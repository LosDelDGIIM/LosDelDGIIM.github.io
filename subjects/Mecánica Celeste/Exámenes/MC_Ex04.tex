\documentclass[12pt]{article}

% Idioma y codificación
\usepackage[spanish, es-tabla]{babel}       %es-tabla para que se titule "Tabla"
\usepackage[utf8]{inputenc}

% Márgenes
\usepackage[a4paper,top=3cm,bottom=2.5cm,left=3cm,right=3cm]{geometry}

% Comentarios de bloque
\usepackage{verbatim}

% Paquetes de links
\usepackage[hidelinks]{hyperref}    % Permite enlaces
\usepackage{url}                    % redirecciona a la web

% Más opciones para enumeraciones
\usepackage{enumitem}

% Personalizar la portada
\usepackage{titling}

% Paquetes de tablas
\usepackage{multirow}


%------------------------------------------------------------------------

%Paquetes de figuras
\usepackage{caption}
\usepackage{subcaption} % Figuras al lado de otras
\usepackage{float}      % Poner figuras en el sitio indicado H.


% Paquetes de imágenes
\usepackage{graphicx}       % Paquete para añadir imágenes
\usepackage{transparent}    % Para manejar la opacidad de las figuras

% Paquete para usar colores
\usepackage[dvipsnames]{xcolor}
\usepackage{pagecolor}      % Para cambiar el color de la página

% Habilita tamaños de fuente mayores
\usepackage{fix-cm}

% Para los gráficos
\usepackage{tikz}

% Para poder situar los nodos en los grafos
\usetikzlibrary{positioning}


%------------------------------------------------------------------------

% Paquetes de matemáticas
\usepackage{mathtools, amsfonts, amssymb, mathrsfs}
\usepackage[makeroom]{cancel}     % Simplificar tachando
\usepackage{polynom}    % Divisiones y Ruffini
\usepackage{units} % Para poner fracciones diagonales con \nicefrac

\usepackage{pgfplots}   %Representar funciones
\pgfplotsset{compat=1.18}  % Versión 1.18

\usepackage{tikz-cd}    % Para usar diagramas de composiciones
\usetikzlibrary{calc}   % Para usar cálculo de coordenadas en tikz

%Definición de teoremas, etc.
\usepackage{amsthm}
%\swapnumbers   % Intercambia la posición del texto y de la numeración

\theoremstyle{plain}

\makeatletter
\@ifclassloaded{article}{
  \newtheorem{teo}{Teorema}[section]
}{
  \newtheorem{teo}{Teorema}[chapter]  % Se resetea en cada chapter
}
\makeatother

\newtheorem{coro}{Corolario}[teo]           % Se resetea en cada teorema
\newtheorem{prop}[teo]{Proposición}         % Usa el mismo contador que teorema
\newtheorem{lema}[teo]{Lema}                % Usa el mismo contador que teorema

\theoremstyle{remark}
\newtheorem*{observacion}{Observación}

\theoremstyle{definition}

\makeatletter
\@ifclassloaded{article}{
  \newtheorem{definicion}{Definición} [section]     % Se resetea en cada chapter
}{
  \newtheorem{definicion}{Definición} [chapter]     % Se resetea en cada chapter
}
\makeatother

\newtheorem*{notacion}{Notación}
\newtheorem*{ejemplo}{Ejemplo}
\newtheorem*{ejercicio*}{Ejercicio}             % No numerado
\newtheorem{ejercicio}{Ejercicio} [section]     % Se resetea en cada section


% Modificar el formato de la numeración del teorema "ejercicio"
\renewcommand{\theejercicio}{%
  \ifnum\value{section}=0 % Si no se ha iniciado ninguna sección
    \arabic{ejercicio}% Solo mostrar el número de ejercicio
  \else
    \thesection.\arabic{ejercicio}% Mostrar número de sección y número de ejercicio
  \fi
}


% \renewcommand\qedsymbol{$\blacksquare$}         % Cambiar símbolo QED
%------------------------------------------------------------------------

% Paquetes para encabezados
\usepackage{fancyhdr}
\pagestyle{fancy}
\fancyhf{}

\newcommand{\helv}{ % Modificación tamaño de letra
\fontfamily{}\fontsize{12}{12}\selectfont}
\setlength{\headheight}{15pt} % Amplía el tamaño del índice


%\usepackage{lastpage}   % Referenciar última pag   \pageref{LastPage}
\fancyfoot[C]{\thepage}

%------------------------------------------------------------------------

% Conseguir que no ponga "Capítulo 1". Sino solo "1."
\makeatletter
\@ifclassloaded{book}{
  \renewcommand{\chaptermark}[1]{\markboth{\thechapter.\ #1}{}} % En el encabezado
    
  \renewcommand{\@makechapterhead}[1]{%
  \vspace*{50\p@}%
  {\parindent \z@ \raggedright \normalfont
    \ifnum \c@secnumdepth >\m@ne
      \huge\bfseries \thechapter.\hspace{1em}\ignorespaces
    \fi
    \interlinepenalty\@M
    \Huge \bfseries #1\par\nobreak
    \vskip 40\p@
  }}
}
\makeatother

%------------------------------------------------------------------------
% Paquetes de cógido
\usepackage{minted}
\renewcommand\listingscaption{Código fuente}

\usepackage{fancyvrb}
% Personaliza el tamaño de los números de línea
\renewcommand{\theFancyVerbLine}{\small\arabic{FancyVerbLine}}

% Estilo para C++
\newminted{cpp}{
    frame=lines,
    framesep=2mm,
    baselinestretch=1.2,
    linenos,
    escapeinside=||
}

% para minted
\definecolor{LightGray}{rgb}{0.95,0.95,0.92}
\setminted{
    linenos=true,
    stepnumber=5,
    numberfirstline=true,
    autogobble,
    breaklines=true,
    breakautoindent=true,
    breaksymbolleft=,
    breaksymbolright=,
    breaksymbolindentleft=0pt,
    breaksymbolindentright=0pt,
    breaksymbolsepleft=0pt,
    breaksymbolsepright=0pt,
    fontsize=\footnotesize,
    bgcolor=LightGray,
    numbersep=10pt
}


\usepackage{listings} % Para incluir código desde un archivo

\renewcommand\lstlistingname{Código Fuente}
\renewcommand\lstlistlistingname{Índice de Códigos Fuente}

% Definir colores
\definecolor{vscodepurple}{rgb}{0.5,0,0.5}
\definecolor{vscodeblue}{rgb}{0,0,0.8}
\definecolor{vscodegreen}{rgb}{0,0.5,0}
\definecolor{vscodegray}{rgb}{0.5,0.5,0.5}
\definecolor{vscodebackground}{rgb}{0.97,0.97,0.97}
\definecolor{vscodelightgray}{rgb}{0.9,0.9,0.9}

% Configuración para el estilo de C similar a VSCode
\lstdefinestyle{vscode_C}{
  backgroundcolor=\color{vscodebackground},
  commentstyle=\color{vscodegreen},
  keywordstyle=\color{vscodeblue},
  numberstyle=\tiny\color{vscodegray},
  stringstyle=\color{vscodepurple},
  basicstyle=\scriptsize\ttfamily,
  breakatwhitespace=false,
  breaklines=true,
  captionpos=b,
  keepspaces=true,
  numbers=left,
  numbersep=5pt,
  showspaces=false,
  showstringspaces=false,
  showtabs=false,
  tabsize=2,
  frame=tb,
  framerule=0pt,
  aboveskip=10pt,
  belowskip=10pt,
  xleftmargin=10pt,
  xrightmargin=10pt,
  framexleftmargin=10pt,
  framexrightmargin=10pt,
  framesep=0pt,
  rulecolor=\color{vscodelightgray},
  backgroundcolor=\color{vscodebackground},
}

%------------------------------------------------------------------------

% Comandos definidos
\newcommand{\bb}[1]{\mathbb{#1}}
\newcommand{\cc}[1]{\mathcal{#1}}

% I prefer the slanted \leq
\let\oldleq\leq % save them in case they're every wanted
\let\oldgeq\geq
\renewcommand{\leq}{\leqslant}
\renewcommand{\geq}{\geqslant}

% Si y solo si
\newcommand{\sii}{\iff}

% Letras griegas
\newcommand{\eps}{\epsilon}
\newcommand{\veps}{\varepsilon}
\newcommand{\lm}{\lambda}

\newcommand{\ol}{\overline}
\newcommand{\ul}{\underline}
\newcommand{\wt}{\widetilde}
\newcommand{\wh}{\widehat}

\let\oldvec\vec
\renewcommand{\vec}{\overrightarrow}

% Derivadas parciales
\newcommand{\del}[2]{\frac{\partial #1}{\partial #2}}
\newcommand{\Del}[3]{\frac{\partial^{#1} #2}{\partial #3^{#1}}}
\newcommand{\deld}[2]{\dfrac{\partial #1}{\partial #2}}
\newcommand{\Deld}[3]{\dfrac{\partial^{#1} #2}{\partial #3^{#1}}}


\newcommand{\AstIg}{\stackrel{(\ast)}{=}}
\newcommand{\Hop}{\stackrel{L'H\hat{o}pital}{=}}

\newcommand{\red}[1]{{\color{red}#1}} % Para integrales, destacar los cambios.

% Método de integración
\newcommand{\MetInt}[2]{
    \left[\begin{array}{c}
        #1 \\ #2
    \end{array}\right]
}

% Declarar aplicaciones
% 1. Nombre aplicación
% 2. Dominio
% 3. Codominio
% 4. Variable
% 5. Imagen de la variable
\newcommand{\Func}[5]{
    \begin{equation*}
        \begin{array}{rrll}
            #1:& #2 & \longrightarrow & #3\\
               & #4 & \longmapsto & #5
        \end{array}
    \end{equation*}
}

%------------------------------------------------------------------------


\newcommand{\R}{\mathbb{R}}
\newcommand{\prodescalar}[2]{\langle #1, #2 \rangle}
\newcommand{\ortogonal}[1]{#1^{\perp}}

\begin{document}

    % 1. Foto de fondo
    % 2. Título
    % 3. Encabezado Izquierdo
    % 4. Color de fondo
    % 5. Coord x del titulo
    % 6. Coord y del titulo
    % 7. Fecha

    
    % 1. Foto de fondo
% 2. Título
% 3. Encabezado Izquierdo
% 4. Color de fondo
% 5. Coord x del titulo
% 6. Coord y del titulo
% 7. Fecha

\newcommand{\portada}[7]{

    \portadaBase{#1}{#2}{#3}{#4}{#5}{#6}{#7}
    \portadaBook{#1}{#2}{#3}{#4}{#5}{#6}{#7}
}

\newcommand{\portadaExamen}[7]{

    \portadaBase{#1}{#2}{#3}{#4}{#5}{#6}{#7}
    \portadaArticle{#1}{#2}{#3}{#4}{#5}{#6}{#7}
}




\newcommand{\portadaBase}[7]{

    % Tiene la portada principal y la licencia Creative Commons
    
    % 1. Foto de fondo
    % 2. Título
    % 3. Encabezado Izquierdo
    % 4. Color de fondo
    % 5. Coord x del titulo
    % 6. Coord y del titulo
    % 7. Fecha
    
    
    \thispagestyle{empty}               % Sin encabezado ni pie de página
    \newgeometry{margin=0cm}        % Márgenes nulos para la primera página
    
    
    % Encabezado
    \fancyhead[L]{\helv #3}
    \fancyhead[R]{\helv \nouppercase{\leftmark}}
    
    
    \pagecolor{#4}        % Color de fondo para la portada
    
    \begin{figure}[p]
        \centering
        \transparent{0.3}           % Opacidad del 30% para la imagen
        
        \includegraphics[width=\paperwidth, keepaspectratio]{assets/#1}
    
        \begin{tikzpicture}[remember picture, overlay]
            \node[anchor=north west, text=white, opacity=1, font=\fontsize{60}{90}\selectfont\bfseries\sffamily, align=left] at (#5, #6) {#2};
            
            \node[anchor=south east, text=white, opacity=1, font=\fontsize{12}{18}\selectfont\sffamily, align=right] at (9.7, 3) {\textbf{\href{https://losdeldgiim.github.io/}{Los Del DGIIM}}};
            
            \node[anchor=south east, text=white, opacity=1, font=\fontsize{12}{15}\selectfont\sffamily, align=right] at (9.7, 1.8) {Doble Grado en Ingeniería Informática y Matemáticas\\Universidad de Granada};
        \end{tikzpicture}
    \end{figure}
    
    
    \restoregeometry        % Restaurar márgenes normales para las páginas subsiguientes
    \pagecolor{white}       % Restaurar el color de página
    
    
    \newpage
    \thispagestyle{empty}               % Sin encabezado ni pie de página
    \begin{tikzpicture}[remember picture, overlay]
        \node[anchor=south west, inner sep=3cm] at (current page.south west) {
            \begin{minipage}{0.5\paperwidth}
                \href{https://creativecommons.org/licenses/by-nc-nd/4.0/}{
                    \includegraphics[height=2cm]{assets/Licencia.png}
                }\vspace{1cm}\\
                Esta obra está bajo una
                \href{https://creativecommons.org/licenses/by-nc-nd/4.0/}{
                    Licencia Creative Commons Atribución-NoComercial-SinDerivadas 4.0 Internacional (CC BY-NC-ND 4.0).
                }\\
    
                Eres libre de compartir y redistribuir el contenido de esta obra en cualquier medio o formato, siempre y cuando des el crédito adecuado a los autores originales y no persigas fines comerciales. 
            \end{minipage}
        };
    \end{tikzpicture}
    
    
    
    % 1. Foto de fondo
    % 2. Título
    % 3. Encabezado Izquierdo
    % 4. Color de fondo
    % 5. Coord x del titulo
    % 6. Coord y del titulo
    % 7. Fecha


}


\newcommand{\portadaBook}[7]{

    % 1. Foto de fondo
    % 2. Título
    % 3. Encabezado Izquierdo
    % 4. Color de fondo
    % 5. Coord x del titulo
    % 6. Coord y del titulo
    % 7. Fecha

    % Personaliza el formato del título
    \pretitle{\begin{center}\bfseries\fontsize{42}{56}\selectfont}
    \posttitle{\par\end{center}\vspace{2em}}
    
    % Personaliza el formato del autor
    \preauthor{\begin{center}\Large}
    \postauthor{\par\end{center}\vfill}
    
    % Personaliza el formato de la fecha
    \predate{\begin{center}\huge}
    \postdate{\par\end{center}\vspace{2em}}
    
    \title{#2}
    \author{\href{https://losdeldgiim.github.io/}{Los Del DGIIM}}
    \date{Granada, #7}
    \maketitle
    
    \tableofcontents
}




\newcommand{\portadaArticle}[7]{

    % 1. Foto de fondo
    % 2. Título
    % 3. Encabezado Izquierdo
    % 4. Color de fondo
    % 5. Coord x del titulo
    % 6. Coord y del titulo
    % 7. Fecha

    % Personaliza el formato del título
    \pretitle{\begin{center}\bfseries\fontsize{42}{56}\selectfont}
    \posttitle{\par\end{center}\vspace{2em}}
    
    % Personaliza el formato del autor
    \preauthor{\begin{center}\Large}
    \postauthor{\par\end{center}\vspace{3em}}
    
    % Personaliza el formato de la fecha
    \predate{\begin{center}\huge}
    \postdate{\par\end{center}\vspace{5em}}
    
    \title{#2}
    \author{\href{https://losdeldgiim.github.io/}{Los Del DGIIM}}
    \date{Granada, #7}
    \thispagestyle{empty}               % Sin encabezado ni pie de página
    \maketitle
    \vfill
}
    \portadaExamen{ffccA4.jpg}{Mecánica Celeste\\Examen IV}{Mecánica Celeste. Examen IV}{MidnightBlue}{-8}{28}{2025}{José Manuel Sánchez Varbas}

    \begin{description}
        \item[Asignatura] Mecánica Celeste.
        \item[Curso Académico] 2024-25.
        \item[Grado] Grado en Matemáticas.
        \item[Grupo] A.
        \item[Profesor] Margarita Arias López.
        \item[Descripción] Primer Parcial.
        \item[Fecha] 25 de Octubre de 2024.
        \item[Duración] 1 hora y 30 minutos.
    \end{description}
    \newpage

    % ------------------------------------
    
    El número entre corchetes es la puntuación máxima de cada ejercicio o apartado.

    \begin{ejercicio}[2 puntos]
        Sea $x : I \to \R^3 \setminus \{0\}$ una solución de la ecuación 
        $$\ddot{x} = - \dfrac{1}{|x|^3} x$$
        Justifica si son verdaderas o falsas las siguientes afirmaciones:
        \begin{itemize}
            \item[a)] [1] Si $x(0) = (0, 0, 1)$ y $\dot{x}(0) = (0, 0, a)$, con $a>0$, $x(t)$ está definida
            para todo $t \geqslant 0$ y su órbita recorre el semieje vertical positivo.
            \item[b)] [1] Si $x(0) = (0, 0, 1)$ y $\dot{x}(0) = (a, 0, 0)$, con $a>0$, la órbita de $x(t)$
            recorre una elipse en el plano $\{x_2 = 0\}$. 
        \end{itemize}
    \end{ejercicio}

    \begin{ejercicio}[2 puntos]
        Sabiendo que el semieje mayor de la órbita de Saturno es aproximadamente de $9,5$ a.u., determina
        su periodo orbital en días.
    \end{ejercicio}

    \begin{ejercicio}[2 puntos]
        Se sabe que la densidad $\rho$ de un planeta, que se supone esférico, es constante y que 
        $\rho = \frac{3}{4 \pi G}$, con $G$ la constante de gravitación. Si un saltador es capaz de 
        impulsarse con una velocidad de $6$ m/s, ¿qué radio debe tener el planeta para que el saltador pueda
        escaparse de un brinco?\footnote{recuerda que el volumen de una esfera es $\frac{4}{3}\pi r^3$}
    \end{ejercicio}

    \begin{ejercicio}[4 puntos]
    Un satélite gira alrededor de un planeta de masa igual al inverso de la constante de gravitación universal
    $MG = 1$ describiendo una trayectoria elíptica. Se sabe que la excentricidad de la elipse que describe es
    $\varepsilon = 0.2$ y que el semieje mayor es $a = 10^3$ km. Se pide:
    \begin{itemize}
        \item[a)] Determinar las distancias máxima y mínima del satélite al planeta.
        \item[b)] Calcular el tiempo que tarda el satélite en dar una vuelta completa a su órbita.
        \item[c)] Determinar la velocidad del satélite en los puntos en que su órbita atraviesa los ejes de 
        la elipse que describe (puntos $Q_i, i=1,2,3,4$ de la Figura \ref{fig:imagenej4})
        
        \begin{figure}[H]
            \begin{center}
                \includegraphics[scale=0.25]{fig12425.png}
            \end{center}
            \caption{Trayectoria elíptica descrita}
            \label{fig:imagenej4}
        \end{figure}
    \end{itemize}
    \end{ejercicio}

    \newpage

    \setcounter{ejercicio}{0}

    \begin{ejercicio}[2 puntos]
        Sea $x : I \to \R^3 \setminus \{0\}$ una solución de la ecuación 
        $$\ddot{x} = - \dfrac{1}{|x|^3} x$$
        Justifica si son verdaderas o falsas las siguientes afirmaciones:
        \begin{itemize}
            \item[a)] [1] Si $x(0) = (0, 0, 1)$ y $\dot{x}(0) = (0, 0, a)$, con $a>0$, $x(t)$ está definida
            para todo $t \geqslant 0$ y su órbita recorre el semieje vertical positivo. \\

            Dado que estamos en un c.f.c. (campo de fuerzas centrales), sabemos que el momento angular será constante.
            Podemos obtener $$c(0) = x(0) \land \dot{x}(0) = \begin{vmatrix}
                0 & 0 & 1 \\
                0 & 0 & a \\
                i & j & k
            \end{vmatrix} = (0, 0, 0) \Longrightarrow |c| = 0$$

            Es decir, el movimiento se da, salvo isometría, a lo largo de una semirrecta. En tal caso,
            $x(t) = r(t) v$, con $v \in \R^3$ y $|v| = 1$. Por teoría, sabemos que hay tres casos posibles (dado
            que el campo es atractivo), aunque en todos ellos el intervalo maximal es de la forma $]\alpha, \omega[$.
            Por ser c.f.c. tenemos que es conservativo, luego la energía total 

            $$E = \dfrac{1}{2} |\dot{x}(t)|^2 - \dfrac{\mu}{|x(t)|}$$

            se conserva. Usando los valores dados en $t=0$, e identificando $\mu = 1$ por la forma de la ecuación
            dada,
            
            $$E = \dfrac{1}{2} |\dot{x}(0)|^2 - \dfrac{1}{|x(0)|} = \dfrac{1}{2} a^2 - \dfrac{1}{1} = 
            \dfrac{a^2}{2} - 1$$


            Una condición necesaria para que $x(t)$ esté definida para todo $t \geqslant 0$ es que 
            $E \geqslant 0$, de lo contrario, $r$ tendría un cambio de monotonía y estaríamos en el caso en que
            $- \infty < \alpha < \omega < +\infty$, luego no estaría definida $[0, +\infty[$. Sin embargo,
            esto no es cierto en general, ya que la condición 
            $$E < 0 \iff \dfrac{a^2}{2} - 1 < 0 \iff \dfrac{a^2}{2} < 1 \iff a^2 < 2 \iff |a| < \sqrt{2} \iff 
            - \sqrt{2} < a < \sqrt{2}$$

            La primera desigualdad está clara, porque $a>0$, pero la segunda no, porque en principio, 

            $$0 < a < \sqrt{2}$$

            y podría ser que $a < \sqrt{2}$, por lo que no es cierta en general.

            \item[b)] [1] Si $x(0) = (0, 0, 1)$ y $\dot{x}(0) = (a, 0, 0)$, con $a>0$, la órbita de $x(t)$
            recorre una elipse en el plano $\{x_2 = 0\}$. \\

            Nuevamente, hallamos el valor del momento angular en $t=0$:

            $$c(0) = x(0) \land \dot{x}(0) = \begin{vmatrix}
                0 & 0 & 1 \\
                a & 0 & 0 \\
                i & j & k
            \end{vmatrix} = (0, a, 0) \Longrightarrow |c| = a > 0$$

            por tanto, como $|c| \neq 0$, por teoría sabemos que el movimiento se hará en el plano $\Pi = \{\ortogonal{c}\}$, es decir,
            aquel plano con vector normal $c$ que pase por el origen. El plano verifica la ecuación
            $$Ax_1 + Bx_2 + Cx_3 + D = 0 \Longrightarrow ax_2 = 0 \stackrel{a>0}{\Longrightarrow} x_2 = 0$$

            En efecto, la órbita de $x(t)$ queda contenida en el plano $\{x_2 = 0\}$. Falta ver que es una elipse. \\

            Para ello, podemos usar la energía igual que en el apartado anterior

            $$h = \dfrac{1}{2} |\dot{x}(0)|^2 - \dfrac{1}{|x(0)|} = \dfrac{a^2}{2} - 1$$

            Sabemos que $h < 0 \iff |e| < 1$, es decir, comprobar que $h<0$ es equivalente a comprobar 
            que $x(t)$ recorre una elipse, pero eso no tiene porqué ser cierto, por el mismo motivo 
            que el apartado anterior. Solo es cierto si $- \sqrt{2} < a < \sqrt{2}$, y como $a>0$, la 
            condición sobre el parámetro $a$ para que esta afirmación sea verdadera es que 
            $$\boxed{0 < a < \sqrt{2}}$$  


        \end{itemize}
    \end{ejercicio}

    \newpage

    \begin{ejercicio}[2 puntos]
        Sabiendo que el semieje mayor de la órbita de Saturno es aproximadamente de $9,5$ a.u., determina
        su periodo orbital en días. \\

        Usaremos la Tercera Ley de Kepler, que nos dice que el cuadrado de los periodos de revolución 
        es proporcional al cubo de las distancias medias de las órbitas. En el caso del sistema solar, esta constante
        de proporcionalidad es la misma para todos los planetas. Por tanto, dado que la Tierra tiene $1$ a.u. y periodo $1$
        año, entonces 

        $$\dfrac{p^2}{a^3} = k = \dfrac{(1\text{ año})^2}{(1\text{ a.u.})^3} = 1$$

        Así, para cualquier planeta del sistema solar, considerando el Sol como el cuerpo central, se tiene que
        $p^2 = a^3$. Dado que ya tenemos la longitud del semieje mayor, y sabemos que si $a$ está en a.u., 
        entonces $p$ está en años, queda usar el factor de conversión
        $1$ año = $365$ días (supondremos que no es bisiesto), y entonces se obtiene el periodo orbital en días

        $$p = a^{3/2} = 9,5^{3/2} \cancel{\text{ años }} \cdot 
        \dfrac{365 \text{ días}}{\cancel{1 \text{ año}}} = 10687,55278 \approx 10678 \text{ días}$$
    \end{ejercicio}

    \newpage

    \begin{ejercicio}[2 puntos]
        Se sabe que la densidad $\rho$ de un planeta, que se supone esférico, es constante y que 
        $\rho = \frac{3}{4 \pi G}$, con $G$ la constante de gravitación. Si un saltador es capaz de 
        impulsarse con una velocidad de $6$ m/s, ¿qué radio debe tener el planeta para que el saltador pueda
        escaparse de un brinco? \\

        \begin{description}
            \item[Opción 1. Usando la fórmula de la velocidad de escape.] ~
            
            Por definición de densidad, $\rho = \frac{M}{V}$, con $M$ la masa del planeta, y $V$ su respectivo volumen. 
            Como conocemos el volumen de la esfera y su densidad, podemos obtener una relación entre $G$, $M$ y el radio $R$ (para que la densidad sea finita, como es el caso, 
            estamos suponiendo que $R>0$), que luego usaremos
            en la fórmula de la velocidad de escape.
    
            $$\rho = \dfrac{M}{V} \iff \dfrac{\cancel{3}}{\cancel{4 \pi} G} = \dfrac{M}{\dfrac{4}{3} \pi R^3} = \dfrac{\cancel{3}M}{\cancel{4 \pi} R^3} \iff \dfrac{1}{G} = \dfrac{M}{R^3} \iff 1 = \dfrac{GM}{R^3}
            \stackrel{R>0}{\iff} R^2 \stackrel{(*)}{=} \dfrac{GM}{R}$$
    
            También sabemos por la Física de Bachillerato que la velocidad de escape tiene por fórmula 
    
            $$v_e = \sqrt{\dfrac{2GM}{R}}$$
    
            y como el saltador puede llegar hasta $6$ m/s, imponiendo que esta sea la velocidad de escape, despejamos el radio, y obtendremos el valor 
            del radio que debe tener el planeta para que el saltador pueda escaparse de un brinco
    
            $$6 \text{ m/s} = v_e = \sqrt{2 \dfrac{GM}{R}} \stackrel{(*)}{=} \sqrt{2 R^2} \stackrel{R>0}{=} \sqrt{2} R$$
    
            de donde, el radio buscado es $$\boxed{R = \dfrac{6}{\sqrt{2}} = 3 \sqrt{2} \approx 4,24 \text{ m}}$$

            Sin embargo, si $0 < R \leqslant 3 \sqrt{2}$, también se cumple (lo cual es
            lógico porque, si con una velocidad inicial determinada escapa
            de un radio, lo hará también para un radio menor con esa misma 
            velocidad inicial), por tanto, los valores que puede tomar
            el radio para que el saltador pueda escapar de un brinco son 

            $$\boxed{0 < R \leqslant 3 \sqrt{2} \text{ m}}$$
            
            \newpage

            \item[Opción 2. No usando la fórmula de la velocidad de escape.] ~
            
            Consideramos la definición de densidad
            $$\rho = \dfrac{M}{V}$$
            Usando que $\rho = \frac{3}{4 \pi G}$, y $V = \frac{4}{3} \pi R^3$, obtenemos
            el valor de la masa $M$
            $$M = \rho V = 
            \dfrac{\cancel{3}}{\cancel{4 \pi} G} \cdot \dfrac{\cancel{4 \pi}}{\cancel{3}} R^3 = \dfrac{R^3}{G}$$
            Teniendo en cuenta que estamos en un campo gravitatorio newtoniano
            de masa $M$, entonces se verifica la ecuación
            $$\ddot{x} = - \dfrac{\mu}{|x|^3}x \quad \mu = GM$$
            es decir, $\mu = GM = G \frac{R^3}{G} = R^3$, y la ecuación resulta ser
            $$\ddot{x} = - \dfrac{R^3}{|x|^3} x$$
            Como el campo newtoniano es un c.f.c., en particular es conservativo,
            con potencial conocido, $V(x) = - \frac{\mu}{|x|}$. Podemos obtener entonces
            la energía total
            $$E = \dfrac{1}{2}|\dot{x}(t)|^2 + V(x(t)) = 
            \dfrac{1}{2}|\dot{x}(t)|^2 - \dfrac{\mu}{|x(t)|} \stackrel{\mu = R^3}{=} \dfrac{1}{2}|\dot{x}(t)|^2 - \frac{R^3}{|x(t)|} $$
            Suponemos que el saltador brinca en una dirección radial, de tal manera que 
            podemos imponer las condiciones iniciales siguientes
            $$x(0) = R \beta, \quad \dot{x}(0) = 6 \beta, \quad |\beta|=1$$
            y concluir que 
            $$c(0) = x(0) \land \dot{x}(0) = 0 \Longrightarrow |c| = 0$$
            porque $x$ y $\dot{x}$ son paralelos, y 
            consecuentemente el movimiento es rectilíneo.
            En tal caso, podemos clasificar dicho movimiento 
            por medio de la energía. Por teoría sabemos que 
            $x(t) = r(t) v$, con $|v|=1$.
            Como $\dot{r}(0) = 6 > 0$, entonces si $E \geqslant 0$,
            $r(t)$ es estrictamente creciente y no acotada, luego el saltador
            escapa. Por tanto hay que imponer que $E \geqslant 0$. 
            De lo contrario, habría un cambio
            de monotonía, y estaríamos en el caso en que 
            $-\infty < \alpha < \omega < +\infty$, y el intervalo maximal 
            de la solución (del movimiento) sería de la forma $]\alpha, \omega[ \subset \R$. 
            Obteniendo la energía en $t=0$, y usando la conservación
            de la energía total

            $$E = \dfrac{1}{2} |\dot{x}(0)|^2 - \dfrac{R^3}{|x(0)|} = 
            \dfrac{1}{2} \cdot 6^2 - \dfrac{R^3}{R} \stackrel{R>0}{=} 
            18 - R^2$$

            Como debe ser $E \geqslant 0$, entonces la condición sobre el radio
            es 
            $$E \geqslant 0 \iff 18 - R^2 \geqslant 0 \iff 18 \geqslant R^2 \iff
            |R| \leqslant \sqrt{18} = 3 \sqrt{2} \iff -3\sqrt{2} \leqslant 
            R \leqslant 3 \sqrt{2}$$

            y por la naturaleza del problema, $R>0$, y entonces para que
            el saltador pueda escapar, debe cumplirse que

            \begin{center}
                \boxed{0 < R \leqslant 3 \sqrt{2} \text{ m}}
            \end{center}

        \end{description}

    \end{ejercicio}

    \newpage

    \begin{ejercicio}[4 puntos]
    Un satélite gira alrededor de un planeta de masa igual al inverso de la constante de gravitación universal
    $MG = 1$ describiendo una trayectoria elíptica. Se sabe que la excentricidad de la elipse que describe es
    $\varepsilon = 0.2$ y que el semieje mayor es $a = 10^3$ km. Se pide:
    \begin{itemize}
        \item[a)] Determinar las distancias máxima y mínima del satélite al planeta. \\
        
        Por teoría, dado que estamos en una órbita elíptica, sabemos que los puntos a los que nos referimos son, respectivamente, el apoastro, y el periastro.
        Aplicando la Primera Ley de Kepler al satélite y al planeta, el planeta está en uno de los focos de la trayectoria elíptica (según la Figura \ref{fig:imagenej4} en el origen de coordenadas). \\

        Si la ecuación de la elipse viene dada por $|x| + \prodescalar{e}{x} = k$, o por su expresión en polares
        
        $$r = \dfrac{k}{1 + \varepsilon \cos(\theta - \omega)}$$
        
        las fórmulas de las distancias del planeta al apoastro y al periastro son, respectivamente

        $$r_{\max} = \dfrac{k}{1-\varepsilon} \quad r_{\min} = \dfrac{k}{1+\varepsilon}$$

        y como $a = \frac{k}{1-\varepsilon^2}$, podemos despejar $k$, y obtener ambas distancias.

        $$a = \dfrac{k}{1-\varepsilon^2} \iff k = a(1- \varepsilon^2) = 10^3(1-0.2^2) = 960 \text{ km}$$

        y

        $$\boxed{r_{\max} = \dfrac{960}{1-0,2} = 1200 \text{ km}} \quad \boxed{r_{\min} = \dfrac{960}{1+0,2} = 800 \text{ km}}$$

        De esta manera 

        \begin{center}
            \boxed{\text{la distancia máxima del satélite al planeta es de 1200 kilómetros}}
        \end{center}


        y

        \begin{center}
            \boxed{\text{la distancia mínima del satélite al planeta es de 800 kilómetros}}
        \end{center}

        \newpage

        \item[b)] Calcular el tiempo que tarda el satélite en dar una vuelta completa a su órbita. \\
        
        Podemos usar la Tercera Ley de Kepler, obteniendo que

        $$p = \dfrac{2 \pi}{\sqrt{\mu}} a^{3/2}$$

        donde $\mu = GM = 1 \text{ m}^3 / \text{s}^2$ (unidades del SI) por el enunciado. Como $a$ está dado en kilómetros, por las unidades de $\mu$, hay que pasarlo a metros, luego $a = 10^3 \text{ km} = 10^6 \text{ m}$, de tal manera que el periodo vendrá expresado en segundos 
        $$p = \dfrac{2 \pi}{\sqrt{1}} (10^6)^{3/2} \approx 6283185307 \text{ s} \approx 104719755 \text{ min} \approx
        1745329 \text{ h} \approx 72722 \text{ días} \approx 199 \text{ años}$$

        \boxed{\text{Así, el satélite tarda aproximadamente 199 años en dar una vuelta completa a su órbita.}}

        \item[c)] Determinar la velocidad del satélite en los puntos en que su órbita atraviesa los ejes de 
        la elipse que describe (puntos $Q_i, i=1,2,3,4$ de la Figura \ref{fig:imagenej4})
        
        \begin{figure}[H]
            \begin{center}
                \includegraphics[scale=0.25]{fig12425.png}
            \end{center}
        \end{figure}

    Usando que estamos en un campo newtoniano, entonces el campo es conservativo, es decir, la energía total

    $$h = \dfrac{1}{2} |\dot{x}(t)|^2 - \dfrac{\mu}{|x(t)|}$$

    se conserva (es constante). Además sabemos por la teoría de clasificación de movimientos en cónicas según su 
    energía, que, como estamos en una órbita elíptica, debe ser $h<0$. De hecho, esa relación sale de 

    $$\mu^2 (\varepsilon^2 - 1) = 2 h |c|^2 \quad (**)$$

    Conocemos también el valor de $k$ de la ecuación de la elipse en polares, que es $k = \frac{c^2}{\mu} \iff c^2 = \mu k$.
    Usando que $k = a(1- \varepsilon^2)$ (visto en el apartado anterior), obtenemos la ecuación vis-viva como sigue. 
    Primero, despejamos $h$ en $(**)$

    $$\mu^2 (\varepsilon^2 - 1) = 2 h |c|^2 \iff h = \dfrac{\mu^2[\varepsilon^2 - 1]}{2 |c|^2} \quad (***)$$

    Sustituimos en $(***)$ que $|c|^2 = c^2 = \mu k$, y luego que $k=a(1- \varepsilon^2)$

    $$h = \dfrac{\mu^2(\varepsilon^2 - 1)}{2 |c|^2} = \dfrac{\mu^{\cancel{2}}(\varepsilon^2 - 1)}{2 \cancel{\mu} k} = 
    \dfrac{\mu(\varepsilon^2 - 1)}{2 a(1-\varepsilon^2)} = - \dfrac{\mu \cancel{(1 - \varepsilon^2)}}{2 a \cancel{(1-\varepsilon^2)}} \stackrel{(1)}{=} - \dfrac{\mu}{2a} < 0 \quad (\mu, a > 0)$$

    donde en $(1)$ hemos usado que el movimiento se da en una elipse, luego $$\varepsilon = |e| < 1 \Longrightarrow \varepsilon^2 < 1 \Longrightarrow 1 - \varepsilon^2 > 0$$

    Igualando ambas expresiones de la energía total, obtenemos
    $$\dfrac{1}{2} |\dot{x}(t)|^2 - \dfrac{\mu}{|x(t)|} = - \dfrac{\mu}{2a} \stackrel{\cdot 2}{\iff} |\dot{x}(t)|^2 - \dfrac{2 \mu}{x(t)} = - \dfrac{\mu}{a} \iff
    |\dot{x}(t)|^2 = \dfrac{2 \mu}{x(t)} - \dfrac{\mu}{a} = \mu \left( \dfrac{2}{x(t)} - \dfrac{1}{a} \right)$$ $$\Longrightarrow \dot{x}(t) = \sqrt{\mu \left( \dfrac{2}{x(t)} - \dfrac{1}{a} \right)}$$

    Ahora, falta obtener $x(t_i)$, donde $t_i$ es el instante en que el planeta se encuentra en el respectivo punto
    $Q_i$, $i=1,2,3,4$, y obtendremos la velocidad en cada punto. Notemos que $Q_1$ y $Q_3$ se corresponden con el periastro y el apoastro, respectivamente. 
    $Q_2$ y $Q_4$ se encuentran a altura módulo la longitud del semieje menor, que sabemos que es $b = a \sqrt{1-\varepsilon^2} = 10^3 \sqrt{1 - 0.2^2} \approx 979,8 \text{ km}$. 
    Por tanto, la distancia hacia ambos puntos de la órbita es la misma, o sea, $d(P,Q_2) = d(P,Q_4) \equiv d$ y verifica el Teorema de Pitágoras $$d^2 = b^2 + (a-800)^2 \Longrightarrow d = \sqrt{b^2 + (a-800)^2}$$
    Este esquema puede verse en la Figura \ref{fig:resimagenej4}

    \begin{figure}[H]
            \begin{center}

            \tikzset{every picture/.style={line width=0.75pt}} %set default line width to 0.75pt        

            \begin{tikzpicture}[x=0.75pt,y=0.75pt,yscale=-1,xscale=1]
            %uncomment if require: \path (0,427); %set diagram left start at 0, and has height of 427

            %Shape: Right Triangle [id:dp9877342649175231] 
            \draw   (319.92,31) -- (500.33,129.25) -- (319.92,129.25) -- cycle ;
            %Shape: Right Triangle [id:dp676544339807461] 
            \draw   (319.92,227.51) -- (500.44,129.25) -- (319.92,129.25) -- cycle ;
            %Shape: Axis 2D [id:dp5567353598720959] 
            \draw  (51.92,128.68) -- (589.92,128.68)(501.33,39.25) -- (501.33,285.25) (582.92,123.68) -- (589.92,128.68) -- (582.92,133.68) (496.33,46.25) -- (501.33,39.25) -- (506.33,46.25)  ;
            %Shape: Ellipse [id:dp20869439735837014] 
            \draw   (83.92,129.12) .. controls (83.92,74.38) and (190.47,30) .. (321.92,30) .. controls (453.36,30) and (559.92,74.38) .. (559.92,129.12) .. controls (559.92,183.87) and (453.36,228.25) .. (321.92,228.25) .. controls (190.47,228.25) and (83.92,183.87) .. (83.92,129.12) -- cycle ;
            %Shape: Brace [id:dp12819468886218754] 
            \draw   (319.92,129.25) .. controls (319.9,133.92) and (322.22,136.26) .. (326.89,136.28) -- (400.1,136.59) .. controls (406.77,136.62) and (410.09,138.96) .. (410.07,143.63) .. controls (410.09,138.96) and (413.43,136.64) .. (420.1,136.67)(417.1,136.66) -- (493.3,136.98) .. controls (497.97,137) and (500.31,134.68) .. (500.33,130.01) ;
            %Shape: Brace [id:dp46107893523462484] 
            \draw   (320.92,260.25) .. controls (320.92,264.92) and (323.25,267.25) .. (327.92,267.25) -- (430.42,267.25) .. controls (437.09,267.25) and (440.42,269.58) .. (440.42,274.25) .. controls (440.42,269.58) and (443.75,267.25) .. (450.42,267.25)(447.42,267.25) -- (552.92,267.25) .. controls (557.59,267.25) and (559.92,264.92) .. (559.92,260.25) ;
            %Shape: Brace [id:dp9377988755670101] 
            \draw   (501.92,189.25) .. controls (501.92,193.92) and (504.25,196.25) .. (508.92,196.25) -- (522.42,196.25) .. controls (529.09,196.25) and (532.42,198.58) .. (532.42,203.25) .. controls (532.42,198.58) and (535.75,196.25) .. (542.42,196.25)(539.42,196.25) -- (555.92,196.25) .. controls (560.59,196.25) and (562.92,193.92) .. (562.92,189.25) ;

            % Text Node
            \draw (513,108) node [anchor=north west][inner sep=0.75pt]   [align=left] {$\displaystyle P$};
            % Text Node
            \draw (305,5) node [anchor=north west][inner sep=0.75pt]   [align=left] {$\displaystyle Q_{2}$};
            % Text Node
            \draw (306,234) node [anchor=north west][inner sep=0.75pt]   [align=left] {$\displaystyle Q_{4}$};
            % Text Node
            \draw (296,166) node [anchor=north west][inner sep=0.75pt]   [align=left] {$\displaystyle -b$};
            % Text Node
            \draw (349,145) node [anchor=north west][inner sep=0.75pt]   [align=left] {$\displaystyle a-d( P,Q_{1})$};
            % Text Node
            \draw (61,107) node [anchor=north west][inner sep=0.75pt]   [align=left] {$\displaystyle Q_{3}$};
            % Text Node
            \draw (562,106) node [anchor=north west][inner sep=0.75pt]   [align=left] {$\displaystyle Q_{1}$};
            % Text Node
            \draw (435,276) node [anchor=north west][inner sep=0.75pt]   [align=left] {$\displaystyle a$};
            % Text Node
            \draw (505,207) node [anchor=north west][inner sep=0.75pt]   [align=left] {$\displaystyle d( P,Q_{1})$};
            % Text Node
            \draw (304,66) node [anchor=north west][inner sep=0.75pt]   [align=left] {$\displaystyle b$};
            % Text Node
            \draw (403,57) node [anchor=north west][inner sep=0.75pt]   [align=left] {$\displaystyle d( P,Q_{2})$};
            % Text Node
            \draw (412.18,181.38) node [anchor=north west][inner sep=0.75pt]   [align=left] {$\displaystyle d( P,Q_{3})$};


            \end{tikzpicture}

            \end{center}
            \caption{Esquema del Apartado c)}
            \label{fig:resimagenej4}
        \end{figure}
    \end{itemize}

    \newpage

    Ya podemos terminar el ejercicio, sin más que sustituir correctamente. Es importante que, al igual que en el apartado anterior, como $[\mu] = \text{m}^3 / \text{s}^2$, entonces tanto las unidades en que expresemos $x(t_i), i=1,2,3,4$ como las de $a$ deben ser igualmente metros.

    \begin{enumerate}
            \item $Q_1$. Se corresponde con el periastro, cuya distancia ya sabemos que es $800$ kilómetros. Por tanto, la velocidad será 
            $$\dot{x}(t_1) = \sqrt{\left( \dfrac{2}{8\cdot10^5} - \dfrac{1}{10^6} \right)} \approx 0,001225 \text{ m/s}$$
            \item $Q_2$ y $Q_4$. La distancia de ambos puntos de la trayectoria elíptica al planeta es $d$, luego 
            $$\dot{x}(t_2) = \dot{x}(t_4) = \sqrt{\left( \dfrac{2}{d \cdot 10^3} - \dfrac{1}{10^6} \right)} = 0,001 \text{ m/s}$$
            \item $Q_3$. Se corresponde con el apoastro, cuya distancia ya sabemos que es $1200$ kilómetros. Por tanto, la velocidad será 
            $$\dot{x}(t_3) = \sqrt{\left( \dfrac{2}{1,2 \cdot 10^6} - \dfrac{1}{10^6} \right)} \approx 0.000816 \text{ m/s}$$
        \end{enumerate}
    \end{ejercicio}

    
\end{document}