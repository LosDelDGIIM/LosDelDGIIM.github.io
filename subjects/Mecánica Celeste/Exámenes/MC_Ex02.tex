\documentclass[12pt]{article}

% Idioma y codificación
\usepackage[spanish, es-tabla]{babel}       %es-tabla para que se titule "Tabla"
\usepackage[utf8]{inputenc}

% Márgenes
\usepackage[a4paper,top=3cm,bottom=2.5cm,left=3cm,right=3cm]{geometry}

% Comentarios de bloque
\usepackage{verbatim}

% Paquetes de links
\usepackage[hidelinks]{hyperref}    % Permite enlaces
\usepackage{url}                    % redirecciona a la web

% Más opciones para enumeraciones
\usepackage{enumitem}

% Personalizar la portada
\usepackage{titling}

% Paquetes de tablas
\usepackage{multirow}


%------------------------------------------------------------------------

%Paquetes de figuras
\usepackage{caption}
\usepackage{subcaption} % Figuras al lado de otras
\usepackage{float}      % Poner figuras en el sitio indicado H.


% Paquetes de imágenes
\usepackage{graphicx}       % Paquete para añadir imágenes
\usepackage{transparent}    % Para manejar la opacidad de las figuras

% Paquete para usar colores
\usepackage[dvipsnames]{xcolor}
\usepackage{pagecolor}      % Para cambiar el color de la página

% Habilita tamaños de fuente mayores
\usepackage{fix-cm}

% Para los gráficos
\usepackage{tikz}

% Para poder situar los nodos en los grafos
\usetikzlibrary{positioning}


%------------------------------------------------------------------------

% Paquetes de matemáticas
\usepackage{mathtools, amsfonts, amssymb, mathrsfs}
\usepackage[makeroom]{cancel}     % Simplificar tachando
\usepackage{polynom}    % Divisiones y Ruffini
\usepackage{units} % Para poner fracciones diagonales con \nicefrac

\usepackage{pgfplots}   %Representar funciones
\pgfplotsset{compat=1.18}  % Versión 1.18

\usepackage{tikz-cd}    % Para usar diagramas de composiciones
\usetikzlibrary{calc}   % Para usar cálculo de coordenadas en tikz

%Definición de teoremas, etc.
\usepackage{amsthm}
%\swapnumbers   % Intercambia la posición del texto y de la numeración

\theoremstyle{plain}

\makeatletter
\@ifclassloaded{article}{
  \newtheorem{teo}{Teorema}[section]
}{
  \newtheorem{teo}{Teorema}[chapter]  % Se resetea en cada chapter
}
\makeatother

\newtheorem{coro}{Corolario}[teo]           % Se resetea en cada teorema
\newtheorem{prop}[teo]{Proposición}         % Usa el mismo contador que teorema
\newtheorem{lema}[teo]{Lema}                % Usa el mismo contador que teorema

\theoremstyle{remark}
\newtheorem*{observacion}{Observación}

\theoremstyle{definition}

\makeatletter
\@ifclassloaded{article}{
  \newtheorem{definicion}{Definición} [section]     % Se resetea en cada chapter
}{
  \newtheorem{definicion}{Definición} [chapter]     % Se resetea en cada chapter
}
\makeatother

\newtheorem*{notacion}{Notación}
\newtheorem*{ejemplo}{Ejemplo}
\newtheorem*{ejercicio*}{Ejercicio}             % No numerado
\newtheorem{ejercicio}{Ejercicio} [section]     % Se resetea en cada section


% Modificar el formato de la numeración del teorema "ejercicio"
\renewcommand{\theejercicio}{%
  \ifnum\value{section}=0 % Si no se ha iniciado ninguna sección
    \arabic{ejercicio}% Solo mostrar el número de ejercicio
  \else
    \thesection.\arabic{ejercicio}% Mostrar número de sección y número de ejercicio
  \fi
}


% \renewcommand\qedsymbol{$\blacksquare$}         % Cambiar símbolo QED
%------------------------------------------------------------------------

% Paquetes para encabezados
\usepackage{fancyhdr}
\pagestyle{fancy}
\fancyhf{}

\newcommand{\helv}{ % Modificación tamaño de letra
\fontfamily{}\fontsize{12}{12}\selectfont}
\setlength{\headheight}{15pt} % Amplía el tamaño del índice


%\usepackage{lastpage}   % Referenciar última pag   \pageref{LastPage}
\fancyfoot[C]{\thepage}

%------------------------------------------------------------------------

% Conseguir que no ponga "Capítulo 1". Sino solo "1."
\makeatletter
\@ifclassloaded{book}{
  \renewcommand{\chaptermark}[1]{\markboth{\thechapter.\ #1}{}} % En el encabezado
    
  \renewcommand{\@makechapterhead}[1]{%
  \vspace*{50\p@}%
  {\parindent \z@ \raggedright \normalfont
    \ifnum \c@secnumdepth >\m@ne
      \huge\bfseries \thechapter.\hspace{1em}\ignorespaces
    \fi
    \interlinepenalty\@M
    \Huge \bfseries #1\par\nobreak
    \vskip 40\p@
  }}
}
\makeatother

%------------------------------------------------------------------------
% Paquetes de cógido
\usepackage{minted}
\renewcommand\listingscaption{Código fuente}

\usepackage{fancyvrb}
% Personaliza el tamaño de los números de línea
\renewcommand{\theFancyVerbLine}{\small\arabic{FancyVerbLine}}

% Estilo para C++
\newminted{cpp}{
    frame=lines,
    framesep=2mm,
    baselinestretch=1.2,
    linenos,
    escapeinside=||
}

% para minted
\definecolor{LightGray}{rgb}{0.95,0.95,0.92}
\setminted{
    linenos=true,
    stepnumber=5,
    numberfirstline=true,
    autogobble,
    breaklines=true,
    breakautoindent=true,
    breaksymbolleft=,
    breaksymbolright=,
    breaksymbolindentleft=0pt,
    breaksymbolindentright=0pt,
    breaksymbolsepleft=0pt,
    breaksymbolsepright=0pt,
    fontsize=\footnotesize,
    bgcolor=LightGray,
    numbersep=10pt
}


\usepackage{listings} % Para incluir código desde un archivo

\renewcommand\lstlistingname{Código Fuente}
\renewcommand\lstlistlistingname{Índice de Códigos Fuente}

% Definir colores
\definecolor{vscodepurple}{rgb}{0.5,0,0.5}
\definecolor{vscodeblue}{rgb}{0,0,0.8}
\definecolor{vscodegreen}{rgb}{0,0.5,0}
\definecolor{vscodegray}{rgb}{0.5,0.5,0.5}
\definecolor{vscodebackground}{rgb}{0.97,0.97,0.97}
\definecolor{vscodelightgray}{rgb}{0.9,0.9,0.9}

% Configuración para el estilo de C similar a VSCode
\lstdefinestyle{vscode_C}{
  backgroundcolor=\color{vscodebackground},
  commentstyle=\color{vscodegreen},
  keywordstyle=\color{vscodeblue},
  numberstyle=\tiny\color{vscodegray},
  stringstyle=\color{vscodepurple},
  basicstyle=\scriptsize\ttfamily,
  breakatwhitespace=false,
  breaklines=true,
  captionpos=b,
  keepspaces=true,
  numbers=left,
  numbersep=5pt,
  showspaces=false,
  showstringspaces=false,
  showtabs=false,
  tabsize=2,
  frame=tb,
  framerule=0pt,
  aboveskip=10pt,
  belowskip=10pt,
  xleftmargin=10pt,
  xrightmargin=10pt,
  framexleftmargin=10pt,
  framexrightmargin=10pt,
  framesep=0pt,
  rulecolor=\color{vscodelightgray},
  backgroundcolor=\color{vscodebackground},
}

%------------------------------------------------------------------------

% Comandos definidos
\newcommand{\bb}[1]{\mathbb{#1}}
\newcommand{\cc}[1]{\mathcal{#1}}

% I prefer the slanted \leq
\let\oldleq\leq % save them in case they're every wanted
\let\oldgeq\geq
\renewcommand{\leq}{\leqslant}
\renewcommand{\geq}{\geqslant}

% Si y solo si
\newcommand{\sii}{\iff}

% Letras griegas
\newcommand{\eps}{\epsilon}
\newcommand{\veps}{\varepsilon}
\newcommand{\lm}{\lambda}

\newcommand{\ol}{\overline}
\newcommand{\ul}{\underline}
\newcommand{\wt}{\widetilde}
\newcommand{\wh}{\widehat}

\let\oldvec\vec
\renewcommand{\vec}{\overrightarrow}

% Derivadas parciales
\newcommand{\del}[2]{\frac{\partial #1}{\partial #2}}
\newcommand{\Del}[3]{\frac{\partial^{#1} #2}{\partial #3^{#1}}}
\newcommand{\deld}[2]{\dfrac{\partial #1}{\partial #2}}
\newcommand{\Deld}[3]{\dfrac{\partial^{#1} #2}{\partial #3^{#1}}}


\newcommand{\AstIg}{\stackrel{(\ast)}{=}}
\newcommand{\Hop}{\stackrel{L'H\hat{o}pital}{=}}

\newcommand{\red}[1]{{\color{red}#1}} % Para integrales, destacar los cambios.

% Método de integración
\newcommand{\MetInt}[2]{
    \left[\begin{array}{c}
        #1 \\ #2
    \end{array}\right]
}

% Declarar aplicaciones
% 1. Nombre aplicación
% 2. Dominio
% 3. Codominio
% 4. Variable
% 5. Imagen de la variable
\newcommand{\Func}[5]{
    \begin{equation*}
        \begin{array}{rrll}
            #1:& #2 & \longrightarrow & #3\\
               & #4 & \longmapsto & #5
        \end{array}
    \end{equation*}
}

%------------------------------------------------------------------------


\newcommand{\R}{\mathbb{R}}
\newcommand{\prodescalar}[2]{\langle #1, #2 \rangle}
\newcommand{\ortogonal}[1]{#1^{\perp}}

\begin{document}

    % 1. Foto de fondo
    % 2. Título
    % 3. Encabezado Izquierdo
    % 4. Color de fondo
    % 5. Coord x del titulo
    % 6. Coord y del titulo
    % 7. Fecha

    
    % 1. Foto de fondo
% 2. Título
% 3. Encabezado Izquierdo
% 4. Color de fondo
% 5. Coord x del titulo
% 6. Coord y del titulo
% 7. Fecha

\newcommand{\portada}[7]{

    \portadaBase{#1}{#2}{#3}{#4}{#5}{#6}{#7}
    \portadaBook{#1}{#2}{#3}{#4}{#5}{#6}{#7}
}

\newcommand{\portadaExamen}[7]{

    \portadaBase{#1}{#2}{#3}{#4}{#5}{#6}{#7}
    \portadaArticle{#1}{#2}{#3}{#4}{#5}{#6}{#7}
}




\newcommand{\portadaBase}[7]{

    % Tiene la portada principal y la licencia Creative Commons
    
    % 1. Foto de fondo
    % 2. Título
    % 3. Encabezado Izquierdo
    % 4. Color de fondo
    % 5. Coord x del titulo
    % 6. Coord y del titulo
    % 7. Fecha
    
    
    \thispagestyle{empty}               % Sin encabezado ni pie de página
    \newgeometry{margin=0cm}        % Márgenes nulos para la primera página
    
    
    % Encabezado
    \fancyhead[L]{\helv #3}
    \fancyhead[R]{\helv \nouppercase{\leftmark}}
    
    
    \pagecolor{#4}        % Color de fondo para la portada
    
    \begin{figure}[p]
        \centering
        \transparent{0.3}           % Opacidad del 30% para la imagen
        
        \includegraphics[width=\paperwidth, keepaspectratio]{assets/#1}
    
        \begin{tikzpicture}[remember picture, overlay]
            \node[anchor=north west, text=white, opacity=1, font=\fontsize{60}{90}\selectfont\bfseries\sffamily, align=left] at (#5, #6) {#2};
            
            \node[anchor=south east, text=white, opacity=1, font=\fontsize{12}{18}\selectfont\sffamily, align=right] at (9.7, 3) {\textbf{\href{https://losdeldgiim.github.io/}{Los Del DGIIM}}};
            
            \node[anchor=south east, text=white, opacity=1, font=\fontsize{12}{15}\selectfont\sffamily, align=right] at (9.7, 1.8) {Doble Grado en Ingeniería Informática y Matemáticas\\Universidad de Granada};
        \end{tikzpicture}
    \end{figure}
    
    
    \restoregeometry        % Restaurar márgenes normales para las páginas subsiguientes
    \pagecolor{white}       % Restaurar el color de página
    
    
    \newpage
    \thispagestyle{empty}               % Sin encabezado ni pie de página
    \begin{tikzpicture}[remember picture, overlay]
        \node[anchor=south west, inner sep=3cm] at (current page.south west) {
            \begin{minipage}{0.5\paperwidth}
                \href{https://creativecommons.org/licenses/by-nc-nd/4.0/}{
                    \includegraphics[height=2cm]{assets/Licencia.png}
                }\vspace{1cm}\\
                Esta obra está bajo una
                \href{https://creativecommons.org/licenses/by-nc-nd/4.0/}{
                    Licencia Creative Commons Atribución-NoComercial-SinDerivadas 4.0 Internacional (CC BY-NC-ND 4.0).
                }\\
    
                Eres libre de compartir y redistribuir el contenido de esta obra en cualquier medio o formato, siempre y cuando des el crédito adecuado a los autores originales y no persigas fines comerciales. 
            \end{minipage}
        };
    \end{tikzpicture}
    
    
    
    % 1. Foto de fondo
    % 2. Título
    % 3. Encabezado Izquierdo
    % 4. Color de fondo
    % 5. Coord x del titulo
    % 6. Coord y del titulo
    % 7. Fecha


}


\newcommand{\portadaBook}[7]{

    % 1. Foto de fondo
    % 2. Título
    % 3. Encabezado Izquierdo
    % 4. Color de fondo
    % 5. Coord x del titulo
    % 6. Coord y del titulo
    % 7. Fecha

    % Personaliza el formato del título
    \pretitle{\begin{center}\bfseries\fontsize{42}{56}\selectfont}
    \posttitle{\par\end{center}\vspace{2em}}
    
    % Personaliza el formato del autor
    \preauthor{\begin{center}\Large}
    \postauthor{\par\end{center}\vfill}
    
    % Personaliza el formato de la fecha
    \predate{\begin{center}\huge}
    \postdate{\par\end{center}\vspace{2em}}
    
    \title{#2}
    \author{\href{https://losdeldgiim.github.io/}{Los Del DGIIM}}
    \date{Granada, #7}
    \maketitle
    
    \tableofcontents
}




\newcommand{\portadaArticle}[7]{

    % 1. Foto de fondo
    % 2. Título
    % 3. Encabezado Izquierdo
    % 4. Color de fondo
    % 5. Coord x del titulo
    % 6. Coord y del titulo
    % 7. Fecha

    % Personaliza el formato del título
    \pretitle{\begin{center}\bfseries\fontsize{42}{56}\selectfont}
    \posttitle{\par\end{center}\vspace{2em}}
    
    % Personaliza el formato del autor
    \preauthor{\begin{center}\Large}
    \postauthor{\par\end{center}\vspace{3em}}
    
    % Personaliza el formato de la fecha
    \predate{\begin{center}\huge}
    \postdate{\par\end{center}\vspace{5em}}
    
    \title{#2}
    \author{\href{https://losdeldgiim.github.io/}{Los Del DGIIM}}
    \date{Granada, #7}
    \thispagestyle{empty}               % Sin encabezado ni pie de página
    \maketitle
    \vfill
}
    \portadaExamen{ffccA4.jpg}{Mecánica Celeste\\Examen II}{Mecánica Celeste. Examen II}{MidnightBlue}{-8}{28}{2025}{José Manuel Sánchez Varbas}

    \begin{description}
        \item[Asignatura] Mecánica Celeste.
        \item[Curso Académico] 2025-26.
        \item[Grado] Grado en Matemáticas.
        \item[Grupo] A.
        \item[Profesor] Margarita Arias López.
        \item[Descripción] Primer Parcial.
        \item[Fecha] 6 de Noviembre de 2025.
        \item[Duración] 1 hora y 30 minutos.
    
    \end{description}
    \newpage


    % ------------------------------------

    \begin{ejercicio}[4 puntos]
        Sea $x : (\alpha, \omega) \to \R^3 \setminus \{0\}$ una solución de la ecuación 
        $$\ddot{x} = - \dfrac{1}{|x|^3} x$$
        Determina de forma razonada si son verdaderas o falsas las siguientes afirmaciones:
        \begin{itemize}
            \item[a)] Si $x(0) = (1, 0, 1)$ y $\dot{x}(0) = (3, 0, 0)$, $x(t)$ está definida
            para todo $t \in \R$ y pertenece al plano $\{x_2 = 0\}$ para todo $t \in \R$.
            \item[b)] Si $x(0) = (1, 0, 1)$ y $\dot{x}(0) = (3, 0, 0)$, la órbita $x(t)$
            está acotada.
            \item[c)] Si $x(0) = (1, 0, 1)$ y $\dot{x}(0) = (3, 0, 0)$, el área que barre el segmento que une $x(t)$ con el origen varía a velocidad $\sqrt{2}/2$.
            \item[d)] Si $x(0) = (1, 0, 1)$ y $\dot{x}(0) = (-3, 0, -3)$, la órbita recorre la semi-recta $\{((1,0,1)s : s>0)\}$, $\omega$ es finito y $x(t)$ se aproxima al origen cuando $t \to \omega$. 
        \end{itemize}
    \end{ejercicio}

    \begin{ejercicio}[3 puntos]
        Desde un observatorio astronómico se está controlando un meteorito que se aproxima a Marte siguiendo la órbita $$\sqrt{x^2 + y^2} + x = k$$
        en un sistema de referencia sobre el plano del movimiento con origen en el centro de masas de Marte y unidades en miles de kilómetros. El meteorito se encuentra aún muy lejos y no se puede precisar con exactitud el valor de $k>0$. Se pide:
        \begin{itemize}
            \item[a)] Determinar el tipo de órbita que describe y la energía total del movimiento.
            \item[b)] Calcular la distancia mínima del meteorito a Marte y su velocidad en el punto de mínima distancia en términos del valor de $k$.\footnote{los resultados se pueden dejar en términos de $\mu$ también.}
            \item[c)] Si se supone que el radio de Marte es aproximadamente de $3,5$ unidades, ¿tienen los marcianos que empezar a preparar las maletas?\footnote{Nos preguntan si se va a chocar el meteorito con el planeta.}   
        \end{itemize} 
    \end{ejercicio}

    \begin{ejercicio}[3 puntos]
        Un satélite describe una órbita circular en sentido anti-horario a altura $1$ alrededor de un planeta de masa $1/G$ y radio $3$.
        \begin{itemize}
            \item[a)] Determina el módulo de la velocidad a la que orbita en cada instante y el tiempo que tarda el satélite en dar una vuelta completa a su órbita.
            \item[b)] En el instante en que el satélite pasa por la vertical del polo norte del planeta se aumenta su velocidad en un $20\%$ manteniendo la misma dirección. Determina cómo se modifica su órbita.  
        \end{itemize}
    \end{ejercicio}

    \newpage

    \setcounter{ejercicio}{0}

    \begin{ejercicio}[4 puntos]
        Sea $x : (\alpha, \omega) \to \R^3 \setminus \{0\}$ una solución de la ecuación 
        $$\ddot{x} = - \dfrac{1}{|x|^3} x$$
        Determina de forma razonada si son verdaderas o falsas las siguientes afirmaciones:
        \begin{itemize}
            \item[a)] Si $x(0) = (1, 0, 1)$ y $\dot{x}(0) = (3, 0, 0)$, $x(t)$ está definida
            para todo $t \in \R$ y pertenece al plano $\{x_2 = 0\}$ para todo $t \in \R$. \\

            Dado que estamos en un c.f.c. (campo de fuerzas centrales), sabemos que el momento angular será constante.
            Podemos obtener $$c(0) = x(0) \land \dot{x}(0) = \begin{vmatrix}
                1 & 0 & 1 \\
                3 & 0 & 0 \\
                i & j & k
            \end{vmatrix} = (0, -3, 0) \Longrightarrow |c| \neq 0.$$

            Por la Primera Ley de Kepler, como estamos en un campo newtoniano, el movimiento estará contenido en una de las tres siguientes cónicas; elipse, parábola o hipérbola, y el intervalo de definición maximal del movimiento será $(\alpha, \omega) = \R$.
            También sabemos por la clasificación de movimientos según el módulo del momento angular, que $x(t) \in \Pi \quad \forall t \in \R$, con $\Pi = \ortogonal{\{c\}}$, es decir, el plano que tiene vector normal $c$, y pasa por el origen. Dicho plano verifica la ecuación
            $$Ax_1 + Bx_2 + Cx_3 = 0 \stackrel{(x_1,x_2,x_3) = (0,-3,0)}{\Longrightarrow} -3x_2 = 0 \stackrel{-3 \neq 0}{\Longrightarrow} x_2 = 0$$
            por lo tanto, la afirmación es \boxed{\text{verdadera.}} 

            \item[b)] Si $x(0) = (1, 0, 1)$ y $\dot{x}(0) = (3, 0, 0)$, la órbita $x(t)$
            está acotada. \\

            Por el apartado anterior sabemos que $|c| \neq 0$, y nuevamente la órbita $x(t)$ quedará contenida en una elipse, parábola o hipérbola. 
            La única manera de que la órbita esté acotada es que $h<0$. De lo contrario, la cónica sería una parábola o hipérbola, ambas con trayectorias abiertas, y, por tanto, no están acotadas.
            Obtenemos entonces la energía total

            $$h = \dfrac{1}{2} |\dot{x}(0)|^2 - \dfrac{1}{|x(0)|} = \dfrac{1}{2} \cdot 3^2 - \dfrac{1}{\sqrt{2}} = \dfrac{9 - \sqrt{2}}{2} \approx 3.79 \geqslant 0$$

            y como la trayectoria no es elíptica (de hecho es hiperbólica), no estará acotada, y la afirmación es \boxed{\text{falsa.}}

            \newpage

            \item[c)] Si $x(0) = (1, 0, 1)$ y $\dot{x}(0) = (3, 0, 0)$, el área que barre el segmento que une $x(t)$ con el origen varía a velocidad $\sqrt{2}/2$. \\
            
            Sabemos, por la Segunda Ley de Kepler, que la velocidad areolar tiene por fórmula
            $$v_{\text{areolar}} = \dfrac{|c|}{2}$$
            así como que por el primer apartado $c = (0,-3,0) \Longrightarrow |c|=3$, y sustituyendo vemos que
            $$v_{\text{areolar}} = \dfrac{3}{2} \neq \dfrac{\sqrt{2}}{2}$$
            luego la afirmación es \boxed{\text{falsa.}}

            \item[d)] Si $x(0) = (1, 0, 1)$ y $\dot{x}(0) = (-3, 0, -3)$, la órbita recorre la semi-recta $\{((1,0,1)s : s>0)\}$, $\omega$ es finito y $x(t)$ se aproxima al origen cuando $t \to \omega$. \\
            
            Primero, vemos que $\dot{x}(0) = -3 x(0)$, es decir, ambos vectores son paralelos, luego $\widehat{x(0),\dot{x}(0)} = 0$, y $|c| = 0$. Por la teoría de clasificación de movimientos según el módulo del momento angular,
            sabemos que el movimiento es rectilíneo, y $x(t) = r(t) v$, con $v \equiv x(t) / |x(t)|$, $|v|=1$, y $x(t) \in \R_{+}v$, es decir, el movimiento se da a lo largo de la semirrecta con extremo inferior en el origen, y la dirección y sentido de $v$.
            Como
            $$v = \dfrac{x(t)}{|x(t)|} = \dfrac{x(0)}{|x(0)|} = \dfrac{(1,0,1)}{\sqrt{2}} \parallel (1,0,1)$$
            la primera parte del enunciado es cierta. Ahora, hay dos opciones para que el resto sea cierto, y es que $h<0$, con lo cual habría un cambio de monotonía en $r$ y se cumplirían ambas cosas, o que $h \geqslant 0$, y $\dot{r} < 0$, ubicándonos
            entonces en el caso en que $-\infty = \alpha < \omega < +\infty$, y $\lim\limits_{t \to \omega} r(t) = 0$. La primera opción se descarta viendo el signo de la energía total

            $$h = \dfrac{1}{2} |\dot{x}(0)|^2 - \dfrac{1}{|x(0)|} = \dfrac{1}{2} \cdot (3 \sqrt{2})^2 - \dfrac{1}{\sqrt{2}} = \dfrac{18 - \sqrt{2}}{2} \approx 8.29 \geqslant 0.$$

            Por teoría, la energía total en este tipo de movimiento también verifica $$h = \dfrac{1}{2} |\dot{r}(t)|^2 - \dfrac{1}{|r(t)|}$$
            y usando que $h > 0$
            $$h = \dfrac{1}{2} |\dot{r}(t)|^2 - \dfrac{1}{|r(t)|} > 0 \iff \dfrac{1}{2} |\dot{r}(t)|^2 > \dfrac{1}{|r(t)|} \iff |\dot{r}(t)| > \sqrt{\dfrac{2}{|r(t)|}}$$
            como $r(t) > 0 \Longrightarrow |r(t)| > 0$ y $\sqrt{\frac{2}{|r(t)|}} > 0$, luego 
            $$|\dot{r}(t)| > \sqrt{\dfrac{2}{|r(t)|}} > 0$$
            
            En particular, $\dot{r}(t) \neq 0$ para cualquier $t>0$. Como $r$ es una función continua (por ser solución), entonces $r$ es estrictamente monótona, con lo que basta obtener $\dot{r}(t_0)$ para algún $t_0 > 0$ para determinar el crecimiento o decrecimiento estricto de $r$.
            Como $$\dot{r}(t) \stackrel{|v|=1}{=} \dot{r}(t) \prodescalar{v}{v} = \prodescalar{\dot{r}(t) v}{v} \stackrel{\dot{x}(t) = \dot{r}(t) v}{=} \prodescalar{\dot{x}(t)}{v}$$ en particular, para $t=0$
            $$\dot{r}(0) = \prodescalar{\dot{x}(0)}{v} = -3 \cdot \dfrac{1}{\sqrt{2}} + 0 \cdot \dfrac{0}{\sqrt{2}} -3 \cdot \dfrac{1}{\sqrt{2}} = \dfrac{-6}{\sqrt{2}} = - 3 \sqrt{2} < 0.$$

            Concluimos finalmente que $\dot{r} < 0$, luego, por lo explicado en el segundo caso de la distinción que hemos hecho, la afirmación es \boxed{\text{verdadera.}}

        \end{itemize}
    \end{ejercicio}

    \newpage

    \begin{ejercicio}[3 puntos]
        Desde un observatorio astronómico se está controlando un meteorito que se aproxima a Marte siguiendo la órbita $$\sqrt{x^2 + y^2} + x = k$$
        en un sistema de referencia sobre el plano del movimiento con origen en el centro de masas de Marte y unidades en miles de kilómetros. El meteorito se encuentra aún muy lejos y no se puede precisar con exactitud el valor de $k>0$. Se pide:
        \begin{itemize}
            \item[a)] Determinar el tipo de órbita que describe y la energía total del movimiento. \\
            
            Para ello, basta determinar el vector de excentricidad, por comparación directa de la ecuación de una cónica con foco en el origen $|x| + \prodescalar{e}{x} = k$, con la dada $\sqrt{x^2 + y^2} + x = k$. Por un lado,
            $|x| = \sqrt{x^2 + y^2}$, luego necesariamente será $\prodescalar{e}{x} = x \iff e = (1,0)$. Así, $\varepsilon = |e| = 1$, y estamos ante un \boxed{\text{movimiento parabólico.}} \\ 
            
            Para la energía total, podríamos calcularla como hemos hecho en el ejercicio anterior,
            o usar la relación $$2h|c|^2 = \mu^2 (\varepsilon^2 - 1) \stackrel{\varepsilon^2 = 1}{=} 0 \stackrel{|c| \neq 0}{\iff} h = 0$$
            y concluir que la energía total del movimiento es $0$ (hemos usado que el movimiento es parabólico, luego $|c| \neq 0$).

            \item[b)] Calcular la distancia mínima del meteorito a Marte y su velocidad en el punto de mínima distancia en términos del valor de $k$.\footnote{los resultados se pueden dejar en términos de $\mu$ también.} \\
            
            Usando la expresión en polares de $|x| + \prodescalar{e}{x} = k$ $$r = \dfrac{k}{1 + \varepsilon \cos(\theta - \omega)}$$
            como $e = \varepsilon (\cos \omega, \sen \omega)$ y $\varepsilon = 1$, por igualación directa tenemos que $e = (1,0) = (\cos \omega, \sen \omega) \iff \cos \omega = 1 \land \sen \omega = 0 \iff \omega = 0$ (salvo múltiplo entero de $2 \pi$). \\

            La ecuación entonces de la parábola es 
            $$r = \dfrac{k}{1 + \cos(\theta)}$$

            y la distancia mínima del meteorito a Marte se obtiene cuando el denominador es máximo, es decir, $\cos(\theta) = 1$, y tal punto es $$r_{\min} = \dfrac{k}{1 + 1} = \dfrac{k}{2} \text{ m}$$ \\

            Por la definición de energía total $$h = \dfrac{1}{2} |\dot{x}(t)|^2 - \dfrac{1}{|x(t)|}$$
            y por ser el movimiento parabólico hemos visto que $h=0$, luego, juntando ambas cosas
            $$0 = \dfrac{1}{2} |\dot{x}(t)|^2 - \dfrac{1}{|x(t)|} \iff \dfrac{1}{2} |\dot{x}(t)|^2 = \dfrac{1}{|x(t)|} \iff |\dot{x}(t)| = \sqrt{\dfrac{2}{|x(t)|}}$$
            y sustituyendo $|x(t)|$ por el periastro $r_{\min}$, obtenemos dicha velocidad, suponiendo que $t_*$ es el instante de tiempo en que el meteorito se encuentra a la mínima distancia
            $$|\dot{x}(t_*)| = \sqrt{\dfrac{2}{\left(\dfrac{k}{2} \right)}} = \sqrt{\dfrac{4}{k}} = \dfrac{2}{\sqrt{k}} = \dfrac{2 \sqrt{k}}{k} \text{ m} / \text{s}$$
            y como $k>0$, dicho valor tiene sentido (usamos las unidades del SI, luego $[k] = [\text{m}]$)
            \item[c)] Si se supone que el radio de Marte es aproximadamente de $3,5$ unidades, ¿tienen los marcianos que empezar a preparar las maletas?\footnote{Nos preguntan si se va a chocar el meteorito con el planeta.} \\
            
            Chocarán en el caso de que $r_{\min} \leqslant R$, con $R$ el radio de marte. Por lo tanto, planteamos la inecuación sustituyendo ambos valores.

            $$\dfrac{k}{2} \text{ m} \leqslant 3,5 \cdot 10^3 \text{ \cancel{km}} \cdot \dfrac{10^3 \text{ m}}{1 \text{ \cancel{km}}} \iff k \text{ m} \leqslant 7 \cdot 10^6 \text{ m}$$

            Así pues \\

            \boxed{\text{si $k \leqslant 7 \cdot 10^6 \text{ m}$, el meteorito chocará con el planeta. De lo contrario, no habrá colisión.}}

        \end{itemize} 
    \end{ejercicio}

    \newpage

    \begin{ejercicio}[3 puntos]
        Un satélite describe una órbita circular en sentido anti-horario a altura $1$ alrededor de un planeta de masa $1/G$ y radio $3$. \\

        La situación es la que se muestra en la Figura \ref{fig:ej3}, con $R = 3$ m, y masa $M = 1 / G$ kg. 

        \begin{figure}[H]
            \begin{center}

            \tikzset{every picture/.style={line width=0.75pt}} %set default line width to 0.75pt        

            \begin{tikzpicture}[x=0.75pt,y=0.75pt,yscale=-1,xscale=1]
            %uncomment if require: \path (0,505); %set diagram left start at 0, and has height of 505

            %Shape: Axis 2D [id:dp9702769508374955] 
            \draw  (95,170.82) -- (480.5,170.82)(269.5,9.49) -- (269.5,307) (473.5,165.82) -- (480.5,170.82) -- (473.5,175.82) (264.5,16.49) -- (269.5,9.49) -- (274.5,16.49)  ;
            %Shape: Circle [id:dp24556951579602282] 
            \draw   (179.76,170.69) .. controls (179.76,121.06) and (219.99,80.82) .. (269.63,80.82) .. controls (319.26,80.82) and (359.5,121.06) .. (359.5,170.69) .. controls (359.5,220.33) and (319.26,260.57) .. (269.63,260.57) .. controls (219.99,260.57) and (179.76,220.33) .. (179.76,170.69) -- cycle ;
            %Straight Lines [id:da9841104437775205] 
            \draw    (269.43,170.55) -- (310.72,93.92) ;
            \draw [shift={(311.67,92.16)}, rotate = 118.32] [color={rgb, 255:red, 0; green, 0; blue, 0 }  ][line width=0.75]    (10.93,-3.29) .. controls (6.95,-1.4) and (3.31,-0.3) .. (0,0) .. controls (3.31,0.3) and (6.95,1.4) .. (10.93,3.29)   ;
            %Shape: Brace [id:dp8708030275487525] 
            \draw   (269,169.57) .. controls (273.1,171.8) and (276.26,170.86) .. (278.49,166.76) -- (293.42,139.23) .. controls (296.6,133.37) and (300.24,131.55) .. (304.35,133.78) .. controls (300.24,131.55) and (299.78,127.51) .. (302.96,121.65)(301.53,124.29) -- (313.82,101.65) .. controls (316.04,97.54) and (315.1,94.38) .. (311,92.16) ;
            %Straight Lines [id:da1878531025359833] 
            \draw  [dash pattern={on 0.84pt off 2.51pt}]  (273.33,50.16) -- (273.33,80.82) ;
            %Straight Lines [id:da12752949608292485] 
            \draw    (270,50.16) -- (290.67,50.16) ;
            %Shape: Brace [id:dp86030864237956] 
            \draw   (270.33,51.16) .. controls (266.23,51.11) and (264.16,53.14) .. (264.12,57.24) -- (264.12,57.24) .. controls (264.05,63.09) and (261.97,66) .. (257.87,65.95) .. controls (261.97,66) and (263.99,68.95) .. (263.92,74.8)(263.95,72.17) -- (263.92,74.8) .. controls (263.87,78.9) and (265.9,80.97) .. (270,81.02) ;
            %Straight Lines [id:da05535381107534876] 
            \draw    (270,80.82) -- (290.67,80.82) ;
            %Shape: Circle [id:dp5674138767177755] 
            \draw   (149.58,170.55) .. controls (149.58,104.36) and (203.24,50.7) .. (269.43,50.7) .. controls (335.62,50.7) and (389.27,104.36) .. (389.27,170.55) .. controls (389.27,236.74) and (335.62,290.4) .. (269.43,290.4) .. controls (203.24,290.4) and (149.58,236.74) .. (149.58,170.55) -- cycle ;
            %Shape: Circle [id:dp7660940700502149] 
            \draw  [color={rgb, 255:red, 9; green, 0; blue, 255 }  ,draw opacity=1 ][fill={rgb, 255:red, 0; green, 237; blue, 255 }  ,fill opacity=1 ] (325.33,71.41) .. controls (325.33,64.46) and (330.97,58.82) .. (337.92,58.82) .. controls (344.87,58.82) and (350.51,64.46) .. (350.51,71.41) .. controls (350.51,78.36) and (344.87,84) .. (337.92,84) .. controls (330.97,84) and (325.33,78.36) .. (325.33,71.41) -- cycle ;
            %Curve Lines [id:da9073228517665418] 
            \draw    (375,89.82) .. controls (376.32,81.9) and (375.03,56.34) .. (330.37,39.66) ;
            \draw [shift={(329,39.16)}, rotate = 19.92] [color={rgb, 255:red, 0; green, 0; blue, 0 }  ][line width=0.75]    (10.93,-3.29) .. controls (6.95,-1.4) and (3.31,-0.3) .. (0,0) .. controls (3.31,0.3) and (6.95,1.4) .. (10.93,3.29)   ;

            % Text Node
            \draw (307.7,126.27) node [anchor=north west][inner sep=0.75pt]   [align=left] {$\displaystyle R$};
            % Text Node
            \draw (245.33,59.67) node [anchor=north west][inner sep=0.75pt]   [align=left] {$\displaystyle 1$};


            \end{tikzpicture}

            \end{center}
            \caption{Esquema Ejercicio 3.}
            \label{fig:ej3}
        \end{figure}

        \begin{itemize}
            \item[a)] Determina el módulo de la velocidad a la que orbita en cada instante y el tiempo que tarda el satélite en dar una vuelta completa a su órbita. \\
            
            En primer lugar, el radio de órbita del satélite será $r = R + 1 = 3 + 1 = 4$ \text{m}. Ahora, consideramos el campo gravitatorio newtoniano dado por la ecuación $$\ddot{x} = - \dfrac{\mu}{|x|^3} x$$

            como $M = 1/G \Longrightarrow \mu = GM = 1 \text{ m}^3 \text{/s}^2$ y la ecuación resulta en

            $$\ddot{x} = - \dfrac{1}{|x|^3} x$$

            Por teoría, sabemos que $x_r(t) = r (\cos(\omega t), \sen (\omega t))$ es solución de esta ecuación si y solo si $$|\omega| = \dfrac{\sqrt{\mu}}{r^{3/2}} = \dfrac{1}{r^{3/2}}$$

            Por la Física de Bachillerato también sabemos que la velocidad lineal es la angular por el radio, es decir, $$v = |\omega| r = \dfrac{1}{r^{3/2}} r = \dfrac{1}{\sqrt{r}} = \dfrac{\sqrt{r}}{r} \text{ m/s}$$

            Como $r=4$ \text{m}, entonces el módulo de la velocidad en cada instante es $$\boxed{v = \dfrac{\sqrt{4}}{4} = \dfrac{2}{4} = \dfrac{1}{2} \text{ m/s}}$$

            El tiempo que tarda un satélite en dar una vuelta completa a su órbita no es más que el periodo, que sabemos que se relaciona con la velocidad angular por medio de $$\boxed{p = \frac{2 \pi}{|\omega|} \text{s} = 2 \pi r^{3/2} = 2 \pi 4^{3/2} = 2 \pi 8 = 16 \pi \text{ s}}$$

            Por lo tanto
            \begin{center}
                \boxed{
                \begin{aligned}
                &\text{El módulo de la velocidad a la que orbita en cada instante } \\[4pt]
                &\text{el satélite es $v$ = 0.5\ \text{m/s} y su periodo es } p = 16\pi\ \text{s.}
                \end{aligned}
                }
            \end{center}

            \item[b)] En el instante en que el satélite pasa por la vertical del polo norte del planeta se aumenta su velocidad en un $20\%$ manteniendo la misma dirección. Determina cómo se modifica su órbita. \\
            
            La situación es la que se muestra en la Figura \ref{fig:ej3b}, denotando con subíndice $0$ y $1$ a las magnitudes antes y después del impulso, respectivamente.

            \begin{figure}[H]
                \begin{center}

                \tikzset{every picture/.style={line width=0.75pt}} %set default line width to 0.75pt        

                \begin{tikzpicture}[x=0.75pt,y=0.75pt,yscale=-1,xscale=1]
                %uncomment if require: \path (0,505); %set diagram left start at 0, and has height of 505

                %Shape: Axis 2D [id:dp9702769508374955] 
                \draw  (7,150.65) -- (239.2,150.65)(120.05,16.49) -- (120.05,263.5) (232.2,145.65) -- (239.2,150.65) -- (232.2,155.65) (115.05,23.49) -- (120.05,16.49) -- (125.05,23.49)  ;
                %Shape: Ellipse [id:dp24556951579602282] 
                \draw   (44.89,150.33) .. controls (44.89,109.12) and (79.03,75.71) .. (121.13,75.71) .. controls (163.23,75.71) and (197.36,109.12) .. (197.36,150.33) .. controls (197.36,191.54) and (163.23,224.95) .. (121.13,224.95) .. controls (79.03,224.95) and (44.89,191.54) .. (44.89,150.33) -- cycle ;
                %Straight Lines [id:da9841104437775205] 
                \draw    (120.96,150.21) -- (155.82,86.88) ;
                \draw [shift={(156.79,85.12)}, rotate = 118.83] [color={rgb, 255:red, 0; green, 0; blue, 0 }  ][line width=0.75]    (10.93,-3.29) .. controls (6.95,-1.4) and (3.31,-0.3) .. (0,0) .. controls (3.31,0.3) and (6.95,1.4) .. (10.93,3.29)   ;
                %Shape: Brace [id:dp8708030275487525] 
                \draw   (120.6,149.4) .. controls (124.68,151.67) and (127.85,150.76) .. (130.11,146.67) -- (141.2,126.67) .. controls (144.43,120.84) and (148.09,119.05) .. (152.17,121.32) .. controls (148.09,119.05) and (147.67,115.01) .. (150.9,109.18)(149.44,111.8) -- (158.95,94.64) .. controls (161.22,90.56) and (160.31,87.39) .. (156.22,85.13) ;
                %Straight Lines [id:da1878531025359833] 
                \draw  [dash pattern={on 0.84pt off 2.51pt}]  (124.27,50.25) -- (124.27,75.71) ;
                %Straight Lines [id:da12752949608292485] 
                \draw    (121.44,50.25) -- (138.97,50.25) ;
                %Shape: Brace [id:dp86030864237956] 
                \draw   (121.73,51.08) .. controls (118.32,51.05) and (116.6,52.73) .. (116.56,56.13) -- (116.56,56.13) .. controls (116.51,60.99) and (114.78,63.4) .. (111.38,63.36) .. controls (114.78,63.4) and (116.45,65.85) .. (116.4,70.72)(116.42,68.53) -- (116.4,70.72) .. controls (116.36,74.12) and (118.04,75.84) .. (121.45,75.88) ;
                %Straight Lines [id:da05535381107534876] 
                \draw    (121.44,75.71) -- (138.97,75.71) ;
                %Shape: Ellipse [id:dp5674138767177755] 
                \draw   (19.3,150.21) .. controls (19.3,95.26) and (64.81,50.71) .. (120.96,50.71) .. controls (177.1,50.71) and (222.62,95.26) .. (222.62,150.21) .. controls (222.62,205.17) and (177.1,249.71) .. (120.96,249.71) .. controls (64.81,249.71) and (19.3,205.17) .. (19.3,150.21) -- cycle ;
                %Shape: Ellipse [id:dp7660940700502149] 
                \draw  [color={rgb, 255:red, 9; green, 0; blue, 255 }  ,draw opacity=1 ][fill={rgb, 255:red, 0; green, 237; blue, 255 }  ,fill opacity=1 ] (168.38,67.9) .. controls (168.38,62.13) and (173.16,57.45) .. (179.06,57.45) .. controls (184.96,57.45) and (189.74,62.13) .. (189.74,67.9) .. controls (189.74,73.67) and (184.96,78.35) .. (179.06,78.35) .. controls (173.16,78.35) and (168.38,73.67) .. (168.38,67.9) -- cycle ;
                %Curve Lines [id:da9073228517665418] 
                \draw    (210.51,83.19) .. controls (211.62,76.64) and (210.54,55.6) .. (173.22,41.75) ;
                \draw [shift={(171.49,41.12)}, rotate = 19.53] [color={rgb, 255:red, 0; green, 0; blue, 0 }  ][line width=0.75]    (10.93,-3.29) .. controls (6.95,-1.4) and (3.31,-0.3) .. (0,0) .. controls (3.31,0.3) and (6.95,1.4) .. (10.93,3.29)   ;
                %Straight Lines [id:da28380246289396105] 
                \draw    (233.38,80.45) -- (298.67,80.01) ;
                \draw [shift={(300.67,80)}, rotate = 179.62] [color={rgb, 255:red, 0; green, 0; blue, 0 }  ][line width=0.75]    (10.93,-3.29) .. controls (6.95,-1.4) and (3.31,-0.3) .. (0,0) .. controls (3.31,0.3) and (6.95,1.4) .. (10.93,3.29)   ;
                %Shape: Axis 2D [id:dp09933319899940374] 
                \draw  (291,150.65) -- (523.2,150.65)(404.05,16.49) -- (404.05,263.5) (516.2,145.65) -- (523.2,150.65) -- (516.2,155.65) (399.05,23.49) -- (404.05,16.49) -- (409.05,23.49)  ;
                %Shape: Ellipse [id:dp5523371017590837] 
                \draw   (328.89,150.33) .. controls (328.89,109.12) and (363.03,75.71) .. (405.13,75.71) .. controls (447.23,75.71) and (481.36,109.12) .. (481.36,150.33) .. controls (481.36,191.54) and (447.23,224.95) .. (405.13,224.95) .. controls (363.03,224.95) and (328.89,191.54) .. (328.89,150.33) -- cycle ;
                %Straight Lines [id:da9702148411626599] 
                \draw    (404.96,150.21) -- (439.82,86.88) ;
                \draw [shift={(440.79,85.12)}, rotate = 118.83] [color={rgb, 255:red, 0; green, 0; blue, 0 }  ][line width=0.75]    (10.93,-3.29) .. controls (6.95,-1.4) and (3.31,-0.3) .. (0,0) .. controls (3.31,0.3) and (6.95,1.4) .. (10.93,3.29)   ;
                %Shape: Brace [id:dp9873937062342023] 
                \draw   (404.6,149.4) .. controls (408.68,151.67) and (411.85,150.76) .. (414.11,146.67) -- (425.2,126.67) .. controls (428.43,120.84) and (432.09,119.05) .. (436.17,121.32) .. controls (432.09,119.05) and (431.67,115.01) .. (434.9,109.18)(433.44,111.8) -- (442.95,94.64) .. controls (445.22,90.56) and (444.31,87.39) .. (440.22,85.13) ;
                %Shape: Ellipse [id:dp8969952940786448] 
                \draw   (303.3,150.21) .. controls (303.3,95.26) and (348.81,50.71) .. (404.96,50.71) .. controls (461.1,50.71) and (506.62,95.26) .. (506.62,150.21) .. controls (506.62,205.17) and (461.1,249.71) .. (404.96,249.71) .. controls (348.81,249.71) and (303.3,205.17) .. (303.3,150.21) -- cycle ;
                %Shape: Ellipse [id:dp7187075666768598] 
                \draw  [color={rgb, 255:red, 9; green, 0; blue, 255 }  ,draw opacity=1 ][fill={rgb, 255:red, 0; green, 237; blue, 255 }  ,fill opacity=1 ] (394.28,50.71) .. controls (394.28,44.94) and (399.06,40.26) .. (404.96,40.26) .. controls (410.86,40.26) and (415.64,44.94) .. (415.64,50.71) .. controls (415.64,56.48) and (410.86,61.16) .. (404.96,61.16) .. controls (399.06,61.16) and (394.28,56.48) .. (394.28,50.71) -- cycle ;
                %Curve Lines [id:da48812672631064924] 
                \draw    (430.67,42) .. controls (421.94,32.3) and (396.27,32.01) .. (377.4,41.13) ;
                \draw [shift={(375.67,42)}, rotate = 332.24] [color={rgb, 255:red, 0; green, 0; blue, 0 }  ][line width=0.75]    (10.93,-3.29) .. controls (6.95,-1.4) and (3.31,-0.3) .. (0,0) .. controls (3.31,0.3) and (6.95,1.4) .. (10.93,3.29)   ;

                % Text Node
                \draw (152.36,111.92) node [anchor=north west][inner sep=0.75pt]   [align=left] {$\displaystyle R$};
                % Text Node
                \draw (99.69,56.62) node [anchor=north west][inner sep=0.75pt]   [align=left] {$\displaystyle 1$};
                % Text Node
                \draw (436.36,111.92) node [anchor=north west][inner sep=0.75pt]   [align=left] {$\displaystyle R$};
                % Text Node
                \draw (221,33) node [anchor=north west][inner sep=0.75pt]   [align=left] {$\displaystyle  \begin{array}{{>{\displaystyle}l}}
                \ \ +20\%\ v\\
                v_{1} \ =\ 1,2v_{0}
                \end{array}$};


                \end{tikzpicture}

                \caption{Esquema Ejercicio 3 Apartado b)}
                \label{fig:ej3b}
                \end{center}
            \end{figure}
            
            Para obtener el módulo del momento angular, primero recordamos que $$|c| = |x(t)||\dot{x}(t)|\sen (\widehat{x(t), \dot{x}(t)}).$$ Como estamos en un MCU, $x(t) = r$, y $\dot{x}(t) = v$, y la velocidad es perpendicular al movimiento, luego $\sen (\widehat{x(t), \dot{x}(t)}) = 1$.
            Antes del impulso
            $$v_0 = \dfrac{1}{2} \text{ m/s}$$
            $$h_0 = \dfrac{1}{2} v_0^2 - \dfrac{1}{r} = \dfrac{1}{2} \cdot \dfrac{1}{4} - \dfrac{1}{4} = - \dfrac{1}{8}$$
            $$|c_0| = rv_0 = 4 \cdot \dfrac{1}{2} = 2$$
            Después del impulso, el módulo de la velocidad se incrementa en un $20\%$, y la dirección se mantiene, luego
            $$v_1 = 1,2 v_0 = 1,2 \dfrac{1}{2} = 0,6 = \dfrac{3}{5}$$
            $$h_1 = \dfrac{1}{2} v_1^2 - \dfrac{1}{r} = \dfrac{1}{2} \cdot \dfrac{9}{25} - \dfrac{1}{4} = - \dfrac{7}{100} < 0$$
            $$|c_1| = rv_1 = 4 \cdot \dfrac{3}{5} = \dfrac{12}{5}$$

            Por la Primera Ley de Kepler, como $|c_1| \neq 0$, entonces el movimiento se da en una cónica, y como $h_1 < 0$, dicha cónica será una elipse. \\

            Podemos obtener las magnitudes y puntos notables que determinan la nueva trayectoria. Primero obtenemos el semieje mayor, usando las siguientes relaciones $$2h|c|^2 = \mu^2 (\varepsilon^2 - 1); \quad k = \dfrac{|c|^2}{\mu} \iff |c|^2 = \mu k; \quad a = \frac{k}{1-\varepsilon^2} \iff k = a(1-\varepsilon^2)$$

            Despejando $h$ de la primera, y sustituyendo $|c|^2$ y $k$ de la segunda y tercera, respectivamente, obtenemos

            $$h = \dfrac{\mu^{2} (\varepsilon^2 - 1)}{2 |c|^2} = \dfrac{\mu^{\cancel{2}} (\varepsilon^2 - 1)}{2 \cancel{\mu} k} = \dfrac{\mu (\varepsilon^2 - 1)}{2 a(1-\varepsilon^2)} = - \dfrac{\mu \cancel{(1 - \varepsilon^2)}}{2 a \cancel{(1-\varepsilon^2)}} \stackrel{(*)}{=} - \dfrac{\mu}{2a}$$
            donde en $(*)$ hemos usado que la trayectoria es elíptica, luego $$\varepsilon < 1 \Longrightarrow \varepsilon^2 < 1 \Longrightarrow \varepsilon^2 - 1 < 0 \Longrightarrow 1 - \varepsilon^2 > 0 \Longrightarrow 1 - \varepsilon^2 \neq 0$$
            El semieje mayor $a_1$ será 
            $$h_1 = -\dfrac{\mu}{2a_1} \iff 2a_1 = - \dfrac{\mu}{h_1} \iff a_1 = - \dfrac{\mu}{2h_1} = - \dfrac{1}{2 \cdot (-7/100)} = \dfrac{50}{7} \approx 7.14 \text{ m}$$
            La excentricidad de la órbita elíptica puede obtenerse despejando en $$2h|c|^2 = \mu^2 (\varepsilon^2 - 1)$$ como sigue
            $$2h|c|^2 = \mu^2 (\varepsilon^2 - 1) \iff 2h|c|^2 = \mu^2 \varepsilon^2 - \mu^2 \iff 2h|c|^2 + \mu^2 = \mu^2 \varepsilon^2 \iff $$
            $$\varepsilon^2 = \dfrac{2h|c|^2 + \mu^2}{\mu^2} \iff \varepsilon = \sqrt{\dfrac{2h|c|^2 + \mu^2}{\mu^2}}$$

            Tomando las magnitudes post-impulso

            $$\varepsilon_1 = \sqrt{\dfrac{2h_1|c_1|^2 + \mu^2}{\mu^2}} = \sqrt{\dfrac{2(-7/100)(12/5)^2 + 1}{1^2}} = \dfrac{11}{25} = 0.44$$

            Podemos obtener también el semieje menor, sustituyendo en la relación $$b_1 = a_1 \sqrt{1-\varepsilon_1^2} = (50/7) \sqrt{1-(11/25)^2} = \dfrac{12 \sqrt{14}}{7} \approx 6.4143 \text{ m}$$

            y el nuevo centro de la elipse

            $$|C_1| = \dfrac{\varepsilon_1 k}{1 - \varepsilon_1^2} = \dfrac{\varepsilon_1 a_1(1-\varepsilon_1^2)}{1 - \varepsilon_1^2} = \dfrac{(11/25)(50/7)(1-(11/25)^2)}{1-(11/25)^2} = \dfrac{22}{7} \approx 3.14 \text{ m}$$

            Por último, las distancias al periastro y apoastro se obtienen, respectivamente, como 

            $$r_{\min} = \dfrac{k}{1+\varepsilon} = \dfrac{a(1-\varepsilon^2)}{1 + \varepsilon} \quad r_{\max} = \dfrac{k}{1-\varepsilon} = \dfrac{a(1-\varepsilon^2)}{1 - \varepsilon}$$

            luego podemos calcularlas

            $$r_{\min_1} = \dfrac{a_1(1-\varepsilon_1^2)}{1 + \varepsilon_1} = \dfrac{(50/7)(1-(11/25)^2)}{1 + (11/25)} = 4 \text{ m}$$

            $$r_{\max_1} = \dfrac{a_1(1-\varepsilon_1^2)}{1 - \varepsilon_1} = \dfrac{(50/7)(1-(11/25)^2)}{1 - (11/25)} = \dfrac{72}{7} \approx 10.29 \text{ m}$$

            y el cambio de órbita se muestra en la Figura \ref{fig:ej3b2}

            \begin{figure}[H]
                \begin{center}

                    \tikzset{every picture/.style={line width=0.75pt}} %set default line width to 0.75pt        

                    \begin{tikzpicture}[x=0.75pt,y=0.75pt,yscale=-1,xscale=1]
                    %uncomment if require: \path (0,505); %set diagram left start at 0, and has height of 505

                    %Shape: Ellipse [id:dp13310176064229318] 
                    \draw   (240.6,150.02) .. controls (240.6,100.14) and (303.24,59.7) .. (380.51,59.7) .. controls (457.78,59.7) and (520.42,100.14) .. (520.42,150.02) .. controls (520.42,199.91) and (457.78,240.35) .. (380.51,240.35) .. controls (303.24,240.35) and (240.6,199.91) .. (240.6,150.02) -- cycle ;
                    %Shape: Axis 2D [id:dp09933319899940374] 
                    \draw  (3,150.79) -- (221.6,150.79)(109.43,23.49) -- (109.43,257.87) (214.6,145.79) -- (221.6,150.79) -- (214.6,155.79) (104.43,30.49) -- (109.43,23.49) -- (114.43,30.49)  ;
                    %Shape: Ellipse [id:dp5523371017590837] 
                    \draw   (38.67,150.49) .. controls (38.67,111.38) and (70.81,79.68) .. (110.44,79.68) .. controls (150.08,79.68) and (182.21,111.38) .. (182.21,150.49) .. controls (182.21,189.59) and (150.08,221.29) .. (110.44,221.29) .. controls (70.81,221.29) and (38.67,189.59) .. (38.67,150.49) -- cycle ;
                    %Straight Lines [id:da9702148411626599] 
                    \draw    (110.28,150.37) -- (143.06,90.37) ;
                    \draw [shift={(144.01,88.61)}, rotate = 118.64] [color={rgb, 255:red, 0; green, 0; blue, 0 }  ][line width=0.75]    (10.93,-3.29) .. controls (6.95,-1.4) and (3.31,-0.3) .. (0,0) .. controls (3.31,0.3) and (6.95,1.4) .. (10.93,3.29)   ;
                    %Shape: Brace [id:dp9873937062342023] 
                    \draw   (109.94,149.6) .. controls (114.03,151.85) and (117.2,150.93) .. (119.45,146.84) -- (129.39,128.78) .. controls (132.6,122.93) and (136.25,121.14) .. (140.34,123.39) .. controls (136.25,121.14) and (135.81,117.09) .. (139.02,111.25)(137.58,113.88) -- (146.25,98.12) .. controls (148.5,94.03) and (147.57,90.86) .. (143.48,88.61) ;
                    %Shape: Ellipse [id:dp8969952940786448] 
                    \draw   (14.58,150.37) .. controls (14.58,98.23) and (57.43,55.96) .. (110.28,55.96) .. controls (163.14,55.96) and (205.99,98.23) .. (205.99,150.37) .. controls (205.99,202.52) and (163.14,244.79) .. (110.28,244.79) .. controls (57.43,244.79) and (14.58,202.52) .. (14.58,150.37) -- cycle ;
                    %Shape: Ellipse [id:dp7187075666768598] 
                    \draw  [color={rgb, 255:red, 9; green, 0; blue, 255 }  ,draw opacity=1 ][fill={rgb, 255:red, 0; green, 237; blue, 255 }  ,fill opacity=1 ] (100.23,55.96) .. controls (100.23,50.48) and (104.73,46.04) .. (110.28,46.04) .. controls (115.84,46.04) and (120.34,50.48) .. (120.34,55.96) .. controls (120.34,61.44) and (115.84,65.88) .. (110.28,65.88) .. controls (104.73,65.88) and (100.23,61.44) .. (100.23,55.96) -- cycle ;
                    %Curve Lines [id:da48812672631064924] 
                    \draw    (134.49,47.7) .. controls (126.27,38.49) and (102.1,38.22) .. (84.34,46.87) ;
                    \draw [shift={(82.71,47.7)}, rotate = 332.06] [color={rgb, 255:red, 0; green, 0; blue, 0 }  ][line width=0.75]    (10.93,-3.29) .. controls (6.95,-1.4) and (3.31,-0.3) .. (0,0) .. controls (3.31,0.3) and (6.95,1.4) .. (10.93,3.29)   ;
                    %Straight Lines [id:da1600520611946855] 
                    \draw    (206.37,84.97) -- (244.37,84.97) ;
                    \draw [shift={(246.37,84.97)}, rotate = 180] [color={rgb, 255:red, 0; green, 0; blue, 0 }  ][line width=0.75]    (10.93,-3.29) .. controls (6.95,-1.4) and (3.31,-0.3) .. (0,0) .. controls (3.31,0.3) and (6.95,1.4) .. (10.93,3.29)   ;
                    %Shape: Axis 2D [id:dp10209187505623352] 
                    \draw  (430.7,10.9) -- (430.7,250.69)(563.63,150.67) -- (229.6,150.67) (435.7,243.69) -- (430.7,250.69) -- (425.7,243.69) (556.63,145.67) -- (563.63,150.67) -- (556.63,155.67)  ;
                    %Shape: Ellipse [id:dp5799042500277464] 
                    \draw   (380.52,150.44) .. controls (380.52,122.5) and (403.08,99.84) .. (430.92,99.84) .. controls (458.76,99.84) and (481.32,122.5) .. (481.32,150.44) .. controls (481.32,178.39) and (458.76,201.04) .. (430.92,201.04) .. controls (403.08,201.04) and (380.52,178.39) .. (380.52,150.44) -- cycle ;
                    %Straight Lines [id:da9478794861791171] 
                    \draw    (432.54,148.38) -- (460.11,112.29) ;
                    \draw [shift={(461.32,110.7)}, rotate = 127.37] [color={rgb, 255:red, 0; green, 0; blue, 0 }  ][line width=0.75]    (10.93,-3.29) .. controls (6.95,-1.4) and (3.31,-0.3) .. (0,0) .. controls (3.31,0.3) and (6.95,1.4) .. (10.93,3.29)   ;
                    %Shape: Brace [id:dp3127346716796223] 
                    \draw   (432.25,147.91) .. controls (435.95,150.76) and (439.22,150.33) .. (442.07,146.63) -- (449.96,136.37) .. controls (454.03,131.08) and (457.91,129.86) .. (461.6,132.71) .. controls (457.91,129.86) and (458.09,125.8) .. (462.15,120.51)(460.32,122.89) -- (462.15,120.51) .. controls (465,116.81) and (464.57,113.54) .. (460.87,110.7) ;
                    %Shape: Ellipse [id:dp7955325203767549] 
                    \draw  [color={rgb, 255:red, 9; green, 0; blue, 255 }  ,draw opacity=1 ][fill={rgb, 255:red, 0; green, 237; blue, 255 }  ,fill opacity=1 ] (501.7,119.62) .. controls (501.7,113.84) and (506.49,109.16) .. (512.38,109.16) .. controls (518.28,109.16) and (523.06,113.84) .. (523.06,119.62) .. controls (523.06,125.39) and (518.28,130.07) .. (512.38,130.07) .. controls (506.49,130.07) and (501.7,125.39) .. (501.7,119.62) -- cycle ;
                    %Curve Lines [id:da549952689278042] 
                    \draw    (530,130.87) .. controls (537.95,113.69) and (527.07,104.7) .. (516.49,96.08) ;
                    \draw [shift={(515,94.87)}, rotate = 39.29] [color={rgb, 255:red, 0; green, 0; blue, 0 }  ][line width=0.75]    (10.93,-3.29) .. controls (6.95,-1.4) and (3.31,-0.3) .. (0,0) .. controls (3.31,0.3) and (6.95,1.4) .. (10.93,3.29)   ;
                    %Shape: Brace [id:dp20072165583131396] 
                    \draw   (538.03,150.3) .. controls (542.7,150.27) and (545.01,147.92) .. (544.98,143.25) -- (544.76,114.6) .. controls (544.71,107.93) and (547.01,104.58) .. (551.68,104.54) .. controls (547.01,104.58) and (544.66,101.27) .. (544.61,94.6)(544.63,97.6) -- (544.39,65.95) .. controls (544.35,61.28) and (542,58.97) .. (537.33,59) ;
                    %Straight Lines [id:da7570386396384341] 
                    \draw    (379.98,129.7) -- (379.7,170.67) ;
                    %Shape: Circle [id:dp2883146230666529] 
                    \draw  [color={rgb, 255:red, 0; green, 0; blue, 0 }  ,draw opacity=1 ][fill={rgb, 255:red, 0; green, 0; blue, 0 }  ,fill opacity=1 ] (427.94,150.44) .. controls (427.94,148.8) and (429.28,147.47) .. (430.92,147.47) .. controls (432.56,147.47) and (433.89,148.8) .. (433.89,150.44) .. controls (433.89,152.08) and (432.56,153.42) .. (430.92,153.42) .. controls (429.28,153.42) and (427.94,152.08) .. (427.94,150.44) -- cycle ;
                    %Shape: Brace [id:dp8734638817204878] 
                    \draw   (432.03,260) .. controls (432.06,264.67) and (434.41,266.98) .. (439.08,266.95) -- (465.69,266.77) .. controls (472.36,266.72) and (475.71,269.03) .. (475.74,273.7) .. controls (475.71,269.03) and (479.02,266.68) .. (485.69,266.63)(482.69,266.66) -- (512.3,266.45) .. controls (516.97,266.42) and (519.28,264.08) .. (519.25,259.41) ;
                    %Shape: Brace [id:dp2335042287457043] 
                    \draw   (430.7,51.4) .. controls (430.73,46.73) and (428.42,44.38) .. (423.75,44.35) -- (345.75,43.81) .. controls (339.08,43.76) and (335.77,41.41) .. (335.8,36.74) .. controls (335.77,41.41) and (332.42,43.72) .. (325.75,43.67)(328.75,43.69) -- (247.75,43.12) .. controls (243.08,43.09) and (240.73,45.4) .. (240.7,50.07) ;
                    %Shape: Circle [id:dp9515653585868795] 
                    \draw  [color={rgb, 255:red, 0; green, 0; blue, 0 }  ,draw opacity=1 ][fill={rgb, 255:red, 0; green, 0; blue, 0 }  ,fill opacity=1 ] (326.9,151.02) .. controls (326.9,149.38) and (328.23,148.05) .. (329.88,148.05) .. controls (331.52,148.05) and (332.85,149.38) .. (332.85,151.02) .. controls (332.85,152.67) and (331.52,154) .. (329.88,154) .. controls (328.23,154) and (326.9,152.67) .. (326.9,151.02) -- cycle ;
                    %Shape: Brace [id:dp8259654533496823] 
                    \draw   (380.7,279.33) .. controls (380.7,284) and (383.03,286.33) .. (387.7,286.33) -- (440.53,286.33) .. controls (447.2,286.33) and (450.53,288.66) .. (450.53,293.33) .. controls (450.53,288.66) and (453.86,286.33) .. (460.53,286.33)(457.53,286.33) -- (513.37,286.33) .. controls (518.04,286.33) and (520.37,284) .. (520.37,279.33) ;

                    % Text Node
                    \draw (139.47,113.58) node [anchor=north west][inner sep=0.75pt]   [align=left] {$\displaystyle R$};
                    % Text Node
                    \draw (461.93,130.54) node [anchor=north west][inner sep=0.75pt]   [align=left] {$\displaystyle R$};
                    % Text Node
                    \draw (554,96) node [anchor=north west][inner sep=0.75pt]   [align=left] {$\displaystyle b$};
                    % Text Node
                    \draw (445.67,296) node [anchor=north west][inner sep=0.75pt]   [align=left] {$\displaystyle a$};
                    % Text Node
                    \draw (369.18,173.31) node [anchor=north west][inner sep=0.75pt]   [align=left] {$\displaystyle C_{1}$};
                    % Text Node
                    \draw (462.67,271.67) node [anchor=north west][inner sep=0.75pt]   [align=left] {$\displaystyle r_{min}$};
                    % Text Node
                    \draw (319.67,21) node [anchor=north west][inner sep=0.75pt]   [align=left] {$\displaystyle r_{max}$};

                    \end{tikzpicture}

                    \caption{Cambio de Órbita del Ejercicio 3 Apartado b)}
                    \label{fig:ej3b2}
                \end{center}
            \end{figure}

            Nótese la rotación de 90º en sentido horario que se ha hecho para tener la representación usual de la elipse, y que se puede hacer por la autonomía de las soluciones del campo newtoniano, y, por tanto, por la invarianza frente a isometrías.
        \end{itemize}
    \end{ejercicio}

    
\end{document}