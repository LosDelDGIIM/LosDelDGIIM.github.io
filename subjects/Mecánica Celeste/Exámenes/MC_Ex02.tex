\documentclass[12pt]{article}

% Idioma y codificación
\usepackage[spanish, es-tabla]{babel}       %es-tabla para que se titule "Tabla"
\usepackage[utf8]{inputenc}

% Márgenes
\usepackage[a4paper,top=3cm,bottom=2.5cm,left=3cm,right=3cm]{geometry}

% Comentarios de bloque
\usepackage{verbatim}

% Paquetes de links
\usepackage[hidelinks]{hyperref}    % Permite enlaces
\usepackage{url}                    % redirecciona a la web

% Más opciones para enumeraciones
\usepackage{enumitem}

% Personalizar la portada
\usepackage{titling}

% Paquetes de tablas
\usepackage{multirow}


%------------------------------------------------------------------------

%Paquetes de figuras
\usepackage{caption}
\usepackage{subcaption} % Figuras al lado de otras
\usepackage{float}      % Poner figuras en el sitio indicado H.


% Paquetes de imágenes
\usepackage{graphicx}       % Paquete para añadir imágenes
\usepackage{transparent}    % Para manejar la opacidad de las figuras

% Paquete para usar colores
\usepackage[dvipsnames]{xcolor}
\usepackage{pagecolor}      % Para cambiar el color de la página

% Habilita tamaños de fuente mayores
\usepackage{fix-cm}

% Para los gráficos
\usepackage{tikz}

% Para poder situar los nodos en los grafos
\usetikzlibrary{positioning}


%------------------------------------------------------------------------

% Paquetes de matemáticas
\usepackage{mathtools, amsfonts, amssymb, mathrsfs}
\usepackage[makeroom]{cancel}     % Simplificar tachando
\usepackage{polynom}    % Divisiones y Ruffini
\usepackage{units} % Para poner fracciones diagonales con \nicefrac

\usepackage{pgfplots}   %Representar funciones
\pgfplotsset{compat=1.18}  % Versión 1.18

\usepackage{tikz-cd}    % Para usar diagramas de composiciones
\usetikzlibrary{calc}   % Para usar cálculo de coordenadas en tikz

%Definición de teoremas, etc.
\usepackage{amsthm}
%\swapnumbers   % Intercambia la posición del texto y de la numeración

\theoremstyle{plain}

\makeatletter
\@ifclassloaded{article}{
  \newtheorem{teo}{Teorema}[section]
}{
  \newtheorem{teo}{Teorema}[chapter]  % Se resetea en cada chapter
}
\makeatother

\newtheorem{coro}{Corolario}[teo]           % Se resetea en cada teorema
\newtheorem{prop}[teo]{Proposición}         % Usa el mismo contador que teorema
\newtheorem{lema}[teo]{Lema}                % Usa el mismo contador que teorema

\theoremstyle{remark}
\newtheorem*{observacion}{Observación}

\theoremstyle{definition}

\makeatletter
\@ifclassloaded{article}{
  \newtheorem{definicion}{Definición} [section]     % Se resetea en cada chapter
}{
  \newtheorem{definicion}{Definición} [chapter]     % Se resetea en cada chapter
}
\makeatother

\newtheorem*{notacion}{Notación}
\newtheorem*{ejemplo}{Ejemplo}
\newtheorem*{ejercicio*}{Ejercicio}             % No numerado
\newtheorem{ejercicio}{Ejercicio} [section]     % Se resetea en cada section


% Modificar el formato de la numeración del teorema "ejercicio"
\renewcommand{\theejercicio}{%
  \ifnum\value{section}=0 % Si no se ha iniciado ninguna sección
    \arabic{ejercicio}% Solo mostrar el número de ejercicio
  \else
    \thesection.\arabic{ejercicio}% Mostrar número de sección y número de ejercicio
  \fi
}


% \renewcommand\qedsymbol{$\blacksquare$}         % Cambiar símbolo QED
%------------------------------------------------------------------------

% Paquetes para encabezados
\usepackage{fancyhdr}
\pagestyle{fancy}
\fancyhf{}

\newcommand{\helv}{ % Modificación tamaño de letra
\fontfamily{}\fontsize{12}{12}\selectfont}
\setlength{\headheight}{15pt} % Amplía el tamaño del índice


%\usepackage{lastpage}   % Referenciar última pag   \pageref{LastPage}
\fancyfoot[C]{\thepage}

%------------------------------------------------------------------------

% Conseguir que no ponga "Capítulo 1". Sino solo "1."
\makeatletter
\@ifclassloaded{book}{
  \renewcommand{\chaptermark}[1]{\markboth{\thechapter.\ #1}{}} % En el encabezado
    
  \renewcommand{\@makechapterhead}[1]{%
  \vspace*{50\p@}%
  {\parindent \z@ \raggedright \normalfont
    \ifnum \c@secnumdepth >\m@ne
      \huge\bfseries \thechapter.\hspace{1em}\ignorespaces
    \fi
    \interlinepenalty\@M
    \Huge \bfseries #1\par\nobreak
    \vskip 40\p@
  }}
}
\makeatother

%------------------------------------------------------------------------
% Paquetes de cógido
\usepackage{minted}
\renewcommand\listingscaption{Código fuente}

\usepackage{fancyvrb}
% Personaliza el tamaño de los números de línea
\renewcommand{\theFancyVerbLine}{\small\arabic{FancyVerbLine}}

% Estilo para C++
\newminted{cpp}{
    frame=lines,
    framesep=2mm,
    baselinestretch=1.2,
    linenos,
    escapeinside=||
}

% para minted
\definecolor{LightGray}{rgb}{0.95,0.95,0.92}
\setminted{
    linenos=true,
    stepnumber=5,
    numberfirstline=true,
    autogobble,
    breaklines=true,
    breakautoindent=true,
    breaksymbolleft=,
    breaksymbolright=,
    breaksymbolindentleft=0pt,
    breaksymbolindentright=0pt,
    breaksymbolsepleft=0pt,
    breaksymbolsepright=0pt,
    fontsize=\footnotesize,
    bgcolor=LightGray,
    numbersep=10pt
}


\usepackage{listings} % Para incluir código desde un archivo

\renewcommand\lstlistingname{Código Fuente}
\renewcommand\lstlistlistingname{Índice de Códigos Fuente}

% Definir colores
\definecolor{vscodepurple}{rgb}{0.5,0,0.5}
\definecolor{vscodeblue}{rgb}{0,0,0.8}
\definecolor{vscodegreen}{rgb}{0,0.5,0}
\definecolor{vscodegray}{rgb}{0.5,0.5,0.5}
\definecolor{vscodebackground}{rgb}{0.97,0.97,0.97}
\definecolor{vscodelightgray}{rgb}{0.9,0.9,0.9}

% Configuración para el estilo de C similar a VSCode
\lstdefinestyle{vscode_C}{
  backgroundcolor=\color{vscodebackground},
  commentstyle=\color{vscodegreen},
  keywordstyle=\color{vscodeblue},
  numberstyle=\tiny\color{vscodegray},
  stringstyle=\color{vscodepurple},
  basicstyle=\scriptsize\ttfamily,
  breakatwhitespace=false,
  breaklines=true,
  captionpos=b,
  keepspaces=true,
  numbers=left,
  numbersep=5pt,
  showspaces=false,
  showstringspaces=false,
  showtabs=false,
  tabsize=2,
  frame=tb,
  framerule=0pt,
  aboveskip=10pt,
  belowskip=10pt,
  xleftmargin=10pt,
  xrightmargin=10pt,
  framexleftmargin=10pt,
  framexrightmargin=10pt,
  framesep=0pt,
  rulecolor=\color{vscodelightgray},
  backgroundcolor=\color{vscodebackground},
}

%------------------------------------------------------------------------

% Comandos definidos
\newcommand{\bb}[1]{\mathbb{#1}}
\newcommand{\cc}[1]{\mathcal{#1}}

% I prefer the slanted \leq
\let\oldleq\leq % save them in case they're every wanted
\let\oldgeq\geq
\renewcommand{\leq}{\leqslant}
\renewcommand{\geq}{\geqslant}

% Si y solo si
\newcommand{\sii}{\iff}

% Letras griegas
\newcommand{\eps}{\epsilon}
\newcommand{\veps}{\varepsilon}
\newcommand{\lm}{\lambda}

\newcommand{\ol}{\overline}
\newcommand{\ul}{\underline}
\newcommand{\wt}{\widetilde}
\newcommand{\wh}{\widehat}

\let\oldvec\vec
\renewcommand{\vec}{\overrightarrow}

% Derivadas parciales
\newcommand{\del}[2]{\frac{\partial #1}{\partial #2}}
\newcommand{\Del}[3]{\frac{\partial^{#1} #2}{\partial #3^{#1}}}
\newcommand{\deld}[2]{\dfrac{\partial #1}{\partial #2}}
\newcommand{\Deld}[3]{\dfrac{\partial^{#1} #2}{\partial #3^{#1}}}


\newcommand{\AstIg}{\stackrel{(\ast)}{=}}
\newcommand{\Hop}{\stackrel{L'H\hat{o}pital}{=}}

\newcommand{\red}[1]{{\color{red}#1}} % Para integrales, destacar los cambios.

% Método de integración
\newcommand{\MetInt}[2]{
    \left[\begin{array}{c}
        #1 \\ #2
    \end{array}\right]
}

% Declarar aplicaciones
% 1. Nombre aplicación
% 2. Dominio
% 3. Codominio
% 4. Variable
% 5. Imagen de la variable
\newcommand{\Func}[5]{
    \begin{equation*}
        \begin{array}{rrll}
            #1:& #2 & \longrightarrow & #3\\
               & #4 & \longmapsto & #5
        \end{array}
    \end{equation*}
}

%------------------------------------------------------------------------


\usetikzlibrary{arrows.meta,decorations.markings}

\newcommand{\R}{\mathbb{R}}
\newcommand{\prodescalar}[2]{\langle #1, #2 \rangle}
\newcommand{\ortogonal}[1]{#1^{\perp}}

\begin{document}

    % 1. Foto de fondo
    % 2. Título
    % 3. Encabezado Izquierdo
    % 4. Color de fondo
    % 5. Coord x del titulo
    % 6. Coord y del titulo
    % 7. Fecha

    
    % 1. Foto de fondo
% 2. Título
% 3. Encabezado Izquierdo
% 4. Color de fondo
% 5. Coord x del titulo
% 6. Coord y del titulo
% 7. Fecha

\newcommand{\portada}[7]{

    \portadaBase{#1}{#2}{#3}{#4}{#5}{#6}{#7}
    \portadaBook{#1}{#2}{#3}{#4}{#5}{#6}{#7}
}

\newcommand{\portadaExamen}[7]{

    \portadaBase{#1}{#2}{#3}{#4}{#5}{#6}{#7}
    \portadaArticle{#1}{#2}{#3}{#4}{#5}{#6}{#7}
}




\newcommand{\portadaBase}[7]{

    % Tiene la portada principal y la licencia Creative Commons
    
    % 1. Foto de fondo
    % 2. Título
    % 3. Encabezado Izquierdo
    % 4. Color de fondo
    % 5. Coord x del titulo
    % 6. Coord y del titulo
    % 7. Fecha
    
    
    \thispagestyle{empty}               % Sin encabezado ni pie de página
    \newgeometry{margin=0cm}        % Márgenes nulos para la primera página
    
    
    % Encabezado
    \fancyhead[L]{\helv #3}
    \fancyhead[R]{\helv \nouppercase{\leftmark}}
    
    
    \pagecolor{#4}        % Color de fondo para la portada
    
    \begin{figure}[p]
        \centering
        \transparent{0.3}           % Opacidad del 30% para la imagen
        
        \includegraphics[width=\paperwidth, keepaspectratio]{assets/#1}
    
        \begin{tikzpicture}[remember picture, overlay]
            \node[anchor=north west, text=white, opacity=1, font=\fontsize{60}{90}\selectfont\bfseries\sffamily, align=left] at (#5, #6) {#2};
            
            \node[anchor=south east, text=white, opacity=1, font=\fontsize{12}{18}\selectfont\sffamily, align=right] at (9.7, 3) {\textbf{\href{https://losdeldgiim.github.io/}{Los Del DGIIM}}};
            
            \node[anchor=south east, text=white, opacity=1, font=\fontsize{12}{15}\selectfont\sffamily, align=right] at (9.7, 1.8) {Doble Grado en Ingeniería Informática y Matemáticas\\Universidad de Granada};
        \end{tikzpicture}
    \end{figure}
    
    
    \restoregeometry        % Restaurar márgenes normales para las páginas subsiguientes
    \pagecolor{white}       % Restaurar el color de página
    
    
    \newpage
    \thispagestyle{empty}               % Sin encabezado ni pie de página
    \begin{tikzpicture}[remember picture, overlay]
        \node[anchor=south west, inner sep=3cm] at (current page.south west) {
            \begin{minipage}{0.5\paperwidth}
                \href{https://creativecommons.org/licenses/by-nc-nd/4.0/}{
                    \includegraphics[height=2cm]{assets/Licencia.png}
                }\vspace{1cm}\\
                Esta obra está bajo una
                \href{https://creativecommons.org/licenses/by-nc-nd/4.0/}{
                    Licencia Creative Commons Atribución-NoComercial-SinDerivadas 4.0 Internacional (CC BY-NC-ND 4.0).
                }\\
    
                Eres libre de compartir y redistribuir el contenido de esta obra en cualquier medio o formato, siempre y cuando des el crédito adecuado a los autores originales y no persigas fines comerciales. 
            \end{minipage}
        };
    \end{tikzpicture}
    
    
    
    % 1. Foto de fondo
    % 2. Título
    % 3. Encabezado Izquierdo
    % 4. Color de fondo
    % 5. Coord x del titulo
    % 6. Coord y del titulo
    % 7. Fecha


}


\newcommand{\portadaBook}[7]{

    % 1. Foto de fondo
    % 2. Título
    % 3. Encabezado Izquierdo
    % 4. Color de fondo
    % 5. Coord x del titulo
    % 6. Coord y del titulo
    % 7. Fecha

    % Personaliza el formato del título
    \pretitle{\begin{center}\bfseries\fontsize{42}{56}\selectfont}
    \posttitle{\par\end{center}\vspace{2em}}
    
    % Personaliza el formato del autor
    \preauthor{\begin{center}\Large}
    \postauthor{\par\end{center}\vfill}
    
    % Personaliza el formato de la fecha
    \predate{\begin{center}\huge}
    \postdate{\par\end{center}\vspace{2em}}
    
    \title{#2}
    \author{\href{https://losdeldgiim.github.io/}{Los Del DGIIM}}
    \date{Granada, #7}
    \maketitle
    
    \tableofcontents
}




\newcommand{\portadaArticle}[7]{

    % 1. Foto de fondo
    % 2. Título
    % 3. Encabezado Izquierdo
    % 4. Color de fondo
    % 5. Coord x del titulo
    % 6. Coord y del titulo
    % 7. Fecha

    % Personaliza el formato del título
    \pretitle{\begin{center}\bfseries\fontsize{42}{56}\selectfont}
    \posttitle{\par\end{center}\vspace{2em}}
    
    % Personaliza el formato del autor
    \preauthor{\begin{center}\Large}
    \postauthor{\par\end{center}\vspace{3em}}
    
    % Personaliza el formato de la fecha
    \predate{\begin{center}\huge}
    \postdate{\par\end{center}\vspace{5em}}
    
    \title{#2}
    \author{\href{https://losdeldgiim.github.io/}{Los Del DGIIM}}
    \date{Granada, #7}
    \thispagestyle{empty}               % Sin encabezado ni pie de página
    \maketitle
    \vfill
}
    \portadaExamen{ffccA4.jpg}{Mecánica Celeste\\Examen II}{Mecánica Celeste. Examen II}{MidnightBlue}{-8}{28}{2025}{José Manuel Sánchez Varbas}

    \begin{description}
        \item[Asignatura] Mecánica Celeste.
        \item[Curso Académico] 2023-24.
        \item[Grado] Grado en Matemáticas.
        \item[Grupo] A.
        \item[Descripción] Segundo Parcial.
        \item[Fecha] 20 de Diciembre de 2023.
        \item[Duración] 1 hora y 30 minutos.
    \end{description}
    \newpage


    % ------------------------------------
    
    El número entre corchetes es la puntuación máxima de cada ejercicio o apartado.

    \begin{ejercicio}[1 punto]
        Decide si es verdadera o falsa la siguiente afirmación: en el problema de $n$ cuerpos existe una solución maximal
        $$
        r = (r_1, r_2, \cdots, r_n) : ]\alpha,\omega[ \to \mathbb{R}^{3n}
        $$
        que cumple
        $$
        |r_i(t) - r_j(t)| \geqslant 1, \quad t \in ]\alpha,\omega[, \quad 1 \leqslant i < j \leqslant n,
        $$
        y tal que $\omega < +\infty$.
    \end{ejercicio}

    \begin{ejercicio}[7 puntos]
        Dos masas, $m_1 = 3\cdot 10^{24}\,\mathrm{Kg}$ y $m_2 = 10^{24}\,\mathrm{Kg}$, 
        se mueven en órbitas circulares coplanarias alrededor de su centro de masas. 
        Se pretende colocar satélites en órbita en los puntos de libración $L_4$ y $L_5$ 
        correspondientes a esas masas primarias, que sabemos que son estables para el problema restringido de los 
        tres cuerpos circular. Se pide:
        \begin{enumerate}
            \item[a)] [1] Determinar la masa $\mu$ de la primaria más pequeña en las unidades apropiadas 
            para que la masa total de las primarias sea $1$.
            \item[b)] [2] Encontrar las coordenadas de los puntos de libración $L_4$ y $L_5$ en el sistema de referencia 
            con origen el centro de masas de las primarias, supuesto que la primaria de mayor masa se sitúa en el 
            punto $P_1 = (-\mu,0)$ y la otra en el $P_2 = (1-\mu,0)$, con $\mu$ el valor obtenido en el apartado anterior.
            \item[c)] [2] Hacer un esbozo del movimiento de los tres cuerpos si el satélite se sitúa en $L_4$.
            \item[d)] [2] Con los valores obtenidos en los apartados anteriores, probar que la función potencial
            $$
            \Phi(z) = \frac{1}{2}|z|^2 + \frac{1-\mu}{|P_1 - z|} + \frac{\mu}{|P_2 - z|} + \frac{1}{2}\mu(1-\mu),
            \quad z \in \mathbb{R}^2 \setminus \{P_1,P_2\},
            $$
            alcanza su mínimo absoluto en los puntos $L_4$ y $L_5$ y calcular el valor de la constante de Jacobi en esos puntos.
        \end{enumerate}
    \end{ejercicio}

    \begin{ejercicio}[2 puntos] 
        En el sistema del ejercicio anterior, un satélite se sitúa a distancia menor que 
        $1/4$ de la primaria de mayor masa con velocidad cero. Demuestra que dicho satélite no se sale de la región
        $$
        \{\, |P_1 - z| < 1/4 \,\}.
        $$
        (Sugerencia: utiliza las regiones de Hill).
    \end{ejercicio}

    \newpage

    \setcounter{ejercicio}{0}

    \begin{ejercicio}[1 punto]
        Decide si es verdadera o falsa la siguiente afirmación: en el problema de $n$ cuerpos existe una solución maximal
        $$
        r = (r_1, r_2, \cdots, r_n) : ]\alpha,\omega[ \to \mathbb{R}^{3n}
        $$
        que cumple
        $$
        |r_i(t) - r_j(t)| \geqslant 1, \quad t \in ]\alpha,\omega[, \quad 1 \leqslant i < j \leqslant n,
        $$
        y tal que $\omega < +\infty$. \\

        Sea $\rho(t) = \displaystyle \min_{1 \leqslant i < j \leqslant n} |r_i(t) - r_j(t)|$. Si $\omega < + \infty$, entonces por un teorema visto en teoría sabemos que $\rho(t) \to 0$ cuando $t \to \omega$. 
        Esto es una contradicción con la hipótesis $|r_i(t) - r_j(t)| \geqslant 1, \quad t \in ]\alpha,\omega[, \quad 1 \leqslant i < j \leqslant n$, por lo que la afirmación necesariamente debe ser falsa.
    \end{ejercicio}

    \begin{ejercicio}[7 puntos]
        Dos masas, $m_1 = 3\cdot 10^{24}\,\mathrm{Kg}$ y $m_2 = 10^{24}\,\mathrm{Kg}$, 
        se mueven en órbitas circulares coplanarias alrededor de su centro de masas. 
        Se pretende colocar satélites en órbita en los puntos de libración $L_4$ y $L_5$ 
        correspondientes a esas masas primarias, que sabemos que son estables para el problema restringido de los 
        tres cuerpos circular. Se pide:
        \begin{enumerate}
            \item[a)] [1] Determinar la masa $\mu$ de la primaria más pequeña en las unidades apropiadas 
            para que la masa total de las primarias sea $1$. \\

            Sabemos que $m_1 = 1 - \mu$, $m_2 = \mu$, $\mu \in ]0, 1/2]$ y $m_1 + m_2 = 1$, por lo que la masa $\mu$ en las unidades apropiadas será

            $$\mu = \dfrac{m_2}{m_1 + m_2} = \dfrac{10^{24}\,\mathrm{Kg}}{3\cdot 10^{24}\,\mathrm{Kg} + 10^{24}\,\mathrm{Kg}} = \dfrac{1}{4}$$

            \item[b)] [2] Encontrar las coordenadas de los puntos de libración $L_4$ y $L_5$ en el sistema de referencia 
            con origen el centro de masas de las primarias, supuesto que la primaria de mayor masa se sitúa en el 
            punto $P_1 = (-\mu,0)$ y la otra en el $P_2 = (1-\mu,0)$, con $\mu$ el valor obtenido en el apartado anterior. \\

            Por teoría sabemos que tanto $L_4$ como $L_5$, colocándolos como vértices, forman un triángulo equilátero de lado $1$ con las primarias. Por lo tanto, 
            los puntos de libración $L_4$ y $L_5$ son aquellos $z \in \mathbb{R}^2$ que verifican $$|z - P_1| = |z-P_2| = |P_1 - P_2| = 1$$
            Deducimos entonces que la abscisa de $L_4$ y $L_5$ está en la mediatriz de las primarias, es decir:
            $$M = \dfrac{P_1+P_2}{2} = \left(\dfrac{-\mu + 1 - \mu}{2},0 \right) = \left(\dfrac{1 - 2 \mu}{2},0 \right) = \left(\dfrac{1}{2} - \mu,0 \right)$$

            Denotando por $z = (x,y)$, entonces $x = \nicefrac{1}{2} - \mu$. \\

            Para obtener la altura, imponemos $|z-P_1| = 1$ (también se podría imponer $|z-P_2| = 1$). Como 
            $$z-P_1 = \left( \dfrac{1}{2} - \mu - (-\mu),y \right) = \left( \dfrac{1}{2},y \right)$$
            Entonces 
            $$|z-P_1| = 1 \iff |z-P_1|^2 = 1 \iff \left( \dfrac{1}{2} \right)^2 + y^2 = 1 \iff y^2 = 1 - \dfrac{1}{4} = \dfrac{3}{4} \iff y = \pm \dfrac{\sqrt{3}}{2}$$

            Consecuentemente 

            $$L_4 = \left( \dfrac{1}{2} - \mu, \dfrac{\sqrt{3}}{2} \right), \quad L_5 = \left( \dfrac{1}{2} - \mu, -\dfrac{\sqrt{3}}{2} \right)$$

            Sustituyendo $\mu = \nicefrac{1}{4}$ del apartado anterior

            $$L_4 = \left( \dfrac{1}{4}, \dfrac{\sqrt{3}}{2} \right), \quad L_5 = \left( \dfrac{1}{4}, -\dfrac{\sqrt{3}}{2} \right)$$

            \newpage

            \item[c)] [2] Hacer un esbozo del movimiento de los tres cuerpos si el satélite se sitúa en $L_4$. \\

            \begin{center}
                \begin{tikzpicture}[scale=2.7, line cap=round, line join=round]
                \shorthandoff{>}
                \definecolor{cM1}{RGB}{120,190,230}   % m1 (azul)
                \definecolor{cM2}{RGB}{ 60,200,110}   % m2 (verde)
                \definecolor{cL4}{RGB}{230,140, 40}   % L4 (naranja)
                \definecolor{tri}{RGB}{210, 70,210}   % triángulo (magenta)

                % --- origen ---
                \coordinate (O) at (0,0);

                \coordinate (m1) at ( 0.70,  0.32);
                \coordinate (m2) at (-1.1264624985, -0.4948832686);

                % equilátero lado 2 con L_4
                \coordinate (L4) at ( 0.4924783624, -1.6692045571);


                % --- tres órbitas concéntricas centradas en O ---
                \draw[dashed, line width=0.9pt, cM1, dash pattern=on 5pt off 4pt]
                (O) circle[radius=0.75]
                node[pos=0.15, above right, white] {$m_1$};

                \draw[dashed, line width=0.9pt, cM2, dash pattern=on 5pt off 4pt]
                (O) circle[radius=1.24]
                node[pos=0.10, above, white] {$m_2$};

                \draw[dashed, line width=0.9pt, cL4, dash pattern=on 5pt off 4pt]
                (O) circle[radius=1.74]
                node[pos=0.05, above left, white] {$L_4$};

                % --- triángulo punteado (si lo quieres mantener) ---
                \draw[dashed, tri, line width=1.0pt, dash pattern=on 4pt off 3pt]
                (m1)--(L4)--(m2)--cycle;

                % --- puntos (colores aproximados) ---
                \fill[cM1] (m1) circle[radius=0.18];
                \fill[cM2] (m2) circle[radius=0.09];
                \fill[cL4] (L4) circle[radius=0.04];
                \fill[black] (O)  circle[radius=0.03];

                % --- etiquetas de los puntos ---
                \node[black] at ($(m1)+(0.2,0.2)$) {$m_1$};
                \node[black] at ($(m2)+(-0.15,-0.15)$) {$m_2$};
                \node[black] at ($(L4)+(0.12,-0.1)$) {$L_4$};

                \draw[-{Stealth[length=2mm, width=2mm]}, black, thick]
                (0.59,0.47) to[bend right=7] (0.45,0.6);

                \draw[-{Stealth[length=2mm, width=2mm]}, black, thick]
                (-1.09,-0.585) to[bend right=3] (-1,-0.74);

                \draw[-{Stealth[length=2mm, width=2mm]}, black, thick]
                (0.535,-1.665) to[bend right=5] (0.65,-1.6);

                \end{tikzpicture}
            \end{center}

            En el sistema inercial (con centro de masas fijo en el origen) las dos primarias describen una órbita circular a la misma velocidad angular, así como
            el satélite situado en $L_4$. De esta manera, en cada instante los tres vértices están a distancia fija $|P_1 - P_2| = 1$, formando un triángulo equilátero de lado $1$,
            que rota rígidamente (sin deformarse) alrededor del centro de masas. 

            \newpage

            \item[d)] [2] Con los valores obtenidos en los apartados anteriores, probar que la función potencial
            $$
            \Phi(z) = \frac{1}{2}|z|^2 + \frac{1-\mu}{|P_1 - z|} + \frac{\mu}{|P_2 - z|} + \frac{1}{2}\mu(1-\mu),
            \quad z \in \mathbb{R}^2 \setminus \{P_1,P_2\},
            $$
            alcanza su mínimo absoluto en los puntos $L_4$ y $L_5$ y calcular el valor de la constante de Jacobi en esos puntos. \\

            Sea $\rho_1 = |z-P_1|$ y $\rho_2 = |z-P_2|$. Buscamos expresar $|z|^2$ en función de $\rho_1$ y $\rho_2$.
            Primero 
            $$|z-P_1|^2 = |z|^2 + |P_1|^2 - 2 z P_1$$
            $$|z-P_2|^2 = |z|^2 + |P_2|^2 - 2 z P_2$$
            Multiplicando la primera por $(1-\mu)$ y la segunda por $\mu$, y sumándolas, obtenemos
            \begin{equation}\label{eq:ec1}
                (1-\mu)\rho_1^2 + \mu \rho_2^2 = |z|^2 + (1 - \mu)|P_1|^2 + \mu |P_2|^2 -
                2z \cdot ((1-\mu)P_1 + \mu P_2)
            \end{equation}

            Como el centro de masas está en el origen, entonces $(1 - \mu)P_1 + \mu P_2 = 0$, y
            además sabemos que $|P_1|^2 = \mu^2$ y $|P_2|^2 = (1-\mu)^2$, de donde
            \begin{equation}\label{eq:ec2}
                (1-\mu)|P_1|^2 + \mu|P_2|^2 = (1-\mu) \mu^2 + \mu (1-\mu)^2 = \mu (1-\mu)
            \end{equation}

            Sustituyendo (\ref{eq:ec2}) en (\ref{eq:ec1})
            \begin{equation}\label{eq:ec3}
                (1-\mu)\rho_1^2 + \mu \rho_2^2 = |z|^2 + \mu(1-\mu) \iff 
                |z|^2 = (1-\mu)\rho_1^2 + \mu \rho_2^2 - \mu (1-\mu)
            \end{equation}

            Recuperando la función potencial dada en el enunciado, multiplicamos por $2$ a ambos lados, obteniendo
            \begin{gather}\label{eq:ec4}
                2\Phi(z) = |z|^2 + \frac{2(1-\mu)}{\rho_1} + \frac{2\mu}{\rho_2} + \mu(1-\mu),
            \quad z \in \mathbb{R}^2 \setminus \{P_1,P_2\},
            \end{gather}

            y sustituimos (\ref{eq:ec3}) en (\ref{eq:ec4}), llegando a

            $$2 \Phi(z) = [(1-\mu)\rho_1^2 + \mu \rho_2^2 - \mu (1-\mu)] + 
            \frac{2(1-\mu)}{\rho_1} + \frac{2\mu}{\rho_2} + \mu(1-\mu) =$$
            $$(1-\mu)\rho_1^2 + \mu \rho_2^2 + \frac{2(1-\mu)}{\rho_1} + \frac{2\mu}{\rho_2}$$
            Factorizando con $(1-\mu)$ y $\mu$, conseguimos la expresión cómoda

            \begin{equation}\label{eq:ec5}
                2 \Phi(z) = (1 - \mu) \left(\rho_1^2 + \dfrac{2}{\rho_1} \right) + \mu 
                \left(\rho_2^2 + \dfrac{2}{\rho_2} \right)
            \end{equation}


            Sea ahora la función \Func{g}{\mathbb{R}^{+}}{\mathbb{R}}{\rho}{\rho^2 + \dfrac{2}{\rho}}
            Vemos que 
            $$g'(\rho) = 2 \rho - \dfrac{2}{\rho^2}, \quad g''(\rho) = 2 + \dfrac{4}{\rho^3} > 0 \quad \forall \rho > 0$$
            Por lo que $g$ es estrictamente convexa, y su mínimo global se alcanza en los puntos críticos $g'(\rho) = 0$,
            es decir,
            $$g'(\rho) = 0 \iff 2 \rho - \dfrac{2}{\rho^2} = 0 \iff \cancel{2} \rho = \dfrac{\cancel{2}}{\rho^2} 
            \iff \rho^3 = 1 \iff \rho = 1$$
            Además, $g(1) = 1+2 = 3$, y, por ser $\rho = 1$ mínimo, $g(\rho) \geqslant 3 \quad \forall \rho > 0$.
            Teniendo esto en cuenta, vemos que de (\ref{eq:ec5}) se deduce que 
            $$2 \Phi(z) \geqslant (1 - \mu) \cdot 3 + \mu \cdot 3 = 3 \Longrightarrow \Phi(z) \geqslant \dfrac{3}{2}$$
            El mínimo de $\Phi$ se alcanza en caso de que $g(\rho_1) = 3 = g(\rho_2)$, es decir, $\rho_1 = 1 = \rho_2$,
            pero como $\rho_1 = |z - P_1| = 1 = |z-P_2| = \rho_2$, y los únicos puntos que verifican
            esto último son $L_4$ y $L_5$, por lo realizado en el apartado anterior, queda demostrado que
            $\Phi$ alcanza su mínimo absoluto en los puntos $L_4$ y $L_5$. \\

            Falta calcular la constante de Jacobi en $L_4$ y $L_5$. Por definición, $$J = 2 \Phi(z(t)) - |\dot{z}(t)|^2$$
            Como $L_4$ y $L_5$ son puntos de equilibrio en el problema restringido circular (demostrado en 
            teoría), entonces la solución $z = z(t)$ es constante, $z(t) \equiv c$, luego $\dot{z}(t) \equiv 0$. 
            Entonces $J(c) = 2 \Phi(c)$, y como hemos visto que el mínimo absoluto de $\Phi$ se alcanza en 
            $L_4$ y $L_5$, y además $\Phi(L_4) = \Phi(L_5) = \nicefrac{3}{2}$, concluimos que
            $$J(L_4) = J(L_5) = 2 \cdot \dfrac{3}{2} = 3$$
        \end{enumerate}
    \end{ejercicio}

    \newpage

    \begin{ejercicio}[2 puntos] 
        En el sistema del ejercicio anterior, un satélite se sitúa a distancia menor que 
        $1/4$ de la primaria de mayor masa con velocidad cero. Demuestra que dicho satélite no se sale de la región
        $$
        \{\, |P_1 - z| < 1/4 \,\}.
        $$
        (Sugerencia: utiliza las regiones de Hill). \\

       Sea $A = \{\, |P_1 - z| < 1/4 \,\}$ y supongamos que $z = z(t)$ define la posición del satélite. Como este se 
       sitúa con velocidad zero, entonces $\dot{z}(0) = 0$, y la constante de Jacobi,
       suponiendo que se sitúa en $z(0) = z_0 \in A$, es $J = 2 \Phi(z_0)$. Como siempre
       
       $$|\dot{z}(t)|^2 = 2 \Phi(z(t)) - J \geqslant 0 \Longrightarrow \Phi(z(t)) \geqslant \dfrac{J}{2} = \Phi(z_0)$$

       Es decir, el movimiento quedaría siempre en la región de Hill asociada al nivel $\Phi(z_0)$, que,
       por definición, es $$H = \{z \in \mathbb{R}^2 \setminus \{P_1,P_2\} : \Phi(z) \geqslant \Phi(z_0)\}$$
       Como $\Phi(z) \to +\infty$ cuando $z \to P_1$, existe una componente conexa $H_{P_1}$ de $H$
       que contiene a $P_1$ y a $z_0$. Además, la curva $\Phi = \Phi(z_0)$ (de velocidad cero)
       es la barrera, pues en el exterior $\Phi < \Phi(z_0)$, lo cual implicaría que $|\dot{z}(t)|^2 < 0$ (contradicción).
       En particular, fijado $z_0 \in A$, la componente conexa $H_{P_1}$ queda contenida en el disco $A$,
       por lo que la trayectoria no puede cruzar la circunferencia $|P_1 - z| = 1/4$. Así pues, concluimos 
       que $z(t) \in A \quad \forall t \in I$, siendo $I$ el intervalo maximal de la solución $z = z(t)$. 
    \end{ejercicio}
\end{document}