\documentclass[12pt]{article}

% Idioma y codificación
\usepackage[spanish, es-tabla]{babel}       %es-tabla para que se titule "Tabla"
\usepackage[utf8]{inputenc}

% Márgenes
\usepackage[a4paper,top=3cm,bottom=2.5cm,left=3cm,right=3cm]{geometry}

% Comentarios de bloque
\usepackage{verbatim}

% Paquetes de links
\usepackage[hidelinks]{hyperref}    % Permite enlaces
\usepackage{url}                    % redirecciona a la web

% Más opciones para enumeraciones
\usepackage{enumitem}

% Personalizar la portada
\usepackage{titling}

% Paquetes de tablas
\usepackage{multirow}


%------------------------------------------------------------------------

%Paquetes de figuras
\usepackage{caption}
\usepackage{subcaption} % Figuras al lado de otras
\usepackage{float}      % Poner figuras en el sitio indicado H.


% Paquetes de imágenes
\usepackage{graphicx}       % Paquete para añadir imágenes
\usepackage{transparent}    % Para manejar la opacidad de las figuras

% Paquete para usar colores
\usepackage[dvipsnames]{xcolor}
\usepackage{pagecolor}      % Para cambiar el color de la página

% Habilita tamaños de fuente mayores
\usepackage{fix-cm}

% Para los gráficos
\usepackage{tikz}

% Para poder situar los nodos en los grafos
\usetikzlibrary{positioning}


%------------------------------------------------------------------------

% Paquetes de matemáticas
\usepackage{mathtools, amsfonts, amssymb, mathrsfs}
\usepackage[makeroom]{cancel}     % Simplificar tachando
\usepackage{polynom}    % Divisiones y Ruffini
\usepackage{units} % Para poner fracciones diagonales con \nicefrac

\usepackage{pgfplots}   %Representar funciones
\pgfplotsset{compat=1.18}  % Versión 1.18

\usepackage{tikz-cd}    % Para usar diagramas de composiciones
\usetikzlibrary{calc}   % Para usar cálculo de coordenadas en tikz

%Definición de teoremas, etc.
\usepackage{amsthm}
%\swapnumbers   % Intercambia la posición del texto y de la numeración

\theoremstyle{plain}

\makeatletter
\@ifclassloaded{article}{
  \newtheorem{teo}{Teorema}[section]
}{
  \newtheorem{teo}{Teorema}[chapter]  % Se resetea en cada chapter
}
\makeatother

\newtheorem{coro}{Corolario}[teo]           % Se resetea en cada teorema
\newtheorem{prop}[teo]{Proposición}         % Usa el mismo contador que teorema
\newtheorem{lema}[teo]{Lema}                % Usa el mismo contador que teorema

\theoremstyle{remark}
\newtheorem*{observacion}{Observación}

\theoremstyle{definition}

\makeatletter
\@ifclassloaded{article}{
  \newtheorem{definicion}{Definición} [section]     % Se resetea en cada chapter
}{
  \newtheorem{definicion}{Definición} [chapter]     % Se resetea en cada chapter
}
\makeatother

\newtheorem*{notacion}{Notación}
\newtheorem*{ejemplo}{Ejemplo}
\newtheorem*{ejercicio*}{Ejercicio}             % No numerado
\newtheorem{ejercicio}{Ejercicio} [section]     % Se resetea en cada section


% Modificar el formato de la numeración del teorema "ejercicio"
\renewcommand{\theejercicio}{%
  \ifnum\value{section}=0 % Si no se ha iniciado ninguna sección
    \arabic{ejercicio}% Solo mostrar el número de ejercicio
  \else
    \thesection.\arabic{ejercicio}% Mostrar número de sección y número de ejercicio
  \fi
}


% \renewcommand\qedsymbol{$\blacksquare$}         % Cambiar símbolo QED
%------------------------------------------------------------------------

% Paquetes para encabezados
\usepackage{fancyhdr}
\pagestyle{fancy}
\fancyhf{}

\newcommand{\helv}{ % Modificación tamaño de letra
\fontfamily{}\fontsize{12}{12}\selectfont}
\setlength{\headheight}{15pt} % Amplía el tamaño del índice


%\usepackage{lastpage}   % Referenciar última pag   \pageref{LastPage}
\fancyfoot[C]{\thepage}

%------------------------------------------------------------------------

% Conseguir que no ponga "Capítulo 1". Sino solo "1."
\makeatletter
\@ifclassloaded{book}{
  \renewcommand{\chaptermark}[1]{\markboth{\thechapter.\ #1}{}} % En el encabezado
    
  \renewcommand{\@makechapterhead}[1]{%
  \vspace*{50\p@}%
  {\parindent \z@ \raggedright \normalfont
    \ifnum \c@secnumdepth >\m@ne
      \huge\bfseries \thechapter.\hspace{1em}\ignorespaces
    \fi
    \interlinepenalty\@M
    \Huge \bfseries #1\par\nobreak
    \vskip 40\p@
  }}
}
\makeatother

%------------------------------------------------------------------------
% Paquetes de cógido
\usepackage{minted}
\renewcommand\listingscaption{Código fuente}

\usepackage{fancyvrb}
% Personaliza el tamaño de los números de línea
\renewcommand{\theFancyVerbLine}{\small\arabic{FancyVerbLine}}

% Estilo para C++
\newminted{cpp}{
    frame=lines,
    framesep=2mm,
    baselinestretch=1.2,
    linenos,
    escapeinside=||
}

% para minted
\definecolor{LightGray}{rgb}{0.95,0.95,0.92}
\setminted{
    linenos=true,
    stepnumber=5,
    numberfirstline=true,
    autogobble,
    breaklines=true,
    breakautoindent=true,
    breaksymbolleft=,
    breaksymbolright=,
    breaksymbolindentleft=0pt,
    breaksymbolindentright=0pt,
    breaksymbolsepleft=0pt,
    breaksymbolsepright=0pt,
    fontsize=\footnotesize,
    bgcolor=LightGray,
    numbersep=10pt
}


\usepackage{listings} % Para incluir código desde un archivo

\renewcommand\lstlistingname{Código Fuente}
\renewcommand\lstlistlistingname{Índice de Códigos Fuente}

% Definir colores
\definecolor{vscodepurple}{rgb}{0.5,0,0.5}
\definecolor{vscodeblue}{rgb}{0,0,0.8}
\definecolor{vscodegreen}{rgb}{0,0.5,0}
\definecolor{vscodegray}{rgb}{0.5,0.5,0.5}
\definecolor{vscodebackground}{rgb}{0.97,0.97,0.97}
\definecolor{vscodelightgray}{rgb}{0.9,0.9,0.9}

% Configuración para el estilo de C similar a VSCode
\lstdefinestyle{vscode_C}{
  backgroundcolor=\color{vscodebackground},
  commentstyle=\color{vscodegreen},
  keywordstyle=\color{vscodeblue},
  numberstyle=\tiny\color{vscodegray},
  stringstyle=\color{vscodepurple},
  basicstyle=\scriptsize\ttfamily,
  breakatwhitespace=false,
  breaklines=true,
  captionpos=b,
  keepspaces=true,
  numbers=left,
  numbersep=5pt,
  showspaces=false,
  showstringspaces=false,
  showtabs=false,
  tabsize=2,
  frame=tb,
  framerule=0pt,
  aboveskip=10pt,
  belowskip=10pt,
  xleftmargin=10pt,
  xrightmargin=10pt,
  framexleftmargin=10pt,
  framexrightmargin=10pt,
  framesep=0pt,
  rulecolor=\color{vscodelightgray},
  backgroundcolor=\color{vscodebackground},
}

%------------------------------------------------------------------------

% Comandos definidos
\newcommand{\bb}[1]{\mathbb{#1}}
\newcommand{\cc}[1]{\mathcal{#1}}

% I prefer the slanted \leq
\let\oldleq\leq % save them in case they're every wanted
\let\oldgeq\geq
\renewcommand{\leq}{\leqslant}
\renewcommand{\geq}{\geqslant}

% Si y solo si
\newcommand{\sii}{\iff}

% Letras griegas
\newcommand{\eps}{\epsilon}
\newcommand{\veps}{\varepsilon}
\newcommand{\lm}{\lambda}

\newcommand{\ol}{\overline}
\newcommand{\ul}{\underline}
\newcommand{\wt}{\widetilde}
\newcommand{\wh}{\widehat}

\let\oldvec\vec
\renewcommand{\vec}{\overrightarrow}

% Derivadas parciales
\newcommand{\del}[2]{\frac{\partial #1}{\partial #2}}
\newcommand{\Del}[3]{\frac{\partial^{#1} #2}{\partial #3^{#1}}}
\newcommand{\deld}[2]{\dfrac{\partial #1}{\partial #2}}
\newcommand{\Deld}[3]{\dfrac{\partial^{#1} #2}{\partial #3^{#1}}}


\newcommand{\AstIg}{\stackrel{(\ast)}{=}}
\newcommand{\Hop}{\stackrel{L'H\hat{o}pital}{=}}

\newcommand{\red}[1]{{\color{red}#1}} % Para integrales, destacar los cambios.

% Método de integración
\newcommand{\MetInt}[2]{
    \left[\begin{array}{c}
        #1 \\ #2
    \end{array}\right]
}

% Declarar aplicaciones
% 1. Nombre aplicación
% 2. Dominio
% 3. Codominio
% 4. Variable
% 5. Imagen de la variable
\newcommand{\Func}[5]{
    \begin{equation*}
        \begin{array}{rrll}
            #1:& #2 & \longrightarrow & #3\\
               & #4 & \longmapsto & #5
        \end{array}
    \end{equation*}
}

%------------------------------------------------------------------------


\usetikzlibrary{arrows.meta,decorations.markings}

\newcommand{\R}{\mathbb{R}}
\newcommand{\prodescalar}[2]{\langle #1, #2 \rangle}
\newcommand{\ortogonal}[1]{#1^{\perp}}

\begin{document}

    % 1. Foto de fondo
    % 2. Título
    % 3. Encabezado Izquierdo
    % 4. Color de fondo
    % 5. Coord x del titulo
    % 6. Coord y del titulo
    % 7. Fecha

    
    % 1. Foto de fondo
% 2. Título
% 3. Encabezado Izquierdo
% 4. Color de fondo
% 5. Coord x del titulo
% 6. Coord y del titulo
% 7. Fecha

\newcommand{\portada}[7]{

    \portadaBase{#1}{#2}{#3}{#4}{#5}{#6}{#7}
    \portadaBook{#1}{#2}{#3}{#4}{#5}{#6}{#7}
}

\newcommand{\portadaExamen}[7]{

    \portadaBase{#1}{#2}{#3}{#4}{#5}{#6}{#7}
    \portadaArticle{#1}{#2}{#3}{#4}{#5}{#6}{#7}
}




\newcommand{\portadaBase}[7]{

    % Tiene la portada principal y la licencia Creative Commons
    
    % 1. Foto de fondo
    % 2. Título
    % 3. Encabezado Izquierdo
    % 4. Color de fondo
    % 5. Coord x del titulo
    % 6. Coord y del titulo
    % 7. Fecha
    
    
    \thispagestyle{empty}               % Sin encabezado ni pie de página
    \newgeometry{margin=0cm}        % Márgenes nulos para la primera página
    
    
    % Encabezado
    \fancyhead[L]{\helv #3}
    \fancyhead[R]{\helv \nouppercase{\leftmark}}
    
    
    \pagecolor{#4}        % Color de fondo para la portada
    
    \begin{figure}[p]
        \centering
        \transparent{0.3}           % Opacidad del 30% para la imagen
        
        \includegraphics[width=\paperwidth, keepaspectratio]{assets/#1}
    
        \begin{tikzpicture}[remember picture, overlay]
            \node[anchor=north west, text=white, opacity=1, font=\fontsize{60}{90}\selectfont\bfseries\sffamily, align=left] at (#5, #6) {#2};
            
            \node[anchor=south east, text=white, opacity=1, font=\fontsize{12}{18}\selectfont\sffamily, align=right] at (9.7, 3) {\textbf{\href{https://losdeldgiim.github.io/}{Los Del DGIIM}}};
            
            \node[anchor=south east, text=white, opacity=1, font=\fontsize{12}{15}\selectfont\sffamily, align=right] at (9.7, 1.8) {Doble Grado en Ingeniería Informática y Matemáticas\\Universidad de Granada};
        \end{tikzpicture}
    \end{figure}
    
    
    \restoregeometry        % Restaurar márgenes normales para las páginas subsiguientes
    \pagecolor{white}       % Restaurar el color de página
    
    
    \newpage
    \thispagestyle{empty}               % Sin encabezado ni pie de página
    \begin{tikzpicture}[remember picture, overlay]
        \node[anchor=south west, inner sep=3cm] at (current page.south west) {
            \begin{minipage}{0.5\paperwidth}
                \href{https://creativecommons.org/licenses/by-nc-nd/4.0/}{
                    \includegraphics[height=2cm]{assets/Licencia.png}
                }\vspace{1cm}\\
                Esta obra está bajo una
                \href{https://creativecommons.org/licenses/by-nc-nd/4.0/}{
                    Licencia Creative Commons Atribución-NoComercial-SinDerivadas 4.0 Internacional (CC BY-NC-ND 4.0).
                }\\
    
                Eres libre de compartir y redistribuir el contenido de esta obra en cualquier medio o formato, siempre y cuando des el crédito adecuado a los autores originales y no persigas fines comerciales. 
            \end{minipage}
        };
    \end{tikzpicture}
    
    
    
    % 1. Foto de fondo
    % 2. Título
    % 3. Encabezado Izquierdo
    % 4. Color de fondo
    % 5. Coord x del titulo
    % 6. Coord y del titulo
    % 7. Fecha


}


\newcommand{\portadaBook}[7]{

    % 1. Foto de fondo
    % 2. Título
    % 3. Encabezado Izquierdo
    % 4. Color de fondo
    % 5. Coord x del titulo
    % 6. Coord y del titulo
    % 7. Fecha

    % Personaliza el formato del título
    \pretitle{\begin{center}\bfseries\fontsize{42}{56}\selectfont}
    \posttitle{\par\end{center}\vspace{2em}}
    
    % Personaliza el formato del autor
    \preauthor{\begin{center}\Large}
    \postauthor{\par\end{center}\vfill}
    
    % Personaliza el formato de la fecha
    \predate{\begin{center}\huge}
    \postdate{\par\end{center}\vspace{2em}}
    
    \title{#2}
    \author{\href{https://losdeldgiim.github.io/}{Los Del DGIIM}}
    \date{Granada, #7}
    \maketitle
    
    \tableofcontents
}




\newcommand{\portadaArticle}[7]{

    % 1. Foto de fondo
    % 2. Título
    % 3. Encabezado Izquierdo
    % 4. Color de fondo
    % 5. Coord x del titulo
    % 6. Coord y del titulo
    % 7. Fecha

    % Personaliza el formato del título
    \pretitle{\begin{center}\bfseries\fontsize{42}{56}\selectfont}
    \posttitle{\par\end{center}\vspace{2em}}
    
    % Personaliza el formato del autor
    \preauthor{\begin{center}\Large}
    \postauthor{\par\end{center}\vspace{3em}}
    
    % Personaliza el formato de la fecha
    \predate{\begin{center}\huge}
    \postdate{\par\end{center}\vspace{5em}}
    
    \title{#2}
    \author{\href{https://losdeldgiim.github.io/}{Los Del DGIIM}}
    \date{Granada, #7}
    \thispagestyle{empty}               % Sin encabezado ni pie de página
    \maketitle
    \vfill
}
    \portadaExamen{ffccA4.jpg}{Mecánica Celeste\\Examen VI}{Mecánica Celeste. Examen VI}{MidnightBlue}{-8}{28}{2025}{José Manuel Sánchez Varbas}

    \begin{description}
        \item[Asignatura] Mecánica Celeste.
        \item[Curso Académico] 2024-25.
        \item[Grado] Grado en Matemáticas.
        \item[Grupo] A.
        \item[Profesor] Margarita Arias López.
        \item[Descripción] Incidencias Segundo Parcial.
        \item[Duración] 1 hora y 30 minutos.
    
    \end{description}
    \newpage


    % ------------------------------------
    
    El número entre corchetes es la puntuación máxima de cada ejercicio o apartado.

    \begin{ejercicio}[2 puntos]
        Pon un ejemplo razonado de un problema de dos cuerpos
        \begin{itemize}
            \item[a)] [1] en el que las dos masas se muevan sobre una recta pero no colisionen.
            \item[b)] [1] en el que las dos masas no sigan un movimiento rectilíneo y se alejen
            infinitamente de un observador que está situado en el origen.
        \end{itemize}
    \end{ejercicio}

    \begin{ejercicio}[2 puntos]
        Desde un observatorio astronómico fijo se está observando el movimiento de dos asteroides aislados
        que se mueven bajo la acción de la ley de gravitación de Newton. Cuando se comienzan las 
        observaciones, sus posiciones respecto al observatorio son $x(0) = (0,1,-1)$, $y(0) = (0,-3,3)$,
        y las velocidades con las que se mueven son $\dot{x}(0) = (1,1,-1)$, $\dot{y}(0) = (1,-3,-1)$.
        Se estima que la masa del asteroide que ocupa la posición $x$ es cuatro veces la del que está en la 
        posición $y$. Se pide:
        \begin{itemize}
            \item[a)] [1] Deduce el comportamiento del centro de masas del sistema que se observa desde la 
            posición del observatorio.
            \item[b)] [1] Determina el movimiento que se observará desde el centro de masas en función de la masa 
            $m$ del asteroide menor.
        \end{itemize}
    \end{ejercicio}

    \begin{ejercicio}[4 puntos]
        Se consideran tres masas iguales en los puntos $P_1 = (1,0)$, $P_2 = (-1/2, \sqrt{3}/2)$ y 
        $P_3 = (-1/2, -\sqrt{3}/2)$ con velocidades iniciales $\dot{r}_1(0) = (0,1/3)$,
        $\dot{r}_2(0) = (-\sqrt{3}/6, -1/6)$ y $\dot{r}_3(0) = (\sqrt{3}/6, -1/6)$.
        Se supone que sobre estas masas actúa únicamente la atracción que cada una ejerce sobre las otras
        según la Ley de Gravitación Universal.
        \begin{itemize}
            \item[a)] [1] Comprueba que el centro de masas permanece fijo en el origen.
            \item[b)] [1] El movimiento resultante, ¿se puede producir sobre una circunferencia?
            \item[c)] [1] ¿Puede haber colapso total?
            \item[d)] [1] Explica de forma intuitiva cuál crees que puede ser el movimiento resultante.    
        \end{itemize}
    \end{ejercicio}

    \begin{ejercicio}[2 puntos]
        En el problema restringido circular
        \begin{itemize}
            \item [a)] [1] ¿dónde colocarías un satélite y con qué velocidad inicial para asegurarte de 
            que se mantiene siempre a igual distancia de cada una de las masas?
            \item [b)] [1] ¿Y si lo que quieres es que no se acerque a una distancia menor de $1$ de una de las primarias?
        \end{itemize}
        Razona tus respuestas.
    \end{ejercicio}

    \newpage

    \setcounter{ejercicio}{0}

    \begin{ejercicio}[2 puntos]
        Pon un ejemplo razonado de un problema de dos cuerpos
        \begin{itemize}
            \item[a)] [1] en el que las dos masas se muevan sobre una recta pero no colisionen. \\
            
            Consideramos $m_1 = m_2 = m > 0$, así como las siguientes condiciones iniciales.
            $$x(0) = (0,0,1), \quad y(0) = (0,0,-1)$$
            $$\dot{x}(0) = (0,0,v), \quad \dot{y}(0) = (0,0,-v)$$

            con $v>0$. Sabemos por teoría que $\ddot{C}(t) = 0$, luego el centro de masas se mueve a velocidad constante,
            $C(t) = \alpha + \beta t$, con $\alpha = C(0), \beta = \dot{C}(0)$\footnote{Aplicando 
            el Teorema de Existencia y Unicidad para Ecuaciones Diferenciales Lineales visto en 
            Ecuaciones Diferenciales I}. Como por definición el centro de masas es 
            $$C(t) = \dfrac{m_1 x(t) + m_2 y(t)}{m_1 + m_2}$$
            y su derivada es 
            $$\dot{C}(t) = \dfrac{m_1 \dot{x}(t) + m_2 \dot{y}(t)}{m_1 + m_2}$$
            Vemos que 
            $$\alpha = C(0) = \dfrac{x(0) + y(0)}{2} = \dfrac{1}{2} (0,0,0) = (0,0,0)$$
            Análogamente, 
            $$\beta = \dfrac{1}{2}(0,0,0) = (0,0,0)$$

            Por lo tanto,
            $$C(t) = \alpha + \beta t \equiv 0$$
            es decir, el centro de masas permanece fijo en el origen. Entonces se cumple que 
            $$m_1 x + m_2 y = 0 \Longrightarrow y = -\frac{m_1}{m_2} x = - x$$

            Ahora, conocemos el problema de Kepler que verifica $x$:

            $$\ddot{x} = - G \dfrac{m_2^3}{(m_1 + m_2)^2} \dfrac{x}{|x|^3} = - G \dfrac{m^3}{4m^2} \dfrac{x}{|x|^3}
            = - \dfrac{Gm}{4} \dfrac{x}{|x|^3} = - \mu \dfrac{x}{|x|^3}$$

            Como
            $$c_x = \tilde{x}(0) \land \dot{\tilde{x}}(0) = \begin{vmatrix}
                0 & 0 & 1 \\
                0 & 0 & v \\
                i & j & k
            \end{vmatrix} = (0, 0, 0) \Longrightarrow |c_x| = 0$$

            \newpage

            el movimiento será rectilíneo en ambos cuerpos, y además sobre la misma recta,
            ya que por una Proposición vista en teoría, el centro de masas siempre está en la
            envolvente convexa de los $n$ cuerpos (en el caso $n=2$, la envolvente convexa es el segmento
            que los une). \\

            Obtenemos la condición de mínima velocidad para que la trayectoria no esté acotada. \\

            La energía total es constante, por lo que la obtenemos en $t=0$

            $$E = \dfrac{1}{2} |\dot{x}(0)|^2 - \dfrac{\mu}{|x(0)|} = \dfrac{1}{2} v^2 - \mu = 
            \dfrac{1}{2} v^2 - \dfrac{Gm}{4}$$

            Imponemos $E \geqslant 0$ para que la trayectoria no sea elíptica

            \begin{equation}\label{eq:ec1}
                E \geqslant 0 \iff \dfrac{1}{2} v^2 - \dfrac{Gm}{4} \geqslant 0 \iff 
                \dfrac{1}{2} v^2 \geqslant \dfrac{Gm}{4} \iff v \geqslant \sqrt{\dfrac{Gm}{2}}
            \end{equation}

            Por tanto, si $v$ verifica (\ref{eq:ec1}), entonces, por la clasificación de movimientos
            rectilíneos en campos de fuerza newtonianos, sabemos que $r(t) = |x(t)|$ es estrictamente
            creciente, pues $\dot{r}(0) > 0$, y sabemos que no hay cambio de monotonía, luego
            $r(t) \to + \infty$ cuando $t \to +\infty$. De esta forma, teniendo en cuenta que
            $$|(x-y)(t)| = 2 |x(t)| = 2 r(t) \geqslant 2 r(0) = 2 > 0 \quad (t \geqslant 0)$$
            no hay colisión en $[0,+\infty[$.

            \item[b)] [1] en el que las dos masas no sigan un movimiento rectilíneo y se alejen
            infinitamente de un observador que está situado en el origen. \\

            Consideramos $m_1 = m_2 = m > 0$, y las siguientes condiciones iniciales.
            $$x(0) = (1,0,0), \quad y(0) = (-1,0,0)$$
            $$\dot{x}(0) = (0,v,0), \quad \dot{y}(0) = (0,-v,0)$$
            con $v>0$. \\

            Igual que en el apartado anterior, $\ddot{C}(t) = 0$, luego $C(t) = \alpha + \beta t$, 
            con $\alpha = C(0)$ y $\beta = \dot{C}(0)$. Como
            $$C(0) = \dfrac{x(0) + y(0)}{2} = (0,0,0), \quad \dot{C}(0) = \dfrac{\dot{x}(0) + \dot{y}(0)}{2} = (0,0,0)$$
            concluimos que $C(t) = C(0) + \dot{C}(0) t \equiv 0$, es decir, el centro de masas permanece fijo
            en el origen. \\

            Ahora, si $c_x \neq 0$, entonces, si $x$ sigue la cónica $|x| + \prodescalar{e}{x} = k$,
            con $e \in \R^3$ y $k>0$, como el centro de masas está en el origen $m_1 x + m_2 y = 0 \Longrightarrow
            y = -x$, y entonces $y$ sigue la cónica $|y| + \prodescalar{-e}{y} = k$. \\

            En efecto
            $$c_x = x(0) \land \dot{x}(0) = \begin{vmatrix}
                1 & 0 & 0 \\
                0 & v & 0\\
                i & j & k
            \end{vmatrix} = (0, 0, v) \Longrightarrow |c_x| = v > 0$$

            El problema de Kepler que sigue $x$ es conocido

            $$\ddot{x} = - \dfrac{Gm}{4} \dfrac{x}{|x|^3} = - \mu \dfrac{x}{|x|^3}$$

            y basta imponer que la energía total sea positiva para que la trayectoria sea hiperbólica.
            $$h = \dfrac{1}{2} |\dot{x}(0)|^2 - \dfrac{\mu}{|x(0)|} = \dfrac{1}{2} v^2 - \dfrac{Gm}{4}$$

            Como la trayectoria es hiperbólica si y solo si $|e| > 1 \iff h > 0$, entonces
            
            \begin{equation}\label{eq:ec2}
                h > 0 \iff \dfrac{1}{2} v^2 - \dfrac{Gm}{4} > 0 \iff \dfrac{1}{2} v^2 > \dfrac{Gm}{4}
                \iff v > \sqrt{\dfrac{Gm}{2}}
            \end{equation}

            Por lo tanto, dada la masa $m > 0$, si $v$ verifica (\ref{eq:ec2}), entonces
            ambos cuerpos siguen una trayectoria hiperbólica (no siguen un movimiento rectilíneo)
            y se alejan infinitamente de un observador que está situado en el origen.



        \end{itemize}
    \end{ejercicio}

    \newpage

    \begin{ejercicio}[2 puntos]
        Desde un observatorio astronómico fijo se está observando el movimiento de dos asteroides aislados
        que se mueven bajo la acción de la ley de gravitación de Newton. Cuando se comienzan las 
        observaciones, sus posiciones respecto al observatorio son $x(0) = (0,1,-1)$, $y(0) = (0,-3,3)$,
        y las velocidades con las que se mueven son $\dot{x}(0) = (1,1,-1)$, $\dot{y}(0) = (1,-3,-1)$.
        Se estima que la masa del asteroide que ocupa la posición $x$ es cuatro veces la del que está en la 
        posición $y$. Se pide:
        \begin{itemize}
            \item[a)] [1] Deduce el comportamiento del centro de masas del sistema que se observa desde la 
            posición del observatorio. \\

            Sabemos por teoría que $\ddot{C}(t) = 0$, o equivalentemente, $C(t) = \alpha + \beta t$, con
            $\alpha = C(t_0)$, $\beta = \dot{C}(t_0)$, siempre que $t_0 \in I$ pertenezca al intervalo
            maximal de definición\footnote{También por el Teorema de Existencia y Unicidad de
            las Ecuaciones Lineales visto en Ecuaciones Diferenciales I}. El enunciado dice cuando ``se comienzan las observaciones'', por lo que 
            ubicamos ahí el origen, y tomamos $t_0 = 0$. Entonces, denotando por $m_1$ a la masa
            del asteroide que ocupa la posición $x$, e igualmente $m_2$ a la masa del asteroide de posición $y$,
            se tiene que $m_1 = 4m_2$. Si $m_2 = m > 0$, se verifica $m_1 + m_2 = 5m$. \\

            Ahora, por definición, el centro de masas es 
            $$C(t) = \dfrac{m_1 x(t) + m_2 y(t)}{m_1 + m_2}$$
            y su derivada es 
            $$\dot{C}(t) = \dfrac{m_1 \dot{x}(t) + m_2 \dot{y}(t)}{m_1 + m_2}$$

            Obtenemos $\alpha = C(0)$ y $\beta = \dot{C}(0)$.
            $$\alpha = C(0) = \dfrac{4 x(0) + y(0)}{5}$$
            Como $4x(0) + y(0) = 4(0,1,-1) + (0,-3,3) = (0,1,-1)$, entonces 
            $$\alpha = \left(0,\dfrac{1}{5},-\dfrac{1}{5} \right)$$
            Análogamente, $4\dot{x}(0) + \dot{y}(0) = 4(1,1,-1) + (1,-3,-1) = (5,1,-5)$, luego
            $$\beta = \dfrac{1}{5}(5,1,-5) = \left(1, \dfrac{1}{5}, -1 \right)$$

            Por lo tanto,

            $$\boxed{C(t) = \left( 0, \dfrac{1}{5}, -\dfrac{1}{5} \right) + \left( 1, \dfrac{1}{5}, -1 \right)t}$$

            \newpage

            \item[b)] [1] Determina el movimiento que se observará desde el centro de masas en función de la masa 
            $m$ del asteroide menor. \\

            Consideramos el sistema con centro de masas fijo en el origen, utilizando el Principio de Relatividad
            de Galileo, $\tilde{x}(t) = x(t) - \alpha - \beta t$, $\tilde{y}(t) = y(t) - \alpha - \beta t$
            donde $C(t) = \alpha + \beta t$ ya calculado en el apartado anterior. \\

            Sabemos que entonces $(\tilde{x}, \tilde{y})$ es solución al mismo problema de los dos cuerpos y el centro de masas en este sistema
            verifica $\tilde{C}(t) \equiv 0$. Consecuentemente 
            $$m_1 \tilde{x} + m_2 \tilde{y} = 0 \Longrightarrow \tilde{y} = - \dfrac{m_1}{m_2} \tilde{x} = - 4 \tilde{x}$$

            Si el asteroide identificado por $x$ sigue la cónica $|x| + \prodescalar{e}{x} = k$ con $e \in \R^3$ y $k>0$
            entonces se ha visto en teoría que el asteroide identificado por $y$ sigue la cónica $|y| + \prodescalar{-e}{y} = 4k$,
            es decir, las dos cónicas son del mismo tipo, y los ejes de excentricidad opuestos. \\

            El asteroide identificado por $x$ (visto en teoría) sigue un problema de Kepler

            $$\ddot{\tilde{x}} = - G\dfrac{m_2^3}{(m_1+m_2)^2} \dfrac{\tilde{x}}{|\tilde{x}|^3} = 
            - G \dfrac{m}{25} \dfrac{\tilde{x}}{|\tilde{x}|^3} = - \mu \dfrac{\tilde{x}}{|\tilde{x}|^3}$$

            Estudiamos uno de los dos asteroides, y el otro ya lo tendremos. Obtenemos el momento angular $c_x$, 
            primero viendo que 
            $$\tilde{x}(0) = x(0) - \alpha = \left(0, \dfrac{4}{5}, - \dfrac{4}{5} \right),
            \quad \dot{\tilde{x}}(0) = \dot{x}(0) - \beta = \left(0, \dfrac{4}{5}, 0 \right)$$ 

            $$c_x = \tilde{x}(0) \land \dot{\tilde{x}}(0) = \begin{vmatrix}
                0 & 4/5 & -4/5 \\
                0 & 4/5 & 0 \\
                i & j & k
            \end{vmatrix} = \left(\dfrac{16}{25}, 0, 0 \right) \Longrightarrow |c_x| \neq 0$$

            Tenemos entonces que ambos cuerpos se moverán sobre una elipse, hipérbola o parábola, con un
            foco en el origen, por la Primera Ley de Kepler. Para determinar la cónica, obtenemos la energía total.

            $$h = \dfrac{1}{2} |\dot{\tilde{x}}(0)|^2 - \dfrac{\mu}{|\tilde{x}(0)|}$$

            Viendo que
            $$|\tilde{x}(0)|^2 = 0^2 + \left( \dfrac{4}{5} \right)^2 + \left( - \dfrac{4}{5} \right)^2 =
            \dfrac{16}{25} + \dfrac{16}{25} = \dfrac{32}{25} \Longrightarrow 
            |\tilde{x}(0)| = \sqrt{\dfrac{32}{25}} = \dfrac{4 \sqrt{2}}{5}$$
            y
            $$|\dot{\tilde{x}}(0)|^2 = 0^2 + \left( \dfrac{4}{5} \right)^2 + 0^2 = \dfrac{16}{25}$$

            Como $$h = \dfrac{1}{2} |\dot{\tilde{x}}(0)|^2 - \dfrac{\mu}{|\tilde{x}(0)|} = 
            \dfrac{1}{2} \cdot \dfrac{16}{25} - \dfrac{Gm}{25} \dfrac{1}{\dfrac{4 \sqrt{2}}{5}} = $$
            $$\dfrac{8}{25} - \dfrac{Gm}{25} \cdot \dfrac{5}{4 \sqrt{2}} = \dfrac{8}{25} - \dfrac{Gm}{20 \sqrt{2}}$$

            el tipo de movimiento visto desde el sistema con centro de masas fijo en el origen dependerá entonces de la 
            masa $m$.
            \begin{itemize}
                \item Será elíptica si y solo si $$h<0 \iff \dfrac{8}{25} - \dfrac{Gm}{20 \sqrt{2}}
                < 0 \iff \dfrac{8}{25} < \dfrac{Gm}{20 \sqrt{2}} \iff \dfrac{32 \sqrt{2}}{5G} = \dfrac{20 \sqrt{2} \cdot 8}{25 G} < m$$
                \item Será parabólica si y solo si
                $$h = 0 \iff m = \dfrac{32 \sqrt{2}}{5G}$$
                \item Será hiperbólica si y solo si
                $$h>0 \iff m < \dfrac{32 \sqrt{2}}{5G}$$
            \end{itemize}

            Y el otro cuerpo seguirá la misma cónica.

        \end{itemize}
    \end{ejercicio}

    \newpage

    \begin{ejercicio}[4 puntos]
        Se consideran tres masas iguales en los puntos $P_1 = (1,0)$, $P_2 = (-1/2, \sqrt{3}/2)$ y 
        $P_3 = (-1/2, -\sqrt{3}/2)$ con velocidades iniciales $\dot{r}_1(0) = (0,1/3)$,
        $\dot{r}_2(0) = (-\sqrt{3}/6, -1/6)$ y $\dot{r}_3(0) = (\sqrt{3}/6, -1/6)$.
        Se supone que sobre estas masas actúa únicamente la atracción que cada una ejerce sobre las otras
        según la Ley de Gravitación Universal.
        \begin{itemize}
            \item[a)] [1] Comprueba que el centro de masas permanece fijo en el origen. \\
            
            Consideramos $m_i = m > 0, i=1,2,3$. Por definición, el centro de masas en el problema de los $n$ cuerpos es $$C(t) \stackrel{def}{=} \dfrac{1}{M} \sum_{i=1}^{n} m_i r_i(t) \quad M = \sum_{j=1}^{n} m_j$$

            Para $n=3$, y las condiciones anteriores, vemos que 

            $$C(t) = \dfrac{1}{3m} \sum_{i=1}^{n} m_i r_i(t) = 
            \dfrac{1}{3\cancel{m}} \cdot \left( \cancel{m} \left((1,0) + 
            \left(\dfrac{-1}{2}, \dfrac{\sqrt{3}}{2} \right) + \left(\dfrac{-1}{2}, \dfrac{-\sqrt{3}}{2} \right) \right) \right) = $$
            $$\dfrac{1}{3} (0,0) = (0,0)$$

            Por tanto, el centro de masas permanece fijo en el origen.

            \item[b)] [1] El movimiento resultante, ¿se puede producir sobre una circunferencia? \\
            
            Para verlo, consideraremos el Teorema de las Soluciones Circulares de Lagrange para el problema de los 
            $3$ cuerpos:

            \begin{teo}\label{teo:t1}
                Sean $z_1, z_2, z_3 \in \R^2$ tres puntos no alineados. Entonces
                $$r_i(t) = R[\omega t] z_i, \quad i=1,2,3$$
                es solución del problema de tres cuerpos si y solo si se cumplen las tres condiciones siguientes:
                \begin{itemize}
                    \item[a)] El centro de masas está en el origen, es decir,
                    $$m_1 z_1 + m_2 z_2 + m_3 z_3 = 0$$
                    \item[b)] Los puntos $z_1, z_2, z_3$ son vértices de un triángulo equilátero de lado $d>0$.
                    \item[c)] $|\omega| = \sqrt{\dfrac{GM}{d^3}}, \quad M = m_1 + m_2 + m_3$  
                \end{itemize}
            \end{teo}

            Denotamos por $z_i \stackrel{not}{=} P_i$ a cada punto del Teorema \ref{teo:t1}:

            \newpage

            \begin{itemize}
                \item[a)] Trivialmente
                $$m((1,0) + (-1/2, \sqrt{3}/2) + (-1/2, -\sqrt{3}/2)) = m (0,0) = (0,0)$$ 
                \item[b)] Hay que comprobar que las longitudes de los tres lados del triángulo sean iguales.
                $$|P_3 - P_1| = |(-1/2, -\sqrt{3}/2) - (1,0)| = |(-3/2, -\sqrt{3}/2)| = 
                \sqrt{\left(-\dfrac{3}{2}\right)^2 + \left(- \dfrac{\sqrt{3}}{2} \right)^2} = $$
                $$\sqrt{\dfrac{9}{4} + \dfrac{3}{4}} = \sqrt{\dfrac{12}{4}} = \sqrt{3}$$
                $$|P_3 - P_2| = |(-1/2, -\sqrt{3}/2) - (-1/2, \sqrt{3}/2)| = |(0, -\sqrt{3})| = \sqrt{3}$$
                $$|P_2 - P_1| = |(-1/2, \sqrt{3}/2) - (1,0)| = |(-3/2, \sqrt{3}/2)| = |(-3/2, -\sqrt{3}/2)| = |P_3-P_1| = \sqrt{3}$$
                \item[c)] Y por último
                
                $$|\omega| = \sqrt{\dfrac{GM}{d^3}} = \sqrt{\dfrac{3Gm}{\sqrt{3}^3}} = \sqrt{\dfrac{\cancel{3}Gm}{\cancel{3}\sqrt{3}}} = \sqrt{\dfrac{Gm}{\sqrt{3}}} = \dfrac{\sqrt{Gm}}{\sqrt[4]{3}} = 3^{-1/4} \sqrt{Gm}$$

                Por lo tanto, en módulo, la velocidad angular a la que debe girar el conjunto para 
                llegar a la solución del problema de los 
                tres cuerpos dependerá de la masa $m$ común a los tres cuerpos, y es
                \begin{center}
                    $\boxed{|\omega| = 3^{-1/4} \sqrt{Gm}}$
                \end{center}
            \end{itemize}
            
            Expresamos entonces $r_i(t) = R[\omega t] z_i, \quad i=1,2,3$. Así, 
            $$\dot{r}_i(t) = \omega J R[\omega t] P_i = \omega J r_i(t) \quad J= \begin{pmatrix}
                0 & -1 \\
                1 & 0
            \end{pmatrix}$$
            En $t=0$, $\dot{r}_i(0) = \omega J P_i$. Para que el movimiento se desarrolle sobre una circunferencia, 
            debe verificarse $\dot{r}_i(0) = \omega J P_i$ con $i=1,2,3$ para un mismo $\omega$. 
            Para ver si tal $\omega$ existe, obtenemos los $JP_i, i=1,2,3$ y vemos 
            si con las velocidades iniciales dadas $\omega$ verifica las tres igualdades. \\
            
            \begin{enumerate}
                \item $JP_1 = \begin{pmatrix}
                    0 & -1 \\
                    1 & 0
                \end{pmatrix} (1,0) = (0,1) \Longrightarrow \dot{r}_1(0) = (0,1/3) = \omega (0,1) \Longrightarrow \omega = \nicefrac{1}{3}$
                \item $JP_2 = \begin{pmatrix}
                    0 & -1 \\
                    1 & 0
                \end{pmatrix} (-1/2, \sqrt{3}/2) = (-\sqrt{3}/2, -1/2) \Longrightarrow \dot{r}_2(0) = (-\sqrt{3}/6, -1/6) = 
                \omega (-\sqrt{3}/2, -1/2) \Longrightarrow \omega = \nicefrac{1}{3}$
                \item $JP_3 = \begin{pmatrix}
                    0 & -1 \\
                    1 & 0
                \end{pmatrix} (-1/2, -\sqrt{3}/2) = (\sqrt{3}/2, -1/2) \Longrightarrow \dot{r}_3(0) = (\sqrt{3}/6, -1/6) = 
                \omega (\sqrt{3}/2, -1/2) \Longrightarrow \omega = \nicefrac{1}{3}$
            \end{enumerate}

            Por lo tanto, a nivel cinemático, sí que puede producirse, al menos inicialmente, un movimiento sobre
            una circunferencia. Si queremos además que la circunferencia no se deforme (el triángulo equilátero
            lo siga siendo a lo largo del tiempo), entonces habría que imponer $m$ de tal manera que 
            $$\dfrac{1}{3} = |\omega| = 3^{-1/4} \sqrt{Gm} \iff 
            \dfrac{1}{9} = 3^{-2/4} G m \iff m = \dfrac{1}{9 \cdot 3 ^{-2/4} \cdot G}$$

            \item[c)] [1] ¿Puede haber colapso total? \\
            
            Buscamos aplicar el Teorema de Sundman, ya que por el apartado a) sabemos que $C(t) \equiv 0$ (permanece fijo en el origen):

            \begin{teo}\label{teo:t2}
                Sea $r = (r_1, \ldots, r_n)$ una solución del problema de $n$ cuerpos con centro de masas fijo en el origen y una colisión (colapso) total. Entonces su momento angular es $c=0$.
            \end{teo}

            Si el momento angular del problema no es cero, entonces no podrá producirse una colisión total. \\

            Por definición, el momento angular en el problema de los $n$ cuerpos es $$c \stackrel{def}{=} \sum_{i=1}^{n} m_i (r_i \land \dot{r}_i)$$

            También sabemos que se verifica la conservación del momento angular en el problema de los $n$ cuerpos, por lo que basta calcularlo en algún instante de tiempo. En este caso, usaremos $t=0$. \\

            \begin{enumerate}
                \item $(1,0) \land (0,1/3) = \begin{vmatrix}
                    1 & 0 & 0 \\
                    0 & 1/3 & 0 \\
                    i & j & k
                \end{vmatrix} = 1/3 k = (0,0,1/3)$
                \item $(-1/2, \sqrt{3}/2) \land (-\sqrt{3}/6, -1/6) = \begin{vmatrix}
                    -1/2 & \sqrt{3}/2 & 0 \\
                    -\sqrt{3}/6 & -1/6 & 0 \\
                    i & j & k
                \end{vmatrix} = (0,0,1/3)$
                \item $(-1/2, -\sqrt{3}/2) \land (\sqrt{3}/6, -1/6) = \begin{vmatrix}
                    -1/2 & -\sqrt{3}/2 & 0 \\
                    \sqrt{3}/6 & -1/6 & 0 \\
                    i & j & k
                \end{vmatrix} = (0,0,1/3)$
            \end{enumerate}

            $$c = \left(0,0,m \cdot (\cancel{3} \cdot 1/\cancel{3}) \right) =
            \left(0,0, m \right) \neq 0$$

            porque $m>0$. Así pues, por el Teorema \ref{teo:t2}, no puede haber colapso total.

            \newpage

            \item[d)] [1] Explica de forma intuitiva cuál crees que puede ser el movimiento resultante. \\
            
            Si $\omega = 1/3$, hay dos opciones.
            \begin{enumerate}
                \item Si $$m = \dfrac{1}{9 \cdot 3 ^{-2/4} \cdot G}$$
                entonces el movimiento se da sobre circunferencias concéntricas centradas en el origen,
                de tal manera que los tres cuerpos forman los vértices de un triángulo equilátero rígido
                de lado $\sqrt{3}$, que no se deforma por ser $r_i(t) = R[\omega t]P_i, i=1,2,3$ solución 
                al problema de los tres cuerpos, y rotan con la misma velocidad angular.
                \item Si $$m \neq \dfrac{1}{9 \cdot 3 ^{-2/4} \cdot G}$$
                entonces, aunque el movimiento inicialmente se pueda dar sobre una circunferencia, con el paso 
                del tiempo el triángulo equilátero se deforma (a priori no se sabe si los cuerpos se acercarán
                o se alejarán). Si se acercan, lo harán de tal manera que nunca se produzca colisión total,
                por lo visto en el apartado c), y si se alejan, cada una divergerá en el infinito.
            \end{enumerate}
        \end{itemize}
    \end{ejercicio}

    \newpage

    \begin{ejercicio}[2 puntos]
        En el problema restringido circular
        \begin{itemize}
            \item [a)] [1] ¿dónde colocarías un satélite y con qué velocidad inicial para asegurarte de 
            que se mantiene siempre a igual distancia de cada una de las masas? \\

            Sabemos por teoría que los puntos de libración $L_4$ y $L_5$ son puntos estables
            en el problema restringido circular, y cada uno forma un triángulo equilátero de lado $1$
            con las primarias, supuestas situadas en $P_1 = (-\mu,0)$, $P_2 = (1-\mu,0)$, con
            $\mu \in ]0,1/2]$. Por lo tanto, basta colocar un satélite en $L_4$ o $L_5$,
            en reposo (con velocidad inicial nula), para asegurarse de que siempre se mantenga a igual
            distancia de cada una de las masas.

            \item [b)] [1] ¿Y si lo que quieres es que no se acerque a una distancia menor de $1$ de una de las primarias? \\
            
            Por teoría sabemos que $L_1$ y $L_3$ están a distancia mayor que $1$ de una de las dos primarias, y 
            además son puntos de equilibrio en el problema restringido circular. Concretamente, 
            si $P_1 = (-\mu,0)$, $P_2 = (1-\mu,0)$, con $\mu \in ]0,1/2]$, entonces:
            \begin{enumerate}
                \item Si queremos que no se acerque a una distancia menor de $1$ de la primaria situada en
                $P_1$, lo colocaríamos desde el reposo en $L_3$.
                \item Si queremos que no se acerque a una distancia menor de $1$ de la primaria situada en
                $P_2$, lo colocaríamos desde el reposo en $L_1$.
            \end{enumerate}
        \end{itemize}
    \end{ejercicio}

\end{document}