\documentclass[12pt]{article}

% Idioma y codificación
\usepackage[spanish, es-tabla]{babel}       %es-tabla para que se titule "Tabla"
\usepackage[utf8]{inputenc}

% Márgenes
\usepackage[a4paper,top=3cm,bottom=2.5cm,left=3cm,right=3cm]{geometry}

% Comentarios de bloque
\usepackage{verbatim}

% Paquetes de links
\usepackage[hidelinks]{hyperref}    % Permite enlaces
\usepackage{url}                    % redirecciona a la web

% Más opciones para enumeraciones
\usepackage{enumitem}

% Personalizar la portada
\usepackage{titling}

% Paquetes de tablas
\usepackage{multirow}


%------------------------------------------------------------------------

%Paquetes de figuras
\usepackage{caption}
\usepackage{subcaption} % Figuras al lado de otras
\usepackage{float}      % Poner figuras en el sitio indicado H.


% Paquetes de imágenes
\usepackage{graphicx}       % Paquete para añadir imágenes
\usepackage{transparent}    % Para manejar la opacidad de las figuras

% Paquete para usar colores
\usepackage[dvipsnames]{xcolor}
\usepackage{pagecolor}      % Para cambiar el color de la página

% Habilita tamaños de fuente mayores
\usepackage{fix-cm}

% Para los gráficos
\usepackage{tikz}

% Para poder situar los nodos en los grafos
\usetikzlibrary{positioning}


%------------------------------------------------------------------------

% Paquetes de matemáticas
\usepackage{mathtools, amsfonts, amssymb, mathrsfs}
\usepackage[makeroom]{cancel}     % Simplificar tachando
\usepackage{polynom}    % Divisiones y Ruffini
\usepackage{units} % Para poner fracciones diagonales con \nicefrac

\usepackage{pgfplots}   %Representar funciones
\pgfplotsset{compat=1.18}  % Versión 1.18

\usepackage{tikz-cd}    % Para usar diagramas de composiciones
\usetikzlibrary{calc}   % Para usar cálculo de coordenadas en tikz

%Definición de teoremas, etc.
\usepackage{amsthm}
%\swapnumbers   % Intercambia la posición del texto y de la numeración

\theoremstyle{plain}

\makeatletter
\@ifclassloaded{article}{
  \newtheorem{teo}{Teorema}[section]
}{
  \newtheorem{teo}{Teorema}[chapter]  % Se resetea en cada chapter
}
\makeatother

\newtheorem{coro}{Corolario}[teo]           % Se resetea en cada teorema
\newtheorem{prop}[teo]{Proposición}         % Usa el mismo contador que teorema
\newtheorem{lema}[teo]{Lema}                % Usa el mismo contador que teorema

\theoremstyle{remark}
\newtheorem*{observacion}{Observación}

\theoremstyle{definition}

\makeatletter
\@ifclassloaded{article}{
  \newtheorem{definicion}{Definición} [section]     % Se resetea en cada chapter
}{
  \newtheorem{definicion}{Definición} [chapter]     % Se resetea en cada chapter
}
\makeatother

\newtheorem*{notacion}{Notación}
\newtheorem*{ejemplo}{Ejemplo}
\newtheorem*{ejercicio*}{Ejercicio}             % No numerado
\newtheorem{ejercicio}{Ejercicio} [section]     % Se resetea en cada section


% Modificar el formato de la numeración del teorema "ejercicio"
\renewcommand{\theejercicio}{%
  \ifnum\value{section}=0 % Si no se ha iniciado ninguna sección
    \arabic{ejercicio}% Solo mostrar el número de ejercicio
  \else
    \thesection.\arabic{ejercicio}% Mostrar número de sección y número de ejercicio
  \fi
}


% \renewcommand\qedsymbol{$\blacksquare$}         % Cambiar símbolo QED
%------------------------------------------------------------------------

% Paquetes para encabezados
\usepackage{fancyhdr}
\pagestyle{fancy}
\fancyhf{}

\newcommand{\helv}{ % Modificación tamaño de letra
\fontfamily{}\fontsize{12}{12}\selectfont}
\setlength{\headheight}{15pt} % Amplía el tamaño del índice


%\usepackage{lastpage}   % Referenciar última pag   \pageref{LastPage}
\fancyfoot[C]{\thepage}

%------------------------------------------------------------------------

% Conseguir que no ponga "Capítulo 1". Sino solo "1."
\makeatletter
\@ifclassloaded{book}{
  \renewcommand{\chaptermark}[1]{\markboth{\thechapter.\ #1}{}} % En el encabezado
    
  \renewcommand{\@makechapterhead}[1]{%
  \vspace*{50\p@}%
  {\parindent \z@ \raggedright \normalfont
    \ifnum \c@secnumdepth >\m@ne
      \huge\bfseries \thechapter.\hspace{1em}\ignorespaces
    \fi
    \interlinepenalty\@M
    \Huge \bfseries #1\par\nobreak
    \vskip 40\p@
  }}
}
\makeatother

%------------------------------------------------------------------------
% Paquetes de cógido
\usepackage{minted}
\renewcommand\listingscaption{Código fuente}

\usepackage{fancyvrb}
% Personaliza el tamaño de los números de línea
\renewcommand{\theFancyVerbLine}{\small\arabic{FancyVerbLine}}

% Estilo para C++
\newminted{cpp}{
    frame=lines,
    framesep=2mm,
    baselinestretch=1.2,
    linenos,
    escapeinside=||
}

% para minted
\definecolor{LightGray}{rgb}{0.95,0.95,0.92}
\setminted{
    linenos=true,
    stepnumber=5,
    numberfirstline=true,
    autogobble,
    breaklines=true,
    breakautoindent=true,
    breaksymbolleft=,
    breaksymbolright=,
    breaksymbolindentleft=0pt,
    breaksymbolindentright=0pt,
    breaksymbolsepleft=0pt,
    breaksymbolsepright=0pt,
    fontsize=\footnotesize,
    bgcolor=LightGray,
    numbersep=10pt
}


\usepackage{listings} % Para incluir código desde un archivo

\renewcommand\lstlistingname{Código Fuente}
\renewcommand\lstlistlistingname{Índice de Códigos Fuente}

% Definir colores
\definecolor{vscodepurple}{rgb}{0.5,0,0.5}
\definecolor{vscodeblue}{rgb}{0,0,0.8}
\definecolor{vscodegreen}{rgb}{0,0.5,0}
\definecolor{vscodegray}{rgb}{0.5,0.5,0.5}
\definecolor{vscodebackground}{rgb}{0.97,0.97,0.97}
\definecolor{vscodelightgray}{rgb}{0.9,0.9,0.9}

% Configuración para el estilo de C similar a VSCode
\lstdefinestyle{vscode_C}{
  backgroundcolor=\color{vscodebackground},
  commentstyle=\color{vscodegreen},
  keywordstyle=\color{vscodeblue},
  numberstyle=\tiny\color{vscodegray},
  stringstyle=\color{vscodepurple},
  basicstyle=\scriptsize\ttfamily,
  breakatwhitespace=false,
  breaklines=true,
  captionpos=b,
  keepspaces=true,
  numbers=left,
  numbersep=5pt,
  showspaces=false,
  showstringspaces=false,
  showtabs=false,
  tabsize=2,
  frame=tb,
  framerule=0pt,
  aboveskip=10pt,
  belowskip=10pt,
  xleftmargin=10pt,
  xrightmargin=10pt,
  framexleftmargin=10pt,
  framexrightmargin=10pt,
  framesep=0pt,
  rulecolor=\color{vscodelightgray},
  backgroundcolor=\color{vscodebackground},
}

%------------------------------------------------------------------------

% Comandos definidos
\newcommand{\bb}[1]{\mathbb{#1}}
\newcommand{\cc}[1]{\mathcal{#1}}

% I prefer the slanted \leq
\let\oldleq\leq % save them in case they're every wanted
\let\oldgeq\geq
\renewcommand{\leq}{\leqslant}
\renewcommand{\geq}{\geqslant}

% Si y solo si
\newcommand{\sii}{\iff}

% Letras griegas
\newcommand{\eps}{\epsilon}
\newcommand{\veps}{\varepsilon}
\newcommand{\lm}{\lambda}

\newcommand{\ol}{\overline}
\newcommand{\ul}{\underline}
\newcommand{\wt}{\widetilde}
\newcommand{\wh}{\widehat}

\let\oldvec\vec
\renewcommand{\vec}{\overrightarrow}

% Derivadas parciales
\newcommand{\del}[2]{\frac{\partial #1}{\partial #2}}
\newcommand{\Del}[3]{\frac{\partial^{#1} #2}{\partial #3^{#1}}}
\newcommand{\deld}[2]{\dfrac{\partial #1}{\partial #2}}
\newcommand{\Deld}[3]{\dfrac{\partial^{#1} #2}{\partial #3^{#1}}}


\newcommand{\AstIg}{\stackrel{(\ast)}{=}}
\newcommand{\Hop}{\stackrel{L'H\hat{o}pital}{=}}

\newcommand{\red}[1]{{\color{red}#1}} % Para integrales, destacar los cambios.

% Método de integración
\newcommand{\MetInt}[2]{
    \left[\begin{array}{c}
        #1 \\ #2
    \end{array}\right]
}

% Declarar aplicaciones
% 1. Nombre aplicación
% 2. Dominio
% 3. Codominio
% 4. Variable
% 5. Imagen de la variable
\newcommand{\Func}[5]{
    \begin{equation*}
        \begin{array}{rrll}
            #1:& #2 & \longrightarrow & #3\\
               & #4 & \longmapsto & #5
        \end{array}
    \end{equation*}
}

%------------------------------------------------------------------------


\newcommand{\R}{\mathbb{R}}
\newcommand{\prodescalar}[2]{\langle #1, #2 \rangle}
\newcommand{\ortogonal}[1]{#1^{\perp}}


\begin{document}

    % 1. Foto de fondo
    % 2. Título
    % 3. Encabezado Izquierdo
    % 4. Color de fondo
    % 5. Coord x del titulo
    % 6. Coord y del titulo
    % 7. Fecha

    
    % 1. Foto de fondo
% 2. Título
% 3. Encabezado Izquierdo
% 4. Color de fondo
% 5. Coord x del titulo
% 6. Coord y del titulo
% 7. Fecha

\newcommand{\portada}[7]{

    \portadaBase{#1}{#2}{#3}{#4}{#5}{#6}{#7}
    \portadaBook{#1}{#2}{#3}{#4}{#5}{#6}{#7}
}

\newcommand{\portadaExamen}[7]{

    \portadaBase{#1}{#2}{#3}{#4}{#5}{#6}{#7}
    \portadaArticle{#1}{#2}{#3}{#4}{#5}{#6}{#7}
}




\newcommand{\portadaBase}[7]{

    % Tiene la portada principal y la licencia Creative Commons
    
    % 1. Foto de fondo
    % 2. Título
    % 3. Encabezado Izquierdo
    % 4. Color de fondo
    % 5. Coord x del titulo
    % 6. Coord y del titulo
    % 7. Fecha
    
    
    \thispagestyle{empty}               % Sin encabezado ni pie de página
    \newgeometry{margin=0cm}        % Márgenes nulos para la primera página
    
    
    % Encabezado
    \fancyhead[L]{\helv #3}
    \fancyhead[R]{\helv \nouppercase{\leftmark}}
    
    
    \pagecolor{#4}        % Color de fondo para la portada
    
    \begin{figure}[p]
        \centering
        \transparent{0.3}           % Opacidad del 30% para la imagen
        
        \includegraphics[width=\paperwidth, keepaspectratio]{assets/#1}
    
        \begin{tikzpicture}[remember picture, overlay]
            \node[anchor=north west, text=white, opacity=1, font=\fontsize{60}{90}\selectfont\bfseries\sffamily, align=left] at (#5, #6) {#2};
            
            \node[anchor=south east, text=white, opacity=1, font=\fontsize{12}{18}\selectfont\sffamily, align=right] at (9.7, 3) {\textbf{\href{https://losdeldgiim.github.io/}{Los Del DGIIM}}};
            
            \node[anchor=south east, text=white, opacity=1, font=\fontsize{12}{15}\selectfont\sffamily, align=right] at (9.7, 1.8) {Doble Grado en Ingeniería Informática y Matemáticas\\Universidad de Granada};
        \end{tikzpicture}
    \end{figure}
    
    
    \restoregeometry        % Restaurar márgenes normales para las páginas subsiguientes
    \pagecolor{white}       % Restaurar el color de página
    
    
    \newpage
    \thispagestyle{empty}               % Sin encabezado ni pie de página
    \begin{tikzpicture}[remember picture, overlay]
        \node[anchor=south west, inner sep=3cm] at (current page.south west) {
            \begin{minipage}{0.5\paperwidth}
                \href{https://creativecommons.org/licenses/by-nc-nd/4.0/}{
                    \includegraphics[height=2cm]{assets/Licencia.png}
                }\vspace{1cm}\\
                Esta obra está bajo una
                \href{https://creativecommons.org/licenses/by-nc-nd/4.0/}{
                    Licencia Creative Commons Atribución-NoComercial-SinDerivadas 4.0 Internacional (CC BY-NC-ND 4.0).
                }\\
    
                Eres libre de compartir y redistribuir el contenido de esta obra en cualquier medio o formato, siempre y cuando des el crédito adecuado a los autores originales y no persigas fines comerciales. 
            \end{minipage}
        };
    \end{tikzpicture}
    
    
    
    % 1. Foto de fondo
    % 2. Título
    % 3. Encabezado Izquierdo
    % 4. Color de fondo
    % 5. Coord x del titulo
    % 6. Coord y del titulo
    % 7. Fecha


}


\newcommand{\portadaBook}[7]{

    % 1. Foto de fondo
    % 2. Título
    % 3. Encabezado Izquierdo
    % 4. Color de fondo
    % 5. Coord x del titulo
    % 6. Coord y del titulo
    % 7. Fecha

    % Personaliza el formato del título
    \pretitle{\begin{center}\bfseries\fontsize{42}{56}\selectfont}
    \posttitle{\par\end{center}\vspace{2em}}
    
    % Personaliza el formato del autor
    \preauthor{\begin{center}\Large}
    \postauthor{\par\end{center}\vfill}
    
    % Personaliza el formato de la fecha
    \predate{\begin{center}\huge}
    \postdate{\par\end{center}\vspace{2em}}
    
    \title{#2}
    \author{\href{https://losdeldgiim.github.io/}{Los Del DGIIM}}
    \date{Granada, #7}
    \maketitle
    
    \tableofcontents
}




\newcommand{\portadaArticle}[7]{

    % 1. Foto de fondo
    % 2. Título
    % 3. Encabezado Izquierdo
    % 4. Color de fondo
    % 5. Coord x del titulo
    % 6. Coord y del titulo
    % 7. Fecha

    % Personaliza el formato del título
    \pretitle{\begin{center}\bfseries\fontsize{42}{56}\selectfont}
    \posttitle{\par\end{center}\vspace{2em}}
    
    % Personaliza el formato del autor
    \preauthor{\begin{center}\Large}
    \postauthor{\par\end{center}\vspace{3em}}
    
    % Personaliza el formato de la fecha
    \predate{\begin{center}\huge}
    \postdate{\par\end{center}\vspace{5em}}
    
    \title{#2}
    \author{\href{https://losdeldgiim.github.io/}{Los Del DGIIM}}
    \date{Granada, #7}
    \thispagestyle{empty}               % Sin encabezado ni pie de página
    \maketitle
    \vfill
}
    \portadaExamen{ffccA4.jpg}{Mecánica Celeste\\Examen V}{Mecánica Celeste. Examen V}{MidnightBlue}{-8}{28}{2025}{José Manuel Sánchez Varbas}

    \begin{description}
        \item[Asignatura] Mecánica Celeste.
        \item[Curso Académico] 2024-25.
        \item[Grado] Grado en Matemáticas.
        \item[Grupo] A.
        \item[Profesor] Margarita Arias López.
        \item[Descripción] Segundo Parcial.
        \item[Fecha] 13 de Diciembre de 2024.
        \item[Duración] 1 hora y 30 minutos.
    
    \end{description}
    \newpage


    % ------------------------------------
    
    El número entre corchetes es la puntuación máxima de cada ejercicio o apartado.

    \begin{ejercicio}[2 puntos]
        Pon un ejemplo razonado de un problema de dos cuerpos
        \begin{itemize}
            \item[a)] [1] en el que las dos masas colisionen, pero que el movimiento no se produzca en una recta.
            \item[b)] [1] en el que las dos masas se muevan sobre una circunferencia.
        \end{itemize}
    \end{ejercicio}

    \begin{ejercicio}[2 puntos]
        Se consideran tres masas iguales, $m_1 = m_2 = m_3 = 1/3G$, situadas inicialmente en los puntos
        $P_1 = (1,0)$, $P_2 = (-1/2, \sqrt{3}/2)$ y $P_3 = (-1/2, -\sqrt{3}/2)$.
        \begin{itemize}
            \item[a)] [1] ¿A qué velocidad angular tiene que girar el conjunto para que las funciones $r_i(t) = R[\omega t]P_i, i=1,2,3$
            constituyan una solución del problema de los tres cuerpos?\footnote{Como es habitual, denotamos por $R[\theta]$ a la matriz
            del giro de ángulo $\theta$}
            \item[b)] [1] En las condiciones del apartado anterior, determina $\dot{r}_i(0), i=1,2,3$. 
        \end{itemize}
    \end{ejercicio}

    \begin{ejercicio}[4 puntos]
        En el ejercicio anterior, suponemos que las velocidades iniciales de las masas son $\dot{r}_1(0) = (0,1/3)$,
        $\dot{r}_2(0) = (-\sqrt{3}/6, -1/6)$ y $\dot{r}_3(0) = (\sqrt{3}/6, -1/6)$.
        \begin{itemize}
            \item[a)] [1] Comprueba que el centro de masas permanece fijo en el origen.
            \item[b)] [1] ¿Se puede producir el movimiento sobre una circunferencia?
            \item[c)] [1] ¿Puede haber colapso total?
            \item[d)] [1] Explica de forma intuitiva lo que crees que puede suceder en este caso.    
        \end{itemize}
    \end{ejercicio}

    \newpage

    \begin{ejercicio}[2 puntos]
        La Figura \ref{fig:ej41}
        \begin{figure}[H]
            \begin{center}
                \includegraphics[scale=0.4]{./Aux/fig424252.png}
                \caption{Situación Inicial del Ejercicio 4}
                \label{fig:ej41}
            \end{center}
        \end{figure}
        representa $\{z \in \R^2 \setminus \{P_1, P_2\} : \Phi(z) = C\}$ para cierto valor de $C>0$, con
        $$2 \Phi(z) = |z|^2 + |P_1 - z|^{-1} + |P_2 - z|^{-1} + 1/4,$$
        correspondiente al problema restringido circular con dos primarias de masas iguales.
        \begin{itemize}
            \item [a)] [1] ¿Dónde colocarías un satélite y con qué velocidad inicial para asegurarte de que no se separa de la primaria
            situada en $P_2$?
            \item [b)] [1] ¿Y si lo que quieres es que no se acerque a ninguna de las primarias?
        \end{itemize}
        Razona tus respuestas.
    \end{ejercicio}

    \newpage

    \setcounter{ejercicio}{0}

    \begin{ejercicio}[2 puntos]
        Pon un ejemplo razonado de un problema de dos cuerpos
        \begin{itemize}
            \item[a)] [1] en el que las dos masas colisionen, pero que el movimiento no se produzca en una recta. \\
            
            Consideramos el centro de masas fijo inicialmente, así como $m_1 = m_2 = m > 0$. Entonces,
            por teoría sabemos que cada cuerpo sigue una trayectoria propia de un campo de fuerzas central newtoniano:

            \begin{equation}\label{eq:ec1}
                \ddot{x} = - \mu \dfrac{x}{|x|^3}, \quad \mu = G \dfrac{m_2^3}{(m_1 + m_2)^2} = 
                G \dfrac{m^3}{4m^2} = \dfrac{Gm}{4}
            \end{equation}

            Además sabemos que $y(t) = -x(t)$
            (por estar el centro de masas fijo en el origen). Sea $e = (0,1,0)$ y sea el instante de colisión 
            $\omega = 1$. Definimos, para $I = [0, \omega[$
            \Func{r}{I}{\R}{t}{r(t) \stackrel{def}{=} \left( \dfrac{9}{2} \mu \right)^{1/3} (1-t)^{2/3}}
            de tal manera que $x(t) = r(t) e$ e $y(t) = -x(t) = -r(t)e$
            
            Entonces $x(t) = r(t)e \to 0$ cuando $t \to \omega$, y análogo con $y(t)$. Además, $r(t)$ verifica (\ref{eq:ec1})

            $$\dot{r}(t) = - \dfrac{2}{3} \left( \dfrac{9}{2} \mu \right)^{1/3} (1-t)^{-1/3} $$
            $$\ddot{r}(t) = \left[ - \dfrac{2}{3} \left( \dfrac{9}{2} \mu \right)^{1/3} \right]
            \dfrac{1}{3} (1-t)^{-4/3} = - \dfrac{2}{9} \left( \dfrac{9}{2} \mu \right)^{1/3} (1-t)^{-4/3}$$
            $$r(t)^2 = \left( \dfrac{9}{2} \mu \right)^{2/3} (1-t)^{4/3}$$

            y

            $$- \dfrac{\mu}{r(t)^2} = -\mu \cdot \left( \dfrac{9}{2} \mu \right)^{-2/3} (1-t)^{-4/3}$$

            Ahora, vemos que 

            $$\mu \cdot \left( \dfrac{9}{2} \mu \right)^{-2/3} = \mu^{1-2/3} \left( \dfrac{9}{2} \right) = 
            \mu^{1/3} \left( \dfrac{2}{9} \right)^{2/3}$$

            y como 

            $$\left(\dfrac{2}{9} \right)^{2/3} = \dfrac{2^{2/3}}{9^{2/3}} = \dfrac{2^{2/3}}{3^{4/3}} = 
            \dfrac{2^{1}}{3^2} \dfrac{2^{2/3 -1}}{3^{4/3-2}} = \dfrac{2}{9} \dfrac{2^{-1/3}}{3^{-2/3}} =
            \dfrac{2}{9} \left( \dfrac{9}{2} \right)^{1/3}$$

            tenemos que

            $$\mu \cdot \left( \dfrac{9}{2} \mu \right)^{-2/3} = 
            \mu^{1/3} \left( \dfrac{2}{9} \right)^{2/3} = \mu^{1/3} \dfrac{2}{9} \left( \dfrac{9}{2} \right)^{1/3} = 
            \dfrac{2}{9} \left( \dfrac{9}{2} \mu \right)^{1/3} $$

            por lo que, juntando todo

            $$- \mu \dfrac{r(t)}{|r(t)|^3} \stackrel{(*)}{=} - \dfrac{\mu}{r(t)^2} = -\mu \cdot \left( \dfrac{9}{2} \mu \right)^{-2/3} (1-t)^{-4/3} = 
            - \dfrac{2}{9} \left( \dfrac{9}{2} \mu \right)^{1/3} (1-t)^{-4/3} = \ddot{r}(t)$$

            donde en $(*)$ se ha usado que $r(t) > 0$ si $t \in I$. Comprobamos ahora que $x = x(t)$ es solución de (\ref{eq:ec1})

            $$\ddot{x}(t) = \ddot{r}(t)e = - \dfrac{\mu}{r(t)^2} e = - \mu \dfrac{r(t)e}{r(t)^3} = 
            - \mu \dfrac{x(t)}{|x(t)|^3} $$

            donde se ha usado que
            $|x(t)| = |r(t)e| = |r(t)||e| \stackrel{|e|=1}{=} |r(t)| = r(t)$, y esto último por lo mismo que se ha utilizado en $(*)$. \\

            Entonces, para que el movimiento no quede contenido en una recta, pero se mantenga la colisión, usamos el Principio
            de Relatividad de Galileo. Tomando $\alpha = 0, \beta = (1,0,0) \nparallel (0,1,0) = e, \alpha, \beta \in 
            \R^3$, se tiene que
            $$\tilde{x}(t) = x(t) + \alpha + \beta t = r(t)e + \alpha + \beta t = (t , r(t), 0)$$
            $$\tilde{y}(t) = y(t) + \alpha + \beta t = -r(t)e + \alpha + \beta t = (t , -r(t), 0)$$

            Como $(\tilde{x} - \tilde{y})(t) = (0, 2r(t), 0) \to 0$ cuando $t \to \omega$, puesto que 
            $r(t) \to 0$ cuando $t \to \omega$, se sigue produciendo la colisión, pero las trayectorias
            de cada cuerpo ya no son rectilíneas, porque la velocidad es $\dot{\tilde{x}}(t) = \dot{r}(t)e + \beta$,
            y como $\dot{r}(t)$ depende del tiempo (no es constante), la dirección del vector
            velocidad cambia con $t$, y consecuentemente las trayectorias
            de cada cuerpo no estarán contenidas en ninguna una recta, y lo mismo ocurre con $\dot{\tilde{y}}(t) = -\dot{r}(t)e + \beta$. \\

            Nótese que ya no es cierto que $\tilde{y}(t) = -\tilde{x}(t)$, dado que antes el centro
            de masas estaba fijo en el origen, pero ahora $\tilde{C}(t) = C(t) + \alpha + \beta t = \alpha + \beta t$, con $\beta = (1,0,0) \neq 0$.

            \newpage

            \item[b)] [1] en el que las dos masas se muevan sobre una circunferencia. \\
            
            Volvemos a considerar $m_1 = m_2 = m > 0$, y trabajamos con el centro de masas en 
            el origen ($C(t) \equiv 0$). Entonces $y(t) = -x(t)$. Nuevamente tenemos un problema de Kepler
            como en el apartado anterior

            $$\ddot{x} = - \mu \dfrac{x}{|x|^3}, \quad \mu = \dfrac{Gm}{4}$$

            Ahora buscamos una órbita circular de radio $r$, de la forma $x_r(t) = r (\cos (\omega t), \sen (\omega t))$
            y sabemos que $x_r(t)$ es solución de $\ddot{x} = -\mu x / |x|^3$
            si y solo si $$|\omega| = \dfrac{\sqrt{\mu}}{r^{3/2}}$$

            Para hacer las cuentas lo más sencillas posibles, tomamos $r=1$, e imponemos $$\mu = 1 \iff \dfrac{Gm}{4} = 1
            \iff m = \dfrac{4}{G}$$
            De esta manera, podemos considerar las soluciones en el plano (por la invarianza frente a isometrías de
            las soluciones)
            $$x(t) = (\cos t, \sen t)$$
            $$y(t) = -x(t) = (- \cos t, - \sen t)$$
            Entonces, ambas masas se mueven sobre la misma circunferencia de radio $1$ centrada en el origen
            (en el que se encuentra el centro de masas fijo).
        \end{itemize}
    \end{ejercicio}

    \newpage

    \begin{ejercicio}[2 puntos]
        Se consideran tres masas iguales, $m_1 = m_2 = m_3 = 1/3G$, situadas inicialmente en los puntos
        $P_1 = (1,0)$, $P_2 = (-1/2, \sqrt{3}/2)$ y $P_3 = (-1/2, -\sqrt{3}/2)$.
        \begin{itemize}
            \item[a)] [1] ¿A qué velocidad angular tiene que girar el conjunto para que las funciones $r_i(t) = R[\omega t]P_i, i=1,2,3$
            constituyan una solución del problema de los tres cuerpos?\footnote{Como es habitual, denotamos por $R[\theta]$ a la matriz
            del giro de ángulo $\theta$}

            Buscamos aplicar el teorema siguiente:

            \begin{teo}\label{teo:t1}
                Sean $z_1, z_2, z_3 \in \R^2$ tres puntos no alineados. Entonces
                $$r_i(t) = R[\omega t] z_i, \quad i=1,2,3$$
                es solución del problema de tres cuerpos si y solo si se cumplen las tres condiciones siguientes:
                \begin{itemize}
                    \item[a)] El centro de masas está en el origen, es decir,
                    $$m_1 z_1 + m_2 z_2 + m_3 z_3 = 0$$
                    \item[b)] Los puntos $z_1, z_2, z_3$ son vértices de un triángulo equilátero de lado $d>0$.
                    \item[c)] $|\omega| = \sqrt{\dfrac{GM}{d^3}}, \quad M = m_1 + m_2 + m_3$  
                \end{itemize}
            \end{teo}

            Los dos primeros apartados deberán cumplirse con los datos dados, y el tercero nos dará la velocidad angular que nos piden.
            Denotamos por $z_i \stackrel{not}{=} P_i$ a cada punto del Teorema \ref{teo:t1}:

            \begin{itemize}
                \item[a)] Trivialmente
                $$\dfrac{1}{3}((1,0) + (-1/2, \sqrt{3}/2) + (-1/2, -\sqrt{3}/2)) = \dfrac{1}{3} G (0,0) = (0,0)$$ 
                \item[b)] Hay que comprobar que las longitudes de los tres lados del triángulo sean iguales.
                $$|P_3 - P_1| = |(-1/2, -\sqrt{3}/2) - | = |(-3/2, -\sqrt{3}/2)| = 
                \sqrt{\left(-\dfrac{3}{2}\right)^2 + \left(- \dfrac{\sqrt{3}}{2} \right)^2} = $$
                $$\sqrt{\dfrac{9}{4} + \dfrac{3}{4}} = \sqrt{\dfrac{12}{4}} = \sqrt{3}$$
                $$|P_3 - P_2| = |(-1/2, -\sqrt{3}/2) - (-1/2, \sqrt{3}/2)| = |(0, -\sqrt{3})| = \sqrt{3}$$
                $$|P_2 - P_1| = |(-1/2, \sqrt{3}/2) - (1,0)| = |(-3/2, \sqrt{3}/2)| = |(-3/2, -\sqrt{3}/2)| = |P_3-P_1| = \sqrt{3}$$
                \item[c)] Y por último
                
                $$|\omega| = \sqrt{\dfrac{GM}{d^3}} = \sqrt{\dfrac{1}{\sqrt{3}^3}} = \sqrt{\dfrac{1}{3\sqrt{3}}} = \sqrt{\dfrac{\sqrt{3}}{9}} = \dfrac{\sqrt[4]{3}}{3} = 3^{-3/4}$$

                Por lo tanto, en módulo, la velocidad angular a la que debe girar el conjunto para llegar a la solución del problema de los 
                tres cuerpos es 
                \begin{center}
                    $\boxed{|\omega| = 3^{-3/4}}$
                \end{center}
            \end{itemize}

            \item[b)] [1] En las condiciones del apartado anterior, determina $\dot{r}_i(0), i=1,2,3$. \\
            
            Ya sabemos por el apartado anterior que $r_i(t) = R[\omega t]P_i, i=1,2,3$, y además sabemos que 
            $\dot{r}_i(t) = \omega J R[\omega t] P_i = \omega J r_i(t)$, con $J= \begin{pmatrix}
                0 & -1 \\
                1 & 0
            \end{pmatrix}$.
            En $t=0$, $\dot{r}_i(0) = \omega J P_i$, de donde
            \begin{enumerate}
                \item $\dot{r}_1(0) = \omega J P_1 = \omega J (1,0) = \omega \begin{pmatrix}
                    0 & -1 \\
                    1 & 0
                \end{pmatrix} (1,0) = \omega (0,1)$ 
                \item $\dot{r}_2(0) = \omega J P_2 = \omega \begin{pmatrix}
                    0 & -1 \\
                    1 & 0
                \end{pmatrix} (-1/2, \sqrt{3}/2) = \omega (-\sqrt{3}/2, -1/2)$
                \item $\dot{r}_3(0) = \omega J P_3 = \omega \begin{pmatrix}
                    0 & -1 \\
                    1 & 0
                \end{pmatrix} (-1/2, -\sqrt{3}/2) = \omega (\sqrt{3}/2, -1/2)$
            \end{enumerate}

            Se ha supuesto que $|\omega| = \omega$, es decir, $\omega = 3^{-3/4}$.
        \end{itemize}
    \end{ejercicio}

    \newpage


    \begin{ejercicio}[4 puntos]
        En el ejercicio anterior, suponemos que las velocidades iniciales de las masas son $\dot{r}_1(0) = (0,1/3)$,
        $\dot{r}_2(0) = (-\sqrt{3}/6, -1/6)$ y $\dot{r}_3(0) = (\sqrt{3}/6, -1/6)$.
        \begin{itemize}
            \item[a)] [1] Comprueba que el centro de masas permanece fijo en el origen. \\
            
            Por definición, el centro de masas en el problema de los $n$ cuerpos es $$C(t) \stackrel{def}{=} \dfrac{1}{M} \sum_{i=1}^{n} m_i r_i(t) \quad M = \sum_{j=1}^{n} m_j$$

            Para $n=3$, y las condiciones anteriores, vemos que 

            $$C(t) = \dfrac{1}{\cancel{3} \cdot \left(\dfrac{1}{\cancel{3}G} \right)} \sum_{i=1}^{n} m_i r_i(t) = 
            \cancel{G} \cdot \left( \dfrac{1}{3\cancel{G}} \left((1,0) + 
            \left(\dfrac{-1}{2}, \dfrac{\sqrt{3}}{2} \right) + \left(\dfrac{-1}{2}, \dfrac{-\sqrt{3}}{2} \right) \right) \right) = $$
            $$\dfrac{1}{3} (0,0) = (0,0)$$

            Por tanto, el centro de masas permanece fijo en el origen.

            \item[b)] [1] ¿Se puede producir el movimiento sobre una circunferencia? \\
            
            Para que el movimiento se desarrolle sobre una circunferencia, debe verificarse $\dot{r}_i(0) = \omega J P_i$ con $i=1,2,3$ para un mismo $\omega$. 
            Para ver si tal $\omega$ existe, usaremos los $JP_i, i=1,2,3$ obtenidos en el ejercicio anterior y veremos si con las velocidades iniciales dadas $\omega$ verifica las tres igualdades. \\
            
            \begin{enumerate}
                \item $JP_1 = (0,1) \Longrightarrow \dot{r}_1(0) = (0,1/3) = \omega (0,1) \Longrightarrow \omega = 1/3$
                \item $JP_2 = (-\sqrt{3}/2, -1/2) \Longrightarrow \dot{r}_2(0) = (-\sqrt{3}/6, -1/6) = \omega (-\sqrt{3}/2, -1/2) \Longrightarrow \omega = 1/3$
                \item $JP_3 = (\sqrt{3}/2, -1/2) \Longrightarrow \dot{r}_3(0) = (\sqrt{3}/6, -1/6) = \omega (\sqrt{3}/2, -1/2) \Longrightarrow \omega = 1/3$
            \end{enumerate}

            Sin embargo, con estas velocidades, $|\omega| = 1/3$, que no coincide con $|\omega| = 3^{-3/4}$, que es la condición para que tengamos una solución
            al problema de los tres cuerpos. Por lo tanto, podemos concluir lo siguiente:
            \begin{center}
                \boxed{
                    \begin{aligned}
                        &\text{Las velocidades iniciales son compatibles con un movimiento} \\
                        &\text{circular rígido (en sentido cinemático), pues existe } \omega=\dfrac{1}{3} \text{ tal } \\
                        &\text{que } \dot r_i(0)=\omega J P_i,\ i=1,2,3. \text{ Sin embargo, dicho movimiento}\\
                        &\text{no es solución del problema de los tres cuerpos, ya que la} \\
                        &\text{condición } |\omega|=\sqrt{\frac{GM}{d^3}} = 3^{-3/4} \text{ no se cumple.}
                    \end{aligned}
                }
            \end{center}

            \newpage

            \item[c)] [1] ¿Puede haber colapso total? \\
            
            Buscamos aplicar el Teorema de Sundman, ya que por el apartado a) sabemos que $C(t) \equiv 0$ (permanece fijo en el origen):

            \begin{teo}\label{teo:t2}
                Sea $r = (r_1, \ldots, r_n)$ una solución del problema de $n$ cuerpos con centro de masas fijo en el origen y una colisión (colapso) total. Entonces su momento angular es $c=0$.
            \end{teo}

            Si el momento angular del problema no es cero, entonces no podrá producirse una colisión total. \\

            Por definición, el momento angular en el problema de los $n$ cuerpos es $$c \stackrel{def}{=} \sum_{i=1}^{n} m_i (r_i \land \dot{r}_i)$$

            También sabemos que se verifica la conservación del momento angular en el problema de los $n$ cuerpos, por lo que basta calcularlo en algún instante de tiempo. En este caso, usaremos $t=0$. \\

            \begin{enumerate}
                \item $(1,0) \land (0,1/3) = \begin{vmatrix}
                    1 & 0 & 0 \\
                    0 & 1/3 & 0 \\
                    i & j & k
                \end{vmatrix} = 1/3 k = (0,0,1/3)$
                \item $(-1/2, \sqrt{3}/2) \land (-\sqrt{3}/6, -1/6) = \begin{vmatrix}
                    -1/2 & \sqrt{3}/2 & 0 \\
                    -\sqrt{3}/6 & -1/6 & 0 \\
                    i & j & k
                \end{vmatrix} = (0,0,1/3)$
                \item $(-1/2, -\sqrt{3}/2) \land (\sqrt{3}/6, -1/6) = \begin{vmatrix}
                    -1/2 & -\sqrt{3}/2 & 0 \\
                    \sqrt{3}/6 & -1/6 & 0 \\
                    i & j & k
                \end{vmatrix} = (0,0,1/3)$
            \end{enumerate}

            $$c = \left(0,0,\dfrac{1}{\cancel{3}G} \cdot (\cancel{3} \cdot 1/3) \right) = \left(0,0, \dfrac{1}{G} \cdot \dfrac{1}{3} \right) = \left(0,0,\dfrac{1}{3G} \right) \neq 0$$

            Así pues, por el Teorema \ref{teo:t2}, no puede haber colapso total.

            \item[d)] [1] Explica de forma intuitiva lo que crees que puede suceder en este caso. \\
            
            Inicialmente, las velocidades hacen que el sistema empiece a girar de forma
            similar a una rotación rígida alrededor del centro de masas.
            Sin embargo, como la velocidad angular no satisface la condición del módulo del 
            Teorema \ref{teo:t1}, para ser solución al problema de los tres cuerpos, 
            el triángulo no permanece rígido: las distancias entre las
            masas varían con el tiempo, produciéndose una deformación progresiva de la
            configuración inicial. El movimiento se asemeja inicialmente al de un triángulo
            girando, pero dicha rotación no se mantiene de forma estable.
        \end{itemize}
    \end{ejercicio}

    \newpage

    \begin{ejercicio}
        
        La Figura \ref{fig:ej41}
        \begin{figure}[H]
            \begin{center}
                \includegraphics[scale=0.4]{./Aux/fig424252.png}
                \caption{Situación Inicial del Ejercicio 4}
            \end{center}
        \end{figure}
        representa $\{z \in \R^2 \setminus \{P_1, P_2\} : \Phi(z) = C\}$ para cierto valor de $C>0$, con
        $$2 \Phi(z) = |z|^2 + |P_1 - z|^{-1} + |P_2 - z|^{-1} + 1/4,$$
        correspondiente al problema restringido circular con dos primarias de masas iguales.
        \begin{itemize}
            \item [a)] [1] ¿Dónde colocarías un satélite y con qué velocidad inicial para asegurarte de que no se separa de la primaria
            situada en $P_2$? \\

            Usaremos las regiones de Hill para abordar el problema, que se muestran en la Figura \ref{fig:ej4}
            \begin{figure}[H]
                \begin{center}
                    \begin{tikzpicture}[scale=3]
                        \shorthandoff{>}

                        % Marco exterior
                        \draw[thick] (-2.2,-2.2) rectangle (2.2,2.2);

                        % Cuadrícula
                        \draw[step=0.25cm,very thin,gray!60] (-2.2,-2.2) grid (2.2,2.2);

                        % Ejes
                        \draw[->,thick] (-2.2,0) -- (2.2,0);
                        \draw[->,thick] (0,-2.2) -- (0,2.2);

                        % Etiquetas numéricas en eje X
                        \foreach \x in {-2,-1,0,1,2}{
                            \draw (\x,0.05) -- (\x,-0.05);
                            \node[below right=-2pt] at (\x,0) {$\x$};
                        }

                        % Etiquetas numéricas en eje Y
                        \foreach \y in {-2,-1,1,2}{
                            \draw (0.05,\y) -- (-0.05,\y);
                            \node[above left=0.5pt] at (0,\y) {$\y$};
                        }

                        % Relleno exterior de R3 con líneas diagonales
                        \begin{scope}
                            \clip (-2.2,-2.2) rectangle (2.2,2.2)
                                (0,0) circle (2);

                            \foreach \k in {-12,-11,...,12}{
                                \draw[blue!70,thin]
                                    (-4, -4 + 0.4*\k) -- (4, 4 + 0.4*\k);
                            }
                        \end{scope}

                        % Círculo grande R3
                        \draw[thick] (0,0) circle (2);
                        \node[blue] at (1.55,1.75) {$R_3$};

                        % Círculo pequeño R1
                        \draw[thick] (-0.5,0) circle (0.2);
                        \fill (-0.5,0) ++(-0.04,-0.04) rectangle ++(0.08,0.08);
                        \node[blue] at (-0.5,0.45) {$R_1$};
                        \node[black] at (-0.5,-0.12) {\scriptsize $P_1$};
                        \draw[->,blue,thick] (-0.5,0.35) to[bend left=30] (-0.5,0.12);

                        % Círculo pequeño R2
                        \draw[thick] (0.5,0) circle (0.2);
                        \fill (0.5,0) ++(-0.04,-0.04) rectangle ++(0.08,0.08);
                        \node[blue] at (0.5,0.45) {$R_2$};
                        \node[black] at (0.5,-0.12) {\scriptsize $P_2$};
                        \draw[->,blue,thick] (0.5,0.35) to[bend left=30] (0.5,0.12);
                    \end{tikzpicture}
                    \caption{Regiones de Hill del Ejercicio 4}
                    \label{fig:ej4}
                \end{center}
            \end{figure}

            Por definición de región de Hill asociada a un nivel $C>0$ $$R(C) = \{z \in \R^2 \setminus
            \{P_1, P_2\} : \Phi(z) \geqslant C\}$$ y por la definición de la constante 
            de Jacobi de una solución $z = z(t)$ del problema restringido circular,
            definida en un intervalo maximal $I$, que es
            $\mathcal{J} = 2 \Phi(z(t)) - |\dot{z}(t)|^2 = cte$,
            entonces como $|\dot{z}(t)|^2 \geqslant 0$, se tiene que $2 \Phi(z(t)) \geqslant \mathcal{J}$.
            Fijando como condición inicial $z(0) = z_0$ y $v = \dot{z}(0)$,
            entonces $\mathcal{J} = 2 \Phi(z(0)) - |v|^2$. Ahora, 
            $$2 \Phi(z(t)) \geqslant \mathcal{J} = 2 \Phi(z(0)) - |v|^2 \quad \forall t \in I
            \iff \Phi(z(t)) \geqslant \Phi(z(0)) - |v|^2 / 2$$
            luego $$z(t) \in R \left(\Phi(z(0)) - \dfrac{|v|^2}{2} \right)$$
            Ahora bien, las componentes conexas de la Región de Hill (en este caso
            hay tres) son $R_1$, $R_2$ y $R_3$. Sabemos por teoría que 
            la trayectoria permanece en la componente conexa donde empieza. \\

            Como $R_2$ es una componente conexa, por el razonamiento
            anterior, $$z(t) \in R_2 \quad \forall t \in I$$ tomando $z_0 \in R_2$ y $\dot{z}(0) = 0$ (se suelta en reposo), de tal manera 
            que el satélite no puede separarse de la primaria situada en $P_2$.

            \item [b)] [1] ¿Y si lo que quieres es que no se acerque a ninguna de las primarias? \\
            
            Procederemos de forma similar, pero con $R_3$. Ubicando el satélite con las mismas condiciones iniciales,
            $z(0) = z_0 \in R_3, \quad \dot{z}(0) = 0$, entonces $$z(t) \in R_3 \quad \forall t \in I$$
            Como puede verse en la Figura \ref{fig:ej4}, $R_3$ está separada de $P_1$ y de $P_2$, 
            luego $\exists \delta > 0$ (basta tomar el mínimo
            entre las dos distancias) tal que $$|z - P_1| \geqslant \delta \quad |z - P_2| \geqslant \delta
            \quad \forall z \in R_3$$
            Consecuentemente $$|z(t) - P_1|, |z(t) - P_2| \geqslant \delta \quad \forall t \in I$$

            puesto que $R_3$ es una componente conexa, 
            por lo que el satélite no se aproxima a ninguna de las dos primarias.


        \end{itemize}
    \end{ejercicio}
    
\end{document}