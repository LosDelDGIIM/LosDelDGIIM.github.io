\documentclass[12pt]{article}

% Idioma y codificación
\usepackage[spanish, es-tabla]{babel}       %es-tabla para que se titule "Tabla"
\usepackage[utf8]{inputenc}

% Márgenes
\usepackage[a4paper,top=3cm,bottom=2.5cm,left=3cm,right=3cm]{geometry}

% Comentarios de bloque
\usepackage{verbatim}

% Paquetes de links
\usepackage[hidelinks]{hyperref}    % Permite enlaces
\usepackage{url}                    % redirecciona a la web

% Más opciones para enumeraciones
\usepackage{enumitem}

% Personalizar la portada
\usepackage{titling}

% Paquetes de tablas
\usepackage{multirow}


%------------------------------------------------------------------------

%Paquetes de figuras
\usepackage{caption}
\usepackage{subcaption} % Figuras al lado de otras
\usepackage{float}      % Poner figuras en el sitio indicado H.


% Paquetes de imágenes
\usepackage{graphicx}       % Paquete para añadir imágenes
\usepackage{transparent}    % Para manejar la opacidad de las figuras

% Paquete para usar colores
\usepackage[dvipsnames]{xcolor}
\usepackage{pagecolor}      % Para cambiar el color de la página

% Habilita tamaños de fuente mayores
\usepackage{fix-cm}

% Para los gráficos
\usepackage{tikz}

% Para poder situar los nodos en los grafos
\usetikzlibrary{positioning}


%------------------------------------------------------------------------

% Paquetes de matemáticas
\usepackage{mathtools, amsfonts, amssymb, mathrsfs}
\usepackage[makeroom]{cancel}     % Simplificar tachando
\usepackage{polynom}    % Divisiones y Ruffini
\usepackage{units} % Para poner fracciones diagonales con \nicefrac

\usepackage{pgfplots}   %Representar funciones
\pgfplotsset{compat=1.18}  % Versión 1.18

\usepackage{tikz-cd}    % Para usar diagramas de composiciones
\usetikzlibrary{calc}   % Para usar cálculo de coordenadas en tikz

%Definición de teoremas, etc.
\usepackage{amsthm}
%\swapnumbers   % Intercambia la posición del texto y de la numeración

\theoremstyle{plain}

\makeatletter
\@ifclassloaded{article}{
  \newtheorem{teo}{Teorema}[section]
}{
  \newtheorem{teo}{Teorema}[chapter]  % Se resetea en cada chapter
}
\makeatother

\newtheorem{coro}{Corolario}[teo]           % Se resetea en cada teorema
\newtheorem{prop}[teo]{Proposición}         % Usa el mismo contador que teorema
\newtheorem{lema}[teo]{Lema}                % Usa el mismo contador que teorema

\theoremstyle{remark}
\newtheorem*{observacion}{Observación}

\theoremstyle{definition}

\makeatletter
\@ifclassloaded{article}{
  \newtheorem{definicion}{Definición} [section]     % Se resetea en cada chapter
}{
  \newtheorem{definicion}{Definición} [chapter]     % Se resetea en cada chapter
}
\makeatother

\newtheorem*{notacion}{Notación}
\newtheorem*{ejemplo}{Ejemplo}
\newtheorem*{ejercicio*}{Ejercicio}             % No numerado
\newtheorem{ejercicio}{Ejercicio} [section]     % Se resetea en cada section


% Modificar el formato de la numeración del teorema "ejercicio"
\renewcommand{\theejercicio}{%
  \ifnum\value{section}=0 % Si no se ha iniciado ninguna sección
    \arabic{ejercicio}% Solo mostrar el número de ejercicio
  \else
    \thesection.\arabic{ejercicio}% Mostrar número de sección y número de ejercicio
  \fi
}


% \renewcommand\qedsymbol{$\blacksquare$}         % Cambiar símbolo QED
%------------------------------------------------------------------------

% Paquetes para encabezados
\usepackage{fancyhdr}
\pagestyle{fancy}
\fancyhf{}

\newcommand{\helv}{ % Modificación tamaño de letra
\fontfamily{}\fontsize{12}{12}\selectfont}
\setlength{\headheight}{15pt} % Amplía el tamaño del índice


%\usepackage{lastpage}   % Referenciar última pag   \pageref{LastPage}
\fancyfoot[C]{\thepage}

%------------------------------------------------------------------------

% Conseguir que no ponga "Capítulo 1". Sino solo "1."
\makeatletter
\@ifclassloaded{book}{
  \renewcommand{\chaptermark}[1]{\markboth{\thechapter.\ #1}{}} % En el encabezado
    
  \renewcommand{\@makechapterhead}[1]{%
  \vspace*{50\p@}%
  {\parindent \z@ \raggedright \normalfont
    \ifnum \c@secnumdepth >\m@ne
      \huge\bfseries \thechapter.\hspace{1em}\ignorespaces
    \fi
    \interlinepenalty\@M
    \Huge \bfseries #1\par\nobreak
    \vskip 40\p@
  }}
}
\makeatother

%------------------------------------------------------------------------
% Paquetes de cógido
\usepackage{minted}
\renewcommand\listingscaption{Código fuente}

\usepackage{fancyvrb}
% Personaliza el tamaño de los números de línea
\renewcommand{\theFancyVerbLine}{\small\arabic{FancyVerbLine}}

% Estilo para C++
\newminted{cpp}{
    frame=lines,
    framesep=2mm,
    baselinestretch=1.2,
    linenos,
    escapeinside=||
}

% para minted
\definecolor{LightGray}{rgb}{0.95,0.95,0.92}
\setminted{
    linenos=true,
    stepnumber=5,
    numberfirstline=true,
    autogobble,
    breaklines=true,
    breakautoindent=true,
    breaksymbolleft=,
    breaksymbolright=,
    breaksymbolindentleft=0pt,
    breaksymbolindentright=0pt,
    breaksymbolsepleft=0pt,
    breaksymbolsepright=0pt,
    fontsize=\footnotesize,
    bgcolor=LightGray,
    numbersep=10pt
}


\usepackage{listings} % Para incluir código desde un archivo

\renewcommand\lstlistingname{Código Fuente}
\renewcommand\lstlistlistingname{Índice de Códigos Fuente}

% Definir colores
\definecolor{vscodepurple}{rgb}{0.5,0,0.5}
\definecolor{vscodeblue}{rgb}{0,0,0.8}
\definecolor{vscodegreen}{rgb}{0,0.5,0}
\definecolor{vscodegray}{rgb}{0.5,0.5,0.5}
\definecolor{vscodebackground}{rgb}{0.97,0.97,0.97}
\definecolor{vscodelightgray}{rgb}{0.9,0.9,0.9}

% Configuración para el estilo de C similar a VSCode
\lstdefinestyle{vscode_C}{
  backgroundcolor=\color{vscodebackground},
  commentstyle=\color{vscodegreen},
  keywordstyle=\color{vscodeblue},
  numberstyle=\tiny\color{vscodegray},
  stringstyle=\color{vscodepurple},
  basicstyle=\scriptsize\ttfamily,
  breakatwhitespace=false,
  breaklines=true,
  captionpos=b,
  keepspaces=true,
  numbers=left,
  numbersep=5pt,
  showspaces=false,
  showstringspaces=false,
  showtabs=false,
  tabsize=2,
  frame=tb,
  framerule=0pt,
  aboveskip=10pt,
  belowskip=10pt,
  xleftmargin=10pt,
  xrightmargin=10pt,
  framexleftmargin=10pt,
  framexrightmargin=10pt,
  framesep=0pt,
  rulecolor=\color{vscodelightgray},
  backgroundcolor=\color{vscodebackground},
}

%------------------------------------------------------------------------

% Comandos definidos
\newcommand{\bb}[1]{\mathbb{#1}}
\newcommand{\cc}[1]{\mathcal{#1}}

% I prefer the slanted \leq
\let\oldleq\leq % save them in case they're every wanted
\let\oldgeq\geq
\renewcommand{\leq}{\leqslant}
\renewcommand{\geq}{\geqslant}

% Si y solo si
\newcommand{\sii}{\iff}

% Letras griegas
\newcommand{\eps}{\epsilon}
\newcommand{\veps}{\varepsilon}
\newcommand{\lm}{\lambda}

\newcommand{\ol}{\overline}
\newcommand{\ul}{\underline}
\newcommand{\wt}{\widetilde}
\newcommand{\wh}{\widehat}

\let\oldvec\vec
\renewcommand{\vec}{\overrightarrow}

% Derivadas parciales
\newcommand{\del}[2]{\frac{\partial #1}{\partial #2}}
\newcommand{\Del}[3]{\frac{\partial^{#1} #2}{\partial #3^{#1}}}
\newcommand{\deld}[2]{\dfrac{\partial #1}{\partial #2}}
\newcommand{\Deld}[3]{\dfrac{\partial^{#1} #2}{\partial #3^{#1}}}


\newcommand{\AstIg}{\stackrel{(\ast)}{=}}
\newcommand{\Hop}{\stackrel{L'H\hat{o}pital}{=}}

\newcommand{\red}[1]{{\color{red}#1}} % Para integrales, destacar los cambios.

% Método de integración
\newcommand{\MetInt}[2]{
    \left[\begin{array}{c}
        #1 \\ #2
    \end{array}\right]
}

% Declarar aplicaciones
% 1. Nombre aplicación
% 2. Dominio
% 3. Codominio
% 4. Variable
% 5. Imagen de la variable
\newcommand{\Func}[5]{
    \begin{equation*}
        \begin{array}{rrll}
            #1:& #2 & \longrightarrow & #3\\
               & #4 & \longmapsto & #5
        \end{array}
    \end{equation*}
}

%------------------------------------------------------------------------


\usetikzlibrary{arrows.meta,decorations.markings}

\newcommand{\R}{\mathbb{R}}
\newcommand{\prodescalar}[2]{\langle #1, #2 \rangle}
\newcommand{\ortogonal}[1]{#1^{\perp}}

\begin{document}

    % 1. Foto de fondo
    % 2. Título
    % 3. Encabezado Izquierdo
    % 4. Color de fondo
    % 5. Coord x del titulo
    % 6. Coord y del titulo
    % 7. Fecha

    
    % 1. Foto de fondo
% 2. Título
% 3. Encabezado Izquierdo
% 4. Color de fondo
% 5. Coord x del titulo
% 6. Coord y del titulo
% 7. Fecha

\newcommand{\portada}[7]{

    \portadaBase{#1}{#2}{#3}{#4}{#5}{#6}{#7}
    \portadaBook{#1}{#2}{#3}{#4}{#5}{#6}{#7}
}

\newcommand{\portadaExamen}[7]{

    \portadaBase{#1}{#2}{#3}{#4}{#5}{#6}{#7}
    \portadaArticle{#1}{#2}{#3}{#4}{#5}{#6}{#7}
}




\newcommand{\portadaBase}[7]{

    % Tiene la portada principal y la licencia Creative Commons
    
    % 1. Foto de fondo
    % 2. Título
    % 3. Encabezado Izquierdo
    % 4. Color de fondo
    % 5. Coord x del titulo
    % 6. Coord y del titulo
    % 7. Fecha
    
    
    \thispagestyle{empty}               % Sin encabezado ni pie de página
    \newgeometry{margin=0cm}        % Márgenes nulos para la primera página
    
    
    % Encabezado
    \fancyhead[L]{\helv #3}
    \fancyhead[R]{\helv \nouppercase{\leftmark}}
    
    
    \pagecolor{#4}        % Color de fondo para la portada
    
    \begin{figure}[p]
        \centering
        \transparent{0.3}           % Opacidad del 30% para la imagen
        
        \includegraphics[width=\paperwidth, keepaspectratio]{assets/#1}
    
        \begin{tikzpicture}[remember picture, overlay]
            \node[anchor=north west, text=white, opacity=1, font=\fontsize{60}{90}\selectfont\bfseries\sffamily, align=left] at (#5, #6) {#2};
            
            \node[anchor=south east, text=white, opacity=1, font=\fontsize{12}{18}\selectfont\sffamily, align=right] at (9.7, 3) {\textbf{\href{https://losdeldgiim.github.io/}{Los Del DGIIM}}};
            
            \node[anchor=south east, text=white, opacity=1, font=\fontsize{12}{15}\selectfont\sffamily, align=right] at (9.7, 1.8) {Doble Grado en Ingeniería Informática y Matemáticas\\Universidad de Granada};
        \end{tikzpicture}
    \end{figure}
    
    
    \restoregeometry        % Restaurar márgenes normales para las páginas subsiguientes
    \pagecolor{white}       % Restaurar el color de página
    
    
    \newpage
    \thispagestyle{empty}               % Sin encabezado ni pie de página
    \begin{tikzpicture}[remember picture, overlay]
        \node[anchor=south west, inner sep=3cm] at (current page.south west) {
            \begin{minipage}{0.5\paperwidth}
                \href{https://creativecommons.org/licenses/by-nc-nd/4.0/}{
                    \includegraphics[height=2cm]{assets/Licencia.png}
                }\vspace{1cm}\\
                Esta obra está bajo una
                \href{https://creativecommons.org/licenses/by-nc-nd/4.0/}{
                    Licencia Creative Commons Atribución-NoComercial-SinDerivadas 4.0 Internacional (CC BY-NC-ND 4.0).
                }\\
    
                Eres libre de compartir y redistribuir el contenido de esta obra en cualquier medio o formato, siempre y cuando des el crédito adecuado a los autores originales y no persigas fines comerciales. 
            \end{minipage}
        };
    \end{tikzpicture}
    
    
    
    % 1. Foto de fondo
    % 2. Título
    % 3. Encabezado Izquierdo
    % 4. Color de fondo
    % 5. Coord x del titulo
    % 6. Coord y del titulo
    % 7. Fecha


}


\newcommand{\portadaBook}[7]{

    % 1. Foto de fondo
    % 2. Título
    % 3. Encabezado Izquierdo
    % 4. Color de fondo
    % 5. Coord x del titulo
    % 6. Coord y del titulo
    % 7. Fecha

    % Personaliza el formato del título
    \pretitle{\begin{center}\bfseries\fontsize{42}{56}\selectfont}
    \posttitle{\par\end{center}\vspace{2em}}
    
    % Personaliza el formato del autor
    \preauthor{\begin{center}\Large}
    \postauthor{\par\end{center}\vfill}
    
    % Personaliza el formato de la fecha
    \predate{\begin{center}\huge}
    \postdate{\par\end{center}\vspace{2em}}
    
    \title{#2}
    \author{\href{https://losdeldgiim.github.io/}{Los Del DGIIM}}
    \date{Granada, #7}
    \maketitle
    
    \tableofcontents
}




\newcommand{\portadaArticle}[7]{

    % 1. Foto de fondo
    % 2. Título
    % 3. Encabezado Izquierdo
    % 4. Color de fondo
    % 5. Coord x del titulo
    % 6. Coord y del titulo
    % 7. Fecha

    % Personaliza el formato del título
    \pretitle{\begin{center}\bfseries\fontsize{42}{56}\selectfont}
    \posttitle{\par\end{center}\vspace{2em}}
    
    % Personaliza el formato del autor
    \preauthor{\begin{center}\Large}
    \postauthor{\par\end{center}\vspace{3em}}
    
    % Personaliza el formato de la fecha
    \predate{\begin{center}\huge}
    \postdate{\par\end{center}\vspace{5em}}
    
    \title{#2}
    \author{\href{https://losdeldgiim.github.io/}{Los Del DGIIM}}
    \date{Granada, #7}
    \thispagestyle{empty}               % Sin encabezado ni pie de página
    \maketitle
    \vfill
}
    \portadaExamen{ffccA4.jpg}{Mecánica Celeste\\Examen VIII}{Mecánica Celeste. Examen VIII}{MidnightBlue}{-8}{28}{2025}{José Manuel Sánchez Varbas}

    \begin{description}
        \item[Asignatura] Mecánica Celeste.
        \item[Curso Académico] 2025-26.
        \item[Grado] Grado en Matemáticas.
        \item[Grupo] A.
        \item[Profesor] Margarita Arias López.
        \item[Descripción] Segundo Parcial.
        \item[Fecha] 18 de Diciembre de 2025.
        \item[Duración] 1 hora y 30 minutos.  
    \end{description}
    \newpage


    % ------------------------------------
    
    El número entre corchetes es la puntuación máxima de cada ejercicio o apartado.

    \begin{ejercicio}[4 puntos]
        Se consideran dos cuerpos sobre los que actúa únicamente la fuerza de gravitación Newtoniana. Determina razonadamente si son verdaderas o falsas las siguientes afirmaciones:
        \begin{itemize}
            \item[a)] [1] Los dos cuerpos se mueven siempre en un mismo plano.
            \item[b)] [1] Si el centro de masas de los dos cuerpos permanece fijo en el origen y su momento angular es cero, se mueven sobre una recta.
            \item[c)] [1] Si los dos cuerpos tienen la misma masa $m = 4/G$ y en $x(0) = (0,0,1)$, $\dot{x}(0) = (0,1,0)$, los cuerpos se alejan infinitamente de un observador situado sobre su centro de masas.
            \item[d)] [1] Si los dos cuerpos tienen la misma masa $m = 4/G$ y en un instante $x(t_0) = (0,0,1)$, $\dot{x}(t_0) = (0,1,0)$, el otro puede estar en $y(t_0) = (0,0,-1)$ con velocidad $\dot{y}(t_0) = (0,-2,0)$    
        \end{itemize}
    \end{ejercicio}

    \begin{ejercicio}[3 puntos]
        Se consideran tres cuerpos de igual masa $m$ sobre los que actúa únicamente la fuerza de gravitación Newtoniana.
        \begin{itemize}
            \item[a)] [1] Determina su masa si cada cuerpo gira en un mismo plano a velocidad angular $3$ alrededor del origen y $|r_i(t) - r_j(t)| = 1 \quad i \neq j$ para todo $t \in \R$
            \item[b)] [2] Supongamos que los tres cuerpos se encuentran siempre a la misma distancia, esto es, $|r_i(t) - r_j(t)| = d, \quad i \neq j$, ¿podemos afirmar que el movimiento se efectúa sobre un plano?
            En el supuesto de que se muevan en un mismo plano, ¿tiene que girar alrededor del origen a velocidad angular constante? De ser posible, pon un ejemplo concreto en cada caso.  
        \end{itemize}
    \end{ejercicio}

    \begin{ejercicio}[3 puntos]
        En un problema de tres cuerpos restringido circular se sabe que la masa de la primaria mayor es $5$ veces la de la primaria más pequeña.
        \begin{itemize}
            \item[a)] [1.5] Determina los puntos de libración $L_4$ y $L_5$ y haz un esbozo de las tres masas si situamos un satélite de masa despreciable sobre $L_5$ con velocidad cero.
            \item[b)] [1.5] Las curvas
            
            \begin{center}
                \begin{tikzpicture}
                    \begin{axis}[
                        width=16cm,height=8cm,
                        axis lines=middle,
                        xmin=-1.5, xmax=1.5,
                        ymin=-1.5, ymax=1.5,
                        axis equal image,
                        grid=both,
                        minor tick num=4,
                        enlargelimits=false,
                        xlabel={$x$}, ylabel={$y$},
                        xtick={-1,1},
                        ytick={-1,1},
                        axis background/.style={fill=white},
                        axis line style={black, line width=1.0pt},
                        tick style={black},
                        ticklabel style={text=black, font=\large},
                        grid style={black!20}
                        ]
                        \pgfmathsetmacro{\m}{1/6}

                        \addplot[black, thick] table {./Aux/hill_arriba.dat};
                        \addplot[black, thick] table {./Aux/hill_abajo.dat};

                        \addplot[only marks, mark=*, mark size=1.8pt]
                        coordinates {(-\m,0) (1-\m,0)};
                        \node[below, text=black] at (axis cs:-\m,0) {$P_1$};
                        \node[below, text=black] at (axis cs:1-\m,0) {$P_2$};

                        \node[anchor=west, xshift=2pt, text=black, font=\large] at (axis cs:-0.05,-0.16) {$0$};
                    \end{axis}
                \end{tikzpicture}
            \end{center}

            representan $\{z \in \R^2 \setminus \{P_1, P_2\} : \Phi(z) = C\}$ para cierto valor de $C > 0$, con 
            $$
            \Phi(z) = \frac{1}{2}|z|^2 + \frac{1-\mu}{|z - P_1|} + \frac{\mu}{|z - P_2|} + \frac{1}{2}\mu(1-\mu),
            $$
            y $\mu$ el valor resultante en el caso que estamos considerando. ¿Podemos obtener alguna información sobre la trayectoria de un satélite si lo soltamos dentro de la región de arriba con velocidad nula?  
        \end{itemize}
        Razona tus respuestas.
    \end{ejercicio}

    \newpage

    \setcounter{ejercicio}{0}

    \begin{ejercicio}[4 puntos]
        Se consideran dos cuerpos sobre los que actúa únicamente la fuerza de gravitación Newtoniana. Determina razonadamente si son verdaderas o falsas las siguientes afirmaciones:
        \begin{itemize}
            \item[a)] [1] Los dos cuerpos se mueven siempre en un mismo plano. \\
            
            Consideramos $m_1 = m_2 = m > 0$. Sabemos por teoría que las trayectorias 

            $$x(t) = (\cos \omega t, \sen \omega t, t), \quad y(t) = (-\cos \omega t, - \sen \omega t, t)$$

            son solución al problema de dos cuerpos si $\omega$ se escoge de forma apropiada, y las órbitas son
            helicoidales. Por lo tanto, la afirmación es \boxed{\text{falsa}}.

            \item[b)] [1] Si el centro de masas de los dos cuerpos permanece fijo en el origen y su momento angular es cero, se mueven sobre una recta. \\
            
            Por teoría sabemos que si el centro de masas está fijo en el origen pero $c_x = 0$ entonces el 
            movimiento de ambas partículas será rectilíneo. Esto se justifica porque el problema 
            de los dos cuerpos se reduce a dos problemas de Kepler, y ahí ya se estudió que si el momento angular 
            es $0$, entonces el movimiento es rectilíneo. Como el centro de masas está en la envolvente
            convexa de sus vértices de posición (en el caso de dos cuerpos, el segmento que los une) y 
            siempre permanece fijo en el origen, entonces necesariamente ambos cuerpos deben moverse sobre una recta.
            Por lo tanto, la afirmación es \boxed{\text{verdadera}}.

            \item[c)] [1] Si los dos cuerpos tienen la misma masa $m = 4/G$ y en $x(0) = (0,0,1)$, $\dot{x}(0) = (0,1,0)$, 
            los cuerpos se alejan infinitamente de un observador situado sobre su centro de masas. \\

            Basta estudiar el cuerpo identificado por $x$, y el otro quedará determinado. Además,
            podemos suponer que el centro de masas está fijo en el origen, ya que el observador
            se encuentra situado sobre su centro de masas. Primero hallamos el
            momento angular

            $$c(0) = x(0) \land \dot{x}(0) = \begin{vmatrix}
                0 & 0 & 1 \\
                0 & 1 & 0 \\
                i & j & k
            \end{vmatrix} = (1, 0, 0) \Longrightarrow |c| = 1 > 0$$

            Como $c \neq 0$, por la Primera Ley de Kepler, $x$ se moverá sobre una cónica; elipse, hipérbola o parábola,
            con un foco en el origen. Además, si la cónica que verifica $x$ sigue la ecuación 
            $|x| + \prodescalar{e}{x} = k$ con $e \in \R^3$ y $k > 0$, entonces la cónica que 
            verifica $y$ es $|y| + \prodescalar{-e}{y} = \lambda k$ con $\lambda = \frac{m_1}{m_2} x$ 
            (viene de suponer que el centro de masas está en el origen
            $m_1 x + m_2 y = 0$). \\

            Para que los cuerpos se alejen infinitamente del centro de masas, la energía total debe ser 
            no negativa (para que la trayectoria no esté acotada). \\
            
            El problema de Kepler que cumple $x$ verifica $\mu = \frac{Gm}{4}$, luego la energía total será
            $$h = \dfrac{1}{2} |\dot{x}(0)|^2 - \dfrac{\mu}{|x(0)|} = \dfrac{1}{2} - \dfrac{Gm}{4} = 
            \dfrac{1}{2} - \dfrac{G}{4} \dfrac{4}{G} = \dfrac{1}{2} - 1 = - \dfrac{1}{2} = - 0.5 < 0$$

            Sabemos que $h < 0 \iff |e| < 1$, por lo que la trayectoria que sigue $x$ será elíptica, así como
            la de $y$, y, en particular, los cuerpos \textbf{no} se alejan infinitamente de un observador situado sobre su centro de masas.
            Por lo tanto, la afirmación es \boxed{\text{falsa}}.

            \item[d)] [1] Si los dos cuerpos tienen la misma masa $m = 4/G$ y en un instante $x(t_0) = (0,0,1)$, 
            $\dot{x}(t_0) = (0,1,0)$, el otro puede estar en $y(t_0) = (0,0,-1)$ con velocidad $\dot{y}(t_0) = (0,-2,0)$ \\
            
            Sabemos que el conjunto $\Omega = (\R^3 \times \R^3) \setminus \Delta$, con $\Delta = \{(\xi, \xi) : \xi \in \R^3\}$
            es un dominio (abierto, conexo y no vacío). Además, como $x(t_0) \neq y(t_0)$, dado que el campo de 
            fuerzas es regular, existe una solución del sistema (función regular) \Func{(x,y)}{I \subset \R}{\Omega}{t}{(x(t),y(t))}
            Por lo tanto, la afirmación es \boxed{\text{verdadera}}.
        \end{itemize}
    \end{ejercicio}

    \newpage

    \begin{ejercicio}[3 puntos]
        Se consideran tres cuerpos de igual masa $m$ sobre los que actúa únicamente la fuerza de gravitación Newtoniana.
        \begin{itemize}
            \item[a)] [1] Determina su masa si cada cuerpo gira en un mismo plano a velocidad angular $3$ alrededor del origen y $|r_i(t) - r_j(t)| = 1 \quad i \neq j$ para todo $t \in \R$ \\
            
            Utilizaremos el siguiente teorema:

            \begin{teo}\label{teo:t1}
                Sean $z_1, z_2, z_3 \in \R^2$ tres puntos no alineados. Entonces
                $$r_i(t) = R[\omega t] z_i, \quad i=1,2,3$$
                es solución del problema de tres cuerpos si y solo si se cumplen las tres condiciones siguientes:
                \begin{itemize}
                    \item[a)] El centro de masas está en el origen, es decir,
                    $$m_1 z_1 + m_2 z_2 + m_3 z_3 = 0$$
                    \item[b)] Los puntos $z_1, z_2, z_3$ son vértices de un triángulo equilátero de lado $d>0$.
                    \item[c)] $|\omega| = \sqrt{\dfrac{GM}{d^3}}, \quad M = m_1 + m_2 + m_3$  
                \end{itemize}
            \end{teo}

            La hipótesis del Teorema \ref{teo:t1} se cumple, puesto que $|r_i(t) - r_j(t)| = 1 \quad i \neq j$ para todo $t \in \R$ impone necesariamente que los tres cuerpos formen un triángulo equilátero de lado $1$.
            En particular, los tres cuerpos no están alineados. También por hipótesis, como se ``consideran tres cuerpos de igual masa $m$ sobre los que actúa únicamente la fuerza de gravitación Newtoniana''
            se está diciendo que estos tres cuerpos son solución al problema de los tres cuerpos. \\

            Así, la masa nos la dará la condición c), dado que ya conocemos el módulo de la velocidad angular, y que los tres cuerpos tienen la misma masa, pongamos $m_i = m > 0 \quad i=1,2,3$, de donde $M = m_1 + m_2 + m_3 = 3m$. La obtenemos

            $3 = |\omega| = \sqrt{\dfrac{GM}{d^3}} = \sqrt{\dfrac{3Gm}{1^3}} = \sqrt{3Gm} \iff 9 = 3 G m \iff m = \dfrac{9}{3G} = \dfrac{3}{G}$

            \newpage

            \item[b)] [2] Supongamos que los tres cuerpos se encuentran siempre a la misma distancia, esto es, $|r_i(t) - r_j(t)| = d, \quad i \neq j$, ¿podemos afirmar que el movimiento se efectúa sobre un plano?
            En el supuesto de que se muevan en un mismo plano, ¿tiene que girar alrededor del origen a velocidad angular constante? De ser posible, pon un ejemplo concreto en cada caso. \\
            
            La primera pregunta se responde con sí. Como $|r_i(t) - r_j(t)| = d > 0, \quad i \neq j$, los tres cuerpos forman en cada instante de tiempo $t$ un triángulo equilátero. 
            Como la distancia entre cuerpos es fija siempre e igual a $d$, el movimiento es rígido. Además, por teoría sabemos que el momento angular total en el problema de los $n$ cuerpos se conserva, 
            lo que necesariamente implica que el movimiento se realiza mediante una rotación rígida alrededor de un eje fijo que pasa por el centro de masas. 
            Por tanto, las trayectorias quedan contenidas en un plano fijo perpendicular a dicho eje. \\

            Un ejemplo concreto serían las soluciones circulares de Lagrange, que son las que tienen la forma del Teorema \ref{teo:t1}. \\

            Para la segunda pregunta, la respuesta es que no. La velocidad angular sí que debe ser constante, en el caso de que $|r_i(t) - r_j(t)| = d, \quad i \neq j$ para todo $t \in \R$. De otra manera las distancias no se
            mantendrían constantes entre cuerpos. Sin embargo, la rotación no tiene por qué ser alrededor del origen, dado que el eje de rotación pasa por el centro de masas, y este último puede no coincidir con el origen. \\

            Por ejemplo, considerando las soluciones circulares de Lagrange desplazadas, $r_i(t) = a + R[\omega t] z_i, \quad i=1,2,3$, para un cierto $a \neq 0$ vector constante, la solución sigue siendo circular de Lagrange
            al problema de los tres cuerpos (verifica todas las hipótesis del Teorema \ref{teo:t1}), pero la rotación no se produce alrededor del origen. 
        \end{itemize}
    \end{ejercicio}

    \newpage

    \begin{ejercicio}[3 puntos]
        En un problema de tres cuerpos restringido circular se sabe que la masa de la primaria mayor es $5$ veces la de la primaria más pequeña.
        \begin{itemize}
            \item[a)] [1.5] Determina los puntos de libración $L_4$ y $L_5$ y haz un esbozo de las tres masas si situamos un satélite de masa despreciable sobre $L_5$ con velocidad cero. \\
            
            Sabemos que $m_1 = 1 - \mu$, $m_2 = \mu$, $\mu \in ]0, 1/2]$ y $m_1 + m_2 = 1$. Como ``la masa de la primaria mayor es $5$ veces la de la primaria más pequeña'', entonces $m_1 = 5 m_2$ y $m_1 + m_2 = 6m_2 > 0$, 
            por lo que la masa $\mu$ en las unidades apropiadas será

            $$\mu = \dfrac{m_2}{m_1 + m_2} = 
            \dfrac{\cancel{m_2}}{6\cancel{m_2}} = \dfrac{1}{6}$$

            Por teoría sabemos que tanto $L_4$ como $L_5$, colocándolos como vértices, forman un triángulo equilátero de lado $1$ con las primarias. Por lo tanto, 
            los puntos de libración $L_4$ y $L_5$ son aquellos $z \in \mathbb{R}^2$ que verifican $$|z - P_1| = |z-P_2| = |P_1 - P_2| = 1$$
            Deducimos entonces que la abscisa de $L_4$ y $L_5$ está en la mediatriz de las primarias, es decir:
            $$M = \dfrac{P_1+P_2}{2} = \left(\dfrac{-\mu + 1 - \mu}{2},0 \right) = \left(\dfrac{1 - 2 \mu}{2},0 \right) = \left(\dfrac{1}{2} - \mu,0 \right)$$

            Denotando por $z = (x,y)$, entonces $x = \nicefrac{1}{2} - \mu$. \\

            Para obtener la altura, imponemos $|z-P_1| = 1$ (también se podría imponer $|z-P_2| = 1$). Como 
            $$z-P_1 = \left( \dfrac{1}{2} - \mu - (-\mu),y \right) = \left( \dfrac{1}{2},y \right)$$
            Entonces 
            $$|z-P_1| = 1 \iff |z-P_1|^2 = 1 \iff \left( \dfrac{1}{2} \right)^2 + y^2 = 1 \iff y^2 = 1 - \dfrac{1}{4} = \dfrac{3}{4} \iff y = \pm \dfrac{\sqrt{3}}{2}$$

            Consecuentemente 

            $$L_4 = \left( \dfrac{1}{2} - \mu, \dfrac{\sqrt{3}}{2} \right), \quad L_5 = \left( \dfrac{1}{2} - \mu, -\dfrac{\sqrt{3}}{2} \right)$$

            Sustituyendo $\mu = \nicefrac{1}{6}$

            $$L_4 = \left( \dfrac{1}{3}, \dfrac{\sqrt{3}}{2} \right), \quad L_5 = \left( \dfrac{1}{3}, -\dfrac{\sqrt{3}}{2} \right)$$

            \newpage

            El esbozo de las tres masas si situamos un satélite de masa despreciable sobre $L_5$ con velocidad cero sería el siguiente:

            \begin{center}
                \begin{tikzpicture}[scale=2.7, line cap=round, line join=round]
                    \shorthandoff{>}
                    \definecolor{cM1}{RGB}{120, 190, 230}   % m1 (azul)
                    \definecolor{cM2}{RGB}{ 60, 200, 110}   % m2 (verde)
                    \definecolor{cL4}{RGB}{230, 140,  40}   % L4 (naranja)
                    \definecolor{cL5}{RGB}{255, 255,   0}   % L5 (amarillo)
                    \definecolor{tri}{RGB}{210, 70,  210}   % triángulo (magenta)

                    % --- origen ---
                    \coordinate (O) at (0,0);

                    % --- primarias ---
                    \coordinate (m1) at ( 0.70,  0.32);
                    \coordinate (m2) at (-1.1264624985, -0.4948832686);

                    % --- puntos triangulares (lado 2 con las primarias) ---
                    \coordinate (L4) at (-0.918940861,  1.494321288); % vértice superior
                    \coordinate (L5) at ( 0.4924783624, -1.6692045571); % vértice inferior

                    % --- radios reales para que las circunferencias concéntricas cuadren con los puntos ---
                    \pgfmathsetmacro{\rMone}{veclen(0.70,0.32)}
                    \pgfmathsetmacro{\rMtwo}{veclen(-1.1264624985,-0.4948832686)}
                    \pgfmathsetmacro{\rLfour}{veclen(-0.918940861,1.494321288)}
                    \pgfmathsetmacro{\rLfive}{veclen(0.4924783624,-1.6692045571)}

                    % --- órbitas (centradas en O) ---
                    \draw[dashed, line width=0.9pt, cM1, dash pattern=on 5pt off 4pt]
                    (O) circle[radius=\rMone];

                    \draw[dashed, line width=0.9pt, cM2, dash pattern=on 5pt off 4pt]
                    (O) circle[radius=\rMtwo];

                    \draw[dashed, line width=0.9pt, red!70!black, dash pattern=on 5pt off 4pt]
                    (O) circle[radius=\rLfive]
                    node[pos=0.10, below right, white] {$L_5$};

                    % --- triángulos punteados (dos equiláteros) ---
                    \draw[dashed, tri, line width=1.0pt, dash pattern=on 4pt off 3pt]
                    (m1)--(L4)--(m2)--cycle;

                    \draw[dashed, tri, line width=1.0pt, dash pattern=on 4pt off 3pt]
                    (m1)--(L5)--(m2)--cycle;

                    % --- puntos ---
                    \fill[cM1] (m1) circle[radius=0.18];
                    \fill[cM2] (m2) circle[radius=0.09];
                    \fill[cL4] (L4) circle[radius=0.045];
                    \fill[cL5] (L5) circle[radius=0.045];
                    \fill[black] (O)  circle[radius=0.03];

                    % --- etiquetas ---
                    \node[black] at ($(m1)+(0.2,0.2)$) {$m_1$};
                    \node[black] at ($(m2)+(-0.15,-0.15)$) {$m_2$};
                    \node[black] at ($(L4)+(0.10,0.15)$) {$L_4$};
                    \node[black] at ($(L5)+(0.12,-0.12)$) {$L_5$};

                    % --- flechas de velocidad (sistema inercial, opcional) ---
                    \draw[-{Stealth[length=2mm, width=2mm]}, black, thick]
                    (0.6,0.48) to[bend right=7] (0.47,0.6); % m_1

                    \draw[-{Stealth[length=2mm, width=2mm]}, black, thick]
                    (-1.085,-0.575) to[bend right=3] (-0.985,-0.74); % m_2

                    \draw[-{Stealth[length=2mm, width=2mm]}, black, thick]
                    (-0.95,1.46) to[bend right=3] (-1.08,1.36); % L_4

                    \draw[-{Stealth[length=2mm, width=2mm]}, black, thick]
                    (0.535,-1.665) to[bend right=5] (0.65,-1.6); % L_5

                \end{tikzpicture}
            \end{center}

            En el sistema de referencia en rotación, los puntos $L_4$ y $L_5$ son equilibrio, luego un satélite situado en ellos con velocidad inicial nula permanece en reposo. 
            En el sistema inercial correspondiente, las primarias y el satélite describen órbitas circulares con la misma velocidad angular, formando un triángulo equilátero rígido de lado $1$. 

            \newpage

            \item[b)] [1.5] Las curvas 
            
            \begin{center}
                \begin{tikzpicture}
                    \begin{axis}[
                        width=16cm,height=8cm,
                        axis lines=middle,
                        xmin=-1.5, xmax=1.5,
                        ymin=-1.5, ymax=1.5,
                        axis equal image,
                        grid=both,
                        minor tick num=4,
                        enlargelimits=false,
                        xlabel={$x$}, ylabel={$y$},
                        xtick={-1,1},
                        ytick={-1,1},
                        axis background/.style={fill=white},
                        axis line style={black, line width=1.0pt},
                        tick style={black},
                        ticklabel style={text=black, font=\large},
                        grid style={black!20}
                        ]
                        \pgfmathsetmacro{\m}{1/6}

                        \addplot[black, thick] table {./Aux/hill_arriba.dat};
                        \addplot[black, thick] table {./Aux/hill_abajo.dat};

                        \addplot[only marks, mark=*, mark size=1.8pt]
                        coordinates {(-\m,0) (1-\m,0)};
                        \node[below, text=black] at (axis cs:-\m,0) {$P_1$};
                        \node[below, text=black] at (axis cs:1-\m,0) {$P_2$};

                        \node[anchor=west, xshift=2pt, text=black, font=\large] at (axis cs:-0.05,-0.16) {$0$};
                    \end{axis}
                \end{tikzpicture}
            \end{center}

            representan $\{z \in \R^2 \setminus \{P_1, P_2\} : \Phi(z) = C\}$ para cierto valor de $C > 0$, con 
            $$
            \Phi(z) = \frac{1}{2}|z|^2 + \frac{1-\mu}{|z - P_1|} + \frac{\mu}{|z - P_2|} + \frac{1}{2}\mu(1-\mu),
            $$
            y $\mu$ el valor resultante en el caso que estamos considerando. ¿Podemos obtener alguna información 
            sobre la trayectoria de un satélite si lo soltamos dentro de la región de arriba con velocidad nula? \\
            
            La ``región de arriba'' se denota por $R_1$ en la Figura \ref{fig:ej3b}

            \begin{figure}[H]
                \begin{center}
                    \begin{tikzpicture}
                        \begin{axis}[
                            width=16cm,height=8cm,
                            axis lines=middle,
                            xmin=-1.5, xmax=1.5,
                            ymin=-1.5, ymax=1.5,
                            axis equal image,
                            grid=both,
                            minor tick num=4,
                            enlargelimits=false,
                            xlabel={$x$}, ylabel={$y$},
                            xtick={-1,1},
                            ytick={-1,1},
                            axis background/.style={fill=white},
                            axis line style={black, line width=1.0pt},
                            tick style={black},
                            ticklabel style={text=black, font=\large},
                            grid style={black!20}
                            ]
                            \pgfmathsetmacro{\m}{1/6}
    
                            \addplot[black, thick] table {./Aux/hill_arriba.dat};
                            \addplot[black, thick] table {./Aux/hill_abajo.dat};
    
                            \addplot[only marks, mark=*, mark size=1.8pt]
                            coordinates {(-\m,0) (1-\m,0)};
                            \node[below, text=black] at (axis cs:-\m,0) {$P_1$};
                            \node[below, text=black] at (axis cs:1-\m,0) {$P_2$};
    
                            \node[anchor=west, xshift=2pt, text=black, font=\large] at (axis cs:-0.05,-0.16) {$0$};

                            \node[anchor=west, xshift=2pt, text=black, font=\large] at (axis cs:0.45,0.7) {$R_1$};

                            \node[anchor=west, xshift=2pt, text=black, font=\large] at (axis cs:0.45,-0.7) {$R_2$};
                        \end{axis}
                    \end{tikzpicture}
                    \caption{Regiones de Hill del Ejercicio 3b)}
                    \label{fig:ej3b}
                \end{center}
            \end{figure}

            Teniendo en cuenta que $R_1$ es una componente conexa de la región de Hill asociada a $C$,
            por teoría sabemos que si una trayectoria empieza en una componente conexa, termina en la misma 
            componente conexa. De esta manera, si se suelta el satélite en reposo (con velocidad nula), 
            entonces está garantizado que la trayectoria del satélite quedará contenida en $R_1$.
        \end{itemize}
    \end{ejercicio}

\end{document}