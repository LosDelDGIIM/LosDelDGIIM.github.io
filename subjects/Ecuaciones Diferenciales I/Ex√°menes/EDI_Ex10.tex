\documentclass[12pt]{article}

% Idioma y codificación
\usepackage[spanish, es-tabla, es-notilde]{babel}       %es-tabla para que se titule "Tabla"
\usepackage[utf8]{inputenc}

% Márgenes
\usepackage[a4paper,top=3cm,bottom=2.5cm,left=3cm,right=3cm]{geometry}

% Comentarios de bloque
\usepackage{verbatim}

% Paquetes de links
\usepackage[hidelinks]{hyperref}    % Permite enlaces
\usepackage{url}                    % redirecciona a la web

% Más opciones para enumeraciones
\usepackage{enumitem}

% Personalizar la portada
\usepackage{titling}

% Paquetes de tablas
\usepackage{multirow}

% Para añadir el símbolo de euro
\usepackage{eurosym}


%------------------------------------------------------------------------

%Paquetes de figuras
\usepackage{caption}
\usepackage{subcaption} % Figuras al lado de otras
\usepackage{float}      % Poner figuras en el sitio indicado H.


% Paquetes de imágenes
\usepackage{graphicx}       % Paquete para añadir imágenes
\usepackage{transparent}    % Para manejar la opacidad de las figuras

% Paquete para usar colores
\usepackage[dvipsnames, table, xcdraw]{xcolor}
\usepackage{pagecolor}      % Para cambiar el color de la página

% Habilita tamaños de fuente mayores
\usepackage{fix-cm}

% Para los gráficos
\usepackage{tikz}
\usepackage{forest}

% Para poder situar los nodos en los grafos
\usetikzlibrary{positioning}


%------------------------------------------------------------------------

% Paquetes de matemáticas
\usepackage{mathtools, amsfonts, amssymb, mathrsfs}
\usepackage[makeroom]{cancel}     % Simplificar tachando
\usepackage{polynom}    % Divisiones y Ruffini
\usepackage{units} % Para poner fracciones diagonales con \nicefrac

\usepackage{pgfplots}   %Representar funciones
\pgfplotsset{compat=1.18}  % Versión 1.18

\usepackage{tikz-cd}    % Para usar diagramas de composiciones
\usetikzlibrary{calc}   % Para usar cálculo de coordenadas en tikz

%Definición de teoremas, etc.
\usepackage{amsthm}
%\swapnumbers   % Intercambia la posición del texto y de la numeración

\theoremstyle{plain}

\makeatletter
\@ifclassloaded{article}{
  \newtheorem{teo}{Teorema}[section]
}{
  \newtheorem{teo}{Teorema}[chapter]  % Se resetea en cada chapter
}
\makeatother

\newtheorem{coro}{Corolario}[teo]           % Se resetea en cada teorema
\newtheorem{prop}[teo]{Proposición}         % Usa el mismo contador que teorema
\newtheorem{lema}[teo]{Lema}                % Usa el mismo contador que teorema
\newtheorem*{lema*}{Lema}

\theoremstyle{remark}
\newtheorem*{observacion}{Observación}

\theoremstyle{definition}

\makeatletter
\@ifclassloaded{article}{
  \newtheorem{definicion}{Definición} [section]     % Se resetea en cada chapter
}{
  \newtheorem{definicion}{Definición} [chapter]     % Se resetea en cada chapter
}
\makeatother

\newtheorem*{notacion}{Notación}
\newtheorem*{ejemplo}{Ejemplo}
\newtheorem*{ejercicio*}{Ejercicio}             % No numerado
\newtheorem{ejercicio}{Ejercicio} [section]     % Se resetea en cada section


% Modificar el formato de la numeración del teorema "ejercicio"
\renewcommand{\theejercicio}{%
  \ifnum\value{section}=0 % Si no se ha iniciado ninguna sección
    \arabic{ejercicio}% Solo mostrar el número de ejercicio
  \else
    \thesection.\arabic{ejercicio}% Mostrar número de sección y número de ejercicio
  \fi
}


% \renewcommand\qedsymbol{$\blacksquare$}         % Cambiar símbolo QED
%------------------------------------------------------------------------

% Paquetes para encabezados
\usepackage{fancyhdr}
\pagestyle{fancy}
\fancyhf{}

\newcommand{\helv}{ % Modificación tamaño de letra
\fontfamily{}\fontsize{12}{12}\selectfont}
\setlength{\headheight}{15pt} % Amplía el tamaño del índice


%\usepackage{lastpage}   % Referenciar última pag   \pageref{LastPage}
%\fancyfoot[C]{%
%  \begin{minipage}{\textwidth}
%    \centering
%    ~\\
%    \thepage\\
%    \href{https://losdeldgiim.github.io/}{\texttt{\footnotesize losdeldgiim.github.io}}
%  \end{minipage}
%}
\fancyfoot[C]{\thepage}
\fancyfoot[R]{\href{https://losdeldgiim.github.io/}{\texttt{\footnotesize losdeldgiim.github.io}}}

%------------------------------------------------------------------------

% Conseguir que no ponga "Capítulo 1". Sino solo "1."
\makeatletter
\@ifclassloaded{book}{
  \renewcommand{\chaptermark}[1]{\markboth{\thechapter.\ #1}{}} % En el encabezado
    
  \renewcommand{\@makechapterhead}[1]{%
  \vspace*{50\p@}%
  {\parindent \z@ \raggedright \normalfont
    \ifnum \c@secnumdepth >\m@ne
      \huge\bfseries \thechapter.\hspace{1em}\ignorespaces
    \fi
    \interlinepenalty\@M
    \Huge \bfseries #1\par\nobreak
    \vskip 40\p@
  }}
}
\makeatother

%------------------------------------------------------------------------
% Paquetes de cógido
\usepackage{minted}
\renewcommand\listingscaption{Código fuente}

\usepackage{fancyvrb}
% Personaliza el tamaño de los números de línea
\renewcommand{\theFancyVerbLine}{\small\arabic{FancyVerbLine}}

% Estilo para C++
\newminted{cpp}{
    frame=lines,
    framesep=2mm,
    baselinestretch=1.2,
    linenos,
    escapeinside=||
}

% para minted
\definecolor{LightGray}{rgb}{0.95,0.95,0.92}
\setminted{
    linenos=true,
    stepnumber=5,
    numberfirstline=true,
    autogobble,
    breaklines=true,
    breakautoindent=true,
    breaksymbolleft=,
    breaksymbolright=,
    breaksymbolindentleft=0pt,
    breaksymbolindentright=0pt,
    breaksymbolsepleft=0pt,
    breaksymbolsepright=0pt,
    fontsize=\footnotesize,
    bgcolor=LightGray,
    numbersep=10pt
}


\usepackage{listings} % Para incluir código desde un archivo

\renewcommand\lstlistingname{Código Fuente}
\renewcommand\lstlistlistingname{Índice de Códigos Fuente}

% Definir colores
\definecolor{vscodepurple}{rgb}{0.5,0,0.5}
\definecolor{vscodeblue}{rgb}{0,0,0.8}
\definecolor{vscodegreen}{rgb}{0,0.5,0}
\definecolor{vscodegray}{rgb}{0.5,0.5,0.5}
\definecolor{vscodebackground}{rgb}{0.97,0.97,0.97}
\definecolor{vscodelightgray}{rgb}{0.9,0.9,0.9}

% Configuración para el estilo de C similar a VSCode
\lstdefinestyle{vscode_C}{
  backgroundcolor=\color{vscodebackground},
  commentstyle=\color{vscodegreen},
  keywordstyle=\color{vscodeblue},
  numberstyle=\tiny\color{vscodegray},
  stringstyle=\color{vscodepurple},
  basicstyle=\scriptsize\ttfamily,
  breakatwhitespace=false,
  breaklines=true,
  captionpos=b,
  keepspaces=true,
  numbers=left,
  numbersep=5pt,
  showspaces=false,
  showstringspaces=false,
  showtabs=false,
  tabsize=2,
  frame=tb,
  framerule=0pt,
  aboveskip=10pt,
  belowskip=10pt,
  xleftmargin=10pt,
  xrightmargin=10pt,
  framexleftmargin=10pt,
  framexrightmargin=10pt,
  framesep=0pt,
  rulecolor=\color{vscodelightgray},
  backgroundcolor=\color{vscodebackground},
}

%------------------------------------------------------------------------

% Comandos definidos
\newcommand{\bb}[1]{\mathbb{#1}}
\newcommand{\cc}[1]{\mathcal{#1}}

% I prefer the slanted \leq
\let\oldleq\leq % save them in case they're every wanted
\let\oldgeq\geq
\renewcommand{\leq}{\leqslant}
\renewcommand{\geq}{\geqslant}

% Si y solo si
\newcommand{\sii}{\iff}

% MCD y MCM
\DeclareMathOperator{\mcd}{mcd}
\DeclareMathOperator{\mcm}{mcm}

% Signo
\DeclareMathOperator{\sgn}{sgn}

% Letras griegas
\newcommand{\eps}{\epsilon}
\newcommand{\veps}{\varepsilon}
\newcommand{\lm}{\lambda}

\newcommand{\ol}{\overline}
\newcommand{\ul}{\underline}
\newcommand{\wt}{\widetilde}
\newcommand{\wh}{\widehat}

\let\oldvec\vec
\renewcommand{\vec}{\overrightarrow}

% Derivadas parciales
\newcommand{\del}[2]{\frac{\partial #1}{\partial #2}}
\newcommand{\Del}[3]{\frac{\partial^{#1} #2}{\partial #3^{#1}}}
\newcommand{\deld}[2]{\dfrac{\partial #1}{\partial #2}}
\newcommand{\Deld}[3]{\dfrac{\partial^{#1} #2}{\partial #3^{#1}}}


\newcommand{\AstIg}{\stackrel{(\ast)}{=}}
\newcommand{\Hop}{\stackrel{L'H\hat{o}pital}{=}}

\newcommand{\red}[1]{{\color{red}#1}} % Para integrales, destacar los cambios.

% Método de integración
\newcommand{\MetInt}[2]{
    \left[\begin{array}{c}
        #1 \\ #2
    \end{array}\right]
}

% Declarar aplicaciones
% 1. Nombre aplicación
% 2. Dominio
% 3. Codominio
% 4. Variable
% 5. Imagen de la variable
\newcommand{\Func}[5]{
    \begin{equation*}
        \begin{array}{rrll}
            \displaystyle #1:& \displaystyle  #2 & \longrightarrow & \displaystyle  #3\\
               & \displaystyle  #4 & \longmapsto & \displaystyle  #5
        \end{array}
    \end{equation*}
}

%------------------------------------------------------------------------



\begin{document}

    % 1. Foto de fondo
    % 2. Título
    % 3. Encabezado Izquierdo
    % 4. Color de fondo
    % 5. Coord x del titulo
    % 6. Coord y del titulo
    % 7. Fecha

    
    % 1. Foto de fondo
% 2. Título
% 3. Encabezado Izquierdo
% 4. Color de fondo
% 5. Coord x del titulo
% 6. Coord y del titulo
% 7. Fecha
% 8. Autor

\newcommand{\portada}[8]{
    \portadaBase{#1}{#2}{#3}{#4}{#5}{#6}{#7}{#8}
    \portadaBook{#1}{#2}{#3}{#4}{#5}{#6}{#7}{#8}
}

\newcommand{\portadaFotoDif}[8]{
    \portadaBaseFotoDif{#1}{#2}{#3}{#4}{#5}{#6}{#7}{#8}
    \portadaBook{#1}{#2}{#3}{#4}{#5}{#6}{#7}{#8}
}

\newcommand{\portadaExamen}[8]{
    \portadaBase{#1}{#2}{#3}{#4}{#5}{#6}{#7}{#8}
    \portadaArticle{#1}{#2}{#3}{#4}{#5}{#6}{#7}{#8}
}

\newcommand{\portadaExamenFotoDif}[8]{
    \portadaBaseFotoDif{#1}{#2}{#3}{#4}{#5}{#6}{#7}{#8}
    \portadaArticle{#1}{#2}{#3}{#4}{#5}{#6}{#7}{#8}
}




\newcommand{\portadaBase}[8]{

    % Tiene la portada principal y la licencia Creative Commons
    
    % 1. Foto de fondo
    % 2. Título
    % 3. Encabezado Izquierdo
    % 4. Color de fondo
    % 5. Coord x del titulo
    % 6. Coord y del titulo
    % 7. Fecha
    % 8. Autor    
    
    \thispagestyle{empty}               % Sin encabezado ni pie de página
    \newgeometry{margin=0cm}        % Márgenes nulos para la primera página
    
    
    % Encabezado
    \fancyhead[L]{\helv #3}
    \fancyhead[R]{\helv \nouppercase{\leftmark}}
    
    
    \pagecolor{#4}        % Color de fondo para la portada
    
    \begin{figure}[p]
        \centering
        \transparent{0.3}           % Opacidad del 30% para la imagen
        
        \includegraphics[width=\paperwidth, keepaspectratio]{../../_assets/#1}
    
        \begin{tikzpicture}[remember picture, overlay]
            \node[anchor=north west, text=white, opacity=1, font=\fontsize{60}{90}\selectfont\bfseries\sffamily, align=left] at (#5, #6) {#2};
            
            \node[anchor=south east, text=white, opacity=1, font=\fontsize{12}{18}\selectfont\sffamily, align=right] at (9.7, 3) {\href{https://losdeldgiim.github.io/}{\textbf{Los Del DGIIM}, \texttt{\footnotesize losdeldgiim.github.io}}};
            
            \node[anchor=south east, text=white, opacity=1, font=\fontsize{12}{15}\selectfont\sffamily, align=right] at (9.7, 1.8) {Doble Grado en Ingeniería Informática y Matemáticas\\Universidad de Granada};
        \end{tikzpicture}
    \end{figure}
    
    
    \restoregeometry        % Restaurar márgenes normales para las páginas subsiguientes
    \nopagecolor      % Restaurar el color de página
    
    
    \newpage
    \thispagestyle{empty}               % Sin encabezado ni pie de página
    \begin{tikzpicture}[remember picture, overlay]
        \node[anchor=south west, inner sep=3cm] at (current page.south west) {
            \begin{minipage}{0.5\paperwidth}
                \href{https://creativecommons.org/licenses/by-nc-nd/4.0/}{
                    \includegraphics[height=2cm]{../../_assets/Licencia.png}
                }\vspace{1cm}\\
                Esta obra está bajo una
                \href{https://creativecommons.org/licenses/by-nc-nd/4.0/}{
                    Licencia Creative Commons Atribución-NoComercial-SinDerivadas 4.0 Internacional (CC BY-NC-ND 4.0).
                }\\
    
                Eres libre de compartir y redistribuir el contenido de esta obra en cualquier medio o formato, siempre y cuando des el crédito adecuado a los autores originales y no persigas fines comerciales. 
            \end{minipage}
        };
    \end{tikzpicture}
    
    
    
    % 1. Foto de fondo
    % 2. Título
    % 3. Encabezado Izquierdo
    % 4. Color de fondo
    % 5. Coord x del titulo
    % 6. Coord y del titulo
    % 7. Fecha
    % 8. Autor


}


\newcommand{\portadaBaseFotoDif}[8]{

    % Tiene la portada principal y la licencia Creative Commons
    
    % 1. Foto de fondo
    % 2. Título
    % 3. Encabezado Izquierdo
    % 4. Color de fondo
    % 5. Coord x del titulo
    % 6. Coord y del titulo
    % 7. Fecha
    % 8. Autor    
    
    \thispagestyle{empty}               % Sin encabezado ni pie de página
    \newgeometry{margin=0cm}        % Márgenes nulos para la primera página
    
    
    % Encabezado
    \fancyhead[L]{\helv #3}
    \fancyhead[R]{\helv \nouppercase{\leftmark}}
    
    
    \pagecolor{#4}        % Color de fondo para la portada
    
    \begin{figure}[p]
        \centering
        \transparent{0.3}           % Opacidad del 30% para la imagen
        
        \includegraphics[width=\paperwidth, keepaspectratio]{#1}
    
        \begin{tikzpicture}[remember picture, overlay]
            \node[anchor=north west, text=white, opacity=1, font=\fontsize{60}{90}\selectfont\bfseries\sffamily, align=left] at (#5, #6) {#2};
            
            \node[anchor=south east, text=white, opacity=1, font=\fontsize{12}{18}\selectfont\sffamily, align=right] at (9.7, 3) {\href{https://losdeldgiim.github.io/}{\textbf{Los Del DGIIM}, \texttt{\footnotesize losdeldgiim.github.io}}};
            
            \node[anchor=south east, text=white, opacity=1, font=\fontsize{12}{15}\selectfont\sffamily, align=right] at (9.7, 1.8) {Doble Grado en Ingeniería Informática y Matemáticas\\Universidad de Granada};
        \end{tikzpicture}
    \end{figure}
    
    
    \restoregeometry        % Restaurar márgenes normales para las páginas subsiguientes
    \nopagecolor      % Restaurar el color de página
    
    
    \newpage
    \thispagestyle{empty}               % Sin encabezado ni pie de página
    \begin{tikzpicture}[remember picture, overlay]
        \node[anchor=south west, inner sep=3cm] at (current page.south west) {
            \begin{minipage}{0.5\paperwidth}
                %\href{https://creativecommons.org/licenses/by-nc-nd/4.0/}{
                %    \includegraphics[height=2cm]{../../_assets/Licencia.png}
                %}\vspace{1cm}\\
                Esta obra está bajo una
                \href{https://creativecommons.org/licenses/by-nc-nd/4.0/}{
                    Licencia Creative Commons Atribución-NoComercial-SinDerivadas 4.0 Internacional (CC BY-NC-ND 4.0).
                }\\
    
                Eres libre de compartir y redistribuir el contenido de esta obra en cualquier medio o formato, siempre y cuando des el crédito adecuado a los autores originales y no persigas fines comerciales. 
            \end{minipage}
        };
    \end{tikzpicture}
    
    
    
    % 1. Foto de fondo
    % 2. Título
    % 3. Encabezado Izquierdo
    % 4. Color de fondo
    % 5. Coord x del titulo
    % 6. Coord y del titulo
    % 7. Fecha
    % 8. Autor


}


\newcommand{\portadaBook}[8]{

    % 1. Foto de fondo
    % 2. Título
    % 3. Encabezado Izquierdo
    % 4. Color de fondo
    % 5. Coord x del titulo
    % 6. Coord y del titulo
    % 7. Fecha
    % 8. Autor

    % Personaliza el formato del título
    \pretitle{\begin{center}\bfseries\fontsize{42}{56}\selectfont}
    \posttitle{\par\end{center}\vspace{2em}}
    
    % Personaliza el formato del autor
    \preauthor{\begin{center}\Large}
    \postauthor{\par\end{center}\vfill}
    
    % Personaliza el formato de la fecha
    \predate{\begin{center}\huge}
    \postdate{\par\end{center}\vspace{2em}}
    
    \title{#2}
    \author{\href{https://losdeldgiim.github.io/}{Los Del DGIIM, \texttt{\large losdeldgiim.github.io}}
    \\ \vspace{0.5cm}#8}
    \date{Granada, #7}
    \maketitle
    
    \tableofcontents
}




\newcommand{\portadaArticle}[8]{

    % 1. Foto de fondo
    % 2. Título
    % 3. Encabezado Izquierdo
    % 4. Color de fondo
    % 5. Coord x del titulo
    % 6. Coord y del titulo
    % 7. Fecha
    % 8. Autor

    % Personaliza el formato del título
    \pretitle{\begin{center}\bfseries\fontsize{42}{56}\selectfont}
    \posttitle{\par\end{center}\vspace{2em}}
    
    % Personaliza el formato del autor
    \preauthor{\begin{center}\Large}
    \postauthor{\par\end{center}\vspace{3em}}
    
    % Personaliza el formato de la fecha
    \predate{\begin{center}\huge}
    \postdate{\par\end{center}\vspace{5em}}
    
    \title{#2}
    \author{\href{https://losdeldgiim.github.io/}{Los Del DGIIM, \texttt{\large losdeldgiim.github.io}}
    \\ \vspace{0.5cm}#8}
    \date{Granada, #7}
    \thispagestyle{empty}               % Sin encabezado ni pie de página
    \maketitle
    \vfill
}
    \portadaExamen{ffccA4.jpg}{Ecuaciones\\Diferenciales I\\Examen X}{Ecuaciones Diferenciales I. Examen X}{MidnightBlue}{-8}{28}{2024-2025}{Arturo Olivares Martos}

    \begin{description}
        \item[Asignatura] Ecuaciones Diferenciales I
        \item[Curso Académico] 2015-16.
        %\item[Grado] Doble Grado en Ingeniería Informática y Matemáticas.
        \item[Grupo] B.
        \item[Profesor] Rafael Ortega Ríos.
        \item[Descripción] Parcial B.
        \item[Fecha] 28 de abril de 2016.
        %\item[Duración] 60 minutos.
    
    \end{description}
    \newpage

    \begin{ejercicio}
        Dada la ecuación diferencial
        \begin{equation*}
            P(x, y) + Q(x, y)y' = 0
        \end{equation*}
        con $P, Q \in C^1(\bb{R}^2)$, ¿bajo qué condiciones existe un factor integrante del tipo $\mu(x, y) = m(x + 2y)$?\\

        Dado $\Omega\subset \bb{R}^2$, un factor integrante $\mu: \Omega \to \bb{R}$ para dicha ecuación diferencial es una función de clase $C^1(\Omega)$ que cumple:
        \begin{itemize}
            \item $\mu(x, y) \neq 0$ para todo $(x, y) \in \Omega$.
            \item Al multiplicar por $\mu$ la ecuación diferencial, se obtiene una ecuación diferencial exacta. Es decir:
            \begin{equation*}
                \dfrac{\partial(\mu P)}{\partial y} = \dfrac{\partial(\mu Q)}{\partial x}.
            \end{equation*}
        \end{itemize}

        Desarrollando dichas derivadas parciales, se tiene:
        \begin{align*}
            \dfrac{\partial(\mu P)}{\partial y} &= \dfrac{\partial \mu}{\partial y}\cdot P + \mu\cdot \dfrac{\partial P}{\partial y}\\
            \dfrac{\partial(\mu Q)}{\partial x} &= \dfrac{\partial \mu}{\partial x}\cdot Q + \mu\cdot \dfrac{\partial Q}{\partial x}
        \end{align*}

        Por tanto, la condición de exactitud queda:
        \begin{equation*}
            \dfrac{\partial \mu}{\partial y}\cdot P + \mu\cdot \dfrac{\partial P}{\partial y}
            = \dfrac{\partial \mu}{\partial x}\cdot Q + \mu\cdot \dfrac{\partial Q}{\partial x}.
        \end{equation*}

        Empleando que $\mu(x, y) = m(x + 2y)$, se tiene que sus derivadas parciales son:
        \begin{align*}
            \dfrac{\partial \mu}{\partial x}(x,y) &= m'(x + 2y)\\
            \dfrac{\partial \mu}{\partial y}(x,y) &= 2m'(x + 2y)
        \end{align*}

        Sustituyendo en la condición de exactitud, se obtiene:
        \begin{align*}
            2m'(x + 2y)P(x,y) + m(x + 2y)\dfrac{\partial P}{\partial y}(x,y) &= m'(x + 2y)Q(x,y) + m(x + 2y)\dfrac{\partial Q}{\partial x}(x,y)\\
            m'(x + 2y)(2P(x,y) - Q(x,y)) &= m(x + 2y)\left(\dfrac{\partial Q}{\partial x}(x,y) - \dfrac{\partial P}{\partial y}(x,y)\right)
        \end{align*}

        Imponemos entonces $2P(x,y) - Q(x,y) \neq 0$ para todo $(x, y) \in \Omega$. Por ser un factor integrante, $m(x + 2y) \neq 0$ para todo $(x, y) \in \Omega$, por lo que la condición de exactitud queda:
        \begin{equation*}
            \dfrac{m'(x + 2y)}{m(x + 2y)} = \dfrac{\dfrac{\partial Q}{\partial x}(x,y) - \dfrac{\partial P}{\partial y}(x,y)}{2P(x,y) - Q(x,y)}.
        \end{equation*}

        EL término izquierdo de la igualdad es función de $x + 2y$. Por tanto, hemos de imponer que el término derecho también lo sea. Es decir, que exista una función $f : \bb{R} \to \bb{R}$ tal que:
        \begin{equation*}
            f(x + 2y) = \dfrac{\dfrac{\partial Q}{\partial x}(x,y) - \dfrac{\partial P}{\partial y}(x,y)}{2P(x,y) - Q(x,y)} \qquad \forall (x, y) \in \Omega
        \end{equation*}

        Por tanto, hemos de imponer, en primer lugar, que ese cociente esté bien definido, lo que se garantiza imponiendo $2P(x,y) - Q(x,y) \neq 0$ para todo $(x, y) \in \bb{R}^2$ y, en segundo lugar, que exista una función $f : \bb{R} \to \bb{R}$ tal que:
        \begin{equation*}
            \dfrac{m'(x+2y)}{m(x+2y)} = f(x+2y) \qquad \forall (x, y) \in \Omega
        \end{equation*}

        Aunque no se pide, calculemos cómo será entonces el factor integrante.
        Sean entonces $\xi$ la variable independiente y $m$ la dependiente. Tenemos la ecuación diferencial:
        \begin{equation*}
            \dfrac{m'}{m} = f(\xi).\qquad \text{con dominio } \bb{R}\times \bb{R}^+
        \end{equation*}
        donde hemos supuesto $m(\xi) > 0$ para todo $\xi \in \bb{R}$ (en caso contrario, obtendríamos otro factor integrante igualmente válido).
        Esta es una ecuación diferencial de variables separables. Integrando ambos lados de la ecuación, notando por $F(\xi)$ a una primitiva de $f(\xi)$, y considerando constante de integración nula (en caso contrario, obtendríamos otro factor integrante igualmente válido), se tiene:
        \begin{align*}
            \int \dfrac{d\ m}{m} &= \int f(\xi)\,d\xi\\
            \ln(m) &= F(\xi)\\
            m(\xi) &= e^{F(\xi)}
        \end{align*}

        Por tanto, el factor integrante será:
        \begin{equation*}
            \mu(x, y) = e^{F(x + 2y)}
        \end{equation*}
    \end{ejercicio}

    \begin{ejercicio}
        Comprueba que la ecuación diferencial
        \begin{equation*}
            \dfrac{e^x}{y+e^x} +2x+\dfrac{1}{y + e^x}y' = 0
        \end{equation*}
        es exacta. Encuentra la solución que cumple $y(0) = 0$.\\

        Como $y(0)=0$, tenemos que $y(0)+e^0=1$. Por tanto, el dominio de la ecuación diferencial es:
        \begin{equation*}
            \Omega=\{(x,y)\in\bb{R}^2\mid y+e^x > 0\}
        \end{equation*}
        
        Comprobemos ahora que la ecuación diferencial es exacta.
        Definimos:
        \Func{P}{\Omega}{\bb{R}}{(x,y)}{\dfrac{e^x}{y+e^x}+2x}
        \Func{Q}{\Omega}{\bb{R}}{(x,y)}{\dfrac{1}{y+e^x}}

        Comprobemos si cumplen la condición de exactitud:
        \begin{align*}
            \dfrac{\partial P}{\partial y}(x,y) = -\dfrac{e^x}{(y+e^x)^2} = -\dfrac{e^x}{(y+e^x)^2}\qquad \forall (x,y)\in\Omega\\
        \end{align*}

        Por tanto, tenemos que es exacta, y además el dominio $\Omega$ es estrellado. Para encontrar la solución, buscaremos una función potencial $U$ tal que $\nabla U = (P,Q)$.
        Integrando la segunda componente de $\nabla U$ con respecto a $y$, obtenemos:
        \begin{equation*}
            U(x,y) = \int Q(x,y)\,dy = \int \dfrac{1}{y+e^x}\,dy = \ln(y+e^x) + \varphi(x)
        \end{equation*}
        donde $\varphi:\pi_1(\Omega)\to\bb{R}$ es una función que depende de $x$ y representa la constante de integración. Derivando $U$ con respecto a $x$, obtenemos:
        \begin{align*}
            \dfrac{\partial U}{\partial x}(x,y) &= \dfrac{\partial}{\partial x}(\ln(y+e^x) + \varphi(x)) = \dfrac{e^x}{y+e^x} + \varphi'(x)\\
            &= P(x,y) = \dfrac{e^x}{y+e^x} + 2x
        \end{align*}

        Por tanto, tenemos que $\varphi'(x) = 2x$, de donde obtenemos $\varphi(x) = x^2$ (notemos que hemos elegido constante de integración nula, puesto que el potencial es único salvo una constante aditiva).
        Por tanto, el potencial $U$ es:
        \begin{equation*}
            U(x,y) = \ln(y+e^x) + x^2
        \end{equation*}

        Por tanto, por la teoría vista en clase, como $Q(0,y(0))=1\neq 0$, la solución de la ecuación diferencial que cumple $y(0)=0$ viene dada implícitamente por la ecuación:
        \begin{equation*}
            U(x,y) = U(0,y(0))\Longrightarrow
            \ln(y+e^x) + x^2 = 0
        \end{equation*}

        Despejando $y(x)$, obtenemos:
        \begin{equation*}
            y(x)+e^x = e^{-x^2}\Longrightarrow y(x) = e^{-x^2}-e^x\qquad \forall x\in\bb{R}
        \end{equation*}
    \end{ejercicio}

    \begin{ejercicio}
        Demuestra que las funciones $f_1(t) = 1$, $f_2(t) = t^2$ y $f_3(t) = |t|^3 t$ son linealmente independientes en $\left]-1, 1\right[$.\\

        Tenemos que $f_1,f_2\in C^\infty(\bb{R})$. Estudiemos $f_3$. Tenemos que:
        \begin{equation*}
            f_3(t) = \begin{cases}
                t^4 & \text{si } t\geq 0\\
                -t^4 & \text{si } t<0
            \end{cases}
        \end{equation*}

        Vemos que $f_3\in C(\bb{R})$. Además, por el carácter local de la derivabilidad, tenemos que $f_3\in C^\infty(\bb{R}\setminus\{0\})$. Para estudiar el caso del origen, calculamos las derivadas de $f_3$:
        \begin{align*}
            f_3'(t) &= \begin{cases}
                4t^3 & \text{si } t>0\\
                -4t^3 & \text{si } t<0
            \end{cases}\\
            f_3''(t) &= \begin{cases}
                12t^2 & \text{si } t>0\\
                -12t^2 & \text{si } t<0
            \end{cases}\\
            f_3'''(t) &= \begin{cases}
                24t & \text{si } t>0\\
                -24t & \text{si } t<0
            \end{cases}
        \end{align*}

        Por tanto, vemos que $f_3$ es $3$ veces derivable en $\bb{R}$, aunque tan solo buscábamos las dos primeras derivadas. El Wronskiano de $f_1$, $f_2$ y $f_3$ es, para cualquier $t\in \left]0,1\right[$:
        \begin{equation*}
            W(f_1,f_2,f_3)(t) = \begin{vmatrix}
                1 & t^2 & t^4\\
                0 & 2t & 4t^3\\
                0 & 2 & 12t^2
            \end{vmatrix} = 2\cdot 4t^2\cdot \begin{vmatrix}
                t & t\\
                1 & 3
            \end{vmatrix} = 8t^2(3t-t) = 16t^3>0
        \end{equation*}
        Por tanto, como $\exists t\in \left]0,1\right[\subset \left]-1,1\right[$ tal que $W(f_1,f_2,f_3)(t)\neq 0$, tenemos que $f_1$, $f_2$ y $f_3$ son linealmente independientes en $\left]-1,1\right[$.
    \end{ejercicio}

    \begin{ejercicio}
        En el intervalo $I = \left]-1, 1\right[$ se dan dos funciones $A \in C^1(I)$, $\beta \in C(I)$ y se define
        \begin{equation*}
            x(t) = 3e^{A(t)} - 2e^{A(t)}\int_0^t e^{-A(s)}\beta(s)\,ds.
        \end{equation*}
        Encuentra una ecuación lineal de primer orden para la que la función $x(t)$ sea solución.\\

        Veamos en primer lugar que $x\in C^1(I)$. En primer lugar, el primer término del integrando es la composición de dos funciones continuas, por lo que es continuo; mientras que el segundo término es continuo por hipótesis. Por tanto, el integrando es continuo en todo compacto $[0,t]\subset I$; y por el Teorema Fundamental del Cálculo, la integral es de clase $C^1$ en $I$. El resto de la expresiónde $x$ es composición, producto y suma de funciones de clase $1$, luego $x\in C^1(I)$. Derivando $x$ con respecto a $t$, obtenemos:
        \begin{align*}
            x'(t) &= 3A'(t)e^{A(t)} - 2A'(t)e^{A(t)}\int_0^t e^{-A(s)}\beta(s)\,ds - 2e^{A(t)}e^{-A(t)}\beta(t)=\\
            &= A'(t)\left(3e^{A(t)} - 2e^{A(t)}\int_0^t e^{-A(s)}\beta(s)\,ds\right) - 2\beta(t)
        \end{align*}

        Usando la definición de $x(t)$, obtenemos:
        \begin{equation*}
            x'(t) = A'(t)x(t) - 2\beta(t)
        \end{equation*}

        Por tanto, la ecuación diferencial lineal de primer orden cuya solución es $x(t)$ es:
        \begin{equation*}
            x'=A'x-2\beta \qquad \text{con dominio } I\times \bb{R}
        \end{equation*}

        Notemos además que:
        \begin{equation*}
            x(t)=e^{A(t)}\left(3+\int_0^t e^{-A(s)}(-2 \beta(s))\,ds\right)
        \end{equation*}

        Posiblemente, al leer el lector la solución de esta forma, recuerde que se trata de la fórmula vista en el Capítulo 2.
    \end{ejercicio}

    \begin{ejercicio}
        Sea una función $f : \bb{R} \to \bb{R}$ de clase $C^1$ y con inversa $g = f^{-1} : \bb{R} \to \bb{R}$ también de clase $C^1$. Para cada $\lambda \in \bb{R}$ se define el cambio de variable en el plano $\varphi_\lm : \bb{R}^2 \to \bb{R}^2$, $(t, x) \mapsto (s, y)$ por las fórmulas
        \begin{equation*}
            s = t, \quad y = f(g(x) + \lambda).
        \end{equation*}
        Demuestra que $\cc{G} = \{\varphi_\lm \mid \lambda \in \bb{R}\}$ es un grupo de difeomorfismos del plano.
    \end{ejercicio}
\end{document}
