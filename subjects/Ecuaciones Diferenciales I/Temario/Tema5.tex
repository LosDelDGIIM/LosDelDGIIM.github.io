\newpage
\chapter{Sistemas Lineales}

\begin{notacion}
    Por comodidad, a lo largo de la sección notaremos al conjunto de matrices de orden $n\times m$ sobre $\mathbb{R}$ por:
    \begin{equation*}
        \mathbb{R}^{n\times m} = M_{n\times m}(\mathbb{R})
    \end{equation*}
\end{notacion}

\noindent
Estudiaremos sistemas lineales de primer orden, es decir, ecuaciones de la forma:
\begin{equation}\label{eq:sel_1orden}
    x' = A(t) x + b(t)
\end{equation}
Con $A:I\rightarrow\mathbb{R}^{d\times d}$ y $b:I\rightarrow\mathbb{R}^d$, funciones continuas\footnote{Recordemos que esto significa que sean continuas coordenada a coordenada.} en un intervalo abierto $I\subseteq \mathbb{R}$. Si notamos por $A = {(a_{ij})}_{i,j}$, $d = (b_i)$ y $x = (x_i)$ a las correspondientes coordenadas de $A$, $b$ y $x$, podemos reescribir~(\ref{eq:sel_1orden}) en forma de sistema, como:
\begin{equation}\label{eq:sistema_1orden}
    \left\{\begin{array}{ccccccccc}
            x_1' & = & a_{11}(t)x_1 & + & \cdots & + & a_{1d}(t)x_d & + & b_1(t) \\
            x_2' & = & a_{21}(t)x_1 & + & \cdots & + & a_{2d}(t)x_d & + & b_2(t) \\
            \vdots & & \vdots & & \ddots & & \vdots & & \vdots \\
            x_d' & = & a_{d1}(t)x_1 & + & \cdots & + & a_{dd}(t)x_d & + & b_d(t) \\
    \end{array}\right.
\end{equation}

\begin{ejemplo}
    Supongamos que estamos en la situación de la Figura~\ref{fig:muelles}, con dos masas $m_1$ y $m_2$, y dos muelles con constantes elásticas $k_1$ y $k_2$. Supongamos además que a la masa $m_2$ se le aplica una fuerza $F(t)$.
\begin{figure}[H]
    \centering
    \begin{tikzpicture}
        % Dibuja la pared
        \draw[thick] (0,0.5) -- (0,1.5);

        % Dibuja el primer muelle
        \draw[thick, decoration={aspect=0.3, segment length=4mm, amplitude=3mm, coil}, decorate] (0,1) -- (4,1);
        % Etiqueta para el primer muelle
        \node[above] at (2,1.2) {$k_1$};

        % Dibuja la primera masa (cuadrado)
        \draw[fill=gray!30] (4,0.75) rectangle (4.5,1.25);
        % Etiqueta para la primera masa
        \node[below] at (4.25,0.75) {$m_1$};

        % Dibuja el segundo muelle (más ancho)
        \draw[thick, decoration={aspect=0.3, segment length=4mm, amplitude=4mm, coil}, decorate] (4.5,1) -- (8,1);
        % Etiqueta para el segundo muelle
        \node[above] at (6.25,1.3) {$k_2$};

        % Dibuja la segunda masa (cuadrado)
        \draw[fill=gray!30] (7.9,0.65) rectangle (8.6,1.35);
        % Etiqueta para la segunda masa
        \node[below] at (8.25,0.65) {$m_2$};

        % Dibuja un vector horizontal desde la masa derecha
        \draw[-{Latex[length=3mm, width=2mm]}, thick] (8.6,1) -- (9.6,1);
        % Etiqueta para el vector
        \node[above] at (9.1,1) {$F(t)$};
    \end{tikzpicture}
    \caption{Dos masas conectadas por muelles.}
    \label{fig:muelles}
\end{figure}
    Describiremos este sistema de forma matemática describiendo $x_1$, la distancia de la masa $m_1$ a su posición de equilibrio; y $x_2$, la distancia de la masa $m_2$ a su posición de equilibrio a lo largo del tiempo $t$.\\

    Suponiendo que inicialmente (en el instante $t_0$) el primer muelle está dilatado (es decir, $x_1(t_0) > 0$) y que el segundo muelle está contraido ($x_2(t_0) - x_1(t_0)<0$), aplicando las leyes de Newton, llegamos al sistema:
    \begin{equation*}
        \left\{\begin{array}{rcl}
                m_1 x_1 '' &=& -k_1 x_1 + k_2 (x_2 - x_1) \\
                m_2 x_2 '' &=& -k_2(x_2 - x_1) + F(t)
        \end{array}\right.
    \end{equation*}
    La máquina descrita sigue estas ecuaciones diferenciales, que no están en la categoría que nos interesa, por ser de segundo orden. Sin embargo, un sistema lineal de cualquier orden se puede hacer siempre de primer orden. Para ello, buscamos transformar dos ecuaciones de segundo orden en 4 ecuaciones de primer orden.\\

    \noindent
    El truco para cambiar orden por dimensión es llamar incógnita a las derivadas. Definimos:
    \begin{equation*}
        y_1 = x_1 \qquad y_2 = x_1' \qquad y_3 = x_2 \qquad y_4 = x_2'
    \end{equation*}
    De esta forma:
    \begin{equation*}
        \left\{\begin{array}{rl}
                y_1' &= y_2 \\
            y_2' &= \dfrac{-k_1}{m_1} y_1 + \dfrac{k_2}{m_1} (y_3-y_1) \\
            y_3' &= y_4 \\
            y_4' &= \dfrac{-k_2}{m_2}(y_3-y_1) + \dfrac{F(t)}{m_2}
        \end{array}\right.
    \end{equation*}
    Obtenemos ya un sistema de ecuaciones lineal de primer orden. Los físicos dicen que hemos pasado del espacio de las configuraciones al espacio de estados.\\

    Tenemos:
    \begin{equation*}
        A(t) = \left(\begin{array}{cccc}
                0 & 1 & 0 & 0 \\
                -\left(\frac{k_1+k_2}{m_1}\right) & 0 & \frac{k_2}{m_2} & 0 \\
                0 & 0 & 0 & 1 \\
                \frac{k_2}{m_2} & 0 & \frac{-k_2}{m_2} & 0
        \end{array}\right) \qquad b(t) = \left(\begin{array}{c}
            0 \\
            0 \\
            0 \\
            \dfrac{F(t)}{m_2}
        \end{array}\right)
    \end{equation*}
\end{ejemplo}

\begin{teo}[Existencia y unicidad de las soluciones]\label{teo:existencia_unicidad_sistemas}
    Dados $t_0\in I$, $x_0\in \mathbb{R}^d$, existe una única solución del sistema:
    \begin{equation*}
        x' = A(t)x + b(t) \qquad x(t_0) = x_0
    \end{equation*}
    definida en \textbf{todo} el intervalo $I$.
\end{teo}
Para su demostración, será necesario repasar varios conceptos ya vistos en otras asignaturas.

\begin{coro}
Ahora, si tenemos una ecuación lineal de orden superior:
\begin{equation*}
    x^{(k)} + a_{k-1}(t) x^{(k-1)} + \cdots + a_1(t)x' + a_0(t)x = b(t)
\end{equation*}
Lo que hacemos es tomar como incógnitas:
\begin{equation*}
    y_1 = x \qquad y_2 = x' \qquad \ldots \qquad y_k = x^{(k-1)}
\end{equation*}
Y plantear el sistema:
\begin{equation*}
    \left\{\begin{array}{rcl}
            y_1' &=& y_2 \\
            y_2' &=& y_3 \\
            \vdots && \vdots \\
            y_{k-1}' &=& y_k \\
            y_k' &=& -a_0(t)y_1 -a_1(t) y_2 - \cdots - a_{k-1}(t)y_k + b(t)
    \end{array}\right.
\end{equation*}
Con lo que el Teorema de existencia y unicidad del Capítulo anterior es un corolario del Teorema~\ref{teo:existencia_unicidad_sistemas}.
\end{coro}

\subsubsection{Normas matriciales}
Sobre $\mathbb{R}^d$, consideramos una norma $\|\cdot \|:\mathbb{R}^d\rightarrow\mathbb{R}$.\\

Esta norma puede inducirse al espacio $\mathbb{R}^{d\times d}$, de forma que:
\begin{equation*}
    \|A\| = \max\{\|Ax\| \mid \|x\|\geq 1\} \qquad \forall A\in \mathbb{R}^{d\times d}
\end{equation*}

De forma geométrica, cada $A$ es una transformación del espacio $\mathbb{R}^d$ en sí mismo. Lo que hacemos es tomar el punto más lejano al origen que es imagen de un vector de norma 1 en el dominio.\\

Notemos que podemos tomar el máximo porque la aplicación $x\mapsto \|Ax\|$ es continua y estamos tomando la imagen de un compacto por una aplicación continua.

\begin{ejemplo}
    Considerando el espacio normado $(\mathbb{R}^2, \|\cdot \|_2)$, si tomamos:
    \begin{equation*}
        A = \left(\begin{array}{cc}
                2 & 0 \\
                0 & \nicefrac{1}{2}
        \end{array}\right)
    \end{equation*}
    % // TODO: Dibujar la transformacion de la bola unidad
    La aplicación asociada a $A$ transforma $\bb{S}^1$ en una elipse de eje mayor 2 y eje menor $\nicefrac{1}{2}$, con lo que:
    \begin{equation*}
        \|A\| = 2
    \end{equation*}
\end{ejemplo}

Las normas matriciales tienen propiedades extra, que ya fueron vistas en Métodos Numéricos I\@:
\begin{prop}
    Se verifica que:
    \begin{enumerate}
        \item $\|I\| = 1$.
        \item $\|AB\| \leq \|A\|\|B\|$, $\forall A,B\in \mathbb{R}^{d\times d}$.
        \item $\|Ax\| \leq \|A\|\|x\|$, $\forall x\in \mathbb{R}^d, A\in \mathbb{R}^{d\times d}$.
    \end{enumerate}
\end{prop}

\subsubsection{Integrales vectoriales}
Supongamos que tenemos $f:[a,b]\rightarrow \mathbb{R}^d$ una función continua en un intervalo compacto, con lo que $f$ tiene $d$ coordenadas: $f=(f_1,\ldots,f_d)$, todas ellas continuas. De esta forma, definimos
\begin{equation*}
    \int_{a}^{b} f(t)~dt  = \left(\begin{array}{c}
        \int_{a}^{b} f_1(t)~dt  \\
        \int_{a}^{b} f_2(t)~dt  \\
        \vdots \\
        \int_{a}^{b} f_d(t)~dt  
    \end{array}\right)
\end{equation*}

\begin{prop}
    Sea $A\in \mathbb{R}^{d\times d}$, entonces:
    \begin{equation*}
        A\left(\int_{a}^{b} f(t)~dt \right) = \int_{a}^{b} A(f(t))~dt
    \end{equation*}
\end{prop}

\begin{prop}
    Se verifica que:
    \begin{equation*}
        \left\|\int_{a}^{b} f(t)~dt \right\| \leq \int_{a}^{b} \|f(t)\|~dt 
    \end{equation*}
    Para cualquier norma.
    \begin{proof}
        Se saca mediante las sumas de Riemann, cogiendo la integral de Riemann.
    \end{proof}
\end{prop}

\subsubsection{Convergencia uniforme}
