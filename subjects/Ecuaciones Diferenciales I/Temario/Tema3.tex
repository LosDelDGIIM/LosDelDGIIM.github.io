\newpage
\chapter{Condición de exactitud y factores integrales}
Se trata de otra forma de resolver ecuaciones diferenciales, mediante ``exactitud''.

\begin{notacion}
    Volveremos a cambiar nuevamente la notación, usando para ello:
    \begin{equation}\label{eq:nueva_notacion}
        P(x,y) + Q(x,y)y' = 0
    \end{equation}
    Donde la variable dependiente será ahora $y$ y la independiente $x$. Ahora, no buscaremos la forma normal de la ecuación.
\end{notacion}

La idea de exactitud consiste en buscar mediante algún procedimiento reescribir la ecuación~(\ref{eq:nueva_notacion}) eso como:
\begin{equation*}
    \dfrac{d}{dx}(algo en funcion de x e y) = 0
\end{equation*}
De donde $algo en funcion de x e y =c$ con $c\in \mathbb{R}$.

\begin{ejemplo}
    Recordamos la ecuación
    \begin{equation*}
        \dfrac{dy}{dx} = \dfrac{y-x}{y+x}
    \end{equation*}
    Que podemos reescribir en el formato~(\ref{eq:nueva_notacion}) como:
    \begin{equation*}
        x-y+(x+y)y' = 0
    \end{equation*}
    Luego tenemos:
    \begin{gather*}
        P(x,y) = x-y \\
        Q(x,y) = x+y
    \end{gather*}
    Y tenemos:
    \begin{equation*}
        x+yy' - y+xy' = 0
    \end{equation*}
    y la parte de la izquierda ($x+yy'$) podemos verla como:
    \begin{equation*}
        \dfrac{d}{dx}\left(\dfrac{x^2+y^2}{2}\right)
    \end{equation*}
    Si también pudiéramos hacerlo en la derecha (algo que parece más difícil, pero que si tuviéramos un $+$ lo tendríamos), llegaríamos a:
    \begin{equation*}
        \dfrac{d}{dx}\left(\dfrac{x^2+y^2}{2}\right) - \text{?} = 0
    \end{equation*}
    Sin embargo, veremos próximamente que este truco no puede hacerse con esta ecuación.

    Existe otro truco distinto que nos permite reseolverlo, dividiendo todo entre $x^2 + y^2$ (algo que ahora parece idea feliz). Este algo es el factor integrante (lo multiplicamos con la ecuación y ya podemos ponerlo bien). Dividimos y:
    \begin{equation*}
        \dfrac{x-y}{x^2+y^2} + \dfrac{x+y}{x^2+y_2}y' = 0
    \end{equation*}
    Vamos a seleccionar cuidadosamente:
    \begin{equation*}
        \dfrac{x+yy'}{x^2+y^2} + \dfrac{-y+xy'}{x^2+y^2}
    \end{equation*}
    Luego:
    \begin{equation*}
        \dfrac{d}{dx}\left(\dfrac{1}{2}\ln(x^2+y^2)\right) + \dfrac{d}{dx}\left(\arctg\left(\dfrac{y}{x}\right)\right) = 0
    \end{equation*}
    % // TODO: Reestructurar
    Comprobemos que lo hemos hecho bien a la derecha, calculando su derivada:
    \begin{equation*}
        \dfrac{1}{1+{\left(\dfrac{y}{x}\right)}^{2}} \dfrac{xy'-y}{x^2}
    \end{equation*}
    Finalmente, llegamos a que:
    \begin{equation*}
        \dfrac{1}{2}\ln(x^2+y^2) + \arctg\left(\dfrac{y}{x}\right) = c \qquad c\in \mathbb{R}
    \end{equation*}
\end{ejemplo}

En primer lugar, encontraremos una condición necesaria y suficiente para tener condicion de exactitud (escribirlas de dicha forma).
Posteriormente, encontrar por qué tenemos que multiplicar (el factor integrante) para tner la condición de exactitud.

\section{} % // TODO: Quizas quitar eso
A continuación, trabajaremos en un abierto conexo $\Omega\subseteq \mathbb{R}^2$ y con dos funciones $P,Q\in C^1(\Omega)$\footnote{Notemos que en el ejemplo anterior $P$ y $Q$ eran de clase $C^1$, por ser polinomios.}.

Si existe $U\in C^1(\Omega)$ de dos variables de forma que la ecuación~(\ref{eq:nueva_notacion}) se transforme en una de la forma:
\begin{equation*}
    \dfrac{d}{dx}(U(x,y)) = 0
\end{equation*}
Derivando de forma implícita:
\begin{equation*}
    \dfrac{\partial U}{\partial x}(x,y) + \dfrac{\partial U}{\partial y}(x,y)y' = 0
\end{equation*}
Lo que buscamos es:
\begin{align*}
    P &= \dfrac{\partial U}{\partial x} \\
    Q &= \dfrac{\partial U}{\partial y}
\end{align*}
Con lo que tenemos que buscar dicho potencial $U$.

Veremos próximamente que esto no es posible en general (sin exigir más condiciones).
Es decir, no tiene por qué existir una $U$ de forma que sus dos parciales sean las funciones dadas.

% //

RAbs del ejemplo:
Ahora, afirmamos que no existe $U\in C^1(\mathbb{R}^2)$ tal que ocurra que:
\begin{gather*}
    \dfrac{\partial U}{\partial x} = P(x,y) = x-y
    \dfrac{\partial U}{\partial y} = Q(x,y) = x+y
\end{gather*}
Si existiera la $U$, en principio sería una función de clase $C^1$, pero sin embargo, deberia ser de clase $C^2$, de hecho, de clase $C^\infty$ (ya que son polinomios).

Hay funciones raras para las cuales las sucesivas derivadas parciales no conmutan. Sin embargo, al ser de clase $C^2$ sí conmutan (se supone teoría vista, aunque no lo ha sido).
Se razona por el cociente incremental por el Teorema del valor medio.
\begin{equation*}
    \dfrac{\partial^2 U}{\partial x\partial y} = \dfrac{\partial^2 U}{\partial y\partial x} 
\end{equation*}
Luego:
\begin{equation*}
    1 = \dfrac{\partial Q}{\partial x} = \dfrac{\partial P}{\partial y} = -1
\end{equation*}
Contradicción, luego no podemos encontrar dicha función $U$.

En general, dadas dos funciones de clase $C^1$ no podemos encontrar una funcion de clase $C^1$ tal que sus derivadas sean aquellas.

Notemos que es el análogoa al TFC en varias variables, pero ahora es falso.

% //

Dadas dos funciones de dos variables, tratamos de buscar un potencial.

\section{} % // TODO: Quizas aqui si

Bajo qué condiciones existe una función $U\in C^1(\Omega)$ de forma que
\begin{equation*}
    \dfrac{\partial U}{\partial x} = P \qquad \dfrac{\partial U}{\partial y} = Q 
\end{equation*}

\begin{prop}
    Si existe $U$ en las condiciones enunciadas, entonces se ha de cumplir la \textbf{condición de exactitud}:
    \begin{equation*}
        \dfrac{\partial P}{\partial y} = \dfrac{\partial Q}{\partial x}
    \end{equation*}
    \begin{proof}
        Por reducción al absurdo, si existiera dicha $U$, esta sería de clase $C^2(\Omega)$, luego:
        \begin{equation*}
            \dfrac{\partial P}{\partial x} = \dfrac{\partial }{\partial x}\left(\dfrac{\partial U}{\partial y}\right)= \dfrac{\partial^2 U}{\partial x\partial y} = \dfrac{\partial^2 U}{\partial y \partial x} = \dfrac{\partial}{\partial y}\left(\dfrac{\partial U}{\partial x}\right) = \dfrac{\partial Q}{\partial y}
        \end{equation*}
    \end{proof}
\end{prop}
Se trata de una condición necesaria para que exista el potencial.

Se trata de una condición suficiente dependiendo de la topología del dominio.

Simplemente conexo (sin agujeros) <=>

\begin{definicion}[Forma de estrella]
    Diremos que $\Omega$ tiene forma de estrella (o que es estrellado\footnote{Tal y como vimos en Topología I.}) si existe $z_\ast \in \Omega$ tal que 
    \begin{equation*}
        \forall z\in \Omega \qquad [z,z_\ast] \in \Omega
    \end{equation*}
    Donde hemos usado $[z,z_\ast]$ para denotar al segmento de extremos $z$ y $z_\ast$
\end{definicion}
Se trata de una propiedad topológica.

Ejemplos de conjuntos estrellados son:
\begin{equation*}
    \mathbb{R}^2 \qquad [0,1]\times [0,1] \qquad \mathbb{S}^1
\end{equation*}
Todos estos son convexos. Sin embargo, existen conjuntos con forma de estrella que no son convexos (como la idea de una estrella). % // TODO: incluir dibujo en geogebra

Recordando la proposición (hacer referencia), mostramos el siguiente teorema, que nos ofrece la otra implicación.
\begin{teo}
    Si $\Omega$ tiene forma de estrella y $P,Q\in C^1(\Omega)$ que cumplen la condición de exactitud, es decir:
    \begin{equation*}
        \dfrac{\partial P}{\partial y} = \dfrac{\partial Q}{\partial x}
    \end{equation*}
    Entonces, existe una función $U\in C^2(\Omega)$ tal que 
    \begin{equation*}
        \dfrac{\partial U}{\partial x} = P \qquad \dfrac{\partial U}{\partial y} = Q
    \end{equation*}
\end{teo}

Preparando la demo:
\begin{equation*}
    F(y) = \int_{y_0}^{y} f(\xi)~d\xi 
\end{equation*}
Sabemos derivar $F$ gracias al TFC\@.

Sin embargo, hay funciones definidas por integrales de diversas maneras, como una función de dos variables que integramos en una:
\begin{equation*}
    F(y) = \int_{a}^{b} f(x,y)~dx 
\end{equation*}
Integrales dependientes de un parámetro, integrales donde el intervalo de integración está fijo.

Lo que vamos a usar para $U$ es derivar una integral dependiente de un parámetro (en la demostracion).

Veremos ahora un Teorema con hipótesis sencillas que nos da algo que necesitamos.

% // Pensar sobre el teorema

Pensaremos que $x$ es una variable discreta:
\begin{equation*}
    F(y) = \sum_{n=1}^{N} f_n(y)
\end{equation*}
Para derivar $F$, la derivada conmuta con el sumatorio. Luego:
\begin{equation*}
    F'(y) = \int_{a}^{b} \dfrac{\partial f}{\partial y}(x,y)~dx 
\end{equation*}

% //

Veremos una versión más débil que la vista en Análisis Matemático II, que cuenta con la funcionalidad justa para demostrar el teorema anterior.
\begin{teo}[Integral dependiente de un parámetro]
Sea $G\subseteq \mathbb{R}^d$ un conjunto abierto, dada una aplicación
\Func{f}{G\times[a,b]}{\mathbb{R}}{(\xi_1, \xi_2, \ldots, \xi_d, t)}{f (\xi_1, \xi_2, \ldots, \xi_d, t)}
De clase $C^1$, definimos una función $F:G\rightarrow\mathbb{R}$ dada por
\begin{equation*}
    F(\xi_1, \xi_2, \ldots, \xi_d) = \int_{a}^{b} f(\xi_1, \xi_2, \ldots, \xi_d, t)~dt
\end{equation*}
Entonces, $F\in C^1(G)$ y 
\begin{equation*}
    \dfrac{\partial F}{\partial \xi_i}(\xi_1, \xi_2, \ldots, \xi_d) = \int_{a}^{b} \dfrac{\partial f}{\partial \xi_i}(\xi_1, \xi_2, \ldots, \xi_d, t)~dt \qquad \forall i \in \{1,\ldots, d\}
\end{equation*}
\end{teo}

\begin{ejemplo}
    Dada la función
    \begin{equation*}
        F(y) = \int_{0}^{1} e^x \sen(x+y^2)~dx 
    \end{equation*}
    La derivada de esta función es:
    \begin{equation*}
        F'(y) = \int_{0}^{1} 2ye^x \cos(x+y^2)~dx = 2y\int_{0}^{1} e^x \cos(x+y^2)~dx \qquad \forall y\in \mathbb{R}
    \end{equation*}
\end{ejemplo}

% // Demostracion del teorema anterior

Dado un conjunto $\Omega$ con forma de estrella, supongamos que $z_\ast = (0,0)$ (para ahorrar notación). Se deja como ejercicio el caso general.
Defino $U:\Omega\rightarrow\mathbb{R}$
\begin{equation*}
    U(x,y) = x\int_{0}^{1} P(\lm x,\lm y)~d\lm + y\int_{0}^{1} Q(\lm x, \lm y)~d\lm  
\end{equation*}
Lo primero es ver que la función está bien definida, es decir, que podemos calcular la integral.

Analíticamente, se trata de integrales de funciones continuas en un compacto.
Geométricamente, se garantiza por ser estrellado.

Podemos aplicar el Teorema de la integral dependiente de un parámetro, llegando a que $U\in C^1(\Omega)$.
Calculamos las parciales:
\begin{equation*}
    \dfrac{\partial U}{\partial x}(x,y) = \int_{0}^{1} P(\lm x,\lm y)~d\lm + x\int_{0}^{1} \lm \dfrac{\partial P}{\partial x}(\lm x,\lm y)~d\lm + y\int_{0}^{1} \lm \dfrac{\partial Q}{\partial x}(\lm x,\lm y)~d\lm 
\end{equation*}
Veamos ahora que es igual a $P$, usando la condición de exactitud:
\begin{equation*}
    \dfrac{\partial U}{\partial x}(x,y) = \int_{0}^{1} P(\lm x,\lm y)~d\lm +  x\int_{0}^{1} \lm\dfrac{\partial P}{\partial x}(\lm x,\lm y)~d\lm  + y\int_{0}^{1} \lm \dfrac{\partial P}{\partial y}(\lm x,\lm y)~d\lm 
\end{equation*}
Escribiéndola como una diferencial exacta:
\begin{equation*}
    = \int_{0}^{1} P(\lm x,\lm y)~d\lm + \int_{0}^{1} \lm \dfrac{d}{d\lm}[P(\lm x, \lm y)]~d\lm  
\end{equation*}
Escribiéndola como una diferencial exacta (otra vez):
\begin{equation*}
    = \int_{0}^{1} \dfrac{d}{d\lm}[\lm P(\lm x,\lm y)]~d\lm 
\end{equation*}
Aplicando Barrow:
\begin{equation*}
    = {[\lm P(\lm x,\lm y)]}_{\lm = 0}^{\lm = 1} = P(x,y)
\end{equation*}
Finalmente, como las derivadas parciales son de clase $C^1$, tenemos que $U\in C^2(\Omega)$.

% // 

Tratamos ahora de buscar la interpretación del teorema anterior, con intención de darle un sentido geométrico, un significado de de dónde viene dicha función $U$.

\subsection{Interpretación de la demostración}
\subsubsection{Campos conservativos y trabajo}
Dada una aplicación $F:G\rightarrow\mathbb{R}^2$ con $G\subseteq \mathbb{R}^2$ un conjunto abierto.
Notemos que un campo en el plano es una aplicación que a cada punto $z=(x,y)$ la hace corresponder un vector (una flecha) $F(z) = (F_1(x,y), F_2(x,y))$.
Pensaremos en este como en un campo de fuerzas, con $F\in C^1(G)$.

\begin{definicion}
    Dieremos que un campo de fuerzas $F$ es conservativo si existe un potencial, es decir, una función $U:G\rightarrow\mathbb{R}$ de clase $C^1$ tal que 
    \begin{equation*}
        \nabla U = F
    \end{equation*}
    Es decir:
    \begin{equation*}
        \dfrac{\partial U}{\partial x} &= F_1 \qquad \dfrac{\partial U}{\partial y} &= F_2
    \end{equation*}
\end{definicion}
Nota: actualmente, $\nabla U = -\nabla V$.

\begin{ejemplo}
    \begin{equation*}
        F(x_1,x_2) = \dfrac{1}{2}(x_1,x_2)
    \end{equation*}
    Que podemos pensar como una homotecia o como un campo vectorial:
    \begin{itemize}
        \item En el origen, tenemos el vector 0.
        \item Dado un punto $(x,y)$, tenemos que dibujar el vector fuerza (gradiente) como la mitad del vector de posición.
    \end{itemize}
    % // TODO: Dibujar
    Notemos que se trata de un campo repulsor, de forma que el vector fuerza se mantiene en circunferencias de un determinado grado, con dirección perpendicular a los radios de la misma.
    Conforme nos alejamos, se incrementa.

    Resulta en un campo de fuerzas conservativo:
    \begin{equation*}
        U(x_1,x_2) = \dfrac{x_1^2 + x_2^2}{4}
    \end{equation*}
\end{ejemplo}

\begin{ejemplo}
    Un ejemplo de campo no conservativo es
    \begin{equation*}
        F(x_1,x_2) = (-x_2, x_1)
    \end{equation*}
    Se trata de un giro de 90º, en sentido antihorario (como agua colándose en un sumidero), de forma que describe vórtices.

    Se trata de un campo no conservativo, por ser imposible hayar un potencial, ya que no cumple la condición de exactitud.

    Por tanto, en este campo la energía no existe.
\end{ejemplo}

Cuando hay un campo de fuerzas no necesariamente hay energía, pero siempre hay trabajo. Definiremos ahora formalmente el trabajo de un campo de fuerzas.

El trabajo es un número que se asigna a cada camino:
En un conjunto abierto $G$, tnemos un camino $\gamma:[a,b]\rightarrow G$, con $\gamma\in C^1$. Asignaremos un trabajo a dicho camino, el trabajo de $F$ a lo largo de $\gamma$.

\begin{definicion}[Trabajo]
    Dado un conjunto abierto $G\subseteq \mathbb{R}^2$ y un camino que será una función $\gamma:[a,b]\rightarrow G$ con $\gamma\in C^1([a,b])$, el trabajo de $F$ a lo largo de $\gamma$ es:
    \begin{equation*}
        \int_{a}^{b} \langle F(\gamma(t)),\gamma'(t) \rangle~dt 
    \end{equation*}
\end{definicion}

Ahora, podemos reformular el teorema anteriormente demostrado:

\begin{teo}
    Si $G\subseteq \mathbb{R}^2$ tiene forma de estrella y sea una función $F:G\rightarrow\mathbb{R}^2$ de clase $C^1$ para la cual se cumple la condición de exactitud:
    \begin{equation*}
        \dfrac{\partial F_1}{\partial x_2} = \dfrac{\partial F_2}{\partial x_1}
    \end{equation*}
    Entonces, $F$ es conservativo.
\end{teo}
Notemos que si tenemos dos caminos distintos, el trabajo no tiene por qué ser necesariamente igual. Esto sólo ocurre si el campo es conservativo.
\begin{gather*}
    \int_{a}^{b} \langle F(\gamma(t)), \gamma'(t) \rangle~dt  = \int_{a}^{b} \left[\dfrac{\partial U}{\partial x_1}(\gamma(t), \gamma_1'(t)) + \dfrac{\partial U}{\partial x_2}(\gamma(t),\gamma_2'(t))\right]~dt  = \\
    = \int_{a}^{b} \dfrac{d}{dt}[U(\gamma(t))]~dt = U(\gamma(b)) - U(\gamma(a))
\end{gather*}
Por tanto, el trabajo para ir de un punto a otro es la diferencia del potencial, gracias a la regla de Barrow.

Teníamos el dominio $G$ con $z_\ast$ y lo que hemos hecho ha sido fijar un nivel de potencial 0. Hemos cogido un camino rectilíneo entre cada punto y $z_\ast$, obteniendo un trabajo para cada camino.
