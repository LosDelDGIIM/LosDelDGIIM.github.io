\section{Cambios de Varible}

\begin{ejercicio}
    Estudie las soluciones de la ecuación
    \begin{equation*}
        x' = \dfrac{t-5}{x^2}
    \end{equation*}
    dando en cada caso su intervalo maximal de definición.\\

    Tenemos que se trata de una ecuación de variables separadas de la forma dada por $x' = p(t)q(x)$, con:
    \Func{p}{I}{\bb{R}}{t}{t-5}
    \Func{q}{J}{\bb{R}}{x}{\dfrac{1}{x^2}}
    donde consideramos $I=\bb{R}$ y, para que el dominio sea conexo, podemos considerar $J=\bb{R}^+$ o $J=\bb{R}^-$.\\

    Usamos por tanto el método de variables separadas. En primer lugar, comprobamos que $q$ no tiene raíces en $J$:
    \begin{equation*}
        q(x)=0 \Longleftrightarrow \dfrac{1}{x^2}=0 \Longleftrightarrow 1=0
    \end{equation*}
    Una vez comprobado esto, procedemos a resolver la ecuación usando el método de variables separadas:
    \begin{align*}
        \dfrac{dx}{dt} = \dfrac{t-5}{x^2} &\Longrightarrow x^2dx = (t-5)dt \Longrightarrow \int x^2dx = \int (t-5)dt \Longrightarrow \\ &\Longrightarrow \dfrac{x^3}{3} = \dfrac{t^2}{2} - 5t + C' \qquad C'\in \bb{R}
    \end{align*}

    Despejando $x$ obtenemos la solución de la ecuación diferencial:
    \begin{equation*}
        x(t) = \sqrt[3]{\dfrac{3}{2}t^2 - 15t + C} \qquad C\in \bb{R}
    \end{equation*}

    Busquemos ahora su intervalo maximal de definición (llamémoslo $\wh{I}\subset I$). Necesitamos que $x(t)\in J$ para todo $t\in \wh{I}$ y que $x$ sea derivable en $\wh{I}$. Distinguimos casos:
    \begin{itemize}
        \item \ul{$J=\bb{R}^+$}: En este caso, necesitamos que $x(t)>0$ para todo $t\in \wh{I}$. Para ello, basta con que el radicando sea positivo:
        \begin{equation*}
            \dfrac{3}{2}t^2 - 15t + C > 0
        \end{equation*}

        Veamos en qué puntos se anula el radicando:
        \begin{align*}
            \dfrac{3}{2}t^2 - 15t + C &= 0 \Longrightarrow t = \dfrac{15\pm\sqrt{225-6C}}{3} = 5\pm\sqrt{25-\dfrac{2C}{3}}
        \end{align*}

        Distinguimos en función de $C$:
        \begin{equation*}
            25 - \dfrac{2C}{3} = 0 \Longrightarrow C = \dfrac{75}{2}
        \end{equation*}
        \begin{itemize}
            \item \ul{$C>\nicefrac{75}{2}$}: En este caso, el último radicando es negativo, luego no se anula el radicando de $x$, y este es siempre positivo. Por tanto, $x(t)>0$ para todo $t\in I$; es decir, $x(t)\in J$ para todo $t\in I$. Además, $x$ es derivable en $I$, luego el intervalo maximal de definición es $I$, $\wh{I}=I$.
            
            \item \ul{$C=\nicefrac{75}{2}$}: En este caso, el último radicando se anula en $t=5$. Por tanto, $x(t)>0$ para $t\in I\setminus \{5\}$. Por tanto, como el intervalo de definición de la solución debe ser conexo, consideramos las dos siguientes opciones:
            \begin{equation*}
                I_1 = \left]-\infty, 5\right[ \qquad I_2 = \left]5, +\infty\right[
            \end{equation*}

            En ambos casos, como $x(t)\in J$ para todo $t\in I_1$ y todo $t\in I_2$, y $x$ es derivable en $I_1$ y $I_2$, el intervalo maximal de definición es $\wh{I}=I_1$ o $\wh{I}=I_2$.

            \item \ul{$C<\nicefrac{75}{2}$}: En este caso, el último radicando es positivo, luego se anula en dos puntos, $t_1$ y $t_2$ dados por:
            \begin{equation*}
                t_1 = 5-\sqrt{25-\dfrac{2C}{3}} \qquad t_2 = 5+\sqrt{25-\dfrac{2C}{3}}
            \end{equation*}

            Por tanto, $x(t)>0$ para $t\in I\setminus [t_1,t_2]$. Por tanto, como el intervalo de definición de la solución debe ser conexo, consideramos las dos siguientes opciones:
            \begin{equation*}
                I_1 = \left]-\infty, t_1\right[ \qquad I_2 = \left]t_2, +\infty\right[
            \end{equation*}

            En todos los casos, como $x(t)\in J$ para todo $t\in I_1$ y todo $t\in I_2$, y $x$ es derivable en $I_1$ y $I_2$, el intervalo maximal de definición es $\wh{I}=I_1$ o $\wh{I}=I_2$.
        \end{itemize}
        
        \item \ul{$J=\bb{R}^-$}: En este caso, necesitamos que $x(t)<0$ para todo $t\in \wh{I}$. Para ello, basta con que el radicando sea negativo:
        \begin{equation*}
            \dfrac{3}{2}t^2 - 15t + C < 0
        \end{equation*}

        Distinguimos en función de $C$:
        \begin{itemize}
            \item \ul{$C>\nicefrac{75}{2}$}: En este caso, el último radicando es negativo, luego no se anula el radicando de $x$. Además, $x(t)>0$ para todo $t\in I$, por lo que no hay solución en este caso.
            
            \item \ul{$C=\nicefrac{75}{2}$}: En este caso, el último radicando se anula en $t=5$. Por tanto, $x(t)>0$ para $t\in I\setminus \{5\}$. Además, como el intervalo de definición de la solución debe ser abierto y conexo, no hay solución en este caso.

            \item \ul{$C<\nicefrac{75}{2}$}: En este caso, el último radicando es positivo, luego se anula en dos puntos, $t_1$ y $t_2$ dados por:
            \begin{equation*}
                t_1 = 5-\sqrt{25-\dfrac{2C}{3}} \qquad t_2 = 5+\sqrt{25-\dfrac{2C}{3}}
            \end{equation*}

            Por tanto, $x(t)<0$ para $t\in [t_1,t_2]$. Como en el abierto es derivable, el intervalo maximal de definición es $\wh{I}=\left]t_1,t_2\right[$.
        \end{itemize}
        % // TODO: Revisar JJ. Joshoccas lo tiene distinto
    \end{itemize}


\end{ejercicio}

\begin{ejercicio}
    En Dinámica de Poblaciones, dos modelos muy conocidos son la ecuación de Verhulst o logística
    \begin{equation*}
        P' = P(\alpha - \beta P)
    \end{equation*}
    y la ecuación de Gompertz
    \begin{equation*}
        P' = P(\alpha - \beta \ln P)
    \end{equation*}
    siendo $P(t)$ la población a tiempo $t$ de una determinada especie y $\alpha, \beta$ parámetros positivos. Calcule en cada caso la solución con condición inicial $P(0) = 100$.\\

    Resolvamos en primer lugar la ecuación de Verhulst. Se trata de una ecuación de variables separadas de la forma $P' = p(t)q(P)$, con:
    \Func{p}{I}{\bb{R}}{t}{1}
    \Func{q}{J}{\bb{R}}{P}{P(\alpha - \beta P)}
    donde consideramos $I=J=\bb{R}$. Comprobamos las raíces de $q$ en $J$:
    \begin{equation*}
        q(P) = 0 \Longleftrightarrow P(\alpha - \beta P) = 0 \Longleftrightarrow P=0, \dfrac{\alpha}{\beta}
    \end{equation*}

    Por tanto, dos soluciones de la ecuación son, para todo $t\in I$:
    \begin{equation*}
        P(t) = 0 \qquad P(t) = \dfrac{\alpha}{\beta}
    \end{equation*}

    Procedemos ahora a resolver la ecuación de variables separadas. Para ello, trabajaremos con $J_1=\bb{R}^-$, $J_2=\left]0, \nicefrac{\alpha}{\beta}\right[$ y $J_3=\left]\nicefrac{\alpha}{\beta}, +\infty\right[$,
    ya que necesitamos que $q(P)\neq 0$ para todo $P$ en el la segunda componente del dominio.
    \begin{itemize}
        \item \ul{$J_1=\bb{R}^-$}:
        
        Como en este caso no cumple que $P(0)=100\in J_1$, no nos interesa este dominio.

        \item \ul{$J_2=\left]0, \nicefrac{\alpha}{\beta}\right[$}:
        
        En este caso, podría ocurrir que $P(0)=100\in J_2$ (dependiendo de los valores de $\alpha$ y $\beta$). Por tanto, resolvemos la ecuación de variables separadas con dominio $I\times J_2$:
        \begin{align*}
            P' = P(\alpha - \beta P) &\Longrightarrow \dfrac{dP}{dt} = P(\alpha - \beta P) \Longrightarrow \dfrac{dP}{P(\alpha - \beta P)} = dt \Longrightarrow \\ &\Longrightarrow \int \dfrac{dP}{P(\alpha - \beta P)} = \int dt
        \end{align*}

        Para resolver la primera integral, aplicamos el método de descomposición en fracciones simples:
        \begin{align*}
            \dfrac{1}{P(\alpha - \beta P)} &= \dfrac{A}{P} + \dfrac{B}{\alpha - \beta P} = \dfrac{A(\alpha - \beta P) + BP}{P(\alpha - \beta P)}
        \end{align*}
        \begin{itemize}
            \item \ul{Para $P=0$}: $1=A \cdot \alpha \Longrightarrow A=\nicefrac{1}{\alpha}$.
            \item \ul{Para $P=\nicefrac{\alpha}{\beta}$}: $1=B \cdot \nicefrac{\alpha}{\beta} \Longrightarrow B=\nicefrac{\beta}{\alpha}$.
        \end{itemize}

        Por tanto, tenemos que:
        \begin{align*}
            \Longrightarrow \int \dfrac{dP}{P(\alpha - \beta P)} = \int dt
            &\Longrightarrow \frac{1}{\alpha}\int \dfrac{dP}{P} + \frac{\beta}{\alpha}\int \dfrac{dP}{\alpha - \beta P} = \int dt \Longrightarrow \\ &\Longrightarrow \dfrac{1}{\alpha}\ln(P) - \dfrac{1}{\alpha}\ln(\alpha - \beta P) = t + C \qquad C\in \bb{R}
        \end{align*}
        donde en la última implicación hemos usado que $P\in J_2$.

        Operando con la solución obtenida, llegamos a:
        \begin{align*}
            \ln\left(\dfrac{P}{\alpha - \beta P}\right) &= \alpha(t + C) \Longrightarrow \dfrac{P}{\alpha - \beta P} = e^{\alpha (t + C)}
            \Longrightarrow\\ &\Longrightarrow P(1+\beta e^{\alpha (t + C)}) = \alpha e^{\alpha (t + C)} \Longrightarrow P = \dfrac{\alpha e^{\alpha (t + C)}}{1+\beta e^{\alpha (t + C)}}
        \end{align*}

        Por tanto, para $P\in J_2$, la familia de soluciones es:
        \begin{equation*}
            P(t) = \dfrac{\alpha e^{\alpha (t + C)}}{1+\beta e^{\alpha (t + C)}} \qquad C\in \bb{R}
        \end{equation*}

        Estableciendo la condición inicial $P(0)=100$, obtenemos:
        \begin{align*}
            P(0) &= 100 = \dfrac{\alpha e^{\alpha C}}{1+\beta e^{\alpha C}} \Longrightarrow 100(1+\beta e^{\alpha C}) = \alpha e^{\alpha C} \Longrightarrow \\ &\Longrightarrow 100 + 100\beta e^{\alpha C} = \alpha e^{\alpha C} \Longrightarrow \\ &\Longrightarrow 100 = \alpha e^{\alpha C} - 100\beta e^{\alpha C} \Longrightarrow \\ &\Longrightarrow 100 = e^{\alpha C}(\alpha - 100\beta) \Longrightarrow \\ &\Longrightarrow e^{\alpha C} = \dfrac{100}{\alpha - 100\beta} \Longrightarrow \\ &\Longrightarrow C = \ln\left(\dfrac{100}{\alpha - 100\beta}\right)
        \end{align*}
        Para esto, es necesario que $\alpha - 100\beta > 0$, comprobémoslo:
        \begin{equation*}
            \alpha - 100\beta > 0 \Longleftrightarrow \alpha > 100\beta \Longleftrightarrow \dfrac{\alpha}{\beta} > 100=P(0)
        \end{equation*}
        Como estamos en el caso de que $P=100\in J_2$, tenemos que esto es cierto. Por tanto, la solución con condición inicial $P(0)=100$ es:
        \begin{equation*}
            P(t) = \dfrac{\alpha e^{\alpha (t + C)}}{1+\beta e^{\alpha (t + C)}}, \quad t\in I, \qquad C=\ln\left(\dfrac{100}{\alpha - 100\beta}\right)
        \end{equation*}
    % // TODO: Arturo Cont
    \end{itemize}

\end{ejercicio}

\begin{ejercicio}
    Nos planteamos resolver la ecuación
    \begin{equation*}
        x' = \cos(t - x)
    \end{equation*}
    Compruebe que el cambio $y = t - x$ nos lleva a una ecuación de variables separadas. Resuelva e invierta el cambio para llegar a una expresión explícita de $x(t)$. Repase el procedimiento por si se ha perdido alguna solución por el camino.
\end{ejercicio}

\begin{ejercicio}
    Experimentalmente, se sabe que la resistencia al aire de un cuerpo en caída libre es proporcional al cuadrado de la velocidad del mismo. Por tanto, si $v(t)$ es la velocidad a tiempo $t$, la ecuación de Newton nos dice que
    \begin{equation*}
        v' + \dfrac{k}{m}v^2 = g,
    \end{equation*}
    donde $m$ es la masa del cuerpo, $g$ es la constante de gravitación universal y $k > 0$ depende de la geometría (aerodinámica) del cuerpo. Si se supone que $v(0) = 0$, calcule la solución explícita y describa el comportamiento a largo plazo.
\end{ejercicio}

\begin{ejercicio}
    Calcule la solución de la ecuación
    \begin{equation*}
        y' = \dfrac{x + y - 3}{x - y - 1}
    \end{equation*}
    que verifica $y(0) = 1$.
\end{ejercicio}

\begin{ejercicio}
    Resuelva los siguientes problemas lineales
    \begin{enumerate}
        \item $x' + 3x = e^{-3t}$, $x(1) = 5$
        \item $x' - \dfrac{x}{t} = \dfrac{1}{1+t^2}$, $x(2) = 0$
        \item $x' = \cosh t \cdot x + \sinh t$, $x(0) = 1$
    \end{enumerate}
\end{ejercicio}

\begin{ejercicio}
    Sean $a, b : \bb{R} \to \bb{R}$ funciones continuas con $a(t) \geq c > 0$ para todo $t$ y
    \begin{equation*}
        \lim_{t \to +\infty} b(t) = 0.
    \end{equation*}
    Demuestre que todas las soluciones de la ecuación $x' = -a(t)x + b(t)$ tienden a cero cuando $t \to +\infty$. (Indicación: regla de L'Hôpital en la fórmula de variación de constantes)
\end{ejercicio}

\begin{ejercicio}
    La ecuación de Bernoulli tiene la forma
    \begin{equation*}
        x' = a(t)x + b(t)x^n,
    \end{equation*}
    donde $a, b : I \to \bb{R}$ son funciones continuas y $n \in \bb{R}$. Compruebe que el cambio de variable $y = x^\alpha$ lleva la ecuación de Bernoulli a una ecuación del mismo tipo, y ajuste el valor de $\alpha$ para que la ecuación obtenida sea lineal ($n = 0$). Usando el cambio anterior, resuelva los problemas de valores iniciales
    \begin{equation*}
        x' = x + t\sqrt{x}, \quad x(0) = 1.
    \end{equation*}
\end{ejercicio}

\begin{ejercicio}
    Se considera la ecuación de Ricatti
    \begin{equation*}
        y' = -\dfrac{1}{x^2} - \dfrac{y}{x} + y^2.
    \end{equation*}
    Encuentre una solución particular de la forma $y(x) = x^\alpha$ con $\alpha \in \bb{R}$. Usando esta solución particular, calcule la solución que cumple $y(1) = 2$ y estudie su intervalo maximal de definición.
\end{ejercicio}

\begin{ejercicio}
    Encuentre una curva $y = y(x)$ que pase por el punto $(1, 2)$ y cumpla la siguiente propiedad: la distancia de cada punto de la curva al origen coincide con la segunda coordenada del punto de corte de la recta tangente y el eje de ordenadas. (C. Sturm, Cours d’Analyse 1859, Vol 2, pag 41).
\end{ejercicio}

\begin{ejercicio}
    Identifique la clase de ecuaciones invariantes por el grupo de transformaciones $s = \lambda t$, $y = \lambda^2 x$, con $\lambda > 0$.
\end{ejercicio}

\begin{ejercicio}
    Resuelva los problemas 42 y 45 (pag. 79) del libro de Nagle-Saff-Sneider.
\end{ejercicio}