\section{Cambios de Varible}

\begin{ejercicio}
    Estudie las soluciones de la ecuación
    \begin{equation*}
        x' = \dfrac{t-5}{x^2}
    \end{equation*}
    dando en cada caso su intervalo maximal de definición.
\end{ejercicio}

\begin{ejercicio}
    En Dinámica de Poblaciones, dos modelos muy conocidos son la ecuación de Verhulst o logística
    \begin{equation*}
        P' = P(\alpha - \beta P)
    \end{equation*}
    y la ecuación de Gompertz
    \begin{equation*}
        P' = P(\alpha - \beta \ln P)
    \end{equation*}
    siendo $P(t)$ la población a tiempo $t$ de una determinada especie y $\alpha, \beta$ parámetros positivos. Calcule en cada caso la solución con condición inicial $P(0) = 100$.
\end{ejercicio}

\begin{ejercicio}
    Nos planteamos resolver la ecuación
    \begin{equation*}
        x' = \cos(t - x)
    \end{equation*}
    Compruebe que el cambio $y = t - x$ nos lleva a una ecuación de variables separadas. Resuelva e invierta el cambio para llegar a una expresión explícita de $x(t)$. Repase el procedimiento por si se ha perdido alguna solución por el camino.
\end{ejercicio}

\begin{ejercicio}
    Experimentalmente, se sabe que la resistencia al aire de un cuerpo en caída libre es proporcional al cuadrado de la velocidad del mismo. Por tanto, si $v(t)$ es la velocidad a tiempo $t$, la ecuación de Newton nos dice que
    \begin{equation*}
        v' + \dfrac{k}{m}v^2 = g,
    \end{equation*}
    donde $m$ es la masa del cuerpo, $g$ es la constante de gravitación universal y $k > 0$ depende de la geometría (aerodinámica) del cuerpo. Si se supone que $v(0) = 0$, calcule la solución explícita y describa el comportamiento a largo plazo.
\end{ejercicio}

\begin{ejercicio}
    Calcule la solución de la ecuación
    \begin{equation*}
        y' = \dfrac{x + y - 3}{x - y - 1}
    \end{equation*}
    que verifica $y(0) = 1$.
\end{ejercicio}

\begin{ejercicio}
    Resuelva los siguientes problemas lineales
    \begin{enumerate}
        \item $x' + 3x = e^{-3t}$, $x(1) = 5$
        \item $x' - \dfrac{x}{t} = \dfrac{1}{1+t^2}$, $x(2) = 0$
        \item $x' = \cosh t \cdot x + \sinh t$, $x(0) = 1$
    \end{enumerate}
\end{ejercicio}

\begin{ejercicio}
    Sean $a, b : \bb{R} \to \bb{R}$ funciones continuas con $a(t) \geq c > 0$ para todo $t$ y
    \begin{equation*}
        \lim_{t \to +\infty} b(t) = 0.
    \end{equation*}
    Demuestre que todas las soluciones de la ecuación $x' = -a(t)x + b(t)$ tienden a cero cuando $t \to +\infty$. (Indicación: regla de L'Hôpital en la fórmula de variación de constantes)
\end{ejercicio}

\begin{ejercicio}
    La ecuación de Bernoulli tiene la forma
    \begin{equation*}
        x' = a(t)x + b(t)x^n,
    \end{equation*}
    donde $a, b : I \to \bb{R}$ son funciones continuas y $n \in \bb{R}$. Compruebe que el cambio de variable $y = x^\alpha$ lleva la ecuación de Bernoulli a una ecuación del mismo tipo, y ajuste el valor de $\alpha$ para que la ecuación obtenida sea lineal ($n = 0$). Usando el cambio anterior, resuelva los problemas de valores iniciales
    \begin{equation*}
        x' = x + t\sqrt{x}, \quad x(0) = 1.
    \end{equation*}
\end{ejercicio}

\begin{ejercicio}
    Se considera la ecuación de Ricatti
    \begin{equation*}
        y' = -\dfrac{1}{x^2} - \dfrac{y}{x} + y^2.
    \end{equation*}
    Encuentre una solución particular de la forma $y(x) = x^\alpha$ con $\alpha \in \bb{R}$. Usando esta solución particular, calcule la solución que cumple $y(1) = 2$ y estudie su intervalo maximal de definición.
\end{ejercicio}

\begin{ejercicio}
    Encuentre una curva $y = y(x)$ que pase por el punto $(1, 2)$ y cumpla la siguiente propiedad: la distancia de cada punto de la curva al origen coincide con la segunda coordenada del punto de corte de la recta tangente y el eje de ordenadas. (C. Sturm, Cours d’Analyse 1859, Vol 2, pag 41).
\end{ejercicio}

\begin{ejercicio}
    Identifique la clase de ecuaciones invariantes por el grupo de transformaciones $s = \lambda t$, $y = \lambda^2 x$, con $\lambda > 0$.
\end{ejercicio}

\begin{ejercicio}
    Resuelva los problemas 42 y 45 (pag. 79) del libro de Nagle-Saff-Sneider.
\end{ejercicio}