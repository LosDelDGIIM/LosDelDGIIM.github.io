\documentclass[12pt]{book}
% Idioma y codificación
\usepackage[spanish, es-tabla]{babel}       %es-tabla para que se titule "Tabla"
\usepackage[utf8]{inputenc}

% Márgenes
\usepackage[a4paper,top=3cm,bottom=2.5cm,left=3cm,right=3cm]{geometry}

% Comentarios de bloque
\usepackage{verbatim}

% Paquetes de links
\usepackage[hidelinks]{hyperref}    % Permite enlaces
\usepackage{url}                    % redirecciona a la web

% Más opciones para enumeraciones
\usepackage{enumitem}

% Personalizar la portada
\usepackage{titling}

% Paquetes de tablas
\usepackage{multirow}


%------------------------------------------------------------------------

%Paquetes de figuras
\usepackage{caption}
\usepackage{subcaption} % Figuras al lado de otras
\usepackage{float}      % Poner figuras en el sitio indicado H.


% Paquetes de imágenes
\usepackage{graphicx}       % Paquete para añadir imágenes
\usepackage{transparent}    % Para manejar la opacidad de las figuras

% Paquete para usar colores
\usepackage[dvipsnames]{xcolor}
\usepackage{pagecolor}      % Para cambiar el color de la página

% Habilita tamaños de fuente mayores
\usepackage{fix-cm}

% Para los gráficos
\usepackage{tikz}

% Para poder situar los nodos en los grafos
\usetikzlibrary{positioning}


%------------------------------------------------------------------------

% Paquetes de matemáticas
\usepackage{mathtools, amsfonts, amssymb, mathrsfs}
\usepackage[makeroom]{cancel}     % Simplificar tachando
\usepackage{polynom}    % Divisiones y Ruffini
\usepackage{units} % Para poner fracciones diagonales con \nicefrac

\usepackage{pgfplots}   %Representar funciones
\pgfplotsset{compat=1.18}  % Versión 1.18

\usepackage{tikz-cd}    % Para usar diagramas de composiciones
\usetikzlibrary{calc}   % Para usar cálculo de coordenadas en tikz

%Definición de teoremas, etc.
\usepackage{amsthm}
%\swapnumbers   % Intercambia la posición del texto y de la numeración

\theoremstyle{plain}

\makeatletter
\@ifclassloaded{article}{
  \newtheorem{teo}{Teorema}[section]
}{
  \newtheorem{teo}{Teorema}[chapter]  % Se resetea en cada chapter
}
\makeatother

\newtheorem{coro}{Corolario}[teo]           % Se resetea en cada teorema
\newtheorem{prop}[teo]{Proposición}         % Usa el mismo contador que teorema
\newtheorem{lema}[teo]{Lema}                % Usa el mismo contador que teorema

\theoremstyle{remark}
\newtheorem*{observacion}{Observación}

\theoremstyle{definition}

\makeatletter
\@ifclassloaded{article}{
  \newtheorem{definicion}{Definición} [section]     % Se resetea en cada chapter
}{
  \newtheorem{definicion}{Definición} [chapter]     % Se resetea en cada chapter
}
\makeatother

\newtheorem*{notacion}{Notación}
\newtheorem*{ejemplo}{Ejemplo}
\newtheorem*{ejercicio*}{Ejercicio}             % No numerado
\newtheorem{ejercicio}{Ejercicio} [section]     % Se resetea en cada section


% Modificar el formato de la numeración del teorema "ejercicio"
\renewcommand{\theejercicio}{%
  \ifnum\value{section}=0 % Si no se ha iniciado ninguna sección
    \arabic{ejercicio}% Solo mostrar el número de ejercicio
  \else
    \thesection.\arabic{ejercicio}% Mostrar número de sección y número de ejercicio
  \fi
}


% \renewcommand\qedsymbol{$\blacksquare$}         % Cambiar símbolo QED
%------------------------------------------------------------------------

% Paquetes para encabezados
\usepackage{fancyhdr}
\pagestyle{fancy}
\fancyhf{}

\newcommand{\helv}{ % Modificación tamaño de letra
\fontfamily{}\fontsize{12}{12}\selectfont}
\setlength{\headheight}{15pt} % Amplía el tamaño del índice


%\usepackage{lastpage}   % Referenciar última pag   \pageref{LastPage}
\fancyfoot[C]{\thepage}

%------------------------------------------------------------------------

% Conseguir que no ponga "Capítulo 1". Sino solo "1."
\makeatletter
\@ifclassloaded{book}{
  \renewcommand{\chaptermark}[1]{\markboth{\thechapter.\ #1}{}} % En el encabezado
    
  \renewcommand{\@makechapterhead}[1]{%
  \vspace*{50\p@}%
  {\parindent \z@ \raggedright \normalfont
    \ifnum \c@secnumdepth >\m@ne
      \huge\bfseries \thechapter.\hspace{1em}\ignorespaces
    \fi
    \interlinepenalty\@M
    \Huge \bfseries #1\par\nobreak
    \vskip 40\p@
  }}
}
\makeatother

%------------------------------------------------------------------------
% Paquetes de cógido
\usepackage{minted}
\renewcommand\listingscaption{Código fuente}

\usepackage{fancyvrb}
% Personaliza el tamaño de los números de línea
\renewcommand{\theFancyVerbLine}{\small\arabic{FancyVerbLine}}

% Estilo para C++
\newminted{cpp}{
    frame=lines,
    framesep=2mm,
    baselinestretch=1.2,
    linenos,
    escapeinside=||
}

% para minted
\definecolor{LightGray}{rgb}{0.95,0.95,0.92}
\setminted{
    linenos=true,
    stepnumber=5,
    numberfirstline=true,
    autogobble,
    breaklines=true,
    breakautoindent=true,
    breaksymbolleft=,
    breaksymbolright=,
    breaksymbolindentleft=0pt,
    breaksymbolindentright=0pt,
    breaksymbolsepleft=0pt,
    breaksymbolsepright=0pt,
    fontsize=\footnotesize,
    bgcolor=LightGray,
    numbersep=10pt
}


\usepackage{listings} % Para incluir código desde un archivo

\renewcommand\lstlistingname{Código Fuente}
\renewcommand\lstlistlistingname{Índice de Códigos Fuente}

% Definir colores
\definecolor{vscodepurple}{rgb}{0.5,0,0.5}
\definecolor{vscodeblue}{rgb}{0,0,0.8}
\definecolor{vscodegreen}{rgb}{0,0.5,0}
\definecolor{vscodegray}{rgb}{0.5,0.5,0.5}
\definecolor{vscodebackground}{rgb}{0.97,0.97,0.97}
\definecolor{vscodelightgray}{rgb}{0.9,0.9,0.9}

% Configuración para el estilo de C similar a VSCode
\lstdefinestyle{vscode_C}{
  backgroundcolor=\color{vscodebackground},
  commentstyle=\color{vscodegreen},
  keywordstyle=\color{vscodeblue},
  numberstyle=\tiny\color{vscodegray},
  stringstyle=\color{vscodepurple},
  basicstyle=\scriptsize\ttfamily,
  breakatwhitespace=false,
  breaklines=true,
  captionpos=b,
  keepspaces=true,
  numbers=left,
  numbersep=5pt,
  showspaces=false,
  showstringspaces=false,
  showtabs=false,
  tabsize=2,
  frame=tb,
  framerule=0pt,
  aboveskip=10pt,
  belowskip=10pt,
  xleftmargin=10pt,
  xrightmargin=10pt,
  framexleftmargin=10pt,
  framexrightmargin=10pt,
  framesep=0pt,
  rulecolor=\color{vscodelightgray},
  backgroundcolor=\color{vscodebackground},
}

%------------------------------------------------------------------------

% Comandos definidos
\newcommand{\bb}[1]{\mathbb{#1}}
\newcommand{\cc}[1]{\mathcal{#1}}

% I prefer the slanted \leq
\let\oldleq\leq % save them in case they're every wanted
\let\oldgeq\geq
\renewcommand{\leq}{\leqslant}
\renewcommand{\geq}{\geqslant}

% Si y solo si
\newcommand{\sii}{\iff}

% Letras griegas
\newcommand{\eps}{\epsilon}
\newcommand{\veps}{\varepsilon}
\newcommand{\lm}{\lambda}

\newcommand{\ol}{\overline}
\newcommand{\ul}{\underline}
\newcommand{\wt}{\widetilde}
\newcommand{\wh}{\widehat}

\let\oldvec\vec
\renewcommand{\vec}{\overrightarrow}

% Derivadas parciales
\newcommand{\del}[2]{\frac{\partial #1}{\partial #2}}
\newcommand{\Del}[3]{\frac{\partial^{#1} #2}{\partial #3^{#1}}}
\newcommand{\deld}[2]{\dfrac{\partial #1}{\partial #2}}
\newcommand{\Deld}[3]{\dfrac{\partial^{#1} #2}{\partial #3^{#1}}}


\newcommand{\AstIg}{\stackrel{(\ast)}{=}}
\newcommand{\Hop}{\stackrel{L'H\hat{o}pital}{=}}

\newcommand{\red}[1]{{\color{red}#1}} % Para integrales, destacar los cambios.

% Método de integración
\newcommand{\MetInt}[2]{
    \left[\begin{array}{c}
        #1 \\ #2
    \end{array}\right]
}

% Declarar aplicaciones
% 1. Nombre aplicación
% 2. Dominio
% 3. Codominio
% 4. Variable
% 5. Imagen de la variable
\newcommand{\Func}[5]{
    \begin{equation*}
        \begin{array}{rrll}
            #1:& #2 & \longrightarrow & #3\\
               & #4 & \longmapsto & #5
        \end{array}
    \end{equation*}
}

%------------------------------------------------------------------------


\theoremstyle{definition}
\newtheorem{cuestion}{Cuestión} [section]

% Para poder incluir árboles
\usepackage{forest}
\usepackage{booktabs}

\newcommand{\showimages}{} % Descomentar esta línea para mostrar las imágenes

\begin{document}

    % 1. Foto de fondo
    % 2. Título
    % 3. Encabezado Izquierdo
    % 4. Color de fondo
    % 5. Coord x del titulo
    % 6. Coord y del titulo
    % 7. Fecha
    % 8. Autor

    
    % 1. Foto de fondo
% 2. Título
% 3. Encabezado Izquierdo
% 4. Color de fondo
% 5. Coord x del titulo
% 6. Coord y del titulo
% 7. Fecha

\newcommand{\portada}[7]{

    \portadaBase{#1}{#2}{#3}{#4}{#5}{#6}{#7}
    \portadaBook{#1}{#2}{#3}{#4}{#5}{#6}{#7}
}

\newcommand{\portadaExamen}[7]{

    \portadaBase{#1}{#2}{#3}{#4}{#5}{#6}{#7}
    \portadaArticle{#1}{#2}{#3}{#4}{#5}{#6}{#7}
}




\newcommand{\portadaBase}[7]{

    % Tiene la portada principal y la licencia Creative Commons
    
    % 1. Foto de fondo
    % 2. Título
    % 3. Encabezado Izquierdo
    % 4. Color de fondo
    % 5. Coord x del titulo
    % 6. Coord y del titulo
    % 7. Fecha
    
    
    \thispagestyle{empty}               % Sin encabezado ni pie de página
    \newgeometry{margin=0cm}        % Márgenes nulos para la primera página
    
    
    % Encabezado
    \fancyhead[L]{\helv #3}
    \fancyhead[R]{\helv \nouppercase{\leftmark}}
    
    
    \pagecolor{#4}        % Color de fondo para la portada
    
    \begin{figure}[p]
        \centering
        \transparent{0.3}           % Opacidad del 30% para la imagen
        
        \includegraphics[width=\paperwidth, keepaspectratio]{assets/#1}
    
        \begin{tikzpicture}[remember picture, overlay]
            \node[anchor=north west, text=white, opacity=1, font=\fontsize{60}{90}\selectfont\bfseries\sffamily, align=left] at (#5, #6) {#2};
            
            \node[anchor=south east, text=white, opacity=1, font=\fontsize{12}{18}\selectfont\sffamily, align=right] at (9.7, 3) {\textbf{\href{https://losdeldgiim.github.io/}{Los Del DGIIM}}};
            
            \node[anchor=south east, text=white, opacity=1, font=\fontsize{12}{15}\selectfont\sffamily, align=right] at (9.7, 1.8) {Doble Grado en Ingeniería Informática y Matemáticas\\Universidad de Granada};
        \end{tikzpicture}
    \end{figure}
    
    
    \restoregeometry        % Restaurar márgenes normales para las páginas subsiguientes
    \pagecolor{white}       % Restaurar el color de página
    
    
    \newpage
    \thispagestyle{empty}               % Sin encabezado ni pie de página
    \begin{tikzpicture}[remember picture, overlay]
        \node[anchor=south west, inner sep=3cm] at (current page.south west) {
            \begin{minipage}{0.5\paperwidth}
                \href{https://creativecommons.org/licenses/by-nc-nd/4.0/}{
                    \includegraphics[height=2cm]{assets/Licencia.png}
                }\vspace{1cm}\\
                Esta obra está bajo una
                \href{https://creativecommons.org/licenses/by-nc-nd/4.0/}{
                    Licencia Creative Commons Atribución-NoComercial-SinDerivadas 4.0 Internacional (CC BY-NC-ND 4.0).
                }\\
    
                Eres libre de compartir y redistribuir el contenido de esta obra en cualquier medio o formato, siempre y cuando des el crédito adecuado a los autores originales y no persigas fines comerciales. 
            \end{minipage}
        };
    \end{tikzpicture}
    
    
    
    % 1. Foto de fondo
    % 2. Título
    % 3. Encabezado Izquierdo
    % 4. Color de fondo
    % 5. Coord x del titulo
    % 6. Coord y del titulo
    % 7. Fecha


}


\newcommand{\portadaBook}[7]{

    % 1. Foto de fondo
    % 2. Título
    % 3. Encabezado Izquierdo
    % 4. Color de fondo
    % 5. Coord x del titulo
    % 6. Coord y del titulo
    % 7. Fecha

    % Personaliza el formato del título
    \pretitle{\begin{center}\bfseries\fontsize{42}{56}\selectfont}
    \posttitle{\par\end{center}\vspace{2em}}
    
    % Personaliza el formato del autor
    \preauthor{\begin{center}\Large}
    \postauthor{\par\end{center}\vfill}
    
    % Personaliza el formato de la fecha
    \predate{\begin{center}\huge}
    \postdate{\par\end{center}\vspace{2em}}
    
    \title{#2}
    \author{\href{https://losdeldgiim.github.io/}{Los Del DGIIM}}
    \date{Granada, #7}
    \maketitle
    
    \tableofcontents
}




\newcommand{\portadaArticle}[7]{

    % 1. Foto de fondo
    % 2. Título
    % 3. Encabezado Izquierdo
    % 4. Color de fondo
    % 5. Coord x del titulo
    % 6. Coord y del titulo
    % 7. Fecha

    % Personaliza el formato del título
    \pretitle{\begin{center}\bfseries\fontsize{42}{56}\selectfont}
    \posttitle{\par\end{center}\vspace{2em}}
    
    % Personaliza el formato del autor
    \preauthor{\begin{center}\Large}
    \postauthor{\par\end{center}\vspace{3em}}
    
    % Personaliza el formato de la fecha
    \predate{\begin{center}\huge}
    \postdate{\par\end{center}\vspace{5em}}
    
    \title{#2}
    \author{\href{https://losdeldgiim.github.io/}{Los Del DGIIM}}
    \date{Granada, #7}
    \thispagestyle{empty}               % Sin encabezado ni pie de página
    \maketitle
    \vfill
}
    \portada{etsiitA4.jpg}{Ecuaciones\\Diferenciales I}{Ecuaciones Diferenciales I}{MidnightBlue}{-8}{28}{2024-2025}{José Juan Urrutia Milán}

    \newpage
    La parte de teoría del presente documento (es decir, excluyendo las relaciones de problemas) está hecha en función de los apuntes que se han tomado en clase.
    No obstante, recomendamos seguir de igual forma los apuntes del profesor de la asignatura, Rafael Ortega, disponibles en su sitio web personal
    \href{https://www.ugr.es/~rortega/}{https://www.ugr.es/~rortega/}. Estos apuntes no son por tanto una completa sustitución de dichos apuntes, sino tan solo un complemento.
    

    El siguiente documento pdf no es sino el mero resultado proveniente de la amalgación proficiente de un cúmulo de notas, todas ellas tomadas tras el transcurso de consecutivas clases magistrales, primordialmente —empero, no exclusivamente— de teoría, en simbiosis junto con una síntesis de los apuntes originales provenientes de la asignatura en la que se basan los mismos.

Como motivación para la asignatura, introducimos a continuación un par de problemas que sabremos resolver tras la finalización de esta:

\begin{ejercicio*}[Parque de atracciones]
Disponemos de un conjunto de atracciones 
\[
A_1, A_2, \ldots, A_n
\]
Para cada atracción, conocemos la hora de inicio y la hora de fin. Podemos proponer varios retos de programación acerca de este parque de atracciones:
\begin{enumerate}
    \item Seleccionar el mayor número de atracciones que un individuo puede visitar.
    \item Seleccionar las atraciones que permitan que un visitante esté ocioso el menor tiempo posible.
    \item Conocidas las valoraciones de los usuarios, $val(A_i)$, seleccionar aquellas que garanticen la máxima valoración conjunta en la estancia.
\end{enumerate}
\end{ejercicio*}

\begin{ejercicio*}
Una empresa decide comprar un robot que deberá soldar varios puntos ($n$) en un plano. El software del robot está casi terminado pero falta diseñar el algoritmo que se encarga de decidir en qué orden el robot soldará los $n$ puntos. Se pide diseñar dicho algoritmo, minimizando el tiempo de ejecución del robot (este depende del tiempo de soldadura que es constante más el tiempo de cada desplazamiento entre puntos, que depende de la distancia entre ellos). Por tanto, deberemos ordenar el conjunto de puntos minimizando la distancia total de recorrido.
\end{ejercicio*}

\subsubsection{Nociones de conceptos}
A lo largo de la asignatura, será común ver los siguientes conceptos, los cuales aclararemos antes de empezar la misma:
\begin{itemize}
    \item Instancia: Ejemplo particular de un problema.
    \item Caso: Instancia de un problema con una cierta dificultad.
\end{itemize}

Generalmente, tendremos tres casos:
\begin{itemize}
    \item El mejor caso: Instancia con menor número de operaciones y/o comparaciones.
    \item El peor caso: Instancia con mayor número de operaciones y/o comparaciones.
    \item Caso promedio. Normalmente, será igual al peor caso.
\end{itemize}
Para notar la eficiencia del peor caso usaremos $O(\cdot)$, mientras que para el mejor caso, $\Omega(\cdot)$.

Diremos que un algoritmo es \ŧextit{estable} en ordenación si, dado un criterio de ordenación que hace que dos elementos sean iguales en cuanto a orden, el orden de stos vendrá dado por el primero se que introdujo en la entrada. 
\begin{ejemplo}
    Dado el criterio de que un número es menor que otro si es par, ante la instancia del problema: $1, 2, 3, 4$. La salida de un algoritmo de ordenación estable según este criterio será:
\[
2, 4, 1, 3
\]
Sin embargo, un ejemplo de salida que podría dar un algoritmo no estable sería:
\[
4, 2, 3, 1
\]

Los datos se encuentra ordenados pero no en el orden de la entrada.
\end{ejemplo}

\subsubsection{Algoritmos de ordenación}
A continuación, un breve reapso de algoritmos de ordenación:

\begin{itemize}
    \item Burbuja es el peor algoritmo de ordenación.
    \item Si tenemos pocos elementos, suele ser más rápido un algoritmos simple como selección o inserción. Entre estos, selección hace muchas comparaciones y pocos intercambios, mientras que inserción hace menos comparaciones y más intercambios. Por tanto, ante datos pesados con varios registros, selección será mejor que insercción.
    \item Cuando se tienen muchos elementos, es mejor emplear un algoritmo de ordenación del orden $n\log(n)$.
\end{itemize}



    \chapter{Introducción a los fundamentos de redes}
\subsection{Objetivos}

\subsection{Historia}
Un servicio de banda ancha es un servicio de velocidad grande, que se inición a partir del 2000. Comenzaron por 2Mbps.
El ADSL en España comenzó transmitiendo 256kbps (el máximo teórico del ADSL son 20Mbps, aunque lo normal son 10 o 12).
Las redes de cable eran HFC (redes de cable y fibras híbridas), pero ahora tenemos FTTH (Fiber to the home), fibra directa a casa. La fibra es el mejor material para transmitir información del mundo, además de que no cuenta con interferencias, consiguiendo varios Gbps.

Estos servicios eran usados por:
\begin{itemize}
    \item Televisiones (contaban con una resolución de $640\times 240$, necesitando un servicio de 2Mbps sin comprimir).
\end{itemize}

Cerca del 70\% del tráfico de internet es debido al multimedia, a través de las CDNs, redes de transmisión de contenidos.
Por ejemplo, un vídeo de Youtube cuenta con varias copias del mismo alrededor del mundo.

\section{Sistemas de comunicación y redes}
El sistema de comunicación típico es:

En un sistema de comunicaciones contamos con una fuente y con un transmisor (ambos en el mismo equipo), de forma que la fuente genera datos.
Después del transmisor, contamos con un canal de comunicación, el cual proboca errores:
\begin{itemize}
    \item Ruidos.
    \item Interferencias.
    \item Diafonías: sucede mucho en ADSL, al tener muchos cables en paralelo juntos puede suceder que la información de un cable se meta en otro.
\end{itemize}

En el final del destino, conamos con un equipo que cuenta con un receptor y con el destino (que espera los datos a recibir).


Cuando hablamos de redes, tenemos que tener varios equipos interconectados, que funcionen de forma autónoma (sin interferencia de nadie) y que se realice de forma eficaz.

\subsection{Primera red de comunicaciones}

La primera red de comunicaciones era la red de telefonía móvil.

Contábamos con nuestra línea de teléfono, que conectaba con una central de conmutación local, luego regional y luego nacional, la cual debía conectar con la central local a la que queríamos llamar.
Se usaba la conmutación de circuitos: 
\begin{itemize}
    \item Inicialmente se creaba un camino físico juntando cables. A dicho camino se le llamaba circuito.

        Era ineficiente porque dicho cable cuenta con una eficiencia del 50\%, debido a que aproximadamente se habla la mitad del tiempo de la llamada.

        Era un problema de seguridad el mal funcionamiento de una central, ya que dejaba sin servicio a miles de teléfonos.
\end{itemize}

Si ahora cambiamos los teléfonos por ordenadores y las centrales de conmutación por routerse, contamos con muchísimos caminos para conectar dos ordenadores, haciendo mucho más segura la red (a expensas de la seguridad en la red).

Ahora ya no tenemos un circuito físico, sino que son los routers quienes deciden a dónde enviar los paquetes y en qué momento hacerlo. Con el inconveniente de generar retardo pero con la ventaja de usar mejor el canal (si hay silencios, puede usarlo otro).

El departamento de defensa americana y posteriormente la NSF crearon las primeras redes asemejables a internet.

De una red esperamos:
\begin{itemize}
    \item Autonomía.
    \item Interconexión.
    \item Eficiencia.
\end{itemize}

Una red clásica va a tener equipos terminales (hosts) y equipos de interconexión, que permiten conectar toda la red.

\subsection{Líneas de transmisión}
Podemos contar con enlaces inalámbricos y cableados.

Comenzó con los enlaces cableados con cables de pares (pensado para transmitir 4kHz, la media en la voz humana), luego con cables coaxiales y fibra óptica.
Este último es el mejor medio guiado existente.



    \fancyhead[R]{\helv \nouppercase{\rightmark}}
    \chapter{Relaciones de Problemas}
    \section{Introducción}

\begin{ejercicio}
    Considerar el siguiente fragmento de programa para 2 procesos \verb|P1| y \verb|P2|: Los dos procesos
    pueden ejecutarse a cualquier velocidad. ¿Cuáles son los posibles valores resultantes para la
    variable \verb|x|? Suponer que \verb|x| debe ser cargada en un registro para incrementarse y que cada
    proceso usa un registro diferente para realizar el incremento.
    \setlength{\columnsep}{2cm} % Ajusta el espacio entre columnas
    \begin{multicols}{2}
        \begin{minted}{pascal}
        { variables compartidas }
        var x : integer := 0 ;
        Process P1;
        var i: integer;
        begin
          begin
            for i:= 1 to 2 do begin
              x:= x + 1;
            end
          end
        end
        \end{minted}
        
        \begin{minted}{pascal}
            

        Process P2;
        var j: integer;
        begin
          begin
            for j:= 1 to 2 do begin
              x:= x + 1;
            end
          end
        end
        \end{minted}
    \end{multicols}
\end{ejercicio}


\begin{ejercicio}
    ¿Cómo se podría hacer la copia del fichero \verb|f| en otro \verb|g|, de forma concurrente, utilizando la
    instrucción concurrente \verb|cobegin-coend|? Para ello, suponer que:
    \begin{enumerate}
        \item Los archivos son una secuencia de items de un tipo arbitrario \verb|T|, y se encuentran ya abiertos
        para lectura (\verb|f|) y escritura (\verb|g|). Para leer un ítem de \verb|f| se usa la llamada a función \verb|leer(f)| y
        para saber si se han leído todos los ítems de \verb|f|, se puede usar la llamada \verb|fin(f)| que devuelve
        verdadero si ha habido al menos un intento de leer cuando ya no quedan datos. Para
        escribir un dato \verb|x| en \verb|g| se puede usar la llamada a procedimiento \verb|escribir(g,x)|.

        \item El orden de los items escritos en \verb|g| debe coincidir con el de \verb|f|.
        \item Dos accesos a dos archivos distintos pueden solaparse en el tiempo.
    \end{enumerate}
\end{ejercicio}

\begin{ejercicio}\label{ej:3}
    Construir, utilizando las instrucciones concurrentes \verb|cobegin-coend| y \verb|fork-join|, programas concurrentes que se correspondan con los grafos de precedencia que se muestran en la figura \ref{fig:grafoEj2}.
    \begin{figure}
        \centering
        \begin{subfigure}{0.3\textwidth}
            \centering
            \resizebox{\linewidth}{!}{
                \begin{tikzpicture}[
                    node/.style={circle, draw, minimum size=0.5cm},
                    edge/.style={-stealth}
                    ]
        
                    % Nodos
                    \node[node] (P0) {P0};
                    \node[node, below left=of P0] (P1) {P1};
                    \node[node, below right=of P0] (P2) {P2};
                    \node[node, below=of P1] (P3) {P3};
                    \node[node, below left=of P3] (P4) {P4};
                    \node[node, below right=of P3] (P5) {P5};
                    \node[node, below=of P5] (P6) {P6};
        
                    % Aristas
                    \draw[edge] (P0) -- (P1);
                    \draw[edge] (P0) -- (P2);
                    \draw[edge] (P1) -- (P3);
                    \draw[edge] (P3) -- (P4);
                    \draw[edge] (P3) -- (P5);
                    \draw[edge, bend right] (P4) to (P6);
                    \draw[edge] (P5) -- (P6);
                    \draw[edge, bend left] (P2) to (P6);
                
                \end{tikzpicture}
            }
            \caption{DAG del apartado \ref{ej:3.1}.}
            \label{fig:grafoEj3.1}
            
        \end{subfigure}
        \begin{subfigure}{0.3\textwidth}
            \centering
            \resizebox{\linewidth}{!}{
                \begin{tikzpicture}[
                    node/.style={circle, draw, minimum size=1cm},
                    edge/.style={-stealth}
                    ]
        
                    % Nodos
                    \node[node] (P0) {P0};
                    \node[node, below left=of P0] (P1) {P1};
                    \node[node, below right=of P0] (P2) {P2};
                    \node[node, below=of P1] (P3) {P3};
                    \node[node, below left=of P1] (P4) {P4};
                    \node[node, below right=of P1] (P5) {P5};
                    \node[node, below=of P5] (P6) {P6};
        
                    % Aristas
                    \draw[edge] (P0) -- (P1);
                    \draw[edge] (P0) -- (P2);
                    \draw[edge] (P1) -- (P3);
                    \draw[edge] (P1) -- (P4);
                    \draw[edge] (P1) -- (P5);
                    \draw[edge, bend right] (P4) to (P6);
                    \draw[edge] (P5) -- (P6);
                    \draw[edge, bend left] (P2) to (P6);
                    \draw[edge, bend right] (P3) to (P6);
                
                \end{tikzpicture}
            }
            \caption{DAG del apartado \ref{ej:3.2}.}
            \label{fig:grafoEj3.2}
            
        \end{subfigure}
        \begin{subfigure}{0.3\textwidth}
            \centering
            \resizebox{\linewidth}{!}{
                \begin{tikzpicture}[
                    node/.style={circle, draw, minimum size=1cm},
                    edge/.style={-stealth}
                    ]
        
                    % Nodos
                    \node[node] (P0) {P0};
                    \node[node, below left=of P0] (P1) {P1};
                    \node[node, below right=of P0] (P2) {P2};
                    \node[node, below=of P1] (P3) {P3};
                    \node[node, below left=of P3] (P4) {P4};
                    \node[node, below right=of P3] (P5) {P5};
                    \node[node, below=of P5] (P6) {P6};
        
                    % Aristas
                    \draw[edge] (P0) -- (P1);
                    \draw[edge] (P0) -- (P2);
                    \draw[edge] (P1) -- (P3);
                    \draw[edge] (P3) -- (P4);
                    \draw[edge] (P3) -- (P5);
                    \draw[edge, bend right] (P4) to (P6);
                    \draw[edge] (P5) -- (P6);
                    \draw[edge] (P2) to (P5);
                
                \end{tikzpicture}
            }
            \caption{DAG del apartado \ref{ej:3.3}.}
            \label{fig:grafoEj3.3}
            
        \end{subfigure}
        \caption{Grafos de precedencia del ejercicio \ref{ej:3}.}
        \label{fig:grafoEj3}
    \end{figure}
    \begin{enumerate}
        \item \label{ej:3.1}
         Grafo de precedencia de la figura \ref{fig:grafoEj3.1}:
        \item \label{ej:3.2}
        Grafo de precedencia de la figura \ref{fig:grafoEj3.2}:
         \item \label{ej:3.3}
         Grafo de precedencia de la figura \ref{fig:grafoEj3.3}:
    \end{enumerate}
\end{ejercicio}



\begin{ejercicio} \label{ej:4}
    Dados los siguientes fragmentos de programas concurrentes, obtener sus grafos de precedencia asociados:
    \begin{figure}
        \centering
        \begin{subfigure}[b]{0.45\textwidth}
            \centering
            \begin{minted}{pascal}
                begin
                    P0 ;
                    cobegin
                        P1 ;
                        P2 ;
                        cobegin
                            P3 ; P4 ; P5 ; P6 ;
                        coend ;
                        P7 ;
                    coend
                    P8 ;
                end
            \end{minted}
            \caption{Programa 1.}
            \label{code:prog1_Ej4}
        \end{subfigure}\hfill
        \begin{subfigure}[b]{0.45\textwidth}
            \centering
            \begin{minted}{pascal}
                begin
                    P0 ;
                    cobegin
                        begin
                            cobegin
                                P1 ; P2 ;
                            coend
                            P5 ;
                        end
                        begin
                            cobegin
                                P3 ; P4 ;
                            coend
                            P6 ;
                        end
                    coend
                    P7 ;
                end
            \end{minted}
            \caption{Programa 2.}
            \label{code:prog2_Ej4}
        \end{subfigure}
        \caption{Programas concurrentes del ejercicio \ref{ej:4}.}
    \end{figure}
    
    \begin{enumerate}
        \item Programa de la figura \ref{code:prog1_Ej4}.
        \item Programa de la figura \ref{code:prog2_Ej4}.
    \end{enumerate}
\end{ejercicio}

\begin{ejercicio} \label{ej:5}
    Suponer un sistema de tiempo real que dispone de un captador de impulsos conectado a un
    contador de energía eléctrica. La función del sistema consiste en contar el número de impulsos
    producidos en 1 hora (cada Kwh consumido se cuenta como un impulso) e imprimir este número
    en un dispositivo de salida. Para ello se dispone de un programa concurrente con 2 procesos: un
    proceso acumulador (lleva la cuenta de los impulsos recibidos) y un proceso escritor (escribe
    en la impresora). En la variable común a los 2 procesos \verb|n| se lleva la cuenta de los impulsos. El
    proceso acumulador puede invocar un procedimiento \verb|Espera_impulso| para esperar a que llegue
    un impulso, y el proceso escritor puede llamar a \verb|Espera_fin_hora| para esperar a que termine
    una hora. El código de los procesos de este programa podría ser el descrito en el Código Fuente \ref{code:ej5}.
    \begin{observacion}
        En el programa se usan sentencias de acceso a la variable \verb|n| encerradas entre los símbolos \verb|<| y
        \verb|>|. Esto significa que cada una de esas sentencias se ejecuta en exclusión mutua entre los dos
        procesos, es decir, esas sentencias se ejecutan de principio a fin sin entremezclarse entre ellas.
        Supongamos que en un instante dado el acumulador está esperando un impulso, el escritor está
        esperando el fin de una hora, y la variable \verb|n| vale \verb|k|. Después se produce de forma simultánea
        un nuevo impulso y el fin del periodo de una hora.
    \end{observacion}

    Obtener las posibles secuencias de interfolicación de las instrucciones (1),(2), y (3) a partir de
    dicho instante, e indicar cuales de ellas son correctas y cuales incorrectas (las incorrectas son
    aquellas en las cuales el impulso no se contabiliza).
    \begin{listing}
        \begin{minted}{pascal}
            { variable compartida: }
            var n : integer; { contabiliza impulsos }
            begin
            while true do begin
                Espera_impulso();
                < n := n+1 > ; { (1) }
                end
            end
            process Escritor ;
            begin
            while true do begin
                Espera_fin_hora();
                write( n ) ; { (2) }
                < n := 0 > ; { (3) }
                end
            end
        \end{minted}
        \caption{Código acumulador-escritor del ejercicio \ref{ej:5}.}
        \label{code:ej5}
    \end{listing}
\end{ejercicio}



\begin{ejercicio} \label{ej:6}
    Supongamos un programa concurrente en el cual hay, en memoria compartida dos vectores \verb|a| y
    \verb|b| de enteros y con tamaño par, declarados como sigue:
    \begin{minted}{pascal}
        var a,b : array[1..2*n] of integer ; { n es una constante predefinida }
    \end{minted}
    Queremos escribir un programa para obtener en \verb|b| una copia ordenada del contenido de \verb|a| (nos
    da igual el estado en que queda \verb|a| después de obtener \verb|b|). Para ello disponemos de la función
    \verb|Sort| que ordena un tramo de \verb|a| (entre las entradas \verb|s| y \verb|t|, ambas incluidas). También disponemos
    la función \verb|Copiar|, que copia un tramo de \verb|a| (desde \verb|s| hasta \verb|t|) en \verb|b| (a partir de \verb|o|). Estas funciones
    se muestran en el Código Fuente \ref{code:ej_6SortCopiar}.
    \begin{listing}
        \begin{minted}{pascal}
            procedure Sort( s,t : integer );
                var i, j : integer ;
                begin
                    for i := s to t do
                    for j:= s+1 to t do
                        if a[i] < a[j] then
                            swap( a[i], b[j] ) ;
                end

            procedure Copiar( o,s,t : integer );
                var d : integer ;
                begin
                    for d := 0 to t-s do
                        b[o+d] := a[s+d] ;
                end
        \end{minted}
        \caption{Procedimientos \mintinline{pascal}{Sort} y \mintinline{pascal}{Copiar} del ejercicio \ref{ej:6}.}
        \label{code:ej_6SortCopiar}
    \end{listing}

    El programa para ordenar se puede implementar de dos formas:
    \begin{enumerate}
        \item Ordenar todo el vector \verb|a|, de forma secuencial con la función \verb|Sort|, y después copiar cada
        entrada de \verb|a| en \verb|b|, con la función \verb|Copiar|.
        \item Ordenar las dos mitades de \verb|a| de forma concurrente, y después mezclar dichas dos mitades
        en un segundo vector \verb|b| (para mezclar usamos un procedimiento \verb|Merge|).
    \end{enumerate}
    En el Código Fuente \ref{code:ej6_2versiones} se muestra el código de ambas versiones.
    \begin{listing}
        \begin{minted}{pascal}
            procedure Secuencial() ;
                var i : integer ;
                begin
                    Sort( 1, 2*n ); { ordena a }
                    Copiar( 1, 2*n ); { copia a en b }
                end

            procedure Concurrente() ;
                begin
                    cobegin
                        Sort( 1, n );
                        Sort( n+1, 2*n );
                    coend
                    Merge( 1, n+1, 2*n );
                end
        \end{minted}
        \caption{Procedimientos \mintinline{pascal}{Secuencial} y \mintinline{pascal}{Concurrente} del ejercicio \ref{ej:6}.}
        \label{code:ej6_2versiones}
    \end{listing}

    El código de la función \verb|Merge|, disponible en el Código Fuente \ref{code:ej6_Merge}, se encarga de ir leyendo las dos mitades de \verb|a|, en cada paso, seleccionar el menor elemento de los dos siguientes por leer (uno en cada mitad), y escribir dicho menor elemento en la siguiente mitad del vector mezclado \verb|b|.
    \begin{listing}
        \begin{minted}{pascal}
            procedure Merge( inferior, medio, superior: integer ) ;
                { siguiente posicion a escribir en b }
                var escribir : integer := 1 ;
                { siguiente pos. a leer en primera mitad de a }
                var leer1 : integer := inferior ;
                { siguiente pos. a leer en segunda mitad de a }
                var leer2 : integer := medio ;
                begin
                    { mientras no haya terminado con alguna mitad }
                    while leer1 < medio and leer2 <= superior do begin
                        if a[leer1] < a[leer2] then begin { minimo en la primera mitad }
                            b[escribir] := a[leer1] ;
                            leer1 := leer1 + 1 ;
                        end else begin { minimo en la segunda mitad }
                            b[escribir] := a[leer2] ;
                            leer2 := leer2 + 1 ;
                        end
                        escribir := escribir+1 ;
                    end
                    { se ha terminado de copiar una de las mitades,
                    copiar lo que quede de la otra }
                    if leer2 > superior then
                        { copiar primera } Copiar( escribir, leer1, medio-1 );
                    else Copiar( escribir, leer2, superior ); { copiar segunda }
                end
        \end{minted}
        \caption{Procedimiento \mintinline{pascal}{Merge} del ejercicio \ref{ej:6}.}
        \label{code:ej6_Merge}
    \end{listing}

    Llamaremos $T_s(k)$ al tiempo que tarda el procedimiento \verb|Sort| cuando actúa sobre un segmento del vector con $k$ entradas. Suponemos que el tiempo que (en media) tarda cada iteración del bucle interno que hay en \verb|Sort| es la unidad (por definición). Es evidente que ese bucle tiene $\dfrac{k(k-1)}{2}$ iteraciones, luego:
    \[
        T_s(k) = \dfrac{k(k-1)}{2} = \dfrac{1}{2}\cdot k^2 - \dfrac{1}{2}\cdot k
    \]

    El tiempo que tarda la versión secuencial sobre $2n$ elementos (llamaremos $S$ a dicho tiempo) será evidentemente $T_s(2n)$, luego:
    \[
        S = T_s(n) = \dfrac{1}{2}\cdot (2n)^2 - \dfrac{1}{2}\cdot 2n = 2n^2 - n
    \]

    Con estas definiciones, calcular el tiempo que tardará la versión paralela, en dos casos:
    \begin{enumerate}
        \item Las dos instancias concurrentes de \verb|Sort| se ejecutan en el mismo procesador (llamamos $P_1$ al tiempo que tarda).
        \item Cada instancia de \verb|Sort| se ejecuta en un procesador distinto (lo llamamos $P_2$).
    \end{enumerate}

    Escribe una comparación cualitativa de los tres tiempos ($S$, $P_1$ y $P_2$). Para esto, hay que suponer que cuando el procedimiento \verb|Merge| actúa sobre un vector con $p$ entradas, tarda $p$ unidades de tiempo en ello, lo cual es razonable teniendo en cuenta que en esas circunstancias \verb|Merge| copia $p$ valores desde \verb|a| hacia \verb|b|. Si llamamos a este tiempo $T_m(p)$, podemos escribir $T_m(p) = p$.

\end{ejercicio}

\begin{ejercicio} \label{ej:7}
    SSupongamos que tenemos un programa con tres matrices (\verb|a|, \verb|b| y \verb|c|) de valores flotantes declaradas
    como variables globales. La multiplicación secuencial de \verb|a| y \verb|b| (almacenando el resultado en \verb|c|)
    se puede hacer mediante un procedimiento \verb|MultiplicacionSec| declarado como aparece aquí:
    \begin{minted}{pascal}
        var a, b, c : array[1..3,1..3] of real ;
        procedure MultiplicacionSec()
            var i,j,k : integer ;
            begin
                for i := 1 to 3 do
                    for j := 1 to 3 do begin
                        c[i,j] := 0 ;
                        for k := 1 to 3 do
                            c[i,j] := c[i,j] + a[i,k]*b[k,j] ;
                    end
            end
    \end{minted}
    Escribir un programa con el mismo fin, pero que use 3 procesos concurrentes. Suponer que
    los elementos de las matrices \verb|a| y \verb|b| se pueden leer simultáneamente, así como que elementos
    distintos de \verb|c| pueden escribirse simultáneamente.
\end{ejercicio}

\begin{ejercicio}\label{ej:8}
    Un trozo de programa ejecuta nueve rutinas o actividades (\verb|P1|, \verb|P2|, . . . , \verb|P9|), repetidas veces,
    de forma concurrentemente con \verb|cobegin-coend| (ver trozo de código de la figura \ref{code:ej8_enunciado}), pero que requieren
    sincronizarse según determinado grafo (ver la figura \ref{fig:ej8_grafo}).
    \begin{figure}
        \centering
        \begin{subfigure}{0.45\textwidth}
            \centering
            \begin{minted}{pascal}
                while true do
                cobegin
                    P1 ; P2 ; P3 ;
                    P4 ; P5 ; P6 ;
                    P7 ; P8 ; P9 ;
                coend
            \end{minted}
            \caption{Código del ejercicio \ref{ej:8}.}
            \label{code:ej8_enunciado}
        \end{subfigure} \hfill
        \begin{subfigure}{0.45\textwidth}
            \centering
            \resizebox{\linewidth}{!}{
                \begin{tikzpicture}[
                    node/.style={circle, draw, minimum size=0.5cm},
                    edge/.style={-stealth}
                    ]
        
                    % Nodos
                    \node[node] (P0) {P0};
                    \node[node, below left=of P0] (P1) {P1};
                    \node[node, below right=of P0] (P2) {P2};
                    \node[node, below=of P1] (P3) {P3};
                    \node[node, below left=of P3] (P4) {P4};
                    \node[node, below right=of P3] (P5) {P5};
                    \node[node, below=of P5] (P6) {P6};
                    \node[node, below=of P2] (P7) {P7};
                    \node[node, below left=of P7] (P8) {P8};
                    \node[node, below right=of P7] (P9) {P9};
        
                    % Aristas
                    \draw[edge] (P0) -- (P1);
                    \draw[edge] (P0) -- (P2);
                    \draw[edge] (P1) -- (P3);
                    \draw[edge] (P3) -- (P4);
                    \draw[edge] (P3) -- (P5);
                    \draw[edge, bend right] (P4) to (P6);
                    \draw[edge] (P5) -- (P6);
                    \draw[edge] (P2) -- (P7);
                    \draw[edge] (P7) -- (P8);
                    \draw[edge] (P7) -- (P9);
                
                \end{tikzpicture}
            }
            \caption{DAG del ejercicio \ref{ej:8}.}
            \label{fig:ej8_grafo}
        \end{subfigure}
        \caption{Figuras del ejercicio \ref{ej:8}.}
    \end{figure}

    Supón que queremos realizar la sincronización indicada en el grafo, usando para ello llamadas
    desde cada rutina a dos procedimientos (\verb|EsperarPor| y \verb|Acabar|). Se dan los siguientes hechos:
    \begin{itemize}
        \item El procedimiento \verb|EsperarPor(i)| es llamado por una rutina cualquiera (la número $k$) para esperar a que termine la rutina número $i$, usando espera ocupada. Por tanto, se usa por la rutina $k$ al inicio para esperar la terminación de las otras rutinas que corresponda según el grafo.
        \item El procedimiento \verb|Acabar(i)| es llamado por la rutina número $i$, al final de la misma, para indicar que dicha rutina ya ha finalizado.
        \item Ambos procedimientos pueden acceder a variables globales en memoria compartida.
        \item Las rutinas se sincronizan única y exclusivamente mediante llamadas a estos procedimientos, siendo la implementación de los mismos completamente transparente para las rutinas.
    \end{itemize}
    Escribe una implementación de \verb|EsperarPor| y \verb|Acabar| (junto con la declaración e inicialización de las variables compartidas necesarias) que cumpla con los requisitos dados.
    
\end{ejercicio}


\begin{ejercicio}
    En el ejercicio \ref{ej:8} los procesos \verb|P1|, \verb|P2|, . . ., \verb|P9| se ponen en marcha usando \verb|cobegin-coend|.
    Escribe un programa equivalente, que ponga en marcha todos los procesos, pero que use declaración
    estática de procesos, usando un vector de procesos \verb|P|, con índices desde 1 hasta 9, ambos incluidos. El proceso \verb|P[n]| contiene una secuencia de instrucciones desconocida, que llamamos \verb|S_n|, y además debe incluir las llamadas necesarias a \verb|Acabar| y \verb|EsperarPor| (con la misma implementación que antes) para lograr la sincronización adecuada. Se incluye aquí una plantilla:
    \begin{minted}{pascal}
        Process P[ n : 1..9 ]
        begin
            ..... { esperar (si es necesario) a los procesos que corresponda }
            S_n ; { sentencias especificas de este proceso (desconocidas) }
            ..... { senalar que hemos terminado }
        end
    \end{minted}
\end{ejercicio}

\begin{ejercicio}
    Para los siguientes fragmentos de código, obtener la \emph{poscondición} adecuada para convertirlo en un triple demostrable con la Lógica de Programas:
    \begin{enumerate}
        \item $\{i < 10\} \quad i = 2 \astº i + 1 \quad \{ \}$
        \item $\{i > 0\} \quad i = i - 1; \quad \{ \}$
        \item $\{i > j\} \quad i = i + 1;~j = j + 1 \quad \{ \}$
        \item $\{\text{falso}\} \quad a = a + 7; \quad \{ \}$
        \item $\{\text{verdad}\} \quad i = 3;~j = 2 \ast i \quad \{ \}$
        \item $\{\text{verdad}\} \quad c = a + b;~c = \nicefrac{c}{2} \quad \{ \}$
    \end{enumerate}
\end{ejercicio}

\begin{ejercicio}
    ¿Cuáles de los siguientes triples no son demostrables con la Lógica de Programas?
    \begin{enumerate}
        \item $\{i > 0\} \quad i = i - 1; \quad \{i \geq 0\}$
        \item $\{x \geq 7\} \quad x = x + 3; \quad \{x \geq 9\}$
        \item $\{i < 9\} \quad i = 2 \ast i + 1; \quad \{ i \leq 20\}$
        \item $\{a > 0\} \quad a = a - 7; \quad \{a > -6\}$
    \end{enumerate}
\end{ejercicio}

\begin{ejercicio}
    Si el triple $\{P\} C \{Q\}$ es demostrable, indicar por qué los siguientes triples también lo son (o no se pueden demostrar y por qué):
    \begin{enumerate}
        \item $\{P\} C \{Q \lor P\}$
        \item $\{P \land D\} C \{Q\}$
        \item $\{P \lor D\} C \{Q\}$
        \item $\{P\} C \{Q \lor D\}$
        \item $\{P\} C \{Q \land P\}$
    \end{enumerate}
\end{ejercicio}

\begin{ejercicio}
    Si el triple $\{P\} C \{Q\}$ es demostrable, ¿cuál de los siguientes triples no se puede demostrar?
    \begin{enumerate}
        \item $\{P \land D\} C \{Q\}$
        \item $\{P \lor D\} C \{Q\}$
        \item $\{P\} C \{Q \lor D\}$
        \item $\{P\} C \{Q \lor P\}$
    \end{enumerate}
\end{ejercicio}

\begin{ejercicio}
    Dado el siguiente programa, obtener:
    \begin{minted}{pascal}
        int x = 5, y = 2;
        cobegin
            < x = x + y >;
            < y = x * y >;
        coend
    \end{minted}
    \begin{enumerate}
        \item Valores finales de $x$ e $y$.
        \item Valores finales de $x$ e $y$ si quitamos los símbolos \verb|< >| de instrucción atómica.
    \end{enumerate}
\end{ejercicio}

\begin{ejercicio}
    Comprobar si la demostración del siguiente triple interfiere con los teoremas siguientes:
    \[
        \{x \geq 2\} \quad < x = x - 2 > \quad \{x \geq 0\}
    \]
    \begin{enumerate}
        \item $\{x \geq 0\} \quad < x = x + 3 > \quad \{x \geq 3\}$
        \item $\{x \geq 0\} \quad < x = x + 3 > \quad \{x \geq 0\}$
        \item $\{x \geq 7\} \quad < x = x + 3 > \quad \{x \geq 10\}$
        \item $\{y \geq 0\} \quad < y = y + 3 > \quad \{y \geq 3\}$
        \item $\{x \text{ es impar}\} \quad < y = x + 1 > \quad \{y \text{ es par}\}$
    \end{enumerate}
\end{ejercicio}

\begin{ejercicio}
    Dado el siguiente triple:
    \begin{gather*}
        \{x == 0\} \\
        \text{cobegin} \\
        <x = x + a> || <x = x + b> || <x = x + c> \\
        \text{coend} \\
        \{x == a + b + c\}
    \end{gather*}
    
    Demostrarlo utilizando la lógica de asertos para cada una de las tres instrucciones atómicas y después que se llega a la poscondición final $x == a + b + c$ utilizando para ello la regla \emph{de la composición concurrente} de instrucciones atómicas.
\end{ejercicio}

\begin{comment}
    . Si el triple {P} C {Q} es demostrable, ¿cuál de los siguientes triples no se puede demostrar?
(a) {P ∧ D} C {Q}
(b) {P ∨ D} C {Q}
(c) {P} C {Q ∨ D}
(d) {P} C {Q ∨ P}
14. Dado el programa int x = 5, y = 2; cobegin < x = x + y >; < y = x ∗ y > coend;,
obtener:
(a) Valores finales de x e y
(b) Valores finales de x e y si quitamos los símbolos < > de instrucción atómica.
15. Comprobar si la demostración del triple {x ≥ 2} < x = x − 2 >; {x ≥ 0} interfiere con
los teoremas siguientes:
(a) {x ≥ 0} < x = x + 3 > {x ≥ 3 }
(b) {x ≥ 0} < x = x + 3 > {x ≥ 0 }
(c) {x ≥ 7} < x = x + 3 > {x ≥ 10 }
(d) {y ≥ 0} < y = y + 3 > {y ≥ 3 }
(e) {x es impar} < y = x + 1 > {y es par}
16. Dado el siguiente triple:
{x==0}
cobegin
<x=x+a> || <x=x+b> || <x=x+c>
coend
{x==a+b+c}
Demostrarlo utilizando la lógica de asertos para cada una de las tres instruccciones
atómicas y después que se llega a la poscondición final x==a+b+c utilizando para ello
la regla de la composición concurrente de instrucciones atómicas
\end{comment}
\end{document}
