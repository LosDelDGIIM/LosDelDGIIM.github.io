\documentclass[12pt]{book}
% Idioma y codificación
\usepackage[spanish, es-tabla]{babel}       %es-tabla para que se titule "Tabla"
\usepackage[utf8]{inputenc}

% Márgenes
\usepackage[a4paper,top=3cm,bottom=2.5cm,left=3cm,right=3cm]{geometry}

% Comentarios de bloque
\usepackage{verbatim}

% Paquetes de links
\usepackage[hidelinks]{hyperref}    % Permite enlaces
\usepackage{url}                    % redirecciona a la web

% Más opciones para enumeraciones
\usepackage{enumitem}

% Personalizar la portada
\usepackage{titling}

% Paquetes de tablas
\usepackage{multirow}


%------------------------------------------------------------------------

%Paquetes de figuras
\usepackage{caption}
\usepackage{subcaption} % Figuras al lado de otras
\usepackage{float}      % Poner figuras en el sitio indicado H.


% Paquetes de imágenes
\usepackage{graphicx}       % Paquete para añadir imágenes
\usepackage{transparent}    % Para manejar la opacidad de las figuras

% Paquete para usar colores
\usepackage[dvipsnames]{xcolor}
\usepackage{pagecolor}      % Para cambiar el color de la página

% Habilita tamaños de fuente mayores
\usepackage{fix-cm}

% Para los gráficos
\usepackage{tikz}

% Para poder situar los nodos en los grafos
\usetikzlibrary{positioning}


%------------------------------------------------------------------------

% Paquetes de matemáticas
\usepackage{mathtools, amsfonts, amssymb, mathrsfs}
\usepackage[makeroom]{cancel}     % Simplificar tachando
\usepackage{polynom}    % Divisiones y Ruffini
\usepackage{units} % Para poner fracciones diagonales con \nicefrac

\usepackage{pgfplots}   %Representar funciones
\pgfplotsset{compat=1.18}  % Versión 1.18

\usepackage{tikz-cd}    % Para usar diagramas de composiciones
\usetikzlibrary{calc}   % Para usar cálculo de coordenadas en tikz

%Definición de teoremas, etc.
\usepackage{amsthm}
%\swapnumbers   % Intercambia la posición del texto y de la numeración

\theoremstyle{plain}

\makeatletter
\@ifclassloaded{article}{
  \newtheorem{teo}{Teorema}[section]
}{
  \newtheorem{teo}{Teorema}[chapter]  % Se resetea en cada chapter
}
\makeatother

\newtheorem{coro}{Corolario}[teo]           % Se resetea en cada teorema
\newtheorem{prop}[teo]{Proposición}         % Usa el mismo contador que teorema
\newtheorem{lema}[teo]{Lema}                % Usa el mismo contador que teorema

\theoremstyle{remark}
\newtheorem*{observacion}{Observación}

\theoremstyle{definition}

\makeatletter
\@ifclassloaded{article}{
  \newtheorem{definicion}{Definición} [section]     % Se resetea en cada chapter
}{
  \newtheorem{definicion}{Definición} [chapter]     % Se resetea en cada chapter
}
\makeatother

\newtheorem*{notacion}{Notación}
\newtheorem*{ejemplo}{Ejemplo}
\newtheorem*{ejercicio*}{Ejercicio}             % No numerado
\newtheorem{ejercicio}{Ejercicio} [section]     % Se resetea en cada section


% Modificar el formato de la numeración del teorema "ejercicio"
\renewcommand{\theejercicio}{%
  \ifnum\value{section}=0 % Si no se ha iniciado ninguna sección
    \arabic{ejercicio}% Solo mostrar el número de ejercicio
  \else
    \thesection.\arabic{ejercicio}% Mostrar número de sección y número de ejercicio
  \fi
}


% \renewcommand\qedsymbol{$\blacksquare$}         % Cambiar símbolo QED
%------------------------------------------------------------------------

% Paquetes para encabezados
\usepackage{fancyhdr}
\pagestyle{fancy}
\fancyhf{}

\newcommand{\helv}{ % Modificación tamaño de letra
\fontfamily{}\fontsize{12}{12}\selectfont}
\setlength{\headheight}{15pt} % Amplía el tamaño del índice


%\usepackage{lastpage}   % Referenciar última pag   \pageref{LastPage}
\fancyfoot[C]{\thepage}

%------------------------------------------------------------------------

% Conseguir que no ponga "Capítulo 1". Sino solo "1."
\makeatletter
\@ifclassloaded{book}{
  \renewcommand{\chaptermark}[1]{\markboth{\thechapter.\ #1}{}} % En el encabezado
    
  \renewcommand{\@makechapterhead}[1]{%
  \vspace*{50\p@}%
  {\parindent \z@ \raggedright \normalfont
    \ifnum \c@secnumdepth >\m@ne
      \huge\bfseries \thechapter.\hspace{1em}\ignorespaces
    \fi
    \interlinepenalty\@M
    \Huge \bfseries #1\par\nobreak
    \vskip 40\p@
  }}
}
\makeatother

%------------------------------------------------------------------------
% Paquetes de cógido
\usepackage{minted}
\renewcommand\listingscaption{Código fuente}

\usepackage{fancyvrb}
% Personaliza el tamaño de los números de línea
\renewcommand{\theFancyVerbLine}{\small\arabic{FancyVerbLine}}

% Estilo para C++
\newminted{cpp}{
    frame=lines,
    framesep=2mm,
    baselinestretch=1.2,
    linenos,
    escapeinside=||
}

% para minted
\definecolor{LightGray}{rgb}{0.95,0.95,0.92}
\setminted{
    linenos=true,
    stepnumber=5,
    numberfirstline=true,
    autogobble,
    breaklines=true,
    breakautoindent=true,
    breaksymbolleft=,
    breaksymbolright=,
    breaksymbolindentleft=0pt,
    breaksymbolindentright=0pt,
    breaksymbolsepleft=0pt,
    breaksymbolsepright=0pt,
    fontsize=\footnotesize,
    bgcolor=LightGray,
    numbersep=10pt
}


\usepackage{listings} % Para incluir código desde un archivo

\renewcommand\lstlistingname{Código Fuente}
\renewcommand\lstlistlistingname{Índice de Códigos Fuente}

% Definir colores
\definecolor{vscodepurple}{rgb}{0.5,0,0.5}
\definecolor{vscodeblue}{rgb}{0,0,0.8}
\definecolor{vscodegreen}{rgb}{0,0.5,0}
\definecolor{vscodegray}{rgb}{0.5,0.5,0.5}
\definecolor{vscodebackground}{rgb}{0.97,0.97,0.97}
\definecolor{vscodelightgray}{rgb}{0.9,0.9,0.9}

% Configuración para el estilo de C similar a VSCode
\lstdefinestyle{vscode_C}{
  backgroundcolor=\color{vscodebackground},
  commentstyle=\color{vscodegreen},
  keywordstyle=\color{vscodeblue},
  numberstyle=\tiny\color{vscodegray},
  stringstyle=\color{vscodepurple},
  basicstyle=\scriptsize\ttfamily,
  breakatwhitespace=false,
  breaklines=true,
  captionpos=b,
  keepspaces=true,
  numbers=left,
  numbersep=5pt,
  showspaces=false,
  showstringspaces=false,
  showtabs=false,
  tabsize=2,
  frame=tb,
  framerule=0pt,
  aboveskip=10pt,
  belowskip=10pt,
  xleftmargin=10pt,
  xrightmargin=10pt,
  framexleftmargin=10pt,
  framexrightmargin=10pt,
  framesep=0pt,
  rulecolor=\color{vscodelightgray},
  backgroundcolor=\color{vscodebackground},
}

%------------------------------------------------------------------------

% Comandos definidos
\newcommand{\bb}[1]{\mathbb{#1}}
\newcommand{\cc}[1]{\mathcal{#1}}

% I prefer the slanted \leq
\let\oldleq\leq % save them in case they're every wanted
\let\oldgeq\geq
\renewcommand{\leq}{\leqslant}
\renewcommand{\geq}{\geqslant}

% Si y solo si
\newcommand{\sii}{\iff}

% Letras griegas
\newcommand{\eps}{\epsilon}
\newcommand{\veps}{\varepsilon}
\newcommand{\lm}{\lambda}

\newcommand{\ol}{\overline}
\newcommand{\ul}{\underline}
\newcommand{\wt}{\widetilde}
\newcommand{\wh}{\widehat}

\let\oldvec\vec
\renewcommand{\vec}{\overrightarrow}

% Derivadas parciales
\newcommand{\del}[2]{\frac{\partial #1}{\partial #2}}
\newcommand{\Del}[3]{\frac{\partial^{#1} #2}{\partial #3^{#1}}}
\newcommand{\deld}[2]{\dfrac{\partial #1}{\partial #2}}
\newcommand{\Deld}[3]{\dfrac{\partial^{#1} #2}{\partial #3^{#1}}}


\newcommand{\AstIg}{\stackrel{(\ast)}{=}}
\newcommand{\Hop}{\stackrel{L'H\hat{o}pital}{=}}

\newcommand{\red}[1]{{\color{red}#1}} % Para integrales, destacar los cambios.

% Método de integración
\newcommand{\MetInt}[2]{
    \left[\begin{array}{c}
        #1 \\ #2
    \end{array}\right]
}

% Declarar aplicaciones
% 1. Nombre aplicación
% 2. Dominio
% 3. Codominio
% 4. Variable
% 5. Imagen de la variable
\newcommand{\Func}[5]{
    \begin{equation*}
        \begin{array}{rrll}
            #1:& #2 & \longrightarrow & #3\\
               & #4 & \longmapsto & #5
        \end{array}
    \end{equation*}
}

%------------------------------------------------------------------------


% Para poder incluir árboles
\usepackage{forest}
\usepackage{booktabs}

% Para poder hacer graficas en polares (espiral de arquimedes)
\usepackage{pgfplots}
\usepgfplotslibrary{polar}

% Definimos \Diff para modo matemático
\DeclareMathOperator{\Diff}{Diff}

% // TODO:
% \newcommand{\showimages}{} % Descomentar esta línea para mostrar las imágenes

\begin{document}

    % 1. Foto de fondo
    % 2. Título
    % 3. Encabezado Izquierdo
    % 4. Color de fondo
    % 5. Coord x del titulo
    % 6. Coord y del titulo
    % 7. Fecha
    % 8. Autor

    
    % 1. Foto de fondo
% 2. Título
% 3. Encabezado Izquierdo
% 4. Color de fondo
% 5. Coord x del titulo
% 6. Coord y del titulo
% 7. Fecha

\newcommand{\portada}[7]{

    \portadaBase{#1}{#2}{#3}{#4}{#5}{#6}{#7}
    \portadaBook{#1}{#2}{#3}{#4}{#5}{#6}{#7}
}

\newcommand{\portadaExamen}[7]{

    \portadaBase{#1}{#2}{#3}{#4}{#5}{#6}{#7}
    \portadaArticle{#1}{#2}{#3}{#4}{#5}{#6}{#7}
}




\newcommand{\portadaBase}[7]{

    % Tiene la portada principal y la licencia Creative Commons
    
    % 1. Foto de fondo
    % 2. Título
    % 3. Encabezado Izquierdo
    % 4. Color de fondo
    % 5. Coord x del titulo
    % 6. Coord y del titulo
    % 7. Fecha
    
    
    \thispagestyle{empty}               % Sin encabezado ni pie de página
    \newgeometry{margin=0cm}        % Márgenes nulos para la primera página
    
    
    % Encabezado
    \fancyhead[L]{\helv #3}
    \fancyhead[R]{\helv \nouppercase{\leftmark}}
    
    
    \pagecolor{#4}        % Color de fondo para la portada
    
    \begin{figure}[p]
        \centering
        \transparent{0.3}           % Opacidad del 30% para la imagen
        
        \includegraphics[width=\paperwidth, keepaspectratio]{assets/#1}
    
        \begin{tikzpicture}[remember picture, overlay]
            \node[anchor=north west, text=white, opacity=1, font=\fontsize{60}{90}\selectfont\bfseries\sffamily, align=left] at (#5, #6) {#2};
            
            \node[anchor=south east, text=white, opacity=1, font=\fontsize{12}{18}\selectfont\sffamily, align=right] at (9.7, 3) {\textbf{\href{https://losdeldgiim.github.io/}{Los Del DGIIM}}};
            
            \node[anchor=south east, text=white, opacity=1, font=\fontsize{12}{15}\selectfont\sffamily, align=right] at (9.7, 1.8) {Doble Grado en Ingeniería Informática y Matemáticas\\Universidad de Granada};
        \end{tikzpicture}
    \end{figure}
    
    
    \restoregeometry        % Restaurar márgenes normales para las páginas subsiguientes
    \pagecolor{white}       % Restaurar el color de página
    
    
    \newpage
    \thispagestyle{empty}               % Sin encabezado ni pie de página
    \begin{tikzpicture}[remember picture, overlay]
        \node[anchor=south west, inner sep=3cm] at (current page.south west) {
            \begin{minipage}{0.5\paperwidth}
                \href{https://creativecommons.org/licenses/by-nc-nd/4.0/}{
                    \includegraphics[height=2cm]{assets/Licencia.png}
                }\vspace{1cm}\\
                Esta obra está bajo una
                \href{https://creativecommons.org/licenses/by-nc-nd/4.0/}{
                    Licencia Creative Commons Atribución-NoComercial-SinDerivadas 4.0 Internacional (CC BY-NC-ND 4.0).
                }\\
    
                Eres libre de compartir y redistribuir el contenido de esta obra en cualquier medio o formato, siempre y cuando des el crédito adecuado a los autores originales y no persigas fines comerciales. 
            \end{minipage}
        };
    \end{tikzpicture}
    
    
    
    % 1. Foto de fondo
    % 2. Título
    % 3. Encabezado Izquierdo
    % 4. Color de fondo
    % 5. Coord x del titulo
    % 6. Coord y del titulo
    % 7. Fecha


}


\newcommand{\portadaBook}[7]{

    % 1. Foto de fondo
    % 2. Título
    % 3. Encabezado Izquierdo
    % 4. Color de fondo
    % 5. Coord x del titulo
    % 6. Coord y del titulo
    % 7. Fecha

    % Personaliza el formato del título
    \pretitle{\begin{center}\bfseries\fontsize{42}{56}\selectfont}
    \posttitle{\par\end{center}\vspace{2em}}
    
    % Personaliza el formato del autor
    \preauthor{\begin{center}\Large}
    \postauthor{\par\end{center}\vfill}
    
    % Personaliza el formato de la fecha
    \predate{\begin{center}\huge}
    \postdate{\par\end{center}\vspace{2em}}
    
    \title{#2}
    \author{\href{https://losdeldgiim.github.io/}{Los Del DGIIM}}
    \date{Granada, #7}
    \maketitle
    
    \tableofcontents
}




\newcommand{\portadaArticle}[7]{

    % 1. Foto de fondo
    % 2. Título
    % 3. Encabezado Izquierdo
    % 4. Color de fondo
    % 5. Coord x del titulo
    % 6. Coord y del titulo
    % 7. Fecha

    % Personaliza el formato del título
    \pretitle{\begin{center}\bfseries\fontsize{42}{56}\selectfont}
    \posttitle{\par\end{center}\vspace{2em}}
    
    % Personaliza el formato del autor
    \preauthor{\begin{center}\Large}
    \postauthor{\par\end{center}\vspace{3em}}
    
    % Personaliza el formato de la fecha
    \predate{\begin{center}\huge}
    \postdate{\par\end{center}\vspace{5em}}
    
    \title{#2}
    \author{\href{https://losdeldgiim.github.io/}{Los Del DGIIM}}
    \date{Granada, #7}
    \thispagestyle{empty}               % Sin encabezado ni pie de página
    \maketitle
    \vfill
}
    \portada{ffccA4.jpg}{Ecuaciones\\Diferenciales I}{Ecuaciones Diferenciales I}{MidnightBlue}{-8}{28}{2024-2025}{José Juan Urrutia Milán\\Arturo Olivares Martos}

    \newpage
    La parte de teoría del presente documento (es decir, excluyendo las relaciones de problemas) está hecha en función de los apuntes que se han tomado en clase.
    No obstante, recomendamos seguir de igual forma los apuntes del profesor de la asignatura, Rafael Ortega, disponibles en su sitio web personal
    \href{https://www.ugr.es/~rortega/}{https://www.ugr.es/~rortega/}. Estos apuntes no son por tanto una completa sustitución de dichos apuntes, sino tan solo un complemento.

    El siguiente documento pdf no es sino el mero resultado proveniente de la amalgación proficiente de un cúmulo de notas, todas ellas tomadas tras el transcurso de consecutivas clases magistrales, primordialmente —empero, no exclusivamente— de teoría, en simbiosis junto con una síntesis de los apuntes originales provenientes de la asignatura en la que se basan los mismos.

Como motivación para la asignatura, introducimos a continuación un par de problemas que sabremos resolver tras la finalización de esta:

\begin{ejercicio*}[Parque de atracciones]
Disponemos de un conjunto de atracciones 
\[
A_1, A_2, \ldots, A_n
\]
Para cada atracción, conocemos la hora de inicio y la hora de fin. Podemos proponer varios retos de programación acerca de este parque de atracciones:
\begin{enumerate}
    \item Seleccionar el mayor número de atracciones que un individuo puede visitar.
    \item Seleccionar las atraciones que permitan que un visitante esté ocioso el menor tiempo posible.
    \item Conocidas las valoraciones de los usuarios, $val(A_i)$, seleccionar aquellas que garanticen la máxima valoración conjunta en la estancia.
\end{enumerate}
\end{ejercicio*}

\begin{ejercicio*}
Una empresa decide comprar un robot que deberá soldar varios puntos ($n$) en un plano. El software del robot está casi terminado pero falta diseñar el algoritmo que se encarga de decidir en qué orden el robot soldará los $n$ puntos. Se pide diseñar dicho algoritmo, minimizando el tiempo de ejecución del robot (este depende del tiempo de soldadura que es constante más el tiempo de cada desplazamiento entre puntos, que depende de la distancia entre ellos). Por tanto, deberemos ordenar el conjunto de puntos minimizando la distancia total de recorrido.
\end{ejercicio*}

\subsubsection{Nociones de conceptos}
A lo largo de la asignatura, será común ver los siguientes conceptos, los cuales aclararemos antes de empezar la misma:
\begin{itemize}
    \item Instancia: Ejemplo particular de un problema.
    \item Caso: Instancia de un problema con una cierta dificultad.
\end{itemize}

Generalmente, tendremos tres casos:
\begin{itemize}
    \item El mejor caso: Instancia con menor número de operaciones y/o comparaciones.
    \item El peor caso: Instancia con mayor número de operaciones y/o comparaciones.
    \item Caso promedio. Normalmente, será igual al peor caso.
\end{itemize}
Para notar la eficiencia del peor caso usaremos $O(\cdot)$, mientras que para el mejor caso, $\Omega(\cdot)$.

Diremos que un algoritmo es \ŧextit{estable} en ordenación si, dado un criterio de ordenación que hace que dos elementos sean iguales en cuanto a orden, el orden de stos vendrá dado por el primero se que introdujo en la entrada. 
\begin{ejemplo}
    Dado el criterio de que un número es menor que otro si es par, ante la instancia del problema: $1, 2, 3, 4$. La salida de un algoritmo de ordenación estable según este criterio será:
\[
2, 4, 1, 3
\]
Sin embargo, un ejemplo de salida que podría dar un algoritmo no estable sería:
\[
4, 2, 3, 1
\]

Los datos se encuentra ordenados pero no en el orden de la entrada.
\end{ejemplo}

\subsubsection{Algoritmos de ordenación}
A continuación, un breve reapso de algoritmos de ordenación:

\begin{itemize}
    \item Burbuja es el peor algoritmo de ordenación.
    \item Si tenemos pocos elementos, suele ser más rápido un algoritmos simple como selección o inserción. Entre estos, selección hace muchas comparaciones y pocos intercambios, mientras que inserción hace menos comparaciones y más intercambios. Por tanto, ante datos pesados con varios registros, selección será mejor que insercción.
    \item Cuando se tienen muchos elementos, es mejor emplear un algoritmo de ordenación del orden $n\log(n)$.
\end{itemize}



    \chapter{Introducción a los fundamentos de redes}
\subsection{Objetivos}

\subsection{Historia}
Un servicio de banda ancha es un servicio de velocidad grande, que se inición a partir del 2000. Comenzaron por 2Mbps.
El ADSL en España comenzó transmitiendo 256kbps (el máximo teórico del ADSL son 20Mbps, aunque lo normal son 10 o 12).
Las redes de cable eran HFC (redes de cable y fibras híbridas), pero ahora tenemos FTTH (Fiber to the home), fibra directa a casa. La fibra es el mejor material para transmitir información del mundo, además de que no cuenta con interferencias, consiguiendo varios Gbps.

Estos servicios eran usados por:
\begin{itemize}
    \item Televisiones (contaban con una resolución de $640\times 240$, necesitando un servicio de 2Mbps sin comprimir).
\end{itemize}

Cerca del 70\% del tráfico de internet es debido al multimedia, a través de las CDNs, redes de transmisión de contenidos.
Por ejemplo, un vídeo de Youtube cuenta con varias copias del mismo alrededor del mundo.

\section{Sistemas de comunicación y redes}
El sistema de comunicación típico es:

En un sistema de comunicaciones contamos con una fuente y con un transmisor (ambos en el mismo equipo), de forma que la fuente genera datos.
Después del transmisor, contamos con un canal de comunicación, el cual proboca errores:
\begin{itemize}
    \item Ruidos.
    \item Interferencias.
    \item Diafonías: sucede mucho en ADSL, al tener muchos cables en paralelo juntos puede suceder que la información de un cable se meta en otro.
\end{itemize}

En el final del destino, conamos con un equipo que cuenta con un receptor y con el destino (que espera los datos a recibir).


Cuando hablamos de redes, tenemos que tener varios equipos interconectados, que funcionen de forma autónoma (sin interferencia de nadie) y que se realice de forma eficaz.

\subsection{Primera red de comunicaciones}

La primera red de comunicaciones era la red de telefonía móvil.

Contábamos con nuestra línea de teléfono, que conectaba con una central de conmutación local, luego regional y luego nacional, la cual debía conectar con la central local a la que queríamos llamar.
Se usaba la conmutación de circuitos: 
\begin{itemize}
    \item Inicialmente se creaba un camino físico juntando cables. A dicho camino se le llamaba circuito.

        Era ineficiente porque dicho cable cuenta con una eficiencia del 50\%, debido a que aproximadamente se habla la mitad del tiempo de la llamada.

        Era un problema de seguridad el mal funcionamiento de una central, ya que dejaba sin servicio a miles de teléfonos.
\end{itemize}

Si ahora cambiamos los teléfonos por ordenadores y las centrales de conmutación por routerse, contamos con muchísimos caminos para conectar dos ordenadores, haciendo mucho más segura la red (a expensas de la seguridad en la red).

Ahora ya no tenemos un circuito físico, sino que son los routers quienes deciden a dónde enviar los paquetes y en qué momento hacerlo. Con el inconveniente de generar retardo pero con la ventaja de usar mejor el canal (si hay silencios, puede usarlo otro).

El departamento de defensa americana y posteriormente la NSF crearon las primeras redes asemejables a internet.

De una red esperamos:
\begin{itemize}
    \item Autonomía.
    \item Interconexión.
    \item Eficiencia.
\end{itemize}

Una red clásica va a tener equipos terminales (hosts) y equipos de interconexión, que permiten conectar toda la red.

\subsection{Líneas de transmisión}
Podemos contar con enlaces inalámbricos y cableados.

Comenzó con los enlaces cableados con cables de pares (pensado para transmitir 4kHz, la media en la voz humana), luego con cables coaxiales y fibra óptica.
Este último es el mejor medio guiado existente.


    \chapter{Sincronización en memoria compartida. Monitores}

Suponiendo que existe una memoria común para los distintos procesos que ejecutan un programa concurrente, este Capítulo trata sobre la sincronización de los mismos usando para ello instrucciones que usan dicha memoria compartida. Nos centraremos en el uso de los monitores, construcciones de alto nivel que nos ofrecen mayor libertad que los semáforos.\\

El concepto de semáforo se desarrolló previamente en el Seminario 1 de prácticas\footnote{Por lo que el lector debería estar familiarizado con ellos.}. Los semáforos presentan dos grandes limitaciones:
\begin{enumerate}
    \item Están basados en variables compartidas del programa, por lo que no fomentan la modularidad de los programas, impidiendo su reutilización.
    \item Las operaciones de los semáforos (\verb|sem_wait| y \verb|sem_signal|) se encuentran dispersas a lo largo del código del programa concurrente. Además, estas instrucciones no solo afectan al bloque de código en el que se encuentran, sino a cualquier otro bloque que use el mismo semáforo.
\end{enumerate}
En definitiva, los semáforos no son un buen mecanismo de programación concurrente, y además la verificación de programas que usan semáforos es muy complicada. 

Era necesario encontrar un nuevo mecanismo de programación concurrente que permitiera la encapsulación de la información y de la sincronización entre procesos, así como programar las operaciones de sincronización (como \verb|wait| o \verb|signal|) dentro de bloques o procedimientos que se ejecuten con instrucciones atómicas, para que las instrucciones de sincronización no se encuentren desperdigadas por el programa.
Fue Charles Antony Richard Hoare quien inventó los monitores, concepto en el que ahondaremos a lo largo de este Capítulo.

\section{Definición de un monitor}
La idea básica de monitor es un módulo que contiene un conjunto de variables a las que llamaremos \textit{variables permanentes}\footnote{A pesar de su nombre, no serán constantes, sino que podremos modificar su valor.}, de forma que dichas variables solo podrán ser alteradas dentro de los procedimientos del módulo monitor. Garantizaremos que la ejecución de cada uno de esos procedimientos se ejecute la mayor parte del tiempo como una única instrucción atómica, salvo que se produzca algo por lo que interrumpir la ejecución del procedimiento.\\

Podemos pensar en un monitor como en un tipo de dato abstracto que define tipos y variables permanentes propias del monitor, así como un conjunto de procedimientos dentro de dicho módulo. No debemos pensar en los monitores como en una clase, ya que no pueden hacer lo mismo que ellas (no se pueden instanciar y tampoco existe polimorfismo o ligadura dinámica).

\subsubsection{Ventajas}
A continuación, los programas concurrentes estarán formados tanto por procesos que se ejecutarán de forma concurrente, como por monitores, los cuales velarán por la sincronización y acceso a variables compartidas de dichos procesos, de forma que no se produzcan condiciones de carrera o comportamientos indeseados. Podremos modelar tantas relaciones de interacción entre los procesos de un programa concurrente como queramos. De esta forma, el uso de los monitores o de procedimientos asociados a monitores no restringen las posibilidades del modelado de un sistema concurrente. 

Los procesos de un programa concurrente no tendrán que llamar a operaciones de sincronización, sino que llamarán a procedimientos del monitor, los cuales realizarán la funcionalidad deseada sobre las variables compartidas garantizando la sincronización entre los procesos.\\

Además, los monitores nos permiten una alta reusabilidad de código, ya que podremos reutilizar un monitor ya creado para resolver problemas similares. Sin embargo, la reutilización de código no es similar a la usada en programación orientada a objetos mediante instancias de una misma clase, sino que se hará por copias parametrizables: tendremos una definición de un monitor basada en parámetros, y cuando necesitemos usar un monitor, crearemos una copia de dicha definición parametrizándola (pasándole los parámetros que necesitemos para resolver nuestro problema). De esta forma, no es reutilización por instanciación, sino por \textit{parametrización}.\\

Los procesos que usemos en los progrmas concurrentes no verán el acceso a las variables compartidas, sino que será realizado por los procedimientos del monitor, garantizando que se hacen como deben hacerse, evitando condiciones de carrera. De esta forma, los monitores garantizan la ocultación de las variables compartidas, haciéndolas transparentes a los procesos del sistema concurrente.\\

Finalmente, existen unos axiomas que nos permiten verificar los programas concurrentes que usen monitores de forma sencilla. Dichas demostraciones estarán basadas en el uso de los invariables globales. Ahondaremos en la verificación de programas concurrentes que utilicen monitores más adelante.

\subsection{Concepto de monitor}
A modo de resumen para comenzar a definir lo que es un monitor, podemos decir que:
\begin{itemize}
    \item Es un módulo con un conjunto de variables permanentes que solo pueden ser modificadas por los procedimientos del monitor.
    \item Cada uno de los procedimientos\footnote{Podemos pensar en ellos como en los ``métodos'' de una clase, haciendo hincapié en que los monitores \textbf{no son} clases.} de un monitor se ejecutan en exclusión mutua (garantizando el acceso a las variables compartidas sin condiciones de carrera). Sin embargo, estos no tienen por qué ejecutarse completamente, sino que pueden interrumpirse y en algún momento futuro seguir ejecutándose en exclusión mutua.
    \item La ejecución de los procedimientos de un monitor modifican el estado interno del mismo (esto es, el conjunto de las variables permanentes asociadas al monitor).
    \item El estado inicial del monitor (de sus variables permanentes) se establece mediante la ejecución de un procedimiento especial, al que llamaremos \textit{código de inicialización}. Este se ejecuta tras la declaración de una variable de tipo monitor y da valores iniciales a las variables permanentes.
\end{itemize}
De esta forma, un monitor puede visualizarse como en la Tabla~\ref{esq:monitor_1}, como un conjunto que englobla:
\begin{itemize}
    \item Un conjunto de variables, llamadas \textit{variables permanentes}, que no son accesibles desde fuera del monitor.
    \item Un conjunto de procedimientos que el monitor proporciona como servicio a los procesos de un programa concurrente (para por ejemplo, acceder a las variables permanentes que serán las variables que compartan dichos procesos), llamados \textit{procedimientos exportados} o \textit{exportables}.
    \item Un procedimiento especial llamado \textit{código de inicialización}, que permite inicializar las variable permanentes.
\end{itemize}

\begin{table}[H]
\centering
\begin{tabular}{|l|}
\hline
Variables \\
permanentes \\
\hline
Procedimientos \\
exportados \\
\hline
Código de \\ 
inicialización \\
\hline
\end{tabular}
\caption{Esquema de un monitor.}
\label{esq:monitor_1}
\end{table}

\begin{ejemplo}
    Aunque todavía no entendemos muy bien qué es un monitor, daremos a continuación un ejemplo de uso del mismo para ilustrar la definición que queremos dar de monitor, pese a que algunas cosas del ejemplo no podamos entenderlas todavía y deberemos dejarlas para más adelante\footnote{Como el tipo de dato \texttt{cond}.}.\\

    En este ejemplo, queremos solventar un problema mediante el paradigma productor/consumidor. Tendremos dos procesos, un productor y un consumidor, de forma que el productor escribirá en un buffer (o vector) que usaremos como cola cíclica (esto es, que si nos pasamos de la posición final, volvemos al inicio y con planificación FIFO), mientras que el consumidor irá leyendo los datos de dicho buffer. Siendo \verb|Buf| una variable de tipo monitor que luego definiremos en este ejemplo, el código del productor y del consumidor será el siguiente (pensando en que tenemos que usar procedimientos del monitor para el acceso a las variables compartidas, en este caso el buffer):
    \begin{minted}[escapeinside=\#\#]{pascal}
        Proceso Prod1::
          var d : tipo_dato;

          while true do begin
            d = producir();
            Buf.insertar(d); {mete d en el buffer}
          end do
    \end{minted}
    \begin{minted}[escapeinside=\#\#]{pascal}
        Proceso Cons1::
          var x : tipo_dato;

          while true do begin
            Buf.retirar(x); {retira del buffer en x}
            consumir(x);
          end do
    \end{minted}
    El código del monitor será el siguiente en pseudo-pascal (hemos omitido el código de inicialización):
    \begin{minted}[escapeinside=\#\#]{pascal}
        Monitor Buf
          var
            -elementos_ocupados : int;
            -frente, atras: 0..N-1;
            -no_vacio, no_lleno : cond;

          +insertar(d : tipo_dato);
          +retirar(var x : tipo_dato);
    \end{minted}
    Donde vemos 5 variables permanentes: \verb|elementos_ocupados|, que mide la cantidad de posiciones ocupadas del buffer, \verb|frente|, que marca la casilla en la que el productor insertará el próximo dato (por tanto, ha de estar siempre vacía), \verb|atras|, que marca la casilla de la que leerá el consumidor, \verb|no_vacio| y \verb|no_lleno|, variables de tipo \verb|cond|, las cuales aprenderemos lo que hacen más adelante.
    
    Contamos además con dos procedimientos: \verb|insertar|, que inserta un dato en el buffer en caso de que haya hueco (si no hay hueco, se bloquea hasta que el consumidor lea un dato y deje un hueco libre):
    \begin{minted}[escapeinside=\#\#]{pascal}
        procedure insertar(d : tipo_dato) begin
          if((frente + 1) mod N = frente) then no_lleno.wait();
          introducir(buf, frente, d);  {inserta d en la posicion frente en el buffer}
          elementos_ocupados += 1;
          frente = (frente + 1) mod N;
          no_vacio.signal();
        end
    \end{minted}
    Y con el procedimiento \verb|retirar|, que retira un dato del buffer y lo devuelve como resultado del procedimiento, siempre que esto sea posible (es decir, si no hay ningún dato que leer en el buffer, se bloquea esperando a que el productor ponga algún dato):
    \begin{minted}[escapeinside=\#\#]{pascal}
        procedure retirar(var x : tipo_dato) begin
          if(frente = atras) then no_vacio.wait();
          eliminar(buf, atras, x);  {inserta buf[atras] en x y lo borra del buffer}
          elementos_ocupados -= 1;
          atras = atras mod N + 1;
          no_lleno.signal();
        end
    \end{minted}
    Como hemos ya comentado mientras mostrábamos los pseudocódigos del ejemplo, hay que establecer condiciones que identifiquen las dos condiciones inseguras del ejemplo: que el buffer esté lleno o que el buffer esté vacío:
    \begin{itemize}
        \item Si \verb|frente = atras|, entonces el último dato que se ha de consumir está en una casilla vacía en la que el productor escribirá. Se trata de la situación en la que el buffer está vacío. Debemos por tanto, evitar que el consumidor lea un dato del buffer.
        \item Si \verb|(frente + 1) mod N = atras|, entonces el siguiente dato a introducir en el buffer está justo delante del dato a consumir. Se trata de la situación en la que el buffer está lleno. Debemos por tanto, evitar que el productor introduzca un dato en el buffer\footnote{Definimos anteriormente que \texttt{frente} siempre apunta a una casilla vacía, por lo que como máximo el buffer tendrá ocupados $N-1$ elementos.}.
    \end{itemize}
    Los procesos del programa llaman a los procedimientos del monitor, y no tienen acceso directo al buffer, por lo que no pueden saber cuándo este está lleno o vacío. De esta forma, lo que sucederá es que los procedimientos internos del monitor realizarán una sincronización interna mediante el uso de llamadas bloqueantes:
    \begin{itemize}
        \item Si el buffer está lleno y el productor se dispone a escribir un dato, quedará el proceso bloqueado hasta que un consumidor lea un dato. Este señalará (\verb|signal|) al productor, desbloqueándolo.
        \item Si el buffer está vacío y el consumidor se disopne a leer un dato, quedará bloqueado el proceso que ejecute el procedimiento del monitor. Cuando el productor escriba un dato, enviará una señal al consumidor, desbloqueándolo.
    \end{itemize}
    Esta funcionalidad se consigue mediante las variables de tipo \verb|cond|. Se verán a continuación, pero para entenderlas por ahora digamos que necesitamos tener una variable de tipo \verb|cond| por cada razón por la que queremos bloquear un proceso\footnote{En el caso de productor/consumidor, queremos bloquear un proceso si sucede alguno de los dos puntos superiores, condiciones inseguras, luego nos harán falta dos variables de tipo \texttt{cond}. En otros problemas, el número de variables de tipo \texttt{cond} podría ser otro.}.\\

    El código de los procedimientos es ejecutado por los propios procesos que ejecutan cada proceso (productor o consumidor, en este caso) del programa concurrente. Por tanto, si el productor ejecuta un procedimiento del monitor con un \verb|wait|, dicho proceso se bloqueará y no podrá ejecutar código hasta desbloquearse.\\

    Para que el código que hemos visto funcione adecuadamente, nos falta introducir un último concepto en los monitores, y es que mientras se ejecuta un procedimiento de un monitor, no se puede ejecutar ningún otro, sino que han de ejecutarse en \textbf{exclusión mutua}.
\end{ejemplo}

\subsection{Características de programación con monitores}
Una vez ilustrado el uso de la herramienta que estamos construyendo en este Capítulo mediante el ejemplo anterior, vamos ahora a introducir la noción de que sólo puede ejecutarse a la vez un único procedimiento de un monitor.\\

Como ya hemos visto, los procedimientos de los monitores no tienen por qué ejecutarse de principio a fin, sino que un proceso puede comenzar a ejecutar un procedimiento, bloquearse (dejando por tanto libre al monitor) y que otro proceso comience a ejecutar un procedimiento de dicho monitor, sucediéndose un entrelazamiento de las trazas de ejecución de los procedimientos.\\

Cuando un proceso se encuentra ejecutando un procedimiento del monitor, decimos que el monitor está \emph{ocupado}. En caso contrario, diremos que este está \emph{libre}. Notemos que si un proceso se bloquea mientras ejecuta un procedimiento del monitor, el monitor tiene que quedar libre, ya que si no no habría forma de volver a despertar a dicho proceso (tenemos que ejecutar un \verb|signal| sobre la misma variable \verb|cond| que bloqueó al proceso\footnote{Se explicará más adelante.}). La situación de bloquear a un proceso y dejar que entre otro al monitor es delicada y debe hacerse con cuidado, para garantizar que solo haya un único proceso ejecutando un procedimiento del monitor al mismo tiempo.\\

Los monitores son objetos \emph{pasivos}. Esto es, no tienen una hebra dentro que ejecute su código, sino que simplemente proporciona código (sus procedimientos) a otros procesos para que sean ellos quien ejecuten el código del monitor.\\

Para implementar una librería con monitores en un lenguaje de programación base, este debe tener la propiedad de ser \emph{reentrante}.
\begin{definicion}
    Un lenguaje de programación tiene la propiedad de ser reentrante si, siempre que tengamos un proceso ejecutando una función y este se bloquea, sea capaz de conservar la siguiente instrucción a ejecutar y el valor de sus variables locales tras desbloquearse. Es decir, el proceso no debe enterarse localmente de que nada haya cambiado mientras estaba bloqueado.
\end{definicion}
Notemos que debemos tener esta propiedad en el lenguaje de programación con el que trabajemos para poder hacer uso de funciones bloqueantes (como \verb|wait|) dentro de los procedimientos de un monitor, algo básico en el funcionamiento de este. Afortunadamente, actualmente todos los lenguajes de programación que encontramos en el mercado son reentrantes.\\

\subsubsection{Instanciación de monitores}
El siguiente ejemplo nos ilustra cómo funciona la instanciación de un monitor:

\begin{ejemplo}
    Aunque los monitores están pensado para programas concurrentes (ya que no tiene sentido su uso en programas secuenciales), usaremos en este ejemplo un monitor en un programa secuencial, ya que sólo nos interesa la forma en la que los monitores se inicializan\footnote{Además, no hemos terminado de desarrollar cómo es que solo puede ejecutarse a la vez un único procedimiento del monitor, por lo que no entendemos hasta ahora cómo es que sirven para sincronizar programas concurrentes.}.\\

    Tenemos un programa en el que necesitamos dos variables, las cuales queremos consultar e incrementar mediante un incremento previamente fijado que no cambiará. Para ello, creamos un monitor de acceso a una variable, con parámetros de entrada, para luego poder crear dos copias parametrizadas del mismo. El código del monitor será algo parecido a:
    \begin{minted}[escapeinside=\#\#]{pascal}
        class monitor VariableProtegida(inicio, incremento : integer);
          var x, inc : integer;

          procedure incremento();
          begin
            x = x + inc;
          end

          procedure valor(var v : integer);
          begin
            v = x;
          end
          
          begin
            x = inicio; inc = incremento;
          end
    \end{minted}
    De esta forma, podemos usar dos copias del monitor de la forma:
    \begin{minted}[escapeinside=\#\#]{pascal}
        var mv1 : VariableProtegida(0,1);   {empieza en 0 e incrementa en 1}
            mv2 : VariableProtegida(10,4);  {empieza en 10 e incrementa en 4}
            a, b : integer;
        begin
          mv1.incremento();  {+=1}
          mv1.valor(a);      {a=1}
          mv2.incremento();  {+=4}
          mv2.valor(b);      {b=14}
        end
    \end{minted}
\end{ejemplo}

\subsubsection{Exclusión mutua en los procedimientos de un monitor}
Si tenemso varios procesos del programa concurrente que quieren hacer uso de procedimientos del monitor a la vez, sólo podremos dejar pasar a un proceso al monitor (suponiendo que este se encuentre libre). Para los otros procesos, almacenamos su llamada al procedimiento. 

Para ello, todos los monitores tienen implementada una cola con planificación FIFO, llamada \emph{cola de entrada al monitor}. Si tenemos dos procesos que quieren acceder a un procedimiento de un monitor libre, sólo podrá hacerlo un proceso. La llamada al procedimiento del monitor del otro proceso quedará almacenada en la cola de entrada al monitor, y este pasará a ejecutar el procedimiento deseado una vez el proceso anterior haya dejado libre el monitor. 

En esta asignatura, supondremos que la cola de entrada al monitor es suficientemente larga como para albergar a todos los procesos que necesiten esperar a que el monitor quede libre.\\

Podemos representar la vida de un proceso de un programa concurrente que hace uso de monitores para sincronizar a sus procesos con el siguiente diagrama:
\begin{figure}[H]
\begin{tikzpicture}[node/.style={rectangle, draw, minimum size=1cm, align=center},
                    edge/.style={-stealth}]
    \node[node] (A) {Procedimiento\\terminado};
    \node[node, left=of A] (B) {\begin{tabular}{c}Ejecutando código\\del proceso\\(fuera del monitor)\end{tabular}};
    \node[node, left=of B] (C) {\begin{tabular}{c}Llamando a\\procedimiento\\del monitor\end{tabular}};
    \node[node, below=of C] (D) {\begin{tabular}{c}Esperando en\\cola de entrada\\al monitor\end{tabular}};
    \node[node, right=of D] (E) {\begin{tabular}{c}Monitor\\libre\end{tabular}};
    \node[node, right=of E, xshift=.6cm] (F) {\begin{tabular}{c}Ejecutando procedimiento\\(dentro del monitor)\end{tabular}};
    
    \draw[edge] (A) -- (B);
    \draw[edge] (B) -- (C);
    \draw[edge] (C) -- (D);
    \draw[edge] (D) -- (E);
    \draw[edge] (E) -- (F);
    \draw[edge] (F) -- (A);
\end{tikzpicture}
\caption{Vida de un proceso en un programa concurrente con monitores.}
\end{figure}

De esta forma, podemos ahora reescribir la descripción gráfica de monitor que hicimos en la tabla~\ref{esq:monitor_1}, incluyendo ahora la cola de entrada al monitor, tal y como vemos en la tabla~\ref{esq:monitor_2}
\begin{table}[H]
\centering
\begin{tabular}{|l|}
\hline
Cola del\\
monitor \\
\hline
Variables \\
permanentes \\
\hline
Procedimientos \\
exportados \\
\hline
Código de \\ 
inicialización \\
\hline
\end{tabular}
\caption{Esquema de un monitor incluyendo la cola de entrada.}
\label{esq:monitor_2}
\end{table}

\subsection{Operaciones de sincronización}
Las operaciones de sincronización entre los procesos de un programa concurrente se programan, como ya hemos visto, dentro de los procedimientos del monitor. Son instrucciones que permiten detener la ejecución de un procedimiento de un monitor y bloquear en una cola al proceso que ha hecho la llamada del procedimiento del monitor. Tenemos para realizar esta acción dos operaciones principales: \verb|wait| y \verb|signal|.\\

Sin embargo, las operaciones \verb|wait| y \verb|signal| que manejamos en monitores no se parecen a las que usábamos en los semáforos:
\begin{itemize}
    \item En los semáforos, la ejecución de \verb|wait| ofrecía la posibilidad de bloquear al proceso, ya que no lo hacía si el entero de dentro del semáforo era mayor estricto que $0$. Por contra, en monitores la llamada \verb|wait| siempre será bloqueante.
    \item Las operaciones \verb|wait| y \verb|signal| eran relativas a un semáforo: hacía falta usar un semáforo por cada razón que tuviéramos dentro de un programa concurrente para bloquear a uno o varios procesos (en el caso del productor/consumidor, usar dos semáforos). Sin embargo, con un solo monitor podremos bloquear procesos por tantas razones como queramos, usando un nuevo tipo de dato.
\end{itemize}

\subsubsection{Tipo de dato \texttt{cond}}
En los monitores, para poder usar las operaciones de \verb|wait| y \verb|signal|, será necesario utilizar una variable de tipo de dato condición, o \verb|cond|. 

Las variables tipo \verb|cond| sólo se encuentran junto con las variables permanentes de un monitor. Estas no se inicializan a ningún valor.\\

En nuestro monitor, tendremos varias razones por las que querremos bloquear a los procesos concurrentes de nuestro programa por alguna determinada razón hasta que se cumpla una condición determinada. Por ejemplo, en el problema del productor/consumidor:
\begin{itemize}
    \item Queremos bloquear a cualquier productor que intente escribir si la estructura de datos intermedia que usamos está llena. Desbloquearemos a un proceso productor cuando se vacíe un hueco en dicha estructura.
    \item Además, queremos bloquear a cualquier consumidor que intente leer de la estructura de datos intermedia cuando esta esté vacía. Desbloquearemos a un consumidor cuando algún productor haya escrito algún dato.
\end{itemize}
Por cada razón o condición distinta por la que queramos bloquear a los procesos de un programa concurrente en relación a una misma variable compartida (para evitar estados inseguros), crearemos una variable de tipo \verb|cond|. Es decir, una variable por cada una de las razones por las que queramos que esperen los procesos. En el ejemplo del productor/consumidor, son necesarias únicamente dos variables de tipo \verb|cond|.\\

Las variables de tipo \verb|cond| admiten 4 métodos (aunque sólo recomendamos usar los dos primeros):
\begin{description}
    \item [wait] Bloquea al proceso que ejecuta este método. Dicho proceso pasa a una cola asociada a la variable condición correspondiente con planificación FIFO.
    \item [signal] En caso de haber algún proceso bloqueado en la cola asociada a la variable condición correspondiente, lo desbloquea. Si esta cola está vacía, es equivalente a una operación nula\footnote{Esto es, equivalente a la instrucción \texttt{;}.}.
    \item [queue] Devuelve un booleano que indica (\verb|true|) si la cola asociada a la variable condición contiene al menos un proceso bloqueado.
    \item [signal\_all] Desbloquea de una sola vez a todos los procesos bloqueados en la cola asociada a la variable condición. El orden de dicha cola no se mantiene para realizar la petición de acceso al monitor, por lo que se produce competencia entre los procesos para entrar al monitor, incumpliendo la propiedad de equidad entre procesos. Depende de la semántica de las señales del lenguaje\footnote{Se explicará más adelante qué es esto.}. Se recomienda \textbf{no usarla}.
\end{description}
De esta forma, la representación gráfica final de un monitor es la que se muestra en la tabla~\ref{esq:monitor_3}:
\begin{table}[H]
\centering
\begin{tabular}{|l|}
\hline
Cola del\\
monitor \\
\hline
Variables \\
permanentes \\
\hline
Variables \\
condición y\\
colas de procesos\\
bloqueados\\
\hline
Procedimientos \\
exportados \\
\hline
Código de \\ 
inicialización \\
\hline
\end{tabular}
\caption{Esquema de un monitor incluyendo las variables condición.}
\label{esq:monitor_3}
\end{table}

\subsubsection{Semántica desplazante}
Como hemos comentado ya, los monitores solo permiten que un único proceso se encuentre ejecutando un procedimiento del mismo. En este caso, decíamos que el monitor está ocupado. En caso de que un proceso que estaba ejecutando el procedimiento ejecute un \verb|wait| (o salga del procedimiento), hay que dejar el monitor libre para dejar pasar a otro. Se trata de un momento muy delicado, ya que se pueden producir condiciones de carrera entre los procesos que quieran conseguir el monitor. Esta situación la hemos solucionado ya con la cola de entrada al monitor, ya que con la planificación FIFO, solo podrá entrar un único proceso al monitor.\\

Si ahora el proceso nuevo que ejecuta el monitor ejecuta un \verb|signal|, desbloqueará al anterior proceso de la cola de la variable condición correspondiente. Si en dicho momento este proceso sale del procedimiento, dejará el monitor abierto, por lo que podría suceder que un proceso de la cola de entrada al monitor entre antes que el proceso recién desbloqueado, sucediéndose un \emph{robo de señal}, y llegaríamos a una situación de falta de equidad.\\

Para solucionar este segundo problema, algunos lenguajes implementan una \emph{semántica desplazante} en las señales: el proceso que ejecuta el \verb|signal| le pasa el monitor al proceso que recibió la señal (el primero en la cola de bloqueados de la variable condición correspondiente), sin liberar en ningún momento el monitor, de forma que el proceso señalado tiene prioridad. Se dice que la señal usada con la operación \verb|signal| tiene \emph{semántica desplazante}.\\

Cabe destacar que \textbf{no todos los lenguajes con monitores tienen señales con semántica desplazante}, por lo que en dichos lenguajes pueden sucederse robos de señales. De hecho, para demostrar luego la corrección de nuestros programas concurrentes que usan monitores, supondremos que estamos usando un \verb|signal| que envía una señal con semántica desplazante.\\

\noindent
Como comentario final a la descripción de un monitor y para motivar la siguiente sección:
\begin{itemize}
    \item Se presupone que el programador de monitores es un programador experto, de forma que el compilador en ningún momento se dedicará a comprobar si hemos programado de forma correcta un monitor o un procedimiento de él, más allá de la sintaxis del código.
    \item No deben programarse operaciones \verb|wait| indebidas ni omitirse operaciones \verb|signal| innecesarias. Para comprobar esto, usaremos nuestro sistema de verificación formal.
\end{itemize}

\section{Verificación de programas con monitores}
En la verificación de los programas concurrentes que hemos manejado hasta ahora, hemos primero demostrado la corrección secuencial de cada proceso que forma parte de un programa secuencial, para luego demostrar la no interferencia entre los mismos. 

Sin embargo, ahora que introducimos los monitores, esto no podrá ser nunca más así, ya que un programador nunca puede conocer a priori la traza que genera un proceso que forma parte de un programa concurrente con monitores, ya que al ejecutar procedimientos de monitores, estos pueden quedar bloqueados y se ejecutarían en medio instrucciones de otros procesos que podrían alterar las variables compartidas del programa, falseando alguna precondición o poscondición del proceso bloqueado, por lo que tras desbloquearse, no podemos esperar nada de dicho proceso.\\

Es por tanto que ahora la estrategia a seguir en las demostraciones es mediante un Invariante de Monitor.

\subsection{Invariante de monitor}
\begin{definicion}[Invariante de Monitor]
    Un Invariante de Monitor (IM) es una relación entre las variables permanentes de un monitor que debe ser cierta en cualquier estado del programa concurrente, excepto cuando un proceso esté ejecutando código de un procedimiento del monitor.
\end{definicion}
De esta forma, un IM puede no ser cierto durante la ejecución de un procedimiento por parte de un proceso, pero este ha de cumplirse antes y después de la ejecución de dicho procedimiento.\\

Si conseguimos probar la existencia de un IM en un programa concurrente, entonces bastará con probar cada una de las secciones de código secuenciales entre llamadas a procedimientos del monitor. Para probar finalmente la corrección de los procesos, usaremos que los IM se mantienen antes y después de las llamadas a procedimientos, para conseguir probar finalmente la corrección de cada uno de los procesos. Si nuestro IM estaba relacionado con la solución al problema, como el acceso a variables compartidas estará controlado por los monitores, al final del programa todos los IMs demostrados se seguirán compliendo, por lo que tendremos probada la corrección de nuestro programa concurrente.

Es decir, primero demostraremos que por cada monitor que usamos se verifica un IM, y luego pasaremos a probar la corrección de cada proceso que interviene en el programa concurrente, usando para ello dichos IMs. Finamente, tendremos probado el programa concurrente.

\subsubsection{Esquema de demostración}
Suponiendo que hemos encontrado una relación matemática entre las variables permanentes de un monitor y queremos probar que se trata de un IM\footnote{A continuación, llamaremos a dicha condición \texttt{IM}, pese a no haber demostrado que se trate de verdad de un IM.}, lo primero será probar que $IM$ se cumple en el estado inicial del monitor, esto es, justo después de la inicialización de las variables permanentes, por lo que tendremos que probar que se verifica el \textbf{triple de inicialización de variables}:
\begin{equation*}
    \{V\}\ \text{código de inicialización}\ \{IM\}
\end{equation*}
Posteriormente, deberemos probar que $IM$ se mantiene antes y después de la llamada a cada procedimiento. Es decir, notando por $IN$ a las precondiciones que tenemos antes de la ejecución de un procedimiento y por $OUT$ a las poscondiciones que deseamos tener tras dicho procedimiento, debemos demostrar los \textbf{triples de procedimientos del monitor}, es decir, demostrar un triple
\begin{equation*}
    \{IM \land IN\}\ \text{procedimiento}\ \{IM \land OUT\}
\end{equation*}
por cada procedimiento que tenga nuestro monitor.

Terminaremos de ver esto más adelante, pero es necesario darnos cuenta de un detalle, y es que si un procedimiento modifica el valor de alguna variable compartida que se usa en otro proceso, debemos demostrar la no interferencia entre dichas instrucciones. Ilustramos esto con el siguiente ejemplo.
\begin{ejemplo}
    Si tenemos un monitor llamado \verb|Buf| con un procedimientos \verb|retirar(x)|, de forma que modifica el valor del parámetro que le pasamos, ante el siguiente código (si \verb|x| es una variable compartida):
    \begin{minted}[escapeinside=\#\#]{pascal}
        cobegin y = x; #||# Buf.retirar(x); coend
    \end{minted}
    Tenemos que probar que al cambiar el valor de \verb|x| con el procedimiento \verb|retirar|, no hay interferencia con la instrucción de la izquierda. Es decir, tenemos que probar:
    \begin{align*}
        &NI(pre(y=x), Buf.retirar(x)) \\
        &NI(pos(y=x), Buf.retirar(x))
    \end{align*}
    Sin embargo, en caso de ejecutar el siguiente código:
    \begin{minted}[escapeinside=\#\#]{pascal}
        z = x;
        cobegin y=z; #||# Buf.retirar(x); coend
    \end{minted}
    No tendríamos que hacerlo, ya que el uso de variables disjuntas nos garantiza la no interferencia entre dichas instrucciones.
\end{ejemplo}

\subsection{Axiomas para operaciones de sincronización con semántica desplazante}
Sabemos ya demostrar toda la corrección de un programa secuencial que usa monitores, salvo por un detalle, y es que no sabemos nada sobre cómo demostrar los triples:
\begin{align*}
    \{P\}\ &c.wait();\ \{Q\} \\
    \{P\}\ &c.signal();\ \{Q\}
\end{align*}
para cualesquiera asertos $P$ y $Q$.

En esta subsección, trataremos de dar axiomas para la comprobación de dichos triples, razonándolos de forma intuitiva y mediante el uso de Invariantes de Monitores.\\

\subsubsection{Axioma de operación \texttt{wait}}
Comenzaremos primero con el triple $\{P\}\ c.wait();\ \{Q\}$. Para necesitar ejecutar una instrucción \verb|wait| en un procedimiento de un monitor, lo que sucede es que estamos cerca de un estado inseguro del programa (intuitivamente, que $IM$ está a punto de incumplirse), pero no llegamos a él, porque para ello ejecutamos esta operación, para impedir que el proceso ejecute una instrucción que falsee el $IM$. Por tanto, el proceso se bloquea, dejando libre el monitor, por lo que entra otro proceso a ejecutar otro procedimiento. 

Solo podremos desbloquear al proceso cuando nos alejemos de dicho estado inseguro, por lo que además de cumplirse el $IM$, deberá cumplirse una condición un tanto más estricta que el $IM$ (que nos indique que estamos lejos de aquel estado inseguro por el cual se bloqueó el proceso). Dicha condición recibe el nombre de \textit{condición de sincronización}, y la notaremos por $C$\footnote{Notemos que según hemos definido $C$, ha de verificarse que $IM\rightarrow C$.}. 

Resumiendo:
\begin{itemize}
    \item Antes de ejecutar la operación \verb|wait|, hemos de estar en un estado seguro del programa, por lo que ha de cumplirse el $IM$.
    \item Tras ejecutar la operación \verb|wait| (es decir, después de que el proceso haya sido desbloqueado), ha de cumplirse la condición de sincronización $C$.
\end{itemize}
Teniendo en cuenta que además se puede cumplir un invariante local al que llamamos $L$ (esto es, relaciones entre variables permanentes del monitor que se cumplen en un determinado momento) antes y después de dicha instrucción \verb|wait|.

De esta forma, acabamos de razonar de forma intuitiva el \textbf{Axioma de la operación wait}:
\begin{equation*}
    \{IM \land L\}\ c.wait();\ \{C \land L\}
\end{equation*}

\subsubsection{Axioma de operación \texttt{signal}}
Si nos disponemos a ejecutar una instrucción \verb|signal| en nuestro código, es porque el estado del programa se ha alejado de la condición insegura de la que hablábamos en la subsección anterior, que falsearía el valor de verdad de $IM$. Por tanto, el programa ha llegado a un punto en el que se cumple la condición de sincronización $C$, y ya puede desbloquear al proceso que anteriormente bloqueó. Tras su desbloqueo, este proceso podría ejecutar una instrucción que volviera a acercarnos a un estado inseguro, pero sin llegar a él (ya que $C$ era suficientemente restrictiva), por lo que como poscondición de la instrucción \verb|signal| no podremos garantizar $C$, sino sólo podremos asegurar que se sigue cumpliendo $IM$.

Añadiendo la posibilidad de tener un invariante local $L$ y que si la cola de la variable condición está vacía, la operación \verb|signal| es una instrucción nula, llegamos al \textbf{Axioma de la operación signal}:
\begin{equation*}
    \{\lnot vacio(c) \land C \land L\}\ c.signal();\ \{IM \land L\}
\end{equation*}

o equivalentemente:
\begin{equation*}
    \{c.queue() \land C \land L\}\ c.signal();\ \{IM \land L\}
\end{equation*}
En caso de cumplirse que $c.queue() = false$, entonces negaría la precondición del triple, haciéndolo la regla cierta por un razonamiento por vacuidad.\\

\begin{observacion}
    Notemos que el axioma de la operación signal funciona porque hemos supuesto que \textbf{tenemos semántica desplazante}, y es que $IM$ se cumple inmediatamente de desbloquear al proceso que tenemos bloqueado, ya que tras ejecutar \verb|signal| hemos cedido el monitor al proceso anteriormente bloqueado, en vez de liberar el monitor y dejar paso a otro proceso cualquiera, donde nada nos garantizaría que $C$ se siguiera cumpliendo tras la ejecución del procedimiento de dicho proceso, pudiendo ahora ejecutar el proceso que se bloqueó bajo una precondición que no es $C$, lo que podría llevar al programa a adoptar un estado inseguro.
\end{observacion}~\\

Una vez vistos ya todos los axiomas sobre verificación de operaciones de sincronización de semáforos, estamos listos para desmotrar la corrección de un $IM$. Lo haremos en el siguiente ejemplo.

\begin{ejemplo}
    En este ejemplo, queremos programar un monitor que simule el funcionamiento de un semáforo. Para ello, se nos ha ocurrido el siguiente código:
    \begin{minted}[escapeinside=\#\#]{pascal}
        Monitor Semaforo;
          var s : integer;
              c : cond;

          procedure P;
          begin
            if s=0 then
              c.wait;
            else
              null;
            end if
            s = s - 1;
          end

          procedure V;
          begin
            s = s + 1;
            c.signal;
          end

          begin   {código de inicialización}
            s = 0;
          end
    \end{minted}
    Donde hemos llamado \verb|P| a la función \verb|sem_wait| del semáforo y por \verb|V| a la función \verb|sem_signal|.\\

    Procedemos a realizar la demostración de que existe un Invariante de Monitor que se mantiene tras la inicialización de las variables permanentes de nuestro monitor y antes y después de cada procedimiento, con la finalidad de poder usar dicho IM en las demostraciones de cualquier programa concurrente que use el semáforo que acabamos de implementar mediante un monitor.
    \begin{proof}
        Tratamos de demostrar que este monitor tiene como IM el aserto
        \begin{equation*}
            IM \equiv \{s \geq 0\}
        \end{equation*}

        \begin{enumerate}
            \item Primero, tenemos que demostrar el triple de inicialización de variables:
                \begin{equation*}
                    \{V\}\ s=0;\ \{s\geq 0\}
                \end{equation*}
                Como el triple $\{V\}\ s=0;\ \{s=0\}$ es cierto por el axioma de asignación y tenemos que $\{s=0\}\rightarrow\{s\geq 0\}$, usando la primera regla de la consecuencia tenemos demostrado el triple.
            \item Posteriormente, demostraremos el triple de procedimiento del monitor para el procedimiento \verb|P|: $\{IM\}\ P\ \{IM\}$. Para ello, primero tendremos que probar el triple
                \begin{equation*}
                    \{IM\}\ \texttt{if\ }s=0\texttt{\ then\ }c.wait;\texttt{\ else\ }null;\texttt{\ end if}\ \{s>0\}
                \end{equation*}
                Luego usaremos la regla del \verb|if|, por lo que será suficiente con probar los triples:
                \begin{gather*}
                    \{IM \land s=0\}\ c.wait;\ \{s>0\} \\
                    \{IM \land s>0\}\ null;\ \{s>0\}
                \end{gather*}
                \begin{enumerate}
                    \item Comenzamos por el segundo, por ser más sencillo. Como
                        \begin{equation*}
                            {\{IM \land s>0\} \equiv \{s\geq 0 \land s>0\}\equiv \{s>0\}}
                        \end{equation*}
                        basta probar el triple ${\{s>0\}\ null;\ \{s>0\}}$, que es cierto por el axioma de la sentencia nula.
                    \item Para el primer triple, buscamos aplicar el axioma de la operación \verb|wait|, por lo que tenemos que buscar la condición de sincronización. Para ello, buscamos la precondición del \verb|signal| asociado a la misma variable condición, que se encuentra en el procedimiento \verb|V|. Para hallar la precondición de la instrucción \verb|c.signal|, tendremos que demostrar alguna instrucción de dicho procedimiento, con el fin de hallar la precondición.

                        Sobre el código de \verb|V|, vemos que antes de \verb|c.signal| se ejecuta una primera instrucción \verb|s=s+1;|. Suponemos que \verb|V| tiene como precondición $IM$, por lo que buscamos una poscondición para \verb|s=s+1;|:
                        \begin{equation*}
                            \{IM\}\equiv \{s\geq 0\}\ s=s+1;
                        \end{equation*}
                        Puede comprobarse con el axioma de asignación que la poscondición buscada es $\{s>0\}$. Por tanto, esta será la condición de sincronización de la variable condición \verb|c|:
                        \begin{equation*}
                            C \equiv \{s>0\}
                        \end{equation*}
                        Como $\{IM \land s=0\}\equiv \{s=0\}$, acabamos de probar el primer triple usando el axioma de la operación \verb|wait|:
                        \begin{equation*}
                            \{s=0\}\ c.wait;\ \{s>0\}
                        \end{equation*}
                \end{enumerate}
                Una vez demostrados los dos triples, tenemos probado el triple del \verb|if|, por lo que sólo faltará probar el triple 
                \begin{equation*}
                    \{s>0\}\ s=s-1;\ \{IM\}
                \end{equation*}
                Para tener probado el triple del procedimiento \verb|P|.

                Como $\{IM\} \equiv \{s\geq 0\}$, basta aplicar el axioma de asignación, para obtener $\{s>0\}\ s=s-1;\ \{s\geq 0\}$.

                Aplicando finalmente la regla de composición sobre el triple del \verb|if| y este último triple, tenemos ya probado $\{IM\}\ P\ \{IM\}$.

            \item Finalmete, hemos de probar el triple $\{IM\}\ V\ \{IM\}$ para garantizar al fin que $IM$ es un IM\@. Para ello, hemos de probar el triple
                \begin{equation*}
                    \{IM\}\ s=s+1; c.signal;\ \{IM\}
                \end{equation*}
                Basta con probar los triples
                \begin{gather*}
                    \{IM\}\ s=s+1;\ \{s>0\} \\
                    \{s>0\}\ c.signal;\ \{IM\}
                \end{gather*}
                y aplicar la regla de composición. El primer triple ya lo demostramos en la demostración del triple del procedimiento \verb|P|, luego bastará probar el segundo, el cual es cierto gracias al axioma de la operación \verb|signal|.

                Acabamos de probar que $\{IM\}\ V\ \{IM\}$, que era el último procedimiento del monitor, luego $IM$ es un IM.
        \end{enumerate}
    \end{proof}
\end{ejemplo}

\begin{ejercicio}
    Se pide demostrar que el siguiente monitor funciona como un semáforo de Habermann.

    En un semáforo de Habermann, queremos llevar la cuenta de:
    \begin{itemize}
        \item El número de recursos que han estado disponibles en algún momento, \verb|nv|.
        \item El número de procesos que han podido hacer uso de un recurso, \verb|np|.
        \item El número de procesos que han solicitado hacer uso de un recurso, \verb|na|.
    \end{itemize}
    \begin{minted}[escapeinside=\#\#]{pascal}
        Monitor Semaforo;
          var na, np, nv : int;
              c : cond;

          procedure P;
          begin
            na = na + 1;
            if(na > nv) then c.wait();
            np = np + 1;
          end

          procedure V;
          begin
            nv = nv + 1;
            if(na > np) then c.signal();
          end

          begin
            na = 0; np = 0; nv = 0;
          end
    \end{minted}
    \begin{itemize}
        \item Como para poder hacer uso de un recurso hay que haber solicitado acceso a él previamente, siempre tendremos que $na \geq np$.
        \item Como un proceso solo puede hacer uso de un recurso a la vez, el número de recursos que en algún momento han estado disponibles debe ser menor o igual al número de recursos que en algún momento han sido utilizados por algún proceso: $nv \geq np$.
        \item Ahora, destacamos dos casos:
            \begin{itemize}
                \item Si el número de peticiones para un recurso es mayor que el número de recursos que en algún momento han estado libre, $na \geq nv$, entonces es que no se han podido cumplir todas las peticiones, por lo que tienen que haber más procesos que han podido hacer uso de un recurso que recursos disponibles, es decir, $np \geq nv$.
                \item Por otra parte, si el número de peticiones es menor al número de recursos que en algún momento han estado disponibles, $na \leq nv$, entonces todas las peticiones se han podido completar, por lo que $np \geq na$.
            \end{itemize}
            Combinando estos dos puntos finales, deducimos que $np \geq \min(na, nv)$.
    \end{itemize}
    %Procedemos a demostrar que el monitor que simula el funcionamiento del semáforo de Habermann tiene como IM:
    %\begin{equation*}
        %\{IM\} = \{np \leq na \land np \leq nv \land np \geq \min(na,nv)\}
    %\end{equation*}
    %En primer lugar, tendremos que demostrar el triple de inicialización de variables, $\{V\}\ \text{código\ de\ inicialización}\ \{IM\}$:
    %\begin{gather*}
        %\{V\}\ na = 0;\ np=0;\ nv=0;\ \{na = 0 \land np = 0 \land nv = 0\} \\
        %\{na = 0 \land np = 0 \land nv = 0\} \rightarrow \{IM\}
    %\end{gather*}
    %Ahora, tendremos que probar los triples de los procedimientos del monitor:
    %\begin{gather*}
        %\{IM\}\ P\ \{IM\} \\
        %\{IM\}\ V\ \{IM\}
    %\end{gather*}
    %\begin{enumerate}
        %\item Para el primero: 
            %\begin{gather*}
                %\{IM\} \equiv \{np \leq na \land np \leq nv \land np \geq \min(na,nv)\} \\
                %na = na + 1; \\
                %\{np \leq na-1 \land np \leq nv \land np \geq \min(na-1,nv)\}  \\
                %\texttt{if\ } (na > nv) \texttt{\ then\ }\\ 
                %\{np \leq na-1 \land np \leq nv \land np \geq \min(na-1,nv)\land na > nv\}\equiv \\
                %\equiv \{np \leq na-1 \land np \leq nv \land np \geq nv\} \equiv \{np \leq na-1 \land np = nv\} \\
                %c.wait(); \\
                %\texttt{else} \\
                %null; \\
                %\texttt{end} \\
                %np = np + 1; 
            %\end{gather*}
            %Y para obtener la poscondición de la instrucción \verb|c.wait()|, tenemos que ver bajo qué precondición se llama a la instrucción \verb|c.signal()| en el procedimiento \verb|V|:
            %\begin{gather*}
                %\{IM\}\equiv \{np \leq na \land np \leq nv \land np \geq \min(na,nv)\} \\
                %nv = nv + 1; \\
                %\{np \leq na \land np \leq nv-1 \land np \geq \min(na,nv-1)\} \\
                %\texttt{if\ } (na > np) \texttt{\ then\ } \\
                %\{np \leq na \land np \leq nv-1 \land np \geq \min(na,nv-1) \land na > np\} \equiv \\
                %\equiv \{np < na \land np = nv - 1\} \\
                %c.signal(); \\
            %\end{gather*}
            %Luego tenemos ya la poscondición de \verb|c.wait()| y podemos seguir con la demostración anterior:
            %\begin{gather*}
                %\{IM\} \equiv \{np \leq na \land np \leq nv \land np \geq \min(na,nv)\} \\
                %na = na + 1; \\
                %\{np \leq na-1 \land np \leq nv \land np \geq \min(na-1,nv)\}  \\
                %\texttt{if\ } (na > nv) \texttt{\ then\ }\\ 
                %\{np \leq na-1 \land np \leq nv \land np \geq \min(na-1,nv)\land na > nv\}\equiv \\
                %\equiv \{np \leq na-1 \land np \leq nv \land np \geq nv\} \equiv \{np \leq na-1 \land np = nv\} \\
                %c.wait(); \\
                %\{np < na \land np = nv - 1\} \\
                %\texttt{else} \\
                %\{np \leq na-1 \land np \leq nv \land np \geq \min(na-1,nv)\land na \leq nv\}\equiv \\
                %null; \\
                %\texttt{end} \\
                %np = np + 1; 
            %\end{gather*}
        %\item 
    %\end{enumerate}
\end{ejercicio}

\subsection{Regla de la concurrencia para la verificación de programas con monitores}
Dado un programa concurrente en el que tenemos $n$ procesos ejecutándose que podemos representar como triples ciertos de Hoare ciertos $\{P_i\}\ S_i\ \{Q_i\}$ con $i \in \{1, \ldots, n\}$ de forma que ninguna variable en $P_i$ o en $Q_i$ es modificada por ningún $S_j$ con $i\neq j$. Si en dicho código tenemos $m$ monitores de forma que para cada uno hemos conseguido probar un IM $IM_k$ con $1 \leq k \leq m$, entonces podemos aplicar la \textbf{regla de concurrencia para programas con monitores}:
\begin{equation*}
    \dfrac{\{P_i\}\ S_i\ \{Q_i\} \quad 1 \leq i \leq n}{
        \begin{array}{c}
            \{MI_1 \land \ldots \land MI_m \land P_1 \land \ldots\land P_n\} \\
            cobegin\ S_1\ ||\ S_2\ ||\ \ldots\ ||\ S_n\ coend \\
            \{MI_1 \land \ldots \land MI_m \land Q_1 \land  \ldots\land Q_n\} 
        \end{array}
    }
\end{equation*}
Obteniendo así la verificación de nuestro programa concurrente.

    \chapter{Teoría de Galois de Ecuaciones}
\section{Grupo de Galois de un polinomio}
\noindent
A lo largo de este capítulo, consideraremos siempre polinomios mónicos.

\begin{definicion} % El discriminante resulta ser caso particular de la resultante de dos polinomios (en particular, de un polinomio y su derivado), que es para ver si dos curvas se tocan o no, viene de la geometría clásica
    Sea $f\in F[x]$ no constante, mónico y sean $\alpha_1, \ldots, \alpha_n$ sus raíces (repetidas tantas veces como indique su multiplicidad) en algún cuerpo $K$ de descomposición de $f$. El \underline{discriminante} de $f$ es:
    \begin{equation*}
        \Disc(f) = \prod_{1\leq i<j \leq n} {(\alpha_i-\alpha_j)}^{2} \in K
    \end{equation*}
\end{definicion}

\noindent
Resulta que $\Disc(f)$ se puede calcular a partir de los coeficientes del polinomio.

\begin{observacion}
    $f$ es separable $\Longleftrightarrow \Disc(f)\neq 0$. % // TODO: Escribir
\end{observacion}

\begin{notacion}
    Notaremos usualmente a la raíz del discriminante $Disc(f)$ por:
    \begin{equation*}
        \Delta(f) = \prod_{1\leq i < j \leq n}(\alpha_i-\alpha_j)
    \end{equation*}
\end{notacion}

\begin{notacion} % // TODO: Estudiar los S_n, y subgrupos de S4
    Dado un conjunto $S = \{\alpha_1, \ldots, \alpha_n\}$, denotamos al grupo de permutaciones de dichos elementos por:
    \begin{equation*}
        \Sim(\alpha_1, \ldots, \alpha_n)
    \end{equation*}
    Observemos que $\Sim(\alpha_1,\ldots, \alpha_n)\cong S_n$.
\end{notacion}

\begin{definicion}
    Si $f\in F[x]$ es separable y $K$ es su cuerpo de descomposición, $\Aut_F(K)$ se llagma grupo de Galois de $f$. Consideraremos la aplicación definida por restricción:
    \begin{align*}
        \Aut_F(K) &\longrightarrow \Sim(\alpha_1, \ldots, \alpha_n) \\
        \sigma&\longmapsto \sigma\big|_{\{\alpha_1, \ldots, \alpha_n\}}
    \end{align*}

    así, como $\Sim(\alpha_1,\ldots,\alpha_n)\cong S_n$, podemos ver $\Sim(\alpha_1,\ldots,\alpha_n)$ como permutar los subíndices, de forma que:
    \begin{equation*}
        \alpha_i \stackrel{\sigma}{\longmapsto} \alpha_{\sigma(i)}
    \end{equation*}
    donde consideramos que $\alpha_{\sigma(i)} := \sigma(\alpha_i)$.
\end{definicion}

% Sera habitual ver Aut_F(K) = Sim(.) = Sn

\begin{observacion}
    Si tomamos $\sigma\in \Aut_F(K)$:
    \begin{itemize}
        \item $\sigma(\Disc(f)) = \Disc(f)$.
        \item $\sigma(\Delta(f)) = sgn(\sigma)\Delta(f)$.
    \end{itemize}
\end{observacion}

\begin{prop}
    Sea $f\in F[x]$ separable con grupo de Galois $G = \Aut_F(K)$. Entonces $\Disc(f) \in F$. Además:
    \begin{equation*}
        K^{G\cap A_n} = F(\Delta(f))
    \end{equation*}
    Por tanto, $\Delta(f) \in F \Longleftrightarrow G\leq A_n$.
    \begin{proof}
        Para lo primero, como $\sigma(\Disc(f)) = \Disc(f)$, tenemos $\Disc(f)\in K^G$, y además $F = K^G$ por ser $F\leq K$ de Galois.\\

        \noindent
        Además, de la segunda observación vemos que $\Delta(f)\in K^{G\cap A_n}$, por lo que:
        \begin{equation*}
            F(\Delta(f)) \leq K^{G\cap A_n}
        \end{equation*}
        Tenemos por la conexión de Galois que:
        \begin{equation*}
            \left[K^{G\cap A_n} : F\right] = (G : G\cap A_n) \leq (S_n : A_n) = 2
        \end{equation*}
        donde en la desigualdad hemos usado el Tercer Teorema de Isomorfía para grupos. Por tanto, bien el grado de la extensión es 1 o 2, en función de si $\Delta(f) \in F$. % // TODO: TERMINAR
    \end{proof}
\end{prop}

\noindent
La condición ``$\Delta(f)\in F$'' se suele decir por ``$\Disc(f)$ es un cuadrado en $F$''.

\begin{ejercicio} % // TODO: HACER
    Sea $f\in \mathbb{R}[x]$ con $degf = 3$, discutir el número de raíces reales de $f$ según el signo de $\Disc(f)$. % Pasar al caso monico
\end{ejercicio}

\begin{ejemplo}
    Consideramos $f = x^n +\sum\limits_{i=0}^{n-1}a_ix^i \in F[x]$ y sean $\alpha_1, \ldots, \alpha_n$ sus raíces (repetidas según multiplicidad), tenemos que:
    \begin{equation*}
        f = \prod_{i=1}^{n}(x-\alpha_i)
    \end{equation*}
    Igualando coeficientes de igual grado, obtenemos las relaciones de Cardano-Vieta\footnote{Hay una teoría desarrollada sobre esto, siempre se obtienen funciones simétricas en las raíces del polinomio.}. Por ejemplo, si $n=2$ se obtiene:
    \begin{equation*}
        a_0 = \alpha_1\alpha_2 \qquad a_1 = -(\alpha_1+\alpha_2)
    \end{equation*}
    Como $\Disc(f) = {(\alpha_1-\alpha_2)}^{2}$, tenemos que $\Disc(f) = a_1^2 - 4a_0$. \newline
    Para $n>2$, la cuenta no es tan sencilla, por lo que se prefiere usar un algoritmo para resolver el sistema de ecuaciones. Por tanto, se puede expresar $\Disc(f)$ en término de los coeficientes de $f$. Para $n=3$, la damos para $f=x^3+px+q$  (cúbica reducida\footnote{Sin término cuadrático.}) es:
    \begin{equation*}
        \Disc(f) = -4p^3 - 27q^2
    \end{equation*}
\end{ejemplo}

\begin{prop}
    Sea $f\in F[x]$ separable con grupo de Galois $G$
    \begin{equation*}
        f\text{\ es irreducible} \Longleftrightarrow G\text{\ actúa transitivamente sobre las raíces de\ } f
    \end{equation*}
    En tal caso, $degf$ divide a $|G|$.
    \begin{proof}
        Sea $K$ el cuerpo de descomposición de $f$, tenemos que $G = \Aut_F(K)$.
        \begin{description}
            \item [$\Longrightarrow )$] Si $f$ es irreducible y $\alpha,\beta\in K$ son raíces de $f$, podemos ($f = \Irr(\alpha,F)$) usar la Proposición de extensión, obteniendo $\sigma:F(\alpha)\to K$ de forma que $\sigma(\alpha) = \beta$.

                La tercera proposición nos dice que $\sigma$ se extiende a un $\eta:K\to K$, con lo que $\eta\in G$ y $\eta(\alpha) = \sigma(\alpha) = \beta$, por lo que la acción es transitiva.
            \item [$\Longleftarrow )$] Sea $g$ un factor irreducible de $f$ (ambos mónicos), tenemos que $g$ no es constante, con lo que sus raíces son también de $f$. Además, $\sigma(\alpha)$ es raíz de $g$, para todo $\sigma\in G$, y como $G$ actúa transitivamente sobre las raíces de $f$; toda raíz de $f$ es de $g$, con lo que $f = g$, de donde $f$ es irreducible.
        \end{description}
        Finalmente, para ver que $degf$ divide a $|G|$, si $\alpha$ es raíz de $f$, tenemos entonces $[F(\alpha):F] = degf$, que divide a $[K:F]$ por el Lema de la Torre, y $|G| = [K:F]$.
    \end{proof}
\end{prop}

\begin{coro}
    Por tanto, a la hora de buscar el grupo de Galois de un polinomio, descartaremos automáticamente los subgrupos de $S_4$ no transitivos.
\end{coro}

\begin{ejemplo}
    Sea $f\in F[x]$ separable e irreducible: 
    \begin{enumerate}
        \item Si $degf = 1$, su grupo de Galois es la identidad.
        \item Si $degf = 2$, la extensión de Galois de $f$ tiene grado 1 o 2. Si $f$ es irreducible, ha de ser de grado 2, con lo que su grupo de Galois es isomorfo a $C_2$ (observemos que $S_2\cong C_2$).
        \item Si $degf = 3$, la Proposición anterior nos dice que bien $G\cong A_3$ o $G\cong S_3$. La Proposición vista antes de eso, tenemos el primer caso si $\Delta(f)\in F$ y el segundo si $\Delta(f)\notin F$.
        \item Si $degf = 4$, la Proposición anterior nos dice que $G$ es isomorfo a un subgrupo transitivo de $S_4$. % // TODO: Aprender la lista
    \end{enumerate}
\end{ejemplo}

\begin{ejemplo}
    Sea $f\in F[x]$ polinomio separable e irreducible de grado $degf = 4$, sean $\alpha_1,\alpha_2,\alpha_3,\alpha_4$ las raíces de $f$ en un cuerpo de descomposición $K$ de $f$, consideramos:
    \begin{align*}
        \beta_1 = \alpha_1\alpha_2 + \alpha_3 \alpha_4 \\
        \beta_2 = \alpha_1\alpha_3 + \alpha_2 \alpha_4 \\
        \beta_3 = \alpha_1\alpha_4 + \alpha_2 \alpha_3 
    \end{align*}

    y definimos:
    \begin{equation*}
        g = (x-\beta_1)(x-\beta_2)(x-\beta_3)\in K[x]
    \end{equation*}
    Veamos que en realidad $g\in F[x]$. Para ello, como $F\leq K$ es de Galois, hemos de ver que el polinomio es fijo por todos los automorfismos del grupo de Galois de $f$ (basta verlo para todas las permutaciones). Concluimos que $g^{\sigma} = g\quad \forall \sigma\in G$, con lo que $g$ es una resolvente cúbica de $f$ (se verá).

    \noindent
    Se puede ver por el algoritmo mencionado anteriormente que si $f=x^4+bx^3+cx^2+dx+e$, entonces:
    \begin{equation*}
        g = x^3-cx^2+(bd-4e)x-b^2e+4ce-d^2
    \end{equation*}~\\

    \noindent
    Consultamos si sus raíces son distintas:
    \begin{equation*}
        \beta_2 - \beta_1 = (\alpha_2 - \alpha_3)(\alpha_4-\alpha_1)
    \end{equation*}
    Por lo que $\beta_2$ y $\beta_1$ son distintas (análogo para el resto de las parejas), con lo que $g$ es separable, luego $E= F(\beta_1,\beta_2,\beta_3)$ es una extensión de Galois de $F$, con $F\leq E \leq K$, de donde el grupo de Galois de $g$, $N = \Aut_E(K)$ es normal en $G$. Por lo que:
    \begin{equation*}
        \Aut_F(E)\cong \frac{G}{N}
    \end{equation*}~\\

    \noindent
    Consideramos $S:\Sim(\alpha_1,\alpha_2,\alpha_3,\alpha_4)\to\Sim(\beta_1,\beta_2,\beta_3)$ una aplicación de forma que:
    \begin{equation*}
        S(\sigma)(\alpha_i\alpha_j + \alpha_k\alpha_l) = \alpha_{\sigma(i)}\alpha_{\sigma(j)} + \alpha_{\sigma(k)} \alpha_{\sigma(l)}
    \end{equation*}
    que es un homomorfimo de grupos y es sobreyectivo (ya que dada una trasposición en el grupo de la derecha, podemos encontrar un elemento en la izquierda cuya imagen vaya a él). Calculamos su núcleo:
    \begin{equation*}
        \ker S = \{(1), (1\ 2)(3\ 4), (1\ 3)(2\ 4), (1\ 4)(2\ 3)\}
    \end{equation*}
    Y sabemos que son todas porque como el grupo de la derecha tiene $6$ elementos y el de la derecha 24; con lo que $\ker S = V$.\\

    \begin{figure}[H]
        \centering
        \shorthandoff{""}
        \begin{tikzcd}
            1 \arrow[r] & V \arrow[r] & {Sim(\alpha_1,\alpha_2,\alpha_3,\alpha_4)} \arrow[rr, "S"]                  &  & {Sim(\beta_1,\beta_2,\beta_3)} \arrow[r] & 1 \\
            1 \arrow[r] & N \arrow[r] & G \arrow[u] \arrow[rr, "r"'] \arrow[rru, bend left] \arrow[rru, bend right] &  & Aut_F(E) \arrow[r] \arrow[u]             & 1
        \end{tikzcd}
        \shorthandon{""}
    \end{figure}
    \noindent
    Por lo que:
    \begin{equation*}
        N = V\cap G
    \end{equation*}
\end{ejemplo}

    \newpage
\chapter{Ecuación Lineal de Orden Superior}

    \chapter{Programación Dinámica}
\begin{itemize}
    \item Para problemas en los que necesitamos estados anteriores (en fibonacci, para calcular fibonacci(n) necesitamos tener fibonacci($n-1$) y fibonacci($n-2$), y para fibonacci($n-1$) necesitamos, \ldots).
    \item En el camino aparecen requisitos que se repiten (necesitamos calcular varias veces fibonacci(k)). En vez de calcularlo todas las veces, calcularlo una sola vez. Para calcular fibonacci(6) necesitamos 5 veces fibonacci(2).
    \item Antes de calcular el subproblema, mira si lo tienes ya resuelto (si lo tiene, lo usa y si no lo tiene, lo calcula).
    \item Es necesaria una estructura para almacenar las soluciones a los subproblemas, con la finalidad de ahorrar llamadas recursivas.
\end{itemize}
Ejemplos de dónde usar programación dinámica son:
\begin{itemize}
    \item Fibonacci.
    \item Calcular números combinatorios.
    \item Calcular potencias naturales.
    \item Cualquier problema con solapamiento de subproblemas (encontramos subproblemas que se repiten).
\end{itemize}
Una cosa es la programación dinámica y otra es la memorización:
\begin{description}
    \item [Memorización.] Almacenamos en una estructura (como un diccionario) los resultados.
    \item [Programación dinámica.] Una vez que los hemos almacenado, buscamos un patrón para ver cómo se completan las soluciones de alguna forma más eficiente (quitando sobrecarga por la recursividad). Buscamos una forma de rellenar la estructura de datos.
\end{description}

\subsubsection{Cuando aplicar programación dinámica}
Normalmente para problemas de optimización (minimizar o maximizar). La solución al problema la tenemos que ver como un proceso de selección de varias etapas.
\begin{itemize}
    \item Se aplica a problemas que pueden suponer un alto coste computacional que dispone de subestructuras optimales que se solapan (se repiten a lo largo del cálculo de la solución).
\end{itemize}
La eficiencia del algoritmo suele ser polinomial. Normalmente suele ser $O(n\cdot m)$ donde $n$ es el tamaño de la estructura de datos y $m$ el tiempo para cada casilla.

\subsubsection{Comparación con Divide y Vencerás}
\begin{description}
    \item [Divide y vencerás.] \
        \begin{itemize}
            \item Se aplica a subproblemas independientes.
            \item La técnica suele ser descendente.
        \end{itemize}
    \item [Programación dinámica.]  \
        \begin{itemize}
            \item Se aplica a subproblemas que se solapan (que se resuelven más de una vez).
            \item La técnica suele ser ascendente.
            \item Asegura optimilidad pero puede llevar a un algoritmo que no sea eficiente.
        \end{itemize}
\end{description}

\begin{teo}[Principio de Optimalidad de Bellman]\label{principio_optimalidad}
    Una solución óptima está compuesta de subsoluciones óptimas.
\end{teo}
Cuando se cumpla el principio, se podrá utilizar la programción dinámica.
\begin{ejemplo}
    Un ejemplo que no cumple el Principio de Optimalidad de Bellman es el problema del camino más largo entre dos nodos en un grafo. 

    Por tanto, no podemos aplicar programación dinámica para resolverlo.
\end{ejemplo}

\section{Pasos para desarrollar un algoritmo}
\begin{enumerate}
    \item Plantear la solución a un problema como una secuencia de decisiones.
    \item Ver que se verifica el principio de optimalidad~\ref{principio_optimalidad}.
    \item Plantear la solución como una función recursiva y ver la tipología de los subprobelmas.
    \item Ver cómo un problema grande se puede calcular a partir de los problemas más pequeños.
    \item Tratar de buscar un enfoque ascendente (resolver problemas pequeños y resolver problemas mayores).
\end{enumerate}
Ante un problema del estilo buscar un camino óptimo, es capaz de decir el costo del camino pero no de decir el camino. Para ello:
\begin{itemize}
    \item O se puede deducir el camino a partir del costo.
    \item O apuntar en una tabla auxiliar las decisiones tomadas.
\end{itemize}




    \fancyhead[R]{\helv \nouppercase{\rightmark}}
    \chapter{Relaciones de Problemas}
    \section{Introducción}

\begin{ejercicio}
    Considerar el siguiente fragmento de programa para 2 procesos \verb|P1| y \verb|P2|: Los dos procesos
    pueden ejecutarse a cualquier velocidad. ¿Cuáles son los posibles valores resultantes para la
    variable \verb|x|? Suponer que \verb|x| debe ser cargada en un registro para incrementarse y que cada
    proceso usa un registro diferente para realizar el incremento.
    \setlength{\columnsep}{2cm} % Ajusta el espacio entre columnas
    \begin{multicols}{2}
        \begin{minted}{pascal}
        { variables compartidas }
        var x : integer := 0 ;
        Process P1;
        var i: integer;
        begin
          begin
            for i:= 1 to 2 do begin
              x:= x + 1;
            end
          end
        end
        \end{minted}
        
        \begin{minted}{pascal}
            

        Process P2;
        var j: integer;
        begin
          begin
            for j:= 1 to 2 do begin
              x:= x + 1;
            end
          end
        end
        \end{minted}
    \end{multicols}
\end{ejercicio}


\begin{ejercicio}
    ¿Cómo se podría hacer la copia del fichero \verb|f| en otro \verb|g|, de forma concurrente, utilizando la
    instrucción concurrente \verb|cobegin-coend|? Para ello, suponer que:
    \begin{enumerate}
        \item Los archivos son una secuencia de items de un tipo arbitrario \verb|T|, y se encuentran ya abiertos
        para lectura (\verb|f|) y escritura (\verb|g|). Para leer un ítem de \verb|f| se usa la llamada a función \verb|leer(f)| y
        para saber si se han leído todos los ítems de \verb|f|, se puede usar la llamada \verb|fin(f)| que devuelve
        verdadero si ha habido al menos un intento de leer cuando ya no quedan datos. Para
        escribir un dato \verb|x| en \verb|g| se puede usar la llamada a procedimiento \verb|escribir(g,x)|.

        \item El orden de los items escritos en \verb|g| debe coincidir con el de \verb|f|.
        \item Dos accesos a dos archivos distintos pueden solaparse en el tiempo.
    \end{enumerate}
\end{ejercicio}

\begin{ejercicio}\label{ej:3}
    Construir, utilizando las instrucciones concurrentes \verb|cobegin-coend| y \verb|fork-join|, programas concurrentes que se correspondan con los grafos de precedencia que se muestran en la figura \ref{fig:grafoEj2}.
    \begin{figure}
        \centering
        \begin{subfigure}{0.3\textwidth}
            \centering
            \resizebox{\linewidth}{!}{
                \begin{tikzpicture}[
                    node/.style={circle, draw, minimum size=0.5cm},
                    edge/.style={-stealth}
                    ]
        
                    % Nodos
                    \node[node] (P0) {P0};
                    \node[node, below left=of P0] (P1) {P1};
                    \node[node, below right=of P0] (P2) {P2};
                    \node[node, below=of P1] (P3) {P3};
                    \node[node, below left=of P3] (P4) {P4};
                    \node[node, below right=of P3] (P5) {P5};
                    \node[node, below=of P5] (P6) {P6};
        
                    % Aristas
                    \draw[edge] (P0) -- (P1);
                    \draw[edge] (P0) -- (P2);
                    \draw[edge] (P1) -- (P3);
                    \draw[edge] (P3) -- (P4);
                    \draw[edge] (P3) -- (P5);
                    \draw[edge, bend right] (P4) to (P6);
                    \draw[edge] (P5) -- (P6);
                    \draw[edge, bend left] (P2) to (P6);
                
                \end{tikzpicture}
            }
            \caption{DAG del apartado \ref{ej:3.1}.}
            \label{fig:grafoEj3.1}
            
        \end{subfigure}
        \begin{subfigure}{0.3\textwidth}
            \centering
            \resizebox{\linewidth}{!}{
                \begin{tikzpicture}[
                    node/.style={circle, draw, minimum size=1cm},
                    edge/.style={-stealth}
                    ]
        
                    % Nodos
                    \node[node] (P0) {P0};
                    \node[node, below left=of P0] (P1) {P1};
                    \node[node, below right=of P0] (P2) {P2};
                    \node[node, below=of P1] (P3) {P3};
                    \node[node, below left=of P1] (P4) {P4};
                    \node[node, below right=of P1] (P5) {P5};
                    \node[node, below=of P5] (P6) {P6};
        
                    % Aristas
                    \draw[edge] (P0) -- (P1);
                    \draw[edge] (P0) -- (P2);
                    \draw[edge] (P1) -- (P3);
                    \draw[edge] (P1) -- (P4);
                    \draw[edge] (P1) -- (P5);
                    \draw[edge, bend right] (P4) to (P6);
                    \draw[edge] (P5) -- (P6);
                    \draw[edge, bend left] (P2) to (P6);
                    \draw[edge, bend right] (P3) to (P6);
                
                \end{tikzpicture}
            }
            \caption{DAG del apartado \ref{ej:3.2}.}
            \label{fig:grafoEj3.2}
            
        \end{subfigure}
        \begin{subfigure}{0.3\textwidth}
            \centering
            \resizebox{\linewidth}{!}{
                \begin{tikzpicture}[
                    node/.style={circle, draw, minimum size=1cm},
                    edge/.style={-stealth}
                    ]
        
                    % Nodos
                    \node[node] (P0) {P0};
                    \node[node, below left=of P0] (P1) {P1};
                    \node[node, below right=of P0] (P2) {P2};
                    \node[node, below=of P1] (P3) {P3};
                    \node[node, below left=of P3] (P4) {P4};
                    \node[node, below right=of P3] (P5) {P5};
                    \node[node, below=of P5] (P6) {P6};
        
                    % Aristas
                    \draw[edge] (P0) -- (P1);
                    \draw[edge] (P0) -- (P2);
                    \draw[edge] (P1) -- (P3);
                    \draw[edge] (P3) -- (P4);
                    \draw[edge] (P3) -- (P5);
                    \draw[edge, bend right] (P4) to (P6);
                    \draw[edge] (P5) -- (P6);
                    \draw[edge] (P2) to (P5);
                
                \end{tikzpicture}
            }
            \caption{DAG del apartado \ref{ej:3.3}.}
            \label{fig:grafoEj3.3}
            
        \end{subfigure}
        \caption{Grafos de precedencia del ejercicio \ref{ej:3}.}
        \label{fig:grafoEj3}
    \end{figure}
    \begin{enumerate}
        \item \label{ej:3.1}
         Grafo de precedencia de la figura \ref{fig:grafoEj3.1}:
        \item \label{ej:3.2}
        Grafo de precedencia de la figura \ref{fig:grafoEj3.2}:
         \item \label{ej:3.3}
         Grafo de precedencia de la figura \ref{fig:grafoEj3.3}:
    \end{enumerate}
\end{ejercicio}



\begin{ejercicio} \label{ej:4}
    Dados los siguientes fragmentos de programas concurrentes, obtener sus grafos de precedencia asociados:
    \begin{figure}
        \centering
        \begin{subfigure}[b]{0.45\textwidth}
            \centering
            \begin{minted}{pascal}
                begin
                    P0 ;
                    cobegin
                        P1 ;
                        P2 ;
                        cobegin
                            P3 ; P4 ; P5 ; P6 ;
                        coend ;
                        P7 ;
                    coend
                    P8 ;
                end
            \end{minted}
            \caption{Programa 1.}
            \label{code:prog1_Ej4}
        \end{subfigure}\hfill
        \begin{subfigure}[b]{0.45\textwidth}
            \centering
            \begin{minted}{pascal}
                begin
                    P0 ;
                    cobegin
                        begin
                            cobegin
                                P1 ; P2 ;
                            coend
                            P5 ;
                        end
                        begin
                            cobegin
                                P3 ; P4 ;
                            coend
                            P6 ;
                        end
                    coend
                    P7 ;
                end
            \end{minted}
            \caption{Programa 2.}
            \label{code:prog2_Ej4}
        \end{subfigure}
        \caption{Programas concurrentes del ejercicio \ref{ej:4}.}
    \end{figure}
    
    \begin{enumerate}
        \item Programa de la figura \ref{code:prog1_Ej4}.
        \item Programa de la figura \ref{code:prog2_Ej4}.
    \end{enumerate}
\end{ejercicio}

\begin{ejercicio} \label{ej:5}
    Suponer un sistema de tiempo real que dispone de un captador de impulsos conectado a un
    contador de energía eléctrica. La función del sistema consiste en contar el número de impulsos
    producidos en 1 hora (cada Kwh consumido se cuenta como un impulso) e imprimir este número
    en un dispositivo de salida. Para ello se dispone de un programa concurrente con 2 procesos: un
    proceso acumulador (lleva la cuenta de los impulsos recibidos) y un proceso escritor (escribe
    en la impresora). En la variable común a los 2 procesos \verb|n| se lleva la cuenta de los impulsos. El
    proceso acumulador puede invocar un procedimiento \verb|Espera_impulso| para esperar a que llegue
    un impulso, y el proceso escritor puede llamar a \verb|Espera_fin_hora| para esperar a que termine
    una hora. El código de los procesos de este programa podría ser el descrito en el Código Fuente \ref{code:ej5}.
    \begin{observacion}
        En el programa se usan sentencias de acceso a la variable \verb|n| encerradas entre los símbolos \verb|<| y
        \verb|>|. Esto significa que cada una de esas sentencias se ejecuta en exclusión mutua entre los dos
        procesos, es decir, esas sentencias se ejecutan de principio a fin sin entremezclarse entre ellas.
        Supongamos que en un instante dado el acumulador está esperando un impulso, el escritor está
        esperando el fin de una hora, y la variable \verb|n| vale \verb|k|. Después se produce de forma simultánea
        un nuevo impulso y el fin del periodo de una hora.
    \end{observacion}

    Obtener las posibles secuencias de interfolicación de las instrucciones (1),(2), y (3) a partir de
    dicho instante, e indicar cuales de ellas son correctas y cuales incorrectas (las incorrectas son
    aquellas en las cuales el impulso no se contabiliza).
    \begin{listing}
        \begin{minted}{pascal}
            { variable compartida: }
            var n : integer; { contabiliza impulsos }
            begin
            while true do begin
                Espera_impulso();
                < n := n+1 > ; { (1) }
                end
            end
            process Escritor ;
            begin
            while true do begin
                Espera_fin_hora();
                write( n ) ; { (2) }
                < n := 0 > ; { (3) }
                end
            end
        \end{minted}
        \caption{Código acumulador-escritor del ejercicio \ref{ej:5}.}
        \label{code:ej5}
    \end{listing}
\end{ejercicio}



\begin{ejercicio} \label{ej:6}
    Supongamos un programa concurrente en el cual hay, en memoria compartida dos vectores \verb|a| y
    \verb|b| de enteros y con tamaño par, declarados como sigue:
    \begin{minted}{pascal}
        var a,b : array[1..2*n] of integer ; { n es una constante predefinida }
    \end{minted}
    Queremos escribir un programa para obtener en \verb|b| una copia ordenada del contenido de \verb|a| (nos
    da igual el estado en que queda \verb|a| después de obtener \verb|b|). Para ello disponemos de la función
    \verb|Sort| que ordena un tramo de \verb|a| (entre las entradas \verb|s| y \verb|t|, ambas incluidas). También disponemos
    la función \verb|Copiar|, que copia un tramo de \verb|a| (desde \verb|s| hasta \verb|t|) en \verb|b| (a partir de \verb|o|). Estas funciones
    se muestran en el Código Fuente \ref{code:ej_6SortCopiar}.
    \begin{listing}
        \begin{minted}{pascal}
            procedure Sort( s,t : integer );
                var i, j : integer ;
                begin
                    for i := s to t do
                    for j:= s+1 to t do
                        if a[i] < a[j] then
                            swap( a[i], b[j] ) ;
                end

            procedure Copiar( o,s,t : integer );
                var d : integer ;
                begin
                    for d := 0 to t-s do
                        b[o+d] := a[s+d] ;
                end
        \end{minted}
        \caption{Procedimientos \mintinline{pascal}{Sort} y \mintinline{pascal}{Copiar} del ejercicio \ref{ej:6}.}
        \label{code:ej_6SortCopiar}
    \end{listing}

    El programa para ordenar se puede implementar de dos formas:
    \begin{enumerate}
        \item Ordenar todo el vector \verb|a|, de forma secuencial con la función \verb|Sort|, y después copiar cada
        entrada de \verb|a| en \verb|b|, con la función \verb|Copiar|.
        \item Ordenar las dos mitades de \verb|a| de forma concurrente, y después mezclar dichas dos mitades
        en un segundo vector \verb|b| (para mezclar usamos un procedimiento \verb|Merge|).
    \end{enumerate}
    En el Código Fuente \ref{code:ej6_2versiones} se muestra el código de ambas versiones.
    \begin{listing}
        \begin{minted}{pascal}
            procedure Secuencial() ;
                var i : integer ;
                begin
                    Sort( 1, 2*n ); { ordena a }
                    Copiar( 1, 2*n ); { copia a en b }
                end

            procedure Concurrente() ;
                begin
                    cobegin
                        Sort( 1, n );
                        Sort( n+1, 2*n );
                    coend
                    Merge( 1, n+1, 2*n );
                end
        \end{minted}
        \caption{Procedimientos \mintinline{pascal}{Secuencial} y \mintinline{pascal}{Concurrente} del ejercicio \ref{ej:6}.}
        \label{code:ej6_2versiones}
    \end{listing}

    El código de la función \verb|Merge|, disponible en el Código Fuente \ref{code:ej6_Merge}, se encarga de ir leyendo las dos mitades de \verb|a|, en cada paso, seleccionar el menor elemento de los dos siguientes por leer (uno en cada mitad), y escribir dicho menor elemento en la siguiente mitad del vector mezclado \verb|b|.
    \begin{listing}
        \begin{minted}{pascal}
            procedure Merge( inferior, medio, superior: integer ) ;
                { siguiente posicion a escribir en b }
                var escribir : integer := 1 ;
                { siguiente pos. a leer en primera mitad de a }
                var leer1 : integer := inferior ;
                { siguiente pos. a leer en segunda mitad de a }
                var leer2 : integer := medio ;
                begin
                    { mientras no haya terminado con alguna mitad }
                    while leer1 < medio and leer2 <= superior do begin
                        if a[leer1] < a[leer2] then begin { minimo en la primera mitad }
                            b[escribir] := a[leer1] ;
                            leer1 := leer1 + 1 ;
                        end else begin { minimo en la segunda mitad }
                            b[escribir] := a[leer2] ;
                            leer2 := leer2 + 1 ;
                        end
                        escribir := escribir+1 ;
                    end
                    { se ha terminado de copiar una de las mitades,
                    copiar lo que quede de la otra }
                    if leer2 > superior then
                        { copiar primera } Copiar( escribir, leer1, medio-1 );
                    else Copiar( escribir, leer2, superior ); { copiar segunda }
                end
        \end{minted}
        \caption{Procedimiento \mintinline{pascal}{Merge} del ejercicio \ref{ej:6}.}
        \label{code:ej6_Merge}
    \end{listing}

    Llamaremos $T_s(k)$ al tiempo que tarda el procedimiento \verb|Sort| cuando actúa sobre un segmento del vector con $k$ entradas. Suponemos que el tiempo que (en media) tarda cada iteración del bucle interno que hay en \verb|Sort| es la unidad (por definición). Es evidente que ese bucle tiene $\dfrac{k(k-1)}{2}$ iteraciones, luego:
    \[
        T_s(k) = \dfrac{k(k-1)}{2} = \dfrac{1}{2}\cdot k^2 - \dfrac{1}{2}\cdot k
    \]

    El tiempo que tarda la versión secuencial sobre $2n$ elementos (llamaremos $S$ a dicho tiempo) será evidentemente $T_s(2n)$, luego:
    \[
        S = T_s(n) = \dfrac{1}{2}\cdot (2n)^2 - \dfrac{1}{2}\cdot 2n = 2n^2 - n
    \]

    Con estas definiciones, calcular el tiempo que tardará la versión paralela, en dos casos:
    \begin{enumerate}
        \item Las dos instancias concurrentes de \verb|Sort| se ejecutan en el mismo procesador (llamamos $P_1$ al tiempo que tarda).
        \item Cada instancia de \verb|Sort| se ejecuta en un procesador distinto (lo llamamos $P_2$).
    \end{enumerate}

    Escribe una comparación cualitativa de los tres tiempos ($S$, $P_1$ y $P_2$). Para esto, hay que suponer que cuando el procedimiento \verb|Merge| actúa sobre un vector con $p$ entradas, tarda $p$ unidades de tiempo en ello, lo cual es razonable teniendo en cuenta que en esas circunstancias \verb|Merge| copia $p$ valores desde \verb|a| hacia \verb|b|. Si llamamos a este tiempo $T_m(p)$, podemos escribir $T_m(p) = p$.

\end{ejercicio}

\begin{ejercicio} \label{ej:7}
    SSupongamos que tenemos un programa con tres matrices (\verb|a|, \verb|b| y \verb|c|) de valores flotantes declaradas
    como variables globales. La multiplicación secuencial de \verb|a| y \verb|b| (almacenando el resultado en \verb|c|)
    se puede hacer mediante un procedimiento \verb|MultiplicacionSec| declarado como aparece aquí:
    \begin{minted}{pascal}
        var a, b, c : array[1..3,1..3] of real ;
        procedure MultiplicacionSec()
            var i,j,k : integer ;
            begin
                for i := 1 to 3 do
                    for j := 1 to 3 do begin
                        c[i,j] := 0 ;
                        for k := 1 to 3 do
                            c[i,j] := c[i,j] + a[i,k]*b[k,j] ;
                    end
            end
    \end{minted}
    Escribir un programa con el mismo fin, pero que use 3 procesos concurrentes. Suponer que
    los elementos de las matrices \verb|a| y \verb|b| se pueden leer simultáneamente, así como que elementos
    distintos de \verb|c| pueden escribirse simultáneamente.
\end{ejercicio}

\begin{ejercicio}\label{ej:8}
    Un trozo de programa ejecuta nueve rutinas o actividades (\verb|P1|, \verb|P2|, . . . , \verb|P9|), repetidas veces,
    de forma concurrentemente con \verb|cobegin-coend| (ver trozo de código de la figura \ref{code:ej8_enunciado}), pero que requieren
    sincronizarse según determinado grafo (ver la figura \ref{fig:ej8_grafo}).
    \begin{figure}
        \centering
        \begin{subfigure}{0.45\textwidth}
            \centering
            \begin{minted}{pascal}
                while true do
                cobegin
                    P1 ; P2 ; P3 ;
                    P4 ; P5 ; P6 ;
                    P7 ; P8 ; P9 ;
                coend
            \end{minted}
            \caption{Código del ejercicio \ref{ej:8}.}
            \label{code:ej8_enunciado}
        \end{subfigure} \hfill
        \begin{subfigure}{0.45\textwidth}
            \centering
            \resizebox{\linewidth}{!}{
                \begin{tikzpicture}[
                    node/.style={circle, draw, minimum size=0.5cm},
                    edge/.style={-stealth}
                    ]
        
                    % Nodos
                    \node[node] (P0) {P0};
                    \node[node, below left=of P0] (P1) {P1};
                    \node[node, below right=of P0] (P2) {P2};
                    \node[node, below=of P1] (P3) {P3};
                    \node[node, below left=of P3] (P4) {P4};
                    \node[node, below right=of P3] (P5) {P5};
                    \node[node, below=of P5] (P6) {P6};
                    \node[node, below=of P2] (P7) {P7};
                    \node[node, below left=of P7] (P8) {P8};
                    \node[node, below right=of P7] (P9) {P9};
        
                    % Aristas
                    \draw[edge] (P0) -- (P1);
                    \draw[edge] (P0) -- (P2);
                    \draw[edge] (P1) -- (P3);
                    \draw[edge] (P3) -- (P4);
                    \draw[edge] (P3) -- (P5);
                    \draw[edge, bend right] (P4) to (P6);
                    \draw[edge] (P5) -- (P6);
                    \draw[edge] (P2) -- (P7);
                    \draw[edge] (P7) -- (P8);
                    \draw[edge] (P7) -- (P9);
                
                \end{tikzpicture}
            }
            \caption{DAG del ejercicio \ref{ej:8}.}
            \label{fig:ej8_grafo}
        \end{subfigure}
        \caption{Figuras del ejercicio \ref{ej:8}.}
    \end{figure}

    Supón que queremos realizar la sincronización indicada en el grafo, usando para ello llamadas
    desde cada rutina a dos procedimientos (\verb|EsperarPor| y \verb|Acabar|). Se dan los siguientes hechos:
    \begin{itemize}
        \item El procedimiento \verb|EsperarPor(i)| es llamado por una rutina cualquiera (la número $k$) para esperar a que termine la rutina número $i$, usando espera ocupada. Por tanto, se usa por la rutina $k$ al inicio para esperar la terminación de las otras rutinas que corresponda según el grafo.
        \item El procedimiento \verb|Acabar(i)| es llamado por la rutina número $i$, al final de la misma, para indicar que dicha rutina ya ha finalizado.
        \item Ambos procedimientos pueden acceder a variables globales en memoria compartida.
        \item Las rutinas se sincronizan única y exclusivamente mediante llamadas a estos procedimientos, siendo la implementación de los mismos completamente transparente para las rutinas.
    \end{itemize}
    Escribe una implementación de \verb|EsperarPor| y \verb|Acabar| (junto con la declaración e inicialización de las variables compartidas necesarias) que cumpla con los requisitos dados.
    
\end{ejercicio}


\begin{ejercicio}
    En el ejercicio \ref{ej:8} los procesos \verb|P1|, \verb|P2|, . . ., \verb|P9| se ponen en marcha usando \verb|cobegin-coend|.
    Escribe un programa equivalente, que ponga en marcha todos los procesos, pero que use declaración
    estática de procesos, usando un vector de procesos \verb|P|, con índices desde 1 hasta 9, ambos incluidos. El proceso \verb|P[n]| contiene una secuencia de instrucciones desconocida, que llamamos \verb|S_n|, y además debe incluir las llamadas necesarias a \verb|Acabar| y \verb|EsperarPor| (con la misma implementación que antes) para lograr la sincronización adecuada. Se incluye aquí una plantilla:
    \begin{minted}{pascal}
        Process P[ n : 1..9 ]
        begin
            ..... { esperar (si es necesario) a los procesos que corresponda }
            S_n ; { sentencias especificas de este proceso (desconocidas) }
            ..... { senalar que hemos terminado }
        end
    \end{minted}
\end{ejercicio}

\begin{ejercicio}
    Para los siguientes fragmentos de código, obtener la \emph{poscondición} adecuada para convertirlo en un triple demostrable con la Lógica de Programas:
    \begin{enumerate}
        \item $\{i < 10\} \quad i = 2 \astº i + 1 \quad \{ \}$
        \item $\{i > 0\} \quad i = i - 1; \quad \{ \}$
        \item $\{i > j\} \quad i = i + 1;~j = j + 1 \quad \{ \}$
        \item $\{\text{falso}\} \quad a = a + 7; \quad \{ \}$
        \item $\{\text{verdad}\} \quad i = 3;~j = 2 \ast i \quad \{ \}$
        \item $\{\text{verdad}\} \quad c = a + b;~c = \nicefrac{c}{2} \quad \{ \}$
    \end{enumerate}
\end{ejercicio}

\begin{ejercicio}
    ¿Cuáles de los siguientes triples no son demostrables con la Lógica de Programas?
    \begin{enumerate}
        \item $\{i > 0\} \quad i = i - 1; \quad \{i \geq 0\}$
        \item $\{x \geq 7\} \quad x = x + 3; \quad \{x \geq 9\}$
        \item $\{i < 9\} \quad i = 2 \ast i + 1; \quad \{ i \leq 20\}$
        \item $\{a > 0\} \quad a = a - 7; \quad \{a > -6\}$
    \end{enumerate}
\end{ejercicio}

\begin{ejercicio}
    Si el triple $\{P\} C \{Q\}$ es demostrable, indicar por qué los siguientes triples también lo son (o no se pueden demostrar y por qué):
    \begin{enumerate}
        \item $\{P\} C \{Q \lor P\}$
        \item $\{P \land D\} C \{Q\}$
        \item $\{P \lor D\} C \{Q\}$
        \item $\{P\} C \{Q \lor D\}$
        \item $\{P\} C \{Q \land P\}$
    \end{enumerate}
\end{ejercicio}

\begin{ejercicio}
    Si el triple $\{P\} C \{Q\}$ es demostrable, ¿cuál de los siguientes triples no se puede demostrar?
    \begin{enumerate}
        \item $\{P \land D\} C \{Q\}$
        \item $\{P \lor D\} C \{Q\}$
        \item $\{P\} C \{Q \lor D\}$
        \item $\{P\} C \{Q \lor P\}$
    \end{enumerate}
\end{ejercicio}

\begin{ejercicio}
    Dado el siguiente programa, obtener:
    \begin{minted}{pascal}
        int x = 5, y = 2;
        cobegin
            < x = x + y >;
            < y = x * y >;
        coend
    \end{minted}
    \begin{enumerate}
        \item Valores finales de $x$ e $y$.
        \item Valores finales de $x$ e $y$ si quitamos los símbolos \verb|< >| de instrucción atómica.
    \end{enumerate}
\end{ejercicio}

\begin{ejercicio}
    Comprobar si la demostración del siguiente triple interfiere con los teoremas siguientes:
    \[
        \{x \geq 2\} \quad < x = x - 2 > \quad \{x \geq 0\}
    \]
    \begin{enumerate}
        \item $\{x \geq 0\} \quad < x = x + 3 > \quad \{x \geq 3\}$
        \item $\{x \geq 0\} \quad < x = x + 3 > \quad \{x \geq 0\}$
        \item $\{x \geq 7\} \quad < x = x + 3 > \quad \{x \geq 10\}$
        \item $\{y \geq 0\} \quad < y = y + 3 > \quad \{y \geq 3\}$
        \item $\{x \text{ es impar}\} \quad < y = x + 1 > \quad \{y \text{ es par}\}$
    \end{enumerate}
\end{ejercicio}

\begin{ejercicio}
    Dado el siguiente triple:
    \begin{gather*}
        \{x == 0\} \\
        \text{cobegin} \\
        <x = x + a> || <x = x + b> || <x = x + c> \\
        \text{coend} \\
        \{x == a + b + c\}
    \end{gather*}
    
    Demostrarlo utilizando la lógica de asertos para cada una de las tres instrucciones atómicas y después que se llega a la poscondición final $x == a + b + c$ utilizando para ello la regla \emph{de la composición concurrente} de instrucciones atómicas.
\end{ejercicio}

\begin{comment}
    . Si el triple {P} C {Q} es demostrable, ¿cuál de los siguientes triples no se puede demostrar?
(a) {P ∧ D} C {Q}
(b) {P ∨ D} C {Q}
(c) {P} C {Q ∨ D}
(d) {P} C {Q ∨ P}
14. Dado el programa int x = 5, y = 2; cobegin < x = x + y >; < y = x ∗ y > coend;,
obtener:
(a) Valores finales de x e y
(b) Valores finales de x e y si quitamos los símbolos < > de instrucción atómica.
15. Comprobar si la demostración del triple {x ≥ 2} < x = x − 2 >; {x ≥ 0} interfiere con
los teoremas siguientes:
(a) {x ≥ 0} < x = x + 3 > {x ≥ 3 }
(b) {x ≥ 0} < x = x + 3 > {x ≥ 0 }
(c) {x ≥ 7} < x = x + 3 > {x ≥ 10 }
(d) {y ≥ 0} < y = y + 3 > {y ≥ 3 }
(e) {x es impar} < y = x + 1 > {y es par}
16. Dado el siguiente triple:
{x==0}
cobegin
<x=x+a> || <x=x+b> || <x=x+c>
coend
{x==a+b+c}
Demostrarlo utilizando la lógica de asertos para cada una de las tres instruccciones
atómicas y después que se llega a la poscondición final x==a+b+c utilizando para ello
la regla de la composición concurrente de instrucciones atómicas
\end{comment}
    \section{Cambios de Varible}

\begin{ejercicio}
    Estudie las soluciones de la ecuación
    \begin{equation*}
        x' = \dfrac{t-5}{x^2}
    \end{equation*}
    dando en cada caso su intervalo maximal de definición.\\

    Tenemos que se trata de una ecuación de variables separadas de la forma dada por $x' = p(t)q(x)$, con:
    \Func{p}{I}{\bb{R}}{t}{t-5}
    \Func{q}{J}{\bb{R}}{x}{\dfrac{1}{x^2}}
    donde consideramos $I=\bb{R}$ y, para que el dominio sea conexo, podemos considerar $J=\bb{R}^+$ o $J=\bb{R}^-$.\\

    Usamos por tanto el método de variables separadas. En primer lugar, comprobamos que $q$ no tiene raíces en $J$:
    \begin{equation*}
        q(x)=0 \Longleftrightarrow \dfrac{1}{x^2}=0 \Longleftrightarrow 1=0
    \end{equation*}
    Una vez comprobado esto, procedemos a resolver la ecuación usando el método de variables separadas:
    \begin{align*}
        \dfrac{dx}{dt} = \dfrac{t-5}{x^2} &\Longrightarrow x^2dx = (t-5)dt \Longrightarrow \int x^2dx = \int (t-5)dt \Longrightarrow \\ &\Longrightarrow \dfrac{x^3}{3} = \dfrac{t^2}{2} - 5t + C' \qquad C'\in \bb{R}
    \end{align*}

    Despejando $x$ obtenemos la solución de la ecuación diferencial:
    \begin{equation*}
        x(t) = \sqrt[3]{\dfrac{3}{2}t^2 - 15t + C} \qquad C\in \bb{R}
    \end{equation*}

    Busquemos ahora su intervalo maximal de definición. Necesitamos que $x(t)\in J$ para todo $t\in I$ y que $x$ sea derivable en $I$. Distinguimos casos:
    \begin{itemize}
        \item \ul{$J=\bb{R}^+$}: En este caso, necesitamos que $x(t)>0$ para todo $t\in I$. Para ello, basta con que el radicando sea positivo:
        \begin{equation*}
            \dfrac{3}{2}t^2 - 15t + C > 0
        \end{equation*}

        Veamos en qué puntos se anula el radicando:
        \begin{align*}
            \dfrac{3}{2}t^2 - 15t + C &= 0 \Longrightarrow t = \dfrac{15\pm\sqrt{225-6C}}{3} = 5\pm\sqrt{25-\dfrac{2C}{3}}
        \end{align*}

        Distinguimos en función de $C$:
        \begin{equation*}
            25 - \dfrac{2C}{3} = 0 \Longrightarrow C = \dfrac{75}{2}
        \end{equation*}
        \begin{itemize}
            \item \ul{$C>\nicefrac{75}{2}$}: En este caso, el último radicando es negativo, luego no se anula el radicando de $x$, y este es siempre positivo. Por tanto, $x(t)>0$ para todo $t\in I$; es decir, $x(t)\in J$ para todo $t\in I$. Además, $x$ es derivable en $I$, luego el intervalo maximal de definición es $I$.
            
            \item \ul{$C=\nicefrac{75}{2}$}: En este caso, el último radicando se anula en $t=5$. Por tanto, $x(t)>0$ para $t\in I\setminus \{5\}$. Por tanto, como el intervalo de definición de la solución debe ser conexo, consideramos las dos siguientes opciones:
            \begin{equation*}
                I_1 = \left]-\infty, 5\right[ \qquad I_2 = \left]5, +\infty\right[
            \end{equation*}

            En ambos casos, como $x(t)\in J$ para todo $t\in I_1$ y todo $t\in I_2$, y $x$ es derivable en $I_1$ y $I_2$, el intervalo maximal de definición es $I_1$ o $I_2$.

            \item \ul{$C<\nicefrac{75}{2}$}: En este caso, el último radicando es positivo, luego se anula en dos puntos, $t_1$ y $t_2$ dados por:
            \begin{equation*}
                t_1 = 5-\sqrt{25-\dfrac{2C}{3}} \qquad t_2 = 5+\sqrt{25-\dfrac{2C}{3}}
            \end{equation*}

            Por tanto, $x(t)>0$ para $t\in I\setminus [t_1,t_2]$. Por tanto, como el intervalo de definición de la solución debe ser conexo, consideramos las dos siguientes opciones:
            \begin{equation*}
                I_1 = \left]-\infty, t_1\right[ \qquad I_2 = \left]t_2, +\infty\right[
            \end{equation*}

            En todos los casos, como $x(t)\in J$ para todo $t\in I_1$ y todo $t\in I_2$, y $x$ es derivable en $I_1$ y $I_2$, el intervalo maximal de definición es $I_1$ o $I_2$.
        \end{itemize}

        % // TODO: Estudiar para J=\bb{R}^-
    \end{itemize}


\end{ejercicio}

\begin{ejercicio}
    En Dinámica de Poblaciones, dos modelos muy conocidos son la ecuación de Verhulst o logística
    \begin{equation*}
        P' = P(\alpha - \beta P)
    \end{equation*}
    y la ecuación de Gompertz
    \begin{equation*}
        P' = P(\alpha - \beta \ln P)
    \end{equation*}
    siendo $P(t)$ la población a tiempo $t$ de una determinada especie y $\alpha, \beta$ parámetros positivos. Calcule en cada caso la solución con condición inicial $P(0) = 100$.
\end{ejercicio}

\begin{ejercicio}
    Nos planteamos resolver la ecuación
    \begin{equation*}
        x' = \cos(t - x)
    \end{equation*}
    Compruebe que el cambio $y = t - x$ nos lleva a una ecuación de variables separadas. Resuelva e invierta el cambio para llegar a una expresión explícita de $x(t)$. Repase el procedimiento por si se ha perdido alguna solución por el camino.
\end{ejercicio}

\begin{ejercicio}
    Experimentalmente, se sabe que la resistencia al aire de un cuerpo en caída libre es proporcional al cuadrado de la velocidad del mismo. Por tanto, si $v(t)$ es la velocidad a tiempo $t$, la ecuación de Newton nos dice que
    \begin{equation*}
        v' + \dfrac{k}{m}v^2 = g,
    \end{equation*}
    donde $m$ es la masa del cuerpo, $g$ es la constante de gravitación universal y $k > 0$ depende de la geometría (aerodinámica) del cuerpo. Si se supone que $v(0) = 0$, calcule la solución explícita y describa el comportamiento a largo plazo.
\end{ejercicio}

\begin{ejercicio}
    Calcule la solución de la ecuación
    \begin{equation*}
        y' = \dfrac{x + y - 3}{x - y - 1}
    \end{equation*}
    que verifica $y(0) = 1$.
\end{ejercicio}

\begin{ejercicio}
    Resuelva los siguientes problemas lineales
    \begin{enumerate}
        \item $x' + 3x = e^{-3t}$, $x(1) = 5$
        \item $x' - \dfrac{x}{t} = \dfrac{1}{1+t^2}$, $x(2) = 0$
        \item $x' = \cosh t \cdot x + \sinh t$, $x(0) = 1$
    \end{enumerate}
\end{ejercicio}

\begin{ejercicio}
    Sean $a, b : \bb{R} \to \bb{R}$ funciones continuas con $a(t) \geq c > 0$ para todo $t$ y
    \begin{equation*}
        \lim_{t \to +\infty} b(t) = 0.
    \end{equation*}
    Demuestre que todas las soluciones de la ecuación $x' = -a(t)x + b(t)$ tienden a cero cuando $t \to +\infty$. (Indicación: regla de L'Hôpital en la fórmula de variación de constantes)
\end{ejercicio}

\begin{ejercicio}
    La ecuación de Bernoulli tiene la forma
    \begin{equation*}
        x' = a(t)x + b(t)x^n,
    \end{equation*}
    donde $a, b : I \to \bb{R}$ son funciones continuas y $n \in \bb{R}$. Compruebe que el cambio de variable $y = x^\alpha$ lleva la ecuación de Bernoulli a una ecuación del mismo tipo, y ajuste el valor de $\alpha$ para que la ecuación obtenida sea lineal ($n = 0$). Usando el cambio anterior, resuelva los problemas de valores iniciales
    \begin{equation*}
        x' = x + t\sqrt{x}, \quad x(0) = 1.
    \end{equation*}
\end{ejercicio}

\begin{ejercicio}
    Se considera la ecuación de Ricatti
    \begin{equation*}
        y' = -\dfrac{1}{x^2} - \dfrac{y}{x} + y^2.
    \end{equation*}
    Encuentre una solución particular de la forma $y(x) = x^\alpha$ con $\alpha \in \bb{R}$. Usando esta solución particular, calcule la solución que cumple $y(1) = 2$ y estudie su intervalo maximal de definición.
\end{ejercicio}

\begin{ejercicio}
    Encuentre una curva $y = y(x)$ que pase por el punto $(1, 2)$ y cumpla la siguiente propiedad: la distancia de cada punto de la curva al origen coincide con la segunda coordenada del punto de corte de la recta tangente y el eje de ordenadas. (C. Sturm, Cours d’Analyse 1859, Vol 2, pag 41).
\end{ejercicio}

\begin{ejercicio}
    Identifique la clase de ecuaciones invariantes por el grupo de transformaciones $s = \lambda t$, $y = \lambda^2 x$, con $\lambda > 0$.
\end{ejercicio}

\begin{ejercicio}
    Resuelva los problemas 42 y 45 (pag. 79) del libro de Nagle-Saff-Sneider.
\end{ejercicio}
    \section{Paso de mensajes}

\begin{ejercicio}\label{ej:rel3_1}
    En un sistema distribuido, 6 procesos clientes necesitan sincronizarse de forma específica para realizar cierta tarea, de forma que dicha tarea sólo podrá ser realizada cuando tres procesos estén preparados para realizarla. Para ello, envían peticiones a un proceso controlador del recurso y esperan respuesta para poder realizar la tarea específica. El proceso controlador se encarga de asegurar la sincronización adecuada. Para ello, recibe y cuenta las peticiones que le llegan de los procesos, las dos primeras no son respondidas y producen la suspensión del proceso que envía la petición (debido a que se bloquea esperando respuesta) pero la tercera petición produce el desbloqueo de los tres procesos pendientes de respuesta. A continuación, una vez desbloqueados los tres procesos que han pedido (al recibir respuesta), inicializa la cuenta y procede cíclicamente de la misma forma sobre otras peticiones. El código de los procesos clientes aparece aquí abajo. Los clientes usan envío asíncrono seguro para realizar su petición, y esperan con una recepción síncrona antes de realizar la tarea:
    \begin{figure}[H]
        \centering
            \begin{minted}{pascal}
                process Cliente[ i : 0..5 ];
                begin
                   while true do begin
                      send(peticion, Controlador);
                      receive(permiso, Controlador);
                      Realiza_tarea_grupal();
                   end
                end
            \end{minted}
        \caption{Código para el Ejercicio~\ref{ej:rel3_1}.}
        \label{fig:cod_1}
    \end{figure}
    Describir en pseudocódigo el comportamiento del proceso controlador, utilizando una orden de espera selectiva que permita implementar la sincronización requerida entre los procesos. Es posible utilizar una sentencia del tipo \verb|select for i=... to ...| para especificar diferentes ramas de una sentencia selectiva que comparten el mismo código dependiente del valor de un índice \verb|i|.\\

    \begin{minted}{pascal}
        process Controlador;
        var contador : integer := 0;
            necesarios : integer := 3;
            esperando : array[0..1] of 0..5;
        begin
           while true do begin
              select
                 for i := 0 to 5
                    when receive(valor, Cliente[i]) do
                       if contador < necesarios-1 then begin
                          esperando[contador] := i;
                          contador := contador + 1;
                       else
                          contador := 0;
                          send(valor, Cliente[i]);
                          send(valor, Cliente[esperando[0]]);
                          send(valor, Cliente[esperando[1]]);
                       end
                    end
              end select
           end do
        end
    \end{minted}
\end{ejercicio}

\begin{ejercicio}\label{ej:rel3_2}
    En un sistema distribuido, 3 procesos productores producen continuamente valores enteros y los envían a un proceso buffer que los almacena temporalmente en un array local de 4 celdas enteras para ir enviándoselos a un proceso consumidor. A su vez, el proceso buffer realiza lo siguiente, sirviendo de forma equitativa al resto de procesos:
    \begin{enumerate}[label=(\alph*)]
        \item Envía enteros al proceso consumidor siempre que su array local tenga al menos dos elementos disponibles.
        \item Acepta envíos de los productores mientras el array no esté lleno, pero no acepta que cualquier productor pueda escribir dos veces consecutivas en el búfer.
    \end{enumerate}
    El código de los procesos productor y consumidor es el siguiente, asumiendo que se usan operaciones síncronas:
    \begin{figure}[H]
        \centering
        \setlength{\columnsep}{1cm}
        \begin{multicols}{2}
            \begin{minted}{pascal}
                process Productor [ i : 0..2 ];
                var dato : integer;
                begin
                   while true do begin
                      dato := Producir();
                      send(dato, Buffer);
                   end
                end
            \end{minted}
            \begin{minted}{pascal}
                process Consumidor;
                begin
                   while true do begin
                      receive(dato, Buffer);
                      Consumir(dato);
                   end
                end
            \end{minted}
        \end{multicols}
        \caption{Código para el Ejercicio~\ref{ej:rel3_2}.}
        \label{fig:cod_2}
    \end{figure}
    Describir en pseudocódigo el comportamiento del proceso \verb|Buffer|, utilizando una orden de espera selectiva que permita implementar la sincronización requerida entre los procesos.\\

    \begin{minted}{pascal}
        process Buffer;
        var buffer : array[0..3] of integer;
            primera_libre, primera_ocupada : integer := 0, 0;
            ocupadas : integer := 0;
            ult_productor : integer := -1;
            dato : integer;
        begin
           while true do begin
              select 
                 when ocupadas >= 2 do
                    dato := buffer[primera_ocupada];
                    primera_ocupada := (primera_ocupada + 1) mod 4;
                    ocupadas := ocupadas - 1;
                    send(dato, Consumidor);
                 end
                 for i := 0 to 2
                    when ocupadas < 4 and i <> ult_productor receive(dato, Productor[i])
                       ult_productor := i;
                       buffer[primera_libre] := dato;
                       primera_libre := (primera_libre + 1) mod 4;
                       ocupadas := ocupadas + 1;
                    end
              end
           end
        end
    \end{minted}
\end{ejercicio}

\begin{ejercicio}\label{ej:rel3_3}
    Suponer un proceso productor y 3 procesos consumidores que comparten un buffer acotado de tamaño \verb|B|. Cada elemento depositado por el proceso productor debe ser retirado por todos los 3 procesos consumidores para ser eliminado del buffer. Cada consumidor retirará los datos del buffer en el mismo orden en el que son depositados, aunque los diferentes consumidores pueden ir retirando los elementos a ritmo diferente unos de otros. Por ejemplo, mientras un consumidor ha retirado los elementos 1, 2 y 3, otro consumidor puede haber retirado solamente el elemento 1. De esta forma, el consumidor más rápido podría retirar hasta B elementos más que el consumidor más lento. Describir en pseudocódigo el comportamiento de un proceso que implemente el buffer de acuerdo con el esquema de interacción descrito usando una construcción de espera selectiva, así como el del proceso productor y de los procesos consumidores. Comenzar identificando qué información es necesario representar, para después resolver las cuestiones de sincronización.
    Una posible implementación del buffer mantendría, para cada proceso consumidor, el puntero de salida y el número de elementos que quedan en el buffer por consumir:

    \begin{figure}[H]
        \centering
    \begin{tikzpicture}
        % Rectángulo principal
        \draw (0, 0) rectangle (8, 1);

        % Divisiones internas del rectángulo
        \foreach \x in {1, 2, 3, 4, 5, 6, 7} {
            \draw (\x, 0) -- (\x, 1);
        }

        % Números dentro de las cajas
        \node at (0.5, 0.5) {1};
        \node at (1.5, 0.5) {2};
        \node at (2.5, 0.5) {3};
        \node at (3.5, 0.5) {4};
        \node at (4.5, 0.5) {5};

        % Tachar el 1
        \draw[thick] (0.2, 0.6) -- (0.8, 0.4);

        % Flecha y etiquetas: out[1]
        \draw[-Stealth] (1.5, -0.6) -- (1.5, -0.1);
        \node[below] at (1.5, -0.6) {\verb|out[1]|};

        % Flecha y etiquetas: out[3]
        \draw[-Stealth] (3.5, -0.6) -- (3.5, -0.1);
        \node[below] at (3.5, -0.6) {\verb|out[3]|};

        % Flecha y etiquetas: out[2]
        \draw[-Stealth] (5.5, -0.6) -- (5.5, -0.1);
        \node[below] at (5.5, -0.6) {\verb|out[2]|};

        % Flecha y etiqueta "in" por encima
        \draw[-Stealth] (5.5, 1.6) -- (5.5, 1.1);
        \node[above] at (5.5, 1.6) {\verb|in|};

     % Texto a la derecha
        \node[right] at (8.5, 1) {\verb|nElems[1] = 4|};
        \node[right] at (8.5, 0.5) {\verb|nElems[2] = 0|};
        \node[right] at (8.5, 0) {\verb|nElems[3] = 2|};
    \end{tikzpicture}        
    \caption{Dibujo para el Ejercicio~\ref{ej:rel3_3}.}
    \label{fig:fig_ej_3}
    \end{figure}

    En primer lugar, describimos los códigos de los procesos productor y consumidores, por ser estos mucho más fáciles que el del proceso intermedio que usaremos para comunicar ambos tipos de procesos, al no requerir estos de sincronización ninguna. Tanto productor como consumidores mandan mensajes al Buffer, y los cosumidores esperan una respuesta del mismo.
    \begin{figure}[H]
        \centering
    \setlength{\columnsep}{1cm}
    \begin{multicols}{2}
        \begin{minted}{pascal}
           process Productor;
           var dato : integer;
           begin
              dato := Producir();
              send(dato, Buffer);
           end
        \end{minted}
        \begin{minted}{pascal}
            process Consumidor[ i : 0..2 ];
            var dato : integer;
            begin
               send(peticion, Buffer);
               receive(dato, Buffer);
               Consumir(dato);
            end
        \end{minted}
    \end{multicols}
    \end{figure}
    Ahora, desarrollamos el código del proceso Buffer, donde \verb|out[i]| indica la siguiente posición a leer del consumidor \verb|i|-ésimo y \verb|nElems[i]| indica la cantidad de datos que quedan por leer al consumidor \verb|i|-ésimo. 
    \begin{minted}{pascal}
        process Buffer;
        var buffer : array[0..B-1] of integer;
            ocupados : integer := 0;
            in : 0..B-1 := 0;
            out : array[0..2] of 0..B-1 := (0, 0, 0);
            nElems : array[0..2] of 0..B-1 := (0, 0, 0);
            dato : integer;
        begin
           while true do begin
              select 
                 when ocupados < B receive(dato, Productor) do
                    buffer[in] := dato;
                    in := (in + 1) mod B;
                    ocupados := ocupados + 1;
                    
                    for i:= 0 to 2 do
                       nElems[i] := nElems[i] + 1;
                 end
                 for i := 0 to 2
                    when nElems[i] > 0 receive(peticion, Consumidor[i]) do
                       dato := buffer[out[i]];
                       out[i] := out[i] + 1 mod B;

                       { Índices de los otros consumidores }
                       j := (i+1) mod 3;
                       k := (j+1) mod 3;

                       { Si nElems es el mayor, era el último en consumir }
                       if nElems[i] > nElems[j] and nElems[i] > nElems[k] then
                          ocupados := ocupados - 1;
                       end

                       nElems[i] := nElems[i] - 1;
                       send(dato, Consumidor[i]);
                    end
              end
           end
        end
    \end{minted}
\end{ejercicio}

\begin{ejercicio}\label{ej:rel3_4}
    Una tribu de antropófagos comparte una olla en la que caben \verb|M| misioneros. Cuando algún salvaje quiere comer, se sirve directamente de la olla, a no ser que ésta esté vacía. Si la olla está vacía, el salvaje despertará al cocinero y esperará a que éste haya rellenado la olla con otros \verb|M| misioneros.
    \begin{figure}[H]
        \centering
        \setlength{\columnsep}{1cm}
        \begin{multicols}{2}
            \begin{minted}{pascal}
                process Salvaje[ i : 0..2 ];
                var peticion : integer := ... ;
                begin
                   while true do begin
                      { esperar a servirse un misionero }
                      ...
                      s_send(peticion, Olla);
                      { comer }
                      Comer();
                   end
                end
            \end{minted}
            \begin{minted}{pascal}
                process Cocinero;
                begin
                   while true do begin
                      { dormir esperando solicitud para rellenar }
                      ...
                      {confirmar que se ha rellenado la olla}
                      ...
                   end
                end
            \end{minted}
        \end{multicols}
        \caption{Código para el Ejercicio~\ref{ej:rel3_4}.}
        \label{fig:cod_4}
    \end{figure}
    Implementar los procesos salvajes y cocinero usando paso de mensajes, usando un proceso olla que incluye una construcción de espera selectiva que sirve peticiones de los salvajes y el cocinero para mantener la sincronización requerida, teniendo en cuenta que:
    \begin{itemize}
        \item La solución no debe producir interbloqueo.
        \item Los salvajes podrán comer siempre que haya comida en la olla.
        \item Solamente se despertará al cocinero cuando la olla esté vacía.
    \end{itemize}
    Mostramos primero los códigos para los procesos salvajes y cocinero:
    \begin{figure}[H]
        \centering
        \setlength{\columnsep}{1cm}
        \begin{multicols}{2}
            \begin{minted}{pascal}
                process Salvaje[ i : 0..2 ];
                begin
                   while true do begin
                      s_send(peticion, Olla);
                      Comer();
                   end
                end
            \end{minted}
            \begin{minted}{pascal}
                process Cocinero;
                begin
                   while true do begin
                      receive(peticion, Olla);
                      send(rellenar, Olla);
                   end
                end
            \end{minted}
        \end{multicols}
    \end{figure}
    A continuación, mostramos el código del proceso Olla:
    \begin{minted}{pascal}
        process Olla;
        var misioneros : integer := 0;
        begin
           while true do begin
              select 
                 when misioneros = 0 do
                    send(valor, Cocinero);
                    receive(valor, Cocinero);
                    misioneros := M;
                 end
                 for i := 0 to 2 
                    when misioneros > 0 receive(valor, Salvaje[i]) do
                       misioneros := misioneros - 1;
                    end
              end
           end
        end
    \end{minted}
\end{ejercicio}

\begin{ejercicio}\label{ej:rel3_5}
    Considerar un conjunto de \verb|N| procesos, \verb|P[i]|, ($i = 0, \ldots, N-1$) que se pasan mensajes cada uno al siguiente (y el primero al último), en forma de anillo. Cada proceso tiene un valor local almacenado en su variable local \verb|mi_valor|. Deseamos calcular la suma de los valores locales almacenados por los procesos de acuerdo con el algoritmo que se expone a continuación.
    \begin{figure}[H]
        \centering

        \setlength{\columnsep}{1cm}
        \begin{multicols}{2}
            
        \begin{tikzpicture}[align=center]
            % Paso 1
            \node[draw, rectangle] (a1) at (0, 0) {mi\_valor = 0\\ suma = 0};
            \node[draw, rectangle, right=of a1] (b1) {mi\_valor = 1\\ suma = 1};
            \node[draw, rectangle, below=of b1] (c1) {mi\_valor = 2\\ suma = 2};
            \node[draw, rectangle, left=of c1] (d1) {mi\_valor = 3\\ suma = 3};

            \draw[-Stealth] (a1) -- (b1) node[midway, above] {0};
            \draw[-Stealth] (b1) -- (c1) node[midway, right] {1};
            \draw[-Stealth] (c1) -- (d1) node[midway, below] {2};
            \draw[-Stealth] (d1) -- (a1) node[midway, left] {3};

            \node[above=of a1] {Paso 1};

            % Paso 3
            \node[draw, rectangle, below=2cm of d1] (a3) {suma = 5};
            \node[draw, rectangle, right=of a3] (b3) {suma = 4};
            \node[draw, rectangle, below=of b3] (c3) {suma = 3};
            \node[draw, rectangle, left=of c3] (d3) {suma = 6};

            \draw[-Stealth] (a3) -- (b3) node[midway, above] {2};
            \draw[-Stealth] (b3) -- (c3) node[midway, right] {3};
            \draw[-Stealth] (c3) -- (d3) node[midway, below] {0};
            \draw[-Stealth] (d3) -- (a3) node[midway, left] {1};

            \node[above=of a3] {Paso 3};
        \end{tikzpicture}

        \begin{tikzpicture}[align=center]

            % Paso 2
            \node[draw, rectangle] (a2) {suma = 3};
            \node[draw, rectangle, right=of a2] (b2) {suma = 1};
            \node[draw, rectangle, below=of b2] (c2) {suma = 3};
            \node[draw, rectangle, left=of c2] (d2) {suma = 5};

            \draw[-Stealth] (a2) -- (b2) node[midway, above] {3};
            \draw[-Stealth] (b2) -- (c2) node[midway, right] {0};
            \draw[-Stealth] (c2) -- (d2) node[midway, below] {1};
            \draw[-Stealth] (d2) -- (a2) node[midway, left] {2};

            \node[above=of a2] {Paso 2};

            % Paso 4
            \node[draw, rectangle, below=3cm of d2] (a4) {suma = 6};
            \node[draw, rectangle, right=of a4] (b4) {suma = 6};
            \node[draw, rectangle, below=of b4] (c4) {suma = 6};
            \node[draw, rectangle, left=of c4] (d4) {suma = 6};

            \node[above=of a4] {Paso 4};
        \end{tikzpicture}
        \end{multicols}
        \caption{Dibujo para el Ejercicio~\ref{ej:rel3_5}.}
        \label{fig:fig_ej_5}
    \end{figure}
    Los procesos realizan una serie de iteraciones para hacer circular sus valores locales por el anillo. En la primera iteración, cada proceso envía su valor local al siguiente proceso del anillo, al mismo tiempo que recibe del proceso anterior el valor local de éste. A continuación acumula la suma de su valor local y el recibido desde el proceso anterior. En las siguientes iteraciones, cada proceso envía al siguiente proceso siguiente el valor recibido en la anterior iteración, al mismo tiempo que recibe del proceso anterior un nuevo valor. Después acumula la suma. Tras un total de \verb|N - 1| iteraciones, cada proceso conocerá la suma de todos los valores locales de los procesos. Dar una descripción en pseudocódigo de los procesos siguiendo un estilo SPMD y usando operaciones de envío y recepción síncronas:
    \begin{minted}{pascal}
        process P[ i : 0..N-1 ];
        var mi_valor : integer := ...; {valor aleatorio, igual a i en la figura}
            suma : integer := mi_valor;
        begin
           for j := 0 to N-1 do begin
              ...
           end
        end
    \end{minted}
    La solución la podemos encontrar en el siguiente código, donde debemos tener cuidado de que no todos los procesos primero envíen y luego reciban, ya que esto puede dar lugar a una situación de interbloqueo por usar envíos y recepciones síncronas. Para ello, obligamos a que al menos un proceso primero reciba y luego envíe, o viceversa:
    \begin{minted}{pascal}
        process P[ i : 0..N-1 ];
        var mi_valor : integer := ...; {valor aleatorio, igual a i en la figura}
            suma : integer := mi_valor;
            siguiente : 0..N-1 := (i+1) mod N;
            anterior : 0..N-1 := (i-1) mod N;
            recibido : integer;
        begin
           for j := 0 to N-1 do begin
              if i == 0 then begin
                 send(mi_valor, siguiente);
                 recv(recibido, anterior);
              else
                 recv(recibido, anterior);
                 send(mi_valor, siguiente);
              end
              mi_valor := recibido;
              suma += recibido;
           end
        end
    \end{minted}
\end{ejercicio}

\begin{ejercicio}\label{ej:rel3_6}
    Considerar un estanco en el que hay tres fumadores y un estanquero. Cada fumador continuamente lía un cigarro y se lo fuma. Para liar un cigarro, el fumador necesita tres ingredientes: tabaco, papel y cerillas. Uno de los fumadores tiene solamente papel, otro tiene solamente tabaco, y el otro tiene solamente cerillas. El estanquero tiene una cantidad infinita de los tres ingredientes.
    \begin{itemize}
        \item El estanquero coloca aleatoriamente dos ingredientes diferentes de los tres que se necesitan para hacer un cigarro, desbloquea al fumador que tiene el tercer ingrediente y después se bloquea. El fumador seleccionado, se puede obtener fácilmente mediante una función \verb|genera_ingredientes| que devuelve el índice (0, 1, ó 2) del fumador escogido.
        \item El fumador desbloqueado toma los dos ingredientes del mostrador, desbloqueando al estanquero, lía un cigarro y fuma durante un tiempo.
        \item El estanquero, una vez desbloqueado, vuelve a poner dos ingredientes aleatorios en el mostrador, y se repite el ciclo.
    \end{itemize}
    Describir una solución distribuida que use envío asíncrono seguro y recepción síncrona, para este problema usando un proceso Estanquero y tres procesos fumadores \verb|Fumador(i)| (con $i=0, 1, 2$).

    \begin{figure}[H]
        \centering
            \begin{minted}{pascal}
                process Estanquero;
                var ing : array[0..1] of ingredientes;
                begin
                   while true do begin
                      { Genera dos ingredientes distintos }
                      ing := generaIngredientes();

                      select
                         for i:=0 to 2
                            when genera_ingredientes(ing) do
                               send(ing, Fumador[i]);
                               receive(confirmacion, Fumador[i]);
                            end
                      end select
                   end
                end
            \end{minted}
            \begin{minted}{pascal}
                process Fumador[ i : 0..2 ];
                var ingredientes : array[0..1] of ingredientes;
                begin
                   while true do begin
                      recv(ingredientes, Estanquero);
                      send(confirmacion, Estanquero);
                      fumar(ingredientes);
                   end
                end
            \end{minted}
        \caption{Código para el Ejercicio~\ref{ej:rel3_6}.}
        \label{fig:cod_6}
    \end{figure}
\end{ejercicio}

\begin{ejercicio}\label{ej:rel3_7}
    En un sistema distribuido, un gran número de procesos clientes usa frecuentemente un determinado recurso y se desea que puedan usarlo simultáneamente el máximo número de procesos. Para ello, los clientes envían peticiones a un proceso controlador para usar el recurso y esperan respuesta para poder usarlo (véase el código de los procesos clientes). Cuando un cliente termina de usar el recurso, envía una solicitud para dejar de usarlo y espera respuesta del Controlador. El proceso controlador se encarga de asegurar la sincronización adecuada imponiendo una única restricción por razones supersticiosas: nunca habrá 13 procesos exactamente usando el recurso al mismo tiempo.
    \begin{figure}[H]
        \centering
            \begin{minted}{pascal}
                process Cli[ i : 0..n ];
                var pet_usar : integer := +1;
                    pet_liberar : integer := -1;
                    permiso : integer := ...;
                begin
                   while true do begin
                      send(pet_usar, Controlador);
                      receive(permiso, Controlador);
                      Usar_recurso();
                      send(pet_liberar, Controlador);
                      receive(permiso, Controlador);
                   end
                end
            \end{minted}
        \caption{Código para el Ejercicio~\ref{ej:rel3_7}.}
        \label{fig:cod_7}
    \end{figure}
    Describir en pseudocódigo el comportamiento del proceso controlador, utilizando una orden de espera selectiva que permita implementar la sincronización requerida entre los procesos. Es posible utilizar una sentencia del tipo \verb|select for i=... to ...| para especificar diferentes ramas de una sentencia selectiva que comparten el mismo código dependiente del valor de un índice \verb|i|.\\

    La solución que planteamos es la siguiente, donde tenemos en cuenta las peticiones de obtención y liberación del recurso que llevan al recurso a ser usado por 13 procesos al mismo tiempo. En dicho caso, guardamos la petición hasta que haya cualquiera otra petición (que puede ser del mismo tipo o distinto).
    \begin{minted}{pascal}
        process Controlador
        var contador : integer := 0;
            pendiente : integer := 0;
            id_pendiente : integer;
            peticion : integer;
            permiso : integer := 100;
        begin
           while true do begin
              select 
                 for i:= 0 to n
                    when receive(peticion, Cli[i]) do
                       { Si no nos sirve el estado al que llega }
                       if contador + pendiente + peticion = 13 then begin
                          { Sabemos que pendiente = 0 }
                          pendiente := peticion;
                          id_pendiente := i;
                       else { Si no, se procesa la petición }
                          contador := contador + pendiente + peticion;
                          send(permiso, Cli[i]);

                          { Si había una pendiente también se acepta }
                          if pendiente <> 0 then begin
                             send(permiso, Cli[id_pendiente]);
                             pendiente := 0;
                          end
                       end
                    end
              end select
           end
        end
    \end{minted}
    
\end{ejercicio}

\begin{ejercicio}\label{ej:rel3_8}
    En un sistema distribuido, tres procesos \verb|Productor| se comunican con un proceso \verb|Impresor| que se encarga de ir imprimiendo en pantalla una cadena con los datos generados por los procesos productores. Cada proceso productor (\verb|Productor[i]|, con $i=0,1,2$) genera continuamente el correspondiente entero \verb|i|, y lo envía al proceso Impresor.

    El proceso Impresor se encarga de ir recibiendo los datos generados por los productores y los imprime por pantalla (usando el procedimiento \verb|imprime(entero)|) generando una cadena de dígitos en la salida. No obstante, los procesos se han de sincronizar adecuadamente para que la impresión por pantalla cumpla las siguientes restricciones:
    \begin{itemize}
        \item Los dígitos 0 y 1 deben aceptarse por el impresor de forma alterna. Es decir, si se acepta un 0 no podrá volver a aceptarse un 0 hasta que se haya aceptado un 1, y viceversa, si se acepta un 1 no podrá volver a aceptarse un 1 hasta que se haya aceptado un 0.
        \item El número total de dígitos 0 o 1 aceptados en un instante no puede superar el doble de número de digitos 2 ya aceptados en dicho instante.
    \end{itemize}
    Cuando un productor envía un digito que no se puede aceptar por el imprersor, el productor quedará bloqueado esperando completar el \verb|s_send|. El pseudocódigo de los procesos productores (\verb|Productor|) se muestra a continuación , asumiendo que se usan operaciones bloqueantes no buferizadas (síncronas):
    \begin{figure}[H]
        \centering
            \begin{minted}{pascal}
                process Productor[ i : 0,1,2 ];
                begin
                   while true do begin
                      s_send(i, Impresor);
                   end
                end
            \end{minted}
        \caption{Código para el Ejercicio~\ref{ej:rel3_8}.}
        \label{fig:cod_8}
    \end{figure}
    Escribir en pseudocódigo el código del proceso \verb|Impresor|, utilizando para ello un bucle infinito con una orden de espera selectiva \verb|select| que permita implementar la sincronización requerida entre los procesos, según el esquema anterior.\\

    El código solicitado es el siguiente, donde \verb|ult0| indica:
    \begin{itemize}
        \item \verb|true| si el último dígito 0 o 1 recibido fue un 0.
        \item \verb|false| si el último dígito 0 o 1 recibido fue un 1.
    \end{itemize}
    Además, notemos que el ejercicio no impone restricciones sobre cuando se pueden recibir dígitos 2.
    \begin{minted}{pascal}
        process Impresor
        var cant0, cant1, cant2 : integer := 0, 0, 0;
            ult0 : boolean := true;
            n : 0..2;
        begin
           while true do begin
              select
                 when (cant0 < 2*cant2-1 and not ult0) receive(n, Productor[0]) do
                    cant0 := cant0 + 1;
                    ult0 := true;
                 end

                 when (cant1 < 2*cant2-1 and ult0) receive(n, Productor[1]) do
                    cant1 := cant1 + 1;
                    ult0 := false;
                 end

                 when receive(n, Productor[2]) do
                    cant2 := cant2 + 1;
                 end
              end

              imprime(n);
           end
        end
    \end{minted}
\end{ejercicio}

\begin{ejercicio}\label{ej:rel3_9}
   En un sistema distribuido hay un vector de \verb|n| procesos iguales que envían con \verb|send| (en un bucle infinito) valores enteros a un proceso receptor, que los imprime. Si en algún momento no hay ningún mensaje pendiente de recibir en el receptor, este proceso debe de imprimir ``no hay mensajes, duermo''; después de bloquearse durante 10 segundos (con \verb|sleep_for(10)|), antes de volver a comprobar si hay mensajes (esto podría hacerse para ahorrar energía, ya que el procesamiento de mensajes se hace en ráfagas separadas por 10 segundos). Este problema no se puede solucionar usando \verb|receive| o \verb|i_receive|. Indica a que se debe esto. Sin embargo, sí se puede hacer con \verb|select|. Diseña una solución a este problema con \verb|select|:
    \begin{figure}[H]
       \centering
           \begin{minted}{pascal}
               process Emisor[ i : 1..n ];
               var dato : integer;
               begin
                  while true do begin
                     dato := Producir();
                     send(dato, Receptor);
                  end
               end
           \end{minted}
       \caption{Código para el Ejercicio~\ref{ej:rel3_9}.}
       \label{fig:cod_9}
   \end{figure}
   El problema no puede resolverse con instrucciones \verb|receive| o \verb|i_receive| porque:
   \begin{itemize}
       \item En el caso de la instrucción \verb|receive|, si no hay mensajes pendientes, el proceso se bloquearía hasta recibir el primero, pero este no es el comportamiento deseado.
       \item En el caso de la instrucción \verb|i_receive|, comenzaría instantáneamente la recepción del mensaje, pero el proceso volvería inmediatamente antes de recibirlo, por lo que no sabríamos si hay o no un mensaje pendiente.
   \end{itemize}
   Sin embargo, podemos hacer uso de las sentencias \verb|else| de las instrucciones \verb|select|, de forma que una instrucción se ejecute cuando todas las guardas no son ejecutables:
   \begin{minted}[escapeinside=\#\#]{pascal}
       process Receptor
       var dato : integer;
       begin
          while true do begin
             select
                for i:= 1 to n
                   when receive(dato, Emisor[i]) do
                      imprime(dato);
                   end
                else begin
                   imprime(#"#No hay mensajes, duermo#"#);
                   sleep_for(10);
                end
             end select
          end
       end
   \end{minted}
\end{ejercicio}

\begin{ejercicio}\label{ej:rel3_10}
    En un sistema tenemos \verb|N| procesos emisores que envían de forma segura un único mensaje cada uno de ellos a un proceso receptor, mensaje que contiene un entero con el número de proceso emisor. El proceso receptor debe de imprimir el número del proceso emisor que inició el envío en primer lugar. Dicho emisor debe terminar, y el resto quedarse bloqueados:
    \begin{figure}[H]
        \centering
        \begin{minted}[escapeinside=\#\#]{pascal}
            process Emisor[ i : 1..N ];
            begin
               s_send(i, Receptor);
            end

            process Receptor;
            var ganador : integer;
            begin
               { calcular ganador }
               ...
               print #"#El primer envio #lo# ha realizado: #"#, ganador;
            end
        \end{minted}
        \caption{Código para el Ejercicio~\ref{ej:rel3_10}.}
        \label{fig:cod_10}
    \end{figure}
    Para cada uno de los siguientes casos, describir razonadamente si es posible diseñar una solución a este problema o no lo es. En caso afirmativo, escribe una posible solución:
    \begin{enumerate}[label=(\alph*)]
        \item el proceso receptor usa exclusivamente recepción mediante una o varias llamadas a \verb|receive|.
        \item el proceso receptor usa exclusivamente recepción mediante una o varias llamadas a \verb|i_receive|.
        \item el proceso receptor usa exclusivamente recepción mediante una o varias instrucciones \verb|select|.
    \end{enumerate}
    Distinguimos casos:
    \begin{enumerate}[label=(\alph*)]
        \item No es posible, ya que si estamos pensando en usar una única instrucción \verb|receive| de forma que el proceso ganador sea aquel que realice una cita con esta instrucción, entonces no nos quedaríamos con el proceso emisor que inició el envío en primer lugar, sino con el emisor del mensaje que primero llegó en el receptor.
        \item Tampoco es posible, porque ahora mantenemos el problema anterior pero además el orden en el que se reciben los mensajes en el receptor no tiene por qué coincidir con el orden con el que se realizan las instrucciones \verb|i_receive|.
        \item Sí que es posible, ya que en caso de que haya más de un mensaje iniciado y preparado para ser recibido, la orden \verb|select| escogerá aquel mensaje cuyo emisor comenzó antes la operación de envío:
            \begin{minted}[escapeinside=\#\#]{pascal}
                process Receptor
                var ganador : integer;
                begin
                   select
                      for i := 1 to N; when receive(ganador, Emisor[i]) do
                         null;
                      end
                   end select
                   print #"#El primer envio #lo# ha realizado: #"#, ganador;
                end
            \end{minted}
    \end{enumerate}
\end{ejercicio}

\begin{ejercicio}\label{ej:rel3_11}
    Supongamos que tenemos \verb|N| procesos concurrentes semejantes.
    Cada proceso produce \verb|N-1| caracteres (con \verb|N-1| llamadas a la función \verb|ProduceCaracter|) y envía cada carácter a los otros \verb|N-1| procesos. Además, cada proceso debe imprimir todos los caracteres recibidos de los otros procesos (el orden en el que se escriben es indiferente).
    \begin{itemize}
        \item Describe razonadamente si es o no posible hacer esto usando exclusivamente \verb|s_send| para los envíos. En caso afirmativo, escribe una solución.
        \item Escribe una solución usando \verb|send| y \verb|receive|.
    \end{itemize}
    Distinguimos casos:
    \begin{itemize}
        \item En el primer caso, es imposible implementar esta funcionalidad usando operaciones \verb|s_send| de envío síncrono, ya que todos los procesos ejecutarían dicha instrucción, llevando a un interbloqueo de todos los procesos, situación que se pone de manifiesto de forma simple si simplemente consideramos dos procesos:
            \begin{figure}[H]
                \setlength{\columnsep}{1cm}
                \begin{multicols}{2}
                    \begin{minted}{pascal}
                       process P1;    
                       var n1, n2 : integer := 1;
                       begin
                          send(n1, P2);
                          receive(n2, P2);
                       end
                    \end{minted}
                    \begin{minted}{pascal}
                       process P2;    
                       var n1, n2 : integer := 2;
                       begin
                          send(n2, P1);
                          receive(n1, P1);
                       end
                    \end{minted}
                \end{multicols}
                \caption{Situación típica de interbloqueo.}
            \end{figure}
        \item Con instrucciones \verb|send| y \verb|receive| sí que se puede resolver de forma sencilla:
            \begin{minted}{pascal}
                process P[ i : 0..N ];
                var n : char;
                begin
                   { Enviamos todos los caracteres }
                   for j := 0 to N do
                      if i <> j then begin
                         n := Producir();
                         send(n, P[j]);
                      end
                   end do
                   { Recibimos todos los caracteres }
                   for j := 0 to N do
                      if i <> j then begin
                         receive(n, P[j]);
                         print(n);
                      end
                   end do
                end
            \end{minted}
    \end{itemize}
\end{ejercicio}

\begin{ejercicio}\label{ej:rel3_12}
    Escribe una nueva solución al Ejercicio~\ref{ej:rel3_11} en la cual se garantize que el orden en el que se imprimen los caracteres es el mismo orden en el que se inician los envíos de dichos caracteres (pista: usa \verb|select| para recibir).\\

    Como bien indica la pista, basta con usar una instrucción \verb|select| para realizar la recepción de los caracteres:
            \begin{minted}{pascal}
                process P[ i : 0..N ];
                var n : char;
                begin
                   { Enviamos todos los caracteres }
                   for j := 0 to N do
                      if i <> j then begin
                         n := Producir();
                         send(n, P[j]);
                      end
                   end do

                   { Recibimos todos los caracteres }
                   for j := 1 to N do
                      select
                         for k := 0 to N
                            when k <> i receive(n, P[k])
                               print(n);
                            end
                      end select
                   end do
            \end{minted}
\end{ejercicio}

\begin{ejercicio}\label{ej:rel3_13}
    Supongamos de nuevo el problema anterior en el cual todos los procesos envían a todos. Ahora cada item de datos a producir y transmitir es un bloque de bytes con muchos valores (por ejemplo, es una imagen que puede tener varios megabytes de tamaño). Se dispone del tipo de datos \verb|TipoBloque| para ello, y el procedimiento \verb|ProducirBloque|, de forma que si b es una variable de tipo \verb|TipoBloque|, entonces la llamada a \verb|ProducirBloque(b)| produce y escribe una secuencia de bytes en \verb|b|. En lugar de imprimir los datos, se deben consumir con una llamada a \verb|ConsumirBloque(b)|.

    Cada proceso se ejecuta en un ordenador, y se garantiza que hay la suficiente memoria en ese ordenador como para contener simultáneamente, al menos, hasta N bloques. Sin embargo, el sistema de paso de mensajes (SPM) podría no tener memoria suficiente como para contener los ${(N-1)}^{2}$ mensajes en tránsito simultáneos que podría llegar a haber en un momento dado con la solución anterior.

    En estas condiciones, si el SPM agota la memoria, debe retrasar los \verb|send| dejando bloqueados los procesos y, en esas circunstancias, se podría producir interbloqueo. Para evitarlo, se pueden usar operaciones inseguras de envío, \verb|i_send|. Escribe dicha solución, usando como orden de recepción el mismo que en el problema anterior.\\

    \noindent
    Para resolver el problema, utilizaremos un array de $N-1$ bloques a enviar (como vamos a usar la operación \verb|i_send| es un envío inseguro, luego optamos por no modificar los los bloques tras enviarlos) y de un bloque para recibir. Además, antes de terminar un proceso tendremos que esperar a que este proceso haya terminado el envío de todos sus bloques:
    \begin{minted}{pascal}
        process P[ i : 0..N ];
        var bloque : array[0..N] of TipoBloque;
            estado : array[0..N] of Estado;
        begin
           { 1. Realizamos todos los envíos }
           for j := 0 to N do
              if i <> j then begin
                 ProducirBloque(bloque[j]);
                 i_send(bloque[j], P[j], estado[j]);
              end
           end

           { 2. Procesamos las recepciones }
           for j := 0 to N
              if i <> j then begin
                 receive(bloque[i], P[j]);
                 ConsumirBloque(bloque[i]);
              end
           end

           { 3. Esperamos en caso de que no se hayan realizado }
           { todos los envíos antes de terminar }
           for j := 0 to N
              if i <> j then begin
                 wait_send(estado[j]);
              end
           end
        end
    \end{minted}
\end{ejercicio}

\begin{ejercicio}\label{ej:rel3_14}
    En los tres problemas anteriores, cada proceso va esperando a recibir un item de datos de cada uno de los otros procesos, consume dicho item, y después pasa a recibir del siguiente emisor (en distintos órdenes). Esto implica que un envío ya iniciado, pero pendiente, no puede completarse hasta que el receptor no haya consumido los anteriores bloques, es decir, se podría estar consumiendo mucha memoria en el SPM por mensajes en tránsito pendientes cuya recepción se ve retrasada. Escribe una solución en la cual cada proceso inicia sus envíos y recepciones y después espera a que se completen todas las recepciones antes de iniciar el primer consumo de un bloque recibido. De esta forma todos los mensajes pueden transferirse potencialmente de forma simultánea. Se debe intentar que la transimisión y las producción de bloques sean lo más simultáneas posible. Suponer que cada proceso puede almacenar como mínimo $2\cdot N$ bloques en su memoria local, y que el orden de recepción o de consumo de los bloques es indiferente.\\

    La solución a este último problema es similar a la del Ejercicio~\ref{ej:rel3_13}, pero en este caso debemos usar la instrucción \verb|i_receive| para recibir, así como un array entero de bloques para realizar dicha recepción. El código sería el siguiente:
    \begin{minted}{pascal}
        process P[ i : 0..N ];
        var bloque_env : array[0..N] of TipoBloque;
            bloque_rec : array[0..N] of TipoBloque;
            estado_env : array[0..N] of Estado;
            estado_rec : array[0..N] of Estado;
        begin
           { Inicializamos las recepciones }
           for j := 0 to N do
              if j <> i then begin
                 i_receive(bloque_rec[j], P[j], estado_rec[j]);
              end
           end

           { Inicializamos los envíos }
           for j := 0 to N do
              if j <> i then begin
                 ProducirBloque(j);
                 i_send(bloque_env[j], P[j], estado_env[j]);
              end
           end

           { Esperar a que terminen todas las recepciones }
           for j := 0 to N do
              if j <> i then begin
                 wait_recv(estado_rec[j]);
              end
           end

           { Procesar todos los bloques }
           for j := 0 to N do
              if j <> i then begin
                 ConsumirBloque(bloque_recv[j]);
              end
           end

           { Esperar a que terminen todos los envíos antes de terminar el proceso }
           for j := 0 to N do
              if j <> i then begin
                 wait_send(estado_env[j]);
              end
           end
        end
    \end{minted}
\end{ejercicio}
\end{document}
