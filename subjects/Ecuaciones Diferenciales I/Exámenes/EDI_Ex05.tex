\documentclass[12pt]{article}

% Idioma y codificación
\usepackage[spanish, es-tabla]{babel}       %es-tabla para que se titule "Tabla"
\usepackage[utf8]{inputenc}

% Márgenes
\usepackage[a4paper,top=3cm,bottom=2.5cm,left=3cm,right=3cm]{geometry}

% Comentarios de bloque
\usepackage{verbatim}

% Paquetes de links
\usepackage[hidelinks]{hyperref}    % Permite enlaces
\usepackage{url}                    % redirecciona a la web

% Más opciones para enumeraciones
\usepackage{enumitem}

% Personalizar la portada
\usepackage{titling}

% Paquetes de tablas
\usepackage{multirow}


%------------------------------------------------------------------------

%Paquetes de figuras
\usepackage{caption}
\usepackage{subcaption} % Figuras al lado de otras
\usepackage{float}      % Poner figuras en el sitio indicado H.


% Paquetes de imágenes
\usepackage{graphicx}       % Paquete para añadir imágenes
\usepackage{transparent}    % Para manejar la opacidad de las figuras

% Paquete para usar colores
\usepackage[dvipsnames]{xcolor}
\usepackage{pagecolor}      % Para cambiar el color de la página

% Habilita tamaños de fuente mayores
\usepackage{fix-cm}

% Para los gráficos
\usepackage{tikz}

% Para poder situar los nodos en los grafos
\usetikzlibrary{positioning}


%------------------------------------------------------------------------

% Paquetes de matemáticas
\usepackage{mathtools, amsfonts, amssymb, mathrsfs}
\usepackage[makeroom]{cancel}     % Simplificar tachando
\usepackage{polynom}    % Divisiones y Ruffini
\usepackage{units} % Para poner fracciones diagonales con \nicefrac

\usepackage{pgfplots}   %Representar funciones
\pgfplotsset{compat=1.18}  % Versión 1.18

\usepackage{tikz-cd}    % Para usar diagramas de composiciones
\usetikzlibrary{calc}   % Para usar cálculo de coordenadas en tikz

%Definición de teoremas, etc.
\usepackage{amsthm}
%\swapnumbers   % Intercambia la posición del texto y de la numeración

\theoremstyle{plain}

\makeatletter
\@ifclassloaded{article}{
  \newtheorem{teo}{Teorema}[section]
}{
  \newtheorem{teo}{Teorema}[chapter]  % Se resetea en cada chapter
}
\makeatother

\newtheorem{coro}{Corolario}[teo]           % Se resetea en cada teorema
\newtheorem{prop}[teo]{Proposición}         % Usa el mismo contador que teorema
\newtheorem{lema}[teo]{Lema}                % Usa el mismo contador que teorema

\theoremstyle{remark}
\newtheorem*{observacion}{Observación}

\theoremstyle{definition}

\makeatletter
\@ifclassloaded{article}{
  \newtheorem{definicion}{Definición} [section]     % Se resetea en cada chapter
}{
  \newtheorem{definicion}{Definición} [chapter]     % Se resetea en cada chapter
}
\makeatother

\newtheorem*{notacion}{Notación}
\newtheorem*{ejemplo}{Ejemplo}
\newtheorem*{ejercicio*}{Ejercicio}             % No numerado
\newtheorem{ejercicio}{Ejercicio} [section]     % Se resetea en cada section


% Modificar el formato de la numeración del teorema "ejercicio"
\renewcommand{\theejercicio}{%
  \ifnum\value{section}=0 % Si no se ha iniciado ninguna sección
    \arabic{ejercicio}% Solo mostrar el número de ejercicio
  \else
    \thesection.\arabic{ejercicio}% Mostrar número de sección y número de ejercicio
  \fi
}


% \renewcommand\qedsymbol{$\blacksquare$}         % Cambiar símbolo QED
%------------------------------------------------------------------------

% Paquetes para encabezados
\usepackage{fancyhdr}
\pagestyle{fancy}
\fancyhf{}

\newcommand{\helv}{ % Modificación tamaño de letra
\fontfamily{}\fontsize{12}{12}\selectfont}
\setlength{\headheight}{15pt} % Amplía el tamaño del índice


%\usepackage{lastpage}   % Referenciar última pag   \pageref{LastPage}
\fancyfoot[C]{\thepage}

%------------------------------------------------------------------------

% Conseguir que no ponga "Capítulo 1". Sino solo "1."
\makeatletter
\@ifclassloaded{book}{
  \renewcommand{\chaptermark}[1]{\markboth{\thechapter.\ #1}{}} % En el encabezado
    
  \renewcommand{\@makechapterhead}[1]{%
  \vspace*{50\p@}%
  {\parindent \z@ \raggedright \normalfont
    \ifnum \c@secnumdepth >\m@ne
      \huge\bfseries \thechapter.\hspace{1em}\ignorespaces
    \fi
    \interlinepenalty\@M
    \Huge \bfseries #1\par\nobreak
    \vskip 40\p@
  }}
}
\makeatother

%------------------------------------------------------------------------
% Paquetes de cógido
\usepackage{minted}
\renewcommand\listingscaption{Código fuente}

\usepackage{fancyvrb}
% Personaliza el tamaño de los números de línea
\renewcommand{\theFancyVerbLine}{\small\arabic{FancyVerbLine}}

% Estilo para C++
\newminted{cpp}{
    frame=lines,
    framesep=2mm,
    baselinestretch=1.2,
    linenos,
    escapeinside=||
}

% para minted
\definecolor{LightGray}{rgb}{0.95,0.95,0.92}
\setminted{
    linenos=true,
    stepnumber=5,
    numberfirstline=true,
    autogobble,
    breaklines=true,
    breakautoindent=true,
    breaksymbolleft=,
    breaksymbolright=,
    breaksymbolindentleft=0pt,
    breaksymbolindentright=0pt,
    breaksymbolsepleft=0pt,
    breaksymbolsepright=0pt,
    fontsize=\footnotesize,
    bgcolor=LightGray,
    numbersep=10pt
}


\usepackage{listings} % Para incluir código desde un archivo

\renewcommand\lstlistingname{Código Fuente}
\renewcommand\lstlistlistingname{Índice de Códigos Fuente}

% Definir colores
\definecolor{vscodepurple}{rgb}{0.5,0,0.5}
\definecolor{vscodeblue}{rgb}{0,0,0.8}
\definecolor{vscodegreen}{rgb}{0,0.5,0}
\definecolor{vscodegray}{rgb}{0.5,0.5,0.5}
\definecolor{vscodebackground}{rgb}{0.97,0.97,0.97}
\definecolor{vscodelightgray}{rgb}{0.9,0.9,0.9}

% Configuración para el estilo de C similar a VSCode
\lstdefinestyle{vscode_C}{
  backgroundcolor=\color{vscodebackground},
  commentstyle=\color{vscodegreen},
  keywordstyle=\color{vscodeblue},
  numberstyle=\tiny\color{vscodegray},
  stringstyle=\color{vscodepurple},
  basicstyle=\scriptsize\ttfamily,
  breakatwhitespace=false,
  breaklines=true,
  captionpos=b,
  keepspaces=true,
  numbers=left,
  numbersep=5pt,
  showspaces=false,
  showstringspaces=false,
  showtabs=false,
  tabsize=2,
  frame=tb,
  framerule=0pt,
  aboveskip=10pt,
  belowskip=10pt,
  xleftmargin=10pt,
  xrightmargin=10pt,
  framexleftmargin=10pt,
  framexrightmargin=10pt,
  framesep=0pt,
  rulecolor=\color{vscodelightgray},
  backgroundcolor=\color{vscodebackground},
}

%------------------------------------------------------------------------

% Comandos definidos
\newcommand{\bb}[1]{\mathbb{#1}}
\newcommand{\cc}[1]{\mathcal{#1}}

% I prefer the slanted \leq
\let\oldleq\leq % save them in case they're every wanted
\let\oldgeq\geq
\renewcommand{\leq}{\leqslant}
\renewcommand{\geq}{\geqslant}

% Si y solo si
\newcommand{\sii}{\iff}

% Letras griegas
\newcommand{\eps}{\epsilon}
\newcommand{\veps}{\varepsilon}
\newcommand{\lm}{\lambda}

\newcommand{\ol}{\overline}
\newcommand{\ul}{\underline}
\newcommand{\wt}{\widetilde}
\newcommand{\wh}{\widehat}

\let\oldvec\vec
\renewcommand{\vec}{\overrightarrow}

% Derivadas parciales
\newcommand{\del}[2]{\frac{\partial #1}{\partial #2}}
\newcommand{\Del}[3]{\frac{\partial^{#1} #2}{\partial #3^{#1}}}
\newcommand{\deld}[2]{\dfrac{\partial #1}{\partial #2}}
\newcommand{\Deld}[3]{\dfrac{\partial^{#1} #2}{\partial #3^{#1}}}


\newcommand{\AstIg}{\stackrel{(\ast)}{=}}
\newcommand{\Hop}{\stackrel{L'H\hat{o}pital}{=}}

\newcommand{\red}[1]{{\color{red}#1}} % Para integrales, destacar los cambios.

% Método de integración
\newcommand{\MetInt}[2]{
    \left[\begin{array}{c}
        #1 \\ #2
    \end{array}\right]
}

% Declarar aplicaciones
% 1. Nombre aplicación
% 2. Dominio
% 3. Codominio
% 4. Variable
% 5. Imagen de la variable
\newcommand{\Func}[5]{
    \begin{equation*}
        \begin{array}{rrll}
            #1:& #2 & \longrightarrow & #3\\
               & #4 & \longmapsto & #5
        \end{array}
    \end{equation*}
}

%------------------------------------------------------------------------



\begin{document}

    % 1. Foto de fondo
    % 2. Título
    % 3. Encabezado Izquierdo
    % 4. Color de fondo
    % 5. Coord x del titulo
    % 6. Coord y del titulo
    % 7. Fecha

    
    % 1. Foto de fondo
% 2. Título
% 3. Encabezado Izquierdo
% 4. Color de fondo
% 5. Coord x del titulo
% 6. Coord y del titulo
% 7. Fecha

\newcommand{\portada}[7]{

    \portadaBase{#1}{#2}{#3}{#4}{#5}{#6}{#7}
    \portadaBook{#1}{#2}{#3}{#4}{#5}{#6}{#7}
}

\newcommand{\portadaExamen}[7]{

    \portadaBase{#1}{#2}{#3}{#4}{#5}{#6}{#7}
    \portadaArticle{#1}{#2}{#3}{#4}{#5}{#6}{#7}
}




\newcommand{\portadaBase}[7]{

    % Tiene la portada principal y la licencia Creative Commons
    
    % 1. Foto de fondo
    % 2. Título
    % 3. Encabezado Izquierdo
    % 4. Color de fondo
    % 5. Coord x del titulo
    % 6. Coord y del titulo
    % 7. Fecha
    
    
    \thispagestyle{empty}               % Sin encabezado ni pie de página
    \newgeometry{margin=0cm}        % Márgenes nulos para la primera página
    
    
    % Encabezado
    \fancyhead[L]{\helv #3}
    \fancyhead[R]{\helv \nouppercase{\leftmark}}
    
    
    \pagecolor{#4}        % Color de fondo para la portada
    
    \begin{figure}[p]
        \centering
        \transparent{0.3}           % Opacidad del 30% para la imagen
        
        \includegraphics[width=\paperwidth, keepaspectratio]{assets/#1}
    
        \begin{tikzpicture}[remember picture, overlay]
            \node[anchor=north west, text=white, opacity=1, font=\fontsize{60}{90}\selectfont\bfseries\sffamily, align=left] at (#5, #6) {#2};
            
            \node[anchor=south east, text=white, opacity=1, font=\fontsize{12}{18}\selectfont\sffamily, align=right] at (9.7, 3) {\textbf{\href{https://losdeldgiim.github.io/}{Los Del DGIIM}}};
            
            \node[anchor=south east, text=white, opacity=1, font=\fontsize{12}{15}\selectfont\sffamily, align=right] at (9.7, 1.8) {Doble Grado en Ingeniería Informática y Matemáticas\\Universidad de Granada};
        \end{tikzpicture}
    \end{figure}
    
    
    \restoregeometry        % Restaurar márgenes normales para las páginas subsiguientes
    \pagecolor{white}       % Restaurar el color de página
    
    
    \newpage
    \thispagestyle{empty}               % Sin encabezado ni pie de página
    \begin{tikzpicture}[remember picture, overlay]
        \node[anchor=south west, inner sep=3cm] at (current page.south west) {
            \begin{minipage}{0.5\paperwidth}
                \href{https://creativecommons.org/licenses/by-nc-nd/4.0/}{
                    \includegraphics[height=2cm]{assets/Licencia.png}
                }\vspace{1cm}\\
                Esta obra está bajo una
                \href{https://creativecommons.org/licenses/by-nc-nd/4.0/}{
                    Licencia Creative Commons Atribución-NoComercial-SinDerivadas 4.0 Internacional (CC BY-NC-ND 4.0).
                }\\
    
                Eres libre de compartir y redistribuir el contenido de esta obra en cualquier medio o formato, siempre y cuando des el crédito adecuado a los autores originales y no persigas fines comerciales. 
            \end{minipage}
        };
    \end{tikzpicture}
    
    
    
    % 1. Foto de fondo
    % 2. Título
    % 3. Encabezado Izquierdo
    % 4. Color de fondo
    % 5. Coord x del titulo
    % 6. Coord y del titulo
    % 7. Fecha


}


\newcommand{\portadaBook}[7]{

    % 1. Foto de fondo
    % 2. Título
    % 3. Encabezado Izquierdo
    % 4. Color de fondo
    % 5. Coord x del titulo
    % 6. Coord y del titulo
    % 7. Fecha

    % Personaliza el formato del título
    \pretitle{\begin{center}\bfseries\fontsize{42}{56}\selectfont}
    \posttitle{\par\end{center}\vspace{2em}}
    
    % Personaliza el formato del autor
    \preauthor{\begin{center}\Large}
    \postauthor{\par\end{center}\vfill}
    
    % Personaliza el formato de la fecha
    \predate{\begin{center}\huge}
    \postdate{\par\end{center}\vspace{2em}}
    
    \title{#2}
    \author{\href{https://losdeldgiim.github.io/}{Los Del DGIIM}}
    \date{Granada, #7}
    \maketitle
    
    \tableofcontents
}




\newcommand{\portadaArticle}[7]{

    % 1. Foto de fondo
    % 2. Título
    % 3. Encabezado Izquierdo
    % 4. Color de fondo
    % 5. Coord x del titulo
    % 6. Coord y del titulo
    % 7. Fecha

    % Personaliza el formato del título
    \pretitle{\begin{center}\bfseries\fontsize{42}{56}\selectfont}
    \posttitle{\par\end{center}\vspace{2em}}
    
    % Personaliza el formato del autor
    \preauthor{\begin{center}\Large}
    \postauthor{\par\end{center}\vspace{3em}}
    
    % Personaliza el formato de la fecha
    \predate{\begin{center}\huge}
    \postdate{\par\end{center}\vspace{5em}}
    
    \title{#2}
    \author{\href{https://losdeldgiim.github.io/}{Los Del DGIIM}}
    \date{Granada, #7}
    \thispagestyle{empty}               % Sin encabezado ni pie de página
    \maketitle
    \vfill
}
    \portadaExamen{ffccA4.jpg}{Ecuaciones\\Diferenciales I\\Examen V}{Ecuaciones Diferenciales I. Examen V}{MidnightBlue}{-8}{28}{2024-2025}{Arturo Olivares Martos}

    \begin{description}
        \item[Asignatura] Ecuaciones Diferenciales I
        \item[Curso Académico] 2018-19.
        %\item[Grado] Doble Grado en Ingeniería Informática y Matemáticas.
        \item[Grupo] A.
        \item[Profesor] Rafael Ortega Ríos.
        \item[Descripción] Primer parcial.
        \item[Fecha] 21 de marzo de 2019.
        %\item[Duración] 60 minutos.
    
    \end{description}
    \newpage

    \begin{ejercicio}
        En el plano con coordenadas $(x, y)$ se considera la familia de curvas dada por la ecuación
        \begin{equation*}
            \dfrac{y^2}{2} + x=c
        \end{equation*}
        donde $c \in \mathbb{R}$ actúa como parámetro. Encuentra la familia de trayectorias ortogonales. Dibuja ambas familias.\\

        La familia descrita también se puede expresar con la ecuación $x= \nicefrac{y^2}{2} + c$, donde vemos más fácilmente que se trata de una parábola con eje de simetría en el eje $y=0$ y vértice en el punto $(c, 0)$. Además, las ramas de la parábola van hacia los valores de $x$ negativos.

        Buscamos ahora una ecuación diferencial que las admita como soluciones, para lo que derivamos implícitamente:
        \begin{equation*}
            1+yy' = 0 \quad \Longrightarrow \quad y' = -\dfrac{1}{y} \quad \text{con dominio }D=left\left\{\begin{array}{rcl}
                D_1&=&\bb{R}\times \bb{R}^+,\\
                &\lor& \\
                D_2&=&\bb{R}\times \bb{R}^-.
            \end{array} \right.
        \end{equation*}

        Usando que el producto de pendientes de rectas ortogonales es $-1$ y la interpretación geométrica de la derivada, podemos concluir que las trayectorias ortogonales vienen descritas por la ecuación diferencial:
        \begin{equation*}
            y' = y  \text{ con dominio } D=\left\{\begin{array}{rcl}
                D_1&=&\bb{R}\times \bb{R}^+,\\
                &\lor& \\
                D_2&=&\bb{R}\times \bb{R}^-.
            \end{array} \right.
        \end{equation*}

        Sus soluciones sabemos, por lo visto en Teoría, que son de la forma:
        \begin{equation*}
            y(x) = c e^x,\quad \forall x\in \bb{R}\quad \text{con } c \in \bb{R}.
        \end{equation*}

        Las dos familias de curvas dibujadas en el plano $(x, y)$ son:
        \begin{figure}[H]
            \centering
            \begin{tikzpicture}
                \begin{axis}[
                    axis lines = center,
                    xlabel = $x$,
                    ylabel = $y$,
                    xmin = -5,
                    ymin = -5,
                    xmax = 5,
                    ymax = 5,
                    xtick = \empty,
                    ytick = \empty
                ]
                % Dibujamos varias funciones de la forma y=\pm sqrt(2(c-x))
                \addplot[domain=-4.5:0.5, samples=100, color=blue]{sqrt(2*(1-x))};
                \addplot[domain=-4.5:0.5, samples=100, color=blue]{-sqrt(2*(1-x))};
                \addplot[domain=0.5:1, samples=100, color=blue]{sqrt(2*(1-x))};
                \addplot[domain=0.5:1, samples=100, color=blue]{-sqrt(2*(1-x))};
                
                \addplot[domain=-4.5:3.5, samples=100, color=blue!50]{sqrt(2*(4-x))};
                \addplot[domain=-4.5:3.5, samples=100, color=blue!50]{-sqrt(2*(4-x))};
                \addplot[domain=3.5:4, samples=100, color=blue!50]{sqrt(2*(4-x))};
                \addplot[domain=3.5:4, samples=100, color=blue!50]{-sqrt(2*(4-x))};
                
                % Dibujamos las soluciones de la ecuación diferencial
                \addplot[domain=-4.5:4.5, samples=100, color=red]{exp(x)};
                \addplot[domain=-4.5:4.5, samples=100, color=red!40]{-exp(x)};
                \addplot[domain=-4.5:4.5, samples=100, color=red!70]{2*exp(x)};

                \end{axis}
            \end{tikzpicture}
        \end{figure}
    \end{ejercicio}
    
    \begin{ejercicio}
        Escribe la ecuación diferencial que modela la desintegración del Radio 226 sabiendo que la masa se reduce a la mitad (periodo de semi-desintegración) en 1600 años.\\

        Sea $m(t)$ la masa de Radio 226 en el año $t$ desde que comenzó la desintegración. Como la velocidad a la que se desintegra es directamente proporcional a la cantidad de masa presente, podemos escribir la ecuación diferencial:
        \begin{equation*}
            m' = -\lm m,
        \end{equation*}
        donde $\lm\in \bb{R}$ es el factor de proporcionalidad. El dominio de esta ecuación diferencial es $\bb{R}^2$, y como condición inicial sabemos que:
        \begin{equation*}
            m(1600)=\dfrac{m(0)}{2}.
        \end{equation*}
    \end{ejercicio}

    \begin{ejercicio}
        Encuentra las órbitas del sistema autónomo
        \begin{equation*}
            \begin{cases}
                x' = (x^2 + 3y^2 + 1)y,\\
                y' = -(x^2 + 3y^2 + 1)x.
            \end{cases}
        \end{equation*}
        ¿Qué tipo de curvas son?\\

        Para encontrar las órbitas del sistema, en primer lugar hemos de asegurar que:
        \begin{equation*}
            (x^2 + 3y^2 + 1)y \neq 0 \Longleftrightarrow
            y\neq 0 \qquad \forall (x, y) \in \bb{R}^2.
        \end{equation*}

        Por tanto, la ecuación diferencial que define las órbitas es:
        \begin{equation*}
            \dfrac{dy}{dx} = \dfrac{-(x^2 + 3y^2 + 1)x}{(x^2 + 3y^2 + 1)y} = -\dfrac{x}{y} \quad \text{con dominio } D = \left\{
                \begin{array}{rcl}
                    D_1 &=& \bb{R}\times \bb{R}^+,\\
                    &\lor& \\
                    D_2 &=& \bb{R}\times \bb{R}^-.
                \end{array}
            \right.
        \end{equation*}

        Se trata de una ecuación diferencial de variables separadas donde la ecuación dependiente de $y$ no se anula, luego:
        \begin{align*}
            ydy &= -xdx \Longrightarrow \dfrac{y^2}{2} = -\dfrac{x^2}{2} + C'
            \Longrightarrow x^2 + y^2 =C
        \end{align*}
        Esto sabemos que define curvas en implícitas, y se trata de las circunferencias de radio $r=\sqrt{C}$ y centro en el origen.
    \end{ejercicio}

    \begin{ejercicio}
        Se considera la transformación $\varphi : \mathbb{R}^2 \to \mathbb{R}^2$, $\varphi(t, x) = (s, y)$ con:
        \begin{equation*}
            s = e^t, \quad y = e^{t}x.
        \end{equation*}
        Determina la imagen $\varphi(\mathbb{R}^2) = \wh{\Omega}$ y prueba que $\varphi$ es un $C^1-$difeomorfismo entre $\Omega = \mathbb{R}^2$ y $\wh{\Omega}$. ¿Es este cambio de variable admisible para la ecuación $x' = t^2 \cos x$?\\

        En primer lugar, definimos el cambio de variable:
        \Func{\varphi=(\varphi_1,\varphi_2)}{\bb{R}^2}{\wh{\Omega}}{(t,x)}{(s,y)=(e^t, e^{t}x)}

        Busquemos en primer lugar su inversa, para lo cual buscamos despejar de forma única $t$ y $x$ en función de $s$ y $y$:
        \begin{align*}
            s &= e^t \Longrightarrow t = \ln s\\
            y &= e^{t}x \Longrightarrow x= e^{-t}y = e^{-\ln s}y = \nicefrac{y}{s}
        \end{align*}

        Por tanto, tenemos que $\varphi$ es biyectiva, y su inversa es:
        \Func{\varphi^{-1}}{\wh{\Omega}}{\bb{R}^2}{(s,y)}{(t,x)=\left(\ln s, \nicefrac{y}{s}\right)}

        Calculemos ahora $\wh{\Omega}$. En primer lugar, para que $\varphi^{-1}$ esté bien definida, necesitamos $s>0$. Por tanto, $\wh{\Omega}\subset \bb{R}^+\times \bb{R}$. Tenemos que:
        \begin{align*}
            \wh{\Omega} &= \varphi(\bb{R}^2) = \left\{(s,y)\in \bb{R}^2 \mid \left(\ln s, \nicefrac{y}{s}\right)\in \bb{R}^2\right\} = \bb{R}^+\times \bb{R}
        \end{align*}
        
        Tenemos que $\varphi,\varphi^{-1}$ son biyectivas, y sus componentes son productos o cocientes de funciones elementales de clase $1$, por lo que $\varphi,\varphi^{-1}\in C^1$. Por tanto, $\varphi$ define un difeomorfismo entre $\bb{R}^2$ y $\wh{\Omega}$.\\

        Para comprobar que el cambio de variable es admisible para la ecuación $x'=t^2\cos x$, tenemos que:
        \begin{align*}
            \dfrac{\partial \varphi_1}{\partial t} + \dfrac{\partial \varphi_1}{\partial x}x' 
            = e^t + 0\cdot x' = e^t > 0 \qquad \forall (t,x)\in \bb{R}^2
        \end{align*}
        Por tanto, el cambio de variable sí es admisible.
    \end{ejercicio}

    \begin{ejercicio}
        Por un argumento visto en clase sabemos que la ecuación
        \begin{equation*}
            x^{55} + x + t = 0
        \end{equation*}
        define una función $x : \mathbb{R} \to \mathbb{R}$, $x = x(t)$, de clase $C^1$. Demuestra que esta función también es de clase $C^2$ y encuentra una ecuación diferencial de segundo orden que la admita como solución.\\

        Para ello, derivamos implícitamente:
        \begin{equation*}
            1+\left(55x^{54}+1\right)x' = 0 \Longrightarrow x' = -\dfrac{1}{55x^{54}+1}
        \end{equation*}

        La función $x'$ es cociente de dos funciones de clase $C^1$ (pues $x$ es de clase $C^1$), y el denominador no se anula puesto que $55x^{54}+1>0$ para todo $x\in \bb{R}$. Por tanto, $x'$ es de clase $C^1$, y por tanto $x$ es de clase $C^2$.\\

        Para hallar una ecuación diferencial de segundo orden que admita a $x$ como solución, derivamos de nuevo:
        \begin{equation*}
            x'' = \dfrac{1}{(55x^4+1)^2}\cdot 55\cdot 54\cdot x^{53}x' = -\dfrac{55\cdot 54\cdot x^{53}}{(55x^{54}+1)^3}
            \quad \text{ con dominio } D = \bb{R}^2.
        \end{equation*}
    \end{ejercicio}
\end{document}