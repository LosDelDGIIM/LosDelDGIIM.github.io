\documentclass[12pt]{article}

% Idioma y codificación
\usepackage[spanish, es-tabla]{babel}       %es-tabla para que se titule "Tabla"
\usepackage[utf8]{inputenc}

% Márgenes
\usepackage[a4paper,top=3cm,bottom=2.5cm,left=3cm,right=3cm]{geometry}

% Comentarios de bloque
\usepackage{verbatim}

% Paquetes de links
\usepackage[hidelinks]{hyperref}    % Permite enlaces
\usepackage{url}                    % redirecciona a la web

% Más opciones para enumeraciones
\usepackage{enumitem}

% Personalizar la portada
\usepackage{titling}

% Paquetes de tablas
\usepackage{multirow}


%------------------------------------------------------------------------

%Paquetes de figuras
\usepackage{caption}
\usepackage{subcaption} % Figuras al lado de otras
\usepackage{float}      % Poner figuras en el sitio indicado H.


% Paquetes de imágenes
\usepackage{graphicx}       % Paquete para añadir imágenes
\usepackage{transparent}    % Para manejar la opacidad de las figuras

% Paquete para usar colores
\usepackage[dvipsnames]{xcolor}
\usepackage{pagecolor}      % Para cambiar el color de la página

% Habilita tamaños de fuente mayores
\usepackage{fix-cm}

% Para los gráficos
\usepackage{tikz}

% Para poder situar los nodos en los grafos
\usetikzlibrary{positioning}


%------------------------------------------------------------------------

% Paquetes de matemáticas
\usepackage{mathtools, amsfonts, amssymb, mathrsfs}
\usepackage[makeroom]{cancel}     % Simplificar tachando
\usepackage{polynom}    % Divisiones y Ruffini
\usepackage{units} % Para poner fracciones diagonales con \nicefrac

\usepackage{pgfplots}   %Representar funciones
\pgfplotsset{compat=1.18}  % Versión 1.18

\usepackage{tikz-cd}    % Para usar diagramas de composiciones
\usetikzlibrary{calc}   % Para usar cálculo de coordenadas en tikz

%Definición de teoremas, etc.
\usepackage{amsthm}
%\swapnumbers   % Intercambia la posición del texto y de la numeración

\theoremstyle{plain}

\makeatletter
\@ifclassloaded{article}{
  \newtheorem{teo}{Teorema}[section]
}{
  \newtheorem{teo}{Teorema}[chapter]  % Se resetea en cada chapter
}
\makeatother

\newtheorem{coro}{Corolario}[teo]           % Se resetea en cada teorema
\newtheorem{prop}[teo]{Proposición}         % Usa el mismo contador que teorema
\newtheorem{lema}[teo]{Lema}                % Usa el mismo contador que teorema

\theoremstyle{remark}
\newtheorem*{observacion}{Observación}

\theoremstyle{definition}

\makeatletter
\@ifclassloaded{article}{
  \newtheorem{definicion}{Definición} [section]     % Se resetea en cada chapter
}{
  \newtheorem{definicion}{Definición} [chapter]     % Se resetea en cada chapter
}
\makeatother

\newtheorem*{notacion}{Notación}
\newtheorem*{ejemplo}{Ejemplo}
\newtheorem*{ejercicio*}{Ejercicio}             % No numerado
\newtheorem{ejercicio}{Ejercicio} [section]     % Se resetea en cada section


% Modificar el formato de la numeración del teorema "ejercicio"
\renewcommand{\theejercicio}{%
  \ifnum\value{section}=0 % Si no se ha iniciado ninguna sección
    \arabic{ejercicio}% Solo mostrar el número de ejercicio
  \else
    \thesection.\arabic{ejercicio}% Mostrar número de sección y número de ejercicio
  \fi
}


% \renewcommand\qedsymbol{$\blacksquare$}         % Cambiar símbolo QED
%------------------------------------------------------------------------

% Paquetes para encabezados
\usepackage{fancyhdr}
\pagestyle{fancy}
\fancyhf{}

\newcommand{\helv}{ % Modificación tamaño de letra
\fontfamily{}\fontsize{12}{12}\selectfont}
\setlength{\headheight}{15pt} % Amplía el tamaño del índice


%\usepackage{lastpage}   % Referenciar última pag   \pageref{LastPage}
\fancyfoot[C]{\thepage}

%------------------------------------------------------------------------

% Conseguir que no ponga "Capítulo 1". Sino solo "1."
\makeatletter
\@ifclassloaded{book}{
  \renewcommand{\chaptermark}[1]{\markboth{\thechapter.\ #1}{}} % En el encabezado
    
  \renewcommand{\@makechapterhead}[1]{%
  \vspace*{50\p@}%
  {\parindent \z@ \raggedright \normalfont
    \ifnum \c@secnumdepth >\m@ne
      \huge\bfseries \thechapter.\hspace{1em}\ignorespaces
    \fi
    \interlinepenalty\@M
    \Huge \bfseries #1\par\nobreak
    \vskip 40\p@
  }}
}
\makeatother

%------------------------------------------------------------------------
% Paquetes de cógido
\usepackage{minted}
\renewcommand\listingscaption{Código fuente}

\usepackage{fancyvrb}
% Personaliza el tamaño de los números de línea
\renewcommand{\theFancyVerbLine}{\small\arabic{FancyVerbLine}}

% Estilo para C++
\newminted{cpp}{
    frame=lines,
    framesep=2mm,
    baselinestretch=1.2,
    linenos,
    escapeinside=||
}

% para minted
\definecolor{LightGray}{rgb}{0.95,0.95,0.92}
\setminted{
    linenos=true,
    stepnumber=5,
    numberfirstline=true,
    autogobble,
    breaklines=true,
    breakautoindent=true,
    breaksymbolleft=,
    breaksymbolright=,
    breaksymbolindentleft=0pt,
    breaksymbolindentright=0pt,
    breaksymbolsepleft=0pt,
    breaksymbolsepright=0pt,
    fontsize=\footnotesize,
    bgcolor=LightGray,
    numbersep=10pt
}


\usepackage{listings} % Para incluir código desde un archivo

\renewcommand\lstlistingname{Código Fuente}
\renewcommand\lstlistlistingname{Índice de Códigos Fuente}

% Definir colores
\definecolor{vscodepurple}{rgb}{0.5,0,0.5}
\definecolor{vscodeblue}{rgb}{0,0,0.8}
\definecolor{vscodegreen}{rgb}{0,0.5,0}
\definecolor{vscodegray}{rgb}{0.5,0.5,0.5}
\definecolor{vscodebackground}{rgb}{0.97,0.97,0.97}
\definecolor{vscodelightgray}{rgb}{0.9,0.9,0.9}

% Configuración para el estilo de C similar a VSCode
\lstdefinestyle{vscode_C}{
  backgroundcolor=\color{vscodebackground},
  commentstyle=\color{vscodegreen},
  keywordstyle=\color{vscodeblue},
  numberstyle=\tiny\color{vscodegray},
  stringstyle=\color{vscodepurple},
  basicstyle=\scriptsize\ttfamily,
  breakatwhitespace=false,
  breaklines=true,
  captionpos=b,
  keepspaces=true,
  numbers=left,
  numbersep=5pt,
  showspaces=false,
  showstringspaces=false,
  showtabs=false,
  tabsize=2,
  frame=tb,
  framerule=0pt,
  aboveskip=10pt,
  belowskip=10pt,
  xleftmargin=10pt,
  xrightmargin=10pt,
  framexleftmargin=10pt,
  framexrightmargin=10pt,
  framesep=0pt,
  rulecolor=\color{vscodelightgray},
  backgroundcolor=\color{vscodebackground},
}

%------------------------------------------------------------------------

% Comandos definidos
\newcommand{\bb}[1]{\mathbb{#1}}
\newcommand{\cc}[1]{\mathcal{#1}}

% I prefer the slanted \leq
\let\oldleq\leq % save them in case they're every wanted
\let\oldgeq\geq
\renewcommand{\leq}{\leqslant}
\renewcommand{\geq}{\geqslant}

% Si y solo si
\newcommand{\sii}{\iff}

% Letras griegas
\newcommand{\eps}{\epsilon}
\newcommand{\veps}{\varepsilon}
\newcommand{\lm}{\lambda}

\newcommand{\ol}{\overline}
\newcommand{\ul}{\underline}
\newcommand{\wt}{\widetilde}
\newcommand{\wh}{\widehat}

\let\oldvec\vec
\renewcommand{\vec}{\overrightarrow}

% Derivadas parciales
\newcommand{\del}[2]{\frac{\partial #1}{\partial #2}}
\newcommand{\Del}[3]{\frac{\partial^{#1} #2}{\partial #3^{#1}}}
\newcommand{\deld}[2]{\dfrac{\partial #1}{\partial #2}}
\newcommand{\Deld}[3]{\dfrac{\partial^{#1} #2}{\partial #3^{#1}}}


\newcommand{\AstIg}{\stackrel{(\ast)}{=}}
\newcommand{\Hop}{\stackrel{L'H\hat{o}pital}{=}}

\newcommand{\red}[1]{{\color{red}#1}} % Para integrales, destacar los cambios.

% Método de integración
\newcommand{\MetInt}[2]{
    \left[\begin{array}{c}
        #1 \\ #2
    \end{array}\right]
}

% Declarar aplicaciones
% 1. Nombre aplicación
% 2. Dominio
% 3. Codominio
% 4. Variable
% 5. Imagen de la variable
\newcommand{\Func}[5]{
    \begin{equation*}
        \begin{array}{rrll}
            #1:& #2 & \longrightarrow & #3\\
               & #4 & \longmapsto & #5
        \end{array}
    \end{equation*}
}

%------------------------------------------------------------------------



\begin{document}

    % 1. Foto de fondo
    % 2. Título
    % 3. Encabezado Izquierdo
    % 4. Color de fondo
    % 5. Coord x del titulo
    % 6. Coord y del titulo
    % 7. Fecha

    
    % 1. Foto de fondo
% 2. Título
% 3. Encabezado Izquierdo
% 4. Color de fondo
% 5. Coord x del titulo
% 6. Coord y del titulo
% 7. Fecha

\newcommand{\portada}[7]{

    \portadaBase{#1}{#2}{#3}{#4}{#5}{#6}{#7}
    \portadaBook{#1}{#2}{#3}{#4}{#5}{#6}{#7}
}

\newcommand{\portadaExamen}[7]{

    \portadaBase{#1}{#2}{#3}{#4}{#5}{#6}{#7}
    \portadaArticle{#1}{#2}{#3}{#4}{#5}{#6}{#7}
}




\newcommand{\portadaBase}[7]{

    % Tiene la portada principal y la licencia Creative Commons
    
    % 1. Foto de fondo
    % 2. Título
    % 3. Encabezado Izquierdo
    % 4. Color de fondo
    % 5. Coord x del titulo
    % 6. Coord y del titulo
    % 7. Fecha
    
    
    \thispagestyle{empty}               % Sin encabezado ni pie de página
    \newgeometry{margin=0cm}        % Márgenes nulos para la primera página
    
    
    % Encabezado
    \fancyhead[L]{\helv #3}
    \fancyhead[R]{\helv \nouppercase{\leftmark}}
    
    
    \pagecolor{#4}        % Color de fondo para la portada
    
    \begin{figure}[p]
        \centering
        \transparent{0.3}           % Opacidad del 30% para la imagen
        
        \includegraphics[width=\paperwidth, keepaspectratio]{assets/#1}
    
        \begin{tikzpicture}[remember picture, overlay]
            \node[anchor=north west, text=white, opacity=1, font=\fontsize{60}{90}\selectfont\bfseries\sffamily, align=left] at (#5, #6) {#2};
            
            \node[anchor=south east, text=white, opacity=1, font=\fontsize{12}{18}\selectfont\sffamily, align=right] at (9.7, 3) {\textbf{\href{https://losdeldgiim.github.io/}{Los Del DGIIM}}};
            
            \node[anchor=south east, text=white, opacity=1, font=\fontsize{12}{15}\selectfont\sffamily, align=right] at (9.7, 1.8) {Doble Grado en Ingeniería Informática y Matemáticas\\Universidad de Granada};
        \end{tikzpicture}
    \end{figure}
    
    
    \restoregeometry        % Restaurar márgenes normales para las páginas subsiguientes
    \pagecolor{white}       % Restaurar el color de página
    
    
    \newpage
    \thispagestyle{empty}               % Sin encabezado ni pie de página
    \begin{tikzpicture}[remember picture, overlay]
        \node[anchor=south west, inner sep=3cm] at (current page.south west) {
            \begin{minipage}{0.5\paperwidth}
                \href{https://creativecommons.org/licenses/by-nc-nd/4.0/}{
                    \includegraphics[height=2cm]{assets/Licencia.png}
                }\vspace{1cm}\\
                Esta obra está bajo una
                \href{https://creativecommons.org/licenses/by-nc-nd/4.0/}{
                    Licencia Creative Commons Atribución-NoComercial-SinDerivadas 4.0 Internacional (CC BY-NC-ND 4.0).
                }\\
    
                Eres libre de compartir y redistribuir el contenido de esta obra en cualquier medio o formato, siempre y cuando des el crédito adecuado a los autores originales y no persigas fines comerciales. 
            \end{minipage}
        };
    \end{tikzpicture}
    
    
    
    % 1. Foto de fondo
    % 2. Título
    % 3. Encabezado Izquierdo
    % 4. Color de fondo
    % 5. Coord x del titulo
    % 6. Coord y del titulo
    % 7. Fecha


}


\newcommand{\portadaBook}[7]{

    % 1. Foto de fondo
    % 2. Título
    % 3. Encabezado Izquierdo
    % 4. Color de fondo
    % 5. Coord x del titulo
    % 6. Coord y del titulo
    % 7. Fecha

    % Personaliza el formato del título
    \pretitle{\begin{center}\bfseries\fontsize{42}{56}\selectfont}
    \posttitle{\par\end{center}\vspace{2em}}
    
    % Personaliza el formato del autor
    \preauthor{\begin{center}\Large}
    \postauthor{\par\end{center}\vfill}
    
    % Personaliza el formato de la fecha
    \predate{\begin{center}\huge}
    \postdate{\par\end{center}\vspace{2em}}
    
    \title{#2}
    \author{\href{https://losdeldgiim.github.io/}{Los Del DGIIM}}
    \date{Granada, #7}
    \maketitle
    
    \tableofcontents
}




\newcommand{\portadaArticle}[7]{

    % 1. Foto de fondo
    % 2. Título
    % 3. Encabezado Izquierdo
    % 4. Color de fondo
    % 5. Coord x del titulo
    % 6. Coord y del titulo
    % 7. Fecha

    % Personaliza el formato del título
    \pretitle{\begin{center}\bfseries\fontsize{42}{56}\selectfont}
    \posttitle{\par\end{center}\vspace{2em}}
    
    % Personaliza el formato del autor
    \preauthor{\begin{center}\Large}
    \postauthor{\par\end{center}\vspace{3em}}
    
    % Personaliza el formato de la fecha
    \predate{\begin{center}\huge}
    \postdate{\par\end{center}\vspace{5em}}
    
    \title{#2}
    \author{\href{https://losdeldgiim.github.io/}{Los Del DGIIM}}
    \date{Granada, #7}
    \thispagestyle{empty}               % Sin encabezado ni pie de página
    \maketitle
    \vfill
}
    \portadaExamen{ffccA4.jpg}{Ecuaciones\\Diferenciales I\\Examen XVII}{Ecuaciones Diferenciales I. Examen XVII}{MidnightBlue}{-8}{28}{2024-2025}{Arturo Olivares Martos}

    \begin{description}
        \item[Asignatura] Ecuaciones Diferenciales I
        \item[Curso Académico] 2015-16.
        % \item[Grado] Doble Grado en Ingeniería Informática y Matemáticas.
        \item[Grupo] B.
        \item[Profesor] Rafael Ortega Ríos.
        \item[Descripción] Parcial C.
        \item[Fecha] 8 de Junio de 2016.
        %\item[Duración] 60 minutos.
    
    \end{description}
    \newpage

    \begin{ejercicio}
        Calcula $e^{tA}$ si $\displaystyle A = \displaystyle
        \begin{pmatrix}
            0 & 1 \\
            1 & 0
        \end{pmatrix}$.\\

        Hay dos opciones:
        \begin{description}
            \item[A partir de la definición:] Tenemos que:
            \begin{equation*}
                e^{tA} = \sum_{n=0}^{\infty} \dfrac{t^nA^n}{n!}
            \end{equation*}

            Calculemos la potencia $n-$ésima de $A$. Para $n=2$, tenemos:
            \begin{equation*}
                A^2 = \displaystyle
                \begin{pmatrix}
                    0 & 1 \\
                    1 & 0
                \end{pmatrix}
                \displaystyle
                \begin{pmatrix}
                    0 & 1 \\
                    1 & 0
                \end{pmatrix}
                = \displaystyle
                \begin{pmatrix}
                    1 & 0 \\
                    0 & 1
                \end{pmatrix} = Id_2
            \end{equation*}

            Por tanto, tenemos que:
            \begin{equation*}
                A^{2n}=Id_2,\quad A^{2n+1}=A\qquad \forall n\in\mathbb{N}
            \end{equation*}

            Por tanto:
            \begin{align*}
                e^{tA} = Id_2 + tA + \dfrac{t^2}{2!}Id_2 + \dfrac{t^3}{3!}A + \dfrac{t^4}{4!}Id_2+\dots
                = \begin{pmatrix}
                    \sum\limits_{n=0}^{\infty} \dfrac{t^{2n}}{(2n)!} & \sum\limits_{n=0}^{\infty} \dfrac{t^{2n+1}}{(2n+1)!} \\
                    \sum\limits_{n=0}^{\infty} \dfrac{t^{2n+1}}{(2n+1)!} & \sum\limits_{n=0}^{\infty} \dfrac{t^{2n}}{(2n)!}
                \end{pmatrix}
            \end{align*}

            Resolver dichas series es complejo, por lo que vamos a intentar resolverlo de otra forma.

            \item[Usando las Ecuaciones Diferenciales:]
            Buscamos oftener una matriz fundamental $\Phi$ del sistema dado por $x'=A(t)x$. Calculamos los valores propios de $A$:
            \begin{equation*}
                P_A(\lambda) = \lm^2-1 = 0 \Longleftrightarrow \lm = \pm 1
            \end{equation*}

            Por tanto, obtenemos la siguiente matriz fundamental:
            \begin{equation*}
                \Phi(t)=\begin{pmatrix}
                    e^t & e^{-t} \\
                    e^t & -e^{-t}
                \end{pmatrix}
            \end{equation*}

            Por tanto, la matriz fundamental del sistema en $t_0=0$ es:
            \begin{align*}
                \Phi(t)\Phi(0)^{-1}
                &= \Phi(t)\begin{pmatrix}
                    1 & 1 \\
                    1 & -1
                \end{pmatrix}^{-1}
                = -\frac{1}{2}\begin{pmatrix}
                    e^t & e^{-t} \\
                    e^t & -e^{-t}
                \end{pmatrix}
                \begin{pmatrix}
                    -1 & -1 \\
                    -1 & 1
                \end{pmatrix}
                =\\&= \frac{1}{2}\begin{pmatrix}
                    e^t & e^{-t} \\
                    e^t & -e^{-t}
                \end{pmatrix}
                \begin{pmatrix}
                    1 & 1 \\
                    1 & -1
                \end{pmatrix}
                =\frac{1}{2}\begin{pmatrix}
                    e^t+e^{-t} & e^t - e^{-t} \\
                    e^t-e^{-t} & e^t+e^{-t}
                \end{pmatrix}
            \end{align*}
        \end{description}

        Por tanto, tenemos que:
        \begin{equation*}
            e^{tA} = \frac{1}{2}\begin{pmatrix}
                e^t+e^{-t} & e^t - e^{-t} \\
                e^t-e^{-t} & e^t+e^{-t}
            \end{pmatrix}
        \end{equation*}
    \end{ejercicio}

    \begin{ejercicio}
        Encuentra una matriz fundamental del sistema
        \begin{equation*}
            x_1' = 3x_1 + x_2, \quad x_2' = 3x_2+x_3,\quad x_3' = 3x_3.
        \end{equation*}

        Definimos $x:\bb{R}\to \bb{R}^3$, $A\in \bb{R}^{3\times 3}$ dados por:
        \begin{equation*}
            x=\begin{pmatrix}
                x_1 \\
                x_2 \\
                x_3
            \end{pmatrix},\quad A=\begin{pmatrix}
                3 & 1 & 0 \\
                0 & 3 & 1 \\
                0 & 0 & 3
            \end{pmatrix}
        \end{equation*}

        Al no haber definido una condición inicial, la solución no será única. Al ser similar a un sistema triangular, procedemos de forma similar obteniendo soluciones de forma escalonada:
        \begin{equation*}
            x'_3=3x_3\Longrightarrow x_3(t)=c_3e^{3t} \qquad c_3\in\bb{R}
        \end{equation*}

        Sustituyendo en la ecuación anterior, obtenemos:
        \begin{align*}
            x'_2=3x_2+c_3e^{3t}\Longrightarrow
            x_2(t)&=e^{3t}\left(c_2+\int e^{-3t}c_3e^{3t}dt\right)
            =\\&= c_2e^{3t}+c_3te^{3t}
        \end{align*}

        Sustituyendo en la ecuación anterior, obtenemos:
        \begin{align*}
            x'_1=3x_1+c_2e^{3t}+c_3te^{3t}\Longrightarrow
            x_1(t)&=e^{3t}\left(c_1+\int e^{-3t}\left(c_2e^{3t}+c_3te^{3t}\right)dt\right)
            =\\&= c_1e^{3t}+c_2te^{3t}+c_3\cdot \frac{t^2}{2}\cdot e^{3t}
        \end{align*}

        Por tanto, tenemos que la solución general del sistema es:
        \begin{equation*}
            x(t)=\begin{pmatrix}
                c_1e^{3t}+c_2te^{3t}+\frac{c_3t^2}{2}e^{3t} \\
                c_2e^{3t}+c_3te^{3t} \\
                c_3e^{3t}
            \end{pmatrix}
            = c_1e^{3t}\begin{pmatrix}
                1 \\
                0 \\
                0
            \end{pmatrix}+c_2e^{3t}\begin{pmatrix}
                t \\
                1 \\
                0
            \end{pmatrix}+c_3e^{3t}\begin{pmatrix}
                \nicefrac{t^2}{2} \\
                t \\
                1
            \end{pmatrix}\qquad c_1,c_2,c_3\in\bb{R}
        \end{equation*}

        Por tanto, una matriz solución del sistema es:
        \begin{equation*}
            \Phi(t)=\begin{pmatrix}
                e^{3t} & te^{3t} & \nicefrac{t^2}{2}e^{3t} \\
                0 & e^{3t} & te^{3t} \\
                0 & 0 & e^{3t}
            \end{pmatrix}
        \end{equation*}

        Además, es matriz fundamental del sistema, ya que:
        \begin{equation*}
            \det\Phi(t)=e^{9t}\neq 0\qquad \forall t\in\bb{R}
        \end{equation*}
    \end{ejercicio}

    \begin{ejercicio}
        Se considera el problema de valores iniciales
        \begin{equation*}
            x' = 3x+\sen t,\quad x(0) = 0
        \end{equation*}
        y se define la correspondiente sucesión de iterantes de Picard ${\{x_n(t)\}}_{n\geq 0}$. Calcula $x_2(t)$.

        Sea $x_n:\bb{R}\to\bb{R}$ la sucesión de iterantes de Picard definida por:
        \begin{align*}
            x_0(t)&=0 \\
            x_{n+1}(t)&=x_0+\int_{t_0}^{t}[a(s)x_n(s)+b(s)]ds \\
            &= \int_{0}^{t}[3x_n(s)+\sen s]ds \qquad n\geq 0
        \end{align*}

        Tenemos que:
        \begin{align*}
            x_1(t)&=\int_{0}^{t}[3x_0(s)+\sen s]ds
            =\int_{0}^{t}\sen sds
            =\left[-\cos s\right]_{0}^{t}=1-\cos t\\
            x_2(t)&=\int_{0}^{t}[3x_1(s)+\sen s]ds
            =\int_{0}^{t}[3(1-\cos s)+\sen s]ds
            =\int_{0}^{t}[3-3\cos s+\sen s]ds
            =\\&= \left[3s-3\sen s-\cos s\right]_{0}^{t}
            =3t-3\sen t-\cos t +1
        \end{align*}
    \end{ejercicio}

    \begin{ejercicio}
        Se emplea la norma Euclídea en $\mathbb{R}^2$ y la norma matricial asociada en $\mathbb{R}^2$. Calcula $\|R\|$ para la matriz $R=\displaystyle
        \begin{pmatrix}
            \cos\theta & -\sen\theta \\
            \sen\theta & \cos\theta
        \end{pmatrix}$.

        Para $x\in \bb{R}^2$, notaremos $x=(x_1,x_2)$. La norma matricial asociada a la norma Euclídea en $\bb{R}^2$ es:
        \begin{align*}
            \|R\|&=\max_{\|x\|=1}\|Rx\|=\max_{\|x\|=1}\left\|\begin{pmatrix}
                \cos\theta & -\sen\theta \\
                \sen\theta & \cos\theta
            \end{pmatrix}\begin{pmatrix}
                x_1 \\
                x_2
            \end{pmatrix}\right\|
            =\\&= \max_{\|x\|=1}\left\|\begin{pmatrix}
                x_1\cos\theta-x_2\sen\theta \\
                x_1\sen\theta+x_2\cos\theta
            \end{pmatrix}\right\|
            =\\&= \max_{\|x\|=1}\sqrt{(x_1\cos\theta-x_2\sen\theta)^2+(x_1\sen\theta+x_2\cos\theta)^2}
            =\\&= \max_{\|x\|=1}\sqrt{x_1^2\cos^2\theta-\cancel{2x_1x_2\cos\theta\sen\theta}+x_2^2\sen^2\theta+x_1^2\sen^2\theta+\cancel{2x_1x_2\cos\theta\sen\theta}+x_2^2\cos^2\theta}
            =\\&= \max_{\|x\|=1}\sqrt{x_1^2(\cos^2\theta+\sen^2\theta)+x_2^2(\sen^2\theta+\cos^2\theta)}
            =\\&= \max_{\|x\|=1}\sqrt{x_1^2+x_2^2}
            =\\&= \max_{\|x\|=1}\|x\|=1
        \end{align*}

        Por tanto, tenemos que $\|R\|=1$. Interpretando $R$ como la matriz asociada a una aplicación lineal, tenemos que se trata de la matriz de rotación de ángulo $\theta$. Por tanto, notando también como $R$ a la aplicación lineal asociada, tenemos que:
        \begin{equation*}
            R(\bb{S}_1)=\bb{S}_1
            \Longrightarrow \|R\|=\max_{\|x\|=1}\|Rx\|=\max_{x\in \bb{S}_1}\|Rx\|
            =\max_{x\in \bb{S}_1}\|x\|=1
        \end{equation*}

        En cualquier caso, vemos que $\|R\|=1$.
    \end{ejercicio}

    \begin{ejercicio}
        Se considera una sucesión de funciones continuas $f_n:[0,1]\rightarrow\mathbb{R}$ que cumplen $f_0(t)=1+t$, $f_1(t)=4+t$,
        \begin{equation*}
            |f_{n+1}(t) - f_n(t)| \leq 7 \int_{0}^{t} |f_n(s)-f_{n-1}(s)|~ds \qquad \text{\ si\ } n\geq 1, t\in [0,1].
        \end{equation*}
        Prueba que la sucesión $\{f_n\}$ converge uniformemente en $[0,1]$.\\

        Usaremos para ello el Test de Weierstrass. Para ello, veamos los primeros términos. Para todo $t\in [0,1]$, tenemos:
        \begin{align*}
            |f_1(t)-f_0(t)|&=|4+t-(1+t)|=3\\
            |f_2(t)-f_1(t)|&\leq 7\int_{0}^{t}|f_1(s)-f_0(s)|ds=7\int_{0}^{t}3ds=7\cdot 3\cdot t \\
            |f_3(t)-f_2(t)|&\leq 7\int_{0}^{t}|f_2(s)-f_1(s)|ds\leq 7\int_{0}^{t}7\cdot 3\cdot sds=7^2\cdot 3\cdot \frac{t^2}{2}
        \end{align*}

        Probemos por tanto por inducción que:
        \begin{equation*}
            |f_{n+1}(t)-f_n(t)|\leq 7^n\cdot 3\cdot \frac{t^n}{n!}
        \end{equation*}
        \begin{itemize}
            \item Para $n=1$, se tiene.
            \item Supongamos que se cumple para $n$, veamos que se cumple para $n+1$:
            \begin{align*}
                |f_{n+2}(t)-f_{n+1}(t)|&\leq 7\int_{0}^{t}|f_{n+1}(s)-f_n(s)|ds
                \leq 7\int_{0}^{t}7^n\cdot 3\cdot \frac{s^n}{n!}ds
                =\\&= 7^{n+1}\cdot 3\cdot \frac{t^{n+1}}{(n+1)!}
            \end{align*}
        \end{itemize}

        Por tanto, acotando $t$ por $1$, tenemos que:
        \begin{equation*}
            |f_{n+1}(t)-f_n(t)|\leq 7^n\cdot 3\cdot \frac{t^n}{n!}
            \leq 3\cdot \frac{7^n}{n!}
        \end{equation*}

        Definimos por tanto la sucesión de números reales:
        \begin{equation*}
            M_n=3\cdot \frac{7^n}{n!}
        \end{equation*}

        De esta forma, por lo visto anteriormente tenemos que:
        \begin{equation*}
            |f_{n+1}(t)-f_n(t)|\leq M_n\qquad \forall t\in [0,1],\forall n\in\bb{N}
        \end{equation*}

        Estudiemos ahora la convergencia de la serie $\sum\limits_{n=0}^{\infty}M_n$:
        \begin{align*}
            \sum_{n=0}^{\infty}M_n
            &=3\sum_{n=0}^{\infty}\frac{7^n}{n!}
            =3e^7<\infty
        \end{align*}

        Por tanto, por el Test de Weierstrass, tenemos que la sucesión $\{f_n\}$ converge uniformemente en $[0,1]$.
    \end{ejercicio}

\end{document}
