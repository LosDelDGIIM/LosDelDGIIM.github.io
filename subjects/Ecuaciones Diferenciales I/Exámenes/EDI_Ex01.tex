\documentclass[12pt]{article}

% Idioma y codificación
\usepackage[spanish, es-tabla]{babel}       %es-tabla para que se titule "Tabla"
\usepackage[utf8]{inputenc}

% Márgenes
\usepackage[a4paper,top=3cm,bottom=2.5cm,left=3cm,right=3cm]{geometry}

% Comentarios de bloque
\usepackage{verbatim}

% Paquetes de links
\usepackage[hidelinks]{hyperref}    % Permite enlaces
\usepackage{url}                    % redirecciona a la web

% Más opciones para enumeraciones
\usepackage{enumitem}

% Personalizar la portada
\usepackage{titling}

% Paquetes de tablas
\usepackage{multirow}


%------------------------------------------------------------------------

%Paquetes de figuras
\usepackage{caption}
\usepackage{subcaption} % Figuras al lado de otras
\usepackage{float}      % Poner figuras en el sitio indicado H.


% Paquetes de imágenes
\usepackage{graphicx}       % Paquete para añadir imágenes
\usepackage{transparent}    % Para manejar la opacidad de las figuras

% Paquete para usar colores
\usepackage[dvipsnames]{xcolor}
\usepackage{pagecolor}      % Para cambiar el color de la página

% Habilita tamaños de fuente mayores
\usepackage{fix-cm}

% Para los gráficos
\usepackage{tikz}

% Para poder situar los nodos en los grafos
\usetikzlibrary{positioning}


%------------------------------------------------------------------------

% Paquetes de matemáticas
\usepackage{mathtools, amsfonts, amssymb, mathrsfs}
\usepackage[makeroom]{cancel}     % Simplificar tachando
\usepackage{polynom}    % Divisiones y Ruffini
\usepackage{units} % Para poner fracciones diagonales con \nicefrac

\usepackage{pgfplots}   %Representar funciones
\pgfplotsset{compat=1.18}  % Versión 1.18

\usepackage{tikz-cd}    % Para usar diagramas de composiciones
\usetikzlibrary{calc}   % Para usar cálculo de coordenadas en tikz

%Definición de teoremas, etc.
\usepackage{amsthm}
%\swapnumbers   % Intercambia la posición del texto y de la numeración

\theoremstyle{plain}

\makeatletter
\@ifclassloaded{article}{
  \newtheorem{teo}{Teorema}[section]
}{
  \newtheorem{teo}{Teorema}[chapter]  % Se resetea en cada chapter
}
\makeatother

\newtheorem{coro}{Corolario}[teo]           % Se resetea en cada teorema
\newtheorem{prop}[teo]{Proposición}         % Usa el mismo contador que teorema
\newtheorem{lema}[teo]{Lema}                % Usa el mismo contador que teorema

\theoremstyle{remark}
\newtheorem*{observacion}{Observación}

\theoremstyle{definition}

\makeatletter
\@ifclassloaded{article}{
  \newtheorem{definicion}{Definición} [section]     % Se resetea en cada chapter
}{
  \newtheorem{definicion}{Definición} [chapter]     % Se resetea en cada chapter
}
\makeatother

\newtheorem*{notacion}{Notación}
\newtheorem*{ejemplo}{Ejemplo}
\newtheorem*{ejercicio*}{Ejercicio}             % No numerado
\newtheorem{ejercicio}{Ejercicio} [section]     % Se resetea en cada section


% Modificar el formato de la numeración del teorema "ejercicio"
\renewcommand{\theejercicio}{%
  \ifnum\value{section}=0 % Si no se ha iniciado ninguna sección
    \arabic{ejercicio}% Solo mostrar el número de ejercicio
  \else
    \thesection.\arabic{ejercicio}% Mostrar número de sección y número de ejercicio
  \fi
}


% \renewcommand\qedsymbol{$\blacksquare$}         % Cambiar símbolo QED
%------------------------------------------------------------------------

% Paquetes para encabezados
\usepackage{fancyhdr}
\pagestyle{fancy}
\fancyhf{}

\newcommand{\helv}{ % Modificación tamaño de letra
\fontfamily{}\fontsize{12}{12}\selectfont}
\setlength{\headheight}{15pt} % Amplía el tamaño del índice


%\usepackage{lastpage}   % Referenciar última pag   \pageref{LastPage}
\fancyfoot[C]{\thepage}

%------------------------------------------------------------------------

% Conseguir que no ponga "Capítulo 1". Sino solo "1."
\makeatletter
\@ifclassloaded{book}{
  \renewcommand{\chaptermark}[1]{\markboth{\thechapter.\ #1}{}} % En el encabezado
    
  \renewcommand{\@makechapterhead}[1]{%
  \vspace*{50\p@}%
  {\parindent \z@ \raggedright \normalfont
    \ifnum \c@secnumdepth >\m@ne
      \huge\bfseries \thechapter.\hspace{1em}\ignorespaces
    \fi
    \interlinepenalty\@M
    \Huge \bfseries #1\par\nobreak
    \vskip 40\p@
  }}
}
\makeatother

%------------------------------------------------------------------------
% Paquetes de cógido
\usepackage{minted}
\renewcommand\listingscaption{Código fuente}

\usepackage{fancyvrb}
% Personaliza el tamaño de los números de línea
\renewcommand{\theFancyVerbLine}{\small\arabic{FancyVerbLine}}

% Estilo para C++
\newminted{cpp}{
    frame=lines,
    framesep=2mm,
    baselinestretch=1.2,
    linenos,
    escapeinside=||
}

% para minted
\definecolor{LightGray}{rgb}{0.95,0.95,0.92}
\setminted{
    linenos=true,
    stepnumber=5,
    numberfirstline=true,
    autogobble,
    breaklines=true,
    breakautoindent=true,
    breaksymbolleft=,
    breaksymbolright=,
    breaksymbolindentleft=0pt,
    breaksymbolindentright=0pt,
    breaksymbolsepleft=0pt,
    breaksymbolsepright=0pt,
    fontsize=\footnotesize,
    bgcolor=LightGray,
    numbersep=10pt
}


\usepackage{listings} % Para incluir código desde un archivo

\renewcommand\lstlistingname{Código Fuente}
\renewcommand\lstlistlistingname{Índice de Códigos Fuente}

% Definir colores
\definecolor{vscodepurple}{rgb}{0.5,0,0.5}
\definecolor{vscodeblue}{rgb}{0,0,0.8}
\definecolor{vscodegreen}{rgb}{0,0.5,0}
\definecolor{vscodegray}{rgb}{0.5,0.5,0.5}
\definecolor{vscodebackground}{rgb}{0.97,0.97,0.97}
\definecolor{vscodelightgray}{rgb}{0.9,0.9,0.9}

% Configuración para el estilo de C similar a VSCode
\lstdefinestyle{vscode_C}{
  backgroundcolor=\color{vscodebackground},
  commentstyle=\color{vscodegreen},
  keywordstyle=\color{vscodeblue},
  numberstyle=\tiny\color{vscodegray},
  stringstyle=\color{vscodepurple},
  basicstyle=\scriptsize\ttfamily,
  breakatwhitespace=false,
  breaklines=true,
  captionpos=b,
  keepspaces=true,
  numbers=left,
  numbersep=5pt,
  showspaces=false,
  showstringspaces=false,
  showtabs=false,
  tabsize=2,
  frame=tb,
  framerule=0pt,
  aboveskip=10pt,
  belowskip=10pt,
  xleftmargin=10pt,
  xrightmargin=10pt,
  framexleftmargin=10pt,
  framexrightmargin=10pt,
  framesep=0pt,
  rulecolor=\color{vscodelightgray},
  backgroundcolor=\color{vscodebackground},
}

%------------------------------------------------------------------------

% Comandos definidos
\newcommand{\bb}[1]{\mathbb{#1}}
\newcommand{\cc}[1]{\mathcal{#1}}

% I prefer the slanted \leq
\let\oldleq\leq % save them in case they're every wanted
\let\oldgeq\geq
\renewcommand{\leq}{\leqslant}
\renewcommand{\geq}{\geqslant}

% Si y solo si
\newcommand{\sii}{\iff}

% Letras griegas
\newcommand{\eps}{\epsilon}
\newcommand{\veps}{\varepsilon}
\newcommand{\lm}{\lambda}

\newcommand{\ol}{\overline}
\newcommand{\ul}{\underline}
\newcommand{\wt}{\widetilde}
\newcommand{\wh}{\widehat}

\let\oldvec\vec
\renewcommand{\vec}{\overrightarrow}

% Derivadas parciales
\newcommand{\del}[2]{\frac{\partial #1}{\partial #2}}
\newcommand{\Del}[3]{\frac{\partial^{#1} #2}{\partial #3^{#1}}}
\newcommand{\deld}[2]{\dfrac{\partial #1}{\partial #2}}
\newcommand{\Deld}[3]{\dfrac{\partial^{#1} #2}{\partial #3^{#1}}}


\newcommand{\AstIg}{\stackrel{(\ast)}{=}}
\newcommand{\Hop}{\stackrel{L'H\hat{o}pital}{=}}

\newcommand{\red}[1]{{\color{red}#1}} % Para integrales, destacar los cambios.

% Método de integración
\newcommand{\MetInt}[2]{
    \left[\begin{array}{c}
        #1 \\ #2
    \end{array}\right]
}

% Declarar aplicaciones
% 1. Nombre aplicación
% 2. Dominio
% 3. Codominio
% 4. Variable
% 5. Imagen de la variable
\newcommand{\Func}[5]{
    \begin{equation*}
        \begin{array}{rrll}
            #1:& #2 & \longrightarrow & #3\\
               & #4 & \longmapsto & #5
        \end{array}
    \end{equation*}
}

%------------------------------------------------------------------------



\begin{document}

    % 1. Foto de fondo
    % 2. Título
    % 3. Encabezado Izquierdo
    % 4. Color de fondo
    % 5. Coord x del titulo
    % 6. Coord y del titulo
    % 7. Fecha

    
    % 1. Foto de fondo
% 2. Título
% 3. Encabezado Izquierdo
% 4. Color de fondo
% 5. Coord x del titulo
% 6. Coord y del titulo
% 7. Fecha

\newcommand{\portada}[7]{

    \portadaBase{#1}{#2}{#3}{#4}{#5}{#6}{#7}
    \portadaBook{#1}{#2}{#3}{#4}{#5}{#6}{#7}
}

\newcommand{\portadaExamen}[7]{

    \portadaBase{#1}{#2}{#3}{#4}{#5}{#6}{#7}
    \portadaArticle{#1}{#2}{#3}{#4}{#5}{#6}{#7}
}




\newcommand{\portadaBase}[7]{

    % Tiene la portada principal y la licencia Creative Commons
    
    % 1. Foto de fondo
    % 2. Título
    % 3. Encabezado Izquierdo
    % 4. Color de fondo
    % 5. Coord x del titulo
    % 6. Coord y del titulo
    % 7. Fecha
    
    
    \thispagestyle{empty}               % Sin encabezado ni pie de página
    \newgeometry{margin=0cm}        % Márgenes nulos para la primera página
    
    
    % Encabezado
    \fancyhead[L]{\helv #3}
    \fancyhead[R]{\helv \nouppercase{\leftmark}}
    
    
    \pagecolor{#4}        % Color de fondo para la portada
    
    \begin{figure}[p]
        \centering
        \transparent{0.3}           % Opacidad del 30% para la imagen
        
        \includegraphics[width=\paperwidth, keepaspectratio]{assets/#1}
    
        \begin{tikzpicture}[remember picture, overlay]
            \node[anchor=north west, text=white, opacity=1, font=\fontsize{60}{90}\selectfont\bfseries\sffamily, align=left] at (#5, #6) {#2};
            
            \node[anchor=south east, text=white, opacity=1, font=\fontsize{12}{18}\selectfont\sffamily, align=right] at (9.7, 3) {\textbf{\href{https://losdeldgiim.github.io/}{Los Del DGIIM}}};
            
            \node[anchor=south east, text=white, opacity=1, font=\fontsize{12}{15}\selectfont\sffamily, align=right] at (9.7, 1.8) {Doble Grado en Ingeniería Informática y Matemáticas\\Universidad de Granada};
        \end{tikzpicture}
    \end{figure}
    
    
    \restoregeometry        % Restaurar márgenes normales para las páginas subsiguientes
    \pagecolor{white}       % Restaurar el color de página
    
    
    \newpage
    \thispagestyle{empty}               % Sin encabezado ni pie de página
    \begin{tikzpicture}[remember picture, overlay]
        \node[anchor=south west, inner sep=3cm] at (current page.south west) {
            \begin{minipage}{0.5\paperwidth}
                \href{https://creativecommons.org/licenses/by-nc-nd/4.0/}{
                    \includegraphics[height=2cm]{assets/Licencia.png}
                }\vspace{1cm}\\
                Esta obra está bajo una
                \href{https://creativecommons.org/licenses/by-nc-nd/4.0/}{
                    Licencia Creative Commons Atribución-NoComercial-SinDerivadas 4.0 Internacional (CC BY-NC-ND 4.0).
                }\\
    
                Eres libre de compartir y redistribuir el contenido de esta obra en cualquier medio o formato, siempre y cuando des el crédito adecuado a los autores originales y no persigas fines comerciales. 
            \end{minipage}
        };
    \end{tikzpicture}
    
    
    
    % 1. Foto de fondo
    % 2. Título
    % 3. Encabezado Izquierdo
    % 4. Color de fondo
    % 5. Coord x del titulo
    % 6. Coord y del titulo
    % 7. Fecha


}


\newcommand{\portadaBook}[7]{

    % 1. Foto de fondo
    % 2. Título
    % 3. Encabezado Izquierdo
    % 4. Color de fondo
    % 5. Coord x del titulo
    % 6. Coord y del titulo
    % 7. Fecha

    % Personaliza el formato del título
    \pretitle{\begin{center}\bfseries\fontsize{42}{56}\selectfont}
    \posttitle{\par\end{center}\vspace{2em}}
    
    % Personaliza el formato del autor
    \preauthor{\begin{center}\Large}
    \postauthor{\par\end{center}\vfill}
    
    % Personaliza el formato de la fecha
    \predate{\begin{center}\huge}
    \postdate{\par\end{center}\vspace{2em}}
    
    \title{#2}
    \author{\href{https://losdeldgiim.github.io/}{Los Del DGIIM}}
    \date{Granada, #7}
    \maketitle
    
    \tableofcontents
}




\newcommand{\portadaArticle}[7]{

    % 1. Foto de fondo
    % 2. Título
    % 3. Encabezado Izquierdo
    % 4. Color de fondo
    % 5. Coord x del titulo
    % 6. Coord y del titulo
    % 7. Fecha

    % Personaliza el formato del título
    \pretitle{\begin{center}\bfseries\fontsize{42}{56}\selectfont}
    \posttitle{\par\end{center}\vspace{2em}}
    
    % Personaliza el formato del autor
    \preauthor{\begin{center}\Large}
    \postauthor{\par\end{center}\vspace{3em}}
    
    % Personaliza el formato de la fecha
    \predate{\begin{center}\huge}
    \postdate{\par\end{center}\vspace{5em}}
    
    \title{#2}
    \author{\href{https://losdeldgiim.github.io/}{Los Del DGIIM}}
    \date{Granada, #7}
    \thispagestyle{empty}               % Sin encabezado ni pie de página
    \maketitle
    \vfill
}
    \portadaExamen{ffccA4.jpg}{Ecuaciones\\Diferenciales I\\Examen I}{Ecuaciones Diferenciales I. Examen I}{MidnightBlue}{-8}{28}{2024-2025}{Arturo Olivares Martos}

    \begin{description}
        \item[Asignatura] Ecuaciones Diferenciales I
        \item[Curso Académico] 2017-18.
        %\item[Grado] Doble Grado en Ingeniería Informática y Matemáticas.
        \item[Grupo] B.
        \item[Profesor] Rafael Ortega Ríos.
        \item[Descripción] Parcial A.
        \item[Fecha] 22 de marzo de 2018.
        %\item[Duración] 60 minutos.
    
    \end{description}
    \newpage
    
\begin{ejercicio}
    Se considera una solución cualquiera \(x(t)\) de la ecuación diferencial
    \[
        x' = 2tx.
    \]

    Se supone que dicha solución está definida en un intervalo abierto \(I\). Demuestra que, para cada $t\in I$, existe \(c \in \bb{R}\) tal que
    \[
        x(t) = cet^2.
    \]

    Definimos la siguiente función auxiliar:
    \Func{f}{I}{\bb{R}}{t}{e^{-t^2}~x(t)}

    Tenemos que \(f\) es derivable en \(I\) por ser producto de funciones derivables. Calculemos su derivada:
    \begin{equation*}
        f'(t) = -2te^{-t^2}~x(t) + e^{-t^2}~x'(t) = -2te^{-t^2}~x(t) + 2te^{-t^2}~x(t) = 0.
    \end{equation*}

    Por tanto, al ser $f'(t)=0$ para todo \(t\in I\), la función \(f\) es constante en \(I\). Es decir:
    \begin{equation*}
        \exists c \in \bb{R} \mid f(t) = c \quad \forall t \in I.
    \end{equation*}

    Multiplicando por $e^{t^2}$ ambos lados de la ecuación anterior, obtenemos que:
    \begin{equation*}
        x(t) = cet^2 \quad \forall t \in I.
    \end{equation*}
\end{ejercicio}

\begin{ejercicio}
    Demuestra que la transformación \(\varphi(t, x) = (s, y)\), \(s = t\), \(y = x + t\) define un difeomorfismo del plano que es compatible con la ecuación
    \[
        x' = (x + t)^2.
    \]

    Encuentra la solución de esta ecuación que cumple \(x(0) = 0\) y especifica su intervalo de definición.\\

    Veamos en primer lugar que es un difeomorfismo.
    Aunque no se menciona, entendemos el dominio maximal de $\varphi$, es decir:
    \Func{\varphi=(\varphi_1, \varphi_2)}{\bb{R}^2}{\bb{R}^2}{(t, x)}{(s,y)=(t,x+t)}

    Comenzamos por demostrar que $\varphi\in C^1(\bb{R}^2)$. Como ambas componentes de $\varphi$ son polinómicas, esto es directo. Veamos ahora que $\varphi$ es biyectiva buscando su inversa.
    Definimos la función:
    \Func{\psi}{\bb{R}^2}{\bb{R}^2}{(s,y)}{(t,x)=(s,y-s)}

    Como $\varphi\circ \psi = \psi\circ \varphi = Id_{\bb{R}^2}$, tenemos que $\psi=\varphi^{-1}$, por lo que $\varphi$ es biyectiva. Además, $\psi\in C^1(\bb{R}^2)$ por la misma razón (ambas componentes son polinómicas).
    Por tanto, $\varphi$ es un difeomorfismo. 
    
    Veamos ahora que es compatible con la ecuación diferencial dada. Definimos la función:
    \Func{f}{\bb{R}^2}{\bb{R}}{(t,x)}{(x+t)^2}

    Nuestra ecuación diferencial es $x' = f(t,x)$, y veamos ahora que el cambio de variable $\varphi$ es compatible con ella probando que:
    \begin{enumerate}
        \item Probar que $f$ es continua, lo que es directo al ser polinómica.
        \item Probar que $\varphi$ es un difeomorfismo, lo que ya hemos hecho.
        \item Comprobar que se cumple la condición de admisibilidad:
        \begin{equation*}
            \del{\varphi_1}{t}(t,x)+ \del{\varphi_1}{x}(t,x)f(t,x) \neq 0 \quad \forall (t,x)\in \bb{R}^2.
        \end{equation*}

        En nuestro caso, tenemos que:
        \begin{equation*}
            1+0\cdot (x+t)^2 = 1 \neq 0 \quad \forall (t,x)\in \bb{R}^2.
        \end{equation*}
    \end{enumerate}
    Por tanto, el cambio de variable es compatible con la ecuación diferencial dada.\\

    Para resolverla aplicando el cambio de variable, tenemos que la ecuación diferencial se transforma en:
    \begin{equation*}
        \dfrac{dy}{ds}(s) = \dfrac{\nicefrac{dy}{dt}(t)}{\nicefrac{ds}{dt}(t)} = \dfrac{dy}{dt}(t) = \del{y}{t}(t) + \del{y}{x}(t)x'(t) = 1 + 1\cdot (x+t)^2 = 1 + y^2
        \Longrightarrow y'=1+y^2.
    \end{equation*}
    El dominio de esta nueva ecuación es $\bb{R}^2$, y vemos que es una ecuación diferencial de variables separadas.

    \begin{comment}
    \begin{itemize}
        \item Soluciones Constantes:
        
        No tiene
        \item Soluciones No Constantes:
        \begin{equation*}
            \int \dfrac{dy}{1+y^2} = \int ds \Longrightarrow \arctan y = s + C
        \end{equation*}

        Usando la condición inicial $x(0)=0$, tenemos que $y(0)=x(0)+0=0$. Evaluando la solución en 0, tenemos que:
        \begin{equation*}
            \arctan y(0) = 0 + C \Longrightarrow C = 0
        \end{equation*}

        Despejando $y$, tenemos que $y(s) = \tan(s)$ para todo $s\in \left]-\dfrac{\pi}{2}, \dfrac{\pi}{2}\right[$ (hemos elegido este intervalo para que la tangente sea continua).
        Deshaciendo el cambio de variable, tenemos que $x(t) = \tan(t)-t$ para todo $t\in \left]-\dfrac{\pi}{2}, \dfrac{\pi}{2}\right[$.
    \end{itemize}
    \end{comment}
\end{ejercicio}

\begin{ejercicio}
    Encuentra un cambio de variable que transforme la ecuación diferencial
    \[
        x' = \dfrac{x + t + 3}{t - x + 2}
    \]
    en una ecuación homogénea.
\end{ejercicio}

\begin{ejercicio}
    Dadas las ecuaciones
    \[
        x = t + e^t, \quad y = 1 + t^4
    \]
    demuestra que la eliminación del parámetro \(t\) nos permite definir una función derivable \(y : \bb{R} \to \bb{R}\), \(x \mapsto y(x)\). Además, la función \(y(x)\) alcanza su mínimo en \(x = 1\).
\end{ejercicio}



\begin{ejercicio}
    Demuestra que la ecuación
    \[
        x - \dfrac{1}{3} \sen x = t
    \]
    define de forma implícita una única función \(x : \bb{R} \to \bb{R}\), \(t \mapsto x(t)\). Además, prueba que se cumple la identidad \(x(t + 2\pi) = x(t) + 2\pi\) para cada \(t \in \bb{R}\).\\

    Para verlo, hemos de demostrar que $x$ es una aplicación; es decir, que para cada valor de $t\in \bb{R}$, existe un único valor $x(t)\in \bb{R}$ tal que cumple dicha ecuación.
    Para ello, fijado $t\in \bb{R}$, definimos la función auxiliar:
    \Func{f_t}{\bb{R}}{\bb{R}}{x}{x - \dfrac{1}{3} \sen x - t}
    Demostrar la existencia y unicidad de $x(t)$ es equivalente a demostrar que $f_t$ tiene un único cero en $\bb{R}$.
    \begin{description}
        \item[Existencia] Tenemos que $f$ es continua, por lo que podemos aplicar el Teorema de Bolzano:
        \begin{equation*}
            \lim_{x\to -\infty} f_t(x) = -\infty, \quad \lim_{x\to +\infty} f_t(x) = +\infty.
        \end{equation*}
        Por tanto, por el Teorema de Bolzano, existe $x(t)\in \bb{R}$ tal que $f_t(x(t)) = 0$.

        \item[Unicidad] Veamos para ello que $f_t$ es estrictamente creciente. Para ello, como es derivable, tenemos que:
        \begin{equation*}
            f_t'(x) = 1 - \dfrac{1}{3} \cos x > 0 \quad \forall x\in \bb{R}.
        \end{equation*}
        Por tanto, $f_t$ es estrictamente creciente, lo que implica que tiene a lo sumo un cero. Por tanto, $x(t)$ es único.
    \end{description}

    % // TODO: Probar que x(t+2\pi) = x(t) + 2\pi
\end{ejercicio}

\end{document}
