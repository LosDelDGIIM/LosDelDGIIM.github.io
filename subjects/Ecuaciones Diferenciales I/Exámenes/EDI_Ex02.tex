\documentclass[12pt]{article}

% Idioma y codificación
\usepackage[spanish, es-tabla]{babel}       %es-tabla para que se titule "Tabla"
\usepackage[utf8]{inputenc}

% Márgenes
\usepackage[a4paper,top=3cm,bottom=2.5cm,left=3cm,right=3cm]{geometry}

% Comentarios de bloque
\usepackage{verbatim}

% Paquetes de links
\usepackage[hidelinks]{hyperref}    % Permite enlaces
\usepackage{url}                    % redirecciona a la web

% Más opciones para enumeraciones
\usepackage{enumitem}

% Personalizar la portada
\usepackage{titling}

% Paquetes de tablas
\usepackage{multirow}


%------------------------------------------------------------------------

%Paquetes de figuras
\usepackage{caption}
\usepackage{subcaption} % Figuras al lado de otras
\usepackage{float}      % Poner figuras en el sitio indicado H.


% Paquetes de imágenes
\usepackage{graphicx}       % Paquete para añadir imágenes
\usepackage{transparent}    % Para manejar la opacidad de las figuras

% Paquete para usar colores
\usepackage[dvipsnames]{xcolor}
\usepackage{pagecolor}      % Para cambiar el color de la página

% Habilita tamaños de fuente mayores
\usepackage{fix-cm}

% Para los gráficos
\usepackage{tikz}

% Para poder situar los nodos en los grafos
\usetikzlibrary{positioning}


%------------------------------------------------------------------------

% Paquetes de matemáticas
\usepackage{mathtools, amsfonts, amssymb, mathrsfs}
\usepackage[makeroom]{cancel}     % Simplificar tachando
\usepackage{polynom}    % Divisiones y Ruffini
\usepackage{units} % Para poner fracciones diagonales con \nicefrac

\usepackage{pgfplots}   %Representar funciones
\pgfplotsset{compat=1.18}  % Versión 1.18

\usepackage{tikz-cd}    % Para usar diagramas de composiciones
\usetikzlibrary{calc}   % Para usar cálculo de coordenadas en tikz

%Definición de teoremas, etc.
\usepackage{amsthm}
%\swapnumbers   % Intercambia la posición del texto y de la numeración

\theoremstyle{plain}

\makeatletter
\@ifclassloaded{article}{
  \newtheorem{teo}{Teorema}[section]
}{
  \newtheorem{teo}{Teorema}[chapter]  % Se resetea en cada chapter
}
\makeatother

\newtheorem{coro}{Corolario}[teo]           % Se resetea en cada teorema
\newtheorem{prop}[teo]{Proposición}         % Usa el mismo contador que teorema
\newtheorem{lema}[teo]{Lema}                % Usa el mismo contador que teorema

\theoremstyle{remark}
\newtheorem*{observacion}{Observación}

\theoremstyle{definition}

\makeatletter
\@ifclassloaded{article}{
  \newtheorem{definicion}{Definición} [section]     % Se resetea en cada chapter
}{
  \newtheorem{definicion}{Definición} [chapter]     % Se resetea en cada chapter
}
\makeatother

\newtheorem*{notacion}{Notación}
\newtheorem*{ejemplo}{Ejemplo}
\newtheorem*{ejercicio*}{Ejercicio}             % No numerado
\newtheorem{ejercicio}{Ejercicio} [section]     % Se resetea en cada section


% Modificar el formato de la numeración del teorema "ejercicio"
\renewcommand{\theejercicio}{%
  \ifnum\value{section}=0 % Si no se ha iniciado ninguna sección
    \arabic{ejercicio}% Solo mostrar el número de ejercicio
  \else
    \thesection.\arabic{ejercicio}% Mostrar número de sección y número de ejercicio
  \fi
}


% \renewcommand\qedsymbol{$\blacksquare$}         % Cambiar símbolo QED
%------------------------------------------------------------------------

% Paquetes para encabezados
\usepackage{fancyhdr}
\pagestyle{fancy}
\fancyhf{}

\newcommand{\helv}{ % Modificación tamaño de letra
\fontfamily{}\fontsize{12}{12}\selectfont}
\setlength{\headheight}{15pt} % Amplía el tamaño del índice


%\usepackage{lastpage}   % Referenciar última pag   \pageref{LastPage}
\fancyfoot[C]{\thepage}

%------------------------------------------------------------------------

% Conseguir que no ponga "Capítulo 1". Sino solo "1."
\makeatletter
\@ifclassloaded{book}{
  \renewcommand{\chaptermark}[1]{\markboth{\thechapter.\ #1}{}} % En el encabezado
    
  \renewcommand{\@makechapterhead}[1]{%
  \vspace*{50\p@}%
  {\parindent \z@ \raggedright \normalfont
    \ifnum \c@secnumdepth >\m@ne
      \huge\bfseries \thechapter.\hspace{1em}\ignorespaces
    \fi
    \interlinepenalty\@M
    \Huge \bfseries #1\par\nobreak
    \vskip 40\p@
  }}
}
\makeatother

%------------------------------------------------------------------------
% Paquetes de cógido
\usepackage{minted}
\renewcommand\listingscaption{Código fuente}

\usepackage{fancyvrb}
% Personaliza el tamaño de los números de línea
\renewcommand{\theFancyVerbLine}{\small\arabic{FancyVerbLine}}

% Estilo para C++
\newminted{cpp}{
    frame=lines,
    framesep=2mm,
    baselinestretch=1.2,
    linenos,
    escapeinside=||
}

% para minted
\definecolor{LightGray}{rgb}{0.95,0.95,0.92}
\setminted{
    linenos=true,
    stepnumber=5,
    numberfirstline=true,
    autogobble,
    breaklines=true,
    breakautoindent=true,
    breaksymbolleft=,
    breaksymbolright=,
    breaksymbolindentleft=0pt,
    breaksymbolindentright=0pt,
    breaksymbolsepleft=0pt,
    breaksymbolsepright=0pt,
    fontsize=\footnotesize,
    bgcolor=LightGray,
    numbersep=10pt
}


\usepackage{listings} % Para incluir código desde un archivo

\renewcommand\lstlistingname{Código Fuente}
\renewcommand\lstlistlistingname{Índice de Códigos Fuente}

% Definir colores
\definecolor{vscodepurple}{rgb}{0.5,0,0.5}
\definecolor{vscodeblue}{rgb}{0,0,0.8}
\definecolor{vscodegreen}{rgb}{0,0.5,0}
\definecolor{vscodegray}{rgb}{0.5,0.5,0.5}
\definecolor{vscodebackground}{rgb}{0.97,0.97,0.97}
\definecolor{vscodelightgray}{rgb}{0.9,0.9,0.9}

% Configuración para el estilo de C similar a VSCode
\lstdefinestyle{vscode_C}{
  backgroundcolor=\color{vscodebackground},
  commentstyle=\color{vscodegreen},
  keywordstyle=\color{vscodeblue},
  numberstyle=\tiny\color{vscodegray},
  stringstyle=\color{vscodepurple},
  basicstyle=\scriptsize\ttfamily,
  breakatwhitespace=false,
  breaklines=true,
  captionpos=b,
  keepspaces=true,
  numbers=left,
  numbersep=5pt,
  showspaces=false,
  showstringspaces=false,
  showtabs=false,
  tabsize=2,
  frame=tb,
  framerule=0pt,
  aboveskip=10pt,
  belowskip=10pt,
  xleftmargin=10pt,
  xrightmargin=10pt,
  framexleftmargin=10pt,
  framexrightmargin=10pt,
  framesep=0pt,
  rulecolor=\color{vscodelightgray},
  backgroundcolor=\color{vscodebackground},
}

%------------------------------------------------------------------------

% Comandos definidos
\newcommand{\bb}[1]{\mathbb{#1}}
\newcommand{\cc}[1]{\mathcal{#1}}

% I prefer the slanted \leq
\let\oldleq\leq % save them in case they're every wanted
\let\oldgeq\geq
\renewcommand{\leq}{\leqslant}
\renewcommand{\geq}{\geqslant}

% Si y solo si
\newcommand{\sii}{\iff}

% Letras griegas
\newcommand{\eps}{\epsilon}
\newcommand{\veps}{\varepsilon}
\newcommand{\lm}{\lambda}

\newcommand{\ol}{\overline}
\newcommand{\ul}{\underline}
\newcommand{\wt}{\widetilde}
\newcommand{\wh}{\widehat}

\let\oldvec\vec
\renewcommand{\vec}{\overrightarrow}

% Derivadas parciales
\newcommand{\del}[2]{\frac{\partial #1}{\partial #2}}
\newcommand{\Del}[3]{\frac{\partial^{#1} #2}{\partial #3^{#1}}}
\newcommand{\deld}[2]{\dfrac{\partial #1}{\partial #2}}
\newcommand{\Deld}[3]{\dfrac{\partial^{#1} #2}{\partial #3^{#1}}}


\newcommand{\AstIg}{\stackrel{(\ast)}{=}}
\newcommand{\Hop}{\stackrel{L'H\hat{o}pital}{=}}

\newcommand{\red}[1]{{\color{red}#1}} % Para integrales, destacar los cambios.

% Método de integración
\newcommand{\MetInt}[2]{
    \left[\begin{array}{c}
        #1 \\ #2
    \end{array}\right]
}

% Declarar aplicaciones
% 1. Nombre aplicación
% 2. Dominio
% 3. Codominio
% 4. Variable
% 5. Imagen de la variable
\newcommand{\Func}[5]{
    \begin{equation*}
        \begin{array}{rrll}
            #1:& #2 & \longrightarrow & #3\\
               & #4 & \longmapsto & #5
        \end{array}
    \end{equation*}
}

%------------------------------------------------------------------------



\begin{document}

    % 1. Foto de fondo
    % 2. Título
    % 3. Encabezado Izquierdo
    % 4. Color de fondo
    % 5. Coord x del titulo
    % 6. Coord y del titulo
    % 7. Fecha

    
    % 1. Foto de fondo
% 2. Título
% 3. Encabezado Izquierdo
% 4. Color de fondo
% 5. Coord x del titulo
% 6. Coord y del titulo
% 7. Fecha

\newcommand{\portada}[7]{

    \portadaBase{#1}{#2}{#3}{#4}{#5}{#6}{#7}
    \portadaBook{#1}{#2}{#3}{#4}{#5}{#6}{#7}
}

\newcommand{\portadaExamen}[7]{

    \portadaBase{#1}{#2}{#3}{#4}{#5}{#6}{#7}
    \portadaArticle{#1}{#2}{#3}{#4}{#5}{#6}{#7}
}




\newcommand{\portadaBase}[7]{

    % Tiene la portada principal y la licencia Creative Commons
    
    % 1. Foto de fondo
    % 2. Título
    % 3. Encabezado Izquierdo
    % 4. Color de fondo
    % 5. Coord x del titulo
    % 6. Coord y del titulo
    % 7. Fecha
    
    
    \thispagestyle{empty}               % Sin encabezado ni pie de página
    \newgeometry{margin=0cm}        % Márgenes nulos para la primera página
    
    
    % Encabezado
    \fancyhead[L]{\helv #3}
    \fancyhead[R]{\helv \nouppercase{\leftmark}}
    
    
    \pagecolor{#4}        % Color de fondo para la portada
    
    \begin{figure}[p]
        \centering
        \transparent{0.3}           % Opacidad del 30% para la imagen
        
        \includegraphics[width=\paperwidth, keepaspectratio]{assets/#1}
    
        \begin{tikzpicture}[remember picture, overlay]
            \node[anchor=north west, text=white, opacity=1, font=\fontsize{60}{90}\selectfont\bfseries\sffamily, align=left] at (#5, #6) {#2};
            
            \node[anchor=south east, text=white, opacity=1, font=\fontsize{12}{18}\selectfont\sffamily, align=right] at (9.7, 3) {\textbf{\href{https://losdeldgiim.github.io/}{Los Del DGIIM}}};
            
            \node[anchor=south east, text=white, opacity=1, font=\fontsize{12}{15}\selectfont\sffamily, align=right] at (9.7, 1.8) {Doble Grado en Ingeniería Informática y Matemáticas\\Universidad de Granada};
        \end{tikzpicture}
    \end{figure}
    
    
    \restoregeometry        % Restaurar márgenes normales para las páginas subsiguientes
    \pagecolor{white}       % Restaurar el color de página
    
    
    \newpage
    \thispagestyle{empty}               % Sin encabezado ni pie de página
    \begin{tikzpicture}[remember picture, overlay]
        \node[anchor=south west, inner sep=3cm] at (current page.south west) {
            \begin{minipage}{0.5\paperwidth}
                \href{https://creativecommons.org/licenses/by-nc-nd/4.0/}{
                    \includegraphics[height=2cm]{assets/Licencia.png}
                }\vspace{1cm}\\
                Esta obra está bajo una
                \href{https://creativecommons.org/licenses/by-nc-nd/4.0/}{
                    Licencia Creative Commons Atribución-NoComercial-SinDerivadas 4.0 Internacional (CC BY-NC-ND 4.0).
                }\\
    
                Eres libre de compartir y redistribuir el contenido de esta obra en cualquier medio o formato, siempre y cuando des el crédito adecuado a los autores originales y no persigas fines comerciales. 
            \end{minipage}
        };
    \end{tikzpicture}
    
    
    
    % 1. Foto de fondo
    % 2. Título
    % 3. Encabezado Izquierdo
    % 4. Color de fondo
    % 5. Coord x del titulo
    % 6. Coord y del titulo
    % 7. Fecha


}


\newcommand{\portadaBook}[7]{

    % 1. Foto de fondo
    % 2. Título
    % 3. Encabezado Izquierdo
    % 4. Color de fondo
    % 5. Coord x del titulo
    % 6. Coord y del titulo
    % 7. Fecha

    % Personaliza el formato del título
    \pretitle{\begin{center}\bfseries\fontsize{42}{56}\selectfont}
    \posttitle{\par\end{center}\vspace{2em}}
    
    % Personaliza el formato del autor
    \preauthor{\begin{center}\Large}
    \postauthor{\par\end{center}\vfill}
    
    % Personaliza el formato de la fecha
    \predate{\begin{center}\huge}
    \postdate{\par\end{center}\vspace{2em}}
    
    \title{#2}
    \author{\href{https://losdeldgiim.github.io/}{Los Del DGIIM}}
    \date{Granada, #7}
    \maketitle
    
    \tableofcontents
}




\newcommand{\portadaArticle}[7]{

    % 1. Foto de fondo
    % 2. Título
    % 3. Encabezado Izquierdo
    % 4. Color de fondo
    % 5. Coord x del titulo
    % 6. Coord y del titulo
    % 7. Fecha

    % Personaliza el formato del título
    \pretitle{\begin{center}\bfseries\fontsize{42}{56}\selectfont}
    \posttitle{\par\end{center}\vspace{2em}}
    
    % Personaliza el formato del autor
    \preauthor{\begin{center}\Large}
    \postauthor{\par\end{center}\vspace{3em}}
    
    % Personaliza el formato de la fecha
    \predate{\begin{center}\huge}
    \postdate{\par\end{center}\vspace{5em}}
    
    \title{#2}
    \author{\href{https://losdeldgiim.github.io/}{Los Del DGIIM}}
    \date{Granada, #7}
    \thispagestyle{empty}               % Sin encabezado ni pie de página
    \maketitle
    \vfill
}
    \portadaExamen{ffccA4.jpg}{Ecuaciones\\Diferenciales I\\Examen II}{Ecuaciones Diferenciales I. Examen II}{MidnightBlue}{-8}{28}{2024-2025}{Arturo Olivares Martos}

    \begin{description}
        \item[Asignatura] Ecuaciones Diferenciales I
        \item[Curso Académico] 2015-16.
        %\item[Grado] Doble Grado en Ingeniería Informática y Matemáticas.
        \item[Grupo] B.
        \item[Profesor] Rafael Ortega Ríos.
        \item[Descripción] Parcial A.
        \item[Fecha] 17 de marzo de 2016.
        %\item[Duración] 60 minutos.
    
    \end{description}
    \newpage
    

    \begin{ejercicio}
        Dadas las siguientes funciones $F$ y $\varphi$:
        \Func{F}{\bb{R}^2}{\bb{R}}{(x,y)}{F(x,y)}
        \Func{\varphi}{\bb{R}}{\bb{R}}{x}{\varphi(x)}
        se define la siguiente función $\phi$:
        \Func{\phi}{\bb{R}}{\bb{R}}{x}{F(x,\varphi(x))}
        Suponiendo que $F$ y $\varphi$ son de clase $C^2$, expresa $\phi''(x)$ en términos de las derivadas sucesivas de $F$ y $\varphi$.\\

        Veamos en primer lugar que $\phi\in C^2(\bb{R})$. Tenemos que:
        \begin{figure}[H]
            \centering
            \shorthandoff{""}
            \begin{tikzcd}
                \bb{R} \arrow[r, "H"] \arrow[rr, "F\circ H = \phi", bend left=49] & \bb{R}^2 \arrow[r, "F"]             & \bb{R}            \\
                x \arrow[r, maps to]                                              & {(x,\varphi(x))} \arrow[r, maps to] & {F(x,\varphi(x))}
            \end{tikzcd}
            \shorthandon{""}
        \end{figure}
        En primer lugar, $H=(Id, \varphi)$, donde $Id$ es la función identidad. Por tanto, se tiene que $H\in C^2(\bb{R})$ por ser ambas componentes $C^2(\bb{R})$. Por otro lado, $F\in C^2(\bb{R}^2)$ por hipótesis. Por tanto, $\phi=F\circ H\in C^2(\bb{R})$ por ser composición de funciones de clase $C^2(\bb{R})$.
        Tenemos por tanto que:
        \begin{equation*}
            \phi'(x) = \del{F}{x}(x,\varphi(x)) + \del{F}{y}(x,\varphi(x))\varphi'(x)
        \end{equation*}
        
        Derivando de nuevo, tenemos:
        \begin{align*}
            \phi''(x) &= \Del{2}{F}{x}(x,\varphi(x)) +
            \dfrac{\partial^2F}{\partial y \partial x} (x,\varphi(x))\varphi'(x) +
            \del{F}{y}(x,\varphi(x))\varphi''(x) +\\&\qquad +
            \varphi'(x)\left(\dfrac{\partial^2 F}{\partial x \partial y}(x,\varphi(x)) + \dfrac{\partial^2 F}{\partial y^2}(x,\varphi(x))\varphi'(x)\right)
        \end{align*}
    \end{ejercicio}

    \begin{ejercicio}
        Encuentra la solución de la siguiente ecuación diferencial, precisando el intervalo $I$ donde está definida:
        \begin{equation*}
            x' = e^{t+x}, \quad x(0) = 0
        \end{equation*}

        El dominio de la ecuación diferencial es $D=\bb{R}^2$.
        Se trata de una ecuación de variables separadas, donde no hay soluciones constantes ya que:
        \begin{equation*}
            e^x\neq 0 \quad \forall x\in \bb{R}
        \end{equation*}

        Resolvemos la ecuación usando el método demostrado en teoría:
        \begin{equation*}
            \int e^{-x}dx = \int e^t dt \Longrightarrow -e^{-x} = e^t + C
            \Longrightarrow x = -\ln(-e^t - C)
        \end{equation*}

        Veamos ahora el intervalo de definición $I\subset \bb{R}$. Para que la solución esté bien definida, necesitamos que el argumento del logaritmo sea positivo. Por tanto, necesitamos que:
        \begin{equation*}
            -e^t - C > 0 \Longrightarrow e^t < -C \Longrightarrow t < \ln(-C)
        \end{equation*}

        Por tanto, la solución está definida en el intervalo $I=]-\infty, \ln(-C)[$.
        \begin{equation*}
            x(t) = -\ln(-e^t - C), \quad t\in I, \quad C\in \bb{R}^-
        \end{equation*}

        Imponiendo la condición inicial, tenemos que:
        \begin{equation*}
            x(0) = -\ln(-e^0 - C) = 0 \Longrightarrow -\ln(-1 - C) = 0 \Longrightarrow -1-C=1 \Longrightarrow C=-2
        \end{equation*}

        Por tanto, la solución es:
        \begin{equation*}
            x(t) = -\ln(-e^t + 2), \quad t\in ]-\infty, \ln(2)[
        \end{equation*}
    \end{ejercicio}

    \begin{ejercicio} \label{ej:3}
        Encuentra una ecuación diferencial para las funciones $y=y(x)$ cuyas gráficas tienen la siguiente propiedad:
        la distancia al origen desde cada punto $(x,y(x))$ coincide con la primera coordenada del punto de corte de la recta tangente y el eje de abscisas.\\

        La representación gráfica de la situación descrita se encuentra en la Figura \ref{fig:ej:3}.
        \begin{figure}
            \centering
            \begin{tikzpicture}
                \begin{axis}[
                    axis lines = center,
                    xlabel = \(x\),
                    ylabel = \(y\),
                    xmin = -1.5, xmax = 2,
                    ymin = -1.5, ymax = 2,
                    xtick = {-1, 1, 2},
                    ytick = {-1, 1, 2},
                    xticklabels = {\(-1\), \(1\)},
                    yticklabels = {\(-1\), \(1\)},
                ]
                \addplot[domain=-2:2, samples=80, color=red]{x^2};

                % Marcamos el punto P(x_0, y(x_0))
                \addplot[mark=*] coordinates {(1, 1)} node[right] {\(P(x, y(x))\)};

                % Marcamos la recta tangente
                \addplot[domain=-2:3, samples=2, color=blue]{2*x - 1};
                % Marcamos el punto de corte de la recta tangente con el eje de abcisas
                \addplot[mark=*] coordinates {(0.5, 0)} node[above right, xshift=0.7em] {\(P_t(x_t, 0)\)};                
                \end{axis}
            \end{tikzpicture}
            \caption{Representación gráfica del enunciado del Ejercicio~\ref{ej:3}.}
            \label{fig:ej:3}
        \end{figure}

        La distancia al origen desde un punto $P=(x,y(x))$, notada por $d(P,O)$, viene dada por:
        \begin{equation*}
            d(P,O) = \sqrt{x^2 + (y(x))^2}
        \end{equation*}

        La recta tangente a la gráfica de $y=y(x)$ en el punto $(x,y(x))$ tiene pendiente $y'(x)$ y pasa por el punto $(x,y(x))$. Siendo $(x_t, 0)$ el punto de corte de la recta tangente con el eje de abscisas, como ambos puntos pertenecen a la misma recta, empleando la fórmula de la pendiente de una recta, tenemos que:
        \begin{equation*}
            y'(x) = \dfrac{y(x) - 0}{x - x_t} = \dfrac{y(x)}{x - x_t}
            \Longrightarrow
            x_t = x - \dfrac{y(x)}{y'(x)}
        \end{equation*}
        Notemos que el caso $x=x_t$ implicaría que la recta tangente es vertical, por lo que no podríamos considerar $y'(x)$ al no ser $y$ derivable en ese punto, por lo que no sería una solución posible.
        Respecto al caso $y'(x)=0$, tendríamos que la recta tangente es horizontal, por lo que no cortaría al eje de abscisas (o sería coincidente con él). En cualquier caso, tampoco se contempla, puesto que $x_t$ no estaría definido o no sería único.

        Imponiendo la condición del enunciado de que $d(P,O) = x_t$, tenemos que:
        \begin{equation*}
            \sqrt{x^2 + (y(x))^2} = -\dfrac{y(x)}{y'(x)} +x
        \end{equation*}

        La ecuación diferencial, en forma no normal, sería:
        \begin{equation*}
            \sqrt{x^2 + y^2} = -\dfrac{y}{y'} + x \qquad \text{con dominio } D=\left\{\begin{array}{c}
                \bb{R}\times \bb{R}\times \bb{R}^+\\
                \lor\\
                \bb{R}\times \bb{R}\times \bb{R}^-
            \end{array}
            \right.
        \end{equation*}

        Si necesitamos expresar la ecuación diferencial en forma normal, despejando $y'$ tenemos:
        \begin{equation*}
            y' = -\dfrac{y}{x - \sqrt{x^2 + y^2}}
            \qquad \text{con dominio } D=\left\{\begin{array}{c}
                \bb{R}\times \bb{R}^+\\
                \lor\\
                \bb{R}\times \bb{R}^-
            \end{array}
            \right.
        \end{equation*}
        donde hemos aplicado que el denominador solo se anula para $y=0$. Despejar $y'$ de la ecuación diferencial para dejarla en forma normal no es recomendable por perder las soluciones en las que $y=0$.
    \end{ejercicio}

    \begin{ejercicio}
        Se considera el cambio de variables $\varphi : s = e^t, y = e^{-t}x$. Demuestra que $\varphi$ define un difeomorfismo entre $\bb{R}^2$ y un dominio $\Omega$ del plano.
        Determina $\Omega$.
        Comprueba que se trata de un cambio admisible para la ecuación $x' = tx^2$ y encuentra la nueva ecuación en las variables $(s,y)$.\\

        En primer lugar, definimos el cambio de variable:
        \Func{\varphi=(\varphi_1,\varphi_2)}{\bb{R}^2}{\Omega}{(t,x)}{(s,y)=(e^t, e^{-t}x)}

        Busquemos en primer lugar su inversa, para lo cual buscamos despejar de forma única $t$ y $x$ en función de $s$ y $y$:
        \begin{align*}
            s &= e^t \Longrightarrow t = \ln s\\
            y &= e^{-t}x \Longrightarrow x= e^ty = e^{\ln s}y = sy
        \end{align*}

        Por tanto, tenemos que $\varphi$ es biyectiva, y su inversa es:
        \Func{\varphi^{-1}}{\Omega}{\bb{R}^2}{(s,y)}{(t,x)=\left(\ln s, sy\right)}

        Calculemos ahora $\Omega$. En primer lugar, para que $\varphi^{-1}$ esté bien definida, necesitamos $s>0$. Por tanto, $\Omega\subset \bb{R}^+\times \bb{R}$. Tenemos que:
        \begin{align*}
            \Omega &= \varphi(\bb{R}^2) = \left\{(s,y)\in \bb{R}^2 \mid \left(\ln s, sy\right)\in \bb{R}^2\right\} = \bb{R}^+\times \bb{R}
        \end{align*}
        
        Tenemos que $\varphi,\varphi^{-1}$ son biyectivas, y sus componentes son productos de funciones elementales de clase $1$, por lo que $\varphi,\varphi^{-1}\in C^1$. Por tanto, $\varphi$ define un difeomorfismo entre $\bb{R}^2$ y $\Omega$.\\

        Para comprobar que el cambio de variable es admisible para la ecuación $x'=tx^2$, tenemos que:
        \begin{align*}
            \dfrac{\partial \varphi_1}{\partial t} + \dfrac{\partial \varphi_1}{\partial x}x' 
            = e^t + 0\cdot x' = e^t > 0 \qquad \forall (t,x)\in \bb{R}^2
        \end{align*}

        La ecuación en las nuevas variables $(s,y)$ es:
        \begin{align*}
            \dfrac{dy}{ds} &= \dfrac{dy}{dt}\dfrac{dt}{ds} = \dfrac{-xe^{-t} + e^{-t}x'}{e^t} =\dfrac{e^{-t}(tx^2-x)}{e^t}
            = \dfrac{e^{-\ln s}(\ln (s)(sy)^2-sy)}{e^{\ln s}} =\\&= \dfrac{1}{s^2}\cdot sy(sy\ln (s)-1) = \dfrac{y(sy\ln (s)-1)}{s}
            \qquad \text{con dominio } \Omega = \bb{R}^+\times \bb{R}
        \end{align*}

    \end{ejercicio}

    \begin{ejercicio}
        Se considera la función seno hiperbólico $f=\senh$ definida como:
        \Func{f}{\bb{R}}{\bb{R}}{t}{f(t)=\senh t = \dfrac{e^t - e^{-t}}{2}}

        Demuestra que $f$ tiene una inversa\footnote{Es costumbre emplear la notación $g(x) = \text{argsh} x$, argumento del seno hiperbólico} $g:\bb{R}\to\bb{R}$, $t=g(x)$ y calcula $g'(x)$.\\

        Para demostrar esto, tenemos dos opciones.
        \begin{description}
            \item[Opción Teórica]
            
            En primer lugar hemos de demostrar que $f$ es biyectiva. Para ello, hemos de demostrar que $f$ es inyectiva y sobreyectiva.
            \begin{description}
                \item[Inyectividad] Como $f\in C^1(\bb{R})$, tenemos que:
                \begin{equation*}
                    f'(t) = \dfrac{e^t+e^{-t}}{2} > 0 \quad \forall t\in \bb{R}
                \end{equation*} 
                Por tanto, $f$ es estrictamente creciente, lo que implica que es inyectiva.

                \item[Sobreyectividad] Para demostrar que $f$ es sobreyectiva, basta con demostrar que su imagen es todo $\bb{R}$. Para ello, como es continua por ser suma de continuas, basta con estudiar sus límites en los extremos:
                \begin{equation*}
                    \lim_{t\to -\infty} f(t) = \dfrac{0-\lim\limits_{t\to \infty}e^{t}}{2} = -\infty, \quad \lim_{t\to +\infty} f(t) = \dfrac{\lim\limits_{t\to \infty}e^{t}-0}{2} = +\infty
                \end{equation*}

                Por tanto, tenemos que $f$ es sobreyectiva.
            \end{description}
            Entonces, $f$ tiene inversa, notada por $g$, tal que $g=f^{-1}$.
            Además, como $f'$ no se anula en ningún punto, tenemos que:
            \begin{equation*}
                g'(x) = \dfrac{1}{f'(g(x))}
                = \dfrac{2}{e^{g(x)}+e^{-g(x)}}
            \end{equation*}

            \item[Opción Directa]
            
            En primer lugar, despejamos $t$ en función de $f(t)=x$:
            \begin{align*}
                x=\dfrac{e^t - e^{-t}}{2} &\Longrightarrow 2x = e^t - e^{-t} \Longrightarrow e^{2t} - 2xe^t - 1 = 0
                \Longrightarrow \\& \Longrightarrow e^t = \dfrac{2x\pm \sqrt{4x^2+4}}{2} = x\pm \sqrt{x^2+1}
            \end{align*}
            Como $e^t>0$, necesitamos quedarnos con la solución positiva. Tenemos que:
            \begin{equation*}
                x<\sqrt{x^2+1} \Longleftrightarrow
                x^2<x^2+1 \Longleftrightarrow
                0<1
            \end{equation*}
            Por tanto, tenemos que:
            \begin{equation*}
                e^t = x + \sqrt{x^2+1} \Longrightarrow t = \ln(x + \sqrt{x^2+1})
            \end{equation*}

            Como hemos podido despejar de forma única $t$ en función de $x$, tenemos que $f$ tiene inversa, notada por $g=f^{-1}$, dada por:
            \Func{g}{\bb{R}}{\bb{R}}{x}{t=g(x) = \ln(x + \sqrt{x^2+1})}

            Tenemos que $g\in C^1(\bb{R})$, con derivada:
            \begin{equation*}
                g'(x) = \dfrac{1}{x + \sqrt{x^2+1}}\cdot \left(1 + \dfrac{x}{\sqrt{x^2+1}}\right) = \dfrac{1}{x+\sqrt{x^2+1}}\cdot \dfrac{\sqrt{x^2+1}+x}{\sqrt{x^2+1}} = \dfrac{1}{\sqrt{x^2+1}} \qquad \forall x\in \bb{R}
            \end{equation*}
        \end{description}
        
    \end{ejercicio}

\end{document}