\documentclass[12pt]{article}

% Idioma y codificación
\usepackage[spanish, es-tabla]{babel}       %es-tabla para que se titule "Tabla"
\usepackage[utf8]{inputenc}

% Márgenes
\usepackage[a4paper,top=3cm,bottom=2.5cm,left=3cm,right=3cm]{geometry}

% Comentarios de bloque
\usepackage{verbatim}

% Paquetes de links
\usepackage[hidelinks]{hyperref}    % Permite enlaces
\usepackage{url}                    % redirecciona a la web

% Más opciones para enumeraciones
\usepackage{enumitem}

% Personalizar la portada
\usepackage{titling}

% Paquetes de tablas
\usepackage{multirow}


%------------------------------------------------------------------------

%Paquetes de figuras
\usepackage{caption}
\usepackage{subcaption} % Figuras al lado de otras
\usepackage{float}      % Poner figuras en el sitio indicado H.


% Paquetes de imágenes
\usepackage{graphicx}       % Paquete para añadir imágenes
\usepackage{transparent}    % Para manejar la opacidad de las figuras

% Paquete para usar colores
\usepackage[dvipsnames]{xcolor}
\usepackage{pagecolor}      % Para cambiar el color de la página

% Habilita tamaños de fuente mayores
\usepackage{fix-cm}

% Para los gráficos
\usepackage{tikz}

% Para poder situar los nodos en los grafos
\usetikzlibrary{positioning}


%------------------------------------------------------------------------

% Paquetes de matemáticas
\usepackage{mathtools, amsfonts, amssymb, mathrsfs}
\usepackage[makeroom]{cancel}     % Simplificar tachando
\usepackage{polynom}    % Divisiones y Ruffini
\usepackage{units} % Para poner fracciones diagonales con \nicefrac

\usepackage{pgfplots}   %Representar funciones
\pgfplotsset{compat=1.18}  % Versión 1.18

\usepackage{tikz-cd}    % Para usar diagramas de composiciones
\usetikzlibrary{calc}   % Para usar cálculo de coordenadas en tikz

%Definición de teoremas, etc.
\usepackage{amsthm}
%\swapnumbers   % Intercambia la posición del texto y de la numeración

\theoremstyle{plain}

\makeatletter
\@ifclassloaded{article}{
  \newtheorem{teo}{Teorema}[section]
}{
  \newtheorem{teo}{Teorema}[chapter]  % Se resetea en cada chapter
}
\makeatother

\newtheorem{coro}{Corolario}[teo]           % Se resetea en cada teorema
\newtheorem{prop}[teo]{Proposición}         % Usa el mismo contador que teorema
\newtheorem{lema}[teo]{Lema}                % Usa el mismo contador que teorema

\theoremstyle{remark}
\newtheorem*{observacion}{Observación}

\theoremstyle{definition}

\makeatletter
\@ifclassloaded{article}{
  \newtheorem{definicion}{Definición} [section]     % Se resetea en cada chapter
}{
  \newtheorem{definicion}{Definición} [chapter]     % Se resetea en cada chapter
}
\makeatother

\newtheorem*{notacion}{Notación}
\newtheorem*{ejemplo}{Ejemplo}
\newtheorem*{ejercicio*}{Ejercicio}             % No numerado
\newtheorem{ejercicio}{Ejercicio} [section]     % Se resetea en cada section


% Modificar el formato de la numeración del teorema "ejercicio"
\renewcommand{\theejercicio}{%
  \ifnum\value{section}=0 % Si no se ha iniciado ninguna sección
    \arabic{ejercicio}% Solo mostrar el número de ejercicio
  \else
    \thesection.\arabic{ejercicio}% Mostrar número de sección y número de ejercicio
  \fi
}


% \renewcommand\qedsymbol{$\blacksquare$}         % Cambiar símbolo QED
%------------------------------------------------------------------------

% Paquetes para encabezados
\usepackage{fancyhdr}
\pagestyle{fancy}
\fancyhf{}

\newcommand{\helv}{ % Modificación tamaño de letra
\fontfamily{}\fontsize{12}{12}\selectfont}
\setlength{\headheight}{15pt} % Amplía el tamaño del índice


%\usepackage{lastpage}   % Referenciar última pag   \pageref{LastPage}
\fancyfoot[C]{\thepage}

%------------------------------------------------------------------------

% Conseguir que no ponga "Capítulo 1". Sino solo "1."
\makeatletter
\@ifclassloaded{book}{
  \renewcommand{\chaptermark}[1]{\markboth{\thechapter.\ #1}{}} % En el encabezado
    
  \renewcommand{\@makechapterhead}[1]{%
  \vspace*{50\p@}%
  {\parindent \z@ \raggedright \normalfont
    \ifnum \c@secnumdepth >\m@ne
      \huge\bfseries \thechapter.\hspace{1em}\ignorespaces
    \fi
    \interlinepenalty\@M
    \Huge \bfseries #1\par\nobreak
    \vskip 40\p@
  }}
}
\makeatother

%------------------------------------------------------------------------
% Paquetes de cógido
\usepackage{minted}
\renewcommand\listingscaption{Código fuente}

\usepackage{fancyvrb}
% Personaliza el tamaño de los números de línea
\renewcommand{\theFancyVerbLine}{\small\arabic{FancyVerbLine}}

% Estilo para C++
\newminted{cpp}{
    frame=lines,
    framesep=2mm,
    baselinestretch=1.2,
    linenos,
    escapeinside=||
}

% para minted
\definecolor{LightGray}{rgb}{0.95,0.95,0.92}
\setminted{
    linenos=true,
    stepnumber=5,
    numberfirstline=true,
    autogobble,
    breaklines=true,
    breakautoindent=true,
    breaksymbolleft=,
    breaksymbolright=,
    breaksymbolindentleft=0pt,
    breaksymbolindentright=0pt,
    breaksymbolsepleft=0pt,
    breaksymbolsepright=0pt,
    fontsize=\footnotesize,
    bgcolor=LightGray,
    numbersep=10pt
}


\usepackage{listings} % Para incluir código desde un archivo

\renewcommand\lstlistingname{Código Fuente}
\renewcommand\lstlistlistingname{Índice de Códigos Fuente}

% Definir colores
\definecolor{vscodepurple}{rgb}{0.5,0,0.5}
\definecolor{vscodeblue}{rgb}{0,0,0.8}
\definecolor{vscodegreen}{rgb}{0,0.5,0}
\definecolor{vscodegray}{rgb}{0.5,0.5,0.5}
\definecolor{vscodebackground}{rgb}{0.97,0.97,0.97}
\definecolor{vscodelightgray}{rgb}{0.9,0.9,0.9}

% Configuración para el estilo de C similar a VSCode
\lstdefinestyle{vscode_C}{
  backgroundcolor=\color{vscodebackground},
  commentstyle=\color{vscodegreen},
  keywordstyle=\color{vscodeblue},
  numberstyle=\tiny\color{vscodegray},
  stringstyle=\color{vscodepurple},
  basicstyle=\scriptsize\ttfamily,
  breakatwhitespace=false,
  breaklines=true,
  captionpos=b,
  keepspaces=true,
  numbers=left,
  numbersep=5pt,
  showspaces=false,
  showstringspaces=false,
  showtabs=false,
  tabsize=2,
  frame=tb,
  framerule=0pt,
  aboveskip=10pt,
  belowskip=10pt,
  xleftmargin=10pt,
  xrightmargin=10pt,
  framexleftmargin=10pt,
  framexrightmargin=10pt,
  framesep=0pt,
  rulecolor=\color{vscodelightgray},
  backgroundcolor=\color{vscodebackground},
}

%------------------------------------------------------------------------

% Comandos definidos
\newcommand{\bb}[1]{\mathbb{#1}}
\newcommand{\cc}[1]{\mathcal{#1}}

% I prefer the slanted \leq
\let\oldleq\leq % save them in case they're every wanted
\let\oldgeq\geq
\renewcommand{\leq}{\leqslant}
\renewcommand{\geq}{\geqslant}

% Si y solo si
\newcommand{\sii}{\iff}

% Letras griegas
\newcommand{\eps}{\epsilon}
\newcommand{\veps}{\varepsilon}
\newcommand{\lm}{\lambda}

\newcommand{\ol}{\overline}
\newcommand{\ul}{\underline}
\newcommand{\wt}{\widetilde}
\newcommand{\wh}{\widehat}

\let\oldvec\vec
\renewcommand{\vec}{\overrightarrow}

% Derivadas parciales
\newcommand{\del}[2]{\frac{\partial #1}{\partial #2}}
\newcommand{\Del}[3]{\frac{\partial^{#1} #2}{\partial #3^{#1}}}
\newcommand{\deld}[2]{\dfrac{\partial #1}{\partial #2}}
\newcommand{\Deld}[3]{\dfrac{\partial^{#1} #2}{\partial #3^{#1}}}


\newcommand{\AstIg}{\stackrel{(\ast)}{=}}
\newcommand{\Hop}{\stackrel{L'H\hat{o}pital}{=}}

\newcommand{\red}[1]{{\color{red}#1}} % Para integrales, destacar los cambios.

% Método de integración
\newcommand{\MetInt}[2]{
    \left[\begin{array}{c}
        #1 \\ #2
    \end{array}\right]
}

% Declarar aplicaciones
% 1. Nombre aplicación
% 2. Dominio
% 3. Codominio
% 4. Variable
% 5. Imagen de la variable
\newcommand{\Func}[5]{
    \begin{equation*}
        \begin{array}{rrll}
            #1:& #2 & \longrightarrow & #3\\
               & #4 & \longmapsto & #5
        \end{array}
    \end{equation*}
}

%------------------------------------------------------------------------



\begin{document}

    % 1. Foto de fondo
    % 2. Título
    % 3. Encabezado Izquierdo
    % 4. Color de fondo
    % 5. Coord x del titulo
    % 6. Coord y del titulo
    % 7. Fecha

    
    % 1. Foto de fondo
% 2. Título
% 3. Encabezado Izquierdo
% 4. Color de fondo
% 5. Coord x del titulo
% 6. Coord y del titulo
% 7. Fecha

\newcommand{\portada}[7]{

    \portadaBase{#1}{#2}{#3}{#4}{#5}{#6}{#7}
    \portadaBook{#1}{#2}{#3}{#4}{#5}{#6}{#7}
}

\newcommand{\portadaExamen}[7]{

    \portadaBase{#1}{#2}{#3}{#4}{#5}{#6}{#7}
    \portadaArticle{#1}{#2}{#3}{#4}{#5}{#6}{#7}
}




\newcommand{\portadaBase}[7]{

    % Tiene la portada principal y la licencia Creative Commons
    
    % 1. Foto de fondo
    % 2. Título
    % 3. Encabezado Izquierdo
    % 4. Color de fondo
    % 5. Coord x del titulo
    % 6. Coord y del titulo
    % 7. Fecha
    
    
    \thispagestyle{empty}               % Sin encabezado ni pie de página
    \newgeometry{margin=0cm}        % Márgenes nulos para la primera página
    
    
    % Encabezado
    \fancyhead[L]{\helv #3}
    \fancyhead[R]{\helv \nouppercase{\leftmark}}
    
    
    \pagecolor{#4}        % Color de fondo para la portada
    
    \begin{figure}[p]
        \centering
        \transparent{0.3}           % Opacidad del 30% para la imagen
        
        \includegraphics[width=\paperwidth, keepaspectratio]{assets/#1}
    
        \begin{tikzpicture}[remember picture, overlay]
            \node[anchor=north west, text=white, opacity=1, font=\fontsize{60}{90}\selectfont\bfseries\sffamily, align=left] at (#5, #6) {#2};
            
            \node[anchor=south east, text=white, opacity=1, font=\fontsize{12}{18}\selectfont\sffamily, align=right] at (9.7, 3) {\textbf{\href{https://losdeldgiim.github.io/}{Los Del DGIIM}}};
            
            \node[anchor=south east, text=white, opacity=1, font=\fontsize{12}{15}\selectfont\sffamily, align=right] at (9.7, 1.8) {Doble Grado en Ingeniería Informática y Matemáticas\\Universidad de Granada};
        \end{tikzpicture}
    \end{figure}
    
    
    \restoregeometry        % Restaurar márgenes normales para las páginas subsiguientes
    \pagecolor{white}       % Restaurar el color de página
    
    
    \newpage
    \thispagestyle{empty}               % Sin encabezado ni pie de página
    \begin{tikzpicture}[remember picture, overlay]
        \node[anchor=south west, inner sep=3cm] at (current page.south west) {
            \begin{minipage}{0.5\paperwidth}
                \href{https://creativecommons.org/licenses/by-nc-nd/4.0/}{
                    \includegraphics[height=2cm]{assets/Licencia.png}
                }\vspace{1cm}\\
                Esta obra está bajo una
                \href{https://creativecommons.org/licenses/by-nc-nd/4.0/}{
                    Licencia Creative Commons Atribución-NoComercial-SinDerivadas 4.0 Internacional (CC BY-NC-ND 4.0).
                }\\
    
                Eres libre de compartir y redistribuir el contenido de esta obra en cualquier medio o formato, siempre y cuando des el crédito adecuado a los autores originales y no persigas fines comerciales. 
            \end{minipage}
        };
    \end{tikzpicture}
    
    
    
    % 1. Foto de fondo
    % 2. Título
    % 3. Encabezado Izquierdo
    % 4. Color de fondo
    % 5. Coord x del titulo
    % 6. Coord y del titulo
    % 7. Fecha


}


\newcommand{\portadaBook}[7]{

    % 1. Foto de fondo
    % 2. Título
    % 3. Encabezado Izquierdo
    % 4. Color de fondo
    % 5. Coord x del titulo
    % 6. Coord y del titulo
    % 7. Fecha

    % Personaliza el formato del título
    \pretitle{\begin{center}\bfseries\fontsize{42}{56}\selectfont}
    \posttitle{\par\end{center}\vspace{2em}}
    
    % Personaliza el formato del autor
    \preauthor{\begin{center}\Large}
    \postauthor{\par\end{center}\vfill}
    
    % Personaliza el formato de la fecha
    \predate{\begin{center}\huge}
    \postdate{\par\end{center}\vspace{2em}}
    
    \title{#2}
    \author{\href{https://losdeldgiim.github.io/}{Los Del DGIIM}}
    \date{Granada, #7}
    \maketitle
    
    \tableofcontents
}




\newcommand{\portadaArticle}[7]{

    % 1. Foto de fondo
    % 2. Título
    % 3. Encabezado Izquierdo
    % 4. Color de fondo
    % 5. Coord x del titulo
    % 6. Coord y del titulo
    % 7. Fecha

    % Personaliza el formato del título
    \pretitle{\begin{center}\bfseries\fontsize{42}{56}\selectfont}
    \posttitle{\par\end{center}\vspace{2em}}
    
    % Personaliza el formato del autor
    \preauthor{\begin{center}\Large}
    \postauthor{\par\end{center}\vspace{3em}}
    
    % Personaliza el formato de la fecha
    \predate{\begin{center}\huge}
    \postdate{\par\end{center}\vspace{5em}}
    
    \title{#2}
    \author{\href{https://losdeldgiim.github.io/}{Los Del DGIIM}}
    \date{Granada, #7}
    \thispagestyle{empty}               % Sin encabezado ni pie de página
    \maketitle
    \vfill
}
    \portadaExamen{ffccA4.jpg}{Ecuaciones\\Diferenciales I\\Examen XV}{Ecuaciones Diferenciales I. Examen XV}{MidnightBlue}{-8}{28}{2024-2025}{Arturo Olivares Martos}

    \begin{description}
        \item[Asignatura] Ecuaciones Diferenciales I
        \item[Curso Académico] 2024-25.
        \item[Grado] Doble Grado en Ingeniería Informática y Matemáticas.
        \item[Grupo] Único.
        \item[Profesor] Rafael Ortega Ríos.
        \item[Descripción] Parcial 1.
        \item[Fecha] 29 de Octubre de 2024.
        \item[Duración] 120 minutos.
    
    \end{description}
    \newpage

    \begin{ejercicio}
        En el plano con coordenadas $(A, B)$ se considera la ecuación
        \begin{equation*}
            A^3 - \cos(AB) = 0.
        \end{equation*}
        ¿Es posible encontrar una función $\varphi : I \to \bb{R}$, $B \mapsto \varphi(B)$ con $\varphi(0) = 1$ y de manera que se cumpla la ecuación para cada $(A, B)$ con $A = \varphi(B)$, $B \in I$?
        
        $I = \left] - \delta, \delta\right[$ para algún $\delta > 0$.\\

        En este caso, $B$ es la variable independiente y $A$ es la variable dependiente. Necesitamos que:
        \begin{equation*}
            \varphi(0)=1\Longrightarrow \left\{
                \begin{aligned}
                    B=0\\
                    A=1
                \end{aligned}
            \right.
        \end{equation*}
        
        Definimos la función:
        \Func{F}{\bb{R}^2}{\bb{R}}{(B,A)}{A^3-\cos(AB)}

        Buscamos aplicar el Teorema de la Función Implícita a $F$ en el punto $(0,1)$. Comprobamos en primer lugar que el punto es solución de la ecuación:
        \begin{equation*}
            F(0,1)=1^3-\cos(0\cdot1)=1-1=0
        \end{equation*}

        Comprobamos ahora que $F$ es de clase $C^1$:
        \begin{equation*}
            \dfrac{\partial F}{\partial B}=A\sen(AB)\qquad
            \dfrac{\partial F}{\partial A}=3A^2+B\sen(AB)
        \end{equation*}
        Como ambas derivadas parciales son continuas, $F\in C^1(\bb{R}^2)$.
        Además, vemos que la derivada parcial de $F$ respecto de $A$ (la variable dependiente) en el punto $(0,1)$ no se anula:
        \begin{equation*}
            \dfrac{\partial F}{\partial A}(0,1)=3\cdot1^2+0\cdot\sen(0\cdot1)=3\neq0
        \end{equation*}

        Aplicando el Teorema de la Función Implícita a $F$ en el punto $(0,1)$, obtenemos que la ecuación $F(B,A)=0$ define una función implícita
        \Func{\varphi}{I}{\bb{R}}{B}{\varphi(B)}
        dofinida en un entorno $I$ de $0$ y tal que $\varphi(0)=1$. Definiendo $A=\varphi(B)$, tenemos que dicha función cumple:
        \begin{equation*}
            F(B,A) = F(B,\varphi(B)) = 0\qquad \forall B\in I
        \end{equation*}
    \end{ejercicio}

    \begin{ejercicio}
        Se considera la familia uniparamétrica de curvas
        \begin{equation*}
            y = \dfrac{x^2}{2} + c, \quad c \in \bb{R}.
        \end{equation*}
        Encuentra la familia de trayectorias ortogonales y dibuja las dos familias en un plano común con coordenadas $(x, y)$.\\

        Derivamos implícitamente dicha familia de curvas para obtener una ecuación diferencial de la cual sean solución:
        \begin{equation*}
            1\cdot y' = \dfrac{2x}{2}\Longrightarrow y'=x
        \end{equation*}

        Por tanto, la familia dada son soluciones de la ecuación diferencial siguiente:
        \begin{equation*}
            y'=x\qquad \text{con dominio}\ D=\bb{R}^2
        \end{equation*}

        Usando que el producto de las pendientes de rectas perpendiculares es $-1$, y la interpretación geométrica de la derivada como la pendiente de la recta tangente a la curva en un punto, obtenemos que las trayectorias ortogonales a la familia dada son soluciones de la ecuación diferencial:
        \begin{equation*}
            y'=-\dfrac{1}{x}\qquad \text{con dominio}\ D=\left\{
                \begin{array}{rcl}
                    D_{-} & = & \bb{R}^-\times \bb{R}\\
                    D_{+} & = & \bb{R}^+\times \bb{R}
                \end{array}
            \right.
        \end{equation*}

        Resolvemos ahora dicha ecuación diferencial, que es un cálculo de primitivas:
        \begin{equation*}
            y(x) = \int-\dfrac{1}{x}dx=-\ln|x|+K\qquad K\in\bb{R}
        \end{equation*}
        donde $|x|$ será $x$ o $-x$ en función del dominio. Tenemos por tanto que la familia de trayectorias ortogonales a la dada son:
        \begin{itemize}
            \item \ul{$D=D_{-}$}: $y(x)=-\ln(-x)+K\qquad \forall x\in \bb{R}^-$, $K\in\bb{R}$
            \item \ul{$D=D_{+}$}: $y(x)=-\ln(x)+K\qquad \forall x\in \bb{R}^+$, $K\in\bb{R}$
        \end{itemize}

        Buscamos ahora dibujar ambas familias.
        \begin{itemize}
            \item La familia dada son parábolas convexas con vértice en el punto $(0,c)$ y eje de simetría en la recta $x=0$. Las pintaremos en azul.
            \item Las trayectorias ortogonales son $y(x)=-\ln|x|$ desplazadas verticalmente. Las pintaremos en rojo.
        \end{itemize}
        \begin{figure}[H]
            \centering
            \begin{tikzpicture}
                \begin{axis}[
                    axis lines = center,
                    xlabel = $x$,
                    ylabel = $y$,
                    xmin = -5, xmax = 5,
                    ymin = -5, ymax = 5,
                    domain=-5:5, 
                    samples=100,
                    legend pos=outer north east, % Cambia la posición de la leyenda
                ]
                % Curvas dadas
                \addplot [color=blue]{x^2/2+1};
                \addlegendentry{$C=1$}
                \addplot [color=blue, dashed]{x^2/2+2};
                \addlegendentry{$C=2$}
                \addplot [color=blue, dotted]{x^2/2+-1};
                \addlegendentry{$C=-1$}
                % Curvas ortogonales
                \addplot [color=red]{-ln(-x)};
                \addplot [color=red]{-ln(x)};
                \addlegendentry{$K=0$}
                \addplot [color=red, dashed]{-ln(-x)+1};
                \addplot [color=red, dashed]{-ln(x)+1};
                \addlegendentry{$K=1$}
                \addplot [color=red, dotted]{-ln(-x)-1};
                \addplot [color=red, dotted]{-ln(x)-1};
                \addlegendentry{$K=-1$}
                \end{axis}
            \end{tikzpicture}
        \end{figure}
    \end{ejercicio}

    \begin{ejercicio}
        Encuentra la solución del problema de valores iniciales
        \begin{equation*}
            x' = \dfrac{1 + x^2}{1 + t^2}, \quad x(0) = 1.
        \end{equation*}
        ¿En qué intervalo está definida?\\

        El dominio de la ecuación diferencial es $\bb{R}^2$.
        Se trata de una ecuación de variables separadas sin soluciones constantes, ya que:
        \begin{equation*}
            1+x^2=0\qquad \text{no tiene solución en }\bb{R}
        \end{equation*}

        Por tanto, podemos separar variables y resolver la ecuación:
        \begin{align*}
            \dfrac{dx}{1+x^2}=\dfrac{dt}{1+t^2}
            &\Longrightarrow \int\dfrac{dx}{1+x^2}=\int\dfrac{dt}{1+t^2}\\
            &\Longrightarrow \arctan x=\arctan t+C\qquad C\in\bb{R}
        \end{align*}

        Aplicamos ahora la condición inicial $x(0)=1$:
        \begin{equation*}
            \arctan 1=\arctan 0+C\Longrightarrow \dfrac{\pi}{4}=0+C
            \Longrightarrow C=\dfrac{\pi}{4}
        \end{equation*}

        Como la inversa de la $\arctan$ es la rama principal de la tangente, obtenemos que la solución del problema de valores iniciales es:
        \begin{equation*}
            x(t)=\tan\left(\arctan t+\dfrac{\pi}{4}\right)\qquad \forall t\in I\subset \bb{R}
        \end{equation*}
        donde $I\subset \bb{R}$ es el intervalo en el que la solución está definida. Para determinar dicho intervalo, como $\arctan(x)=\arctan(t)+\nicefrac{\pi}{4}\in \left]\nicefrac{-\pi}{2},\nicefrac{\pi}{2}\right[$, tenemos que:
        \begin{equation*}
            -\dfrac{\pi}{2}<\arctan(t)+\dfrac{\pi}{4}<\dfrac{\pi}{2}\Longrightarrow -\dfrac{3\pi}{4}<\arctan(t)<\dfrac{\pi}{4}
        \end{equation*}
        Resolvemos ambas desigualdades por separado para obtener $I$:
        \begin{itemize}
            \item Por un lado, tenemos que:
            \begin{equation*}
                -\dfrac{3\pi}{4}<-\dfrac{\pi}{2}<\arctan(t)\qquad \forall t\in \bb{R}
            \end{equation*}
            Por tanto, esta desigualdad no restringe el intervalo.

            \item Por otro lado, tenemos:
            \begin{equation*}
                \arctan(t)<\dfrac{\pi}{4}
            \end{equation*}
            Usando que la rama principal de la tangente es creciente, obtenemos que:
            \begin{equation*}
                t<\tan\left(\dfrac{\pi}{4}\right)=1
            \end{equation*}
            Por tanto, tenemos $t<1$.
        \end{itemize}
        Por tanto, el intervalo de definición de la solución es:
        \begin{equation*}
            I=\left]-\infty,1\right[
        \end{equation*}        
    \end{ejercicio}

    \begin{ejercicio}
        Se considera la transformación
        \begin{equation*}
            \varphi : D \to \bb{R}^2, \quad \varphi(t, x) = (e^{t}x, \arctan x),
        \end{equation*}
        donde $D = \bb{R} \times \left]0, \infty\right[$. Se pide:
        \begin{enumerate}
            \item Describe el conjunto $D_1 = \varphi(D)$ y prueba que $\varphi$ es un difeomorfismo entre $D$ y $D_1$.\\
            
            Tenemos que la transformación en cuestión es:
            \Func{\varphi=(\varphi_1,\varphi_2)}{D}{\bb{R}^2}{(t,x)}{(s,y)=(e^tx,\arctan x)}

            Sea $D_1=\bb{R}^+\times\left]0,\nicefrac{\pi}{2}\right[$. Probaremos que $D_1=\varphi(D)$ por doble inclusión:
            \begin{description}
                \item[$\subset)$] Sea $(s,y)\in D_1$, y para ver que $(s,y)\in\varphi(D)$, necesitamos encontrar $(t,x)\in D$ tal que $\varphi(t,x)=(s,y)$. Definimos:
                \begin{equation*}
                    t=\ln\left(\dfrac{s}{\tan y}\right)\qquad x=\tan y
                \end{equation*}
                Veamos que estos valores están bien definidos y que $(t,x)\in D$:
                \begin{itemize}
                    \item Como $y\in \left]0,\nicefrac{\pi}{2}\right[$, tenemos que $x=\tan y$ está bien definido y $x=\tan y>0$.
                    \item Como $\tan y>0$, tenemos que el denominador no se anula y es siempre positivo. Además, como $s>0$, tenemos que el cociente es positivo, luego $t=\ln\left(\dfrac{s}{\tan y}\right)$ está bien definido.
                \end{itemize}
                Por tanto, tenemos $(t,x)\in D$. Comprobemos que $\varphi(t,x)=(s,y)$:
                \begin{align*}
                    \varphi(t,x)
                    &=(e^t x,\arctan x)\\
                    &=(e^{\ln\left(\frac{s}{\tan y}\right)}\tan y,\arctan\tan y)\\
                    &=\left(\dfrac{s}{\tan y}\cdot \tan y,\arctan(\tan y)\right)\\
                    &\AstIg(s,y)
                \end{align*}
                donde en $(\ast)$ hemos usado que, como $y\in ]\nicefrac{-\pi}{2},\nicefrac{\pi}{2}[$, tenemos que $\arctan(\tan y)=y$.

                \item[$\supset)$] Sea $(s,y)\in \varphi(D)$, por lo que $\exists (t,x)\in D$ tal que $\varphi(t,x)=(s,y)$. Por tanto, tenemos:
                \begin{equation*}
                    \left\{
                        \begin{aligned}
                            s=e^tx\\
                            y=\arctan x
                        \end{aligned}
                    \right.
                \end{equation*}
                Veamos que $(s,y)\in D_1$:
                \begin{itemize}
                    \item Como $x\in \bb{R}^+$ y $e^t>0~\forall t\in\bb{R}$, tenemos que $s=e^tx>0$.
                    \item Como $x>0$, usando que $\arctan$ es creciente en $\bb{R}$, tenemos que $y=\arctan x>\arctan 0=0$. Además, por la definición de $\arctan$, tenemos que $y=\arctan x<\nicefrac{\pi}{2}$. Por tanto, $y\in\left]0,\nicefrac{\pi}{2}\right[$.
                \end{itemize}
                Por tanto, $(s,y)\in D_1$.
            \end{description}

            Veamos ahora que se trata de un difeomorfismo. En primer lugar, como $D_1=\varphi(D)$, tenemos que la restricción que consideramos es sobreyectiva. Además, como $\varphi_2$ es inyectiva, tenemos que $\varphi$ es inyectiva. Por tanto, $\varphi$ es biyectiva, con inversa:
            \Func{\varphi^{-1}=((\varphi^{-1})_1,(\varphi^{-1})_2)}{D_1}{D}{(s,y)}{(t,x)=\left(\ln\left(\dfrac{s}{\tan y}\right),\tan y\right)}

            Además, tenemos que $\varphi$ y $\varphi^{-1}$ son de clase $1$ en sus respectivos dominios por serlo cada una de las componentes. Por tanto, $\varphi$ es un difeomorfismo entre $D$ y $D_1$.

            \item Dada una ecuación $x' = f(t, x)$ con $f : D \to \bb{R}$, ¿qué condiciones hay que imponer para que se pueda asegurar que el cambio $(s, y) = \varphi(t, x)$ es admisible?
            
            Para que el cambio de variable sea admisible, necesitamos en primer que $\varphi$ sea un difeomorfismo entre $D$ y $D_1$, algo que ya hemos probado en el apartado anterior. La condición de admisibilidad es:
            \begin{equation*}
                \dfrac{\partial \varphi_1}{\partial t}+\dfrac{\partial \varphi_1}{\partial x}f(t,x)\neq 0\qquad \forall (t,x)\in D
            \end{equation*}

            En nuestro caso, tenemos:
            \begin{equation*}
                \dfrac{\partial \varphi_1}{\partial t}=e^tx\qquad \dfrac{\partial \varphi_1}{\partial x}=e^t
            \end{equation*}

            Por tanto, aparte de imponer que $f$ sea continua en $D$, necesitamos que:
            \begin{equation*}
                e^tx+e^tf(t,x)\neq 0\Longleftrightarrow
                x+f(t,x)\neq 0\qquad \forall (t,x)\in D
            \end{equation*}
            donde en la doble implicación hemos usado que $e^t>0~\forall t\in\bb{R}$.
        \end{enumerate}
    \end{ejercicio}

    \begin{ejercicio}
        Dada una función $\phi : \bb{R} \to \bb{R}$ de clase $C^1$, se considera el cambio de variable
        \begin{equation*}
            s = t + \phi(x), \quad y = x.
        \end{equation*}
        \begin{enumerate}
            \item Prueba que $\varphi : \bb{R}^2 \to \bb{R}^2$, $\varphi(t, x) = (s, y)$ es un difeomorfismo.
            
            El cambio de variable es:
            \Func{\varphi=(\varphi_1,\varphi_2)}{\bb{R}^2}{\bb{R}^2}{(t,x)}{(s,y)=(t+\phi(x),x)}

            En primer lugar, tenemos que $\varphi$ es inyectiva por serlo $\varphi_2$. Veamos ahora que es sobreyectiva. Dado $(s,y)\in\bb{R}^2$, necesitamos encontrar $(t,x)\in\bb{R}^2$ tal que $\varphi(t,x)=(s,y)$. Definimos:
            \begin{equation*}
                t=s-\phi(y)\qquad x=y
            \end{equation*}
            De forma directa se tiene que estos valores están bien definidos y que $(t,x)\in\bb{R}^2$. Comprobamos que $\varphi(t,x)=(s,y)$:
            \begin{align*}
                \varphi(t,x)
                &=(t+\phi(x),x)\\
                &=(s-\phi(y)+\phi(y),y)\\
                &=(s,y)
            \end{align*}

            Por tanto, $\varphi$ es sobreyectiva. Por tanto, $\varphi$ es biyectiva, con inversa:
            \Func{\varphi^{-1}}{\bb{R}^2}{\bb{R}^2}{(s,y)}{(t,x)=(s-\phi(y),y)}

            Además, tenemos que $\varphi,\varphi^{-1}\in C^1(\bb{R}^2)$ por serlo las proyecciones, la suma, diferencia, composisión y la función $\phi$. Por tanto, $\varphi$ es un difeomorfismo del plano.

            \item Encuentra una función $\phi$ en las condiciones anteriores que permita transformar la ecuación $x' = x$ en la ecuación
            \begin{equation*}
                \dfrac{dy}{ds} = \dfrac{y}{1 + y \cos y}.
            \end{equation*}
            Describe los dominios sobre los que este cambio es admisible.\\

            Supongamos que el cambio es admisible (más adelante veremos las condiciones que hay que imponer). Entonces, la ecuación transformada queda:
            \begin{equation*}
                \dfrac{dy}{ds}=\dfrac{dy}{dt}\cdot \dfrac{dt}{ds}=x'\cdot \dfrac{1}{1+\phi'(x)x}
            \end{equation*}
            Usando que $x'=x$, $x=y$ y la ecuación impuesta por el enunciado, obtenemos:
            \begin{equation*}
                y'=\dfrac{y}{1+y\phi'(y)} = \dfrac{y}{1+y\cos y}
            \end{equation*}

            Por tanto, $\phi$ es una función tal que $\phi'(y)=\cos y$ para todo $y\in\bb{R}$. Por el Teorema Fundamental del Cálculo, como $\sen y$ es una primitiva de $\cos y$, tenemos que:
            \Func{\phi}{\bb{R}}{\bb{R}}{y}{\sen y}
            es una función válida, ya que $\phi\in C^1(\bb{R})$ y cumple que la ecuación transformada usando el cambio de variable descrito es la ecuación dada.

            Veamos ahora la condición de admisibilidad. Necesitamos que:
            \begin{equation*}
                \dfrac{\partial \varphi_1}{\partial t}+\dfrac{\partial \varphi_1}{\partial x}f(t,x)\neq 0\qquad \forall (t,x)\in D
            \end{equation*}
            siendo $D$ el dominio en el que este cambio es admisible. En nuestro caso, tenemos:
            \begin{equation*}
                \dfrac{\partial \varphi_1}{\partial t}=1\qquad \dfrac{\partial \varphi_1}{\partial x}=\varphi'(x)=\cos x\qquad f(t,x)=x
            \end{equation*}

            Por tanto, necesitamos que:
            \begin{equation*}
                1+x\cos x\neq 0\qquad \forall (t,x)\in D
            \end{equation*}

            Por tanto, como sobre $t$ no imponemos condiciones, tenemos que $D=\bb{R}\times D_X$, donde $D_X\subset \bb{R}$ es un abierto y conexo de forma que:
            \begin{equation*}
                D_X\subset \{x\in\bb{R} \mid 1+x\cos x\neq 0\}
            \end{equation*}

            La ecuación $1+x\cos x=0$ es una ecuación trascentendental cuyas soluciones no podemos encontrar, por lo que no podemos determinar el dominio de forma explícita. No obstante, para intuirlos, tenemos que:
            \begin{equation*}
                1+x\cos x=0\Longleftrightarrow \cos x=-\dfrac{1}{x}
            \end{equation*}
            Por tanto, como estas dos funciones sí sabemos graficarlas, las soluciones serán las abcisas de los puntos de corte de dichas dos gráficas, que se muestran en la Figura~\ref{fig:graficas}.
            \begin{figure}[H]
                \centering
                \begin{tikzpicture}
                    \begin{axis}[
                        axis lines = center,
                        xlabel = $x$,
                        ylabel = $y$,
                        xmin = -10, xmax = 10,
                        ymin = -10, ymax = 10,
                        samples=200,
                        % Que no haya marcas en los ejes
                        xtick=\empty,
                        ytick=\empty,
                        legend pos=outer north east, % Cambia la posición de la leyenda
                    ]
                    % Curvas dadas
                    
                    \addplot [color=red, domain=-20:20]{cos(deg(x))};
                    \addlegendentry{$y=\cos x$}
                    \addplot [color=blue, domain=-20:0]{-1/x};
                    \addplot [color=blue, domain=0:20]{-1/x};
                    \addlegendentry{$y=\nicefrac{-1}{x}$}
                    \end{axis}
                \end{tikzpicture}
                \caption{Gráficas de $y=\nicefrac{-1}{x}$ y $y=\cos x$.}
                \label{fig:graficas}
            \end{figure}
            \begin{comment}

            Definimos la función auxiliar:
            \Func{f}{\bb{R}}{\bb{R}}{x}{1+x\cos x}

            Tenemos que $f$ es continua. Veamos que, para cada $k\in \bb{Z}\setminus \{1\}$, $\exists! x_k\inI_k$, donde $I_k$ es el intervalo abierto de extremos $\pi\cdot k$ y $\pi\cdot(k+1),$ tal que $f(x_k)=0$. Distinguimos en función de $k$:
            \begin{itemize}
                \item Si $k$ es par:
                \begin{align*}
                    f(\pi k) &= 1+\pi k\cos(\pi k) = 1+\pi k\cos 0 = 1+\pi k\\
                    f(\pi(k+1)) &= 1+\pi(k+1)\cos(\pi(k+1)) = 1+\pi(k+1)\cos \pi = 1-\pi(k+1)
                \end{align*}

                Por tanto, tenemos que:
                \begin{align*}
                    f(\pi k)f(\pi(k+1))&=(1+\pi k)(1-\pi(k+1))=(1+\pi k)(1-\pi k-\pi)
                    =\\&=(1+\pi k)(1-\pi k)-\pi(1+\pi k)=1-\pi^2k^2-\pi-\pi^2k
                \end{align*}

                Veamos qué hemos de imponer a $k$ para que se tenga que ese producto es negativo. Resolvemos en primer lugar la ecuación:
                \begin{align*}
                    1-\pi^2k^2-\pi-\pi^2k=0\Longleftrightarrow k&=\dfrac{\pi^2\pm \sqrt{\pi^4+4\pi^2(1-\pi)}}{-2\pi^2}
                    =\dfrac{\pi\pm \sqrt{\pi^2+4-4\pi}}{-2\pi}
                    =\\&= \dfrac{\pi\pm \sqrt{(\pi-2)^2}}{-2\pi}
                    = \dfrac{\pi\pm (\pi-2)}{-2\pi}
                    =\\&=\left\{
                        \begin{array}{rcl}
                            k_1 & = & -\dfrac{1}{\pi}\approx -0.31\\ \\
                            k_2 & = & \dfrac{1-\pi}{\pi}\approx-0.68
                        \end{array}
                    \right.
                \end{align*}
                Evaluamos en $\dfrac{k_1+k_2}{2}$:
                \begin{equation*}
                    \dfrac{k_1+k_2}{2}=\dfrac{-1+1-\pi}{2\pi}=-\dfrac{\pi}{2\pi}=-\dfrac{1}{2}
                    \Longrightarrow
                    1-\dfrac{\pi^2}{4} -\pi+\dfrac{\pi^2}{2}=1+\dfrac{\pi^2}{4}-\pi\approx 0.32>0
                \end{equation*}
                Por tanto, tenemos que $f(\pi k)f(\pi(k+1))<0$ para $k\in\bb{Z}$ par.

                \item Si $k$ es impar:
                \begin{align*}
                    f(\pi k) &= 1+\pi k\cos(\pi k) = 1+\pi k\cos (\pi) = 1-\pi k\\
                    f(\pi(k+1)) &= 1+\pi(k+1)\cos(\pi(k+1)) = 1+\pi(k+1)\cos 0 = 1+\pi(k+1)
                \end{align*}

                Por tanto, tenemos que:
                \begin{align*}
                    f(\pi k)f(\pi(k+1))&=(1-\pi k)(1+\pi(k+1))=(1-\pi k)(1+\pi k+\pi)
                    =\\&=(1-\pi k)(1+\pi k)+\pi(1-\pi k)=1-\pi^2k^2+\pi-\pi^2k
                \end{align*}
                Veamos qué hemos de imponer a $k$ para que se tenga que ese producto es negativo. Resolvemos en primer lugar la ecuación:
                \begin{align*}
                    1-\pi^2k^2+\pi-\pi^2k=0\Longleftrightarrow k&=\dfrac{\pi^2\pm \sqrt{\pi^4+4\pi^2(1+\pi)}}{-2\pi^2}
                    =\dfrac{\pi\pm \sqrt{\pi^2+4+4\pi}}{-2\pi}
                    =\\&= \dfrac{\pi\pm \sqrt{(\pi+2)^2}}{-2\pi}
                    = \dfrac{\pi\pm (\pi+2)}{-2\pi}
                    =\\&=\left\{
                        \begin{array}{rcl}
                            k_1 & = & -\dfrac{1+\pi}{\pi}\approx -1.31<-1 \\
                            k_2 & = & \dfrac{1}{\pi}\approx 0.31>0
                        \end{array}
                    \right.
                \end{align*}
                Evaluamos en $0$:
                \begin{equation*}
                    1-\pi^2\cdot 0^2+\pi-\pi^2\cdot 0=1+\pi>0
                \end{equation*}
                Por tanto, tenemos que $f(\pi k)f(\pi(k+1))<0$ para $k\in\bb{Z}\setminus \{1\}$ impar.
            \end{itemize}
            Por tanto, por el Teorema de Bolzano, para cada $k\in\bb{Z}\setminus \{1\}$, $\exists x_k\in I_k$ tal que $f(x_k)=0$. Veamos ahora que dicho $x_k$ es el único punto de corte de las dos funciones, para lo cual consideramos la derivada de $f$:
            \begin{equation*}
                f'(x)=\cos x-x\sen x
            \end{equation*}
            
            ara ello, distinguimos en función de $k$:
            \begin{itemize}
                \item Si $k$ es par, $k\geq 0$:
                
                Tenemos que $f(\pi k)=1+\pi k>0$. 
            \end{itemize}
        \end{comment}

        Viendo las gráficas, es fácil intuir que hay infinitas soluciones de $1+x\cos x=0$. Notando cada uno de ellos como $x_k$ para cada $n\in \bb{R}$ de forma que $x_k<x_{k+1}$, tenemos que los posibles dominios son:
        \begin{equation*}
            D_k=\bb{R}\times \left]x_k,x_{k+1}\right[
        \end{equation*}
        \end{enumerate}
    \end{ejercicio}
\end{document}
