\documentclass[12pt]{article}

% Idioma y codificación
\usepackage[spanish, es-tabla]{babel}       %es-tabla para que se titule "Tabla"
\usepackage[utf8]{inputenc}

% Márgenes
\usepackage[a4paper,top=3cm,bottom=2.5cm,left=3cm,right=3cm]{geometry}

% Comentarios de bloque
\usepackage{verbatim}

% Paquetes de links
\usepackage[hidelinks]{hyperref}    % Permite enlaces
\usepackage{url}                    % redirecciona a la web

% Más opciones para enumeraciones
\usepackage{enumitem}

% Personalizar la portada
\usepackage{titling}

% Paquetes de tablas
\usepackage{multirow}


%------------------------------------------------------------------------

%Paquetes de figuras
\usepackage{caption}
\usepackage{subcaption} % Figuras al lado de otras
\usepackage{float}      % Poner figuras en el sitio indicado H.


% Paquetes de imágenes
\usepackage{graphicx}       % Paquete para añadir imágenes
\usepackage{transparent}    % Para manejar la opacidad de las figuras

% Paquete para usar colores
\usepackage[dvipsnames]{xcolor}
\usepackage{pagecolor}      % Para cambiar el color de la página

% Habilita tamaños de fuente mayores
\usepackage{fix-cm}

% Para los gráficos
\usepackage{tikz}

% Para poder situar los nodos en los grafos
\usetikzlibrary{positioning}


%------------------------------------------------------------------------

% Paquetes de matemáticas
\usepackage{mathtools, amsfonts, amssymb, mathrsfs}
\usepackage[makeroom]{cancel}     % Simplificar tachando
\usepackage{polynom}    % Divisiones y Ruffini
\usepackage{units} % Para poner fracciones diagonales con \nicefrac

\usepackage{pgfplots}   %Representar funciones
\pgfplotsset{compat=1.18}  % Versión 1.18

\usepackage{tikz-cd}    % Para usar diagramas de composiciones
\usetikzlibrary{calc}   % Para usar cálculo de coordenadas en tikz

%Definición de teoremas, etc.
\usepackage{amsthm}
%\swapnumbers   % Intercambia la posición del texto y de la numeración

\theoremstyle{plain}

\makeatletter
\@ifclassloaded{article}{
  \newtheorem{teo}{Teorema}[section]
}{
  \newtheorem{teo}{Teorema}[chapter]  % Se resetea en cada chapter
}
\makeatother

\newtheorem{coro}{Corolario}[teo]           % Se resetea en cada teorema
\newtheorem{prop}[teo]{Proposición}         % Usa el mismo contador que teorema
\newtheorem{lema}[teo]{Lema}                % Usa el mismo contador que teorema

\theoremstyle{remark}
\newtheorem*{observacion}{Observación}

\theoremstyle{definition}

\makeatletter
\@ifclassloaded{article}{
  \newtheorem{definicion}{Definición} [section]     % Se resetea en cada chapter
}{
  \newtheorem{definicion}{Definición} [chapter]     % Se resetea en cada chapter
}
\makeatother

\newtheorem*{notacion}{Notación}
\newtheorem*{ejemplo}{Ejemplo}
\newtheorem*{ejercicio*}{Ejercicio}             % No numerado
\newtheorem{ejercicio}{Ejercicio} [section]     % Se resetea en cada section


% Modificar el formato de la numeración del teorema "ejercicio"
\renewcommand{\theejercicio}{%
  \ifnum\value{section}=0 % Si no se ha iniciado ninguna sección
    \arabic{ejercicio}% Solo mostrar el número de ejercicio
  \else
    \thesection.\arabic{ejercicio}% Mostrar número de sección y número de ejercicio
  \fi
}


% \renewcommand\qedsymbol{$\blacksquare$}         % Cambiar símbolo QED
%------------------------------------------------------------------------

% Paquetes para encabezados
\usepackage{fancyhdr}
\pagestyle{fancy}
\fancyhf{}

\newcommand{\helv}{ % Modificación tamaño de letra
\fontfamily{}\fontsize{12}{12}\selectfont}
\setlength{\headheight}{15pt} % Amplía el tamaño del índice


%\usepackage{lastpage}   % Referenciar última pag   \pageref{LastPage}
\fancyfoot[C]{\thepage}

%------------------------------------------------------------------------

% Conseguir que no ponga "Capítulo 1". Sino solo "1."
\makeatletter
\@ifclassloaded{book}{
  \renewcommand{\chaptermark}[1]{\markboth{\thechapter.\ #1}{}} % En el encabezado
    
  \renewcommand{\@makechapterhead}[1]{%
  \vspace*{50\p@}%
  {\parindent \z@ \raggedright \normalfont
    \ifnum \c@secnumdepth >\m@ne
      \huge\bfseries \thechapter.\hspace{1em}\ignorespaces
    \fi
    \interlinepenalty\@M
    \Huge \bfseries #1\par\nobreak
    \vskip 40\p@
  }}
}
\makeatother

%------------------------------------------------------------------------
% Paquetes de cógido
\usepackage{minted}
\renewcommand\listingscaption{Código fuente}

\usepackage{fancyvrb}
% Personaliza el tamaño de los números de línea
\renewcommand{\theFancyVerbLine}{\small\arabic{FancyVerbLine}}

% Estilo para C++
\newminted{cpp}{
    frame=lines,
    framesep=2mm,
    baselinestretch=1.2,
    linenos,
    escapeinside=||
}

% para minted
\definecolor{LightGray}{rgb}{0.95,0.95,0.92}
\setminted{
    linenos=true,
    stepnumber=5,
    numberfirstline=true,
    autogobble,
    breaklines=true,
    breakautoindent=true,
    breaksymbolleft=,
    breaksymbolright=,
    breaksymbolindentleft=0pt,
    breaksymbolindentright=0pt,
    breaksymbolsepleft=0pt,
    breaksymbolsepright=0pt,
    fontsize=\footnotesize,
    bgcolor=LightGray,
    numbersep=10pt
}


\usepackage{listings} % Para incluir código desde un archivo

\renewcommand\lstlistingname{Código Fuente}
\renewcommand\lstlistlistingname{Índice de Códigos Fuente}

% Definir colores
\definecolor{vscodepurple}{rgb}{0.5,0,0.5}
\definecolor{vscodeblue}{rgb}{0,0,0.8}
\definecolor{vscodegreen}{rgb}{0,0.5,0}
\definecolor{vscodegray}{rgb}{0.5,0.5,0.5}
\definecolor{vscodebackground}{rgb}{0.97,0.97,0.97}
\definecolor{vscodelightgray}{rgb}{0.9,0.9,0.9}

% Configuración para el estilo de C similar a VSCode
\lstdefinestyle{vscode_C}{
  backgroundcolor=\color{vscodebackground},
  commentstyle=\color{vscodegreen},
  keywordstyle=\color{vscodeblue},
  numberstyle=\tiny\color{vscodegray},
  stringstyle=\color{vscodepurple},
  basicstyle=\scriptsize\ttfamily,
  breakatwhitespace=false,
  breaklines=true,
  captionpos=b,
  keepspaces=true,
  numbers=left,
  numbersep=5pt,
  showspaces=false,
  showstringspaces=false,
  showtabs=false,
  tabsize=2,
  frame=tb,
  framerule=0pt,
  aboveskip=10pt,
  belowskip=10pt,
  xleftmargin=10pt,
  xrightmargin=10pt,
  framexleftmargin=10pt,
  framexrightmargin=10pt,
  framesep=0pt,
  rulecolor=\color{vscodelightgray},
  backgroundcolor=\color{vscodebackground},
}

%------------------------------------------------------------------------

% Comandos definidos
\newcommand{\bb}[1]{\mathbb{#1}}
\newcommand{\cc}[1]{\mathcal{#1}}

% I prefer the slanted \leq
\let\oldleq\leq % save them in case they're every wanted
\let\oldgeq\geq
\renewcommand{\leq}{\leqslant}
\renewcommand{\geq}{\geqslant}

% Si y solo si
\newcommand{\sii}{\iff}

% Letras griegas
\newcommand{\eps}{\epsilon}
\newcommand{\veps}{\varepsilon}
\newcommand{\lm}{\lambda}

\newcommand{\ol}{\overline}
\newcommand{\ul}{\underline}
\newcommand{\wt}{\widetilde}
\newcommand{\wh}{\widehat}

\let\oldvec\vec
\renewcommand{\vec}{\overrightarrow}

% Derivadas parciales
\newcommand{\del}[2]{\frac{\partial #1}{\partial #2}}
\newcommand{\Del}[3]{\frac{\partial^{#1} #2}{\partial #3^{#1}}}
\newcommand{\deld}[2]{\dfrac{\partial #1}{\partial #2}}
\newcommand{\Deld}[3]{\dfrac{\partial^{#1} #2}{\partial #3^{#1}}}


\newcommand{\AstIg}{\stackrel{(\ast)}{=}}
\newcommand{\Hop}{\stackrel{L'H\hat{o}pital}{=}}

\newcommand{\red}[1]{{\color{red}#1}} % Para integrales, destacar los cambios.

% Método de integración
\newcommand{\MetInt}[2]{
    \left[\begin{array}{c}
        #1 \\ #2
    \end{array}\right]
}

% Declarar aplicaciones
% 1. Nombre aplicación
% 2. Dominio
% 3. Codominio
% 4. Variable
% 5. Imagen de la variable
\newcommand{\Func}[5]{
    \begin{equation*}
        \begin{array}{rrll}
            #1:& #2 & \longrightarrow & #3\\
               & #4 & \longmapsto & #5
        \end{array}
    \end{equation*}
}

%------------------------------------------------------------------------



\begin{document}

    % 1. Foto de fondo
    % 2. Título
    % 3. Encabezado Izquierdo
    % 4. Color de fondo
    % 5. Coord x del titulo
    % 6. Coord y del titulo
    % 7. Fecha

    
    % 1. Foto de fondo
% 2. Título
% 3. Encabezado Izquierdo
% 4. Color de fondo
% 5. Coord x del titulo
% 6. Coord y del titulo
% 7. Fecha

\newcommand{\portada}[7]{

    \portadaBase{#1}{#2}{#3}{#4}{#5}{#6}{#7}
    \portadaBook{#1}{#2}{#3}{#4}{#5}{#6}{#7}
}

\newcommand{\portadaExamen}[7]{

    \portadaBase{#1}{#2}{#3}{#4}{#5}{#6}{#7}
    \portadaArticle{#1}{#2}{#3}{#4}{#5}{#6}{#7}
}




\newcommand{\portadaBase}[7]{

    % Tiene la portada principal y la licencia Creative Commons
    
    % 1. Foto de fondo
    % 2. Título
    % 3. Encabezado Izquierdo
    % 4. Color de fondo
    % 5. Coord x del titulo
    % 6. Coord y del titulo
    % 7. Fecha
    
    
    \thispagestyle{empty}               % Sin encabezado ni pie de página
    \newgeometry{margin=0cm}        % Márgenes nulos para la primera página
    
    
    % Encabezado
    \fancyhead[L]{\helv #3}
    \fancyhead[R]{\helv \nouppercase{\leftmark}}
    
    
    \pagecolor{#4}        % Color de fondo para la portada
    
    \begin{figure}[p]
        \centering
        \transparent{0.3}           % Opacidad del 30% para la imagen
        
        \includegraphics[width=\paperwidth, keepaspectratio]{assets/#1}
    
        \begin{tikzpicture}[remember picture, overlay]
            \node[anchor=north west, text=white, opacity=1, font=\fontsize{60}{90}\selectfont\bfseries\sffamily, align=left] at (#5, #6) {#2};
            
            \node[anchor=south east, text=white, opacity=1, font=\fontsize{12}{18}\selectfont\sffamily, align=right] at (9.7, 3) {\textbf{\href{https://losdeldgiim.github.io/}{Los Del DGIIM}}};
            
            \node[anchor=south east, text=white, opacity=1, font=\fontsize{12}{15}\selectfont\sffamily, align=right] at (9.7, 1.8) {Doble Grado en Ingeniería Informática y Matemáticas\\Universidad de Granada};
        \end{tikzpicture}
    \end{figure}
    
    
    \restoregeometry        % Restaurar márgenes normales para las páginas subsiguientes
    \pagecolor{white}       % Restaurar el color de página
    
    
    \newpage
    \thispagestyle{empty}               % Sin encabezado ni pie de página
    \begin{tikzpicture}[remember picture, overlay]
        \node[anchor=south west, inner sep=3cm] at (current page.south west) {
            \begin{minipage}{0.5\paperwidth}
                \href{https://creativecommons.org/licenses/by-nc-nd/4.0/}{
                    \includegraphics[height=2cm]{assets/Licencia.png}
                }\vspace{1cm}\\
                Esta obra está bajo una
                \href{https://creativecommons.org/licenses/by-nc-nd/4.0/}{
                    Licencia Creative Commons Atribución-NoComercial-SinDerivadas 4.0 Internacional (CC BY-NC-ND 4.0).
                }\\
    
                Eres libre de compartir y redistribuir el contenido de esta obra en cualquier medio o formato, siempre y cuando des el crédito adecuado a los autores originales y no persigas fines comerciales. 
            \end{minipage}
        };
    \end{tikzpicture}
    
    
    
    % 1. Foto de fondo
    % 2. Título
    % 3. Encabezado Izquierdo
    % 4. Color de fondo
    % 5. Coord x del titulo
    % 6. Coord y del titulo
    % 7. Fecha


}


\newcommand{\portadaBook}[7]{

    % 1. Foto de fondo
    % 2. Título
    % 3. Encabezado Izquierdo
    % 4. Color de fondo
    % 5. Coord x del titulo
    % 6. Coord y del titulo
    % 7. Fecha

    % Personaliza el formato del título
    \pretitle{\begin{center}\bfseries\fontsize{42}{56}\selectfont}
    \posttitle{\par\end{center}\vspace{2em}}
    
    % Personaliza el formato del autor
    \preauthor{\begin{center}\Large}
    \postauthor{\par\end{center}\vfill}
    
    % Personaliza el formato de la fecha
    \predate{\begin{center}\huge}
    \postdate{\par\end{center}\vspace{2em}}
    
    \title{#2}
    \author{\href{https://losdeldgiim.github.io/}{Los Del DGIIM}}
    \date{Granada, #7}
    \maketitle
    
    \tableofcontents
}




\newcommand{\portadaArticle}[7]{

    % 1. Foto de fondo
    % 2. Título
    % 3. Encabezado Izquierdo
    % 4. Color de fondo
    % 5. Coord x del titulo
    % 6. Coord y del titulo
    % 7. Fecha

    % Personaliza el formato del título
    \pretitle{\begin{center}\bfseries\fontsize{42}{56}\selectfont}
    \posttitle{\par\end{center}\vspace{2em}}
    
    % Personaliza el formato del autor
    \preauthor{\begin{center}\Large}
    \postauthor{\par\end{center}\vspace{3em}}
    
    % Personaliza el formato de la fecha
    \predate{\begin{center}\huge}
    \postdate{\par\end{center}\vspace{5em}}
    
    \title{#2}
    \author{\href{https://losdeldgiim.github.io/}{Los Del DGIIM}}
    \date{Granada, #7}
    \thispagestyle{empty}               % Sin encabezado ni pie de página
    \maketitle
    \vfill
}
    \portadaExamen{ffccA4.jpg}{Ecuaciones\\Diferenciales I\\Examen XIV}{Ecuaciones Diferenciales I. Examen XIV}{MidnightBlue}{-8}{28}{2024-2025}{Arturo Olivares Martos}

    \begin{description}
        \item[Asignatura] Ecuaciones Diferenciales I
        \item[Curso Académico] 2023-24.
        \item[Grado] Doble Grado en Ingeniería Informática y Matemáticas.
        \item[Grupo] Único.
        \item[Profesor] Rafael Ortega Ríos.
        \item[Descripción] Parcial 2.
        \item[Fecha] 19 de Diciembre de 2023.
        %\item[Duración] 60 minutos.
    \end{description}
    \newpage

    \begin{ejercicio}
        Se consideran las funciones $f_1, f_2: \left]0, 1\right[ \to \bb{R}$ dadas por:
        \begin{equation*}
            f_1(t) = 1
            \quad f_2(t) = \begin{cases}
                1 & \text{si } t \in \left]0, \nicefrac{2}{3}\right],\\
                0 & \text{si } t \in \left]\nicefrac{2}{3}, 1\right[.
            \end{cases}
        \end{equation*}
        ¿Son estas funciones linealmente independientes en el intervalo $\left]0, 1\right[$?\\

        Aplicando la definición de independencia lineal, buscamos $c_1, c_2 \in \bb{R}$ tales que:
        \begin{equation*}
            c_1f_1(t) + c_2f_2(t) = 0, \quad \forall t \in \left]0, 1\right[.
        \end{equation*}

        Tomando $t\in \left]\nicefrac{2}{3},1\right]$, se tiene:
        \begin{equation*}
            0 = c_1f_1(t) + c_2f_2(t) = c_1\cdot 1 + c_2\cdot 0 = c_1
            \Longrightarrow c_1 = 0.
        \end{equation*}

        Tomando $t\in \left]0,\nicefrac{2}{3}\right]$, se tiene:
        \begin{equation*}
            0 = c_1f_1(t) + c_2f_2(t) = 0\cdot 1 + c_2\cdot 1 = c_2
            \Longrightarrow c_2 = 0.
        \end{equation*}

        Por tanto, $c_1=c_2=0$, lo que implica que $f_1$ y $f_2$ son linealmente independientes en el intervalo $\left]0, 1\right[$.
    \end{ejercicio}

    \begin{ejercicio}
        Se considera la ecuación diferencial
        \begin{equation*}
            ax + by + (cx + dy)y' = 0,
        \end{equation*}
        con $a, b, c, d \in \bb{R}^+$. ¿En qué casos se puede afirmar que $\mu(x, y) = e^{x+y}$ es un factor integrante?\\

        Definimos:
        \Func{P}{\bb{R}^2}{\bb{R}}{(x, y)}{ax + by}
        \Func{Q}{\bb{R}^2}{\bb{R}}{(x, y)}{cx + dy}

        Sea $\Omega$ el dominio del factor integrante. Para que $\mu(x, y)$ sea un factor integrante de la ecuación diferencial, se debe cumplir que:
        \begin{itemize}
            \item $\mu(x, y) \neq 0, \quad \forall (x, y) \in \Omega$. Si $\mu(x, y) = e^{x+y}$, lo tenemos garantizado.
            \item Se cumpla la condición de exactitud tras multiplicar la ecuación diferencial por $\mu(x, y)$:
            \begin{equation*}
                \dfrac{\partial (\mu P)}{\partial y} = \dfrac{\partial (\mu Q)}{\partial x}.
            \end{equation*}
        \end{itemize}

        Calculamos las derivadas parciales de la condición de exactitud:
        \begin{align*}
            \dfrac{\partial (\mu P)}{\partial y} &= \dfrac{\partial \mu}{\partial y}P + \mu\dfrac{\partial P}{\partial y},\\
            \dfrac{\partial (\mu Q)}{\partial x} &= \dfrac{\partial \mu}{\partial x}Q + \mu\dfrac{\partial Q}{\partial x}.
        \end{align*}

        Por tanto, la condición de exactitud se traduce en:
        \begin{equation*}
            \dfrac{\partial \mu}{\partial y}P  - \dfrac{\partial \mu}{\partial x}Q = \mu\left(\dfrac{\partial Q}{\partial x} - \dfrac{\partial P}{\partial y}\right).
        \end{equation*}

        En nuestro caso concreto, las derivadas son:
        \begin{align*}
            \dfrac{\partial \mu}{\partial y}(x,y) &= e^{x+y}=\mu(x,y), & \dfrac{\partial \mu}{\partial x}(x,y) &= e^{x+y}=\mu(x,y),\\
            \dfrac{\partial P}{\partial y}(x,y) &= b, & \dfrac{\partial Q}{\partial x}(x,y) &= c.
        \end{align*}

        Por tanto, en nuestro caso concreto, la condición de exactitud se traduce en:
        \begin{equation*}
            \mu(x,y)(ax+by-cx-dy)=\mu(x,y)(c-b)\qquad \forall (x,y)\in\Omega.
        \end{equation*}

        Como $\mu(x,y)\neq 0$ para todo $(x,y)\in \Omega$, la condición de exactitud queda:
        \begin{equation*}
            x(a-c)+y(b-d)=c-b\qquad \forall (x,y)\in \Omega
        \end{equation*}

        Por tanto, lo único que hemos de imponer sobre los coeficientes de la ecuación diferencial es que se cumpla la ecuación siguiente:
        \begin{equation*}
            x(a-c)+y(b-d)=c-b\qquad \forall (x,y)\in \Omega
        \end{equation*}

        Como $1,x,y$ son linealmente independientes, tenemos que:
        \begin{equation*}
            \begin{cases}
                a-c &= 0,\\
                b-d &= 0,\\
                c-b &= 0.
            \end{cases}
        \end{equation*}

        Por tanto, tenemos que $a=b=c=d$. Por tanto, lo único que hemos de imponer sobre los coeficientes de la ecuación diferencial es:
        \begin{equation*}
            a=b=c=d.
        \end{equation*}
    \end{ejercicio}

    \begin{ejercicio}
        Dada una función $a \in C(\bb{R})$, se supone que $\varphi_1, \varphi_2$ son las soluciones de la ecuación $x'' + a(t)x = 0$ que cumplen las condiciones iniciales
        \begin{align*}
            \varphi_1(0) &= 1, & \varphi_1'(0) &= 0,\\
            \varphi_2(0) &= 0, & \varphi_2'(0) &= 1.
        \end{align*}
        Demuestra que la función
        \begin{equation*}
            x(t) = \varphi_2(t) \int_0^t e^{s}\varphi_1(s)~ds - \varphi_1(t) \int_0^t e^{s}\varphi_2(s)~ds + 2024\varphi_2(t)
        \end{equation*}
        pertenece a $C^2(\bb{R})$ y encuentra una ecuación diferencial de la que es solución.\\

        El dominio de la ecuación diferencial descrita en el enunciado es $\bb{R}^2$.
        Por tanto, por ser $\varphi_1,\varphi_2$ las soluciones de dicha ecuación diferencial para distintas condiciones iniciales, por el Teorema de Existencia y Unicidad visto en el Capítulo 4, tenemos que dichas soluciones están definidas en todo $\bb{R}$. Además, como $\varphi_1,\varphi_2$ son soluciones, tenemos que:
        \begin{equation*}
            \varphi_1,\varphi_2\in C^2(\bb{R}).
        \end{equation*}

        En particular, por ser $\varphi_1,\varphi_2\in C(\bb{R})$, por el Teorema Fundamental del Cálculo tenemos que dichas integrales son de clase $1$. Por tanto, al $x$ suma de productos de funciones de clase $C^1$, tenemos que $x\in C^1(\bb{R})$. Para argumentar que $x\in C^2(\bb{R})$, hemos de calcular su derivada (notemos que para derivar las integrales usamos el Teorema Fundamental del Cálculo):
        \begin{align*}
            x'(t)=&\varphi_2'(t)\int_0^t e^{s}\varphi_1(s)~ds+\cancel{\varphi_2(t)e^{t}\varphi_1(t)}-\varphi_1'(t)\int_0^t e^{s}\varphi_2(s)~ds-\cancel{\varphi_1(t)e^{t}\varphi_2(t)}+2024\varphi_2'(t)
        \end{align*}

        En primer lugar, tenemos que $\varphi_1',\varphi_2'\in C^1(\bb{R})$. Además, como los integrandos son producto de funciones continuas, tenemos que las integrales son de clase $C^1$. Por tanto, $x'\in C^1(\bb{R})$, de forma que $x\in C^2(\bb{R})$.
        Calculamos ahora $x''(t)$:
        \begin{align*}
            \hspace{-3cm}
            x''(t)&= \varphi_2''(t)\int_0^t e^{s}\varphi_1(s)~ds+\varphi_2'(t)e^{t}\varphi_1(t)-\varphi_1''(t)\int_0^t e^{s}\varphi_2(s)~ds-\varphi_1'(t)e^{t}\varphi_2(t)+2024\varphi_2''(t)\\
            &\AstIg -a(t)\varphi_2(t)\int_0^t e^{s}\varphi_1(s)~ds+\varphi_2'(t)e^{t}\varphi_1(t)+a(t)\varphi_1(t)\int_0^t e^{s}\varphi_2(s)~ds-\varphi_1'(t)e^{t}\varphi_2(t)-2024a(t)\varphi_2(t)\\
            &= -a(t)\left[\varphi_2(t)\int_0^t e^{s}\varphi_1(s)~ds-\varphi_1(t)\int_0^t e^{s}\varphi_2(s)~ds+2024\varphi_2(t)\right] + e^t[\varphi_2'(t)\varphi_1(t)-\varphi_1'(t)\varphi_2(t)]\\
            &\stackrel{(\ast\ast)}{=} -a(t)x(t) + e^t[\varphi_2'(t)\varphi_1(t)-\varphi_1'(t)\varphi_2(t)]
        \end{align*}
        donde en $(\ast)$ hemos usado que $\varphi_1,\varphi_2$ son soluciones de la ecuación diferencial, y en $(\ast\ast)$ hemos usado la definición de $x(t)$.
        Por tanto, una ecuación diferencial de la que $x(t)$ es solución es:
        \begin{equation*}
            x''=-a(t)x+e^t[\varphi_2'(t)\varphi_1(t)-\varphi_1'(t)\varphi_2(t)]\qquad \text{con dominio }\bb{R}^2
        \end{equation*}

        No obstante, veamos ahora que se puede simplificar aún más, ya que podemos conseguir que no dependa de $\varphi_1$ ni $\varphi_2$, puesto que ese término es constante.
        Tenemos dos opciones:
        \begin{description}
            \item[Derivando:]
            
            Derivemos dicho término, que sabemos que es de clase $1$ en $\bb{R}$ por ser producto y restas de funciones de clase $C^1$.
            \begin{equation*}
                \dfrac{d}{dt}\left(\varphi_2'\varphi_1-\varphi_1'\varphi_2\right)
                = \varphi_2''\varphi_1+\cancel{\varphi_2'\varphi_1'}-\varphi_1''\varphi_2-\cancel{\varphi_1'\varphi_2'}
                \AstIg
                = -a\varphi_2\varphi_1 + a\varphi_1\varphi_2=0
            \end{equation*}

            Por tanto, al ser dicha derivada nula en todo $\bb{R}$, tenemos que dicho término es constante. Evaluando en $0$, tenemos:
            \begin{equation*}
                \varphi_2'(0)\varphi_1(0)-\varphi_1'(0)\varphi_2(0)
                =1\cdot 1-0\cdot 0=1
            \end{equation*}

            \item[Usando la Fórmula de Jacobi-Liouville:]
            
            Tenemos que:
            \begin{equation*}
                W(\varphi_1,\varphi_2)=\begin{vmatrix}
                    \varphi_1& \varphi_2\\
                    \varphi_1' & \varphi_2'
                \end{vmatrix}
                = \varphi_1\varphi_2'-\varphi_1'\varphi_2
            \end{equation*}

            Por tanto, el término que estamos estudiando es dicho Wronskiano.
            Evaluando en $0$, tenemos:
            \begin{equation*}
                W(\varphi_1,\varphi_2)(0)=\begin{vmatrix}
                    \varphi_1(0) & \varphi_2(0)\\
                    \varphi_1'(0) & \varphi_2'(0)
                \end{vmatrix}
                = \begin{vmatrix}
                    1 & 0\\
                    0 & 1
                \end{vmatrix} = 1
            \end{equation*}

            Por la Fórmula de Jacobi-Liouville, como $\varphi_1,\varphi_2$ son soluciones de la ecuación diferencial, tenemos que:
            \begin{equation*}
                W(\varphi_1,\varphi_2)(t)=W(\varphi_1,\varphi_2)(0)\cdot \exp\left(\int_0^t 0~ds\right)=1\cdot e^0=1\qquad \forall t\in \bb{R}
            \end{equation*}
            donde hemos empleado que el coeficiente que acompaña a $x'$ en la ecuación original es $0$.
        \end{description}

        En cualquier caso, hemos probado que dicho término es constantemente igual a~$1$:
        \begin{equation*}
            \varphi_2'(t)\varphi_1(t)-\varphi_1'(t)\varphi_2(t)
            = 1\qquad \forall t\in \bb{R}
        \end{equation*}

        Por tanto, la ecuación diferencial de la que $x(t)$ es solución es:
        \begin{equation*}
            x''=-a(t)x+e^t\qquad \text{con dominio }\bb{R}^2
        \end{equation*}

        No obstante, esta es no es la única solución de dicha ecuación. Aunque no sea necesario darlas, considerando la condición inicial:
        \begin{equation*}
            x(0)=0\qquad x'(0)=2024
        \end{equation*}
        tenemos que $x(t)$ es la única solución de la ecuación diferencial descrita que cumple dichas condiciones iniciales.
    \end{ejercicio}

    \begin{ejercicio}
        Encuentra todas las funciones continuas $f: \bb{R} \to \bb{R}$ que cumplen las desigualdades
        \begin{equation*}
            0 \leq f(t) \leq \dfrac{1}{1 + t^2}F(t), \quad \forall t \in \bb{R},
        \end{equation*}
        con $F(t) = \int_0^t f(s)~ds$.\\

        Distinguimos en función del valor de $t$:
        \begin{itemize}
            \item Restringuiendo a $\bb{R}^-$, veamos que $f\Big|_{\bb{R}^-}=0$.
            Como $f(t)\geq 0$ para todo $t\in \bb{R}$, tenemos que:
            \begin{equation*}
                \int_a^b f(t)~dt\geq 0\qquad \forall a,b\in \bb{R}, a<b
            \end{equation*}
            Por tanto, para $t<0$, tenemos que:
            \begin{equation*}
                F(t)=\int_0^t f(s)~ds=-\int_t^0 f(s)~ds\leq 0
            \end{equation*}
            Por tanto, tenemos que:
            \begin{equation*}
                0\leq f(t)\leq 0\qquad \forall t<0
            \end{equation*}
            Por tanto, $f\Big|_{\bb{R}^-}=0$.

            \item Restringuiendo a $\bb{R}^+_0$, veamos también que $f\Big|_{\bb{R}^+}=0$.
            
            Como $t^2\geq 0$ para todo $t\in \bb{R}$, tenemos que:
            \begin{equation*}
                \dfrac{1}{1+t^2}\leq \dfrac{1}{1+0} = 1\qquad \forall t\in \bb{R}
            \end{equation*}

            Por tanto, buscamos las funciones $f$ continuas tales que:
            \begin{equation*}
                0\leq f(t)\leq 1\cdot F(t)\qquad \forall t\in \bb{R}
            \end{equation*}

            Para $t>0$, tenemos que:
            \begin{equation*}
                0\leq f(t)\leq 1\cdot |F(t)|\qquad \forall t\geq 0
            \end{equation*}

            Por tanto, y usando un Lema visto en la demostración del Teorema de Existencia y Unicidad del Capítulo 5, tenemos que:
            \begin{equation*}
                f(t)=0\qquad \forall t\geq 0
            \end{equation*}

            Por tanto, $f\Big|_{\bb{R}^+_0}=0$.
        \end{itemize}

        Por tanto, la única función continua que cumple las desigualdades dadas es la función nula.
    \end{ejercicio}

    \begin{ejercicio}
        El espacio vectorial de soluciones de la ecuación $x'' + 4x = 0$ se denota por $Z_x$. De igual modo, $Z_y$ será el espacio vectorial de soluciones de $y'' + 2y' + 5y = 0$. Demuestra que la transformación
        \begin{equation*}
            \Psi: Z_x \to Z_y, \quad x \mapsto y, \quad y(t) = e^{-t}x(t)
        \end{equation*}
        define un isomorfismo. Encuentra bases de $Z_x$ y $Z_y$ y calcula la matriz que representa a $\Psi$ en esas bases.\\

        Buscamos en primer lugar base de $Z_x$. El polinomio característico de la primera ecuación es:
        \begin{equation*}
            \lambda^2+4=0\Longleftrightarrow
            \lambda^2=-4\Longleftrightarrow
            \lambda=\pm 2i
        \end{equation*}

        Trabajamos con el valor propio $\lm=2i$. Sabemos que $e^{2it}$ es solución (compleja) de la ecuación diferencial.
        Tenemos que:
        \begin{equation*}
            e^{2it}=\cos(2t)+i\sen(2t)
        \end{equation*}

        Por tanto, dos soluciones reales de la primera ecuación diferencial son:
        \begin{equation*}
            \begin{cases}
                x_1(t)=\cos(2t)\\
                x_2(t)=\sen(2t)
            \end{cases}
        \end{equation*}

        Además, son linealmente independientes, ya que:
        \begin{equation*}
            W(x_1,x_2)(t)=\begin{vmatrix}
                \cos(2t) & \sen(2t)\\
                -2\sen(2t) & 2\cos(2t)
            \end{vmatrix}=2\begin{vmatrix}
                \cos(2t) & \sen(2t)\\
                -\sen(2t) & \cos(2t)
            \end{vmatrix}=2\left(\cos^2(2t)+\sen^2(2t)\right)=2\neq 0
        \end{equation*}

        Por tanto, tenemos que:
        \begin{equation*}
            \cc{B}_x=\left\{\cos(2t),\sen(2t)\right\}
            \qquad
            Z_x=\cc{L}\left\{\cc{B}_x\right\}
        \end{equation*}

        Buscamos ahora base de $Z_y$. El polinomio característico de la segunda ecuación es:
        \begin{equation*}
            \lambda^2+2\lambda+5=0\Longleftrightarrow
            \lambda=\dfrac{-2\pm\sqrt{4-20}}{2}=-1\pm 2i
        \end{equation*}

        Trabajamos con el valor propio $\lm=-1+2i$. Sabemos que $e^{(-1+2i)t}$ es solución (compleja) de la ecuación diferencial.
        Tenemos que:
        \begin{equation*}
            e^{(-1+2i)t}=e^{-t}(\cos(2t)+i\sen(2t))
        \end{equation*}

        Por tanto, dos soluciones reales de la segunda ecuación diferencial son:
        \begin{equation*}
            \begin{cases}
                y_1(t)=e^{-t}\cos(2t)\\
                y_2(t)=e^{-t}\sen(2t)
            \end{cases}
        \end{equation*}

        Además, son linealmente independientes. Por tanto, tenemos que:
        \begin{equation*}
            \cc{B}_y=\left\{e^{-t}\cos(2t),e^{-t}\sen(2t)\right\}
            \qquad
            Z_y=\cc{L}\left\{\cc{B}_y\right\}
        \end{equation*}

        Veamos ahora que $\Psi$ es una aplicación lineal. Dados $\lm_1,\lm_2\in \bb{R}$ y $x_1,x_2\in Z_x$, tenemos que:
        \begin{align*}
            \Psi(\lm_1x_1+\lm_2x_2)&=e^{-t}(\lm_1x_1+\lm_2x_2)\\
            &=\lm_1e^{-t}x_1+\lm_2e^{-t}x_2\\
            &=\lm_1\Psi(x_1)+\lm_2\Psi(x_2)
        \end{align*}
        Por tanto, $\Psi$ es una aplicación lineal. Además, tenemos que:
        \begin{align*}
            \Psi(x_1(t))&=e^{-t}\cos(2t)=y_1(t)\\
            \Psi(x_2(t))&=e^{-t}\sen(2t)=y_2(t)
        \end{align*}

        Por tanto, la matriz que representa a $\Psi$ en las bases dadas es:
        \begin{equation*}
            \cc{M}(\Psi, \cc{B}_x, \cc{B}_y)=\begin{pmatrix}
                1 & 0\\
                0 & 1
            \end{pmatrix}
            =Id_2
        \end{equation*}

        Por tanto, como $\Psi$ es lineal con $|\Psi|=1\neq 0$, tenemos que $\Psi$ es un isomorfismo.
    \end{ejercicio}
\end{document}
