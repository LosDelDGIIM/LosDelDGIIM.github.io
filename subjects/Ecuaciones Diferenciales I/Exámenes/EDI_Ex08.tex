\documentclass[12pt]{article}

% Idioma y codificación
\usepackage[spanish, es-tabla]{babel}       %es-tabla para que se titule "Tabla"
\usepackage[utf8]{inputenc}

% Márgenes
\usepackage[a4paper,top=3cm,bottom=2.5cm,left=3cm,right=3cm]{geometry}

% Comentarios de bloque
\usepackage{verbatim}

% Paquetes de links
\usepackage[hidelinks]{hyperref}    % Permite enlaces
\usepackage{url}                    % redirecciona a la web

% Más opciones para enumeraciones
\usepackage{enumitem}

% Personalizar la portada
\usepackage{titling}

% Paquetes de tablas
\usepackage{multirow}


%------------------------------------------------------------------------

%Paquetes de figuras
\usepackage{caption}
\usepackage{subcaption} % Figuras al lado de otras
\usepackage{float}      % Poner figuras en el sitio indicado H.


% Paquetes de imágenes
\usepackage{graphicx}       % Paquete para añadir imágenes
\usepackage{transparent}    % Para manejar la opacidad de las figuras

% Paquete para usar colores
\usepackage[dvipsnames]{xcolor}
\usepackage{pagecolor}      % Para cambiar el color de la página

% Habilita tamaños de fuente mayores
\usepackage{fix-cm}

% Para los gráficos
\usepackage{tikz}

% Para poder situar los nodos en los grafos
\usetikzlibrary{positioning}


%------------------------------------------------------------------------

% Paquetes de matemáticas
\usepackage{mathtools, amsfonts, amssymb, mathrsfs}
\usepackage[makeroom]{cancel}     % Simplificar tachando
\usepackage{polynom}    % Divisiones y Ruffini
\usepackage{units} % Para poner fracciones diagonales con \nicefrac

\usepackage{pgfplots}   %Representar funciones
\pgfplotsset{compat=1.18}  % Versión 1.18

\usepackage{tikz-cd}    % Para usar diagramas de composiciones
\usetikzlibrary{calc}   % Para usar cálculo de coordenadas en tikz

%Definición de teoremas, etc.
\usepackage{amsthm}
%\swapnumbers   % Intercambia la posición del texto y de la numeración

\theoremstyle{plain}

\makeatletter
\@ifclassloaded{article}{
  \newtheorem{teo}{Teorema}[section]
}{
  \newtheorem{teo}{Teorema}[chapter]  % Se resetea en cada chapter
}
\makeatother

\newtheorem{coro}{Corolario}[teo]           % Se resetea en cada teorema
\newtheorem{prop}[teo]{Proposición}         % Usa el mismo contador que teorema
\newtheorem{lema}[teo]{Lema}                % Usa el mismo contador que teorema

\theoremstyle{remark}
\newtheorem*{observacion}{Observación}

\theoremstyle{definition}

\makeatletter
\@ifclassloaded{article}{
  \newtheorem{definicion}{Definición} [section]     % Se resetea en cada chapter
}{
  \newtheorem{definicion}{Definición} [chapter]     % Se resetea en cada chapter
}
\makeatother

\newtheorem*{notacion}{Notación}
\newtheorem*{ejemplo}{Ejemplo}
\newtheorem*{ejercicio*}{Ejercicio}             % No numerado
\newtheorem{ejercicio}{Ejercicio} [section]     % Se resetea en cada section


% Modificar el formato de la numeración del teorema "ejercicio"
\renewcommand{\theejercicio}{%
  \ifnum\value{section}=0 % Si no se ha iniciado ninguna sección
    \arabic{ejercicio}% Solo mostrar el número de ejercicio
  \else
    \thesection.\arabic{ejercicio}% Mostrar número de sección y número de ejercicio
  \fi
}


% \renewcommand\qedsymbol{$\blacksquare$}         % Cambiar símbolo QED
%------------------------------------------------------------------------

% Paquetes para encabezados
\usepackage{fancyhdr}
\pagestyle{fancy}
\fancyhf{}

\newcommand{\helv}{ % Modificación tamaño de letra
\fontfamily{}\fontsize{12}{12}\selectfont}
\setlength{\headheight}{15pt} % Amplía el tamaño del índice


%\usepackage{lastpage}   % Referenciar última pag   \pageref{LastPage}
\fancyfoot[C]{\thepage}

%------------------------------------------------------------------------

% Conseguir que no ponga "Capítulo 1". Sino solo "1."
\makeatletter
\@ifclassloaded{book}{
  \renewcommand{\chaptermark}[1]{\markboth{\thechapter.\ #1}{}} % En el encabezado
    
  \renewcommand{\@makechapterhead}[1]{%
  \vspace*{50\p@}%
  {\parindent \z@ \raggedright \normalfont
    \ifnum \c@secnumdepth >\m@ne
      \huge\bfseries \thechapter.\hspace{1em}\ignorespaces
    \fi
    \interlinepenalty\@M
    \Huge \bfseries #1\par\nobreak
    \vskip 40\p@
  }}
}
\makeatother

%------------------------------------------------------------------------
% Paquetes de cógido
\usepackage{minted}
\renewcommand\listingscaption{Código fuente}

\usepackage{fancyvrb}
% Personaliza el tamaño de los números de línea
\renewcommand{\theFancyVerbLine}{\small\arabic{FancyVerbLine}}

% Estilo para C++
\newminted{cpp}{
    frame=lines,
    framesep=2mm,
    baselinestretch=1.2,
    linenos,
    escapeinside=||
}

% para minted
\definecolor{LightGray}{rgb}{0.95,0.95,0.92}
\setminted{
    linenos=true,
    stepnumber=5,
    numberfirstline=true,
    autogobble,
    breaklines=true,
    breakautoindent=true,
    breaksymbolleft=,
    breaksymbolright=,
    breaksymbolindentleft=0pt,
    breaksymbolindentright=0pt,
    breaksymbolsepleft=0pt,
    breaksymbolsepright=0pt,
    fontsize=\footnotesize,
    bgcolor=LightGray,
    numbersep=10pt
}


\usepackage{listings} % Para incluir código desde un archivo

\renewcommand\lstlistingname{Código Fuente}
\renewcommand\lstlistlistingname{Índice de Códigos Fuente}

% Definir colores
\definecolor{vscodepurple}{rgb}{0.5,0,0.5}
\definecolor{vscodeblue}{rgb}{0,0,0.8}
\definecolor{vscodegreen}{rgb}{0,0.5,0}
\definecolor{vscodegray}{rgb}{0.5,0.5,0.5}
\definecolor{vscodebackground}{rgb}{0.97,0.97,0.97}
\definecolor{vscodelightgray}{rgb}{0.9,0.9,0.9}

% Configuración para el estilo de C similar a VSCode
\lstdefinestyle{vscode_C}{
  backgroundcolor=\color{vscodebackground},
  commentstyle=\color{vscodegreen},
  keywordstyle=\color{vscodeblue},
  numberstyle=\tiny\color{vscodegray},
  stringstyle=\color{vscodepurple},
  basicstyle=\scriptsize\ttfamily,
  breakatwhitespace=false,
  breaklines=true,
  captionpos=b,
  keepspaces=true,
  numbers=left,
  numbersep=5pt,
  showspaces=false,
  showstringspaces=false,
  showtabs=false,
  tabsize=2,
  frame=tb,
  framerule=0pt,
  aboveskip=10pt,
  belowskip=10pt,
  xleftmargin=10pt,
  xrightmargin=10pt,
  framexleftmargin=10pt,
  framexrightmargin=10pt,
  framesep=0pt,
  rulecolor=\color{vscodelightgray},
  backgroundcolor=\color{vscodebackground},
}

%------------------------------------------------------------------------

% Comandos definidos
\newcommand{\bb}[1]{\mathbb{#1}}
\newcommand{\cc}[1]{\mathcal{#1}}

% I prefer the slanted \leq
\let\oldleq\leq % save them in case they're every wanted
\let\oldgeq\geq
\renewcommand{\leq}{\leqslant}
\renewcommand{\geq}{\geqslant}

% Si y solo si
\newcommand{\sii}{\iff}

% Letras griegas
\newcommand{\eps}{\epsilon}
\newcommand{\veps}{\varepsilon}
\newcommand{\lm}{\lambda}

\newcommand{\ol}{\overline}
\newcommand{\ul}{\underline}
\newcommand{\wt}{\widetilde}
\newcommand{\wh}{\widehat}

\let\oldvec\vec
\renewcommand{\vec}{\overrightarrow}

% Derivadas parciales
\newcommand{\del}[2]{\frac{\partial #1}{\partial #2}}
\newcommand{\Del}[3]{\frac{\partial^{#1} #2}{\partial #3^{#1}}}
\newcommand{\deld}[2]{\dfrac{\partial #1}{\partial #2}}
\newcommand{\Deld}[3]{\dfrac{\partial^{#1} #2}{\partial #3^{#1}}}


\newcommand{\AstIg}{\stackrel{(\ast)}{=}}
\newcommand{\Hop}{\stackrel{L'H\hat{o}pital}{=}}

\newcommand{\red}[1]{{\color{red}#1}} % Para integrales, destacar los cambios.

% Método de integración
\newcommand{\MetInt}[2]{
    \left[\begin{array}{c}
        #1 \\ #2
    \end{array}\right]
}

% Declarar aplicaciones
% 1. Nombre aplicación
% 2. Dominio
% 3. Codominio
% 4. Variable
% 5. Imagen de la variable
\newcommand{\Func}[5]{
    \begin{equation*}
        \begin{array}{rrll}
            #1:& #2 & \longrightarrow & #3\\
               & #4 & \longmapsto & #5
        \end{array}
    \end{equation*}
}

%------------------------------------------------------------------------



\begin{document}

    % 1. Foto de fondo
    % 2. Título
    % 3. Encabezado Izquierdo
    % 4. Color de fondo
    % 5. Coord x del titulo
    % 6. Coord y del titulo
    % 7. Fecha

    
    % 1. Foto de fondo
% 2. Título
% 3. Encabezado Izquierdo
% 4. Color de fondo
% 5. Coord x del titulo
% 6. Coord y del titulo
% 7. Fecha

\newcommand{\portada}[7]{

    \portadaBase{#1}{#2}{#3}{#4}{#5}{#6}{#7}
    \portadaBook{#1}{#2}{#3}{#4}{#5}{#6}{#7}
}

\newcommand{\portadaExamen}[7]{

    \portadaBase{#1}{#2}{#3}{#4}{#5}{#6}{#7}
    \portadaArticle{#1}{#2}{#3}{#4}{#5}{#6}{#7}
}




\newcommand{\portadaBase}[7]{

    % Tiene la portada principal y la licencia Creative Commons
    
    % 1. Foto de fondo
    % 2. Título
    % 3. Encabezado Izquierdo
    % 4. Color de fondo
    % 5. Coord x del titulo
    % 6. Coord y del titulo
    % 7. Fecha
    
    
    \thispagestyle{empty}               % Sin encabezado ni pie de página
    \newgeometry{margin=0cm}        % Márgenes nulos para la primera página
    
    
    % Encabezado
    \fancyhead[L]{\helv #3}
    \fancyhead[R]{\helv \nouppercase{\leftmark}}
    
    
    \pagecolor{#4}        % Color de fondo para la portada
    
    \begin{figure}[p]
        \centering
        \transparent{0.3}           % Opacidad del 30% para la imagen
        
        \includegraphics[width=\paperwidth, keepaspectratio]{assets/#1}
    
        \begin{tikzpicture}[remember picture, overlay]
            \node[anchor=north west, text=white, opacity=1, font=\fontsize{60}{90}\selectfont\bfseries\sffamily, align=left] at (#5, #6) {#2};
            
            \node[anchor=south east, text=white, opacity=1, font=\fontsize{12}{18}\selectfont\sffamily, align=right] at (9.7, 3) {\textbf{\href{https://losdeldgiim.github.io/}{Los Del DGIIM}}};
            
            \node[anchor=south east, text=white, opacity=1, font=\fontsize{12}{15}\selectfont\sffamily, align=right] at (9.7, 1.8) {Doble Grado en Ingeniería Informática y Matemáticas\\Universidad de Granada};
        \end{tikzpicture}
    \end{figure}
    
    
    \restoregeometry        % Restaurar márgenes normales para las páginas subsiguientes
    \pagecolor{white}       % Restaurar el color de página
    
    
    \newpage
    \thispagestyle{empty}               % Sin encabezado ni pie de página
    \begin{tikzpicture}[remember picture, overlay]
        \node[anchor=south west, inner sep=3cm] at (current page.south west) {
            \begin{minipage}{0.5\paperwidth}
                \href{https://creativecommons.org/licenses/by-nc-nd/4.0/}{
                    \includegraphics[height=2cm]{assets/Licencia.png}
                }\vspace{1cm}\\
                Esta obra está bajo una
                \href{https://creativecommons.org/licenses/by-nc-nd/4.0/}{
                    Licencia Creative Commons Atribución-NoComercial-SinDerivadas 4.0 Internacional (CC BY-NC-ND 4.0).
                }\\
    
                Eres libre de compartir y redistribuir el contenido de esta obra en cualquier medio o formato, siempre y cuando des el crédito adecuado a los autores originales y no persigas fines comerciales. 
            \end{minipage}
        };
    \end{tikzpicture}
    
    
    
    % 1. Foto de fondo
    % 2. Título
    % 3. Encabezado Izquierdo
    % 4. Color de fondo
    % 5. Coord x del titulo
    % 6. Coord y del titulo
    % 7. Fecha


}


\newcommand{\portadaBook}[7]{

    % 1. Foto de fondo
    % 2. Título
    % 3. Encabezado Izquierdo
    % 4. Color de fondo
    % 5. Coord x del titulo
    % 6. Coord y del titulo
    % 7. Fecha

    % Personaliza el formato del título
    \pretitle{\begin{center}\bfseries\fontsize{42}{56}\selectfont}
    \posttitle{\par\end{center}\vspace{2em}}
    
    % Personaliza el formato del autor
    \preauthor{\begin{center}\Large}
    \postauthor{\par\end{center}\vfill}
    
    % Personaliza el formato de la fecha
    \predate{\begin{center}\huge}
    \postdate{\par\end{center}\vspace{2em}}
    
    \title{#2}
    \author{\href{https://losdeldgiim.github.io/}{Los Del DGIIM}}
    \date{Granada, #7}
    \maketitle
    
    \tableofcontents
}




\newcommand{\portadaArticle}[7]{

    % 1. Foto de fondo
    % 2. Título
    % 3. Encabezado Izquierdo
    % 4. Color de fondo
    % 5. Coord x del titulo
    % 6. Coord y del titulo
    % 7. Fecha

    % Personaliza el formato del título
    \pretitle{\begin{center}\bfseries\fontsize{42}{56}\selectfont}
    \posttitle{\par\end{center}\vspace{2em}}
    
    % Personaliza el formato del autor
    \preauthor{\begin{center}\Large}
    \postauthor{\par\end{center}\vspace{3em}}
    
    % Personaliza el formato de la fecha
    \predate{\begin{center}\huge}
    \postdate{\par\end{center}\vspace{5em}}
    
    \title{#2}
    \author{\href{https://losdeldgiim.github.io/}{Los Del DGIIM}}
    \date{Granada, #7}
    \thispagestyle{empty}               % Sin encabezado ni pie de página
    \maketitle
    \vfill
}
    \portadaExamen{ffccA4.jpg}{Ecuaciones\\Diferenciales I\\Examen VIII}{Ecuaciones Diferenciales I. Examen VIII}{MidnightBlue}{-8}{28}{2024-2025}{Arturo Olivares Martos}

    \begin{description}
        \item[Asignatura] Ecuaciones Diferenciales I
        \item[Curso Académico] 2022-23.
        \item[Grado] Doble Grado en Ingeniería Informática y Matemáticas.
        \item[Grupo] Único.
        \item[Profesor] Rafael Ortega Ríos.
        \item[Descripción] Primer parcial.
        \item[Fecha] 7 de noviembre de 2022.
        \item[Duración] 120 minutos.    
    \end{description}
    \newpage

\begin{ejercicio}
    Encuentra la ecuación diferencial de las curvas $(x,y(x))$ con la siguiente propiedad geométrica: en cada punto de la curva, su segunda coordenada coincide con la suma de las coordenadas del punto de intersección de la recta tangente con la bisectriz del primer cuadrante.\\

    Dado un punto $P=(x,(y(x)))$, necesitamos calcular el punto de corte de la recta tangente a $y(x)$ por $P$ con la bisectriz del primer cuadrante. Usando la ecuación punto$-$pendiente en las variables $(u,v)$, tenemos que la recta tangente a $y(x)$ por $u$ es:
    \begin{equation*}
        y'(x)=\dfrac{v-y(x)}{u-x}
        \Longrightarrow
        r_t\equiv v = y'(x)(u-x) + y(x)
    \end{equation*}
    La bisectriz del primer cuadrante, en las variables $(u,v)$, es:
    \begin{equation*}
        b\equiv v=u
    \end{equation*}

    Por tanto, el sistema a resolver es:
    \begin{equation*}
        \begin{cases}
            r_t\equiv v = y'(x)(u-x) + y(x)\\
            b\equiv v=u
        \end{cases}
    \end{equation*}

    Igualando las componentes $v$, tenemos que:
    \begin{equation*}
        y'(x)(u-x) + y(x)=u \Longrightarrow
        u=v=\dfrac{y(x)-y'(x)x}{1-y'(x)}
    \end{equation*}

    Por tanto, la ecuación diferencial planteada es:
    \begin{equation*}
        y=2\cdot \dfrac{y-y'x}{1-y'}
    \end{equation*}

    En forma normal (aunque no es deseado por reducir el dominio), esta es:
    \begin{equation*}
        y'=\dfrac{y-2y}{-2x+y}
    \end{equation*}
\end{ejercicio}

\begin{ejercicio}
    Resuelve el problema de valores iniciales siguiente:
    \begin{equation*}
        y^3e^x+3y^2e^xy'=e^{-x},\quad y(0)=1.
    \end{equation*}
    Para ello, usa un cambio de variable del tipo $y=u^{\alpha}$ para $\alpha$ adecuada. Estudia el intervalo maximal de definición de la solución.
\end{ejercicio}

\begin{ejercicio}
    Responda a los siguientes apartados:
    \begin{enumerate}
        \item Sean $P,Q$ funciones de clase $C^1$ definidas en un dominio del plano. Argumenta la veracidad o falsedad de la siguiente afirmación: si $\mu(x)$ es un factor integrante para la ecuación $P(x,y)dx+Q(x,y)y'=0$, entonces también lo es para la ecuación $P(x,y)+Q(x,y)y'=h(x)$, con $h$ función real de variable real de clase $C^1$.\\
        
        Sea $\Omega\subset \bb{R}^2$ el dominio del plano descrito. Por ser $\mu$ factor integrante de la ecuación $P(x,y)dx+Q(x,y)y'=0$, se tiene que las siguientes derivadas parciales son iguales:
        \begin{align*}
            \dfrac{\partial(\mu P)}{\partial y}&=\dfrac{\partial(\mu)}{\partial y}P+\mu\dfrac{\partial(P)}{\partial y}=0\cdot P+\mu\left(\dfrac{\partial(P)}{\partial y}\right) = \mu\left(\dfrac{\partial(P)}{\partial y}\right)\\
            \dfrac{\partial(\mu Q)}{\partial x}&=\dfrac{\partial(\mu)}{\partial x}Q+\mu\dfrac{\partial(Q)}{\partial x}=m'Q + \mu\dfrac{\partial(Q)}{\partial x}
        \end{align*}

        Definimos $\wt{P}$ como sigue:
        \Func{\wt{P}}{\Omega}{\bb{R}}{(x,y)}{P(x,y)-h(x)}

        De esta forma, $\wt{P}\in C^1(\Omega)$ y la ecuación $P(x,y)+Q(x,y)y'=h(x)$ se puede reescribir como $\wt{P}(x,y)+Q(x,y)y'=0$. Por tanto, para que $\mu$ sea factor integrante de la ecuación $P(x,y)+Q(x,y)y'=h(x)$, debe cumplir que:
        \begin{itemize}
            \item $\mu(x)>0$ en $\pi_1(\Omega)$. Esto se tiene por ser $\mu$ factor integrante de la ecuación $P(x,y)dx+Q(x,y)y'=0$.
            \item Tras multiplicar por $\mu$, la ecuación $\wt{P}(x,y)+Q(x,y)y'=0$ debe ser exacta.
        \end{itemize}

        Veamos si se da la condición de exactitud tras multiplicar por $\mu$:
        \begin{align*}
            \dfrac{\partial(\mu \wt{P})}{\partial y}&=\dfrac{\partial(\mu)}{\partial y}\wt{P}+\mu\dfrac{\partial(\wt{P})}{\partial y}=0\cdot \wt{P}+\mu\left(\dfrac{\partial(P)}{\partial y}\right) = \mu\left(\dfrac{\partial(P)}{\partial y}\right)\\
            \dfrac{\partial(\mu Q)}{\partial x}&=\dfrac{\partial(\mu)}{\partial x}Q+\mu\dfrac{\partial(Q)}{\partial x}=m'Q + \mu\dfrac{\partial(Q)}{\partial x}
        \end{align*}

        Por tanto, por ser $\mu$ factor integrante de la ecuación $P(x,y)dx+Q(x,y)y'=0$, hemos visto anteriormente que estas dos derivadas parciales son iguales. Por tanto, $\mu$ es factor integrante de la ecuación $P(x,y)+Q(x,y)y'=h(x)$.
        
        \item Encuentra un factor integrante de la forma $\mu\left(\nicefrac{x}{y}\right)$ para la ecuación
        \begin{equation*}
            x+y+(y-x)y'=0.
        \end{equation*}

        Definimos:
        \Func{P}{\bb{R}^2}{\bb{R}}{(x,y)}{x+y}
        \Func{Q}{\bb{R}^2}{\bb{R}}{(x,y)}{y-x}

        Por tanto, la ecuación dada es $P(x,y)+Q(x,y)y'=0$. Tras multiplicarla el factor integrante $\mu(x,y)$, ha de cumplirse la condición de exactitud. Las derivadas que aparecen en la condición de exactitud son:
        \begin{align*}
            \dfrac{\partial(\mu P)}{\partial y}&=\dfrac{\partial(\mu)}{\partial y}P+\mu\dfrac{\partial(P)}{\partial y}\\
            \dfrac{\partial(\mu Q)}{\partial x}&=\dfrac{\partial(\mu)}{\partial x}Q+\mu\dfrac{\partial(Q)}{\partial x}
        \end{align*}

        Igualando, llegamos a:
        \begin{equation*}
            \dfrac{\partial(\mu)}{\partial y}P-\dfrac{\partial(\mu)}{\partial x}Q=\mu\left(\dfrac{\partial(Q)}{\partial x}-\dfrac{\partial(P)}{\partial y}\right)
        \end{equation*}

        Calculemos las derivadas parciales necesarias, usando que $\mu$ es de la forma $\mu(x,y)=m\left(\nicefrac{x}{y}\right)$ para alguna función $m$. Además, de aquí en adelante consideramos un dominio $\Omega\subset \bb{R}^2$ en el que $y$ no se anula, para poder considerar ese cociente:
        \begin{align*}
            \dfrac{\partial(P)}{\partial y}(x,y)&=1\qquad &\dfrac{\partial(Q)}{\partial x}(x,y)&=-1\\
            \dfrac{\partial(\mu)}{\partial y}(x,y)&=m'\left(\frac{x}{y}\right)\left(-\frac{x}{y^2}\right) &\dfrac{\partial(\mu)}{\partial x}(x,y)&=m'\left(\frac{x}{y}\right)\left(\frac{1}{y}\right)
        \end{align*}

        Por tanto, tenemos que:
        \begin{align*}
            m'\left(\frac{x}{y}\right)\left(-\frac{x}{y^2}\right)(x+y)-m'\left(\frac{x}{y}\right)\left(\frac{1}{y}\right)(y-x)&=m\left(\frac{x}{y}\right)(-1-1)\\
            m'\left(\frac{x}{y}\right)\left[\left(-\frac{x}{y^2}\right)(x+y)-\left(\frac{1}{y}\right)(y-x)\right]&=-2m\left(\frac{x}{y}\right)\\
            m'\left(\frac{x}{y}\right)\left(\frac{-x^2-\cancel{xy}-y^2+\cancel{xy}}{y^2}\right)&=-2m\left(\frac{x}{y}\right)\\
            -m'\left(\frac{x}{y}\right)\left[\left(\frac{x}{y}\right)^2+1\right]&=-2m\left(\frac{x}{y}\right)
        \end{align*}

        Por tanto, como el factor integrante no se puede anular, y el otro término es siempre positiva, notando $\xi=\nicefrac{x}{y}$, llegamos a que la función $m(\xi)$ buscada ha de ser solución de la siguiente ecuación diferencial:
        \begin{equation*}
            m'=2m\cdot \frac{1}{1+\xi^2}
        \end{equation*}

        Se trata de una ecuación diferencial en variables separadas. Suponiendo $m(\xi)>0$ para todo $\xi$ (en caso contrario, tendríamos un factor integrante igualmente válido), se tiene que:
        \begin{equation*}
            m(\xi)=\exp(\arctan(\xi))
        \end{equation*}
        donde hemos elegido como constante de integración $e^{-2}$ por simplicidad. Por tanto, el factor integrante buscado es:
        \begin{equation*}
            \mu(x,y)=\exp\left(\arctan\left(\frac{x}{y}\right)\right)
        \end{equation*}
    \end{enumerate}
\end{ejercicio}

\begin{ejercicio}
    Sean $P,Q$ funciones de clase $C^1$ definidas en un dominio del plano que verifican la condición de exactitud. Se define la función
    \begin{equation*}
        U(x,y)=\int_{0}^{1}[xP(\lm x,\lm^2 y)+2\lm yQ(\lm x,\lm^2 y)]~d\lm.
    \end{equation*}
    Calcula $\dfrac{\partial U}{\partial x}$.\\

    Sea $\Omega$ el dominio del plano descrito. Definimos la función auxiliar:
    \Func{F}{\Omega\times [0,1]}{\bb{R}}{(x,y,\lm)}{xP(\lm x,\lm^2 y)+2\lm yQ(\lm x,\lm^2 y)}

    Por ser $P,Q$ funciones de clase $C^1$ en $\Omega$, la función $F$ es de clase $C^1$ en su dominio. Por tanto, por el Teorema de la Derivación de las Integrales dependientes de Parámetros, tenemos que $U\in C^1(\Omega)$, y en particular:
    \begin{align*}
        \dfrac{\partial U}{\partial x}(x,y)&=\int_{0}^{1}\dfrac{\partial F}{\partial x}(x,y,\lm)~d\lm\\
        &=\int_{0}^{1}\left[P(\lm x,\lm^2 y)+x\lm \dfrac{\partial P}{\partial x}(\lm x,\lm^2 y)+2\lm^2 y\dfrac{\partial Q}{\partial x}(\lm x,\lm^2 y)\right]~d\lm\\
        & \AstIg\int_{0}^{1}\left[P(\lm x,\lm^2 y)+x\lm \dfrac{\partial P}{\partial x}(\lm x,\lm^2 y)+2\lm^2 y\dfrac{\partial P}{\partial y}(\lm x,\lm^2 y)\right]~d\lm\\
        & = \int_{0}^{1}\left[P(\lm x,\lm^2 y)+\lm \dfrac{\partial P}{\partial \lm}(\lm x,\lm^2 y)\right]~d\lm\\
        & = \int_{0}^{1}\dfrac{d}{d\lm}\left[\lm P(\lm x,\lm^2 y)\right]~d\lm\\
        & = \left.\lm P(\lm x,\lm^2 y)\right|_{0}^{1}\\
        & = P(x,y)
    \end{align*}
    donde en $(\ast)$ hemos usado que cumplen la condición de exactitud, y en la penúltima igualdad hemos usado la Regla de Barrow.
\end{ejercicio}

    
\end{document}