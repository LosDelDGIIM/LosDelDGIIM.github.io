\documentclass[12pt]{article}

% Idioma y codificación
\usepackage[spanish, es-tabla]{babel}       %es-tabla para que se titule "Tabla"
\usepackage[utf8]{inputenc}

% Márgenes
\usepackage[a4paper,top=3cm,bottom=2.5cm,left=3cm,right=3cm]{geometry}

% Comentarios de bloque
\usepackage{verbatim}

% Paquetes de links
\usepackage[hidelinks]{hyperref}    % Permite enlaces
\usepackage{url}                    % redirecciona a la web

% Más opciones para enumeraciones
\usepackage{enumitem}

% Personalizar la portada
\usepackage{titling}

% Paquetes de tablas
\usepackage{multirow}


%------------------------------------------------------------------------

%Paquetes de figuras
\usepackage{caption}
\usepackage{subcaption} % Figuras al lado de otras
\usepackage{float}      % Poner figuras en el sitio indicado H.


% Paquetes de imágenes
\usepackage{graphicx}       % Paquete para añadir imágenes
\usepackage{transparent}    % Para manejar la opacidad de las figuras

% Paquete para usar colores
\usepackage[dvipsnames]{xcolor}
\usepackage{pagecolor}      % Para cambiar el color de la página

% Habilita tamaños de fuente mayores
\usepackage{fix-cm}

% Para los gráficos
\usepackage{tikz}

% Para poder situar los nodos en los grafos
\usetikzlibrary{positioning}


%------------------------------------------------------------------------

% Paquetes de matemáticas
\usepackage{mathtools, amsfonts, amssymb, mathrsfs}
\usepackage[makeroom]{cancel}     % Simplificar tachando
\usepackage{polynom}    % Divisiones y Ruffini
\usepackage{units} % Para poner fracciones diagonales con \nicefrac

\usepackage{pgfplots}   %Representar funciones
\pgfplotsset{compat=1.18}  % Versión 1.18

\usepackage{tikz-cd}    % Para usar diagramas de composiciones
\usetikzlibrary{calc}   % Para usar cálculo de coordenadas en tikz

%Definición de teoremas, etc.
\usepackage{amsthm}
%\swapnumbers   % Intercambia la posición del texto y de la numeración

\theoremstyle{plain}

\makeatletter
\@ifclassloaded{article}{
  \newtheorem{teo}{Teorema}[section]
}{
  \newtheorem{teo}{Teorema}[chapter]  % Se resetea en cada chapter
}
\makeatother

\newtheorem{coro}{Corolario}[teo]           % Se resetea en cada teorema
\newtheorem{prop}[teo]{Proposición}         % Usa el mismo contador que teorema
\newtheorem{lema}[teo]{Lema}                % Usa el mismo contador que teorema

\theoremstyle{remark}
\newtheorem*{observacion}{Observación}

\theoremstyle{definition}

\makeatletter
\@ifclassloaded{article}{
  \newtheorem{definicion}{Definición} [section]     % Se resetea en cada chapter
}{
  \newtheorem{definicion}{Definición} [chapter]     % Se resetea en cada chapter
}
\makeatother

\newtheorem*{notacion}{Notación}
\newtheorem*{ejemplo}{Ejemplo}
\newtheorem*{ejercicio*}{Ejercicio}             % No numerado
\newtheorem{ejercicio}{Ejercicio} [section]     % Se resetea en cada section


% Modificar el formato de la numeración del teorema "ejercicio"
\renewcommand{\theejercicio}{%
  \ifnum\value{section}=0 % Si no se ha iniciado ninguna sección
    \arabic{ejercicio}% Solo mostrar el número de ejercicio
  \else
    \thesection.\arabic{ejercicio}% Mostrar número de sección y número de ejercicio
  \fi
}


% \renewcommand\qedsymbol{$\blacksquare$}         % Cambiar símbolo QED
%------------------------------------------------------------------------

% Paquetes para encabezados
\usepackage{fancyhdr}
\pagestyle{fancy}
\fancyhf{}

\newcommand{\helv}{ % Modificación tamaño de letra
\fontfamily{}\fontsize{12}{12}\selectfont}
\setlength{\headheight}{15pt} % Amplía el tamaño del índice


%\usepackage{lastpage}   % Referenciar última pag   \pageref{LastPage}
\fancyfoot[C]{\thepage}

%------------------------------------------------------------------------

% Conseguir que no ponga "Capítulo 1". Sino solo "1."
\makeatletter
\@ifclassloaded{book}{
  \renewcommand{\chaptermark}[1]{\markboth{\thechapter.\ #1}{}} % En el encabezado
    
  \renewcommand{\@makechapterhead}[1]{%
  \vspace*{50\p@}%
  {\parindent \z@ \raggedright \normalfont
    \ifnum \c@secnumdepth >\m@ne
      \huge\bfseries \thechapter.\hspace{1em}\ignorespaces
    \fi
    \interlinepenalty\@M
    \Huge \bfseries #1\par\nobreak
    \vskip 40\p@
  }}
}
\makeatother

%------------------------------------------------------------------------
% Paquetes de cógido
\usepackage{minted}
\renewcommand\listingscaption{Código fuente}

\usepackage{fancyvrb}
% Personaliza el tamaño de los números de línea
\renewcommand{\theFancyVerbLine}{\small\arabic{FancyVerbLine}}

% Estilo para C++
\newminted{cpp}{
    frame=lines,
    framesep=2mm,
    baselinestretch=1.2,
    linenos,
    escapeinside=||
}

% para minted
\definecolor{LightGray}{rgb}{0.95,0.95,0.92}
\setminted{
    linenos=true,
    stepnumber=5,
    numberfirstline=true,
    autogobble,
    breaklines=true,
    breakautoindent=true,
    breaksymbolleft=,
    breaksymbolright=,
    breaksymbolindentleft=0pt,
    breaksymbolindentright=0pt,
    breaksymbolsepleft=0pt,
    breaksymbolsepright=0pt,
    fontsize=\footnotesize,
    bgcolor=LightGray,
    numbersep=10pt
}


\usepackage{listings} % Para incluir código desde un archivo

\renewcommand\lstlistingname{Código Fuente}
\renewcommand\lstlistlistingname{Índice de Códigos Fuente}

% Definir colores
\definecolor{vscodepurple}{rgb}{0.5,0,0.5}
\definecolor{vscodeblue}{rgb}{0,0,0.8}
\definecolor{vscodegreen}{rgb}{0,0.5,0}
\definecolor{vscodegray}{rgb}{0.5,0.5,0.5}
\definecolor{vscodebackground}{rgb}{0.97,0.97,0.97}
\definecolor{vscodelightgray}{rgb}{0.9,0.9,0.9}

% Configuración para el estilo de C similar a VSCode
\lstdefinestyle{vscode_C}{
  backgroundcolor=\color{vscodebackground},
  commentstyle=\color{vscodegreen},
  keywordstyle=\color{vscodeblue},
  numberstyle=\tiny\color{vscodegray},
  stringstyle=\color{vscodepurple},
  basicstyle=\scriptsize\ttfamily,
  breakatwhitespace=false,
  breaklines=true,
  captionpos=b,
  keepspaces=true,
  numbers=left,
  numbersep=5pt,
  showspaces=false,
  showstringspaces=false,
  showtabs=false,
  tabsize=2,
  frame=tb,
  framerule=0pt,
  aboveskip=10pt,
  belowskip=10pt,
  xleftmargin=10pt,
  xrightmargin=10pt,
  framexleftmargin=10pt,
  framexrightmargin=10pt,
  framesep=0pt,
  rulecolor=\color{vscodelightgray},
  backgroundcolor=\color{vscodebackground},
}

%------------------------------------------------------------------------

% Comandos definidos
\newcommand{\bb}[1]{\mathbb{#1}}
\newcommand{\cc}[1]{\mathcal{#1}}

% I prefer the slanted \leq
\let\oldleq\leq % save them in case they're every wanted
\let\oldgeq\geq
\renewcommand{\leq}{\leqslant}
\renewcommand{\geq}{\geqslant}

% Si y solo si
\newcommand{\sii}{\iff}

% Letras griegas
\newcommand{\eps}{\epsilon}
\newcommand{\veps}{\varepsilon}
\newcommand{\lm}{\lambda}

\newcommand{\ol}{\overline}
\newcommand{\ul}{\underline}
\newcommand{\wt}{\widetilde}
\newcommand{\wh}{\widehat}

\let\oldvec\vec
\renewcommand{\vec}{\overrightarrow}

% Derivadas parciales
\newcommand{\del}[2]{\frac{\partial #1}{\partial #2}}
\newcommand{\Del}[3]{\frac{\partial^{#1} #2}{\partial #3^{#1}}}
\newcommand{\deld}[2]{\dfrac{\partial #1}{\partial #2}}
\newcommand{\Deld}[3]{\dfrac{\partial^{#1} #2}{\partial #3^{#1}}}


\newcommand{\AstIg}{\stackrel{(\ast)}{=}}
\newcommand{\Hop}{\stackrel{L'H\hat{o}pital}{=}}

\newcommand{\red}[1]{{\color{red}#1}} % Para integrales, destacar los cambios.

% Método de integración
\newcommand{\MetInt}[2]{
    \left[\begin{array}{c}
        #1 \\ #2
    \end{array}\right]
}

% Declarar aplicaciones
% 1. Nombre aplicación
% 2. Dominio
% 3. Codominio
% 4. Variable
% 5. Imagen de la variable
\newcommand{\Func}[5]{
    \begin{equation*}
        \begin{array}{rrll}
            #1:& #2 & \longrightarrow & #3\\
               & #4 & \longmapsto & #5
        \end{array}
    \end{equation*}
}

%------------------------------------------------------------------------



\begin{document}

    % 1. Foto de fondo
    % 2. Título
    % 3. Encabezado Izquierdo
    % 4. Color de fondo
    % 5. Coord x del titulo
    % 6. Coord y del titulo
    % 7. Fecha

    
    % 1. Foto de fondo
% 2. Título
% 3. Encabezado Izquierdo
% 4. Color de fondo
% 5. Coord x del titulo
% 6. Coord y del titulo
% 7. Fecha

\newcommand{\portada}[7]{

    \portadaBase{#1}{#2}{#3}{#4}{#5}{#6}{#7}
    \portadaBook{#1}{#2}{#3}{#4}{#5}{#6}{#7}
}

\newcommand{\portadaExamen}[7]{

    \portadaBase{#1}{#2}{#3}{#4}{#5}{#6}{#7}
    \portadaArticle{#1}{#2}{#3}{#4}{#5}{#6}{#7}
}




\newcommand{\portadaBase}[7]{

    % Tiene la portada principal y la licencia Creative Commons
    
    % 1. Foto de fondo
    % 2. Título
    % 3. Encabezado Izquierdo
    % 4. Color de fondo
    % 5. Coord x del titulo
    % 6. Coord y del titulo
    % 7. Fecha
    
    
    \thispagestyle{empty}               % Sin encabezado ni pie de página
    \newgeometry{margin=0cm}        % Márgenes nulos para la primera página
    
    
    % Encabezado
    \fancyhead[L]{\helv #3}
    \fancyhead[R]{\helv \nouppercase{\leftmark}}
    
    
    \pagecolor{#4}        % Color de fondo para la portada
    
    \begin{figure}[p]
        \centering
        \transparent{0.3}           % Opacidad del 30% para la imagen
        
        \includegraphics[width=\paperwidth, keepaspectratio]{assets/#1}
    
        \begin{tikzpicture}[remember picture, overlay]
            \node[anchor=north west, text=white, opacity=1, font=\fontsize{60}{90}\selectfont\bfseries\sffamily, align=left] at (#5, #6) {#2};
            
            \node[anchor=south east, text=white, opacity=1, font=\fontsize{12}{18}\selectfont\sffamily, align=right] at (9.7, 3) {\textbf{\href{https://losdeldgiim.github.io/}{Los Del DGIIM}}};
            
            \node[anchor=south east, text=white, opacity=1, font=\fontsize{12}{15}\selectfont\sffamily, align=right] at (9.7, 1.8) {Doble Grado en Ingeniería Informática y Matemáticas\\Universidad de Granada};
        \end{tikzpicture}
    \end{figure}
    
    
    \restoregeometry        % Restaurar márgenes normales para las páginas subsiguientes
    \pagecolor{white}       % Restaurar el color de página
    
    
    \newpage
    \thispagestyle{empty}               % Sin encabezado ni pie de página
    \begin{tikzpicture}[remember picture, overlay]
        \node[anchor=south west, inner sep=3cm] at (current page.south west) {
            \begin{minipage}{0.5\paperwidth}
                \href{https://creativecommons.org/licenses/by-nc-nd/4.0/}{
                    \includegraphics[height=2cm]{assets/Licencia.png}
                }\vspace{1cm}\\
                Esta obra está bajo una
                \href{https://creativecommons.org/licenses/by-nc-nd/4.0/}{
                    Licencia Creative Commons Atribución-NoComercial-SinDerivadas 4.0 Internacional (CC BY-NC-ND 4.0).
                }\\
    
                Eres libre de compartir y redistribuir el contenido de esta obra en cualquier medio o formato, siempre y cuando des el crédito adecuado a los autores originales y no persigas fines comerciales. 
            \end{minipage}
        };
    \end{tikzpicture}
    
    
    
    % 1. Foto de fondo
    % 2. Título
    % 3. Encabezado Izquierdo
    % 4. Color de fondo
    % 5. Coord x del titulo
    % 6. Coord y del titulo
    % 7. Fecha


}


\newcommand{\portadaBook}[7]{

    % 1. Foto de fondo
    % 2. Título
    % 3. Encabezado Izquierdo
    % 4. Color de fondo
    % 5. Coord x del titulo
    % 6. Coord y del titulo
    % 7. Fecha

    % Personaliza el formato del título
    \pretitle{\begin{center}\bfseries\fontsize{42}{56}\selectfont}
    \posttitle{\par\end{center}\vspace{2em}}
    
    % Personaliza el formato del autor
    \preauthor{\begin{center}\Large}
    \postauthor{\par\end{center}\vfill}
    
    % Personaliza el formato de la fecha
    \predate{\begin{center}\huge}
    \postdate{\par\end{center}\vspace{2em}}
    
    \title{#2}
    \author{\href{https://losdeldgiim.github.io/}{Los Del DGIIM}}
    \date{Granada, #7}
    \maketitle
    
    \tableofcontents
}




\newcommand{\portadaArticle}[7]{

    % 1. Foto de fondo
    % 2. Título
    % 3. Encabezado Izquierdo
    % 4. Color de fondo
    % 5. Coord x del titulo
    % 6. Coord y del titulo
    % 7. Fecha

    % Personaliza el formato del título
    \pretitle{\begin{center}\bfseries\fontsize{42}{56}\selectfont}
    \posttitle{\par\end{center}\vspace{2em}}
    
    % Personaliza el formato del autor
    \preauthor{\begin{center}\Large}
    \postauthor{\par\end{center}\vspace{3em}}
    
    % Personaliza el formato de la fecha
    \predate{\begin{center}\huge}
    \postdate{\par\end{center}\vspace{5em}}
    
    \title{#2}
    \author{\href{https://losdeldgiim.github.io/}{Los Del DGIIM}}
    \date{Granada, #7}
    \thispagestyle{empty}               % Sin encabezado ni pie de página
    \maketitle
    \vfill
}
    \portadaExamen{ffccA4.jpg}{Ecuaciones\\Diferenciales I\\Examen VI}{Ecuaciones Diferenciales I. Examen VI}{MidnightBlue}{-8}{28}{2024-2025}{Arturo Olivares Martos}

    \begin{description}
        \item[Asignatura] Ecuaciones Diferenciales I
        \item[Curso Académico] 2023-24.
        \item[Grado] Doble Grado en Ingeniería Informática y Matemáticas.
        \item[Grupo] Único.
        \item[Profesor] Rafael Ortega Ríos.
        \item[Descripción] Convocatoria Ordinaria
        \item[Fecha] 10 de enero de 2024.
        %\item[Duración] 60 minutos.
    
    \end{description}
    \newpage

\begin{ejercicio}
    Resuelve el problema de valores iniciales siguiente, indicando si la solución está definida en todo $\bb{R}$:
    \begin{equation*}
        x'= -\frac{x}{x+t}, \quad x(0)=-1.
    \end{equation*}

    Hay dos opciones:
    \begin{description}
        \item[Razonar de forma no rigorsa]
        
        Tenemos que se trata una ecuación homogénea con dominio $$D=\{(t,x)\in\bb{R}^2\mid x+t<0\}.$$
        Podríamos intentar resolverlo aplicando la teoría, pero no podemos aplicar el cambio de variable $y=\nicefrac{x}{t}$ para la condición inicial dada.
        Resolvemos por tanto el problema sin tener en cuenta la condición inicial. Para poder aplicar dicho cambio de variable, tomamos como dominio
        $$D'=\{(t,x)\in\bb{R}^2\mid x+t< 0,t<0\}$$
        Aplicamos el cambio de variable siguiente:
        \Func{\varphi=(\varphi_1,\varphi_2)}{D'}{D_1'}{(t,x)}{(s,y)=(t,\nicefrac{x}{t})}

        Calculamos la inversa de $\varphi$:
        \Func{\varphi^{-1}}{D_1'}{D'}{(s,y)}{(t,x)=(s,sy)}

        Tenemos que $\varphi$ es un difeomorfismo entre $D'$ y $D_1'$ por ser $\varphi$ y $\varphi^{-1}$ biyectivas y de clase $C^1$.
        Además, es admisible puesto que no modifica la primera variable. Por tanto, la ecuación transformada es:
        \begin{equation*}
            y'=-\dfrac{x}{t^2} + \dfrac{x'}{t}
            = -\dfrac{y}{t} + \dfrac{1}{t}\cdot \left(-\dfrac{y}{y+1}\right)
            = \dfrac{1}{t}\cdot \dfrac{-2y-y^2}{y+1}.
        \end{equation*}

        Esta nueva ecuación diferencial es de variables separadas, con solución constante:
        \begin{equation*}
            y(t)=-2\qquad \forall t\in \bb{R}^-.
        \end{equation*}

        Para obtener las soluciones no constantes teniendo en cuenta que consideramos $-2y-y^2>0$, tenemos que:
        \begin{align*}
            \int \dfrac{y+1}{-2y-y^2}dy &= \int \dfrac{dt}{t}
            \Longrightarrow -\dfrac{1}{2}\ln(-2y-y^2) = \ln(-t) + C
            \Longrightarrow \\&\Longrightarrow \ln(-2y-y^2) = -\ln(t^2)-2C
            \Longrightarrow -2y-y^2 = \dfrac{1}{t^2}e^{-2C}
            \Longrightarrow \\&\Longrightarrow y^2+2y+\dfrac{K}{t^2}=0
            \Longrightarrow y(t)=-1\pm\sqrt{1-\dfrac{K}{t^2}},\qquad t\in \left]-\sqrt{K},0\right[.
        \end{align*}

        Deshaciendo el cambio de variable, obtenemos la solución en el dominio original:
        \begin{equation*}
            x(t)=ty(t)=t\left(-1\pm\sqrt{1-\dfrac{K}{t^2}}\right)=-t\mp\sqrt{t^2-K},\qquad t\in \left]-\sqrt{K},0\right[.
        \end{equation*}

        Retomamos ahora nuestro problema de valores iniciales, suponiendo que $t=0$ pertenece al dominio, veamos el valor de $K$:
        \begin{equation*}
            x(0)=-1=-0-\sqrt{0-K}\Longrightarrow K=-1.
        \end{equation*}
        

        Por tanto, y a modo de heurística\footnote{Esto es igual de válido, ya que hemos demostrado que efectivamente es una solución. Los pasos hasta llegar a esta función pueden serles útiles al lector.}, consideramos la función $$x(t)=-t-\sqrt{t^2+1}\qquad \forall t\in \bb{R}$$
        que cumple $x(0)=-1$. Veamos que $x(t)$ es solución de la ecuación diferencial:
        \begin{itemize}
            \item En primer lugar tenemos que $x$ es derivable en todo $\bb{R}$.
            \item Veamos ahora que $(t,x(t))\in D$ para todo $t\in\bb{R}$:
            \begin{align*}
                (t,x(t))\in D &\Longleftrightarrow x(t)+t<0
                \Longleftrightarrow -t-\sqrt{t^2+1}+t<0
                \Longleftrightarrow -\sqrt{t^2+1}<0
            \end{align*}
            \item Por último, es necesario ver que $x'(t)= -\dfrac{x(t)}{x(t)+t}$ para todo $t\in\bb{R}$:
            \begin{equation*}
                x'(t)=-1-\dfrac{t}{\sqrt{t^2+1}}=\dfrac{-t-\sqrt{t^2+1}}{\sqrt{t^2+1}}=-\dfrac{x(t)}{x(t)+t} \quad \forall t\in\bb{R}.
            \end{equation*}
        \end{itemize}
        Por tanto, la solución del problema de valores iniciales es $x(t)=-t-\sqrt{t^2+1}$, que está definida en todo $\bb{R}$.

        \item[Razonar de forma rigurosa]
        
        Tiene como dominio $D=\{(t,x)\in\bb{R}^2\mid x+t<0\}$. Aplicamos el cambio de variable $y=x+t$, de forma que:
        \Func{\varphi}{D}{D_1}{(t,x)}{(s,y)=(t,x+t)}

        Calculamos la inversa de $\varphi$:
        \Func{\varphi^{-1}}{D_1}{D}{(s,y)}{(t,x)=(s,y-s)}

        Calculamos ahora $D_1=\varphi(D)$:
        \begin{align*}
            D_1 &= \{(s,y)\in\bb{R}^2\mid (s,y-s)\in D\}
            = \{(s,y)\in\bb{R}^2\mid y<0\}
        \end{align*}

        Por tanto, $\varphi$ es un difeomorfismo entre $D$ y $D_1$ y es admisible al no modificar la primera variable. La ecuación transformada es:
        \begin{equation*}
            y'=\dfrac{dy}{dt}=1+x'=1-\dfrac{x}{x+t}=1-\dfrac{y-s}{y}=\dfrac{s}{y}\qquad \text{con dominio }D_1.
        \end{equation*}
        Tenemos que $y'$ es una ecuación de variables separadas que no tiene soluciones constantes. Por tanto, la solución general es:
        \begin{equation*}
            \int ydy = \int sds \Longrightarrow y^2 - s^2 =C
            \Longrightarrow y(s)=-\sqrt{s^2+C},\qquad \text{con dominio }\bb{R}
        \end{equation*}

        Deshaciendo el cambio de variable, obtenemos la solución en el dominio original:
        \begin{equation*}
            x(t)=y-s=-t-\sqrt{t^2+C},\qquad t\in \bb{R}.
        \end{equation*}

        Para obtener la solución del problema de valores iniciales, calculamos $C$:
        \begin{equation*}
            x(0)=-1=-0-\sqrt{0+C}\Longrightarrow C=1.
        \end{equation*}

        Por tanto, la solución del problema de valores iniciales es $x(t)=-t-\sqrt{t^2+1}$, que está definida en todo $\bb{R}$.
    \end{description}
\end{ejercicio}

\begin{ejercicio}
    Se considera la transformación
    \Func{\varphi}{\bb{R}^2}{\bb{R}^2}{(t,x)}{(s,y)=(-2e^x,e^{-3t})}
    Determina $\Omega=\varphi(\bb{R}^2)$ y prueba que $\varphi$ define un difeomorfismo entre $\bb{R}^2$ y $\Omega$.
    Se considera la ecuación diferencial
    \begin{equation*}
        x'=f(t,x)
    \end{equation*}
    con $f:\bb{R}^2\to\bb{R}$ continua. ¿Bajo qué condiciones sobre $f$ se puede asegurar que el difeomorfismo es admisible para esta ecuación? Encuentra la ecuación transportada al dominio $\Omega$.\\

    Buscamos la inversa de $\varphi$, para lo cual despejamos $t$ y $x$ en función de $s$ e $y$:
    \begin{align*}
        s=-2e^x &\Longrightarrow x=\ln\left(-\dfrac{s}{2}\right),\\
        y=e^{-3t} &\Longrightarrow t=-\dfrac{1}{3}\ln y.
    \end{align*}
    Por tanto, la inversa es:
    \Func{\varphi^{-1}}{\Omega}{\bb{R}^2}{(s,y)}{(t,x)=\left(-\dfrac{1}{3}\ln y,\ln\left(-\dfrac{s}{2}\right)\right)}

    Para que $\varphi^{-1}$ esté bien definida, es necesario que $y>0$ y $s<0$. Por tanto, $\Omega\subset \bb{R}^- \times \bb{R}^+$.
    \begin{align*}
        \Omega &= \varphi(\bb{R}^2)=\{(s,y)\in \bb{R}^2\mid (\nicefrac{-1}{3}\ln y,\ln(\nicefrac{-s}{2}))\in \bb{R}^2\}=\bb{R}^-\times \bb{R}^+
    \end{align*}

    Para que $\varphi$ defina un difeomorfismo entre $\bb{R}^2$ y $\Omega$, es necesario que $\varphi$ sea biyectiva y que $\varphi$ y $\varphi^{-1}$ sean de clase $C^1$. Lo primer es directo por haber despejado de forma única $t$ y $x$ en función de $s$ e $y$. Para lo segundo, como el logaritmo lo es y la composición de funciones es de clase $C^1$, $\varphi$ y $\varphi^{-1}$ son de clase $C^1$. Por tanto, $\varphi$ define un difeomorfismo entre $\bb{R}^2$ y $\Omega$.\\

    Para que sea admisible, considerando $\varphi=(\varphi_1,\varphi_2)$, es necesario que:
    \begin{equation*}
        \dfrac{\partial \varphi_1}{\partial t} + \dfrac{\partial \varphi_1}{\partial x}f(t,x)
        = -2e^x\cdot f(t,x) \neq 0 \Longrightarrow f(t,x)\neq 0\qquad \forall (t,x)\in\bb{R}^2.
    \end{equation*}

    La ecuación transportada al dominio $\Omega$ es:
    \begin{equation*}
        y'=\dfrac{dy}{dt}\cdot \dfrac{dt}{ds}=\dfrac{-3e^{-3t}}{-2e^x f(t,x)}=\dfrac{-3y}{sf\left(-\dfrac{1}{3}\ln y,\ln\left(-\dfrac{s}{2}\right)\right)},\qquad \text{con dominio }\Omega.
    \end{equation*}
\end{ejercicio}

\begin{ejercicio}
    Se considera la ecuación
    \begin{equation*}
        x''+a(t)x=0
    \end{equation*}
    donde $a:I\to\bb{R}$ es una función continua en un intervalo abierto $I$. Se supone que $\varphi$ es una solución que cumple
    \begin{equation*}
        \varphi(t)>0\quad\forall t\in I.
    \end{equation*}
    \begin{enumerate}
        \item Demuestra que existe una única función $\psi:I\to\bb{R}$ que cumple
        \begin{equation*}
            W(\varphi,\psi)(t)=7,\quad t\in I,\quad \psi(0)=0.
        \end{equation*}
        \item Demuestra que la pareja $\varphi,\psi$ forma un sistema fundamental de la ecuación de partida.
    \end{enumerate}
\end{ejercicio}

\begin{ejercicio}
    Responda a las siguientes cuestiones:
    \begin{enumerate}
        \item Calcula $e^A$ para la matriz
        \begin{equation*}
            A=\begin{pmatrix}
                0 & a & b\\
                0 & 0 & c\\
                0 & 0 & 0
            \end{pmatrix},
        \end{equation*}
        con $a,b,c\in\bb{R}$.
        \item Encuentra una matriz fundamental del sistema
        \begin{align*}
            x_1'&=x_1+ax_2+bx_3,\\
            x_2'&=x_2+cx_3,\\
            x_3'&=x_3.
        \end{align*}
    \end{enumerate}
\end{ejercicio}
    
\end{document}