\documentclass[12pt]{article}

% Idioma y codificación
\usepackage[spanish, es-tabla]{babel}       %es-tabla para que se titule "Tabla"
\usepackage[utf8]{inputenc}

% Márgenes
\usepackage[a4paper,top=3cm,bottom=2.5cm,left=3cm,right=3cm]{geometry}

% Comentarios de bloque
\usepackage{verbatim}

% Paquetes de links
\usepackage[hidelinks]{hyperref}    % Permite enlaces
\usepackage{url}                    % redirecciona a la web

% Más opciones para enumeraciones
\usepackage{enumitem}

% Personalizar la portada
\usepackage{titling}

% Paquetes de tablas
\usepackage{multirow}


%------------------------------------------------------------------------

%Paquetes de figuras
\usepackage{caption}
\usepackage{subcaption} % Figuras al lado de otras
\usepackage{float}      % Poner figuras en el sitio indicado H.


% Paquetes de imágenes
\usepackage{graphicx}       % Paquete para añadir imágenes
\usepackage{transparent}    % Para manejar la opacidad de las figuras

% Paquete para usar colores
\usepackage[dvipsnames]{xcolor}
\usepackage{pagecolor}      % Para cambiar el color de la página

% Habilita tamaños de fuente mayores
\usepackage{fix-cm}

% Para los gráficos
\usepackage{tikz}

% Para poder situar los nodos en los grafos
\usetikzlibrary{positioning}


%------------------------------------------------------------------------

% Paquetes de matemáticas
\usepackage{mathtools, amsfonts, amssymb, mathrsfs}
\usepackage[makeroom]{cancel}     % Simplificar tachando
\usepackage{polynom}    % Divisiones y Ruffini
\usepackage{units} % Para poner fracciones diagonales con \nicefrac

\usepackage{pgfplots}   %Representar funciones
\pgfplotsset{compat=1.18}  % Versión 1.18

\usepackage{tikz-cd}    % Para usar diagramas de composiciones
\usetikzlibrary{calc}   % Para usar cálculo de coordenadas en tikz

%Definición de teoremas, etc.
\usepackage{amsthm}
%\swapnumbers   % Intercambia la posición del texto y de la numeración

\theoremstyle{plain}

\makeatletter
\@ifclassloaded{article}{
  \newtheorem{teo}{Teorema}[section]
}{
  \newtheorem{teo}{Teorema}[chapter]  % Se resetea en cada chapter
}
\makeatother

\newtheorem{coro}{Corolario}[teo]           % Se resetea en cada teorema
\newtheorem{prop}[teo]{Proposición}         % Usa el mismo contador que teorema
\newtheorem{lema}[teo]{Lema}                % Usa el mismo contador que teorema

\theoremstyle{remark}
\newtheorem*{observacion}{Observación}

\theoremstyle{definition}

\makeatletter
\@ifclassloaded{article}{
  \newtheorem{definicion}{Definición} [section]     % Se resetea en cada chapter
}{
  \newtheorem{definicion}{Definición} [chapter]     % Se resetea en cada chapter
}
\makeatother

\newtheorem*{notacion}{Notación}
\newtheorem*{ejemplo}{Ejemplo}
\newtheorem*{ejercicio*}{Ejercicio}             % No numerado
\newtheorem{ejercicio}{Ejercicio} [section]     % Se resetea en cada section


% Modificar el formato de la numeración del teorema "ejercicio"
\renewcommand{\theejercicio}{%
  \ifnum\value{section}=0 % Si no se ha iniciado ninguna sección
    \arabic{ejercicio}% Solo mostrar el número de ejercicio
  \else
    \thesection.\arabic{ejercicio}% Mostrar número de sección y número de ejercicio
  \fi
}


% \renewcommand\qedsymbol{$\blacksquare$}         % Cambiar símbolo QED
%------------------------------------------------------------------------

% Paquetes para encabezados
\usepackage{fancyhdr}
\pagestyle{fancy}
\fancyhf{}

\newcommand{\helv}{ % Modificación tamaño de letra
\fontfamily{}\fontsize{12}{12}\selectfont}
\setlength{\headheight}{15pt} % Amplía el tamaño del índice


%\usepackage{lastpage}   % Referenciar última pag   \pageref{LastPage}
\fancyfoot[C]{\thepage}

%------------------------------------------------------------------------

% Conseguir que no ponga "Capítulo 1". Sino solo "1."
\makeatletter
\@ifclassloaded{book}{
  \renewcommand{\chaptermark}[1]{\markboth{\thechapter.\ #1}{}} % En el encabezado
    
  \renewcommand{\@makechapterhead}[1]{%
  \vspace*{50\p@}%
  {\parindent \z@ \raggedright \normalfont
    \ifnum \c@secnumdepth >\m@ne
      \huge\bfseries \thechapter.\hspace{1em}\ignorespaces
    \fi
    \interlinepenalty\@M
    \Huge \bfseries #1\par\nobreak
    \vskip 40\p@
  }}
}
\makeatother

%------------------------------------------------------------------------
% Paquetes de cógido
\usepackage{minted}
\renewcommand\listingscaption{Código fuente}

\usepackage{fancyvrb}
% Personaliza el tamaño de los números de línea
\renewcommand{\theFancyVerbLine}{\small\arabic{FancyVerbLine}}

% Estilo para C++
\newminted{cpp}{
    frame=lines,
    framesep=2mm,
    baselinestretch=1.2,
    linenos,
    escapeinside=||
}

% para minted
\definecolor{LightGray}{rgb}{0.95,0.95,0.92}
\setminted{
    linenos=true,
    stepnumber=5,
    numberfirstline=true,
    autogobble,
    breaklines=true,
    breakautoindent=true,
    breaksymbolleft=,
    breaksymbolright=,
    breaksymbolindentleft=0pt,
    breaksymbolindentright=0pt,
    breaksymbolsepleft=0pt,
    breaksymbolsepright=0pt,
    fontsize=\footnotesize,
    bgcolor=LightGray,
    numbersep=10pt
}


\usepackage{listings} % Para incluir código desde un archivo

\renewcommand\lstlistingname{Código Fuente}
\renewcommand\lstlistlistingname{Índice de Códigos Fuente}

% Definir colores
\definecolor{vscodepurple}{rgb}{0.5,0,0.5}
\definecolor{vscodeblue}{rgb}{0,0,0.8}
\definecolor{vscodegreen}{rgb}{0,0.5,0}
\definecolor{vscodegray}{rgb}{0.5,0.5,0.5}
\definecolor{vscodebackground}{rgb}{0.97,0.97,0.97}
\definecolor{vscodelightgray}{rgb}{0.9,0.9,0.9}

% Configuración para el estilo de C similar a VSCode
\lstdefinestyle{vscode_C}{
  backgroundcolor=\color{vscodebackground},
  commentstyle=\color{vscodegreen},
  keywordstyle=\color{vscodeblue},
  numberstyle=\tiny\color{vscodegray},
  stringstyle=\color{vscodepurple},
  basicstyle=\scriptsize\ttfamily,
  breakatwhitespace=false,
  breaklines=true,
  captionpos=b,
  keepspaces=true,
  numbers=left,
  numbersep=5pt,
  showspaces=false,
  showstringspaces=false,
  showtabs=false,
  tabsize=2,
  frame=tb,
  framerule=0pt,
  aboveskip=10pt,
  belowskip=10pt,
  xleftmargin=10pt,
  xrightmargin=10pt,
  framexleftmargin=10pt,
  framexrightmargin=10pt,
  framesep=0pt,
  rulecolor=\color{vscodelightgray},
  backgroundcolor=\color{vscodebackground},
}

%------------------------------------------------------------------------

% Comandos definidos
\newcommand{\bb}[1]{\mathbb{#1}}
\newcommand{\cc}[1]{\mathcal{#1}}

% I prefer the slanted \leq
\let\oldleq\leq % save them in case they're every wanted
\let\oldgeq\geq
\renewcommand{\leq}{\leqslant}
\renewcommand{\geq}{\geqslant}

% Si y solo si
\newcommand{\sii}{\iff}

% Letras griegas
\newcommand{\eps}{\epsilon}
\newcommand{\veps}{\varepsilon}
\newcommand{\lm}{\lambda}

\newcommand{\ol}{\overline}
\newcommand{\ul}{\underline}
\newcommand{\wt}{\widetilde}
\newcommand{\wh}{\widehat}

\let\oldvec\vec
\renewcommand{\vec}{\overrightarrow}

% Derivadas parciales
\newcommand{\del}[2]{\frac{\partial #1}{\partial #2}}
\newcommand{\Del}[3]{\frac{\partial^{#1} #2}{\partial #3^{#1}}}
\newcommand{\deld}[2]{\dfrac{\partial #1}{\partial #2}}
\newcommand{\Deld}[3]{\dfrac{\partial^{#1} #2}{\partial #3^{#1}}}


\newcommand{\AstIg}{\stackrel{(\ast)}{=}}
\newcommand{\Hop}{\stackrel{L'H\hat{o}pital}{=}}

\newcommand{\red}[1]{{\color{red}#1}} % Para integrales, destacar los cambios.

% Método de integración
\newcommand{\MetInt}[2]{
    \left[\begin{array}{c}
        #1 \\ #2
    \end{array}\right]
}

% Declarar aplicaciones
% 1. Nombre aplicación
% 2. Dominio
% 3. Codominio
% 4. Variable
% 5. Imagen de la variable
\newcommand{\Func}[5]{
    \begin{equation*}
        \begin{array}{rrll}
            #1:& #2 & \longrightarrow & #3\\
               & #4 & \longmapsto & #5
        \end{array}
    \end{equation*}
}

%------------------------------------------------------------------------



\begin{document}

    % 1. Foto de fondo
    % 2. Título
    % 3. Encabezado Izquierdo
    % 4. Color de fondo
    % 5. Coord x del titulo
    % 6. Coord y del titulo
    % 7. Fecha

    
    % 1. Foto de fondo
% 2. Título
% 3. Encabezado Izquierdo
% 4. Color de fondo
% 5. Coord x del titulo
% 6. Coord y del titulo
% 7. Fecha

\newcommand{\portada}[7]{

    \portadaBase{#1}{#2}{#3}{#4}{#5}{#6}{#7}
    \portadaBook{#1}{#2}{#3}{#4}{#5}{#6}{#7}
}

\newcommand{\portadaExamen}[7]{

    \portadaBase{#1}{#2}{#3}{#4}{#5}{#6}{#7}
    \portadaArticle{#1}{#2}{#3}{#4}{#5}{#6}{#7}
}




\newcommand{\portadaBase}[7]{

    % Tiene la portada principal y la licencia Creative Commons
    
    % 1. Foto de fondo
    % 2. Título
    % 3. Encabezado Izquierdo
    % 4. Color de fondo
    % 5. Coord x del titulo
    % 6. Coord y del titulo
    % 7. Fecha
    
    
    \thispagestyle{empty}               % Sin encabezado ni pie de página
    \newgeometry{margin=0cm}        % Márgenes nulos para la primera página
    
    
    % Encabezado
    \fancyhead[L]{\helv #3}
    \fancyhead[R]{\helv \nouppercase{\leftmark}}
    
    
    \pagecolor{#4}        % Color de fondo para la portada
    
    \begin{figure}[p]
        \centering
        \transparent{0.3}           % Opacidad del 30% para la imagen
        
        \includegraphics[width=\paperwidth, keepaspectratio]{assets/#1}
    
        \begin{tikzpicture}[remember picture, overlay]
            \node[anchor=north west, text=white, opacity=1, font=\fontsize{60}{90}\selectfont\bfseries\sffamily, align=left] at (#5, #6) {#2};
            
            \node[anchor=south east, text=white, opacity=1, font=\fontsize{12}{18}\selectfont\sffamily, align=right] at (9.7, 3) {\textbf{\href{https://losdeldgiim.github.io/}{Los Del DGIIM}}};
            
            \node[anchor=south east, text=white, opacity=1, font=\fontsize{12}{15}\selectfont\sffamily, align=right] at (9.7, 1.8) {Doble Grado en Ingeniería Informática y Matemáticas\\Universidad de Granada};
        \end{tikzpicture}
    \end{figure}
    
    
    \restoregeometry        % Restaurar márgenes normales para las páginas subsiguientes
    \pagecolor{white}       % Restaurar el color de página
    
    
    \newpage
    \thispagestyle{empty}               % Sin encabezado ni pie de página
    \begin{tikzpicture}[remember picture, overlay]
        \node[anchor=south west, inner sep=3cm] at (current page.south west) {
            \begin{minipage}{0.5\paperwidth}
                \href{https://creativecommons.org/licenses/by-nc-nd/4.0/}{
                    \includegraphics[height=2cm]{assets/Licencia.png}
                }\vspace{1cm}\\
                Esta obra está bajo una
                \href{https://creativecommons.org/licenses/by-nc-nd/4.0/}{
                    Licencia Creative Commons Atribución-NoComercial-SinDerivadas 4.0 Internacional (CC BY-NC-ND 4.0).
                }\\
    
                Eres libre de compartir y redistribuir el contenido de esta obra en cualquier medio o formato, siempre y cuando des el crédito adecuado a los autores originales y no persigas fines comerciales. 
            \end{minipage}
        };
    \end{tikzpicture}
    
    
    
    % 1. Foto de fondo
    % 2. Título
    % 3. Encabezado Izquierdo
    % 4. Color de fondo
    % 5. Coord x del titulo
    % 6. Coord y del titulo
    % 7. Fecha


}


\newcommand{\portadaBook}[7]{

    % 1. Foto de fondo
    % 2. Título
    % 3. Encabezado Izquierdo
    % 4. Color de fondo
    % 5. Coord x del titulo
    % 6. Coord y del titulo
    % 7. Fecha

    % Personaliza el formato del título
    \pretitle{\begin{center}\bfseries\fontsize{42}{56}\selectfont}
    \posttitle{\par\end{center}\vspace{2em}}
    
    % Personaliza el formato del autor
    \preauthor{\begin{center}\Large}
    \postauthor{\par\end{center}\vfill}
    
    % Personaliza el formato de la fecha
    \predate{\begin{center}\huge}
    \postdate{\par\end{center}\vspace{2em}}
    
    \title{#2}
    \author{\href{https://losdeldgiim.github.io/}{Los Del DGIIM}}
    \date{Granada, #7}
    \maketitle
    
    \tableofcontents
}




\newcommand{\portadaArticle}[7]{

    % 1. Foto de fondo
    % 2. Título
    % 3. Encabezado Izquierdo
    % 4. Color de fondo
    % 5. Coord x del titulo
    % 6. Coord y del titulo
    % 7. Fecha

    % Personaliza el formato del título
    \pretitle{\begin{center}\bfseries\fontsize{42}{56}\selectfont}
    \posttitle{\par\end{center}\vspace{2em}}
    
    % Personaliza el formato del autor
    \preauthor{\begin{center}\Large}
    \postauthor{\par\end{center}\vspace{3em}}
    
    % Personaliza el formato de la fecha
    \predate{\begin{center}\huge}
    \postdate{\par\end{center}\vspace{5em}}
    
    \title{#2}
    \author{\href{https://losdeldgiim.github.io/}{Los Del DGIIM}}
    \date{Granada, #7}
    \thispagestyle{empty}               % Sin encabezado ni pie de página
    \maketitle
    \vfill
}
    \portadaExamen{ffccA4.jpg}{Ecuaciones\\Diferenciales I\\Examen XII}{Ecuaciones Diferenciales I. Examen XII}{MidnightBlue}{-8}{28}{2024-2025}{Arturo Olivares Martos}

    \begin{description}
        \item[Asignatura] Ecuaciones Diferenciales I
        \item[Curso Académico] 2017-18.
        %\item[Grado] Doble Grado en Ingeniería Informática y Matemáticas.
        \item[Grupo] B.
        \item[Profesor] Rafael Ortega Ríos.
        \item[Descripción] Parcial B.
        \item[Fecha] 10 de Mayo de 2018.
        %\item[Duración] 60 minutos.
    
    \end{description}
    \newpage

    \begin{ejercicio}
        Se considera el campo de fuerzas siguiente:
        \Func{F}{\bb{R}\times \bb{R}^+}{\bb{R}^2}{(x,y)}{\left(\dfrac{2x}{y},-\dfrac{x^2}{y^2}\right)}

        ¿Admite un potencial? Calcula el trabajo a lo largo de la curva dada por:
        \begin{equation*}
            \gamma(\theta) = (\cos \theta, 1 + \sen \theta), \quad \theta \in [0,\pi].
        \end{equation*}

        Notamos $F=(F_1,F_2)$. En primer lugar, vemos que $F_1,F_2\in C^1(\bb{R}\times \bb{R}^+)$ por ser cociente de funciones polinómicas en las que no se anula el denominador. Veamos ahora que $\bb{R}\times \bb{R}^+$ es convexo.
        \begin{itemize}
            \item Sean $(x_1,y_1),(x_2,y_2)\in \bb{R}\times \bb{R}^+$ y $t \in [0,1]$; y veamos que el segmento que une $(x_1,y_1)$ y $(x_2,y_2)$, notado por $[(x_1,y_1),(x_2,y_2)]$, está contenido en $\bb{R}\times \bb{R}^+$. Tenemos que:
            \begin{align*}
                [(x_1,y_1),(x_2,y_2)]
                &= \{t(x_1,y_1) + (1-t)(x_2,y_2) \mid t \in [0,1]\}
                =\\&= \{(tx_1 + (1-t)x_2,ty_1 + (1-t)y_2) \mid t \in [0,1]\}.
            \end{align*}

            Para que se tenga que $[(x_1,y_1),(x_2,y_2)]\subseteq \bb{R}\times \bb{R}^+$, es necesario que:
            \begin{align*}
                ty_1 + (1-t)y_2 > 0
            \end{align*}
            Tenemos que ambos sumandos son no-negativos. Además, en el caso de que uno de ellos se anule el otro no se anula, luego la suma es positiva. Por tanto, $[(x_1,y_1),(x_2,y_2)]\subseteq \bb{R}\times \bb{R}^+$.
        \end{itemize}

        Por tanto, tenemos que $\bb{R}\times \bb{R}^+$ es convexo, luego en particular es conexo. Veamos ahora si cumple la condición de exactitud:
        \begin{align*}
            \dfrac{\partial F_2}{\partial x}(x,y) = -\dfrac{2x}{y^2} = \dfrac{\partial F_1}{\partial y}(x,y).
        \end{align*}

        Por tanto, también cumple la condición de exactitud. Por tanto, sabemos que sí existe un potencial $U:\bb{R}\times \bb{R}^+\to \bb{R}$ tal que $\nabla U = F$.\\

        Como existe un potencial, el trabajo a lo largo de la curva $\gamma$ se puede calcular como:
        \begin{align*}
            W(\gamma) = U(\gamma(\pi)) - U(\gamma(0)).
        \end{align*}

        Veamos en primer lugar que no es una trayectoria cerrada, ya que en ese caso habríamos terminado. Tenemos que:
        \begin{align*}
            \gamma(\pi) = (\cos \pi, 1 + \sen \pi) = (-1,1) \neq (1,1) = \gamma(0).
        \end{align*}

        Por tanto, vemos que es necesario calcular $U$. Como $\dfrac{\partial U}{\partial x} = F_1$, tenemos que:
        \begin{align*}
            U(x,y) = \int F_1(x,y)\,dx = \int \dfrac{2x}{y}\,dx = \dfrac{x^2}{y} + \varphi(y).
        \end{align*}
        donde $\varphi:\bb{R}^+\to \bb{R}$ es una función que representa la constante de integración en función de $y$. Además, como $U\in C^2(\bb{R}\times \bb{R}^+)$, tenemos que $\varphi \in C^1(\bb{R}^+)$.
        Ahora, como $\dfrac{\partial U}{\partial y} = F_2$, tenemos que:
        \begin{align*}
            \dfrac{\partial U}{\partial y}(x,y) = -\dfrac{x^2}{y^2} + \varphi'(y) = -\dfrac{x^2}{y^2} \Longrightarrow \varphi'(y) = 0  \qquad \forall y \in \bb{R}^+.
        \end{align*}
        Como $\varphi\in C^1(\bb{R}^+)$, tenemos que $\varphi$ es constante. Supongamos por ejemplo $\varphi=0$, aunque podríamos haber elegido cualquier otro valor (el potencial es único salvo una constante aditiva). Por tanto:
        \begin{align*}
            U(x,y) = \dfrac{x^2}{y} \qquad \forall (x,y) \in \bb{R}\times \bb{R}^+.
        \end{align*}

        Por tanto, el trabajo a lo largo de la curva $\gamma$ es:
        \begin{align*}
            W(\gamma) &= U(\gamma(\pi)) - U(\gamma(0)) = U(\cos \pi, 1 + \sen \pi) - U(\cos 0, 1 + \sen 0) =\\
            &= U(-1,1) - U(1,1) = \dfrac{(-1)^2}{1} - \dfrac{1^2}{1} = 1 - 1 = 0.
        \end{align*}

    \end{ejercicio}

    \begin{ejercicio}
        Demuestra que la ecuación diferencial
        \begin{equation*}
            x' = (x - t)^2
        \end{equation*}
        admite una solución polinómica de grado uno. Encuentra un cambio de variable que transforme esta ecuación en una ecuación lineal.
    \end{ejercicio}

    \begin{ejercicio}
        Dadas dos funciones $P,Q \in C^1(\bb{R}^2)$ que cumplen $\dfrac{\partial P}{\partial y} = \dfrac{\partial Q}{\partial x}$, demuestra que la función
        \begin{equation*}
            U(x,y) = \int_0^y Q(0,s)\,ds + \int_0^x P(s,y)\,ds
        \end{equation*}
        es solución de las ecuaciones:
        \begin{equation*}
            \dfrac{\partial U}{\partial x} = P(x,y), \quad \dfrac{\partial U}{\partial y} = Q(x,y).
        \end{equation*}

        Como $P,Q$ son continuas en $\bb{R}^2$, por el Teorema Fundamental del Cálculo, las integrales que encontramos son de clase $C^1(\bb{R})$, por lo que $U\in C^1(\bb{R}^2)$. Calculamos las derivadas parciales de cada una de las integrales.
        En primer lugar, de forma directa tenemos que:
        \begin{equation*}
            \dfrac{\partial}{\partial x}\left(\int_0^y Q(0,s)\,ds\right)(x,y) = 0
        \end{equation*}

        Usando el Teorema Fundamental del Cálculo, tenemos que:
        \begin{equation*}
            \dfrac{\partial}{\partial y}\left(\int_0^y Q(0,s)\,ds\right)(x,y) = Q(0,y)\qquad 
            \dfrac{\partial}{\partial x}\left(\int_0^x P(s,y)\,ds\right)(x,y) = P(x,y)
        \end{equation*}

        Por último, empleado el Teorema de la Derivada de Integrales dependientes de un parámetro, tenemos que:
        \begin{equation*}
            \dfrac{\partial}{\partial y}\left(\int_0^x P(s,y)\,ds\right)(x,y) = \int_0^x \dfrac{\partial P}{\partial y}(s,y)\,ds
            \AstIg \int_0^x \dfrac{\partial Q}{\partial x}(s,y)\,ds \stackrel{(\ast\ast)}{=} Q(x,y)-Q(0,y)
        \end{equation*}
        donde en $(\ast)$ hemos usado la hipótesis del enunciado y en $(\ast\ast)$ la Regla de Barrow. Por tanto, tenemos que:
        \begin{equation*}
            \dfrac{\partial U}{\partial x}(x,y) = P(x,y), \qquad \dfrac{\partial U}{\partial y}(x,y) = Q(0,y) + Q(x,y) - Q(0,y) = Q(x,y).
        \end{equation*}

        Por tanto, $U$ es solución de las ecuaciones dadas.
    \end{ejercicio}

    \begin{ejercicio}
        Considera las funciones $f_1,f_2:\bb{R}\to \bb{R}$ dadas por:
        \begin{equation*}
            f_1(t) = \begin{cases}
                0 & \text{si } t \geq 0,\\
                t^3 & \text{si } t < 0,
            \end{cases} \quad f_2(t) = \begin{cases}
                t^3 & \text{si } t \geq 0,\\
                0 & \text{si } t < 0.
            \end{cases}
        \end{equation*}
        ¿Son linealmente independientes? Calcula su Wronskiano.\\

        Calculamos su Wronskiano en primer lugar. Por el carácter local de la derivabilidad, tenemos que:
        \begin{equation*}
            f_1'(t) = \begin{cases}
                0 & \text{si } t > 0,\\
                3t^2 & \text{si } t < 0,
            \end{cases}\quad
            f_2'(t) = \begin{cases}
                3t^2 & \text{si } t > 0,\\
                0 & \text{si } t < 0.
            \end{cases}
        \end{equation*}

        Además, como los límites laterales en el origen coinciden, tenemos que $f_1,f_2$ son derivables en todo $\bb{R}$, por lo que podemos considerar su Wronskiano.
        Para $t>0$, tenemos que:
        \begin{equation*}
            W(f_1,f_2)(t) = \begin{vmatrix}
                0 & t^3\\
                0 & 3t^2
            \end{vmatrix} = 0.
        \end{equation*}

        Para $t<0$, tenemos que:
        \begin{equation*}
            W(f_1,f_2)(t) = \begin{vmatrix}
                t^3 & 0\\
                3t^2 & 0
            \end{vmatrix} = 0.
        \end{equation*}

        Por tanto, tenemos que:
        \begin{equation*}
            W(f_1,f_2)(t) = 0 \qquad \forall t \in \bb{R}.
        \end{equation*}

        Por tanto, no nos es de ayuda para determinar si son linealmente independientes, por lo que debemos recurrir a la definición. Buscamos por tanto $c_1,c_2\in \bb{R}$ tales que:
        \begin{equation*}
            c_1f_1(t) + c_2f_2(t) = 0 \qquad \forall t \in \bb{R}.
        \end{equation*}
        \begin{itemize}
            \item Para $t=1$, tenemos que:
            \begin{equation*}
                0 = c_1f_1(1) + c_2f_2(1) = c_1\cdot 0 + c_2\cdot 1 = c_2
                \Longrightarrow c_2 = 0.
            \end{equation*}

            \item Para $t=-1$, tenemos que:
            \begin{equation*}
                0 = c_1f_1(-1) + c_2f_2(-1) = c_1\cdot (-1)^3 + c_2\cdot 0 = -c_1
                \Longrightarrow c_1 = 0.
            \end{equation*}
        \end{itemize}

        Por tanto, tenemos que $c_1=c_2=0$, por lo que $f_1,f_2$ son linealmente independientes.
    \end{ejercicio}

    \begin{ejercicio}
        Dada una función continua $a:\bb{R}\to \bb{R}$, $t \mapsto a(t)$, se denota por $Z$ al conjunto de funciones $x \in C^1(\bb{R})$, $x = x(t)$, que satisfacen la ecuación integro-diferencial
        \begin{equation}\label{eq:ed1}
            x'(t) + x(t) = \int_0^t a(s)x(s)\,ds, \quad t \in \bb{R}.
        \end{equation}
        Demuestra que $Z$ admite una estructura de espacio vectorial. ¿Qué dimensión tiene?\\

        Como $Z\subset C^1(\bb{R})\subset \bb{F}(\bb{R},\bb{R})$, tan solo tendremos que probar que $Z$ es un subespacio vectorial de $\bb{F}(\bb{R},\bb{R})$. Para ello, tendremos que probar que $Z\neq 0$ y que es cerrado para la suma y el producto por escalares.\\

        Sea $x\equiv 0$, la función constantemente nula. Tenemos que:
        \begin{equation*}
            x'(t) + x(t) = 0 + 0 = 0 = \int_0^t 0\,ds = \int_0^t a(s)\cdot 0\,ds\qquad \forall t \in \bb{R}.
        \end{equation*}

        Por tanto, $x\in Z$, luego $Z\neq 0$. Veamos ahora que es cerrado para la suma y el producto por escalares.
        \begin{description}
            \item[Suma] Sean $x_1,x_2\in Z$. Tenemos que:
            \begin{align*}
                (x_1+x_2)'(t) + (x_1+x_2)(t) &= x_1'(t) + x_2'(t) + x_1(t) + x_2(t) =\\
                = \int_0^t a(s)x_1(s)\,ds + \int_0^t a(s)x_2(s)\,ds &= \int_0^t a(s)(x_1(s)+x_2(s))\,ds.
            \end{align*}
            Por tanto, $x_1+x_2\in Z$.

            \item[Producto por escalares] Sean $x\in Z$ y $\lambda \in \bb{R}$. Tenemos que:
            \begin{align*}
                (\lambda x)'(t) + (\lambda x)(t) &= \lambda x'(t) + \lambda x(t) = \lambda(x'(t) + x(t)) =\\
                = \lambda \int_0^t a(s)x(s)\,ds &= \int_0^t a(s)(\lambda x(s))\,ds.
            \end{align*}
            Por tanto, $\lambda x\in Z$.
        \end{description}

        Por tanto, $Z$ es un subespacio vectorial de $\bb{F}(\bb{R},\bb{R})$, por lo que $Z$ tiene estructura de espacio vectorial. Estudiar su dimensión es algo más complejo. Como no se han estudiado las ecuaciones integro-diferenciales en la asignatura, derivamos la ecuación para obtener una ecuación diferencial. Usando el Teorema Fundamental del Cálculo, tenemos que:
        \begin{equation}\label{eq:ed2}
            x''(t) + x'(t)= a(t)x(t)
        \end{equation}

        Notemos que, si $x$ es una solución de la Ecuación~\eqref{eq:ed1}, entonces $x$ es solución de la Ecuación~\eqref{eq:ed2}, pero el recíproco no lo tenemos asegurado. Sea $x$ una solución de la Ecuación~\eqref{eq:ed2}. Integrando la Ecuación~\eqref{eq:ed2} en el intervalo $[0,t]$, tenemos que:
        \begin{equation*}
            \int_{0}^t x''(s) + x'(s)\,ds = \int_{0}^t a(s)x(s)\,ds \Longrightarrow x'(t) - x'(0)  + x(t) - x(0) = \int_{0}^t a(s)x(s)\,ds.
        \end{equation*}

        Por tanto, dada $x(t)$ una solución de la Ecuación~\eqref{eq:ed2}, tenemos que:
        \begin{equation*}
            x(t) \text{ es solución de la Ecuación~\eqref{eq:ed1}} \Longleftrightarrow x'(0) = -x(0).
        \end{equation*}

        Como anteriormente hemos visto que, si $x$ es solución de la Ecuación~\eqref{eq:ed1}, entonces $x$ es solución de la Ecuación~\eqref{eq:ed2}, tenemos que, dada $x:\bb{R}\to \bb{R}$:
        \begin{equation*}
            x(t) \text{ es solución de la Ecuación~\eqref{eq:ed1}} \Longleftrightarrow x'(0) = -x(0)
        \end{equation*}

        Estamos ahora para probar que $\dim Z=1$. para ello, consideramos la siguiente función:
        \Func{\Phi}{Z}{\cc{L}\{(1,-1)\}}{x}{(x(0),x'(0))}

        En primer lugar, hemos de ver que está bien definida. Veamos que, dado $x\in Z$, tenemos que $\Phi(x)\in \cc{L}\{(1,-1)\}$. Tenemos que:
        \begin{equation*}
            (x(0),x'(0)) = (x(0),-x(0)) = x(0)(1,-1) \in \cc{L}\{(1,-1)\}.
        \end{equation*}
        Por tanto, $\Phi$ está bien definida. Veamos que es biyectiva.
        \begin{description}
            \item[Inyectividad] Sean $x_1,x_2\in Z$ tales que $\Phi(x_1)=\Phi(x_2)$. Entonces, sabemos que $x_1,x_2$ son solución de la Ecuación~\eqref{eq:ed2}, y tenemos que:
            \begin{equation*}
                \Phi(x_1) = \Phi(x_2) \Longrightarrow (x_1(0),x_1'(0)) = (x_2(0),x_2'(0)) \Longrightarrow x_1(0) = x_2(0), x_1'(0) = x_2'(0).
            \end{equation*}

            por tanto, $x_1,x_2$ son dos soluciones con las mismas condiciones iniciales de la Ecuación lineal~\eqref{eq:ed2}. Por el Teorema de Unicidad del Capítulo 4, tenemos que $x_1=x_2$, luego $\Phi$ es inyectiva.

            \item[Sobreyectividad] Sea $v\in \cc{L}\{(1,-1)\}$, y veamos que $\exists x\in Z$ tal que $\Phi(x)=v$. Como $v\in \cc{L}\{(1,-1)\}$, tenemos que $\exists \alpha \in \bb{R}$ tal que $v=\alpha(1,-1)$. Por el Teorema de Existencia del Capítulo 4, tenemos que existe una solución del sistema de ecuaciones diferenciales lineales~\eqref{eq:ed2} con condiciones iniciales $(x(0),x'(0)) = \alpha(1,-1)$. Como $x(0)=-x'(0)$, tenemos que $x\in Z$ y $\Phi(x)=v$. Por tanto, $\Phi$ es sobreyectiva.
        \end{description}

        Por tanto, $\Phi$ es biyectiva, por lo que:
        \begin{equation*}
            \dim Z = \dim \cc{L}\{(1,-1)\} = 1.
        \end{equation*}
    \end{ejercicio}
\end{document}
