\documentclass[12pt]{article}

% Idioma y codificación
\usepackage[spanish, es-tabla]{babel}       %es-tabla para que se titule "Tabla"
\usepackage[utf8]{inputenc}

% Márgenes
\usepackage[a4paper,top=3cm,bottom=2.5cm,left=3cm,right=3cm]{geometry}

% Comentarios de bloque
\usepackage{verbatim}

% Paquetes de links
\usepackage[hidelinks]{hyperref}    % Permite enlaces
\usepackage{url}                    % redirecciona a la web

% Más opciones para enumeraciones
\usepackage{enumitem}

% Personalizar la portada
\usepackage{titling}

% Paquetes de tablas
\usepackage{multirow}


%------------------------------------------------------------------------

%Paquetes de figuras
\usepackage{caption}
\usepackage{subcaption} % Figuras al lado de otras
\usepackage{float}      % Poner figuras en el sitio indicado H.


% Paquetes de imágenes
\usepackage{graphicx}       % Paquete para añadir imágenes
\usepackage{transparent}    % Para manejar la opacidad de las figuras

% Paquete para usar colores
\usepackage[dvipsnames]{xcolor}
\usepackage{pagecolor}      % Para cambiar el color de la página

% Habilita tamaños de fuente mayores
\usepackage{fix-cm}

% Para los gráficos
\usepackage{tikz}

% Para poder situar los nodos en los grafos
\usetikzlibrary{positioning}


%------------------------------------------------------------------------

% Paquetes de matemáticas
\usepackage{mathtools, amsfonts, amssymb, mathrsfs}
\usepackage[makeroom]{cancel}     % Simplificar tachando
\usepackage{polynom}    % Divisiones y Ruffini
\usepackage{units} % Para poner fracciones diagonales con \nicefrac

\usepackage{pgfplots}   %Representar funciones
\pgfplotsset{compat=1.18}  % Versión 1.18

\usepackage{tikz-cd}    % Para usar diagramas de composiciones
\usetikzlibrary{calc}   % Para usar cálculo de coordenadas en tikz

%Definición de teoremas, etc.
\usepackage{amsthm}
%\swapnumbers   % Intercambia la posición del texto y de la numeración

\theoremstyle{plain}

\makeatletter
\@ifclassloaded{article}{
  \newtheorem{teo}{Teorema}[section]
}{
  \newtheorem{teo}{Teorema}[chapter]  % Se resetea en cada chapter
}
\makeatother

\newtheorem{coro}{Corolario}[teo]           % Se resetea en cada teorema
\newtheorem{prop}[teo]{Proposición}         % Usa el mismo contador que teorema
\newtheorem{lema}[teo]{Lema}                % Usa el mismo contador que teorema

\theoremstyle{remark}
\newtheorem*{observacion}{Observación}

\theoremstyle{definition}

\makeatletter
\@ifclassloaded{article}{
  \newtheorem{definicion}{Definición} [section]     % Se resetea en cada chapter
}{
  \newtheorem{definicion}{Definición} [chapter]     % Se resetea en cada chapter
}
\makeatother

\newtheorem*{notacion}{Notación}
\newtheorem*{ejemplo}{Ejemplo}
\newtheorem*{ejercicio*}{Ejercicio}             % No numerado
\newtheorem{ejercicio}{Ejercicio} [section]     % Se resetea en cada section


% Modificar el formato de la numeración del teorema "ejercicio"
\renewcommand{\theejercicio}{%
  \ifnum\value{section}=0 % Si no se ha iniciado ninguna sección
    \arabic{ejercicio}% Solo mostrar el número de ejercicio
  \else
    \thesection.\arabic{ejercicio}% Mostrar número de sección y número de ejercicio
  \fi
}


% \renewcommand\qedsymbol{$\blacksquare$}         % Cambiar símbolo QED
%------------------------------------------------------------------------

% Paquetes para encabezados
\usepackage{fancyhdr}
\pagestyle{fancy}
\fancyhf{}

\newcommand{\helv}{ % Modificación tamaño de letra
\fontfamily{}\fontsize{12}{12}\selectfont}
\setlength{\headheight}{15pt} % Amplía el tamaño del índice


%\usepackage{lastpage}   % Referenciar última pag   \pageref{LastPage}
\fancyfoot[C]{\thepage}

%------------------------------------------------------------------------

% Conseguir que no ponga "Capítulo 1". Sino solo "1."
\makeatletter
\@ifclassloaded{book}{
  \renewcommand{\chaptermark}[1]{\markboth{\thechapter.\ #1}{}} % En el encabezado
    
  \renewcommand{\@makechapterhead}[1]{%
  \vspace*{50\p@}%
  {\parindent \z@ \raggedright \normalfont
    \ifnum \c@secnumdepth >\m@ne
      \huge\bfseries \thechapter.\hspace{1em}\ignorespaces
    \fi
    \interlinepenalty\@M
    \Huge \bfseries #1\par\nobreak
    \vskip 40\p@
  }}
}
\makeatother

%------------------------------------------------------------------------
% Paquetes de cógido
\usepackage{minted}
\renewcommand\listingscaption{Código fuente}

\usepackage{fancyvrb}
% Personaliza el tamaño de los números de línea
\renewcommand{\theFancyVerbLine}{\small\arabic{FancyVerbLine}}

% Estilo para C++
\newminted{cpp}{
    frame=lines,
    framesep=2mm,
    baselinestretch=1.2,
    linenos,
    escapeinside=||
}

% para minted
\definecolor{LightGray}{rgb}{0.95,0.95,0.92}
\setminted{
    linenos=true,
    stepnumber=5,
    numberfirstline=true,
    autogobble,
    breaklines=true,
    breakautoindent=true,
    breaksymbolleft=,
    breaksymbolright=,
    breaksymbolindentleft=0pt,
    breaksymbolindentright=0pt,
    breaksymbolsepleft=0pt,
    breaksymbolsepright=0pt,
    fontsize=\footnotesize,
    bgcolor=LightGray,
    numbersep=10pt
}


\usepackage{listings} % Para incluir código desde un archivo

\renewcommand\lstlistingname{Código Fuente}
\renewcommand\lstlistlistingname{Índice de Códigos Fuente}

% Definir colores
\definecolor{vscodepurple}{rgb}{0.5,0,0.5}
\definecolor{vscodeblue}{rgb}{0,0,0.8}
\definecolor{vscodegreen}{rgb}{0,0.5,0}
\definecolor{vscodegray}{rgb}{0.5,0.5,0.5}
\definecolor{vscodebackground}{rgb}{0.97,0.97,0.97}
\definecolor{vscodelightgray}{rgb}{0.9,0.9,0.9}

% Configuración para el estilo de C similar a VSCode
\lstdefinestyle{vscode_C}{
  backgroundcolor=\color{vscodebackground},
  commentstyle=\color{vscodegreen},
  keywordstyle=\color{vscodeblue},
  numberstyle=\tiny\color{vscodegray},
  stringstyle=\color{vscodepurple},
  basicstyle=\scriptsize\ttfamily,
  breakatwhitespace=false,
  breaklines=true,
  captionpos=b,
  keepspaces=true,
  numbers=left,
  numbersep=5pt,
  showspaces=false,
  showstringspaces=false,
  showtabs=false,
  tabsize=2,
  frame=tb,
  framerule=0pt,
  aboveskip=10pt,
  belowskip=10pt,
  xleftmargin=10pt,
  xrightmargin=10pt,
  framexleftmargin=10pt,
  framexrightmargin=10pt,
  framesep=0pt,
  rulecolor=\color{vscodelightgray},
  backgroundcolor=\color{vscodebackground},
}

%------------------------------------------------------------------------

% Comandos definidos
\newcommand{\bb}[1]{\mathbb{#1}}
\newcommand{\cc}[1]{\mathcal{#1}}

% I prefer the slanted \leq
\let\oldleq\leq % save them in case they're every wanted
\let\oldgeq\geq
\renewcommand{\leq}{\leqslant}
\renewcommand{\geq}{\geqslant}

% Si y solo si
\newcommand{\sii}{\iff}

% Letras griegas
\newcommand{\eps}{\epsilon}
\newcommand{\veps}{\varepsilon}
\newcommand{\lm}{\lambda}

\newcommand{\ol}{\overline}
\newcommand{\ul}{\underline}
\newcommand{\wt}{\widetilde}
\newcommand{\wh}{\widehat}

\let\oldvec\vec
\renewcommand{\vec}{\overrightarrow}

% Derivadas parciales
\newcommand{\del}[2]{\frac{\partial #1}{\partial #2}}
\newcommand{\Del}[3]{\frac{\partial^{#1} #2}{\partial #3^{#1}}}
\newcommand{\deld}[2]{\dfrac{\partial #1}{\partial #2}}
\newcommand{\Deld}[3]{\dfrac{\partial^{#1} #2}{\partial #3^{#1}}}


\newcommand{\AstIg}{\stackrel{(\ast)}{=}}
\newcommand{\Hop}{\stackrel{L'H\hat{o}pital}{=}}

\newcommand{\red}[1]{{\color{red}#1}} % Para integrales, destacar los cambios.

% Método de integración
\newcommand{\MetInt}[2]{
    \left[\begin{array}{c}
        #1 \\ #2
    \end{array}\right]
}

% Declarar aplicaciones
% 1. Nombre aplicación
% 2. Dominio
% 3. Codominio
% 4. Variable
% 5. Imagen de la variable
\newcommand{\Func}[5]{
    \begin{equation*}
        \begin{array}{rrll}
            #1:& #2 & \longrightarrow & #3\\
               & #4 & \longmapsto & #5
        \end{array}
    \end{equation*}
}

%------------------------------------------------------------------------



\begin{document}

    % 1. Foto de fondo
    % 2. Título
    % 3. Encabezado Izquierdo
    % 4. Color de fondo
    % 5. Coord x del titulo
    % 6. Coord y del titulo
    % 7. Fecha

    
    % 1. Foto de fondo
% 2. Título
% 3. Encabezado Izquierdo
% 4. Color de fondo
% 5. Coord x del titulo
% 6. Coord y del titulo
% 7. Fecha

\newcommand{\portada}[7]{

    \portadaBase{#1}{#2}{#3}{#4}{#5}{#6}{#7}
    \portadaBook{#1}{#2}{#3}{#4}{#5}{#6}{#7}
}

\newcommand{\portadaExamen}[7]{

    \portadaBase{#1}{#2}{#3}{#4}{#5}{#6}{#7}
    \portadaArticle{#1}{#2}{#3}{#4}{#5}{#6}{#7}
}




\newcommand{\portadaBase}[7]{

    % Tiene la portada principal y la licencia Creative Commons
    
    % 1. Foto de fondo
    % 2. Título
    % 3. Encabezado Izquierdo
    % 4. Color de fondo
    % 5. Coord x del titulo
    % 6. Coord y del titulo
    % 7. Fecha
    
    
    \thispagestyle{empty}               % Sin encabezado ni pie de página
    \newgeometry{margin=0cm}        % Márgenes nulos para la primera página
    
    
    % Encabezado
    \fancyhead[L]{\helv #3}
    \fancyhead[R]{\helv \nouppercase{\leftmark}}
    
    
    \pagecolor{#4}        % Color de fondo para la portada
    
    \begin{figure}[p]
        \centering
        \transparent{0.3}           % Opacidad del 30% para la imagen
        
        \includegraphics[width=\paperwidth, keepaspectratio]{assets/#1}
    
        \begin{tikzpicture}[remember picture, overlay]
            \node[anchor=north west, text=white, opacity=1, font=\fontsize{60}{90}\selectfont\bfseries\sffamily, align=left] at (#5, #6) {#2};
            
            \node[anchor=south east, text=white, opacity=1, font=\fontsize{12}{18}\selectfont\sffamily, align=right] at (9.7, 3) {\textbf{\href{https://losdeldgiim.github.io/}{Los Del DGIIM}}};
            
            \node[anchor=south east, text=white, opacity=1, font=\fontsize{12}{15}\selectfont\sffamily, align=right] at (9.7, 1.8) {Doble Grado en Ingeniería Informática y Matemáticas\\Universidad de Granada};
        \end{tikzpicture}
    \end{figure}
    
    
    \restoregeometry        % Restaurar márgenes normales para las páginas subsiguientes
    \pagecolor{white}       % Restaurar el color de página
    
    
    \newpage
    \thispagestyle{empty}               % Sin encabezado ni pie de página
    \begin{tikzpicture}[remember picture, overlay]
        \node[anchor=south west, inner sep=3cm] at (current page.south west) {
            \begin{minipage}{0.5\paperwidth}
                \href{https://creativecommons.org/licenses/by-nc-nd/4.0/}{
                    \includegraphics[height=2cm]{assets/Licencia.png}
                }\vspace{1cm}\\
                Esta obra está bajo una
                \href{https://creativecommons.org/licenses/by-nc-nd/4.0/}{
                    Licencia Creative Commons Atribución-NoComercial-SinDerivadas 4.0 Internacional (CC BY-NC-ND 4.0).
                }\\
    
                Eres libre de compartir y redistribuir el contenido de esta obra en cualquier medio o formato, siempre y cuando des el crédito adecuado a los autores originales y no persigas fines comerciales. 
            \end{minipage}
        };
    \end{tikzpicture}
    
    
    
    % 1. Foto de fondo
    % 2. Título
    % 3. Encabezado Izquierdo
    % 4. Color de fondo
    % 5. Coord x del titulo
    % 6. Coord y del titulo
    % 7. Fecha


}


\newcommand{\portadaBook}[7]{

    % 1. Foto de fondo
    % 2. Título
    % 3. Encabezado Izquierdo
    % 4. Color de fondo
    % 5. Coord x del titulo
    % 6. Coord y del titulo
    % 7. Fecha

    % Personaliza el formato del título
    \pretitle{\begin{center}\bfseries\fontsize{42}{56}\selectfont}
    \posttitle{\par\end{center}\vspace{2em}}
    
    % Personaliza el formato del autor
    \preauthor{\begin{center}\Large}
    \postauthor{\par\end{center}\vfill}
    
    % Personaliza el formato de la fecha
    \predate{\begin{center}\huge}
    \postdate{\par\end{center}\vspace{2em}}
    
    \title{#2}
    \author{\href{https://losdeldgiim.github.io/}{Los Del DGIIM}}
    \date{Granada, #7}
    \maketitle
    
    \tableofcontents
}




\newcommand{\portadaArticle}[7]{

    % 1. Foto de fondo
    % 2. Título
    % 3. Encabezado Izquierdo
    % 4. Color de fondo
    % 5. Coord x del titulo
    % 6. Coord y del titulo
    % 7. Fecha

    % Personaliza el formato del título
    \pretitle{\begin{center}\bfseries\fontsize{42}{56}\selectfont}
    \posttitle{\par\end{center}\vspace{2em}}
    
    % Personaliza el formato del autor
    \preauthor{\begin{center}\Large}
    \postauthor{\par\end{center}\vspace{3em}}
    
    % Personaliza el formato de la fecha
    \predate{\begin{center}\huge}
    \postdate{\par\end{center}\vspace{5em}}
    
    \title{#2}
    \author{\href{https://losdeldgiim.github.io/}{Los Del DGIIM}}
    \date{Granada, #7}
    \thispagestyle{empty}               % Sin encabezado ni pie de página
    \maketitle
    \vfill
}
    \portadaExamen{ffccA4.jpg}{Ecuaciones\\Diferenciales I\\Examen VII}{Ecuaciones Diferenciales I. Examen VII}{MidnightBlue}{-8}{28}{2024-2025}{Arturo Olivares Martos}

    \begin{description}
        \item[Asignatura] Ecuaciones Diferenciales I.
        \item[Curso Académico] 2018-2019.
        %\item[Grado] Doble Grado en Ingeniería Informática y Matemáticas.
        %\item[Grupo] Único.
        \item[Profesor] Rafael Ortega Ríos.
        \item[Descripción] Convocatoria Ordinaria
        \item[Fecha] 18 de junio de 2019.
        %\item[Duración] 60 minutos.
    
    \end{description}
    \newpage

    \section{Primera Parte}
\begin{ejercicio}[6 puntos]
    Se considera una ecuación diferencial del tipo
    \begin{equation}\label{eq:1}
        x' = a(t)x^5 + b(t)x,
    \end{equation}
    donde $a, b : I \to \bb{R}$ son funciones continuas definidas en un intervalo abierto $I$.
    \begin{enumerate}
        \item (2 puntos) Dado $\alpha \in \bb{R} \setminus \{0\}$, se definen las nuevas variables
        \[
            s = t, \quad y = x^\alpha.
        \]
        Encuentra dominios del plano $D$ y $D'$ de manera que la transformación dada por $(t, x) \in D \mapsto (s, y) \in D'$ sea un cambio admisible para la ecuación \eqref{eq:1}.\\

        El cambio de variable viene definido por:
        \Func{\varphi=(\varphi_1,\varphi_2)}{D}{D'}{(t,x)}{(s,y)=(t,x^\alpha)}

        Su inversa no obstante depende de los valores de $\alpha$, al igual que los dominios. Distinguimos en los siguientes casos:
        \begin{itemize}
            \item Si $\alpha\in \bb{R}^+$: Hay que volver a distinguir casos:
            \begin{itemize}
                \item Si $\alpha\in \bb{R}^+\cap \bb{Z}$: Hay que distinguir entre si es par o impar.
                \begin{itemize}
                    \item Si $\alpha\in \bb{R}^+\cap \bb{Z}$ y es par:
                    
                    Tenemos que la función $y=x^\alpha$ está definida en todo $\bb{R}$, pero tan solo es inyectiva en $\bb{R}^+$ o en $\bb{R}^-$. Consideramos por tanto:
                    \begin{itemize}
                        \item $x\in \bb{R}^+$:
                        
                        En este caso, tenemos que la inversa, por poder despejar de forma única $t,x$ en función de $s,y$, es:
                        \Func{\varphi^{-1}}{D'}{D}{(s,y)}{(t,x)=(s,y^{1/\alpha})=(s,\sqrt[\alpha]{y})}

                        Los dominios son:
                        \begin{align*}
                            D=I\times \bb{R}^+,\quad D'=I\times \bb{R}^+.
                        \end{align*}

                        \item $x\in \bb{R}^-$:
                        
                        En este caso, tenemos que la inversa, por poder despejar de forma única $t,x$ en función de $s,y$, es:
                        \Func{\varphi^{-1}}{D'}{D}{(s,y)}{(t,x)=(s,-y^{1/\alpha})=(s,-\sqrt[\alpha]{y})}

                        Los dominios son:
                        \begin{align*}
                            D=I \times \bb{R}^-,\quad D'=I \times \bb{R}^+.
                        \end{align*}
                    \end{itemize}

                    \item Si $\alpha\in \bb{R}^+\cap \bb{Z}$ y es impar:
                    
                    Tenemos que la función $y=x^\alpha$ está definida en todo $\bb{R}$ y es biyectiva, por lo que podemos despejar de forma única $t,x$ en función de $s,y$.
                    Por tanto, la inversa es:
                    \Func{\varphi^{-1}}{D'}{D}{(s,y)}{(t,x)=(s,y^{1/\alpha})=(s,\sqrt[\alpha]{y})}

                    Los dominios son:
                    \begin{align*}
                        D=D'=\bb{R}^2.
                    \end{align*}
                \end{itemize}

                \item Si $\alpha\in \bb{R}^+\setminus \bb{Z}$:
                
                En este caso, tenemos que la función $y=x^\alpha$ tan solo está definida en $\bb{R}^+$, por lo que la inversa es:
                \Func{\varphi^{-1}}{D'}{D}{(s,y)}{(t,x)=(s,y^{1/\alpha})}

                Los dominios son:
                \begin{align*}
                    D=D'=I \times \bb{R}^+.
                \end{align*}
            \end{itemize}

            \item Si $\alpha\in \bb{R}^-$:
            
            Volvemos a distinguir casos:
            \begin{itemize}
                \item Si $\alpha\in \bb{R}^-\cap \bb{Z}$: Hay que distinguir entre si es par o impar.
                \begin{itemize}
                    \item Si $\alpha\in \bb{R}^-\cap \bb{Z}$ y es par:
                    
                    Tenemos que la función $y=x^\alpha$ está definida en $\bb{R}^\ast$, y el dominio ha de ser un intervalo. Distinguimos:
                    \begin{itemize}
                        \item $x\in \bb{R}^+$:
                        
                        En este caso, tenemos que la inversa, por poder despejar de forma única $t,x$ en función de $s,y$, es:
                        \Func{\varphi^{-1}}{D'}{D}{(s,y)}{(t,x)=(s,y^{1/\alpha})=\left(s,\dfrac{1}{\sqrt[-\alpha]{y}}\right)}

                        Los dominios son:
                        \begin{align*}
                            D=I \times \bb{R}^+,\quad D'=I \times \bb{R}^+.
                        \end{align*}

                        \item $x\in \bb{R}^-$:
                        
                        En este caso, tenemos que la inversa, por poder despejar de forma única $t,x$ en función de $s,y$, es:
                        \Func{\varphi^{-1}}{D'}{D}{(s,y)}{(t,x)=(s,-y^{1/\alpha})=\left(s,-\dfrac{1}{\sqrt[-\alpha]{y}}\right)}

                        Los dominios son:
                        \begin{align*}
                            D=I \times \bb{R}^-,\quad D'=I \times \bb{R}^+.
                        \end{align*}
                    \end{itemize}

                    \item Si $\alpha\in \bb{R}^-\cap \bb{Z}$ y es impar:
                    
                    Tenemos que la función $y=x^\alpha$ está definida en $\bb{R}^\ast$ y es biyectiva.
                    Por tanto, la inversa es:
                    \Func{\varphi^{-1}}{D'}{D}{(s,y)}{(t,x)=(s,y^{1/\alpha})=\left(s,\dfrac{1}{\sqrt[-\alpha]{y}}\right)}

                    Hay dos posibles dominios, ya que ha de ser conexo:
                    \begin{align*}
                        D_1=D_1'=I \times \bb{R}^+,\\
                        D_2=D_2'=I \times \bb{R}^-.
                    \end{align*}
                \end{itemize}

                \item Si $\alpha\in \bb{R}^-\setminus \bb{Z}$:
                
                En este caso, tenemos que la función $y=x^\alpha$ está definida en $\bb{R}^+$, por lo que la inversa es:
                \Func{\varphi^{-1}}{D'}{D}{(s,y)}{(t,x)=(s,y^{1/\alpha})=\left(s,\dfrac{1}{\sqrt[-\alpha]{y}}\right)}

                Los dominios son:
                \begin{align*}
                    D=D'=I \times \bb{R}^+.
                \end{align*}
            \end{itemize}
        \end{itemize}

        En cualquier caso, tenemos que es admisible puesto que no modifica la variable independiente.

        \item (2 puntos) Transforma la ecuación \eqref{eq:1} mediante el cambio del apartado anterior. ¿Para qué valores de $\alpha$ se obtiene una ecuación lineal?
        
        La ecuación transformada es:
        \begin{align*}
            \dfrac{dy}{ds}&=\dfrac{dy}{dx} = \alpha x^{\alpha-1}x' = \alpha x^{\alpha-1}\left(a(t)x^5+b(t)x\right) = \alpha x^{4+\alpha}a(t)+\alpha x^{\alpha}b(t)=\\
            &=\alpha yx^4a(t)+\alpha yb(t).
        \end{align*}

        Para despejar $x^4$ en función de $y$, tenemos que $x=\pm y^{1/\alpha}$, dependiendo del caso. No obstante, como es $x^4$, no importa el signo, por lo que:
        \begin{equation*}
            y'=y\cdot y^{\nicefrac{4}{\alpha}}a(t)+yb(t)=y^{\nicefrac{4+\alpha}{\alpha}}a(t)+yb(t)\qquad \text{con dominio }D'.
        \end{equation*}

        Para que sea lineal, ha de ser $4+\alpha=0$, es decir, $\alpha=-4$. Por tanto, la ecuación lineal es:
        \begin{equation*}
            y'=a+by \qquad \text{con dominio }D'=I\times \bb{R}^+.
        \end{equation*}
        
        \item (2 puntos) Se supone $a(t) = \nicefrac{-1}{4}t$, $b(t) = \nicefrac{1}{4}$, $I = \bb{R}$. Encuentra la solución de \eqref{eq:1} que cumple $x(2) = 1$. ¿En qué intervalo está definida?
        
        La ecuación a resolver es:
        \begin{equation*}
            x'=-\dfrac{1}{4}tx^5+\dfrac{1}{4}x\qquad \text{con dominio }D=\bb{R}\times \bb{R}^+.
        \end{equation*}

        Tomando el cambio de variable anterior para $\alpha=-4$, llegamos a:
        \begin{equation*}
            y'=-\dfrac{1}{4}s+\dfrac{1}{4}y, \qquad \text{con dominio }D'=\bb{R}\times \bb{R}^+.
        \end{equation*}

        La solución de esta ecuación es:
        \begin{equation*}
            y(s)=e^{\nicefrac{s}{4}}\left(\int e^{\nicefrac{-z}{4}}\cdot \frac{-z}{4}dz+C\right)
        \end{equation*}

        Para resolver dicha ecuación aplicamos el Método de Integración por Partes:
        \begin{multline*}
            \MetInt{u(z)=z,\qquad u'(z)=1}{v'(z)=\nicefrac{-1}{4}e^{\nicefrac{-z}{4}},~v(z)=e^{\nicefrac{-z}{4}}}
            \Longrightarrow
            \int e^{\nicefrac{-z}{4}}\cdot \frac{-z}{4}dz=ze^{\nicefrac{-z}{4}}-\int e^{\nicefrac{-z}{4}}dz=\\=ze^{\nicefrac{-z}{4}}+4e^{\nicefrac{-z}{4}}+C
        \end{multline*}

        Por tanto, la solución de la ecuación diferencial es:
        \begin{equation*}
            y(s)=e^{\nicefrac{s}{4}}\left(se^{\nicefrac{-s}{4}}+4e^{\nicefrac{-s}{4}}+C\right)=s+4+e^{\nicefrac{s}{4}}C.
        \end{equation*}

        Deshaciendo el cambio de variable, llegamos a:
        \begin{equation*}
            x(t)=\dfrac{1}{\sqrt[4]{t+4+e^{\nicefrac{t}{4}}C}}
        \end{equation*}

        Aplicando la condición inicial de $x(2)=1$, llegamos a:
        \begin{equation*}
            x(2)=1=\dfrac{1}{\sqrt[4]{2+4+e^{\nicefrac{1}{2}}C}}\Longrightarrow C=-\dfrac{5}{e^{\nicefrac{1}{2}}}
        \end{equation*}

        Por tanto, la solución de la ecuación diferencial es:
        \begin{equation*}
            x(t)=\dfrac{1}{\sqrt[4]{t+4-5e^{\nicefrac{t-2}{4}}}}
        \end{equation*}

        Para que sea una solución válida en el dominio $\bb{R}\times \bb{R}^+$, ha de tenerse que el argumento de la raíz sea positivo, por lo que:
        \begin{equation*}
            t+4-5e^{\nicefrac{t-2}{4}}>0
        \end{equation*}
    \end{enumerate}
\end{ejercicio}

\begin{ejercicio}[4 puntos]
    Se considera el sistema autónomo definido en $\bb{R}^2 \setminus \{(0, 0)\}$,
    \begin{equation*}
        \begin{cases}
            x' = \dfrac{2y}{x^2 + y^2},\\
            y' = \dfrac{-3x}{x^2 + y^2}.
        \end{cases}
    \end{equation*}
    \begin{enumerate}
        \item (2 puntos) Determina la ecuación de las órbitas y encuentra todas sus soluciones.
        \item (2 puntos) Dibuja las órbitas del sistema.
    \end{enumerate}
\end{ejercicio}

\section{Segunda Parte}
\begin{ejercicio}[5 puntos]
    Se considera la ecuación diferencial
    \begin{equation*}
        e^y + \cos x + (xe^y + 2y)y' = 0.
    \end{equation*}
    \begin{enumerate}
        \item (2.5 puntos) Encuentra la ecuación que define en forma implícita a la solución que cumple la condición inicial $y(0) = 1$. (Se debe justificar que dicha ecuación define una función en un entorno de $x = 0$).
        \item (2.5 puntos) ¿Existe una solución de la ecuación diferencial que cumpla $y(0) = 0$?
    \end{enumerate}
\end{ejercicio}

\begin{ejercicio}[5 puntos]
    Calcula en cada caso el trabajo para la fuerza $F$ y el camino $\gamma$.
    \begin{enumerate}
        \item (2.5 puntos) $F(x, y) = (-y, x)$, $\gamma(t) = (\cos t, \sen t)$, $t \in [0, 2\pi]$.
        \item (2.5 puntos) $F(x, y) = (4x^3 + y, 2y + x)$, $\gamma(t) = (\ln t, t)$, $t \in [1, e]$.
    \end{enumerate}
\end{ejercicio}

\section{Tercera Parte}
\begin{ejercicio}[6 puntos]
    Se considera la ecuación
    \begin{equation*}
        t^2x'' + tx' + x = \sen(\ln t), \quad t > 0,
    \end{equation*}
    y se pide:
    \begin{enumerate}
        \item (2 puntos) Encuentra una constante $\alpha > 0$ de manera que las funciones
        \[
            \phi_1(t) = \sen(\alpha \ln t), \quad \phi_2(t) = \cos(\alpha \ln t), \quad t > 0
        \]
        formen un sistema fundamental de la ecuación homogénea asociada.
        \item (2 puntos) Usando la fórmula de variación de constantes, encuentra una solución particular de la ecuación completa.
        \item (2 puntos) Encuentra la solución de la ecuación completa que cumple $x(1) = 0$, $x'(1) = 0$.
    \end{enumerate}
\end{ejercicio}

\begin{ejercicio}[4 puntos]
    Encuentra la solución general de la ecuación
    \begin{equation*}
        y''' + 2y'' + y' = 5e^t + \cos t.
    \end{equation*}
\end{ejercicio}


\end{document}