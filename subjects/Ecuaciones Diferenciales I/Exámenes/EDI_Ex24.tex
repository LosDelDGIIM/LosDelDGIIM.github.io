\documentclass[12pt]{article}

% Idioma y codificación
\usepackage[spanish, es-tabla]{babel}       %es-tabla para que se titule "Tabla"
\usepackage[utf8]{inputenc}

% Márgenes
\usepackage[a4paper,top=3cm,bottom=2.5cm,left=3cm,right=3cm]{geometry}

% Comentarios de bloque
\usepackage{verbatim}

% Paquetes de links
\usepackage[hidelinks]{hyperref}    % Permite enlaces
\usepackage{url}                    % redirecciona a la web

% Más opciones para enumeraciones
\usepackage{enumitem}

% Personalizar la portada
\usepackage{titling}

% Paquetes de tablas
\usepackage{multirow}


%------------------------------------------------------------------------

%Paquetes de figuras
\usepackage{caption}
\usepackage{subcaption} % Figuras al lado de otras
\usepackage{float}      % Poner figuras en el sitio indicado H.


% Paquetes de imágenes
\usepackage{graphicx}       % Paquete para añadir imágenes
\usepackage{transparent}    % Para manejar la opacidad de las figuras

% Paquete para usar colores
\usepackage[dvipsnames]{xcolor}
\usepackage{pagecolor}      % Para cambiar el color de la página

% Habilita tamaños de fuente mayores
\usepackage{fix-cm}

% Para los gráficos
\usepackage{tikz}

% Para poder situar los nodos en los grafos
\usetikzlibrary{positioning}


%------------------------------------------------------------------------

% Paquetes de matemáticas
\usepackage{mathtools, amsfonts, amssymb, mathrsfs}
\usepackage[makeroom]{cancel}     % Simplificar tachando
\usepackage{polynom}    % Divisiones y Ruffini
\usepackage{units} % Para poner fracciones diagonales con \nicefrac

\usepackage{pgfplots}   %Representar funciones
\pgfplotsset{compat=1.18}  % Versión 1.18

\usepackage{tikz-cd}    % Para usar diagramas de composiciones
\usetikzlibrary{calc}   % Para usar cálculo de coordenadas en tikz

%Definición de teoremas, etc.
\usepackage{amsthm}
%\swapnumbers   % Intercambia la posición del texto y de la numeración

\theoremstyle{plain}

\makeatletter
\@ifclassloaded{article}{
  \newtheorem{teo}{Teorema}[section]
}{
  \newtheorem{teo}{Teorema}[chapter]  % Se resetea en cada chapter
}
\makeatother

\newtheorem{coro}{Corolario}[teo]           % Se resetea en cada teorema
\newtheorem{prop}[teo]{Proposición}         % Usa el mismo contador que teorema
\newtheorem{lema}[teo]{Lema}                % Usa el mismo contador que teorema

\theoremstyle{remark}
\newtheorem*{observacion}{Observación}

\theoremstyle{definition}

\makeatletter
\@ifclassloaded{article}{
  \newtheorem{definicion}{Definición} [section]     % Se resetea en cada chapter
}{
  \newtheorem{definicion}{Definición} [chapter]     % Se resetea en cada chapter
}
\makeatother

\newtheorem*{notacion}{Notación}
\newtheorem*{ejemplo}{Ejemplo}
\newtheorem*{ejercicio*}{Ejercicio}             % No numerado
\newtheorem{ejercicio}{Ejercicio} [section]     % Se resetea en cada section


% Modificar el formato de la numeración del teorema "ejercicio"
\renewcommand{\theejercicio}{%
  \ifnum\value{section}=0 % Si no se ha iniciado ninguna sección
    \arabic{ejercicio}% Solo mostrar el número de ejercicio
  \else
    \thesection.\arabic{ejercicio}% Mostrar número de sección y número de ejercicio
  \fi
}


% \renewcommand\qedsymbol{$\blacksquare$}         % Cambiar símbolo QED
%------------------------------------------------------------------------

% Paquetes para encabezados
\usepackage{fancyhdr}
\pagestyle{fancy}
\fancyhf{}

\newcommand{\helv}{ % Modificación tamaño de letra
\fontfamily{}\fontsize{12}{12}\selectfont}
\setlength{\headheight}{15pt} % Amplía el tamaño del índice


%\usepackage{lastpage}   % Referenciar última pag   \pageref{LastPage}
\fancyfoot[C]{\thepage}

%------------------------------------------------------------------------

% Conseguir que no ponga "Capítulo 1". Sino solo "1."
\makeatletter
\@ifclassloaded{book}{
  \renewcommand{\chaptermark}[1]{\markboth{\thechapter.\ #1}{}} % En el encabezado
    
  \renewcommand{\@makechapterhead}[1]{%
  \vspace*{50\p@}%
  {\parindent \z@ \raggedright \normalfont
    \ifnum \c@secnumdepth >\m@ne
      \huge\bfseries \thechapter.\hspace{1em}\ignorespaces
    \fi
    \interlinepenalty\@M
    \Huge \bfseries #1\par\nobreak
    \vskip 40\p@
  }}
}
\makeatother

%------------------------------------------------------------------------
% Paquetes de cógido
\usepackage{minted}
\renewcommand\listingscaption{Código fuente}

\usepackage{fancyvrb}
% Personaliza el tamaño de los números de línea
\renewcommand{\theFancyVerbLine}{\small\arabic{FancyVerbLine}}

% Estilo para C++
\newminted{cpp}{
    frame=lines,
    framesep=2mm,
    baselinestretch=1.2,
    linenos,
    escapeinside=||
}

% para minted
\definecolor{LightGray}{rgb}{0.95,0.95,0.92}
\setminted{
    linenos=true,
    stepnumber=5,
    numberfirstline=true,
    autogobble,
    breaklines=true,
    breakautoindent=true,
    breaksymbolleft=,
    breaksymbolright=,
    breaksymbolindentleft=0pt,
    breaksymbolindentright=0pt,
    breaksymbolsepleft=0pt,
    breaksymbolsepright=0pt,
    fontsize=\footnotesize,
    bgcolor=LightGray,
    numbersep=10pt
}


\usepackage{listings} % Para incluir código desde un archivo

\renewcommand\lstlistingname{Código Fuente}
\renewcommand\lstlistlistingname{Índice de Códigos Fuente}

% Definir colores
\definecolor{vscodepurple}{rgb}{0.5,0,0.5}
\definecolor{vscodeblue}{rgb}{0,0,0.8}
\definecolor{vscodegreen}{rgb}{0,0.5,0}
\definecolor{vscodegray}{rgb}{0.5,0.5,0.5}
\definecolor{vscodebackground}{rgb}{0.97,0.97,0.97}
\definecolor{vscodelightgray}{rgb}{0.9,0.9,0.9}

% Configuración para el estilo de C similar a VSCode
\lstdefinestyle{vscode_C}{
  backgroundcolor=\color{vscodebackground},
  commentstyle=\color{vscodegreen},
  keywordstyle=\color{vscodeblue},
  numberstyle=\tiny\color{vscodegray},
  stringstyle=\color{vscodepurple},
  basicstyle=\scriptsize\ttfamily,
  breakatwhitespace=false,
  breaklines=true,
  captionpos=b,
  keepspaces=true,
  numbers=left,
  numbersep=5pt,
  showspaces=false,
  showstringspaces=false,
  showtabs=false,
  tabsize=2,
  frame=tb,
  framerule=0pt,
  aboveskip=10pt,
  belowskip=10pt,
  xleftmargin=10pt,
  xrightmargin=10pt,
  framexleftmargin=10pt,
  framexrightmargin=10pt,
  framesep=0pt,
  rulecolor=\color{vscodelightgray},
  backgroundcolor=\color{vscodebackground},
}

%------------------------------------------------------------------------

% Comandos definidos
\newcommand{\bb}[1]{\mathbb{#1}}
\newcommand{\cc}[1]{\mathcal{#1}}

% I prefer the slanted \leq
\let\oldleq\leq % save them in case they're every wanted
\let\oldgeq\geq
\renewcommand{\leq}{\leqslant}
\renewcommand{\geq}{\geqslant}

% Si y solo si
\newcommand{\sii}{\iff}

% Letras griegas
\newcommand{\eps}{\epsilon}
\newcommand{\veps}{\varepsilon}
\newcommand{\lm}{\lambda}

\newcommand{\ol}{\overline}
\newcommand{\ul}{\underline}
\newcommand{\wt}{\widetilde}
\newcommand{\wh}{\widehat}

\let\oldvec\vec
\renewcommand{\vec}{\overrightarrow}

% Derivadas parciales
\newcommand{\del}[2]{\frac{\partial #1}{\partial #2}}
\newcommand{\Del}[3]{\frac{\partial^{#1} #2}{\partial #3^{#1}}}
\newcommand{\deld}[2]{\dfrac{\partial #1}{\partial #2}}
\newcommand{\Deld}[3]{\dfrac{\partial^{#1} #2}{\partial #3^{#1}}}


\newcommand{\AstIg}{\stackrel{(\ast)}{=}}
\newcommand{\Hop}{\stackrel{L'H\hat{o}pital}{=}}

\newcommand{\red}[1]{{\color{red}#1}} % Para integrales, destacar los cambios.

% Método de integración
\newcommand{\MetInt}[2]{
    \left[\begin{array}{c}
        #1 \\ #2
    \end{array}\right]
}

% Declarar aplicaciones
% 1. Nombre aplicación
% 2. Dominio
% 3. Codominio
% 4. Variable
% 5. Imagen de la variable
\newcommand{\Func}[5]{
    \begin{equation*}
        \begin{array}{rrll}
            #1:& #2 & \longrightarrow & #3\\
               & #4 & \longmapsto & #5
        \end{array}
    \end{equation*}
}

%------------------------------------------------------------------------



\begin{document}

    % 1. Foto de fondo
    % 2. Título
    % 3. Encabezado Izquierdo
    % 4. Color de fondo
    % 5. Coord x del titulo
    % 6. Coord y del titulo
    % 7. Fecha

    
    % 1. Foto de fondo
% 2. Título
% 3. Encabezado Izquierdo
% 4. Color de fondo
% 5. Coord x del titulo
% 6. Coord y del titulo
% 7. Fecha

\newcommand{\portada}[7]{

    \portadaBase{#1}{#2}{#3}{#4}{#5}{#6}{#7}
    \portadaBook{#1}{#2}{#3}{#4}{#5}{#6}{#7}
}

\newcommand{\portadaExamen}[7]{

    \portadaBase{#1}{#2}{#3}{#4}{#5}{#6}{#7}
    \portadaArticle{#1}{#2}{#3}{#4}{#5}{#6}{#7}
}




\newcommand{\portadaBase}[7]{

    % Tiene la portada principal y la licencia Creative Commons
    
    % 1. Foto de fondo
    % 2. Título
    % 3. Encabezado Izquierdo
    % 4. Color de fondo
    % 5. Coord x del titulo
    % 6. Coord y del titulo
    % 7. Fecha
    
    
    \thispagestyle{empty}               % Sin encabezado ni pie de página
    \newgeometry{margin=0cm}        % Márgenes nulos para la primera página
    
    
    % Encabezado
    \fancyhead[L]{\helv #3}
    \fancyhead[R]{\helv \nouppercase{\leftmark}}
    
    
    \pagecolor{#4}        % Color de fondo para la portada
    
    \begin{figure}[p]
        \centering
        \transparent{0.3}           % Opacidad del 30% para la imagen
        
        \includegraphics[width=\paperwidth, keepaspectratio]{assets/#1}
    
        \begin{tikzpicture}[remember picture, overlay]
            \node[anchor=north west, text=white, opacity=1, font=\fontsize{60}{90}\selectfont\bfseries\sffamily, align=left] at (#5, #6) {#2};
            
            \node[anchor=south east, text=white, opacity=1, font=\fontsize{12}{18}\selectfont\sffamily, align=right] at (9.7, 3) {\textbf{\href{https://losdeldgiim.github.io/}{Los Del DGIIM}}};
            
            \node[anchor=south east, text=white, opacity=1, font=\fontsize{12}{15}\selectfont\sffamily, align=right] at (9.7, 1.8) {Doble Grado en Ingeniería Informática y Matemáticas\\Universidad de Granada};
        \end{tikzpicture}
    \end{figure}
    
    
    \restoregeometry        % Restaurar márgenes normales para las páginas subsiguientes
    \pagecolor{white}       % Restaurar el color de página
    
    
    \newpage
    \thispagestyle{empty}               % Sin encabezado ni pie de página
    \begin{tikzpicture}[remember picture, overlay]
        \node[anchor=south west, inner sep=3cm] at (current page.south west) {
            \begin{minipage}{0.5\paperwidth}
                \href{https://creativecommons.org/licenses/by-nc-nd/4.0/}{
                    \includegraphics[height=2cm]{assets/Licencia.png}
                }\vspace{1cm}\\
                Esta obra está bajo una
                \href{https://creativecommons.org/licenses/by-nc-nd/4.0/}{
                    Licencia Creative Commons Atribución-NoComercial-SinDerivadas 4.0 Internacional (CC BY-NC-ND 4.0).
                }\\
    
                Eres libre de compartir y redistribuir el contenido de esta obra en cualquier medio o formato, siempre y cuando des el crédito adecuado a los autores originales y no persigas fines comerciales. 
            \end{minipage}
        };
    \end{tikzpicture}
    
    
    
    % 1. Foto de fondo
    % 2. Título
    % 3. Encabezado Izquierdo
    % 4. Color de fondo
    % 5. Coord x del titulo
    % 6. Coord y del titulo
    % 7. Fecha


}


\newcommand{\portadaBook}[7]{

    % 1. Foto de fondo
    % 2. Título
    % 3. Encabezado Izquierdo
    % 4. Color de fondo
    % 5. Coord x del titulo
    % 6. Coord y del titulo
    % 7. Fecha

    % Personaliza el formato del título
    \pretitle{\begin{center}\bfseries\fontsize{42}{56}\selectfont}
    \posttitle{\par\end{center}\vspace{2em}}
    
    % Personaliza el formato del autor
    \preauthor{\begin{center}\Large}
    \postauthor{\par\end{center}\vfill}
    
    % Personaliza el formato de la fecha
    \predate{\begin{center}\huge}
    \postdate{\par\end{center}\vspace{2em}}
    
    \title{#2}
    \author{\href{https://losdeldgiim.github.io/}{Los Del DGIIM}}
    \date{Granada, #7}
    \maketitle
    
    \tableofcontents
}




\newcommand{\portadaArticle}[7]{

    % 1. Foto de fondo
    % 2. Título
    % 3. Encabezado Izquierdo
    % 4. Color de fondo
    % 5. Coord x del titulo
    % 6. Coord y del titulo
    % 7. Fecha

    % Personaliza el formato del título
    \pretitle{\begin{center}\bfseries\fontsize{42}{56}\selectfont}
    \posttitle{\par\end{center}\vspace{2em}}
    
    % Personaliza el formato del autor
    \preauthor{\begin{center}\Large}
    \postauthor{\par\end{center}\vspace{3em}}
    
    % Personaliza el formato de la fecha
    \predate{\begin{center}\huge}
    \postdate{\par\end{center}\vspace{5em}}
    
    \title{#2}
    \author{\href{https://losdeldgiim.github.io/}{Los Del DGIIM}}
    \date{Granada, #7}
    \thispagestyle{empty}               % Sin encabezado ni pie de página
    \maketitle
    \vfill
}
    \portadaExamen{ffccA4.jpg}{Ecuaciones\\Diferenciales I\\Examen XXIV}{Ecuaciones Diferenciales I. Examen XXIV}{MidnightBlue}{-8}{28}{2024-2025}{Arturo Olivares Martos}

    \begin{description}
        \item[Asignatura] Ecuaciones Diferenciales I
        \item[Curso Académico] 2018-19.
        % \item[Grado] Doble Grado en Ingeniería Informática y Matemáticas.
        \item[Grupo] A.
        \item[Profesor] Rafael Ortega Ríos.
        \item[Descripción] Parcial C.
        \item[Fecha] 23 de Mayo de 2019.
        %\item[Duración] 60 minutos.    
    \end{description}
    \newpage

    \begin{ejercicio}
    Se considera la ecuación diferencial de segundo orden
    \begin{equation*}
        (1+2t+t^2)x''+2(1+t)x'-2x=0.
    \end{equation*}
    \begin{enumerate}
        \item Encuentre una solución del tipo $x=at+b$, con $a,b\in\bb{R}$ adecuadas. ¿Es esta solución única? ¿Se puede formar un sistema findamental con soluciones de este tipo?
        
        Para que $x=at+b$ sea solución de la ecuación, calculamos sus derivadas:
        \begin{equation*}
            x'(t)=a,\quad x''(t)=0.
        \end{equation*}

        Sustituyendo en la ecuación, obtenemos
        \begin{equation*}
            2(1+t)a-2(at+b)=0
            \Longrightarrow 2(a-a)t+2(a-b)=0
        \end{equation*}

        Por tanto, para cada $a\in \bb{R}$, tenemos que:
        \begin{equation*}
            x_a(t)=a(t+1)\qquad \forall t\in\bb{R}.
        \end{equation*}
        es una solución de la ecuación. Por tanto, tenemos que no es única. No obstante, no podemos formar un sistema fundamental con soluciones de este tipo, ya que dados $a_1,a_2\in\bb{R}$, veamos $x_{a_1}(t),x_{a_2}(t)$ que son linealmente dependientes.
        \begin{itemize}
            \item Si $a_2=0$, entonces se tiene de forma directa que es linealmente dependiente.
            \item Si $a_2\neq 0$, entonces:
            \begin{equation*}
                x_{a_1}(t)=\dfrac{a_1}{a_2}\cdot a_2 (t+1)=\dfrac{a_1}{a_2}x_{a_2}(t).
            \end{equation*}
            Por tanto, también son linealmente dependientes.
        \end{itemize}

        \item Use la fórmula de Liouville para completar un sistema fundamental de la ecuación.
        
        Sea $x_1(t)=t+1$, y sea $\varphi(t)$ una solución linealmente independiente de $x_1(t)$, que sabemos que existe por ser $\dim Z=2$. Calculemos su Wronskiano:
        \begin{equation*}
            W(x_1,\varphi)(t)=\begin{vmatrix}
                t+1 & \varphi(t)\\
                1 & \varphi'(t)
            \end{vmatrix}=(t+1)\varphi'(t)-\varphi(t).
        \end{equation*}
        
        En primer lugar, para que $\{x_1,\varphi\}$ sean linealmente independientes, imponemos que, fijado $t_0=0$, se tenga $W(x_1,\varphi)(0)=1$.        
        Por tanto, por la fórmula de Jacobi-Liouville, tenemos que la solución buscada cumple la ecuación, válida para todo $t\in\bb{R}\setminus \{-1\}$:
        \begin{align*}
            W(x_1,\varphi)(t)&=W(x_1,\varphi)(0)\exp\left(-\int_0^t a_{k-1}(s)\ ds\right)\\
            W(x_1,\varphi)(t)&=1\exp\left(-\int_0^t \dfrac{2(1+s)}{(1+2s+s^2)}\ ds\right)\\
            W(x_1,\varphi)(t)&=\exp\left(-\int_0^t \dfrac{2}{1+s}\ ds\right)
            = \exp\left(-2\ln(|1+t|)\right)=\dfrac{1}{(1+t)^2}.
        \end{align*}

        Por tanto, la solución buscada cumple la ecuación diferencial:
        \begin{equation*}
            (t+1)\varphi'-\varphi=\dfrac{1}{(1+t)^2}
            \Longrightarrow
            \varphi'=\dfrac{\varphi}{t+1}+\dfrac{1}{(t+1)^3}
        \end{equation*}

        Por tanto, y tomando como constante de integración $0$ (ya que solo buscamos una solución), tenemos que:
        \begin{align*}
            \varphi(t)&=e^{\ln|t+1|}\int e^{-\ln|s+1|}\dfrac{1}{(s+1)^3}\ ds\\
            &= |t+1|\int \dfrac{1}{|s+1|(s+1)^3}\ ds \\
            &= -|t+1|\cdot \frac{1}{3}\cdot \frac{1}{|t+1|(t+1)^2}
            = -\dfrac{1}{3(t+1)^2}.
        \end{align*}

        Por tanto, un sistema fundamental de la ecuación (considerando un dominio $\Omega\subset \bb{R}\setminus \{-1\}$) es:
        \begin{equation*}
            \left\{x_1(t)=t+1,\ \varphi(t)=-\dfrac{1}{3(t+1)^2}\right\}.
        \end{equation*}
    \end{enumerate}
    \end{ejercicio}

    \begin{ejercicio}
        Encuentre la solución general de la ecuación diferencial de segundo orden
        \begin{equation*}
            y''-3y'+2y=xe^x +2x.
        \end{equation*}
        por el método de variación de constantes
    \end{ejercicio}

    \begin{ejercicio}
        Encuentre la solución general del sistema
        \begin{equation*}
            x'=Ax+b,
        \end{equation*}
        donde $A=\begin{pmatrix}
            1 & 2\\
            0 & 0
        \end{pmatrix}$ y $b=\begin{pmatrix}
            1\\
            2
        \end{pmatrix}$. ¿Tiene este sistema soluciones constantes?\\

        Estudiemos en primer lugar si tiene soluciones constantes. Sean $x_1,x_2\in \bb{R}$, de forma que:
        \begin{equation*}
            x(t)=\begin{pmatrix}
                x_1\\
                x_2
            \end{pmatrix}
            \qquad x'(t)=\begin{pmatrix}
                0\\
                0
            \end{pmatrix}
            \qquad \forall t\in\bb{R}.
        \end{equation*}

        De esta forma, buscamos $x_1,x_2$ tales que:
        \begin{equation*}
            A\begin{pmatrix}
                x_1\\
                x_2
            \end{pmatrix}=-\begin{pmatrix}
                1\\
                2
            \end{pmatrix}
        \end{equation*}
        Como $|A|=0$, tenemos que ese sistema es incompatible, por lo que no tiene soluciones constantes. Busquemos por tanto resolver el sistema. Notando a $x=(x_1,x_2)^t$, tenemos que:
        \begin{equation*}
            \begin{cases}
                x_1'=x_1+2x_2+1\\
                x_2'=2
            \end{cases}
        \end{equation*}

        Por tanto, tenemos que:
        \begin{equation*}
            x_2(t)=2t+c_2
        \end{equation*}

        Resolvemos ahora por tanto la ecuación para $x_1$:
        \begin{align*}
            x_1'=x_1+2(2t+c_2)+1
            \Longrightarrow x_1(t)&=e^t\left(c_1+\int e^{-s}\left(4s+2c_2+1\right)\ dt\right)\\
            &= e^t\left(c_1-e^{-t}(4(t+1)+2c_2+1)\right)\\
            &= c_1e^t -(4t+2c_2+5)
        \end{align*}

        Por tanto, la solución general del sistema es:
        \begin{equation*}
            x(t)=c_1\begin{pmatrix}
                e^t\\
                0
            \end{pmatrix}+c_2\begin{pmatrix}
                -2\\
                1
            \end{pmatrix}+\begin{pmatrix}
                -(4t+5)\\
                2t
            \end{pmatrix}
        \end{equation*}
    \end{ejercicio}

    \begin{ejercicio}
    Calcule (en función de $A$) la matriz fundamental principal en cero del sistema $x'=Ax$, sabiendo que $A^2=A$.\\

    Calculemos en primer lugar $e^{tA}$:
    \begin{align*}
        e^{tA}&=\sum_{n=0}^\infty \dfrac{(tA)^n}{n!}
        =\sum_{n=0}^\infty \dfrac{t^nA^n}{n!}
    \end{align*}

    Usando que $A^n=A$ para todo $n\in\bb{N}$ y $A^0=I$, tenemos que:
    \begin{align*}
        e^{tA}&=\sum_{n=0}^\infty \dfrac{t^nA}{n!}
        =\sum_{n=0}^0 \dfrac{t^nI}{n!}+\sum_{n=1}^\infty \dfrac{t^nA}{n!}
        =I+A\cdot \sum_{n=1}^\infty \dfrac{t^n}{n!}
        =I+A\left(\sum_{n=0}^\infty \dfrac{t^n}{n!}-\sum_{n=0}^0 \dfrac{t^n}{n!}\right)\\
        &=I+A(e^t-1)
    \end{align*}

    Por tanto, usando lo visto en Teoría, tenemos que la matriz fundamental principal en cero del sistema es:
    \begin{equation*}
        \Phi(t)=e^{tA}=I+A(e^t-1)
    \end{equation*}
    \end{ejercicio}
\end{document}
