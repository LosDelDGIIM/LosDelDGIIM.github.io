\documentclass[12pt]{article}

% Idioma y codificación
\usepackage[spanish, es-tabla]{babel}       %es-tabla para que se titule "Tabla"
\usepackage[utf8]{inputenc}

% Márgenes
\usepackage[a4paper,top=3cm,bottom=2.5cm,left=3cm,right=3cm]{geometry}

% Comentarios de bloque
\usepackage{verbatim}

% Paquetes de links
\usepackage[hidelinks]{hyperref}    % Permite enlaces
\usepackage{url}                    % redirecciona a la web

% Más opciones para enumeraciones
\usepackage{enumitem}

% Personalizar la portada
\usepackage{titling}

% Paquetes de tablas
\usepackage{multirow}


%------------------------------------------------------------------------

%Paquetes de figuras
\usepackage{caption}
\usepackage{subcaption} % Figuras al lado de otras
\usepackage{float}      % Poner figuras en el sitio indicado H.


% Paquetes de imágenes
\usepackage{graphicx}       % Paquete para añadir imágenes
\usepackage{transparent}    % Para manejar la opacidad de las figuras

% Paquete para usar colores
\usepackage[dvipsnames]{xcolor}
\usepackage{pagecolor}      % Para cambiar el color de la página

% Habilita tamaños de fuente mayores
\usepackage{fix-cm}

% Para los gráficos
\usepackage{tikz}

% Para poder situar los nodos en los grafos
\usetikzlibrary{positioning}


%------------------------------------------------------------------------

% Paquetes de matemáticas
\usepackage{mathtools, amsfonts, amssymb, mathrsfs}
\usepackage[makeroom]{cancel}     % Simplificar tachando
\usepackage{polynom}    % Divisiones y Ruffini
\usepackage{units} % Para poner fracciones diagonales con \nicefrac

\usepackage{pgfplots}   %Representar funciones
\pgfplotsset{compat=1.18}  % Versión 1.18

\usepackage{tikz-cd}    % Para usar diagramas de composiciones
\usetikzlibrary{calc}   % Para usar cálculo de coordenadas en tikz

%Definición de teoremas, etc.
\usepackage{amsthm}
%\swapnumbers   % Intercambia la posición del texto y de la numeración

\theoremstyle{plain}

\makeatletter
\@ifclassloaded{article}{
  \newtheorem{teo}{Teorema}[section]
}{
  \newtheorem{teo}{Teorema}[chapter]  % Se resetea en cada chapter
}
\makeatother

\newtheorem{coro}{Corolario}[teo]           % Se resetea en cada teorema
\newtheorem{prop}[teo]{Proposición}         % Usa el mismo contador que teorema
\newtheorem{lema}[teo]{Lema}                % Usa el mismo contador que teorema

\theoremstyle{remark}
\newtheorem*{observacion}{Observación}

\theoremstyle{definition}

\makeatletter
\@ifclassloaded{article}{
  \newtheorem{definicion}{Definición} [section]     % Se resetea en cada chapter
}{
  \newtheorem{definicion}{Definición} [chapter]     % Se resetea en cada chapter
}
\makeatother

\newtheorem*{notacion}{Notación}
\newtheorem*{ejemplo}{Ejemplo}
\newtheorem*{ejercicio*}{Ejercicio}             % No numerado
\newtheorem{ejercicio}{Ejercicio} [section]     % Se resetea en cada section


% Modificar el formato de la numeración del teorema "ejercicio"
\renewcommand{\theejercicio}{%
  \ifnum\value{section}=0 % Si no se ha iniciado ninguna sección
    \arabic{ejercicio}% Solo mostrar el número de ejercicio
  \else
    \thesection.\arabic{ejercicio}% Mostrar número de sección y número de ejercicio
  \fi
}


% \renewcommand\qedsymbol{$\blacksquare$}         % Cambiar símbolo QED
%------------------------------------------------------------------------

% Paquetes para encabezados
\usepackage{fancyhdr}
\pagestyle{fancy}
\fancyhf{}

\newcommand{\helv}{ % Modificación tamaño de letra
\fontfamily{}\fontsize{12}{12}\selectfont}
\setlength{\headheight}{15pt} % Amplía el tamaño del índice


%\usepackage{lastpage}   % Referenciar última pag   \pageref{LastPage}
\fancyfoot[C]{\thepage}

%------------------------------------------------------------------------

% Conseguir que no ponga "Capítulo 1". Sino solo "1."
\makeatletter
\@ifclassloaded{book}{
  \renewcommand{\chaptermark}[1]{\markboth{\thechapter.\ #1}{}} % En el encabezado
    
  \renewcommand{\@makechapterhead}[1]{%
  \vspace*{50\p@}%
  {\parindent \z@ \raggedright \normalfont
    \ifnum \c@secnumdepth >\m@ne
      \huge\bfseries \thechapter.\hspace{1em}\ignorespaces
    \fi
    \interlinepenalty\@M
    \Huge \bfseries #1\par\nobreak
    \vskip 40\p@
  }}
}
\makeatother

%------------------------------------------------------------------------
% Paquetes de cógido
\usepackage{minted}
\renewcommand\listingscaption{Código fuente}

\usepackage{fancyvrb}
% Personaliza el tamaño de los números de línea
\renewcommand{\theFancyVerbLine}{\small\arabic{FancyVerbLine}}

% Estilo para C++
\newminted{cpp}{
    frame=lines,
    framesep=2mm,
    baselinestretch=1.2,
    linenos,
    escapeinside=||
}

% para minted
\definecolor{LightGray}{rgb}{0.95,0.95,0.92}
\setminted{
    linenos=true,
    stepnumber=5,
    numberfirstline=true,
    autogobble,
    breaklines=true,
    breakautoindent=true,
    breaksymbolleft=,
    breaksymbolright=,
    breaksymbolindentleft=0pt,
    breaksymbolindentright=0pt,
    breaksymbolsepleft=0pt,
    breaksymbolsepright=0pt,
    fontsize=\footnotesize,
    bgcolor=LightGray,
    numbersep=10pt
}


\usepackage{listings} % Para incluir código desde un archivo

\renewcommand\lstlistingname{Código Fuente}
\renewcommand\lstlistlistingname{Índice de Códigos Fuente}

% Definir colores
\definecolor{vscodepurple}{rgb}{0.5,0,0.5}
\definecolor{vscodeblue}{rgb}{0,0,0.8}
\definecolor{vscodegreen}{rgb}{0,0.5,0}
\definecolor{vscodegray}{rgb}{0.5,0.5,0.5}
\definecolor{vscodebackground}{rgb}{0.97,0.97,0.97}
\definecolor{vscodelightgray}{rgb}{0.9,0.9,0.9}

% Configuración para el estilo de C similar a VSCode
\lstdefinestyle{vscode_C}{
  backgroundcolor=\color{vscodebackground},
  commentstyle=\color{vscodegreen},
  keywordstyle=\color{vscodeblue},
  numberstyle=\tiny\color{vscodegray},
  stringstyle=\color{vscodepurple},
  basicstyle=\scriptsize\ttfamily,
  breakatwhitespace=false,
  breaklines=true,
  captionpos=b,
  keepspaces=true,
  numbers=left,
  numbersep=5pt,
  showspaces=false,
  showstringspaces=false,
  showtabs=false,
  tabsize=2,
  frame=tb,
  framerule=0pt,
  aboveskip=10pt,
  belowskip=10pt,
  xleftmargin=10pt,
  xrightmargin=10pt,
  framexleftmargin=10pt,
  framexrightmargin=10pt,
  framesep=0pt,
  rulecolor=\color{vscodelightgray},
  backgroundcolor=\color{vscodebackground},
}

%------------------------------------------------------------------------

% Comandos definidos
\newcommand{\bb}[1]{\mathbb{#1}}
\newcommand{\cc}[1]{\mathcal{#1}}

% I prefer the slanted \leq
\let\oldleq\leq % save them in case they're every wanted
\let\oldgeq\geq
\renewcommand{\leq}{\leqslant}
\renewcommand{\geq}{\geqslant}

% Si y solo si
\newcommand{\sii}{\iff}

% Letras griegas
\newcommand{\eps}{\epsilon}
\newcommand{\veps}{\varepsilon}
\newcommand{\lm}{\lambda}

\newcommand{\ol}{\overline}
\newcommand{\ul}{\underline}
\newcommand{\wt}{\widetilde}
\newcommand{\wh}{\widehat}

\let\oldvec\vec
\renewcommand{\vec}{\overrightarrow}

% Derivadas parciales
\newcommand{\del}[2]{\frac{\partial #1}{\partial #2}}
\newcommand{\Del}[3]{\frac{\partial^{#1} #2}{\partial #3^{#1}}}
\newcommand{\deld}[2]{\dfrac{\partial #1}{\partial #2}}
\newcommand{\Deld}[3]{\dfrac{\partial^{#1} #2}{\partial #3^{#1}}}


\newcommand{\AstIg}{\stackrel{(\ast)}{=}}
\newcommand{\Hop}{\stackrel{L'H\hat{o}pital}{=}}

\newcommand{\red}[1]{{\color{red}#1}} % Para integrales, destacar los cambios.

% Método de integración
\newcommand{\MetInt}[2]{
    \left[\begin{array}{c}
        #1 \\ #2
    \end{array}\right]
}

% Declarar aplicaciones
% 1. Nombre aplicación
% 2. Dominio
% 3. Codominio
% 4. Variable
% 5. Imagen de la variable
\newcommand{\Func}[5]{
    \begin{equation*}
        \begin{array}{rrll}
            #1:& #2 & \longrightarrow & #3\\
               & #4 & \longmapsto & #5
        \end{array}
    \end{equation*}
}

%------------------------------------------------------------------------



\begin{document}

    % 1. Foto de fondo
    % 2. Título
    % 3. Encabezado Izquierdo
    % 4. Color de fondo
    % 5. Coord x del titulo
    % 6. Coord y del titulo
    % 7. Fecha

    
    % 1. Foto de fondo
% 2. Título
% 3. Encabezado Izquierdo
% 4. Color de fondo
% 5. Coord x del titulo
% 6. Coord y del titulo
% 7. Fecha

\newcommand{\portada}[7]{

    \portadaBase{#1}{#2}{#3}{#4}{#5}{#6}{#7}
    \portadaBook{#1}{#2}{#3}{#4}{#5}{#6}{#7}
}

\newcommand{\portadaExamen}[7]{

    \portadaBase{#1}{#2}{#3}{#4}{#5}{#6}{#7}
    \portadaArticle{#1}{#2}{#3}{#4}{#5}{#6}{#7}
}




\newcommand{\portadaBase}[7]{

    % Tiene la portada principal y la licencia Creative Commons
    
    % 1. Foto de fondo
    % 2. Título
    % 3. Encabezado Izquierdo
    % 4. Color de fondo
    % 5. Coord x del titulo
    % 6. Coord y del titulo
    % 7. Fecha
    
    
    \thispagestyle{empty}               % Sin encabezado ni pie de página
    \newgeometry{margin=0cm}        % Márgenes nulos para la primera página
    
    
    % Encabezado
    \fancyhead[L]{\helv #3}
    \fancyhead[R]{\helv \nouppercase{\leftmark}}
    
    
    \pagecolor{#4}        % Color de fondo para la portada
    
    \begin{figure}[p]
        \centering
        \transparent{0.3}           % Opacidad del 30% para la imagen
        
        \includegraphics[width=\paperwidth, keepaspectratio]{assets/#1}
    
        \begin{tikzpicture}[remember picture, overlay]
            \node[anchor=north west, text=white, opacity=1, font=\fontsize{60}{90}\selectfont\bfseries\sffamily, align=left] at (#5, #6) {#2};
            
            \node[anchor=south east, text=white, opacity=1, font=\fontsize{12}{18}\selectfont\sffamily, align=right] at (9.7, 3) {\textbf{\href{https://losdeldgiim.github.io/}{Los Del DGIIM}}};
            
            \node[anchor=south east, text=white, opacity=1, font=\fontsize{12}{15}\selectfont\sffamily, align=right] at (9.7, 1.8) {Doble Grado en Ingeniería Informática y Matemáticas\\Universidad de Granada};
        \end{tikzpicture}
    \end{figure}
    
    
    \restoregeometry        % Restaurar márgenes normales para las páginas subsiguientes
    \pagecolor{white}       % Restaurar el color de página
    
    
    \newpage
    \thispagestyle{empty}               % Sin encabezado ni pie de página
    \begin{tikzpicture}[remember picture, overlay]
        \node[anchor=south west, inner sep=3cm] at (current page.south west) {
            \begin{minipage}{0.5\paperwidth}
                \href{https://creativecommons.org/licenses/by-nc-nd/4.0/}{
                    \includegraphics[height=2cm]{assets/Licencia.png}
                }\vspace{1cm}\\
                Esta obra está bajo una
                \href{https://creativecommons.org/licenses/by-nc-nd/4.0/}{
                    Licencia Creative Commons Atribución-NoComercial-SinDerivadas 4.0 Internacional (CC BY-NC-ND 4.0).
                }\\
    
                Eres libre de compartir y redistribuir el contenido de esta obra en cualquier medio o formato, siempre y cuando des el crédito adecuado a los autores originales y no persigas fines comerciales. 
            \end{minipage}
        };
    \end{tikzpicture}
    
    
    
    % 1. Foto de fondo
    % 2. Título
    % 3. Encabezado Izquierdo
    % 4. Color de fondo
    % 5. Coord x del titulo
    % 6. Coord y del titulo
    % 7. Fecha


}


\newcommand{\portadaBook}[7]{

    % 1. Foto de fondo
    % 2. Título
    % 3. Encabezado Izquierdo
    % 4. Color de fondo
    % 5. Coord x del titulo
    % 6. Coord y del titulo
    % 7. Fecha

    % Personaliza el formato del título
    \pretitle{\begin{center}\bfseries\fontsize{42}{56}\selectfont}
    \posttitle{\par\end{center}\vspace{2em}}
    
    % Personaliza el formato del autor
    \preauthor{\begin{center}\Large}
    \postauthor{\par\end{center}\vfill}
    
    % Personaliza el formato de la fecha
    \predate{\begin{center}\huge}
    \postdate{\par\end{center}\vspace{2em}}
    
    \title{#2}
    \author{\href{https://losdeldgiim.github.io/}{Los Del DGIIM}}
    \date{Granada, #7}
    \maketitle
    
    \tableofcontents
}




\newcommand{\portadaArticle}[7]{

    % 1. Foto de fondo
    % 2. Título
    % 3. Encabezado Izquierdo
    % 4. Color de fondo
    % 5. Coord x del titulo
    % 6. Coord y del titulo
    % 7. Fecha

    % Personaliza el formato del título
    \pretitle{\begin{center}\bfseries\fontsize{42}{56}\selectfont}
    \posttitle{\par\end{center}\vspace{2em}}
    
    % Personaliza el formato del autor
    \preauthor{\begin{center}\Large}
    \postauthor{\par\end{center}\vspace{3em}}
    
    % Personaliza el formato de la fecha
    \predate{\begin{center}\huge}
    \postdate{\par\end{center}\vspace{5em}}
    
    \title{#2}
    \author{\href{https://losdeldgiim.github.io/}{Los Del DGIIM}}
    \date{Granada, #7}
    \thispagestyle{empty}               % Sin encabezado ni pie de página
    \maketitle
    \vfill
}
    \portadaExamen{ffccA4.jpg}{Ecuaciones\\Diferenciales I\\Examen XI}{Ecuaciones Diferenciales I. Examen XI}{MidnightBlue}{-8}{28}{2024-2025}{Arturo Olivares Martos}

    \begin{description}
        \item[Asignatura] Ecuaciones Diferenciales I
        \item[Curso Académico] 2016-17.
        %\item[Grado] Doble Grado en Ingeniería Informática y Matemáticas.
        \item[Grupo] A.
        \item[Profesor] Rafael Ortega Ríos.
        \item[Descripción] Parcial B.
        \item[Fecha] 27 de abril de 2017.
        %\item[Duración] 60 minutos.
    
    \end{description}
    \newpage

\begin{ejercicio}
    Encuentra la solución del problema siguiente, indicando el intervalo en el que está definida:
    \begin{equation*}
        y - 4x^3 + (2y + x)y' = 0, \quad y(0) = -1.
    \end{equation*}

    Sea el dominio de la ecuación $\Omega=\bb{R}^2$. Definimos las funciones:
    \Func{P}{\Omega}{\bb{R}}{(x,y)}{y-4x^3}
    \Func{Q}{\Omega}{\bb{R}}{(x,y)}{2y+x}

    Tenemos que $P,Q\in C^1(\Omega)$. Además, tenemos que:
    \begin{equation*}
        \frac{\partial P}{\partial y}(x,y) = 1 = \frac{\partial Q}{\partial x}(x,y)\qquad \forall (x,y)\in \Omega.
    \end{equation*}

    Por tanto, la ecuación es exacta. Buscamos una función potencial $U:\Omega\to \bb{R}$ tal que $\nabla U = (P,Q)$.
    Integrando la primera componente de $\nabla U$ respecto de $x$ obtenemos:
    \begin{equation*}
        U(x,y) = \int P(x,y)\,dx = \int (y-4x^3)\,dx = xy - x^4 + \varphi(y),
    \end{equation*}
    donde $\varphi:\bb{R}\to \bb{R}$ es una función derivable respecto de $y$ que representa la constante de integración. Derivando respecto de $y$ obtenemos:
    \begin{align*}
        \frac{\partial U}{\partial y}(x,y) &= x + \varphi'(y)\\
        &= Q(x,y) = 2y + x.
    \end{align*}

    Por tanto, $\varphi'(y) = 2y$, de lo que obtenemos $\varphi(y) = y^2$ (por ejemplo. Notemos que el potencial es úncio salvo una constante aditiva). Por tanto, el potencial es:
    \begin{equation*}
        U(x,y) = xy - x^4 + y^2.
    \end{equation*}
    
    De esta forma, la ecuación diferencial se puede expresar como:
    \begin{equation*}
        \dfrac{d}{dx}(U(x,y(x))) = 0.
    \end{equation*}

    Por tanto, como $U\in C^2(\bb{R}^2)$, tenemos que:
    \begin{equation*}
        U(x,y) = C \quad \forall (x,y)\in \Omega,
    \end{equation*}
    Para cierto $C$. Como se pide la condición inicial $y(0) = -1$, tenemos que:
    \begin{equation*}
        U(0,-1) = 1 = C.
    \end{equation*}

    Por tanto, la solución del problema de valores iniciales viene dado de forma implícita por la ecuación:
    \begin{equation*}
        xy - x^4 + y^2 = 1.
    \end{equation*}

    Obtengamos los valores de $y(x)$ (donde nos quedamos con la raíz negativa ya que $y(0)=-1$):
    \begin{equation*}
        y(x)=\dfrac{-x- \sqrt{x^2+4\cdot \dot (1+x^4)}}{2}
        \qquad \forall x\in \bb{R}.
    \end{equation*}
\end{ejercicio}

\begin{ejercicio}
    Encuentra un factor integrante del tipo $\mu(t,x) = m(t)$ para la ecuación
    \begin{equation*}
        2t + t^2x + x\dot{x} = 0.
    \end{equation*}
\end{ejercicio}

\begin{ejercicio}
    Demuestra que las funciones $f_1,f_2,f_3:\bb{R}\to \bb{R}$ dadas por:
    \begin{equation*}
        f_1(t) = e^t, \quad f_2(t) = e^{2t}, \quad f_3(t) = e^{3t},
    \end{equation*}
    son linealmente independientes.
\end{ejercicio}

\begin{ejercicio}
    Demuestra que la función $F:\bb{R}\to \bb{R}$ definida por la integral
    \begin{equation*}
        F(x) = \int_0^1 e^{\theta x^2}\cos^2(\theta)\,d\theta
    \end{equation*}
    es derivable y cumple $F'(0) = 0$.\\

    Definimos la siguiente función auxiliar:
    \Func{G}{\bb{R}\times [0,1]}{\bb{R}}{(x,\theta)}{e^{\theta x^2}\cos^2(\theta).}

    Comprobemos en primer lugar que $G\in C^1(\bb{R}\times [0,1])$. Esto es directo, por ser producto y composición de funciones de clase $C^1$. Entonces, por el Teorema de la Derivación de Funciones dependientes de un Parámetro, la función $F$ es derivable en $\bb{R}$ con:
    \begin{equation*}
        F'(x) = \int_0^1 \frac{\partial G}{\partial x}(x,\theta)\,d\theta.
    \end{equation*}

    Calculemos su derivada parcial de primer orden respecto de $x$:
    \begin{align*}
        \frac{\partial G}{\partial x}(x,\theta) &= 2x\theta e^{\theta x^2}\cos^2(\theta)
    \end{align*}
    
    Por tanto, la derivada de $F$ es:
    \begin{align*}
        F'(x) &= \int_0^1 2x\theta e^{\theta x^2}\cos^2(\theta)\,d\theta\\
        &= 2x\int_0^1 \theta e^{\theta x^2}\cos^2(\theta)\,d\theta.
    \end{align*}

    Evaluando en $x=0$ obtenemos:
    \begin{align*}
        F'(0) &= 2\cdot 0\int_0^1 \theta e^{\theta\cdot 0^2}\cos^2(\theta)\,d\theta = 0
    \end{align*}
\end{ejercicio}

\begin{ejercicio}
    Dada una función $\ell\in C^1(\bb{R})$ que cumple $\ell(t) > 0$ para cada $t\in \bb{R}$ se define la transformación del plano
    \Func{\varphi}{\bb{R}^2}{\bb{R}^2}{(t,x)}{(t,\ell(t)x).}

    Demuestra que el conjunto de estas transformaciones es un grupo de difeomorfismos. Encuentra el subgrupo que deja invariante la ecuación $x' = 2t^2x$.
\end{ejercicio}
\end{document}
