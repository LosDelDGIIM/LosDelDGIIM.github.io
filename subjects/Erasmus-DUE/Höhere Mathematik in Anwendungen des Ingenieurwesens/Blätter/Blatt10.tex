\section{Potenzialfelder und Mehrfachintegrale}

\begin{ejercicio}[Kurvenintegral einer Funktion mit Potenzial]

    Im $\bb{R}^2$ ist eine Kurve gegeben durch die Parameterdarstellung
    \Func{\varphi}{\left[0, \nicefrac{\pi}{2}\right]}{\bb{R}^2}{t}{\left(\sin t, t\right)}

    Berechnen Sie das Kurvenintegral $\displaystyle \int_{\varphi} \vec{f} \cdot d\vec{s}$ für:
    \Func{\vec{f}}{\bb{R}^2}{\bb{R}^2}{(x,y)}{\begin{matrix}
2xy + 2\cos(y) \\ x^2 - 2x\sin(y) + 1
    \end{matrix}}
    \begin{observacion}
        Prüfen Sie, ob $\vec{f}$ ein Potenzial hat und nutzen Sie es ggf. aus.
    \end{observacion}
\end{ejercicio}


\begin{ejercicio}[Doppelintegral mit e-Funktion]
    Berechnen Sie
    \begin{equation*}
    I = \int_0^1 \int_0^{y^2} e^{\nicefrac{x}{y}} \, dx \, dy
    \end{equation*}
\end{ejercicio}


\begin{ejercicio}[Flächenberechnung mit Doppelintegral]
    Ein Flächenstück wird durch die Kurven $x=0$, $y=2x$ $y=\nicefrac{1}{a}x^2 + a$, $a > 0$ berandet.
    Berechnen Sie den Flächeninhalt $A$ mit Hilfe eines Doppelintegrals.
\end{ejercicio}

\begin{ejercicio}[Flächeninhalt mit Polarkoordinaten]
    Die Randkurve eines Gebiets in der Ebene wird durch die Gleichung
    \[r = 2(\cos(\varphi) + \sin(\varphi))\]
    in Polarkoordinaten beschrieben. Berechnen Sie den Flächeninhalt 
    $A$ dieses Gebiets.
\end{ejercicio}

\begin{ejercicio}[Dreifachintegral in Zylinderkoordinaten]
    Berechnen Sie das folgende in Zylinderkoordinaten gegebene Integral:
    \[
    I = \int_{\pi}^{2\pi} \int_0^1 \int_{r}^{r^2}rz \cdot \sin(\varphi) \, dz \, dr \, d\varphi.
    \]
\end{ejercicio}