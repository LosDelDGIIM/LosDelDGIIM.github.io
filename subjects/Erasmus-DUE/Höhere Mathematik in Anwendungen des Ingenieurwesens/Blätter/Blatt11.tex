\section{Datenanalyse und Fehlerrechnung}

\begin{ejercicio}[Ein neues Schwimmbecken]
    Bei einem Schwimmbadbauer wird ein quaderförmiges Becken mit einer Länge $l = \unit[50]{m}$, einer Breite $b = \unit[15]{m}$ und einer Tiefe $h = \unit[2]{m}$ in Auftrag gegeben. Erfahrungsgemäß baut
    die Firma ziemlich ungenau mit statistisch verteilten Fehlern von $\sigma_l = \unit[0.4]{m}$, $\sigma_b = \unit[0.2]{m}$ sowie $\sigma_h = \unit[0.1]{m}$ für Länge, Breite bzw. Tiefe.
    Berechnen Sie die das Volumen des Schwimmbeckens und den zu erwartenden relativen Fehler nach der Gaußschen Fehlerfortpflanzung. Wieviel Wasser (in $\unit{m^3}$) sollte bereitstehen, damit das Becken auf jeden Fall bis zum Rand gefüllt werden kann?\\

    The volume $V$ of the swimming pool is given by the formula:
    \begin{equation*}
        V = l \cdot b \cdot h
        = \unit[50]{m} \cdot \unit[15]{m} \cdot \unit[2]{m}
        = \unit[1500]{m^3}
    \end{equation*}

    To calculate the relative error in the volume using Gaussian error propagation, we use the formula:
    \begin{align*}
        \sigma_V &= \sqrt{\left(\dfrac{\partial V}{\partial l} \sigma_l\right)^2 + \left(\dfrac{\partial V}{\partial b} \sigma_b\right)^2 + \left(\dfrac{\partial V}{\partial h} \sigma_h\right)^2}
        =\\&= \sqrt{\left(b \cdot h \cdot \sigma_l\right)^2 + \left(l \cdot h \cdot \sigma_b\right)^2 + \left(l \cdot b \cdot \sigma_h\right)^2}
        =\\&= \sqrt{\left(\unit[15]{m} \cdot \unit[2]{m} \cdot \unit[0.4]{m}\right)^2 + \left(\unit[50]{m} \cdot \unit[2]{m} \cdot \unit[0.2]{m}\right)^2 + \left(\unit[50]{m} \cdot \unit[15]{m} \cdot \unit[0.1]{m}\right)^2}
        =\\&= \sqrt{\left(\unit[12]{m^3}\right)^2 + \left(\unit[20]{m^3}\right)^2 + \left(\unit[75]{m^3}\right)^2}
        =\\&= \sqrt{\unit[144]{m^6} + \unit[400]{m^6} + \unit[5625]{m^6}}
        =\\&= \sqrt{\unit[6169]{m^6}}
        \approx \unit[78.55]{m^3}
    \end{align*}

    The relative error in the volume is then given by:
    \begin{equation*}
        \dfrac{\sigma_V}{V} = \dfrac{\unit[78.55]{m^3}}{\unit[1500]{m^3}} \approx 0.0524 \approx 5.24\%
    \end{equation*}

    Therefore, the volume of the swimming pool is $\unit[1500]{m^3}$ with an expected relative error of approximately $5.24\%$. To ensure that the pool can be filled to the brim with a level of confidence of around $99.7\%$ (which corresponds to $3\sigma$), we should prepare the following amount of water:
    \begin{equation*}
        V + 3\sigma_V = \unit[1500]{m^3} + 3 \cdot \unit[78.55]{m^3} \approx \unit[1735.65]{m^3}
    \end{equation*}
\end{ejercicio}


\begin{ejercicio}[Selbstinduktion einer Doppelleitung]

    Eine elektrische Doppelleitung besteht aus zwei parallelen, elektrisch leitenden Drähten mit der Länge $l$ und dem Radius $r$. Der Abstand ihrer Mittelpunkte beträgt $a$. Die Selbstinduktivität $L$ dieser Doppelleitung in Luft wird dabei nach der Formel
    \begin{equation*}
        L(l, r, a) = \dfrac{\mu_0}{\pi} \, l \cdot \ln\left(\dfrac{a - r}{r}+ \dfrac{1}{4}\right)
    \end{equation*}
    berechnet, mit der magnetischen Feldkonstante $\mu_0 = \unitfrac[4\pi\cdot 10^{-7}]{Vs}{Am}$.
    \begin{enumerate}
        \item Welche Selbstinduktivität $L$ besitzt eine Doppelleitung, deren Dimensionen wie folgt gemessen wurden?
        \begin{equation*}
            l = (\unit[2000 \pm 10]{m}); \quad r = (\unit[2 \pm 0.05]{mm}); \quad a = (\unit[30 \pm 0.3]{cm})
        \end{equation*}

        The self-inductance $L$ of the twin-lead can be calculated using the given formula:
        \begin{align*}
            L &= \dfrac{\mu_0}{\pi} \, l \cdot \ln\left(\dfrac{a - r}{r}+ \dfrac{1}{4}\right)
            =\\&= \dfrac{\unitfrac[4\pi\cdot 10^{-7}]{Vs}{Am}}{\pi} \cdot \unit[2000]{m} \cdot \ln\left(\dfrac{\unit[0.30]{m} - \unit[0.002]{m}}{\unit[0.002]{m}} + \dfrac{1}{4}\right)
            =\\&= \unitfrac[8 \cdot 10^{-4}]{Vs}{A} \ln\left(149 + 0.25\right)
            \approx \unitfrac[8 \cdot 10^{-4}]{Vs}{A} \cdot 5.005622
            =\\&= \unit[4.004498]{mH}
        \end{align*}


        \item Wie groß ist die absolute bzw. prozentuale Messunsicherheit des Mittelwertes von $L$ (d. h. der absolute bzw. prozentuale mittlere Fehler des Mittelwertes von $L$)?
        
        The absolute uncertainty in $L$ can be calculated using Gaussian error propagation:
        \begin{align*}
            \Delta L &= \sqrt{\left(\dfrac{\partial L}{\partial l} \Delta l\right)^2 + \left(\dfrac{\partial L}{\partial r} \Delta r\right)^2 + \left(\dfrac{\partial L}{\partial a} \Delta a\right)^2}
            =\\&= \sqrt{\left(\dfrac{\mu_0}{\pi} \ln\left(\dfrac{a - r}{r}+ \dfrac{1}{4}\right) \Delta l\right)^2 + \left(\dfrac{\mu_0}{\pi} l \cdot \dfrac{1}{\dfrac{a}{r}-\dfrac{3}{4}} \cdot \dfrac{a}{r^2} \Delta r\right)^2 + \left(\dfrac{\mu_0}{\pi} l \cdot \dfrac{1}{\dfrac{a}{r}-\dfrac{3}{4}} \cdot \dfrac{1}{r} \Delta a\right)^2}
            =\\&= \sqrt{\left(\dfrac{\mu_0}{\pi} \ln\left(\dfrac{a}{r}- \dfrac{3}{4}\right) \Delta l\right)^2 + \left(\dfrac{\mu_0}{\pi} l \cdot \dfrac{a}{ar-\dfrac{3r^2}{4}} \Delta r\right)^2 + \left(\dfrac{\mu_0}{\pi} l \cdot \dfrac{1}{a-\dfrac{3r}{4}}\Delta a\right)^2}
            \approx\\&\approx \unit[2.95 \cdot 10^{-5}]{H}
            = \\&= \unit[0.0295]{mH}
        \end{align*}

        The percentage uncertainty in $L$ is then given by:
        \begin{equation*}
            \dfrac{\Delta L}{L} = \dfrac{\unit[0.0295]{mH}}{\unit[4.004498]{mH}} \approx 0.00737 \approx 0.737\%
        \end{equation*}
    \end{enumerate}
\end{ejercicio}


\begin{ejercicio}[Regressionsanalyse]
    Gegeben die folgende Wertetabelle:
    \begin{equation*}
        \begin{array}{c|c|c|c|c|c}
            t & -2 & -1 & 0 & 1 & 2 \\
            \hline
            y & 13.1 & -2 & -2.2 & -4 & -15.9
        \end{array}
    \end{equation*}
    Finden Sie $x_1$ und $x_2$ für die beste Kurve der Form $y = f(t) = x_1 + x_2 t^3$ zur Anpassung der Messdaten. Welchen Wert hat das zugehörige Bestimmtheitsmaß $R^2$?
\end{ejercicio}