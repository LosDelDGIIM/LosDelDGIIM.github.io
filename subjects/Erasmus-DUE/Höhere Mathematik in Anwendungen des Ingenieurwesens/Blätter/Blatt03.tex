\setcounter{section}{2}
\section{Rechnen mit physikalischen Größen}


\begin{ejercicio}[Fährverbindung]
    Eine Fähre bewegt sich mit der Eigengeschwindigkeit $v_0=\unitfrac[4]{m}{s}$ (relativ zum Fluss) vom Uferpunkt $A$ aus auf kürzestem Weg zum gegenüberliegenden Flussufer (Punkt $B$; Abb.~\ref{fig:ferry}).
    \begin{figure}
        \centering
        \begin{tikzpicture}
            % Coordenadas de los puntos
            \coordinate (A) at (0,-0.2);
            \coordinate (B) at (0,2);
            \coordinate (O) at (0,0.25);
            
            % Ejes y etiquetas
            \draw[thick,-Stealth] (A) -- (5,-0.2) node[right] {$x$};
            \draw[thick,-Stealth] (A) -- (0,3.5) node[above] {$y$};
            
            % Vectores
            \draw[thick,-Stealth] (O) -- (1, 0.25) node[midway, above right] {$\vec{v_s}$};
            \draw[thick,-Stealth] (O) -- (-1.5,1.5) node[midway, left] {$\vec{v_0}$};
            \draw[thick,-Stealth] (O) -- (0,1.5) node[midway, right] {$\vec{v_r}$};
            
            % Ángulo
            \draw (O) -- (0,0.7) arc (90:115:1) node[midway, above] {$\alpha$};
            
            % Líneas de "Ufer" (orillas)
            \draw[pattern=north east lines, pattern color=gray] (-2,-1.2) rectangle (4.5,-0.2);
            \draw[pattern=north east lines, pattern color=gray] (-2,2) rectangle (4.5,3);
            
            % Etiquetas
            \node at (-0.7,0.25) {};
            \draw[fill=black] (O) circle (0.1) node[left, xshift=-0.25em] {Fähre};
            \node at (2,2.5) {Ufer};
            \node at (2,0.5) {Fluss};
            \node at (2,-0.7) {Ufer};
            \draw[fill=black] (A) circle (0.05) node[below left] {A};
            \draw[fill=black] (B) circle (0.05) node[above left] {B};
            \node at (3,1.5) {Strömung};
            
            % Flecha de Strömung
            \draw[thick,-Stealth] (2.5,1) -- (3.5,1);
        \end{tikzpicture}
        \caption{Fährverbindung über einen Fluss mit Strömung.}
        \label{fig:ferry}
    \end{figure}
    \begin{enumerate}
        \item Unter welchem Winkel $\alpha$ muss die Fähre gegen die Strömung gesteuert werden, wenn die Geschwindigkeit der Strömung den Betrag $v_s=\unitfrac[1.5]{m}{s}$ hat?
        
        The Ferry wants to get to $B$ in an straight line. Therefore, it must compensate the flow velocity by steering at an angle $\alpha$ against the flow. It is then needed that:
        \begin{equation*}
            v_s = v_0 \sin(\alpha) \implies \alpha = \arcsin\left(\frac{v_s}{v_0}\right) = \arcsin\left(\frac{1.5}{4}\right) \approx \unit[0.3844]{rad}
            \approx \unit[22.024]{^\circ}
        \end{equation*}

        \item Wie groß ist dann die resultierende Geschwindigkeit $v_r$ der Fähre?
        
        The resulting velocity $v_r$ of the ferry is given by:
        \begin{equation*}
            v_r = v_0\cos(\alpha) = 4 \cos(0.3844) \approx \unitfrac[3.7081]{m}{s}
        \end{equation*}
    \end{enumerate}
\end{ejercicio}


\begin{ejercicio}[Elektrische Punktladungen im Koordinatensystem]
    Eine Punktladung $q_1=\unit[6.0]{\mu C}$ befindet sich in einem kartesischen Koordinatensystem bei $x_1=\unit[1.0]{m}$, $y_1=\unit[0.5]{m}$. Eine zweite Ladung $q_2=\unit[-2.5]{\mu C}$ befindet sich in dessen Ursprung. Ein Elektron, d.h. eine dritte Punktladung, ist in einem Punkt mit den Koordinaten $(x_e,y_e)$. Berechnen Sie die Werte für $x_e$ und $y_e$, bei denen sich das Elektron im Gleichgewicht befindet, d.h. bei dem die Gesamtkraft auf das Elektron verschwindet.
    \begin{figure}
        \centering
        \begin{tikzpicture}
            % Ejes
            \draw[thick,-Stealth] (-1,0) -- (4,0) node[right] {$x$};
            \draw[thick,-Stealth] (0,-1) -- (0,4) node[above] {$y$};
            
            % Cargas
            \draw[fill=black] (0,0) circle (0.1) node[below left] {$q_2=-2.5\,\mu C$};
            \draw[fill=black] (1,0.5) circle (0.1) node[above right] {$q_1=6.0\,\mu C$};
            \draw[fill=black] (2.5,2) circle (0.1) node[above right] {$e$\ Elektron};
        \end{tikzpicture}
        \caption{Punktladungen im Koordinatensystem.}
        \label{fig:point_charges}
    \end{figure}

    The forces acting on the electron due to the other two charges must cancel each other out for equilibrium. Let $F_{ei}$ be the force on the electron due to charge $q_i$. The forces can be expressed as:
    \begin{equation*}
        \vec{F}_{e1} + \vec{F}_{e2} = 0
    \end{equation*}
    where, using that $q_e=-e<0$ (the charge of the electron):
    \begin{equation*}
        \vec{F}_{e1} = -k_e \frac{|q_1 e|}{r_{e1}^2} \ \hat{r}_{e1}, \quad \vec{F}_{e2} = k_e \frac{|q_2 e|}{r_{e2}^2} \ \hat{r}_{e2}
    \end{equation*}
    Here, $k_e$ is Coulomb's constant, $e$ is the elementary charge, $r_{ei}$ is the distance between the electron and charge $q_i$, and $\hat{r}_{ei}$ is the unit vector pointing from charge $q_i$ to the electron. Using the values of $r_{e1}$ and $r_{e2}$ based on the coordinates of the charges and the electron, we have:
    \begin{equation*}
        \vec{r}_{e1} = (x_e - 1, y_e - 0.5), \quad \vec{r}_{e2} = (x_e, y_e)
    \end{equation*}

    Therefore, the equilibrium condition becomes:
    \begin{equation*}
        -k_e \frac{|q_1 e|}{((x_e - 1)^2 + (y_e - 0.5)^2)} \cdot \frac{(x_e - 1, y_e - 0.5)}{\sqrt{(x_e - 1)^2 + (y_e - 0.5)^2}} + k_e \frac{|q_2 e|}{(x_e^2 + y_e^2)} \cdot \frac{(x_e, y_e)}{\sqrt{x_e^2 + y_e^2}} = 0
    \end{equation*}

    This vector equation can be separated into its $x$ and $y$ components, leading to a system of two equations with two unknowns ($x_e$ and $y_e$). Solving this system will yield the coordinates of the electron in equilibrium.
    \begin{align*}
        & -k_e \frac{|q_1 e| (x_e - 1)}{((x_e - 1)^2 + (y_e - 0.5)^2)^{3/2}} + k_e \frac{|q_2 e| x_e}{(x_e^2 + y_e^2)^{3/2}} = 0 \\
        & -k_e \frac{|q_1 e| (y_e - 0.5)}{((x_e - 1)^2 + (y_e - 0.5)^2)^{3/2}} + k_e \frac{|q_2 e| y_e}{(x_e^2 + y_e^2)^{3/2}} = 0
    \end{align*}

    In an easier way:
    \begin{align*}
        \frac{|q_1| (x_e - 1)}{((x_e - 1)^2 + (y_e - 0.5)^2)^{3/2}} &= \frac{|q_2| x_e}{(x_e^2 + y_e^2)^{3/2}}\\
        \frac{|q_1| (y_e - 0.5)}{((x_e - 1)^2 + (y_e - 0.5)^2)^{3/2}} &= \frac{|q_2| y_e}{(x_e^2 + y_e^2)^{3/2}}
    \end{align*}

    % // TODO: Cont
    
\end{ejercicio}


\begin{ejercicio}[Die magnetische Kraft]
    Ein punktförmiges Teilchen mit einer Ladung $q=\unit[-3.64]{nC}$ bewegt sich mit einer Geschwindigkeit $\vec{v}=\unitfrac[2.75 \cdot 10^{6}]{m}{s} \ \vec{e_x}$, d.h. entlang der x-Achse. Berechnen Sie die Kraft, die folgende Felder auf das Teilchen ausüben:
    \begin{enumerate}
        \item $\vec{B}=\unit[0.38]{T} \ \vec{e_y}$
        \begin{align*}
            \vec{F} &= q\cdot (\vec{v} \times \vec{B}) = q\cdot \begin{vmatrix}
                \vec{e_x} & \vec{e_y} & \vec{e_z} \\
                v_x & 0 & 0 \\
                0 & B_y & 0
            \end{vmatrix} = q \cdot v_x \cdot B_y\cdot \vec{e_z} =\\&= -3.64 \cdot 10^{-9} \cdot 2.75 \cdot 10^{6} \cdot 0.38 \ \vec{e_z} = \unit[-3.8038 \cdot 10^{-3}]{N}\ \vec{e_z}
        \end{align*}
        \item $\vec{B}=\unit[T]{0.75} \ \vec{e_x} + \unit[T]{0.75} \ \vec{e_y}$
        \begin{align*}
            \vec{F} &= q\cdot (\vec{v} \times \vec{B}) = q\cdot \begin{vmatrix}
                \vec{e_x} & \vec{e_y} & \vec{e_z} \\
                v_x & 0 & 0 \\
                B_x & B_y & 0
            \end{vmatrix} = q \cdot v_x \cdot B_y\cdot \vec{e_z} =\\&= -3.64 \cdot 10^{-9} \cdot 2.75 \cdot 10^{6} \cdot 0.75 \ \vec{e_z} = \unit[-7.5075 \cdot 10^{-3}]{N}\ \vec{e_z}
        \end{align*}
    \end{enumerate}
\end{ejercicio}


