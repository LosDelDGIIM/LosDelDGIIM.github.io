\section{Kurvenintegrale}

\begin{ejercicio}[Kurvenintegral erster Art]
Berechnen Sie das Kurvenintegral erster Art $\displaystyle \int_{\varphi_1} f \, dt$ für die skalarwertige Funktion
\Func{f}{\bb{R}^3}{\bb{R}}{(x,y,z)}{\sqrt{x^2+y^2+z^2}}
über die parametrisierte Kurve
\Func{\varphi_1}{[0,2\pi]}{\bb{R}^3}{t}{(t\cos(t), t\sin(t), t)}

    First of all, we compute the derivative of the parameterization:
    \begin{align*}
        \varphi_1'(t) &= \begin{pmatrix}
            \cos(t) - t\sin(t) \\
            \sin(t) + t\cos(t) \\
            1
        \end{pmatrix}
    \end{align*}

    We then realize that $\phi$ is differentiable, and given that the third component is constant and non-zero, we can conclude that the parameterization is regular. Therefore, in order to compute the norm of the derivative, we have:
    \begin{align*}
        \|\varphi_1'(t)\| &= \sqrt{(\cos(t) - t\sin(t))^2 + (\sin(t) + t\cos(t))^2 + 1^2} \\
        &= \sqrt{\cos^2(t) - \cancel{2t\sin(t)\cos(t)} + t^2\sin^2(t) + \sin^2(t) + \cancel{2t\sin(t)\cos(t)} + t^2\cos^2(t) + 1} \\
        &= \sqrt{1 + t^2 + 1}
        = \sqrt{t^2 + 2}
    \end{align*}

    Therefore, we can now compute the curve integral:
    \begin{align*}
        \int_{\varphi_1} f \, dt &= \int_0^{2\pi} f(\varphi_1(t)) \|\varphi_1'(t)\| \, dt \\
        &= \int_0^{2\pi} \sqrt{t^2 + t^2}
        \cdot
        \sqrt{t^2 + 2} \, dt \\
        &= \sqrt{2} \int_0^{2\pi} t \sqrt{t^2 + 2}\, dt
    \end{align*}

    We use the substitution given by:
    \Func{\Phi}{[2,4\pi^2+2]}{[0,2\pi]}{u}{\sqrt{u-2}}

    Thus, using $t=\Phi(u)$, we have:
    \begin{align*}
        \int_{\varphi_1} f \, dt &= \sqrt{2} \int_{\Phi^{-1}(0)}^{\Phi^{-1}(2\pi)} \Phi(u) \sqrt{\Phi(u)^2 + 2} \cdot \Phi'(u) \, du \\
        &= \sqrt{2} \int_{2}^{4\pi^2 + 2} \sqrt{u-2} \sqrt{u} \cdot \frac{1}{2\sqrt{u-2}} \, du
        = \frac{\sqrt{2}}{2} \int_{2}^{4\pi^2 + 2} \sqrt{u} \, du
        = \frac{\sqrt{2}}{2} \left[ \frac{2}{3} u^{\nicefrac{3}{2}} \right]_{2}^{4\pi^2 + 2}
        =\\&= 
        \frac{\sqrt{2}}{3} \left[ u^{\nicefrac{3}{2}} \right]_{2}^{4\pi^2 + 2}
        = \frac{\sqrt{2}}{3} \left( (4\pi^2 + 2)^{\nicefrac{3}{2}} - 2^{\nicefrac{3}{2}} \right)
    \end{align*}
\end{ejercicio}

\begin{ejercicio}[Kurvenintegral zweiter Art]
Berechnen Sie das Kurvenintegral zweiter Art $\displaystyle \int_{\varphi_2} \vec{f} \, d\vec{s}$ für das Vektorfeld
\Func{\vec{f}}{\bb{R}^3}{\bb{R}^3}{(x,y,z)}{(x-z, y^2, 1)}
über die parametrisierte Kurve
\Func{\varphi_2}{[0,1]}{\bb{R}^3}{t}{\left(t^2, t, e^t\right)}

    We first compute the derivative of the parameterization:
    \begin{align*}
        \varphi_2'(t) &= (2t, 1, e^t)
    \end{align*}

    We then realize that $\phi$ is differentiable, and given that the third component is non-zero, we can conclude that the parameterization is regular. Then, the dot product $\vec{f}(\varphi_2(t)) \cdot \varphi_2'(t)$ is given by:
    \begin{align*}
        \vec{f}(\varphi_2(t)) \cdot \varphi_2'(t) &= (t^2 - e^t, t^2, 1) \cdot (2t, 1, e^t) \\
        &= 2t^3 - 2t e^t + t^2 + e^t
    \end{align*}

    Therefore, we can now compute the curve integral:
    \begin{align*}
        \int_{\varphi_2} \vec{f} \, d\vec{s} &= \int_0^1 \vec{f}(\varphi_2(t)) \cdot \varphi_2'(t) \, dt \\
        &= \int_0^1 2t^3 - 2t e^t + t^2 + e^t \, dt
    \end{align*}

    Let's firstly compute the integral of $t e^t$ using integration by parts:
    \begin{equation*}
        \MetInt{u(t)=t\qquad u'(t)=1}{v(t)=e^t\qquad v'(t)=e^t}
    \end{equation*}
    \begin{align*}
        \int_0^1 t e^t \, dt &= \left[ t e^t \right]_0^1 - \int_0^1 e^t \, dt
    \end{align*}

    Now, we can compute the entire integral:
    \begin{align*}
        \int_{\varphi_2} \vec{f} \, d\vec{s} &= \left[2 \cdot \frac{t^4}{4} - 2 \left( t e^t - e^t \right) + \frac{t^3}{3} + e^t \right]_0^1
        = \left[\frac{t^4}{2} - 2te^t +3e^t + \frac{t^3}{3}\right]_0^1
        =\\&= \dfrac{1}{2} - 2e + 3e + \frac{1}{3} - 3
        = -\dfrac{2}{3} + e
    \end{align*}
\end{ejercicio}

\begin{ejercicio}[Länge einer Kurve]
Berechnen Sie die Länge der Kurve $C$ mit der Parameterdarstellung
\Func{\varphi}{[-1,1]}{\bb{R}^2}{t}{(x(t), y(t))}
\Func{x}{[-1,1]}{\bb{R}}{t}{t^3}
\Func{y}{[-1,1]}{\bb{R}}{t}{1 - t^2}

\begin{observacion}
Man denke an die Voraussetzung zur Längenberechnung mittels Kurvenintegralen, dass die Parameterdarstellung regulär sein muss, d.h. $\varphi'(t) \neq \vec{0}$. Ist dies nicht für alle Punkte der Kurve der Fall, berechnet man die Länge abschnittsweise.
\end{observacion}

    Given that $x(t)$ is inyective, we realize that $\varphi$ is also inyective. Let's compute the derivative of the parameterization:
    \begin{align*}
        \varphi'(t) &= (3t^2, -2t)
    \end{align*}

    We realize that $\varphi$ is differentiable, and that $\varphi'(t) = \vec{0}$ only at $t=0$. Therefore, we will compute the length of the curve in two parts: in $[-1,0]$ and in $[0,1]$. Thus, we compute the norm of the derivative:
    \begin{align*}
        \|\varphi'(t)\| &= \sqrt{(3t^2)^2 + (-2t)^2} = \sqrt{9t^4 + 4t^2} = \sqrt{t^2(9t^2 + 4)} = |t| \sqrt{9t^2 + 4}
    \end{align*}

    Therefore, we can now compute the length of the curve:
    \begin{align*}
        L &= \int_{-1}^{0} \|\varphi'(t)\| \, dt + \int_{0}^{1} \|\varphi'(t)\| \, dt \\
        &= \int_{-1}^{0} -t \sqrt{9t^2 + 4} \, dt + \int_{0}^{1} t \sqrt{9t^2 + 4} \, dt
    \end{align*}

    We use the substitution given by:
    \Func{\Phi}{\bb{R}}{\bb{R}}{t}{u=9t^2 + 4}

    In both intervals, $\Phi$ es biyective. Thus, using $t=\Phi^{-1}(u)$, we have:
    \begin{align*}
        L &= \int_{\Phi(-1)}^{\Phi(0)} -\Phi^{-1}(u) \sqrt{u} \cdot (\Phi^{-1})'(u) \, du + \int_{\Phi(0)}^{\Phi(1)} \Phi^{-1}(u) \sqrt{u} \cdot (\Phi^{-1})'(u) \, du \\
        &= \int_{13}^{4}-\Phi^{-1}(u) \sqrt{u} \cdot \dfrac{1}{2\Phi^{-1}(u)}\cdot \dfrac{1}{9}\, du + \int_{4}^{13} \Phi^{-1}(u) \sqrt{u} \cdot \dfrac{1}{2\Phi^{-1}(u)}\cdot \dfrac{1}{9} \, du
        =\\&= 2\cdot \dfrac{1}{18} \int_{4}^{13} \sqrt{u} \, du
        = \dfrac{1}{9} \left[ \dfrac{2}{3} u^{\nicefrac{3}{2}} \right]_{4}^{13}
        = \dfrac{2}{27} \left( 13^{\nicefrac{3}{2}} - 4^{\nicefrac{3}{2}} \right)
        = \dfrac{26\sqrt{13} - 16}{27}
    \end{align*}
\end{ejercicio}

\begin{ejercicio}[Flächeninhalt einer Ellipse]
Zeigen Sie, dass der Flächeninhalt $F$ einer Ellipse $E$, die durch die Gleichung
\[\frac{x^2}{a^2} + \frac{y^2}{b^2} = 1\]
beschrieben wird, $F = \pi a b$ ist.
\begin{observacion}
Verwenden Sie die Parameterdarstellung
\Func{\varphi}{[0,2\pi]}{\bb{R}^2}{t}{\left(a \cos t, b \sin t\right)}
\end{observacion}

    We first compute the derivative of the parameterization:
    \begin{align*}
        \varphi'(t) &= (-a \sin t, b \cos t)
    \end{align*}

    We then realize that $\phi$ is differentiable, and given that $a,b > 0$ and that both $\sin t$ and $\cos t$ cannot be zero at the same time, we can conclude that the parameterization is regular. Given that the curve goes around the ellipse counter-clockwise, we can use the following formula to compute the area:
    \begin{align*}
        F &= \dfrac{1}{2} \int_\varphi (-y,x)\cdot d\vec{x}
        = \dfrac{1}{2} \int_0^{2\pi} (-b \sin t, a \cos t) \cdot ( -a \sin t, b \cos t) \, dt \\
        &= \dfrac{1}{2} \int_0^{2\pi} ab \sin^2 t + ab \cos^2 t \, dt
        = \dfrac{ab}{2} \int_0^{2\pi} 1 \, dt
        = \dfrac{ab}{2} \cdot 2\pi
        = \pi ab
    \end{align*}
\end{ejercicio}