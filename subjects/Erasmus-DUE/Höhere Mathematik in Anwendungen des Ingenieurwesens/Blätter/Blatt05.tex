\setcounter{section}{4}
\section{Komplexe Rechnung in physikalisch-technischen Zusammenhängen}


\begin{ejercicio}

    Zeigen Sie durch Rechnung in komplexer Form: Drei gleichfrequente, sinusförmige Wechselströme $i_1$, $i_2$ und $i_3$ mit den Amplituden $i_0$, der Kreisfrequenz $\omega > 0$ und den Nullphasenwinkeln $\varphi_1 = 0$, $\varphi_2 = \nicefrac{2}{3} \pi$ und $\varphi_3 = \nicefrac{4}{3} \pi$ löschen sich bei ungestörter Überlagerung gegenseitig aus, d.h. $i_1 + i_2 + i_3 = 0$.\\

    Let $\ul{i}_1(t)$, $\ul{i}_2(t)$ und $\ul{i}_3(t)$ be:
    \begin{align*}
        \ul{i}_1(t) &= i_0 \cdot e^{\jmath(\omega t + 0)} = i_0 \cdot e^{\jmath \omega t} \\
        \ul{i}_2(t) &= i_0 \cdot e^{\jmath(\omega t + \nicefrac{2}{3} \pi)} = i_0 \cdot e^{\jmath \omega t} \cdot e^{\jmath \nicefrac{2}{3} \pi} \\
        \ul{i}_3(t) &= i_0 \cdot e^{\jmath(\omega t + \nicefrac{4}{3} \pi)} = i_0 \cdot e^{\jmath \omega t} \cdot e^{\jmath \nicefrac{4}{3} \pi}
    \end{align*}

    Therefore:
    \begin{align*}
        \ul{i}_1(t) + \ul{i}_2(t) + \ul{i}_3(t) &= i_0 \cdot e^{\jmath \omega t} \left(e^{\jmath 0} + e^{\jmath \nicefrac{2}{3} \pi} + e^{\jmath \nicefrac{4}{3} \pi}\right)
        =\\&= i_0 \cdot e^{\jmath \omega t} \left(1 + \left(-\frac{1}{2} + \jmath \frac{\sqrt{3}}{2}\right) + \left(-\frac{1}{2} - \jmath \frac{\sqrt{3}}{2}\right)\right)
        =\\&= i_0 \cdot e^{\jmath \omega t} \cdot 0 = 0
    \end{align*}

    Therefore, we have shown that:
    \[
        i_1(t) + i_2(t) + i_3(t) = \Im\left(\ul{i}_1(t) + \ul{i}_2(t) + \ul{i}_3(t)\right) = \Im(0) = 0
    \]
\end{ejercicio}