\setcounter{section}{4}
\section{Komplexe Rechnung in physikalisch-technischen Zusammenhängen}


\begin{ejercicio}

    Zeigen Sie durch Rechnung in komplexer Form: Drei gleichfrequente, sinusförmige Wechselströme $i_1$, $i_2$ und $i_3$ mit den Amplituden $i_0$, der Kreisfrequenz $\omega > 0$ und den Nullphasenwinkeln $\varphi_1 = 0$, $\varphi_2 = \nicefrac{2}{3} \pi$ und $\varphi_3 = \nicefrac{4}{3} \pi$ löschen sich bei ungestörter Überlagerung gegenseitig aus, d.h. $i_1 + i_2 + i_3 = 0$.\\

    Let $\ul{i}_1(t)$, $\ul{i}_2(t)$ und $\ul{i}_3(t)$ be:
    \begin{align*}
        \ul{i}_1(t) &= i_0 \cdot e^{\jmath(\omega t + 0)} = i_0 \cdot e^{\jmath \omega t} \\
        \ul{i}_2(t) &= i_0 \cdot e^{\jmath(\omega t + \nicefrac{2}{3} \pi)} = i_0 \cdot e^{\jmath \omega t} \cdot e^{\jmath \nicefrac{2}{3} \pi} \\
        \ul{i}_3(t) &= i_0 \cdot e^{\jmath(\omega t + \nicefrac{4}{3} \pi)} = i_0 \cdot e^{\jmath \omega t} \cdot e^{\jmath \nicefrac{4}{3} \pi}
    \end{align*}

    Therefore:
    \begin{align*}
        \ul{i}_1(t) + \ul{i}_2(t) + \ul{i}_3(t) &= i_0 \cdot e^{\jmath \omega t} \left(e^{\jmath 0} + e^{\jmath \nicefrac{2}{3} \pi} + e^{\jmath \nicefrac{4}{3} \pi}\right)
        =\\&= i_0 \cdot e^{\jmath \omega t} \left(1 + \left(-\frac{1}{2} + \jmath \frac{\sqrt{3}}{2}\right) + \left(-\frac{1}{2} - \jmath \frac{\sqrt{3}}{2}\right)\right)
        =\\&= i_0 \cdot e^{\jmath \omega t} \cdot 0 = 0
    \end{align*}

    Therefore, we have shown that:
    \[
        i_1(t) + i_2(t) + i_3(t) = \Im\left(\ul{i}_1(t) + \ul{i}_2(t) + \ul{i}_3(t)\right) = \Im(0) = 0
    \]
\end{ejercicio}

\begin{ejercicio}[Resonanz im Parallelschwingkreis]\label{ej:1.5.2}
    \begin{figure}
        \centering
        \shorthandoff{"<>}
        \begin{tikzpicture}
            % Paths, nodes and wires:
            \draw (8.875, 6.125) to[ammeter, l={$I$}] (8.875, 4.875);
            \node[ocirc] at (4, 4){};
            \node[ocirc](N1) at (4.75, 4){} node[anchor=north east] at (N1.south west){$U$};
            \draw (5.995, 4) to[closing switch, l_={$S$}] (7.495, 4);
            \draw (5.5, 8.25) to[american resistor, l={$R$}] (7.5, 8.25);
            \draw (5.625, 7.504) to[cute inductor, l_={$L$}] (7.375, 7.504);
            \draw (6, 9.5) to[capacitor, l={$C$}] (7, 9.5);
            \node[shape=rectangle, minimum width=0.965cm, minimum height=0.465cm] at (7.25, 4.5){} node[anchor=north west, align=left, text width=2cm, inner sep=6pt] at (6.75, 4.75){\small $t=0$};
            \node[shape=rectangle, rotate=-90, draw, line width=1pt, dash pattern={on 4pt off 4pt}, minimum width=1.09cm, minimum height=0.34cm](N2) at (2.063, 5.562){} node[anchor=east] at (N2.south){$R_\ell$};
            \draw[-latex] (3.5, 9.5) -- (4.25, 9.5);
            \draw[-latex] (3.5, 8.25) -- (4.25, 8.25);
            \draw[-latex] (3.5, 7.5) -- (4.25, 7.5);
            \node[circ] at (8, 8.25){};
            \node[circ] at (3, 8.25){};
            \draw (3.5, 9.5) -- (3, 9.5) -| (3, 8.25);
            \draw (3.5, 8.25) -- (3, 8.25);
            \draw (3.5, 7.5) |- (3, 7.5) -| (3, 8.25);
            \draw (4.25, 9.5) -- (6, 9.5);
            \draw (5.5, 8.25) -- (4.25, 8.25);
            \draw (7, 9.5) -| (8, 8.25);
            \draw (7.5, 8.25) -- (8, 8.25);
            \draw (8, 8.25) -| (8.875, 6.125);
            \draw (8.875, 4.875) |- (7.495, 4);
            \draw (5.995, 4) |- (4.75, 4);
            \draw (4, 4) -| (2.063, 5);
            \draw (2.063, 6.125) |- (3, 8.25);
            \draw (2.063, 5) -- (2.063, 6.125);
            \draw (4.25, 7.5) -| (5.625, 7.493);
            \draw (7.375, 7.504) -| (8, 8.25);
            \node[shape=rectangle, minimum width=0.715cm, minimum height=0.465cm] at (3.875, 9.75){} node[anchor=north west, align=left, text width=0.327cm, inner sep=6pt] at (3.5, 10.2){\footnotesize $I_C$};
            \node[shape=rectangle, minimum width=0.715cm, minimum height=0.465cm] at (3.875, 8.5){} node[anchor=north west, align=left, text width=0.327cm, inner sep=6pt] at (3.5, 8.9){\footnotesize $I_R$};
            \node[shape=rectangle, minimum width=0.715cm, minimum height=0.465cm] at (3.875, 7.25){} node[anchor=north west, align=left, text width=0.327cm, inner sep=6pt] at (3.5, 7.5){\footnotesize $I_L$};
        \end{tikzpicture}
        \shorthandon{"<>}
        \caption{Skizze zur Aufgabe~\ref{ej:1.5.2}}
        \label{fig:parallelschwingkreis}
    \end{figure}
    Der in Abbildung~\ref{fig:parallelschwingkreis} skizzierte Parallelschwingkreis mit dem ohmschen Widerstand $R = \unit[10]{\Omega}$, der Induktivität $L = \unit[0.2]{H}$ und der Kapazität $C = \unit[10]{mF}$ wird durch eine Wechselstromquelle mit dem Effektivwert $I = \unit[10]{A}$ und der variablen Kreisfrequenz $\omega$ zu elektromagnetischen Schwingungen angeregt.
    \begin{enumerate}
        \item Im Resonanzfall sind der Gesamtstrom $I$ und die angelegte Spannung $U$ in Phase, d.h. $\varphi_I - \varphi_U = 0$. Bei welcher Kreisfrequenz $\omega_0$ tritt dieser Fall ein? Wie groß ist dann der komplexe Gesamtwiderstand $Z$?
        
        The complex resistance is:
        \begin{align*}
            \ul{Z} = \dfrac{\ul{u}(t)}{\ul{i}(t)}
            = \dfrac{\sqrt{2}U e^{\jmath(\omega t + \varphi_U)}}{\sqrt{2}I e^{\jmath(\omega t + \varphi_I)}}
            = \dfrac{U}{I} e^{\jmath(\varphi_U - \varphi_I)}
            \AstIg \dfrac{U}{I} e^{\jmath 0}
            = \dfrac{U}{I} \in \mathbb{R}
        \end{align*}
        where in $(\ast)$ we used the fact that in resonance $\varphi_U = \varphi_I$.\\

        We can calculate the complex resistance as:
        \begin{align*}
            \dfrac{1}{\ul{Z}} &= \dfrac{1}{\ul{Z}_R} + \dfrac{1}{\ul{Z}_L} + \dfrac{1}{\ul{Z}_C} \\
            &= \dfrac{1}{R} + \dfrac{1}{\jmath \omega L} + \jmath \omega C \\
            &\Longrightarrow
            \ul{Z} = \dfrac{1}{\dfrac{1}{R} - \jmath \dfrac{1}{\omega L} + \jmath \omega C}
            = \dfrac{1}{\dfrac{1}{R} + \jmath \left(-\dfrac{1}{\omega L} + \omega C\right)}
        \end{align*}

        Given that $\ul{Z}\in \mathbb{R}$, we have that the imaginary part must be zero:
        \begin{align*}
            -\dfrac{1}{\omega_0 L} + \omega_0 C = 0
            \Longrightarrow& \omega_0C = \dfrac{1}{\omega_0 L}
            \Longrightarrow \omega_0^2 = \dfrac{1}{LC}
            \Longrightarrow\\\Longrightarrow & \omega_0 = \sqrt{\dfrac{1}{LC}}
            = \sqrt{\dfrac{1}{0.2 \cdot 10 \cdot 10^{-3}}}
            = \sqrt{500} \approx \unitfrac[22.36]{rad}{s}
        \end{align*}

        Therefore, we have that:
        \begin{equation*}
            \ul{Z} = Z = R = \unit[10]{\Omega}
        \end{equation*}
        

        \item Wie ändert sich die Situation, wenn die Zuleitungen nicht mehr als ideal angenommen werden und einen ohmschen Widerstand $R_\ell$ beitragen? Welche Werte haben $\omega_0$ und $Z$ dann?
        
        In this case, we have:
        \begin{align*}
            \ul{Z} = \dfrac{1}{\dfrac{1}{R} + \jmath \left(-\dfrac{1}{\omega L} + \omega C\right)}  + R_{\ell}
        \end{align*}

        Therefore, $\omega_0$ does not change, but:
        \begin{equation*}
            \ul{Z} = Z = R + R_{\ell} = 10 + R_{\ell}
        \end{equation*}
    \end{enumerate}
\end{ejercicio}

\begin{ejercicio}[Wechselstrommessbrücke]\label{ej:1.5.3}
    \begin{figure}
        \centering
        \begin{tikzpicture}
            % Paths, nodes and wires:
            \draw (4, 6) to[sinusoidal voltage source] (4, 4.5);
            \draw[-latex] (3.25, 6) -- (3.25, 4.5);
            \node[shape=rectangle, minimum width=0.715cm, minimum height=0.715cm] at (2.875, 5.25){} node[anchor=north west, align=left, text width=0.327cm, inner sep=6pt] at (2.5, 5.625){$\underline{U}$};
            \draw (4, 3.5) to[closing switch, l_={$S$}] (4, 2);
            \draw (6.25, 7.5) to[european resistor, l_={$\underline{Z}_1=\underline{Z}_X$}] (6.25, 5.25);
            \draw (6.25, 4.25) to[european resistor, l_={$\underline{Z}_2$}] (6.25, 1);
            \draw (9, 7.5) to[european resistor, l={$\underline{Z}_3$}] (9, 5.25);
            \draw (9, 4.25) to[variable european resistor, invert, l={$\underline{Z}_4 \text{(variable)}$}] (9, 1);
            \draw (6.25, 4.5) to[ammeter, l_={$\underline{Z}_5$}, label distance=-0.24cm] (9, 4.5);
            \node[circ] at (6.25, 4.5){};
            \node[circ] at (9, 4.5){};
            \node[circ] at (6.25, 7.5){};
            \node[circ] at (6.25, 1){};
            \draw[-latex] (4, 7.5) -- (5.25, 7.5);
            \draw[-latex] (6.25, 7.5) -- (6.25, 7);
            \draw[-latex] (6.25, 7.5) -- (7.25, 7.5);
            \draw[-latex] (9, 7.5) -- (9, 7);
            \draw[-latex] (6.25, 4.5) -- (7, 4.5);
            \draw[-latex] (6.25, 2) -- (6.25, 1.25);
            \draw[-latex] (9, 2) -- (9, 1.25);
            \draw[-latex] (6.25, 1) -- (5, 1);
            \draw (7.25, 7.5) -- (9, 7.5);
            \draw (9, 5.25) -| (9, 4.5);
            \draw (6.25, 5.25) -| (6.25, 4.5);
            \draw (6.25, 4.5) -| (6.25, 4.25);
            \draw (9, 4.5) -| (9, 4.25);
            \draw (9, 1) -- (6.25, 1);
            \draw (5, 1) |- (4, 1) -| (4, 2);
            \draw (4, 3.5) -| (4, 4.5);
            \node[shape=rectangle, minimum width=0.715cm, minimum height=0.715cm] at (5.25, 7.875){} node[anchor=north west, align=left, text width=0.327cm, inner sep=6pt] at (4.875, 8.25){$\underline{I}$};
            \node[shape=rectangle, minimum width=0.715cm, minimum height=0.715cm] at (7.125, 7.875){} node[anchor=north west, align=left, text width=0.327cm, inner sep=6pt] at (6.75, 8.25){$\underline{I}_3$};
            \node[shape=rectangle, minimum width=0.715cm, minimum height=0.715cm] at (9.25, 7.25){} node[anchor=north west, align=left, text width=0.327cm, inner sep=6pt] at (8.875, 7.625){$\underline{I}_3$};
            \node[shape=rectangle, minimum width=0.715cm, minimum height=0.715cm] at (6.5, 7.25){} node[anchor=north west, align=left, text width=0.327cm, inner sep=6pt] at (6.125, 7.625){$\underline{I}_1$};
            \node[shape=rectangle, minimum width=0.715cm, minimum height=0.715cm] at (6.75, 4.25){} node[anchor=north west, align=left, text width=0.327cm, inner sep=6pt] at (6.375, 4.625){$\underline{I}_5$};
            \node[shape=rectangle, minimum width=0.715cm, minimum height=0.715cm] at (6.5, 1.625){} node[anchor=north west, align=left, text width=0.327cm, inner sep=6pt] at (6.125, 2){$\underline{I}_2$};
            \node[shape=rectangle, minimum width=0.715cm, minimum height=0.715cm] at (9.25, 1.625){} node[anchor=north west, align=left, text width=0.327cm, inner sep=6pt] at (8.875, 2){$\underline{I}_4$};
            \node[shape=rectangle, minimum width=0.715cm, minimum height=0.715cm] at (5.25, 1.25){} node[anchor=north west, align=left, text width=0.327cm, inner sep=6pt] at (4.875, 1.625){$\underline{I}$};
            \node[shape=rectangle, minimum width=0.715cm, minimum height=0.715cm] at (4.5, 2.625){} node[anchor=north west, align=left, text width=0.327cm, inner sep=6pt] at (4.125, 3){\footnotesize $t=0$};
            \draw[-latex] (7.5, 6.5) -- (7, 6) -- (7.5, 5.5) -- (8, 6) -- (7.5, 6.5);
            \draw[-latex] (7.5, 3.053) -- (7, 2.553) -- (7.5, 2.053) -- (8, 2.553) -- (7.5, 3.053);
            \node[shape=rectangle, minimum width=0.715cm, minimum height=0.715cm] at (7.607, 6.018){} node[anchor=north west, align=left, text width=0.327cm, inner sep=6pt] at (7.232, 6.393){I};
            \node[shape=rectangle, minimum width=0.715cm, minimum height=0.715cm] at (7.536, 2.535){} node[anchor=north west, align=left, text width=0.327cm, inner sep=6pt] at (7.161, 2.91){II};
            \node[shape=rectangle, minimum width=0.715cm, minimum height=0.715cm] at (6, 4.5){} node[anchor=north west, align=left, text width=0.327cm, inner sep=6pt] at (5.625, 4.875){A};
            \node[shape=rectangle, minimum width=0.715cm, minimum height=0.715cm] at (9.375, 4.5){} node[anchor=north west, align=left, text width=0.327cm, inner sep=6pt] at (9, 4.875){B};
            \draw (4, 6) -- (4, 7.5);
            \draw (5.25, 7.5) -- (6.25, 7.5);
        \end{tikzpicture}
        \caption{Skizze zur Aufgabe~\ref{ej:1.5.3}.}
        \label{fig:1.5.3}
    \end{figure}

    Mit der in Abbildung~\ref{fig:1.5.3} dargestellten Brückenschaltung lässt sich ein unbekannter komplexer Widerstand $\ul{Z}_1 = \ul{Z}_X$ wie folgt bestimmen: Bei vorgegebenen (komplexen) Widerständen $\ul{Z}_2$ und $\ul{Z}_3$ wird der stetig veränderbare komplexe Widerstand $\ul{Z}_4$ so eingestellt, dass der Brückenzweig A–B stromlos wird. Das in die Brücke geschaltete Wechselstromamperemeter mit dem (bekannten) Innenwiderstand $\ul{Z}_5$ dient dabei lediglich als Nullindikator.
    \begin{enumerate}
        \item Wie lautet die sogenannte Abgleichbedingung, d.h. die Bedingung für die Stromlosigkeit des Brückenzweiges A-B?
        \item In einem konkreten Fall haben die festen Widerstände $\ul{Z}_2$ und $\ul{Z}_3$ folgende Werte:
        \begin{align*}
            \ul{Z}_2 &= 10\Omega - \jmath \cdot 2\Omega,\\
            \ul{Z}_3 &= 8\Omega - \jmath \cdot 6\Omega.
        \end{align*}
        Die Brücke A-B wird dabei genau dann stromlos, wenn der variable Widerstand $\ul{Z}_4$ auf den Wert
        \begin{equation*}
            \ul{Z}_4 = 5\Omega - \jmath \cdot 2\Omega
        \end{equation*}
        eingestellt wird. Welchen Wert besitzt dann der (zunächst noch unbekannte) Widerstand $\ul{Z}_X$?
    \end{enumerate}
    
\end{ejercicio}



\begin{ejercicio}[Wechselstromparadoxon]\label{ej:1.5.4}

    \begin{figure}
        \centering
        \begin{tikzpicture}
            % Paths, nodes and wires:
            \draw (0, 8) to[american voltage source] (0, 9.5);
            \draw[-latex] (0, 9.5) -- (0, 10.5);
            \draw[-latex] (-0.75, 9.5) -- (-0.75, 8);
            \draw (0.5, 11.75) to[american resistor, l={$R$}] (2.5, 11.75);
            \draw (2.5, 11.75) to[ammeter] (4.5, 11.75);
            \draw[-latex] (5.25, 11.75) -- (5.25, 10.75);
            \node[circ] at (5.25, 10.75){};
            \draw[-latex] (4.25, 10.75) -- (4.25, 10);
            \draw[-latex] (6.25, 10.75) -- (6.25, 10);
            \draw (4.25, 10) to[american resistor, l={$R_X$}] (4.25, 8.25);
            \draw (6.25, 10) to[capacitor, l={$C$}] (6.25, 8.25);
            \draw (4.25, 8.25) to[closing switch, l={$S$}] (4.25, 6.75);
            \node[circ] at (5.25, 6.5){};
            \draw[-latex] (3.75, 5.5) -- (2.5, 5.5);
            \draw (0, 10.5) -| (0, 11.75) -- (0.5, 11.75);
            \draw (4.5, 11.75) -- (5.25, 11.75);
            \draw (5.25, 10.75) -- (4.25, 10.75);
            \draw (5.25, 10.75) -- (6.25, 10.75);
            \draw (4.25, 6.75) -| (4.25, 6.5) -- (5.25, 6.5) -| (6.25, 8.25);
            \draw (3.75, 5.5) -| (5.25, 6.5);
            \draw (2.5, 5.5) -- (0, 5.5) -| (0, 8);
            \node[shape=rectangle, minimum width=0.715cm, minimum height=0.715cm] at (-1.086, 8.694){} node[anchor=north west, align=left, text width=0.327cm, inner sep=6pt] at (-1.461, 9.069){$\underline{U}$};
            \node[shape=rectangle, minimum width=0.715cm, minimum height=0.715cm] at (-0.204, 10.383){} node[anchor=north west, align=left, text width=0.327cm, inner sep=6pt] at (-0.579, 10.758){$\underline{I}$};
            \node[shape=rectangle, minimum width=0.715cm, minimum height=0.715cm] at (2.663, 5.819){} node[anchor=north west, align=left, text width=0.327cm, inner sep=6pt] at (2.288, 6.194){$\underline{I}$};
            \node[shape=rectangle, minimum width=0.715cm, minimum height=0.715cm] at (5.5, 11.125){} node[anchor=north west, align=left, text width=0.327cm, inner sep=6pt] at (5.125, 11.5){$\underline{I}$};
            \node[shape=rectangle, minimum width=0.715cm, minimum height=0.715cm] at (4, 10.375){} node[anchor=north west, align=left, text width=0.327cm, inner sep=6pt] at (3.625, 10.75){$\underline{I}_1$};
            \node[shape=rectangle, minimum width=0.715cm, minimum height=0.715cm] at (6.5, 10.375){} node[anchor=north west, align=left, text width=0.327cm, inner sep=6pt] at (6.125, 10.75){$\underline{I}_2$};
        \end{tikzpicture}
        \caption{Skizze zur Aufgabe~\ref{ej:1.5.4}.}
        \label{fig:1.5.4}
    \end{figure}

    Der in Bild~\ref{fig:1.5.4} dargestellte Wechselstromkreis enthält die ohmschen Widerstände $R$ und $R_x$ und einen zu $R_x$ parallel geschalteten Kondensator mit der Kapazität $C$. Beim Anlegen einer Wechselspannung $\ul{U}$ mit der Kreisfrequenz $\omega$ fließt der Gesamtstrom $\ul{I}$, dessen Effektivwert $I$ durch das zugeschaltete Wechselstrommessgerät $A$ gemessen wird. Der Innenwiderstand $R_i$ des Geräts sei im Widertstand $R$ bereits enthalten. Zeigen Sie: Der ohmsche Widerstand $R_x$ lässt sich so wählen, dass die Stromanzeige unabhängig ist von der Stellung des Schalters $S$ (geschlossen oder offen; sog Wechselstromparadoxon).\\

    We have two options:
    \begin{itemize}
        \item \ul{Switch open}: In this case, everything is connected in serial and $\ul{I}_1=0$. Therefore:
        \begin{align*}
            \ul{Z}_{\text{open}} = R + \dfrac{1}{\jmath \omega C} = R -\frac{1}{\omega C}\cdot \jmath
        \end{align*}

        \item \ul{Switch closed}:
        \begin{align*}
            \ul{Z}_{\text{closed}} &= R + \dfrac{1}{\dfrac{1}{R_X} + \jmath\omega C}
            = \left(R + \dfrac{\frac{1}{R_X}}{\frac{1}{R_X^2} + \omega^2C^2}\right) - \dfrac{\omega C}{\frac{1}{R_X^2} + \omega^2C^2}\cdot \jmath
            =\\&= \left(R + \dfrac{R_X}{1 + (R_X\omega C)^2}\right) - \dfrac{R_X^2\omega C}{1 + (R_X\omega C)^2}\cdot \jmath
            =\\&= \dfrac{R(1 + (R_X\omega C)^2) + R_X}{1 + (R_X\omega C)^2}- \dfrac{R_X^2\omega C}{1 + (R_X\omega C)^2}\cdot \jmath
        \end{align*}
    \end{itemize}

    In order that the current does not depend on the position of the switch, it is needed:
    \begin{align*}
        \wh{I}_{\text{closed}} = \wh{I}_{\text{open}}
        \Longrightarrow
        |\ul{I}_{\text{closed}}| = |\ul{I}_{\text{open}}|
        \Longrightarrow
        \left|\dfrac{\ul{U}_{\text{closed}}}{\ul{Z}_{\text{closed}}}\right|
        = \left|\dfrac{\ul{U}_{\text{open}}}{\ul{Z}_{\text{open}}}\right|
    \end{align*}

    Given that $\ul{U}_{\text{open}}= \ul{U}_{\text{closed}}$, we only need that:

    \begin{align*}
        &|\ul{Z}_{\text{closed}}| = |\ul{Z}_{\text{open}}|
        \Longrightarrow
        |\ul{Z}_{\text{closed}}|^2 = |\ul{Z}_{\text{open}}|^2
        \Longrightarrow\\
        \Longrightarrow& R^2 + \dfrac{1}{\omega^2C^2}
        =  \dfrac{(R(1 + (R_X\omega C)^2) + R_X)^2+ (R_X^2\omega C)^2}{(1 + (R_X\omega C)^2)^2}
        \Longrightarrow\\
        \Longrightarrow& (R(1 + (R_X\omega C)^2))^2 + \dfrac{(1 + (R_X\omega C)^2)^2}{\omega^2C^2}
        =  (R(1 + (R_X\omega C)^2))^2 + R_X^2 + 2R_X\ R(1 + (R_X\omega C)^2) + (R_X^2\omega C)^2
        \Longrightarrow\\
        \Longrightarrow& \dfrac{(1 + (R_X\omega C)^2)^2}{\omega^2C^2}
        =  R_X^2 + 2R_X\ R(1 + (R_X\omega C)^2) + (R_X^2\omega C)^2
        \Longrightarrow\\
        \Longrightarrow& 1 + 2(R_X\omega C)^2= (R_X\omega C)^2 + 2\omega^2C^2 R_X\ R(1 + (R_X\omega C)^2)
        \Longrightarrow\\
        \Longrightarrow& 1 + (R_X\omega C)^2= 2\omega^2C^2 R_X\ R(1 + (R_X\omega C)^2)
        \Longrightarrow\\
        \Longrightarrow& (2\omega^2C^2 R_X\ R-1)(1 + (R_X\omega C)^2) = 0
    \end{align*}

    Therefore, the only possible solution is:
    \begin{equation*}
        R_X = \dfrac{1}{2R (\omega C)^2}
    \end{equation*}

    Using that value of $R_X$, we will measure the same current value with the switch opened and closed.
    
\end{ejercicio}