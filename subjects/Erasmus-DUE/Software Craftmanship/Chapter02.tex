\chapter{VCS With Git}

In this chapter, we will focus on the insights of using Git as a Version Control System (VCS). However, a depper understanding of the concepts behind Git can be found in the Chapter 2 of the contents of the \myhref{https://losdeldgiim.github.io/subjects/Erasmus-DUE/Application\%20Management/Notes.pdf}{Application Management} course.

\begin{definicion}[Version Control System]
A Version Control System (VCS) is a software tool that helps manage changes to source code over time. It allows multiple developers to collaborate on a project, track changes, and maintain a history of modifications.
\end{definicion}

Git is a distributed version control system that provides a powerful and flexible way to manage code changes. There is a lot of reasons why Git is widely used in the software development industry, including its speed, efficiency, and support for branching and merging. Even when someone works alone on a project, using Git can provide benefits such as keeping a history of changes, allowing to revert to previous versions, and facilitating the integration of new features or bug fixes.


\section{Git Data Model}

Git is based on three different types of objects: blobs, trees, and commits. All of these objects are identified by a unique SHA-1 hash, which is generated based on the content of the object, and saved in the \verb|.git/objects| folder.
\begin{enumerate}
    \item \ul{Blobs}: A blob (binary large object) is a file that contains the contents of a file in the repository. It does not contain any metadata, such as the file name or permissions.
    \item \ul{Trees}: A tree is a directory that contains references to blobs and other trees. It represents the structure of the repository at a given point in time. It also contains metadata, such as the file name and permissions.
    \item \ul{Commits}: A commit is a snapshot of the repository at a specific point in time. It contains a reference to a tree, which represents the state of the repository at the time of the commit. It also contains metadata, such as the author, date, and commit message.
    
    In order to refer commits in a more human-readable way, Git uses references, such as branches and tags. The most important ones are:
    \begin{itemize}
        \item \texttt{HEAD}: A reference to the current commit that the user is working on.
        \item \texttt{master} or \texttt{main}: The default branch in Git, which typically represents the main development line.
    \end{itemize}

    Commits always have a comment. There are good and bad practices for writing commit messages. A good commit message should be concise and should describe the changes made in the commit.

    Another important concept in Git is commit granularity. Although some people have a different opinion on this topic, it is generally recommended to have small and focused commits that represent a single logical change. This makes it easier to understand the history of the repository and to revert changes if necessary.
\end{enumerate}

In order to save the history of changes, Git uses a DAG (Directed Acyclic Graph) data structure. Each commit points to its parent commit(s), creating a chain of commits that represents the history of the repository. This allows Git to efficiently manage and track changes over time, enabling features like branching and merging.


Lastly, the idea of branches in Git is fundamental to its workflow. A branch is a pointer to a specific commit, allowing developers to work on different features or bug fixes without affecting the main codebase. Branches can be easily created, merged, and deleted, making it a powerful tool for managing parallel development efforts.

\section{Working with Others}

Git is a distributed VCS, so it uses a centralized collaboration. This means that there is a central repository that serves as the main source of truth for the project, and developers can clone this repository to their local machines, make changes, and then push those changes back to the central repository. This central repository can be hosted on platforms like GitHub, GitLab, or Bitbucket, which provide additional features for collaboration, such as pull requests, code reviews, and issue tracking.