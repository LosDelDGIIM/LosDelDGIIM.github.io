\section{Version Control System (VCS)}
\label{sec:vcs}

\begin{ejercicio}
    Beschreiben Sie die grundlegenden Unterschiede zwischen Git und SVN hinsichtlich ihrer Arbeitsweise und ihres Datenmodells. Erläutern Sie, was ein verteiltes Versionskontrollsystem (Git) und ein zentrales Versionskontrollsystem (SVN) sind.\\

    Both Git and SVN are version control systems, but they differ in their architecture and how they manage data. Both architectures have a central repository in a server, however:
    \begin{itemize}
        \item In a centralized version control system like SVN, the central repository is the only repository, and all operations (commits, updates, etc.) are performed directly on this central repository. This means that developers need to be connected to the central repository to perform most operations, and the history of changes is stored in this central location.
        \item In a distributed version control system like Git, each developer has also a complete copy of that main repository, including its entire history. This allows developers to work offline and perform operations such as commits, branching, and merging locally. Changes can be shared between repositories through push and pull operations, but the local repositories are fully functional on their own.
    \end{itemize}
\end{ejercicio}~\\

Zu den folgenden Aufgaben finden Sie \myhref{https://github.com/LosDelDGIIM/LosDelDGIIM.github.io/blob/main/subjects/Erasmus-DUE/Application\%20Management/Blatt_Material/blatt02_material.zip}{hier} ein \verb|zip|-Datei mit den notwendigen Ressourcen.
\begin{ejercicio}~\label{ej:git-branching}
    \begin{enumerate}
        \item \label{itm:git-branching-commit}
        Nehmen Sie an, Sie arbeiten mit einem  kleinen Team an der Echtzeit-Messaging-App für die
        „Universität der Zukunft“. Sie haben noch lokale Änderungen, die noch nicht committet sind.
        Erstellen Sie bitte einen Commit und schließen Sie somit die Entwicklung des neuen Features ab.

        The code needed to solve this exercise is shown in Listing~\ref{lst:git-branching-commit}.
        \begin{listing}
            \begin{minted}[breaklines=true,fontsize=\small]{shell}
$ git status
On branch cool_stuff
Changes not staged for commit:
  (use "git add <file>..." to update what will be committed)
  (use "git restore <file>..." to discard changes in working directory)
	modified:   text.py

no changes added to commit (use "git add" and/or "git commit -a")
$ git add text.py 
$ git commit -m "New Feature"
[cool_stuff 00c5b2a] New Feature
 1 file changed, 3 insertions(+), 2 deletions(-)
            \end{minted}
            \caption{Lösung zu Übung~\ref{ej:git-branching}.\ref{itm:git-branching-commit}}
            \label{lst:git-branching-commit}
        \end{listing}
        \item \label{itm:git-branching-merge}
        Da das neue Feature nun fertig implementiert ist, möchten Sie dafür sorgen, dass es auch in den
        aktuellen master aufgenommen wird. Der übliche Prozess in Ihrem Team ist, einen Merge Request
        (MR) zu erstellen, der den Feature-Branch in den master übernimmt. Es gibt die Richtlinie, dass
        MRs nur dann akzeptiert werden, wenn sie mit dem master aktuell sind und keine Konflikte
        erzeugen. Sorgen Sie dafür, dass der Feature-Branch \verb|cool_stuff| mit dem master-Branch aktuell
        ist.

        The code needed to solve this exercise is shown in Listing~\ref{lst:git-branching-merge}.
        \begin{listing}
            \begin{minted}[breaklines=true,fontsize=\small]{shell}
$ git branch
* cool_stuff
  master
$ git checkout master 
Switched to branch 'master'
$ git merge cool_stuff 
Merge made by the 'ort' strategy.
 text.py | 12 +++++++++++-
 1 file changed, 11 insertions(+), 1 deletion(-)
            \end{minted}
            \caption{Lösung zu Übung~\ref{ej:git-branching}.\ref{itm:git-branching-merge}}
            \label{lst:git-branching-merge}
        \end{listing}


    \end{enumerate}
\end{ejercicio}

\begin{ejercicio}\label{ej:git-prepare-mr}
    Ihr MR wurde akzeptiert und das neue Feature ist im master. Parallel dazu haben Sie im \verb|other_cool_stuff|-Branch noch an einem ähnlichen Feature gearbeitet. Bereiten Sie auch diesen Branch für den MR in den
    master vor.

    At first it is needed to execute the code shown in Listing~\ref{lst:git-prepare-mr}. After manually solving the conflict, the code shown in Listing~\ref{lst:git-prepare-mr-2} is executed.
    \begin{listing}
        \begin{minted}[breaklines=true,fontsize=\small]{shell}
$ git branch
  master
* other_cool_stuff
$ git checkout master 
Switched to branch 'master'
$ git merge other_cool_stuff 
Auto-merging text.py
CONFLICT (content): Merge conflict in text.py
Automatic merge failed; fix conflicts and then commit the result.
        \end{minted}
        \caption{Erster Teil der Lösung zu Übung~\ref{ej:git-prepare-mr}.}
        \label{lst:git-prepare-mr}
    \end{listing}
    \begin{listing}
        \begin{minted}[breaklines=true,fontsize=\small]{shell}
$ git add text.py 
$ git commit
[master 9abac85] Merge branch 'other_cool_stuff'
        \end{minted}
        \caption{Zweiter Teil der Lösung zu Übung~\ref{ej:git-prepare-mr}.}
        \label{lst:git-prepare-mr-2}
    \end{listing}
\end{ejercicio}

\begin{ejercicio}
    Ein Kollege von Ihnen ist erst seit Kurzem im Team und bittet Sie um Hilfe, da er lokale Änderungen vorgenommen hat, die er jedoch nicht committen kann. Helfen Sie ihm, alle geänderten und neu hinzugefügten Dateien zu committen.

    The problem here lies in the fact that there is a \verb|pre-commit| hook, the shown in Listing~\ref{lst:git-pre-commit-hook}, that prevents committing new files. The solution is just to delete that hook from the \verb|.git/hooks| directory.
    \begin{listing}
        \begin{minted}[breaklines=true,fontsize=\small]{bash}
#!/bin/sh

git diff --cached --name-only --diff-filter=A \
| while read -r name; do
    git rm --cached "$name" > /dev/null 2> /dev/null
done

exit 0
        \end{minted}
        \caption{\texttt{pre-commit} Hook that prevents committing new files.}
        \label{lst:git-pre-commit-hook}
    \end{listing}
\end{ejercicio}