\section{Version Control System (VCS)}
\label{sec:vcs}

\begin{ejercicio}
    Beschreiben Sie die grundlegenden Unterschiede zwischen Git und SVN hinsichtlich ihrer Arbeitsweise und ihres Datenmodells. Erläutern Sie, was ein verteiltes Versionskontrollsystem (Git) und ein zentrales Versionskontrollsystem (SVN) sind.
\end{ejercicio}~\\

Zu den Aufgaben finden Sie \href{https://github.com/LosDelDGIIM/LosDelDGIIM.github.io/blob/main/subjects/Erasmus-DUE/Application%20Management/Blatt_Material/blatt02_material.zip}{hier} ein \verb|zip|-Datei mit den notwendigen Ressourcen.
\begin{ejercicio}~\label{ej:git-branching}
    \begin{enumerate}
        \item \label{itm:git-branching-commit}
        Nehmen Sie an, Sie arbeiten mit einem  kleinen Team an der Echtzeit-Messaging-App für die
        „Universität der Zukunft“. Sie haben noch lokale Änderungen, die noch nicht committet sind.
        Erstellen Sie bitte einen Commit und schließen Sie somit die Entwicklung des neuen Features ab.

        The code needed to solve this exercise is shown in Listing~\ref{lst:git-branching-commit}.
        \begin{listing}
            \begin{minted}[breaklines=true,fontsize=\small]{shell}
$ git status
On branch cool_stuff
Changes not staged for commit:
  (use "git add <file>..." to update what will be committed)
  (use "git restore <file>..." to discard changes in working directory)
	modified:   text.py

no changes added to commit (use "git add" and/or "git commit -a")
$ git add text.py 
$ git commit -m "New Feature"
[cool_stuff 00c5b2a] New Feature
 1 file changed, 3 insertions(+), 2 deletions(-)
$ 
            \end{minted}
            \caption{Lösung zu Übung~\ref{ej:git-branching}.\ref{itm:git-branching-commit}}
            \label{lst:git-branching-commit}
        \end{listing}
        \item \label{itm:git-branching-merge}
        Da das neue Feature nun fertig implementiert ist, möchten Sie dafür sorgen, dass es auch in den
        aktuellen master aufgenommen wird. Der übliche Prozess in Ihrem Team ist, einen Merge Request
        (MR) zu erstellen, der den Feature-Branch in den master übernimmt. Es gibt die Richtlinie, dass
        MRs nur dann akzeptiert werden, wenn sie mit dem master aktuell sind und keine Konflikte
        erzeugen. Sorgen Sie dafür, dass der Feature-Branch \verb|cool_stuff| mit dem master-Branch aktuell
        ist.
    \end{enumerate}
\end{ejercicio}

\begin{ejercicio}\label{ej:git-prepare-mr}
    Ihr MR wurde akzeptiert und das neue Feature ist im master. Parallel dazu haben Sie im \verb|other_cool_stuff|
    -Branch noch an einem ähnlichen Feature gearbeitet. Bereiten Sie auch diesen Branch für den MR in den
    master vor.
\end{ejercicio}

\begin{ejercicio}
    Ein Kollege von Ihnen ist erst seit Kurzem im Team und bittet Sie um Hilfe, da er lokale Änderungen
    vorgenommen hat, die er jedoch nicht committen kann. Helfen Sie ihm, alle geänderten und neu
    hinzugefügten Dateien zu committen.
\end{ejercicio}