\section{Fuzzing \& Z3 2}

Fur die nachsten Aufgaben müssen Sie ein paar Python-Pakete installieren. Sie finden \verb|z3|-Beispiele in Moodle.
\begin{enumerate}
    \item Öffnen Sie das Terminal (\verb|ctrl + Alt + T|)
    \item Installieren Sie den Package Installer for Python ( \verb|$ sudo apt install python3-pip python3-venv| )
    \item Erstellen Sie ein neues Virtual Environment ( \verb|$ python3 -m venv ./venv| )
    \item Konfigurieren Sie die aktuelle Shell, damit sie das \verb|venv| verwendet. ( \verb|$ source ./venv/bin/activate| )
    \item Installieren Sie \verb|z3| ( \verb|$ pip3 install z3-solver| )
    \item Atheris können Sie in einem Ubuntu 24.04 Container installieren ( \verb|$ pip3 install atheris| )
\end{enumerate}

\begin{ejercicio}~
    \begin{enumerate}
        \item Nutzen Sie \verb|atheris| und instrumentieren Sie die \verb|validate|-Funktion in \verb|fuzz.py|.
        \item Zeichnen Sie einen Kontrollflussgraphen für die Funktion. Nutzen Sie als Knotennamen die Zeilen-nummern.
        \item Bestimmen Sie die prozentuale Coverage, wenn der Code mit \verb|[246,63,103,121]| aufgerufen wird. Markieren Sie außerdem die erreichten Knoten im Kontrollflussgraphen.
        \item Instrumentieren Sie die \verb|validate|-Funktion in \verb|fuzz2.py|
        \item Suchen Sie mittels \verb|z3| nach einer Lösung für \verb|validate|.
    \end{enumerate}
\end{ejercicio}

\begin{ejercicio}
Manche Programmierer nutzen Bit-Tricks, um auf spezifische Hardware zu optimieren. Nutzen Sie \verb|z3|, um zu zeigen, dass die Tricks auf einem 64-Bit-System dasselbe Verhalten haben wie eine naive Implementierung.
    \begin{enumerate}
        \item Tauschen zweier Variablen mit XOR (Swapping Values with XOR)
        \item Tauschen zweier Variablen mit Subtraktion und Addition (Swapping values with subtraction and addition)
        \item Die Bitreihenfolge innerhalb eines Bytes umkehren (Reverse the bits in a byte with 4 operations)
    \end{enumerate}
\end{ejercicio}


\begin{ejercicio}
Nutzen Sie \verb|z3|, um generische Sudokus zu lösen.
\begin{observacion}
    Sollten Sie nicht weiterkommen, kann eine Internetsuche weiterhelfen.
\end{observacion}
\end{ejercicio}

\begin{ejercicio}
Ein Freund von Ihnen behauptet, dass er sein eigenes sicheres Krypto-System entwickelt hat. Zum System gehören zwei Komponenten: eine asymmetrische und eine symmetrische Chiffre. Zeigen Sie, dass beide Verfahren nicht sicher sind.
\end{ejercicio}