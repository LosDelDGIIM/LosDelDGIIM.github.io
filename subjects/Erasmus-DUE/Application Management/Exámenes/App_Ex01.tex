\documentclass[12pt]{article}

% Idioma y codificación
\usepackage[spanish, es-tabla]{babel}       %es-tabla para que se titule "Tabla"
\usepackage[utf8]{inputenc}

% Márgenes
\usepackage[a4paper,top=3cm,bottom=2.5cm,left=3cm,right=3cm]{geometry}

% Comentarios de bloque
\usepackage{verbatim}

% Paquetes de links
\usepackage[hidelinks]{hyperref}    % Permite enlaces
\usepackage{url}                    % redirecciona a la web

% Más opciones para enumeraciones
\usepackage{enumitem}

% Personalizar la portada
\usepackage{titling}

% Paquetes de tablas
\usepackage{multirow}


%------------------------------------------------------------------------

%Paquetes de figuras
\usepackage{caption}
\usepackage{subcaption} % Figuras al lado de otras
\usepackage{float}      % Poner figuras en el sitio indicado H.


% Paquetes de imágenes
\usepackage{graphicx}       % Paquete para añadir imágenes
\usepackage{transparent}    % Para manejar la opacidad de las figuras

% Paquete para usar colores
\usepackage[dvipsnames]{xcolor}
\usepackage{pagecolor}      % Para cambiar el color de la página

% Habilita tamaños de fuente mayores
\usepackage{fix-cm}

% Para los gráficos
\usepackage{tikz}

% Para poder situar los nodos en los grafos
\usetikzlibrary{positioning}


%------------------------------------------------------------------------

% Paquetes de matemáticas
\usepackage{mathtools, amsfonts, amssymb, mathrsfs}
\usepackage[makeroom]{cancel}     % Simplificar tachando
\usepackage{polynom}    % Divisiones y Ruffini
\usepackage{units} % Para poner fracciones diagonales con \nicefrac

\usepackage{pgfplots}   %Representar funciones
\pgfplotsset{compat=1.18}  % Versión 1.18

\usepackage{tikz-cd}    % Para usar diagramas de composiciones
\usetikzlibrary{calc}   % Para usar cálculo de coordenadas en tikz

%Definición de teoremas, etc.
\usepackage{amsthm}
%\swapnumbers   % Intercambia la posición del texto y de la numeración

\theoremstyle{plain}

\makeatletter
\@ifclassloaded{article}{
  \newtheorem{teo}{Teorema}[section]
}{
  \newtheorem{teo}{Teorema}[chapter]  % Se resetea en cada chapter
}
\makeatother

\newtheorem{coro}{Corolario}[teo]           % Se resetea en cada teorema
\newtheorem{prop}[teo]{Proposición}         % Usa el mismo contador que teorema
\newtheorem{lema}[teo]{Lema}                % Usa el mismo contador que teorema

\theoremstyle{remark}
\newtheorem*{observacion}{Observación}

\theoremstyle{definition}

\makeatletter
\@ifclassloaded{article}{
  \newtheorem{definicion}{Definición} [section]     % Se resetea en cada chapter
}{
  \newtheorem{definicion}{Definición} [chapter]     % Se resetea en cada chapter
}
\makeatother

\newtheorem*{notacion}{Notación}
\newtheorem*{ejemplo}{Ejemplo}
\newtheorem*{ejercicio*}{Ejercicio}             % No numerado
\newtheorem{ejercicio}{Ejercicio} [section]     % Se resetea en cada section


% Modificar el formato de la numeración del teorema "ejercicio"
\renewcommand{\theejercicio}{%
  \ifnum\value{section}=0 % Si no se ha iniciado ninguna sección
    \arabic{ejercicio}% Solo mostrar el número de ejercicio
  \else
    \thesection.\arabic{ejercicio}% Mostrar número de sección y número de ejercicio
  \fi
}


% \renewcommand\qedsymbol{$\blacksquare$}         % Cambiar símbolo QED
%------------------------------------------------------------------------

% Paquetes para encabezados
\usepackage{fancyhdr}
\pagestyle{fancy}
\fancyhf{}

\newcommand{\helv}{ % Modificación tamaño de letra
\fontfamily{}\fontsize{12}{12}\selectfont}
\setlength{\headheight}{15pt} % Amplía el tamaño del índice


%\usepackage{lastpage}   % Referenciar última pag   \pageref{LastPage}
\fancyfoot[C]{\thepage}

%------------------------------------------------------------------------

% Conseguir que no ponga "Capítulo 1". Sino solo "1."
\makeatletter
\@ifclassloaded{book}{
  \renewcommand{\chaptermark}[1]{\markboth{\thechapter.\ #1}{}} % En el encabezado
    
  \renewcommand{\@makechapterhead}[1]{%
  \vspace*{50\p@}%
  {\parindent \z@ \raggedright \normalfont
    \ifnum \c@secnumdepth >\m@ne
      \huge\bfseries \thechapter.\hspace{1em}\ignorespaces
    \fi
    \interlinepenalty\@M
    \Huge \bfseries #1\par\nobreak
    \vskip 40\p@
  }}
}
\makeatother

%------------------------------------------------------------------------
% Paquetes de cógido
\usepackage{minted}
\renewcommand\listingscaption{Código fuente}

\usepackage{fancyvrb}
% Personaliza el tamaño de los números de línea
\renewcommand{\theFancyVerbLine}{\small\arabic{FancyVerbLine}}

% Estilo para C++
\newminted{cpp}{
    frame=lines,
    framesep=2mm,
    baselinestretch=1.2,
    linenos,
    escapeinside=||
}

% para minted
\definecolor{LightGray}{rgb}{0.95,0.95,0.92}
\setminted{
    linenos=true,
    stepnumber=5,
    numberfirstline=true,
    autogobble,
    breaklines=true,
    breakautoindent=true,
    breaksymbolleft=,
    breaksymbolright=,
    breaksymbolindentleft=0pt,
    breaksymbolindentright=0pt,
    breaksymbolsepleft=0pt,
    breaksymbolsepright=0pt,
    fontsize=\footnotesize,
    bgcolor=LightGray,
    numbersep=10pt
}


\usepackage{listings} % Para incluir código desde un archivo

\renewcommand\lstlistingname{Código Fuente}
\renewcommand\lstlistlistingname{Índice de Códigos Fuente}

% Definir colores
\definecolor{vscodepurple}{rgb}{0.5,0,0.5}
\definecolor{vscodeblue}{rgb}{0,0,0.8}
\definecolor{vscodegreen}{rgb}{0,0.5,0}
\definecolor{vscodegray}{rgb}{0.5,0.5,0.5}
\definecolor{vscodebackground}{rgb}{0.97,0.97,0.97}
\definecolor{vscodelightgray}{rgb}{0.9,0.9,0.9}

% Configuración para el estilo de C similar a VSCode
\lstdefinestyle{vscode_C}{
  backgroundcolor=\color{vscodebackground},
  commentstyle=\color{vscodegreen},
  keywordstyle=\color{vscodeblue},
  numberstyle=\tiny\color{vscodegray},
  stringstyle=\color{vscodepurple},
  basicstyle=\scriptsize\ttfamily,
  breakatwhitespace=false,
  breaklines=true,
  captionpos=b,
  keepspaces=true,
  numbers=left,
  numbersep=5pt,
  showspaces=false,
  showstringspaces=false,
  showtabs=false,
  tabsize=2,
  frame=tb,
  framerule=0pt,
  aboveskip=10pt,
  belowskip=10pt,
  xleftmargin=10pt,
  xrightmargin=10pt,
  framexleftmargin=10pt,
  framexrightmargin=10pt,
  framesep=0pt,
  rulecolor=\color{vscodelightgray},
  backgroundcolor=\color{vscodebackground},
}

%------------------------------------------------------------------------

% Comandos definidos
\newcommand{\bb}[1]{\mathbb{#1}}
\newcommand{\cc}[1]{\mathcal{#1}}

% I prefer the slanted \leq
\let\oldleq\leq % save them in case they're every wanted
\let\oldgeq\geq
\renewcommand{\leq}{\leqslant}
\renewcommand{\geq}{\geqslant}

% Si y solo si
\newcommand{\sii}{\iff}

% Letras griegas
\newcommand{\eps}{\epsilon}
\newcommand{\veps}{\varepsilon}
\newcommand{\lm}{\lambda}

\newcommand{\ol}{\overline}
\newcommand{\ul}{\underline}
\newcommand{\wt}{\widetilde}
\newcommand{\wh}{\widehat}

\let\oldvec\vec
\renewcommand{\vec}{\overrightarrow}

% Derivadas parciales
\newcommand{\del}[2]{\frac{\partial #1}{\partial #2}}
\newcommand{\Del}[3]{\frac{\partial^{#1} #2}{\partial #3^{#1}}}
\newcommand{\deld}[2]{\dfrac{\partial #1}{\partial #2}}
\newcommand{\Deld}[3]{\dfrac{\partial^{#1} #2}{\partial #3^{#1}}}


\newcommand{\AstIg}{\stackrel{(\ast)}{=}}
\newcommand{\Hop}{\stackrel{L'H\hat{o}pital}{=}}

\newcommand{\red}[1]{{\color{red}#1}} % Para integrales, destacar los cambios.

% Método de integración
\newcommand{\MetInt}[2]{
    \left[\begin{array}{c}
        #1 \\ #2
    \end{array}\right]
}

% Declarar aplicaciones
% 1. Nombre aplicación
% 2. Dominio
% 3. Codominio
% 4. Variable
% 5. Imagen de la variable
\newcommand{\Func}[5]{
    \begin{equation*}
        \begin{array}{rrll}
            #1:& #2 & \longrightarrow & #3\\
               & #4 & \longmapsto & #5
        \end{array}
    \end{equation*}
}

%------------------------------------------------------------------------


\usetikzlibrary{arrows.meta, positioning, shapes.multipart}
\tikzset{
    git/commit/.style={
        circle, 
        draw, 
        minimum size=1.2cm, 
        font=\small\ttfamily
    },
    % Estilo base para refs
    git/ref/.style={
        draw,
        rectangle split,
        rectangle split parts=#1, 
        minimum width=1.5cm,
        font=\small\ttfamily,
        align=center,
        fill=white
    },
    % Valor por defecto para el argumento #1
    git/ref/.default=1,
    git/history/.style={
        -{Stealth}, 
        thick
    },
    git/pointer/.style={
        -{Stealth}, 
        dashed, 
        thick
    }
}



\begin{document}

    % 1. Foto de fondo
    % 2. Título
    % 3. Encabezado Izquierdo
    % 4. Color de fondo
    % 5. Coord x del titulo
    % 6. Coord y del titulo
    % 7. Fecha

    % 1. Foto de fondo
% 2. Título
% 3. Encabezado Izquierdo
% 4. Color de fondo
% 5. Coord x del titulo
% 6. Coord y del titulo
% 7. Fecha

\newcommand{\portada}[7]{

    \portadaBase{#1}{#2}{#3}{#4}{#5}{#6}{#7}
    \portadaBook{#1}{#2}{#3}{#4}{#5}{#6}{#7}
}

\newcommand{\portadaExamen}[7]{

    \portadaBase{#1}{#2}{#3}{#4}{#5}{#6}{#7}
    \portadaArticle{#1}{#2}{#3}{#4}{#5}{#6}{#7}
}




\newcommand{\portadaBase}[7]{

    % Tiene la portada principal y la licencia Creative Commons
    
    % 1. Foto de fondo
    % 2. Título
    % 3. Encabezado Izquierdo
    % 4. Color de fondo
    % 5. Coord x del titulo
    % 6. Coord y del titulo
    % 7. Fecha
    
    
    \thispagestyle{empty}               % Sin encabezado ni pie de página
    \newgeometry{margin=0cm}        % Márgenes nulos para la primera página
    
    
    % Encabezado
    \fancyhead[L]{\helv #3}
    \fancyhead[R]{\helv \nouppercase{\leftmark}}
    
    
    \pagecolor{#4}        % Color de fondo para la portada
    
    \begin{figure}[p]
        \centering
        \transparent{0.3}           % Opacidad del 30% para la imagen
        
        \includegraphics[width=\paperwidth, keepaspectratio]{assets/#1}
    
        \begin{tikzpicture}[remember picture, overlay]
            \node[anchor=north west, text=white, opacity=1, font=\fontsize{60}{90}\selectfont\bfseries\sffamily, align=left] at (#5, #6) {#2};
            
            \node[anchor=south east, text=white, opacity=1, font=\fontsize{12}{18}\selectfont\sffamily, align=right] at (9.7, 3) {\textbf{\href{https://losdeldgiim.github.io/}{Los Del DGIIM}}};
            
            \node[anchor=south east, text=white, opacity=1, font=\fontsize{12}{15}\selectfont\sffamily, align=right] at (9.7, 1.8) {Doble Grado en Ingeniería Informática y Matemáticas\\Universidad de Granada};
        \end{tikzpicture}
    \end{figure}
    
    
    \restoregeometry        % Restaurar márgenes normales para las páginas subsiguientes
    \pagecolor{white}       % Restaurar el color de página
    
    
    \newpage
    \thispagestyle{empty}               % Sin encabezado ni pie de página
    \begin{tikzpicture}[remember picture, overlay]
        \node[anchor=south west, inner sep=3cm] at (current page.south west) {
            \begin{minipage}{0.5\paperwidth}
                \href{https://creativecommons.org/licenses/by-nc-nd/4.0/}{
                    \includegraphics[height=2cm]{assets/Licencia.png}
                }\vspace{1cm}\\
                Esta obra está bajo una
                \href{https://creativecommons.org/licenses/by-nc-nd/4.0/}{
                    Licencia Creative Commons Atribución-NoComercial-SinDerivadas 4.0 Internacional (CC BY-NC-ND 4.0).
                }\\
    
                Eres libre de compartir y redistribuir el contenido de esta obra en cualquier medio o formato, siempre y cuando des el crédito adecuado a los autores originales y no persigas fines comerciales. 
            \end{minipage}
        };
    \end{tikzpicture}
    
    
    
    % 1. Foto de fondo
    % 2. Título
    % 3. Encabezado Izquierdo
    % 4. Color de fondo
    % 5. Coord x del titulo
    % 6. Coord y del titulo
    % 7. Fecha


}


\newcommand{\portadaBook}[7]{

    % 1. Foto de fondo
    % 2. Título
    % 3. Encabezado Izquierdo
    % 4. Color de fondo
    % 5. Coord x del titulo
    % 6. Coord y del titulo
    % 7. Fecha

    % Personaliza el formato del título
    \pretitle{\begin{center}\bfseries\fontsize{42}{56}\selectfont}
    \posttitle{\par\end{center}\vspace{2em}}
    
    % Personaliza el formato del autor
    \preauthor{\begin{center}\Large}
    \postauthor{\par\end{center}\vfill}
    
    % Personaliza el formato de la fecha
    \predate{\begin{center}\huge}
    \postdate{\par\end{center}\vspace{2em}}
    
    \title{#2}
    \author{\href{https://losdeldgiim.github.io/}{Los Del DGIIM}}
    \date{Granada, #7}
    \maketitle
    
    \tableofcontents
}




\newcommand{\portadaArticle}[7]{

    % 1. Foto de fondo
    % 2. Título
    % 3. Encabezado Izquierdo
    % 4. Color de fondo
    % 5. Coord x del titulo
    % 6. Coord y del titulo
    % 7. Fecha

    % Personaliza el formato del título
    \pretitle{\begin{center}\bfseries\fontsize{42}{56}\selectfont}
    \posttitle{\par\end{center}\vspace{2em}}
    
    % Personaliza el formato del autor
    \preauthor{\begin{center}\Large}
    \postauthor{\par\end{center}\vspace{3em}}
    
    % Personaliza el formato de la fecha
    \predate{\begin{center}\huge}
    \postdate{\par\end{center}\vspace{5em}}
    
    \title{#2}
    \author{\href{https://losdeldgiim.github.io/}{Los Del DGIIM}}
    \date{Granada, #7}
    \thispagestyle{empty}               % Sin encabezado ni pie de página
    \maketitle
    \vfill
}
    \portadaExamenFotoDif{../../dueA4.jpg}{Application\\Management\\Exam I}{App. Management. Exam I}{MidnightBlue}{-8}{28}{2025-2026}{Arturo Olivares Martos}

    \begin{description}
        \item[Asignatura] Application Management.
        \item[Curso Académico] Winter Semester 2024-25.
        %\item[Grado] Grado en Matemáticas.
        %\item[Grupo] B.
        %\item[Profesor] José María Espinar García.
        %\item[Descripción] Parcial de los Temas 2 y 3.
        \item[Fecha] 13 de Febrero de 2025.
        % \item[Duración] 60 minutos.
    
    \end{description}
    \newpage
    
    \begin{ejercicio}[Application Lifecycle Management (ALM)]~
        \begin{enumerate}
            \item (1 Punkt) Nennen Sie vier Phasen des ALM, welche auch dem Wasserfallmodell zugeordnet werden können.
            \item (2 Punkte) Welche(n) Vorteil(e) bietet die Wahl einer permissiven Lizenz wie Apache~2.0 oder MIT?
            \begin{enumerate}
                \item Förderung der breiten Nutzung und Akzeptanz.
                \item Strikte Kontrolle über alle Modifikationen.
                \item Geringere Hürden für kommerzielle Nutzung.
                \item Verpflichtung zur Offenlegung aller Änderungen.
            \end{enumerate}

            \item (2 Punkte) Welche der folgenden Aussagen zum Application-Lifecycle-Management (ALM) trifft/treffen zu?
            \begin{enumerate}
                \item ALM umfasst neben Entwicklung auch Betriebs- und Wartungsprozesse.
                \item ALM ist nur relevant für agile Softwareprojekte.
                \item ALM setzt sich häufig aus kontinuierlichem Monitoring und Feedback zusammen.
                \item Ein zentrales Ziel von ALM ist die Transparenz über die gesamte Lebensdauer einer Software.
            \end{enumerate}

            \item (2 Punkte) Welche der folgenden Aussagen trifft/treffen auf die Shift-Left-Teststrategie zu?
            \begin{enumerate}
                \item Tests möglichst spät durchführen, damit mehr Zeit für die Entwicklung bleibt.
                \item Probleme möglichst früh erkennen und beheben, um Kosten zu senken.
                \item Den Testaufwand im Betrieb auf ein Minimum reduzieren.
                \item Nur kritische Funktionen in einer frühen Phase testen.
            \end{enumerate}

            \item (2 Punkte) Warum ist es wichtig, den gesamten Lebenszyklus einer Software zu betrachten?
            
            \item (2 Punkte) Was versteht man unter 'Flakiness' bei Tests?
            
            \item (2 Punkte) Beschreiben Sie, welche Einfluss 'flaky' Tests auf eine CI/CD haben.
            
            \item (4 Punkte) Nennen Sie vier Stakeholder, welche typischerweise in den ALM-Prozess involviert sind und beschreiben Sie diese kurz.
        \end{enumerate}
    \end{ejercicio}





    \begin{ejercicio}[Versionskontrolle]~
        \begin{enumerate}
            \item (2 Punkte) Welche der folgenden Vorteile bietet ein verteiltes Versionskontrollsystem?
            \begin{enumerate}
                \item Jeder Entwickler hat eine vollständige Kopie des Repositories.
                \item Änderungen werden automatisch mit allen anderen synchronisiert.
                \item Es gibt immer einen zentralen Server, der alle Änderungen speichert.
                \item Arbeiten an Branches können unabhängig voneinander erfolgen.
            \end{enumerate}
            \item (2 Punkte) Welche Rolle spielen Hashes bei Git?
            \begin{enumerate}
                \item Sie identifizieren eindeutig jede Version einer Datei oder eines Commits.
                \item Sie verhindern Datenverlust bei Netzwerkausfällen.
                \item Sie ermöglichen es Entwicklern, Konflikte automatisch zu lösen.
                \item Sie gewährleisten die Integrität der gespeicherten Daten.
            \end{enumerate}
            \item (6 Punkte) Diskutieren Sie die Vor- und Nachteile von Rebase vs. Merge in Bezug auf Historie, Konfliktmanagement und Teamabsprachen aus der Sicht:
            \begin{itemize}
                \item Der Übersichtlichkeit der Historie
                \item Des Konfliktmanagements
                \item Der Teamabsprachen (z. B. wann force push nötig wird).
            \end{itemize}

            \item (5 Punkte) Es wurde folgender Befehl ausgeführt: \verb|git checkout HEAD^2|. Auf welchen Commit zeigt \verb|HEAD|? Begründen Sie Ihre Antwort. In Figur~\ref{fig:commit-history} ist die Commit-Historie dargestellt, und in Listing~\ref{lst:head-history} die Historie von \verb|HEAD|.
            \begin{figure}
                \centering
                \begin{tikzpicture}[node distance=1.5cm]
                    % Commits
                    \node[git/commit] (9ca0fb) {\verb|9ca0fb|};
                    \node[git/commit, right=of 9ca0fb] (517f73) {\verb|517f73|};
                    \node[git/commit, above =of 517f73] (84a640) {\verb|84a640|};
                    \node[git/commit, above =of 84a640] (f59d4c) {\verb|f59d4c|};
                    \node[git/commit, right=of 84a640] (f2f27d) {\verb|f2f27d|};
                    \node[git/commit, below=of f2f27d, xshift=3cm] (15ce75) {\verb|15ce75|};

                    % HEAD
                    \node[git/ref, right=0.5cm of 15ce75] (HEAD) {\verb|HEAD|};

                    % Verbindungen
                    \draw[git/history] (517f73) -- (9ca0fb);
                    \draw[git/history] (84a640) -- (9ca0fb);
                    \draw[git/history] (f59d4c) -| (9ca0fb);
                    \draw[git/history] (f2f27d) -- (84a640);
                    \draw[git/history] (15ce75) |- (f59d4c);
                    \draw[git/history] (15ce75) -- (f2f27d);
                    \draw[git/history] (15ce75) -- (517f73);

                    \draw[git/pointer] (HEAD) -- (15ce75);

                \end{tikzpicture}
                \caption{Commit-Historie}
                \label{fig:commit-history}
            \end{figure}

            \begin{listing}
                \begin{minted}{text}
                    HEAD Historie:
                    15ce75 merge f2f27d, f59d4c
                    517f73 checkout
                    f2f27d commit
                    84a640 checkout
                    f59d4c commit
                    9ca0fb checkout
                \end{minted}
                \caption{\texttt{HEAD} Historie}
                \label{lst:head-history}
            \end{listing}

            \item (4 Punkte) Ein Entwicklerteam verwendet Git für ein Webprojekt. Aus Versehen hat ein Teammitglied in einem früheren Commit eine Datei \texttt{.env} eingecheckt, in der sensible Zugangsdaten (API-Keys, Datenbankpasswörter) enthalten sind. Das ist dem Team erst nach einigen Tagen aufgefallen, nachdem bereits weitere Commits und Branches erstellt wurden. Das Repository ist öffentlich auf GitHub verfügbar. Die Git-Dokumentation beschreibt den Befehl \texttt{git revert} wie in Listing~\ref{lst:git-revert} dargestellt.
            \begin{listing}
                \begin{minted}{text}
1. git revert [--no-edit] [-n] [-m <parent-number>] [-s] [-S[keyid]] <commit>...
2. git revert [--continue | --skip | --abort | --quit]

> Wenn ein oder mehrere bestehende Commits angegeben werden, werden die durch die zugehörigen Patches eingeführten Änderungen rückgängig gemacht. Dabei entstehen neue Commits, die diesen Vorgang festhalten. Voraussetzung dafür ist, dass dein Working Tree sauber ist.
                \end{minted}
                \caption{Git Revert Befehl}
                \label{lst:git-revert}
            \end{listing}
            Begründen Sie, wie und ob dieser Befehl dazu beitragen kann, das Problem zu beheben. Sollten der Befehl nützlich sein, geben Sie außerdem den abgeätzten Befehl an, welcher das Problem behebt.

            \item (2 Punkte) Reicht das Entfernen aus Git aus, um die Sicherheit vollständig wiederherzustellen? Begründen Sie Ihre Antwort.
            
            \item (2 Punkte) Welche Maßnahmen würden Sie dem Team vorschlagen, um solche Vorfälle in Zukunft zu vermeiden? Nennen Sie mindestens zwei.
        \end{enumerate}
        
    \end{ejercicio}



    \begin{ejercicio}[Deployment und Delivery]~
        \begin{enumerate}
            \item (2 Punkte) Welche der folgenden Aussagen trifft/treffen auf Continuous Deployment zu?
            \begin{enumerate}
                \item Jede Codeänderung durchläuft manuelle Tests, bevor sie deployed wird.
                \item Jede erfolgreich getestete Codeänderung wird automatisch in Produktion deployed.
                \item Continuous Deployment erfolgt nur bei erfolgreichen Software-Releases.
                \item Continuous Deployment setzt keine vollständige Automatisierung voraus.
            \end{enumerate}

            \item (2 Punkte) Welche der folgenden Aussagen zu Canary Releases ist korrekt?
            \begin{enumerate}
                \item Alle Benutzer erhalten sofort die neue Version.
                \item Die neue Version wird erst einer kleinen Nutzergruppe zur Verfügung gestellt.
                \item Ein Rollback ist nicht möglich.
                \item Canary Releases sind nicht für sicherheitskritische Systeme geeignet.
            \end{enumerate}

            \item (3 Punkte) Welche Herausforderungen müssen Unternehmen bei der Umstellung auf automatisiertes Deployment berücksichtigen?
            
            \item (3 Punkte) Continuous Integration, Continuous Delivery und Continuous Deployment sind zentrale Prinzipien moderner Softwareentwicklung. Erklären Sie die Unterschiede dieser Konzepte.
            
            \item (2 Punkte) Nennen Sie zwei Vorteile und zwei Herausforderungen von CI/CD Pipelines
            
            \item (6 Punkte) Sie haben eine Binary (MagicWeb), welche einen Webserver auf Port 8080 startet. Diese Binary haben Sie frisch erstellt. Da diese Binary Ihre kompletten Geheimnisse enthält, möchten Sie nicht, dass die Binary im Internet verfügbar ist, daher haben Sie dafür gesorgt, dass Ihr Rechner nicht mit dem Internet verbunden ist. Sie möchten diese Binary aber zweimal ausführen, deshalb haben Sie Docker gerade frisch installiert. Dazu wollen Sie die Docker-Compose Datei wie in Listing~\ref{lst:docker-compose} nutzen.
            \begin{listing}
                \begin{minted}{yaml}
services:
    magicweb1:
        image: MagicWeb
        ports:
            - "8081:8080"
        container_name: magicweb_container1
    magicweb2:
        image: MagicWeb
        ports:
            - "8082:8080"
        container_name: magicweb_container2
                \end{minted}
                \caption{Docker-Compose Datei}
                \label{lst:docker-compose}
            \end{listing}

            Wenn Sie nun \texttt{docker compose up} ausführen, beschreiben Sie was nun passiert. Sollte der Befehl nicht funktionieren, beschreiben Sie, welche Schritte man vornehmen müsste, um das Problem zu beheben.
            
        \end{enumerate}
        
    \end{ejercicio}






    \begin{ejercicio}[Secure Software Development]~
        \begin{enumerate}
            \item (2 Punkte) Welche der folgenden Aussagen trifft/treffen auf Netflix Chaos Monkey zu?
            \begin{enumerate}
                \item Es simuliert und testet zufällige Ausfälle von Systemen, um die Infrastruktur zu verbessern.
                \item Es dient dazu, sensible Daten in Cloud-Umgebungen zu verschlüsseln und vor Bedrohungen zu schützen.
                \item Es ist ein Tool zur automatisierten Schwachstellenanalyse in Webanwendungen.
                \item Es analysiert den Softwarecode auf unsichere Abhängigkeiten und veraltete Bibliotheken.
            \end{enumerate}

            \item (2 Punkte) Welche der folgenden Aussagen trifft/treffen zu?
            \begin{enumerate}
                \item Mutational Fuzzing ist eine Technik, bei der Eingaben durch gezielte synthetische Constraints generiert werden.
                \item Symbolic Execution eignet sich besonders für Programme mit vielen Verzweigungen.
                \item Coverage-Guided Fuzzing nutzt Feedback über erreichte Code-Pfade, um Eingaben systematisch zu generieren.
                \item Symbolic Execution kann auch auf Binärprogramme angewendet werden, ohne zusätzliche Modellierung.
            \end{enumerate}

            \item (2 Punkte) Warum ist Fuzzing ein effektives Mittel zur Identifizierung von Sicherheitslücken?
            
            \item (3 Punkte) Beschreiben Sie, wie Symbolische Ausführung (SymExe) mit einem SMT-Solver wie Z3 systematisch Pfade in einem Programm analysiert.
            
            \item (2 Punkte) Falls die symbolische Ausführung einen kritischen Pfad entdeckt, welche Informationen würde Z3 dazu typischerweise liefern?
            
            \item (4 Punkte) Nennen Sie zwei Security-Testing-Methoden und geben Sie für jede ein Beispiel.
            
            \item (4 Punkte) Wie verbessert eine mehrstufige Zertifikatskette die Sicherheit?
            
            \item Was ist binary provenance und was kann passieren wenn dies nicht gewährt ist? Nennen Sie zwei Fälle.
        \end{enumerate}
        
    \end{ejercicio}
\end{document}