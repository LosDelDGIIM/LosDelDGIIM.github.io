\section{Docker}

\begin{ejercicio}
    In Moodle finden Sie eine ZIP-Datei, die ausführbare Dateien für verschiedene Architekturen enthält. Wenn Sie die Ubuntu-VM nutzen, ist die Datei \texttt{server\_linux\_x86\_64} die richtige. Alle anderen Dateien sind ungetestet und nicht garantiert zu funktionieren. Sie können die Datei mit \texttt{\$ ./server\_linux\_x86\_64} ausführen und den Webserver im Browser unter \texttt{localhost:8080} erreichen. Ihre Aufgabe ist es, zwei Instanzen des Webservers parallel auf Ihrem System auszuführen.
\end{ejercicio}

\begin{ejercicio}
    Sie sind Teil des Entwicklerteams für die Echtzeit-Messaging-App der „Universität der Zukunft“. Da Ihr Team in einer heterogenen Umgebung arbeitet und sicherstellen muss, dass das entwickelte Python-Skript unter verschiedenen Python-Versionen ordnungsgemäß ausgeführt wird, ist es Ihre Aufgabe, Docker-Container für diese Tests zu erstellen. Ihr Manager hat Sie gebeten, zu testen, ob die App mit allen offiziell unterstützten Python-Versionen ausführbar ist.
\end{ejercicio}

\begin{ejercicio}~
    \begin{enumerate}
        \item Was ist Docker und wie unterscheidet es sich von Hypervisor-basierten Virtualisierungstechnologien?
        \item Erläutern Sie den Begriff ``Container'' im Kontext von Docker.
        \item Wie kann Docker in einer Continuous-Integration/Continuous-Deployment-(CI/CD)-Pipeline eingesetzt werden?
        \item Erläutern Sie den Begriff ``Docker Registry'' und erläutern Sie, warum er für CI/CD wichtig ist.
    \end{enumerate}
\end{ejercicio}

\begin{ejercicio}
    In dieser Aufgabe lernen Sie einen der Grundmechanismen der \emph{historischen} Containerisolierung kennen. Dazu verwenden Sie das UNIX-Werkzeug \texttt{chroot}, das den sichtbaren Root-Ordner eines Prozesses ändert. In der ZIP-Datei aus Moodle finden Sie die Datei \texttt{alpine-rootfs.tar}. Diese Datei enthält ein minimales Linux-Dateisystem, das ursprünglich aus einem Docker-Container (\texttt{alpine}) exportiert wurde.
    \begin{enumerate}
        \item Entpacken Sie das Root-Dateisystem in ein Verzeichnis Ihrer Wahl (z.B. \texttt{\$HOME/alpine-rootfs}).
        \item Starten Sie eine Shell innerhalb dieses Dateisystems mithilfe von \texttt{chroot}.
        \item Untersuchen Sie das Verhalten innerhalb der Umgebung:
        \begin{itemize}
            \item Führen Sie Befehle wie \texttt{ls}, \texttt{pwd} und \texttt{ps} aus. Was fällt Ihnen auf?
            \item Starten Sie in einem separaten Terminal \texttt{ps -ef} aus. Können Sie den Prozess aus der chroot-Umgebung sehen?
        \end{itemize}
        \item Schreiben Sie ein kleines Programm/Script, das die Schritte des Entpackens sowie das Starten eines Befehls in der Chroot-Umgebung automatisiert.
        \item Begründen Sie, warum \texttt{chroot} \emph{keine vollständige Isolation} bietet.
    \end{enumerate}
\end{ejercicio}