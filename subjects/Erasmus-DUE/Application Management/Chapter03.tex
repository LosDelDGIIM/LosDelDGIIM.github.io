\chapter{Build Engineering und Continuous Integration}

The main aim of this chapter is to introduce the concept of Continuous Integration (CI) and its significance in modern software development. However, before diving into CI, a good understanding of build engineering is essential, as it forms the foundation for effective CI practices.

\section{Build Engineering}

Build engineering refers to the process of compiling source code into executable programs. This process should ideally be automated to ensure consistency, efficiency, and reliability. It is crucial in order to increase productivity and reduce human error during the build process.\\

To start with the building process, engineering teams typically start from available scripts (e.g., Ant, Maven, Make) that automate the build process. However, these scripts often need to be adapted to support Quality Assurance (QA) and deployment on production systems. This is where the role of a Build Engineer becomes vital.\\

There are usually two types of methologies for build engineering:
\begin{description}
    \item[Using IDEs]: Integrated Development Environments (IDEs) provide built-in tools for building and managing projects. However, using them can lead to inconsistencies, as different developers may have different IDE configurations.
    
    \item[Command-Line Build]: Command-line build is usually preferred in professional environments. It allows for greater control and automation, ensuring that builds are consistent across different environments. It also lets the build scripts be version-controlled alongside the source code.
\end{description}

One important aspect of build engineering is the security of the build process. In order to ensure that the build process is secure, there are three concepts that need to be taken into account:
\begin{itemize}
    \item \textbf{Automation}: The build process should be fully automated to minimize human intervention and reduce the risk of errors. These scripts should also follow the ``Failing Fast'' principle, which means that if an error occurs during the build process, it should stop immediately and report the error.
    \item \textbf{Secure Supply Chain}: The build process should ensure that all dependencies and components used in the build are secure and trustworthy. This includes verifying the integrity of third-party libraries and tools. Isolated build environments (e.g., using containers) can help mitigate risks associated with compromised dependencies.
    \item \textbf{Secure Trusted Base}: In the event of cyberattacks, it is crucial to accurately identify the software that has been compromised. This includes knowing which version of the software was deployed and whether it was deployed correctly. Methods for achieving this include using version numbers, hash functions, and creating a Manifest file that contains all configuration parameters.
\end{itemize}


\section{Continuous Integration (CI)}