\section{Secure Deployment}

\begin{ejercicio}
    Die \emph{Echtzeit-Messaging-App} für den Campus der ``Universität der Zukunft'' soll regelmäßig aktualisiert werden, dabei aber sichere Deployments garantiert werden.
    \begin{enumerate}
        \item Erklären Sie kurz die Begriffe hermetic build, reproducible build und verifiable build.
        
        The building process should be hermetic, reproducible, and verifiable.
        \begin{itemize}
            \item \ul{Hermetic}: The build process is isolated from the outside environment, ensuring that it does not rely on any external factors that could introduce variability or security risks.
            \item \ul{Reproducible}: The build process can be repeated with the same inputs and produce the same outputs, allowing for consistent and reliable builds.
            \item \ul{Verifiable}: The build process can be independently verified to ensure that it has not been tampered with and that the resulting build artifacts are trustworthy.
        \end{itemize}
        \item Nennen Sie zwei typische Fallstricke, die reproducible Builds verhindern.
        
        \begin{itemize}
            \item \ul{Timestamps}: If the build process includes timestamps, it can lead to non-reproducible builds, as the output will differ each time due to the changing timestamps.
            \item \ul{Non-deterministic inputs}: If the build process relies on non-deterministic inputs, such as random numbers or environment variables, it can also lead to non-reproducible builds.
        \end{itemize}
    \end{enumerate}
\end{ejercicio}

\begin{ejercicio}
    Die Universität überlegt, eine eigene CA für interne Services einzurichten.
    \begin{enumerate}
        \item Erklären Sie, wofür eine CA benötigt wird, und welche Rolle Zertifikate beim sicheren Schlüsselaustausch spielen.
        
            A CA (Certificate Authority) is needed to issue digital certificates that verify the identity of entities (such as servers, users, or devices) and facilitate secure communication. Certificates play a crucial role in secure key exchange by providing a way to establish trust between parties, as they garantee that the public key contained in the certificate belongs to the entity it claims to represent. This allows for secure communication through encryption and authentication.
        \item Skizzieren Sie den Ausstellungsprozess eines Zertifikats.
        
            The certificate issuance process typically involves the following steps:
            \begin{enumerate}
                \item The entity (e.g., a server) generates a public-private key pair and creates a Certificate Signing Request (CSR) that includes the public key and identifying information about the entity.
                \item The CSR is sent to the CA, which verifies the identity of the entity through various means (e.g., email verification, domain ownership verification).
                \item Once the CA is satisfied with the verification, it signs the CSR with its private key, creating a digital certificate that binds the public key to the entity's identity.
                \item The signed certificate is then returned to the entity, which can use it for secure communication and authentication.
            \end{enumerate}
        \item Nennen Sie drei Risiken beim Betrieb einer CA und mögliche Gegenmaßnahmen.
        
            \begin{itemize}
                \item \ul{Compromise of the CA's private key}: If an attacker gains access to the CA's private key, they can issue fraudulent certificates. To mitigate this risk, the CA's private key should be stored securely, such as in a hardware security module (HSM), and access should be strictly controlled.
                \item \ul{Misissuance of certificates}: If the CA issues certificates to unauthorized entities, it can lead to security breaches. To prevent this, the CA should implement strict verification processes for certificate requests and regularly audit issued certificates.
                \item \ul{Revocation of certificates}: If a certificate is compromised or no longer valid, it needs to be revoked. The CA should have a robust certificate revocation mechanism in place, such as a Certificate Revocation List (CRL) or Online Certificate Status Protocol (OCSP), to ensure that revoked certificates are not trusted.
            \end{itemize}
    \end{enumerate}
\end{ejercicio}

\begin{ejercicio}
    Angenommen, ein Angreifer hat Zugriff auf einen CI-Runner erlangt, der Builds ausführt.
    \begin{enumerate}
        \item Welche zwei Angriffe sind dadurch besonders naheliegend?
        
            \begin{itemize}
                \item \ul{Injection of malicious code}: The attacker could inject malicious code into the build process, which could then be included in the final build artifacts and potentially distributed to users.
                \item \ul{Exfiltration of sensitive information}: The attacker could access sensitive information such as API keys, credentials, or proprietary code that is used during the build process.
            \end{itemize}
        \item Welche Sofortmaßnahme (erste 24h) würden Sie einleiten?
        
            The immediate response would be to isolate the compromised CI-Runner to prevent further damage. This could involve taking the runner offline, revoking any credentials that were used on that runner, and conducting a thorough investigation to determine the extent of the breach and identify any potential vulnerabilities that were exploited. Additionally, it would be important to communicate with stakeholders about the incident and implement measures to prevent similar attacks in the future.
    \end{enumerate}
\end{ejercicio}

\begin{ejercicio}
    Schauen Sie sich \myhref{https://slsa.dev/}{SLSA} an. Worum handelt es sich dabei?

    SLSA (Supply-chain Levels for Software Artifacts) is a security framework that provides a set of guidelines and best practices for securing the software supply chain. It defines different levels of security assurance for software artifacts, ranging from basic to advanced, based on the measures taken to ensure the integrity and authenticity of the software throughout its lifecycle. The framework aims to help organizations improve the security of their software supply chain and reduce the risk of vulnerabilities and attacks.
\end{ejercicio}