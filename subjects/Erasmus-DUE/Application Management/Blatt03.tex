\tikzset{
    git/commit/.style={
        circle, 
        draw, 
        minimum size=1.2cm, 
        font=\small\ttfamily
    },
    git/ref/.style={
        rectangle split, 
        rectangle split parts=2, 
        draw, 
        minimum width=1.5cm,
        font=\small\ttfamily,
        align=center,
        fill=white % Para que no se transparente si pasa sobre una línea
    },
    git/history/.style={
        -{Stealth}, 
        thick
    },
    git/pointer/.style={
        -{Stealth}, 
        dashed, 
        thick
    }
}


\section{Distributed Git und Internals}

\begin{ejercicio}
    In Aufgaben~\ref{ej:git-branching}.\ref{itm:git-branching-merge} und \ref{ej:git-prepare-mr} des vorherigen Aufgabenblatts sollten Sie die Änderungen des master-Branches in den aktuellen Feature-Branch übernehmen. Überlegen Sie sich eine weitere Möglichkeit, die Änderungen zu übernehmen.
\end{ejercicio}


\begin{ejercicio}
    Überlegen Sie sich mögliche Vor- und Nachteile der drei vorgestellten Distributed Workflows.
    \begin{enumerate}
        \item Dictator and Lieutenants Workflow
        \item Integration-Manager Workflow
        \item Centralized Workflow
    \end{enumerate}
\end{ejercicio}

\begin{ejercicio}
    Der Chef des Teams, zuständig für die Entwicklung der Echtzeit-Messaging-App hat wenig Ahnung von Softwareentwicklung. Er hat damals einfach irgendwelche Regeln bezüglich des Merge-Prozesses festgelegt, weiß aber nicht wirklich, was diese bedeuten, und bittet Sie, einen sinnvollen Workflow für das Projekt zu wählen. Wählen Sie einen passenden Workflow aus und begründen Sie Ihre Wahl.
\end{ejercicio}

\begin{ejercicio}
    Ihr Kollege bittet Sie erneut um Hilfe. Dieses mal sei es ernst, er hat all seine Arbeit der letzten Wochen verloren. Da er sich nicht gut mit Git auskennt, committet er selten. Als er fertig war, wollte er committen. Dafür hat er seine Änderungen mit \verb|git add -A| in den Staging-Bereich gepackt. Doch da ist ihm eingefallen, dass er vorher noch Änderungen vom Remoteserver herunterladen muss. Also führt er git fetch aus und sieht, dass Remote-Änderungen übernommen wurden. Doch als er die Änderungen dann endlich in seinem lokalen Branch hat, sind alle seine eigenen Änderungen verschwunden. Helfen Sie Ihm die verlorenen Dateien wiederherzustellen, nutzen Sie dafür das im Moodle bereitgestellte Repository.
\end{ejercicio}


\begin{ejercicio}\label{ej:git-graphs}
    Nehmen Sie an, Sie haben ein Git-Repository, welches die in Abbildung~\ref{fig:git-repo-initial} dargestellte Commit-Graphen hat. Nehmen Sie nun an, Sie führen die untenstehenden Git-Befehle aus. Zeichnen Sie den Commit-Graphen nach jedem Befehl. Wenn ein neuer Commit-Hash berechnet wurde, wählen Sie bitte eine eindeutige zufällige Nummer.
    \begin{figure}
        \centering
        \begin{tikzpicture}
            \node[git/commit] (c1) {4b6f56};
            \node[git/commit] (c2) [right=1.5cm of c1] {402a33};
            \node[git/commit] (c3) [right=1.5cm of c2] {8dd752};

            \node[git/ref] (head) [above=0.8cm of c3] {
                HEAD
                \nodepart{second} master
            };

            \draw[git/history] (c2) -- (c1);
            \draw[git/history] (c3) -- (c2);
            \draw[git/pointer] (head) -- (c3);
        \end{tikzpicture}
        \caption{Git-Repository mit drei Commits und einem Branch \texttt{master}.}
        \label{fig:git-repo-initial}
    \end{figure}
    \begin{enumerate}
        \item \verb|git checkout HEAD~1|
        \item \verb|git checkout -b 'feature_branch'|
        \item \verb|git commit -m 'new feature'|
        \item Zeichnen Sie bitte beide Graphen
        \begin{enumerate}
            \item \verb|git merge master|
            \item \verb|git rebase master|
        \end{enumerate}
        \item Führen Sie abschließend für beide Graphen einen Merge von \texttt{feature\_branch} in den \texttt{master} aus und löschen Sie den obsoleten Branch.
    \end{enumerate}
\end{ejercicio}

\begin{ejercicio}
    Nutzen Sie das Git-Repository aus Moodle.
    \begin{enumerate}
        \item Angenommen, Sie haben gerade Ihre neuesten Änderungen committet und möchten vor dem Pushen testen, ob alles noch funktioniert. Dabei fällt Ihnen auf, dass ein Anführungszeichen fehlt. Sie haben es bereits hinzugefügt, finden es jedoch unnötig, dafür einen neuen Commit anzulegen. Fügen Sie die Änderung Ihrem lokalen Commit hinzu.
        \item Nachdem Sie die Änderung vorgenommen haben, stellen Sie fest, dass die Commit-Nachricht nicht alle Änderungen widerspiegelt. Sie möchten der Nachricht hinzufügen, dass die neue Funktion auch in \verb|main.py| aufgenommen wurde.
    \end{enumerate}
\end{ejercicio}