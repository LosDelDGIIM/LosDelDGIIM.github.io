\section{Secure Deployment 2}

\begin{ejercicio}
    Erläutern Sie, inwiefern die folgenden Maßnahmen vor einem Benign Insider oder einem Malicious Adversary absichern. Gegen welche der beiden Arten von Akteuren sind die Maßnahmen effektiver und warum?
    \begin{enumerate}
        \item Code Reviews
        
        This measure is more effective against a Benign Insider, as it can help identify unintentional mistakes or oversights in the code that could lead to security vulnerabilities.
        \item Geheimnisse schützen (durch Key-Management-Systeme, Zugriffskontrollen etc.)
        
        This measure is effective against both Benign Insiders and Malicious Adversaries, as it helps prevent unauthorized access to sensitive information. However, it may be more effective against Malicious Adversaries, as they are more likely to intentionally seek out and exploit secrets for malicious purposes.
        \item Automatisierung der CI/CD-Pipeline
        
        This measure can help reduce the risk of human error and improve the consistency and reliability of the deployment process. It can be effective against both Benign Insiders and Malicious Adversaries, as it can help prevent mistakes and ensure that security best practices are followed. However, it may be more effective against Benign Insiders, as they are more likely to make unintentional mistakes that could lead to security vulnerabilities.
        \item Verifiable Builds
        
        This measure is more effective against a Malicious Adversary, as it allows for the verification of the integrity and authenticity of the build process. It can help detect any tampering or unauthorized modifications to the build, which is more likely to be done by a Malicious Adversary with malicious intent.
    \end{enumerate}
\end{ejercicio}

\begin{ejercicio}~
    \begin{enumerate}
        \item Welche Daten sollten in einer Binary Provenance für unsere Echtzeit-Messaging-App des Campus der ``Universität der Zukunft'' enthalten sein?
        
        In a Binary Provenance for our real-time messaging app, we should include the following data:
        \begin{itemize}
            \item Source code repository URL and commit hash
            \item Build environment details (e.g., operating system, compiler version, dependencies)
            \item Build configuration and parameters
            \item Build timestamp
            \item List of all dependencies and their versions
            \item Hashes of the built binaries for integrity verification
            \item Information about the build process (e.g., build scripts used, build tools)
        \end{itemize}
        \item Welche Verbindung besteht zwischen Hermetic, Reproducible und Verifiable Builds und Binary Provenance?
        
        Hermetic, Reproducible, and Verifiable Builds are closely related to Binary Provenance as they all contribute to ensuring the integrity and authenticity of the build process. Hermetic builds ensure that the build process is isolated from external factors, Reproducible builds ensure that the same source code and build environment will produce the same binary, and Verifiable builds allow for the verification of the build process. Binary Provenance provides the necessary data to support these concepts by documenting the details of the build process, allowing for traceability and accountability in software development.
    \end{enumerate}
\end{ejercicio}