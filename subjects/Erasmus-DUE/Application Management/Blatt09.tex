\section{Secure Development}

\begin{ejercicio}~
    \begin{enumerate}
        \item Erläutern Sie, was ein Security Champion ist.
        
        A security champion is a member of the development team who takes on the responsibility of promoting and advocating for security best practices within the team. They act as a liaison between the security team and the development team, helping to ensure that security considerations are integrated into the software development process. Security champions are typically passionate about security and have a good understanding of secure coding practices, threat modeling, and vulnerability management. They play a crucial role in fostering a culture of security awareness and helping to identify and mitigate security risks early in the development lifecycle.
        \item Welche Vor- und Nachteile hat dieses Konzept?
        
        The advantages of having a security champion include:
        \begin{itemize}
            \item Improved Security Awareness: Security champions can help raise awareness about security issues among the development team, leading to better security practices and a stronger security culture.
            \item Early Identification of Security Issues: By being involved in the development process, security champions can identify potential security issues early on, which can save time and resources in the long run.
            \item Better Communication: Security champions can facilitate communication between the security team and the development team, ensuring that security requirements are clearly understood and implemented.
        \end{itemize}
        However, there are also some disadvantages to consider:
        \begin{itemize}
            \item Additional Responsibility: Being a security champion can add extra responsibilities to the individual's workload, which may lead to burnout if not managed properly.
            \item Responsibilities Shift: Members of the team may rely too heavily on the security champion to handle all security-related issues, which can lead to a lack of shared responsibility for security across the team.
            \item Skill Requirements: Security champions need to have a good understanding of security concepts and practices, which may require additional training and development for the individual taking on this role.
        \end{itemize}
    \end{enumerate}
\end{ejercicio}

\begin{ejercicio}
    Nehmen Sie wieder an, Sie seien Entwickler im Team der Echtzeit-Messaging-App der ``Universität der Zukunft''. Ihr Team denkt über das Thema Sicherheit nach und möchte das \verb|OWASP SAMM| einführen.
    \begin{enumerate}
        \item Schauen Sie sich das \myhref{https://owaspsamm.org/model/}{\texttt{OWASP SAMM}} an.
        \item Überlegen Sie sich verschiedene Maßnahmen auf unterschiedlichen Ebenen, um eine sichere Benutzerauthentifizierung sicherzustellen. (Mögliche Hilfestellung kann das \myhref{https://cheatsheetseries.owasp.org/index.html}{\texttt{OWASP CheatSheet}} bieten.)
        
        Some measures to ensure secure user authentication could include:
        \begin{itemize}
            \item Implementing multi-factor authentication (MFA) to add an extra layer of security.
            \item Using strong password policies, such as requiring a minimum length and complexity for passwords.
            \item Implementing account lockout mechanisms to prevent brute-force attacks.
            \item Regularly reviewing and updating authentication mechanisms to address new threats and vulnerabilities.
        \end{itemize}
        \item Überlegen Sie, zu welcher Domäne des \verb|OWASP SAMM|s die Maßnahmen passen und inwiefern Sie den Reifegrad verbessern können.
        
        The measures mentioned above would primarily fall under the "Implementation" domain of the OWASP SAMM, which focuses on secure coding and code review. By implementing these measures, we can improve the maturity level of our authentication practices, moving from a basic level (where security is an afterthought) to a more advanced level (where security is integrated into the development process). This would help us to better protect our users' accounts and reduce the risk of unauthorized access.
        

    \end{enumerate}
\end{ejercicio}