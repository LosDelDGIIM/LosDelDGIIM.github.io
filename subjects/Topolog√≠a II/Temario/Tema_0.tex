\chapter*{Tema 0. Conexión por arcos}

\section{Conexión}

\begin{notacion}
    Notaremos por e.t al espacio topológico $(X,\cc{T})$ o diremos $X$ es un e.t.
\end{notacion}

\begin{definicion}
    Se dice que un e.t $X$ es {no} conexo si existen $U$ y $V$ abiertos disjuntos y no vacíos tales que $X=U\cup V$. 
\end{definicion}

\begin{prop}
    Dado un e.t. $X$ equivalen las siguientes afirmaciones:
    \begin{enumerate}
        \item[(i)] $X$ es conexo.
        \item[(ii)] Los únicos subconjuntos de $X$ que son abiertos y cerrados a la vez son el vacío y el total. 
        \item[(iii)] Los únicos subconjuntos de $X$ con frontera vacía son el vacío y el total. 
    \end{enumerate}
\end{prop}

\begin{teo}
    El ser conexo se conserva por aplicaciones continuas. En particular, ser conexo es una propiedad topológica (se conserva por homeomorfismos).
\end{teo}

\begin{teo}
    La unión de una colección de subconjuntos conexos que tienen un punto común de un e.t. $X$ es también conexa.
\end{teo}

\begin{teo}
    Si $A$ es un subconjunto del e.t. $X$ y $A$ es conexo, entonces dado $B$ con $A\subset B\subset\overline{A}$, entonces se tiene que $B$ también es conexo. En particular, la adherencia de un conexo siempre es un conjunto conexo.
\end{teo}

\begin{teo}
    Dados dos espacios topológicos $X,Y$ se cumple que $X\times Y$ es conexo (con la topología producto) si y solo si $X$ e $Y$ son conexos.
\end{teo}

\begin{teo}
    Los conjuntos conexos de $\bb{R}$ con la topología usual son exactamente los intervalos (incluyendo los puntos).
\end{teo}

\begin{definicion}
    Dados un e.t. $X$ y un punto $x_0$ se define la componente conexa de $x_0$ es $X$ como el mayor conexo de $X$ que contiene a $x_0$
\end{definicion}

\begin{teo}
    Las componentes conexas de un e.t. $X$ forman una partición de $X$ es conjuntos conexos maximales y cerrados.
\end{teo}

\section{Conexión por arcos}

\begin{definicion}
    Un \textbf{arco} (o camino) en un espacio topológico $X$ es una aplicación continua $\alpha:[0,1]\to X$. Si además $\alpha(0) = \alpha(1)$ diremos que $\alpha$ es un lazo.\\

    Diremos que un arco $\alpha:[0,1]\to X$ une $x$ con $y$ si se verifica que $\alpha(0)=x$ y $\alpha(1) = y$. Si $\alpha$ es un arco, diremos que está basado en $x$ (o su punto base es $x$) si $\alpha(0)=x=\alpha(1)$.\\

    Denotaremos por 
    \begin{gather*}
        \Omega(X;x,y) = \{\alpha:[0,1]\to X \text{ continua } : \alpha(0)=x, \ \ \alpha(1)=y\}
    \end{gather*}
    al \textbf{conjunto de arcos} que unen $x$ con $y$. Denotaremos además por  
    \begin{gather*}
        \Omega(X;x) = \{\alpha:[0,1]\to X \text{ continua } : \alpha(0)=x=\alpha(1)\}
    \end{gather*}
    al \textbf{conjunto de lazos} basados en $x$.
\end{definicion}

\begin{ejemplo}\
    \begin{enumerate}
        \item Dados un e.t. $X$ y un punto $x_0\in X$ siempre se tiene que
        \begin{align*}
            \veps_{x_0}:[0,1]&\to X\\
            t &\mapsto x_0
        \end{align*}
        es un lazo basado en $x_0$ al que llamaremos \textbf{arco constante}. De hecho, si $X$ tiene la topología discreta, entonces los únicos arcos que hay en $X$ son los arcos constantes.
        \begin{proof}
            Si $X$ tiene la topología discreta, entonces como $\alpha$ es continua $\alpha^{-1}(\{x_0\})$ será abierto y cerrado y por tanto $\alpha^{-1}(\{x_0\})\in\{\emptyset, X\}$ por ser $X$ conexo.
        \end{proof}


        \item Sean $x,y,z\in X$, $\alpha:[0,1]\to X$ un arco uniendo $x$ con $y$ y $\beta:[0,1]\to X$ un arco uniendo $y$ con $z$. Buscaremos ahora un arco formado a partir de estos dos de la siguiente forma:
        \begin{align*}
            \alpha\ast\beta : [0,1]\to X : (\alpha\ast\beta)(t) = \left\{ 
                \begin{array}{l c c}
                    \alpha(2t) & \text{ si } & 0 \leq t \leq \nicefrac{1}{2} \\
                    \beta(2t-1) & \text{ si } & \nicefrac{1}{2}\leq t \leq 1
                \end{array}
            \right.
        \end{align*}
        Entonces $\alpha\ast\beta$ es continua ya que $(\alpha\ast\beta)_{|_{[0,\nicefrac{1}{2}]}}$ y $(\alpha\ast\beta)_{|_{[\nicefrac{1}{2},1]}}$ lo son y para $t=\nicefrac{1}{2}$ se tiene que
        \begin{gather*}
            \alpha \left(2 \cdot \frac{1}{2}\right) = \alpha(1) = \beta(0) = \beta\left(2\cdot \frac{1}{2} -1\right)
        \end{gather*}
        con $\left[0, \frac{1}{2}\right]$ y $\left[\frac{1}{2}, 1\right]$ cerrados. Aplicando el lema de pegado\footnote{visto en Topología I} tenemos que $\alpha\ast \beta$ es continua.

        \item Si $\alpha:[0,1]\to X$ es un arco uniendo $x$ con $y$, entonces
        \begin{align*}
            \tilde{\alpha}:[0,1]&\to X\\
            t & \mapsto \alpha(1-t)
        \end{align*}
        es un arco que une $y$ con $x$.
    \end{enumerate}
\end{ejemplo}

\begin{definicion}
    Decimos que un e.t. $X$ es \textbf{arcoconexo} (o \textbf{conexo por arcos}) si para cualesquiera $x,y\in X$ existe un arco en $X$ que une el punto $x$ con el punto $y$.\\

    Si $X$ es un e.t. y $A\subset X$, diremos que $A$ es arcoconexo si $A$ es arcoconexo con la topología de inducida de $X$.
\end{definicion}

\begin{teo}
    Todo espacio topológico arcoconexo es conexo.
    \begin{proof}
        Dado $x_0\in X$ fijo y otro punto $x\in X$ cualquiera, sabemos que existe $\alpha:[0,1]\to X$ un arco tal que $\alpha(0)=x_0$ y $\alpha(1)=x$. En particular, como el intervalo $[0,1]$ es conexo y $\alpha$ es continua, entonces se tiene que $\alpha([0,1])$ es conexo y podremos escribir 
        \begin{gather*}
            X=\bigcup\limits_{x\in X}\{x\}\subseteq \bigcup\limits_{x\in X} \alpha_x ([0,1])\in X \Rightarrow X = \bigcup\limits_{x\in X}\alpha_x\{[0,1]\}
        \end{gather*}
        y además $x_0 \in \bigcap\limits_{x\in X}\alpha_x\{[0,1]\}$ por lo que $X$ es conexo (por el lema del peine).
    \end{proof}
\end{teo}

\begin{ejemplo}
    Veamos que la otra implicación no es cierta en general. Para ello consideramos los siguientes conjuntos: 
    \begin{gather*}
        X_0=\{1\}\times [0,1] \hspace{0.5cm}\text{ y } \hspace{0.5cm} X_n = [0,1]\times \left\{\frac{1}{n}\right\},\ n\in \bb{N}
    \end{gather*}
    Llamamos $X=\{(0,0)\}\cup \left(\bigcup\limits_{n\in \bb{N}\cup \{0\}}X_n\right)$ y queremos ver que $X$ es conexo pero no es arcoconexo.\\

    Si tomamos la segunda parte de esta unión, es decir, 
    \begin{gather*}
        Y=\bigcup\limits_{n\in \bb{N}\cup \{0\}}X_n
    \end{gather*}
    tenemos que $Y$ es es conexo porque es unión de los $X_n$ que son todos conexos y que intersecan con $X_0$. Entonces, como $Y\subset X \subset \overline{Y}$ tenemos que $X$ es conexo. Veamos sin embargo que $X$ no es arcoconexo.\\

    Para ello vamos a demostrar que si $\alpha:[0,1]\to X : \alpha$ es continua con $\alpha(0)=(0,0)$, entonces $\alpha(t)=(0,0)$ para todo $t\in [0,1]$.\\

    Podemos escribir la curva como $\alpha(t) = (x(t), y(t))\in \bb{R}^2$. Como $\alpha(0)=(0,0)$, si tomamos $((-\nicefrac{1}{2}, \nicefrac{1}{2})\times (-\nicefrac{1}{2}, \nicefrac{1}{2})) \cap X$ un abierto que contiene al origen, entonces $\exists \veps > 0$ tal que $\alpha([0,\veps))\subseteq ((-\nicefrac{1}{2}, \nicefrac{1}{2})\times (-\nicefrac{1}{2}, \nicefrac{1}{2})) \cap X$ por ser $\alpha$ continua. Como $y(t)$ es continua y se tiene que $y([0,\veps))\subseteq \{0\}\cup\left(\bigcup\limits_{n>2} \left\{\frac{1}{n}\right\}\right)$. Por el teorema del valor intermedio tenemos que $y([0,\veps)) = \{0\}$ por lo que $\alpha([0,\veps))=\{(0,0)\}$. De esta forma hemos probado que no hay ningún arco que conecte $(0,0)$ con un punto distinto (el único arco es el constante).
\end{ejemplo}

\begin{teo}
    Si $A\subseteq \bb{R}^n$ es estrellado, entonces $A$ es arcoconexo.
    \begin{proof}
        Como $A$ es estrellado existe un $x_0\in A$ tal que para cualquier $x\in A$, el segmento que los une, $(1-t)x + tx_0\in A$ para todo $t\in [0,1]$ y entonces $\alpha(t)=(1-t)x+tx_0$ es una curva continua uniendo $x$ con $x_0$. Dados dos puntos cualesquiera $x,y \in A$ se tiene que $\alpha_x\ast \tilde{\alpha}_y$ es una curva continua que une $x$ con $y$.
    \end{proof}
\end{teo}

\begin{coro}
    Cualquier conjunto $A$ de $\bb{R}^n$ convexo es arcoconexo. Por ejemplo, las bolas abiertas o las bolas cerradas de $\bb{R}^n$.
\end{coro}

\begin{coro}
    En $\bb{R}$ coinciden los conjuntos conexos y arcoconexos (son solo los intervalos).
\end{coro}

\begin{teo}
    La imagen mediante una aplicación continua de un arcoconexo es un arcoconexo. En particular, ser arcoconexo es una propiedad topológica, es decir, se conserva por homeomorfismos.
    \begin{proof}
        Dados $x,y\in f(X)$, entonces existen $x_0,y_0\in X$ tal que $f(x_0)=x$ y $f(y_0) = y$. Por ser $X$ arcoconexo, entonces existe un arco $\alpha:[0,1]\to X$ con $\alpha(0)=x_0$ y $\alpha(1)=y_0$. Entonces tenemos que $f\circ \alpha: [0,1]\to f(X)$ es continua y $(f\circ \alpha)(0) = f(x_0) = x$ y $(f\circ\alpha)(1) = f(y_0) = y$ y tenemos demostrado el resultado que buscábamos.
    \end{proof}
\end{teo}

\begin{teo}
    Sean $X$ un e.t y $\{A_i\}_{i\in I}$ una familia de arcoconexos de $X$. Si $\bigcap\limits_{i\in I}A_i \neq \emptyset$, entones $\bigcup\limits_{i\in I}A_i$ es arcoconexo.
\end{teo}

\begin{observacion}
    Hay resultados de conexión que no son ciertos para arcoconexión. Por ejemplo, si en un e.t. $X$ se tiene que $A\subseteq X$ es arcoconexo podría ocurrir que si $A\subseteq B \subseteq \overline{A}$, se diera que $B$ no sea arcoconexo (como en el ejemplo anterior).
\end{observacion}

\begin{ejemplo}
    Veamos que $\bb{S}^n = \{x\in \bb{R}^{n+1} : |x| = 1\}$ es arcoconexo. Podemos hacerlo sabiendo que $\bb{S}^n\setminus \{N\}$, con $N=(0,\dots,0,1)$ es homeomorfo a $\bb{R}^n$ que es arcoconexo por lo que $\bb{S}^n \setminus\{N\}$ es arcoconexo. Análogamente, $\bb{S}^n \setminus\{S\}$ con $S=(0,\dots,0,-1)$ es arcoconexo y podemos escribir
    \begin{gather*}
        \bb{S}^n = (\bb{S}^n \setminus\{N\}) \cup (\bb{S}^n \setminus\{S\})
    \end{gather*}
    por lo que $\bb{S}^n$ es unión de arcoconexos con puntos en común, luego es arcoconexo.
\end{ejemplo}

\begin{teo}
    Sean $X,Y$ e.t., entonces $X\times Y$ es arcoconexo (con la topología producto) si y solo si $X$ e $Y$ son arcoconexos.
\end{teo}

\begin{teo}
    Dado un e.t. $X$, se tiene que $X$ es arcoconexo si y solo si $X$ es conexo y todo punto $x\in X$ tiene un entorno suyo arcoconexo.

    \begin{proof}\
        \begin{itemize}
            \item[$\Rightarrow$)] Hemos visto que si $X$ es arcoconexo, entonces $X$ es conexo. Además, $X$ es entorno de cualquier punto suyo luego todo punto tiene un entorno conexo.
            \item[$\Leftarrow$)] Elegimos un $x\in X$ fijo y definimos $A = \{y\in X : \exists \alpha_y \text{ arco uniendo $x$ con $y$ }\}$
            
            Como $x\in A$ tenemos que $A\neq \emptyset$. Si probamos que $A$ es abierto y cerrado, entonces como $X$ es conexo tendremos que $A=X$, es decir podremos unir $x$ con cualquier otro punto $y\in X$. Esto se deja como ejercicio para el lector.\\
        \end{itemize}
    \end{proof}
\end{teo}