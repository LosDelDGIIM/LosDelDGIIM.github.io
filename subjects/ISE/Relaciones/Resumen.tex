\chapter{Resumen de fórmulas}
Este apartado trata de ser un resumen de la mayoría de las fórmulas vistas en cada tema de la asignatura, está pensado para recordar de forma rápida las fórmulas vistas en la asignatura, y no debe usarse como medio de estudio.

\section{Introducción a la Ingeniería de Servidores}
Los dos conceptos más importantes del tema son:
\begin{itemize}
    \item Tiempo de respuesta (latencia).
    \item Productividad (ancho de banda).
\end{itemize}
Si queremos comparar dos dispositivos $A$ y $B$ con tiempos $t_A$ y $t_B$, la ganancia en velocidad de $A$ respecto a $B$ viene dada por:
\begin{equation*}
    S_B(A) = \dfrac{t_A}{t_B}
\end{equation*}
Si tenemos un dispositivo que tarda en ejecutar una tarea un tiempo $T_0$ y mejoramos dicho dispositivo sustituyendoo un compoente que se usa durante una fracción de tiempo $f\in [0,1]$ de forma que mejoramos dicho componentes $k$ veces, el nuevo tiempo vendrá dado por:
\begin{equation*}
    T_M = (1-f)T_0 + \dfrac{fT_0}{k}
\end{equation*}
Bajo las mismas condiciones, la Ley de Amdahl nos dice que:
\begin{equation*}
    S = \dfrac{T_0}{T_M} = \dfrac{T_0}{(1-f)T_0 + \dfrac{fT_0}{k}} = \dfrac{1}{1-f+\dfrac{f}{k}}
\end{equation*}

\setcounter{section}{3}
\section{Análisis comparativo del rendimiento}
Si sabemos las instrucciones de un programa ($NI$), cuántas de ellas son en coma flotante ($FL$) y el tiempo que tarda un dispositivo en ejecutar el programa ($t$), podremos calcular:
\begin{align*}
    \text{MIPS} &= \dfrac{NI}{t\cdot 10^6} \\
    \text{MFLOPS} &= \dfrac{FL}{t\cdot 10^6}
\end{align*}
Si para un cierto programa conocemos el número de instrucciones necesarias para su ejecución ($NI$), el número medio de ciclos por instrucción de la CPU ($CPI$) y la frecuencia del procesador ($f$), podremos calcular el tiempo que tarda la CPU en ejecutar el programa, mediante:
\begin{equation*}
    T_{CPU} = NI\cdot CPI \cdot \dfrac{1}{f}
\end{equation*}
Si al ejecutar un benchmark de $n$ programas obtenemos las puntuaciones $t_1,t_2,\ldots,t_n$ y las puntuaciones de referencia eran $t_{REF_1}, t_{REF_2}, \ldots, t_{REF_n}$, podemos calcular el índice SPEC mediante:
\begin{equation*}
    SPEC = \sqrt[n]{\dfrac{t_{REF_1}}{t_1} \cdot \dfrac{t_{REF_2}}{t_2} \cdot \ldots \cdot \dfrac{t_{REF_n}}{t_n}}
\end{equation*}

\subsection{Distribución t-Student}
Si extraemos $n$ muestras de ejecución de varios programas por dos dispositivos o programas distintos y consideramos la diferencia de los datos obtenidos: $d_1,d_2,\ldots,d_n$, estamos interesados en calcular si las dos muestras tienen diferencias significativas con un grado de significatividad mayor al $95\%$. Para ello, lo que haremos será suponer la hipótesis nula $H_0$:
\begin{center}
    Las dos máquinas/programas tienen rendimientos equivalentes
\end{center}
Definimos:
\begin{itemize}
    \item La media de las diferencias:
        \begin{equation*}
            \ol{d} = \dfrac{1}{n}\sum_{i=1}^{n}d_i
        \end{equation*}
    \item La desviación típica muestral:
        \begin{equation*}
            s = \sqrt{\dfrac{\sum\limits_{i=1}^{n}{(d_i - \ol{d})}^{2}}{n-1}}
        \end{equation*}
    \item El error estándar:
        \begin{equation*}
            \dfrac{s}{\sqrt{n}}
        \end{equation*}
\end{itemize}
Supuesta la hipótesis nula $H_0$, tras un estudio podremos rechazar la hipótesis nula (por lo que los datos no son significativamente distintos) o no podremos rechazarla. Este estudio se puede realizar de 3 formas distintas. Fijado un grado de significatividad usualmente de $\alpha=0.5$ (para el $95\%$):
\begin{enumerate}
    \item Una vez calculados $\ol{d},s$ y el error estándar, calculamos:
        \begin{equation*}
            t_{exp} = \dfrac{\ol{d}}{\nicefrac{s}{\sqrt{n}}}
        \end{equation*}
        Tras lo cual podemos calcular el $p-$value:
        \begin{equation*}
            P(|t| \geq |t_{exp}|)
        \end{equation*}
        Donde consideramos la distribución de probabilidad de $t-$Student de $n-1$ grados de libertad. Si $P(|t|\geq |t_{exp}|) < \alpha$, entonces podremos rechazar $H_0$.
    \item Conocidos $\alpha$ y $n$, si calculamos $t_{\frac{\alpha}{2},n-1}$ que cumple:
        \begin{equation*}
            P(|t| \geq |t_{\frac{\alpha}{2},n-1}|) = \alpha
        \end{equation*}
        Para una distribución de $t-$Student de $n-1$ grados de libertad, si calculamos $t_{exp}$ y:
        \begin{equation*}
            t_{exp} \notin \left[-t_{\frac{\alpha}{2},n-1}, t_{\frac{\alpha}{2},n-1}\right]
        \end{equation*}
        Entonces podremos rechazar $H_0$.
    \item Si calculamos $t_{\frac{\alpha}{2},n-1}$, si:
        \begin{equation*}
            0\notin \left[\ol{d}-\frac{s}{\sqrt{n}} \cdot t_{\frac{\alpha}{2},n-1},~\ol{d}+\frac{s}{\sqrt{n}} \cdot t_{\frac{\alpha}{2},n-1}\right]
        \end{equation*}
        Entonces podremos rechazar $H_0$.
\end{enumerate}

\section{Optimización del rendimiento}
Fórmulas pendientes de subir.
