\setcounter{section}{2}
\section{Monitorización}

\begin{ejercicio}
    En un sistema Linux se ha ejecutado la orden \verb|uptime| tres veces en momentos diferentes. El resultado, de forma resumida, se muestra en el siguiente listado:
    \begin{minted}[linenos=false]{shell}
... load average: 6.85, 7.37, 7.83
... load average: 8.50, 10.93, 8.61
... load average: 37.34, 9.47, 3.30
    \end{minted}

    Indique si la carga crece, decrece, se mantiene estacionaria o bien no puede decidir sobre ello.\\

    No hay una tendencia clara en los valores de las medidas, por lo que no podemos concluir nada realmente. Sería necesario saber cuánto tiempo ha pasado entre cada ejecución para estudiar así en detalle la situación. Tan solo podríamos afirmar que, en el último instante, la carga ha crecido de forma notable respecto a las dos anteriores.
\end{ejercicio}
\begin{comment}
    Solución: No se puede decidir sobre la evolución porque no hay una tendencia clara en los valores de las medidas.
\end{comment}

\begin{ejercicio}
    En un sistema Linux se ha ejecutado la siguiente orden:
    \begin{minted}[linenos=false]{shell}
$ time quicksort
real 0m40.2s
user 0m17.1s
sys 0m3.2s
    \end{minted}

    Indique si el sistema está soportando mucha o poca carga. Razone la respuesta.\\

    En este caso, la carga del sistema es considerable, puesto que el tiempo de ejecución del programa \verb|quicksort| es de $\unit[20.3]{s}$ (suma de los tiempos \verb|user| y \verb|sys|) y el tiempo real que ha tardado en ejecutarse es de $\unit[40.2]{s}$. Esto da a entender dos posibilidades:
    \begin{itemize}
        \item El sistema está soportando una carga considerable, ya que el proceso ha estado esperando mucho tiempo para que el sistema le proporcionara los recursos necesarios para su ejecución.
        \item El proceso \verb|quicksort| ha estado bloqueado por E/S (\verb|I/O blocked|) durante una parte importante de su ejecución, lo que ha provocado que el tiempo real sea mucho mayor que el tiempo de CPU consumido.
    \end{itemize}
\end{ejercicio}
\begin{comment}
    Solución: Parece que se trata de un sistema con una carga considerable, ya que hay una notable diferencia entre el tiempo de respuesta que experimenta el usuario (parámetro real) y el tiempo que realmente tarda en ejecutarse su programa (suma de los parámetros user y sys). También podría ser que el proceso \verb|quicksort| usara mucha E/S y que estuviera buena parte del tiempo bloqueado por E/S (I/O blocked).
\end{comment}

\begin{ejercicio}
    Se sabe que la sobrecarga (\emph{overhead}) de CPU de un monitor software en un determinado servidor es del 4\%. Si el monitor se activa cada 2 segundos, ¿cuánto tiempo tarda el monitor en ejecutarse por cada activación?\\

    Sea $T$ el tiempo que tarda el monitor en ejecutarse por cada activación. Si la sobrecarga es del 4\%, entonces tenemos que:
    \begin{align*}
        4 = \frac{T}{2} \cdot 100
        \Longrightarrow T = \frac{4 \cdot 2}{100} = \unit[0.08]{s} = \unit[80]{ms}
    \end{align*}
\end{ejercicio}
\begin{comment}
    Solución: El monitor tarda en ejecutarse 80 milisegundos por activación.
\end{comment}

\begin{ejercicio}
    A continuación se muestra el resultado obtenido tras ejecutar la orden \verb|top| en un sistema informático que emplea Linux como sistema operativo:
    \begin{minted}[linenos=false]{shell}
2:52pm  up 17 days, 3:41, 1 user, load average: 0.15, 0.27, 0.32
54 processes: 51 sleeping, 3 running, 0 zombie, 0 stopped
%Cpu(s): 23.8% user, 14.0% system, 0.0% nice, 17.0% idle, 45.2% wa
Mem:  257124K av, 253052K used,  4072K free,  8960K shrd, 182972K buff
Swap: 261496K av,  21396K used,240100K free, 26344K cached

  PID USER      PRI  NI  VIRT  RSS  SHARE STAT LC %CPU %MEM   TIME COMMAND
  807 joan       0   0  5708 5708   532   R  N  0 23.0  2.2   6:16 p_exec
  809 joan       0   0  5708 5708   532   R  N  0 14.0  2.2   3:42 p_exec
  185 tomi       0   0   824  824   632   R     0  0.5  0.3   0:00 top
  201 xp         0   0  1272 1208   644   S     0  0.1  0.4   5:49 xp_stat
    1 root       0   0    60   56    36   S     0  0.0  0.0   0:03 init
    2 root       0   0     0    0     0   SW    0  0.0  0.0   0:13 kflushd
    7 root       0   0     0    0     0   SW    0  0.0  0.0   0:00 nfsiod
  194 root       0   0    72    4     4   S     0  0.0  0.0   0:00 migetty
  195 root       0   0    68    0     0   SW    0  0.0  0.0   0:00 migetty
  179 root       0   0   532  312   236   S     0  0.0  0.1   0:00 sndmail
    \end{minted}

    \begin{enumerate}
        \item ¿Cuánta memoria física tiene la máquina?
        
        Este es el parámetro \verb|Mem: 257124K av|, que indica que la máquina tiene $\unit[257124]{KiB}\approx 251\text{ MiB}$ de memoria física.
        \item ¿Qué porcentaje de la memoria física está marcada como usada según el monitor?
        
        Este es el parámetro \verb|Mem: 253052K used|, que indica que $\unit[253052]{KiB}\approx 247\text{ MiB}$ de memoria física está usada. Por lo tanto, el porcentaje de memoria física usada es:
        \begin{align*}
            \frac{253052}{257124} \cdot 100 \approx 98.41\%
        \end{align*}
        \item ¿Cuál es la utilización media del procesador?
        
        Este parámetro se calcula como la totalidad $(100\%)$ menos el tiempo que el procesador está inactivo (reflejado por los parámetros \verb|idle| y \verb|wa|):
        \begin{align*}
            100\% - 17.0\% - 45.2\% = 37.8\%
        \end{align*}
        \item ¿Cómo es la evolución de la carga media del sistema, ascendente o descendente?
        
        Esta es descendente, y se puede ver en el parámetro \verb|load average: 0.15, 0.27, 0.32|, que indica que la carga media del sistema ha ido disminuyendo desde $0.32$ a $0.15$.
        \item ¿Cuánta memoria física ocupa el monitor?
        
        La memoria física ocupada por el monitor es la que se muestra en el parámetro \verb|RSS| (\emph{Resident Set Size}) de la línea del proceso \verb|top|, que es de $\unit[824]{KiB}$.
    \end{enumerate}
\end{ejercicio}
\begin{comment}
    Solución:
    \begin{enumerate}
        \item 257124KiB.
        \item 98.4\%.
        \item 37.8\%.
        \item Descendente.
        \item 824KiB.
    \end{enumerate}
\end{comment}

\begin{ejercicio}
    Considere las órdenes siguientes ejecutadas en un sistema Linux:
    \begin{minted}[linenos=false]{shell}
$ time simulador_original
real 0m24.2s
user 0m15.1s
sys 0m1.6s
$ time simulador_mejorado
real 0m32.8s
user 0m10.7s
sys 0m2.1s
    \end{minted}

    \begin{enumerate}
        \item ¿Cuál es el tiempo de ejecución de ambos simuladores?
        
        El tiempo de ejecución del simulador original es de $\unit[16.7]{s}$ (suma de los tiempos \verb|user| y \verb|sys|) y el del simulador mejorado es de $\unit[12.8]{s}$. Notemos que el tiempo real representa el tiempo total que ha tardado en ejecutarse el programa, incluyendo el tiempo de espera por parte del sistema operativo para que le proporcione los recursos necesarios y el tiempo atendiendo interrupciones de otros procesos. Es por esto que el simulador mejorado tarda más tiempo en ejecutarse que el original, puesto que el sistema en ese momento tiene más carga de trabajo.
        \item Calcule, si es el caso, la mejora en el tiempo de ejecución del simulador mejorado respecto del original.
        
        La mejora en el tiempo de ejecución del simulador mejorado respecto del original es:
        \begin{align*}
            \frac{16.7}{12.8} \approx 1.30469
        \end{align*}

        Por lo tanto, el simulador mejorado es aproximadamente $1.3$ veces más rápido que el original.
    \end{enumerate}
\end{ejercicio}
\begin{comment}
    Solución: El simulador original se ejecuta en 16,7 segundos y el mejorado en 12,8 segundos (en realidad, los datos muestran que al sistema le cuesta más tiempo porque ha de atender otras tareas). El simulador mejorado es 1,3 veces más rápido que el original.
\end{comment}

\begin{ejercicio}
    El monitor \verb|sar| \emph{(system activity reporter)} de un computador se activa cada 15 minutos y tarda $\unit[750]{ms}$ en ejecutarse por cada activación. Se pide:
    \begin{enumerate}
        \item Calcular la sobrecarga que genera este monitor sobre el sistema informático.
        
        La sobrecarga que genera el monitor \verb|sar| se calcula como:
        \begin{align*}
            \dfrac{750}{15\cdot 60\cdot 10^3} \cdot 100 = \dfrac{1}{12}\approx 0.08333\%
        \end{align*}
        \item Si la información generada en cada activación ocupa $\unit[8192]{bytes}$, ¿la monitorización de cuántos días completos se pueden almacenar en el directorio \verb|/var/log/sysstat| si se dispone únicamente de $\unit[200]{MiB}$ de capacidad libre?\\

        Suponemos que la totalidad de capacidad disponible se puede emplear para guardar recursos (en realidad, también ha de guardarse información como el nombre de los ficheros, por ejemplo). Calculemos por tanto en primer lugar los registros de cuántas activaciones pueden almacenarse:
        \begin{equation*}
            \unit[200]{MiB}\cdot \dfrac{\unit[2^{20}]{bytes}}{\unit[1]{MiB}} \cdot \dfrac{\unit[1]{registro}}{\unit[8192]{bytes}} = \unit[25600]{registros}
        \end{equation*}

        Calculemos ahora la monitorización de cuántos días completos se pueden almacenar:
        \begin{align*}
            \unit[25600]{registros} \cdot \dfrac{\unit[15]{minutos}}{\unit[1]{registro}} \cdot \dfrac{\unit[1]{hora}}{\unit[60]{minutos}} \cdot \dfrac{\unit[1]{dia}}{\unit[24]{horas}} = \unit[266.67]{dias}
        \end{align*}

        Por tanto, se pueden almacenar la información de $\unit[266]{dias}$ completos.
    \end{enumerate}
\end{ejercicio}
\begin{comment}
    Solución: La sobrecarga es del 0,083 \% y se pueden almacenar la información de 266 días completos.
\end{comment}

\begin{ejercicio}
    El día 8 de octubre se ha ejecutado la siguiente orden en un sistema Linux:
    \begin{minted}[linenos=false]{shell}
% ls /var/log/sysstat
-rw-r--r-- 1 root root 3049952 Oct  6 23:50 sa06
-rw-r--r-- 1 root root 3049952 Oct  7 23:50 sa07
-rw-r--r-- 1 root root 2372184 Oct  8 18:40 sa08
    \end{minted}

    Suponiendo que la primera muestra se toma a las $0:00$ de cada día y que \verb|sadc| se ejecuta con un tiempo de muestreo constante, ¿cada cuánto tiempo se activa el monitor \verb|sar|? ¿Cuál es la anchura de entrada del monitor?\\

    Puesto que las últimas activaciones del día 6 y del día 7 se hacen a las $23:50$ horas, podemos concluir que el monitor se activa cada 10 minutos. Por tanto, el número de veces que se activa el monitor en un día es:
    \begin{align*}
        \frac{24\cdot 60}{10} = 144
    \end{align*}

    La anchura de entrada del monitor se calcula como el tamaño del fichero dividido por el número de activaciones:
    \begin{align*}
        \frac{\unit[3049952]{bytes}}{\unit[144]{activaciones}} \approx \unitfrac[21180.22]{B}{entrada}
        \approx \unitfrac[20.68]{KiB}{entrada}
    \end{align*}

    Por tanto, la anchura de entrada del monitor es de aproximadamente $\unit[20.68]{KiB}$.
\end{ejercicio}
\begin{comment}
    Solución: El monitor se activa cada 10 minutos (la última activación del día se hace a las $23:50$ horas). La información generada en cada activación (anchura de entrada) ocupa aproximadamente 20,7 KiB o 21,2 KB.
\end{comment}

\begin{ejercicio}
    Indique el resultado que produce la ejecución de las siguientes órdenes sobre un sistema Linux con el monitor \verb|sar| instalado:
    \begin{enumerate}
        \item \verb|sar|
        
        Informa sobre la utilización del procesador durante el día actual (opción por defecto, \verb|-u|).
        \item \verb|sar -A|
        
        Informa sobre toda la información recogida durante el día actual.
        \item \verb|sar -u 1 30|
        
        Informa sobre la utilización del procesador en el momento actual, mostrando 30 medidas tomadas con un período de un segundo.
        \item \verb|sar -uB -f /var/log/sysstat/08|
        
        Informa sobre la utilización del procesador y paginación de la memoria virtual durante el día 8 del mes.
        \item \verb|sar -d -s 12:30:00 -e 18:15:00 -f /var/log/sysstat/08|
        
        Informa sobre las transferencias de disco desde las $12:30$ hasta las $18:15$ horas del día 8 del mes.
    \end{enumerate}
\end{ejercicio}
\begin{comment}
    Solución:
    \begin{enumerate}
        \item Utilización del procesador durante el día actual.
        \item Toda la información recogida durante el día actual.
        \item Utilización actual del procesador: 30 medidas tomadas con un período de un segundo.
        \item Utilización del procesador y paginación de la memoria virtual durante el día 8 del mes.
        \item Transferencias de disco desde las $12:30$ hasta las $18:15$ horas del día 8 del mes. 
    \end{enumerate}
\end{comment}

\begin{ejercicio}
    Después de instrumentar un programa con la herramienta \verb|gprof| el resultado obtenido ha sido el siguiente:
    \begin{minted}[linenos=false]{text}
Flat profile:
Each sample counts as 0.01 seconds.
  %   cumulative   self              self     total           
 time   seconds   seconds    calls  s/call   s/call  name    
 59.36     27.72    27.72        3    9.24    14.39  reduce  
 33.08     43.17    15.45        6    2.57     2.57  invierte
  7.56     46.70     3.53        2    1.76     1.76  calcula
    \end{minted}

    El grafo de dependencias muestra que \verb|invierte()| es llamado desde el procedimiento \verb|reduce()|.
    \begin{enumerate}
        \item ¿Cuál es el procedimiento cuyo código propio sería más conveniente optimizar?
        
        Se debe optimizar la función con mayor tiempo propio (\verb|self seconds|), que en este caso es \verb|reduce()|, ya que consume casi el 60\% del tiempo de CPU del programa.
        \item Si el código propio de \verb|reduce()| se sustituye por otro tres veces más rápido, ¿cuánto tiempo tardará en ejecutarse el programa?
        
        El tiempo que tardaría sería:
        \begin{align*}
            27.72 \cdot \frac{1}{3} + 15.45 + 3.53 = \unit[28.22]{s}
        \end{align*}
        \item Si el procedimiento \verb|invierte()| se sustituye por una nueva versión cuatro veces más rápida, ¿qué mejora se obtendrá en el tiempo de ejecución?
        
        La mejora sería:
        \begin{align*}
            S = \dfrac{T_o}{T_m} = \dfrac{27.72 + 15.45 + 3.53}{27.72 + \frac{15.45}{4} + 3.53}\approx 1.33
        \end{align*}
        \item Calcule cuál es la ganancia en velocidad máxima que se podría conseguir en el tiempo de ejecución mediante la optimización del código del procedimiento \verb|invierte()|.
        
        Sea $k$ el factor de mejora del procedimiento \verb|invierte()|, entonces la ganancia en velocidad máxima sería:
        \begin{align*}
            \lim_{k\to\infty} S = \lim_{k\to\infty} \dfrac{27.72 + 15.45 + 3.53}{27.72 + \frac{15.45}{k} + 3.53} \approx 1.49
        \end{align*}
        Por tanto, la máxima ganancia en velocidad que se podría conseguir es de aproximadamente $1.49$.
    \end{enumerate}
\end{ejercicio}
\begin{comment}
    Solución:
    \begin{enumerate}
        \item Deberíamos optimizar \verb|reduce()| ya que su código propio consume casi el 60\% de tiempo de CPU del programa.
        \item El programa se ejecutaría en 28,22 s.
        \item El programa se ejecutaría 1,33 veces más rápidamente.
        \item La máxima ganancia en velocidad que se podría conseguir es 1,49.
    \end{enumerate}
\end{comment}

\begin{ejercicio}
    Un informático desea evaluar el rendimiento de un computador por medio del benchmark SPEC CPU 2017. Una vez compilados todos los programas del paquete y lanzado su ejecución monitoriza el sistema con la orden \verb|vmstat 1 5|. El resultado de las medidas de este monitor es el siguiente:
    \begin{minted}[linenos=false]{text}
  procs  ----------memory---------  --swap--  ----io---- ---system---  ----cpu----
  r  b   swpd   free   buff  cache   si   so    bi    bo   in   cs  us  sy  id  wa
  0  0     8  14916  92292 833828    0    0     0     3    0    7   3   1  96   0
  1  0     8  14916  92292 833828    0    0     0     0 1022   40 100   0   0   0
  3  0     8  14916  92292 833828    2    1    16     3 1016   34  99   1   0   0
  1  0     8  14916  92292 833828    0    4     0     8 1035   36  98   2   0   0
  2  0     8  14916  92292 833828    1    5     4    28 1035   36  99   1   0   0
    \end{minted}

    Indique si, a la vista de los datos anteriores, los resultados obtenidos en la prueba de evaluación serán correctos o no. Justifique la respuesta.\\

    No aparentan ser correctos, puesto que este benchmark tiene como objetivo medir el rendimiento de la CPU, y no se producen entonces intercambio con el disco. Por tanto, deberá haber otros programas en ejecución que los provoquen, alterando los resultados de la medición.
\end{ejercicio}
\begin{comment}
    Solución: El sistema operativo presenta actividad de intercambio con el disco y ninguno de los programas del benchmark debería provocarlos.
\end{comment}

\begin{ejercicio}
    La monitorización de un programa de dibujo en tres dimensiones mediante la herramienta \verb|gprof| ha proporcionado la siguiente información (por errores en la transmisión hay valores que no están disponibles):
    \begin{minted}[linenos=false]{text}
Flat profile:
  %     cumulative   self              self     total           
 time     seconds   seconds    calls   s/call   s/call  name    
xxxxx       xxxxx    15.47        3     5.16     5.16  colorea 
xxxxx       xxxxx     1.89        5     0.38     0.38  interpola
xxxxx       xxxxx     1.76        1     1.76     3.65  traza   
xxxxx       xxxxx     0.46                             main

Call graph:
index % time    self  children    called     name
[1]    100.0    0.46    19.12                main [1]
               15.47     0.00      3/3          colorea [2]
                1.76     1.89      1/1          traza [3]

               15.47     0.00      3/3          main [1]
[2]     79.0   15.47     0.00      3         colorea [2]
                
                1.76     1.89      1/1          main [1]
[3]     18.6    1.76     1.89      1         traza [3]
                1.89     0.00      5/5          interpola [4]
                
                1.89     0.00      5/5          traza [3]
[4]      9.7    1.89     0.00      5         interpola [4]
    \end{minted}
    \begin{enumerate}
        \item ¿En cuánto tiempo se ejecuta el programa de dibujo?
        
        Hay varias formas de verlo, y evidentemente todas ellas coinciden.
        \begin{itemize}
            \item En el \verb|flat profile|, podemos obtener el tiempo total de ejecución como la suma de los \verb|self seconds|:
            \begin{align*}
                15.47 + 1.89 + 1.76 + 0.46 = \unit[19.58]{s}
            \end{align*}
            \item Desde el \verb|call graph|, como el método \verb|main| se ejecuta el 100\% del tiempo, podemos calcular este tiempo como la suma de su tiempo propio de ejecución más el de sus hijos:
            \begin{align*}
                0.46 + 19.12 = \unit[19.58]{s}
            \end{align*}
        \end{itemize}
        \item Indique cuánto tiempo tarda en ejecutarse el código propio de \verb|main()|.

        El código propio de \verb|main()| se ejecuta en $\unit[0.46]{s}$, que es el valor del campo \verb|self seconds| de la línea correspondiente al procedimiento \verb|main()| en el \verb|flat profile|.
        \item Establezca la relación de llamadas entre los procedimientos del programa así como el número de veces que se ejecuta cada uno de ellos.

        El procedimiento \verb|main()| llama 3 veces al procedimiento \verb|colorea()| y una vez al procedimiento \verb|traza()|; a su vez, el procedimiento \verb|traza()| llama 5 veces al procedimiento \verb|interpola()|.
        \item Calcule el nuevo tiempo de ejecución del programa si se elimina el código propio de \verb|main()| y se reduce a la mitad el tiempo de ejecución del código propio del procedimiento \verb|traza()|.
        
        El nuevo tiempo de ejecución del programa sería:
        \begin{align*}
            15.47 + 1.89 + \frac{1.76}{2} + 0 = \unit[18.24]{s}
        \end{align*}
        \item Proponga y justifique numéricamente una acción sobre el programa original que no afecte el procedimiento \verb|colorea()| (ni su código ni el número de veces que es ejecutado) con el fin de conseguir que el programa se ejecute en 10 segundos.

        Esto no es posible, puesto que el procedimiento \verb|colorea()| ya consume más de 10 segundos de tiempo de ejecución propio, y no se puede reducir su tiempo de ejecución sin afectar a su código o al número de veces que se ejecuta. Por tanto, cualquier acción que se proponga para reducir el tiempo de ejecución del programa afectará necesariamente al procedimiento \verb|colorea()|.
    \end{enumerate}
\end{ejercicio}
\begin{comment}
    Solución:
    \begin{enumerate}
        \item El programa se ejecuta en 19,58 s.
        \item El código propio de \verb|main()| se ejecuta en 0,46 s.
        \item El procedimiento \verb|main()| llama 3 veces a \verb|colorea()| y una vez a \verb|traza()|; a su vez, \verb|traza()| llama 5 veces a \verb|interpola()|.
        \item El tiempo de ejecución sería de 18,24 s.
        \item El tiempo de ejecución no se puede reducir a 10 segundos sin afectar el procedimiento \verb|colorea()| porque su contribución ya sobrepasa este valor. 
    \end{enumerate}
\end{comment}