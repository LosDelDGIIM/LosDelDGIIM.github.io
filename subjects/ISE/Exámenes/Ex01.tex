\documentclass[12pt]{article}

% Idioma y codificación
\usepackage[spanish, es-tabla]{babel}       %es-tabla para que se titule "Tabla"
\usepackage[utf8]{inputenc}

% Márgenes
\usepackage[a4paper,top=3cm,bottom=2.5cm,left=3cm,right=3cm]{geometry}

% Comentarios de bloque
\usepackage{verbatim}

% Paquetes de links
\usepackage[hidelinks]{hyperref}    % Permite enlaces
\usepackage{url}                    % redirecciona a la web

% Más opciones para enumeraciones
\usepackage{enumitem}

% Personalizar la portada
\usepackage{titling}

% Paquetes de tablas
\usepackage{multirow}


%------------------------------------------------------------------------

%Paquetes de figuras
\usepackage{caption}
\usepackage{subcaption} % Figuras al lado de otras
\usepackage{float}      % Poner figuras en el sitio indicado H.


% Paquetes de imágenes
\usepackage{graphicx}       % Paquete para añadir imágenes
\usepackage{transparent}    % Para manejar la opacidad de las figuras

% Paquete para usar colores
\usepackage[dvipsnames]{xcolor}
\usepackage{pagecolor}      % Para cambiar el color de la página

% Habilita tamaños de fuente mayores
\usepackage{fix-cm}

% Para los gráficos
\usepackage{tikz}

% Para poder situar los nodos en los grafos
\usetikzlibrary{positioning}


%------------------------------------------------------------------------

% Paquetes de matemáticas
\usepackage{mathtools, amsfonts, amssymb, mathrsfs}
\usepackage[makeroom]{cancel}     % Simplificar tachando
\usepackage{polynom}    % Divisiones y Ruffini
\usepackage{units} % Para poner fracciones diagonales con \nicefrac

\usepackage{pgfplots}   %Representar funciones
\pgfplotsset{compat=1.18}  % Versión 1.18

\usepackage{tikz-cd}    % Para usar diagramas de composiciones
\usetikzlibrary{calc}   % Para usar cálculo de coordenadas en tikz

%Definición de teoremas, etc.
\usepackage{amsthm}
%\swapnumbers   % Intercambia la posición del texto y de la numeración

\theoremstyle{plain}

\makeatletter
\@ifclassloaded{article}{
  \newtheorem{teo}{Teorema}[section]
}{
  \newtheorem{teo}{Teorema}[chapter]  % Se resetea en cada chapter
}
\makeatother

\newtheorem{coro}{Corolario}[teo]           % Se resetea en cada teorema
\newtheorem{prop}[teo]{Proposición}         % Usa el mismo contador que teorema
\newtheorem{lema}[teo]{Lema}                % Usa el mismo contador que teorema

\theoremstyle{remark}
\newtheorem*{observacion}{Observación}

\theoremstyle{definition}

\makeatletter
\@ifclassloaded{article}{
  \newtheorem{definicion}{Definición} [section]     % Se resetea en cada chapter
}{
  \newtheorem{definicion}{Definición} [chapter]     % Se resetea en cada chapter
}
\makeatother

\newtheorem*{notacion}{Notación}
\newtheorem*{ejemplo}{Ejemplo}
\newtheorem*{ejercicio*}{Ejercicio}             % No numerado
\newtheorem{ejercicio}{Ejercicio} [section]     % Se resetea en cada section


% Modificar el formato de la numeración del teorema "ejercicio"
\renewcommand{\theejercicio}{%
  \ifnum\value{section}=0 % Si no se ha iniciado ninguna sección
    \arabic{ejercicio}% Solo mostrar el número de ejercicio
  \else
    \thesection.\arabic{ejercicio}% Mostrar número de sección y número de ejercicio
  \fi
}


% \renewcommand\qedsymbol{$\blacksquare$}         % Cambiar símbolo QED
%------------------------------------------------------------------------

% Paquetes para encabezados
\usepackage{fancyhdr}
\pagestyle{fancy}
\fancyhf{}

\newcommand{\helv}{ % Modificación tamaño de letra
\fontfamily{}\fontsize{12}{12}\selectfont}
\setlength{\headheight}{15pt} % Amplía el tamaño del índice


%\usepackage{lastpage}   % Referenciar última pag   \pageref{LastPage}
\fancyfoot[C]{\thepage}

%------------------------------------------------------------------------

% Conseguir que no ponga "Capítulo 1". Sino solo "1."
\makeatletter
\@ifclassloaded{book}{
  \renewcommand{\chaptermark}[1]{\markboth{\thechapter.\ #1}{}} % En el encabezado
    
  \renewcommand{\@makechapterhead}[1]{%
  \vspace*{50\p@}%
  {\parindent \z@ \raggedright \normalfont
    \ifnum \c@secnumdepth >\m@ne
      \huge\bfseries \thechapter.\hspace{1em}\ignorespaces
    \fi
    \interlinepenalty\@M
    \Huge \bfseries #1\par\nobreak
    \vskip 40\p@
  }}
}
\makeatother

%------------------------------------------------------------------------
% Paquetes de cógido
\usepackage{minted}
\renewcommand\listingscaption{Código fuente}

\usepackage{fancyvrb}
% Personaliza el tamaño de los números de línea
\renewcommand{\theFancyVerbLine}{\small\arabic{FancyVerbLine}}

% Estilo para C++
\newminted{cpp}{
    frame=lines,
    framesep=2mm,
    baselinestretch=1.2,
    linenos,
    escapeinside=||
}

% para minted
\definecolor{LightGray}{rgb}{0.95,0.95,0.92}
\setminted{
    linenos=true,
    stepnumber=5,
    numberfirstline=true,
    autogobble,
    breaklines=true,
    breakautoindent=true,
    breaksymbolleft=,
    breaksymbolright=,
    breaksymbolindentleft=0pt,
    breaksymbolindentright=0pt,
    breaksymbolsepleft=0pt,
    breaksymbolsepright=0pt,
    fontsize=\footnotesize,
    bgcolor=LightGray,
    numbersep=10pt
}


\usepackage{listings} % Para incluir código desde un archivo

\renewcommand\lstlistingname{Código Fuente}
\renewcommand\lstlistlistingname{Índice de Códigos Fuente}

% Definir colores
\definecolor{vscodepurple}{rgb}{0.5,0,0.5}
\definecolor{vscodeblue}{rgb}{0,0,0.8}
\definecolor{vscodegreen}{rgb}{0,0.5,0}
\definecolor{vscodegray}{rgb}{0.5,0.5,0.5}
\definecolor{vscodebackground}{rgb}{0.97,0.97,0.97}
\definecolor{vscodelightgray}{rgb}{0.9,0.9,0.9}

% Configuración para el estilo de C similar a VSCode
\lstdefinestyle{vscode_C}{
  backgroundcolor=\color{vscodebackground},
  commentstyle=\color{vscodegreen},
  keywordstyle=\color{vscodeblue},
  numberstyle=\tiny\color{vscodegray},
  stringstyle=\color{vscodepurple},
  basicstyle=\scriptsize\ttfamily,
  breakatwhitespace=false,
  breaklines=true,
  captionpos=b,
  keepspaces=true,
  numbers=left,
  numbersep=5pt,
  showspaces=false,
  showstringspaces=false,
  showtabs=false,
  tabsize=2,
  frame=tb,
  framerule=0pt,
  aboveskip=10pt,
  belowskip=10pt,
  xleftmargin=10pt,
  xrightmargin=10pt,
  framexleftmargin=10pt,
  framexrightmargin=10pt,
  framesep=0pt,
  rulecolor=\color{vscodelightgray},
  backgroundcolor=\color{vscodebackground},
}

%------------------------------------------------------------------------

% Comandos definidos
\newcommand{\bb}[1]{\mathbb{#1}}
\newcommand{\cc}[1]{\mathcal{#1}}

% I prefer the slanted \leq
\let\oldleq\leq % save them in case they're every wanted
\let\oldgeq\geq
\renewcommand{\leq}{\leqslant}
\renewcommand{\geq}{\geqslant}

% Si y solo si
\newcommand{\sii}{\iff}

% Letras griegas
\newcommand{\eps}{\epsilon}
\newcommand{\veps}{\varepsilon}
\newcommand{\lm}{\lambda}

\newcommand{\ol}{\overline}
\newcommand{\ul}{\underline}
\newcommand{\wt}{\widetilde}
\newcommand{\wh}{\widehat}

\let\oldvec\vec
\renewcommand{\vec}{\overrightarrow}

% Derivadas parciales
\newcommand{\del}[2]{\frac{\partial #1}{\partial #2}}
\newcommand{\Del}[3]{\frac{\partial^{#1} #2}{\partial #3^{#1}}}
\newcommand{\deld}[2]{\dfrac{\partial #1}{\partial #2}}
\newcommand{\Deld}[3]{\dfrac{\partial^{#1} #2}{\partial #3^{#1}}}


\newcommand{\AstIg}{\stackrel{(\ast)}{=}}
\newcommand{\Hop}{\stackrel{L'H\hat{o}pital}{=}}

\newcommand{\red}[1]{{\color{red}#1}} % Para integrales, destacar los cambios.

% Método de integración
\newcommand{\MetInt}[2]{
    \left[\begin{array}{c}
        #1 \\ #2
    \end{array}\right]
}

% Declarar aplicaciones
% 1. Nombre aplicación
% 2. Dominio
% 3. Codominio
% 4. Variable
% 5. Imagen de la variable
\newcommand{\Func}[5]{
    \begin{equation*}
        \begin{array}{rrll}
            #1:& #2 & \longrightarrow & #3\\
               & #4 & \longmapsto & #5
        \end{array}
    \end{equation*}
}

%------------------------------------------------------------------------



\begin{document}

    % 1. Foto de fondo
    % 2. Título
    % 3. Encabezado Izquierdo
    % 4. Color de fondo
    % 5. Coord x del titulo
    % 6. Coord y del titulo
    % 7. Fecha

    
    % 1. Foto de fondo
% 2. Título
% 3. Encabezado Izquierdo
% 4. Color de fondo
% 5. Coord x del titulo
% 6. Coord y del titulo
% 7. Fecha

\newcommand{\portada}[7]{

    \portadaBase{#1}{#2}{#3}{#4}{#5}{#6}{#7}
    \portadaBook{#1}{#2}{#3}{#4}{#5}{#6}{#7}
}

\newcommand{\portadaExamen}[7]{

    \portadaBase{#1}{#2}{#3}{#4}{#5}{#6}{#7}
    \portadaArticle{#1}{#2}{#3}{#4}{#5}{#6}{#7}
}




\newcommand{\portadaBase}[7]{

    % Tiene la portada principal y la licencia Creative Commons
    
    % 1. Foto de fondo
    % 2. Título
    % 3. Encabezado Izquierdo
    % 4. Color de fondo
    % 5. Coord x del titulo
    % 6. Coord y del titulo
    % 7. Fecha
    
    
    \thispagestyle{empty}               % Sin encabezado ni pie de página
    \newgeometry{margin=0cm}        % Márgenes nulos para la primera página
    
    
    % Encabezado
    \fancyhead[L]{\helv #3}
    \fancyhead[R]{\helv \nouppercase{\leftmark}}
    
    
    \pagecolor{#4}        % Color de fondo para la portada
    
    \begin{figure}[p]
        \centering
        \transparent{0.3}           % Opacidad del 30% para la imagen
        
        \includegraphics[width=\paperwidth, keepaspectratio]{assets/#1}
    
        \begin{tikzpicture}[remember picture, overlay]
            \node[anchor=north west, text=white, opacity=1, font=\fontsize{60}{90}\selectfont\bfseries\sffamily, align=left] at (#5, #6) {#2};
            
            \node[anchor=south east, text=white, opacity=1, font=\fontsize{12}{18}\selectfont\sffamily, align=right] at (9.7, 3) {\textbf{\href{https://losdeldgiim.github.io/}{Los Del DGIIM}}};
            
            \node[anchor=south east, text=white, opacity=1, font=\fontsize{12}{15}\selectfont\sffamily, align=right] at (9.7, 1.8) {Doble Grado en Ingeniería Informática y Matemáticas\\Universidad de Granada};
        \end{tikzpicture}
    \end{figure}
    
    
    \restoregeometry        % Restaurar márgenes normales para las páginas subsiguientes
    \pagecolor{white}       % Restaurar el color de página
    
    
    \newpage
    \thispagestyle{empty}               % Sin encabezado ni pie de página
    \begin{tikzpicture}[remember picture, overlay]
        \node[anchor=south west, inner sep=3cm] at (current page.south west) {
            \begin{minipage}{0.5\paperwidth}
                \href{https://creativecommons.org/licenses/by-nc-nd/4.0/}{
                    \includegraphics[height=2cm]{assets/Licencia.png}
                }\vspace{1cm}\\
                Esta obra está bajo una
                \href{https://creativecommons.org/licenses/by-nc-nd/4.0/}{
                    Licencia Creative Commons Atribución-NoComercial-SinDerivadas 4.0 Internacional (CC BY-NC-ND 4.0).
                }\\
    
                Eres libre de compartir y redistribuir el contenido de esta obra en cualquier medio o formato, siempre y cuando des el crédito adecuado a los autores originales y no persigas fines comerciales. 
            \end{minipage}
        };
    \end{tikzpicture}
    
    
    
    % 1. Foto de fondo
    % 2. Título
    % 3. Encabezado Izquierdo
    % 4. Color de fondo
    % 5. Coord x del titulo
    % 6. Coord y del titulo
    % 7. Fecha


}


\newcommand{\portadaBook}[7]{

    % 1. Foto de fondo
    % 2. Título
    % 3. Encabezado Izquierdo
    % 4. Color de fondo
    % 5. Coord x del titulo
    % 6. Coord y del titulo
    % 7. Fecha

    % Personaliza el formato del título
    \pretitle{\begin{center}\bfseries\fontsize{42}{56}\selectfont}
    \posttitle{\par\end{center}\vspace{2em}}
    
    % Personaliza el formato del autor
    \preauthor{\begin{center}\Large}
    \postauthor{\par\end{center}\vfill}
    
    % Personaliza el formato de la fecha
    \predate{\begin{center}\huge}
    \postdate{\par\end{center}\vspace{2em}}
    
    \title{#2}
    \author{\href{https://losdeldgiim.github.io/}{Los Del DGIIM}}
    \date{Granada, #7}
    \maketitle
    
    \tableofcontents
}




\newcommand{\portadaArticle}[7]{

    % 1. Foto de fondo
    % 2. Título
    % 3. Encabezado Izquierdo
    % 4. Color de fondo
    % 5. Coord x del titulo
    % 6. Coord y del titulo
    % 7. Fecha

    % Personaliza el formato del título
    \pretitle{\begin{center}\bfseries\fontsize{42}{56}\selectfont}
    \posttitle{\par\end{center}\vspace{2em}}
    
    % Personaliza el formato del autor
    \preauthor{\begin{center}\Large}
    \postauthor{\par\end{center}\vspace{3em}}
    
    % Personaliza el formato de la fecha
    \predate{\begin{center}\huge}
    \postdate{\par\end{center}\vspace{5em}}
    
    \title{#2}
    \author{\href{https://losdeldgiim.github.io/}{Los Del DGIIM}}
    \date{Granada, #7}
    \thispagestyle{empty}               % Sin encabezado ni pie de página
    \maketitle
    \vfill
}
    \portadaExamen{etsiitA4.jpg}{Ingeniería de\\Servidores\\Examen I}{Ingeniería de Servidores. Examen I}{MidnightBlue}{-8}{28}{2025}{Arturo Olivares Martos}

    \begin{description}
        \item[Asignatura] Ingeniería de Servidores.
        \item[Curso Académico] 2024/25.
        \item[Grado] Doble Grado en Ingeniería Informática y Matemáticas.
        \item[Grupo] Único.
        \item[Profesor] Pablo García Sánchez.
        \item[Descripción] Examen de Incidencias de la Convocatoria Ordinaria.
        \item[Fecha] 10 de Junio de 2025.
        \item[Duración] 2 horas y media.
    
    \end{description}
    \newpage


    % ------------------------------------
    
    El examen iba acompañadado de 20 preguntas tipo test de verdadero o falso, que se encuentran en el banco de preguntas ofrecido en la web.


    \begin{ejercicio}[1 punto]
        Se decide reemplazar el disco duro del servidor, y ahora el programa que hacía uso del disco (y otros) tardan la mitad de tiempo en ejecutarse. Además, ahora ese programa accede al disco duro el $25\%$ del tiempo.
        \begin{enumerate}
            \item Calcule la porción de tiempo que el proceso consumía antes accediento al disco antiguo (0.5 puntos).
            \item ¿Cuántas veces es más rápida la nueva unidad SSD que el antiguo? (0.5 puntos).
        \end{enumerate}
    \end{ejercicio}


    \begin{ejercicio}[2 puntos]
        Comparando dos discos duros se ha obtenido el tiempo en almacenar distintos archivos. ¿Cuál debería usar nuestro servidor? Queremos estar seguros al $95\%$, se adjunta la tabla para calcular $t_{\frac{\alpha}{2},n-1}$.
        \begin{table}[H]
        \centering
        \begin{tabular}{c|c|c}
            Archivo & $D_A$ & $D_B$ \\
            \hline
            1     & 100 & 50 \\
            2     & 200 & 100 \\
            3     & 50 & 100 \\
            4     & 100 & 150 \\
            5     & 200 & 100
        \end{tabular}
        \caption{Tiempo de almacenamiento de archivos.}
        \end{table}
        \begin{table}[H]
        \centering
        \begin{tabular}{c|c|c|c|c|c|}
            df & 0.2 & 0.10 & 0.05 & 0.02 & 0.01 \\
            \hline
            1 & 3.077 & 6.3138 & 12.7062 & 31.8205 & 63.6567 \\
            2 & 1.8556 & 2.92 & 4.3027 & 6.9646 & 0.9248 \\
            3 & 1.6377 & 2.3534 & 3.1824 & 4.5407 & 5.8409 \\
            4 & 1.5332 & 2.1318 & 2.7764 & 3.7469 & 4.6041 \\
            5 & 1.475 & 2.015 & 2.5706 & 3.3649 & 4.0321 \\
            6 & 1.4398 & 1.9432 & 2.4469 & 3.1427 & 3.7074
        \end{tabular}
        \caption{Tabla de distribución $t-$Student.}
        \end{table}
    \end{ejercicio}

    \begin{ejercicio}[1 punto]
        Partiendo de la Ley de Utilización demuestre de manera razonada que la productividad media máxima de un servidor viene dada por la inversa de la demanda de servicio medio del cuello de botella.
    \end{ejercicio}

    \begin{ejercicio}[2 puntos]
        Un servidor recibe una media de $0.3$ peticiones por segundo y es modelado con (tabla en segundos):
        \begin{table}[H]
        \centering
        \begin{tabular}{c|c|c}
            & $S_i$ & $V_i$ \\
            \hline
            CPU & 0.2 & 10 \\
            Disco A & 0.07 & 6 \\
            Disco B & 0.02 & 8
        \end{tabular}
        \end{table}

        \begin{enumerate}
            \item ¿Qué unidad provoca el cuello de botella? ¿Está el servidor saturado? (0.25 puntos).
            \item Calcule valores máximos y mínimos globales (cualquier valor de la tasa de llegada al servidor) tanto del tiempo de respuesta medio del servidor como su productividad media (0.5 puntos).
            \item Calcule la utilización del disco A. Indique las leyes usadas (0.5 puntos).
            \item ¿Cuántos trabajos hay en el sistema de media? Suponemos que se cumple $W_i = N_i\cdot S_i$. Indique las leyes usadas (0.75 puntos).
        \end{enumerate}
    \end{ejercicio}


    \newpage
    \setcounter{ejercicio}{0}
    \begin{ejercicio}[1 punto]
        Se decide reemplazar el disco duro del servidor, y ahora el programa que hacía uso del disco (y otros) tardan la mitad de tiempo en ejecutarse. Además, ahora ese programa accede al disco duro el $25\%$ del tiempo.
        \begin{enumerate}
            \item Calcule la porción de tiempo que el proceso consumía antes accediento al disco antiguo (0.5 puntos).
            
            Sea $T_o$ el tiempo que el proceso consumía antes accediendo al disco antiguo, y $T_m$ el tiempo que consume ahora. Entonces, tenemos que:
            \begin{align*}
                T_m &= \frac{T_o}{2}
            \end{align*}

            Sea ahora $f$ la fracción de tiempo que el proceso accedía al disco duro antes del cambio, y sea $k$ el factor de mejora del disco duro. Por la Ley de Amdahl, tenemos que:
            \begin{align*}
                T_m = T_o \cdot (1 - f) + \frac{f \cdot T_o}{k}
            \end{align*}

            Por último, puesto que ahora el proceso accede al disco duro el $25\%$ del tiempo, tenemos que:
            \begin{align*}
                \dfrac{T_m}{4} = \frac{fT_o}{k}
            \end{align*}

            Resolvemos ahora el sistema:
            \begin{equation*}
                T_o\cdot (1 - f) + \dfrac{T_o}{8}
                = \dfrac{T_o}{2}
                \Longrightarrow
                1-f = \dfrac{3}{8}
                \Longrightarrow f = \dfrac{5}{8} = 0.625 = 62.5\%
            \end{equation*}
            \item ¿Cuántas veces es más rápida la nueva unidad SSD que el antiguo? (0.5 puntos).
            \begin{align*}
                \dfrac{T_o}{8} = \dfrac{fT_o}{k}
                \Longrightarrow k = 8f = 5
            \end{align*}
        \end{enumerate}
    \end{ejercicio}

    \begin{ejercicio}[2 puntos]
        Comparando dos discos duros se ha obtenido el tiempo en almacenar distintos archivos. ¿Cuál debería usar nuestro servidor? Queremos estar seguros al $95\%$, se adjunta la tabla para calcular $t_{\frac{\alpha}{2},n-1}$.
        \begin{table}[H]
        \centering
        \begin{tabular}{c|c|c}
            Archivo & $D_A$ & $D_B$ \\
            \hline
            1     & 100 & 50 \\
            2     & 200 & 100 \\
            3     & 50 & 100 \\
            4     & 100 & 150 \\
            5     & 200 & 100
        \end{tabular}
        \caption{Tiempo de almacenamiento de archivos.}
        \end{table}
        \begin{table}[H]
        \centering
        \begin{tabular}{c|c|c|c|c|c|}
            df & 0.2 & 0.10 & 0.05 & 0.02 & 0.01 \\
            \hline
            1 & 3.077 & 6.3138 & 12.7062 & 31.8205 & 63.6567 \\
            2 & 1.8556 & 2.92 & 4.3027 & 6.9646 & 0.9248 \\
            3 & 1.6377 & 2.3534 & 3.1824 & 4.5407 & 5.8409 \\
            4 & 1.5332 & 2.1318 & 2.7764 & 3.7469 & 4.6041 \\
            5 & 1.475 & 2.015 & 2.5706 & 3.3649 & 4.0321 \\
            6 & 1.4398 & 1.9432 & 2.4469 & 3.1427 & 3.7074
        \end{tabular}
        \caption{Tabla de distribución $t-$Student.}
        \end{table}

        Buscamos obtener si las diferencias son o no significativas. Para ello, calculamos la media y la desviación típica de las diferencias, sabiendo que hay $5-1=4$ grados de libertad.
        \begin{align*}
            \ol{d} &= \frac{50 + 100 - 50 - 50 + 100}{5} = \frac{150}{5} = \unit[30]{ms}\\
            s &= \sqrt{\frac{(50 - 30)^2 + (100 - 30)^2 + (-50 - 30)^2 + (-50 - 30)^2 + (100 - 30)^2}{5-1}}
            =\\&= \sqrt{\frac{(20)^2 + 2(70)^2 + 2(80)^2}{4}}
            = 5\sqrt{230} \approx \unit[75.8287]{ms}
        \end{align*}

        Calculamos el valor de $t_{exp}$ haciendo uso de la Hipótesis nula $H_0$: suponemos que las máquinas tienen rendimientos equivalentes, luego $\ol{d}_{real}=0$.
        \begin{align*}
            t_{exp} &= \frac{\ol{d} - 0}{\frac{s}{\sqrt{n}}}
            = \frac{30\sqrt{5}}{5\sqrt{230}}
            = \frac{6}{\sqrt{46}} \approx 0.8846
        \end{align*}

        Hacemos ahora uso de que $\alpha=0.05$ (puesto que es el nivel de confianza del $95\%$) y $n-1=4$ grados de libertad, por lo que buscamos en la tabla de distribución $t-$Student:
        \begin{equation*}
            t_{\frac{\alpha}{2},n-1} = 2.7764
        \end{equation*}

        Como $|t_{exp}| < t_{\frac{\alpha}{2},n-1}$, no podemos rechazar la hipótesis nula, por lo que no hay diferencias significativas entre los discos. Por tanto, podemos usar cualquiera de los dos discos. A efectos prácticos, probablemente emplearemos el disco más barato.
    \end{ejercicio}

    \begin{ejercicio}[1 punto]
        Partiendo de la Ley de Utilización demuestre de manera razonada que la productividad media máxima de un servidor viene dada por la inversa de la demanda de servicio medio del cuello de botella.\\

        Consideramos el dispositivo $i-$ésimo de un servidor. Por la Ley de Utilización, tenemos que:
        \begin{align*}
            U_i &= X_i\cdot S_i
        \end{align*}

        Haciendo uso de la Ley de Flujo Reforzado, tenemos que:
        \begin{align*}
            X_i &= X_0\cdot V_i
        \end{align*}

        Por tanto, la utilización del dispositivo $i-$ésimo queda como:
        \begin{align*}
            U_i &= X_0\cdot V_i\cdot S_i
        \end{align*}

        De ladefinición de demanda de servicio, tenemos que:
        \begin{align*}
            D_i &= V_i\cdot S_i
        \end{align*}

        Por tanto, llegamos así a demostrar la Relación Utilización-Demanda de Servicio:
        \begin{align*}
            U_i &= X_0\cdot D_i
        \end{align*}

        Consideramos ahora el cuello de botella del servidor (emplearemos por tanto el subíndice $b$). La productividad media máxima del servidor se alcanza cuando el cuello de botella está saturado, es decir, cuando $U_b = 1$. Por tanto, tenemos que:
        \begin{align*}
            1 = U_b^{max} = X_0^{max}\cdot D_b
            \Longrightarrow
            X_0^{max} &= \frac{1}{D_b}
        \end{align*}
    \end{ejercicio}

    \begin{ejercicio}[2 puntos]
        Un servidor recibe una media de $0.3$ peticiones por segundo y es modelado con (tabla en segundos):
        \begin{table}[H]
        \centering
        \begin{tabular}{c|c|c}
            & $S_i$ & $V_i$ \\
            \hline
            CPU & 0.2 & 10 \\
            Disco A & 0.07 & 6 \\
            Disco B & 0.02 & 8
        \end{tabular}
        \end{table}

        \begin{enumerate}
            \item ¿Qué unidad provoca el cuello de botella? ¿Está el servidor saturado? (0.25 puntos).
            
            Para determinar el cuello de botella, calculamos la demanda de servicio de cada dispositivo:
            \begin{align*}
                D_{CPU} &= V_{CPU}\cdot S_{CPU} = 10\cdot 0.2 = 2\\
                D_{Disco A} &= V_{Disco A}\cdot S_{Disco A} = 6\cdot 0.07 = 0.42\\
                D_{Disco B} &= V_{Disco B}\cdot S_{Disco B} = 8\cdot 0.02 = 0.16
            \end{align*}

            El cuello de botella es el dispositivo con mayor demanda de servicio, que en este caso es la CPU, con $D_{CPU} = 2$.
            Para determinar si el servidor está saturado, notemos que no podemos calcular la utilización de la CPU puesto que no sabemos la productividad (no podemos asumir que esté en equilibrio).
            La productividad máxima del servidor es:
            \begin{align*}
                X_0^{max} &= \frac{1}{D_b} = \frac{1}{D_{CPU}} = \frac{1}{2} = \unitfrac[0.5]{trabajos}{s}
            \end{align*}

            Como la tasa de llegada al servidor es de $0.3$ trabajos por segundo y es menor que la productividad máxima del servidor, podemos concluir que el servidor no está saturado.
            \item Calcule valores máximos y mínimos globales (cualquier valor de la tasa de llegada al servidor) tanto del tiempo de respuesta medio del servidor como su productividad media (0.5 puntos).
            
            Como hemos visto antes, la productividad media máxima del servidor es:
            \begin{align*}
                X_0^{max} &= \frac{1}{D_b} = \frac{1}{D_{CPU}} = \frac{1}{2} = \unitfrac[0.5]{trabajos}{s}
            \end{align*}

            El tiempo mínimo de respuesta del servidor se alcanza cuando todas las colas están vacías:
            \begin{align*}
                R_{0}^{min} &= \sum_{i=1}^{n} V_i\cdot R_i = \sum_{i=1}^{n} V_i\cdot (S_i + \cancelto{0}{W_i})
                = \sum_{i=1}^{n} V_i\cdot S_i
                = \sum_{i=1}^{n} D_i
                =\\&= D_{CPU} + D_{Disco A} + D_{Disco B} = 2 + 0.42 + 0.16 = \unit[2.58]{s}
            \end{align*}

            Aunque carecen de sentido y entendemos que no tendrían que calcularse, la productividad mínima media es de $0$ trabajos por segundo (caso en el que no hay trabajos en el sistema), y el tiempo de respuesta máximo es infinito (caso en el que las colas se desbordan completamente).
            \item Calcule la utilización del disco A. Indique las leyes usadas (0.5 puntos).
            
            Suponemos que la tasa de llegada al servidor media ha sido siempre constante (no se han llenado las colas anteriormente), luego el servidor no está saturado y, por tanto, está en equilibrio, luego:
            \begin{equation*}
                X_0 \approx \lm_0 = \unitfrac[0.3]{trabajos}{s}
            \end{equation*}

            Por la Relación Utilización-Demanda de Servicio, tenemos que:
            \begin{align*}
                U_{Disco A} &= X_0\cdot D_{Disco A} = 0.3\cdot 0.42 = 0.126 = 12.6\%
            \end{align*}

            Por tanto, la utilización del disco A es del $12.6\%$.
            
            
            \item ¿Cuántos trabajos hay en el sistema de media? Suponemos que se cumple $W_i = N_i\cdot S_i$. Indique las leyes usadas (0.75 puntos).
            
            Por la Ley de Little, tenemos que:
            \begin{align*}
                N_0 &= X_0\cdot R_0
            \end{align*}

            Como el valor de $X_0$ es conocido, necesitamos calcular el tiempo de respuesta medio del servidor. Por la Ley General del Tiempo de Respuesta, tenemos que:
            \begin{align*}
                R_0 &= \sum_{i=1}^{n} V_i\cdot R_i
            \end{align*}


            Por tanto, tenemos que:
            \begin{align*}
                N_0 &= X_0\cdot \sum_{i=1}^{n} V_i\cdot R_i
                = \sum_{i=1}^{n} X_0\cdot V_i\cdot R_i
            \end{align*}

            Aplicando ahora la Ley de Flujo Reforzado, tenemos que:
            \begin{align*}
                X_i &= X_0\cdot V_i
            \end{align*}

            Por tanto, tenemos que:
            \begin{align*}
                N_0 &= \sum_{i=1}^{n} X_i\cdot R_i
            \end{align*}

            Aplicando ahora la Ley de Little en cada dispositivo, tenemos que:
            \begin{align*}
                N_i = X_i\cdot R_i
            \end{align*}

            Por tanto, tenemos que:
            \begin{align*}
                N_0 &= \sum_{i=1}^{n} N_i
            \end{align*}

            Por tanto, se basa en calcular el número de trabajos medio en cada dispositivo, y luego sumarlos para obtener el número de trabajos medio en el sistema.

            En vistas de aplicar la Ley de Little en cada dispositivo, necesitamos calcular el tiempo de respuesta medio de cada dispositivo. Por su definición, tenemos que:
            \begin{align*}
                R_i &= S_i + W_i = S_i + N_i\cdot S_i
            \end{align*}

            Por tanto:
            \begin{align*}
                N_i = X_i\cdot R_i
                = X_i\cdot (S_i + N_i\cdot S_i)
                \Longrightarrow N_i &= \frac{X_i\cdot S_i}{1 - X_i\cdot S_i}
            \end{align*}

            Calculamos ahora la productividad de cada dispositivo, aplicando para ello la Ley del Flujo Reforzado:
            \begin{align*}
                X_i &= X_0\cdot V_i
            \end{align*}

            Por tanto, tenemos que:
            \begin{align*}
                N_{i} &= \frac{X_0\cdot V_i\cdot S_i}{1 - X_0\cdot V_i\cdot S_i} = \frac{X_0\cdot D_i}{1 - X_0\cdot D_i}\\
                N_{CPU} &= \frac{0.3\cdot 2}{1 - 0.3\cdot 2} = \unit[1.5]{trabajos}\\
                N_{Disco A} &= \frac{0.3\cdot 0.42}{1 - 0.3\cdot 0.42} \approx \unit[0.1442]{trabajos}\\
                N_{Disco B} &= \frac{0.3\cdot 0.16}{1 - 0.3\cdot 0.16} \approx \unit[0.05042]{trabajos}
            \end{align*}

            Por tanto, el número de trabajos medio en el sistema es:
            \begin{align*}
                N_0 &= N_{CPU} + N_{Disco A} + N_{Disco B}\\
                &= 1.5 + 0.1442 + 0.05042 \approx \unit[1.69462]{trabajos}
            \end{align*}
        \end{enumerate}
    \end{ejercicio}
\end{document}
