\documentclass[a4paper,12pt]{article}
% Idioma y codificación
\usepackage[spanish, es-tabla, es-notilde]{babel}       %es-tabla para que se titule "Tabla"
\usepackage[utf8]{inputenc}

% Márgenes
\usepackage[a4paper,top=3cm,bottom=2.5cm,left=3cm,right=3cm]{geometry}

% Comentarios de bloque
\usepackage{verbatim}

% Paquetes de links
\usepackage[hidelinks]{hyperref}    % Permite enlaces
\usepackage{url}                    % redirecciona a la web

% Más opciones para enumeraciones
\usepackage{enumitem}

% Personalizar la portada
\usepackage{titling}

% Paquetes de tablas
\usepackage{multirow}

% Para añadir el símbolo de euro
\usepackage{eurosym}


%------------------------------------------------------------------------

%Paquetes de figuras
\usepackage{caption}
\usepackage{subcaption} % Figuras al lado de otras
\usepackage{float}      % Poner figuras en el sitio indicado H.


% Paquetes de imágenes
\usepackage{graphicx}       % Paquete para añadir imágenes
\usepackage{transparent}    % Para manejar la opacidad de las figuras

% Paquete para usar colores
\usepackage[dvipsnames, table, xcdraw]{xcolor}
\usepackage{pagecolor}      % Para cambiar el color de la página

% Habilita tamaños de fuente mayores
\usepackage{fix-cm}

% Para los gráficos
\usepackage{tikz}
\usepackage{forest}

% Para poder situar los nodos en los grafos
\usetikzlibrary{positioning}


%------------------------------------------------------------------------

% Paquetes de matemáticas
\usepackage{mathtools, amsfonts, amssymb, mathrsfs}
\usepackage[makeroom]{cancel}     % Simplificar tachando
\usepackage{polynom}    % Divisiones y Ruffini
\usepackage{units} % Para poner fracciones diagonales con \nicefrac

\usepackage{pgfplots}   %Representar funciones
\pgfplotsset{compat=1.18}  % Versión 1.18

\usepackage{tikz-cd}    % Para usar diagramas de composiciones
\usetikzlibrary{calc}   % Para usar cálculo de coordenadas en tikz

%Definición de teoremas, etc.
\usepackage{amsthm}
%\swapnumbers   % Intercambia la posición del texto y de la numeración

\theoremstyle{plain}

\makeatletter
\@ifclassloaded{article}{
  \newtheorem{teo}{Teorema}[section]
}{
  \newtheorem{teo}{Teorema}[chapter]  % Se resetea en cada chapter
}
\makeatother

\newtheorem{coro}{Corolario}[teo]           % Se resetea en cada teorema
\newtheorem{prop}[teo]{Proposición}         % Usa el mismo contador que teorema
\newtheorem{lema}[teo]{Lema}                % Usa el mismo contador que teorema
\newtheorem*{lema*}{Lema}

\theoremstyle{remark}
\newtheorem*{observacion}{Observación}

\theoremstyle{definition}

\makeatletter
\@ifclassloaded{article}{
  \newtheorem{definicion}{Definición} [section]     % Se resetea en cada chapter
}{
  \newtheorem{definicion}{Definición} [chapter]     % Se resetea en cada chapter
}
\makeatother

\newtheorem*{notacion}{Notación}
\newtheorem*{ejemplo}{Ejemplo}
\newtheorem*{ejercicio*}{Ejercicio}             % No numerado
\newtheorem{ejercicio}{Ejercicio} [section]     % Se resetea en cada section


% Modificar el formato de la numeración del teorema "ejercicio"
\renewcommand{\theejercicio}{%
  \ifnum\value{section}=0 % Si no se ha iniciado ninguna sección
    \arabic{ejercicio}% Solo mostrar el número de ejercicio
  \else
    \thesection.\arabic{ejercicio}% Mostrar número de sección y número de ejercicio
  \fi
}


% \renewcommand\qedsymbol{$\blacksquare$}         % Cambiar símbolo QED
%------------------------------------------------------------------------

% Paquetes para encabezados
\usepackage{fancyhdr}
\pagestyle{fancy}
\fancyhf{}

\newcommand{\helv}{ % Modificación tamaño de letra
\fontfamily{}\fontsize{12}{12}\selectfont}
\setlength{\headheight}{15pt} % Amplía el tamaño del índice


%\usepackage{lastpage}   % Referenciar última pag   \pageref{LastPage}
%\fancyfoot[C]{%
%  \begin{minipage}{\textwidth}
%    \centering
%    ~\\
%    \thepage\\
%    \href{https://losdeldgiim.github.io/}{\texttt{\footnotesize losdeldgiim.github.io}}
%  \end{minipage}
%}
\fancyfoot[C]{\thepage}
\fancyfoot[R]{\href{https://losdeldgiim.github.io/}{\texttt{\footnotesize losdeldgiim.github.io}}}

%------------------------------------------------------------------------

% Conseguir que no ponga "Capítulo 1". Sino solo "1."
\makeatletter
\@ifclassloaded{book}{
  \renewcommand{\chaptermark}[1]{\markboth{\thechapter.\ #1}{}} % En el encabezado
    
  \renewcommand{\@makechapterhead}[1]{%
  \vspace*{50\p@}%
  {\parindent \z@ \raggedright \normalfont
    \ifnum \c@secnumdepth >\m@ne
      \huge\bfseries \thechapter.\hspace{1em}\ignorespaces
    \fi
    \interlinepenalty\@M
    \Huge \bfseries #1\par\nobreak
    \vskip 40\p@
  }}
}
\makeatother

%------------------------------------------------------------------------
% Paquetes de cógido
\usepackage{minted}
\renewcommand\listingscaption{Código fuente}

\usepackage{fancyvrb}
% Personaliza el tamaño de los números de línea
\renewcommand{\theFancyVerbLine}{\small\arabic{FancyVerbLine}}

% Estilo para C++
\newminted{cpp}{
    frame=lines,
    framesep=2mm,
    baselinestretch=1.2,
    linenos,
    escapeinside=||
}

% para minted
\definecolor{LightGray}{rgb}{0.95,0.95,0.92}
\setminted{
    linenos=true,
    stepnumber=5,
    numberfirstline=true,
    autogobble,
    breaklines=true,
    breakautoindent=true,
    breaksymbolleft=,
    breaksymbolright=,
    breaksymbolindentleft=0pt,
    breaksymbolindentright=0pt,
    breaksymbolsepleft=0pt,
    breaksymbolsepright=0pt,
    fontsize=\footnotesize,
    bgcolor=LightGray,
    numbersep=10pt
}


\usepackage{listings} % Para incluir código desde un archivo

\renewcommand\lstlistingname{Código Fuente}
\renewcommand\lstlistlistingname{Índice de Códigos Fuente}

% Definir colores
\definecolor{vscodepurple}{rgb}{0.5,0,0.5}
\definecolor{vscodeblue}{rgb}{0,0,0.8}
\definecolor{vscodegreen}{rgb}{0,0.5,0}
\definecolor{vscodegray}{rgb}{0.5,0.5,0.5}
\definecolor{vscodebackground}{rgb}{0.97,0.97,0.97}
\definecolor{vscodelightgray}{rgb}{0.9,0.9,0.9}

% Configuración para el estilo de C similar a VSCode
\lstdefinestyle{vscode_C}{
  backgroundcolor=\color{vscodebackground},
  commentstyle=\color{vscodegreen},
  keywordstyle=\color{vscodeblue},
  numberstyle=\tiny\color{vscodegray},
  stringstyle=\color{vscodepurple},
  basicstyle=\scriptsize\ttfamily,
  breakatwhitespace=false,
  breaklines=true,
  captionpos=b,
  keepspaces=true,
  numbers=left,
  numbersep=5pt,
  showspaces=false,
  showstringspaces=false,
  showtabs=false,
  tabsize=2,
  frame=tb,
  framerule=0pt,
  aboveskip=10pt,
  belowskip=10pt,
  xleftmargin=10pt,
  xrightmargin=10pt,
  framexleftmargin=10pt,
  framexrightmargin=10pt,
  framesep=0pt,
  rulecolor=\color{vscodelightgray},
  backgroundcolor=\color{vscodebackground},
}

%------------------------------------------------------------------------

% Comandos definidos
\newcommand{\bb}[1]{\mathbb{#1}}
\newcommand{\cc}[1]{\mathcal{#1}}

% I prefer the slanted \leq
\let\oldleq\leq % save them in case they're every wanted
\let\oldgeq\geq
\renewcommand{\leq}{\leqslant}
\renewcommand{\geq}{\geqslant}

% Si y solo si
\newcommand{\sii}{\iff}

% MCD y MCM
\DeclareMathOperator{\mcd}{mcd}
\DeclareMathOperator{\mcm}{mcm}

% Signo
\DeclareMathOperator{\sgn}{sgn}

% Letras griegas
\newcommand{\eps}{\epsilon}
\newcommand{\veps}{\varepsilon}
\newcommand{\lm}{\lambda}

\newcommand{\ol}{\overline}
\newcommand{\ul}{\underline}
\newcommand{\wt}{\widetilde}
\newcommand{\wh}{\widehat}

\let\oldvec\vec
\renewcommand{\vec}{\overrightarrow}

% Derivadas parciales
\newcommand{\del}[2]{\frac{\partial #1}{\partial #2}}
\newcommand{\Del}[3]{\frac{\partial^{#1} #2}{\partial #3^{#1}}}
\newcommand{\deld}[2]{\dfrac{\partial #1}{\partial #2}}
\newcommand{\Deld}[3]{\dfrac{\partial^{#1} #2}{\partial #3^{#1}}}


\newcommand{\AstIg}{\stackrel{(\ast)}{=}}
\newcommand{\Hop}{\stackrel{L'H\hat{o}pital}{=}}

\newcommand{\red}[1]{{\color{red}#1}} % Para integrales, destacar los cambios.

% Método de integración
\newcommand{\MetInt}[2]{
    \left[\begin{array}{c}
        #1 \\ #2
    \end{array}\right]
}

% Declarar aplicaciones
% 1. Nombre aplicación
% 2. Dominio
% 3. Codominio
% 4. Variable
% 5. Imagen de la variable
\newcommand{\Func}[5]{
    \begin{equation*}
        \begin{array}{rrll}
            \displaystyle #1:& \displaystyle  #2 & \longrightarrow & \displaystyle  #3\\
               & \displaystyle  #4 & \longmapsto & \displaystyle  #5
        \end{array}
    \end{equation*}
}

%------------------------------------------------------------------------

\usepackage{fvextra}
\setminted{
  breaklines=true,
  breakanywhere=true,
  linenos=false,
  breakindent=8em  % Puedes ajustar la sangría aquí
}
\title{Memoria Prácticas ISE.\\Monitorización de un Servidor Linux}
\author{Arturo Olivares Martos}
\date{\today}

% Footer con fancyhdr
\fancyhead[L]{Monitorización de un Servidor Linux}

\begin{document}
\maketitle

\begin{abstract}
    En esta pequeña memoria, se incluirán lan capturas más relevantes que pongan de manifiesto la monitorización de un servidor Linux, empleando para ello el software Node Exporter de Prometheus.
\end{abstract}


\tableofcontents
\newpage

\section{Node Exporter}

En la presente sección, describiremos cómo hemos instalado y configurado el software Node Exporter de Prometheus en la Máquina Virtual Rocky Linux.
\begin{enumerate}
    \item En primer lugar, hemos de descargar y descomprimir el \verb|node_exporter|, y guardarlo en \verb|/usr/local/bin/|.
    \begin{minted}{shell}
        $ cd /tmp
        $ curl -LO https://github.com/prometheus/node_exporter/releases/download/v0.18.1/node_exporter-0.18.1.linux-amd64.tar.gz
        $ tar -xvf node_exporter-0.18.1.linux-amd64.tar.gz
        $ sudo mv node_exporter-0.18.1.linux-amd64/node_exporter /usr/local/bin/
    \end{minted}

    Este proceso se muestra en la Figura~\ref{fig:1-Node_Exporter}.
    \begin{figure}[h]
        \centering
        \includegraphics[width=\textwidth]{Images/1-Node_Exporter.png}
        \caption{Descarga y descompresión de Node Exporter}
        \label{fig:1-Node_Exporter}
    \end{figure}


    \item A continuación, hemos de crear un usuario específico para ejecutar el \verb|node_exporter|, de forma que no se ejecute con permisos de \verb|text|. Esto es una buena práctica de seguridad, ya que limita los permisos del servicio. También cambiamos los permisos del binario para que el usuario creado sea el propietario.
    \begin{minted}{shell}
        $ sudo useradd -rs /bin/false node_exporter
        $ sudo chown node_exporter:node_exporter /usr/local/bin/node_exporter
    \end{minted}


    También le indicamos a SELinux que este fichero es un ejecutable:
    \begin{minted}{shell}
        $ sudo chcon -t bin_t /usr/local/bin/node_exporter
    \end{minted}

    \item Ahora, hemos de crear un archivo de configuración para el servicio \verb|node_exporter|. Este archivo se encuentra en \verb|/etc/systemd/system/node_exporter.service| y contiene la configuración del servicio. El contenido del archivo es el siguiente:
    \begin{minted}{ini}
    [Unit]
    Description=Node Exporter
    After=network.target

    [Service]
    User=node_exporter
    Group=node_exporter
    Type=simple
    ExecStart=/usr/local/bin/node_exporter --collector.systemd --collector.processes

    [Install]
    WantedBy=multi-user.target
    \end{minted}


    \item A continuación, hemos de abrir el puerto 9100 en el firewall para permitir el acceso al servicio \verb|node_exporter| (puesto que, por defecto, este servicio escucha en el puerto 9100). Recordamos que también debemos hacer uso de \verb|SELinux| para evitar problemas de permisos. Para ello, ejecutamos los siguientes comandos:
    \begin{minted}{shell}
        $ sudo firewall-cmd --add-port=9100/tcp --permanent
        $ sudo firewall-cmd --reload
        $ sudo semanage port -a -t http_port_t -p tcp 9100
    \end{minted}
    La Figura~\ref{fig:2-Node_Exporter} muestra el proceso de apertura del puerto 9100 en el firewall.
    \begin{figure}[h]
        \centering
        \includegraphics[width=\textwidth]{Images/2-Node_Exporter.png}
        \caption{Apertura del puerto 9100 en el firewall}
        \label{fig:2-Node_Exporter}
    \end{figure}

    \item Ahora, hemos de recargar el demonio \verb|systemd| para que reconozca el nuevo servicio y luego iniciar el servicio \verb|node_exporter|. Comprobamos que el servicio se ha iniciado correctamente y que está escuchando en el puerto 9100:
    \begin{minted}{shell}
        $ sudo systemctl daemon-reload
        $ sudo systemctl start node_exporter
        $ sudo systemctl status node_exporter
    \end{minted}
    La Figura~\ref{fig:3-Node_Exporter} muestra el estado del servicio \verb|node_exporter|.
    \begin{figure}[h]
        \centering
        \includegraphics[width=\textwidth]{Images/3-Node_Exporter.png}
        \caption{Estado del servicio Node Exporter}
        \label{fig:3-Node_Exporter}
    \end{figure}
    Habilitamos el servicio para que se inicie automáticamente al arrancar la máquina:
    \begin{minted}{shell}
        $ sudo systemctl enable node_exporter
    \end{minted}

    Una vez realizado todo, podemos comprobar que el servicio \verb|node_exporter| está funcionando correctamente accediendo desde un navegador a la dirección dada por \verb|http://<IP Maquina Virtual>:9100/metrics|. Aquí vemos una lista de métricas en formato texto, que son las que \verb|node_exporter| está exponiendo. La Figura~\ref{fig:4-Node_Exporter} muestra un ejemplo de estas métricas.
    \begin{figure}[h]
        \centering
        \includegraphics[width=\textwidth]{Images/4-Node_Exporter.png}
        \caption{Métricas expuestas por Node Exporter}
        \label{fig:4-Node_Exporter}
    \end{figure}

    También podemos comprobar que el servicio funciona correctamente desde Prometheus, comprobando que funciona correctamente. La Figura~\ref{fig:5-Node_Exporter} muestra cómo se ve el servicio en la interfaz de Prometheus.
    \begin{figure}[h]
        \centering
        \includegraphics[width=\textwidth]{Images/5-Node_Exporter.png}
        \caption{Node Exporter en Prometheus}
        \label{fig:5-Node_Exporter}
    \end{figure}
\end{enumerate}

\section{Panel de Servicios}

En el siguiente panel, mostramos si dos servicios (en este caso \verb|SSHD| y \verb|HTTPD|) están activos o no. Varias pruebas se ven en las Figuras~\ref{fig:Grafana-Services_1} y~\ref{fig:Grafana-Services_2}.
\begin{figure}[h]
    \centering
    \includegraphics[width=\textwidth]{Images/Grafana-Services_1.png}
    \caption{Panel de Servicios. SSHD activo y HTTPD desinstalado.}
    \label{fig:Grafana-Services_1}
\end{figure}
\begin{figure}[h]
    \centering
    \includegraphics[width=\textwidth]{Images/Grafana-Services_2.png}
    \caption{Panel de Servicios. SSHD inactivo y HTTPD activo.}
    \label{fig:Grafana-Services_2}
\end{figure}

\section{Uso de la CPU}

En el siguiente panel, mostramos el uso de la CPU del servidor. Añadimos también el hecho de poner una alerta si el uso de la CPU supera el 75\% durante más de 5 minutos.

\subsection{Lanzamiento Carga CPU}

El lanzamiento de carga en la CPU se ha realizado con el siguiente comando:
\begin{minted}{shell}
    $ stress-ng --cpu 4 --timeout 1000s
\end{minted}

La Figura~\ref{fig:Grafana-CPU_1} muestra el uso de la CPU antes y después de lanzar la carga. Además, en la parte inferior del panel, se muestra el comando que se ha ejecutado para lanzar la carga en la CPU.
\begin{figure}[h]
    \centering
    \includegraphics[width=\textwidth]{Images/Grafana-CPU_1.png}
    \caption{Uso de la CPU antes y después de lanzar la carga.}
    \label{fig:Grafana-CPU_1}
\end{figure}
\subsection{Alerta de Uso de la CPU}
La Figura~\ref{fig:Grafana-CPU_2} muestra el disparo de alerta que se ha generado al superar el 75\% de uso de la CPU durante más de 5 minutos. Como vemos, se detectó rápidamente el uso elevado de la CPU y se generó una alerta pendiente. Cuando está por encima del 75\% durante más de 5 minutos, se genera una alerta pendiente.
\begin{figure}[h]
    \centering
    \includegraphics[width=\textwidth]{Images/Grafana-CPU_2.png}
    \caption{Alerta de uso elevado de la CPU.}
    \label{fig:Grafana-CPU_2}
\end{figure}



\end{document}