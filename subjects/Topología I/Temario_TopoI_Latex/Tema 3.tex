\chapter{Conexión y Compacidad}

\section{Conexión}

\begin{definicion}[Conexión]
    Decimos que un espacio topológico $(X,\T)$ es conexo si no existen dos abiertos disjuntos $U_1,U_2\subset \T,~U_1\cap U_2=\emptyset$, tales que $X=U_1\cup U_2$; salvo el vacío y el total.

    En el caso de que un conjunto no sea conexo, diremos que es disconexo.
\end{definicion}

\begin{prop}[Caracterización de la Conexión]
    Sea $(X,\T)$ un espacio topológico. Entonces, las siguientes afirmaciones son equivalentes:
    \begin{enumerate}
        \item $(X,\T)$ es conexo.
        \item No existen dos cerrados disjuntos $C_1,C_2\subset C_{\T},~C_1\cap C_2=\emptyset$, tales que $X=C_1\cup C_2$; salvo el vacío y el total.
        \item Si $A\subset X$ es un conjunto abierto y cerrado, entonces $A=\emptyset$ o bien $A=X$.
        \item Toda aplicación continua $f:(X,\T)\to \left(\{0,1\},\T_{disc}\right)$ es constante.
    \end{enumerate}
\end{prop}
\begin{proof}\
    \begin{description}
        \item[$1)\Longrightarrow 2)$]
        Demostramos el contrarrecíproco. Supongamos que existen dos cerrados disjuntos $C_1,C_2\subset C_{\T},~C_1\cap C_2=\emptyset$, tales que $X=C_1\cup C_2$; con $C_1,C_2\neq \emptyset, X$.
        Entonces, sean $U_1=X\setminus C_1,~U_2=X\setminus C_2$. Tenemos que $U_1,U_2\in \T$, ya que $C_1,C_2\in C_{\T}$.
        Además, estos abiertos son disjuntos, ya que:
        $$U_1\cap U_2=X\setminus C_1 \cap X\setminus C_2 = X\setminus (C_1\cup C_2) = X\setminus X = \emptyset$$.
        Además, como $C_1,C_2\neq \emptyset, X$, entonces $U_1,U_2\neq X, \emptyset$.

        Por tanto, $(X,\T)$ es disconexo, por lo que queda demostrado el contrarrecíproco.

        \item [$2)\Longrightarrow 3)$] Sea $A$ un conjunto abierto y cerrado, y consideramos $C_1=A,~C_2=X\setminus A$. Tenemos que $C_1,C_2\in C_{\T}$, y $C_1\cap C_2=\emptyset$.
        Además, $X=C_1\cup C_2$. Por tanto, como $(X,\T)$ es conexo, $C_1=A=\emptyset$ o $C_1=A=X$.

        \item [$3)\Longrightarrow 1)$] Demostramos el contrarrecíproco. Supongamos que existen dos abiertos disjuntos $U_1,U_2\subset \T,~U_1\cap U_2=\emptyset$, tales que $X=U_1\cup U_2$; con $U_1,U_2\neq \emptyset, X$.
        Entonces, como $X=U_1\cup U_2$ y $U_1\cap U_2=\emptyset$, tenemos que $U_2=X\setminus U_1$, por lo que es cerrado. Por tanto, $U_2\neq \emptyset, X$ es abierto y cerrado a la vez, por lo que queda demostrado el contrarrecíproco.

        \item [$1)\Longrightarrow 4)$] Demostramos el contrarrecíproco.
        Si existe $f:(X,\T)\to \left(\{0,1\},\T_{disc}\right)$ continua y no constante, entonces definimos los siguientes conjuntos:
        \begin{equation*}
            U_1=f^{-1}(\{0\})\in \T
            \qquad
            U_2=f^{-1}(\{1\})\in \T
        \end{equation*}

        Tenemos que $U_1,U_2\notin \{\emptyset, X\}$ ya que no es constante; y se tiene que $U_1\cap U_2=\emptyset$, $U_1\cup U_2 = X$. Por tanto, $(X,\T)$ es disconexo, por lo que queda demostrado el contrarrecíproco.

        \item [$4)\Longrightarrow 1)$] Demostramos el contrarrecíproco. Como $X$ no es conexo, existen dos abiertos $U_1,U_2\in \T$, con $U_1,U_2\notin \{\emptyset, X\}$ tal que $U_1\cap U_2=\emptyset$, $U_1\cup U_2=X$.

        Entonces, definimos $f$ de la siguiente forma:
        \begin{equation*}
            f(x) = \left\{\begin{array}{ccl}
                0 & \text{si} & x\in U_1 \\
                1 & \text{si} & x\notin U_1 \Longleftrightarrow x\in U_2\Longleftrightarrow x\in X\setminus U_1 \\
            \end{array}\right.
        \end{equation*}

        Tenemos que $f$ no es constante, ya que ninguno de los conjuntos es el vacío. Además, $f$ es continua, ya que:
        \begin{equation*}
            f^{-1}(\emptyset)=\emptyset
            \qquad 
            f^{-1}(\{0,1\})=X
            \qquad 
            f^{-1}(\{0\})=U_1
            \qquad 
            f^{-1}(\{1\})=U_2
        \end{equation*}
        Por tanto, $f$ es continua, ya que la imagen inversa de todo abierto es un abierto; pero no es constante, por lo que queda demostrado el contrarrecíproco.
    \end{description}
\end{proof}



\begin{observacion}
    Si $(X,\T)$ es un espacio topológico y $A\subset X$, decimos que $A$ es conexo si $(A,\T_A)$ es conexo. Es decir, cuando se habla de un subconjunto, implícitamente nos referimos a la topología inducida.
\end{observacion}

\begin{ejemplo}
    Veamos algunos ejemplos de conjuntos conexos:
    \begin{enumerate}
        \item $(X,\T_t)$ es conexo, ya que los únicos abiertos son el vacío y el total.

        \item Sea $X$ un conjunto tal que $|X|>1$. Entonces, $(X,\T_{disc})$ no es conexo, ya que fijado $x_0\in X$, entonces:
        \begin{equation*}
            X = (X\setminus \{x_0\})\cup \{x_0\}
        \end{equation*}

        \item $(X,\T_{CF})$. Si $X$ es finito, tenemos la topología discreta, por lo que no es conexo. Si $X$ es infinito, sí es conexo. Veámoslo:

        Sea $C\in \T_{CF}\cap C_{\T_{CF}}$, y supongamos que no es ni el vacío ni el total. Entonces, por ser cerrado es finito, y su complementario es también finito por ser abierto. Entonces, $X=C\cup (X\setminus C)$ es finito, por lo que llegamos a una contradicción. Por tanto, es conexo.

        \item Sea la hipérbola $H=\{(x,y)\in \bb{R}^2\mid xy=1\}$.

        Tenemos que $H$ no es conexo. Para verlo, sea $H^+$ el siguiente conjunto:
        \begin{equation*}
            H^+ = \{(x,y)\in \bb{R}^+\mid xy=1,~ x>0\}
        \end{equation*}

        Tenemos que:
        \begin{align*}
            H^+ &= H\cap (\bb{R}^+\times \bb{R}^+)\in {\T_u}_H\\
            H^+ &= H\cap (\bb{R}^+_0\times \bb{R}^+_0)\in C_{{\T_u}_H}
        \end{align*}

        Por tanto, $H^+$ es un abierto y cerrado. Como $H^+\neq \emptyset, H$, entonces $H$ no es conexo.
    \end{enumerate}
\end{ejemplo}

\begin{prop}
    Los conjuntos conexos de $(\bb{R}, \T_u)$ son exactamente los intervalos. Es decir, dado $A\subset \bb{R}$, tenemos que:
    \begin{equation*}
        A \text{ es conexo} \Longleftrightarrow A \text{ es un intervalo}
    \end{equation*}
\end{prop}
\begin{proof}
    Demostramos por doble implicación:
    \begin{description}
        \item[$\Longrightarrow)$] Sea $A$ conexo, y supongamos que no es un intervalo. Entonces, existen $a,b\in A$, $x\in X\setminus A$, tales que $a<x<b$. Entonces, consideramos los siguientes conjuntos:
        \begin{align*}
            U_1 &= A\cap ]-\infty, x[ \\
            U_2 &= A\cap ]x, \infty[
        \end{align*}

        Tenemos que $U_1,U_2\in {\T_u}_A$, ya que son intersecciones de abiertos de $T_u$ con $A$.
        Además, $U_1\cap U_2=\emptyset$. Además, $U_1\cup U_2=A\setminus \{x\}=A$.

        Como $a\in U_1, b\notin U_1$, entonces $U_1\neq \emptyset, A$. Análogamente, se demuestra que $U_2\neq \emptyset, A$. Por tanto, $A$ es disconexo, llegando a una contradicción, por lo que $A$ es un intervalo.
        
        \item[$\Longleftarrow)$] Supongamos que $A\subset \bb{R}$ es un intervalo, y sea $f:(A,\T_A)\to \left(\{0,1\},\T_{disc}\right)$ continua. Tenemos que demostrar que $f$ es constante.
        
        Por el Teorema del Valor Intermedio visto en Cálculo I, tenemos que que $f(A)$ es un intervalo. Pero como $f(A)\subset \{0,1\}$, entonces $f(A)=\{0\}$ o bien $f(A)=\{1\}$, por lo que $f$ es constante.
    \end{description}
\end{proof}


\begin{teo}
    La imagen mediante una aplicación continua de un conexo es un conexo.
    
    Es decir, si $f:(X,\T)\to (Y,\T')$ es continua y $(X,\T)$ es conexo, entonces $f(X)$ es conexo.
\end{teo}
\begin{proof}
    Veamos el contrarrecíproco. Si $f(X)$ no fuese conexo, entonces por la caracterización de los conjuntos conexos, $\exists g:f(X)\to \left(\{0,1\},\T_{disc}\right)$ continua y no constante. Por tanto,
    \begin{equation*}
        g\circ f:(X,\T)\to \left(\{0,1\},\T_{disc}\right)
    \end{equation*}
    es continua (por ser composición de continuas) y no constante (ya que $g$ no lo es), por lo que $(X,\T)$ no es conexo, llegando a una contradicción.
\end{proof}

Como consecuencia inmediata, tenemos que si $f$ es un homeomorfismo, se trata de una doble implicación, ya que se puede considerar $f^{-1}$, también continua.
\begin{coro}
    La conexión es una propiedad topológica. Es decir, si $f:(X,\T)\to (Y,\T')$ es un homeomorfismo, entonces $f(X)=Y$ es conexo si y solo si $X$ es conexo.
\end{coro}

Para el caso de los espacios cocientes, tenemos el siguiente corolario:
\begin{coro}
    Sea $(X,\T)$ un espacio topológico. Entonces, todos sus espacios cocientes son conexos.
\end{coro}
\begin{proof}
    Consideramos $p:(X,\T)\to (X/\cc{R}, \T/\cc{R})$ la proyección canónica en el cociente. Entonces, por el teorema anterior, como $p$ es continua, $p(X)$ es conexo. Pero como $p(X)=X/\cc{R}$ por ser $p$ sobreyectiva, entonces $X/\cc{R}$ es conexo.
\end{proof}

\begin{ejemplo}
    Veamos algunos resultados sobre el teorema anterior:
    \begin{enumerate}
        \item $(\bb{S}^1, {\T_u}_{\bb{S}^1})$ es conexo.

        Sabemos que $\bb{R}$ es conexo, por ser un intervalo. Nos definimos la siguiente aplicación:
        \Func{f}{(\bb{R}, \T_u)}{(\bb{S}^1, {\T_u}_{\bb{S}^1})}{x}{(\cos x, \sen x)}

        Tenemos que $f$ es continua y sobreyectiva, por lo que $f(\bb{R})=\bb{S}^1$ es conexo.

        \item $[a,b],~[a,b[, ~]a,b], ~]a,b[$ con $a<b$ no son homeomorfos.

        Supongamos que sí, lo son, por lo que existe el siguiente homeomorfismo:
        \begin{equation*}
            h:[a,b]\to [a,b[
        \end{equation*}

        Como $h$ es inyectiva, $h(a)$ o bien $h(b)$ no son $a$. Suponemos sin pérdida de generalidad que $h(b)\neq a$. Entonces:
        \begin{equation*}
            h_{\big| [a,b[}:[a,b[\longrightarrow [a,b[\setminus \{h(b)\}
        \end{equation*}
        es un homeomorfismo (luego continua). Pero $[a,b[$ es conexo y $[a,b[\setminus \{h(b)\}$ no lo es (no es un intervalo), por lo que llegamos a una contradicción.

        Para el resto de casos, se demuestra de forma análoga.

        \item $(\bb{R}, \T_u)$ no es homeomorfo a $(\bb{S}^1, {\T_u}_{\bb{S}^1})$.

        Supongamos que sí existe $h:\bb{R}\to \bb{S}^1$ homeomorfismo. Entonces, consideramos la siguiente restricción:
        \begin{equation*}
            h_{\big | \bb{R}\setminus \{0\}}: \bb{R}\setminus \{0\} \to \bb{S}^1\setminus \{h(0)\}
        \end{equation*}
        que también es un homeomorfismo. Como $\bb{S}^1\setminus \{h(0)\}$ es homeomorfo a $\bb{R}$, que es conexo, tenemos que $\bb{S}^1\setminus \{h(0)\}$ es conexo. No obstante, $\bb{R}\setminus \{0\}$ no lo es, por lo que llegamos a una contradicción.
    \end{enumerate}
\end{ejemplo}

\begin{coro}
    Sea un intervalo abierto $]a,b[\subset \bb{R}$, y sea $A\subset \bb{R}$. Entonces, Si $A$ es homeomorfo a $]a,b[$~, entonces $A$ es un intervalo abierto.
\end{coro}
\begin{proof}
    Si $A$ es homeomorfo a $]a,b[$, entonces $A$ debe ser conexo y así $A$ es un intervalo. Entonces, $A$ es uno de los siguientes tipos:
    \begin{align*}
        ]c,d[ &\quad -\infty \leq c<d\leq \infty\\
        [c,d[ &\quad -\infty < c<d\leq \infty \\
        ]c,d] &\quad -\infty \leq c<d< \infty \\
        [c,d] &\quad -\infty < c<d< \infty
    \end{align*}
    Por el apartado 2 del resultado anterior, tenemos que $A$ no es homeomorfo a los intervalos cerrados por uno o ambos extremos,
    por lo que:
    \begin{align*}
        A=]c,d[ &\quad -\infty \leq c<d\leq \infty \\& \hfill \qedhere
    \end{align*}
\end{proof}


Como consecuencia de lo visto, tenemos el siguiente resultado:
\begin{teo}[Invarianza del Dominio]
    Sean $A,B\subset \bb{R}$. Si $A,B$ son homeomorfos con la topología usual y $A$ es abierto, entonces $B$ es abierto.
\end{teo}
\begin{proof}
    Sea $h:A\to B$ un homeomorfismo entre ambos. Si $A\in \T_u$, como los intervalos son una base de la topología usual, entonces:
    \begin{equation*}
        A=\bigcup_{i\in I}~]c_i, d_i[ \text{\hspace{1cm} con } c_i<d_i,~ c_i,d_i\in \bb{R}
    \end{equation*}

    Entonces, la siguiente restricción es otro homomorfismo:
    \begin{equation*}
        h_{\big | ]c_i,d_i[}:]c_i,d_i[\to h(]c_i,d_i[)
    \end{equation*}

    Por consecuencia del corolario anterior, $h(]c_i,d_i[)$ es un intervalo abierto de $\bb{R}$, por lo que:
    \begin{equation*}
        B=h(A)=h\left(\bigcup_{i\in I}~]c_i, d_i[\right)=\bigcup_{i\in I} h(]c_i,d_i[)\in \T_u
    \end{equation*}
\end{proof}


\begin{teo}[del Valor Intermedio]
    Sea $f:(X,\T)\to (\bb{R}, \T_u)$ una aplicación continua con $(X,\T)$ conexo. Si existen $x_0,x_1\in X$ tales que $f(x_0)=a<b=f(x_1)$, entonces para todo $c\in ]a,b[$, existe $x\in X\mid f(x)=c$.
\end{teo}
\begin{proof}
    Supongamos que no existe $x\in X$ tal que $f(x)=c$.
    Entonces, $f(X)$ no es un intervalo, ya que $a < c < b$, con $a,b\in f(X)$ pero $c\notin f(X)$.
    Por tanto, $f(X)$ no es conexo, llegando a una contradicción.
\end{proof}


\begin{teo}[Borsuk-Ulam]
    Sea $f:(\bb{S}^n, {\T_u}_{\bb{S}^n})\to (\bb{R}, \T_u)$ una aplicación continua. Entonces, existe $p\in \bb{S}^n$ tal que $f(p)=f(-p)$.
\end{teo}
\begin{proof}
    Definimos la siguiente aplicación, que es continua por ser diferencia de continuas:
    \Func{g}{\bb{S}^n}{\bb{R}}{x}{f(x)-f(-x)}
    El teorema quedaría demostrado si encontramos $p\in \bb{S}^n$ tal que $g(p)=0$.
    
    Sea $q\in \bb{S}^n$. Si $g(q)=0$, $q$ es el valor buscado.
    En caso contrario, supongamos sin pérdida de generalidad que $g(q)>0$. Entonces,
    $-q\in \bb{S}^n$ y $$g(-q)=f(-q)-f(q)=-g(q)<0$$
    
    Por tanto, por el teorema del valor intermedio, como $g$ es continua y $\bb{S}^n$ es conexo (se demuestra más adelante, en el apartado \ref{ex:SnConexo} del ejemplo de la página \pageref{ex:SnConexo}), $\exists p\in \bb{S}^n$ tal que $g(p)=0$, es decir, $f(p)=f(-p)$.
\end{proof}


\begin{prop}
    Sea $(X,\T)$ un espacio topológico, y $\{A_i\}_{i\in I}$ una familia de conjuntos conexos con $A_i\subset X$ para todo $i\in I$.
    \begin{enumerate}
        \item Si $\bigcap\limits_{i\in I}A_i\neq \emptyset$, entonces $\bigcup\limits_{i\in I}A_i$ es conexo.
        \item Si existe $i_o\in I$ tal que $A_i\cap A_{i_0}\neq \emptyset~ \forall i\in I$, entonces $\bigcup\limits_{i\in I}A_i$ es conexo.
        \item Si la familia $A_i$ fuese numerable; es decir, podemos escribirla como $\{A_{i_n}\}_{n\in \bb{N}}$ (o cantidad finita) y se cumple que $A_{i_n}\cap A_{i_{n+1}}\neq \emptyset$, entonces $\bigcup\limits_{i\in I}A_i= \bigcup\limits_{n\in \bb{N}}A_{i_n}$ es conexo.
    \end{enumerate}
    % // TODO: Dibujos de las uniones
\end{prop}
\begin{proof} Demostramos cada una por separado:
    \begin{enumerate}
        \item Sea $f:\bigcup\limits_{i\in I}A_i \to (\{0,1\}, \T_{disc})$ continua. Tenemos que demostrar que es constante. Como cada $A_i$ es conexo, entonces:
        \begin{equation*}
            f_{\big |A_i}:A_i\to (\{0,1\}, \T_{disc})
        \end{equation*}
        es constante. Además, existe $x_0\in \bigcap\limits_{i\in I}A_i$, por lo que $f(x)=f(x_0)~ \forall x\in A_i$.
    
        Uniendo ambos resultados, tenemos que $f(x)=f(x_0)~\forall x\in \bigcup\limits_{i\in I}A_i$, por lo que $f$ es constante.

        \item Sea $f:\bigcup\limits_{i\in I}A_i \to (\{0,1\}, \T_{disc})$ continua. Tenemos que demostrar que es constante. Como cada $A_i$ es conexo, entonces:
        \begin{equation*}
            f_{\big |A_i}:A_i\to (\{0,1\}, \T_{disc})
        \end{equation*}
        es constante. En particular, $f_{\big |A_{i_0}}$ es constante. Sea $x_{i_0}\in A_{i_0}$.
        
        Para cada $i\in I$, $\exists x_i\in A_i\cap A_{i_0}$, por lo que $f(x_i)=f(x_{i_0})$.
        Por tanto, uniendo ambos resultados, tenemos que $f(x)=f(x_{i_0})~\forall x\in \bigcup\limits_{i\in I}A_i$, por lo que $f$ es constante.

        \item Definimos $Y_n = \bigcup\limits_{\substack{k\in \bb{N}\\k\leq n}}A_{i_k}$. Demostremos que $Y_n$ es conexo para todo $n\in \bb{N}$:        
        \begin{itemize}
            \item $n=1$. Es trivial, ya que $Y_1 = A_{i_1}$, que es conexo.
            \item Supuesto cierto para $n$, demostramos para $n+1$.
            
            Tenemos que $Y_{n+1}=Y_n\cup A_{i_{n+1}}$, ambos conexos. Además,
            \begin{equation*}
                Y_n \cap A_{i_{n+1}} = \left(\bigcup\limits_{\substack{k\in \bb{N}\\k\leq n}}A_{i_k}\right) \bigcap A_{i_{n+1}}
                = \bigcup\limits_{\substack{k\in \bb{N}\\k\leq n}}\left(A_{i_k} \cap A_{i_{n+1}}\right)
                \supset A_{i_{n}}\cap A_{i_{n+1}}\neq \emptyset
            \end{equation*}
            Por tanto, por el apartado 1 de esta proposición, $Y_{n+1}$ es conexo.
        \end{itemize}

        Como $\emptyset\neq A_{i_1}\subset \bigcap\limits_{n\in \bb{N}}Y_n$, entonces por el apartado 1 de esta proposición, $\bigcup\limits_{n\in \bb{N}}Y_n = \bigcup\limits_{n\in \bb{N}}A_{i_n}$ es conexo.
    \end{enumerate}
\end{proof}

\begin{definicion}[Conjunto estrellado]
    Decimos que un conjunto $C\subset \bb{R}^n$ es estrellado desde $x_0\in C$ si para todo $x\in C$, se cumple que el segmento que une $x$ con $x_0$, $[x,x_0]$, está en $C$. Es decir,
    \begin{equation*}
        (1-t)x + tx_0 \in C  \qquad \forall t\in [0,1],\forall x\in C
    \end{equation*}
    %// TODO: Dibujo conjunto estrellado. Estrella

    Decimos que un conjunto $C\subset \bb{R}^n$ es estrellado si es estrellado desde algún $x_0\in C$.
\end{definicion}
\begin{ejemplo}
    Veamos algunos ejemplos de conjuntos estrellados:
    \begin{enumerate}
        \item $\bb{R}^n$ es estrellado. De hecho, es estrellado desde cualquier punto.
        \item La semirrecta con inicio en el origen que pasa por $x$, $S_x = \{\lm x \mid \lm > 0\}$, es estrellado.
        
        En efecto, dado $x_0\in S_x$, entonces $x_0=\lm x$, con $\lm > 0$. Entonces, para todo $t\in [0,1]$:
        $$(1-t)x+tx_0=(1-t)x+t\lm x=(1-t+t\lm)x\in S_x \Longrightarrow
        1-t+t\lm > 0$$
        que, como $t\in [0,1]$, se cumple.
        
        Por tanto, $S_x$ es estrellado desde $x_0$.
    \end{enumerate}
\end{ejemplo}

\begin{prop}
    Todo conjunto estrellado $C\subset \bb{R}^n$ es conexo. En particular, $\bb{R}^n$ es conexo.
\end{prop}
\begin{proof} 
    Cada segmento cerrado uniendo dos puntos $x,x_0\in \bb{R}^n$ es un conjunto conexo, ya que son la imagen de la siguiente aplicación:
    \Func{f}{[0,1]}{\bb{R}^n}{t}{(1-t)x+tx_0}
    que es continua. Como $[0,1]$ es conexo, entonces todos los segmentos dichos son conexos.

    Entonces, como $C$ es estrellado, supongamos que lo es desde $x_0\in C$:
    \begin{equation*}
        C = \bigcup_{x\in C}\{x\} \subset \bigcup_{x\in C}[x,x_0] \stackrel{(\ast)}{\subset} C
    \end{equation*}
    donde en $(\ast)$ hemos usado que $C$ es estrellado. Por tanto, como $x_0\in \bigcap\limits_{x\in C}[x,x_0]$, por el primer apartado del resultado anterior tenemos que $C$ es conexo.
\end{proof}

\begin{ejemplo}
    Otros ejemplos de conjuntos conexos son:
    \begin{enumerate}
        \item $\bb{R}^n$ es conexo por ser estrellado.
        \item \label{ex:SnConexo}
        La esfera $\bb{S}^n$ es conexo con la ${\T_u}_{\bb{S}^n}$.

        Tenemos que:
        \begin{equation*}
            \bb{S}^n=(\bb{S}^n\setminus \{N\}) \cup (\bb{S}^n\setminus \{S\})
        \end{equation*}
    
        Ambos conjuntos son conexos por ser homeomorfos a $\bb{R}^n$, y su intersección es no vacía. Por tanto, $\bb{S}^n$ es conexo.

        \item $\bb{R}^n \setminus \{0\}$ es conexo para $n\geq 2$.

        Podemos ver $\bb{R}^n\setminus \{0\}$ como la unión de $\bb{S}^{n-1}$ con todas las semirrectas $S_x$ para cada $x\in \bb{S}^{n-1}$. La esfera tenemos que es conexa, y las semirrectas $S_x$ son estrellados, luego conexos.

        Como cada $S_x\cap \bb{S}^{n-1}=\{x\}\neq \emptyset$, entonces $\left(\bigcup\limits_{x\in \bb{S}^{n-1}}S_x\right)\cup \bb{S}^{n-1}=\bb{R}\setminus \{0\}$ es conexo.
    \end{enumerate}
\end{ejemplo}


\begin{teo}
    Sea $(X,\T)$ un espacio topológico, y $A\subset X$ un conjunto conexo. Entonces, para cualquier $B\subset X$ tal que $A\subset B\subset \ol{A}$ se tiene que $B$ es conexo.

    En particular, la adherencia de un conexo también es conexo.
\end{teo}
\begin{proof}
    Sea $f:B\to (\{0,1\},\T_{disc})$ continua. Tenemos que demostrar que $f$ es constante. Consideramos la restricción a $A$:
    \begin{equation*}
        f_{\big| A}: A \to (\{0,1\},\T_{disc})
    \end{equation*}
    Tenemos que dicha restricción es constante por ser $A$ conexo.
    Estudiamos ahora en $B\setminus A$.

    Sea $b\in B\setminus A$, y consideramos $\{f(b)\}\in \T_{disc}$. Por ser $f$ continua, también es continua en $b$,
    por lo que $\exists N\in N_b$ tal que $f(N)\subset \{f(b)\}$. Es decir, $f$ es constante en un entorno de $b$.

    Como $b\in B\setminus A \subset \ol{A}\setminus A$, entonces $b\in \partial A$. Por tanto, $\exists x\in A\cap N$. Como $f$ es constante en $N$, $f(x)=f(b)$. Además, como $x\in A$ y $f$ es constante en $A$, tenemos que $f(b)=f(x)$ para todo $x\in A$.

    Por tanto, $\forall b\in B\setminus A$, tenemos que $f(b)$ es constante e igual a $f(x)~\forall x\in A$.

    Por tanto, $f$ es constante en $B$, por lo que $B$ es conexo.
\end{proof}


\begin{prop}
    Sean $(X,\T), (Y,\T')$ dos espacios topológicos. Entonces, el producto $(X\times Y, \T\times \T')$ es conexo si y solo si $(X,\T), (Y,\T')$ son conexos.
\end{prop}
\begin{proof}
    Demostramos por doble implicación:
    \begin{description}
        \item[$\Longrightarrow)$]
        Si $X\times Y$ es conexo, como la proyección sobre $X$, $\pi_X:X\times Y\to X$ es continua y sobreyectiva, entonces $\pi_X(X\times Y)=X$ es conexo.

        Análogamente, se demuestra que $Y$ es conexo.

        \item[$\Longleftarrow)$] Veamos que $X\times Y$ es unión de conexos de forma que todos corten a uno fijo.

        Para cada $y\in Y$, consideramos los siguientes conjuntos:
        \begin{equation*}
            X_y = X\times \{y\},\qquad \text{con } y\in Y
        \end{equation*}
        Tenemos que $X_y$ es homeomorfo a $X$ para todo $y\in Y$, por lo que es conexo.

        Además, fijado $x_0\in X$, consideramos e siguiente conjunto homeomorfo a $Y$, luego conexo:
        \begin{equation*}
            Y_{x_0} = \{x_0\}\times Y
        \end{equation*}

        Por tanto, tenemos que $X\times Y = \bigcup\limits_{y\in Y}X_y = \left(\bigcup\limits_{y\in Y}X_y\right) \cup Y_{x_0}$, y la intersección $X_y\cap Y_{x_0}=\{(x_0,y)\}\neq \emptyset$ es no vacía para todo $y\in Y$.

        Por tanto, $X\times Y$ es conexo.
    \end{description}
\end{proof}


\begin{ejemplo}
    Ejemplos de conjuntos producto que son conexos son:
    \begin{enumerate}
        \item El cilindro $\bb{S}^1\times \bb{R}=\{(x,y,z)\in \bb{R}^3\mid x^2 + y^2=1\}$ es conexo por ser producto de conexos.

        \item El toro es conexo por ser homeomorfo a $\bb{S}^1\times \bb{S}^1$, que es producto de conexos.

        \item El espacio proyectivo es también conexo.
    \end{enumerate}
\end{ejemplo}


\subsection{Componentes Conexas}
\begin{definicion}
    Sea $(X,\T)$ un espacio topológico, y consideramos $x_0\in X$. Llamamos componente conexa de $x_0$ al mayor conexo que contiene a $x_0$. Lo denotaremos por $C(x_0)$.
\end{definicion}
\begin{ejemplo}
    Veamos algunos ejemplos de componentes conexas:
    \begin{enumerate}
        \item En $\bb{R}\setminus \{0\}$, tenemos que $C(1)=\bb{R}^+$.
        \item En $\bb{Q}$, tenemos que $C(1)=\{1\}$.
    \end{enumerate}
\end{ejemplo}


\subsubsection{Propiedades}
\begin{enumerate}
    \item Las componentes conexas de un espacio topológico $(X,\T)$ dan una partición de $X$. Es decir, dos componentes conexas distintas son disjuntas, y la unión de todas las componentes conexas es $X$.\\

    Tomamos dos puntos distintos $x,y\in X$. Veamos que $C(x)=C(y)$ o bien $C(x)\cap C(y)=\emptyset$.

    Supongamos que $C(x)\cap C(y)\neq \emptyset$. Entonces, $C(x)\cup C(y)$ es conexo. Por tanto,
    \begin{equation*}
        x\in C(x)\subset C(x)\cup C(y) \stackrel{(\ast)}{\subset} C(x)
    \end{equation*}
    donde en $(\ast)$ se ha aplicado que $C(x)$ es el mayor conexo que contiene a $x$. Por tanto, $C(x)=C(x)\cup C(y)$. Análogamente, se demuestra que $C(y)=C(c)\cup C(y)$. Por tanto, $C(x)=C(y)$.

    Es claro que:
    \begin{equation*}
        X=\bigcup_{x\in X}\{x\} \subset \bigcup_{x\in X}C(x)\subset X \Longrightarrow X=\bigcup_{x\in X}C(x)
    \end{equation*}
    Por tanto, las componentes conexas dan una partición de $X$.

    \item Un espacio topológico $(X,\T)$ es conexo si y solo si tiene una única componente conexa.

    Consecuencia directa de la propiedad anterior.

    \item Cada componente conexa de $(X,\T)$ es un cerrado de $(X,\T)$.

    Si $C(x)$ es la componente conexa de $x\in X$, sabemos que $\ol{C(x)}$ es un conexo. Pero:
    \begin{equation*}
        x\in C(x)\subset \ol{C(x)} \stackrel{(\ast)}{\subset} C(x)
    \end{equation*}
    donde en $(\ast)$ hemos aplicado que $C(x)$ es el mayor conexo que contiene a $x$. Entonces, $C(x)=\ol{C(x)}$, por lo que $C(x)$ es cerrado.

    \item Si ${\{A_i\}}_{i\in I}$ es una partición de $(X,\T)$ donde cada $A_i\in \T$ y es conexo, entonces la familia ${\{A_i\}}_{i\in I}$ es el conjunto de todas las componentes conexas de $(X,\T)$.
\end{enumerate}


\begin{ejemplo}
    Algunas consecuencias directas de dichas propiedades son:
    \begin{enumerate}
        \item $X=\bb{R}\setminus \left\{\dfrac{1}{n}\mid n\in \bb{N}\right\}$.

        Tenemos que sus componentes conexas son:
        \begin{equation*}
            ]-\infty, 0] \qquad ]1,\infty[ \qquad \left]\dfrac{1}{n+1}, \dfrac{1}{n}\right[~~n\in \bb{N}
        \end{equation*}
    \end{enumerate}
\end{ejemplo}



\begin{teo}
    El número de componentes conexas se conserva por homeomorfismos.
\end{teo}