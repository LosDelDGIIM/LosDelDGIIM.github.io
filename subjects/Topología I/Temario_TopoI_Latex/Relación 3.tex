\section{Conexión y Compacidad}\label{sec:Rel3}

\begin{ejercicio}
Demuestra que \( \bb{S}^1 \) no es homeomorfo a ningún subconjunto de \( \mathbb{R} \) ambos considerados con la topología usual.
\end{ejercicio}

\begin{ejercicio}
Demuestra que \( \mathbb{R}^{n+1} \setminus \bb{S}^n \) no es conexo.
\end{ejercicio}

\begin{ejercicio}
Sea \((X, \mathcal{T})\) un espacio topológico. Demuestra que son equivalentes:
\begin{enumerate}
\item \((X, \mathcal{T})\) es conexo.
\item Para todo \(A \subseteq X\) tal que \(\partial A = \emptyset\) se tiene \(A = X\) o \(A = \emptyset\).
\end{enumerate}
\end{ejercicio}

\begin{ejercicio}
Sean \(A\) y \(B\) subconjuntos conexos de un espacio topológico \((X, \mathcal{T})\) tales que \(\ol{A} \cap B = \emptyset\). Demuestra que \(A \cup B\) es conexo.


Como $\ol{A}\cap B\neq \emptyset$, entonces $\exists b_0\in B\cap \ol{A}$. $A\cup \{b_0\}$ es conexo, porque $A$ es conexo y $A\subset A\cup \{b_0\}\subset \ol{A}$.

Tenemos que $A\cup B = \left(A\cup \{b_0\}\right)\cup B$, siendo ambos conexos y $b_0\in \left(A\cup \{b_0\}\right)\cap B\neq \emptyset$. Por tanto, $A\cup B$ es conexo. 
\end{ejercicio}

\begin{ejercicio}
Sean \(A\) y \(B\) subconjuntos cerrados y no vacíos de un espacio \((X, \mathcal{T})\) tales que \(A \cup B\) y \(A \cap B\) son conexos. Demuestra que \(A\) y \(B\) son conexos.
\end{ejercicio}

\begin{ejercicio}
Sean \(Y_1\) y \(Y_2\) subespacios de \((X, \mathcal{T})\) y sea \(A \subseteq Y_1 \cap Y_2\). Demuestra que si \(A\) es abierto (respectivamente cerrado) en \(Y_1\) y en \(Y_2\), entonces \(A\) es abierto (respectivamente cerrado) en \(Y_1 \cap Y_2\) y en \(Y_1 \cup Y_2\).
\end{ejercicio}

\begin{ejercicio}
Sean \((X, \mathcal{T})\) un espacio conexo y \(A\) un subconjunto conexo y no vacío de \(X\). Sea \(U\) un abierto y cerrado en \(X \setminus A\). Demuestra que \(A \cup U\) es conexo.



El más difícil de conexión es el 7, que se basa en el 6.

Para el 8 hace falta el 7.
\end{ejercicio}

\begin{ejercicio}
Sean \((X, \mathcal{T})\) un espacio conexo y \(A\) un subconjunto conexo y no vacío de \(X\). Sea \(C\) una componente conexa de \(X \setminus A\). Demuestra que \(X \setminus C\) es conexo.
\end{ejercicio}

\begin{ejercicio}
Prueba que el interior y la frontera de un subconjunto conexo no son en general conexos.
\end{ejercicio}

\begin{ejercicio}
Sean \((X, \mathcal{T})\) y \((X', \mathcal{T'})\) dos espacios topológicos conexos y \(A \subseteq X\), \(A\neq X\), \(B \subseteq X'\), \(B\neq X'\). ¿Es \(X \times X' \setminus A \times B\) conexo?
\end{ejercicio}

\begin{ejercicio}
Sea \(X\) el conjunto de los puntos de \( \mathbb{R}^2 \) con alguna coordenada racional. Prueba que \(X\) con la topología inducida es un subconjunto conexo.
\end{ejercicio}

\begin{ejercicio}
Prueba que no existen aplicaciones continuas e inyectivas de \( \mathbb{R}^2 \) en \( \mathbb{R} \).
\end{ejercicio}

\begin{ejercicio}
Sea \((X, \mathcal{T})\) un espacio topológico y \(\{A_i\}_{i \in I}\) una partición por subconjuntos conexos y abiertos de \((X, \mathcal{T})\). Entonces \(\{A_i\}_{i \in I}\) es la familia de componentes conexas de \(X\).
\end{ejercicio}



\begin{ejercicio}
Sea \( X = \{a, b, c, d, e\} \) y \( \mathcal{T} = \{X, \emptyset, \{a\}, \{c, d\}, \{a, c, d\}, \{b, c, d, e\}\} \). Calcula las componentes conexas de \((X, \mathcal{T})\).
\end{ejercicio}

\begin{ejercicio}
Denotemos por \( C((a, b), r) \) la circunferencia en \( \mathbb{R}^2 \) con centro \((a, b)\) y radio \(r\). Demuestra que ninguno de los siguientes espacios topológicos es homeomorfo a cualquier otro:
\begin{enumerate}
    \item \( X_1 = C((0, 0), 1) \)
    \item \( X_2 = C((-1, 0), 1) \cup C((1, 0), 1) \)
    \item \( X_3 = C((-1, 0), 1) \cup C((1, 0), 1) \cup C((0, \sqrt{3}), 1) \)
    \item \( X_4 = C((-1, 0), 1) \cup C((1, 0), 1) \cup (\mathbb{R} \times \{1\}) \).
\end{enumerate}
\end{ejercicio}

\begin{ejercicio}
Decide cuáles de los siguientes subespacios de \( \mathbb{R} \) y \( \mathbb{R}^2 \) son compactos. Razona la respuesta:
\begin{enumerate}
    \item \( [0, \infty[ \)
    \item \( [0, 1] \cap \mathbb{Q} \)
    \item \( \{(x, y) \in \mathbb{R}^2 \mid x \geq 1, 0 \leq y \leq \frac{1}{x}\} \)
    \item \( \{(x, y) \in \mathbb{R}^2 \mid |x| + |y| \leq 1\} \)
    \item \( \bb{S}^1 \setminus \left\{\left(\frac{\sqrt{2}}{2}, -\frac{\sqrt{2}}{2}\right)\right\} \)
    \item \( (]0, 1[, \mathcal{T}) \) donde \( \mathcal{T} = \{\emptyset, X\} \cup \{]0, 1 - \frac{1}{n}[~\mid n \geq 2\} \)
    \item \( (\mathbb{R}, \mathcal{T}) \) donde \( \mathcal{T} = \{O \subseteq \mathbb{R} \mid O = U \setminus B, U \in \mathcal{T}_u, B \subseteq \left\{\frac{1}{n} \mid n \in \mathbb{N}\right\}\} \)
    \item La recta de Sorgenfrey.
    \item \( (\mathbb{Q}, {\mathcal{T}_u}_{\big |\mathbb{Q}}) \)
    \item \( X \) es un conjunto, \( p \in X \) un punto fijo y \( \mathcal{T} = \{O \subseteq X \mid p \in O\} \cup \{\emptyset\} \).
    \item \( X \) es un conjunto, \( p \in X \) un punto fijo y \( \mathcal{T} = \{O \subseteq X \mid p \in O\} \cup \{X\} \).
    \item \( \mathbb{N} \) con la topología \( \mathcal{T} = \{\emptyset, \mathbb{N}\} \cup \{\{1, \ldots, n\} \mid n \in \mathbb{N}\} \).
\end{enumerate}
\end{ejercicio}

\begin{ejercicio}
Sea \((X, \mathcal{T})\) un espacio topológico. Prueba que si \( A, A' \subseteq X \) son subespacios compactos, entonces \( A \cup A' \) también es compacto.
\end{ejercicio}


\begin{ejercicio}
Sea \((X, \mathcal{T})\) un espacio topológico Hausdorff. Prueba que si \(\{A_i\}_{i \in I}\) son subespacios compactos de \(X\), entonces \(\bigcap\limits_{i \in I} A_i\) también es compacto.
\end{ejercicio}

\begin{ejercicio}
Sea \((X, \mathcal{T})\) un espacio topológico compacto y supongamos que para cada \(n \in \mathbb{N}\), \(C_n\) es un subconjunto cerrado y no vacío tal que \(C_{n+1} \subseteq C_n\). Prueba que \(\bigcap\limits_{n=1}^{\infty} C_n \neq \emptyset\).
\end{ejercicio}

\begin{ejercicio}
Sea \((X, \mathcal{T})\) un espacio Hausdorff y compacto y \(f: X \to X\) una aplicación continua. Prueba que existe \(A \subseteq X\) un subconjunto cerrado y no vacío tal que \(f(A) = A\).
\end{ejercicio}

\begin{ejercicio}
Demuestra el siguiente resultado conocido como el \emph{teorema de la aplicación contractiva}: si \(X\) es un espacio métrico compacto y \(f: X \to X\) es una aplicación contractiva (es decir, existe \(K < 1\) tal que \(d(f(x), f(y)) \leq K \cdot d(x, y)\) para todo \(x, y \in X\)), entonces existe un único punto \(x \in X\) tal que \(f(x) = x\).
\end{ejercicio}

\begin{ejercicio}
Sea \((X, \mathcal{T})\) un espacio topológico Hausdorff y \(\{K_n\}_{n \in \mathbb{N}}\) una sucesión estrictamente decreciente de compactos no vacíos en \(X\). Demuestra que \(K = \bigcap\limits_{n \in \mathbb{N}} K_n \neq \emptyset\) y que si \(U \in \mathcal{T}\) tal que \(K \subseteq U\), entonces existe \(m \in \mathbb{N}\) tal que \(K_m \subseteq U\).
\end{ejercicio}

\begin{ejercicio}
Encuentra contraejemplos de las afirmaciones siguientes:
\begin{enumerate}
\item En cualquier espacio métrico toda bola cerrada es un subespacio compacto.
\item En cualquier espacio métrico ninguna bola abierta puede ser un subespacio compacto.
\end{enumerate}
\end{ejercicio}

\begin{ejercicio}
Demuestra que no existe ninguna función continua \[f: ([0, 1], \mathcal{T}_u|[0,1]) \to (\mathbb{R}, \mathcal{T}_u)\] que verifique \(f(x) \in \mathbb{R} \setminus \mathbb{Q}\) si \(x \in \mathbb{Q}\) y \(f(x) \in \mathbb{Q}\) si \(x \in \mathbb{R} \setminus \mathbb{Q}\).
\end{ejercicio}

\begin{ejercicio}
Sean \((X, \mathcal{T})\) e \((Y, \mathcal{T'})\) espacios topológicos con \((X, \mathcal{T})\) compacto. Prueba que la proyección \(\pi_Y: X \times Y \to Y\) es cerrada.
\end{ejercicio}

\begin{ejercicio}
Se dice que \(C \subseteq \mathbb{R}^2\) es una curva de Jordan si \(C = f([0, 1])\) para \(f: [0, 1] \to \mathbb{R}^2\) una aplicación continua e inyectiva. Prueba que no existe una curva de Jordan que rellene el cuadrado \([0, 1] \times [0, 1]\).
\end{ejercicio}


\begin{ejercicio}
    Prueba que los subconjuntos compactos de la recta de Sorgenfrey son necesariamente
numerables.
\end{ejercicio}