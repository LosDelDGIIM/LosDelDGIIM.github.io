\section{Conexión y Compacidad}

\begin{ejercicio}
    Ejercicio 4.

    Como $\ol{A}\cap B\neq \emptyset$, entonces $\exists b_0\in B\cap \ol{A}$. $A\cup \{b_0\}$ es conexo, porque $A$ es conexo y $A\subset A\cup \{b_0\}\subset \ol{A}$.

    Tenemos que $A\cup B = \left(A\cup \{b_0\}\right)\cup B$, siendo ambos conexos y $b_0\in \left(A\cup \{b_0\}\right)\cap B\neq \emptyset$. Por tanto, $A\cup B$ es conexo. 
\end{ejercicio}



El más difícil de conexión es el 7, que se basa en el 6.

Para el 8 hace falta el 7.

\begin{ejercicio}
    
\end{ejercicio}