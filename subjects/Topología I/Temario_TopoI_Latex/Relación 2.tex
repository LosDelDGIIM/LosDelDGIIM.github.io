\section{Aplicaciones entre Espacios Topológicos}

\begin{ejercicio}
    Sean $(X, d)$ e $(Y, d')$ espacios métricos. Diremos que una aplicación $f : (X, d) \to (Y, d')$ es lipschitziana si existe $K > 0$ tal que $d'(f(x), f(y)) \leq Kd(x, y)$ para todo $x, y \in X$. Prueba que toda aplicación lipschitziana es una aplicación continua.
\end{ejercicio}

\begin{ejercicio}
    Sean $(X, d)$ un espacio métrico y $A \subset X$. Demuestra que la aplicación $f : (X, \T_d) \to (\bb{R},\T_u)$ dada por $f(x) = d(x, A) = \inf\{d(x, a) \mid a \in A\}$ es continua.
\end{ejercicio}

\begin{ejercicio}
    Sean $(X, \T)$ un espacio topológico y $f, g:(X, \T) \to (\bb{R},\T_u)$ aplicaciones continuas. Demostrar que las siguientes aplicaciones son continuas:
    \Func{f+g}{(X,\T)}{(\bb{R},\T_u)}{x}{f(x)+g(x)}
    \Func{f\cdot g}{(X,\T)}{(\bb{R},\T_u)}{x}{f(x)\cdot g(x)}

    Veámoslo en primer lugar para el caso de la suma. Escribiremos $f+g$ como una composición. Sean las siguientes aplicaciones entre espacios topológicos:
    \Func{F}{(X,\T)}{\bb{R}^2,\T_u}{x}{(f(x,g(x)))}
    \Func{s}{(\bb{R}^2,\T_u)}{\bb{R},\T_u}{(x,y)}{x+y}

    Tenemos que:
    \begin{equation*}
        (f+g)(x):=f(x)+g(x)=s(f(x),g(x))=s(F(x))=(s\circ F)(x) \qquad \forall  x\in X
    \end{equation*}
    Por tanto, se tiene que $f+g=s\circ F$. Sabemos que $s$ es continua, veamos ahora que $F$ lo es. En $(\bb{R}^2,\T_u)$, tenemos que $\{]a,b[\times ]c,d[~ \mid a<b,~ c<d\}$ es una base. Entonces:
    \begin{equation*}
        F^{-1}(]a,b[\times ]c,d[)
        = \left\{
        x\in X\mid F(x)\in~  ]a,b[\times]c,d[
        \right\}
        = \left\{
        x\in X\mid a<f(x)<b \quad \land \quad c<g(x)<d
        \right\}
        = f^{-1}]a,b[~ \cap g^{-1}]c,d[\in \T
    \end{equation*}
    donde tenemos que es un abierto por ser la intersección de dos abiertos, y estos dos lo son por ser $f,g$ continuas.
\end{ejercicio}


\begin{ejercicio}
    Sea $f : (X, \T ) \to (Y, \T')$ una aplicación. Demuestra que equivalen:
\begin{enumerate}
    \item $f$ es continua.
    \item $f^{-1}(B^\circ) \subset [f^{-1}(B)]^\circ, \forall B \subset Y$.
    \item $\partial (f^{-1}(B)) \subset f^{-1}(\partial B), \forall B \subset Y$.
\end{enumerate}
\end{ejercicio}

\begin{ejercicio}
    Sean $(X, \T ), (Y, \T')$, dos espacios topológicos, $f : (X, \T ) \to (Y, \T')$ una aplicación continua y sobreyectiva. Demuestra que si $D \subset X$ es un subconjunto denso, entonces $f(D)$ es denso en $Y$. Demuestra, mediante un contraejemplo, que si $f(D)$ es denso, $D$ no tiene por qué serlo.

    Demostremos que $f(D)$ es denso. Para ello, demostramos que $\ol{f(D)}=Y$.
    \begin{description}
        \item[$\subset)$] Tenemos que $f(D)\subset Y$. Por tanto, $\ol{f(D)}\subset \ol{Y}=Y$, ya que $Y\in C_{\T'}$.

        \item[$\supset)$] Por la caracterización de continuidad, por ser $f$ continua tenemos que $f(\ol{D})\subset \ol{f(D)}$. Como $X$ es denso ($X=\ol{D}$), tenemos que $f(X)=f(\ol{D})$. Además, tenemos por ser $f$ sobreyectiva tenemos que $f(X)=Y$. Por tanto,
        \begin{equation*}
            Y=f(X)=f(\ol{D})\subset \ol{f(D)}
        \end{equation*}
    \end{description}


    Para ver que el recíproco no es cierto, sea $f:(X,\T_{disc})\to (Y, \T_t)$, considerando $X=\{0,1\}$ e $Y=\{y_0\}$, la aplicación constante en $y_0$. Tenemos que $f$ es continua y sobreyectiva. No obstante,

    TERMINAR
\end{ejercicio}

\begin{ejercicio}
    Sean $(X, \T ), (Y, \T')$, dos espacios topológicos y $f : (X, \T ) \to (Y, \T')$ una aplicación.
    \begin{enumerate}
        \item Demuestra que si $f$ es continua y $\{x_n\}_{n\in \bb{N}}$ es una sucesión en $X$ que converge a $x_0$ entonces $\{f(x_n)\}_{n\in \bb{N}}$ es una sucesión en $Y$ que converge a $f(x_0)$.
    
        \item \label{ej:3.2.6.2}  Demuestra que si $(X, \T )$ es un espacio topológico 1AN tal que para toda sucesión
        $\{x_n\}_{n\in \bb{N}}$ que converge a $x_0$ se tiene que $\{f(x_n)\}_{n\in \bb{N}}$ es una sucesión que converge a $f(x_0)$, entonces $f$ es continua.
    
        \item Demuestra que \ref{ej:3.2.6.2}) no es cierto en general si se elimina la condición 1AN.
    \end{enumerate}
\end{ejercicio}

\begin{ejercicio}
    Se considera en $\bb{N}$ la topología $\T$ del ejercicio \ref{ej:3.1.10} de la Relación 1. Caracteriza las aplicaciones continuas de $(\bb{N},\T)$ en sí mismo.
\end{ejercicio}

\begin{ejercicio}
    Sean $(X, \T ), (Y, \T')$, dos espacios topológicos, $f : (X, \T ) \to (Y, \T')$ una aplicación. Si $A \subset X$, entonces $f_{\big| A}$ puede ser continua sin que $f$ sea continua en los puntos de $A$.


    Funcion característica de Q.
\end{ejercicio}

\begin{ejercicio}
    Sean $(X, \T ), (Y, \T')$, dos espacios topológicos y $f : (X, \T ) \to (Y, \T')$ una aplicación.
    Demuestra que $f$ es continua en $x_0$ si y solo si existe $U \in \T$ con $x_0 \in U$ tal que $f_{\big|U} : (U, \T_U ) \to (Y, \T')$ es continua en $x_0$. ¿Es cierta la equivalencia anterior si sustituimos $U$ abierto conteniendo a $x0$ por $C$ cerrado conteniendo a $x_0$?
\end{ejercicio}

\begin{ejercicio}
    Demuestra que una aplicación $f : (X, \T_{x_0}) \to (Y, \T_{x_0})$ es continua si y solo si es constante o $f(x_0) = y_0$. Deduce que $(X, \T_{x_0}) \cong (X, \T_{x_1})$ para todo par de puntos $x_0, x_1 \in X$.
\end{ejercicio}

\begin{ejercicio}
    Demuestra que todo subespacio afín $S \subset \bb{R}^n$ es un cerrado de $(\bb{R}^n, \T_u)$.
    
    Si $S$ es un subespacio afín de $\bb{R}^n$, entonces $S$ viene dado por unas ecuaciones implícitas en las coordenadas usuales de $\bb{R}^n$:
    \begin{equation*}
        \left\{
            \begin{array}{ccc}
                 a_{11}x_1 + \dots + a_{1n}x_n &=& b_1 \\
                 a_{r1}x_1 + \dots + a_{\bb{R}^n}x_n &=& b_r
            \end{array}
        \right.
    \end{equation*}

    Definimos las siguientes aplicaciones para $i=1,\dots,r$:
    \Func{f_i}{\qquad (\bb{R}^n, \T_u)}{(\bb{R},\T_u)}{(x_i,\dots,x_n)}{a_{i1}x_1+\dots+a_{1n}x_n}

    Tenemos que $S=\bigcap\limits_{i=1,\dots,r}f^{-1}(b_r)$ es cerrado en $(\bb{R}^n,\T_u)$, ya que cada funcion es continua. TERMINAR
\end{ejercicio}

\begin{ejercicio}
    Consideremos el espacio $(X, \T )$ donde $X = \{a, b, c, d\}$ y
    \begin{equation*}
        \T = \{\emptyset, X, \{a\}, \{b\}, \{a, b\}, \{b, c, d\}\}.
    \end{equation*}
    Sea $f : (X, \T ) \to (X, \T )$ la aplicación dada por $f(a) = b$, $f(b) = d$, $f(c) = b$, $f(d) = c$.
    Estudia en qué puntos la aplicación $f$ es continua. ¿Es $f$ abierta o cerrada?

    Veamos en primer lugar que no es continua. Dado $\{b\}\in \T$, tenemos que $f^{-1}\{b\}=\{a,c\}\notin \T$. Por tanto, como hemos encontrado un abierto cuya preimagen no es un abierto, tenemos que $f$ no es continua.

    Veamos ahora que $f$ no es abierta. Dado $\{b\}\in \T$, tenemos que $f\{b\}=\{d\}\notin \T$. Por tanto, como hemos encontrado un abierto cuya imagen no es un abierto, tenemos que $f$ no es abierta.

    Tenemos que los cerrados son:
    \begin{equation*}
        C_\T=\{\emptyset, X, \{b,c,d\}, \{a,c,d\}, \{c,d\}, \{a\}\}
    \end{equation*}
    Veamos que $f$ no es cerrada. Dado $\{a\}\in C_\T$, tenemos que $f\{a\}=\{b\}\notin C_\T$. Por tanto, como hemos encontrado un cerrado cuya imagen no es un cerrado, tenemos que $f$ no es cerrada.
\end{ejercicio}


\begin{ejercicio}
    Se considera $f : (\bb{R},\T) \to (\bb{R},\T)$ dada por $f(x) = \sen(x)$, siendo $(\bb{R},\T)$ la recta diseminada (Ejercicio \ref{ej:3.1.16} de la Relación 1). Estudia si $f$ es continua, abierta o cerrada.

    Veamos si $f$ es continua. Comamos como abierto en $\T$ el conjunto $W=\{\sen 1\}\subset \bb{R}\setminus \bb{Q}$. Tenemos que:
    \begin{equation*}
        f^{-1}(W)=\{1+2\pi k\mid k\in \bb{Z}\} \cup \{(\pi-1)+2\pi k\mid k\in \bb{Z}\}
    \end{equation*}
    Tenemos que $f^{-1}(W)\in \T$, ya que si fuese un abierto, entonces $1\in f^{-1}(W)=U\cup V$, con $U\in \T_u$, $V\subset \bb{R}\setminus \bb{Q}$. Como $1\in \bb{Q}$, tenemos que ha de ser que $1\in U\in \T_u$. Por la definición de la topología usual, $\exists \veps\in \bb{R}^+\mid 1\in B(1,\veps)\subset U\subset f^{-1}(W)$. No obstante, esto no es posible, ya que $f^{-1}$ es discreto. TERMINAR

    Veamos que no es abierta. Tenemos que $W=\left\{\dfrac{\pi}{2}\right\}\in \T$, pero $f(W)=\{1\}\notin \T$.

    Veamos que no es cerrada. Para ello, vemos en primer lugar cuáles son los cerrados. TERMINAR
    \begin{equation*}
        C_\T=\{C\cap F\mid C\in \T_u, \bb{Q}\subset F\}
    \end{equation*}
\end{ejercicio}

\begin{ejercicio}
    Sea $\chi_{\left[0,\frac{1}{2}\right]}: ([0, 1],(\T_u)_{[0,1]}) \to (\{0, 1\}, \T_{disc})$ la función característica del intervalo $\left[0,\frac{1}{2}\right]$.
    Demuestra que $\chi_{\left[0,\frac{1}{2}\right]}$ es sobreyectiva, abierta, cerrada, pero no es continua.
\end{ejercicio}


\begin{ejercicio}
    Demuestra que las proyecciones $p_i: (\bb{R}^n, \T_u) \to (\bb{R}, \T_u)$ dadas por $p_i(x_1, \dots , x_n) = x_i$, $i = 1, \dots, n$, son aplicaciones abiertas pero no cerradas.
\end{ejercicio}

\begin{ejercicio}
    Demuestra que la aplicación $f : (\bb{R}^n, \T_u) \to ([0, +\infty[,(\T_u)_{[0,+\infty[})$ dada por $f(x) = \|x\|$ es abierta, y que $g : (\bb{R}^n, \T_u) \to (\bb{R}, \T_u)$ dada por $g(x) = \|x\|$ no lo es.
\end{ejercicio}

\begin{ejercicio}
    Sea $A \subset \bb{R}^n$ con $A \in C_{\T_u}$ y $f : \left(A,{(\T_u)}_{\big|A}\right) \to (\bb{R}^m, \T_u)$ continua, y tal que $f^{-1}(B)$ es acotado en $\bb{R}^n$ para cada $B \subset \bb{R}^m$ acotado. Demuestra que entonces $f$ es cerrada.
    Deduce que la función $g$ del ejercicio anterior y las funciones polinómicas $p : (\bb{R}, \T_u) \to (\bb{R}, \T_u)$ son cerradas.
\end{ejercicio}

\begin{ejercicio}
    Demuestra que toda aplicación afín biyectiva en $\bb{R}^n$ de la forma $f : (\bb{R}^n, \T_u) \to (\bb{R}^n, \T_u)$ es un homeomorfismo.\\

    Por ser una aplicación afín, tenemos que $f$ es:
    \Func{f}{(\bb{R}^n,\T_u)}{(\bb{R}^n,\T_u)}{\left(\begin{array}{c}
        x_1\\ \vdots \\ x_n
    \end{array}\right)}{A\left(\begin{array}{c}
        x_1\\ \vdots \\ x_n
    \end{array}\right) + b}
    con $A\in \cc{M}_{n\times n}(\bb{R})$, $b\in \bb{R}^n$ fijo. Como $f$ es biyectiva, tenemos que $|A|\neq 0$.

    Es directo ver que $f$ es continua, ya que... TERMINAR. Además, su inversa (que también es una aplicación afín) es también continua. Por tanto, $f$ es un homeomorfismo.\\
    
    
    Utiliza este resultado para construir un homeomorfismo:
    \begin{enumerate}
        \item Entre cualesquiera bolas abiertas, cualesquiera bolas cerradas y cualesquiera esferas de $(\bb{R}^n, d_u)$.

        Sean $x,x'\in \bb{R}^n$ fijos, y sean $\veps,\veps'\in \bb{R}^+$.
        
        Sea la aplicación lineal buscada una traslación según el vector $v=\vec{x'x}$ compuesto con una homotecia de centro $x'$ y razón $\frac{\veps'}{\veps}$. Es decir:
        \Func{f}{(\bb{R}^n,\T_u)}{(\bb{R}^n,\T_u)}{\left(\begin{array}{c}
        x_1\\ \vdots \\ x_n
        \end{array}\right)}{A\left(\begin{array}{c}
            x_1\\ \vdots \\ x_n
        \end{array}\right) + b= \frac{\veps'}{\veps}\left(\begin{array}{c}
            x_1\\ \vdots \\ x_n
        \end{array}\right) + b}

        Calculamos $b$:
        \begin{equation*}
            \frac{\veps'}{\veps} x + b=x'\Longrightarrow b=c'-\frac{\veps'}{\veps}x
        \end{equation*}

        Veamos ahora que $f[B(x,\veps)]=B(x',\veps')$.

        TERMINAR

        

        


        
        \item El cilindro circular $\bb{S}^1 \times \bb{R}$ y el cilindro elíptico $C = \{(x, y, z) \in \bb{R}^3 \mid x^2 +\frac{y^2}{4} = 1\}$ de $\bb{R}^3$.

        Tomando como sistema de referencia el usual, tenemos que la aplicación afín ha de de cumplir que $f(O)=O$, $f(e_1)=e_1$, $f(e_2)=2e_2$, $f(e_3)=e_3$. En definitiva, tenemos que $f(x,y,z)=(x,2y,z)$.

        Veamos ahora que $f(\bb{S}^1\times \bb{R})=C$.
        \begin{description}
            \item[$\subset)$]
            Sea $(x,y,z)\in \bb{S}^1\times \bb{R}$, y veamos si $f(x,y,z)\in C$. Como $(x,y,z)\in \bb{S}^1\times \bb{R}$, tenemos que $x^2+y^2=1$. Además, $f(x,y,z)=(x,2y,z)$. Por tanto, como se tiene que:
            \begin{equation*}
                x^2 + \frac{(2y)^2}{4}=x^2+y^2 = 1
            \end{equation*}
            Por tanto, tenemos que $(x,2y,z)\in C$.

            \item[$\supset)$] Sea $(u,v,w)\in C$, y veamos si $\exists (x,y,z)\in \bb{S}^1\times \bb{R}$ tal que $f(x,y,z)=(u,v,w)$.

            Como $(u,v,w)\in C$, tenemos que $u^2 + \frac{v^2}{4}=1$. Consideramos $(x,y,z)=\left(u,\dfrac{v}{2}, w\right)$. Tenemos claramente que $f\left(u,\dfrac{v}{2}, w\right)=(u,v,w)$. Veamos ahora que $\left(u,\dfrac{v}{2}, w\right)\in \bb{S}^1\times \bb{R}$.
            $$u^2 + \left(\frac{v}{2}\right)^2=u^2 + \frac{v^2}{4}=1$$
        \end{description}
    \end{enumerate}
\end{ejercicio}

\begin{ejercicio}
    Sea $X$ un conjunto. Demuestra que toda aplicación biyectiva $f : (X, \T_{CF}) \to (X, \T_{CF})$ es un homeomorfismo.
\end{ejercicio}

\begin{ejercicio}
    Encuentra un contraejemplo que demuestre que la siguiente afirmación es falsa: Si existen aplicaciones continuas e inyectivas $f : (X,\T) \to (Y, \T')$ y $g : (Y, \T') \to (X,\T)$ entonces $(X,\T)$ e $(Y, \T')$ son homeomorfos.
\end{ejercicio}

\begin{ejercicio}
    Demuestra que toda aplicación $f : (\bb{R}, \T_u) \to (\bb{R}, \T_u)$ estrictamente creciente (decreciente) y continua es un embebimiento.
\end{ejercicio}


\begin{ejercicio}
    Sea $A \subset \bb{R}^n$, $A\neq \emptyset$ y $f:(A,{(\T_u)}_A) \to (\bb{R}, \T_u)$ una función continua. Se define el \textbf{grafo} de $f$ como el como el subconjunto de $\bb{R}^{n+1}$ dado por:
    \begin{equation*}
        G(f) = \{(x, f(x)) \mid x \in A\} .
    \end{equation*}

    Demuestra que:
    \begin{enumerate}
        \item  ${(A,(\T_u)}_A)$ es homeomorfo a $(G(f),{(\T_u)}_{G(f)})$.

        Hay que probar que la siguiente aplicación es un homeomorfismo:
        \Func{H}{A}{G(f)}{x}{(x,f(x))}

        Tenemos que $H$ es continua, ya que cada una de sus coordenadas lo es (estamos en $\T_u$). Además, tenemos que $H$ es biyectiva. Su inversa es:
        \Func{H^{-1}}{G(f)}{A}{(x,y)}{x}

        Tenemos que $H^{-1}=F_{\big| G(f)}$, donde $F$ es la siguiente aplicación:
        \Func{F}{\bb{R}^{n+1}}{\bb{R}^n}{(x_0,\dots,x_n,x_{n+1})}{(x_0,\dots,x_n)}
        que es continua. Por tanto, $H$ es un homeomorfismo.
        
        \item Notamos $\bb{S}^n=S(0,n)$. La bola cerrada $\ol{B}(0, 1) \subset \bb{R}^n$ es homeomorfa al conjunto $\bb{S}^+ = \{(x, t) \in \bb{S}^n \mid t \geq 0\} \subset\bb{R}^{n+1}$.
        \item Las cuádricas $C_1,C_2,C_3$ son homeomorfas a $\bb{R}^2$:
        \begin{equation*}\begin{split}
            C_1 &= \{(x, y, z) \in \bb{R}^3 \mid z = x^2+y^2\}\\
            C_2 &= \{(x, y, z) \in \bb{R}^3 \mid z = x^2-y^2\}\\
            C_3 &= \{(x, y, z) \in \bb{R}^3 \mid x^2 + y^2 - z^2 = 0,~ z \geq 0\}
        \end{split}\end{equation*}
    \end{enumerate}
\end{ejercicio}

\begin{ejercicio}
    Demuestra que la siguiente aplicación es continua y biyectiva pero no es un homeomorfismo.
    \Func{f}{([0, 1[,{(\T_u)}_{[0,1[})}{(\bb{S}^1,{(\T_u)}_{\bb{S}^1})}{t}{(\cos(2\pi t),\sen(2\pi t))}
\end{ejercicio}

\begin{ejercicio}
    Sea $(\bb{R},\T_S)$ la recta de Sorgenfrey. Se define la aplicación $f : \bb{R} \to \bb{R}$ dada por
    \begin{equation*}
        f(x)=\left\{
            \begin{array}{ccc}
                e^x & \text{si} & x<0,\\
                3 & \text{si} & x\geq 0.
            \end{array}
        \right.
    \end{equation*}
    \begin{enumerate}
        \item Estudia la continuidad de $f : (\bb{R}, \T_u) \to (\bb{R}, \T_u)$, $f : (\bb{R},\T_S) \to (\bb{R},\T_S)$, $f :(\bb{R}, \T_u) \to (\bb{R},\T_S$) y $f : (\bb{R},\T_S) \to (\bb{R}, \T_u)$.
        \item Estudia si las aplicaciones anteriores son abiertas o cerradas.
    \end{enumerate}
\end{ejercicio}

\begin{ejercicio}
    Demuestra que “ser metrizable” es una propiedad topológica.
\end{ejercicio}

\begin{ejercicio}
    Sean $(X,\T)$ e $(Y, \T')$ espacios topológicos. Demuestra que si $(X,\T)$ e $(Y, \T')$ son dos espacios topológicos metrizables si y solo si $(X \times Y, \T \times \T')$ es un espacio topológico metrizable.
\end{ejercicio}

\begin{ejercicio}
    Sean $X = \{a, b, c\}$, $Y = \{u, v\}$, $\T_X = \{\emptyset, X, \{a\}, \{b, c\}\}$, $\T_Y = \{\emptyset, Y, \{u\}\}$. Halla la topología producto $\T_X\times \T_Y$.
\end{ejercicio}

\begin{ejercicio}
    Encuentra tres espacios topológicos $(X,\T)$, $(Y, \T')$ y $(Z, \T'')$ tales que $(X \times Y, \T \times \T') \cong (X \times Z, \T \times \T'')$ pero $(Y, \T')\not \cong (Z, \T'')$.
\end{ejercicio}


\begin{ejercicio}
    Sean$ (X,\T)$ e $(Y, \T')$ espacios topológicos y sean $A \subset X$ y $B \subset Y$. Demuestra que:
    \begin{enumerate}
        \item $\operatorname{int}_{X\times Y}(A \times B) = \operatorname{int}_X(A) \times \operatorname{int}_Y (B)$.
        \item $\operatorname{cl}_{X\times Y} (A \times B) = \operatorname{cl}_X(A) \times \operatorname{cl}_Y (B)$.
        \item $\partial_{X\times Y} (A \times B) = (\operatorname{cl}_X(A) \times \partial_Y (B)) \cup (\partial_X(A) \times \operatorname{cl}_Y (B))$.
        \item $(\T \times \T')_{A\times B} = \T_A \times \T'_B$.
        \item $A \times B \in \T \times \T'$ si y solo si $A \in \T$ y $B \in \T'$.
        \item $A \times B \in C_{\T \times \T'}$ si y solo si $A \in C_\T$ y $B \in C_{\T'}$.
        \item $A \times B$ es denso en $X \times Y$ si y solo si $A$ es denso en $X$ y $B$ es denso en $Y$.
    \end{enumerate}
\end{ejercicio}

\begin{ejercicio}
    Sean $(X, \T_{CF})$ e $(Y, \T_{CF})$ espacios topológicos con la topología cofinita. Demuestra que $\T_{CF}\times \T_{CF}$ no tiene por qué ser la topología cofinita en $X \times Y$.

    En el caso de que $X,Y$ sean finitos, tenemos que ambas son la discreta, por lo que no nos sirve.

    Sea $X=Y=\bb{R}$. Tenemos que $\bb{R}^\ast\times \bb{R}^\ast \in (\bb{R}\times \bb{R}, \T_{CF}\times \T_{CF})$.

    No obstante, ese conjunto no es en $(\bb{R}^2, \T_{CF})$.
\end{ejercicio}


\begin{ejercicio}
    Sean $(X, \T_{x_0})$ e $(Y, \T_{y_0})$ espacios topológicos con la topología del punto incluido. Demuestra que $\T_{x_0} \times \T_{y_0}$ no tiene por qué ser la topología $\T_{(x_0,y_0)}$ en $X \times Y$.

    Consideramos $(\bb{R},\T_{0})$. Sea $U=\{(0,0), (1,1)\}$. Tenemos que $U\in (\bb{R}^2, \T_{0,0})$, pero $U\notin (\bb{R}\times \bb{R}, \T_0\times \T_0)$. Veámoslo.

    
\end{ejercicio}


\begin{ejercicio}
    En el espacio topológico producto $(\bb{R} \times \bb{R}, \T_ \times \T_S)$ calcula la clausura, el interior y la frontera del conjunto $A=[1, 2[ \times [1, 2[$. Estudia también si la aplicación $f : (\bb{R}^2, \T_u) \to(\bb{R}^2, \T_u \times \T_S)$ dada por $f(x, y) = (y, x)$ es continua.

    Tenemos que $\ol{A}=[1,2]\times [1,2[$. Además, $A^\circ = ]1,2[\times [1,2[$.

    Veamos ahora si $f$ es continua. Para ello, componemos o usamos un contraejemplo.

    Sea $]1,2[\times [1,2[$. Su preimagen es $[1,2[\times ]1,2[$.
    No es continua.
\end{ejercicio}


\begin{ejercicio}
    Sea $f : (X,\T) \to (Y, \T')$ una aplicación continua, abierta y sobreyectiva. Entonces, $(Y, \T')$ es $T2$ si y sólo si $\Delta f = \{(x, y) \in X \times X \mid f(x) = f(y)\}$ es un subconjunto cerrado de $(X \times X, \T \times \T )$. Deduce de aquí que un espacio topológico $(X,\T)$ es T2 si y sólo si el conjunto $\Delta = \{(x, x) \in X \times X \mid x \in X\}$ es cerrado en $(X \times X, \T \times \T )$.
\end{ejercicio}

\begin{ejercicio}
    Sean $(X,\T)$ e $(Y, \T')$ dos espacios topológicos con $(Y, \T')$ Hausdorff y sean $f, g :(X,\T) \to (Y, \T')$ aplicaciones continuas. Si existe un subconjunto $A \subset X$ tal que $f(x) = g(x)$ para todo $x \in A$ entonces $f(x) = g(x)$ para todo $x \in \ol{A}$. Demuestra que si $A$ es denso en $X$ entonces $f=g$.

    Demostramos por reducción al absurdo. Supongamos que $\exists x\in\ol{A}$ tal que $f(x)\neq g(x)\in Y$. Como $Y$ es T2, $\exists U,V\in \T$ tal que $f(x)\in U$, $g(x)\in V$, y $U\cap V=\emptyset$.

    Como $f,g$ son continuas, entonces $f^{-1}(U), g^{-1}(V)\in T$. Además, $x\in f^{-1}(U), g^{-1}(V)$. Entonces, $f^{-1}(U)\cap g^{-1}(V)\neq \emptyset$.

    Como $x\in \ol{A}$, entonces $f^{-1}(U)\cap A\neq \emptyset$, $g^{-1}(V)\cap A\neq \emptyset$. Por tanto, tenemos que $f^{-1}(U)\cap g^{-1}(V)\cap A \neq \emptyset$. Sea $x'\in f^{-1}(U)\cap g^{-1}(V)\cap A$. Como $x\in A$, entonces $f(x')=g(x')\in U\cap V$, por lo que llegamos a una contradicción.\\

    Veamos ahora que si $A$ es denso en $X$, entonces $f=g$. Como $A$ es denso, entonces $\ol{A}=X$. Por lo demostrado antes, $f(x)=g(x)$ para todo $x\in \ol{A}=X$, por lo que $f=g$.
\end{ejercicio}

\begin{ejercicio}
    Consideremos el espacio topológico $(\bb{R}^2, \T_{disc} \times \T )$, donde se tiene que $\T = \{\emptyset, \bb{R}, \bb{Q}, \bb{R}\setminus \bb{Q}\}$.
    \begin{enumerate}
        \item Encuentra una base de entornos, si es posible numerable, de cada punto de $\bb{R}^2$.

        Sea $(x,y)\in \bb{R}^2$, y busquemos $\beta_{(x,y)}$. Tenemos que:
        \begin{equation*}
            \beta_{(x,y)} = 
            \left\{
            \begin{array}{ccc}
                \{\{x\}\times \bb{Q}\} & \text{si} & y\in \bb{Q} \\
                \{\{x\}\times \bb{R}\setminus\bb{Q}\} & \text{si} & y\in \bb{R}\setminus\bb{Q} 
            \end{array}
            \right.
        \end{equation*}
        
        \item Encuentra un subconjunto no vacío $A \not \subset \bb{R}^2$ que sea abierto y cerrado a la vez.

        Por ejemplo, $A=\bb{Q}\times \bb{Q}$.
        
        \item Sea $L = \{(x, y) \in \bb{R}^2\mid y = x\}$. ¿Es cerrado $L$? ¿Cuál es la topología producto $(\T_{disc} \times \T )_L$?

        Veamos que $L$ no es un cerrado. Veamos si el complementario es abierto. Sea $(1,2)\in \bb{R}^2\setminus L$, y sea $U=\{1\}\times \bb{Q}$.  tenemos que $U\cap L\neq \emptyset$, ya que $(1,1)\in U\cap L$.

        Tenemos que $(\T_{disc}\times \T)_L = (\T_{disc})_L$. Esto se debe a que $\{(x,x)\}\in (\T_{disc}\times \T)_L$, ya que:
        \begin{equation*}
            \{(x,x)=(\{x\}\times \bb{R}\cap L)\}
        \end{equation*}
    \end{enumerate}
\end{ejercicio}


\begin{ejercicio}
    Sea $f : (\bb{R}, \T_u) \to (\bb{R}, \T_u)$ una aplicación continua que verifica la siguiente igualdad $f(x + y) = f(x)f(y),~\forall x, y \in \bb{R}$. Demuestra que $f\equiv 0$ o $f(x) = a^x$ para algún $a > 0$.
\end{ejercicio}

\begin{ejercicio}
    Sea $f : (X,\T) \to (Y, \T')$ una aplicación continua y sobreyectiva tal que para cada $y \in Y$ existe un entorno $N$ verificando que $f_{\big|f^{-1}(N)} : f^{-1}(N) \to N$ es una identificación. Demuestra que $f$ es una identificación.
\end{ejercicio}

\begin{ejercicio}
    Sean $(X,\T)$ e $(Y, \T')$ espacios topológicos. Demuestra que si $f : (X,\T) \to (Y, \T')$ es una aplicación continua, sobreyectiva y admite una inversa continua por la derecha (es decir, existe $g : (Y, \T') \to (X,\T)$ tal que $f \circ g = Id_Y$) entonces $f$ es una identificación.
\end{ejercicio}

\begin{ejercicio}
    En $\bb{R}^2$ consideramos la siguiente relación de equivalencia
    \begin{equation*}
        (x, y)R(x', y') \Longleftrightarrow x^2 + y = (x')^2 + y'
    \end{equation*}
    Demuestra que $(\bb{R}^2/R, \T_u/R)$ es homeomorfo a $(\bb{R}, \T_u)$.
\end{ejercicio}

\begin{ejercicio}
    Demuestra que la proyección $p : X \to X/A$ es una biyección continua de $X \setminus A$ en su imagen. Demuestra también que es un homeomorfismo si $A$ es abierto o cerrado.
\end{ejercicio}

\begin{ejercicio}
    Da un ejemplo de un espacio topológico $(X,\T)$ y un subconjunto $A \subset X$ ni abierto ni cerrado tales que $X \setminus A$ no sea homeomorfo a $X/A \setminus \{[A]\}$.
\end{ejercicio}

\begin{ejercicio}
    Da un ejemplo de un espacio topológico $(X,\T)$ y una relación de equivalencia $R$ en $X$ tal que $(X,\T)$ sea Hausdorff pero $(X/R, \T/R)$ no lo sea.
\end{ejercicio}

\begin{ejercicio}
    Da un ejemplo de un espacio topológico $(X,\T)$ y una relación de equivalencia $R$ en $X$ tal que $(X,\T)$ sea 2AN pero $(X/R, \T /R)$ no lo sea.
\end{ejercicio}

\begin{ejercicio}
    Sea $(X,\T)$ un espacio topológico e $I = [0, 1]$. Se denomina cono de $X$ al espacio topológico cociente
    \begin{equation*}
        \left(\frac{X\times I}{X\times \{0\}},\frac{\T,\T_u}{X\times \{0\}}\right)
    \end{equation*}
    Demuestra que que el cono de $(\bb{S}^n,(\T_u)\bb{S}^n)$ es homeomorfo a $(\ol{B^{n+1}},(\T_u)_{\ol{B^{n+1}}})$ para $n \geq 0$.
\end{ejercicio}

\begin{ejercicio}
    Sea $X = [0, 2]$ y $A = \{0, 1, 2\}$. Demuestra que $(X/A,(\T_u)_{X/A})$ es homeomorfo a $(C_1 \cup C_{-1},(\T_u)_{C_1 \cup C_{-1}})$, donde $C_1$ es la circunferencia de radio $1$ centrada en $(1, 0)$ y $C-1$ es la circunferencia de radio $1$ centrada en $(-1, 0)$.
\end{ejercicio}

\begin{ejercicio}
    ¿Qué espacio se obtiene si en una banda de Möbius se identifican todos los puntos de su borde?
\end{ejercicio}

\begin{ejercicio}
    Ver que $\bb{R}\bb{P}^2$ es homeomorfo al cociente $((I \times I)/R,(\T \times \T )/R)$ donde $R$ es la menor relación que contiene a $(t, 0)R(1 - t, 1)$ y $(0, s)R(1, 1 - s)$ y $\T$ es la topología usual de I.
\end{ejercicio}

\begin{ejercicio}
    Sea $X = \{(x, y) \in \bb{R}^2\mid 1 \leq \|(x, y)\| \leq 2\}$. Se define una relación de equivalencia en $X$ de la siguiente forma: $(x, y)R(x', y')$ si y sólo si $(x, y) = (x', y')$ o $\|(x, y)\| - \|(x', y')\| = \pm 1$ y $(x, y) = \lambda(x', y')$, $\lambda > 0$. Demuestra que el espacio cociente es homeomorfo al toro.
\end{ejercicio}