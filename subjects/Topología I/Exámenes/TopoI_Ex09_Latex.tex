\documentclass[12pt]{article}

% Idioma y codificación
\usepackage[spanish, es-tabla]{babel}       %es-tabla para que se titule "Tabla"
\usepackage[utf8]{inputenc}

% Márgenes
\usepackage[a4paper,top=3cm,bottom=2.5cm,left=3cm,right=3cm]{geometry}

% Comentarios de bloque
\usepackage{verbatim}

% Paquetes de links
\usepackage[hidelinks]{hyperref}    % Permite enlaces
\usepackage{url}                    % redirecciona a la web

% Más opciones para enumeraciones
\usepackage{enumitem}

% Personalizar la portada
\usepackage{titling}

% Paquetes de tablas
\usepackage{multirow}


%------------------------------------------------------------------------

%Paquetes de figuras
\usepackage{caption}
\usepackage{subcaption} % Figuras al lado de otras
\usepackage{float}      % Poner figuras en el sitio indicado H.


% Paquetes de imágenes
\usepackage{graphicx}       % Paquete para añadir imágenes
\usepackage{transparent}    % Para manejar la opacidad de las figuras

% Paquete para usar colores
\usepackage[dvipsnames]{xcolor}
\usepackage{pagecolor}      % Para cambiar el color de la página

% Habilita tamaños de fuente mayores
\usepackage{fix-cm}

% Para los gráficos
\usepackage{tikz}

% Para poder situar los nodos en los grafos
\usetikzlibrary{positioning}


%------------------------------------------------------------------------

% Paquetes de matemáticas
\usepackage{mathtools, amsfonts, amssymb, mathrsfs}
\usepackage[makeroom]{cancel}     % Simplificar tachando
\usepackage{polynom}    % Divisiones y Ruffini
\usepackage{units} % Para poner fracciones diagonales con \nicefrac

\usepackage{pgfplots}   %Representar funciones
\pgfplotsset{compat=1.18}  % Versión 1.18

\usepackage{tikz-cd}    % Para usar diagramas de composiciones
\usetikzlibrary{calc}   % Para usar cálculo de coordenadas en tikz

%Definición de teoremas, etc.
\usepackage{amsthm}
%\swapnumbers   % Intercambia la posición del texto y de la numeración

\theoremstyle{plain}

\makeatletter
\@ifclassloaded{article}{
  \newtheorem{teo}{Teorema}[section]
}{
  \newtheorem{teo}{Teorema}[chapter]  % Se resetea en cada chapter
}
\makeatother

\newtheorem{coro}{Corolario}[teo]           % Se resetea en cada teorema
\newtheorem{prop}[teo]{Proposición}         % Usa el mismo contador que teorema
\newtheorem{lema}[teo]{Lema}                % Usa el mismo contador que teorema

\theoremstyle{remark}
\newtheorem*{observacion}{Observación}

\theoremstyle{definition}

\makeatletter
\@ifclassloaded{article}{
  \newtheorem{definicion}{Definición} [section]     % Se resetea en cada chapter
}{
  \newtheorem{definicion}{Definición} [chapter]     % Se resetea en cada chapter
}
\makeatother

\newtheorem*{notacion}{Notación}
\newtheorem*{ejemplo}{Ejemplo}
\newtheorem*{ejercicio*}{Ejercicio}             % No numerado
\newtheorem{ejercicio}{Ejercicio} [section]     % Se resetea en cada section


% Modificar el formato de la numeración del teorema "ejercicio"
\renewcommand{\theejercicio}{%
  \ifnum\value{section}=0 % Si no se ha iniciado ninguna sección
    \arabic{ejercicio}% Solo mostrar el número de ejercicio
  \else
    \thesection.\arabic{ejercicio}% Mostrar número de sección y número de ejercicio
  \fi
}


% \renewcommand\qedsymbol{$\blacksquare$}         % Cambiar símbolo QED
%------------------------------------------------------------------------

% Paquetes para encabezados
\usepackage{fancyhdr}
\pagestyle{fancy}
\fancyhf{}

\newcommand{\helv}{ % Modificación tamaño de letra
\fontfamily{}\fontsize{12}{12}\selectfont}
\setlength{\headheight}{15pt} % Amplía el tamaño del índice


%\usepackage{lastpage}   % Referenciar última pag   \pageref{LastPage}
\fancyfoot[C]{\thepage}

%------------------------------------------------------------------------

% Conseguir que no ponga "Capítulo 1". Sino solo "1."
\makeatletter
\@ifclassloaded{book}{
  \renewcommand{\chaptermark}[1]{\markboth{\thechapter.\ #1}{}} % En el encabezado
    
  \renewcommand{\@makechapterhead}[1]{%
  \vspace*{50\p@}%
  {\parindent \z@ \raggedright \normalfont
    \ifnum \c@secnumdepth >\m@ne
      \huge\bfseries \thechapter.\hspace{1em}\ignorespaces
    \fi
    \interlinepenalty\@M
    \Huge \bfseries #1\par\nobreak
    \vskip 40\p@
  }}
}
\makeatother

%------------------------------------------------------------------------
% Paquetes de cógido
\usepackage{minted}
\renewcommand\listingscaption{Código fuente}

\usepackage{fancyvrb}
% Personaliza el tamaño de los números de línea
\renewcommand{\theFancyVerbLine}{\small\arabic{FancyVerbLine}}

% Estilo para C++
\newminted{cpp}{
    frame=lines,
    framesep=2mm,
    baselinestretch=1.2,
    linenos,
    escapeinside=||
}

% para minted
\definecolor{LightGray}{rgb}{0.95,0.95,0.92}
\setminted{
    linenos=true,
    stepnumber=5,
    numberfirstline=true,
    autogobble,
    breaklines=true,
    breakautoindent=true,
    breaksymbolleft=,
    breaksymbolright=,
    breaksymbolindentleft=0pt,
    breaksymbolindentright=0pt,
    breaksymbolsepleft=0pt,
    breaksymbolsepright=0pt,
    fontsize=\footnotesize,
    bgcolor=LightGray,
    numbersep=10pt
}


\usepackage{listings} % Para incluir código desde un archivo

\renewcommand\lstlistingname{Código Fuente}
\renewcommand\lstlistlistingname{Índice de Códigos Fuente}

% Definir colores
\definecolor{vscodepurple}{rgb}{0.5,0,0.5}
\definecolor{vscodeblue}{rgb}{0,0,0.8}
\definecolor{vscodegreen}{rgb}{0,0.5,0}
\definecolor{vscodegray}{rgb}{0.5,0.5,0.5}
\definecolor{vscodebackground}{rgb}{0.97,0.97,0.97}
\definecolor{vscodelightgray}{rgb}{0.9,0.9,0.9}

% Configuración para el estilo de C similar a VSCode
\lstdefinestyle{vscode_C}{
  backgroundcolor=\color{vscodebackground},
  commentstyle=\color{vscodegreen},
  keywordstyle=\color{vscodeblue},
  numberstyle=\tiny\color{vscodegray},
  stringstyle=\color{vscodepurple},
  basicstyle=\scriptsize\ttfamily,
  breakatwhitespace=false,
  breaklines=true,
  captionpos=b,
  keepspaces=true,
  numbers=left,
  numbersep=5pt,
  showspaces=false,
  showstringspaces=false,
  showtabs=false,
  tabsize=2,
  frame=tb,
  framerule=0pt,
  aboveskip=10pt,
  belowskip=10pt,
  xleftmargin=10pt,
  xrightmargin=10pt,
  framexleftmargin=10pt,
  framexrightmargin=10pt,
  framesep=0pt,
  rulecolor=\color{vscodelightgray},
  backgroundcolor=\color{vscodebackground},
}

%------------------------------------------------------------------------

% Comandos definidos
\newcommand{\bb}[1]{\mathbb{#1}}
\newcommand{\cc}[1]{\mathcal{#1}}

% I prefer the slanted \leq
\let\oldleq\leq % save them in case they're every wanted
\let\oldgeq\geq
\renewcommand{\leq}{\leqslant}
\renewcommand{\geq}{\geqslant}

% Si y solo si
\newcommand{\sii}{\iff}

% Letras griegas
\newcommand{\eps}{\epsilon}
\newcommand{\veps}{\varepsilon}
\newcommand{\lm}{\lambda}

\newcommand{\ol}{\overline}
\newcommand{\ul}{\underline}
\newcommand{\wt}{\widetilde}
\newcommand{\wh}{\widehat}

\let\oldvec\vec
\renewcommand{\vec}{\overrightarrow}

% Derivadas parciales
\newcommand{\del}[2]{\frac{\partial #1}{\partial #2}}
\newcommand{\Del}[3]{\frac{\partial^{#1} #2}{\partial #3^{#1}}}
\newcommand{\deld}[2]{\dfrac{\partial #1}{\partial #2}}
\newcommand{\Deld}[3]{\dfrac{\partial^{#1} #2}{\partial #3^{#1}}}


\newcommand{\AstIg}{\stackrel{(\ast)}{=}}
\newcommand{\Hop}{\stackrel{L'H\hat{o}pital}{=}}

\newcommand{\red}[1]{{\color{red}#1}} % Para integrales, destacar los cambios.

% Método de integración
\newcommand{\MetInt}[2]{
    \left[\begin{array}{c}
        #1 \\ #2
    \end{array}\right]
}

% Declarar aplicaciones
% 1. Nombre aplicación
% 2. Dominio
% 3. Codominio
% 4. Variable
% 5. Imagen de la variable
\newcommand{\Func}[5]{
    \begin{equation*}
        \begin{array}{rrll}
            #1:& #2 & \longrightarrow & #3\\
               & #4 & \longmapsto & #5
        \end{array}
    \end{equation*}
}

%------------------------------------------------------------------------

\newcommand{\T}[0]{\cc{T}}

\newcounter{ejercicio}[section] % Define el contador de ejercicio y lo reinicia con cada sección

\newcounter{ejercicio}
\newcommand{\resetearcontador}{%
  \setcounter{ejercicio}{0} % Resetea el contador de ejercicios a 0
}

\renewcommand{\theejercicio}{\arabic{ejercicio}}


\begin{document}

    % 1. Foto de fondo
    % 2. Título
    % 3. Encabezado Izquierdo
    % 4. Color de fondo
    % 5. Coord x del titulo
    % 6. Coord y del titulo
    % 7. Fecha

    
    % 1. Foto de fondo
% 2. Título
% 3. Encabezado Izquierdo
% 4. Color de fondo
% 5. Coord x del titulo
% 6. Coord y del titulo
% 7. Fecha

\newcommand{\portada}[7]{

    \portadaBase{#1}{#2}{#3}{#4}{#5}{#6}{#7}
    \portadaBook{#1}{#2}{#3}{#4}{#5}{#6}{#7}
}

\newcommand{\portadaExamen}[7]{

    \portadaBase{#1}{#2}{#3}{#4}{#5}{#6}{#7}
    \portadaArticle{#1}{#2}{#3}{#4}{#5}{#6}{#7}
}




\newcommand{\portadaBase}[7]{

    % Tiene la portada principal y la licencia Creative Commons
    
    % 1. Foto de fondo
    % 2. Título
    % 3. Encabezado Izquierdo
    % 4. Color de fondo
    % 5. Coord x del titulo
    % 6. Coord y del titulo
    % 7. Fecha
    
    
    \thispagestyle{empty}               % Sin encabezado ni pie de página
    \newgeometry{margin=0cm}        % Márgenes nulos para la primera página
    
    
    % Encabezado
    \fancyhead[L]{\helv #3}
    \fancyhead[R]{\helv \nouppercase{\leftmark}}
    
    
    \pagecolor{#4}        % Color de fondo para la portada
    
    \begin{figure}[p]
        \centering
        \transparent{0.3}           % Opacidad del 30% para la imagen
        
        \includegraphics[width=\paperwidth, keepaspectratio]{assets/#1}
    
        \begin{tikzpicture}[remember picture, overlay]
            \node[anchor=north west, text=white, opacity=1, font=\fontsize{60}{90}\selectfont\bfseries\sffamily, align=left] at (#5, #6) {#2};
            
            \node[anchor=south east, text=white, opacity=1, font=\fontsize{12}{18}\selectfont\sffamily, align=right] at (9.7, 3) {\textbf{\href{https://losdeldgiim.github.io/}{Los Del DGIIM}}};
            
            \node[anchor=south east, text=white, opacity=1, font=\fontsize{12}{15}\selectfont\sffamily, align=right] at (9.7, 1.8) {Doble Grado en Ingeniería Informática y Matemáticas\\Universidad de Granada};
        \end{tikzpicture}
    \end{figure}
    
    
    \restoregeometry        % Restaurar márgenes normales para las páginas subsiguientes
    \pagecolor{white}       % Restaurar el color de página
    
    
    \newpage
    \thispagestyle{empty}               % Sin encabezado ni pie de página
    \begin{tikzpicture}[remember picture, overlay]
        \node[anchor=south west, inner sep=3cm] at (current page.south west) {
            \begin{minipage}{0.5\paperwidth}
                \href{https://creativecommons.org/licenses/by-nc-nd/4.0/}{
                    \includegraphics[height=2cm]{assets/Licencia.png}
                }\vspace{1cm}\\
                Esta obra está bajo una
                \href{https://creativecommons.org/licenses/by-nc-nd/4.0/}{
                    Licencia Creative Commons Atribución-NoComercial-SinDerivadas 4.0 Internacional (CC BY-NC-ND 4.0).
                }\\
    
                Eres libre de compartir y redistribuir el contenido de esta obra en cualquier medio o formato, siempre y cuando des el crédito adecuado a los autores originales y no persigas fines comerciales. 
            \end{minipage}
        };
    \end{tikzpicture}
    
    
    
    % 1. Foto de fondo
    % 2. Título
    % 3. Encabezado Izquierdo
    % 4. Color de fondo
    % 5. Coord x del titulo
    % 6. Coord y del titulo
    % 7. Fecha


}


\newcommand{\portadaBook}[7]{

    % 1. Foto de fondo
    % 2. Título
    % 3. Encabezado Izquierdo
    % 4. Color de fondo
    % 5. Coord x del titulo
    % 6. Coord y del titulo
    % 7. Fecha

    % Personaliza el formato del título
    \pretitle{\begin{center}\bfseries\fontsize{42}{56}\selectfont}
    \posttitle{\par\end{center}\vspace{2em}}
    
    % Personaliza el formato del autor
    \preauthor{\begin{center}\Large}
    \postauthor{\par\end{center}\vfill}
    
    % Personaliza el formato de la fecha
    \predate{\begin{center}\huge}
    \postdate{\par\end{center}\vspace{2em}}
    
    \title{#2}
    \author{\href{https://losdeldgiim.github.io/}{Los Del DGIIM}}
    \date{Granada, #7}
    \maketitle
    
    \tableofcontents
}




\newcommand{\portadaArticle}[7]{

    % 1. Foto de fondo
    % 2. Título
    % 3. Encabezado Izquierdo
    % 4. Color de fondo
    % 5. Coord x del titulo
    % 6. Coord y del titulo
    % 7. Fecha

    % Personaliza el formato del título
    \pretitle{\begin{center}\bfseries\fontsize{42}{56}\selectfont}
    \posttitle{\par\end{center}\vspace{2em}}
    
    % Personaliza el formato del autor
    \preauthor{\begin{center}\Large}
    \postauthor{\par\end{center}\vspace{3em}}
    
    % Personaliza el formato de la fecha
    \predate{\begin{center}\huge}
    \postdate{\par\end{center}\vspace{5em}}
    
    \title{#2}
    \author{\href{https://losdeldgiim.github.io/}{Los Del DGIIM}}
    \date{Granada, #7}
    \thispagestyle{empty}               % Sin encabezado ni pie de página
    \maketitle
    \vfill
}
    \portadaExamen{ffccA4.jpg}{Topología I\\Examen IX}{Topología I. Examen IX}{MidnightBlue}{-8}{28}{2023-2024}{José Juan Urrutia Milán}

    \begin{description}
        \item[Asignatura] Topología I.
        \item[Curso Académico] 2023-24.
        \item[Grado] Doble Grado en Ingeniería Informática y Matemáticas.
        \item[Grupo] Único.
        \item[Profesor] Jose Antonio Gálvez López.
        \item[Descripción] Convocatoria Extraordinaria.
        \item[Fecha] 14 de febrero de 2023.
        \item[Duración] 3 horas.
    
    \end{description}
    \newpage

    \begin{ejercicio}[4.5 puntos]
        Sea $X = \bb{R}^2 \setminus \{(0,0)\}$. Para cada $(x,y) \in X$ denotamos:
        $$B_{(x,y)} = \{(tx, ty) \mid 0 < t \leq 1 \}$$
        En $X$, consideramos la familia de subconjuntos:
        $$\mathcal{B} = \{B_{(x,y)} \mid (x,y) \in X\}$$
        \begin{enumerate}[label=(\alph*)]
            \item Prueba que $\mathcal{B}$ es base para alguna topología $\mathcal{T}$ en $X$.
            \item ¿Verifica $(X, \mathcal{T})$ el segundo axioma de numerabilidad?
            \item Calcula el interior, la clausura y la frontera de $A = \{ (x,y) \in \bb{R}^2 \mid 0 < y \leq 1\}$.
            \item En $X$ consideramos la relación de equivalencia dada por:
                $$(x,y)R(x',y') \Leftrightarrow \exists \lambda > 0 \mid (x,y) = \lambda(x',y')$$
                Prueba que $(X/R, \mathcal{T}/R)$ es homeomorfo a $(\bb{S}^1, \mathcal{T}_D)$ donde $\mathcal{T}_D$ es la topología discreta.
        \end{enumerate}
    \end{ejercicio}
        
    \begin{ejercicio}[2.5 puntos]
        Elige una de las siguiente preguntas (2a o 2b):
        \begin{enumerate}
            \item[2a.] Sean $(X, \mathcal{T})$ e $(Y, \mathcal{T}')$ espacios topológicos. Prueba que $(X\times Y, \mathcal{T}\times \mathcal{T}')$ es conexo si y sólo si $(X, \mathcal{T})$ e $(Y, \mathcal{T}')$ son conexos.
            \item[2b.] Prueba que toda aplicación continua e inyectiva $f:(\bb{S}^1, \mathcal{T}_{u|\bb{S}^1}) \rightarrow (\bb{S}^1, \mathcal{T}_{u|\bb{S}^1})$ es un homeomorfismo.
        \end{enumerate}
    \end{ejercicio}


    \begin{ejercicio}[3 puntos]
        Consideremos $X = \bb{R} \cup \{p\}$, donde $p$ es un punto que no pertenece a $\bb{R}^2$ y 
        $$\mathcal{T} = \mathcal{T}_u \cup \{O \subseteq X \mid p \in O \land \bb{R}\setminus O \mbox{ es un compacto de } \mathcal{T}_u\}$$
        donde $\mathcal{T}_u$ denota la topología usual de $\bb{R}$. Demuéstrese que:
        \begin{enumerate}[label=(\alph*)]
            \item $(X,\mathcal{T})$ es un espacio topológico.
            \item $(X,\mathcal{T})$ es compacto.
            \item $(X,\mathcal{T})$ es $T_2$ y conexo.
        \end{enumerate}
    \end{ejercicio}
    
    \newpage
    \           % ------------------------------------------------------------------------------
    \newpage
    \resetearcontador

    \begin{ejercicio}[4.5 puntos]
        Sea $X = \bb{R}^2 \setminus \{(0,0)\}$. Para cada $(x,y) \in X$ denotamos:
        $$B_{(x,y)} = \{(tx, ty) \mid 0 < t \leq 1 \}$$
        En $X$, consideramos la familia de subconjuntos:
        $$\mathcal{B} = \{B_{(x,y)} \mid (x,y) \in X\}$$
        \begin{enumerate}[label=(\alph*)]
            \item Prueba que $\mathcal{B}$ es base para alguna topología $\mathcal{T}$ en $X$.\\

                \noindent
                Para una mayor intuición de la base dada, notemos que $\mathcal{B}_{(x,y)}$ es el segmento que une $(0,0)$ con $(x,y)$ sin incluir el $(0,0)$.\\

                \noindent
                Para ver que $\mathcal{B}$ es base, veamos que verifican B1 y B2:
                \begin{enumerate}
                    \item[B1)] $\displaystyle \bigcup_{B_{(x,y)} \in \mathcal{B}} B_{(x,y)} = X$\\

                        \noindent
                        Por ser $B_{(x,y)} \subseteq X ~~\forall B_{(x,y)} \in \mathcal{B}$, sabemos que $\displaystyle \bigcup_{B_{(x,y)} \in \mathcal{B}} B_{(x,y)} \subseteq X$.\newline
                        Sea $(x,y) \in X$, $(x,y) \in B_{(x,y)} \Rightarrow (x,y) \in \displaystyle \bigcup_{B_{(x,y)} \in \mathcal{B}} B_{(x,y)}$.\newline
                        Luego $X \subseteq \displaystyle \bigcup_{B_{(x,y)} \in \mathcal{B}} B_{(x,y)}$.\newline
                        $\displaystyle \bigcup_{B_{(x,y)} \in \mathcal{B}} B_{(x,y)} = X$ y se tiene B1.
                    \item[B2)] Sean $B_1$, $B_2 \in \mathcal{B}$. Si $x \in B_1 \cap B_2 \Rightarrow \exists B_3 \in \mathcal{B} \mid x \in B_3 \subseteq B_1 \cap B_2$\\

                        \noindent
                        Sean $B_{(x,y)}$, $B_{(x', y')} \in \mathcal{B}$. Sea $(u,v) \in B_{(x,y)} \cap B_{(x', y')} \Rightarrow$
                        $$\Rightarrow \exists \alpha, \lambda \in ]0,1] \mid (u,v) = (\alpha x, \alpha y) = (\lambda x', \lambda y') \Rightarrow (u,v) = \alpha (x,y) = \lambda (x',y') \Rightarrow$$
                        $$\Rightarrow \dfrac{\alpha}{\lambda}(x,y) = (x',y')$$
                        Si se intersecan, $(x,y)$ y $(x',y')$ son vectores proporcionales con factor de proporcionalidad positivo, luego es fácil ver que:
                        $$B_{(x,y)} \cap B_{(x', y')} = B = \{ (tx'', ty'') \mid 0 < t \leq 1\}$$
                        Donde:
                        $$(x'', y'') = \left\{ \begin{tabular}{lcl}
                                $(x,y)$ & \mbox{si} & \|(x,y)\| \leq \|(x',y')\| \\
                                $(x',y')$ & \mbox{si} & \|(x',y')\| < \|(x,y)\| \\
                        \end{tabular}\right.$$
                        Por lo que:
                        $$B = \left\{ \begin{tabular}{lcl}
                                $B_{(x,y)}$ & \mbox{si} & \|(x,y)\| \leq \|(x',y')\| \\
                                $B_{(x', y')}$ & \mbox{si} & \|(x',y')\| < \|(x,y)\| \\
                        \end{tabular}\right.$$
                        
                        $$B \in \mathcal{B} \mbox{ y además, } x \in B = B_{(x,y)} \cap B_{(x', y')}$$
                        Por lo que se tiene B2.
                \end{enumerate}

            \item ¿Verifica $(X, \mathcal{T})$ el segundo axioma de numerabilidad?\\

                \noindent \textbf{No:}\newline
                Supongamos que sí, luego $\exists \mathcal{B}'$ base numerable de $(X, \mathcal{T})$.\newline
                Sea $(x,y) \in X$, consideramos el abierto $B_{(x,y)}$. $\mathcal{B}'$ es base de $(X, \mathcal{T})$, luego:
                $$\exists B \in \mathcal{B}' \mid (x,y) \in B \subseteq B_{(x,y)}$$
                Pero como $B$ es el segmento que une el $(0,0)$ (sin él) con un punto, y contiene al $(x,y)$, ha de ser $B_{(x,y)}$ o algo mayor. Por estar contenido en $B_{(x,y)}$ sabemos que:
                $$B = B_{(x,y)}$$
                Por este razonamiento, podemos definir una aplicación $f:X \rightarrow \mathcal{B}'$ dada por:
                $$f(x,y) = B_{(x,y)}~~~\forall (x,y) \in X$$
                Que es claramente inyectiva, luego $|X| \leq |\mathcal{B}'|$. Pero $\mathcal{B}'$ era numerable y $X$ no lo es, \underline{contradicción}.
                
            \item Calcula el interior, la clausura y la frontera de $A = \{ (x,y) \in \bb{R}^2 \mid 0 < y \leq 1\}$.\\

                \noindent
                $$A = \bb{R} \times ]0,1]$$
                Veamos que $A$ es abierto:\newline
                Sea $(x,y) \in A$, $(x,y) \in B_{(x,y)} \in \mathcal{B}$. Veamos que $B_{(x,y)} \subseteq A$:\newline
                Sea $(tx, ty) \in B_{(x,y)}$ con $t \in ]0,1]$. Por ser $(x,y) \in A \Rightarrow y \in ]0,1]$.\newline
                Entonces, $ty \in ]0,1]$ por estar ambos en $]0,1]$ \footnote{Nociones de Cálculo I}.\newline
                Luego $(tx, ty) \in A~~\forall (tx, ty) \in B_{(x,y)}$.\newline
                Por lo que $(x,y) \in B_{(x,y)} \subseteq A~~\forall (x,y) \in A \Rightarrow A \in \mathcal{T}$.
                $$\mathrm{int}(A) = A$$

                \noindent
                Para la clausura, intuimos que será el conjunto $\bb{R} \times ]0, +\infty[$. Veamos primero que este conjunto es cerrado:\newline
                Consideramos $B = X\setminus (\bb{R} \times ]0,+\infty[) = (\bb{R} \times ]-\infty, 0]) \setminus \{(0,0)\}$ y veamos que es abierto:\newline
                Sea $(x,y) \in B$. Si $y = 0 \Rightarrow (x,0) \in B_{(x,0)} \subseteq (\bb{R} \times \{0\})\setminus \{(0,0)\} \subseteq B$.\newline
                Si $y \neq 0 \Rightarrow (x,y) \in \bb{R} \times ]-\infty, 0[ \Rightarrow y < 0$. $(x,y) \in B_{(x,y)} = \{(tx,ty) \mid t \in ]0,1]\}$.\newline
                Luego, sea $(tx, ty) \in B_{(x,y)} \Rightarrow ty < 0 \Rightarrow (tx, ty) \in \bb{R}\times ]-\infty, 0[ \subseteq B$.\newline
                Por tanto, $(x,y) \in B_{(x,y)} \subseteq B~~\forall (x,y) \in B \Rightarrow B \in \mathcal{T} \Rightarrow X\setminus B \in \mathcal{C_T}$.\\

                \noindent
                Para concluir que $\overline{A} = \bb{R} \times ]0, +\infty[$, veamos que $\forall (x,y) \in \bb{R} \times ]0, +\infty[$:
                $$A \cap B \neq \emptyset ~~\forall B \in \mathcal{B} \mid (x,y) \in B$$
                Sea $(x,y) \in \bb{R} \times ]0, +\infty[$, sea $B_{(x', y')} \in \mathcal{B} \mid (x,y) \in B_{(x', y')}$\newline
                Sea $(u,v) = \dfrac{(x',y')}{\|(x',y')\|} \in \bb{S}^1 \cap (\bb{R} \times ]0, +\infty[) \subseteq A$, proporcional a (x',y')\newline
                Luego: $(u,v) \in B_{(x', y')} \Rightarrow A \cap B_{(x', y')} \neq \emptyset$.
                $$\overline{A} = \bb{R} \times ]0, +\infty[$$

                \noindent
                Por tanto:
                $$\partial A = \overline{A}\setminus \mathrm{int}(A) = (\bb{R} \times ]0, +\infty[) \setminus (\bb{R} \times ]0, 1]) = \bb{R} \times ]1, +\infty[$$

            \item En $X$ consideramos la relación de equivalencia dada por:
                $$(x,y)R(x',y') \Leftrightarrow \exists \lambda > 0 \mid (x,y) = \lambda(x',y')$$
                Prueba que $(X/R, \mathcal{T}/R)$ es homeomorfo a $(\bb{S}^1, \mathcal{T}_D)$ donde $\mathcal{T}_D$ es la topología discreta.\\

                \noindent
                TODO.
        \end{enumerate}
    \end{ejercicio}
        
    \begin{ejercicio}[2.5 puntos]
        Elige una de las siguiente preguntas (2a o 2b):
        \begin{enumerate}
            \item[2a.] Sean $(X, \mathcal{T})$ e $(Y, \mathcal{T}')$ espacios topológicos. Prueba que $(X\times Y, \mathcal{T}\times \mathcal{T}')$ es conexo si y sólo si $(X, \mathcal{T})$ e $(Y, \mathcal{T}')$ son conexos.\\

                \noindent
                $\Rightarrow)$ Supongamos que $(X \times Y, \mathcal{T}, \mathcal{T}')$ es conexo. Recordamos que $\pi_X$ y $\pi_Y$ son continuas y que la imagen de un conexo por una aplicación continua es un conjunto conexo. Como:
                $$\pi_X(X\times Y) = X~~~~~~~~\pi_Y(X \times Y) = Y$$
                Se tiene que $(X, \mathcal{T})$ e $(Y, \mathcal{T}')$ son conexos.\\

                \noindent
                $\Leftarrow)$ Supongamos que $(X, \mathcal{T})$ e $(Y, \mathcal{T}')$ son conexos. Como la conexión es una propiedad topológica (se conserva por homeomorfismos) y tenemos que, para todo $x \in X$ e $y \in Y$:
                $$X \cong X \times \{y\}~~~~~~~~ Y \cong \{x\} \times Y$$
                Se tiene que $X \times \{y\}$ y $\{x\} \times Y$ son conexos $\forall x \in X, y \in Y$.\newline
                Recordamos además que la unión de conexos que intersecan a uno fijo es un conjunto conexo, luego (fijado un $y \in Y$):
                $$X \times Y = \left( \bigcup_{x \in X} \{x\} \times Y\right) \cup (X \times \{y\})$$
                $$ (x,y) \in (\{x\} \times Y) \cap (X \times \{y\}) \neq \emptyset~~~~\forall x \in X, y \in Y$$
                Es un conjunto conexo.
                
            \item[2b.] Prueba que toda aplicación continua e inyectiva $f:(\bb{S}^1, \mathcal{T}_{u|\bb{S}^1}) \rightarrow (\bb{S}^1, \mathcal{T}_{u|\bb{S}^1})$ es un homeomorfismo.\\

                \noindent
                Sea $f:(\bb{S}^1, \mathcal{T}_{u|\bb{S}^1}) \rightarrow (\bb{S}^1, \mathcal{T}_{u|\bb{S}^1})$ continua e inyectiva, falta ver que es sobreyectiva y abierta (o cerrada o que su inversa es continua) para llegar a que es un homeomorfismo.

                \noindent
                Comenzamos viendo que es cerrada:\newline
                Sabemos que $(\bb{S}^1, \mathcal{T}_{u|\bb{S}^1})$ es compacta y $T_2$ (visto en clase), luego $f$ es cerrada por un teorema visto en teoria.

                \noindent
                Veamos ahora que es sobreyectiva:\newline
                Supongamos que no, luego $\exists p \in \bb{S}^1 \mid$ Im$(f) \subseteq \bb{S}^1 \setminus \{p\}$\newline
                TODO


        \end{enumerate}
    \end{ejercicio}


    \begin{ejercicio}[3 puntos]
        Consideremos $X = \bb{R} \cup \{p\}$, donde $p$ es un punto que no pertenece a $\bb{R}^2$ y 
        $$\mathcal{T} = \mathcal{T}_u \cup \{O \subseteq X \mid p \in O \land \bb{R}\setminus O \mbox{ es un compacto de } \mathcal{T}_u\}$$
        donde $\mathcal{T}_u$ denota la topología usual de $\bb{R}$. Demuéstrese que:
        \begin{enumerate}[label=(\alph*)]
            \item $(X,\mathcal{T})$ es un espacio topológico.\\

                \noindent
                Notaremos al segundo conjunto como:
                $$\Lambda = \{O \subseteq X \mid p \in O \land \bb{R}\setminus O \mbox{ es un compacto de } \mathcal{T}_u\} $$

                \noindent
                Primero, reescribimos $\Lambda$ para tener una mayor intuición de lo que tenemos que demostrar:
                $$\Lambda = \{O \subseteq X \mid p \in O \land \bb{R}\setminus O \mbox{ es cerrado y acotado en } \mathcal{T}_u\} $$
                $$ = \{ O \subseteq X \mid p \in O \land O \setminus \{p\} \in \mathcal{T}_u \mbox{ y no está mayorado ni minorado }\}$$
                Veamos que se verifican A1, A2 y A3 para tener el apartado probado:

                \begin{enumerate}
                    \item[A1)] $\emptyset$, $X \in \mathcal{T}$\\
                        
                        \noindent
                        $\emptyset \in \mathcal{T}$

                        \noindent
                        $\bb{R} \in \mathcal{T}_u$ y $A = ]-\infty, -1[ \cup \{p\} \cup ]1, +\infty[ \in \Lambda$\newline
                        Luego: $X = \bb{R} \cup A \in \mathcal{T}$
                    \item[A2)] Si $\{U_i\}_{i \in I} \subseteq \mathcal{T} \Rightarrow \displaystyle \bigcup_{i \in I} U_i \in \mathcal{T}$\\

                        \noindent
                        Sea $J = \{i \in I \mid p \notin U_i\} = \{i \in I \mid U_i \notin \Lambda\}$
                        $$U_j \in \mathcal{T}_u ~~\forall j \in J \Rightarrow \bigcup_{j \in J} U_j \in \mathcal{T}_u \subseteq \mathcal{T}$$
                        $$U_i \in \Lambda~~\forall i \in I \setminus J \mbox{ .Veamos si } \bigcup_{i \in I\setminus J} U_i \in \Lambda$$
                        $p \in U_i~~\forall i \in I \setminus J \Rightarrow p \in \displaystyle \bigcup_{i \in I \setminus J} U_i$\newline
                        $U_i \setminus \{p\} \in \mathcal{T}_u ~~ \forall i \in I \setminus J \Rightarrow \displaystyle \bigcup_{i \in I \setminus J } (U_i \setminus \{p\}) \in \mathcal{T}_u$\newline
                        Sea $h \in I \setminus J$, tenemos que $U_h$ no está mayorado ni minorado y por ser $U_h \subseteq \displaystyle \bigcup_{i \in I \setminus J} (U_i \setminus \{p\}) \Rightarrow \displaystyle \bigcup_{i \in I \setminus J} (U_i \setminus \{p\})$ no está mayorado ni minorado.
                        
                        \noindent
                        Por tanto, $ \displaystyle \bigcup_{i \in I \setminus J} (U_i \setminus \{p\}) \in \Lambda$ 
                        $$\displaystyle \bigcup_{i \in I} U_i = \bigcup_{j \in J}U_j \cup \bigcup_{i \in I \setminus J} U_i \in \mathcal{T}_u \cup \Lambda = \mathcal{T}$$

                    \item[A3)] Si $U, V \in \mathcal{T} \Rightarrow U \cap V \in \mathcal{T}$\\

                        \noindent
                        Si $U$, $V \in \mathcal{T}_u \Rightarrow U \cap V \in \mathcal{T}_u \subseteq \mathcal{T}$.\\

                        \noindent
                        Si $U \in \mathcal{T}_u$ y $V \in \Lambda \Rightarrow U \cap V = U \cap (V \setminus \{p\}) \in \mathcal{T}_u \subseteq \mathcal{T}$\\

                        \noindent
                        Si $U, V \in \Lambda$:\newline
                        $p \in U \land p \in V \Rightarrow p \in U \cap V$\newline
                        $\bb{R}\setminus U \in \mathcal{C_T} \land \bb{R}\setminus V \in \mathcal{C_T} \Rightarrow \bb{R}\setminus U \cup \bb{R}\setminus V = \bb{R}\setminus (U \cap V)\in \mathcal{C_T}$\newline
                        $\bb{R}\setminus U \mbox{ acotado } \land \bb{R}\setminus V \mbox{ acotado } \Rightarrow \bb{R}\setminus U \cup \bb{R}\setminus V = \bb{R}\setminus (U \cap V)\mbox{ acotado }$\newline
                        Luego $U \cap V \in \Lambda$.
                \end{enumerate}

            \item $(X,\mathcal{T})$ es compacto.
            \item $(X,\mathcal{T})$ es $T_2$ y conexo.
        \end{enumerate}
    \end{ejercicio}

\end{document}
