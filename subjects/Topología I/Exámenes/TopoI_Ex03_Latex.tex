\documentclass[12pt]{article}

% Idioma y codificación
\usepackage[spanish, es-tabla]{babel}       %es-tabla para que se titule "Tabla"
\usepackage[utf8]{inputenc}

% Márgenes
\usepackage[a4paper,top=3cm,bottom=2.5cm,left=3cm,right=3cm]{geometry}

% Comentarios de bloque
\usepackage{verbatim}

% Paquetes de links
\usepackage[hidelinks]{hyperref}    % Permite enlaces
\usepackage{url}                    % redirecciona a la web

% Más opciones para enumeraciones
\usepackage{enumitem}

% Personalizar la portada
\usepackage{titling}

% Paquetes de tablas
\usepackage{multirow}


%------------------------------------------------------------------------

%Paquetes de figuras
\usepackage{caption}
\usepackage{subcaption} % Figuras al lado de otras
\usepackage{float}      % Poner figuras en el sitio indicado H.


% Paquetes de imágenes
\usepackage{graphicx}       % Paquete para añadir imágenes
\usepackage{transparent}    % Para manejar la opacidad de las figuras

% Paquete para usar colores
\usepackage[dvipsnames]{xcolor}
\usepackage{pagecolor}      % Para cambiar el color de la página

% Habilita tamaños de fuente mayores
\usepackage{fix-cm}

% Para los gráficos
\usepackage{tikz}

% Para poder situar los nodos en los grafos
\usetikzlibrary{positioning}


%------------------------------------------------------------------------

% Paquetes de matemáticas
\usepackage{mathtools, amsfonts, amssymb, mathrsfs}
\usepackage[makeroom]{cancel}     % Simplificar tachando
\usepackage{polynom}    % Divisiones y Ruffini
\usepackage{units} % Para poner fracciones diagonales con \nicefrac

\usepackage{pgfplots}   %Representar funciones
\pgfplotsset{compat=1.18}  % Versión 1.18

\usepackage{tikz-cd}    % Para usar diagramas de composiciones
\usetikzlibrary{calc}   % Para usar cálculo de coordenadas en tikz

%Definición de teoremas, etc.
\usepackage{amsthm}
%\swapnumbers   % Intercambia la posición del texto y de la numeración

\theoremstyle{plain}

\makeatletter
\@ifclassloaded{article}{
  \newtheorem{teo}{Teorema}[section]
}{
  \newtheorem{teo}{Teorema}[chapter]  % Se resetea en cada chapter
}
\makeatother

\newtheorem{coro}{Corolario}[teo]           % Se resetea en cada teorema
\newtheorem{prop}[teo]{Proposición}         % Usa el mismo contador que teorema
\newtheorem{lema}[teo]{Lema}                % Usa el mismo contador que teorema

\theoremstyle{remark}
\newtheorem*{observacion}{Observación}

\theoremstyle{definition}

\makeatletter
\@ifclassloaded{article}{
  \newtheorem{definicion}{Definición} [section]     % Se resetea en cada chapter
}{
  \newtheorem{definicion}{Definición} [chapter]     % Se resetea en cada chapter
}
\makeatother

\newtheorem*{notacion}{Notación}
\newtheorem*{ejemplo}{Ejemplo}
\newtheorem*{ejercicio*}{Ejercicio}             % No numerado
\newtheorem{ejercicio}{Ejercicio} [section]     % Se resetea en cada section


% Modificar el formato de la numeración del teorema "ejercicio"
\renewcommand{\theejercicio}{%
  \ifnum\value{section}=0 % Si no se ha iniciado ninguna sección
    \arabic{ejercicio}% Solo mostrar el número de ejercicio
  \else
    \thesection.\arabic{ejercicio}% Mostrar número de sección y número de ejercicio
  \fi
}


% \renewcommand\qedsymbol{$\blacksquare$}         % Cambiar símbolo QED
%------------------------------------------------------------------------

% Paquetes para encabezados
\usepackage{fancyhdr}
\pagestyle{fancy}
\fancyhf{}

\newcommand{\helv}{ % Modificación tamaño de letra
\fontfamily{}\fontsize{12}{12}\selectfont}
\setlength{\headheight}{15pt} % Amplía el tamaño del índice


%\usepackage{lastpage}   % Referenciar última pag   \pageref{LastPage}
\fancyfoot[C]{\thepage}

%------------------------------------------------------------------------

% Conseguir que no ponga "Capítulo 1". Sino solo "1."
\makeatletter
\@ifclassloaded{book}{
  \renewcommand{\chaptermark}[1]{\markboth{\thechapter.\ #1}{}} % En el encabezado
    
  \renewcommand{\@makechapterhead}[1]{%
  \vspace*{50\p@}%
  {\parindent \z@ \raggedright \normalfont
    \ifnum \c@secnumdepth >\m@ne
      \huge\bfseries \thechapter.\hspace{1em}\ignorespaces
    \fi
    \interlinepenalty\@M
    \Huge \bfseries #1\par\nobreak
    \vskip 40\p@
  }}
}
\makeatother

%------------------------------------------------------------------------
% Paquetes de cógido
\usepackage{minted}
\renewcommand\listingscaption{Código fuente}

\usepackage{fancyvrb}
% Personaliza el tamaño de los números de línea
\renewcommand{\theFancyVerbLine}{\small\arabic{FancyVerbLine}}

% Estilo para C++
\newminted{cpp}{
    frame=lines,
    framesep=2mm,
    baselinestretch=1.2,
    linenos,
    escapeinside=||
}

% para minted
\definecolor{LightGray}{rgb}{0.95,0.95,0.92}
\setminted{
    linenos=true,
    stepnumber=5,
    numberfirstline=true,
    autogobble,
    breaklines=true,
    breakautoindent=true,
    breaksymbolleft=,
    breaksymbolright=,
    breaksymbolindentleft=0pt,
    breaksymbolindentright=0pt,
    breaksymbolsepleft=0pt,
    breaksymbolsepright=0pt,
    fontsize=\footnotesize,
    bgcolor=LightGray,
    numbersep=10pt
}


\usepackage{listings} % Para incluir código desde un archivo

\renewcommand\lstlistingname{Código Fuente}
\renewcommand\lstlistlistingname{Índice de Códigos Fuente}

% Definir colores
\definecolor{vscodepurple}{rgb}{0.5,0,0.5}
\definecolor{vscodeblue}{rgb}{0,0,0.8}
\definecolor{vscodegreen}{rgb}{0,0.5,0}
\definecolor{vscodegray}{rgb}{0.5,0.5,0.5}
\definecolor{vscodebackground}{rgb}{0.97,0.97,0.97}
\definecolor{vscodelightgray}{rgb}{0.9,0.9,0.9}

% Configuración para el estilo de C similar a VSCode
\lstdefinestyle{vscode_C}{
  backgroundcolor=\color{vscodebackground},
  commentstyle=\color{vscodegreen},
  keywordstyle=\color{vscodeblue},
  numberstyle=\tiny\color{vscodegray},
  stringstyle=\color{vscodepurple},
  basicstyle=\scriptsize\ttfamily,
  breakatwhitespace=false,
  breaklines=true,
  captionpos=b,
  keepspaces=true,
  numbers=left,
  numbersep=5pt,
  showspaces=false,
  showstringspaces=false,
  showtabs=false,
  tabsize=2,
  frame=tb,
  framerule=0pt,
  aboveskip=10pt,
  belowskip=10pt,
  xleftmargin=10pt,
  xrightmargin=10pt,
  framexleftmargin=10pt,
  framexrightmargin=10pt,
  framesep=0pt,
  rulecolor=\color{vscodelightgray},
  backgroundcolor=\color{vscodebackground},
}

%------------------------------------------------------------------------

% Comandos definidos
\newcommand{\bb}[1]{\mathbb{#1}}
\newcommand{\cc}[1]{\mathcal{#1}}

% I prefer the slanted \leq
\let\oldleq\leq % save them in case they're every wanted
\let\oldgeq\geq
\renewcommand{\leq}{\leqslant}
\renewcommand{\geq}{\geqslant}

% Si y solo si
\newcommand{\sii}{\iff}

% Letras griegas
\newcommand{\eps}{\epsilon}
\newcommand{\veps}{\varepsilon}
\newcommand{\lm}{\lambda}

\newcommand{\ol}{\overline}
\newcommand{\ul}{\underline}
\newcommand{\wt}{\widetilde}
\newcommand{\wh}{\widehat}

\let\oldvec\vec
\renewcommand{\vec}{\overrightarrow}

% Derivadas parciales
\newcommand{\del}[2]{\frac{\partial #1}{\partial #2}}
\newcommand{\Del}[3]{\frac{\partial^{#1} #2}{\partial #3^{#1}}}
\newcommand{\deld}[2]{\dfrac{\partial #1}{\partial #2}}
\newcommand{\Deld}[3]{\dfrac{\partial^{#1} #2}{\partial #3^{#1}}}


\newcommand{\AstIg}{\stackrel{(\ast)}{=}}
\newcommand{\Hop}{\stackrel{L'H\hat{o}pital}{=}}

\newcommand{\red}[1]{{\color{red}#1}} % Para integrales, destacar los cambios.

% Método de integración
\newcommand{\MetInt}[2]{
    \left[\begin{array}{c}
        #1 \\ #2
    \end{array}\right]
}

% Declarar aplicaciones
% 1. Nombre aplicación
% 2. Dominio
% 3. Codominio
% 4. Variable
% 5. Imagen de la variable
\newcommand{\Func}[5]{
    \begin{equation*}
        \begin{array}{rrll}
            #1:& #2 & \longrightarrow & #3\\
               & #4 & \longmapsto & #5
        \end{array}
    \end{equation*}
}

%------------------------------------------------------------------------

\newcommand{\T}[0]{\cc{T}}

\begin{document}

    % 1. Foto de fondo
    % 2. Título
    % 3. Encabezado Izquierdo
    % 4. Color de fondo
    % 5. Coord x del titulo
    % 6. Coord y del titulo
    % 7. Fecha

    
    % 1. Foto de fondo
% 2. Título
% 3. Encabezado Izquierdo
% 4. Color de fondo
% 5. Coord x del titulo
% 6. Coord y del titulo
% 7. Fecha

\newcommand{\portada}[7]{

    \portadaBase{#1}{#2}{#3}{#4}{#5}{#6}{#7}
    \portadaBook{#1}{#2}{#3}{#4}{#5}{#6}{#7}
}

\newcommand{\portadaExamen}[7]{

    \portadaBase{#1}{#2}{#3}{#4}{#5}{#6}{#7}
    \portadaArticle{#1}{#2}{#3}{#4}{#5}{#6}{#7}
}




\newcommand{\portadaBase}[7]{

    % Tiene la portada principal y la licencia Creative Commons
    
    % 1. Foto de fondo
    % 2. Título
    % 3. Encabezado Izquierdo
    % 4. Color de fondo
    % 5. Coord x del titulo
    % 6. Coord y del titulo
    % 7. Fecha
    
    
    \thispagestyle{empty}               % Sin encabezado ni pie de página
    \newgeometry{margin=0cm}        % Márgenes nulos para la primera página
    
    
    % Encabezado
    \fancyhead[L]{\helv #3}
    \fancyhead[R]{\helv \nouppercase{\leftmark}}
    
    
    \pagecolor{#4}        % Color de fondo para la portada
    
    \begin{figure}[p]
        \centering
        \transparent{0.3}           % Opacidad del 30% para la imagen
        
        \includegraphics[width=\paperwidth, keepaspectratio]{assets/#1}
    
        \begin{tikzpicture}[remember picture, overlay]
            \node[anchor=north west, text=white, opacity=1, font=\fontsize{60}{90}\selectfont\bfseries\sffamily, align=left] at (#5, #6) {#2};
            
            \node[anchor=south east, text=white, opacity=1, font=\fontsize{12}{18}\selectfont\sffamily, align=right] at (9.7, 3) {\textbf{\href{https://losdeldgiim.github.io/}{Los Del DGIIM}}};
            
            \node[anchor=south east, text=white, opacity=1, font=\fontsize{12}{15}\selectfont\sffamily, align=right] at (9.7, 1.8) {Doble Grado en Ingeniería Informática y Matemáticas\\Universidad de Granada};
        \end{tikzpicture}
    \end{figure}
    
    
    \restoregeometry        % Restaurar márgenes normales para las páginas subsiguientes
    \pagecolor{white}       % Restaurar el color de página
    
    
    \newpage
    \thispagestyle{empty}               % Sin encabezado ni pie de página
    \begin{tikzpicture}[remember picture, overlay]
        \node[anchor=south west, inner sep=3cm] at (current page.south west) {
            \begin{minipage}{0.5\paperwidth}
                \href{https://creativecommons.org/licenses/by-nc-nd/4.0/}{
                    \includegraphics[height=2cm]{assets/Licencia.png}
                }\vspace{1cm}\\
                Esta obra está bajo una
                \href{https://creativecommons.org/licenses/by-nc-nd/4.0/}{
                    Licencia Creative Commons Atribución-NoComercial-SinDerivadas 4.0 Internacional (CC BY-NC-ND 4.0).
                }\\
    
                Eres libre de compartir y redistribuir el contenido de esta obra en cualquier medio o formato, siempre y cuando des el crédito adecuado a los autores originales y no persigas fines comerciales. 
            \end{minipage}
        };
    \end{tikzpicture}
    
    
    
    % 1. Foto de fondo
    % 2. Título
    % 3. Encabezado Izquierdo
    % 4. Color de fondo
    % 5. Coord x del titulo
    % 6. Coord y del titulo
    % 7. Fecha


}


\newcommand{\portadaBook}[7]{

    % 1. Foto de fondo
    % 2. Título
    % 3. Encabezado Izquierdo
    % 4. Color de fondo
    % 5. Coord x del titulo
    % 6. Coord y del titulo
    % 7. Fecha

    % Personaliza el formato del título
    \pretitle{\begin{center}\bfseries\fontsize{42}{56}\selectfont}
    \posttitle{\par\end{center}\vspace{2em}}
    
    % Personaliza el formato del autor
    \preauthor{\begin{center}\Large}
    \postauthor{\par\end{center}\vfill}
    
    % Personaliza el formato de la fecha
    \predate{\begin{center}\huge}
    \postdate{\par\end{center}\vspace{2em}}
    
    \title{#2}
    \author{\href{https://losdeldgiim.github.io/}{Los Del DGIIM}}
    \date{Granada, #7}
    \maketitle
    
    \tableofcontents
}




\newcommand{\portadaArticle}[7]{

    % 1. Foto de fondo
    % 2. Título
    % 3. Encabezado Izquierdo
    % 4. Color de fondo
    % 5. Coord x del titulo
    % 6. Coord y del titulo
    % 7. Fecha

    % Personaliza el formato del título
    \pretitle{\begin{center}\bfseries\fontsize{42}{56}\selectfont}
    \posttitle{\par\end{center}\vspace{2em}}
    
    % Personaliza el formato del autor
    \preauthor{\begin{center}\Large}
    \postauthor{\par\end{center}\vspace{3em}}
    
    % Personaliza el formato de la fecha
    \predate{\begin{center}\huge}
    \postdate{\par\end{center}\vspace{5em}}
    
    \title{#2}
    \author{\href{https://losdeldgiim.github.io/}{Los Del DGIIM}}
    \date{Granada, #7}
    \thispagestyle{empty}               % Sin encabezado ni pie de página
    \maketitle
    \vfill
}
    \portadaExamen{ffccA4.jpg}{Topología I\\Examen III}{Topología I. Examen III}{MidnightBlue}{-8}{28}{2023-2024}{Arturo Olivares Martos}

    \begin{description}
        \item[Asignatura] Topología I.
        \item[Curso Académico] 2023-24.
        \item[Grado] Grado en Matemáticas.
        \item[Grupo] B.
        \item[Profesor] Miguel Ortega Titos.
        \item[Descripción] Parcial 1.
        \item[Fecha] 30 de octubre de 2023.
        %\item[Duración] 60 minutos.
    
    \end{description}
    \newpage
    
    En $\bb{R}$, se considera la topología de Sorgenfrey, $\T_S$. En $\bb{R}^2$, se considera la topología producto $\T=\T_S\times \T_S$.
    \begin{ejercicio}
        Dado el conjunto $A=\{(x,y)\in \bb{R}^2\mid x^2+y^2<1, x+y\leq 0\}$, calcula:
        \begin{enumerate}
            \item (2 puntos) El interior de $A$.
            
            Representemos en primer lugar el conjunto $A$:
            \begin{figure}[H]
                \centering
                \begin{tikzpicture}[scale=1.4]
                    
                    \coordinate (circ_init) at (135:1);
                    \coordinate (circ_end) at (315:1);

                    % Arco de circunferencia centrada en el origen
                    \draw[thick, dashed, fill=blue!30] (circ_init) arc (135:315:1);

                    % Cierre de la circunferencia
                    \draw[thick] (circ_end) -- (circ_init);

                    % Ejes coordenados
                    \draw[-stealth, dashed] (-1.5,0) -- (1.5,0) node[right] {$x$};
                    \draw[-stealth, dashed] (0,-1.5) -- (0,1.5) node[above] {$y$};
                \end{tikzpicture}
                \caption{Representación de $A$.}
            \end{figure}

            Tenemos que, dado $(x,y)\in \bb{R}^2$, una base de entornos de $(x,y)$ en $(\bb{R}^2,\T)$ es:
            \begin{equation*}
                \beta_{(x,y)} = \left\{[x,x+\veps[~\times~[y,y+\veps'[~\mid~\veps,\veps'\in \bb{R}^+\right\}
            \end{equation*}

            Por tanto, demostraremos que $A^\circ = \wt{A}$, con:
            \begin{equation*}
                \wt{A}=\{(x,y)\in \bb{R}^2\mid x^2+y^2<1, x+y< 0\}
            \end{equation*}
            \begin{description}
                \item[$\supset)$] Veamos en primer lugar que $\T_u \subset \T$. Una base de $\T$ es:
                \begin{equation*}
                    \cc{B}_S = \left\{[a,b[~\times ~[c,d[~\mid~a,b,c,d\in \bb{R},~ a<b,~c<d\right\}
                \end{equation*}
                Una base de $(\bb{R}^2, \T_u)$ es:
                \begin{equation*}
                    \cc{B}_u = \left\{]a,b[~\times ~]c,d[~\mid~a,b,c,d\in \bb{R},~ a<b,~c<d\right\}
                \end{equation*}

                Sea $]a,b[~\times ~]c,d[\in \cc{B}_u$, y sea $(x,y)\in ]a,b[~\times ~]c,d[$. Entonces, como $\T_u\subset \T_S$,
                $\exists [a',b'[, [c',d'[$ tal que $x\in [a',b'[\subset ]a,b[$ y $y\in [c',d'[\subset ]c,d[$. Por tanto, $(x,y)\in [a',b'[\times [c',d'[\subset ]a,b[~\times ~]c,d[$. Por tanto, $\T_u \subset \T$.

                Como $\wt{A}\in \T_u$ por ser intersección de dos abiertos, $\wt{A}\in \T$. Además, como $\wt{A}\subset A$, tenemos que $\wt{A}\subset A^\circ$.

                \item[$\supset)$] Veremos que, dado $(x,y)\in A\setminus \wt{A}$, entonces $(x,y)\notin A^\circ$.
                Como $(x,y)\in A\setminus \wt{A}$, entonces $x^2+y^2<1$ y $x+y=0$, es decir, $y=-x$. Veamos que $(x,-x)\notin A^\circ$.

                Supongamos que $\exists V\in \beta_{(x,y)}$ tal que $V\subset A$. Entonces, $\exists \veps,\veps'\in \bb{R}^+$ tal que:
                \begin{equation*}
                    V=[x,x+\veps[~\times~[y,y+\veps'[=[x,x+\veps[~\times~[-x,-x+\veps'[ ~ \subset A
                \end{equation*}

                De esta forma, $\left(x+\frac{\veps}{2},-x+\frac{\veps'}{2}\right)\in V\subset A$, pero:
                \begin{equation*}
                    x+\frac{\veps}{2} + \left(-x+\frac{\veps'}{2}\right) = \frac{\veps+\veps'}{2} > 0
                \end{equation*}
                Por tanto, llegamos a una contradicción, y $(x,y)\notin A^\circ$.
                De esta forma, $A^\circ \subset \wt{A}$.
            \end{description}

            \item (2 puntos) La frontera de $A$.
            
            Para calcular la frontera de $A$, calcularemos primero el cierre de $A$. Para ello, veremos que $\ol{A}=\wh{A}$, con:
            \begin{equation*}
                \wh{A}=\{(x,y)\in \bb{R}^2\mid x^2+y^2\leq 1, x+y\leq 0\}
            \end{equation*}

            \begin{description}
                \item[$\subset)$] Veamos que $\ol{A} \subset \wh{A}$. Como $\wh{A}\in C_{\T_u}$ y $\T_u\subset \T$, entonces $\wh{A}\in C_\T$.
                Por tanto, como además $A\subset \wh{A}$, se tiene que $\ol{A}\subset \wh{A}$.

                \item[$\supset)$] Veamos que $\wh{A}\subset \ol{A}$. Dado $(x,y)\in \wh{A}$, veremos que $\forall V\in \beta_{(x,y)}$, $V\cap A\neq \emptyset$.
                Como $A\subset \wh{A}$, tomaremos $(x,y)\in \wh{A}\setminus A$, ya que en el primer caso es trivial que $(x,y)\in V\cap A\neq \emptyset$.
                Por tanto, sea $(x,y)\in \wh{A}\setminus A$. Entonces, $x^2+y^2=1$ y $x+y<0$. Veamos que $(x,y)\in \ol{A}$.

                Sea $V\in \beta_{(x,y)}$, por lo que $V=[x,x+\veps[~\times~[y,y+\veps'[~\subset \bb{R}^2$ con $\veps,\veps'\in \bb{R}^+$.
                Sea $\delta=\dfrac{\min\{\veps, \veps'\}}{2}$.
                Veamos que $(x,y)-\delta(x,y)\in A\cap V$.
                \begin{enumerate}
                    \item Como $\delta \leq \veps,\veps'$, tenemos que pertenece a $V$.
                    \item Veamos que pertenece a $A$. Veamos primero que $x+y\leq 0$:
                    \begin{equation*}
                        x-x\cdot (x,y)+ y -y (x,y)= (x+y)\left(1-(x,y)\right)
                    \end{equation*}
                \end{enumerate}

                \begin{enumerate}
                    \item Supongamos $y<0$. Entonces, veamos que $\exists \delta \in \bb{R}^+$, $0<\delta<\veps'$ tal que $\left(x, y+\delta\right)\in V\cap A$.
                    Veamos que pertenece a $A$:
                    \begin{gather*}
                        x^2 + \left(y+\delta\right)^2 = x^2 + y^2 + 2y\delta + \delta^2 = 1 + 2y\delta + \delta^2 < 1 \Longleftrightarrow \\
                        \hspace{3cm}\Longleftrightarrow \delta(2y+\delta) < 0 \Longleftrightarrow \delta < -2y \\
                        x + y + \delta \leq 0 \Longleftrightarrow \delta \leq -x-y = -(x+y)
                    \end{gather*}
                    Como $y<0$, entonces $-2y>0$, y como $x+y<0$, entonces sea $\delta=\min\{-2y, -(x+y)\}$, y sin pérdida de generalidad suponemos $\delta < \veps$, que en caso contrario tomaríamos $0<\delta'<\delta$ y se tendría.

                    \item Supongamos $y\geq 0$. Entonces, veamos que $\exists \delta \in \bb{R}^+$, $0<\delta<\veps'$ tal que $\left(x+\delta, y\right)\in V\cap A$.
                    Como $x+y<0$ e $y\geq 0$, entonces $x<0$. Entonces, para que pertenezca a $A$:
                    \begin{gather*}
                        \left(x+\delta\right)^2 + y^2  = x^2 + y^2 + 2x\delta + \delta^2 = 1 + 2x\delta + \delta^2 < 1 \Longleftrightarrow \\
                        \hspace{3cm}\Longleftrightarrow \delta(2x+\delta) < 0 \Longleftrightarrow \delta < -2x \\
                        x + y + \delta \leq 0 \Longleftrightarrow \delta \leq -x-y = -(x+y)
                    \end{gather*}
                    De manera análoga, tomando $\delta=\min\{-2x, -(x+y)\}$ y suponiendo $\delta<\veps$, tenemos que pertenece a $A\cap V$.
                \end{enumerate}
            \end{description}
        \end{enumerate}
    \end{ejercicio}

    \begin{ejercicio}[1 punto]
        Estudia si el espacio topológico $(\bb{R}^2,\T)$ es o no T2.
    \end{ejercicio}

    \begin{ejercicio}[1.5 puntos]
        Encuentra un subconjunto $B\subset \bb{R}^2$ tal que la topología inducida $\T_{B}$ sea la discreta en $B$, pero la topología ${(\T_u^2)}_B$ no sea la discreta.
    \end{ejercicio}

    \begin{ejercicio}
        Estudia si el espacio topológico es:
        \begin{enumerate}
            \item (1 punto) 1AN.
            \item (1 punto) 2AN.
        \end{enumerate}
    \end{ejercicio}

    \begin{ejercicio}[1.5 puntos]
        Un subconjunto $C$ se dice frontera si $C\subset \partial C$. Encuentra un subconjunto $C\subset \bb{R}^2$ que sea frontera, infinito y que no esté incluido en $B$.
    \end{ejercicio}
\end{document}