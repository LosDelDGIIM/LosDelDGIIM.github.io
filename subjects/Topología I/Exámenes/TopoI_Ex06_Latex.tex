\documentclass[12pt]{article}

% Idioma y codificación
\usepackage[spanish, es-tabla]{babel}       %es-tabla para que se titule "Tabla"
\usepackage[utf8]{inputenc}

% Márgenes
\usepackage[a4paper,top=3cm,bottom=2.5cm,left=3cm,right=3cm]{geometry}

% Comentarios de bloque
\usepackage{verbatim}

% Paquetes de links
\usepackage[hidelinks]{hyperref}    % Permite enlaces
\usepackage{url}                    % redirecciona a la web

% Más opciones para enumeraciones
\usepackage{enumitem}

% Personalizar la portada
\usepackage{titling}

% Paquetes de tablas
\usepackage{multirow}


%------------------------------------------------------------------------

%Paquetes de figuras
\usepackage{caption}
\usepackage{subcaption} % Figuras al lado de otras
\usepackage{float}      % Poner figuras en el sitio indicado H.


% Paquetes de imágenes
\usepackage{graphicx}       % Paquete para añadir imágenes
\usepackage{transparent}    % Para manejar la opacidad de las figuras

% Paquete para usar colores
\usepackage[dvipsnames]{xcolor}
\usepackage{pagecolor}      % Para cambiar el color de la página

% Habilita tamaños de fuente mayores
\usepackage{fix-cm}

% Para los gráficos
\usepackage{tikz}

% Para poder situar los nodos en los grafos
\usetikzlibrary{positioning}


%------------------------------------------------------------------------

% Paquetes de matemáticas
\usepackage{mathtools, amsfonts, amssymb, mathrsfs}
\usepackage[makeroom]{cancel}     % Simplificar tachando
\usepackage{polynom}    % Divisiones y Ruffini
\usepackage{units} % Para poner fracciones diagonales con \nicefrac

\usepackage{pgfplots}   %Representar funciones
\pgfplotsset{compat=1.18}  % Versión 1.18

\usepackage{tikz-cd}    % Para usar diagramas de composiciones
\usetikzlibrary{calc}   % Para usar cálculo de coordenadas en tikz

%Definición de teoremas, etc.
\usepackage{amsthm}
%\swapnumbers   % Intercambia la posición del texto y de la numeración

\theoremstyle{plain}

\makeatletter
\@ifclassloaded{article}{
  \newtheorem{teo}{Teorema}[section]
}{
  \newtheorem{teo}{Teorema}[chapter]  % Se resetea en cada chapter
}
\makeatother

\newtheorem{coro}{Corolario}[teo]           % Se resetea en cada teorema
\newtheorem{prop}[teo]{Proposición}         % Usa el mismo contador que teorema
\newtheorem{lema}[teo]{Lema}                % Usa el mismo contador que teorema

\theoremstyle{remark}
\newtheorem*{observacion}{Observación}

\theoremstyle{definition}

\makeatletter
\@ifclassloaded{article}{
  \newtheorem{definicion}{Definición} [section]     % Se resetea en cada chapter
}{
  \newtheorem{definicion}{Definición} [chapter]     % Se resetea en cada chapter
}
\makeatother

\newtheorem*{notacion}{Notación}
\newtheorem*{ejemplo}{Ejemplo}
\newtheorem*{ejercicio*}{Ejercicio}             % No numerado
\newtheorem{ejercicio}{Ejercicio} [section]     % Se resetea en cada section


% Modificar el formato de la numeración del teorema "ejercicio"
\renewcommand{\theejercicio}{%
  \ifnum\value{section}=0 % Si no se ha iniciado ninguna sección
    \arabic{ejercicio}% Solo mostrar el número de ejercicio
  \else
    \thesection.\arabic{ejercicio}% Mostrar número de sección y número de ejercicio
  \fi
}


% \renewcommand\qedsymbol{$\blacksquare$}         % Cambiar símbolo QED
%------------------------------------------------------------------------

% Paquetes para encabezados
\usepackage{fancyhdr}
\pagestyle{fancy}
\fancyhf{}

\newcommand{\helv}{ % Modificación tamaño de letra
\fontfamily{}\fontsize{12}{12}\selectfont}
\setlength{\headheight}{15pt} % Amplía el tamaño del índice


%\usepackage{lastpage}   % Referenciar última pag   \pageref{LastPage}
\fancyfoot[C]{\thepage}

%------------------------------------------------------------------------

% Conseguir que no ponga "Capítulo 1". Sino solo "1."
\makeatletter
\@ifclassloaded{book}{
  \renewcommand{\chaptermark}[1]{\markboth{\thechapter.\ #1}{}} % En el encabezado
    
  \renewcommand{\@makechapterhead}[1]{%
  \vspace*{50\p@}%
  {\parindent \z@ \raggedright \normalfont
    \ifnum \c@secnumdepth >\m@ne
      \huge\bfseries \thechapter.\hspace{1em}\ignorespaces
    \fi
    \interlinepenalty\@M
    \Huge \bfseries #1\par\nobreak
    \vskip 40\p@
  }}
}
\makeatother

%------------------------------------------------------------------------
% Paquetes de cógido
\usepackage{minted}
\renewcommand\listingscaption{Código fuente}

\usepackage{fancyvrb}
% Personaliza el tamaño de los números de línea
\renewcommand{\theFancyVerbLine}{\small\arabic{FancyVerbLine}}

% Estilo para C++
\newminted{cpp}{
    frame=lines,
    framesep=2mm,
    baselinestretch=1.2,
    linenos,
    escapeinside=||
}

% para minted
\definecolor{LightGray}{rgb}{0.95,0.95,0.92}
\setminted{
    linenos=true,
    stepnumber=5,
    numberfirstline=true,
    autogobble,
    breaklines=true,
    breakautoindent=true,
    breaksymbolleft=,
    breaksymbolright=,
    breaksymbolindentleft=0pt,
    breaksymbolindentright=0pt,
    breaksymbolsepleft=0pt,
    breaksymbolsepright=0pt,
    fontsize=\footnotesize,
    bgcolor=LightGray,
    numbersep=10pt
}


\usepackage{listings} % Para incluir código desde un archivo

\renewcommand\lstlistingname{Código Fuente}
\renewcommand\lstlistlistingname{Índice de Códigos Fuente}

% Definir colores
\definecolor{vscodepurple}{rgb}{0.5,0,0.5}
\definecolor{vscodeblue}{rgb}{0,0,0.8}
\definecolor{vscodegreen}{rgb}{0,0.5,0}
\definecolor{vscodegray}{rgb}{0.5,0.5,0.5}
\definecolor{vscodebackground}{rgb}{0.97,0.97,0.97}
\definecolor{vscodelightgray}{rgb}{0.9,0.9,0.9}

% Configuración para el estilo de C similar a VSCode
\lstdefinestyle{vscode_C}{
  backgroundcolor=\color{vscodebackground},
  commentstyle=\color{vscodegreen},
  keywordstyle=\color{vscodeblue},
  numberstyle=\tiny\color{vscodegray},
  stringstyle=\color{vscodepurple},
  basicstyle=\scriptsize\ttfamily,
  breakatwhitespace=false,
  breaklines=true,
  captionpos=b,
  keepspaces=true,
  numbers=left,
  numbersep=5pt,
  showspaces=false,
  showstringspaces=false,
  showtabs=false,
  tabsize=2,
  frame=tb,
  framerule=0pt,
  aboveskip=10pt,
  belowskip=10pt,
  xleftmargin=10pt,
  xrightmargin=10pt,
  framexleftmargin=10pt,
  framexrightmargin=10pt,
  framesep=0pt,
  rulecolor=\color{vscodelightgray},
  backgroundcolor=\color{vscodebackground},
}

%------------------------------------------------------------------------

% Comandos definidos
\newcommand{\bb}[1]{\mathbb{#1}}
\newcommand{\cc}[1]{\mathcal{#1}}

% I prefer the slanted \leq
\let\oldleq\leq % save them in case they're every wanted
\let\oldgeq\geq
\renewcommand{\leq}{\leqslant}
\renewcommand{\geq}{\geqslant}

% Si y solo si
\newcommand{\sii}{\iff}

% Letras griegas
\newcommand{\eps}{\epsilon}
\newcommand{\veps}{\varepsilon}
\newcommand{\lm}{\lambda}

\newcommand{\ol}{\overline}
\newcommand{\ul}{\underline}
\newcommand{\wt}{\widetilde}
\newcommand{\wh}{\widehat}

\let\oldvec\vec
\renewcommand{\vec}{\overrightarrow}

% Derivadas parciales
\newcommand{\del}[2]{\frac{\partial #1}{\partial #2}}
\newcommand{\Del}[3]{\frac{\partial^{#1} #2}{\partial #3^{#1}}}
\newcommand{\deld}[2]{\dfrac{\partial #1}{\partial #2}}
\newcommand{\Deld}[3]{\dfrac{\partial^{#1} #2}{\partial #3^{#1}}}


\newcommand{\AstIg}{\stackrel{(\ast)}{=}}
\newcommand{\Hop}{\stackrel{L'H\hat{o}pital}{=}}

\newcommand{\red}[1]{{\color{red}#1}} % Para integrales, destacar los cambios.

% Método de integración
\newcommand{\MetInt}[2]{
    \left[\begin{array}{c}
        #1 \\ #2
    \end{array}\right]
}

% Declarar aplicaciones
% 1. Nombre aplicación
% 2. Dominio
% 3. Codominio
% 4. Variable
% 5. Imagen de la variable
\newcommand{\Func}[5]{
    \begin{equation*}
        \begin{array}{rrll}
            #1:& #2 & \longrightarrow & #3\\
               & #4 & \longmapsto & #5
        \end{array}
    \end{equation*}
}

%------------------------------------------------------------------------

\newcommand{\T}[0]{\cc{T}}

\begin{document}

    % 1. Foto de fondo
    % 2. Título
    % 3. Encabezado Izquierdo
    % 4. Color de fondo
    % 5. Coord x del titulo
    % 6. Coord y del titulo
    % 7. Fecha

    
    % 1. Foto de fondo
% 2. Título
% 3. Encabezado Izquierdo
% 4. Color de fondo
% 5. Coord x del titulo
% 6. Coord y del titulo
% 7. Fecha

\newcommand{\portada}[7]{

    \portadaBase{#1}{#2}{#3}{#4}{#5}{#6}{#7}
    \portadaBook{#1}{#2}{#3}{#4}{#5}{#6}{#7}
}

\newcommand{\portadaExamen}[7]{

    \portadaBase{#1}{#2}{#3}{#4}{#5}{#6}{#7}
    \portadaArticle{#1}{#2}{#3}{#4}{#5}{#6}{#7}
}




\newcommand{\portadaBase}[7]{

    % Tiene la portada principal y la licencia Creative Commons
    
    % 1. Foto de fondo
    % 2. Título
    % 3. Encabezado Izquierdo
    % 4. Color de fondo
    % 5. Coord x del titulo
    % 6. Coord y del titulo
    % 7. Fecha
    
    
    \thispagestyle{empty}               % Sin encabezado ni pie de página
    \newgeometry{margin=0cm}        % Márgenes nulos para la primera página
    
    
    % Encabezado
    \fancyhead[L]{\helv #3}
    \fancyhead[R]{\helv \nouppercase{\leftmark}}
    
    
    \pagecolor{#4}        % Color de fondo para la portada
    
    \begin{figure}[p]
        \centering
        \transparent{0.3}           % Opacidad del 30% para la imagen
        
        \includegraphics[width=\paperwidth, keepaspectratio]{assets/#1}
    
        \begin{tikzpicture}[remember picture, overlay]
            \node[anchor=north west, text=white, opacity=1, font=\fontsize{60}{90}\selectfont\bfseries\sffamily, align=left] at (#5, #6) {#2};
            
            \node[anchor=south east, text=white, opacity=1, font=\fontsize{12}{18}\selectfont\sffamily, align=right] at (9.7, 3) {\textbf{\href{https://losdeldgiim.github.io/}{Los Del DGIIM}}};
            
            \node[anchor=south east, text=white, opacity=1, font=\fontsize{12}{15}\selectfont\sffamily, align=right] at (9.7, 1.8) {Doble Grado en Ingeniería Informática y Matemáticas\\Universidad de Granada};
        \end{tikzpicture}
    \end{figure}
    
    
    \restoregeometry        % Restaurar márgenes normales para las páginas subsiguientes
    \pagecolor{white}       % Restaurar el color de página
    
    
    \newpage
    \thispagestyle{empty}               % Sin encabezado ni pie de página
    \begin{tikzpicture}[remember picture, overlay]
        \node[anchor=south west, inner sep=3cm] at (current page.south west) {
            \begin{minipage}{0.5\paperwidth}
                \href{https://creativecommons.org/licenses/by-nc-nd/4.0/}{
                    \includegraphics[height=2cm]{assets/Licencia.png}
                }\vspace{1cm}\\
                Esta obra está bajo una
                \href{https://creativecommons.org/licenses/by-nc-nd/4.0/}{
                    Licencia Creative Commons Atribución-NoComercial-SinDerivadas 4.0 Internacional (CC BY-NC-ND 4.0).
                }\\
    
                Eres libre de compartir y redistribuir el contenido de esta obra en cualquier medio o formato, siempre y cuando des el crédito adecuado a los autores originales y no persigas fines comerciales. 
            \end{minipage}
        };
    \end{tikzpicture}
    
    
    
    % 1. Foto de fondo
    % 2. Título
    % 3. Encabezado Izquierdo
    % 4. Color de fondo
    % 5. Coord x del titulo
    % 6. Coord y del titulo
    % 7. Fecha


}


\newcommand{\portadaBook}[7]{

    % 1. Foto de fondo
    % 2. Título
    % 3. Encabezado Izquierdo
    % 4. Color de fondo
    % 5. Coord x del titulo
    % 6. Coord y del titulo
    % 7. Fecha

    % Personaliza el formato del título
    \pretitle{\begin{center}\bfseries\fontsize{42}{56}\selectfont}
    \posttitle{\par\end{center}\vspace{2em}}
    
    % Personaliza el formato del autor
    \preauthor{\begin{center}\Large}
    \postauthor{\par\end{center}\vfill}
    
    % Personaliza el formato de la fecha
    \predate{\begin{center}\huge}
    \postdate{\par\end{center}\vspace{2em}}
    
    \title{#2}
    \author{\href{https://losdeldgiim.github.io/}{Los Del DGIIM}}
    \date{Granada, #7}
    \maketitle
    
    \tableofcontents
}




\newcommand{\portadaArticle}[7]{

    % 1. Foto de fondo
    % 2. Título
    % 3. Encabezado Izquierdo
    % 4. Color de fondo
    % 5. Coord x del titulo
    % 6. Coord y del titulo
    % 7. Fecha

    % Personaliza el formato del título
    \pretitle{\begin{center}\bfseries\fontsize{42}{56}\selectfont}
    \posttitle{\par\end{center}\vspace{2em}}
    
    % Personaliza el formato del autor
    \preauthor{\begin{center}\Large}
    \postauthor{\par\end{center}\vspace{3em}}
    
    % Personaliza el formato de la fecha
    \predate{\begin{center}\huge}
    \postdate{\par\end{center}\vspace{5em}}
    
    \title{#2}
    \author{\href{https://losdeldgiim.github.io/}{Los Del DGIIM}}
    \date{Granada, #7}
    \thispagestyle{empty}               % Sin encabezado ni pie de página
    \maketitle
    \vfill
}
    \portadaExamen{ffccA4.jpg}{Topología I\\Examen VI}{Topología I. Examen VI}{MidnightBlue}{-8}{28}{2023-2024}{Arturo Olivares Martos}

    \begin{description}
        \item[Asignatura] Topología I.
        \item[Curso Académico] 2023-24.
        \item[Grado] Grado en Matemáticas.
        \item[Grupo] A.
        \item[Profesor] Leonor Ferrer Martínez.
        \item[Descripción] Segundo Parcial.
        \item[Fecha] 5 de diciembre de 2023.
        %\item[Duración] 3 horas.
    
    \end{description}
    \newpage
    
    \begin{ejercicio}[4 puntos] Consideramos en $\bb{R}$ la topología del punto incluido para el $0$, $\T_0=\{U\subseteq\bb{R}\mid 0\in U\}\cup \{\emptyset\}$ y
        $\T_S$ la topología de Sorgenfrey.
        \begin{enumerate}
            \item Estudia en qué puntos es continua la aplicación $f:(\bb{R},\T_0)\to(\bb{R},\T_S)$ definida por $f(x)=x^2$.\\
            
            Dado $x\in \bb{R}$, una base de entornos de $x$ en $\T_S$ es:
            \begin{equation*}
                \beta_x^S = \{[x,x+\varepsilon[\mid \varepsilon>0\}
            \end{equation*}

            Por otro lado, una base de entornos de $x$ en $\T_0$ es:
            \begin{equation*}
                \beta_x^0 = \{\{x,0\}\}
            \end{equation*}

            Por tanto, $f$ es continua en $x$ si y solo si para todo $V\in \beta_{f(x)}^S$ existe un $U\in \beta_x^0$ tal que $f(U)\subseteq V$.
            Como la única posibilidad es que $U=\{x,0\}$,
            tenemos que $f$ es continua en $x$ si y solo si para cualquier $V\in \beta_{f(x)}^S$ se tiene que $f(x), f(0)\in V$.

            \begin{itemize}
                \item Veamos que es continua en $x=0$:
                
                Sea $V\in \beta_{f(0)}^S$, entonces $V=[0,\varepsilon[$ para algún $\varepsilon>0$.
                Como $f(0)=f(x)=0$, tenemos que $f(x),f(0)\in V$ y por tanto $f$ es continua en $x=0$.
                
                \item Veamos que no es continua en $x\neq 0$:
                
                Sea $V\in \beta_{f(x)}^S$, entonces $V=[x^2,x^2+\varepsilon[$ para algún $\varepsilon>0$.
                Como $x\neq 0$, entonces $f(x)=x^2> 0$. Por tanto, $f(0)=0\notin V$ y por tanto $f$ no es continua en $x\neq 0$.
            \end{itemize}
            

            \item Calcula la clausura de $A=[0,1]\times [0,1]$ y el interior del conjunto dado por $B=\{(x,y)\in\bb{R}^2\mid x+y\geq 0\}$ en 
            el espacio topológico $(\bb{R}^2,\T_0\times \T_S)$.\\

            Sabemos que $\ol{A}=\ol{[0,1]}\times \ol{[0,1]}$. En $(\bb{R},\T_0)$, tenemos que $C_{\T_0}=\{C\subseteq \bb{R}\mid 0\notin C\}\cup \{\bb{R}\}$.
            Por tanto, como $\ol{[0,1]}\in C_{\T_0}$, tenemos que:
            \begin{equation*}
                0\in[0,1]\subset  \ol{[0,1]} \in C_{\T_0} \Longrightarrow \ol{[0,1]} = \bb{R}
            \end{equation*}

            Calculamos ahora $\ol{[0,1]}$ en $(\bb{R},\T_S)$. Como $C_{\T_u}\subset C_{\T_S}$ y $[0,1]\in C_{\T_u}$, tenemos que $[0,1]\in C_{\T_S}$,
            por lo que $\ol{[0,1]}=[0,1]$. Por tanto,
            \begin{equation*}
                \ol{A}=\ol{[0,1]}\times \ol{[0,1]} = \bb{R} \times [0,1]
            \end{equation*}\\

            Calculamos ahora el interior de $B$. Veamos la forma de $B$:
            \begin{figure}[H]
                \centering
                \begin{tikzpicture}
                    \def\a{3}
                    \def\b{3}
                    \def\neg{2}

                    % Conjunto B
                    \filldraw[blue!30] (-\neg,\b) -- (-\neg,\neg) -- (\neg,-\neg) -- (\a,-\neg) -- (\a,\b) -- (-\neg,\b);
                    
                    % Límite de B
                    \draw[thick] (-\neg,\neg) -- (\neg,-\neg);


                    % Ejes de coordenadas
                    \draw[-stealth, dashed] (-\neg,0) -- (\a,0) node[right] {$x$};
                    \draw[-stealth, dashed] (0,-\neg) -- (0,\b) node[above] {$y$};
                \end{tikzpicture}
                \caption{Conjunto $B=\{(x,y)\in\bb{R}^2\mid x+y\geq 0\}$.}
            \end{figure}
            

            Los abiertos básicos de $\T_S$ son de la forma $[a,b[$, con $a<b$,
            y los abiertos básicos de $\T_0$ son de la forma $\{x,0\}$, con $x\in \bb{R}$.
            Por tanto, una base de $\T_0\times \T_S$ es:
            \begin{align*}
                \cc{B} &= \{\{x,0\} \times [a,b[ ~\mid a<b,~ x\in \bb{R}\}
                =\\&= \{\{(0,y), (x,y)\} ~\mid x\in \bb{R},~y\in [a,b[\}
            \end{align*}

            Intuitivamente, vemos que si $y\leq 0$, entonces el punto $(0,y)$ no está en $B$, y por tanto no puede estar en su interior.
            Por tanto, demostremos que $B^{\circ} = \wt{B}$, con:
            \begin{equation*}
                \wt{B} = \{(x,y)\in B\mid y\geq 0\}
            \end{equation*}
            \begin{description}
                \item[$\supset)$] Sea $(x,y)\in \wt{B}\subset B$, y buscamos ver que $(x,y)\in B^{\circ}$.
                Veamos que $\exists U\in \cc{B}$ tal que $(x,y)\in U\subset B$.

                Tenemos que $\{x,0\}\times [y,y+1[\in \cc{B}$, y veamos que $(x,y)\in \{x,0\}\times [y,y+1[~\subset B$.
                Es evidente que $(x,y)\in \{x,0\}\times [y,y+1[$. Veamos que $\{x,0\}\times [y,y+1[\subset B$.
                \begin{description}
                    \item[$\subset)$] Tenemos que $\{x,0\}\times [y,y+1[~= \{x\}\times [y,y+1[~\cup~ \{0\}\times [y,y+1[$.
                    Tenemos en cuenta que $0+y'\geq 0$ para cualquier $y'\in [y,y+1[$, por lo que $\{0\}\times [y,y+1[\subset B$.
                    Por otro lado, $x+y'\geq x+y \geq 0$ para cualquier $y'\in [y,y+1[$, por lo que $\{x\}\times [y,y+1[\subset B$.

                    Por tanto, $\{x,0\}\times [y,y+1[~= \{x\}\times [y,y+1[~\cup~ \{0\}\times [y,y+1[\subset B$.
                \end{description}

                Por tanto, tenemos que $(x,y)\in B^{\circ}$.

                \item[$\subset)$] En este caso, no probaremos directamente la inclusión. Como $B^\circ,\wt{B} \subset B$, probaremos
                que dado $(x,y)\in B$, si $(x,y)\notin \wt{B}$, entonces $(x,y)\notin B^\circ$. Es decir, probamos el recíproco.
                
                Sea $(x,y)\in B\setminus \wt{B}$, por lo que $(x,y)\in B$ e $y<0$. Veamos que $(x,y)\notin B^\circ$.
                Para ver esto, por reducción al absurdo supongamos que $\exists U\in \cc{B}$ tal que $(x,y)\in U\subset B$.
                Entonces, $U=\{x,0\}\times [a,b[$, para algún $a<b$, $y\in [a,b[$. Por tanto, $\{0\}\times [a,b[\subset U\subset B$, con $y\in [a,b[$.
                Por tanto, $(0,y)\in B$, por lo que $0+y=y\geq 0$, lo cual es una contradicción, ya que $y<0$.

                Por tanto, $(x,y)\notin B^\circ$.
            \end{description}



        \end{enumerate}
        
    \end{ejercicio}


    \begin{ejercicio}[3 puntos]
        En $X=[-2,2]\times \{-1,0,1\}\subset \bb{R}^2$ se considera la relación de equivalencia
        \begin{equation*}
            (t,s)\cc{R}(t',s')\iff \left\{
                \begin{array}{c}
                    (t,s)=(t',s')\\
                    \lor \\
                    t,t'\leq -1 \\
                    \lor \\
                    t,t'\geq 1
                \end{array}
            \right.
        \end{equation*}
        \begin{enumerate}
            \item Prueba que $(X/\cc{R},{\T_u}_{|X} / {\cc{R}})$ es homeomorfo a $\left(\bb{S}^1 \cup ([-1,1]\times \{0\}), {\T_u}_{|\bb{S}^1 \cup ([-1,1]\times \{0\})}\right)$.
            
            Veamos qué puntos identifica la relación de equivalencia:
            \begin{figure}[H]
                \centering
                \begin{tikzpicture}
                    % Ejes de coordenadas
                    \draw[-stealth, dashed] (-3,0) -- (3,0) node[right] {$t$};
                    \draw[-stealth, dashed] (0,-2) -- (0,2) node[above] {$s$};

                    % Conjunto X
                    \draw[thick] (-2,-1) -- (2,-1);
                    \draw[thick] (-2,0) -- (2,0);
                    \draw[thick] (-2,1) -- (2,1);

                    % t \leq -1
                    \draw[ultra thick, red] (-2,-1) -- (-1,-1);
                    \draw[ultra thick, red] (-2,0) -- (-1,0);
                    \draw[ultra thick, red] (-2,1) -- (-1,1);

                    % t \geq 1
                    \draw[ultra thick, blue] (2,-1) -- (1,-1);
                    \draw[ultra thick, blue] (2,0) -- (1,0);
                    \draw[ultra thick, blue] (2,1) -- (1,1);


                \end{tikzpicture}
            \end{figure}


            Nos definimos entonces la siguiente aplicación:
            \Func{f}{X}{\bb{S}^1 \cup ([-1,1]\times \{0\})}{(t,s)}{\left\{
                \begin{array}{ll}
                    (-1,0) & \text{si } t\leq -1 \\
                    \left(t,s\sqrt{1-t^2}\right) & \text{si } t\in [-1,1] \\
                    (1,0) & \text{si } t\geq 1
                \end{array}
            \right.}

            Para demostrar su continuidad, buscamos aplicar el lema de pegado. Tenemos que $X$ se puede expresar como la unión de los siguientes conjuntos:
            \begin{equation*}
                    \begin{array}{l}
                        X_1 = \{(t,s)\in X\mid t\leq -1\} \\
                        X_2 = \{(t,s)\in X\mid t\in [-1,1]\} \\
                        X_3 = \{(t,s)\in X\mid t\geq 1\}
                    \end{array}
            \end{equation*}

            Tenemos que $X=X_1\cup X_2\cup X_3$. Además, $X_i\in C_{\T_u}$, ya que $X_i$ es la imagen inversa mediante la proyección en la primera coordenada (continua) de un cerrado de $\bb{R}$.
            Además, podemos escribir $f$ como:
            \begin{equation*}
                f(t,s) = \left\{
                    \begin{array}{ll}
                        f_1(t,s) & \text{si } (t,s)\in X_1 \\
                        f_2(t,s) & \text{si } (t,s)\in X_2 \\
                        f_3(t,s) & \text{si } (t,s)\in X_3
                    \end{array}
                \right.
            \end{equation*}
            donde $f_1(t,s)=(-1,0)$, $f_2(t,s)=(t,s\sqrt{1-t^2})$ y $f_3(t,s)=(1,0)$.
            Tenemos que $f_1,f_3$ son claramente continuas por ser constantes.
            Además, $f_2$ es continua, ya que ambas componentes lo son. La segunda componente es continua por
            ser producto de funciones continuas. La raíz es continua por ser composición de
            una polinómica que toma valores en $[0,1]$ y la función raíz que es continua.

            Veamos ahora $f_1=f_2$ en $X_1\cap X_2$. Tenemos que $f_1(-1,s)=(-1,0)$ y $f_2(-1,s)=(-1,-\sqrt{1-1})=(-1,0)$.
            Además, veamos que $f_2=f_3$ en $X_2\cap X_3$. Tenemos que $f_2(1,s)=(1,s\sqrt{1-1})=(1,0)$ y $f_3(1,s)=(1,0)$.

            Por tanto, por el lema de pegado, $f$ es continua. Veamos ahora que $f$ es sobreyectiva. Sea $(t,s)\in \bb{S}^1 \cup ([-1,1]\times \{0\})$.
            \begin{itemize}
                \item Si $(t,s)\in \bb{S}^1$, con $s>0$, entonces $f(t,1) = (t, \sqrt{1-t^2}) = (t,s)$, donde he empleado que $t^2+s^2=1$.
                \item Si $(t,s)\in \bb{S}^1$, con $s<0$, entonces $f(t,-1) = (t, -\sqrt{1-t^2}) = (t,s)$, donde he empleado que $t^2+s^2=1$, por lo que $s=-\sqrt{1-t^2}$.
                \item Si $(t,s)\in [-1,1]\times \{0\}$, entonces $f(t,0) = (t, 0) = (t,s)$.
            \end{itemize}

            Por tanto, $f$ es sobreyectiva. Veamos ahora que $f$ es cerrada. Como claramente $[-2,2]\in C_{\T_u}$ y $\{-1,0,1\}$ es la unión de tres cerrados de $\T_u$, tenemos que $X$
            es producto de dos cerrados, por lo que es cerrado en la topología producto $\bb{R}^3$. Además, claramente $X$ es acotado, ya que $X\subset B(0, 15)$ (por ejemplo).
            Por tanto, como $X\subset \bb{R}^3$ es cerrado y acotado, entonces $f$ es cerrada.
            
            Como $f$ es continua, sobreyectiva y cerrada, entonces $f$ es una identificación. Veamos que $\cc{R}=\cc{R}_f$. Sea $(t,s), (t',s')\in X$. Entonces,
            \begin{equation*}
                (t,s)\cc{R} (t',s') \iff f(t,s)=f(t',s')
            \end{equation*}
            \begin{description}
                \item[$\Longrightarrow)$] Supongamos que $(t,s)\cc{R} (t',s')$.
                \begin{itemize}
                    \item Si $(t,s)=(t',s')$, entonces $f(t,s)=f(t',s')$ por ser $f$ una aplicación.
                    \item Si $t,t'\leq -1$, entonces $f(t,s)=f(t',s')=(-1,0)$.
                    \item Si $t,t'\geq 1$, entonces $f(t,s)=f(t',s')=(1,0)$.
                \end{itemize}

                \item[$\Longleftarrow)$] Supongamos que $f(t,s)=f(t',s')$.
                \begin{itemize}
                    \item Supongamos $t\leq -1$.
                    
                    Entonces $f(t,s)=(-1,0)=f(t',s')$, y por tanto $t'\leq -1$. Por tanto, $(t,s)\cc{R}(t',s')$.

                    \item Supongamos $t\geq 1$.
                    
                    Entonces $f(t,s)=(1,0)=f(t',s')$, y por tanto $t'\geq 1$. Por tanto, $(t,s)\cc{R}(t',s')$.

                    \item Supongamos $t\in ]-1,1[$.
                    
                    Como $(t,s\sqrt{1-t^2})=(t',s'\sqrt{1-{t'}^2})$, tenemos de forma directa que $t=t'$, y por tanto $s\sqrt{1-t^2}=s'\sqrt{1-t^2}$,
                    y como $t\neq -1,1$, entonces $s=s'$. Por tanto, $(t,s)=(t',s')$, por lo que $(t,s)\cc{R}(t',s')$.
                \end{itemize}
            \end{description}

            Por tanto, $f$ induce un homeomorfismo entre $(X/\cc{R},{\T_u}_{|X} / {\cc{R}})$ y el conjunto $\left(\bb{S}^1 \cup ([-1,1]\times \{0\}), {\T_u}_{|\bb{S}^1 \cup ([-1,1]\times \{0\})}\right)$.


            

            \item Prueba que la proyección $p:(X,{\T_u}_{|X})\to (X/\cc{R},{\T_u}_{|X} / {\cc{R}})$ es cerrada.
            
            Para ello, dado $C\in C_{\T_{|X}}$, tenemos que ver que $p(C)\in C_{\T_{|X}/\cc{R}}$.
            Para ello, habrá que ver que $p^{-1}(p(C))\in C_{\T_{|X}}$.
            \begin{enumerate}
                \item Si $(C\cap X_1)\cup (C\cap X_3)=\emptyset$, entonces $$p{-1}(p(C))=C$$
                \item Si $(C\cap X_1)\neq \emptyset,~ (C\cap X_3)= \emptyset$, entonces $$p{-1}(p(C))=p^{-1}((-1,0)\cup C) = X_1 \cup C$$
                \item Si $(C\cap X_1)= \emptyset,~ (C\cap X_3)\neq \emptyset$, entonces $$p{-1}(p(C))=p^{-1}((1,0)\cup C) = X_3 \cup C$$
                \item Si $(C\cap X_1)\neq \emptyset,~ (C\cap X_3)\neq \emptyset$, entonces $$p{-1}(p(C))=p^{-1}((1,0)\cup (-1,0)\cup C) = X_1 \cup X_3 \cup C$$
            \end{enumerate}
            En cualquier caso, $p^{-1}(p(C))\in C_{\T_{|X}}$ por ser unión finita de cerrados de $\T_{|X}$, por lo que $p$ es cerrada.
        \end{enumerate}
    \end{ejercicio}

    \begin{ejercicio}[3 puntos]
        Sean $(X,\T)$, $(Y_1,\T_1)$, $(Y_2,\T_2)$ espacios topológicos y las aplicaciones continuas $f_i:(X,\T)\to (Y_i,\T_i)$ para $i=1,2$.
        Se dice que la familia $\cc{F}=\{f_1,f_2\}$ \emph{separa puntos de cerrados} si para cada punto $x\in X$ y cada $C$ cerrado de $\T$ tal que $x\notin C$,
        existe un $i\in \{1,2\}$ tal que $f_i(x)\notin f_i(C)$. Consideremos la aplicación $f:(X,\T)\to (Y_1\times Y_2,\T_1\times \T_2)$ definida por
        $f(x)=(f_1(x),f_2(x))$. Prueba que:
        \begin{enumerate}
            \item Si $\cc{F}$ separa puntos de cerrados, entonces $f:(X,\T)\to \left(f(X),(\T_1\times \T_2)_{f(X)}\right)$ es abierta.
            
            Sea $U\in \T$. Tenemos que ver que $f(U)=f_1(U)\times f_2(U)\in (\T_1\times \T_2)_{f(X)}$.
            % // TODO:  F separa puntos de cerrados

            \item Si $\cc{F}$ separa puntos de cerrados y $(X,\T)$ es T1, entonces la aplicación dada por $f:(X,\T) \to (Y_1\times Y_2,\T_1\times \T_2)$ es un embebimiento.
            
            Como cada una de sus componentes $(f_1,f_2)$ son continuas, entonces sabemos que $f$ es continua. Veamos que $f$ es inyectiva. Sean $x,x'\in X$, con $x\neq x'$.
            Entonces, $x\notin \{x'\}$, y como $(X,\T)$ es T1, entonces $\{x'\}$ es cerrado. Por tanto, como $\cc{F}$ separa puntos de cerrados, existe un $i\in \{1,2\}$ tal que
            $f_i(x)\notin f_i(\{x'\})$, y por tanto $f_i(x)\neq f_i(x')$. Por tanto, $f(x)\neq f(x')$, y por tanto es inyectiva.
            
            Además, por el apartado anterior, $f$ es abierta, por lo que $f$ es un homeomorfismo sobre su imagen, es decir, $f$ es un embebimiento.
        \end{enumerate}
    \end{ejercicio}
    

\end{document}