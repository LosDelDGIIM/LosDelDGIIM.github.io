\documentclass[12pt]{article}

% Idioma y codificación
\usepackage[spanish, es-tabla]{babel}       %es-tabla para que se titule "Tabla"
\usepackage[utf8]{inputenc}

% Márgenes
\usepackage[a4paper,top=3cm,bottom=2.5cm,left=3cm,right=3cm]{geometry}

% Comentarios de bloque
\usepackage{verbatim}

% Paquetes de links
\usepackage[hidelinks]{hyperref}    % Permite enlaces
\usepackage{url}                    % redirecciona a la web

% Más opciones para enumeraciones
\usepackage{enumitem}

% Personalizar la portada
\usepackage{titling}

% Paquetes de tablas
\usepackage{multirow}


%------------------------------------------------------------------------

%Paquetes de figuras
\usepackage{caption}
\usepackage{subcaption} % Figuras al lado de otras
\usepackage{float}      % Poner figuras en el sitio indicado H.


% Paquetes de imágenes
\usepackage{graphicx}       % Paquete para añadir imágenes
\usepackage{transparent}    % Para manejar la opacidad de las figuras

% Paquete para usar colores
\usepackage[dvipsnames]{xcolor}
\usepackage{pagecolor}      % Para cambiar el color de la página

% Habilita tamaños de fuente mayores
\usepackage{fix-cm}

% Para los gráficos
\usepackage{tikz}

% Para poder situar los nodos en los grafos
\usetikzlibrary{positioning}


%------------------------------------------------------------------------

% Paquetes de matemáticas
\usepackage{mathtools, amsfonts, amssymb, mathrsfs}
\usepackage[makeroom]{cancel}     % Simplificar tachando
\usepackage{polynom}    % Divisiones y Ruffini
\usepackage{units} % Para poner fracciones diagonales con \nicefrac

\usepackage{pgfplots}   %Representar funciones
\pgfplotsset{compat=1.18}  % Versión 1.18

\usepackage{tikz-cd}    % Para usar diagramas de composiciones
\usetikzlibrary{calc}   % Para usar cálculo de coordenadas en tikz

%Definición de teoremas, etc.
\usepackage{amsthm}
%\swapnumbers   % Intercambia la posición del texto y de la numeración

\theoremstyle{plain}

\makeatletter
\@ifclassloaded{article}{
  \newtheorem{teo}{Teorema}[section]
}{
  \newtheorem{teo}{Teorema}[chapter]  % Se resetea en cada chapter
}
\makeatother

\newtheorem{coro}{Corolario}[teo]           % Se resetea en cada teorema
\newtheorem{prop}[teo]{Proposición}         % Usa el mismo contador que teorema
\newtheorem{lema}[teo]{Lema}                % Usa el mismo contador que teorema

\theoremstyle{remark}
\newtheorem*{observacion}{Observación}

\theoremstyle{definition}

\makeatletter
\@ifclassloaded{article}{
  \newtheorem{definicion}{Definición} [section]     % Se resetea en cada chapter
}{
  \newtheorem{definicion}{Definición} [chapter]     % Se resetea en cada chapter
}
\makeatother

\newtheorem*{notacion}{Notación}
\newtheorem*{ejemplo}{Ejemplo}
\newtheorem*{ejercicio*}{Ejercicio}             % No numerado
\newtheorem{ejercicio}{Ejercicio} [section]     % Se resetea en cada section


% Modificar el formato de la numeración del teorema "ejercicio"
\renewcommand{\theejercicio}{%
  \ifnum\value{section}=0 % Si no se ha iniciado ninguna sección
    \arabic{ejercicio}% Solo mostrar el número de ejercicio
  \else
    \thesection.\arabic{ejercicio}% Mostrar número de sección y número de ejercicio
  \fi
}


% \renewcommand\qedsymbol{$\blacksquare$}         % Cambiar símbolo QED
%------------------------------------------------------------------------

% Paquetes para encabezados
\usepackage{fancyhdr}
\pagestyle{fancy}
\fancyhf{}

\newcommand{\helv}{ % Modificación tamaño de letra
\fontfamily{}\fontsize{12}{12}\selectfont}
\setlength{\headheight}{15pt} % Amplía el tamaño del índice


%\usepackage{lastpage}   % Referenciar última pag   \pageref{LastPage}
\fancyfoot[C]{\thepage}

%------------------------------------------------------------------------

% Conseguir que no ponga "Capítulo 1". Sino solo "1."
\makeatletter
\@ifclassloaded{book}{
  \renewcommand{\chaptermark}[1]{\markboth{\thechapter.\ #1}{}} % En el encabezado
    
  \renewcommand{\@makechapterhead}[1]{%
  \vspace*{50\p@}%
  {\parindent \z@ \raggedright \normalfont
    \ifnum \c@secnumdepth >\m@ne
      \huge\bfseries \thechapter.\hspace{1em}\ignorespaces
    \fi
    \interlinepenalty\@M
    \Huge \bfseries #1\par\nobreak
    \vskip 40\p@
  }}
}
\makeatother

%------------------------------------------------------------------------
% Paquetes de cógido
\usepackage{minted}
\renewcommand\listingscaption{Código fuente}

\usepackage{fancyvrb}
% Personaliza el tamaño de los números de línea
\renewcommand{\theFancyVerbLine}{\small\arabic{FancyVerbLine}}

% Estilo para C++
\newminted{cpp}{
    frame=lines,
    framesep=2mm,
    baselinestretch=1.2,
    linenos,
    escapeinside=||
}

% para minted
\definecolor{LightGray}{rgb}{0.95,0.95,0.92}
\setminted{
    linenos=true,
    stepnumber=5,
    numberfirstline=true,
    autogobble,
    breaklines=true,
    breakautoindent=true,
    breaksymbolleft=,
    breaksymbolright=,
    breaksymbolindentleft=0pt,
    breaksymbolindentright=0pt,
    breaksymbolsepleft=0pt,
    breaksymbolsepright=0pt,
    fontsize=\footnotesize,
    bgcolor=LightGray,
    numbersep=10pt
}


\usepackage{listings} % Para incluir código desde un archivo

\renewcommand\lstlistingname{Código Fuente}
\renewcommand\lstlistlistingname{Índice de Códigos Fuente}

% Definir colores
\definecolor{vscodepurple}{rgb}{0.5,0,0.5}
\definecolor{vscodeblue}{rgb}{0,0,0.8}
\definecolor{vscodegreen}{rgb}{0,0.5,0}
\definecolor{vscodegray}{rgb}{0.5,0.5,0.5}
\definecolor{vscodebackground}{rgb}{0.97,0.97,0.97}
\definecolor{vscodelightgray}{rgb}{0.9,0.9,0.9}

% Configuración para el estilo de C similar a VSCode
\lstdefinestyle{vscode_C}{
  backgroundcolor=\color{vscodebackground},
  commentstyle=\color{vscodegreen},
  keywordstyle=\color{vscodeblue},
  numberstyle=\tiny\color{vscodegray},
  stringstyle=\color{vscodepurple},
  basicstyle=\scriptsize\ttfamily,
  breakatwhitespace=false,
  breaklines=true,
  captionpos=b,
  keepspaces=true,
  numbers=left,
  numbersep=5pt,
  showspaces=false,
  showstringspaces=false,
  showtabs=false,
  tabsize=2,
  frame=tb,
  framerule=0pt,
  aboveskip=10pt,
  belowskip=10pt,
  xleftmargin=10pt,
  xrightmargin=10pt,
  framexleftmargin=10pt,
  framexrightmargin=10pt,
  framesep=0pt,
  rulecolor=\color{vscodelightgray},
  backgroundcolor=\color{vscodebackground},
}

%------------------------------------------------------------------------

% Comandos definidos
\newcommand{\bb}[1]{\mathbb{#1}}
\newcommand{\cc}[1]{\mathcal{#1}}

% I prefer the slanted \leq
\let\oldleq\leq % save them in case they're every wanted
\let\oldgeq\geq
\renewcommand{\leq}{\leqslant}
\renewcommand{\geq}{\geqslant}

% Si y solo si
\newcommand{\sii}{\iff}

% Letras griegas
\newcommand{\eps}{\epsilon}
\newcommand{\veps}{\varepsilon}
\newcommand{\lm}{\lambda}

\newcommand{\ol}{\overline}
\newcommand{\ul}{\underline}
\newcommand{\wt}{\widetilde}
\newcommand{\wh}{\widehat}

\let\oldvec\vec
\renewcommand{\vec}{\overrightarrow}

% Derivadas parciales
\newcommand{\del}[2]{\frac{\partial #1}{\partial #2}}
\newcommand{\Del}[3]{\frac{\partial^{#1} #2}{\partial #3^{#1}}}
\newcommand{\deld}[2]{\dfrac{\partial #1}{\partial #2}}
\newcommand{\Deld}[3]{\dfrac{\partial^{#1} #2}{\partial #3^{#1}}}


\newcommand{\AstIg}{\stackrel{(\ast)}{=}}
\newcommand{\Hop}{\stackrel{L'H\hat{o}pital}{=}}

\newcommand{\red}[1]{{\color{red}#1}} % Para integrales, destacar los cambios.

% Método de integración
\newcommand{\MetInt}[2]{
    \left[\begin{array}{c}
        #1 \\ #2
    \end{array}\right]
}

% Declarar aplicaciones
% 1. Nombre aplicación
% 2. Dominio
% 3. Codominio
% 4. Variable
% 5. Imagen de la variable
\newcommand{\Func}[5]{
    \begin{equation*}
        \begin{array}{rrll}
            #1:& #2 & \longrightarrow & #3\\
               & #4 & \longmapsto & #5
        \end{array}
    \end{equation*}
}

%------------------------------------------------------------------------

\newcommand{\T}[0]{\cc{T}}

% \newcounter{ejercicio}[section] % Define el contador de ejercicio y lo reinicia con cada sección

% \newcounter{ejercicio}
% \newcommand{\resetearcontador}{%
%   \setcounter{ejercicio}{0} % Resetea el contador de ejercicios a 0
% }

\renewcommand{\theejercicio}{\arabic{ejercicio}}


\begin{document}

    % 1. Foto de fondo
    % 2. Título
    % 3. Encabezado Izquierdo
    % 4. Color de fondo
    % 5. Coord x del titulo
    % 6. Coord y del titulo
    % 7. Fecha

    
    % 1. Foto de fondo
% 2. Título
% 3. Encabezado Izquierdo
% 4. Color de fondo
% 5. Coord x del titulo
% 6. Coord y del titulo
% 7. Fecha

\newcommand{\portada}[7]{

    \portadaBase{#1}{#2}{#3}{#4}{#5}{#6}{#7}
    \portadaBook{#1}{#2}{#3}{#4}{#5}{#6}{#7}
}

\newcommand{\portadaExamen}[7]{

    \portadaBase{#1}{#2}{#3}{#4}{#5}{#6}{#7}
    \portadaArticle{#1}{#2}{#3}{#4}{#5}{#6}{#7}
}




\newcommand{\portadaBase}[7]{

    % Tiene la portada principal y la licencia Creative Commons
    
    % 1. Foto de fondo
    % 2. Título
    % 3. Encabezado Izquierdo
    % 4. Color de fondo
    % 5. Coord x del titulo
    % 6. Coord y del titulo
    % 7. Fecha
    
    
    \thispagestyle{empty}               % Sin encabezado ni pie de página
    \newgeometry{margin=0cm}        % Márgenes nulos para la primera página
    
    
    % Encabezado
    \fancyhead[L]{\helv #3}
    \fancyhead[R]{\helv \nouppercase{\leftmark}}
    
    
    \pagecolor{#4}        % Color de fondo para la portada
    
    \begin{figure}[p]
        \centering
        \transparent{0.3}           % Opacidad del 30% para la imagen
        
        \includegraphics[width=\paperwidth, keepaspectratio]{assets/#1}
    
        \begin{tikzpicture}[remember picture, overlay]
            \node[anchor=north west, text=white, opacity=1, font=\fontsize{60}{90}\selectfont\bfseries\sffamily, align=left] at (#5, #6) {#2};
            
            \node[anchor=south east, text=white, opacity=1, font=\fontsize{12}{18}\selectfont\sffamily, align=right] at (9.7, 3) {\textbf{\href{https://losdeldgiim.github.io/}{Los Del DGIIM}}};
            
            \node[anchor=south east, text=white, opacity=1, font=\fontsize{12}{15}\selectfont\sffamily, align=right] at (9.7, 1.8) {Doble Grado en Ingeniería Informática y Matemáticas\\Universidad de Granada};
        \end{tikzpicture}
    \end{figure}
    
    
    \restoregeometry        % Restaurar márgenes normales para las páginas subsiguientes
    \pagecolor{white}       % Restaurar el color de página
    
    
    \newpage
    \thispagestyle{empty}               % Sin encabezado ni pie de página
    \begin{tikzpicture}[remember picture, overlay]
        \node[anchor=south west, inner sep=3cm] at (current page.south west) {
            \begin{minipage}{0.5\paperwidth}
                \href{https://creativecommons.org/licenses/by-nc-nd/4.0/}{
                    \includegraphics[height=2cm]{assets/Licencia.png}
                }\vspace{1cm}\\
                Esta obra está bajo una
                \href{https://creativecommons.org/licenses/by-nc-nd/4.0/}{
                    Licencia Creative Commons Atribución-NoComercial-SinDerivadas 4.0 Internacional (CC BY-NC-ND 4.0).
                }\\
    
                Eres libre de compartir y redistribuir el contenido de esta obra en cualquier medio o formato, siempre y cuando des el crédito adecuado a los autores originales y no persigas fines comerciales. 
            \end{minipage}
        };
    \end{tikzpicture}
    
    
    
    % 1. Foto de fondo
    % 2. Título
    % 3. Encabezado Izquierdo
    % 4. Color de fondo
    % 5. Coord x del titulo
    % 6. Coord y del titulo
    % 7. Fecha


}


\newcommand{\portadaBook}[7]{

    % 1. Foto de fondo
    % 2. Título
    % 3. Encabezado Izquierdo
    % 4. Color de fondo
    % 5. Coord x del titulo
    % 6. Coord y del titulo
    % 7. Fecha

    % Personaliza el formato del título
    \pretitle{\begin{center}\bfseries\fontsize{42}{56}\selectfont}
    \posttitle{\par\end{center}\vspace{2em}}
    
    % Personaliza el formato del autor
    \preauthor{\begin{center}\Large}
    \postauthor{\par\end{center}\vfill}
    
    % Personaliza el formato de la fecha
    \predate{\begin{center}\huge}
    \postdate{\par\end{center}\vspace{2em}}
    
    \title{#2}
    \author{\href{https://losdeldgiim.github.io/}{Los Del DGIIM}}
    \date{Granada, #7}
    \maketitle
    
    \tableofcontents
}




\newcommand{\portadaArticle}[7]{

    % 1. Foto de fondo
    % 2. Título
    % 3. Encabezado Izquierdo
    % 4. Color de fondo
    % 5. Coord x del titulo
    % 6. Coord y del titulo
    % 7. Fecha

    % Personaliza el formato del título
    \pretitle{\begin{center}\bfseries\fontsize{42}{56}\selectfont}
    \posttitle{\par\end{center}\vspace{2em}}
    
    % Personaliza el formato del autor
    \preauthor{\begin{center}\Large}
    \postauthor{\par\end{center}\vspace{3em}}
    
    % Personaliza el formato de la fecha
    \predate{\begin{center}\huge}
    \postdate{\par\end{center}\vspace{5em}}
    
    \title{#2}
    \author{\href{https://losdeldgiim.github.io/}{Los Del DGIIM}}
    \date{Granada, #7}
    \thispagestyle{empty}               % Sin encabezado ni pie de página
    \maketitle
    \vfill
}
    \portadaExamen{ffccA4.jpg}{Topología I\\Examen XI}{Topología I. Examen XI}{MidnightBlue}{-8}{28}{2024-2025}{Jesús Muñoz Velasco}

    \begin{description}
        \item[Asignatura] Topología I.
        \item[Curso Académico] 2024-25.
        \item[Grado] Doble Grado en Ingeniería Informática y Matemáticas.
        \item[Grupo] Único.
        \item[Profesor] Antonio Alarcón.
        \item[Descripción] Segundo Parcial.
        \item[Fecha] 13 de diciembre de 2024.
        \item[Duración] 90 minutos.
    
    \end{description}
    \newpage

    \begin{ejercicio}[3 puntos]
        Dados espacios topológicos $(X, \cc{T})$ e $(Y, \cc{T}')$, demuestra que la proyección 
        \begin{align*}
            \pi_Y : (X, Y, \cc{T}\times \cc{T}')
        \end{align*}
        es continua y abierta. Da un ejemplo que demuestre que, en general, no es cerrada.\\

        Veamos en primer lugar que $\pi_Y$ es continua. Sea $U\in \cc{T}'$, tenemos que $\pi_Y^{-1}(U')=\{(x,y)\in X\times Y : y\in U'\} = X\times U'\in \cc{T}\times \cc{T}'$ por lo que es continua.\\

        Veamos ahora que es abierta. Para ello consideramos la siguiente base de la topología producto:
        \begin{gather*}
            \cc{B}_{\cc{T}\times \cc{T}'}=\{U\times U' : U\in \cc{T}, U'\in \cc{T}'\}
        \end{gather*}
        Sea $U\times U'\in \cc{B}_{\cc{T}\times \cc{T}'}$, entonces tenemos $\pi_Y(U\times U')=U'\in \cc{T}'$.\\

        Veamos además que en general no es cerrada. Para ello consideramos la topología $(\bb{R}^2, \cc{T}_u)$ que es la topología producto de $(\bb{R}, \cc{T}_u)$ consigo misma. Consideramos además la función $f:\bb{R} \to ]\nicefrac{-\pi}{2}, \nicefrac{\pi}{2}[$ dada por $f(x)=\arctg(x)$. Consideramos ahora el grafo de $f$ definido como $G(f)=\{(x,f(x)):x\in \bb{R}\}\in \cc{C}_{\cc{T}_u}$. Sin embargo tenemos que $\pi_Y(G(f)) = ]\nicefrac{-\pi}{2}, \nicefrac{\pi}{2}[\notin \cc{C}_{\cc{T}_u}$ por lo que en esta topología la proyección no es cerrada ya que $\exists C\in \cc{C}_{\cc{T}}$ tal que $\pi_Y(C)\notin \cc{C}_{\cc{T}}$.
    \end{ejercicio}

    \begin{ejercicio}[3 puntos]
        Sea $(X, \cc{T})$ un espacio tologógico y supongamos que para todo espacio topológico $(Y, \cc{T}')$ se tiene que toda aplicación $f:(X, \cc{T})\to (Y, \cc{T}')$ es continua. Demuestra que $\cc{T}$ es la topología discreta.\\

        Tomamos $(Y, \cc{T}')=(X, \cc{T}_{disc})$ y $f=Id_X$. Como la aplicación $Id_X : (X, \cc{T})\to (X, \cc{T}_{disc})$ es continua por hipótesis, entonces $\cc{T}_{disc}\leq \cc{T}$. Como siempre se da la otra inclusión ($\cc{T}_{disc}\geq \cc{T}$) tenemos que $\cc{T}=\cc{T}_{disc}$.
    \end{ejercicio}
    
    \begin{ejercicio}[4 puntos]
        En la circunferencia $\bb{S}^1=\{(x,y)\in \bb{R}^2 : x^2 + y^2 = 1\}\subset \bb{R}^2$ consideramos la topología $\cc{T}_u|_{\bb{S}^1}$ inducida por la topología usual $\cc{T}_u$ de $\bb{R}^2$.
        \begin{enumerate}
            \item Sea $R$ la relación de equivalencia en $\bb{S}^1$ dada por 
            \begin{align*}
                (x,y)R(x',y') \sii x=x'.
            \end{align*}
            Demuestra que el espacio topológico cociente $(\bb{S}^1/R, \cc{T}_u|_{\bb{S}^1}/R)$ es homeomorfo a $([-1,1], \cc{T}_u|_{[-1,1]})$, donde $\cc{T}_u|_{[-1.1]}$ es la topología en el intervalo $[-1,1]\subset \bb{R}$ inducida por la topología usual $\cc{T}_u$ de $\bb{R}$.\\

            Consideramos la siguiente aplicación
            \begin{align*}
                f:(\bb{S}^1, \cc{T}_u|_{\bb{S}^1}) &\to ([-1,1], \cc{T}_u)\\
                (x,y) &\mapsto x
            \end{align*}
            y tenemos lo siguiente:
            \begin{itemize}
                \item $f$ sobreyectiva trivialmente.
                \item $f$ continua ya que $f=\pi_X|_{\bb{S}^1}$ y $\pi_X$ es continua y la restricción en el dominio de una función continua sigue siendo continua.\footnote{También se puede ver que es continua con argumentos de análisis.}
                \item $f$ es cerrada ya que su dominio, $\bb{S}^1$ es cerrado y acotado en un espacio euclídeo y su codominio, $[-1,1]$ es un subespacio de un espacio euclídeo. Por el lema visto en clase tenemos que $f$ es cerrada.
            \end{itemize}

            Tenemos que $f$ es sobreyectiva, continua y cerrada luego $f$ es una identificación. Tenemos el siguiente diagrama:

            \begin{figure}[H]
                \centering
                \shorthandoff{""}
                \begin{tikzcd}
                    \bb{S}^1 \arrow[d, "\pi_f"'] \arrow[r, "f"]    & \text{[}-1,1\text{]} \\
                    \bb{S}^1/R_f \arrow[ru, "\tilde{f}", dashed] &  
                \end{tikzcd}
                \shorthandon{""}
            \end{figure}
            Por tanto $\exists \tilde{f}:(\bb{S}^1/R_f, \cc{T}_u|_{\bb{S}^1}/R_f)\to ([-1,1], \cc{T}_u|_{[-1,1]})$ homeomorfismo con $f=\tilde{f}\circ \pi_f$. Nos queda comprobar que $R_f=R$.

            En efecto, si $(x,y),(x',y')\in \bb{S}^1$, entonces 
            \begin{align*}
                (x,y)R(x',y') \sii f(x,y)=f(x',y') \sii x=x' \sii (x,y)R(x',y')
            \end{align*}

            \item Sea $R'$ la relación de equivalencia en $\bb{S}^1$ dada por
            \begin{align*}
                (x,y)R'(x',y')\sii (x,y)=(x',y') \text{ o } x=x'\neq 0.
            \end{align*}
            Demuestra que los espacios topológicos cociente $(\bb{S}^1/R, \cc{T}_u|_{\bb{S}^1}/R)$ y $(\bb{S}^1/R', \cc{T}_u|_{\bb{S}^1}/R')$ no son homeomorfos, donde $R$ es la relación de equivalencia del apartado anterior.\\

            Por el apartado anterior teníamos que $(\bb{S^1}/R, \cc{T}_u|_{\bb{S}^1}/R)\cong ([-1,1], \cc{T}_u|_{[-1,1]})$ que es T2, luego $(\bb{S^1}/R, \cc{T}_u|_{\bb{S}^1}/R)$ es T2. Como ser T2 es un invariante topológico, si comprobamos que $(\bb{S}^1/R', \cc{T}_u|_{\bb{S}^1})$ no es T2 habremos probado lo que queríamos.

            Si $(x,y)\in \bb{S}^1$, tenemos que su clase de equivalencia es
            \begin{itemize}
                \item $[(x,y)]=\{(x,y),(x,-y)\}$ si $x\neq 0$
                \item $[(0,1)]=\{(0,1)\}$
                \item $[(0,-1)]=\{(0,-1)\}$
            \end{itemize}
            Veamos que la propiedad T2 no se cumple para los puntos $[(0,-1)]$ y $[(0,-1)]$ que son distintos en $\bb{S}^1/R'$. Sean $\tilde{U}_1$ y $\tilde{U}_{-1}$ entornos de $[(0,1)]$ y $[(0,-1)]$ en $\bb{S}^1/R'$ respectivamente. Entonces $\tilde{U}_1 = p_{R'}(U_1)$ con $p_{R'}:\bb{S}^1 \to \bb{S}^1/R'$ la proyección y $U_1$ un entorno $p_{R'}-$saturado de $(0,1)$ en $\bb{S}^1$. Como $U_1$ es entorno de $(0,1)$, tenemos que existe $\veps >0$ (si queremos $\veps <1$) tal que $\{(x, \sqrt{1-x^2}):x\in (-\veps, \veps)\}\subset U_1$.\\

            Como además $U_1$ es $p_{R'}-$saturado, los puntos de $\bb{S}^1$ que están $R'-$relacionados con esos también están en $U_1$, luego $\{(x, -\sqrt{1-x^2}):x\in (-\veps, \veps), x\neq 0\}\subset U_1$. De esta forma tenemos que 
            \begin{gather*}
                (\{x, \sqrt{1-x^2}:x\in (-\veps, \veps)\}\cup \{(x, -\sqrt{1-x^2}) : x\in (-\veps, \veps), x\neq 0\})\subset U_1
            \end{gather*}
            Análogamente, tenemos que $\exists \veps'> 0$ tal que
            \begin{gather*}
                (\{x, \sqrt{1-x^2}:x\in (-\veps', \veps')\}\cup \{(x, -\sqrt{1-x^2}) : x\in (-\veps', \veps'), x\neq 0\})\subset U_{-1}
            \end{gather*}
            luego $U_1\cap U_{-1}\neq \emptyset$.\\

            Esto implica que $\tilde{U}_1\cap \tilde{U}_{-1}=p_{R'}(U_1)\cap p_{R'}(U_{-1})\neq \emptyset$ puesto que
            \begin{gather*}
                \emptyset \neq p_{R'}(U_1 \cap U_{-1})\subset p_{R'}(U_1)\cap p_{R'}(U_{-1})
            \end{gather*}

            Hemos comprobado entonces que $(\bb{S}^1/R', \cc{T}_u|_{\bb{S}^1})$ no es T2, luego no puede ser homeomorfo a $(\bb{S}^1/R, \cc{T}_u|_{\bb{S}^1}/R)$.
        \end{enumerate}
    \end{ejercicio}

\end{document}
