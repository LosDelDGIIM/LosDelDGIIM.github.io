\documentclass[12pt]{article}

% Idioma y codificación
\usepackage[spanish, es-tabla]{babel}       %es-tabla para que se titule "Tabla"
\usepackage[utf8]{inputenc}

% Márgenes
\usepackage[a4paper,top=3cm,bottom=2.5cm,left=3cm,right=3cm]{geometry}

% Comentarios de bloque
\usepackage{verbatim}

% Paquetes de links
\usepackage[hidelinks]{hyperref}    % Permite enlaces
\usepackage{url}                    % redirecciona a la web

% Más opciones para enumeraciones
\usepackage{enumitem}

% Personalizar la portada
\usepackage{titling}

% Paquetes de tablas
\usepackage{multirow}


%------------------------------------------------------------------------

%Paquetes de figuras
\usepackage{caption}
\usepackage{subcaption} % Figuras al lado de otras
\usepackage{float}      % Poner figuras en el sitio indicado H.


% Paquetes de imágenes
\usepackage{graphicx}       % Paquete para añadir imágenes
\usepackage{transparent}    % Para manejar la opacidad de las figuras

% Paquete para usar colores
\usepackage[dvipsnames]{xcolor}
\usepackage{pagecolor}      % Para cambiar el color de la página

% Habilita tamaños de fuente mayores
\usepackage{fix-cm}

% Para los gráficos
\usepackage{tikz}

% Para poder situar los nodos en los grafos
\usetikzlibrary{positioning}


%------------------------------------------------------------------------

% Paquetes de matemáticas
\usepackage{mathtools, amsfonts, amssymb, mathrsfs}
\usepackage[makeroom]{cancel}     % Simplificar tachando
\usepackage{polynom}    % Divisiones y Ruffini
\usepackage{units} % Para poner fracciones diagonales con \nicefrac

\usepackage{pgfplots}   %Representar funciones
\pgfplotsset{compat=1.18}  % Versión 1.18

\usepackage{tikz-cd}    % Para usar diagramas de composiciones
\usetikzlibrary{calc}   % Para usar cálculo de coordenadas en tikz

%Definición de teoremas, etc.
\usepackage{amsthm}
%\swapnumbers   % Intercambia la posición del texto y de la numeración

\theoremstyle{plain}

\makeatletter
\@ifclassloaded{article}{
  \newtheorem{teo}{Teorema}[section]
}{
  \newtheorem{teo}{Teorema}[chapter]  % Se resetea en cada chapter
}
\makeatother

\newtheorem{coro}{Corolario}[teo]           % Se resetea en cada teorema
\newtheorem{prop}[teo]{Proposición}         % Usa el mismo contador que teorema
\newtheorem{lema}[teo]{Lema}                % Usa el mismo contador que teorema

\theoremstyle{remark}
\newtheorem*{observacion}{Observación}

\theoremstyle{definition}

\makeatletter
\@ifclassloaded{article}{
  \newtheorem{definicion}{Definición} [section]     % Se resetea en cada chapter
}{
  \newtheorem{definicion}{Definición} [chapter]     % Se resetea en cada chapter
}
\makeatother

\newtheorem*{notacion}{Notación}
\newtheorem*{ejemplo}{Ejemplo}
\newtheorem*{ejercicio*}{Ejercicio}             % No numerado
\newtheorem{ejercicio}{Ejercicio} [section]     % Se resetea en cada section


% Modificar el formato de la numeración del teorema "ejercicio"
\renewcommand{\theejercicio}{%
  \ifnum\value{section}=0 % Si no se ha iniciado ninguna sección
    \arabic{ejercicio}% Solo mostrar el número de ejercicio
  \else
    \thesection.\arabic{ejercicio}% Mostrar número de sección y número de ejercicio
  \fi
}


% \renewcommand\qedsymbol{$\blacksquare$}         % Cambiar símbolo QED
%------------------------------------------------------------------------

% Paquetes para encabezados
\usepackage{fancyhdr}
\pagestyle{fancy}
\fancyhf{}

\newcommand{\helv}{ % Modificación tamaño de letra
\fontfamily{}\fontsize{12}{12}\selectfont}
\setlength{\headheight}{15pt} % Amplía el tamaño del índice


%\usepackage{lastpage}   % Referenciar última pag   \pageref{LastPage}
\fancyfoot[C]{\thepage}

%------------------------------------------------------------------------

% Conseguir que no ponga "Capítulo 1". Sino solo "1."
\makeatletter
\@ifclassloaded{book}{
  \renewcommand{\chaptermark}[1]{\markboth{\thechapter.\ #1}{}} % En el encabezado
    
  \renewcommand{\@makechapterhead}[1]{%
  \vspace*{50\p@}%
  {\parindent \z@ \raggedright \normalfont
    \ifnum \c@secnumdepth >\m@ne
      \huge\bfseries \thechapter.\hspace{1em}\ignorespaces
    \fi
    \interlinepenalty\@M
    \Huge \bfseries #1\par\nobreak
    \vskip 40\p@
  }}
}
\makeatother

%------------------------------------------------------------------------
% Paquetes de cógido
\usepackage{minted}
\renewcommand\listingscaption{Código fuente}

\usepackage{fancyvrb}
% Personaliza el tamaño de los números de línea
\renewcommand{\theFancyVerbLine}{\small\arabic{FancyVerbLine}}

% Estilo para C++
\newminted{cpp}{
    frame=lines,
    framesep=2mm,
    baselinestretch=1.2,
    linenos,
    escapeinside=||
}

% para minted
\definecolor{LightGray}{rgb}{0.95,0.95,0.92}
\setminted{
    linenos=true,
    stepnumber=5,
    numberfirstline=true,
    autogobble,
    breaklines=true,
    breakautoindent=true,
    breaksymbolleft=,
    breaksymbolright=,
    breaksymbolindentleft=0pt,
    breaksymbolindentright=0pt,
    breaksymbolsepleft=0pt,
    breaksymbolsepright=0pt,
    fontsize=\footnotesize,
    bgcolor=LightGray,
    numbersep=10pt
}


\usepackage{listings} % Para incluir código desde un archivo

\renewcommand\lstlistingname{Código Fuente}
\renewcommand\lstlistlistingname{Índice de Códigos Fuente}

% Definir colores
\definecolor{vscodepurple}{rgb}{0.5,0,0.5}
\definecolor{vscodeblue}{rgb}{0,0,0.8}
\definecolor{vscodegreen}{rgb}{0,0.5,0}
\definecolor{vscodegray}{rgb}{0.5,0.5,0.5}
\definecolor{vscodebackground}{rgb}{0.97,0.97,0.97}
\definecolor{vscodelightgray}{rgb}{0.9,0.9,0.9}

% Configuración para el estilo de C similar a VSCode
\lstdefinestyle{vscode_C}{
  backgroundcolor=\color{vscodebackground},
  commentstyle=\color{vscodegreen},
  keywordstyle=\color{vscodeblue},
  numberstyle=\tiny\color{vscodegray},
  stringstyle=\color{vscodepurple},
  basicstyle=\scriptsize\ttfamily,
  breakatwhitespace=false,
  breaklines=true,
  captionpos=b,
  keepspaces=true,
  numbers=left,
  numbersep=5pt,
  showspaces=false,
  showstringspaces=false,
  showtabs=false,
  tabsize=2,
  frame=tb,
  framerule=0pt,
  aboveskip=10pt,
  belowskip=10pt,
  xleftmargin=10pt,
  xrightmargin=10pt,
  framexleftmargin=10pt,
  framexrightmargin=10pt,
  framesep=0pt,
  rulecolor=\color{vscodelightgray},
  backgroundcolor=\color{vscodebackground},
}

%------------------------------------------------------------------------

% Comandos definidos
\newcommand{\bb}[1]{\mathbb{#1}}
\newcommand{\cc}[1]{\mathcal{#1}}

% I prefer the slanted \leq
\let\oldleq\leq % save them in case they're every wanted
\let\oldgeq\geq
\renewcommand{\leq}{\leqslant}
\renewcommand{\geq}{\geqslant}

% Si y solo si
\newcommand{\sii}{\iff}

% Letras griegas
\newcommand{\eps}{\epsilon}
\newcommand{\veps}{\varepsilon}
\newcommand{\lm}{\lambda}

\newcommand{\ol}{\overline}
\newcommand{\ul}{\underline}
\newcommand{\wt}{\widetilde}
\newcommand{\wh}{\widehat}

\let\oldvec\vec
\renewcommand{\vec}{\overrightarrow}

% Derivadas parciales
\newcommand{\del}[2]{\frac{\partial #1}{\partial #2}}
\newcommand{\Del}[3]{\frac{\partial^{#1} #2}{\partial #3^{#1}}}
\newcommand{\deld}[2]{\dfrac{\partial #1}{\partial #2}}
\newcommand{\Deld}[3]{\dfrac{\partial^{#1} #2}{\partial #3^{#1}}}


\newcommand{\AstIg}{\stackrel{(\ast)}{=}}
\newcommand{\Hop}{\stackrel{L'H\hat{o}pital}{=}}

\newcommand{\red}[1]{{\color{red}#1}} % Para integrales, destacar los cambios.

% Método de integración
\newcommand{\MetInt}[2]{
    \left[\begin{array}{c}
        #1 \\ #2
    \end{array}\right]
}

% Declarar aplicaciones
% 1. Nombre aplicación
% 2. Dominio
% 3. Codominio
% 4. Variable
% 5. Imagen de la variable
\newcommand{\Func}[5]{
    \begin{equation*}
        \begin{array}{rrll}
            #1:& #2 & \longrightarrow & #3\\
               & #4 & \longmapsto & #5
        \end{array}
    \end{equation*}
}

%------------------------------------------------------------------------

\newcommand{\T}[0]{\cc{T}}

\begin{document}

    % 1. Foto de fondo
    % 2. Título
    % 3. Encabezado Izquierdo
    % 4. Color de fondo
    % 5. Coord x del titulo
    % 6. Coord y del titulo
    % 7. Fecha

    
    % 1. Foto de fondo
% 2. Título
% 3. Encabezado Izquierdo
% 4. Color de fondo
% 5. Coord x del titulo
% 6. Coord y del titulo
% 7. Fecha

\newcommand{\portada}[7]{

    \portadaBase{#1}{#2}{#3}{#4}{#5}{#6}{#7}
    \portadaBook{#1}{#2}{#3}{#4}{#5}{#6}{#7}
}

\newcommand{\portadaExamen}[7]{

    \portadaBase{#1}{#2}{#3}{#4}{#5}{#6}{#7}
    \portadaArticle{#1}{#2}{#3}{#4}{#5}{#6}{#7}
}




\newcommand{\portadaBase}[7]{

    % Tiene la portada principal y la licencia Creative Commons
    
    % 1. Foto de fondo
    % 2. Título
    % 3. Encabezado Izquierdo
    % 4. Color de fondo
    % 5. Coord x del titulo
    % 6. Coord y del titulo
    % 7. Fecha
    
    
    \thispagestyle{empty}               % Sin encabezado ni pie de página
    \newgeometry{margin=0cm}        % Márgenes nulos para la primera página
    
    
    % Encabezado
    \fancyhead[L]{\helv #3}
    \fancyhead[R]{\helv \nouppercase{\leftmark}}
    
    
    \pagecolor{#4}        % Color de fondo para la portada
    
    \begin{figure}[p]
        \centering
        \transparent{0.3}           % Opacidad del 30% para la imagen
        
        \includegraphics[width=\paperwidth, keepaspectratio]{assets/#1}
    
        \begin{tikzpicture}[remember picture, overlay]
            \node[anchor=north west, text=white, opacity=1, font=\fontsize{60}{90}\selectfont\bfseries\sffamily, align=left] at (#5, #6) {#2};
            
            \node[anchor=south east, text=white, opacity=1, font=\fontsize{12}{18}\selectfont\sffamily, align=right] at (9.7, 3) {\textbf{\href{https://losdeldgiim.github.io/}{Los Del DGIIM}}};
            
            \node[anchor=south east, text=white, opacity=1, font=\fontsize{12}{15}\selectfont\sffamily, align=right] at (9.7, 1.8) {Doble Grado en Ingeniería Informática y Matemáticas\\Universidad de Granada};
        \end{tikzpicture}
    \end{figure}
    
    
    \restoregeometry        % Restaurar márgenes normales para las páginas subsiguientes
    \pagecolor{white}       % Restaurar el color de página
    
    
    \newpage
    \thispagestyle{empty}               % Sin encabezado ni pie de página
    \begin{tikzpicture}[remember picture, overlay]
        \node[anchor=south west, inner sep=3cm] at (current page.south west) {
            \begin{minipage}{0.5\paperwidth}
                \href{https://creativecommons.org/licenses/by-nc-nd/4.0/}{
                    \includegraphics[height=2cm]{assets/Licencia.png}
                }\vspace{1cm}\\
                Esta obra está bajo una
                \href{https://creativecommons.org/licenses/by-nc-nd/4.0/}{
                    Licencia Creative Commons Atribución-NoComercial-SinDerivadas 4.0 Internacional (CC BY-NC-ND 4.0).
                }\\
    
                Eres libre de compartir y redistribuir el contenido de esta obra en cualquier medio o formato, siempre y cuando des el crédito adecuado a los autores originales y no persigas fines comerciales. 
            \end{minipage}
        };
    \end{tikzpicture}
    
    
    
    % 1. Foto de fondo
    % 2. Título
    % 3. Encabezado Izquierdo
    % 4. Color de fondo
    % 5. Coord x del titulo
    % 6. Coord y del titulo
    % 7. Fecha


}


\newcommand{\portadaBook}[7]{

    % 1. Foto de fondo
    % 2. Título
    % 3. Encabezado Izquierdo
    % 4. Color de fondo
    % 5. Coord x del titulo
    % 6. Coord y del titulo
    % 7. Fecha

    % Personaliza el formato del título
    \pretitle{\begin{center}\bfseries\fontsize{42}{56}\selectfont}
    \posttitle{\par\end{center}\vspace{2em}}
    
    % Personaliza el formato del autor
    \preauthor{\begin{center}\Large}
    \postauthor{\par\end{center}\vfill}
    
    % Personaliza el formato de la fecha
    \predate{\begin{center}\huge}
    \postdate{\par\end{center}\vspace{2em}}
    
    \title{#2}
    \author{\href{https://losdeldgiim.github.io/}{Los Del DGIIM}}
    \date{Granada, #7}
    \maketitle
    
    \tableofcontents
}




\newcommand{\portadaArticle}[7]{

    % 1. Foto de fondo
    % 2. Título
    % 3. Encabezado Izquierdo
    % 4. Color de fondo
    % 5. Coord x del titulo
    % 6. Coord y del titulo
    % 7. Fecha

    % Personaliza el formato del título
    \pretitle{\begin{center}\bfseries\fontsize{42}{56}\selectfont}
    \posttitle{\par\end{center}\vspace{2em}}
    
    % Personaliza el formato del autor
    \preauthor{\begin{center}\Large}
    \postauthor{\par\end{center}\vspace{3em}}
    
    % Personaliza el formato de la fecha
    \predate{\begin{center}\huge}
    \postdate{\par\end{center}\vspace{5em}}
    
    \title{#2}
    \author{\href{https://losdeldgiim.github.io/}{Los Del DGIIM}}
    \date{Granada, #7}
    \thispagestyle{empty}               % Sin encabezado ni pie de página
    \maketitle
    \vfill
}
    \portadaExamen{ffccA4.jpg}{Topología I\\Examen VIII}{Topología I. Examen VIII}{MidnightBlue}{-8}{28}{2023-2024}{Arturo Olivares Martos}

    \begin{description}
        \item[Asignatura] Topología I.
        \item[Curso Académico] 2023-24.
        \item[Grado] Doble Grado en Ingeniería Informática y Matemáticas.
        \item[Grupo] Único.
        \item[Profesor] Jose Antonio Gálvez López\footnote{El examen lo pone el departamento.}.
        \item[Descripción] Convocatoria Ordinaria.
        \item[Fecha] 18 de enero de 2023.
        \item[Duración] 3 horas.
    
    \end{description}
    \newpage
    
    \begin{ejercicio}[5 puntos]
        Sea $\| \cdot \|$ la norma euclídea usual en $\mathbb{R}^2$. Se consideran los conjuntos
        $X = \{(x,y) \in \mathbb{R}^2 \mid \| (x,y) \| \leq 2\}$ y $S = \{(x,y) \in \mathbb{R}^2 \mid \| (x,y) \| = 2\}$.
        Para cada $p = (x,y) \in X$, se considera la siguiente familia:
        \begin{itemize}
          \item Si $p \in S$, $\beta_p = \{B((0,0), r) \cup \{p\} \mid 0 < r < 1\}$.
          \item Si $p \notin S$, $\beta_p = \{B(p,r) \mid 0 < r < 2 - \| p \|\}$.
        \end{itemize}
        Responde razonadamente a las siguientes preguntas:
        \begin{enumerate}
          \item(1.5 puntos) Demuestra que existe una única topología $\T$ en $X$ tal que las familias anteriores forman una base de entornos de cada punto. En los siguientes apartados, se considera el espacio topológico $(X,\T)$.
          
          Demostramos las 4 condiciones del Teorema corespondiente. En primer lugar, sea $p\in S$:
          \begin{enumerate}
            \item[V1)] Veamos que $\beta_p \neq \emptyset$. Como $]0,1[~\neq \emptyset$, entonces
            $\beta_p \neq \emptyset$. Por ejemplo, $B(0,\nicefrac{1}{2}) \cup \{p\} \in \beta_p$.

            \item[V2)] Dado $V\in \beta_p$, de forma directa se tiene que $p\in V$.
            
            \item[V3)] Sean $V_1, V_2 \in \beta_p$. Tenemos que ver que $\exists V_3\in \beta_p$ tal que
            $V_3 \subset V_1 \cap V_2$.

            Sean $V_1=B(0, r_1) \cup \{p\}$ y $V_2=B(0, r_2) \cup \{p\}$, con $r_1, r_2 \in ]0,1[$.
            Entonces, tomando $r_3 = \min\{r_1, r_2\}$, tenemos que $V_3=B(0, r_3) \cup \{p\} \in \beta_p$,
            y además $V_3 = V_1 \cap V_2$, por lo que $V_3$ cumple lo que buscamos.

            \item[V4)] Sea $V\in \beta_p$. Comprobemos que $\exists V'\in \beta_p$ tal que
            $V' \subset V$ y, para todo $q\in V'$, $\exists V_q\in \beta_q$ tal que $V_q \subset V$.

            Sea $V=B(0, r) \cup \{p\}$, con $r\in ]0,1[$. Entonces, tomando $r' = \nicefrac{r}{2}$, sea
            $V'=B(0, r') \cup \{p\} \in \beta_p$. Como $r' < r$, entonces $V' \subset V$. Sea ahora $q\in V'$:
            \begin{itemize}
              \item Si $q=p$, tomamos $V_q = V'$, y tenemos que $V_q \subset V$.
              \item Si $q\neq p$, entonces $q\in B(0, r')$, por lo que $q\notin S$. Por tanto, tenemos que
              $V_q = B(q, r_q) \in \beta_q$, para cierto $0<r_q < 2 - \|q\|$. Calculemos $r_q$ de forma que
              $V_q = B(q, r_q) \subset V$. Para ello, necesitamos que, para todo $x\in B(q, r_q)$, se tenga que $\|x\|<r$:
              \begin{equation*}
                \|x-q\| < r_q \Longrightarrow \|x\| < r_q + \|q\| < r_q + r' = r_q + \frac{r}{2} < r
                \Longleftrightarrow r_q < \frac{r}{2}
              \end{equation*}
              Por tanto, tomando $r_q = \dfrac{\min\left\{\frac{r}{2},~2 - \|q\|\right\}}{2}$, tenemos lo buscado.
            \end{itemize}
          \end{enumerate}

          Sea ahora $p\in X\setminus S$, por lo que $p\in B(0,2)$. Veamos que $\beta_p$ cumple las condiciones
            del Teorema:
            \begin{enumerate}
              \item[V1)] Veamos que $\beta_p \neq \emptyset$. Como $\|p\|<2$,
              tenemos que $]0,2-\|p\|[~\neq \emptyset$, por lo que $\beta_p \neq \emptyset$.
              Por ejemplo, $B(p,nicefrac{2-\|p\|}{2}) \in \beta_p$.

              \item[V2)] Dado $V\in \beta_p$, de forma directa se tiene que $p\in V$.
              
              \item[V3)] Sean $V_1, V_2 \in \beta_p$. Tenemos que ver que $\exists V_3\in \beta_p$ tal que
              $V_3 \subset V_1 \cap V_2$.

              Sean $V_1=B(p, r_1)$ y $V_2=B(p, r_2)$, con $r_1, r_2 \in ]0,2-\|p\|[$.
              Entonces, tomando $r_3 = \min\{r_1, r_2\}$, tenemos que $V_3=B(p, r_3) \in \beta_p$,
              y además $V_3 = V_1 \cap V_2$, por lo que $V_3$ cumple lo que buscamos.

              \item[V4)] Sea $V\in \beta_p$. Comprobemos que $\exists V'\in \beta_p$ tal que
              $V' \subset V$ y, para todo $q\in V'$, $\exists V_q\in \beta_q$ tal que $V_q \subset V$.

              Sea $V=B(p, r)$, con $r\in~]0,2-\|p\|[$. Entonces, tomando $r' = \nicefrac{r}{2}$, sea
              $V'=B(p, r') \in \beta_p$. Como $r' < r$, entonces $V' \subset V$. Sea ahora $q\in V'$:
              \begin{itemize}
                \item Si $\|q\|\geq 2$:
                
                Veamos que este caso no se puede dar.
                Como $q\in V'=B(p, r')$, entonces $\|q-p\|<r'<r<2-\|p\|$. Además, por
                la desigualdad triangular, tenemos que:
                \begin{equation*}
                  \|q\| - \|p\| \leq \|q-p\| < 2 - \|p\|
                \end{equation*}
                Por tanto, tenemos que $\|q\| - \|p\| < 2 - \|p\|$, por lo que $\|q\| < 2$.
                Pero esto es una contradicción, ya que habíamos supuesto que $\|q\|\geq 2$.

                \item Si $\|q\|<2$:
                
                Tenemos que $V_q=B(q, r_q) \in \beta_q$, para cierto $0<r_q < 2 - \|q\|$.
                Calculemos $r_q$ de forma que $V_q = B(q, r_q) \subset V$. Para ello, necesitamos que,
                para todo $x\in B(q, r_q)$, se tenga que $\|x\|<r$:
                \begin{equation*}
                  \|x-q\| < r_q \Longrightarrow \|x\| < r_q + \|q\| < r_q + r' = r_q + \frac{r}{2} < r
                  \Longleftrightarrow r_q < \frac{r}{2}
                \end{equation*}
                Por tanto, tomando $r_q = \dfrac{\min\left\{\frac{r}{2},~2 - \|q\|\right\}}{2}$, tenemos lo buscado.
              \end{itemize}
            
            \end{enumerate}

          \item(1 punto) Dado el subconjunto $A = \{(x,y) \in X : x > 0\}$, calcula el interior y la frontera de $A$.
          
          Representamos en la siguiente figura el conjunto $A$:
          \begin{figure}[H]
            \centering
            \begin{tikzpicture}
              \draw[red, ultra thick, fill=red!30] (0,-2) arc(-90:90:2);
              \draw[red, ultra thick, dashed] (0,-2) -- (0,2);
              
              % Ejes de coordenadas
              \draw[-stealth, dashed] (-1,0) -- (3,0) node[right] {$x$};
              \draw[-stealth, dashed] (0,-2.5) -- (0,2.5) node[above] {$y$};
            \end{tikzpicture}
          \end{figure}

          Calcularemos en primer lugar $A^\circ$. Tenemos que:
          \begin{equation*}
            p\in A^\circ \Longleftrightarrow \exists V\in \beta_p \mid V \subset A
          \end{equation*}

          Veamos que, si $p\in S$, entonces $p\notin A^\circ$. Sea $p\in S$,
          por lo que dado $V\in \beta_p$, se tiene que $V=B(0,r)\cup \{p\}$,
          con $r\in ]0,1[$. Por tanto, tenemos que $\left(-\frac{r}{2},0\right)\in V$,
          pero $\left(-\frac{r}{2},0\right)\notin A$, por lo que $V\not\subset A$. Por tanto,
          $p\notin A^\circ$.

          Veamos entonces que $A^\circ = \wt{A}$, con:
          \begin{equation*}
            \wt{A} = A\cap B(0,2) = A\setminus S(0,2) = \{(x,y)\in X \mid x>0,~\|(x,y)\|<2\}
          \end{equation*}

          \begin{description}
            \item[$\supset)$] Sea $p=(x,y)\in \wt{A}$, y veamos que $p\in A^\circ$. Como $p\in X$, $p\notin S$,
            entonces dado $V\in \beta_p$, se tiene que $V=B(p,r)$, con $r\in ]0,2-\|p\|[$. Buscamos el valor de $r$ tal que
            $V=B(p,r)\subset A$. Para ello, necesitamos que, para todo $(x',y')\in B(p,r)$, se tenga que $(x',y')\in A$; es decir,
            $x'>0$ y $\|(x',y')\|\leq 2$. Tomamos entonces $r=\min\left\{2-\|p\|,~\nicefrac{x}{2}\right\}$, y veamos que
            $B(p,r)\subset A$:
            \begin{description}
              \item[$\subset)$] Sea $q=(x',y')\in B(p,r)$, y veamos que $q\in A$. En primer lugar, comprobamos que
              $q\in X$, es decir, $\|q\|\leq 2$. Como $q\in B(p,r)$, entonces $\|q-p\|<r$, por lo que:
              \begin{equation*}
                \|q-p\|< r \Longrightarrow \|q\| < r+\|p\| < 2-\|p\|+\|p\| = 2
              \end{equation*}
              Por tanto, $q\in X$. Veamos ahora que $q\in A$, es decir, $x'>0$. Supongamos que $x'\leq 0$, y veamos que
              esto es una contradicción. Como $q\in B(p,r)$, entonces $\|q-p\|<r$, por lo que:
              \begin{multline*}
                \|q-p\|= \sqrt{(x'-x)^2+(y'-y)^2} < r \Longrightarrow |x'-x| < r
                \Longrightarrow \\ \Longrightarrow
                x-r < x' < x+r \Longrightarrow x-r < x'
              \end{multline*}
              No obstante, esto es una contradicción, ya que: $$x-r > x-\nicefrac{x}{2} = \nicefrac{x}{2} > 0$$
              Por tanto, $x'>0$, por lo que $q\in A$.
            \end{description}

            Por tanto, como $\exists r\in ]0,2-\|p\|[$ tal que $B(p,r)\subset A$, entonces $p\in A^\circ$.

            \item[$\subset)$] Sea $p\in A^\circ$, y veamos que $p\in \wt{A}$. Sabemos que $p\in A$, y supongamos que
            $p\in A\cap S$. Como hemos visto antes, esto es una contradicción, ya que $p\notin A^\circ$. Por tanto,
            $p\in A\setminus S=\wt{A}$.
          \end{description}

          Por tanto, queda demostrado que $A^\circ = \wt{A}$. Calculemos ahora el cierre. Para ello, sabemos que:
          \begin{equation*}
            p\in \ol{A} \Longleftrightarrow \forall V\in \beta_p,~V\cap A \neq \emptyset
          \end{equation*}
          
          Veamos entonces que $\ol{A}=\wh{A}$, con:
          \begin{equation*}
            \wh{A} = A \cup S \cup \{(0,y)\in X\}
          \end{equation*}
          \begin{description}
            \item[$\supset)$] Sea $p=(x,y)\in \wh{A}$, y veamos que $p\in \ol{A}$. Distinguimos en
            función de si $p\in S$ o no:
            \begin{itemize}
              \item Si $p\in S$, entonces dado $V\in \beta_p$, se tiene que $V=B(0,r)\cup \{p\}$.
              Por tanto, tenemos que:
              \begin{equation*}
                \left(\frac{r}{2}, 0\right) \in V\cap A \subset B(0,r)\cap A \neq \emptyset
              \end{equation*}

              \item Si $p\in A\setminus S$, entonces dado $V\in \beta_p$, se tiene que $V=B(p,r)$, con $r\in ]0,2-\|p\|[$. Por tanto,
              \begin{equation*}
                p\in V\cap A = B(p,r)\cap A \neq \emptyset
              \end{equation*}

              \item Si $p=(0,y),~y\neq \pm 2$, entonces dado $V\in \beta_p$, se tiene que $V=B(p,r)$, con $r\in ]0,2-\|p\|[$.
              Por tanto, tomando $\delta = \min \{\nicefrac{r}{2}, \sqrt{4-\|p\|^2}\}$, tenemos que:
              \begin{equation*}
                q = \left(\delta, y\right) \in V\cap A = B(p,r)\cap A \neq \emptyset
              \end{equation*}
              En primer lugar, tenemos que $q\in V$ ya que $\|q-p\|=|\delta|<r$. Veamos ahora que $q\in X$:
              \begin{equation*}
                \|q\| = \sqrt{\delta^2 + y^2}
                \leq \sqrt{{4-\|p\|^2} + y^2}
                = \sqrt{4} = 2
              \end{equation*}
              Por tanto, $q\in X$. Además, como $\delta > 0$, entonces $q\in A$. Por tanto, $q\in V\cap A$.

            \end{itemize}

            Por tanto, en los tres casos, $\forall V\in \beta_p,~V\cap A \neq \emptyset$, por lo que $p\in \ol{A}$.

            \item[$\subset)$] Sea $p\in \ol{A}$, y veamos que $p\in \wh{A}$. Supongamos que $p\in X$ pero $p\notin \wh{A}$.
            Entonces, $p=(x,y)$, con $x<0$ y $\|p\|<2$. Veamos que esto es una contradicción. Como $\|p\|<2$,
            entonces para todo $V\in \beta_p$, se tiene que $V=B(p,r)$, con $r\in ]0,2-\|p\|[$. Consideramos
            entonces el valor $r=\min\{2-\|p\|, \nicefrac{-x}{2}\}$, y veamos que
            $V\cap A = \emptyset$. Para ello, sea $q=(x',y')\in V$, y veamos que $q\notin A$. Tenemos que:
            \begin{multline*}
              \|q-p\| = \sqrt{(x'-x)^2+(y'-y)^2} < r \Longrightarrow |x'-x| < r \Longrightarrow \\
              \Longrightarrow x-r < x' < x+r \Longrightarrow x' < x+r
            \end{multline*}
            No obstante, tenemos que $x+r \leq x+\nicefrac{-x}{2} = \nicefrac{x}{2} < 0$, por lo que $x'<0$,
            por lo que $q\notin A$. Por tanto, $V\cap A = \emptyset$, por lo que $p\notin \ol{A}$, lo cual es una contradicción.
            Por tanto, $p\in \wh{A}$.
          \end{description}

          Por tanto, queda demostrado que $\ol{A}=\wh{A}$. Por tanto, tenemos que:
          \begin{equation*}
            \partial A = \ol{A}\setminus A^\circ = \wh{A}\setminus \wt{A}
            = \left(A \cup S \cup \{(0,y)\in X\}\right) \setminus \left(A\cap B(0,2)\right)
            = S \cup \{(0,y)\in X\}
          \end{equation*}

          \item(0.5 puntos) Estudia si $(X,\T)$ Hausdorff o no.
          
          Veamos que no lo es. Sean $p,q\in S\subset X$, con $p\neq q$. Entonces tenemos que,
          para todo $V\in \beta_p$, $U\in \beta_q$, se tiene que $0\in V\cap U$. Por tanto,
          $\nexists V\in \beta_p$, $U\in \beta_q$ tal que $V\cap U = \emptyset$, por lo que
          $(X,\T)$ no es Hausdorff.

          \item(1 punto) Estudia si $(X,\T)$ es conexo o no.
          
          Veamos en primer lugar que $V_p = B(p, \nicefrac{1}{2})$ es conexo para todo
          $p\in X, \|p\|\leq \nicefrac{1}{2}$. Veamos que $\T_{V_p} \subset {\T_u}_{V_p}$:
          \begin{description}
            \item[$\subset)$] Sea $q\in V_p$, y un entorno básico de $q$ en $\T_{V_p}$ será de la forma
            $B(q,r)\cap V_p$, con $r\in ]0,2-\|q\|[$. Veamos que dicho conjunto es un entorno de
            $q$ en ${\T_u}_{V_p}$. Como $B(q,r)\in \T_u$, entonces $B(q,r)\cap V_p \in {\T_u}_{V_p}$.
            Además, como $q\in B(q,r)\cap V_p$, entonces $B(q,r)\cap V_p$ es un entorno de $q$ en ${\T_u}_{V_p}$,
            por lo que se tiene que $\T_{V_p} \subset {\T_u}_{V_p}$.
          \end{description}
          Por tanto, supongamos que $V_p$ no es conexo, es decir, que existen
          $U,V\in \T_{V_p}$ no triviales tales que $U\cap V = \emptyset$ y $V_p = U\cup V$.
          Como $\T_{V_p} \subset {\T_u}_{V_p}$, entonces $U,V\in {\T_u}_{V_p}$, por lo que
          $\left(V_p, {\T_u}_{V_p}\right)$ no es conexo, lo cual es una contradicción, ya que $V_p$ es
          una bola abierta y sabemos que es conexa por ser estrellada. Por tanto, $V_p$ es conexo.\\
          
          Sea ahora $q\in S$, y consideramos $V_q=B(0, \nicefrac{1}{2})\cup \{q\} \in \beta_q$.
          Veamos que $V_q$ es conexo. Para ello, supongamos que $V_q$ no es conexo, es decir, que existen
          $U,V\in \T$ no triviales tales que $U\cap V = \emptyset$ y $V_q = U\cup V$.
          Supongamos sin pérdida de generalidad que $q\in U$. Como $U$ es abierto, entonces
          $U$ es entorno de $q$, por lo que $\exists r\in ]0,\nicefrac{1}{2}[$ tal que $B(0,r)\subset U$.
          Como $U\cap V = \emptyset$, entonces $V\subset B(0, \nicefrac{1}{2})\setminus B(0,r)$.

          Consideramos ahora $V_0=B(0, \nicefrac{1}{2})$, y tenemos que $\T_{V_0} \subset {\T_u}_{V_0}$.
          Sean $\wt{U}=U\cap V_0$, $\wt{V}=V\cap V_0$, y tenemos que $\wt{U}, \wt{V}\in \T_{V_0}$. Además,
          \begin{align*}
            \wt{U}\cap \wt{V} &= (U\cap V_0)\cap (V\cap V_0) = U\cap V \cap V_0 = \emptyset\\
            \wt{U}\cup \wt{V} &= (U\cap V_0)\cup (V\cap V_0) = (U\cup V)\cap V_0 = V_q \cap V_0 = V_0
          \end{align*}
          Por tanto, tenemos que $V_0$ es disconexo. No obstante, esto es una contradicción, ya que
          antes hemos visto que $V_0$ era conexo. Por tanto, $V_q$ es conexo.\\

          De esta forma, tenemos que:
          \begin{multline*}
            X = S\cup (X\setminus S) = \left(\bigcup_{p\in S} \{s\}\right) \cup X\setminus S=            
            \bigcup_{s\in S} \left(B\left(0, \nicefrac{1}{2}\right) \cup \{s\}\right)
            \cup X\setminus S \subset \\
            \subset \bigcup_{s\in S} \left(B\left(0, \nicefrac{1}{2}\right) \cup \{s\}\right)
            \bigcup_{\substack{p\in X\\ \|p\|=\nicefrac{1}{2}}} B\left(p, \frac{1}{2}\right)\subset X
          \end{multline*}

          Por tanto, tenemos que:
          \begin{equation*}
            X = B(0, \nicefrac{1}{2}) \cup
            \left(\bigcup_{s\in S} \left(B\left(0, \nicefrac{1}{2}\right) \cup \{s\}\right)\right)
            \bigcup_{\substack{p\in X\\ \|p\|=\nicefrac{1}{2}}} B\left(p, \frac{1}{2}\right)
          \end{equation*}
          Entonces, $X$ es la unión de conjuntos conexos en $(X,\T)$, y todos ellos intersecan a $B(0, \nicefrac{1}{2})$,
          por lo que $(X,\T)$ es conexo.
          
          
          \item(1 punto) Estudia si $(X,\T)$ es compacto o no.
          
          Veamos que $X\setminus S$ es abierto. Sea $p\in X\setminus S$, y veamos que $\exists V\in \beta_p$ tal que
          que $V\subset X\setminus S$. Como $p\in X\setminus S$, entonces $\|p\|<2$. Por tanto, dado $V\in \beta_p$,
          se tiene que $V=B(p,r)$, con $r\in ]0,2-\|p\|[$. Tomamos entonces $r=\dfrac{2-\|p\|}{2}$, y veamos que
          $V\subset X\setminus S$. Para ello, sea $q\in V$, y veamos que $q\in X\setminus S$. Como $q\in V$, entonces:
          \begin{equation*}
            \|q-p\| < r = \frac{2-\|p\|}{2} = 1 - \frac{\|p\|}{2} \Longrightarrow
            \|q\| < 1 - \frac{\|p\|}{2} + \|p\| = 1 + \frac{\|p\|}{2} < 2 \Longleftrightarrow \|p\| < 2
          \end{equation*}
          Por tanto, $q\in X\setminus S$, por lo que $V\subset X\setminus S$. Por tanto, $X\setminus S$ es abierto.

          Veamos ahora que, dado $p\in S$, $U_p = B(0, \nicefrac{1}{2})\cup \{p\}$ es abierto. Para ello, veamos que
          dado $q\in U_p$, $\exists V\in \beta_q$ tal que $V\subset U_p$. Distinguimos en función de si $q=p$ o no:
          \begin{itemize}
            \item Si $q=p$, entonces $V=U_p$ cumple lo que buscamos.

            \item Si $q\neq p$, entonces $q\in B(0, \nicefrac{1}{2})$, por lo que $\|q\|<\nicefrac{1}{2}$.
            Por tanto, dado $V\in \beta_q$, se tiene que $V=B(q,r)$, con $r\in ]0,2-\|q\|[$. Por tanto, tomando $r$
            lo suficientemente pequeño, tenemos que $V\subset B(0, \nicefrac{1}{2})$, por lo que $V\subset U_p$.
          \end{itemize}

          Por tanto, $U_p$ es abierto. Por tanto, consideramos el siguiente recubrimiento de $X$ mediante abiertos:
          \begin{equation*}
            X = \left(\bigcup_{p\in S} U_p\right) \cup (X\setminus S)
          \end{equation*}

          Supomgamos entonces que $X$ es compacto. Entonces, dado dicho recubrimiento, existe un subrecubrimiento finito.
          Sea entonces $S_0\subset S$ finito tal que:
          \begin{equation*}
            X = \left(\bigcup_{p\in S_0} U_p\right) \cup (X\setminus S)
          \end{equation*}
          No obstante, esto es una contradicción, ya que $S$ es no numerable (infinito), y $S_0$ es finito. Por tanto,
          hay puntos de $S$ que no están en $S_0$, por lo que no están en dicho subrecubrimiento recubrimiento, lo cual es una contradicción.

          Por tanto, $X$ no es compacto.
        \end{enumerate}

    \end{ejercicio}

    \begin{ejercicio}[2.5 puntos]
        Elige una de las siguientes preguntas (2a o 2b):
        \begin{enumerate}
          \item[2a.] Da una definición de subespacio compacto en un espacio topológico y prueba las siguientes afirmaciones:
          \begin{enumerate}
            \item Un subespacio cerrado de un espacio topológico compacto es compacto.
            \item Todo subespacio compacto de un espacio topológico Hausdorff es cerrado.
          \end{enumerate}

          \item[2b.] Razona si son verdaderas o falsas las siguientes afirmaciones:
          \begin{enumerate}
            \item Todo entorno de un punto de un espacio topológico es abierto.
            
            Esto es falso. Como contraejemplo, consideremos el espacio topológico $(\mathbb{R}, \T_u)$, donde $\T_u$ es la topología usual.
            Entonces, el conjunto $[-2,2]$ es un entorno de $0$, ya que existe una bola abierta $]-1,1[$ tal que se tiene que $0 \in ]-1,1[ \subset [-2,2]$.
            No obstante, este conjunto no es abierto, ya que no es entorno del $2$.

            \item Si $(X, \T)$ es un espacio topológico entonces la aplicación identidad dada por $Id:~(X, \T) \to (X, \T_{\text{CF}})$ es continua si y solo si $(X,\T)$ es T1. Aquí, $\T_{\text{CF}}$ denota la topología cofinita.
            
            Tenemos que $Id$ es continua si y solo si $\T_{\text{CF}} \subset \T$. Veamos por tanto que es cierta
            mediante doble implicación:
            \begin{description}
              \item[$\Longrightarrow)$] Supongamos que $Id$ es continua, por lo que $\T_{\text{CF}} \subset \T$.
              Dados $x,y\in X$ tales que $x \neq y$, entonces como $(X,\T_{\text{CF}})$ es T1, entonces
              existen $U\in \T_{\text{CF}}$ con $x\in U$, $y\notin U$. Como $\T_{\text{CF}} \subset \T$,
              entonces $U\in \T$, por lo que $(X,\T)$ es T1.

              \item[$\Longleftarrow)$] Supongamos que $(X,\T)$ es T1, y veamos que $\T_{\text{CF}} \subset \T$.
              Para ello, veamos que $C_{\text{CF}} \subset C_{\T}$. Sea $C\in C_{\text{CF}}$.
              Entonces, $C$ es finito, por lo que $\exists n\in \bb{N}$ tal que $C = \{x_1, \dots, x_n\}$:
              \begin{equation*}
                C = \bigcup_{i=1}^n \{x_i\}
              \end{equation*}
              Como $(X,\T)$ es T1, entonces $\{x\}\in C_{\T}$ para todo $x\in X$, por lo que $C$ es una unión finita
              de cerrados, por lo que $C\in C_{\T}$. Por tanto, $C_{\text{CF}} \subset C_{\T}$,
              por lo que $\T_{\text{CF}} \subset \T$ e $Id:~(X, \T) \to (X, \T_{\text{CF}})$ es continua.
            \end{description}

            \item Si $f : (X, \T) \to (Y, \T')$ es una aplicación biyectiva entre espacios compactos y T2, entonces $f$ es continua si, y solo si, $f^{-1}$ es continua.
            
            Demostramos que es cierta mediante doble implicación:
            \begin{description}
              \item[$\Longrightarrow)$] Supongamos que $f$ es continua, y tenemos que $f$ es biyectiva.
              Además, como $(X,\T)$ es compacto y $(Y,\T')$ es T2, entonces $f$ es cerrada.
              Como $f$ es continua, cerrada y biyectiva, entonces $f$ es un homeomorfismo, por lo que $f^{-1}$ es continua.

              \item[$\Longleftarrow)$] De manera análoga, como $f^{-1}$ es continua, cerrada y biyectiva, entonces $f^{-1}$ es un homeomorfismo, por lo que $f$ es continua.            \end{description}
          \end{enumerate}
        \end{enumerate}
    \end{ejercicio}
        
    \begin{ejercicio}[2.5 puntos]
        En $X = \mathbb{R} \times \{-1,1\}$ consideramos la relación de equivalencia dada por:
        \[(x_1,y_1)\cc{R}(x_2,y_2) \iff \left\{
            \begin{array}{ll}
            (x_1, y_1) = (x_2, y_2), & \text{o bien} \\
            x_1, x_2 > 1, & \text{o bien} \\
            x_1, x_2 < -1.
            \end{array}
        \right.\]
        Si sobre $X$ consideramos la topología usual inducida de $\mathbb{R}^2$, estudiar si:
        \begin{enumerate}
            \item La proyección canónica $p : X \to X/\cc{R}$ es una aplicación abierta,
            
            Veamos qué puntos identifica la relación de equivalencia:
            \begin{figure}[H]
                \centering
                \begin{tikzpicture}
                    % Ejes de coordenadas
                    \draw[-stealth, dashed] (-3,0) -- (3,0) node[right] {$x_1$};
                    \draw[-stealth, dashed] (0,-2) -- (0,2) node[above] {$x_2$};

                    % Conjunto X
                    \draw[thick] (-2,-1) -- (2,-1);
                    \draw[thick] (-2,1) -- (2,1);

                    % t < -1
                    \draw[ultra thick, red] (-2,-1) -- (-1,-1);
                    \draw[ultra thick, red] (-2,1) -- (-1,1);

                    % t > 1
                    \draw[ultra thick, blue] (2,-1) -- (1,-1);
                    \draw[ultra thick, blue] (2,1) -- (1,1);

                    % Circulos abiertos en los puntos (1,\pm 1) y (-1,\pm 1)
                    \draw[fill=white] (1,1) circle (0.1);
                    \draw[fill=white] (1,-1) circle (0.1);
                    \draw[fill=white] (-1,1) circle (0.1);
                    \draw[fill=white] (-1,-1) circle (0.1);

                    % Marca en $x=1$ y $x=-1$
                    \draw (1,0.1) -- node[below] {$1$} (1,-0.1);
                    \draw (-1,0.1) -- node[below] {$-1$} (-1,-0.1);

                \end{tikzpicture}
            \end{figure}

            Demostremos ahora que es abierta. Para ello, una base de
            $(X,\T)$ es:
            \begin{equation*}
              \cc{B} = \left\{
                \left]a,b\right[\times \{1\}, \left]a,b\right[\times \{-1\}
                \mid
                a,b\in \bb{R},~a<b
              \right\}
            \end{equation*}

            Demostramos para el caso de $\left]a,b\right[\times \{1\}$, y el otro caso es análogo.            
            Distingumos entonces en función de $a,b$:
            \begin{itemize}
              \item Si $]a,b[ \subset ]1, +\infty[$:
              
              En este caso, tenemos que:
              \begin{equation*}
                p(\left]a,b\right[\times \{1\}) = \{[~]1, +\infty[ \times \{1,-1\}~]\} \in \T/\cc{R}
              \end{equation*}

              \item Si $]a,b[ \subset ]-\infty, -1[$:
              
              En este caso, tenemos que:
              \begin{equation*}
                p(\left]a,b\right[\times \{1\}) = \{[~]-\infty, -1[ \times \{1,-1\}~\} \in \T/\cc{R}
              \end{equation*}

              \item Si $]a,b[ \subset ]-1,1[$:
              
              En este caso, tenemos que $]a,b[\times \{1\}\in \T_X$. Además,
              \begin{equation*}
                p(\left]a,b\right[\times \{1\}) = ]a,b[ \times \{1\}
                \qquad
                p^{-1}(p(\left]a,b\right[\times \{1\})) = ]a,b[ \times \{1\}
              \end{equation*}
              Por tanto, $]a,b[\times \{1\}$ es saturado, por lo que
              $p(]a,b[\times \{1\})\in \T/\cc{R}$.

              \item Si $-1<a<1$ y $b>1$:
              
              Tenemos que:
              \begin{equation*}
                p(U) = ]a, 1] \times \{1\} \cup \{[~]1, +\infty[ \times \{1,-1\}~]\}
              \end{equation*}
              
              Veamos que dicho conjunto es abierto. Sea el siguiente conjunto:
              $$V=]a,+\infty[\times \{1\}~\cup~]1, +\infty[ \times \{-1\}\in \T_X$$
              Tenemos que $p^{-1}(p(V)) = V$, por lo que $p(V)$ es abierto en $\T/\cc{R}$.
              Además, $p(U) = p(V)$, por lo que $p(U)$ es abierto en $\T/\cc{R}$.

              \item Si $a<-1$ y $-1<b<1$:
              
              Tenemos que:
              \begin{equation*}
                p(U) = \{[~]-\infty, -1[ \times \{1,-1\}~]\}~\cup~[-1,b[ \times \{1\}
              \end{equation*}

              Veamos que dicho conjunto es abierto. Sea el siguiente conjunto:
              $$V=]-\infty, -1[ \times \{-1\}~\cup~]-\infty,b[\times \{1\}\in \T_X$$
              Tenemos que $p^{-1}(p(V)) = V$, por lo que $p(V)$ es abierto en $\T/\cc{R}$.
              Además, $p(U) = p(V)$, por lo que $p(U)$ es abierto en $\T/\cc{R}$.

              \item Si $a<-1$ y $b>1$:
              
              Tenemos que:
              \begin{equation*}
                p(U) =
                \{[~]-\infty, -1[ \times \{1,-1\}~]\} \cup
                \{[~]1, +\infty[ \times \{1,-1\}~]\} \cup
                [-1,1] \times \{1\}
              \end{equation*}

              Veamos que dicho conjunto es abierto. Sea el siguiente conjunto:
              $$V=\bb{R}\times \{1\} ~\cup~
              ]-\infty, -1[ \times \{-1\}~\cup~]1, +\infty[ \times \{-1\}\in \T_X$$
              Tenemos que $p^{-1}(p(V)) = V$, por lo que $p(V)$ es abierto en $\T/\cc{R}$.
              Además, $p(U) = p(V)$, por lo que $p(U)$ es abierto en $\T/\cc{R}$.
            \end{itemize}

            Por tanto, como para todos los abiertos de la base $\cc{B}$ se tiene que $p(U)\in \T/\cc{R}$,
            entonces $p$ es abierta.

            \begin{comment}
              Consideramos ahora $U=]2,4[\times \{1\} \subset X$.
              Tenemos que $U=\left(]2,4[\times ]0,2[ \right)\cap X$,
              por lo que $U\in \T_X$. Tenemos que ver que $p(U)\notin \T/\cc{R}$, y para ello veremos que
              $U$ no es saturado. Tenemos que:
              \begin{equation*}
                  p(U) = \left\{\left[~]1, +\infty[ \times \{1,-1\}~\right]\right\}
                  \qquad
                  p^{-1}(p(U)) = ]1, +\infty[ \times \{1,-1\}
              \end{equation*}

              Como $p^{-1}(p(U)) \neq U$, entonces $U$ no es saturado,
              por lo que $p(U)\notin \T/\cc{R}$ y por tanto $p$ no es abierta.
            \end{comment}

            \item $X/\cc{R}$ es T2,
            
            En primer lugar, necesitamos calcular $X/\cc{R}$. Tenemos que:
            \begin{multline*}
                X/\cc{R} =\left\{
                  \left[~]1, +\infty[ \times \{1,-1\}~\right],
                  \left[~]-\infty, -1[ \times \{1,-1\}~\right]
                \right\}
                \bigcup \\ \bigcup
                \left\{
                  A \times \{1\}, A \times \{-1\}, A \times \{1,-1\}
                  \mid
                  A\subset [-1,1]
                \right\}
            \end{multline*}
            
            \begin{comment}
            Además, tenemos que:
            \begin{multline*}
                \T/\cc{R} = \left\{
                  \left[~]1, +\infty[ \times \{1,-1\}~\right],
                  \left[~]-\infty, -1[ \times \{1,-1\}~\right]
                \right\}
                \bigcup \\ \bigcup
                \left\{
                  A \times \{1\}, A \times \{-1\}, A \times \{1,-1\}
                  \mid
                  A\in \T_{]-1,1[}
                \right\}
            \end{multline*}
            \end{comment}


            Sea $x=\{1\}\times \{1\}=(1,1) \in X/\cc{R}$. Buscamos $U\in \T_X$ saturado
            tal que $\{x\}\in p(U)$. Como $p^{-1}(\{x\}) = x$, buscamos $U\in \T_X$ saturado tal que $x\in U$,
            por lo que $U$ es entorno de $x$, y entonces $\exists \veps\in \bb{R}^+$ tal que
            $B((1,1),\veps)\cap X \subset U$. Por tanto, se tiene que $U\cap \left(]1, +\infty[ \times \{1,-1\}\right) \neq \emptyset$.
            Por tanto, como $U$ es saturado,
            \begin{equation*}
              \left(]1, +\infty[ \times \{1,-1\}\right) \cup \left(1,1\right) \subset U
            \end{equation*}
            Por tanto, tenemos que, tomando $x=\left[(1,1)\right]$ e $y=[~]1, +\infty[ \times \{1,-1\}~]$, se tiene que
            $\forall U\in \T/\cc{R}$ con $x\in U$:
            $$\{y\} \cap U \subset \{y\} \cap \left[~\left(]1, +\infty[ \times \{1,-1\}\right) \cup \left(1,1\right)~\right]
            = \{y\} \cap \left(\{y\}\cup \{x\} \right) = \{y\} \neq \emptyset$$

            Por tanto, no es T2.\\

            Otra opción de ver que no es T2 es ver que tampoco es T1. Para esto, veamos que
            $\{[~]1, +\infty[ \times \{1,-1\}~]\}$ no es cerrado en $X/\cc{R}$.
            
            Buscamos $U\subset X$ saturado y cerrado tal que $p(U) = \{[~]1, +\infty[ \times \{1,-1\}~]\}$.
            El único conjunto saturado de $X$ tal que $p(U) = \{[~]1, +\infty[ \times \{1,-1\}~]\}$ es
            $U = ]1, +\infty[ \times \{1,-1\}$, que no es cerrado, por lo que no existe dicho
            conjunto saturado y cerrado. Por tanto, $\{[~]1, +\infty[ \times \{1,-1\}~]\}$ no es cerrado,
            por lo que $X/\cc{R}$ no es T1, por lo que tampoco es T2.

            \item $X/\cc{R}$ es compacto,
            
            Sabemos que $X$ no es compacto por no ser $\bb{R}$ acotado, por lo que el hecho de que la proyección
            $p$ sea continua no nos sirve para demostrar si $X/\cc{R}$ es compacto.

            \begin{description}
              \item[Opción 1:] De forma directa.
              
              Para ver que sí es compacto, demostraremos que $X/\cc{R}$ es unión finita de compactos.
              Para ello, sean los siguientes conjuntos:
              \begin{align*}
                A_1 &= \{[~]1, +\infty[ \times \{1,-1\}~]\}&
                A_2 &= \{[~]-\infty, -1[ \times \{1,-1\}~]\}
                \\
                A_3 &= [-1,1] \times \{1\}&
                A_4 &= [-1,1] \times \{-1\}
              \end{align*}

              Es directo ver que $X/\cc{R} = A_1 \cup A_2 \cup A_3 \cup A_4$.
              Además, $A_1,A_2$ son compactos por ser conjuntos unitarios.
              Veamos ahora que $A_3$ es compacto. En primer lugar, sabemos que
              es cerrado y acotado en $\bb{R}^2$, por lo que es compacto en $\bb{R}^2$ y,
              como es cerrado en $X$, entonces es compacto en $X$. Además, como
              ${\T/\cc{R}}_{\big|A_3} = \T_{\big|A_3}$, entonces $A_3$ es compacto en
              $X/\cc{R}$. De forma análoga, se demuestra que $A_4$ es compacto.

              Por tanto, $X/\cc{R}$ es unión finita de compactos, por lo que es compacto.

              \item[Opción 2:] Usando que la proyección canónica es continua.
              
              Sea $C=[-2,2] \times \{-1,1\}\subset X$ un conjunto compacto en $X$. por ser unión de dos compactos.
              Como $p$ es continua, tenemos que $p(C)=X/\cc{R}$ es compacto.
            \end{description}

            \item $X/\cc{R}$ es conexo.
            
            Sabemos que $X$ no es conexo por no ser producto de conexos (ya que $\{-1,1\}$ no es conexo),
            por lo que el hecho de que la proyección $p$ sea continua no nos sirve para demostrar si $X/\cc{R}$ es conexo.

            \begin{description}
              \item[Opción 1:] De forma directa.
              
              Para ver si es conexo, buscamos $p(U)\in \T/\cc{R}\cap C_{\T/\cc{R}}$.
              Es decir, buscamos $U$ saturado tal que $U\in \T_X\cap C_{\T_X}$.
              Como $\bb{R}$ es conexo, los únicos abiertos y cerrados en $(\bb{R}, \T_u)$ son $\emptyset$ y $\bb{R}$. Por tanto,
              los conjuntos abiertos y cerrados en $X$ son:
              \begin{equation*}
                  \emptyset,~X,~\bb{R}\times \{1\},~\bb{R}\times \{-1\}
              \end{equation*}
              No obstante, los dos últimos no son saturados, por lo que
              los únicos abiertos y cerrados en $X$ saturados son $\emptyset$ y $X$.
              Por tanto, $U=X, \emptyset$ por lo que $p(U)=X/\cc{R}, \emptyset$.
              Por tanto, tenemos que $X/\cc{R}$ es conexo.

              \item[Opción 2:] Usando que la proyección canónica es continua.
              
              Sea $A=[-2,2] \times \{-1\}, B=[-2,2] \times \{1\}\subset X$ dos conjuntos conexos en $X$. Como $p$ es continua, tenemos que
              $p(A), p(B)$ son conexos en $X/\cc{R}$. Además, tenemos que:
              \begin{equation*}
                X/\cc{R} = p(A) \cup p(B)
              \end{equation*}

              Además, $p(A)\cap p(B) \neq \emptyset$, ya que $p(2,1)=p(2,-1)=[~]1, +\infty[ \times \{1,-1\}~]\in p(A)\cap p(B)$.
              Por tanto, como $X/\cc{R}$ es unión de dos conjuntos conexos con intersección no vacía, entonces $X/\cc{R}$ es conexo.
            \end{description}
        \end{enumerate}
    \end{ejercicio}
\end{document}