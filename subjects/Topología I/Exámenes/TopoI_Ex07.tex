\documentclass[12pt]{article}

% Idioma y codificación
\usepackage[spanish, es-tabla]{babel}       %es-tabla para que se titule "Tabla"
\usepackage[utf8]{inputenc}

% Márgenes
\usepackage[a4paper,top=3cm,bottom=2.5cm,left=3cm,right=3cm]{geometry}

% Comentarios de bloque
\usepackage{verbatim}

% Paquetes de links
\usepackage[hidelinks]{hyperref}    % Permite enlaces
\usepackage{url}                    % redirecciona a la web

% Más opciones para enumeraciones
\usepackage{enumitem}

% Personalizar la portada
\usepackage{titling}

% Paquetes de tablas
\usepackage{multirow}


%------------------------------------------------------------------------

%Paquetes de figuras
\usepackage{caption}
\usepackage{subcaption} % Figuras al lado de otras
\usepackage{float}      % Poner figuras en el sitio indicado H.


% Paquetes de imágenes
\usepackage{graphicx}       % Paquete para añadir imágenes
\usepackage{transparent}    % Para manejar la opacidad de las figuras

% Paquete para usar colores
\usepackage[dvipsnames]{xcolor}
\usepackage{pagecolor}      % Para cambiar el color de la página

% Habilita tamaños de fuente mayores
\usepackage{fix-cm}

% Para los gráficos
\usepackage{tikz}

% Para poder situar los nodos en los grafos
\usetikzlibrary{positioning}


%------------------------------------------------------------------------

% Paquetes de matemáticas
\usepackage{mathtools, amsfonts, amssymb, mathrsfs}
\usepackage[makeroom]{cancel}     % Simplificar tachando
\usepackage{polynom}    % Divisiones y Ruffini
\usepackage{units} % Para poner fracciones diagonales con \nicefrac

\usepackage{pgfplots}   %Representar funciones
\pgfplotsset{compat=1.18}  % Versión 1.18

\usepackage{tikz-cd}    % Para usar diagramas de composiciones
\usetikzlibrary{calc}   % Para usar cálculo de coordenadas en tikz

%Definición de teoremas, etc.
\usepackage{amsthm}
%\swapnumbers   % Intercambia la posición del texto y de la numeración

\theoremstyle{plain}

\makeatletter
\@ifclassloaded{article}{
  \newtheorem{teo}{Teorema}[section]
}{
  \newtheorem{teo}{Teorema}[chapter]  % Se resetea en cada chapter
}
\makeatother

\newtheorem{coro}{Corolario}[teo]           % Se resetea en cada teorema
\newtheorem{prop}[teo]{Proposición}         % Usa el mismo contador que teorema
\newtheorem{lema}[teo]{Lema}                % Usa el mismo contador que teorema

\theoremstyle{remark}
\newtheorem*{observacion}{Observación}

\theoremstyle{definition}

\makeatletter
\@ifclassloaded{article}{
  \newtheorem{definicion}{Definición} [section]     % Se resetea en cada chapter
}{
  \newtheorem{definicion}{Definición} [chapter]     % Se resetea en cada chapter
}
\makeatother

\newtheorem*{notacion}{Notación}
\newtheorem*{ejemplo}{Ejemplo}
\newtheorem*{ejercicio*}{Ejercicio}             % No numerado
\newtheorem{ejercicio}{Ejercicio} [section]     % Se resetea en cada section


% Modificar el formato de la numeración del teorema "ejercicio"
\renewcommand{\theejercicio}{%
  \ifnum\value{section}=0 % Si no se ha iniciado ninguna sección
    \arabic{ejercicio}% Solo mostrar el número de ejercicio
  \else
    \thesection.\arabic{ejercicio}% Mostrar número de sección y número de ejercicio
  \fi
}


% \renewcommand\qedsymbol{$\blacksquare$}         % Cambiar símbolo QED
%------------------------------------------------------------------------

% Paquetes para encabezados
\usepackage{fancyhdr}
\pagestyle{fancy}
\fancyhf{}

\newcommand{\helv}{ % Modificación tamaño de letra
\fontfamily{}\fontsize{12}{12}\selectfont}
\setlength{\headheight}{15pt} % Amplía el tamaño del índice


%\usepackage{lastpage}   % Referenciar última pag   \pageref{LastPage}
\fancyfoot[C]{\thepage}

%------------------------------------------------------------------------

% Conseguir que no ponga "Capítulo 1". Sino solo "1."
\makeatletter
\@ifclassloaded{book}{
  \renewcommand{\chaptermark}[1]{\markboth{\thechapter.\ #1}{}} % En el encabezado
    
  \renewcommand{\@makechapterhead}[1]{%
  \vspace*{50\p@}%
  {\parindent \z@ \raggedright \normalfont
    \ifnum \c@secnumdepth >\m@ne
      \huge\bfseries \thechapter.\hspace{1em}\ignorespaces
    \fi
    \interlinepenalty\@M
    \Huge \bfseries #1\par\nobreak
    \vskip 40\p@
  }}
}
\makeatother

%------------------------------------------------------------------------
% Paquetes de cógido
\usepackage{minted}
\renewcommand\listingscaption{Código fuente}

\usepackage{fancyvrb}
% Personaliza el tamaño de los números de línea
\renewcommand{\theFancyVerbLine}{\small\arabic{FancyVerbLine}}

% Estilo para C++
\newminted{cpp}{
    frame=lines,
    framesep=2mm,
    baselinestretch=1.2,
    linenos,
    escapeinside=||
}

% para minted
\definecolor{LightGray}{rgb}{0.95,0.95,0.92}
\setminted{
    linenos=true,
    stepnumber=5,
    numberfirstline=true,
    autogobble,
    breaklines=true,
    breakautoindent=true,
    breaksymbolleft=,
    breaksymbolright=,
    breaksymbolindentleft=0pt,
    breaksymbolindentright=0pt,
    breaksymbolsepleft=0pt,
    breaksymbolsepright=0pt,
    fontsize=\footnotesize,
    bgcolor=LightGray,
    numbersep=10pt
}


\usepackage{listings} % Para incluir código desde un archivo

\renewcommand\lstlistingname{Código Fuente}
\renewcommand\lstlistlistingname{Índice de Códigos Fuente}

% Definir colores
\definecolor{vscodepurple}{rgb}{0.5,0,0.5}
\definecolor{vscodeblue}{rgb}{0,0,0.8}
\definecolor{vscodegreen}{rgb}{0,0.5,0}
\definecolor{vscodegray}{rgb}{0.5,0.5,0.5}
\definecolor{vscodebackground}{rgb}{0.97,0.97,0.97}
\definecolor{vscodelightgray}{rgb}{0.9,0.9,0.9}

% Configuración para el estilo de C similar a VSCode
\lstdefinestyle{vscode_C}{
  backgroundcolor=\color{vscodebackground},
  commentstyle=\color{vscodegreen},
  keywordstyle=\color{vscodeblue},
  numberstyle=\tiny\color{vscodegray},
  stringstyle=\color{vscodepurple},
  basicstyle=\scriptsize\ttfamily,
  breakatwhitespace=false,
  breaklines=true,
  captionpos=b,
  keepspaces=true,
  numbers=left,
  numbersep=5pt,
  showspaces=false,
  showstringspaces=false,
  showtabs=false,
  tabsize=2,
  frame=tb,
  framerule=0pt,
  aboveskip=10pt,
  belowskip=10pt,
  xleftmargin=10pt,
  xrightmargin=10pt,
  framexleftmargin=10pt,
  framexrightmargin=10pt,
  framesep=0pt,
  rulecolor=\color{vscodelightgray},
  backgroundcolor=\color{vscodebackground},
}

%------------------------------------------------------------------------

% Comandos definidos
\newcommand{\bb}[1]{\mathbb{#1}}
\newcommand{\cc}[1]{\mathcal{#1}}

% I prefer the slanted \leq
\let\oldleq\leq % save them in case they're every wanted
\let\oldgeq\geq
\renewcommand{\leq}{\leqslant}
\renewcommand{\geq}{\geqslant}

% Si y solo si
\newcommand{\sii}{\iff}

% Letras griegas
\newcommand{\eps}{\epsilon}
\newcommand{\veps}{\varepsilon}
\newcommand{\lm}{\lambda}

\newcommand{\ol}{\overline}
\newcommand{\ul}{\underline}
\newcommand{\wt}{\widetilde}
\newcommand{\wh}{\widehat}

\let\oldvec\vec
\renewcommand{\vec}{\overrightarrow}

% Derivadas parciales
\newcommand{\del}[2]{\frac{\partial #1}{\partial #2}}
\newcommand{\Del}[3]{\frac{\partial^{#1} #2}{\partial #3^{#1}}}
\newcommand{\deld}[2]{\dfrac{\partial #1}{\partial #2}}
\newcommand{\Deld}[3]{\dfrac{\partial^{#1} #2}{\partial #3^{#1}}}


\newcommand{\AstIg}{\stackrel{(\ast)}{=}}
\newcommand{\Hop}{\stackrel{L'H\hat{o}pital}{=}}

\newcommand{\red}[1]{{\color{red}#1}} % Para integrales, destacar los cambios.

% Método de integración
\newcommand{\MetInt}[2]{
    \left[\begin{array}{c}
        #1 \\ #2
    \end{array}\right]
}

% Declarar aplicaciones
% 1. Nombre aplicación
% 2. Dominio
% 3. Codominio
% 4. Variable
% 5. Imagen de la variable
\newcommand{\Func}[5]{
    \begin{equation*}
        \begin{array}{rrll}
            #1:& #2 & \longrightarrow & #3\\
               & #4 & \longmapsto & #5
        \end{array}
    \end{equation*}
}

%------------------------------------------------------------------------

\newcommand{\T}[0]{\cc{T}}

\begin{document}

    % 1. Foto de fondo
    % 2. Título
    % 3. Encabezado Izquierdo
    % 4. Color de fondo
    % 5. Coord x del titulo
    % 6. Coord y del titulo
    % 7. Fecha

    
    % 1. Foto de fondo
% 2. Título
% 3. Encabezado Izquierdo
% 4. Color de fondo
% 5. Coord x del titulo
% 6. Coord y del titulo
% 7. Fecha

\newcommand{\portada}[7]{

    \portadaBase{#1}{#2}{#3}{#4}{#5}{#6}{#7}
    \portadaBook{#1}{#2}{#3}{#4}{#5}{#6}{#7}
}

\newcommand{\portadaExamen}[7]{

    \portadaBase{#1}{#2}{#3}{#4}{#5}{#6}{#7}
    \portadaArticle{#1}{#2}{#3}{#4}{#5}{#6}{#7}
}




\newcommand{\portadaBase}[7]{

    % Tiene la portada principal y la licencia Creative Commons
    
    % 1. Foto de fondo
    % 2. Título
    % 3. Encabezado Izquierdo
    % 4. Color de fondo
    % 5. Coord x del titulo
    % 6. Coord y del titulo
    % 7. Fecha
    
    
    \thispagestyle{empty}               % Sin encabezado ni pie de página
    \newgeometry{margin=0cm}        % Márgenes nulos para la primera página
    
    
    % Encabezado
    \fancyhead[L]{\helv #3}
    \fancyhead[R]{\helv \nouppercase{\leftmark}}
    
    
    \pagecolor{#4}        % Color de fondo para la portada
    
    \begin{figure}[p]
        \centering
        \transparent{0.3}           % Opacidad del 30% para la imagen
        
        \includegraphics[width=\paperwidth, keepaspectratio]{assets/#1}
    
        \begin{tikzpicture}[remember picture, overlay]
            \node[anchor=north west, text=white, opacity=1, font=\fontsize{60}{90}\selectfont\bfseries\sffamily, align=left] at (#5, #6) {#2};
            
            \node[anchor=south east, text=white, opacity=1, font=\fontsize{12}{18}\selectfont\sffamily, align=right] at (9.7, 3) {\textbf{\href{https://losdeldgiim.github.io/}{Los Del DGIIM}}};
            
            \node[anchor=south east, text=white, opacity=1, font=\fontsize{12}{15}\selectfont\sffamily, align=right] at (9.7, 1.8) {Doble Grado en Ingeniería Informática y Matemáticas\\Universidad de Granada};
        \end{tikzpicture}
    \end{figure}
    
    
    \restoregeometry        % Restaurar márgenes normales para las páginas subsiguientes
    \pagecolor{white}       % Restaurar el color de página
    
    
    \newpage
    \thispagestyle{empty}               % Sin encabezado ni pie de página
    \begin{tikzpicture}[remember picture, overlay]
        \node[anchor=south west, inner sep=3cm] at (current page.south west) {
            \begin{minipage}{0.5\paperwidth}
                \href{https://creativecommons.org/licenses/by-nc-nd/4.0/}{
                    \includegraphics[height=2cm]{assets/Licencia.png}
                }\vspace{1cm}\\
                Esta obra está bajo una
                \href{https://creativecommons.org/licenses/by-nc-nd/4.0/}{
                    Licencia Creative Commons Atribución-NoComercial-SinDerivadas 4.0 Internacional (CC BY-NC-ND 4.0).
                }\\
    
                Eres libre de compartir y redistribuir el contenido de esta obra en cualquier medio o formato, siempre y cuando des el crédito adecuado a los autores originales y no persigas fines comerciales. 
            \end{minipage}
        };
    \end{tikzpicture}
    
    
    
    % 1. Foto de fondo
    % 2. Título
    % 3. Encabezado Izquierdo
    % 4. Color de fondo
    % 5. Coord x del titulo
    % 6. Coord y del titulo
    % 7. Fecha


}


\newcommand{\portadaBook}[7]{

    % 1. Foto de fondo
    % 2. Título
    % 3. Encabezado Izquierdo
    % 4. Color de fondo
    % 5. Coord x del titulo
    % 6. Coord y del titulo
    % 7. Fecha

    % Personaliza el formato del título
    \pretitle{\begin{center}\bfseries\fontsize{42}{56}\selectfont}
    \posttitle{\par\end{center}\vspace{2em}}
    
    % Personaliza el formato del autor
    \preauthor{\begin{center}\Large}
    \postauthor{\par\end{center}\vfill}
    
    % Personaliza el formato de la fecha
    \predate{\begin{center}\huge}
    \postdate{\par\end{center}\vspace{2em}}
    
    \title{#2}
    \author{\href{https://losdeldgiim.github.io/}{Los Del DGIIM}}
    \date{Granada, #7}
    \maketitle
    
    \tableofcontents
}




\newcommand{\portadaArticle}[7]{

    % 1. Foto de fondo
    % 2. Título
    % 3. Encabezado Izquierdo
    % 4. Color de fondo
    % 5. Coord x del titulo
    % 6. Coord y del titulo
    % 7. Fecha

    % Personaliza el formato del título
    \pretitle{\begin{center}\bfseries\fontsize{42}{56}\selectfont}
    \posttitle{\par\end{center}\vspace{2em}}
    
    % Personaliza el formato del autor
    \preauthor{\begin{center}\Large}
    \postauthor{\par\end{center}\vspace{3em}}
    
    % Personaliza el formato de la fecha
    \predate{\begin{center}\huge}
    \postdate{\par\end{center}\vspace{5em}}
    
    \title{#2}
    \author{\href{https://losdeldgiim.github.io/}{Los Del DGIIM}}
    \date{Granada, #7}
    \thispagestyle{empty}               % Sin encabezado ni pie de página
    \maketitle
    \vfill
}
    \portadaExamen{ffccA4.jpg}{Topología I\\Examen VII}{Topología I. Examen VII}{MidnightBlue}{-8}{28}{2023-2024}{Arturo Olivares Martos}

    \begin{description}
        \item[Asignatura] Topología I.
        \item[Curso Académico] 2023-24.
        \item[Grado] Grado en Matemáticas.
        \item[Grupo] A.
        \item[Profesor] Leonor Ferrer Martínez.
        \item[Descripción] Primer Parcial.
        \item[Fecha] 9 de noviembre de 2023.
        %\item[Duración] 3 horas.
    
    \end{description}
    \newpage
    
    Para cada $x\in \bb{R}$, y cada $\veps\in \bb{R}^+$, se define
    el subconjunto:
    \begin{equation*}
        G(x,\veps) = \{x\} ~\bigcup~ \left(]x-\veps, x+\veps[~\cap \bb{Q}\right)
    \end{equation*}

    En $\bb{R}$ consideramos ahora para cada $x\in \bb{R}$ la familia de subconjuntos:
    \begin{equation*}
        \beta_x = \{G(x,\veps) \mid \veps\in \bb{R}^+\}
    \end{equation*}
    
    \begin{ejercicio}[4 puntos]
        Demuestra que $\beta_x$ es base de entornos para una única topología $\T$
        en $\bb{R}$. Prueba que los subconjuntos $G(x,\veps)$ son abiertos de la topología $\T$.\\

        Para ello, hemos de comprobar las 4 hipótesis del correspondiente teorema. Dado $x\in \bb{R}$, tenemos que:
        \begin{enumerate}
            \item[V1)] $\beta_x\neq \emptyset$, pues $G(x,1)\in \beta_x$, por ejemplo.
            \item[V2)] Sea $G(x,\veps) \in \beta_x$. Trivialmente, se tiene que $x\in G(x,\veps)$ por la definición de $G(x,\veps)$.
            \item[V3)] Sean $G(x,\veps_1), G(x,\veps_2)\in \beta_x$. Buscamos $\veps_3\in \bb{R}^+$ tal que $G(x,\veps_3)\subset G(x,\veps_1)\cap G(x,\veps_2)$.
            Sea $\veps_3 = \min\{\veps_1,\veps_2\}$ y se tiene de forma directa.
            \item[V4)] Sea $G(x,\veps)\in \beta_x$. Buscamos $\veps'\in \bb{R}^+$ tal que $G(x,\veps')\subset G(x,\veps)$ y, para todo $y\in G(x,\veps')$, buscamos $\veps''\in \bb{R}^+$ tal que $G(y,\veps'')\subset G(x,\veps)$.
            
            Sea $\veps' = \veps$, y la primera inclusión es trivial.
            Sea ahora $y\in G(x,\veps)\subset]x-\veps,x+\veps[$. Como $]x-\veps,x+\veps[$ es abierto en $(\bb{R},\T)$,
            tenemos que $\exists \veps''\in \bb{R}^+$ tal que $]y-\veps'',y+\veps''[~\subset~]x-\veps,x+\veps[$.
            Por tanto, como $y\in G(x,\veps)$ tenemos que $G(y,\veps'')\subset G(x,\veps)$.
        \end{enumerate}

        Por tanto, $\beta_x$ es base de entornos para una única topología $\T$ en $\bb{R}$.
        Veamos ahora que $G(x,\veps)$ es abierto de la topología $\T$. Para ello, veremos que $G(x,\veps)$ es entorno de cada uno de sus puntos.

        Sea $y\in G(x,\veps)$. Hemos visto que $\exists \veps'\in \bb{R}^+$ tal que $G(y,\veps')\subset G(x,\veps)$. Como $G(y,\veps')\in N_y$ y $G(y,\veps')\subset G(x,\veps)$, tenemos que $G(x,\veps)\in N_y$. Por tanto, $G(x,\veps)$ es entorno de cada uno de sus puntos, y por tanto es abierto de la topología $\T$.
    \end{ejercicio}

    \begin{ejercicio}[1 punto]
        Da una base de $\T$.\\

        Sea el siguiente conjunto, y demostraremos que es base de $\T$:
        \begin{equation*}
            \cc{B} = \{G(x,\veps)\mid x\in \bb{R}, \veps\in \bb{R}^+\}
        \end{equation*}

        En primer lugar, por lo visto al final del ejercicio anterior, tenemos que $\cc{B}\subset \T$.
        Veamos ahora que, para todo $U\in \cc{\T}$ y para todo $x\in U$, existe $B\in \cc{B}$ tal que $x\in B\subset U$.
        Como $x\in U\in \T$, entonces $U\in N_x$, y como $\beta_x$ es una base de enornos de $x$, entonces $\exists B\in \beta_x$ tal que $x\in B\subset U$.
    \end{ejercicio}

    \begin{ejercicio}[1 punto]
        Prueba que $\T_u\subsetneq \T$. ¿Es el espacio $(\bb{R},\T)$ Haussdorf?

        \begin{description}
            \item[$\subset)$] Sea $U\in \T_u$. Entonces, veamos que $U$ es entorno de todos sus puntos en $\T$.
            Sea $x\in U$. Como $U\in \T_u$, entonces $\exists \veps\in \bb{R}^+$ tal que $]x-\veps,x+\veps[\subset U$.
            Por tanto
            \begin{equation*}
                G(x,\veps) = \{x\} ~\bigcup~ \left(]x-\veps, x+\veps[~\cap \bb{Q}\right) \subset ]x-\veps,x+\veps[\subset U
            \end{equation*}
            Como $x\in G(x,\veps)\subset U$, y $G(x,\veps)\in N_x$, entonces $U\in N_x$ en $\T$, y por tanto $U\in \T$. 
        \end{description}

        Para ver que $\T_u\neq \T$, sea $G(0,1)\in \T_u$.
        \begin{equation*}
            G(0,1) = \bb{Q}\cap ]-1,1[
        \end{equation*}

        Como $0\in G(0,1)$, pero $\nexists \veps\in \bb{R}^+$ tal que $B(0,\veps)\subset G(0,1)$, entonces $G(0,1)\notin N_0$ en $\T_u$, y por tanto $G(0,1)\notin \T_u$.

        Veamos ahora que $(\bb{R},\T)$ es T2. Dados $x,y\in \bb{R}$, $x\neq y$,
        como $(\bb{R},\T_u)$ es T2, $\exists U,V\in \T_u\subset \T$, con $x\in U,~y\in V$ y $U\cap V=\emptyset$.  
        Por tanto, $(\bb{R},\T)$ sí es T2.
    \end{ejercicio}

    \begin{ejercicio}[1 punto]
        Calcula la clausura de $G(x,\veps)$ en $(\bb{R},\T)$.\\

        Veamos que $\ol{G(x,\veps)}=[x-\veps, x+\veps]$.
        \begin{description}
            \item[$\subset)$] Claramente, $G(x,\veps)\subset [x-\veps, x+\veps]$,
            y $[x-\veps, x+\veps]\in C_{\T_u}\subset C_{\T}$. Por tanto,
            \begin{equation*}
                \ol{G(x,\veps)}\subset [x-\veps, x+\veps]
            \end{equation*}
            
            \item[$\supset)$] Sea $y\in [x-\veps, x+\veps]$. Veamos que $y\in \ol{G(x,\veps)}$. Basta comprobar
            que, para todo $\delta\in \bb{R}^+$, se tiene que $G(y,\delta)\cap G(x,\veps)\neq \emptyset$.

            Como $y\in \ol{B(x,\veps)}$, en $(\bb{R},\T_u)$ tenemos que $B(x,\veps)\cap B(y,\delta)\in \T_u$ por ser intersección de abiertos.
            Por la densidad de $\bb{Q}$ en $\bb{R}$, tenemos que:
            \begin{equation*}
                B(x,\veps)\cap B(y,\delta)\cap \bb{Q} =~ ]x-\veps,x+\veps[~\cap~]y-\delta,y+\delta[~\cap \bb{Q}~\neq \emptyset
                \qquad \forall \delta\in \bb{R}^+
            \end{equation*}

            Por tanto,
            \begin{equation*}
                \emptyset \neq ~]x-\veps,x+\veps[~\cap~]y-\delta,y+\delta[~\cap \bb{Q} \subset G(x,\veps)\cap G(y,\delta)
                \qquad \forall \delta\in \bb{R}^+
            \end{equation*}
            y tenemos entonces que $y\in \ol{G(x,\veps)}$.
        \end{description}
    \end{ejercicio}

    \begin{ejercicio}[1 punto]
        Describe la topología $\T_{\bb{R}\setminus \bb{Q}}$.\\

        Veamos que, para todo $x\in \bb{R}\setminus \bb{Q}$, se tiene que $\{x\}\in \T_{\bb{R}\setminus \bb{Q}}$.
        \begin{equation*}
            G(x,\veps)\cap \bb{R}\setminus \bb{Q} =
            \left(\{x\}\cup ]x-\veps,x+\veps[~\cap \bb{Q}\right)\cap \bb{R}\setminus \bb{Q} = \{x\} \cap \bb{R}\setminus \bb{Q} = \{x\}
        \end{equation*}
        
        Por tanto, como $G(x,\veps)\in \T$, entonces $G(x,\veps)\cap \bb{R}\setminus \bb{Q}\in \T_{\bb{R}\setminus \bb{Q}}$. Por tanto, $\{x\}\in \T_{\bb{R}\setminus \bb{Q}}$.
        Como la unión de abiertos es abierta, entonces:
        \begin{equation*}
            \T_{\bb{R}\setminus \bb{Q}} = \{\{x\}\mid x\in \bb{R}\setminus \bb{Q}\} = \cc{P}(\bb{R}\setminus \bb{Q}) = {\T_{disc}}_{\big| \bb{R}\setminus \bb{Q}}
        \end{equation*}

        Por tanto, tenemos que es la topología discreta en $\bb{R}\setminus \bb{Q}$.
    \end{ejercicio}

    \begin{ejercicio}[2 puntos]
        Estudia si $(\bb{R},\T)$ es 1AN o 2AN.\\

        Veamos que $(\bb{R},\T)$ es 1AN. Para todo $x\in \bb{R}$, veamos que el siguiente conjunto es una base de entornos de $x$:
        \begin{equation*}
            \beta_x' = \left\{G\left(x,\frac{1}{n}\right)\mid n\in \bb{N}\right\}
        \end{equation*}
        Como $\nicefrac{1}{n}\in \bb{R}$ para todo $n\in \bb{N}$, entonces $\beta_x'\subset \beta_x$. Veamos ahora que, para todo $\veps\in \bb{R}^+$, existe $n\in \bb{N}$ tal que $G\left(x,\nicefrac{1}{n}\right)\subset G(x,\veps)$.
        Esto es cierto ya que consideramos $n\in \bb{N}$ tal que $\nicefrac{1}{n}<\veps$, o equivalentemente $n>\nicefrac{1}{\veps}$. Esto siempre es posible ya que $\left\{\nicefrac{1}{n}\right\}\to 0$.

        Por tanto, $\beta_x'$ es base de entornos de $x$ numerable, y por tanto $(\bb{R},\T)$ es 1AN.\\

        Veamos ahora que no es 2AN. Por reducción al absurdo, supongamos que $(\bb{R},\T)$ es 2AN. Entonces, como
        esta propiedad es hereditaria, $(\bb{R}\setminus \bb{Q},\T_{\bb{R}\setminus \bb{Q}})$ sería 2AN. No obstante, su base más económica es
        $\cc{B}'=\{\{x\}\mid x\in \bb{R}\setminus \bb{Q}\}$, que claramente no es numerable por no serlo $\bb{R}\setminus \bb{Q}$.
        Por tanto, llegamos a un absurdo, y $(\bb{R},\T)$ no es 2AN.
    \end{ejercicio}

    \begin{ejercicio}[1 punto extra]
        Prueba que para toda sucesión $\{x_n\}_{n\in \bb{N}}\subset \bb{R}\setminus \bb{Q}$
        que converja a un punto $x_0\in \bb{R}$, existe $n_0\in \bb{N}$ tal que
        $x_n=x_0$ para todo $n\geq n_0$, y por tanto $x_0\in \bb{R}\setminus \bb{Q}$.\\

        Como la sucesión converge a $x_0$, entonces dado $\veps\in \bb{R}^+$, $\exists n_0\in \bb{N}$ tal que $x_n\in G(x_0,\veps)$ para todo $n\geq n_0$.
        No obstante, $x_n\in \bb{R}\setminus \bb{Q}$ para todo $n\in \bb{N}$, y además $G(x_0,\veps)\setminus \{x_0\}\subset \bb{Q}$,
        por lo que $x_n=x_0$ para todo $n\geq n_0$.
        
        Además, como $x_n\in \bb{R}\setminus \bb{Q}$ para todo $n\in \bb{N}$, entonces $x_0\in \bb{R}\setminus \bb{Q}$.
    \end{ejercicio}

\end{document}