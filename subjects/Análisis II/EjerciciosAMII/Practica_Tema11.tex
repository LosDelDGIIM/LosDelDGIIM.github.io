\section{Cálculo de Integrales Simples}

\begin{ejercicio}
    En cada uno de los siguientes casos, probar que la función $f$ es integrable en el
    intervalo $J$ y calcular su integral:
    \begin{enumerate}
        \item $f(x)=x^2\ln x \qquad \forall x\in J=\left]0,1\right[$.
        \item $f(x)=e^{-x}\cos(2x) \qquad \forall x\in J=\bb{R}^+$.
        \item $f(x)=\dfrac{1}{x^4-1} \qquad \forall x\in J=\left]2,+\infty\right[$.
        \item $f(x)=\dfrac{1}{e^x+e^{-x}} \qquad \forall x\in J=\bb{R}$.
        \item $f(x)=\dfrac{1}{x^2 + \sqrt{x}} \qquad \forall x\in J=\left]0,1\right[$.
        \item $f(x)=\dfrac{1}{x^2\sqrt{1+x^2}} \qquad \forall x\in J=\left]1,+\infty\right[$.  
        \item $f(x)=\dfrac{1}{1+\cos x + \sen x} \qquad \forall x\in J=\left]0,\dfrac{\pi}{2}\right[$.
        \item $f(x)=\dfrac{1}{x^3\sqrt{x^2-1}} \qquad \forall x\in J=\left]1,+\infty\right[$.
    \end{enumerate}
\end{ejercicio}



\begin{ejercicio}
    En cada uno de los siguientes casos, estudiar la integrabilidad de la función $f$ en
    el intervalo $J$:
    \begin{enumerate}
        \item $f(x)=\dfrac{x^2}{e^x-1}\qquad \forall x\in J=\bb{R}^+.\hspace{1cm}(a\in \bb{R})$
        \item $f(x)=x^ne^{-x^2}\cos x \qquad \forall x\in J=\bb{R}.\hspace{1cm}(n\in \bb{N})$
        \item $f(x)=\dfrac{x^\rho}{1-\cos x} \qquad \forall x\in J=\left]0,\pi\right[.\hspace{1cm}(\rho\in \bb{R})$
        \item $f(x)=\dfrac{x^a(1-x)^b~\ln(1+x^2)}{(\ln x)^2} \qquad \forall x\in J=\left]0,1\right[.\hspace{1cm}(a,b\in \bb{R})$
    \end{enumerate}
\end{ejercicio}