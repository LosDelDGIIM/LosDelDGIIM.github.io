\section{Cálculo de Integrales Simples}

\subsection{Repaso teórico}
Repasamos ahora los conceptos teóricos necesarios para realizar el cálculo de integrales simples. Destacamos 4 reglas que nos ayudan a resolver el cálculo de integrales, todas ellas tendrán su versión elemental\footnote{Las cuales ya se vieron en Cálculo II.} y su versión general.

\subsubsection{Regla de Barrow}
\begin{teo}[Regla de Barrow versión elemental]\ \\
    Si $f:J\rightarrow\mathbb{R}$ es una función continua y $G$ una primitiva de $f$, se tiene:
    \begin{equation*}
        \int_{a}^{b} f(x)~dx  = G(b) - G(a) = {[G(x)]}_a^b\qquad \forall a,b\in J
    \end{equation*}
\end{teo}

\begin{teo}[Versión general de la Regla de Barrow]\ \\
    Si $f\in \cc{L}_1(J)$ y $G:J\rightarrow\mathbb{R}$ es una primitiva de $f$, entonces $G$ tiene límite, tanto en $\alpha$ como el $\beta$, y se verifica que:
    \begin{equation*}
        \displaystyle\int_{\alpha}^{\beta} f(t)~dt = \lim_{x\to\beta} G(x) -\lim_{x\to\alpha} G(x) = {[G(x)]}_\alpha^\beta
    \end{equation*}
\end{teo}
Notemos que ninguno de los dos teoremas anteriores nos permite averiguar si $f$ es integrable o no, ya que suponen que lo es para llegar a la tesis. A continuación, vemos un criterio que nos permite comprobar esto, como consecuencia de la versión general de la regla de Barrow.

\begin{teo}[Criterio de integrabilidad]\ \\
    Dada una función $f:J\rightarrow\red{\mathbb{R}^+_0}$, sea $G$ una primitiva de $f$. Entonces $f\in \cc{L}_1$ si, y sólo si, $G$ tiene límite en $\alpha$ y $\beta$, en cuyo caso:
    \begin{equation*}
        \displaystyle\int_{\alpha}^{\beta} f(t) ~dt  = {[G(x)]}_\alpha^\beta
    \end{equation*}
\end{teo}
Notemos que sólo es válido para funciones con codominio $\mathbb{R}^+_0$. Sin embargo, si tenemos una función con imagen negativa $f$ de forma que $-f$ (función con imagen positiva) cumpla las hipótesis del criterio, $-f$ será integrable, luego $f$ también. De esta forma, si tenemos una función que pasa de ser positiva a negativa (o viceversa) un número finito de veces, podemos, en cada trozo donde el signo de su imagen es constante, aplicar el criterio, obteniendo que la función es integrable (en caso de que cada uno de sus ``trozos'' cumpla con las hipótesis del criterio).

\begin{ejercicio*}
    Dados $s\in \mathbb{R}$ y $c\in \mathbb{R}^+$, la función $x\rightarrow x^s$ es:
    \begin{itemize}
        \item $\cc{L}_1^{loc}$.
        \item integrable en $]0,c[$ si, y sólo si, $s > -1$.
        \item integrable en $]c,+\infty[$ si, y sólo si, $s < -1$.
    \end{itemize}
\end{ejercicio*}

\subsubsection{Criterio de comparación}
\begin{prop}[Criterio de comparación por paso al límite]\ \\
    Sea $I=[a, \beta[$ con $a\in \mathbb{R}$ y $a<\beta \leq +\infty$ y $f,g\in \cc{L}_1^{loc}(I)$ con $g(x)\neq 0$ para todo $x\in I$
    \begin{itemize}
        \item Si $\lim\limits_{x\to\beta}\dfrac{|f(x)|}{|g(x)|} = L \in \mathbb{R}^+ $, entonces $f\in \cc{L}_1(I) \Longleftrightarrow g\in \cc{L}_1(I)$
        \item Si $\lim\limits_{x\to\beta} \dfrac{|f(x)|}{|g(x)|} = 0 $, entonces $g\in \cc{L}_1I()\Longrightarrow f\in \cc{L}_1(I)$
        \item Si $\dfrac{|f(x)|}{|g(x)|}\rightarrow +\infty (x\rightarrow\beta)$, entonces: $g\notin \cc{L}_1(I) \Longrightarrow f\notin\cc{L}_1(I)$
    \end{itemize}
    En el caso $I=\left]\alpha,b\right]$ con $b\in \mathbb{R}$ y $-\infty\leq \alpha < b$, se verifica el resultado análogo, con $\alpha$ en lugar de $\beta$.
\end{prop}

\subsubsection{Integración por partes}
\begin{teo}[Fórmula de integración por partes versión elemental]\ \\
    Si $F,G:J\rightarrow\mathbb{R}$ son funciones de clase $C^1$ en $J$, se tiene:
    \begin{equation*}
        \displaystyle\int_{a}^{b} F(t)G'(t)~dt  = {[F(x)G(x)]}_a^b - \displaystyle\int_{a}^{b} F'(t)G(t)~dt  \qquad a,b\in J
    \end{equation*}
\end{teo}

A continuación, destacamos el resultado general, que nombramos como ``primera versión''.

\begin{teo}[Fórmula de integración por partes (primera versión general)]\ \\
    Dadas $F,G:J\rightarrow\mathbb{R}$, supongamos que, para cada intervalo compacto $K\subset J$ las restricciones $F_{\big|K}$ y $G_{\big|K}$ son absolutamente continuas. Entonces, $FG'$ y $GF'$ son localmente integrables en $J$ con:
    \begin{equation*}
        \displaystyle\int_{a}^{b} F(t)G'(t)~dt = {[F(x)G(x)]}_a^b - \displaystyle\int_{a}^{b} F'(t)G(t)~dt \qquad a,b\in J
    \end{equation*}
\end{teo}

Como podemos ver, pese a ser satisfactorio teóricamente, no nos aporta utilidad práctica. Por ello, destacamos el siguiente teorema, más débil pero de gran utilidad práctica.

\begin{teo}[Fórmula de Integración por partes (segunda versión general)]\ \\
    Sean $F,G:J\rightarrow\mathbb{R}$ dos funciones derivables en $J=\left]\alpha,\beta\right[$, tales que $F'G$ y $FG'$ son integrables en $J$. Entonces, $FG$ tiene límite, tanto en $\alpha$ como en $\beta$, y se verifica que:
    \begin{equation*}
        \displaystyle\int_{\alpha}^{\beta} F(t)G'(t)~dt = {[F(x)G(x)]}_\alpha^\beta - \displaystyle\int_{\alpha}^{\beta} F'(t)G(t)~dt 
    \end{equation*}
\end{teo}

\subsubsection{Cambio de variable}
\begin{teo}[Cambio de variable versión elemental]\ \\
    Dados dos intervalos no triviales $I,J\subset \mathbb{R}$, sea $\varphi:I\rightarrow J$ una función de clase $C^1$ en $I$ y $f:J\rightarrow\mathbb{R}$ una función continua. Entonces:
    \begin{equation*}
        \displaystyle\int_{\varphi(a)}^{\varphi(b)} f(x)~dx = \displaystyle\int_{a}^{b} f(\varphi(t))\varphi'(t)~dt\qquad a,b\in I 
    \end{equation*}
\end{teo}

\begin{teo}[Versión general de la fórmula del cambio de variable]\ \\
    Dados dos intervalos no triviales $I,J\subset \mathbb{R}$, sea $\varphi:I\rightarrow J$ tal que $\varphi_{\big|H}$ es absolutamente continua, para todo intervalo compacto $H\subset I$.\newline
    Si $f:I\rightarrow\mathbb{R}$ es localmente integrable en $J$ y verifica que $(f\circ \varphi)\varphi'$ es localmente integrable en $I$, entonces:
    \begin{equation*}
        \displaystyle\int_{\varphi(a)}^{\varphi(b)} f(x)~dx = \displaystyle\int_{a}^{b} f(\varphi(t))\varphi'(t)~dt\qquad \forall a,b\in I
    \end{equation*}
\end{teo}

Sin embargo, al igual que sucedía antes, carece de utilidad práctica. Vemos ahora otro segundo teorema que podemos usar, que además caracteriza la integabilidad de la función $f$ con la función $(f\circ \varphi)\varphi'$. Sin embargo, es menos débil que el anterior, como antes sucedía.

\begin{teo}[Teorema de cambio de variable]\ \\
    Dado un intervalo abierto no vacío $I\subset \mathbb{R}$, sea $\varphi:I\rightarrow\mathbb{R}$ una función de clase $C^1$ en $I$, con $\varphi'(t)\neq 0$ para todo $t\in I$, y sea $J=\varphi(I)$.\newline
    Entonces, una función $f:J\rightarrow\mathbb{R}$ es integrable en $J$ si, y sólo si, $(f\circ \varphi)\varphi'$ es integrable en $I$, en cuyo caso:
    \begin{equation*}
        \int_J f = \int_I (f\circ\varphi)|\varphi'|
    \end{equation*}
\end{teo}

Este teorema suele usarse de la siguiente forma:\newline
$I=\left]\alpha,\beta\right[$ con $-\infty\leq \alpha<\beta\leq +\infty$ y $J=\left]\gamma,\delta\right[$ con $-\infty\leq \gamma<\delta\leq +\infty$\newline
$\{\gamma,\delta\} = \{\wt{\alpha},\wt{\beta}\}$ con $\varphi(t)\to\wt{\alpha}$ $(t\to\alpha)$ y $\varphi(t)\to\wt{\beta}$ $(t\to\beta)$\newline
Entonces:
\begin{equation*}
    \displaystyle\int_{\wt{\alpha}}^{\wt{\beta}} f(x)~dx = \displaystyle\int_{\alpha}^{\beta} f(\varphi(t))\varphi'(t)~dt 
\end{equation*}

\newpage
\subsection{Ejercicios}
\begin{ejercicio}
    En cada uno de los siguientes casos, probar que la función $f$ es integrable en el intervalo $J$ y calcular su integral:
    \begin{enumerate}
        \item $f(x)=x^2\ln x \qquad \forall x\in J=\left]0,1\right[$.
        
        Para buscar una primitiva de $f$ y así poder demostrar que $f\in \cc{L}_1(J)$, usamos la integración por partes.
        Supongamos entonces que $f$ es integrable en $J$ (algo que demostraremos).
        Sean $F,G:J\to \bb{R}$ las funciones dadas por:
        \begin{equation*}
            F(x)=\ln x \hspace{1cm} G(x)=\dfrac{x^3}{3}
        \end{equation*}

        Tenemos que $F,G$ son derivables en $J$, con:
        \begin{equation*}
            F'(x)=\dfrac{1}{x} \hspace{1cm} G'(x)=x^2
        \end{equation*}

        Tenemos que $FG'=f\in \cc{L}_1(J)$ por hipótesis, y $F'G=\frac{x^2}{3}\in \cc{L}_1(J)$ por ser una función continua en $\ol{J}$.
        Por tanto, por el Teorema de Integración por Partes, tenemos que:
        \begin{equation*}
            \int_0^1 x^2\ln x~dx
            = \left[\dfrac{x^3}{3}\ln x\right]_0^1 - \int_0^1 \dfrac{x^3}{3}\dfrac{1}{x}~dx
            \AstIg \frac{1}{3}\left[x^3\ln x-\frac{x^3}{3}\right]_0^1
            = \frac{1}{3}\left[-\frac{1}{3}\right] = -\frac{1}{9}
        \end{equation*}
        donde en $(\ast)$ hemos empleado la versión elemental de la Regla de Barrow. Además, en la última igualdad, hemos usado que:
        \begin{equation*}
            \lim_{x\to 0} x^3\ln x = \lim_{x\to 0} \dfrac{\ln x}{x^{-3}} \Hop \lim_{x\to 0} \dfrac{\frac{1}{x}}{-3x^{-4}} = \lim_{x\to 0} -\dfrac{x^3}{3} = 0
        \end{equation*}~\\

        Probaremos ahora que $f\in \cc{L}_1(J)$.
        Probemos que $G$ es una primitiva de $f$, donde:
        \Func{G}{J}{\bb{R}}{x}{\frac{1}{3}\left[x^3\ln x-\frac{x^3}{3}\right]}
        
        Tenemos que $G$ es derivable en $J$, con:
        \begin{equation*}
            G'(x) = \frac{1}{3}\left[3x^2\ln x + x^2 - x^2\right] = x^2\ln x = f(x) \qquad \forall x\in J
        \end{equation*}

        Por tanto, tenemos que $G$ es una primitiva de $f$, y hemos visto que $G$ tiene límite en $0$ y $1$.
        Empleando el Criterio de Integrabilidad de la Regla de Barrow para $-f$, tenemos que $f\in \cc{L}_1(J)$, como queríamos demostrar.

        \begin{observacion}
            Notemos que hemos tenido que suponer que $f$ era integrable para poder aplicar el método de integración por partes y poder así hallar
            la primitiva. Una vez hallada, hemos demostrado que efectivamente era su primitiva, con lo que hemos podido demostrar que $f$ era integrable.
        \end{observacion}

        \item $f(x)=e^{-x}\cos(2x) \qquad \forall x\in J=\bb{R}^+$.
        
        Sabiendo que la función $x\mapsto e^{-x}$ es interable en $\bb{R}^+$, tenemos que:
        \begin{equation*}
            \int_{\bb{R}} |f| \leq \int_{\bb{R}} e^{-x} < +\infty
        \end{equation*}

        Por tanto, tenemos que $f\in \cc{L}_1(\bb{R}^+)$. Para calcular su integral, usamos la fórmula de integración por partes.
        Sean $F,G:\bb{R}^+\to \bb{R}$ las funciones dadas por:
        \begin{equation*}
            F(x) = \cos(2x) \hspace{1cm} G(x) = -e^{-x}
        \end{equation*}

        Tenemos que $F,G$ son derivables en $\bb{R}^+$, con:
        \begin{equation*}
            F'(x) = -2\sen(2x) \hspace{1cm} G'(x) = e^{-x}
        \end{equation*}

        Tenemos que $FG' = f\in \cc{L}_1(\bb{R}^+)$ como hemos visto antes, y $F'G = 2e^{-x}\sen(2x)\in \cc{L}_1(\bb{R}^+)$ por el mismo razonamiento que $f$.
        Por tanto, por la fórmula de integración por partes, tenemos que:
        \begin{align*}
            \int_0^{+\infty} e^{-x}\cos(2x)~dx
            &= \left[-e^{-x}\cos(2x)\right]_0^{+\infty} - \int_0^{+\infty} -2e^{-x}\sen(2x)~dx
            =\\&= \left[-e^{-x}\cos(2x)\right]_0^{+\infty} + 2\int_0^{+\infty} e^{-x}\sen(2x)~dx
        \end{align*}

        Usamos ahora la fórmula de integración por partes para la integral que nos queda.
        Sean $G_1=G$ y $F_1:\bb{R}^+\to \bb{R}$ la función dada por $F_1(x) = \sen(2x)$.
        Tenemos que $F_1,G_1$ son derivables en $\bb{R}^+$, con:
        \begin{equation*}
            F_1'(x) = 2\cos(2x) \hspace{1cm} G_1'(x) = -e^{-x}
        \end{equation*}

        Tenemos entonces que:
        \begin{align*}
            \int_0^{+\infty} e^{-x}\cos(2x)~dx
            &= \left[-e^{-x}\cos(2x)\right]_0^{+\infty} + 2\int_0^{+\infty} e^{-x}\sen(2x)~dx
            =\\&= \left[-e^{-x}\cos(2x)\right]_0^{+\infty} + 2\left[-e^{-x}\sen(2x)\right]_0^{+\infty} + 4\int_0^{+\infty}e^{-x}\cos(2x)~dx
        \end{align*}

        % // TODO: Continuar


        \item $f(x)=\dfrac{1}{x^4-1} \qquad \forall x\in J=\left]2,+\infty\right[$.
        \item $f(x)=\dfrac{1}{e^x+e^{-x}} \qquad \forall x\in J=\bb{R}$.
        \item $f(x)=\dfrac{1}{x^2 + \sqrt{x}} \qquad \forall x\in J=\left]0,1\right[$.
        \item $f(x)=\dfrac{1}{x^2\sqrt{1+x^2}} \qquad \forall x\in J=\left]1,+\infty\right[$.  
        \item $f(x)=\dfrac{1}{1+\cos x + \sen x} \qquad \forall x\in J=\left]0,\dfrac{\pi}{2}\right[$.
        \item $f(x)=\dfrac{1}{x^3\sqrt{x^2-1}} \qquad \forall x\in J=\left]1,+\infty\right[$.
    \end{enumerate}
\end{ejercicio}

\begin{ejercicio}
    En cada uno de los siguientes casos, estudiar la integrabilidad de la función $f$ en el intervalo $J$:
    \begin{enumerate}
        \item $f(x)=\dfrac{x^2}{e^x-1}\qquad \forall x\in J=\bb{R}^+.\hspace{1cm}(a\in \bb{R})$
        \item $f(x)=x^n e^{-x^2}\cos x \qquad \forall x\in J=\bb{R}.\hspace{1cm}(n\in \bb{N})$
        \item $f(x)=\dfrac{x^\rho}{1-\cos x} \qquad \forall x\in J=\left]0,\pi\right[.\hspace{1cm}(\rho\in \bb{R})$
        \item $f(x)=\dfrac{x^a{(1-x)}^b~\ln(1+x^2)}{{(\ln x)}^2} \qquad \forall x\in J=\left]0,1\right[.\hspace{1cm}(a,b\in \bb{R})$
    \end{enumerate}
\end{ejercicio}
