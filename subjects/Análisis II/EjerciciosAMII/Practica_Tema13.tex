\section{Teorema de Fubini}

\begin{ejercicio}
    Probar que el siguiente conjunto es medible y calcular su área.
    \[ E = \left\{ (x, y) \in \mathbb{R}^2 \mid 0 \leq x \leq \min \left\{ e^y , 1 , e^{1-y} \right\} \right\} \subset \mathbb{R}^2 \]

    Tenemos que $E$ es la intersección de tres cerrados, luego $E$ es cerrado y, por tanto, $E\in \cc{M}_2$.
    Para calcular el área de $E$, en primer lugar veamos cuál es el mímino de las tres funciones que definen $E$.
    Tenemos que:
    \begin{align*}
        e^y \leq 1 &\iff e^y \leq e^0 \iff y \leq 0 \\
        e^y \leq e^{1-y} &\iff y \leq 1-y \iff y \leq \nicefrac{1}{2} \\
        e^{1-y} \leq 1 &\iff e^{1-y} \leq e^0 \iff 1-y \leq 0 \iff y \geq 1
    \end{align*}

    Entonces, tenemos que:
    \begin{itemize}
        \item \ul{Si $y \leq 0$}: $e^y \leq 1\leq e^{1-y}$, luego $\min \left\{ e^y , 1 , e^{1-y} \right\} = e^y$.
        \item \ul{Si $0 \leq y \leq \nicefrac{1}{2}$}: $1\leq e^y \leq e^{1-y}$, luego $\min \left\{ e^y , 1 , e^{1-y} \right\} = 1$.
        \item \ul{Si $\nicefrac{1}{2} \leq y \leq 1$}: $1\leq e^{1-y} \leq e^y$, luego $\min \left\{ e^y , 1 , e^{1-y} \right\} = 1$.
        \item \ul{Si $1\leq y$}: $1\leq e^{1-y} \leq e^{y}$, luego $\min \left\{ e^y , 1 , e^{1-y} \right\} = e^{1-y}$.
    \end{itemize}

    Por tanto, definimos $f:\bb{R} \to \bb{R}_0^+$ como:
    \[ f(y) = \begin{cases}
        e^y & \text{si } y \leq 0 \\
        1 & \text{si } 0 \leq y \leq 1 \\
        e^{1-y} & \text{si } 1 \leq y
    \end{cases} \]

    Mediante la función $f$ podemos reescribir $E$ como:
    \[ E = \left\{ (x, y) \in \mathbb{R}^2 \mid 0 \leq x \leq f(y) \right\}\in \cc{M}_2 \]

    Visualmente, el conjunto $E$ es el siguiente:
    \begin{figure}[H]
        \centering
        \begin{tikzpicture}
            \begin{axis}[
                domain=-4:6, xmin=-1, xmax=4, ymin=-4, ymax=4,
                samples=100,
                axis lines=center,
                xlabel={$x$},
                ylabel={$y$},
                clip=true,
                axis equal,
                legend cell align=left
            ]
                % Dibuja la curva de y = \ln(x)
                \addplot [name path=A, blue, thick] {ln(x)};
                \addlegendentry{$x=e^y$}

                % Dibuja la curva de y = 1-\ln(x)
                \addplot [name path=B, red, thick] {1-ln(x)};
                \addlegendentry{$x=e^{1-y}$}
            
                % Dibuja la línea vertical en x=1
                \addplot [name path=C, teal, thick] (1,x);
                \addlegendentry{$x=1$}
            
                % Dibuja la línea vertical en x=0
                % \path [name path=D] (axis cs:0,-4) -- (axis cs:0,4);
            
                % Rellena el área bajo la curva entre x=0 y x=1
                \addplot [
                    thick,
                    color=orange,
                    fill=orange,
                    fill opacity=0.4
                ]
                fill between [
                    of=A and B,
                    soft clip={domain=0:1},
                ];
                \addlegendentry{$E$}
            \end{axis}
          \end{tikzpicture}
    \end{figure}

    Como $E$ es la subgráfica de $f$ (función continua, luego medible), entonces el área de $E$ es:
    \begin{equation*}
        \lm_2(E) = \int_{-\infty}^{\infty} f(y)~dy
    \end{equation*}

    Usando la aditividad de la integral, tenemos que:
    \begin{align*}
        \lm_2(E) &= \int_{-\infty}^0 e^y~dy + \int_0^1 1~dy + \int_1^{\infty} e^{1-y}~dy \\
        &= \left[ e^y \right]_{-\infty}^0 + \left[ y \right]_0^1 + \left[ -e^{1-y} \right]_1^{\infty} \\
        &= e^0 - 0 + 1 - 0 - 0 + e^0 = 3
    \end{align*}

    Por tanto, el área de $E$ es $\lm_2(E) = 3$.
\end{ejercicio}

\begin{ejercicio}
    En cada uno de los siguientes casos, probar que la función \( f \) es integrable en \( \Omega \) y calcular su integral.
    \begin{enumerate}
        \item \(\Omega = \left\{ (x, y) \in \mathbb{R}^2 \mid 0 \leq x \leq 2,~y^2 \leq 2x \right\} \),
        \[ f(x, y) = \frac{x}{\sqrt{1 + x^2 + y^2}} \quad \forall (x, y) \in \Omega \]

        Veamos gráficamente el conjunto $\Omega$:
        \begin{figure}[H]
            \centering
            \begin{tikzpicture}
                \begin{axis}[
                    domain=-1:4, xmin=-1, xmax=8, ymin=-4, ymax=4,
                    samples=100,
                    axis lines=center,
                    xlabel={$x$},
                    ylabel={$y$},
                    clip=true,
                    axis equal,
                    legend cell align=left
                ]
                    % Dibuja la curva de y = \sqrt{2x}
                    \addplot [name path=A, blue, thick, forget plot] {sqrt(2*x)};
                
                    % Dibuja la curva de y = -\sqrt{2x}
                    \addplot [name path=B, blue, thick] {-sqrt(2*x)};
                    \addlegendentry{$x=\nicefrac{y^2}{2}$}
                
                    % Dibuja la línea vertical en x=2
                    \addplot [name path=C, teal, thick] (2,-4) -- (2,4);
                    \addlegendentry{$x=2$}
                
                    % Rellena el área bajo la curva entre x=0 y x=2
                    \addplot [
                        thick,
                        color=orange,
                        fill=orange,
                        fill opacity=0.4
                    ]
                    fill between [
                        of=A and B,
                        soft clip={domain=0:2},
                    ];
                    \addlegendentry{$\Omega$}
                \end{axis}
              \end{tikzpicture}
        \end{figure}

        Veamos en primer lugar que $\Omega$ está acotado. Sabemos que es medible por ser intersección de cerrados, y está acotado ya que
        para todo $(x, y) \in \Omega$ se tiene que:
        \begin{equation*}
            |x|\leq 2
            \hspace{2cm}
            |y|\leq \sqrt{2x} \leq \sqrt{2\cdot 2} = 2
        \end{equation*}

        Por tanto, $\lm_2(\Omega) \leq 4^2 < \infty$. Tenemos además que $f$ es continua, luego medible. Veamos que está acotada:
        \begin{equation*}
            |f(x,y)| = \left| \frac{x}{\sqrt{1 + x^2 + y^2}} \right| \leq |x| \leq 2
            \qquad \forall (x, y) \in \Omega
        \end{equation*}

        Por tanto, veamos que $f$ es integrable en $\Omega$:
        \begin{equation*}
            \int_{\Omega} |f(x, y)|~d(x, y) \leq \int_{\Omega} 2~d(x, y) = 2 \lm_2(\Omega) < \infty
        \end{equation*}

        \item \(\Omega = \left\{ (x, y) \in \mathbb{R}^2 \mid x^2 + y^2 \leq 1,~x^2 + y^2 \leq 2x \right\} \),
        \[ f(x, y) = x \quad \forall (x, y) \in \Omega \]

        \item \(\Omega = \left\{ (x, y, z) \in \mathbb{R}^3 \mid 0 < x < y < z \right\} \),
        \[ f(x, y, z) = e^{-(x+y+z)} \quad \forall (x, y, z) \in \Omega \]
    \end{enumerate}

\end{ejercicio}

\begin{ejercicio}
    En cada uno de los siguientes casos, estudiar la integrabilidad de la función \( f \) en el conjunto \( \Omega \).
    \begin{enumerate}
        \item $f(x, y) = \dfrac{\cos (x y)}{(1 + y^2) \sqrt{\sen x}} \hspace{1cm} \forall (x, y) \in \Omega= \left]0, \nicefrac{\pi}{2}\right[ \times \mathbb{R}^+$,
        
        \item $f(x, y) = (x - y) e^{-(x-y)^2} \hspace{1cm} \forall (x, y) \in \Omega= \mathbb{R}^+ \times \mathbb{R}^+$,
        \item $f(x, y, z) = \dfrac{\cos x + \cos y + \cos z}{(1 + x^2 + y^2 + z^2)^3} \hspace{1cm} \forall (x, y, z) \in \Omega= \mathbb{R}^3$.
    \end{enumerate}
\end{ejercicio}

\begin{ejercicio}
    Probar que el siguiente conjunto es medible y calcular su volumen.
    \[ E = \left\{ (x, y, z) \in \left(\mathbb{R}^+_0\right)^3 \mid x + y + z \leq 1 \right\} \subset \mathbb{R}^3 \]
\end{ejercicio}
    