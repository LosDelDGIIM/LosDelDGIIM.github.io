\section{Teorema de Cambio de Variable}

\begin{ejercicio}
    En cada uno de los siguientes casos, probar que la función \( f \) es integrable en el conjunto \( A \) y calcular su integral:

    \begin{enumerate}
        \item \( A = \left\{ (x, y) \in \mathbb{R}^2 \mid x^2 + y^2 > 1 ,~y > 0 \right\} \)
        \[ f(x, y) = \frac{x + y}{(x^2 + y^2)^\alpha} \quad \forall (x, y) \in A \quad (\alpha \in \mathbb{R},~\alpha > \nicefrac{3}{2}) \]

        Aplicaremos el cambio de variable a coordenadas polares. Para ello, en primer
        lugar obtenemos el conjunto $E$ siguiente:
        \begin{align*}
            E &= \left\{ (\rho, \theta) \in \bb{R}^+\times \left]-\pi, \pi\right[ ~\mid (\rho\cos\theta, \rho\sen\theta) \in A \right\} \\
            &= \left\{ (\rho, \theta) \in \bb{R}^+\times \left]-\pi, \pi\right[ ~\mid \rho > 1 ,~\sen\theta > 0 \right\} \\
            &= \left] 1, +\infty \right[ \times \left] 0, \pi \right[
        \end{align*}

        Definimos por tanto la función $g: E \to \bb{R}^2$ como:
        \begin{equation*}
            g(\rho, \theta) = \rho f(\rho\cos\theta, \rho\sen\theta) = \frac{\rho^2(\cos\theta + \sen\theta)}{\rho^{2\alpha}}
        \end{equation*}

        Por el Teorema de Cambio de Variable, tenemos que $f\in \cc{L}_1(A)$ si y solo si $g\in \cc{L}_1(E)$, y en tal caso:
        \begin{equation*}
            \int_A f = \int_E g
        \end{equation*}

        Veamos si $g\in \cc{L}_1(E)$. Por el Teorema de Tonelli, tenemos que:
        \begin{align*}
            \int_E |g(\rho, \theta)|~d(\rho, \theta)
            &\leq 2\int_E \rho^{2-2\alpha}~d(\rho, \theta)
            = 2\int_0^\pi \left( \int_1^{+\infty} \rho^{2-2\alpha}~d\rho \right)~d\theta
            =\\&= 2\pi \int_1^{+\infty} \rho^{2-2\alpha}~d\rho
            = 2\pi \left[ \frac{\rho^{3-2\alpha}}{3-2\alpha} \right]_1^{+\infty}
            \AstIg \frac{2\pi}{2\alpha-3} < \infty
        \end{align*}
        donde, en $(\ast)$, hemos usado que $\alpha > \nicefrac{3}{2}$, por lo que $3-2\alpha < 0$.
        Por tanto, $g\in \cc{L}_1(E)$, y por el Teorema de Cambio de Variable, tenemos que $f\in \cc{L}_1(A)$ con:
        \begin{align*}
            \int_A f &= \int_E g = \int_0^\pi \left( \int_1^{+\infty} \frac{\rho^2(\cos\theta + \sen\theta)}{\rho^{2\alpha}}~d\rho \right)~d\theta =\\
            &= \int_0^\pi \left( \cos\theta + \sen\theta \right) \left( \int_1^{+\infty} \rho^{2-2\alpha}~d\rho \right)~d\theta =\\
            &= \int_0^\pi \left( \cos\theta + \sen\theta \right) \left[ \frac{\rho^{3-2\alpha}}{3-2\alpha} \right]_1^{+\infty}~d\theta =\\
            &= \frac{1}{2\alpha-3} \int_0^\pi \left( \cos\theta + \sen\theta \right) ~d\theta =\\
            &= \frac{1}{2\alpha-3} \left[ \sen\theta - \cos\theta \right]_0^\pi = \frac{2}{2\alpha-3}
        \end{align*}

        \item \( A = \left\{ (x, y, z) \in \mathbb{R}^3 \mid 0 < x^2 + y^2 < 1 ,~z > 1 \right\} \)
        \[ f(x, y, z) = z^\alpha (x^2 + y^2)^\beta \quad \forall (x, y, z) \in A \quad (\alpha, \beta \in \mathbb{R},~\alpha < -1 < \beta) \]

        Aplicaremos el cambio de variable a coordenadas cilíndricas. Para ello, en primer
        obtener el conjunto $E$ siguiente:
        \begin{align*}
            E &= \left\{ (\rho, \theta, z) \in \bb{R}^+\times \left]-\pi, \pi\right[ \times \bb{R} ~\mid (\rho\cos\theta, \rho\sen\theta, z) \in A \right\} \\
            &= \left\{ (\rho, \theta, z) \in \bb{R}^+\times \left]-\pi, \pi\right[ \times \bb{R} ~\mid 0 < \rho < 1 ,~z > 1 \right\} \\
            &= \left] 0, 1 \right[ \times \left] -\pi, \pi \right[ \times \left] 1, +\infty \right[
        \end{align*}

        Definimos por tanto la función $g: E \to \bb{R}^3$ como:
        \begin{equation*}
            g(\rho, \theta, z) = \rho f(\rho\cos\theta, \rho\sen\theta, z) = z^\alpha \rho^{2\beta+1}
        \end{equation*}

        Por el Teorema de Cambio de Variable para funciones medibles positivas, junto con el Teorema de Tonelli, tenemos que:
        \begin{align*}
            \int_A f &= \int_E g = \int_{-\pi}^\pi \left( \int_0^1 \left( \int_1^{+\infty} z^\alpha \rho^{2\beta+1}~dz \right)~d\rho \right)~d\theta =\\
            &= \left(\int_{-\pi}^\pi ~d\theta \right) \left( \int_0^1 \rho^{2\beta+1}~d\rho \right) \left( \int_1^{+\infty} z^\alpha ~dz \right) =\\
            &= 2\pi \left[ \frac{\rho^{2\beta+2}}{2\beta+2} \right]_0^1 \left[ \frac{z^{\alpha+1}}{\alpha+1} \right]_1^{+\infty} =\\
            &= -\frac{2\pi}{2\beta+2}\cdot \frac{1}{\alpha+1}
            = -\frac{\pi}{(\beta+1)(\alpha+1)}
        \end{align*}

        Por tanto, como la integral es finita, $f\in \cc{L}_1(A)$ y su valor es el calculado.

        \item \( A = \left\{ (x, y, z) \in (\mathbb{R}^+)^3 \mid x^2 + y^2 + z^2 > 1 \right\} \)
        \[ f(x, y, z) = \frac{x y z}{(x^2 + y^2 + z^2)^4} \quad \forall (x, y, z) \in A \]

        Aplicaremos el cambio de variable a coordenadas esféricas. Para ello, en primer
        lugar obtenemos el conjunto $E$ siguiente:
        \begin{align*}
            E &= \left\{ (r, \theta, \varphi) \in \bb{R}^+\times \left]-\pi, \pi\right[ \times \left] \nicefrac{-\pi}{2},\nicefrac{\pi}{2} \right[ ~\mid (r\cos\varphi\cos\theta, r\cos\varphi\sen\theta, r\sen\varphi) \in A \right\} \\
            &= \left\{ (r, \theta, \varphi) \in \bb{R}^+\times \left]-\pi, \pi\right[ \times \left] \nicefrac{-\pi}{2},\nicefrac{\pi}{2} \right[ ~\mid \rho > 1 \right\} \\
            &= \left] 1, +\infty \right[ \times \left] -\pi, \pi \right[ \times \left] \nicefrac{-\pi}{2},\nicefrac{\pi}{2} \right[
        \end{align*}

        Definimos por tanto la función $g: E \to \bb{R}^3$ como:
        \begin{align*}
            g(r, \theta, \varphi) &= r^2\cos\varphi\cdot f(r\cos\varphi\cos\theta, r\cos\varphi\sen\theta, r\sen\varphi)
            = \frac{r^5\cos^3\varphi\cos\theta\sen\theta\sen\varphi}{r^8} =\\
            &= \frac{\cos^3\varphi\cos\theta\sen\theta\sen\varphi}{r^3}
        \end{align*}

        Veamos que $g\in \cc{L}_1(E)$ usando el Teorema de Tonelli:
        \begin{align*}
            \int_E |g(r, \theta, \varphi)|~d(r, \theta, \varphi)
            &\leq \int_E \frac{1}{r^3}~d(r, \theta, \varphi)
            = \int_{\nicefrac{-\pi}{2}}^{\nicefrac{\pi}{2}} \left( \int_{-\pi}^\pi \left( \int_1^{+\infty} \frac{dr}{r^3} \right)~d\theta \right)~d\varphi
            =\\&= 2\pi^2 \int_{1}^{+\infty} \frac{dr}{r^3}
            = 2\pi^2 \left[ -\frac{1}{2r^2} \right]_1^{+\infty} = \pi^2 < \infty
        \end{align*}

        Por tanto, $g\in \cc{L}_1(E)$, y por el Teorema de Cambio de Variable, tenemos que $f\in \cc{L}_1(A)$. Para calcular la integral, hay dos formas de proceder:
        \begin{description}
            \item[Opción 1.] Calcular la integral de forma normal, usando el Teorema de Cambio de Variable y el Teorema de Fubini.
            \begin{align*}
                \int_A f &= \int_E g = \int_{\nicefrac{-\pi}{2}}^{\nicefrac{\pi}{2}} \left( \int_{-\pi}^\pi \left( \int_1^{+\infty} \frac{\cos^3\varphi\cos\theta\sen\theta\sen\varphi}{r^3}~dr \right)~d\theta \right)~d\varphi =\\
                &= \left( \int_{\nicefrac{-\pi}{2}}^{\nicefrac{\pi}{2}} \cos^3\varphi\sen\varphi~d\varphi \right) \left( \int_{-\pi}^\pi \cos\theta\sen\theta~d\theta \right) \left( \int_1^{+\infty} \frac{dr}{r^3} \right) =\\
                &= \left[\dfrac{\cos^4\varphi}{4}\right]_{\nicefrac{-\pi}{2}}^{\nicefrac{\pi}{2}} \cdot \left[-\dfrac{\cos(2\theta)}{4}\right]_{-\pi}^{\pi} \cdot \left[-\dfrac{1}{2r^2}\right]_1^{+\infty} =\\
                &= 0 \cdot 0 \cdot \dfrac{1}{2} = 0
            \end{align*}

            \item[Opción 2.] Usar que es simétrica respecto de la variable $x$. Para ello, definimos en primer lugar $A_0,~A_+$ y $A_-$ como:
            \begin{align*}
                A_0 &= \left\{ (x, y, z) \in A \mid x=0 \right\} \\
                A_+ &= \left\{ (x, y, z) \in A \mid x > 0 \right\} \\
                A_- &= \left\{ (x, y, z) \in A \mid x < 0 \right\}
            \end{align*}

            Por ser $A_0$ un hiperplano en $\bb{R}^3$, tenemos que $\lm_3(A_0) = 0$. Por otro lado, tenemos que:
            \begin{gather*}
                A_- = \left\{ (-x, y, z) \in \bb{R}^3 \mid (x, y, z) \in A_+ \right\} \\
                f(-x, y, z) = -f(x, y, z) \quad \forall (x, y, z) \in A_+
            \end{gather*}

            Por tanto, como $A=A_+ \uplus A_- \uplus A_0$, por la aditividad de la integral de Lebesgue, tenemos que:
            \begin{align*}
                \int_A f &= \int_{A_+} f + \int_{A_-} f + \int_{A_0} f = \int_{A_+} f - \int_{A_+} f = 0
            \end{align*}
        \end{description}
    \end{enumerate}
\end{ejercicio}

\begin{ejercicio}
    En cada uno de los siguientes casos, estudiar la integrabilidad de la función \( f \) en el conjunto \( A \):

    \begin{enumerate}
        \item \( A = \mathbb{R}^2 \setminus \{(0, 0)\} \)
        \[ f(x, y) = \frac{\sen x \sen y}{(x^2 + y^2)^{\nicefrac{3}{2}}} \quad \forall (x, y) \in A \]

        En primer lugar, descomponemos el conjunto $A$ en dos conjuntos disjuntos:
        \begin{align*}
            A_1 &= \left\{ (x, y) \in \bb{R}^2 \mid x^2 + y^2 > 1 \right\} \\
            A_2 &= \left\{ (x, y) \in \bb{R}^2 \mid x^2 + y^2 \leq 1 \right\}
        \end{align*}

        tenemos que $A=A_1 \uplus A_2$. Por tanto, $f\in \cc{L}_1(A)$ si y solo si $f\in \cc{L}_1(A_1)$ y $f\in \cc{L}_1(A_2)$. Estudiemos en primer lugar la integrabilidad de $f$ en $A_1$.
        Para ello, aplicaremos el cambio de variable a coordenadas polares. Obtenemos el conjunto $E_1$ siguiente:
        \begin{align*}
            E_1 &= \left\{ (\rho, \theta) \in \bb{R}^+\times \left]-\pi, \pi\right[ ~\mid (\rho\cos\theta, \rho\sen\theta) \in A_1 \right\} \\
            &= \left\{ (\rho, \theta) \in \bb{R}^+\times \left]-\pi, \pi\right[ ~\mid \rho > 1 \right\} \\
            &= \left]1,\infty\right[\times ]-\pi, \pi[
        \end{align*}

        Definimos por tanto la función $g_1: E_1 \to \bb{R}$ como:
        \begin{equation*}
            g_1(\rho, \theta) = \rho f(\rho\cos\theta, \rho\sen\theta) = \rho\cdot \frac{\sen(\rho\cos\theta) \sen(\rho\sen\theta)}{\rho^3}
            = \frac{\sen(\rho\cos\theta) \sen(\rho\sen\theta)}{\rho^2}
        \end{equation*}

        Por el Teorema de Cambio de Variable, tenemos que $f\in \cc{L}_1(A_1)$ si y solo si $g_1\in \cc{L}_1(E_1)$. Veamos si $g_1\in \cc{L}_1(E_1)$:
        \begin{align*}
            \int_{E_1} |g_1(\rho, \theta)|~d(\rho, \theta)
            &\leq \int_{E_1} \frac{1}{\rho^2}~d(\rho, \theta)
            = \int_{-\pi}^\pi \left( \int_1^{+\infty} \frac{d\rho}{\rho^2} \right)~d\theta
            = 2\pi \int_1^{+\infty} \frac{d\rho}{\rho^2} =\\
            &= 2\pi \left[ -\frac{1}{\rho} \right]_1^{+\infty} = 2\pi < \infty
        \end{align*}
        
        Por tanto, $g_1\in \cc{L}_1(E_1)$, y por el Teorema de Cambio de Variable, tenemos que $f\in \cc{L}_1(A_1)$.
        \begin{observacion}
            Es cierto que no es fácil que se nos ocurra descomponer $A$ en $A_1$ y $A_2$. No obstante,
            el razonamiento seguido sería el lógico, y si no se hubiese descompuesto en ambos conjuntos aquí nos encontraríamos
            con un problema, puesto que la función $x\mapsto \frac{1}{x^2}$ no es integrable en $\left]0,c\right[$ para ningún $c\in \bb{R}^+$.
            Por esto, tomando $c=1$, conseguimos que ahora sí sea integrable en $A_1$. Para $A_2$ será necesario un razonamiento más sutil.
        \end{observacion}

        Estudiemos ahora la integrabilidad de $f$ en $A_2$. Para ello, aplicamos el cambio de variable a coordenadas polares. Obtenemos el conjunto $E_2$ siguiente:
        \begin{align*}
            E_2 &= \left\{ (\rho, \theta) \in \bb{R}^+\times \left]-\pi, \pi\right[ ~\mid (\rho\cos\theta, \rho\sen\theta) \in A_2 \right\} \\
            &= \left\{ (\rho, \theta) \in \bb{R}^+\times \left]-\pi, \pi\right[ ~\mid \rho \leq 1 \right\} \\
            &= \left]0,1\right]\times ]-\pi, \pi[
        \end{align*}

        Definimos por tanto la función $g_2: E_2 \to \bb{R}$ como:
        \begin{equation*}
            g_2(\rho, \theta) = \rho f(\rho\cos\theta, \rho\sen\theta) = \rho\cdot \frac{\sen(\rho\cos\theta) \sen(\rho\sen\theta)}{\rho^3}
            = \frac{\sen(\rho\cos\theta) \sen(\rho\sen\theta)}{\rho^2}
        \end{equation*}

        Por el Teorema de Cambio de Variable, tenemos que $f\in \cc{L}_1(A_1)$ si y solo si $g_2\in \cc{L}_1(E_2)$. Para acotar $g_2$,
        hacemos uso de la Desigualdad del Valor Medio para funciones de una variable, de la que deducimos que:
        \begin{equation*}
            |\sen(t) - \sen(0)| \leq |t-0| \quad \forall t\in \bb{R} \Longrightarrow |\sen(t)| \leq |t| \quad \forall t\in \bb{R}
        \end{equation*}

        Por tanto, tenemos que:
        \begin{align*}
            |g_2(\rho, \theta)| &= \left| \frac{\sen(\rho\cos\theta) \sen(\rho\sen\theta)}{\rho^2} \right|
            \leq \frac{|\rho\cos\theta| \cdot |\rho\sen\theta|}{\rho^2}
            \leq 1 \quad \forall (\rho, \theta) \in E_2
        \end{align*}

        Por tanto, tenemos que:
        \begin{align*}
            \int_{E_2} |g_2(\rho, \theta)|~d(\rho, \theta)
            &\leq \int_{E_2} 1~d(\rho, \theta)
            = \lm(E_2) = \lm\left(\left]0,1\right]\right) \cdot \lm\left(]-\pi, \pi[\right) = 2\pi < \infty
        \end{align*}

        Por tanto, $g_2\in \cc{L}_1(E_2)$, y por el Teorema de Cambio de Variable, tenemos que $f\in \cc{L}_1(A_2)$.
        Como $f\in \cc{L}_1(A_1)$ y $f\in \cc{L}_1(A_2)$, tenemos que $f\in \cc{L}_1(A)$.

        \item \( A = \left\{ (x, y, z) \in \mathbb{R}^3 \mid z > x^2 + y^2 \right\} \)
        \[ f(x, y, z) = (x^3 + y^3) \cos (xy) e^{-z} \quad \forall (x, y, z) \in A \]

        Aplicaremos el cambio de variable a coordenadas cilíndricas. Para ello, en primer
        lugar obtenemos el conjunto $E$ siguiente:
        \begin{align*}
            E &= \left\{ (\rho, \theta, z) \in \bb{R}^+\times \left]-\pi, \pi\right[ \times \bb{R} ~\mid (\rho\cos\theta, \rho\sen\theta, z) \in A \right\} \\
            &= \left\{ (\rho, \theta, z) \in \bb{R}^+\times \left]-\pi, \pi\right[ \times \bb{R} ~\mid z > \rho^2 \right\}
        \end{align*}

        Definimos por tanto la función $g: E \to \bb{R}^3$ para cada $(\rho, \theta, z) \in E$ como:
        \begin{equation*}
            g(\rho, \theta, z) = \rho f(\rho\cos\theta, \rho\sen\theta, z) = \rho^4(\cos^3\theta + \sen^3\theta) \cos(\rho^2\cos\theta\sen\theta)e^{-z}
        \end{equation*}

        Por el Teorema de Cambio de Variable, tenemos que $f\in \cc{L}_1(A)$ si y solo si $g\in \cc{L}_1(E)$. Para estudiar la integrabilidad de $g$, veamos si $g\in \cc{L}_1(E)$,
        para lo cual usaremos el Teorema de Tonelli dos veces:
        \begin{align*}
            \int_E |g(\rho, \theta, z)|~d(\rho, \theta, z)
            &\leq \int_E \rho^4e^{-z}~d(\rho, \theta, z)
            = \int_{-\pi}^\pi \left( \int_0^{+\infty} \left( \int_{\rho^2}^{+\infty} \rho^4e^{-z}~dz \right)~d\rho \right)~d\theta =\\
            &= 2\pi \int_0^{+\infty}\rho^4 \left( \int_{\rho^2}^{+\infty}e^{-z}~dz \right)~d\rho
            = 2\pi \int_0^{+\infty}\rho^4 \left[ -e^{-z} \right]_{\rho^2}^{+\infty}~d\rho =\\
            &= 2\pi \int_0^{+\infty}\rho^4e^{-\rho^2}~d\rho
        \end{align*}

        Para ver si dicha última función es integrable, podemos hacer uso del Criterio de Comparación con la función $x\mapsto e^{-x}$, integrable en $\bb{R}^+_0$.
        Como ambas son continuas, sabemos que son localmente integrables en $\bb{R}^+_0$. Tenemos el siguiente límite:
        \begin{equation*}
            \lim_{\rho\to+\infty} \frac{\rho^4e^{-\rho^2}}{e^{-\rho}} = \lim_{\rho\to+\infty} \rho^4e^{-\rho} = 0
        \end{equation*}

        Por tanto, por el Criterio de Comparación, como la función empleada para comparar es integrable, tenemos que:
        \begin{equation*}
            \int_0^{+\infty}\rho^4e^{-\rho^2}~d\rho < \infty
        \end{equation*}

        Por tanto, tenemos que:
        \begin{equation*}
            \int_E |g(\rho, \theta, z)|~d(\rho, \theta, z) = 2\pi \int_0^{+\infty}\rho^4e^{-\rho^2}~d\rho < \infty
        \end{equation*}

        Por tanto, $g\in \cc{L}_1(E)$, y por el Teorema de Cambio de Variable, tenemos que $f\in \cc{L}_1(A)$.

        \item \( A = \mathbb{R}^3 \)
        \[ f(x, y, z) = \frac{1}{\left(1 + x^2 + y^2 + z^2\right)^\alpha} \quad \forall (x, y, z) \in A \quad (\alpha \in \mathbb{R}^+) \]

        Apliquemos el cambio de variable a coordenadas esféricas. Para ello, en primer
        lugar obtenemos el conjunto $E$ siguiente:
        \begin{align*}
            E &= \left\{ (r, \theta, \varphi) \in \bb{R}^+\times \left]-\pi, \pi\right[ \times \left] \nicefrac{-\pi}{2},\nicefrac{\pi}{2} \right[ ~\mid (r\cos\varphi\cos\theta, r\cos\varphi\sen\theta, r\sen\varphi) \in A \right\} \\
            &= \bb{R}^+\times \left]-\pi, \pi\right[ \times \left] \nicefrac{-\pi}{2},\nicefrac{\pi}{2} \right[
        \end{align*}

        Definimos por tanto la función $g: E \to \bb{R}$ para cada $(r, \theta, \varphi) \in E$ como:
        \begin{equation*}
            g(r, \theta, \varphi) = r^2\cos\varphi f(r\cos\varphi\cos\theta, r\cos\varphi\sen\theta, r\sen\varphi) =
            \frac{r^2\cos\varphi}{\left(1 + r^2\right)^\alpha}
        \end{equation*}

        Por el Teorema de Cambio de Variable, tenemos que $f\in \cc{L}_1(A)$ si y solo si $g\in \cc{L}_1(E)$. Para estudiar la integrabilidad de $g$, usamos el Teorema de Tonelli:
        \begin{align*}
            \int_A f(x, y, z)~d(x, y, z) &= \int_E g(r, \theta, \varphi)~d(r, \theta, \varphi) =\\
            &= \int_{-\pi}^\pi \left( \int_{\nicefrac{-\pi}{2}}^{\nicefrac{\pi}{2}} \left( \int_0^{+\infty} \frac{r^2\cos\varphi}{\left(1 + r^2\right)^\alpha}~dr \right)~d\varphi \right)~d\theta =\\
            &= 2\pi \int_{\nicefrac{-\pi}{2}}^{\nicefrac{\pi}{2}} \left( \int_0^{+\infty} \frac{r^2\cos\varphi}{\left(1 + r^2\right)^\alpha}~dr \right)~d\varphi =\\
            &= 2\pi \left(\int_{\nicefrac{-\pi}{2}}^{\nicefrac{\pi}{2}} \cos\varphi~d\varphi\right) \left(\int_0^{+\infty} \frac{r^2}{\left(1 + r^2\right)^\alpha}~dr\right) =\\
            &= 4\pi \left(\int_0^{+\infty} \frac{r^2}{\left(1 + r^2\right)^\alpha}~dr\right)
        \end{align*}

        Para estudiar la integrabilidad de la última integral, como tiene extensión continua en $\bb{R}^0_0$ sabemos que será integrable en $[0,1]$,
        luego nos centraremos en estudiar su integrabilidad en $[1,+\infty[$. Para ello,
        podemos hacer uso de la función $r\mapsto r^{2-2\alpha}$.
        Como ambas son continuas, sabemos que son localmente integrables en $[1,+\infty[$. Tenemos el siguiente límite:
        \begin{equation*}
            \lim_{r\to+\infty} \dfrac{\frac{r^2}{\left(1 + r^2\right)^\alpha}}{r^{2-2\alpha}}
            = \lim_{r\to+\infty} \dfrac{r^{2\alpha}}{\left(1 + r^2\right)^\alpha} = 1
        \end{equation*}

        Por tanto, por el Criterio de Comparación, tenemos que l función cuya integrabilidad queremos estudiar es integrable si y solo si lo es la función $r\mapsto r^{2-2\alpha}$.
        Esta última sabemos que es integrable en $[1,+\infty[$ si y solo si $2-2\alpha < -1 \iff \alpha > \nicefrac{3}{2}$.
        Por tanto, la función $f$ es integrable en $A$ si y solo si $\alpha > \nicefrac{3}{2}$.
    \end{enumerate}
\end{ejercicio}

\begin{ejercicio}
    Calcular el volumen de la llamada \emph{bóveda de Viviani}:
    \[ B = \left\{ (x, y, z) \in \mathbb{R}^3 \mid (2x - 1)^2 + 4y^2 \leq 1 ,~x^2 + y^2 + z^2 \leq 1 ,~z \geq 0 \right\} \]
\end{ejercicio}
    