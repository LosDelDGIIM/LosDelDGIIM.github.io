\chapter{Ejercicios Voluntarios}

\begin{teo}[Aproximación de Weierstrass]
    Sea $f:[0,1]\to \bb{R}$ una función continua. Entonces, existe una sucesión de polinomios $\{P_n\}$ de manera que $\{P_n\}$ converge uniformemente a $f$ en $[0,1]$.
\end{teo}
\begin{proof}
    Definimos la sucesión de polinomios de Bernstein como:
    \begin{equation*}
        B_n(f)(x) = \sum_{k=0}^n f\left(\frac{k}{n}\right)\binom{n}{k} x^k(1-x)^{n-k}
        \qquad 0\leq x\leq 1
    \end{equation*}
    Tenemos claramente que $k, n-k\in \bb{N}$, por lo que $B_n(f)(x)$ es un polinomio.
    Veamos ahora que $\{B_n(f)\}$ converge uniformemente a $f$ en $[0,1]$. Para ello, usaremos el siguiente lema
    relacionado con el binomio de Newton:
    \begin{lema}\label{lema:AproxWeierstrass}
        Para $x\in \bb{R}$, $n\in \bb{N}$, se tiene que:
        \begin{enumerate}
            \item $\displaystyle \sum_{k=0}^n \binom{n}{k} x^k(1-x)^{n-k} = 1$.
            \item $\displaystyle \sum_{k=0}^n \left(x-\frac{k}{n}\right)^2 \binom{n}{k} x^k(1-x)^{n-k} = \frac{x(1-x)}{n}$.
        \end{enumerate}
        \begin{proof}
            Demostramos cada uno de los apartados por separado:
            \begin{enumerate}
                \item Usamos la fórmula del binomio de Newton:
                \begin{equation}\label{eq:BinomioNewton}
                    (p+q)^n = \sum_{k=0}^n \binom{n}{k} p^k q^{n-k} \qquad p,q\in \bb{R}
                \end{equation}

                En concreto, para $p=x$ y $q=1-x$, se tiene que:
                \begin{equation}\label{eq:BinomioNewtonSolo}
                    1 = (x+(1-x))^n = \sum_{k=0}^n \binom{n}{k} x^k(1-x)^{n-k}
                \end{equation}

                \item Derivando la fórmula del binomio de Newton (Ecuación \ref{eq:BinomioNewton}) respecto de $p$, se tiene que:
                \begin{equation}\label{eq:BinomioNewton_kn}
                    n(p+q)^{n-1} = \sum_{k=0}^n \binom{n}{k} k p^{k-1} q^{n-k}
                    \Longrightarrow
                    p(p+q)^{n-1} = \sum_{k=0}^n \binom{n}{k} \frac{k}{n}\cdot p^k q^{n-k}
                \end{equation}

                Derivando ahora la Ecuación \ref{eq:BinomioNewton_kn} respecto de $p$, se tiene que:
                \begin{equation*}
                    (p+q)^{n-1} + p(n-1)(p+q)^{n-2} = \sum_{k=0}^n \binom{n}{k} \frac{k^2}{n}\cdot p^{k-1} q^{n-k}
                \end{equation*}

                Multiplicando todo por $p$ y diviendo por $n$, se tiene que:
                \begin{equation}\label{eq:BinomioNewton_kn2}
                    \frac{p}{n}\cdot(p+q)^{n-1} + \frac{p^2}{n}(n-1)(p+q)^{n-2} = \sum_{k=0}^n \binom{n}{k} \frac{k^2}{n^2}\cdot p^{k} q^{n-k}
                \end{equation}

                Por tanto, tenemos que:
                \begin{align*}
                    \sum_{k=0}^n & \left(x-\frac{k}{n}\right)^2 \binom{n}{k} x^k(1-x)^{n-k}
                    =\\&= \sum_{k=0}^n \binom{n}{k} \left(x^2 - 2x\cdot \frac{k}{n} + \frac{k^2}{n^2}\right)\cdot x^k(1-x)^{n-k}
                    =\\&= x^2 \sum_{k=0}^n \binom{n}{k} x^{k}(1-x)^{n-k} - 2x\sum_{k=0}^n \binom{n}{k}\cdot \frac{k}{n} x^k(1-x)^{n-k} +\\&\hspace{3cm}+ \sum_{k=0}^n \binom{n}{k} \frac{k^2}{n^2} x^k(1-x)^{n-k}
                    \AstIg\\&\AstIg x^2 - 2x\cdot x(x+1-x)^{n-1} + \frac{x}{n}\cdot (x+1-x)^{n-1} + \frac{x^2}{n}(n-1)(x+1-x)^{n-2}
                    =\\&= x^2 - 2x^2 + \frac{x}{n} + \frac{x^2}{n}(n-1)
                    = -x^2 + \frac{x}{n} + x^2 - \frac{x^2}{n} = \frac{x-x^2}{n} = \frac{x(1-x)}{n}
                \end{align*}
                donde en $(\ast)$ se han usado las Ecuaciones \ref{eq:BinomioNewtonSolo}, \ref{eq:BinomioNewton_kn} y \ref{eq:BinomioNewton_kn2}.


            \end{enumerate}
        \end{proof}
    \end{lema}


    Fijado $n\in \bb{N}$, $x\in [0,1]$, la acotación entonces la obtenemos de la siguiente manera:
    \begin{align*}
        \left| B_n(f)(x) - f(x) \right|
        &= \left| \sum_{k=0}^n f\left(\frac{k}{n}\right)\binom{n}{k} x^k(1-x)^{n-k} - f(x)\cdot 1 \right| \stackrel{\text{Ec. \ref{eq:BinomioNewtonSolo}}}{=}\\
        &\stackrel{\text{Ec. \ref{eq:BinomioNewtonSolo}}}{=} \left| \sum_{k=0}^n f\left(\frac{k}{n}\right)\binom{n}{k} x^k(1-x)^{n-k} - f(x)\sum_{k=0}^n \binom{n}{k} x^k(1-x)^{n-k} \right| =\\
        &= \left| \sum_{k=0}^n f\left(\frac{k}{n}\right)\binom{n}{k} x^k(1-x)^{n-k} - \sum_{k=0}^n f(x)\binom{n}{k} x^k(1-x)^{n-k} \right| =\\
        &= \left| \sum_{k=0}^n \left( f\left(\frac{k}{n}\right) - f(x) \right)\binom{n}{k} x^k(1-x)^{n-k} \right| \leq\\
        &\leq \sum_{k=0}^n \left| f\left(\frac{k}{n}\right) - f(x) \right|\binom{n}{k} x^k(1-x)^{n-k}
    \end{align*}
    donde en la última desigualdad se usó la desigualdad triangular. Ahora, usamos el Teorema de Heine
    para afirmar que, como $f$ es continua en $[0,1]$ (cerrado y acotado), es uniformemente continua en $[0,1]$. Por lo tanto,
    \begin{equation*}
        \forall \varepsilon>0\quad \exists \delta>0\quad \text{tal que si}\quad |x-y|<\delta\quad \text{entonces}\quad |f(x)-f(y)|<\varepsilon
    \end{equation*}

    Fijado $\varepsilon>0$, consideramos el $\delta$ dado por la continuidad uniforme para $\nicefrac{\veps}{2}$.
    Consideramos el siguiente conjunto:
    \begin{equation*}
        F = \left\{ k\in \{0,\dots,n\} : \left|x-\frac{k}{n}\right|<\delta \right\}
    \end{equation*}
    Veamos qué ocurre en los puntos de $F$ y en los que no están en $F$:
    \begin{itemize}
        \item Si $k\in F$, entonces $\left|x-\nicefrac{k}{n}\right|<\delta$, por lo que $|f(x)-f\left(\nicefrac{k}{n}\right)|<\nicefrac{\varepsilon}{2}$.
        \item Si $k\notin F$, el razonamiento es algo más complejo. Por el Teorema de Weierstrass,
        sabemos que $f$ es acotada en $[0,1]$, es decir, existe $M>0$ tal que $|f(x)|\leq M$ para todo $x\in [0,1]$.
        Además, como $k\notin F$, se tiene que:
        \begin{equation*}
            \left|x-\frac{k}{n}\right|\geq\delta\quad \Longrightarrow\quad \left(x-\frac{k}{n}\right)^2\geq\delta^2
            \Longrightarrow \dfrac{\left(x-\frac{k}{n}\right)^2}{\delta^2}\geq 1
        \end{equation*}

        Uniendo ambos resultados, se tiene que:
        \begin{equation*}
            \left| f(x) - f\left(\frac{k}{n}\right) \right|
            \leq |f(x)| + \left|f\left(\frac{k}{n}\right)\right|
            \leq 2M \leq 2M\left(\frac{\left(x-\frac{k}{n}\right)^2}{\delta^2}\right)
        \end{equation*}
    \end{itemize}

    Por tanto, en función de si $k\in F$ o no, tenemos que:
    \begin{align*}
        \left| B_n(f)(x) - f(x) \right| &\leq \sum_{k=0}^n \left| f\left(\frac{k}{n}\right) - f(x) \right|\binom{n}{k} x^k(1-x)^{n-k} =\\
        &= \sum_{k\in F} \left| f\left(\frac{k}{n}\right) - f(x) \right|\binom{n}{k} x^k(1-x)^{n-k} +\\&\hspace{3cm} +\sum_{k\notin F} \left| f\left(\frac{k}{n}\right) - f(x) \right|\binom{n}{k} x^k(1-x)^{n-k} \leq\\
        &\leq \sum_{k\in F} \frac{\varepsilon}{2}\binom{n}{k} x^k(1-x)^{n-k} + \sum_{k\notin F} 2M\frac{\left(x-\frac{k}{n}\right)^2}{\delta^2}\binom{n}{k} x^k(1-x)^{n-k} <\\
        &< \frac{\varepsilon}{2}\sum_{k=0}^n \binom{n}{k} x^k(1-x)^{n-k} + \frac{2M}{\delta^2}\sum_{k=0}^n \left(x-\frac{k}{n}\right)^2\binom{n}{k} x^k(1-x)^{n-k} \AstIg\\
        &\AstIg \frac{\varepsilon}{2} + \frac{2M}{\delta^2}\cdot \frac{x(1-x)}{n} \stackrel{(\ast\ast)}{\leq}
        \frac{\varepsilon}{2} + \frac{2M}{\delta^2}\cdot \frac{1}{4n} = \frac{\varepsilon}{2} + \frac{M}{2n\delta^2}
        \qquad \forall x\in [0,1], n\in \bb{N}
    \end{align*}
    donde en $(\ast)$ se usó el Lema \ref{lema:AproxWeierstrass} y en $(\ast\ast)$ se usó que la función $g:[0,1]\to \bb{R}$ dada por $g(x)=(x)=x(1-x)=x-x^2$ es una parábola con imagen $g([0,1])=[0,\nicefrac{1}{4}]$.\\

    Por tanto, buscamos que $\dfrac{M}{2n\delta^2}<\dfrac{\varepsilon}{2}$:
    \begin{equation*}
        \dfrac{M}{2n\delta^2}<\dfrac{\varepsilon}{2}
        \Longleftrightarrow
        \frac{1}{n} < \frac{\varepsilon\delta^2}{M}
        \Longleftrightarrow
        n > \frac{M}{\varepsilon\delta^2}
    \end{equation*}
    Sea $m=E\left(\dfrac{M}{\varepsilon\delta^2}\right)+1$ el primer natural que cumple la condición. Entonces, para $n\geq m$,
    se tiene que:
    \begin{equation*}
        \left| B_n(f)(x) - f(x) \right| < \varepsilon
        \qquad \forall x\in [0,1]
    \end{equation*}

    queda así demostrado que $\{B_n(f)\}$ converge uniformemente a $f$ en $[0,1]$.
\end{proof}

\begin{definicion}
    Un monstruo de Weierstrass es una función continua en todos sus puntos que no es derivable en ningún punto.
\end{definicion}

\begin{ejercicio*}
    Encontrar un monstruo de Weierstrass y demostrar que lo es.

    Un ejemplo es el siguiente:
    \begin{equation*}
        \sum_{n=0}^\infty \frac{1}{n!}\cos\left((n!)^2 x\right)
    \end{equation*}
\end{ejercicio*}