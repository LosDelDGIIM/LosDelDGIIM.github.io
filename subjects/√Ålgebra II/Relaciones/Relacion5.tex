\section{Grupos resolubles}

\begin{ejercicio}
    Sea $N\lhd G$ un subgrupo normal y simple de un grupo $G$. Demostrar que si $G/N$ tiene una serie de composición entonces $G$ tiene una serie de composición.\\
    
    Consideramos la serie de composición de $G/N$:
    \[
        G/N = N_0 \rhd N_1 \rhd \cdots \rhd N_{r-1} \rhd N_r = \{1N\}.
    \]

    Por el Tercer Teorema de Isomorfía, como $N_i<G/N$ para todo $i\in \{0,1,\ldots,r\}$ existe $G_i<G$ tal que $N\lhd G_i$ cumpliendo que:
    \begin{equation*}
        N_i = G_i/N \quad \forall i\in \{0,1,\ldots,r\}.
    \end{equation*}

    Para considerar la serie buscada, hemos de probar que $G_i < G_{i-1}$ para todo $i\in \{1,\ldots,r\}$. Como $N_i\subset N_{i-1}$ por la biyección del Tercer Teorema de Isomorfía se tiene que $G_i\subset G_{i-1}$; y como $G_i$ es un grupo, se tiene que $G_i<G_{i-1}$. Consideramos por tanto la siguiente serie (donde notemos que hemos añadido el $\{1\}$):
    \[
        G = G_0 > G_1 > \cdots > G_{r-1} > G_r = N > \{1\}
    \]

    Nos falta ahora por ver que $G_i \lhd G_{i-1}$ para todo $i\in \{1,\ldots,r\}$. Por el Tercer Teorema de Isomorfía, como $G_i/N \lhd G/N$, se tiene que $G_i \lhd G$. Como además $G_i<G_{i-1}$, se tiene que $G_i \lhd G_{i-1}$. Por último, sabemos que $\{1\} \lhd N$.    
    Por tanto, consideramos la siguiente serie normal:
    \[
        G = G_0 \rhd G_1 \rhd \cdots \rhd G_{r-1} \rhd G_r = N \rhd \{1\}.
    \]

    Veamos que dicha serie es de composición. Para ello, hemos de ver que los factores son simples. Por ser la serie de partida de composición, sabemos que el siguiente grupo cociente es simple:
    \begin{equation*}
        \dfrac{G_{i-1}/N}{G_i/N}\qquad \forall i\in \{1,\ldots,r\}
    \end{equation*}

    Por el Tercer Teorema de Isomorfía, se tiene que:
    \begin{equation*}
        \dfrac{G_{i-1}}{G_i} \cong \dfrac{G_{i-1}/N}{G_i/N}\qquad \forall i\in \{1,\ldots,r\}
    \end{equation*}

    Como el ``ser simple'' es una propiedad que se conserva bajo isomorfismos, se tiene que:
    \begin{equation*}
        G_{i-1}/G_i \text{ es simple } \forall i\in \{1,\ldots,r\}.
    \end{equation*}

    Falta por compronar que $N/\{1\}$ es simple, algo que se tiene de forma directa puesto que $N/\{1\}\cong N$ y $N$ es simple. Por tanto, la serie
    \[
        G = G_0 \rhd G_1 \rhd \cdots \rhd G_{r-1} \rhd G_r = N \rhd \{1\}
    \]
    tiene todos sus factores simples, y por tanto es de composición. Notemos además que $l(G)=l(G/N)+1$.    
\end{ejercicio}

\begin{ejercicio}
    Sea $G$ un grupo abeliano. Demostrar que $G$ tiene series de composición si y sólo si $G$ es finito.
    \begin{description}
        \item[$\Longleftarrow$)] Si $G$ es finito, entonces hemos visto que $G$ tiene series de composición.
        
        \item[$\Longrightarrow$)] Si $G$ tiene una serie de composición, veamos ahora que $G$ es finito. Consideramos la serie de composición:
        \[
            G = G_0 \rhd G_1 \rhd \cdots \rhd G_{r-1} \rhd G_r = \{1\}.
        \]

        Como $G$ es abeliano, todos sus subgrupos son abelianos, y por tanto todos sus factores son abelianos. Además, por ser serie de composición, todos los factores son simples. Por la caracterización de los grupos abelianos y simples, todos los factores son de orden primo (en particular, finitos).\\

        A continuación, desarrollamos la siguiente idea. Dado un grupo $A$ y un subgrupo suyo $B\lhd A$, si $B$ es finito y $A/B$ es finito, entonces $A$ es finito. Notemos que a priori no podemos aplicar el Teorema de Lagrange, puesto que $A$ no es necesariamente finito. Sin embargo como las clases de equivalencia del cociente $A/B$ forman una partición de $A$, se tiene que:
        \begin{equation*}
            A=\bigcup_{i=1}^{|A/B|} a_iB
        \end{equation*}
        Como $B$ es finito, entonces $|a_iB|=|B|$ para todo $i\in \{1,\ldots,|A/B|\}$; luego $|A|=|B|\cdot |A/B|$, y en particular $A$ es finito.\\

        Aplicando esta idea a la serie de composición de $G$, obtenemos en primer que $G_{r-1}$ es finito, puesto que $G_{r-1}/\{1\}$ y $\{1\}$ son finitos. Análogamente, $G_{r-2}$ es finito, puesto que $G_{r-2}/G_{r-1}$ y $G_{r-1}$ son finitos. Por inducción sobre $r$, se tiene que $G_{0}=G$ es finito.
    \end{description}
\end{ejercicio}

\begin{ejercicio}\label{ej:5.3}
    Sea $H$ un subgrupo normal de un grupo finito $G$. Demostrar que existe una serie de composición de $G$ uno de cuyos términos es $H$.\\

    Como $G$ es finito, entonces $G$ tiene una serie de composición. Consideramos ahora la siguiente serie normal:
    \[
        G \rhd H \rhd \{1\}.
    \]

    Como $G$ admite una serie de composición, por el Teorema de Jordan-Holder dicha serie normal puede refinarse a una serie de composición.
\end{ejercicio}

\begin{ejercicio}\label{ej:5.4}
    Se define la longitud de un grupo finito $G$, denotada $l(G)$, como la longitud de cualquiera de sus series de composición. Demostrar que si $H$ es un subgrupo normal de $G$ entonces:
    \begin{equation*}
        l(G) = l(H) + l(G/H)\qquad \fact(G)=\fact(H)\cup\fact(G/H).
    \end{equation*}
    \begin{observacion}
        Notemos que los factores de composición de $G$ no tienen por qué ser únicos, por lo que a priori no podemos hablar de $\fact(G)$ como un conjunto. No obstante, son únicos salvo isomorfismos (y reordenamientos, pero al trabajar con conjuntos no es necesario tener en cuenta el orden). Por tanto, dos conjuntos $\fact(G)$ pueden ser distintos, pero sus elementos son isomorfos entre sí.
    \end{observacion}

    Por el Ejercicio~\ref{ej:5.3}, $G$ tiene una serie de composición que contiene a $H$. Consideramos la serie de composición:
    \[
        G = G_0 \rhd G_1 \rhd \cdots \rhd G_{r-1} \rhd G_r = H \rhd G_{r+1} \rhd \cdots \rhd G_{r+m-1} \rhd G_{r+m} = \{1\}.
    \]

    Vemos claramente que $l(G)=r+m$ y $l(H)=r+m-r=m$. Además:
    \begin{equation*}
        \fact(H)=\bigcup_{i=r+1}^{r+m-1} G_i/G_{i+1} \qquad \fact(G)=\bigcup_{i=0}^{r+m-1} G_i/G_{i+1}.
    \end{equation*}    
    
    
    Hemos de calcular ahora una serie de composición de $G/H$. Como $H\lhd G$ se tiene que $H\lhd G_i$ para todo $i\in \{0,1,\ldots,r\}$. Por el Tercer Teorema de Isomorfía, como $G_{i-1}\rhd G_i$, se tiene que $G_{i-1}/H \rhd G_i/H$ para todo $i\in \{1,\ldots,r\}$. Por tanto, la serie
    \[
        G/H = G_0/H \rhd G_1/H \rhd \cdots \rhd G_{r-1}/H \rhd G_r/H = H/H = \{1H\}
    \]
    es una serie normal de $G/H$. Además, por el Tercer Teorema de Isomorfía, se tiene que:
    \begin{equation*}
        \dfrac{G_{i-1}/H}{G_i/H} \cong {G_{i-1}/}{G_i} \qquad \forall i\in \{1,\ldots,r\}.
    \end{equation*}
    Como $G_{i-1}/G_i$ es simple por ser un factor de composición, y los factores se conservan bajo isomorfismos, se tiene que los factores de la serie de composición de $G/H$ son simples. Por tanto, la serie de $G/H$ es de composición, luego se cumple que $l(G/H)=r$ y se tiene que:
    \begin{equation*}
        l(H) + l(G/H) = m + r = l(G).
    \end{equation*}

    Por otro lado, los factores de la serie de composición de $G/H$ son isomorfos por el Tercer Teorema de Isomorfía a:
    \begin{equation*}
        \fact(G/H)=\bigcup_{i=0}^r G_i/G_{i+1}
    \end{equation*}

    Por tanto, se tiene que:
    \begin{equation*}
        \fact(G/H)\cup \fact(H) = \bigcup_{i=0}^{r+m-1} G_i/G_{i+1} = \fact(G).
    \end{equation*}
\end{ejercicio}

\begin{ejercicio}
    Encontrar todas las series de composición, calcular la longitud y la lista de factores de composición de los siguientes grupos:
    \begin{enumerate}
        \item El grupo diédrico $D_4$.
        
        Conviene tener presente el Diagrama de Hasse de $D_4$, presente en la Figura~\ref{fig:ej11_D4}. Simplemente lo usaremos para buscar todas las series normales de $D_4$ que no admitan refinamientos, consiguiendo así todas las series de composición. Para ello, iremos desde $D_4$ hasta $\{1\}$ por el grafo del retículo sin saltarnos vértices (evitando así los refinamientos) y yendo solo por los subgrupos normales en el anterior.\\

        En este caso, como todos los índices de un grupo en su subgrupo adyacente son $2$, todas las relaciones de inclusión dadas en el grafo son en realidad de normalidad. De hecho, todas las series de composición son las siguientes:
        \begin{align*}
            D_4 &\rhd \langle r^2, s \rangle \rhd  \langle s \rangle  \rhd \{1\} \\
            D_4 &\rhd \langle r^2, s \rangle \rhd \langle sr^2 \rangle  \rhd \{1\} \\
            D_4 &\rhd \langle r^2, s \rangle \rhd \langle r^2 \rangle  \rhd \{1\} \\
            D_4 &\rhd \langle r^2, sr \rangle  \rhd \langle r^2 \rangle  \rhd \{1\} \\
            D_4 &\rhd \langle r^2, sr \rangle  \rhd \langle sr \rangle  \rhd \{1\} \\
            D_4 &\rhd \langle r^2, sr \rangle  \rhd \langle sr^3 \rangle  \rhd \{1\} \\
            D_4 &\rhd \langle r \rangle \rhd \langle r^2 \rangle  \rhd \{1\}
        \end{align*}

        Como vemos:
        \begin{align*}
            l(D_4) &= 3 \\
            \fact(D_4) &= \{\bb{Z}_2, \bb{Z}_2, \bb{Z}_2\}
        \end{align*}
        \begin{observacion}
            Notemos que, puesto que dos series de composición de un mismo grupo $G$ son isomorfas, para calcular la lista de factores de composición de un grupo $G$ o su longitud basta con calcular una serie de composición, no todas. Se calculan todas puesto que es parte del ejercicio.
        \end{observacion}
        \item El grupo alternado $A_4$.
        
        El Diagrama de Hasse de $A_4$ está presente en la Figura~\ref{fig:ej11_A4}. Además, se vió que el único subgrupo normal propio de $A_4$ es $V$. Por tanto, las series de composición son las siguientes:
        \begin{align*}
            A_4 &\rhd V \rhd \langle (1\ 2)(3\ 4) \rangle  \rhd \{1\} \\
            A_4 &\rhd V \rhd \langle (1\ 3)(2\ 4) \rangle  \rhd \{1\} \\
            A_4 &\rhd V \rhd \langle (1\ 4)(2\ 3) \rangle  \rhd \{1\}
        \end{align*}

        Como vemos:
        \begin{align*}
            l(A_4) &= 3 \\
            \fact(A_4) &= \{\bb{Z}_3, \bb{Z}_2, \bb{Z}_2\}
        \end{align*}
        \item El grupo simétrico $S_4$.
        
        En primer lugar, sabemos que las siguientes son series de composición de $S_4$:
        \begin{align*}
            S_4 &\rhd A_4 \rhd V \rhd \langle (1\ 2)(3\ 4) \rangle  \rhd \{1\} \\
            S_4 &\rhd A_4 \rhd V \rhd \langle (1\ 3)(2\ 4) \rangle  \rhd \{1\} \\
            S_4 &\rhd A_4 \rhd V \rhd \langle (1\ 4)(2\ 3) \rangle  \rhd \{1\}
        \end{align*}

        No obstante, podría suceder que tuviese más grupos normales. Supongamos que existe $N\lhd S_4$ tal que $N\neq A_4$ y $N\neq \{1\}$.
        \begin{itemize}
            \item Si este contiene a un $n-$ciclo $\gamma\in N$, veamos que contiene a todos los $n-$ciclos. Dado otro $n-$ciclo $\sigma\in S_4$, sean:
            \begin{align*}
                \gamma &= (x_1\ \dots\ x_n)\in N \\
                \sigma &= (y_1\ \dots\ y_n)\in S_4
            \end{align*}

            Definimos ahora $\tau\in S_4$ como:
            \begin{align*}
                \tau(x_k) &= y_k\qquad \forall k\in \{1,\ldots,n\} \\
                \tau(k) &= k\qquad \forall k\in \{1,\ldots,4\}\setminus\{x_1,\ldots,x_n\}
            \end{align*}

            De esta forma, tenemos que $\sigma=\tau\gamma\tau^{-1}$. Como $N$ es normal, se tiene que $\sigma=\tau\gamma\tau^{-1}\in N$. Por tanto, $N$ contiene todos los $n-$ciclos.

            \item Si $N$ contiene un producto de dos transposiciones disjuntas $\gamma\in N$, veamos que contiene a todos los productos de dos transposiciones disjuntas. Sea $\gamma=\gamma_1\gamma_2\in N$ un producto de dos transposiciones disjuntas, y sea $\sigma=\sigma_1\sigma_2\in S_4$ un producto de dos transposiciones disjuntas.
            \begin{align*}
                \gamma &= \gamma_1\gamma_2=(x_1\ x_2)(x_3\ x_4)\in N \\
                \sigma &= \sigma_1\sigma_2=(y_1\ y_2)(y_3\ y_4)\in S_4
            \end{align*}
            
            Definimos $\tau\in S_4$ como:
            \begin{align*}
                \tau(x_n) &= y_n\qquad \forall n\in \{1,\ldots,4\} \\
                \tau(k) &= k\qquad \forall k\in \{1,\ldots,4\}\setminus\{x_1,x_2,y_1,y_2\}
            \end{align*}
            De esta forma, tenemos que:
            \begin{equation*}
                \tau\gamma\tau^{-1} = \tau\gamma_1\tau^{-1}\tau\gamma_2\tau^{-1} = (\tau(x_1)\ \tau(x_2))(\tau(x_3)\ \tau(x_4)) = (y_1\ y_2)(y_3\ y_4)=\sigma\in N.
            \end{equation*}
            Por tanto, $N$ contiene todos los productos de dos transposiciones disjuntas.
        \end{itemize}
        Este es el concepto de clase de conjugación, concepto que no se ha tratado pero no es difícil de entender. En $S_4$ hay:
        \begin{itemize}
            \item $1$ $1-$ciclo (la identidad).
            \item $6$ $2-$ciclos.
            \item $8$ $3-$ciclos.
            \item $6$ $4-$ciclos.
            \item $3$ productos de dos transposiciones disjuntas.
        \end{itemize}

        Efecetivamente, se tiene que $|A_4|=12=1+8+3$. Sea entonces $N$ un subgrupo normal propio de $S_4$.
        \begin{itemize}
            \item Supongamos que $N$ contiene un $2-$ciclo. Entonces, $|N|\geq 1+6=7$. Como $|N|$ es divisor de $|S_4|=24$, se tiene que $|N|=12$, luego faltarían $5$ elementos. No obstante, esto no es posible (puesto que no hay ninguna clase de conjugación con $5$ elementos). Por tanto, ningún $2-$ciclo pertenece a $N$.
            \item De forma análoga, se ve que no hay $4-$ciclos en $N$.
        \end{itemize}
        Como no hay $2-$ciclos ni $4-$ciclos, se tiene que $N\subset A_4$. Como $N$ es un grupo, se tiene que $N<A_4$. Si $N$ no es normal en $A_4$, entonces tampoco lo es en $S_4$, por lo que $N$ es normal en $A_4$ y entonces será necesario pasar por $A_4$ en la serie de composición.
        Por tanto, las únicas series de composición de $S_4$ son las anteriormente vistas:
        \begin{align*}
            S_4 &\rhd A_4 \rhd V \rhd \langle (1\ 2)(3\ 4) \rangle  \rhd \{1\} \\
            S_4 &\rhd A_4 \rhd V \rhd \langle (1\ 3)(2\ 4) \rangle  \rhd \{1\} \\
            S_4 &\rhd A_4 \rhd V \rhd \langle (1\ 4)(2\ 3) \rangle  \rhd \{1\}
        \end{align*}

        Como vemos:
        \begin{align*}
            l(S_4) &= 4 \\
            \fact(S_4) &= \{\bb{Z}_3, \bb{Z}_2, \bb{Z}_2, \bb{Z}_2\}
        \end{align*}
        \item El grupo diédrico $D_5$.
        
        Calculamos el orden de cada elemento de $D_5$:
        \begin{align*}
            O(r) &= O(r^2) = O(r^3) = O(r^4) = 5 \\
            O(s) &= O(sr) = O(sr^2) = O(sr^3) = O(sr^4) = 2
        \end{align*}

        Todo subgrupo de $D_5$ será de orden primo, luego será cíclico. El diagrama de Hasse de $D_5$ está presente en la Figura~\ref{fig:Hasse_D5}.
        \begin{figure}
            \centering
            \begin{tikzpicture}[node distance=1cm]
                \node (D5) {$D_5$};
                \node (r) [below left=of D5] {$\langle r \rangle$};
                \node (s) [below right=of r] {$\langle s \rangle$};
                \node (sr) [right=of s] {$\langle sr \rangle$};
                \node (sr2) [right=of sr] {$\langle sr^2 \rangle$};
                \node (sr3) [right=of sr2] {$\langle sr^3 \rangle$};
                \node (sr4) [right=of sr3] {$\langle sr^4 \rangle$};
                \node (1) [below=of s] {$\{1\}$};


                \draw (D5) -- (r);
                \draw (D5) -- (s);
                \draw (D5) -- (sr);
                \draw (D5) -- (sr2);
                \draw (D5) -- (sr3);
                \draw (D5) -- (sr4);
                \draw (r) -- (1);
                \draw (s) -- (1);
                \draw (sr) -- (1);
                \draw (sr2) -- (1);
                \draw (sr3) -- (1);
                \draw (sr4) -- (1);
            \end{tikzpicture}    
            \caption{Diagrama de Hasse para los subgrupos del grupo $D_5$.}
            \label{fig:Hasse_D5}
        \end{figure}     
        
        Veamos que los subgrupos generados por elementos de orden $2$ no son normales.
        \begin{itemize}
            \item $r\ s\ r^4 = sr^3 \notin \langle s \rangle$.
            \item $r\ sr \ r^4 = sr^4 \notin \langle sr \rangle$.
            \item $r\ sr^2 \ r^4 = sr^{10} = s \notin \langle sr^2 \rangle$.
            \item $r\ sr^3 \ r^4 = sr^{11} = sr \notin \langle sr^3 \rangle$.
            \item $r\ sr^4 \ r^4 = sr^{12} = sr^2 \notin \langle sr^4 \rangle$.
        \end{itemize}

        Por tanto, el único subgrupo normal de $D_5$ es $\langle r\rangle$. Por tanto, la única serie de composición es la siguiente:
        \begin{align*}
            D_5 &\rhd \langle r \rangle \rhd \{1\}
        \end{align*}

        Como vemos:
        \begin{align*}
            l(D_5) &= 2 \\
            \fact(D_5) &= \{\bb{Z}_2, \bb{Z}_5\}
        \end{align*}
        \item El grupo de cuaterniones $Q_2$.
        
        El Diagrama de Hasse de $Q_2$ está presente en la Figura~\ref{fig:ej11_Q2}. Como todos los índices son $2$, todas las relaciones de inclusión son de normalidad. Por tanto, las series de composición son las siguientes:
        \begin{align*}
            Q_2 &\rhd \langle i \rangle \rhd \{-1\} \rhd \{1\} \\
            Q_2 &\rhd \langle j \rangle \rhd \{-1\} \rhd \{1\} \\
            Q_2 &\rhd \langle k \rangle \rhd \{-1\} \rhd \{1\}
        \end{align*}

        Como vemos:
        \begin{align*}
            l(Q_2) &= 3 \\
            \fact(Q_2) &= \{\bb{Z}_2, \bb{Z}_2, \bb{Z}_2\}
        \end{align*}
        \item El grupo cíclico $C_{24}$.
        
        Sabemos que los subgrupos de $C_{24}$ son cíclicos, y por tanto abelianos. Por tanto, todos los subgrupos son normales. El Diagrama de Hasse de $C_{24}$ está presente en la Figura~\ref{fig:Hasse_C24}.
        \begin{figure}
            \centering
            \begin{tikzpicture}[node distance=1cm]
                \node (C24) {$C_{24}$};
                \node (C12) [below right=of C24, xshift=-1cm, yshift=0.2cm] {$\langle x^2 \rangle \cong C_{12}$};
                \node (C8) [below left=of C12, yshift=0.5cm] {$\langle x^3 \rangle \cong C_8$};
                \node (C6) [below right=of C8, yshift=0.5cm] {$\langle x^4 \rangle \cong C_6$};
                \node (C4) [below left=of C6, yshift=0.5cm] {$\langle x^6 \rangle \cong C_4$};
                \node (C3) [below right=of C4, yshift=0.5cm] {$\langle x^8 \rangle \cong C_3$};
                \node (C2) [below left=of C3, yshift=0.5cm] {$\langle x^{12} \rangle \cong C_2$};
                \node (C1) [below=of C2, xshift=1cm, yshift=0.5cm] {$\{1\}$};

                \draw (C24) -- (C12) -- (C6) -- (C3) -- (C1);
                \draw (C24) -- (C8) -- (C4) -- (C2) -- (C1);

                \draw (C12) -- (C4);
                \draw (C6) -- (C2);

            \end{tikzpicture}
            \caption{Diagrama de Hasse para los subgrupos del grupo $C_{24}$.}
            \label{fig:Hasse_C24}
        \end{figure}
        
        Las series de composición son, por tanto, las siguientes:
        \begin{align*}
            C_{24} &\rhd \langle C_{12} \rangle \rhd \langle C_{6} \rangle \rhd \langle C_{3} \rangle \rhd \{1\} \\
            C_{24} &\rhd \langle C_{12} \rangle \rhd \langle C_{6} \rangle \rhd \langle C_{2} \rangle \rhd \{1\} \\
            C_{24} &\rhd \langle C_{12} \rangle \rhd \langle C_{4} \rangle \rhd \langle C_{2} \rangle \rhd \{1\} \\
            C_{24} &\rhd \langle C_{8} \rangle \rhd \langle C_{4} \rangle \rhd \langle C_{2} \rangle \rhd \{1\}
        \end{align*}

        Como vemos:
        \begin{align*}
            l(C_{24}) &= 4 \\
            \fact(C_{24}) &= \{\bb{Z}_2, \bb{Z}_2, \bb{Z}_2, \bb{Z}_3\}
        \end{align*}

        \item El grupo simétrico $S_5$.
        
        Como $A_5$ es normal en $S_5$, se tiene que la siguiente es una serie normal:
        \begin{align*}
            S_5 &\rhd A_5 \rhd \{1\}
        \end{align*}

        Además, $S_5/A_5$ es simple por ser de orden primo, mientras que $A_5/\{1\}\cong A_5$ es simple por el Lema de Abel. Por tanto, la serie es de composición.

        No obstante, podría suceder que tuviese más grupos normales. Supongamos que existe $N\lhd S_5$ tal que $N\neq A_5$ y $N\neq \{1\}$. Por una demostración análoga a la de $S_4$, las clases de conjugación de $S_5$ son las siguientes:
        \begin{itemize}
            \item $1$ $1-$ciclo (la identidad).
            \item $10$ $2-$ciclos.
            \item $20$ $3-$ciclos.
            \item $30$ $4-$ciclos.
            \item $24$ $5-$ciclos.
            \item $15$ productos de dos transposiciones disjuntas.
            \item $20$ productos de un $2-$ciclo y un $3-$ciclo.
        \end{itemize}

        Efecetivamente, se tiene que $|A_5|=60=1+20+24+15$. Sea entonces $N$ un subgrupo normal propio de $S_4$.
        \begin{itemize}
            \item Supongamos que $N$ contiene un $2-$ciclo. Entonces, $|N|\geq 1+10=11$, que no divide a $120$. Como la siguiente clase de conjugación más pequeña es de $15$ elementos, sabemos que $|N|>26$. Por tanto, $|N|\in \{30,40,60\}$. Para, desde $11$ podemos sumár un múltiplo de $10$, es necesario que contenga a los $24$ $5-$ciclos y a los $15$ productos de dos transposiciones disjuntas, luego $|N|\geq 11+15+24=50$, luego $|N|=60$, por lo que tan solo nos falta por determinar $10$ elementos. No obstante, todas las clases restantes son de más de $10$ elementos. Por tanto, no puede contener ningún $2-$ciclo.
            
            \item Supongamos que $N$ contiene un $4-$ciclo. Entonces, $|N|\geq 1+30=31$, que no divide a $120$. Por tanto, $|N|\in \{40,60\}$. Para, desde $31$ podemos sumar un múltiplo de $10$, es necesario que contenga a los $24$ $5-$ciclos y a los $15$ productos de dos transposiciones disjuntas, pero $31+24+15>60$. Por tanto, no puede contener ningún $4-$ciclo.
            

            \item Supongamos que $N$ contiene un producto de un $2-$ciclo y un $3-$ciclo. Entonces, $|N|\geq 1+20=21$, que no divide a $120$. Como la siguiente clase de conjugación más pequeña es de $10$ elementos, sabemos que $|N|>31$. Por tanto, $|N|\in \{40,60\}$. Para, desde $21$ podemos sumar un múltiplo de $10$, es necesario que contenga a los $24$ $5-$ciclos y a los $15$ productos de dos transposiciones disjuntas, luego $|N|\geq 21+15+24=60$, luego $|N|=60$. Por tanto, $N$ está formado por:
            \begin{itemize}
                \item $1$ $1-$ciclo (la identidad).
                \item $20$ productos de un $2-$ciclo y un $3-$ciclo.
                \item $24$ $5-$ciclos.
                \item $15$ productos de dos transposiciones disjuntas.
            \end{itemize}

            No obstante, veamos que $N$ no es un subgrupo de $S_5$ puesto que no es cerrado por producto:
            \begin{equation*}
                (1\ 2)(3\ 4\ 5)\ (1\ 2)(3\ 4) = (1\ 2)(1\ 2)(3\ 4\ 5)(3\ 4)
                = (3\ 4\ 5)(3\ 4) = (3\ 5)\notin N
            \end{equation*}
            Por tanto, no puede contener ningún producto de un $2-$ciclo y un $3-$ciclo.
        \end{itemize}
        Por tanto, $N\subset A_5$. Como $N$ es un grupo, se tiene que $N<A_5$. Si $N$ no es normal en $A_5$, entonces tampoco lo es en $S_5$, por lo que $N$ es normal en $A_5$. No obstante, $A_5$ es simple, luego $N=A_5$.
        Por tanto, la única serie de composición de $S_5$ es la siguiente:
        \begin{align*}
            S_5 &\rhd A_5 \rhd \{1\}
        \end{align*}

        Como vemos:
        \begin{align*}
            l(S_5) &= 2 \\
            \fact(S_5) &= \{\bb{Z}_2, A_5\}
        \end{align*}
    \end{enumerate}
\end{ejercicio}

\begin{ejercicio}\label{ej:5.6}
    Sea $G$ un grupo finito, y
    \[
        G = G_0 \rhd G_1 \rhd \cdots \rhd G_{r-1} \rhd G_r = \{1\}
    \]
    una serie normal de $G$. Demostrar que
    \[
        l(G) = \sum_{i=0}^{r-1} l\left(\frac{G_i}{G_{i+1}}\right), \quad \fact(G) = \bigcup_{i=0}^{r-1} \fact\left(\frac{G_i}{G_{i+1}}\right).
    \]

    Como $G$ es finito y $G_1\lhd G$, por el Ejercicio~\ref{ej:5.4} se tiene que:
    \begin{align*}
        l(G) &= l(G_1) + l(G/G_1) \\
        \fact(G) &= \fact(G_1)\cup\fact(G/G_1).
    \end{align*}

    Como $G_1$ es finito y $G_2\lhd G_1$, por el Ejercicio~\ref{ej:5.4} se tiene que:
    \begin{align*}
        l(G) = l(G_1) + l(G/G_1) &= l(G_2) + l(G_1/G_2) + l(G/G_1) \\
        &= l(G_2) + l(G_1/G_2) + l(G/G_1) \\
        \fact(G) &= \fact(G_1)\cup\fact(G/G_1) = \fact(G_2)\cup\fact(G_1/G_2)\cup\fact(G/G_1).
    \end{align*}

    Iterando hasta usar que $G_{r}\lhd G_{r-1}$, se tiene que:
    \begin{align*}
        l(G) &= \sum_{i=0}^{r-1} l(G_i/G_{i+1}) + \cancel{l(G_r)} \\
        \fact(G) &= \bigcup_{i=0}^{r-1} \fact(G_i/G_{i+1}) \cup \cancel{\fact(G_r)}.
    \end{align*}
\end{ejercicio}

\begin{ejercicio}\label{ej:5.7}
    Si $G_1, G_2, \ldots, G_r$ son grupos finitos, demostrar que
    \[
        l(G_1 \times G_2 \times \cdots \times G_r) = \sum_{i=1}^{r} l(G_i), \quad \fact(G_1 \times G_2 \times \cdots \times G_r) = \bigcup_{i=1}^{r} \fact(G_i).
    \]

    Demostramos por inducción sobre $r$.
    \begin{itemize}
        \item Para $r=1$ se tiene trivialmente.
        \item Supuesto cierto para $r$, demostrémoslo para $r+1$.\\
        
        Buscamos demostrarlo aplicando el Ejercicio~\ref{ej:5.4}. Para ello, necesitamos un subgrupo normal de $G_1 \times \cdots \times G_r \times G_{r+1}$. Definimos:
        \Func{\pi}{G_1 \times \cdots \times G_r \times G_{r+1}}{G_1\times \cdots \times G_r}{(g_1, g_2, \ldots, g_r, g_{r+1})}{(g_1, g_2, \ldots, g_r)}

        Tenemos que $\pi$ es un homomorfismo con:
        \begin{align*}
            \ker(\pi) &= \{1\}\times \cdots \times \{1\} \times G_{r+1} \\
            Im(\pi) &= G_1\times \cdots \times G_r.
        \end{align*}

        Por el Primer Teorema de Isomorfía, se tiene que:
        \begin{align*}
            \dfrac{G_1 \times \cdots \times G_r \times G_{r+1}}{\{1\}\times \cdots \times \{1\} \times G_{r+1}} &\cong G_1\times \cdots \times G_r.
        \end{align*}

        Veamos ahora que $\{1\}\times \cdots \times \{1\} \times G_{r+1}$ es isomorfo a $G_{r+1}$. Definimos:
        \Func{\phi}{\{1\}\times \cdots \times \{1\} \times G_{r+1}}{G_{r+1}}{(1, \ldots, 1, g_{r+1})}{g_{r+1}}

        Vemos claramente que $\phi$ es un isomorfismo, luego $\{1\}\times \cdots \times \{1\} \times G_{r+1} \cong G_{r+1}$.\\

        Vistos ambos aspectos, como $\{1\}\times \cdots \times \{1\} \times G_{r+1}=\ker(\pi) \lhd G_1 \times \cdots \times G_r \times G_{r+1}$ por el Ejercicio~\ref{ej:5.4}, se tiene que:
        \begin{align*}
            l(G_1 \times \cdots \times G_r \times G_{r+1}) &= l(\{1\}\times \cdots \times \{1\} \times G_{r+1}) + l\left(\dfrac{G_1 \times \cdots \times G_r \times G_{r+1}}{\{1\}\times \cdots \times \{1\} \times G_{r+1}}\right)
        \end{align*}

        Como las series de composición de dos grupos isomorfas son isomorfas, tenemos que:
        \begin{align*}
            l(G_1 \times \cdots \times G_r \times G_{r+1}) &= l(G_{r+1}) + l\left(G_1 \times \cdots \times G_r\right) \AstIg
             l(G_{r+1}) + \sum_{i=1}^{r} l(G_i)
            = \sum_{i=1}^{r+1} l(G_i).
        \end{align*}
        donde en $(\ast)$ hemos usado la hipótesis de inducción.\\

        De igual forma, usando de nuevo el Ejercicio~\ref{ej:5.4} se tiene que:
        \begin{align*}
            \fact(G_1 \times \cdots \times G_r \times G_{r+1}) &= \fact(\{1\}\times \cdots \times \{1\} \times G_{r+1}) \cup \fact\left(\dfrac{G_1 \times \cdots \times G_r \times G_{r+1}}{\{1\}\times \cdots \times \{1\} \times G_{r+1}}\right)\\
            &\AstIg \fact(G_{r+1})\cup\fact(G_1\times\cdots\times G_r)\\
            &\stackrel{(\ast\ast)}{=} \fact(G_{r+1})\cup \bigcup_{i=1}^{r} \fact(G_i) = \bigcup_{i=1}^{r+1} \fact(G_i).
        \end{align*}
        donde en $(\ast\ast)$ hemos usado la hipótesis de inducción y en $(\ast)$ hemos empleado que las series de composición de dos grupos isomorfas son isomorfas, luego sus factores de composición son isomorfos y por tanto el conjunto $\fact$ de ambos grupos es el mismo (salvo la observacón que hicimos de isomorfismos en el Ejercicio~\ref{ej:5.4}).
    \end{itemize}

    Por tanto, se ha demostrado el resultado por inducción.
\end{ejercicio}

\begin{ejercicio}\label{ej:5.8}
    Sea $G$ un grupo cíclico de orden $p^n$ con $p$ primo. Demostrar que $l(G) = n$ y que $\fact(G) = (\bb{Z}_p, \bb{Z}_p, \stackrel{(n)}{\ldots}, \bb{Z}_p)$ ($n$ veces).\\

    Conviene tener presente el diagrama de Hasse de los subgrupos de $G=\langle g\rangle$, presente en la Figura~\ref{fig:ej12}. Además, como $G$ es cíclico, en particular es abeliano y todos sus subgrupos son abelianos, luego todas las relaciones de inclusión son de normalidad. Por tanto, la única serie de composición es la siguiente:
    \begin{align*}
        G=\langle g^{p^0}\rangle &\rhd \langle g^{p} \rangle \rhd \langle g^{p^2} \rangle \rhd \cdots \rhd \langle g^{p^{n-1}} \rangle \rhd \langle g^{p^n} \rangle = \{1\}
    \end{align*}

    De esta serie de composición se deduce que $l(G)=n$. Veamos cuáles son los factores de composición:
    \begin{equation*}
        \left|\dfrac{\langle g^{p^i} \rangle}{\langle g^{p^{i+1}} \rangle}\right| = \dfrac{|\langle g^{p^i} \rangle|}{|\langle g^{p^{i+1}} \rangle|} = \dfrac{O(g^{p^i})}{O(g^{p^{i+1}})} = \dfrac{\nicefrac{p^n}{\mcd(p^n, p^i)}}{\nicefrac{p^n}{\mcd(p^n, p^{i+1})}} = \dfrac{\mcd(p^n, p^{i+1})}{\mcd(p^n, p^i)} = \dfrac{p^{i+1}}{p^i} = p.\quad \forall i\in\{0,\ldots,n-1\}
    \end{equation*}

    Por tanto, se tiene que:
    \begin{equation*}
        \dfrac{\langle g^{p^i} \rangle}{\langle g^{p^{i+1}} \rangle} \cong \bb{Z}_p\qquad \forall i\in\{0,\ldots,n-1\}
    \end{equation*}

    Por tanto, los factores de composición son:
    \begin{equation*}
        \fact(G) = \left(\bb{Z}_p, \bb{Z}_p, \stackrel{(n)}{\ldots}, \bb{Z}_p\right).
    \end{equation*}
\end{ejercicio}

\begin{ejercicio}\label{ej:5.9}
    Sea $G$ un grupo cíclico de orden $n$. Si la descomposición de $n$ en factores primos es $n = p_1^{e_1} p_2^{e_2} \cdots p_r^{e_r}$, demostrar que
    \[
        l(G) = e_1 + e_2 + \cdots + e_r,
    \]
    y que
    \[
        \fact(G) = (\bb{Z}_{p_1}, \stackrel{(e_1)}{\ldots}\bb{Z}_{p_1}, \dots, \bb{Z}_{p_r},\stackrel{(e_r)}{\ldots}, \bb{Z}_{p_r}).
    \]
    Aplica el resultado cuando $n = 12$ y compara su longitud y factores de composición con los del grupo $\bb{Z}_2 \times \bb{Z}_6$.\\

    Sabemos que $\mcd(p_1,\dots,p_r)=1$, luego $\mcd(p_1^{e_1},\dots,p_r^{e_r})=1$. Por tanto, se tiene que:
    \begin{equation*}
        \prod_{i=1}^{r} C_{p_i^{e_i}}\qquad \text{es cíclico}
    \end{equation*}

    Además, se tiene que:
    \begin{equation*}
        \left|\prod_{i=1}^{r} C_{p_i^{e_i}}\right| = \prod_{i=1}^{r} |C_{p_i^{e_i}}| = \prod_{i=1}^{r} p_i^{e_i} = n.
    \end{equation*}

    Por tanto, $G\cong \prod\limits_{i=1}^{r} C_{p_i^{e_i}}$. Como dos grupos isomorfos tienen series de composición isomorfas, se tiene que:
    \begin{align*}
        l(G) &= l\left(\prod_{i=1}^{r} C_{p_i^{e_i}}\right) 
        \AstIg \sum_{i=1}^{r} l\left(C_{p_i^{e_i}}\right) 
        \stackrel{(\ast\ast)}{=} \sum_{i=1}^{r} e_i
    \end{align*}
    donde en $(\ast)$ hemos usado el Ejercicio~\ref{ej:5.7} y en $(\ast\ast)$ el Ejercicio~\ref{ej:5.8}.\\

    Veamos ahora cuáles son los factores de composición. Como las series de composición de dos grupos isomorfos son isomorfas y, por tanto, sus factores de composición son isomorfos, se tiene que:
    \begin{align*}
        \fact(G) &= \fact\left(\prod_{i=1}^{r} C_{p_i^{e_i}}\right) 
        \AstIg \bigcup_{i=1}^{r} \fact\left(C_{p_i^{e_i}}\right) 
        \stackrel{(\ast\ast)}{=} \bigcup_{i=1}^{r} \left(\bb{Z}_{p_i}, \bb{Z}_{p_i}, \stackrel{(e_i)}{\ldots}, \bb{Z}_{p_i}\right)\\
        &= (\bb{Z}_{p_1}, \stackrel{(e_1)}{\ldots}\bb{Z}_{p_1}, \dots, \bb{Z}_{p_r},\stackrel{(e_r)}{\ldots}, \bb{Z}_{p_r}).
    \end{align*}
    donde en $(\ast)$ hemos usado el Ejercicio~\ref{ej:5.7} y en $(\ast\ast)$ el Ejercicio~\ref{ej:5.8}. Por tanto, se ha demostrado el resultado.\\

    Aplicándolo ahora a $n=12$, se tiene que $12=2^2\cdot 3^1$, luego:
    \begin{align*}
        l(\bb{Z}_{12}) &= 2+1 = 3 \\
        \fact(\bb{Z}_{12}) &= (\bb{Z}_2, \bb{Z}_2, \bb{Z}_3).
    \end{align*}

    Queremos calcular ahora la longitud y factores de composición de $\bb{Z}_2 \times \bb{Z}_6$. Como este no es cíclico, calculamos su longitud y factores de composición usando el Ejercicio~\ref{ej:5.7}:
    \begin{align*}
        l(\bb{Z}_2 \times \bb{Z}_6) &= l(\bb{Z}_2) + l(\bb{Z}_6) = 1 + 1+1 = 3 \\
        \fact(\bb{Z}_2 \times \bb{Z}_6) &= \fact(\bb{Z}_2)\cup\fact(\bb{Z}_6) = (\bb{Z}_2)\cup(\bb{Z}_2, \bb{Z}_3) = (\bb{Z}_2, \bb{Z}_2, \bb{Z}_3).
    \end{align*}

    Comprobamos por tanto que, aun no siendo isomorfos (puesto que uno es cíclico y el otro no), se cumple que:
    \begin{align*}
        l(\bb{Z}_{12}) &= l(\bb{Z}_2 \times \bb{Z}_6) = 3 \\
        \fact(\bb{Z}_{12}) &= \fact(\bb{Z}_2 \times \bb{Z}_6) = (\bb{Z}_2, \bb{Z}_2, \bb{Z}_3).
    \end{align*}

    Notemos que si dos grupos son isomorfos entonces tienen la misma longitud y los mismos factores de composición, pero el recíproco no es cierto.
\end{ejercicio}

\begin{ejercicio}
    Sea $D_n$ el grupo diédrico de orden $2n$. Si la descomposición de $n$ en factores primos es $n = p_1^{e_1} p_2^{e_2} \cdots p_r^{e_r}$, demostrar que
    \[
        l(D_n) = e_1 + e_2 + \cdots + e_r + 1,
    \]
    y que
    \[
        \fact(D_n) = (\bb{Z}_{p_1}, \stackrel{(e_1)}{\ldots}\bb{Z}_{p_1}, \dots, \bb{Z}_{p_r},\stackrel{(e_r)}{\ldots}, \bb{Z}_{p_r}, \bb{Z}_2).
    \]

    Sabemos que la siguiente serie es una serie normal de $D_n$:
    \begin{align*}
        D_n &\rhd \langle r \rangle \rhd \{1\}
    \end{align*}

    Por tanto, por el Ejercicio~\ref{ej:5.6} se tiene que:
    \begin{align*}
        l(D_n) &= l\left(\dfrac{D_n}{\langle r \rangle}\right) + l\left(\dfrac{\langle r \rangle}{\{1\}}\right) \\
        \fact(D_n) &= \fact\left(\dfrac{D_n}{\langle r \rangle}\right) \cup \fact\left(\dfrac{\langle r \rangle}{\{1\}}\right)
    \end{align*}

    Sabemos que $|D_n/\langle r \rangle|=\nicefrac{2n}{n}=2$, luego $D_n/\langle r \rangle \cong \bb{Z}_2$. Por otro lado, sabemos que $\langle r \rangle$ es cíclico de orden $n$, luego $\langle r \rangle \cong \bb{Z}_n$. Como la longitud y los factores se mantienen bajo isomorfismos, y usando el Ejercicio~\ref{ej:5.9} con $n = p_1^{e_1} p_2^{e_2} \cdots p_r^{e_r}$ y $2$ primo, se tiene que:
    \begin{align*}
        l(D_n) &= l\left(\dfrac{D_n}{\langle r \rangle}\right) + l\left(\dfrac{\langle r \rangle}{\{1\}}\right) 
        = l(\bb{Z}_2) + l(\bb{Z}_n) 
        = 1 + \left(e_1 + e_2 + \cdots + e_r\right) \\
        &= e_1 + e_2 + \cdots + e_r + 1\\
        \fact(D_n) &= \fact\left(\dfrac{D_n}{\langle r \rangle}\right) \cup \fact\left(\dfrac{\langle r \rangle}{\{1\}}\right)
        = \fact(\bb{Z}_2) \cup \fact(\bb{Z}_n) \\
        &= (\bb{Z}_2) \cup \left(\bb{Z}_{p_1}, \stackrel{(e_1)}{\ldots}\bb{Z}_{p_1}, \dots, \bb{Z}_{p_r},\stackrel{(e_r)}{\ldots}, \bb{Z}_{p_r}\right)\\
        &= (\bb{Z}_{p_1}, \stackrel{(e_1)}{\ldots}\bb{Z}_{p_1}, \dots, \bb{Z}_{p_r},\stackrel{(e_r)}{\ldots}, \bb{Z}_{p_r}, \bb{Z}_2).
    \end{align*}
\end{ejercicio}

\begin{ejercicio}
    Demostrar que $D_n$, $S_2$, $S_3$ y $S_4$ son grupos resolubles.
    \begin{enumerate}
        \item $D_n$.
        
        Una serie normal de $D_n$ es la siguiente:
        \begin{align*}
            D_n &\rhd \langle r \rangle \rhd \{1\}
        \end{align*}

        Sus factores son:
        \begin{align*}
            \dfrac{D_n}{\langle r \rangle} &\cong \bb{Z}_2 \\
            \dfrac{\langle r \rangle}{\{1\}} &\cong \langle r \rangle \cong \bb{Z}_n
        \end{align*}

        Por tanto, todos sus factores son abelianos, luego $D_n$ es resoluble.

        \item $S_2$.
        
        La serie derivada de $S_2$ es la siguiente:
        \begin{align*}
            S_2 &\rhd \{1\}
        \end{align*}

        Donde he empleado que $S_2\cong C_2$ es abeliano, luego $[S_2, S_2]=\{1\}$.
        Por tanto, $S_2$ es resoluble.

        \item $S_3$.
        
        Sabemos que $S_3'=[S_3, S_3]=A_3\cong C_3$ abeliano, luego la serie derivada de $S_3$ es la siguiente:
        \begin{align*}
            S_3 &\rhd A_3 \rhd \{1\}
        \end{align*}
        Por tanto, $S_3$ es resoluble.

        \item $S_4$.
        
        Una serie normal de $S_4$ es la siguiente:
        \begin{align*}
            S_4 &\rhd A_4 \rhd V \rhd \{1\}
        \end{align*}

        Sus factores son:
        \begin{align*}
            \dfrac{S_4}{A_4} &\cong \bb{Z}_2 \\
            \dfrac{A_4}{V} &\cong \bb{Z}_3 \\
            \dfrac{V}{\{1\}} &\cong V
        \end{align*}
        Donde $V$ es el grupo de Klein, que es abeliano. Por tanto, todos sus factores son abelianos, luego $S_4$ es resoluble.        
    \end{enumerate}
\end{ejercicio}

\begin{ejercicio}
    Sean $H$ y $K$ subgrupos normales de un grupo $G$ tales que $G/H$ y $G/K$ son ambos resolubles. Demostrar que $G/(H \cap K)$ también es resoluble.\\


    Por el Segundo Teorema de Isomorfía, como $H\lhd G$, tenemos $(H\cap K)\lhd K$ y:
    \begin{align*}
        \dfrac{K}{H\cap K} &\cong \dfrac{KH}{H}
    \end{align*}

    Este Teorema también afirma que $KH<G$, luego $KH/H<G/H$. Como $G/H$ es resoluble, se tiene que $KH/H$ es resoluble. Por tanto, $K/(H\cap K)$ es resoluble.\\

    Por otro lado, como $K,H\lhd G$, se tiene que $(H\cap K)\lhd G$. Como $(H\cap K)\subset K$ y $K\lhd G$, por el Tercer Teorema de Isomorfía se tiene que $H/(H\cap K)\lhd G/(H\cap K)$ y:
    \begin{align*}
        \dfrac{G/(H\cap K)}{K/(H\cap K)} &\cong \dfrac{G}{K}
    \end{align*}

    Como $G/K$ es resoluble, se tiene que $G/(H\cap K)/K/(H\cap K)$ es resoluble (puesto que esta propiedad se mantiene por isomorfismo).\\

    Como $\dfrac{G/(H\cap K)}{K/(H\cap K)}$ y $K/(H\cap K)$ son ambos resolubles, entonces $G/(H\cap K)$ es resoluble.
\end{ejercicio}

\begin{ejercicio}
    Sea $G$ un grupo resoluble y sea $H$ un subgrupo normal no trivial de $G$. Demostrar que existe un subgrupo no trivial $A$ de $H$ que es abeliano y normal en $G$.\\

    Como $H<G$, entonces $H$ es resoluble. Consideramos su serie derivada:
    \begin{align*}
        H &\rhd H' \rhd H'' \rhd \cdots \rhd H^{(n)} = \{1\}
    \end{align*}
    Como $H\neq \{1\}$, $n\neq 0$. Sea ahora $A=H^{(n-1)}$ (que podemos considerarlo puesto que $n\neq 0$). Como $[A,A]=[H^{(n-1)},H^{(n-1)}]=H^{(n)}=\{1\}$, se tiene que $A$ es abeliano. Nos falta por ver que $A\lhd G$.\\


    Consideramos la siguiente serie normal de $G$:
    \begin{align*}
        G &\rhd H\rhd H' \rhd H'' \rhd \cdots \rhd H^{(n)} = \{1\}
    \end{align*}
    
    Veamos que $H^{(i)}\lhd G$ para todo $i\in\{0,\ldots,n\}$.
    \begin{itemize}
        \item Para $i=0$, $G\lhd H$, luego se tiene que $H^0\lhd G$.
        \item Supuesto cierto para $i$, veamos que se cumple para $i+1$.
        
        Sabemos que $H^{(i)}\lhd G$, y queremos ver que $[H^{(i)},H^{(i)}]\lhd G$. Como se tiene que $[H^{(i)},H^{(i)}]=\langle [x,y] \mid x,y\in H^{(i)}\rangle$ y $[x,y]^{-1}=[y,x]$, tan solo es necesario comprobarlo sobre los generadores. Por tanto, sea $x,y\in H^{(i)}$, $g\in G$. Entonces:
        \begin{equation*}
            g[x,y]g^{-1} = [gxg^{-1},gyg^{-1}]
        \end{equation*}
        Como $H^{(i)}\lhd G$, se tiene que $gxg^{-1},gyg^{-1}\in H^{(i)}$, luego concluimos que $[gxg^{-1},gyg^{-1}]\in [H^{(i)},H^{(i)}]$. Por tanto, $H^{(i+1)}= [H^{(i)},H^{(i)}]\lhd G$.
    \end{itemize}

    Por tanto, $H^{(i)}\lhd G$ para todo $i\in\{0,\ldots,n\}$. En particular, $A=H^{(n-1)}\lhd G$.
\end{ejercicio}

\begin{ejercicio}\label{ej:5.14}
    Demuestra que todo $p$-grupo finito es resoluble.


    Esto equivale a probar que, para todo $n\in\bb{N}$, todo grupo de orden $p^n$ es resoluble. Vamos a demostrarlo por inducción sobre $n$.
    \begin{itemize}
        \item \ul{Caso base}: $n=1$.
        
        Sea $G$ un grupo de orden $p$. Entonces, $G$ es cíclico, luego abeliano y por tanto resoluble.

        \item \ul{Paso inductivo}: Supongamos que todo grupo de orden $p^k$, con $k\in \bb{N}$, $k<n$ es resoluble. Demostremos que todo grupo de orden $p^{n}$ es resoluble.
        
        Sea $G$ un grupo de orden $p^{n}$. El procedimiento será ver que $Z(G)$, $Z/Z(G)$ son ambos resolubles, y por tanto $G$ es resoluble.
        

        Sabemos en primer lugar que $Z(G)$ es abeliano, luego resoluble. Por otro lado, sabemos que $|Z(G)|\mid |G|$, luego $|Z(G)|=p^k$ para algún $k\in\{0,\ldots,n\}$.
        Además, como $G$ es un $p$-grupo, en particular su centro es no trivial, luego $k\neq 0$.
        Por tanto, tenemos que:
        \begin{equation*}
            \left|G/Z(G)\right| = \dfrac{|G|}{|Z(G)|} = \dfrac{p^{n}}{p^k} = p^{n-k}
        \end{equation*}
        Como $n-k<n$, por la hipótesis de inducción se tiene que $G/Z(G)$ es resoluble.

        Como $Z(G)$ y $G/Z(G)$ son ambos resolubles, entonces $G$ es resoluble.
    \end{itemize}

    Por tanto, se ha demostrado que todo grupo de orden $p^n$ es resoluble para todo $n\in\bb{N}$.
\end{ejercicio}

\begin{ejercicio}\label{ej:5.15}
    Demuestra que todo grupo de orden $pq$, con $p$ y $q$ primos, es un grupo resoluble.\\

    Sea $G$ un grupo de orden $pq$, y supongamos sin pérdida de generalidad que $p\geq q$. Sea $n_p$ el número de $p-$subgrupos de Sylow de $G$. Por el Segundo Teorema de Sylow, se tiene que:
    \begin{align*}
        n_p &\equiv 1 \mod p \\
        n_p &\mid q
    \end{align*}

    Como $n_p\mid q$, entonces $n_p\leq q\leq p$. Como $n_p\equiv 1 \mod p$, concluimos que $n_p=1$. Sea $P$ el único $p$-subgrupo de Sylow de $G$. Como es el único, entonces $P\lhd G$. Por ser un $p-$subgrupo de Sylow de $G$, entonces $|P|=p$. Por tanto, tenemos que:
    \begin{itemize}
        \item $|P|=p$ primo, luego $P$ es cíclico y abeliano, luego $P$ es resoluble.
        \item $|G/P|=q$ primo, luego $G/P$ es cíclico y abeliano, luego $G/P$ es resoluble.
    \end{itemize}
    Por tanto, como $P$ y $G/P$ son ambos resolubles, entonces $G$ es resoluble.
\end{ejercicio}

\begin{ejercicio}
    Demuestra que todo grupo de orden $p^2q$, con $p$ y $q$ primos, es un grupo resoluble.\\

    \begin{description}
        \item [Opción 1.]~\\
    Sea $G$ un grupo de orden $p^2q$. Sea $n_p$ el número de $p-$subgrupos de Sylow de $G$. Por el Segundo Teorema de Sylow, se tiene que:
    \begin{align*}
        n_p &\equiv 1 \mod p \\
        n_p &\mid q
    \end{align*}

    Por la segunda condición, tenemos que $n_p\in \{1,q\}$.
    Hay dos opciones:
    \begin{itemize}
        \item \ul{$n_p=1$}.
        
        En tal caso, sea $P$ el único $p$-subgrupo de Sylow de $G$. Como es el único, entonces $P\lhd G$. Por ser un $p-$subgrupo de Sylow de $G$, entonces $|P|=p^2$. Por tanto, tenemos que:
        \begin{itemize}
            \item $|P|=p^2$, luego $P$ es un $p-$grupo, luego resoluble.
            \item $|G/P|=q$, luego $G/P$ es un $q-$grupo, luego resoluble.
        \end{itemize}
        Por tanto, como $P$ y $G/P$ son ambos resolubles, entonces $G$ es resoluble.
        
        \item \ul{$n_p=q$:} Sea $n_q$ el número de $q-$subgrupos de Sylow de $G$. Por el Segundo Teorema de Sylow, se tiene que:
        \begin{align*}
            n_q &\equiv 1 \mod q \\
            n_q &\mid p^2
        \end{align*}
        Por la segunda condición, tenemos que $n_q\in \{1,p,p^2\}$.
        \begin{itemize}
            \item Si $n_q=p$, como $n_q\equiv 1 \mod q$, entonces $\exists k\in\bb{N}$ tal que:
            \begin{equation*}
                p = 1+ kq
            \end{equation*}

            Como $n_p\mid q$, $\exists k'\in\bb{N}$ tal que $q=k'n_p$, luego:
            \begin{equation*}
                p = 1 + k'kn_p
            \end{equation*}

            Por último, como $n_p\equiv 1 \mod p$, $\exists k'''\in\bb{N}$ tal que $n_p = 1 + k''p$, luego:
            \begin{equation*}
                p = 1 + k'k''(1+k''')p > p
            \end{equation*}
            Lo cual es una contradicción, luego $n_q\neq p$.

            \item Si $n_q=p^2$, tenemos que hay $p^2$ $q$-subgrupos de Sylow de $G$; es decir, hay $p^2$ subgrupos de orden $q$. Sean estos $Q_1,\ldots,Q_{p^2}$. Fijado $i\in \{1,\ldots,p^2\}$, tenemos que $|Q_i|=q$, luego todo elemento distinto de $1$ de $Q_i$ tiene orden $q$. Por tanto, cada $Q_i$ tiene $q-1$ elementos de orden $q$.
            \begin{itemize}
                \item Supongamos que $\exists i,j\in\{1,\ldots,p^2\}$ tales que $i\neq j$ y $Q_i\cap Q_j \neq \{1\}$. Como $Q_i\cap Q_j< Q_i$ y el orden de un subgrupo divide el orden del grupo, tenemos que $Q_i\cap Q_j=Q_i$, luego $Q_i=Q_j$, en contradicción con que $i\neq j$.
            \end{itemize}
            Por tanto, $Q_i\cap Q_j=\{1\}$ para todo $i,j\in\{1,\ldots,p^2\}$ tales que $i\neq j$. Como hay $p^2$ subgrupos de orden $q$ y cada uno de ellos tiene $q-1$ elementos de orden $q$ todos ellos distintos, sabemos que hay $p^2(q-1)$ elementos de orden $q$ en $G$.

            Por otro lado, como $n_p=q$, tenemos que hay $q$ $p$-subgrupos de Sylow de $G$. Sean estos $P_1,\ldots,P_q$. Fijado $i\in \{1,\ldots,q\}$, tenemos que $|P_i|=p^2$, luego todo elemento distinto de $1$ de $P_i$ tiene orden $p$ o $p^2$. En este caso no podemos garantizar que las intersecciones sean triviales (puesto que podrían tener orden $p$), pero fijado $i\in \{1,\ldots,q\}$, $P_i$ tiene $p^2-1$ elementos de orden $p$ o $p_2$. Además, sabemos que $\exists j\in\{1,\ldots,q\}$ tal que $P_j\cap P_i\neq P_i$ (pues si no, $P_i$ sería el único $p$-subgrupo de Sylow de $G$ y por tanto $n_p=1$). Por tanto, sabemos que, al menos, hay $p^2-1$ elementos de orden $p$ o $p^2$ en $G$ (los de $P_i$) pero no son los únicos (pues $P_j$ tiene algún elemento de orden $p$ o $p^2$ distinto de los de $P_i$).

            En conclusión, hemos demostrado que hay $p^2(q-1)$ elementos de orden $q$ y hay más de $p^2-1$ elementos de orden $p$ o $p^2$ en $G$. Por tanto, tenemos que:
            \begin{equation*}
                |G| = p^2q > p^2-1 + p^2(q-1) + 1 = p^2-1 + p^2q - p^2 + 1 = p^2q
            \end{equation*}
            Lo cual es una contradicción, luego $n_q\neq p^2$.
        \end{itemize}

        Por tanto, $n_q=1$. Sea $Q$ el único $q$-subgrupo de Sylow de $G$. Como es el único, entonces $Q\lhd G$. Por ser un $q$-subgrupo de Sylow de $G$, entonces $|Q|=q$. Por tanto, tenemos que:
        \begin{itemize}
            \item $|Q|=q$ primo, luego $Q$ es un $q$-grupo, luego resoluble.
            \item $|G/Q|=p^2$, luego $G/Q$ es un $p$-grupo, luego resoluble.
        \end{itemize}
        Por tanto, como $Q$ y $G/Q$ son ambos resolubles, entonces $G$ es resoluble.
    \end{itemize}
    
    En cualquier caso, hemos visto que $G$ es resoluble.
        \item [Opción 2.]~\\
            Distinguimos casos:
            \begin{itemize}
                \item Si $p = q$, entonces tenemos un grupo de orden $|G| = p^3$:
                    \begin{itemize}
                        \item Si $G$ es abeliano, entonces será resoluble.
                        \item Si $G$ no es abeliano, entonces sabemos por Burnside que $|Z(G)| \geq p$ y que $|Z(G)| < p^2$, por lo que $|Z(G)| = p$, luego $Z(G) \cong \mathbb{Z}_p$, luego $Z(G)$ es resoluble. Además, tenemos que $Z(G) \lhd G$, con:
                            \begin{equation*}
                                |G/Z(G)| = p^2
                            \end{equation*}
                            Y sabemos que todo grupo de orden $p^2$ es abeliano, luego resoluble. En definitiva, $G$ es resoluble, por ser $Z(G)$ y $G/Z(G)$ resolubles.
                    \end{itemize}
                \item Si $p>q$, el Segundo Teorema de Sylow nos dice que:
                    \begin{equation*}
                        \left.\begin{array}{r}
                            n_p \mid q \\
                            n_p \equiv 1 \mod p
                        \end{array}\right\} \Longrightarrow 
                        \left\{\begin{array}{r}
                            n_p < q < p \\
                            n_p \equiv 1 \mod p
                        \end{array}\right. \Longrightarrow n_p = 1
                    \end{equation*}
                    Como tenemos un único $p-$subgrupo de Sylow de $G$, llamémoslo $P<G$, será $P\lhd G$, con $|P| = p^2$, luego $P$ es abeliano y en particular, resoluble. Como:
                    \begin{equation*}
                        |G/P| = q
                    \end{equation*}
                    Tendremos que $G/P\cong \mathbb{Z}_q$, luego también será resoluble, de donde $G$ será resoluble.
                    \item Si $p<q$, el Segundo Teorema de Sylow nos dice que:
                        \begin{equation*}
                            \left.\begin{array}{r}
                                n_q \mid p^2 \\
                                n_q \equiv 1 \mod q
                            \end{array}\right\} \Longrightarrow n_q \in \{1,p,p^2\}
                        \end{equation*}
                        Distinguimos casos:
                        \begin{itemize}
                            \item Si $n_q = p$, entonces tenemos que $p\equiv 1 \mod q$, de donde $q\mid p-1$ con $p-1<p<q$, \underline{contradicción}.
                            \item Si $n_q = p^2$, de la misma forma tendremos que $q\mid p^2-1 = (p-1)(p+1)$. Como $q$ es primo, ha de ser $q\mid p-1$ o $q\mid p+1$, por definición de primo. El caso $q\mid p-1$ lo hemos discutido en el punto superior y es imposible.

                                Suponemos por tanto que $q\mid p+1$, y como $p<q$, ha de ser $q=p+1$. Como solo hay dos primos consecutivos, tendremos que $q = 3$ y $p = 2$, por lo que estamos en un grupo de orden 12 con más de un $3-$subgrupo de Sylow ($n_3 = 2^2 = 4$), luego $G\cong A_4$ (por un ejercicio de la relación de $p-$grupos), y $A_4$ es resoluble:
                                \begin{equation*}
                                    A_4 \rhd V \rhd \{1\}
                                \end{equation*}
                                Luego $G$ también lo será.
                            \item Si $n_q = 1$, tenemos entonces la existencia de un único $q-$subgrupo de Sylow $Q\lhd G$ con $|Q| = q$, luego este será resoluble. Además:
                                \begin{equation*}
                                    |G/Q| = p^2
                                \end{equation*}
                                También será resoluble, luego $G$ es resoluble.
                        \end{itemize}
            \end{itemize}
    \end{description}

\end{ejercicio}

\begin{ejercicio}\label{ej:5.16}
    Demuestra que si $p_1$, $p_2$, $p_3$ son tres primos tales que $p_3 > p_1 p_2$ entonces cualquier grupo de orden $p_1 p_2 p_3$ es resoluble.\\


    Sea $G$ un grupo de orden $p_1 p_2 p_3$. Sea $n_{p_3}$ el número de $p_3$-subgrupos de Sylow de $G$. Por el Segundo Teorema de Sylow, se tiene que:
    \begin{align*}
        n_{p_3} &\equiv 1 \mod p_3 \\
        n_{p_3} &\mid p_1 p_2
    \end{align*}
    Por la segunda condición, tenemos que $n_{p_3}\in \{1,p_1,p_2,p_1p_2\}$.
    \begin{itemize}
        \item Si $n_{p_3}=p_1$:
        
        Como $n_{p_3}\equiv 1 \mod p_3$, entonces $\exists k\in\bb{N}$ tal que:
        \begin{equation*}
            p_1 = 1 + k p_3\Longrightarrow p_1p_2 = p_2 + k p_2 p_3>p_3
        \end{equation*}
        Lo cual es una contradicción, luego $n_{p_3}\neq p_1$.
        \item Si $n_{p_3}=p_2$:
        
        Como $n_{p_3}\equiv 1 \mod p_3$, entonces $\exists k\in\bb{N}$ tal que:
        \begin{equation*}
            p_2 = 1 + k p_3\Longrightarrow p_1p_2 = p_1 + k p_1 p_3>p_3
        \end{equation*}
        Lo cual es una contradicción, luego $n_{p_3}\neq p_2$.

        \item Si $n_{p_3}=p_1p_2$:
        
        Como $n_{p_3}\equiv 1 \mod p_3$, entonces $\exists k\in\bb{N}$ tal que:
        \begin{equation*}
            p_1p_2 = 1 + k p_3 > p_3
        \end{equation*}
        Lo cual es una contradicción, luego $n_{p_3}\neq p_1p_2$.
    \end{itemize}

    Por tanto, $n_{p_3}=1$. Sea $P$ el único $p_3$-subgrupo de Sylow de $G$. Como es el único, entonces $P\lhd G$. Por ser un $p_3$-subgrupo de Sylow de $G$, entonces $|P|=p_3$. Por tanto, tenemos que:
    \begin{itemize}
        \item $|P|=p_3$ primo, luego $P$ es un $p_3$-grupo, luego resoluble.
        \item $|G/P|=p_1 p_2$, y por tanto es resoluble por el Ejercicio~\ref{ej:5.15}.
    \end{itemize}
    Por tanto, como $P$ y $G/P$ son ambos resolubles, entonces $G$ es resoluble.
\end{ejercicio}

\begin{ejercicio}~
    \begin{enumerate}
        \item Demuestra que todo grupo de orden $70$ es resoluble.
        
        Tenemos que $70=2\cdot 5\cdot 7$. Por el Segundo Teorema de Sylow, se tiene que:
        \begin{align*}
            n_7 &\equiv 1 \mod 7 \\
            n_7 &\mid 10
        \end{align*}

        Por tanto, $n_7=1$. Entonces existe $P$ único $7$-subgrupo de Sylow de $G$. Como es el único, entonces $P\lhd G$. Por ser un $7$-subgrupo de Sylow de $G$, entonces $|P|=7$. Por tanto, tenemos que:
        \begin{itemize}
            \item $|P|=7$ primo, luego $P$ es un $7$-grupo, luego resoluble.
            \item $|G/P|=10=5\cdot 2$, y por tanto es resoluble por el Ejercicio~\ref{ej:5.15}.
        \end{itemize}

        Por tanto, como $P_7$ y $G/P_7$ son ambos resolubles, entonces $G$ es resoluble.
        \item Demuestra que todo grupo de orden $24$ es resoluble.
        
        Sea $G$ un grupo de orden $24$. Sea $n_3$ el número de $3-$subgrupos de Sylow de $G$. Por el Segundo Teorema de Sylow, se tiene que:
        \begin{align*}
            n_3 &\equiv 1 \mod 3 \\
            n_3 &\mid 8
        \end{align*}
        Por tanto, $n_3\in \{1,4\}$.
        \begin{itemize}
            \item Si $n_3=1$, entonces existe $P$ único $3$-subgrupo de Sylow de $G$. Como es el único, entonces $P\lhd G$. Por ser un $3$-subgrupo de Sylow de $G$, entonces $|P|=3$. Por tanto, tenemos que:
            \begin{itemize}
                \item $|P|=3$ primo, luego $P$ es un $3$-grupo, luego resoluble.
                \item $|G/P|=8$, luego $G/P$ es un $2$-grupo, luego resoluble.
            \end{itemize}
            Por tanto, como $P$ y $G/P$ son ambos resolubles, entonces $G$ es resoluble.


            \item Si $n_3=4$, entonces hay $4$ $3$-subgrupos de Sylow de $G$:
            \begin{equation*}
                \Syl_3(G) = \{P_1, P_2, P_3, P_4\}
            \end{equation*}

            Consideramos ahora la acción por conjugación de $G$ sobre $\Syl_3(G)$. Sea $\phi$ su representación por permutaciones:
            \Func{\phi}{G}{\Perm(\Syl_3(G))}{g}{\phi_g}

            Como $|\Syl_3(G)|=4$, tenemos que $\Perm(\Syl_3(G))=S_4$. Por tanto, $\phi$ es un homomorfismo de $G$ en $S_4$.
            \begin{itemize}
                \item Si $\ker(\phi)=\{1\}$, entonces por el Primer Teorema de Isomorfía:
                \begin{equation*}
                    G \cong G/\{1\} = G/\ker(\phi) \cong Im(\phi)
                \end{equation*}

                Como $Im(\phi)< S_4$ y $S_4$ es resoluble, entonces $Im(\phi)$ es resoluble, y como esta propiedad se mantiene por isomorfismo, entonces $G$ es resoluble.

                \item Si $\ker(\phi)=G$, entonces $\phi$ es el homomorfismo trivial, luego $G$:
                \begin{align*}
                    \phi_g(P_i) = \prescript{g}{}{P_i} = gP_i g^{-1} = P_i\qquad \forall g\in G, \forall P_i\in \Syl_3(G)
                \end{align*}

                Como $P_i=gP_i g^{-1}$ para todo $g\in G$, entonces $P_i\lhd G$, luego $P_i$ es el único $3$-subgrupo de Sylow de $G$, lo que contradice que $n_3=4$. Este caso no es posible.

                \item Si $\ker(\phi)\neq \{1\},G$, entonces:
                \begin{itemize}
                    \item Como $\ker(\phi)<G$, sabemos que $|\ker(\phi)|\mid |G|=24$, luego:
                    \begin{equation*}
                        |\ker(\phi)|\in \{2,3,4,6,8,12\}
                    \end{equation*}

                    Veamos si $\ker(\phi)$ es resoluble:
                    \begin{itemize}
                        \item Si $|\ker(\phi)|\in \{2,3,4,8\}$, entonces $\ker(\phi)$ es un $p$-grupo, luego resoluble.
                        \item Si $|\ker(\phi)|=6=3\cdot 2$, es resoluble por el Ejercicio~\ref{ej:5.15}.
                        \item Si $|\ker(\phi)|=12=3\cdot 2^2$, es resoluble por el Ejercicio~\ref{ej:5.16}.
                    \end{itemize}
                    Por tanto, en cualquiera de estos casos, $\ker(\phi)$ es resoluble.

                    \item Por el Primer Teorema de Isomorfía, tenemos que:
                    \begin{equation*}
                        G/\ker(\phi) \cong Im(\phi)
                    \end{equation*}
                    Como $Im(\phi)<S_4$ y $S_4$ es resoluble, entonces $Im(\phi)$ es resoluble, y como esta propiedad se mantiene por isomorfismo, entonces $G/\ker(\phi)$ es resoluble.
                \end{itemize}
                Por tanto, como $\ker(\phi)$ y $G/\ker(\phi)$ son ambos resolubles, entonces $G$ es resoluble.
            \end{itemize}

            En cualquier caso, $G$ es resoluble.
        \end{itemize}

        En conclusión, hemos visto que $G$ es resoluble.
        \item Demuestra que todo grupo de orden $100$ es resoluble.
        
        Tenemos que $100=2^2\cdot 5^2$. Por el Segundo Teorema de Sylow, se tiene que:
        \begin{align*}
            n_5 &\equiv 1 \mod 5 \\
            n_5 &\mid 4
        \end{align*}
        Por tanto, $n_5=1$. Entonces existe $P$ único $5$-subgrupo de Sylow de $G$. Como es el único, entonces $P\lhd G$. Por ser un $5$-subgrupo de Sylow de $G$, entonces $|P|=25$. Por tanto, tenemos que:
        \begin{itemize}
            \item $|P|=25=5^2$, luego $P$ es un $5$-grupo, luego resoluble.
            \item $|G/P|=4$, luego $G/P$ es un $2$-grupo, luego resoluble.
        \end{itemize}
        Por tanto, como $P$ y $G/P$ son ambos resolubles, entonces $G$ es resoluble.
        \item Demuestra que todo grupo de orden $48$ es resoluble.
        
        Tenemos que $48=2^4\cdot 3$. Por el Segundo Teorema de Sylow, se tiene que:
        \begin{align*}
            n_3 &\equiv 1 \mod 3 \\
            n_3 &\mid 16
        \end{align*}
        Por tanto, $n_3\in \{1,4,16\}$. Por otro lado:
        \begin{align*}
            n_2 &\equiv 1 \mod 2 \\
            n_2 &\mid 3
        \end{align*}
        Por tanto, $n_2\in \{1,3\}$.
        Distinguimos en función de los valores de $n_2$:
        \begin{itemize}
            \item Si $n_2=1$, entonces existe $Q$ único $2$-subgrupo de Sylow de $G$. Como es el único, entonces $Q\lhd G$. Por ser un $2$-subgrupo de Sylow de $G$, entonces $|Q|=16$. Por tanto, tenemos que:
            \begin{itemize}
                \item $|Q|=16=2^4$, luego $Q$ es un $2$-grupo, luego resoluble.
                \item $|G/Q|=3$, luego $G/Q$ es un $3$-grupo, luego resoluble.
            \end{itemize}
            Por tanto, como $Q$ y $G/Q$ son ambos resolubles, entonces $G$ es resoluble.

            \item Si $n_2=3$, entonces hay tres subgrupos de Sylow de orden $2$. Sea $\Syl_2(G)=\{Q_1,Q_2,Q_3\}$. Consideramos ahora la acción por conjugación de $G$ sobre $\Syl_2(G)$. Sea $\phi$ su representación por permutaciones:
            \Func{\phi}{G}{\Perm(\Syl_2(G))}{g}{\phi_g}

            Como $|\Syl_2(G)|=3$, tenemos que $\Perm(\Syl_2(G))=S_3$. Por tanto, $\phi$ es un homomorfismo de $G$ en $S_3$.
            \begin{itemize}
                \item Si $\ker(\phi)=\{1\}$, entonces por el Primer Teorema de Isomorfía:
                \begin{equation*}
                    G \cong G/\{1\} = G/\ker(\phi) \cong Im(\phi)
                \end{equation*}

                Como $Im(\phi)< S_3$ y $S_3$ es resoluble, entonces $Im(\phi)$ es resoluble, y como esta propiedad se mantiene por isomorfismo, entonces $G$ es resoluble.

                \item Si $\ker(\phi)=G$, entonces $\phi$ es el homomorfismo trivial, luego:
                \begin{align*}
                    \phi_g(Q_i) = \prescript{g}{}{Q_i} = gQ_i g^{-1} = Q_i\qquad \forall g\in G, \forall Q_i\in \Syl_2(G)
                \end{align*}
                Como $Q_i=gQ_i g^{-1}$ para todo $g\in G$, entonces $Q_i\lhd G$, luego $Q_i$ es el único $2$-subgrupo de Sylow de $G$, lo que contradice que $n_2=3$. Este caso no es posible.

                \item Si $\ker(\phi)\neq \{1\},G$, entonces:
                \begin{itemize}
                    \item Como $\ker(\phi)<G$, sabemos que $|\ker(\phi)|\mid |G|=48$, luego:
                    \begin{equation*}
                        |\ker(\phi)|\in \{2,3,4,6,8,12,16,24\}
                    \end{equation*}

                    Veamos si $\ker(\phi)$ es resoluble:
                    \begin{itemize}
                        \item Si $|\ker(\phi)|\in \{2,3,4,8,16\}$, entonces $\ker(\phi)$ es un $p$-grupo, luego resoluble.
                        \item Si $|\ker(\phi)|=6=3\cdot 2$, es resoluble por el Ejercicio~\ref{ej:5.15}.
                        \item Si $|\ker(\phi)|=12=3\cdot 2^2$, es resoluble por el Ejercicio~\ref{ej:5.16}.
                        \item Si $|\ker(\phi)|=24$, es resoluble por el segundo apartado de este ejercicio.
                    \end{itemize}
                    Por tanto, en cualquiera de estos casos, $\ker(\phi)$ es resoluble.

                    \item Por el Primer Teorema de Isomorfía, tenemos que:
                    \begin{equation*}
                        G/\ker(\phi) \cong Im(\phi)
                    \end{equation*}
                    Como $Im(\phi)<S_3$ y $S_3$ es resoluble, entonces $Im(\phi)$ es resoluble, y como esta propiedad se mantiene por isomorfismo, entonces $G/\ker(\phi)$ es resoluble.
                \end{itemize}
                Por tanto, como $\ker(\phi)$ y $G/\ker(\phi)$ son ambos resolubles, entonces $G$ es resoluble.
            \end{itemize}
            En cualquier caso, $G$ es resoluble.
        \end{itemize}
        En conclusión, hemos visto que $G$ es resoluble.
        \item Sea $G$ un grupo de orden $200$. Demuestra que $G \times D_{41}$ es resoluble.
        
        Veamos que todo grupo de orden $200$ es resoluble. Tenemos que $200=2^3\cdot 5^2$. Por el Segundo Teorema de Sylow, se tiene que:
        \begin{align*}
            n_5 &\equiv 1 \mod 5 \\
            n_5 &\mid 8
        \end{align*}
        Por tanto, $n_5=1$. Entonces existe $P$ único $5$-subgrupo de Sylow de $G$. Como es el único, entonces $P\lhd G$. Por ser un $5$-subgrupo de Sylow de $G$, entonces $|P|=25$. Por tanto, tenemos que:
        \begin{itemize}
            \item $|P|=25=5^2$, luego $P$ es un $5$-grupo, luego resoluble.
            \item $|G/P|=8$, luego $G/P$ es un $2$-grupo, luego resoluble.
        \end{itemize}
        Por tanto, como $P$ y $G/P$ son ambos resolubles, entonces $G$ es resoluble.\\

        Como $|D_{41}|=82=2\cdot 41$ y $41$ es primo, por el Ejercicio~\ref{ej:5.15} sabemos que $D_{41}$ es resoluble.\\

        Por tanto, como $G$ y $D_{41}$ son ambos resolubles, entonces $G\times D_{41}$ es resoluble.
        \item Demuestra que todo grupo de orden $63$ es soluble (sin usar que es un caso particular de un grupo de orden $p^2q$ con $p$ y $q$ primos).
        
        Sea $G$ un grupo de orden $63=3^2\cdot 7$. Sea $n_7$ el número de $7-$subgrupos de Sylow de $G$. Por el Segundo Teorema de Sylow, se tiene que:
        \begin{align*}
            n_7 &\equiv 1 \mod 7 \\
            n_7 &\mid 9
        \end{align*}
        Por tanto, $n_7=1$. Entonces existe $P$ único $7$-subgrupo de Sylow de $G$. Como es el único, entonces $P\lhd G$. Por ser un $7$-subgrupo de Sylow de $G$, entonces $|P|=7$. Por tanto, tenemos que:
        \begin{itemize}
            \item $|P|=7$ primo, luego $P$ es un $7$-grupo, luego resoluble.
            \item $|G/P|=9=3^2$, luego $G/P$ es un $3$-grupo, luego resoluble.
        \end{itemize}
        Por tanto, como $P$ y $G/P$ son ambos resolubles, entonces $G$ es resoluble.
    \end{enumerate}
\end{ejercicio}
