

\begin{ejercicio}
    Define el concepto de subgrupo normal de un grupo. Demuestra el Tercer Teorema de Isomorfía para grupos (esto es, describe de qué forma son los subgrupos de un grupo cociente y cómo son los cocientes de un grupo cociente).
    \begin{definicion}
        Sea \(G\) un grupo. Un subgrupo \(N<G\) se dice \emph{normal} en $G$, notado por $N\lhd G$, si:
        \begin{equation*}
            xN = Nx \quad \forall x \in G.
        \end{equation*}
        En tal caso, tendremos que los conjuntos conciente $G/\prescript{}{N}{\sim}$ y $G/{\sim}_{N}$ son iguales, por lo que podemos considerar simplemente el cociente \(G/N\), cuyos elementos llamaremos \emph{clases laterales} de \(N\) en \(G\).
    \end{definicion}

    Cabe resaltar la propiedad más relevante de los subgrupos normales: dotan al conjunto cociente \(G/N\) de una estructura de grupo, como se puede ver en el siguiente teorema.
    \begin{teo}
        Sea \(G\) un grupo y \(N\) un subgrupo normal de \(G\). Entonces, el conjunto cociente \(G/N\) es un grupo con la operación:
        \Func{\cdot}{G/N\times G/N}{G/N}{(xN, yN)}{(xy)N}
        Este grupo se llama \emph{grupo cociente} de \(G\) por \(N\). Además, la proyección canónica \(p: G \to G/N\) es un homomorfismo de grupos.
    \end{teo}

    Vistos estos conceptos, enunciamos y demostramos el Tercer Teorema de Isomorfía para grupos, también llamado Teorema del Doble Cociente.
    \begin{teo}[Tercer Teorema de Isomorfía]
        Sea \(G\) un grupo y \(N\lhd G\) un subgrupo normal de \(G\). Entonces, hay una biyección entre los subgrupos de $G$ que contienen a \(N\) y los subgrupos de \(G/N\), dada por $H\mapsto H/N$.\\

        Además, $H\lhd G\iff H/N\lhd G/N$, en cuyo caso:
        \begin{equation*}
            \frac{G/N}{H/N} \cong G/H.
        \end{equation*}
        \begin{proof}
            Dada la proyección canónica \(p: G \to G/N\), consideramos las aplicaciones imagen directa e imagen inversa por $p$, dadas por:
            \Func{p_\ast}{\cc{P}(G)}{\cc{P}(G/N)}{H}{\{p(h) \mid h \in H\}}
            \Func{p^\ast}{\cc{P}(G/N)}{\cc{P}(G)}{J}{\{x\in G \mid p(x) \in J\}}

            Definimos ahora los siguientes conjuntos:
            \begin{align*}
                \cc{A} &= \{H<G \mid N\subseteq H\} \\
                \cc{B} &= \{J<G/N\}
            \end{align*}

            Consideramos ahora las restricciones de \(p_\ast\) y \(p^\ast\) a los conjuntos \(\cc{A}\) y \(\cc{B}\) respectivamente, aunque las notaremos de la misma forma.
            \begin{align*}
                p_\ast: \cc{A} &\to \cc{B} \\
                p^\ast: \cc{B} &\to \cc{A}
            \end{align*}

            Veamos en primer lugar que están bien definidas:
            \begin{itemize}
                \item Sea $H\in \cc{A}$, por lo que $H<G$ y $N\subseteq H$. Veamos que \(p_\ast(H)<G/N\).
                
                Como $H<G$ y $p$ es un homomorfismo, tenemos que \(p_\ast(H)<p_\ast(G) = G/N\), luego $p_\ast(H)\in \cc{B}$ y está bien definida.\\

                Veamos qué forma tiene \(p_\ast(H)\). Como $N\lhd G$ y $N\subset H$, tenemos que \(N\lhd H\), luego:
                \begin{align*}
                    p_\ast(H) &= \{p(h) \mid h \in H\} = \{hN \mid h \in H\} = H/N.
                \end{align*}

                \item Sea \(J\in \cc{B}\), por lo que \(J<G/N\). Veamos que \(p^\ast(J)<G\) y que \(N\subseteq p^\ast(J)\).
                
                Como \(J<G/N\) y \(p\) es un homomorfismo, tenemos que \(p^\ast(J)<p^\ast(G/N) = G\). Veamos ahora que \(N\subseteq p^\ast(J)\). Sea ahora $n\in N$, luego:
                \begin{equation*}
                    p(n) = nN = N \in J \implies n \in p^\ast(J).
                \end{equation*}
                donde hemos usado que \(N\in J\) puesto que $N$ es el elemento neutro de \(G/N\). Por tanto, \(p^\ast(J)\in \cc{A}\) y está bien definida.
            \end{itemize}

            Veamos ahora que \(p_\ast\) y \(p^\ast\) son inversas entre sí, demostrando que componen la identidad en sus respectivos conjuntos:
            \begin{itemize}
                \item Sea \(H\in \cc{A}\), y tenemos que:
                \begin{align*}
                    p^\ast(p_\ast(H)) &= p^\ast(H/N) = \{x\in G \mid p(x) \in H/N\} = \{x\in G \mid xN \in H/N\}
                    =\\&= \{x\in G \mid \exists h\in H: xN = hN\}
                    = \{x\in G \mid x \in H\} = H
                \end{align*}

                \item Sea \(J\in \cc{B}\), y veamos que:
                \begin{align*}
                    p_\ast(p^\ast(J)) &= p_\ast(\{x\in G \mid p(x) \in J\}) \AstIg J
                \end{align*}
                donde en $(\ast)$ hemos empleado que \(p\) es sobreyectiva.
            \end{itemize}

            Por tanto, \(p_\ast\) y \(p^\ast\) son inversas entre sí, luego son biyecciones. La biyección mencionada en el enunciado es:
            \Func{p_\ast}{\cc{A}}{\cc{B}}{H}{H/N}

            Veamos ahora la segunda parte del enunciado.
            \begin{description}
                \item[$\Longrightarrow$)] Sea \(H\lhd G\), luego \(H/N < G/N\). Sean ahora \(x\in G\) y \(h\in H\), tenemos que:
                \begin{equation*}
                    xN\ hN\ (xN)^{-1} = (xhx^{-1})N \AstIg h'N\in H/N
                \end{equation*}
                donde en \((\ast)\) hemos empleado que \(H\lhd G\) luego \(xhx^{-1}=h'\in H\). Por tanto, \(H/N\lhd G/N\).

                \item[$\Longleftarrow$)] Sea \(H/N\lhd G/N\), luego \(H < G\). Sea \(x\in G\) y \(h\in H\), tenemos que:
                \begin{equation*}
                    xN\ hN\ (xN)^{-1} = (xhx^{-1})N \in H/N
                    \Longrightarrow
                    xhx^{-1} \in H.
                \end{equation*}
                Por tanto, \(H\lhd G\).
            \end{description}

            Tan solo nos queda demostrar la isomorfía mencionada en el enunciado. Consideramos las proyecciones canónicas \(p_N: G \to G/N\) y \(p_H: H \to H/N\). Como $N\subset H=\ker(p_H)$, por la Propiedad Universal del Conjunto Cociente, tenemos existe un único homomorfismo $\varphi: G/N \to G/H$ tal que \(p_H = \varphi \circ p_N\), como se muestra en la Figura~\ref{fig:teo_propiedad_universal}.
            \begin{figure}[h]
                \centering
                \shorthandoff{""}
                \begin{tikzcd}
                G \arrow[r, "p_N"] \arrow[rd, "p_H"'] & G/N \arrow[d, "\varphi", dotted] \\
                                                & G/H                              
                \end{tikzcd}
                \shorthandon{""}
                \caption{Situación de la Propiedad Universal del Conjunto Cociente.}
                \label{fig:teo_propiedad_universal}
            \end{figure}
            
            
            Aplicando el Primer Teorema de Isomorfía a $\varphi$, tenemos que:
            \begin{equation*}
                \frac{G/N}{\ker(\varphi)} \cong Im(\varphi)
            \end{equation*}
            Como $p_H$ es sobreyectiva, $\varphi$ también lo es, luego $Im(\varphi)=G/H$. Veamos ahora que $\ker(\varphi) = H/N$:
            \begin{align*}
                \ker(\varphi) &= \{xN\in G/N \mid \varphi(xN) = H\} \\
                &= \{xN\in G/N \mid p_H(x) = H\} \\
                &= \{xN\in G/N \mid x\in H\} = H/N.
            \end{align*}
            Por tanto, tenemos que:
            \begin{equation*}
                \frac{G/N}{H/N} \cong G/H.
            \end{equation*}
        \end{proof}
    \end{teo}

    Por tanto, dado un grupo \(G\) y un subgrupo normal \(N\lhd G\), todos los subgrupos de $G/H$ son de la forma \(H/N\) para algún subgrupo \(H<G\) con $N\subset H$. Además, dado un subgrupo normal $H/N \lhd G/N$, hemos visto que \(H\lhd G\) y que:
    \begin{equation*}
        \frac{G/N}{H/N} \cong G/H.
    \end{equation*}
\end{ejercicio}

\newpage
\begin{ejercicio}
    Define los conceptos de serie de composición y de serie derivada de un grupo y da dos definiciones de grupo resoluble demostrando su equivalencia. Razona que \(S_4\) es resoluble pero que \(S_5\) no lo es.\\

    Antes de definir serie de composición, hemos de realizar varias definiciones:
    \begin{definicion}[Serie]
        Sea \(G\) un grupo. Una serie de $G$ es una cadena de subgrupos $G_0,\dots,G_r$ tales que:
        \begin{equation*}
            G = G_0 > G_1 > \ldots > G_r = \{1\}.
        \end{equation*}
        En tal caso, diremos que es una serie de \(G\) de longitud \(r\).
    \end{definicion}
    \begin{definicion}[Serie Normal]
        Sea \(G\) un grupo. Una serie de \(G\) es normal si todos sus subgrupos son normales, es decir, si:
        \begin{equation*}
            G_i \rhd G_{i+1} \quad \forall i = 0, \ldots, r-1.
        \end{equation*}

        En tal caso, notaremos la serie como:
        \begin{equation*}
            G = G_0 \rhd G_1 \rhd \ldots \rhd G_r = \{1\}.
        \end{equation*}

        Además, a los cocientes \(G_i/G_{i+1}\) de la serie los llamaremos \emph{factores de la serie}.
    \end{definicion}

    \begin{definicion}[Refinamiento de una serie]
        Sea \(G\) un grupo, y sea la siguiente serie de \(G\):
        \begin{equation}\label{eq:serie1}
            G = G_0 > G_1 > \ldots > G_r = \{1\}.
        \end{equation}
        
        Un refinamiento de la serie \eqref{eq:serie1} es una serie de \(G\) de la forma:
        \begin{equation}\label{eq:serie2}
            G = H_0 > H_1 > \ldots > H_s = \{1\}
        \end{equation}
        de forma que, para todo $i\in \{0,\ldots,r-1\}$, existe un \(j\in \{0,\ldots,s-1\}\) tal que $G_i = H_j$. Es decir, si todo subgrupo de la serie \eqref{eq:serie1} aperece en la serie \eqref{eq:serie2} (por lo que \(s\geq r\)).
        \begin{itemize}
            \item Un refinamiento se dice \emph{normal} si la serie \eqref{eq:serie2} es normal.
            \item Un refinamiento se dice \emph{propio} si, al menos, se ha añadido un subgrupo; es decir, $\exists j\in \{0,\ldots,s-1\}$ tal que $H_j \neq G_i$ para todo $i\in \{0,\ldots,r-1\}$.
        \end{itemize}
    \end{definicion}

    \begin{definicion}[Serie de composición]
        Sea \(G\) un grupo. Una serie de composición de \(G\) es una serie normal sin refinamientos propios.
    \end{definicion}

    \begin{comment}
    Además, presenta una importante caracterización, para lo que hemos de definir el concepto de subgrupo simple.
    \begin{definicion}[Subgrupo simple]
        Un grupo \(G\) se dice \emph{simple} si no tiene subgrupos normales propios.
    \end{definicion}
    \begin{prop}[Caracterización de serie de composición]
        Sea \(G\) un grupo. Entonces, una serie normal de \(G\) es una serie de composición si y solo si todos sus factores son grupos simples.
    \end{prop}
    \end{comment}

    Veamos ahora la definición de serie derivada. Previamente, hemos de definir algunos conceptos previos.
    \begin{definicion}[Conmutador]
        Sea \(G\) un grupo. El conmutador de dos elementos \(x, y \in G\) es el elemento:
        \begin{equation*}
            [x, y] = xyx^{-1}y^{-1}\in G
        \end{equation*}
    \end{definicion}
    \begin{definicion}[Subgrupo conmutador]
        Sea \(G\) un grupo. El subgrupo conmutador de \(G\), denotado por \([G, G]\), es el subgrupo generado por todos los conmutadores de \(G\):
        \begin{equation*}
            [G, G] = \langle [x, y] \mid x, y \in G\rangle.
        \end{equation*}
    \end{definicion}


    % // TODO: Quitar?
    Algunas propiedades importantes del subgrupo conmutador que usaremos en la demostración son las siguientes:
    \begin{prop}
        Sea \(G\) un grupo. Entonces:
        \begin{enumerate}
            \item \(G/[G, G]\) es abeliano.
            \item $[G, G]\lhd G$.
            \item Si \(N\lhd G\), entonces:
            \begin{equation*}
                G/N\ \text{es abeliano} \iff [G, G] < N.
            \end{equation*}
            \item Si $H<G$ es un subgrupo de \(G\), entonces:
            \begin{equation*}
                [H, H] < [G, G].
            \end{equation*}
            \item $[G, G]=\{1\}$ si y solo si \(G\) es abeliano.
        \end{enumerate}
    \end{prop}

    \begin{definicion}[Serie derivada]
        Sea \(G\) un grupo. La serie derivada de \(G\) es la serie normal siguiente:
        \begin{equation*}
            G = G^0 \rhd G' \rhd G'' \rhd \ldots \rhd G^{(k)} \rhd \ldots
        \end{equation*}
        donde \(G^{(0)} = G\) y \(G^{(k+1)} = [G^{(k)}, G^{(k)}]\) para todo \(k\geq 0\).
    \end{definicion}

    Notemos que no tiene por qué existir un $k\in \mathbb{N}$ tal que \(G^{(k)} = \{1\}\), por lo que la serie derivada puede ser infinita. En el caso de que exista, será un grupo resoluble.
    \begin{definicion}[Grupo resoluble]
        Sea \(G\) un grupo. Diremos que \(G\) es \emph{resoluble} si su serie derivada es finita; es decir, si existe un \(k\in \mathbb{N}\) tal que \(G^{(k)} = \{1\}\).
    \end{definicion}

    Veamos ahora la caracterización buscada.
    \begin{prop}[Caracterización de grupo resoluble]
        Un grupo \(G\) es resoluble si y solo si tiene una serie normal con factores abelianos.
        \begin{proof}~
            \begin{description}
                \item[$\Longrightarrow$)] Sea \(G\) un grupo resoluble, luego su serie derivada es finita. Por tanto, existe un \(k\in \mathbb{N}\) tal que \(G^{(k)} = \{1\}\). Sea esta:
                \begin{equation*}
                    G = G^0 \rhd G' \rhd G'' \rhd \ldots \rhd G^{(k)} = \{1\}.
                \end{equation*}
                Esta serie es normal, y sus factores son:
                \begin{equation*}
                    G^{(i)} / G^{(i+1)} = G^{(i)} / [G^{(i)}, G^{(i)}]\qquad \forall i = 0, \ldots, k-1.
                \end{equation*}
                Estos factores ya hemos mencionado anteriormente que son abelianos.


                \item[$\Longleftarrow$)] Sea \(G\) un grupo con una serie normal con factores abelianos. Entonces, existe una serie normal de la forma:
                \begin{equation*}
                    G = G_0 \rhd G_1 \rhd G_2 \rhd \ldots \rhd G_r = \{1\},
                \end{equation*}
                donde todos los factores son abelianos. Demostremos que $G^{(k)}< G_k$ para todo \(k\in \{0, \ldots, r\}\):
                \begin{itemize}
                    \item Para \(k = 0\), queremos ver que $G^{(0)} = G < G_0$, lo cual es cierto.
                    \item Supuesto cierto para \(k\), veamos que $G^{(k+1)} < G_{k+1}$:
                    
                    Como $G_{k}\rhd G_{k+1}$ y $G_k/G_{k+1}$ es abeliano, tenemos que:
                    \begin{equation*}
                        [G_k, G_k] < G_{k+1}
                    \end{equation*}

                    Por tanto:
                    \begin{align*}
                        G^{(k+1)} &= [G^{(k)}, G^{(k)}] < [G_k, G_k] < G_{k+1}.
                    \end{align*}
                    donde en la primera desigualdad hemos empleado que \(G^{(k)} < G_k\) por la hipótesis de inducción.
                \end{itemize}

                Por tanto, para todo \(k\in \{0, \ldots, r\}\) tenemos que \(G^{(k)} < G_k\). Empleando el reultado en $r$, como \(G_r = \{1\}\), tenemos que:
                \begin{equation*}
                    G^{(r)} < G_r = \{1\}\Longrightarrow G^{(r)} = \{1\}.
                \end{equation*}

                Por tanto, la serie derivada de \(G\) es finita, luego \(G\) es resoluble.
            \end{description}
        \end{proof}
    \end{prop}

    Veamos ahora el ejemplo de los grupos \(S_n\) que se menciona en el enunciado.
    Queremos ver que \(S_4\) es resoluble pero que \(S_5\) no lo es.
    \begin{itemize}
        \item Veamos que \(S_4\) es resoluble.
        
        Para ello, buscamos una serie normal suya con factores abelianos. Consideramos la serie:
        \begin{equation*}
            S_4 \rhd A_4 \rhd V \rhd \{1\},
        \end{equation*}
        Veamos en primer lugar que es una serie normal:
        \begin{itemize}
            \item \(A_4\lhd S_4\) pues $[S_4: A_4] = 2$.
            \item Veamos que \(V\lhd A_4\). Dado $\sigma\in A_4$, y \(\gamma\in V\), si $\gamma=1$ entonces:
            $$\sigma\gamma\sigma^{-1} = 1\in V$$

            Si $\gamma\neq 1$, entonces tomamos $x_1, x_2,x_3,x_4\in \{1,2,3,4\}$ distintos tales que $\gamma = (x_1\ x_2)(x_3\ x_4)$, y tenemos que:
            \begin{align*}
                \sigma\gamma\sigma^{-1} &= \sigma(x_1\ x_2)(x_3\ x_4)\sigma^{-1}
                = (\sigma(x_1)\ \sigma(x_2))(\sigma(x_3)\ \sigma(x_4))
            \end{align*}
            Como $\sigma\in A_4$, en particular es inyectiva, luego $\sigma(x_1), \sigma(x_2), \sigma(x_3), \sigma(x_4)$ son distintos, por lo que la permutación obtenida es un producto de 2 transposiciones disjuntas, luego:
            \begin{equation*}
                \sigma\gamma\sigma^{-1} \in V.
            \end{equation*}

            En cualquier caso, tenemos que $\sigma\gamma\sigma^{-1} \in V$, luego \(V\lhd A_4\).

            \item Por último, \(V\lhd \{1\}\) es trivial.
        \end{itemize}

        Veamos ahora que los factores son abelianos:
        \begin{itemize}
            \item \(|S_4/A_4| = 2\) primo, luego \(S_4/A_4\cong \bb{Z}_2\), que es abeliano.
            \item \(|A_4/V| = 3\) primo, luego \(A_4/V\cong \bb{Z}_3\), que es abeliano.
            \item $V$ es el subgrupo de Klein, que es abeliano.
        \end{itemize}

        Por tanto, \(S_4\) es resoluble.

        \item Veamos ahora que \(S_5\) no es resoluble.
        
        En este caso, será más fácil aplicar la definición. Veamos que $[S_n, S_n] = A_n$ para todo \(n\geq 3\).
        \begin{description}
            \item[$\subset$)] Como $A_n\lhd S_n$ y $S_n/A_n\cong \bb{Z}_2$ clíclico (abeliano), tenemos que:
            \begin{equation*}
                [S_n, S_n] < A_n.
            \end{equation*}

            \item[$\supset$)] Empleamos que:
            \begin{equation*}
                A_n = \langle (x\ y\ z) : x, y,z \in \{1, \ldots, n\}, x\neq y, x\neq z, y\neq z \rangle.
            \end{equation*}

            Fijados \(x, y, z\in \{1, \ldots, n\}\) distintos, tenemos que:
            \begin{align*}
                [(x\ y), (x\ z)] = (x\ y)(x\ z)(x\ y)^{-1}(x\ z)^{-1}
                = (x\ y)(x\ z)(x\ y)(x\ z)
                = (x\ y\ z) \in [S_n, S_n].
            \end{align*}
            Por tanto, \(A_n\subset [S_n, S_n]\).
        \end{description}

        Por tanto, \(A_n = [S_n, S_n]\). Entonces:
        \begin{equation*}
            S_5' = [S_5, S_5] = A_5.
        \end{equation*}

        Como $A_5$ es simple por el Teorema de Abel, tenemos que los únicos subgrupos normales suyos son \(A_5\) y \(\{1\}\). Como \(S_5\) no es abeliano, tenemos \([A_5, A_5]\neq \{1\}\), luego:
        \begin{equation*}
            A_5' = [A_5, A_5] = A_5.
        \end{equation*}

        Por tanto, la serie derivada de \(S_5\) es infinita:
        \begin{equation*}
            S_5 \rhd A_5 \rhd A_5 \rhd A_5 \rhd \ldots
        \end{equation*}
        Por tanto, \(S_5\) no es resoluble.
    \end{itemize}
\end{ejercicio}

\newpage
\begin{ejercicio}
    Define, para un grupo \(G\), los conceptos de \(G\)-conjunto \(X\) y de órbita y estabilizador de un elemento \(x \in X\). Demuestra los resultados requeridos que conduzcan, en las condiciones oportunas, a la llamada fórmula de clases \[|G| = |Z(G)| + \ldots\]

    \begin{definicion}[Acción]
        Sea \(G\) un grupo y \(X\) un conjunto. Una acción de \(G\) sobre \(X\) es una aplicación:
        \Func{ac}{G\times X}{X}{(g, x)}{ac(g, x)}
        tal que:
        \begin{enumerate}
            \item \(ac(1, x) = x\) para todo \(x\in X\).
            \item \(ac(g_1g_2, x) = ac(g_1, ac(g_2, x))\) para todo \(g_1, g_2\in G\) y \(x\in X\).
        \end{enumerate}
        En tal caso, diremos que \(G\) actúa (por la izquierda) sobre \(X\) y notaremos la acción como:
        \begin{equation*}
            ac(g,x)=\prescript{g}{ }{x} \quad \forall g\in G, x\in X.
        \end{equation*}

        En este caso, diremos que \(X\) es un \(G-\)conjunto.
    \end{definicion}

    \begin{prop}[Órbita]
        Sea $G$ un grupo y \(X\) un \(G-\)conjunto. Definimos la siguiente relación en \(X\):
        \begin{equation*}
            x \sim y \iff \exists g\in G: y = \prescript{g}{}{x}
        \end{equation*}
        Entonces, \(\sim\) es una relación de equivalencia en \(X\). A la clase de equivalencia de \(x\) la llamaremos \emph{órbita} de \(x\) y la denotaremos por $\Orb(x)$:
        \begin{equation*}
            \Orb(x) = \{y\in X \mid \exists g\in G: y = \prescript{g}{}{x}\}.
        \end{equation*}
    \end{prop}

    \begin{prop}[Estabilizador]
        Sea $G$ un grupo y \(X\) un \(G-\)conjunto. Definimos el estabilizador de un elemento \(x\in X\) como el conjunto siguiente:
        \begin{equation*}
            \Stab_G(x) = \{g\in G \mid \prescript{g}{}{x} = x\}.
        \end{equation*}
        Entonces, $\Stab_G(x)<G$ es un subgrupo de \(G\) que llamaremos subgrupo estabilizador de \(x\) en \(G\).
        \begin{proof}
            Fijado $x\in X$, es claro que $\Stab_G(x)\subset G$. Sean $g_1, g_2\in \Stab_G(x)$, luego:
            \begin{align*}
                \prescript{g_1g_2^{-1}}{}{x} =
                \prescript{g_1}{}{\left(\prescript{g_2^{-1}}{}{x}\right)}
                \AstIg
                \prescript{g_1}{}{\left(\prescript{g_2^{-1}}{}{\left(\prescript{g_2}{}{x}\right)}\right)}
                = \prescript{g_1}{}{\prescript{g_2^{-1}g_2}{}{x}}
                = \prescript{g_1}{}{\left(\prescript{1}{}{x}\right)}
                = \prescript{g_1}{}{x} \AstIg x.
            \end{align*}
            donde en \((\ast)\) hemos empleado que $g_1, g_2\in \Stab_G(x)$.
            Por tanto, \(g_1g_2^{-1}\in \Stab_G(x)\), luego \(\Stab_G(x)\) es un subgrupo de \(G\).
        \end{proof}
    \end{prop}

    \begin{definicion}[Punto Fijo]
        Sea $G$ un grupo y \(X\) un \(G-\)conjunto. Un elemento \(x\in X\) se dice \emph{punto fijo} de la acción de \(G\) sobre \(X\) si:
        \begin{equation*}
            \prescript{g}{}{x} = x \quad \forall g\in G.
        \end{equation*}
        Al conjunto de todos los puntos fijos de la acción de \(G\) sobre \(X\) lo llamaremos \emph{conjunto de puntos fijos} y lo denotaremos por $\Fix(X)$:
        \begin{equation*}
            \Fix(X) = \{x\in X \mid \prescript{g}{}{x} = x\quad \forall g\in G\}.
        \end{equation*}
    \end{definicion}

    Con vistas a demostrar la fórmula de clases, consideramos $X$ finito. Entonces, puesto que $\sim$ es una relación de equivalencia, las órbitas de la acción de \(G\) sobre \(X\) forman una partición de \(X\). Sea $\Gamma$ el conjunto formado por un un único representante de cada órbita de la acción de \(G\) sobre \(X\). Entonces, tenemos que:
    \begin{equation*}
        |X| = \sum_{x\in \Gamma} |\Orb(x)|.
    \end{equation*}

    Buscamos ahora simplificar esta expresión. Para ello, tenemos en cuenta la siguiente proposición, de demostración directa a partir de las definiciones anteriores.
    \begin{prop}
        Sea \(G\) un grupo finito y \(X\) un \(G-\)conjunto. Entonces, para todo \(x\in X\), equivalen:
        \begin{equation*}
            x\in \Fix(X) \iff \Orb(x) = \{x\} \iff \Stab_G(x) = G.
        \end{equation*}
    \end{prop}

    Con esta proposición, podemos simplificar la expresión anterior considerando los puntos fijos. De esta forma, sea $\Gamma'$ el conjunto formado por un único representante de cada órbita de la acción de \(G\) sobre \(X\) que no sea unitaria (que no sea un punto fijo), de forma que $\Gamma'\subset X\setminus \Fix(X)$. Entonces, tenemos que:
    \begin{equation*}
        |X| = |\Fix(X)| + \sum_{x\in \Gamma'} |\Orb(x)|.
    \end{equation*}

    Por último, podemos evitarnos también el cálculo de las órbitas haciendo uso de la siguiente proposición.
    \begin{prop}
        Sea \(G\) un grupo finito y \(X\) un \(G-\)conjunto. Entonces:
        \begin{equation*}
            |\Orb(x)| = [G: \Stab_G(x)] \quad \forall x\in X.
        \end{equation*}
        \begin{proof}
            Sea \(x\in X\), y consideramos el conjunto $G/\prescript{}{\Stab_G(x)}{\sim}$ (no podemos considerar el grupo cociente puesto que no tenemos que \(\Stab_G(x)\lhd G\)). Definimos la aplicación siguiente:
            \Func{\varphi}{G/\prescript{}{\Stab_G(x)}{\sim}}{\Orb(x)}{g\Stab_G(x)}{\prescript{g}{}{x}}
            \begin{itemize}
                \item Veamos que está bien definida. Sea \(g_1\Stab_G(x) = g_2\Stab_G(x)\), por lo que $\exists h\in \Stab_G(x)$ tal que \(g_1 = g_2 h\). Entonces:
                \begin{align*}
                    \prescript{g_1}{}{x} &= \prescript{g_2 h}{}{x} = \prescript{g_2}{}{\left(\prescript{h}{}{x}\right)} = \prescript{g_2}{}{x},
                \end{align*}
                donde en la última igualdad hemos empleado que \(h\in \Stab_G(x)\), luego \(\prescript{h}{}{x} = x\).
                \item Veamos que es sobreyectiva. Sea \(y\in \Orb(x)\), luego existe \(g\in G\) tal que \(y = \prescript{g}{}{x}\). Por tanto:
                \begin{align*}
                    \varphi(g\Stab_G(x)) &= \prescript{g}{}{x} = y.
                \end{align*}
                \item Veamos que es inyectiva. Sea \(g_1\Stab_G(x), g_2\Stab_G(x)\in G/\prescript{}{\Stab_G(x)}{\sim}\) tales que:
                \begin{align*}
                    \varphi(g_1\Stab_G(x)) &= \varphi(g_2\Stab_G(x))
                    \Longrightarrow \prescript{g_1}{}{x} = \prescript{g_2}{}{x}
                \end{align*}
                Veamos que \(g_1\Stab_G(x) = g_2\Stab_G(x)\):
                \begin{align*}
                    \prescript{g_1^{-1}g_2}{}{x} &= \prescript{g_1^{-1}}{}{\left(\prescript{g_2}{}{x}\right)} = \prescript{g_1^{-1}}{}{\left(\prescript{g_1}{}{x}\right)} = \prescript{g_1^{-1}g_1}{}{x} = \prescript{1}{}{x} = x
                    \Longrightarrow g_1^{-1}g_2\in \Stab_G(x)
                \end{align*}
                Por tanto, $g_1\Stab_G(x) = g_2\Stab_G(x)$, luego \(\varphi\) es inyectiva.
            \end{itemize}
            Por tanto, \(\varphi\) es un biyectiva. Hasta aquí, notemos que no hemos empleado que \(G\) sea finito. En tal caso, empleando la definición de índice, tenemos que:
            \begin{align*}
                |\Orb(x)| &= |G/\prescript{}{\Stab_G(x)}{\sim}|= [G: \Stab_G(x)].
            \end{align*}
        \end{proof}
    \end{prop}

    De esta forma, la fórmula de clases queda como sigue:
    \begin{equation*}
        |X| = |\Fix(X)| + \sum_{x\in \Gamma'} [G: \Stab_G(x)].
    \end{equation*}

    Para llegar a la fórmula de clases mencionada en el enunciado, se emplea una acción en concreto, la acción de \(G\) sobre sí mismo por conjugación.
    \begin{prop}[Acción por Conjugación]
        Sea \(G\) un grupo. Entonces, \(G\) actúa sobre sí mismo por conjugación de la forma siguiente:
        \Func{ac}{G\times G}{G}{(g, x)}{\prescript{g}{}{x} = g x g^{-1}}
    \end{prop}
    % // TODO: Demostrar que es una acción

    Calculemos los subconjuntos que aparecen en la fórmula de clases:
    \begin{align*}
        \Fix(G) &= \{x\in G \mid g x g^{-1} = x\quad \forall g\in G\}
        = \{x\in G \mid g x= x g\quad \forall g\in G\} =: Z(G)\\
        \Stab_G(x) &= \{g\in G \mid g x g^{-1} = x\}
        = \{g\in G \mid g x = x g\} =: C_G(x)\qquad \forall x\in G.
    \end{align*}

    Por tanto, la fórmula de clases buscada es (suponiendo que $G$ es finito):
    \begin{equation*}
        |G| = |Z(G)| + \sum_{x\in \Gamma'} [G: C_G(x)].
    \end{equation*}
\end{ejercicio}

\newpage
\begin{ejercicio}
    Demuestra el Teorema de Cauchy (Si \(G\) es un grupo finito y \(p\) es un primo que divide a \(|G|\), entonces \(G\) tiene un elemento de orden \(p\)). Concluye que, si \(G\) es finito, entonces \(G\) es un \(p-\)grupo si y solo si su orden es una potencia de \(p\).
    \begin{teo}[de Cauchy]
        Sea \(G\) un grupo finito y \(p\) un primo que divide a \(|G|\). Entonces, \(G\) tiene un elemento de orden \(p\).
        \begin{proof}
            Consideramos el siguiente conjunto:
            \begin{equation*}
                X = \{(a_1,\dots, a_p)\in G^p \mid a_1 a_2\cdots a_p = 1\}.
            \end{equation*}

            Sea ahora $\sigma=(1\ 2\ \ldots\ p)\in S_p$, y consideramos el siguiente grupo:
            \begin{equation*}
                H = \langle \sigma \rangle = \{\sigma^k \mid k = 0, \ldots, p-1\} < S_p.
            \end{equation*}

            Consideramos ahora la acción de \(H\) sobre \(X\) dada por:
            \Func{ac}{H\times X}{X}{(\sigma^k, (a_1,\dots, a_p))}{\prescript{\sigma^k}{}{(a_1,\dots, a_p)} = \left(a_{\sigma^k(1)},\dots, a_{\sigma^k(p)}\right)}
            \begin{itemize}
                \item Para cada $(a_1,\dots, a_p)\in X$, tenemos que:
                \begin{align*}
                    \prescript{1}{}{(a_1,\dots, a_p)} &= (a_1,\dots, a_p)
                \end{align*}
                \item Para cada \(k, l\in \{0, \ldots, p-1\}\), tenemos que:
                \begin{align*}
                    \prescript{\sigma^k}{}{\left(\prescript{\sigma^l}{}{(a_1,\dots, a_p)}\right)}
                    &= \prescript{\sigma^k}{}{\left(a_{\sigma^l(1)},\dots, a_{\sigma^l(p)}\right)}
                    = \left(a_{\sigma^k(\sigma^l(1))},\dots, a_{\sigma^k(\sigma^l(p))}\right) \\
                    &= \left(a_{\sigma^{k+l}(1)},\dots, a_{\sigma^{k+l}(p)}\right)
                    = \prescript{\sigma^{k+l}}{}{(a_1,\dots, a_p)}
                    = \prescript{\sigma^k\sigma^l}{}{(a_1,\dots, a_p)}.
                \end{align*}
            \end{itemize}

            Queremos trabajar con las órbitas de la acción de \(H\) sobre \(X\). En primer lugar, para cada \((a_1,\dots, a_p)\in X\), tenemos que:
            \begin{align*}
                |\Orb((a_1,\dots, a_p))| = [H: \Stab_H((a_1,\dots, a_p))]&\Longrightarrow
                |\Orb((a_1,\dots, a_p))| \mid |H|=p
                \Longrightarrow \\&\Longrightarrow
                |\Orb((a_1,\dots, a_p))| \in \{1, p\}
            \end{align*}


            Veamos cómo son las órbitas de la acción de \(H\) sobre \(X\). Sea \((a_1,\dots, a_p)\in X\). Entonces:
            \begin{align*}
                \Orb((a_1,\dots, a_p)) &= \left\{\prescript{\sigma^k}{}{(a_1,\dots, a_p)} \mid k = 0, \ldots, p-1\right\} \\
                &= \left\{(a_1,\dots, a_p), (a_2, a_3, \ldots, a_p, a_1), \ldots, (a_p, a_1, a_2, \ldots, a_{p-1})\right\}.
            \end{align*}

            Por tanto, la órbita será unitaria si y solo si $a_1 = a_2 = \ldots = a_p$, en cuyo caso será de la forma $(a, a, \ldots, a)\in X$, luego $a^p=1$. Sabemos que al menos hay una órbita unitaria, pues $(1, 1, \ldots, 1)\in X$. Queremos encontrar más. En vistas de aplicar la fórmula de clases, sea $r=|\Fix(X)|$ el número de órbitas unitarias de la acción de \(H\) sobre \(X\) y sea $\Gamma$ el conjunto formado por un único representante de cada órbita no unitaria (de orden $p$) de la acción de \(H\) sobre \(X\). Entonces, por la fórmula de clases, tenemos que:
            \begin{equation*}
                |X| = |\Fix(X)| + \sum_{x\in \Gamma}|\Orb(x)|
                = r + |\Gamma|p.
            \end{equation*}

            Por otro lado, razonemos el cardinal de $X$. Las primeras $p-1$ componentes de un elemento de \(X\) pueden ser elegidas de forma arbitraria, mientras que la última queda determinada por las anteriores como:
            \begin{equation*}
                a_p = (a_1 a_2\cdots a_{p-1})^{-1}.
            \end{equation*}

            Por tanto, tenemos que:
            \begin{equation*}
                |X| = |G|^{p-1}.
            \end{equation*}

            Uniendo ambas formas de calcular el cardinal de \(X\), tenemos que:
            \begin{equation*}
                |X| = |G|^{p-1} = r + |\Gamma|p
                \Longrightarrow r= |G|^{p-1} - |\Gamma|p.
            \end{equation*}

            Por hipótesis, $p\mid |G|$, luego $p\mid r$. Como $p$ es primo ($p>1$), tenemos que $r>1$. Por tanto, $\exists a\in G$ tal que:
            \begin{equation*}
                (1,1,\ldots, 1) \neq (a, a, \ldots, a) \in X.
            \end{equation*}

            Por tanto, $a\neq 1$ verifica que $a^p = 1$, luego $\ord(a)\mid p$. Como \(p\) es primo y \(a\neq 1\), tenemos que \(\ord(a) = p\). Por tanto, tenemos lo pedido.
        \end{proof}
    \end{teo}
    \begin{coro}
        Sea \(G\) un grupo finito y $p$ un número primo. Entonces, \(G\) es un \(p-\)grupo si y solo si \(|G|\) es una potencia de \(p\).
        \begin{proof}~
            \begin{description}
                \item[$\Longrightarrow$)] Sea $q$ un primo que divide a \(|G|\). Por el Teorema de Cauchy, \(\exists g\in G\) tal que \(\ord(g) = q\). Como $G$ es un \(p-\)grupo, tenemos que \(\exists k\in \mathbb{N}\) tal que \(q = p^k\), pero como $p$ y \(q\) son primos, tenemos que \(p = q\) y $k=1$.
                
                Por tanto, tenemos que el único primo que divide a \(|G|\) es \(p\), y considerando la descomposición en factores primos de \(|G|\), tenemos que $\exists n\in \mathbb{N}$ tal que:
                \begin{equation*}
                    |G| = p^n.
                \end{equation*}

                \item[$\Longleftarrow$)] Sea \(|G| = p^n\) para algún \(n\in \mathbb{N}\). Como el orden de todo elemento de \(G\) divide a \(|G|\), tenemos que, para todo \(g\in G\), \(\ord(g)\mid p^n\), luego \(\ord(g) = p^k\) para algún \(k\in \{0, \ldots, n\}\). Por tanto, \(G\) es un \(p-\)grupo.
            \end{description}
        \end{proof}
    \end{coro}
\end{ejercicio}

\newpage
\begin{ejercicio}
    Demuestra el Teorema de Burnside (el centro de un \(p-\)grupo finito es no trivial) y concluye, como consecuencia, que todo grupo de orden \(p^2\) es abeliano. Clasifica entonces, salvo isomorfismo, todos los grupos de órdenes \(4\), \(9\) y \(841\).
    \begin{teo}[de Burnside]
        Sea \(G\) un \(p-\)grupo finito. Entonces, $Z(G)\neq \{1\}$.
        \begin{proof}
            Sea \(G\) un \(p-\)grupo finito.
            \begin{itemize}
                \item Aunque no sería necesario, distinguimos el caso de $G$ abeliano. En tal caso, \(Z(G) = G\), luego \(Z(G)\neq \{1\}\). Esto nos permite centrarnos en el caso de \(G\) no abeliano, y por tanto la acción de \(G\) sobre sí mismo por conjugación no es trivial.
            \end{itemize}

            Consideramos la fórmula de clases de la acción de \(G\) sobre sí mismo por conjugación. Sea $\Gamma\subset G\setminus Z(G)$ el conjunto formado por un único representante de cada órbita no unitaria. Entonces, tenemos que:
            \begin{equation*}
                |G| = |Z(G)| + \sum_{x\in \Gamma} [G: C_G(x)].
            \end{equation*}

            Como \(G\) es un \(p-\)grupo finito, tenemos que \(|G| = p^n\) para algún \(n\in \mathbb{N}\). Por otro lado, como como $G$ es finito, para cada $x\in \Gamma$ se tiene que $[G: C_G(x)]$ divide a $|G|=p^n$, luego $\exists k_x\in \{0,1,\dots,n\}$ tal que:
            \begin{equation*}
                [G: C_G(x)] = p^{k_x}
            \end{equation*}

            \begin{itemize}
                \item Si $\exists x\in \Gamma$ tal que \(k_x = 0\), entonces $|G|=|C_G(x)|$ y \(C_G(x) = G\), luego \(x\in Z(G)\), lo cual contradice que \(x\in \Gamma\subset G\setminus Z(G)\).
            \end{itemize} 

            Por tanto, tenemos que \(k_x\in \{1, \ldots, n\}\) para todo \(x\in \Gamma\). Por tanto, despejando de la fórmula de clases, tenemos que:
            \begin{align*}
                |Z(G)| &= |G| - \sum_{x\in \Gamma} [G: C_G(x)]
                = p^n - \sum_{x\in \Gamma} p^{k_x}
            \end{align*}

            Como \(k_x > 0\) para todo \(x\in \Gamma\), el sumatorio es múltiplo de \(p\), y por tanto $p\mid |Z(G)|$, de donde $|Z(G)|\geq p$. En particular, tenemos que \(Z(G)\) no es trivial.
        \end{proof}
    \end{teo}

    \begin{lema}
        Sea \(G\) un grupo. Si $G/Z(G)$ es cíclico, entonces \(G\) es abeliano.
        \begin{proof}
            Supuesto que \(G/Z(G)\) es cíclico, \(\exists gZ(G)\in G/Z(G)\) tal que:
            \begin{equation*}
                G/Z(G) = \langle gZ(G)\rangle.
            \end{equation*}

            Como las clases de equivalencia del cociente \(G/Z(G)\) forman una partición de \(G\), para cada \(x\in G\) $\exists k\in \mathbb{Z}$ tal que:
            \begin{equation*}
                x\in g^{k} Z(G).
            \end{equation*}

            En vistas de demostrar que \(G\) es abeliano, sean $a,b\in G$. Entonces, $\exists k_1, k_2\in \mathbb{Z}$ tales que:
            \begin{align*}
                a&\in g^{k_1} Z(G)\\
                b&\in g^{k_2} Z(G).
            \end{align*}
            Por tanto, $\exists z_a, z_b\in Z(G)$ tales que:
            \begin{align*}
                a &= g^{k_1} z_a\\
                b &= g^{k_2} z_b.
            \end{align*}

            Entonces, puesto que $z_a, z_b\in Z(G)$:
            \begin{align*}
                ab &= g^{k_1} z_a g^{k_2} z_b = g^{k_1+k_2} z_a z_b
                = g^{k_2} z_b g^{k_1} z_a = b a,
            \end{align*}
            Puesto que \(a\) y \(b\) son arbitrarios, tenemos que \(G\) es abeliano.
        \end{proof}
    \end{lema}
    \begin{coro}
        Sea \(G\) un \(p-\)grupo finito con \(|G| = p^n\) para algún \(n\in \mathbb{N}\). Entonces:
        \begin{equation*}
            |Z(G)| \neq p^{n-1}.
        \end{equation*}
        \begin{proof}
            Supongamos que \(|Z(G)| = p^{n-1}\).
            Como $Z(G)\lhd G$, considerando el cociente tenemos que:
            \begin{equation*}
                \left|G/Z(G)\right| = \frac{|G|}{|Z(G)|} = \frac{p^n}{|p^{n-1}|} = p.
            \end{equation*}

            Por tanto, \(G/Z(G)\) es un grupo de orden primo, luego es cíclico. Como $G/Z(G)$ es cíclico, por el lema anterior \(G\) es abeliano, luego \(Z(G) = G\) y \(|Z(G)| = |G| = p^n\), lo cual contradice la hipótesis de que \(|Z(G)| = p^{n-1}\).
        \end{proof}
    \end{coro}

    Sea ahora \(p\) un primo, y $G$ un \(p-\)grupo finito con \(|G| = p^2\). Como $Z(G)<G$ y el orden de todo subgrupo de \(G\) divide a \(|G|\), tenemos que \(|Z(G)|\in \{1, p, p^2\}\):
    \begin{itemize}
        \item Si \(|Z(G)| = 1\), entonces \(Z(G) = \{1\}\), en contradicción con el Teorema de Burnside.
        \item Si \(|Z(G)| = p\), entonces \(Z(G)=p^{2-1}\), en contradicción con el Corolario anterior.
    \end{itemize}

    Por tanto, \(|Z(G)| = p^2\), luego \(Z(G) = G\) y \(G\) es abeliano.\\

    Clasificamos ahora los grupos de órdenes \(4\), \(9\) y \(841\) haciendo uso de que:
    \begin{align*}
        4= 2^2 \qquad 9= 3^2 \qquad 841 = 29^2.
    \end{align*}

    Por el corolario anterior, todo grupo de dichos órdenes es abeliano (y es finito), luego empleamos el Teorema de Estructura de los Grupos Abelianos Finitos. Como el número $2$ tan solo tiene dos particiones:
    \begin{equation*}
        \begin{array}{rcl}
            2 &=& 2\\
            2 &=& 1+1
        \end{array}
    \end{equation*}

    Entonces, podemos afirmar lo que sigue:
    \begin{itemize}
        \item Los grupos de orden \(4\) salvo isomorfismo son:
        \begin{equation*}
            \mathbb{Z}_4 \qquad \mathbb{Z}_2\oplus\mathbb{Z}_2.
        \end{equation*}
        \item Los grupos de orden \(9\) salvo isomorfismo son:
        \begin{equation*}
            \mathbb{Z}_9 \qquad \mathbb{Z}_3\oplus\mathbb{Z}_3.
        \end{equation*}
        \item Los grupos de orden \(841\) salvo isomorfismo son:
        \begin{equation*}
            \mathbb{Z}_{841} \qquad \mathbb{Z}_{29}\oplus\mathbb{Z}_{29}
        \end{equation*}
    \end{itemize}
\end{ejercicio}

\newpage
\begin{ejercicio}
    (Teoremas de Sylow) Demuestra que, si \(G\) es un grupo finito, para cualquier potencia de un primo \(p\) que divida al orden del grupo existe un subgrupo cuyo orden es esa potencia de \(p\). Define entonces el concepto de \(p-\)subgrupo de Sylow de un grupo finito \(G\) y concluye la existencia de \(p-\)subgrupos de Sylow de \(G\).

    \begin{teo}
        Sea $G$ un grupo finito y \(p\) un primo. Entonces, para todo $k\in \mathbb{N}$ tal que \(p^k\mid |G|\), existe un subgrupo \(H<G\) tal que \(|H| = p^k\).

        
        \begin{proof}
            Demostramos por inducción sobre \(k\).
            \begin{itemize}
                \item \ul{Para $k=1$}:
                
                Por el Teorema de Cauchy, $\exists g\in G$ tal que \(\ord(g) = p\). Entonces, el subgrupo \(H = \langle g\rangle\) es un subgrupo de \(G\) de orden \(p\).

                \item \ul{Para $k\in \mathbb{N}$, $k>1$}: Suponemos el resultado cierto para todo $l<k$. Esto es:
                \begin{equation*}
                    p^l\mid |G| \Longrightarrow \exists H<G: |H| = p^l \qquad \forall l<k.
                \end{equation*}
                Demostrémoslo para \(k\). Como \(p^k\mid |G|\), $\exists n\in \mathbb{N}$ tal que:
                \begin{equation*}
                    |G| = p^k n
                \end{equation*}

                Demostramos ahora el resultado para todo $n\in \mathbb{N}$ por inducción.
                \begin{itemize}
                    \item \ul{Para $n=1$}:
                    
                    Entonces, $|G|=p^k$, luego sea $H=G$. Entonces, \(|H| = p^k\), luego el resultado es cierto.

                    \item \ul{Para $n\in \bb{N},\ n>1$}:
                    
                    Suponemos el resultado cierto para todo $m<n$. Esto es:
                    \begin{equation*}
                        |G| = p^k m \Longrightarrow \exists H<G: |H| = p^k \qquad \forall m<n.
                    \end{equation*}

                    Veamos qué ocurre si $|G|=p^k n$. Distinguimos dos casos:
                    \begin{itemize}
                        \item Si $\exists K<G$, $K\neq G$, tal que \(p\nmid [G:K]\).
                        
                        En tal caso, como $|G|=[G:K]|K|$, y $p^k\mid |G|$, tenemos que:
                        \begin{equation*}
                            p^k\mid |K| \Longrightarrow \exists s\in \mathbb{N}\ \text{ tal que } |K| = p^k s.
                        \end{equation*}

                        Además, puesto que $K\neq G$, entonces $s<n$, luego por la segunda hipótesis de inducción, tenemos que:
                        \begin{equation*}
                            \exists H<K<G: |H| = p^k.
                        \end{equation*}

                        \item Si para todo $K<G$, $K\neq G$, se tiene que $p\mid [G:K]$.
                        
                        Con vistas a aplicar la fórmula de clases de la acción de \(G\) sobre sí mismo por conjugación, definimos $\Gamma$ como el conjunto formado por un único representante de cada órbita no unitaria de la acción de \(G\) sobre sí mismo por conjugación. Para cada $x\in \Gamma$, tenemos que:
                        \begin{equation*}
                            x\in \Gamma \Longrightarrow x\notin Z(G) \Longrightarrow C_G(x) \neq G
                            \Longrightarrow p\mid [G: C_G(x)].
                        \end{equation*}

                        Por tanto, por la fórmula de clases, tenemos que:
                        \begin{align*}
                            |G| &= |Z(G)| + \sum_{x\in \Gamma} [G: C_G(x)]
                            \Longrightarrow |Z(G)| = |G| - \sum_{x\in \Gamma} [G: C_G(x)].
                        \end{align*}
                        Como \(p\mid |G|\) y \(p\mid [G: C_G(x)]\) para todo \(x\in \Gamma\), tenemos que:
                        \begin{equation*}
                            p\mid |Z(G)|
                        \end{equation*}

                        Por el Teorema de Cauchy, $\exists K<Z(G)$ tal que $|K| = p$. Como $K\subset Z(G)$, tenemos que \(K\lhd G\), por lo que podemos considerar el cociente \(G/K\). Entonces, tenemos que:
                        \begin{equation*}
                            |G/K| = \frac{|G|}{|K|} = \frac{p^k n}{p} = p^{k-1} n.
                        \end{equation*}

                        Por tanto, $p^{k-1}\mid |G/K|$, luego por la primera hipótesis de inducción, tenemos que:
                        \begin{equation*}
                            \exists L<G/K: |L| = p^{k-1}.
                        \end{equation*}

                        Por el Tercer Teorema de Isomorfía, considerando $H=p^*(L)$, tenemos que $K\lhd H<G$, con:
                        \begin{equation*}
                            L=\frac{H}{K} \Longrightarrow |H| = |L||K| = p^{k-1} p = p^k.
                        \end{equation*}
                        Por tanto, hemos encontrado un subgrupo \(H<G\) tal que \(|H| = p^k\).
                    \end{itemize}
                \end{itemize}
            \end{itemize}
        \end{proof}
    \end{teo}

    Una vez dado dicho teorema, definimos el concepto de \(p-\)subgrupo de Sylow de un grupo finito \(G\).
    \begin{definicion}[\(p-\)subgrupo de Sylow]
        Sea \(G\) un grupo finito y \(p\) un primo. Un \(p-\)subgrupo de Sylow de \(G\) es un \(p-\)subgrupo de $G$ cuyo orden es la mayor potencia de \(p\) que divide a \(|G|\).
        Es decir, si \(|G| = p^k m\) con \(\mcd(p, m) = 1\), entonces un \(p-\)subgrupo $H<G$ es un \(p-\)subgrupo de Sylow de \(G\) si \(|H| = p^k\).
    \end{definicion}

    El siguiente corolario, consecuencia directa del Teorema anterior, nos garantiza la existencia de \(p-\)subgrupos de Sylow en un grupo finito \(G\).
    \begin{coro}
        Sea \(G\) un grupo finito y \(p\) un primo tal que $p\mid |G|$. Entonces, \(G\) tiene un \(p-\)subgrupo de Sylow.
        \begin{proof}
            Como $p\mid |G|$, tenemos que \(\exists k,m\in \mathbb{N}\) tales que:
            \begin{equation*}
                |G| = p^k m \quad \text{con } \mcd(p, m) = 1.
            \end{equation*}
            Por el Teorema anterior, como \(p^k\mid |G|\), tenemos que:
            \begin{equation*}
                \exists H<G: |H| = p^k.
            \end{equation*}
            Por tanto, \(H\) es un \(p-\)subgrupo de \(G\). Además, como \(p^k\) es la mayor potencia de \(p\) que divide a \(|G|\), tenemos que \(H\) es un \(p-\)subgrupo de Sylow de \(G\).
        \end{proof}
    \end{coro}
\end{ejercicio}


\newpage
\begin{ejercicio}
    (Teoremas de Sylow) Demuestra que todo \(p-\)subgrupo de un grupo finito \(G\) (con \(|G| = p^k m\) y \(\mcd(p,m) = 1\)) está contenido en un \(p-\)subgrupo de Sylow y que el número \(n_p\) de \(p-\)subgrupos de Sylow de \(G\) satisface que \(n_p \mid m\) y que \(n_p \equiv 1 \mod p\).

    \begin{lema}
        Sea $G$ un grupo finito y \(P<G\) un \(p-\)subgrupo de Sylow de \(G\). Si $H<N_G(P)$ es un \(p-\)subgrupo de \(N_G(P)\), entonces \(H\subset P\). Es decir, los \(p-\)subgrupos del normalizador de un \(p-\)subgrupo de Sylow estarán contenidos en el propio \(p-\)subgrupo de Sylow.
        \begin{proof}
            Como $P\lhd N_G(P)$, y $H<N_G(P)$, aplicamos el Segundo Teorema de Isomorfía como muestra la Figura \ref{fig:segundo_teorema_isomorfia}.
            \begin{figure}[h]
                \centering
                \shorthandoff{""}
                \begin{tikzcd}
                                    & G \arrow[d, no head]                                                        &                       \\
                                    & N_G(P) \arrow[d, no head] \arrow[ldd, no head] \arrow[rdd, "\rhd", no head] &                       \\
                                    & HP \arrow[rd, no head] \arrow[ld, no head]                                  &                       \\
                H \arrow[rd, no head] &                                                                             & P \arrow[ld, no head] \\
                                    & H\cap P \arrow[d, no head]                                                  &                       \\
                                    & \{1\}                                                                       &                      
                \end{tikzcd}
                \shorthandon{""}
                \caption{Segundo Teorema de Isomorfía}
                \label{fig:segundo_teorema_isomorfia}
            \end{figure}

            De esta forma, obtenemos:
            \begin{itemize}
                \item $HP<N_G(P)$.
                \item $H\cap P\lhd H$, $P\lhd HP$.
                \item El siguiente isomorfismo:
                \begin{equation*}
                    \frac{H}{H\cap P} \cong \frac{HP}{P}.
                \end{equation*}
            \end{itemize}

            Gracias a dicho isomorfismo, definimos:
            \begin{equation*}
                r = [H: H\cap P] = [HP: P]\geq 1.
            \end{equation*}

            Distinguimos dos casos:
            \begin{itemize}
                \item Si \(r = 1\), entonces \(HP=P\), luego \(H\subset P\).
                \item Si \(r>1\), tenemos que:
                \begin{equation*}
                    P<HP< G
                    \Longrightarrow [G: P] = [G: HP][HP: P]=[G: HP]r.
                \end{equation*}
                Supuesto que $p\mid r$, entonces $p\mid [G: P]$. No obstante, esto contradice que \(P\) es un \(p-\)subgrupo de Sylow de \(G\), luego hemos llegado a una contradicción. Por tanto, $p\nmid r$.

                Por otro lado, como la intersección de \(p-\)grupos es un \(p-\)grupo (basta ver la definición de \(p-\)grupo) y $r>1$, tenemos que $\exists i\in \mathbb{N}$, \(i>0\), tal que:
                \begin{align*}
                    r = [H:H\cap P] = \left|\dfrac{H}{H\cap P}\right| &= p^i\Longrightarrow p\mid r.
                \end{align*}

                Por tanto, hemos demostrado que \(p\mid r\) y \(p\nmid r\), lo cual es una contradicción. Por tanto, este caso no puede ocurrir.
            \end{itemize}

            Por tanto, concluimos que \(H\subset P\).
        \end{proof}
    \end{lema}

    Buscamos ahora demostrar en sí el Segundo Teorema de Sylow.
    \begin{teo}[Segundo Teorema de Sylow]
        \item Sea $G$ un grupo finito y $p$ un número primo. Supongamos que $|G|=p^k m$ con \(\mcd(p, m) = 1\). Sea además $n_p$ el número de \(p-\)subgrupos de Sylow de \(G\). Entonces:
        \begin{enumerate}
            \item Todo \(p-\)subgrupo de \(G\) está contenido en un \(p-\)subgrupo de Sylow de \(G\).
            \item Cualesquiera dos \(p-\)subgrupos de Sylow de \(G\) son conjugados entre sí.
            \item $n_p\mid m$ y \(n_p \equiv 1 \mod p\).
        \end{enumerate}
        \begin{proof}
            Demostramos cada resultado por separado:
            \begin{enumerate}
                \item Previa a la demostración en si, veamos un resultado que nos será de ayuda. Definimos la acción de \(G\) sobre $\Syl_p(G)$ por conjugación:
                \Func{ac}{G\times \Syl_p(G)}{\Syl_p(G)}{(g, P)}{gPg^{-1}}
                Veamos que está bien definida. Es necesario ver que, fijado $g\in G,\ P\in \Syl_p(G)$, se tiene que \(gPg^{-1}\in \Syl_p(G)\). Veamos que la siguiente aplicación es biyectiva:
                \Func{f}{P}{gPg^{-1}}{p}{gpg^{-1}}
                \begin{itemize}
                    \item Si $\wt{x}\in gPg^{-1}$, entonces $\exists p\in P$ tal que $\wt{x} = gpg^{-1}$. De esta forma:
                    \begin{equation*}
                        f(p) = gpg^{-1} = \wt{x}
                    \end{equation*}
                    Por tanto, \(f\) es sobreyectiva.
                    \item Dados $p_1,p_2\in P$, si \(f(p_1) = f(p_2)\), entonces:
                    \begin{align*}
                        g p_1 g^{-1} &= g p_2 g^{-1}
                        \Longrightarrow p_1 = p_2,
                    \end{align*}
                    luego \(f\) es inyectiva.
                \end{itemize}
                Por tanto, \(f\) es biyectiva, luego $|P|=|gPg^{-1}|$, y como $P\in \Syl_p(G)$, tenemos que $gPg^{-1}\in \Syl_p(G)$, luego la acción de \(G\) sobre \(\Syl_p(G)\) por conjugación está bien definida.\\

                Fijamos ahora para toda la demostración $P_1\in \Syl_p(G)$, y estudiamos su órbita y su estabilizador:
                \begin{align*}
                    \Orb(P_1) &= \{gP_1g^{-1}\mid g\in G\}\\
                    \Stab_{G}(P_1) &= \{g\in G\mid gP_1g^{-1} = P_1\} = N_G(P_1).
                \end{align*}

                Como $P_1\lhd N_G(P_1)<G$, por el Teorema de Lagrange, tenemos que:
                \begin{equation*}
                    [G:P_1] = [G:N_G(P_1)][N_G(P_1):P_1]
                \end{equation*}
                Por un lado, como $P_1\in \Syl_p(G)$, y $|G| = p^k m$ con \(\mcd(p, m) = 1\), tenemos que $[G:P_1] = m$. Además, $[G:N_G(P_1)]=[G:\Stab_{G}(P_1)]=|\Orb(P_1)|$, luego:
                \begin{equation}\label{eq:orbita}
                    m = |\Orb(P_1)|[N_G(P_1):P_1].
                \end{equation}

                Por tanto:
                \begin{align*}
                    1=\mcd(p, m) &= \mcd\left(p, |\Orb(P_1)|[N_G(P_1):P_1]\right)
                    \Longrightarrow
                    \mcd\left(p, |\Orb(P_1)|\right) = 1
                \end{align*}
                de donde deducimos que $p\nmid |\Orb(P_1)|$. Visto esto, podemos comenzar con la demostración en sí.\\

                Sea $H$ un \(p-\)subgrupo de \(G\). Definimos la siguiente acción:
                \Func{ac}{H\times \Orb(P_1)}{\Orb(P_1)}{(h, P)}{hPh^{-1}}
                Veamos que está bien definida. Sea \(P\in \Orb(P_1)\), entonces $\exists g\in G$ tal que \(P = gP_1g^{-1}\). Entonces, tenemos que:
                \begin{align*}
                    hPh^{-1} &= h(gP_1g^{-1})h^{-1} = hgP_1g^{-1}h^{-1} = (hg)P_1(hg)^{-1}.
                \end{align*}
                Como \(hg\in G\), tenemos que \(hPh^{-1}\in \Orb(P_1)\), luego está bien definida. Para todo $P\in \Orb(P_1)$, tenemos:
                \begin{align*}
                    \Stab_{H}(P) &= \{h\in H\mid hPh^{-1} = P\} = \{h\in H\mid hP=Ph\} = H\cap N_G(P)
                \end{align*}

                Por un lado, $H\cap N_G(P)<H$. Por otro lado, como $P\in \Orb(P_1)\subset \Syl_p(G)$ y $H\cap N_G(P)<N_G(P)$ es un $p-$subgrupo de \(N_G(P)\), por el Lema anteriormente demostrado tenemos que $H\cap N_G(P)<P$. Por tanto, tenemos que:
                \begin{equation*}
                    \Stab_{H}(P) = H\cap N_G(P) < H\cap P
                \end{equation*}
                Como además $P<N_G(P)$, tenemos que:
                \begin{equation*}
                    \Stab_{H}(P) = H\cap N_G(P) = H\cap P
                \end{equation*}

                En vistas de aplicar la fórmula de clases, sea $\Gamma$ el conjunto formado por un único representante de cada órbita de la acción de \(H\) sobre \(\Orb(P_1)\). Entonces, tenemos que:
                \begin{equation}\label{eq:orbita2}
                    |\Orb(P_1)| = \sum_{P\in \Gamma} |\Orb(P)|
                    = \sum_{P\in \Gamma} [H:\Stab_{H}(P)]
                    = \sum_{P\in \Gamma} [H:H\cap P]
                \end{equation}

                Tenemos que cada sumando divide a $|H|$, que como es un \(p-\)subgrupo, es una potencia de \(p\). Como cada sumando divide a una potencia de $p$ y, por lo demostrado al inicio de la demostración, $p\nmid |\Orb(P_1)|$, $\exists \wt{P}\in \Gamma$ tal que:
                \begin{equation*}
                    [H:H\cap \wt{P}] = 1 \Longrightarrow H\cap \wt{P} = H\Longrightarrow H< \wt{P}.
                \end{equation*}
                Como $\wt{P}\in \Gamma\subset \Orb(P_1)\subset \Syl_p(G)$, tenemos que \(H\) está contenido en un \(p-\)subgrupo de Sylow de \(G\).

                \item Sean \(P_1, P_2\in \Syl_p(G)\). Como $P_2$ es un \(p-\)subgrupo de $G$, por el apartado anterior, tenemos que \(P_1\) está contenido en un \(p-\)subgrupo de Sylow de \(G\), que llamaremos $\wt{P}$ por seguir la misma notación. Además, por la demostración anterior $\wt{P}\in \Orb(P_1)$, luego $\exists g\in G$ tal que:
                \begin{equation*}
                    \wt{P} = gP_1g^{-1}.
                \end{equation*}

                Por tanto, $P_2<\wt{P}$ pero, por ser ambos \(p-\)subgrupos de Sylow de \(G\), $|P_2|=|\wt{P}|$, luego:
                \begin{equation*}
                    P_2 = \wt{P} = gP_1g^{-1}
                \end{equation*}
                
                Por tanto, $P_1$ y \(P_2\) son conjugados entre sí.

                \item Sea \(n_p\) el número de \(p-\)subgrupos de Sylow de \(G\). Por el apartado anterior, tenemos que $\Orb(P_1)=\Syl_p(G)$, luego:
                \begin{equation*}
                    n_p = |\Syl_p(G)| = |\Orb(P_1)|
                \end{equation*}

                Por la Ecuación~\eqref{eq:orbita}, tenemos que $n_p\mid m$. Para ver que \(n_p \equiv 1 \mod p\), empleamos la Ecuación~\eqref{eq:orbita2} con $H=P_1$:
                \begin{align*}
                    n_p &= |\Orb(P_1)| = \sum_{P\in \Gamma} [P_1:P_1\cap P]
                \end{align*}
                Tenemos que $[P_1:P_1\cap P]$ es una potencia de \(p\) para todo \(P\in \Gamma\).
                \begin{itemize}
                    \item Si $[P_1:P_1\cap P] = 1$, entonces \(P_1\cap P = P_1\), luego \(P_1\subset P\). Como $P\in \Syl_p(G)$, tenemos que \(P_1 = P\). Por tanto, \(P_1\) es el único elemento de \(\Gamma\) tal que $[P_1:P_1\cap P] = 1$; el resto de sumandos son potencias de \(p\) mayores que \(1\).
                \end{itemize}
                Por tanto:
                \begin{equation*}
                    n_p = 1 + \sum_{P\in \Gamma, P\neq P_1} [P_1:P_1\cap P]
                    \Longrightarrow p\mid n_p-1
                    \Longrightarrow n_p \equiv 1 \mod p.
                \end{equation*}
            \end{enumerate}
        \end{proof}
    \end{teo}
\end{ejercicio}

\newpage
\begin{ejercicio}
    Prueba que todos los grupos de orden \(2p\) siendo \(p\) un primo impar y también todos los de orden \(pq\) siendo \(p, q\) primos con \(p > q\) y \(q \nmid p - 1\) son producto semidirecto. Clasifica los grupos de estos órdenes y concluye entonces con la clasificación de todos los grupos de órdenes \(6\), \(10\), \(14\), \(15\), \(161\) y \(1994\).\\

    Comenzamos con el caso de \(2p\) siendo \(p\) un primo impar. Sea \(G\) un grupo de orden \(2p\). Sea $n_2$ el número de \(2-\)subgrupos de Sylow de \(G\) y $n_p$ el número de \(p-\)subgrupos de Sylow de \(G\). Por el Segundo Teorema de Sylow, tenemos que:
    \begin{equation*}
        n_p\mid 2 \quad \text{y} \quad n_p \equiv 1 \mod p.
    \end{equation*}
    Por tanto, tenemos que \(n_p = 1\), sea este $P$. Puesto que $P$ es el único \(p-\)subgrupo de Sylow de \(G\), tenemos que \(P\lhd G\). Además, puesto que $|P|=p$ primo, tenemos que $P\cong C_p$, luego tiene $p-1$ elementos de orden \(p\).

    Por otro lado, por el Primer Teorema de Sylow, tenemos que $n_2\geq 1$. Sea $Q\in \Syl_2(G)$ un \(2-\)subgrupo de Sylow de \(G\). Como $|Q|=2$, tenemos que \(Q\cong C_2\).\\
        
    Buscamos demostrar que $G\cong P\rtimes Q$ es un producto semidirecto.
    \begin{itemize}
        \item En primer lugar, $P\lhd G$.
        \item $Q\cap P = \{1\}$, pues todos los elementos distintos del neutro de \(P\) son de orden \(p\) y todos los elementos distintos del neutro de \(Q\) son de orden \(2\), luego no pueden coincidir.
        \item Por otro lado, como $P\lhd G$, aplicamos el Segundo Teorema de Isomorfía y obtenemos $PQ<G$ con:
        \begin{align*}
            \dfrac{PQ}{P} \cong \dfrac{Q}{P\cap Q}
            \Longrightarrow
            |PQ| = \dfrac{|Q|\cdot |P|}{|P\cap Q|}
            = \dfrac{2p}{1} = 2p.
        \end{align*}
        Como $|PQ| = |G|$, tenemos que \(PQ=G\).
    \end{itemize}
    Por tanto, $G\cong P\rtimes_{\theta} Q$ es un producto semidirecto, donde $\theta$ es la representación de $G$ por permutaciones haciendo uso de la acción de \(Q\) sobre \(P\) por conjugación. Es decir, tenemos que:
    \Func{\theta}{Q}{\Aut(P)}{q}{\theta(q)}
    \Func{\theta(q)}{P}{P}{p}{q p q^{-1}}


    Buscamos ahora otro enfoque para determinar que son producto semidirecto. Como \(P\cong C_p\), tenemos que \(\Aut(P)\cong C_{p-1}\). Por tanto:
    \begin{align*}
        P\cong C_p
        \Longrightarrow \exists &x_p\in P\text{ tal que } P=\langle x_p\mid x_p^p=1\rangle\\
        Q\cong C_2
        \Longrightarrow \exists &x_q\in Q\text{ tal que } Q=\langle x_q\mid x_q^2=1\rangle\\
        \Aut(P)\cong C_{p-1}
        \Longrightarrow \exists &\wt{x}_p\in \Aut(P)\text{ tal que } \Aut(P)=\langle \wt{x}_p\mid \wt{x}_p^{p-1}=1\rangle.
    \end{align*}

    Buscamos ahora los elementos de $\Aut(P)$.
    Como $P=\langle x_p\rangle$, por el Teorema de Dyck tenemos que los posibles isomorfismos son los que lleven \(x_p\) a un generador de \(P\) (que como \(P\) es cíclico, son todos los elementos de orden \(p\)). Por tanto, estos son los siguientes:
    \begin{align*}
        x_p & \mapsto \alpha_1(x_p) = x_p\\
        x_p & \mapsto \alpha_2(x_p) = x_p^2\\
        &\vdots\\
        x_p & \mapsto \alpha_{p-1}(x_p) = x_p^{p-1}.
    \end{align*}

    De todos estos, tan solo buscamos quedarnos con los que nos son válidos.
    Como $\theta$ es un homomorfismo y $O(x_q)=2$, tenemos que $O(\theta(x_q))\mid 2$, luego \(\ord(\theta(x_q))\in \{1,2\}\).
    \begin{itemize}
        \item Sea \(\ord(\theta(x_q)) = 1\). Entonces \(\theta(x_q) = Id_P\), luego $\theta$ es el homomorfismo trivial, y por tanto \(G\cong P\times Q\cong C_p\times C_2\cong C_{2p}\).
        \item Sea \(\ord(\theta(x_q)) = 2\). Buscamos por tanto los elementos de \(\Aut(P)\) de orden \(2\). Como en $C_{p-1}$ tan solo hay un elemento de orden \(2\) $\left(\wt{x}_p^{\frac{p-1}{2}}\right)$ (donde usamos que \(p\) es primo impar), entonces tenemos que tan solo hay una elección posible para \(\theta(x_q)\). Esta es la siguiente:
        \Func{\alpha}{P}{P}{x}{x^{-1}}

        Como \(\alpha=\alpha_{p-1}\), es un homomorfismo. Como esta elección es única, uniendo con el homomorfismo por conjugación anteriormente encontrado, tenemos que:
        \begin{equation*}
            qpq^{-1} = \theta(q)(p) = p^{-1}
            \Longrightarrow qp=p^{-1}q\qquad \forall p\in P,\ q\in Q.
        \end{equation*}

        Buscamos establecer ahora un isomorfismo entre $D_p$ y $G$. Vemos que:
        \begin{gather*}
            x_p^{p} = 1\qquad
            x_q^{2} = 1\\
            x_q x_p = x_p^{-1} x_q\quad (x_q\in Q,\ x_p\in P).
        \end{gather*}
        Por tanto, por el Teorema de Dyck, existe un homomorfismo entre $D_p$ y $G$. Como además $G=PQ=\langle x_p, x_q\rangle$ y $|G|=2p=|D_p|$, tenemos que se trata de un isomorfismo. Por tanto, tenemos que:
        \begin{equation*}
            G\cong D_p.
        \end{equation*}
    \end{itemize}

    Visto esto, podemos realizar las siguientes clasificaciones:
    \begin{itemize}
        \item Los grupos de orden \(6=2\cdot 3\) son $C_6$ y \(D_3\).
        \item Los grupos de orden \(10=2\cdot 5\) son \(C_{10}\) y \(D_5\).
        \item Los grupos de orden \(14=2\cdot 7\) son \(C_{14}\) y \(D_7\).
        \item Los grupos de orden \(1994=2\cdot 997\) son \(C_{1994}\) y \(D_{997}\).
    \end{itemize}~\\

    Ahora, consideramos el caso de \(|G|=pq\) siendo \(p, q\) primos con \(p > q\) y \(q \nmid p - 1\). Sea $n_p$ el número de \(p-\)subgrupos de Sylow de \(G\) y $n_q$ el número de \(q-\)subgrupos de Sylow de \(G\). Por el Segundo Teorema de Sylow, tenemos que:
    \begin{equation*}
        n_p\mid q \quad \text{y} \quad n_p \equiv 1 \mod p.
    \end{equation*}
    Como $n_p\leq q<p$, entonces $n_p=1$, luego sea \(P\) el único \(p-\)subgrupo de Sylow de \(G\). Como \(P\) es el único \(p-\)subgrupo de Sylow de \(G\), tenemos que \(P\lhd G\). Además, puesto que \(|P|=p\) primo, tenemos que $P\cong C_p$.\\

    Por otro lado, por el Segundo Teorema de Sylow, tenemos que:
    \begin{equation*}
        n_q\mid p \quad \text{y} \quad n_q \equiv 1 \mod q.
    \end{equation*}
    Por tanto, como $n_q\mid p$, tenemos que $n_q\in \{1, p\}$.
    \begin{itemize}
        \item Si \(n_q = p\), entonces:
        \begin{equation*}
            p\equiv 1 \mod q
            \Longrightarrow \exists k\in \mathbb{N}\text{ tal que } p-1 = kq
            \Longrightarrow q\mid p-1.
        \end{equation*}
        No obstante, esto contradice la hipótesis de que \(q \nmid p - 1\). Por tanto, este caso no puede ocurrir.
    \end{itemize}
    Por tanto, tenemos que \(n_q = 1\), sea \(Q\) el único \(q-\)subgrupo de Sylow de \(G\). Como \(Q\) es el único \(q-\)subgrupo de Sylow de \(G\), tenemos que \(Q\lhd G\). Además, puesto que \(|Q|=q\) primo, tenemos que $Q\cong C_q$.\\

    Como todos los subgrupos de Sylow de \(G\) son únicos, tenemos que:
    \begin{equation*}
        G\cong P\times Q\cong C_p\times C_q \cong C_{pq}.
    \end{equation*}

    Por tanto, \(G\cong C_{pq}\). Ya podemos realizar las siguientes clasificaciones:
    \begin{itemize}
        \item El único grupo de orden \(15=3\cdot 5\) $(3\nmid 4)$ es \(C_{15}\).
        \item El único grupo de orden \(161=7\cdot 23\) $(7\nmid 22)$ es \(C_{161}\).
    \end{itemize}

\end{ejercicio}

\newpage
\begin{ejercicio}
    Clasifica, salvo isomorfismo, todos los grupos de orden \(8\).\\

    Sea $G$ un grupo de orden \(8=2^3\). Si $G$ es abeliano, entonces por el Teorema de Estructura de los Grupos Abelianos Finitos, están las siguientes posibilidades:
    \begin{gather*}
        C_8 \qquad C_4\oplus C_2 \qquad C_2\oplus C_2\oplus C_2.
    \end{gather*}

    Supongamos de aquí en adelante que \(G\) no es abeliano. Como $|G|=8$, todos los elementos distintos del neutro de \(G\) son de orden \(2, 4\) u \(8\). Como \(G\) no es abeliano, en particular no es cíclico, luego no puede haber un elemento de orden \(8\). Por tanto, todos los elementos distintos del neutro de \(G\) son de orden \(2\) o \(4\).\\

    Supongamos que todos los elementos distintos del neutro de \(G\) son de orden \(2\). Dados $a,b\in G$, como $a,b\in G$, tenemos que:
    \begin{align*}
        1 = (ab)^2 &= abab
        \Longrightarrow
        a=bab\Longrightarrow
        ab = ba.
    \end{align*}
    Por tanto, \(G\) es abeliano, lo cual contradice el caso en el que estamos. Por tanto, no todos los elementos distintos del neutro de \(G\) son de orden \(2\), luego $\exists a\in G$ tal que \(\ord(a) = 4\). Sea ahora $b\in H\setminus \langle a\rangle$. Veamos que $G=\langle a, b\rangle$:
    \begin{itemize}
        \item Como $G$ es un grupo, entonces $\langle a, b\rangle \subset G$. Además, como $b\notin \langle a\rangle$, tenemos que $|\langle a, b\rangle|>|\langle a\rangle|=4$. Como \(|G|=8\), tenemos que \(|\langle a, b\rangle| = 8\), luego \(G=\langle a, b\rangle\).
    \end{itemize}

    Por tanto, tenemos que $G=\langle a, b\rangle$ con \(\ord(a) = 4\) y \(b\notin \langle a\rangle\).
    \begin{itemize}
        \item Como $\langle a\rangle\lhd G$ por ser $[G:\langle a\rangle]=2$, tenemos que \(ba b^{-1}\in \langle a\rangle\). Además, como el orden se mantiene invariante por conjugación, tenemos que:
        \begin{equation*}
            \ord(ba b^{-1}) = \ord(a) = 4.
        \end{equation*}

        Por tanto, \(ba b^{-1}\in \{a, a^3\}\). Veamos qué ocurre en cada caso:
        \begin{itemize}
            \item Si \(bab^{-1} = a\), entonces \(ab=ba\), y como $G=\langle a, b\rangle$, tenemos que \(G\) es abeliano, lo cual contradice el supuesto inicial.
        \end{itemize}
        Por tanto, tenemos que $bab^{-1}=a^3=a^{-1}$.
    \end{itemize}

    Sea ahora $b^2$, y veamos si $b^2\in Hb$:
    \begin{equation*}
        b^2\in Hb
        \Longrightarrow b^2 = a^ib \Longrightarrow b=a^i \qquad i\in \{0, 1, 2, 3\}.
    \end{equation*}
    En cualquiera de los casos, llegamos a $b\in H$, lo cual contradice que \(b\in G\setminus H\). Por tanto, tenemos que \(b^2\notin Hb\), luego \(b^2\in H\). Estudiemos en función del valor de \(b^2\):
    \begin{itemize}
        \item Si $b^2=a$, entonces $O(b)=2\cdot O(a)=8$, lo que es una contradicción.
        \item Si $b^2=a^3$, entonces $O(b)=2\cdot O(a^3)=2\cdot 4=8$, lo que es una contradicción.
        \item Supongamos $b^2=1$. En este caso:
        \begin{equation*}
            a^4 = 1\qquad
            b^2 = 1\qquad
            bab^{-1} = a^{-1}.
        \end{equation*}

        Por tanto, por el Teorema de Dyck, tenemos que \(G\cong D_4\).
        \item Si $b^2=a^2$, entonces:
        \begin{equation*}
            a^4 = 1\qquad
            b^2 = a^2\qquad
            bab^{-1} = a^{-1}.
        \end{equation*}
        Por tanto, por el Teorema de Dyck, tenemos que \(G\cong Q_2\).
    \end{itemize}

    Por tanto, los grupos de orden \(8\) no abelianos son:
    \begin{gather*}
        D_4 \qquad Q_2.
    \end{gather*}

    Resumiendo, los grupos de orden \(8\) son:
    \begin{gather*}
        C_8 \qquad C_4\oplus C_2 \qquad C_2\oplus C_2\oplus C_2 \qquad D_4 \qquad Q_2.
    \end{gather*}
\end{ejercicio}

\newpage
\begin{ejercicio}
    Prueba que todo grupo de orden \(12\) es un producto semidirecto y clasifica, salvo isomorfismo, todos los grupos de orden \(12\) identificándolos con productos semidirectos.\\

    Sea \(G\) un grupo de orden \(12=2^2\cdot 3\). Sea $n_2$ el número de \(2-\)subgrupos de Sylow de \(G\) y $n_3$ el número de \(3-\)subgrupos de Sylow de \(G\). Por el Segundo Teorema de Sylow, tenemos que:
    \begin{equation*}
        n_2\mid 3 \quad \text{y} \quad n_2 \equiv 1 \mod 2.
    \end{equation*}
    Por tanto, tenemos que \(n_2\in \{1, 3\}\). En cualquier caso, cada \(2-\)subgrupo de Sylow de \(G\) tiene tres elementos de orden \(2\) o \(4\).\\

    De igual forma, tenemos que:
    \begin{equation*}
        n_3\mid 4 \quad \text{y} \quad n_3 \equiv 1 \mod 3.
    \end{equation*}
    Por tanto, tenemos que \(n_3\in \{1, 4\}\). En cualquier caso, cada \(3-\)subgrupo de Sylow de \(G\) tiene dos elementos de orden \(3\).\\

    Veamos que el siguiente caso no puede darse:
    \begin{itemize}
        \item \ul{$n_3=4,\ n_2=3$}:
        
        Como $n_3=4$, tenemos que \(G\) tiene cuatro \(3-\)subgrupos de Sylow, cada uno de los cuales tiene dos elementos de orden \(3\). Además, tienen intersección trivial, puesto que si tuviesen algún elemento de orden \(3\) en común, entonces sería un generador de ambos \(3-\)subgrupos de Sylow, luego serían iguales. Por tanto, hay \(4\cdot 2=8\) elementos de orden \(3\) en \(G\). Los 4 elementos de orden \(1,2\) y \(4\) restantes deben formar un \(2-\)subgrupo de Sylow de \(G\), que sería único y por tanto \(n_2=1\). No obstante, esto contradice que \(n_2=3\). Por tanto, este caso no puede ocurrir.
    \end{itemize}

    Por tanto, dado $P_2\in \Syl_2(G)$ y $P_3\in \Syl_3(G)$, uno de ellos será único y, por tanto, normal en \(G\). Como $|P_2|=4$ y $|P_3|=3$, razonando por órdenes vemos que $P_2\cap P_2=\{1\}$ y, por el Segundo Teoría de Isomorfía (puesto que uno de los subgrupos es normal), tenemos que $P_2P_3<G$ con:
    \begin{equation*}
        |P_2P_3| = \frac{|P_2|\cdot |P_3|}{|P_2\cap P_3|} = \frac{4\cdot 3}{1} = 12.
    \end{equation*}
    Por tanto, \(P_2P_3=G\) y, entonces, $G\cong P_i\rtimes P_j$ (donde $P_i$ es el grupo normal) es un producto semidirecto.\\

    Para la clasificación, distinguimos en función de los valores de \(n_2\) y \(n_3\):
    \begin{enumerate}
        \item \ul{$n_2=1,\ n_3=1$}:
        
        En este caso, tenemos que \(P_2\) y \(P_3\) son únicos, luego son normales en \(G\). Por tanto, tenemos que:
        \begin{equation*}
            G\cong P_2\times P_3
        \end{equation*}

        Como $|P_2|=4=2^2$, es abeliano, y como $P_3\cong C_3$, también es abeliano. Por tanto, \(G\) es abeliano. Por el Teorema de Estructura de los Grupos Abelianos Finitos, tenemos que:
        \begin{equation*}
            G\cong C_4\oplus C_3\cong C_{12}
            \quad \lor \quad
            G\cong C_2\oplus C_2\oplus C_3 \cong C_2\oplus C_6.
        \end{equation*}

        \item \ul{$n_3=1,\ n_2=3$}:
        
        En este caso, tenemos que \(P_3\lhd G\) y \(P_2\) no es normal en \(G\). Como $G\cong P_3\rtimes P_2$, buscamos los siguientes homomorfismos:
        \Func{\theta}{P_2}{\Aut(P_3)}{p}{\theta(p)}

        Como $P_3\cong C_3$, $\exists x\in P_3$ tal que:
        \begin{equation*}
            P_3 = \langle x\mid x^3=1\rangle.
        \end{equation*}

        Veamos en primer lugar cuántos elementos tiene \(\Aut(P_3)\):
        \begin{equation*}
            |\Aut(P_3)| = \varphi(3) = 2.
        \end{equation*}

        Por tanto, tan solo tiene el isomorfismo identidad, y el isomorfismo $x\mapsto x^{-1}$, que llamaremos $\alpha$. Como $|P_2|=4$, hay dos opciones:
        \begin{itemize}
            \item $P_2\cong C_4$:
            

            Por tanto, $\exists y\in P_2$ tal que:
            \begin{equation*}
                P_2 = \langle y\mid y^4=1\rangle.
            \end{equation*}
            
            Como hemos visto antes, tan solo hay dos homomorfismos posibles:
            \begin{itemize}
                \item Si \(\theta(y) = Id_{P_3}\), entonces $G\cong P_3\times P_2$, luego \(G\) es abeliano y por tanto $n_2=1$, lo cual contradice el supuesto inicial.
            \end{itemize}

            Por tanto, tenemos que:
            \begin{align*}
                yxy^{-1} &= \theta(y)(x) = x^{-1}\\
                x^3 &= 1\\
                y^4 &= 1
            \end{align*}

            Como $G=\langle x,y\rangle$, por el Teorema de Dyck, tenemos que:
            \begin{equation*}
                G\cong \langle x, y\mid x^3=1,\ y^4=1,\ yxy^{-1}=x^{-1}\rangle = Q_3
            \end{equation*}

            \item $P_2\cong C_2\oplus C_2$:
            
            En este caso, $P_2$ tiene dos generadores, sean $y,z\in P_2$ tales que:
            \begin{equation*}
                P_2 = \langle y, z\mid y^2=1,\ z^2=1\rangle.
            \end{equation*}

            Aunque hay tres homomorfismos no triviales, todos ellos dan a grupos isomorfos entre sí. Uno de ellos es:
            \begin{align*}
                \theta(y) &= Id_{P_3}\\
                \theta(z) &= \alpha.
            \end{align*}

            Por tanto, tenemos que:
            \begin{align*}
                yxy^{-1} &= \theta(y)(x) = x\\
                zxz^{-1} &= \theta(z)(x) = x^{-1}\\
                zyz^{-1} &= \theta(z)(y) = y^{-1}\\
                y^2 &= 1\\
                z^2 &= 1
            \end{align*}

            Tenemos que $G=\langle xy,yz\rangle$, y además:
            \begin{align*}
                (xy)^6 &= x^6y^6 = 1\\
                (yz)^2 &= y^2z^2 = 1\\
                (xy)(yz) &= xz\\
                (yz)(xy)^{-1} &= yzyx^2=zx^2
            \end{align*}
            Como $zx^2=xz$, tenemos que cumple las relaciones de $D_6$. Como $|G|=|D_6|=12$, por el Teorema de Dyck, tenemos que:
            \begin{equation*}
                G\cong D_6.
            \end{equation*}
        \end{itemize}

        \item \ul{$n_2=1,\ n_3=4$}:
        
        En este caso, tenemos que \(P_2\lhd G\) y \(P_3\) no es normal en \(G\). Como $G\cong P_2\rtimes P_3$, buscamos los siguientes homomorfismos:
        \Func{\theta}{P_3}{\Aut(P_2)}{p}{\theta(p)}

        En este caso, como $P_2\cong C_4$, o $P_2\cong C_2\oplus C_2$, hay dos opciones:
        \begin{enumerate}
            \item $G\cong C_4\rtimes P_3$:
            
            El único automorfismo no trivial de \(C_4\) es el que lleva \(x\mapsto x^{-1}\), cuyo orden es \(2\). Por tanto, no puede ser la imagen de ningún elemento de \(P_3\), luego no hay automorfismos no triviales. Este caso no se contempla por tanto.

            \item $G\cong C_2\oplus C_2\rtimes P_3$:
            
            Como $|G|=12$ y $n_3\geq 1$, sabemos que $G\cong A_4$.
        \end{enumerate}
    \end{enumerate}

    Resumiendo, los grupos de orden \(12\) son:
    \begin{gather*}
        C_{12} \qquad C_2\oplus C_6 \qquad D_6 \qquad A_4\qquad Q_3
    \end{gather*}
\end{ejercicio}