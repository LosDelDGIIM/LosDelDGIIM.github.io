\chapter{Espacio Proyectivo}

En el presente tema, vamos a estudiar el espacio proyectivo, que es de gran utilidad para trabajar con perspectiva.

En lo que sigue, $V(\bb{R})$ es un espacio vectorial real de dimensión $\dim V = n+1$, $n\in \bb{N}$, y consideramos $V^\ast = V\setminus \left\{\vec{0}\right\}$.

\begin{definicion}[Espacio Proyectivo]
    En $V^\ast$ definimos la siguiente relación de equivalencia:
    \begin{equation*}
        v_1 \sim v_2 \Longleftrightarrow \exists \lm \in \bb{R}^\ast \mid v_1=\lm v_2
    \end{equation*}
    
    Definimos el espacio proyectivo de dimensión $n$, notado por $P^n(V)$, como el espacio vectorial cociente mediante la relación de equivalencia $\sim$ descrita:
    \begin{equation*}
        P^n(V) = V^\ast/\sim = \{[v]\mid v\in V^\ast\}
    \end{equation*}

    Consideramos además la proyección al cociente, denotada por $\pi$:
    \Func{\pi}{V^\ast}{P^n(V)}{v}{[v]}
\end{definicion}

Veamos qué puntos forman la clase de equivalencia de $v\in V^\ast$:
\begin{equation*}
    [v] = \{v_0\in V^\ast \mid v\sim v_0\} = \{\lm v \mid \lm \in \bb{R}^\ast\} = \left(\cc{L}\{v\}\right)^\ast
\end{equation*}
Por esta razón, los puntos $[v]$ de $P(V)$ son referidos como rectas vectoriales o direcciones de $V$.


\begin{notacion}
    En el caso de $V=\bb{R}^{n+1}$, el espacio $P(\bb{R}^{n+1})$ se denota por $\bb{P}^n$.
\end{notacion}
\begin{observacion}
    En la asignatura de Topología I, se ha visto que $\bb{P}^n \cong \bb{S}^n/\cc{R}$, donde $\cc{R}$ es la relación de equivalencia que identifica los antípodas de $\bb{S}^n$:
    \begin{equation*}
        x \cc{R} y \Longleftrightarrow y = \pm x
    \end{equation*}

    Aunque en Geometría no se van a tratar homeomorfismos, si el lector conoce dicho concepto es de utilidad entender que $\bb{P}^n$ es homeomorfo a $\bb{S}^n/\cc{R}$.
\end{observacion}


\section{Subespacios Proyectivos}

En esta sección, dado $X\subset P(V)$, notaremos por $\wt{X}$ al conjunto $$\wt{X}:=\pi^{-1}(X)\cup \left\{\vec{0}\right\}.$$

\begin{definicion}[Subespacio Proyectivo]
    Sea el espacio proyectivo $P(V)$ y un subconjunto $X\subset P(V)$. Diremos que $X$ es un subespacio proyectivo (o una variedad proyectiva) si el conjunto 
    $\wt{X}$ es un subespacio vectorial.

    La dimensión de $X$ se define por:
    $$\dim X = \dim \wt{X}-1.$$
\end{definicion}

\begin{observacion}
    Por convenio, $\dim \emptyset = -1$.
\end{observacion}

Al igual que ocurría con la geometría afín o vectorial, los subespacios proyectivos de dimensión $0,1,2$ y $n-1$ tienen un nombre especial. Sea $X\subset P(V)$ un subespacio proyectivo, tenemos que:
\begin{itemize}
    \item $\dim X=0$: Punto proyectivo.
    \item $\dim X=1$: Recta proyectiva.
    \item $\dim X=2$: Plano proyectivo.
    \item $\dim X=\dim P(V)-1$: Hiperplano proyectivo.
\end{itemize}

Veamos la dimensión de $P(V)$. Puesto que $\pi^{-1}(P(V))\cup \left\{\vec{0}\right\} = V$, tenemos que:
\begin{equation*}
    \dim P(V) = \dim V - 1
\end{equation*}

Como la aplicación $\pi$ es sobreyectiva, tenemos que $\pi(\wt{X}^\ast)=\pi(\pi^{-1}(X))=X$ para cualquier $X\subset P(V)$.
Por tanto, cualquier subespacio proyectivo $X$ es de la forma $X=\pi(\wt{X}^\ast)$ para cierto $\wt{X}^\ast\subset V^\ast$. 
\begin{prop}
    Sea $U\subset V$ un subespacio vectorial de $V$ de dimensión $\dim U \geq 1$. Entonces, $\pi(U^\ast)$ es un subespacio proyectivo de $P(V)$.
\end{prop}
\begin{proof}
    Probaremos que $\pi^{-1}(\pi(U^\ast))\cup \left\{\vec{0}\right\}=U$.
    \begin{description}
        \item[$\supset)$] De forma directa, tenemos que $U^\ast \subset \pi^{-1}(\pi(U^\ast))$, por lo que:
        \begin{equation*}
            U = U^\ast \cup \left\{\vec{0}\right\} \subset \pi^{-1}(\pi(U^\ast))\cup \left\{\vec{0}\right\}
        \end{equation*}

        \item[$\subset)$] Sea $v\in \pi^{-1}(\pi(U^\ast))\cup \left\{\vec{0}\right\}$.
        Si $v\neq \vec{0}$, entonces $[v]=[w]$ para cierto $w\in U^\ast$. Por tanto, $v=\lm w$ para cierto $\lm \in \bb{R}^\ast$. Por tanto, $v\in U^\ast$.

        Por tanto, $\pi^{-1}(\pi(U^\ast))\cup \left\{\vec{0}\right\} \subset U$.
    \end{description}
\end{proof}
\begin{observacion}
    En esta sección, hemos visto que cada subespacio proyectivo $X$ tiene asociado un único
    subespacio vectorial $\wt{X}$ tal que $X=\pi(\wt{X}^\ast)$. Por tanto,
    de ahora en adelante será habitual referirse a $\wt{X}$ como el subespacio vectorial asociado a $X$, y
    determinar $\wt{X}$ será equivalente a determinar $X$.
\end{observacion}

\subsubsection{Operaciones entre espacios proyectivos}

En esta sección vamos a definir, al igual que en el caso vectorial y afín, las operaciones de intersección y suma de subespacios proyectivos.
\begin{prop}[Intersección]
    Sea $P^n(V)$ un espacio proyectivo, y sean $X,Y\subset P(V)$ dos subespacios proyectivos. Entonces, $X\cap Y$ es vacío, o bien es un subespacio proyectivo de $P(V)$ con:
    \begin{equation*}
        \wt{X\cap Y} = \wt{X} \cap \wt{Y}
    \end{equation*}
\end{prop}
\begin{proof}
    Tenemos que:
    \begin{equation*}
        \wt{X\cap Y} = \pi^{-1}(X\cap Y) \cup \left\{\vec{0}\right\} = \left(\pi^{-1}(X)\cap \pi^{-1}(Y)\right) \cup \left\{\vec{0}\right\} = \wt{X} \cap \wt{Y}
    \end{equation*}
\end{proof}

\begin{definicion}
    Sea $C\subset P(V)$. Llamamos subespacio proyectivo generado por $C$, notado por $\langle C \rangle$, al menor subespacio proyectivo que lo contenga. Es decir,
    \begin{equation*}
        \langle C \rangle = \bigcap \{X\supset C\mid X\text{ es un subespacio proyectivo}\}
    \end{equation*}
\end{definicion}

\begin{prop}
    Sea $C\subset P(V)$ un subconjunto no vacío de $P(V)$. Entonces:
    \begin{equation*}
        \wt{\langle C \rangle} = \cc{L}\left\{\pi^{-1}(C)\right\}
    \end{equation*}
\end{prop}
\begin{proof}
    Probaremos que $\langle C \rangle = \pi\left(\cc{L}\left\{\pi^{-1}(C)\right\}^\ast\right)$:
    \begin{description}
        \item[$\subset)$] Como $\pi^{-1}(C)\subset \cc{L}\left\{\pi^{-1}(C)\right\}$, tenemos que
        $C\AstIg \pi(\pi^{-1}(C))\subset \pi\left(\cc{L}\left\{\pi^{-1}(C)\right\}^\ast\right)$, donde en $(\ast)$ se ha empleado que $\pi$ es sobreyectiva.
        Por tanto, como $\langle C \rangle$ es el menor subespacio proyectivo que contiene a $C$, tenemos que $$\langle C \rangle \subset \pi\left(\cc{L}\left\{\pi^{-1}(C)\right\}^\ast\right)$$

        \item[$\supset)$] Sea $X$ un subespacio proyectivo que contiene a $C$,
        es decir, $C\subset X$. Entonces, $\pi^{-1}(C)\subset \pi^{-1}(X)\subset \wt{X}$, por lo que
        $\cc{L}\left\{\pi^{-1}(C)\right\}\subset \wt{X}$, ya que $\wt{X}$ es un subespacio vectorial y contiene a $\pi^{-1}(C)$.
        Por tanto, $\pi\left(\cc{L}\left\{\pi^{-1}(C)\right\}^\ast\right)\subset \pi(\wt{X}^\ast)=X$,
        y como $X$ es un subespacio proyectivo arbitrario que contiene a $C$, tenemos que:
        \begin{equation*}
            \pi\left(\cc{L}\left\{\pi^{-1}(C)\right\}^\ast\right) \subset \langle C \rangle
        \end{equation*}
    \end{description}
\end{proof}

Definimos entonces la suma de subespacios proyectivos como sigue:
\begin{definicion}[Suma]
    Sean $X,Y\subset P(V)$ dos subespacios proyectivos. Definimos la suma de $X$ e $Y$ como:
    \begin{equation*}
        X+Y = \langle X\cup Y\rangle
    \end{equation*}
\end{definicion}

\begin{prop}
    Sean $X,Y\subset P(V)$ dos subespacios proyectivos. Entonces:
    \begin{equation*}
        \wt{X+Y} = \wt{X} + \wt{Y}
    \end{equation*}
\end{prop}
\begin{proof}
    Tenemos que:
    \begin{equation*}
        X+Y = \langle X\cup Y\rangle = \bigcap \{Z\supset (X\cup Y)\mid Z\text{ es un subespacio proyectivo}\}
    \end{equation*}

    Por tanto, $\wt{X+Y} = \bigcap \{\wt{Z}\supset (\wt{X}\cup \wt{Y})\mid \wt{Z}\text{ es un subespacio vectorial}\}$. Por tanto, $\wt{X+Y} = \wt{X} + \wt{Y}$.
    Otra forma de verlo es como sigue:
    \begin{equation*}
        \wt{X+Y} = \cc{L}\left\{\pi^{-1}(X\cup Y)\right\} = \cc{L}\left\{\pi^{-1}(X) \cup \pi^{-1}(Y)\right\} = \wt{X} + \wt{Y}
    \end{equation*}
\end{proof}

Una vez definida la suma y la intersección, tenemos la siguiente fórmula de las dimensiones, análoga al caso vectorial:
\begin{prop}[Fórmula de las dimensiones]
    Sean $X,Y\subset P(V)$ dos subespacios proyectivos. Entonces:
    \begin{equation*}
        \dim (X+Y) + \dim (X\cap Y) = \dim X + \dim Y
    \end{equation*}

    Recordamos de que si $X\cap Y = \emptyset$, entonces $\dim (X\cap Y) = -1$.
\end{prop}
\begin{proof}
    Usando la fórmula de las dimensiones vectoriales, tenemos:
    \begin{equation*}
        \begin{split}
            \dim (X+Y) +& \dim (X\cap Y) =\\
            &= \dim (\wt{X+Y}) -1 + \dim (\wt{X\cap Y}) -1 =\\
            &= \dim (\wt{X} + \wt{Y}) -1 + \dim (\wt{X} \cap \wt{Y}) -1 =\\
            &= \dim \wt{X} + \dim \wt{Y}  - \cancel{\dim (\wt{X} \cap \wt{Y})} -1 + \cancel{\dim (\wt{X} \cap \wt{Y})} -1  =\\
            &= \dim X + \dim Y
        \end{split}
    \end{equation*}
\end{proof}

El siguiente resultado nos es útil para determinar cuándo dos subespacios proyectivos son iguales:
\begin{prop}
    Sean $X,Y\subset P(V)$ dos subespacios proyectivos. Si $X\subset Y$, entonces $\dim X \leq \dim Y$.

    Además, si $\dim X = \dim Y$, entonces $X=Y$.
\end{prop}
\begin{proof}
    Sea $X\subset Y$. Entonces, $\pi^{-1}(X)\subset \pi^{-1}(Y)$, por lo que, $\wt{X} \subset \wt{Y}$. Por tanto, $\dim \wt{X} \leq \dim \wt{Y}$, por lo que:
    \begin{equation*}
        \dim X = \dim \wt{X} - 1 \leq \dim \wt{Y} - 1 = \dim Y
    \end{equation*}

    Si $\dim X = \dim Y$, entonces $\dim \wt{X} = \dim \wt{Y}$. Por tanto, $\wt{X}=\wt{Y}$, por lo que $X=Y$.
\end{proof}

\begin{ejemplo}
    Sean dos rectas proyectivas $X,Y\subset P^2({V})$ en un plano proyectivo. Entonces:
    \begin{equation*}
        \dim (X+Y) + \dim (X\cap Y) = \dim X + \dim Y = 1+1 = 2
    \end{equation*}

    Además, como $X+Y\subset P^2({V})$, tenemos que $\dim (X+Y) \leq 2$. Por tanto:
    \begin{equation*}
        \dim (X\cap Y) = 2 - \dim (X+Y) \geq 2 - 2 = 0
    \end{equation*}

    Por tanto, como $\dim (X\cap Y) \geq 0$, tenemos que $X\cap Y\neq \emptyset$. Es decir, \emph{dos rectas proyectivas en un plano proyectivo siempre se cortan}.
    Vemos que la geometría proyectiva es un ejemplo de geometría no-euclídea.
\end{ejemplo}



\section{Coordenadas Homogéneas}

Sea $\cc{B}$ una base de $V^{n+1}$, y consideramos $v \in V^\ast$. Sea $v=(x_0,\dots,x_n)_{\cc{B}}$.

Definimos las coordenadas homogéneas de $[v]\in P(V)$ en la base $\cc{B}$ de $V$ como las coordenadas de dicho $v$ en $\cc{B}$:
\begin{equation*}
    [v] \equiv (x_0:\dots :x_n)_{\cc{B}}
\end{equation*}

Notemos que, como $[v]=[\lm v],~\lm \in \bb{R}^\ast$, dichas coordenada son únicas \ul{salvo factores} \ul{de proporcionalidad},
por lo que:
\begin{equation*}
    [v] = \{\lm v \mid \lm \in \bb{R}^\ast\} \equiv \{(\lm x_0:\dots :\lm x_n)_{\cc{B}} \mid \lm \in \bb{R}^\ast\}
\end{equation*}
Además, como $v\neq \vec{0}$, tenemos que las coordenadas homogéneas no son todas nulas.

\begin{definicion}
    Llamaremos coordenadas homogéneas usuales (o canónicas) de un punto $[v]\in \bb{P}^n$ a las coordenadas de $v$ en la base usual $\cc{B}_u$ de $\bb{R}^{n+1}$.
\end{definicion}


Una vez se han definido las coordenadas homogéneas, se pueden entonces definir las ecuaciones implícitas y paramétricas de un subespacio proyectivo:
\begin{definicion}[Ecuaciones Paramétricas e Implícitas]
    Sea $X\subset P(V)$ un subespacio proyectivo y $\cc{B}$ una base de $V$.
    Entonces, las ecuaciones parmétricas (respectivamente implícitas) de $X$ en la base $\cc{B}$ son las ecuaciones parmétricas (respectivamente implícitas) que definen al subespacio vectorial $\wt{X}$ en la base $\cc{B}$.
\end{definicion}
\begin{ejemplo}
    Sea $p=(1:2:3),~q=(0:1:2)\in \bb{P}^2$. Calcular la recta proyectiva que une ambos puntos.

    Tenemos que $p=[(1,2,3)]$, $q=[(0,1,2)]$. Entonces:
    \begin{equation*}
        p+q = \pi((\cc{L}\{(1,2,3), (0,1,2)\})^\ast)
    \end{equation*}

    Por tanto, tenemos que:
    \begin{equation*}
        (x:y:z)\in p+q \Longleftrightarrow (x,y,z)=\alpha(1,2,3) + \beta(0,1,2)
    \end{equation*}

    Es decir, las ecuaciones paramétricas de $p+q$ son:
    \begin{equation*}
        \begin{cases}
            x = \alpha\\
            y = 2\alpha + \beta\\
            z = 3\alpha + 2\beta
        \end{cases} \hspace{1cm} \alpha,\beta \in \bb{R}
    \end{equation*}

    La ecuación implícita es:
    \begin{equation*}
        \left|\begin{array}{ccc}
            1 & 0 & x \\
            2 & 1 & y\\
            3 & 2 & z
        \end{array}\right| = 0 = x+2y-z
    \end{equation*}
\end{ejemplo}


\section{Proyectividades}

En esta sección, vamos a estudiar un tipo de aplicaciones entre espacios proyectivos, llamadas proyectividades, que son análogas a las aplicaciones lineales en el caso vectorial o a las afinidades en el caso afín.
En lo que sigue, sean $V_1,V_2$ dos espacios vectoriales reales, y consideramos $P(V_1),P(V_2)$ los espacios proyectivos asociados.
Notamos por $\pi_1,\pi_2$ a las proyecciones al cociente de $V_1^\ast$ y $V_2^\ast$ respectivamente.

\begin{definicion}[Proyectividad]
    Diremos que una aplicación $f:P(V_1)\to P(V_2)$ es una proyectividad si y solo si existe $\wt{f}:V_1\to V_2$ lineal e inyectiva tal que
    \begin{equation*}
        f([v])=\left[\wt{f}(v)\right]
    \end{equation*}

    Diremos que $\wt{f}$ es la lineal asociada a $f$.

    \begin{figure}[H]
        \centering
        \shorthandoff{"}
        \begin{tikzcd}
            V_1^\ast \arrow[r, "\wt{f}"] \arrow[d, "\pi_1"] & V_2^\ast \arrow[d, "\pi_2"] \\
            P(V_1) \arrow[r, "f"]             & P(V_2)       
        \end{tikzcd}
        \shorthandon{"}
    \end{figure}
\end{definicion}
Notemos que se pide que sea inyectiva para poder proyectar, ya que para $v\neq \vec{0}$, entonces $\wt{f}(v)\neq \vec{0}$.
Además, como el proyectivo es el conjunto de rectas vectoriales, necesitamos que $\wt{f}$ lleve rectas vectoriales en rectas vectoriales,
por lo que $\wt{f}$ debe ser inyectiva.

Además, la lineal asociada a una proyectividad $f$ es única salvo factores de proporcionalidad, como se ve en el siguiente resultado:
\begin{prop}
    Sea $f:P(V_1)\to P(V_2)$ una proyectividad con lineal asociada $\wt{f}$.
    Entonces, si $\wt{g}$ es otra lineal asociada a $f$, entonces $\wt{g}=\lm \wt{f}$ para cierto $\lm \in \bb{R}^\ast$.
\end{prop}
\begin{proof}
    Sea $v\in V^\ast$. Entonces, por ser $\wt{f}$ una lineal asociada a $f$, tenemos que $f([v])=\left[\wt{f}(v)\right]$.
    Por ser $\wt{g}$ otra lineal asociada a $f$, tenemos que $f([v])=[\wt{g}(v)]$. Por tanto,
    \begin{equation*}
        \left[\wt{f}(v)\right]=[\wt{g}(v)] \Longrightarrow \exists \lm_v \in \bb{R}^\ast \mid \wt{f}(v)=\lm_v \wt{g}(v)
    \end{equation*}

    Escogemos ahora $w\in V^\ast$ linealmente independiente de $v$. Entonces,
    \begin{equation*}
        \begin{split}
            \wt{f}(v+w) &= \wt{f}(v) + \wt{f}(w) = \lm_v \wt{g}(v) + \lm _w \wt{g}(w)\\
            &= \lm_{v+w} \wt{g}(v+w) = \lm_{v+w} (\wt{g}(v) + \wt{g}(w))
        \end{split}
    \end{equation*}

    Por ser $\wt{g}$ lineal tenemos que, como $\{v,w\}$ son linealmente independientes, sus imágenes $\{\wt{g}(v), \wt{g}(v)\}$ son linealmente independientes. Por tanto, $\lm_{v+w} = \lm_v = \lm_w$.
    Repitiendo este proceso para una base de $V$, tenemos que $\exists \lm \in \bb{R}^\ast \mid \wt{f}=\lm \wt{g}$.
\end{proof}

Veamos nuestro primer ejemplo de proyectividad:
\begin{ejemplo}
    Sea $P(V)$ un espacio proyectivo. La identidad $Id:P(V)\to P(V)$ es una proyectividad, con lineal asociada $Id:V\to V$.
    Veámoslo:
    \begin{equation*}
        Id([v]) = [v] = \left[Id(v)\right] \qquad \forall v\in V^\ast
    \end{equation*}
\end{ejemplo}


\begin{prop}
    Sean $V_1,V_2$ dos espacios vectoriales reales, y consideramos una proyectividad $f:P(V_1)\to P(V_2)$. Entonces, $f$ es inyectiva.
\end{prop}
\begin{proof}
    Sea $[v_1],[v_2]\in P(V_1)$ tales que $f([v_1])=f([v_2])$. Entonces, por definición de proyectividad, $\left[\wt{f}(v_1)\right]=\left[\wt{f}(v_2)\right]$, por lo que $\wt{f}(v_1)=\lm \wt{f}(v_2)$ para cierto $\lm \in \bb{R}^\ast$.
    Como $\wt{f}$ es lineal, tenemos que $\wt{f}(v_1) = \wt{f}(\lm v_2)$, y como $\wt{f}$ es inyectiva, tenemos que $v_1 = \lm v_2$. Por tanto, $[v_1]=[v_2]$, demostrando así que $f$ es inyectiva.
\end{proof}




La siguiente proposición afirma que las proyectividades conservan la dimensión:
\begin{prop}
    Sea $f:P(V_1)\to P(V_2)$ una proyectividad, y sea $X\subset P(V_1)$ un subespacio proyectivo. Entonces, $f(X)$ es un subespacio proyectivo de $P(V_2)$ con:
    \begin{equation*}
        \wt{f}\left(\wt{X}\right) = \wt{f(X)}
    \end{equation*}

    Además, $\dim f(X) = \dim X$.
\end{prop}
\begin{proof}
    Tenemos que: $$f(X)=\{f([v]) \mid [v]\in X\} = \left\{\left[\wt{f}(v)\right] \mid v\in \wt{X}^\ast\right\} = \pi_2\left(\wt{f}\left(\wt{X}^\ast\right)\right) = \pi_2\left(\left(\wt{f}\left(\wt{X}\right)\right)^\ast\right)$$

    Como $\wt{X}$ es un subespacio vectorial, tenemos que $\wt{f}\left(\wt{X}\right)$ es un subespacio vectorial, por lo que $f(X)$ es un subespacio proyectivo de $P(V_2)$ con $\wt{f(X)}=\wt{f}\left(\wt{X}\right)$.

    Respecto a las dimensiones, tenemos que: $$\dim f(X) = \dim \wt{f(X)} - 1 = \dim \wt{f}\left(\wt{X}\right) - 1 \AstIg \dim \wt{X} - 1 = \dim X$$
    donde en $(\ast)$ se ha empleado que $\dim \wt{f}\left(\wt{X}\right) = \dim \wt{X}$. Veamos esto último. Como $\wt{f}$ es inyectiva y estamos restringiendo el codominio a la imagen, tenemos que tamién es sobreyectiva. Por tanto, $\dim \wt{f}\left(\wt{X}\right) = \dim \wt{X}$.
\end{proof}
\begin{observacion}
    El contrarrecíproco es también de gran utilidad, ya que si dos lineales
    asociadas a proyectividades no son proporcionales, entonces dichas proyectividades
    no son iguales.
\end{observacion}

Introducimos ahora el concepto de homografía, equivalente al de isomorfismo en el caso vectorial:
\begin{definicion}[Homografía]
    Sea $f:P(V_1)\to P(V_2)$ una proyectividad. Diremos que $f$ es una homografía si y solo si $\wt{f}$ es un isomorfismo.
\end{definicion}

Usando que $\wt{f},f$ son inyectivas, haciendo uso del diagrama de composición de las proyectividades, la siguiente caracterización de las homografías es de inmediata comprobación:
\begin{prop}
    Sean $V_1,V_2$ dos espacios vectoriales reales, y consideramos una proyectividad $f:P(V_1)\to P(V_2)$. Equivalen:
    \begin{enumerate}
        \item $f$ es una homografía.
        \item $f$ es una proyectividad biyectiva.
        \item $\dim P(V_1) = \dim P(V_2)$.
    \end{enumerate}
\end{prop}

Veamos que la composición de proyectividades es una proyectividad:
\begin{prop}
    Sean $V_1,V_2,V_3$ tres espacios vectoriales reales, y consideramos proyectividades
    $f:P(V_1)\to P(V_2)$ y $g:P(V_2)\to P(V_3)$. Entonces, $g\circ f$ es una proyectividad.
\end{prop}
\begin{proof}
    Sea $\wt{f}$ la lineal asociada a $f$, y sea $\wt{g}$ la lineal asociada a $g$.
    Entonces, $\wt{g}\circ \wt{f}$ es lineal. Veamos que $\wt{g}\circ \wt{f}$ es la
    lineal asociada a $g\circ f$:
    \begin{equation*}
        (g\circ f)([v]) = g(f([v])) = g\left(\left[\wt{f}(v)\right]\right)
        = \left[\wt{g}(\wt{f}(v))\right] = \left[(\wt{g}\circ \wt{f})(v)\right] \qquad \forall v\in V_1^\ast
    \end{equation*}
\end{proof}

\subsection{Determinación de una proyectividad}

En esta sección, vamos a ver cómo determinar una proyectividad.
En primer lugar, es evidente que determinar $f$ es equivalente a determinar $\wt{f}$, ya que $f([v])=\left[\wt{f}(v)\right]$. Por tanto, nos centraremos en determinar $\wt{f}$.

\begin{definicion}[Independencia Proyectiva]
    Sea $P(V)$ un espacio proyectivo, y sea $X=\{[v_1],\dots [v_{k+1}]\}\subset P(V)$ un conjunto de $k+1$ puntos proyectivos.
    Diremos que $X$ es un conjunto de puntos proyectivamente independientes si y solo si $\dim \langle X \rangle = k$. Equivalentemente, esto es si:
    \begin{equation*}
        \dim \wt{\langle X \rangle} = k +1 = \dim \cc{L}\{v_1,\dots,v_{k+1}\}
    \end{equation*}
    Esto es equivalente a que $\{v_1,\dots,v_{k+1}\}$ sea un conjunto de vectores linealmente independientes.

    En caso contrario, diremos que $X$ es un conjunto de puntos proyectivamente dependientes o alineados.    
\end{definicion}


Buscamos ahora demostrar el Teorema Fundamental de la Geometría Proyectiva, que nos informa de la unicidad de una proyectividad dada una serie de puntos proyectivos independientes.
Para ello, antes demostraremos el siguiente teorema:
\begin{teo}
    Considerenos $P^n(V)$ y $P^m(V')$ dos espacios proyectivos con dimensiones $\dim P^n(V) = n \leq m = \dim P^m(V')$,
    y sean $\{p_0, \dots, p_n\}\subset P^n(V)$ y $\{p_0', \dots, p_n'\}\subset P^m(V')$ dos conjuntos de puntos proyectivamente independientes.

    Entonces, existe una proyectividad $f:P^n(V)\to P^m(V')$ tal que:
    \begin{equation*}
        f(p_i) = p_i',\qquad i=0,\dots,n
    \end{equation*}

    Si además $n = m$, entonces $f$ es una homografía.
\end{teo}
\begin{proof}
    Como $\pi,\pi'$ son sobreyectivas, sean $v_0,\dots,v_n\in V^\ast$ y $v_0',\dots,v_n'\in V'^\ast$ tales que:
    \begin{equation*}
        \pi(v_i) = p_i,\qquad \pi'(v_i') = p_i',\hspace{1cm} i=0,\dots,n
    \end{equation*}

    Por ser ambos conjuntos de puntos proyectivamente independientes, tenemos que $\{v_0,\dots,v_n\}$ es un conjunto de vectores linealmente independientes de $V$.
    Además, como $\dim V = n+1$, tenemos que $\{v_0,\dots,v_n\}$ es una base de $V$. Por tanto, por el Teorema Fundamental del Álgebra Lineal, $\exists! \wt{f}:V\to V'$ lineal tal que:
    \begin{equation*}
        \wt{f}(v_i) = v_i',\hspace{1cm} i=0,\dots,n
    \end{equation*}

    No obstante, como $\left[\wt{f}(v_i)\right] = [v_i'] = [\lm_i v_i']$, tenemos que, para cada elección
    de $\lm_i \in \bb{R}^\ast$, $i=1, \dots, n$, tenemos que existe una única lineal $\wt{f}$ que cumple que:
    \begin{equation*}
        \wt{f}(v_i) = \lm_i v_i',\hspace{1cm} i=0,\dots,n
    \end{equation*}
    
    Por último, podemos suponer sin pérdida de generalidad que $\lm_0 = 1$, ya que, en caso contrario, dividimos por $\lm_0$ y obtenemos:
    \begin{equation*}
        \wt{f}(v_0) = v_0' \qquad \wt{f}(v_i) = \lm_i v_i',\hspace{1cm} i=1,\dots,n
    \end{equation*}

    Además, como por hipótesis $\{v_0',\dots,v_n'\}$ es un conjunto de vectores linealmente independientes,
    tenemos que $\wt{f}$ es inyectiva, por lo que existe $f$ una proyectividad con lineal asociada $\wt{f}$.

    Si $n=m$, entonces $\dim V = \dim V'$, por lo que $\wt{f}$ es sobreyectiva, por lo que $f$ es una homografía.
\end{proof}

No obstante, esta no tiene por qué ser única, como se ve en el siguiente ejemplo:
\begin{ejemplo}
    Sean $P^1(\bb{R})$ y $P^1(\bb{R})$ dos espacios proyectivos. Sean $[v_0], [v_1] \in P^1(\bb{R})$ dos puntos proyectivamente independientes
    y sean $[v_0'], [v_1'] \in P^1(\bb{R})$ otros dos puntos proyectivamente independientes.

    Sea $f:P^1(\bb{R})\to P^1(\bb{R})$ una proyectividad tal que:
    \begin{equation*}
        f([v_i]) = [v_i'] = \left[\wt{f}(v_i)\right] \qquad i=0,1
    \end{equation*}

    Por tanto, $f$ es una proyectividad. No obstante, veamos que estas dos son
    lineales asociadas a dicha proyectividad:
    \begin{equation*}
        \wt{f}(v_0) = v_0' \qquad \wt{f}(v_1) = v_1' \\
        \wt{g}(v_0) = 2v_0' \qquad \wt{g}(v_1) = 3v_1'
    \end{equation*}

    Veamos que ambas lineales no son proporcionales:
    \begin{equation*}
        \wt{f}(v_0+v_1) = v_0' + v_1' \qquad \wt{g}(v_0+v_1) = 2v_0' + 3v_1'
    \end{equation*}

    Por tanto, $\wt{f}$ y $\wt{g}$ no son proporcionales, por lo que sus proyectividades asociadas no son iguales.
    Por tanto, la proyectividad $f$ no es única.
\end{ejemplo}

Veamos ahora el siguiente teorema de gran importancia, que nos informa también de la unicidad.
No obstante, para ello necesitamos definir el concepto de sistema de referencia proyectivo:
\begin{definicion}[Sistema de Referencia Proyectivo]
    Sea $P(V)$ un espacio proyectivo de dimensión $n$. Dados $n+2$ puntos proyectivos $\{[v_0], \dots, [v_{n+1}]\}\subset P(V)$,
    diremos que es un sistema de referencia proyectivo si y solo si
    todo subconjunto de $n+1$ puntos es proyectivamente independiente.
\end{definicion}

\begin{teo}[Fundamemtal de la Geometría Proyectiva]
    Sean $P^n(V)$ y $P^n(V')$ dos espacios proyectivos $n-$dimensionales. 
    Sean también $\{[v_0], \dots, [v_{n+1}]\}\subset P^n(V)$ y $\{[v_0'], \dots, [v_{n+1}']\}\subset P^n(V')$ dos sistemas de referencia proyectivos.
    Entonces, existe una única homografía $f:P^n(V)\to P^n(V')$ tal que:
    \begin{equation*}
        f([v_i]) = [v_i'],\qquad i=0,\dots,n+1
    \end{equation*}
\end{teo}
\begin{proof}
    Por la demostración del teorema anterior, tenemos que una proyectividad $f$ tal que $f([v_i]) = [v_i']$, $i=0,\dots,n+1$,
    viene asociada a una lineal inyectiva $\wt{f}$ tal que:
    \begin{equation*}
        \wt{f}(v_0) = v_0' \qquad \wt{f}(v_i) = \lm_i v_i',\hspace{1cm} i=1,\dots,n
    \end{equation*}
    donde:
    \begin{itemize}
        \item $\cc{B}= \{v_0,\dots,v_n\}$ es una base de $V$, ya que son $n+1$ vectores, y por ser parte de un sistema de referencia proyectivo, son linealmente independientes.
        \item $\cc{B}'= \{v_0',\dots,v_n'\}$ es una base de $V'$, ya que son $n+1$ vectores, y por ser parte de un sistema de referencia proyectivo, son linealmente independientes.
        \item $\lm_i \in \bb{R}^\ast$, $i=1,\dots,n$.
    \end{itemize}

    Demostrar que existe una única elección de $\lm_i$, $i=1,\dots,n$ equivaldría a demostrar que $f$ es única.
    Para ello, como $\cc{B}$, $\cc{B}'$ son bases de $V$, $V'$ respectivamente, tenemos que:
    \begin{equation*}
        v_{n+1} = \sum_{i=0}^n \mu_i v_i \qquad v_{n+1}' = \sum_{i=0}^n \mu_i' v_i' \qquad \mu_i,\mu_i' \in \bb{R}
    \end{equation*}

    Además, como todo subconjunto de $n+1$ vectores es linealmente independiente por ser sistemas de referencia, tenemos que $\mu_i, \mu_i' \neq 0$, $\forall i=0,\dots,n$,
    ya que en caso contrario podríamos obtener una combinación lineal de $n+1$ vectores que fuera nula, contradiciendo la independencia lineal.
    Por tanto, tenemos que $\mu_i, \mu_i' \in \bb{R}^\ast$, $\forall i=0,\dots,n$. Como $f([v_{n+1}]) = [v_{n+1}'] = [\wt{f}(v_{n+1})]$, tenemos que:
    \begin{equation*}
        \wt{f}\left(\sum_{i=0}^n \mu_i v_i\right) = \mu \left(\sum_{i=0}^n \mu_i' v_i'\right) \qquad \mu \in \bb{R}^\ast
    \end{equation*}

    Además, podemos super sin pérdida de generalidad que $\mu_0 = \mu_0'$, ya que se tiene que $[v_{n+1}'] = \left[\frac{\mu_0}{\mu_0'}v_{n+1}\right]$. Entonces, uniendo el
    resultado anterior junto con la definición de $\wt{f}$, tenemos que:
    \begin{equation*}
        \wt{f}\left(\sum_{i=0}^n \mu_i v_i\right)
        = \sum_{i=0}^n \mu_i \wt{f}(v_i)
        = \mu_0v_0' + \sum_{i=1}^n \mu_i \lm_i v_i'
        = \mu \left(\sum_{i=0}^n \mu_i' v_i'\right) \qquad \mu \in \bb{R}^\ast
    \end{equation*}

    Por tanto, en coodenadas en $\cc{B}'$ tenemos que:
    \begin{equation*}
        \left(\mu_0, \mu_1 \lm_1, \dots, \mu_n \lm_n\right)_{\cc{B}'}
        = \mu \left(\mu_0', \mu_1', \dots, \mu_n'\right)_{\cc{B}'}
    \end{equation*}
    Por la unicidad de las coordenadas en una base vectorial, igualando la primera coordenada tenemos que $\mu = \mu_0/\mu_0'$, y por tanto $\mu=1$.
    Por tanto, usando que $f([v_{n+1}]) = [v_{n+1}'] = [\wt{f}(v_{n+1})]$, tenemos que:
    \begin{equation*}
        \mu_0v_0' + \sum_{i=1}^n \mu_i \lm_i v_i'
        = \sum_{i=0}^n \mu_i' v_i'
    \end{equation*}

    Expresándolo en coordenadas homogéneas respecto de $\cc{B}'$, tenemos que:
    \begin{equation*}
        \left(\mu_0: \mu_1 \lm_1: \dots: \mu_n \lm_n\right)_{\cc{B}'}
        = \left(\mu_0': \mu_1': \dots: \mu_n'\right)_{\cc{B}'}
    \end{equation*}

    Por tanto, los valores de $\lm_i$ quedan fijados a las coordenadas de $v_{n+1}',~v_{n+1}$, teniendo que:
    \begin{equation*}
        \lm_i = \frac{\mu_i'}{\mu_i} \qquad i=1,\dots,n
    \end{equation*}

    Como esta es la única elección posible de $\lm_i$, $i=1,\dots,n$, tenemos que $f$ es única.
\end{proof}

\section{Relación entre Geometría Proyectiva y Geometría Afín}

En esta sección, vamos a ver cómo relacionar la geometría proyectiva con la geometría afín.
Lo estudiaremos en concreto para el espacio proyectivo $\bb{P}^n$ y el espacio afín $\bb{R}^{n+1}$,
aunque el resultado es análogo para cualquier espacio proyectivo y afín de dimensión $n$ y $n+1$ respectivamente.

Sea la aplicación afín inyectiva dada por:
\Func{i}{\bb{R}^n}{\bb{R}^{n+1}}{(x_1,\dots,x_n)}{(1,x_1,\dots,x_n)}
y la aplicación proyección natural $\pi:\bb{R}^{n+1}-\left\{\vec{0}\right\}\to \bb{P}^n$.
Consideramos ahora la aplicación $\varphi = \pi \circ i:\bb{R}^n \to \bb{P}^n$ dada por:
\begin{equation*}
    \varphi(x_1,\dots,x_n) = \pi(1,x_1,\dots,x_n) = [(1,x_1,\dots,x_n)] =
    (1:x_1:\dots:x_n)_{\cc{B}_u}
\end{equation*}
Veamos que $\varphi$ es una aplicación inyectiva. Sean $x,y \in \bb{R}^n$ tales que $\varphi(x)=\varphi(y)$.
Entonces, por definición de $\varphi$, $[(1,x_1,\dots,x_n)] = [(1,y_1,\dots,y_n)]$, por lo que $\exists \lm \in \bb{R}^\ast$ tal que
$(1,x_1,\dots,x_n) = \lm (1,y_1,\dots,y_n)$, y como las primeras coordenadas son $1$, tenemos que $\lm =1$, por lo que $x=y$.
Por tanto, $\varphi$ es inyectiva, e incluye $\bb{R}^n$ en $\bb{P}^n$.

\begin{prop}[Hiperplano del infinito]
    Haciendo uso de la aplicación $\varphi$, podemos definir el siguiente conjunto, notado por $H_{\infty}$:
    \begin{equation*}
        H_{\infty} := \bb{P}^n \setminus \varphi(\bb{R}^n) = \{(y_0:y_1:\dots :y_n) \in \bb{P}^n \mid y_0 = 0\}
    \end{equation*}
    Tenemos que $H_{\infty}$ es un hiperplano proyectivo de $\bb{P}^n$, denominado \ul{hiperplano del infinito}.
\end{prop}
\begin{proof}
    Veamos en primer lugar que dicha igualdad es cierta. Calculemos $\varphi(\bb{R}^n)$:
    \begin{equation*}
        \varphi(\bb{R}^n) = \{(1:y_1:\dots:y_n)_{\cc{B}_u} \mid (y_1,\dots,y_n) \in \bb{R}\}
    \end{equation*}
    Como las coordenadas son únicas salvo factores de proporcionalidad, multiplicando por $y_0 \in \bb{R}^\ast$ tenemos que:
    \begin{align*}
        \varphi(\bb{R}^n) &= \{(y_0:y_1:\dots:y_n)_{\cc{B}_u} \mid (y_0,y_1,\dots,y_n) \in \bb{R}^{n+1}\} =\\
        &= \{(y_0:y_1:\dots:y_n) \in \bb{P}^n \mid y_0 \neq 0\}
    \end{align*}
    Por tanto, la igualdad es cierta, y tenemos que la ecuación implícita de $H_{\infty}$ es $y_0=0$.
    Por tanto, $H_{\infty}$ es un hiperplano proyectivo de $\bb{P}^n$.
\end{proof}

\begin{prop}
    Dado $S\subset \bb{R}^n$ un subespacio afín de dimensión $k\geq 1$,
    se tiene que:
    \begin{equation*}
        \wh{S} := \varphi(S) \cup  \left\{(0:v_1:\dots:v_n)\in \bb{P}^n \mid (v_1,\dots,v_n)\in \vec{S}^\ast\right\}
    \end{equation*}
    es un subespacio proyectivo de $\bb{P}^n$ de dimensión $k$. Además,
    \begin{equation*}
        \wh{S}_\infty := \wh{S}\cap H_{\infty} = \left\{(0:v_1:\dots:v_n)\in \bb{P}^n \mid (v_1,\dots,v_n)\in \vec{S}^\ast\right\}
    \end{equation*}
    es un subespacio proyectivo de $\bb{P}^n$ de dimensión $k-1$.
\end{prop}
% // TODO: Demo de incluir un subespacio afín en el espacio proyectivo

De manera intuitiva, esta proposición nos dice que, dado un subespacio afín $S\subset \bb{R}^n$ de dimensión $k\geq 1$,
el subespacio proyectivo $\wh{S}$ se puede ver como $S$ ($\varphi(S)$ es simplemente una inclusión en el proyectivo) junto con todas las direcciones de $\vec{S}$.
Además, $\wh{S}_\infty$ se puede ver como todas las direcciones de $\vec{S}$. En particular, para $S=\bb{R}^n$,
tenemos que $\bb{P}^n$ se puede ver como $\bb{R}^n$ junto con todas las direcciones de $\bb{R}^n$.

\subsection{Proyectivización de subespacios afines}
%// TODO: Dar formalidad a proyectivización de subespacios afines

En esta sección, veremos cómo, dado un subespacio afín $S\subset \bb{R}^n$ de dimensión $k\geq 1$, obtener el subespacio proyectivo $\wh{S}$.
Sea $S=p+\vec{S}$, con $\cc{B}= \{v_1,\dots,v_k\}$ una base de $\vec{S}$.

Sean las ecuaciones paramétricas de $S$ en la base $\cc{B}$ las siguientes:
\begin{equation*}
    \left(\begin{array}{c}
        x_1 \\ \vdots \\ x_n
    \end{array}\right)
    = \left(\begin{array}{c}
        p_1 \\ \vdots \\ p_n
    \end{array}\right)
    + \lm_1 \left(\begin{array}{c}
        v_{11} \\ \vdots \\ v_{n1}
    \end{array}\right)
    + \dots
    + \lm_k \left(\begin{array}{c}
        v_{1k} \\ \vdots \\ v_{nk}
    \end{array}\right) \hspace{1cm} \lm_1,\dots,\lm_k \in \bb{R}
\end{equation*}
Por tanto, las ecuaciones paramétricas de $\wh{S}$ en la base $\cc{B}$ son:
\begin{equation*}
    \left(\begin{array}{c}
        x_0 \\ x_1 \\ \vdots \\ x_n
    \end{array}\right)
    = \lm_0 \left(\begin{array}{c}
        1 \\ p_1 \\ \vdots \\ p_n
    \end{array}\right)
    + \lm_1 \left(\begin{array}{c}
        0 \\ v_{11} \\ \vdots \\ v_{n1}
    \end{array}\right)
    + \dots
    + \lm_k \left(\begin{array}{c}
        0 \\ v_{1k} \\ \vdots \\ v_{nk}
    \end{array}\right) \hspace{1cm} \lm_1,\dots,\lm_k \in \bb{R}, \lm_0 \in \bb{R}^\ast
\end{equation*}
\vspace{1cm}

De igual forma, si las ecuaciones implícitas de $S$ en la base $\cc{B}$ son:
\begin{equation*}
    \left\{
        \begin{array}{c}
            a_{11}x_1 + \dots + a_{1n}x_n = b_1\\
            \vdots\\
            a_{(n-k)1}x_1 + \dots + a_{(n-k)n}x_n = b_{n-k}
        \end{array}
    \right.
\end{equation*}
entonces, las ecuaciones implícitas de $\wh{S}$ en la base $\cc{B}$ son:
\begin{equation*}
    \left\{
        \begin{array}{c}
            a_{11}x_1 + \dots + a_{1n}x_n = b_1x_0\\
            \vdots\\
            a_{(n-k)1}x_1 + \dots + a_{(n-k)n}x_n = b_{n-k}x_0            
        \end{array}
    \right.
\end{equation*}




\section{Teoremas de Pappus y Desargues Proyectivos}

Ambos teoremas vistos en la Sección \ref{sec:TeoremasPappusDesarguesAfines}
tienen su versión proyectiva, que trataremos en esta sección. Para tratar el Teorema
de Desargues, es necesario antes dar la siguiente definición:
\begin{definicion}[Triángulo Proyectivo]
    Sea $P(V)$ un espacio proyectivo.
    Dados $A,B,C \in {P}(V)$, diremos que $\{A,B,C\}\subset P(V)$ es un triángulo proyectivo si y solo si
    $\{A,B,C\}$ es un conjunto de puntos proyectivamente independientes.
\end{definicion}

Recordamos también que una familia de rectas $\{R_1,\dots,R_n\}\subset P(V)$ se dicen
\emph{concurrentes} si y solo si $\bigcap\limits_{i=1}^n R_i \neq \emptyset$.

El Teorema de Desargues trata sobre triángulos perspectivos desde un punto y desde una recta, por lo que introduzcamos ambos conceptos:
\begin{definicion}
    Sea $P(V)$ un espacio proyectivo.
    Dados dos triángulos proyectivos $T = \{A_1, A_2, A_3\}\subset P(V)$ y $T' = \{A_1', A_2', A_3'\}\subset P(V)$,
    diremos que $T$ y $T'$ son triángulos proyectivamente (o perspectivamente) equivalentes desde un punto $O \in P(V)$ si y solo si:
    \begin{equation*}
        O, A_i, A_i' \text{ están alineados } \forall i=1,2,3
    \end{equation*}
\end{definicion}

\begin{definicion}
    Sea $P(V)$ un espacio proyectivo.
    Dados dos triángulos proyectivos $T = \{A_1, A_2, A_3\}\subset P(V)$ y $T' = \{A_1', A_2', A_3'\}\subset P(V)$,
    diremos que $T$ y $T'$ son triángulos proyectivamente (o perspectivamente) equivalentes desde una recta $R \subset P(V)$ si y solo si:
    \begin{equation*}
        \{A_i + A_j, A_i' + A_j', R\} \text{ son concurrentes } \forall i,j=1,2,3~i\neq j.
    \end{equation*}
\end{definicion}

%En la siguiente figura, se pueden ver dos triángulos proyectivamente equivalentes desde un punto y desde una recta:
% // TODO: Imagen de triángulos proyectivamente equivalentes desde un punto y desde una recta


TERMINAR %// TODO: Teoremas de Pappus y Desargues Proyectivos




\section{Relación de Ejercicios}

Para ver ejercicios relacionados con este tema, consultar la sección \ref{Rel:Tema4}.