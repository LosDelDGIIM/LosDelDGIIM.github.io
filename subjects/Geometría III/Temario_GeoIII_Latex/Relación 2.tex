\section{El Espacio Afín Euclídeo.}\label{Rel:Tema2}

\begin{ejercicio}
    Sean $\cc{A}$ un espacio afín euclídeo y S un subespacio afín suyo. Dado un punto $p \in \cc{A}$ demuestra que existe $q_0 \in S$ tal que
    \begin{equation*}
        d(p, q_0) = d(p, S) := \inf\{d(p, q) : q \in S\}.
    \end{equation*}

    Demostramos en primer lugar que existe $q_0 \in S$ tal que $\vec{pq_0}\in \vec{S}^\perp$.
    Notemos el subespacio $S$ como $S=q+\vec{S}$.
    
    Como se tiene que $\vec{\cc{A}}=\vec{S}\oplus \vec{S}^\perp$, entonces $\exists_1 u\in \vec{S}, v\in \vec{S}^\perp$ tal que:
    \begin{equation*}
        \vec{pq} = u+v
    \end{equation*}

    Sea $q_0=q-u\in S$. Entonces:
    \begin{equation*}
        v = \vec{pq}-u = q-p-u = q_0-p = \vec{pq_0} \in \vec{S}^\perp
    \end{equation*}

    Veamos ahora que $d(p,q_0)\leq d(p,q)~ \forall q\in S$:
    \begin{multline*}
        d^2(p,q) = \langle\vec{pq}, \vec{pq}\rangle
        = \langle\vec{pq_0} + \vec{q_0q}, \vec{pq}+ \vec{q_0q}\rangle
        = \|\vec{pq_0}\|^2 + \|\vec{q_0q}\|^2 + 2\cancel{\langle\vec{pq_0}, \vec{q_0q}\rangle} =\\= \|\vec{pq_0}\|^2 + \|\vec{q_0q}\|^2 \geq \|\vec{pq_0}\|^2 = d^2(p,q_0) \qquad \forall q\in S
    \end{multline*}

    Por tanto, $d(p,q_0)$ es un minorante del conjunto. Como además $q_0\in S$, pertenece al conjunto, por lo que es el ínfimo.

    Por tanto, se tiene.
\end{ejercicio}

\begin{ejercicio}[Hiperplano afín de puntos equidistantes] Dados tres puntos $p, q, r \in~\bb{R}^n$, demostrar que se cumple la igualdad
    \begin{equation*}
        d(p,q)^2 - d(q,r)^2 = 2\langle\vec{rm_{pq}}, \vec{qp}\rangle,
    \end{equation*}
    donde $m_{pq}$ es el punto medio entre $p$ y $q$. Utilizar esta igualdad para probar lo siguiente: si $p \neq q$, entonces el conjunto de los puntos de $\bb{R}^n$ que se encuentran a la misma distancia de $p$ y de $q$ coincide con el hiperplano $m_{pq} + \cc{L}(\{\vec{p q}\})^\perp$.
\end{ejercicio}

\begin{ejercicio}
    Dados los siguientes pares de rectas de $\bb{R}^2$, estudia su posición relativa. Si se cortan, determina el ángulo que forman; en otro caso, calcula la distancia entre ellas.
    \begin{enumerate}
        \item $R=\{(x,y)\in \bb{R}^2 \mid x=y\}$, $S=\{(x,y)\in \bb{R}^2\mid 2x-y=0\}$.

        Tenemos que $R=(0,0)+\cc{L}\{(1,1)\}$, $S=(0,0)+\cc{L}(1,2)$. 

        Por tanto, son secantes. El origen $(0,0)\in S\cap R$ está en la intersección. Veamos el ángulo entre las rectas:
        \begin{equation*}
            \frac{\langle (1,1), (1,2)\rangle}{\|(1,1)\|~\|(1,2)\|} = \frac{1+2}{\sqrt{2}\cdot \sqrt{5}} = \frac{3}{\sqrt{10}}
            \hspace{2cm}
            \frac{\langle (1,1), -(1,2)\rangle}{\|(1,1)\|~\|-(1,2)\|} = -\frac{3}{\sqrt{10}}
        \end{equation*}

        Por tanto, tenemos que el ángulo es:
        \begin{equation*}
            \alpha = \measuredangle (S,R) = \arccos \left(\frac{3}{\sqrt{10}}\right) \approx 0.32~ \text{rad}
        \end{equation*}
        
        \item $R=\{(x,y)\in \bb{R}^2 \mid x-y=1\}$, $S=\{(2\lambda,1+2\lambda) \mid \lambda\in \bb{R}\}$.

        Tenemos que $S=\{(x,y)\in \bb{R}^2\mid y=1+x\}=\{(x,y)\in \bb{R}^2\mid x-y=-1\}$. Por tanto, como las ecuaciones cartesianas de $S\cap R$ forman un SI, tenemos que $S\cap R=\emptyset$. Calculamos la distancia entre ambas rectas.

        Tenemos que $R=\{(\mu, \mu-1)\mid \mu\in \bb{R}\}=(0,-1)+\cc{L}\{(1,1)\}$. Además, $S=(0,1)+\cc{L}\{(1,1)\}$. Por tanto, un vector arbitrario que una ambas rectas es $v=\vec{(\mu,\mu-1)(2\lm, 1+2\lm)}=(2\lm-\mu, 2\lm -\mu + 2)$. Como $v\in \vec{R}^\perp$, entonces:
        \begin{equation*}
            \langle (2\lm-\mu, 2\lm -\mu + 2), (1,1)\rangle = 0
            \Longleftrightarrow 4\lm -2\mu +2 = 0
        \end{equation*}

        Por ejemplo, para $\lm = 0$, $\mu=1$ se tiene que $v\in \vec{R}^\perp = \vec{S}^\perp$, tenemos que:
        \begin{equation*}
            d(R,S) = \|v\| = \|(-1, 1)\| = \sqrt{2}
        \end{equation*}
    \end{enumerate}
\end{ejercicio}

\begin{ejercicio}
    En el espacio vectorial $\cc{P}_2(\bb{R})$, consideramos el producto escalar definido como $\langle p(x), q(x)\rangle = \int_0^1 p(x)q(x)~dx$.
    Dotamos $\cc{P}_2(\bb{R})$ de la estructura afín canónica (que lo convierte en un espacio afín euclídeo) y consideramos las siguientes rectas afines:
    \begin{equation*}
        S = \{p(x) \in \cc{P}_2(\bb{R}) \mid p(0) = 5, p''(8) = 4\}
        \qquad \text{y} \qquad
        T = \langle\{5x^2 - 2x, 2x^2 - x + 1\}\rangle.
    \end{equation*} 
    Comprueba si $S$ y $T$ se cortan en un punto, y en dicho caso calcula el ángulo que forman.\\

    Buscamos obtener las ecuaciones cartesianas de $S$. Tenemos que $p\in \cc{P}_2(\bb{R})$ arbitrario es $p(x)=a_0+a_1x+a_2x^2$, con $a_0,a_1,a_2\in \bb{R}$. Además, $p'(x)=a_1+2a_2x$ y $p''(x)=2a_2$. Por tanto, las condiciones dadas son $a_0=5,~ a_2=2$. Por tanto, las ecuaciones paramétricas de $S$ en el sistema de referencia $\cc{R}=\{0,\cc{B}_u=\{1,x,x^2\}\}$ son:
    \begin{equation*}
        S = \{(a_0,a_1,a_2)_{\cc{R}}\in \cc{P}_2(\bb{R})\mid a_0=5,~a_2=2\}
    \end{equation*}

    Respecto a $T$, tenemos que el vector director es:
    $$\vec{(5x^2-2x)(2x^2-x+1)} = (2x^2-x+1) - (5x^2-2x) = -3x^2+x+1$$

    Por tanto, $T=5x^2-2x + \cc{L}\{-3x^2+x+1\}$. Para calcular las ecuaciones cartesianas en $\cc{R}$, el siguiente determinante ha de ser nulo:
    \begin{equation*}
        \left|
            \begin{array}{cc}
                1 & 0-a_0 \\
                1 & -2-a_1\\
                -3 & 5-a_2
            \end{array}
        \right| = 0
    \end{equation*}

    Para ello, ha de ser:
    \begin{equation*}
        \left|
            \begin{array}{cc}
                1 & 0-a_0 \\
                1 & -2-a_1\\
            \end{array}
        \right| = 0 = -2-a_1 + a_0 = 0
        \qquad \qquad
        \left|
            \begin{array}{cc}
                1 & 0-a_0 \\
                -3 & 5-a_2
            \end{array}
        \right| = 0 = 5-a_2-3a_0 = 0
    \end{equation*}

    Por tanto, las ecuaciones cartesianas en $\cc{R}$ son:
    \begin{equation*}
        T = \{(a_0,a_1,a_2)_{\cc{R}}\in \cc{P}_2(\bb{R})\mid a_0-a_1=2,~3a_0+a_2=5\}
    \end{equation*}

    Por tanto, las ecuaciones cartesianas de $S\cap T$ en $\cc{R}$ son:
    \begin{equation*}
        S\cap T = \{(a_0,a_1,a_2)_{\cc{R}}\in \cc{P}_2(\bb{R})\mid a_0 = 5,~a_1=3,~a_2=2,~3a_0+a_2=5\}
    \end{equation*}

    Por tanto, $S\cap T = \emptyset$, por lo que no se cortan.
\end{ejercicio}

\begin{ejercicio}
    En el espacio vectorial $\cc{M}_{2\times 2}(\bb{R})$, consideramos el producto escalar definido como $\langle M, N\rangle = tr(M^tN)$. Dotamos $\cc{M}_{2\times 2}(\bb{R})$ de la estructura afín canónica (que lo convierte en un espacio afín euclídeo). Calcula el ángulo que forman los siguientes hiperplanos afines:
    \begin{equation*}
        S = \{M \in \cc{M}_{2\times 2}(\bb{R}) \mid tr(M) = 2\}
        \quad \text{y}\quad 
        T = \left\{\left(\begin{array}{cc}
            a & b \\ c & d
        \end{array}\right) \in \cc{M}_{2\times 2}(\bb{R}) \mid a = 1 \right\}
    \end{equation*}


    Tenemos que $S$ viene dado por:
    \begin{equation*}
        S = \left\{\left(\begin{array}{cc}
            a & b \\ c & d
        \end{array}\right) \in \cc{M}_{2\times 2}(\bb{R}) \mid a + d = 2 \right\}
    \end{equation*}

    Tenemos que los vectores normales a $S$ y $T$, que son:
    \begin{equation*}
        n_S=\left(\begin{array}{cc}
            1 & 0 \\ 0 & 1
        \end{array}\right)
        \hspace{1cm}
        n_T=\left(\begin{array}{cc}
            1 & 0 \\ 0 & 0
        \end{array}\right)
    \end{equation*}

    Entonces $\measuredangle(S,T) = \measuredangle (n_S, n_T)$, por lo que:
    \begin{equation*}
        \measuredangle(S,T) = \arccos \left(\frac{tr(n_S^tn_T)}{tr(n_S^tn_S) \cdot tr(n_T^tn_T)}\right)
        = \arccos \left(\frac{1}{2 \cdot 1}\right) = \frac{\pi}{3}
    \end{equation*}
\end{ejercicio}


\begin{ejercicio}
    En $\bb{R}^2$ consideramos dos triángulos $T_1 = \{a_1, a_2, a_3\}$ y $T_2=~\{b_1, b_2, b_3\}$. Demuestra que:
    \begin{enumerate}
        \item Existen seis aplicaciones afines de $\bb{R}^2$ en $\bb{R}^2$ que llevan $T_1$ en $T_2$.

        Son las distintas formas de reordenar el conjunto de 3 elementos; es decir, las permutaciones, que son $3!=6$.

        \item Una de las aplicaciones afines anteriores $f : \bb{R}^2 \to \bb{R}^2$ es isometría si y sólo si
        \begin{equation*}
            d(a_i, a_j ) = d(f(a_i), f(a_j )), \qquad \forall i, j \in \{1, 2, 3\},
        \end{equation*}
        donde $d(\cdot, \cdot)$ es la función distancia de $\bb{R}^2$.

        \begin{description}
            \item[$\Longrightarrow)$] Por ser isometría, se tiene de forma directa.
            \item[$\Longleftarrow)$] Veamos que $\vec{f}$ es una isometría. Como $T_1$ son tres puntos afínmente independientes, sea $\cc{B}=\{\vec{a_1a_2}, \vec{a_1a_3}\}$. Entonces, tenemos que:
            \begin{equation*}
                \begin{split}
                    \langle u,v\rangle &= \langle \alpha\vec{a_1a_2} + \beta\vec{a_1a_3}, \alpha\vec{a_1a_2} + \beta\vec{a_1a_3}\rangle 
                    = \alpha^2\|\vec{a_1a_2}\| + \beta^2\|\vec{a_1a_3}\| + 2ab\langle \vec{a_1a_2}, \vec{a_1a_3}\rangle
                \end{split}
            \end{equation*}

            TERMINAR
        \end{description}
    \end{enumerate}
\end{ejercicio}

\begin{ejercicio}
    En $\bb{R}^3$, considera el plano afín $\Pi = \{(x, y, z) \in \bb{R}^3 \mid x + y - z = -1\}$. Sea $s : \bb{R}^3 \to \bb{R}^3$ la simetría respecto de $\Pi$. Calcula la imagen mediante $s$ de la recta dada por las ecuaciones $r\equiv \frac{x-1}{2} =\frac{y+1}{-3} = z - 1$.\\

    Sabemos que $r=(1,-1,1)+\cc{L}(2, -3, 1)$. Como el vector director de la recta no es un vector de $\vec{\bb{R}^3}$, tenemos que son secantes. Además, como son complementarios entonces su intersección es un único punto.
    
    En este caso, da la casualidad de que $p_0 = (1, -1, 1)\in \Pi$, por lo que: $$r\cap \Pi = \{(1,-1,1)\} = \{p_0\}$$

    Por tanto, tenemos que:
    \begin{equation*}
        s(q) = p_0 + \sigma_{\vec{\Pi}}(\vec{p_0q}) \qquad \forall q\in \bb{R}^3
    \end{equation*}

    Como tan solo piden la imagen de la recta dada, tenemos que $q=p_0+\lambda v$, con $v=(2,-3,1)$, $\lambda=\bb{R}$. La imagen pedida es:
    \begin{equation*}
        s(r) = s(p_0+\lm v) = p_0 + \sigma_{\vec{\Pi}}(\vec{p_0(p_0+\lm v)})
        = p_0 + \sigma_{\vec{\Pi}}(\lm v) = p_0 + \lm \sigma_{\vec{\Pi}}(v)
        \qquad \forall \lm \in \bb{R}
    \end{equation*}

    Por tanto, tan solo es necesario calcular $\sigma_{\vec{\Pi}}(v)$. Para ello, lo descomponemos como $v=v_t+v_n$, con $v_t\in \vec{\Pi}$, $v_n\in \vec{\Pi}^\perp$. Veamos dos formas para hacerlo:
    \begin{description}
        \item[Forma 1:] Más rápida y con menos cálculos, pero con un razonamiento más complejo.

        Como $\vec{\Pi}^\perp = \cc{L}\{(1,1,-1)\} = \cc{L}\{N\}$, tenemos que $v_n = \alpha N$, con $\alpha\in \bb{R}$. Calculemos el valor de $\alpha$:
        \begin{equation*}
            \langle v,N\rangle = \langle v_t,N\rangle + \alpha\langle N,N\rangle \Longrightarrow \alpha = \frac{\langle v,N\rangle}{\langle N,N\rangle} = \frac{\langle v,N\rangle}{\|N\|^2} = \frac{-2}{3}
        \end{equation*}
        
        Además, tenemos que $v_t = v-\alpha N$. Entonces, sabiendo la definición de simetría vectorial, tenemos que:
        \begin{equation*}
            \sigma_{\vec{\Pi}}(v) = \sigma_{\vec{\Pi}}(v_t + \alpha N) = v_t - \alpha N = v-2\alpha N
            = (2,-3,1) + \frac{4}{3}(1,1,-1) = \frac{1}{3}(10,-5,-1)
        \end{equation*}

        \item[Forma 2:] Forma usual, resolviendo un sistema de ecuaciones con 3 ecuaciones. Unos cálculos bastante más complejos (hay que resolver un sistema $3\times 3$), pero razonamiento más sencillo.

        Tenemos que $\vec{\Pi}^\perp = \cc{L}\{(1,1,-1)\}$ y $\vec{\Pi} = \cc{L}\{(1,0,1), (1,-1,0)\}$. Buscamos las coordenadas de $v$ en la base $\cc{B} = \{(1,0,1), (1,-1,0), (1,1,-1)\}$:
        \begin{equation*}
            (2,-3,1) = \alpha (1,0,1) + \beta(1,-1,0) + \gamma(1,1,-1)
        \end{equation*}

        Por tanto, el sistema a resolver es:
        \begin{equation*}
            \left\{\begin{array}{r}
                \alpha+\beta+\gamma = 2\\
                -\beta+\gamma = -3\\
                \alpha - \gamma = 1
            \end{array}\right\} \Longrightarrow \left\{\begin{array}{l}
                \alpha = \nicefrac{1}{3} \\
                \beta = \nicefrac{7}{3} \\
                \gamma = \nicefrac{-2}{3}
            \end{array}\right.
        \end{equation*}

        Por tanto, $v=\frac{1}{3}(1,0,1) +\frac{7}{3} (1,-1,0) -\frac{2}{3}(1,1,-1)$. Entonces:
        \begin{equation*}
            \sigma_{\vec{\Pi}}(v) = \frac{1}{3}(1,0,1) +\frac{7}{3} (1,-1,0) +\frac{2}{3}(1,1,-1) = \frac{1}{3}(10, -5, -1)
        \end{equation*}
    \end{description}

    Por tanto, y sabiendo el valor de $\sigma_{\vec{\Pi}}(v)$, tenemos que:
    \begin{equation*}
        s(r) = (1,-1,1) + \cc{L}\left\{(10,-5,-1)\right\}
    \end{equation*}
    
    
    
\end{ejercicio}

\begin{ejercicio}
     Una aplicación afín $f : (\cc{A}, \langle , \rangle) \to (\cc{A}', \langle , \rangle')$ entre espacios afines euclidianos se dice que preserva la ortogonalidad si para cualesquiera rectas secantes $R, S$ en $\cc{A}$ tales que $R \perp S$, entonces $f({R}) \perp f(S)$.

     Probar que si $f : (\cc{A}, \langle , \rangle) \to (\cc{A}', \langle , \rangle')$ es una aplicación afín biyectiva, entonces $f$ es una semejanza (composición de un movimiento rígido y una homotecia) si, y sólo si, $f$ preserva la ortogonalidad.

     \begin{description}
         \item[$\Longrightarrow)$] Supongamos $R\perp S$, por lo que $\vec{R}\perp \vec{S}$; es decir, $\langle u,v\rangle = 0$ para todo $u\in \vec{R},~v\in \vec{S}$. Veamos si $f(R)\perp f(S)$, o equivalentemente $\vec{f}(\vec{R})\perp \vec{f}(\vec{S})$.

         Sean $u_f\in \vec{f}(\vec{R}), v_f\in \vec{f}(\vec{S})$. Como $f$ es una biyección, tenemos que $\vec{f}$ también lo es, por lo que $\exists_1~u\in \vec{R}, v\in \vec{S}\mid \vec{f}(u) = u_f,~\vec{f}(v) = v_f$.

         Veamos si $\langle u_f,v_f\rangle = 0$:
         \begin{equation*}
             \langle u_f, v_f\rangle = \langle \vec{f}(u), \vec{f}(v)\rangle
         \end{equation*}

         Veamos ahora el valor de $\vec{f}$. Como $f$ es la composición de un movimiento rígido con una homotecia, sea $f=g\circ h$, con $h$ homotecia y $g$ movimiento rígido. Entonces, tenemos que $\vec{h}=kId_{\vec{\cc{A}}}$, mientras que $\vec{f}$ es una isometría vectorial. Por tanto:
         \begin{multline*}
             \left\langle u_f, v_f\right\rangle = \left\langle \vec{f}(u), \vec{f}(v)\right\rangle
             = \left\langle \vec{g}(\vec{h}(u)), \vec{g}(\vec{h}(v))\right\rangle
             = \left\langle \vec{g}(ku), \vec{g}(kv)\right\rangle
             \AstIg\\\AstIg \left\langle ku, kv\right\rangle = k^2\left\langle u, v\right\rangle = 0 \qquad \qquad \forall u_f\in \vec{R},~v_f\in \vec{S}
         \end{multline*} 
         donde en $(\ast)$ he aplicado que $\vec{g}$ es una isometría vectorial.

         Por tanto, hemos demostrado que si $f$ es una semejanza, entonces preserva la ortogonalidad.

         \item[$\Longleftarrow)$] TERMINAR
     \end{description}
\end{ejercicio}

\begin{ejercicio}
    Encuentra, si existe, un movimiento rígido de $\bb{R}^2$ que lleve la recta $\{(x, y) \in \bb{R}^2 \mid x = 0\}$ en la recta $\{(x, y) \in \bb{R}^2 \mid y = 1\}$, y también lleve la recta $\{(x, y) \in \bb{R}^2 \mid y =0\}$ en la recta $\{(x, y) \in \bb{R}^2 \mid x = 1\}$.

    \begin{figure}[H]
        \centering
        \begin{tikzpicture}
            
            % Recta x=1
            \draw[red] (1,-1) -- (1,2) node[above right] {$x=1$};

            % Recta x=0
            \draw[blue] (0,-1) -- (0,2) node[above left] {$x=0$};
            
            % Recta y=0
            \draw[red] (-1,0) -- (4,0) node[right] {$y=0$};

            % Recta y=1
            \draw[blue] (-1,1) -- (4,1) node[right] {$y=1$};

            \fill (0,0) circle[radius=2pt] node[below left] {$(0,0)$};
        \end{tikzpicture}
    \end{figure}

    \begin{description}
        \item[Opción 1:] Composición de un giro con una traslación.

        Sean las las siguientes rectas, con sus respectivas imágenes:
        \begin{gather*}
            r=(0,0)+\cc{L}\{(0,1)\} \longrightarrow f(r)=(1,1)+\cc{L}\{(-1,0)\} \\
            s=(0,0)+\cc{L}\{(1,0)\} \longrightarrow f(s)=(1,1)+\cc{L}\{(0,1)\}
        \end{gather*}

        Podemos ver que una posible solución es que $f$ sea la composición de un giro de 90 grados centrado en el origen junto con una traslación según el vector $(1,1)$.
        
        \item[Opción 2:] Simetría axial.

        También podemos ver las rectas, con sus respectivas imágenes, de la siguiente forma:
        \begin{gather*}
            r=(0,1)+\cc{L}\{(0,1)\} \longrightarrow f(r)=(0,1)+\cc{L}\{(-1,0)\} \\
            s=(1,0)+\cc{L}\{(1,0)\} \longrightarrow f(s)=(1,0)+\cc{L}\{(0,-1)\}
        \end{gather*}

        Podemos ver que hay dos puntos fijos, $(0,1), (1,0)$. Tenemos que se trata de la reflexión axial respecto de $L=(0,1)+\cc{L}\{(1,-1)\}$; es decir, $L\equiv y=-x+1$.
    \end{description}
\end{ejercicio}

\begin{ejercicio}
    Sean $f_1, f_2$ las simetrías (ortogonales) de $\bb{R}^2$ respecto de las rectas $R_1 = \{(x, y) \in \bb{R}^2 \mid x - y = 2\}$ y $R_2=\{(x, y) \in \bb{R}^2 \mid x - 2y = 1\}$, respectivamente. Calcula $f_1 \circ f_2$ y descríbela geométricamente.
\end{ejercicio}

\begin{ejercicio}
    Considera un espacio afín euclídeo $\cc{A}$ de dimensión 3, y sea $f$ un movimiento rígido de $\cc{A}$ tal que $f(1, 0, 1) = (2, -3, 1)$ en coordenadas de un sistema de referencia euclídeo fijo. Si sabemos que $f$ es la simetría respecto de un plano, calcula dicho plano.
\end{ejercicio}

\begin{ejercicio}
    Sea $\cc{A}$ un espacio afín euclídeo de dimensión 2, y sean $R_1$ y $R_2$ dos rectas de $\cc{A}$. Prueba que siempre es posible encontrar un movimiento rígido $f : \cc{A} \to \cc{A}$ que lleve $R_1$ en $R_2$. Estudia de qué tipo es $f$, según la posición relativa de $R_1$ y $R_2$. ¿Se puede elegir siempre directo? ¿E inverso?
\end{ejercicio}

\begin{ejercicio}
    Sean $p$ y $q$ dos puntos distintos en un espacio afín euclídeo. Demuestra que existe una única simetría respecto de un hiperplano que lleva $p$ en $q$.
\end{ejercicio}

\begin{ejercicio}
     Sean $R_1$ y $R_2$ dos rectas que se cruzan en un espacio afín euclídeo tridimensional $\cc{A}$. Demuestra que existe una única recta afín $R$ que interseca de manera ortogonal a $R_1$ y $R_2$. Prueba además que la distancia de $R_1$ a $R_2$ es exactamente la distancia entre los puntos dados por $R_1 \cap R$ y $R_2 \cap R$.
\end{ejercicio}

\begin{ejercicio}
    Consideremos la aplicación $f : \bb{R}^2 \to \bb{R}^2$ que viene dada por $f(x, y) = (y -2, x+ 1)$. ¿Es $f$ un movimiento rígido? En tal caso, clasifícalo.
\end{ejercicio}

\begin{ejercicio}
     Demostrar que si $p$ y $q$ son dos puntos de un espacio afín euclídeo $\cc{A}$, entonces siempre existe un movimiento rígido $f : \cc{A} \to \cc{A}$ tal que $f(p) = q$. De forma más general, probar que si $\cc{A}$ tiene dimensión finita y $S$, $S'$ son dos subespacios afines de $\cc{A}$ de dimensión $m$, entonces existe un movimiento rígido $f : \cc{A} \to \cc{A}$ tal que $f(S) = S_0$.
\end{ejercicio}

\begin{ejercicio}
     Consideremos la aplicación $f : \bb{R}^2 \to \bb{R}^2$ que viene dada por $f(x, y) = (2y - 1, -2x + 3)$. ¿Es $f$ un movimiento rígido? En tal caso, clasifícalo.
\end{ejercicio}

\begin{ejercicio}
    Sean $f_1, f_2 : \bb{R}^2\to \bb{R}^2$ las isometrías dadas, respectivamente, por las simetrías respecto de las rectas de ecuación $x + y = 0$ y $x + 2y = 2$.
    \begin{enumerate}
        \item Calcula explícitamente $f_1$ y $f_2$ en coordenadas usuales.
        \item Clasifica el movimiento rígido $g = f_1 \circ f_2$.
    \end{enumerate}
\end{ejercicio}

\begin{ejercicio}
    Demuestra que las siguientes aplicaciones son movimientos rígidos del plano y clasifícalos.
    \begin{enumerate}
        \item $f\left(x, y\right) = \left(3 - \dfrac{3x}{5} + \dfrac{4y}{5}, 1 - \dfrac{4x}{5} - \dfrac{3y}{5}\right).$

        Tenemos que:
        \begin{equation*}
            M(f,\cc{R}_0) = \left(
            \begin{array}{c|cc}
                1 & 0 & 0 \\ \hline
                3 & \nicefrac{-3}{5} & \nicefrac{4}{5} \\
                1 & \nicefrac{-4}{5} & \nicefrac{-3}{5} \\
            \end{array}
            \right)
        \end{equation*}

        Como $\cc{R}_0$ es un sistema de referencia ortonormal, para ver si $f$ es una isometría (equivalentemente vemos que $\vec{f}$ lo es) basta con probar que $M(\vec{f}, \cc{B}_u)\in O(2)$. Tenemos que:
        \begin{equation*}
            M(\vec{f},\cc{B}_u)M(\vec{f},\cc{B}_u)^t
            = 
            \left(
            \begin{array}{cc}
                \nicefrac{-3}{5} & \nicefrac{4}{5} \\
                \nicefrac{-4}{5} & \nicefrac{-3}{5} \\
            \end{array}
            \right)
            \left(
            \begin{array}{cc}
                \nicefrac{-3}{5} & \nicefrac{-4}{5} \\
                \nicefrac{4}{5} & \nicefrac{-3}{5} \\
            \end{array}
            \right)
            = 
            \left(
            \begin{array}{cc}
                1 & 0 \\
                0 & 1 \\
            \end{array}
            \right) = Id_2
        \end{equation*}
        Por tanto, $\vec{f}$ es una isometría y por tanto, también lo es $f$. Como $|f|=1$, tenemos que $f$ es una isometría directa en el plano. Calculamos sus puntos fijos $(x,y)\in \bb{R}^2$:
        \begin{equation*}
            \left(
            \begin{array}{cc}
                \nicefrac{-3}{5}-1 & \nicefrac{4}{5} \\
                \nicefrac{-4}{5} & \nicefrac{-3}{5}-1 \\
            \end{array}
            \right)
            \left(
            \begin{array}{c}
                x\\y
            \end{array}
            \right)
            =\left(
            \begin{array}{cc}
                \nicefrac{-8}{5} & \nicefrac{4}{5} \\
                \nicefrac{-4}{5} & \nicefrac{-8}{5} \\
            \end{array}
            \right)
            \left(
            \begin{array}{c}
                x\\y
            \end{array}
            \right)
            = \left(
            \begin{array}{c}
                -3\\-1
            \end{array}
            \right)
        \end{equation*}

        Equivalentemente,
        \begin{equation*}
            \left(
            \begin{array}{cc}
                -8 & 4 \\
                -4 & -8 \\
            \end{array}
            \right)
            \left(
            \begin{array}{c}
                x\\y
            \end{array}
            \right)
            = \left(
            \begin{array}{c}
                -15\\-5
            \end{array}
            \right)
        \end{equation*}

        Por tanto, tenemos que solo hay un punto fijo, el punto $o=\frac{1}{4}(7,-1)$. Por tanto, se traza de un giro en el plano centrado en el punto $o$. Para calcular el ángulo no orientado sabemos que:
        \begin{equation*}
            2\cos\theta = tr(M(\vec{f}, \cc{B}_u)) = -\frac{6}{5} \Longrightarrow \cos\theta = -\frac{6}{10} = -\frac{3}{5} \Longrightarrow \theta \approx 2.21\text{ rad}
        \end{equation*}

        Por tanto, se trata de un giro en el plano centrado en el punto $o=\frac{1}{4}(7,-1)$ y de ángulo $\theta \approx 2.21\text{ rad}$.

        
        
        \item $f\left(x, y\right) = \left(\dfrac{x}{2} -\dfrac{\sqrt{3} y}{2} + 1,\dfrac{\sqrt{3} x}{2}+ \dfrac{y}{2} + 2\right).$

        Tenemos que:
        \begin{equation*}
            M(f,\cc{R}_0) = \left(
            \begin{array}{c|cc}
                1 & 0 & 0 \\ \hline
                1 & \nicefrac{1}{2} & \nicefrac{-\sqrt{3}}{2} \\
                2 & \nicefrac{\sqrt{3}}{2} & \nicefrac{1}{2} \\
            \end{array}
            \right)
        \end{equation*}

        Como $\cc{R}_0$ es un sistema de referencia ortonormal, para ver si $f$ es una isometría (equivalentemente vemos que $\vec{f}$ lo es) basta con probar que $M(\vec{f}, \cc{B}_u)\in O(2)$. Tenemos que:
        \begin{equation*}
            M(\vec{f},\cc{B}_u)M(\vec{f},\cc{B}_u)^t
            = 
            \left(
            \begin{array}{cc}
                \nicefrac{1}{2} & \nicefrac{-\sqrt{3}}{2} \\
                \nicefrac{\sqrt{3}}{2} & \nicefrac{1}{2}
            \end{array}
            \right)
            \left(
            \begin{array}{cc}
                \nicefrac{1}{2} & \nicefrac{\sqrt{3}}{2} \\
                \nicefrac{-\sqrt{3}}{2} & \nicefrac{1}{2}
            \end{array}
            \right)
            = 
            \left(
            \begin{array}{cc}
                1 & 0 \\
                0 & 1 \\
            \end{array}
            \right) = Id_2
        \end{equation*}
        Por tanto, $\vec{f}$ es una isometría y por tanto, también lo es $f$. Como $|f|=1$, tenemos que $f$ es una isometría directa en el plano. Calculamos sus puntos fijos $(x,y)\in \bb{R}^2$:
        \begin{equation*}
            \left(
            \begin{array}{cc}
                \nicefrac{1}{2}-1 & \nicefrac{-\sqrt{3}}{2} \\
                \nicefrac{\sqrt{3}}{2} & \nicefrac{1}{2}-1
            \end{array}
            \right)
            \left(
            \begin{array}{c}
                x\\y
            \end{array}
            \right)
            =\left(
            \begin{array}{cc}
                \nicefrac{-1}{2} & \nicefrac{-\sqrt{3}}{2} \\
                \nicefrac{\sqrt{3}}{2} & \nicefrac{-1}{2}
            \end{array}
            \right)
            \left(
            \begin{array}{c}
                x\\y
            \end{array}
            \right)
            = \left(
            \begin{array}{c}
                -1\\-2
            \end{array}
            \right)
        \end{equation*}

        Equivalentemente,
        \begin{equation*}
            \left(
            \begin{array}{cc}
                -1 & -\sqrt{3} \\
                \sqrt{3} & -1 \\
            \end{array}
            \right)
            \left(
            \begin{array}{c}
                x\\y
            \end{array}
            \right)
            = \left(
            \begin{array}{c}
                -2\\-4
            \end{array}
            \right)
        \end{equation*}

        Por tanto, tenemos que solo hay un punto fijo, el punto $$o=\frac{1}{2}(-2\sqrt{3}+1,\sqrt{3}+2)$$ Por tanto, se traza de un giro en el plano centrado en el punto $o$. Para calcular el ángulo no orientado sabemos que:
        \begin{equation*}
            2\cos\theta = tr(M(\vec{f}, \cc{B}_u)) = 1 \Longrightarrow \cos\theta = \frac{1}{2} \Longrightarrow \theta =\frac{\pi}{3}\text{ rad}
        \end{equation*}

        Por tanto, se trata de un giro en el plano de ángulo $\theta =\frac{\pi}{3}\text{ rad}$ y centrado en el punto $o=\frac{1}{2}(-2\sqrt{3}+1,\sqrt{3}+2)$.


        
        \item $f\left(x, y\right) = \left(-\dfrac{x}{2} + \dfrac{\sqrt{3}y}{2} + 1,\dfrac{\sqrt{3} x}{2} + \dfrac{y}{2} - 1\right).$

        Tenemos que:
        \begin{equation*}
            M(f,\cc{R}_0) = \left(
            \begin{array}{c|cc}
                1 & 0 & 0 \\ \hline
                1 & \nicefrac{-1}{2} & \nicefrac{\sqrt{3}}{2} \\
                -1 & \nicefrac{\sqrt{3}}{2} & \nicefrac{1}{2} \\
            \end{array}
            \right)
        \end{equation*}

        Como $\cc{R}_0$ es un sistema de referencia ortonormal, para ver si $f$ es una isometría (equivalentemente vemos que $\vec{f}$ lo es) basta con probar que $M(\vec{f}, \cc{B}_u)\in O(2)$. Tenemos que:
        \begin{equation*}
            M(\vec{f},\cc{B}_u)M(\vec{f},\cc{B}_u)^t
            = 
            \left(
            \begin{array}{cc}
                \nicefrac{-1}{2} & \nicefrac{\sqrt{3}}{2} \\
                \nicefrac{\sqrt{3}}{2} & \nicefrac{1}{2}
            \end{array}
            \right)
            \left(
            \begin{array}{cc}
                \nicefrac{-1}{2} & \nicefrac{\sqrt{3}}{2} \\
                \nicefrac{\sqrt{3}}{2} & \nicefrac{1}{2}
            \end{array}
            \right)
            = 
            \left(
            \begin{array}{cc}
                1 & 0 \\
                0 & 1 \\
            \end{array}
            \right) = Id_2
        \end{equation*}
        Por tanto, $\vec{f}$ es una isometría y por tanto, también lo es $f$. Como $|f|=-1$, tenemos que $f$ es una isometría inversa en el plano. Calculamos sus puntos fijos $(x,y)\in \bb{R}^2$:
        \begin{equation*}
            \left(
            \begin{array}{cc}
                \nicefrac{-1}{2}-1 & \nicefrac{\sqrt{3}}{2} \\
                \nicefrac{\sqrt{3}}{2} & \nicefrac{1}{2}-1
            \end{array}
            \right)
            \left(
            \begin{array}{c}
                x\\y
            \end{array}
            \right)
            =\left(
            \begin{array}{cc}
                \nicefrac{-3}{2} & \nicefrac{\sqrt{3}}{2} \\
                \nicefrac{\sqrt{3}}{2} & \nicefrac{-1}{2}
            \end{array}
            \right)
            \left(
            \begin{array}{c}
                x\\y
            \end{array}
            \right)
            = \left(
            \begin{array}{c}
                -1\\1
            \end{array}
            \right)
        \end{equation*}

        Equivalentemente,
        \begin{equation*}
            \left(
            \begin{array}{cc}
                -3 & \sqrt{3} \\
                \sqrt{3} & -1 \\
            \end{array}
            \right)
            \left(
            \begin{array}{c}
                x\\y
            \end{array}
            \right)
            = \left(
            \begin{array}{c}
                -2\\2
            \end{array}
            \right)
        \end{equation*}

        Por tanto, tenemos que $f$ no tiene puntos fijos, ya que ese sistema es un SI. Por tanto, tenemos que es una simetría axial con deslizamiento.

        Calculamos en primer lugar su eje de simetría $L$, buscando primero $\vec{L}$. Como $\vec{f}=\sigma_{\vec{L}}$, tenemos que $\vec{L}=V_1$, por lo que:
        \begin{equation*}
            \begin{split}
                \vec{L} &= \{ (x,y)\in \bb{R}^2 \mid (M(\vec{f}, \cc{B}_u) - Id) (x,y)^t = (0,0)^t \} =\\
                &= \left\{ (x,y)\in \bb{R}^2 \mid \left(
                \begin{array}{cc}
                    \nicefrac{-1}{2}-1 & \nicefrac{\sqrt{3}}{2} \\
                    \nicefrac{\sqrt{3}}{2} & \nicefrac{1}{2}-1
                \end{array}
                \right)
                \left(
                \begin{array}{c}
                    x\\y
                \end{array}
                \right)
                = \left(
                \begin{array}{c}
                    0\\0
                \end{array}
                \right)
                \right\} =\\
                &= \left\{ (x,y)\in \bb{R}^2 \mid \left(
                \begin{array}{cc}
                    -3 & \sqrt{3} \\
                    \sqrt{3} & -1 \\
                \end{array}
                \right)
                \left(
                \begin{array}{c}
                    x\\y
                \end{array}
                \right)
                = \left(
                \begin{array}{c}
                    0\\0
                \end{array}
                \right)
                \right\} =\\
                &= \left\{ (x,y)\in \bb{R}^2 \mid x = \frac{\sqrt{3}}{3} y
                \right\} = \cc{L}\left\{\left(\frac{\sqrt{3}}{3}, 1\right)\right\}
            \end{split}
        \end{equation*}

        Falta ahora calcular un punto $(x,y)\in S$. Para ello, hay dos opciones:
        \begin{description}
            \item[Opción 1:] Método General. Este método también es válido para los movimientos helicoidales.



            Sea $(x,y)\in L$, por lo que $\vec{(x,y)f(x,y)}\in \vec{L}$. Aplicando esta condición, tenemos que:
        \begin{multline*}
            \vec{(x,y)f(x,y)} = f(x,y) - (x,y) = \left(-\dfrac{3}{2}x + \dfrac{\sqrt{3}y}{2} + 1,\dfrac{\sqrt{3} x}{2} - \dfrac{y}{2} - 1\right) \in \vec{L} \Longrightarrow \\\Longrightarrow 
            -\dfrac{3}{2}x + \dfrac{\sqrt{3}y}{2} + 1 = \frac{\sqrt{3}}{3}\left(\dfrac{\sqrt{3} x}{2} - \dfrac{y}{2} - 1\right)
            \Longrightarrow 
            -\dfrac{3}{2}x + \dfrac{\sqrt{3}y}{2} + 1 = \dfrac{x}{2} - \dfrac{\sqrt{3}}{6}y - \frac{\sqrt{3}}{3} \Longrightarrow \\ \Longrightarrow
            -2x + \dfrac{2\sqrt{3}}{3}y + 1 + \frac{\sqrt{3}}{3} = 0
            \Longrightarrow x = \dfrac{\sqrt{3}}{3}y + \frac{1}{2} + \frac{\sqrt{3}}{6}
        \end{multline*}

        Esta es la ecuación cartesiana del eje en el sistema de referencia $\cc{R}_0$. Por tanto, tenemos que un punto del eje es $\left(\nicefrac{1}{2},\nicefrac{-1}{2}\right)$.


            \item[Opción 2]: Método concreto para las reflexiones con deslizamiento.
            
            
            Tenemos que $m_{pf(p)}\in L$ por ser una simetría con desplazamiento. Por tanto, dado $p=(0,0)$, tenemos que $f(p)=(1,-1)$, por lo que:
            \begin{equation*}
                m_{pf(p)} = \left(\frac{1}{2}, -\frac{1}{2}\right)\in S
            \end{equation*}
        \end{description}

        
        Es decir,
        \begin{equation*}
            L = \left(\frac{1}{2}, -\frac{1}{2}\right) + \cc{L}\left\{\left(\frac{\sqrt{3}}{3}, 1\right)\right\}
        \end{equation*}

        Falta ahora por calcular el vector de desplazamiento, tenemos que este se calcula como $v=\vec{pf(p)}$ para cualquier $p\in L$.
        \begin{equation*}
            v = f\left(\frac{1}{2}, -\frac{1}{2}\right)-\left(\frac{1}{2}, -\frac{1}{2}\right)
            = \left(\frac{1-\sqrt{3}}{4},\frac{-3+\sqrt{3}}{4}\right)
        \end{equation*}
        

        
        \item $f\left(x, y\right) = \left(\dfrac{3x}{5} + \dfrac{4y}{5} + 2, \dfrac{4x}{5} - \dfrac{3y}{5} + 5\right).$

        Tenemos que:
        \begin{equation*}
            M(f,\cc{R}_0) = \left(
            \begin{array}{c|cc}
                1 & 0 & 0 \\ \hline
                2 & \nicefrac{3}{5} & \nicefrac{4}{5} \\
                5 & \nicefrac{4}{5} & \nicefrac{-3}{5} \\
            \end{array}
            \right)
        \end{equation*}

        Como $\cc{R}_0$ es un sistema de referencia ortonormal, para ver si $f$ es una isometría (equivalentemente vemos que $\vec{f}$ lo es) basta con probar que $M(\vec{f}, \cc{B}_u)\in O(2)$. Tenemos que:
        \begin{equation*}
            M(\vec{f},\cc{B}_u)M(\vec{f},\cc{B}_u)^t
            = 
            \left(
            \begin{array}{cc}
                \nicefrac{3}{5} & \nicefrac{4}{5} \\
                \nicefrac{4}{5} & \nicefrac{-3}{5} 
            \end{array}
            \right)
            \left(
            \begin{array}{cc}
                \nicefrac{3}{5} & \nicefrac{4}{5} \\
                \nicefrac{4}{5} & \nicefrac{-3}{5}
            \end{array}
            \right)
            = 
            \left(
            \begin{array}{cc}
                1 & 0 \\
                0 & 1 \\
            \end{array}
            \right) = Id_2
        \end{equation*}
        Por tanto, $\vec{f}$ es una isometría y por tanto, también lo es $f$. Como $|f|=-1$, tenemos que $f$ es una isometría inversa en el plano. Calculamos sus puntos fijos $(x,y)\in \bb{R}^2$:
        \begin{equation*}
            \left(
            \begin{array}{cc}
                \nicefrac{3}{5}-1 & \nicefrac{4}{5} \\
                \nicefrac{4}{5} & \nicefrac{-3}{5}-1
            \end{array}
            \right)
            \left(
            \begin{array}{c}
                x\\y
            \end{array}
            \right)
            =\left(
            \begin{array}{cc}
                \nicefrac{-2}{5} & \nicefrac{4}{5} \\
                \nicefrac{4}{5} & \nicefrac{-8}{5}
            \end{array}
            \right)
            \left(
            \begin{array}{c}
                x\\y
            \end{array}
            \right)
            = \left(
            \begin{array}{c}
                -2\\-5
            \end{array}
            \right)
        \end{equation*}

        Equivalentemente,
        \begin{equation*}
            \left(
            \begin{array}{cc}
                -2&4 \\
                4&-8
            \end{array}
            \right)
            \left(
            \begin{array}{c}
                x\\y
            \end{array}
            \right)
            = \left(
            \begin{array}{c}
                -10\\-25
            \end{array}
            \right)
        \end{equation*}

        Por tanto, tenemos que $f$ no tiene puntos fijos, ya que ese sistema es un SI. Por tanto, tenemos que es una simetría axial con deslizamiento.

        Sea $L$ el eje de simetría, y calculemos $\vec{L}$. Como $\vec{f}=\sigma_{\vec{L}}$, tenemos que $\vec{L}=V_1$, por lo que:
        \begin{equation*}
            \begin{split}
                \vec{L} &= \{ (x,y)\in \bb{R}^2 \mid (M(\vec{f}, \cc{B}_u) - Id) (x,y)^t = (0,0)^t \} =\\
                &= \left\{ (x,y)\in \bb{R}^2 \mid \left(
                \begin{array}{cc}
                    -2&4 \\
                    4&-8
                \end{array}
                \right)
                \left(
                \begin{array}{c}
                    x\\y
                \end{array}
                \right)
                = \left(
                \begin{array}{c}
                    0\\0
                \end{array}
                \right)
                \right\} =\\
                &= \left\{ (x,y)\in \bb{R}^2 \mid -x+2y=0
                \right\} = \cc{L}\left\{\left(2,1\right)\right\}
            \end{split}
        \end{equation*}

        Busquemos ahora un punto de $L$. tenemos que $m_{pf(p)}\in L$ para todo $p\in \bb{R}^2$. Sea $p=(0,0)$, $f(p)=(2,5)$. Tenemos que:
        \begin{equation*}
            m_{pf(p)} = \left(1,\nicefrac{5}{2}\right)\in L
        \end{equation*}

        Por tanto, el eje es $L=(1,\nicefrac{5}{2}) + \cc{L}\left\{\left(2,1\right)\right\}$. El vector de desplazamiento es $v=\vec{pf(p)}$ para todo $p\in L$. Entonces:
        \begin{equation*}
            v = \vec{pf(p)} = f(1,\nicefrac{5}{2}) - (1,\nicefrac{5}{2})
            = \left(\frac{18}{5}, \frac{9}{5}\right)
        \end{equation*}
    \end{enumerate}
\end{ejercicio}

\begin{ejercicio}
    Demuestra que las siguientes aplicaciones son movimientos rígidos del espacio y clasifícalos.
    \begin{enumerate}
        \item $f\left(x, y, z\right) = \left(2 + y, x, 1 + z\right)$.

        Tenemos que:
        \begin{equation*}
            M(f,\cc{R}_0) = \left(\begin{array}{c|ccc}
                1 & 0 & 0 & 0 \\ \hline
                2 & 0 & 1 & 0 \\
                0 & 1 & 0 & 0 \\
                1 & 0 & 0 & 1 
            \end{array}\right)
        \end{equation*}

        Como $\cc{R}_0$ es un sistema de referencia ortonormal, para ver si $f$ es una isometría (equivalentemente vemos que $\vec{f}$ lo es) basta con probar que $M(\vec{f}, \cc{B}_u)\in O(3)$. Tenemos que:
        \begin{equation*}
            M(\vec{f},\cc{B}_u)M(\vec{f},\cc{B}_u)^t
            = 
            \left(
            \begin{array}{ccc}
                0 & 1 & 0 \\
                1 & 0 & 0 \\
                0 & 0 & 1 
            \end{array}
            \right)
            \left(
            \begin{array}{ccc}
                0 & 1 & 0 \\
                1 & 0 & 0 \\
                0 & 0 & 1 
            \end{array}
            \right)
            = 
            \left(
            \begin{array}{ccc}
                1 & 0 & 0\\
                0 & 1 & 0\\
                0 & 0 & 1
            \end{array}
            \right) = Id_3
        \end{equation*}

        Por tanto, tenemos que $\vec{f}$ es una isometría, y por tanto $f$ lo es también. Como $|\vec{f}|=-1$, tenemos que se trata de una isometría inversa. Calculemos los puntos fijos:
        \begin{equation*}
            -\left(
            \begin{array}{ccc}
                -1 & 1 & 0 \\
                1 & -1 & 0 \\
                0 & 0 & 0 
            \end{array}
            \right)
            \left(
            \begin{array}{c}
                x \\ y \\ z
            \end{array}
            \right)
            = \left(
            \begin{array}{c}
                2 \\ 0 \\ 1
            \end{array}
            \right)
        \end{equation*}

        Por tanto, $f$ no tiene puntos fijos, por lo que se trata de una simetría especular con deslizamiento. Calculemos el plano de simetría $\pi$. Obtenemos en primer lugar $\vec{\pi}$, que como $\vec{f}=\sigma_{\vec{\pi}}$, tenemos que es $V_1$:
        \begin{equation*}
            \begin{split}
                \vec{\pi} &= \{ (x,y,z)\in \bb{R}^3 \mid (M(\vec{f}, \cc{B}_u) - Id) (x,y,z)^t = (0,0, 0)^t \} =\\
                &= \left\{ (x,y,z)\in \bb{R}^3 \mid \left(
                \begin{array}{ccc}
                    -1 & 1 & 0\\
                    1 & -1 & 0\\
                    0 & 0 & 0
                \end{array}
                \right)
                \left(
                \begin{array}{c}
                    x\\y\\z
                \end{array}
                \right)
                = \left(
                \begin{array}{c}
                    0\\0\\0
                \end{array}
                \right)
                \right\} =\\
                &= \left\{ (x,y,z)\in \bb{R}^3 \mid x-y=0
                \right\} = \cc{L}\left\{\left(1,1,0\right), \left(0,0,1\right)\right\}
            \end{split}
        \end{equation*}

        Busquemos ahora un punto de $\pi$. tenemos que $m_{pf(p)}\in \pi$ para todo $p\in \bb{R}^3$. Sea $p=(0,0,0)$, $f(p)=(2,0,1)$. Tenemos que:
        \begin{equation*}
            m_{pf(p)} = \left(1,0,\frac{1}{2}\right)\in \pi
        \end{equation*}

        Por tanto, el plano de simetría es $\pi=\left(1,0,\frac{1}{2}\right) + \cc{L}\left\{\left(1,1,0\right), \left(0,0,1\right)\right\}$. El vector de desplazamiento es $v=\vec{pf(p)}$ para todo $p\in \pi$. Entonces:
        \begin{equation*}
            v = \vec{pf(p)} = f\left(1,0,\frac{1}{2}\right) - \left(1,0,\frac{1}{2}\right)
            = \left(2,1,\frac{3}{2}\right)-\left(1,0,\frac{1}{2}\right)
            = \left(1,1,1\right)
        \end{equation*}
        
        \item $f\left(x, y, z\right) = \left(\dfrac{x}{2} -\dfrac{\sqrt{3} z}{2} + 2, y + 2,\dfrac{\sqrt{3} x}{2} + \dfrac{z}{2} + 2\right)$.

        Tenemos que:
        \begin{equation*}
            M(f,\cc{R}_0) = \left(\begin{array}{c|ccc}
                1 & 0 & 0 & 0 \\ \hline
                2 & \nicefrac{1}{2} & 0 & \nicefrac{-\sqrt{3}}{2} \\
                2 & 0 & 1 & 0 \\
                2 & \nicefrac{\sqrt{3}}{2} & 0 & \nicefrac{1}{2} 
            \end{array}\right)
        \end{equation*}

        Como $\cc{R}_0$ es un sistema de referencia ortonormal, para ver si $f$ es una isometría (equivalentemente vemos que $\vec{f}$ lo es) basta con probar que $M(\vec{f}, \cc{B}_u)\in O(3)$. Tenemos que:
        \begin{equation*}
            M(\vec{f},\cc{B}_u)M(\vec{f},\cc{B}_u)^t
            = 
            \left(
            \begin{array}{ccc}
                \nicefrac{1}{2} & 0 & \nicefrac{-\sqrt{3}}{2} \\
                0 & 1 & 0 \\
                \nicefrac{\sqrt{3}}{2} & 0 & \nicefrac{1}{2} 
            \end{array}
            \right)
            \left(
            \begin{array}{ccc}
                \nicefrac{1}{2} & 0 & \nicefrac{\sqrt{3}}{2} \\
                0 & 1 & 0 \\
                \nicefrac{-\sqrt{3}}{2} & 0 & \nicefrac{1}{2} 
            \end{array}
            \right)
            = 
            \left(
            \begin{array}{ccc}
                1 & 0 & 0\\
                0 & 1 & 0\\
                0 & 0 & 1
            \end{array}
            \right) = Id_3
        \end{equation*}

        Por tanto, tenemos que $\vec{f}$ es una isometría, y por tanto $f$ lo es también. Como $|\vec{f}|=1$, tenemos que se trata de una isometría directa. Calculemos los puntos fijos:
        \begin{equation*}
            -\left(
            \begin{array}{ccc}
                -\nicefrac{1}{2} & 0 & \nicefrac{-\sqrt{3}}{2} \\
                0 & 0 & 0 \\
                \nicefrac{\sqrt{3}}{2} & 0 & -\nicefrac{1}{2} 
            \end{array}
            \right)
            \left(
            \begin{array}{c}
                x \\ y \\ z
            \end{array}
            \right)
            = \left(
            \begin{array}{c}
                2 \\ 2 \\ 2
            \end{array}
            \right)
        \end{equation*}

        Por tanto, $f$ no tiene puntos fijos, por lo que se trata de una traslación o de un movimiento helicoidal. Como $\vec{f}\neq Id_{\bb{R}^3}$, tenemos que $f$ es un movimiento helicoidal. Busquemos el eje de giro $L$, el ángulo de giro $\theta$, y el vector de deslizamiento $v$.

        Como $\vec{f}=G_{\theta, \vec{L}}$, tenemos que $\vec{L}=V_1$. Por tanto,
        \begin{equation*}
            \begin{split}
                \vec{L} &= \{ (x,y,z)\in \bb{R}^3 \mid (M(\vec{f}, \cc{B}_u) - Id) (x,y,z)^t = (0,0, 0)^t \} =\\
                &= \left\{ (x,y,z)\in \bb{R}^3 \mid \left(
                \begin{array}{ccc}
                    -\nicefrac{1}{2} & 0 & \nicefrac{-\sqrt{3}}{2} \\
                    0 & 0 & 0 \\
                    \nicefrac{\sqrt{3}}{2} & 0 & -\nicefrac{1}{2} 
                \end{array}
                \right)
                \left(
                \begin{array}{c}
                    x\\y\\z
                \end{array}
                \right)
                = \left(
                \begin{array}{c}
                    0\\0\\0
                \end{array}
                \right)
                \right\} =\\
                &= \cc{L}\left\{\left(0,1,0\right)\right\}
            \end{split}
        \end{equation*}

        Para obtener un punto del eje, sabemos que dado $(x,y,z)\in L$, entonces $\vec{(x,y,z)f(x,y,z)}\in \vec{L}$. Por tanto,
        \begin{multline*}
            \vec{(x,y,z)f(x,y,z)} = f(x,y,z) - (x,y,z) = \left(-\dfrac{x}{2} -\dfrac{\sqrt{3} z}{2} + 2, 2,\dfrac{\sqrt{3} x}{2} -\dfrac{z}{2} + 2\right) \in \vec{L} 
            \Longleftrightarrow \\ \Longleftrightarrow
            -\dfrac{x}{2} -\dfrac{\sqrt{3} z}{2} + 2 = 0 = \dfrac{\sqrt{3} x}{2} -\dfrac{z}{2} + 2 \Longleftrightarrow x=1-\sqrt{3},~~z=1+\sqrt{3}
        \end{multline*}
        Por tanto, tenemos que el eje es:
        \begin{equation*}
            L = (1-\sqrt{3}, 0,1+\sqrt{3}) + \cc{L}\left\{\left(0,1,0\right)\right\}
        \end{equation*}

        Para calcular el ángulo de giro, sabemos que:
        \begin{equation*}
            2\cos\theta +1 = tr(M(\vec{f}, \cc{B}_u)) = 2 \Longrightarrow \cos\theta = \frac{1}{2} \Longrightarrow \theta = \frac{\pi}{3}
        \end{equation*}

        Por último, tan solo falta calcular el vector de desplazamiento. Tenemos que $v=\vec{pf(p)}$ para cualquier $p\in L$. Tomando $p=(1-\sqrt{3}, 0, 1+\sqrt{3})$, tenemos que $f(p) = (1-\sqrt{3}, 2, 1+\sqrt{3})$. Por tanto,
        \begin{equation*}
            v = (0, 2, 0)
        \end{equation*}

        
        \item $f\left(x, y, z\right) = \left(-\dfrac{4x}{5} + \dfrac{3z}{5} + 3, y, \dfrac{3x}{5} + \dfrac{4z}{5} - 1\right)$.



        Tenemos que:
        \begin{equation*}
            M(f,\cc{R}_0) = \left(\begin{array}{c|ccc}
                1 & 0 & 0 & 0 \\ \hline
                3 & \nicefrac{-4}{5} & 0 & \nicefrac{3}{5} \\
                0 & 0 & 1 & 0 \\
                -1 & \nicefrac{3}{5} & 0 & \nicefrac{4}{5} 
            \end{array}\right)
        \end{equation*}

        Como $\cc{R}_0$ es un sistema de referencia ortonormal, para ver si $f$ es una isometría (equivalentemente vemos que $\vec{f}$ lo es) basta con probar que $M(\vec{f}, \cc{B}_u)\in O(3)$. Tenemos que:
        \begin{equation*}
            M(\vec{f},\cc{B}_u)M(\vec{f},\cc{B}_u)^t
            = 
            \left(
            \begin{array}{ccc}
                \nicefrac{-4}{5} & 0 & \nicefrac{3}{5} \\
                0 & 1 & 0 \\
                \nicefrac{3}{5} & 0 & \nicefrac{4}{5}
            \end{array}
            \right)
            \left(
            \begin{array}{ccc}
                \nicefrac{-4}{5} & 0 & \nicefrac{3}{5} \\
                0 & 1 & 0 \\
                \nicefrac{3}{5} & 0 & \nicefrac{4}{5}
            \end{array}
            \right)
            = 
            \left(
            \begin{array}{ccc}
                1 & 0 & 0\\
                0 & 1 & 0\\
                0 & 0 & 1
            \end{array}
            \right) = Id_3
        \end{equation*}

        Por tanto, tenemos que $\vec{f}$ es una isometría, y por tanto $f$ lo es también. Como $|\vec{f}|=-1$, tenemos que se trata de una isometría inversa. Calculemos los puntos fijos:
        \begin{equation*}
            -\left(
            \begin{array}{ccc}
                \nicefrac{-9}{5} & 0 & \nicefrac{3}{5} \\
                0 & 0 & 0 \\
                \nicefrac{3}{5} & 0 & \nicefrac{-1}{5} 
            \end{array}
            \right)
            \left(
            \begin{array}{c}
                x \\ y \\ z
            \end{array}
            \right)
            = \left(
            \begin{array}{c}
                3 \\ 0 \\ -1
            \end{array}
            \right) \Longrightarrow x=\frac{5}{3} + \frac{1}{3}z
        \end{equation*}

        Es decir, $f$ tiene un plano de puntos fijos, por lo que se trata de una simetría especular. El plano es:
        \begin{equation*}
            \pi = \left(\frac{5}{3}, 0, 0\right) + \cc{L}\{(0,1,0), (1,0,3)\}
        \end{equation*}

        
        \item $f\left(x, y, z\right) = \left(-\dfrac{4x}{5} + \dfrac{3z}{5} + 3, y + 4, \dfrac{3x}{5} + \dfrac{4z}{5} - 1\right)$.

        Tenemos que $f(x,y,z)=f'(x,y,z) + (0,4,0)$, donde $f'$ es la reflexión especular del apartado anterior.

        Por tanto, tenemos que se trata una reflexión especular en el plano $\pi = \left(\frac{5}{3}, 0, 0\right) + \cc{L}\{(0,1,0), (1,0,3)\}$ con deslizamiento según el vector $v=(0,4,0)$.
        
        \item $f\left(x, y, z\right) = \left(\dfrac{2y}{\sqrt{5}} + \dfrac{z}{\sqrt{5}},\dfrac{\sqrt{5}x}{3} - \dfrac{2y}{3\sqrt{5}} + \dfrac{4z}{3\sqrt{5}}, -\dfrac{2x}{3} - \dfrac{y}{3} + \dfrac{2z}{3}\right)$.

        Tenemos que:
        \begin{equation*}
            M(f,\cc{R}_0) = \left(\begin{array}{c|ccc}
                1 & 0 & 0 & 0 \\ \hline
                0 & 0 & \nicefrac{2}{\sqrt{5}} & \nicefrac{1}{\sqrt{5}} \\
                0 & \nicefrac{\sqrt{5}}{3} & \nicefrac{-2}{3\sqrt{5}} & \nicefrac{4}{3\sqrt{5}} \\
                0 & \nicefrac{-2}{3} & \nicefrac{-1}{3} & \nicefrac{2}{3} 
            \end{array}\right)
        \end{equation*}

        Como $\cc{R}_0$ es un sistema de referencia ortonormal, para ver si $f$ es una isometría (equivalentemente vemos que $\vec{f}$ lo es) basta con probar que $M(\vec{f}, \cc{B}_u)\in O(3)$. Tenemos que:
        \begin{equation*}
            \left(
            \begin{array}{ccc}
                0 & \nicefrac{2}{\sqrt{5}} & \nicefrac{1}{\sqrt{5}} \\
                \nicefrac{\sqrt{5}}{3} & \nicefrac{-2}{3\sqrt{5}} & \nicefrac{4}{3\sqrt{5}} \\
                \nicefrac{-2}{3} & \nicefrac{-1}{3} & \nicefrac{2}{3}
            \end{array}
            \right)
            \left(
            \begin{array}{ccc}
                0 & \nicefrac{\sqrt{5}}{3} &  \nicefrac{-2}{3}\\
                \nicefrac{2}{\sqrt{5}} & \nicefrac{-2}{3\sqrt{5}} & \nicefrac{-1}{3} \\
                \nicefrac{1}{\sqrt{5}} & \nicefrac{4}{3\sqrt{5}} & \nicefrac{2}{3}
            \end{array}
            \right)
            = 
            \left(
            \begin{array}{ccc}
                1 & 0 & 0\\
                0 & 1 & 0\\
                0 & 0 & 1
            \end{array}
            \right) = Id_3
        \end{equation*}

        Por tanto, tenemos que $\vec{f}$ es una isometría, y por tanto $f$ lo es también. Como $|\vec{f}|=-1$, tenemos que se trata de una isometría inversa. Además, como $f(0,0,0)=(0,0,0)$, sabemos que al menos hay un punto fijo. Veamos si hay más:
        \begin{equation*}
            -\left(
            \begin{array}{ccc}
                -1 & \nicefrac{2}{\sqrt{5}} & \nicefrac{1}{\sqrt{5}} \\
                \nicefrac{\sqrt{5}}{3} & \nicefrac{-2}{3\sqrt{5}}-1 & \nicefrac{4}{3\sqrt{5}} \\
                \nicefrac{-2}{3} & \nicefrac{-1}{3} & \nicefrac{-1}{3}
            \end{array}
            \right)
            \left(
            \begin{array}{c}
                x \\ y \\ z
            \end{array}
            \right)
            = \left(
            \begin{array}{c}
                0 \\ 0 \\ 0
            \end{array}
            \right)
        \end{equation*}
        Como el rango de la matriz de coeficientes es 3, tenemos que el sistema es SCD, por lo que tan solo hay una solución, que sabemos que es el origen. Por tanto, se trata de un giro con simetría. Sabemos que el ángulo de giro $\theta$ cumple la siguiente condición:
        \begin{equation*}
            2\cos\theta -1 = tr(M(\vec{f},\cc{B}_u)) = \frac{2}{3} - \frac{2}{3\sqrt{5}} \Longrightarrow \cos\theta = \frac{25 -2\sqrt{5}}{30} \Longrightarrow \theta \approx 0.817\text{ rad}.
        \end{equation*}

        Calculemos ahora el eje $L$. Sabemos que $\vec{f}$ es un giro con simetría vectoriales, y $\vec{L}=V_{-1}$. Entonces:
        \begin{equation*}
            \begin{split}
                \vec{L} &= \{ (x,y,z)\in \bb{R}^3 \mid (M(\vec{f}, \cc{B}_u) + Id) (x,y,z)^t = (0,0, 0)^t \} =\\
                &= \left\{ (x,y,z)\in \bb{R}^3 \mid \left(
                \begin{array}{ccc}
                    1 & \nicefrac{2}{\sqrt{5}} & \nicefrac{1}{\sqrt{5}} \\
                    \nicefrac{\sqrt{5}}{3} & \nicefrac{-2}{3\sqrt{5}}+1 & \nicefrac{4}{3\sqrt{5}} \\
                    \nicefrac{-2}{3} & \nicefrac{-1}{3} & \nicefrac{5}{3}
                \end{array}
                \right)
                \left(
                \begin{array}{c}
                    x\\y\\z
                \end{array}
                \right)
                = \left(
                \begin{array}{c}
                    0\\0\\0
                \end{array}
                \right)
                \right\} =\\
                &= \cc{L}\left\{\left(\sqrt{5}+4, -2\sqrt{5}-3, 1\right)\right\}
            \end{split}
        \end{equation*}

        Por tanto, tenemos que el eje de giro es:
        \begin{equation*}
            L = (0,0,0) + \cc{L}\left\{\left(\sqrt{5}+4, -2\sqrt{5}-3, 1\right)\right\}
        \end{equation*}

        
        \item $f\left(x, y, z\right) = \left(\dfrac{\sqrt{5} x}{3} - \dfrac{2y}{3\sqrt{5}} + \dfrac{4z}{3\sqrt{5}}, \dfrac{2y}{\sqrt{5}} + \dfrac{z}{\sqrt{5}}, -\dfrac{2x}{3}- \dfrac{y}{3} + \dfrac{2z}{3}\right)$.

        Tenemos que:
        \begin{equation*}
            M(f,\cc{R}_0) = \left(\begin{array}{c|ccc}
                1 & 0 & 0 & 0 \\ \hline
                0 & \nicefrac{\sqrt{5}}{3} & \nicefrac{-2}{3\sqrt{5}} & \nicefrac{4}{3\sqrt{5}} \\
                0 & 0 & \nicefrac{2}{\sqrt{5}} & \nicefrac{1}{\sqrt{5}} \\
                0 & \nicefrac{-2}{3} & \nicefrac{-1}{3} & \nicefrac{2}{3} 
            \end{array}\right)
        \end{equation*}

        Como $\cc{R}_0$ es un sistema de referencia ortonormal, para ver si $f$ es una isometría (equivalentemente vemos que $\vec{f}$ lo es) basta con probar que $M(\vec{f}, \cc{B}_u)\in O(3)$. Tenemos que:
        \begin{equation*}
            \left(
            \begin{array}{ccc}
                \nicefrac{\sqrt{5}}{3} & \nicefrac{-2}{3\sqrt{5}} & \nicefrac{4}{3\sqrt{5}} \\
                0 & \nicefrac{2}{\sqrt{5}} & \nicefrac{1}{\sqrt{5}} \\
                \nicefrac{-2}{3} & \nicefrac{-1}{3} & \nicefrac{2}{3} 
            \end{array}
            \right)
            \left(
            \begin{array}{ccc}
                \nicefrac{\sqrt{5}}{3} & 0 & \nicefrac{-2}{3} \\
                \nicefrac{-2}{3\sqrt{5}} & \nicefrac{2}{\sqrt{5}} & \nicefrac{-1}{3} \\
                \nicefrac{4}{3\sqrt{5}} & \nicefrac{1}{\sqrt{5}} & \nicefrac{2}{3}
            \end{array}
            \right)
            = 
            \left(
            \begin{array}{ccc}
                1 & 0 & 0\\
                0 & 1 & 0\\
                0 & 0 & 1
            \end{array}
            \right) = Id_3
        \end{equation*}

        Por tanto, tenemos que $\vec{f}$ es una isometría, y por tanto $f$ lo es también. Como $|\vec{f}|=1$, tenemos que se trata de una isometría directa. Además, como $f(0,0,0)=(0,0,0)$, sabemos que al menos hay un punto fijo. Veamos si hay más:
        \begin{equation*}
            -\left(
            \begin{array}{ccc}
                \nicefrac{\sqrt{5}}{3}-1 & \nicefrac{-2}{3\sqrt{5}} & \nicefrac{4}{3\sqrt{5}} \\
                0 & \nicefrac{2}{\sqrt{5}}-1 & \nicefrac{1}{\sqrt{5}} \\
                \nicefrac{-2}{3} & \nicefrac{-1}{3} & \nicefrac{-1}{3} 
            \end{array}
            \right)
            \left(
            \begin{array}{c}
                x \\ y \\ z
            \end{array}
            \right)
            = \left(
            \begin{array}{c}
                0 \\ 0 \\ 0
            \end{array}
            \right) \Longrightarrow \left\{\begin{array}{l}
                x = \frac{-\sqrt{5}-3}{2} \cdot \lm \\
                y = (\sqrt{5}+2)\cdot \lm
                z = \lm
            \end{array}\right.
        \end{equation*}

        Por tanto, tenemos que hay una recta de puntos fijos, $$L=(0,0,0)+\cc{L}\left\{\left(\frac{-\sqrt{5}-3}{2}, \sqrt{5}+2, 1\right)\right\}$$

        Veamos ahora el ángulo de giro $\theta$. Tenemos que:
        \begin{equation*}
            2\cos\theta +1 = tr(M(\vec{f}, \cc{B}_u)) = \frac{10+11\sqrt{5}}{15} \Longrightarrow \cos\theta = \frac{-5+11\sqrt{5}}{30} \Longrightarrow \theta \approx 0.86\text{ rad.}
        \end{equation*}        
    \end{enumerate}
\end{ejercicio}

\begin{ejercicio}
    Calcula en coordenadas usuales de $\bb{R}^2$ el giro centrado en el punto $c = (1, 2)$ y de ángulo $\frac{2\pi}{3}$.

    Sea el sistema de referencia $\cc{R}=\{c, \cc{B}_u\}$, y sea $\theta = \frac{2\pi}{3}$. Entonces, tenemos que:
    \begin{equation*}
        M(f, \cc{R}) = \left(\begin{array}{c|cc}
            1 & 0 & 0 \\ \hline
            0 & \cos\theta & -\sen\theta \\
            0 & \sen\theta & \cos\theta
        \end{array}\right)
        = \left(\begin{array}{c|cc}
            1 & 0 & 0 \\ \hline
            0 & \nicefrac{-1}{2} & \nicefrac{-\sqrt{3}}{2} \\
            0 & \nicefrac{\sqrt{3}}{2} & \nicefrac{-1}{2}
        \end{array}\right)
    \end{equation*}

    Para expresarlo en coordenadas usuales, tenemos que:
    \begin{equation*}
        \begin{split}
            M(f,\cc{R}_0) &= M(Id_{\bb{R}^2}, \cc{R}, \cc{R}_0) \cdot M(f,\cc{R}) \cdot M(Id_{\bb{R}^2}, \cc{R}_0, \cc{R}) =\\
            &= M(Id_{\bb{R}^2}, \cc{R}, \cc{R}_0) \cdot M(f,\cc{R}) \cdot M(Id_{\bb{R}^2}, \cc{R}, \cc{R}_0)^{-1} =\\
            &=\left(\begin{array}{c|cc}
                1 & 0 & 0 \\ \hline
                1 & 1 & 0 \\
                2 & 0 & 1
            \end{array}\right)
            \left(\begin{array}{c|cc}
                1 & 0 & 0 \\ \hline
                0 & \nicefrac{-1}{2} & \nicefrac{-\sqrt{3}}{2} \\
                0 & \nicefrac{\sqrt{3}}{2} & \nicefrac{-1}{2}
            \end{array}\right)
            \left(\begin{array}{c|cc}
                1 & 0 & 0 \\ \hline
                1 & 1 & 0 \\
                2 & 0 & 1
            \end{array}\right)^{-1} =\\
            &= \left(\begin{array}{c|cc}
                1 & 0 & 0 \\ \hline
                \nicefrac{2\sqrt{3}+3}{2} & \nicefrac{-1}{2} & \nicefrac{-\sqrt{3}}{2} \\
                \nicefrac{-\sqrt{3}+6}{2} & \nicefrac{\sqrt{3}}{2} & \nicefrac{-1}{2}
            \end{array}\right)
        \end{split}
    \end{equation*}
\end{ejercicio}

\begin{ejercicio}
    Calcula la simetría con deslizamiento respecto de la recta dada por la ecuación $L\equiv x - y = 1$ de $\bb{R}^2$ y vector de desplazamiento $v = (-2, -2)$.\\

    Tenemos que $L=(1,0)+\cc{L}\{(1,1)\}$, por lo que $\vec{L}^\perp = \cc{L}\{(1,-1)\}$. Sea entonces el sistema de referencia $\cc{R}=\{(1,0), \cc{B}=\{(1,1), (1,-1)\}\}$.

    Tenemos que $f(1,0) = (1,0) + (-2, -2) = (-1, -2)$. Calculamos sus coordenadas en $\cc{R}$:
    \begin{equation*}
        (-1, -2) = (1, 0) + \alpha (1,1) + \beta(1, -1) = (1+\alpha + \beta, \alpha-\beta)
    \end{equation*}
    Por tanto, $\alpha=-2$, $\beta=0$, por lo que $f(1,0) = (-2, 0)_{\cc{R}}$. Entonces,
    \begin{equation*}
        M(f, \cc{R}) =
        \left(\begin{array}{c|cc}
            1 & 0 & 0 \\ \hline
            -2 & 1 & 0 \\
            0 & 0 & -1
        \end{array}\right)
    \end{equation*}


    Para expresarlo en coordenadas usuales (aunque en este caso no lo piden), tenemos que:
    \begin{equation*}
        \begin{split}
            M(f,\cc{R}_0) &= M(Id_{\bb{R}^2}, \cc{R}, \cc{R}_0) \cdot M(f,\cc{R}) \cdot M(Id_{\bb{R}^2}, \cc{R}_0, \cc{R}) =\\
            &= M(Id_{\bb{R}^2}, \cc{R}, \cc{R}_0) \cdot M(f,\cc{R}) \cdot M(Id_{\bb{R}^2}, \cc{R}, \cc{R}_0)^{-1} =\\
            &=\left(\begin{array}{c|cc}
                1 & 0 & 0 \\ \hline
                1 & 1 & 1 \\
                0 & 1 & -1
            \end{array}\right)
            \left(\begin{array}{c|cc}
                1 & 0 & 0 \\ \hline
                -2 & 1 & 0 \\
                0 & 0 & -1
            \end{array}\right)
            \left(\begin{array}{c|cc}
                1 & 0 & 0 \\ \hline
                1 & 1 & 1 \\
                0 & 1 & -1
            \end{array}\right)^{-1} =\\
            &= \left(\begin{array}{c|cc}
                1 & 0 & 0 \\ \hline
                -1 & 0 & 1 \\
                -3 & 1 & 0
            \end{array}\right)
        \end{split}
    \end{equation*}
\end{ejercicio}

\begin{ejercicio}
    Sea $f : \bb{R}^2\to \bb{R}^2$ la aplicación afín dada por
    \begin{equation*}
        f(-1, -1) = (0, 0),\qquad  f(-1, -2) = (1, 0),\qquad f(0, -1) = (0, 1).
    \end{equation*}
    Demuestra que $f$ es un movimiento rígido y clasifícalo.\\

    Definimos en primer lugar el sistema de referencia dado por $\cc{R}=\{(-1,-1), (-1,-2), (0,-1)\}$. Tenemos que su base asociada es $\cc{B} = \{(0, -1), (1, 0)\}$. Calculemos las imágenes de los vectores de la base mediante $\vec{f}$:
    \begin{gather*}
        \vec{f}(0, -1) = \vec{f(-1, -1)f(-1,-2)} = \vec{(0,0)(1,0)} = (1,0) \\
        \vec{f}(1,0) = \vec{f(-1, -1)f(0, -1)} = \vec{(0,0)(0,1)} = (0,1)
    \end{gather*}

    Tenemos que su matriz asociada es:
    \begin{equation*}
        M(f, \cc{R}, \cc{R}_0) = \left(\begin{array}{c|cc}
            1 & 0 & 0 \\ \hline
            0 & 1 & 0 \\
            0 & 0 & 1
        \end{array}\right)
    \end{equation*}

    En el sistema de referencia usual, tenemos que es:
    \begin{equation*}
        \begin{split}
            M(f, \cc{R}_0) &= M(f, \cc{R}, \cc{R}_0) \cdot M(Id_{\bb{R}^2}, \cc{R}_0, \cc{R}) =\\
            &= M(f, \cc{R}, \cc{R}_0) \cdot M(Id_{\bb{R}^2}, \cc{R}, \cc{R}_0)^{-1} =\\
            &=\left(\begin{array}{c|cc}
                1 & 0 & 0 \\ \hline
                0 & 1 & 0 \\
                0 & 0 & 1
            \end{array}\right)
            \left(\begin{array}{c|cc}
                1 & 0 & 0 \\ \hline
                -1 & 0 & 1 \\
                -1 & -1 & 0
            \end{array}\right)^{-1}
            = \left(\begin{array}{c|cc}
                1 & 0 & 0 \\ \hline
                -1 & 0 & -1 \\
                1 & 1 & 0
            \end{array}\right)
        \end{split}
    \end{equation*}

    Es fácil comprobar que cumple las tres condiciones dadas por el enunciado. Como $\cc{R}_0$ es un sistema de referencia ortonormal, tenemos que para comprobar que $f$ es un movimiento rígido basta con comprobar que $M(\vec{f}, \cc{B}_u)\in O(2)$. Tenemos que:
    \begin{equation*}
        M(\vec{f}, \cc{B}_u)M(\vec{f}, \cc{B}_u)^t =
        \left(\begin{array}{cc}
                0 & -1 \\
                1 & 0
            \end{array}\right)
            \left(\begin{array}{cc}
                0 & 1 \\
                -1 & 0
            \end{array}\right) = Id_2
    \end{equation*}

    Por tanto, tenemos que es una isometría. Como $|\vec{f}| = 1$, tenemos que $f$ es una isometría directa. Calculemos los puntos fijos:
    \begin{equation*}
        -\left(\begin{array}{cc}
            -1 & -1 \\
            1 & -1
        \end{array}\right)
        \left(\begin{array}{c}
            x\\y
        \end{array}\right)
        = \left(\begin{array}{c}
            -1 \\1
        \end{array}\right) \Longrightarrow \left\{\begin{array}{l}
            x = -1\\
            y=0
        \end{array}\right.
    \end{equation*}

    Es decir, es una isometría directa con un único punto fijo, por lo que se trata de un giro de ángulo $\theta \neq 0$. Calculemos su ángulo de giro:
    \begin{equation*}
        2\cos\theta = tr(M(\vec{f}, \cc{B}_u)) \Longrightarrow \cos\theta  = 0 \Longrightarrow \theta = \frac{\pi}{2}
    \end{equation*}

    Es decir, se trata de giro centrado en el punto $(-1,0)$ y de ángulo $\frac{\pi}{2}$ rad.
\end{ejercicio}

\begin{ejercicio}
    Sea $\cc{R}$ el sistema de referencia de $\bb{R}^2$ con origen en el punto $(1, 1)$ y base asociada $\{(1, 1),(-1, 1)\}$. Consideremos la aplicación afín $f$ tal que, si $(x, y)$ son las coordenadas de un punto genérico $p$ en el sistema de referencia $\cc{R}$, entonces las coordenadas de $f(p)$ en el sistema de referencia usual vienen dadas por
    \begin{equation*}
        \left(\begin{array}{c}
            -1 \\ 3
        \end{array}\right)+
        \left(\begin{array}{cc}
            1 & 1 \\ 1 & -1
        \end{array}\right)
        \left(\begin{array}{c}
            x \\ y
        \end{array}\right).
    \end{equation*}
    ¿Es $f$ un movimiento rígido? En caso afirmativo, clasifícalo.\\

    Tenemos que $\cc{R}=\{(1,1), \cc{B}\}$, con $\cc{B} = \{(1, 1),(-1, 1)\}$. La matriz asociada de $f$ viene dada por:
    \begin{equation*}
        M(f, \cc{R}, \cc{R}_0) = \left(\begin{array}{c|cc}
            1 & 0 & 0 \\ \hline
            -1 & 1 & 1 \\
            3 & 1 & -1
        \end{array}\right)
    \end{equation*}

    En el sistema de referencia usual, tenemos que es:
    \begin{equation*}
        \begin{split}
            M(f, \cc{R}_0) &= M(f, \cc{R}, \cc{R}_0) \cdot M(Id_{\bb{R}^2}, \cc{R}_0, \cc{R}) =\\
            &= M(f, \cc{R}, \cc{R}_0) \cdot M(Id_{\bb{R}^2}, \cc{R}, \cc{R}_0)^{-1} =\\
            &=\left(\begin{array}{c|cc}
                1 & 0 & 0 \\ \hline
                -1 & 1 & 1 \\
                3 & 1 & -1
            \end{array}\right)
            \left(\begin{array}{c|cc}
                1 & 0 & 0 \\ \hline
                1 & 1 & -1 \\
                1 & 1 & 1
            \end{array}\right)^{-1}
            = \left(\begin{array}{c|cc}
                1 & 0 & 0 \\ \hline
                -2 & 0 & 1 \\
                2 & 1 & 0
            \end{array}\right)
        \end{split}
    \end{equation*}

    Claramente, $\vec{f}$ es una isometría; y como $|\vec{f}| = -1$, tenemos que es una isometría inversa. Calculemos sus puntos fijos:
    \begin{equation*}
        -\left(\begin{array}{cc}
            -1 & 1 \\
            1 & -1
        \end{array}\right)
        \left(\begin{array}{c}
            x\\y
        \end{array}\right) = 
        \left(\begin{array}{c}
            -2\\2
        \end{array}\right)
    \end{equation*}

    Por tanto, tenemos que hay una recta de puntos fijos: $L\equiv x-y=-2$. Es decir, $L = (-1,1) + \cc{L}\{(1,1)\}$. Como $f$ es una isometría inversa, tenemos que $f$ es una reflexión axial sobre la recta $L$.
\end{ejercicio}

\begin{ejercicio}
    Sean $f_1, f_2 : \bb{R}^3\to \bb{R}^3$ las isometrías dadas respectivamente por las simetrías respecto de los planos de ecuaciones $x + y = 1$ y $x - z = 2$.
    \begin{enumerate}
        \item Calcula explícitamente $f_1$ y $f_2$ en coordenadas usuales.

        Sea $f_1$ la simetría especular respecto de $\pi_1=(0,1,1) + \cc{L}\{(1,-1,0), (0,0,1)\}$. Tenemos que $\vec{\pi_1}^\perp = \cc{L}\{(1,1,0)\}$; por lo que consideramos el sistema de referencia $\cc{R}=\{(0,1,1), \cc{B}\}$, con $\cc{B}=\{(1,-1,0), (0,0,1), (1,1,0)\}$. Tenemos que:
        \begin{equation*}
            M(f_1,\cc{R}) = \left(
            \begin{array}{c|ccc}
                1 & 0 & 0 & 0 \\ \hline
                0 & 1 & 0 & 0 \\
                0 & 0 & 1 & 0 \\
                0 & 0 & 0 & -1
            \end{array}
            \right)
        \end{equation*}

        En las coordenadas usuales, esta es:
        \begin{equation*}
            \begin{split}
                M(f_1,\cc{R}_0) &= M(Id_{\bb{R}^3},\cc{R}, \cc{R}_0) M(f_1,\cc{R}) M(Id_{\bb{R}^3},\cc{R}_0, \cc{R}) =\\
                &= 
                \left(
                \begin{array}{c|ccc}
                    1 & 0 & 0 & 0 \\ \hline
                    0 & 1 & 0 & 1 \\
                    1 & -1 & 0 & 1 \\
                    1 & 0 & 1 & 0
                \end{array}
                \right)
                \left(
                \begin{array}{c|ccc}
                    1 & 0 & 0 & 0 \\ \hline
                    0 & 1 & 0 & 0 \\
                    0 & 0 & 1 & 0 \\
                    0 & 0 & 0 & -1
                \end{array}
                \right)
                \left(
                \begin{array}{c|ccc}
                    1 & 0 & 0 & 0 \\ \hline
                    0 & 1 & 0 & 1 \\
                    1 & -1 & 0 & 1 \\
                    1 & 0 & 1 & 0
                \end{array}
                \right)^{-1}
                =\\
                &=  \left(
                \begin{array}{c|ccc}
                    1 & 0 & 0 & 0 \\ \hline
                    1 & 0 & -1 & 0 \\
                    1 & -1 & 0 & 0 \\
                    0 & 0 & 0 & 1
                \end{array}
                \right)
            \end{split} 
        \end{equation*}



        Sea $f_2$ la simetría especular respecto de $\pi_2=(1,0,-1) + \cc{L}\{(1,0,1), (0,1,0)\}$. Tenemos que $\vec{\pi_2}^\perp = \cc{L}\{(1,0,-1)\}$; por lo que consideramos el sistema de referencia $\cc{R}'=\{(1,0,-1), \cc{B}'\}$, con $\cc{B}'=\{(1,0,1), (0,1,0), (1,0,-1)\}$. Tenemos que:
        \begin{equation*}
            M(f_2,\cc{R}') = \left(
            \begin{array}{c|ccc}
                1 & 0 & 0 & 0 \\ \hline
                0 & 1 & 0 & 0 \\
                0 & 0 & 1 & 0 \\
                0 & 0 & 0 & -1
            \end{array}
            \right)
        \end{equation*}

        En las coordenadas usuales, esta es:
        \begin{equation*}
            \begin{split}
                M(f_2,\cc{R}_0) &= M(Id_{\bb{R}^3},\cc{R}', \cc{R}_0) M(f_2,\cc{R}') M(Id_{\bb{R}^3},\cc{R}_0, \cc{R}') =\\
                &= 
                \left(
                \begin{array}{c|ccc}
                    1 & 0 & 0 & 0 \\ \hline
                    1 & 1 & 0 & 1 \\
                    0 & 0 & 1 & 0 \\
                    -1 & 1 & 0 & -1
                \end{array}
                \right)
                \left(
                \begin{array}{c|ccc}
                    1 & 0 & 0 & 0 \\ \hline
                    0 & 1 & 0 & 0 \\
                    0 & 0 & 1 & 0 \\
                    0 & 0 & 0 & -1
                \end{array}
                \right)
                \left(
                \begin{array}{c|ccc}
                    1 & 0 & 0 & 0 \\ \hline
                    1 & 1 & 0 & 1 \\
                    0 & 0 & 1 & 0 \\
                    -1 & 1 & 0 & -1
                \end{array}
                \right)^{-1}
                =\\
                &=  \left(
                \begin{array}{c|ccc}
                    1 & 0 & 0 & 0 \\ \hline
                    2 & 0 & 0 & 1 \\
                    0 & 0 & 1 & 0 \\
                    -2 & 1 & 0 & 0
                \end{array}
                \right)
            \end{split} 
        \end{equation*}
        
        \item Clasifica el movimiento rígido $g = f_1 \circ f_2$.

        Tenemos que $|g| = |f_1|\cdot |f_2| = 1$, por lo que se trata de una isometría directa en el espacio. Además, tenemos que la recta dada por la intersección de ambos planos es una recta de puntos fijos. Es decir, la recta $L$ dada por:
        \begin{equation*}
            L = \left\{(x,y,z)\in \bb{R}^3 \left|\begin{array}{l}
                x+y=1\\
                x-z=2
            \end{array}\right.\right\} = (0, 1, -2) + \cc{L}\{(1,-1,1)\}
        \end{equation*}
        es una recta de puntos fijos. Como $\vec{f}\neq Id_{\bb{R}^3}$, tenemos que no hay más puntos fijos, por lo que se trata de un giro sobre el eje $L$. Para obtener el ángulo de giro, obtenemos $M(\vec{g},\cc{B}_u)$:
        \begin{equation*}
            \begin{split}
                M(\vec{g},\cc{B}_u) &= M(\vec{f_1\circ f_2},\cc{B}_u) = M(\vec{f_1}\circ \vec{f_2},\cc{B}_u) = M(\vec{f_1},\cc{B}_u)\cdot M(\vec{f_2},\cc{B}_u) =\\
                &= \left(
                \begin{array}{ccc}
                    0 & -1 & 0 \\
                    -1 & 0 & 0 \\
                    0 & 0 & 1
                \end{array}
                \right)
                \left(
                \begin{array}{ccc}
                    0 & 0 & 1 \\
                    0 & 1 & 0 \\
                    1 & 0 & 0
                \end{array}
                \right)
                = \left(
                \begin{array}{ccc}
                    0 & -1 & 0 \\
                    0 & 0 & -1 \\
                    1 & 0 & 0
                \end{array}
                \right)
            \end{split}
        \end{equation*}

        Por tanto, si $\theta$ es el ángulo de giro, tenemos que:
        \begin{equation*}
             1+2\cos\theta = tr(M(\vec{g}, \cc{B}_u)) = 0\Longrightarrow \cos\theta = -\frac{1}{2} \Longrightarrow \theta = \frac{2\pi}{3}
        \end{equation*}
    \end{enumerate}
\end{ejercicio}

\begin{ejercicio}
    Calcula en coordenadas usuales la isometría de $\bb{R}^3$ dada por el movimiento helicoidal alrededor de la recta $R = (1, 2, 1) + \cc{L}\{(1, 0, -1)\}$ con giro de ángulo $\theta = \frac{\pi}{2}$ y vector de traslación $v = (-2, 0, 2)$.\\

    Sea $\vec{R}^\perp = \cc{L}\{(0,1,0), (1,0,1)\}$. Consideramos entonces el sistema de referencia dado por $\cc{R}=\{(1,2,1), \cc{B}\}$, con $\cc{B}=\{(1,0,-1), (0,1,0), (1,0,1)\}$. Calculemos las coordenadas de $v$ en $\cc{B}$:
    \begin{equation*}
        v = (-2,0,2) = \alpha(1,0,-1) +\beta(0,1,0) + \gamma(1,0,1)
    \end{equation*}

    Entonces, $\alpha=-2$, $\beta=0$ $\gamma=0$, por lo que $v=(-2,0,0)_{\cc{B}}$. Entonces, tenemos que:
    \begin{equation*}
        M(f, \cc{R}) = \left(
        \begin{array}{c|ccc}
            1 & 0 & 0 & 0 \\ \hline
            -2 & 1 & 0 & 0 \\
            0 & 0 & \cos\theta & -\sen\theta \\
            0 & 0 & \sen\theta & \cos\theta
        \end{array}
        \right)
    \end{equation*}
\end{ejercicio}

\begin{ejercicio}
     Clasifica el siguiente movimiento rígido de $\bb{R}^3$:
     \begin{equation*}
         f(x, y, z) = \frac{1}{3}(2x + 2y + z + 2, x - 2y + 2z - 2, 2x - y - 2z - 4)
     \end{equation*}
     y calcula sus elementos notables (o geométricos).
\end{ejercicio}

\begin{ejercicio}
     Calcula la simetría con deslizamiento respecto del plano de ecuación $x + y + z = 1$ de $\bb{R}^3$ y con vector de traslación $v = (2, -1, -1)$.
\end{ejercicio}

\begin{ejercicio}
     Clasifica el movimiento rígido de $\bb{R}^3$ dado por
     \begin{equation*}
         f(x,y,z) = \left(
         \frac{2x+2y+z+3}{3}, \frac{-2x+y+2z}{3}, \frac{-x+2y-2z-3}{3}
         \right)
     \end{equation*}
\end{ejercicio}

\begin{ejercicio}
    Calcula la isometría de $\bb{R}^3$ dada por la composición de un giro de ángulo $\frac{\pi}{2}$ respecto del eje $R \equiv (1, 2, 0) + \cc{L}(\{(0, 1, 0)\})$ y la simetría respecto del plano $y = -1$.
\end{ejercicio}

\begin{ejercicio}
    Para cada $\alpha \in \bb{R}$ se considera el movimiento rígido de $\bb{R}^3$ dado por:
    \begin{equation*}
        f_\alpha(x,y,z)=\frac{1}{3}(-x + 2y + 2z + 2, 2x + 2y - z - 1, 2x - y + 2z - \alpha).
    \end{equation*}
    Clasificar, según los valores de $\alpha$, qué tipo de movimiento es $f_\alpha$, calculando en cada caso el conjunto de puntos fijos.
\end{ejercicio}

\begin{ejercicio}
    Sea $\cc{R}$ es el sistema de referencia con origen en el punto $(1, 0, 1)$ y base asociada $\{(1, 0, 0),(1, 1, 0),(1, 1, 1)\}$. Determina si la siguiente aplicación afín $f : \bb{R}^3\to \bb{R}^3$, que en coordenadas respecto del sistema de referencia afín $\cc{R}$ está dada por
    \begin{equation*}
        f\left(\begin{array}{c}
            x \\ y \\ z
        \end{array}\right) = \frac{1}{5}
        \left[
        \left(\begin{array}{c}
            -9 \\ 16 \\ -7
        \end{array}\right) +
        \left(\begin{array}{ccc}
            5 & 2 & -2 \\
            0 & 7 & 8 \\
            0 & -4 & -1
        \end{array}\right)
        \left(\begin{array}{c}
            x \\ y \\ z
        \end{array}\right)
        \right]
    \end{equation*}
    es un movimiento rígido y, en caso afirmativo, clasifícalo.
\end{ejercicio}

\begin{ejercicio}
    Decide de forma razonada qué tipo de movimiento rígido es:
    \begin{enumerate}
        \item La composición de dos simetrías ortogonales en el plano euclídeo $\bb{R}^2$.
        \item La composición de dos simetrías ortogonales con deslizamiento en el plano euclídeo $\bb{R}^2$.
        \item La composición de un giro y una simetría en el plano euclídeo $\bb{R}^2$.
        \item La composición de un giro y una simetría con deslizamiento en el plano euclídeo~$\bb{R}^2$.
        \item La composición de dos simetrías ortogonales en el espacio euclídeo $\bb{R}^3$.
        \item La composición de un giro y una simetría en el espacio euclídeo $\bb{R}^3$.
        \item La composición de un giro y una traslación en el espacio euclídeo $\bb{R}^3$.
        \item La composición de dos simetrías centrales en el espacio euclídeo $\bb{R}^3$.
    \end{enumerate}
\end{ejercicio}

\begin{ejercicio}
    Demuestra que la composición de dos simetrías respecto de dos puntos distintos es una traslación.
\end{ejercicio}

\begin{ejercicio}
    Sea $T$ un triángulo en un espacio afín euclídeo $\cc{A}$ con vértices $a, b, c \in \cc{A}$. La recta que pasa por el vértice $a$ y con vector director
    \begin{equation*}
        v_a=\frac{1}{\|\vec{ab}\|} \vec{ab} + \frac{1}{\|\vec{ac}\|}\vec{ac}
    \end{equation*}
    la llamamos bisectriz que pasa por $a$. Si se definen de manera análoga las bisectrices que pasan por los vértices $b$ y $c$, prueba que las tres rectas se cortan en un mismo punto, que llamaremos incentro del triángulo.
\end{ejercicio}

\begin{ejercicio}
    Calcula el baricentro, ortocentro, circuncentro e incentro del triángulo de $\bb{R}^2$ que tiene por vértices a los puntos $(0, 0)$,$(1, 0)$ y $(0, 1)$.
\end{ejercicio}

\begin{ejercicio}
    ¿Está el incentro de cualquier triángulo alineado con el baricentro, ortocentro y circuncentro?
\end{ejercicio}

\begin{ejercicio}[Ejercicio de Examen 2022-23]
    Demostrar que, en un triángulo isósceles, los 4 puntos notables de un triángulo están alineados.


    Tenemos que ver que los 4 puntos están en una misma recta. Esto se debe a que la misma altura
\end{ejercicio}