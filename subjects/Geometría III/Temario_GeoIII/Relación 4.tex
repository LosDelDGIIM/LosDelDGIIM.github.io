\section{Espacio Proyectivo.}\label{Rel:Tema4}


\begin{ejercicio}
    En el plano proyectivo $\bb{P}^2$, consideramos las rectas:
    \begin{equation*}
        R = \left\{ (x_0:x_1:x_2) \in \bb{P}^2 \mid x_0+x_1 = 0 \right\}, \quad S = \left\{ (x_0:x_1:x_2) \in \bb{P}^2 \mid x_1+x_2 = 0 \right\}
    \end{equation*}
    y el punto $p_0=(1:1:1)\in \bb{P}^2$. Calcular la aplicación $f:R \to S$ tal que a cada
    punto $p\in R$ le hace corresponder el único punto de corte entre
    las rectas $p_0+p$ y $S$. ¿Es $f$ una proyectividad de $R$ en $S$?\\

    Sabemos que un punto genérico de $R$ es de la forma $p=(\alpha:-\alpha:\beta)$, por lo que el
    espacio vectorial asociado a $p_0+p$ es:
    \begin{equation*}
        \wt{p_0+p} = \cc{L}\{(1,1,1), (\alpha, -\alpha, \beta)\}
    \end{equation*}

    Por tanto, la ecuación implícita de $\wt{p_0+p}$ es:
    \begin{equation*}
        \begin{vmatrix}
            1 & \alpha & x_0 \\
            1 & -\alpha & x_1 \\
            1 & \beta & x_2
        \end{vmatrix} = 0
        = -\alpha x_2 + \alpha x_1 + \beta x_0 + \alpha x_0 - \beta x_1 - \alpha x_2
        = (\alpha+\beta)x_0 + (\alpha-\beta)x_1 -2 \alpha x_2
    \end{equation*}

    Calculamos por tanto la intersección de $\wt{p_0+p}$ con $\wt{S}$:
    \begin{multline*}
        \left\{
            \begin{array}{ll}
                (\alpha+\beta)x_0 + (\alpha-\beta)x_1 -2 \alpha x_2 &= 0 \\
                x_1+x_2 &= 0
            \end{array}
        \right\}
        \Longrightarrow
        \left\{
            \begin{array}{ll}
                (\alpha+\beta)x_0 + (\alpha-\beta)x_1 +2 \alpha x_1 &= 0 \\
                x_2 &= -x_1
            \end{array}
        \right\}
        \Longrightarrow \\ \Longrightarrow
        \left\{
            \begin{array}{ll}
                (\alpha+\beta)x_0 + (3\alpha-\beta)x_1&= 0 \\
                x_2 &= -x_1
            \end{array}
        \right\}
        \Longrightarrow
        \left\{
            \begin{array}{ll}
                x_0 = \frac{3\alpha-\beta}{\alpha+\beta} \lm \\
                x_1 = -\lm \\
                x_2 = \lm
            \end{array} 
        \right. \qquad \lm \in \bb{R}
    \end{multline*}

    Por tanto, deducimos que:
    \begin{equation*}
        f(p) = f(\alpha:-\alpha:\beta)
        = (p_0+p)\cap S = \left( \frac{3\alpha-\beta}{\alpha+\beta} : -1 : 1 \right)
    \end{equation*}

    Para ver si es una proyectividad, veremos si existe una aplicación lineal
    inyectiva $\wt{f}:\wt{R}\to \wt{S}$ tal que $f([v]) = \left[\wt{f}(v)\right]$.
    Veamos si la siguiente aplicación es inyectiva:
    \begin{equation*}
        \vec{f}(\alpha, -\alpha, \beta) = \left( 3\alpha-\beta, -\alpha-\beta, {\alpha+\beta} \right)
    \end{equation*}

    Para ver que lo es, basta ver que el núcleo es trivial:
    \begin{equation*}
        \left\{
            \begin{array}{ll}
                3\alpha-\beta &= 0 \\
                -\alpha-\beta &= 0 \\
                \alpha+\beta &= 0
            \end{array}
        \right\}
        \Longrightarrow
        \left\{
            \begin{array}{ll}
                \alpha &= 0 \\
                \beta &= 0
            \end{array}
        \right\}
    \end{equation*}

    Por tanto, $\vec{f}$ es una lineal inyectiva. Además, se tiene que $f([v]) = \left[\vec{f}(v)\right]$, por lo que $f$ es una proyectividad.
\end{ejercicio}



\begin{ejercicio}
    En $\bb{P}^2$, determinar la expresión matricial en la base canónica $\cc{B}_u$ de $\bb{R}^3$ de todas las
    homografías $h:\bb{P}^2\to \bb{P}^2$ satisfaciendo:
    \begin{equation*}
        h(1:1:1) = (1:0:1), \quad h(1:1:0) = (1:0:-1), \quad h(1:0:0) = (1:1:0)
    \end{equation*}

    Supongamos que $h$ es una homografía, y sea $\wt{h}$ la aplicación lineal asociada. Tenemos que, para cierto $\lm_1, \lm_2, \lm_3 \in \bb{R}^\ast$:
    \begin{equation*}
        \wt{h}(1,1,1) = \lm_1 (1,0,1), \quad \wt{h}(1,1,0) = \lm_2 (1,0,-1), \quad \wt{h}(1,0,0) = \lm_3 (1,1,0)
    \end{equation*}

    Por tanto, dada la base $\cc{B} = \{(1,1,1), (1,1,0), (1,0,0)\}$, la matriz de $\wt{h}$ en la base $\cc{B}$ es:
    \begin{equation*}
        M\left(\wt{h}, \cc{B}, \cc{B}_u\right) = \begin{pmatrix}
            \lm_1 & \lm_2 & \lm_3 \\
            0 & 0 & \lm_3 \\
            \lm_1 & -\lm_2 & 0
        \end{pmatrix}
    \end{equation*}

    Por tanto, la matriz de $h$ en la base $\cc{B}_u$ es:
    \begin{align*}
        M\left(\wt{h}, \cc{B}_u\right) &= M\left(\wt{h}, \cc{B}, \cc{B}_u\right) \cdot M\left(Id, \cc{B}_u, \cc{B}\right) = \\
        &= \begin{pmatrix}
            \lm_1 & \lm_2 & \lm_3 \\
            0 & 0 & \lm_3 \\
            \lm_1 & -\lm_2 & 0
        \end{pmatrix} \cdot \begin{pmatrix}
            1 & 1 & 1 \\
            1 & 1 & 0 \\
            1 & 0 & 0
        \end{pmatrix}^{-1}
        =\\&= \begin{pmatrix}
            \lambda_3 & \lambda_2-\lambda_3 & \lambda_1-\lambda_2 \\
            \lambda_3 & -\lambda_3 & 0 \\
            0 & -\lambda_2 & \lambda_1+\lambda_2
            \end{pmatrix} \qquad \lambda_1, \lambda_2, \lambda_3 \in \bb{R}^\ast
    \end{align*}

    Por tanto, tenemos que las homografías $h$ son las que
    tienen por lineal asociada $\wt{h}$ con dicha matriz, para ciertos $\lm_1, \lm_2, \lm_3 \in \bb{R}^\ast$.
    Además, como es inyectiva y la dimensión del dominio y el codominio coinciden, es biyectiva, por lo que es una homografía.
\end{ejercicio}