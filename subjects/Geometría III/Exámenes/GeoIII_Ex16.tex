\documentclass[12pt]{article}

% Idioma y codificación
\usepackage[spanish, es-tabla]{babel}       %es-tabla para que se titule "Tabla"
\usepackage[utf8]{inputenc}

% Márgenes
\usepackage[a4paper,top=3cm,bottom=2.5cm,left=3cm,right=3cm]{geometry}

% Comentarios de bloque
\usepackage{verbatim}

% Paquetes de links
\usepackage[hidelinks]{hyperref}    % Permite enlaces
\usepackage{url}                    % redirecciona a la web

% Más opciones para enumeraciones
\usepackage{enumitem}

% Personalizar la portada
\usepackage{titling}

% Paquetes de tablas
\usepackage{multirow}


%------------------------------------------------------------------------

%Paquetes de figuras
\usepackage{caption}
\usepackage{subcaption} % Figuras al lado de otras
\usepackage{float}      % Poner figuras en el sitio indicado H.


% Paquetes de imágenes
\usepackage{graphicx}       % Paquete para añadir imágenes
\usepackage{transparent}    % Para manejar la opacidad de las figuras

% Paquete para usar colores
\usepackage[dvipsnames]{xcolor}
\usepackage{pagecolor}      % Para cambiar el color de la página

% Habilita tamaños de fuente mayores
\usepackage{fix-cm}

% Para los gráficos
\usepackage{tikz}

% Para poder situar los nodos en los grafos
\usetikzlibrary{positioning}


%------------------------------------------------------------------------

% Paquetes de matemáticas
\usepackage{mathtools, amsfonts, amssymb, mathrsfs}
\usepackage[makeroom]{cancel}     % Simplificar tachando
\usepackage{polynom}    % Divisiones y Ruffini
\usepackage{units} % Para poner fracciones diagonales con \nicefrac

\usepackage{pgfplots}   %Representar funciones
\pgfplotsset{compat=1.18}  % Versión 1.18

\usepackage{tikz-cd}    % Para usar diagramas de composiciones
\usetikzlibrary{calc}   % Para usar cálculo de coordenadas en tikz

%Definición de teoremas, etc.
\usepackage{amsthm}
%\swapnumbers   % Intercambia la posición del texto y de la numeración

\theoremstyle{plain}

\makeatletter
\@ifclassloaded{article}{
  \newtheorem{teo}{Teorema}[section]
}{
  \newtheorem{teo}{Teorema}[chapter]  % Se resetea en cada chapter
}
\makeatother

\newtheorem{coro}{Corolario}[teo]           % Se resetea en cada teorema
\newtheorem{prop}[teo]{Proposición}         % Usa el mismo contador que teorema
\newtheorem{lema}[teo]{Lema}                % Usa el mismo contador que teorema

\theoremstyle{remark}
\newtheorem*{observacion}{Observación}

\theoremstyle{definition}

\makeatletter
\@ifclassloaded{article}{
  \newtheorem{definicion}{Definición} [section]     % Se resetea en cada chapter
}{
  \newtheorem{definicion}{Definición} [chapter]     % Se resetea en cada chapter
}
\makeatother

\newtheorem*{notacion}{Notación}
\newtheorem*{ejemplo}{Ejemplo}
\newtheorem*{ejercicio*}{Ejercicio}             % No numerado
\newtheorem{ejercicio}{Ejercicio} [section]     % Se resetea en cada section


% Modificar el formato de la numeración del teorema "ejercicio"
\renewcommand{\theejercicio}{%
  \ifnum\value{section}=0 % Si no se ha iniciado ninguna sección
    \arabic{ejercicio}% Solo mostrar el número de ejercicio
  \else
    \thesection.\arabic{ejercicio}% Mostrar número de sección y número de ejercicio
  \fi
}


% \renewcommand\qedsymbol{$\blacksquare$}         % Cambiar símbolo QED
%------------------------------------------------------------------------

% Paquetes para encabezados
\usepackage{fancyhdr}
\pagestyle{fancy}
\fancyhf{}

\newcommand{\helv}{ % Modificación tamaño de letra
\fontfamily{}\fontsize{12}{12}\selectfont}
\setlength{\headheight}{15pt} % Amplía el tamaño del índice


%\usepackage{lastpage}   % Referenciar última pag   \pageref{LastPage}
\fancyfoot[C]{\thepage}

%------------------------------------------------------------------------

% Conseguir que no ponga "Capítulo 1". Sino solo "1."
\makeatletter
\@ifclassloaded{book}{
  \renewcommand{\chaptermark}[1]{\markboth{\thechapter.\ #1}{}} % En el encabezado
    
  \renewcommand{\@makechapterhead}[1]{%
  \vspace*{50\p@}%
  {\parindent \z@ \raggedright \normalfont
    \ifnum \c@secnumdepth >\m@ne
      \huge\bfseries \thechapter.\hspace{1em}\ignorespaces
    \fi
    \interlinepenalty\@M
    \Huge \bfseries #1\par\nobreak
    \vskip 40\p@
  }}
}
\makeatother

%------------------------------------------------------------------------
% Paquetes de cógido
\usepackage{minted}
\renewcommand\listingscaption{Código fuente}

\usepackage{fancyvrb}
% Personaliza el tamaño de los números de línea
\renewcommand{\theFancyVerbLine}{\small\arabic{FancyVerbLine}}

% Estilo para C++
\newminted{cpp}{
    frame=lines,
    framesep=2mm,
    baselinestretch=1.2,
    linenos,
    escapeinside=||
}

% para minted
\definecolor{LightGray}{rgb}{0.95,0.95,0.92}
\setminted{
    linenos=true,
    stepnumber=5,
    numberfirstline=true,
    autogobble,
    breaklines=true,
    breakautoindent=true,
    breaksymbolleft=,
    breaksymbolright=,
    breaksymbolindentleft=0pt,
    breaksymbolindentright=0pt,
    breaksymbolsepleft=0pt,
    breaksymbolsepright=0pt,
    fontsize=\footnotesize,
    bgcolor=LightGray,
    numbersep=10pt
}


\usepackage{listings} % Para incluir código desde un archivo

\renewcommand\lstlistingname{Código Fuente}
\renewcommand\lstlistlistingname{Índice de Códigos Fuente}

% Definir colores
\definecolor{vscodepurple}{rgb}{0.5,0,0.5}
\definecolor{vscodeblue}{rgb}{0,0,0.8}
\definecolor{vscodegreen}{rgb}{0,0.5,0}
\definecolor{vscodegray}{rgb}{0.5,0.5,0.5}
\definecolor{vscodebackground}{rgb}{0.97,0.97,0.97}
\definecolor{vscodelightgray}{rgb}{0.9,0.9,0.9}

% Configuración para el estilo de C similar a VSCode
\lstdefinestyle{vscode_C}{
  backgroundcolor=\color{vscodebackground},
  commentstyle=\color{vscodegreen},
  keywordstyle=\color{vscodeblue},
  numberstyle=\tiny\color{vscodegray},
  stringstyle=\color{vscodepurple},
  basicstyle=\scriptsize\ttfamily,
  breakatwhitespace=false,
  breaklines=true,
  captionpos=b,
  keepspaces=true,
  numbers=left,
  numbersep=5pt,
  showspaces=false,
  showstringspaces=false,
  showtabs=false,
  tabsize=2,
  frame=tb,
  framerule=0pt,
  aboveskip=10pt,
  belowskip=10pt,
  xleftmargin=10pt,
  xrightmargin=10pt,
  framexleftmargin=10pt,
  framexrightmargin=10pt,
  framesep=0pt,
  rulecolor=\color{vscodelightgray},
  backgroundcolor=\color{vscodebackground},
}

%------------------------------------------------------------------------

% Comandos definidos
\newcommand{\bb}[1]{\mathbb{#1}}
\newcommand{\cc}[1]{\mathcal{#1}}

% I prefer the slanted \leq
\let\oldleq\leq % save them in case they're every wanted
\let\oldgeq\geq
\renewcommand{\leq}{\leqslant}
\renewcommand{\geq}{\geqslant}

% Si y solo si
\newcommand{\sii}{\iff}

% Letras griegas
\newcommand{\eps}{\epsilon}
\newcommand{\veps}{\varepsilon}
\newcommand{\lm}{\lambda}

\newcommand{\ol}{\overline}
\newcommand{\ul}{\underline}
\newcommand{\wt}{\widetilde}
\newcommand{\wh}{\widehat}

\let\oldvec\vec
\renewcommand{\vec}{\overrightarrow}

% Derivadas parciales
\newcommand{\del}[2]{\frac{\partial #1}{\partial #2}}
\newcommand{\Del}[3]{\frac{\partial^{#1} #2}{\partial #3^{#1}}}
\newcommand{\deld}[2]{\dfrac{\partial #1}{\partial #2}}
\newcommand{\Deld}[3]{\dfrac{\partial^{#1} #2}{\partial #3^{#1}}}


\newcommand{\AstIg}{\stackrel{(\ast)}{=}}
\newcommand{\Hop}{\stackrel{L'H\hat{o}pital}{=}}

\newcommand{\red}[1]{{\color{red}#1}} % Para integrales, destacar los cambios.

% Método de integración
\newcommand{\MetInt}[2]{
    \left[\begin{array}{c}
        #1 \\ #2
    \end{array}\right]
}

% Declarar aplicaciones
% 1. Nombre aplicación
% 2. Dominio
% 3. Codominio
% 4. Variable
% 5. Imagen de la variable
\newcommand{\Func}[5]{
    \begin{equation*}
        \begin{array}{rrll}
            #1:& #2 & \longrightarrow & #3\\
               & #4 & \longmapsto & #5
        \end{array}
    \end{equation*}
}

%------------------------------------------------------------------------


\usepackage{pgfplots}
\pgfplotsset{compat=1.15}
\usepackage{mathrsfs}
\usetikzlibrary{arrows}



\begin{document}
	
	% 1. Foto de fondo
	% 2. Título
	% 3. Encabezado Izquierdo
	% 4. Color de fondo
	% 5. Coord x del titulo
	% 6. Coord y del titulo
	% 7. Fecha
	
	
	% 1. Foto de fondo
% 2. Título
% 3. Encabezado Izquierdo
% 4. Color de fondo
% 5. Coord x del titulo
% 6. Coord y del titulo
% 7. Fecha

\newcommand{\portada}[7]{

    \portadaBase{#1}{#2}{#3}{#4}{#5}{#6}{#7}
    \portadaBook{#1}{#2}{#3}{#4}{#5}{#6}{#7}
}

\newcommand{\portadaExamen}[7]{

    \portadaBase{#1}{#2}{#3}{#4}{#5}{#6}{#7}
    \portadaArticle{#1}{#2}{#3}{#4}{#5}{#6}{#7}
}




\newcommand{\portadaBase}[7]{

    % Tiene la portada principal y la licencia Creative Commons
    
    % 1. Foto de fondo
    % 2. Título
    % 3. Encabezado Izquierdo
    % 4. Color de fondo
    % 5. Coord x del titulo
    % 6. Coord y del titulo
    % 7. Fecha
    
    
    \thispagestyle{empty}               % Sin encabezado ni pie de página
    \newgeometry{margin=0cm}        % Márgenes nulos para la primera página
    
    
    % Encabezado
    \fancyhead[L]{\helv #3}
    \fancyhead[R]{\helv \nouppercase{\leftmark}}
    
    
    \pagecolor{#4}        % Color de fondo para la portada
    
    \begin{figure}[p]
        \centering
        \transparent{0.3}           % Opacidad del 30% para la imagen
        
        \includegraphics[width=\paperwidth, keepaspectratio]{assets/#1}
    
        \begin{tikzpicture}[remember picture, overlay]
            \node[anchor=north west, text=white, opacity=1, font=\fontsize{60}{90}\selectfont\bfseries\sffamily, align=left] at (#5, #6) {#2};
            
            \node[anchor=south east, text=white, opacity=1, font=\fontsize{12}{18}\selectfont\sffamily, align=right] at (9.7, 3) {\textbf{\href{https://losdeldgiim.github.io/}{Los Del DGIIM}}};
            
            \node[anchor=south east, text=white, opacity=1, font=\fontsize{12}{15}\selectfont\sffamily, align=right] at (9.7, 1.8) {Doble Grado en Ingeniería Informática y Matemáticas\\Universidad de Granada};
        \end{tikzpicture}
    \end{figure}
    
    
    \restoregeometry        % Restaurar márgenes normales para las páginas subsiguientes
    \pagecolor{white}       % Restaurar el color de página
    
    
    \newpage
    \thispagestyle{empty}               % Sin encabezado ni pie de página
    \begin{tikzpicture}[remember picture, overlay]
        \node[anchor=south west, inner sep=3cm] at (current page.south west) {
            \begin{minipage}{0.5\paperwidth}
                \href{https://creativecommons.org/licenses/by-nc-nd/4.0/}{
                    \includegraphics[height=2cm]{assets/Licencia.png}
                }\vspace{1cm}\\
                Esta obra está bajo una
                \href{https://creativecommons.org/licenses/by-nc-nd/4.0/}{
                    Licencia Creative Commons Atribución-NoComercial-SinDerivadas 4.0 Internacional (CC BY-NC-ND 4.0).
                }\\
    
                Eres libre de compartir y redistribuir el contenido de esta obra en cualquier medio o formato, siempre y cuando des el crédito adecuado a los autores originales y no persigas fines comerciales. 
            \end{minipage}
        };
    \end{tikzpicture}
    
    
    
    % 1. Foto de fondo
    % 2. Título
    % 3. Encabezado Izquierdo
    % 4. Color de fondo
    % 5. Coord x del titulo
    % 6. Coord y del titulo
    % 7. Fecha


}


\newcommand{\portadaBook}[7]{

    % 1. Foto de fondo
    % 2. Título
    % 3. Encabezado Izquierdo
    % 4. Color de fondo
    % 5. Coord x del titulo
    % 6. Coord y del titulo
    % 7. Fecha

    % Personaliza el formato del título
    \pretitle{\begin{center}\bfseries\fontsize{42}{56}\selectfont}
    \posttitle{\par\end{center}\vspace{2em}}
    
    % Personaliza el formato del autor
    \preauthor{\begin{center}\Large}
    \postauthor{\par\end{center}\vfill}
    
    % Personaliza el formato de la fecha
    \predate{\begin{center}\huge}
    \postdate{\par\end{center}\vspace{2em}}
    
    \title{#2}
    \author{\href{https://losdeldgiim.github.io/}{Los Del DGIIM}}
    \date{Granada, #7}
    \maketitle
    
    \tableofcontents
}




\newcommand{\portadaArticle}[7]{

    % 1. Foto de fondo
    % 2. Título
    % 3. Encabezado Izquierdo
    % 4. Color de fondo
    % 5. Coord x del titulo
    % 6. Coord y del titulo
    % 7. Fecha

    % Personaliza el formato del título
    \pretitle{\begin{center}\bfseries\fontsize{42}{56}\selectfont}
    \posttitle{\par\end{center}\vspace{2em}}
    
    % Personaliza el formato del autor
    \preauthor{\begin{center}\Large}
    \postauthor{\par\end{center}\vspace{3em}}
    
    % Personaliza el formato de la fecha
    \predate{\begin{center}\huge}
    \postdate{\par\end{center}\vspace{5em}}
    
    \title{#2}
    \author{\href{https://losdeldgiim.github.io/}{Los Del DGIIM}}
    \date{Granada, #7}
    \thispagestyle{empty}               % Sin encabezado ni pie de página
    \maketitle
    \vfill
}
	\portadaExamen{ffccA4.jpg}{Geometría III\\Examen XVI}{Geometría III. Examen XVI}{MidnightBlue}{-8}{28}{2025}{Roxana Acedo Parra}
	
	
	\begin{description}
		\item[Asignatura] Geometría III.
		\item[Curso Académico] 2025-26.
		\item[Grado] Doble Grado en Ingeniería Informática y Matemáticas.
		\item[Grupo] Único.
		\item[Profesor] Antonio Ros Mulero.
		\item[Descripción] Examen Parcial 1.
		\item[Fecha] 22 de Octubre del 2025.
		\item[Duración] 1 hora.
		
	\end{description}
	\newpage
	
	\begin{ejercicio}[5 puntos] En el plano afín $\cc{A}$, dado un triángulo $ABC$, consideramos la mediana $A \vee A'$ y una recta $r$ paralela a la base que corta a los lados en los puntos distintos $D$ y $E$.
		\begin{figure}[H]
			\centering
			\begin{tikzpicture}[scale=0.5]
				
				\filldraw[black] (0,0) circle (1.5pt) node[anchor=north]{$B$};
				\filldraw[black] (22,0) circle (1.5pt) node[anchor=north]{$C$};
				\filldraw[black] (3.5,8.9) circle (1.5pt) node[anchor=south]{$A$};
				\filldraw[black] (11,0) circle (1.5pt) node[anchor=north]{$A'$};
				\filldraw[black] (8.564,6.463) circle (1.5pt) node[anchor=south]{$E$};
				\filldraw[black] (2.542,6.463) circle (1.5pt);
				
				\draw(2.2,6.463) node[anchor=south]{$D$};
				\draw(13,6.463) node[anchor=south]{$r$};
				
				\draw (0,0) -- (22,0);
				\draw (0,0) -- (3.5, 8.9);
				\draw (3.5,8.9) -- (22,0);
				\draw (15,6.463) -- (1,6.463);
				
				\draw[thin, dashed] (3.5,8.9) -- (11,0);
				\draw[thin, dashed] (0,0) -- (8.564,6.463);
				\draw[thin, dashed] (22,0) -- (2.542,6.463);
			\end{tikzpicture}
		\end{figure}
		
		\begin{enumerate}[label=\alph*)]
			\item Sea $f : \cc{A} \to \cc{A}$ la afinidad definida por la imagen de los vértices $f(A) = A$, $f(B) = C$ y $f(C) = B$. Razonar que $f$ es una involución. Calcular los puntos fijos de $f$ y demostrar que
			$f(r) = r$.
			
			\item Usando $f$ y sus propiedades, concluir que (si $B \vee E$ y $C \vee D$ no son paralelas) las rectas
			$A \vee A'$, $B \vee E$ y $C \vee D$ son concurrentes.
		\end{enumerate}
	\end{ejercicio}
	
	\begin{ejercicio}[5 puntos] En el espacio afín $\cc{A}^3$ y respecto del sistema de referencia $\cc{R}$, consideramos los puntos $A = (1,0,1)$, $B = (0,1,2)$ y el plano $\Pi \equiv x + y + z = 1$.
		
		\begin{enumerate}[label=\alph*)]
			\item Calcular la intersección de la recta $A \vee B$ con el plano $\Pi$.
			
			\item Encontrar el centro y la razón de la homotecia $h : \cc{A} \to \cc{A}$ que lleva $A$ en $B$ y deja invariante el plano $\Pi$: $h(A) = B$ y $h(\Pi) = \Pi$.
		\end{enumerate}
	\end{ejercicio}
	
	
	\newpage
	\setcounter{ejercicio}{0}
	
	\begin{ejercicio}[5 puntos] En el plano afín $\cc{A}$, dado un triángulo $ABC$, consideramos la mediana $A \vee A'$ y una recta $r$ paralela a la base que corta a los lados en los puntos distintos $D$ y $E$.
		
		\begin{enumerate}[label=\alph*)]
			\item Sea $f : \cc{A} \to \cc{A}$ la afinidad definida por la imagen de los vértices $f(A) = A$, $f(B) = C$ y $f(C) = B$. Razonar que $f$ es una involución. Calcular los puntos fijos de $f$ y demostrar que
			$f(r) = r$.
			
			La afinidad $f \circ f$ fija los tres vértices del triángulo:
			$$f \circ f(A) = A, \qquad f \circ f(B) = B, \qquad f \circ f(C) = C.$$
			Por tanto, $f \circ f = \mathrm{Id}$ y $f$ es una involución.
			
			El conjunto de puntos fijos de $f$ solo puede ser el vacío, un punto, una recta o todo el plano.
			La afinidad $f$ fija el vértice $A$. Veamos que también fija el punto medio $A'$:
			$$ f(A') = f\left(\dfrac{1}{2}B + \dfrac{1}{2}C\right)
			= \dfrac{1}{2}f(B) + \dfrac{1}{2}f(C)
			= \dfrac{1}{2}C + \dfrac{1}{2}B = A'. $$
			Como $f$ no es la identidad, concluimos que el conjunto de puntos fijos es la mediana $A \vee A'$.
			
			Veamos que $f(r) = r$.\\
			
			
			\textit{Primera demostración.} 
			
			La base $B \vee C$ es una recta fija de $f$. Como la recta $r$ es paralela a $B \vee C$, se sigue que $f(r)$ es paralela a $r$. Sea $P$ el punto donde se cortan $r$ y $A \vee A'$. Entonces $f(P) = P$ (todos los puntos de la mediana son puntos fijos).  
			Por tanto, las rectas paralelas $f(r)$ y $r$ se cortan en $P$ y concluimos que $f(r) = r$.\\
			
			
			\textit{Segunda demostración. }
			
			Por el teorema de Tales, existe un escalar $a \neq 0$ tal que $\overrightarrow{AE} = a\,\overrightarrow{AC} \quad \text{y} \quad
			\overrightarrow{AD} = a\,\overrightarrow{AB}.$ Ahora calculemos la imagen de $D$.
			
			Como $f$ intercambia $B$ y $C$, se tiene $f(\overrightarrow{AB}) = \overrightarrow{AC}$ y $f(\overrightarrow{AC}) = \overrightarrow{AB}$. Entonces:
			
			$$ f(D) = f(A + \overrightarrow{AD})
			= f(A) + a\,f(\overrightarrow{AB})
			= A + a\,\overrightarrow{AC} = A + \overrightarrow{AE}=E. $$
			
			Entonces $f(r) \cap r \neq \emptyset$ y se sigue que $f(r) = r$.
			
			\item Usando $f$ y sus propiedades, concluir que (si $B \vee E$ y $C \vee D$ no son paralelas) las rectas
			$A \vee A'$, $B \vee E$ y $C \vee D$ son concurrentes.
			
			Como $f$ fija la recta $r$ e intercambia los lados $A \vee B$ y $A \vee C$, tenemos que $f(D) = E$ y $f(E) = D$.  
			Por tanto, $f$ intercambia las rectas $B \vee E$ y $C \vee D$.
			
			Sea $Q = (B \vee E) \cap (C \vee D)$ el punto donde se cortan dos de las tres rectas. Tenemos que:
			$$
			f(Q) = f(B \vee E) \cap f(C \vee D)
			= (C \vee D) \cap (B \vee E)
			= Q.
			$$
			Entonces $Q$ es un punto fijo y, por tanto, pertenece a la tercera recta $A \vee A'$.
		\end{enumerate}
	\end{ejercicio}
	
	
	\begin{ejercicio}[5 puntos] En el espacio afín $\cc{A}^3$ y respecto del sistema de referencia $\cc{R}$, consideramos los puntos $A = (1,0,1)$, $B = (0,1,2)$ y el plano $\Pi \equiv x + y + z = 1$.
		
		\begin{enumerate}[label=\alph*)]
			\item Calcular la intersección de la recta $A \vee B$ con el plano $\Pi$. 
			
			Obtengamos primero las ecuaciones de la recta $A \vee B$
			$$A \vee B = A + \cc{L}\left\{\overrightarrow{AB}\right\} = (1,0,1) + \cc{L}\left\{(-1,1,1)\right\} =(1,0,1) + \lambda(-1,1,1)$$
			De ello concluimos que
			$$ x=1-\lambda \quad y=\lambda \quad z=1+y \, \Rightarrow \, \lambda = y = z-1 = 1-x$$
			Consecuentemente
			$$A \vee B\equiv\begin{cases}
				x+y=1 \\
				y-z=-1
			\end{cases}$$
			
			Si llamamos $P$ al punto de intersección entre el plano $\Pi$ y $A \vee B$, sabemos que $P$ es el único punto que cumple las ecuaciones de ambos, es decir, la solución del sistema
			
			$$P=\begin{cases}
				x+y=1 \\
				y-z=-1 \\
				x + y + z = 1
			\end{cases} \Rightarrow P = (2,-1,0)$$
			
			Entonces, $\Pi\cap A \vee B = P = (2,-1,0)$.
			
			\item Encontrar el centro y la razón de la homotecia $h : \cc{A} \to \cc{A}$ que lleva $A$ en $B$ y deja invariante el plano $\Pi$: $h(A) = B$ y $h(\Pi) = \Pi$.
			
			Recordando la expresión de una homotecia de centro $C$ y razón $\lambda$
			
			$$h(P)_{C,\, \lambda}=\lambda\overrightarrow{CP} + C = \lambda P + (1-\lambda)C$$
			
			Expresándolo en el sistema de referencia dado y denotando $C=(c_1,c_2,c_3)$
			
			$$h\begin{pmatrix}
				x \\
				y \\
				z
			\end{pmatrix} = \begin{pmatrix}
			\lambda & 0 & 0 \\
			0 & \lambda & 0 \\
			0 & 0 & \lambda
			\end{pmatrix}\begin{pmatrix}
			x \\
			y \\
			z
			\end{pmatrix} + \begin{pmatrix}
			1-\lambda & 0 & 0 \\
			0 & 1-\lambda & 0 \\
			0 & 0 & 1-\lambda
			\end{pmatrix}\begin{pmatrix}
			c_1 \\
			c_2 \\
			c_3
			\end{pmatrix}$$
			
			Ahora podemos usar que $h(A) = B$ y $h(\Pi) = \Pi$ para encontrar ecuaciones y resolver para $C$ y $\lambda$
			
			$$h\begin{pmatrix}
				1 \\
				0 \\
				1
			\end{pmatrix} = \begin{pmatrix}
				\lambda & 0 & 0 \\
				0 & \lambda & 0 \\
				0 & 0 & \lambda
			\end{pmatrix}\begin{pmatrix}
				1 \\
				0 \\
				1
			\end{pmatrix} + \begin{pmatrix}
				1-\lambda & 0 & 0 \\
				0 & 1-\lambda & 0 \\
				0 & 0 & 1-\lambda
			\end{pmatrix}\begin{pmatrix}
				c_1 \\
				c_2 \\
				c_3
			\end{pmatrix} = $$
			$$\begin{pmatrix}
			\lambda +(1-\lambda)c_1 \\
			(1-\lambda)c_2 \\
			\lambda +(1-\lambda)c_3
			\end{pmatrix}=\begin{pmatrix}
			0 \\
			1 \\
			2
			\end{pmatrix}$$
			
			También sabemos que si $P \in \Pi \Rightarrow h(P) \in \Pi$, sea $P=(x,y,z) \in \Pi$
			
			$$h(P)=\begin{pmatrix}
				\lambda x +(1-\lambda)c_1 \\
				\lambda y + (1-\lambda)c_2 \\
				\lambda z +(1-\lambda)c_3
			\end{pmatrix} \in \Pi \Rightarrow$$
			$$ \lambda x +(1-\lambda)c_1 + \lambda y + (1-\lambda)c_2 + \lambda z +(1-\lambda)c_3 = 1 \iff$$
			$$ \lambda(\underbrace{x+y+z}_{\text{$P \in \Pi$}}) + (1-\lambda)c_1 +(1-\lambda)c_2 +(1-\lambda)c_3 = 1 \iff $$
			$$ \lambda + (1-\lambda)c_1 +(1-\lambda)c_2 +(1-\lambda)c_3 = 1$$
			
			Y nos quedan las ecuaciones
			$$\begin{cases}
				\lambda +(1-\lambda)c_1 = 0\\
				(1-\lambda)c_2 = 1\\
				\lambda +(1-\lambda)c_3 = 2 \\
				\lambda + (1-\lambda)c_1 +(1-\lambda)c_2 +(1-\lambda)c_3 = 1
				\end{cases}$$
			Para resolverlo podemos, por ejemplo, sustituir en la última las dos primeras
			$$ 0 + 1 + (1-\lambda)c_3 = 1 \Rightarrow (1-\lambda)c_3 = 0$$
			Sustituimos este resultado en la tercera, obteniendo $\boxed{\lambda = 2}$. Con este dato obtenemos que $\boxed{C=(2,-1,0)}$. Es decir, la homotecia que cumple las condiciones pedidas tiene como centro el punto de intersección entre la recta $A \vee B$ y el plano $\Pi$ y razón 2.
		\end{enumerate}
	\end{ejercicio}
\end{document}