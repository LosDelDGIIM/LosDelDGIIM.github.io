\documentclass[12pt]{article}

% Idioma y codificación
\usepackage[spanish, es-tabla]{babel}       %es-tabla para que se titule "Tabla"
\usepackage[utf8]{inputenc}

% Márgenes
\usepackage[a4paper,top=3cm,bottom=2.5cm,left=3cm,right=3cm]{geometry}

% Comentarios de bloque
\usepackage{verbatim}

% Paquetes de links
\usepackage[hidelinks]{hyperref}    % Permite enlaces
\usepackage{url}                    % redirecciona a la web

% Más opciones para enumeraciones
\usepackage{enumitem}

% Personalizar la portada
\usepackage{titling}

% Paquetes de tablas
\usepackage{multirow}


%------------------------------------------------------------------------

%Paquetes de figuras
\usepackage{caption}
\usepackage{subcaption} % Figuras al lado de otras
\usepackage{float}      % Poner figuras en el sitio indicado H.


% Paquetes de imágenes
\usepackage{graphicx}       % Paquete para añadir imágenes
\usepackage{transparent}    % Para manejar la opacidad de las figuras

% Paquete para usar colores
\usepackage[dvipsnames]{xcolor}
\usepackage{pagecolor}      % Para cambiar el color de la página

% Habilita tamaños de fuente mayores
\usepackage{fix-cm}

% Para los gráficos
\usepackage{tikz}

% Para poder situar los nodos en los grafos
\usetikzlibrary{positioning}


%------------------------------------------------------------------------

% Paquetes de matemáticas
\usepackage{mathtools, amsfonts, amssymb, mathrsfs}
\usepackage[makeroom]{cancel}     % Simplificar tachando
\usepackage{polynom}    % Divisiones y Ruffini
\usepackage{units} % Para poner fracciones diagonales con \nicefrac

\usepackage{pgfplots}   %Representar funciones
\pgfplotsset{compat=1.18}  % Versión 1.18

\usepackage{tikz-cd}    % Para usar diagramas de composiciones
\usetikzlibrary{calc}   % Para usar cálculo de coordenadas en tikz

%Definición de teoremas, etc.
\usepackage{amsthm}
%\swapnumbers   % Intercambia la posición del texto y de la numeración

\theoremstyle{plain}

\makeatletter
\@ifclassloaded{article}{
  \newtheorem{teo}{Teorema}[section]
}{
  \newtheorem{teo}{Teorema}[chapter]  % Se resetea en cada chapter
}
\makeatother

\newtheorem{coro}{Corolario}[teo]           % Se resetea en cada teorema
\newtheorem{prop}[teo]{Proposición}         % Usa el mismo contador que teorema
\newtheorem{lema}[teo]{Lema}                % Usa el mismo contador que teorema

\theoremstyle{remark}
\newtheorem*{observacion}{Observación}

\theoremstyle{definition}

\makeatletter
\@ifclassloaded{article}{
  \newtheorem{definicion}{Definición} [section]     % Se resetea en cada chapter
}{
  \newtheorem{definicion}{Definición} [chapter]     % Se resetea en cada chapter
}
\makeatother

\newtheorem*{notacion}{Notación}
\newtheorem*{ejemplo}{Ejemplo}
\newtheorem*{ejercicio*}{Ejercicio}             % No numerado
\newtheorem{ejercicio}{Ejercicio} [section]     % Se resetea en cada section


% Modificar el formato de la numeración del teorema "ejercicio"
\renewcommand{\theejercicio}{%
  \ifnum\value{section}=0 % Si no se ha iniciado ninguna sección
    \arabic{ejercicio}% Solo mostrar el número de ejercicio
  \else
    \thesection.\arabic{ejercicio}% Mostrar número de sección y número de ejercicio
  \fi
}


% \renewcommand\qedsymbol{$\blacksquare$}         % Cambiar símbolo QED
%------------------------------------------------------------------------

% Paquetes para encabezados
\usepackage{fancyhdr}
\pagestyle{fancy}
\fancyhf{}

\newcommand{\helv}{ % Modificación tamaño de letra
\fontfamily{}\fontsize{12}{12}\selectfont}
\setlength{\headheight}{15pt} % Amplía el tamaño del índice


%\usepackage{lastpage}   % Referenciar última pag   \pageref{LastPage}
\fancyfoot[C]{\thepage}

%------------------------------------------------------------------------

% Conseguir que no ponga "Capítulo 1". Sino solo "1."
\makeatletter
\@ifclassloaded{book}{
  \renewcommand{\chaptermark}[1]{\markboth{\thechapter.\ #1}{}} % En el encabezado
    
  \renewcommand{\@makechapterhead}[1]{%
  \vspace*{50\p@}%
  {\parindent \z@ \raggedright \normalfont
    \ifnum \c@secnumdepth >\m@ne
      \huge\bfseries \thechapter.\hspace{1em}\ignorespaces
    \fi
    \interlinepenalty\@M
    \Huge \bfseries #1\par\nobreak
    \vskip 40\p@
  }}
}
\makeatother

%------------------------------------------------------------------------
% Paquetes de cógido
\usepackage{minted}
\renewcommand\listingscaption{Código fuente}

\usepackage{fancyvrb}
% Personaliza el tamaño de los números de línea
\renewcommand{\theFancyVerbLine}{\small\arabic{FancyVerbLine}}

% Estilo para C++
\newminted{cpp}{
    frame=lines,
    framesep=2mm,
    baselinestretch=1.2,
    linenos,
    escapeinside=||
}

% para minted
\definecolor{LightGray}{rgb}{0.95,0.95,0.92}
\setminted{
    linenos=true,
    stepnumber=5,
    numberfirstline=true,
    autogobble,
    breaklines=true,
    breakautoindent=true,
    breaksymbolleft=,
    breaksymbolright=,
    breaksymbolindentleft=0pt,
    breaksymbolindentright=0pt,
    breaksymbolsepleft=0pt,
    breaksymbolsepright=0pt,
    fontsize=\footnotesize,
    bgcolor=LightGray,
    numbersep=10pt
}


\usepackage{listings} % Para incluir código desde un archivo

\renewcommand\lstlistingname{Código Fuente}
\renewcommand\lstlistlistingname{Índice de Códigos Fuente}

% Definir colores
\definecolor{vscodepurple}{rgb}{0.5,0,0.5}
\definecolor{vscodeblue}{rgb}{0,0,0.8}
\definecolor{vscodegreen}{rgb}{0,0.5,0}
\definecolor{vscodegray}{rgb}{0.5,0.5,0.5}
\definecolor{vscodebackground}{rgb}{0.97,0.97,0.97}
\definecolor{vscodelightgray}{rgb}{0.9,0.9,0.9}

% Configuración para el estilo de C similar a VSCode
\lstdefinestyle{vscode_C}{
  backgroundcolor=\color{vscodebackground},
  commentstyle=\color{vscodegreen},
  keywordstyle=\color{vscodeblue},
  numberstyle=\tiny\color{vscodegray},
  stringstyle=\color{vscodepurple},
  basicstyle=\scriptsize\ttfamily,
  breakatwhitespace=false,
  breaklines=true,
  captionpos=b,
  keepspaces=true,
  numbers=left,
  numbersep=5pt,
  showspaces=false,
  showstringspaces=false,
  showtabs=false,
  tabsize=2,
  frame=tb,
  framerule=0pt,
  aboveskip=10pt,
  belowskip=10pt,
  xleftmargin=10pt,
  xrightmargin=10pt,
  framexleftmargin=10pt,
  framexrightmargin=10pt,
  framesep=0pt,
  rulecolor=\color{vscodelightgray},
  backgroundcolor=\color{vscodebackground},
}

%------------------------------------------------------------------------

% Comandos definidos
\newcommand{\bb}[1]{\mathbb{#1}}
\newcommand{\cc}[1]{\mathcal{#1}}

% I prefer the slanted \leq
\let\oldleq\leq % save them in case they're every wanted
\let\oldgeq\geq
\renewcommand{\leq}{\leqslant}
\renewcommand{\geq}{\geqslant}

% Si y solo si
\newcommand{\sii}{\iff}

% Letras griegas
\newcommand{\eps}{\epsilon}
\newcommand{\veps}{\varepsilon}
\newcommand{\lm}{\lambda}

\newcommand{\ol}{\overline}
\newcommand{\ul}{\underline}
\newcommand{\wt}{\widetilde}
\newcommand{\wh}{\widehat}

\let\oldvec\vec
\renewcommand{\vec}{\overrightarrow}

% Derivadas parciales
\newcommand{\del}[2]{\frac{\partial #1}{\partial #2}}
\newcommand{\Del}[3]{\frac{\partial^{#1} #2}{\partial #3^{#1}}}
\newcommand{\deld}[2]{\dfrac{\partial #1}{\partial #2}}
\newcommand{\Deld}[3]{\dfrac{\partial^{#1} #2}{\partial #3^{#1}}}


\newcommand{\AstIg}{\stackrel{(\ast)}{=}}
\newcommand{\Hop}{\stackrel{L'H\hat{o}pital}{=}}

\newcommand{\red}[1]{{\color{red}#1}} % Para integrales, destacar los cambios.

% Método de integración
\newcommand{\MetInt}[2]{
    \left[\begin{array}{c}
        #1 \\ #2
    \end{array}\right]
}

% Declarar aplicaciones
% 1. Nombre aplicación
% 2. Dominio
% 3. Codominio
% 4. Variable
% 5. Imagen de la variable
\newcommand{\Func}[5]{
    \begin{equation*}
        \begin{array}{rrll}
            #1:& #2 & \longrightarrow & #3\\
               & #4 & \longmapsto & #5
        \end{array}
    \end{equation*}
}

%------------------------------------------------------------------------



\begin{document}

    % 1. Foto de fondo
    % 2. Título
    % 3. Encabezado Izquierdo
    % 4. Color de fondo
    % 5. Coord x del titulo
    % 6. Coord y del titulo
    % 7. Fecha

    
    % 1. Foto de fondo
% 2. Título
% 3. Encabezado Izquierdo
% 4. Color de fondo
% 5. Coord x del titulo
% 6. Coord y del titulo
% 7. Fecha

\newcommand{\portada}[7]{

    \portadaBase{#1}{#2}{#3}{#4}{#5}{#6}{#7}
    \portadaBook{#1}{#2}{#3}{#4}{#5}{#6}{#7}
}

\newcommand{\portadaExamen}[7]{

    \portadaBase{#1}{#2}{#3}{#4}{#5}{#6}{#7}
    \portadaArticle{#1}{#2}{#3}{#4}{#5}{#6}{#7}
}




\newcommand{\portadaBase}[7]{

    % Tiene la portada principal y la licencia Creative Commons
    
    % 1. Foto de fondo
    % 2. Título
    % 3. Encabezado Izquierdo
    % 4. Color de fondo
    % 5. Coord x del titulo
    % 6. Coord y del titulo
    % 7. Fecha
    
    
    \thispagestyle{empty}               % Sin encabezado ni pie de página
    \newgeometry{margin=0cm}        % Márgenes nulos para la primera página
    
    
    % Encabezado
    \fancyhead[L]{\helv #3}
    \fancyhead[R]{\helv \nouppercase{\leftmark}}
    
    
    \pagecolor{#4}        % Color de fondo para la portada
    
    \begin{figure}[p]
        \centering
        \transparent{0.3}           % Opacidad del 30% para la imagen
        
        \includegraphics[width=\paperwidth, keepaspectratio]{assets/#1}
    
        \begin{tikzpicture}[remember picture, overlay]
            \node[anchor=north west, text=white, opacity=1, font=\fontsize{60}{90}\selectfont\bfseries\sffamily, align=left] at (#5, #6) {#2};
            
            \node[anchor=south east, text=white, opacity=1, font=\fontsize{12}{18}\selectfont\sffamily, align=right] at (9.7, 3) {\textbf{\href{https://losdeldgiim.github.io/}{Los Del DGIIM}}};
            
            \node[anchor=south east, text=white, opacity=1, font=\fontsize{12}{15}\selectfont\sffamily, align=right] at (9.7, 1.8) {Doble Grado en Ingeniería Informática y Matemáticas\\Universidad de Granada};
        \end{tikzpicture}
    \end{figure}
    
    
    \restoregeometry        % Restaurar márgenes normales para las páginas subsiguientes
    \pagecolor{white}       % Restaurar el color de página
    
    
    \newpage
    \thispagestyle{empty}               % Sin encabezado ni pie de página
    \begin{tikzpicture}[remember picture, overlay]
        \node[anchor=south west, inner sep=3cm] at (current page.south west) {
            \begin{minipage}{0.5\paperwidth}
                \href{https://creativecommons.org/licenses/by-nc-nd/4.0/}{
                    \includegraphics[height=2cm]{assets/Licencia.png}
                }\vspace{1cm}\\
                Esta obra está bajo una
                \href{https://creativecommons.org/licenses/by-nc-nd/4.0/}{
                    Licencia Creative Commons Atribución-NoComercial-SinDerivadas 4.0 Internacional (CC BY-NC-ND 4.0).
                }\\
    
                Eres libre de compartir y redistribuir el contenido de esta obra en cualquier medio o formato, siempre y cuando des el crédito adecuado a los autores originales y no persigas fines comerciales. 
            \end{minipage}
        };
    \end{tikzpicture}
    
    
    
    % 1. Foto de fondo
    % 2. Título
    % 3. Encabezado Izquierdo
    % 4. Color de fondo
    % 5. Coord x del titulo
    % 6. Coord y del titulo
    % 7. Fecha


}


\newcommand{\portadaBook}[7]{

    % 1. Foto de fondo
    % 2. Título
    % 3. Encabezado Izquierdo
    % 4. Color de fondo
    % 5. Coord x del titulo
    % 6. Coord y del titulo
    % 7. Fecha

    % Personaliza el formato del título
    \pretitle{\begin{center}\bfseries\fontsize{42}{56}\selectfont}
    \posttitle{\par\end{center}\vspace{2em}}
    
    % Personaliza el formato del autor
    \preauthor{\begin{center}\Large}
    \postauthor{\par\end{center}\vfill}
    
    % Personaliza el formato de la fecha
    \predate{\begin{center}\huge}
    \postdate{\par\end{center}\vspace{2em}}
    
    \title{#2}
    \author{\href{https://losdeldgiim.github.io/}{Los Del DGIIM}}
    \date{Granada, #7}
    \maketitle
    
    \tableofcontents
}




\newcommand{\portadaArticle}[7]{

    % 1. Foto de fondo
    % 2. Título
    % 3. Encabezado Izquierdo
    % 4. Color de fondo
    % 5. Coord x del titulo
    % 6. Coord y del titulo
    % 7. Fecha

    % Personaliza el formato del título
    \pretitle{\begin{center}\bfseries\fontsize{42}{56}\selectfont}
    \posttitle{\par\end{center}\vspace{2em}}
    
    % Personaliza el formato del autor
    \preauthor{\begin{center}\Large}
    \postauthor{\par\end{center}\vspace{3em}}
    
    % Personaliza el formato de la fecha
    \predate{\begin{center}\huge}
    \postdate{\par\end{center}\vspace{5em}}
    
    \title{#2}
    \author{\href{https://losdeldgiim.github.io/}{Los Del DGIIM}}
    \date{Granada, #7}
    \thispagestyle{empty}               % Sin encabezado ni pie de página
    \maketitle
    \vfill
}
    \portadaExamen{ffccA4.jpg}{Geometría III\\Examen VI}{Geometría III. Examen VI}{MidnightBlue}{-8}{28}{2023-2024}{Arturo Olivares Martos}

    \begin{description}
        \item[Asignatura] Geometría III.
        \item[Curso Académico] 2022-23.
        \item[Grado] Doble Grado en Ingeniería Informática y Matemáticas.
        \item[Grupo] Único.
        \item[Profesor] Antonio Martínez López.
        \item[Descripción] Convocatoria Ordinaria\footnote{El examen lo pone el departamento.}.
        \item[Fecha] 20 de enero de 2023.
        \item[Duración] 3 horas.
    
    \end{description}
    \newpage

    \begin{ejercicio}[2.5 puntos]
        Contesta a dos de los tres siguientes apartados:
        \begin{enumerate}
            \item Razona si la siguiente afirmación es verdadera o falsa:
            ``En un espacio afín $3-$dimensional
            la intersección de tres planos distintos o es vacía, o un punto o una recta afín''.

            Sabemos que la intersección de subespacios afines es, o vacía, o un subespacio afín de dimensión menor o igual a la dimensión de los subespacios afines que se intersecan.
            Por tanto, sabemos que la intersección puede ser vacía, un punto, una recta o un plano afín.

            No obstante, veamos que no puede ser un plano afín. Si lo fuese, para cada uno de los planos que intersecan tendríamos que
            tienen la misma dimensión que la intersección y la contienen, por tanto, serían la intersección. No obstante, esto es
            una contradicción, ya que son planos distintos.

            \item Razona qué movimiento rígido resulta al componer un giro con una simetría axial en
            un plano euclídeo.

            En este caso, tenemos que $f = G_{\theta, p} \circ \sigma_{R}$, con $p\in \bb{R}^2$, $\theta \in ]0, \pi]$ y $R\subset \bb{R}^2$.
            Tenemos que $|f| = -1$, por lo que se trata de una isometría inversa.
            Por tanto, se trata de una simetría axial o una simetría axial con deslizamiento. Supongamos que
            es una simetría axial con deslizamiento, $t_v\circ \sigma_{S}$, con $v\in \vec{S}\setminus \{0\}$. Tenemos que:
            \begin{equation*}
                G_{\theta, p} \circ \sigma_{R} = t_v\circ \sigma_{S} \Longrightarrow
                G_{\theta, p} = t_v\circ \sigma_{S} \circ \sigma_{R}
            \end{equation*}

            Distinguimos en funión de la posición relativa de $S$ y $R$:
            \begin{itemize}
                \item Si $S=R$ (son coincidentes), entonces sabemos que $\sigma_{S} \circ \sigma_{R}= Id_{\bb{R}^2}$,
                por lo que $G_{\theta, p}=t_v$, llegando a una contradicción, ya que el giro tiene puntos fijos pero la traslación no.

                \item Si $S \| R$, $S \neq R$ (son paralelas), entonces sabemos que $\sigma_{S} \circ \sigma_{R}=t_w$, con $w\in \vec{S}^\perp \setminus \{0\}$.
                Como $w\in \vec{S}^\perp$ y $v\in \vec{S}$, tenemos que $w\perp v$, por lo que $v+w \neq 0$.
                por lo que $G_{\theta, p}=t_{v+w}$. Esto es una contradicción, ya que el giro tiene puntos fijos pero la traslación no.

                \item Si $S\cap R \neq \emptyset$, $S \cancel{\|} R$ (son secantes),
                sabemos que $\sigma_{S} \circ \sigma_{R}$ es un giro de centro $q$ de ángulo no orientado $\theta' \in ]0, \pi]$.
                Por tanto, tenemos que $G_{\theta, p} = t_v\circ G_{\theta', q}$,
                que también es una contradicción, ya que $p$ no se mantiene fijo en el caso de la derecha.
            \end{itemize}
            
            Por tanto, se trata de una simetría axial.

            \item Enuncia y demuestra el Teorema de Thales.
        \end{enumerate}
    \end{ejercicio}





    \begin{ejercicio}[2.5 puntos]
        En $\bb{R}^3$, se consideran los planos $\Pi\equiv -x-y+z=1$ y $\Pi'\equiv x+y+z=-1$
        y las rectas $r=(-1,0,0) + \cc{L}\{(1,1,0)\}$ y $r'=(0,0,-1) + \cc{L}\{(1,0,1)\}$.
        \begin{enumerate}
            \item Prueba la existencia de una afinidad $f:\bb{R}^3 \to \bb{R}^3$ que verifique $f(\Pi)=\Pi'$ y $f(r)=r'$.
            Determina sus ecuaciones en el sistema de referencia canónico.

            Sean los sistemas de referencia dados por $\cc{R}=\{(-1,0,0), \{(1,1,0), (1,0,1), (0,1,1)\}\}$
            y $\cc{R}'=\{(0,0,-1), \{(1,0,1), (1,0,-1), (0,1,-1)\}\}$.

            Consideramos ahora entonces $f$ cuya matriz asociada es la siguiente:
            \begin{equation*}
                M\left(f, \cc{R}, \cc{R}'\right) = \left(
                    \begin{array}{c|ccc}
                        1 & 0 & 0 & 0 \\
                        \hline
                        0 & 1 & 0 & 0 \\
                        0 & 0 & 1 & 0 \\
                        0 & 0 & 0 & 1
                    \end{array}
                \right)
            \end{equation*}

            Tenemos que:
            \begin{align*}
                f(r)&
                = f\left((-1,0,0) + \cc{L}\{(1,1,0)\}\right)
                = f\left((-1,0,0)\right) + \vec{f}\left(\cc{L}\{(1,1,0)\}\right)
                =\\&= (0,0,-1) + \cc{L}\{(1,0,1)\} = r'
            \end{align*}
            \begin{align*}
                f(\Pi)&
                = f\left((-1,0,0) + \cc{L}\{(1,0,1), (0,1,1)\}\right)
                = f\left((-1,0,0)\right) + \vec{f}\left(\cc{L}\{(1,0,1), (0,1,1)\}\right)
                =\\&= (0,0,-1) + \cc{L}\{(1,0,-1), (0,1,-1)\} = \Pi'
            \end{align*}

            Calculamos su expresión en el sistema de referencia canónico:
            \begin{align*}
                M(f, \cc{R}_0) &= M\left(Id_{\bb{R}^3}, \cc{R}', \cc{R}_0\right) \cdot M\left(f, \cc{R}, \cc{R}'\right) \cdot M\left(Id_{\bb{R}^3}, \cc{R}_0, \cc{R}\right) = \\
                &= M\left(Id_{\bb{R}^3}, \cc{R}', \cc{R}_0\right) \cdot M\left(f, \cc{R}, \cc{R}'\right) \cdot M\left(Id_{\bb{R}^3}, \cc{R}, \cc{R}_0\right)^{-1} = \\
                &=\left(
                    \begin{array}{c|ccc}
                        1 & 0 & 0 & 0 \\
                        \hline
                        -1 & 1 & 1 & 0 \\
                        0 & 1 & 0 & 1 \\
                        0 & 0 & 1 & 1
                    \end{array}
                \right)
                Id_{4}
                \left(
                    \begin{array}{c|ccc}
                        1 & 0 & 0 & 0 \\
                        \hline
                        0 & 1 & 1 & 0 \\
                        0 & 1 & 0 & 1 \\
                        -1 & 0 & -1 & -1
                    \end{array}
                \right)^{-1} = \\
                &=\left(
                    \begin{array}{c|ccc}
                        1 & 0 & 0 & 0 \\
                        \hline
                        -1 & 1 & 0 & 0 \\
                        0 & 0 & 1 & 0 \\
                        -1 & 0 & 0 & -1
                    \end{array}
                \right)
            \end{align*}

            \item Demuestra que $r$ y $r'$ se cruzan y calcula la distancia entre ellas.
            
            Calculamos la intersección entre $r$ y $r'$. Para ello, tenemos que:
            \begin{equation*}
                r = \left\{(-1+\lm, \lm, 0) \mid \lm \in \bb{R}\right\} \qquad
                r' = \left\{(\mu, 0, -1+\mu) \mid \mu \in \bb{R}\right\}
            \end{equation*}

            Para que se intersequen, han de existir $\lm, \mu \in \bb{R}$ tales que:
            \begin{equation*}
                (-1+\lm, \lm, 0) = (\mu, 0, -1+\mu) \Longrightarrow
                \left\{
                    \begin{array}{l}
                        -1+\lm = \mu \\
                        \lm = 0 \\
                        0 = -1+\mu
                    \end{array}
                \right.
            \end{equation*}
            Por tanto, de la tercera ecuación deducimos que $\mu = 1$, y sustituyendo en la primera, tenemos que $\lm = 2$.
            Por tanto, vemos que no se intersecan. Calculamos ahora la distancia entre ambas rectas.
            Necesitamos buscar $p\in r$ y $p'\in r'$ tales que $\vec{pp'}\in \vec{r}^\perp \cap \vec{r'}^\perp$.
            Tenemos que:
            \begin{equation*}
                \vec{pp'} = (\mu+1-\lm, -\lm, -1+\mu),\qquad \mu, \lm \in \bb{R}
            \end{equation*}

            Por tanto, imponiendo que $\vec{pp'}\in \vec{r}^\perp \cap \vec{r'}^\perp$, tenemos que:
            \begin{align*}
                \left\langle (\mu+1-\lm, -\lm, -1+\mu), (1,1,0) \right\rangle &= 0
                = \mu+1-\lm -\lm =\\&\hspace{2cm}= \mu -2\lm =-1 \\
                \left\langle (\mu+1-\lm, -\lm, -1+\mu), (1,0,1) \right\rangle &= 0
                = \mu+1-\lm -1+\mu =\\&\hspace{2cm}= 2\mu -\lm =0
            \end{align*}

            Por tanto, deducimos que $\mu=\frac{2}{3}$, $\lm=\frac{1}{3}$, por lo que $p=\left(-\frac{2}{3}, \frac{1}{3}, 0\right)$ y además $p'=\left(\frac{2}{3}, 0, -\frac{1}{3}\right)$.
            Tenemos que:
            \begin{equation*}
                d(r,r') = d(p, p') = \sqrt{\frac{4^2}{3^2} +2\cdot \frac{1}{3^2}} = \sqrt{2}
            \end{equation*}
        \end{enumerate}
    \end{ejercicio}



    \begin{ejercicio}[2.5 puntos] En $\bb{R}^3$ y respecto del sistema de referencia euclídeo usual, calcula las ecuaciones
        de la simetría axial con deslizamiento, $f:\bb{R}^3 \to \bb{R}^3$, que verifica $f(0,0,0)=(1,1,2)$ y $\vec{f}(1,1,0)=(1,1,0)$.

        En este caso, consideremos el plano de $\bb{R}^3$ dado por $\Pi\equiv z=1$, y el vector dado por $v=(1,1,0)\in \vec{\Pi}$.
        Sea $f$ la simetría axial respecto de la recta $R=(0,0,1) + \cc{L}\{v\}$ compuesto con la traslación de vector $v$.
        En la base usual, tenemos que:
        \begin{gather*}
            f(0,0,0)=(0,0,2) + (1,1,0) = (1,1,2) \qquad
            \vec{f}(1,0,0) = (0,1,0) \\
            \vec{f}(0,1,0) = (1,0,0) \qquad
            \vec{f}(0,0,1) = (0,0,-1)
        \end{gather*}

        Por tanto, tenemos que:
        \begin{equation*}
            M(f, \cc{R}_0) = \left(
                \begin{array}{c|ccc}
                    1 & 0 & 0 & 0 \\
                    \hline
                    1 & 0 & 1 & 0 \\
                    1 & 1 & 0 & 0 \\
                    2 & 0 & 0 & -1
                \end{array}
            \right)
        \end{equation*}

        Comprobamos las condiciones del enunciado. Tenemos que $f(0,0,0)=(1,1,2)$, y además:
        \begin{equation*}
            \vec{f}(e_1+e_2) = \vec{f}(e_1) + \vec{f}(e_2) = (0,1,0) + (1,0,0) = (1,1,0)
        \end{equation*}
        
    \end{ejercicio}


   \begin{ejercicio}[2.5 puntos]
    En $\bb{R}^3$ consideramos el punto $F = (0, 0, 1)$ y el plano afín $S$ de ecuación $x-z = 0$. Definimos el conjunto:
    \begin{equation*}
        C = \{p \in \bb{R}^3 \mid d(p, F) = d(p, S)\}.
    \end{equation*}
    Demostrar que $C$ es una cuádrica y clasificarla.\\


    Sea el vector unitario $v_3=\dfrac{\vec{\pi_S(F)F}}{\left\|\vec{\pi_S(F)F}\right\|}\in \vec{S}^\perp \subset \bb{R}^3$
    y sea $v_1,v_2\in \bb{R}^3$ tal que $\cc{B}=\{v_1,v_2,v_3\}$ es una base ortonormal orientada positivamente.
    Como $v_3\in \vec{S}^\perp$, tenemos que $v_1,v_2\in \vec{S}$.

    Consideramos el sistema de referencia ortonormal $\cc{R}=\{m_{F\pi_S(F)},\cc{B}\}$. Las coordenadas de $F$ en $\cc{R}$ son:
    \begin{align*}
        F &= F + \frac{1}{2}\vec{F\pi_S(F)} + \frac{1}{2}\vec{\pi_S(F)F} 
        = m_{F \pi_S(F)} + \frac{1}{2}\vec{\pi_S(F)F} \cdot \dfrac{\left\|\vec{\pi_S(F)F}\right\|}{\left\|\vec{\pi_S(F)F}\right\|}
        =\\&= \left(0,0, \dfrac{\left\|\vec{\pi_S(F)F}\right\|}{2}\right)_{\cc{R}}
    \end{align*}

    Además, dado $ p\in \bb{R}^3$, $p=(x,y,z)_{\cc{R}}$, calculamos las coordenadas en $\cc{R}$ de $\pi_S(p)$:
    \begin{align*}
        \pi_S(p) &= \pi_S(m_{F\pi_S(F)} + xv_1 + yv_2+zv_3)
        =\\&= \pi_S\left(\pi_S(F) + \frac{1}{2}\vec{\pi_S(F)F} + xv_1 + yv_2+zv_3\right)
        =\\&= \pi_S(\pi_S(F)) + \pi_{\vec{R}}\left(\frac{1}{2}\vec{\pi_S(F)F} + xv_1 + yv_2+zv_3\right) = \pi_S(F) + xv_1 + yv_2
        =\\&= m_{F\pi_S(F)} -\frac{1}{2}\vec{\pi_S(F)F} + xv_1 + yv_2
        = \left(x,y, -\dfrac{\left\|\vec{\pi_S(F)F}\right\|}{2}\right)_{\cc{R}}
    \end{align*}
    donde he hecho uso de que $v_1,v_2\in \vec{S},~~v_3, \vec{F\pi_S(F)} \in \vec{S}^\perp$. Por tanto, tenemos que:
    \begin{align*}
        p\in H &\Longleftrightarrow
        d(p,F) = d(p,S) \Longleftrightarrow d(p, F) = d(p,\pi_S(p)) \\ & \Longleftrightarrow
        \sqrt{x^2 +y^2+ \left(z-\frac{\left\|\vec{\pi_S(F)F}\right\|}{2}\right)^2} =\\&\hspace{2cm}=\sqrt{(x-x)^2 +(y-y)^2 + \left(z+\frac{\left\|\vec{\pi_S(F)F}\right\|}{2}\right)^2} \Longleftrightarrow \\ & \Longleftrightarrow
        x^2+y^2 + \left(z-\frac{\left\|\vec{\pi_S(F)F}\right\|}{2}\right)^2 = \left(z+\frac{\left\|\vec{\pi_S(F)F}\right\|}{2}\right)^2 \Longleftrightarrow \\ & \Longleftrightarrow
        x^2 +y^2 +\cancel{z^2} - z\left\|\vec{\pi_S(F)F}\right\| + \cancel{\frac{\left\|\vec{\pi_S(F)F}\right\|^2}{4}} =\\&\hspace{3cm}=\cancel{z^2} + z\left\|\vec{\pi_S(F)F}\right\| + \cancel{\frac{\left\|\vec{\pi_S(F)F}\right\|^2}{4}} \Longleftrightarrow \\ & \Longleftrightarrow
        2z = \frac{x^2+y^2}{\left\|\vec{\pi_S(F)F}\right\|}
    \end{align*}
    Por tanto, tenemos que se trata de un paraboloide elíptico.\\


    Una vez resuelto el ejercicio desde un enfoque más teórico y abstracto, vamos a obtener los resultados numéricos, sabiendo el valor de $F$ y de $S$.
    Tenemos que $S=(0,0,0) + \cc{L}\left\{(0,1,0), \frac{1}{\sqrt{2}}(1,0,1)\right\}$. Por tanto, $\vec{S}^\perp=\cc{L}\left\{\frac{1}{\sqrt{2}}(1,0,-1)\right\}$.
    Veamos si tomando $v_1=(0,1,0)$, $v_2=\frac{1}{\sqrt{2}}(1,0,1)$ y $v_3=\frac{1}{\sqrt{2}}(1,0,-1)$, tenemos que la base $\cc{B}=\{v_1,v_2,v_3\}$ es una base ortonormal orientada positivamente.
    Es directo ver que es ortonormal, por lo que comprobemos si es positivamente orientada:
    \begin{equation*}
        \frac{1}{2}\begin{vmatrix}
            0 & 1 & 1 \\
            1 & 0 & 0 \\
            0 & 1 & -1
        \end{vmatrix} = 1 > 0
    \end{equation*}
    Por tanto, $\cc{B}$ es una base ortonormal orientada positivamente. Calculamos ahora $\pi_S(F)$, para lo cual nos definimos un sistema de referencia auxiliar (no es el buscado) para
    realizar de forma más sencilla los cálculos. Sea este $\cc{R}_{aux}=\{0,\cc{B}\}$:
    \begin{equation*}
        \pi_S(0,0,1) = (0,0,0) + \pi_{\vec{S}} \left(0,0,1\right) = (0,0,0) + \left(0, \frac{\sqrt{2}}{2}, 0\right)_{\cc{B}} = \left(0, \frac{\sqrt{2}}{2}, 0\right)_{\cc{R}_{aux}}
    \end{equation*}
    para lo cual he tenido que calcular que $(0,0,1)=\left(0, \frac{\sqrt{2}}{2}, -\frac{\sqrt{2}}{2}\right)_{\cc{B}}$. Por tanto, calculamos ahora $m_{F \pi_S(F)}$:
    \begin{multline*}
        m_{F \pi_S(F)} = (0,0,1) + \frac{1}{2}\vec{F\pi_S(F)} = \left(0, \frac{\sqrt{2}}{2}, -\frac{\sqrt{2}}{2}\right)_{\cc{R}_{aux}} + \frac{1}{2}\left(0, 0, \frac{\sqrt{2}}{2}\right)_{\cc{B}} =\\= \left(0, \frac{\sqrt{2}}{2}, -\frac{\sqrt{2}}{4}\right)_{\cc{R}_{aux}}
    \end{multline*}

    Calculamos ahora las coordenadas de $m_{F \pi_S(F)}$ en el sistema de referencia usual:
    \begin{multline*}
        m_{F \pi_S(F)} = \left(0, \frac{\sqrt{2}}{2}, -\frac{\sqrt{2}}{4}\right)_{\cc{R}_{aux}}
        = \frac{\sqrt{2}}{2}\cdot \frac{1}{\sqrt{2}}(1,0,1) - \frac{\sqrt{2}}{4}\cdot \frac{1}{\sqrt{2}}(1,0,-1)
        =\\= \frac{1}{2}(1,0,1) - \frac{1}{4}(1,0,-1) = \left(\frac{1}{4}, 0, \frac{3}{4}\right)
    \end{multline*}

    Sea entonces $\cc{R}'=\{m_{F \pi_S(F)},\cc{B}\}$ el sistema de referencia buscado. Tenemos entonces que, en dicho sistema la ecuación asociada a $C$ es:
    \begin{equation*}
        2z = \frac{x^2+y^2}{\left\|\vec{\pi_S(F)F}\right\|} \Longrightarrow 2z = \frac{x^2+y^2}{\nicefrac{\sqrt{2}}{2}} \Longleftrightarrow \sqrt{2}z = x^2+y^2
    \end{equation*}
    Vemos que, efectivamente, es un paraboloide elíptico.
   \end{ejercicio} 
     
\end{document}