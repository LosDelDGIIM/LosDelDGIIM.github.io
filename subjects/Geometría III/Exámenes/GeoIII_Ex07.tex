\documentclass[12pt]{article}

% Idioma y codificación
\usepackage[spanish, es-tabla, es-notilde]{babel}       %es-tabla para que se titule "Tabla"
\usepackage[utf8]{inputenc}

% Márgenes
\usepackage[a4paper,top=3cm,bottom=2.5cm,left=3cm,right=3cm]{geometry}

% Comentarios de bloque
\usepackage{verbatim}

% Paquetes de links
\usepackage[hidelinks]{hyperref}    % Permite enlaces
\usepackage{url}                    % redirecciona a la web

% Más opciones para enumeraciones
\usepackage{enumitem}

% Personalizar la portada
\usepackage{titling}

% Paquetes de tablas
\usepackage{multirow}

% Para añadir el símbolo de euro
\usepackage{eurosym}


%------------------------------------------------------------------------

%Paquetes de figuras
\usepackage{caption}
\usepackage{subcaption} % Figuras al lado de otras
\usepackage{float}      % Poner figuras en el sitio indicado H.


% Paquetes de imágenes
\usepackage{graphicx}       % Paquete para añadir imágenes
\usepackage{transparent}    % Para manejar la opacidad de las figuras

% Paquete para usar colores
\usepackage[dvipsnames, table, xcdraw]{xcolor}
\usepackage{pagecolor}      % Para cambiar el color de la página

% Habilita tamaños de fuente mayores
\usepackage{fix-cm}

% Para los gráficos
\usepackage{tikz}
\usepackage{forest}

% Para poder situar los nodos en los grafos
\usetikzlibrary{positioning}


%------------------------------------------------------------------------

% Paquetes de matemáticas
\usepackage{mathtools, amsfonts, amssymb, mathrsfs}
\usepackage[makeroom]{cancel}     % Simplificar tachando
\usepackage{polynom}    % Divisiones y Ruffini
\usepackage{units} % Para poner fracciones diagonales con \nicefrac

\usepackage{pgfplots}   %Representar funciones
\pgfplotsset{compat=1.18}  % Versión 1.18

\usepackage{tikz-cd}    % Para usar diagramas de composiciones
\usetikzlibrary{calc}   % Para usar cálculo de coordenadas en tikz

%Definición de teoremas, etc.
\usepackage{amsthm}
%\swapnumbers   % Intercambia la posición del texto y de la numeración

\theoremstyle{plain}

\makeatletter
\@ifclassloaded{article}{
  \newtheorem{teo}{Teorema}[section]
}{
  \newtheorem{teo}{Teorema}[chapter]  % Se resetea en cada chapter
}
\makeatother

\newtheorem{coro}{Corolario}[teo]           % Se resetea en cada teorema
\newtheorem{prop}[teo]{Proposición}         % Usa el mismo contador que teorema
\newtheorem{lema}[teo]{Lema}                % Usa el mismo contador que teorema
\newtheorem*{lema*}{Lema}

\theoremstyle{remark}
\newtheorem*{observacion}{Observación}

\theoremstyle{definition}

\makeatletter
\@ifclassloaded{article}{
  \newtheorem{definicion}{Definición} [section]     % Se resetea en cada chapter
}{
  \newtheorem{definicion}{Definición} [chapter]     % Se resetea en cada chapter
}
\makeatother

\newtheorem*{notacion}{Notación}
\newtheorem*{ejemplo}{Ejemplo}
\newtheorem*{ejercicio*}{Ejercicio}             % No numerado
\newtheorem{ejercicio}{Ejercicio} [section]     % Se resetea en cada section


% Modificar el formato de la numeración del teorema "ejercicio"
\renewcommand{\theejercicio}{%
  \ifnum\value{section}=0 % Si no se ha iniciado ninguna sección
    \arabic{ejercicio}% Solo mostrar el número de ejercicio
  \else
    \thesection.\arabic{ejercicio}% Mostrar número de sección y número de ejercicio
  \fi
}


% \renewcommand\qedsymbol{$\blacksquare$}         % Cambiar símbolo QED
%------------------------------------------------------------------------

% Paquetes para encabezados
\usepackage{fancyhdr}
\pagestyle{fancy}
\fancyhf{}

\newcommand{\helv}{ % Modificación tamaño de letra
\fontfamily{}\fontsize{12}{12}\selectfont}
\setlength{\headheight}{15pt} % Amplía el tamaño del índice


%\usepackage{lastpage}   % Referenciar última pag   \pageref{LastPage}
%\fancyfoot[C]{%
%  \begin{minipage}{\textwidth}
%    \centering
%    ~\\
%    \thepage\\
%    \href{https://losdeldgiim.github.io/}{\texttt{\footnotesize losdeldgiim.github.io}}
%  \end{minipage}
%}
\fancyfoot[C]{\thepage}
\fancyfoot[R]{\href{https://losdeldgiim.github.io/}{\texttt{\footnotesize losdeldgiim.github.io}}}

%------------------------------------------------------------------------

% Conseguir que no ponga "Capítulo 1". Sino solo "1."
\makeatletter
\@ifclassloaded{book}{
  \renewcommand{\chaptermark}[1]{\markboth{\thechapter.\ #1}{}} % En el encabezado
    
  \renewcommand{\@makechapterhead}[1]{%
  \vspace*{50\p@}%
  {\parindent \z@ \raggedright \normalfont
    \ifnum \c@secnumdepth >\m@ne
      \huge\bfseries \thechapter.\hspace{1em}\ignorespaces
    \fi
    \interlinepenalty\@M
    \Huge \bfseries #1\par\nobreak
    \vskip 40\p@
  }}
}
\makeatother

%------------------------------------------------------------------------
% Paquetes de cógido
\usepackage{minted}
\renewcommand\listingscaption{Código fuente}

\usepackage{fancyvrb}
% Personaliza el tamaño de los números de línea
\renewcommand{\theFancyVerbLine}{\small\arabic{FancyVerbLine}}

% Estilo para C++
\newminted{cpp}{
    frame=lines,
    framesep=2mm,
    baselinestretch=1.2,
    linenos,
    escapeinside=||
}

% para minted
\definecolor{LightGray}{rgb}{0.95,0.95,0.92}
\setminted{
    linenos=true,
    stepnumber=5,
    numberfirstline=true,
    autogobble,
    breaklines=true,
    breakautoindent=true,
    breaksymbolleft=,
    breaksymbolright=,
    breaksymbolindentleft=0pt,
    breaksymbolindentright=0pt,
    breaksymbolsepleft=0pt,
    breaksymbolsepright=0pt,
    fontsize=\footnotesize,
    bgcolor=LightGray,
    numbersep=10pt
}


\usepackage{listings} % Para incluir código desde un archivo

\renewcommand\lstlistingname{Código Fuente}
\renewcommand\lstlistlistingname{Índice de Códigos Fuente}

% Definir colores
\definecolor{vscodepurple}{rgb}{0.5,0,0.5}
\definecolor{vscodeblue}{rgb}{0,0,0.8}
\definecolor{vscodegreen}{rgb}{0,0.5,0}
\definecolor{vscodegray}{rgb}{0.5,0.5,0.5}
\definecolor{vscodebackground}{rgb}{0.97,0.97,0.97}
\definecolor{vscodelightgray}{rgb}{0.9,0.9,0.9}

% Configuración para el estilo de C similar a VSCode
\lstdefinestyle{vscode_C}{
  backgroundcolor=\color{vscodebackground},
  commentstyle=\color{vscodegreen},
  keywordstyle=\color{vscodeblue},
  numberstyle=\tiny\color{vscodegray},
  stringstyle=\color{vscodepurple},
  basicstyle=\scriptsize\ttfamily,
  breakatwhitespace=false,
  breaklines=true,
  captionpos=b,
  keepspaces=true,
  numbers=left,
  numbersep=5pt,
  showspaces=false,
  showstringspaces=false,
  showtabs=false,
  tabsize=2,
  frame=tb,
  framerule=0pt,
  aboveskip=10pt,
  belowskip=10pt,
  xleftmargin=10pt,
  xrightmargin=10pt,
  framexleftmargin=10pt,
  framexrightmargin=10pt,
  framesep=0pt,
  rulecolor=\color{vscodelightgray},
  backgroundcolor=\color{vscodebackground},
}

%------------------------------------------------------------------------

% Comandos definidos
\newcommand{\bb}[1]{\mathbb{#1}}
\newcommand{\cc}[1]{\mathcal{#1}}

% I prefer the slanted \leq
\let\oldleq\leq % save them in case they're every wanted
\let\oldgeq\geq
\renewcommand{\leq}{\leqslant}
\renewcommand{\geq}{\geqslant}

% Si y solo si
\newcommand{\sii}{\iff}

% MCD y MCM
\DeclareMathOperator{\mcd}{mcd}
\DeclareMathOperator{\mcm}{mcm}

% Signo
\DeclareMathOperator{\sgn}{sgn}

% Letras griegas
\newcommand{\eps}{\epsilon}
\newcommand{\veps}{\varepsilon}
\newcommand{\lm}{\lambda}

\newcommand{\ol}{\overline}
\newcommand{\ul}{\underline}
\newcommand{\wt}{\widetilde}
\newcommand{\wh}{\widehat}

\let\oldvec\vec
\renewcommand{\vec}{\overrightarrow}

% Derivadas parciales
\newcommand{\del}[2]{\frac{\partial #1}{\partial #2}}
\newcommand{\Del}[3]{\frac{\partial^{#1} #2}{\partial #3^{#1}}}
\newcommand{\deld}[2]{\dfrac{\partial #1}{\partial #2}}
\newcommand{\Deld}[3]{\dfrac{\partial^{#1} #2}{\partial #3^{#1}}}


\newcommand{\AstIg}{\stackrel{(\ast)}{=}}
\newcommand{\Hop}{\stackrel{L'H\hat{o}pital}{=}}

\newcommand{\red}[1]{{\color{red}#1}} % Para integrales, destacar los cambios.

% Método de integración
\newcommand{\MetInt}[2]{
    \left[\begin{array}{c}
        #1 \\ #2
    \end{array}\right]
}

% Declarar aplicaciones
% 1. Nombre aplicación
% 2. Dominio
% 3. Codominio
% 4. Variable
% 5. Imagen de la variable
\newcommand{\Func}[5]{
    \begin{equation*}
        \begin{array}{rrll}
            \displaystyle #1:& \displaystyle  #2 & \longrightarrow & \displaystyle  #3\\
               & \displaystyle  #4 & \longmapsto & \displaystyle  #5
        \end{array}
    \end{equation*}
}

%------------------------------------------------------------------------



\begin{document}

    % 1. Foto de fondo
    % 2. Título
    % 3. Encabezado Izquierdo
    % 4. Color de fondo
    % 5. Coord x del titulo
    % 6. Coord y del titulo
    % 7. Fecha

    
    % 1. Foto de fondo
% 2. Título
% 3. Encabezado Izquierdo
% 4. Color de fondo
% 5. Coord x del titulo
% 6. Coord y del titulo
% 7. Fecha
% 8. Autor

\newcommand{\portada}[8]{
    \portadaBase{#1}{#2}{#3}{#4}{#5}{#6}{#7}{#8}
    \portadaBook{#1}{#2}{#3}{#4}{#5}{#6}{#7}{#8}
}

\newcommand{\portadaFotoDif}[8]{
    \portadaBaseFotoDif{#1}{#2}{#3}{#4}{#5}{#6}{#7}{#8}
    \portadaBook{#1}{#2}{#3}{#4}{#5}{#6}{#7}{#8}
}

\newcommand{\portadaExamen}[8]{
    \portadaBase{#1}{#2}{#3}{#4}{#5}{#6}{#7}{#8}
    \portadaArticle{#1}{#2}{#3}{#4}{#5}{#6}{#7}{#8}
}

\newcommand{\portadaExamenFotoDif}[8]{
    \portadaBaseFotoDif{#1}{#2}{#3}{#4}{#5}{#6}{#7}{#8}
    \portadaArticle{#1}{#2}{#3}{#4}{#5}{#6}{#7}{#8}
}




\newcommand{\portadaBase}[8]{

    % Tiene la portada principal y la licencia Creative Commons
    
    % 1. Foto de fondo
    % 2. Título
    % 3. Encabezado Izquierdo
    % 4. Color de fondo
    % 5. Coord x del titulo
    % 6. Coord y del titulo
    % 7. Fecha
    % 8. Autor    
    
    \thispagestyle{empty}               % Sin encabezado ni pie de página
    \newgeometry{margin=0cm}        % Márgenes nulos para la primera página
    
    
    % Encabezado
    \fancyhead[L]{\helv #3}
    \fancyhead[R]{\helv \nouppercase{\leftmark}}
    
    
    \pagecolor{#4}        % Color de fondo para la portada
    
    \begin{figure}[p]
        \centering
        \transparent{0.3}           % Opacidad del 30% para la imagen
        
        \includegraphics[width=\paperwidth, keepaspectratio]{../../_assets/#1}
    
        \begin{tikzpicture}[remember picture, overlay]
            \node[anchor=north west, text=white, opacity=1, font=\fontsize{60}{90}\selectfont\bfseries\sffamily, align=left] at (#5, #6) {#2};
            
            \node[anchor=south east, text=white, opacity=1, font=\fontsize{12}{18}\selectfont\sffamily, align=right] at (9.7, 3) {\href{https://losdeldgiim.github.io/}{\textbf{Los Del DGIIM}, \texttt{\footnotesize losdeldgiim.github.io}}};
            
            \node[anchor=south east, text=white, opacity=1, font=\fontsize{12}{15}\selectfont\sffamily, align=right] at (9.7, 1.8) {Doble Grado en Ingeniería Informática y Matemáticas\\Universidad de Granada};
        \end{tikzpicture}
    \end{figure}
    
    
    \restoregeometry        % Restaurar márgenes normales para las páginas subsiguientes
    \nopagecolor      % Restaurar el color de página
    
    
    \newpage
    \thispagestyle{empty}               % Sin encabezado ni pie de página
    \begin{tikzpicture}[remember picture, overlay]
        \node[anchor=south west, inner sep=3cm] at (current page.south west) {
            \begin{minipage}{0.5\paperwidth}
                \href{https://creativecommons.org/licenses/by-nc-nd/4.0/}{
                    \includegraphics[height=2cm]{../../_assets/Licencia.png}
                }\vspace{1cm}\\
                Esta obra está bajo una
                \href{https://creativecommons.org/licenses/by-nc-nd/4.0/}{
                    Licencia Creative Commons Atribución-NoComercial-SinDerivadas 4.0 Internacional (CC BY-NC-ND 4.0).
                }\\
    
                Eres libre de compartir y redistribuir el contenido de esta obra en cualquier medio o formato, siempre y cuando des el crédito adecuado a los autores originales y no persigas fines comerciales. 
            \end{minipage}
        };
    \end{tikzpicture}
    
    
    
    % 1. Foto de fondo
    % 2. Título
    % 3. Encabezado Izquierdo
    % 4. Color de fondo
    % 5. Coord x del titulo
    % 6. Coord y del titulo
    % 7. Fecha
    % 8. Autor


}


\newcommand{\portadaBaseFotoDif}[8]{

    % Tiene la portada principal y la licencia Creative Commons
    
    % 1. Foto de fondo
    % 2. Título
    % 3. Encabezado Izquierdo
    % 4. Color de fondo
    % 5. Coord x del titulo
    % 6. Coord y del titulo
    % 7. Fecha
    % 8. Autor    
    
    \thispagestyle{empty}               % Sin encabezado ni pie de página
    \newgeometry{margin=0cm}        % Márgenes nulos para la primera página
    
    
    % Encabezado
    \fancyhead[L]{\helv #3}
    \fancyhead[R]{\helv \nouppercase{\leftmark}}
    
    
    \pagecolor{#4}        % Color de fondo para la portada
    
    \begin{figure}[p]
        \centering
        \transparent{0.3}           % Opacidad del 30% para la imagen
        
        \includegraphics[width=\paperwidth, keepaspectratio]{#1}
    
        \begin{tikzpicture}[remember picture, overlay]
            \node[anchor=north west, text=white, opacity=1, font=\fontsize{60}{90}\selectfont\bfseries\sffamily, align=left] at (#5, #6) {#2};
            
            \node[anchor=south east, text=white, opacity=1, font=\fontsize{12}{18}\selectfont\sffamily, align=right] at (9.7, 3) {\href{https://losdeldgiim.github.io/}{\textbf{Los Del DGIIM}, \texttt{\footnotesize losdeldgiim.github.io}}};
            
            \node[anchor=south east, text=white, opacity=1, font=\fontsize{12}{15}\selectfont\sffamily, align=right] at (9.7, 1.8) {Doble Grado en Ingeniería Informática y Matemáticas\\Universidad de Granada};
        \end{tikzpicture}
    \end{figure}
    
    
    \restoregeometry        % Restaurar márgenes normales para las páginas subsiguientes
    \nopagecolor      % Restaurar el color de página
    
    
    \newpage
    \thispagestyle{empty}               % Sin encabezado ni pie de página
    \begin{tikzpicture}[remember picture, overlay]
        \node[anchor=south west, inner sep=3cm] at (current page.south west) {
            \begin{minipage}{0.5\paperwidth}
                %\href{https://creativecommons.org/licenses/by-nc-nd/4.0/}{
                %    \includegraphics[height=2cm]{../../_assets/Licencia.png}
                %}\vspace{1cm}\\
                Esta obra está bajo una
                \href{https://creativecommons.org/licenses/by-nc-nd/4.0/}{
                    Licencia Creative Commons Atribución-NoComercial-SinDerivadas 4.0 Internacional (CC BY-NC-ND 4.0).
                }\\
    
                Eres libre de compartir y redistribuir el contenido de esta obra en cualquier medio o formato, siempre y cuando des el crédito adecuado a los autores originales y no persigas fines comerciales. 
            \end{minipage}
        };
    \end{tikzpicture}
    
    
    
    % 1. Foto de fondo
    % 2. Título
    % 3. Encabezado Izquierdo
    % 4. Color de fondo
    % 5. Coord x del titulo
    % 6. Coord y del titulo
    % 7. Fecha
    % 8. Autor


}


\newcommand{\portadaBook}[8]{

    % 1. Foto de fondo
    % 2. Título
    % 3. Encabezado Izquierdo
    % 4. Color de fondo
    % 5. Coord x del titulo
    % 6. Coord y del titulo
    % 7. Fecha
    % 8. Autor

    % Personaliza el formato del título
    \pretitle{\begin{center}\bfseries\fontsize{42}{56}\selectfont}
    \posttitle{\par\end{center}\vspace{2em}}
    
    % Personaliza el formato del autor
    \preauthor{\begin{center}\Large}
    \postauthor{\par\end{center}\vfill}
    
    % Personaliza el formato de la fecha
    \predate{\begin{center}\huge}
    \postdate{\par\end{center}\vspace{2em}}
    
    \title{#2}
    \author{\href{https://losdeldgiim.github.io/}{Los Del DGIIM, \texttt{\large losdeldgiim.github.io}}
    \\ \vspace{0.5cm}#8}
    \date{Granada, #7}
    \maketitle
    
    \tableofcontents
}




\newcommand{\portadaArticle}[8]{

    % 1. Foto de fondo
    % 2. Título
    % 3. Encabezado Izquierdo
    % 4. Color de fondo
    % 5. Coord x del titulo
    % 6. Coord y del titulo
    % 7. Fecha
    % 8. Autor

    % Personaliza el formato del título
    \pretitle{\begin{center}\bfseries\fontsize{42}{56}\selectfont}
    \posttitle{\par\end{center}\vspace{2em}}
    
    % Personaliza el formato del autor
    \preauthor{\begin{center}\Large}
    \postauthor{\par\end{center}\vspace{3em}}
    
    % Personaliza el formato de la fecha
    \predate{\begin{center}\huge}
    \postdate{\par\end{center}\vspace{5em}}
    
    \title{#2}
    \author{\href{https://losdeldgiim.github.io/}{Los Del DGIIM, \texttt{\large losdeldgiim.github.io}}
    \\ \vspace{0.5cm}#8}
    \date{Granada, #7}
    \thispagestyle{empty}               % Sin encabezado ni pie de página
    \maketitle
    \vfill
}
    \portadaExamen{ffccA4.jpg}{Geometría III\\Examen VII}{Geometría III. Examen VII}{MidnightBlue}{-8}{28}{2023-2024}{Arturo Olivares Martos\\Irina Kuzyshyn Basarab}

    \begin{description}
        \item[Asignatura] Geometría III.
        \item[Curso Académico] 2023-24.
        \item[Grado] Doble Grado en Ingeniería Informática y Matemáticas.
        \item[Grupo] Único.
        \item[Profesor] Antonio Martínez López.
        \item[Descripción] Convocatoria Ordinaria\footnote{El examen lo pone el departamento.}.
        \item[Fecha] 22 de enero de 2023.
        \item[Duración] 3 horas.
    
    \end{description}
    \newpage

    \begin{ejercicio}[3 puntos]
        Razona:
        
        \begin{enumerate}
        \item (1,5 puntos) Sean \( p_1\neq p_2 \) dos puntos distintos de un plano afin euclideo \( \cc{A} \).
        Prueba que 
        \[
        \{ p \in \cc{A} : \langle \vec{p p_1}, \vec{p p_2} \rangle = 0 \}
        \]
        es una circunferencia. Calcula el centro y el radio de la misma.

        \noindent
        %Resulucion ej 1.1 ------------------------------------------------------------------------------------

        Sea $v_1=\dfrac{\vec{m_{p_1p_2}p_2}}{\|\vec{m_{p_1p_2}p_2}\|}$ vector normalizado y $v_2$ el único vector tal que $\cc{B} = \{v_1,v_2\}$ es una base ortonormal positivamente orientada. 

        \noindent
        Sea el sistema de referencia euclídeo $R=\{m_{p_1p_2},\cc{B}\}$. Sea $r:=\dfrac{\|\vec{p_1p_2\|}}{2}$

        Tenemos que $p_{1_R}=(-r,0), p_{2_R}=(r,0)$. Sea $p_R = (x,y)$. Tenemos que: 
        $$0 = \langle\vec{pp_1},\vec{pp_2}\rangle = \langle(-r-x,-y),(r-x,-y)\rangle $$
        $$ -(r+x)(r-x) + y^2 = -(r^2-x^2)+y^2 = x^2 + y^2 - r^2 = 0$$
        $$\Longrightarrow x^2 + y^2 = r^2$$

        Por tanto, se trata de una circunferencia de centro $m_{p_1p_2}$ y radio $r$.
        %--------------------------------------------------------------------------------------------------------
        
        \item (1,5 puntos) Sean \( S_1, S_2 \subset \bb{R}^3 \) dos planos afines distintos y \( f_1, f_2 : \mathbb{R}^3 \to \mathbb{R}^3 \) las simetrias especulares respecto a \( S_1 \) y \( S_2 \) respectivamente. Clasificar \( f = f_1 \circ f_2 \).
        % Resolucion 1.2 ---------------------------------------------------------------------------------------
        Como nos dicen que los planos son distintos, distinguimos varios casos, que estén paralelos y que se corten, concretamente en una recta y en este último caso distinguiremos el caso de que sean ortogonales.
        \begin{enumerate}
            \item Empecemos por el caso de que sean paralelos. En este caso tendremos una traslación. Veámoslo: \\
                Sea $p\in S_2$ y una base ortonormal $\cc{B}$ tal que los dos primeros vectores generen los planos. Sea entonces $\cc{R} = \{p,\cc{B}\}$ nuestro sistema de referencia. Tenemos entonces que las matrices asociadas a las dos reflexiones son las siguientes:
                $$ M(f_2,\cc{R})=
                \begin{pmatrix}
                    1 & 0 & 0 & 0 \\
                    0 & 1 & 0 & 0 \\
                    0 & 0 & 1 & 0 \\
                    0 & 0 & 0 & -1 \\
                \end{pmatrix}$$    
                Sea $x = d(S_1,S_2)$ la distancia entre los dos planos:
                $$ M(f_1,\cc{R})=
                \begin{pmatrix}
                    1 & 0 & 0 & 0 \\
                    0 & 1 & 0 & 0 \\
                    0 & 0 & 1 & 0 \\
                    2x & 0 & 0 & -1 \\
                \end{pmatrix} $$   
                Donde $(0,0,2x)$ es la imagen por la segunda reflexión del origen de nuestro sistema de referencia. Ahora bien, sabemos que la composición de aplicaciones afines viene dada por el producto de las matrices asociadas:
                $$f_1\circ f_2=M(f_1,\cc{R})\cdot M(f_2,\cc{R}) = 
                \begin{pmatrix}
                    1 & 0 & 0 & 0 \\
                    0 & 1 & 0 & 0 \\
                    0 & 0 & 1 & 0 \\
                    2x & 0 & 0 & 1\\
                \end{pmatrix} $$   
                Tenemos entonces una traslación de vector ortogonal a los planos y de módulo el doble de la distancia entre los planos.
            \item Veamos ahora el caso de que los planos sean ortogonales. En este caso tenemos una reflexión axial. Veámoslo:\\
                Sea $r=S_1\cap S_2$, la recta de intersección entre los dos planos. Tomo como sistema de referencia $\cc{R}=\{p,\cc{B}\}$ tal que $p\in r$ y $\cc{B}$ es una base ortonormal cuyo primer vector esta en la recta $r$ y los otros dos en los planos $S_1$ y $S_2$ respectivamente. 
                Las matrices asociadas a las dos reflexiones que tratamos en el sistema de referencia que acabamos de definir son las siguientes:
                $$\cc{M}(f_1,\cc{R})=
                \begin{pmatrix}
                    1 & 0 & 0 & 0 \\
                    0 & 1 & 0 & 0 \\
                    0 & 0 & 1 & 0 \\
                    0 & 0 & 0 & -1 \\
                \end{pmatrix}$$    
                $$\cc{M}(f_2,\cc{R})=
                \begin{pmatrix}
                    1 & 0 & 0 & 0 \\
                    0 & 1 & 0 & 0 \\
                    0 & 0 & -1 & 0 \\
                    0 & 0 & 0 & 1 \\
                \end{pmatrix}$$    
                Igual que antes, para sacar la matriz de la composición multiplicamos las matrices de las aplicaciones:

                $$f_1\circ f_2=\cc{M}(f_1,\cc{R})\cdot \cc{M}(f_2,\cc{R}) = 
                \begin{pmatrix}
                    1 & 0 & 0 & 0 \\
                    0 & 1 & 0 & 0 \\
                    0 & 0 & -1 & 0 \\
                    0 & 0 & 0 & -1\\
                \end{pmatrix} $$   
                Vemos así que es una reflexión axial que deja fija a la recta $r$.
            
            \item Por último veamos el caso de que los planos se corten pero no sean ortogonales. En este caso tendremos un giro. Veámoslo: \\
                Sea $r=S_1\cap S_2$, la recta de intersección entre los dos planos. Tenemos que la recta es invariante, pues los planos son invariantes para cada simetría. Sea $\theta$ el ángulo entre los dos planos (viéndolo desde el primer plano al segundo que reflejamos). Tenemos pues que nuestra $f$ es un giro de ángulo $2\theta$ respecto a la recta $r$. \\
                Podemos también verlo de la seguiente manera: si cortamos nuestros planos por uno perpendicular a ambos tenemos dos rectas que se cortan en un punto $p\in r$ y nuestra $f$ en ese plano sabemos que efectivamente es un giro de ángulo $2\theta$ respecto del punto de corte. 

        \end{enumerate}
        %---------------------------------------------------------------------------------------
        \end{enumerate}
        \end{ejercicio}
        
        \begin{ejercicio}[2 puntos]
        Se considera la aplicación \( f : P_2(\mathbb{R}) \to \mathbb{R}^3 \), siendo \( P_2(\mathbb{R}) \) el espacio de polinomios de orden 2 con coeficientes reales,
        \[
        f(p(x)) = (p(0) + 1, p(1), p'(1)-1).
        \]
        
        \begin{enumerate}
        \item (1 punto) Demuestra que \( f \) es afín y encuentra la expresión matricial de \( f \) respecto de los sistemas de referencia canónicos
        \( \cc{R}_0' \) de \( P_2(\mathbb{R}) \) y \( \cc{R}_0 \) de \( \mathbb{R}^3 \). (En \( \cc{R}_0' \), el polinomio $0$ representa el origen del sistema de referencia y
        \( \cc{B}_0' = \{1, x, x^2\} \) la base asociada.)\\
        %Resolucion ej 2.1 --------------------------------------------------------------------
        Sea $p(x)=a+bx+cx^2 \in P_2(\bb{R})$ con $a,b,c \in \bb{R}$. Luego tenemos:
        $$f(p(x))=(a+1,a+b+c,2c+b-1)$$
        Pera ver que es afín tenemos que ver que la asociada es lineal. Para ello sean: 
        $$p(x)=a+bx+cx^2, \mbox{    } q(x)=a'+b'x+c'x^2$$ 
        $$s(x)=d+ex+fx^2, \mbox{    } t(x)=d'+e'x+f'x^2$$
        $$\vec{f}(\vec{pq})=\vec{f(p)f(q)} = f(q) - f(p) = (a'-a,a'-a + b'-b + c'-c, 2c'-2c + b'-b)$$
        Si ahora llamamos $u=a'-a, v=b'-b, w=c'-c$ tenemos que:
        $$\vec{f}(\vec{pq}) = (u,u+v+w,2w+v)$$
        $$\vec{f}(\vec{st})=\vec{f(s)f(t)} = f(t) - f(s) = (d'-d,d'-d + e'-e + f'-f, 2f'-2f + e'-e)$$
        Si ahora llamamos $u'=d'-d, v'=e'-e, w'=f'-f$ tenemos que:
        $$\vec{f}(\vec{st}) = (u',u'+v'+w',2w'+v')$$
        Tenemos que ver que la $\vec{f}$ es lineal. 
        $$\vec{pq} + \vec{st} = (u+u') + (v+v')x + (w+w')x^2$$
        $$\vec{f}(\vec{pq} + \vec{st})= (u+u',u+u'+v+v'+w+w',2(w+w')+v+v') = $$ 
        $$(u,u+v+w,2w+v) + (u',u'+v'+w',2w'+v') = \vec{f}(\vec{pq}) + \vec{f}(\vec{st})$$
        $$\vec{f}(\lambda\vec{pq}) = (\lambda u,\lambda u+\lambda v+\lambda w,\lambda2w+\lambda v)= $$
        $$ \lambda (u,u+v+w,2w+v) = \lambda \vec{f}(\vec{pq})$$
        Tenemos así demostrado que la asociada es lineal y que por tanto $f$ es afín.\\
        Calculamos ahora la expresión matricial de $f$.
        $$f(0) = (1,0,-1)$$
        $$\vec{f}(1)=(1,1,0)$$
        $$\vec{f}(x)=(0,1,1)$$
        $$\vec{f}(x^2)=(0,1,2)$$
        $$\cc{M}(f,\cc{R}'_0,\cc{R}_0) = 
                \begin{pmatrix}
                    1 & 0 & 0 & 0 \\
                    1 & 1 & 0 & 0 \\
                    0 & 1 & 1 & 1 \\
                    -1 & 0 & 1 & 2 \\
                \end{pmatrix}$$    
        %--------------------------------------------------------------------------------------------
        \item (1 punto) Comprueba que \( S = \{p(x) \in P_2(\mathbb{R}) : p(0)+p(1) = 2\} \) es un subespacio de \( P_2(\mathbb{R}) \) y determina sus ecuaciones implícitas en \( \cc{R}_0' \).\\
        %Resolucion 2.1-----------------------------------------------------------------------------
        Sea $p(x)=a+bx+cx^2\in S$; $p(0) = a$, $p(1) = a+b+c$. Luego tenemos que se cumple lo siguiente: 
        $$2a+b+c = 2 \mbox{ lo cual es la ecuación implícita.}$$
        Veamos ahora que es subespacio afín. Tenemos que escribir $S$ de la siguiente forma:
        $$S=q+U \mbox{ con U subespacio vectorial. }$$
        Claramente $q(x)=1 \in S$ y sea $U$ tal que se cumple la ecuación implícita homogénea $2a+b+c=0$. Veamos la igualdad por doble inclusión. 
        \begin{description}
            \item [$\subseteq )$] Sea $p(x)= a + bx + cx^2 \in S \implies 2a+b+c = 2$ \\
                $$\vec{1p} = p-1 = (a-1)+bx+cx^2 \mbox{ ¿}\in U?$$
               $$2(a-1) + b + c = -2 + 2a + b + c = -2 + 2= 0 \implies \vec{1p}\in U$$ 
            \item [$\supseteq )$] Sea $v= a+bx+cx^2\in U \implies 2a+b+c = 0$
                $$p = 1 + v = 1 + a + bx + cx^2 \mbox{ ¿}\in S?$$
                $$2(a+1) + b  + c = 2a + 2 + b + c = 0 + 2 = 2 \implies p \in S$$
        \end{description}
        \end{enumerate}
        \end{ejercicio}
        
        \begin{ejercicio}[3 puntos] Clasifica euclídeamente la siguiente cónica encontrando el sistema de referencia euclídeo en el que adopta su ecuación reducida. Calcula sus elementos euclídeos (ejes, centro, focos, asíntotas):
        \[
        -7 - 4x + 2x^2 + 4y + 8xy + 2y^2 = 0.
        \]
        Tenemos que la matriz que representa la cónica es la siguiente:
        $$M = \begin{pmatrix}
                    -7 & -2 & 2\\
                    -2 & 2 & 4\\
                    2 & 4 & 2\\
            \end{pmatrix}
            \qquad
        A = \begin{pmatrix}
            2 & 4 \\
            4 & 2 \\
        \end{pmatrix} $$
        Calculamos el polinomio característico de $A$:
        $$\cc{P}_A(\lambda) = (2-\lambda)^2 - 16 = 4 + \lambda^2 -4\lambda - 16 =$$ 
        $$\lambda^2 - 4\lambda - 12 = 0 \sii \lambda = 6 \mbox{ ó } \lambda = -2$$
        Tenemos un valor propio positivo y otro negativo, tenemos por tanto una hipérbola. Faltaría ahora calcular los elementos euclídeos: los ejes, el centro, los focos y asíntotas.
        \end{ejercicio}
        %--------------------------------------------------------------------------------------------
        
        \begin{ejercicio}[2 puntos] Estudia si existe una proyectividad \( f : \mathbb{P}^2 \to \mathbb{P}^2 \) del plano proyectivo real en el plano proyectivo real, verificando
        \begin{align*}
            f(0 : 1 : 0) &= (1 : 1 : 1), \\
            f(0 : 0 : 1) &= (1 : 0 : 0), \\
            f(1 : 0 : -1) &= (0 : 1 : 0), \\
            f(2 : -2 : 1) &= (0 : 0 : 1).
        \end{align*}
        
        En caso afirmativo calcula su expresión en coordenadas homogéneas usuales y decide si es o no biyectiva (homografía).
        \end{ejercicio}
        
     
\end{document}
