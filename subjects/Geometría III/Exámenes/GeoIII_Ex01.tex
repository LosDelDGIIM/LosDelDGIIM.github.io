\documentclass[12pt]{article}

% Idioma y codificación
\usepackage[spanish, es-tabla]{babel}       %es-tabla para que se titule "Tabla"
\usepackage[utf8]{inputenc}

% Márgenes
\usepackage[a4paper,top=3cm,bottom=2.5cm,left=3cm,right=3cm]{geometry}

% Comentarios de bloque
\usepackage{verbatim}

% Paquetes de links
\usepackage[hidelinks]{hyperref}    % Permite enlaces
\usepackage{url}                    % redirecciona a la web

% Más opciones para enumeraciones
\usepackage{enumitem}

% Personalizar la portada
\usepackage{titling}

% Paquetes de tablas
\usepackage{multirow}


%------------------------------------------------------------------------

%Paquetes de figuras
\usepackage{caption}
\usepackage{subcaption} % Figuras al lado de otras
\usepackage{float}      % Poner figuras en el sitio indicado H.


% Paquetes de imágenes
\usepackage{graphicx}       % Paquete para añadir imágenes
\usepackage{transparent}    % Para manejar la opacidad de las figuras

% Paquete para usar colores
\usepackage[dvipsnames]{xcolor}
\usepackage{pagecolor}      % Para cambiar el color de la página

% Habilita tamaños de fuente mayores
\usepackage{fix-cm}

% Para los gráficos
\usepackage{tikz}

% Para poder situar los nodos en los grafos
\usetikzlibrary{positioning}


%------------------------------------------------------------------------

% Paquetes de matemáticas
\usepackage{mathtools, amsfonts, amssymb, mathrsfs}
\usepackage[makeroom]{cancel}     % Simplificar tachando
\usepackage{polynom}    % Divisiones y Ruffini
\usepackage{units} % Para poner fracciones diagonales con \nicefrac

\usepackage{pgfplots}   %Representar funciones
\pgfplotsset{compat=1.18}  % Versión 1.18

\usepackage{tikz-cd}    % Para usar diagramas de composiciones
\usetikzlibrary{calc}   % Para usar cálculo de coordenadas en tikz

%Definición de teoremas, etc.
\usepackage{amsthm}
%\swapnumbers   % Intercambia la posición del texto y de la numeración

\theoremstyle{plain}

\makeatletter
\@ifclassloaded{article}{
  \newtheorem{teo}{Teorema}[section]
}{
  \newtheorem{teo}{Teorema}[chapter]  % Se resetea en cada chapter
}
\makeatother

\newtheorem{coro}{Corolario}[teo]           % Se resetea en cada teorema
\newtheorem{prop}[teo]{Proposición}         % Usa el mismo contador que teorema
\newtheorem{lema}[teo]{Lema}                % Usa el mismo contador que teorema

\theoremstyle{remark}
\newtheorem*{observacion}{Observación}

\theoremstyle{definition}

\makeatletter
\@ifclassloaded{article}{
  \newtheorem{definicion}{Definición} [section]     % Se resetea en cada chapter
}{
  \newtheorem{definicion}{Definición} [chapter]     % Se resetea en cada chapter
}
\makeatother

\newtheorem*{notacion}{Notación}
\newtheorem*{ejemplo}{Ejemplo}
\newtheorem*{ejercicio*}{Ejercicio}             % No numerado
\newtheorem{ejercicio}{Ejercicio} [section]     % Se resetea en cada section


% Modificar el formato de la numeración del teorema "ejercicio"
\renewcommand{\theejercicio}{%
  \ifnum\value{section}=0 % Si no se ha iniciado ninguna sección
    \arabic{ejercicio}% Solo mostrar el número de ejercicio
  \else
    \thesection.\arabic{ejercicio}% Mostrar número de sección y número de ejercicio
  \fi
}


% \renewcommand\qedsymbol{$\blacksquare$}         % Cambiar símbolo QED
%------------------------------------------------------------------------

% Paquetes para encabezados
\usepackage{fancyhdr}
\pagestyle{fancy}
\fancyhf{}

\newcommand{\helv}{ % Modificación tamaño de letra
\fontfamily{}\fontsize{12}{12}\selectfont}
\setlength{\headheight}{15pt} % Amplía el tamaño del índice


%\usepackage{lastpage}   % Referenciar última pag   \pageref{LastPage}
\fancyfoot[C]{\thepage}

%------------------------------------------------------------------------

% Conseguir que no ponga "Capítulo 1". Sino solo "1."
\makeatletter
\@ifclassloaded{book}{
  \renewcommand{\chaptermark}[1]{\markboth{\thechapter.\ #1}{}} % En el encabezado
    
  \renewcommand{\@makechapterhead}[1]{%
  \vspace*{50\p@}%
  {\parindent \z@ \raggedright \normalfont
    \ifnum \c@secnumdepth >\m@ne
      \huge\bfseries \thechapter.\hspace{1em}\ignorespaces
    \fi
    \interlinepenalty\@M
    \Huge \bfseries #1\par\nobreak
    \vskip 40\p@
  }}
}
\makeatother

%------------------------------------------------------------------------
% Paquetes de cógido
\usepackage{minted}
\renewcommand\listingscaption{Código fuente}

\usepackage{fancyvrb}
% Personaliza el tamaño de los números de línea
\renewcommand{\theFancyVerbLine}{\small\arabic{FancyVerbLine}}

% Estilo para C++
\newminted{cpp}{
    frame=lines,
    framesep=2mm,
    baselinestretch=1.2,
    linenos,
    escapeinside=||
}

% para minted
\definecolor{LightGray}{rgb}{0.95,0.95,0.92}
\setminted{
    linenos=true,
    stepnumber=5,
    numberfirstline=true,
    autogobble,
    breaklines=true,
    breakautoindent=true,
    breaksymbolleft=,
    breaksymbolright=,
    breaksymbolindentleft=0pt,
    breaksymbolindentright=0pt,
    breaksymbolsepleft=0pt,
    breaksymbolsepright=0pt,
    fontsize=\footnotesize,
    bgcolor=LightGray,
    numbersep=10pt
}


\usepackage{listings} % Para incluir código desde un archivo

\renewcommand\lstlistingname{Código Fuente}
\renewcommand\lstlistlistingname{Índice de Códigos Fuente}

% Definir colores
\definecolor{vscodepurple}{rgb}{0.5,0,0.5}
\definecolor{vscodeblue}{rgb}{0,0,0.8}
\definecolor{vscodegreen}{rgb}{0,0.5,0}
\definecolor{vscodegray}{rgb}{0.5,0.5,0.5}
\definecolor{vscodebackground}{rgb}{0.97,0.97,0.97}
\definecolor{vscodelightgray}{rgb}{0.9,0.9,0.9}

% Configuración para el estilo de C similar a VSCode
\lstdefinestyle{vscode_C}{
  backgroundcolor=\color{vscodebackground},
  commentstyle=\color{vscodegreen},
  keywordstyle=\color{vscodeblue},
  numberstyle=\tiny\color{vscodegray},
  stringstyle=\color{vscodepurple},
  basicstyle=\scriptsize\ttfamily,
  breakatwhitespace=false,
  breaklines=true,
  captionpos=b,
  keepspaces=true,
  numbers=left,
  numbersep=5pt,
  showspaces=false,
  showstringspaces=false,
  showtabs=false,
  tabsize=2,
  frame=tb,
  framerule=0pt,
  aboveskip=10pt,
  belowskip=10pt,
  xleftmargin=10pt,
  xrightmargin=10pt,
  framexleftmargin=10pt,
  framexrightmargin=10pt,
  framesep=0pt,
  rulecolor=\color{vscodelightgray},
  backgroundcolor=\color{vscodebackground},
}

%------------------------------------------------------------------------

% Comandos definidos
\newcommand{\bb}[1]{\mathbb{#1}}
\newcommand{\cc}[1]{\mathcal{#1}}

% I prefer the slanted \leq
\let\oldleq\leq % save them in case they're every wanted
\let\oldgeq\geq
\renewcommand{\leq}{\leqslant}
\renewcommand{\geq}{\geqslant}

% Si y solo si
\newcommand{\sii}{\iff}

% Letras griegas
\newcommand{\eps}{\epsilon}
\newcommand{\veps}{\varepsilon}
\newcommand{\lm}{\lambda}

\newcommand{\ol}{\overline}
\newcommand{\ul}{\underline}
\newcommand{\wt}{\widetilde}
\newcommand{\wh}{\widehat}

\let\oldvec\vec
\renewcommand{\vec}{\overrightarrow}

% Derivadas parciales
\newcommand{\del}[2]{\frac{\partial #1}{\partial #2}}
\newcommand{\Del}[3]{\frac{\partial^{#1} #2}{\partial #3^{#1}}}
\newcommand{\deld}[2]{\dfrac{\partial #1}{\partial #2}}
\newcommand{\Deld}[3]{\dfrac{\partial^{#1} #2}{\partial #3^{#1}}}


\newcommand{\AstIg}{\stackrel{(\ast)}{=}}
\newcommand{\Hop}{\stackrel{L'H\hat{o}pital}{=}}

\newcommand{\red}[1]{{\color{red}#1}} % Para integrales, destacar los cambios.

% Método de integración
\newcommand{\MetInt}[2]{
    \left[\begin{array}{c}
        #1 \\ #2
    \end{array}\right]
}

% Declarar aplicaciones
% 1. Nombre aplicación
% 2. Dominio
% 3. Codominio
% 4. Variable
% 5. Imagen de la variable
\newcommand{\Func}[5]{
    \begin{equation*}
        \begin{array}{rrll}
            #1:& #2 & \longrightarrow & #3\\
               & #4 & \longmapsto & #5
        \end{array}
    \end{equation*}
}

%------------------------------------------------------------------------



\begin{document}

    % 1. Foto de fondo
    % 2. Título
    % 3. Encabezado Izquierdo
    % 4. Color de fondo
    % 5. Coord x del titulo
    % 6. Coord y del titulo
    % 7. Fecha

    
    % 1. Foto de fondo
% 2. Título
% 3. Encabezado Izquierdo
% 4. Color de fondo
% 5. Coord x del titulo
% 6. Coord y del titulo
% 7. Fecha

\newcommand{\portada}[7]{

    \portadaBase{#1}{#2}{#3}{#4}{#5}{#6}{#7}
    \portadaBook{#1}{#2}{#3}{#4}{#5}{#6}{#7}
}

\newcommand{\portadaExamen}[7]{

    \portadaBase{#1}{#2}{#3}{#4}{#5}{#6}{#7}
    \portadaArticle{#1}{#2}{#3}{#4}{#5}{#6}{#7}
}




\newcommand{\portadaBase}[7]{

    % Tiene la portada principal y la licencia Creative Commons
    
    % 1. Foto de fondo
    % 2. Título
    % 3. Encabezado Izquierdo
    % 4. Color de fondo
    % 5. Coord x del titulo
    % 6. Coord y del titulo
    % 7. Fecha
    
    
    \thispagestyle{empty}               % Sin encabezado ni pie de página
    \newgeometry{margin=0cm}        % Márgenes nulos para la primera página
    
    
    % Encabezado
    \fancyhead[L]{\helv #3}
    \fancyhead[R]{\helv \nouppercase{\leftmark}}
    
    
    \pagecolor{#4}        % Color de fondo para la portada
    
    \begin{figure}[p]
        \centering
        \transparent{0.3}           % Opacidad del 30% para la imagen
        
        \includegraphics[width=\paperwidth, keepaspectratio]{assets/#1}
    
        \begin{tikzpicture}[remember picture, overlay]
            \node[anchor=north west, text=white, opacity=1, font=\fontsize{60}{90}\selectfont\bfseries\sffamily, align=left] at (#5, #6) {#2};
            
            \node[anchor=south east, text=white, opacity=1, font=\fontsize{12}{18}\selectfont\sffamily, align=right] at (9.7, 3) {\textbf{\href{https://losdeldgiim.github.io/}{Los Del DGIIM}}};
            
            \node[anchor=south east, text=white, opacity=1, font=\fontsize{12}{15}\selectfont\sffamily, align=right] at (9.7, 1.8) {Doble Grado en Ingeniería Informática y Matemáticas\\Universidad de Granada};
        \end{tikzpicture}
    \end{figure}
    
    
    \restoregeometry        % Restaurar márgenes normales para las páginas subsiguientes
    \pagecolor{white}       % Restaurar el color de página
    
    
    \newpage
    \thispagestyle{empty}               % Sin encabezado ni pie de página
    \begin{tikzpicture}[remember picture, overlay]
        \node[anchor=south west, inner sep=3cm] at (current page.south west) {
            \begin{minipage}{0.5\paperwidth}
                \href{https://creativecommons.org/licenses/by-nc-nd/4.0/}{
                    \includegraphics[height=2cm]{assets/Licencia.png}
                }\vspace{1cm}\\
                Esta obra está bajo una
                \href{https://creativecommons.org/licenses/by-nc-nd/4.0/}{
                    Licencia Creative Commons Atribución-NoComercial-SinDerivadas 4.0 Internacional (CC BY-NC-ND 4.0).
                }\\
    
                Eres libre de compartir y redistribuir el contenido de esta obra en cualquier medio o formato, siempre y cuando des el crédito adecuado a los autores originales y no persigas fines comerciales. 
            \end{minipage}
        };
    \end{tikzpicture}
    
    
    
    % 1. Foto de fondo
    % 2. Título
    % 3. Encabezado Izquierdo
    % 4. Color de fondo
    % 5. Coord x del titulo
    % 6. Coord y del titulo
    % 7. Fecha


}


\newcommand{\portadaBook}[7]{

    % 1. Foto de fondo
    % 2. Título
    % 3. Encabezado Izquierdo
    % 4. Color de fondo
    % 5. Coord x del titulo
    % 6. Coord y del titulo
    % 7. Fecha

    % Personaliza el formato del título
    \pretitle{\begin{center}\bfseries\fontsize{42}{56}\selectfont}
    \posttitle{\par\end{center}\vspace{2em}}
    
    % Personaliza el formato del autor
    \preauthor{\begin{center}\Large}
    \postauthor{\par\end{center}\vfill}
    
    % Personaliza el formato de la fecha
    \predate{\begin{center}\huge}
    \postdate{\par\end{center}\vspace{2em}}
    
    \title{#2}
    \author{\href{https://losdeldgiim.github.io/}{Los Del DGIIM}}
    \date{Granada, #7}
    \maketitle
    
    \tableofcontents
}




\newcommand{\portadaArticle}[7]{

    % 1. Foto de fondo
    % 2. Título
    % 3. Encabezado Izquierdo
    % 4. Color de fondo
    % 5. Coord x del titulo
    % 6. Coord y del titulo
    % 7. Fecha

    % Personaliza el formato del título
    \pretitle{\begin{center}\bfseries\fontsize{42}{56}\selectfont}
    \posttitle{\par\end{center}\vspace{2em}}
    
    % Personaliza el formato del autor
    \preauthor{\begin{center}\Large}
    \postauthor{\par\end{center}\vspace{3em}}
    
    % Personaliza el formato de la fecha
    \predate{\begin{center}\huge}
    \postdate{\par\end{center}\vspace{5em}}
    
    \title{#2}
    \author{\href{https://losdeldgiim.github.io/}{Los Del DGIIM}}
    \date{Granada, #7}
    \thispagestyle{empty}               % Sin encabezado ni pie de página
    \maketitle
    \vfill
}
    \portadaExamen{ffccA4.jpg}{Geometría III\\Examen I}{Geometría III. Examen I}{MidnightBlue}{-8}{28}{2023-2024}{Arturo Olivares Martos}

    \begin{description}
        \item[Asignatura] Geometría III.
        \item[Curso Académico] 2023-24.
        \item[Grado] Grado en Matemáticas.
        \item[Grupo] B.
        \item[Profesor] José María Espinar García.
        \item[Descripción] Parcial de los Temas 2 y 3.
        \item[Fecha] 11 de diciembre de 2023.
        % \item[Duración] 60 minutos.
    
    \end{description}
    \newpage
    
    \begin{ejercicio}[3 puntos]
        Clasificar afínmente las cónicas que se obtienen al cortar el cono de ecuación
        $x^2+y^2=z^2$ con un plano arbitrario $ax+bz+c=0$, con $a,b,c\in \bb{R}$.\\

        Sea $H\equiv x^2+y^2-z^2=0$ el cono, y $\pi\equiv ax+bz+c=0$ el plano. Tenemos que:
        \begin{equation*}
            H \cap \pi = \left\{(x,y,z)\in \bb{R}^3\mid x^2+y^2=z^2, ax+bz+c=0\right\}
        \end{equation*}

        Distinguimos en función de los siguientes casos:
        \begin{itemize}
            \item \ul{$b=0$}:
            
            En este caso, $\pi\equiv ax+c=0$, y por tanto, $x=-\nicefrac{c}{a}$.
            Sustituyendo en la ecuación del cono, tenemos que:
            \begin{equation*}
                \left(-\frac{c}{a}\right)^2+y^2=z^2
                \Longrightarrow
                y^2-z^2 = -\frac{c^2}{a^2}
            \end{equation*}

            Por tanto, tenemos que:
            \begin{equation*}
                H\cap \pi = \left\{(x,y,z)\in \bb{R}^3\mid y^2-z^2=-\frac{c^2}{a^2}, x=-\frac{c}{a}\right\}
            \end{equation*}

            \begin{itemize}
                \item \ul{$a = b=c=0$}:
                Tenemos que el plano no tiene ecuación, por lo que $\pi=\bb{R}^3$ y, por tanto,
                la intersección con $H$ es $H$.

                \item \ul{$a=b=0$, $c\neq 0$}:
                La ecuación del plano es $0=-c \neq 0$, por lo que $\pi=\emptyset$ y, por tanto,
                $H\cap \pi=\emptyset$.

                \item \ul{$b=c=0$, $a\neq 0$}:
                En este caso, vemos que $H\cap \pi\equiv y^2-z^2=0$, por lo que
                se trata de un par de rectas secantes y perpediculares en el plano $x=0$.

                \item \ul{$b=0$, $a,c\neq 0$}:
                En este caso, vemos que se trata de una hipérbola en el plano $x=-\nicefrac{c}{a}$.
            \end{itemize}


            \item \ul{$b\neq 0$, $a=0$}:
            
            En este caso, $\pi\equiv bz+c=0$, y por tanto, $z=-\nicefrac{c}{b}$.
            Sustituyendo en la ecuación del cono, tenemos que:
            \begin{equation*}
                x^2+y^2=\left(-\frac{c}{b}\right)^2 = 0
                \Longrightarrow
                x^2+y^2 = \frac{c^2}{b^2}
            \end{equation*}

            Por tanto, tenemos que:
            \begin{equation*}
                H\cap \pi = \left\{(x,y,z)\in \bb{R}^3\mid x^2+y^2=\frac{c^2}{b^2}, z=-\frac{c}{b}\right\}
            \end{equation*}

            \begin{itemize}
                \item \ul{$a=c=0$, $b\neq 0$}:
                En este caso, vemos que $H\cap \pi\equiv x^2+y^2=0$, por lo que
                se trata de un punto en el plano $z=0$, es decir, el origen.

                \item \ul{$a=0$, $b,c\neq 0$}:
                En este caso, vemos que se trata de una circunferencia en el plano
                $z=-\nicefrac{c}{b}$.
            \end{itemize}

            \item \ul{$a,b\neq 0$}:
            
            En este caso, $\pi\equiv ax+bz+c=0$, y por tanto, $z=-\nicefrac{c+ax}{b}$.
            Sustituyendo en la ecuación del cono, tenemos que:
            \begin{multline*}
                x^2+y^2=\left(-\frac{c+ax}{b}\right)^2 = \frac{a^2x^2+c^2+2axc}{b^2}
                \Longrightarrow
                b^2x^2+b^2y^2=a^2x^2+c^2+2axc
                \Longrightarrow \\ \Longrightarrow
                 (b^2-a^2)x^2+b^2y^2-2axc-c^2=0
            \end{multline*}

            Distinguimos en función de los valores de $c$:
            \begin{itemize}
                \item \ul{$a,b\neq 0$, $c=0$}:
                
                En este caso, tenemos que:
                \begin{equation*}
                    H\cap \pi = \left\{(x,y,z)\in \bb{R}^3\mid (b^2-a^2)x^2+b^2y^2=0, z=-\frac{ax}{b}\right\}
                \end{equation*}
                Dintinguimos en funión de los valores de $a$ y $b$:
                \begin{itemize}
                    \item \ul{$0<|a|<|b|$, $c=0$}:
                    
                    Tenemos que $b^2-a^2>0$, por lo que $H\cap \pi$ es un punto, el origen.

                    \item \ul{$0<|b|<|a|$, $c=0$}:
                    
                    Tenemos que $b^2-a^2<0$, por lo que $H\cap \pi$ es un par de rectas secantes.

                    \item \ul{$0<|a|=|b|$, $c=0$}:
                    
                    Tenemos que $b^2-a^2=0$, por lo que $H\cap \pi$ es una recta dada por
                    $y=0$, $z=-\nicefrac{ax}{b}$.
                \end{itemize}

                \item \ul{$a,b, c\neq 0$}:
                
                En este caso, tenemos que:
                \begin{equation*}
                    H\cap \pi = \left\{(x,y,z)\in \bb{R}^3\mid (b^2-a^2)x^2+b^2y^2-2axc-c^2=0, z=-\frac{c+ax}{b}\right\}
                \end{equation*}

                \begin{itemize}
                    \item \ul{$0<|a|=|b|$, $c\neq 0$}:
                    
                    Tenemos que $b^2-a^2=0$, por lo que $H\cap \pi\equiv b^2y^2-2axc-c^2=0$.
                    Por tanto, se trata de una parábola.

                    \item \ul{$0<|a|\neq |b|$, $c\neq 0$}:
                    
                    En este caso, podemos dividir entre $b^2-a^2$ y completar cuadrados para
                    obtener:
                    \begin{align*}
                        (b^2-a^2)\left[x^2 -2zx \cdot \frac{a}{b^2-a^2}\right] + b^2y^2 - c^2 &= 0\\
                        (b^2-a^2)\left(x-\frac{za}{b^2-a^2}\right)^2 - \frac{z^2a^2}{(b^2-a^2)^2} + b^2y^2 - c^2 &= 0\\
                        (b^2-a^2)\left(x-\frac{za}{b^2-a^2}\right)^2 + b^2y^2 &= c^2 + \frac{z^2a^2}{(b^2-a^2)^2} \\
                    \end{align*}

                    El término independiente de la ecuación siempre es positivo, por lo que distinguimos en función de los valores de $a, b$:
                    \begin{itemize}
                        \item \ul{$0<|a|<|b|$, $c\neq 0$}:
                        
                        Tenemos que $b^2-a^2>0$, por lo que $H\cap \pi$ es una elipse.

                        \item \ul{$0<|b|<|a|$, $c\neq 0$}:
                        
                        Tenemos que $b^2-a^2<0$, por lo que $H\cap \pi$ es una hipérbola.
                    \end{itemize}
                        
                \end{itemize}
            \end{itemize}
        \end{itemize}
    \end{ejercicio}

    \begin{ejercicio}[4 puntos]
        Sea la aplicación afín $f:\bb{R}^3\to \bb{R}^3$ dada por
        \begin{equation*}
            f(x,y,z)=\left(-\frac{4}{5}x + \frac{3}{5}z + 3, y+4,\frac{3}{5}x + \frac{4}{5}z-1\right)
        \end{equation*}
        \begin{enumerate}
            \item Demostrar que es una isometría.
            
            Obtenemos su matriz respecto de la base canónica:
            \begin{equation*}
                M(f,\cc{B}_u) = \frac{1}{5} \left(
                    \begin{array}{c|ccc}
                        5 & 0 & 0 & 0\\ \hline
                        15 & -4 & 0 & 3\\
                        20 & 0 & 5 & 0\\
                        -5 & 3 & 0 & 4
                    \end{array}
                \right)
            \end{equation*}

            Comprobemos que es una isometría. Como $\cc{B}_u$ es una 
            base ortonormal, bastará con probar que las matrices son ortogonales:
            \begin{equation*}
                \frac{1}{5^2}\left(
                    \begin{array}{ccc}
                        -4 & 0 & 3\\
                        0 & 5 & 0\\
                        3 & 0 & 4
                    \end{array}
                \right)\left(
                    \begin{array}{ccc}
                        -4 & 0 & 3\\
                        0 & 5 & 0\\
                        3 & 0 & 4
                    \end{array}
                \right) = Id_3
            \end{equation*}

            Por tanto, $M(f,\cc{B}_u)$ es una matriz ortogonal, y por tanto, $f$ es una isometría.

            \item Clasifícala.
            
            Tenemos que:
            \begin{equation*}
                \left|M\left(\vec{f}, \cc{B}_u\right)\right|
                = \frac{1}{5^3}\cdot 5 \cdot (-16-9) = -1
            \end{equation*}

            Por tanto, como $\left|M\left(\vec{f}, \cc{B}_u\right)\right|=-1$, tenemos que se trata de una isometría inversa.

            \item Calcular el conjunto de puntos fijos.
            
            Tenemos que los puntos fijos cumplen el siguiente sistema:
            \begin{equation*}
                \begin{pmatrix}
                    -9 & 0 & 3\\
                    0 & 0 & 0\\
                    3 & 0 & -1
                \end{pmatrix}
                \begin{pmatrix}
                    x\\y\\z
                \end{pmatrix}
                =
                \begin{pmatrix}
                    -15\\-20\\5
                \end{pmatrix}
            \end{equation*}

            Por tanto, dicho sistema no tiene solución, por lo que:
            \begin{equation*}
                \cc{P}_f = \emptyset
            \end{equation*}

            Por tanto, como es un movimiento inverso en el espacio sin puntos fijos,
            tenemos que se trata de reflexión especular con deslizamiento.

            \item Calcular las ecuaciones que representan a $f$ respecto de los sistemas de referencia $\cc{R}$ (en el dominio) y $\cc{R}'$ (en el codominio); siendo:
            \begin{gather*}
                \cc{R}:=\{(1,-1,0), (0,0,1), (1,0,-1), (2,-2,1)\}\\
                \cc{R}':=\{(0,0,1), (1,-1,0), (0,0,0), (-2,0,1)\}
            \end{gather*}

            Tenemos que $\cc{R}=\{(1,-1,0), \{v_1, v_2, v_3\}\}$, y tenemos que:
            \begin{align*}
                f(1,-1,0) &= \frac{1}{5}(11,15,-2) \\
                \vec{f}(v_1) &= \vec{f(1,-1,0)f(0,0,1)} = \frac{1}{5} \vec{(11,15,-2)~(18,20,-1)} = \frac{1}{5}(7,5,1) \\
                \vec{f}(v_2) &= \vec{f(1,-1,0)f(1,0,-1)} = \frac{1}{5} \vec{(11,15,-2)~(8,20,-6)} = \frac{1}{5}(-3,5,-4) \\
                \vec{f}(v_3) &= \vec{f(1,-1,0)f(2,-2,1)} = \frac{1}{5} \vec{(11,15,-2)~(10,10,5)} = \frac{1}{5}(-1,-5,7)
            \end{align*}

            Por tanto, tenemos que:
            \begin{equation*}
                M(f,\cc{R}, \cc{R}_0) = \frac{1}{5}\left(
                    \begin{array}{c|ccc}
                        5 & 0 & 0 & 0\\ \hline
                        11 & 7 & -3 & -1\\
                        15 & 5 & 5 & -5\\
                        -2 & 1 & -4 & 7
                    \end{array}
                \right)
            \end{equation*}

            Haciendo uso de las matrices de cambio de base, tenemos que:
            \begin{align*}
                M(f, \cc{R}_0) &= M(Id_4, \cc{R}_0, \cc{R}') \cdot M(f, \cc{R}, \cc{R}_0) = \\
               &= M(Id_4, \cc{R}', \cc{R}_0)^{-1} \cdot M(f, \cc{R}, \cc{R}_0) = \\
                &=  \frac{1}{5}\cdot \left(
                    \begin{array}{c|ccc}
                        1 & 0 & 0 & 0\\ \hline
                        0 & 1 & 0 & -2\\
                        0 & -1 & 0 & 0\\
                        1 & -1 & -1 & 0
                    \end{array}
                \right)^{-1}\left(
                    \begin{array}{c|ccc}
                        5 & 0 & 0 & 0\\ \hline
                        11 & 7 & -3 & -1\\
                        15 & 5 & 5 & -5\\
                        -2 & 1 & -4 & 7
                    \end{array}
                \right) =\\
                &= \frac{1}{5}\left(
                    \begin{array}{c|ccc}
                        5 & 0 & 0 & 0\\ \hline
                        -20 & 0 & -5 & 0 \\
                        30 & -3 & 5 & -4 \\
                        \nicefrac{-35}{2} & 2 & \nicefrac{-5}{2} & \nicefrac{-3}{2}
                    \end{array}
                \right)
            \end{align*}
        \end{enumerate}
    \end{ejercicio}

    \begin{ejercicio}[3 puntos]
        En $\bb{R}^2$, para todo $a,b\in \bb{R}^\ast$, demostrar que toda hipérbola $H=\left\{(x,y)\in \bb{R}^2\mid \left(\nicefrac{x}{a}\right)^2-\left(\nicefrac{y}{b}\right)^2=1\right\}\subset \bb{R}^2$ admite dos rectas $R_1,R_2\subset \bb{R}^2$, con $(R_1\cup R_2)\cap H=\emptyset$, verificando que $\forall \veps\in \bb{R}^+$, $\exists p\in H$ tal que:
        \begin{equation*}
            \text{dist}(p,R_1\cup R_2)\leq \veps
        \end{equation*}

        Sean dichas rectas el par de rectas $\left(\nicefrac{x}{a}\right)^2-\left(\nicefrac{y}{b}\right)^2=0$.
        Es decir, sea:
        \begin{equation*}
            R_1\equiv \frac{x}{a} - \frac{y}{b} = 0 \equiv y = \frac{b}{a}x
            \qquad R_2\equiv \frac{x}{a} + \frac{y}{b} = 0 \equiv y = -\frac{b}{a}x
        \end{equation*}

        Veamos en primer lugar que la intersección de $H$ con $R_1$ es vacía:
        \begin{equation*}
            \frac{x^2}{a^2} - \frac{\left(\frac{b^2}{a^2}x^2\right)}{b^2} = 1
            \Longrightarrow
            \frac{x^2}{a^2} - \frac{x^2}{a^2} = 1
            \Longrightarrow
            0 = 1
        \end{equation*}
        Por tanto, $H\cap R_1 = \emptyset$. Análogamente, se ve que $H\cap R_2 = \emptyset$,
        por lo que la intersección de ambas es vacía, es decir, $(R_1\cup R_2)\cap H = \emptyset$.
        Veamos ahora la última parte. Por simetría, consideraremos tan solo los puntos del primer cuadrante, es decir, la recta $R_1$ y el primer cuadrante de $H$.
        Un punto cualquiera de la asíntota en dicho cuadrante cumple que:
        \begin{equation*}
            \frac{x^2}{a^2} - \frac{y^2}{b^2} = 1 \Longrightarrow
            y^2 = b^2\left(\frac{x^2}{a^2}-1\right)
            \Longrightarrow y = |b|\sqrt{\frac{x^2}{a^2}-1}
        \end{equation*}

        Calculamos ahora la distancia entre un punto de la recta y un punto de la hipérbola que tengan la misma coordenada $x$:
        \begin{equation*}
            d(p, p_{R_1}) = \left|\frac{b}{a} x - |b|\sqrt{\frac{x^2}{a^2}-1}\right|
        \end{equation*}

        Tomando límite cuando $x\to \infty$, tenemos que:
        \begin{align*}
            \lim_{x\to \infty} d(p, p_{R_1})
            &= \lim_{x\to \infty} \frac{b}{a} x - |b|\sqrt{\frac{x^2}{a^2}-1}
            = \lim_{x\to \infty} \frac{b}{a} x - |b|\sqrt{\frac{x^2}{a^2}-1}
            \cdot \frac{\frac{b}{a} x + |b|\sqrt{\frac{x^2}{a^2}-1}}{\frac{b}{a} x + |b|\sqrt{\frac{x^2}{a^2}-1}}
            =\\= \lim_{x\to \infty} &\frac{\frac{b^2}{a^2} x^2 - b^2\cdot \left(\frac{x^2}{a^2}-1\right)}{\frac{b}{a} x + |b|\sqrt{\frac{x^2}{a^2}-1}}
            = \lim_{x\to \infty} \frac{b^2}{\frac{b}{a} x + |b|\sqrt{\frac{x^2}{a^2}-1}}=0
        \end{align*}

        Por tanto, $\forall \veps\in \bb{R}^+$, $\exists M \in \bb{R}^+$ tal que si $x>M$, entonces si $p=\left(x, \frac{b}{a}x\right)\in R_1$, $p_{R_1}=\left(x, |b|\sqrt{\frac{x^2}{a^1}-1}\right)\in H$, entonces $d(p, p_{R_1})<\veps$.
        Como $d(p, R_1)=\inf\{d(p, q)\mid q\in R_1\}$, tenemos que $d(p, R_1)\leq d(p, p_{R_1})<\veps$.
    \end{ejercicio}
\end{document}