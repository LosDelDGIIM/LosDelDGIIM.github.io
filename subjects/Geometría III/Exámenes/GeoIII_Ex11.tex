\documentclass[12pt]{article}

% Idioma y codificación
\usepackage[spanish, es-tabla]{babel}       %es-tabla para que se titule "Tabla"
\usepackage[utf8]{inputenc}

% Márgenes
\usepackage[a4paper,top=3cm,bottom=2.5cm,left=3cm,right=3cm]{geometry}

% Comentarios de bloque
\usepackage{verbatim}

% Paquetes de links
\usepackage[hidelinks]{hyperref}    % Permite enlaces
\usepackage{url}                    % redirecciona a la web

% Más opciones para enumeraciones
\usepackage{enumitem}

% Personalizar la portada
\usepackage{titling}

% Paquetes de tablas
\usepackage{multirow}


%------------------------------------------------------------------------

%Paquetes de figuras
\usepackage{caption}
\usepackage{subcaption} % Figuras al lado de otras
\usepackage{float}      % Poner figuras en el sitio indicado H.


% Paquetes de imágenes
\usepackage{graphicx}       % Paquete para añadir imágenes
\usepackage{transparent}    % Para manejar la opacidad de las figuras

% Paquete para usar colores
\usepackage[dvipsnames]{xcolor}
\usepackage{pagecolor}      % Para cambiar el color de la página

% Habilita tamaños de fuente mayores
\usepackage{fix-cm}

% Para los gráficos
\usepackage{tikz}

% Para poder situar los nodos en los grafos
\usetikzlibrary{positioning}


%------------------------------------------------------------------------

% Paquetes de matemáticas
\usepackage{mathtools, amsfonts, amssymb, mathrsfs}
\usepackage[makeroom]{cancel}     % Simplificar tachando
\usepackage{polynom}    % Divisiones y Ruffini
\usepackage{units} % Para poner fracciones diagonales con \nicefrac

\usepackage{pgfplots}   %Representar funciones
\pgfplotsset{compat=1.18}  % Versión 1.18

\usepackage{tikz-cd}    % Para usar diagramas de composiciones
\usetikzlibrary{calc}   % Para usar cálculo de coordenadas en tikz

%Definición de teoremas, etc.
\usepackage{amsthm}
%\swapnumbers   % Intercambia la posición del texto y de la numeración

\theoremstyle{plain}

\makeatletter
\@ifclassloaded{article}{
  \newtheorem{teo}{Teorema}[section]
}{
  \newtheorem{teo}{Teorema}[chapter]  % Se resetea en cada chapter
}
\makeatother

\newtheorem{coro}{Corolario}[teo]           % Se resetea en cada teorema
\newtheorem{prop}[teo]{Proposición}         % Usa el mismo contador que teorema
\newtheorem{lema}[teo]{Lema}                % Usa el mismo contador que teorema

\theoremstyle{remark}
\newtheorem*{observacion}{Observación}

\theoremstyle{definition}

\makeatletter
\@ifclassloaded{article}{
  \newtheorem{definicion}{Definición} [section]     % Se resetea en cada chapter
}{
  \newtheorem{definicion}{Definición} [chapter]     % Se resetea en cada chapter
}
\makeatother

\newtheorem*{notacion}{Notación}
\newtheorem*{ejemplo}{Ejemplo}
\newtheorem*{ejercicio*}{Ejercicio}             % No numerado
\newtheorem{ejercicio}{Ejercicio} [section]     % Se resetea en cada section


% Modificar el formato de la numeración del teorema "ejercicio"
\renewcommand{\theejercicio}{%
  \ifnum\value{section}=0 % Si no se ha iniciado ninguna sección
    \arabic{ejercicio}% Solo mostrar el número de ejercicio
  \else
    \thesection.\arabic{ejercicio}% Mostrar número de sección y número de ejercicio
  \fi
}


% \renewcommand\qedsymbol{$\blacksquare$}         % Cambiar símbolo QED
%------------------------------------------------------------------------

% Paquetes para encabezados
\usepackage{fancyhdr}
\pagestyle{fancy}
\fancyhf{}

\newcommand{\helv}{ % Modificación tamaño de letra
\fontfamily{}\fontsize{12}{12}\selectfont}
\setlength{\headheight}{15pt} % Amplía el tamaño del índice


%\usepackage{lastpage}   % Referenciar última pag   \pageref{LastPage}
\fancyfoot[C]{\thepage}

%------------------------------------------------------------------------

% Conseguir que no ponga "Capítulo 1". Sino solo "1."
\makeatletter
\@ifclassloaded{book}{
  \renewcommand{\chaptermark}[1]{\markboth{\thechapter.\ #1}{}} % En el encabezado
    
  \renewcommand{\@makechapterhead}[1]{%
  \vspace*{50\p@}%
  {\parindent \z@ \raggedright \normalfont
    \ifnum \c@secnumdepth >\m@ne
      \huge\bfseries \thechapter.\hspace{1em}\ignorespaces
    \fi
    \interlinepenalty\@M
    \Huge \bfseries #1\par\nobreak
    \vskip 40\p@
  }}
}
\makeatother

%------------------------------------------------------------------------
% Paquetes de cógido
\usepackage{minted}
\renewcommand\listingscaption{Código fuente}

\usepackage{fancyvrb}
% Personaliza el tamaño de los números de línea
\renewcommand{\theFancyVerbLine}{\small\arabic{FancyVerbLine}}

% Estilo para C++
\newminted{cpp}{
    frame=lines,
    framesep=2mm,
    baselinestretch=1.2,
    linenos,
    escapeinside=||
}

% para minted
\definecolor{LightGray}{rgb}{0.95,0.95,0.92}
\setminted{
    linenos=true,
    stepnumber=5,
    numberfirstline=true,
    autogobble,
    breaklines=true,
    breakautoindent=true,
    breaksymbolleft=,
    breaksymbolright=,
    breaksymbolindentleft=0pt,
    breaksymbolindentright=0pt,
    breaksymbolsepleft=0pt,
    breaksymbolsepright=0pt,
    fontsize=\footnotesize,
    bgcolor=LightGray,
    numbersep=10pt
}


\usepackage{listings} % Para incluir código desde un archivo

\renewcommand\lstlistingname{Código Fuente}
\renewcommand\lstlistlistingname{Índice de Códigos Fuente}

% Definir colores
\definecolor{vscodepurple}{rgb}{0.5,0,0.5}
\definecolor{vscodeblue}{rgb}{0,0,0.8}
\definecolor{vscodegreen}{rgb}{0,0.5,0}
\definecolor{vscodegray}{rgb}{0.5,0.5,0.5}
\definecolor{vscodebackground}{rgb}{0.97,0.97,0.97}
\definecolor{vscodelightgray}{rgb}{0.9,0.9,0.9}

% Configuración para el estilo de C similar a VSCode
\lstdefinestyle{vscode_C}{
  backgroundcolor=\color{vscodebackground},
  commentstyle=\color{vscodegreen},
  keywordstyle=\color{vscodeblue},
  numberstyle=\tiny\color{vscodegray},
  stringstyle=\color{vscodepurple},
  basicstyle=\scriptsize\ttfamily,
  breakatwhitespace=false,
  breaklines=true,
  captionpos=b,
  keepspaces=true,
  numbers=left,
  numbersep=5pt,
  showspaces=false,
  showstringspaces=false,
  showtabs=false,
  tabsize=2,
  frame=tb,
  framerule=0pt,
  aboveskip=10pt,
  belowskip=10pt,
  xleftmargin=10pt,
  xrightmargin=10pt,
  framexleftmargin=10pt,
  framexrightmargin=10pt,
  framesep=0pt,
  rulecolor=\color{vscodelightgray},
  backgroundcolor=\color{vscodebackground},
}

%------------------------------------------------------------------------

% Comandos definidos
\newcommand{\bb}[1]{\mathbb{#1}}
\newcommand{\cc}[1]{\mathcal{#1}}

% I prefer the slanted \leq
\let\oldleq\leq % save them in case they're every wanted
\let\oldgeq\geq
\renewcommand{\leq}{\leqslant}
\renewcommand{\geq}{\geqslant}

% Si y solo si
\newcommand{\sii}{\iff}

% Letras griegas
\newcommand{\eps}{\epsilon}
\newcommand{\veps}{\varepsilon}
\newcommand{\lm}{\lambda}

\newcommand{\ol}{\overline}
\newcommand{\ul}{\underline}
\newcommand{\wt}{\widetilde}
\newcommand{\wh}{\widehat}

\let\oldvec\vec
\renewcommand{\vec}{\overrightarrow}

% Derivadas parciales
\newcommand{\del}[2]{\frac{\partial #1}{\partial #2}}
\newcommand{\Del}[3]{\frac{\partial^{#1} #2}{\partial #3^{#1}}}
\newcommand{\deld}[2]{\dfrac{\partial #1}{\partial #2}}
\newcommand{\Deld}[3]{\dfrac{\partial^{#1} #2}{\partial #3^{#1}}}


\newcommand{\AstIg}{\stackrel{(\ast)}{=}}
\newcommand{\Hop}{\stackrel{L'H\hat{o}pital}{=}}

\newcommand{\red}[1]{{\color{red}#1}} % Para integrales, destacar los cambios.

% Método de integración
\newcommand{\MetInt}[2]{
    \left[\begin{array}{c}
        #1 \\ #2
    \end{array}\right]
}

% Declarar aplicaciones
% 1. Nombre aplicación
% 2. Dominio
% 3. Codominio
% 4. Variable
% 5. Imagen de la variable
\newcommand{\Func}[5]{
    \begin{equation*}
        \begin{array}{rrll}
            #1:& #2 & \longrightarrow & #3\\
               & #4 & \longmapsto & #5
        \end{array}
    \end{equation*}
}

%------------------------------------------------------------------------


\usepackage{pgfplots}
\pgfplotsset{compat=1.15}
\usepackage{mathrsfs}
\usetikzlibrary{arrows}



\begin{document}

    % 1. Foto de fondo
    % 2. Título
    % 3. Encabezado Izquierdo
    % 4. Color de fondo
    % 5. Coord x del titulo
    % 6. Coord y del titulo
    % 7. Fecha

    
    % 1. Foto de fondo
% 2. Título
% 3. Encabezado Izquierdo
% 4. Color de fondo
% 5. Coord x del titulo
% 6. Coord y del titulo
% 7. Fecha

\newcommand{\portada}[7]{

    \portadaBase{#1}{#2}{#3}{#4}{#5}{#6}{#7}
    \portadaBook{#1}{#2}{#3}{#4}{#5}{#6}{#7}
}

\newcommand{\portadaExamen}[7]{

    \portadaBase{#1}{#2}{#3}{#4}{#5}{#6}{#7}
    \portadaArticle{#1}{#2}{#3}{#4}{#5}{#6}{#7}
}




\newcommand{\portadaBase}[7]{

    % Tiene la portada principal y la licencia Creative Commons
    
    % 1. Foto de fondo
    % 2. Título
    % 3. Encabezado Izquierdo
    % 4. Color de fondo
    % 5. Coord x del titulo
    % 6. Coord y del titulo
    % 7. Fecha
    
    
    \thispagestyle{empty}               % Sin encabezado ni pie de página
    \newgeometry{margin=0cm}        % Márgenes nulos para la primera página
    
    
    % Encabezado
    \fancyhead[L]{\helv #3}
    \fancyhead[R]{\helv \nouppercase{\leftmark}}
    
    
    \pagecolor{#4}        % Color de fondo para la portada
    
    \begin{figure}[p]
        \centering
        \transparent{0.3}           % Opacidad del 30% para la imagen
        
        \includegraphics[width=\paperwidth, keepaspectratio]{assets/#1}
    
        \begin{tikzpicture}[remember picture, overlay]
            \node[anchor=north west, text=white, opacity=1, font=\fontsize{60}{90}\selectfont\bfseries\sffamily, align=left] at (#5, #6) {#2};
            
            \node[anchor=south east, text=white, opacity=1, font=\fontsize{12}{18}\selectfont\sffamily, align=right] at (9.7, 3) {\textbf{\href{https://losdeldgiim.github.io/}{Los Del DGIIM}}};
            
            \node[anchor=south east, text=white, opacity=1, font=\fontsize{12}{15}\selectfont\sffamily, align=right] at (9.7, 1.8) {Doble Grado en Ingeniería Informática y Matemáticas\\Universidad de Granada};
        \end{tikzpicture}
    \end{figure}
    
    
    \restoregeometry        % Restaurar márgenes normales para las páginas subsiguientes
    \pagecolor{white}       % Restaurar el color de página
    
    
    \newpage
    \thispagestyle{empty}               % Sin encabezado ni pie de página
    \begin{tikzpicture}[remember picture, overlay]
        \node[anchor=south west, inner sep=3cm] at (current page.south west) {
            \begin{minipage}{0.5\paperwidth}
                \href{https://creativecommons.org/licenses/by-nc-nd/4.0/}{
                    \includegraphics[height=2cm]{assets/Licencia.png}
                }\vspace{1cm}\\
                Esta obra está bajo una
                \href{https://creativecommons.org/licenses/by-nc-nd/4.0/}{
                    Licencia Creative Commons Atribución-NoComercial-SinDerivadas 4.0 Internacional (CC BY-NC-ND 4.0).
                }\\
    
                Eres libre de compartir y redistribuir el contenido de esta obra en cualquier medio o formato, siempre y cuando des el crédito adecuado a los autores originales y no persigas fines comerciales. 
            \end{minipage}
        };
    \end{tikzpicture}
    
    
    
    % 1. Foto de fondo
    % 2. Título
    % 3. Encabezado Izquierdo
    % 4. Color de fondo
    % 5. Coord x del titulo
    % 6. Coord y del titulo
    % 7. Fecha


}


\newcommand{\portadaBook}[7]{

    % 1. Foto de fondo
    % 2. Título
    % 3. Encabezado Izquierdo
    % 4. Color de fondo
    % 5. Coord x del titulo
    % 6. Coord y del titulo
    % 7. Fecha

    % Personaliza el formato del título
    \pretitle{\begin{center}\bfseries\fontsize{42}{56}\selectfont}
    \posttitle{\par\end{center}\vspace{2em}}
    
    % Personaliza el formato del autor
    \preauthor{\begin{center}\Large}
    \postauthor{\par\end{center}\vfill}
    
    % Personaliza el formato de la fecha
    \predate{\begin{center}\huge}
    \postdate{\par\end{center}\vspace{2em}}
    
    \title{#2}
    \author{\href{https://losdeldgiim.github.io/}{Los Del DGIIM}}
    \date{Granada, #7}
    \maketitle
    
    \tableofcontents
}




\newcommand{\portadaArticle}[7]{

    % 1. Foto de fondo
    % 2. Título
    % 3. Encabezado Izquierdo
    % 4. Color de fondo
    % 5. Coord x del titulo
    % 6. Coord y del titulo
    % 7. Fecha

    % Personaliza el formato del título
    \pretitle{\begin{center}\bfseries\fontsize{42}{56}\selectfont}
    \posttitle{\par\end{center}\vspace{2em}}
    
    % Personaliza el formato del autor
    \preauthor{\begin{center}\Large}
    \postauthor{\par\end{center}\vspace{3em}}
    
    % Personaliza el formato de la fecha
    \predate{\begin{center}\huge}
    \postdate{\par\end{center}\vspace{5em}}
    
    \title{#2}
    \author{\href{https://losdeldgiim.github.io/}{Los Del DGIIM}}
    \date{Granada, #7}
    \thispagestyle{empty}               % Sin encabezado ni pie de página
    \maketitle
    \vfill
}
    \portadaExamen{ffccA4.jpg}{Geometría III\\Examen XI}{Geometría III. Examen XI}{MidnightBlue}{-8}{28}{2023-2024}{Jesús Muñoz Velasco\\Arturo Olivares Martos}

    
    \begin{description}
        \item[Asignatura] Geometría III.
        \item[Curso Académico] 2021-2022.
        %\item[Grado] Doble Grado en Ingeniería Informática y Matemáticas.
        %\item[Grupo] Único.
        %\item[Profesor] %Antonio Martínez López.
        \item[Descripción] Prueba de Clase. Temas 2 y 3.
        \item[Fecha] 20 de diciembre de 2021.
        %\item[Duración] 1 hora.
    
    \end{description}
    \newpage

    \begin{ejercicio}[4 puntos]
        Clasifica el siguiente movimiento rígido de $\bb{R}^3$:
        \begin{gather*}
            f(x,y,z)=(1-z,y,1-x)
        \end{gather*}
        y calcula sus elementos notables (o geométricos).\\

        Podemos escribir $f$ matricialmente como
        \begin{gather*}
            f(x,y,z)= 
            \begin{pmatrix}
                1 \\ 0 \\ 1
            \end{pmatrix} +
            \begin{pmatrix}
                0&0&-1\\
                0&1&0\\
                -1&0&0
            \end{pmatrix}
            \begin{pmatrix}
                x\\y\\z
            \end{pmatrix}\\\\
            \text{Como }\  \begin{vmatrix}
                0&0&-1\\
                0&1&0\\
                -1&0&0
            \end{vmatrix}=-1 \Rightarrow \text{ Movimiento inverso}
        \end{gather*}

        Veamos si $f$ tiene puntos fijos:
        \begin{gather*}
            \left.
            \begin{array}{l}
                x=1-z\\
                y=y\\
                z=1-x
            \end{array}
            \right\} \Rightarrow \left.
            \begin{array}{l}
                x=\lambda\\
                y=\mu\\
                z=1-\lambda
            \end{array}
            \right\} \Rightarrow \cc{P}_f=\{(x,y,z)\in \bb{R}^3: z=1-x\}=\Pi
        \end{gather*}

        Por tanto tenemos que $f$ es un movimiento inverso con un plano de puntos fijos. 
        Entonces $f$ es una reflexión especular respecto al plano $\Pi$.

    \end{ejercicio}

    \begin{ejercicio}[4 puntos]
        Explica razonadamente si es posible obtener lo siguiente, teniendo en cuenta
        la siguiente observación:
        \begin{observacion}
            Recordemos que una simetría axial es una simetría ortogonal respecto de una recta.
        \end{observacion}
        

        \begin{enumerate}
            \item (2 puntos) una traslación en $\bb{R}^2$ como composición de dos simetrías axiales.

            Sí, basta tomar dos rectas paralelas que sean perpendiculares al vector de traslación y la distancia entre ellas sea igual a la mitad del módulo del vector de traslación.
            Veámolo:

            Sea $v=(v_1,v_2)$ el vector de traslación. Entonces podemos tomar $\cc{S}_1$ y $\cc{S}_2$, dos rectas de $\bb{R}^2$ como
            \begin{gather*}
                \begin{array}{rl}
                    \cc{S}_1&=(p_1,p_2) + \cc{L}\{v^\perp\}\\
                    \cc{S}_2&=(q_1,q_2) + \cc{L}\{v^\perp\}\\
                \end{array}\qquad
                \text{ donde } d(\cc{S}_1,\cc{S}_2)=\frac{\|v\|}{2}
            \end{gather*}
            Podemos considerar el sistema de referencia $\cc{R}=\{p=(p_1,p_2),\{v, v^\perp \}\}$. 
            En este sistema nos queda:
            \begin{gather*}
                {M}(t_v;\cc{R})=\left(\begin{array}{c|cc}
                    1 & 0 & 0\\
                    \hline
                    \|v\| & 1 & 0\\
                    0 & 0 & 1
                \end{array}\right)=A\\
                \\
                M(\sigma_{\cc{S}_1}; \cc{R})=\left(\begin{array}{c|cc}
                    1 & 0 & 0\\
                    \hline
                    0 & -1 & 0\\
                    0 & 0 & 1
                \end{array}\right)=B \hspace{2cm}
                M(\sigma_{\cc{S}_2}; \cc{R})=\left(\begin{array}{c|cc}
                    1 & 0 & 0\\
                    \hline
                    a_1 & -1 & 0\\
                    a_2 & 0 & 1
                \end{array}\right)=C
            \end{gather*}
            Por tanto, Como $C\cdot B= A$, tenemos que:
            \begin{gather*}
                C\cdot B = \left(\begin{array}{c|cc}
                    1 & 0 & 0\\
                    \hline
                    a_1 & 1 & 0\\
                    a_2 & 0 & 1
                \end{array}\right) = A \Leftrightarrow 
                \begin{array}{c}
                    a_1=\|v\| \\ 
                    a_2=0
                \end{array}
            \end{gather*}
            Sabemos que $(a_1,a_2)=\sigma_{\cc{S}_2}(p_1,p_2)_{\cc{R}_0}=\sigma_{\cc{S}_2}(0,0)_\cc{R}=(2d(\cc{S}_1,\cc{S}_2),0)$ como se puede ver en el siguiente gráfico (que está en el sistema de referencia $\cc{R}$):
            \begin{figure}[H]
                \centering
                \begin{tikzpicture}[line cap=round,line join=round,x=1cm,y=1cm]
                    \begin{axis}[
                        x=0.7cm,y=0.7cm,
                        axis lines=middle,
                        xmin=-2,
                        xmax=6,
                        ymin=-3,
                        ymax=4,
                        xtick={-10,-9,...,13},
                        ytick={-4,-3,...,5},
                        xticklabels={}, % quitar numeración del eje x
                        yticklabels={},  % quitar numeración del eje y
                        xlabel={$\cc{L}\{v\}$}, % etiqueta del eje x
                        ylabel={$\cc{L}\{v^\perp\}$}, % etiqueta del eje y
                    ] 
                        \clip(-10.29002028804219,-4.659091377730705) rectangle (13.952614239917073,5.151809059689284);
                        \draw [ultra thick,color=red] (0,-4.659091377730705) -- (0,5.151809059689284);
                        \draw [ultra thick,color=blue] (2,-4.659091377730705) -- (2,5.151809059689284);
                        \begin{scriptsize}
                            \draw[color=red] (0.16824331817228005,2.012291135380127) node {$\ \ \ S_1$};
                            \draw[color=blue] (2.5094270135505695,2.012291135380127) node {$S_2$};
                            \draw [fill=teal] (0,0) circle (2.5pt);
                            \draw[color=teal] (0.19011328055915167,-0.46519740703455727) node {$\ \ p$};
                            \draw [fill=teal] (4,0) circle (2.5pt);
                            \draw[color=teal] (4.0859066327734315,-0.46519740703455727) node {$\sigma_{S_2}(p)$};
                        \end{scriptsize}
                    \end{axis}
                \end{tikzpicture}
            \end{figure}
            Además, hemos definido $d(\cc{S}_1,\cc{S}_2)=\frac{\|v\|}{2}$ por lo que $(a_1,a_2)=\left(2\cdot \frac{\|v\|}{2},0\right) = (\|v\|,0)$ y se verifica
            \begin{gather*}
                M(\sigma_{\cc{S}_2}; \cc{R}) \cdot M(\sigma_{\cc{S}_1}; \cc{R}) = M(t_v; \cc{R}) \Rightarrow t_v=\sigma_{\cc{S}_2} \circ \sigma_{\cc{S}_1}
            \end{gather*}

            \item (2 puntos) una simetría axial de $\bb{R}^2$ como composición de dos traslaciones.

            Esto no va a ser posible ya que la traslación es un movimiento directo, y por tanto su composición también lo será. 
            Sin embargo, la simetría axial es inverso y por tanto no se puede conseguir.
        \end{enumerate}
        
    \end{ejercicio}

    \begin{ejercicio}[2 puntos]
        Si sabemos que $f$ es la simetría respecto de un plano $\Pi$ de $\bb{R}^3$ y $f(1,2,3)=(3,4,5)$, calcula una ecuación implícita del plano $\Pi$.\\

        Tomo en primer lugar el vector $\vec{(1,2,3),(3,4,5)}=(2,2,2)$ que es perpendicular al plano.
        Puedo entonces encontrar dos vectores perpendiculares a este que me darán el espacio de direcciones de $\Pi$.
        Por ejemplo, $(1,-1,0)$ y $(1,0,-1)$. Como son linealmente independientes entre sí nos bastan para ser generadores de $\Pi$.
        Nos queda encontrar un punto que pertenezca al plano. Este será el punto medio entre $(1,2,3)$ y $(3,4,5)$,
        \begin{gather*}
            m_{(1,2,3),(3,4,5)}=\dfrac{(1,2,3)+(3,4,5)}{2} = \dfrac{(4,6,8)}{2} = (2,3,4) % Mejorar notación
        \end{gather*}
        Nos queda entonces $\Pi=(2,3,4)+\cc{L}\{(1,-1,0),(1,0,-1)\}$. Busquemos una ecuación implícita del plano.\\
        \begin{gather*}
            (x,y,z)\in \Pi \Leftrightarrow (x,y,z)=(2,3,4)+\lambda(1,-1,0)+\mu (1,0,-1)\ \ \ \ \lambda,\mu \in \bb{R}\\\\
            \left.\begin{array}{r c l}
                x=2+\lambda + \mu&&\\
                y=3-\lambda &\Rightarrow & \lambda = 3-y\\
                z=4-\mu &\Rightarrow & \mu = 4-z\\
            \end{array}\right\} x = 9 -y-z \Rightarrow x+y+z=9
        \end{gather*}

        Por tanto, $\Pi \equiv x+y+z=9$
    \end{ejercicio}
     
\end{document}
