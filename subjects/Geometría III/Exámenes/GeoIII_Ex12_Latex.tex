\documentclass[12pt]{article}

% Idioma y codificación
\usepackage[spanish, es-tabla]{babel}       %es-tabla para que se titule "Tabla"
\usepackage[utf8]{inputenc}

% Márgenes
\usepackage[a4paper,top=3cm,bottom=2.5cm,left=3cm,right=3cm]{geometry}

% Comentarios de bloque
\usepackage{verbatim}

% Paquetes de links
\usepackage[hidelinks]{hyperref}    % Permite enlaces
\usepackage{url}                    % redirecciona a la web

% Más opciones para enumeraciones
\usepackage{enumitem}

% Personalizar la portada
\usepackage{titling}

% Paquetes de tablas
\usepackage{multirow}


%------------------------------------------------------------------------

%Paquetes de figuras
\usepackage{caption}
\usepackage{subcaption} % Figuras al lado de otras
\usepackage{float}      % Poner figuras en el sitio indicado H.


% Paquetes de imágenes
\usepackage{graphicx}       % Paquete para añadir imágenes
\usepackage{transparent}    % Para manejar la opacidad de las figuras

% Paquete para usar colores
\usepackage[dvipsnames]{xcolor}
\usepackage{pagecolor}      % Para cambiar el color de la página

% Habilita tamaños de fuente mayores
\usepackage{fix-cm}

% Para los gráficos
\usepackage{tikz}

% Para poder situar los nodos en los grafos
\usetikzlibrary{positioning}


%------------------------------------------------------------------------

% Paquetes de matemáticas
\usepackage{mathtools, amsfonts, amssymb, mathrsfs}
\usepackage[makeroom]{cancel}     % Simplificar tachando
\usepackage{polynom}    % Divisiones y Ruffini
\usepackage{units} % Para poner fracciones diagonales con \nicefrac

\usepackage{pgfplots}   %Representar funciones
\pgfplotsset{compat=1.18}  % Versión 1.18

\usepackage{tikz-cd}    % Para usar diagramas de composiciones
\usetikzlibrary{calc}   % Para usar cálculo de coordenadas en tikz

%Definición de teoremas, etc.
\usepackage{amsthm}
%\swapnumbers   % Intercambia la posición del texto y de la numeración

\theoremstyle{plain}

\makeatletter
\@ifclassloaded{article}{
  \newtheorem{teo}{Teorema}[section]
}{
  \newtheorem{teo}{Teorema}[chapter]  % Se resetea en cada chapter
}
\makeatother

\newtheorem{coro}{Corolario}[teo]           % Se resetea en cada teorema
\newtheorem{prop}[teo]{Proposición}         % Usa el mismo contador que teorema
\newtheorem{lema}[teo]{Lema}                % Usa el mismo contador que teorema

\theoremstyle{remark}
\newtheorem*{observacion}{Observación}

\theoremstyle{definition}

\makeatletter
\@ifclassloaded{article}{
  \newtheorem{definicion}{Definición} [section]     % Se resetea en cada chapter
}{
  \newtheorem{definicion}{Definición} [chapter]     % Se resetea en cada chapter
}
\makeatother

\newtheorem*{notacion}{Notación}
\newtheorem*{ejemplo}{Ejemplo}
\newtheorem*{ejercicio*}{Ejercicio}             % No numerado
\newtheorem{ejercicio}{Ejercicio} [section]     % Se resetea en cada section


% Modificar el formato de la numeración del teorema "ejercicio"
\renewcommand{\theejercicio}{%
  \ifnum\value{section}=0 % Si no se ha iniciado ninguna sección
    \arabic{ejercicio}% Solo mostrar el número de ejercicio
  \else
    \thesection.\arabic{ejercicio}% Mostrar número de sección y número de ejercicio
  \fi
}


% \renewcommand\qedsymbol{$\blacksquare$}         % Cambiar símbolo QED
%------------------------------------------------------------------------

% Paquetes para encabezados
\usepackage{fancyhdr}
\pagestyle{fancy}
\fancyhf{}

\newcommand{\helv}{ % Modificación tamaño de letra
\fontfamily{}\fontsize{12}{12}\selectfont}
\setlength{\headheight}{15pt} % Amplía el tamaño del índice


%\usepackage{lastpage}   % Referenciar última pag   \pageref{LastPage}
\fancyfoot[C]{\thepage}

%------------------------------------------------------------------------

% Conseguir que no ponga "Capítulo 1". Sino solo "1."
\makeatletter
\@ifclassloaded{book}{
  \renewcommand{\chaptermark}[1]{\markboth{\thechapter.\ #1}{}} % En el encabezado
    
  \renewcommand{\@makechapterhead}[1]{%
  \vspace*{50\p@}%
  {\parindent \z@ \raggedright \normalfont
    \ifnum \c@secnumdepth >\m@ne
      \huge\bfseries \thechapter.\hspace{1em}\ignorespaces
    \fi
    \interlinepenalty\@M
    \Huge \bfseries #1\par\nobreak
    \vskip 40\p@
  }}
}
\makeatother

%------------------------------------------------------------------------
% Paquetes de cógido
\usepackage{minted}
\renewcommand\listingscaption{Código fuente}

\usepackage{fancyvrb}
% Personaliza el tamaño de los números de línea
\renewcommand{\theFancyVerbLine}{\small\arabic{FancyVerbLine}}

% Estilo para C++
\newminted{cpp}{
    frame=lines,
    framesep=2mm,
    baselinestretch=1.2,
    linenos,
    escapeinside=||
}

% para minted
\definecolor{LightGray}{rgb}{0.95,0.95,0.92}
\setminted{
    linenos=true,
    stepnumber=5,
    numberfirstline=true,
    autogobble,
    breaklines=true,
    breakautoindent=true,
    breaksymbolleft=,
    breaksymbolright=,
    breaksymbolindentleft=0pt,
    breaksymbolindentright=0pt,
    breaksymbolsepleft=0pt,
    breaksymbolsepright=0pt,
    fontsize=\footnotesize,
    bgcolor=LightGray,
    numbersep=10pt
}


\usepackage{listings} % Para incluir código desde un archivo

\renewcommand\lstlistingname{Código Fuente}
\renewcommand\lstlistlistingname{Índice de Códigos Fuente}

% Definir colores
\definecolor{vscodepurple}{rgb}{0.5,0,0.5}
\definecolor{vscodeblue}{rgb}{0,0,0.8}
\definecolor{vscodegreen}{rgb}{0,0.5,0}
\definecolor{vscodegray}{rgb}{0.5,0.5,0.5}
\definecolor{vscodebackground}{rgb}{0.97,0.97,0.97}
\definecolor{vscodelightgray}{rgb}{0.9,0.9,0.9}

% Configuración para el estilo de C similar a VSCode
\lstdefinestyle{vscode_C}{
  backgroundcolor=\color{vscodebackground},
  commentstyle=\color{vscodegreen},
  keywordstyle=\color{vscodeblue},
  numberstyle=\tiny\color{vscodegray},
  stringstyle=\color{vscodepurple},
  basicstyle=\scriptsize\ttfamily,
  breakatwhitespace=false,
  breaklines=true,
  captionpos=b,
  keepspaces=true,
  numbers=left,
  numbersep=5pt,
  showspaces=false,
  showstringspaces=false,
  showtabs=false,
  tabsize=2,
  frame=tb,
  framerule=0pt,
  aboveskip=10pt,
  belowskip=10pt,
  xleftmargin=10pt,
  xrightmargin=10pt,
  framexleftmargin=10pt,
  framexrightmargin=10pt,
  framesep=0pt,
  rulecolor=\color{vscodelightgray},
  backgroundcolor=\color{vscodebackground},
}

%------------------------------------------------------------------------

% Comandos definidos
\newcommand{\bb}[1]{\mathbb{#1}}
\newcommand{\cc}[1]{\mathcal{#1}}

% I prefer the slanted \leq
\let\oldleq\leq % save them in case they're every wanted
\let\oldgeq\geq
\renewcommand{\leq}{\leqslant}
\renewcommand{\geq}{\geqslant}

% Si y solo si
\newcommand{\sii}{\iff}

% Letras griegas
\newcommand{\eps}{\epsilon}
\newcommand{\veps}{\varepsilon}
\newcommand{\lm}{\lambda}

\newcommand{\ol}{\overline}
\newcommand{\ul}{\underline}
\newcommand{\wt}{\widetilde}
\newcommand{\wh}{\widehat}

\let\oldvec\vec
\renewcommand{\vec}{\overrightarrow}

% Derivadas parciales
\newcommand{\del}[2]{\frac{\partial #1}{\partial #2}}
\newcommand{\Del}[3]{\frac{\partial^{#1} #2}{\partial #3^{#1}}}
\newcommand{\deld}[2]{\dfrac{\partial #1}{\partial #2}}
\newcommand{\Deld}[3]{\dfrac{\partial^{#1} #2}{\partial #3^{#1}}}


\newcommand{\AstIg}{\stackrel{(\ast)}{=}}
\newcommand{\Hop}{\stackrel{L'H\hat{o}pital}{=}}

\newcommand{\red}[1]{{\color{red}#1}} % Para integrales, destacar los cambios.

% Método de integración
\newcommand{\MetInt}[2]{
    \left[\begin{array}{c}
        #1 \\ #2
    \end{array}\right]
}

% Declarar aplicaciones
% 1. Nombre aplicación
% 2. Dominio
% 3. Codominio
% 4. Variable
% 5. Imagen de la variable
\newcommand{\Func}[5]{
    \begin{equation*}
        \begin{array}{rrll}
            #1:& #2 & \longrightarrow & #3\\
               & #4 & \longmapsto & #5
        \end{array}
    \end{equation*}
}

%------------------------------------------------------------------------


\usepackage{pgfplots}
\pgfplotsset{compat=1.15}
\usepackage{mathrsfs}
\usetikzlibrary{arrows}



\begin{document}

    % 1. Foto de fondo
    % 2. Título
    % 3. Encabezado Izquierdo
    % 4. Color de fondo
    % 5. Coord x del titulo
    % 6. Coord y del titulo
    % 7. Fecha

    
    % 1. Foto de fondo
% 2. Título
% 3. Encabezado Izquierdo
% 4. Color de fondo
% 5. Coord x del titulo
% 6. Coord y del titulo
% 7. Fecha

\newcommand{\portada}[7]{

    \portadaBase{#1}{#2}{#3}{#4}{#5}{#6}{#7}
    \portadaBook{#1}{#2}{#3}{#4}{#5}{#6}{#7}
}

\newcommand{\portadaExamen}[7]{

    \portadaBase{#1}{#2}{#3}{#4}{#5}{#6}{#7}
    \portadaArticle{#1}{#2}{#3}{#4}{#5}{#6}{#7}
}




\newcommand{\portadaBase}[7]{

    % Tiene la portada principal y la licencia Creative Commons
    
    % 1. Foto de fondo
    % 2. Título
    % 3. Encabezado Izquierdo
    % 4. Color de fondo
    % 5. Coord x del titulo
    % 6. Coord y del titulo
    % 7. Fecha
    
    
    \thispagestyle{empty}               % Sin encabezado ni pie de página
    \newgeometry{margin=0cm}        % Márgenes nulos para la primera página
    
    
    % Encabezado
    \fancyhead[L]{\helv #3}
    \fancyhead[R]{\helv \nouppercase{\leftmark}}
    
    
    \pagecolor{#4}        % Color de fondo para la portada
    
    \begin{figure}[p]
        \centering
        \transparent{0.3}           % Opacidad del 30% para la imagen
        
        \includegraphics[width=\paperwidth, keepaspectratio]{assets/#1}
    
        \begin{tikzpicture}[remember picture, overlay]
            \node[anchor=north west, text=white, opacity=1, font=\fontsize{60}{90}\selectfont\bfseries\sffamily, align=left] at (#5, #6) {#2};
            
            \node[anchor=south east, text=white, opacity=1, font=\fontsize{12}{18}\selectfont\sffamily, align=right] at (9.7, 3) {\textbf{\href{https://losdeldgiim.github.io/}{Los Del DGIIM}}};
            
            \node[anchor=south east, text=white, opacity=1, font=\fontsize{12}{15}\selectfont\sffamily, align=right] at (9.7, 1.8) {Doble Grado en Ingeniería Informática y Matemáticas\\Universidad de Granada};
        \end{tikzpicture}
    \end{figure}
    
    
    \restoregeometry        % Restaurar márgenes normales para las páginas subsiguientes
    \pagecolor{white}       % Restaurar el color de página
    
    
    \newpage
    \thispagestyle{empty}               % Sin encabezado ni pie de página
    \begin{tikzpicture}[remember picture, overlay]
        \node[anchor=south west, inner sep=3cm] at (current page.south west) {
            \begin{minipage}{0.5\paperwidth}
                \href{https://creativecommons.org/licenses/by-nc-nd/4.0/}{
                    \includegraphics[height=2cm]{assets/Licencia.png}
                }\vspace{1cm}\\
                Esta obra está bajo una
                \href{https://creativecommons.org/licenses/by-nc-nd/4.0/}{
                    Licencia Creative Commons Atribución-NoComercial-SinDerivadas 4.0 Internacional (CC BY-NC-ND 4.0).
                }\\
    
                Eres libre de compartir y redistribuir el contenido de esta obra en cualquier medio o formato, siempre y cuando des el crédito adecuado a los autores originales y no persigas fines comerciales. 
            \end{minipage}
        };
    \end{tikzpicture}
    
    
    
    % 1. Foto de fondo
    % 2. Título
    % 3. Encabezado Izquierdo
    % 4. Color de fondo
    % 5. Coord x del titulo
    % 6. Coord y del titulo
    % 7. Fecha


}


\newcommand{\portadaBook}[7]{

    % 1. Foto de fondo
    % 2. Título
    % 3. Encabezado Izquierdo
    % 4. Color de fondo
    % 5. Coord x del titulo
    % 6. Coord y del titulo
    % 7. Fecha

    % Personaliza el formato del título
    \pretitle{\begin{center}\bfseries\fontsize{42}{56}\selectfont}
    \posttitle{\par\end{center}\vspace{2em}}
    
    % Personaliza el formato del autor
    \preauthor{\begin{center}\Large}
    \postauthor{\par\end{center}\vfill}
    
    % Personaliza el formato de la fecha
    \predate{\begin{center}\huge}
    \postdate{\par\end{center}\vspace{2em}}
    
    \title{#2}
    \author{\href{https://losdeldgiim.github.io/}{Los Del DGIIM}}
    \date{Granada, #7}
    \maketitle
    
    \tableofcontents
}




\newcommand{\portadaArticle}[7]{

    % 1. Foto de fondo
    % 2. Título
    % 3. Encabezado Izquierdo
    % 4. Color de fondo
    % 5. Coord x del titulo
    % 6. Coord y del titulo
    % 7. Fecha

    % Personaliza el formato del título
    \pretitle{\begin{center}\bfseries\fontsize{42}{56}\selectfont}
    \posttitle{\par\end{center}\vspace{2em}}
    
    % Personaliza el formato del autor
    \preauthor{\begin{center}\Large}
    \postauthor{\par\end{center}\vspace{3em}}
    
    % Personaliza el formato de la fecha
    \predate{\begin{center}\huge}
    \postdate{\par\end{center}\vspace{5em}}
    
    \title{#2}
    \author{\href{https://losdeldgiim.github.io/}{Los Del DGIIM}}
    \date{Granada, #7}
    \thispagestyle{empty}               % Sin encabezado ni pie de página
    \maketitle
    \vfill
}
    \portadaExamen{ffccA4.jpg}{Geometría III\\Examen XII}{Geometría III. Examen XII}{MidnightBlue}{-8}{28}{2023-2024}{Jesús Muñoz Velasco\\Arturo Olivares Martos}

    
    \begin{description}
        \item[Asignatura] Geometría III.
        \item[Curso Académico] 2023-24.
        \item[Grado] Grado en Matemáticas.
        \item[Grupo] A.
        \item[Profesor] María Magdalena Rodríguez Pérez.
        \item[Descripción] Prueba de clase (Tema 1).
        \item[Fecha] 14 de noviembre de 2023.
        %\item[Duración] 1 hora.
    
    \end{description}
    \newpage

    \begin{ejercicio}[3 puntos]
        Encuentra un sistema de referencia $\cc{R}$ de $\bb{R}^2$ en el que los puntos $(1,2)$ y $(3,4)$ tengan coordenadas, respectivamente, $(0,0)$ y $(0,1)$. Calcula el cambio de coordenadas de $\cc{R}$ al sistema de referencia usual.\\

        Como $\cc{R}$ es un sistema de referencia de $\bb{R}^2$, será de la forma
        \begin{gather*}
            \cc{R}=\{p\ ,\ \cc{B}=\{e_1,e_2\}\}\ \ p \in \bb{R}^2,\ \ e_1,e_2 \in \vec{\bb{R}^2}
        \end{gather*}
        Sabemos que $p=(1,2)$, ya que sus coordenadas son el $(0,0)_\cc{R}$. Además, el segundo vector de la base de $\cc{R}$ es el $\vec{(1,2)(3,4)}=(2,2)$, ya que $(3,4)=(0,1)_\cc{R}$, y por tanto $(3,4)=(0,0)_\cc{R} + 0 \cdot e_1 + 1\cdot e_2 = (1,2)+e_2 \Rightarrow e_2 = (2,2)$. Nos faltará encontrar solo $e_1$. Como no se impone ninguna condición adicional en el enunciado bastará con que sea linealmente independiente de $e_2$. El $(1,0)$ nos vale. Consideremos entonces
        \begin{gather*}
            \cc{R}=\{(1,2),\{(1,0),(2,2)\}\}
        \end{gather*}

        Con esto nos queda
        \begin{gather*}
            M(\cc{R}_0 \leftarrow \cc{R})=\left(
            \begin{array}{r|rr}
                1 & 0 & 0\\
                \hline
                1 & 1 & 2\\
                2 & 0 & 2
            \end{array}\right) \Rightarrow (x,y)_{\cc{R}_0}=
            \begin{pmatrix}
                1&2\\
                0&2
            \end{pmatrix} \cdot
            \begin{pmatrix}
                x\\y
            \end{pmatrix}_{\cc{R}} + 
            \begin{pmatrix}
                1\\2
            \end{pmatrix}\\\\
            \Rightarrow (x,y)_{\cc{R}_0}=(x+2y+1,2y+2)_\cc{R}
        \end{gather*}

    \end{ejercicio}
    
    \begin{ejercicio}[2 puntos]
        Demuestra que si $a,b,c,d$ son cuatro puntos de un espacio afín $\cc{A}$ tales que $\vec{ab}=\vec{cd}$, entonces se cumple que $\vec{ac}= \vec{bd}$.\\

        Esta es la identidad del paralelogramo. Su demostración es sencilla:
        \begin{gather*}
            \vec{ac}=\vec{ab}+\vec{bc}=\vec{cd}+\vec{bc}=\vec{bc}+\vec{cd}=\vec{bd}
        \end{gather*}
    \end{ejercicio}

    \begin{ejercicio}[3 puntos]
        Calcula la expresión explícita en coordenadas usuales de una aplicación afín de $\bb{R}^2$ cuyo conjunto de puntos fijos sea la recta de ecuación implícita $x=1$ (en coordenadas de sistema de referencia usual) y tal que la imagen del origen sea el punto $(2,2)$. ¿De qué aplicación se trata?\\

        % En primer lugar veamos la representación de esta aplicación:
        % \begin{figure}[H]
        %     \centering
        %     \input{graficas_Ex12_Latex.tex}
        % \end{figure}

        Como $f(0,0)=(2,2)$ y sabiendo que $f$ es una aplicación afín de $\bb{R}^2$ podemos considerar
        \begin{gather*}
            A=M(f;\cc{R}_0)=\left(
            \begin{array}{r|rr}
                1 & 0 & 0 \\
                \hline
                2 & a & b \\
                2 & c & d \\
            \end{array}\right)
        \end{gather*}

        Sabemos que tiene una recta de puntos fijos. Es decir, $(A-I)(1,p)=0 \Leftrightarrow p\in P_f$.
        \begin{gather*}
            (A-I)=\left(
                \begin{array}{c|cc}
                    0 & 0 & 0 \\
                    \hline
                    2 & a-1 & b \\
                    2 & c & d-1 \\
                \end{array}\right) \Rightarrow 
                \left(
                \begin{array}{c|cc}
                    0 & 0 & 0 \\
                    \hline
                    2 & a-1 & b \\
                    2 & c & d-1 \\
                \end{array}\right)
                \begin{pmatrix}
                    1\\1\\p_2
                \end{pmatrix} = 
                \begin{pmatrix}
                    0\\0\\0
                \end{pmatrix}\\\\
                \Rightarrow \left.
                \begin{array}{c}
                    2 + a -1 +bp_2 =0\\
                    2+c+p_2(d-1)=0
                \end{array} \right\} a=-1,\ b=0,\ c=-2,\ d= 1
        \end{gather*}

        Esta solución verifica la condición $\forall p_2 \in \bb{R}^2$. Por tanto nos queda
        \begin{gather*}
            A=M(f;\cc{R}_0)=\left(
            \begin{array}{r|rr}
                1 & 0 & 0 \\
                \hline
                2 & -1 & 0 \\
                2 & -2 & 1 \\
            \end{array}\right)
        \end{gather*}

        Veamos ahora qué aplicación es. Tenemos que $|A|=-1$ por lo que tenemos un movimiento inverso con una recta de puntos fijos. Se trata de una reflexión axial con respecto a la recta dada por $x=1$. Sin embargo no es ortogonal, como se puede ver al calcular $f(0,0)$. Estará orientada con respecto al vector $\vec{(0,0)f(0,0)}=\vec{(0,0)(2,2)}=(2,2)$. %No sé muy bien qué aplicación es 
    \end{ejercicio}

    \begin{ejercicio}[2 puntos]
        Sean $\cc{A}$ un espacio afín, $f: \cc{A}\rightarrow \cc{A}$ una aplicación afín y $\cc{S}=p_0+\cc{L}(\{v_0\})$ una recta en $\cc{A}$. Demuestra que $f(\cc{S})=\cc{S}$ si, y solo si, $\vec{p_0f(p_0)}\in \cc{L}(\{v_0\})$ y $v_0$ es un vector propio de $\vec{f}$ de autovalor no nulo.\\

        Veamos ambas implicaciones:
        \begin{itemize}
            \item [$\Rightarrow$)] Sea $p_0 \in \cc{S}$. Entonces, $f(p_0) \in f(\cc{S})=\cc{S}$, por lo que $\vec{p_0f(p_0)}\in \vec{\cc{S}}=\cc{L}(\{v_0\})$.
            Falta por ver que $v_0$ es un vector propio de $\vec{f}$ de autovalor no nulo.
            Como $v_0\in \vec{S}$, entonces $\vec{f}(v_0)\in \vec{f}(\cc{S})=\vec{f(\cc{S})}=\vec{\cc{S}}=\cc{L}(\{v_0\})$. Por tanto, se tiene que $\vec{f}(v_0)=\lambda v_0$, con $\lambda \neq 0$.
            
            \item [$\Leftarrow$)] Sea $p_0\in \cc{S}$, y veamos que $f(\cc{S})=\cc{S}$.
            Tenemos que $f(S)=f(p_0)+\cc{L}\{\vec{f}(v_0)\}$, y como
            $v_0$ es un vector propio de $\vec{f}$ de autovalor no nulo, entonces se tiene que
            $f(S)=f(p_0)+\cc{L}\{v_0\}$.
            
            Como $p_0 \in \cc{S}$ y $\vec{p_0f(p_0)}\in \vec{\cc{S}}$, entonces
            $f(p_0) \in \cc{S}$, por lo que $f(\cc{S})=\cc{S}$.
        \end{itemize}
    \end{ejercicio}
     
\end{document}
