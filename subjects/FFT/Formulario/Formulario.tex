\documentclass[12pt]{book}

% Idioma y codificación
\usepackage[spanish, es-tabla]{babel}       %es-tabla para que se titule "Tabla"
\usepackage[utf8]{inputenc}

% Márgenes
\usepackage[a4paper,top=3cm,bottom=2.5cm,left=3cm,right=3cm]{geometry}

% Comentarios de bloque
\usepackage{verbatim}

% Paquetes de links
\usepackage[hidelinks]{hyperref}    % Permite enlaces
\usepackage{url}                    % redirecciona a la web

% Más opciones para enumeraciones
\usepackage{enumitem}

% Personalizar la portada
\usepackage{titling}

% Paquetes de tablas
\usepackage{multirow}


%------------------------------------------------------------------------

%Paquetes de figuras
\usepackage{caption}
\usepackage{subcaption} % Figuras al lado de otras
\usepackage{float}      % Poner figuras en el sitio indicado H.


% Paquetes de imágenes
\usepackage{graphicx}       % Paquete para añadir imágenes
\usepackage{transparent}    % Para manejar la opacidad de las figuras

% Paquete para usar colores
\usepackage[dvipsnames]{xcolor}
\usepackage{pagecolor}      % Para cambiar el color de la página

% Habilita tamaños de fuente mayores
\usepackage{fix-cm}

% Para los gráficos
\usepackage{tikz}

% Para poder situar los nodos en los grafos
\usetikzlibrary{positioning}


%------------------------------------------------------------------------

% Paquetes de matemáticas
\usepackage{mathtools, amsfonts, amssymb, mathrsfs}
\usepackage[makeroom]{cancel}     % Simplificar tachando
\usepackage{polynom}    % Divisiones y Ruffini
\usepackage{units} % Para poner fracciones diagonales con \nicefrac

\usepackage{pgfplots}   %Representar funciones
\pgfplotsset{compat=1.18}  % Versión 1.18

\usepackage{tikz-cd}    % Para usar diagramas de composiciones
\usetikzlibrary{calc}   % Para usar cálculo de coordenadas en tikz

%Definición de teoremas, etc.
\usepackage{amsthm}
%\swapnumbers   % Intercambia la posición del texto y de la numeración

\theoremstyle{plain}

\makeatletter
\@ifclassloaded{article}{
  \newtheorem{teo}{Teorema}[section]
}{
  \newtheorem{teo}{Teorema}[chapter]  % Se resetea en cada chapter
}
\makeatother

\newtheorem{coro}{Corolario}[teo]           % Se resetea en cada teorema
\newtheorem{prop}[teo]{Proposición}         % Usa el mismo contador que teorema
\newtheorem{lema}[teo]{Lema}                % Usa el mismo contador que teorema

\theoremstyle{remark}
\newtheorem*{observacion}{Observación}

\theoremstyle{definition}

\makeatletter
\@ifclassloaded{article}{
  \newtheorem{definicion}{Definición} [section]     % Se resetea en cada chapter
}{
  \newtheorem{definicion}{Definición} [chapter]     % Se resetea en cada chapter
}
\makeatother

\newtheorem*{notacion}{Notación}
\newtheorem*{ejemplo}{Ejemplo}
\newtheorem*{ejercicio*}{Ejercicio}             % No numerado
\newtheorem{ejercicio}{Ejercicio} [section]     % Se resetea en cada section


% Modificar el formato de la numeración del teorema "ejercicio"
\renewcommand{\theejercicio}{%
  \ifnum\value{section}=0 % Si no se ha iniciado ninguna sección
    \arabic{ejercicio}% Solo mostrar el número de ejercicio
  \else
    \thesection.\arabic{ejercicio}% Mostrar número de sección y número de ejercicio
  \fi
}


% \renewcommand\qedsymbol{$\blacksquare$}         % Cambiar símbolo QED
%------------------------------------------------------------------------

% Paquetes para encabezados
\usepackage{fancyhdr}
\pagestyle{fancy}
\fancyhf{}

\newcommand{\helv}{ % Modificación tamaño de letra
\fontfamily{}\fontsize{12}{12}\selectfont}
\setlength{\headheight}{15pt} % Amplía el tamaño del índice


%\usepackage{lastpage}   % Referenciar última pag   \pageref{LastPage}
\fancyfoot[C]{\thepage}

%------------------------------------------------------------------------

% Conseguir que no ponga "Capítulo 1". Sino solo "1."
\makeatletter
\@ifclassloaded{book}{
  \renewcommand{\chaptermark}[1]{\markboth{\thechapter.\ #1}{}} % En el encabezado
    
  \renewcommand{\@makechapterhead}[1]{%
  \vspace*{50\p@}%
  {\parindent \z@ \raggedright \normalfont
    \ifnum \c@secnumdepth >\m@ne
      \huge\bfseries \thechapter.\hspace{1em}\ignorespaces
    \fi
    \interlinepenalty\@M
    \Huge \bfseries #1\par\nobreak
    \vskip 40\p@
  }}
}
\makeatother

%------------------------------------------------------------------------
% Paquetes de cógido
\usepackage{minted}
\renewcommand\listingscaption{Código fuente}

\usepackage{fancyvrb}
% Personaliza el tamaño de los números de línea
\renewcommand{\theFancyVerbLine}{\small\arabic{FancyVerbLine}}

% Estilo para C++
\newminted{cpp}{
    frame=lines,
    framesep=2mm,
    baselinestretch=1.2,
    linenos,
    escapeinside=||
}

% para minted
\definecolor{LightGray}{rgb}{0.95,0.95,0.92}
\setminted{
    linenos=true,
    stepnumber=5,
    numberfirstline=true,
    autogobble,
    breaklines=true,
    breakautoindent=true,
    breaksymbolleft=,
    breaksymbolright=,
    breaksymbolindentleft=0pt,
    breaksymbolindentright=0pt,
    breaksymbolsepleft=0pt,
    breaksymbolsepright=0pt,
    fontsize=\footnotesize,
    bgcolor=LightGray,
    numbersep=10pt
}


\usepackage{listings} % Para incluir código desde un archivo

\renewcommand\lstlistingname{Código Fuente}
\renewcommand\lstlistlistingname{Índice de Códigos Fuente}

% Definir colores
\definecolor{vscodepurple}{rgb}{0.5,0,0.5}
\definecolor{vscodeblue}{rgb}{0,0,0.8}
\definecolor{vscodegreen}{rgb}{0,0.5,0}
\definecolor{vscodegray}{rgb}{0.5,0.5,0.5}
\definecolor{vscodebackground}{rgb}{0.97,0.97,0.97}
\definecolor{vscodelightgray}{rgb}{0.9,0.9,0.9}

% Configuración para el estilo de C similar a VSCode
\lstdefinestyle{vscode_C}{
  backgroundcolor=\color{vscodebackground},
  commentstyle=\color{vscodegreen},
  keywordstyle=\color{vscodeblue},
  numberstyle=\tiny\color{vscodegray},
  stringstyle=\color{vscodepurple},
  basicstyle=\scriptsize\ttfamily,
  breakatwhitespace=false,
  breaklines=true,
  captionpos=b,
  keepspaces=true,
  numbers=left,
  numbersep=5pt,
  showspaces=false,
  showstringspaces=false,
  showtabs=false,
  tabsize=2,
  frame=tb,
  framerule=0pt,
  aboveskip=10pt,
  belowskip=10pt,
  xleftmargin=10pt,
  xrightmargin=10pt,
  framexleftmargin=10pt,
  framexrightmargin=10pt,
  framesep=0pt,
  rulecolor=\color{vscodelightgray},
  backgroundcolor=\color{vscodebackground},
}

%------------------------------------------------------------------------

% Comandos definidos
\newcommand{\bb}[1]{\mathbb{#1}}
\newcommand{\cc}[1]{\mathcal{#1}}

% I prefer the slanted \leq
\let\oldleq\leq % save them in case they're every wanted
\let\oldgeq\geq
\renewcommand{\leq}{\leqslant}
\renewcommand{\geq}{\geqslant}

% Si y solo si
\newcommand{\sii}{\iff}

% Letras griegas
\newcommand{\eps}{\epsilon}
\newcommand{\veps}{\varepsilon}
\newcommand{\lm}{\lambda}

\newcommand{\ol}{\overline}
\newcommand{\ul}{\underline}
\newcommand{\wt}{\widetilde}
\newcommand{\wh}{\widehat}

\let\oldvec\vec
\renewcommand{\vec}{\overrightarrow}

% Derivadas parciales
\newcommand{\del}[2]{\frac{\partial #1}{\partial #2}}
\newcommand{\Del}[3]{\frac{\partial^{#1} #2}{\partial #3^{#1}}}
\newcommand{\deld}[2]{\dfrac{\partial #1}{\partial #2}}
\newcommand{\Deld}[3]{\dfrac{\partial^{#1} #2}{\partial #3^{#1}}}


\newcommand{\AstIg}{\stackrel{(\ast)}{=}}
\newcommand{\Hop}{\stackrel{L'H\hat{o}pital}{=}}

\newcommand{\red}[1]{{\color{red}#1}} % Para integrales, destacar los cambios.

% Método de integración
\newcommand{\MetInt}[2]{
    \left[\begin{array}{c}
        #1 \\ #2
    \end{array}\right]
}

% Declarar aplicaciones
% 1. Nombre aplicación
% 2. Dominio
% 3. Codominio
% 4. Variable
% 5. Imagen de la variable
\newcommand{\Func}[5]{
    \begin{equation*}
        \begin{array}{rrll}
            #1:& #2 & \longrightarrow & #3\\
               & #4 & \longmapsto & #5
        \end{array}
    \end{equation*}
}

%------------------------------------------------------------------------

\usepackage{booktabs}



\begin{document}

    % 1. Foto de fondo
    % 2. Título
    % 3. Encabezado Izquierdo
    % 4. Color de fondo
    % 5. Coord x del titulo
    % 6. Coord y del titulo
    % 7. Fecha
    % 8. Autor

    
    % 1. Foto de fondo
% 2. Título
% 3. Encabezado Izquierdo
% 4. Color de fondo
% 5. Coord x del titulo
% 6. Coord y del titulo
% 7. Fecha

\newcommand{\portada}[7]{

    \portadaBase{#1}{#2}{#3}{#4}{#5}{#6}{#7}
    \portadaBook{#1}{#2}{#3}{#4}{#5}{#6}{#7}
}

\newcommand{\portadaExamen}[7]{

    \portadaBase{#1}{#2}{#3}{#4}{#5}{#6}{#7}
    \portadaArticle{#1}{#2}{#3}{#4}{#5}{#6}{#7}
}




\newcommand{\portadaBase}[7]{

    % Tiene la portada principal y la licencia Creative Commons
    
    % 1. Foto de fondo
    % 2. Título
    % 3. Encabezado Izquierdo
    % 4. Color de fondo
    % 5. Coord x del titulo
    % 6. Coord y del titulo
    % 7. Fecha
    
    
    \thispagestyle{empty}               % Sin encabezado ni pie de página
    \newgeometry{margin=0cm}        % Márgenes nulos para la primera página
    
    
    % Encabezado
    \fancyhead[L]{\helv #3}
    \fancyhead[R]{\helv \nouppercase{\leftmark}}
    
    
    \pagecolor{#4}        % Color de fondo para la portada
    
    \begin{figure}[p]
        \centering
        \transparent{0.3}           % Opacidad del 30% para la imagen
        
        \includegraphics[width=\paperwidth, keepaspectratio]{assets/#1}
    
        \begin{tikzpicture}[remember picture, overlay]
            \node[anchor=north west, text=white, opacity=1, font=\fontsize{60}{90}\selectfont\bfseries\sffamily, align=left] at (#5, #6) {#2};
            
            \node[anchor=south east, text=white, opacity=1, font=\fontsize{12}{18}\selectfont\sffamily, align=right] at (9.7, 3) {\textbf{\href{https://losdeldgiim.github.io/}{Los Del DGIIM}}};
            
            \node[anchor=south east, text=white, opacity=1, font=\fontsize{12}{15}\selectfont\sffamily, align=right] at (9.7, 1.8) {Doble Grado en Ingeniería Informática y Matemáticas\\Universidad de Granada};
        \end{tikzpicture}
    \end{figure}
    
    
    \restoregeometry        % Restaurar márgenes normales para las páginas subsiguientes
    \pagecolor{white}       % Restaurar el color de página
    
    
    \newpage
    \thispagestyle{empty}               % Sin encabezado ni pie de página
    \begin{tikzpicture}[remember picture, overlay]
        \node[anchor=south west, inner sep=3cm] at (current page.south west) {
            \begin{minipage}{0.5\paperwidth}
                \href{https://creativecommons.org/licenses/by-nc-nd/4.0/}{
                    \includegraphics[height=2cm]{assets/Licencia.png}
                }\vspace{1cm}\\
                Esta obra está bajo una
                \href{https://creativecommons.org/licenses/by-nc-nd/4.0/}{
                    Licencia Creative Commons Atribución-NoComercial-SinDerivadas 4.0 Internacional (CC BY-NC-ND 4.0).
                }\\
    
                Eres libre de compartir y redistribuir el contenido de esta obra en cualquier medio o formato, siempre y cuando des el crédito adecuado a los autores originales y no persigas fines comerciales. 
            \end{minipage}
        };
    \end{tikzpicture}
    
    
    
    % 1. Foto de fondo
    % 2. Título
    % 3. Encabezado Izquierdo
    % 4. Color de fondo
    % 5. Coord x del titulo
    % 6. Coord y del titulo
    % 7. Fecha


}


\newcommand{\portadaBook}[7]{

    % 1. Foto de fondo
    % 2. Título
    % 3. Encabezado Izquierdo
    % 4. Color de fondo
    % 5. Coord x del titulo
    % 6. Coord y del titulo
    % 7. Fecha

    % Personaliza el formato del título
    \pretitle{\begin{center}\bfseries\fontsize{42}{56}\selectfont}
    \posttitle{\par\end{center}\vspace{2em}}
    
    % Personaliza el formato del autor
    \preauthor{\begin{center}\Large}
    \postauthor{\par\end{center}\vfill}
    
    % Personaliza el formato de la fecha
    \predate{\begin{center}\huge}
    \postdate{\par\end{center}\vspace{2em}}
    
    \title{#2}
    \author{\href{https://losdeldgiim.github.io/}{Los Del DGIIM}}
    \date{Granada, #7}
    \maketitle
    
    \tableofcontents
}




\newcommand{\portadaArticle}[7]{

    % 1. Foto de fondo
    % 2. Título
    % 3. Encabezado Izquierdo
    % 4. Color de fondo
    % 5. Coord x del titulo
    % 6. Coord y del titulo
    % 7. Fecha

    % Personaliza el formato del título
    \pretitle{\begin{center}\bfseries\fontsize{42}{56}\selectfont}
    \posttitle{\par\end{center}\vspace{2em}}
    
    % Personaliza el formato del autor
    \preauthor{\begin{center}\Large}
    \postauthor{\par\end{center}\vspace{3em}}
    
    % Personaliza el formato de la fecha
    \predate{\begin{center}\huge}
    \postdate{\par\end{center}\vspace{5em}}
    
    \title{#2}
    \author{\href{https://losdeldgiim.github.io/}{Los Del DGIIM}}
    \date{Granada, #7}
    \thispagestyle{empty}               % Sin encabezado ni pie de página
    \maketitle
    \vfill
}
    \portada{etsiitA4.jpg}{Formulario\\Circuitos Eléctricos\\FFT}{Formulario FFT}{MidnightBlue}{-9}{28}{2025}{Carlos Salazar Castillo}
    
\chapter{Circuitos en Corriente Continua (CC)}
\label{sec:cc}

\begin{table}[h]
\centering
\caption{Fórmulas de Circuitos en Corriente Continua}
\label{tab:cc}
\begin{tabular}{lcl}
\toprule
\textbf{Concepto} & \textbf{Fórmula} & \textbf{Unidades Típicas} \\
\midrule
Intensidad de Corriente ($I$) & $\mathbf{I = \dfrac{dQ}{dt}}$ & Amperio ($\text{A}$) \\
Ley de Ohm (Tensión en Resistencia) & $\mathbf{V = IR}$ & Voltio ($\text{V}$) \\
Potencia Consumida (Efecto Joule) & $\mathbf{P = IV = I^2R = \dfrac{V^2}{R}}$ & Vatio ($\text{W}$) \\
Energía Consumida ($U$) & $\mathbf{U = P t}$ & Julio ($\text{J}$) \\
Ley de Corrientes de Kirchhoff (LKC) & $\mathbf{\sum\limits I_{\text{entrantes}} = \sum\limits I_{\text{salientes}}}$ & $\text{A}$ \\
Ley de Tensiones de Kirchhoff (LKV) & $\mathbf{\sum\limits V_{\text{caídas}} = \sum\limits V_{\text{elevaciones}}}$ & $\text{V}$ \\
Resistencias en Serie & $\mathbf{R_{\text{eq}} = \sum\limits R_i}$ & Ohm ($\Omega$) \\
Resistencias en Paralelo & $\mathbf{\dfrac{1}{R_{\text{eq}}} = \sum\limits \dfrac{1}{R_i}}$ & $\Omega$ \\
Relación Thévenin-Norton & $\mathbf{I_n = \dfrac{V_{\text{th}}}{R_{\text{th}}}}$ & $\text{A}$ \\
\bottomrule
\end{tabular}
\end{table}

\clearpage

\chapter{Circuitos No Estacionarios (Transitorios)}
\section{Condensadores ($C$)}
\begin{table}[h]
\centering
\caption{Fórmulas de Condensadores}
\label{tab:condensadores}
\begin{tabular}{ll}
\toprule
\textbf{Concepto} & \textbf{Fórmula} \\
\midrule
Relación I-V & $\mathbf{i(t) = C \dfrac{dv(t)}{dt}}$ \\
Energía Almacenada & $\mathbf{u(t) = \dfrac{1}{2} C v(t)^2}$ \\
Capacitancia en Serie & $\mathbf{\dfrac{1}{C_{\text{eq}}} = \sum\limits \dfrac{1}{C_i}}$ \\
Capacitancia en Paralelo & $\mathbf{C_{\text{eq}} = \sum\limits C_i}$ \\
Constante de Tiempo $\tau$ (circuito RC) & $\mathbf{\tau = R C}$ \\
Tensión en Carga (sol. transitoria) & $\mathbf{v_C(t) = V_{\text{final}} \left( 1 - e^{\nicefrac{-t}{\tau}} \right) + v_C(0) e^{\nicefrac{-t}{\tau}}}$ \\
\bottomrule
\end{tabular}
\end{table}

\section{Inductores ($L$)}
\begin{table}[h]
\centering
\caption{Fórmulas de Inductores}
\label{tab:inductores}
\begin{tabular}{ll}
\toprule
\textbf{Concepto} & \textbf{Fórmula} \\
\midrule
Relación V-I & $\mathbf{v(t) = L \dfrac{di(t)}{dt}}$ \\
Energía Almacenada & $\mathbf{u(t) = \dfrac{1}{2} L i(t)^2}$ \\
Inductancia en Serie & $\mathbf{L_{\text{eq}} = \sum\limits L_i}$ \\
Inductancia en Paralelo & $\mathbf{\dfrac{1}{L_{\text{eq}}} = \sum\limits \dfrac{1}{L_i}}$ \\
Constante de Tiempo $\tau$ (circuito RL) & $\mathbf{\tau = L/R}$ \\
Corriente en Carga (sol. transitoria) & $\mathbf{i_L(t) = I_{\text{final}} \left( 1 - e^{\nicefrac{-t}{\tau}} \right) + i_L(0) e^{\nicefrac{-t}{\tau}}}$ \\
\bottomrule
\end{tabular}
\end{table}

\clearpage

\chapter{Corriente Alterna (CA) y Análisis Fasorial}
\begin{table}[h]
\centering
\caption{Fórmulas de CA y Fasores}
\label{tab:ca}
\begin{tabular}{ll}
\toprule
\textbf{Concepto} & \textbf{Fórmula} \\
\midrule
Señal Sinusoidal & $\mathbf{v(t) = V_0 \cos (\omega t + \phi)}$ \\
Valor Eficaz (r.m.s.) & $\mathbf{V_{\text{rms}} = \dfrac{V_0}{\sqrt{2}}}$ \\
Impedancia (General) & $\mathbf{Z = R + jX}$ \\
Reactancia Inductiva & $\mathbf{Z_L = j\omega L}$ \\
Reactancia Capacitiva & $\mathbf{Z_C = \dfrac{1}{j\omega C} = \dfrac{-j}{\omega C}}$ \\
Ley de Ohm Fasorial & $\mathbf{\mathbf{V} = \mathbf{I} \mathbf{Z}}$ \\
Potencia Compleja ($\mathbf{S}$) & $\mathbf{\mathbf{S} = \mathbf{V} \mathbf{I}^* = P + jQ}$ \\
Potencia Activa (Real) ($P$) & $\mathbf{P = V_{\text{rms}} I_{\text{rms}} \cos \phi}$ \\
Potencia Reactiva ($Q$) & $\mathbf{Q = V_{\text{rms}} I_{\text{rms}} \sen \phi}$ \\
Factor de Potencia (FP) & $\mathbf{FP = \cos \phi = \dfrac{P}{|\mathbf{S}|}}$ \\
\bottomrule
\end{tabular}
\end{table}

\chapter{Dispositivos Semiconductores (Básicos)}
\section{Diodo}
\begin{table}[h]
\centering
\caption{Fórmulas de Diodo}
\label{tab:diodo}
\begin{tabular}{ll}
\toprule
\textbf{Concepto} & \textbf{Fórmula} \\
\midrule
Ecuación de Shockley (I-V) & $\mathbf{I_D = I_s \left( e^{\dfrac{V_D}{\eta V_T}} - 1 \right)}$ \\
Voltaje Térmico ($V_T$) & $\mathbf{V_T = \dfrac{kT}{q}}$ \\
Modelo Simplificado (Encendido) & $\mathbf{V_D \approx V_{\gamma}}$ (Tensión de umbral) \\
\bottomrule
\end{tabular}
\end{table}

\section{Transistor MOSFET (NMOSFET, con modulación de canal $\lambda$)}
\begin{table}[h]
\centering
\caption{Fórmulas de Transistor MOSFET (NMOSFET)}
\label{tab:mosfet}
\begin{tabular}{ll}
\toprule
\textbf{Concepto} & \textbf{Fórmula} \\
\midrule
Condición de Corte & $\mathbf{V_{\text{GS}} < V_{\text{th}}}$ \\
Corriente $I_D$ en Saturación & $\mathbf{I_D = \dfrac{1}{2} k'_n \dfrac{W}{L} (V_{\text{GS}} - V_{\text{th}})^2 (1 + \lambda V_{\text{DS}})}$ \\
Parámetro de Transconductancia & $\mathbf{k'_n = \mu_n C_{\text{ox}}}$ \\
\bottomrule
\end{tabular}
\end{table}

\chapter{Análisis de Sistemas y Medidas (Diagramas de Bode)}
\begin{table}[h]
\centering
\caption{Fórmulas de Bode y Medidas}
\label{tab:bode}
\begin{tabular}{ll}
\toprule
\textbf{Concepto} & \textbf{Fórmula} \\
\midrule
Módulo en Decibelios ($\text{dB}$) & $\mathbf{|\mathbf{T}(\omega)|_{\text{dB}} = 20 \log_{10} |\mathbf{T}(\omega)|}$ \\
Suma de Módulos en Bode & $\mathbf{20 \log_{10} |\mathbf{T}(\omega)| = \sum\limits_i 20 \log_{10} |\mathbf{T}_i(\omega)|}$ \\
Suma de Argumentos en Bode & $\mathbf{\arg \mathbf{T} (\omega) = \sum\limits_i \arg \mathbf{T}_i (\omega)}$ \\
Error Relativo Porcentual & $\mathbf{\epsilon_{\%, x} = \dfrac{\Delta x}{\bar{x}} \cdot 100}$ \\
Propagación de Error ($F=f(x, y)$) & $\mathbf{\Delta F = \left| \dfrac{\partial f}{\partial x} \right| \Delta x + \left| \dfrac{\partial f}{\partial y} \right| \Delta y}$ \\
\bottomrule
\end{tabular}
\end{table}

\end{document}
