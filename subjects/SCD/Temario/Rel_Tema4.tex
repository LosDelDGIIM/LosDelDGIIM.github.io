\section{Sistemas de Tiempo Real}

\begin{ejercicio}\label{ej:rel4_1}
    Dado el conjunto de tareas periódicas y sus atributos temporales que se indica en la Tabla~\ref{tab:4_1}, determinar si se puede planificar el conjunto de dichas tareas utilizando un esquema de planificación basado en planificación cíclica. Diseña el plan cíclico determinando el marco secundario, y el entrelazamiento de las tareas sobre un cronograma.
    \begin{table}[H]
    \centering
    \begin{tabular}{|c|c|c|c|}
        \hline
        Tarea & $C_i$ & $T_i$ & $D_i$ \\
        \hline
        T1 & 10 & 40 & 40 \\
        \hline
        T2 & 18 & 50 & 50 \\
        \hline
        T3 & 10 & 200 & 200 \\
        \hline
        T4 & 20 & 200 & 200 \\
        \hline
    \end{tabular}
    \caption{Tareas periódicas y sus atributos temporales.}
    \label{tab:4_1}
    \end{table}
\end{ejercicio}

\begin{ejercicio}\label{ej:rel4_2}
    El siguiente conjunto de tareas periódicas se puede planificar con ejecutivos cíclicos. Determina si esto es cierto calculando el marco secundario que debería tener. Dibuja el cronograma que muestre las ocurrencias de cada tarea y su entrelazamiento. ¿Cómo se tendría que implementar? (escribe el pseudo-código de la implementación)
    \begin{table}[H]
    \centering
    \begin{tabular}{|c|c|c|c|}
        \hline
        Tarea & $C_i$ & $T_i$ & $D_i$ \\
        \hline
        T1 & 2 & 6 & 6 \\
        \hline
        T2 & 2 & 8 & 8 \\
        \hline
        T3 & 3 & 12 & 12 \\
        \hline
    \end{tabular}
    \caption{Tareas periódicas y sus atributos temporales.}
    \label{tab:4_2}
    \end{table}
\end{ejercicio}

\begin{ejercicio}\label{ej:rel4_3}
    Comprobar si el conjunto de procesos periódicos que se muestra en la siguiente tabla es planificable con el algoritmo RMS utilizando el test basado en el factor de utilización del tiempo del procesador. Si el test no se cumple, ¿debemos descartar que el sistema sea planificable?    
    \begin{table}[H]
    \centering
    \begin{tabular}{|c|c|c|}
        \hline
        Tarea & $C_i$ & $T_i$ \\
        \hline
        T1 & 9 & 30 \\
        \hline
        T2 & 10 & 40 \\
        \hline
        T3 & 10 & 50 \\
        \hline
    \end{tabular}
    \caption{Tareas periódicas y sus atributos temporales.}
    \label{tab:4_3}
    \end{table}
\end{ejercicio}

\begin{ejercicio}\label{ej:rel4_4}
    Considérese el siguiente conjunto de tareas compuesto por tres tareas periódicas:
    \begin{table}[H]
    \centering
    \begin{tabular}{|c|c|c|}
        \hline
        Tarea & $C_i$ & $T_i$ \\
        \hline
        T1 & 10 & 40 \\
        \hline
        T2 & 20 & 60 \\
        \hline
        T3 & 20 & 80 \\
        \hline
    \end{tabular}
    \caption{Tareas periódicas y sus atributos temporales.}
    \label{tab:4_4}
    \end{table}
    Comprueba la planificabilidad del conjunto de tareas con el algoritmo RMS utilizando el test basado en el factor de utilización. Calcular el hiperperiodo y construir el correspondiente cronograma.
\end{ejercicio}

\begin{ejercicio}\label{ej:rel4_5}
    Comprobar la planificabilidad y construir el cronograma de acuerdo al algoritmo de planificación RMS del siguiente conjunto de tareas periódicas.
    \begin{table}[H]
    \centering
    \begin{tabular}{|c|c|c|}
        \hline
        Tarea & $C_i$ & $T_i$ \\
        \hline
        T1 & 20 & 60 \\
        \hline
        T2 & 20 & 80 \\
        \hline
        T3 & 20 & 120 \\
        \hline
    \end{tabular}
    \caption{Tareas periódicas y sus atributos temporales.}
    \label{tab:4_5}
    \end{table}
\end{ejercicio}

\begin{ejercicio}\label{ej:rel4_6}
    Determinar si el siguiente conjunto de tareas puede planificarse con la política de planificación RMS y con la política EDF, utilizando los tests de planificabilidad adecuados para cada uno de los dos casos. Comprobar también la planificabilidad en ambos casos construyendo los dos cronogramas.
    \begin{table}[H]
    \centering
    \begin{tabular}{|c|c|c|}
        \hline
        Tarea & $C_i$ & $T_i$ \\
        \hline
        T1 & 1 & 5 \\
        \hline
        T2 & 1 & 10 \\
        \hline
        T3 & 2 & 20 \\
        \hline
        T4 & 10 & 20 \\
        \hline
        T5 & 7 & 100 \\
        \hline
    \end{tabular}
    \caption{Tareas periódicas y sus atributos temporales.}
    \label{tab:4_6}
    \end{table}
\end{ejercicio}

\begin{ejercicio}\label{ej:rel4_7}
    Describe razonadamente si el siguiente conjunto de tareas puede planificarse o no puede planificarse en un sistema monoprocesador usando un ejecutivo cíclico o usando algún algoritmo basado en prioridades estáticas o dinámicas.
    \begin{table}[H]
    \centering
    \begin{tabular}{|c|c|c|}
        \hline
        Tarea & $C_i$ & $T_i$ \\
        \hline
        T1 & 1 & 5 \\
        \hline
        T2 & 1 & 10 \\
        \hline
        T3 & 2 & 10 \\
        \hline
        T4 & 11 & 20 \\
        \hline
        T5 & 5 & 100 \\
        \hline
    \end{tabular}
    \caption{Tareas periódicas y sus atributos temporales.}
    \label{tab:4_7}
    \end{table}
\end{ejercicio}

\subsubsection{Problemas adicionales}

\begin{ejercicio}\label{ej:rel4_8}
    Para el conjunto de tareas cuyos datos se muestran más abajo, se pide:
    \begin{itemize}
        \item Dibujar el gráfico de ejecución y obtener el tiempo de respuesta de cada tarea.
        \item Determinar, mediante inspección del gráfico anterior, cuántas veces interfiere la tarea $\mathcal{T}_1$ a la tarea $\mathcal{T}_3$ durante un intervalo temporal dado por el tiempo de respuesta de esta última tarea.
        \item Hacer lo mismo que en el apartado anterior pero para las tareas $\mathcal{T}_1$ y $\mathcal{T}_2$.
    \end{itemize}
    \begin{table}[H]
    \centering
    \begin{tabular}{|c|c|c|c|}
        \hline
        Tarea & $C_i$ & $T_i$ & $D_i$ \\
        \hline
        T1 & 1 & 3 & 2 \\
        \hline
        T2 & 3 & 6 & 5 \\
        \hline
        T3 & 2 & 13 & 13 \\
        \hline
    \end{tabular}
    \caption{Tareas periódicas y sus atributos temporales.}
    \label{tab:4_8}
    \end{table}
\end{ejercicio}

\begin{ejercicio}\label{ej:rel4_9}
    Verificar la planificabilidad del siguiente conjunto de tareas utilizando para ello el algoritmo del ``primero el del tiempo límite más cercano'' (EDF).
    \begin{table}[H]
    \centering
    \begin{tabular}{|c|c|c|}
        \hline
        Tarea & $C_i$ & $T_i$ \\
        \hline
        T1 & 1 & 4 \\
        \hline
        T2 & 2 & 6 \\
        \hline
        T3 & 3 & 8 \\
        \hline
    \end{tabular}
    \caption{Tareas periódicas y sus atributos temporales.}
    \label{tab:4_9}
    \end{table}
\end{ejercicio}

\begin{ejercicio}\label{ej:rel4_10}
    Verificar la planificabilidad utilizando el algoritmo EDF de asignación dinámica de prioridades a las tareas y construir el diagrama de ejecución de tareas del siguiente conjunto:
    \begin{table}[H]
    \centering
    \begin{tabular}{|c|c|c|c|}
        \hline
        Tarea & $C_i$ & $D_i$ & $T_i$ \\
        \hline
        T1 & 2 & 5 & 6 \\
        \hline
        T2 & 2 & 4 & 8 \\
        \hline
        T3 & 4 & 8 & 12 \\
        \hline
    \end{tabular}
    \caption{Tareas periódicas y sus atributos temporales.}
    \label{tab:4_10}
    \end{table}
\end{ejercicio}

\begin{ejercicio}\label{ej:rel4_11}
    Verificar la planificabilidad del conjunto de tareas descrito en el Ejercicio~\ref{ej:rel4_10} utilizando el algoritmo del ``plazo de respuesta máximo (D)'' (algoritmo \textit{deadline monotomic} o DM).
\end{ejercicio}

\begin{ejercicio}\label{ej:rel4_12}
    Indicar cuáles de las siguientes afirmaciones son correctas respecto de los algoritmos para resolver el problema de \textit{inversión de prioridad} de las tareas en sistemas de tiempo real de \textit{misión crítica}:
    \begin{enumerate}[label=(\alph*)]
        \item Suponiendo que las tareas se planifican con el protocolo de \textit{herencia de prioridad}: la prioridad heredada por una tarea sólo se mantiene mientras dicha tarea esté utilizando un recurso compartido con otra tarea más prioritaria.
        \item Con el protocolo de \textit{techo de prioridad}, cuando una tarea adquiere un recurso no puede verse interrumpida, hasta que termine su ejecución, por otras tareas que se activan después que ésta y que vayan a utilizar en el futuro un recurso con límite de prioridad igual o inferior.
        \item Con el protocolo de techo de prioridad (PPP), una tarea no puede comenzar a ejecutarse si no están libres todos los recursos que va a utilizar durante su primer ciclo.
        \item Si consideramos una tarea periódica que utilice el protocolo de techo de prioridad inmediato para cambiar su prioridad dinámica cuando accede a recursos, siempre se cumplirá que dicha tarea no puede ser interrumpida por otra menos prioritaria que ella.
        \item Con el protocolo de techo de prioridad las tareas más prioritarias del sistema pueden ser interrumpidas durante cada ciclo de su ejecución como máximo 1 vez cuando acceden a recursos que comparten con otras tareas menos prioritarias.
        \item El protocolo de techo de prioridad original producirá siempre tiempos de respuesta menores para las tareas que el algoritmo de \textit{herencia de prioridad}.
    \end{enumerate}
\end{ejercicio}

\begin{ejercicio}\label{ej:rel4_13}
    Calcular la utilización máxima del procesador que se puede asignar al \textit{Servidor Esporádico} para garantizar la planificabilidad del siguiente conjunto de tareas periódicas utilizando RM.
    \begin{table}[H]
    \centering
    \begin{tabular}{|c|c|c|}
        \hline
        $\mathcal{T}_1$ & 1 & 5 \\
        \hline
        $\mathcal{T}_2$ & 2 & 8 \\
        \hline
    \end{tabular}
    \caption{Conjunto de tareas periódicas.}
    \label{tab:4_13}
    \end{table}
\end{ejercicio}

\begin{ejercicio}\label{ej:rel4_14}
    Calcular la utilización máxima del procesador que puede ser asignada al \textit{Servidor Diferido} (SD) para garantizar la planificabilidad del conjunto de tareas periódicas dado en el Ejercicio~\ref{ej:rel4_13} anterior.
\end{ejercicio}

\begin{ejercicio}\label{ej:rel4_15}
    Junto con las tareas periódicas que se muestran en el Ejercicio~\ref{ej:rel4_13} definir un plan para planificar las siguientes tareas aperiódicas utilizando un \textit{SD} (tarea sondeante) que posea una utilización máxima del tiempo del procesador y prioridad intermedia:
    \begin{table}[H]
    \centering
    \begin{tabular}{|c|c|c|}
        \hline
        & $t_a$ & $C_i$ \\
        \hline
        $J_1$ & 2 & 2 \\
        \hline
        $J_2$ & 7 & 2 \\
        \hline
        $J_3$ & 17 & 1 \\
        \hline
    \end{tabular}
    \caption{Tareas periódicas y sus atributos temporales.}
    \label{tab:4_15}
    \end{table}
\end{ejercicio}

\begin{ejercicio}\label{ej:rel4_16}
    Resolver ahora el mismo problema de planificación descrito en el Ejercicio~\ref{ej:rel4_15} utilizando ahora un \textit{Servidor Esporádico} que tenga una utilización máxima y prioridad intermedia.
\end{ejercicio}

\begin{ejercicio}\label{ej:rel4_17}
    Resolver el mismo problema de planificación descrito en el Ejercicio~\ref{ej:rel4_15} utilizando ahora un \textit{Servidor Diferido} que tenga una utilización máxima y prioridad intermedia.
\end{ejercicio}

\begin{ejercicio}\label{ej:rel4_18}
    Utilizar un \textit{Servidor Esporádico} con capacidad $C_s=2$ y periodo $T_s=6$ para planificar las siguiente tareas:
    \begin{table}[H]
    \centering
    \begin{tabular}{|c|c|c|}
        \hline
        & $C_i$ & $T_i$ \\
        \hline
        $\mathcal{T}_1$ & 1 & 4 \\
        \hline
        $\mathcal{T}_2$ & 2 & 6 \\
        \hline
                        & $a_i$ & $C_i$ \\
        \hline
        $J_1$ & 2 & 2 \\
        \hline
        $J_2$ & 5 & 1 \\
        \hline
        $J_3$ & 10 & 2 \\
        \hline
    \end{tabular}
    \caption{Tareas periódicas y sus atributos temporales.}
    \label{tab:4_18}
    \end{table}
\end{ejercicio}
