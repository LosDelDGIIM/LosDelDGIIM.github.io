\section{Paso de mensajes}

\begin{ejercicio}\label{ej:rel3_1}
    En un sistema distribuido, 6 procesos clientes necesitan sincronizarse de forma específica para realizar cierta tarea, de forma que dicha tarea sólo podrá ser realizada cuando tres procesos estén preparados para realizarla. Para ello, envían peticiones a un proceso controlador del recurso y esperan respuesta para poder realizar la tarea específica. El proceso controlador se encarga de asegurar la sincronización adecuada. Para ello, recibe y cuenta las peticiones que le llegan de los procesos, las dos primeras no son respondidas y producen la suspensión del proceso que envía la petición (debido a que se bloquea esperando respuesta) pero la tercera petición produce el desbloqueo de los tres procesos pendientes de respuesta. A continuación, una vez desbloqueados los tres procesos que han pedido (al recibir respuesta), inicializa la cuenta y procede cíclicamente de la misma forma sobre otras peticiones. El código de los procesos clientes aparece aquí abajo. Los clientes usan envío asíncrono seguro para realizar su petición, y esperan con una recepción síncrona antes de realizar la tarea:
    \begin{figure}[H]
        \centering
        \setlength{\columnsep}{1cm}
        \begin{multicols}{2}
            \begin{minted}{pascal}
                process Cliente[ i : 0..5 ];
                begin
                   while true do begin
                      send(peticion, Controlador);
                      receive(permiso, Controlador);
                      Realiza_tarea_grupal();
                   end
                end
            \end{minted}
            \begin{minted}{pascal}
                process Controlador;
                begin
                   while true do begin
                      ...
                   end
                end
            \end{minted}
        \end{multicols}
        \caption{Código para el Ejercicio~\ref{ej:rel3_1}.}
        \label{fig:cod_1}
    \end{figure}
    Describir en pseudocódigo el comportamiento del proceso controlador, utilizando una orden de espera selectiva que permita implementar la sincronización requerida entre los procesos. Es posible utilizar una sentencia del tipo \verb|select for i=... to ...| para especificar diferentes ramas de una sentencia selectiva que comparten el mismo código dependiente del valor de un índice \verb|i|.\\

    \begin{minted}{pascal}
        process Controlador;
        var contador : integer := 0;
            necesarios : integer := 3;
            esperando : array[0..1] of 0..5;
        begin
           while true do begin
              select
                 for i := 0 to 5
                    when receive(valor, Cliente[i]) do
                       if contador < necesarios-1 then begin
                          esperando[contador] := i;
                          contador := contador + 1;
                       else
                          contador := 0;
                          send(valor, Cliente[i]);
                          send(valor, Cliente[esperando[0]]);
                          send(valor, Cliente[esperando[1]]);
                       end
                    end
              end select
           end do
        end
    \end{minted}
\end{ejercicio}

\begin{ejercicio}\label{ej:rel3_2}
    En un sistema distribuido, 3 procesos productores producen continuamente valores enteros y los envían a un proceso buffer que los almacena temporalmente en un array local de 4 celdas enteras para ir enviándoselos a un proceso consumidor. A su vez, el proceso buffer realiza lo siguiente, sirviendo de forma equitativa al resto de procesos:
    \begin{enumerate}[label=(\alph*)]
        \item Envía enteros al proceso consumidor siempre que su array local tenga al menos dos elementos disponibles.
        \item Acepta envíos de los productores mientras el array no esté lleno, pero no acepta que cualquier productor pueda escribir dos veces consecutivas en el búfer.
    \end{enumerate}
    El código de los procesos productor y consumidor es el siguiente, asumiendo que se usan operaciones síncronas:
    \begin{figure}[H]
        \centering
        \setlength{\columnsep}{1cm}
        \begin{multicols}{2}
            \begin{minted}{pascal}
                process Productor [ i : 0..2 ];
                var dato : integer;
                begin
                   while true do begin
                      dato := Producir();
                      send(dato, Buffer);
                   end
                end
            \end{minted}
            \begin{minted}{pascal}
                process Consumidor;
                begin
                   while true do begin
                      receive(dato, Buffer);
                      Consumir(dato);
                   end
                end
            \end{minted}
        \end{multicols}
        \caption{Código para el Ejercicio~\ref{ej:rel3_2}.}
        \label{fig:cod_2}
    \end{figure}
    Describir en pseudocódigo el comportamiento del proceso \verb|Buffer|, utilizando una orden de espera selectiva que permita implementar la sincronización requerida entre los procesos.\\

    \begin{minted}{pascal}
        process Buffer;
        var buffer : array[0..3] of integer;
            primera_libre, primera_ocupada : integer := 0, 0;
            ocupadas : integer := 0;
            ult_productor : integer := -1;
            dato : integer;
        begin
           while true do begin
              select 
                 when ocupadas >= 2 do
                    dato := buffer[primera_ocupada];
                    primera_ocupada := (primera_ocupada + 1) mod 4;
                    ocupadas := ocupadas - 1;
                    send(dato, Consumidor;)
                 end
                 for i := 0 to 2
                    when ocupadas < 4 and i <> ult_productor receive(dato, Productor[i])
                       ult_productor := i;
                       buffer[primera_libre] := dato;
                       primera_libre := (primera_libre + 1) mod 4;
                       ocupadas := ocupadas + 1;
                    end
              end
           end
        end
    \end{minted}
\end{ejercicio}

\begin{ejercicio}\label{ej:rel3_3}
    Suponer un proceso productor y 3 procesos consumidores que comparten un buffer acotado de tamaño \verb|B|. Cada elemento depositado por el proceso productor debe ser retirado por todos los 3 procesos consumidores para ser eliminado del buffer. Cada consumidor retirará los datos del buffer en el mismo orden en el que son depositados, aunque los diferentes consumidores pueden ir retirando los elementos a ritmo diferente unos de otros. Por ejemplo, mientras un consumidor ha retirado los elementos 1, 2 y 3, otro consumidor puede haber retirado solamente el elemento 1. De esta forma, el consumidor más rápido podría retirar hasta B elementos más que el consumidor más lento. Describir en pseudocódigo el comportamiento de un proceso que implemente el buffer de acuerdo con el esquema de interacción descrito usando una construcción de espera selectiva, así como el del proceso productor y de los procesos consumidores. Comenzar identificando qué información es necesario representar, para después resolver las cuestiones de sincronización.
    Una posible implementación del buffer mantendría, para cada proceso consumidor, el puntero de salida y el número de elementos que quedan en el buffer por consumir:

    \begin{figure}[H]
        \centering
    \begin{tikzpicture}
        % Rectángulo principal
        \draw (0, 0) rectangle (8, 1);

        % Divisiones internas del rectángulo
        \foreach \x in {1, 2, 3, 4, 5, 6, 7} {
            \draw (\x, 0) -- (\x, 1);
        }

        % Números dentro de las cajas
        \node at (0.5, 0.5) {1};
        \node at (1.5, 0.5) {2};
        \node at (2.5, 0.5) {3};
        \node at (3.5, 0.5) {4};
        \node at (4.5, 0.5) {5};

        % Tachar el 1
        \draw[thick] (0.2, 0.6) -- (0.8, 0.4);

        % Flecha y etiquetas: out[1]
        \draw[-Stealth] (1.5, -0.6) -- (1.5, -0.1);
        \node[below] at (1.5, -0.6) {\verb|out[1]|};

        % Flecha y etiquetas: out[3]
        \draw[-Stealth] (3.5, -0.6) -- (3.5, -0.1);
        \node[below] at (3.5, -0.6) {\verb|out[3]|};

        % Flecha y etiquetas: out[2]
        \draw[-Stealth] (5.5, -0.6) -- (5.5, -0.1);
        \node[below] at (5.5, -0.6) {\verb|out[2]|};

        % Flecha y etiqueta "in" por encima
        \draw[-Stealth] (5.5, 1.6) -- (5.5, 1.1);
        \node[above] at (5.5, 1.6) {\verb|in|};

     % Texto a la derecha
        \node[right] at (8.5, 1) {\verb|nElems[1] = 4|};
        \node[right] at (8.5, 0.5) {\verb|nElems[2] = 0|};
        \node[right] at (8.5, 0) {\verb|nElems[3] = 2|};
    \end{tikzpicture}        
    \caption{Dibujo para el Ejercicio~\ref{ej:rel3_3}.}
    \label{fig:fig_ej_3}
    \end{figure}

    En primer lugar, describimos los códigos de los procesos productor y consumidores, por ser estos mucho más fáciles que el del proceso intermedio que usaremos para comunicar ambos tipos de procesos, al no requerir estos de sincronización ninguna. Tanto productor como consumidores mandan mensajes al Buffer, y los cosumidores esperan una respuesta del mismo.
    \begin{figure}[H]
        \centering
    \setlength{\columnsep}{1cm}
    \begin{multicols}{2}
        \begin{minted}{pascal}
           process Productor;
           var dato : integer;
           begin
              dato := Producir();
              send(dato, Buffer);
           end
        \end{minted}
        \begin{minted}{pascal}
            process Consumidor[ i : 0..2 ];
            var dato : integer;
            begin
               send(peticion, Buffer);
               receive(dato, Buffer);
               Consumir(dato);
            end
        \end{minted}
    \end{multicols}
    \end{figure}
    Ahora, desarrollamos el código del proceso Buffer, donde \verb|out[i]| indica la siguiente posición a leer del consumidor \verb|i|-ésimo y \verb|nElems[i]| indica la cantidad de datos que quedan por leer al consumidor \verb|i|-ésimo. 
    \begin{minted}{pascal}
        process Buffer;
        var buffer : array[0..B-1] of integer;
            ocupados : integer := 0;
            in : 0..B-1 := 0;
            out : array[0..2] of 0..B-1 := (0, 0, 0);
            nElems : array[0..2] of 0..B-1 := (0, 0, 0);
            dato : integer;
        begin
           while true do begin
              select 
                 when ocupados < B receive(dato, Productor) do
                    buffer[in] := dato;
                    in := (in + 1) mod B;
                    ocupados := ocupados + 1;
                    
                    for i:= 0 to 2 do
                       nElems[i] := nElems[i] + 1;
                 end
                 for i := 0 to 2
                    when ocupados > 0 and nElems[i] > 0 receive(peticion, Consumidor[i]) do
                       dato := buffer[out[i]];
                       out[i] := out[i] + 1 mod B;

                       { Índices de los otros consumidores }
                       j := (i+1) mod 3;
                       k := (j+1) mod 3;

                       { Si nElems es el mayor, era el último en consumir }
                       if nElems[i] > nElems[j] and nElems[i] > nElems[k] then
                          ocupados := ocupados - 1;
                       end

                       nElems[i] := nElems[i] - 1;
                       send(dato, Consumidor[i]);
                    end
              end
           end
        end
    \end{minted}
\end{ejercicio}

\begin{ejercicio}\label{ej:rel3_4}
    Una tribu de antropófagos comparte una olla en la que caben \verb|M| misioneros. Cuando algún salvaje quiere comer, se sirve directamente de la olla, a no ser que ésta esté vacía. Si la olla está vacía, el salvaje despertará al cocinero y esperará a que éste haya rellenado la olla con otros \verb|M| misioneros.
    \begin{figure}[H]
        \centering
        \setlength{\columnsep}{1cm}
        \begin{multicols}{2}
            \begin{minted}{pascal}
                process Salvaje[ i : 0..2 ];
                var peticion : integer := ... ;
                begin
                   while true do begin
                      { esperar a servirse un misionero }
                      ...
                      s_send(peticion, Olla);
                      { comer }
                      Comer();
                   end
                end
            \end{minted}
            \begin{minted}{pascal}
                process Cocinero;
                begin
                   while true do begin
                      { dormir esperando solicitud para rellenar }
                      ...
                      {confirmar que se ha rellenado la olla}
                      ...
                   end
                end
            \end{minted}
        \end{multicols}
        \caption{Código para el Ejercicio~\ref{ej:rel3_4}.}
        \label{fig:cod_4}
    \end{figure}
    Implementar los procesos salvajes y cocinero usando paso de mensajes, usando un proceso olla que incluye una construcción de espera selectiva que sirve peticiones de los salvajes y el cocinero para mantener la sincronización requerida, teniendo en cuenta que:
    \begin{itemize}
        \item La solución no debe producir interbloqueo.
        \item Los salvajes podrán comer siempre que haya comida en la olla.
        \item Solamente se despertará al cocinero cuando la olla esté vacía.
    \end{itemize}
    Mostramos primero los códigos para los procesos salvajes y cocinero:
    \begin{figure}[H]
        \centering
        \setlength{\columnsep}{1cm}
        \begin{multicols}{2}
            \begin{minted}{pascal}
                process Salvaje[ i : 0..2 ];
                var peticion : integer := 0;
                begin
                   while true do begin
                      s_send(peticion, Olla);
                      Comer();
                   end
                end
            \end{minted}
            \begin{minted}{pascal}
                process Cocinero;
                var confirmacion : integer := 0;
                begin
                   while true do begin
                      receive(peticion, Olla);
                      Rellenar();
                      send(confirmacion, Olla);
                   end
                end
            \end{minted}
        \end{multicols}
    \end{figure}
    A continuación, mostramos el código del proceso Olla:
    \begin{minted}{pascal}
        process Olla;
        var misioneros : integer := 0;
            peticion : integer := 0;
        begin
           while true do begin
              select 
                 when misioneros = 0 do
                    send(peticion, Cocinero);
                    receive(confirmacion, Cocinero);
                    misioneros := M;
                 end
                 for i := 0 to 2 
                    when misioneros > 0 receive(pet, Salvaje[i]) do
                       misioneros := misioneros - 1;
                    end
              end
           end
        end
    \end{minted}
\end{ejercicio}

\begin{ejercicio}\label{ej:rel3_5}
    Considerar un conjunto de \verb|N| procesos, \verb|P[i]|, ($i = 0, \ldots, N-1$) que se pasan mensajes cada uno al siguiente (y el primero al último), en forma de anillo. Cada proceso tiene un valor local almacenado en su variable local \verb|mi_valor|. Deseamos calcular la suma de los valores locales almacenados por los procesos de acuerdo con el algoritmo que se expone a continuación.
    \begin{figure}[H]
        \centering

        \setlength{\columnsep}{1cm}
        \begin{multicols}{2}
            
        \begin{tikzpicture}[align=center]
            % Paso 1
            \node[draw, rectangle] (a1) at (0, 0) {mi\_valor = 0\\ suma = 0};
            \node[draw, rectangle, right=of a1] (b1) {mi\_valor = 1\\ suma = 1};
            \node[draw, rectangle, below=of b1] (c1) {mi\_valor = 2\\ suma = 2};
            \node[draw, rectangle, left=of c1] (d1) {mi\_valor = 3\\ suma = 3};

            \draw[-Stealth] (a1) -- (b1) node[midway, above] {0};
            \draw[-Stealth] (b1) -- (c1) node[midway, right] {1};
            \draw[-Stealth] (c1) -- (d1) node[midway, below] {2};
            \draw[-Stealth] (d1) -- (a1) node[midway, left] {3};

            \node[above=of a1] {Paso 1};

            % Paso 3
            \node[draw, rectangle, below=2cm of d1] (a3) {suma = 5};
            \node[draw, rectangle, right=of a3] (b3) {suma = 4};
            \node[draw, rectangle, below=of b3] (c3) {suma = 3};
            \node[draw, rectangle, left=of c3] (d3) {suma = 6};

            \draw[-Stealth] (a3) -- (b3) node[midway, above] {2};
            \draw[-Stealth] (b3) -- (c3) node[midway, right] {3};
            \draw[-Stealth] (c3) -- (d3) node[midway, below] {0};
            \draw[-Stealth] (d3) -- (a3) node[midway, left] {1};

            \node[above=of a3] {Paso 3};
        \end{tikzpicture}

        \begin{tikzpicture}[align=center]

            % Paso 2
            \node[draw, rectangle] (a2) {suma = 3};
            \node[draw, rectangle, right=of a2] (b2) {suma = 1};
            \node[draw, rectangle, below=of b2] (c2) {suma = 3};
            \node[draw, rectangle, left=of c2] (d2) {suma = 5};

            \draw[-Stealth] (a2) -- (b2) node[midway, above] {3};
            \draw[-Stealth] (b2) -- (c2) node[midway, right] {0};
            \draw[-Stealth] (c2) -- (d2) node[midway, below] {1};
            \draw[-Stealth] (d2) -- (a2) node[midway, left] {2};

            \node[above=of a2] {Paso 2};

            % Paso 4
            \node[draw, rectangle, below=3cm of d2] (a4) {suma = 6};
            \node[draw, rectangle, right=of a4] (b4) {suma = 6};
            \node[draw, rectangle, below=of b4] (c4) {suma = 6};
            \node[draw, rectangle, left=of c4] (d4) {suma = 6};

            \node[above=of a4] {Paso 4};
        \end{tikzpicture}
        \end{multicols}
        \caption{Dibujo para el Ejercicio~\ref{ej:rel3_5}.}
        \label{fig:fig_ej_5}
    \end{figure}
    Los procesos realizan una serie de iteraciones para hacer circular sus valores locales por el anillo. En la primera iteración, cada proceso envía su valor local al siguiente proceso del anillo, al mismo tiempo que recibe del proceso anterior el valor local de éste. A continuación acumula la suma de su valor local y el recibido desde el proceso anterior. En las siguientes iteraciones, cada proceso envía al siguiente proceso siguiente el valor recibido en la anterior iteración, al mismo tiempo que recibe del proceso anterior un nuevo valor. Después acumula la suma. Tras un total de \verb|N - 1| iteraciones, cada proceso conocerá la suma de todos los valores locales de los procesos. Dar una descripción en pseudocódigo de los procesos siguiendo un estilo SPMD y usando operaciones de envío y recepción síncronas:
    \begin{minted}{pascal}
        process P[ i : 0..N-1 ];
        var mi_valor : integer := ...; {valor aleatorio, igual a i en la figura}
            suma : integer := mi_valor;
        begin
           for j := 0 to N-1 do begin
              ...
           end
        end
    \end{minted}
    La solución la podemos encontrar en el siguiente código, donde debemos tener cuidado de que no todos los procesos primero envíen y luego reciban, ya que esto puede dar lugar a una situación de interbloqueo por usar envíos y recepciones síncronas. Para ello, obligamos a que al menos un proceso primero reciba y luego envíe, o viceversa:
    \begin{minted}{pascal}
        process P[ i : 0..N-1 ];
        var mi_valor : integer := ...; {valor aleatorio, igual a i en la figura}
            suma : integer := mi_valor;
            siguiente : 0..N-1 := (i+1) mod N;
            anterior : 0..N-1 := (i-1) mod N;
            recibido : integer;
        begin
           for j := 0 to N-1 do begin
              if i == 0 then begin
                 send(mi_valor, siguiente);
                 recv(recibido, anterior);
              else
                 recv(recibido, anterior);
                 send(mi_valor, siguiente);
              end
              mi_valor := recibido;
              suma += recibido;
           end
        end
    \end{minted}
\end{ejercicio}

\begin{ejercicio}\label{ej:rel3_6}
    Considerar un estanco en el que hay tres fumadores y un estanquero. Cada fumador continuamente lía un cigarro y se lo fuma. Para liar un cigarro, el fumador necesita tres ingredientes: tabaco, papel y cerillas. Uno de los fumadores tiene solamente papel, otro tiene solamente tabaco, y el otro tiene solamente cerillas. El estanquero tiene una cantidad infinita de los tres ingredientes.
    \begin{itemize}
        \item El estanquero coloca aleatoriamente dos ingredientes diferentes de los tres que se necesitan para hacer un cigarro, desbloquea al fumador que tiene el tercer ingrediente y después se bloquea. El fumador seleccionado, se puede obtener fácilmente mediante una función \verb|genera_ingredientes| que devuelve el índice (0, 1, ó 2) del fumador escogido.
        \item El fumador desbloqueado toma los dos ingredientes del mostrador, desbloqueando al estanquero, lía un cigarro y fuma durante un tiempo.
        \item El estanquero, una vez desbloqueado, vuelve a poner dos ingredientes aleatorios en el mostrador, y se repite el ciclo.
    \end{itemize}
    Describir una solución distribuida que use envío asíncrono seguro y recepción síncrona, para este problema usando un proceso Estanquero y tres procesos fumadores \verb|Fumador(i)| (con $i=0, 1, 2$).

    \begin{figure}[H]
        \centering
            \begin{minted}{pascal}
                process Estanquero;
                var ing1, ing2 : ingredientes;
                begin
                   while true do begin
                      ing1, ing2 := generaIngredientes();

                      select
                         for i:=0 to 2
                            when genera_ingredientes(ing1, ing2) receive(peticion, Fumador[i])
                            do
                               send(ing1, ing2, Fumador[i])
                            end
                      end select
                   end
                end
            \end{minted}
            \begin{minted}{pascal}
                process Fumador[ i : 0..2 ];
                begin
                   while true do begin
                      send(peticion, Estanquero);
                      recv(ingredientes, Estanquero);
                      fumar(ingredientes);
                   end
                end
            \end{minted}
        \caption{Código para el Ejercicio~\ref{ej:rel3_6}.}
        \label{fig:cod_6}
    \end{figure}
\end{ejercicio}

\begin{ejercicio}\label{ej:rel3_7}
    En un sistema distribuido, un gran número de procesos clientes usa frecuentemente un determinado recurso y se desea que puedan usarlo simultáneamente el máximo número de procesos. Para ello, los clientes envían peticiones a un proceso controlador para usar el recurso y esperan respuesta para poder usarlo (véase el código de los procesos clientes). Cuando un cliente termina de usar el recurso, envía una solicitud para dejar de usarlo y espera respuesta del Controlador. El proceso controlador se encarga de asegurar la sincronización adecuada imponiendo una única restricción por razones supersticiosas: nunca habrá 13 procesos exactamente usando el recurso al mismo tiempo.
    \begin{figure}[H]
        \centering
        \setlength{\columnsep}{1cm}
        \begin{multicols}{2}
            \begin{minted}{pascal}
                process Cli[ i : 0..n ];
                var pet_usar : integer := +1;
                    pet_liberar : integer := -1;
                    permiso : integer := ...;
                begin
                   while true do begin
                      send(pet_usar, Controlador);
                      receive(permiso, Controlador);
                      Usar_recurso();
                      send(pet_liberar, Controlador);
                      receive(permiso, Controlador);
                   end
                end
            \end{minted}
            \begin{minted}{pascal}
                process Controlador;
                begin
                   while true do begin
                      select
                         ...
                      end
                   end
                end
            \end{minted}
        \end{multicols}
        \caption{Código para el Ejercicio~\ref{ej:rel3_7}.}
        \label{fig:cod_7}
    \end{figure}
    Describir en pseudocódigo el comportamiento del proceso controlador, utilizando una orden de espera selectiva que permita implementar la sincronización requerida entre los procesos. Es posible utilizar una sentencia del tipo \verb|select for i=... to ...| para especificar diferentes ramas de una sentencia selectiva que comparten el mismo código dependiente del valor de un índice \verb|i|.\\

    La solución que planteamos es la siguiente, donde tenemos en cuenta las peticiones de obtención y liberación del recurso que llevan al recurso a ser usado por 13 procesos al mismo tiempo. En dicho caso, guardamos la petición hasta que haya cualquiera otra petición (que puede ser del mismo tipo o distinto).
    \begin{minted}{pascal}
        process Controlador
        var usando : integer := 0;
            esperando : integer := 0;
            cola : array[0..n-13] of 0..n;
            ultimo : integer := 0;
            siguiente : integer := 0;
            permiso : integer := 1;
            peticion : integer := ...;
        begin
           while true do begin
              select 
                 for i := 0 to n-1
                     when receive(peticion,Cli[i]) do
                        if peticion == +1 then begin
                           if usando == 13 then begin
                              cola[ultimo] := i;
                              ultimo := (ultimo + 1) mod (n-13);
                              esperando := esperando + 1;
                           else
                              usando := usando + 1;
                              send(permiso, Cli[i]);
                           end
                        else if peticion == -1 then begin
                           usando := usando - 1;
                           send(permiso, Cli[i]);

                           if esperando > 0 then begin
                              esperando := esperando - 1;
                              send(permiso, Cli[cola[siguiente]]);
                              siguiente := (siguiente + 1) mod (n-13);
                              usando := usando + 1;
                           end
                        end
              end select
           end
        end
    \end{minted}
\end{ejercicio}

\begin{ejercicio}\label{ej:rel3_8}
    En un sistema distribuido, tres procesos \verb|Productor| se comunican con un proceso \verb|Impresor| que se encarga de ir imprimiendo en pantalla una cadena con los datos generados por los procesos productores. Cada proceso productor (\verb|Productor[i]|, con $i=0,1,2$) genera continuamente el correspondiente entero \verb|i|, y lo envía al proceso Impresor.

    El proceso Impresor se encarga de ir recibiendo los datos generados por los productores y los imprime por pantalla (usando el procedimiento \verb|imprime(entero)|) generando una cadena dígitos en la salida. No obstante, los procesos se han de sincronizar adecuadamente para que la impresión por pantalla cumpla las siguientes restricciones:
    \begin{itemize}
        \item Los dígitos 0 y 1 deben aceptarse por el impresor de forma alterna. Es decir, si se acepta un 0 no podrá volver a aceptarse un 0 hasta que se haya aceptado un 1, y viceversa, si se acepta un 1 no podrá volver a aceptarse un 1 hasta que se haya aceptado un 0.
        \item El número total de dígitos 0 o 1 aceptados en un instante no puede superar el doble de número de digitos 2 ya aceptados en dicho instante.
    \end{itemize}
    Cuando un productor envía un digito que no se puede aceptar por el imprersor, el productor quedará bloqueado esperando completar el \verb|s_send|. El pseudocódigo de los procesos productores (\verb|Productor|) se muestra a continuación , asumiendo que se usan operaciones bloqueantes no buferizadas (síncronas):
    \begin{figure}[H]
        \centering
            \begin{minted}{pascal}
                process Productor[ i : 0,1,2 ];
                begin
                   while true do begin
                      s_send(i, Impresor);
                   end
                end
            \end{minted}
        \caption{Código para el Ejercicio~\ref{ej:rel3_8}.}
        \label{fig:cod_8}
    \end{figure}
    Escribir en pseudocódigo el código del proceso \verb|Impresor|, utilizando para ello un bucle infinito con una orden de espera selectiva \verb|select| que permita implementar la sincronización requerida entre los procesos, según el esquema anterior.\\

    El código solicitado es el siguiente, donde \verb|ult0| indica:
    \begin{itemize}
        \item \verb|true| si el último dígito 0 o 1 recibido fue un 0.
        \item \verb|false| si el último dígito 0 o 1 recibido fue un 1.
    \end{itemize}
    Además, notemos que el ejercicio no impone restricciones sobre cuando se pueden recibir dígitos 2.
    \begin{minted}{pascal}
        process Impresor
        var cant0, cant1, cant2 : integer := 0, 0, 0;
            ult0 : boolean := true;
            n : 0..2;
        begin
           while true do begin
              select
                 when (cant0 < 2*cant2-1 and not ult0) receive(n, Productor[0]) do
                    cant0 := cant0 + 1;
                    ult0 := true;
                 end

                 when (cant1 < 2*cant2-1 and ult0) receive(n, Productor[1]) do
                    cant1 := cant1 + 1;
                    ult0 := false;
                 end

                 when receive(n, Productor[2]) do
                    cant2 := cant2 + 1;
                 end
              end
           end
        end
    \end{minted}
\end{ejercicio}

\begin{ejercicio}\label{ej:rel3_9}
   En un sistema distribuido hay un vector de \verb|n| procesos iguales que envían con send (en un bucle infinito) valores enteros a un proceso receptor, que los imprime. Si en algún momento no hay ningún mensaje pendiente de recibir en el receptor, este proceso debe de imprimir ``no hay mensajes, duermo''; después de bloquearse durante 10 segundos (con \verb|sleep_for(10)|), antes de volver a comprobar si hay mensajes (esto podría hacerse para ahorrar energía, ya que el procesamiento de mensajes se hace en ráfagas separadas por 10 segundos). Este problema no se puede solucionar usando \verb|receive| o \verb|i_receive|. Indica a que se debe esto. Sin embargo, sí se puede hacer con \verb|select|. Diseña una solución a este problema con \verb|select|:
    \begin{figure}[H]
       \centering
           \begin{minted}{pascal}
               process Emisor[ i : 1..n ];
               var dato : integer;
               begin
                  while true do begin
                     dato := Producir();
                     send(dato, Receptor);
                  end
               end
           \end{minted}
       \caption{Código para el Ejercicio~\ref{ej:rel3_9}.}
       \label{fig:cod_9}
   \end{figure}
   El problema no puede resolverse con instrucciones \verb|receive| o \verb|i_receive| porque:
   \begin{itemize}
       \item En el caso de la instrucción \verb|receive|, si no hay mensajes pendientes, el proeso se bloquearía hasta recepcionar el primero, pero este no es el comportamiento deseado.
       \item En el caso de la instrucción \verb|i_receive|, comenzaría instantáneamente la recepción del mensaje, pero el proceso volvería inmediatamente antes de recibirlo, por lo que no sabríamos si hay o no un mensaje pendiente.
   \end{itemize}
   Sin embargo, podemos hacer uso de las sentencias \verb|else| de las instrucciones \verb|select|, de forma que una instrucción se ejecute cuando todas las guardas no son ejecutables:
   \begin{minted}[escapeinside=\#\#]{pascal}
       process Receptor
       var dato : integer;
       begin
          while true do begin
             select
                for i:= 1 to n
                   when receive(dato, Emisor[i]) do
                      imprime(dato);
                   end
                else begin
                   imprime(#"#No hay mensajes, duermo#"#);
                   sleep_for(10);
                end
             end select
          end
       end
   \end{minted}
\end{ejercicio}

\begin{ejercicio}\label{ej:rel3_10}
    En un sistema tenemos \verb|N| procesos emisores que envían de forma segura un único mensaje cada uno de ellos a un proceso receptor, mensaje que contiene un entero con el número de proceso emisor. El proceso receptor debe de imprimir el número del proceso emisor que inició el envío en primer lugar. Dicho emisor debe terminar, y el resto quedarse bloqueados:
    \begin{figure}[H]
        \centering
        \begin{minted}[escapeinside=\#\#]{pascal}
            process Emisor[ i : 1..N ];
            begin
               s_send(i, Receptor);
            end

            process Receptor;
            var ganador : integer;
            begin
               { calcular ganador }
               ...
               print #"#El primer envio #lo# ha realizado: #"#, ganador;
            end
        \end{minted}
        \caption{Código para el Ejercicio~\ref{ej:rel3_10}.}
        \label{fig:cod_10}
    \end{figure}
    Para cada uno de los siguientes casos, describir razonadamente si es posible diseñar una solución a este problema o no lo es. En caso afirmativo, escribe una posible solución:
    \begin{enumerate}[label=(\alph*)]
        \item el proceso receptor usa exclusivamente recepción mediante una o varias llamadas a \verb|receive|.
        \item el proceso receptor usa exclusivamente recepción mediante una o varias llamadas a \verb|i_receive|.
        \item el proceso receptor usa exclusivamente recepción mediante una o varias instrucciones \verb|select|.
    \end{enumerate}
    Distinguimos casos:
    \begin{enumerate}[label=(\alph*)]
        \item No es posible, ya que si estamos pensando en usar una única instrucción \verb|receive| de forma que el proceso ganador sea aquel que realice una cita con esta instrucción, entonces no nos quedaríamos con el proceso emisor que inició el envío en primer lugar, sino con el emisor del mensaje que primero llegó en el receptor.
        \item Tampoco es posible, porque ahora mantenemos el problema anterior pero además el orden en el que se reciben los mensajes en el receptor no tiene por qué coincidir con el orden con el que se realizan las instrucciones \verb|i_receive|.
        \item Sí que es posible, ya que en caso de que haya más de un mensaje iniciado y preparado para ser recibido, la orden \verb|select| escogerá aquel mensaje cuyo emisor comenzó antes la operación de envío:
            \begin{minted}[escapeinside=\#\#]{pascal}
                process Receptor
                var ganador : integer;
                begin
                   select
                      for i := 1 to N; when receive(ganador, Emisor[i]) do
                         null;
                      end
                   end select
                   print #"#El primer envio #lo# ha realizado: #"#, ganador;
                end
            \end{minted}
    \end{enumerate}
\end{ejercicio}

\begin{ejercicio}\label{ej:rel3_11}
    Supongamos que tenemos \verb|N| procesos concurrentes semejantes:

    Cada proceso produce \verb|N-1| caracteres (con \verb|N-1| llamadas a la función \verb|ProduceCaracter|) y envía cada carácter a los otros \verb|N-1| procesos. Además, cada proceso debe imprimir todos los caracteres recibidos de los otros procesos (el orden en el que se escriben es indiferente).
    \begin{itemize}
        \item Describe razonadamente si es o no posible hacer esto usando exclusivamente \verb|s_send| para los envíos. En caso afirmativo, escribe una solución.
        \item Escribe una solución usando \verb|send| y \verb|receive|.
    \end{itemize}
    Distinguimos casos:
    \begin{itemize}
        \item En el primer caso, es imposible implementar esta funcionalidad usando operaciones \verb|s_send| de envío síncrono, ya que todos los procesos ejecutarían dicha instrucción, llevando a un interbloqueo de todos los procesos, situación que se pone de manifiesto de forma simple si simplemente consideramos dos procesos:
            \begin{figure}[H]
                \setlength{\columnsep}{1cm}
                \begin{multicols}{2}
                    \begin{minted}{pascal}
                       process P1;    
                       var n1 : integer := 1;
                       begin
                          send(n1, P2);
                       end
                    \end{minted}
                    \begin{minted}{pascal}
                       process P2;    
                       var n2 : integer := 2;
                       begin
                          send(n2, P1);
                       end
                    \end{minted}
                \end{multicols}
                \caption{Situación típica de interbloqueo.}
            \end{figure}
        \item Con instrucciones \verb|send| y \verb|receive| sí que se puede resolver de forma sencilla:
            \begin{minted}{pascal}
                process P[ i : 0..N ];
                var n : char;
                begin
                   { Enviamos todos los caracteres }
                   for j := 0 to N do
                      if i <> j then begin
                         n := Producir();
                         send(n, P[j]);
                      end
                   end do
                   { Recibimos todos los caracteres }
                   for j := 0 to N do
                      if i <> j then begin
                         receive(n, P[j]);
                         print(n);
                      end
                   end do
                end
            \end{minted}
    \end{itemize}
\end{ejercicio}

\begin{ejercicio}\label{ej:rel3_12}
    Escribe una nueva solución al Ejercicio~\ref{ej:rel3_11} en la cual se garantize que el orden en el que se imprimen los caracteres es el mismo orden en el que se inician los envíos de dichos caracteres (pista: usa \verb|select| para recibir).\\

    Como bien indica la pista, basta con usar una instrucción \verb|select| para realizar la recepción de los caracteres:
            \begin{minted}{pascal}
                process P[ i : 0..N ];
                var n : char;
                begin
                   { Enviamos todos los caracteres }
                   for j := 0 to N do
                      if i <> j then begin
                         n := Producir();
                         send(n, P[j]);
                      end
                   end do

                   { Recibimos todos los caracteres }
                   for j := 1 to N do
                      select
                         for k := 0 to N
                            when k <> i receive(n, P[k])
                               print(n);
                            end
                      end select
                   end do
            \end{minted}
\end{ejercicio}

\begin{ejercicio}\label{ej:rel3_13}
    Supongamos de nuevo el problema anterior en el cual todos los procesos envían a todos. Ahora cada item de datos a producir y transmitir es un bloque de bytes con muchos valores (por ejemplo, es una imagen que puede tener varios megabytes de tamaño). Se dispone del tipo de datos \verb|TipoBloque| para ello, y el procedimiento \verb|ProducirBloque|, de forma que si b es una variable de tipo \verb|TipoBloque|, entonces la llamada a \verb|ProducirBloque(b)| produce y escribe una secuencia de bytes en \verb|b|. En lugar de imprimir los datos, se deben consumir con una llamada a \verb|ConsumirBloque(b)|.

    Cada proceso se ejecuta en un ordenador, y se garantiza que hay la suficiente memoria en ese ordenador como para contener simultáneamente, al menos, hasta N bloques. Sin embargo, el sistema de paso de mensajes (SPM) podría no tener memoria suficiente como para contener los ${(N-1)}^{2}$ mensajes en tránsito simultáneos que podría llegar a haber en un momento dado con la solución anterior.

    En estas condiciones, si el SPM agota la memoria, debe retrasar los \verb|send| dejando bloqueados los procesos y, en esas circunstancias, se podría producir interbloqueo. Para evitarlo, se pueden usar operaciones inseguras de envío, \verb|i_send|. Escribe dicha solución, usando como orden de recepción el mismo que en el problema anterior.\\

    \noindent
    Para resolver el problema, utilizaremos un array de $N-1$ bloques a enviar (como vamos a usar la operación \verb|i_send| es un envío inseguro, luego optamos por no modificar los los bloques tras enviarlos) y de un bloque para recibir. Además, antes de terminar un proceso tendremos que esperar a que este proceso ha terminado el envío de todos sus bloques:
    \begin{minted}{pascal}
        process P[ i : 0..N ];
        var bloque : array[0..N] of TipoBloque;
            estado : array[0..N] of Estado;
        begin
           { 1. Realizamos todos los envíos }
           for j := 0 to N do
              if i <> j then begin
                 ProducirBloque(bloque[j]);
                 i_send(bloque[j], P[j], estado[j]);
              end
           end

           { 2. Procesamos las recepciones }
           for j := 0 to N
              if i <> j then begin
                 receive(bloque[i], P[j]);
                 ConsumirBloque(bloque[i]);
              end
           end

           { 3. Esperamos en caso de que no se hayan realizado }
           { todos los envíos antes de terminar }
           for j := 0 to N
              if i <> j then begin
                 wait_send(estado[j]);
              end
           end
        end
    \end{minted}
\end{ejercicio}

\begin{ejercicio}\label{ej:rel3_14}
    En los tres problemas anteriores, cada proceso va esperando a recibir un item de datos de cada uno de los otros procesos, consume dicho item, y después pasa a recibir del siguiente emisor (en distintos órdenes). Esto implica que un envío ya iniciado, pero pendiente, no puede completarse hasta que el receptor no haya consumido los anteriores bloques, es decir, se podría estar consumiendo mucha memoria en el SPM por mensajes en tránsito pendientes cuya recepción se ve retrasada. Escribe una solución en la cual cada proceso inicia sus envíos y recepciones y después espera a que se completen todas las recepciones antes de iniciar el primer consumo de un bloque recibido. De esta forma todos los mensajes pueden transferirse potencialmente de forma simultánea. Se debe intentar que la transimisión y las producción de bloques sean lo más simultáneas posible. Suponer que cada proceso puede almacenar como mínimo $2\cdot N$ bloques en su memoria local, y que el orden de recepción o de consumo de los bloques es indiferente.\\

    La solución a este último problema es similar a la del Ejercicio~\ref{ej:rel3_13}, pero en este caso debemos usar la instrucción \verb|i_receive| para recibir, así como un array entero de bloques para realizar dicha recepción. El código sería el siguiente:
    \begin{minted}{pascal}
        process P[ i : 0..N ];
        var bloque_env : array[0..N] of TipoBloque;
            bloque_rec : array[0..N] of TipoBloque;
            estado_env : array[0..N] of Estado;
            estado_rec : array[0..N] of Estado;
        begin
           { Inicializamos las recepciones }
           for j := 0 to N do
              if j <> i then begin
                 i_receive(bloque_rec[j], P[j], estado_rec[j]);
              end
           end

           { Inicializamos los envíos }
           for j := 0 to N do
              if j <> i then begin
                 ProducirBloque(j);
                 i_send(bloque_env[j], P[j], estado_env[j]);
              end
           end

           { Esperar a que terminen todas las recepciones }
           for j := 0 to N do
              if j <> i then begin
                 wait_recv(estado_rec[j]);
              end
           end

           { Procesar todos los bloques }
           for j := 0 to N do
              if j <> i then begin
                 ConsumirBloque(bloque[j]);
              end
           end

           { Esperar a que terminen todos los envíos antes de terminar el proceso }
           for j := 0 to N do
              if j <> i then begin
                 wait_send(estado_env[j]);
              end
           end
        end
    \end{minted}
\end{ejercicio}

