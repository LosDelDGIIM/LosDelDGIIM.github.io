\documentclass[12pt]{book}

% Idioma y codificación
\usepackage[spanish, es-tabla]{babel}       %es-tabla para que se titule "Tabla"
\usepackage[utf8]{inputenc}

% Márgenes
\usepackage[a4paper,top=3cm,bottom=2.5cm,left=3cm,right=3cm]{geometry}

% Comentarios de bloque
\usepackage{verbatim}

% Paquetes de links
\usepackage[hidelinks]{hyperref}    % Permite enlaces
\usepackage{url}                    % redirecciona a la web

% Más opciones para enumeraciones
\usepackage{enumitem}

% Personalizar la portada
\usepackage{titling}

% Paquetes de tablas
\usepackage{multirow}


%------------------------------------------------------------------------

%Paquetes de figuras
\usepackage{caption}
\usepackage{subcaption} % Figuras al lado de otras
\usepackage{float}      % Poner figuras en el sitio indicado H.


% Paquetes de imágenes
\usepackage{graphicx}       % Paquete para añadir imágenes
\usepackage{transparent}    % Para manejar la opacidad de las figuras

% Paquete para usar colores
\usepackage[dvipsnames]{xcolor}
\usepackage{pagecolor}      % Para cambiar el color de la página

% Habilita tamaños de fuente mayores
\usepackage{fix-cm}

% Para los gráficos
\usepackage{tikz}

% Para poder situar los nodos en los grafos
\usetikzlibrary{positioning}


%------------------------------------------------------------------------

% Paquetes de matemáticas
\usepackage{mathtools, amsfonts, amssymb, mathrsfs}
\usepackage[makeroom]{cancel}     % Simplificar tachando
\usepackage{polynom}    % Divisiones y Ruffini
\usepackage{units} % Para poner fracciones diagonales con \nicefrac

\usepackage{pgfplots}   %Representar funciones
\pgfplotsset{compat=1.18}  % Versión 1.18

\usepackage{tikz-cd}    % Para usar diagramas de composiciones
\usetikzlibrary{calc}   % Para usar cálculo de coordenadas en tikz

%Definición de teoremas, etc.
\usepackage{amsthm}
%\swapnumbers   % Intercambia la posición del texto y de la numeración

\theoremstyle{plain}

\makeatletter
\@ifclassloaded{article}{
  \newtheorem{teo}{Teorema}[section]
}{
  \newtheorem{teo}{Teorema}[chapter]  % Se resetea en cada chapter
}
\makeatother

\newtheorem{coro}{Corolario}[teo]           % Se resetea en cada teorema
\newtheorem{prop}[teo]{Proposición}         % Usa el mismo contador que teorema
\newtheorem{lema}[teo]{Lema}                % Usa el mismo contador que teorema

\theoremstyle{remark}
\newtheorem*{observacion}{Observación}

\theoremstyle{definition}

\makeatletter
\@ifclassloaded{article}{
  \newtheorem{definicion}{Definición} [section]     % Se resetea en cada chapter
}{
  \newtheorem{definicion}{Definición} [chapter]     % Se resetea en cada chapter
}
\makeatother

\newtheorem*{notacion}{Notación}
\newtheorem*{ejemplo}{Ejemplo}
\newtheorem*{ejercicio*}{Ejercicio}             % No numerado
\newtheorem{ejercicio}{Ejercicio} [section]     % Se resetea en cada section


% Modificar el formato de la numeración del teorema "ejercicio"
\renewcommand{\theejercicio}{%
  \ifnum\value{section}=0 % Si no se ha iniciado ninguna sección
    \arabic{ejercicio}% Solo mostrar el número de ejercicio
  \else
    \thesection.\arabic{ejercicio}% Mostrar número de sección y número de ejercicio
  \fi
}


% \renewcommand\qedsymbol{$\blacksquare$}         % Cambiar símbolo QED
%------------------------------------------------------------------------

% Paquetes para encabezados
\usepackage{fancyhdr}
\pagestyle{fancy}
\fancyhf{}

\newcommand{\helv}{ % Modificación tamaño de letra
\fontfamily{}\fontsize{12}{12}\selectfont}
\setlength{\headheight}{15pt} % Amplía el tamaño del índice


%\usepackage{lastpage}   % Referenciar última pag   \pageref{LastPage}
\fancyfoot[C]{\thepage}

%------------------------------------------------------------------------

% Conseguir que no ponga "Capítulo 1". Sino solo "1."
\makeatletter
\@ifclassloaded{book}{
  \renewcommand{\chaptermark}[1]{\markboth{\thechapter.\ #1}{}} % En el encabezado
    
  \renewcommand{\@makechapterhead}[1]{%
  \vspace*{50\p@}%
  {\parindent \z@ \raggedright \normalfont
    \ifnum \c@secnumdepth >\m@ne
      \huge\bfseries \thechapter.\hspace{1em}\ignorespaces
    \fi
    \interlinepenalty\@M
    \Huge \bfseries #1\par\nobreak
    \vskip 40\p@
  }}
}
\makeatother

%------------------------------------------------------------------------
% Paquetes de cógido
\usepackage{minted}
\renewcommand\listingscaption{Código fuente}

\usepackage{fancyvrb}
% Personaliza el tamaño de los números de línea
\renewcommand{\theFancyVerbLine}{\small\arabic{FancyVerbLine}}

% Estilo para C++
\newminted{cpp}{
    frame=lines,
    framesep=2mm,
    baselinestretch=1.2,
    linenos,
    escapeinside=||
}

% para minted
\definecolor{LightGray}{rgb}{0.95,0.95,0.92}
\setminted{
    linenos=true,
    stepnumber=5,
    numberfirstline=true,
    autogobble,
    breaklines=true,
    breakautoindent=true,
    breaksymbolleft=,
    breaksymbolright=,
    breaksymbolindentleft=0pt,
    breaksymbolindentright=0pt,
    breaksymbolsepleft=0pt,
    breaksymbolsepright=0pt,
    fontsize=\footnotesize,
    bgcolor=LightGray,
    numbersep=10pt
}


\usepackage{listings} % Para incluir código desde un archivo

\renewcommand\lstlistingname{Código Fuente}
\renewcommand\lstlistlistingname{Índice de Códigos Fuente}

% Definir colores
\definecolor{vscodepurple}{rgb}{0.5,0,0.5}
\definecolor{vscodeblue}{rgb}{0,0,0.8}
\definecolor{vscodegreen}{rgb}{0,0.5,0}
\definecolor{vscodegray}{rgb}{0.5,0.5,0.5}
\definecolor{vscodebackground}{rgb}{0.97,0.97,0.97}
\definecolor{vscodelightgray}{rgb}{0.9,0.9,0.9}

% Configuración para el estilo de C similar a VSCode
\lstdefinestyle{vscode_C}{
  backgroundcolor=\color{vscodebackground},
  commentstyle=\color{vscodegreen},
  keywordstyle=\color{vscodeblue},
  numberstyle=\tiny\color{vscodegray},
  stringstyle=\color{vscodepurple},
  basicstyle=\scriptsize\ttfamily,
  breakatwhitespace=false,
  breaklines=true,
  captionpos=b,
  keepspaces=true,
  numbers=left,
  numbersep=5pt,
  showspaces=false,
  showstringspaces=false,
  showtabs=false,
  tabsize=2,
  frame=tb,
  framerule=0pt,
  aboveskip=10pt,
  belowskip=10pt,
  xleftmargin=10pt,
  xrightmargin=10pt,
  framexleftmargin=10pt,
  framexrightmargin=10pt,
  framesep=0pt,
  rulecolor=\color{vscodelightgray},
  backgroundcolor=\color{vscodebackground},
}

%------------------------------------------------------------------------

% Comandos definidos
\newcommand{\bb}[1]{\mathbb{#1}}
\newcommand{\cc}[1]{\mathcal{#1}}

% I prefer the slanted \leq
\let\oldleq\leq % save them in case they're every wanted
\let\oldgeq\geq
\renewcommand{\leq}{\leqslant}
\renewcommand{\geq}{\geqslant}

% Si y solo si
\newcommand{\sii}{\iff}

% Letras griegas
\newcommand{\eps}{\epsilon}
\newcommand{\veps}{\varepsilon}
\newcommand{\lm}{\lambda}

\newcommand{\ol}{\overline}
\newcommand{\ul}{\underline}
\newcommand{\wt}{\widetilde}
\newcommand{\wh}{\widehat}

\let\oldvec\vec
\renewcommand{\vec}{\overrightarrow}

% Derivadas parciales
\newcommand{\del}[2]{\frac{\partial #1}{\partial #2}}
\newcommand{\Del}[3]{\frac{\partial^{#1} #2}{\partial #3^{#1}}}
\newcommand{\deld}[2]{\dfrac{\partial #1}{\partial #2}}
\newcommand{\Deld}[3]{\dfrac{\partial^{#1} #2}{\partial #3^{#1}}}


\newcommand{\AstIg}{\stackrel{(\ast)}{=}}
\newcommand{\Hop}{\stackrel{L'H\hat{o}pital}{=}}

\newcommand{\red}[1]{{\color{red}#1}} % Para integrales, destacar los cambios.

% Método de integración
\newcommand{\MetInt}[2]{
    \left[\begin{array}{c}
        #1 \\ #2
    \end{array}\right]
}

% Declarar aplicaciones
% 1. Nombre aplicación
% 2. Dominio
% 3. Codominio
% 4. Variable
% 5. Imagen de la variable
\newcommand{\Func}[5]{
    \begin{equation*}
        \begin{array}{rrll}
            #1:& #2 & \longrightarrow & #3\\
               & #4 & \longmapsto & #5
        \end{array}
    \end{equation*}
}

%------------------------------------------------------------------------

\usetikzlibrary{positioning}

\usepackage{multicol}

\renewcommand{\theFancyVerbLine}{\sffamily
\textcolor[rgb]{0.5,0.5,1.0}{\scriptsize
\oldstylenums{\arabic{FancyVerbLine}}}}

\definecolor{LightGray}{rgb}{0.95,0.95,0.92}
\setminted{
    linenos=true,
    stepnumber=5,
    numberfirstline=true,
    autogobble,
    breaklines=true,
    breakautoindent=true,
    breaksymbolleft=,
    breaksymbolright=,
    breaksymbolindentleft=0pt,
    breaksymbolindentright=0pt,
    breaksymbolsepleft=0pt,
    breaksymbolsepright=0pt,
    fontsize=\footnotesize,
    bgcolor=LightGray,
    numbersep=10pt
}


\begin{document}

    % 1. Foto de fondo
    % 2. Título
    % 3. Encabezado Izquierdo
    % 4. Color de fondo
    % 5. Coord x del titulo
    % 6. Coord y del titulo
    % 7. Fecha
    % 8. Autor

    
    % 1. Foto de fondo
% 2. Título
% 3. Encabezado Izquierdo
% 4. Color de fondo
% 5. Coord x del titulo
% 6. Coord y del titulo
% 7. Fecha

\newcommand{\portada}[7]{

    \portadaBase{#1}{#2}{#3}{#4}{#5}{#6}{#7}
    \portadaBook{#1}{#2}{#3}{#4}{#5}{#6}{#7}
}

\newcommand{\portadaExamen}[7]{

    \portadaBase{#1}{#2}{#3}{#4}{#5}{#6}{#7}
    \portadaArticle{#1}{#2}{#3}{#4}{#5}{#6}{#7}
}




\newcommand{\portadaBase}[7]{

    % Tiene la portada principal y la licencia Creative Commons
    
    % 1. Foto de fondo
    % 2. Título
    % 3. Encabezado Izquierdo
    % 4. Color de fondo
    % 5. Coord x del titulo
    % 6. Coord y del titulo
    % 7. Fecha
    
    
    \thispagestyle{empty}               % Sin encabezado ni pie de página
    \newgeometry{margin=0cm}        % Márgenes nulos para la primera página
    
    
    % Encabezado
    \fancyhead[L]{\helv #3}
    \fancyhead[R]{\helv \nouppercase{\leftmark}}
    
    
    \pagecolor{#4}        % Color de fondo para la portada
    
    \begin{figure}[p]
        \centering
        \transparent{0.3}           % Opacidad del 30% para la imagen
        
        \includegraphics[width=\paperwidth, keepaspectratio]{assets/#1}
    
        \begin{tikzpicture}[remember picture, overlay]
            \node[anchor=north west, text=white, opacity=1, font=\fontsize{60}{90}\selectfont\bfseries\sffamily, align=left] at (#5, #6) {#2};
            
            \node[anchor=south east, text=white, opacity=1, font=\fontsize{12}{18}\selectfont\sffamily, align=right] at (9.7, 3) {\textbf{\href{https://losdeldgiim.github.io/}{Los Del DGIIM}}};
            
            \node[anchor=south east, text=white, opacity=1, font=\fontsize{12}{15}\selectfont\sffamily, align=right] at (9.7, 1.8) {Doble Grado en Ingeniería Informática y Matemáticas\\Universidad de Granada};
        \end{tikzpicture}
    \end{figure}
    
    
    \restoregeometry        % Restaurar márgenes normales para las páginas subsiguientes
    \pagecolor{white}       % Restaurar el color de página
    
    
    \newpage
    \thispagestyle{empty}               % Sin encabezado ni pie de página
    \begin{tikzpicture}[remember picture, overlay]
        \node[anchor=south west, inner sep=3cm] at (current page.south west) {
            \begin{minipage}{0.5\paperwidth}
                \href{https://creativecommons.org/licenses/by-nc-nd/4.0/}{
                    \includegraphics[height=2cm]{assets/Licencia.png}
                }\vspace{1cm}\\
                Esta obra está bajo una
                \href{https://creativecommons.org/licenses/by-nc-nd/4.0/}{
                    Licencia Creative Commons Atribución-NoComercial-SinDerivadas 4.0 Internacional (CC BY-NC-ND 4.0).
                }\\
    
                Eres libre de compartir y redistribuir el contenido de esta obra en cualquier medio o formato, siempre y cuando des el crédito adecuado a los autores originales y no persigas fines comerciales. 
            \end{minipage}
        };
    \end{tikzpicture}
    
    
    
    % 1. Foto de fondo
    % 2. Título
    % 3. Encabezado Izquierdo
    % 4. Color de fondo
    % 5. Coord x del titulo
    % 6. Coord y del titulo
    % 7. Fecha


}


\newcommand{\portadaBook}[7]{

    % 1. Foto de fondo
    % 2. Título
    % 3. Encabezado Izquierdo
    % 4. Color de fondo
    % 5. Coord x del titulo
    % 6. Coord y del titulo
    % 7. Fecha

    % Personaliza el formato del título
    \pretitle{\begin{center}\bfseries\fontsize{42}{56}\selectfont}
    \posttitle{\par\end{center}\vspace{2em}}
    
    % Personaliza el formato del autor
    \preauthor{\begin{center}\Large}
    \postauthor{\par\end{center}\vfill}
    
    % Personaliza el formato de la fecha
    \predate{\begin{center}\huge}
    \postdate{\par\end{center}\vspace{2em}}
    
    \title{#2}
    \author{\href{https://losdeldgiim.github.io/}{Los Del DGIIM}}
    \date{Granada, #7}
    \maketitle
    
    \tableofcontents
}




\newcommand{\portadaArticle}[7]{

    % 1. Foto de fondo
    % 2. Título
    % 3. Encabezado Izquierdo
    % 4. Color de fondo
    % 5. Coord x del titulo
    % 6. Coord y del titulo
    % 7. Fecha

    % Personaliza el formato del título
    \pretitle{\begin{center}\bfseries\fontsize{42}{56}\selectfont}
    \posttitle{\par\end{center}\vspace{2em}}
    
    % Personaliza el formato del autor
    \preauthor{\begin{center}\Large}
    \postauthor{\par\end{center}\vspace{3em}}
    
    % Personaliza el formato de la fecha
    \predate{\begin{center}\huge}
    \postdate{\par\end{center}\vspace{5em}}
    
    \title{#2}
    \author{\href{https://losdeldgiim.github.io/}{Los Del DGIIM}}
    \date{Granada, #7}
    \thispagestyle{empty}               % Sin encabezado ni pie de página
    \maketitle
    \vfill
}
    \portada{etsiitA4.jpg}{Sistemas\\Concurrentes y\\Distribuidos}{SCD}{MidnightBlue}{-8.5}{28.3}{2024-2025}{José Juan Urrutia Milán}

    \chapter{Introducción a los fundamentos de redes}
\subsection{Objetivos}

\subsection{Historia}
Un servicio de banda ancha es un servicio de velocidad grande, que se inición a partir del 2000. Comenzaron por 2Mbps.
El ADSL en España comenzó transmitiendo 256kbps (el máximo teórico del ADSL son 20Mbps, aunque lo normal son 10 o 12).
Las redes de cable eran HFC (redes de cable y fibras híbridas), pero ahora tenemos FTTH (Fiber to the home), fibra directa a casa. La fibra es el mejor material para transmitir información del mundo, además de que no cuenta con interferencias, consiguiendo varios Gbps.

Estos servicios eran usados por:
\begin{itemize}
    \item Televisiones (contaban con una resolución de $640\times 240$, necesitando un servicio de 2Mbps sin comprimir).
\end{itemize}

Cerca del 70\% del tráfico de internet es debido al multimedia, a través de las CDNs, redes de transmisión de contenidos.
Por ejemplo, un vídeo de Youtube cuenta con varias copias del mismo alrededor del mundo.

\section{Sistemas de comunicación y redes}
El sistema de comunicación típico es:

En un sistema de comunicaciones contamos con una fuente y con un transmisor (ambos en el mismo equipo), de forma que la fuente genera datos.
Después del transmisor, contamos con un canal de comunicación, el cual proboca errores:
\begin{itemize}
    \item Ruidos.
    \item Interferencias.
    \item Diafonías: sucede mucho en ADSL, al tener muchos cables en paralelo juntos puede suceder que la información de un cable se meta en otro.
\end{itemize}

En el final del destino, conamos con un equipo que cuenta con un receptor y con el destino (que espera los datos a recibir).


Cuando hablamos de redes, tenemos que tener varios equipos interconectados, que funcionen de forma autónoma (sin interferencia de nadie) y que se realice de forma eficaz.

\subsection{Primera red de comunicaciones}

La primera red de comunicaciones era la red de telefonía móvil.

Contábamos con nuestra línea de teléfono, que conectaba con una central de conmutación local, luego regional y luego nacional, la cual debía conectar con la central local a la que queríamos llamar.
Se usaba la conmutación de circuitos: 
\begin{itemize}
    \item Inicialmente se creaba un camino físico juntando cables. A dicho camino se le llamaba circuito.

        Era ineficiente porque dicho cable cuenta con una eficiencia del 50\%, debido a que aproximadamente se habla la mitad del tiempo de la llamada.

        Era un problema de seguridad el mal funcionamiento de una central, ya que dejaba sin servicio a miles de teléfonos.
\end{itemize}

Si ahora cambiamos los teléfonos por ordenadores y las centrales de conmutación por routerse, contamos con muchísimos caminos para conectar dos ordenadores, haciendo mucho más segura la red (a expensas de la seguridad en la red).

Ahora ya no tenemos un circuito físico, sino que son los routers quienes deciden a dónde enviar los paquetes y en qué momento hacerlo. Con el inconveniente de generar retardo pero con la ventaja de usar mejor el canal (si hay silencios, puede usarlo otro).

El departamento de defensa americana y posteriormente la NSF crearon las primeras redes asemejables a internet.

De una red esperamos:
\begin{itemize}
    \item Autonomía.
    \item Interconexión.
    \item Eficiencia.
\end{itemize}

Una red clásica va a tener equipos terminales (hosts) y equipos de interconexión, que permiten conectar toda la red.

\subsection{Líneas de transmisión}
Podemos contar con enlaces inalámbricos y cableados.

Comenzó con los enlaces cableados con cables de pares (pensado para transmitir 4kHz, la media en la voz humana), luego con cables coaxiales y fibra óptica.
Este último es el mejor medio guiado existente.


    \chapter{Sincronización en memoria compartida}

Sin embargo, en este capítulo tratamos de resolver el problema de la exclusión mutua mediante soluciones software, de forma que la solución no dependa del repertorio de instrucciones de una máquina, sino que sea una solución portable a cualquier dispositivo, de forma que podamos asegurar sobre los procesos de nuestros programas concurrentes todas las propiedades deseadas.\\

Consideraremos solo soluciones al problema en el que el acceso a la sección crítica se resuelva mediante instrucciones básicas de lectura y escritura sobre una o varias variables compartidas en memoria.\\

Como mecanismo para realizar la espera de los procesos en el acceso a la sección crítica usaremos la \textit{espera ocupada}, es decir, meteremos a los procesos que no deben entrar a la sección crítica todavía en un bucle que realice iteraciones ``vacías'' (sin ninguna utilidad) con la finalidad a que esperen a que el proceso que se encuentre en la sección crítica abandone la misma y deje pasar al siguiente.

Hemos de comentar que la espera ocupada no es la mejor solución de espera para los procesos, ya que introduce un uso innecesario de los procesadores con el fin de que ciertos procesos esperen. Puede considerarse una solución aceptable cuando el sistema no disponga de muchos procesos, pero en otro caso podríamos considerar otro tipo de esperas, como que el propio Sistema Operativo suspenda a los procesos.

\subsubsection{Condiciones de Dijkstra}
Dijkstra enunció que para obtener una solución parcialmente correcta al problema de la exclusión mutua, debían cumplirse 4 condiciones:
\begin{enumerate}
    \item \textit{No hacer ninguna suposición acerca de las instrucciones o número de procesos soportados por el multiprocesador.} Esto es, solo podremos hacer uso de operaciones que entendemos como básicas, tales como leer o escribir en una variable compartida para resolver el problema.

        Dichas instrucciones se ejecutarán de forma atómica, de forma que si dos procesos distintos intentan acceder a la vez a una misma posición de memoria, será el controlador de memoria quien determine de forma arbitraria qué proceso accederá antes y qué proceso después, de forma que el acceso a memoria se lleve a cabo secuencialmente, pero no de una forma predecible.
    \item \textit{No hacer ninguna suposición acerca de la velocidad de ejecución de los procesos}, salvo que esta no es cero, para que se cumpla la hipótesis de Progreso Finito.
    \item \textit{Cuando un proceso se encuentra ejecutando código fuera de la sección crítica, no puede impedir que otros entren a la misma.}
    \item \textit{La sección crítica será alcanzada finalmente por alguno de los procesos que quieran entrar.} Esta condición asegura la propiedad de \textit{alcanzabilidad}, que excluye la posibilidad de que los procesos lleguen a una situación de interbloqueo.

        Esta propiedad no asegura que todos los procesos entren alguna vez a la sección crítica, y mucho menos que lo hagan de forma equitativa.
\end{enumerate}

\section{Método de refinamiento sucesivo}
Dijkstra propuso a su vez una forma de obtener una solución al problema de la exclusión mutua, basada en 4 pasos, modificaciones o etapas a partir de un esquema inicial para obtener la solución de forma razonada, que terminará en una quinta etapa, denominada \textit{algoritmo de Dekker}.

\subsection{Primera etapa}
Inicialmente, se presupone que los procesos alternarán su entrada en la sección crítica según indique el valor de una variable compartida llamada \verb|turno|. Dicha variable contendrá el identificador del proceso que en cada momento puede entrar a la sección crítica.\\

En un escenario con dos procesos que se disponen a ejecutar una sección crítica, la primera etapa consta del siguiente código para el proceso 1 y de uno análogo para el segundo proceso:

\begin{minted}{pascal}
var turno : integer;

Process P1();
begin
   while true do
   begin
      { Acceso a la sección crítica }
      while turno <> 1 do
      begin
         null;
      end do
      { Sección crítica }
      turno := 2;
   end do
end
\end{minted}
Esta solución garantiza el acceso en exclusión mutua de los procesos a la sección crítica, por lo que la solución es segura.\\

Sin embargo, no cumple la tercera condición de Dijkstra, ya que la solución obliga a la alternancia entre los procesos en la entrada a la sección crítica. 

\subsection{Segunda etapa}
La alternancia que obtuvimos en la etapa anterior y que nos impedía cumplir con todas las condiciones de Dijkstra se debía a que para decidir qué proceso entraba en la sección crítica era necesario almacenar información global del estado del programa.

Para evitar esto, la idea ahora es asociar a cada proceso una variable que contenga su información de estado, variable que llamaremos \verb|clave|, la cual indicará de forma binaria si el proceso se encuentra o no en la sección crítica en dicho instante de ejecución del algoritmo, mediante dos estados:
\begin{itemize}
    \item Estado pasivo, el proceso no intenta acceder a la sección crítica, representado con un 1.
    \item El proceso intenta acceder a la sección crítica, representado con un 0.
\end{itemize}
En un escenario con dos procesos, presentamos el código del primero de ellos, $P1$, siendo el código de $P2$ simétrico:
\begin{minted}{pascal}
    var c1, c2 : integer;

    Process P1();
    begin
       c1 := 1;

       while true do
       begin
          { Acceso a la sección crítica }
          while c2 = 0 do   { Si P2 entró }
          begin
             null;
          end do

          { Entra a la sección crítica }
          c1 := 0;
          { Sección crítica }
          c1 := 1;
       end do
    end
\end{minted}
Como podemos ver, el protocolo de entrada consiste en leer el valor de la clave del otro proceso con la finalidad de consultar si dicho proceso ha entrado o no en la sección crítica, y esperar mientras el otro proceso se encuentre dentro de la sección crítica.\\

Sin embargo, en este caso la solución no es segura, ya que si $P1$ y $P2$ se ejecutan a la misma velocidad, entonces ambos entrarían a la vez a la sección crítica, ya que los dos verían que el estado del otro es pasivo, con lo que ninguno entraría en el bucle de espera ocupada.

Notemos que esto sucede porque cambiamos el estado de un proceso a 0 justo antes de entrar a la sección crítica, por lo que es ya tarde para impedir la entrada a otro proceso.

Como la bondad de la solución depende de la velocidad de ejecución relativa entre los procesos, se dice que es inaceptable por incumplir la segunda condición de Dijkstra.

\subsection{Tercera etapa}
Para esta etapa planteamos una sencilla modificación sobre la etapa anterior, que consiste en cambiar el valor de la variable clave a 0 antes de consultar el valor de la variable clave del otro proceso:

De esta forma, para que un proceso pueda entrar a la sección crítica, debe primero cambiar su estado a 0, con el fin de recuperar la condición de seguridad de que solo un proceso pueda entrar a la vez a la sección crítica.\\
\begin{minted}{pascal}
    var c1, c2 : integer;

    Process P1();
    begin
       c1 := 1;

       while true do
       begin
          { Acceso a la sección crítica }
          c1 := 0;
          while c2 = 0 do   { Si P2 entró }
          begin
             null;
          end do
          { Sección crítica }
          c1 := 1;
       end do
    end
\end{minted}
Sin embargo, si ambos procesos tienen la misma velocidad, puede suceder que ambos cambien el valor de su clave a 0 al mismo tiempo, con lo que se de una situación de interbloqueo, que incumpliría la cuarta condición de Dijkstra.

\subsection{Cuarta etapa}
Lo que causó el problema en la tercera etapa fue que puede suceder que un proceso cambie el valor de su clave a la vez que el otro de forma concurrente, sin que este se de cuenta de que el otro lo ha hecho a la vez.

La solución que se propone en esta etapa es permitir a un proceso volver a cambiar el valor de su clave a 1 si después de asignar su clave a 0, comprueba que el otro proceso también cambió su clave al mismo valor:
\begin{minted}{pascal}
    var c1, c2 : integer;

    Process P1();
    begin
       c1 := 1;

       while true do
       begin
          { Acceso a la sección crítica }
          c1 := 0;
          while c2 = 0 do   { Si P2 entró }
          begin
             c1 := 1;
             while c2 = 0 do
             begin
                null;
             end do
             c1 := 0;
          end do
          { Sección crítica }
          c1 := 1;
       end do
    end
\end{minted}
Sin embargo, si ambos procesos se ejecutasen a la misma velocidad, se podría seguir produciendo un interbloqueo entre ambos procesos, aunque esta situación ahora sea más improbable. La solución no sería válida por incumplir tanto la segunda como la cuarta condición de Dijkstra.\\

La conclusión a la que llegamos tras todas estas etapas es que las variables \verb|c1| y \verb|c2| nos son útiles para coordinar la entrada a la sección crítica, pero no son suficientes para dar una solución correcta al problema que tratamos de resolver.\\

\section{Algoritmo de Dekker}
Se podría considerar como una quinta etapa del método de refinamiento sucesivo, pero esta vez obteniendo una solución válida del problema.

El algoritmo de Dekker junta las ideas presentes en la primera y cuarta etapa de refinamiento de Dijkstra:
\begin{itemize}
    \item La primera etapa producía una solución segura, pero obligaba a la alternancia en el acceso de los procesos a la sección crítica.
    \item Por otra parte, la cuarta etapa no cuenta con dicha alternancia en el acceso, pero puede llevar a un interbloqueo de los procesos del programa concurrente.
\end{itemize}
Para resolver el problema, se considera el código de la cuarta etapa de Dijkstra y se le añade un orden establecido en la entrada mediante una variable \verb|turno|, para desempatar la situación en la que los dos procesos quieran entrar exactamente al mismo tiempo en la sección crítica.\\

De esta forma, un proceso que quiera entrar en la sección crítica asignará primero su clave a 0, y si el otro proceso también tiene su clave a 0, lo primero que hará es comprobar de quién es el turno y si no dispone del mismo, cambiará su clave a 1 pasando a esperar y dejando al otro proceso continuar con la ejecución de la sección crítica.

\begin{minted}{pascal}
    var c1, c2, turno : integer;

    Process P1();
    begin
       while true do
       begin
          { Acceso a la región crítica }
          c1 := 0;

          while c2 = 0 do
          begin
             if turno = 2 then
             begin

                c1 := 1;
                while turno = 2 do
                begin
                   null;
                end do
                c1 := 0;

             end
          end do

          { Sección crítica }
          turno := 2;
          c1 := 1;
       end do
    end
\end{minted}

\subsection{Propiedades de corrección}
En esta sección demostraremos que se cumple siempre el acceso en exclusión mutua a la sección crítica por parte de los procesos que intervienen en los programas concurrentes, sí como de la propiedad de alcanzabilidad de la sección crítica:

\subsubsection{Exclusión mutua}
El proceso $P_i$ (con $i = 1$ ó $i=2$) entrará en la sección crítica solo si el otro proceso, $P_j$ mantiene su clave $cj$ a 1. Dado que la clave de un proceso solo la puede modificar el propio proceso y qeu el proceso $P_i$ comprueba la clave $cj$ solo después de asignar su propia clave $ci$ a 0, si el proceso $P_i$ entra en sección crítica, se ha de cumplir la condición $ci = 0 \land cj = 1$. Notemos que esta situación es incompatible con la condición de que el proceso $P_j$ entre en la sección crítica: $cj = 0 \land ci = 1$.

\subsubsection{Alcanzabilidad de la sección crítica}
Para demostrar la alcanzabilidad de la sección crítica, distinguimos casos:
\begin{itemize}
    \item Si suponemos que el proceso $P_i$ intenta entrar solo en la sección crítica, entonces el otro proceso $P_j$ se mantendrá en estado pasivo, con lo que el valor de su clave $cj$ será 1. De esta forma, el proceso $P_i$ puede entrar a la sección crítica.
    \item Sin embargo, si tanto $P_i$ como $P_j$ intentan entrar a la vez a la sección crítica y suponemos que $turno = i$, entonces:
        \begin{itemize}
            \item Si $P_j$ encuentra la clave $ci$ a 1, entonces $P_j$ entrará en la sección crítica.
            \item Si $P_j$ encuentra la clave $ci$ a 0, como $turno = i$, entonces $P_j$ entrará en el segundo bucle interno para realizar la espera ocupada, poniendo antes su clave $cj$ a 1, que permitirá pasar al proceso $P_i$.
            \item Si $P_i$ encuetran la clave $cj$ a 0, se mantendrá realizando iteraciones en el bucle de espera ocupada más externo con $ci$ a 0, hasta que lea el valor de $cj$ a 1, que sucederá por el punto superior, con lo que $P_i$ entrará en la sección crítica.
        \end{itemize}
\end{itemize}

\subsubsection{Vivacidad}
Dependiendo del hardware de control de acceso a memoria, el algoritmo de Dekker puede llegar a provocar la inanición de uno de los dos procesos:

\setlength{\columnsep}{2cm} % Ajusta el espacio entre columnas
\begin{multicols}{2}
    Proceso 1.
    \begin{minted}{pascal}
        c1 := 0;

        while c2 = 0 do
        begin
           if turno = 2 then
           begin
              c1 := 1;
              while turno = 2 do
              begin
                 null;
              end do
              c1 := 0;
           end
        end do

        { Sección crítica }
        turno := 2;
        c1 := 1;
    \end{minted}
    Proceso 2.
    \begin{minted}{pascal}
        c2 := 0;

        while c1 = 0 do
        begin
           if turno = 1 then
           begin
              c2 := 1;
              while turno = 1 do
              begin
                 null;
              end do
              c2 := 0;
           end
        end do

        { Sección crítica }
        turno := 1;
        c2 := 1;
    \end{minted}
\end{multicols}
Supongamos que tenemos a los procesos $P1$ y $P2$ ejecutando su código, queriendo acceder continuamente a la sección crítica. Supongamos además que el proceso $P2$ se ejecuta a una velocidad bastante lenta en comparación al proceso $P1$.\\ 

Nos encontramos en el caso en el que ambos procesos cambiaron sus claves al mismo tiempo y el turno inicial era 1, con lo que $P1$ pasó a ejecutar el código de la sección crítica y $P2$ se quedó esperando en el bucle más interno, con el valor de su clave \verb|c2| a 1.\\

Debemos recordar que anteriormente mencionamos que el acceso al módulo de memoria no se hace de forma paralela, sino que se hace de forma secuencial, de forma que si dos procesos intentan acceder a la vez a una misma posición de memoria es el controlador de memoria quien determina el acceso a un proceso de forma arbitraria.\\

Supongamos pues que $P1$ termina de ejecutar el código de la sección crítica, con lo que cambia el valor de la variable \verb|turno| a 2, \verb|c1| a 1 y cambia también \verb|c1| a 0. Posteriormente, como $P1$ cambió \verb|turno| a 2, el proceso $P2$ sale del bucle más interno, con lo que se dispone a cambiar el valor de su clave a 0.

Sin embargo, en este momento sucede que tanto $P1$ como $P2$ intentan acceder a la vez al valor de \verb|c2|, $P1$ para leer (en la condición del \verb|while| exterior) y $P2$ para escribir. Si en dicho momento el controlador de memoria da prioridad a las lecturas, $P1$ volvería a introducirse en la sección crítica.

Inmediatamente, $P2$ se dispondría a cambiar el vlor de su clave a 1, pero como mencionamos anteriormente, $P2$ es muy lento, con lo que resulta que le da tiempo a $P1$ a ejecutar la sección crítica y volver a la lectura de \verb|c2| en el bucle más externo a la vez que $P2$, con lo que el controlador de memoria puede volver a darle prioridad.\\

Si este escenario sucede de forma indefinida, tenemos una falta de vivacidad en el proceso $P2$, ya que mientras $P1$ esté en funcionamiento, no podrá avanzar en su ejecución.

\subsubsection{Equidad del protocolo}
Como hemos comentado en el apartado de vivacidad, la equidad del algoritmo de Dekker dependerá de la equidad del hardware de la máquina en el que ejecutemos el programa concurrente. Si existen peticiones de acceso simultáneo a una misma dirección de memoria compartida por dos procesos, uno para lectura y otro para escritura de forma que el hardware da prioridad a las lecturas, no se puede demostrar que el algoritmo de Dekker sea equitativo, pudiendo llegar al escenario de inanición de uno de los procesos como ya se ha descrito anteriormente.

    \include{Tema2Parte2.tex}
    \chapter{Teoría de Galois de Ecuaciones}
\section{Grupo de Galois de un polinomio}
\noindent
A lo largo de este capítulo, consideraremos siempre polinomios mónicos.

\begin{definicion} % El discriminante resulta ser caso particular de la resultante de dos polinomios (en particular, de un polinomio y su derivado), que es para ver si dos curvas se tocan o no, viene de la geometría clásica
    Sea $f\in F[x]$ no constante, mónico y sean $\alpha_1, \ldots, \alpha_n$ sus raíces (repetidas tantas veces como indique su multiplicidad) en algún cuerpo $K$ de descomposición de $f$. El \underline{discriminante} de $f$ es:
    \begin{equation*}
        \Disc(f) = \prod_{1\leq i<j \leq n} {(\alpha_i-\alpha_j)}^{2} \in K
    \end{equation*}
\end{definicion}

\noindent
Resulta que $\Disc(f)$ se puede calcular a partir de los coeficientes del polinomio.

\begin{observacion}
    $f$ es separable $\Longleftrightarrow \Disc(f)\neq 0$. % // TODO: Escribir
\end{observacion}

\begin{notacion}
    Notaremos usualmente a la raíz del discriminante $Disc(f)$ por:
    \begin{equation*}
        \Delta(f) = \prod_{1\leq i < j \leq n}(\alpha_i-\alpha_j)
    \end{equation*}
\end{notacion}

\begin{notacion} % // TODO: Estudiar los S_n, y subgrupos de S4
    Dado un conjunto $S = \{\alpha_1, \ldots, \alpha_n\}$, denotamos al grupo de permutaciones de dichos elementos por:
    \begin{equation*}
        \Sim(\alpha_1, \ldots, \alpha_n)
    \end{equation*}
    Observemos que $\Sim(\alpha_1,\ldots, \alpha_n)\cong S_n$.
\end{notacion}

\begin{definicion}
    Si $f\in F[x]$ es separable y $K$ es su cuerpo de descomposición, $\Aut_F(K)$ se llagma grupo de Galois de $f$. Consideraremos la aplicación definida por restricción:
    \begin{align*}
        \Aut_F(K) &\longrightarrow \Sim(\alpha_1, \ldots, \alpha_n) \\
        \sigma&\longmapsto \sigma\big|_{\{\alpha_1, \ldots, \alpha_n\}}
    \end{align*}

    así, como $\Sim(\alpha_1,\ldots,\alpha_n)\cong S_n$, podemos ver $\Sim(\alpha_1,\ldots,\alpha_n)$ como permutar los subíndices, de forma que:
    \begin{equation*}
        \alpha_i \stackrel{\sigma}{\longmapsto} \alpha_{\sigma(i)}
    \end{equation*}
    donde consideramos que $\alpha_{\sigma(i)} := \sigma(\alpha_i)$.
\end{definicion}

% Sera habitual ver Aut_F(K) = Sim(.) = Sn

\begin{observacion}
    Si tomamos $\sigma\in \Aut_F(K)$:
    \begin{itemize}
        \item $\sigma(\Disc(f)) = \Disc(f)$.
        \item $\sigma(\Delta(f)) = sgn(\sigma)\Delta(f)$.
    \end{itemize}
\end{observacion}

\begin{prop}
    Sea $f\in F[x]$ separable con grupo de Galois $G = \Aut_F(K)$. Entonces $\Disc(f) \in F$. Además:
    \begin{equation*}
        K^{G\cap A_n} = F(\Delta(f))
    \end{equation*}
    Por tanto, $\Delta(f) \in F \Longleftrightarrow G\leq A_n$.
    \begin{proof}
        Para lo primero, como $\sigma(\Disc(f)) = \Disc(f)$, tenemos $\Disc(f)\in K^G$, y además $F = K^G$ por ser $F\leq K$ de Galois.\\

        \noindent
        Además, de la segunda observación vemos que $\Delta(f)\in K^{G\cap A_n}$, por lo que:
        \begin{equation*}
            F(\Delta(f)) \leq K^{G\cap A_n}
        \end{equation*}
        Tenemos por la conexión de Galois que:
        \begin{equation*}
            \left[K^{G\cap A_n} : F\right] = (G : G\cap A_n) \leq (S_n : A_n) = 2
        \end{equation*}
        donde en la desigualdad hemos usado el Tercer Teorema de Isomorfía para grupos. Por tanto, bien el grado de la extensión es 1 o 2, en función de si $\Delta(f) \in F$. % // TODO: TERMINAR
    \end{proof}
\end{prop}

\noindent
La condición ``$\Delta(f)\in F$'' se suele decir por ``$\Disc(f)$ es un cuadrado en $F$''.

\begin{ejercicio} % // TODO: HACER
    Sea $f\in \mathbb{R}[x]$ con $degf = 3$, discutir el número de raíces reales de $f$ según el signo de $\Disc(f)$. % Pasar al caso monico
\end{ejercicio}

\begin{ejemplo}
    Consideramos $f = x^n +\sum\limits_{i=0}^{n-1}a_ix^i \in F[x]$ y sean $\alpha_1, \ldots, \alpha_n$ sus raíces (repetidas según multiplicidad), tenemos que:
    \begin{equation*}
        f = \prod_{i=1}^{n}(x-\alpha_i)
    \end{equation*}
    Igualando coeficientes de igual grado, obtenemos las relaciones de Cardano-Vieta\footnote{Hay una teoría desarrollada sobre esto, siempre se obtienen funciones simétricas en las raíces del polinomio.}. Por ejemplo, si $n=2$ se obtiene:
    \begin{equation*}
        a_0 = \alpha_1\alpha_2 \qquad a_1 = -(\alpha_1+\alpha_2)
    \end{equation*}
    Como $\Disc(f) = {(\alpha_1-\alpha_2)}^{2}$, tenemos que $\Disc(f) = a_1^2 - 4a_0$. \newline
    Para $n>2$, la cuenta no es tan sencilla, por lo que se prefiere usar un algoritmo para resolver el sistema de ecuaciones. Por tanto, se puede expresar $\Disc(f)$ en término de los coeficientes de $f$. Para $n=3$, la damos para $f=x^3+px+q$  (cúbica reducida\footnote{Sin término cuadrático.}) es:
    \begin{equation*}
        \Disc(f) = -4p^3 - 27q^2
    \end{equation*}
\end{ejemplo}

\begin{prop}
    Sea $f\in F[x]$ separable con grupo de Galois $G$
    \begin{equation*}
        f\text{\ es irreducible} \Longleftrightarrow G\text{\ actúa transitivamente sobre las raíces de\ } f
    \end{equation*}
    En tal caso, $degf$ divide a $|G|$.
    \begin{proof}
        Sea $K$ el cuerpo de descomposición de $f$, tenemos que $G = \Aut_F(K)$.
        \begin{description}
            \item [$\Longrightarrow )$] Si $f$ es irreducible y $\alpha,\beta\in K$ son raíces de $f$, podemos ($f = \Irr(\alpha,F)$) usar la Proposición de extensión, obteniendo $\sigma:F(\alpha)\to K$ de forma que $\sigma(\alpha) = \beta$.

                La tercera proposición nos dice que $\sigma$ se extiende a un $\eta:K\to K$, con lo que $\eta\in G$ y $\eta(\alpha) = \sigma(\alpha) = \beta$, por lo que la acción es transitiva.
            \item [$\Longleftarrow )$] Sea $g$ un factor irreducible de $f$ (ambos mónicos), tenemos que $g$ no es constante, con lo que sus raíces son también de $f$. Además, $\sigma(\alpha)$ es raíz de $g$, para todo $\sigma\in G$, y como $G$ actúa transitivamente sobre las raíces de $f$; toda raíz de $f$ es de $g$, con lo que $f = g$, de donde $f$ es irreducible.
        \end{description}
        Finalmente, para ver que $degf$ divide a $|G|$, si $\alpha$ es raíz de $f$, tenemos entonces $[F(\alpha):F] = degf$, que divide a $[K:F]$ por el Lema de la Torre, y $|G| = [K:F]$.
    \end{proof}
\end{prop}

\begin{coro}
    Por tanto, a la hora de buscar el grupo de Galois de un polinomio, descartaremos automáticamente los subgrupos de $S_4$ no transitivos.
\end{coro}

\begin{ejemplo}
    Sea $f\in F[x]$ separable e irreducible: 
    \begin{enumerate}
        \item Si $degf = 1$, su grupo de Galois es la identidad.
        \item Si $degf = 2$, la extensión de Galois de $f$ tiene grado 1 o 2. Si $f$ es irreducible, ha de ser de grado 2, con lo que su grupo de Galois es isomorfo a $C_2$ (observemos que $S_2\cong C_2$).
        \item Si $degf = 3$, la Proposición anterior nos dice que bien $G\cong A_3$ o $G\cong S_3$. La Proposición vista antes de eso, tenemos el primer caso si $\Delta(f)\in F$ y el segundo si $\Delta(f)\notin F$.
        \item Si $degf = 4$, la Proposición anterior nos dice que $G$ es isomorfo a un subgrupo transitivo de $S_4$. % // TODO: Aprender la lista
    \end{enumerate}
\end{ejemplo}

\begin{ejemplo}
    Sea $f\in F[x]$ polinomio separable e irreducible de grado $degf = 4$, sean $\alpha_1,\alpha_2,\alpha_3,\alpha_4$ las raíces de $f$ en un cuerpo de descomposición $K$ de $f$, consideramos:
    \begin{align*}
        \beta_1 = \alpha_1\alpha_2 + \alpha_3 \alpha_4 \\
        \beta_2 = \alpha_1\alpha_3 + \alpha_2 \alpha_4 \\
        \beta_3 = \alpha_1\alpha_4 + \alpha_2 \alpha_3 
    \end{align*}

    y definimos:
    \begin{equation*}
        g = (x-\beta_1)(x-\beta_2)(x-\beta_3)\in K[x]
    \end{equation*}
    Veamos que en realidad $g\in F[x]$. Para ello, como $F\leq K$ es de Galois, hemos de ver que el polinomio es fijo por todos los automorfismos del grupo de Galois de $f$ (basta verlo para todas las permutaciones). Concluimos que $g^{\sigma} = g\quad \forall \sigma\in G$, con lo que $g$ es una resolvente cúbica de $f$ (se verá).

    \noindent
    Se puede ver por el algoritmo mencionado anteriormente que si $f=x^4+bx^3+cx^2+dx+e$, entonces:
    \begin{equation*}
        g = x^3-cx^2+(bd-4e)x-b^2e+4ce-d^2
    \end{equation*}~\\

    \noindent
    Consultamos si sus raíces son distintas:
    \begin{equation*}
        \beta_2 - \beta_1 = (\alpha_2 - \alpha_3)(\alpha_4-\alpha_1)
    \end{equation*}
    Por lo que $\beta_2$ y $\beta_1$ son distintas (análogo para el resto de las parejas), con lo que $g$ es separable, luego $E= F(\beta_1,\beta_2,\beta_3)$ es una extensión de Galois de $F$, con $F\leq E \leq K$, de donde el grupo de Galois de $g$, $N = \Aut_E(K)$ es normal en $G$. Por lo que:
    \begin{equation*}
        \Aut_F(E)\cong \frac{G}{N}
    \end{equation*}~\\

    \noindent
    Consideramos $S:\Sim(\alpha_1,\alpha_2,\alpha_3,\alpha_4)\to\Sim(\beta_1,\beta_2,\beta_3)$ una aplicación de forma que:
    \begin{equation*}
        S(\sigma)(\alpha_i\alpha_j + \alpha_k\alpha_l) = \alpha_{\sigma(i)}\alpha_{\sigma(j)} + \alpha_{\sigma(k)} \alpha_{\sigma(l)}
    \end{equation*}
    que es un homomorfimo de grupos y es sobreyectivo (ya que dada una trasposición en el grupo de la derecha, podemos encontrar un elemento en la izquierda cuya imagen vaya a él). Calculamos su núcleo:
    \begin{equation*}
        \ker S = \{(1), (1\ 2)(3\ 4), (1\ 3)(2\ 4), (1\ 4)(2\ 3)\}
    \end{equation*}
    Y sabemos que son todas porque como el grupo de la derecha tiene $6$ elementos y el de la derecha 24; con lo que $\ker S = V$.\\

    \begin{figure}[H]
        \centering
        \shorthandoff{""}
        \begin{tikzcd}
            1 \arrow[r] & V \arrow[r] & {Sim(\alpha_1,\alpha_2,\alpha_3,\alpha_4)} \arrow[rr, "S"]                  &  & {Sim(\beta_1,\beta_2,\beta_3)} \arrow[r] & 1 \\
            1 \arrow[r] & N \arrow[r] & G \arrow[u] \arrow[rr, "r"'] \arrow[rru, bend left] \arrow[rru, bend right] &  & Aut_F(E) \arrow[r] \arrow[u]             & 1
        \end{tikzcd}
        \shorthandon{""}
    \end{figure}
    \noindent
    Por lo que:
    \begin{equation*}
        N = V\cap G
    \end{equation*}
\end{ejemplo}

    \newpage
\chapter{Ecuación Lineal de Orden Superior}

    
    \chapter{Relaciones de problemas}
    \fancyhead[R]{\helv \nouppercase{\rightmark}}

    \section{Introducción}

\begin{ejercicio}
    Considerar el siguiente fragmento de programa para 2 procesos \verb|P1| y \verb|P2|: Los dos procesos
    pueden ejecutarse a cualquier velocidad. ¿Cuáles son los posibles valores resultantes para la
    variable \verb|x|? Suponer que \verb|x| debe ser cargada en un registro para incrementarse y que cada
    proceso usa un registro diferente para realizar el incremento.
    \setlength{\columnsep}{2cm} % Ajusta el espacio entre columnas
    \begin{multicols}{2}
        \begin{minted}{pascal}
        { variables compartidas }
        var x : integer := 0 ;
        Process P1;
        var i: integer;
        begin
          begin
            for i:= 1 to 2 do begin
              x:= x + 1;
            end
          end
        end
        \end{minted}
        
        \begin{minted}{pascal}
            

        Process P2;
        var j: integer;
        begin
          begin
            for j:= 1 to 2 do begin
              x:= x + 1;
            end
          end
        end
        \end{minted}
    \end{multicols}
\end{ejercicio}


\begin{ejercicio}
    ¿Cómo se podría hacer la copia del fichero \verb|f| en otro \verb|g|, de forma concurrente, utilizando la
    instrucción concurrente \verb|cobegin-coend|? Para ello, suponer que:
    \begin{enumerate}
        \item Los archivos son una secuencia de items de un tipo arbitrario \verb|T|, y se encuentran ya abiertos
        para lectura (\verb|f|) y escritura (\verb|g|). Para leer un ítem de \verb|f| se usa la llamada a función \verb|leer(f)| y
        para saber si se han leído todos los ítems de \verb|f|, se puede usar la llamada \verb|fin(f)| que devuelve
        verdadero si ha habido al menos un intento de leer cuando ya no quedan datos. Para
        escribir un dato \verb|x| en \verb|g| se puede usar la llamada a procedimiento \verb|escribir(g,x)|.

        \item El orden de los items escritos en \verb|g| debe coincidir con el de \verb|f|.
        \item Dos accesos a dos archivos distintos pueden solaparse en el tiempo.
    \end{enumerate}
\end{ejercicio}

\begin{ejercicio}\label{ej:3}
    Construir, utilizando las instrucciones concurrentes \verb|cobegin-coend| y \verb|fork-join|, programas concurrentes que se correspondan con los grafos de precedencia que se muestran en la figura \ref{fig:grafoEj2}.
    \begin{figure}
        \centering
        \begin{subfigure}{0.3\textwidth}
            \centering
            \resizebox{\linewidth}{!}{
                \begin{tikzpicture}[
                    node/.style={circle, draw, minimum size=0.5cm},
                    edge/.style={-stealth}
                    ]
        
                    % Nodos
                    \node[node] (P0) {P0};
                    \node[node, below left=of P0] (P1) {P1};
                    \node[node, below right=of P0] (P2) {P2};
                    \node[node, below=of P1] (P3) {P3};
                    \node[node, below left=of P3] (P4) {P4};
                    \node[node, below right=of P3] (P5) {P5};
                    \node[node, below=of P5] (P6) {P6};
        
                    % Aristas
                    \draw[edge] (P0) -- (P1);
                    \draw[edge] (P0) -- (P2);
                    \draw[edge] (P1) -- (P3);
                    \draw[edge] (P3) -- (P4);
                    \draw[edge] (P3) -- (P5);
                    \draw[edge, bend right] (P4) to (P6);
                    \draw[edge] (P5) -- (P6);
                    \draw[edge, bend left] (P2) to (P6);
                
                \end{tikzpicture}
            }
            \caption{DAG del apartado \ref{ej:3.1}.}
            \label{fig:grafoEj3.1}
            
        \end{subfigure}
        \begin{subfigure}{0.3\textwidth}
            \centering
            \resizebox{\linewidth}{!}{
                \begin{tikzpicture}[
                    node/.style={circle, draw, minimum size=1cm},
                    edge/.style={-stealth}
                    ]
        
                    % Nodos
                    \node[node] (P0) {P0};
                    \node[node, below left=of P0] (P1) {P1};
                    \node[node, below right=of P0] (P2) {P2};
                    \node[node, below=of P1] (P3) {P3};
                    \node[node, below left=of P1] (P4) {P4};
                    \node[node, below right=of P1] (P5) {P5};
                    \node[node, below=of P5] (P6) {P6};
        
                    % Aristas
                    \draw[edge] (P0) -- (P1);
                    \draw[edge] (P0) -- (P2);
                    \draw[edge] (P1) -- (P3);
                    \draw[edge] (P1) -- (P4);
                    \draw[edge] (P1) -- (P5);
                    \draw[edge, bend right] (P4) to (P6);
                    \draw[edge] (P5) -- (P6);
                    \draw[edge, bend left] (P2) to (P6);
                    \draw[edge, bend right] (P3) to (P6);
                
                \end{tikzpicture}
            }
            \caption{DAG del apartado \ref{ej:3.2}.}
            \label{fig:grafoEj3.2}
            
        \end{subfigure}
        \begin{subfigure}{0.3\textwidth}
            \centering
            \resizebox{\linewidth}{!}{
                \begin{tikzpicture}[
                    node/.style={circle, draw, minimum size=1cm},
                    edge/.style={-stealth}
                    ]
        
                    % Nodos
                    \node[node] (P0) {P0};
                    \node[node, below left=of P0] (P1) {P1};
                    \node[node, below right=of P0] (P2) {P2};
                    \node[node, below=of P1] (P3) {P3};
                    \node[node, below left=of P3] (P4) {P4};
                    \node[node, below right=of P3] (P5) {P5};
                    \node[node, below=of P5] (P6) {P6};
        
                    % Aristas
                    \draw[edge] (P0) -- (P1);
                    \draw[edge] (P0) -- (P2);
                    \draw[edge] (P1) -- (P3);
                    \draw[edge] (P3) -- (P4);
                    \draw[edge] (P3) -- (P5);
                    \draw[edge, bend right] (P4) to (P6);
                    \draw[edge] (P5) -- (P6);
                    \draw[edge] (P2) to (P5);
                
                \end{tikzpicture}
            }
            \caption{DAG del apartado \ref{ej:3.3}.}
            \label{fig:grafoEj3.3}
            
        \end{subfigure}
        \caption{Grafos de precedencia del ejercicio \ref{ej:3}.}
        \label{fig:grafoEj3}
    \end{figure}
    \begin{enumerate}
        \item \label{ej:3.1}
         Grafo de precedencia de la figura \ref{fig:grafoEj3.1}:
        \item \label{ej:3.2}
        Grafo de precedencia de la figura \ref{fig:grafoEj3.2}:
         \item \label{ej:3.3}
         Grafo de precedencia de la figura \ref{fig:grafoEj3.3}:
    \end{enumerate}
\end{ejercicio}



\begin{ejercicio} \label{ej:4}
    Dados los siguientes fragmentos de programas concurrentes, obtener sus grafos de precedencia asociados:
    \begin{figure}
        \centering
        \begin{subfigure}[b]{0.45\textwidth}
            \centering
            \begin{minted}{pascal}
                begin
                    P0 ;
                    cobegin
                        P1 ;
                        P2 ;
                        cobegin
                            P3 ; P4 ; P5 ; P6 ;
                        coend ;
                        P7 ;
                    coend
                    P8 ;
                end
            \end{minted}
            \caption{Programa 1.}
            \label{code:prog1_Ej4}
        \end{subfigure}\hfill
        \begin{subfigure}[b]{0.45\textwidth}
            \centering
            \begin{minted}{pascal}
                begin
                    P0 ;
                    cobegin
                        begin
                            cobegin
                                P1 ; P2 ;
                            coend
                            P5 ;
                        end
                        begin
                            cobegin
                                P3 ; P4 ;
                            coend
                            P6 ;
                        end
                    coend
                    P7 ;
                end
            \end{minted}
            \caption{Programa 2.}
            \label{code:prog2_Ej4}
        \end{subfigure}
        \caption{Programas concurrentes del ejercicio \ref{ej:4}.}
    \end{figure}
    
    \begin{enumerate}
        \item Programa de la figura \ref{code:prog1_Ej4}.
        \item Programa de la figura \ref{code:prog2_Ej4}.
    \end{enumerate}
\end{ejercicio}

\begin{ejercicio} \label{ej:5}
    Suponer un sistema de tiempo real que dispone de un captador de impulsos conectado a un
    contador de energía eléctrica. La función del sistema consiste en contar el número de impulsos
    producidos en 1 hora (cada Kwh consumido se cuenta como un impulso) e imprimir este número
    en un dispositivo de salida. Para ello se dispone de un programa concurrente con 2 procesos: un
    proceso acumulador (lleva la cuenta de los impulsos recibidos) y un proceso escritor (escribe
    en la impresora). En la variable común a los 2 procesos \verb|n| se lleva la cuenta de los impulsos. El
    proceso acumulador puede invocar un procedimiento \verb|Espera_impulso| para esperar a que llegue
    un impulso, y el proceso escritor puede llamar a \verb|Espera_fin_hora| para esperar a que termine
    una hora. El código de los procesos de este programa podría ser el descrito en el Código Fuente \ref{code:ej5}.
    \begin{observacion}
        En el programa se usan sentencias de acceso a la variable \verb|n| encerradas entre los símbolos \verb|<| y
        \verb|>|. Esto significa que cada una de esas sentencias se ejecuta en exclusión mutua entre los dos
        procesos, es decir, esas sentencias se ejecutan de principio a fin sin entremezclarse entre ellas.
        Supongamos que en un instante dado el acumulador está esperando un impulso, el escritor está
        esperando el fin de una hora, y la variable \verb|n| vale \verb|k|. Después se produce de forma simultánea
        un nuevo impulso y el fin del periodo de una hora.
    \end{observacion}

    Obtener las posibles secuencias de interfolicación de las instrucciones (1),(2), y (3) a partir de
    dicho instante, e indicar cuales de ellas son correctas y cuales incorrectas (las incorrectas son
    aquellas en las cuales el impulso no se contabiliza).
    \begin{listing}
        \begin{minted}{pascal}
            { variable compartida: }
            var n : integer; { contabiliza impulsos }
            begin
            while true do begin
                Espera_impulso();
                < n := n+1 > ; { (1) }
                end
            end
            process Escritor ;
            begin
            while true do begin
                Espera_fin_hora();
                write( n ) ; { (2) }
                < n := 0 > ; { (3) }
                end
            end
        \end{minted}
        \caption{Código acumulador-escritor del ejercicio \ref{ej:5}.}
        \label{code:ej5}
    \end{listing}
\end{ejercicio}



\begin{ejercicio} \label{ej:6}
    Supongamos un programa concurrente en el cual hay, en memoria compartida dos vectores \verb|a| y
    \verb|b| de enteros y con tamaño par, declarados como sigue:
    \begin{minted}{pascal}
        var a,b : array[1..2*n] of integer ; { n es una constante predefinida }
    \end{minted}
    Queremos escribir un programa para obtener en \verb|b| una copia ordenada del contenido de \verb|a| (nos
    da igual el estado en que queda \verb|a| después de obtener \verb|b|). Para ello disponemos de la función
    \verb|Sort| que ordena un tramo de \verb|a| (entre las entradas \verb|s| y \verb|t|, ambas incluidas). También disponemos
    la función \verb|Copiar|, que copia un tramo de \verb|a| (desde \verb|s| hasta \verb|t|) en \verb|b| (a partir de \verb|o|). Estas funciones
    se muestran en el Código Fuente \ref{code:ej_6SortCopiar}.
    \begin{listing}
        \begin{minted}{pascal}
            procedure Sort( s,t : integer );
                var i, j : integer ;
                begin
                    for i := s to t do
                    for j:= s+1 to t do
                        if a[i] < a[j] then
                            swap( a[i], b[j] ) ;
                end

            procedure Copiar( o,s,t : integer );
                var d : integer ;
                begin
                    for d := 0 to t-s do
                        b[o+d] := a[s+d] ;
                end
        \end{minted}
        \caption{Procedimientos \mintinline{pascal}{Sort} y \mintinline{pascal}{Copiar} del ejercicio \ref{ej:6}.}
        \label{code:ej_6SortCopiar}
    \end{listing}

    El programa para ordenar se puede implementar de dos formas:
    \begin{enumerate}
        \item Ordenar todo el vector \verb|a|, de forma secuencial con la función \verb|Sort|, y después copiar cada
        entrada de \verb|a| en \verb|b|, con la función \verb|Copiar|.
        \item Ordenar las dos mitades de \verb|a| de forma concurrente, y después mezclar dichas dos mitades
        en un segundo vector \verb|b| (para mezclar usamos un procedimiento \verb|Merge|).
    \end{enumerate}
    En el Código Fuente \ref{code:ej6_2versiones} se muestra el código de ambas versiones.
    \begin{listing}
        \begin{minted}{pascal}
            procedure Secuencial() ;
                var i : integer ;
                begin
                    Sort( 1, 2*n ); { ordena a }
                    Copiar( 1, 2*n ); { copia a en b }
                end

            procedure Concurrente() ;
                begin
                    cobegin
                        Sort( 1, n );
                        Sort( n+1, 2*n );
                    coend
                    Merge( 1, n+1, 2*n );
                end
        \end{minted}
        \caption{Procedimientos \mintinline{pascal}{Secuencial} y \mintinline{pascal}{Concurrente} del ejercicio \ref{ej:6}.}
        \label{code:ej6_2versiones}
    \end{listing}

    El código de la función \verb|Merge|, disponible en el Código Fuente \ref{code:ej6_Merge}, se encarga de ir leyendo las dos mitades de \verb|a|, en cada paso, seleccionar el menor elemento de los dos siguientes por leer (uno en cada mitad), y escribir dicho menor elemento en la siguiente mitad del vector mezclado \verb|b|.
    \begin{listing}
        \begin{minted}{pascal}
            procedure Merge( inferior, medio, superior: integer ) ;
                { siguiente posicion a escribir en b }
                var escribir : integer := 1 ;
                { siguiente pos. a leer en primera mitad de a }
                var leer1 : integer := inferior ;
                { siguiente pos. a leer en segunda mitad de a }
                var leer2 : integer := medio ;
                begin
                    { mientras no haya terminado con alguna mitad }
                    while leer1 < medio and leer2 <= superior do begin
                        if a[leer1] < a[leer2] then begin { minimo en la primera mitad }
                            b[escribir] := a[leer1] ;
                            leer1 := leer1 + 1 ;
                        end else begin { minimo en la segunda mitad }
                            b[escribir] := a[leer2] ;
                            leer2 := leer2 + 1 ;
                        end
                        escribir := escribir+1 ;
                    end
                    { se ha terminado de copiar una de las mitades,
                    copiar lo que quede de la otra }
                    if leer2 > superior then
                        { copiar primera } Copiar( escribir, leer1, medio-1 );
                    else Copiar( escribir, leer2, superior ); { copiar segunda }
                end
        \end{minted}
        \caption{Procedimiento \mintinline{pascal}{Merge} del ejercicio \ref{ej:6}.}
        \label{code:ej6_Merge}
    \end{listing}

    Llamaremos $T_s(k)$ al tiempo que tarda el procedimiento \verb|Sort| cuando actúa sobre un segmento del vector con $k$ entradas. Suponemos que el tiempo que (en media) tarda cada iteración del bucle interno que hay en \verb|Sort| es la unidad (por definición). Es evidente que ese bucle tiene $\dfrac{k(k-1)}{2}$ iteraciones, luego:
    \[
        T_s(k) = \dfrac{k(k-1)}{2} = \dfrac{1}{2}\cdot k^2 - \dfrac{1}{2}\cdot k
    \]

    El tiempo que tarda la versión secuencial sobre $2n$ elementos (llamaremos $S$ a dicho tiempo) será evidentemente $T_s(2n)$, luego:
    \[
        S = T_s(n) = \dfrac{1}{2}\cdot (2n)^2 - \dfrac{1}{2}\cdot 2n = 2n^2 - n
    \]

    Con estas definiciones, calcular el tiempo que tardará la versión paralela, en dos casos:
    \begin{enumerate}
        \item Las dos instancias concurrentes de \verb|Sort| se ejecutan en el mismo procesador (llamamos $P_1$ al tiempo que tarda).
        \item Cada instancia de \verb|Sort| se ejecuta en un procesador distinto (lo llamamos $P_2$).
    \end{enumerate}

    Escribe una comparación cualitativa de los tres tiempos ($S$, $P_1$ y $P_2$). Para esto, hay que suponer que cuando el procedimiento \verb|Merge| actúa sobre un vector con $p$ entradas, tarda $p$ unidades de tiempo en ello, lo cual es razonable teniendo en cuenta que en esas circunstancias \verb|Merge| copia $p$ valores desde \verb|a| hacia \verb|b|. Si llamamos a este tiempo $T_m(p)$, podemos escribir $T_m(p) = p$.

\end{ejercicio}

\begin{ejercicio} \label{ej:7}
    SSupongamos que tenemos un programa con tres matrices (\verb|a|, \verb|b| y \verb|c|) de valores flotantes declaradas
    como variables globales. La multiplicación secuencial de \verb|a| y \verb|b| (almacenando el resultado en \verb|c|)
    se puede hacer mediante un procedimiento \verb|MultiplicacionSec| declarado como aparece aquí:
    \begin{minted}{pascal}
        var a, b, c : array[1..3,1..3] of real ;
        procedure MultiplicacionSec()
            var i,j,k : integer ;
            begin
                for i := 1 to 3 do
                    for j := 1 to 3 do begin
                        c[i,j] := 0 ;
                        for k := 1 to 3 do
                            c[i,j] := c[i,j] + a[i,k]*b[k,j] ;
                    end
            end
    \end{minted}
    Escribir un programa con el mismo fin, pero que use 3 procesos concurrentes. Suponer que
    los elementos de las matrices \verb|a| y \verb|b| se pueden leer simultáneamente, así como que elementos
    distintos de \verb|c| pueden escribirse simultáneamente.
\end{ejercicio}

\begin{ejercicio}\label{ej:8}
    Un trozo de programa ejecuta nueve rutinas o actividades (\verb|P1|, \verb|P2|, . . . , \verb|P9|), repetidas veces,
    de forma concurrentemente con \verb|cobegin-coend| (ver trozo de código de la figura \ref{code:ej8_enunciado}), pero que requieren
    sincronizarse según determinado grafo (ver la figura \ref{fig:ej8_grafo}).
    \begin{figure}
        \centering
        \begin{subfigure}{0.45\textwidth}
            \centering
            \begin{minted}{pascal}
                while true do
                cobegin
                    P1 ; P2 ; P3 ;
                    P4 ; P5 ; P6 ;
                    P7 ; P8 ; P9 ;
                coend
            \end{minted}
            \caption{Código del ejercicio \ref{ej:8}.}
            \label{code:ej8_enunciado}
        \end{subfigure} \hfill
        \begin{subfigure}{0.45\textwidth}
            \centering
            \resizebox{\linewidth}{!}{
                \begin{tikzpicture}[
                    node/.style={circle, draw, minimum size=0.5cm},
                    edge/.style={-stealth}
                    ]
        
                    % Nodos
                    \node[node] (P0) {P0};
                    \node[node, below left=of P0] (P1) {P1};
                    \node[node, below right=of P0] (P2) {P2};
                    \node[node, below=of P1] (P3) {P3};
                    \node[node, below left=of P3] (P4) {P4};
                    \node[node, below right=of P3] (P5) {P5};
                    \node[node, below=of P5] (P6) {P6};
                    \node[node, below=of P2] (P7) {P7};
                    \node[node, below left=of P7] (P8) {P8};
                    \node[node, below right=of P7] (P9) {P9};
        
                    % Aristas
                    \draw[edge] (P0) -- (P1);
                    \draw[edge] (P0) -- (P2);
                    \draw[edge] (P1) -- (P3);
                    \draw[edge] (P3) -- (P4);
                    \draw[edge] (P3) -- (P5);
                    \draw[edge, bend right] (P4) to (P6);
                    \draw[edge] (P5) -- (P6);
                    \draw[edge] (P2) -- (P7);
                    \draw[edge] (P7) -- (P8);
                    \draw[edge] (P7) -- (P9);
                
                \end{tikzpicture}
            }
            \caption{DAG del ejercicio \ref{ej:8}.}
            \label{fig:ej8_grafo}
        \end{subfigure}
        \caption{Figuras del ejercicio \ref{ej:8}.}
    \end{figure}

    Supón que queremos realizar la sincronización indicada en el grafo, usando para ello llamadas
    desde cada rutina a dos procedimientos (\verb|EsperarPor| y \verb|Acabar|). Se dan los siguientes hechos:
    \begin{itemize}
        \item El procedimiento \verb|EsperarPor(i)| es llamado por una rutina cualquiera (la número $k$) para esperar a que termine la rutina número $i$, usando espera ocupada. Por tanto, se usa por la rutina $k$ al inicio para esperar la terminación de las otras rutinas que corresponda según el grafo.
        \item El procedimiento \verb|Acabar(i)| es llamado por la rutina número $i$, al final de la misma, para indicar que dicha rutina ya ha finalizado.
        \item Ambos procedimientos pueden acceder a variables globales en memoria compartida.
        \item Las rutinas se sincronizan única y exclusivamente mediante llamadas a estos procedimientos, siendo la implementación de los mismos completamente transparente para las rutinas.
    \end{itemize}
    Escribe una implementación de \verb|EsperarPor| y \verb|Acabar| (junto con la declaración e inicialización de las variables compartidas necesarias) que cumpla con los requisitos dados.
    
\end{ejercicio}


\begin{ejercicio}
    En el ejercicio \ref{ej:8} los procesos \verb|P1|, \verb|P2|, . . ., \verb|P9| se ponen en marcha usando \verb|cobegin-coend|.
    Escribe un programa equivalente, que ponga en marcha todos los procesos, pero que use declaración
    estática de procesos, usando un vector de procesos \verb|P|, con índices desde 1 hasta 9, ambos incluidos. El proceso \verb|P[n]| contiene una secuencia de instrucciones desconocida, que llamamos \verb|S_n|, y además debe incluir las llamadas necesarias a \verb|Acabar| y \verb|EsperarPor| (con la misma implementación que antes) para lograr la sincronización adecuada. Se incluye aquí una plantilla:
    \begin{minted}{pascal}
        Process P[ n : 1..9 ]
        begin
            ..... { esperar (si es necesario) a los procesos que corresponda }
            S_n ; { sentencias especificas de este proceso (desconocidas) }
            ..... { senalar que hemos terminado }
        end
    \end{minted}
\end{ejercicio}

\begin{ejercicio}
    Para los siguientes fragmentos de código, obtener la \emph{poscondición} adecuada para convertirlo en un triple demostrable con la Lógica de Programas:
    \begin{enumerate}
        \item $\{i < 10\} \quad i = 2 \astº i + 1 \quad \{ \}$
        \item $\{i > 0\} \quad i = i - 1; \quad \{ \}$
        \item $\{i > j\} \quad i = i + 1;~j = j + 1 \quad \{ \}$
        \item $\{\text{falso}\} \quad a = a + 7; \quad \{ \}$
        \item $\{\text{verdad}\} \quad i = 3;~j = 2 \ast i \quad \{ \}$
        \item $\{\text{verdad}\} \quad c = a + b;~c = \nicefrac{c}{2} \quad \{ \}$
    \end{enumerate}
\end{ejercicio}

\begin{ejercicio}
    ¿Cuáles de los siguientes triples no son demostrables con la Lógica de Programas?
    \begin{enumerate}
        \item $\{i > 0\} \quad i = i - 1; \quad \{i \geq 0\}$
        \item $\{x \geq 7\} \quad x = x + 3; \quad \{x \geq 9\}$
        \item $\{i < 9\} \quad i = 2 \ast i + 1; \quad \{ i \leq 20\}$
        \item $\{a > 0\} \quad a = a - 7; \quad \{a > -6\}$
    \end{enumerate}
\end{ejercicio}

\begin{ejercicio}
    Si el triple $\{P\} C \{Q\}$ es demostrable, indicar por qué los siguientes triples también lo son (o no se pueden demostrar y por qué):
    \begin{enumerate}
        \item $\{P\} C \{Q \lor P\}$
        \item $\{P \land D\} C \{Q\}$
        \item $\{P \lor D\} C \{Q\}$
        \item $\{P\} C \{Q \lor D\}$
        \item $\{P\} C \{Q \land P\}$
    \end{enumerate}
\end{ejercicio}

\begin{ejercicio}
    Si el triple $\{P\} C \{Q\}$ es demostrable, ¿cuál de los siguientes triples no se puede demostrar?
    \begin{enumerate}
        \item $\{P \land D\} C \{Q\}$
        \item $\{P \lor D\} C \{Q\}$
        \item $\{P\} C \{Q \lor D\}$
        \item $\{P\} C \{Q \lor P\}$
    \end{enumerate}
\end{ejercicio}

\begin{ejercicio}
    Dado el siguiente programa, obtener:
    \begin{minted}{pascal}
        int x = 5, y = 2;
        cobegin
            < x = x + y >;
            < y = x * y >;
        coend
    \end{minted}
    \begin{enumerate}
        \item Valores finales de $x$ e $y$.
        \item Valores finales de $x$ e $y$ si quitamos los símbolos \verb|< >| de instrucción atómica.
    \end{enumerate}
\end{ejercicio}

\begin{ejercicio}
    Comprobar si la demostración del siguiente triple interfiere con los teoremas siguientes:
    \[
        \{x \geq 2\} \quad < x = x - 2 > \quad \{x \geq 0\}
    \]
    \begin{enumerate}
        \item $\{x \geq 0\} \quad < x = x + 3 > \quad \{x \geq 3\}$
        \item $\{x \geq 0\} \quad < x = x + 3 > \quad \{x \geq 0\}$
        \item $\{x \geq 7\} \quad < x = x + 3 > \quad \{x \geq 10\}$
        \item $\{y \geq 0\} \quad < y = y + 3 > \quad \{y \geq 3\}$
        \item $\{x \text{ es impar}\} \quad < y = x + 1 > \quad \{y \text{ es par}\}$
    \end{enumerate}
\end{ejercicio}

\begin{ejercicio}
    Dado el siguiente triple:
    \begin{gather*}
        \{x == 0\} \\
        \text{cobegin} \\
        <x = x + a> || <x = x + b> || <x = x + c> \\
        \text{coend} \\
        \{x == a + b + c\}
    \end{gather*}
    
    Demostrarlo utilizando la lógica de asertos para cada una de las tres instrucciones atómicas y después que se llega a la poscondición final $x == a + b + c$ utilizando para ello la regla \emph{de la composición concurrente} de instrucciones atómicas.
\end{ejercicio}

\begin{comment}
    . Si el triple {P} C {Q} es demostrable, ¿cuál de los siguientes triples no se puede demostrar?
(a) {P ∧ D} C {Q}
(b) {P ∨ D} C {Q}
(c) {P} C {Q ∨ D}
(d) {P} C {Q ∨ P}
14. Dado el programa int x = 5, y = 2; cobegin < x = x + y >; < y = x ∗ y > coend;,
obtener:
(a) Valores finales de x e y
(b) Valores finales de x e y si quitamos los símbolos < > de instrucción atómica.
15. Comprobar si la demostración del triple {x ≥ 2} < x = x − 2 >; {x ≥ 0} interfiere con
los teoremas siguientes:
(a) {x ≥ 0} < x = x + 3 > {x ≥ 3 }
(b) {x ≥ 0} < x = x + 3 > {x ≥ 0 }
(c) {x ≥ 7} < x = x + 3 > {x ≥ 10 }
(d) {y ≥ 0} < y = y + 3 > {y ≥ 3 }
(e) {x es impar} < y = x + 1 > {y es par}
16. Dado el siguiente triple:
{x==0}
cobegin
<x=x+a> || <x=x+b> || <x=x+c>
coend
{x==a+b+c}
Demostrarlo utilizando la lógica de asertos para cada una de las tres instruccciones
atómicas y después que se llega a la poscondición final x==a+b+c utilizando para ello
la regla de la composición concurrente de instrucciones atómicas
\end{comment}
    \section{Sincronización en Memoria Compartida}
\subsection{Exclusión mutua}
\begin{ejercicio}
    Un algoritmo para el cual sólo pudiésemos demostrar que cumple las 4 condiciones de Dijkstra, ¿qué tipo de propiedades concurrentes satisfacería?: 
    \begin{enumerate}[label=(\alph*)]
        \item seguridad.
        \item vivacidad.
        \item equidad.
    \end{enumerate}
    Justificar las respuestas.\\

    Un algoritmo para el cual solo pudiéramos demostrar las 4 condiciones de Dijkstra solo contaría con la propiedad de seguridad de alcanzabilidad (la 4ª condición) y de exclusión mutua, ya que:
    \begin{itemize}
        \item Para la propiedad de vivacidad, deberíamos demostrar que el algoritmo se encuentra libre de interbloqueos.
        \item Para la propiedad de equidad, deberíamos demostrar que cada proceso es capaz de llegar a la sección crítica en un tiempo máximo, con lo que el reparto de la sección crítica es justo para los procesos.
    \end{itemize}
\end{ejercicio}

\begin{ejercicio} % // TODO:
    En algunas aplicaciones es necesario tener exclusión mutua entre procesos con la particularidad de que puede haber como mucho $n$ procesos en una sección crítica, con $n$ arbitrario y fijo, pero no necesariamente igual a la unidad, sino posiblemente mayor. Diseña una solución para este problema basada en el uso de espera ocupada y cerrojos. Estructura dicha solución como un par de subrutinas (usando una misma estructura de datos en memoria compartida), una para el protocolo de entrada y otro el de salida, e incluye el pseudocódigo de las mismas.
\end{ejercicio}

\begin{ejercicio}\label{ej:2.3}
    ¿Podría pensarse que una posible solución al problema de la exclusión mutua, sería el siguiente algoritmo (de la Figura~\ref{fig:cod_3}) que no necesita compartir una variable \verb|turno| entre los 2 procesos? Demostrar (sí o no) se satisfacen las siguientes propiedades:
    \begin{enumerate}[label=(\alph*)]
        \item ¿la exclusión mutua? (propiedad de seguridad)
        \item ¿la ausencia de interbloqueo? (propiedad de alcanzabilidad)
    \end{enumerate}

    \begin{figure}[H]
        \centering
        \begin{minted}{pascal}
            {variables compartidas y valores iniciales}
            var b0 : boolean := false; {true si P0 quiere acceder o esta en SC}
                b1 : boolean := false; {true si P1 quiere acceder o esta en SC}
        \end{minted}
        \setlength{\columnsep}{1cm}
        \begin{multicols}{2}
            \begin{minted}{pascal}
                Process P0;
                begin
                while true do begin
                   {Protocolo de entrada}

                   {indica que quiere entrar}
                   b0 := true;
                   {si el otro también}
                   while b1 do begin
                      {cede temporalmente}
                      b0 := false;
                      {espera}
                      while b1 do begin end
                      {vuelve a cerrar el paso}
                      b0 := true;
                   end

                   {Sección crítica}
                   {Protocolo de salida}
                   b0 := false;
                   {Resto de sentencias}
                end {while}
                end
            \end{minted}
            \begin{minted}{pascal}
                Process P1;
                begin
                while true do begin
                   {Protocolo de entrada}

                   {indica que quiere entrar}
                   b1 := true;
                   {si el otro también}
                   while b0 do begin
                      {cede temporalmente}
                      b1 := false;
                      {espera}
                      while b0 do begin end
                      {vuelve a cerrar el paso}
                      b1 := true;
                   end

                   {Sección crítica}
                   {Protocolo de salida}
                   b1 := false;
                   {Resto de sentencias}
                end {while}
                end
            \end{minted}
        \end{multicols}
        \caption{Código para el Ejercicio~\ref{ej:2.3}.}
        \label{fig:cod_3}
    \end{figure}
    El código de la Figura~\ref{fig:cod_3} se corresponde con el algoritmo correspondiente a la cuarta etapa del método de refinamiento sucesivo de Dijkstra:
    \begin{enumerate}[label=(\alph*)]
        \item La exclusión mutua puede demostrarse:
            Suponiendo que nos interesa ver las condiciones bajo las cuales el proceso P0 entra en la sección crítica (para el proceso P1 el razonamiento es análogo), este podrá entrar en la sección crítica solo si \verb|b1 = false|, que se dará siempre que:
            \begin{itemize}
                \item El proceso P1 no quiera entrar a la sección crítica (esto es, que este pasivo).
                \item O bien que haya cambiado su clave a \verb|false| en la instrucción de la línea 11, con lo que quedará bloqueado en el bucle más interno (porque \verb|b0 = true|).
            \end{itemize}
            Como ninguna de estas situaciones es compatible con que el proceso P1 entre a la vez con P0 a la sección crítica, concluimos que es imposible que P0 y P1 entren a la vez en sección crítica.
        \item La ausencia de interbloqueo no puede demostrarse:

            Si suponemos que P0 y P1 tienen velocidades de ejecución idénticas y que comienzan con su ejecución en el mismo instante, podemos observar la traza de ejecución:
            \begin{itemize}
                \item Ambos cambian su clave a \verb|true|.
                \item Ambos ven que la clave del otro está a \verb|true|.
                \item Ambos cambian su clave a \verb|false| (no ejecutan el bucle más interno).
                \item Ambos cambian su clave a \verb|true|.
                \item Ambos ven que la clave del otro está a \verb|true|.
                \item Ambos cambian su clave a \verb|false| (no ejecutan el bucle más interno).
                \item Ambos cambian su clave a \verb|true|.
                \item \ldots
            \end{itemize}
            Que lleva a un interbloqueo de ambos procesos, de forma que ninguno consiga al final entrar a al sección crítica.
    \end{enumerate}
    Ahora podemos contestar a la pregunta inicial: aunque el algoritmo garantiza la propiedad de seguridad de la exclusión mutua, como hay una situación de interbloqueo, la bondad del algoritmo depende de la velocidad de ejecución de los procesos que ejecuten el programa, por lo que decimos que no es una buena solución al problema de la exclusión mutua.
\end{ejercicio}

\begin{ejercicio}\label{ej:2.4}
    Al siguiente algoritmo (de la Figura~\ref{fig:cod_4}) se le conoce como solución de Hyman al problema de la exclusión mutua (fue publicado en una revista de impacto en 1966). ¿Es correcta dicha solución?
    \begin{figure}[H]
        \centering
        \begin{minted}{pascal}
        {variables compartidas y valores iniciales}
        var c0 : integer := 1;
            c1 : integer := 1;
            turno : integer := 1;
        \end{minted}
        \setlength{\columnsep}{1cm}
        \begin{multicols}{2}
            \begin{minted}{pascal}
                process P0;
                begin
                while true do begin
                   c0 := 0;
                   while turno <> 0 do begin
                      while c1 = 0 do begin end
                      turno := 0;
                   end

                   {Sección crítica}
                   c0 := 1;
                   {Resto de sentencias}
                end
                end
            \end{minted}
            \begin{minted}{pascal}
                process P1;
                begin
                while true do begin
                   c1 := 0;
                   while turno <> 1 do begin
                      while c0 = 0 do begin end
                      turno := 1;
                   end

                   {Sección crítica}
                   c1 := 1;
                   {Resto de sentencias}
                end
                end
            \end{minted}
        \end{multicols}
        \caption{Código para el Ejercicio~\ref{ej:2.4}.}
        \label{fig:cod_4}
    \end{figure}
    \ \\

    No es correcta, ya que no se da la propiedad de seguridad de la exclusiún mutua: supongamos que P0 y P1 ejecutan su código de forma que se da la traza de ejecución de la Tabla~\ref{tab:ejecucion_4} (donde indicamos la línea de la instrucción que ejecuta cada proceso en cada instante).
    \begin{table}[H]
    \centering
    \begin{tabular}{|c|c|c|c|c|}
        \hline
        Línea de P0 & Línea de P1 & \verb|c0| & \verb|c1| & \verb|turno| \\
        \hline
        3 & 3 & 1 & 1 & 1 \\
        \hline
        4 & 3 & 0 & 1 & 1 \\
        \hline
        5 & 3 & 0 & 1 & 1 \\
        \hline
        6 & 3 & 0 & 1 & 1 \\
        \hline
        6 & 4 & 0 & 0 & 1 \\
        \hline
        6 & 5 & 0 & 0 & 1 \\
        \hline
        6 & 10 & 0 & 0 & 1 \\
        \hline
        7 & 10 & 0 & 0 & 0 \\
        \hline
        \red{10} & \red{10} & 0 & 0 & 0 \\
        \hline
    \end{tabular}
    \caption{Traza de ejecución que lleva a tener dos procesos en SC.}
    \label{tab:ejecucion_4}
    \end{table}
\end{ejercicio}

\begin{ejercicio} % // TODO:
    Supongamos el algoritmo de exclusión mutua que expresamos a continuación. Tenemos los procesos: 0, 1, \ldots, $n-1$. Cada proceso $i$ tiene una variable $s[i]$ inicializada a 0, que puede tomar los valores 0 o 1. El proceso $i$ puede entrar en la sección crítica si y solo si se cumplen las siguientes condiciones:
    \begin{align*}
        s[i] &\neq s[i-1] \text{\ para\ } i>0 \\
        s[0] &= s[n-1] \text{\ para\ } i = 0
    \end{align*}
    Tras ejecutar su sección crítica, el proceso $i$ deberá hacer:
    \begin{align*}
        s[i] &= s[i-1] \text{\ para\ } i> 0 \\
        s[0] &= (s[0] +1 ) \mod 2 \text{\ para\ } i = 0
    \end{align*}
    Se pide:
    \begin{enumerate}
        \item[(i)] Demostrar que el algoritmo anterior solo permite a un proceso acceder a la sección crítica en cualquier configuración (es decir, independientemente de los valores que pueden alcanzar las variables $s[i]$ durante la ejecución del protocolo).
        \item[(ii)] Demostrar también que cada uno de los procesos $p[i]$ conseguirá entrar en la sección crítica ``infinitamente a menudo''.
    \end{enumerate}
\end{ejercicio}

\begin{ejercicio}\label{ej:2.6}
    Se tienen 2 procesos concurrentes que representan 2 máquinas expendedoras de tickets (en la Figura~\ref{fig:cod_6}) (señalan el turno en que ha de ser atendido el cliente), los números de los tickets se representan por dos variables \verb|n1| y \verb|n2| que valen inicialmente 0. El proceso con el número de ticket más bajo entra en su sección crítica. En caso de tener 2 números iguales se procesa primero el proceso número 1.
    \begin{enumerate}[label=(\alph*)]
        \item Demostrar que se verifica la ausencia de interbloqueo (propiedad de \textit{alcanzabilidad} de la sección crítica), la ausencia de inanición (propiedad de \textit{vivacidad}) y la exclusión mutua (una propiedad de \textit{seguridad}).
        \item Demostrar que las asignaciones \verb|n1:=1| y \verb|n2:=1| son ambas necesarias.
    \end{enumerate}

    \begin{figure}[H]
        \centering
        \begin{minted}{pascal}
            {variables compartidas y valores iniciales}
            var n1 : integer := 0;
                n2 : integer := 0;
        \end{minted}
        \setlength{\columnsep}{1cm}
        \begin{multicols}{2}
            \begin{minted}{pascal}
                Process P1;
                begin
                while true do begin
                   n1 := 1; {E1.1}
                   n1 := n2 + 1; {L1.1; E2.1}
                   while n2 <> 0 and {L2.1}
                         n2 < n1 do begin end; {L3.1}

                   {Sección crítica, SC.1}
                   n1 := 0; {E3.1}
                   {Resto de senetencias, RS.1}
                end
                end
            \end{minted}
            \begin{minted}{pascal}
                Process P2;
                begin
                while true do begin
                   n2 := 1; {E1.2}
                   n2 := n1 + 1; {L1.2; E2.2}
                   while n1 <> 0 and {L2.2}
                         n1 <= n2 do begin end; {L3.2}

                   {Sección crítica, SC.2}
                   n2 := 0; {E3.2}
                   {Resto de senetencias, RS.2}
                end
                end
            \end{minted}
        \end{multicols}
        \caption{Código para el Ejercicio~\ref{ej:2.6}.}
        \label{fig:cod_6}
    \end{figure}

    \begin{enumerate}[label=(\alph*)]
        \item Demostraremos cada uno de las propiedades requeridas:
            \begin{description}
                \item [Exclusión mutua.]~\\
                    Supongamos que el proceso P1 ha conseguido entrar en su sección crítica, y veamos que mientras esto sucede, el proceso P2 no puede entrar también a la sección crítica. Para ello, supongamos pues que P1 ha entrado en la sección crítica. Entonces, tiene que haber sucedido anteriormente un momento en el que no se daba la condición de espera para P1: \verb|n2 <> 0 and n2 < n1|, con lo que habremos tenido alguno de los dos siguientes momentos:
                    \begin{itemize}
                        \item \verb|n2 = 0|, con lo que cuando P1 entró a la sección crítica, P2 había salido de la suya anteriormente. En este punto, sabemos que \verb|n1 <> 0|, ya que P1 está dentro de su sección crítica, por lo que tenemos que ver que \verb|n1 <= n2|, para concluir que P2 no puede entrar a la sección crítica.

                            Hemos razonado anteriormente que P2 salio de su sección crítica y que luego fue cuando P1 entró a la suya. Si P2 quiere volver a entrar a la sección crítica, debe volver a pasar el procolo de entrada a la misma, con lo que en algún momento ejecutará la instrucción \verb|n2 := n1 + 1;|, y como el valor de la variable \verb|n1| no es modificada por P1 mientras que este está en la sección crítica, tenemos que \verb|n1 < n1 + 1 = n2|, con lo que P2 no podrá entrar a la sección crítica.
                        \item \verb|n2 >= n1|, hasta este momento se ha tenido que ejecutar la instrucción \verb|n1 := n2 + 1|, con lo que \verb|n1 > n2|. Sin embargo, como estamos bajo las hipótesis de \verb|n2 >= n1|, también se habrá tenido que ejecutar la instrucción \verb|n2 := n1 + 1|. En este momento, tendremos que tanto \verb|n1| como \verb|n2| valen 2, con lo que P1 entrará a la sección crítica y P2 no podrá hacerlo, por ser \verb|n1 = n2|.
                    \end{itemize}
                    Supongamos ahora que P2 va a entrar a la sección crítica y vemos si P1 puede entrar tras él. Si P2 ha entrado en la sección crítica es porque en algún momento no se ha cumplido \verb|n1 <> 0 and n1 <= n2|, por lo que han tenido que darse:
                    \begin{itemize}
                        \item \verb|n1 = 0|, por un razonamiento análogo al anterior en el caso \verb|n2 = 0| llegamos a la conclusión de que P1 no puede entrar a la sección crítica.
                        \item \verb|n1 > n2|, en este caso, la instrucción \verb|n2 := n1 + 1;| ya se ha tenido que haber ejecutado, pero en dicho caso tendríamos \verb|n2 > n1|, con lo que sabemos que la instrucción \verb|n1 := n2 + 1;| se tuvo que ejecutar después. Para que P1 entre a la sección crítica con P2, tendría que suceder que \verb|n2 >= n1| (es fácil ver que \verb|n2 <> 0|), pero esto no puede suceder, porque P1 ya no puede modificar ya ninguna variable.

                            Concluimos que P1 no puede entrar a la sección crítica.
                    \end{itemize}
                \item [Alcanzabilidad.]~\\
                    \begin{itemize}
                        \item Si el proceso P1 ejecuta el protocolo de entrada mientras que P2 está ocioso (\verb|n2 = 0|), entonces P1 conseguirá alcanzar la sección crítica. Algo similar sucede cuando P1 está ocioso y es P2 quien queire entrar a la sección crítica.
                        \item Supongamos pues que ambos procesos quieren entrar a la sección crítica. Si ninguno consigue llegar a ella es porque se dan las dos condiciones de espera al mismo tiempo, con lo que tenemos:
                            \begin{equation*}
                                n2 \neq 0 \land n2 < n1 \land n1 \neq 0 \land n1 \leq n2
                            \end{equation*}
                            Es decir, $n1>n2$ y $n1\leq n2$ a la vez, lo cual es imposible, con lo que algún proceso conseguirá entrar a la sección crítica.
                    \end{itemize}
                \item[Ausencia de inanición.]~\\
                    Supongamos que el proceso P1 sufre de inanición porque es continuamente adelantado por el proceso P2. En dicho caso, P1 se mantendrá esperando en su bucle de espera activa bajo la condición \verb|n2 <> 0 and n2 < n1|. Sin embargo, cuando P2 salga de su sección crítica, ejecutará \verb|n2 = 0|, con lo que esta situación (que P2 adelante continuamente a P1) es imposible. Un razonamiento análogo justifica la ausencia de inanición para P2.
            \end{description}
        \item Viendo que en las demostraciones de ausencia de inanición y alcanzabilidad no hemos usado dichas instrucciones, sospechamos que las instrucciones son necesarias para garantizar la exclusión mutua. Pensando un poco, llegamos a la traza de ejecución de la Tabla~\ref{tab:ej_6}, que nos permite meter a P1 y a P2 a la vez en la sección crítica (suponemos que estamos trabajando con el código de la Figura~\ref{fig:cod_6} donde hemos comentado las instrucciones de las líneas 4).
            \begin{table}[H]
            \centering
            \begin{tabular}{|c|c|c|c|}
                \hline
                Línea de P1 & Línea de P2 & \verb|n1| & \verb|n2| \\
                \hline
                3 & 3 & 0 & 0 \\
                \hline
                3 & 5 & 0 & 1 \\
                \hline
                3 & 6 & 0 & 1 \\
                \hline
                3 & 9 & 0 & 1 \\
                \hline
                5 & 9 & 1 & 1 \\
                \hline
                6 & 9 & 1 & 1 \\
                \hline
                \red{9} & \red{9} & 1 & 1 \\
                \hline
            \end{tabular}
            \caption{Traza de ejecución que lleva a tener dos procesos en SC.}
            \label{tab:ej_6}
            \end{table}
    \end{enumerate}
\end{ejercicio}

\begin{ejercicio}\label{ej:2.7}
    El siguiente programa (en la Figura~\ref{fig:cod_7}) es una solución al problema de la exclusión mutua para 2 procesos.  Discutir la corrección de esta solución: si es correcta, entonces probarlo. Si no fuese correcta, escribir escenarios que demuestren que la solución es incorrecta.
    \begin{figure}[H]
        \centering
        \begin{minted}{pascal}
            {variables compartidas y valores iniciales}
            var c0 : integer := 1;
                c1 : integer := 1;
        \end{minted}
        \setlength{\columnsep}{1cm}
        \begin{multicols}{2}
            \begin{minted}{pascal}
                Process P0;
                begin
                while true do begin
                   repeat 
                      c0 := 1 - c1;
                   until c1 <> 0;

                   {Sección crítica}
                   c0 := 1;
                   {Resto de sentencias}
                end
                end
            \end{minted}
            \begin{minted}{pascal}
                Process P1;
                begin
                while true do begin
                   repeat 
                      c1 := 1 - c0;
                   until c0 <> 0;

                   {Sección crítica}
                   c1 := 1;
                   {Resto de sentencias}
                end
                end
            \end{minted}
        \end{multicols}
        \caption{Código para el Ejercicio~\ref{ej:2.7}.}
        \label{fig:cod_7}
    \end{figure}

    \ \\

    El algoritmo de la Figura~\ref{fig:cod_7} no es una solución correcta al problema de la exclusión mutua, debido a que la instrucción \verb|x := 1 - y| no es atómica, con lo que se podría suceder la traza de ejecución de P0 y P1 de la Tabla~\ref{tab:ej_7} (donde indicamos en cada caso la línea de la instrucción que se ejecuta, siendo \verb|l(5)| la lectura de \verb|c1| en P0 y \verb|e(5)| la escritura en \verb|c0| de P0, de forma análoga con P1), que nos lleva a tener tanto a P0 como a P1 en la sección crítica a la vez, con lo que no cumple con la propiedad de seguridad exigida. 

    \begin{table}[H]
    \centering
    \begin{tabular}{|c|c|c|c|}
        \hline
        Línea de P0 & Línea de P1 & \verb|c0| & \verb|c1| \\
        \hline
        4 & 4 & 1 & 1 \\
        \hline
        \verb|l(5)| & 4 & 1 & \textbf{1} \\
        \hline
        \verb|l(5)| & \verb|l(5)| & \textbf{1} & 1 \\
        \hline
        \verb|e(5)| & \verb|l(5)| & 0 & 1 \\
        \hline
        \verb|e(5)| & \verb|e(5)| & 0 & 0 \\
        \hline
        6 & 6 & 0 & 0 \\
        \hline
        \red{8} & \red{8} & 0 & 0 \\
        \hline
    \end{tabular}
    \caption{Traza de ejecución que lleva a tener dos procesos en SC.}
    \label{tab:ej_7}
    \end{table}
\end{ejercicio}

\begin{ejercicio}
    Con respecto al algoritmo de Peterson para $N$ procesos: ¿sería posible que llegaran 2 procesos a la etapa $N-2$, 0 procesos a la etapa $N-3$ y en todas las etapas anteriores existiera al menos 1 proceso? Justificar la respuesta.\\

    No sería posible, ya que contradiría el Lema 3 que demostramos en para demostrar las propiedades del algoritmo de Peterson para $N$ procesos:

    \textit{Si tenemos al menos dos procesos en la etapa $j$, entonces ha de haber, al menos, un proceso en cada etapa anterior.}
\end{ejercicio}

\begin{ejercicio}
    En el \textit{algoritmo de Peterson} para $N$ procesos y considerando cualquier escenario de ejecución de dicho algoritmo, el número máximo de turnos que tiene que esperar cualquier proceso para entrar en sección crítica es $N-1$ turnos.\\

    Falso, en la teoría vimos que el número máximo de turnos que tiene que esperar cualquier proceso para entrar en sección crítica es de $\frac{N(N-1)}{2}$ turnos.
\end{ejercicio}

\begin{ejercicio}\label{ej:2.10}
Con respecto al algoritmo de la Figura~\ref{fig:cod_10} (algoritmo de Dijkstra para $N$ procesos), demostrar la falsedad de la siguiente proposición:\newline

\begin{centering}
\textit{si un conjunto de procesos está intentando pasar simultáneamente el primer bucle} (el de la línea 11), \textit{y el proceso que tiene el turno está pasivo, entonces siempre conseguirá entrar primero en sección crítica el proceso de dicho grupo que consiga asignar la variable turno en último lugar.}
\end{centering}

    \begin{figure}
        \centering
        \begin{minted}{pascal}
            var turno : 0..N-1;
                flag : array[0..N-1] of (pasivo, solicitando, enSC);
            flag := pasivo:

            Process P(i);
            begin
               {Resto de instrucciones}
               repeat
                  flag[i] := solicitando;

                  while turno <> i do begin
                     if flag[turno] = pasivo then 
                        turno := i;
                  end

                  flag[i] := enSC;
                  j := 0;

                  while j < N and ((j=i) or flag[j] <> enSC) do begin
                     j := j + 1;
                  end
               until (j >= N);

               {Sección crítica}
               flag[i] := pasivo;
            end
        \end{minted}
        \caption{Código para el Ejercicio~\ref{ej:2.10}.}
        \label{fig:cod_10}
    \end{figure}
    \ \\

    Para demostrar la falsedad de la afirmación basta con dar una traza de ejecución en la que esta no se cumpla. Para ello, supongamos que tenemos dos procesos, P0 y P1 ejecutando el protocolo de entrada de la sección crítica, de forma que se da la siguiente trzada de ejecución:
    \begin{itemize}
        \item Ambos ejecutan la instrucción de la línea 9, cambiando su flag a \verb|solicitando|.
        \item Ambos consultan que su identificador (\verb|i|) es distinto del turno, con lo que ejecutan el bucle, viendo ambos que \verb|flag[turno] = pasivo|.
        \item Ambos modifican el valor de la variable compartida \verb|turno|, de forma que es el proceso P1 quien modifica dicha variable en último lugar.
        \item Mientras P1 terminaba de modificar la variable \verb|turno|, a P0 le da tiempo a cambiar su flag a \verb|enSC| y de ejecutar el bucle de la línea 19, con lo que le da tiempo a comprobar que ningún otro proceso tiene su flag a \verb|enSC|, saliendo del bucle con un valor de \verb|j = N|, lo que le permite entrar en la sección crítica.
        \item P0 y P1 intentaron pasar simultáneamente el primer bucle, con el proceso del turno en estado pasivo, P1 fue el último en asignar la variable \verb|turno| y al final P0 fue el primer proceso que consiguió entrar en la sección crítica.
    \end{itemize}
\end{ejercicio}

\begin{ejercicio}\label{ej:2.11}
    El algoritmo de la Figura~\ref{fig:cod_11} (algoritmo de Knuth para $N$ procesos) resuelve el problema de la exclusión mutua para $N$ procesos, pra lo cual utiliza $N$ variables booleanas \verb|flag|, una variable \verb|turn| y la variable local \verb|j|.
    \begin{enumerate}[label=(\alph*)]
        \item Demostrar que el algoritmo de Knuth verifica todas las propiedades exigibles a un programa concurrente, incluyendo la de equidad.
        \item Escribir un escenario en el que 2 procesos consiguen pasar el bucle de la instrucción de la línea 13, suponiendo que el turno lo tiene inicialmente el proceso \verb|p(0)|.
    \end{enumerate}

    \begin{figure}
        \centering
        \begin{minted}{pascal}
            var flag : array[0..N-1] of (pasivo, solicitando, enSC);
                turn := 0..N-1;
            flag := pasivo;
            turn := 0;

            Process P(i);
            var j : integer;
            begin
               {Resto de instrucciones}
               repeat
                  flag[i] := solicitando;
                  j := turn;
                  while j <> i do begin
                     if flag[j] <> pasivo then
                        j := turn;
                     else
                        j := (j - 1) mod N;
                     endif;
                  end

                  flag[i] := enSC;
                  j := 0;

                  while (j < N) and ((j=i) or flag[j] <> enSC) do begin
                     j := j + 1;
                  end
               until (j >= N);

               turn := i;
               {Seccion crítica}
               j := (turn + 1) mod N;
               turn := j;
               flag[i] := pasivo;
            end
        \end{minted}
        \caption{Código para el Ejercicio~\ref{ej:2.11}}
        \label{fig:cod_11}
    \end{figure}
    \begin{enumerate}[label=(\alph*)]
        \item Todas estas propiedades fueron ya demostradas en los apuntes de teoría.
        \item Para ello, describiremos a continuación un escenario en el que los procesos P1 y P2 consiguen pasar el bucle de la línea 13 (es decir, llegar a la línea 21), suponiendo que \verb|turno = 0|. El escenario es el contemplado en la Tabla~\ref{tab:ej_10}, donde suponemos que en todo momento el proceso P0 está pasivo, es decir, \verb|flag[0] = pasivo|.

            \begin{table}[H]
            \centering
            \begin{tabular}{|c|c|c|c|}
                \hline
                Línea de P1 & Línea de P2 & \verb|j|$_1$ & \verb|j|$_2$ \\
                \hline
                12 & 12 & 0 & 0 \\
                \hline
                13 & 13 & 0 & 0 \\
                \hline
                14 & 14 & 0 & 0 \\
                \hline
                17 & 14 & 2 & 0 \\
                \hline
                13 & 14 & 2 & 0 \\
                \hline
                14 & 14 & 2 & 0 \\
                \hline
                17 & 14 & 1 & 0 \\
                \hline
                13 & 17 & 1 & 2 \\
                \hline
                21 & 13 & 1 & 2 \\
                \hline
                21 & 21 & 1 & 2 \\
                \hline
            \end{tabular}
            \caption{Traza de ejecución en la que 2 procesos pasan el bucle de la línea 13.}
            \label{tab:ej_10}
            \end{table}
    \end{enumerate}
\end{ejercicio}

\newpage
\begin{ejercicio}
    Si en el algoritmo de Dijkstra de la Figura~\ref{fig:cod_10} se cambia la instrucción de la línea 12 por esta otra: \verb|if (flag[turno] <> enSC)|, entonces el algoritmo dejaría de ser correcto. Indicar qué propiedad(es) de corrección faltaría(n) y justificar por qué.\\

    El algoritmo de Dijkstra tiene las propiedades de garantizar la exclusión mutua y de alcanzabilidad. Si observamos la demostración de la propiedad de exclusión mutua que hicimos en teoría, esta no dependía del bucle de la línea 11, con lo que su modificación no dará lugar a que más de un proceso ejecute la sección crítica al mismo tiempo.\\

    Por tanto, podemos pensar que la única propiedad de corrección que faltaría sería la de alcanzabilidad de la sección crítica. Para ver que esta no se da, supongamos un escenario en el que tenemos dos procesos, P0 y P1 ejecutando el protocolo de entrada a la sección crítica del código modificado, de forma que se da la traza de ejecución de la Tabla~\ref{tab:ej_12}. Si la traza del programa continúa siendo las que aparecen entre las dos instruccioens destacadas, estas se repetirían de forma infinita, dándose un interbloqueo entre los dos procesos, ya que después de que uno cambie el turno, el otro proceso comprueba que no tiene el turno y lo cambia antes de que el que lo cambió vea que el turno es igual a su identificador. 

    Al darse una situación de interbloqueo, el algoritmo no cuenta con la propiedad de alcanzabilidad.
    \begin{table}[H]
    \centering
    \begin{tabular}{|c|c|c|}
        \hline
        Línea de P0 & Línea de P1 & turno \\
        \hline
        9 & 9 & 2 \\
        \hline
        11 & 11 & 2 \\
        \hline
        12 & 11 & 2 \\
        \hline
        \textbf{13} & \textbf{11} & \textbf{0} \\
        \hline
        13 & 12 & 0 \\
        \hline
        13 & 13 & 1 \\
        \hline
        11 & 13 & 1 \\
        \hline
        12 & 13 & 1 \\
        \hline
        13 & 13 & 0 \\
        \hline
        \textbf{13} & \textbf{11} & \textbf{0} \\
        \hline
    \end{tabular}
    \caption{Traza de ejecución que da lugar a un interbloqueo.}
    \label{tab:ej_12}
    \end{table}
\end{ejercicio}

\begin{ejercicio}
    Si en el algoritmo de Knuth de la Figura~\ref{fig:cod_11} se hacen las siguientes sustituciones:
    \begin{itemize}
        \item La condición de la instrucción \verb|until| de la línea 27 por la condición \newline \verb|(j >= N) and (turno = i or flag[turno] = pasivo)|.
        \item Se inserta el siguiente bucle después de la instrucción de la línea 31:
            \begin{minted}{pascal}
                while (j <> turn) and (flag[j] = pasivo) do begin
                   j := j + 1;
                end
            \end{minted}
    \end{itemize}
    \begin{enumerate}[label=(\alph*)]
        \item Verificar las propiedades de exclusión mutua, alcanzabilidad de la sección crítica, vivacidad y equidad del algoritmo.
        \item Calcular el número de turnos máximo que puede llegar a tener que esperar un proceso que quiera entrar en su sección crítica con el algoritmo anterior.
    \end{enumerate}
\end{ejercicio}

\begin{ejercicio}\label{ej:2.14} % // TODO:
    Demostrar que las instrucciones entre las líneas 18 y 28 del algoritmo de exclusión mutua distribuido de Ricart-Agrawala (de la Figura~\ref{fig:cod_14}) no necesitan ser protegidas dentro de la sección crítica definida por las operaciones \verb|wait()|, \verb|signal()| del semáforo \verb|s|.
    \begin{figure}[H]
        \centering
        \begin{minted}{pascal}
            var token_presente : boolean := false;
                enSC : boolean := false;
                peticion : array[1..n] of boolean := false;
        \end{minted}
        \setlength{\columnsep}{1cm}
        \begin{multicols}{2}
            \begin{minted}{pascal}
                Process P(i);
                begin
                   wait(s);
                   if not token_presente then begin
                      broadcast(pet, i);
                      receive(acceso);
                      token_presente := true;
                   end

                   enSC := true;
                   signal(s);

                   {Sección crítica}

                   enSC := false;
                   wait(s);

                   for j := i+1 to n, 1 to i-1 do
                      if peticion[j] and 
                         token_presente then begin

                         token_presente := false;
                         send(j,acceso);
                         peticion[j] := false;
                      end
                   end

                   signal(s);
                end
            \end{minted}
            \begin{minted}{pascal}
                Process Pet(i);
                begin
                   receive(pet, j);
                   wait(s);
                   peticion[j] := true;
                   if token_presente and not enSC then
                      {Repetir líneas 18-28}
                end
            \end{minted}
        \end{multicols}
        \caption{Código para el Ejercicio~\ref{ej:2.14}}
        \label{fig:cod_14}
    \end{figure}
\end{ejercicio}

\begin{ejercicio}\label{ej:2.15} % // TODO:
    Suponer que el algoritmo de Suzuki-Kasami para resolver el problema de la exclusión mutua distribuida para $n$-procesos se modifica como aparece en la siguiente figura. Explicar por qué dejaría de ser correcto el algoritmo, relacionándolo con cada una de las propiedades de corrección que se demuestran para el algoritmo original.
    \begin{figure}
        \centering
        \begin{minted}{pascal}
            var token_presente : boolean := false;
                enSC : boolean := false;
                peticion : array[1..n] of boolean := false;
            {En el algoritmo original -> peticion : array[1..n] of 0..+INF}
            {además se declara otro array -> token : array[1..n] of 0..+INF}
        \end{minted}
        \setlength{\columnsep}{1cm}
        \begin{multicols}{2}
            \begin{minted}{pascal}
                Process P(i);
                begin
                   wait(s);
                   if not token_presente then begin
                      broadcast(pet, i);
                      receive(acceso);
                      token_presente := true;
                   end

                   enSC := true;
                   signal(s);

                   { SC }

                   enSC := false;
                   wait(s);

                   for j := i+1 to n, 1 to i-1 do
                      if peticion[j] and
                         token_presente then begin
                         token_presente := false;
                         send(j, acceso);
                         peticion[j] := false;
                      end
                   end
                   signal(s);
                end
            \end{minted}
            \begin{minted}{pascal}
                Process Pet(i);
                begin
                   receive(pet, j);
                   wait(s);
                   peticion[j] := true;
                   if token_presente and
                      not enSC then begin
                      for j := i+1 to n, 1 to i-1 do
                         if peticion[j] and
                            token_presente then begin
                            token_presente := false;
                            send(j, acceso);
                            peticion[j] := false;
                         end
                      end
                      signal(s);
                   end
                end
            \end{minted}
        \end{multicols}
        \caption{Código para el Ejercicio~\ref{ej:2.15}}
        \label{fig:cod_15}
    \end{figure}
\end{ejercicio}

    \include{Rel_Tema2Parte2}
    \section{Paso de mensajes}

\begin{ejercicio}\label{ej:rel3_1}
    En un sistema distribuido, 6 procesos clientes necesitan sincronizarse de forma específica para realizar cierta tarea, de forma que dicha tarea sólo podrá ser realizada cuando tres procesos estén preparados para realizarla. Para ello, envían peticiones a un proceso controlador del recurso y esperan respuesta para poder realizar la tarea específica. El proceso controlador se encarga de asegurar la sincronización adecuada. Para ello, recibe y cuenta las peticiones que le llegan de los procesos, las dos primeras no son respondidas y producen la suspensión del proceso que envía la petición (debido a que se bloquea esperando respuesta) pero la tercera petición produce el desbloqueo de los tres procesos pendientes de respuesta. A continuación, una vez desbloqueados los tres procesos que han pedido (al recibir respuesta), inicializa la cuenta y procede cíclicamente de la misma forma sobre otras peticiones. El código de los procesos clientes aparece aquí abajo. Los clientes usan envío asíncrono seguro para realizar su petición, y esperan con una recepción síncrona antes de realizar la tarea:
    \begin{figure}[H]
        \centering
            \begin{minted}{pascal}
                process Cliente[ i : 0..5 ];
                begin
                   while true do begin
                      send(peticion, Controlador);
                      receive(permiso, Controlador);
                      Realiza_tarea_grupal();
                   end
                end
            \end{minted}
        \caption{Código para el Ejercicio~\ref{ej:rel3_1}.}
        \label{fig:cod_1}
    \end{figure}
    Describir en pseudocódigo el comportamiento del proceso controlador, utilizando una orden de espera selectiva que permita implementar la sincronización requerida entre los procesos. Es posible utilizar una sentencia del tipo \verb|select for i=... to ...| para especificar diferentes ramas de una sentencia selectiva que comparten el mismo código dependiente del valor de un índice \verb|i|.\\

    \begin{minted}{pascal}
        process Controlador;
        var contador : integer := 0;
            necesarios : integer := 3;
            esperando : array[0..1] of 0..5;
        begin
           while true do begin
              select
                 for i := 0 to 5
                    when receive(valor, Cliente[i]) do
                       if contador < necesarios-1 then begin
                          esperando[contador] := i;
                          contador := contador + 1;
                       else
                          contador := 0;
                          send(valor, Cliente[i]);
                          send(valor, Cliente[esperando[0]]);
                          send(valor, Cliente[esperando[1]]);
                       end
                    end
              end select
           end do
        end
    \end{minted}
\end{ejercicio}

\begin{ejercicio}\label{ej:rel3_2}
    En un sistema distribuido, 3 procesos productores producen continuamente valores enteros y los envían a un proceso buffer que los almacena temporalmente en un array local de 4 celdas enteras para ir enviándoselos a un proceso consumidor. A su vez, el proceso buffer realiza lo siguiente, sirviendo de forma equitativa al resto de procesos:
    \begin{enumerate}[label=(\alph*)]
        \item Envía enteros al proceso consumidor siempre que su array local tenga al menos dos elementos disponibles.
        \item Acepta envíos de los productores mientras el array no esté lleno, pero no acepta que cualquier productor pueda escribir dos veces consecutivas en el búfer.
    \end{enumerate}
    El código de los procesos productor y consumidor es el siguiente, asumiendo que se usan operaciones síncronas:
    \begin{figure}[H]
        \centering
        \setlength{\columnsep}{1cm}
        \begin{multicols}{2}
            \begin{minted}{pascal}
                process Productor [ i : 0..2 ];
                var dato : integer;
                begin
                   while true do begin
                      dato := Producir();
                      send(dato, Buffer);
                   end
                end
            \end{minted}
            \begin{minted}{pascal}
                process Consumidor;
                begin
                   while true do begin
                      receive(dato, Buffer);
                      Consumir(dato);
                   end
                end
            \end{minted}
        \end{multicols}
        \caption{Código para el Ejercicio~\ref{ej:rel3_2}.}
        \label{fig:cod_2}
    \end{figure}
    Describir en pseudocódigo el comportamiento del proceso \verb|Buffer|, utilizando una orden de espera selectiva que permita implementar la sincronización requerida entre los procesos.\\

    \begin{minted}{pascal}
        process Buffer;
        var buffer : array[0..3] of integer;
            primera_libre, primera_ocupada : integer := 0, 0;
            ocupadas : integer := 0;
            ult_productor : integer := -1;
            dato : integer;
        begin
           while true do begin
              select 
                 when ocupadas >= 2 do
                    dato := buffer[primera_ocupada];
                    primera_ocupada := (primera_ocupada + 1) mod 4;
                    ocupadas := ocupadas - 1;
                    send(dato, Consumidor);
                 end
                 for i := 0 to 2
                    when ocupadas < 4 and i <> ult_productor receive(dato, Productor[i])
                       ult_productor := i;
                       buffer[primera_libre] := dato;
                       primera_libre := (primera_libre + 1) mod 4;
                       ocupadas := ocupadas + 1;
                    end
              end
           end
        end
    \end{minted}
\end{ejercicio}

\begin{ejercicio}\label{ej:rel3_3}
    Suponer un proceso productor y 3 procesos consumidores que comparten un buffer acotado de tamaño \verb|B|. Cada elemento depositado por el proceso productor debe ser retirado por todos los 3 procesos consumidores para ser eliminado del buffer. Cada consumidor retirará los datos del buffer en el mismo orden en el que son depositados, aunque los diferentes consumidores pueden ir retirando los elementos a ritmo diferente unos de otros. Por ejemplo, mientras un consumidor ha retirado los elementos 1, 2 y 3, otro consumidor puede haber retirado solamente el elemento 1. De esta forma, el consumidor más rápido podría retirar hasta B elementos más que el consumidor más lento. Describir en pseudocódigo el comportamiento de un proceso que implemente el buffer de acuerdo con el esquema de interacción descrito usando una construcción de espera selectiva, así como el del proceso productor y de los procesos consumidores. Comenzar identificando qué información es necesario representar, para después resolver las cuestiones de sincronización.
    Una posible implementación del buffer mantendría, para cada proceso consumidor, el puntero de salida y el número de elementos que quedan en el buffer por consumir:

    \begin{figure}[H]
        \centering
    \begin{tikzpicture}
        % Rectángulo principal
        \draw (0, 0) rectangle (8, 1);

        % Divisiones internas del rectángulo
        \foreach \x in {1, 2, 3, 4, 5, 6, 7} {
            \draw (\x, 0) -- (\x, 1);
        }

        % Números dentro de las cajas
        \node at (0.5, 0.5) {1};
        \node at (1.5, 0.5) {2};
        \node at (2.5, 0.5) {3};
        \node at (3.5, 0.5) {4};
        \node at (4.5, 0.5) {5};

        % Tachar el 1
        \draw[thick] (0.2, 0.6) -- (0.8, 0.4);

        % Flecha y etiquetas: out[1]
        \draw[-Stealth] (1.5, -0.6) -- (1.5, -0.1);
        \node[below] at (1.5, -0.6) {\verb|out[1]|};

        % Flecha y etiquetas: out[3]
        \draw[-Stealth] (3.5, -0.6) -- (3.5, -0.1);
        \node[below] at (3.5, -0.6) {\verb|out[3]|};

        % Flecha y etiquetas: out[2]
        \draw[-Stealth] (5.5, -0.6) -- (5.5, -0.1);
        \node[below] at (5.5, -0.6) {\verb|out[2]|};

        % Flecha y etiqueta "in" por encima
        \draw[-Stealth] (5.5, 1.6) -- (5.5, 1.1);
        \node[above] at (5.5, 1.6) {\verb|in|};

     % Texto a la derecha
        \node[right] at (8.5, 1) {\verb|nElems[1] = 4|};
        \node[right] at (8.5, 0.5) {\verb|nElems[2] = 0|};
        \node[right] at (8.5, 0) {\verb|nElems[3] = 2|};
    \end{tikzpicture}        
    \caption{Dibujo para el Ejercicio~\ref{ej:rel3_3}.}
    \label{fig:fig_ej_3}
    \end{figure}

    En primer lugar, describimos los códigos de los procesos productor y consumidores, por ser estos mucho más fáciles que el del proceso intermedio que usaremos para comunicar ambos tipos de procesos, al no requerir estos de sincronización ninguna. Tanto productor como consumidores mandan mensajes al Buffer, y los cosumidores esperan una respuesta del mismo.
    \begin{figure}[H]
        \centering
    \setlength{\columnsep}{1cm}
    \begin{multicols}{2}
        \begin{minted}{pascal}
           process Productor;
           var dato : integer;
           begin
              dato := Producir();
              send(dato, Buffer);
           end
        \end{minted}
        \begin{minted}{pascal}
            process Consumidor[ i : 0..2 ];
            var dato : integer;
            begin
               send(peticion, Buffer);
               receive(dato, Buffer);
               Consumir(dato);
            end
        \end{minted}
    \end{multicols}
    \end{figure}
    Ahora, desarrollamos el código del proceso Buffer, donde \verb|out[i]| indica la siguiente posición a leer del consumidor \verb|i|-ésimo y \verb|nElems[i]| indica la cantidad de datos que quedan por leer al consumidor \verb|i|-ésimo. 
    \begin{minted}{pascal}
        process Buffer;
        var buffer : array[0..B-1] of integer;
            ocupados : integer := 0;
            in : 0..B-1 := 0;
            out : array[0..2] of 0..B-1 := (0, 0, 0);
            nElems : array[0..2] of 0..B-1 := (0, 0, 0);
            dato : integer;
        begin
           while true do begin
              select 
                 when ocupados < B receive(dato, Productor) do
                    buffer[in] := dato;
                    in := (in + 1) mod B;
                    ocupados := ocupados + 1;
                    
                    for i:= 0 to 2 do
                       nElems[i] := nElems[i] + 1;
                 end
                 for i := 0 to 2
                    when nElems[i] > 0 receive(peticion, Consumidor[i]) do
                       dato := buffer[out[i]];
                       out[i] := out[i] + 1 mod B;

                       { Índices de los otros consumidores }
                       j := (i+1) mod 3;
                       k := (j+1) mod 3;

                       { Si nElems es el mayor, era el último en consumir }
                       if nElems[i] > nElems[j] and nElems[i] > nElems[k] then
                          ocupados := ocupados - 1;
                       end

                       nElems[i] := nElems[i] - 1;
                       send(dato, Consumidor[i]);
                    end
              end
           end
        end
    \end{minted}
\end{ejercicio}

\begin{ejercicio}\label{ej:rel3_4}
    Una tribu de antropófagos comparte una olla en la que caben \verb|M| misioneros. Cuando algún salvaje quiere comer, se sirve directamente de la olla, a no ser que ésta esté vacía. Si la olla está vacía, el salvaje despertará al cocinero y esperará a que éste haya rellenado la olla con otros \verb|M| misioneros.
    \begin{figure}[H]
        \centering
        \setlength{\columnsep}{1cm}
        \begin{multicols}{2}
            \begin{minted}{pascal}
                process Salvaje[ i : 0..2 ];
                var peticion : integer := ... ;
                begin
                   while true do begin
                      { esperar a servirse un misionero }
                      ...
                      s_send(peticion, Olla);
                      { comer }
                      Comer();
                   end
                end
            \end{minted}
            \begin{minted}{pascal}
                process Cocinero;
                begin
                   while true do begin
                      { dormir esperando solicitud para rellenar }
                      ...
                      {confirmar que se ha rellenado la olla}
                      ...
                   end
                end
            \end{minted}
        \end{multicols}
        \caption{Código para el Ejercicio~\ref{ej:rel3_4}.}
        \label{fig:cod_4}
    \end{figure}
    Implementar los procesos salvajes y cocinero usando paso de mensajes, usando un proceso olla que incluye una construcción de espera selectiva que sirve peticiones de los salvajes y el cocinero para mantener la sincronización requerida, teniendo en cuenta que:
    \begin{itemize}
        \item La solución no debe producir interbloqueo.
        \item Los salvajes podrán comer siempre que haya comida en la olla.
        \item Solamente se despertará al cocinero cuando la olla esté vacía.
    \end{itemize}
    Mostramos primero los códigos para los procesos salvajes y cocinero:
    \begin{figure}[H]
        \centering
        \setlength{\columnsep}{1cm}
        \begin{multicols}{2}
            \begin{minted}{pascal}
                process Salvaje[ i : 0..2 ];
                begin
                   while true do begin
                      s_send(peticion, Olla);
                      Comer();
                   end
                end
            \end{minted}
            \begin{minted}{pascal}
                process Cocinero;
                begin
                   while true do begin
                      receive(peticion, Olla);
                      send(rellenar, Olla);
                   end
                end
            \end{minted}
        \end{multicols}
    \end{figure}
    A continuación, mostramos el código del proceso Olla:
    \begin{minted}{pascal}
        process Olla;
        var misioneros : integer := 0;
        begin
           while true do begin
              select 
                 when misioneros = 0 do
                    send(valor, Cocinero);
                    receive(valor, Cocinero);
                    misioneros := M;
                 end
                 for i := 0 to 2 
                    when misioneros > 0 receive(valor, Salvaje[i]) do
                       misioneros := misioneros - 1;
                    end
              end
           end
        end
    \end{minted}
\end{ejercicio}

\begin{ejercicio}\label{ej:rel3_5}
    Considerar un conjunto de \verb|N| procesos, \verb|P[i]|, ($i = 0, \ldots, N-1$) que se pasan mensajes cada uno al siguiente (y el primero al último), en forma de anillo. Cada proceso tiene un valor local almacenado en su variable local \verb|mi_valor|. Deseamos calcular la suma de los valores locales almacenados por los procesos de acuerdo con el algoritmo que se expone a continuación.
    \begin{figure}[H]
        \centering

        \setlength{\columnsep}{1cm}
        \begin{multicols}{2}
            
        \begin{tikzpicture}[align=center]
            % Paso 1
            \node[draw, rectangle] (a1) at (0, 0) {mi\_valor = 0\\ suma = 0};
            \node[draw, rectangle, right=of a1] (b1) {mi\_valor = 1\\ suma = 1};
            \node[draw, rectangle, below=of b1] (c1) {mi\_valor = 2\\ suma = 2};
            \node[draw, rectangle, left=of c1] (d1) {mi\_valor = 3\\ suma = 3};

            \draw[-Stealth] (a1) -- (b1) node[midway, above] {0};
            \draw[-Stealth] (b1) -- (c1) node[midway, right] {1};
            \draw[-Stealth] (c1) -- (d1) node[midway, below] {2};
            \draw[-Stealth] (d1) -- (a1) node[midway, left] {3};

            \node[above=of a1] {Paso 1};

            % Paso 3
            \node[draw, rectangle, below=2cm of d1] (a3) {suma = 5};
            \node[draw, rectangle, right=of a3] (b3) {suma = 4};
            \node[draw, rectangle, below=of b3] (c3) {suma = 3};
            \node[draw, rectangle, left=of c3] (d3) {suma = 6};

            \draw[-Stealth] (a3) -- (b3) node[midway, above] {2};
            \draw[-Stealth] (b3) -- (c3) node[midway, right] {3};
            \draw[-Stealth] (c3) -- (d3) node[midway, below] {0};
            \draw[-Stealth] (d3) -- (a3) node[midway, left] {1};

            \node[above=of a3] {Paso 3};
        \end{tikzpicture}

        \begin{tikzpicture}[align=center]

            % Paso 2
            \node[draw, rectangle] (a2) {suma = 3};
            \node[draw, rectangle, right=of a2] (b2) {suma = 1};
            \node[draw, rectangle, below=of b2] (c2) {suma = 3};
            \node[draw, rectangle, left=of c2] (d2) {suma = 5};

            \draw[-Stealth] (a2) -- (b2) node[midway, above] {3};
            \draw[-Stealth] (b2) -- (c2) node[midway, right] {0};
            \draw[-Stealth] (c2) -- (d2) node[midway, below] {1};
            \draw[-Stealth] (d2) -- (a2) node[midway, left] {2};

            \node[above=of a2] {Paso 2};

            % Paso 4
            \node[draw, rectangle, below=3cm of d2] (a4) {suma = 6};
            \node[draw, rectangle, right=of a4] (b4) {suma = 6};
            \node[draw, rectangle, below=of b4] (c4) {suma = 6};
            \node[draw, rectangle, left=of c4] (d4) {suma = 6};

            \node[above=of a4] {Paso 4};
        \end{tikzpicture}
        \end{multicols}
        \caption{Dibujo para el Ejercicio~\ref{ej:rel3_5}.}
        \label{fig:fig_ej_5}
    \end{figure}
    Los procesos realizan una serie de iteraciones para hacer circular sus valores locales por el anillo. En la primera iteración, cada proceso envía su valor local al siguiente proceso del anillo, al mismo tiempo que recibe del proceso anterior el valor local de éste. A continuación acumula la suma de su valor local y el recibido desde el proceso anterior. En las siguientes iteraciones, cada proceso envía al siguiente proceso siguiente el valor recibido en la anterior iteración, al mismo tiempo que recibe del proceso anterior un nuevo valor. Después acumula la suma. Tras un total de \verb|N - 1| iteraciones, cada proceso conocerá la suma de todos los valores locales de los procesos. Dar una descripción en pseudocódigo de los procesos siguiendo un estilo SPMD y usando operaciones de envío y recepción síncronas:
    \begin{minted}{pascal}
        process P[ i : 0..N-1 ];
        var mi_valor : integer := ...; {valor aleatorio, igual a i en la figura}
            suma : integer := mi_valor;
        begin
           for j := 0 to N-1 do begin
              ...
           end
        end
    \end{minted}
    La solución la podemos encontrar en el siguiente código, donde debemos tener cuidado de que no todos los procesos primero envíen y luego reciban, ya que esto puede dar lugar a una situación de interbloqueo por usar envíos y recepciones síncronas. Para ello, obligamos a que al menos un proceso primero reciba y luego envíe, o viceversa:
    \begin{minted}{pascal}
        process P[ i : 0..N-1 ];
        var mi_valor : integer := ...; {valor aleatorio, igual a i en la figura}
            suma : integer := mi_valor;
            siguiente : 0..N-1 := (i+1) mod N;
            anterior : 0..N-1 := (i-1) mod N;
            recibido : integer;
        begin
           for j := 0 to N-1 do begin
              if i == 0 then begin
                 send(mi_valor, siguiente);
                 recv(recibido, anterior);
              else
                 recv(recibido, anterior);
                 send(mi_valor, siguiente);
              end
              mi_valor := recibido;
              suma += recibido;
           end
        end
    \end{minted}
\end{ejercicio}

\begin{ejercicio}\label{ej:rel3_6}
    Considerar un estanco en el que hay tres fumadores y un estanquero. Cada fumador continuamente lía un cigarro y se lo fuma. Para liar un cigarro, el fumador necesita tres ingredientes: tabaco, papel y cerillas. Uno de los fumadores tiene solamente papel, otro tiene solamente tabaco, y el otro tiene solamente cerillas. El estanquero tiene una cantidad infinita de los tres ingredientes.
    \begin{itemize}
        \item El estanquero coloca aleatoriamente dos ingredientes diferentes de los tres que se necesitan para hacer un cigarro, desbloquea al fumador que tiene el tercer ingrediente y después se bloquea. El fumador seleccionado, se puede obtener fácilmente mediante una función \verb|genera_ingredientes| que devuelve el índice (0, 1, ó 2) del fumador escogido.
        \item El fumador desbloqueado toma los dos ingredientes del mostrador, desbloqueando al estanquero, lía un cigarro y fuma durante un tiempo.
        \item El estanquero, una vez desbloqueado, vuelve a poner dos ingredientes aleatorios en el mostrador, y se repite el ciclo.
    \end{itemize}
    Describir una solución distribuida que use envío asíncrono seguro y recepción síncrona, para este problema usando un proceso Estanquero y tres procesos fumadores \verb|Fumador(i)| (con $i=0, 1, 2$).

    \begin{figure}[H]
        \centering
            \begin{minted}{pascal}
                process Estanquero;
                var ing : array[0..1] of ingredientes;
                begin
                   while true do begin
                      { Genera dos ingredientes distintos }
                      ing := generaIngredientes();

                      select
                         for i:=0 to 2
                            when genera_ingredientes(ing) do
                               send(ing, Fumador[i]);
                               receive(confirmacion, Fumador[i]);
                            end
                      end select
                   end
                end
            \end{minted}
            \begin{minted}{pascal}
                process Fumador[ i : 0..2 ];
                var ingredientes : array[0..1] of ingredientes;
                begin
                   while true do begin
                      recv(ingredientes, Estanquero);
                      send(confirmacion, Estanquero);
                      fumar(ingredientes);
                   end
                end
            \end{minted}
        \caption{Código para el Ejercicio~\ref{ej:rel3_6}.}
        \label{fig:cod_6}
    \end{figure}
\end{ejercicio}

\begin{ejercicio}\label{ej:rel3_7}
    En un sistema distribuido, un gran número de procesos clientes usa frecuentemente un determinado recurso y se desea que puedan usarlo simultáneamente el máximo número de procesos. Para ello, los clientes envían peticiones a un proceso controlador para usar el recurso y esperan respuesta para poder usarlo (véase el código de los procesos clientes). Cuando un cliente termina de usar el recurso, envía una solicitud para dejar de usarlo y espera respuesta del Controlador. El proceso controlador se encarga de asegurar la sincronización adecuada imponiendo una única restricción por razones supersticiosas: nunca habrá 13 procesos exactamente usando el recurso al mismo tiempo.
    \begin{figure}[H]
        \centering
            \begin{minted}{pascal}
                process Cli[ i : 0..n ];
                var pet_usar : integer := +1;
                    pet_liberar : integer := -1;
                    permiso : integer := ...;
                begin
                   while true do begin
                      send(pet_usar, Controlador);
                      receive(permiso, Controlador);
                      Usar_recurso();
                      send(pet_liberar, Controlador);
                      receive(permiso, Controlador);
                   end
                end
            \end{minted}
        \caption{Código para el Ejercicio~\ref{ej:rel3_7}.}
        \label{fig:cod_7}
    \end{figure}
    Describir en pseudocódigo el comportamiento del proceso controlador, utilizando una orden de espera selectiva que permita implementar la sincronización requerida entre los procesos. Es posible utilizar una sentencia del tipo \verb|select for i=... to ...| para especificar diferentes ramas de una sentencia selectiva que comparten el mismo código dependiente del valor de un índice \verb|i|.\\

    La solución que planteamos es la siguiente, donde tenemos en cuenta las peticiones de obtención y liberación del recurso que llevan al recurso a ser usado por 13 procesos al mismo tiempo. En dicho caso, guardamos la petición hasta que haya cualquiera otra petición (que puede ser del mismo tipo o distinto).
    \begin{minted}{pascal}
        process Controlador
        var contador : integer := 0;
            pendiente : integer := 0;
            id_pendiente : integer;
            peticion : integer;
            permiso : integer := 100;
        begin
           while true do begin
              select 
                 for i:= 0 to n
                    when receive(peticion, Cli[i]) do
                       { Si no nos sirve el estado al que llega }
                       if contador + pendiente + peticion = 13 then begin
                          { Sabemos que pendiente = 0 }
                          pendiente := peticion;
                          id_pendiente := i;
                       else { Si no, se procesa la petición }
                          contador := contador + pendiente + peticion;
                          send(permiso, Cli[i]);

                          { Si había una pendiente también se acepta }
                          if pendiente <> 0 then begin
                             send(permiso, Cli[id_pendiente]);
                             pendiente := 0;
                          end
                       end
                    end
              end select
           end
        end
    \end{minted}
    
\end{ejercicio}

\begin{ejercicio}\label{ej:rel3_8}
    En un sistema distribuido, tres procesos \verb|Productor| se comunican con un proceso \verb|Impresor| que se encarga de ir imprimiendo en pantalla una cadena con los datos generados por los procesos productores. Cada proceso productor (\verb|Productor[i]|, con $i=0,1,2$) genera continuamente el correspondiente entero \verb|i|, y lo envía al proceso Impresor.

    El proceso Impresor se encarga de ir recibiendo los datos generados por los productores y los imprime por pantalla (usando el procedimiento \verb|imprime(entero)|) generando una cadena de dígitos en la salida. No obstante, los procesos se han de sincronizar adecuadamente para que la impresión por pantalla cumpla las siguientes restricciones:
    \begin{itemize}
        \item Los dígitos 0 y 1 deben aceptarse por el impresor de forma alterna. Es decir, si se acepta un 0 no podrá volver a aceptarse un 0 hasta que se haya aceptado un 1, y viceversa, si se acepta un 1 no podrá volver a aceptarse un 1 hasta que se haya aceptado un 0.
        \item El número total de dígitos 0 o 1 aceptados en un instante no puede superar el doble de número de digitos 2 ya aceptados en dicho instante.
    \end{itemize}
    Cuando un productor envía un digito que no se puede aceptar por el imprersor, el productor quedará bloqueado esperando completar el \verb|s_send|. El pseudocódigo de los procesos productores (\verb|Productor|) se muestra a continuación , asumiendo que se usan operaciones bloqueantes no buferizadas (síncronas):
    \begin{figure}[H]
        \centering
            \begin{minted}{pascal}
                process Productor[ i : 0,1,2 ];
                begin
                   while true do begin
                      s_send(i, Impresor);
                   end
                end
            \end{minted}
        \caption{Código para el Ejercicio~\ref{ej:rel3_8}.}
        \label{fig:cod_8}
    \end{figure}
    Escribir en pseudocódigo el código del proceso \verb|Impresor|, utilizando para ello un bucle infinito con una orden de espera selectiva \verb|select| que permita implementar la sincronización requerida entre los procesos, según el esquema anterior.\\

    El código solicitado es el siguiente, donde \verb|ult0| indica:
    \begin{itemize}
        \item \verb|true| si el último dígito 0 o 1 recibido fue un 0.
        \item \verb|false| si el último dígito 0 o 1 recibido fue un 1.
    \end{itemize}
    Además, notemos que el ejercicio no impone restricciones sobre cuando se pueden recibir dígitos 2.
    \begin{minted}{pascal}
        process Impresor
        var cant0, cant1, cant2 : integer := 0, 0, 0;
            ult0 : boolean := true;
            n : 0..2;
        begin
           while true do begin
              select
                 when (cant0 < 2*cant2-1 and not ult0) receive(n, Productor[0]) do
                    cant0 := cant0 + 1;
                    ult0 := true;
                 end

                 when (cant1 < 2*cant2-1 and ult0) receive(n, Productor[1]) do
                    cant1 := cant1 + 1;
                    ult0 := false;
                 end

                 when receive(n, Productor[2]) do
                    cant2 := cant2 + 1;
                 end
              end

              imprime(n);
           end
        end
    \end{minted}
\end{ejercicio}

\begin{ejercicio}\label{ej:rel3_9}
   En un sistema distribuido hay un vector de \verb|n| procesos iguales que envían con \verb|send| (en un bucle infinito) valores enteros a un proceso receptor, que los imprime. Si en algún momento no hay ningún mensaje pendiente de recibir en el receptor, este proceso debe de imprimir ``no hay mensajes, duermo''; después de bloquearse durante 10 segundos (con \verb|sleep_for(10)|), antes de volver a comprobar si hay mensajes (esto podría hacerse para ahorrar energía, ya que el procesamiento de mensajes se hace en ráfagas separadas por 10 segundos). Este problema no se puede solucionar usando \verb|receive| o \verb|i_receive|. Indica a que se debe esto. Sin embargo, sí se puede hacer con \verb|select|. Diseña una solución a este problema con \verb|select|:
    \begin{figure}[H]
       \centering
           \begin{minted}{pascal}
               process Emisor[ i : 1..n ];
               var dato : integer;
               begin
                  while true do begin
                     dato := Producir();
                     send(dato, Receptor);
                  end
               end
           \end{minted}
       \caption{Código para el Ejercicio~\ref{ej:rel3_9}.}
       \label{fig:cod_9}
   \end{figure}
   El problema no puede resolverse con instrucciones \verb|receive| o \verb|i_receive| porque:
   \begin{itemize}
       \item En el caso de la instrucción \verb|receive|, si no hay mensajes pendientes, el proceso se bloquearía hasta recibir el primero, pero este no es el comportamiento deseado.
       \item En el caso de la instrucción \verb|i_receive|, comenzaría instantáneamente la recepción del mensaje, pero el proceso volvería inmediatamente antes de recibirlo, por lo que no sabríamos si hay o no un mensaje pendiente.
   \end{itemize}
   Sin embargo, podemos hacer uso de las sentencias \verb|else| de las instrucciones \verb|select|, de forma que una instrucción se ejecute cuando todas las guardas no son ejecutables:
   \begin{minted}[escapeinside=\#\#]{pascal}
       process Receptor
       var dato : integer;
       begin
          while true do begin
             select
                for i:= 1 to n
                   when receive(dato, Emisor[i]) do
                      imprime(dato);
                   end
                else begin
                   imprime(#"#No hay mensajes, duermo#"#);
                   sleep_for(10);
                end
             end select
          end
       end
   \end{minted}
\end{ejercicio}

\begin{ejercicio}\label{ej:rel3_10}
    En un sistema tenemos \verb|N| procesos emisores que envían de forma segura un único mensaje cada uno de ellos a un proceso receptor, mensaje que contiene un entero con el número de proceso emisor. El proceso receptor debe de imprimir el número del proceso emisor que inició el envío en primer lugar. Dicho emisor debe terminar, y el resto quedarse bloqueados:
    \begin{figure}[H]
        \centering
        \begin{minted}[escapeinside=\#\#]{pascal}
            process Emisor[ i : 1..N ];
            begin
               s_send(i, Receptor);
            end

            process Receptor;
            var ganador : integer;
            begin
               { calcular ganador }
               ...
               print #"#El primer envio #lo# ha realizado: #"#, ganador;
            end
        \end{minted}
        \caption{Código para el Ejercicio~\ref{ej:rel3_10}.}
        \label{fig:cod_10}
    \end{figure}
    Para cada uno de los siguientes casos, describir razonadamente si es posible diseñar una solución a este problema o no lo es. En caso afirmativo, escribe una posible solución:
    \begin{enumerate}[label=(\alph*)]
        \item el proceso receptor usa exclusivamente recepción mediante una o varias llamadas a \verb|receive|.
        \item el proceso receptor usa exclusivamente recepción mediante una o varias llamadas a \verb|i_receive|.
        \item el proceso receptor usa exclusivamente recepción mediante una o varias instrucciones \verb|select|.
    \end{enumerate}
    Distinguimos casos:
    \begin{enumerate}[label=(\alph*)]
        \item No es posible, ya que si estamos pensando en usar una única instrucción \verb|receive| de forma que el proceso ganador sea aquel que realice una cita con esta instrucción, entonces no nos quedaríamos con el proceso emisor que inició el envío en primer lugar, sino con el emisor del mensaje que primero llegó en el receptor.
        \item Tampoco es posible, porque ahora mantenemos el problema anterior pero además el orden en el que se reciben los mensajes en el receptor no tiene por qué coincidir con el orden con el que se realizan las instrucciones \verb|i_receive|.
        \item Sí que es posible, ya que en caso de que haya más de un mensaje iniciado y preparado para ser recibido, la orden \verb|select| escogerá aquel mensaje cuyo emisor comenzó antes la operación de envío:
            \begin{minted}[escapeinside=\#\#]{pascal}
                process Receptor
                var ganador : integer;
                begin
                   select
                      for i := 1 to N; when receive(ganador, Emisor[i]) do
                         null;
                      end
                   end select
                   print #"#El primer envio #lo# ha realizado: #"#, ganador;
                end
            \end{minted}
    \end{enumerate}
\end{ejercicio}

\begin{ejercicio}\label{ej:rel3_11}
    Supongamos que tenemos \verb|N| procesos concurrentes semejantes.
    Cada proceso produce \verb|N-1| caracteres (con \verb|N-1| llamadas a la función \verb|ProduceCaracter|) y envía cada carácter a los otros \verb|N-1| procesos. Además, cada proceso debe imprimir todos los caracteres recibidos de los otros procesos (el orden en el que se escriben es indiferente).
    \begin{itemize}
        \item Describe razonadamente si es o no posible hacer esto usando exclusivamente \verb|s_send| para los envíos. En caso afirmativo, escribe una solución.
        \item Escribe una solución usando \verb|send| y \verb|receive|.
    \end{itemize}
    Distinguimos casos:
    \begin{itemize}
        \item En el primer caso, es imposible implementar esta funcionalidad usando operaciones \verb|s_send| de envío síncrono, ya que todos los procesos ejecutarían dicha instrucción, llevando a un interbloqueo de todos los procesos, situación que se pone de manifiesto de forma simple si simplemente consideramos dos procesos:
            \begin{figure}[H]
                \setlength{\columnsep}{1cm}
                \begin{multicols}{2}
                    \begin{minted}{pascal}
                       process P1;    
                       var n1, n2 : integer := 1;
                       begin
                          send(n1, P2);
                          receive(n2, P2);
                       end
                    \end{minted}
                    \begin{minted}{pascal}
                       process P2;    
                       var n1, n2 : integer := 2;
                       begin
                          send(n2, P1);
                          receive(n1, P1);
                       end
                    \end{minted}
                \end{multicols}
                \caption{Situación típica de interbloqueo.}
            \end{figure}
        \item Con instrucciones \verb|send| y \verb|receive| sí que se puede resolver de forma sencilla:
            \begin{minted}{pascal}
                process P[ i : 0..N ];
                var n : char;
                begin
                   { Enviamos todos los caracteres }
                   for j := 0 to N do
                      if i <> j then begin
                         n := Producir();
                         send(n, P[j]);
                      end
                   end do
                   { Recibimos todos los caracteres }
                   for j := 0 to N do
                      if i <> j then begin
                         receive(n, P[j]);
                         print(n);
                      end
                   end do
                end
            \end{minted}
    \end{itemize}
\end{ejercicio}

\begin{ejercicio}\label{ej:rel3_12}
    Escribe una nueva solución al Ejercicio~\ref{ej:rel3_11} en la cual se garantize que el orden en el que se imprimen los caracteres es el mismo orden en el que se inician los envíos de dichos caracteres (pista: usa \verb|select| para recibir).\\

    Como bien indica la pista, basta con usar una instrucción \verb|select| para realizar la recepción de los caracteres:
            \begin{minted}{pascal}
                process P[ i : 0..N ];
                var n : char;
                begin
                   { Enviamos todos los caracteres }
                   for j := 0 to N do
                      if i <> j then begin
                         n := Producir();
                         send(n, P[j]);
                      end
                   end do

                   { Recibimos todos los caracteres }
                   for j := 1 to N do
                      select
                         for k := 0 to N
                            when k <> i receive(n, P[k])
                               print(n);
                            end
                      end select
                   end do
            \end{minted}
\end{ejercicio}

\begin{ejercicio}\label{ej:rel3_13}
    Supongamos de nuevo el problema anterior en el cual todos los procesos envían a todos. Ahora cada item de datos a producir y transmitir es un bloque de bytes con muchos valores (por ejemplo, es una imagen que puede tener varios megabytes de tamaño). Se dispone del tipo de datos \verb|TipoBloque| para ello, y el procedimiento \verb|ProducirBloque|, de forma que si b es una variable de tipo \verb|TipoBloque|, entonces la llamada a \verb|ProducirBloque(b)| produce y escribe una secuencia de bytes en \verb|b|. En lugar de imprimir los datos, se deben consumir con una llamada a \verb|ConsumirBloque(b)|.

    Cada proceso se ejecuta en un ordenador, y se garantiza que hay la suficiente memoria en ese ordenador como para contener simultáneamente, al menos, hasta N bloques. Sin embargo, el sistema de paso de mensajes (SPM) podría no tener memoria suficiente como para contener los ${(N-1)}^{2}$ mensajes en tránsito simultáneos que podría llegar a haber en un momento dado con la solución anterior.

    En estas condiciones, si el SPM agota la memoria, debe retrasar los \verb|send| dejando bloqueados los procesos y, en esas circunstancias, se podría producir interbloqueo. Para evitarlo, se pueden usar operaciones inseguras de envío, \verb|i_send|. Escribe dicha solución, usando como orden de recepción el mismo que en el problema anterior.\\

    \noindent
    Para resolver el problema, utilizaremos un array de $N-1$ bloques a enviar (como vamos a usar la operación \verb|i_send| es un envío inseguro, luego optamos por no modificar los los bloques tras enviarlos) y de un bloque para recibir. Además, antes de terminar un proceso tendremos que esperar a que este proceso haya terminado el envío de todos sus bloques:
    \begin{minted}{pascal}
        process P[ i : 0..N ];
        var bloque : array[0..N] of TipoBloque;
            estado : array[0..N] of Estado;
        begin
           { 1. Realizamos todos los envíos }
           for j := 0 to N do
              if i <> j then begin
                 ProducirBloque(bloque[j]);
                 i_send(bloque[j], P[j], estado[j]);
              end
           end

           { 2. Procesamos las recepciones }
           for j := 0 to N
              if i <> j then begin
                 receive(bloque[i], P[j]);
                 ConsumirBloque(bloque[i]);
              end
           end

           { 3. Esperamos en caso de que no se hayan realizado }
           { todos los envíos antes de terminar }
           for j := 0 to N
              if i <> j then begin
                 wait_send(estado[j]);
              end
           end
        end
    \end{minted}
\end{ejercicio}

\begin{ejercicio}\label{ej:rel3_14}
    En los tres problemas anteriores, cada proceso va esperando a recibir un item de datos de cada uno de los otros procesos, consume dicho item, y después pasa a recibir del siguiente emisor (en distintos órdenes). Esto implica que un envío ya iniciado, pero pendiente, no puede completarse hasta que el receptor no haya consumido los anteriores bloques, es decir, se podría estar consumiendo mucha memoria en el SPM por mensajes en tránsito pendientes cuya recepción se ve retrasada. Escribe una solución en la cual cada proceso inicia sus envíos y recepciones y después espera a que se completen todas las recepciones antes de iniciar el primer consumo de un bloque recibido. De esta forma todos los mensajes pueden transferirse potencialmente de forma simultánea. Se debe intentar que la transimisión y las producción de bloques sean lo más simultáneas posible. Suponer que cada proceso puede almacenar como mínimo $2\cdot N$ bloques en su memoria local, y que el orden de recepción o de consumo de los bloques es indiferente.\\

    La solución a este último problema es similar a la del Ejercicio~\ref{ej:rel3_13}, pero en este caso debemos usar la instrucción \verb|i_receive| para recibir, así como un array entero de bloques para realizar dicha recepción. El código sería el siguiente:
    \begin{minted}{pascal}
        process P[ i : 0..N ];
        var bloque_env : array[0..N] of TipoBloque;
            bloque_rec : array[0..N] of TipoBloque;
            estado_env : array[0..N] of Estado;
            estado_rec : array[0..N] of Estado;
        begin
           { Inicializamos las recepciones }
           for j := 0 to N do
              if j <> i then begin
                 i_receive(bloque_rec[j], P[j], estado_rec[j]);
              end
           end

           { Inicializamos los envíos }
           for j := 0 to N do
              if j <> i then begin
                 ProducirBloque(j);
                 i_send(bloque_env[j], P[j], estado_env[j]);
              end
           end

           { Esperar a que terminen todas las recepciones }
           for j := 0 to N do
              if j <> i then begin
                 wait_recv(estado_rec[j]);
              end
           end

           { Procesar todos los bloques }
           for j := 0 to N do
              if j <> i then begin
                 ConsumirBloque(bloque_recv[j]);
              end
           end

           { Esperar a que terminen todos los envíos antes de terminar el proceso }
           for j := 0 to N do
              if j <> i then begin
                 wait_send(estado_env[j]);
              end
           end
        end
    \end{minted}
\end{ejercicio}
    \section{Sistemas de Tiempo Real}

\begin{ejercicio}\label{ej:rel4_1}
    Dado el conjunto de tareas periódicas y sus atributos temporales que se indica en la Tabla~\ref{tab:4_1}, determinar si se puede planificar el conjunto de dichas tareas utilizando un esquema de planificación basado en planificación cíclica. Diseña el plan cíclico determinando el marco secundario, y el entrelazamiento de las tareas sobre un cronograma.
    \begin{table}[H]
    \centering
    \begin{tabular}{|c|c|c|c|}
        \hline
        Tarea & $C_i$ & $T_i$ & $D_i$ \\
        \hline
        T1 & 10 & 40 & 40 \\
        \hline
        T2 & 18 & 50 & 50 \\
        \hline
        T3 & 10 & 200 & 200 \\
        \hline
        T4 & 20 & 200 & 200 \\
        \hline
    \end{tabular}
    \caption{Tareas periódicas y sus atributos temporales.}
    \label{tab:4_1}
    \end{table}
\end{ejercicio}

\begin{ejercicio}\label{ej:rel4_2}
    El siguiente conjunto de tareas periódicas se puede planificar con ejecutivos cíclicos. Determina si esto es cierto calculando el marco secundario que debería tener. Dibuja el cronograma que muestre las ocurrencias de cada tarea y su entrelazamiento. ¿Cómo se tendría que implementar? (escribe el pseudo-código de la implementación)
    \begin{table}[H]
    \centering
    \begin{tabular}{|c|c|c|c|}
        \hline
        Tarea & $C_i$ & $T_i$ & $D_i$ \\
        \hline
        T1 & 2 & 6 & 6 \\
        \hline
        T2 & 2 & 8 & 8 \\
        \hline
        T3 & 3 & 12 & 12 \\
        \hline
    \end{tabular}
    \caption{Tareas periódicas y sus atributos temporales.}
    \label{tab:4_2}
    \end{table}
\end{ejercicio}

\begin{ejercicio}\label{ej:rel4_3}
    Comprobar si el conjunto de procesos periódicos que se muestra en la siguiente tabla es planificable con el algoritmo RMS utilizando el test basado en el factor de utilización del tiempo del procesador. Si el test no se cumple, ¿debemos descartar que el sistema sea planificable?    
    \begin{table}[H]
    \centering
    \begin{tabular}{|c|c|c|}
        \hline
        Tarea & $C_i$ & $T_i$ \\
        \hline
        T1 & 9 & 30 \\
        \hline
        T2 & 10 & 40 \\
        \hline
        T3 & 10 & 50 \\
        \hline
    \end{tabular}
    \caption{Tareas periódicas y sus atributos temporales.}
    \label{tab:4_3}
    \end{table}
\end{ejercicio}

\begin{ejercicio}\label{ej:rel4_4}
    Considérese el siguiente conjunto de tareas compuesto por tres tareas periódicas:
    \begin{table}[H]
    \centering
    \begin{tabular}{|c|c|c|}
        \hline
        Tarea & $C_i$ & $T_i$ \\
        \hline
        T1 & 10 & 40 \\
        \hline
        T2 & 20 & 60 \\
        \hline
        T3 & 20 & 80 \\
        \hline
    \end{tabular}
    \caption{Tareas periódicas y sus atributos temporales.}
    \label{tab:4_4}
    \end{table}
    Comprueba la planificabilidad del conjunto de tareas con el algoritmo RMS utilizando el test basado en el factor de utilización. Calcular el hiperperiodo y construir el correspondiente cronograma.
\end{ejercicio}

\begin{ejercicio}\label{ej:rel4_5}
    Comprobar la planificabilidad y construir el cronograma de acuerdo al algoritmo de planificación RMS del siguiente conjunto de tareas periódicas.
    \begin{table}[H]
    \centering
    \begin{tabular}{|c|c|c|}
        \hline
        Tarea & $C_i$ & $T_i$ \\
        \hline
        T1 & 20 & 60 \\
        \hline
        T2 & 20 & 80 \\
        \hline
        T3 & 20 & 120 \\
        \hline
    \end{tabular}
    \caption{Tareas periódicas y sus atributos temporales.}
    \label{tab:4_5}
    \end{table}
\end{ejercicio}

\begin{ejercicio}\label{ej:rel4_6}
    Determinar si el siguiente conjunto de tareas puede planificarse con la política de planificación RMS y con la política EDF, utilizando los tests de planificabilidad adecuados para cada uno de los dos casos. Comprobar también la planificabilidad en ambos casos construyendo los dos cronogramas.
    \begin{table}[H]
    \centering
    \begin{tabular}{|c|c|c|}
        \hline
        Tarea & $C_i$ & $T_i$ \\
        \hline
        T1 & 1 & 5 \\
        \hline
        T2 & 1 & 10 \\
        \hline
        T3 & 2 & 20 \\
        \hline
        T4 & 10 & 20 \\
        \hline
        T5 & 7 & 100 \\
        \hline
    \end{tabular}
    \caption{Tareas periódicas y sus atributos temporales.}
    \label{tab:4_6}
    \end{table}
\end{ejercicio}

\begin{ejercicio}\label{ej:rel4_7}
    Describe razonadamente si el siguiente conjunto de tareas puede planificarse o no puede planificarse en un sistema monoprocesador usando un ejecutivo cíclico o usando algún algoritmo basado en prioridades estáticas o dinámicas.
    \begin{table}[H]
    \centering
    \begin{tabular}{|c|c|c|}
        \hline
        Tarea & $C_i$ & $T_i$ \\
        \hline
        T1 & 1 & 5 \\
        \hline
        T2 & 1 & 10 \\
        \hline
        T3 & 2 & 10 \\
        \hline
        T4 & 11 & 20 \\
        \hline
        T5 & 5 & 100 \\
        \hline
    \end{tabular}
    \caption{Tareas periódicas y sus atributos temporales.}
    \label{tab:4_7}
    \end{table}
\end{ejercicio}

\subsubsection{Problemas adicionales}

\begin{ejercicio}\label{ej:rel4_8}
    Para el conjunto de tareas cuyos datos se muestran más abajo, se pide:
    \begin{itemize}
        \item Dibujar el gráfico de ejecución y obtener el tiempo de respuesta de cada tarea.
        \item Determinar, mediante inspección del gráfico anterior, cuántas veces interfiere la tarea $\mathcal{T}_1$ a la tarea $\mathcal{T}_3$ durante un intervalo temporal dado por el tiempo de respuesta de esta última tarea.
        \item Hacer lo mismo que en el apartado anterior pero para las tareas $\mathcal{T}_1$ y $\mathcal{T}_2$.
    \end{itemize}
    \begin{table}[H]
    \centering
    \begin{tabular}{|c|c|c|c|}
        \hline
        Tarea & $C_i$ & $T_i$ & $D_i$ \\
        \hline
        T1 & 1 & 3 & 2 \\
        \hline
        T2 & 3 & 6 & 5 \\
        \hline
        T3 & 2 & 13 & 13 \\
        \hline
    \end{tabular}
    \caption{Tareas periódicas y sus atributos temporales.}
    \label{tab:4_8}
    \end{table}
\end{ejercicio}

\begin{ejercicio}\label{ej:rel4_9}
    Verificar la planificabilidad del siguiente conjunto de tareas utilizando para ello el algoritmo del ``primero el del tiempo límite más cercano'' (EDF).
    \begin{table}[H]
    \centering
    \begin{tabular}{|c|c|c|}
        \hline
        Tarea & $C_i$ & $T_i$ \\
        \hline
        T1 & 1 & 4 \\
        \hline
        T2 & 2 & 6 \\
        \hline
        T3 & 3 & 8 \\
        \hline
    \end{tabular}
    \caption{Tareas periódicas y sus atributos temporales.}
    \label{tab:4_9}
    \end{table}
\end{ejercicio}

\begin{ejercicio}\label{ej:rel4_10}
    Verificar la planificabilidad utilizando el algoritmo EDF de asignación dinámica de prioridades a las tareas y construir el diagrama de ejecución de tareas del siguiente conjunto:
    \begin{table}[H]
    \centering
    \begin{tabular}{|c|c|c|c|}
        \hline
        Tarea & $C_i$ & $D_i$ & $T_i$ \\
        \hline
        T1 & 2 & 5 & 6 \\
        \hline
        T2 & 2 & 4 & 8 \\
        \hline
        T3 & 4 & 8 & 12 \\
        \hline
    \end{tabular}
    \caption{Tareas periódicas y sus atributos temporales.}
    \label{tab:4_10}
    \end{table}
\end{ejercicio}

\begin{ejercicio}\label{ej:rel4_11}
    Verificar la planificabilidad del conjunto de tareas descrito en el Ejercicio~\ref{ej:rel4_10} utilizando el algoritmo del ``plazo de respuesta máximo (D)'' (algoritmo \textit{deadline monotomic} o DM).
\end{ejercicio}

\begin{ejercicio}\label{ej:rel4_12}
    Indicar cuáles de las siguientes afirmaciones son correctas respecto de los algoritmos para resolver el problema de \textit{inversión de prioridad} de las tareas en sistemas de tiempo real de \textit{misión crítica}:
    \begin{enumerate}[label=(\alph*)]
        \item Suponiendo que las tareas se planifican con el protocolo de \textit{herencia de prioridad}: la prioridad heredada por una tarea sólo se mantiene mientras dicha tarea esté utilizando un recurso compartido con otra tarea más prioritaria.
        \item Con el protocolo de \textit{techo de prioridad}, cuando una tarea adquiere un recurso no puede verse interrumpida, hasta que termine su ejecución, por otras tareas que se activan después que ésta y que vayan a utilizar en el futuro un recurso con límite de prioridad igual o inferior.
        \item Con el protocolo de techo de prioridad (PPP), una tarea no puede comenzar a ejecutarse si no están libres todos los recursos que va a utilizar durante su primer ciclo.
        \item Si consideramos una tarea periódica que utilice el protocolo de techo de prioridad inmediato para cambiar su prioridad dinámica cuando accede a recursos, siempre se cumplirá que dicha tarea no puede ser interrumpida por otra menos prioritaria que ella.
        \item Con el protocolo de techo de prioridad las tareas más prioritarias del sistema pueden ser interrumpidas durante cada ciclo de su ejecución como máximo 1 vez cuando acceden a recursos que comparten con otras tareas menos prioritarias.
        \item El protocolo de techo de prioridad original producirá siempre tiempos de respuesta menores para las tareas que el algoritmo de \textit{herencia de prioridad}.
    \end{enumerate}
\end{ejercicio}

\begin{ejercicio}\label{ej:rel4_13}
    Calcular la utilización máxima del procesador que se puede asignar al \textit{Servidor Esporádico} para garantizar la planificabilidad del siguiente conjunto de tareas periódicas utilizando RM.
    \begin{table}[H]
    \centering
    \begin{tabular}{|c|c|c|}
        \hline
        $\mathcal{T}_1$ & 1 & 5 \\
        \hline
        $\mathcal{T}_2$ & 2 & 8 \\
        \hline
    \end{tabular}
    \caption{Conjunto de tareas periódicas.}
    \label{tab:4_13}
    \end{table}
\end{ejercicio}

\begin{ejercicio}\label{ej:rel4_14}
    Calcular la utilización máxima del procesador que puede ser asignada al \textit{Servidor Diferido} (SD) para garantizar la planificabilidad del conjunto de tareas periódicas dado en el Ejercicio~\ref{ej:rel4_13} anterior.
\end{ejercicio}

\begin{ejercicio}\label{ej:rel4_15}
    Junto con las tareas periódicas que se muestran en el Ejercicio~\ref{ej:rel4_13} definir un plan para planificar las siguientes tareas aperiódicas utilizando un \textit{SD} (tarea sondeante) que posea una utilización máxima del tiempo del procesador y prioridad intermedia:
    \begin{table}[H]
    \centering
    \begin{tabular}{|c|c|c|}
        \hline
        & $t_a$ & $C_i$ \\
        \hline
        $J_1$ & 2 & 2 \\
        \hline
        $J_2$ & 7 & 2 \\
        \hline
        $J_3$ & 17 & 1 \\
        \hline
    \end{tabular}
    \caption{Tareas periódicas y sus atributos temporales.}
    \label{tab:4_15}
    \end{table}
\end{ejercicio}

\begin{ejercicio}\label{ej:rel4_16}
    Resolver ahora el mismo problema de planificación descrito en el Ejercicio~\ref{ej:rel4_15} utilizando ahora un \textit{Servidor Esporádico} que tenga una utilización máxima y prioridad intermedia.
\end{ejercicio}

\begin{ejercicio}\label{ej:rel4_17}
    Resolver el mismo problema de planificación descrito en el Ejercicio~\ref{ej:rel4_15} utilizando ahora un \textit{Servidor Diferido} que tenga una utilización máxima y prioridad intermedia.
\end{ejercicio}

\begin{ejercicio}\label{ej:rel4_18}
    Utilizar un \textit{Servidor Esporádico} con capacidad $C_s=2$ y periodo $T_s=6$ para planificar las siguiente tareas:
    \begin{table}[H]
    \centering
    \begin{tabular}{|c|c|c|}
        \hline
        & $C_i$ & $T_i$ \\
        \hline
        $\mathcal{T}_1$ & 1 & 4 \\
        \hline
        $\mathcal{T}_2$ & 2 & 6 \\
        \hline
                        & $a_i$ & $C_i$ \\
        \hline
        $J_1$ & 2 & 2 \\
        \hline
        $J_2$ & 5 & 1 \\
        \hline
        $J_3$ & 10 & 2 \\
        \hline
    \end{tabular}
    \caption{Tareas periódicas y sus atributos temporales.}
    \label{tab:4_18}
    \end{table}
\end{ejercicio}

\end{document}
