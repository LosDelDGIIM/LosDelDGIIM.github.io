\section{Capa de red}

\begin{ejercicio}
    Estime el tiempo involucrado en la transmisión de un mensaje de datos para la técnicas de conmutación de circuitos (CC) y de paquetes mediante datagramas (CDP) y mediante circuitos virtuales (CPCV) considerando los siguientes parámetros:
    \begin{itemize}
        \item $M$: longitud en bits del mensaje a enviar.
        \item $V$: velocidad de transmisión de las líneas en bps.
        \item $P$: longitud en bits de los paquetes, tanto en CPD como en CPCV\@.
        \item $H_d$: bits de cabecera de los paquetes en CPD\@.
        \item $H_c$: bits de cabecera de los paquetes en CPCV\@.
        \item $T$: longitud en bits de los mensajes intercambiados para el establecimiento y cierre de conexión, tanto en CC como en CPCV\@.
        \item $N$: número de nodos intermedios entre las estaciones finales.
        \item $D$: tiempo de procesamiento en segundos en cada nodo, tanto en CC como en CPD y en CPCV\@.
        \item $R$: retardo de propagación, en segundos, asociado a cada enlace, en CC, en CPD y en CPCV\@.
    \end{itemize}
\end{ejercicio}

\begin{ejercicio}
    Un mensaje de $\unit[64]{kB}$ se transmite a lo largo de dos saltos de una red. Ésta limita la longitud máxima de los paquetes a $\unit[2]{kB}$ y cada paquete tiene una cabecera de $\unit[32]{B}$. Las líneas de transmisión de la red no presentan errores y tienen una capacidad de $\unit[50]{Mbps}$. Cada salto corresponde a una distancia de $\unit[1000]{km}$. ¿Qué tiempo se emplea en la transmisión del mensaje mediante datagramas?\\

    Como las cabeceras son de $\unit[32]{B}$ y el \acrshort{MTU} es de $\unit[2]{kB}$, cada paquete podrá enviar, como máximo:
    \begin{equation*}
        \unit[2]{kB} - \unit[32]{B} = \unit[2028]{B} - \unit[32]{B} = \unit[1996]{B}
    \end{equation*}

    Además, y aunque no se especifica en el enunciado, se asume\footnote{En el caso de no realizar esta asunción, sería similar solo que trabajando con el valor anteriormente calculado.} que estos se envían empleando el protocolo \acrshort{IP} y que la cabecera es mayor debido al uso de campos opcionales. Como el campo de desplazamiento de la cabecera se mide en múltiplos de $\unit[8]{B}$, el tamaño de los datos ha de ser múltiplo de $\unit[8]{B}$, por lo que como máximo se podrán enviar $\unit[1992]{B}$ de datos (que provocará que el desplazamiento aumente en $249$ unidades en cada paquete). Por tanto, en cada paquete se enviarán $\unit[1992]{B}$ de datos y $\unit[32]{B}$ de cabecera.\\

    Veamos cuántos paquetes se envían:
    \begin{equation*}
        \unit[64]{kB} = 32\cdot \unit[1992]{B} + \unit[1792]{B}
    \end{equation*}
    Por tanto, se envían $33$ paquetes; el último de ellos con $\unit[1792]{B}$ de datos y $\unit[32]{B}$ de cabecera.\\

    Busquemos por tanto ahora el tiempo. Para ello, primero calculamos el tiempo de transmisión de un paquete:
    \begin{equation*}
        T_t = \frac{\unit[L]{B} + \unit[32]{B}}{\unit[50]{Mbps}} = \begin{cases}
            \unit[0.3238]{ms} & \text{para los primeros $32$ paquetes} \\
            \unit[0.2918]{ms} & \text{para el último paquete}
        \end{cases}
    \end{equation*}

    Veamos ahora el tiempo de propagación entre dos nodos, donde suponemos que la red es cableada y, por tanto, la velocidad de propagación es de $\nicefrac{2}{3}\cdot c=\unitfrac[2\cdot 10^5]{km}{s}$:
    \begin{equation*}
        T_p = \frac{\unit[1000]{km}}{\unitfrac[2\cdot 10^5]{km}{s}} = \unit[5]{ms}
    \end{equation*}

    Calculemos ahora el tiempo total que tarda en llegar tan sólo el primer paquete al segundo nodo (recorrido completo). Este será\footnote{Como mínimo, pues consideramos despreciable tiempos como el de procesamiento.}:
    \begin{equation*}
        T_{\text{total}}^1 = 2T_t + 2T_p
    \end{equation*}

    Una vez llegado este, cada paquete más tan sólo supondrá el retardo provocado por la transmisión, por lo que el tiempo total que tardan en llegar los $33$ paquetes será, como mínimo:
    \begin{align*}
        T_{\text{total}} &= T_{\text{total}}^1 + 31T_t + T_t
        = 33T_t + 2T_p + T_t
        =\\&= 33\cdot \unit[0.3238]{ms} + 2\cdot \unit[5]{ms} + \unit[0.2918]{ms}
        = \unit[20.9772]{ms}
    \end{align*}
\end{ejercicio}

\begin{ejercicio}
    Suponga que una red de datagramas usa cabeceras de $H$ bytes y que una red de paquetes de circuitos virtuales utiliza cabeceras de $h$ bytes. Determine la longitud $M$ de un mensaje que se consigue transmitir más rápido haciendo uso de la técnica de conmutación de circuitos virtuales que mediante la de datagramas. Suponga que los paquetes tienen la misma longitud en ambas redes y que los retardos de procesamiento son idénticos
\end{ejercicio}

\begin{ejercicio}\label{ej:4}
    Una aplicación audiovisual en tiempo real hace uso de conmutación de paquetes para transmitir voz a $\unit[32]{kbps}$ y vídeo a $\unit[64]{kbps}$ a través de la conexión de red de la figura~\ref{fig:ej4}.  Se consideran paquetes de voz e información de audio con dos longitudes distintas: $\unit[10]{ms}$ y $\unit[100]{ms}$. Cada paquete tiene además una cabecera de $\unit[40]{B}$.
    \begin{figure}
        \centering
        \resizebox{1.1\linewidth}{!}{
            \begin{tikzpicture}[node distance=2.5cm]
                \node[laptop] (L1) {};
                \node[router, right=of L1] (R1) {};
                \node[router, right=of R1] (R2) {};
                \node[router, right=of R2] (R3) {};
                \node[laptop, right=of R3] (L2) {};

                % Bucle for para marcar los routers
                \foreach \i in {1,...,3} {
                    \node[yshift=-0.8em] at (R\i) {\color{white} \textbf{R\i}};
                }

                % Bucle for para marcar los laptops
                \foreach \i in {1,...,2} {
                    \node[xshift=0.7em, yshift=0.7em] at (L\i) {\color{white} \textbf{L\i}};
                }

                % Enlaces
                \draw (L1) -- node[above] {$\unit[1]{km}$} node[below] {$\unit[10]{Mbps}$} (R1);
                \draw (R1) -- node[above] {$\unit[3000]{km}$} node[below] {$45\ \text{o}\ \unit[1.5]{Mbps}$} (R2);
                \draw (R2) -- node[above] {$\unit[1000]{km}$} node[below] {$45\ \text{o}\ \unit[1.5]{Mbps}$} (R3);
                \draw (R3) -- node[above] {$\unit[1]{km}$} node[below] {$\unit[10]{Mbps}$} (L2);

                
            \end{tikzpicture}
        }
        \caption{Red de conmutación de paquetes del ejercicio~\ref{ej:4}.}
        \label{fig:ej4}
    \end{figure}

    \begin{enumerate}
        \item Encuentre para ambos casos el porcentaje de bits suplementarios que supone la cabecera.
        
        Veamos en primer lugar cuántos datos hemos de enviar:
        \begin{itemize}
            \item Para la voz, se envían $\unit[32]{kbps}\cdot \unit[10]{ms} = \unit[320]{b}$ de datos.
            \item Para el vídeo, se envían $\unit[64]{kbps}\cdot \unit[100]{ms} = \unit[6400]{b}$ de datos.
        \end{itemize}

        Por tanto, el porcentaje de bits suplementarios que supone la cabecera es:
        \begin{itemize}
            \item Para la voz:
            \begin{equation*}
                \frac{\unit[(40\cdot 8)]{b}}{\unit[(40\cdot 8)]{b}+\unit[320]{b}} = 0.5\Longrightarrow 50\%
            \end{equation*}
            \item Para el vídeo:
            \begin{equation*}
                \frac{\unit[(40\cdot 8)]{b}}{\unit[(40\cdot 8)]{b}+\unit[6400]{b}} = 0.0476\Longrightarrow 4.76\%
            \end{equation*}
        \end{itemize}
        \item Dibuje un diagrama temporal e identifique todas las componentes del retardo extremo a extremo en la conexión anterior. Recuerde que un paquete no puede ser transmitido hasta que esté completo y que no se puede retransmitir hasta que no se haya recibido completamente. Suponga despreciables los errores a nivel de bit.
        \item Evalúe todas las componentes del retardo de las que se dispone suficiente información. Considere las dos longitudes de paquete aceptadas. Suponga que la señal se propaga a una velocidad de 1 km/5 microsegundos y considere dos velocidades para la red troncal: 45 Mbps y 1,5 Mbps. Resuma el resultado para los cuatro posibles casos en una tabla con cuatro entradas.
        \item ¿Cuál de las componentes anteriores implica la existencia de retardos de cola?
    \end{enumerate}

\end{ejercicio}

\begin{ejercicio}\label{ej:5}
    Imagine la situación descrita en la figura~\ref{fig:ej5}, donde se muestra una red. El router 5 a Internet a través de la puerta de acceso especificada por el \acrshort{ISP}. Especifique las tablas de encaminamiento en los routers. Asigne a voluntad las direcciones IP e interfaces necesarias.
    \begin{figure}
    \centering
    \begin{tikzpicture}[node distance=3cm]
        \node[cloud] (Internet) {Internet};

        \node[router, below of=Internet, yshift=0.35cm] (R5){};
        \node[switch, below of=R5, yshift=0.35cm] (SW2) {};
        \node[router, left of=SW2] (R4) {};
        \node[router, right of=SW2] (R3) {};
        \node[switch, below of=R4, yshift=0.35cm] (SW1) {};
        \node[router, left of=SW1] (R1) {};
        \node[router, right of=SW1] (R2) {};
        \node[cloud, below of=R1, yshift=0.35cm] (LAN_A) {LAN A};
        \node[cloud, below of=R2, yshift=0.35cm] (LAN_B) {LAN B};
        \node[cloud, below of=R3, yshift=0.35cm] (LAN_C) {LAN C};

        % Bucle for para marcar los routers
        \foreach \i in {1,...,5} {
            \node[yshift=-0.8em] at (R\i) {\color{white} \textbf{R\i}};
        }

        % Bucle for para marcar los switches
        \foreach \i in {1,2} {
            \node[yshift=-0.7em, xshift=-0.3em] at (SW\i) {\color{white} \textbf{SW\i}};
        }

        % Conexiones
        \draw (Internet) -- (R5);
        \draw (R5) -- node[right] {192.168.0.0/24} (SW2);
        \draw (SW2) -- (R4);
        \draw (SW2) -- (R3);
        \draw (R4) -- node[left] {192.168.1.0/24} (SW1);    
        \draw (R3) -- node[right] {192.168.4.0/24} 
        (LAN_C);
        \draw (SW1) -- (R1);
        \draw (SW1) -- (R2);
        \draw (R1) -- node[left] {192.168.2.0/24} 
        (LAN_A);
        \draw (R2) -- node[right] {192.168.3.0/24} 
        (LAN_B);
    \end{tikzpicture}
    \caption{Situación del Ejercicio~\ref{ej:5}.}
    \label{fig:ej5}
    \end{figure}

    En primer lugar, asignaremos las direcciones IP y las interfaces de cada router. Cada router presenta dos conexiones, por lo que sólo usaremos dos interfaces por cada router, luego asociaremos dos direcciones IP a cada router.

    Por comodidad, asociaremos las conexiones más cercanas a Internet de cada router a la interfaz \textit{ether0}, y a las más lejanas la interfaz \textit{ether1}.

    Una vez añadidas las interfaces, procederemos a asocciar direcciones IP a cada interfaz de cada router:
    \begin{itemize}
        \item R1\@:
            \begin{itemize}
                \item Tendrá IP 192.168.2.1 en la red 192.168.2.0/24.
                \item Tendrá IP 192.168.1.2 en la red 192.168.1.0/24.
            \end{itemize}
        \item R2\@:
            \begin{itemize}
                \item Tendrá IP 192.168.3.1 en la red 192.168.3.0/24.
                \item Tendrá IP 192.168.1.3 en la red 192.168.1.0/24.
            \end{itemize}
        \item R3\@:
            \begin{itemize}
                \item Tendrá IP 192.168.1.1 en la red 192.168.1.0/24.
                \item Tendrá IP 192.168.0.2 en la red 192.168.0.0/24.
            \end{itemize}
        \item R4\@: 
            \begin{itemize}
                \item Tendrá IP 192.168.4.1 en la red 192.168.4.0/24.
                \item Tendrá IP 192.168.0.3 en la red 192.168.0.0/24.
            \end{itemize}
        \item R5\@:
            \begin{itemize}
                \item Tendrá IP 192.168.0.1 en la red 192.168.0.0/24.
                \item Su IP en la red que le conecta con Internet la proveerá el \acrshort{ISP}. Supongamos que es 33.33.0.0/16.
            \end{itemize}
    \end{itemize}

    Procedemos ahora a rellenar las tablas de encaminamiento de cada router:
    \begin{table}
    \centering
    \begin{tabular}{|c|c|c|c|}
        \hline
        Red destino & Máscara & Siguiente salto & Interfaz \\
        \hline
        192.168.2.0 & /24 & * & ether1 \\
        \hline
        192.168.1.0 & /24 & * & ether0 \\
        \hline
        192.168.3.0 & /24 & 192.168.1.3 (R2) & ether0 \\
        \hline
        \verb|default| & /0 & 192.168.1.1 (R4) & ether0 \\
        \hline
    \end{tabular}\\
    \caption{Tabla de encaminamiento para R1.}
    \end{table}

    \begin{table}
        \centering
        \begin{tabular}{|c|c|c|c|}
            \hline
            Red destino & Máscara & Siguiente salto & Interfaz \\
            \hline
            192.168.3.0 & /24 & * & ether1 \\
            \hline
            192.168.1.0 & /24 & * & ether0 \\
            \hline
            192.168.2.0 & /24 & 192.168.1.2 (R1) & ether0 \\
            \hline
            \verb|default| & /0 & 192.168.1.1 (R4) & ether0 \\
            \hline
        \end{tabular}\\
        \caption{Tabla de encaminamiento para R2.}
        \end{table}

    \begin{table}
        \centering
        \begin{tabular}{|c|c|c|c|}
            \hline
            Red destino & Máscara & Siguiente salto & Interfaz \\
            \hline
            192.168.4.0 & /24 & * & ether1 \\
            \hline
            192.168.0.0 & /24 & * & ether0 \\
            \hline
            192.168.0.0 & \textbf{/22} & 192.168.0.2 (R4) & ether0 \\
            \hline
            \verb|default| & /0 & 192.168.0.1 (R5) & ether0 \\
            \hline
        \end{tabular}\\
        \caption{Tabla de encaminamiento para R3.}
    \end{table}
    Donde hemos agrupado la Red A (192.168.2.0/24), B (192.168.3.0/24) y la red 192.168.1.0/24 en la superred 192.168.0.0/22.
    Notemos que dentro de las direcciones de la superred se encuentran las direcciones de la forma 192.168.0.x, que no se encuentran en dicha superred. Sin embargo, tenemos una entrada específica para dichas direcciones, con una máscara de mayor prioridad (más 1s), por lo que no tenemos problema\footnote{Si no tuviéramos dicha entrada, tendríamos un problema, ya que si mandamos un paquete a 192.168.0.26, por ejemplo, iría a la superred que hemos definido pero algún router se daría cuenta de que no sabe llegar a 192.168.0.0/24.}.

    \begin{table}
        \centering
        \begin{tabular}{|c|c|c|c|}
            \hline
            Red destino & Máscara & Siguiente salto & Interfaz \\
            \hline
            192.168.1.0 & /24 & * & ether1 \\
            \hline
            192.168.0.0 & /24 & * & ether0 \\
            \hline
            192.168.2.0 & /24 & 192.168.1.2 (R1) & ether1 \\
            \hline
            192.168.3.0 & /24 & 192.168.1.3 (R2) & ether1 \\
            \hline
            192.168.4.0 & /24 & 192.168.0.3 (R3) & ether0 \\
            \hline
            \verb|default| & /0 & 192.168.0.1 (R5) & ether0 \\
            \hline
        \end{tabular}\\
        \caption{Tabla de encaminamiento para R4.}
    \end{table}

    \begin{table}
        \centering
        \begin{tabular}{|c|c|c|c|}
            \hline
            Red destino & Máscara & Siguiente salto & Interfaz \\
            \hline
            192.168.0.0 & /24 & * & ether1 \\
            \hline
            33.33.0.0 & /16 & * & ether0 \\
            \hline
            192.168.4.0 & /24 & 192.168.0.3 (R3) & ether1 \\
            \hline
            192.168.0.0 & \textbf{/22} & 192.168.0.2 (R4) & ether1 \\
            \hline
            \verb|default| & /0 & Router\_ISP\_Gateway & ether0 \\
            \hline
        \end{tabular}\\
        \caption{Tabla de encaminamiento para R5.}
    \end{table}
    Donde hemos vuelto a usar la superred 192.168.0.0/22 que engloba a Red A, B y 192.168.1.0/24.
\end{ejercicio}

\begin{ejercicio}
    Asigne las direcciones de subred en la siguiente topología a partir de 192.168.0.0 para minimizar el número de entradas en las tablas de encaminamiento, asumiendo que en las redes LAN puede haber hasta 50 PCs.
\begin{figure}
    \centering
    \begin{tikzpicture}[
        node/.style={circle, draw, minimum size=1cm}, 
        switch/.style={rectangle, draw, minimum size=1cm}, 
        red/.style={ellipse, draw, minimum size=1cm},
        edge/.style={-}
        ]

        \node[red] (A) {Internet};
        \node[node, right=of A] (B) {R};
        \node[switch, right=of B] (C) {SW};
        \node[node, right=of C] (D) {R};
        \node[node, above=4cm of D] (E) {R};
        \node[node, below=4cm of D] (F) {R};

        \node[switch, right=of E] (G) {SW};
        \node[node, above=.5cm of G] (H) {R};
        \node[node, below=.5cm of G] (I) {R};

        \node[switch, right=of D] (J) {SW};
        \node[node, above=.5cm of J] (K) {R};
        \node[node, below=.5cm of J] (L) {R};

        \node[switch, right=of F] (M) {SW};
        \node[node, above=.5cm of M] (N) {R};
        \node[node, below=.5cm of M] (O) {R};

        \node[red, right=of H] (P) {Lan A};
        \node[red, right=of I] (Q) {Lan B};
        \node[red, right=of K] (R) {Lan C};
        \node[red, right=of L] (S) {Lan D};
        \node[red, right=of N] (T) {Lan E};
        \node[red, right=of O] (U) {Lan F};


        \draw[edge] (A) -- (B);
        \draw[edge] (B) -- (C);
        \draw[edge] (C) -- (D);
        \draw[edge] (C) -- (E);
        \draw[edge] (C) -- (F);

        \draw[edge] (E) -- (G);
        \draw[edge] (G) -- (H);
        \draw[edge] (G) -- (I);

        \draw[edge] (D) -- (J);
        \draw[edge] (J) -- (K);
        \draw[edge] (J) -- (L);

        \draw[edge] (F) -- (M);
        \draw[edge] (M) -- (N);
        \draw[edge] (M) -- (O);

        \draw[edge] (H) -- (P);
        \draw[edge] (I) -- (Q);
        \draw[edge] (K) -- (R);
        \draw[edge] (L) -- (S);
        \draw[edge] (N) -- (T);
        \draw[edge] (O) -- (U);
    \end{tikzpicture}
    \caption{Situación del ejercicio 6}
    \end{figure}
\end{ejercicio}

\begin{ejercicio}
    Un datagrama de 4020 bytes pasa de una red Token Ring con THT 8 ms (MTU 4400) a una Ethernet (MTU 1500) y después pasa por un enlace PPP con bajo retardo (MTU 296). Si ese mismo datagrama pasara directamente de la red Token Ring al enlace PPP (sin pasa por la red Ethernet) ¿habría alguna diferencia en la forma como se produce la fragmentación? Especifique en ambos casos los fragmentos obtenidos.
\end{ejercicio}

\begin{ejercicio}
    ¿Cómo podría utilizar ICMP para hacer una estimación de la latencia entre dos entidades finales? ¿Y para estimar la latencia de un enlace en particular entre dos routers?
\end{ejercicio}

\begin{ejercicio}
    Considere la subred de la figura. Se utiliza el algoritmo de encaminamiento de vector distancia, habiéndose recibido en el encaminador C los siguientes vectores de encaminamiento: desde B (5, 0, 8, 12, 6, 2), desde D (16, 12, 6, 0, 9, 10) y desde E (7, 6, 3, 9, 0, 4). Los retardos medidos a B, D y E son, respectivamente, 6, 3 y 5. ¿Cuál es la nueva tabla de encaminamiento de C\@? Indique la línea de salida y el retardo esperado.
% // TODO: Copiar figura
\end{ejercicio}

\begin{ejercicio}
    Considere la red mostrada en la figura~\ref{grafo:ej10}, en la que se representan 4 nodos unidos con enlaces. En cada enlace se indica el retardo sufrido por los mensajes al atravesarlo. Los nodos utilizan un protocolo de encaminamiento dinámico de tipo distribuido en el que la métrica está basada en el retardo. Se pide lo siguiente:
    \begin{enumerate}[label=\alph*.]
        \item Escriba las tablas de encaminamiento para todos los nodos de la red una vez haya pasado el tiempo suficiente para que dichas tablas se construyan de forma estable.
        \item Considere que los nodos envían y actualizan sus tablas cada 5 segundos, siendo la primera actualización en $t=0$ s. Suponga que, en $t=12$ s., el enlace BD pasa a tener un retardo de 3 s. ¿Cuál será el encaminamiento desde el nodo A hasta el nodo B cuando las tablas se estabilicen de nuevo?
        \item ¿En qué instante comenzará dicho encaminamiento a funcionar? Justifique su respuesta explicando qué sucederá desde $t=12$ s.\ hasta dicho instante y también después del mismo.
    \end{enumerate}

    \begin{figure}[H]
    \centering
    \begin{tikzpicture}[
        node/.style={circle, draw, minimum size=.5cm},
        edge/.style={-}
        ]
        \node[node] (A) {A};
        \node[node, above right=1cm of A] (B) {B};
        \node[node, below right=1cm of A] (C) {C};
        \node[node, above right=1cm of C] (D) {D};
        
        \draw[edge] (A) -- node[left] {4} (B);
        \draw[edge] (A) -- node[left] {1} (C);
        \draw[edge] (B) -- node[right] {4} (C);
        \draw[edge] (B) -- node[right] {1} (D);
        \draw[edge] (C) -- node[right] {1} (D);
    \end{tikzpicture}
    \caption{Grafo para el ejercicio 10.}
    \label{grafo:ej10}
    \end{figure}
\end{ejercicio}

\begin{ejercicio}
En la topología de red adjunta se indica la capacidad, en Kbps, de las líneas de transmisión entre los distintos nodos intermedios. Considérese al respecto que los enlaces son \textit{full‐duplex} y que la velocidad es la misma para cada uno de los sentidos.  Por otra parte, la tabla anexa especifica el tráfico, en paquetes/segundo, entre cada par de nodos. Además, en cursiva se indica la ruta (secuencia de nodos) seguida en la transmisión. Teniendo en cuenta todo lo anterior, determine el retardo medio en el envío de un paquete sobre la red global.    

\begin{figure}[H]
\centering
\begin{tikzpicture}[
    node/.style={circle, draw, minimum size=.5cm},
    edge/.style={-}
    ]
    \node[node] (A) {2};
    \node[node, right=2cm of A] (B) {5};
    \node[node, below left=of A] (C) {1};
    \node[node, below right=of A] (D) {4};
    \node[node, below left=of D] (E) {3};
    
    \draw[edge] (A) -- node[above] {10} (B);
    \draw[edge] (A) -- node[left] {20} (C);
    \draw[edge] (A) -- node[right] {15} (D);
    \draw[edge] (B) -- node[right] {20} (D);
    \draw[edge] (C) -- node[above] {40} (D);
    \draw[edge] (C) -- node[right] {30} (E);
    \draw[edge] (D) -- node[right] {15} (E);
\end{tikzpicture}
\caption{Grafo para el ejercicio 11.}
\end{figure}

\begin{table}[H]
\centering
\begin{tabular}{|c|c|c|c|c|c|c|}
    \hline
    \multicolumn{2}{|c|}{} & \multicolumn{5}{c|}{\textbf{Nodo destino}} \\ % Fila de título que agrupa las 5 columnas
    \hline
    \multicolumn{2}{|c|}{}   & 1 & 2 & 3 & 4 & 5 \\
    \hline
    \multirow{5}{*}{\rotatebox[origin=c]{90}{\textbf{Nodo origen}}} & 1 &   & 2-\textit{12} & 3-\textit{13} & 1-\textit{14} & 2-\textit{145} \\
    \cline{2-7}
    & 2 & 2-\textit{21} &   & 4-\textit{243} & 2-\textit{24} & 2-\textit{25} \\
    \cline{2-7}
    & 3 & 3-\textit{31} & 4-\textit{342} &  & 3-\textit{34} & 5-\textit{345} \\
    \cline{2-7}
    & 4 & 1-\textit{41} & 2-\textit{42}  & 3-\textit{43} &  & 1-\textit{45} \\
    \cline{2-7}
    & 5 & 2-\textit{541} & 2-\textit{52} & 5-\textit{543} & 1-\textit{54} &  \\
    \hline
\end{tabular}
\end{table}

\end{ejercicio}
