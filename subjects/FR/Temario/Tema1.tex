\chapter{Introducción a los fundamentos de redes}
\subsubsection{Objetivos}
\begin{itemize}
    \item Conocer y comprender los principios básicos de las comunicaciones.
    \item Entender el diseño funiconal en capas de las redes y los conceptos y terminología fundamentales involucrados. 
    \item Comprender desde un punto de vista teórico-conceptual el modelo de referencia OSI y su correspondencia con el modelo de capas usado en Internet.
\end{itemize}

\subsection{Introducción}
La arquitectura lógica de Internet está diseñada por capas. Veamos el model TCP/IP (también veremos el modelo OSI):
\begin{table}[h]
    \centering
    \begin{tabular}{|c|}
        \hline
        Aplicación que hace uso de la red\\ \hline
        Transporte \textbf{(TCP/UDP)} \\ \hline
        Red \textbf{(IP)} \\ \hline
        Enlace \\ \hline
        Física \\ \hline
    \end{tabular}
    \caption{Modelo de capas del protocolo TCP/IP}
    \label{table:_tabla_de_capas}
\end{table}

Las dos últimas capas están implementadas en Hardware y las tres primeras en Software, también llamado \textbf{NOS} (Network Operative System). En la asignatura veremos las capas altas. 

\section{Sistemas de comunicación y redes}

\begin{definicion}[Sistema de comunicación]
    Es una infraestructura (hw + sw) que permite el intercambio de información. Un sistema típico es el siguiente:
    \begin{itemize}
        \item Tenemos una fuente y un transmisor en un mismo equipo (que es el que va a mandar la información). La fuente genera la información y el transmisor adapta la información al medio.
        \item Después tenemos el canal de comunicación, el cual produce errores: ruidos, interferencias, diafonías (cuando hay muchos cables en paralelo juntos, puede suceder que la información de un cable se meta en otro). 
        \item Al final tenemos un receptor y el destino (en un mismo equipo). El receptor adapta la información para el destino y éste espera los datos a recibir. 
    \end{itemize}
\end{definicion}


\subsection{Motivación para usar redes}
Para verlo, vamos a hablar de la primera red de comunicaciones que era una red de telefonía móvil. 
Cada usuario contaba con su línea de teléfono, que conectaba con una central de conmutación local, luego regional y luego nacional, la cual debía conectarse con la central local del destino. Se usaba conmutación de circuitos. 
\begin{itemize}
    \item Inicialemnte se creaba un camino físico juntando cables, llamado circuito. 
    \item Era ineficiente porque no se está hablando todo el tiempo, y los tiempos de silencio el circuito se desaprovecha. 
    \item Era un problema de seguridad el mal funcionamiento de una central, pues dejaba a miles de telefónos sin servicio. 
\end{itemize}

Si ahora pensamos en ordenadores (o equipos más generales, móviles, PCs, portátiles, móviles\ldots) en vez de móviles y cambiamos las centrales de conmutación por routers, contamos con muchísimos caminos para conectar dos ordenadores, haciendo más segura la red.\\

Ahora ya no tenemos un camino físico, sino que son los routers quienes deciden por dónde enviar los paquetes y en qué momento. Los routers tienen colas, lo que genera algo de retardo, pero tiene la ventaja de que se usa mucho mejor el canal y hay más seguridad, pues hay más caminos que solo uno.\\ 

De una red esperamos:
\begin{itemize}
    \item Autonomía.
    \item Interconexión.
    \item Intercambio de información con eficacia y transparencia. 
\end{itemize}



\begin{definicion}[Red]
    Sistema de comunicación con sistemas finales o terminales autónomos (con capacidad para procesar información) que facilita el intercambio eficaz y transparente de información. Concretamente tenemos:

    \begin{itemize}
        \item\textbf{Hosts:} sistemas finales o terminales autónomos. Los que transmiten y reciben datos. 
        \item\textbf{Subred:} infraestructura para el transporte de información, formada por líneas de transmisión y nodos o elementos de conmutación: routers y switches.   
    \end{itemize}

    En cuanto a medios de transmisión, originalmente se usaban cables de pares (pensado para transmitir 4kHz, la media de la voz humana), luego cables coaxiales, que mejoraron mucho; y fibra óptica que puede transmitir sin interferencias, por lo que es el mejor medio guiado existente. Los cables trenzados son para distancias más cortas, Ethernet por ejemplo. 
\end{definicion}

\subsection{Topologías de redes}
\begin{description}
    \item[En bus:] es la más sencilla, pero como el medio es común, todos intentan acceder y se producen colisiones. 
    \item[En anillo:] un círculo en el que tenemos los distintos computadores. Es similar al bus pues el medio es compartido. Una versión habitual es \textbf{token ring}, testigo de anillo. Como el medio es común, se van pasando el testigo evitando así colisiones. 
    \item[En estrella:] todos están conectados a un centro principal, típicamente un switch. 

    A diferencia del bus, en este caso cada cable es independiente del resto. Si un PC pone algo en una toma el resto no lo escuchan. Cada línea tienen una cola para guardar a dónde enviar los paquetes y el switch tiene un procesador que coge los paquetes de dichas colas y los envían a las salidas. Es una topología mucho más segura por el hecho de no compartir el medio. 

    \item[En árbol:] típica en redes empresariales. Se suele estructurar en tres niveles:
    \begin{itemize}
        \item Primer nivel: red troncal.
        \item Segundo nivel: red de división.
        \item Tercer nivel: red de acceso. 
    \end{itemize}
    Los equipos de primer y segundo nivel suelen ser switches. Pueden aparecer ciclos en el árbol. Ethernet no tiene ningún mecanismo para evitar que un paquete se mueva en círculo, lo que echa la red abajo. 

    El protocolo STP hace que cualquier topología quiten enlaces redundantes que forman bucles. 

    \item[Mallada:] todos los nodos están conectados entre sí por medios independientes. Es muy fiable, si se cae un enlace tienes más caminos para llegar a tu destino. Pero no es escalable, si metemos un nodo n-ésimo, hay que meter $n-1$ enlaces. Pero para redes pequeñas está bien. 
    Dentro de una empresa la red troncal puede seguir esta topología para evitar caídas importantes. 

    \item[Híbrida:] se usa una mezcla de todas. Es la más utilizada.   
\end{description}

%TODO: ver si dejo esto
\begin{definicion}[CSMA/CD]
    Ethernet para compartir un medio común como un bus utiliza CSMA/CD\@: acceso múltiple sintiendo la portadora (Carrier Sense). CD viene de ``detecta colisiones'': si lo que hay en el cable no es lo que ha puesto (si hay más ruido), da error.
\end{definicion}

\begin{definicion}[CSMA/CA]
    Wifi usa este protocolo, no puede escuchar el medio, primero escucha que no haya nadie, envía el mensaje y recibe confirmación. Si no recibe confirmación, hay colisión. 
\end{definicion}



\subsection{Clasificación de redes}
\begin{description}
    \item[Según tamaño y extensión:] \
    \begin{itemize}
        \item LAN (Local Area Network). Red de área local, suele ser el mismo edificio. 
        \item MAN (Metropolitan Area Network). Red de área metropolitana, para conectar un campus o una ciudad. 
        \item WAN (Wide Area Network). Red de área extensa, redes desponibles en todo el país, como las redes telefónicas.
        \item PAN (Personal Area Network). Red de área personal, todo lo que puede tener una persona, relojes, portátiles, cascos\ldots
    \end{itemize}
    \item[Según tecnología de transmisión:]\
    \begin{itemize}
        \item Difusión: lo que pone alguien en el medio le llega a todos. HUB\@.
        \item Punto a punto: Sólo estoy unido a un nodo. Switch.
    \end{itemize}
    \item[Según el tipo de tranferencia de datos:] \
    \begin{itemize}
        \item Simple: solo transmite o recibe. Por ejemplo los TDT (para que una televisión analógica reciba señal digital).
        \item Half-duplex: transmite y recibe pero no simultáneamente. Por ejemplo el WIFI, aunque cambia muy rápido, por lo que no nos damos cuenta.
        \item Full-duplex: transmite y recibe simultáneamente. Por ejemplo Ethernet. 
    \end{itemize}
\end{description}

\subsection{Nomenclatura típica en figuras (Iconos)}
\begin{description}
    \item[HUB:] es un concentrador: permite centralizar los nodos de una red de computadoras, se implementa mediante un bus. 
    \item[Bridge:] funciona como un switch, pero uniendo tecnologías distintas, funciona a nivel 2. 
    \item[Switch:] tiene muchas bocas y tiene LAN\@. Necesitan un router para conectarse a otro switch.
    \item[Router:] tiene pocas bocas y lo que hace es conectar distintas redes. 
    \item[Cortafuegos:] bloquea el acceso no autorizado, permitiendo el autorizado. 
    \item[NAT:] traducción de direcciones de red. 
    \item[Switch multicapa:] (no entra) todas las bocas de un switch son de la misma LAN, pero esto a veces no nos interesa. Podemos hacer redes virtuales (VLAN) dividiendo un mismo switch en varias redes, permitiéndonos esto conectar dos switches distintos con la misma VLAN\@. Pero esto no nos permite movernos por distintas redes en el mismo switch, habría que pasar por un router, ya que necesitamos movernos a nivel de red para cambiar de red. Esto sí se puede hacer con un switch multicapa. 
\end{description}

\section{Diseño y estandarización de redes}

La idea principal que se sigue al diseñar redes es solucionar los problemas en capas. Se estandarizan Modelos de Referencia (no son implementaciones, solo una referencia): definición de las capas y las funcionalidades de cada una. Los principios que se siguen son que las funcionalidades distintas tienen que estar en distintas capas y minimizar el flujo de información entre las capas.\\

A continuación detallamos los problemas que tiene que resolver una red por capas: 
\begin{description}
    \item [A nivel físico:] hay que ver como transmitir los datos. Hay distintos tipos de codificaciones para enviar bits de información. 
    \item [Capa de enlace:] se encarga de los mecanismos de acceso al medio. Si hay un medio común, antes de transmitir datos tiene que asegurarse de que ningún equipo está transmitiendo. Suele seguir dos protocolos: 
        \begin{itemize}
            \item MAC, control de acceso al medio.
            \item LLC, control de acceso lógico para las primeras retransmisiones. Si algún paquete llega mal, retransmite varias veces. 
        \end{itemize}
    
    \item [Capa de red:] una vez llegado a este punto, se asume que no han habido colisiones en la comunicación. Esta capa se encarga principalmente de: 
        \begin{itemize}
            \item El direccionamiento: saber dar una dirección y tener un identificador dentro de la red. 
            \item El encaminamiento: saber cómo llegar al destino. 
        \end{itemize}

    \item [Capa de transporte:] se encarga de recuperar los paquetes que en la capa de enlace no se ha podido, es la capa de la fiabilidad. 
        \begin{itemize}
            \item Corrige errores.
            \item Gestiona la congestión. 
            \item Control de flujos: si hay un receptor más lento que el emisor, debe decirle al emisor que disminuya la velocidad de emisión, para adecuarse a la del receptor. 
        \end{itemize}
    Además se encarga de la multiplexación de datos: mediante puertos (los veremos más adelante) le indica al SO a qué aplicación corresponde cada paquete. 
    \item [Capa de aplicación:] los clientes y los servidores deben buscar alguna forma de comunicarse. 
\end{description}

\subsection{Modelo OSI vs TCP/IP}
\begin{table}[H]
\centering
\begin{tabular}{|c|}
    \hline
    Aplicación \\\hline
    Presentación \\\hline
    Sesión \\ \hline
    Transporte \\ \hline
    Red \\\hline
    Enlace \\\hline
    Física \\\hline
\end{tabular}
\caption{Modelo OSI.}
\label{tab:mod_osi}
\end{table}

\begin{table}[H]
\centering
\begin{tabular}{|c|}
    \hline
    Aplicación \\\hline
    Transporte \\ \hline
    Red \\\hline
    Red subyacente \\\hline
\end{tabular}
\caption{Modelo TCP/IP.}
\label{tab:mod_tcp}
\end{table}

El modelo OSI fue propuesto por la ISO y el TCP/IP por el Internet Engineering Task Force. Las tres primeras capas del modelo OSI se corresponden con la capa de aplicación, y las dos última con la red subyacente, que es la parte física. Esta última en el modelo TCP/IP depende un poco de la tecnología está implementada de una forma u otra, pero la comunicación con la capa de red no puede variar, pues la capa de red si está estandarizada. 

\begin{itemize}
    
    \item Las capas físicas solo se encargan de hacer la primera conexión.
    \item La capa de red, salto a salto se encarga de llegar al destino, usando routers y sus tablas de encaminamiento. 
    \item Una vez hecho el encaminamiento, las capas superiores son solo de los extremos, son los computadores de los extremos los que se comunican. 
    \item Por tanto, los computadores tienen las 5 capas (en el caso de TCP/IP) y los routers solo las 3 más bajas. 
\end{itemize}


\section{Terminología, conceptos y servicios}

\begin{figure}
    \centering
    \includegraphics[width=0.8\linewidth]{./images/comunicacionHosts.png}
    \caption{Comunicación real frente a comunicación virtual.}
    \label{fig:comHosts}
\end{figure}

En la Figura~\ref{fig:comHosts} vemos el camino que siguen los datos que le manda un emisor a un receptor. Cada capa le pone una cabecera a lo que le llega de arriba. A este proceso se le llama encapsulamiento. La capa física por su parte manda señales eléctricas en vez de bits. \\

Por otra parte, cuando la información va subiendo por las capas del modelo, cada capa le quita la cabecera correspondiente y ve si tiene que hacer algo. \\

Las flechas continuas es por donde realmente pasa la información y las discontinuas representan la comunicación virtual u horizontal, que en breves definiremos. \\

Las capas del mismo tipo en distintos equipos hablan directamente en sentido abstracto por medio de otras capas. Aunque solo se habla entre las capas del mismo tipo se usan recursos que nos proporcionan otras capas adyacentes. \\

\subsection{Definiciones importantes}
\begin{description}
    \item [SDU:] unidad de datos de servicio. Son los datos de una determinada capa. 
    \item [PDU:] unidad de datos de paquete. SDU + cabecera. 
    \item [Comunicación real:] o vertical. Es por donde realmente va la información. 
    \item [Comunicación virtual:] u horizontal. Es la comunicación entre capas del mismo tipo en distinto equipo. No es una comunicación directa, sino a través de otras capas. 
    \item [Entidades pares:] entidades iguales, las que hablan entre sí. 
    \item [Protocolo:] cómo se comunican las entidades pares, los paquetes que se mandan.
    \item [Interfaz:] cómo hablan las entidades adyacentes. 
    \item [Servicio:] lo que proporciona una capa a otra. 
    \item [Capa proveedora/usuaria del servicio:] una capa es proveedora de la de arriba y usuaria de la de abajo. 
    \item [Pila de protocolos:] los protocolos de las distintas capas.
    \item [Arquitectura de red:] modelo de referencia + pila de protocolos.
\end{description}

\subsection{Retardos en la comunicación}
Supongamos que queremos transmitir un paquete entre dos equipos (terminales), por medio de otro, un router. Veamos los retardos que tenemos en la comunicación. 
\begin{itemize}
    \item Para empezar, tenemos que saber el tamaño del paquete: en Ethernet un paquete típico son $1500 \text{Bytes}\cdot 8\nicefrac{\text{bits}}{\text{Byte}}$. Luego el tiempo que se tarda en transmitir el paquete sale como resultado de dividirlo entre los bps. Esto es el \textbf{tiempo de transmisión} que depende exclusivamente de la tarjeta de red. 
    \item Una vez transmitido, hay un retardo, que se corresponde con el \textbf{tiempo de propagación} ($\nicefrac{\text{distancia}}{\text{velocidad de transmisión}}$), que es el tiempo que que ocurre entre que se escribe el primer bit y que llega el primer bit al siguiente equipo. La velocidad de transmisión inalámbrica es a la velocidad de la luz y la de un cable suele rondar $\nicefrac{2}{3}$ de la velocidad de la luz. En los cables se pierde un bit por cada millón de bits, y suele deberse a colas llenas. 
    \item Cuando el paquete llega a un equipo intermedio, en nuestro caso el router; éste lo mete en una cola hasta que pueda procesarlo. El tiempo que está el paquete en dicha cola depende de la situación del equipo. Luego el tiempo que está el paquete en dicho equipo es el \textbf{tiempo de cola} más el \textbf{tiempo de procesamiento} (del orden de milisegundos) más el \textbf{tiempo de acceso al medio}.
    \item Ahora se envía el paquete al equipo destino, el tiempo de propagación, pese a calcularse de forma análoga al anterior, probablemente no coincida debido a distintas distancias, tarjetas de red\ldots. 
\end{itemize}

\subsection{Tipos de servicios}
Relacionado con el nivel de transporte, hay dos clasificaciones importantes: orientadas a conexión y a fiabilidad. 
\begin{itemize}
    \item Orientado a conexión (SOC): antes de mandar un paquete comprueba que el otro equipo esté encendido. 
        
    \item Orientados a la fiabilidad: se asegura de que todo funcione bien (que todos los bits de un archivo estén bien). Además, si algo falla la conexión se termina. 

     Para tener un protocolo fiable, contamos con los siguientes mecanismos: 
    \begin{itemize}
        \item Control de conexión. Ser fiable implica ser orientado a conexión. 
        \item Control de errores. 
        \item Control de congestión. Hablamos de congestión cuando las colas de los routers se llenan y empiezan a descartar paquetes. 
        \item Control de flujo. 
        \item Entrega ordenada. Si se envían muchos paquetes, estos deben llegar en orden. 
    \end{itemize}

    Algunos protocolos que estudiaremos más adelante: 
    \begin{itemize}
        \item TCP es un servicio orientado a conexión y fiable.
        \item UDP es un servicio no orientado a conexión y no es fiable. La finalidad de este protocolo es ser rápido. 
    \end{itemize}
\end{itemize}


\section{Internet: topología y direccionamiento}
Internet tiene dos cosas importantes:
\begin{itemize}
    \item Los protocolos de comunicación.
    \item Cómo se gestiona Internet. Las direcciones IP son únicas en todo el mundo (salvo algunas, ya lo veremos).
\end{itemize}

\subsection{Organización topológica}

Los operadores se establecen según la siguiente jerarquía:
\begin{itemize}
    \item \textbf{Tier 3}: los más cercanos a los usuarios, ISPs (Internet Service Provider).
    \item \textbf{Tier 2}: necesitan pasar por una Tier 1 para llegar a toda Internet y ofrecen servicios de conectividad a operadores de Tier 3.
    \item \textbf{Tier 1}: los que componen la estructura troncal de Internet. Están todos comunicados entre sí y hay como mínimo en dos continentes. 
\end{itemize}

Hay dos tipos de relaciones entre operadores:
\begin{itemize}
    \item \textbf{Tránsito:} conexiones entre distintos tier. Por ejemplo un tier 3 paga a un tier superior para enviar datos. 
    \item \textbf{Peering:} conexiones entre el mismo tier. 
\end{itemize}

Antiguamente, para que un ISP de un país hable con otro del mismo país había que ir hasta EEUU por falta de recursos. Más tarde se pusieron puntos neutros en cada país para comunicar operadores dentro de un mismo país. 


\subsection{Red Iris}

Es la red española para investigación. Todas las universidades públicas y centros de investigación están conectados a ella. \\

La red se divide según autonomía y por otro lado tiene conexiones externas con la red científica europea. 


\subsection{Direccionamiento por capas}

\begin{itemize}
    \item Enlace: depende de la tarjeta de red. Nesesitamos conocer la dirección MAC del siguiente punto. Las direcciones MAC son de la siguiente forma: AA\@:BB\@:CC\@:DD\@:EE\@:FF\@. Son en teoría únicas en todo el mundo. Pero realmente se ponen aleatorias para evitar seguimientos (puesto que si sabemos la dirección MAC de una tarjeta podemos hacer seguimiento de paquetes). 
    \item Red: direcciones IP\@: A\@.B\@.C\@.D\@. Las públicas son únicas en todo el mundo, las privadas no. 
    \item Transporte: direccionamiento a través de puertos, que identifican a qué proceso va un determinado paquete. 
    \item Aplicación: Nombres de dominio mediante DNS\@. 
\end{itemize}

