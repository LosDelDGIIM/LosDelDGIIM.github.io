\chapter{Introducción a los fundamentos de redes}
\subsection{Objetivos}

\subsection{Historia}
Un servicio de banda ancha es un servicio de velocidad grande, que se inición a partir del 2000. Comenzaron por 2Mbps.
El ADSL en España comenzó transmitiendo 256kbps (el máximo teórico del ADSL son 20Mbps, aunque lo normal son 10 o 12).
Las redes de cable eran HFC (redes de cable y fibras híbridas), pero ahora tenemos FTTH (Fiber to the home), fibra directa a casa. La fibra es el mejor material para transmitir información del mundo, además de que no cuenta con interferencias, consiguiendo varios Gbps.

Estos servicios eran usados por:
\begin{itemize}
    \item Televisiones (contaban con una resolución de $640\times 240$, necesitando un servicio de 2Mbps sin comprimir).
\end{itemize}

Cerca del 70\% del tráfico de internet es debido al multimedia, a través de las CDNs, redes de transmisión de contenidos.
Por ejemplo, un vídeo de Youtube cuenta con varias copias del mismo alrededor del mundo.

\section{Sistemas de comunicación y redes}
El sistema de comunicación típico es:

En un sistema de comunicaciones contamos con una fuente y con un transmisor (ambos en el mismo equipo), de forma que la fuente genera datos.
Después del transmisor, contamos con un canal de comunicación, el cual proboca errores:
\begin{itemize}
    \item Ruidos.
    \item Interferencias.
    \item Diafonías: sucede mucho en ADSL, al tener muchos cables en paralelo juntos puede suceder que la información de un cable se meta en otro.
\end{itemize}

En el final del destino, conamos con un equipo que cuenta con un receptor y con el destino (que espera los datos a recibir).


Cuando hablamos de redes, tenemos que tener varios equipos interconectados, que funcionen de forma autónoma (sin interferencia de nadie) y que se realice de forma eficaz.

\subsection{Primera red de comunicaciones}

La primera red de comunicaciones era la red de telefonía móvil.

Contábamos con nuestra línea de teléfono, que conectaba con una central de conmutación local, luego regional y luego nacional, la cual debía conectar con la central local a la que queríamos llamar.
Se usaba la conmutación de circuitos: 
\begin{itemize}
    \item Inicialmente se creaba un camino físico juntando cables. A dicho camino se le llamaba circuito.

        Era ineficiente porque dicho cable cuenta con una eficiencia del 50\%, debido a que aproximadamente se habla la mitad del tiempo de la llamada.

        Era un problema de seguridad el mal funcionamiento de una central, ya que dejaba sin servicio a miles de teléfonos.
\end{itemize}

Si ahora cambiamos los teléfonos por ordenadores y las centrales de conmutación por routerse, contamos con muchísimos caminos para conectar dos ordenadores, haciendo mucho más segura la red (a expensas de la seguridad en la red).

Ahora ya no tenemos un circuito físico, sino que son los routers quienes deciden a dónde enviar los paquetes y en qué momento hacerlo. Con el inconveniente de generar retardo pero con la ventaja de usar mejor el canal (si hay silencios, puede usarlo otro).

El departamento de defensa americana y posteriormente la NSF crearon las primeras redes asemejables a internet.

De una red esperamos:
\begin{itemize}
    \item Autonomía.
    \item Interconexión.
    \item Eficiencia.
\end{itemize}

Una red clásica va a tener equipos terminales (hosts) y equipos de interconexión, que permiten conectar toda la red.

\subsection{Líneas de transmisión}
Podemos contar con enlaces inalámbricos y cableados.

Comenzó con los enlaces cableados con cables de pares (pensado para transmitir 4kHz, la media en la voz humana), luego con cables coaxiales y fibra óptica.
Este último es el mejor medio guiado existente.

