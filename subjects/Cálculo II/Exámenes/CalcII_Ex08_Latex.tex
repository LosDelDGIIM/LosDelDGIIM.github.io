\documentclass[12pt]{article}

% Idioma y codificación
\usepackage[spanish, es-tabla]{babel}       %es-tabla para que se titule "Tabla"
\usepackage[utf8]{inputenc}

% Márgenes
\usepackage[a4paper,top=3cm,bottom=2.5cm,left=3cm,right=3cm]{geometry}

% Comentarios de bloque
\usepackage{verbatim}

% Paquetes de links
\usepackage[hidelinks]{hyperref}    % Permite enlaces
\usepackage{url}                    % redirecciona a la web

% Más opciones para enumeraciones
\usepackage{enumitem}

% Personalizar la portada
\usepackage{titling}

% Paquetes de tablas
\usepackage{multirow}


%------------------------------------------------------------------------

%Paquetes de figuras
\usepackage{caption}
\usepackage{subcaption} % Figuras al lado de otras
\usepackage{float}      % Poner figuras en el sitio indicado H.


% Paquetes de imágenes
\usepackage{graphicx}       % Paquete para añadir imágenes
\usepackage{transparent}    % Para manejar la opacidad de las figuras

% Paquete para usar colores
\usepackage[dvipsnames]{xcolor}
\usepackage{pagecolor}      % Para cambiar el color de la página

% Habilita tamaños de fuente mayores
\usepackage{fix-cm}

% Para los gráficos
\usepackage{tikz}

% Para poder situar los nodos en los grafos
\usetikzlibrary{positioning}


%------------------------------------------------------------------------

% Paquetes de matemáticas
\usepackage{mathtools, amsfonts, amssymb, mathrsfs}
\usepackage[makeroom]{cancel}     % Simplificar tachando
\usepackage{polynom}    % Divisiones y Ruffini
\usepackage{units} % Para poner fracciones diagonales con \nicefrac

\usepackage{pgfplots}   %Representar funciones
\pgfplotsset{compat=1.18}  % Versión 1.18

\usepackage{tikz-cd}    % Para usar diagramas de composiciones
\usetikzlibrary{calc}   % Para usar cálculo de coordenadas en tikz

%Definición de teoremas, etc.
\usepackage{amsthm}
%\swapnumbers   % Intercambia la posición del texto y de la numeración

\theoremstyle{plain}

\makeatletter
\@ifclassloaded{article}{
  \newtheorem{teo}{Teorema}[section]
}{
  \newtheorem{teo}{Teorema}[chapter]  % Se resetea en cada chapter
}
\makeatother

\newtheorem{coro}{Corolario}[teo]           % Se resetea en cada teorema
\newtheorem{prop}[teo]{Proposición}         % Usa el mismo contador que teorema
\newtheorem{lema}[teo]{Lema}                % Usa el mismo contador que teorema

\theoremstyle{remark}
\newtheorem*{observacion}{Observación}

\theoremstyle{definition}

\makeatletter
\@ifclassloaded{article}{
  \newtheorem{definicion}{Definición} [section]     % Se resetea en cada chapter
}{
  \newtheorem{definicion}{Definición} [chapter]     % Se resetea en cada chapter
}
\makeatother

\newtheorem*{notacion}{Notación}
\newtheorem*{ejemplo}{Ejemplo}
\newtheorem*{ejercicio*}{Ejercicio}             % No numerado
\newtheorem{ejercicio}{Ejercicio} [section]     % Se resetea en cada section


% Modificar el formato de la numeración del teorema "ejercicio"
\renewcommand{\theejercicio}{%
  \ifnum\value{section}=0 % Si no se ha iniciado ninguna sección
    \arabic{ejercicio}% Solo mostrar el número de ejercicio
  \else
    \thesection.\arabic{ejercicio}% Mostrar número de sección y número de ejercicio
  \fi
}


% \renewcommand\qedsymbol{$\blacksquare$}         % Cambiar símbolo QED
%------------------------------------------------------------------------

% Paquetes para encabezados
\usepackage{fancyhdr}
\pagestyle{fancy}
\fancyhf{}

\newcommand{\helv}{ % Modificación tamaño de letra
\fontfamily{}\fontsize{12}{12}\selectfont}
\setlength{\headheight}{15pt} % Amplía el tamaño del índice


%\usepackage{lastpage}   % Referenciar última pag   \pageref{LastPage}
\fancyfoot[C]{\thepage}

%------------------------------------------------------------------------

% Conseguir que no ponga "Capítulo 1". Sino solo "1."
\makeatletter
\@ifclassloaded{book}{
  \renewcommand{\chaptermark}[1]{\markboth{\thechapter.\ #1}{}} % En el encabezado
    
  \renewcommand{\@makechapterhead}[1]{%
  \vspace*{50\p@}%
  {\parindent \z@ \raggedright \normalfont
    \ifnum \c@secnumdepth >\m@ne
      \huge\bfseries \thechapter.\hspace{1em}\ignorespaces
    \fi
    \interlinepenalty\@M
    \Huge \bfseries #1\par\nobreak
    \vskip 40\p@
  }}
}
\makeatother

%------------------------------------------------------------------------
% Paquetes de cógido
\usepackage{minted}
\renewcommand\listingscaption{Código fuente}

\usepackage{fancyvrb}
% Personaliza el tamaño de los números de línea
\renewcommand{\theFancyVerbLine}{\small\arabic{FancyVerbLine}}

% Estilo para C++
\newminted{cpp}{
    frame=lines,
    framesep=2mm,
    baselinestretch=1.2,
    linenos,
    escapeinside=||
}

% para minted
\definecolor{LightGray}{rgb}{0.95,0.95,0.92}
\setminted{
    linenos=true,
    stepnumber=5,
    numberfirstline=true,
    autogobble,
    breaklines=true,
    breakautoindent=true,
    breaksymbolleft=,
    breaksymbolright=,
    breaksymbolindentleft=0pt,
    breaksymbolindentright=0pt,
    breaksymbolsepleft=0pt,
    breaksymbolsepright=0pt,
    fontsize=\footnotesize,
    bgcolor=LightGray,
    numbersep=10pt
}


\usepackage{listings} % Para incluir código desde un archivo

\renewcommand\lstlistingname{Código Fuente}
\renewcommand\lstlistlistingname{Índice de Códigos Fuente}

% Definir colores
\definecolor{vscodepurple}{rgb}{0.5,0,0.5}
\definecolor{vscodeblue}{rgb}{0,0,0.8}
\definecolor{vscodegreen}{rgb}{0,0.5,0}
\definecolor{vscodegray}{rgb}{0.5,0.5,0.5}
\definecolor{vscodebackground}{rgb}{0.97,0.97,0.97}
\definecolor{vscodelightgray}{rgb}{0.9,0.9,0.9}

% Configuración para el estilo de C similar a VSCode
\lstdefinestyle{vscode_C}{
  backgroundcolor=\color{vscodebackground},
  commentstyle=\color{vscodegreen},
  keywordstyle=\color{vscodeblue},
  numberstyle=\tiny\color{vscodegray},
  stringstyle=\color{vscodepurple},
  basicstyle=\scriptsize\ttfamily,
  breakatwhitespace=false,
  breaklines=true,
  captionpos=b,
  keepspaces=true,
  numbers=left,
  numbersep=5pt,
  showspaces=false,
  showstringspaces=false,
  showtabs=false,
  tabsize=2,
  frame=tb,
  framerule=0pt,
  aboveskip=10pt,
  belowskip=10pt,
  xleftmargin=10pt,
  xrightmargin=10pt,
  framexleftmargin=10pt,
  framexrightmargin=10pt,
  framesep=0pt,
  rulecolor=\color{vscodelightgray},
  backgroundcolor=\color{vscodebackground},
}

%------------------------------------------------------------------------

% Comandos definidos
\newcommand{\bb}[1]{\mathbb{#1}}
\newcommand{\cc}[1]{\mathcal{#1}}

% I prefer the slanted \leq
\let\oldleq\leq % save them in case they're every wanted
\let\oldgeq\geq
\renewcommand{\leq}{\leqslant}
\renewcommand{\geq}{\geqslant}

% Si y solo si
\newcommand{\sii}{\iff}

% Letras griegas
\newcommand{\eps}{\epsilon}
\newcommand{\veps}{\varepsilon}
\newcommand{\lm}{\lambda}

\newcommand{\ol}{\overline}
\newcommand{\ul}{\underline}
\newcommand{\wt}{\widetilde}
\newcommand{\wh}{\widehat}

\let\oldvec\vec
\renewcommand{\vec}{\overrightarrow}

% Derivadas parciales
\newcommand{\del}[2]{\frac{\partial #1}{\partial #2}}
\newcommand{\Del}[3]{\frac{\partial^{#1} #2}{\partial #3^{#1}}}
\newcommand{\deld}[2]{\dfrac{\partial #1}{\partial #2}}
\newcommand{\Deld}[3]{\dfrac{\partial^{#1} #2}{\partial #3^{#1}}}


\newcommand{\AstIg}{\stackrel{(\ast)}{=}}
\newcommand{\Hop}{\stackrel{L'H\hat{o}pital}{=}}

\newcommand{\red}[1]{{\color{red}#1}} % Para integrales, destacar los cambios.

% Método de integración
\newcommand{\MetInt}[2]{
    \left[\begin{array}{c}
        #1 \\ #2
    \end{array}\right]
}

% Declarar aplicaciones
% 1. Nombre aplicación
% 2. Dominio
% 3. Codominio
% 4. Variable
% 5. Imagen de la variable
\newcommand{\Func}[5]{
    \begin{equation*}
        \begin{array}{rrll}
            #1:& #2 & \longrightarrow & #3\\
               & #4 & \longmapsto & #5
        \end{array}
    \end{equation*}
}

%------------------------------------------------------------------------



\begin{document}

    % 1. Foto de fondo
    % 2. Título
    % 3. Encabezado Izquierdo
    % 4. Color de fondo
    % 5. Coord x del titulo
    % 6. Coord y del titulo
    % 7. Fecha

    
    % 1. Foto de fondo
% 2. Título
% 3. Encabezado Izquierdo
% 4. Color de fondo
% 5. Coord x del titulo
% 6. Coord y del titulo
% 7. Fecha

\newcommand{\portada}[7]{

    \portadaBase{#1}{#2}{#3}{#4}{#5}{#6}{#7}
    \portadaBook{#1}{#2}{#3}{#4}{#5}{#6}{#7}
}

\newcommand{\portadaExamen}[7]{

    \portadaBase{#1}{#2}{#3}{#4}{#5}{#6}{#7}
    \portadaArticle{#1}{#2}{#3}{#4}{#5}{#6}{#7}
}




\newcommand{\portadaBase}[7]{

    % Tiene la portada principal y la licencia Creative Commons
    
    % 1. Foto de fondo
    % 2. Título
    % 3. Encabezado Izquierdo
    % 4. Color de fondo
    % 5. Coord x del titulo
    % 6. Coord y del titulo
    % 7. Fecha
    
    
    \thispagestyle{empty}               % Sin encabezado ni pie de página
    \newgeometry{margin=0cm}        % Márgenes nulos para la primera página
    
    
    % Encabezado
    \fancyhead[L]{\helv #3}
    \fancyhead[R]{\helv \nouppercase{\leftmark}}
    
    
    \pagecolor{#4}        % Color de fondo para la portada
    
    \begin{figure}[p]
        \centering
        \transparent{0.3}           % Opacidad del 30% para la imagen
        
        \includegraphics[width=\paperwidth, keepaspectratio]{assets/#1}
    
        \begin{tikzpicture}[remember picture, overlay]
            \node[anchor=north west, text=white, opacity=1, font=\fontsize{60}{90}\selectfont\bfseries\sffamily, align=left] at (#5, #6) {#2};
            
            \node[anchor=south east, text=white, opacity=1, font=\fontsize{12}{18}\selectfont\sffamily, align=right] at (9.7, 3) {\textbf{\href{https://losdeldgiim.github.io/}{Los Del DGIIM}}};
            
            \node[anchor=south east, text=white, opacity=1, font=\fontsize{12}{15}\selectfont\sffamily, align=right] at (9.7, 1.8) {Doble Grado en Ingeniería Informática y Matemáticas\\Universidad de Granada};
        \end{tikzpicture}
    \end{figure}
    
    
    \restoregeometry        % Restaurar márgenes normales para las páginas subsiguientes
    \pagecolor{white}       % Restaurar el color de página
    
    
    \newpage
    \thispagestyle{empty}               % Sin encabezado ni pie de página
    \begin{tikzpicture}[remember picture, overlay]
        \node[anchor=south west, inner sep=3cm] at (current page.south west) {
            \begin{minipage}{0.5\paperwidth}
                \href{https://creativecommons.org/licenses/by-nc-nd/4.0/}{
                    \includegraphics[height=2cm]{assets/Licencia.png}
                }\vspace{1cm}\\
                Esta obra está bajo una
                \href{https://creativecommons.org/licenses/by-nc-nd/4.0/}{
                    Licencia Creative Commons Atribución-NoComercial-SinDerivadas 4.0 Internacional (CC BY-NC-ND 4.0).
                }\\
    
                Eres libre de compartir y redistribuir el contenido de esta obra en cualquier medio o formato, siempre y cuando des el crédito adecuado a los autores originales y no persigas fines comerciales. 
            \end{minipage}
        };
    \end{tikzpicture}
    
    
    
    % 1. Foto de fondo
    % 2. Título
    % 3. Encabezado Izquierdo
    % 4. Color de fondo
    % 5. Coord x del titulo
    % 6. Coord y del titulo
    % 7. Fecha


}


\newcommand{\portadaBook}[7]{

    % 1. Foto de fondo
    % 2. Título
    % 3. Encabezado Izquierdo
    % 4. Color de fondo
    % 5. Coord x del titulo
    % 6. Coord y del titulo
    % 7. Fecha

    % Personaliza el formato del título
    \pretitle{\begin{center}\bfseries\fontsize{42}{56}\selectfont}
    \posttitle{\par\end{center}\vspace{2em}}
    
    % Personaliza el formato del autor
    \preauthor{\begin{center}\Large}
    \postauthor{\par\end{center}\vfill}
    
    % Personaliza el formato de la fecha
    \predate{\begin{center}\huge}
    \postdate{\par\end{center}\vspace{2em}}
    
    \title{#2}
    \author{\href{https://losdeldgiim.github.io/}{Los Del DGIIM}}
    \date{Granada, #7}
    \maketitle
    
    \tableofcontents
}




\newcommand{\portadaArticle}[7]{

    % 1. Foto de fondo
    % 2. Título
    % 3. Encabezado Izquierdo
    % 4. Color de fondo
    % 5. Coord x del titulo
    % 6. Coord y del titulo
    % 7. Fecha

    % Personaliza el formato del título
    \pretitle{\begin{center}\bfseries\fontsize{42}{56}\selectfont}
    \posttitle{\par\end{center}\vspace{2em}}
    
    % Personaliza el formato del autor
    \preauthor{\begin{center}\Large}
    \postauthor{\par\end{center}\vspace{3em}}
    
    % Personaliza el formato de la fecha
    \predate{\begin{center}\huge}
    \postdate{\par\end{center}\vspace{5em}}
    
    \title{#2}
    \author{\href{https://losdeldgiim.github.io/}{Los Del DGIIM}}
    \date{Granada, #7}
    \thispagestyle{empty}               % Sin encabezado ni pie de página
    \maketitle
    \vfill
}
    \portadaExamen{ffccA4.jpg}{Cálculo II\\Examen VIII}{Cálculo II. Examen VIII}{MidnightBlue}{-8}{28}{2023}

    \begin{description}
        \item[Asignatura] Cálculo II.
        \item[Curso Académico] 2017-18.
        \item[Grado] Doble Grado en Ingeniería Informática y Matemáticas.
        \item[Grupo] Único.
        %\item[Profesor] María Victoria Velasco Collado.
        \item[Descripción] Primer Parcial. Derivación.
        %\item[Fecha] 24 de abril de 2019.
        %\item[Duración] 60 minutos.
    
    \end{description}
    \newpage
    
    \begin{ejercicio} Función uniformemente continua. Teorema de Heine.
\begin{definicion} [Continuidad Uniforme]
    Sea $f:A\to \bb{R}$ una función. Se dice que $f$ es uniformemente continua si:
    \begin{equation*}
        \forall {\varepsilon} > 0, \exists \delta>0 \mid \text{Si } x,y\in A, \text{ con } |x-y|<\delta \Longrightarrow |f(x)-f(y)|<{\varepsilon}
    \end{equation*}
\end{definicion}

\begin{teo}[Teorema de Heine]
    \begin{equation*}
        f:[a,b]\to \bb{R} \text{ uniformemente continua en } [a,b] \Longleftrightarrow f:[a,b]\to \bb{R} \text{ continua en } [a,b] 
    \end{equation*}
    \end{teo}

    \begin{proof}
        Procedemos mediante doble implicación.
        \begin{description}
            \item [$\Longrightarrow$)]
            Trivialmente, tomando $y=a$, tenemos que $f$ es continua en todo $y\in [a,b]$, por lo que $f$ es continua.

            \item [$\Longleftarrow$)]
            Demostramos mediante reducción al absurdo. Suponemos que $f$ es continua pero que no es uniformemente continua.

            Por tanto, por no ser uniformemente continua, podemos encontrar $x_n,y_n\in~[a,b]$ con $y_n-x_n\to 0$ y $|f(y_n)-f(x_n)|\geq \varepsilon_0$.

            Como $x_n,y_n$ están acotadas, por el Teorema de Bolzano-Weierstrass podemos encontrar una parcial $\sigma$ tal que $x_{\sigma(n)}\to x$ y $y_{\sigma(n)}\to y$.

            Puesto que $y_n-x_n=0$ y $y_{\sigma(n)}-x_{\sigma(n)}\to y-x$, por la unicidad del límite, concluimos que $y-x=0\Longrightarrow x=y$.

            Por la continuidad de $f$, ha de ser que $f(y_{\sigma(n)})-f(x_{\sigma(n)})\to f(y)-f(x)=0$ (por ser $x=y$). Esto, sin embargo, contradice que $|f(y_n)|-f(x_n)|\geq \varepsilon_0$.

            Por tal contradiccón, deducimos que la hipótesis es falsa, y que por tanto $f$ es uniformemente continua.
        \end{description}
    \end{proof}
    
\end{ejercicio}

\begin{ejercicio}
    Decir si son verdaderas o falsas las siguientes cuestiones, justificando la respuesta:

    \begin{enumerate}
        \item Sean $f:A\to \bb{R},g:B\to \bb{R}$ con $f(A)\subset B$ dos funciones uniformemente continuas. Entonces $g\circ f$ es uniformemente continua.

        Tenemos que $f$ es uniformemente continua; es decir:
        \begin{equation*}
            \forall \hat{\varepsilon} > 0, \exists \delta>0 \mid \text{Si } x,y\in A, \text{ con } |x-y|<\delta \Longrightarrow |f(x)-f(y)|<\hat{\varepsilon}
        \end{equation*}

        Tenemos que $g$ es uniformemente continua; es decir:
        \begin{equation*}
            \forall {\varepsilon} > 0, \exists \hat{\delta}>0 \mid \text{Si } x,y\in B, \text{ con } |x-y|<\hat{\delta} \Longrightarrow |g(x)-g(y)|<{\varepsilon}
        \end{equation*}

        En particular, tomando $\hat{\varepsilon} = \hat{\delta}$, y usando que $f(A)\subset B$, la continuidad uniforme de $g$ queda:
        \begin{equation*}
            \forall {\varepsilon} > 0, \exists \hat{\varepsilon}>0 \mid \text{Si } f(x),f(y)\in B, \text{ con } |f(x)-f(y)|<\hat{\varepsilon} \Longrightarrow |g(f(x))-g(f(y))|<{\varepsilon}
        \end{equation*}

        Uniendo lo que tenemos, queda:
        \begin{equation*}
            \forall \varepsilon>0, \exists \delta>0\mid  \text{Si } x,y\in A, \text{ con } |x-y|<\delta \Longrightarrow |(g\circ f)(x)-(g\circ f)(y)|<{\varepsilon}
        \end{equation*}

        Es decir, se ha demostrado que $g\circ f$ es uniformemente continua, por lo que es \textbf{cierto}.

        \item Toda función $f:\bb{R}^+_0\to \bb{R}$ uniformemente continua está acotada.

        Tomando $f(x)=x$, tenemos que es lipsitchziana por ser $f'(x)=1$ acotada. Al ser $f$ lipsitchziana, tenemos que $f$ es uniformemente continua.

        No obstante, $f$ no está acotada, por lo que el enunciado es \textbf{falso}.

        \item Si $f:\bb{R}^+\to \bb{R}$ es uniformemente continua y tiene límite en $+\infty$, es una función acotada.

        Tomando $f(x)=x$, tenemos que es lipsitchziana por ser $f'(x)=1$ acotada. Al ser $f$ lipsitchziana, tenemos que $f$ es uniformemente continua.

        No obstante, $f$ no está acotada, por lo que el enunciado es \textbf{falso}.

        \item Si $f:\bb{R}^+\to \bb{R}$ es uniformemente continua y tiene límite en $+\infty$, es una función acotada.

        Por reducción al absurdo, supongamos que $f$ no está acotada. Entonces, $$\exists \{x_n\}\;(x_n\in \bb{R}^+\;\forall n\in \bb{N}) \mid \{|f(x_n)|\}\to +\infty$$

        Como $f$ tiene límite en $+\infty$, se tiene que: $$\forall \{x_n'\}\to +\infty\;(x_n'\in \bb{R}^+\;\forall n\in \bb{N}) \Longrightarrow \{f(x_n')\}\to L\in \bb{R}$$

        Por tanto, tenemos que $\{x_n\}$ no diverge, por lo que está acotada.
        \begin{equation*}
            \exists M>0 \mid |x_n|\leq M \qquad \forall n\in \bb{N}
        \end{equation*}

        Consideramos ahora $f_{]0,M[}$. Como $f$ es uniformemente continua, transforma conjuntos acotados en conjuntos acotados. Por tanto, $f(]0,M[)$ está acotado.

        Por tanto, tenemos que $f(x_n)\in f(]0,M[) \quad \forall n\in \bb{N}$. Por tanto, $f(x_n)$ acotado. Pero como $\{|f(x_n)|\}$ diverge, llegamos a una contradicción.

        Por tanto, $f$ está acotada.

        \begin{observacion}
            Tenemos que $\forall \{x_n\}\to +\infty$, tenemos que la imagen de la cola está acotada por $L+\varepsilon$, y que la imagen de los primeros términos, por ser un conjunto acotado y ser $f$ uniformemente continua, también es uniformemente continua.
        \end{observacion}
        

        \item Toda función integrable tiene una primitiva. Pon un ejemplo.

        Si $f$ fuese continua, es cierto. No obstante, hay función integrables que no son continuas. Ejemplo de esto es la función de las palomitas o la función parte entera. Estas son integrables pero no admiten una primitiva.

        De admitir la función parte entera $E(x)$ una primitiva, tendríamos que $E(x)$ sería la derivada de una función continua, y esto no es posible por ser $E(x)$ discontinua.
    \end{enumerate}
\end{ejercicio}

\begin{ejercicio}
    Calcula los extremos relativos de la función $F:\bb{R}\to \bb{R}$ dada por:
    \begin{equation*}
        F(x)=\int_{-x}^x \frac{t^2(1-t^2)}{e^{t^2}} dt \qquad (x\in \bb{R})
    \end{equation*}

    Tenemos que el integrando es una función par y que es localmente integrable, por lo que:
    \begin{equation*}
        F(x)=\int_{-x}^x \frac{t^2(1-t^2)}{e^{t^2}} dt
        = 2\int^{x}_0 \frac{t^2(1-t^2)}{e^{t^2}} dt
    \end{equation*}

    Como el intervalo es Riemman Integrable, tenemos que, por el TFC tenemos que:
    \begin{equation*}
        F'(x)=2\cdot \frac{x^2(1-x^2)}{e^{x^2}} = 0\Longleftrightarrow x=0,\pm 1
    \end{equation*}
    Estudiamos la monotonía:
    \begin{itemize}
        \item \underline{Para $x<-1$}: $F'(x)<0 \Longrightarrow F(x)$ estrictamente decreciente.
        \item \underline{Para $-1<x<0$}: $F'(x)>0 \Longrightarrow F(x)$ estrictamente creciente.
        \item \underline{Para $0<x<1$}: $F'(x)>0 \Longrightarrow F(x)$ estrictamente creciente.
        \item \underline{Para $1<x$}: $F'(x)<0 \Longrightarrow F(x)$ estrictamente decreciente.
    \end{itemize}

    Por tanto, tenemos un mínimo relativo en $x=-1$ y un máximo relativo en $x=1$.
\end{ejercicio}

\begin{ejercicio}
    Calcula:
    \begin{enumerate}
        \item Una primitiva de la función $f(x)=\displaystyle \frac{\sen^3 x}{\sqrt{\cos x}}$ en el intervalo $\left[-\frac{\pi}{2},\frac{\pi}{2}\right]$.

        Se pide la integral indefinida:
        \begin{multline*}
            \int \frac{\sen^3 x}{\sqrt{\cos x}}\;dx =\MetInt{\cos x=t\quad x\in [-\frac{\pi}{2},\frac{\pi}{2}]}{\sen x\;dx = dt}
            = \int \frac{\sen^3 x}{\sqrt{t}}\;\frac{dt}{\sen x}
            = \int \frac{\sen^2 x}{\sqrt{t}}\;dt
            =\\
            = \int \frac{1-t^2}{\sqrt{t}}\;dt
            = 2\int \frac{1}{2\sqrt{t}}\;dt -\int t^{\frac{3}{2}}\;dx
            = 2\sqrt{t} - \frac{t^{5/2}}{\frac{5}{2}} +C
            =\\= 2\sqrt{t} -\frac{2}{5}t\sqrt{t} +C
            = 2\sqrt{\cos x} -\frac{2}{5}\cos x\sqrt{\cos x} +C
        \end{multline*}

        Por tanto, una primitiva de $f(x)$ es:
        \begin{equation*}
            F(x)=2\sqrt{\cos x} -\frac{2}{5}\cos x\sqrt{\cos x}
        \end{equation*}

        \item $\displaystyle \int\sen(x)e^{-2x}dx$ 

        \begin{multline*}
            \int\sen(x)e^{-2x}dx=\MetInt{u(x)=e^{-2x}\quad u'(x)=-2e^{-2x}}{v'(x)=\sen x \quad v(x)=-\cos x} = -\cos x e^{-2x} -2\int \cos x e^{-2x}\;dx
            =\\=\MetInt{u(x)=e^{-2x}\quad u'(x)=-2e^{-2x}}{v'(x)=\cos x \quad v(x)=\sen x}
            = -\cos x e^{-2x}-2\sen x e^{-2x}+4\int e^{-2x}\sen x\;dx
        \end{multline*}

        Por tanto, se trata de una integral cíclica. Tenemos:
        \begin{equation*}
            \int e^{-2x}\sen x\;dx = \frac{e^{-2x}}{3}\left(\cos x +2\sen x\right)+C
        \end{equation*}
    \end{enumerate}
\end{ejercicio}
    



\end{document}