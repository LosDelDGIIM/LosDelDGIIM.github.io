\documentclass[12pt]{article}

% Idioma y codificación
\usepackage[spanish, es-tabla]{babel}       %es-tabla para que se titule "Tabla"
\usepackage[utf8]{inputenc}

% Márgenes
\usepackage[a4paper,top=3cm,bottom=2.5cm,left=3cm,right=3cm]{geometry}

% Comentarios de bloque
\usepackage{verbatim}

% Paquetes de links
\usepackage[hidelinks]{hyperref}    % Permite enlaces
\usepackage{url}                    % redirecciona a la web

% Más opciones para enumeraciones
\usepackage{enumitem}

% Personalizar la portada
\usepackage{titling}

% Paquetes de tablas
\usepackage{multirow}


%------------------------------------------------------------------------

%Paquetes de figuras
\usepackage{caption}
\usepackage{subcaption} % Figuras al lado de otras
\usepackage{float}      % Poner figuras en el sitio indicado H.


% Paquetes de imágenes
\usepackage{graphicx}       % Paquete para añadir imágenes
\usepackage{transparent}    % Para manejar la opacidad de las figuras

% Paquete para usar colores
\usepackage[dvipsnames]{xcolor}
\usepackage{pagecolor}      % Para cambiar el color de la página

% Habilita tamaños de fuente mayores
\usepackage{fix-cm}

% Para los gráficos
\usepackage{tikz}

% Para poder situar los nodos en los grafos
\usetikzlibrary{positioning}


%------------------------------------------------------------------------

% Paquetes de matemáticas
\usepackage{mathtools, amsfonts, amssymb, mathrsfs}
\usepackage[makeroom]{cancel}     % Simplificar tachando
\usepackage{polynom}    % Divisiones y Ruffini
\usepackage{units} % Para poner fracciones diagonales con \nicefrac

\usepackage{pgfplots}   %Representar funciones
\pgfplotsset{compat=1.18}  % Versión 1.18

\usepackage{tikz-cd}    % Para usar diagramas de composiciones
\usetikzlibrary{calc}   % Para usar cálculo de coordenadas en tikz

%Definición de teoremas, etc.
\usepackage{amsthm}
%\swapnumbers   % Intercambia la posición del texto y de la numeración

\theoremstyle{plain}

\makeatletter
\@ifclassloaded{article}{
  \newtheorem{teo}{Teorema}[section]
}{
  \newtheorem{teo}{Teorema}[chapter]  % Se resetea en cada chapter
}
\makeatother

\newtheorem{coro}{Corolario}[teo]           % Se resetea en cada teorema
\newtheorem{prop}[teo]{Proposición}         % Usa el mismo contador que teorema
\newtheorem{lema}[teo]{Lema}                % Usa el mismo contador que teorema

\theoremstyle{remark}
\newtheorem*{observacion}{Observación}

\theoremstyle{definition}

\makeatletter
\@ifclassloaded{article}{
  \newtheorem{definicion}{Definición} [section]     % Se resetea en cada chapter
}{
  \newtheorem{definicion}{Definición} [chapter]     % Se resetea en cada chapter
}
\makeatother

\newtheorem*{notacion}{Notación}
\newtheorem*{ejemplo}{Ejemplo}
\newtheorem*{ejercicio*}{Ejercicio}             % No numerado
\newtheorem{ejercicio}{Ejercicio} [section]     % Se resetea en cada section


% Modificar el formato de la numeración del teorema "ejercicio"
\renewcommand{\theejercicio}{%
  \ifnum\value{section}=0 % Si no se ha iniciado ninguna sección
    \arabic{ejercicio}% Solo mostrar el número de ejercicio
  \else
    \thesection.\arabic{ejercicio}% Mostrar número de sección y número de ejercicio
  \fi
}


% \renewcommand\qedsymbol{$\blacksquare$}         % Cambiar símbolo QED
%------------------------------------------------------------------------

% Paquetes para encabezados
\usepackage{fancyhdr}
\pagestyle{fancy}
\fancyhf{}

\newcommand{\helv}{ % Modificación tamaño de letra
\fontfamily{}\fontsize{12}{12}\selectfont}
\setlength{\headheight}{15pt} % Amplía el tamaño del índice


%\usepackage{lastpage}   % Referenciar última pag   \pageref{LastPage}
\fancyfoot[C]{\thepage}

%------------------------------------------------------------------------

% Conseguir que no ponga "Capítulo 1". Sino solo "1."
\makeatletter
\@ifclassloaded{book}{
  \renewcommand{\chaptermark}[1]{\markboth{\thechapter.\ #1}{}} % En el encabezado
    
  \renewcommand{\@makechapterhead}[1]{%
  \vspace*{50\p@}%
  {\parindent \z@ \raggedright \normalfont
    \ifnum \c@secnumdepth >\m@ne
      \huge\bfseries \thechapter.\hspace{1em}\ignorespaces
    \fi
    \interlinepenalty\@M
    \Huge \bfseries #1\par\nobreak
    \vskip 40\p@
  }}
}
\makeatother

%------------------------------------------------------------------------
% Paquetes de cógido
\usepackage{minted}
\renewcommand\listingscaption{Código fuente}

\usepackage{fancyvrb}
% Personaliza el tamaño de los números de línea
\renewcommand{\theFancyVerbLine}{\small\arabic{FancyVerbLine}}

% Estilo para C++
\newminted{cpp}{
    frame=lines,
    framesep=2mm,
    baselinestretch=1.2,
    linenos,
    escapeinside=||
}

% para minted
\definecolor{LightGray}{rgb}{0.95,0.95,0.92}
\setminted{
    linenos=true,
    stepnumber=5,
    numberfirstline=true,
    autogobble,
    breaklines=true,
    breakautoindent=true,
    breaksymbolleft=,
    breaksymbolright=,
    breaksymbolindentleft=0pt,
    breaksymbolindentright=0pt,
    breaksymbolsepleft=0pt,
    breaksymbolsepright=0pt,
    fontsize=\footnotesize,
    bgcolor=LightGray,
    numbersep=10pt
}


\usepackage{listings} % Para incluir código desde un archivo

\renewcommand\lstlistingname{Código Fuente}
\renewcommand\lstlistlistingname{Índice de Códigos Fuente}

% Definir colores
\definecolor{vscodepurple}{rgb}{0.5,0,0.5}
\definecolor{vscodeblue}{rgb}{0,0,0.8}
\definecolor{vscodegreen}{rgb}{0,0.5,0}
\definecolor{vscodegray}{rgb}{0.5,0.5,0.5}
\definecolor{vscodebackground}{rgb}{0.97,0.97,0.97}
\definecolor{vscodelightgray}{rgb}{0.9,0.9,0.9}

% Configuración para el estilo de C similar a VSCode
\lstdefinestyle{vscode_C}{
  backgroundcolor=\color{vscodebackground},
  commentstyle=\color{vscodegreen},
  keywordstyle=\color{vscodeblue},
  numberstyle=\tiny\color{vscodegray},
  stringstyle=\color{vscodepurple},
  basicstyle=\scriptsize\ttfamily,
  breakatwhitespace=false,
  breaklines=true,
  captionpos=b,
  keepspaces=true,
  numbers=left,
  numbersep=5pt,
  showspaces=false,
  showstringspaces=false,
  showtabs=false,
  tabsize=2,
  frame=tb,
  framerule=0pt,
  aboveskip=10pt,
  belowskip=10pt,
  xleftmargin=10pt,
  xrightmargin=10pt,
  framexleftmargin=10pt,
  framexrightmargin=10pt,
  framesep=0pt,
  rulecolor=\color{vscodelightgray},
  backgroundcolor=\color{vscodebackground},
}

%------------------------------------------------------------------------

% Comandos definidos
\newcommand{\bb}[1]{\mathbb{#1}}
\newcommand{\cc}[1]{\mathcal{#1}}

% I prefer the slanted \leq
\let\oldleq\leq % save them in case they're every wanted
\let\oldgeq\geq
\renewcommand{\leq}{\leqslant}
\renewcommand{\geq}{\geqslant}

% Si y solo si
\newcommand{\sii}{\iff}

% Letras griegas
\newcommand{\eps}{\epsilon}
\newcommand{\veps}{\varepsilon}
\newcommand{\lm}{\lambda}

\newcommand{\ol}{\overline}
\newcommand{\ul}{\underline}
\newcommand{\wt}{\widetilde}
\newcommand{\wh}{\widehat}

\let\oldvec\vec
\renewcommand{\vec}{\overrightarrow}

% Derivadas parciales
\newcommand{\del}[2]{\frac{\partial #1}{\partial #2}}
\newcommand{\Del}[3]{\frac{\partial^{#1} #2}{\partial #3^{#1}}}
\newcommand{\deld}[2]{\dfrac{\partial #1}{\partial #2}}
\newcommand{\Deld}[3]{\dfrac{\partial^{#1} #2}{\partial #3^{#1}}}


\newcommand{\AstIg}{\stackrel{(\ast)}{=}}
\newcommand{\Hop}{\stackrel{L'H\hat{o}pital}{=}}

\newcommand{\red}[1]{{\color{red}#1}} % Para integrales, destacar los cambios.

% Método de integración
\newcommand{\MetInt}[2]{
    \left[\begin{array}{c}
        #1 \\ #2
    \end{array}\right]
}

% Declarar aplicaciones
% 1. Nombre aplicación
% 2. Dominio
% 3. Codominio
% 4. Variable
% 5. Imagen de la variable
\newcommand{\Func}[5]{
    \begin{equation*}
        \begin{array}{rrll}
            #1:& #2 & \longrightarrow & #3\\
               & #4 & \longmapsto & #5
        \end{array}
    \end{equation*}
}

%------------------------------------------------------------------------



\begin{document}

    % 1. Foto de fondo
    % 2. Título
    % 3. Encabezado Izquierdo
    % 4. Color de fondo
    % 5. Coord x del titulo
    % 6. Coord y del titulo
    % 7. Fecha

    
    % 1. Foto de fondo
% 2. Título
% 3. Encabezado Izquierdo
% 4. Color de fondo
% 5. Coord x del titulo
% 6. Coord y del titulo
% 7. Fecha

\newcommand{\portada}[7]{

    \portadaBase{#1}{#2}{#3}{#4}{#5}{#6}{#7}
    \portadaBook{#1}{#2}{#3}{#4}{#5}{#6}{#7}
}

\newcommand{\portadaExamen}[7]{

    \portadaBase{#1}{#2}{#3}{#4}{#5}{#6}{#7}
    \portadaArticle{#1}{#2}{#3}{#4}{#5}{#6}{#7}
}




\newcommand{\portadaBase}[7]{

    % Tiene la portada principal y la licencia Creative Commons
    
    % 1. Foto de fondo
    % 2. Título
    % 3. Encabezado Izquierdo
    % 4. Color de fondo
    % 5. Coord x del titulo
    % 6. Coord y del titulo
    % 7. Fecha
    
    
    \thispagestyle{empty}               % Sin encabezado ni pie de página
    \newgeometry{margin=0cm}        % Márgenes nulos para la primera página
    
    
    % Encabezado
    \fancyhead[L]{\helv #3}
    \fancyhead[R]{\helv \nouppercase{\leftmark}}
    
    
    \pagecolor{#4}        % Color de fondo para la portada
    
    \begin{figure}[p]
        \centering
        \transparent{0.3}           % Opacidad del 30% para la imagen
        
        \includegraphics[width=\paperwidth, keepaspectratio]{assets/#1}
    
        \begin{tikzpicture}[remember picture, overlay]
            \node[anchor=north west, text=white, opacity=1, font=\fontsize{60}{90}\selectfont\bfseries\sffamily, align=left] at (#5, #6) {#2};
            
            \node[anchor=south east, text=white, opacity=1, font=\fontsize{12}{18}\selectfont\sffamily, align=right] at (9.7, 3) {\textbf{\href{https://losdeldgiim.github.io/}{Los Del DGIIM}}};
            
            \node[anchor=south east, text=white, opacity=1, font=\fontsize{12}{15}\selectfont\sffamily, align=right] at (9.7, 1.8) {Doble Grado en Ingeniería Informática y Matemáticas\\Universidad de Granada};
        \end{tikzpicture}
    \end{figure}
    
    
    \restoregeometry        % Restaurar márgenes normales para las páginas subsiguientes
    \pagecolor{white}       % Restaurar el color de página
    
    
    \newpage
    \thispagestyle{empty}               % Sin encabezado ni pie de página
    \begin{tikzpicture}[remember picture, overlay]
        \node[anchor=south west, inner sep=3cm] at (current page.south west) {
            \begin{minipage}{0.5\paperwidth}
                \href{https://creativecommons.org/licenses/by-nc-nd/4.0/}{
                    \includegraphics[height=2cm]{assets/Licencia.png}
                }\vspace{1cm}\\
                Esta obra está bajo una
                \href{https://creativecommons.org/licenses/by-nc-nd/4.0/}{
                    Licencia Creative Commons Atribución-NoComercial-SinDerivadas 4.0 Internacional (CC BY-NC-ND 4.0).
                }\\
    
                Eres libre de compartir y redistribuir el contenido de esta obra en cualquier medio o formato, siempre y cuando des el crédito adecuado a los autores originales y no persigas fines comerciales. 
            \end{minipage}
        };
    \end{tikzpicture}
    
    
    
    % 1. Foto de fondo
    % 2. Título
    % 3. Encabezado Izquierdo
    % 4. Color de fondo
    % 5. Coord x del titulo
    % 6. Coord y del titulo
    % 7. Fecha


}


\newcommand{\portadaBook}[7]{

    % 1. Foto de fondo
    % 2. Título
    % 3. Encabezado Izquierdo
    % 4. Color de fondo
    % 5. Coord x del titulo
    % 6. Coord y del titulo
    % 7. Fecha

    % Personaliza el formato del título
    \pretitle{\begin{center}\bfseries\fontsize{42}{56}\selectfont}
    \posttitle{\par\end{center}\vspace{2em}}
    
    % Personaliza el formato del autor
    \preauthor{\begin{center}\Large}
    \postauthor{\par\end{center}\vfill}
    
    % Personaliza el formato de la fecha
    \predate{\begin{center}\huge}
    \postdate{\par\end{center}\vspace{2em}}
    
    \title{#2}
    \author{\href{https://losdeldgiim.github.io/}{Los Del DGIIM}}
    \date{Granada, #7}
    \maketitle
    
    \tableofcontents
}




\newcommand{\portadaArticle}[7]{

    % 1. Foto de fondo
    % 2. Título
    % 3. Encabezado Izquierdo
    % 4. Color de fondo
    % 5. Coord x del titulo
    % 6. Coord y del titulo
    % 7. Fecha

    % Personaliza el formato del título
    \pretitle{\begin{center}\bfseries\fontsize{42}{56}\selectfont}
    \posttitle{\par\end{center}\vspace{2em}}
    
    % Personaliza el formato del autor
    \preauthor{\begin{center}\Large}
    \postauthor{\par\end{center}\vspace{3em}}
    
    % Personaliza el formato de la fecha
    \predate{\begin{center}\huge}
    \postdate{\par\end{center}\vspace{5em}}
    
    \title{#2}
    \author{\href{https://losdeldgiim.github.io/}{Los Del DGIIM}}
    \date{Granada, #7}
    \thispagestyle{empty}               % Sin encabezado ni pie de página
    \maketitle
    \vfill
}
    \portadaExamen{ffccA4.jpg}{Cálculo II\\Examen IV}{Cálculo II. Examen IV}{MidnightBlue}{-8}{28}{2023}

    \begin{description}
        \item[Asignatura] Cálculo II.
        \item[Curso Académico] 2021-22.
        \item[Grado] Doble Grado en Ingeniería Informática y Matemáticas.
        \item[Grupo] Único.
        \item[Profesor] María Victoria Velasco Collado.
        \item[Descripción] Primer Parcial. Derivación. Temas 1-4.
        \item[Fecha] 22 de abril de 2022.
        %\item[Duración] 60 minutos.
    
    \end{description}
    \newpage
    
    \begin{ejercicio}\textbf{[2 puntos]}
Dada $f:\bb{R}^+ \to \bb{R}^+$, demostrar las siguientes afirmaciones:
\begin{enumerate}
    \item Si $f$ es derivable en $\bb{R}^+$ y $|f'(x)|\leq f(x)$ para cada $x\in \bb{R}^+$, entonces:
    \begin{equation*}
        |\ln (f(y)) -\ln(f(x))| \leq |y-x| \qquad \forall x,y\in \bb{R}^+
    \end{equation*}

    Sea $g:\bb{R}^+ \to \bb{R}$ dada por $g(x)=\ln x$, y consideramos $g\circ f$.
    \begin{equation*}
        (g\circ f)'(x) = g'(f(x))\cdot f'(x) = \frac{f'(x)}{f(x)} 
    \end{equation*}

    Usando que $|f'(x)|\leq f(x)$, tenemos que:
    \begin{equation*}
        \left|(g\circ f)'(x)\right| = \left|\frac{f'(x)}{f(x)}\right| \leq \frac{f(x)}{|f(x)|} \leq 1
    \end{equation*}

    Por tanto, como su derivada está acotada, tengo que $g\circ f$ es lipschitziana, con constante de Lipschitz menor o igual a la cota de la derivada; es decir, $M_0 \leq 1$.

    Por tanto, por ser lipschitziana, tengo que:
    \begin{equation*}
        |(g\circ f)(y)-(g\circ f)(y)| \leq M_0 |y-x| \leq  |y-x| \qquad \forall x,y\in \bb{R}^+
    \end{equation*}

    Es decir, ha quedado demostrado.

    \item Si $f$ es dos veces derivable en $\bb{R}^+$, entonces $\ln(f(x))$ es cóncava hacia arriba si y solo si:
    \begin{equation*}
        f''(x)f(x)-f'(x)f'(x)\geq 0 \qquad \forall x\in \bb{R}^+
    \end{equation*}

    Sea $g:\bb{R}^+ \to \bb{R}$ dada por $g(x)=\ln x$, y consideramos $g\circ f$.

    Como $f$ es dos veces derivable, tengo que $g$ también lo es. Sé que $(g\circ f)(x)$ es cóncava hacia abajo si y solo si $(g\circ f)''(x)\geq 0\;\forall x\in \bb{R}^+$. Calculo por tanto la segunda derivada:
    \begin{equation*}
        (g\circ f)'(x) = g'(f(x))\cdot f'(x) = \frac{f'(x)}{f(x)} 
    \end{equation*}
    \begin{equation*}
        (g\circ f)''(x) = \frac{f''(x)f(x)-(f'(x))^2}{(f(x))^2} \geq 0 \Longleftrightarrow
        f''(x)f(x)-(f'(x))^2 \geq 0 \qquad \forall x\in \bb{R}^+
    \end{equation*}

    Por tanto, tenemos que el resultado es cierto.
\end{enumerate}
    
\end{ejercicio}

\begin{ejercicio}\textbf{[2 puntos]}
    Probar que $\arctan \left(\frac{1-x}{1+x}\right) + \arctan x = \frac{\pi}{4}$, para cada $x>-1$. ¿Es constante la función $f(x):=\arctan \left(\frac{1-x}{1+x}\right) + \arctan x$ en su dominio de definición?\\

    Su dominio de definición es su dominio maximal. El dominio de $\arctan x$ es $\bb{R}$, por lo que:
    \begin{equation*}
        Dom(f)=\bb{R}-\{-1\}
    \end{equation*}

    Veamos ahora si es constante. $f$ es derivable en $\bb{R}-\{-1\}$, por lo que calculo su primera derivada:
    \begin{multline*}
        f'(x) = \frac{1}{1+\left(\frac{1-x}{1+x}\right)^2} \cdot \frac{-1(1+x)-(1-x)}{(1+x)^2} + \frac{1}{1+x^2}
        = \frac{1}{1+\left(\frac{1-x}{1+x}\right)^2} \cdot \frac{-2}{(1+x)^2} + \frac{1}{1+x^2}
        =\\=
        \frac{-2}{(1+x)^2+(1-x)^2} + \frac{1}{1+x^2}
        = \frac{-2}{2(1+x^2)} + \frac{1}{1+x^2} = 0 \qquad \forall x \in \bb{R}-\{-1\}
    \end{multline*}
    Por tanto, tengo que $f$ es constante en cada una de las restricciones a $]-1, +\infty[$ y $]-\infty, -1[$. No obstante, no tenemos asegurado que la imagen sea igual a ambos lados de $x=-1$.
    \begin{gather*}
        f(0)=\arctan 1 + \arctan 0 = \frac{\pi}{4} + 0 = \frac{\pi}{4} \\
        f(-2) = \arctan(-3) + \arctan (-2) < 0
    \end{gather*}

    Por tanto, tenemos que \textbf{no} es constante en su dominio de definición.
    \begin{equation*}
        Im(f_{|]-1, +\infty[}) = \frac{\pi}{4} \qquad Im(f_{|]-\infty, -1[}) = \arctan(-3) + \arctan (-2) < 0
    \end{equation*}

    Además, podemos afirmar que, efectivamente, $\arctan \left(\frac{1-x}{1+x}\right) + \arctan x = \frac{\pi}{4}$, para cada $x>-1$.
\end{ejercicio}

\begin{ejercicio}\textbf{[2 puntos]} Mediante el desarrollo de Taylor, calcular
    \begin{equation*}
        \lim_{x\to 0} \frac{\sen (x^3) - \tan^3(x)}{3x^5}.
    \end{equation*}
    
    Definimos $f(x)=\sen (x^3) - \tan^3(x)$. Entonces:
    \begin{equation*}
        \lim_{x\to 0} \frac{\sen (x^3) - \tan^3(x)}{3x^5} = \frac{1}{3}\lim_{x\to 0} \frac{f(x)}{x^5}
        = \frac{1}{3}\lim_{x\to 0} \cancel{\frac{f(x)-P_{5,0}^f(x)}{x^5}} + \frac{P_{5,0}^f(x)}{x^5}
        = \frac{1}{3}\lim_{x\to 0} \frac{P_{5,0}^f(x)}{x^5}
    \end{equation*}
    
    Calculamos por tanto los polinomios de Taylor:
    \begin{equation*}
        P_{2, 0}^{\sen x}(x) = x
        \Longrightarrow
        P_{6, 0}^{\sen (x^3)}(x) = x^3 = P_{5, 0}^{\sen (x^3)}(x)
    \end{equation*}
    \begin{equation*}
        P_{5, 0}^{\tan x}(x) = x+\frac{1}{3}x^3 + \frac{2}{15}x^5
        \Longrightarrow
        P_{5, 0}^{\tan^3 x}(x) = \left[\left( x+\frac{1}{3}x^3 + \frac{2}{15}x^5\right)^3\right]_{n=5} = x^3 + 3\cdot \frac{1}{3}x^5 = x^3+x^5
    \end{equation*}
    Por tanto, 
    \begin{equation*}
        P_{5,0}^f(x) = P_{5, 0}^{\sen (x^3)}(x) - P_{5, 0}^{\tan^3 x}(x) = x^3 - x^3 -x^5 = -x^5
    \end{equation*}

    En conclusión, tenemos que:
    \begin{equation*}
        \lim_{x\to 0} \frac{\sin (x^3) - \tan^3(x)}{3x^5}
        = \frac{1}{3}\lim_{x\to 0} \frac{P_{5,0}^f(x)}{x^5}
        = \frac{1}{3}\lim_{x\to 0} \frac{-x^5}{x^5}
        = \frac{1}{3}\lim_{x\to 0} -1 = -\frac{1}{3}
    \end{equation*}
\end{ejercicio}

\begin{ejercicio}\textbf{[2 puntos]} Sean $a,b\in \bb{R}^+$. Estudiar el comportamiento en $x=0$ de la función $f:\bb{R}^+ \to \bb{R}$ dada por
\begin{equation*}
    f(x)=\left(\frac{2a^x + 3b^x}{5}\right)^{\frac{1}{x}} \qquad (x\in \bb{R}^+).
\end{equation*}

    Estudiar su comportamiento en el 0 equivale a calcular el límite si $x\to 0^+$. Por tanto,
    \begin{equation*}
        \lim_{x\to 0^+}f(x) = \lim_{x\to 0^+} \left(\frac{2a^x + 3b^x}{5}\right)^{\frac{1}{x}} = [1^\infty] = \lim_{x\to 0^+} e^{\frac{\ln\left(\frac{2a^x + 3b^x}{5}\right)}{x}} \stackrel{Ec.\;\ref{Ej4.Ind}}{=}e^{\ln \left(\sqrt[5]{a^2b^3}\right)} = \sqrt[5]{a^2b^3}
    \end{equation*}

    donde previamente he tenido que resolver el siguiente límite:
    \begin{multline}\label{Ej4.Ind}
        \lim_{x\to 0^+}  \frac{\ln\left(\frac{2a^x + 3b^x}{5}\right)}{x}
        = \lim_{x\to 0^+}  \frac{\ln\left(2a^x + 3b^x\right)-\ln 5}{x}
        = \left[\frac{0}{0}\right] \Hop\\=
        \lim_{x\to 0^+} \frac{2a^x\ln a + 3b^x \ln b}{2a^x+3b^x} = \frac{2\ln a + 3\ln b}{5} = \ln \left(\sqrt[5]{a^2b^3}\right)
    \end{multline}
\end{ejercicio}

\begin{ejercicio}\textbf{[2 puntos]} Para qué punto $P$ del eje de abcisas el perímetro del triángulo de vértices $P=(x,0),\;A=(0,2)$ y $B=(10,3)$ es menor?
    \begin{figure}[H]
        \centering
        \begin{tikzpicture}
        \begin{axis}[
            xlabel=$x$,
            ylabel=$y$,
            xmin=-1,
            xmax=12,
            ymin=-1,
            ymax=7,
            axis lines=middle,
            width=12cm,
            height=5cm,
            samples=90 % número de muestras para la función
        ]

        \def\x{4}
        
        \addplot[fill=orange,fill opacity=0.25] coordinates {(\x,0) (0,2) (10,3) (\x,0)};

        \addplot[only marks,mark=*,mark size=2pt,color=red] coordinates {(\x,0)};
        \node[label={above: $P(x, 0)$}] at (axis cs:\x,0) {};

        \addplot[only marks,mark=*,mark size=2pt,color=red] coordinates {(0,2)};
        \node[label={above right: $A(0,2)$}] at (axis cs:0,2) {};

        \addplot[only marks,mark=*,mark size=2pt,color=red] coordinates {(10, 3)};
        \node[label={above right: $B(10,3)$}] at (axis cs:10-0.3,3-0.3) {};

            
        \end{axis}
        \end{tikzpicture}
    \end{figure}

    Calculamos la longitud de cada lado del triángulo:
    \begin{gather*}
        \overline{AP} = \sqrt{x^2+2^2} = \sqrt{x^2+4}
        \qquad
        \overline{AB} = \sqrt{10^2+1^2} = \sqrt{101}
        \\
        \overline{PB} = \sqrt{(10-x)^2+3^2} = \sqrt{109 +x^2 -20x}
    \end{gather*}

    Sea $P$ la función que determina el perímetro:
    \begin{equation*}
        \begin{array}{rl}
            P:\bb{R} & \longrightarrow \bb{R}\\
                    x & \longrightarrow P(x) = \overline{AP} + \overline{AB}+\overline{PB} = \sqrt{x^2+4} + \sqrt{101} + \sqrt{109+x^2-20x}
        \end{array}
    \end{equation*}

    Como los discriminantes son positivos $\forall x\in \bb{R}$, es derivable.
    \begin{multline*}
        P'(x) = \frac{x}{\sqrt{x^2+4}} +\frac{x-10}{\sqrt{109+x^2-20x}} = 0
        \Longleftrightarrow \\ \Longleftrightarrow 
        x\sqrt{109+x^2-20x} = (10-x)\sqrt{x^2+4}
        \stackrel{(\ast)}{\Longleftrightarrow}
        x^2(109+x^2-20x) = (10-x)^2(x^2+4)
        \Longleftrightarrow  \\ \Longleftrightarrow 
        109x^2+\cancel{x^4}-\bcancel{20x^3} = \cancel{x^4}+4x^2+100x^2+400-\bcancel{20x^3}-80x \Longleftrightarrow 5x^2+80x-400 = 0 
        \Longleftrightarrow \\ \Longleftrightarrow
        x^2+16x-80=0 \Longleftrightarrow \left\{\begin{array}{l}
             x=4 \\
             \cancel{x=-20} \text{, ya que no cumple $(\ast)$}
        \end{array} \right.
    \end{multline*}

    Comprobamos ahora si, efectivamente, es un mínimo relativo.
    \begin{itemize}
        \item \underline{Para $x<4$}: $P'(x)<0 \Longrightarrow P(x)$ estrictamente decreciente.
        \item \underline{Para $x>4$}: $P'(x)>0 \Longrightarrow P(x)$ estrictamente creciente.
    \end{itemize}

    Por tanto, se confirma que es un mínimo relativo. También es absoluto, ya que es el único extremo de una función de clase $1$ definida en un intervalo.

    Por tanto, el punto que minimiza el perímetro del triángulo es:
    \begin{equation*}
        P(x,0) = P(4,0)
    \end{equation*}
    
\end{ejercicio}



\end{document}