\documentclass[12pt]{article}

% Idioma y codificación
\usepackage[spanish, es-tabla]{babel}       %es-tabla para que se titule "Tabla"
\usepackage[utf8]{inputenc}

% Márgenes
\usepackage[a4paper,top=3cm,bottom=2.5cm,left=3cm,right=3cm]{geometry}

% Comentarios de bloque
\usepackage{verbatim}

% Paquetes de links
\usepackage[hidelinks]{hyperref}    % Permite enlaces
\usepackage{url}                    % redirecciona a la web

% Más opciones para enumeraciones
\usepackage{enumitem}

% Personalizar la portada
\usepackage{titling}

% Paquetes de tablas
\usepackage{multirow}


%------------------------------------------------------------------------

%Paquetes de figuras
\usepackage{caption}
\usepackage{subcaption} % Figuras al lado de otras
\usepackage{float}      % Poner figuras en el sitio indicado H.


% Paquetes de imágenes
\usepackage{graphicx}       % Paquete para añadir imágenes
\usepackage{transparent}    % Para manejar la opacidad de las figuras

% Paquete para usar colores
\usepackage[dvipsnames]{xcolor}
\usepackage{pagecolor}      % Para cambiar el color de la página

% Habilita tamaños de fuente mayores
\usepackage{fix-cm}

% Para los gráficos
\usepackage{tikz}

% Para poder situar los nodos en los grafos
\usetikzlibrary{positioning}


%------------------------------------------------------------------------

% Paquetes de matemáticas
\usepackage{mathtools, amsfonts, amssymb, mathrsfs}
\usepackage[makeroom]{cancel}     % Simplificar tachando
\usepackage{polynom}    % Divisiones y Ruffini
\usepackage{units} % Para poner fracciones diagonales con \nicefrac

\usepackage{pgfplots}   %Representar funciones
\pgfplotsset{compat=1.18}  % Versión 1.18

\usepackage{tikz-cd}    % Para usar diagramas de composiciones
\usetikzlibrary{calc}   % Para usar cálculo de coordenadas en tikz

%Definición de teoremas, etc.
\usepackage{amsthm}
%\swapnumbers   % Intercambia la posición del texto y de la numeración

\theoremstyle{plain}

\makeatletter
\@ifclassloaded{article}{
  \newtheorem{teo}{Teorema}[section]
}{
  \newtheorem{teo}{Teorema}[chapter]  % Se resetea en cada chapter
}
\makeatother

\newtheorem{coro}{Corolario}[teo]           % Se resetea en cada teorema
\newtheorem{prop}[teo]{Proposición}         % Usa el mismo contador que teorema
\newtheorem{lema}[teo]{Lema}                % Usa el mismo contador que teorema

\theoremstyle{remark}
\newtheorem*{observacion}{Observación}

\theoremstyle{definition}

\makeatletter
\@ifclassloaded{article}{
  \newtheorem{definicion}{Definición} [section]     % Se resetea en cada chapter
}{
  \newtheorem{definicion}{Definición} [chapter]     % Se resetea en cada chapter
}
\makeatother

\newtheorem*{notacion}{Notación}
\newtheorem*{ejemplo}{Ejemplo}
\newtheorem*{ejercicio*}{Ejercicio}             % No numerado
\newtheorem{ejercicio}{Ejercicio} [section]     % Se resetea en cada section


% Modificar el formato de la numeración del teorema "ejercicio"
\renewcommand{\theejercicio}{%
  \ifnum\value{section}=0 % Si no se ha iniciado ninguna sección
    \arabic{ejercicio}% Solo mostrar el número de ejercicio
  \else
    \thesection.\arabic{ejercicio}% Mostrar número de sección y número de ejercicio
  \fi
}


% \renewcommand\qedsymbol{$\blacksquare$}         % Cambiar símbolo QED
%------------------------------------------------------------------------

% Paquetes para encabezados
\usepackage{fancyhdr}
\pagestyle{fancy}
\fancyhf{}

\newcommand{\helv}{ % Modificación tamaño de letra
\fontfamily{}\fontsize{12}{12}\selectfont}
\setlength{\headheight}{15pt} % Amplía el tamaño del índice


%\usepackage{lastpage}   % Referenciar última pag   \pageref{LastPage}
\fancyfoot[C]{\thepage}

%------------------------------------------------------------------------

% Conseguir que no ponga "Capítulo 1". Sino solo "1."
\makeatletter
\@ifclassloaded{book}{
  \renewcommand{\chaptermark}[1]{\markboth{\thechapter.\ #1}{}} % En el encabezado
    
  \renewcommand{\@makechapterhead}[1]{%
  \vspace*{50\p@}%
  {\parindent \z@ \raggedright \normalfont
    \ifnum \c@secnumdepth >\m@ne
      \huge\bfseries \thechapter.\hspace{1em}\ignorespaces
    \fi
    \interlinepenalty\@M
    \Huge \bfseries #1\par\nobreak
    \vskip 40\p@
  }}
}
\makeatother

%------------------------------------------------------------------------
% Paquetes de cógido
\usepackage{minted}
\renewcommand\listingscaption{Código fuente}

\usepackage{fancyvrb}
% Personaliza el tamaño de los números de línea
\renewcommand{\theFancyVerbLine}{\small\arabic{FancyVerbLine}}

% Estilo para C++
\newminted{cpp}{
    frame=lines,
    framesep=2mm,
    baselinestretch=1.2,
    linenos,
    escapeinside=||
}

% para minted
\definecolor{LightGray}{rgb}{0.95,0.95,0.92}
\setminted{
    linenos=true,
    stepnumber=5,
    numberfirstline=true,
    autogobble,
    breaklines=true,
    breakautoindent=true,
    breaksymbolleft=,
    breaksymbolright=,
    breaksymbolindentleft=0pt,
    breaksymbolindentright=0pt,
    breaksymbolsepleft=0pt,
    breaksymbolsepright=0pt,
    fontsize=\footnotesize,
    bgcolor=LightGray,
    numbersep=10pt
}


\usepackage{listings} % Para incluir código desde un archivo

\renewcommand\lstlistingname{Código Fuente}
\renewcommand\lstlistlistingname{Índice de Códigos Fuente}

% Definir colores
\definecolor{vscodepurple}{rgb}{0.5,0,0.5}
\definecolor{vscodeblue}{rgb}{0,0,0.8}
\definecolor{vscodegreen}{rgb}{0,0.5,0}
\definecolor{vscodegray}{rgb}{0.5,0.5,0.5}
\definecolor{vscodebackground}{rgb}{0.97,0.97,0.97}
\definecolor{vscodelightgray}{rgb}{0.9,0.9,0.9}

% Configuración para el estilo de C similar a VSCode
\lstdefinestyle{vscode_C}{
  backgroundcolor=\color{vscodebackground},
  commentstyle=\color{vscodegreen},
  keywordstyle=\color{vscodeblue},
  numberstyle=\tiny\color{vscodegray},
  stringstyle=\color{vscodepurple},
  basicstyle=\scriptsize\ttfamily,
  breakatwhitespace=false,
  breaklines=true,
  captionpos=b,
  keepspaces=true,
  numbers=left,
  numbersep=5pt,
  showspaces=false,
  showstringspaces=false,
  showtabs=false,
  tabsize=2,
  frame=tb,
  framerule=0pt,
  aboveskip=10pt,
  belowskip=10pt,
  xleftmargin=10pt,
  xrightmargin=10pt,
  framexleftmargin=10pt,
  framexrightmargin=10pt,
  framesep=0pt,
  rulecolor=\color{vscodelightgray},
  backgroundcolor=\color{vscodebackground},
}

%------------------------------------------------------------------------

% Comandos definidos
\newcommand{\bb}[1]{\mathbb{#1}}
\newcommand{\cc}[1]{\mathcal{#1}}

% I prefer the slanted \leq
\let\oldleq\leq % save them in case they're every wanted
\let\oldgeq\geq
\renewcommand{\leq}{\leqslant}
\renewcommand{\geq}{\geqslant}

% Si y solo si
\newcommand{\sii}{\iff}

% Letras griegas
\newcommand{\eps}{\epsilon}
\newcommand{\veps}{\varepsilon}
\newcommand{\lm}{\lambda}

\newcommand{\ol}{\overline}
\newcommand{\ul}{\underline}
\newcommand{\wt}{\widetilde}
\newcommand{\wh}{\widehat}

\let\oldvec\vec
\renewcommand{\vec}{\overrightarrow}

% Derivadas parciales
\newcommand{\del}[2]{\frac{\partial #1}{\partial #2}}
\newcommand{\Del}[3]{\frac{\partial^{#1} #2}{\partial #3^{#1}}}
\newcommand{\deld}[2]{\dfrac{\partial #1}{\partial #2}}
\newcommand{\Deld}[3]{\dfrac{\partial^{#1} #2}{\partial #3^{#1}}}


\newcommand{\AstIg}{\stackrel{(\ast)}{=}}
\newcommand{\Hop}{\stackrel{L'H\hat{o}pital}{=}}

\newcommand{\red}[1]{{\color{red}#1}} % Para integrales, destacar los cambios.

% Método de integración
\newcommand{\MetInt}[2]{
    \left[\begin{array}{c}
        #1 \\ #2
    \end{array}\right]
}

% Declarar aplicaciones
% 1. Nombre aplicación
% 2. Dominio
% 3. Codominio
% 4. Variable
% 5. Imagen de la variable
\newcommand{\Func}[5]{
    \begin{equation*}
        \begin{array}{rrll}
            #1:& #2 & \longrightarrow & #3\\
               & #4 & \longmapsto & #5
        \end{array}
    \end{equation*}
}

%------------------------------------------------------------------------


\begin{document}
	
	% 1. Foto de fondo
	% 2. Título
	% 3. Encabezado Izquierdo
	% 4. Color de fondo
	% 5. Coord x del titulo
	% 6. Coord y del titulo
	% 7. Fecha
	
	
	% 1. Foto de fondo
% 2. Título
% 3. Encabezado Izquierdo
% 4. Color de fondo
% 5. Coord x del titulo
% 6. Coord y del titulo
% 7. Fecha

\newcommand{\portada}[7]{

    \portadaBase{#1}{#2}{#3}{#4}{#5}{#6}{#7}
    \portadaBook{#1}{#2}{#3}{#4}{#5}{#6}{#7}
}

\newcommand{\portadaExamen}[7]{

    \portadaBase{#1}{#2}{#3}{#4}{#5}{#6}{#7}
    \portadaArticle{#1}{#2}{#3}{#4}{#5}{#6}{#7}
}




\newcommand{\portadaBase}[7]{

    % Tiene la portada principal y la licencia Creative Commons
    
    % 1. Foto de fondo
    % 2. Título
    % 3. Encabezado Izquierdo
    % 4. Color de fondo
    % 5. Coord x del titulo
    % 6. Coord y del titulo
    % 7. Fecha
    
    
    \thispagestyle{empty}               % Sin encabezado ni pie de página
    \newgeometry{margin=0cm}        % Márgenes nulos para la primera página
    
    
    % Encabezado
    \fancyhead[L]{\helv #3}
    \fancyhead[R]{\helv \nouppercase{\leftmark}}
    
    
    \pagecolor{#4}        % Color de fondo para la portada
    
    \begin{figure}[p]
        \centering
        \transparent{0.3}           % Opacidad del 30% para la imagen
        
        \includegraphics[width=\paperwidth, keepaspectratio]{assets/#1}
    
        \begin{tikzpicture}[remember picture, overlay]
            \node[anchor=north west, text=white, opacity=1, font=\fontsize{60}{90}\selectfont\bfseries\sffamily, align=left] at (#5, #6) {#2};
            
            \node[anchor=south east, text=white, opacity=1, font=\fontsize{12}{18}\selectfont\sffamily, align=right] at (9.7, 3) {\textbf{\href{https://losdeldgiim.github.io/}{Los Del DGIIM}}};
            
            \node[anchor=south east, text=white, opacity=1, font=\fontsize{12}{15}\selectfont\sffamily, align=right] at (9.7, 1.8) {Doble Grado en Ingeniería Informática y Matemáticas\\Universidad de Granada};
        \end{tikzpicture}
    \end{figure}
    
    
    \restoregeometry        % Restaurar márgenes normales para las páginas subsiguientes
    \pagecolor{white}       % Restaurar el color de página
    
    
    \newpage
    \thispagestyle{empty}               % Sin encabezado ni pie de página
    \begin{tikzpicture}[remember picture, overlay]
        \node[anchor=south west, inner sep=3cm] at (current page.south west) {
            \begin{minipage}{0.5\paperwidth}
                \href{https://creativecommons.org/licenses/by-nc-nd/4.0/}{
                    \includegraphics[height=2cm]{assets/Licencia.png}
                }\vspace{1cm}\\
                Esta obra está bajo una
                \href{https://creativecommons.org/licenses/by-nc-nd/4.0/}{
                    Licencia Creative Commons Atribución-NoComercial-SinDerivadas 4.0 Internacional (CC BY-NC-ND 4.0).
                }\\
    
                Eres libre de compartir y redistribuir el contenido de esta obra en cualquier medio o formato, siempre y cuando des el crédito adecuado a los autores originales y no persigas fines comerciales. 
            \end{minipage}
        };
    \end{tikzpicture}
    
    
    
    % 1. Foto de fondo
    % 2. Título
    % 3. Encabezado Izquierdo
    % 4. Color de fondo
    % 5. Coord x del titulo
    % 6. Coord y del titulo
    % 7. Fecha


}


\newcommand{\portadaBook}[7]{

    % 1. Foto de fondo
    % 2. Título
    % 3. Encabezado Izquierdo
    % 4. Color de fondo
    % 5. Coord x del titulo
    % 6. Coord y del titulo
    % 7. Fecha

    % Personaliza el formato del título
    \pretitle{\begin{center}\bfseries\fontsize{42}{56}\selectfont}
    \posttitle{\par\end{center}\vspace{2em}}
    
    % Personaliza el formato del autor
    \preauthor{\begin{center}\Large}
    \postauthor{\par\end{center}\vfill}
    
    % Personaliza el formato de la fecha
    \predate{\begin{center}\huge}
    \postdate{\par\end{center}\vspace{2em}}
    
    \title{#2}
    \author{\href{https://losdeldgiim.github.io/}{Los Del DGIIM}}
    \date{Granada, #7}
    \maketitle
    
    \tableofcontents
}




\newcommand{\portadaArticle}[7]{

    % 1. Foto de fondo
    % 2. Título
    % 3. Encabezado Izquierdo
    % 4. Color de fondo
    % 5. Coord x del titulo
    % 6. Coord y del titulo
    % 7. Fecha

    % Personaliza el formato del título
    \pretitle{\begin{center}\bfseries\fontsize{42}{56}\selectfont}
    \posttitle{\par\end{center}\vspace{2em}}
    
    % Personaliza el formato del autor
    \preauthor{\begin{center}\Large}
    \postauthor{\par\end{center}\vspace{3em}}
    
    % Personaliza el formato de la fecha
    \predate{\begin{center}\huge}
    \postdate{\par\end{center}\vspace{5em}}
    
    \title{#2}
    \author{\href{https://losdeldgiim.github.io/}{Los Del DGIIM}}
    \date{Granada, #7}
    \thispagestyle{empty}               % Sin encabezado ni pie de página
    \maketitle
    \vfill
}
	\portadaExamen{ffccA4.jpg}{Cálculo II\\Examen XVII}{Cálculo II. Examen XVII}{MidnightBlue}{-8}{28}{2025}{Roxana Acedo Parra}
	
	\begin{description}
		\item[Asignatura] Cálculo II.
		\item[Curso Académico] 2024-25.
		\item[Grado] Doble Grado en Ingeniería Informática y Matemáticas.
		\item[Grupo] Único.
		\item[Profesor] José Luis Gámez Ruiz\footnote{El examen lo puso el departamento.}.
		\item[Fecha] 1 de julio de 2025.
		\item[Duración] 3 horas.
		\item[Descripción] Convocatoria Extraordinaria.
	\end{description}
	\newpage
	
	\begin{ejercicio}[2 puntos]
		Tema a desarrollar: Seno y coseno como series de potencias.
	\end{ejercicio}
	
	\begin{ejercicio}[2 puntos]
		Sea $g:\bb{R}\longrightarrow\bb{R}$ una función derivable en $\bb{R}$ con $g(0)=0$, tal que $\exists g''(0)$. Estudia la derivabilidad de la función $f:\bb{R}\longrightarrow\bb{R}$ definida como \\
		$$ f(x) = 
		\begin{cases}
			\displaystyle \frac{\int_{0}^{x} \frac{g(t)}{t}\,dt}{x}, & \text{si } x \neq 0 \\
			\\
			g'(0), & \text{si } x = 0
		\end{cases}$$
		Nota: En el enunciado se afirma que $g$ es dos veces derivable en 0 pero no necesariamente tiene que ser dos veces derivable en los $x \neq 0$.
	\end{ejercicio}
	
	\begin{ejercicio}[2 puntos]
		Sea $q \in \bb{N}.$
		\begin{enumerate}[label=\alph*)]
			\item Demuestra que, para $x \in [0, 1],$ se cumple que 
			$$ \int_{0}^{x} \frac{1}{1 + t^q} \, dt = \sum_{n=0}^{+\infty} \frac{(-1)^n x^{qn+1}}{qn+1}.$$
			 \item Calcula $\displaystyle \sum_{n=0}^{+\infty} \frac{(-1)^n}{qn+1},$ para $q=1,$ $q=2,$ y $q=3.$
		\end{enumerate}
	\end{ejercicio}
	
	\begin{ejercicio}[2 puntos]
		Sea $F:\bb{R}\longrightarrow\bb{R}$ la función definida por $\displaystyle \int_{0}^{x} e^{-t^2} \, dt$, para $x \in \bb{R}$. Calcula una aproximación de $F(\nicefrac{1}{2})$ con $|$error$|$ menor que $10^{-1}$.
	\end{ejercicio}
	
	\begin{ejercicio}[2 puntos]
		Demuestra que $\forall x \in [0, \frac{\pi}{2}]$ se verifica que
		$$ \int_{\frac{1}{e}}^{\text{tan}(x)} \frac{t}{1+t^2}\, dt + \int_{\frac{1}{e}}^{\frac{1}{\text{tan}(x)}} \frac{1}{t(1+t^2)}\, dt = 1$$
	\end{ejercicio}
	
	
	
	\newpage
	
	
	\setcounter{ejercicio}{0}
	% Revisar (1)
	\begin{ejercicio}[2 puntos]
		Tema a desarrollar: Seno y coseno como series de potencias. \\
		
		La función $f:\bb{R}\longrightarrow\bb{R}$ representará al sen$(x)$ y $g:\bb{R}\longrightarrow\bb{R}$ al cos$(x)$ en este desarrollo.
		Consecuentemente deberán cumplir las siguientes condiciones para poder construir las series de pontencias centradas en 0 acertadas:
		\begin{enumerate}
			\item $f(0)=0$.
			\item $f'(x) = g(x)$.
			\item $g(0)=1$.
			\item $g'(x)=-f(x)$.
		\end{enumerate}
		
		La serie de potencias que converge a $f(x)$ será
		$$ f(x) = \sum_{n=0}^{+\infty} a_n\cdot (x-0)^n = \sum_{n=0}^{+\infty} a_n\cdot x^n$$
		y la que converge a $g(x)$
		$$ g(x)= \sum_{n=0}^{+\infty} b_n\cdot (x-0)^n = \sum_{n=0}^{+\infty} b_n\cdot x^n$$
		
		Gracias a la primera y tercera condición ya sabemos que $a_0=0$ y $b_0=1$.
		
		Derivando ambas series término a término tenemos que
		
		$$f'(x) = \sum_{n=1}^{+\infty} a_n\cdot n \cdot x^{n-1} \underset{m=n-1}{=} \sum_{m=0}^{+\infty} a_{m+1}\cdot (m+1) \cdot x^{m}$$
		$$g'(x) = \sum_{n=1}^{+\infty} b_n\cdot n \cdot x^{n-1} \underset{m=n-1}{=} \sum_{m=0}^{+\infty} b_{m+1}\cdot (m+1) \cdot x^{m}$$
		
		Ahora igualamos término a término siguiendo la segunda condición
		
		$$ a_{m+1}\cdot (m+1) = b_m $$
		
		y la cuarta
		
		$$ b_{m+1}\cdot (m+1) = -a_m$$
		
		Realizando los cálculos convenientes, nos queda que:
		\begin{itemize}
			\item $a_{m+2}\cdot (m+2) \cdot (m+1) = -a_m$
			\item $b_{m+2}\cdot (m+2) \cdot (m+1) = -b_m$
		\end{itemize}
		
		Desarrollando unos cuantos términos de la serie de $f(x)$
		
		$$ a_0=0, \, a_1 = \frac{b_0}{0+1} = 1, \, a_2 = 0, $$ 
		$$ a_3 = \frac{-a_1}{(1+1) \cdot (1+2)} = \frac{-1}{3!}, \, a_4=0, \, a_5 = \frac{-a_3}{(3+1)\cdot(3+2)} = \frac{1}{5!} \, \ldots $$
		
		es más sencillo ver cómo únicamente aparecen los términos impares y estos van cambiando de signo, por ello podemos reescribir la serie de la siguiente manera
		
		$$ f(x) = \sum_{n=0}^{+\infty} \frac{(-1)^n \cdot x^{2n+1}}{(2n+1)!} $$
		
		Para reescribir la de $g(x)$ es más sencillo derivar la de $f(x)$, sin embargo desarrollaremos también los 6 primeros términos.
		
		$$b_0=1, \, b_1=\frac{-a_0}{0+1}=0,\, b_2=\frac{-b_0}{(0+1)\cdot (0+2)}= \frac{-1}{2!}, \, $$
		$$b_3 = 0, \, b_4=\frac{-b_2}{(2+1)\cdot (2+2)}= \frac{1}{4!}, \, b_5=0 \, \ldots$$
		
		Confirmamos así que 
		
		$$g(x)= \sum_{n=0}^{+\infty} \frac{(-1)^n \cdot x^{2n}}{(2n)!}$$
		
		Calculemos ahora sus radios de convergencia.
		
		\begin{itemize}
			\item De $f(x)$:
			$$ \left\{\frac{|a_{n+1}|}{|a_n|}\right\} = \left\{\frac{\frac{1}{(2n+3)!}}{\frac{1}{(2n+1)!}}\right\}=\left\{\frac{1}{(2n+3)\cdot(2n+2)}\right\} \longrightarrow 0 $$
			Al ser $L=0 \Rightarrow R=\infty \, \wedge \, I_c=\bb{R}$ y es una serie analítica.
			
			\item De $g(x)$:
			$$ \left\{\frac{|b_{n+1}|}{|b_n|}\right\} = \left\{\frac{\frac{1}{(2n+2)!}}{\frac{1}{(2n)!}}\right\}=\left\{\frac{1}{(2n+2)\cdot(2n+1)}\right\} \longrightarrow 0 $$
			Al ser $L=0 \Rightarrow R=\infty \, \wedge \, I_c=\bb{R}$ y es una serie analítica.
		\end{itemize}
		
		Por último podemos demostrar que $\text{cos}(x)^2+\text{sen}(x)^2=1$ haciendo uso de $f(x)$ y $g(x)$. Llamamos $h(x)=f(x)^2+g(x)^2$, derivándola
		$$h'(x)=2 \cdot f(x) \cdot f'(x) + 2 \cdot g(x) \cdot g'(x) \underset{ \text{ usando 2 y 4 }}{=} 2 \cdot f(x) \cdot g(x) - 2 \cdot g(x) \cdot f(x)=0$$
		Por lo tanto, al ser su derivada nula, $h(x)$ es constante. Evaluándola en $x=0$
		$$h(0)= f(0)^2+g(0)^2 = 0^2+1^2=1 \Rightarrow f(x)^2+g(x)^2 = 1$$
		ya lo habríamos demostrado.
		
	\end{ejercicio}
	
	% Resuelto (2) 
	\begin{ejercicio}[2 puntos]
		Sea $g:\bb{R}\longrightarrow\bb{R}$ una función derivable en $\bb{R}$ con $g(0)=0$, tal que $\exists g''(0)$. Estudia la derivabilidad de la función $f:\bb{R}\longrightarrow\bb{R}$ definida como \\
		$$ f(x) = 
		\begin{cases}
			\displaystyle \frac{\int_{0}^{x} \frac{g(t)}{t}\,dt}{x}, & \text{si } x \neq 0 \\
			\\
			g'(0), & \text{si } x = 0
		\end{cases}$$
		Nota: En el enunciado se afirma que $g$ es dos veces derivable en 0 pero no necesariamente tiene que ser dos veces derivable en los $x \neq 0$. \\
		
		Antes de ver si es derivable, conviene ver si es continua, puesto que si no es continua no será derivable. Para ello debemos calcular los límites laterales, ver que coinciden entre ellos y que son iguales que el propio valor de la función en $x=0$. \\
		
		Además, debido al T.F.C. sabemos que $\exists H:\bb{R}\longrightarrow\bb{R} \, | \, H'(t)= \displaystyle \frac{g(t)}{t}$.
		
		$$ \lim_{x\rightarrow0^+} \frac{\int_{0}^{x} \frac{g(t)}{t}\,dt}{x} \underset{\text{ Aplicamos L$'$Hôpital }}{=} \lim_{x\rightarrow0^+} \frac{H'(x)}{1} = \lim_{x\rightarrow0^+} \frac{g(x)}{x}$$
		
		$$\underset{\text{ Como $g(0)=0$ }}{=} \lim_{x\rightarrow0^+} \frac{g(x)-g(0)}{x-0} \underset{\text{ $g$ derivable en 0 }}{=} g'(0) = f(0)$$
		
		El límite lateral izquierdo se calcula exactamente igual.
		$$ \lim_{x\rightarrow0^-} \frac{\int_{0}^{x} \frac{g(t)}{t}\,dt}{x} \underset{\text{ Aplicamos L$'$Hôpital }}{=} \lim_{x\rightarrow0^+} \frac{H'(x)}{1} = \lim_{x\rightarrow0^+} \frac{g(x)}{x}$$

		$$\underset{\text{ Como $g(0)=0$ }}{=} \lim_{x\rightarrow0^+} \frac{g(x)-g(0)}{x-0} \underset{\text{ $g$ derivable en 0 }}{=} g'(0) = f(0)$$
		
		Por ende son iguales y la función $f(x)$ es continua. Veamos si es derivable. \\
		
		Para que sea derivable en $0$, necesitamos que sus derivadas laterales en $0$ sean iguales. Derivemos la función:
		$$ f'(x) = 
		\begin{cases}
			\displaystyle \frac{H'(x)\cdot x - H(x)\cdot 1}{x^2}, & \text{si } x \neq 0 \\
			\\
			0, \text{{\tiny $g'(0)$ es una constante}} & \text{si } x = 0
		\end{cases}$$
		
		Siendo las derivadas laterales
		
		$$ \lim_{x\rightarrow0^+} \frac{H'(x)\cdot x - H(x)\cdot 1}{x^2} = \lim_{x\rightarrow0^+} \frac{H'(x)}{x} - \frac{H(x)\cdot 1}{x^2} = 
		\lim_{x\rightarrow0^+} \frac{g(x)}{x^2} - \frac{H(x)\cdot 1}{x^2}$$
		
		Les aplicamos L'Hôpital por separado y si los límites existen los restamos entre ellos.
		
		$$\lim_{x\rightarrow0^+} \frac{g(x)}{x^2} \underset{\text{ Aplicamos L$'$Hôpital dos veces}}{=} \lim_{x\rightarrow0^+} \frac{g''(x)}{2}$$
		
		$$\text{Como } \exists g''(0) \Rightarrow \exists \lim_{x\rightarrow0^+} \frac{g(x)}{x^2}= \frac{g''(0)}{2}$$
		
		Para el otro
		$$\lim_{x\rightarrow0^+} \frac{H(x)}{x^2} \underset{\text{ Aplicamos L$'$Hôpital}}{=} \lim_{x\rightarrow0^+} \frac{H'(x)}{2x} = \lim_{x\rightarrow0^+} \frac{g(x)}{2x^2} = \frac{1}{2}\cdot \lim_{x\rightarrow0^+} \frac{g(x)}{x^2} = \frac{g''(0)}{4}$$
		
		Consecuentemente, 
		$$\lim_{x\rightarrow0^+} \frac{g(x)}{x^2} - \frac{H(x)\cdot 1}{x^2} = \frac{g''(0)}{2}-\frac{g''(0)}{4}=\frac{g''(0)}{4}$$
		
		El procedimiento para calcular la derivada lateral izquierda es exactamente el mismo (no ha llegado a afectar que la estuviésemos haciendo por la derecha en ningún momento) y tiene el mismo valor. Como existen todas la derivadas laterales que tienen sentido y son iguales entre sí, entonces $f(x)$ es derivable en 0 y por ende en todo su dominio.
		
	\end{ejercicio}
	
	% Resuelto (3)
	\begin{ejercicio}[2 puntos]
		Sea $q \in \bb{N}.$
		\begin{enumerate}[label=\alph*)]
			\item Demuestra que, para $x \in [0, 1],$ se cumple que 
			$$ \int_{0}^{x} \frac{1}{1 + t^q} \, dt = \sum_{n=0}^{+\infty} \frac{(-1)^n x^{qn+1}}{qn+1}.$$
			
			Antes de nada, recordemos la serie geométrica: \\
			$$ \sum_{n=0}^{+\infty}r ^n = \frac{1}{1-r} \Leftrightarrow |r|<1.$$
			
			Podemos crear una función 
			$$ h(x)= \int_{0}^{x} \frac{1}{1 + t^q} \, dt - \sum_{n=0}^{+\infty} \frac{(-1)^n x^{qn+1}}{qn+1}$$
			y demostrar que es constante ($h'(x)=0 \quad \forall x \in [0, 1]$) con valor 0. Derivemos por un lado la integral y por el otro la serie.
			
			\begin{itemize}
				\item Por el T.F.C. podemos derivar la integral, siendo su derivada:
				$$ \frac{1}{1 + x^q}$$
				\item Ahora derivamos término a término de la serie (el radio de convergencia no cambia):
				$$\sum_{n=0}^{+\infty} (-x^q)^n = \frac{1}{1+x^q}$$
				(y el resultado también vale en $t=1$ por convergencia uniforme si $q \in \bb{N}$).
			\end{itemize}
			
			Es sencillo ver que $h'(x)=0$, evaluémosla en $x=0$,
			\begin{equation*}
				h(0)= \int_{0}^{0} \frac{1}{1 + t^q} \, dt - \sum_{n=0}^{+\infty} \frac{(-1)^n 0^{qn+1}}{qn+1} = 0
				\Longrightarrow \int_{0}^{x} \frac{1}{1 + t^q} \, dt = \sum_{n=0}^{+\infty} \frac{(-1)^n x^{qn+1}}{qn+1}.
			\end{equation*}
			Y queda demostrado el enunciado.
			
			\item  Calcula $\displaystyle \sum_{n=0}^{+\infty} \frac{(-1)^n}{qn+1},$ para $q=1,$ $q=2,$ y $q=3.$\\
			Usamos el resultado de la parte anterior con $x=1$:
			$$ \sum_{n=0}^{\infty} \frac{(-1)^n}{qn + 1} = \int_0^1 \frac{1}{1 + t^q} \, dt$$
				
			\begin{itemize}
				\item Para $q = 1$:
				$$ \int_0^1 \frac{1}{1 + t} \, dt = \ln(2)
				\Rightarrow \sum_{n=0}^{\infty} \frac{(-1)^n}{n + 1} = \ln(2)$$
					
				\item Para $q = 2$:
				$$ \int_0^1 \frac{1}{1 + t^2} \, dt = \arctan(1) = \frac{\pi}{4}
					\Rightarrow \sum_{n=0}^{\infty} \frac{(-1)^n}{2n + 1} = \frac{\pi}{4}$$
					
				\item Para $q = 3$:
				$$\sum_{n=0}^{\infty} \frac{(-1)^n}{3n + 1} = \int_0^1 \frac{1}{1 + t^3} \, dt$$
				En este caso la integral no es inmediata. Para resolverla seguiremos la integración de cocientes.
				
				$$\int_0^1 \frac{1}{1 + t^3} \, dt = \int_0^1 \frac{1}{(t+1)(t^2-t+1)} \, dt = \int_0^1 \frac{A}{(t+1)}+\frac{Bt+C}{(t^2-t+1)} \, dt$$
				
				Resolvemos el siguiente sistema de ecuaciones:
				$$ \begin{cases}
					A+B=0 & (t^2)\\
					-A+B+C = 0 & (t)\\
					A+C=1 
				\end{cases}
				\Rightarrow A=\frac{1}{3}, \quad B=\frac{-1}{3}, \quad
				C=\frac{2}{3}$$
				
				Ya sabemos que $\displaystyle \int_0^1 \frac{\frac{1}{3}}{(t+1)} \, dt = \left[\frac{1}{3}\cdot \ln(t+1)\right]^1_0$, sigamos con el otro trozo.
				\begin{align*}
					\int_0^1 \frac{\frac{-1}{3}t+\frac{2}{3}}{(t^2-t+1)} \, dt &= \frac{1}{3}\int_0^1 \frac{-t+2-\frac{3}{2}+\frac{3}{2}}{(t^2-t+1)} \, dt =\\&= \frac{1}{3}\int_0^1 \frac{-t+\frac{1}{2}}{(t^2-t+1)} \, dt +  \frac{1}{3}\int_0^1 \frac{\frac{3}{2}}{(t^2-t+1)} \, dt
				\end{align*}
				
				Enfocándonos en 
				$\displaystyle \frac{1}{3}\int_0^1 \frac{-t+\frac{1}{2}}{(t^2-t+1)} \, dt$ tenemos que su integral es
				$$\frac{-1}{3\cdot 2}\int_0^1 \frac{2t-1}{(t^2-t+1)} \, dt = \left[\frac{-1}{6}\cdot\ln(t^2-t+1)\right]^1_0$$
				
				La integral del otro sumando sería:
				\begin{align*}
					\frac{1}{3}&\int_0^1 \frac{\frac{3}{2}}{(t^2-t+1)} \, dt = \frac{1}{\cancel{3}}\cdot\frac{\cancel{3}}{2}\int_0^1 \frac{1}{(t^2-t+1)} \, dt = \frac{1}{2}\int_0^1 \frac{1}{\frac{3}{4}+(t-\frac{1}{2})^2} \, dt
					=\\&= \frac{1}{2}\int_0^1 \frac{1}{\frac{3}{4}} \cdot \frac{1}{1+\frac{(t-\frac{1}{2})^2}{\frac{3}{4}}} \, dt = \frac{2}{3}\int_0^1 \frac{1}{1+\left(\frac{t-\frac{1}{2}}{\frac{\sqrt{3}}{2}}\right)^2} \, dt = \frac{2}{3}\int_0^1 \frac{1}{1+(\frac{2t-1}{\sqrt{3}})^2} \, dt
					=\\&=
					\left[\frac{\sqrt{3}}{3}\arctan\left(\frac{2t-1}{\sqrt{3}}\right)\right]^1_0
				\end{align*}
				
				En consecuencia la integral al completo es:
				\begin{align*}
					&\left[\frac{1}{3}\cdot \ln(t+1)-\frac{1}{6}\cdot\ln(t^2-t+1)+\frac{\sqrt{3}}{3}\arctan\left(\frac{2t-1}{\sqrt{3}}\right)\right]^1_0
					=\\&= \frac{\ln 2}{3} + \frac{\sqrt{3}}{3}\arctan\left(\frac{1}{\sqrt{3}}\right) - \frac{\sqrt{3}}{3}\arctan\left(-\frac{1}{\sqrt{3}}\right)
					=\\&= \frac{\ln 2}{3} + \frac{2\sqrt{3}}{3}\arctan\left(\frac{1}{\sqrt{3}}\right)\approx 0.8356
				\end{align*}

				Por tanto, tenemos que:
				\begin{equation*}
					\sum_{n=0}^{\infty} \frac{(-1)^n}{3n + 1}= \frac{\ln 2}{3} + \frac{2\sqrt{3}}{3}\arctan\left(\frac{1}{\sqrt{3}}\right)\approx 0.8356
				\end{equation*}
				\end{itemize}
		\end{enumerate}
	\end{ejercicio}
	
	% Revisar (4)
	\begin{ejercicio}[2 puntos]
		Sea $F:\bb{R}\longrightarrow\bb{R}$ la función definida por $\displaystyle \int_{0}^{x} e^{-t^2} \, dt$, para $x \in \bb{R}$. Calcula una aproximación de $F(\nicefrac{1}{2})$ con $|$error$|$ menor que $10^{-1}$. \\
		
		Para resolver este ejercicio lo más sencillo es utilizar el polinomio de Taylor centrado en $a=0$ y $x=\nicefrac{1}{2}$ ($T_n[F, 0](\nicefrac{1}{2})$) y la fórmula del resto de Lagrange.
		$$\exists c \in [0, \nicefrac{1}{2}] : R_n[F,0](\nicefrac{1}{2})=\frac{F^{(n+1)}(c)}{(n+1)!}\cdot\frac{1}{2^{n+1}}$$
		
		Derivemos varias veces $F(x)$ (es derivable por el T.F.C.) e intentemos acotar sus derivadas, teniendo en cuenta que el error estaría en valor absoluto.
		\begin{align*}
			F'(x)&=e^{-x^2}\\
			F''(x)&=e^{-x^2}\cdot (-2x)\\
			F'''(x)&=e^{-x^2}\cdot (-2+4x^2)\\
			F''''(x)&= e^{-x^2}\cdot (12x-8x^3)\\
			F'''''(x)&= e^{-x^2}\cdot (12-48x^2+16x^4)
		\end{align*}
		
		Es sencillo ver que la primera y segunda derivadas están $|$acotadas$|$ por 1, sin embargo la tercera ya no (por 2). Es algo más tedioso ver que la tercera, cuarta y quinta están $|$acotadas$|$ por 12. \\
		
		Probemos con $n=2$, entonces sabemos que el valor absoluto de la tercera derivada en $[0, \nicefrac{1}{2}]$ es menor que 2. Haciendo uso de la fórmula del resto
		
		$$ \frac{2}{6\cdot8} < 10^{-1} \iff 1 < 2.4$$
		
		De esta manera ya sabemos que
		$$ T_2[F,0](x)=\sum_{k=0}^{2} \frac{F^{(k)}(0)}{k!}\cdot x^k = \frac{F(0)}{0!} + \frac{F'(0)}{1!}\cdot x^1 + \frac{F''(0)}{2!}\cdot x^2 = 0 + x + 0$$
		
		Evaluándolo en $x=\nicefrac{1}{2}$ tenemos
		$$ T_2[F,0](\nicefrac{1}{2}) = \frac{1}{2} = 0.5$$
	
	\end{ejercicio}
	
	% Resuelto (5)
	\begin{ejercicio}[2 puntos]
		Demuestra que $\forall x \in [0, \frac{\pi}{2}]$ se verifica que
		$$ \int_{\frac{1}{e}}^{\text{tan}(x)} \frac{t}{1+t^2}\, dt + \int_{\frac{1}{e}}^{\frac{1}{\text{tan}(x)}} \frac{1}{t(1+t^2)}\, dt = 1$$
		
		Para resolver este ejercicio, llamaremos $h(x)$ a la suma de ambas integrales, derivaremos esta nueva función para comprobar que su derivada es nula. Posteriormente, usaremos el siguiente corolario.
		\begin{coro}
			Sea $I \subseteq \bb{R}$ un intervalo y $f:I\longrightarrow \bb{R}$ derivable. Entonces:
			$$ f \text{ es constante } \Leftrightarrow f'(x)= 0 \quad \forall x \in I$$
		\end{coro}
		
		Recordemos que es derivable gracias al T.F.C.
		\begin{align*}
			h'(x) &= \frac{\text{tan}(x)}{1+\text{tan}(x)^2}\cdot\left(\frac{1}{1+x^2}\right) + \frac{1}{\frac{1}{\text{tan}(x)}(1+\frac{1}{\text{tan}(x)^2})} \cdot \frac{-1}{\text{tan}(x)^2} \cdot\left(\frac{1}{1+x^2}\right)
			=\\&= \left(\frac{1}{1+x^2}\right)\cdot \left(\frac{\text{tan}(x)}{1+\text{tan}(x)^2} - \frac{1}{\frac{1}{\cancel{\text{tan}(x)}}(1+\frac{1}{\text{tan}(x)^2})} \cdot \frac{1}{\text{tan}(x)^{\cancel{2}}}\right)
			=\\&= \left(\frac{1}{1+x^2}\right)\cdot \left(\frac{\text{tan}(x)}{1+\text{tan}(x)^2} - \frac{1}{(\text{tan}(x)+\frac{\cancel{\text{tan}(x)}}{\text{tan}(x)^{\cancel{2}}})} \right)
			=\\&= \left(\frac{1}{1+x^2}\right)\cdot \left(\frac{\text{tan}(x)}{1+\text{tan}(x)^2} - \frac{1}{(\frac{\text{tan}(x)^2}{\text{tan}(x)}+\frac{1}{\text{tan}(x)})} \right)
			=\\&= \left(\frac{1}{1+x^2}\right)\cdot \left(\frac{\text{tan}(x)}{1+\text{tan}(x)^2} - \frac{\text{tan}(x)}{1+\text{tan}(x)^2} \right) = 0
		\end{align*}
		
		Al haber comprobado que su derivada es nula, sólo nos queda evaluarla en un punto conveniente y fácil de calcular. Para ello lo más simple es elegir $x=\frac{\pi}{4}$ ya que $\text{tan}(\frac{\pi}{4})=1$ y así ambos extremos de las integrales serán iguales ($[a=\frac{1}{e}, b=1]$).
		\begin{align*}
			h\left(\frac{\pi}{4}\right) &= \int_{\frac{1}{e}}^{\text{tan}(\frac{\pi}{4})} \frac{t}{1+t^2}\, dt + \int_{\frac{1}{e}}^{\frac{1}{\text{tan}(\frac{\pi}{4})}} \frac{1}{t(1+t^2)}\, dt
			=\\&= \int_{\frac{1}{e}}^{1} \frac{t}{1+t^2}\, dt + \int_{\frac{1}{e}}^{1} \frac{1}{t(1+t^2)}\, d
			=\\&= \int_{\frac{1}{e}}^{1} \frac{t}{1+t^2} + \frac{1}{t(1+t^2)}\, dt = \int_{\frac{1}{e}}^{1} \frac{t}{1+t^2}\cdot \frac{t}{t} + \frac{1}{t(1+t^2)}\, dt
			=\\&= \int_{\frac{1}{e}}^{1} \frac{\cancel{1+t^2}}{t\cancel{(1+t^2)}}\, dt = \left[\ln(t)\right]^1_{e^{-1}}= 0 - (-1) = 1
		\end{align*}
		
		Por lo tanto queda el enunciado demostrado.
		
	\end{ejercicio}






	
	
\end{document}