\documentclass[12pt]{article}

% Idioma y codificación
\usepackage[spanish, es-tabla]{babel}       %es-tabla para que se titule "Tabla"
\usepackage[utf8]{inputenc}

% Márgenes
\usepackage[a4paper,top=3cm,bottom=2.5cm,left=3cm,right=3cm]{geometry}

% Comentarios de bloque
\usepackage{verbatim}

% Paquetes de links
\usepackage[hidelinks]{hyperref}    % Permite enlaces
\usepackage{url}                    % redirecciona a la web

% Más opciones para enumeraciones
\usepackage{enumitem}

% Personalizar la portada
\usepackage{titling}

% Paquetes de tablas
\usepackage{multirow}


%------------------------------------------------------------------------

%Paquetes de figuras
\usepackage{caption}
\usepackage{subcaption} % Figuras al lado de otras
\usepackage{float}      % Poner figuras en el sitio indicado H.


% Paquetes de imágenes
\usepackage{graphicx}       % Paquete para añadir imágenes
\usepackage{transparent}    % Para manejar la opacidad de las figuras

% Paquete para usar colores
\usepackage[dvipsnames]{xcolor}
\usepackage{pagecolor}      % Para cambiar el color de la página

% Habilita tamaños de fuente mayores
\usepackage{fix-cm}

% Para los gráficos
\usepackage{tikz}

% Para poder situar los nodos en los grafos
\usetikzlibrary{positioning}


%------------------------------------------------------------------------

% Paquetes de matemáticas
\usepackage{mathtools, amsfonts, amssymb, mathrsfs}
\usepackage[makeroom]{cancel}     % Simplificar tachando
\usepackage{polynom}    % Divisiones y Ruffini
\usepackage{units} % Para poner fracciones diagonales con \nicefrac

\usepackage{pgfplots}   %Representar funciones
\pgfplotsset{compat=1.18}  % Versión 1.18

\usepackage{tikz-cd}    % Para usar diagramas de composiciones
\usetikzlibrary{calc}   % Para usar cálculo de coordenadas en tikz

%Definición de teoremas, etc.
\usepackage{amsthm}
%\swapnumbers   % Intercambia la posición del texto y de la numeración

\theoremstyle{plain}

\makeatletter
\@ifclassloaded{article}{
  \newtheorem{teo}{Teorema}[section]
}{
  \newtheorem{teo}{Teorema}[chapter]  % Se resetea en cada chapter
}
\makeatother

\newtheorem{coro}{Corolario}[teo]           % Se resetea en cada teorema
\newtheorem{prop}[teo]{Proposición}         % Usa el mismo contador que teorema
\newtheorem{lema}[teo]{Lema}                % Usa el mismo contador que teorema

\theoremstyle{remark}
\newtheorem*{observacion}{Observación}

\theoremstyle{definition}

\makeatletter
\@ifclassloaded{article}{
  \newtheorem{definicion}{Definición} [section]     % Se resetea en cada chapter
}{
  \newtheorem{definicion}{Definición} [chapter]     % Se resetea en cada chapter
}
\makeatother

\newtheorem*{notacion}{Notación}
\newtheorem*{ejemplo}{Ejemplo}
\newtheorem*{ejercicio*}{Ejercicio}             % No numerado
\newtheorem{ejercicio}{Ejercicio} [section]     % Se resetea en cada section


% Modificar el formato de la numeración del teorema "ejercicio"
\renewcommand{\theejercicio}{%
  \ifnum\value{section}=0 % Si no se ha iniciado ninguna sección
    \arabic{ejercicio}% Solo mostrar el número de ejercicio
  \else
    \thesection.\arabic{ejercicio}% Mostrar número de sección y número de ejercicio
  \fi
}


% \renewcommand\qedsymbol{$\blacksquare$}         % Cambiar símbolo QED
%------------------------------------------------------------------------

% Paquetes para encabezados
\usepackage{fancyhdr}
\pagestyle{fancy}
\fancyhf{}

\newcommand{\helv}{ % Modificación tamaño de letra
\fontfamily{}\fontsize{12}{12}\selectfont}
\setlength{\headheight}{15pt} % Amplía el tamaño del índice


%\usepackage{lastpage}   % Referenciar última pag   \pageref{LastPage}
\fancyfoot[C]{\thepage}

%------------------------------------------------------------------------

% Conseguir que no ponga "Capítulo 1". Sino solo "1."
\makeatletter
\@ifclassloaded{book}{
  \renewcommand{\chaptermark}[1]{\markboth{\thechapter.\ #1}{}} % En el encabezado
    
  \renewcommand{\@makechapterhead}[1]{%
  \vspace*{50\p@}%
  {\parindent \z@ \raggedright \normalfont
    \ifnum \c@secnumdepth >\m@ne
      \huge\bfseries \thechapter.\hspace{1em}\ignorespaces
    \fi
    \interlinepenalty\@M
    \Huge \bfseries #1\par\nobreak
    \vskip 40\p@
  }}
}
\makeatother

%------------------------------------------------------------------------
% Paquetes de cógido
\usepackage{minted}
\renewcommand\listingscaption{Código fuente}

\usepackage{fancyvrb}
% Personaliza el tamaño de los números de línea
\renewcommand{\theFancyVerbLine}{\small\arabic{FancyVerbLine}}

% Estilo para C++
\newminted{cpp}{
    frame=lines,
    framesep=2mm,
    baselinestretch=1.2,
    linenos,
    escapeinside=||
}

% para minted
\definecolor{LightGray}{rgb}{0.95,0.95,0.92}
\setminted{
    linenos=true,
    stepnumber=5,
    numberfirstline=true,
    autogobble,
    breaklines=true,
    breakautoindent=true,
    breaksymbolleft=,
    breaksymbolright=,
    breaksymbolindentleft=0pt,
    breaksymbolindentright=0pt,
    breaksymbolsepleft=0pt,
    breaksymbolsepright=0pt,
    fontsize=\footnotesize,
    bgcolor=LightGray,
    numbersep=10pt
}


\usepackage{listings} % Para incluir código desde un archivo

\renewcommand\lstlistingname{Código Fuente}
\renewcommand\lstlistlistingname{Índice de Códigos Fuente}

% Definir colores
\definecolor{vscodepurple}{rgb}{0.5,0,0.5}
\definecolor{vscodeblue}{rgb}{0,0,0.8}
\definecolor{vscodegreen}{rgb}{0,0.5,0}
\definecolor{vscodegray}{rgb}{0.5,0.5,0.5}
\definecolor{vscodebackground}{rgb}{0.97,0.97,0.97}
\definecolor{vscodelightgray}{rgb}{0.9,0.9,0.9}

% Configuración para el estilo de C similar a VSCode
\lstdefinestyle{vscode_C}{
  backgroundcolor=\color{vscodebackground},
  commentstyle=\color{vscodegreen},
  keywordstyle=\color{vscodeblue},
  numberstyle=\tiny\color{vscodegray},
  stringstyle=\color{vscodepurple},
  basicstyle=\scriptsize\ttfamily,
  breakatwhitespace=false,
  breaklines=true,
  captionpos=b,
  keepspaces=true,
  numbers=left,
  numbersep=5pt,
  showspaces=false,
  showstringspaces=false,
  showtabs=false,
  tabsize=2,
  frame=tb,
  framerule=0pt,
  aboveskip=10pt,
  belowskip=10pt,
  xleftmargin=10pt,
  xrightmargin=10pt,
  framexleftmargin=10pt,
  framexrightmargin=10pt,
  framesep=0pt,
  rulecolor=\color{vscodelightgray},
  backgroundcolor=\color{vscodebackground},
}

%------------------------------------------------------------------------

% Comandos definidos
\newcommand{\bb}[1]{\mathbb{#1}}
\newcommand{\cc}[1]{\mathcal{#1}}

% I prefer the slanted \leq
\let\oldleq\leq % save them in case they're every wanted
\let\oldgeq\geq
\renewcommand{\leq}{\leqslant}
\renewcommand{\geq}{\geqslant}

% Si y solo si
\newcommand{\sii}{\iff}

% Letras griegas
\newcommand{\eps}{\epsilon}
\newcommand{\veps}{\varepsilon}
\newcommand{\lm}{\lambda}

\newcommand{\ol}{\overline}
\newcommand{\ul}{\underline}
\newcommand{\wt}{\widetilde}
\newcommand{\wh}{\widehat}

\let\oldvec\vec
\renewcommand{\vec}{\overrightarrow}

% Derivadas parciales
\newcommand{\del}[2]{\frac{\partial #1}{\partial #2}}
\newcommand{\Del}[3]{\frac{\partial^{#1} #2}{\partial #3^{#1}}}
\newcommand{\deld}[2]{\dfrac{\partial #1}{\partial #2}}
\newcommand{\Deld}[3]{\dfrac{\partial^{#1} #2}{\partial #3^{#1}}}


\newcommand{\AstIg}{\stackrel{(\ast)}{=}}
\newcommand{\Hop}{\stackrel{L'H\hat{o}pital}{=}}

\newcommand{\red}[1]{{\color{red}#1}} % Para integrales, destacar los cambios.

% Método de integración
\newcommand{\MetInt}[2]{
    \left[\begin{array}{c}
        #1 \\ #2
    \end{array}\right]
}

% Declarar aplicaciones
% 1. Nombre aplicación
% 2. Dominio
% 3. Codominio
% 4. Variable
% 5. Imagen de la variable
\newcommand{\Func}[5]{
    \begin{equation*}
        \begin{array}{rrll}
            #1:& #2 & \longrightarrow & #3\\
               & #4 & \longmapsto & #5
        \end{array}
    \end{equation*}
}

%------------------------------------------------------------------------



\begin{document}

    % 1. Foto de fondo
    % 2. Título
    % 3. Encabezado Izquierdo
    % 4. Color de fondo
    % 5. Coord x del titulo
    % 6. Coord y del titulo
    % 7. Fecha

    
    % 1. Foto de fondo
% 2. Título
% 3. Encabezado Izquierdo
% 4. Color de fondo
% 5. Coord x del titulo
% 6. Coord y del titulo
% 7. Fecha

\newcommand{\portada}[7]{

    \portadaBase{#1}{#2}{#3}{#4}{#5}{#6}{#7}
    \portadaBook{#1}{#2}{#3}{#4}{#5}{#6}{#7}
}

\newcommand{\portadaExamen}[7]{

    \portadaBase{#1}{#2}{#3}{#4}{#5}{#6}{#7}
    \portadaArticle{#1}{#2}{#3}{#4}{#5}{#6}{#7}
}




\newcommand{\portadaBase}[7]{

    % Tiene la portada principal y la licencia Creative Commons
    
    % 1. Foto de fondo
    % 2. Título
    % 3. Encabezado Izquierdo
    % 4. Color de fondo
    % 5. Coord x del titulo
    % 6. Coord y del titulo
    % 7. Fecha
    
    
    \thispagestyle{empty}               % Sin encabezado ni pie de página
    \newgeometry{margin=0cm}        % Márgenes nulos para la primera página
    
    
    % Encabezado
    \fancyhead[L]{\helv #3}
    \fancyhead[R]{\helv \nouppercase{\leftmark}}
    
    
    \pagecolor{#4}        % Color de fondo para la portada
    
    \begin{figure}[p]
        \centering
        \transparent{0.3}           % Opacidad del 30% para la imagen
        
        \includegraphics[width=\paperwidth, keepaspectratio]{assets/#1}
    
        \begin{tikzpicture}[remember picture, overlay]
            \node[anchor=north west, text=white, opacity=1, font=\fontsize{60}{90}\selectfont\bfseries\sffamily, align=left] at (#5, #6) {#2};
            
            \node[anchor=south east, text=white, opacity=1, font=\fontsize{12}{18}\selectfont\sffamily, align=right] at (9.7, 3) {\textbf{\href{https://losdeldgiim.github.io/}{Los Del DGIIM}}};
            
            \node[anchor=south east, text=white, opacity=1, font=\fontsize{12}{15}\selectfont\sffamily, align=right] at (9.7, 1.8) {Doble Grado en Ingeniería Informática y Matemáticas\\Universidad de Granada};
        \end{tikzpicture}
    \end{figure}
    
    
    \restoregeometry        % Restaurar márgenes normales para las páginas subsiguientes
    \pagecolor{white}       % Restaurar el color de página
    
    
    \newpage
    \thispagestyle{empty}               % Sin encabezado ni pie de página
    \begin{tikzpicture}[remember picture, overlay]
        \node[anchor=south west, inner sep=3cm] at (current page.south west) {
            \begin{minipage}{0.5\paperwidth}
                \href{https://creativecommons.org/licenses/by-nc-nd/4.0/}{
                    \includegraphics[height=2cm]{assets/Licencia.png}
                }\vspace{1cm}\\
                Esta obra está bajo una
                \href{https://creativecommons.org/licenses/by-nc-nd/4.0/}{
                    Licencia Creative Commons Atribución-NoComercial-SinDerivadas 4.0 Internacional (CC BY-NC-ND 4.0).
                }\\
    
                Eres libre de compartir y redistribuir el contenido de esta obra en cualquier medio o formato, siempre y cuando des el crédito adecuado a los autores originales y no persigas fines comerciales. 
            \end{minipage}
        };
    \end{tikzpicture}
    
    
    
    % 1. Foto de fondo
    % 2. Título
    % 3. Encabezado Izquierdo
    % 4. Color de fondo
    % 5. Coord x del titulo
    % 6. Coord y del titulo
    % 7. Fecha


}


\newcommand{\portadaBook}[7]{

    % 1. Foto de fondo
    % 2. Título
    % 3. Encabezado Izquierdo
    % 4. Color de fondo
    % 5. Coord x del titulo
    % 6. Coord y del titulo
    % 7. Fecha

    % Personaliza el formato del título
    \pretitle{\begin{center}\bfseries\fontsize{42}{56}\selectfont}
    \posttitle{\par\end{center}\vspace{2em}}
    
    % Personaliza el formato del autor
    \preauthor{\begin{center}\Large}
    \postauthor{\par\end{center}\vfill}
    
    % Personaliza el formato de la fecha
    \predate{\begin{center}\huge}
    \postdate{\par\end{center}\vspace{2em}}
    
    \title{#2}
    \author{\href{https://losdeldgiim.github.io/}{Los Del DGIIM}}
    \date{Granada, #7}
    \maketitle
    
    \tableofcontents
}




\newcommand{\portadaArticle}[7]{

    % 1. Foto de fondo
    % 2. Título
    % 3. Encabezado Izquierdo
    % 4. Color de fondo
    % 5. Coord x del titulo
    % 6. Coord y del titulo
    % 7. Fecha

    % Personaliza el formato del título
    \pretitle{\begin{center}\bfseries\fontsize{42}{56}\selectfont}
    \posttitle{\par\end{center}\vspace{2em}}
    
    % Personaliza el formato del autor
    \preauthor{\begin{center}\Large}
    \postauthor{\par\end{center}\vspace{3em}}
    
    % Personaliza el formato de la fecha
    \predate{\begin{center}\huge}
    \postdate{\par\end{center}\vspace{5em}}
    
    \title{#2}
    \author{\href{https://losdeldgiim.github.io/}{Los Del DGIIM}}
    \date{Granada, #7}
    \thispagestyle{empty}               % Sin encabezado ni pie de página
    \maketitle
    \vfill
}
    \portadaExamen{ffccA4.jpg}{Cálculo II\\Examen V}{Cálculo II. Examen V}{MidnightBlue}{-8}{28}{2023}

    \begin{description}
        \item[Asignatura] Cálculo II.
        \item[Curso Académico] 2020-21.
        \item[Grado] Doble Grado en Ingeniería Informática y Matemáticas.
        \item[Grupo] Único.
        \item[Profesor] María Victoria Velasco Collado.
        \item[Descripción] Primer Parcial. Derivación. Temas 1-4.
        \item[Fecha] 3 de mayo de 2021.
        %\item[Duración] 60 minutos.
    
    \end{description}
    \newpage
    
    \begin{ejercicio} \textbf{[1 punto]}. Demostrar que $|\cos x - \cos y| \leq |x-y|$, para cada $x,y\in \bb{R}$.

Fijo $y=y_0$. Calculamos la imagen de:
\begin{equation*}
    f:\bb{R}\to \bb{R} \qquad f(x)=|\cos x - \cos y| - |x-y|
\end{equation*}

\begin{itemize}
    \item \underline{\textbf{Opción 1}: Usando teoría de funciones lipschitzianas}.

    Sea $f(x)=\cos(x)$. Como $f\in C^\infty(\bb{R})$ y tenemos que $f'$ está acotada, tenemos que $f$ es lipschitziana, por lo que:
    \begin{equation*}
        |\cos x - \cos y| \leq M|x-y|
    \end{equation*}

    El valor mínimo de $M$ que se puede emplear es denominado constante de Lipschitz, y en este caso tenemos que es la cota de la derivada. Como $|f'(x)|\leq 1 \;\forall x\in \bb{R}$, tenemos que $M\geq 1$. Por tanto,
    \begin{equation*}
        |\cos x - \cos y| \leq 1\cdot |x-y| \leq M|x-y|
    \end{equation*}

    En conclusión, queda demostrado que 
    \begin{equation*}
        |\cos x - \cos y| \leq |x-y| \qquad \forall x,y\in \bb{R}
    \end{equation*}

    \item \underline{\textbf{Opción 2}: Calculando las imágenes de las funciones}.

    Dividimos nuestro estudio en 4 casos, para así no trabajar con el valor absoluto.
    \begin{itemize}
        \item \underline{Suponemos $\cos x \geq \cos y \qquad \land \qquad x\geq y$}:
        \begin{equation*}
            f:\bb{R}\to \bb{R} \qquad f(x)=\cos x - \cos y - x+y
        \end{equation*}
        \begin{equation*}
            f'(x) = -\sen x -1 \leq 0 \qquad \forall x\in \bb{R}
        \end{equation*}
        Por tanto, tenemos que $f$ es decreciente. Como, en este caso, la función está definida en $[y,+\infty[$, y $f(y)=0$, tenemos que:
        \begin{equation*}
            f(x)\leq 0 \qquad \forall x\geq y
        \end{equation*}
    
        \item \underline{Suponemos $\cos x < \cos y \qquad \land \qquad x\geq y$}:
        \begin{equation*}
            f:\bb{R}\to \bb{R} \qquad f(x)=-\cos x + \cos y - x+y
        \end{equation*}
        \begin{equation*}
            f'(x) = \sen x -1 \leq 0 \qquad \forall x\in \bb{R}
        \end{equation*}
        Por tanto, tenemos que $f$ es decreciente. Como, en este caso, la función está definida en $[y,+\infty[$, y $f(y)=0$, tenemos que:
        \begin{equation*}
            f(x)\leq 0 \qquad \forall x\geq y
        \end{equation*}
    
        \item \underline{Suponemos $\cos x < \cos y \qquad \land \qquad x< y$}:
        \begin{equation*}
            f:\bb{R}\to \bb{R} \qquad f(x)=-\cos x + \cos y + x-y
        \end{equation*}
        \begin{equation*}
            f'(x) = \sen x +1 \geq 0 \qquad \forall x\in \bb{R}
        \end{equation*}
        Por tanto, tenemos que $f$ es creciente. Como, en este caso, la función está definida en $]-\infty, y[$, y $\lim_{x\to y^-}f(x)=0$, tenemos que:
        \begin{equation*}
            f(x)\leq 0 \qquad \forall x< y
        \end{equation*}
    
        \item \underline{Suponemos $\cos x \geq \cos y \qquad \land \qquad x< y$}:
        \begin{equation*}
            f:\bb{R}\to \bb{R} \qquad f(x)=\cos x - \cos y + x-y
        \end{equation*}
        \begin{equation*}
            f'(x) = -\sen x +1 \geq 0 \qquad \forall x\in \bb{R}
        \end{equation*}
        Por tanto, tenemos que $f$ es creciente. Como, en este caso, la función está definida en $]-\infty,y[$, y $\lim_{x\to y^-}f(x)=0$, tenemos que:
        \begin{equation*}
            f(x)\leq 0 \qquad \forall x< y
        \end{equation*}
    \end{itemize}
    
    Por tanto, independientemente del caso en el que estemos, tenemos que $f(x)\leq~0\quad \forall x\in~\bb{R}$. Por tanto,
    \begin{equation*}
        f(x)=|\cos x -\cos y|-|x-y|\leq 0 \Longrightarrow  |\cos x -\cos y|\leq |x-y| \qquad \forall x\in \bb{R}
    \end{equation*}
    
    Como el razonamiento es válido $\forall y\in \bb{R}$ (simplemente tendríamos que cambiar el valor de $y_0$ fijado), tenemos que:
    \begin{equation*}
        |\cos x -\cos y|\leq |x-y| \qquad \forall x,y\in \bb{R}
    \end{equation*}
\end{itemize}



\end{ejercicio}


\begin{ejercicio} \textbf{[2 puntos]}. Sea $a>0$.
\begin{enumerate}
    \item Determinar (en función del parámetro $a$) la imagen de la función $f_a:\bb{R}^+\to \bb{R}$ dada por $f_a(x):=x\ln a - a\ln x$, para cada $x\in \bb{R}^+$.

    Necesitamos calcular la imagen de la función. Para ello, en primer lugar, calculamos los extremos relativos sabiendo que la función es continua y derivable.
    \begin{equation*}
        f'_a(x) = \ln a -\frac{a}{x} = 0 \Longleftrightarrow x=\frac{a}{\ln a}
    \end{equation*}
    \begin{equation*}
        f''_a(x) = \frac{a}{x^2} >0 \qquad \forall x\in \bb{R}^+
    \end{equation*}
    Por tanto, tenemos que $x=\frac{a}{\ln a}$ es un mínimo relativo.
    \begin{equation*}
        f_a\left(\frac{a}{\ln a}\right) = \frac{a}{\ln a} \cdot \ln a - a\ln \frac{a}{\ln a} = a-a\ln \frac{a}{\ln a} =a\left(1-\ln \frac{a}{\ln a}\right)
    \end{equation*}
    
    Calculamos ahora el comportamiento de la función en $x=0$ y en $+\infty$.

    \begin{equation*}
        \lim_{x\to 0} f_a(x) = \lim_{x\to 0} x\ln a - a\ln x = -a\ln(0) = +\infty
    \end{equation*}
    
    \begin{equation*}
        \lim_{x\to \infty} f_a(x) = \lim_{x\to \infty} x\ln a - a\ln x 
        = \lim_{x\to \infty} \ln \left(\frac{a^x}{x^a}\right)
    \end{equation*}
    Por tanto, para estudiar el comportamiento en $+\infty$, diferencio en función del valor de $a$:
    \begin{itemize}
        \item \underline{Para $a=1$}:
        \begin{equation*}
            \lim_{x\to \infty} f_a(x) 
            = \lim_{x\to \infty} \ln \left(\frac{1}{x}\right) = \ln 0 = -\infty
        \end{equation*}

        \item \underline{Para $a<1$}:
        \begin{equation*}
            \lim_{x\to \infty} f_a(x) 
            = \lim_{x\to \infty} \ln \left(\frac{a^x}{x^a}\right) =\ln \frac{0}{\infty} = \ln 0 = -\infty
        \end{equation*}

        \item \underline{Para $a>1$}:
        \begin{equation*}
            \lim_{x\to \infty} f_a(x) 
            = \lim_{x\to \infty} \ln \left(\frac{a^x}{x^a}\right) =\ln \infty = \infty
        \end{equation*}
    \end{itemize}

    Por tanto, para $a>1$, tenemos que $Im(f_a)=\left[a\left(1-\ln \frac{a}{\ln a}\right),+\infty\right[$.
    
    Para $a\leq 1$, tenemos que $Im(f_a)=\bb{R}$.

    \item Determinar los valores de $a>0$ que son tales que $x\ln a \geq a\ln x$, para cada $x\in \bb{R}^+$.

    Para ello, necesito que $Im(f_a)\subseteq \bb{R}^+_0$. Por tanto, descartamos los $a\leq 1$.

    Para que $Im(f_a)\subseteq \bb{R}^+_0$, necesitamos que:
    \begin{equation*}
        a\left(1-\ln \frac{a}{\ln a}\right) \geq 0 \Longleftrightarrow \ln \frac{a}{\ln a} \leq 1 \Longleftrightarrow \frac{a}{\ln a} \leq e \Longleftrightarrow e -\frac{a}{\ln a} \geq 0
    \end{equation*}

    Calculo por tanto la imagen de $g:]1,+\infty[\to \bb{R}$ dado por $g(x)=e-\frac{x}{\ln x}$.

    \begin{equation*}
        g'(x) = \frac{-\ln x +1}{\ln^2 x} = 0 \Longleftrightarrow \ln x=1 \Longleftrightarrow x=e
    \end{equation*}
    \begin{itemize}
        \item \underline{Para $x<e$}: $g'(x)>0 \Longrightarrow g$ estrictamente creciente.
        \item \underline{Para $x>e$}: $g'(x)<0 \Longrightarrow g$ estrictamente decreciente.
    \end{itemize}

    Por tanto, tenemos que $x=e$ es un máximo relativo. Sabemos que $g(e)=0$. Vemos ahora el comportamiento en $x=1^+$ y en $+\infty$:
    \begin{equation*}
        \lim_{x\to 1^+}g(x) = -\frac{1}{\ln 1^+} = -\frac{1}{0^+} = -\infty
        \qquad
        \lim_{x\to \infty}g(x) = e-\lim_{x\to \infty} \frac{x}{\ln x} = e-\infty = -\infty
    \end{equation*}

    Por tanto, tenemos que $Im(g)=\bb{R}^-_0$. Por tanto, el único valor de $x$ que hace que $g(x)\geq 0$ es $x=e$.
    
    Por tanto, el valor de $a>0$ que hace que $x\ln a \geq a\ln \;\forall x\in \bb{R}^+$ es $a=e$.

\end{enumerate}
    
\end{ejercicio}



\begin{ejercicio} \textbf{[2.5 puntos]} Sea $f:\bb{R}\to \bb{R}$ la función dada por $f(x)=4-x^2$.
\begin{enumerate}
    \item Estudiar la concavidad de $f$. ¿Posee algún punto de inflexión? Justifíquese la respuesta.

    Sabemos que $f\in C^\infty (\bb{R})$. Por tanto,
    \begin{equation*}
        f'(x)=-2x \qquad f''(x)=-2
    \end{equation*}

    Como $f$ es continua y derivable en $\bb{R}$, tenemos que la concavidad la determina la segunda derivada. Como $f''(x)=-2<0\;\forall x$, tenemos que $f$ es cóncava hacia abajo en todos los reales.

    Además, al ser $f$ dos veces derivable, una condición necesaria de punto de inflexión es que, en él, se anule la segunda derivada. Como $\nexists x\in \bb{R}\mid f''(x)=0$, entonces podemos afirmar que no hay puntos de inflexión.

    \item Determinar el punto $(a, f(a))$ de la gráfica de $f(x)=4-x^2$ cuya recta tangente corta en el primer cuadrante tanto al eje $OX$ como al eje $OY$, determinando un triángulo de área mínima.

    \begin{figure}[H]
        \centering
        \begin{tikzpicture}
        \begin{axis}[
            xlabel=$x$,
            xmin=-1,
            xmax=5,
            ymin=-1,
            ymax=7,
            axis lines=middle,
            width=7cm,
            height=5cm,
            samples=90 % número de muestras para la función
        ]
        \addplot[blue,thick,domain=-1:5] {4-x^2};
        \addplot[thick,domain=-1:5] {-2*x+5};

        \addplot[only marks,mark=*,mark size=2pt,color=red] coordinates {(1,3)};
        \node[label={above right: $P(a, f(a))$}] at (axis cs:1-0.3,3-0.5) {};

        \addplot[only marks,mark=*,mark size=2pt,color=red] coordinates {(0,5)};
        \node[label={above right: $H(0,h)$}] at (axis cs:0-0.3,5-0.5) {};

        \addplot[only marks,mark=*,mark size=2pt,color=red] coordinates {(2.5,0)};
        \node[label={above right: $B(b,0)$}] at (axis cs:2.5-0.3,0-0.5) {};

        \addplot[fill=orange,fill opacity=0.25] coordinates {(0,0) (0,5) (2.5,0)};
            
        \end{axis}
        \end{tikzpicture}
    \end{figure}

    Por la interpretación geométrica de la derivada, tenemos que $f'(a)=-2a=~m_t$. Por tanto, la ecuación de la recta que pasa por el punto $P$ es:
    \begin{equation*}
        f(x)=m_tx +n = -2ax+h
    \end{equation*}

    Como el punto $P$ pertenece tanto a la recta como a la parábola,
    \begin{equation*}
        f(a)=f(a)\Longrightarrow -2a^2+h = 4-a^2 \Longrightarrow h=4+a^2
    \end{equation*}

    Además, para $y=0$, tenemos:
    \begin{equation*}
        f(b)=0 = -2ab+h \Longrightarrow b=\frac{h}{2a} = \frac{4+a^2}{2a}
    \end{equation*}

    por tanto, una vez establecidas las ecuaciones de ligadura, tenemos:
    \begin{equation*}
        \begin{array}{rl}
            A:]0, 2[ & \longrightarrow \bb{R}\\
                    a & \longrightarrow A(a) = \displaystyle \frac{bh}{2} = \frac{\frac{4+a^2}{2a} \cdot (4+a^2)}{2} = \frac{(4+a^2)^2}{4a}
        \end{array}
    \end{equation*}

    Para calcular el área mínima, minimizamos $A(a)$. Para ello, calculamos el mínimo de la función.
    \begin{multline*}
        A'(a) = \frac{(4a)^2(4+a^2) -4(4+a^2)^2}{(4a)^2} = 0 \Longleftrightarrow
        (4+a^2)(16a^2-4(4+a^2)) =\\= (4+a^2)(12a^2-16) = 0 \Longleftrightarrow a=\sqrt{\frac{16}{12}}=\sqrt{\frac{4}{3}} = \frac{2\sqrt{3}}{3}
    \end{multline*}

    Comprobemos ahora si el punto crítico es, efectivamente, un mínimo relativo.
    \begin{itemize}
        \item \underline{Para $0<a<\frac{2\sqrt{3}}{3}$}: $A'(a)<0\Longrightarrow A'$ decreciente.
        \item \underline{Para $\frac{2\sqrt{3}}{3}<a<2$}: $A'(a)>0\Longrightarrow A'$ creciente.
    \end{itemize}

    Por tanto, confirmamos que es un mínimo relativo. También es absoluto en ese intervalo, ya que es una función continua en un intervalo.

    \begin{equation*}
        f\left(\frac{2\sqrt{3}}{3}\right) = 4-\left(\frac{2\sqrt{3}}{3}\right)^2 = 4-\frac{4}{3} = \frac{12-4}{3} = \frac{8}{3}
    \end{equation*}

    Por tanto, el punto que minimiza el área es:
    \begin{equation*}
        (a,f(a)) = \left(\frac{2\sqrt{3}}{3}, f\left(\frac{2\sqrt{3}}{3}\right)\right) = \left(\frac{2\sqrt{3}}{3}, \frac{8}{3}\right)
    \end{equation*}
\end{enumerate}
\end{ejercicio}

\begin{ejercicio} \textbf{[3 puntos]}
    \begin{enumerate}
        \item Calcular el polinomio de Taylor de orden 10 centrado en el origen de las funciones $f(x)=\sen x$ y $g(x)=\cos x$.

        Las derivadas sucesivas del seno son:
        \begin{equation*}
            f^{(2k)}(0) = 0 \qquad f^{(4k+1)}(0) = 1 \qquad f^{(4k+3)}(0) = -1 \qquad \forall k\in \bb{N}
        \end{equation*}

        Por tanto,
        \begin{equation*}
            P_{10,0}^f(x) = x - \frac{x^3}{3!} + \frac{x^5}{5!} - \frac{x^7}{7!} + \frac{x^9}{9!}
        \end{equation*}

        Respecto al coseno, tenemos:
        \begin{equation*}
            f^{(4k)}(0) = 1 \qquad f^{(2k+1)}(0) = 0 \qquad f^{(4k+2)}(0) = -1 \qquad \forall k\in \bb{N}
        \end{equation*}

        Por tanto,
        \begin{equation*}
            P_{10,0}^g(x) = 1 - \frac{x^2}{2!} + \frac{x^4}{4!} - \frac{x^6}{6!} + \frac{x^8}{8!} - \frac{x^{10}}{10!}
        \end{equation*}

        \item Determinar $P_{3,0}^{\sen x} \left(\frac{\pi}{18}\right)$ y $P_{3,0}^{\cos x} \left(\frac{\pi}{18}\right)$ y dar la estimación del error cometido al aproximar $\sen \left(\frac{\pi}{18}\right)$ y $\cos \left(\frac{\pi}{18}\right)$ por dichos valores, respectivamente (esto es una aproximación del seno y del coseno del ángulo de $10^0$).

        En primer lugar, trabajo con el seno.
        \begin{equation*}
            P_{3,0}^{\sen x} \left(\frac{\pi}{18}\right) = \frac{\pi}{18} -\frac{\pi^3}{18^3\cdot 3!} \approx 0.1736
        \end{equation*}
        
        Usando el Resto de Lagrange, el error cometido es, para algún $c\in \left[0, \frac{\pi}{18}\right]$:
        \begin{equation*}
            R_{3,0}^{\sen x} \left(\frac{\pi}{18}\right) = \frac{f^{4)}(c)}{4!} \left(\frac{\pi}{18}\right)^4 = \frac{\sen c}{4!}  \left(\frac{\pi}{18}\right)^4 \leq \frac{\pi^4}{4! \cdot 18^4} = 3.8663\cdot 10^{-5}
        \end{equation*}


        Respecto al coseno, la aproximación es:
        \begin{equation*}
            P_{3,0}^{\cos x} \left(\frac{\pi}{18}\right) = 1 -\frac{\pi^2}{18^2\cdot 2} \approx 0.984769
        \end{equation*}
        
        Usando el Resto de Lagrange, el error cometido es, para algún $c\in \left[0, \frac{\pi}{18}\right]$:
        \begin{equation*}
            R_{3,0}^{\cos x} \left(\frac{\pi}{18}\right) = \frac{g^{4)}(c)}{4!} \left(\frac{\pi}{18}\right)^4 = \frac{\cos c}{4!}  \left(\frac{\pi}{18}\right)^4 \leq \frac{\pi^4}{4! \cdot 18^4} = 3.8663\cdot 10^{-5}
        \end{equation*}

        No obstante, aunque la acotación del error a la que hemos llegado es la misma, como para $x\in [0,\frac{\pi}{4}[$ se tiene que $\sen x < \cos x$, entonces podemos afirmar que la aproximación del seno es mejor.

        \item Haciendo uso del polinomio de Taylor, calcular $\lim_{x\to 0} \frac{(2x-\sen x)(\cos x -1)}{x^3}$.

        Sea $h(x)=(2x-\sen x)(\cos x -1)$.
        \begin{equation*}
            \lim_{x\to 0} \frac{h(x)}{x^3}
            = \lim_{x\to 0} \cancelto{0}{\frac{h(x)-P_{3,0}^h(x)}{x^3}} +\frac{P_{3,0}^h(x)}{x^3} = \lim_{x\to 0} \frac{P_{3,0}^h(x)}{x^3}
            \stackrel{Ec.\;\ref{Ej4:Taylor3}}{=} -\frac{1}{2}
        \end{equation*}

        Calculamos ahora el polinomio de Taylor necesario:
        \begin{equation} \label{Ej4:Taylor3}
            P_{3,0}^h(x) = \left[\left(2x-x+\frac{x^3}{3!}\right) \left(1-\frac{x^2}{2!} -1\right)\right]_{n=3}
            = \left[\left(x+\frac{x^3}{6}\right) \left(-\frac{x^2}{2}\right)\right]_{n=3} = -\frac{x^3}{2}
        \end{equation}
    \end{enumerate}
\end{ejercicio}

\begin{ejercicio} \textbf{[1.5 puntos]}
Calcular
\begin{equation*}
    \lim_{x\to 0} (1+x)^{\frac{1}{x}} \left( \frac{1}{x(1+x)} -\frac{\ln (1+x)}{x^2}\right).
\end{equation*}

Calculo el límite por partes. Resuelvo en primer la primera parte
\begin{equation} \label{Ej5:Parte1}
    \lim_{x\to 0} (1+x)^{\frac{1}{x}}
    = \lim_{x\to 0} e^{\frac{\ln (1+x)}{x}} \stackrel{Ec.\;\ref{Ej5:Ind1}}{=} e^1 = e
\end{equation}

donde he tenido que resolver previamente:
\begin{equation} \label{Ej5:Ind1}
    \lim_{x\to 0} \frac{\ln (1+x)}{x} \Hop \lim_{x\to 0} \frac{1}{1+x} = 1
\end{equation}

Resuelvo ahora la segunda parte:
\begin{multline} \label{Ej5:Parte2}
    \lim_{x\to 0} \left( \frac{1}{x(1+x)} -\frac{\ln (1+x)}{x^2}\right)
     = \lim_{x\to 0} \frac{x  -(1+x)\ln(1+x)}{x^2(1+x)}
     \Hop\\=
     \lim_{x\to 0} \frac{1  -\ln(1+x)-\frac{1+x}{1+x}}{2x(1+x)+x^2}
     =
      - \lim_{x\to 0} \frac{\ln(1+x)}{2x+3x^2}
      \Hop
      - \lim_{x\to 0} \frac{\frac{1}{1+x}}{2+6x} = -\frac{1}{2}
\end{multline}

Por tanto, uniendo ambas partes, tenemos:
\begin{equation*}
    \lim_{x\to 0} (1+x)^{\frac{1}{x}} \left( \frac{1}{x(1+x)} -\frac{\ln (1+x)}{x^2}\right) \stackrel{Ec.\;\ref{Ej5:Parte1},\ref{Ej5:Parte2}} =-\frac{e}{2}
\end{equation*}


\end{ejercicio}



\end{document}