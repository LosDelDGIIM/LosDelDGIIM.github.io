\documentclass[12pt]{article}

% Idioma y codificación
\usepackage[spanish, es-tabla, es-notilde]{babel}       %es-tabla para que se titule "Tabla"
\usepackage[utf8]{inputenc}

% Márgenes
\usepackage[a4paper,top=3cm,bottom=2.5cm,left=3cm,right=3cm]{geometry}

% Comentarios de bloque
\usepackage{verbatim}

% Paquetes de links
\usepackage[hidelinks]{hyperref}    % Permite enlaces
\usepackage{url}                    % redirecciona a la web

% Más opciones para enumeraciones
\usepackage{enumitem}

% Personalizar la portada
\usepackage{titling}

% Paquetes de tablas
\usepackage{multirow}

% Para añadir el símbolo de euro
\usepackage{eurosym}


%------------------------------------------------------------------------

%Paquetes de figuras
\usepackage{caption}
\usepackage{subcaption} % Figuras al lado de otras
\usepackage{float}      % Poner figuras en el sitio indicado H.


% Paquetes de imágenes
\usepackage{graphicx}       % Paquete para añadir imágenes
\usepackage{transparent}    % Para manejar la opacidad de las figuras

% Paquete para usar colores
\usepackage[dvipsnames, table, xcdraw]{xcolor}
\usepackage{pagecolor}      % Para cambiar el color de la página

% Habilita tamaños de fuente mayores
\usepackage{fix-cm}

% Para los gráficos
\usepackage{tikz}
\usepackage{forest}

% Para poder situar los nodos en los grafos
\usetikzlibrary{positioning}


%------------------------------------------------------------------------

% Paquetes de matemáticas
\usepackage{mathtools, amsfonts, amssymb, mathrsfs}
\usepackage[makeroom]{cancel}     % Simplificar tachando
\usepackage{polynom}    % Divisiones y Ruffini
\usepackage{units} % Para poner fracciones diagonales con \nicefrac

\usepackage{pgfplots}   %Representar funciones
\pgfplotsset{compat=1.18}  % Versión 1.18

\usepackage{tikz-cd}    % Para usar diagramas de composiciones
\usetikzlibrary{calc}   % Para usar cálculo de coordenadas en tikz

%Definición de teoremas, etc.
\usepackage{amsthm}
%\swapnumbers   % Intercambia la posición del texto y de la numeración

\theoremstyle{plain}

\makeatletter
\@ifclassloaded{article}{
  \newtheorem{teo}{Teorema}[section]
}{
  \newtheorem{teo}{Teorema}[chapter]  % Se resetea en cada chapter
}
\makeatother

\newtheorem{coro}{Corolario}[teo]           % Se resetea en cada teorema
\newtheorem{prop}[teo]{Proposición}         % Usa el mismo contador que teorema
\newtheorem{lema}[teo]{Lema}                % Usa el mismo contador que teorema
\newtheorem*{lema*}{Lema}

\theoremstyle{remark}
\newtheorem*{observacion}{Observación}

\theoremstyle{definition}

\makeatletter
\@ifclassloaded{article}{
  \newtheorem{definicion}{Definición} [section]     % Se resetea en cada chapter
}{
  \newtheorem{definicion}{Definición} [chapter]     % Se resetea en cada chapter
}
\makeatother

\newtheorem*{notacion}{Notación}
\newtheorem*{ejemplo}{Ejemplo}
\newtheorem*{ejercicio*}{Ejercicio}             % No numerado
\newtheorem{ejercicio}{Ejercicio} [section]     % Se resetea en cada section


% Modificar el formato de la numeración del teorema "ejercicio"
\renewcommand{\theejercicio}{%
  \ifnum\value{section}=0 % Si no se ha iniciado ninguna sección
    \arabic{ejercicio}% Solo mostrar el número de ejercicio
  \else
    \thesection.\arabic{ejercicio}% Mostrar número de sección y número de ejercicio
  \fi
}


% \renewcommand\qedsymbol{$\blacksquare$}         % Cambiar símbolo QED
%------------------------------------------------------------------------

% Paquetes para encabezados
\usepackage{fancyhdr}
\pagestyle{fancy}
\fancyhf{}

\newcommand{\helv}{ % Modificación tamaño de letra
\fontfamily{}\fontsize{12}{12}\selectfont}
\setlength{\headheight}{15pt} % Amplía el tamaño del índice


%\usepackage{lastpage}   % Referenciar última pag   \pageref{LastPage}
%\fancyfoot[C]{%
%  \begin{minipage}{\textwidth}
%    \centering
%    ~\\
%    \thepage\\
%    \href{https://losdeldgiim.github.io/}{\texttt{\footnotesize losdeldgiim.github.io}}
%  \end{minipage}
%}
\fancyfoot[C]{\thepage}
\fancyfoot[R]{\href{https://losdeldgiim.github.io/}{\texttt{\footnotesize losdeldgiim.github.io}}}

%------------------------------------------------------------------------

% Conseguir que no ponga "Capítulo 1". Sino solo "1."
\makeatletter
\@ifclassloaded{book}{
  \renewcommand{\chaptermark}[1]{\markboth{\thechapter.\ #1}{}} % En el encabezado
    
  \renewcommand{\@makechapterhead}[1]{%
  \vspace*{50\p@}%
  {\parindent \z@ \raggedright \normalfont
    \ifnum \c@secnumdepth >\m@ne
      \huge\bfseries \thechapter.\hspace{1em}\ignorespaces
    \fi
    \interlinepenalty\@M
    \Huge \bfseries #1\par\nobreak
    \vskip 40\p@
  }}
}
\makeatother

%------------------------------------------------------------------------
% Paquetes de cógido
\usepackage{minted}
\renewcommand\listingscaption{Código fuente}

\usepackage{fancyvrb}
% Personaliza el tamaño de los números de línea
\renewcommand{\theFancyVerbLine}{\small\arabic{FancyVerbLine}}

% Estilo para C++
\newminted{cpp}{
    frame=lines,
    framesep=2mm,
    baselinestretch=1.2,
    linenos,
    escapeinside=||
}

% para minted
\definecolor{LightGray}{rgb}{0.95,0.95,0.92}
\setminted{
    linenos=true,
    stepnumber=5,
    numberfirstline=true,
    autogobble,
    breaklines=true,
    breakautoindent=true,
    breaksymbolleft=,
    breaksymbolright=,
    breaksymbolindentleft=0pt,
    breaksymbolindentright=0pt,
    breaksymbolsepleft=0pt,
    breaksymbolsepright=0pt,
    fontsize=\footnotesize,
    bgcolor=LightGray,
    numbersep=10pt
}


\usepackage{listings} % Para incluir código desde un archivo

\renewcommand\lstlistingname{Código Fuente}
\renewcommand\lstlistlistingname{Índice de Códigos Fuente}

% Definir colores
\definecolor{vscodepurple}{rgb}{0.5,0,0.5}
\definecolor{vscodeblue}{rgb}{0,0,0.8}
\definecolor{vscodegreen}{rgb}{0,0.5,0}
\definecolor{vscodegray}{rgb}{0.5,0.5,0.5}
\definecolor{vscodebackground}{rgb}{0.97,0.97,0.97}
\definecolor{vscodelightgray}{rgb}{0.9,0.9,0.9}

% Configuración para el estilo de C similar a VSCode
\lstdefinestyle{vscode_C}{
  backgroundcolor=\color{vscodebackground},
  commentstyle=\color{vscodegreen},
  keywordstyle=\color{vscodeblue},
  numberstyle=\tiny\color{vscodegray},
  stringstyle=\color{vscodepurple},
  basicstyle=\scriptsize\ttfamily,
  breakatwhitespace=false,
  breaklines=true,
  captionpos=b,
  keepspaces=true,
  numbers=left,
  numbersep=5pt,
  showspaces=false,
  showstringspaces=false,
  showtabs=false,
  tabsize=2,
  frame=tb,
  framerule=0pt,
  aboveskip=10pt,
  belowskip=10pt,
  xleftmargin=10pt,
  xrightmargin=10pt,
  framexleftmargin=10pt,
  framexrightmargin=10pt,
  framesep=0pt,
  rulecolor=\color{vscodelightgray},
  backgroundcolor=\color{vscodebackground},
}

%------------------------------------------------------------------------

% Comandos definidos
\newcommand{\bb}[1]{\mathbb{#1}}
\newcommand{\cc}[1]{\mathcal{#1}}

% I prefer the slanted \leq
\let\oldleq\leq % save them in case they're every wanted
\let\oldgeq\geq
\renewcommand{\leq}{\leqslant}
\renewcommand{\geq}{\geqslant}

% Si y solo si
\newcommand{\sii}{\iff}

% MCD y MCM
\DeclareMathOperator{\mcd}{mcd}
\DeclareMathOperator{\mcm}{mcm}

% Signo
\DeclareMathOperator{\sgn}{sgn}

% Letras griegas
\newcommand{\eps}{\epsilon}
\newcommand{\veps}{\varepsilon}
\newcommand{\lm}{\lambda}

\newcommand{\ol}{\overline}
\newcommand{\ul}{\underline}
\newcommand{\wt}{\widetilde}
\newcommand{\wh}{\widehat}

\let\oldvec\vec
\renewcommand{\vec}{\overrightarrow}

% Derivadas parciales
\newcommand{\del}[2]{\frac{\partial #1}{\partial #2}}
\newcommand{\Del}[3]{\frac{\partial^{#1} #2}{\partial #3^{#1}}}
\newcommand{\deld}[2]{\dfrac{\partial #1}{\partial #2}}
\newcommand{\Deld}[3]{\dfrac{\partial^{#1} #2}{\partial #3^{#1}}}


\newcommand{\AstIg}{\stackrel{(\ast)}{=}}
\newcommand{\Hop}{\stackrel{L'H\hat{o}pital}{=}}

\newcommand{\red}[1]{{\color{red}#1}} % Para integrales, destacar los cambios.

% Método de integración
\newcommand{\MetInt}[2]{
    \left[\begin{array}{c}
        #1 \\ #2
    \end{array}\right]
}

% Declarar aplicaciones
% 1. Nombre aplicación
% 2. Dominio
% 3. Codominio
% 4. Variable
% 5. Imagen de la variable
\newcommand{\Func}[5]{
    \begin{equation*}
        \begin{array}{rrll}
            \displaystyle #1:& \displaystyle  #2 & \longrightarrow & \displaystyle  #3\\
               & \displaystyle  #4 & \longmapsto & \displaystyle  #5
        \end{array}
    \end{equation*}
}

%------------------------------------------------------------------------

\usepackage{pgfplots}
\usepackage{amsmath}
\pgfplotsset{compat=1.17}


\begin{document}

    % 1. Foto de fondo
    % 2. Título
    % 3. Encabezado Izquierdo
    % 4. Color de fondo
    % 5. Coord x del titulo
    % 6. Coord y del titulo
    % 7. Fecha

    
    % 1. Foto de fondo
% 2. Título
% 3. Encabezado Izquierdo
% 4. Color de fondo
% 5. Coord x del titulo
% 6. Coord y del titulo
% 7. Fecha
% 8. Autor

\newcommand{\portada}[8]{
    \portadaBase{#1}{#2}{#3}{#4}{#5}{#6}{#7}{#8}
    \portadaBook{#1}{#2}{#3}{#4}{#5}{#6}{#7}{#8}
}

\newcommand{\portadaFotoDif}[8]{
    \portadaBaseFotoDif{#1}{#2}{#3}{#4}{#5}{#6}{#7}{#8}
    \portadaBook{#1}{#2}{#3}{#4}{#5}{#6}{#7}{#8}
}

\newcommand{\portadaExamen}[8]{
    \portadaBase{#1}{#2}{#3}{#4}{#5}{#6}{#7}{#8}
    \portadaArticle{#1}{#2}{#3}{#4}{#5}{#6}{#7}{#8}
}

\newcommand{\portadaExamenFotoDif}[8]{
    \portadaBaseFotoDif{#1}{#2}{#3}{#4}{#5}{#6}{#7}{#8}
    \portadaArticle{#1}{#2}{#3}{#4}{#5}{#6}{#7}{#8}
}




\newcommand{\portadaBase}[8]{

    % Tiene la portada principal y la licencia Creative Commons
    
    % 1. Foto de fondo
    % 2. Título
    % 3. Encabezado Izquierdo
    % 4. Color de fondo
    % 5. Coord x del titulo
    % 6. Coord y del titulo
    % 7. Fecha
    % 8. Autor    
    
    \thispagestyle{empty}               % Sin encabezado ni pie de página
    \newgeometry{margin=0cm}        % Márgenes nulos para la primera página
    
    
    % Encabezado
    \fancyhead[L]{\helv #3}
    \fancyhead[R]{\helv \nouppercase{\leftmark}}
    
    
    \pagecolor{#4}        % Color de fondo para la portada
    
    \begin{figure}[p]
        \centering
        \transparent{0.3}           % Opacidad del 30% para la imagen
        
        \includegraphics[width=\paperwidth, keepaspectratio]{../../_assets/#1}
    
        \begin{tikzpicture}[remember picture, overlay]
            \node[anchor=north west, text=white, opacity=1, font=\fontsize{60}{90}\selectfont\bfseries\sffamily, align=left] at (#5, #6) {#2};
            
            \node[anchor=south east, text=white, opacity=1, font=\fontsize{12}{18}\selectfont\sffamily, align=right] at (9.7, 3) {\href{https://losdeldgiim.github.io/}{\textbf{Los Del DGIIM}, \texttt{\footnotesize losdeldgiim.github.io}}};
            
            \node[anchor=south east, text=white, opacity=1, font=\fontsize{12}{15}\selectfont\sffamily, align=right] at (9.7, 1.8) {Doble Grado en Ingeniería Informática y Matemáticas\\Universidad de Granada};
        \end{tikzpicture}
    \end{figure}
    
    
    \restoregeometry        % Restaurar márgenes normales para las páginas subsiguientes
    \nopagecolor      % Restaurar el color de página
    
    
    \newpage
    \thispagestyle{empty}               % Sin encabezado ni pie de página
    \begin{tikzpicture}[remember picture, overlay]
        \node[anchor=south west, inner sep=3cm] at (current page.south west) {
            \begin{minipage}{0.5\paperwidth}
                \href{https://creativecommons.org/licenses/by-nc-nd/4.0/}{
                    \includegraphics[height=2cm]{../../_assets/Licencia.png}
                }\vspace{1cm}\\
                Esta obra está bajo una
                \href{https://creativecommons.org/licenses/by-nc-nd/4.0/}{
                    Licencia Creative Commons Atribución-NoComercial-SinDerivadas 4.0 Internacional (CC BY-NC-ND 4.0).
                }\\
    
                Eres libre de compartir y redistribuir el contenido de esta obra en cualquier medio o formato, siempre y cuando des el crédito adecuado a los autores originales y no persigas fines comerciales. 
            \end{minipage}
        };
    \end{tikzpicture}
    
    
    
    % 1. Foto de fondo
    % 2. Título
    % 3. Encabezado Izquierdo
    % 4. Color de fondo
    % 5. Coord x del titulo
    % 6. Coord y del titulo
    % 7. Fecha
    % 8. Autor


}


\newcommand{\portadaBaseFotoDif}[8]{

    % Tiene la portada principal y la licencia Creative Commons
    
    % 1. Foto de fondo
    % 2. Título
    % 3. Encabezado Izquierdo
    % 4. Color de fondo
    % 5. Coord x del titulo
    % 6. Coord y del titulo
    % 7. Fecha
    % 8. Autor    
    
    \thispagestyle{empty}               % Sin encabezado ni pie de página
    \newgeometry{margin=0cm}        % Márgenes nulos para la primera página
    
    
    % Encabezado
    \fancyhead[L]{\helv #3}
    \fancyhead[R]{\helv \nouppercase{\leftmark}}
    
    
    \pagecolor{#4}        % Color de fondo para la portada
    
    \begin{figure}[p]
        \centering
        \transparent{0.3}           % Opacidad del 30% para la imagen
        
        \includegraphics[width=\paperwidth, keepaspectratio]{#1}
    
        \begin{tikzpicture}[remember picture, overlay]
            \node[anchor=north west, text=white, opacity=1, font=\fontsize{60}{90}\selectfont\bfseries\sffamily, align=left] at (#5, #6) {#2};
            
            \node[anchor=south east, text=white, opacity=1, font=\fontsize{12}{18}\selectfont\sffamily, align=right] at (9.7, 3) {\href{https://losdeldgiim.github.io/}{\textbf{Los Del DGIIM}, \texttt{\footnotesize losdeldgiim.github.io}}};
            
            \node[anchor=south east, text=white, opacity=1, font=\fontsize{12}{15}\selectfont\sffamily, align=right] at (9.7, 1.8) {Doble Grado en Ingeniería Informática y Matemáticas\\Universidad de Granada};
        \end{tikzpicture}
    \end{figure}
    
    
    \restoregeometry        % Restaurar márgenes normales para las páginas subsiguientes
    \nopagecolor      % Restaurar el color de página
    
    
    \newpage
    \thispagestyle{empty}               % Sin encabezado ni pie de página
    \begin{tikzpicture}[remember picture, overlay]
        \node[anchor=south west, inner sep=3cm] at (current page.south west) {
            \begin{minipage}{0.5\paperwidth}
                %\href{https://creativecommons.org/licenses/by-nc-nd/4.0/}{
                %    \includegraphics[height=2cm]{../../_assets/Licencia.png}
                %}\vspace{1cm}\\
                Esta obra está bajo una
                \href{https://creativecommons.org/licenses/by-nc-nd/4.0/}{
                    Licencia Creative Commons Atribución-NoComercial-SinDerivadas 4.0 Internacional (CC BY-NC-ND 4.0).
                }\\
    
                Eres libre de compartir y redistribuir el contenido de esta obra en cualquier medio o formato, siempre y cuando des el crédito adecuado a los autores originales y no persigas fines comerciales. 
            \end{minipage}
        };
    \end{tikzpicture}
    
    
    
    % 1. Foto de fondo
    % 2. Título
    % 3. Encabezado Izquierdo
    % 4. Color de fondo
    % 5. Coord x del titulo
    % 6. Coord y del titulo
    % 7. Fecha
    % 8. Autor


}


\newcommand{\portadaBook}[8]{

    % 1. Foto de fondo
    % 2. Título
    % 3. Encabezado Izquierdo
    % 4. Color de fondo
    % 5. Coord x del titulo
    % 6. Coord y del titulo
    % 7. Fecha
    % 8. Autor

    % Personaliza el formato del título
    \pretitle{\begin{center}\bfseries\fontsize{42}{56}\selectfont}
    \posttitle{\par\end{center}\vspace{2em}}
    
    % Personaliza el formato del autor
    \preauthor{\begin{center}\Large}
    \postauthor{\par\end{center}\vfill}
    
    % Personaliza el formato de la fecha
    \predate{\begin{center}\huge}
    \postdate{\par\end{center}\vspace{2em}}
    
    \title{#2}
    \author{\href{https://losdeldgiim.github.io/}{Los Del DGIIM, \texttt{\large losdeldgiim.github.io}}
    \\ \vspace{0.5cm}#8}
    \date{Granada, #7}
    \maketitle
    
    \tableofcontents
}




\newcommand{\portadaArticle}[8]{

    % 1. Foto de fondo
    % 2. Título
    % 3. Encabezado Izquierdo
    % 4. Color de fondo
    % 5. Coord x del titulo
    % 6. Coord y del titulo
    % 7. Fecha
    % 8. Autor

    % Personaliza el formato del título
    \pretitle{\begin{center}\bfseries\fontsize{42}{56}\selectfont}
    \posttitle{\par\end{center}\vspace{2em}}
    
    % Personaliza el formato del autor
    \preauthor{\begin{center}\Large}
    \postauthor{\par\end{center}\vspace{3em}}
    
    % Personaliza el formato de la fecha
    \predate{\begin{center}\huge}
    \postdate{\par\end{center}\vspace{5em}}
    
    \title{#2}
    \author{\href{https://losdeldgiim.github.io/}{Los Del DGIIM, \texttt{\large losdeldgiim.github.io}}
    \\ \vspace{0.5cm}#8}
    \date{Granada, #7}
    \thispagestyle{empty}               % Sin encabezado ni pie de página
    \maketitle
    \vfill
}
    \portadaExamen{ffccA4.jpg}{Cálculo II\\Examen XVI}{Cálculo II. Examen XVI}{MidnightBlue}{-8}{28}{2024}{Roberto González Lugo}

    \begin{description}
        \item[Asignatura] Cálculo II.
        \item[Curso Académico] 2024-25.
        \item[Grado] Doble Grado en Ingeniería Informática y Matemáticas.
        \item[Grupo] Único.
        \item[Profesor] José Luis Gámez Ruiz.
        \item[Fecha] 4 de junio de 2025.
        \item[Descripción] Convocatoria Ordinaria.
    \end{description}
    \newpage

    \begin{ejercicio}[2 puntos]
        Tema a desarrollar: Teorema del Valor Medio y consecuencias sobre el crecimiento.
    \end{ejercicio}

    \begin{ejercicio}[2 puntos]\ 
        \begin{enumerate}[label=\alph*)]
            \item Sea $f:\mathbb{R}\setminus \{0\}\longrightarrow \mathbb{R} $ definida por $f(x) = \arctan(x) + \arctan\left(\frac{1}{x}\right)$. Calcula la imagen de $f$.
            \item Sea $a\neq 0$ y sea $g:\mathbb{R}\setminus \left\{\frac{1}{a}\right\}\longrightarrow \mathbb{R}$ definida por
                \begin{equation*}
                    g(x) = \arctan(a) + \arctan(x) - \arctan\left(\dfrac{a+x}{1-ax}\right)
                \end{equation*}
                Calcula la imagen de $g$.
        \end{enumerate}
    \end{ejercicio}

    \begin{ejercicio}[2 puntos]
        Sea $f:\mathbb{R}^+_0\longrightarrow\mathbb{R}^+_0$ continua con $f(\mathbb{R}^+) \subseteq \mathbb{R}^+$ cumpliendo:
        \begin{equation*}
            f(x)^2 = 2\int_{0}^{x} f(t)~dt \qquad \forall x\geq 0
        \end{equation*}
        \begin{enumerate}[label=\alph*)]
            \item Demuestra que $f$ es derivable en $\mathbb{R}^+$ y calcula su derivada.
            \item Determina qué función (o funciones) cumplen la condición anterior.
        \end{enumerate}
    \end{ejercicio}

    \begin{ejercicio}[2 puntos]
        Calcula el área del recinto interior delimitado por la elipse de fórmula $\frac{x^2}{a^2}+\frac{y^2}{b^2} = 1$ $(a,b>0)$, así como el volumen del sólido de revolución generado por rotar dicha elipse alrededor del eje $OX$.
    \end{ejercicio}

    \begin{ejercicio}[2 puntos]
        Sea $F:\mathbb{R}^+_0\longrightarrow \mathbb{R}$ definida por:
        \begin{equation*}
            F(x) = \int_{2}^{x^2+2} te^{-t}~dt 
        \end{equation*}
        \begin{enumerate}[label=\alph*)]
            \item Calcula $\lim\limits_{x\to0^+}\dfrac{F(x)}{x\log(1+x)}$
            \item Estudia el crecimiento de $F$ y calcula su imagen.
        \end{enumerate}
    \end{ejercicio}
    
    
    
    \newpage


\setcounter{ejercicio}{0}
\begin{ejercicio}
    Tema a desarrollar: Teorema del Valor Medio y consecuencias sobre el crecimiento.


    \begin{teo}[del Valor Medio] Sea \( g: [a, b] \to \mathbb{R} \), continua en \( [a, b] \) y derivable en \( ]a, b[ \).  
    Entonces, existe \( c \in ]a, b[ \) tal que
    \[
    g(b) - g(a) = g'(c)(b - a)
    \]
    \begin{proof}Sea la recta que pasa por \( (a, g(a)) \) y \( (b, g(b)) \).
    \[
    r(x) = g(a) + \frac{g(b) - g(a)}{b - a}(x - a)
        \]

    Definimos
    \[
    h(x) = g(x) - r(x)
    \]
    \begin{itemize}
        \item \( g \) es continua en \( [a, b] \) y derivable en \( ]a, b[ \)
        \item \( r \) es continua y derivable en todo \( \mathbb{R} \)
        \item Entonces, \( h \) es continua en \( [a, b] \) y derivable en \( ]a, b[ \)
    \end{itemize}

    Además:
    \[
    h(a) = g(a) - r(a) = 0, \quad h(b) = g(b) - r(b) = 0 \Rightarrow h(a) = h(b)
    \]

    Hacemos ahora uso del Teorema de Rolle:
    \begin{teo}[Rolle]  Sea \( f: [a, b] \to \mathbb{R} \), continua en \( [a, b] \), derivable en \( ]a, b[ \), y tal que \( f(a) = f(b) \).  
    Entonces, existe \( c \in ]a, b[ \) tal que \( f'(c) = 0 \).
    \end{teo}

    Aplicamos el Teorema de Rolle a \( h\), por lo que \( \exists\, c \in ]a, b[ \) tal que:
    \[
    h'(c) = 0 \Rightarrow g'(c) - r'(c) = 0 \Rightarrow g'(c) = \frac{g(b) - g(a)}{b - a}
    \]
    \end{proof}
    \end{teo}
    \textbf{Consecuencia.} Sea \( I \) un intervalo no trivial y \( f \in C(I) \cap D(I^\circ) \). Entonces:
    \[
    f'(x) = 0 \quad \forall x \in I^\circ \quad \Longleftrightarrow \quad f \text{ es constante}
    \]

    La implicación hacia la izquierda es obvia, la interesante es la otra, pues nos dice que una función derivable en un intervalo queda determinada cuando conocemos su función derivada, salvo una constante aditiva. En efecto, si \( I \) es un intervalo no trivial y \( f, g \in D(I) \) verifican que \( f' = g' \), entonces \( g - f \) es constante: existe \( \lambda \in \mathbb{R} \) tal que \( g(x) = f(x) + \lambda \) para todo \( x \in I \).

    \begin{coro}[Criterio de crecimiento] Sea \( I \) un intervalo y \( f \in C(I) \cap D(I^\circ) \).
    \begin{itemize}
        \item Si \( f'(x) > 0 \ \forall x \in I^\circ \), entonces \( f \) es estrictamente creciente en \( I \).
        \item Si \( f'(x) < 0 \ \forall x \in I^\circ \), entonces \( f \) es estrictamente decreciente en \( I \).
        \item Si \( f'(x) = 0 \ \forall x \in I^\circ \), entonces \( f \) es constante en \( I \).
    \end{itemize}
    \end{coro}
    
\end{ejercicio}
\newpage
\begin{ejercicio}\
\begin{enumerate}[label=\alph*)]
    \item Sea $f:\mathbb{R}\setminus \{0\}\longrightarrow \mathbb{R} $ definida por $f(x) = \arctan(x) + \arctan\left(\frac{1}{x}\right)$. Calcula la imagen de $f$.
           
    
    En primer lugar, calculamos $$f'(x) =  \frac1{1+x^2} + \left(\frac{-1}{x^2+1}\right) = 0, \forall x \in \mathbb{R}\setminus\{0\}$$
    Esto nos permite saber que $f(x)$ es constante para todo $x\neq 0$ \newline
    Veamos qué ocurre a la izquierda y derecha del 0, utilizando valores conocidos de la arctan:\begin{itemize}
        \item $\lim\limits_{x\to -\infty}\arctan(x)=\nicefrac{-\pi}{2}$
        \item $\lim\limits_{x\to +\infty}\arctan(x)=\nicefrac{\pi}{2}$
        \item $\arctan(0)=0$
    
    \end{itemize} 
    Luego: \begin{itemize}
        \item \textbf{Izquierda}$\lim\limits_{x\to -\infty} f(x)=\nicefrac{-\pi}{2} + 0 = \nicefrac{-\pi}{2}$
        \item \textbf{Derecha}$\lim\limits_{x\to +\infty} f(x)=\nicefrac{\pi}{2} + 0 = \nicefrac{\pi}{2}$
        \item \textbf{En 0} La función NO está definida en $x=0$ (Condición inicial del enunciado)
    \end{itemize}
    Por tanto, la imagen es la siguiente, y se puede observar en la Figura~\ref{fig:imagen-f}:
    $$Im(f) = \left\{\nicefrac{-\pi}{2}, \nicefrac{\pi}{2}\right\}$$
    \begin{figure}
        \centering
        \begin{tikzpicture}
            \begin{axis}[
                axis lines=middle,
                xlabel={$x$},
                ylabel={$f(x)$},
                ymin=-3, ymax=3,
                xmin=-8, xmax=8,
                samples=1000,
                domain=-8:8,
                restrict y to domain=-10:10,
                enlargelimits=true,
                grid=both,
                width=14cm,
                height=8cm,
                legend pos=south east
            ]
            
            % Lado izquierdo de la discontinuidad
            \addplot [very thick, green!60!black, domain=-10:-0.01] {rad(atan(x) + atan(1/x))};
            
            % Lado derecho de la discontinuidad
            \addplot [very thick, green!60!black, domain=0.01:10] {rad(atan(x) + atan(1/x))};
            
            \end{axis}
        \end{tikzpicture}
        \caption{Imagen de la función $f(x) = \arctan(x) + \arctan\left(\frac{1}{x}\right)$}
        \label{fig:imagen-f}
    \end{figure}
    
     \item Sea $a\neq 0$ y sea $g:\mathbb{R}\setminus \left\{\frac{1}{a}\right\}\longrightarrow \mathbb{R}$ definida por
                \begin{equation*}
                    g(x) = \arctan(a) + \arctan(x) - \arctan\left(\dfrac{a+x}{1-ax}\right)
                \end{equation*}
                Calcula la imagen de $g$.\\

    Siguiendo el mismo procedimiento, calculamos:
    \begin{multline*}
        g'(x) = \frac1{1+x^2}-\frac{1}{1 + \left( \frac{a + x}{1 - ax} \right)^2} \cdot
        \frac{(1 - ax) - (a + x)(-a)}{(1 - ax)^2} = \frac1{1+x^2}-\frac1{1+x^2} = 0,
        \\\forall x \in \mathbb{R}\setminus\{\nicefrac{1}{a}\}, \forall a \neq0
    \end{multline*}
Luego la función es constante en $]-\infty,\nicefrac{1}{a}[ \text{ y en } ]\nicefrac{1}{a}, +\infty[$
\newline
Estudiamos los valores a la izquierda y a la derecha:
\begin{enumerate}
    \item \textbf{Izquierda: } $\lim\limits_{x\to -\infty}{g(x)}=\arctan(a)  - \nicefrac{\pi}{2} - \arctan(\nicefrac{-1}{a})$ \newline Y sabiendo que $\arctan$ es una función impar $(f(-x) = -f(x))$: $$\lim\limits_{x\to -\infty}{g(x)}=\arctan(a)+\arctan(\nicefrac{1}{a}) - \frac \pi2 = f(a) - \frac \pi2$$
    \item \textbf{Derecha: } $\lim\limits_{x\to -\infty}{g(x)}=\arctan(a) + \nicefrac{\pi}{2} - \arctan(\nicefrac{-1}{a}) \overset{\text{Análogamente}}{=} f(a) +\frac \pi2$ 
\end{enumerate}
Por último, diferenciamos casos:
\begin{enumerate}
    \item Si $a<0 \implies f(a) = \nicefrac{-\pi}{2} \implies Im(g) = \{-\pi,0\}$
    \item si $a>0 \implies f(a) = \nicefrac{\pi}{2} \implies Im(g) = \{0, \pi\}$
\end{enumerate}
\end{enumerate}
\end{ejercicio}



    \begin{ejercicio}
        Sea $f:\mathbb{R}^+_0\longrightarrow\mathbb{R}^+_0$ continua con $f(\mathbb{R}^+) \subseteq \mathbb{R}^+$ cumpliendo:
        \begin{equation*}
            f(x)^2 = 2\int_{0}^{x} f(t)~dt \qquad \forall x\geq 0
        \end{equation*}
        \begin{enumerate}[label=\alph*)]
            \item Demuestra que $f$ es derivable en $\mathbb{R}^+$ y calcula su derivada.
            \item Determina qué función (o funciones) cumplen la condición anterior.
            En primer lugar, observemos que $$f(x) = \sqrt{2\int_0^xf(t)dt} \quad \forall x \geq0$$
            Es una raíz (que es derivable si su interior es no negativo y derivable) de una constante positiva mutliplicada por una función no negativa, ya que es una integral de otra función no negativa (por la condición del enunciado)
            Visto esto, veamos:
            $$\frac{f(x)^2}{2} = \int_0^xf(t)dt \quad \forall x \geq0 $$ $$\text{Sea: } F(x)= \frac{f(x)^2}{2}$$
            Por el TFC: $F'(x) = f(x) \iff f(x) f'(x) = f(x)$
            
            Usamos ahora que $f:\mathbb{R}^+_0\longrightarrow\mathbb{R}^+_0$ y separamos:
            \begin{enumerate}
                \item Si $f(x) > 0 \implies f'(x) = 1 \implies f(x) = Id(x)+c \quad c\in \mathbb{R}$
                \item Si $f(x) = 0 \implies \int_0^xf(t)dt = 0$ nos da dos opciones \begin{enumerate}
                    \item $f(x)$ es constantemente 0
                    \item $f(x)$ no es constantemente 0, luego $x=0$
                \end{enumerate}
                
            \end{enumerate}
             Sabemos que $f(0) = 0 $, luego:
            \newline juntando todo lo anterior, $f(x) = Id(x), \quad  \forall x \in \mathbb{R}_0^+$

            En resumen, hemos comprobado que $f(x)$ es constantemente $0$, o bien la identidad. Pero el enunciado establece que la imagen de $f$ está contenida en $\mathbb{R}^+$, luego \fbox{$f(x)=x \quad , \forall x \in \mathbb{R}_0^+$}
            

        \end{enumerate}
    \end{ejercicio}

 \begin{ejercicio}
    Calcula el área del recinto interior delimitado por la elipse de fórmula $\frac{x^2}{a^2}+\frac{y^2}{b^2} = 1$ $(a,b>0)$, así como el volumen del sólido de revolución generado por rotar dicha elipse alrededor del eje $OX$.
    
    \begin{enumerate}[label=\alph*)]
        \item \textbf{Área encerrada por la elipse}

        Despejamos $y$ de la ecuación de la elipse:
        \[
            y = b \sqrt{1 - \frac{x^2}{a^2}}
        \]
        El área total del recinto delimitado por la elipse es:
        \[
            A = 2 \int_{-a}^{a} b \sqrt{1 - \frac{x^2}{a^2}} \, dx
        \]
        Sacamos constantes y utilizamos la simetría:
        \[
            A = 4b \int_{0}^{a} \sqrt{1 - \frac{x^2}{a^2}} \, dx
        \]
        Usamos el cambio de variable \( x = a \sin\theta \), entonces \( dx = a \cos\theta d\theta \), y los límites cambian:
        \[
            x = 0 \Rightarrow \theta = 0, \quad x = a \Rightarrow \theta = \frac{\pi}{2}
        \]
        La integral queda:
        \[
            A = 4b \int_{0}^{\frac{\pi}{2}} \sqrt{1 - \sin^2\theta} \cdot a \cos\theta \, d\theta = 4ab \int_{0}^{\frac{\pi}{2}} \cos^2\theta \, d\theta
        \]
        Usamos la identidad: \( \cos^2\theta = \frac{1 + \cos(2\theta)}{2} \)
        \[
            A = 4ab \cdot \int_{0}^{\frac{\pi}{2}} \frac{1 + \cos(2\theta)}{2} \, d\theta = 4ab \cdot \frac{1}{2} \left[\theta + \frac{\sin(2\theta)}{2} \right]_0^{\frac{\pi}{2}} = 2ab \cdot \left( \frac{\pi}{2} \right) = \boxed{\pi ab}
        \]

        \item \textbf{Volumen del sólido de revolución generado al rotar la elipse respecto del eje $OX$}

        Usamos el método de discos sobre la parte superior de la elipse:
        \[
            V = \pi \int_{-a}^{a} \left( b \sqrt{1 - \frac{x^2}{a^2}} \right)^2 dx = \pi b^2 \int_{-a}^{a} \left(1 - \frac{x^2}{a^2} \right) dx
        \]
        Como la función es par:
        \[
            V = 2\pi b^2 \int_{0}^{a} \left(1 - \frac{x^2}{a^2} \right) dx = 2\pi b^2 \left[ x - \frac{x^3}{3a^2} \right]_0^a = 2\pi b^2 \left(a - \frac{a}{3} \right) = \boxed{\frac{4\pi a b^2}{3}}
        \]
    \end{enumerate}
\end{ejercicio}


\begin{ejercicio}
        Sea $F:\mathbb{R}^+_0\longrightarrow \mathbb{R}$ definida por:
        \begin{equation*}
            F(x) = \int_{2}^{x^2+2} te^{-t}~dt 
        \end{equation*}
        \begin{enumerate}[label=\alph*)]
            \item Calcula $\lim\limits_{x\to0^+}\dfrac{F(x)}{x\log(1+x)}$

            Vemos que 
            $\lim\limits_{x\to0^+}\dfrac{F(x)}{x\log(1+x)} = \frac{\int_{2}^{2} te^{-t}~dt}{0} = \frac 00 $
            ¿Podemos usar L'Hôpital? Sí, puesto que el numerador y el denominador son funciones continuas, derivables y ambos tienden a 0\newline \newline
            Sea $g(x)=x\log(1+x):$
            \begin{enumerate}
                \item $F'(x) \overset{TFC \ y \ regla \ cadena}{=} 2x(x^2 + 2)e^{-(x^2 + 2)}$
                \item $g(x)'\overset{Regla\ producto} = \log(1 + x) + \frac{x}{1 + x}$
            \end{enumerate}
            Y vemos ahora el nuevo limite:
            $$\lim_{x \to 0^+} \frac{2x(x^2 + 2)e^{-(x^2 + 2)}}{\log(1 + x) + \frac{x}{1 + x}} = \frac00$$ Volvemos a aplicar L'Hôpital
            \begin{enumerate}
                \item Calculamos $F''(x)$:
                \begin{align*}
                    F''(x)&=(6x^2+4)e^{-(x^2+2)}+e^{-(x^2+2)}\cdot -2x \cdot (2x^3+4x)=\\&= e^{-x^2-2}[(6x^2+4)+(-4x^4-8x^2)]
                \end{align*}
                \item $g''(x) = \frac1{1+x}+\frac1{(1+x)^2}$
            \end{enumerate}
            Estudiamos ahora este nuevo límite:
            $$\lim\limits_{x\to0^+}\frac{e^{-x^2-2}[(6x^2+4)+(-4x^4-8x^2)]}{\frac1{1+x}+\frac1{(1+x)^2}} = \frac{4e^{-2}}2 = \frac2{e^2}$$
            Concluimos por ende que 
            $$\boxed{\lim\limits_{x\to0^+}\dfrac{F(x)}{g(x)} \overset{\text{L'Hôpital}}{=}\lim\limits_{x\to0^+}\dfrac{F'(x)}{g'(x)}\overset{\text{L'Hôpital}}{=}\lim\limits_{x\to0^+}\dfrac{F''(x)}{g''(x)} = \frac2{e^2}}$$
            
            \item Estudia el crecimiento de $F$ y calcula su imagen.
            \newline
            Ya sabemos que $F(x)$ es derivable así que podemos estudiar el signo de su derivada:
            $$F'(x) = 2x(x^2+2)e^{-(x^2+2)}$$
            Vemos que $F'(x)\geq 0 \forall x \geq 0 $ por ser cociente de dos términos no negativos; y además $F'(x) > 0 \forall x >0$; como $F$ está definida en $\mathbb{R}_0^+$, y sabemos que $F(0) = 0$ (integral con limites de integración iguales) Podemos afirmar que $F(x)$ es estrictamente creciente en $\mathbb{R}^+$ y $F(0)=0$.\newline Solo falta ver el $\lim\limits_{x\to+\infty} F(x)$:

            $$\lim\limits_{x\to+\infty} F(x) = \lim\limits_{x\to+\infty} \int_{2}^{x^2+2} te^{-t}~dt = \int_{2}^{\infty} te^{-t}~dt$$
            Podemos calcular esta integral impropia aplicando los cambios:
            \begin{itemize}
                \item $u=t$
                \item $du=1$
                \item $dv=e^{-t}$
                \item $v=e^{-t}$
            \end{itemize}
            $$\int udv = uv-\int vdu$$
            Luego, aplicando:
            $$\int_{2}^{\infty} te^{-t}~dt = te^{-t}]_2^\infty - \int_{2}^{\infty} e^{-t}dt =te^{-t}-e^{-t}]_2^\infty =\lim\limits_{x\to\infty}\left(\frac{-t}{e^t}-\frac1{e^t}\right)-\left(\frac{-2}{e^2}-\frac1{e^2}\right) \AstIg \boxed{\frac{3}{e^2}}$$
            Donde en $(\ast)$ hemos usado que $\lim\limits_{x\to\infty}(\frac{-t}{e^t}-\frac1{e^t}) = 0$ por la escala de infinitos\footnote{Si no se ve claro, esto se puede comprobar fácilmente por L'Hôpital}.\\
            \centering\fbox{En conclusión $Im(F) = [0,\frac{3}{e^2}[$}
                        
        \end{enumerate}
    \end{ejercicio}

    


\end{document}
