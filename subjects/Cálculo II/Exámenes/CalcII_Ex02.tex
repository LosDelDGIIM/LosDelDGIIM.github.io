\documentclass[12pt]{article}

% Idioma y codificación
\usepackage[spanish, es-tabla]{babel}       %es-tabla para que se titule "Tabla"
\usepackage[utf8]{inputenc}

% Márgenes
\usepackage[a4paper,top=3cm,bottom=2.5cm,left=3cm,right=3cm]{geometry}

% Comentarios de bloque
\usepackage{verbatim}

% Paquetes de links
\usepackage[hidelinks]{hyperref}    % Permite enlaces
\usepackage{url}                    % redirecciona a la web

% Más opciones para enumeraciones
\usepackage{enumitem}

% Personalizar la portada
\usepackage{titling}

% Paquetes de tablas
\usepackage{multirow}


%------------------------------------------------------------------------

%Paquetes de figuras
\usepackage{caption}
\usepackage{subcaption} % Figuras al lado de otras
\usepackage{float}      % Poner figuras en el sitio indicado H.


% Paquetes de imágenes
\usepackage{graphicx}       % Paquete para añadir imágenes
\usepackage{transparent}    % Para manejar la opacidad de las figuras

% Paquete para usar colores
\usepackage[dvipsnames]{xcolor}
\usepackage{pagecolor}      % Para cambiar el color de la página

% Habilita tamaños de fuente mayores
\usepackage{fix-cm}

% Para los gráficos
\usepackage{tikz}

% Para poder situar los nodos en los grafos
\usetikzlibrary{positioning}


%------------------------------------------------------------------------

% Paquetes de matemáticas
\usepackage{mathtools, amsfonts, amssymb, mathrsfs}
\usepackage[makeroom]{cancel}     % Simplificar tachando
\usepackage{polynom}    % Divisiones y Ruffini
\usepackage{units} % Para poner fracciones diagonales con \nicefrac

\usepackage{pgfplots}   %Representar funciones
\pgfplotsset{compat=1.18}  % Versión 1.18

\usepackage{tikz-cd}    % Para usar diagramas de composiciones
\usetikzlibrary{calc}   % Para usar cálculo de coordenadas en tikz

%Definición de teoremas, etc.
\usepackage{amsthm}
%\swapnumbers   % Intercambia la posición del texto y de la numeración

\theoremstyle{plain}

\makeatletter
\@ifclassloaded{article}{
  \newtheorem{teo}{Teorema}[section]
}{
  \newtheorem{teo}{Teorema}[chapter]  % Se resetea en cada chapter
}
\makeatother

\newtheorem{coro}{Corolario}[teo]           % Se resetea en cada teorema
\newtheorem{prop}[teo]{Proposición}         % Usa el mismo contador que teorema
\newtheorem{lema}[teo]{Lema}                % Usa el mismo contador que teorema

\theoremstyle{remark}
\newtheorem*{observacion}{Observación}

\theoremstyle{definition}

\makeatletter
\@ifclassloaded{article}{
  \newtheorem{definicion}{Definición} [section]     % Se resetea en cada chapter
}{
  \newtheorem{definicion}{Definición} [chapter]     % Se resetea en cada chapter
}
\makeatother

\newtheorem*{notacion}{Notación}
\newtheorem*{ejemplo}{Ejemplo}
\newtheorem*{ejercicio*}{Ejercicio}             % No numerado
\newtheorem{ejercicio}{Ejercicio} [section]     % Se resetea en cada section


% Modificar el formato de la numeración del teorema "ejercicio"
\renewcommand{\theejercicio}{%
  \ifnum\value{section}=0 % Si no se ha iniciado ninguna sección
    \arabic{ejercicio}% Solo mostrar el número de ejercicio
  \else
    \thesection.\arabic{ejercicio}% Mostrar número de sección y número de ejercicio
  \fi
}


% \renewcommand\qedsymbol{$\blacksquare$}         % Cambiar símbolo QED
%------------------------------------------------------------------------

% Paquetes para encabezados
\usepackage{fancyhdr}
\pagestyle{fancy}
\fancyhf{}

\newcommand{\helv}{ % Modificación tamaño de letra
\fontfamily{}\fontsize{12}{12}\selectfont}
\setlength{\headheight}{15pt} % Amplía el tamaño del índice


%\usepackage{lastpage}   % Referenciar última pag   \pageref{LastPage}
\fancyfoot[C]{\thepage}

%------------------------------------------------------------------------

% Conseguir que no ponga "Capítulo 1". Sino solo "1."
\makeatletter
\@ifclassloaded{book}{
  \renewcommand{\chaptermark}[1]{\markboth{\thechapter.\ #1}{}} % En el encabezado
    
  \renewcommand{\@makechapterhead}[1]{%
  \vspace*{50\p@}%
  {\parindent \z@ \raggedright \normalfont
    \ifnum \c@secnumdepth >\m@ne
      \huge\bfseries \thechapter.\hspace{1em}\ignorespaces
    \fi
    \interlinepenalty\@M
    \Huge \bfseries #1\par\nobreak
    \vskip 40\p@
  }}
}
\makeatother

%------------------------------------------------------------------------
% Paquetes de cógido
\usepackage{minted}
\renewcommand\listingscaption{Código fuente}

\usepackage{fancyvrb}
% Personaliza el tamaño de los números de línea
\renewcommand{\theFancyVerbLine}{\small\arabic{FancyVerbLine}}

% Estilo para C++
\newminted{cpp}{
    frame=lines,
    framesep=2mm,
    baselinestretch=1.2,
    linenos,
    escapeinside=||
}

% para minted
\definecolor{LightGray}{rgb}{0.95,0.95,0.92}
\setminted{
    linenos=true,
    stepnumber=5,
    numberfirstline=true,
    autogobble,
    breaklines=true,
    breakautoindent=true,
    breaksymbolleft=,
    breaksymbolright=,
    breaksymbolindentleft=0pt,
    breaksymbolindentright=0pt,
    breaksymbolsepleft=0pt,
    breaksymbolsepright=0pt,
    fontsize=\footnotesize,
    bgcolor=LightGray,
    numbersep=10pt
}


\usepackage{listings} % Para incluir código desde un archivo

\renewcommand\lstlistingname{Código Fuente}
\renewcommand\lstlistlistingname{Índice de Códigos Fuente}

% Definir colores
\definecolor{vscodepurple}{rgb}{0.5,0,0.5}
\definecolor{vscodeblue}{rgb}{0,0,0.8}
\definecolor{vscodegreen}{rgb}{0,0.5,0}
\definecolor{vscodegray}{rgb}{0.5,0.5,0.5}
\definecolor{vscodebackground}{rgb}{0.97,0.97,0.97}
\definecolor{vscodelightgray}{rgb}{0.9,0.9,0.9}

% Configuración para el estilo de C similar a VSCode
\lstdefinestyle{vscode_C}{
  backgroundcolor=\color{vscodebackground},
  commentstyle=\color{vscodegreen},
  keywordstyle=\color{vscodeblue},
  numberstyle=\tiny\color{vscodegray},
  stringstyle=\color{vscodepurple},
  basicstyle=\scriptsize\ttfamily,
  breakatwhitespace=false,
  breaklines=true,
  captionpos=b,
  keepspaces=true,
  numbers=left,
  numbersep=5pt,
  showspaces=false,
  showstringspaces=false,
  showtabs=false,
  tabsize=2,
  frame=tb,
  framerule=0pt,
  aboveskip=10pt,
  belowskip=10pt,
  xleftmargin=10pt,
  xrightmargin=10pt,
  framexleftmargin=10pt,
  framexrightmargin=10pt,
  framesep=0pt,
  rulecolor=\color{vscodelightgray},
  backgroundcolor=\color{vscodebackground},
}

%------------------------------------------------------------------------

% Comandos definidos
\newcommand{\bb}[1]{\mathbb{#1}}
\newcommand{\cc}[1]{\mathcal{#1}}

% I prefer the slanted \leq
\let\oldleq\leq % save them in case they're every wanted
\let\oldgeq\geq
\renewcommand{\leq}{\leqslant}
\renewcommand{\geq}{\geqslant}

% Si y solo si
\newcommand{\sii}{\iff}

% Letras griegas
\newcommand{\eps}{\epsilon}
\newcommand{\veps}{\varepsilon}
\newcommand{\lm}{\lambda}

\newcommand{\ol}{\overline}
\newcommand{\ul}{\underline}
\newcommand{\wt}{\widetilde}
\newcommand{\wh}{\widehat}

\let\oldvec\vec
\renewcommand{\vec}{\overrightarrow}

% Derivadas parciales
\newcommand{\del}[2]{\frac{\partial #1}{\partial #2}}
\newcommand{\Del}[3]{\frac{\partial^{#1} #2}{\partial #3^{#1}}}
\newcommand{\deld}[2]{\dfrac{\partial #1}{\partial #2}}
\newcommand{\Deld}[3]{\dfrac{\partial^{#1} #2}{\partial #3^{#1}}}


\newcommand{\AstIg}{\stackrel{(\ast)}{=}}
\newcommand{\Hop}{\stackrel{L'H\hat{o}pital}{=}}

\newcommand{\red}[1]{{\color{red}#1}} % Para integrales, destacar los cambios.

% Método de integración
\newcommand{\MetInt}[2]{
    \left[\begin{array}{c}
        #1 \\ #2
    \end{array}\right]
}

% Declarar aplicaciones
% 1. Nombre aplicación
% 2. Dominio
% 3. Codominio
% 4. Variable
% 5. Imagen de la variable
\newcommand{\Func}[5]{
    \begin{equation*}
        \begin{array}{rrll}
            #1:& #2 & \longrightarrow & #3\\
               & #4 & \longmapsto & #5
        \end{array}
    \end{equation*}
}

%------------------------------------------------------------------------



\begin{document}

    % 1. Foto de fondo
    % 2. Título
    % 3. Encabezado Izquierdo
    % 4. Color de fondo
    % 5. Coord x del titulo
    % 6. Coord y del titulo
    % 7. Fecha

    
    % 1. Foto de fondo
% 2. Título
% 3. Encabezado Izquierdo
% 4. Color de fondo
% 5. Coord x del titulo
% 6. Coord y del titulo
% 7. Fecha

\newcommand{\portada}[7]{

    \portadaBase{#1}{#2}{#3}{#4}{#5}{#6}{#7}
    \portadaBook{#1}{#2}{#3}{#4}{#5}{#6}{#7}
}

\newcommand{\portadaExamen}[7]{

    \portadaBase{#1}{#2}{#3}{#4}{#5}{#6}{#7}
    \portadaArticle{#1}{#2}{#3}{#4}{#5}{#6}{#7}
}




\newcommand{\portadaBase}[7]{

    % Tiene la portada principal y la licencia Creative Commons
    
    % 1. Foto de fondo
    % 2. Título
    % 3. Encabezado Izquierdo
    % 4. Color de fondo
    % 5. Coord x del titulo
    % 6. Coord y del titulo
    % 7. Fecha
    
    
    \thispagestyle{empty}               % Sin encabezado ni pie de página
    \newgeometry{margin=0cm}        % Márgenes nulos para la primera página
    
    
    % Encabezado
    \fancyhead[L]{\helv #3}
    \fancyhead[R]{\helv \nouppercase{\leftmark}}
    
    
    \pagecolor{#4}        % Color de fondo para la portada
    
    \begin{figure}[p]
        \centering
        \transparent{0.3}           % Opacidad del 30% para la imagen
        
        \includegraphics[width=\paperwidth, keepaspectratio]{assets/#1}
    
        \begin{tikzpicture}[remember picture, overlay]
            \node[anchor=north west, text=white, opacity=1, font=\fontsize{60}{90}\selectfont\bfseries\sffamily, align=left] at (#5, #6) {#2};
            
            \node[anchor=south east, text=white, opacity=1, font=\fontsize{12}{18}\selectfont\sffamily, align=right] at (9.7, 3) {\textbf{\href{https://losdeldgiim.github.io/}{Los Del DGIIM}}};
            
            \node[anchor=south east, text=white, opacity=1, font=\fontsize{12}{15}\selectfont\sffamily, align=right] at (9.7, 1.8) {Doble Grado en Ingeniería Informática y Matemáticas\\Universidad de Granada};
        \end{tikzpicture}
    \end{figure}
    
    
    \restoregeometry        % Restaurar márgenes normales para las páginas subsiguientes
    \pagecolor{white}       % Restaurar el color de página
    
    
    \newpage
    \thispagestyle{empty}               % Sin encabezado ni pie de página
    \begin{tikzpicture}[remember picture, overlay]
        \node[anchor=south west, inner sep=3cm] at (current page.south west) {
            \begin{minipage}{0.5\paperwidth}
                \href{https://creativecommons.org/licenses/by-nc-nd/4.0/}{
                    \includegraphics[height=2cm]{assets/Licencia.png}
                }\vspace{1cm}\\
                Esta obra está bajo una
                \href{https://creativecommons.org/licenses/by-nc-nd/4.0/}{
                    Licencia Creative Commons Atribución-NoComercial-SinDerivadas 4.0 Internacional (CC BY-NC-ND 4.0).
                }\\
    
                Eres libre de compartir y redistribuir el contenido de esta obra en cualquier medio o formato, siempre y cuando des el crédito adecuado a los autores originales y no persigas fines comerciales. 
            \end{minipage}
        };
    \end{tikzpicture}
    
    
    
    % 1. Foto de fondo
    % 2. Título
    % 3. Encabezado Izquierdo
    % 4. Color de fondo
    % 5. Coord x del titulo
    % 6. Coord y del titulo
    % 7. Fecha


}


\newcommand{\portadaBook}[7]{

    % 1. Foto de fondo
    % 2. Título
    % 3. Encabezado Izquierdo
    % 4. Color de fondo
    % 5. Coord x del titulo
    % 6. Coord y del titulo
    % 7. Fecha

    % Personaliza el formato del título
    \pretitle{\begin{center}\bfseries\fontsize{42}{56}\selectfont}
    \posttitle{\par\end{center}\vspace{2em}}
    
    % Personaliza el formato del autor
    \preauthor{\begin{center}\Large}
    \postauthor{\par\end{center}\vfill}
    
    % Personaliza el formato de la fecha
    \predate{\begin{center}\huge}
    \postdate{\par\end{center}\vspace{2em}}
    
    \title{#2}
    \author{\href{https://losdeldgiim.github.io/}{Los Del DGIIM}}
    \date{Granada, #7}
    \maketitle
    
    \tableofcontents
}




\newcommand{\portadaArticle}[7]{

    % 1. Foto de fondo
    % 2. Título
    % 3. Encabezado Izquierdo
    % 4. Color de fondo
    % 5. Coord x del titulo
    % 6. Coord y del titulo
    % 7. Fecha

    % Personaliza el formato del título
    \pretitle{\begin{center}\bfseries\fontsize{42}{56}\selectfont}
    \posttitle{\par\end{center}\vspace{2em}}
    
    % Personaliza el formato del autor
    \preauthor{\begin{center}\Large}
    \postauthor{\par\end{center}\vspace{3em}}
    
    % Personaliza el formato de la fecha
    \predate{\begin{center}\huge}
    \postdate{\par\end{center}\vspace{5em}}
    
    \title{#2}
    \author{\href{https://losdeldgiim.github.io/}{Los Del DGIIM}}
    \date{Granada, #7}
    \thispagestyle{empty}               % Sin encabezado ni pie de página
    \maketitle
    \vfill
}
    \portadaExamen{ffccA4.jpg}{Cálculo II\\Examen II}{Cálculo II. Examen II}{MidnightBlue}{-8}{28}{2023}{Arturo Olivares Martos}

    \begin{description}
        \item[Asignatura] Cálculo II.
        \item[Curso Académico] 2021-22.
        \item[Grado] Doble Grado en Ingeniería Informática y Matemáticas.
        \item[Grupo] Único.
        \item[Profesor] María Victoria Velasco Collado.
        \item[Descripción] Convocatoria Extraordinaria.
        \item[Fecha] 11 de julio de 2022.
        %\item[Duración] 60 minutos.
    
    \end{description}
    \newpage
    
    \begin{ejercicio}\textbf{[1.5 puntos]}
Determinar, en función de los valores de $c\in \bb{R}$, el número de soluciones que la ecuación $x^3-3x+c=0$ tiene en el intervalo $[0,1]$.

Definimos $f(x)=x^3-3x+c=0$, y buscamos cuántas raíces tiene en dicho intervalo.
\begin{equation*}
    f(0)=c \hspace{1cm} f(1)=1-3+c=-2+c
\end{equation*}

Estudiamos ahora la monotonía en dicho intervalo:
\begin{equation*}
    f'(x)=3x^2-3=0 \Longleftrightarrow x^2=1 \Longleftrightarrow x=\pm 1    
\end{equation*}
\begin{itemize}
    \item \underline{Para $x<-1$}: $f'(x)>0\Longrightarrow f$ estrictamente creciente.
    \item \underline{Para $-1<x<1$}: $f'(x)< 0\Longrightarrow f$ estrictamente decreciente.
    \item \underline{Para $x>1$}: $f'(x)>0\Longrightarrow f$ estrictamente creciente.
\end{itemize}

Por tanto, en nuestro intervalo $[0,1]$ tenemos que es continua y estrictamente decreciente.
\begin{itemize}
    \item \underline{Para $0 \leq c \leq 2$}:

    Tenemos que $f(0)\geq 0$ y $f(1)\leq 0$, por lo que por el Teorema de Bolzano hay una solución en $[0,1]$. Además, como es estrictamente decreciente, tenemos que la solución es única.

    \item \underline{Para $0 <c$}:

    Tenemos que $f(0)<0$, y al ser estrictamente decreciente, es el máximo absoluto del intervalo. Por tanto, no hay soluciones en el intervalo $[0,1]$.

    \item \underline{Para $2<c$}:

    Tenemos que $f(1)>0$, y al ser estrictamente decreciente, es el mínimo absoluto del intervalo. Por tanto, no hay soluciones en el intervalo $[0,1]$.
\end{itemize}

\begin{comment}

Por tanto, tenemos que $x=-1$ es un máximo relativo y $x=1$ es un mínimo relativo. Calculamos sus imágenes:
\begin{equation*}
    f(-1)=-1+3+c=2+c
    \qquad
    f(1)=1-3+c=-2+c
\end{equation*}

Además, tenemos que:
\begin{equation*}
    \lim_{x\to -\infty}f(x)=-\infty
    \qquad
    \lim_{x\to \infty}f(x)=\infty
\end{equation*}

Realizamos la siguiente distinción:
\begin{itemize}
    \item \underline{Para $c<-2$}:

    Tengo que $f(-1)<0$ y $x=-1$ es el máximo absoluto en la restricción de $f$ a $]-\infty, 1]$, por lo que no hay raíces en dicho intervalo. No obstante, como $f$ es estrictamente creciente para $x>1$ y diverge positivamente, tenemos que hay una raíz en dicho intervalo.

    Por tanto, para $c<-2$ tiene una única solución.

    \item \underline{Para $-2<c<2$}:

    Tengo que $f(-1)>0, f(1)<0$. Por el Teorema de Bolzano, tenemos que hay al menos una raíz en cada uno de los intervalos. Además, por ser estrictamente monótona en cada uno de ellos, solo hay una raíz en cada intervalo. Por tanto, para $-2<c<2$ hay tres soluciones.

    \item \underline{Para $2<c$}:

    Tengo que $f(-1)>0,$, por lo que hay una solución en $],-\infty, -1]$. Además, $f(1)>0$, y $x=1$ es el mínimo absoluto en la restricción de $f$ a $[1, +\infty[$, por lo que no hay raíces en dicho intervalo.

    Por tanto, para $2<c$ tiene una única solución.

    \item \underline{Para $c=-2$}:

    Tengo que $f(-1)=0$ y $x=-1$ es el máximo absoluto en la restricción de $f$ a $]-\infty, 1]$, por lo que solo hay una raíz en dicho intervalo.

    Además, $f(1)=-4$, y como $f$ es estrictamente creciente para $x>1$ y diverge positivamente, tenemos que hay una raíz en dicho intervalo.

    Por tanto, para $c=-2$ hay dos raíces.

    \item \underline{Para $c=2$}:

    Tengo que $f(1)=0$ y $x=1$ es el mínimo absoluto en la restricción de $f$ a $[1,+\infty[$, por lo que solo hay una raíz en dicho intervalo.

    Además, $f(-1)=4$, y como $f$ es estrictamente creciente para $x<-1$ y diverge negativamente, tenemos que hay una raíz en dicho intervalo.

    Por tanto, para $c=2$ hay dos raíces.
\end{itemize}

Por tanto, tenemos que:
\begin{equation*}
    \begin{array}{cc}
        |c|=2 & 2\text{ soluciones} \\
        |c|<2 & 3\text{ soluciones} \\
        |c|>2 & 1\text{ solución} \\
    \end{array}
\end{equation*}
\end{comment}

\end{ejercicio}

\begin{ejercicio}\textbf{[2 puntos]}
De todos los puntos $(a,b)$ de la circunferencia de radio 2, ¿cuáles son los que tienen la propiedad de que la recta tangente a la circunferencia por dichos puntos determina con los ejes coordenados un triángulo de área mínima?


Restrinjo mi procedimiento al primer cuadrante; es decir, a calcular la recta formada con los semiejes positivos.
\begin{figure}[H]
    \centering
    \begin{tikzpicture}[scale=1]
         % Radio de la circunferencia
        \def\r{2};
        \def\h{sqrt(2)*\r};
        \def\a{\r/sqrt(2)};
        
        \begin{axis}[
            xlabel=$x$,
            xmin=-\r-1,
            xmax=\r+3,
            ymin=-\r-1,
            ymax=\r+3,
            axis lines=middle,
            samples=100 % número de muestras para la función
        ]
        \addplot[blue,thick,domain=-1:\h+1] {-x+\h};
        
        % Coordenadas del centro de la circunferencia
        \coordinate (O) at (0,0);
        
        % Dibujo de la circunferencia
        \draw[ultra thick, red] (O) circle (\r);
        
        % Punto de tangencia
        \coordinate (T) at ({\a},{\a});

        % Base y altura
        \coordinate (B) at ({\h},{0});
        \coordinate (H) at ({0},{\h});
        
        % Puntos
        \fill (O) circle (2pt) node[below right] {$O$};
        \fill (T) circle (2pt) node[above right] {$T(a,b)$};
        \fill[red] (H) circle (2pt) node[left] {$H(0,h)$};
        \fill[red] (B) circle (2pt) node[above right] {$B(l,0)$};
        

        \addplot[fill=orange,fill opacity=0.25] coordinates {(0,0) (0,\h) (\h,0)};
        
        \end{axis}
    \end{tikzpicture}
\end{figure}

En el primer cuadrante, tenemos que la ecuación de la circunferencia es $y=~\sqrt{r^2-x^2}$. Sea la recta buscada es $t(x)=m_tx+n$. Para hallar $m_t$, empleamos la interpretación geométrica de la derivada:
\begin{equation*}
    y'(x)=\frac{-2x}{2\sqrt{r^2-x^2}} \Longrightarrow y'(a)=m_t=\frac{-a}{\sqrt{r^2-a^2}}
\end{equation*}

Para hallar $n$, tenemos que el punto $T$ pertenece a la circunferencia, por lo que:
$$b=\sqrt{r^2-a^2}$$

No obstante, tenemos que
\begin{equation*}
    t(a)=b = \frac{-a}{\sqrt{r^2-a^2}}\cdot a +n
\end{equation*}

Igualando ambas expresiones de $b$, obtenemos que:
\begin{equation*}
    \frac{-a}{\sqrt{r^2-a^2}}\cdot a +n = \sqrt{r^2-a^2} \Longrightarrow -a^2+ \sqrt{r^2-a^2}n = r^2-a^2 \Longrightarrow
    n = \frac{r^2}{\sqrt{r^2-a^2}}
\end{equation*}

Por tanto, la recta es:
\begin{equation*}
    t(x)=\frac{-a}{\sqrt{r^2-a^2}}\cdot x + \frac{r^2}{\sqrt{r^2-a^2}}
\end{equation*}

Para hallar $h,l$, como $t(0)=h$ y $0=t(l)$, tenemos que:
\begin{equation*}
    t(l)=0 \Longleftrightarrow \frac{a}{\sqrt{r^2-a^2}}\cdot l = \frac{r^2}{\sqrt{r^2-a^2}} \Longleftrightarrow al=r^2 \Longleftrightarrow l=\frac{r^2}{a} 
\end{equation*}
\begin{equation*}
    t(0)=h \Longleftrightarrow h = \frac{r^2}{\sqrt{r^2-a}}
\end{equation*}

La función a minimizar es, por tanto:
\begin{equation*}
    \begin{array}{rl}
        A:\left]0, r\right[ & \longrightarrow \bb{R}\\
                a & \longrightarrow A(a) = \displaystyle \frac{1}{2}lh = \frac{r^4}{2a\sqrt{r^2-a^2}} = \frac{1}{2}\sqrt{\frac{r^8}{a^2(r^2-a^2)}}
                = \frac{1}{2}\sqrt{\frac{r^8}{-a^4 +a^2r^2}}
    \end{array}
\end{equation*}

Como $A(a)\geq 0$, tenemos que minimizar $A$ equivale a minimizar $A^2$. Por tanto,
\begin{equation*}
    (A^2)'(a)=-\frac{r^8(-4a^3+2ar^2)}{4a^4(r^2-a^2)^2} = 0 \Longleftrightarrow 4a^3=2ar^2 \Longleftrightarrow a=\sqrt{\frac{r^2}{2}} = \frac{r}{\sqrt{2}}
\end{equation*}

Comprobemos que se trata de un mínimo relativo:
\begin{itemize}
    \item \underline{Para $0<a<\frac{r}{\sqrt{2}}$}:
    \begin{equation*}
        (A^2)'(a)<0 \Longleftrightarrow -4a^3+2ar^2>0 \Longleftrightarrow 2ar^2>4a^3
        \Longleftrightarrow r^2>2a^2
        \Longleftrightarrow r>\sqrt{2}a 
        \Longleftrightarrow \frac{r}{\sqrt{2}}>a 
    \end{equation*}

    Por tanto, tenemos que es estrictamente decreciente en este intervalo.

    \item \underline{Para $a>\frac{r}{\sqrt{2}}$}:
    \begin{equation*}
        (A^2)'(a)>0 \Longleftrightarrow -4a^3+2ar^2<0 \Longleftrightarrow 2ar^2<4a^3
        \Longleftrightarrow r^2<2a^2
        \Longleftrightarrow r<\sqrt{2}a 
        \Longleftrightarrow \frac{r}{\sqrt{2}}<a 
    \end{equation*}

    Por tanto, tenemos que es estrictamente creciente en este intervalo.
\end{itemize}

Por tanto, se ha demostrado que $a=\frac{r}{\sqrt{2}}$ es un mínimo relativo. Además, es absoluto, ya que es el único extremo relativo de una función continua.

Por tanto, tenemos que:
\begin{equation*}
    a=\frac{r}{\sqrt{2}}=\sqrt{2}
    \hspace{1cm}
    b=\sqrt{r^2-a^2}=\sqrt{r^2-\frac{r^2}{2}} = \sqrt{\frac{r^2}{2}} = \frac{r}{\sqrt{2}}=\sqrt{2}
\end{equation*}

Por tanto, tenemos que el único punto en el primer cuadrante con la propiedad buscada es el punto $(\sqrt{2}, \sqrt{2})$. Como la circunferencia se encuentra centrada en el $(0,0)$, tenemos que el resto de puntos son:
\begin{equation*}
    (\pm \sqrt{2}, \pm \sqrt{2})
\end{equation*}


\end{ejercicio}

\begin{ejercicio}\textbf{[2 puntos]}
Sea $f:\bb{R}\to \bb{R}$  una función de $C^\infty(\bb{R})$ tal que $f(0)=3$. Supongamos $0\leq f(x)\leq 4-x$ y que $f'(x)=f(x)+x$, para cada $x\in \bb{R}$. Calcular $f\left(\frac{1}{10}\right)$ con tres cifras decimales exactas.

Al pedir tres cifras decimales exactas, tenemos que el error de aproximación ha de ser menor que $10^{-4}$. Por el resto de Lagrange tenemos que:
\begin{equation*}
    R_{n,0}^f(x)=\frac{f^{n+1)}(c)}{(n+1)!}\cdot x^{n+1} \qquad c\in [0,x]
\end{equation*}

Acotamos para $x=\frac{1}{10}$:
\begin{multline*}
    \left|R_{n,0}^f\left(\frac{1}{10}\right)\right|
    =\left|\frac{f^{n+1)}(c)}{(n+1)!}\cdot \frac{1}{10^{n+1}}\right|
    =\frac{\left|f^{n+1)}(c)\right|}{(n+1)!}\cdot \frac{1}{10^{n+1}} \leq \frac{1}{10^4}
    \Longleftrightarrow \\ \Longleftrightarrow 
    \left|f^{n+1)}(c)\right| \leq (n+1)!10^{n-3}
\end{multline*}

Acotamos ahora cada derivada en el intervalo $\left[0,\frac{1}{10}\right]$:
\begin{itemize}
    \item \underline{Para $n=1$}:

    Tenemos que $f'(x)=f(x)+x$. Por tanto,
    \begin{equation*}
        f'(x)=f(x)+x\leq 4-x+x=4
        \qquad
        f'(x)=f(x)+x\geq x \geq 0
    \end{equation*}

    Por tanto, $0\leq f'(x)\leq 4$.

    \item \underline{Para $n=2$}:

    Tenemos que $f''(x)=f'(x)+1$. Por tanto, $1\leq f''(x)\leq 5$.

    \item \underline{Para $n\in \bb{N}, n\geq 3$}:

    Se demuestra por inducción que $f^{n)}(x)=f^{n-1)}(x)$. Por tanto, como $1~\leq~f''(x)\leq~5$, se tiene que $1\leq f^{n)}(x)\leq 5$.
\end{itemize}

Por tanto, como $1\leq f^{n)}(x)\leq 5$ para todo $n\geq 3$ natural, calculemos un valor de $n$ que cumple la desigualdad obtenida por la acotación del resto de Lagrange:
\begin{equation*}
    \left|f^{n+1)}(c)\right| \leq 5 \leq (n+1)!10^{n-3}
\end{equation*}

Vemos que $n=3$ lo cumple, ya que $(4)!\cdot 10^0 = 4!=24\geq 5$. Por tanto, tenemos que una aproximación con tres cifras decimales exactas se obtendría con el polinomio de grado 3 centrado en el 0.
\begin{equation*}
    P_{3,0}^f(x)=f(0)+f'(0)x + \frac{f''(0)}{2}x^2 +\frac{f'''(0)}{6}x^3
\end{equation*}

Tenemos que $f(0)=3$ por el enunciado. Por tanto,
\begin{equation*}
    f'(0)=f(0)+0=3
    \qquad
    f''(0)=f'(0)+1=4
    \qquad
    f'''(0)=f''(0)=4
\end{equation*}

Por tanto,
\begin{equation*}
    P_{3,0}^f(x)=3+3x + 2x^2 +\frac{2}{3}x^3
\end{equation*}

La aproximación buscada es:
\begin{equation*}
    P_{3,0}^f \left(\frac{1}{10}\right) = 3+\frac{3}{10} + \frac{2}{100} + \frac{2}{3\cdot 10^3}=3.320
\end{equation*}
    
\end{ejercicio}

\begin{ejercicio}\textbf{[2.5 puntos]}
Sea $F:[0,+\infty[\to \bb{R}$ la función dada por $F(x)=~\int_x^{2x}e^{-t^2}\;dt$ para cada $x\geq 0$.
\begin{enumerate}
    \item Determinar los puntos en los que $F$ alcanza sus extremos relativos y/o absolutos.

    Veamos en primer lugar que el integrando está acotado. Sea $f(x)=e^{-x^2}$. Tenemos que $f'(x)=-2xe^{-x^2}<0$, por lo que es estrictamente decreciente. Además, $f(0)=e^0=1$, y $\lim_{x\to \infty}f(x)=e^{-\infty}=0$. Por tanto, está acotada por 1. Como también es continua, tenemos que es Riemman Integrable. Por tanto, tenemos que:
    \begin{equation*}
        F(x)=\int_x^{2x}e^{-t^2}\;dt
        = \int_0^{2x}e^{-t^2}\;dt
        -
        \int_0^{x}e^{-t^2}\;dt
    \end{equation*}

    Como es Riemman Integrable y continua, por el TFC, tenemos que $F$ es continua y derivable en $\bb{R}^+_0$ con:
    \begin{multline*}
        F'(x)=2e^{-(2x)^2} -e^{-x^2}
        = 2e^{-4x^2} -e^{-x^2} = 0 \Longleftrightarrow 2e^{-4x^2}=e^{-x^2}
        \Longleftrightarrow \\ \Longleftrightarrow
        2e^{-4x^2+x^2} = 1
        \Longleftrightarrow -3x^2 = \ln\left(\frac{1}{2}\right) = -\ln 2 \Longleftrightarrow x=\sqrt{\frac{\ln 2}{3}}
    \end{multline*}

    Estudiamos su monotonía:
    \begin{itemize}
        \item \underline{Para $x<\sqrt{\frac{\ln 2}{3}}$}:
        \begin{multline*}
            F'(x)
            = 2e^{-4x^2} -e^{-x^2} > 0 \Longleftrightarrow 2e^{-4x^2}>e^{-x^2}
            \Longleftrightarrow \\ \Longleftrightarrow
            2e^{-4x^2+x^2} > 1
            \Longleftrightarrow -3x^2 > \ln\left(\frac{1}{2}\right) = -\ln 2 \Longleftrightarrow x<\sqrt{\frac{\ln 2}{3}}
        \end{multline*}
        Por tanto, tenemos que $F'(x)$ es estrictamente creciente.

        \item \underline{Para $x>\sqrt{\frac{\ln 2}{3}}$}:
        \begin{multline*}
            F'(x)
            = 2e^{-4x^2} -e^{-x^2} < 0 \Longleftrightarrow 2e^{-4x^2}>e^{-x^2}
            \Longleftrightarrow \\ \Longleftrightarrow
            2e^{-4x^2+x^2} < 1
            \Longleftrightarrow -3x^2 < \ln\left(\frac{1}{2}\right) = -\ln 2 \Longleftrightarrow x>\sqrt{\frac{\ln 2}{3}}
        \end{multline*}
        Por tanto, tenemos que $F'(x)$ es estrictamente decreciente.
    \end{itemize}

    Por tanto, $x=\sqrt{\frac{\ln 2}{3}}$ es un máximo relativo. Por ser $F$ continua y ser el único extremo relativo, tenemos que es también máximo absoluto.

    Como $F(0)=0$, veamos si es un mínimo absoluto. Veamos si $\exists c\in \bb{R}\mid F(c)<~0$.
    
    Como $f(x)> 0 \;\;\forall x\in \bb{R}^+$, tenemos que $\int_a^b f(x)\;dx > 0 \;\;\forall\; b>a$. Por tanto, como $2x>x \;\;\forall x\in \bb{R}^+$, tenemos que:
    \begin{equation*}
        0< F(x) \quad \forall x\in \bb{R}^+
    \end{equation*}

    Por tanto, como $F(x)>0 \;\;\forall x\in \bb{R}^+$ y $F(0)=0$, tenemos que $x=0$ es un mínimo absoluto.

    \item Estudiar la concavidad de $F$. ¿Tiene $F$ algún punto de inflexión?

    Tenemos que $F$ es dos veces derivable, con:
    \begin{multline*}
        F''(x)=-8x\cdot 2e^{-4x^2} +2xe^{-x^2} = 0 \Longleftrightarrow
        e^{-x^2}=8e^{-4x^2} \Longleftrightarrow -x^2=\ln (8)- 4x^2
        \Longleftrightarrow \\ \Longleftrightarrow
        3x^2-\ln 8=0 \Longleftrightarrow x=\sqrt{\frac{\ln 8}{3}}
    \end{multline*}
    donde he supuesto que $x=0$ no puede ser un punto de inflexión, ya que no es un punto interior.

    Veamos si efectivamente ese valor es un punto de inflexión estudiando su convexidad:
    \begin{itemize}
        \item \underline{Para $x<\sqrt{\frac{\ln 8}{3}}$}:
        \begin{multline*}
            F''(x)=-8x\cdot 2e^{-4x^2} +2xe^{-x^2} < 0 \Longleftrightarrow
            e^{-x^2}<8e^{-4x^2} \Longleftrightarrow -x^2<\ln (8)- 4x^2
            \Longleftrightarrow \\ \Longleftrightarrow
            3x^2-\ln 8<0 \Longleftrightarrow x<\sqrt{\frac{\ln 8}{3}}
        \end{multline*}

        Por tanto, tenemos que en este intervalo es cóncava hacia abajo.

        \item \underline{Para $x>\sqrt{\frac{\ln 8}{3}}$}:
        \begin{multline*}
            F''(x)=-8x\cdot 2e^{-4x^2} +2xe^{-x^2} > 0 \Longleftrightarrow
            e^{-x^2}>8e^{-4x^2} \Longleftrightarrow -x^2>\ln (8)- 4x^2
            \Longleftrightarrow \\ \Longleftrightarrow
            3x^2-\ln 8>0 \Longleftrightarrow x>\sqrt{\frac{\ln 8}{3}}
        \end{multline*}

        Por tanto, tenemos que en este intervalo es cóncava hacia arriba.
    \end{itemize}

    Por tanto, como hay un cambio de curvatura, tenemos que $x=\sqrt{\frac{\ln 8}{3}}$ es un punto de inflexión.

    \item Demostrar que $0\leq F(x)\leq xe^{-x^2}$, para cada $x\geq 0$. ¿Cuánto vale $\displaystyle \lim_{x\to \infty}F(x)$?

    En el primer apartado de este ejercicio se ha demostrado que $F(x)>0\quad \forall x\in~\bb{R}^+$ y $F(0)=0$, por lo que se tiene la primera desigualdad.

    Además, por el Teorema del Valor Medio para las integrales, tenemos que $\forall x\in \bb{R}^+_0, \exists c\in [x,2x]$ tal que:
    \begin{equation*}
        F(x)=\int_x^{2x}f(t)\;dt =f(c)(2x-x)=xf(c)
    \end{equation*}

    Además, se ha demostrado que $f(t)$ es estrictamente decreciente. Como $c\in~[x,2x]$, tenemos que $x\leq c\Longrightarrow f(x)\geq f(c)$. Por tanto,
    \begin{equation*}
        F(x)=xf(c) \leq xf(x)
    \end{equation*}

    Uniendo lo demostrado, tenemos que:
    \begin{equation*}
        0\leq F(x)\leq xf(x) \hspace{2cm} \forall x\geq 0
    \end{equation*}
    como queríamos demostrar.\vspace{1cm}

    Para calcular el límite en $+\infty$ de $F(x)$, calculamos primero el siguiente límite:
    \begin{equation*}
        \lim_{x\to \infty}xf(x)
        = \lim_{x\to \infty}\frac{x}{e^{x^2}}
        \Hop
        \lim_{x\to \infty}\frac{2}{2x e^{x^2}} = 0
    \end{equation*}

    Por tanto, por el Teorema del Sandwich, tenemos que:
    \begin{equation*}
        \lim_{x\to \infty}F(x) = 0
    \end{equation*}

    \item Estudiar la continuidad uniforme de la función $H(x):=\frac{F(x)}{\sqrt{x}}$ en el intervalo $]0,1[$.

    \begin{equation*}
        \lim_{x\to 0}H(x)=\left[\frac{0}{0}\right]\Hop
        \lim_{x\to 0}\frac{F'(x)}{\frac{1}{2\sqrt{x}}}
        = \lim_{x\to 0}2\sqrt{x}\left(2e^{-4x^2}-e^{-x^2}\right)
        = 0\cdot (2e^0-e^0)=0
    \end{equation*}

    \begin{equation*}
        \lim_{x\to 1}H(x)=F(1)
    \end{equation*}

    Por tanto, definiendo $H(1)=F(1)\in \bb{R}$ y $H(0)=0$, tenemos que $H$ es una función continua. Por tanto, como existe una ampliación del dominio continua, tenemos que $H$ es uniformemente continua en $]0,1[$.
\end{enumerate}
    
\end{ejercicio}

\begin{ejercicio}\textbf{[2 puntos]}
Calcular el área de la región acotada limitada por la gráfica de la curva de ecuación $\displaystyle \left(\frac{x}{5}\right)^2 +\left(\frac{y}{4}\right)^{\frac{2}{3}}=1$.\\

Tenemos que la función es:
\begin{equation*}
    y^{\frac{2}{3}} = 4^{\frac{2}{3}}\left(1-\frac{x^2}{25}\right) \Longrightarrow y^2=16\left(1-\frac{x^2}{25}\right)^3 \Longrightarrow y=\pm4 \sqrt{\left(1-\frac{x^2}{25}\right)^3 }
\end{equation*}

Calculo solo el área que se encuentra por encima del eje OX, que será la mitad del área. Los puntos de corte con el eje OX son:
\begin{equation*}
    0=\pm4 \sqrt{\left(1-\frac{x^2}{25}\right)^3 }
    \Longleftrightarrow 1-\frac{x^2}{25}=0 \Longleftrightarrow x=\pm 5
\end{equation*}

Como la función es par, basta con hallar el área del primer cuadrante, que será un cuarto del área total.

\begin{multline*}
    \frac{A}{4} = \int_0^5 4 \sqrt{\left(1-\frac{x^2}{25}\right)^3 }
    \;dx= 4\int_0^5 \sqrt{\left(\frac{25-x^3}{25}\right)^3 }\;dx
    = \frac{4}{125}\int_0^5 \sqrt{\left(25-x^2\right)^3 }
    \;dx=\\=
    \MetInt{5\sen t=x\quad t\in \left[0,\frac{\pi}{2}\right]}{5\cos t \;dt=dx}
    = \frac{4}{125}\int_0^{\frac{\pi}{2}} \sqrt{(25-25\sen^2t)^3}\;5\cos t \;dt
    =\\= \frac{4}{125}\cdot 5\cdot 125 \int_0^{\frac{\pi}{2}} \sqrt{(1-\sen^2t)^3}\;\cos t \;dt
    = 20 \int_0^{\frac{\pi}{2}} \cos^4 t \;dt
\end{multline*}

Tenemos que $\cos^2 x = \frac{1+\cos(2x)}{2}$. Por tanto,
\begin{equation*}
    \cos^4 x = \left(\frac{1+\cos(2x)}{2}\right)^2
    = \frac{1+\cos^2(2x)+2\cos(2x)}{4}
    = \frac{1+\frac{1+\cos(4x)}{2}+2\cos(2x)}{4}
\end{equation*}

Por tanto,
\begin{multline*}
    \frac{A}{4}= 20 \int_0^{\frac{\pi}{2}} \cos^4 t \;dt
    = 20 \int_0^{\frac{\pi}{2}} \frac{1}{4} + \frac{1}{8} + \frac{1}{8}\cos(4t) + \frac{1}{2}\cos(2t) \;dt =\\= 20\left[\frac{3t}{8}+\frac{1}{32}\sen(4t) +\frac{1}{4}\sen(2t)\right]_0^{\frac{\pi}{2}} =
    20\left[\frac{3\pi}{16}\right] = \frac{15\pi}{4}
\end{multline*}

Por tanto, tenemos que el área es $A=15\pi\;u^2$.

\end{ejercicio}



\end{document}