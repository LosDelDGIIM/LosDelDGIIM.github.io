\documentclass[12pt]{article}

% Idioma y codificación
\usepackage[spanish, es-tabla]{babel}       %es-tabla para que se titule "Tabla"
\usepackage[utf8]{inputenc}

% Márgenes
\usepackage[a4paper,top=3cm,bottom=2.5cm,left=3cm,right=3cm]{geometry}

% Comentarios de bloque
\usepackage{verbatim}

% Paquetes de links
\usepackage[hidelinks]{hyperref}    % Permite enlaces
\usepackage{url}                    % redirecciona a la web

% Más opciones para enumeraciones
\usepackage{enumitem}

% Personalizar la portada
\usepackage{titling}

% Paquetes de tablas
\usepackage{multirow}


%------------------------------------------------------------------------

%Paquetes de figuras
\usepackage{caption}
\usepackage{subcaption} % Figuras al lado de otras
\usepackage{float}      % Poner figuras en el sitio indicado H.


% Paquetes de imágenes
\usepackage{graphicx}       % Paquete para añadir imágenes
\usepackage{transparent}    % Para manejar la opacidad de las figuras

% Paquete para usar colores
\usepackage[dvipsnames]{xcolor}
\usepackage{pagecolor}      % Para cambiar el color de la página

% Habilita tamaños de fuente mayores
\usepackage{fix-cm}

% Para los gráficos
\usepackage{tikz}

% Para poder situar los nodos en los grafos
\usetikzlibrary{positioning}


%------------------------------------------------------------------------

% Paquetes de matemáticas
\usepackage{mathtools, amsfonts, amssymb, mathrsfs}
\usepackage[makeroom]{cancel}     % Simplificar tachando
\usepackage{polynom}    % Divisiones y Ruffini
\usepackage{units} % Para poner fracciones diagonales con \nicefrac

\usepackage{pgfplots}   %Representar funciones
\pgfplotsset{compat=1.18}  % Versión 1.18

\usepackage{tikz-cd}    % Para usar diagramas de composiciones
\usetikzlibrary{calc}   % Para usar cálculo de coordenadas en tikz

%Definición de teoremas, etc.
\usepackage{amsthm}
%\swapnumbers   % Intercambia la posición del texto y de la numeración

\theoremstyle{plain}

\makeatletter
\@ifclassloaded{article}{
  \newtheorem{teo}{Teorema}[section]
}{
  \newtheorem{teo}{Teorema}[chapter]  % Se resetea en cada chapter
}
\makeatother

\newtheorem{coro}{Corolario}[teo]           % Se resetea en cada teorema
\newtheorem{prop}[teo]{Proposición}         % Usa el mismo contador que teorema
\newtheorem{lema}[teo]{Lema}                % Usa el mismo contador que teorema

\theoremstyle{remark}
\newtheorem*{observacion}{Observación}

\theoremstyle{definition}

\makeatletter
\@ifclassloaded{article}{
  \newtheorem{definicion}{Definición} [section]     % Se resetea en cada chapter
}{
  \newtheorem{definicion}{Definición} [chapter]     % Se resetea en cada chapter
}
\makeatother

\newtheorem*{notacion}{Notación}
\newtheorem*{ejemplo}{Ejemplo}
\newtheorem*{ejercicio*}{Ejercicio}             % No numerado
\newtheorem{ejercicio}{Ejercicio} [section]     % Se resetea en cada section


% Modificar el formato de la numeración del teorema "ejercicio"
\renewcommand{\theejercicio}{%
  \ifnum\value{section}=0 % Si no se ha iniciado ninguna sección
    \arabic{ejercicio}% Solo mostrar el número de ejercicio
  \else
    \thesection.\arabic{ejercicio}% Mostrar número de sección y número de ejercicio
  \fi
}


% \renewcommand\qedsymbol{$\blacksquare$}         % Cambiar símbolo QED
%------------------------------------------------------------------------

% Paquetes para encabezados
\usepackage{fancyhdr}
\pagestyle{fancy}
\fancyhf{}

\newcommand{\helv}{ % Modificación tamaño de letra
\fontfamily{}\fontsize{12}{12}\selectfont}
\setlength{\headheight}{15pt} % Amplía el tamaño del índice


%\usepackage{lastpage}   % Referenciar última pag   \pageref{LastPage}
\fancyfoot[C]{\thepage}

%------------------------------------------------------------------------

% Conseguir que no ponga "Capítulo 1". Sino solo "1."
\makeatletter
\@ifclassloaded{book}{
  \renewcommand{\chaptermark}[1]{\markboth{\thechapter.\ #1}{}} % En el encabezado
    
  \renewcommand{\@makechapterhead}[1]{%
  \vspace*{50\p@}%
  {\parindent \z@ \raggedright \normalfont
    \ifnum \c@secnumdepth >\m@ne
      \huge\bfseries \thechapter.\hspace{1em}\ignorespaces
    \fi
    \interlinepenalty\@M
    \Huge \bfseries #1\par\nobreak
    \vskip 40\p@
  }}
}
\makeatother

%------------------------------------------------------------------------
% Paquetes de cógido
\usepackage{minted}
\renewcommand\listingscaption{Código fuente}

\usepackage{fancyvrb}
% Personaliza el tamaño de los números de línea
\renewcommand{\theFancyVerbLine}{\small\arabic{FancyVerbLine}}

% Estilo para C++
\newminted{cpp}{
    frame=lines,
    framesep=2mm,
    baselinestretch=1.2,
    linenos,
    escapeinside=||
}

% para minted
\definecolor{LightGray}{rgb}{0.95,0.95,0.92}
\setminted{
    linenos=true,
    stepnumber=5,
    numberfirstline=true,
    autogobble,
    breaklines=true,
    breakautoindent=true,
    breaksymbolleft=,
    breaksymbolright=,
    breaksymbolindentleft=0pt,
    breaksymbolindentright=0pt,
    breaksymbolsepleft=0pt,
    breaksymbolsepright=0pt,
    fontsize=\footnotesize,
    bgcolor=LightGray,
    numbersep=10pt
}


\usepackage{listings} % Para incluir código desde un archivo

\renewcommand\lstlistingname{Código Fuente}
\renewcommand\lstlistlistingname{Índice de Códigos Fuente}

% Definir colores
\definecolor{vscodepurple}{rgb}{0.5,0,0.5}
\definecolor{vscodeblue}{rgb}{0,0,0.8}
\definecolor{vscodegreen}{rgb}{0,0.5,0}
\definecolor{vscodegray}{rgb}{0.5,0.5,0.5}
\definecolor{vscodebackground}{rgb}{0.97,0.97,0.97}
\definecolor{vscodelightgray}{rgb}{0.9,0.9,0.9}

% Configuración para el estilo de C similar a VSCode
\lstdefinestyle{vscode_C}{
  backgroundcolor=\color{vscodebackground},
  commentstyle=\color{vscodegreen},
  keywordstyle=\color{vscodeblue},
  numberstyle=\tiny\color{vscodegray},
  stringstyle=\color{vscodepurple},
  basicstyle=\scriptsize\ttfamily,
  breakatwhitespace=false,
  breaklines=true,
  captionpos=b,
  keepspaces=true,
  numbers=left,
  numbersep=5pt,
  showspaces=false,
  showstringspaces=false,
  showtabs=false,
  tabsize=2,
  frame=tb,
  framerule=0pt,
  aboveskip=10pt,
  belowskip=10pt,
  xleftmargin=10pt,
  xrightmargin=10pt,
  framexleftmargin=10pt,
  framexrightmargin=10pt,
  framesep=0pt,
  rulecolor=\color{vscodelightgray},
  backgroundcolor=\color{vscodebackground},
}

%------------------------------------------------------------------------

% Comandos definidos
\newcommand{\bb}[1]{\mathbb{#1}}
\newcommand{\cc}[1]{\mathcal{#1}}

% I prefer the slanted \leq
\let\oldleq\leq % save them in case they're every wanted
\let\oldgeq\geq
\renewcommand{\leq}{\leqslant}
\renewcommand{\geq}{\geqslant}

% Si y solo si
\newcommand{\sii}{\iff}

% Letras griegas
\newcommand{\eps}{\epsilon}
\newcommand{\veps}{\varepsilon}
\newcommand{\lm}{\lambda}

\newcommand{\ol}{\overline}
\newcommand{\ul}{\underline}
\newcommand{\wt}{\widetilde}
\newcommand{\wh}{\widehat}

\let\oldvec\vec
\renewcommand{\vec}{\overrightarrow}

% Derivadas parciales
\newcommand{\del}[2]{\frac{\partial #1}{\partial #2}}
\newcommand{\Del}[3]{\frac{\partial^{#1} #2}{\partial #3^{#1}}}
\newcommand{\deld}[2]{\dfrac{\partial #1}{\partial #2}}
\newcommand{\Deld}[3]{\dfrac{\partial^{#1} #2}{\partial #3^{#1}}}


\newcommand{\AstIg}{\stackrel{(\ast)}{=}}
\newcommand{\Hop}{\stackrel{L'H\hat{o}pital}{=}}

\newcommand{\red}[1]{{\color{red}#1}} % Para integrales, destacar los cambios.

% Método de integración
\newcommand{\MetInt}[2]{
    \left[\begin{array}{c}
        #1 \\ #2
    \end{array}\right]
}

% Declarar aplicaciones
% 1. Nombre aplicación
% 2. Dominio
% 3. Codominio
% 4. Variable
% 5. Imagen de la variable
\newcommand{\Func}[5]{
    \begin{equation*}
        \begin{array}{rrll}
            #1:& #2 & \longrightarrow & #3\\
               & #4 & \longmapsto & #5
        \end{array}
    \end{equation*}
}

%------------------------------------------------------------------------



\begin{document}

    % 1. Foto de fondo
    % 2. Título
    % 3. Encabezado Izquierdo
    % 4. Color de fondo
    % 5. Coord x del titulo
    % 6. Coord y del titulo
    % 7. Fecha

    
    % 1. Foto de fondo
% 2. Título
% 3. Encabezado Izquierdo
% 4. Color de fondo
% 5. Coord x del titulo
% 6. Coord y del titulo
% 7. Fecha

\newcommand{\portada}[7]{

    \portadaBase{#1}{#2}{#3}{#4}{#5}{#6}{#7}
    \portadaBook{#1}{#2}{#3}{#4}{#5}{#6}{#7}
}

\newcommand{\portadaExamen}[7]{

    \portadaBase{#1}{#2}{#3}{#4}{#5}{#6}{#7}
    \portadaArticle{#1}{#2}{#3}{#4}{#5}{#6}{#7}
}




\newcommand{\portadaBase}[7]{

    % Tiene la portada principal y la licencia Creative Commons
    
    % 1. Foto de fondo
    % 2. Título
    % 3. Encabezado Izquierdo
    % 4. Color de fondo
    % 5. Coord x del titulo
    % 6. Coord y del titulo
    % 7. Fecha
    
    
    \thispagestyle{empty}               % Sin encabezado ni pie de página
    \newgeometry{margin=0cm}        % Márgenes nulos para la primera página
    
    
    % Encabezado
    \fancyhead[L]{\helv #3}
    \fancyhead[R]{\helv \nouppercase{\leftmark}}
    
    
    \pagecolor{#4}        % Color de fondo para la portada
    
    \begin{figure}[p]
        \centering
        \transparent{0.3}           % Opacidad del 30% para la imagen
        
        \includegraphics[width=\paperwidth, keepaspectratio]{assets/#1}
    
        \begin{tikzpicture}[remember picture, overlay]
            \node[anchor=north west, text=white, opacity=1, font=\fontsize{60}{90}\selectfont\bfseries\sffamily, align=left] at (#5, #6) {#2};
            
            \node[anchor=south east, text=white, opacity=1, font=\fontsize{12}{18}\selectfont\sffamily, align=right] at (9.7, 3) {\textbf{\href{https://losdeldgiim.github.io/}{Los Del DGIIM}}};
            
            \node[anchor=south east, text=white, opacity=1, font=\fontsize{12}{15}\selectfont\sffamily, align=right] at (9.7, 1.8) {Doble Grado en Ingeniería Informática y Matemáticas\\Universidad de Granada};
        \end{tikzpicture}
    \end{figure}
    
    
    \restoregeometry        % Restaurar márgenes normales para las páginas subsiguientes
    \pagecolor{white}       % Restaurar el color de página
    
    
    \newpage
    \thispagestyle{empty}               % Sin encabezado ni pie de página
    \begin{tikzpicture}[remember picture, overlay]
        \node[anchor=south west, inner sep=3cm] at (current page.south west) {
            \begin{minipage}{0.5\paperwidth}
                \href{https://creativecommons.org/licenses/by-nc-nd/4.0/}{
                    \includegraphics[height=2cm]{assets/Licencia.png}
                }\vspace{1cm}\\
                Esta obra está bajo una
                \href{https://creativecommons.org/licenses/by-nc-nd/4.0/}{
                    Licencia Creative Commons Atribución-NoComercial-SinDerivadas 4.0 Internacional (CC BY-NC-ND 4.0).
                }\\
    
                Eres libre de compartir y redistribuir el contenido de esta obra en cualquier medio o formato, siempre y cuando des el crédito adecuado a los autores originales y no persigas fines comerciales. 
            \end{minipage}
        };
    \end{tikzpicture}
    
    
    
    % 1. Foto de fondo
    % 2. Título
    % 3. Encabezado Izquierdo
    % 4. Color de fondo
    % 5. Coord x del titulo
    % 6. Coord y del titulo
    % 7. Fecha


}


\newcommand{\portadaBook}[7]{

    % 1. Foto de fondo
    % 2. Título
    % 3. Encabezado Izquierdo
    % 4. Color de fondo
    % 5. Coord x del titulo
    % 6. Coord y del titulo
    % 7. Fecha

    % Personaliza el formato del título
    \pretitle{\begin{center}\bfseries\fontsize{42}{56}\selectfont}
    \posttitle{\par\end{center}\vspace{2em}}
    
    % Personaliza el formato del autor
    \preauthor{\begin{center}\Large}
    \postauthor{\par\end{center}\vfill}
    
    % Personaliza el formato de la fecha
    \predate{\begin{center}\huge}
    \postdate{\par\end{center}\vspace{2em}}
    
    \title{#2}
    \author{\href{https://losdeldgiim.github.io/}{Los Del DGIIM}}
    \date{Granada, #7}
    \maketitle
    
    \tableofcontents
}




\newcommand{\portadaArticle}[7]{

    % 1. Foto de fondo
    % 2. Título
    % 3. Encabezado Izquierdo
    % 4. Color de fondo
    % 5. Coord x del titulo
    % 6. Coord y del titulo
    % 7. Fecha

    % Personaliza el formato del título
    \pretitle{\begin{center}\bfseries\fontsize{42}{56}\selectfont}
    \posttitle{\par\end{center}\vspace{2em}}
    
    % Personaliza el formato del autor
    \preauthor{\begin{center}\Large}
    \postauthor{\par\end{center}\vspace{3em}}
    
    % Personaliza el formato de la fecha
    \predate{\begin{center}\huge}
    \postdate{\par\end{center}\vspace{5em}}
    
    \title{#2}
    \author{\href{https://losdeldgiim.github.io/}{Los Del DGIIM}}
    \date{Granada, #7}
    \thispagestyle{empty}               % Sin encabezado ni pie de página
    \maketitle
    \vfill
}
    \portadaExamen{ffccA4.jpg}{Cálculo II\\Examen IX}{Cálculo II. Examen IX}{MidnightBlue}{-8}{28}{2023}{Arturo Olivares Martos}

    \begin{description}
        \item[Asignatura] Cálculo II.
        \item[Curso Académico] 2017-18.
        \item[Grado] Doble Grado en Ingeniería Informática y Matemáticas.
        \item[Grupo] Único.
        %\item[Profesor] María Victoria Velasco Collado.
        \item[Descripción] Convocatoria Ordinaria.
        %\item[Fecha] 24 de abril de 2019.
        %\item[Duración] 60 minutos.
    
    \end{description}
    \newpage
    
    \begin{ejercicio} [\textbf{2 puntos}]
Teorema de Rolle. Teorema del Valor Medio.

\begin{teo} [Teorema de Rolle]
    Sea $f:[a,b] \to \bb{R}$ continua en $[a.b]$ y derivable en $]a,b[$ tal que $f(a)=f(b)$. Entonces, $$\exists c\in ]a,b[ \mid f'(c)=0$$
\end{teo}
\begin{proof}
    En el caso de que $f$ sea constante, tenemos que $f'(x)=0\;\forall x\in [a,b]$, por lo que se tiene el resultado.

    Supongamos $f$ no constante. Al ser continua en $[a,b]$, por el Teorema de Bolzano-Weierstrass tenemos que $\exists m,M\in \bb{R}$, con $m<M$ tal que $f([a,b])=[m,M]$. Sean $c_1, c_2\in \bb{R}$ tal que $f(c_1)=m$, $f(c_2)=M$.
    \begin{itemize}
        \item \underline{$m\neq (f(a)=f(b))$}: Tenemos que $c_1\in ]a,b[$, y es un mínimo relativo, por lo que $f'(c_1)=0$.

        \item \underline{$M\neq (f(a)=f(b))$}: Tenemos que $c_2\in ]a,b[$, y es un máximo relativo, por lo que $f'(c_2)=0$.
    \end{itemize}
    Puesto que $m\neq M$, seguro que estamos al menos ante uno de los dos casos anteriores, por lo que $\exists c\in ]a,b[\mid f'(c)=0$.
\end{proof}

\begin{teo} [Teorema del Valor Medio]
    Sea $f:[a,b]\to \bb{R}$ continua en $[a,b]$ y derivable en $]a,b[$. Entonces,
    $$\exists c\in ]a,b[ \mid f(b)-f(a)=f'(c)(b-a)$$
\end{teo}
\begin{proof}
    Sea $r(x)$ la recta que pasa por los puntos $(a,f(a)),(b,f(b))$:
    \begin{equation*}
        r(x)=f(a)+\frac{f(b)-f(a)}{b-a}(x-a)
    \end{equation*}

    Consideramos ahora la función $h(x)=f(x)-r(x)$.
    \begin{gather*}
        h(a)=f(a)-r(a)=f(a)-f(a)+0 = 0
        \\
        h(b)=f(b)-f(a)-f(b)+f(a) = 0
    \end{gather*}

    Como $h(a)=h(b)$, estamos ante las condiciones de Rolle. Por tanto, $\exists c\in ]a,b[$ tal que:
    \begin{equation*}
        0=h'(c) = f'(c)-r'(c) = f'(c)-\frac{f(b)-f(a)}{b-a}
    \end{equation*}

    Por tanto,
    \begin{equation*}
        \exists c\in ]a,b[ \mid f(b)-f(a)=f'(c)(b-a)
    \end{equation*}
\end{proof}
    
\end{ejercicio}

\begin{ejercicio} [\textbf{2 puntos}]
    Decir si son verdaderas o falsas las siguientes cuestiones, justificando la respuesta:

    \begin{enumerate}
        \item Toda función cóncava hacia arriba es uniformemente continua.

        Esto es falso, ya que $f:\bb{R}\to \bb{R}$ dada por $f(x)=x^2$ es cóncava hacia arriba pero no es uniformemente continua. Por tanto, el enunciado es \textbf{falso}.

        \item Si $A\subset \bb{R}$ es un conjunto tal que todos los puntos de $A$ son de acumulación de $A$, $f:A\to \bb{R}$ es derivable y $f'$ no se anula, entonces $f$ es inyectiva.

        \begin{itemize}
            \item \underline{\textbf{Opción 1}: Usando el Teorema de Rolle}:

            Sabemos que $f$ es derivable en $A=A\cap A'$, por lo que también es continua en $A$.

            Supongamos que $\exists x,y\in A$, con $x<y$, tal que $f(x)=f(y)$. Entonces, por el Teorema de Rolle, $\exists c\in ]x,y[\mid f'(c)=0$. No obstante, esto no es posible ya que $f'$ no se anula, por lo que deducimos que $\nexists x,y\in A,\;x<y$, con $f(x)=f(y)$.
            
            Por tanto, $f(x)=f(y) \Longleftrightarrow x=y$, por lo que $f$ es inyectiva.

            \underline{\textbf{Opción 2}: Sin usar el Teorema de Rolle}:

            Como es derivable en $A=A\cap A'$, y como $f'$ no se anula, tenemos que $f$ es estrictamente monótona. Por tanto, dados $x,y\in \bb{R}$ con $x<y$, tenemos que $f(x)<f(y)$ o $f(x)>f(y)$. En cualquier caso, tenemos que $x\neq y \Longrightarrow f(x)\neq f(y)$.
            
            Por tanto,
            \begin{equation*}
                f(x)=f(y)\Longleftrightarrow x=y
            \end{equation*}
            por lo que $f$ es inyectiva.
        \end{itemize}

        Por tanto, tenemos que el enunciado es \textbf{cierto}.

        \item Sea $I$ un intervalo no trivial y $f:I\to \bb{R}$ una función derivable tal que $f'$ tiene un único cero en $x_0$ y $f$ tiene un extremo relativo en $x_0$. ¿Tiene $f$ un extremo absoluto en $x_0$?

        El enunciado es \textbf{cierto}. Veámoslo.

        \begin{itemize}
            \item \underline{Suponemos que $x_0$ mínimo relativo}:

            Es decir, $\exists r>0\mid ]x_0-r, x_0+r[\subseteq A$ y que, $\forall x\in ]x_0-r, x_0+r[$ se tiene que $f(x)>f(x_0)$.

            Para que $x_0$ no sea mínimo absoluto, es necesario que $\exists x_m\in I\mid f(x_m)\leq~f(x_0)$. Por tanto, como la función es continua, es necesario que exista un $x_0'\in \bb{R}$ entre $x_m$ y $x_0$ en el que $f(x_0')=f(x_0)$. Por Rolle, tenemos que $\exists c\neq x_0\mid f'(c)=0$, en contradicción con que $f'$ solo tiene un cero.

            Por tanto, tenemos que $x_0$ es un mínimo absoluto.

            \item \underline{Suponemos que $x_0$ máximo relativo}:

            Es decir, $\exists r>0\mid ]x_0-r, x_0+r[\subseteq A$ y que, $\forall x\in ]x_0-r, x_0+r[$ se tiene que $f(x)<f(x_0)$.

            Para que $x_0$ no sea máximo absoluto, es necesario que $\exists x_m\in I\mid f(x_m)\geq~f(x_0)$. Por tanto, como la función es continua, es necesario que exista un $x_0'\in \bb{R}$ entre $x_m$ y $x_0$ en el que $f(x_0')=f(x_0)$. Por Rolle, tenemos que $\exists c\neq x_0\mid f'(c)=0$, en contradicción con que $f'$ solo tiene un cero.

            Por tanto, tenemos que $x_0$ es un máximo absoluto.
        \end{itemize}
        Por tanto, en ambos casos tenemos que $x_0$ es un extremo relativo.

        \item La función $f:\bb{R}^+ \to \bb{R}$ dada por:
        \begin{equation*}
            f(x)=\int_0^x \sqrt{t^5+t^2+1}dt \qquad (x\in \bb{R}^+)
        \end{equation*}
        tiene límite en $+\infty$.\\

        Consideramos $f_I(t)=\sqrt{t^5+t^2+1}$, por lo que $f(x)=\int_0^x f_I(t)\;dt$.

        Sea ahora $g_I(t)=\sqrt{t}$, y definamos $g(x)=\int_0^x g_I(t)\;dt$.
        \begin{equation*}
            g(x) = \int_0^x g_I(t)\;dt = \left[\frac{2}{3}t\sqrt{t}\right]_0^x = \frac{2}{3}x\sqrt{x}
        \end{equation*}
        \begin{equation*}
            \lim_{x\to \infty}g(x)=\lim_{x\to \infty}\frac{2}{3}x\sqrt{x} = +\infty
        \end{equation*}
        Por tanto, tenemos que $g(x)$ diverge positivamente.

        Como tenemos que $0\leq g_I(x)\leq f_I(x)$ y todas ellas son continuas por lo que localmente integrables en $\bb{R}^+$, tenemos que:
        \begin{equation*}
            f(x)=\int_0^\infty f_I(t)\;dt \text{ convergente}
            \Longrightarrow
            g(x)=\int_0^\infty g_I(t)\;dt \text{ convergente}
        \end{equation*}

        Como $g(x)$ no es convergente, tenemos que $f(x)$ tampoco. Por tanto, es \textbf{falso}.
        
    \end{enumerate}    
\end{ejercicio}

\begin{ejercicio} [\textbf{2 puntos}]
    Cada tangente a la circunferencia unidad en un punto cualquiera del primer cuadrante corta a los dos ejes en dos puntos de la forma $(x_1, 0)$ y $(0,y_1)$. Halla la ecuación de la recta tangente para que la suma $x_1+y_1$ sea mínima.

    \begin{figure}[H]
        \centering
        \begin{tikzpicture}
        \begin{axis}[
            xmin=-1.3,
            xmax=2,
            ymin=-1.3,
            ymax=2,
            axis lines=middle,
            width=7cm,
            height=7cm,
            samples=90 % número de muestras para la función
        ]

        \pgfmathsetmacro{\a}{sqrt(2)}
        \pgfmathsetmacro{\x}{cos(45)}

        \addplot[domain=0:360, samples=100, color=red] ({cos(x)}, {sin(x)});
        \addplot[thick,domain=-1:5] {-x+\a};

        \addplot[only marks,mark=*,mark size=2pt,color=red] coordinates {(\x,\x)};
        \node[label={above right: $P(a, f(a))$}] at (axis cs:\x,\x-0.1) {};

        \addplot[only marks,mark=*,mark size=2pt,color=red] coordinates {(0,\a)};
        \node[label={above right: $(0,y_1)$}] at (axis cs:0,\a-0.2) {};

        \addplot[only marks,mark=*,mark size=2pt,color=red] coordinates {(\a,0)};
        \node[label={above: $(x_1,0)$}] at (axis cs:\a+0.2,0) {};
            
        \end{axis}
        \end{tikzpicture}
    \end{figure}

    Trabajamos en primer lugar con la ecuación de la circunferencia. La ecuación de la circunferencia unidad es:
    \begin{equation*}
        x^2+y^2=1 \Longrightarrow y=\pm\sqrt{1-x^2}
    \end{equation*}

    Como estamos trabajando en el primer cuadrante, con valores de $y\geq 0$, tenemos que:
    \begin{equation*}
        f(x)=\sqrt{1-x^2} \qquad f'(x)=-\frac{x}{\sqrt{1-x^2}}
    \end{equation*}

    Como el punto $P$ pertenece a la circunferencia, tenemos que:
    \begin{equation*}
        f(a)=\sqrt{1-a^2}
    \end{equation*}

    \vspace{1cm}
    Trabajamos ahora con la recta. Por la interpretación geométrica, tenemos que:
    \begin{equation*}
        m_t = f'(a) = -\frac{a}{\sqrt{1-a^2}}
    \end{equation*}

    Tenemos que la recta que pasa por los dos puntos de corte es:
    \begin{equation*}
        r(x)=y_1 + m_tx = y_1 -\frac{a}{\sqrt{1-a^2}} \cdot x
    \end{equation*}
    
    Como el punto $P$ pertenece a la recta, tenemos que:
    \begin{equation*}
        r(a)=y_1 -\frac{a^2}{\sqrt{1-a^2}}
    \end{equation*}

    Usando que el punto $P$ pertenece tanto a la circunferencia como a la recta,
    \begin{multline*}
        f(a)=r(a) \Longrightarrow \sqrt{1-a^2} = y_1 -\frac{a^2}{\sqrt{1-a^2}} \Longrightarrow 1-\cancel{a^2} = y_1\cdot \sqrt{1-a^2} -\cancel{a^2}
        \Longrightarrow \\ \Longrightarrow
        y_1 = \frac{1}{\sqrt{1-a^2}}
    \end{multline*}

    Por tanto, la recta queda:
    \begin{equation*}
        r(x) = y_1 -\frac{a}{\sqrt{1-a^2}} \cdot x =  \frac{1}{\sqrt{1-a^2}} -\frac{a}{\sqrt{1-a^2}} \cdot x = \frac{1-ax}{\sqrt{1-a^2}}
    \end{equation*}

    Para $x=x_1$, tenemos que $r(x_1)=0$:
    \begin{equation*}
        r(x_1) = 0 \Longleftrightarrow 1=ax_1 \Longleftrightarrow x_1=\frac{1}{a}
    \end{equation*}

    Por tanto, la función a minimizar es:
    \begin{equation*}
        \begin{array}{rl}
            S:]0, 1[ & \longrightarrow \bb{R}\\
                    a & \longrightarrow S(a) = x_1 + y_1 = \displaystyle \frac{1}{a} + \frac{1}{\sqrt{1-a^2}}
        \end{array}
    \end{equation*}

    \begin{multline*}
        S'(a)=-\frac{1}{a^2} -\frac{1}{1-a^2} \cdot \frac{-2a}{2\sqrt{1-a^2}} = -\frac{1}{a^2} +\frac{a}{(1-a^2)\sqrt{1-a^2}} = 0
        \Longleftrightarrow \\\Longleftrightarrow
        a^3 = (1-a^2)\sqrt{1-a^2} \Longleftrightarrow a^6 = (1-a^2)^3 \Longleftrightarrow a^2 = 1-a^2
        \Longleftrightarrow \\\Longleftrightarrow
        a^2 = \frac{1}{2} \Longleftrightarrow a=\frac{\sqrt{2}}{2}
    \end{multline*}

    Comprobemos que el punto crítico es un mínimo relativo.
    \begin{itemize}
        \item \underline{Para $a<\frac{\sqrt{2}}{2}$}: $S'(a)<0\Longrightarrow S(a)$ estrictamente decreciente.

        \item \underline{Para $a<\frac{\sqrt{2}}{2}$}: $S'(a)>0\Longrightarrow S(a)$ estrictamente creciente.
    \end{itemize}
    Por tanto, tenemos que $a=\frac{\sqrt{2}}{2}$ es un mínimo relativo. Como $A$ es derivable y es definida en un intervalo, como es un mínimo relativo también es un mínimo absoluto.

    \begin{equation*}
        f\left(\frac{\sqrt{2}}{2}\right)=\sqrt{1-\left(\frac{\sqrt{2}}{2}\right)^2} = \sqrt{1-\frac{1}{2}} = \frac{\sqrt{2}}{2}
    \end{equation*}

    Por tanto, el punto $(a,f(a))$ que minimiza esa suma es:
    \begin{equation*}
        P\left(\frac{\sqrt{2}}{2},\frac{\sqrt{2}}{2}\right)
    \end{equation*}

    Además, tenemos que:
    \begin{equation*}
        x_1=y_1 = \sqrt{2}
    \end{equation*}
    
    
    
\end{ejercicio}

\begin{ejercicio} [\textbf{2 puntos}]
    Calcula los siguientes límites:
    \begin{enumerate}
        \item $\displaystyle \lim_{x\to 0} \frac{\displaystyle \int_0^x \arcsen(t)\arctan(t)dt}{(\ln (1+x))^3}$

        Como tenemos que el integrando es una función continua y acotada, tenemos que es Riemman Integrable. Por tanto, se puede emplear el TFC para calcular la derivada del numerador. Se empleará en la resolución del límite.
        \begin{multline*}
            \lim_{x\to 0} \frac{\displaystyle \int_0^x \arcsen(t)\arctan(t)dt}{(\ln (1+x))^3} = \left[\frac{0}{0}\right] \Hop
            \lim_{x\to 0} \frac{\arcsen(x)\arctan(x)}{3\frac{\ln^2(1+x)}{1+x}}
            =\\= \lim_{x\to 0} \frac{(1+x)\arcsen(x)\arctan(x)}{3 \ln^2(1+x)}
            = \left[\frac{0}{0}\right] \Hop
            \\= \lim_{x\to 0} \frac{\arcsen x\arctan x + \frac{(1+x)\arctan x}{\sqrt{1-x^2}} + \frac{(1+x)\arcsen x}{1+x^2}}{6 \frac{\ln(1+x)}{1+x}}
        \end{multline*}

        Aplicando que el límite de la suma es la suma de los límites, tenemos:
        \begin{multline*}
            \lim_{x\to 0} \frac{\displaystyle \int_0^x \arcsen(t)\arctan(t)dt}{(\ln (1+x))^3}
            =\\= \lim_{x\to 0} \frac{(1+x)\arcsen x\arctan x}{6\ln(1+x)} + \lim_{x\to 0} \frac{(1+x)^2\arctan x}{6\sqrt{1-x^2}\ln(1+x)}
            + \lim_{x\to 0} \frac{(1+x)^2\arcsen x }{6(1+x^2)\ln (1+x)}
        \end{multline*}

        Aplicando que el límite del producto es el producto de los límites, tenemos:
        \begin{multline*}
            \lim_{x\to 0} \frac{\displaystyle \int_0^x \arcsen(t)\arctan(t)dt}{(\ln (1+x))^3}
            =\\= 
            \lim_{x\to 0} \frac{(1+x)\arcsen x}{6}\lim_{x\to 0} \frac{\arctan x}{\ln(1+x)}
            + \lim_{x\to 0} \frac{(1+x)^2}{6\sqrt{1-x^2}} \lim_{x\to 0} \frac{\arctan x}{\ln(1+x)}
            +\\ + \lim_{x\to 0} \frac{(1+x)^2}{6(1+x^2)}
            \lim_{x\to 0} \frac{\arcsen x }{\ln (1+x)}
        \end{multline*}


        Calculo los siguientes límites:
        \begin{equation*}
            \lim_{x\to 0}\frac{\arctan x}{\ln 1+x} \Hop
            \lim_{x\to 0}\frac{1+x}{1+x^2} = 1
        \end{equation*}
        \begin{equation*}
            \lim_{x\to 0}\frac{\arcsen x}{\ln 1+x} \Hop
            \lim_{x\to 0}\frac{1+x}{\sqrt{1-x^2}} = 1
        \end{equation*}
        
        Por tanto, usando los límites calculados, tenemos que:
        \begin{multline*}
            \lim_{x\to 0} \frac{\displaystyle \int_0^x \arcsen(t)\arctan(t)dt}{(\ln (1+x))^3}
            =\\= 
            \lim_{x\to 0} \frac{(1+x)\arcsen x}{6}\lim_{x\to 0} \frac{\arctan x}{\ln(1+x)}
            + \lim_{x\to 0} \frac{(1+x)^2}{6\sqrt{1-x^2}} \lim_{x\to 0} \frac{\arctan x}{\ln(1+x)}
            +\\ + \lim_{x\to 0} \frac{(1+x)^2}{6(1+x^2)}
            \lim_{x\to 0} \frac{\arcsen x }{\ln (1+x)}
            = 0\cdot 1 + \frac{1}{6}\cdot 1 + \frac{1}{6}\cdot 1 = \frac{1}{3}
        \end{multline*}


        

        \item $\displaystyle \lim_{x\to 0} \left(\frac{2+\sen x}{2-\sen x}\right)^{\frac{1}{x}}$

        \begin{equation*}
            \lim_{x\to 0} \left(\frac{2+\sen x}{2-\sen x}\right)^{\frac{1}{x}}
            = \lim_{x\to 0} e^{\frac{\ln\left(\frac{2+\sen x}{2-\sen x}\right)}{x}} \stackrel{Ec.\;\ref{Ej4.1.Ind}}{=} e^1=e
        \end{equation*}
        donde previamente he resuelto esta indeterminación:
        \begin{multline}\label{Ej4.1.Ind}
            \lim_{x\to 0} \frac{\ln\left(\frac{2+\sen x}{2-\sen x}\right)}{x} = \left[\frac{0}{0}\right] \Hop
            \lim_{x\to 0} \frac{\frac{\cos x(2-\sen x)+\cos x(2+\sen x)}{(2-\sen x)^2}}{\frac{2+\sen x}{2-\sen x}}
            = \lim_{x\to 0} \frac{\frac{\cos x(2-\sen x)+\cos x(2+\sen x)}{2-\sen x}}{2+\sen x}
            =\\=
            \lim_{x\to 0} \frac{\cos x(2-\sen x)+\cos x(2+\sen x)}{(2+\sen x)(2-\sen x)}
            =
            \lim_{x\to 0} \frac{\cos x(2-\sen x+2+\sen x)}{4-\sen^2(x)}
            =\\=
            \lim_{x\to 0} \frac{4\cos x}{4-\sen^2(x)} = 1
        \end{multline}
    \end{enumerate}
\end{ejercicio}

\begin{ejercicio}
    Sea $f:I\to \bb{R}$ de clase $C^{n+1}(I),\;n\in \bb{N}$ y $P_{n,a}^f(x)$ su polinomio de Taylor de grado $n$ centrado en el punto $a\in I$. Probar que $\forall x\in I,\;x\neq a$ se cumple:
    \begin{equation*}
        f(x) - P_{n,a}^f(x) = \frac{1}{n!}\int_a^x f^{n+1)}(t) (x-t)^n\;dt
    \end{equation*}
    (Indicación: Inducción e integración por partes)\\

    Por el TFC, como $f$ es de clase $n+2$, $f^{n+2)}(x)$ es continua y por tanto Riemman Integrable en I, y por el TFC tenemos que:
        \begin{equation*}
            \int f^{p+1)}(x)\;dx = f^{p)}(x) \qquad \forall p\leq n+1
        \end{equation*}


    Demostramos ahora la igualdad por inducción sobre $n$:
    \begin{itemize}

        \item \underline{Para $n=0$}:
    
            Resolvemos la integral del término de la derecha.
            \begin{equation*}
                \int_a^x f'(t)\;dt = \left[f(t)\right]_a^x
            \end{equation*}
    
    
            Por tanto, la igualdad a demostrar es:
            \begin{equation*}
                f(x) - P_{0,a}^f(x) = \frac{1}{0!}\left[f(t)\right]_a^x
                \Longleftrightarrow 
                f(x)- \frac{f(a)}{0!}x^0 = f(x)-f(a)
            \end{equation*}

            Por tanto, tenemos que es cierto para $n=0$.
            
        \item \underline{Supuesto cierto para $n-1$, demostramos para $n$}:

        Resolvemos la integral del término de la derecha.
        \begin{multline*}
            \int_a^x f^{n+1)}(t) (x-t)^n\;dt = \MetInt{u(t)=(x-t)^n \qquad u'(t)=-n(x-t)^{n-1}}{v'(t)=f^{n+1)}(t) \qquad v(t)=f^{n)}(t)}
            =\\=
            \left[f^{n)}(t) (x-t)^n\right]_a^x +n\int_a^x f^{n)}(t)(x-t)^{n-1}\;dt
        \end{multline*}
        Usando la hipótesis de inducción, resolvemos la integral restante:
        \begin{multline*}
            \int_a^x f^{n+1)}(t) (x-t)^n\;dt =
            \left[f^{n)}(t) (x-t)^n\right]_a^x +n(n-1)!\left[f(x)-P_{n-1, a}^f (x)\right]
            =\\=
            -f^{n)}(a) (x-a)^n +n!\left[f(x)-P_{n-1, a}^f (x)\right]
        \end{multline*}

        Pasando el factorial dividiendo, tenemos que:
        \begin{equation*}
            \frac{1}{n!}\int_a^x f^{n+1)}(t) (x-t)^n\;dt
            =
            -\frac{f^{n)}(a)}{n!}(x-a)^n +\left[f(x)-P_{n-1, a}^f (x)\right]
        \end{equation*}

        No obstante, tenemos que el término $n-$ésimo del polinomio de Taylor en cuestión, por lo que tenemos:
        \begin{equation*}
            \frac{1}{n!}\int_a^x f^{n+1)}(t) (x-t)^n\;dt
            =
            f(x)-P_{n, a}^f (x)
        \end{equation*}

        Por tanto, lo tenemos demostrado para $n$.
    \end{itemize}
\end{ejercicio}
    



\end{document}