\documentclass[12pt]{article}

% Idioma y codificación
\usepackage[spanish, es-tabla]{babel}       %es-tabla para que se titule "Tabla"
\usepackage[utf8]{inputenc}

% Márgenes
\usepackage[a4paper,top=3cm,bottom=2.5cm,left=3cm,right=3cm]{geometry}

% Comentarios de bloque
\usepackage{verbatim}

% Paquetes de links
\usepackage[hidelinks]{hyperref}    % Permite enlaces
\usepackage{url}                    % redirecciona a la web

% Más opciones para enumeraciones
\usepackage{enumitem}

% Personalizar la portada
\usepackage{titling}

% Paquetes de tablas
\usepackage{multirow}


%------------------------------------------------------------------------

%Paquetes de figuras
\usepackage{caption}
\usepackage{subcaption} % Figuras al lado de otras
\usepackage{float}      % Poner figuras en el sitio indicado H.


% Paquetes de imágenes
\usepackage{graphicx}       % Paquete para añadir imágenes
\usepackage{transparent}    % Para manejar la opacidad de las figuras

% Paquete para usar colores
\usepackage[dvipsnames]{xcolor}
\usepackage{pagecolor}      % Para cambiar el color de la página

% Habilita tamaños de fuente mayores
\usepackage{fix-cm}

% Para los gráficos
\usepackage{tikz}

% Para poder situar los nodos en los grafos
\usetikzlibrary{positioning}


%------------------------------------------------------------------------

% Paquetes de matemáticas
\usepackage{mathtools, amsfonts, amssymb, mathrsfs}
\usepackage[makeroom]{cancel}     % Simplificar tachando
\usepackage{polynom}    % Divisiones y Ruffini
\usepackage{units} % Para poner fracciones diagonales con \nicefrac

\usepackage{pgfplots}   %Representar funciones
\pgfplotsset{compat=1.18}  % Versión 1.18

\usepackage{tikz-cd}    % Para usar diagramas de composiciones
\usetikzlibrary{calc}   % Para usar cálculo de coordenadas en tikz

%Definición de teoremas, etc.
\usepackage{amsthm}
%\swapnumbers   % Intercambia la posición del texto y de la numeración

\theoremstyle{plain}

\makeatletter
\@ifclassloaded{article}{
  \newtheorem{teo}{Teorema}[section]
}{
  \newtheorem{teo}{Teorema}[chapter]  % Se resetea en cada chapter
}
\makeatother

\newtheorem{coro}{Corolario}[teo]           % Se resetea en cada teorema
\newtheorem{prop}[teo]{Proposición}         % Usa el mismo contador que teorema
\newtheorem{lema}[teo]{Lema}                % Usa el mismo contador que teorema

\theoremstyle{remark}
\newtheorem*{observacion}{Observación}

\theoremstyle{definition}

\makeatletter
\@ifclassloaded{article}{
  \newtheorem{definicion}{Definición} [section]     % Se resetea en cada chapter
}{
  \newtheorem{definicion}{Definición} [chapter]     % Se resetea en cada chapter
}
\makeatother

\newtheorem*{notacion}{Notación}
\newtheorem*{ejemplo}{Ejemplo}
\newtheorem*{ejercicio*}{Ejercicio}             % No numerado
\newtheorem{ejercicio}{Ejercicio} [section]     % Se resetea en cada section


% Modificar el formato de la numeración del teorema "ejercicio"
\renewcommand{\theejercicio}{%
  \ifnum\value{section}=0 % Si no se ha iniciado ninguna sección
    \arabic{ejercicio}% Solo mostrar el número de ejercicio
  \else
    \thesection.\arabic{ejercicio}% Mostrar número de sección y número de ejercicio
  \fi
}


% \renewcommand\qedsymbol{$\blacksquare$}         % Cambiar símbolo QED
%------------------------------------------------------------------------

% Paquetes para encabezados
\usepackage{fancyhdr}
\pagestyle{fancy}
\fancyhf{}

\newcommand{\helv}{ % Modificación tamaño de letra
\fontfamily{}\fontsize{12}{12}\selectfont}
\setlength{\headheight}{15pt} % Amplía el tamaño del índice


%\usepackage{lastpage}   % Referenciar última pag   \pageref{LastPage}
\fancyfoot[C]{\thepage}

%------------------------------------------------------------------------

% Conseguir que no ponga "Capítulo 1". Sino solo "1."
\makeatletter
\@ifclassloaded{book}{
  \renewcommand{\chaptermark}[1]{\markboth{\thechapter.\ #1}{}} % En el encabezado
    
  \renewcommand{\@makechapterhead}[1]{%
  \vspace*{50\p@}%
  {\parindent \z@ \raggedright \normalfont
    \ifnum \c@secnumdepth >\m@ne
      \huge\bfseries \thechapter.\hspace{1em}\ignorespaces
    \fi
    \interlinepenalty\@M
    \Huge \bfseries #1\par\nobreak
    \vskip 40\p@
  }}
}
\makeatother

%------------------------------------------------------------------------
% Paquetes de cógido
\usepackage{minted}
\renewcommand\listingscaption{Código fuente}

\usepackage{fancyvrb}
% Personaliza el tamaño de los números de línea
\renewcommand{\theFancyVerbLine}{\small\arabic{FancyVerbLine}}

% Estilo para C++
\newminted{cpp}{
    frame=lines,
    framesep=2mm,
    baselinestretch=1.2,
    linenos,
    escapeinside=||
}

% para minted
\definecolor{LightGray}{rgb}{0.95,0.95,0.92}
\setminted{
    linenos=true,
    stepnumber=5,
    numberfirstline=true,
    autogobble,
    breaklines=true,
    breakautoindent=true,
    breaksymbolleft=,
    breaksymbolright=,
    breaksymbolindentleft=0pt,
    breaksymbolindentright=0pt,
    breaksymbolsepleft=0pt,
    breaksymbolsepright=0pt,
    fontsize=\footnotesize,
    bgcolor=LightGray,
    numbersep=10pt
}


\usepackage{listings} % Para incluir código desde un archivo

\renewcommand\lstlistingname{Código Fuente}
\renewcommand\lstlistlistingname{Índice de Códigos Fuente}

% Definir colores
\definecolor{vscodepurple}{rgb}{0.5,0,0.5}
\definecolor{vscodeblue}{rgb}{0,0,0.8}
\definecolor{vscodegreen}{rgb}{0,0.5,0}
\definecolor{vscodegray}{rgb}{0.5,0.5,0.5}
\definecolor{vscodebackground}{rgb}{0.97,0.97,0.97}
\definecolor{vscodelightgray}{rgb}{0.9,0.9,0.9}

% Configuración para el estilo de C similar a VSCode
\lstdefinestyle{vscode_C}{
  backgroundcolor=\color{vscodebackground},
  commentstyle=\color{vscodegreen},
  keywordstyle=\color{vscodeblue},
  numberstyle=\tiny\color{vscodegray},
  stringstyle=\color{vscodepurple},
  basicstyle=\scriptsize\ttfamily,
  breakatwhitespace=false,
  breaklines=true,
  captionpos=b,
  keepspaces=true,
  numbers=left,
  numbersep=5pt,
  showspaces=false,
  showstringspaces=false,
  showtabs=false,
  tabsize=2,
  frame=tb,
  framerule=0pt,
  aboveskip=10pt,
  belowskip=10pt,
  xleftmargin=10pt,
  xrightmargin=10pt,
  framexleftmargin=10pt,
  framexrightmargin=10pt,
  framesep=0pt,
  rulecolor=\color{vscodelightgray},
  backgroundcolor=\color{vscodebackground},
}

%------------------------------------------------------------------------

% Comandos definidos
\newcommand{\bb}[1]{\mathbb{#1}}
\newcommand{\cc}[1]{\mathcal{#1}}

% I prefer the slanted \leq
\let\oldleq\leq % save them in case they're every wanted
\let\oldgeq\geq
\renewcommand{\leq}{\leqslant}
\renewcommand{\geq}{\geqslant}

% Si y solo si
\newcommand{\sii}{\iff}

% Letras griegas
\newcommand{\eps}{\epsilon}
\newcommand{\veps}{\varepsilon}
\newcommand{\lm}{\lambda}

\newcommand{\ol}{\overline}
\newcommand{\ul}{\underline}
\newcommand{\wt}{\widetilde}
\newcommand{\wh}{\widehat}

\let\oldvec\vec
\renewcommand{\vec}{\overrightarrow}

% Derivadas parciales
\newcommand{\del}[2]{\frac{\partial #1}{\partial #2}}
\newcommand{\Del}[3]{\frac{\partial^{#1} #2}{\partial #3^{#1}}}
\newcommand{\deld}[2]{\dfrac{\partial #1}{\partial #2}}
\newcommand{\Deld}[3]{\dfrac{\partial^{#1} #2}{\partial #3^{#1}}}


\newcommand{\AstIg}{\stackrel{(\ast)}{=}}
\newcommand{\Hop}{\stackrel{L'H\hat{o}pital}{=}}

\newcommand{\red}[1]{{\color{red}#1}} % Para integrales, destacar los cambios.

% Método de integración
\newcommand{\MetInt}[2]{
    \left[\begin{array}{c}
        #1 \\ #2
    \end{array}\right]
}

% Declarar aplicaciones
% 1. Nombre aplicación
% 2. Dominio
% 3. Codominio
% 4. Variable
% 5. Imagen de la variable
\newcommand{\Func}[5]{
    \begin{equation*}
        \begin{array}{rrll}
            #1:& #2 & \longrightarrow & #3\\
               & #4 & \longmapsto & #5
        \end{array}
    \end{equation*}
}

%------------------------------------------------------------------------



\begin{document}

    % 1. Foto de fondo
    % 2. Título
    % 3. Encabezado Izquierdo
    % 4. Color de fondo
    % 5. Coord x del titulo
    % 6. Coord y del titulo
    % 7. Fecha

    
    % 1. Foto de fondo
% 2. Título
% 3. Encabezado Izquierdo
% 4. Color de fondo
% 5. Coord x del titulo
% 6. Coord y del titulo
% 7. Fecha

\newcommand{\portada}[7]{

    \portadaBase{#1}{#2}{#3}{#4}{#5}{#6}{#7}
    \portadaBook{#1}{#2}{#3}{#4}{#5}{#6}{#7}
}

\newcommand{\portadaExamen}[7]{

    \portadaBase{#1}{#2}{#3}{#4}{#5}{#6}{#7}
    \portadaArticle{#1}{#2}{#3}{#4}{#5}{#6}{#7}
}




\newcommand{\portadaBase}[7]{

    % Tiene la portada principal y la licencia Creative Commons
    
    % 1. Foto de fondo
    % 2. Título
    % 3. Encabezado Izquierdo
    % 4. Color de fondo
    % 5. Coord x del titulo
    % 6. Coord y del titulo
    % 7. Fecha
    
    
    \thispagestyle{empty}               % Sin encabezado ni pie de página
    \newgeometry{margin=0cm}        % Márgenes nulos para la primera página
    
    
    % Encabezado
    \fancyhead[L]{\helv #3}
    \fancyhead[R]{\helv \nouppercase{\leftmark}}
    
    
    \pagecolor{#4}        % Color de fondo para la portada
    
    \begin{figure}[p]
        \centering
        \transparent{0.3}           % Opacidad del 30% para la imagen
        
        \includegraphics[width=\paperwidth, keepaspectratio]{assets/#1}
    
        \begin{tikzpicture}[remember picture, overlay]
            \node[anchor=north west, text=white, opacity=1, font=\fontsize{60}{90}\selectfont\bfseries\sffamily, align=left] at (#5, #6) {#2};
            
            \node[anchor=south east, text=white, opacity=1, font=\fontsize{12}{18}\selectfont\sffamily, align=right] at (9.7, 3) {\textbf{\href{https://losdeldgiim.github.io/}{Los Del DGIIM}}};
            
            \node[anchor=south east, text=white, opacity=1, font=\fontsize{12}{15}\selectfont\sffamily, align=right] at (9.7, 1.8) {Doble Grado en Ingeniería Informática y Matemáticas\\Universidad de Granada};
        \end{tikzpicture}
    \end{figure}
    
    
    \restoregeometry        % Restaurar márgenes normales para las páginas subsiguientes
    \pagecolor{white}       % Restaurar el color de página
    
    
    \newpage
    \thispagestyle{empty}               % Sin encabezado ni pie de página
    \begin{tikzpicture}[remember picture, overlay]
        \node[anchor=south west, inner sep=3cm] at (current page.south west) {
            \begin{minipage}{0.5\paperwidth}
                \href{https://creativecommons.org/licenses/by-nc-nd/4.0/}{
                    \includegraphics[height=2cm]{assets/Licencia.png}
                }\vspace{1cm}\\
                Esta obra está bajo una
                \href{https://creativecommons.org/licenses/by-nc-nd/4.0/}{
                    Licencia Creative Commons Atribución-NoComercial-SinDerivadas 4.0 Internacional (CC BY-NC-ND 4.0).
                }\\
    
                Eres libre de compartir y redistribuir el contenido de esta obra en cualquier medio o formato, siempre y cuando des el crédito adecuado a los autores originales y no persigas fines comerciales. 
            \end{minipage}
        };
    \end{tikzpicture}
    
    
    
    % 1. Foto de fondo
    % 2. Título
    % 3. Encabezado Izquierdo
    % 4. Color de fondo
    % 5. Coord x del titulo
    % 6. Coord y del titulo
    % 7. Fecha


}


\newcommand{\portadaBook}[7]{

    % 1. Foto de fondo
    % 2. Título
    % 3. Encabezado Izquierdo
    % 4. Color de fondo
    % 5. Coord x del titulo
    % 6. Coord y del titulo
    % 7. Fecha

    % Personaliza el formato del título
    \pretitle{\begin{center}\bfseries\fontsize{42}{56}\selectfont}
    \posttitle{\par\end{center}\vspace{2em}}
    
    % Personaliza el formato del autor
    \preauthor{\begin{center}\Large}
    \postauthor{\par\end{center}\vfill}
    
    % Personaliza el formato de la fecha
    \predate{\begin{center}\huge}
    \postdate{\par\end{center}\vspace{2em}}
    
    \title{#2}
    \author{\href{https://losdeldgiim.github.io/}{Los Del DGIIM}}
    \date{Granada, #7}
    \maketitle
    
    \tableofcontents
}




\newcommand{\portadaArticle}[7]{

    % 1. Foto de fondo
    % 2. Título
    % 3. Encabezado Izquierdo
    % 4. Color de fondo
    % 5. Coord x del titulo
    % 6. Coord y del titulo
    % 7. Fecha

    % Personaliza el formato del título
    \pretitle{\begin{center}\bfseries\fontsize{42}{56}\selectfont}
    \posttitle{\par\end{center}\vspace{2em}}
    
    % Personaliza el formato del autor
    \preauthor{\begin{center}\Large}
    \postauthor{\par\end{center}\vspace{3em}}
    
    % Personaliza el formato de la fecha
    \predate{\begin{center}\huge}
    \postdate{\par\end{center}\vspace{5em}}
    
    \title{#2}
    \author{\href{https://losdeldgiim.github.io/}{Los Del DGIIM}}
    \date{Granada, #7}
    \thispagestyle{empty}               % Sin encabezado ni pie de página
    \maketitle
    \vfill
}
    \portadaExamen{ffccA4.jpg}{Cálculo II\\Examen III}{Cálculo II. Examen III}{MidnightBlue}{-8}{28}{2023}{Arturo Olivares Martos}

    \begin{description}
        \item[Asignatura] Cálculo II.
        \item[Curso Académico] 2021-22.
        \item[Grado] Doble Grado en Ingeniería Informática y Matemáticas.
        \item[Grupo] Único.
        \item[Profesor] María Victoria Velasco Collado.
        \item[Descripción] Convocatoria Ordinaria.
        \item[Fecha] 22 de junio de 2022.
        %\item[Duración] 60 minutos.
    
    \end{description}
    \newpage
    
    \begin{ejercicio}\textbf{[2.5 puntos]}
    Sea $f:\bb{R}^+\to \bb{R}$ una función continua. Dado $a>0$, sea:
    \begin{equation*}
        g(x):=e^{-ax}\int_0^xe^{at}f(t)\;dt
    \end{equation*}


    \begin{enumerate}
        \item Demostrar que $g$ es constante si y solo si se verifica $f(x)=ag(x),\;\forall x\in \bb{R}^+$.

        Como $f$ es continua y la exponencial también, tenemos que el integrando es continuo y Riemman Integrable. Por tanto, por el TFC, tenemos que:$$\left(\int_0^xe^{at}f(t)\;dt\right)' = e^{ax}f(x)$$

        Por tanto, como la integral es derivable y la exponencial también, tenemos que $g$ es derivable en $\bb{R}^+$, con:
        \begin{equation*}
            g'(x)=-ae^{-ax}\int_0^xe^{at}f(t)\;dt + e^{-ax}e^{ax}f(x)
            =-ae^{-ax}\int_0^xe^{at}f(t)\;dt + f(x)
        \end{equation*}

        Como $g$ es derivable en $\bb{R}^+$, tenemos que $g$ es constante si su primer derivada es nula. Entonces:
        \begin{equation*}
            g'(x)=0 \;\;\forall x\in \bb{R}^+\Longleftrightarrow f(x)=ae^{-ax}\int_0^xe^{at}f(t)\;dt = ag(x) \qquad \forall x\in \bb{R}^+
        \end{equation*}

        Por tanto, se ha demostrado lo pedido.

        \item Sabiendo que $\displaystyle \lim_{x\to \infty}f(x)=L>0$, estudiar la convergencia de la integral $\displaystyle \int_0^\infty e^{at}f(t)\;dt$ y calcular $\displaystyle \lim_{x\to \infty}g(x)$.

        \begin{observacion}
            Aunque en el examen no está especificado, creo que se puede suponer que $L\in \bb{R}$, por la notación usual de los límites convergentes.
        \end{observacion}

        Tenemos:
        \begin{equation*}
            \lim_{x\to \infty} \frac{e^{ax}f(x)}{e^{ax}}
            = \lim_{x\to \infty} f(x) = L>0
        \end{equation*}

        Por tanto, tenemos que
        \begin{equation*}
            \int_0^\infty e^{at}f(t)\;dt \text{ diverge positivamente}
            \Longleftrightarrow
            \int_0^\infty e^{at}\;dt \text{ diverge positivamente}
        \end{equation*}

        Calculamos por tanto la segunda integral para ver si converge:
        \begin{equation*}
            \int_0^\infty e^{at}\;dt
            = \lim_{c\to \infty} \frac{1}{a}\int_0^c ae^{at}\;dt
            = \lim_{c\to \infty} \frac{1}{a}\left[e^{at}\right]_0^c
            = \frac{1}{a}\left[\infty - e^0 \right] = \infty
        \end{equation*}

        Por tanto, tenemos que $\displaystyle \int_0^\infty e^{at}f(t)\;dt$ diverge positivamente.

        Además, tenemos que:
        \begin{equation*}
            \lim_{x\to \infty}g(x)=
            \lim_{x\to \infty}\frac{\int_0^x e^{at}f(t)\;dt}{e^{ax}} = \left[\frac{\infty}{\infty}\right]
            \Hop
            \lim_{x\to \infty}\frac{e^{ax}f(x)}{ae^{ax}}
            = \lim_{x\to \infty}\frac{f(x)}{a}= \frac{L}{a}
        \end{equation*}

        \item Demostrar que si $f(\bb{R}^+)\subseteq [m,M]$, entonces:
        \begin{equation*}
            \frac{m}{a}\frac{(e^{ax}-1)}{e^{ax}}\leq g(x)\leq \frac{M}{a}\frac{(e^{ax}-1)}{e^{ax}}
        \end{equation*}
        (por lo que $g$ es una función acotada cuando $f$ lo es).

        Tenemos que:
        \begin{equation*}
            m\leq f(x)\leq M
            \Longrightarrow 
            e^{ax}m\leq e^{ax}f(x)\leq e^{ax}M \hspace{1.5cm} \forall x\in \bb{R}^+
        \end{equation*}

        donde he usado que, como $a>0$, $e^{ax}>0$. Como el operador integral mantiene el orden:
        \begin{equation*}
            \int_0^x e^{at}m\;dt\leq \int_0^x e^{at}f(t)\;dt\leq \int_0^x e^{at}M\;dt
        \end{equation*}

        Tenemos que:
        \begin{equation*}
            \int_0^x e^{at}\;dt = \frac{1}{a}\left[e^{ax}-1\right]
        \end{equation*}

        Por tanto:
        \begin{equation*}
            \frac{m}{a}\left[e^{ax}-1\right] \leq \int_0^x e^{at}f(t)\;dt\leq \frac{M}{a}\left[e^{ax}-1\right]
        \end{equation*}

        Para obtener $g(x)$, divido por $e^{ax}>0$:
        \begin{equation*}
            \frac{m}{a}\frac{(e^{ax}-1)}{e^{ax}} \leq g(x)\leq \frac{M}{a}\frac{(e^{ax}-1)}{e^{ax}}
        \end{equation*}

        Por tanto, se ha demostrado lo pedido.
    \end{enumerate}
\end{ejercicio}


\begin{ejercicio}\textbf{[1.5 puntos]}
Determinar las dimensiones del rectángulo de área máxima inscrito en un círculo de radio $r=\frac{1}{2}$.

    \begin{figure}[H]
        \centering
        \begin{tikzpicture}

            \def\r{2}
            \def\angle{30}
            
            \draw (0,0) circle (\r); % Circulo de radio \r
            
            \coordinate (A) at ({\r*cos(\angle)},{\r*sin(\angle)}); % Vértice superior derecho del rectángulo
            \coordinate (B) at ({\r*cos(\angle)},{-\r*sin(\angle)}); % Vértice inferior derecho del rectángulo
            \coordinate (C) at ({-\r*cos(\angle)},{-\r*sin(\angle)}); % Vértice inferior izquierdo del rectángulo
            \coordinate (D) at ({-\r*cos(\angle)},{\r*sin(\angle)}); % Vértice superior izquierdo del rectángulo
            
            \draw (A) -- node[left]{$h$} (B) -- (C) -- (D) -- node[above]{$b$} cycle; % Rectángulo
            
            % Líneas para ilustrar el círculo
            \draw[dashed] (0,0) -- (A);
            \draw[dashed] (0,0) -- (B);
            \draw[dashed] (0,0) -- (C);
            \draw[dashed] (0,0) -- (D);

            \draw[-stealth] (0,0) -- node[below left]{$r$} (0, -\r);
            
            % Etiquetas de los vértices
            \node[above right] at (A) {A};
            \node[below right] at (B) {B};
            \node[below left] at (C) {C};
            \node[above left] at (D) {D};
        \end{tikzpicture}
    \end{figure}

    En este caso, tenemos que la ecuación de ligadura viene dada por el Teorema de Pitágoras:
    \begin{equation*}
        b^2+h^2=(2r)^2=4r^2 \Longrightarrow b=\sqrt{4r^2-h^2}
    \end{equation*}

    Por tanto, la función a maximizar es:
    \begin{equation*}
        \begin{array}{rl}
            A:\bb{R^+} & \longrightarrow \bb{R}\\
                    h & \longmapsto A(h)=bh=h\sqrt{4r^2-h^2}
        \end{array}
    \end{equation*}

    Como $A(h)\geq 0\quad \forall h\in \bb{R}^+$, tenemos que $A(h)$ y $A^2(h)$ alcanzan los extremos relativos en los mismos valores de las abcisas. Por tanto, maximizo $A^2(h)$:
    \begin{multline*}
        A^2(h)=h^2(4r^2-h^2) = 4h^2r^2-h^4 \Longrightarrow
        \frac{\partial A^2}{\partial h} = 8hr^2-4h^3 = 0 \Longleftrightarrow 4h(2r^2-h^2)=0
        \Longleftrightarrow\\ \Longleftrightarrow h=0,\sqrt{2}r
    \end{multline*}

    Por tanto, como $h=0$ no pertenece al dominio de la función, tenemos que $h=\sqrt{2}r$ es el único candidato a extremo relativo. Estudiemos la monotonía.
    \begin{itemize}
        \item \underline{Para $h<\sqrt{2}r$}: $\frac{\partial A^2}{\partial h}>0\Longrightarrow A^2(h)$ estrictamente creciente.
        \item \underline{Para $h>\sqrt{2}r$}: $\frac{\partial A^2}{\partial h}<0\Longrightarrow A^2(h)$ estrictamente decreciente.
    \end{itemize}

    Por tanto, tenemos que $h=\sqrt{2}r$ es un máximo relativo. Además, es absoluto por el el único extremo relativo y ser continua. Por tanto, las dimensiones pedidas son:
    \begin{equation*}
        h=\sqrt{2}r = \frac{\sqrt{2}}{2}
        \qquad
        b=\sqrt{4r^2-h^2}=\sqrt{4r^2-2r^2}=\sqrt{2r^2} = \frac{\sqrt{2}}{2}
    \end{equation*}
    
\end{ejercicio}

\begin{ejercicio}\textbf{[1 punto]}
Demostrar que:
\begin{equation*}
    |\arctan x - \arctan y|\leq |x-y|, \hspace{1cm} \forall x,y\in \bb{R}.
\end{equation*}

¿Es $f(x)=\arctan x$ una función uniformemente continua?\\

Demostramos en primer lugar que $f$ es lipschitziana. Como es derivable en $\bb{R}$, veamos si su derivada está acotada.
\begin{equation*}
    0\leq f'(x)=\frac{1}{1+x^2} \leq 1
\end{equation*}

Por tanto, tenemos que su derivada está acotada por $1$. Por tanto, $f$ es lipschitziana y, por tanto, uniformemente continua. Además, como la constante de Lipschitiz es 1, tenemos que:
\begin{equation*}
|f(y)<f(x)| \leq M|x-y| \leq 1\cdot |x-y| \qquad \forall x\in \bb{R}
\end{equation*}

Por tanto, se ha demostrado lo pedido.

\end{ejercicio}


\begin{ejercicio}\textbf{[3 puntos]}
Sea $f(x)=\ln (1+x)$.
\begin{enumerate}
    \item Calcular el Polinomio de Taylor de $f(x)$ de grado $n$ centrado en el origen, dando una expresión del resto.\\

    Demostramos en primer lugar mediante inducción la derivada $n-$ésima de $f(x)$:
    \begin{equation*}
        f^{(n)}(x)=(-1)^{n-1}\cdot \frac{(n-1)!}{(1+x)^n}
    \end{equation*}
    \begin{itemize}
        \item \underline{Demostramos para $n=1$}:
        \begin{equation*}
            f'(x)=\frac{1}{1+x} = 1^0\cdot \frac{0!}{(1+x)^1}
        \end{equation*}

        Por tanto, para $n=1$ se tiene.

        \item \underline{Supuesto cierto para $n$, se comprueba para $n+1$}:
        \begin{equation*}
            f^{(n+1)}(x)=(-1)^{n-1}\cdot (n-1)!\cdot \frac{-n(1+x)^{n-1}}{(1+x)^{2n}}
            = (-1)^{n}\cdot (n)!\cdot \frac{1}{(1+x)^{n+1}}
            = (-1)^{n}\cdot \frac{n!}{(1+x)^{n+1}}
        \end{equation*}
        Por tanto, se ha demostrado para $n+1$.
    \end{itemize}

    Por tanto, el polinomio de Taylor de grado $n$ centrado en el origen es:
    \begin{equation*}
        P_{n,0}^f(x)=f(0) + \sum_{k=1}^n \frac{f^{(n)}(0)}{k!}(x)^k
        =\ln 1 + \sum_{k=1}^n \frac{(-1)^{k-1}\cdot \frac{(k-1)!}{(1)^k}}{k!}(x)^k
        =\sum_{k=1}^n (-1)^{k-1}\cdot \frac{x^k}{k}
    \end{equation*}

    El resto de Lagrage es, para cierto $c\in ]-1,\infty[$:
    \begin{multline*}
        R_{n,0}^{f}(x):=f(x)-P_{n,0}^f(x) = \frac{f^{(n+1)}(c)}{(n+1)!}x^{n+1}
        = \frac{(-1)^{n}\cdot \frac{n!}{(1+c)^{n+1}}}{(n+1)!}x^{n+1}
        =\\= (-1)^{n}\cdot \frac{x^{n+1}}{(n+1)(1+c)^{n+1}}
    \end{multline*}

    \item Calcular $\displaystyle \lim_{x\to 0}\frac{x\ln (1+x)-x^2+\frac{1}{2}x^3-\frac{1}{3}x^4}{x^5}$.
    \begin{multline*}
        \lim_{x\to 0}\frac{x\ln (1+x)-x^2+\frac{1}{2}x^3-\frac{1}{3}x^4}{x^5}
        = \lim_{x\to 0}\frac{\ln (1+x)-x+\frac{1}{2}x^2-\frac{1}{3}x^3 {\color{red} -\frac{1}{4}x^4 +\frac{1}{4}x^4}}{x^4}
        =\\= \lim_{x\to 0}\frac{f(x)-P_{4,0}^f(x) -\frac{1}{4}x^4}{x^4}
        = \lim_{x\to 0}\frac{f(x)-P_{4,0}^f(x)}{x^4}
        - \frac{\frac{1}{4}x^4}{x^4}
        = \lim_{x\to 0}\frac{f(x)-P_{4,0}^f(x)}{x^4}
        - \frac{1}{4}
        =\\= 0- \frac{1}{4}=- \frac{1}{4}
    \end{multline*}

    donde he empleado el Teorema Infinetesimal del Resto.

    \item Calcular el área de la región determinada por la gráfica de $f(x)$ en el intervalo $\left[-\frac{1}{2},1\right]$.

    Veamos en qué valores corta la gráfica al eje $X$:
    \begin{equation*}
        f(x)=0 \Longleftrightarrow \ln (1+x)=0 \Longleftrightarrow x=0
    \end{equation*}

    Por tanto, a partir de la gráfica del $\ln x$, sacamos que la gráfica de $f(x)$ es la siguiente:
    \begin{figure}[H]
        \centering
        
        \begin{tikzpicture}[scale=1.5]

            \def\a{-0.5};
            \def\b{1};
            
            % Eje de abscisas
            \draw[-stealth] (\a-1,0) -- (\b+1,0) node[right] {$x$};
            % Eje de ordenadas
            \draw[-stealth] (0,-1) -- (0,2) node[above] {$y$};
            
            % Gráfica de f(x) = ln(1+x)
            \draw[domain=\a-0.3:\b+0.5,smooth,variable=\x,blue] plot ({\x},{ln(1+\x)});
            
            % Recta x = a
            \draw[dashed] (\a,-1) -- (\a,2) node[above left] {$x=-\frac{1}{2}$};
            
            % Recta x = a
            \draw[dashed] (\b,-1) -- (\b,2) node[above] {$x=1$};

            
            % Región acotada
            \fill[gray!30,domain=\a:0,smooth,variable=\x] plot ({\x},{ln(1+\x)}) -- (0,0) -- (\a,0) -- cycle;
            \fill[gray!30,domain=0:\b,smooth,variable=\x] plot ({\x},{ln(1+\x)}) -- (\b,0) -- (0,0) -- cycle;
            
            % Puntos de intersección
            \filldraw[black] (\b,{ln(1+\b)}) circle (2pt);
            \filldraw[black] (\a,{ln(1+\a)}) circle (2pt);
            \filldraw[black] (0,0) circle (2pt);
            
        \end{tikzpicture}
    \end{figure}

    En primer lugar, resolvemos la integral indefinida:
    \begin{multline*}
        \int \ln (1+x)\;dx = \MetInt{u(x)=\ln (1+x) \qquad u'(x)=\frac{1}{1+x}}{v'(x)=1 \qquad v(x)=x}
        = x\ln (1+x) -\int \frac{x {\color{red}+1-1}}{1+x}\;dx
        =\\= x\ln (1+x) -\int 1-\frac{1}{1+x}\;dx
        = x\ln (1+x) -x +\ln|1+x|+C
    \end{multline*}

    Por tanto, tenemos que el área es:
    \begin{multline*}
        A = \int_{-\frac{1}{2}}^1 |f(x)|\;dx
        = \int_{-\frac{1}{2}}^0 -f(x)\;dx
        + \int_{0}^1 f(x)\;dx
        =\\= \left[x\ln (1+x) -x +\ln|1+x|\right]^{-\frac{1}{2}}_0
        + \left[x\ln (1+x) -x +\ln|1+x|\right]_0^1
        =\\= -\frac{1}{2}\ln \left(\frac{1}{2}\right) +\frac{1}{2}+\ln \left(\frac{1}{2}\right) +\ln 2 -1 +\ln 2
        = \frac{1}{2}\ln \left(\frac{1}{2}\right) -\frac{1}{2}+ 2\ln 2
        = -\frac{1}{2}\ln 2 -\frac{1}{2}+ 2\ln 2
        =\\= \frac{3}{2}\ln 2 -\frac{1}{2} \quad u^2
    \end{multline*}
    
\end{enumerate}
\end{ejercicio}

\begin{ejercicio}
    \textbf{[2 puntos]} Calcular la longitud de la curva $f(x)=\ln \frac{e^x+1}{e^x-1}$ entre $x=1$ y $x=2$.

    Tenemos que $f$ restringida al intervalo [1,2] es continua. Calculemos su derivada.
    \begin{multline*}
        f'(x)=\frac{e^x-1}{e^x+1}\cdot \frac{e^x(e^x-1) -e^x(e^x+1)}{(e^x-1)^2}
        = \frac{e^x-1}{e^x+1}\cdot \frac{e^x(e^x-1-e^x -1)}{(e^x-1)^2}
        = \frac{-2e^x}{(e^x-1)(e^x+1)}
        =\\
        = \frac{-2e^x}{e^{2x}-1}
    \end{multline*}

    Por tanto, como tanto $f$ como $f'$ son continuas en $[1,2]$, tenemos que $f\in C^1[1,2]$. Por tanto, su longitud en este intervalo es:
    \begin{multline*}
        l=\int_1^2 \sqrt{1+[f'(x)]^2}\;dx
        =\int_1^2 \sqrt{1+\frac{4e^{2x}}{(e^{2x}-1)^2}}\;dx = \MetInt{e^{2x}=t \quad t\in [e^2, e^4]}{2e^{2x}dx=dt}
        =\\
        = \int_{e^2}^{e^4} \sqrt{1+\frac{4t}{(t-1)^2}}\;\frac{dt}{2t}
        = \int_{e^2}^{e^4} \sqrt{\frac{t^2+1-2t+4t}{(t-1)^2}}\;\frac{dt}{2t}
        = \int_{e^2}^{e^4} \sqrt{\frac{(t+1)^2}{(t-1)^2}}\;\frac{dt}{2t}
        = \int_{e^2}^{e^4} \left|\frac{t+1}{t-1}\right|\;\frac{dt}{2t} =\\
        = \frac{1}{2}\int_{e^2}^{e^4} \frac{t}{t-1}\;\frac{dt}{t}
        + \frac{1}{2}\int_{e^2}^{e^4} \frac{1}{t-1}\;\frac{dt}{t}
        = \frac{1}{2}\int_{e^2}^{e^4} \frac{1}{t-1}\;dt
        + \frac{1}{2}\int_{e^2}^{e^4} \frac{1}{t(t-1)}\;dt
    \end{multline*}

    Aplico el método de coeficientes indeterminados para resolver la integral:
    \begin{equation*}
        \frac{1}{t(t-1)} = \frac{A}{t} + \frac{B}{t-1} = \frac{A(t-1)+Bt}{t(t-1)}
    \end{equation*}
    \begin{itemize}
        \item \underline{Para $t=0$}: $1=-A\Longrightarrow A=-1$
        \item \underline{Para $t=1$}: $1=B$
    \end{itemize}

    Por tanto, tenemos que:
    \begin{multline*}
        l= \frac{1}{2}\int_{e^2}^{e^4} \frac{1}{t-1}\;dt
        + \frac{1}{2}\int_{e^2}^{e^4} \frac{1}{t(t-1)}\;dt = \frac{1}{2}\left[\ln |t-1| -\ln t +\ln |t-1|\right]_{e^2}^{e^4}=\\
        = \left[\ln |t-1| -\frac{1}{2}\ln t \right]_{e^2}^{e^4}
        = \left[\ln |t-1| -\ln \sqrt{t} \right]_{e^2}^{e^4} = \ln(e^4-1) -2 -\ln (e^2-1) +1
        =\\= \ln \frac{e^4-1}{e^2-1} -1
        = \ln(e^2+1) -1
    \end{multline*}

    Por tanto, tenemos que la longitud de esa curva es rectificable, con: $$l=\ln (e^2+1) -1\;u.$$
\end{ejercicio}



\end{document}