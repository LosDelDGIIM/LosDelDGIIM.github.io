\documentclass[12pt]{article}

% Idioma y codificación
\usepackage[spanish, es-tabla]{babel}       %es-tabla para que se titule "Tabla"
\usepackage[utf8]{inputenc}

% Márgenes
\usepackage[a4paper,top=3cm,bottom=2.5cm,left=3cm,right=3cm]{geometry}

% Comentarios de bloque
\usepackage{verbatim}

% Paquetes de links
\usepackage[hidelinks]{hyperref}    % Permite enlaces
\usepackage{url}                    % redirecciona a la web

% Más opciones para enumeraciones
\usepackage{enumitem}

% Personalizar la portada
\usepackage{titling}

% Paquetes de tablas
\usepackage{multirow}


%------------------------------------------------------------------------

%Paquetes de figuras
\usepackage{caption}
\usepackage{subcaption} % Figuras al lado de otras
\usepackage{float}      % Poner figuras en el sitio indicado H.


% Paquetes de imágenes
\usepackage{graphicx}       % Paquete para añadir imágenes
\usepackage{transparent}    % Para manejar la opacidad de las figuras

% Paquete para usar colores
\usepackage[dvipsnames]{xcolor}
\usepackage{pagecolor}      % Para cambiar el color de la página

% Habilita tamaños de fuente mayores
\usepackage{fix-cm}

% Para los gráficos
\usepackage{tikz}

% Para poder situar los nodos en los grafos
\usetikzlibrary{positioning}


%------------------------------------------------------------------------

% Paquetes de matemáticas
\usepackage{mathtools, amsfonts, amssymb, mathrsfs}
\usepackage[makeroom]{cancel}     % Simplificar tachando
\usepackage{polynom}    % Divisiones y Ruffini
\usepackage{units} % Para poner fracciones diagonales con \nicefrac

\usepackage{pgfplots}   %Representar funciones
\pgfplotsset{compat=1.18}  % Versión 1.18

\usepackage{tikz-cd}    % Para usar diagramas de composiciones
\usetikzlibrary{calc}   % Para usar cálculo de coordenadas en tikz

%Definición de teoremas, etc.
\usepackage{amsthm}
%\swapnumbers   % Intercambia la posición del texto y de la numeración

\theoremstyle{plain}

\makeatletter
\@ifclassloaded{article}{
  \newtheorem{teo}{Teorema}[section]
}{
  \newtheorem{teo}{Teorema}[chapter]  % Se resetea en cada chapter
}
\makeatother

\newtheorem{coro}{Corolario}[teo]           % Se resetea en cada teorema
\newtheorem{prop}[teo]{Proposición}         % Usa el mismo contador que teorema
\newtheorem{lema}[teo]{Lema}                % Usa el mismo contador que teorema

\theoremstyle{remark}
\newtheorem*{observacion}{Observación}

\theoremstyle{definition}

\makeatletter
\@ifclassloaded{article}{
  \newtheorem{definicion}{Definición} [section]     % Se resetea en cada chapter
}{
  \newtheorem{definicion}{Definición} [chapter]     % Se resetea en cada chapter
}
\makeatother

\newtheorem*{notacion}{Notación}
\newtheorem*{ejemplo}{Ejemplo}
\newtheorem*{ejercicio*}{Ejercicio}             % No numerado
\newtheorem{ejercicio}{Ejercicio} [section]     % Se resetea en cada section


% Modificar el formato de la numeración del teorema "ejercicio"
\renewcommand{\theejercicio}{%
  \ifnum\value{section}=0 % Si no se ha iniciado ninguna sección
    \arabic{ejercicio}% Solo mostrar el número de ejercicio
  \else
    \thesection.\arabic{ejercicio}% Mostrar número de sección y número de ejercicio
  \fi
}


% \renewcommand\qedsymbol{$\blacksquare$}         % Cambiar símbolo QED
%------------------------------------------------------------------------

% Paquetes para encabezados
\usepackage{fancyhdr}
\pagestyle{fancy}
\fancyhf{}

\newcommand{\helv}{ % Modificación tamaño de letra
\fontfamily{}\fontsize{12}{12}\selectfont}
\setlength{\headheight}{15pt} % Amplía el tamaño del índice


%\usepackage{lastpage}   % Referenciar última pag   \pageref{LastPage}
\fancyfoot[C]{\thepage}

%------------------------------------------------------------------------

% Conseguir que no ponga "Capítulo 1". Sino solo "1."
\makeatletter
\@ifclassloaded{book}{
  \renewcommand{\chaptermark}[1]{\markboth{\thechapter.\ #1}{}} % En el encabezado
    
  \renewcommand{\@makechapterhead}[1]{%
  \vspace*{50\p@}%
  {\parindent \z@ \raggedright \normalfont
    \ifnum \c@secnumdepth >\m@ne
      \huge\bfseries \thechapter.\hspace{1em}\ignorespaces
    \fi
    \interlinepenalty\@M
    \Huge \bfseries #1\par\nobreak
    \vskip 40\p@
  }}
}
\makeatother

%------------------------------------------------------------------------
% Paquetes de cógido
\usepackage{minted}
\renewcommand\listingscaption{Código fuente}

\usepackage{fancyvrb}
% Personaliza el tamaño de los números de línea
\renewcommand{\theFancyVerbLine}{\small\arabic{FancyVerbLine}}

% Estilo para C++
\newminted{cpp}{
    frame=lines,
    framesep=2mm,
    baselinestretch=1.2,
    linenos,
    escapeinside=||
}

% para minted
\definecolor{LightGray}{rgb}{0.95,0.95,0.92}
\setminted{
    linenos=true,
    stepnumber=5,
    numberfirstline=true,
    autogobble,
    breaklines=true,
    breakautoindent=true,
    breaksymbolleft=,
    breaksymbolright=,
    breaksymbolindentleft=0pt,
    breaksymbolindentright=0pt,
    breaksymbolsepleft=0pt,
    breaksymbolsepright=0pt,
    fontsize=\footnotesize,
    bgcolor=LightGray,
    numbersep=10pt
}


\usepackage{listings} % Para incluir código desde un archivo

\renewcommand\lstlistingname{Código Fuente}
\renewcommand\lstlistlistingname{Índice de Códigos Fuente}

% Definir colores
\definecolor{vscodepurple}{rgb}{0.5,0,0.5}
\definecolor{vscodeblue}{rgb}{0,0,0.8}
\definecolor{vscodegreen}{rgb}{0,0.5,0}
\definecolor{vscodegray}{rgb}{0.5,0.5,0.5}
\definecolor{vscodebackground}{rgb}{0.97,0.97,0.97}
\definecolor{vscodelightgray}{rgb}{0.9,0.9,0.9}

% Configuración para el estilo de C similar a VSCode
\lstdefinestyle{vscode_C}{
  backgroundcolor=\color{vscodebackground},
  commentstyle=\color{vscodegreen},
  keywordstyle=\color{vscodeblue},
  numberstyle=\tiny\color{vscodegray},
  stringstyle=\color{vscodepurple},
  basicstyle=\scriptsize\ttfamily,
  breakatwhitespace=false,
  breaklines=true,
  captionpos=b,
  keepspaces=true,
  numbers=left,
  numbersep=5pt,
  showspaces=false,
  showstringspaces=false,
  showtabs=false,
  tabsize=2,
  frame=tb,
  framerule=0pt,
  aboveskip=10pt,
  belowskip=10pt,
  xleftmargin=10pt,
  xrightmargin=10pt,
  framexleftmargin=10pt,
  framexrightmargin=10pt,
  framesep=0pt,
  rulecolor=\color{vscodelightgray},
  backgroundcolor=\color{vscodebackground},
}

%------------------------------------------------------------------------

% Comandos definidos
\newcommand{\bb}[1]{\mathbb{#1}}
\newcommand{\cc}[1]{\mathcal{#1}}

% I prefer the slanted \leq
\let\oldleq\leq % save them in case they're every wanted
\let\oldgeq\geq
\renewcommand{\leq}{\leqslant}
\renewcommand{\geq}{\geqslant}

% Si y solo si
\newcommand{\sii}{\iff}

% Letras griegas
\newcommand{\eps}{\epsilon}
\newcommand{\veps}{\varepsilon}
\newcommand{\lm}{\lambda}

\newcommand{\ol}{\overline}
\newcommand{\ul}{\underline}
\newcommand{\wt}{\widetilde}
\newcommand{\wh}{\widehat}

\let\oldvec\vec
\renewcommand{\vec}{\overrightarrow}

% Derivadas parciales
\newcommand{\del}[2]{\frac{\partial #1}{\partial #2}}
\newcommand{\Del}[3]{\frac{\partial^{#1} #2}{\partial #3^{#1}}}
\newcommand{\deld}[2]{\dfrac{\partial #1}{\partial #2}}
\newcommand{\Deld}[3]{\dfrac{\partial^{#1} #2}{\partial #3^{#1}}}


\newcommand{\AstIg}{\stackrel{(\ast)}{=}}
\newcommand{\Hop}{\stackrel{L'H\hat{o}pital}{=}}

\newcommand{\red}[1]{{\color{red}#1}} % Para integrales, destacar los cambios.

% Método de integración
\newcommand{\MetInt}[2]{
    \left[\begin{array}{c}
        #1 \\ #2
    \end{array}\right]
}

% Declarar aplicaciones
% 1. Nombre aplicación
% 2. Dominio
% 3. Codominio
% 4. Variable
% 5. Imagen de la variable
\newcommand{\Func}[5]{
    \begin{equation*}
        \begin{array}{rrll}
            #1:& #2 & \longrightarrow & #3\\
               & #4 & \longmapsto & #5
        \end{array}
    \end{equation*}
}

%------------------------------------------------------------------------



\begin{document}

    % 1. Foto de fondo
    % 2. Título
    % 3. Encabezado Izquierdo
    % 4. Color de fondo
    % 5. Coord x del titulo
    % 6. Coord y del titulo
    % 7. Fecha

    
    % 1. Foto de fondo
% 2. Título
% 3. Encabezado Izquierdo
% 4. Color de fondo
% 5. Coord x del titulo
% 6. Coord y del titulo
% 7. Fecha

\newcommand{\portada}[7]{

    \portadaBase{#1}{#2}{#3}{#4}{#5}{#6}{#7}
    \portadaBook{#1}{#2}{#3}{#4}{#5}{#6}{#7}
}

\newcommand{\portadaExamen}[7]{

    \portadaBase{#1}{#2}{#3}{#4}{#5}{#6}{#7}
    \portadaArticle{#1}{#2}{#3}{#4}{#5}{#6}{#7}
}




\newcommand{\portadaBase}[7]{

    % Tiene la portada principal y la licencia Creative Commons
    
    % 1. Foto de fondo
    % 2. Título
    % 3. Encabezado Izquierdo
    % 4. Color de fondo
    % 5. Coord x del titulo
    % 6. Coord y del titulo
    % 7. Fecha
    
    
    \thispagestyle{empty}               % Sin encabezado ni pie de página
    \newgeometry{margin=0cm}        % Márgenes nulos para la primera página
    
    
    % Encabezado
    \fancyhead[L]{\helv #3}
    \fancyhead[R]{\helv \nouppercase{\leftmark}}
    
    
    \pagecolor{#4}        % Color de fondo para la portada
    
    \begin{figure}[p]
        \centering
        \transparent{0.3}           % Opacidad del 30% para la imagen
        
        \includegraphics[width=\paperwidth, keepaspectratio]{assets/#1}
    
        \begin{tikzpicture}[remember picture, overlay]
            \node[anchor=north west, text=white, opacity=1, font=\fontsize{60}{90}\selectfont\bfseries\sffamily, align=left] at (#5, #6) {#2};
            
            \node[anchor=south east, text=white, opacity=1, font=\fontsize{12}{18}\selectfont\sffamily, align=right] at (9.7, 3) {\textbf{\href{https://losdeldgiim.github.io/}{Los Del DGIIM}}};
            
            \node[anchor=south east, text=white, opacity=1, font=\fontsize{12}{15}\selectfont\sffamily, align=right] at (9.7, 1.8) {Doble Grado en Ingeniería Informática y Matemáticas\\Universidad de Granada};
        \end{tikzpicture}
    \end{figure}
    
    
    \restoregeometry        % Restaurar márgenes normales para las páginas subsiguientes
    \pagecolor{white}       % Restaurar el color de página
    
    
    \newpage
    \thispagestyle{empty}               % Sin encabezado ni pie de página
    \begin{tikzpicture}[remember picture, overlay]
        \node[anchor=south west, inner sep=3cm] at (current page.south west) {
            \begin{minipage}{0.5\paperwidth}
                \href{https://creativecommons.org/licenses/by-nc-nd/4.0/}{
                    \includegraphics[height=2cm]{assets/Licencia.png}
                }\vspace{1cm}\\
                Esta obra está bajo una
                \href{https://creativecommons.org/licenses/by-nc-nd/4.0/}{
                    Licencia Creative Commons Atribución-NoComercial-SinDerivadas 4.0 Internacional (CC BY-NC-ND 4.0).
                }\\
    
                Eres libre de compartir y redistribuir el contenido de esta obra en cualquier medio o formato, siempre y cuando des el crédito adecuado a los autores originales y no persigas fines comerciales. 
            \end{minipage}
        };
    \end{tikzpicture}
    
    
    
    % 1. Foto de fondo
    % 2. Título
    % 3. Encabezado Izquierdo
    % 4. Color de fondo
    % 5. Coord x del titulo
    % 6. Coord y del titulo
    % 7. Fecha


}


\newcommand{\portadaBook}[7]{

    % 1. Foto de fondo
    % 2. Título
    % 3. Encabezado Izquierdo
    % 4. Color de fondo
    % 5. Coord x del titulo
    % 6. Coord y del titulo
    % 7. Fecha

    % Personaliza el formato del título
    \pretitle{\begin{center}\bfseries\fontsize{42}{56}\selectfont}
    \posttitle{\par\end{center}\vspace{2em}}
    
    % Personaliza el formato del autor
    \preauthor{\begin{center}\Large}
    \postauthor{\par\end{center}\vfill}
    
    % Personaliza el formato de la fecha
    \predate{\begin{center}\huge}
    \postdate{\par\end{center}\vspace{2em}}
    
    \title{#2}
    \author{\href{https://losdeldgiim.github.io/}{Los Del DGIIM}}
    \date{Granada, #7}
    \maketitle
    
    \tableofcontents
}




\newcommand{\portadaArticle}[7]{

    % 1. Foto de fondo
    % 2. Título
    % 3. Encabezado Izquierdo
    % 4. Color de fondo
    % 5. Coord x del titulo
    % 6. Coord y del titulo
    % 7. Fecha

    % Personaliza el formato del título
    \pretitle{\begin{center}\bfseries\fontsize{42}{56}\selectfont}
    \posttitle{\par\end{center}\vspace{2em}}
    
    % Personaliza el formato del autor
    \preauthor{\begin{center}\Large}
    \postauthor{\par\end{center}\vspace{3em}}
    
    % Personaliza el formato de la fecha
    \predate{\begin{center}\huge}
    \postdate{\par\end{center}\vspace{5em}}
    
    \title{#2}
    \author{\href{https://losdeldgiim.github.io/}{Los Del DGIIM}}
    \date{Granada, #7}
    \thispagestyle{empty}               % Sin encabezado ni pie de página
    \maketitle
    \vfill
}
    \portadaExamen{ffccA4.jpg}{Cálculo II\\Examen VI}{Cálculo II. Examen VI}{MidnightBlue}{-8}{28}{2023}{Arturo Olivares Martos}

    \begin{description}
        \item[Asignatura] Cálculo II.
        \item[Curso Académico] 2020-21.
        \item[Grado] Doble Grado en Ingeniería Informática y Matemáticas.
        \item[Grupo] Único.
        \item[Profesor] María Victoria Velasco Collado.
        \item[Descripción] Segundo Parcial. Integración. Temas 5-7.
        %\item[Fecha] 3 de mayo de 2021.
        %\item[Duración] 60 minutos.
    
    \end{description}
    \newpage
    
    \begin{ejercicio}\textbf{[2 puntos]}
Justificar de forma razonada si las siguientes funciones son o no uniformemente continuas y/o lipschitzianas.

\begin{enumerate}
    \item $f:]0,1[\to \bb{R}$ tal que $f(x)=x\ln x$, para cada $x\in ]0,1[$.

    Tenemos que es derivable en $]0,1[$, con:
    \begin{equation*}
        f'(x)=\ln x + 1 \qquad Im(f')=]-\infty, 1[
    \end{equation*}

    Por tanto, como la derivada de $f$ no está acotada en $]0,1[$, tenemos que no es lipschitziana. Veamos si es uniformemente continua.
    \begin{equation*}
        \lim_{x\to 0} f(x)
        = \lim_{x\to 0} \frac{\ln x}{\frac{1}{x}}
        \Hop
        = \lim_{x\to 0} \frac{\frac{1}{x}}{-\frac{1}{x^2}}
        = \lim_{x\to 0} \frac{-x^2}{x}
        = \lim_{x\to 0} -x=0
    \end{equation*}
    \begin{equation*}
        \lim_{x\to 1} f(x)=0
    \end{equation*}

    Por tanto, definiendo $f(0)=f(1)=0$, tenemos una ampliación continua del dominio de $f$, por lo que $f$ sí es uniformemente continua.

    \item $F:[1,3]\to \bb{R}$ tal que $F(x)=\int_1^x g(t)\;dt$, donde $g:[1,3]\to \bb{R}$ es una función monótona.

    Demostramos en primer lugar que $g$ está acotada. Como $g$ es monótona, supongamos $g$ creciente (no es restrictivo). Entonces:
    \begin{equation*}
        g(1)\leq g(x)\leq g(3)\qquad \forall x\in [1,3]
    \end{equation*}
    Por tanto, tenemos que está acotada.

    Por ser $g$ monótona y acotada, tenemos que es Riemman Integrable. Además, por ser acotada, tenemos que $|g(x)|\leq M\in \bb{R}$. Entonces:
    \begin{equation*}
        |F(y)-F(x)|=\left|\int_1^y g(t)\;dt -\int_1^x g(t)\;dt\right|
        = \left|\int_x^y g(t)\;dt\right|
        \leq M|y-x|
    \end{equation*}

    Por tanto, tenemos que $F(x)$ es uniformemente continua y, por tanto, lipschitziana y continua.
\end{enumerate}
\end{ejercicio}


\begin{ejercicio}\textbf{[1 punto]}
    Calcular el siguiente límite:
    \begin{equation*}
        L=\lim_{n\to \infty} \frac{1}{n} \left(
        \left(\frac{n+1}{n}\right)^2+
        \left(\frac{n+2}{n}\right)^2
        +\dots +
        \left(\frac{n+n}{n}\right)^2
        \right)
    \end{equation*}

    Definimos $f(x)=(1+x)^2$. Entonces:
    \begin{equation*}
        L=\lim_{n\to \infty} \frac{1}{n}\sum_{k=1}^n f\left(\frac{k}{n}\right) = \int_0^1 f(x)\;dx
        = \int_0^1 x^2 +2x +1 \;dx
        = \left[\frac{x^3}{3}+x^2 +x\right]_0^1
        = 2+\frac{1}{3} = \frac{7}{3}
    \end{equation*}
\end{ejercicio}

\begin{ejercicio}\textbf{[2 puntos]}
    Determina para qué valores de $a>0$, el área de la región acotada limitada por la gráfica de la función $f(x)=\ln x$, el eje de abcisas y las rectas $x=1, x=a$ es igual a 1.
    
    \begin{figure}[H]
        \centering
        
        \begin{tikzpicture}

            \def\a{3};
            
            % Eje de abscisas
            \draw[-stealth] (-1,0) -- (\a+1,0) node[right] {$x$};
            % Eje de ordenadas
            \draw[-stealth] (0,-1) -- (0,2) node[above] {$y$};
            
            % Gráfica de f(x) = ln(x)
            \draw[domain=0.1:\a+0.5,smooth,variable=\x,blue] plot ({\x},{ln(\x)});
            
            % Recta x = 1
            \draw[dashed] (1,-1) -- (1,2) node[above] {$x=1$};
            
            % Recta x = a
            \draw[dashed] (\a,-1) -- (\a,2) node[above] {$x=a$};
            
            
            % Región acotada
            \fill[gray!30,domain=1:\a,smooth,variable=\x] plot ({\x},{ln(\x)}) -- (\a,0) -- (1,0) -- cycle;
            
            % Puntos de intersección
            \filldraw[black] (1,0) circle (2pt) node[below] {$(1,0)$};
            \filldraw[black] (\a,{ln(\a)}) circle (2pt) node[right] {$(a,\ln a)$};
        \end{tikzpicture}
    \end{figure}
    
    \begin{itemize}
        \item \underline{Supongamos $a>1$}:
        
        Tenemos que el área delimitada es:
        \begin{multline*}
            A=\int_1^a \ln x\;dx = \MetInt{u(x)=\ln \quad u'(x)=\frac{1}{x}}{v'(x)=1 \quad v(x)=x} = \left[x\ln x\right]_1^a -\int_1^a x\cdot \frac{1}{x}\;dx = \left[x\ln x -x\right]_1^a
            =\\= a\ln a -a -\ln(1)+1 = 1\;u^2 \Longleftrightarrow a\ln a -a = 0 \Longleftrightarrow \ln a =1 \Longleftrightarrow a=e
        \end{multline*}
    
        Por tanto, tenemos que el valor buscado es $a=e$.

        \item \underline{Supongamos $0<a\leq 1$}:
        \begin{equation*}
            A= \int_a^1 -\ln x\;dx
            = -\int_a^1 \ln x\;dx
            = \int_1^a \ln x\;dx=1
            \Longleftrightarrow a=e
        \end{equation*}

        donde he usado el resultado del apartado anterior. Pero $a\leq 1$, por lo que no es posible.
        
        \begin{observacion}
        \begin{equation*}
            A=\int_0^1-\ln x\;dx =-\lim_{c\to 0}\int_c^1 \ln x 
            =-\lim_{c\to 0} \left[x\ln x -x\right]_c^1 = 1+\lim_{c\to 0} c\ln c -c = 1
        \end{equation*}

        donde he usado que $\displaystyle \lim_{x\to 0}x\ln x =0$.

        Por tanto, tenemos que el valor pedido sería $a=0$, pero tenemos que $a>0$.
        \end{observacion}
    \end{itemize}
\end{ejercicio}


\begin{ejercicio}\textbf{[2 puntos]}
    Calcular la longitud de la elipse $\displaystyle x^2+\frac{y^2}{2^2}=1$.\\

    Tenemos que se trata de la elipse centrada en el punto $P(0,0)$ con los ejes paralelos a los ejes cartesianos. El semieje horizontal mide $1$, y el semieje vertical mide $2$ unidades. Por tanto, la representación es esta:
    \begin{figure}[H]
        \centering
        
        \begin{tikzpicture}

            % Ejes
            \draw[-stealth] (-3,0) -- (3,0) node[right] {$x$};
            \draw[-stealth] (0,-2.5) -- (0,2.5) node[above] {$y$};
            
            % Puntos de corte con los ejes
            \filldraw (1,0) circle (2pt) node[below right] {$(1, 0)$};
            \filldraw (-1,0) circle (2pt) node[below left] {$(-1, 0)$};
            \filldraw (0,2) circle (2pt) node[above right] {$(0, 2)$};
            \filldraw (0,-2) circle (2pt) node[below right] {$(0, -2)$};
            
            % Elipse
            \draw[blue] (0,0) ellipse (1 and 2);
        
        \end{tikzpicture}
    \end{figure}

    Al ser simétrica la figura, calculo solo la longitud contenida en el primer cuadrante, que es $\frac{1}{4}$ de la longitud total.

    En el primer cuadrante, tengo que $x,y>0$. Por tanto, la función queda:
    \begin{equation*}
        y^2 = 4(1-x^2) \Longrightarrow y=2\sqrt{1-x^2}
    \end{equation*}

    Por tanto, la longitud de la elipse en el primer cuadrante es:
    \begin{equation*}
        l=\int_0^1 \sqrt{1+[y'(x)]^2}\;dx
    \end{equation*}

    Calculo la derivada:
    \begin{equation*}
        y'(x)=2\cdot \frac{-2x}{2\sqrt{1-x^2}}= -\frac{2x}{\sqrt{1-x^2}} \Longrightarrow [y'(x)]^2 = \frac{4x^2}{1-x^2}
    \end{equation*}

    Por tanto,
    \begin{equation*}
        l=\int_0^1 \sqrt{1+\frac{4x^2}{1-x^2}}\;dx
        = \int_0^1 \sqrt{\frac{1-x^2 + 4x^2}{1-x^2}}\;dx
        = \int_0^1 \sqrt{\frac{1+3x^2}{1-x^2}}\;dx
    \end{equation*}

    \begin{observacion}
        Este ejercicio se impugnó ya que con los conocimientos de la asignatura no es posible calcular dicha integral.
    \end{observacion}
\end{ejercicio}

\begin{ejercicio}\textbf{[2 puntos]} Sea $F:\bb{R}^+_0\to \bb{R}$ la función dada por $F(x):=\int_0^x e^{t^2}\;dt$. Estudiar su monotonía, y determinar su imagen. Calcular
\begin{equation*}
    \lim_{x\to \infty} \frac{\displaystyle \left(\int_0^x e^{t^2}\;dt\right)^2+\int_0^x e^{t^2}\;dt}{\displaystyle \int_0^{\sqrt{x}} te^{2t^4}\;dt}
\end{equation*}

Tenemos que el integrando es una función continua, por lo que es localmente integrable. Por tanto, por el TFC, tenemos que $F(x)$ es continua, con $F'(x)=e^{x^2}$. Como $F'(x)>0$, tenemos que $F(x)$ es estrictamente creciente en $\bb{R}^+_0$. Para calcular la imagen, antes veamos su si el límite en $+\infty$ converge. Tenemos que $e^x < e^{x^2} \quad \forall x\in \bb{R}^+$, por lo que:
\begin{equation*}
    \int_0^\infty e^{t^2}\;dt \text{ convergente} \Longrightarrow
    \int_0^\infty e^{t}\;dt \text{ convergente}
\end{equation*}

Como la integral de $e^x$ no converge, tenemos que la integral que buscamos tampoco. Por tanto, tenemos que diverge positivamente.
\begin{equation*}
    \lim_{x\to \infty} F(x)=\infty
\end{equation*}

Además, tenemos que $F(0)=0$. Por tanto, tenemos que:
\begin{equation*}
    Im(F)=\bb{R}^+_0
\end{equation*}

\hspace{1cm}

Para resolver el límite, tenemos que el numerador diverge positivamente. Veamos el comportamiento del denominador en $+\infty$. Como el integrando diverge positivamente, tenemos que la integral también. Además, como el integrando es positivo, puedo aplicar el TFC para calcular la derivada. Por tanto,
\begin{multline*}
    \lim_{x\to \infty} \frac{\displaystyle \left(\int_0^x e^{t^2}\;dt\right)^2+\int_0^x e^{t^2}\;dt}{\displaystyle \int_0^{\sqrt{x}} te^{2t^4}\;dt}
    = \left[\frac{\infty}{\infty}\right]
    \Hop
    \lim_{x\to \infty} \frac{2F(x)F'(x) + F'(x)}{\displaystyle \sqrt{x}e^{2x^2}\cdot \frac{1}{2\sqrt{x}}}
    =\\= \lim_{x\to \infty} \frac{2F(x)\cancel{e^{x^2}} + \cancel{e^{x^2}}}{\displaystyle \frac{e^{x^2}\cdot \cancel{e^{x^2}}}{2}}
    = \lim_{x\to \infty} \frac{4F(x)+2}{e^{x^2}}
    = \left[\frac{\infty}{\infty}\right]
    \Hop
    \lim_{x\to \infty} \frac{4e^{x^2}}{2xe^{x^2}}
    = \lim_{x\to \infty} \frac{2}{x} = 0
\end{multline*}
    
\end{ejercicio}


\begin{ejercicio}
    Demostrar que $0\leq \int_2^\infty \frac{1}{x^2(1+e^x)}\;dx \leq \frac{1}{2}$

    Definimos $f(x)=\frac{1}{x^2(1+e^x)}$, $g(x)=\frac{1}{x^2}$. Veamos en primer lugar el valor de la siguiente integral:
    \begin{equation*}
        \int_2^\infty g(x)\;dx
        = \lim_{b\to \infty} \int_2^b g(x)\;dx
        = \lim_{b\to \infty} \left[-\frac{1}{x}\right]_2^b = 
        = \frac{1}{2} - \lim_{b\to \infty} -\frac{1}{b} = \frac{1}{2}
    \end{equation*}

    Para razonar la desigualdad, vemos la siguiente desigualdad:
    \begin{equation*}
        f(x)\leq g(x) \Longleftrightarrow \frac{1}{x^2(1+e^x)}\leq \frac{1}{x^2} \Longleftrightarrow 1+e^x>1 \Longleftrightarrow e^x >0
    \end{equation*}
    
    Por tanto, tenemos que $0\leq f(x)\leq g(x)$. Como son Riemman Integrables y la integral conserva el orden, tenemos que:
    \begin{equation*}
        \int_2^\infty 0\;dx \leq \int_2^\infty f(x)\;dx \leq \int_2^\infty g(x)\;dx
    \end{equation*}

    Usando el valor calculado de la integral de $g(x)$, tenemos que:
    \begin{equation*}
        0\leq \int_2^\infty \frac{1}{x^2(1+e^x)}\;dx \leq \frac{1}{2}
    \end{equation*}
\end{ejercicio}



\end{document}