\documentclass[12pt]{article}

% Idioma y codificación
\usepackage[spanish, es-tabla]{babel}       %es-tabla para que se titule "Tabla"
\usepackage[utf8]{inputenc}

% Márgenes
\usepackage[a4paper,top=3cm,bottom=2.5cm,left=3cm,right=3cm]{geometry}

% Comentarios de bloque
\usepackage{verbatim}

% Paquetes de links
\usepackage[hidelinks]{hyperref}    % Permite enlaces
\usepackage{url}                    % redirecciona a la web

% Más opciones para enumeraciones
\usepackage{enumitem}

% Personalizar la portada
\usepackage{titling}

% Paquetes de tablas
\usepackage{multirow}


%------------------------------------------------------------------------

%Paquetes de figuras
\usepackage{caption}
\usepackage{subcaption} % Figuras al lado de otras
\usepackage{float}      % Poner figuras en el sitio indicado H.


% Paquetes de imágenes
\usepackage{graphicx}       % Paquete para añadir imágenes
\usepackage{transparent}    % Para manejar la opacidad de las figuras

% Paquete para usar colores
\usepackage[dvipsnames]{xcolor}
\usepackage{pagecolor}      % Para cambiar el color de la página

% Habilita tamaños de fuente mayores
\usepackage{fix-cm}

% Para los gráficos
\usepackage{tikz}

% Para poder situar los nodos en los grafos
\usetikzlibrary{positioning}


%------------------------------------------------------------------------

% Paquetes de matemáticas
\usepackage{mathtools, amsfonts, amssymb, mathrsfs}
\usepackage[makeroom]{cancel}     % Simplificar tachando
\usepackage{polynom}    % Divisiones y Ruffini
\usepackage{units} % Para poner fracciones diagonales con \nicefrac

\usepackage{pgfplots}   %Representar funciones
\pgfplotsset{compat=1.18}  % Versión 1.18

\usepackage{tikz-cd}    % Para usar diagramas de composiciones
\usetikzlibrary{calc}   % Para usar cálculo de coordenadas en tikz

%Definición de teoremas, etc.
\usepackage{amsthm}
%\swapnumbers   % Intercambia la posición del texto y de la numeración

\theoremstyle{plain}

\makeatletter
\@ifclassloaded{article}{
  \newtheorem{teo}{Teorema}[section]
}{
  \newtheorem{teo}{Teorema}[chapter]  % Se resetea en cada chapter
}
\makeatother

\newtheorem{coro}{Corolario}[teo]           % Se resetea en cada teorema
\newtheorem{prop}[teo]{Proposición}         % Usa el mismo contador que teorema
\newtheorem{lema}[teo]{Lema}                % Usa el mismo contador que teorema

\theoremstyle{remark}
\newtheorem*{observacion}{Observación}

\theoremstyle{definition}

\makeatletter
\@ifclassloaded{article}{
  \newtheorem{definicion}{Definición} [section]     % Se resetea en cada chapter
}{
  \newtheorem{definicion}{Definición} [chapter]     % Se resetea en cada chapter
}
\makeatother

\newtheorem*{notacion}{Notación}
\newtheorem*{ejemplo}{Ejemplo}
\newtheorem*{ejercicio*}{Ejercicio}             % No numerado
\newtheorem{ejercicio}{Ejercicio} [section]     % Se resetea en cada section


% Modificar el formato de la numeración del teorema "ejercicio"
\renewcommand{\theejercicio}{%
  \ifnum\value{section}=0 % Si no se ha iniciado ninguna sección
    \arabic{ejercicio}% Solo mostrar el número de ejercicio
  \else
    \thesection.\arabic{ejercicio}% Mostrar número de sección y número de ejercicio
  \fi
}


% \renewcommand\qedsymbol{$\blacksquare$}         % Cambiar símbolo QED
%------------------------------------------------------------------------

% Paquetes para encabezados
\usepackage{fancyhdr}
\pagestyle{fancy}
\fancyhf{}

\newcommand{\helv}{ % Modificación tamaño de letra
\fontfamily{}\fontsize{12}{12}\selectfont}
\setlength{\headheight}{15pt} % Amplía el tamaño del índice


%\usepackage{lastpage}   % Referenciar última pag   \pageref{LastPage}
\fancyfoot[C]{\thepage}

%------------------------------------------------------------------------

% Conseguir que no ponga "Capítulo 1". Sino solo "1."
\makeatletter
\@ifclassloaded{book}{
  \renewcommand{\chaptermark}[1]{\markboth{\thechapter.\ #1}{}} % En el encabezado
    
  \renewcommand{\@makechapterhead}[1]{%
  \vspace*{50\p@}%
  {\parindent \z@ \raggedright \normalfont
    \ifnum \c@secnumdepth >\m@ne
      \huge\bfseries \thechapter.\hspace{1em}\ignorespaces
    \fi
    \interlinepenalty\@M
    \Huge \bfseries #1\par\nobreak
    \vskip 40\p@
  }}
}
\makeatother

%------------------------------------------------------------------------
% Paquetes de cógido
\usepackage{minted}
\renewcommand\listingscaption{Código fuente}

\usepackage{fancyvrb}
% Personaliza el tamaño de los números de línea
\renewcommand{\theFancyVerbLine}{\small\arabic{FancyVerbLine}}

% Estilo para C++
\newminted{cpp}{
    frame=lines,
    framesep=2mm,
    baselinestretch=1.2,
    linenos,
    escapeinside=||
}

% para minted
\definecolor{LightGray}{rgb}{0.95,0.95,0.92}
\setminted{
    linenos=true,
    stepnumber=5,
    numberfirstline=true,
    autogobble,
    breaklines=true,
    breakautoindent=true,
    breaksymbolleft=,
    breaksymbolright=,
    breaksymbolindentleft=0pt,
    breaksymbolindentright=0pt,
    breaksymbolsepleft=0pt,
    breaksymbolsepright=0pt,
    fontsize=\footnotesize,
    bgcolor=LightGray,
    numbersep=10pt
}


\usepackage{listings} % Para incluir código desde un archivo

\renewcommand\lstlistingname{Código Fuente}
\renewcommand\lstlistlistingname{Índice de Códigos Fuente}

% Definir colores
\definecolor{vscodepurple}{rgb}{0.5,0,0.5}
\definecolor{vscodeblue}{rgb}{0,0,0.8}
\definecolor{vscodegreen}{rgb}{0,0.5,0}
\definecolor{vscodegray}{rgb}{0.5,0.5,0.5}
\definecolor{vscodebackground}{rgb}{0.97,0.97,0.97}
\definecolor{vscodelightgray}{rgb}{0.9,0.9,0.9}

% Configuración para el estilo de C similar a VSCode
\lstdefinestyle{vscode_C}{
  backgroundcolor=\color{vscodebackground},
  commentstyle=\color{vscodegreen},
  keywordstyle=\color{vscodeblue},
  numberstyle=\tiny\color{vscodegray},
  stringstyle=\color{vscodepurple},
  basicstyle=\scriptsize\ttfamily,
  breakatwhitespace=false,
  breaklines=true,
  captionpos=b,
  keepspaces=true,
  numbers=left,
  numbersep=5pt,
  showspaces=false,
  showstringspaces=false,
  showtabs=false,
  tabsize=2,
  frame=tb,
  framerule=0pt,
  aboveskip=10pt,
  belowskip=10pt,
  xleftmargin=10pt,
  xrightmargin=10pt,
  framexleftmargin=10pt,
  framexrightmargin=10pt,
  framesep=0pt,
  rulecolor=\color{vscodelightgray},
  backgroundcolor=\color{vscodebackground},
}

%------------------------------------------------------------------------

% Comandos definidos
\newcommand{\bb}[1]{\mathbb{#1}}
\newcommand{\cc}[1]{\mathcal{#1}}

% I prefer the slanted \leq
\let\oldleq\leq % save them in case they're every wanted
\let\oldgeq\geq
\renewcommand{\leq}{\leqslant}
\renewcommand{\geq}{\geqslant}

% Si y solo si
\newcommand{\sii}{\iff}

% Letras griegas
\newcommand{\eps}{\epsilon}
\newcommand{\veps}{\varepsilon}
\newcommand{\lm}{\lambda}

\newcommand{\ol}{\overline}
\newcommand{\ul}{\underline}
\newcommand{\wt}{\widetilde}
\newcommand{\wh}{\widehat}

\let\oldvec\vec
\renewcommand{\vec}{\overrightarrow}

% Derivadas parciales
\newcommand{\del}[2]{\frac{\partial #1}{\partial #2}}
\newcommand{\Del}[3]{\frac{\partial^{#1} #2}{\partial #3^{#1}}}
\newcommand{\deld}[2]{\dfrac{\partial #1}{\partial #2}}
\newcommand{\Deld}[3]{\dfrac{\partial^{#1} #2}{\partial #3^{#1}}}


\newcommand{\AstIg}{\stackrel{(\ast)}{=}}
\newcommand{\Hop}{\stackrel{L'H\hat{o}pital}{=}}

\newcommand{\red}[1]{{\color{red}#1}} % Para integrales, destacar los cambios.

% Método de integración
\newcommand{\MetInt}[2]{
    \left[\begin{array}{c}
        #1 \\ #2
    \end{array}\right]
}

% Declarar aplicaciones
% 1. Nombre aplicación
% 2. Dominio
% 3. Codominio
% 4. Variable
% 5. Imagen de la variable
\newcommand{\Func}[5]{
    \begin{equation*}
        \begin{array}{rrll}
            #1:& #2 & \longrightarrow & #3\\
               & #4 & \longmapsto & #5
        \end{array}
    \end{equation*}
}

%------------------------------------------------------------------------



\begin{document}

    % 1. Foto de fondo
    % 2. Título
    % 3. Encabezado Izquierdo
    % 4. Color de fondo
    % 5. Coord x del titulo
    % 6. Coord y del titulo
    % 7. Fecha

    
    % 1. Foto de fondo
% 2. Título
% 3. Encabezado Izquierdo
% 4. Color de fondo
% 5. Coord x del titulo
% 6. Coord y del titulo
% 7. Fecha

\newcommand{\portada}[7]{

    \portadaBase{#1}{#2}{#3}{#4}{#5}{#6}{#7}
    \portadaBook{#1}{#2}{#3}{#4}{#5}{#6}{#7}
}

\newcommand{\portadaExamen}[7]{

    \portadaBase{#1}{#2}{#3}{#4}{#5}{#6}{#7}
    \portadaArticle{#1}{#2}{#3}{#4}{#5}{#6}{#7}
}




\newcommand{\portadaBase}[7]{

    % Tiene la portada principal y la licencia Creative Commons
    
    % 1. Foto de fondo
    % 2. Título
    % 3. Encabezado Izquierdo
    % 4. Color de fondo
    % 5. Coord x del titulo
    % 6. Coord y del titulo
    % 7. Fecha
    
    
    \thispagestyle{empty}               % Sin encabezado ni pie de página
    \newgeometry{margin=0cm}        % Márgenes nulos para la primera página
    
    
    % Encabezado
    \fancyhead[L]{\helv #3}
    \fancyhead[R]{\helv \nouppercase{\leftmark}}
    
    
    \pagecolor{#4}        % Color de fondo para la portada
    
    \begin{figure}[p]
        \centering
        \transparent{0.3}           % Opacidad del 30% para la imagen
        
        \includegraphics[width=\paperwidth, keepaspectratio]{assets/#1}
    
        \begin{tikzpicture}[remember picture, overlay]
            \node[anchor=north west, text=white, opacity=1, font=\fontsize{60}{90}\selectfont\bfseries\sffamily, align=left] at (#5, #6) {#2};
            
            \node[anchor=south east, text=white, opacity=1, font=\fontsize{12}{18}\selectfont\sffamily, align=right] at (9.7, 3) {\textbf{\href{https://losdeldgiim.github.io/}{Los Del DGIIM}}};
            
            \node[anchor=south east, text=white, opacity=1, font=\fontsize{12}{15}\selectfont\sffamily, align=right] at (9.7, 1.8) {Doble Grado en Ingeniería Informática y Matemáticas\\Universidad de Granada};
        \end{tikzpicture}
    \end{figure}
    
    
    \restoregeometry        % Restaurar márgenes normales para las páginas subsiguientes
    \pagecolor{white}       % Restaurar el color de página
    
    
    \newpage
    \thispagestyle{empty}               % Sin encabezado ni pie de página
    \begin{tikzpicture}[remember picture, overlay]
        \node[anchor=south west, inner sep=3cm] at (current page.south west) {
            \begin{minipage}{0.5\paperwidth}
                \href{https://creativecommons.org/licenses/by-nc-nd/4.0/}{
                    \includegraphics[height=2cm]{assets/Licencia.png}
                }\vspace{1cm}\\
                Esta obra está bajo una
                \href{https://creativecommons.org/licenses/by-nc-nd/4.0/}{
                    Licencia Creative Commons Atribución-NoComercial-SinDerivadas 4.0 Internacional (CC BY-NC-ND 4.0).
                }\\
    
                Eres libre de compartir y redistribuir el contenido de esta obra en cualquier medio o formato, siempre y cuando des el crédito adecuado a los autores originales y no persigas fines comerciales. 
            \end{minipage}
        };
    \end{tikzpicture}
    
    
    
    % 1. Foto de fondo
    % 2. Título
    % 3. Encabezado Izquierdo
    % 4. Color de fondo
    % 5. Coord x del titulo
    % 6. Coord y del titulo
    % 7. Fecha


}


\newcommand{\portadaBook}[7]{

    % 1. Foto de fondo
    % 2. Título
    % 3. Encabezado Izquierdo
    % 4. Color de fondo
    % 5. Coord x del titulo
    % 6. Coord y del titulo
    % 7. Fecha

    % Personaliza el formato del título
    \pretitle{\begin{center}\bfseries\fontsize{42}{56}\selectfont}
    \posttitle{\par\end{center}\vspace{2em}}
    
    % Personaliza el formato del autor
    \preauthor{\begin{center}\Large}
    \postauthor{\par\end{center}\vfill}
    
    % Personaliza el formato de la fecha
    \predate{\begin{center}\huge}
    \postdate{\par\end{center}\vspace{2em}}
    
    \title{#2}
    \author{\href{https://losdeldgiim.github.io/}{Los Del DGIIM}}
    \date{Granada, #7}
    \maketitle
    
    \tableofcontents
}




\newcommand{\portadaArticle}[7]{

    % 1. Foto de fondo
    % 2. Título
    % 3. Encabezado Izquierdo
    % 4. Color de fondo
    % 5. Coord x del titulo
    % 6. Coord y del titulo
    % 7. Fecha

    % Personaliza el formato del título
    \pretitle{\begin{center}\bfseries\fontsize{42}{56}\selectfont}
    \posttitle{\par\end{center}\vspace{2em}}
    
    % Personaliza el formato del autor
    \preauthor{\begin{center}\Large}
    \postauthor{\par\end{center}\vspace{3em}}
    
    % Personaliza el formato de la fecha
    \predate{\begin{center}\huge}
    \postdate{\par\end{center}\vspace{5em}}
    
    \title{#2}
    \author{\href{https://losdeldgiim.github.io/}{Los Del DGIIM}}
    \date{Granada, #7}
    \thispagestyle{empty}               % Sin encabezado ni pie de página
    \maketitle
    \vfill
}
    \portadaExamen{ffccA4.jpg}{Cálculo II\\Examen VII}{Cálculo II. Examen VII}{MidnightBlue}{-8}{28}{2023}{Arturo Olivares Martos}

    \begin{description}
        \item[Asignatura] Cálculo II.
        \item[Curso Académico] 2018-19.
        \item[Grado] Doble Grado en Ingeniería Informática y Matemáticas.
        \item[Grupo] Único.
        %\item[Profesor] María Victoria Velasco Collado.
        \item[Descripción] Primera Parte. Derivación.
        \item[Fecha] 24 de abril de 2019.
        %\item[Duración] 60 minutos.
    
    \end{description}
    \newpage
    
    \begin{ejercicio}
    Calcula el siguiente límite:
    \begin{equation*}
        \lim_{x\to 0}\left(\frac{\sen x +\tan x}{2x}\right) ^\frac{1}{x}
    \end{equation*}
    
    \begin{itemize}
        \item \underline{\textbf{Opción 1}: Sin usar la fórmula de Euler/Zapato}:
        \begin{equation*}
            \lim_{x\to 0}\left(\frac{\sen x +\tan x}{2x}\right) ^\frac{1}{x} \stackrel{Ec.\;\ref{Ej1.Ind1}}{=} [1^\infty] = \lim_{x\to 0}e^{\frac{\ln\left(\frac{\sen x +\tan x}{2x}\right)}{x}}
            \stackrel{Ec.\;\ref{Ej1.Ind2}}{=} e^0 = 1
        \end{equation*}
    
        donde he tenido que hacer uso de los siguientes límites:
        \begin{equation} \label{Ej1.Ind1}
            \lim_{x\to 0} \frac{\sen x +\tan x}{2x} = \left[\frac{0}{0}\right] \Hop \lim_{x\to 0} \frac{\cos x + 1+\tan^2 x}{2} = \frac{2}{2} = 1
        \end{equation}
        \begin{equation}\begin{split} \label{Ej1.Ind2}
            &\lim_{x\to 0}\frac{\ln\left(\frac{\sen x +\tan x}{2x}\right)}{x} \stackrel{Ec.\;\ref{Ej1.Ind1}}{=} \left[\frac{0}{0}\right] \Hop
            \lim_{x\to 0} \frac{\frac{(\cos x +1+\tan^2 x)2x -2(\sen x + \tan x)}{4x^2}}{\frac{\sen x +\tan x}{2x}} 
            =\\&=
            \lim_{x\to 0} \frac{(\cos x +1+\tan^2 x)x -(\sen x + \tan x)}{x(\sen x +\tan x)}
            = \left[\frac{0}{0}\right]
            \Hop\\&=
            \lim_{x\to 0} \frac{[-\sen x +2\tan x(1+\tan^2x)]x + \cancel{(\cos x + 1 + \tan^2 x)}\cancel{-\cos x -(1+\tan^2x)} }{(\sen x + \tan x) +x[\cos x +(1+\tan^2 x)]}
            =\\&=
            \lim_{x\to 0} \frac{[-\sen x +2\tan x(1+\tan^2x)]x}{(\sen x + \tan x) +x[\cos x +(1+\tan^2 x)]}
            =\\&=
            \lim_{x\to 0} \frac{(-\sen x +2\tan x +2\tan^3x)x}{(\sen x + \tan x) +x[\cos x +(1+\tan^2 x)]}
            = \left[\frac{0}{0}\right]
            \Hop\\&=
            \lim_{x\to 0} \frac{[-\cos x +2(1+\tan^2 x) +6\tan^2x (1+\tan^2 x)]x +(-\sen x +2\tan x +2\tan^3x)}{\cos x+(1+\tan^2 x)+[\cos x +(1+\tan^2 x)] +x[-\sen x +2\tan x(1+\tan^2 x)]}
            =\\&=
            \frac{0+0}{1+1+1+1+0} = \frac{0}{4} = 0
        \end{split}\end{equation}

        \item \underline{\textbf{Opción 2}: Usando la fórmula de Euler/Zapato}:
        \begin{equation*}
            \lim_{x\to 0}\left(\frac{\sen x +\tan x}{2x}\right) ^\frac{1}{x} \stackrel{Ec.\;\ref{Ej1.Ind1}}{=} [1^\infty] \stackrel{Euler}{=} e^{\lim_{x\to 0} \left(\frac{\sen x +\tan x}{2x} -1\right)\cdot \frac{1}{x}} \stackrel{Ec.\;\ref{Ej1.Ind3}}{=} e^0 = 1
        \end{equation*}
    
        donde he tenido que aplicar el criterio de Euler o del Zapato:
        \begin{multline}\label{Ej1.Ind3}
            \lim_{x\to 0} \left(\frac{\sen x +\tan x}{2x} -1\right)\cdot \frac{1}{x}
            = \lim_{x\to 0} \frac{\sen x +\tan x -2x}{2x^2}
            = \left[\frac{0}{0}\right]
            \Hop\\=
            \lim_{x\to 0} \frac{\cos x +1+\tan^2 x -2}{4x}
            = \left[\frac{0}{0}\right]
            \Hop\\=
            \lim_{x\to 0} \frac{-\sen x +2\tan x(1+\tan^2 x)}{4} =\frac{0}{4}= 0
        \end{multline}
    \end{itemize}

    
\end{ejercicio}

\begin{ejercicio}
    Sea $f:[0,1]\to [0,1]$ derivable y tal que $f'(x)\neq 1,\;\forall x\in [0,1]$. Prueba que $\exists!\; c\in [0,1] \mid f(c)=c$.\\

    Defino la función auxiliar $g:[0,1]\to[-1, 2] \quad g(x)=f(x)-x$.
    
    Demuestro, en primer lugar, que $\exists c\in [0,1] \mid g(c)=0$. 
    \begin{itemize}
        \item \underline{Supongo $f(0)=0$}: El valor buscado es $c=0$.
        \item \underline{Supongo $f(1)=1$}: El valor buscado es $c=1$.
        \item \underline{Supongo $f(0)\neq 0 \land f(1)\neq 1$}:
        \begin{equation*}
            g(0)=f(0)>0 \qquad g(1)=f(1)-1<0
        \end{equation*}
        Como $g$ es continua por ser $f$ derivable y continua, por el Teorema de Bolzano tengo que:
        \begin{equation*}
            \exists c\in ]0,1[\mid g(c)=0=f(c)-c \Longrightarrow f(c)=c
        \end{equation*}
    \end{itemize}

    Demuestro ahora la unicidad de $c$. Supongamos que $\exists c'\in [0,1]$ tal que $f(c')=c'$. Por tanto, $g'(c')=0$. Como $g(c)=g(c')$, por el Teorema de Rolle (sabiendo que $g$ es derivable), tenemos que:
    \begin{equation*}
        \exists d\in ]0,1[\mid g'(d)=f'(d)-1=0 \Longrightarrow f'(d)=1
    \end{equation*}

    Sin embargo, llegamos a una contradicción, ya que $f'(x)\neq 1 \;\forall x\in [0,1]$. Por tanto, la hipótesis es falsa y tenemos que $c$ es único.
\end{ejercicio}

\begin{ejercicio}
    Demuestra la siguiente desigualdad $\forall x\in \bb{R}^+$:
    \begin{equation*}
        \frac{\arctan x}{1+x} < \ln(1+x)
    \end{equation*}

    Definimos $g:\bb{R}^+ \to \bb{R}$ dada por:
    \begin{equation*}
        g(x)=\frac{\arctan x}{1+x} - \ln(1+x) \qquad \forall x \in \bb{R}^+
    \end{equation*}

    Necesitamos calcular la imagen de $g$.
    \begin{equation*}
        g'(x)=\frac{\frac{1}{1+x^2}(1+x)-\arctan x}{(1+x)^2} -\frac{1}{1+x}
    \end{equation*}

    Calculamos los puntos que anulan a la primera derivada:
    \begin{multline*}
        g'(x)=0 \Longleftrightarrow \frac{1+x}{1+x^2} = \arctan x +(1+x) \Longleftrightarrow 1+x = (1+x^2)[\arctan x +(1+x)]
        \Longleftrightarrow \\ \Longleftrightarrow
        \cancel{1+x} = \arctan x +\cancel{1+x} + x^2 \arctan x + x^2 + x^3 \Longleftrightarrow 0=(1+x^2)\arctan (x) +x^2 + x^3
    \end{multline*}

    Tenemos que $x=0$ es una solución. Además, para $x>0$, tenemos que todos los sumandos son positivos ($\arctan x>0 \;\forall x>0$). Por tanto, tenemos que el único punto crítico es $x=0$. No obstante, $x=0\notin \bb{R}^+$, por lo que no tiene puntos críticos.

    Como $g'(x)<0 \;\forall x\in \bb{R}^+$, tenemos que $g'$ es estrictamente decreciente.
    \begin{equation*}
        \lim_{x\to 0^+} g(x) = \lim_{x\to 0^+} \frac{\arctan x}{1+x} - \ln(1+x) = \frac{0}{1} -\ln 1 = 0
    \end{equation*}
    \begin{equation*}
        \lim_{x\to +\infty} g(x) = \lim_{x\to +\infty} \frac{\arctan x}{1+x} - \ln(1+x) = \frac{\frac{\pi}{2}}{\infty} - \infty = 0-\infty = -\infty
    \end{equation*}

    Como $g$ es estrictamente decreciente, y tenemos que $g\in C^1(\bb{R^+})$, sabiendo el valor de los límites en 0 y en $+\infty$ tenemos que $Im(g)=\bb{R}^-$. Por tanto,
    \begin{equation*}
        \frac{\arctan x}{1+x} - \ln(1+x)<0 \qquad \forall x \in \bb{R}^+
        \Longrightarrow 
        \frac{\arctan x}{1+x} < \ln(1+x) \qquad \forall x \in \bb{R}^+
    \end{equation*}
\end{ejercicio}

\begin{ejercicio}
    Halla el rectángulo de mayor área que puede inscribirse en un semicírculo de radio $R>0$, teniendo la base inferior sobre el diámetro.
    \begin{figure}[H]
        \centering
        \begin{tikzpicture}
            \def\r{3}
            \def\a{30}
            
            \pgfmathsetmacro{\b}{\r*cos(\a)};
            \pgfmathsetmacro{\h}{\r*sin(\a)};

            %\draw (0,0) circle (\r);
            \draw[dashed] (-\r,0) arc (180:360:\r);
            \draw[thick] (\r,0) arc (0:180:\r);

            \draw[ultra thick] (\b,0) -- (\b,\h) -- (-\b, \h) -- (-\b, 0) -- (\b, 0);

            \draw[dashed] (-\r-0.5, 0) -- (\r + 0.5, 0);
            \draw[dashed] (0,-\r-0.5) -- (0,\r + 0.5);

            \draw[dashed, -stealth] (0,0) -- node[above right]{$R$} (\b,-\h);

            \filldraw[fill=red, draw=red] (\b,\h) node[above right, red] {$P\left(\frac{b}{2}, h\right)$} circle (0.08);
        \end{tikzpicture}
    \end{figure}

    La circunferencia de radio $R$ tiene por ecuación $x^2+y^2 = R^2$. Por tanto, como el punto $P$ pertenece a la circunferencia, la ecuación de ligadura es:
    \begin{equation*}
        \frac{b^2}{4} + h^2 = R^2 \Longrightarrow b=2\sqrt{R^2 - h^2}
    \end{equation*}

    Por tanto, la ecuación que indica el área del rectángulo es:
    \begin{equation*}
        \begin{array}{rl}
            A:]0,R[ & \longrightarrow \bb{R}^+\\
                    h & \longmapsto A(h) = bh = 2h\sqrt{R^2 - h^2}
        \end{array}
    \end{equation*}

    Como $A(h)\geq 0\forall h\in ]0,R[$, maximizar $A(h)$ equivale a maximizar $A^2(h)$:
    \begin{equation*}
        A^2(h) = 4h^2(R^2-h^2) = 4h^2R^2 - 4h^4
    \end{equation*}
    \begin{equation*}
        (A^2)'(h) = 8R^2h - 16h^3 = 0 \Longleftrightarrow h(8R^2-16h^2) = 0 \Longleftrightarrow 8R^2 = 16h^2 \Longleftrightarrow h=\sqrt{\frac{R^2}{2}} = \frac{R}{\sqrt{2}}
    \end{equation*}

    Como $(A^2)''(h)=8R^2 -48h^2$, tenemos que:
    \begin{equation*}
        (A^2)''\left(\frac{R}{\sqrt{2}}\right) = 8R^2 -48\cdot \frac{R^2}{2} = 8R^2 - 24R^2 = -16R^2 <0
    \end{equation*}
    Por tanto, $A^2(h)$ tiene un máximo relativo en $h=\frac{R}{\sqrt{2}}$. Como $A(h)\geq 0\;\forall h\in ]0,R[$, tenemos que $A(x)$ también tiene un máximo relativo en el mismo punto. Además, como es el único extremo relativo de una función de clase $1$, tenemos que es máximo absoluto.

    Por tanto, el rectángulo de mayor área tiene las siguientes dimensiones:
    \begin{equation*}
        h=\frac{R}{\sqrt{2}}=\frac{\sqrt{2}}{2}R \qquad \qquad b=2\sqrt{\frac{R^2}{2}} = \sqrt{2}R
    \end{equation*}
\end{ejercicio}



\end{document}