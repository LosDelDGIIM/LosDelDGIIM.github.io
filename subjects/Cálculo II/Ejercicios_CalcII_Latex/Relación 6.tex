\section{Cálculo Integral Práctico}

\renewcommand{\labelenumi}{\alph{enumi})}
\begin{ejercicio}\label{Ejercicio.Ej1}
Calcular las siguientes integrales:
\begin{enumerate}
    \item $\displaystyle \int \frac{1}{\sqrt{x}}e^{\sqrt{x}}dx = \red{2}\int \frac{1}{\red{2}\sqrt{x}}e^{\sqrt{x}}dx = 2e^{\sqrt{x}} +C$

    \item $\displaystyle \int \frac{dx}{(1-x)^2} = \frac{1}{1-x} +C$

    \item $\displaystyle \int (\tg^3x + \tg^5x) dx
    = \int [\tg^3x(1+\tg^2x)]dx
    = \red{\frac{1}{4}}\int [4\tg^3x(1+\tg^2x)]dx = \frac{\tg^4x}{4} +C$

    \item $\displaystyle \int \sen^3 x\; dx$
    \begin{multline*}
        \int \sen^3 x\; dx
        = \int \sen x (1-\cos^2 x)dx
        = \int \sen x\;dx -\int \cos^2x\sen x\;dx =\\= -\cos x +\frac{\cos^3x}{3}+C
    \end{multline*}

    \item $\displaystyle \int \tg^2x\;dx = \int \tg^2x \red{+1-1}\;dx = \tg x -x + C$

    \item $\displaystyle \int \frac{2^x}{1+4^x}dx$

    Tenemos que $4^x = (2^2)^x = 2^{2x} = (2^x)^2$. Por tanto,
    \begin{multline*}
        \int \frac{2^x}{1+4^x}dx
        = \int \frac{2^x}{1+(2^x)^2}dx
        = \MetInt{2^x=t}{2^x\ln 2\;dx = dt}
        =\int \frac{\cancel{2^x}}{1+t^2}\cdot \frac{dt}{\cancel{2^x}\ln 2}
        =\\= \frac{1}{\ln 2}\arctan t + C
        = \frac{1}{\ln 2}\arctan 2^x + C
    \end{multline*}

    \item $\displaystyle \int \frac{e^x}{\sqrt{1-e^x}}dx = \MetInt{1-e^x = t}{dx = -\frac{dt}{e^x}}
    = \int \frac{-dt}{\sqrt{t}}dx = -2\sqrt{t} +C = -2\sqrt{1-e^x}+C$

    \item $\displaystyle \int \frac{x^2}{9+x^6}dx = \frac{1}{9}\int \frac{x^2}{1+\frac{x^6}{9}}dx = \frac{1}{9} \int \frac{x^2}{1+\left(\frac{x^3}{3}\right)^2} dx
    = \frac{1}{9} \arctan \left(\frac{x^3}{3}\right) +C$

    \item $\displaystyle \int \ln x\; dx = \MetInt{u(x)=\ln x \quad v'(x)=1}{u'(x)=\frac{1}{x} \quad v(x)=x} = x\ln x -\int x\cdot \frac{1}{x}\;dx = x\ln x -x + C$ 

    \item $\displaystyle \int \frac{1}{1+\sqrt{x}}\;dx$

    Aplico el siguiente cambio de variable:
    \begin{equation*}
        \MetInt{\sqrt{x}=t}{\frac{1}{2\sqrt{x}}dx = dt \Longrightarrow dx=2t\;dt}
    \end{equation*}

    Por tanto, la integral queda:
    \begin{equation*}
        \int\frac{2t}{1+t}dt = 2\int \frac{\red{1}+t\red{-1}}{1+t}dt = 2t-2\ln |1+t|+C
    \end{equation*}

    Deshaciendo el cambio de variable,
    \begin{equation*}
        \int \frac{1}{1+\sqrt{x}}\;dx = 2\sqrt{x}-2\ln (1+\sqrt{x})+C
    \end{equation*}


    \item $\displaystyle \int \tg(2x)\;dx = \int \frac{\sen 2x}{\cos 2x}\;dx = \red{-\frac{1}{2}} \int \frac{\red{-2}\sen 2x}{\cos 2x}\;dx = -\frac{1}{2}\ln |\cos 2x|+C$

    \item $\displaystyle \int (x^2+5)e^{-x}dx = -e^{-x}(x^2+5)-\int 2xe^{-x}$

    Aplico el método de integración por partes con:
    \begin{equation*}
         \MetInt{u(x)=x^2+5 \quad u'(x)=2x}{v'(x)=e^{-x}\quad v(x)=-e^{-x}}
    \end{equation*}
    \begin{equation*}
        \int (x^2+5)e^{-x}dx = -e^{-x}(x^2+5)+\int 2xe^{-x}\;dx
    \end{equation*}

    Vuelvo a aplicar el método de integración por partes con:
    \begin{equation*}
         \MetInt{u(x)=2x \quad u'(x)=2}{v'(x)=e^{-x}\quad v(x)=-e^{-x}}
    \end{equation*}
    \begin{multline*}
        \int (x^2+5)e^{-x}dx = -e^{-x}(x^2+5)+\int 2xe^{-x}\;dx
        = -e^{-x}(x^2+5) -e^{-x}(2x)+\int 2e^{-x}
        =\\=
        -e^{-x}(x^2+5) -e^{-x}(2x) -2e^{-x} +C
        = -e^{-x}(x^2+2x+7) +C
    \end{multline*}


    \item[ll)] $\displaystyle \int x^3\sen(3x)\;dx$

    Aplico el método de integración por partes con:
    \begin{equation*}
        \MetInt{u(x)=x^3 \quad u'(x)=3x^2}{v'(x)=\sen(3x) \quad v(x)=-\frac{1}{3}\cos(3x)}
    \end{equation*}
    \begin{equation*}
        \int x^3\sen(3x)\;dx = -\frac{x^3}{3}\cos(3x)+\int 3x^2\frac{1}{3}\cos(3x)\;dx
        = -\frac{x^3}{3}\cos(3x)+\int x^2\cos(3x)\;dx
    \end{equation*}

    Aplico de nuevo el método de integración por partes con:
    \begin{equation*}
        \MetInt{u(x)=x^2 \quad u'(x)=2x}{v'(x)=\cos(3x) \quad v(x)=\frac{1}{3}\sen(3x)}
    \end{equation*}
    \begin{multline*}
        \int x^3\sen(3x)\;dx
        = -\frac{x^3}{3}\cos(3x)+\int x^2\cos(3x)\;dx
        =\\=  -\frac{x^3}{3}\cos(3x) +\frac{x^2}{3}\sen(3x)-\int 2x\cdot \frac{1}{3}\sen(3x)\;dx
        =\\= -\frac{x^3}{3}\cos(3x) +\frac{x^2}{3}\sen(3x)-\frac{2}{3}\int x\sen(3x)\;dx
    \end{multline*}

    Aplico por tercera vez el método de integración por partes con:
    \begin{equation*}
        \MetInt{u(x)=x \quad u'(x)=1}{v'(x)=\sen(3x) \quad v(x)=-\frac{1}{3}\cos(3x)}
    \end{equation*}
    \begin{multline*}
        \int x^3\sen(3x)\;dx
        = -\frac{x^3}{3}\cos(3x) +\frac{x^2}{3}\sen(3x)-\frac{2}{3}\int x\sen(3x)\;dx
        =\\=
        -\frac{x^3}{3}\cos(3x) +\frac{x^2}{3}\sen(3x)-\frac{2}{3}\left[-\frac{x}{3}\cos(3x)+\int\frac{1}{3}\cos(3x) \right]
        =\\=
        -\frac{x^3}{3}\cos(3x) +\frac{x^2}{3}\sen(3x)-\frac{2}{3}\left[-\frac{x}{3}\cos(3x)+\frac{1}{3\cdot \red{3}}\int \red{3}\cos(3x) \right]
        =\\=
        -\frac{x^3}{3}\cos(3x) +\frac{x^2}{3}\sen(3x)-\frac{2}{3}\left[-\frac{x}{3}\cos(3x)+\frac{1}{9}\sen(3x) \right] +C
        =\\= \left(-\frac{x^3}{3}+\frac{2x}{9}\right)\cos(3x) +\left(\frac{x^2}{3}-\frac{2}{27}\right)\sen(3x)
    \end{multline*}

    \item $\displaystyle \int x\ln(1+x^2)\;dx$

    Aplico el método de integración por partes con:
    \begin{equation*}
        \MetInt{u(x)=\ln(1+x^2) \quad u'(x)=\frac{2x}{1+x^2}}{v'(x)=x \qquad v(x)=\frac{x^2}{2}}
    \end{equation*}
    \begin{equation*}
        \int x\ln(1+x^2)\;dx = \frac{x^2}{2}\ln(1+x^2)-\int\frac{x^2}{2}\cdot \frac{2x}{1+x^2}\;dx
        = \frac{x^2}{2}\ln(1+x^2)-\int \frac{x^3}{1+x^2}\;dx
    \end{equation*}

    Para calcular la integral, realizamos la división de polinomios:
    $$\polylongdiv[style=C]{x^3}{x^2+1}$$
    \begin{multline*}
        \int x\ln(1+x^2)\;dx
        = \frac{x^2}{2}\ln(1+x^2)-\int \frac{x^3}{1+x^2}\;dx
        = \frac{x^2}{2}\ln(1+x^2)-\int x-\frac{x}{1+x^2}\;dx
        =\\=
        \frac{x^2}{2}\ln(1+x^2)-\frac{x^2}{2} +\red{\frac{1}{2}}\int \frac{\red{2}x}{1+x^2}\;dx
        = \frac{x^2}{2}\ln(1+x^2)-\frac{x^2}{2} +\frac{1}{2}\ln(1+x^2) +C
        =\\= \ln(1+x^2) \left(\frac{x^2+1}{2}\right) - \frac{x^2}{2} + C
    \end{multline*}

    \item $\displaystyle \int \frac{x+1}{x^4 -1} dx$

    Factorizo en primer lugar el denominador.
    $$\polyhornerscheme[x=-1]{x^4-1} \Longrightarrow x^4-1=(x+1)(x^3-x^2+x-1)$$
    $$\polyhornerscheme[x=1]{x^3-x^2+x-1} \Longrightarrow x^4-1=(x+1)(x-1)(x^2+1)$$

    Por tanto,
    \begin{equation*}
        \int \frac{x+1}{x^4 -1} dx = \int \frac{\cancel{x+1}}{\cancel{(x+1)}(x-1)(x^2+1)}dx
        = \int \frac{dx}{(x-1)(x^2+1)}dx
    \end{equation*}

    Aplico el método de los coeficientes indeterminados:
    \begin{equation*}
        \frac{1}{(x-1)(x^2+1)} = \frac{A}{x-1}+\frac{Bx+C}{x^2+1} \Longrightarrow 1=A(x^2+1)+(Bx+C)(x-1)
    \end{equation*}
    \begin{itemize}
        \item \underline{$x=1$} $\Longrightarrow 1=2A \Longrightarrow A=\frac{1}{2}$
        \item \underline{$x=0$} $\Longrightarrow 1=A-C \Longrightarrow C=A-1 = -\frac{1}{2}$
        \item \underline{$x=-1$} $\Longrightarrow 1=2A+2B-2C \Longrightarrow B=\frac{1}{2}-A+C=\frac{1}{2}-1 = -\frac{1}{2}$
    \end{itemize}
    
    Por tanto,
    \begin{multline*}
        \int \frac{x+1}{x^4 -1} dx = \int \frac{\frac{1}{2}}{x-1}+\frac{-\frac{1}{2}x-\frac{1}{2}}{x^2+1}\;dx
        = \frac{1}{2}\ln|x-1| -\frac{1}{2}\int\frac{x}{x^2+1}\;dx -\frac{1}{2}\int \frac{1}{x^2+1}\;dx
        =\\=
        \frac{1}{2}\ln|x-1| -\frac{1}{4}\ln{(x^2+1)} -\frac{1}{2} \arctan x +C
    \end{multline*}

    \item $\displaystyle \int\frac{x-2}{x(x-1)(x+1)}\;dx$

    Aplico el método de los coeficientes indeterminados:
    \begin{equation*}
        \frac{x-2}{x(x-1)(x+1)} = \frac{A}{x}+\frac{B}{x-1} + \frac{C}{x+1} \Longrightarrow x-2=A(x-1)(x+1)+Bx(x+1)+Cx(x-1)
    \end{equation*}
    \begin{itemize}
        \item \underline{$x=1$} $\Longrightarrow -1=2B \Longrightarrow B=-\frac{1}{2}$
        \item \underline{$x=0$} $\Longrightarrow -2=-A \Longrightarrow A=2$
        \item \underline{$x=-1$} $\Longrightarrow -3=2C \Longrightarrow C=-\frac{3}{2}$
    \end{itemize}

    Por tanto,
    \begin{multline*}
        \int\frac{x-2}{x(x-1)(x+1)}\;dx
        = 2\ln |x|-\frac{1}{2}\ln |x-1|-\frac{3}{2}\ln|x+1|+C
        = \ln \frac{x^2}{\sqrt{|(x-1)(x+1)^3|}}+C
    \end{multline*}

    \item $\displaystyle \int \frac{x^4-3x^3-3x-2}{x^3-x^2-2x}dx$

    Realizo la división de polinomios:
    $$\polylongdiv[style=C]{x^4-3x^3-3x-2}{x^3-x^2-2x}$$

    \begin{equation*}
        \int \frac{x^4-3x^3-3x-2}{x^3-x^2-2x}dx = \int (x-2)dx -\int \frac{7x+2}{x^3-x^2-2x}dx 
    \end{equation*}

    Para calcular la integral restante, factorizo el denominador.
    $$x^2-x-2=0 \Longleftrightarrow x=\{2, -1\}$$

    \begin{equation*}
        \int \frac{x^4-3x^3-3x-2}{x^3-x^2-2x}dx = \frac{x^2}{2}-2x -\int \frac{7x+2}{x(x-2)(x+1)}dx 
    \end{equation*}

    Aplico el método de coeficientes indeterminados:
    \begin{equation*}
        \frac{7x+2}{x(x-2)(x+1)} = \frac{A}{x} + \frac{B}{x-2} + \frac{C}{x+1} \Longrightarrow 7x+2=A(x+1)(x-2)+Bx(x+1)+Cx(x-2)
    \end{equation*}
    \begin{itemize}
        \item \underline{Para $x=0$} $\Longrightarrow 2=-2A \Longrightarrow A=-1$

        \item \underline{Para $x=-1$} $\Longrightarrow -5=3C \Longrightarrow C=-\frac{5}{3}$

        \item \underline{Para $x=2$} $\Longrightarrow 16=6B \Longrightarrow B=\frac{8}{3}$
    \end{itemize}

    Por tanto,
    \begin{multline*}
        \int \frac{x^4-3x^3-3x-2}{x^3-x^2-2x}dx = \frac{x^2}{2}-2x -\int \frac{7x+2}{x(x-2)(x+1)}dx 
        =\\=
        \frac{x^2}{2}-2x +\ln|x|-\frac{8}{3}\ln |x-2|+\frac{5}{3}\ln|x+1|+C
    \end{multline*}

    \item $\displaystyle \int \frac{x+1}{x^2-3x+3}\;dx$

    Como $\Delta=3^2-4\cdot 3 <0$, no tiene raíces reales.
    \begin{multline*}
        \int \frac{x+1}{x^2-3x+3}\;dx =\red{\frac{1}{2}} \int \frac{\red{2}x+\red{2}\cdot 1}{x^2-3x+3}\;dx
        = \frac{1}{2} \int \frac{2x-3}{x^2-3x+3} +\frac{5}{x^2-3x+3}\;dx
        =\\=\frac{1}{2}\ln|x^2-3x+3|+\frac{5}{2}\int \frac{1}{x^2-3x+3}\;dx
    \end{multline*}

    Busco ahora un binomio al cuadrado en el denominador.
    \begin{multline*}
        x^2-3x+3 = \left(x-\frac{3}{2}\right)^2-\frac{9}{4}+3
        = \left(x-\frac{3}{2}\right)^2-\frac{9}{4}+3
        = \left(x-\frac{3}{2}\right)^2+\frac{3}{4}
        =\\=
        \frac{3}{4}\left[\frac{4}{3}\left(x-\frac{3}{2}\right)^2+1\right]
        =
        \frac{3}{4}\left[\left(\frac{2x}{\sqrt{3}}-\frac{3}{\sqrt{3}}\right)^2+1\right]
        =
        \frac{3}{4}\left[\left(\frac{2x}{\sqrt{3}}-\sqrt{3}\right)^2+1\right]
    \end{multline*}

    Por tanto, tenemos que:
    \begin{multline*}
        \int \frac{x+1}{x^2-3x+3}\;dx =\frac{1}{2}\ln|x^2-3x+3|+\frac{5}{2}\int \frac{1}{x^2-3x+3}\;dx
        =\\=
        \frac{1}{2}\ln|x^2-3x+3|+\frac{5}{2}\int \frac{1}{\frac{3}{4}\left[\left(\frac{2x}{\sqrt{3}}-\sqrt{3}\right)^2+1\right]}\;dx
        =\\=
        \frac{1}{2}\ln|x^2-3x+3|+\frac{5}{2}\cdot \frac{4}{3}\int \frac{1}{1+\left(\frac{2x}{\sqrt{3}}-\sqrt{3}\right)^2}\;dx
        =\\=
        \frac{1}{2}\ln|x^2-3x+3|+\frac{5}{2}\cdot \frac{4}{3}\cdot \red{\frac{\sqrt{3}}{2}}\int \frac{\red{\frac{2}{\sqrt{3}}}}{1+\left(\frac{2x}{\sqrt{3}}-\sqrt{3}\right)^2}\;dx
        =\\=
        \frac{1}{2}\ln|x^2-3x+3|+\frac{5\sqrt{3}}{3} \arctan \left(\frac{2x}{\sqrt{3}}-\sqrt{3}\right) + C
    \end{multline*}

    \item $\displaystyle \int \frac{\cos x}{\sen^3x +2\cos^2x\sen x}\;dx$

    Como la función es impar en el coseno, aplico el cambio de variable $\sen x = t$.
    \begin{multline*}
        \int \frac{\cos x}{\sen^3x +2\cos^2x\sen x}\;dx = \MetInt{\sen x= t}{\cos x \;dx = dt} = 
        \int \frac{dt}{t^3 +2(1-t^2)t}\;dx
        =\\= 
        \int \frac{dt}{t^3 +2t-2t^3}\;dx
        = 
        \int \frac{dt}{t(-t^2+2)}\;dx
        = 
        \int \frac{dt}{t(\sqrt{2}+t)(\sqrt{2}-t)}\;dx
    \end{multline*}

    Aplico ahora el método de los coeficientes indeterminados:
    \begin{equation*}
        \frac{1}{t(\sqrt{2}+t)(\sqrt{2}-t)} = \frac{A}{t} + \frac{B}{\sqrt{2}+t} + \frac{C}{\sqrt{2}-t} \Longrightarrow 1=A(\sqrt{2}+t)(\sqrt{2}-t) + Bt(\sqrt{2}-t) + Ct(\sqrt{2}+t)
    \end{equation*}
    \begin{itemize}
        \item \underline{Para $t=0$} $\Longrightarrow 1=2A\Longrightarrow A=\frac{1}{2}$
        \item \underline{Para $t=\sqrt{2}$} $\Longrightarrow 1=C\sqrt{2}\cdot 2\sqrt{2}\Longrightarrow 1=4C\Longrightarrow C=\frac{1}{4}$
        \item \underline{Para $t=-\sqrt{2}$} $\Longrightarrow 1=-B\sqrt{2}\cdot 2\sqrt{2}\Longrightarrow 1=-4b\Longrightarrow B=-\frac{1}{4}$
    \end{itemize}

    Por tanto, tenemos que:
    \begin{multline*}
        \int \frac{dt}{t^3 +2(1-t^2)t}\;dx
        = 
        \int \frac{dt}{t(\sqrt{2}+t)(\sqrt{2}-t)}\;dx
        = \frac{1}{2}\ln|t| -\frac{1}{4}\ln |\sqrt{2}+t|
        -\frac{1}{4}\ln |\sqrt{2}-t|+C
    \end{multline*}

    Deshaciendo el cambio de variable,
    \begin{multline*}
        \int \frac{\cos x}{\sen^3x +2\cos^2x\sen x}\;dx
        = \frac{1}{2}\ln|\sen x| -\frac{1}{4}\ln |\sqrt{2}+\sen x|
        -\frac{1}{4}\ln |\sqrt{2}-\sen x|+C
        =\\=
        \frac{1}{2}\ln|\sen x| -\frac{1}{4}\ln (2-\sen^2 x)+C
    \end{multline*}

    \item $\displaystyle \int \sen(3x) \cos (4x)\;dx$

    Tenemos que:
    \begin{equation*}
        \left.\begin{array}{c}
            \sen (a+b) = \sen (a) \cos (b) + \cos (a)\sen (b) \\
            \sen (a-b) = \sen (a) \cos (b) - \cos (a)\sen (b)
        \end{array}\right\}
        \Longrightarrow
        \frac{\sen(a+b)+\sen(a-b)}{2} = \sen (a) \cos (b)
    \end{equation*}

    Por tanto,
    \begin{multline*}
        \int \sen(3x) \cos (4x)\;dx = \int \frac{\sen (7x) +\sen (-x)}{2}
        = -\frac{1}{14}\cos(7x) + \frac{1}{2}\cos(-x)+C
        =\\= -\frac{1}{14}\cos(7x) + \frac{1}{2}\cos(x)+C
    \end{multline*}


    \item $\displaystyle \int \sec x \;dx = \int\frac{dx}{\cos x}$ \label{Ej1.Apartado_s}

    Como la función es impar en el coseno, aplico el cambio de variable $\sen x =t$:
    \begin{equation*}
        \MetInt{\sen x =t}{\cos x\;dx= dt}
    \end{equation*}

    Por tanto,
    \begin{equation*}
        \int \sec x \;dx = \int\frac{dx}{\cos x}
        =\int \frac{dt}{\cos^2 x}
        =\int \frac{dt}{1-\sen^2 x}
        =\int \frac{dt}{1-t^2}
    \end{equation*}

    Aplico el método de coeficientes indeterminados:
    \begin{equation*}
        \frac{1}{1-t^2} = \frac{1}{(1+t)(1-t)} = \frac{A}{1+t}+\frac{B}{1-t} \Longrightarrow 1=A(1-t)+B(1+t)
    \end{equation*}
    \begin{itemize}
        \item \underline{Para $t=1$} $\Longrightarrow 1=2B \Longrightarrow B=\frac{1}{2}$
        \item \underline{Para $t=-1$} $\Longrightarrow 1=2A \Longrightarrow A=\frac{1}{2}$
    \end{itemize}

    Por tanto,
    \begin{equation*}
        \int \frac{dt}{1-t^2} = \frac{1}{2}\ln |1+t|-\frac{1}{2}\ln |1-t| +C
        = \frac{1}{2}\ln \left(\frac{1+t}{1-t}\right)+C
    \end{equation*}

    Deshaciendo el cambio de variable,
    \begin{equation*}
        \int \sec x \;dx = \frac{1}{2}\ln \left(\frac{1+\sen x}{1-\sen x}\right)+C
    \end{equation*}

    \item $\displaystyle \int \frac{dx}{\sqrt{x^2+x+1}}$

    Busco un binomio en el denominador.
    \begin{multline*}
        x^2 + x +1 = \left(x+\frac{1}{2}\right)^2 -\frac{1}{4}+1 = \left(x+\frac{1}{2}\right)^2 +\frac{3}{4}
        = \frac{3}{4}\left[\frac{4}{3}\left(x+\frac{1}{2}\right)^2 +1\right]
        =\\= \frac{3}{4}\left[\left(\frac{2x+1}{\sqrt{3}}\right)^2 +1\right]
    \end{multline*}

    Por tanto,
    \begin{equation*}
        \int \frac{dx}{\sqrt{x^2+x+1}}
        = \int \frac{dx}{\sqrt{\frac{3}{4}\left[\left(\frac{2x+1}{\sqrt{3}}\right)^2 +1\right]}}
        = \int \frac{\frac{2}{\sqrt{3}}\;dx}{\sqrt{\left(\frac{2x+1}{\sqrt{3}}\right)^2 +1}}
    \end{equation*}

    Aplico el siguiente cambio de variable. Es válido ya que es $\tg:\left[-\frac{\pi}{2},\frac{\pi}{2}\right]\to \bb{R}$ es una biyección de clase 1.
   \begin{equation*}
        \MetInt{\frac{2x+1}{\sqrt{3}}=\tg t\qquad t\in\left[-\frac{\pi}{2},\frac{\pi}{2}\right]}{\frac{2}{\sqrt{3}}\;dx = (1+\tg^2 t)\;dt}
   \end{equation*} 

   Usando que $\tg^2t+1=\sec^2 t$, la integral queda:
   \begin{equation*}
       \int \frac{dx}{\sqrt{x^2+x+1}} 
       = \int \frac{1+\tg^2 t}{\sqrt{\tg^2 t +1}}\;dt
       = \int \frac{\sec^2 t}{\sqrt{\sec^2 t}}\;dt
       =\int |\sec t|\;dt =\int \sec t\;dt
   \end{equation*}
   donde he aplicado que $t\in \left[-\frac{\pi}{2},\frac{\pi}{2}\right]$, por lo que el coseno es positivo y por tanto la tangente también. Usando el ejercicio 1.s (apartado anteror), tenemos que:
   \begin{equation*}
       \int \frac{dx}{\sqrt{x^2+x+1}} 
       = \int \sec t\;dt = \frac{1}{2}\ln \left(\frac{1+\sen t}{1-\sen t}\right)+C
       = \frac{1}{2}\ln \left(\frac{1+\sen\left[ \arctan \left(\frac{2x+1}{\sqrt{3}}\right)\right]}{1-\sen \left[ \arctan \left(\frac{2x+1}{\sqrt{3}}\right)\right]}\right)+C
   \end{equation*}


   \item $\displaystyle \int \frac{dx}{x[\ln^3 x-2\ln^2x-\ln x+2]} = \MetInt{\ln x =t}{\frac{dx}{x}=dt}
   = \int \frac{dt}{t^3-2t^2- t+2}$

   Factorizamos en primer lugar el denominador:
   $$\polyhornerscheme[x=1]{x^3-2x^2-x+2} \Longrightarrow t^3-2t^2-t+2=(t-1)(t^2-t-2)$$
   $$\polyhornerscheme[x=-1]{x^2-x-2} \Longrightarrow t^3-2t^2-t+2=(t-1)(t+1)(t-2)$$

   Por tanto,
   \begin{equation*}
       \int \frac{dt}{t^3-2t^2- t+2} = \int \frac{dt}{(t-1)(t+1)(t-2)}
   \end{equation*}

   Aplico el método de los coeficientes indeterminados:
   \begin{equation*}
       \frac{1}{(t-1)(t+1)(t-2)} = \frac{A}{t-1}+\frac{B}{t+1}+\frac{C}{t-2} \Longrightarrow 1=A(t+1)(t-2)+B(t-1)(t-2)+C(t^2-1)
   \end{equation*}
   \begin{itemize}
       \item \underline{Para $t=1$} $\Longrightarrow 1=-2A\Longrightarrow A=-\frac{1}{2}$
       \item \underline{Para $t=-1$} $\Longrightarrow 1=6B\Longrightarrow B=\frac{1}{6}$
       \item \underline{Para $t=2$} $\Longrightarrow 1=3C\Longrightarrow C=\frac{1}{3}$
   \end{itemize}

   Por tanto,
   \begin{equation*}
       \int \frac{dt}{t^3-2t^2- t+2} = -\frac{1}{2}\ln|t-1|+\frac{1}{6}\ln|t+1|+\frac{1}{3}\ln|t-2|+C
   \end{equation*}

   Deshaciendo el cambio de variable, tenemos que:
   \begin{equation*}
       \int \frac{dx}{x[\ln^3 x-2\ln^2x-\ln x+2]}
       = -\frac{1}{2}\ln|\ln (x)-1|+\frac{1}{6}\ln|\ln (x)+1|+\frac{1}{3}\ln|\ln (x)-2|+C 
   \end{equation*}

   \item $\displaystyle \int \frac{1+\sen x}{\sen x \cos^2 x}\;dx
   = \int \frac{1}{\sen x \cos^2 x}\;dx+\int \frac{dx}{\cos^2 x}$

   Para la primera integral:
   \begin{multline*}
       \int \frac{1}{\sen x \cos^2 x}\;dx = \MetInt{\cos x =t}{-\sen x\;dx = dt}
       = -\int \frac{1}{\sen x \cos^2 x}\cdot \frac{dt}{\sen x}
       = -\int \frac{dt}{(1-t^2)t^2}
       =\\=
       \int \frac{dt}{(t+1)(t-1)t^2}
   \end{multline*}

   Aplico ahora el método de los coeficientes indeterminados:
   \begin{multline*}
       \frac{1}{(t+1)(t-1)t^2} = \frac{A}{t}+\frac{B}{t^2}+\frac{C}{t+1}+\frac{D}{t-1}
       \Longrightarrow \\ \Longrightarrow
       1= At(t+1)(t-1) + B(t+1)(t-1)+Ct^2(t-1)+Dt^2(t+1)
   \end{multline*}
   \begin{itemize}
       \item \underline{Para $t=0$} $\Longrightarrow 1=-B\Longrightarrow B=-1$
       \item \underline{Para $t=1$} $\Longrightarrow 1=2D\Longrightarrow D=\frac{1}{2}$
       \item \underline{Para $t=-1$} $\Longrightarrow 1=-2C\Longrightarrow C=-\frac{1}{2}$
       \item \underline{Para $t=2$} $\Longrightarrow 1=6A+3B+4C+12D\Longrightarrow A=0$
   \end{itemize}

   Por tanto,
   \begin{equation*}
       -\int \frac{dt}{(1-t^2)t^2} = \frac{1}{t} -\frac{1}{2}\ln|t+1|+\frac{1}{2}\ln|t-1| +C = \frac{1}{t}+\frac{1}{2}\ln \left|\frac{t-1}{t+1}\right|+C
   \end{equation*}

   Deshaciendo el cambio de variable, tenemos que la primera integral es:
   \begin{equation*}
       \int \frac{1}{\sen x \cos^2 x}\;dx
       = \frac{1}{\cos (x)}+\frac{1}{2}\ln \left|\frac{\cos (x)-1}{\cos (x)+1}\right|+C
   \end{equation*}

   Por tanto, el resultado final es:
   \begin{equation*}
       \int \frac{1+\sen x}{\sen x \cos^2 x}\;dx
       = \int \frac{1}{\sen x \cos^2 x}\;dx+\int \frac{dx}{\cos^2 x}
       = \frac{1}{\cos (x)}+\frac{1}{2}\ln \left|\frac{\cos (x)-1}{\cos (x)+1}\right| +\tg x +C
   \end{equation*}
   
\end{enumerate}


\end{ejercicio}

\begin{ejercicio} Sean $a,b\in \bb{R}^\ast$. Calcular:
    \begin{enumerate}
        \item $\displaystyle
            \int \frac{dx}{a^2\sen^2 x + b^2 \cos ^2 x}
            = \int \frac{\red{\frac{1}{\cos^2 x}}dx}{\red{\frac{1}{\cos^2 x}}[a^2\sen^2 x + b^2 \cos ^2 x]}
            = \int \frac{{\frac{1}{\cos^2 x}}dx}{a^2\tg^2 x + b^2}$

        Aplico el siguiente cambio de variable:
        \begin{equation*}
            \MetInt{\tg x =t}{\frac{1}{\cos^2 x}dx = dt}
        \end{equation*}

        Por tanto, la integral queda:
        \begin{equation*}
            \int \frac{dt}{a^2t^2+b^2} = \frac{1}{b^2} \int \frac{dt}{\left(\frac{a}{b}t\right)^2+1}
            = \frac{1}{b^2}\cdot\red{\frac{b}{a}} \int \frac{\red{\frac{a}{b}}dt}{\left(\frac{a}{b}t\right)^2+1} = \frac{1}{ab} \arctan \left(\frac{a}{b}t\right)+C
        \end{equation*}

        Deshaciendo el cambio de variable,
        \begin{equation*}
            \int \frac{dx}{a^2\sen^2 x + b^2 \cos ^2 x}
            = \frac{1}{ab} \arctan \left(\frac{a}{b}\tg x\right)+C
        \end{equation*}

        \item $\displaystyle \int e^{ax}\cos(bx)\;dx$

        Aplico en primer lugar el método de integración por partes:
        $$\MetInt{u(x)=e^{ax}\quad u'(x)=ae^{ax}}{v'(x)=\cos (bx)\quad v(x)=\frac{1}{b}\sen(bx)}$$

        La integral, por tanto, queda:
        \begin{equation*}
            \int e^{ax}\cos(bx)\;dx = \frac{1}{b}e^{ax}\sen(bx)-\int \frac{a}{b}e^{ax}\sen (bx)\;dx
            = \frac{1}{b}e^{ax}\sen(bx)-\frac{a}{b}\int e^{ax}\sen (bx)\;dx
        \end{equation*}

        Aplico de nuevo el método de integración por partes:
        $$\MetInt{u(x)=e^{ax}\quad u'(x)=ae^{ax}}{v'(x)=\sen (bx)\quad v(x)=-\frac{1}{b}\cos(bx)}$$
        
        La integral, por tanto, queda:
        \begin{multline*}
            \int e^{ax}\cos(bx)\;dx
            = \frac{1}{b}e^{ax}\sen(bx)-\frac{a}{b}\int e^{ax}\sen (bx)\;dx
            =\\=
            \frac{1}{b}e^{ax}\sen(bx)+\frac{a}{b^2}e^{ax}\cos(bx)-\frac{a^2}{b^2}\int e^{ax}\cos(bx)\;dx
        \end{multline*}

        Despejando la integral, tenemos que:
        \begin{equation*}
            \int e^{ax}\cos(bx)\;dx = \frac{\frac{1}{b}e^{ax}\sen(bx)+\frac{a}{b^2}e^{ax}\cos(bx)}{1+\frac{a^2}{b^2}}+C
            = \frac{be^{ax}\sen(bx)+ae^{ax}\cos(bx)}{b^2+a^2}+C
        \end{equation*}
        
    \end{enumerate}
\end{ejercicio}

\begin{ejercicio}\label{Ej3} Obtener una fórmula recurrente para las siguientes integrales:
    \begin{enumerate}
        \item $\displaystyle
            \int x^n e^{-x} dx = \MetInt{u(x)=x^n \quad u'(x)=nx^{n-1}}{v'(x)=e^{-x}\quad v(x)=-e^{-x}} = -x^ne^{-x} +\int nx^{n-1}e^{-x}dx =$\\$=\displaystyle-\frac{x^n}{e^x} +n\int x^{n-1}e^{-x}\;dx$

        Definiendo la sucesión $I_n=\int x^n e^{-x}dx$, tenemos que:
        \begin{equation*}
            I_n = -\frac{x^n}{e^x} +nI_{n-1} \qquad \forall n=1,2,\dots
        \end{equation*}
            
        Calculamos el primer elemento de la sucesión, $I_0$:
        \begin{equation*}
            I_0 = \int e^{-x}\;dx = -e^{-x}+C = -\frac{1}{e^x}+C
        \end{equation*}

        \item $\displaystyle \int_0^{\frac{\pi}{2}}\cos^n (x)\;dx$

        Aplicamos el método de integración por partes:
        \begin{equation*}
            \MetInt{u(x)=\cos^{n-1}(x) \qquad u'(x)=-(n-1)\cos^{n-2}(x)\sen(x)}{v'(x)=\cos x \qquad v(x)=\sen x}
        \end{equation*}

        Por tanto,
        \begin{multline*}
            \int_0^{\frac{\pi}{2}}\cos^n (x)\;dx
            = \cancelto{0}{\left[\sen(x)\cos^{n-1}(x)\right]_0^{\frac{\pi}{2}}}
            + \int_0^{\frac{\pi}{2}} (n-1)\cos^{n-2}(x)\sen^2(x)\;dx =
            \\=
            (n-1) \int_0^{\frac{\pi}{2}} \cos^{n-2}(x)(1-\cos^2(x))\;dx =
            \\=
            (n-1)\int_0^{\frac{\pi}{2}} \cos^{n-2}(x)-\cos^n(x)\;dx
        \end{multline*}

        Definiendo la sucesión $I_n=\int_0^{\frac{\pi}{2}}\cos^n (x)\;dx$, tenemos que:
        \begin{equation*}
            I_n = (n-1)(I_{n-2}-I_n) = (n-1)I_{n-2} -(n-1)I_n \Longrightarrow I_n = \frac{(n-1)I_{n-2}}{n} \qquad \forall n=2,3,\dots
        \end{equation*}

        Calculamos $I_0$ y $I_1$:
        \begin{equation*}
            I_0 = \int_{0}^{\frac{\pi}{2}} dx = \frac{\pi}{2}
            \qquad
            I_1 = \int_{0}^{\frac{\pi}{2}} \cos x dx = \left[\sen x\right]_0^{\frac{\pi}{2}}
            = 1
        \end{equation*}
    \end{enumerate}
\end{ejercicio}


\begin{ejercicio}
    Calcular los siguientes límites:
    \begin{enumerate}
        \item $\displaystyle \lim_{x\to 0}\frac{\int_0^x \sen^2(t)\;dt}{x^3}$

        \begin{itemize}
            \item \textbf{\underline{Opción 1: Resolviendo la integral}}
            
            Para resolver esa integral, aplico las siguientes identidades trigonométricas:
            \begin{equation*}
                \left.\begin{array}{c}
                    \sen^2 t + \cos^2 t =1 \\
                    \cos^2 t - \sen^2 t = \cos(2t) 
                \end{array}\right\} \Longrightarrow
                \sen^2 t = \frac{1-\cos(2t)}{2}
            \end{equation*}
        
            Por tanto,
            \begin{multline*}
                \lim_{x\to 0}\frac{\int_0^x \sen^2(t)\;dt}{x^3}
                = \lim_{x\to 0}\frac{\int_0^x \frac{1-\cos(2t)}{2}\;dt}{x^3}
                = \lim_{x\to 0}\frac{\int_0^x 1-\cos(2t)\;dt}{2x^3}
                =\\= \lim_{x\to 0}\frac{\left[t-\frac{1}{2}\sen(2t)\right]_0^x}{2x^3}
                = \lim_{x\to 0}\frac{x-\frac{1}{2}\sen(2x)}{2x^3}
                \Hop
                \lim_{x\to 0}\frac{1-\cos(2x)}{6x^2}
                =\\\Hop
                \lim_{x\to 0}\frac{2\sen(2x)}{12x}
                \Hop
                \lim_{x\to 0}\frac{4\cos(2x)}{12} = \frac{1}{3}
            \end{multline*}


            \item \textbf{\underline{Opción 2: Aplicando el Teorema Fundamental del Cálculo}}

            Sea $f(t)=\sen^2 t$. Como $f(\bb{R})=[0,1]$, tenemos que está acotada. Además, $f$ es continua en $\bb{R}$.

            Definimos $F(x)=\int_0^x f(t)\;dt$, y por el TFC tengo que $F(x)$ es derivable en $\bb{R}$, con $F'(x)=f(x)$.

            Por tanto,
            \begin{multline*}
                \lim_{x\to 0}\frac{\int_0^x \sen^2(t)\;dt}{x^3}
                = \lim_{x\to 0}\frac{F(x)}{x^3}
                = \left[\frac{0}{0}\right]
                \Hop
                \lim_{x\to 0}\frac{F'(x)}{3x^2}
                = \lim_{x\to 0}\frac{f(x)}{3x^2}
                =\\= \lim_{x\to 0}\frac{\sen^2 x}{3x^2}
                \Hop
                \lim_{x\to 0}\frac{\sen (2x)}{6x}
                \Hop
                \lim_{x\to 0}\frac{2\cos (2x)}{6} = \frac{1}{3}
            \end{multline*}
            
        \end{itemize}


        \item $\displaystyle \lim_{x\to 0}\frac{\int_0^{x^2} e^{\sen t}\;dt}{x^2}$

        Sea $f(t)=e^{\sen t}$. Como $f(\bb{R})=[e^{-1},e]$, tenemos que está acotada. Además, $f$ es continua en $\bb{R}$.

            Definimos $F(x^2)=\int_0^x f(t)\;dt$, y por el TFC tengo que $F(x^2)$ es derivable en $\bb{R}$, con $F'(x^2)=f(x) \Longrightarrow F'(x)=2f(x)$.

            Por tanto,
            \begin{multline*}
                \lim_{x\to 0}\frac{\int_0^{x^2} e^{\sen t}\;dt}{x^2}
                = \lim_{x\to 0}\frac{F(x^2)}{x^2}
                = \left[\frac{0}{0}\right]
                \Hop
                \lim_{x\to 0}\frac{F'(x^2)}{2x}
                = \lim_{x\to 0}\frac{2xf(x^2)}{2x}
                =\\= \lim_{x\to 0} f(x^2)
                = \lim_{x\to 0} = e^{\sen x^2} = e^0 = 1
            \end{multline*}
    \end{enumerate}
\end{ejercicio}

\begin{ejercicio}
    Calcular las siguientes integrales:
    \begin{enumerate}
        \item $\displaystyle \int_0^1 \frac{x^2}{\sqrt[4]{x^3+1}}\;dx$

        \item $\displaystyle \int_0^\frac{\pi}{4} \sqrt{\cos x}\sen x\;dx$

        \item $\displaystyle \int_1^2 \frac{x^2}{1-2x^3}\;dx$

        \item $\displaystyle \int_e^{e^2} \frac{dx}{x\ln x}$

        \item $\displaystyle \int_0^1 xe^{ax^2+b}\;dx \qquad (a,b\in \bb{R})$

        \item $\displaystyle \int_0^1 a^{2x}\;dx \qquad (a\in \bb{R}^+)$

        \item $\displaystyle \int_0^1 \frac{dx}{1+e^x}$

        \item $\displaystyle \int_0^{\frac{\pi}{4}} \frac{1+\sen x}{\cos^2 x}\;dx$

        \item $\displaystyle \int_0^1 \frac{x}{x^4+3}\;dx$

        \item $\displaystyle \int_{\frac{\pi}{6}}^{\frac{\pi}{3}} \frac{dx}{\sen^2 x}$

        \item $\displaystyle \int_0^1 e^x \cos(e^x)\;dx$

        \item $\displaystyle \int_0^{\frac{1}{2}} \frac{x^2}{\sqrt{1-x^6}}\;dx$

        \item[ll)] $\displaystyle \int_{\frac{1}{4}}^{\frac{3}{4}} \frac{\arcsen x}{\sqrt{1-x^2}}\;dx$

        \item $\displaystyle \int_0^{\frac{1}{2}} \frac{1}{\sqrt{20+8x-x^2}}\;dx$

        \item $\displaystyle \int_0^1 \frac{x+3}{\sqrt{3+4x-4x^2}}\;dx$

        \item $\displaystyle \int_0^1 x^3 e^{2x}\;dx$

        \item $\displaystyle \int_0^{\frac{\pi}{2}} x\cos x\;dx$

        \item $\displaystyle \int_1^e \ln x\;dx$

        \item $ \displaystyle \int_a^b \frac{dx}{(x+a)(x+b)} \qquad (0<a<b)$.
        \begin{comment}
        \begin{equation*}
            \int_a^b \frac{dx}{(x+a)(x+b)} = \int_a^b \frac{A}{x+a}dx + \int_a^b\frac{B}{x+b}dx = \frac{1}{b-a}\int_a^b \frac{dx}{x+a} - \frac{1}{b-a}\int_a^b \frac{dx}{x+b} = \left[\frac{1}{b-a}\ln |x+a| -\frac{1}{b-a}\ln |x+b|\right]^b_a = \left[\frac{1}{b-a}\ln \left|\frac{x+a}{x+b}\right|\right]^b_a = \frac{1}{b-a}\left(\ln\left(\frac{b+a}{2b}\right)-\ln\left(\frac{2a}{a+b}\right)\right)
        \end{equation*}

        Aplico el método de los coeficientes indeterminados:
        \begin{equation*}
            A(x+b) +B(x+a)=1
        \end{equation*}
        \end{comment}

        \item $\displaystyle \int_0^1 \frac{dx}{x^3+1}$

        \begin{comment}
        \begin{equation*}
            \int_0^1 \frac{dx}{x^3+1}= \int_0^1 \frac{dx}{(x+1)\left[\left(x-\frac{1}{2}\right)^2\right]}
        \end{equation*}
        \end{comment}
        

        \item $\displaystyle \int_1^2 \frac{x-1}{x(x^2+1)^2}\;dx$
    \end{enumerate}
\end{ejercicio}

\begin{ejercicio} Estudiar la convergencia y, cuando la haya, calcular el valor las siguientes integrales:
\begin{enumerate}
    \item $\displaystyle \int_a^{+\infty} x^n\;dx \quad (a>0)$
    \begin{itemize}
        \item \underline{Para $n\neq -1$}:
        \begin{equation*}
            \int_a^{+\infty} x^n\;dx = \lim_{c\to \infty}\left[\frac{x^{n+1}}{n+1}\right]_a^c = \lim_{c\to \infty} \frac{c^{n+1}-a^{n+1}}{n+1} = \left\{\begin{array}{cc}
                \infty & n>-1 \\
                -\frac{a^{n+1}}{n+1} & n<-1
            \end{array}\right.
        \end{equation*}

        \item \underline{Para $n= -1$}:
        \begin{equation*}
            \int_a^{+\infty} \frac{1}{x}\;dx = \lim_{c\to \infty}\left[\ln x\right]_a^c = \lim_{c\to \infty} \ln \frac{c}{a} = \infty
        \end{equation*}
    \end{itemize}
    

    Por tanto, tenemos que:
    \begin{equation*}
        \int_a^{+\infty} x^n\;dx = \left\{\begin{array}{cc}
            \infty & n\geq -1 \\
            \displaystyle -\frac{a^{n+1}}{n+1} & n<-1
        \end{array}\right.
    \end{equation*}
    
    \item $\displaystyle \int_{-\infty}^{0} e^x\;dx  = \lim_{c\to -\infty}\left[e^x\right]_c^0 = 1$
    
    \item $\displaystyle \int_0^{+\infty} \frac{dx}{\sqrt{e^x}} = \lim_{c\to \infty} \int_0^c (e^x)^{-\frac{1}{2}}\;dx
        = \lim_{c\to \infty} \int_0^c e^{-\frac{x}{2}}\;dx
        = \lim_{c\to \infty} \left[-2e^{-\frac{x}{2}}\right]_0^c = 2$
    
    
    \item $\displaystyle \int_{-\infty}^{+\infty} e^{-a|x|}\;dx \quad (a\in \bb{R})$
    
    \begin{itemize}
        \item \underline{Para $a=0$}:
        \begin{equation*}
            \int_{-\infty}^{+\infty} e^{0}\;dx
            = \int_{-\infty}^{+\infty}\;dx = \infty = \lim_{c\to -\infty} [t]_c^0 + \lim_{c\to \infty}[t]_0^c = \infty
        \end{equation*}

        \item \underline{Para $a>0$}:
        \begin{multline*}
            \int_{-\infty}^{+\infty} e^{-a|x|}\;dx
            = \int_{-\infty}^{0} e^{ax}\;dx
            + \int_{0}^{+\infty} e^{-ax}\;dx
            = \lim_{c\to -\infty}\left[\frac{1}{a}e^{ax}\right]_c^0
            + \lim_{c\to \infty}\left[-\frac{1}{a}e^{-ax}\right]_0^c
            =\\=
            \frac{1}{a} - 0 +0 +\frac{1}{a} = \frac{2}{a}
        \end{multline*}

        \item \underline{Para $a<0$}:
        \begin{multline*}
            \int_{-\infty}^{+\infty} e^{-a|x|}\;dx
            = \int_{-\infty}^{0} e^{ax}\;dx
            + \int_{0}^{+\infty} e^{-ax}\;dx
            = \lim_{c\to -\infty}\left[\frac{1}{a}e^{ax}\right]_c^0
            + \lim_{c\to \infty}\left[-\frac{1}{a}e^{-ax}\right]_0^c
            =\\=
            \frac{1}{a} + \infty +\infty +\frac{1}{a} = \infty
        \end{multline*}
    \end{itemize}

    
    \item $\displaystyle \int_{0}^{+\infty} \frac{dx}{e^x+e^{-x}}$
    \begin{multline*}
        \int_{0}^{+\infty} \frac{dx}{e^x+e^{-x}} = \MetInt{e^x = t}{dx=\frac{dt}{t}} = \lim_{c\to \infty}\int_1^c \frac{dt}{t(t+t^{-1})}
        = \lim_{c\to \infty}\int_1^c \frac{dt}{1+t^2}
        =\\= \lim_{c\to \infty}\left[\arctan t\right]_1^c=\frac{\pi}{2} - \frac{\pi}{4} = \frac{\pi}{4}
    \end{multline*}
    \item $\displaystyle \int_{2}^{+\infty} \frac{dx}{x(\ln x)^8}$
    \begin{multline*}
        \int_{2}^{+\infty} \frac{dx}{x(\ln x)^8} = \MetInt{\ln x = t}{\frac{dx}{x} = dt} = \lim_{c\to \infty} \int_{\ln 2}^c \frac{dt}{t^8} = \lim_{c\to \infty}\left[\frac{t^{-7}}{-7}\right]_{\ln 2}^c
        = \lim_{c\to \infty}\left[\frac{1}{-7t^7}\right]_{\ln 2}^c = \frac{1}{7\ln^7 2}
    \end{multline*}

    \item $\displaystyle \int_{-\infty}^{+\infty} e^{x-e^x}\;dx$
    \begin{multline*}
        \int_{-\infty}^{+\infty} e^{x-e^x}\;dx
        = \int_{-\infty}^{+\infty} e^{x}e^{-e^x}\;dx
        = \lim_{c\to \infty}\int_{-c}^c e^{x}e^{-e^x}\;dx
        = \MetInt{e^x = t}{e^x dx = dt}
        =\\
        =\lim_{c\to \infty}\int_{e^{-c}}^{e^c} e^{-t}\;dt
        =\lim_{c\to \infty} \left[-e^{-t}\right]_{e^{-c}}^{e^c}
        = -e^{-\infty} +e^0 = 1
    \end{multline*}
    
    \item $\displaystyle \int_{0}^{+\infty} e^{-x}\sen x\;dx$
    \begin{multline*}
        \int_{0}^{+\infty} e^{-x}\sen x\;dx
        = \lim_{c\to \infty} \int_{0}^{c} e^{-x}\sen x\;dx
        = \MetInt{u(x) = e^{-x}\quad u'(x)=-e^{-x}}{v'(x)=\sen x \quad v(x)=-\cos x}
        =\\= \lim_{c\to \infty}\left[\left.-e^{-x}\cos x\right|_0^c -\int_0^c e^{-x}\cos x\;dx\right]
        = \MetInt{u(x) = e^{-x}\quad u'(x)=-e^{-x}}{v'(x)=\cos x \quad v(x)=\sen x}
        =\\= \lim_{c\to \infty}\left[\left.-e^{-x}\cos x-e^{-x}\sen x\right|_0^c -\int_0^c e^{-x}\sen x\;dx\right]
    \end{multline*}

    Por tanto, tenemos que:
    \begin{equation*}
        2\lim_{c\to \infty} \int_{0}^{c} e^{-x}\sen x\;dx = \lim_{c\to \infty}\left[\left.-e^{-x}\cos x-e^{-x}\sen x\right|\right]_0^c \Longrightarrow \int_{0}^{+\infty} e^{-x}\sen x\;dx = \frac{1}{2}
    \end{equation*}

    \item $\displaystyle \int_{0}^{+\infty} x^ne^{-x}\;dx \quad (n\in \bb{N})$

    \begin{multline*}
        \int_{0}^{+\infty} x^ne^{-x}\;dx = \lim_{c\to \infty} \int_{0}^{c} x^ne^{-x}\;dx
        = \MetInt{u(x)=x^n \quad u'(x)=nx^{n-1}}{v'(x)=e^{-x}\quad v(x)=-e^{-x}}
        =\\=
        \lim_{c\to \infty}\left.-x^ne^{-x}\right|_0^c +\int_0^c nx^{n-1}e^{-x}dx
        = \lim_{c\to \infty}
        \left.-\frac{x^n}{e^x}\right|_0^c +n\int_0^c x^{n-1}e^{-x}\;dx
        =\\
        = \lim_{c\to \infty}
        -\frac{c^n}{e^c} +n\int_0^c x^{n-1}e^{-x}\;dx
        = \lim_{c\to \infty}
        n\int_0^c x^{n-1}e^{-x}\;dx
        = n\int_0^{\infty}x^{n-1}e^{-x}\;dx
    \end{multline*}

        Definiendo la sucesión $\displaystyle I_n=\int_0^{\infty} x^n e^{-x}dx$, tenemos que:
        \begin{equation*}
            I_n = nI_{n-1} = n(n-1)I_{n-2} = \dots = n!I_0 = n!
        \end{equation*}
            
        Esto se debe a que $I_0 = 1$:
        \begin{equation*}
            I_0 = \int_0^\infty e^{-x}\;dx
            =\lim_{c\to \infty} \int_0^c e^{-x}\;dx
            =\lim_{c\to \infty} \left[-e^{-x}\right]_0^c = 1
        \end{equation*}

    
    \item $\displaystyle \int_{0}^{+\infty} \frac{dx}{4+x^2}$
    \item $\displaystyle \int_{-\infty}^{+\infty} \frac{x^2-x+2}{x^4+10x^2+9}\;dx$
    \item $\displaystyle \int_{1}^{+\infty} \frac{\ln x}{x^2}\;dx$
    \item[ll)] $\displaystyle \int_{0}^3 \frac{dx}{\sqrt{9-x^2}}$
    \item $\displaystyle \int_{a}^{b} \frac{dx}{(b-x)^\alpha}\quad (\alpha\in \bb{R}, a<b)$
    \item $\displaystyle \int_{0}^{4} \frac{dx}{(x-1)^2}$
    \item $\displaystyle \int_{-1}^{1} \frac{dx}{\sqrt[3]{x}}$
    \item $\displaystyle \int_{0}^{4} \frac{dx}{\sqrt[3]{x-1}}$
    \item $\displaystyle \int_{0}^{\frac{\pi}{2}} \frac{\cos x}{\sqrt{\sqrt{1-\sen x}}}\;dx$
    \item $\displaystyle \int_{0}^{1} \frac{\ln x}{\sqrt{x}}\;dx$
    \item $\displaystyle \int_{0}^{1} \frac{\arcsen x}{\sqrt{1-x^2}}\;dx$
    \item $\displaystyle \int_{0}^{+\infty} \frac{e^{-\sqrt{x}}}{\sqrt{x}}\;dx$
    
\end{enumerate}
\end{ejercicio}

\begin{ejercicio}
    Determinar los valores de $\alpha\in \bb{R}$ para los cuales la siguiente integral converge, calculando el valor de la misma:
    \begin{equation*}
        \int_0^{+\infty} \frac{x}{2x^2+2\alpha}-\frac{\alpha}{x+1}dx
    \end{equation*}

    Tenemos que:
    \begin{multline*}
        \int_0^{+\infty} \frac{x}{2x^2+2\alpha}-\frac{\alpha}{x+1}dx
        = \lim_{c\to \infty} \int_0^{c} \frac{x}{2x^2+2\alpha}-\frac{\alpha}{x+1}dx
        = \lim_{c\to \infty} \left[\frac{1}{4}\ln |2x^2+2\alpha| - \alpha \ln |x+1|\right]_0^c
        =\\=
        \lim_{c\to \infty} \left[ \ln \left|\frac{\sqrt[4]{|2x^2+2\alpha|}}{(x+1)^\alpha}\right|\right]_0^c
        = \lim_{c\to \infty} \ln \left|\frac{\sqrt[4]{|2c^2+2\alpha|}}{(c+1)^\alpha}\right| - \ln \left|\sqrt[4]{|2\alpha|}\right|
        = \lim_{c\to \infty} \ln \left|\frac{\sqrt[4]{|2c^2+2\alpha|}}{(c+1)^\alpha\sqrt[4]{|2\alpha|}}\right|
        =\\=
        \lim_{c\to \infty} \ln \sqrt[4]{\left|\frac{2c^2+2\alpha}{(c+1)^{4\alpha}\cdot  2\alpha}\right|}
        = \lim_{c\to \infty} \ln \sqrt[4]{\left|\frac{c^2+\alpha}{(c+1)^{4\alpha}\cdot  \alpha}\right|} = \left\{
        \begin{array}{lll}
            \ln \infty = \infty & \text{si} & 2>4\alpha \\
            \ln \sqrt[4]{\frac{1}{\alpha}} = \ln \sqrt[4]{2} & \text{si} & 2=4\alpha \\
            \ln 0=-\infty & \text{si} & 2<4\alpha \\
        \end{array}
        \right.
    \end{multline*}

    Por tanto, como tenemos que el integrando es localmente integrable, tenemos que solo converge para $\alpha=\frac{1}{2}$, valor para el cual la integral vale $\ln \sqrt[4]{2}$.
\end{ejercicio}

\begin{ejercicio}
    Determinar los valores de $\alpha,\beta\in \bb{R}$ tales que
    \begin{equation*}
        \int_1^{+\infty} \left(\frac{2x^2+\beta x+\alpha}{x(2x+\alpha)}-1\right)\;dx = 1.
    \end{equation*}

    Resuelvo en primer lugar la integral indefinida:
    \begin{equation*}\label{Ej8.Div}
        \int \left(\frac{2x^2+\beta x+\alpha}{x(2x+\alpha)}-1\right)\;dx
        =
        \int \frac{(\beta-\alpha)x + \alpha}{x(2x+\alpha)}\;dx
    \end{equation*}
    

    \begin{itemize}
        \item \underline{Para $\alpha\neq 0$}:
        Por tanto, aplicamos el método de los coeficientes indeterminados:
        \begin{equation*}
            \frac{1}{x(2x+\alpha)} = \frac{A'}{x} + \frac{B'}{2x+\alpha} = \frac{A'(2x+\alpha) +B'x}{x(2x+\alpha)}
        \end{equation*}
        \begin{itemize}
            \item Para $x=0\Longrightarrow 1 = A'\alpha \Longrightarrow A'=\frac{1}{\alpha}$
            \item Para $x=-\frac{\alpha}{2}\Longrightarrow 1=-B'\frac{\alpha}{2} \Longrightarrow B'=-\frac{2}{\alpha}$
        \end{itemize}
        Por tanto,
        \begin{multline*}
            \int \left(\frac{2x^2+\beta x+\alpha}{x(2x+\alpha)}-1\right)\;dx
            \stackrel{Ec.\;\ref{Ej8.Div}}{=}
            \int \frac{(\beta-\alpha)x + \alpha}{x(2x+\alpha)}\;dx
            =\\= (\beta - \alpha)\int \frac{x}{x(2x + \alpha)}\;dx +\alpha \left[\frac{1}{\alpha}\int \frac{dx}{x}\; -\frac{2}{\alpha}\int \frac{dx}{2x+\alpha}\right] =
            \\= \frac{(\beta-\alpha)}{\red{4}} \int \frac{\red{4}x}{x(2x + \alpha)}\;dx +\int \frac{dx}{x}\; -2\int \frac{dx}{2x+\alpha} =
            \\= \frac{(\beta-\alpha)}{4}\ln |2x^2+\alpha x| +\ln |x| - \ln |2x+\alpha| + C
            \\= \frac{(\beta-\alpha)}{4}\ln |2x^2+\alpha x| +\ln \left|\frac{x}{2x+\alpha}\right| + C
        \end{multline*}
        
        \item \underline{Para $\alpha = 0$}:
        \begin{equation*}
            \int \left(\frac{2x^2+\beta x}{2x^2}-1\right)\;dx
            \stackrel{Ec.\;\ref{Ej8.Div}}{=}
            \int \frac{\beta x}{2x^2}\;dx = \frac{\beta}{4}\ln |2x^2| + C
        \end{equation*}
    \end{itemize}
    
    \vspace{1cm}
    Una vez tenemos la integral indefinida, calculamos la integral definida.
    \begin{itemize}
        \item \underline{Para $\alpha = 0$}:
        \begin{equation*}
            \int_1^{+\infty} \left(\frac{2x^2+\beta x}{2x^2}-1\right)\;dx
            = \frac{\beta}{4}\lim_{c\to \infty} \left[\ln 2x^2\right]_1^c
            = \frac{\beta}{4}\lim_{c\to \infty}\ln 2c^2
        \end{equation*}
        Por tanto, para $\alpha=0$ tenemos que converge si y solo si $\beta = 0$. No obstante, tenemos que:
        \begin{equation*}
            \int_1^{+\infty} \left(\frac{2x^2}{2x^2}-1\right)\;dx = 0
        \end{equation*}
        Por tanto, para $\alpha = 0$ no se puede dar.


        \item \underline{Supuesto $\alpha \neq 0$}:

        Veamos antes si la siguiente integral definida converge:
        \begin{multline*}
           \int_{\left|\frac{\alpha}{2}\right|+1}^{+\infty} \left(\frac{2x^2+\beta x+\alpha}{x(2x+\alpha)}-1\right)\;dx
            = \lim_{c\to +\infty} \left[\frac{(\beta-\alpha)}{4}\ln |2x^2+\alpha x| +\ln \left|\frac{x}{2x+\alpha}\right|\right]_{\left|\frac{\alpha}{2}\right|+1}^{c}
       \end{multline*}

       Como en el valor de la primitiva en ${\left|\frac{\alpha}{2}\right|+1}$ es finito, ya que se trata de un valor del dominio de dicha función continua, para que esa integral converja es necesario que el límite en $+\infty$ de la integral converja.
       \begin{equation*}
           \left[\lim_{c\to +\infty} \frac{(\beta-\alpha)}{4}\ln |2c^2+\alpha c| +\ln \left|\frac{c}{2c+\alpha}\right|\right] \in \bb{R}\Longleftrightarrow \beta -\alpha = 0\Longleftrightarrow \beta = \alpha
       \end{equation*}

       Por tanto, si $\alpha \neq 0$ es necesario que $\alpha = \beta$ para que la integral de partida converja, ya que:
       \begin{multline*}
           \int_1^{+\infty} \left(\frac{2x^2+\beta x+\alpha}{x(2x+\alpha)}-1\right)\;dx
           =\\= \int^{\left|\frac{\alpha}{2}\right|+1}_1 \left(\frac{2x^2+\beta x+\alpha}{x(2x+\alpha)}-1\right)\;dx
           +
           \int_{\left|\frac{\alpha}{2}\right|+1}^{+\infty} \left(\frac{2x^2+\beta x+\alpha}{x(2x+\alpha)}-1\right)\;dx
       \end{multline*}
       Como observación, cabe destacar que en el caso de que $-\frac{\alpha}{2}>1$, la primera integral sería impropia y también sería necesario dividirla en dos integrales. No obstante, como la segunda parte ya vemos que solo converge para $\alpha=~\beta$, tenemos que esto es condición necesaria para que la integral de partida converja. Por tanto, procedemos suponiendo $\alpha = \beta\neq 0$:
       \begin{itemize}
           \item \underline{Supuesto $-\frac{\alpha}{2}\leq 1\Longleftrightarrow -\alpha \leq 2 \Longleftrightarrow \alpha > -2$}.

            Tenemos que el integrando no tiene ninguna asíntota vertical. Por tanto:
            \begin{multline*}
               \int_1^{+\infty} \left(\frac{2x^2+\beta x+\alpha}{x(2x+\alpha)}-1\right)\;dx
               =
               \lim_{c\to +\infty} \int_1^{c} \left(\frac{2x^2+\beta x+\alpha}{x(2x+\alpha)}-1\right)\;dx =\\= \lim_{c\to +\infty} \left[\ln \left|\frac{x}{2x+\alpha}\right|\right]_1^{c} = \ln \frac{1}{2} - \ln \frac{1}{2+\alpha} = \ln \frac{2+\alpha}{2} = 1 \Longleftrightarrow \frac{2+\alpha}{2} = e \Longleftrightarrow \\ \Longleftrightarrow
               \alpha = 2e-2
           \end{multline*}
    
           Además, tenemos que $2e-2 >-2 \Longleftrightarrow 2e>0$, por lo que estamos en esta suposición. Por tanto, tenemos que si $\alpha = \beta = 2e-2$, la integral impropia vale 1.




           \item \underline{Supuesto $-\frac{\alpha}{2}\geq 1\Longleftrightarrow -\alpha \geq 2 \Longleftrightarrow \alpha < -2$}.

           Como la asíntota pertenece al intervalo de integración, rompemos la integral en ese punto.
           \begin{equation*}
               \int_1^{-\frac{\alpha}{2}} \left(\frac{2x^2+\beta x+\alpha}{x(2x+\alpha)}-1\right)\;dx
               + \int_{-\frac{\alpha}{2}}^{+\infty} \left(\frac{2x^2+\beta x+\alpha}{x(2x+\alpha)}-1\right)\;dx
           \end{equation*}

            Veamos no obstante si la primera integral es convergente:
           \begin{multline*}
               \int_1^{-\frac{\alpha}{2}} \left(\frac{2x^2+\beta x+\alpha}{x(2x+\alpha)}-1\right)\;dx =
               \lim_{c\to -\frac{\alpha}{2}^-} \int_1^{c} \left(\frac{2x^2+\beta x+\alpha}{x(2x+\alpha)}-1\right)\;dx
               =\\=
               \lim_{c\to -\frac{\alpha}{2}^-} \left[\ln \left|\frac{x}{2x+\alpha}\right|\right]_1^{c} 
               = \infty -\ln \left|\frac{1}{2+\alpha}\right| = \infty
           \end{multline*}

           Por tanto, tenemos que en este caso la integral no es convergente.
       \end{itemize}
    \end{itemize}

    Por tanto concluimos que:
    \begin{equation*}
        \int_1^{+\infty} \left(\frac{2x^2+\beta x+\alpha}{x(2x+\alpha)}-1\right)\;dx = 1 \Longleftrightarrow \alpha = \beta = 2e-2
    \end{equation*}
    
\end{ejercicio}

\begin{ejercicio}
    Se define la función gamma como la función $\Gamma:\bb{R}\to \bb{R}$ dada por:
    \begin{equation*}
        \Gamma(x) = \int_0^{+\infty} t^{x-1} e^{-t}\;dt
    \end{equation*}

    \begin{enumerate}
        \item Probar que dicha integral converge para $x>0$ y diverge para $x\leq 0$.
        \item Probar que $\Gamma(x+1)=x\Gamma(x)$, para cada $x>0$.
        \item Deducir que $\Gamma(n) = (n-1)!$ para cada $n\in \bb{N}$.
    \end{enumerate}
\end{ejercicio}



\begin{ejercicio} Justificar la convergencia o divergencia de las siguientes integrales (sin necesidad de resolverlas):
\begin{enumerate}

    \item $\displaystyle \int_1^{+\infty} \frac{x^2+3x+1}{x^4+x^3+\sqrt{x}}\;dx$

    Sea $f(x)=\frac{x^2+3x+1}{x^4+x^3+\sqrt{x}}$.
    Sea $g(x)=\frac{x^2+3x+1}{x^4}$. Veamos que $\int_1^{+\infty} g(x)\;dx$ converge:
    \begin{multline*}
        \int_1^{+\infty} g(x)\;dx
        = \lim_{t\to +\infty} \int_1^t \frac{x^2+3x+1}{x^4}\;dx
        = \lim_{t\to +\infty} \int_1^t \frac{x^2}{x^4} +\frac{3x}{x^4} +\frac{1}{x^4}\;dx
        =\\= \lim_{t\to +\infty} \int_1^t \frac{1}{x^2} +3\cdot \frac{1}{x^3} +\frac{1}{x^4}\;dx
        = \lim_{t\to +\infty} \left[\frac{x^{-1}}{-1}+3\cdot \frac{x^{-2}}{-2} +\frac{x^{-3}}{-3}\right]_1^t
        =\\=\lim_{t\to +\infty} -\frac{1}{t} -\frac{3}{2}\cdot \frac{1}{t^2} -\frac{1}{3}\cdot \frac{1}{t^3} +1+\frac{3}{2}+\frac{1}{3} = \frac{17}{6}\in \bb{R}
    \end{multline*}

    Como ese límite es $L=\frac{17}{6}\in \bb{R}$, tenemos que converge. Probamos ahora que $g(x)\geq f(x) \quad \forall x\in [1,+\infty[$:
    \begin{multline*}
        g(x)\geq f(x) \Longleftrightarrow \frac{x^2+3x+1}{x^4} \geq \frac{x^2+3x+1}{x^4+x^3+\sqrt{x}}
        \Longleftrightarrow x^4+x^3+\sqrt{x} \geq x^4 \Longleftrightarrow \\ \Longleftrightarrow
        x^3+\sqrt{x} \geq 0 \Longleftrightarrow x\geq 0 \qquad \text{Cierto.}
    \end{multline*}
    Por tanto, tenemos que $g(x)\geq f(x)\quad \forall x\in [1,+\infty[$.\\

    Como $f(x)\leq g(x) \qquad \forall x\in [1, +\infty[$ y el orden se conserva en las integrales de Riemman, por el Criterio de Comparación tenemos que:
    \begin{equation*}
        \int_1^{+\infty} g(x)\;dx \text{ converge} \Longrightarrow 
        \int_1^{+\infty} f(x)\;dx \text{ converge}
    \end{equation*}

    Por tanto, tenemos que  $\displaystyle \int_1^{+\infty} \frac{x^2+3x+1}{x^4+x^3+\sqrt{x}}\;dx$ sí converge.




    \item $\displaystyle \int_1^{+\infty} \frac{1}{2x+\sqrt[3]{x+1} +5}\;dx$

    Sea $f(x)= \frac{1}{2x+\sqrt[3]{x+1} +5}$.
    Sea $g(x)=\frac{1}{x}$. Veamos que $\int_1^{+\infty} g(x)\;dx$ diverge positivamente:
    \begin{equation*}
        \int_1^{+\infty} g(x)\;dx
        = \lim_{t\to +\infty} \int_1^t \frac{1}{x};dx
        = \lim_{t\to +\infty} [\ln |x|]_1^t
        = \lim_{t\to +\infty} \ln t = +\infty
    \end{equation*}

    Calculamos ahora el siguiente límite:
    \begin{equation*}
        \lim_{x\to \infty} \frac{f(x)}{g(x)}
        = \lim_{x\to \infty} \frac{x}{{2x+\sqrt[3]{x+1} +5}} = \frac{1}{2}=L\in \bb{R}^\ast
    \end{equation*}
    
    Como $L\in \bb{R}^\ast$, tenemos que, por el Criterio Límite de Comparación,
    \begin{equation*}
        \int_1^{+\infty} g(x)\;dx \text{ diverge positivamente} \Longleftrightarrow 
        \int_1^{+\infty} f(x)\;dx \text{ diverge positivamente}
    \end{equation*}

    Por tanto, tenemos que  $\displaystyle \int_1^{+\infty} \frac{1}{2x+\sqrt[3]{x+1} +5}\;dx$ diverge positivamente.


    \item $\displaystyle \int_0^{+\infty} \frac{x}{\sqrt{x^4+1}}\;dx$

    Sea $f(x)= \frac{x}{\sqrt{x^4+1}}$.
    Sea $g(x)=\frac{1}{x+1}$. Veamos que $\int_0^{+\infty} g(x)\;dx$ diverge positivamente:
    \begin{equation*}
        \int_0^{+\infty} g(x)\;dx
        = \lim_{t\to +\infty} \int_0^t \frac{1}{x+1};dx
        = \lim_{t\to +\infty} [\ln |x+1|]_0^t
        = \lim_{t\to +\infty} \ln t = +\infty
    \end{equation*}

    Calculamos ahora el siguiente límite:
    \begin{equation*}
        \lim_{x\to \infty} \frac{f(x)}{g(x)}
        = \lim_{x\to \infty} \frac{x^2+x}{\sqrt{x^4+1}} = 1=L\in \bb{R}^\ast
    \end{equation*}
    
    Como $L\in \bb{R}^\ast$, tenemos que, por el Criterio Límite de Comparación,
    \begin{equation*}
        \int_0^{+\infty} g(x)\;dx \text{ diverge positivamente} \Longleftrightarrow 
        \int_0^{+\infty} f(x)\;dx \text{ diverge positivamente}
    \end{equation*}

    Por tanto, tenemos que  $\displaystyle \int_0^{+\infty} \frac{x}{\sqrt{x^4+1}}\;dx$ diverge positivamente.



    \item $\displaystyle \int_0^{+\infty} \frac{x^2}{(a^2+x^2)^\frac{3}{2}}\;dx$

    Sea $f(x)=\frac{x^2}{(a^2+x^2)^\frac{3}{2}}$.
    Sea $g(x)=\frac{1}{x+1}$, y por el ejercicio anterior tenemos que $\int_0^{+\infty} g(x)\;dx$ diverge positivamente. Calculamos el siguiente límite:
    \begin{equation*}
        \lim_{x\to \infty} \frac{f(x)}{g(x)}
        = \lim_{x\to \infty} \frac{x^3+x}{(a^2+x^2)^\frac{3}{2}} = 1=L\in \bb{R}^\ast
    \end{equation*}
    
    Como $L\in \bb{R}^\ast$, tenemos que, por el Criterio Límite de Comparación,
    \begin{equation*}
        \int_0^{+\infty} g(x)\;dx \text{ diverge positivamente} \Longleftrightarrow 
        \int_0^{+\infty} f(x)\;dx \text{ diverge positivamente}
    \end{equation*}

    Por tanto, tenemos que  $\displaystyle \int_0^{+\infty} \frac{x^2}{(a^2+x^2)^\frac{3}{2}}\;dx$ diverge positivamente.




    \item $\displaystyle \int_{-\infty}^{+\infty} e^{-x^2}\;dx$

    Vemos, en primer lugar, que la función $f(x)=e^{-x^2}$ es par. Por tanto, tenemos que:
    \begin{equation*}
        \int_{-\infty}^{+\infty} e^{-x^2}\;dx
        = \int_{-\infty}^{-1} e^{-x^2}\;dx
        + \int_{-1}^{1} e^{-x^2}\;dx
        + \int_{1}^{+\infty} e^{-x^2}\;dx
        = \int_{-1}^{1} e^{-x^2}\;dx
        + 2\int_{1}^{+\infty} e^{-x^2}\;dx
    \end{equation*}

    Como $f(x)$ está acotada en $[-1,1]$ y es Riemman Integrable, tenemos que esa parte de la integral converge. Veamos la integral impropia restante. Definimos $g(x)=e^{-x}$, y veamos que su integral entre 1 y $+\infty$ converge:
    \begin{equation*}
        \int_1^\infty e^{-x}dx = \lim_{t\to \infty} \int_1^te^{-x}dx
        = \lim_{t\to \infty} \left[-e^{-x}\right]_1^t
        = \lim_{t\to \infty} -e^{-t} +\frac{1}{e} = \frac{1}{e}
    \end{equation*}

    Como tenemos que $L=\frac{1}{e}\in \bb{R}$, tenemos que dicha integral impropia converge. Vemos ahora que $g(x)\geq f(x)$:
    \begin{equation*}
        g(x)\geq f(x)
        \Longleftrightarrow 
        e^{-x} \geq e^{-x^2}
        \Longleftrightarrow
        -x \geq -x^2
        \Longleftrightarrow
        x \leq x^2 \qquad \text{Cierto } (x\in [1,+\infty])
    \end{equation*}

    Por tanto, por el Criterio de Comparación, como $\int_1^{+\infty}g(x)\;dx$ converge, tenemos que $\int_1^{+\infty}f(x)\;dx$.

    Por lo mencionado anteriormente, tenemos que la integral buscada es suma de integrales convergentes, por lo que converge a la suma de los límites. Es decir, $\displaystyle \int_{-\infty}^{+\infty} e^{-x^2}\;dx$ converge.

    
    \item $\displaystyle \int_0^1 \frac{dx}{\sqrt{1-x^4}}$

    Sea $\displaystyle f(x)=\frac{1}{\sqrt{1-x^4}}$. Como $\displaystyle \lim_{x\to 1}f(x)\to +\infty$, tenemos que $f(x)$ no está acotada, por lo que se trata de una integral impropia.

    Sea $g(x)=\frac{1}{\sqrt{1-x}}$. Veamos que $\int_0^1 g(x)\;dx$ converge:
    \begin{multline*}
        \int_0^1 g(x)\;dx
        = \lim_{t\to 1} \int_0^t \frac{dx}{\sqrt{1-x}}
        = \lim_{t\to 1} \red{-2}\int_0^t \frac{\red{-}1}{\red{2}\sqrt{1-x}}\;dx
        = \lim_{t\to 1} \left[-2\sqrt{1-x}\right]_0^t 
        =\\=
        \lim_{t\to 1} -2\sqrt{1-t} +2\sqrt{1} = 2
    \end{multline*}

    Como ese límite es $L=2\in \bb{R}$, tenemos que converge. Probamos ahora que $g(x)\geq f(x) \quad \forall x\in [0,1]$:
    \begin{multline*}
        g(x)\geq f(x) \Longleftrightarrow \frac{1}{\sqrt{1-x}} \geq \frac{1}{\sqrt{1-x^4}}
        \Longleftrightarrow \sqrt{1-x^4} \geq \sqrt{1-x} \Longleftrightarrow \\ \Longleftrightarrow
        1-x^4\geq 1-x
        \Longleftrightarrow x\geq x^4 \qquad \text{Cierto, ya que $x\in [0,1]$.}
    \end{multline*}

    Como $f(x)\leq g(x) \qquad \forall x\in [0,1[$ y el orden se conserva en las integrales de Riemman, por el Criterio de Comparación tenemos que:
    \begin{equation*}
        \int_0^1 g(x)\;dx \text{ converge} \Longrightarrow 
        \int_0^1 f(x)\;dx \text{ converge}
    \end{equation*}

    Por tanto, tenemos que  $\displaystyle \int_0^1 \frac{dx}{\sqrt{1-x^4}}$ sí converge.


    \item $\displaystyle \int_0^2 \frac{dx}{(1+x^2)\sqrt{4-x^2}}$

    Sea $\displaystyle f(x)=\frac{1}{(1+x^2)\sqrt{4-x^2}}$. Como $\displaystyle \lim_{x\to 2}f(x)\to +\infty$, tenemos que $f(x)$ no está acotada, por lo que se trata de una integral impropia.

    Sea $g(x)=\frac{1}{\sqrt{2-x}}$. Veamos que $\int_0^2 g(x)\;dx$ converge:
    \begin{multline*}
        \int_0^2 g(x)\;dx
        = \lim_{t\to 2} \int_0^t \frac{dx}{\sqrt{2-x}}
        = \lim_{t\to 2} \red{-2}\int_0^t \frac{\red{-}dx}{\red{2}\sqrt{2-x}}
        = \lim_{t\to 2} \left[-2\sqrt{2-x}\right]_0^t 
        =\\=
        \lim_{t\to 2} -2\sqrt{2-t} +2\sqrt{2} = 2\sqrt{2}
    \end{multline*}

    Como ese límite es $L=2\sqrt{2}\in \bb{R}$, tenemos que converge. Probamos ahora que $g(x)\geq f(x) \quad \forall x\in [0,2]$:
    \begin{multline*}
        g(x)\geq f(x) \Longleftrightarrow \frac{1}{\sqrt{2-x}} \geq \frac{1}{(1+x^2)\sqrt{4-x^2}}
        \Longleftrightarrow {(1+x^2)\sqrt{4-x^2}} \geq \sqrt{2-x} \Longleftrightarrow \\ \Longleftrightarrow
        {(1+x^2)\sqrt{2+x}} \geq 1
    \end{multline*}
    donde esto es cierto, ya que al ser $x\in [0,2]$, tenemos que $\sqrt{2+x}\geq \sqrt{2}>1$ y $(1+x^2)\geq 1$. Por tanto, tenemos que $g(x)\geq f(x)\quad \forall x\in [0,2]$.\\

    Como $f(x)\leq g(x) \qquad \forall x\in [0,2[$ y el orden se conserva en las integrales de Riemman, por el Criterio de Comparación tenemos que:
    \begin{equation*}
        \int_0^2 g(x)\;dx \text{ converge} \Longrightarrow 
        \int_0^2 f(x)\;dx \text{ converge}
    \end{equation*}

    Por tanto, tenemos que  $\displaystyle \int_0^2 \frac{dx}{(1+x^2)\sqrt{4-x^2}}$ sí converge.




    \item $\displaystyle \int_2^3 \frac{dx}{\sqrt{(3-x)(x-2)}}$

    Sea $\displaystyle f(x)=\frac{dx}{\sqrt{(3-x)(x-2)}}$. Como $\displaystyle \lim_{x\to 3}f(x)\to +\infty$, tenemos que $f(x)$ no está acotada, por lo que se trata de una integral impropia.

    Sea $g(x)=\frac{1}{\sqrt{3-x}}$. Veamos que $\int_2^3 g(x)\;dx$ converge:
    \begin{multline*}
        \int_2^3 g(x)\;dx
        = \lim_{t\to 3} \int_2^t \frac{dx}{\sqrt{3-x}}
        = \lim_{t\to 3} \red{-2}\int_2^t \frac{\red{-}dx}{\red{2}\sqrt{3-x}}
        = \lim_{t\to 3} \left[-2\sqrt{3-x}\right]_2^t 
        =\\=
        \lim_{t\to 3} -2\sqrt{3-t} +2\sqrt{1} = 2
    \end{multline*}

    Como ese límite es $L=2\in \bb{R}$, tenemos que converge.

    Resolvemos el siguiente límite:
    \begin{equation*}
        \lim_{x\to 3} \frac{f(x)}{g(x)} = \lim_{x\to 3}\frac{\sqrt{3-x}}{\sqrt{(3-x)(x-2)}}
        = \lim_{x\to 3}\frac{1}{\sqrt{x-2}} = 1 = L\in \bb{R}^\ast
    \end{equation*}

    Como $L\in \bb{R}^\ast$, tenemos que, por el Criterio Límite de Comparación,
    \begin{equation*}
        \int_2^3 g(x)\;dx \text{ converge} \Longleftrightarrow 
        \int_2^3 f(x)\;dx \text{ converge}
    \end{equation*}

    Por tanto, tenemos que  $\displaystyle \int_2^3 \frac{1}{\sqrt{(3-x)(x-2)}}\;dx$ converge.



    \item $\displaystyle \int_0^{\frac{\pi}{2}} \frac{1-\cos x}{x^m} \;dx\qquad (m\in \bb{N})$.

    Sea $\displaystyle f(x)=\frac{1-\cos x}{x^m}$.
    
    
    \begin{itemize}
        \item \underline{Para $m\neq 1$}:

        Veamos en primer lugar que, por la acotación del coseno,
    \begin{equation*}
        f(x)=\frac{1-\cos x}{x^m} \leq \frac{2}{x^m} \qquad \forall x\in \left[0,\frac{\pi}{2}\right]
    \end{equation*}

    Sea $g(x)=\frac{2}{x^m}$. Veamos que $\int_0^{\frac{\pi}{2}} g(x)\;dx$ converge:
    \begin{multline*}
        \int_0^{\frac{\pi}{2}} g(x)\;dx
        = \lim_{t\to \frac{\pi}{2}} \int_0^t \frac{2}{x^m}\;dx
        = \lim_{t\to \frac{\pi}{2}} \left[2\cdot \frac{x^{-m+1}}{-m+1}\right]_0^t 
        =\\=
        \lim_{t\to \frac{\pi}{2}} 2\cdot \frac{t^{-m+1}}{-m+1} = L\in \bb{R}
    \end{multline*}

    Como ese límite es $L \in \bb{R}$, tenemos que converge.

    Por el criterio de comparación, como 
    \begin{equation*}
        \int_0^{\frac{\pi}{2}} g(x)\;dx \text{ converge} \Longleftrightarrow 
        \int_0^{\frac{\pi}{2}} \frac{1-\cos x}{x^m} \;dx \text{ converge}
    \end{equation*}

    Por tanto, tenemos que  $\displaystyle \int_0^{\frac{\pi}{2}} \frac{1-\cos x}{x^m} \;dx$ converge para $m\neq 1$.


    \underline{Para $m=1$}:

    Sea $g(x)=1$. Es fácil ver que su integral converge en $\left[0,\frac{\pi}{2}\right]$. Además,
    \begin{equation*}
        \lim_{x\to \frac{\pi}{2}} \frac{f(x)}{g(x)}
        = \lim_{x\to \frac{\pi}{2}} \frac{1-\cos x}{x} = \frac{2}{\pi} \in \bb{R}^\ast
    \end{equation*}

    Como $L\in \bb{R}^\ast$, y tenemos que la integral de $g(x)$ es convergente en dicho intervalo, por el Criterio Límite de comparación se tiene que la integral de $f(x)$ es convergente.


    Es decir, $\displaystyle \int_0^{\frac{\pi}{2}}\frac{1-\cos x}{x}\;dx$ es convergente.
    \end{itemize}

    Por tanto, hemos llegado a que dicha integral es convergente $\forall m\in \bb{N}$.
\end{enumerate}

\renewcommand{\labelenumi}{\arabic{enumi}.}
    
\end{ejercicio}