\documentclass[12pt]{article}

% Idioma y codificación
\usepackage[spanish, es-tabla]{babel}       %es-tabla para que se titule "Tabla"
\usepackage[utf8]{inputenc}

% Márgenes
\usepackage[a4paper,top=3cm,bottom=2.5cm,left=3cm,right=3cm]{geometry}

% Comentarios de bloque
\usepackage{verbatim}

% Paquetes de links
\usepackage[hidelinks]{hyperref}    % Permite enlaces
\usepackage{url}                    % redirecciona a la web

% Más opciones para enumeraciones
\usepackage{enumitem}

% Personalizar la portada
\usepackage{titling}

% Paquetes de tablas
\usepackage{multirow}


%------------------------------------------------------------------------

%Paquetes de figuras
\usepackage{caption}
\usepackage{subcaption} % Figuras al lado de otras
\usepackage{float}      % Poner figuras en el sitio indicado H.


% Paquetes de imágenes
\usepackage{graphicx}       % Paquete para añadir imágenes
\usepackage{transparent}    % Para manejar la opacidad de las figuras

% Paquete para usar colores
\usepackage[dvipsnames]{xcolor}
\usepackage{pagecolor}      % Para cambiar el color de la página

% Habilita tamaños de fuente mayores
\usepackage{fix-cm}

% Para los gráficos
\usepackage{tikz}

% Para poder situar los nodos en los grafos
\usetikzlibrary{positioning}


%------------------------------------------------------------------------

% Paquetes de matemáticas
\usepackage{mathtools, amsfonts, amssymb, mathrsfs}
\usepackage[makeroom]{cancel}     % Simplificar tachando
\usepackage{polynom}    % Divisiones y Ruffini
\usepackage{units} % Para poner fracciones diagonales con \nicefrac

\usepackage{pgfplots}   %Representar funciones
\pgfplotsset{compat=1.18}  % Versión 1.18

\usepackage{tikz-cd}    % Para usar diagramas de composiciones
\usetikzlibrary{calc}   % Para usar cálculo de coordenadas en tikz

%Definición de teoremas, etc.
\usepackage{amsthm}
%\swapnumbers   % Intercambia la posición del texto y de la numeración

\theoremstyle{plain}

\makeatletter
\@ifclassloaded{article}{
  \newtheorem{teo}{Teorema}[section]
}{
  \newtheorem{teo}{Teorema}[chapter]  % Se resetea en cada chapter
}
\makeatother

\newtheorem{coro}{Corolario}[teo]           % Se resetea en cada teorema
\newtheorem{prop}[teo]{Proposición}         % Usa el mismo contador que teorema
\newtheorem{lema}[teo]{Lema}                % Usa el mismo contador que teorema

\theoremstyle{remark}
\newtheorem*{observacion}{Observación}

\theoremstyle{definition}

\makeatletter
\@ifclassloaded{article}{
  \newtheorem{definicion}{Definición} [section]     % Se resetea en cada chapter
}{
  \newtheorem{definicion}{Definición} [chapter]     % Se resetea en cada chapter
}
\makeatother

\newtheorem*{notacion}{Notación}
\newtheorem*{ejemplo}{Ejemplo}
\newtheorem*{ejercicio*}{Ejercicio}             % No numerado
\newtheorem{ejercicio}{Ejercicio} [section]     % Se resetea en cada section


% Modificar el formato de la numeración del teorema "ejercicio"
\renewcommand{\theejercicio}{%
  \ifnum\value{section}=0 % Si no se ha iniciado ninguna sección
    \arabic{ejercicio}% Solo mostrar el número de ejercicio
  \else
    \thesection.\arabic{ejercicio}% Mostrar número de sección y número de ejercicio
  \fi
}


% \renewcommand\qedsymbol{$\blacksquare$}         % Cambiar símbolo QED
%------------------------------------------------------------------------

% Paquetes para encabezados
\usepackage{fancyhdr}
\pagestyle{fancy}
\fancyhf{}

\newcommand{\helv}{ % Modificación tamaño de letra
\fontfamily{}\fontsize{12}{12}\selectfont}
\setlength{\headheight}{15pt} % Amplía el tamaño del índice


%\usepackage{lastpage}   % Referenciar última pag   \pageref{LastPage}
\fancyfoot[C]{\thepage}

%------------------------------------------------------------------------

% Conseguir que no ponga "Capítulo 1". Sino solo "1."
\makeatletter
\@ifclassloaded{book}{
  \renewcommand{\chaptermark}[1]{\markboth{\thechapter.\ #1}{}} % En el encabezado
    
  \renewcommand{\@makechapterhead}[1]{%
  \vspace*{50\p@}%
  {\parindent \z@ \raggedright \normalfont
    \ifnum \c@secnumdepth >\m@ne
      \huge\bfseries \thechapter.\hspace{1em}\ignorespaces
    \fi
    \interlinepenalty\@M
    \Huge \bfseries #1\par\nobreak
    \vskip 40\p@
  }}
}
\makeatother

%------------------------------------------------------------------------
% Paquetes de cógido
\usepackage{minted}
\renewcommand\listingscaption{Código fuente}

\usepackage{fancyvrb}
% Personaliza el tamaño de los números de línea
\renewcommand{\theFancyVerbLine}{\small\arabic{FancyVerbLine}}

% Estilo para C++
\newminted{cpp}{
    frame=lines,
    framesep=2mm,
    baselinestretch=1.2,
    linenos,
    escapeinside=||
}

% para minted
\definecolor{LightGray}{rgb}{0.95,0.95,0.92}
\setminted{
    linenos=true,
    stepnumber=5,
    numberfirstline=true,
    autogobble,
    breaklines=true,
    breakautoindent=true,
    breaksymbolleft=,
    breaksymbolright=,
    breaksymbolindentleft=0pt,
    breaksymbolindentright=0pt,
    breaksymbolsepleft=0pt,
    breaksymbolsepright=0pt,
    fontsize=\footnotesize,
    bgcolor=LightGray,
    numbersep=10pt
}


\usepackage{listings} % Para incluir código desde un archivo

\renewcommand\lstlistingname{Código Fuente}
\renewcommand\lstlistlistingname{Índice de Códigos Fuente}

% Definir colores
\definecolor{vscodepurple}{rgb}{0.5,0,0.5}
\definecolor{vscodeblue}{rgb}{0,0,0.8}
\definecolor{vscodegreen}{rgb}{0,0.5,0}
\definecolor{vscodegray}{rgb}{0.5,0.5,0.5}
\definecolor{vscodebackground}{rgb}{0.97,0.97,0.97}
\definecolor{vscodelightgray}{rgb}{0.9,0.9,0.9}

% Configuración para el estilo de C similar a VSCode
\lstdefinestyle{vscode_C}{
  backgroundcolor=\color{vscodebackground},
  commentstyle=\color{vscodegreen},
  keywordstyle=\color{vscodeblue},
  numberstyle=\tiny\color{vscodegray},
  stringstyle=\color{vscodepurple},
  basicstyle=\scriptsize\ttfamily,
  breakatwhitespace=false,
  breaklines=true,
  captionpos=b,
  keepspaces=true,
  numbers=left,
  numbersep=5pt,
  showspaces=false,
  showstringspaces=false,
  showtabs=false,
  tabsize=2,
  frame=tb,
  framerule=0pt,
  aboveskip=10pt,
  belowskip=10pt,
  xleftmargin=10pt,
  xrightmargin=10pt,
  framexleftmargin=10pt,
  framexrightmargin=10pt,
  framesep=0pt,
  rulecolor=\color{vscodelightgray},
  backgroundcolor=\color{vscodebackground},
}

%------------------------------------------------------------------------

% Comandos definidos
\newcommand{\bb}[1]{\mathbb{#1}}
\newcommand{\cc}[1]{\mathcal{#1}}

% I prefer the slanted \leq
\let\oldleq\leq % save them in case they're every wanted
\let\oldgeq\geq
\renewcommand{\leq}{\leqslant}
\renewcommand{\geq}{\geqslant}

% Si y solo si
\newcommand{\sii}{\iff}

% Letras griegas
\newcommand{\eps}{\epsilon}
\newcommand{\veps}{\varepsilon}
\newcommand{\lm}{\lambda}

\newcommand{\ol}{\overline}
\newcommand{\ul}{\underline}
\newcommand{\wt}{\widetilde}
\newcommand{\wh}{\widehat}

\let\oldvec\vec
\renewcommand{\vec}{\overrightarrow}

% Derivadas parciales
\newcommand{\del}[2]{\frac{\partial #1}{\partial #2}}
\newcommand{\Del}[3]{\frac{\partial^{#1} #2}{\partial #3^{#1}}}
\newcommand{\deld}[2]{\dfrac{\partial #1}{\partial #2}}
\newcommand{\Deld}[3]{\dfrac{\partial^{#1} #2}{\partial #3^{#1}}}


\newcommand{\AstIg}{\stackrel{(\ast)}{=}}
\newcommand{\Hop}{\stackrel{L'H\hat{o}pital}{=}}

\newcommand{\red}[1]{{\color{red}#1}} % Para integrales, destacar los cambios.

% Método de integración
\newcommand{\MetInt}[2]{
    \left[\begin{array}{c}
        #1 \\ #2
    \end{array}\right]
}

% Declarar aplicaciones
% 1. Nombre aplicación
% 2. Dominio
% 3. Codominio
% 4. Variable
% 5. Imagen de la variable
\newcommand{\Func}[5]{
    \begin{equation*}
        \begin{array}{rrll}
            #1:& #2 & \longrightarrow & #3\\
               & #4 & \longmapsto & #5
        \end{array}
    \end{equation*}
}

%------------------------------------------------------------------------


\DeclareMathOperator{\GL}{GL}
\DeclareMathOperator{\fact}{fact}
\DeclareMathOperator{\Aut}{Aut}
\DeclareMathOperator{\Syl}{Syl}
\begin{document}

    % 1. Foto de fondo
    % 2. Título
    % 3. Encabezado Izquierdo
    % 4. Color de fondo
    % 5. Coord x del titulo
    % 6. Coord y del titulo
    % 7. Fecha

    
    % 1. Foto de fondo
% 2. Título
% 3. Encabezado Izquierdo
% 4. Color de fondo
% 5. Coord x del titulo
% 6. Coord y del titulo
% 7. Fecha

\newcommand{\portada}[7]{

    \portadaBase{#1}{#2}{#3}{#4}{#5}{#6}{#7}
    \portadaBook{#1}{#2}{#3}{#4}{#5}{#6}{#7}
}

\newcommand{\portadaExamen}[7]{

    \portadaBase{#1}{#2}{#3}{#4}{#5}{#6}{#7}
    \portadaArticle{#1}{#2}{#3}{#4}{#5}{#6}{#7}
}




\newcommand{\portadaBase}[7]{

    % Tiene la portada principal y la licencia Creative Commons
    
    % 1. Foto de fondo
    % 2. Título
    % 3. Encabezado Izquierdo
    % 4. Color de fondo
    % 5. Coord x del titulo
    % 6. Coord y del titulo
    % 7. Fecha
    
    
    \thispagestyle{empty}               % Sin encabezado ni pie de página
    \newgeometry{margin=0cm}        % Márgenes nulos para la primera página
    
    
    % Encabezado
    \fancyhead[L]{\helv #3}
    \fancyhead[R]{\helv \nouppercase{\leftmark}}
    
    
    \pagecolor{#4}        % Color de fondo para la portada
    
    \begin{figure}[p]
        \centering
        \transparent{0.3}           % Opacidad del 30% para la imagen
        
        \includegraphics[width=\paperwidth, keepaspectratio]{assets/#1}
    
        \begin{tikzpicture}[remember picture, overlay]
            \node[anchor=north west, text=white, opacity=1, font=\fontsize{60}{90}\selectfont\bfseries\sffamily, align=left] at (#5, #6) {#2};
            
            \node[anchor=south east, text=white, opacity=1, font=\fontsize{12}{18}\selectfont\sffamily, align=right] at (9.7, 3) {\textbf{\href{https://losdeldgiim.github.io/}{Los Del DGIIM}}};
            
            \node[anchor=south east, text=white, opacity=1, font=\fontsize{12}{15}\selectfont\sffamily, align=right] at (9.7, 1.8) {Doble Grado en Ingeniería Informática y Matemáticas\\Universidad de Granada};
        \end{tikzpicture}
    \end{figure}
    
    
    \restoregeometry        % Restaurar márgenes normales para las páginas subsiguientes
    \pagecolor{white}       % Restaurar el color de página
    
    
    \newpage
    \thispagestyle{empty}               % Sin encabezado ni pie de página
    \begin{tikzpicture}[remember picture, overlay]
        \node[anchor=south west, inner sep=3cm] at (current page.south west) {
            \begin{minipage}{0.5\paperwidth}
                \href{https://creativecommons.org/licenses/by-nc-nd/4.0/}{
                    \includegraphics[height=2cm]{assets/Licencia.png}
                }\vspace{1cm}\\
                Esta obra está bajo una
                \href{https://creativecommons.org/licenses/by-nc-nd/4.0/}{
                    Licencia Creative Commons Atribución-NoComercial-SinDerivadas 4.0 Internacional (CC BY-NC-ND 4.0).
                }\\
    
                Eres libre de compartir y redistribuir el contenido de esta obra en cualquier medio o formato, siempre y cuando des el crédito adecuado a los autores originales y no persigas fines comerciales. 
            \end{minipage}
        };
    \end{tikzpicture}
    
    
    
    % 1. Foto de fondo
    % 2. Título
    % 3. Encabezado Izquierdo
    % 4. Color de fondo
    % 5. Coord x del titulo
    % 6. Coord y del titulo
    % 7. Fecha


}


\newcommand{\portadaBook}[7]{

    % 1. Foto de fondo
    % 2. Título
    % 3. Encabezado Izquierdo
    % 4. Color de fondo
    % 5. Coord x del titulo
    % 6. Coord y del titulo
    % 7. Fecha

    % Personaliza el formato del título
    \pretitle{\begin{center}\bfseries\fontsize{42}{56}\selectfont}
    \posttitle{\par\end{center}\vspace{2em}}
    
    % Personaliza el formato del autor
    \preauthor{\begin{center}\Large}
    \postauthor{\par\end{center}\vfill}
    
    % Personaliza el formato de la fecha
    \predate{\begin{center}\huge}
    \postdate{\par\end{center}\vspace{2em}}
    
    \title{#2}
    \author{\href{https://losdeldgiim.github.io/}{Los Del DGIIM}}
    \date{Granada, #7}
    \maketitle
    
    \tableofcontents
}




\newcommand{\portadaArticle}[7]{

    % 1. Foto de fondo
    % 2. Título
    % 3. Encabezado Izquierdo
    % 4. Color de fondo
    % 5. Coord x del titulo
    % 6. Coord y del titulo
    % 7. Fecha

    % Personaliza el formato del título
    \pretitle{\begin{center}\bfseries\fontsize{42}{56}\selectfont}
    \posttitle{\par\end{center}\vspace{2em}}
    
    % Personaliza el formato del autor
    \preauthor{\begin{center}\Large}
    \postauthor{\par\end{center}\vspace{3em}}
    
    % Personaliza el formato de la fecha
    \predate{\begin{center}\huge}
    \postdate{\par\end{center}\vspace{5em}}
    
    \title{#2}
    \author{\href{https://losdeldgiim.github.io/}{Los Del DGIIM}}
    \date{Granada, #7}
    \thispagestyle{empty}               % Sin encabezado ni pie de página
    \maketitle
    \vfill
}
    \portadaExamen{ffccA4.jpg}{Álgebra II\\Examen VI}{Álgebra II. Examen VI}{MidnightBlue}{-8}{28}{2025}{Arturo Olivares Martos}

    \begin{description}
        \item[Asignatura] Álgebra II.
        \item[Curso Académico] 2022-23.
        \item[Grado] Doble Grado en Ingeniería Informática y Matemáticas.
        \item[Grupo] Único.
        \item[Profesor] Manuel Bullejos Lorenzo.
        \item[Descripción] Convocatoria Extraordinaria.
        %\item[Fecha] 21 de mayo del 2025.
        %\item[Duración] 2 horas.
    
    \end{description}
    \newpage

    \begin{ejercicio}
        Da la descomposición en ciclos disjuntos y en transposiciones de la permutación $\sigma = (1\ 2\ 3\ 4)(2\ 3\ 5)(1\ 2)$. Calcula su orden.
    \end{ejercicio}

    \begin{ejercicio}
        Sea $G$ un grupo no abeliano de orden $27$. Razona que su centro $Z(G)$ es un grupo cíclico.
    \end{ejercicio}

    \begin{ejercicio}
        Prueba que todo grupo de orden $18$ es un producto semidirecto.
    \end{ejercicio}

    \begin{ejercicio}
        Da las descomposiciones cíclica y cíclica primaria del grupo $\bb{Z}_{20} \oplus \bb{Z}_6$.
    \end{ejercicio}

    \begin{ejercicio}
        Sea $G$ un grupo de orden $18$. ¿Puedes asegurar que $G$ tiene un elemento de orden $3$? ¿Y de orden $9$? Razona las respuestas.
    \end{ejercicio}

    \begin{ejercicio}
        Sea $G$ un grupo de orden $18$ no abeliano y $g \in G$ un elemento de orden $9$. Razona que $G$ es isomorfo a $D_9$.
    \end{ejercicio}

    \begin{ejercicio}
        ¿Es el grupo $D_8 /Z(D_8)$ abeliano?
    \end{ejercicio}

    \begin{ejercicio}
        Demuestra si se tiene que $S_3 \cong \Aut(C_9)$ o $C_6 \cong \Aut(C_9)$. Da el isomorfismo.
    \end{ejercicio}

    \begin{ejercicio}
        Considera los grupos $C_3 = \langle a\mid a^3 = 1\rangle$, $K = \langle b, c\mid b^2 = c^2 = (bc)^2 = 1\rangle$ y la acción de $C_3$ sobre $K$ determinada por $\prescript{a}{}{b} = bc$ y $\prescript{a}{}{c} = b$. En el producto semidirecto $G = K \rtimes C_3$ calcula el producto $(b, a)^{-1}(bc, a^2)(c, a)^{-1}$.
    \end{ejercicio}

    \begin{ejercicio}
        Para el grupo $G = K \rtimes C_3$ del ejercicio anterior, calcula el conmutador $[G, G]$.
    \end{ejercicio}



    \newpage
    \setcounter{ejercicio}{0}

    \begin{ejercicio}
        Da la descomposición en ciclos disjuntos y en transposiciones de la permutación $\sigma = (1\ 2\ 3\ 4)(2\ 3\ 5)(1\ 2)$. Calcula su orden.
        \begin{equation*}
            \sigma = (1\ 4)(3\ 5)\Longrightarrow
            O(\sigma) = \text{mcm}(2,2) = 2.
        \end{equation*}
    \end{ejercicio}

    \begin{ejercicio}
        Sea $G$ un grupo no abeliano de orden $27$. Razona que su centro $Z(G)$ es un grupo cíclico.\\

        Sea $Z(G)$ el centro de $G$. Como $|G|=27=3^3$, $G$ es un $3-$grupo. Como $Z(G)<G$, sabemos que $|Z(G)|\mid 27$, luego $|Z(G)|\in \{1,3,9,27\}$.
        \begin{itemize}
            \item Si $|Z(G)|=1$, entonces $Z(G)=\{1\}$, pero $G$ es un $3-$grupo, luego llegamos a una contradicción con el Teorema de Burnside.
            \item Como $G$ es no abeliano, $Z(G)\neq G$, luego $|Z(G)|\neq 27$.
            \item Supongamos que $|Z(G)|=9$. Entonces, como $Z(G)\lhd G$, tenemos que:
            \begin{equation*}
                |G/Z(G)| = \frac{|G|}{|Z(G)|} = \frac{27}{9} = 3
            \end{equation*}
            Luego $G/Z(G)$ es cíclico, y por tanto $G$ es abeliano, lo cual contradice la hipótesis.
        \end{itemize}

        Por tanto, $|Z(G)|=3$, luego $Z(G)\cong C_3$ cíclico.
    \end{ejercicio}

    \begin{ejercicio}\label{ej:grupo-orden-18}
        Prueba que todo grupo de orden $18$ es un producto semidirecto.\\

        Sea $G$ un grupo de orden $18=2\cdot 3^2$. Sea $n_p$ el número de $p$-subgrupos de Sylow de $G$. Por el Segundo Teorema de Sylow, tenemos que:
        \begin{equation*}
            \left.\begin{array}{l}
                n_3 \equiv 1 \mod 3\\
                n_3 \mid 2
            \end{array}\right\}
            \Longrightarrow n_3=1
        \end{equation*}

        Por tanto, hay un único $3$-subgrupo de Sylow, que llamaremos $P_3$. Como $n_3=1$, $P_3$ es normal en $G$. Por el Primer Teorema de Sylow, tenemos $n_2\geq 1$, luego sea $P_2$ un $2$-subgrupo de Sylow de $G$.
        \begin{itemize}
            \item $P_3\lhd G$ y $P_2<G$.
            \item Como $|P_3|=9$ y $|P_2|=2$, razonando por órdenes tenemos que $P_3\cap P_2\{1\}$.
            \item Por el Segundo Teorema de Isomorfía, tenemos que:
            \begin{equation*}
                \dfrac{P_2}{P_2\cap P_3} \cong \dfrac{P_2P_3}{P_3}
                \Longrightarrow |P_2P_3| = |P_2|\cdot |P_3| = 2\cdot 9 = 18
            \end{equation*}
            Por tanto, $P_2P_3=G$.
        \end{itemize}

        Concluimos por tanto que $G\cong P_3\rtimes P_2$.
    \end{ejercicio}

    \begin{ejercicio}
        Da las descomposiciones cíclica y cíclica primaria del grupo $\bb{Z}_{20} \oplus \bb{Z}_6$.

        La descomposición cíclica primaria de $\bb{Z}_{20} \oplus \bb{Z}_6$ es:
        \begin{align*}
            \bb{Z}_{20} \oplus \bb{Z}_6 & \cong C_4 \oplus C_5 \oplus C_2 \oplus C_3
        \end{align*}

        La descomposición cíclica de $\bb{Z}_{20} \oplus \bb{Z}_6$ es:
        \begin{align*}
            \bb{Z}_{20} \oplus \bb{Z}_6 & \cong C_{60} \oplus C_2
        \end{align*}
    \end{ejercicio}

    \begin{ejercicio}
        Sea $G$ un grupo de orden $18$. ¿Puedes asegurar que $G$ tiene un elemento de orden $3$? ¿Y de orden $9$? Razona las respuestas.\\


        Por el Teorema de Cauchy, como $3$ es primo y $3\mid 18$, tenemos que $\exists g\in G$ tal que $O(g)=3$. Por tanto, $G$ tiene un elemento de orden $3$.\\

        Por otro lado, sea $G=\bb{Z}_3 \oplus \bb{Z}_3 \oplus \bb{Z}_2$. Supongamos que $\exists (a,b,c)\in G$ tal que $O((a,b,c))=9$. Entonces:
        \begin{equation*}
            9 = \mcd(O(a), O(b), O(c))
        \end{equation*}

        No obstante, tenemos que $O(a), O(b)\in \{1,3\}$ y $O(c)\in \{1,2\}$. Por tanto, no puede ser que $\mcd(O(a), O(b), O(c))=9$. Por tanto, para este grupo $G$ no hay elementos de orden $9$.
    \end{ejercicio}

    \begin{ejercicio}\label{ej:grupo-orden-18-isomorfo-D9}
        Sea $G$ un grupo de orden $18$ no abeliano y $g \in G$ un elemento de orden $9$. Razona que $G$ es isomorfo a $D_9$.\\

        Por el Ejercicio \ref{ej:grupo-orden-18}, sabemos que $G$ tiene un único $3$-subgrupo de Sylow. Como $O(g)=9$, entonces $|\langle G\rangle|=9$, luego $\langle g\rangle$ es el único $3$-subgrupo de Sylow de $G$. Por tanto:
        \begin{equation*}
            G\cong \langle g\rangle \rtimes P_2
        \end{equation*}

        Buscamos ahora los homomorfismos $\theta: P_2 \to \Aut(\langle g\rangle )$. Para ello, veamos en primer lugar los generadores de $\langle g\rangle $.
        \begin{equation*}
            |\langle g\rangle | = 9 \Longrightarrow \varphi(9)=2\cdot 3=6
        \end{equation*}

        Por tanto, $\langle g\rangle $ tiene $6$ generadores. Estos son:
        \begin{equation*}
            \langle g\rangle = \langle g^2\rangle = \langle g^4\rangle = \langle g^5\rangle = \langle g^7\rangle = \langle g^8\rangle
        \end{equation*}

        Por tanto, para cada $i\in \{1,2,4,5,7,8\}$, tenemos el automorfismo:
        \Func{\varphi_i}{\langle g\rangle }{\langle g\rangle }{g}{g^i}

        Ahora, hemos de ver qué automorfismos son válidos. Como $P_2$ es cíclico de orden $2$, sea $P_2=\langle b\mid b^2=1\rangle$. Como $O(b)=2$, entonces $O(\theta(b))\mid 2$, luego $O(\theta(b))\in \{1,2\}$.
        \begin{itemize}
            \item Si $O(\theta(b))=1$, entonces $\theta(b)=Id$, luego $G\cong \langle g\rangle \times P_2$ es abeliano, lo cual contradice la hipótesis de que $G$ es no abeliano.
        \end{itemize}

        Por tanto, $O(\theta(b))=2$. Veamos ahora qué automorfismos son de orden $2$:
        \begin{align*}
            (\varphi_i \circ \varphi_i)(g) & = \varphi_i(\varphi_i(g)) = \varphi_i(g^i) = g^{i^2} = g\iff i^2 \equiv 1 \mod 9\\
        \end{align*}

        Por tanto, tenemos que $i\in \{1,8\}$. Pero como $O(\theta(b))\neq 1$, tenemos que $i\neq 1$. Por tanto, $i=8$. Por tanto, el único automorfismo válido es $\varphi_8$, luego:
        \begin{equation*}
            bgb^{-1} = \varphi_8(g) = g^8 = g^{-1}\Longrightarrow bg = g^{-1}b
        \end{equation*}

        Por el Teorema de Dyck, como $b^2=1$, $g^9=1$ y $bg=g^{-1}b$, tenenemos que $f:D_9 \to G$ dada por:
        \begin{equation*}
            f(s) = b\qquad f(r)= g
        \end{equation*}
        es un homomorfismo. Como $b,g$ son generadores de $P_2$ y $\langle g\rangle $ respectivamente, tenemos que $G=\langle g,b\rangle $, por lo que es sobreyectivo. Además, como $|G|=18=|D_9|$, tenemos que $f$ es un isomorfismo. Por tanto, $G\cong D_9$.
    \end{ejercicio}

    \begin{ejercicio}
        ¿Es el grupo $D_8 /Z(D_8)$ abeliano?\\

        Por un ejercicio de las Relaciones, sabemos que $Z(D_8) = \{1, r^4\}$. Veamos que $D_8 /Z(D_8)$ no es abeliano. Para ello, vemos que:
        \begin{align*}
            srZ(D_8) & = \{sr, sr^5\}\\
            rsZ(D_8) & = \{rs, rsr^4\} = \{sr^7, sr^7r^4\} = \{sr^7, sr^3\}
        \end{align*}

        Como $srZ(D_8)\neq rsZ(D_8)$, tenemos que $D_8 /Z(D_8)$ no es abeliano.
    \end{ejercicio}

    \begin{ejercicio}
        Demuestra si se tiene que $S_3 \cong \Aut(C_9)$ o $C_6 \cong \Aut(C_9)$. Da el isomorfismo.\\

        Sea $C_9=\langle g\mid g^9=1\rangle$. En el Ejercicio~\ref{ej:grupo-orden-18-isomorfo-D9} vimos los $6$ elementos de $\Aut(C_9)$. Consideramos $\varphi_2\in \Aut(C_9)$, que es el automorfismo dado por:
        \Func{\varphi_2}{C_9}{C_9}{g}{g^2}

        Aplicando $\varphi_2$, tenemos que:
        \begin{equation*}
            g \mapsto g^2 \mapsto g^4 \mapsto g^8 \mapsto g^7 \mapsto g^5 \mapsto g
        \end{equation*}
        Por tanto, $O(\varphi_2)=6$, y como el orden se mantiene por isomorfismos y $S_3$ no tiene elementos de orden $6$, tenemos que $\Aut(C_9)\not\cong S_3$. Por tanto, hemos de ver si $\Aut(C_9)\cong C_6$. El isomorfismo es:
        \Func{\varphi}{C_6}{\Aut(C_9)}{g}{\varphi_2}

        Como $\varphi_2^2=Id$, por el Teorema de Dyck tenemos que $\varphi$ es un homomorfismo. Además, como $\varphi_2$ es un generador de $\Aut(C_9)$, tenemos que $\varphi$ es sobreyectivo. Finalmente, como $|C_6|=6=|\Aut(C_9)|$, tenemos que $\varphi$ es un isomorfismo. Por tanto, $\Aut(C_9)\cong C_6$.
    \end{ejercicio}

    \begin{ejercicio}
        Considera los grupos $C_3 = \langle a\mid a^3 = 1\rangle$, $K = \langle b, c\mid b^2 = c^2 = (bc)^2 = 1\rangle$ y la acción de $C_3$ sobre $K$ determinada por $\prescript{a}{}{b} = bc$ y $\prescript{a}{}{c} = b$. En el producto semidirecto $G = K \rtimes C_3$ calcula el producto $(b, a)^{-1}(bc, a^2)(c, a)^{-1}$.\\

        Calculamos cada parte por separado:
        \begin{align*}
            (b, a)^{-1} & = \left(\prescript{a^{-1}}{}{b^{-1}}, a^{-1}\right) = \left(\prescript{a^2}{}{b}, a^2\right) = \left(\prescript{a}{}{bc}, a^2\right)
            = \left(\prescript{a}{}{b}\prescript{a}{}{c}, a^2\right) =\\&= (bcb, a^2)
            = (c^{-1}, a^2) = (c, a^2)\\
            (c, a)^{-1} & = \left(\prescript{a^{-1}}{}{c^{-1}}, a^{-1}\right) = \left(\prescript{a^2}{}{c}, a^2\right) = \left(\prescript{a}{}{b}, a^2\right)
            = (bc, a^2)\\
            (bc, a^2)(bc,a^2) & = \left(bc\prescript{a^2}{}{bc}, a^2\cdot a^2\right) = \left(bc\prescript{a^2}{}{b}\prescript{a^2}{}{c}, a\right)
            = \left(bcc\prescript{a}{}{b}, a\right) = \left(bbc, a\right) = (c,a)\\
            (b, a)^{-1}(bc, a^2)(c, a)^{-1} & = (c, a^2)(c, a) = \left(c\prescript{a^2}{}{c}, a^2\cdot a\right) = \left(c\prescript{a}{}{b}, 1\right)
            = (cbc, 1) = (b, 1)
        \end{align*}
    \end{ejercicio}

    \begin{ejercicio}
        Para el grupo $G = K \rtimes C_3$ del ejercicio anterior, calcula el conmutador $[G, G]$.\\

        Sabemos que $G' = [G, G]$ es el menor subgrupo normal de $G$ tal que $G/G'$ es abeliano. Por definición de producto semidirecto, sabemos que $K \lhd G$ y $|G/K| = 3$, luego $G/K$ es abeliano. Por tanto, sabemos que:
        \begin{equation*}
            [G, G] \leq K
        \end{equation*}

        Calculemos $[(b, a), (b, 1)]$:
        \begin{align*}
            (b, a)(b, 1)(b, a)^{-1} (b, 1)^{-1} & = (c, a)(c, a^2)(b, 1)\\
            & = (bc, 1)(b, 1) = (c, 1)
        \end{align*}

        Por lo que concluimos que, en primer lugar, $G$ no es abeliano (es decir, $[G, G] \neq 1$) y en segundo lugar que $\langle c \rangle \leq [G, G]$.
        Comprobamos que $\langle c \rangle$ no es normal en $G$:
        \begin{align*}
            (1, a)(c, 1)(1, a)^{-1} & = (b, a)(1, a^2) = (b, 1) \not\in \langle c \rangle
        \end{align*}
        Por tanto, $[G, G] \neq \langle c \rangle$, y como $\langle c \rangle \leq [G, G] \leq K$ y $|K/\langle c \rangle| = 2$ que es primo (es decir, no existe ningún otro grupo normal distinto que contenga a $\langle c \rangle$ y esté contenido en $K$), se tiene que $[G, G] = K$.
    \end{ejercicio}
\end{document}
