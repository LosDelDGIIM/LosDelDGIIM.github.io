\documentclass[12pt]{article}

% Idioma y codificación
\usepackage[spanish, es-tabla, es-notilde]{babel}       %es-tabla para que se titule "Tabla"
\usepackage[utf8]{inputenc}

% Márgenes
\usepackage[a4paper,top=3cm,bottom=2.5cm,left=3cm,right=3cm]{geometry}

% Comentarios de bloque
\usepackage{verbatim}

% Paquetes de links
\usepackage[hidelinks]{hyperref}    % Permite enlaces
\usepackage{url}                    % redirecciona a la web

% Más opciones para enumeraciones
\usepackage{enumitem}

% Personalizar la portada
\usepackage{titling}

% Paquetes de tablas
\usepackage{multirow}

% Para añadir el símbolo de euro
\usepackage{eurosym}


%------------------------------------------------------------------------

%Paquetes de figuras
\usepackage{caption}
\usepackage{subcaption} % Figuras al lado de otras
\usepackage{float}      % Poner figuras en el sitio indicado H.


% Paquetes de imágenes
\usepackage{graphicx}       % Paquete para añadir imágenes
\usepackage{transparent}    % Para manejar la opacidad de las figuras

% Paquete para usar colores
\usepackage[dvipsnames, table, xcdraw]{xcolor}
\usepackage{pagecolor}      % Para cambiar el color de la página

% Habilita tamaños de fuente mayores
\usepackage{fix-cm}

% Para los gráficos
\usepackage{tikz}
\usepackage{forest}

% Para poder situar los nodos en los grafos
\usetikzlibrary{positioning}


%------------------------------------------------------------------------

% Paquetes de matemáticas
\usepackage{mathtools, amsfonts, amssymb, mathrsfs}
\usepackage[makeroom]{cancel}     % Simplificar tachando
\usepackage{polynom}    % Divisiones y Ruffini
\usepackage{units} % Para poner fracciones diagonales con \nicefrac

\usepackage{pgfplots}   %Representar funciones
\pgfplotsset{compat=1.18}  % Versión 1.18

\usepackage{tikz-cd}    % Para usar diagramas de composiciones
\usetikzlibrary{calc}   % Para usar cálculo de coordenadas en tikz

%Definición de teoremas, etc.
\usepackage{amsthm}
%\swapnumbers   % Intercambia la posición del texto y de la numeración

\theoremstyle{plain}

\makeatletter
\@ifclassloaded{article}{
  \newtheorem{teo}{Teorema}[section]
}{
  \newtheorem{teo}{Teorema}[chapter]  % Se resetea en cada chapter
}
\makeatother

\newtheorem{coro}{Corolario}[teo]           % Se resetea en cada teorema
\newtheorem{prop}[teo]{Proposición}         % Usa el mismo contador que teorema
\newtheorem{lema}[teo]{Lema}                % Usa el mismo contador que teorema
\newtheorem*{lema*}{Lema}

\theoremstyle{remark}
\newtheorem*{observacion}{Observación}

\theoremstyle{definition}

\makeatletter
\@ifclassloaded{article}{
  \newtheorem{definicion}{Definición} [section]     % Se resetea en cada chapter
}{
  \newtheorem{definicion}{Definición} [chapter]     % Se resetea en cada chapter
}
\makeatother

\newtheorem*{notacion}{Notación}
\newtheorem*{ejemplo}{Ejemplo}
\newtheorem*{ejercicio*}{Ejercicio}             % No numerado
\newtheorem{ejercicio}{Ejercicio} [section]     % Se resetea en cada section


% Modificar el formato de la numeración del teorema "ejercicio"
\renewcommand{\theejercicio}{%
  \ifnum\value{section}=0 % Si no se ha iniciado ninguna sección
    \arabic{ejercicio}% Solo mostrar el número de ejercicio
  \else
    \thesection.\arabic{ejercicio}% Mostrar número de sección y número de ejercicio
  \fi
}


% \renewcommand\qedsymbol{$\blacksquare$}         % Cambiar símbolo QED
%------------------------------------------------------------------------

% Paquetes para encabezados
\usepackage{fancyhdr}
\pagestyle{fancy}
\fancyhf{}

\newcommand{\helv}{ % Modificación tamaño de letra
\fontfamily{}\fontsize{12}{12}\selectfont}
\setlength{\headheight}{15pt} % Amplía el tamaño del índice


%\usepackage{lastpage}   % Referenciar última pag   \pageref{LastPage}
%\fancyfoot[C]{%
%  \begin{minipage}{\textwidth}
%    \centering
%    ~\\
%    \thepage\\
%    \href{https://losdeldgiim.github.io/}{\texttt{\footnotesize losdeldgiim.github.io}}
%  \end{minipage}
%}
\fancyfoot[C]{\thepage}
\fancyfoot[R]{\href{https://losdeldgiim.github.io/}{\texttt{\footnotesize losdeldgiim.github.io}}}

%------------------------------------------------------------------------

% Conseguir que no ponga "Capítulo 1". Sino solo "1."
\makeatletter
\@ifclassloaded{book}{
  \renewcommand{\chaptermark}[1]{\markboth{\thechapter.\ #1}{}} % En el encabezado
    
  \renewcommand{\@makechapterhead}[1]{%
  \vspace*{50\p@}%
  {\parindent \z@ \raggedright \normalfont
    \ifnum \c@secnumdepth >\m@ne
      \huge\bfseries \thechapter.\hspace{1em}\ignorespaces
    \fi
    \interlinepenalty\@M
    \Huge \bfseries #1\par\nobreak
    \vskip 40\p@
  }}
}
\makeatother

%------------------------------------------------------------------------
% Paquetes de cógido
\usepackage{minted}
\renewcommand\listingscaption{Código fuente}

\usepackage{fancyvrb}
% Personaliza el tamaño de los números de línea
\renewcommand{\theFancyVerbLine}{\small\arabic{FancyVerbLine}}

% Estilo para C++
\newminted{cpp}{
    frame=lines,
    framesep=2mm,
    baselinestretch=1.2,
    linenos,
    escapeinside=||
}

% para minted
\definecolor{LightGray}{rgb}{0.95,0.95,0.92}
\setminted{
    linenos=true,
    stepnumber=5,
    numberfirstline=true,
    autogobble,
    breaklines=true,
    breakautoindent=true,
    breaksymbolleft=,
    breaksymbolright=,
    breaksymbolindentleft=0pt,
    breaksymbolindentright=0pt,
    breaksymbolsepleft=0pt,
    breaksymbolsepright=0pt,
    fontsize=\footnotesize,
    bgcolor=LightGray,
    numbersep=10pt
}


\usepackage{listings} % Para incluir código desde un archivo

\renewcommand\lstlistingname{Código Fuente}
\renewcommand\lstlistlistingname{Índice de Códigos Fuente}

% Definir colores
\definecolor{vscodepurple}{rgb}{0.5,0,0.5}
\definecolor{vscodeblue}{rgb}{0,0,0.8}
\definecolor{vscodegreen}{rgb}{0,0.5,0}
\definecolor{vscodegray}{rgb}{0.5,0.5,0.5}
\definecolor{vscodebackground}{rgb}{0.97,0.97,0.97}
\definecolor{vscodelightgray}{rgb}{0.9,0.9,0.9}

% Configuración para el estilo de C similar a VSCode
\lstdefinestyle{vscode_C}{
  backgroundcolor=\color{vscodebackground},
  commentstyle=\color{vscodegreen},
  keywordstyle=\color{vscodeblue},
  numberstyle=\tiny\color{vscodegray},
  stringstyle=\color{vscodepurple},
  basicstyle=\scriptsize\ttfamily,
  breakatwhitespace=false,
  breaklines=true,
  captionpos=b,
  keepspaces=true,
  numbers=left,
  numbersep=5pt,
  showspaces=false,
  showstringspaces=false,
  showtabs=false,
  tabsize=2,
  frame=tb,
  framerule=0pt,
  aboveskip=10pt,
  belowskip=10pt,
  xleftmargin=10pt,
  xrightmargin=10pt,
  framexleftmargin=10pt,
  framexrightmargin=10pt,
  framesep=0pt,
  rulecolor=\color{vscodelightgray},
  backgroundcolor=\color{vscodebackground},
}

%------------------------------------------------------------------------

% Comandos definidos
\newcommand{\bb}[1]{\mathbb{#1}}
\newcommand{\cc}[1]{\mathcal{#1}}

% I prefer the slanted \leq
\let\oldleq\leq % save them in case they're every wanted
\let\oldgeq\geq
\renewcommand{\leq}{\leqslant}
\renewcommand{\geq}{\geqslant}

% Si y solo si
\newcommand{\sii}{\iff}

% MCD y MCM
\DeclareMathOperator{\mcd}{mcd}
\DeclareMathOperator{\mcm}{mcm}

% Signo
\DeclareMathOperator{\sgn}{sgn}

% Letras griegas
\newcommand{\eps}{\epsilon}
\newcommand{\veps}{\varepsilon}
\newcommand{\lm}{\lambda}

\newcommand{\ol}{\overline}
\newcommand{\ul}{\underline}
\newcommand{\wt}{\widetilde}
\newcommand{\wh}{\widehat}

\let\oldvec\vec
\renewcommand{\vec}{\overrightarrow}

% Derivadas parciales
\newcommand{\del}[2]{\frac{\partial #1}{\partial #2}}
\newcommand{\Del}[3]{\frac{\partial^{#1} #2}{\partial #3^{#1}}}
\newcommand{\deld}[2]{\dfrac{\partial #1}{\partial #2}}
\newcommand{\Deld}[3]{\dfrac{\partial^{#1} #2}{\partial #3^{#1}}}


\newcommand{\AstIg}{\stackrel{(\ast)}{=}}
\newcommand{\Hop}{\stackrel{L'H\hat{o}pital}{=}}

\newcommand{\red}[1]{{\color{red}#1}} % Para integrales, destacar los cambios.

% Método de integración
\newcommand{\MetInt}[2]{
    \left[\begin{array}{c}
        #1 \\ #2
    \end{array}\right]
}

% Declarar aplicaciones
% 1. Nombre aplicación
% 2. Dominio
% 3. Codominio
% 4. Variable
% 5. Imagen de la variable
\newcommand{\Func}[5]{
    \begin{equation*}
        \begin{array}{rrll}
            \displaystyle #1:& \displaystyle  #2 & \longrightarrow & \displaystyle  #3\\
               & \displaystyle  #4 & \longmapsto & \displaystyle  #5
        \end{array}
    \end{equation*}
}

%------------------------------------------------------------------------


\DeclareMathOperator{\GL}{GL}
\DeclareMathOperator{\fact}{fact}
\DeclareMathOperator{\Aut}{Aut}
\DeclareMathOperator{\Syl}{Syl}
\begin{document}

    % 1. Foto de fondo
    % 2. Título
    % 3. Encabezado Izquierdo
    % 4. Color de fondo
    % 5. Coord x del titulo
    % 6. Coord y del titulo
    % 7. Fecha

    
    % 1. Foto de fondo
% 2. Título
% 3. Encabezado Izquierdo
% 4. Color de fondo
% 5. Coord x del titulo
% 6. Coord y del titulo
% 7. Fecha
% 8. Autor

\newcommand{\portada}[8]{
    \portadaBase{#1}{#2}{#3}{#4}{#5}{#6}{#7}{#8}
    \portadaBook{#1}{#2}{#3}{#4}{#5}{#6}{#7}{#8}
}

\newcommand{\portadaFotoDif}[8]{
    \portadaBaseFotoDif{#1}{#2}{#3}{#4}{#5}{#6}{#7}{#8}
    \portadaBook{#1}{#2}{#3}{#4}{#5}{#6}{#7}{#8}
}

\newcommand{\portadaExamen}[8]{
    \portadaBase{#1}{#2}{#3}{#4}{#5}{#6}{#7}{#8}
    \portadaArticle{#1}{#2}{#3}{#4}{#5}{#6}{#7}{#8}
}

\newcommand{\portadaExamenFotoDif}[8]{
    \portadaBaseFotoDif{#1}{#2}{#3}{#4}{#5}{#6}{#7}{#8}
    \portadaArticle{#1}{#2}{#3}{#4}{#5}{#6}{#7}{#8}
}




\newcommand{\portadaBase}[8]{

    % Tiene la portada principal y la licencia Creative Commons
    
    % 1. Foto de fondo
    % 2. Título
    % 3. Encabezado Izquierdo
    % 4. Color de fondo
    % 5. Coord x del titulo
    % 6. Coord y del titulo
    % 7. Fecha
    % 8. Autor    
    
    \thispagestyle{empty}               % Sin encabezado ni pie de página
    \newgeometry{margin=0cm}        % Márgenes nulos para la primera página
    
    
    % Encabezado
    \fancyhead[L]{\helv #3}
    \fancyhead[R]{\helv \nouppercase{\leftmark}}
    
    
    \pagecolor{#4}        % Color de fondo para la portada
    
    \begin{figure}[p]
        \centering
        \transparent{0.3}           % Opacidad del 30% para la imagen
        
        \includegraphics[width=\paperwidth, keepaspectratio]{../../_assets/#1}
    
        \begin{tikzpicture}[remember picture, overlay]
            \node[anchor=north west, text=white, opacity=1, font=\fontsize{60}{90}\selectfont\bfseries\sffamily, align=left] at (#5, #6) {#2};
            
            \node[anchor=south east, text=white, opacity=1, font=\fontsize{12}{18}\selectfont\sffamily, align=right] at (9.7, 3) {\href{https://losdeldgiim.github.io/}{\textbf{Los Del DGIIM}, \texttt{\footnotesize losdeldgiim.github.io}}};
            
            \node[anchor=south east, text=white, opacity=1, font=\fontsize{12}{15}\selectfont\sffamily, align=right] at (9.7, 1.8) {Doble Grado en Ingeniería Informática y Matemáticas\\Universidad de Granada};
        \end{tikzpicture}
    \end{figure}
    
    
    \restoregeometry        % Restaurar márgenes normales para las páginas subsiguientes
    \nopagecolor      % Restaurar el color de página
    
    
    \newpage
    \thispagestyle{empty}               % Sin encabezado ni pie de página
    \begin{tikzpicture}[remember picture, overlay]
        \node[anchor=south west, inner sep=3cm] at (current page.south west) {
            \begin{minipage}{0.5\paperwidth}
                \href{https://creativecommons.org/licenses/by-nc-nd/4.0/}{
                    \includegraphics[height=2cm]{../../_assets/Licencia.png}
                }\vspace{1cm}\\
                Esta obra está bajo una
                \href{https://creativecommons.org/licenses/by-nc-nd/4.0/}{
                    Licencia Creative Commons Atribución-NoComercial-SinDerivadas 4.0 Internacional (CC BY-NC-ND 4.0).
                }\\
    
                Eres libre de compartir y redistribuir el contenido de esta obra en cualquier medio o formato, siempre y cuando des el crédito adecuado a los autores originales y no persigas fines comerciales. 
            \end{minipage}
        };
    \end{tikzpicture}
    
    
    
    % 1. Foto de fondo
    % 2. Título
    % 3. Encabezado Izquierdo
    % 4. Color de fondo
    % 5. Coord x del titulo
    % 6. Coord y del titulo
    % 7. Fecha
    % 8. Autor


}


\newcommand{\portadaBaseFotoDif}[8]{

    % Tiene la portada principal y la licencia Creative Commons
    
    % 1. Foto de fondo
    % 2. Título
    % 3. Encabezado Izquierdo
    % 4. Color de fondo
    % 5. Coord x del titulo
    % 6. Coord y del titulo
    % 7. Fecha
    % 8. Autor    
    
    \thispagestyle{empty}               % Sin encabezado ni pie de página
    \newgeometry{margin=0cm}        % Márgenes nulos para la primera página
    
    
    % Encabezado
    \fancyhead[L]{\helv #3}
    \fancyhead[R]{\helv \nouppercase{\leftmark}}
    
    
    \pagecolor{#4}        % Color de fondo para la portada
    
    \begin{figure}[p]
        \centering
        \transparent{0.3}           % Opacidad del 30% para la imagen
        
        \includegraphics[width=\paperwidth, keepaspectratio]{#1}
    
        \begin{tikzpicture}[remember picture, overlay]
            \node[anchor=north west, text=white, opacity=1, font=\fontsize{60}{90}\selectfont\bfseries\sffamily, align=left] at (#5, #6) {#2};
            
            \node[anchor=south east, text=white, opacity=1, font=\fontsize{12}{18}\selectfont\sffamily, align=right] at (9.7, 3) {\href{https://losdeldgiim.github.io/}{\textbf{Los Del DGIIM}, \texttt{\footnotesize losdeldgiim.github.io}}};
            
            \node[anchor=south east, text=white, opacity=1, font=\fontsize{12}{15}\selectfont\sffamily, align=right] at (9.7, 1.8) {Doble Grado en Ingeniería Informática y Matemáticas\\Universidad de Granada};
        \end{tikzpicture}
    \end{figure}
    
    
    \restoregeometry        % Restaurar márgenes normales para las páginas subsiguientes
    \nopagecolor      % Restaurar el color de página
    
    
    \newpage
    \thispagestyle{empty}               % Sin encabezado ni pie de página
    \begin{tikzpicture}[remember picture, overlay]
        \node[anchor=south west, inner sep=3cm] at (current page.south west) {
            \begin{minipage}{0.5\paperwidth}
                %\href{https://creativecommons.org/licenses/by-nc-nd/4.0/}{
                %    \includegraphics[height=2cm]{../../_assets/Licencia.png}
                %}\vspace{1cm}\\
                Esta obra está bajo una
                \href{https://creativecommons.org/licenses/by-nc-nd/4.0/}{
                    Licencia Creative Commons Atribución-NoComercial-SinDerivadas 4.0 Internacional (CC BY-NC-ND 4.0).
                }\\
    
                Eres libre de compartir y redistribuir el contenido de esta obra en cualquier medio o formato, siempre y cuando des el crédito adecuado a los autores originales y no persigas fines comerciales. 
            \end{minipage}
        };
    \end{tikzpicture}
    
    
    
    % 1. Foto de fondo
    % 2. Título
    % 3. Encabezado Izquierdo
    % 4. Color de fondo
    % 5. Coord x del titulo
    % 6. Coord y del titulo
    % 7. Fecha
    % 8. Autor


}


\newcommand{\portadaBook}[8]{

    % 1. Foto de fondo
    % 2. Título
    % 3. Encabezado Izquierdo
    % 4. Color de fondo
    % 5. Coord x del titulo
    % 6. Coord y del titulo
    % 7. Fecha
    % 8. Autor

    % Personaliza el formato del título
    \pretitle{\begin{center}\bfseries\fontsize{42}{56}\selectfont}
    \posttitle{\par\end{center}\vspace{2em}}
    
    % Personaliza el formato del autor
    \preauthor{\begin{center}\Large}
    \postauthor{\par\end{center}\vfill}
    
    % Personaliza el formato de la fecha
    \predate{\begin{center}\huge}
    \postdate{\par\end{center}\vspace{2em}}
    
    \title{#2}
    \author{\href{https://losdeldgiim.github.io/}{Los Del DGIIM, \texttt{\large losdeldgiim.github.io}}
    \\ \vspace{0.5cm}#8}
    \date{Granada, #7}
    \maketitle
    
    \tableofcontents
}




\newcommand{\portadaArticle}[8]{

    % 1. Foto de fondo
    % 2. Título
    % 3. Encabezado Izquierdo
    % 4. Color de fondo
    % 5. Coord x del titulo
    % 6. Coord y del titulo
    % 7. Fecha
    % 8. Autor

    % Personaliza el formato del título
    \pretitle{\begin{center}\bfseries\fontsize{42}{56}\selectfont}
    \posttitle{\par\end{center}\vspace{2em}}
    
    % Personaliza el formato del autor
    \preauthor{\begin{center}\Large}
    \postauthor{\par\end{center}\vspace{3em}}
    
    % Personaliza el formato de la fecha
    \predate{\begin{center}\huge}
    \postdate{\par\end{center}\vspace{5em}}
    
    \title{#2}
    \author{\href{https://losdeldgiim.github.io/}{Los Del DGIIM, \texttt{\large losdeldgiim.github.io}}
    \\ \vspace{0.5cm}#8}
    \date{Granada, #7}
    \thispagestyle{empty}               % Sin encabezado ni pie de página
    \maketitle
    \vfill
}
    \portadaExamen{ffccA4.jpg}{Álgebra II\\Examen IV}{Álgebra II. Examen IV}{MidnightBlue}{-8}{28}{2025}{Arturo Olivares Martos}

    \begin{description}
        \item[Asignatura] Álgebra II.
        \item[Curso Académico] 2023-24.
        \item[Grado] Doble Grado en Ingeniería Informática y Matemáticas.
        \item[Grupo] Único.
        \item[Profesor] Manuel Bullejos Lorenzo.
        \item[Descripción] Convocatoria Ordinaria.
        %\item[Fecha] 21 de mayo del 2025.
        %\item[Duración] 2 horas.
    
    \end{description}
    \newpage

    \begin{ejercicio}~
        \begin{enumerate}
            \item (1 punto) Dado el grupo abeliano siguiente, calcula sus descomposiciones cíclica y cíclica primaria, el orden de $A$ y el rango de su parte libre:
                \[
                    A = \left\langle x, y, z \left|\begin{array}{rcl}
                        6x - 4y + 4z & = & 0 \\
                        8x + 4y + 6z & = & 0 \\
                        6x + 4y + 4z & = & 0
                    \end{array}\right.\right\rangle
                \]
            \item (1 punto) Escribe las descomposiciones cíclicas y cíclicas primarias de todos los grupos abelianos de orden $108$.
            \item (1 punto) Calcula la descomposición cíclica y cíclica primaria del grupo abeliano $\Aut(C_{16})$.
        \end{enumerate}
    \end{ejercicio}

    \begin{ejercicio}~
        \begin{enumerate}
            \item ($0.5$ puntos) Sea $\alpha=(2\ 3\ 4)(1\ 2\ 3) \in S_5$. Calcula $\alpha^{123}$.
            \item ($1.5$ puntos) Calcula el número de $3$-subgrupos de Sylow de $S_5$.
        \end{enumerate}
    \end{ejercicio}

    \begin{ejercicio}~
        \begin{enumerate}
            \item (2 puntos) Demuestra que hay un único grupo de orden $885$ que además es abeliano.
            \item ($1.5$ puntos) Demuestra que todo grupo de orden $351$ es un producto semidirecto.
            \item ($1.5$ puntos) Calcula todos los productos semidirectos $C_{13} \rtimes C_{27}$. ¿Cuántos hay salvo isomorfismo?
        \end{enumerate}
    \end{ejercicio}



    \newpage
    \setcounter{ejercicio}{0}

    \begin{ejercicio}~
        \begin{enumerate}
            \item (1 punto) Dado el grupo abeliano siguiente, calcula sus descomposiciones cíclica y cíclica primaria, el orden de $A$ y el rango de su parte libre:
                \[
                    A = \left\langle x, y, z \left|\begin{array}{rcl}
                        6x - 4y + 4z & = & 0 \\
                        8x + 4y + 6z & = & 0 \\
                        6x + 4y + 4z & = & 0
                    \end{array}\right.\right\rangle
                \]

                Consideramos su matriz de relaciones:
                \[
                    M = \begin{pmatrix}
                        6 & -4 & 4 \\
                        8 & 4 & 6 \\
                        6 & 4 & 4
                    \end{pmatrix}
                \]

                Calculamos su forma normal de Smith:
                \begin{multline*}
                    M = \begin{pmatrix}
                        6 & -4 & 4 \\
                        8 & 4 & 6 \\
                        6 & 4 & 4
                    \end{pmatrix}
                    \xrightarrow{C_1'=C_1-C_3}
                    \begin{pmatrix}
                        2 & -4 & 4 \\
                        2 & 4 & 6 \\
                        2 & 4 & 4
                    \end{pmatrix}
                    \xrightarrow[F_3'=F_3-F_1]{F_2'=F_2-F_1}
                    \begin{pmatrix}
                        2 & -4 & 4 \\
                        0 & 8 & 2 \\
                        0 & 8 & 0
                    \end{pmatrix}
                    \xrightarrow[C_2'=C_2+2C_1]{C_3'=C_3-2C_1}\\
                    \begin{pmatrix}
                        2 & 0 & 0 \\
                        0 & 8 & 2 \\
                        0 & 8 & 0
                    \end{pmatrix}
                    \xrightarrow{C_2\leftrightarrow C_3}
                    \begin{pmatrix}
                        2 & 0 & 0 \\
                        0 & 2 & 8 \\
                        0 & 0 & 8
                    \end{pmatrix}
                    \xrightarrow{C_3'=C_3-4C_2}
                    \begin{pmatrix}
                        2 & 0 & 0 \\
                        0 & 2 & 0 \\
                        0 & 0 & 8
                    \end{pmatrix}
                \end{multline*}

                Por tanto, la forma normal de Smith de $M$ es
                \[
                    \begin{pmatrix}
                        2 & 0 & 0 \\
                        0 & 2 & 0 \\
                        0 & 0 & 8
                    \end{pmatrix}
                \]

                Por tanto, tanto su descomposición cíclica como su descomposición cíclica primaria son:
                \begin{equation*}
                    A \cong C_2 \oplus C_2 \oplus C_8
                \end{equation*}

                Como vemos, el orden de $A$ es $32$ y su parte libre tiene rango $0$.
            \item (1 punto) Escribe las descomposiciones cíclicas y cíclicas primarias de todos los grupos abelianos de orden $108$.
            
            Mostrado en la Tabla~\ref{tab:grupos_abelianos_108}.
        \begin{table}[h]
            \centering
            \begin{tabular}{c|c|c|c|c}
                & \textbf{Fact. Inv.} & \textbf{Div. element.} & \textbf{Desc. cíclica primaria} & \textbf{Desc. cíclica} \\
                \hline
                $\begin{pmatrix}
                    2^2\\
                    3^3
                \end{pmatrix}
                $ & $d_1=108$ & $\{2^2; 3^3\}$ & $C_4 \oplus C_{27}$ & $C_{108}$ \\ \hline
                $\begin{pmatrix}
                    2 & 2\\
                    3^3 & 1
                \end{pmatrix}
                $ & $\begin{array}{l}
                    d_1=54 \\
                    d_2=2
                \end{array}$ & $\{2; 2; 3^3\}$ & $C_2 \oplus C_2 \oplus C_{27}$ & $C_{54} \oplus C_2$ \\ \hline
                $\begin{pmatrix}
                    2^2 & 1\\
                    3^2 & 3
                \end{pmatrix}
                $ & $\begin{array}{l}
                    d_1=36 \\
                    d_2=3
                \end{array}$ & $\{2^2; 3^2; 3\}$ & $C_4 \oplus C_9 \oplus C_3$ & $C_{36} \oplus C_3$ \\ \hline
                $\begin{pmatrix}
                    2 & 2\\
                    3^2 & 3
                \end{pmatrix}
                $ & $\begin{array}{l}
                    d_1=18 \\
                    d_2=6
                \end{array}$ & $\{2; 2; 3^2; 3\}$ & $C_2 \oplus C_2 \oplus C_9 \oplus C_3$ & $C_{18} \oplus C_6$ \\ \hline
                $\begin{pmatrix}
                    2^2 & 1 & 1\\
                    3 & 3 & 3
                \end{pmatrix}
                $ & $\begin{array}{l}
                    d_1=12 \\
                    d_2=3 \\
                    d_3=3
                \end{array}$ & $\{2^2; 3; 3; 3\}$ & $C_4 \oplus C_3 \oplus C_3 \oplus C_3$ & $C_{12} \oplus C_3 \oplus C_3$ \\ \hline
                $\begin{pmatrix}
                    2 & 2 & 1\\
                    3 & 3 & 3
                \end{pmatrix}
                $ & $\begin{array}{l}
                    d_1=6 \\
                    d_2=6 \\
                    d_3=3
                \end{array}$ & $\{2; 2; 3; 3; 3\}$ & $C_2 \oplus C_2 \oplus C_3 \oplus C_3 \oplus C_3$ & $C_6 \oplus C_6 \oplus C_3$
            \end{tabular}
            \caption{Grupos abelianos de orden $108$.}
            \label{tab:grupos_abelianos_108}
        \end{table}





            \item (1 punto) Calcula la descomposición cíclica y cíclica primaria del grupo abeliano $\Aut(C_{16})$.\\
            
            Sea $C_{16}=\langle x\mid x^{16}=1\rangle$ el grupo cíclico de orden $16$. Tenemos que:
            \begin{align*}
                |\Aut(C_{16})| & = \varphi(16) = \varphi(2^4) = 1\cdot 2^3=8
            \end{align*}

            Veamos cuáles son. Por el Teorema de Dyck, construir estos automorfismos basta con enviar un generador de $C_{16}$ a otro generador. Los generadores de $C_{16}$ son los elementos de orden $16$, que son aquellos que son coprimos con $16$.
            \begin{align*}
                C_16 = \langle x\rangle
                = \langle x^3\rangle
                = \langle x^5\rangle
                = \langle x^7\rangle
                = \langle x^9\rangle
                = \langle x^{11}\rangle
                = \langle x^{13}\rangle
                = \langle x^{15}\rangle
            \end{align*}

            Por tanto, los automorfismos de $C_{16}$ son:
            \begin{align*}
                x &\mapsto \varphi_1(x) = x \\
                x &\mapsto \varphi_3(x) = x^3 \\
                x &\mapsto \varphi_5(x) = x^5 \\
                x &\mapsto \varphi_7(x) = x^7 \\
                x &\mapsto \varphi_9(x) = x^9 \\
                x &\mapsto \varphi_{11}(x) = x^{11} \\
                x &\mapsto \varphi_{13}(x) = x^{13} \\
                x &\mapsto \varphi_{15}(x) = x^{15}
            \end{align*}

            Veamos que $\Aut(C_{16})$ es abeliano. Dados $\varphi_i, \varphi_j \in \Aut(C_{16})$, tenemos que:
            \begin{align*}
                (\varphi_i \circ \varphi_j)(x) & = \varphi_i(\varphi_j(x)) = \varphi_i(x^j) = x^{ij} = x^{ji} = \varphi_j(\varphi_i(x)) = (\varphi_j \circ \varphi_i)(x)
            \end{align*}
            Por tanto, como la composición conmuta para un generador de $C_{16}$, se cumple:
            \begin{align*}
                \varphi_i \circ \varphi_j & = \varphi_j \circ \varphi_i\qquad \forall \varphi_i, \varphi_j \in \Aut(C_{16})
            \end{align*}
            Por tanto, $\Aut(C_{16})$ es abeliano. 
            Por la estructura de los grupos finitos abelianos, tenemos que hay dos posibilidades:
            \begin{align*}
                \Aut(C_{16}) & \cong C_8
                \qquad \lor \qquad
                \Aut(C_{16}) \cong C_4 \oplus C_2
                \qquad \lor \qquad
                \Aut(C_{16}) \cong C_2 \oplus C_2 \oplus C_2
            \end{align*}

            Para determinar la correcta, hemos de razonar por órdenes. Los órdenes de los elementos de $C_8\cong \bb{Z}_8$ son:
            \begin{gather*}
                O(0)=1,\quad
                O(1)=O(3)=O(5)=O(7)=8,\quad
                O(2)=O(6)=4,\quad
                O(4)=2
            \end{gather*}

            Los órdenes de los elementos de $C_4\oplus C_2\cong \bb{Z}_4\oplus \bb{Z}_2$ son:
            \begin{gather*}
                O(0,0)=1,\quad
                O(1,0)=O(3,0)=O(1,1)=O(3,1)=4,\quad
                O(2,0)=O(1,0)=O(2,1)=2
            \end{gather*}

            Los órdenes de los elementos de $C_2\oplus C_2\oplus C_2\cong \bb{Z}_2\oplus \bb{Z}_2\oplus \bb{Z}_2$ son:
            \begin{gather*}
                O(0,0,0)=1,\quad
                O(x,y,z)=2\qquad \forall (x,y,z)\in \bb{Z_2}^3\setminus \{(0,0,0)\}
            \end{gather*}

            Para determinar cuál es la correcta, hemos de ver qué órdenes tienen los elementos de $\Aut(C_{16})$. No es necesario calcular el orden de todos los elementos, sino que nos basta con ver cuántos tienen orden $2$:
            \begin{align*}
                (\varphi_i \circ \varphi_i)(x) & = \varphi_i(\varphi_i(x)) = \varphi_i(x^i) = x^{i^2}
            \end{align*}

            Por tanto, tenemos que $\varphi_i$ tiene orden $2$ si y solo si $i^2 \equiv 1 \mod 16$, es decir, si y solo si $i\in \{7,9,15\}$. Por tanto, hay $3$ elementos de orden $2$ en $\Aut(C_{16})$. Además:
            \begin{equation*}
                (\varphi_3\circ \varphi_3)(x) = \varphi_3(x^3) = x^{3^2} = x^9\neq x
                \Longrightarrow O(\varphi_3)\neq 2
            \end{equation*}
            
            De aquí, deducimos que la descomposición cíclica (y cíclica primaria) de $\Aut(C_{16})$ es:
            \begin{align*}
                \Aut(C_{16}) & \cong C_4 \oplus C_2
            \end{align*}

        \end{enumerate}
    \end{ejercicio}

    \begin{ejercicio}~
        \begin{enumerate}
            \item ($0.5$ puntos) Sea $\alpha=(2\ 3\ 4)(1\ 2\ 3) \in S_5$. Calcula $\alpha^{123}$.
            
            Hallamos la descomposición de $\alpha$ en ciclos disjuntos:
            \begin{equation*}
                \alpha = (2\ 3\ 4)(1\ 2\ 3) = (1\ 3)(2\ 4)
                \Longrightarrow O(\alpha)=\mcm(O(1\ 3),O(2\ 4)) = \mcm(2,2) = 2
            \end{equation*}

            Por tanto:
            \begin{align*}
                \alpha^{123} & = \alpha = (1\ 3)(2\ 4)
            \end{align*}
            \item ($1.5$ puntos) Calcula el número de $3$-subgrupos de Sylow de $S_5$.
            
            Sabemos que $|S_5|=5!=2^3\cdot 3\cdot 5$. Notando por $n_3$ el número de $3$-subgrupos de Sylow de $S_5$, por el Segundo Teorema de Sylow, tenemos que:
            \begin{align*}
                n_3 & \equiv 1 \mod 3 \\
                n_3 & \mid 2^3\cdot 5=40
            \end{align*}

            Como $n_3$ es un divisor de $40$, sus posibles valores son:
            \begin{align*}
                n_3 & \in \{1,2,4,5,8,10,20,40\}
            \end{align*}

            Como además $n_3 \equiv 1 \mod 3$, tenemos que:
            \begin{align*}
                n_3 & \in \{1,4,10,40\}
            \end{align*}

            Sea $P_3\in \Syl_3(S_5)$ un $3$-subgrupo de Sylow de $S_5$. Por tanto, $|P_3|=3$, luego $P_3$ es cíclico y contiene dos elementos de orden $3$. Los únicos elementos de orden $3$ en $S_5$ son los ciclos de longitud $3$; veamos cuántos hay:
            \begin{align*}
                \text{Número de ciclos de longitud } 3 & = \dfrac{5\cdot 4\cdot 3}{3} = 20
            \end{align*}

            Cada elemento de orden $3$ pertenece a un $3-$subgrupo de Sylow de $S_5$, ya que cualquier otro subconjunto de $S_5$ no va a tener cardinal múltiplo de $3$. Además, dados dos $3$-subgrupos de Sylow distintos, tienen que tener intersección trivial, ya que si no, tendrían un elemento de orden $3$ en común, y por tanto serían el mismo subgrupo.\\
            
            Por tanto, cada $3$-subgrupo de Sylow de $S_5$ contiene exactamente dos elementos de orden $3$, y como hay $20$ elementos de orden $3$, tenemos que:
            \begin{align*}
                n_3 & = \dfrac{20}{2} = 10
            \end{align*}

            Por tanto, el número de $3$-subgrupos de Sylow de $S_5$ es $10$.
        \end{enumerate}
    \end{ejercicio}

    \begin{ejercicio}~
        \begin{enumerate}
            \item (2 puntos) Demuestra que hay un único grupo de orden $885$ que además es abeliano.
            
            Sea $G$ un grupo de orden $885$. Notamos que:
            \begin{align*}
                885 & = 3\cdot 5\cdot 59
            \end{align*}
            Calculamos el número de subgrupos de Sylow de $G$, notando por $n_p$ el número de $p$-subgrupos de Sylow de $G$. Por el Segundo Teorema de Sylow, tenemos que:
            \begin{equation*}
                \left.\begin{array}{rcl}
                    n_3 & \equiv & 1 \mod 3 \\
                    n_3 & \mid & 5\cdot 59 = 295
                \end{array}\right\}
                \Longrightarrow n_3 \in \left\{1,\cancel{5},\cancel{59},295\right\}
            \end{equation*}

            De nuevo, por el Segundo Teorema de Sylow, tenemos que:
            \begin{equation*}
                \left.\begin{array}{rcl}
                    n_5 & \equiv & 1 \mod 5 \\
                    n_5 & \mid & 3\cdot 59 = 177
                \end{array}\right\}
                \Longrightarrow n_5 \in \left\{1,\cancel{3},\cancel{59},\cancel{177}\right\}
            \end{equation*}
            Por tanto, $n_5=1$, luego existe un único $5$-subgrupo de Sylow de $G$, que denotamos por $P_5$, que además es normal en $G$ ($P_5 \lhd G$). Como $|P_5|=5$, tenemos que $P_5 \cong C_5$.\\

            Por último, por el Segundo Teorema de Sylow, tenemos que:
            \begin{equation*}
                \left.\begin{array}{rcl}
                    n_{59} & \equiv & 1 \mod 59 \\
                    n_{59} & \mid & 3\cdot 5 = 15
                \end{array}\right\}
                \Longrightarrow n_{59}=1
            \end{equation*}
            Por tanto, existe un único $59$-subgrupo de Sylow de $G$, que denotamos por $P_{59}$, que además es normal en $G$ ($P_{59} \lhd G$). Como $|P_{59}|=59$, tenemos que $P_{59} \cong C_{59}$.\\

            Tenemos que $P_{59}\cap P_5=\{1\}$. Como $P_{5}\lhd G$, por el Segundo Teorema de Isomorfía, $P_5P_{59}< G$, con:
            \begin{equation*}
                \dfrac{P_5P_{59}}{P_{5}} \cong \dfrac{P_{59}}{P_{5}\cap P_{59}}
                \Longrightarrow |P_5P_{59}| = |P_5|\cdot |P_{59}| = 5\cdot 59 = 295
            \end{equation*}

            Sean $n_p'$ el número de $p$-subgrupos de Sylow de $P_5P_{59}$. Por el mismo razonamiento que antes, tenemos que $n_5'=1=n_{59}'=1$, luego $P_5P_{59}$ tiene un único $5$-subgrupo de Sylow ($P_5$) y un único $59$-subgrupo de Sylow ($P_{59}$). Por tanto:
            \begin{align*}
                P_5P_{59} & \cong C_5 \oplus C_{59}\cong C_{295}
            \end{align*}

            Veamos que $P_5P_{59}\lhd G$. Sea $g\in G$ y $xy\in P_5P_{59}$, con $x\in P_5$ y $y\in P_{59}$. Entonces:
            \begin{align*}
                gxyg^{-1} & = gxg^{-1}gyg^{-1}\in P_5P_{59}
            \end{align*}
            donde hemos empleado que $P_5\lhd G$ y $P_{59}\lhd G$. Por tanto, $P_5P_{59}\lhd G$. Tenemos que:
            \begin{itemize}
                \item $P_5P_{59}\lhd G$.
                \item $P_3\cap P_5P_{59}=\{1\}$, puesto que $P_5P_{59}$ no tiene elementos de orden $3$ puesto que $3\nmid 5\cdot 59$.
                \item Por el Segundo Teorema de Isomorfía, tenemos que:
                \begin{align*}
                    \dfrac{P_3}{P_3\cap P_5P_{59}} & \cong \dfrac{P_3P_5P_{59}}{P_5P_{59}}\Longrightarrow |P_3P_5P_{59}| = |P_3|\cdot |P_5P_{59}| = 3\cdot 5\cdot 59 = 885
                \end{align*}
                Por tanto, $P_3P_5P_{59}=G$.
            \end{itemize}

            Por tanto, $G\cong P_5P_{59}\rtimes_{\theta}P_3$, donde:
            \Func{\theta}{P_3}{\Aut(P_5P_{59})}{x}{\theta(x)}
            \Func{\theta(x)}{P_5P_{59}}{P_5P_{59}}{y}{xyx^{-1}}

            Veamos ahora cuántos productos semidirectos hay. Para ello, hemos de encontrar homomorfismos $\theta:P_3\to \Aut(P_5P_{59})$. Notamos que:
            \begin{align*}
                P_3\cong C_3 &\Longrightarrow \exists x\in P_3 \text{ tal que } P_3=\langle x\mid x^3=1\rangle\\
                P_5P_{59}\cong C_{295} &\Longrightarrow \exists y\in P_5P_{59} \text{ tal que } P_5P_{59}=\langle y\mid y^{295}=1\rangle
            \end{align*}

            Veamos en primer lugar cuántos automorfismos tiene $P_5P_{59}$. Por el Teorema de Dyck, dar un automorfismo de $P_5P_{59}$ equivale a dar la imagen del generador, garantizando que esta imagen es un generador. Calculamos cuántos generadores tiene $P_5P_{59}\cong C_{295}$:
            \begin{align*}
                \varphi(295) & = \varphi(5\cdot 59) = \varphi(5)\cdot \varphi(59) = (5-1)(59-1) = 4\cdot 58 = 232
            \end{align*}

            Por tanto, $|\Aut(P_5P_{59})|=232$.
            Como $O(x)=3$, tenemos que $O(\theta(x))\mid 3$, luego $O(\theta(x))\in \{1,3\}$.
            \begin{itemize}
                \item Supongamos que $O(\theta(x))=3$. Entonces, como el orden de todo elemento divide al orden del grupo, tenemos que:
                \begin{align*}
                    3\mid 232
                \end{align*}
                No obstante, esto no es cierto, luego llegamos a una contradicción.
            \end{itemize}
            
            Por tanto, $O(\theta(x))=1$, luego $\theta(x)$ es el automorfismo identidad. Por tanto, $\theta$ es el homomorfismo trivial. Por tanto, como $G\cong P_5P_{59}\rtimes_{\theta}P_3$ y $\theta$ es el homomorfismo trivial, tenemos que:
            \begin{align*}
                G & \cong P_5P_{59} \times P_3 \cong C_{295} \times C_3\cong C_{885}
            \end{align*}

            Por tanto, hay un único grupo de orden $885$ que además es abeliano, que es el grupo cíclico $C_{885}$.
            
            \item ($1.5$ puntos) Demuestra que todo grupo de orden $351$ es un producto semidirecto.
            
            Sea $G$ un grupo de orden $351$. Notamos que:
            \begin{align*}
                351 & = 3^3\cdot 13
            \end{align*}
            Sea $n_p$ el número de $p$-subgrupos de Sylow de $G$. Por el Segundo Teorema de Sylow, tenemos que:
            \begin{equation*}
                \left.\begin{array}{rcl}
                    n_3 & \equiv & 1 \mod 3 \\
                    n_3 & \mid & 13
                \end{array}\right\}
                \Longrightarrow n_3 \in \left\{1,13\right\}
            \end{equation*}

            De nuevo, por el Segundo Teorema de Sylow, tenemos que:
            \begin{equation*}
                \left.\begin{array}{rcl}
                    n_{13} & \equiv & 1 \mod 13 \\
                    n_{13} & \mid & 3^3=27
                \end{array}\right\}
                \Longrightarrow n_{13} \in \left\{1,\cancel{3},\cancel{9},27\right\}
            \end{equation*}

            Supongamos que $n_3=13$ y $n_{13}=27$. Como $n_{13}=27$, hay $27$ $13$-subgrupos de Sylow de $G$, que por ser de orden primo son cíclicos, y por tanto tienen $12$ elementos de orden $13$. Además, fijados dos $13$-subgrupos de Sylow distintos, tienen intersección trivial, ya que si no, tendrían un elemento de orden $13$ en común, y por tanto serían el mismo subgrupo. Por tanto, hay $27\cdot 12=324$ elementos de orden $13$ en $G$.\\


            Por otro lado, como $n_3=13$, hay $13$ $3$-subgrupos de Sylow de $G$, pero no tenemos garantizado que tengan intersección trivial. No obstante, cada uno tiene al menos $26$ elementos de orden $3$, $9$ o $27$. Por tanto, hay al menos $26$ elemenos nuevos en $G$. Como hay más de un $3$-subgrupo de Sylow, tenemos que hay algún elemento de orden $3$, $9$ o $27$ más, luego hay más de $26$ elementos de orden $3$, $9$ o $27$. Por tanto:
            \begin{equation*}
                |G| = 351 < 324 + 26 + 1 = 351
            \end{equation*}
            Hemos llegado a una contradicción, luego no puede ser que $n_3=13$ y $n_{13}=27$ simultáneamente. Por tanto, tenemos que $n_3=1$ o $n_{13}=1$.
            \begin{itemize}
                \item Si $n_3=1$, entonces existe un único $3$-subgrupo de Sylow de $G$, que denotamos por $P_3$, que además es normal en $G$ ($P_3 \lhd G$). Sea además $P_13\in \Syl_{13}(G)$ un $13$-subgrupo de Sylow de $G$.
                \begin{itemize}
                    \item $P_{3}\lhd G$.
                    \item Razonando por órdenes, vemos que $P_3\cap P_{13}=\{1\}$, ya que $P_3$ no tiene elementos de orden $13$ y $P_{13}$ no tiene elementos de orden múltiplo de $3$.
                    \item Por el Segundo Teorema de Isomorfía, tenemos que:
                    \begin{align*}
                        \dfrac{P_{13}}{P_{3}\cap P_{13}} & \cong \dfrac{P_{13}P_3}{P_3} \Longrightarrow |P_{13}P_3| = |P_{13}|\cdot |P_3| = 13\cdot 3^3 = 351
                    \end{align*}
                    Por tanto, $P_{13}P_3=G$.
                \end{itemize}

                Por tanto, $G\cong P_3 \rtimes_{\theta} P_{13}$, donde:
                \Func{\theta}{P_{13}}{\Aut(P_3)}{x}{\theta(x)}
                \Func{\theta(x)}{P_3}{P_3}{y}{xyx^{-1}}
                
                \item Si $n_{13}=1$, entonces existe un único $13$-subgrupo de Sylow de $G$, que denotamos por $P_{13}$, que además es normal en $G$ ($P_{13} \lhd G$). Sea además $P_3\in \Syl_3(G)$ un $3$-subgrupo de Sylow de $G$.
                \begin{itemize}
                    \item $P_{13}\lhd G$.
                    \item Razonando por órdenes, vemos que $P_{13}\cap P_{3}=\{1\}$, ya que $P_{13}$ no tiene elementos de orden múltiplo de $3$ y $P_3$ no tiene elementos de orden $13$.
                    \item Por el Segundo Teorema de Isomorfía, tenemos que:
                    \begin{align*}
                        \dfrac{P_{3}}{P_{13}\cap P_{3}} & \cong \dfrac{P_{3}P_{13}}{P_{13}} \Longrightarrow |P_{3}P_{13}| = |P_{3}|\cdot |P_{13}| = 3^3\cdot 13 = 351
                    \end{align*}
                    Por tanto, $P_{3}P_{13}=G$.
                \end{itemize}
                Por tanto, $G\cong P_{13} \rtimes_{\theta} P_{3}$, donde:
                \Func{\theta}{P_{3}}{\Aut(P_{13})}{x}{\theta(x)}
                \Func{\theta(x)}{P_{13}}{P_{13}}{y}{xyx^{-1}}
            \end{itemize}
            \item ($1.5$ puntos) Calcula todos los productos semidirectos $C_{13} \rtimes C_{27}$. ¿Cuántos hay salvo isomorfismo?
            
            Tenemos que:
            \begin{align*}
                C_{13}&=\langle x\mid x^{13}=1\rangle\\
                C_{27}&=\langle y\mid y^{27}=1\rangle
            \end{align*}

            Dar un producto semidirecto $C_{13} \rtimes C_{27}$ equivale a dar un homomorfismo de la forma $\theta:C_{27}\to \Aut(C_{13})$.

            Veamos en primer lugar cuántos automorfismos tiene $C_{13}$. Por el Teorema de Dyck, dar un automorfismo de $C_{13}$ equivale a dar la imagen del generador, garantizando que esta imagen es un generador. Calculamos cuántos generadores tiene $C_{13}$:
            \begin{align*}
                \varphi(13) & = 13-1 = 12
            \end{align*}
            Por tanto, $|\Aut(C_{13})|=12$. Para cada $i\in \{1,\ldots,12\}$, consideramos:
            \begin{align*}
                \varphi_i:C_{13} & \to C_{13} \\
                x & \mapsto x^i
            \end{align*}
            De esta forma, tenemos que $\Aut(C_{13})=\{\varphi_1,\ldots,\varphi_{12}\}$. De estos $12$, veamos cuáles nos sirven. Dar un homomorfismo $\theta:C_{27}\to \Aut(C_{13})$ equivale a dar la imagen del generador $y\in C_{27}$. Como $O(y)=27$, tenemos que $O(\theta(y))\mid 27$, luego $O(\theta(y))\in \{1,3,9,27\}$.  Además, puesto que $|\Aut(C_{13})|=12$, descartamos los automorfismos de orden $9$ y $27$. Por tanto, hemos de ver cuáles de los automorfismos $\varphi_i$ tienen orden $1$ o $3$.
            \begin{itemize}
                \item Si $O(\theta(y))=1$, entonces $\theta(y)=\varphi_1$.
                \item Si $O(\theta(y))=3$, entonces:
                \begin{equation*}
                    x = \varphi_i^3(x) = \varphi_i(\varphi_i(\varphi_i(x))) = x^{i^3}
                    \Longrightarrow i^3 \equiv 1 \mod 13
                    \Longrightarrow i\in \{3,9\}
                \end{equation*}
            \end{itemize}

            Por tanto, los automorfismos que nos sirven son:
            \begin{enumerate}
                \item $\theta(y)=\varphi_1=Id$.
                
                En este caso, el producto semidirecto es el producto directo:
                \begin{align*}
                    C_{13} \rtimes_{\theta} C_{27} & \cong C_{13} \times C_{27} \cong C_{351}
                \end{align*}

                \item $\theta(y)=\varphi_3$.
                
                En este caso, tenemos que:
                \begin{equation*}
                    yxy^{-1} = \varphi_3(x) = x^3
                \end{equation*}

                Por tanto:
                \begin{align*}
                    C_{13} \rtimes_{\varphi_3} C_{27} &\cong \langle x,y\mid x^{13}=1, y^{27}=1, yxy^{-1}=x^3\rangle
                    \cong C_{13} \rtimes_{\varphi_3} C_{27}
                \end{align*}
                
                \item $\theta(y)=\varphi_9$.
                
                En este caso, tenemos que:
                \begin{equation*}
                    yxy^{-1} = \varphi_9(x) = x^9
                \end{equation*}

                Por tanto:
                \begin{align*}
                    C_{13} \rtimes_{\varphi_9} C_{27} &\cong \langle x,y\mid x^{13}=1, y^{27}=1, yxy^{-1}=x^9\rangle
                \end{align*}
            \end{enumerate}

            Veamos ahora que $C_{13} \rtimes_{\varphi_3} C_{27}$ y $C_{13} \rtimes_{\varphi_9} C_{27}$ son isomorfos entre sí. Como $x,y^2$ son generadores de $C_{13}$ y $C_{27}$ respectivamente, tenemos que:
            \begin{align*}
                C_{13} \rtimes_{\varphi_3} C_{27} & = \langle x,y^2\rangle
            \end{align*}

            Sea $\varphi:C_{13} \rtimes_{\varphi_3} C_{27} \to C_{13} \rtimes_{\varphi_9} C_{27}$ el homomorfismo dado por:
            \begin{align*}
                \varphi(x) & = x \\
                \varphi(y) & = y^2
            \end{align*}

            Veamos que $x,y^2$ cumplen con las relaciones del grupo $C_{13} \rtimes_{\varphi_9} C_{27}$:
            \begin{align*}
                \varphi(x)^{13} & = x^{13} = 1 \\
                \varphi(y)^{27} & = (y^2)^{27} = (y^{27})^2 = 1^2 = 1 \\
                \varphi(y)\varphi(x)\varphi(y)^{-1} & = y^2 x (y^2)^{-1} = yx^3y^{-1}
                = x^3yx^2y^{-1} = x^6yxy^{-1} = x^9 = \varphi(x)^2
            \end{align*}

            Por el Teorema de Dyck, como $x,y^2$ cumplen con las relaciones del grupo $C_{13} \rtimes_{\varphi_9} C_{27}$, y son un generador, es sobreyectiva. Además, como el cardinal de $C_{13} \rtimes_{\varphi_3} C_{27}$ no depende del homomorfismo, tenemos que es inyectiva. Por tanto, $\varphi$ es un isomorfismo, luego:
            \begin{align*}
                C_{13} \rtimes_{\varphi_3} C_{27} & \cong C_{13} \rtimes_{\varphi_9} C_{27}
            \end{align*}

            Por tanto, los productos semidirectos $C_{13} \rtimes C_{27}$, salvo isomorfismo, son:
            \begin{align*}
                C_{13} \rtimes_{\varphi_1} C_{27} & \cong C_{351} \\
                C_{13} \rtimes_{\varphi_3} C_{27} & \cong \left\langle x,y\mid x^{13}=1, y^{27}=1, yxy^{-1}=x^3\right\rangle
            \end{align*}
        \end{enumerate}
    \end{ejercicio}
\end{document}
