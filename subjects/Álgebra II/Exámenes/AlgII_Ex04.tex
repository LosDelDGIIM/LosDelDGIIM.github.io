\documentclass[12pt]{article}

% Idioma y codificación
\usepackage[spanish, es-tabla]{babel}       %es-tabla para que se titule "Tabla"
\usepackage[utf8]{inputenc}

% Márgenes
\usepackage[a4paper,top=3cm,bottom=2.5cm,left=3cm,right=3cm]{geometry}

% Comentarios de bloque
\usepackage{verbatim}

% Paquetes de links
\usepackage[hidelinks]{hyperref}    % Permite enlaces
\usepackage{url}                    % redirecciona a la web

% Más opciones para enumeraciones
\usepackage{enumitem}

% Personalizar la portada
\usepackage{titling}

% Paquetes de tablas
\usepackage{multirow}


%------------------------------------------------------------------------

%Paquetes de figuras
\usepackage{caption}
\usepackage{subcaption} % Figuras al lado de otras
\usepackage{float}      % Poner figuras en el sitio indicado H.


% Paquetes de imágenes
\usepackage{graphicx}       % Paquete para añadir imágenes
\usepackage{transparent}    % Para manejar la opacidad de las figuras

% Paquete para usar colores
\usepackage[dvipsnames]{xcolor}
\usepackage{pagecolor}      % Para cambiar el color de la página

% Habilita tamaños de fuente mayores
\usepackage{fix-cm}

% Para los gráficos
\usepackage{tikz}

% Para poder situar los nodos en los grafos
\usetikzlibrary{positioning}


%------------------------------------------------------------------------

% Paquetes de matemáticas
\usepackage{mathtools, amsfonts, amssymb, mathrsfs}
\usepackage[makeroom]{cancel}     % Simplificar tachando
\usepackage{polynom}    % Divisiones y Ruffini
\usepackage{units} % Para poner fracciones diagonales con \nicefrac

\usepackage{pgfplots}   %Representar funciones
\pgfplotsset{compat=1.18}  % Versión 1.18

\usepackage{tikz-cd}    % Para usar diagramas de composiciones
\usetikzlibrary{calc}   % Para usar cálculo de coordenadas en tikz

%Definición de teoremas, etc.
\usepackage{amsthm}
%\swapnumbers   % Intercambia la posición del texto y de la numeración

\theoremstyle{plain}

\makeatletter
\@ifclassloaded{article}{
  \newtheorem{teo}{Teorema}[section]
}{
  \newtheorem{teo}{Teorema}[chapter]  % Se resetea en cada chapter
}
\makeatother

\newtheorem{coro}{Corolario}[teo]           % Se resetea en cada teorema
\newtheorem{prop}[teo]{Proposición}         % Usa el mismo contador que teorema
\newtheorem{lema}[teo]{Lema}                % Usa el mismo contador que teorema

\theoremstyle{remark}
\newtheorem*{observacion}{Observación}

\theoremstyle{definition}

\makeatletter
\@ifclassloaded{article}{
  \newtheorem{definicion}{Definición} [section]     % Se resetea en cada chapter
}{
  \newtheorem{definicion}{Definición} [chapter]     % Se resetea en cada chapter
}
\makeatother

\newtheorem*{notacion}{Notación}
\newtheorem*{ejemplo}{Ejemplo}
\newtheorem*{ejercicio*}{Ejercicio}             % No numerado
\newtheorem{ejercicio}{Ejercicio} [section]     % Se resetea en cada section


% Modificar el formato de la numeración del teorema "ejercicio"
\renewcommand{\theejercicio}{%
  \ifnum\value{section}=0 % Si no se ha iniciado ninguna sección
    \arabic{ejercicio}% Solo mostrar el número de ejercicio
  \else
    \thesection.\arabic{ejercicio}% Mostrar número de sección y número de ejercicio
  \fi
}


% \renewcommand\qedsymbol{$\blacksquare$}         % Cambiar símbolo QED
%------------------------------------------------------------------------

% Paquetes para encabezados
\usepackage{fancyhdr}
\pagestyle{fancy}
\fancyhf{}

\newcommand{\helv}{ % Modificación tamaño de letra
\fontfamily{}\fontsize{12}{12}\selectfont}
\setlength{\headheight}{15pt} % Amplía el tamaño del índice


%\usepackage{lastpage}   % Referenciar última pag   \pageref{LastPage}
\fancyfoot[C]{\thepage}

%------------------------------------------------------------------------

% Conseguir que no ponga "Capítulo 1". Sino solo "1."
\makeatletter
\@ifclassloaded{book}{
  \renewcommand{\chaptermark}[1]{\markboth{\thechapter.\ #1}{}} % En el encabezado
    
  \renewcommand{\@makechapterhead}[1]{%
  \vspace*{50\p@}%
  {\parindent \z@ \raggedright \normalfont
    \ifnum \c@secnumdepth >\m@ne
      \huge\bfseries \thechapter.\hspace{1em}\ignorespaces
    \fi
    \interlinepenalty\@M
    \Huge \bfseries #1\par\nobreak
    \vskip 40\p@
  }}
}
\makeatother

%------------------------------------------------------------------------
% Paquetes de cógido
\usepackage{minted}
\renewcommand\listingscaption{Código fuente}

\usepackage{fancyvrb}
% Personaliza el tamaño de los números de línea
\renewcommand{\theFancyVerbLine}{\small\arabic{FancyVerbLine}}

% Estilo para C++
\newminted{cpp}{
    frame=lines,
    framesep=2mm,
    baselinestretch=1.2,
    linenos,
    escapeinside=||
}

% para minted
\definecolor{LightGray}{rgb}{0.95,0.95,0.92}
\setminted{
    linenos=true,
    stepnumber=5,
    numberfirstline=true,
    autogobble,
    breaklines=true,
    breakautoindent=true,
    breaksymbolleft=,
    breaksymbolright=,
    breaksymbolindentleft=0pt,
    breaksymbolindentright=0pt,
    breaksymbolsepleft=0pt,
    breaksymbolsepright=0pt,
    fontsize=\footnotesize,
    bgcolor=LightGray,
    numbersep=10pt
}


\usepackage{listings} % Para incluir código desde un archivo

\renewcommand\lstlistingname{Código Fuente}
\renewcommand\lstlistlistingname{Índice de Códigos Fuente}

% Definir colores
\definecolor{vscodepurple}{rgb}{0.5,0,0.5}
\definecolor{vscodeblue}{rgb}{0,0,0.8}
\definecolor{vscodegreen}{rgb}{0,0.5,0}
\definecolor{vscodegray}{rgb}{0.5,0.5,0.5}
\definecolor{vscodebackground}{rgb}{0.97,0.97,0.97}
\definecolor{vscodelightgray}{rgb}{0.9,0.9,0.9}

% Configuración para el estilo de C similar a VSCode
\lstdefinestyle{vscode_C}{
  backgroundcolor=\color{vscodebackground},
  commentstyle=\color{vscodegreen},
  keywordstyle=\color{vscodeblue},
  numberstyle=\tiny\color{vscodegray},
  stringstyle=\color{vscodepurple},
  basicstyle=\scriptsize\ttfamily,
  breakatwhitespace=false,
  breaklines=true,
  captionpos=b,
  keepspaces=true,
  numbers=left,
  numbersep=5pt,
  showspaces=false,
  showstringspaces=false,
  showtabs=false,
  tabsize=2,
  frame=tb,
  framerule=0pt,
  aboveskip=10pt,
  belowskip=10pt,
  xleftmargin=10pt,
  xrightmargin=10pt,
  framexleftmargin=10pt,
  framexrightmargin=10pt,
  framesep=0pt,
  rulecolor=\color{vscodelightgray},
  backgroundcolor=\color{vscodebackground},
}

%------------------------------------------------------------------------

% Comandos definidos
\newcommand{\bb}[1]{\mathbb{#1}}
\newcommand{\cc}[1]{\mathcal{#1}}

% I prefer the slanted \leq
\let\oldleq\leq % save them in case they're every wanted
\let\oldgeq\geq
\renewcommand{\leq}{\leqslant}
\renewcommand{\geq}{\geqslant}

% Si y solo si
\newcommand{\sii}{\iff}

% Letras griegas
\newcommand{\eps}{\epsilon}
\newcommand{\veps}{\varepsilon}
\newcommand{\lm}{\lambda}

\newcommand{\ol}{\overline}
\newcommand{\ul}{\underline}
\newcommand{\wt}{\widetilde}
\newcommand{\wh}{\widehat}

\let\oldvec\vec
\renewcommand{\vec}{\overrightarrow}

% Derivadas parciales
\newcommand{\del}[2]{\frac{\partial #1}{\partial #2}}
\newcommand{\Del}[3]{\frac{\partial^{#1} #2}{\partial #3^{#1}}}
\newcommand{\deld}[2]{\dfrac{\partial #1}{\partial #2}}
\newcommand{\Deld}[3]{\dfrac{\partial^{#1} #2}{\partial #3^{#1}}}


\newcommand{\AstIg}{\stackrel{(\ast)}{=}}
\newcommand{\Hop}{\stackrel{L'H\hat{o}pital}{=}}

\newcommand{\red}[1]{{\color{red}#1}} % Para integrales, destacar los cambios.

% Método de integración
\newcommand{\MetInt}[2]{
    \left[\begin{array}{c}
        #1 \\ #2
    \end{array}\right]
}

% Declarar aplicaciones
% 1. Nombre aplicación
% 2. Dominio
% 3. Codominio
% 4. Variable
% 5. Imagen de la variable
\newcommand{\Func}[5]{
    \begin{equation*}
        \begin{array}{rrll}
            #1:& #2 & \longrightarrow & #3\\
               & #4 & \longmapsto & #5
        \end{array}
    \end{equation*}
}

%------------------------------------------------------------------------


\DeclareMathOperator{\GL}{GL}
\DeclareMathOperator{\SL}{GL}
\DeclareMathOperator{\Z}{Z}
\DeclareMathOperator{\Q}{Q}
\begin{document}

    % 1. Foto de fondo
    % 2. Título
    % 3. Encabezado Izquierdo
    % 4. Color de fondo
    % 5. Coord x del titulo
    % 6. Coord y del titulo
    % 7. Fecha

    
    % 1. Foto de fondo
% 2. Título
% 3. Encabezado Izquierdo
% 4. Color de fondo
% 5. Coord x del titulo
% 6. Coord y del titulo
% 7. Fecha

\newcommand{\portada}[7]{

    \portadaBase{#1}{#2}{#3}{#4}{#5}{#6}{#7}
    \portadaBook{#1}{#2}{#3}{#4}{#5}{#6}{#7}
}

\newcommand{\portadaExamen}[7]{

    \portadaBase{#1}{#2}{#3}{#4}{#5}{#6}{#7}
    \portadaArticle{#1}{#2}{#3}{#4}{#5}{#6}{#7}
}




\newcommand{\portadaBase}[7]{

    % Tiene la portada principal y la licencia Creative Commons
    
    % 1. Foto de fondo
    % 2. Título
    % 3. Encabezado Izquierdo
    % 4. Color de fondo
    % 5. Coord x del titulo
    % 6. Coord y del titulo
    % 7. Fecha
    
    
    \thispagestyle{empty}               % Sin encabezado ni pie de página
    \newgeometry{margin=0cm}        % Márgenes nulos para la primera página
    
    
    % Encabezado
    \fancyhead[L]{\helv #3}
    \fancyhead[R]{\helv \nouppercase{\leftmark}}
    
    
    \pagecolor{#4}        % Color de fondo para la portada
    
    \begin{figure}[p]
        \centering
        \transparent{0.3}           % Opacidad del 30% para la imagen
        
        \includegraphics[width=\paperwidth, keepaspectratio]{assets/#1}
    
        \begin{tikzpicture}[remember picture, overlay]
            \node[anchor=north west, text=white, opacity=1, font=\fontsize{60}{90}\selectfont\bfseries\sffamily, align=left] at (#5, #6) {#2};
            
            \node[anchor=south east, text=white, opacity=1, font=\fontsize{12}{18}\selectfont\sffamily, align=right] at (9.7, 3) {\textbf{\href{https://losdeldgiim.github.io/}{Los Del DGIIM}}};
            
            \node[anchor=south east, text=white, opacity=1, font=\fontsize{12}{15}\selectfont\sffamily, align=right] at (9.7, 1.8) {Doble Grado en Ingeniería Informática y Matemáticas\\Universidad de Granada};
        \end{tikzpicture}
    \end{figure}
    
    
    \restoregeometry        % Restaurar márgenes normales para las páginas subsiguientes
    \pagecolor{white}       % Restaurar el color de página
    
    
    \newpage
    \thispagestyle{empty}               % Sin encabezado ni pie de página
    \begin{tikzpicture}[remember picture, overlay]
        \node[anchor=south west, inner sep=3cm] at (current page.south west) {
            \begin{minipage}{0.5\paperwidth}
                \href{https://creativecommons.org/licenses/by-nc-nd/4.0/}{
                    \includegraphics[height=2cm]{assets/Licencia.png}
                }\vspace{1cm}\\
                Esta obra está bajo una
                \href{https://creativecommons.org/licenses/by-nc-nd/4.0/}{
                    Licencia Creative Commons Atribución-NoComercial-SinDerivadas 4.0 Internacional (CC BY-NC-ND 4.0).
                }\\
    
                Eres libre de compartir y redistribuir el contenido de esta obra en cualquier medio o formato, siempre y cuando des el crédito adecuado a los autores originales y no persigas fines comerciales. 
            \end{minipage}
        };
    \end{tikzpicture}
    
    
    
    % 1. Foto de fondo
    % 2. Título
    % 3. Encabezado Izquierdo
    % 4. Color de fondo
    % 5. Coord x del titulo
    % 6. Coord y del titulo
    % 7. Fecha


}


\newcommand{\portadaBook}[7]{

    % 1. Foto de fondo
    % 2. Título
    % 3. Encabezado Izquierdo
    % 4. Color de fondo
    % 5. Coord x del titulo
    % 6. Coord y del titulo
    % 7. Fecha

    % Personaliza el formato del título
    \pretitle{\begin{center}\bfseries\fontsize{42}{56}\selectfont}
    \posttitle{\par\end{center}\vspace{2em}}
    
    % Personaliza el formato del autor
    \preauthor{\begin{center}\Large}
    \postauthor{\par\end{center}\vfill}
    
    % Personaliza el formato de la fecha
    \predate{\begin{center}\huge}
    \postdate{\par\end{center}\vspace{2em}}
    
    \title{#2}
    \author{\href{https://losdeldgiim.github.io/}{Los Del DGIIM}}
    \date{Granada, #7}
    \maketitle
    
    \tableofcontents
}




\newcommand{\portadaArticle}[7]{

    % 1. Foto de fondo
    % 2. Título
    % 3. Encabezado Izquierdo
    % 4. Color de fondo
    % 5. Coord x del titulo
    % 6. Coord y del titulo
    % 7. Fecha

    % Personaliza el formato del título
    \pretitle{\begin{center}\bfseries\fontsize{42}{56}\selectfont}
    \posttitle{\par\end{center}\vspace{2em}}
    
    % Personaliza el formato del autor
    \preauthor{\begin{center}\Large}
    \postauthor{\par\end{center}\vspace{3em}}
    
    % Personaliza el formato de la fecha
    \predate{\begin{center}\huge}
    \postdate{\par\end{center}\vspace{5em}}
    
    \title{#2}
    \author{\href{https://losdeldgiim.github.io/}{Los Del DGIIM}}
    \date{Granada, #7}
    \thispagestyle{empty}               % Sin encabezado ni pie de página
    \maketitle
    \vfill
}
    \portadaExamen{ffccA4.jpg}{Álgebra II\\Examen III}{Álgebra II. Examen III}{MidnightBlue}{-8}{28}{2025}{Arturo Olivares Martos}

    \begin{description}
        \item[Asignatura] Álgebra II.
        \item[Curso Académico] 2018-19.
        \item[Grado] Doble Grado en Ingeniería Informática y Matemáticas.
        \item[Grupo] Único.
        %\item[Profesor] José María Espinar García.
        \item[Descripción] Parcial I.
        \item[Fecha] Octubre de 2018.
        % \item[Duración] 60 minutos.
    
    \end{description}
    \newpage

    \begin{ejercicio}~
        \begin{enumerate}
            \item Sea $f: S_4 \to S_6$ la aplicación dada por $f(\sigma) = \tau$ donde $\tau$ actúa igual que $\sigma$ sobre los elementos $\{1, 2, 3, 4\}$ y los elementos $\{5, 6\}$ los fija si $\sigma$ es par o bien los intercambia si $\sigma$ es impar. Demuestra que $f$ es un homomorfismo inyectivo de grupos y que su imagen está contenida en $A_6$.
            \item Considera los grupos $Q_2 = \langle x, y \mid x^4 = 1, y^2 = x^2, yx = x^{-1}y \rangle$ y $S_4$. Demuestra que la asignación
            \begin{align*}
                x &\mapsto (12)(34),\\
                y &\mapsto (34)
            \end{align*}
            determina un homomorfismo de grupos. Calcula su imagen y su núcleo, dando todos sus elementos.
        \end{enumerate}
    \end{ejercicio}

    \begin{ejercicio}
        Razona cuál es la respuesta correcta en cada una de las siguientes cuestiones:
        \begin{enumerate}
            \item Sean $C_8$ y $C_{12}$ los grupos cíclicos de órdenes 8 y 12 respectivamente. El número de homomorfismos de grupos de $C_8$ en $C_{12}$ es:
            \begin{enumerate}
                \item Dos.
                \item Tres.
                \item Cuatro.
            \end{enumerate}

            \item Si $\sigma = (2\ 5\ 8\ 4\ 1\ 3)(4\ 6\ 7\ 8\ 5)(8\ 10\ 11)$ en $S_{11}$, entonces la permutación $\sigma^{1000}$:
            \begin{enumerate}
                \item Es impar.
                \item Tiene orden 3.
                \item Es un 6-ciclo.
            \end{enumerate}

            \item La ecuación $x(1\ 2\ 3)x^{-1} = (1\ 3)(5\ 7\ 8)$ en $S_8$:
            \begin{enumerate}
                \item No tiene solución.
                \item Tiene una única solución.
                \item Tiene solución pero no es única.
            \end{enumerate}


            \item La ecuación $x(1\ 2)(3\ 4)x^{-1} = (5\ 6)(1\ 3)$ en $S_6$:
            \begin{enumerate}
                \item No tiene solución.
                \item Tiene una única solución.
                \item Tiene solución pero no es única.
            \end{enumerate}

            \item Si $G \neq 1$ es un grupo cíclico que tiene un solo generador entonces:
            \begin{enumerate}
                \item $G$ es infinito.
                \item No existe $G$ en esas condiciones.
                \item $G$ tiene como mucho 2 elementos.
            \end{enumerate}

            \item Sea $G \neq 1$ un grupo. Entonces:
            \begin{enumerate}
                \item $G$ puede tener un subgrupo propio isomorfo a $G$.
                \item Si todos los subgrupos propios de $G$ son abelianos entonces $G$ es abeliano.
                \item Si todos los subgrupos propios de $G$ son cíclicos entonces $G$ es cíclico.
            \end{enumerate}

            \item El grupo simétrico $S_4$:
            \begin{enumerate}
                \item Es cíclico.
                \item No es cíclico pero se puede generar por dos elementos.
                \item No tiene subgrupos de orden 6.
            \end{enumerate}

            \item Si se consideran los grupos aditivos $\Z$ de los enteros, $\Q$ de los racionales y $\Z_n$ de los enteros módulo $n=2,5,10$, se tiene que:
            \begin{enumerate}
                \item Los grupos $\Z \times \Z_2$ y $\Z$ son isomorfos.
                \item Los grupos $\Q \times \Z_{10}$ y $\Z_2 \times \Q \times \Z_5$ son isomorfos.
                \item Los grupos $\Z \times \Z_2$ y $\Q \times \Z_2$ son isomorfos.
            \end{enumerate}

            \item El subgrupo $\SL_3(\Z_2) < \GL_3(\Z_2)$ de las matrices invertibles $3\times 3$ con entradas en $\Z_2$ y de determinante 1:
            \begin{enumerate}
                \item Es un subgrupo impropio.
                \item Es un grupo abeliano de orden 168.
                \item Es un grupo no abeliano de orden 84.
            \end{enumerate}

            % // TODO: CONTINUAR
        \end{enumerate}
    \end{ejercicio}

    


    \begin{comment}
        1. Sea f : S4 ! S6 la aplicaci´on dada por f() =  donde  act´ua igual
que  sobre los elementos {1, 2, 3, 4} y los elementos {5, 6} los fija si
 es par o bien los intercambia si  es impar. Demuestra que f es un
homomorfismo inyectivo de grupos y que su imagen est´a contenida en
A6.
2. Considera los grupos Q2 =< x, y | x4 = 1, y2 = x2, yx = x1y > y S4.
Demuestra que la asignaci´on
x 7! (12)(34) , y 7! (34)
determina un homomorfismo de grupos. Calcula su imagen y su n´ucleo,
dando todos sus elementos.






Razona cual es la respuesta correcta en cada una de las siguientes cuestiones:
1. Sean C8 y C12 los grupos c´ıclicos de ´ordenes 8 y 12 respectivamente.
El n´umero de homomorfismos de grupos de C8 en C12 es:
a) Dos.
b) Tres.
c) Cuatro.
2. Si  = (2 5 8 4 1 3)(4 6 7 8 5)(8 10 11) 2 S11, entonces la permutaci´on
1000:
a) Es impar.
b) Tiene orden 3.
c) Es un 6-ciclo.
3. La ecuaci´on x(1 2 3)x1 = (1 3)(5 7 8) en S8:
a) No tiene soluci´on.
b) Tiene una ´unica soluci´on.
c) Tiene soluci´on pero no es ´unica.
4. La ecuaci´on x(1 2)(3 4)x1 = (5 6)(1 3) en S6:
a) No tiene soluci´on.
b) Tiene una ´unica soluci´on.
c) Tiene soluci´on pero no es ´unica.
5. Si G 6= 1 es un grupo c´ıclico que tiene un solo generador entonces:
a) G es infinito.
b) No existe G en esas condiciones.
c) G tiene como mucho 2 elementos.
6. Sea G 6= 1 un grupo. Entonces:
a) G puede tener un subgrupo propio isomorfo a G.
b) Si todos los subgrupos propios de G son abelianos entonces G es
abeliano.
c) Si todos los subgrupos propios de G son c´ıclicos entonces G es
c´ıclico.
7. El grupo sim´etrico S4:
a) Es c´ıclico.
b) No es c´ıclico pero se puede generar por dos elementos.
) No tiene subgrupos de orden 6.
8. Si se consideran los grupos aditivos Z de los enteros, Q de los racionales
y Zn de los enteros m´odulo n=2,5,10, se tiene que:
a) Los grupos Z ⇥ Z2 y Z son isomorfos.
b) Los grupos Q ⇥ Z10 y Z2 ⇥ Q ⇥ Z5 son isomorfos.
c) Los grupos Z ⇥ Z2 y Q ⇥ Z2 son isomorfos.
9. El subgrupo SL3(Z2) < GL3(Z2) de las matrices invertibles 3⇥3 con
entradas en Z2 y de determinante 1:
a) Es un subgrupo impropio.
b) Es un grupo abeliano de orden 168.
c) Es un grupo no abeliano de orden 84.
10. a) El grupo Z ⇥ Z5 es c´ıclico.
b) El grupo Z ⇥ Z5 tiene todos sus elementos de orden infinito.
c) El grupo Z ⇥ Z5 es finitamente generado.
11. Sea C120 =< x | x120 = 1 > y se consideran sus subgrupos H =< x42 >
y K =< x36 >. Entonces se tiene que:
a) K<H.
b) H<K.
c) H = K.
12. Sea f : G ! H un homomorfismo de grupos. Entonces:
a) Si f es inyectivo y G es abeliano entonces H es abeliano.
b) Si f es inyectivo y H es abeliano entonces G es abeliano.
c) Ninguno de los enunciados anteriores es cierto.
13. Dados los grupos C8 =< a | a8 = 1 > y D4 =< x, y | x4 = 1, y2 =
1, yx = x1y >, se tiene que la asignaci´on x 7! a2, y 7! a4:
a) Determina un homomorfismo de grupos sobreyectivo.
b) Determina un homomorfismo de grupos pero no es sobreyectivo.
c) No determina un homomorfismo de grupos.
14. Se considera el subgrupo de S5, H =< (123),(4, 5) >. Entonces:
a) H es un grupo abeliano pero no es c´ıclico.
b) H es un grupo c´ıclico.
6
8
c) S5 es un grupo no abeliano y por tanto H tampoco es abeliano.
15. Sea f : G ! H un homomorfismo de grupos. Entonces:
a) Si f es sobreyectivo y G es abeliano entonces H es abeliano.
b) Si f es sobreyectivo y H es abeliano entonces G es abeliano.
c) Ninguno de los enunciados anteriores es cierto.
    \end{comment}
\end{document}