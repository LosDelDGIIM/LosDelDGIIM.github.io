\documentclass[12pt]{article}

% Idioma y codificación
\usepackage[spanish, es-tabla]{babel}       %es-tabla para que se titule "Tabla"
\usepackage[utf8]{inputenc}

% Márgenes
\usepackage[a4paper,top=3cm,bottom=2.5cm,left=3cm,right=3cm]{geometry}

% Comentarios de bloque
\usepackage{verbatim}

% Paquetes de links
\usepackage[hidelinks]{hyperref}    % Permite enlaces
\usepackage{url}                    % redirecciona a la web

% Más opciones para enumeraciones
\usepackage{enumitem}

% Personalizar la portada
\usepackage{titling}

% Paquetes de tablas
\usepackage{multirow}


%------------------------------------------------------------------------

%Paquetes de figuras
\usepackage{caption}
\usepackage{subcaption} % Figuras al lado de otras
\usepackage{float}      % Poner figuras en el sitio indicado H.


% Paquetes de imágenes
\usepackage{graphicx}       % Paquete para añadir imágenes
\usepackage{transparent}    % Para manejar la opacidad de las figuras

% Paquete para usar colores
\usepackage[dvipsnames]{xcolor}
\usepackage{pagecolor}      % Para cambiar el color de la página

% Habilita tamaños de fuente mayores
\usepackage{fix-cm}

% Para los gráficos
\usepackage{tikz}

% Para poder situar los nodos en los grafos
\usetikzlibrary{positioning}


%------------------------------------------------------------------------

% Paquetes de matemáticas
\usepackage{mathtools, amsfonts, amssymb, mathrsfs}
\usepackage[makeroom]{cancel}     % Simplificar tachando
\usepackage{polynom}    % Divisiones y Ruffini
\usepackage{units} % Para poner fracciones diagonales con \nicefrac

\usepackage{pgfplots}   %Representar funciones
\pgfplotsset{compat=1.18}  % Versión 1.18

\usepackage{tikz-cd}    % Para usar diagramas de composiciones
\usetikzlibrary{calc}   % Para usar cálculo de coordenadas en tikz

%Definición de teoremas, etc.
\usepackage{amsthm}
%\swapnumbers   % Intercambia la posición del texto y de la numeración

\theoremstyle{plain}

\makeatletter
\@ifclassloaded{article}{
  \newtheorem{teo}{Teorema}[section]
}{
  \newtheorem{teo}{Teorema}[chapter]  % Se resetea en cada chapter
}
\makeatother

\newtheorem{coro}{Corolario}[teo]           % Se resetea en cada teorema
\newtheorem{prop}[teo]{Proposición}         % Usa el mismo contador que teorema
\newtheorem{lema}[teo]{Lema}                % Usa el mismo contador que teorema

\theoremstyle{remark}
\newtheorem*{observacion}{Observación}

\theoremstyle{definition}

\makeatletter
\@ifclassloaded{article}{
  \newtheorem{definicion}{Definición} [section]     % Se resetea en cada chapter
}{
  \newtheorem{definicion}{Definición} [chapter]     % Se resetea en cada chapter
}
\makeatother

\newtheorem*{notacion}{Notación}
\newtheorem*{ejemplo}{Ejemplo}
\newtheorem*{ejercicio*}{Ejercicio}             % No numerado
\newtheorem{ejercicio}{Ejercicio} [section]     % Se resetea en cada section


% Modificar el formato de la numeración del teorema "ejercicio"
\renewcommand{\theejercicio}{%
  \ifnum\value{section}=0 % Si no se ha iniciado ninguna sección
    \arabic{ejercicio}% Solo mostrar el número de ejercicio
  \else
    \thesection.\arabic{ejercicio}% Mostrar número de sección y número de ejercicio
  \fi
}


% \renewcommand\qedsymbol{$\blacksquare$}         % Cambiar símbolo QED
%------------------------------------------------------------------------

% Paquetes para encabezados
\usepackage{fancyhdr}
\pagestyle{fancy}
\fancyhf{}

\newcommand{\helv}{ % Modificación tamaño de letra
\fontfamily{}\fontsize{12}{12}\selectfont}
\setlength{\headheight}{15pt} % Amplía el tamaño del índice


%\usepackage{lastpage}   % Referenciar última pag   \pageref{LastPage}
\fancyfoot[C]{\thepage}

%------------------------------------------------------------------------

% Conseguir que no ponga "Capítulo 1". Sino solo "1."
\makeatletter
\@ifclassloaded{book}{
  \renewcommand{\chaptermark}[1]{\markboth{\thechapter.\ #1}{}} % En el encabezado
    
  \renewcommand{\@makechapterhead}[1]{%
  \vspace*{50\p@}%
  {\parindent \z@ \raggedright \normalfont
    \ifnum \c@secnumdepth >\m@ne
      \huge\bfseries \thechapter.\hspace{1em}\ignorespaces
    \fi
    \interlinepenalty\@M
    \Huge \bfseries #1\par\nobreak
    \vskip 40\p@
  }}
}
\makeatother

%------------------------------------------------------------------------
% Paquetes de cógido
\usepackage{minted}
\renewcommand\listingscaption{Código fuente}

\usepackage{fancyvrb}
% Personaliza el tamaño de los números de línea
\renewcommand{\theFancyVerbLine}{\small\arabic{FancyVerbLine}}

% Estilo para C++
\newminted{cpp}{
    frame=lines,
    framesep=2mm,
    baselinestretch=1.2,
    linenos,
    escapeinside=||
}

% para minted
\definecolor{LightGray}{rgb}{0.95,0.95,0.92}
\setminted{
    linenos=true,
    stepnumber=5,
    numberfirstline=true,
    autogobble,
    breaklines=true,
    breakautoindent=true,
    breaksymbolleft=,
    breaksymbolright=,
    breaksymbolindentleft=0pt,
    breaksymbolindentright=0pt,
    breaksymbolsepleft=0pt,
    breaksymbolsepright=0pt,
    fontsize=\footnotesize,
    bgcolor=LightGray,
    numbersep=10pt
}


\usepackage{listings} % Para incluir código desde un archivo

\renewcommand\lstlistingname{Código Fuente}
\renewcommand\lstlistlistingname{Índice de Códigos Fuente}

% Definir colores
\definecolor{vscodepurple}{rgb}{0.5,0,0.5}
\definecolor{vscodeblue}{rgb}{0,0,0.8}
\definecolor{vscodegreen}{rgb}{0,0.5,0}
\definecolor{vscodegray}{rgb}{0.5,0.5,0.5}
\definecolor{vscodebackground}{rgb}{0.97,0.97,0.97}
\definecolor{vscodelightgray}{rgb}{0.9,0.9,0.9}

% Configuración para el estilo de C similar a VSCode
\lstdefinestyle{vscode_C}{
  backgroundcolor=\color{vscodebackground},
  commentstyle=\color{vscodegreen},
  keywordstyle=\color{vscodeblue},
  numberstyle=\tiny\color{vscodegray},
  stringstyle=\color{vscodepurple},
  basicstyle=\scriptsize\ttfamily,
  breakatwhitespace=false,
  breaklines=true,
  captionpos=b,
  keepspaces=true,
  numbers=left,
  numbersep=5pt,
  showspaces=false,
  showstringspaces=false,
  showtabs=false,
  tabsize=2,
  frame=tb,
  framerule=0pt,
  aboveskip=10pt,
  belowskip=10pt,
  xleftmargin=10pt,
  xrightmargin=10pt,
  framexleftmargin=10pt,
  framexrightmargin=10pt,
  framesep=0pt,
  rulecolor=\color{vscodelightgray},
  backgroundcolor=\color{vscodebackground},
}

%------------------------------------------------------------------------

% Comandos definidos
\newcommand{\bb}[1]{\mathbb{#1}}
\newcommand{\cc}[1]{\mathcal{#1}}

% I prefer the slanted \leq
\let\oldleq\leq % save them in case they're every wanted
\let\oldgeq\geq
\renewcommand{\leq}{\leqslant}
\renewcommand{\geq}{\geqslant}

% Si y solo si
\newcommand{\sii}{\iff}

% Letras griegas
\newcommand{\eps}{\epsilon}
\newcommand{\veps}{\varepsilon}
\newcommand{\lm}{\lambda}

\newcommand{\ol}{\overline}
\newcommand{\ul}{\underline}
\newcommand{\wt}{\widetilde}
\newcommand{\wh}{\widehat}

\let\oldvec\vec
\renewcommand{\vec}{\overrightarrow}

% Derivadas parciales
\newcommand{\del}[2]{\frac{\partial #1}{\partial #2}}
\newcommand{\Del}[3]{\frac{\partial^{#1} #2}{\partial #3^{#1}}}
\newcommand{\deld}[2]{\dfrac{\partial #1}{\partial #2}}
\newcommand{\Deld}[3]{\dfrac{\partial^{#1} #2}{\partial #3^{#1}}}


\newcommand{\AstIg}{\stackrel{(\ast)}{=}}
\newcommand{\Hop}{\stackrel{L'H\hat{o}pital}{=}}

\newcommand{\red}[1]{{\color{red}#1}} % Para integrales, destacar los cambios.

% Método de integración
\newcommand{\MetInt}[2]{
    \left[\begin{array}{c}
        #1 \\ #2
    \end{array}\right]
}

% Declarar aplicaciones
% 1. Nombre aplicación
% 2. Dominio
% 3. Codominio
% 4. Variable
% 5. Imagen de la variable
\newcommand{\Func}[5]{
    \begin{equation*}
        \begin{array}{rrll}
            #1:& #2 & \longrightarrow & #3\\
               & #4 & \longmapsto & #5
        \end{array}
    \end{equation*}
}

%------------------------------------------------------------------------


\DeclareMathOperator{\GL}{GL}
\begin{document}

    % 1. Foto de fondo
    % 2. Título
    % 3. Encabezado Izquierdo
    % 4. Color de fondo
    % 5. Coord x del titulo
    % 6. Coord y del titulo
    % 7. Fecha

    
    % 1. Foto de fondo
% 2. Título
% 3. Encabezado Izquierdo
% 4. Color de fondo
% 5. Coord x del titulo
% 6. Coord y del titulo
% 7. Fecha

\newcommand{\portada}[7]{

    \portadaBase{#1}{#2}{#3}{#4}{#5}{#6}{#7}
    \portadaBook{#1}{#2}{#3}{#4}{#5}{#6}{#7}
}

\newcommand{\portadaExamen}[7]{

    \portadaBase{#1}{#2}{#3}{#4}{#5}{#6}{#7}
    \portadaArticle{#1}{#2}{#3}{#4}{#5}{#6}{#7}
}




\newcommand{\portadaBase}[7]{

    % Tiene la portada principal y la licencia Creative Commons
    
    % 1. Foto de fondo
    % 2. Título
    % 3. Encabezado Izquierdo
    % 4. Color de fondo
    % 5. Coord x del titulo
    % 6. Coord y del titulo
    % 7. Fecha
    
    
    \thispagestyle{empty}               % Sin encabezado ni pie de página
    \newgeometry{margin=0cm}        % Márgenes nulos para la primera página
    
    
    % Encabezado
    \fancyhead[L]{\helv #3}
    \fancyhead[R]{\helv \nouppercase{\leftmark}}
    
    
    \pagecolor{#4}        % Color de fondo para la portada
    
    \begin{figure}[p]
        \centering
        \transparent{0.3}           % Opacidad del 30% para la imagen
        
        \includegraphics[width=\paperwidth, keepaspectratio]{assets/#1}
    
        \begin{tikzpicture}[remember picture, overlay]
            \node[anchor=north west, text=white, opacity=1, font=\fontsize{60}{90}\selectfont\bfseries\sffamily, align=left] at (#5, #6) {#2};
            
            \node[anchor=south east, text=white, opacity=1, font=\fontsize{12}{18}\selectfont\sffamily, align=right] at (9.7, 3) {\textbf{\href{https://losdeldgiim.github.io/}{Los Del DGIIM}}};
            
            \node[anchor=south east, text=white, opacity=1, font=\fontsize{12}{15}\selectfont\sffamily, align=right] at (9.7, 1.8) {Doble Grado en Ingeniería Informática y Matemáticas\\Universidad de Granada};
        \end{tikzpicture}
    \end{figure}
    
    
    \restoregeometry        % Restaurar márgenes normales para las páginas subsiguientes
    \pagecolor{white}       % Restaurar el color de página
    
    
    \newpage
    \thispagestyle{empty}               % Sin encabezado ni pie de página
    \begin{tikzpicture}[remember picture, overlay]
        \node[anchor=south west, inner sep=3cm] at (current page.south west) {
            \begin{minipage}{0.5\paperwidth}
                \href{https://creativecommons.org/licenses/by-nc-nd/4.0/}{
                    \includegraphics[height=2cm]{assets/Licencia.png}
                }\vspace{1cm}\\
                Esta obra está bajo una
                \href{https://creativecommons.org/licenses/by-nc-nd/4.0/}{
                    Licencia Creative Commons Atribución-NoComercial-SinDerivadas 4.0 Internacional (CC BY-NC-ND 4.0).
                }\\
    
                Eres libre de compartir y redistribuir el contenido de esta obra en cualquier medio o formato, siempre y cuando des el crédito adecuado a los autores originales y no persigas fines comerciales. 
            \end{minipage}
        };
    \end{tikzpicture}
    
    
    
    % 1. Foto de fondo
    % 2. Título
    % 3. Encabezado Izquierdo
    % 4. Color de fondo
    % 5. Coord x del titulo
    % 6. Coord y del titulo
    % 7. Fecha


}


\newcommand{\portadaBook}[7]{

    % 1. Foto de fondo
    % 2. Título
    % 3. Encabezado Izquierdo
    % 4. Color de fondo
    % 5. Coord x del titulo
    % 6. Coord y del titulo
    % 7. Fecha

    % Personaliza el formato del título
    \pretitle{\begin{center}\bfseries\fontsize{42}{56}\selectfont}
    \posttitle{\par\end{center}\vspace{2em}}
    
    % Personaliza el formato del autor
    \preauthor{\begin{center}\Large}
    \postauthor{\par\end{center}\vfill}
    
    % Personaliza el formato de la fecha
    \predate{\begin{center}\huge}
    \postdate{\par\end{center}\vspace{2em}}
    
    \title{#2}
    \author{\href{https://losdeldgiim.github.io/}{Los Del DGIIM}}
    \date{Granada, #7}
    \maketitle
    
    \tableofcontents
}




\newcommand{\portadaArticle}[7]{

    % 1. Foto de fondo
    % 2. Título
    % 3. Encabezado Izquierdo
    % 4. Color de fondo
    % 5. Coord x del titulo
    % 6. Coord y del titulo
    % 7. Fecha

    % Personaliza el formato del título
    \pretitle{\begin{center}\bfseries\fontsize{42}{56}\selectfont}
    \posttitle{\par\end{center}\vspace{2em}}
    
    % Personaliza el formato del autor
    \preauthor{\begin{center}\Large}
    \postauthor{\par\end{center}\vspace{3em}}
    
    % Personaliza el formato de la fecha
    \predate{\begin{center}\huge}
    \postdate{\par\end{center}\vspace{5em}}
    
    \title{#2}
    \author{\href{https://losdeldgiim.github.io/}{Los Del DGIIM}}
    \date{Granada, #7}
    \thispagestyle{empty}               % Sin encabezado ni pie de página
    \maketitle
    \vfill
}
    \portadaExamen{ffccA4.jpg}{Álgebra III\\Examen III}{Álgebra III. Examen III}{MidnightBlue}{-8}{28}{2025}{José Juan Urrutia Milán}

    \begin{description}
        \item[Asignatura] Álgebra II.
        \item[Curso Académico] 2024-25.
        \item[Grado] Doble Grado en Ingeniería Informática y Matemáticas.
        \item[Grupo] Único.
        \item[Profesor] Aurora del Río Cabeza.
        \item[Descripción] Parcial II.
        \item[Fecha] 21 de mayo del 2025.
        \item[Duración] 2 horas.
    
    \end{description}
    \newpage

    \begin{ejercicio}[5 puntos]
        Responda \textbf{VERDADERO} o \textbf{FALSO} a cada una de las siguientes cuestiones, junto con una breve justificación de la respuesta.
        \begin{enumerate}
            \item Si $f:G\to G'$ es un homomorfismo de gupos y $N\subseteq G'$ es un subgrupo normal, entonces $f^\ast(N)\subseteq G$ es un subgrupo normal.
            \item Todo grupo $G$ actúa sobre sí mismo por conjugación, y en ese caso la acción es fiel.
            \item Si $H$ es un subgrupo normal en un grupo $G$, entonces $Z(H)$ es un subgrupo normal en $Z(G)$.
            \item Si todos los subgrupos de un grupo $G$ son normales entonces $G$ es abeliano.
            \item No hay puntos fijos bajo cualquier acción no trivial de $Q_2$ sobre un conjunto de 21 elementos.
            \item El grupo producto directo $S_5\times A_5$ tiene una única serie de composición de longitud 3.
            \item Si $H$ y $K$ son subgrupos de $G$, con $K$ normal y $H$ resoluble, entonces el cociente $HK/K$ es resoluble.
            \item Sea $G$ un grupo tal que $|G/Z(G)| = pq$, $p<q$, primos. Entonces $q\equiv 1 \mod p$.
            \item El centralizador $C_{S_4}((1\ 3)(2\ 4))$ es isomorfo a $D_4$.
            \item Todos los $p-$subgrupos de Sylow de $A_5$ son cíclicos.
        \end{enumerate}
    \end{ejercicio}

    \begin{ejercicio}[2.5 puntos]
        Considera el grupo $G=\langle a,b\mid a^8 = b^2 = 1, bab = a^3 \rangle $:
        \begin{enumerate}[label=(\alph*)]
            \item Calcula el orden de $ab$.
            \item ¿Es el subgrupo $H = \langle ab \rangle $ normal?
            \item Prueba que el subgrupo $K = \langle a^4 \rangle $ es normal.
            \item ¿Se puede dar un morfismo $f:G/K\to S_4$ tal que $f(aK)=(1\ 2\ 3\ 4)$ y $f(bK) = (2\ 4)$?
        \end{enumerate}
    \end{ejercicio}

    \begin{ejercicio}[2.5 puntos]
        Demuestra que un grupo de orden $5175$ tiene al menos dos subgrupos normales y que todo grupo de este orden es resoluble. ¿Tienen todos los grupos de este orden la misma longitud?
    \end{ejercicio}

    \newpage
    \setcounter{ejercicio}{0}
    \begin{ejercicio}[5 puntos]
        Contestamos a cada pregunta, razonando la respuesta:
        \begin{enumerate}
            \item Si $f:G\to G'$ es un homomorfismo de gupos y $N\subseteq G'$ es un subgrupo normal, entonces $f^\ast(N)\subseteq G$ es un subgrupo normal.

                \textbf{Verdadero.} Si $x\in G$ y $y\in f^\ast(N)$, entonces $f(y) \in N$. Para ver que $f^\ast(N)\lhd G$ queremos ver que $xyx^{-1}\in f^\ast(N)$, es decir, que $f(xyx^{-1})\in N$:
                \begin{equation*}
                    f(xyx^{-1}) = f(x)f(y)f(x^{-1}) = f(x)f(y){(f(x))}^{-1} \in N
                \end{equation*}
                Que pertenece a $N$ por ser $f(x),{(f(x))}^{-1}\in G'$ y $f(y)\in N$, siendo $N\lhd G'$. En definitiva, $xyx^{-1}\in f^\ast(N)$ para todo $x\in G$ y para todo $y\in f^\ast(N)$, por lo que $N\lhd G$.
            \item Todo grupo $G$ actúa sobre sí mismo por conjugación, y en ese caso la acción es fiel.

                \textbf{Falso.} Si consideramos la aplicación:
                \Func{ac}{G\times G}{G}{(g,h)}{ghg^{-1}}
                Veamos en primer lugar que es una acción:
                \begin{align*}
                    &\prescript{1}{}{x} = 1x1 = x \qquad \forall x\in G \\
                    &\prescript{gh}{}{x} = ghx{(gh)}^{-1} = ghxh^{-1}g^{-1} = \prescript{g}{}{hxh^{-1}} = \prescript{g}{}{(\prescript{h}{}{x})} \qquad \forall g,h,x\in G
                \end{align*}
                Si consideramos la representación por permutaciones:
                \Func{\Phi}{G}{Perm(G)}{g}{\Phi_g}
                donde cada $\Phi_g:G\to G$ viene dada por:
                \begin{equation*}
                    \Phi_g(h) = ghg^{-1} \qquad \forall h\in G
                \end{equation*}
                Veamos si $\ker(\Phi) = \{1\}$, en cuyo caso será una acción fiel:
                \begin{multline*}
                    \ker(\Phi) = \{g\in G \mid \Phi_g = id\} = \{g\in G \mid ghg^{-1} = h \quad \forall h\in G\} \\ = \{g\in G \mid gh = hg \quad \forall h\in G\} = Z(G)
                \end{multline*}
                Como hay grupos para los que $Z(G) \neq \{1\}$ (por ejemplo, cualquier grupo abeliano), en algunos casos la acción no será fiel.
            \item Si $H$ es un subgrupo normal en un grupo $G$, entonces $Z(H)$ es un subgrupo normal en $Z(G)$.
                
                \textbf{Falso.} Veamos un ejemplo en el que ni siquiera es subgrupo:\newline Sea $G = D_3 = \langle r,s\mid r^3 = s^2 = 1, sr = r^{-1}s \rangle $ y $H = \langle r \rangle $, tenemos que $H\lhd G$ y como $H$ es abeliano, $Z(H) = H$. Veamos ahora que $r\notin Z(G)$, ya que:
                \begin{align*}
                    r(sr) &= rr^{-1}s = s \\
                    (sr)r &= sr^2
                \end{align*}
                Y como $s \neq sr^{2}$, $r\notin Z(G)$, por lo que $Z(H)\nsubseteq Z(G)$.
            \item Si todos los subgrupos de un grupo $G$ son normales entonces $G$ es abeliano.

                \textbf{Falso.} Por ejemplo, todos los subgrupos de $Q_2 = \{\pm 1, \pm i,\pm j, \pm k\}$ son normales:
                \begin{figure}[H]
                    \centering
                    \begin{tikzpicture}[node distance=2cm]
                        \node (Q2) {$Q_2$};
                        \node (1) [below of=Q2, xshift=-2cm] {$\left\langle i\right\rangle$};
                        \node (2) [below of=Q2] {$\left\langle j\right\rangle$};
                        \node (3) [below of=Q2, xshift=2cm] {$\left\langle k\right\rangle$};
                        \node (4) [below of=2] {$\left\langle -1\right\rangle$};
                        \node (5) [below of=4] {$\{1\}$};

                        \draw (Q2) -- node[above left] {$2$} (1);
                        \draw (Q2) -- node[left] {$2$} (2);
                        \draw (Q2) -- node[above right] {$2$} (3);
                        \draw (1) -- node[below left] {$2$} (4);
                        \draw (2) -- node[left] {$2$} (4);
                        \draw (3) -- node[below right] {$2$} (4);
                        \draw (4) -- node[left] {$2$} (5);
                    \end{tikzpicture}
                    \caption{Diagrama de Hasse para los subgrupos del grupo de los cuaternios.}
                \end{figure}
                Ya que $[G:\langle i \rangle ] = [G:\langle j \rangle ] = [G:\langle k \rangle ] = 2$, $\{1\}\lhd Q_2$ y $\langle -1 \rangle \lhd G $ porque:
                \begin{align*}
                    i(-1)i^3 &= -i^4 = -1 \\
                    i^3(-1)i &= -i^4 = -1 \\
                    j(-1)j^3 &= -j^4 = -1 \\
                    j^3(-1)j &= -j^4 = -1 
                \end{align*}
                Y como $Q_2 = \langle i,j \rangle $, $\langle -1 \rangle \lhd Q_2$. Sin embargo, $Q_2$ no es abeliano, puesto que:
                \begin{align*}
                    ij &= k \\
                    ji &= -k
                \end{align*}
            \item No hay puntos fijos bajo cualquier acción no trivial de $Q_2$ sobre un conjunto de 21 elementos.
                
                \textbf{Falso.} En el ejercicio 12 de la relación de $p-$grupos vimos que si $G$ es un $p-$grupo que actúa sobre un conjunto finito $X$, entonces:
                \begin{equation*}
                    |X| \equiv |Fix(X)| \mod p
                \end{equation*}
                Como $Q_2$ es un $2-grupo$ (por ser $|Q_2| = 8 = 2^3$), si $X$ es un $Q_2-$conjunto con $|X| = 21$, tenemos que:
                \begin{equation*}
                    21 = |X| \equiv |Fix(X)| \mod 2
                \end{equation*}
                Por lo que $|Fix(X)|$ será impar y, en particular, $Fix(X)\neq \emptyset $, por lo que cualquier acción no trivial de $Q_2$ sobre cualquier conjunto de 21 elementos tendrá siempre al menos un punto fijo.
            \item El grupo producto directo $S_5\times A_5$ tiene una única serie de composición de longitud 3.

                \textbf{Falso.} Las dos siguientes series son dos series de composición distintas de $S_5\times A_5$ de longitud 3:
                \begin{align*}
                    &S_5\times A_5 \rhd A_5\times A_5 \rhd \{1\}\times A_5 \rhd \{1\} \\
                    &S_5\times A_5 \rhd S_5\times \{1\} \rhd A_5\times \{1\} \rhd \{1\}
                \end{align*}
            \item Si $H$ y $K$ son subgrupos de $G$, con $K$ normal y $H$ resoluble, entonces el cociente $HK/K$ es resoluble.

                \textbf{Verdadero.} Estamos en las condiciones de aplicar el Segundo Teorema de Isomorfía:
                \begin{figure}[H]
                    \centering
                    \begin{tikzpicture}
                        \node (G) {$G$};
                        \node (HK) [below=of G] {$HK$};
                        \node (H) [below=of HK, xshift=-1cm] {$H$};
                        \node (K) [below=of HK, xshift=1cm] {$K$};
                        \node (HcapK) [below=of HK, yshift=-1.5cm] {$H \cap K$};

                        \draw (G) -- (HK);
                        \draw (HK) -- (H);
                        \draw (HK) -- node[left] {$\rhd$} (K);
                        \draw (H) -- node[left] {$\rhd$} (HcapK);
                        \draw (K) -- (HcapK);
                        \draw (G) -- node[right] {$\rhd$} (K);
                    \end{tikzpicture}
                \end{figure}
                Obteniendo que $HK/K \cong H/(H\cap K)$. Como $H$ es resoluble, también lo será cualquier cociente suyo, por lo que $H/(H\cap K)$ será resoluble, y como esta propiedad se mantiene por isomorfismos, $HK/K$ será resoluble.
            \item Sea $G$ un grupo tal que $|G/Z(G)| = pq$, $p<q$, primos. Entonces $q\equiv 1 \mod p$.

                \textbf{Verdadero.} Por el Primer Teorema de Sylow, sabemos de la existencia de, al menos, un $p-$subgrupo de Sylow de $G/Z(G)$ de orden $p$ y de un $q-$subgrupo de Sylow de $G/Z(G)$ de orden $q$. Por el Segundo Teorema de Sylow, si denotamos por $n_t$ al número de $t-$subgrupos de Sylow de $G/Z(G)$:
                \begin{equation*}
                    \left.\begin{array}{l}
                        n_q \mid p \Longrightarrow n_p \leq p < q \\
                        n_q \equiv 1 \mod q
                    \end{array}\right\} \Longrightarrow n_q = 1
                \end{equation*}
                Solo hay un único $q-$subgrupo de Sylow de $G/Z(G)$: $P_q$, que será normal en $G/Z(G)$. Si calculamos el número de $p-$subgrupos de Sylow:
                \begin{equation*}
                    \left.\begin{array}{l}
                        n_p \mid q \\
                        n_p \equiv 1 \mod p
                    \end{array}\right\} \Longrightarrow n_p \in \{1,q\}
                \end{equation*}
                Si suponemos que $n_p = 1$, entonces también habrá un único $p-$subgrupo de Sylow de $G/Z(G)$: $P_p$, que también será normal en $G/Z(G)$. Bajo estas condiciones, un resultado visto en teoría nos dice que $G/Z(G)$ es producto directo interno de sus únicos subgrupos de Sylow:
                \begin{equation*}
                    G/Z(G) \cong P_p \times P_q
                \end{equation*}
                Sin embargo, como $|P_p| = p$, será $P_p \cong \mathbb{Z}_p$ y análogamente obtenemos que $P_q\cong \mathbb{Z}_q$, de donde:
                \begin{equation*}
                    G/Z(G) \cong P_p \times P_q \cong \mathbb{Z}_p \oplus \mathbb{Z}_q
                \end{equation*}
                Que será un grupo cíclico, por ser $\mathbb{Z}_p$ y $\mathbb{Z}_q$ cíclicos con $\mcd(p,q) = 1$, por lo que (por el ejercicio 4 de la relación de grupos directos) $G$ será abeliano, de donde $Z(G) = G$ y $p=q=1$, \underline{contradicción}, ya que $p$ y $q$ eran primos con $p < q$. La contradicción viene de suponer que $n_p = 1$, por lo que será $n_p = q$ y recordamos que:
                \begin{equation*}
                    q = n_p \equiv 1 \mod p
                \end{equation*}
            \item El centralizador $C_{S_4}((1\ 3)(2\ 4))$ es isomorfo a $D_4$.

                \textbf{Verdadero.} Si esribimos la definición del centralizador:
                \begin{equation*}
                    C_{S_4}((1\ 3)(2\ 4)) = \{\sigma\in S_4 \mid (\sigma(1)\ \sigma(3))(\sigma(2)\ \sigma(4)) = \sigma(1\ 3)(2\ 4)\sigma^{-1} = (1\ 3)(2\ 4)\}
                \end{equation*}
                Las únicas posibilidades para $\sigma$ son:
                \begin{equation*}
                    C_{S_4}((1\ 3)(2\ 4)) = \{1, (1\ 3)(2\ 4), (1\ 2)(3\ 4), (1\ 4)(2\ 3), (1\ 3), (2\ 4), (1\ 2\ 3\ 4), (1\ 4\ 3\ 2)\} 
                \end{equation*}
                Y sabemos que no hay más (porque $|S_4| = 24 = 2^3 \cdot 3$, tenemos ya 8 y si añadimos un elemento más ya tenemos que ir a $C_{S_4}((1\ 3)(2\ 4)) = S_4$, que sabemos que es falso). Si pensamos en:
                \begin{equation*}
                    D_4 = \langle r, s \mid r^4 = s^2 = 1, rs = sr^{3} \rangle 
                \end{equation*}
                Podemos pensar en los elementos $(1\ 2\ 3\ 4)$ (como $r$) y en $(1\ 3)$ (como $s$), que cumplen todas las relaciones de los generadores de $D_4$:
                \begin{align*}
                    &(1\ 2\ 3\ 4)^4 = 1 \\
                    &(1\ 3)^2 = 1 \\
                    &(1\ 3)(1\ 2\ 3\ 4)(1\ 3) = (3\ 2\ 1\ 4) = (1\ 2\ 3\ 4)^3
                \end{align*}
                Por el Teorema de Dyck, existe un homomorfismo $f:D_4\to C_{S_4}((1\ 3)(2\ 4))$. Además, $f$ será un isomorfismo por ser $C_{S_4}((1\ 3)(2\ 4)) = \langle (1\ 2\ 3\ 4)(1\ 3) \rangle $ y $|D_4| = |C_{S_4}((1\ 3)(2\ 4))| = 8$.
            \item Todos los $p-$subgrupos de Sylow de $A_5$ son cíclicos.

                \textbf{Falso.} Como $|A_5| = \nicefrac{5!}{2} = 5\cdot 4\cdot 3 = 2^2 \cdot 3 \cdot 5$, cualquier subgrupo de orden $4$ de $A_5$ será un $2-$subgrupo de Sylow suyo. En particular, lo será:
                \begin{equation*}
                    V = \{1, (1\ 3)(2\ 4), (1\ 2)(3\ 4), (1\ 4)(2\ 3)\}
                \end{equation*}
                Y $V$ no es cíclico, ya que todos sus elementos (salvo el 1) tienen orden 2.
        \end{enumerate}
    \end{ejercicio}

    \begin{ejercicio}[2.5 puntos]
        Considera el grupo $G=\langle a,b\mid a^8 = b^2 = 1, bab = a^3 \rangle $. \newline
        Podemos verlo también como:
        \begin{equation*}
            G = \langle a,b \mid a^8 = b^2 = 1, ba = a^3b \rangle  = \langle a,b\mid a^8 = b^2 = 1, ab = ba^3 \rangle 
        \end{equation*}
        \begin{enumerate}[label=(\alph*)]
            \item Calcula el orden de $ab$.

                Como $ab \neq 1$, buscamos el menor natural $n\in \mathbb{N}\setminus \{0\}$ de forma que ${(ab)}^{n} = 1$:
                \begin{align*}
                    {(ab)}^{2} &= abab = aa^3bb = a^4 \neq 1 \\
                    {(ab)}^{3} &= {(ab)}^{2}ab = a^4ab = a^5b \neq 1 \\
                    {(ab)}^{4} &= {(ab)}^{2}{(ab)}^{2} = a^4a^4 = a^8 = 1
                \end{align*}
                Por lo que será $O(ab) = 4$.
            \item ¿Es el subgrupo $H = \langle ab \rangle $ normal?

                Como $O(ab) = 4$ y usando el apartado anterior, $H= \langle ab \rangle = \{1, ab, a^4, a^5b\}$. Sin embargo, como:
                \begin{align*}
                    babb &= ba = a^3b \notin H
                \end{align*}
                $H$ no podrá ser normal en $G$.
            \item Prueba que el subgrupo $K = \langle a^4 \rangle $ es normal.

                Como $a^8 = 1$, tenemos que $K = \langle a^4 \rangle = \{1,a^4\} $.
                Basta probar que $xnx^{-1}\in K$ para todo $n\in K$ (para $n=1$ es trivial) y $x\in \{a,b,a^{-1},b^{-1}\}$, ya que $G = \langle a,b \rangle $:
                \begin{align*}
                    aa^4a^{-1} = aa^4a^7 = a^{12}= a^4 \in K \\
                    a^{-1}a^4a = a^7a^4a = a^{12} = a^4 \in K \\
                    ba^4b = ba^3ab = abab = a^4 \in K
                \end{align*}
                Por lo que $H\lhd G$.
            \item ¿Se puede dar un morfismo $f:G/K\to S_4$ tal que $f(aK)=(1\ 2\ 3\ 4)$ y $f(bK) = (2\ 4)$?

                Sí, como $(1\ 2\ 3\ 4)$ y $(2\ 4)$ cumplen las relaciones que aparecen en la presentación de $G$ (pensando en $(1\ 2\ 3\ 4)$ como $a$ y en $(2\ 4)$ como $b$):
                \begin{align*}
                    &{(1\ 2\ 3\ 4)}^{8} = {\left({(1\ 2\ 3\ 4)}^{4}\right)}^{2} = 1^2 = 1 \\
                    &{(2\ 4)}^{2} = 1 \\
                    &(2\ 4)(1\ 2\ 3\ 4){(2\ 4)}^{-1} = (1\ 4\ 3\ 2) = {(1\ 2\ 3\ 4)}^{3}
                \end{align*}
                Por el Teorema de Dyck, existe un homomorfismo $g:G\to S_4$ de forma que:
                \begin{align*}
                    g(a) &= (1\ 2\ 3\ 4) \\
                    g(b) &= (2\ 4)
                \end{align*}

                En vistas de aplicar la Propiedad Universal del grupo cociente, necesitamos que $K\lhd G$ y que $K=\{1,a^4\}\subset \ker(g)$. Comprobemos esto segundo:
                \begin{equation*}
                    g(1) = 1 \qquad g(a^4) = g(a)^4 = (1\ 2\ 3\ 4)^4 = 1
                \end{equation*}

                Por lo que $K\subset \ker(g)$. Por tanto, por la Propiedad Universal del grupo cociente: 
                \begin{figure}[H]
                    \centering
                    \shorthandoff{""}
                    \begin{tikzcd}
                    G \arrow[r, "p"] \arrow[rd, "g"'] & G/K \arrow[d, "f", dotted] \\
                                                      & S_4                              
                    \end{tikzcd}
                    \shorthandon{""}
                \end{figure}
                tenemos que existe un homomorfismo $f:G/K\to S_4$ que viene dado por:
                \begin{equation*}
                    f(xK) = g(x) \qquad \forall xK\in G/K
                \end{equation*}
                De esta forma:
                \begin{align*}
                    f(aK) &= g(a) = (1\ 2\ 3\ 4) \\
                    f(bK) &= g(b) = (2\ 4)
                \end{align*}
        \end{enumerate}
    \end{ejercicio}

    \begin{ejercicio}[2.5 puntos]
        Demuestra que un grupo de orden $5175$ tiene al menos dos subgrupos normales y que todo grupo de este orden es resoluble. ¿Tienen todos los grupos de este orden la misma longitud?\\

        \noindent
        Sea $G$ un grupo con $|G| = 5175 = 3^2 \cdot 5^2 \cdot 23$, por el Primer Teorema de Sylow sabemos de la existencia de $3-$subgrupos de Sylow, $5-$subgrupos de Sylow y $23-$subgrupos de Sylow de $G$. Si denotamos por $n_t$ a la cantidad de $t-$subgrupos de Sylow de $G$, aplicando el Segundo Teorema de Sylow, obtenemos que:
        \begin{equation*}
            \left.\begin{array}{r}
                n_{23} \mid 3^2 \cdot 5^2 = 225 \\
                n_{23} \equiv 1 \mod 23
            \end{array}\right\} \Longrightarrow n_{23} = 1
        \end{equation*}
        Por lo que solo habrá un único $23-$subgrupos de Sylow de $G$, $P_{23}$, que será normal en $G$ por ser el único $23-$subgrupo de Sylow. Además:
        \begin{equation*}
            \left.\begin{array}{r}
                n_5 \mid 3^2 \cdot 23 = 207 \\
                n_5 \equiv 1 \mod 5
            \end{array}\right\} \Longrightarrow n_5 = 1
        \end{equation*}
        Por lo que también habrá un único $5-$subgrupo de Sylow de $G$, $P_5$, que también será normal en $G$.\\

        \noindent
        Como $|P_5| = 5^2 = 25$, $P_5$ será resoluble. Además, como:
        \begin{equation*}
            |G/P_5| = |G|/|P_5| = 3^2\cdot 23
        \end{equation*}
        Tendremos que $G/P_5$ también será resoluble, de donde $G$ será resoluble.\\

        \noindent
        Finalmente, todos los grupos de este orden tendrán la misma longitud, ya que por ser $G$ resoluble, sus factores de composición serán grupos cíclicos de orden primo, y las únicas posibilidades a considerar como factores son cada uno de los grupos cíclicos de orden primo asociados a la descomposición de $5175$ en primos, es decir:
        \begin{equation*}
            \mathbb{Z}_3, \quad \mathbb{Z}_3, \quad \mathbb{Z}_5, \quad \mathbb{Z}_5, \quad \mathbb{Z}_{23}
        \end{equation*}
        Por lo que la longitud de una serie de composición de $G$ será siempre 5.
    \end{ejercicio}

\end{document}
