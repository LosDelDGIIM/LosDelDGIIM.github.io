\documentclass[12pt]{article}

% Idioma y codificación
\usepackage[spanish, es-tabla]{babel}       %es-tabla para que se titule "Tabla"
\usepackage[utf8]{inputenc}

% Márgenes
\usepackage[a4paper,top=3cm,bottom=2.5cm,left=3cm,right=3cm]{geometry}

% Comentarios de bloque
\usepackage{verbatim}

% Paquetes de links
\usepackage[hidelinks]{hyperref}    % Permite enlaces
\usepackage{url}                    % redirecciona a la web

% Más opciones para enumeraciones
\usepackage{enumitem}

% Personalizar la portada
\usepackage{titling}

% Paquetes de tablas
\usepackage{multirow}


%------------------------------------------------------------------------

%Paquetes de figuras
\usepackage{caption}
\usepackage{subcaption} % Figuras al lado de otras
\usepackage{float}      % Poner figuras en el sitio indicado H.


% Paquetes de imágenes
\usepackage{graphicx}       % Paquete para añadir imágenes
\usepackage{transparent}    % Para manejar la opacidad de las figuras

% Paquete para usar colores
\usepackage[dvipsnames]{xcolor}
\usepackage{pagecolor}      % Para cambiar el color de la página

% Habilita tamaños de fuente mayores
\usepackage{fix-cm}

% Para los gráficos
\usepackage{tikz}

% Para poder situar los nodos en los grafos
\usetikzlibrary{positioning}


%------------------------------------------------------------------------

% Paquetes de matemáticas
\usepackage{mathtools, amsfonts, amssymb, mathrsfs}
\usepackage[makeroom]{cancel}     % Simplificar tachando
\usepackage{polynom}    % Divisiones y Ruffini
\usepackage{units} % Para poner fracciones diagonales con \nicefrac

\usepackage{pgfplots}   %Representar funciones
\pgfplotsset{compat=1.18}  % Versión 1.18

\usepackage{tikz-cd}    % Para usar diagramas de composiciones
\usetikzlibrary{calc}   % Para usar cálculo de coordenadas en tikz

%Definición de teoremas, etc.
\usepackage{amsthm}
%\swapnumbers   % Intercambia la posición del texto y de la numeración

\theoremstyle{plain}

\makeatletter
\@ifclassloaded{article}{
  \newtheorem{teo}{Teorema}[section]
}{
  \newtheorem{teo}{Teorema}[chapter]  % Se resetea en cada chapter
}
\makeatother

\newtheorem{coro}{Corolario}[teo]           % Se resetea en cada teorema
\newtheorem{prop}[teo]{Proposición}         % Usa el mismo contador que teorema
\newtheorem{lema}[teo]{Lema}                % Usa el mismo contador que teorema

\theoremstyle{remark}
\newtheorem*{observacion}{Observación}

\theoremstyle{definition}

\makeatletter
\@ifclassloaded{article}{
  \newtheorem{definicion}{Definición} [section]     % Se resetea en cada chapter
}{
  \newtheorem{definicion}{Definición} [chapter]     % Se resetea en cada chapter
}
\makeatother

\newtheorem*{notacion}{Notación}
\newtheorem*{ejemplo}{Ejemplo}
\newtheorem*{ejercicio*}{Ejercicio}             % No numerado
\newtheorem{ejercicio}{Ejercicio} [section]     % Se resetea en cada section


% Modificar el formato de la numeración del teorema "ejercicio"
\renewcommand{\theejercicio}{%
  \ifnum\value{section}=0 % Si no se ha iniciado ninguna sección
    \arabic{ejercicio}% Solo mostrar el número de ejercicio
  \else
    \thesection.\arabic{ejercicio}% Mostrar número de sección y número de ejercicio
  \fi
}


% \renewcommand\qedsymbol{$\blacksquare$}         % Cambiar símbolo QED
%------------------------------------------------------------------------

% Paquetes para encabezados
\usepackage{fancyhdr}
\pagestyle{fancy}
\fancyhf{}

\newcommand{\helv}{ % Modificación tamaño de letra
\fontfamily{}\fontsize{12}{12}\selectfont}
\setlength{\headheight}{15pt} % Amplía el tamaño del índice


%\usepackage{lastpage}   % Referenciar última pag   \pageref{LastPage}
\fancyfoot[C]{\thepage}

%------------------------------------------------------------------------

% Conseguir que no ponga "Capítulo 1". Sino solo "1."
\makeatletter
\@ifclassloaded{book}{
  \renewcommand{\chaptermark}[1]{\markboth{\thechapter.\ #1}{}} % En el encabezado
    
  \renewcommand{\@makechapterhead}[1]{%
  \vspace*{50\p@}%
  {\parindent \z@ \raggedright \normalfont
    \ifnum \c@secnumdepth >\m@ne
      \huge\bfseries \thechapter.\hspace{1em}\ignorespaces
    \fi
    \interlinepenalty\@M
    \Huge \bfseries #1\par\nobreak
    \vskip 40\p@
  }}
}
\makeatother

%------------------------------------------------------------------------
% Paquetes de cógido
\usepackage{minted}
\renewcommand\listingscaption{Código fuente}

\usepackage{fancyvrb}
% Personaliza el tamaño de los números de línea
\renewcommand{\theFancyVerbLine}{\small\arabic{FancyVerbLine}}

% Estilo para C++
\newminted{cpp}{
    frame=lines,
    framesep=2mm,
    baselinestretch=1.2,
    linenos,
    escapeinside=||
}

% para minted
\definecolor{LightGray}{rgb}{0.95,0.95,0.92}
\setminted{
    linenos=true,
    stepnumber=5,
    numberfirstline=true,
    autogobble,
    breaklines=true,
    breakautoindent=true,
    breaksymbolleft=,
    breaksymbolright=,
    breaksymbolindentleft=0pt,
    breaksymbolindentright=0pt,
    breaksymbolsepleft=0pt,
    breaksymbolsepright=0pt,
    fontsize=\footnotesize,
    bgcolor=LightGray,
    numbersep=10pt
}


\usepackage{listings} % Para incluir código desde un archivo

\renewcommand\lstlistingname{Código Fuente}
\renewcommand\lstlistlistingname{Índice de Códigos Fuente}

% Definir colores
\definecolor{vscodepurple}{rgb}{0.5,0,0.5}
\definecolor{vscodeblue}{rgb}{0,0,0.8}
\definecolor{vscodegreen}{rgb}{0,0.5,0}
\definecolor{vscodegray}{rgb}{0.5,0.5,0.5}
\definecolor{vscodebackground}{rgb}{0.97,0.97,0.97}
\definecolor{vscodelightgray}{rgb}{0.9,0.9,0.9}

% Configuración para el estilo de C similar a VSCode
\lstdefinestyle{vscode_C}{
  backgroundcolor=\color{vscodebackground},
  commentstyle=\color{vscodegreen},
  keywordstyle=\color{vscodeblue},
  numberstyle=\tiny\color{vscodegray},
  stringstyle=\color{vscodepurple},
  basicstyle=\scriptsize\ttfamily,
  breakatwhitespace=false,
  breaklines=true,
  captionpos=b,
  keepspaces=true,
  numbers=left,
  numbersep=5pt,
  showspaces=false,
  showstringspaces=false,
  showtabs=false,
  tabsize=2,
  frame=tb,
  framerule=0pt,
  aboveskip=10pt,
  belowskip=10pt,
  xleftmargin=10pt,
  xrightmargin=10pt,
  framexleftmargin=10pt,
  framexrightmargin=10pt,
  framesep=0pt,
  rulecolor=\color{vscodelightgray},
  backgroundcolor=\color{vscodebackground},
}

%------------------------------------------------------------------------

% Comandos definidos
\newcommand{\bb}[1]{\mathbb{#1}}
\newcommand{\cc}[1]{\mathcal{#1}}

% I prefer the slanted \leq
\let\oldleq\leq % save them in case they're every wanted
\let\oldgeq\geq
\renewcommand{\leq}{\leqslant}
\renewcommand{\geq}{\geqslant}

% Si y solo si
\newcommand{\sii}{\iff}

% Letras griegas
\newcommand{\eps}{\epsilon}
\newcommand{\veps}{\varepsilon}
\newcommand{\lm}{\lambda}

\newcommand{\ol}{\overline}
\newcommand{\ul}{\underline}
\newcommand{\wt}{\widetilde}
\newcommand{\wh}{\widehat}

\let\oldvec\vec
\renewcommand{\vec}{\overrightarrow}

% Derivadas parciales
\newcommand{\del}[2]{\frac{\partial #1}{\partial #2}}
\newcommand{\Del}[3]{\frac{\partial^{#1} #2}{\partial #3^{#1}}}
\newcommand{\deld}[2]{\dfrac{\partial #1}{\partial #2}}
\newcommand{\Deld}[3]{\dfrac{\partial^{#1} #2}{\partial #3^{#1}}}


\newcommand{\AstIg}{\stackrel{(\ast)}{=}}
\newcommand{\Hop}{\stackrel{L'H\hat{o}pital}{=}}

\newcommand{\red}[1]{{\color{red}#1}} % Para integrales, destacar los cambios.

% Método de integración
\newcommand{\MetInt}[2]{
    \left[\begin{array}{c}
        #1 \\ #2
    \end{array}\right]
}

% Declarar aplicaciones
% 1. Nombre aplicación
% 2. Dominio
% 3. Codominio
% 4. Variable
% 5. Imagen de la variable
\newcommand{\Func}[5]{
    \begin{equation*}
        \begin{array}{rrll}
            #1:& #2 & \longrightarrow & #3\\
               & #4 & \longmapsto & #5
        \end{array}
    \end{equation*}
}

%------------------------------------------------------------------------


\DeclareMathOperator{\GL}{GL}
\begin{document}

    % 1. Foto de fondo
    % 2. Título
    % 3. Encabezado Izquierdo
    % 4. Color de fondo
    % 5. Coord x del titulo
    % 6. Coord y del titulo
    % 7. Fecha

    
    % 1. Foto de fondo
% 2. Título
% 3. Encabezado Izquierdo
% 4. Color de fondo
% 5. Coord x del titulo
% 6. Coord y del titulo
% 7. Fecha

\newcommand{\portada}[7]{

    \portadaBase{#1}{#2}{#3}{#4}{#5}{#6}{#7}
    \portadaBook{#1}{#2}{#3}{#4}{#5}{#6}{#7}
}

\newcommand{\portadaExamen}[7]{

    \portadaBase{#1}{#2}{#3}{#4}{#5}{#6}{#7}
    \portadaArticle{#1}{#2}{#3}{#4}{#5}{#6}{#7}
}




\newcommand{\portadaBase}[7]{

    % Tiene la portada principal y la licencia Creative Commons
    
    % 1. Foto de fondo
    % 2. Título
    % 3. Encabezado Izquierdo
    % 4. Color de fondo
    % 5. Coord x del titulo
    % 6. Coord y del titulo
    % 7. Fecha
    
    
    \thispagestyle{empty}               % Sin encabezado ni pie de página
    \newgeometry{margin=0cm}        % Márgenes nulos para la primera página
    
    
    % Encabezado
    \fancyhead[L]{\helv #3}
    \fancyhead[R]{\helv \nouppercase{\leftmark}}
    
    
    \pagecolor{#4}        % Color de fondo para la portada
    
    \begin{figure}[p]
        \centering
        \transparent{0.3}           % Opacidad del 30% para la imagen
        
        \includegraphics[width=\paperwidth, keepaspectratio]{assets/#1}
    
        \begin{tikzpicture}[remember picture, overlay]
            \node[anchor=north west, text=white, opacity=1, font=\fontsize{60}{90}\selectfont\bfseries\sffamily, align=left] at (#5, #6) {#2};
            
            \node[anchor=south east, text=white, opacity=1, font=\fontsize{12}{18}\selectfont\sffamily, align=right] at (9.7, 3) {\textbf{\href{https://losdeldgiim.github.io/}{Los Del DGIIM}}};
            
            \node[anchor=south east, text=white, opacity=1, font=\fontsize{12}{15}\selectfont\sffamily, align=right] at (9.7, 1.8) {Doble Grado en Ingeniería Informática y Matemáticas\\Universidad de Granada};
        \end{tikzpicture}
    \end{figure}
    
    
    \restoregeometry        % Restaurar márgenes normales para las páginas subsiguientes
    \pagecolor{white}       % Restaurar el color de página
    
    
    \newpage
    \thispagestyle{empty}               % Sin encabezado ni pie de página
    \begin{tikzpicture}[remember picture, overlay]
        \node[anchor=south west, inner sep=3cm] at (current page.south west) {
            \begin{minipage}{0.5\paperwidth}
                \href{https://creativecommons.org/licenses/by-nc-nd/4.0/}{
                    \includegraphics[height=2cm]{assets/Licencia.png}
                }\vspace{1cm}\\
                Esta obra está bajo una
                \href{https://creativecommons.org/licenses/by-nc-nd/4.0/}{
                    Licencia Creative Commons Atribución-NoComercial-SinDerivadas 4.0 Internacional (CC BY-NC-ND 4.0).
                }\\
    
                Eres libre de compartir y redistribuir el contenido de esta obra en cualquier medio o formato, siempre y cuando des el crédito adecuado a los autores originales y no persigas fines comerciales. 
            \end{minipage}
        };
    \end{tikzpicture}
    
    
    
    % 1. Foto de fondo
    % 2. Título
    % 3. Encabezado Izquierdo
    % 4. Color de fondo
    % 5. Coord x del titulo
    % 6. Coord y del titulo
    % 7. Fecha


}


\newcommand{\portadaBook}[7]{

    % 1. Foto de fondo
    % 2. Título
    % 3. Encabezado Izquierdo
    % 4. Color de fondo
    % 5. Coord x del titulo
    % 6. Coord y del titulo
    % 7. Fecha

    % Personaliza el formato del título
    \pretitle{\begin{center}\bfseries\fontsize{42}{56}\selectfont}
    \posttitle{\par\end{center}\vspace{2em}}
    
    % Personaliza el formato del autor
    \preauthor{\begin{center}\Large}
    \postauthor{\par\end{center}\vfill}
    
    % Personaliza el formato de la fecha
    \predate{\begin{center}\huge}
    \postdate{\par\end{center}\vspace{2em}}
    
    \title{#2}
    \author{\href{https://losdeldgiim.github.io/}{Los Del DGIIM}}
    \date{Granada, #7}
    \maketitle
    
    \tableofcontents
}




\newcommand{\portadaArticle}[7]{

    % 1. Foto de fondo
    % 2. Título
    % 3. Encabezado Izquierdo
    % 4. Color de fondo
    % 5. Coord x del titulo
    % 6. Coord y del titulo
    % 7. Fecha

    % Personaliza el formato del título
    \pretitle{\begin{center}\bfseries\fontsize{42}{56}\selectfont}
    \posttitle{\par\end{center}\vspace{2em}}
    
    % Personaliza el formato del autor
    \preauthor{\begin{center}\Large}
    \postauthor{\par\end{center}\vspace{3em}}
    
    % Personaliza el formato de la fecha
    \predate{\begin{center}\huge}
    \postdate{\par\end{center}\vspace{5em}}
    
    \title{#2}
    \author{\href{https://losdeldgiim.github.io/}{Los Del DGIIM}}
    \date{Granada, #7}
    \thispagestyle{empty}               % Sin encabezado ni pie de página
    \maketitle
    \vfill
}
    \portadaExamen{ffccA4.jpg}{Álgebra II\\Examen II}{Álgebra II. Examen II}{MidnightBlue}{-8}{28}{2025}{Arturo Olivares Martos}

    \begin{description}
        \item[Asignatura] Álgebra II.
        \item[Curso Académico] 2024-25.
        \item[Grado] Doble Grado en Ingeniería Informática y Matemáticas.
        \item[Grupo] Único.
        \item[Profesor] Aurora del Río Cabeza.
        \item[Descripción] Parcial I.
        \item[Fecha] 26 de marzo del 2025.
        \item[Duración] Dos partes de 45 minutos.
    
    \end{description}
    \newpage

    \begin{ejercicio}[5 puntos]
        Responda \textbf{VERDADERO} o \textbf{FALSO} a cada una de las siguientes cuestiones, junto con una breve justificación de la respuesta (todo grafo mencionado es simple: sin lazos ni lados paralelos).
        \begin{enumerate}
            \item Todo grafo tiene dos vértices con el mismo grado.
            \item Todo grafo bipartido completo tiene dos componentes conexas no vacías.
            \item Sea $f:G\to H$ un monomorfismo de grupos con $H$ un grupo abeliano, entonces $G$ es abeliano.
            \item Sean $H_1,H_2<G$ dos subgrupos con $|H_1|$ y $|H_2|$ primos relativos, entonces $H_1\cap H_2 = \{1\}$.
            \item Sean $H,K<G$ dos subgrupos, entonces $HK$ es un subgrupo de $G$.
            \item Si dos grupos tienen dos subgrupos propios isomorfos, entonces los grupos son isomorfos.
            \item En un grupo cíclico, todo elemento que no es el neutro es un generador.
            \item Sea $f:G\to G$ con $G$ un grupo y $x\in G$, la aplicación $f_x(y) = xyx^{-1}$ es un automorfismo.
            \item Sea $G$ un grupo y $H\subseteq G$, si $H$ es cerrado para la operación de $G$, entonces $H$ es un subgrupo de $G$.
            \item Si en un grupo $G$, $a=a^{-1}$, entonces $a$ es el elemento neutro del grupo.
        \end{enumerate}
    \end{ejercicio}

    \begin{ejercicio}[5 puntos]
        Fue el ejercicio 41 de la relación 1 (la de grafos).
    \end{ejercicio}

    \newpage
    \setcounter{ejercicio}{0}
    \begin{ejercicio}
        Contestamos a cada pregunta, razonando la respuesta:
        \begin{enumerate}
            \item Todo grafo tiene dos vértices con el mismo grado.

                \textbf{Verdadero.} Sea $G=(V,E)$ un grafo con $|V| =n$, los vértices de $G$ pueden tomar los grados del conjunto $GR(n) = \{0,1,\ldots,n-1\}$. Sin embargo, si un vértice se conecta con todos los demás (es decir, tiene grado $n-1$), entonces no podrá haber vértices de grado 0; y viceversa: si un vértice tiene grado 0 (no se conecta con ningún otro), entonces no podrá haber vértices de grado $n-1$. Por tanto, tendremos que:
                \begin{equation*}
                    |\{gr(u) \mid u\in V\}| < n
                \end{equation*}
                Por el Lema del Palomar, como cada vétice tiene que tener un grado (tenemos $n$ vértices y $n-1$ grados posibles), concluimos que $\exists u,v\in G$ con $gr(u) = gr(v)$.

            \item Todo grafo bipartido completo tiene dos componentes conexas no vacías.

                \textbf{Falso.} Por ejemplo, $K_{2,2}$ es un grafo bipartido completo pero solo tiene una componente conexa (ya que es conexo), tal y como vemos en la Figura~\ref{fig:k22}.

                \begin{figure}[H]
                \centering
                \begin{tikzpicture}[node distance=2cm]
                    % Nodos con posiciones relativas
                    \node[draw, fill=black, circle, minimum size=0.2cm] (A) {};
                    \node[draw, fill=black, circle, minimum size=0.2cm, right=of A] (B) {};
                    \node[draw, fill=black, circle, minimum size=0.2cm, below=of A] (C) {};
                    \node[draw, fill=black, circle, minimum size=0.2cm, below=of B] (D) {};
                    
                    % Aristas
                    \draw (A) -- (C);
                    \draw (A) -- (D);
                    \draw (B) -- (D);
                    \draw (C) -- (B);
                \end{tikzpicture}
                \caption{Grafo $K_{2,2}$.}
                \label{fig:k22}
            \end{figure}

            \item Sea $f:G\to H$ un monomorfismo de grupos con $H$ un grupo abeliano, entonces $G$ es abeliano.

                \textbf{Verdadero.} Sean $a,b\in G$:
                \begin{equation*}
                    f(ab) = f(a)f(b) = f(b)f(a) = f(ba) \Longrightarrow ab = ba
                \end{equation*}
            \item Sean $H_1,H_2<G$ dos subgrupos con $|H_1|$ y $|H_2|$ primos relativos, entonces $H_1\cap H_2 = \{1\}$.

                \textbf{Verdadero.} Veamos las dos inclusiones:
                \begin{description}
                    \item [$\supseteq)$] Como $H_1$ y $H_2$ son subgrupos, tenemos que $1\in H_1\cap H_2$.
                    \item [$\subseteq)$] Sea $x\in H_1\cap H_2$, entonces $x\in H_1$ y $x\in H_2$, con lo que:
                        \begin{equation*}
                            \left.\begin{array}{r}
                                O(x) \mid |H_1| \\
                                O(x) \mid |H_2| \\
                                mcd(|H_1|,|H_2|) = 1
                            \end{array}\right\} \Longrightarrow  O(x) = 1
                        \end{equation*}
                \end{description}
            \item Sean $H,K<G$ dos subgrupos, entonces $HK$ es un subgrupo de $G$.

                \textbf{Falso.} Por ejemplo, en $S_3$ consideramos:
                \begin{gather*}
                    H = \langle (1\ 2) \rangle = \{1, (1\ 2)\} < S_4 \\
                    K = \langle (1\ 3) \rangle  = \{1, (1\ 3)\} < S_4
                \end{gather*}
                Como:
                \begin{equation*}
                    (1\ 2)(1\ 3) = (1\ 3\ 2) 
                \end{equation*}
                Tenemos que:
                \begin{align*}
                    HK = \{1\cdot 1, 1\cdot (1\ 3), (1\ 2)\cdot 1, (1\ 2)(1\ 3) \} = \{1, (1\ 3),(1\ 2),  (1\ 3\ 2)\}
                \end{align*}
                Que claramente no es un subgrupo de $S_3$, ya que ${(1\ 3\ 2)}^{-1} = (1\ 2\ 3) \notin HK$.

            \item Si dos grupos tienen dos subgrupos propios isomorfos, entonces los grupos son isomorfos.

                \textbf{Falso.} Por ejemplo, si consideramos $V$, el grupo de Klein y $S_3$, tenemos que:
                \begin{gather*}
                    H = \langle (1\ 2)(3\ 4) \rangle = \{1, (1\ 2)(3\ 4)\} < V \\
                    K = \langle s \rangle = \{1, s\} < S_2
                \end{gather*}
                Son isomorfos, basta considerar $f:H\to G$ de forma que:
                \begin{align*}
                    1 &\longmapsto 1 \\
                    (1\ 2)(3\ 4) &\longmapsto s
                \end{align*}
                Para tener el isomorfismo, pero $V$ y $S_3$ no son isomorfos, ya que:
                \begin{equation*}
                    |V| = 4 \neq 6 = |S_3|
                \end{equation*}
            \item En un grupo cíclico, todo elemento que no es el neutro es un generador.

                \textbf{Falso.} Por ejemplo, en:
                \begin{equation*}
                    C_4 = \langle x \mid x^4 = 1 \rangle = \{1,x,x^2,x^3\}
                \end{equation*}
                Tenemos que $1\neq x^2\in C_4$, con:
                \begin{equation*}
                    \langle x^2 \rangle = \{1,x^2\} \neq C_4
                \end{equation*}
            \item Sea $f:G\to G$ con $G$ un grupo y $x\in G$, la aplicación $f_x(y) = xyx^{-1}$ es un automorfismo.

                \textbf{Verdadero.} Vemos que su dominio coincide con su codominio. Veamos que es un isomorfismo:
                    \begin{itemize}
                        \item Para ver que es un homomorfismo:
                            \begin{equation*}
                                f_x(yz) = xyzx^{-1} = xyx^{-1}xzx^{-1} = f_x(y)f_x(z) \qquad \forall y,z\in G
                            \end{equation*}
                        \item Para ver que es inyectiva, sean $y,z\in G$ de forma que:
                            \begin{equation*}
                                f_x(y) = xyx^{-1} = xzx^{-1} = f_x(z)
                            \end{equation*}
                            Entonces, aplicando dos veces la propiedad cancelativa, tenemos que:
                            \begin{equation*}
                                xyx^{-1} = xzx^{-1} \Longrightarrow xy = xz \Longrightarrow y = z
                            \end{equation*}
                        \item Para ver que es sobreyectiva, sea $y\in G$, tomamos:
                            \begin{equation*}
                                z = x^{-1}yx
                            \end{equation*}
                            Y tenemos que:
                            \begin{equation*}
                                f_x(z) = f_x(x^{-1}yx) = xx^{-1}yxx^{-1} = y
                            \end{equation*}
                    \end{itemize}
                    Concluimos que $f_x$ es un automorfismo.
            \item Sea $G$ un grupo y $H\subseteq G$, si $H$ es cerrado para la operación de $G$, entonces $H$ es un subgrupo de $G$.

                \textbf{Falso.} Por ejemplo, $(\mathbb{Z},+)$ es un grupo y $\mathbb{N}\subseteq \mathbb{Z}$ es un conjunto de forma que:
                \begin{equation*}
                    m+n \in \mathbb{N} \qquad \forall m,n\in \mathbb{N}
                \end{equation*}
                Es decir, que es cerrado para la suma de $\mathbb{Z}$. Sin embargo, $\mathbb{N}$ no es un grupo, por no ser cerrado para opuestos.

            \item Si en un grupo $G$, $a=a^{-1}$, entonces $a$ es el elemento neutro del grupo.
                
                \textbf{Falso.} Por ejemplo, en $S_3$ tenemos que $s\in S_3$ con $s^{-1} = s$ pero $s\neq 1$.
        \end{enumerate}
    \end{ejercicio}

\end{document}
