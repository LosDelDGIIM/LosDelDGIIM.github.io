\documentclass[12pt]{article}

% Idioma y codificación
\usepackage[spanish, es-tabla]{babel}       %es-tabla para que se titule "Tabla"
\usepackage[utf8]{inputenc}

% Márgenes
\usepackage[a4paper,top=3cm,bottom=2.5cm,left=3cm,right=3cm]{geometry}

% Comentarios de bloque
\usepackage{verbatim}

% Paquetes de links
\usepackage[hidelinks]{hyperref}    % Permite enlaces
\usepackage{url}                    % redirecciona a la web

% Más opciones para enumeraciones
\usepackage{enumitem}

% Personalizar la portada
\usepackage{titling}

% Paquetes de tablas
\usepackage{multirow}


%------------------------------------------------------------------------

%Paquetes de figuras
\usepackage{caption}
\usepackage{subcaption} % Figuras al lado de otras
\usepackage{float}      % Poner figuras en el sitio indicado H.


% Paquetes de imágenes
\usepackage{graphicx}       % Paquete para añadir imágenes
\usepackage{transparent}    % Para manejar la opacidad de las figuras

% Paquete para usar colores
\usepackage[dvipsnames]{xcolor}
\usepackage{pagecolor}      % Para cambiar el color de la página

% Habilita tamaños de fuente mayores
\usepackage{fix-cm}

% Para los gráficos
\usepackage{tikz}

% Para poder situar los nodos en los grafos
\usetikzlibrary{positioning}


%------------------------------------------------------------------------

% Paquetes de matemáticas
\usepackage{mathtools, amsfonts, amssymb, mathrsfs}
\usepackage[makeroom]{cancel}     % Simplificar tachando
\usepackage{polynom}    % Divisiones y Ruffini
\usepackage{units} % Para poner fracciones diagonales con \nicefrac

\usepackage{pgfplots}   %Representar funciones
\pgfplotsset{compat=1.18}  % Versión 1.18

\usepackage{tikz-cd}    % Para usar diagramas de composiciones
\usetikzlibrary{calc}   % Para usar cálculo de coordenadas en tikz

%Definición de teoremas, etc.
\usepackage{amsthm}
%\swapnumbers   % Intercambia la posición del texto y de la numeración

\theoremstyle{plain}

\makeatletter
\@ifclassloaded{article}{
  \newtheorem{teo}{Teorema}[section]
}{
  \newtheorem{teo}{Teorema}[chapter]  % Se resetea en cada chapter
}
\makeatother

\newtheorem{coro}{Corolario}[teo]           % Se resetea en cada teorema
\newtheorem{prop}[teo]{Proposición}         % Usa el mismo contador que teorema
\newtheorem{lema}[teo]{Lema}                % Usa el mismo contador que teorema

\theoremstyle{remark}
\newtheorem*{observacion}{Observación}

\theoremstyle{definition}

\makeatletter
\@ifclassloaded{article}{
  \newtheorem{definicion}{Definición} [section]     % Se resetea en cada chapter
}{
  \newtheorem{definicion}{Definición} [chapter]     % Se resetea en cada chapter
}
\makeatother

\newtheorem*{notacion}{Notación}
\newtheorem*{ejemplo}{Ejemplo}
\newtheorem*{ejercicio*}{Ejercicio}             % No numerado
\newtheorem{ejercicio}{Ejercicio} [section]     % Se resetea en cada section


% Modificar el formato de la numeración del teorema "ejercicio"
\renewcommand{\theejercicio}{%
  \ifnum\value{section}=0 % Si no se ha iniciado ninguna sección
    \arabic{ejercicio}% Solo mostrar el número de ejercicio
  \else
    \thesection.\arabic{ejercicio}% Mostrar número de sección y número de ejercicio
  \fi
}


% \renewcommand\qedsymbol{$\blacksquare$}         % Cambiar símbolo QED
%------------------------------------------------------------------------

% Paquetes para encabezados
\usepackage{fancyhdr}
\pagestyle{fancy}
\fancyhf{}

\newcommand{\helv}{ % Modificación tamaño de letra
\fontfamily{}\fontsize{12}{12}\selectfont}
\setlength{\headheight}{15pt} % Amplía el tamaño del índice


%\usepackage{lastpage}   % Referenciar última pag   \pageref{LastPage}
\fancyfoot[C]{\thepage}

%------------------------------------------------------------------------

% Conseguir que no ponga "Capítulo 1". Sino solo "1."
\makeatletter
\@ifclassloaded{book}{
  \renewcommand{\chaptermark}[1]{\markboth{\thechapter.\ #1}{}} % En el encabezado
    
  \renewcommand{\@makechapterhead}[1]{%
  \vspace*{50\p@}%
  {\parindent \z@ \raggedright \normalfont
    \ifnum \c@secnumdepth >\m@ne
      \huge\bfseries \thechapter.\hspace{1em}\ignorespaces
    \fi
    \interlinepenalty\@M
    \Huge \bfseries #1\par\nobreak
    \vskip 40\p@
  }}
}
\makeatother

%------------------------------------------------------------------------
% Paquetes de cógido
\usepackage{minted}
\renewcommand\listingscaption{Código fuente}

\usepackage{fancyvrb}
% Personaliza el tamaño de los números de línea
\renewcommand{\theFancyVerbLine}{\small\arabic{FancyVerbLine}}

% Estilo para C++
\newminted{cpp}{
    frame=lines,
    framesep=2mm,
    baselinestretch=1.2,
    linenos,
    escapeinside=||
}

% para minted
\definecolor{LightGray}{rgb}{0.95,0.95,0.92}
\setminted{
    linenos=true,
    stepnumber=5,
    numberfirstline=true,
    autogobble,
    breaklines=true,
    breakautoindent=true,
    breaksymbolleft=,
    breaksymbolright=,
    breaksymbolindentleft=0pt,
    breaksymbolindentright=0pt,
    breaksymbolsepleft=0pt,
    breaksymbolsepright=0pt,
    fontsize=\footnotesize,
    bgcolor=LightGray,
    numbersep=10pt
}


\usepackage{listings} % Para incluir código desde un archivo

\renewcommand\lstlistingname{Código Fuente}
\renewcommand\lstlistlistingname{Índice de Códigos Fuente}

% Definir colores
\definecolor{vscodepurple}{rgb}{0.5,0,0.5}
\definecolor{vscodeblue}{rgb}{0,0,0.8}
\definecolor{vscodegreen}{rgb}{0,0.5,0}
\definecolor{vscodegray}{rgb}{0.5,0.5,0.5}
\definecolor{vscodebackground}{rgb}{0.97,0.97,0.97}
\definecolor{vscodelightgray}{rgb}{0.9,0.9,0.9}

% Configuración para el estilo de C similar a VSCode
\lstdefinestyle{vscode_C}{
  backgroundcolor=\color{vscodebackground},
  commentstyle=\color{vscodegreen},
  keywordstyle=\color{vscodeblue},
  numberstyle=\tiny\color{vscodegray},
  stringstyle=\color{vscodepurple},
  basicstyle=\scriptsize\ttfamily,
  breakatwhitespace=false,
  breaklines=true,
  captionpos=b,
  keepspaces=true,
  numbers=left,
  numbersep=5pt,
  showspaces=false,
  showstringspaces=false,
  showtabs=false,
  tabsize=2,
  frame=tb,
  framerule=0pt,
  aboveskip=10pt,
  belowskip=10pt,
  xleftmargin=10pt,
  xrightmargin=10pt,
  framexleftmargin=10pt,
  framexrightmargin=10pt,
  framesep=0pt,
  rulecolor=\color{vscodelightgray},
  backgroundcolor=\color{vscodebackground},
}

%------------------------------------------------------------------------

% Comandos definidos
\newcommand{\bb}[1]{\mathbb{#1}}
\newcommand{\cc}[1]{\mathcal{#1}}

% I prefer the slanted \leq
\let\oldleq\leq % save them in case they're every wanted
\let\oldgeq\geq
\renewcommand{\leq}{\leqslant}
\renewcommand{\geq}{\geqslant}

% Si y solo si
\newcommand{\sii}{\iff}

% Letras griegas
\newcommand{\eps}{\epsilon}
\newcommand{\veps}{\varepsilon}
\newcommand{\lm}{\lambda}

\newcommand{\ol}{\overline}
\newcommand{\ul}{\underline}
\newcommand{\wt}{\widetilde}
\newcommand{\wh}{\widehat}

\let\oldvec\vec
\renewcommand{\vec}{\overrightarrow}

% Derivadas parciales
\newcommand{\del}[2]{\frac{\partial #1}{\partial #2}}
\newcommand{\Del}[3]{\frac{\partial^{#1} #2}{\partial #3^{#1}}}
\newcommand{\deld}[2]{\dfrac{\partial #1}{\partial #2}}
\newcommand{\Deld}[3]{\dfrac{\partial^{#1} #2}{\partial #3^{#1}}}


\newcommand{\AstIg}{\stackrel{(\ast)}{=}}
\newcommand{\Hop}{\stackrel{L'H\hat{o}pital}{=}}

\newcommand{\red}[1]{{\color{red}#1}} % Para integrales, destacar los cambios.

% Método de integración
\newcommand{\MetInt}[2]{
    \left[\begin{array}{c}
        #1 \\ #2
    \end{array}\right]
}

% Declarar aplicaciones
% 1. Nombre aplicación
% 2. Dominio
% 3. Codominio
% 4. Variable
% 5. Imagen de la variable
\newcommand{\Func}[5]{
    \begin{equation*}
        \begin{array}{rrll}
            #1:& #2 & \longrightarrow & #3\\
               & #4 & \longmapsto & #5
        \end{array}
    \end{equation*}
}

%------------------------------------------------------------------------


\newcommand{\Z}{\mathbb{Z}}
\newcommand{\N}{\mathbb{N}}
\newcommand{\Aut}{\text{Aut}}
\newcommand{\Fix}{\text{Fix}}
\newcommand{\Stab}{\text{Stab}}
\newcommand{\Orb}{\text{Orb}}

\begin{document}

    % 1. Foto de fondo
    % 2. Título
    % 3. Encabezado Izquierdo
    % 4. Color de fondo
    % 5. Coord x del titulo
    % 6. Coord y del titulo
    % 7. Fecha

    
    % 1. Foto de fondo
% 2. Título
% 3. Encabezado Izquierdo
% 4. Color de fondo
% 5. Coord x del titulo
% 6. Coord y del titulo
% 7. Fecha

\newcommand{\portada}[7]{

    \portadaBase{#1}{#2}{#3}{#4}{#5}{#6}{#7}
    \portadaBook{#1}{#2}{#3}{#4}{#5}{#6}{#7}
}

\newcommand{\portadaExamen}[7]{

    \portadaBase{#1}{#2}{#3}{#4}{#5}{#6}{#7}
    \portadaArticle{#1}{#2}{#3}{#4}{#5}{#6}{#7}
}




\newcommand{\portadaBase}[7]{

    % Tiene la portada principal y la licencia Creative Commons
    
    % 1. Foto de fondo
    % 2. Título
    % 3. Encabezado Izquierdo
    % 4. Color de fondo
    % 5. Coord x del titulo
    % 6. Coord y del titulo
    % 7. Fecha
    
    
    \thispagestyle{empty}               % Sin encabezado ni pie de página
    \newgeometry{margin=0cm}        % Márgenes nulos para la primera página
    
    
    % Encabezado
    \fancyhead[L]{\helv #3}
    \fancyhead[R]{\helv \nouppercase{\leftmark}}
    
    
    \pagecolor{#4}        % Color de fondo para la portada
    
    \begin{figure}[p]
        \centering
        \transparent{0.3}           % Opacidad del 30% para la imagen
        
        \includegraphics[width=\paperwidth, keepaspectratio]{assets/#1}
    
        \begin{tikzpicture}[remember picture, overlay]
            \node[anchor=north west, text=white, opacity=1, font=\fontsize{60}{90}\selectfont\bfseries\sffamily, align=left] at (#5, #6) {#2};
            
            \node[anchor=south east, text=white, opacity=1, font=\fontsize{12}{18}\selectfont\sffamily, align=right] at (9.7, 3) {\textbf{\href{https://losdeldgiim.github.io/}{Los Del DGIIM}}};
            
            \node[anchor=south east, text=white, opacity=1, font=\fontsize{12}{15}\selectfont\sffamily, align=right] at (9.7, 1.8) {Doble Grado en Ingeniería Informática y Matemáticas\\Universidad de Granada};
        \end{tikzpicture}
    \end{figure}
    
    
    \restoregeometry        % Restaurar márgenes normales para las páginas subsiguientes
    \pagecolor{white}       % Restaurar el color de página
    
    
    \newpage
    \thispagestyle{empty}               % Sin encabezado ni pie de página
    \begin{tikzpicture}[remember picture, overlay]
        \node[anchor=south west, inner sep=3cm] at (current page.south west) {
            \begin{minipage}{0.5\paperwidth}
                \href{https://creativecommons.org/licenses/by-nc-nd/4.0/}{
                    \includegraphics[height=2cm]{assets/Licencia.png}
                }\vspace{1cm}\\
                Esta obra está bajo una
                \href{https://creativecommons.org/licenses/by-nc-nd/4.0/}{
                    Licencia Creative Commons Atribución-NoComercial-SinDerivadas 4.0 Internacional (CC BY-NC-ND 4.0).
                }\\
    
                Eres libre de compartir y redistribuir el contenido de esta obra en cualquier medio o formato, siempre y cuando des el crédito adecuado a los autores originales y no persigas fines comerciales. 
            \end{minipage}
        };
    \end{tikzpicture}
    
    
    
    % 1. Foto de fondo
    % 2. Título
    % 3. Encabezado Izquierdo
    % 4. Color de fondo
    % 5. Coord x del titulo
    % 6. Coord y del titulo
    % 7. Fecha


}


\newcommand{\portadaBook}[7]{

    % 1. Foto de fondo
    % 2. Título
    % 3. Encabezado Izquierdo
    % 4. Color de fondo
    % 5. Coord x del titulo
    % 6. Coord y del titulo
    % 7. Fecha

    % Personaliza el formato del título
    \pretitle{\begin{center}\bfseries\fontsize{42}{56}\selectfont}
    \posttitle{\par\end{center}\vspace{2em}}
    
    % Personaliza el formato del autor
    \preauthor{\begin{center}\Large}
    \postauthor{\par\end{center}\vfill}
    
    % Personaliza el formato de la fecha
    \predate{\begin{center}\huge}
    \postdate{\par\end{center}\vspace{2em}}
    
    \title{#2}
    \author{\href{https://losdeldgiim.github.io/}{Los Del DGIIM}}
    \date{Granada, #7}
    \maketitle
    
    \tableofcontents
}




\newcommand{\portadaArticle}[7]{

    % 1. Foto de fondo
    % 2. Título
    % 3. Encabezado Izquierdo
    % 4. Color de fondo
    % 5. Coord x del titulo
    % 6. Coord y del titulo
    % 7. Fecha

    % Personaliza el formato del título
    \pretitle{\begin{center}\bfseries\fontsize{42}{56}\selectfont}
    \posttitle{\par\end{center}\vspace{2em}}
    
    % Personaliza el formato del autor
    \preauthor{\begin{center}\Large}
    \postauthor{\par\end{center}\vspace{3em}}
    
    % Personaliza el formato de la fecha
    \predate{\begin{center}\huge}
    \postdate{\par\end{center}\vspace{5em}}
    
    \title{#2}
    \author{\href{https://losdeldgiim.github.io/}{Los Del DGIIM}}
    \date{Granada, #7}
    \thispagestyle{empty}               % Sin encabezado ni pie de página
    \maketitle
    \vfill
}
    \portadaExamen{ffccA4.jpg}{Álgebra II\\Examen VII}{Álgebra II. Examen VII}{MidnightBlue}{-8}{28}{2025}{José Manuel Sánchez Varbas}

    \begin{description}
        \item[Asignatura] Álgebra II.
        \item[Curso Académico] 2024-25.
        \item[Grado] Doble Grado en Ingeniería Informática y Matemáticas.
        \item[Grupo] Único.
        \item[Profesor] Aurora Inés del Río Cabeza.
        \item[Descripción] Convocatoria Ordinaria.
        \item[Fecha] 18 de Junio de 2025.
        \item[Duración] 2 horas y 30 minutos.
    
    \end{description}
    \newpage


    % ------------------------------------
    
    \begin{ejercicio}[1 punto]
        Prueba, utilizando el algoritmo explicado en clase, que la sucesión
        $$3 \ge 3 \ge 2 \ge 2 \ge 2 \ge 2 \ge 2$$
        es gráfica y, utilizando dicho algoritmo, encuentra un grafo en que los grados de sus vértices sean los términos de esa sucesión. Prueba que el grafo es plano y que satisface el teorema de la característica de Euler.
    \end{ejercicio}

    \begin{ejercicio}[3 puntos]
        Se considera el grupo diédrico $$D_9 = \langle r, s \mid r^9 = s^2 = 1,\; rs = sr^{8} \rangle$$
        \begin{enumerate}[label=(\alph*)]
            \item Calcula el orden de cada uno de los elementos de $D_9$.
            \item Calcula el número de subgrupos de Sylow de $D_9$ y descríbelos.
            \item ¿Es $D_9$ resoluble? En caso afirmativo, calcula la longitud y los factores de composición de $D_9$.
            \item Demuestra que el centralizador de $r^i$, con $i = 1,\dots,8$, es el subgrupo $\langle r \rangle$.
            \item Calcula el centro de $D_9$.
            \item ¿Es normal el subgrupo $H = \langle s \rangle \subseteq D_9$? En caso contrario, calcula su normalizador en $D_9$.
            \item Consideremos el subgrupo $N = \langle r^3, s \rangle \subseteq D_9$, prueba que $N$ es isomorfo a $D_3$.
            \item ¿Es $N$ un subgrupo normal de $D_9$?
        \end{enumerate}
    \end{ejercicio}

    \begin{ejercicio}[2 puntos]
        Una presentación del grupo abeliano $A$ está dada por:
        $$
            A = \scalebox{1.2}{$\Biggl\langle$}x,y,z,t \;\scalebox{1.2}{$\Biggm|$}\;
            \begin{alignedat}{7}
            12&x& \quad &+& \quad 18&y& \quad &+& \quad  6&z& \phantom{6t} &\;=\; &0\\
            9&x& &+&  9&y& &+&  9&z& \quad + \quad 6t &\;=\; &0\\
            6&x& &+&  9&y& &+& 27&z& \quad + \quad 6t &\;=\; &0
            \end{alignedat}
            \scalebox{1.2}{$\Biggr\rangle$}
        $$
        \begin{enumerate}[label=(\alph*)]
            \item Calcula, de forma razonada, el rango (de la parte libre) y las descomposiciones cíclica y cíclica primaria de $A$ y, si $T(A)$ denota el grupo de torsión de $A$, determina su longitud y sus factores de composición.
            \item Clasifica, dando sus descomposiciones cíclica y cíclica primaria, todos los grupos abelianos cuyo orden sea el orden de $T(A)$ e identifica cuál de ellos es justamente $T(A)$.
        \end{enumerate}

    \end{ejercicio}

    \begin{ejercicio}[2 puntos]
        Sea $G$ un grupo de orden 28.
        \begin{enumerate}[label=(\alph*)]
            \item Razona que $G$ es el producto semidirecto $P_7 \rtimes P_2$ con $P_7$ y $P_2$ un $7$-subgrupo y un $2$-subgrupo de Sylow de $G$, respectivamente.
            \item Razona que si $G$ tiene un elemento de orden 4, entonces hay exactamente dos productos semidirectos (no isomorfos) $P_7 \rtimes P_2$: uno de ellos abeliano y el otro no abeliano, da una presentación de este último.
            \item Concluye que hay sólo 4 grupos de orden 28, dos abelianos y dos no abelianos. Da las descomposiciones cíclicas de los abelianos y presentaciones de los no abelianos.
        \end{enumerate}
    \end{ejercicio}

    \begin{ejercicio}[2 puntos]
        Demuestra el Teorema de Cauchy. Concluye que, si $G$ es finito, entonces $G$ es un $p$-grupo si y sólo si su orden es una potencia de $p$.
    \end{ejercicio}

    \newpage
    \setcounter{ejercicio}{0}

    \begin{ejercicio}
        Aplicamos el Algoritmo de Havel-Hakimi, y posteriormente construimos el grafo correspondiente, que se muestra en la Figura~\ref{fig:ej1}.

        \begin{table}[H]
            \centering
            \begin{tabular}{ccccccc|l}
                3 & 3 & 2 & 2 & 2 & 2 & 2 & Eliminamos el 3 y restamos uno a los 3 términos siguientes\\
                  & 2 & 1 & 1 & 2 & 2 & 2 & Reordenamos los términos\\
                  & 2 & 2 & 2 & 2 & 1 & 1 & Eliminamos el 2 y restamos uno a los 2 términos siguientes\\
                  &   & 1 & 1 & 2 & 1 & 1 & Reordenamos los términos\\
                  &   & 2 & 1 & 1 & 1 & 1 & Eliminamos el 2 y restamos uno a los 2 términos siguientes\\
                  &   &   & 0 & 0 & 1 & 1 & Reordenamos los términos\\
                  &   &   & 1 & 1 & 0 & 0 & Eliminamos el 1 y restamos uno a los 2 términos siguientes\\
                  &   &   &   & 0 & 0 & 0 & Obtenemos una sucesión gráfica
            \end{tabular}
        \end{table}
        Por el Teorema de Havel-Hakimi, la sucesión de partida $$3 \ge 3 \ge 2 \ge 2 \ge 2 \ge 2 \ge 2$$ es gráfica. Reconstruimos el grafo añadiendo un vértice y las correspondientes aristas en cada paso.

        \begin{figure}[H]
            \centering
            \begin{tikzpicture}
                % Nodos con posiciones relativas
                \node[draw, fill=black, circle, minimum size=0.1cm] (1) {};
                \node[draw, fill=black, circle, minimum size=0.1cm, right=of 1] (2) {};
                \node[draw, fill=black, circle, minimum size=0.1cm, right=of 2] (3) {};
            \end{tikzpicture}
            \label{fig:ej11}
        \end{figure}

        \begin{figure}[H]
            \centering
            \begin{tikzpicture}
                % Nodos con posiciones relativas
                \node[draw, fill=black, circle, minimum size=0.1cm] (1) {};
                \node[draw, fill=black, circle, minimum size=0.1cm, right=of 1] (2) {};
                \node[draw, fill=black, circle, minimum size=0.1cm, right=of 2] (3) {};
                \node[draw, fill=black, circle, minimum size=0.1cm, right=of 3] (4) {};

                % Aristas
                \draw (1) -- (2);
            \end{tikzpicture}
            \label{fig:ej12}
        \end{figure}

        \begin{figure}[H]
            \centering
            \begin{tikzpicture}
                % Nodos con posiciones relativas
                \node[draw, fill=black, circle, minimum size=0.1cm] (1) {};
                \node[draw, fill=black, circle, minimum size=0.1cm, right=of 1] (2) {};
                \node[draw, fill=black, circle, minimum size=0.1cm, right=of 2] (3) {};
                \node[draw, fill=black, circle, minimum size=0.1cm, right=of 3] (4) {};
                \node[draw, fill=black, circle, minimum size=0.1cm, above=of 2] (5) {};

                % Aristas
                \draw (1) -- (2);
                \draw (2) -- (3);
                \draw (3) -- (4);
                \draw (1) -- (5);
            \end{tikzpicture}
            \label{fig:ej13}
        \end{figure}

        \begin{figure}[H]
            \centering
            \begin{tikzpicture}
                % Nodos con posiciones relativas
                \node[draw, fill=black, circle, minimum size=0.1cm] (1) {};
                \node[draw, fill=black, circle, minimum size=0.1cm, right=of 1] (2) {};
                \node[draw, fill=black, circle, minimum size=0.1cm, right=of 2] (3) {};
                \node[draw, fill=black, circle, minimum size=0.1cm, right=of 3] (4) {};
                \node[draw, fill=black, circle, minimum size=0.1cm, above=of 2] (5) {};
                \node[draw, fill=black, circle, minimum size=0.1cm, above=of 3] (6) {};

                % Aristas
                \draw (1) -- (2);
                \draw (1) -- (5);
                \draw (2) -- (3);
                \draw (3) -- (4);
                \draw (5) -- (6);
            \end{tikzpicture}
            \label{fig:ej14}
        \end{figure}

        \begin{figure}[H]
            \centering
            \begin{tikzpicture}
                % Nodos con posiciones relativas
                \node[draw, fill=black, circle, minimum size=0.1cm] (1) {};
                \node[draw, fill=black, circle, minimum size=0.1cm, right=of 1] (2) {};
                \node[draw, fill=black, circle, minimum size=0.1cm, right=of 2] (3) {};
                \node[draw, fill=black, circle, minimum size=0.1cm, right=of 3] (4) {};
                \node[draw, fill=black, circle, minimum size=0.1cm, above=of 2] (5) {};
                \node[draw, fill=black, circle, minimum size=0.1cm, above=of 3] (6) {};
                \node[draw, fill=black, circle, minimum size=0.1cm, above=of 4] (7) {};

                % Aristas
                \draw (1) -- (2);
                \draw (1) -- (5);
                \draw (2) -- (3);
                \draw (3) -- (4);
                \draw (5) -- (6);
                \draw (6) -- (7);
                \draw (7) -- (4);
                \draw (1) to[bend right=90, looseness=0.5] (4);
            \end{tikzpicture}
            \caption{Grafo Correspondiente a la Sucesión Gráfica $3 \ge 3 \ge 2 \ge 2 \ge 2 \ge 2 \ge 2$}
            \label{fig:ej1}
        \end{figure}

        Como vemos, el grafo es plano, puesto que se da una representación en la que no se cruzan las aristas. Por último, verificar la característica de Euler significa que $v - a + c = 2$, es decir, la suma del número de vértices menos el número de caras más el número de aristas debe ser igual a $2$.
        Tenemos $v=7$ vértices, $a = 8$ aristas, y $c=3$ caras (dos interiores, y la cara exterior), por lo que $$v - a + c = 7 - 8 + 3 = 2$$ y se verifica la característica de Euler, como se pedía comprobar.
    \end{ejercicio}

    \begin{ejercicio}
        Lo resolvemos por apartados.
        \begin{enumerate}[label=(\alph*)]
            \item Calcula el orden de cada uno de los elementos de $D_9$.
            
            En primer lugar, por el Tº de Lagrange, vemos que $$x \in D_9 \Longrightarrow O(x) \mid |D_9| = 18 \Longrightarrow O(x) \in \{1,2,3,6,9,18\}$$

            Ahora, antes de hallar el orden de cada elemento, probamos que $$O(sr^{i}) = 2 \quad \forall i \in \{1, \ldots, 9\}$$
            
            Basta comprobar que $$(sr^i)^2 = (sr^i)(sr^i) \stackrel{(*)}{=} (r^{-i}s)(sr^i) = r^{-i}(ss)r^i = r^{-i}(s^2)r^i \stackrel{(**)}{=} r^{-i}r^{i} = r^{-i+i} = 1$$

            donde en $(*)$ se ha usado que $sr^i = r^{-i}s, \quad \forall i \in \{1, \ldots, 9\}$, y en $(**)$ se ha usado que $s^2 = 1$. Como $sr^i \neq 1 \quad \forall i \in \{1, \ldots, 9\}$, tenemos que $$O(sr^i) = 2 \quad \forall i \in \{1, \ldots, 9\}$$

            Con lo anterior, ya podemos calcular el orden de cada elemento de $D_9$:

            \begin{itemize}
                \item $O(x) \neq 18 \quad \forall x \in D_9$, ya que en caso contrario habría un elemento de orden $18$, lo cual implicaría que $D_9$ es cíclico, luego abeliano, cosa que sabemos que es falsa.
                \item $O(x) = 1 \iff x = 1$.
                \item $O(x) = 2 \iff x = sr^i \quad \forall i \in \{1, \ldots, 9\}$.
                \item $O(x) \in \{3,6,9\} \iff x = r^i \quad \forall i \in \{1, \ldots, 9\}$.
                \item Hallamos el orden de cada $r^{i}$: $$O(r^i) = \dfrac{9}{\text{mcd}(9,i)} \Longrightarrow \begin{cases}
                    O(r^i) = 9 \quad i \in \{1,2,4,5,7,8\} \\
                    O(r^i) = 3 \quad i \in \{3,6\}
                \end{cases}$$
                Vemos que no hay elementos de orden $6$ en $D_9$.
            \end{itemize}

            \item Calcula el número de subgrupos de Sylow de $D_9$ y descríbelos.
            
            Factorizamos $$|D_9| = 18 = 2 \cdot 3^2$$

            Por el Primer Teorema de Sylow, tenemos garantizada la existencia de $2$-subgrupos y $3$-subgrupos de Sylow en $D_9$. Vamos a obtenerlos:
            \begin{itemize}
                \item $3$-subgrupos de Sylow. Sea $n_3$ el número de $3$-subgrupos de Sylow en $D_9$. Por el Segundo Teorema de Sylow, tenemos que 
                $$n_3 \mid 2 \quad \land \quad n_3 \equiv 1 \mod 3 \Longrightarrow n_3 = 1$$
                Hay un único $3$-subgrupo de Sylow, pongamos $P_3$, que además, por ser el único $3$-subgrupo de Sylow, será normal, $P_3 \vartriangleleft D_9$, y $|P_3| = 3^2 = 9$, luego, por el Corolario del Teorema de Burnside, será abeliano (tiene orden cuadrado de un primo),
                y este debe ser $P_3 \cong \langle r \rangle \cong C_9$ (los únicos $9$ elementos de $D_9$ que forman un grupo abeliano son los generados por las potencias de $r$. Además, $P_3$ es un $3$-grupo, y como hemos visto en el apartado a) los únicos elementos de órdenes potencias de $3$, 
                $(1,3,9)$, en $D_9$ son las rotaciones $r^{i} \quad i \in \{1, \ldots, 9\}$)
                \item $2$-subgrupos de Sylow. Sea $n_2$ el número de $2$-subgrupos de Sylow en $D_9$. Por el Segundo Teorema de Sylow, tenemos que
                $$n_2 \mid 9 \quad \land \quad n_2 \equiv 1 \mod 2 \Longrightarrow n_2 \in \{1,3,9\}$$
                Ahora, si $P_2$ es un $2$-subgrupo de Sylow, entonces $|P_2| = 2$. En particular $P_2$ es cíclico, luego abeliano. Recordando que en el apartado a) habíamos probado que $O(x) = 2 \iff x = sr^i \quad \forall i \in \{1, \ldots, 9\}$, y teniendo en cuenta que
                $sr^i \neq sr^j \quad \forall i \neq j \hspace{0.2cm} i,j \in \{1, \ldots, 9\}$, entonces, cada $\langle sr^i \rangle$ es un $2$-subgrupo de Sylow distinto, y por tanto, tenemos que $n_2 = 9$, y no puede haber más $2$-subgrupos de Sylow,
                por lo recién encontrado con el Segundo Teorema de Sylow. Así pues, hay nueve $2$-subgrupos de Sylow, y cada uno es de la forma $\langle sr^i \rangle \quad i \in \{1, \ldots, 9\}$. 
            \end{itemize}

            \item ¿Es $D_9$ resoluble? En caso afirmativo, calcula la longitud y los factores de composición de $D_9$.
            
            En el apartado b) hemos visto que $P_3$ era el único $3$-subgrupo de Sylow, y que, por tanto, era normal en $D_9$, $P_3 \vartriangleleft D_9$. Entonces $D_9$ sí es resoluble, ya que encontramos la siguiente serie de composición
            $$D_9 \vartriangleright P_3 \vartriangleright \{1\}$$
            que alcanza el $\{1\}$ y todos sus factores, $D_9 / P_3 \cong C_2$ y $P_3 / \{1\} \cong C_9$, son cíclicos, y por tanto, abelianos. Por definición, $D_9$ es resoluble, y vemos que la longitud de la serie es $2$, y sus factores de composición son $C_2$ y $C_9$. Aunque no se pida, 
            sus índices son, respectivamente, $\nicefrac{|D_9|}{|P_3|} = 2$ y $\nicefrac{|P_3|}{|\{1\}|} = 9$, por lo que la serie quedaría como sigue:

            $$D_9 \stackrel{2}{\vartriangleright} P_3 \stackrel{9}{\vartriangleright} \{1\}$$

            \newpage

            \item Demuestra que el centralizador de $r^i$, con $i = 1,\dots,8$, es el subgrupo $\langle r \rangle$.
            
            Por definición, el centralizador en $D_9$ de $r^i$ es, para cada $i \in \{1, \ldots, 8\}$, el siguiente $$C_{D_9}(r^i) = \{x \in D_9 : xr^i = r^ix\}$$

            Vamos a probar que $C_{D_9}(r^i) = \langle r \rangle$ por doble contenido.

            \begin{itemize}
                \item[$\supseteq)$] Sea $x \in \langle r \rangle \Longrightarrow x = r^j$ para cierto $j \in \{1, \ldots, 9\}$. Entonces $$r^{j}r^{i} = r^{j+i} = r^{i+j} = r^{i}r^{j} \Longrightarrow x \in C_{D_9}(r^i)$$
                \item[$\subseteq)$] Sea $x \in C_{D_9}(r^i)$ para un $i \in \{1, \ldots, 8\}$ fijo. Distinguimos en función de la forma de los elementos de $D_9$. Si $x \in D_9$, puede ser $x = r^k$ con $k \in \{1, \ldots, 9\}$, o $x = sr^k$, con $k \in \{1, \ldots, 9\}$.
                \begin{itemize}
                    \item Si $x = r^k$ con $k \in \{1, \ldots, 9\}$, entonces $x \in \langle r \rangle$, y ya hemos terminado este caso.
                    \item Vamos ahora a llegar a contradicción con que $x \in C_{D_9}(r^i)$ si $x = sr^k$, con $k \in \{1, \ldots, 9\}$. Por reducción al absurdo, supongamos que $\exists x = sr^k \in C_{D_9}(r^i)$. Entonces por estar $x$ en el centralizador se verifica que $(sr^k)(r^i) = (r^i)(sr^k)$.
                    Recordando que $sr^j = r^{-j} s \quad \forall j \in \{1, \ldots, 9\}$, vemos que $$\begin{cases}
                        (sr^k)r^i = sr^{k+i} = r^{-(k+i)}s \\
                        r^i(sr^k) = r^i r^{-k} s = r^{(i-k)}s
                    \end{cases}$$
                    Igualando ambas expresiones, se obtiene $$r^{-(k+i)} s = r^{(i-k)}s \stackrel{(1)}{\iff} r^{-(k+i)} = r^{i-k} \stackrel{(2)}{\iff} r^{(i-k) + (k+i)} = 1 \iff r^{2i} = 1$$
                    donde en $(1)$ se ha multiplicado por la derecha en ambos miembros de la igualdad por $s$, y se ha usado que $s^2 = 1$, y en $(2)$ se ha hecho lo mismo, pero con $r^{k+i}$. \\
                    
                    Ahora, como $O(r) = 9$, necesariamente debe ser $9 \mid 2i$, para algún $i \in \{1, \ldots, 8\}$, pero como además $\mcd(9,2) = 1$, debe ser $9 \mid i$, con $i \in \{1, \ldots, 8\}$. Contradicción que viene de suponer que $\exists x = sr^k \in C_{D_9}(r^i)$.
                \end{itemize}  
            \end{itemize}

            Como $i$ era fijo, pero arbitrario, concluimos que $C_{D_9}(r^i) = \langle r \rangle$ para cada $i \in \{1, \ldots, 8\}$.

            \newpage

            \item Calcula el centro de $D_9$.
            
            Por definición, el centro de $D_9$ es $$Z(D_9) = \{x \in D_9 : xy = yx \quad \forall y \in D_9\}$$ 

            Sabemos que $Z(D_9) < D_9$ (además es normal), luego $\{1\} \subseteq Z(D_9)$. Veamos ahora que $Z(D_9) \subseteq \{1\}$, siguiendo un razonamiento similar al que se ha hecho en el apartado d), pero encontrando contraejemplos.

            \begin{itemize}
                \item Si $x = sr^k$ con $k \in \{1, \ldots, 9\}$, entonces, si $x \in Z(D_9)$, en particular $x$ conmutaría con $r$, pero:
                $$(sr^k)r = sr^{k+1} = r^{-(k+1)}s$$
                $$r(sr^k) = r(r^{-k}s) = r^{1-k}s$$
                Igual que en el apartado d) $$r^{-(k+1)}s = r^{1-k}s \iff r^{-(k+1)} = r^{1-k} \iff r^{(1-k)+(k+1)} = 1 \iff r^{2} = 1$$
                Cosa que sabemos que es falsa. Por tanto, $x \notin Z(D_9)$.
                \item Si $x = r^k$ con $k \in \{1, \ldots, 9\}$, entonces, si $x \in Z(D_9)$, en particular, $x$ conmutaría con $s$, luego $r^k s = s r^k$, pero, por otro lado, sabemos que $sr^k = r^{-k}s$, por lo que, igualando ambas expresiones de $sr^k$, llegamos a que:
                $$r^{k}s = sr^k = r^{-k}s \iff r^k = r^{-k} \iff r^{2k} = 1$$
                Aplicando el mismo argumento que en el apartado d), $9 \mid 2k \quad \land \quad \mcd(9,2) = 1 \Longrightarrow 9|k$, con $k \in \{1, \ldots, 9\}$. Necesariamente entonces $k=9$, y $x = r^{9} = 1$. 
            \end{itemize}

            Concluimos entonces que $Z(D_9) = \{1\}$. Nótese que si $D_9$ fuera un $p$-grupo, por el Teorema de Burnside, $Z(D_9)$ sería no trivial.

            \item ¿Es normal el subgrupo $H = \langle s \rangle \subseteq D_9$? En caso contrario, calcula su normalizador en $D_9$.
            
            Usando la caracterización de los subgrupos normales, $H \vartriangleleft D_9 \iff \forall h \in H, \forall x \in D_9 \Longrightarrow xhx^{-1} \in H$. Tomando $s = h \in H$, y $r = x \in D_9$, vemos que
            $$rsr^{-1} = rsr^{8} = rr^{-8}s = r^{-7}s = sr^{7} \notin H = \langle s \rangle = \{1,s\}$$

            Por lo tanto, $H$ no es normal en $D_9$. Por definición, su normalizador es $$N_{D_9}(H) \stackrel{def}{=} \{x \in D_9 : xH = Hx\} = \{x \in D_9 : xHx^{-1} = H\}$$

            Sabemos que el normalizador se caracteriza como el mayor subgrupo (en este caso de $D_9$) en el que $H$ es normal. Así, $x \in N_{D_9}(H) \iff \forall h \in H, \exists h' \in H : xhx^{-1} = h'$ \\

            Basta comprobar la conjugación para los dos generadores de $D_9$, $r$ y $s$. Como la conjugación preserva órdenes, y $O(s) = 2$, entonces $xsx^{-1} = s \iff xs = sx$. Distinguimos entre los dos posibles tipos de elementos de $D_9$, como venimos haciendo hasta ahora.

            \begin{itemize}
                \item Si $x = r^k$ con $k \in \{1, \ldots, 8\}$, entonces $$r^k s = s r^k = r^{-k} s \Longrightarrow r^{2k} = 1 \Longrightarrow 9 \mid 2k \Longrightarrow 9 \mid k$$ lo cual es imposible. Así, $r^{k} \notin N_{D_9}(H)$ para cada $k \in \{1, \ldots, 8\}$.
                \item Si $x = sr^k$ con $k \in \{1, \ldots, 9\}$, entonces $$(sr^k)s = s(sr^k) \iff (sr^k)s = r^k \iff (r^{-k}s)s = r^k \iff r^{2k} = 1$$
                lo que implica que $9 \mid 2k \Longrightarrow 9 \mid k \Longrightarrow k=9$, de donde $x = sr^9 = s$. Así, $s \in N_{D_9}(H)$, y vemos que, junto con el neutro, no hay más elementos en $N_{D_9}(H)$, por lo que concluimos que $N_{D_9}(H) = H = \langle s \rangle = \{1,s\}$
             \end{itemize}

            \item Consideremos el subgrupo $N = \langle r^3, s \rangle \subseteq D_9$, prueba que $N$ es isomorfo a $D_3$.
            
            Vemos que los elementos de $N$ serán de la forma $r^{3i} \quad i \in \{1, \ldots, 9\}$, o bien $sr^{3i} \quad i \in \{1, \ldots, 9\}$. Como por el apartado a) sabemos que $O(r^3) = 3$, entonces $|\langle r^3 \rangle| = 3$, y $\langle r^3 \rangle = \{1, r^3, r^6\}$. Ahora, añadiéndole
            el generador $s$, tenemos los otros $3$ elementos, y podemos dar $N$ por extensión: $N = \{1, r^3, r^6, s, sr^3, sr^6\}$ Como $sr^{i} \neq sr^j \quad \forall i\neq j, \hspace{0.2cm} i,j \in \{1, \ldots, 9\}$, también será $sr^{3i} \neq sr^{3j}$ puesto que $3i \neq 3j \iff i \neq j$, 
            entonces los tres elementos $s, sr^3, sr^6$ son distintos, por lo que $|N| = 6 = 2 \cdot 3 = |D_3|$. \\

            Buscamos aplicar el Teorema de Dyck, para poder encontrar un homomorfismo $f : N \to D_3$. Para ello, veamos que $r^3$ y $s$ verifican las relaciones de $r$ y $s$ en $D_3$, que son $r^3 = 1$, $s^2 = 1$ y $rs = sr^{-1} = sr^2$.

            $$(r^3)^3 = 1^3 = 1$$
            $$s^2 = 1$$
            $$r^3s = s(r^3)^2 = sr^6 = r^{-6}s \iff r^3 = r^{-6} = r^{-3} \cdot r^{-3} = (r^{3})^{-1} \cdot (r^{3})^{-1} = 1$$
            Donde hemos usado que $O(r^3) = 1$ en $D_3$ para la primera y tercera relación. Así pues, pues el Teorema de Dyck, existe un homomorfismo $f : N \to D_3$ de tal manera que $f(r^3) = r \in D_3$ y $f(s) = s \in D_3$. Como $D_3 = \langle r,s \rangle$, $f$ es un epimorfismo y como
            $|N| = |D_3| = 6$, entonces $f$ será un isomorfismo, luego $$N \cong D_3$$

            \item ¿Es $N$ un subgrupo normal de $D_9$?
            
            Usando nuevamente la caracterización de subgrupo normal de un grupo, se tiene que $N \vartriangleleft D_9 \iff \forall n \in N, \forall x \in D_9 \Longrightarrow xnx^{-1} \in N$. Tomando $s = n \in N$, y $r = x \in D_9$, vemos que
            $$rsr^{-1} = (sr^{8})r^{-1} = sr^{7} \notin N$$
            porque $sr^i \in N \iff i \in \{3,6,9\}$. Como $i=7$, $rsr^{-1} \notin N$, luego $N$ no es un subgrupo normal de $D_9$.
        \end{enumerate}
    \end{ejercicio}

    \newpage

    \begin{ejercicio}
        \begin{enumerate}[label=(\alph*)]
            \item Calcula, de forma razonada, el rango (de la parte libre) y las descomposiciones cíclica y cíclica primaria de $A$ y, si $T(A)$ denota el grupo de torsión de $A$, determina su longitud y sus factores de composición.
            
            La matriz de relaciones es 
            $M = \begin{pmatrix}
                12 & 18 & 6 & 0 \\
                9 & 9 & 9 & 6 \\
                6 & 9 & 27 & 6
            \end{pmatrix} = (m_{ij})_{\substack{1\le i\le 3 \\\\ 1\le j\le 4}}$ \\\\
            Como $\mcd((m_{ij})_{\substack{1\le i\le 3 \\\\ 1\le j\le 4}}) = \mcd(12,18,6,0,9,27) = 3 $, buscamos que $$m_{11} = 3 \quad \land \quad m_{1j} = 0 \quad \land \quad m_{i1} = 0, \quad i,j>1$$

            $$\begin{pmatrix}
                12 & 18 & 6 & 0 \\
                9 & 9 & 9 & 6 \\
                6 & 9 & 27 & 6
            \end{pmatrix} 
            \xrightarrow{F_1 \leftrightarrow F_3} 
            \begin{pmatrix}
                6 & 9 & 27 & 6 \\
                9 & 9 & 9 & 6 \\
                12 & 18 & 6 & 0
            \end{pmatrix} 
            \underset{F_3-2F_1}{\xrightarrow{F_2-F_1}}
            \begin{pmatrix}
                6 & 9 & 27 & 6 \\
                3 & 0 & -18 & 0 \\
                0 & 0 & -48 & -12
            \end{pmatrix}
            \xrightarrow{F_1 \leftrightarrow F_2}$$
            $$\begin{pmatrix}
                3 & 0 & -18 & 0 \\
                6 & 9 & 27 & 6 \\
                0 & 0 & -48 & -12
            \end{pmatrix}
            \xrightarrow{C_3 + 6C_1}
            \begin{pmatrix}
                3 & 0 & 0 & 0 \\
                6 & 9 & 63 & 6 \\
                0 & 0 & -48 & -12
            \end{pmatrix}
            \xrightarrow{F_2 - 2F_1}
            \begin{pmatrix}
                3 & 0 & 0 & 0 \\
                0 & 9 & 63 & 6 \\
                0 & 0 & -48 & -12
            \end{pmatrix}
            $$
            Ahora como $\mcd((m_{ij})_{\substack{2\le i\le 3 \\\\ 2\le j\le 4}}) = \mcd(9,63,6,0,-48,-12) = 3$, buscamos que $$m_{22} = 3 \quad \land \quad m_{2j} = 0 \quad \land \quad m_{i2} = 0, \quad i,j>2$$

            $$\begin{pmatrix}
                3 & 0 & 0 & 0 \\
                0 & 9 & 63 & 6 \\
                0 & 0 & -48 & -12
            \end{pmatrix}
            \xrightarrow{C_2 - C_4}
            \begin{pmatrix}
                3 & 0 & 0 & 0 \\
                0 & 3 & 63 & 6 \\
                0 & 12 & -48 & -12
            \end{pmatrix} 
            \xrightarrow{F_3 - 4F_2}
            \begin{pmatrix}
                3 & 0 & 0 & 0 \\
                0 & 3 & 63 & 6 \\
                0 & 0 & -300 & -36
            \end{pmatrix}
            \underset{C_3-21C_2}{\xrightarrow{C_4-2C_2}}$$
            $$
            \begin{pmatrix}
                3 & 0 & 0 & 0 \\
                0 & 3 & 0 & 0 \\
                0 & 0 & -300 & -36
            \end{pmatrix}
            $$

            Por último, vemos que $\mcd((m_{ij})_{\substack{3\le i\le 3 \\\\ 3\le j\le 4}}) = \mcd(-300, -36) = 12$, buscamos que $m_{33} = 12$ y que $m_{34} = 0$

            $$
            \begin{pmatrix}
                3 & 0 & 0 & 0 \\
                0 & 3 & 0 & 0 \\
                0 & 0 & -300 & -36
            \end{pmatrix}
            \xrightarrow{-F_3}
            \begin{pmatrix}
                3 & 0 & 0 & 0 \\
                0 & 3 & 0 & 0 \\
                0 & 0 & 300 & 36
            \end{pmatrix}
            \xrightarrow{C_3 - 8C_4}
            \begin{pmatrix}
                3 & 0 & 0 & 0 \\
                0 & 3 & 0 & 0 \\
                0 & 0 & 12 & 36
            \end{pmatrix}
            \xrightarrow{C_4 - 3C_3}$$
            $$
            \begin{pmatrix}
                3 & 0 & 0 & 0 \\
                0 & 3 & 0 & 0 \\
                0 & 0 & 12 & 0
            \end{pmatrix}
            $$

            De esta manera, como tenemos $4$ generadores, y el rango de la matriz es $r=3$, tendremos que $$A \cong \Z \oplus \Z_3 \oplus \Z_3 \oplus \Z_{12}$$
            Esta es su descomposición cíclica. El rango de la parte libre es $1$, y su grupo de torsión es:
            $$T(A) \cong \Z_3 \oplus \Z_3 \oplus \Z_{12} \cong \Z_3 \oplus \Z_3 \oplus \left( \Z_3 \oplus \Z_4 \right) \cong \Z_4 \oplus \Z_9$$
            con $|T(A)| = 3 \cdot 3 \cdot 12 = 108 = 2^{3} \cdot 3^3$ \\

            La descomposición cíclica primaria de $A$ es $A \cong \Z \oplus \Z_4 \oplus \Z_9$ \\

            Ahora, hallamos la longitud y los factores de composición de $T(A)$. Para ello, usaremos que, dado un grupo, una serie normal es de composición si, y sólo si, sus factores son grupos simples, y que un grupo es abeliano y simple si, y sólo si, es de orden primo.

            Definimos:
            \begin{enumerate}
                \item $G_0 = T(A) \cong \Z_4 \oplus \Z_3 \oplus \Z_3 \oplus \Z_3$
                \item $G_1 = \Z_4 \oplus \Z_3 \oplus \Z_3$
                \item $G_2 = \Z_4 \oplus \Z_3$
                \item $G_3 = \Z_4$
                \item $G_4 = 2\Z_4 \cong \Z_2$
                \item $G_5 = \{0\}$
            \end{enumerate}

            Cada cociente $G_{k-1} / G_{k}$ es abeliano y simple para $k \in \{1,2,3\}$, porque es de orden primo $3$, y para $k \in \{4,5\}$, igual, pero siendo el orden primo $2$.
            
            Por construcción, tenemos la siguiente serie de composición:

            $$G_0 \stackrel{3}{\vartriangleright} G_1 \stackrel{3}{\vartriangleright} G_2 \stackrel{3}{\vartriangleright} G_3 \stackrel{2}{\vartriangleright} G_4 \stackrel{2}{\vartriangleright} G_5 = \{0\}$$

            De esta manera, la longitud de composición es $5$.
                
            \item Clasifica, dando sus descomposiciones cíclica y cíclica primaria, todos los grupos abelianos cuyo orden sea el orden de $T(A)$ e identifica cuál de ellos es justamente $T(A)$.
            
            Sea $G$ un grupo abeliano, verificando que $|G| = |T(A)| = 108 = 2^2 \cdot 3^3$. Clasificamos:
            \begin{equation*}
                \begin{array}{c|c|c|c}
                    \text{Divisores elementales} & \text{desc. cíclica primaria} & \text{factores invariantes} & \text{desc. cíclica} \\
                    \hline
                    \{2^2, 3^3\} & C_4 \oplus C_{27} & d_1= 2^2\cdot 3^3 = 108 & C_{108} \\
                    \hline
                    \{2^2,3,3^2\} & C_4\oplus C_3\oplus C_9 & \begin{array}{c}
                            d_1 = 2^2 \cdot 3^2 = 36 \\
                            d_2 = 3
                    \end{array}& C_{36}\oplus C_3 \\
                    \hline
                    \{2^2, 3, 3, 3\} & C_4\oplus C_3\oplus C_3\oplus C_3 & \begin{array}{c}
                            d_1 = 2^2 \cdot 3 = 12 \\
                            d_2 = 3 \\
                            d_3 = 3
                    \end{array}& C_{12} \oplus C_3 \oplus C_3 \\
                    \hline
                    \{2, 2, 3^3\} & C_2\oplus C_2\oplus C_{27} & \begin{array}{c}
                            d_1 = 2 \cdot 3^3 = 54 \\
                            d_2 = 2 \\
                    \end{array}& C_{54} \oplus C_2 \\
                    \hline
                    \{2, 2, 3, 3^2\} & C_2\oplus C_2\oplus C_{3} \oplus C_{9} & \begin{array}{c}
                            d_1 = 2 \cdot 3^2 = 18 \\
                            d_2 = 2 \cdot 3 = 6 \\
                    \end{array}& C_{18} \oplus C_6 \\
                    \hline
                    \{2, 2, 3, 3, 3\} & C_2\oplus C_2\oplus C_{3} \oplus C_{3} \oplus C_3 & \begin{array}{c}
                            d_1 = 2 \cdot 3 = 6 \\
                            d_2 = 2 \cdot 3 = 6 \\
                            d_3 = 3
                    \end{array}& C_{6} \oplus C_6 \oplus C_3 \\
                \end{array}
            \end{equation*}

            Por medio de los factores invariantes, vemos que el grupo que coincide con $T(A)$ es el que tiene factores invariantes $(3,3,12)$, que sería el resultante de considerar como divisores elementales, según la clasificación anterior, $\{2^2, 3, 3, 3\}$, es decir:

            $$T(A) \cong C_{12} \oplus C_3 \oplus C_3 \cong C_3 \oplus C_3 \oplus C_{12}$$

        \end{enumerate}
    \end{ejercicio}

    \begin{ejercicio}
        \begin{enumerate}[label=(\alph*)]
            \item Razona que $G$ es el producto semidirecto $P_7 \rtimes P_2$ con $P_7$ y $P_2$ un $7$-subgrupo y un $2$-subgrupo de Sylow de $G$, respectivamente.
            
            Como $|G| = 28 = 2^2 \cdot 7$, por el Primer Teorema de Sylow tenemos garantizada la existencia de $2$-subgrupos de Sylow y 7-subgrupos de Sylow en $G$. Sea $n_7$ el número de $7$-subgrupos de Sylow que hay en $G$. Entonces, por el Segundo Teorema de Sylow, tenemos que
            $$n_7 \mid 4 \quad \land \quad n_7 \equiv 1 \mod 7 \Longrightarrow n_7 = 1$$
            Hay un único $7$-subgrupo de Sylow, pongamos $P_7$, que además, por ser el único $7$-subgrupo de Sylow, será normal, $P_7 \vartriangleleft G$ con $|P_7| = 7$. \\

            Ahora, consideramos un $2$-subgrupo de Sylow, $P_2$. Buscamos aplicar la Propiedad Universal del Producto Semidirecto. Primero, veamos por reducción al absurdo que $P_2 \cap P_7 = \{1\}$. 
            Si $P_2 \cap P_7 \neq \{1\} \Longrightarrow \exists x \in P_2 \cap P_7$ tal que $x \neq 1$. Entonces, por el Teorema de Lagrange, $O(x) \mid |P_2| = 4 \quad \land \quad O(x) \mid |P_7| = 7$, por lo que $O(x) \in \{1,2,4\} \cap \{1,7\} \Longrightarrow O(x) = 1 \iff x = 1$, 
            contradicción que viene de suponer que $P_2 \cap P_7 \neq \{1\}$. Así, debe ser $P_2 \cap P_7 = \{1\}$. Ahora, por el Segundo Teorema de Isomorfía:

            $$\dfrac{P_2}{P_2 \cap P_7} \cong \dfrac{P_2 P_7}{P_7} \Longrightarrow |P_2P_7| = \dfrac{|P_2| |P_7|}{|P_2 \cap P_7|} = 4 \cdot 7 = 28 = |G|$$

            de esta manera, $P_2 P_7 = G$, y como $P_7 \vartriangleleft G$, entonces $G$ es el producto semidirecto de $P_7$ y $P_2$, es decir, $G \cong P_7 \rtimes P_2$.

            \item Razona que si $G$ tiene un elemento de orden 4, entonces hay exactamente dos productos semidirectos (no isomorfos) $P_7 \rtimes P_2$: uno de ellos abeliano y el otro no abeliano, da una presentación de este último.
            
            Sea $y \in G$, con $O(y) = 4$. Entonces, por el Teorema de Lagrange, $O(y) \mid |G|$, y vemos que $|\langle y \rangle| = 4 = |P_2|$, por lo que $ \langle y \rangle$ es un $2$-subgrupo de Sylow, y además $ \langle y \rangle \cong C_4$.
            Consideramos $\langle y \rangle \cong P_2$, así como $P_7 \cong C_7$. Ahora, buscamos homomorfismos $\theta : C_4 \to \Aut(C_7)$. Primero veamos los generadores de $C_7$. Sea $a \in C_7$ tal que $\langle a \rangle \cong C_7$. Como $\varphi(7) = 6$, habrá 6 generadores, que son aquellos
            coprimos con $7$, es decir:

            $$\langle a^{i} \rangle \quad i \in \{1, \ldots, 6\}$$ Para cada $i \in \{1, \ldots, 6\}$, definimos entonces el automorfismo \Func{\varphi_i}{\langle a \rangle}{\langle a \rangle}{a}{a^i}

            Ahora, hay que ver cuáles de entre todos estos son automorfismos válidos. Trabajamos con $i \in \{1, \ldots, 6\}$ fijo, y vemos que dado que $O(y) = 4 \Longrightarrow O(\theta(y)) \mid 4 \Longrightarrow O(\theta(y)) \in \{1, 2, 4\}$ \\

            Veamos que $O(\theta(y)) \neq 4$. Como $\theta(y) \in \Aut(C_7) \Longrightarrow O(\theta(y)) \mid |\Aut(C_7)| = 6$. Por tanto, tenemos que $O(\theta(y)) \mid 4 \quad \land \quad O(\theta(y)) \mid 6 \Longrightarrow O(\theta(y)) \mid \mcd(4,6) = 2$. Así, $O(\theta(y)) \in \{1,2\}$. Distinguimos
            en función de estos dos valores.

            \begin{itemize}
                \item Si $O(\theta(y)) = 1$, entonces $\theta(y) = \varphi_1$, y por tanto, $\theta$ es el homomorfismo trivial, y $G \cong P_7 \times P_2$, que es abeliano, por ser producto directo de abelianos ($P_7 \cong C_7$, luego cíclico, luego abeliano, y $P_2$ tiene orden cuadrado de un primo,
                luego, por el Corolario del Teorema de Burnside, es abeliano).
                \item Si $O(\theta(y)) = 2$, hay que comprobar cuáles son los automorfismos que tienen orden $2$:
                $$(\varphi_i \circ \varphi_i)(a) = \varphi_i(\varphi_i(a)) = \varphi_i(a^i) = (a^i)^i = a^{i \cdot i} = a^{i^2}$$
                Ahora, $O(\varphi_i) = 2 \iff a^{i^2} = a \iff i^2 \equiv 1 \mod 7 \iff i \in \{1,6\}$. Como $O(\theta(y)) \neq 1 \Longrightarrow i \neq 1$, luego $i=6$ necesariamente, y el único automorfismo válido sería $\varphi_6$. Por tanto:
                $$yay^{-1} = \varphi_6(a) = a^6 = a^{-1} \Longrightarrow ya = a^{-1}y$$
                
                Así pues, obtenemos que $G$ es isomorfo a un grupo no abeliano con la siguiente presentación $G \cong \langle a,y : a^7 = 1,  y^4 = 1, ya = a^{-1}y \rangle$.
            \end{itemize}

            \newpage

            \item Concluye que hay sólo 4 grupos de orden 28, dos abelianos y dos no abelianos. Da las descomposiciones cíclicas de los abelianos y presentaciones de los no abelianos.
            
            Como $|G| = 28 = 2^2 \cdot 7$, y hemos visto en el apartado a) que $G \cong P_7 \rtimes P_2$, los posibles isomorfismos a grupos abelianos o no abelianos vendrán determinados por el homomorfismo que se tome, y de a quién sea isomorfo $P_2$. Como $|P_2| = 4$, hay dos posibles isomorfismos
            que son $P_2 \cong C_4$ o $P_2 \cong C_2 \times C_2 \cong V^{abs}$ (nótese que ambos isomorfismos son distintos, puesto que $C_4$ es cíclico, y $V^{abs}$ no). También hemos visto en el apartado b) que los únicos homomorfismos válidos eran aquellos que llegaban a los automorfismos
            $\varphi_1$ (el trivial), o $\varphi_6$ (el no trivial). Con esto, distinguimos, en primer lugar, si se toma la acción trivial o no, y, en caso de no tomarse, del isomorfismo que se tome de $P_2$. Recordemos que dado un grupo $G$ y un conjunto no vacío $X$, dar una acción de $G$ sobre $X$
            equivale a dar un homomorfismo de grupos de $G$ en $\text{Perm}(X)$ (este homomorfismo es la representación de $G$ por permutaciones).
            
            \begin{itemize}
                \item Si se toma la acción trivial, entonces $G \cong P_7 \times P_2$, y como $P_7 \cong C_7$ y $P_2 \cong C_4$, entonces $G$ es abeliano, por ser producto directo de grupos cíclicos, luego abelianos. Clasificamos como en el apartado b) del ejercicio 3:
                
                \begin{equation*}
                    \begin{array}{c|c|c|c}
                        \text{Divisores elementales} & \text{desc. cíclica primaria} & \text{factores invariantes} & \text{desc. cíclica} \\
                        \hline
                        \{2^2, 7\} & C_4 \oplus C_{7} & d_1 = 2^2\cdot 7 = 28 & C_{28} \\
                        \hline
                        \{2, 2, 7\} & C_2\oplus C_2\oplus C_7 & \begin{array}{c}
                                d_1 = 2 \cdot 7 = 14 \\
                                d_2 = 2
                        \end{array}& C_{14}\oplus C_2
                    \end{array}
                \end{equation*}
                \item En caso de no tomar la acción trivial, distinguimos según los isomorfismos de $P_2$ que se tomen, $P_2 \cong C_4$ o $P_2 \cong V^{abs}$.
                \begin{itemize}
                    \item Si $P_2 \cong C_4$, entonces, por el apartado b), $G$ es isomorfo a un grupo no abeliano con la presentación siguiente $$G \cong \langle a,y : a^7 = 1,  y^4 = 1, ya = a^{-1}y \rangle$$
                    \item Si $P_2 \cong C_2 \times C_2 \cong V^{abs}$, recordamos que $V^{abs}$ se puede presentar como $V^{abs} = \langle b,c : b^2 = c^2 = 1, bc = cb \rangle = \{1, b, c, bc\}$
                    Sabemos que para $n \geqslant 3$, si $\theta : C_2 \to \Aut(C_n)$, dado por $\theta(y)(x) = x^{-1} \quad \forall y \in C_2$, $\forall x \in C_n$, entonces $C_n \rtimes_{\theta} C_2 \cong D_n$.
                    Este homomorfismo coincide con el homomorfismo $\theta: P_2 \to \Aut(C_7)$, que llega al automorfismo $\varphi_6$, y entonces $G \cong P_7 \rtimes P_2 \cong C_7 \rtimes_{\theta} (C_2 \times C_2) \cong (C_7 \rtimes_{\theta} C_2) \times C_2 \cong 
                    D_7 \times C_2$, de tal manera que $G$ no es abeliano, por no serlo $D_7$. \\

                    La presentación de este último grupo no abeliano será entonces $$G \cong \langle a,b,c : a^7 = 1, b^2 = c^2 = 1, bab^{-1} = a^{-1}, algo \rangle$$

                    Como $\theta : V^{abs} \to \Aut(C_7)$, entonces $\text{Im}(\theta) = \langle \varphi_6 \rangle$, y ya hemos visto que $O(\varphi_6) = 2$ en el apartado b), por lo que $|\langle \varphi_6 \rangle| = 2$, y entonces, por el Primer Teorema de Isomorfía:
                    
                    $$\dfrac{V^{abs}}{\ker \theta} \cong \text{Im}(\theta) \cong C_2 \Longrightarrow |\ker \theta| = \dfrac{|V^{abs}|}{|C_2|} = \dfrac{4}{2} = 2$$

                    Ahora, hay tomando $b,c \in V^{abs}$ los dos generadores, hay cuatro posibilidades para el par $(\theta(b), \theta(c))$:

                    \begin{itemize}
                        \item Si $\theta(b) = \theta(c) = \varphi_1$, estamos en el caso de la acción trivial, ya estudiado.
                        \item Si $\theta(b) = \varphi_6$, entonces $\text{Im}(\theta) = \langle \theta(b) \rangle$, y entonces $\theta(c) = \varphi_1$, luego $\ker(\theta) = \langle c \rangle$. En tal caso,
                        como $$\ker \theta = \{h \in V^{abs} : \theta(h) = \varphi_1\} = \{h \in V^{abs} : \theta(h)(k) = k \quad \forall k \in P_7\} = $$$$ = \{h \in V^{abs} : hkh^{-1} = k \quad \forall k \in P_7\}$$
                        por estar $c \in \ker \theta$, tomando $a \in P_7$, se tiene que $cac^{-1} = a \iff ca = ac \iff [c,a] = 1$. Por la presentación de $V^{abs}$, ya sabemos que $bc = cb$ luego $[b,c] = [c,b] = 1$, y obtenemos la presentación final del último grupo no abeliano.
                        $$G \cong \langle a,b,c : a^7=1,b^2 = c^2=1, bab^{-1} = a^{-1}, [c,a] = [c,b] = 1 \rangle$$
                        \item Si $\theta(c) = \varphi_6$, entonces $\text{Im}(\theta) = \langle \theta(c) \rangle$, y entonces $\theta(b) = \varphi_1$, luego $\ker(\theta) = \langle b \rangle$, y el resultado que se obtendría sería el mismo que en el caso anterior cambiando los papeles de $b$ con los de $c$.
                        \item Si $\theta(b) = \theta(c) = \varphi_6$, entonces 
                        $$\theta(bc) = \theta(b)\theta(c) = \varphi_6^2 = \varphi_1$$
                        donde en la primera igualdad se ha usado que $\theta$ es un homomorfismo, y en la tercera que $O(\varphi_6) = 2$. Entonces, $\ker(\theta) = \langle bc \rangle$, y $\text{Im}(\theta) = \langle \varphi_6 \rangle$ Definiendo $c' = bc$ y $b' = b$ como nuevos generadores,
                        comprobamos que cumplen todas las relaciones: $O(b') = 2$, pues $b' = b$, y $O(b) = 2$. $O(c') = 2$, pues $$(c') ^ 2 = (c')(c') = (bc)(bc) = bccb = bc^2b = b^2 = 1$$ donde en la tercera igualdad se ha usado la relación de $V^{abs}$. \\

                        Como $\theta(b') = \theta(b) = \varphi_6$, se sigue cumpliendo que $b'a(b')^{-1} = a^{-1}$. \\ 

                        También se verifica que $$c'a(c')^{-1} = (bc)a(bc)^{-1} = (bc)a(c^{-1}b^{-1}) = b(cac^{-1})b^{-1} = b(a^{-1})b^{-1} = $$ $$ = (bab^{-1})^{-1} = (a^{-1})^{-1} = a$$

                        Vemos por último que $[b',c'] = 1$, pues $$b'c' = b(bc) = c$$ $$c'b' = (bc)b = (cb)b = c$$ donde en la penúltima igualdad se ha usado la relación de $V^{abs}$. En definitiva, la presentación resultante es la misma pero con otros nombres:
                        
                        $$G \cong \langle a,b',c' : a^7=1, (b')^2 = (c')^2=1, (b')a(b')^{-1} = a^{-1}, [c',a] = [c',b'] = 1 \rangle$$
                    \end{itemize}
                \end{itemize}
            \end{itemize}
        \end{enumerate}
    \end{ejercicio}

    \begin{ejercicio}
        Primero definimos qué es un $p$-grupo.

        \begin{definicion}[$p$-grupo]
            Si $p$ es un número primo, un grupo $G$ se dice que es un $p$-grupo si todo elemento de $G$ distinto del neutro tiene orden una potencia de $p$. Si $G$ es un grupo, diremos que $H<G$ es un $p$-subgrupo si $H$ es un $p$-grupo.
        \end{definicion}

        Necesitaremos el siguiente lema para la demostración:

        \begin{lema}[Relación Entre los Conjuntos Notables $\Orb$ y $\Stab$ de un $G$-conjunto]\label{lema:OrbStab}
            Sea $G$ un grupo finito que actúa sobre $X$, entonces para cada $x \in X$, $\Orb(x)$ es un conjunto finito y:
            $$|\Orb(x)| = [G : \Stab_G(x)]$$
            En particular, el cardinal de la órbita es un divisor del orden de $G$.
        \end{lema}

        Ahora enunciamos y demostramos el Teorema de Cauchy.

        \begin{teo}[Teorema de Cauchy]\label{teo:Cauchy}
            Si $G$ es un grupo finito, y $p$ es un primo que divide a $|G|$, entonces $G$ tiene un elemento de orden $p$, y, por tanto, tendrá un $p$-subgrupo de orden $p$.
        \end{teo}

        \begin{proof}
            Consideramos:
            $$X = \{(a_1, \ldots, a_p) \in G^{p} : a_1 \cdots a_p = 1\}$$
            Si $|G| = n$, entonces $X = n^{p-1}$, ya que elegimos libremente las $p-1$ primeras coordenadas (variación con repetición):
            $$a_1, \ldots, a_{p-1} \in G \quad \text{arbitrarios}$$
            Y la última viene condicionada por:
            $$a_p = (a_1, \ldots, a_{p-1})^{-1}$$
            Sea ahora $\sigma = (1 \hspace{0.1cm} 2 \hspace{0.1cm} \cdots \hspace{0.1cm} p) \in S_p$. Consideramos $H = \langle \sigma \rangle = \{1, \sigma, \ldots, \sigma^{p-1}\} \subseteq S_p$. Consideramos también la acción $ac : H \times X \to X$:
            $$ac(\sigma^{k}, (a_1, \ldots, a_p)) = (a_{\sigma^{k}(1)}, \ldots, a_{\sigma^{k}(p)}) \quad \forall(a_1,\ldots,a_p) \in X, \quad \forall \sigma^{k} \in H$$
            En efecto, es una acción, pues:
            \begin{itemize}
                \item Tomando como neutro $1 = \sigma^{0}$, tenemos que, para $x = (a_1, \ldots, a_p) \in X$ arbitrario: $$ac(\sigma^{0}, (a_1, \ldots, a_p)) = (a_{\sigma^{0}(1)}, \ldots, a_{\sigma^{0}(p)}) = (a_1, \ldots, a_p)$$ que es la identidad de $X$.
                \item Si tenemos $\sigma^{k}, \sigma^{l} \in H$, es decir, $k,l \in \{0, \ldots, p-1\}$ arbitrarios, y $x=(a_1, \ldots, a_p) \in X$, entonces:
                $$ac(\sigma^{k} \sigma^{l}, x) = ac(\sigma^{k+l}, x) = (a_{\sigma^{k+l}(1)}, \ldots, a_{\sigma^{k+l}(p)}) = (a_{\sigma^{k}(\sigma^{l}(1))}, \ldots, a_{\sigma^{k}(\sigma^{l}(p))}) = $$
                $$= ac(\sigma^{k}, (a_{\sigma^{l}(1)}, \ldots, a_{\sigma^{l}(p)})) = ac(\sigma^{k}, ac(\sigma^{l}, x))$$
            \end{itemize}
            Por el Lema \ref{lema:OrbStab}, tenemos que:
            $$|\Orb(z)| = [H : \Stab_H(z)] = \dfrac{|H|}{|\Stab_H(z)|} \quad \forall z \in X$$
            De donde tenemos que $|\Orb(a_1, \ldots, a_p)|$ es un divisor de $H \quad \forall (a_1, \ldots, a_p) \in X$. En dicho caso, $|\Orb(a_1, \ldots, a_p)| \in \{1,p\}$, por ser $|H| = p$ (y ser $p$ primo). Por tanto, las órbitas de un elemento serán unitarias o bien tendrán cardinal $p$. \\

            Así, sean $r$ el número de órbitas con un elemento, y $s$ el número de órbitas con $p$ elementos, entonces (con $|\Gamma| = s$):
            $$n^{p-1} = |X| = |\Fix(X)| + \sum_{y \in \Gamma} |\Orb(y)| = r + \sum_{y \in \Gamma} p = r + sp$$
            Veamos ahora cómo son los elementos de $\Orb(a_1, \ldots, a_p)$:
            $$\Orb(a_1, \ldots, a_p) = \{\prescript{\sigma^{k}}{}{(a_1, \ldots, a_p)} : k \in \{0, \ldots, p-1\}\} = $$ 
            $$\{(a_1, \ldots, a_p), (a_2, \ldots, a_p, a_1), \ldots, (a_p, a_1, \ldots, a_{p-1})\}$$
            Por tanto, la órbita será unitaria si, y sólo si, $a_1 = \cdots = a_p$. Además, sabemos de la existencia de órbitas con un elemento $(r \geqslant 1)$, como $\Orb(1, \ldots, 1)$. Busquemos más: por hipótesis, $p | n$, y además $r = n^{p-1} - sp$, de donde $p | r$, luego $r \geqslant 2$, ya que lo divide un primo. \\

            En conclusión, $\exists a \in G \setminus \{1\}$ de forma que $\Orb(a, \ldots, a)$ es unitaria, de donde $a^{p} = 1$, luego $O(a) | p$, y sabemos que $O(a) \neq 1$ (porque $O(a) = 1 \iff a = 1$, y $a \in G \setminus \{1\}$). Así pues, debe ser $O(a) = p$. \\

            Finalmente, sea $x \in \langle a \rangle \setminus \{1\}$, tenemos entonces que $1 \neq O(x) | p$, por lo que $O(x) = p$, y tenemos consecuentemente que todo elemento del subgrupo $\langle a \rangle$ es de orden $p$. En definitiva, $\langle a \rangle$ es un $p$-subgrupo de $G$ de orden $p$, como queríamos probar.
        \end{proof}

        Ahora, concluimos que, si $G$ es finito, entonces $G$ es un $p$-grupo si y sólo si su orden es una potencia de $p$. Lo enunciamos como corolario:

        \begin{coro}[Corolario del Teorema de Cauchy]
            Sea $G$ un grupo finito y $p$ un número primo:
            $$G \text{ es un $p$-subgrupo} \iff \exists n \in \N : |G| = p^{n}$$
        \end{coro}

        \begin{proof}
            Lo probaremos por doble implicación:
            \begin{itemize}
                \item[$\Longleftarrow)$] Si $|G|=p^n$ para cierto $n \in \N$, entonces tendremos que $O(x) | p^n$ para todo $x \in G$, de donde $O(x) = p^{k}$ para cierto $k \in \N$, luego $G$ es un $p$-subgrupo por definición. 
                \item[$\Longrightarrow)$] Suponemos que $q$ es un primo que divide al orden de $|G|$ (por el Teorema Fundamental de la Aritmética, todo número entero mayor que $1$, en este caso, $|G|>1$, tiene al menos un factor primo). Por el Teorema de Cauchy (Teorema \ref{teo:Cauchy}), debe existir $x \in G$ de forma que $O(x) = q$. En tal caso, como $G$ es un $p$-grupo (por hipótesis), $q=p^{r}$ para cierto $r \in \N$, de donde (dado que $q$ y $p$ son primos), $r = 1$ y $q = p$. De esta manera, el único primo que divide a $|G|$ es $p$, por lo que será $|G| = p^{n}$ para algún $n \in \N$, como queríamos probar.
                \end{itemize}
        \end{proof}
    \end{ejercicio}
\end{document}