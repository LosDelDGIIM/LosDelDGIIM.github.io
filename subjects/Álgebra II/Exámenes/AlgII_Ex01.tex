\documentclass[12pt]{article}

% Idioma y codificación
\usepackage[spanish, es-tabla]{babel}       %es-tabla para que se titule "Tabla"
\usepackage[utf8]{inputenc}

% Márgenes
\usepackage[a4paper,top=3cm,bottom=2.5cm,left=3cm,right=3cm]{geometry}

% Comentarios de bloque
\usepackage{verbatim}

% Paquetes de links
\usepackage[hidelinks]{hyperref}    % Permite enlaces
\usepackage{url}                    % redirecciona a la web

% Más opciones para enumeraciones
\usepackage{enumitem}

% Personalizar la portada
\usepackage{titling}

% Paquetes de tablas
\usepackage{multirow}


%------------------------------------------------------------------------

%Paquetes de figuras
\usepackage{caption}
\usepackage{subcaption} % Figuras al lado de otras
\usepackage{float}      % Poner figuras en el sitio indicado H.


% Paquetes de imágenes
\usepackage{graphicx}       % Paquete para añadir imágenes
\usepackage{transparent}    % Para manejar la opacidad de las figuras

% Paquete para usar colores
\usepackage[dvipsnames]{xcolor}
\usepackage{pagecolor}      % Para cambiar el color de la página

% Habilita tamaños de fuente mayores
\usepackage{fix-cm}

% Para los gráficos
\usepackage{tikz}

% Para poder situar los nodos en los grafos
\usetikzlibrary{positioning}


%------------------------------------------------------------------------

% Paquetes de matemáticas
\usepackage{mathtools, amsfonts, amssymb, mathrsfs}
\usepackage[makeroom]{cancel}     % Simplificar tachando
\usepackage{polynom}    % Divisiones y Ruffini
\usepackage{units} % Para poner fracciones diagonales con \nicefrac

\usepackage{pgfplots}   %Representar funciones
\pgfplotsset{compat=1.18}  % Versión 1.18

\usepackage{tikz-cd}    % Para usar diagramas de composiciones
\usetikzlibrary{calc}   % Para usar cálculo de coordenadas en tikz

%Definición de teoremas, etc.
\usepackage{amsthm}
%\swapnumbers   % Intercambia la posición del texto y de la numeración

\theoremstyle{plain}

\makeatletter
\@ifclassloaded{article}{
  \newtheorem{teo}{Teorema}[section]
}{
  \newtheorem{teo}{Teorema}[chapter]  % Se resetea en cada chapter
}
\makeatother

\newtheorem{coro}{Corolario}[teo]           % Se resetea en cada teorema
\newtheorem{prop}[teo]{Proposición}         % Usa el mismo contador que teorema
\newtheorem{lema}[teo]{Lema}                % Usa el mismo contador que teorema

\theoremstyle{remark}
\newtheorem*{observacion}{Observación}

\theoremstyle{definition}

\makeatletter
\@ifclassloaded{article}{
  \newtheorem{definicion}{Definición} [section]     % Se resetea en cada chapter
}{
  \newtheorem{definicion}{Definición} [chapter]     % Se resetea en cada chapter
}
\makeatother

\newtheorem*{notacion}{Notación}
\newtheorem*{ejemplo}{Ejemplo}
\newtheorem*{ejercicio*}{Ejercicio}             % No numerado
\newtheorem{ejercicio}{Ejercicio} [section]     % Se resetea en cada section


% Modificar el formato de la numeración del teorema "ejercicio"
\renewcommand{\theejercicio}{%
  \ifnum\value{section}=0 % Si no se ha iniciado ninguna sección
    \arabic{ejercicio}% Solo mostrar el número de ejercicio
  \else
    \thesection.\arabic{ejercicio}% Mostrar número de sección y número de ejercicio
  \fi
}


% \renewcommand\qedsymbol{$\blacksquare$}         % Cambiar símbolo QED
%------------------------------------------------------------------------

% Paquetes para encabezados
\usepackage{fancyhdr}
\pagestyle{fancy}
\fancyhf{}

\newcommand{\helv}{ % Modificación tamaño de letra
\fontfamily{}\fontsize{12}{12}\selectfont}
\setlength{\headheight}{15pt} % Amplía el tamaño del índice


%\usepackage{lastpage}   % Referenciar última pag   \pageref{LastPage}
\fancyfoot[C]{\thepage}

%------------------------------------------------------------------------

% Conseguir que no ponga "Capítulo 1". Sino solo "1."
\makeatletter
\@ifclassloaded{book}{
  \renewcommand{\chaptermark}[1]{\markboth{\thechapter.\ #1}{}} % En el encabezado
    
  \renewcommand{\@makechapterhead}[1]{%
  \vspace*{50\p@}%
  {\parindent \z@ \raggedright \normalfont
    \ifnum \c@secnumdepth >\m@ne
      \huge\bfseries \thechapter.\hspace{1em}\ignorespaces
    \fi
    \interlinepenalty\@M
    \Huge \bfseries #1\par\nobreak
    \vskip 40\p@
  }}
}
\makeatother

%------------------------------------------------------------------------
% Paquetes de cógido
\usepackage{minted}
\renewcommand\listingscaption{Código fuente}

\usepackage{fancyvrb}
% Personaliza el tamaño de los números de línea
\renewcommand{\theFancyVerbLine}{\small\arabic{FancyVerbLine}}

% Estilo para C++
\newminted{cpp}{
    frame=lines,
    framesep=2mm,
    baselinestretch=1.2,
    linenos,
    escapeinside=||
}

% para minted
\definecolor{LightGray}{rgb}{0.95,0.95,0.92}
\setminted{
    linenos=true,
    stepnumber=5,
    numberfirstline=true,
    autogobble,
    breaklines=true,
    breakautoindent=true,
    breaksymbolleft=,
    breaksymbolright=,
    breaksymbolindentleft=0pt,
    breaksymbolindentright=0pt,
    breaksymbolsepleft=0pt,
    breaksymbolsepright=0pt,
    fontsize=\footnotesize,
    bgcolor=LightGray,
    numbersep=10pt
}


\usepackage{listings} % Para incluir código desde un archivo

\renewcommand\lstlistingname{Código Fuente}
\renewcommand\lstlistlistingname{Índice de Códigos Fuente}

% Definir colores
\definecolor{vscodepurple}{rgb}{0.5,0,0.5}
\definecolor{vscodeblue}{rgb}{0,0,0.8}
\definecolor{vscodegreen}{rgb}{0,0.5,0}
\definecolor{vscodegray}{rgb}{0.5,0.5,0.5}
\definecolor{vscodebackground}{rgb}{0.97,0.97,0.97}
\definecolor{vscodelightgray}{rgb}{0.9,0.9,0.9}

% Configuración para el estilo de C similar a VSCode
\lstdefinestyle{vscode_C}{
  backgroundcolor=\color{vscodebackground},
  commentstyle=\color{vscodegreen},
  keywordstyle=\color{vscodeblue},
  numberstyle=\tiny\color{vscodegray},
  stringstyle=\color{vscodepurple},
  basicstyle=\scriptsize\ttfamily,
  breakatwhitespace=false,
  breaklines=true,
  captionpos=b,
  keepspaces=true,
  numbers=left,
  numbersep=5pt,
  showspaces=false,
  showstringspaces=false,
  showtabs=false,
  tabsize=2,
  frame=tb,
  framerule=0pt,
  aboveskip=10pt,
  belowskip=10pt,
  xleftmargin=10pt,
  xrightmargin=10pt,
  framexleftmargin=10pt,
  framexrightmargin=10pt,
  framesep=0pt,
  rulecolor=\color{vscodelightgray},
  backgroundcolor=\color{vscodebackground},
}

%------------------------------------------------------------------------

% Comandos definidos
\newcommand{\bb}[1]{\mathbb{#1}}
\newcommand{\cc}[1]{\mathcal{#1}}

% I prefer the slanted \leq
\let\oldleq\leq % save them in case they're every wanted
\let\oldgeq\geq
\renewcommand{\leq}{\leqslant}
\renewcommand{\geq}{\geqslant}

% Si y solo si
\newcommand{\sii}{\iff}

% Letras griegas
\newcommand{\eps}{\epsilon}
\newcommand{\veps}{\varepsilon}
\newcommand{\lm}{\lambda}

\newcommand{\ol}{\overline}
\newcommand{\ul}{\underline}
\newcommand{\wt}{\widetilde}
\newcommand{\wh}{\widehat}

\let\oldvec\vec
\renewcommand{\vec}{\overrightarrow}

% Derivadas parciales
\newcommand{\del}[2]{\frac{\partial #1}{\partial #2}}
\newcommand{\Del}[3]{\frac{\partial^{#1} #2}{\partial #3^{#1}}}
\newcommand{\deld}[2]{\dfrac{\partial #1}{\partial #2}}
\newcommand{\Deld}[3]{\dfrac{\partial^{#1} #2}{\partial #3^{#1}}}


\newcommand{\AstIg}{\stackrel{(\ast)}{=}}
\newcommand{\Hop}{\stackrel{L'H\hat{o}pital}{=}}

\newcommand{\red}[1]{{\color{red}#1}} % Para integrales, destacar los cambios.

% Método de integración
\newcommand{\MetInt}[2]{
    \left[\begin{array}{c}
        #1 \\ #2
    \end{array}\right]
}

% Declarar aplicaciones
% 1. Nombre aplicación
% 2. Dominio
% 3. Codominio
% 4. Variable
% 5. Imagen de la variable
\newcommand{\Func}[5]{
    \begin{equation*}
        \begin{array}{rrll}
            #1:& #2 & \longrightarrow & #3\\
               & #4 & \longmapsto & #5
        \end{array}
    \end{equation*}
}

%------------------------------------------------------------------------


\DeclareMathOperator{\GL}{GL}
\begin{document}

    % 1. Foto de fondo
    % 2. Título
    % 3. Encabezado Izquierdo
    % 4. Color de fondo
    % 5. Coord x del titulo
    % 6. Coord y del titulo
    % 7. Fecha

    
    % 1. Foto de fondo
% 2. Título
% 3. Encabezado Izquierdo
% 4. Color de fondo
% 5. Coord x del titulo
% 6. Coord y del titulo
% 7. Fecha

\newcommand{\portada}[7]{

    \portadaBase{#1}{#2}{#3}{#4}{#5}{#6}{#7}
    \portadaBook{#1}{#2}{#3}{#4}{#5}{#6}{#7}
}

\newcommand{\portadaExamen}[7]{

    \portadaBase{#1}{#2}{#3}{#4}{#5}{#6}{#7}
    \portadaArticle{#1}{#2}{#3}{#4}{#5}{#6}{#7}
}




\newcommand{\portadaBase}[7]{

    % Tiene la portada principal y la licencia Creative Commons
    
    % 1. Foto de fondo
    % 2. Título
    % 3. Encabezado Izquierdo
    % 4. Color de fondo
    % 5. Coord x del titulo
    % 6. Coord y del titulo
    % 7. Fecha
    
    
    \thispagestyle{empty}               % Sin encabezado ni pie de página
    \newgeometry{margin=0cm}        % Márgenes nulos para la primera página
    
    
    % Encabezado
    \fancyhead[L]{\helv #3}
    \fancyhead[R]{\helv \nouppercase{\leftmark}}
    
    
    \pagecolor{#4}        % Color de fondo para la portada
    
    \begin{figure}[p]
        \centering
        \transparent{0.3}           % Opacidad del 30% para la imagen
        
        \includegraphics[width=\paperwidth, keepaspectratio]{assets/#1}
    
        \begin{tikzpicture}[remember picture, overlay]
            \node[anchor=north west, text=white, opacity=1, font=\fontsize{60}{90}\selectfont\bfseries\sffamily, align=left] at (#5, #6) {#2};
            
            \node[anchor=south east, text=white, opacity=1, font=\fontsize{12}{18}\selectfont\sffamily, align=right] at (9.7, 3) {\textbf{\href{https://losdeldgiim.github.io/}{Los Del DGIIM}}};
            
            \node[anchor=south east, text=white, opacity=1, font=\fontsize{12}{15}\selectfont\sffamily, align=right] at (9.7, 1.8) {Doble Grado en Ingeniería Informática y Matemáticas\\Universidad de Granada};
        \end{tikzpicture}
    \end{figure}
    
    
    \restoregeometry        % Restaurar márgenes normales para las páginas subsiguientes
    \pagecolor{white}       % Restaurar el color de página
    
    
    \newpage
    \thispagestyle{empty}               % Sin encabezado ni pie de página
    \begin{tikzpicture}[remember picture, overlay]
        \node[anchor=south west, inner sep=3cm] at (current page.south west) {
            \begin{minipage}{0.5\paperwidth}
                \href{https://creativecommons.org/licenses/by-nc-nd/4.0/}{
                    \includegraphics[height=2cm]{assets/Licencia.png}
                }\vspace{1cm}\\
                Esta obra está bajo una
                \href{https://creativecommons.org/licenses/by-nc-nd/4.0/}{
                    Licencia Creative Commons Atribución-NoComercial-SinDerivadas 4.0 Internacional (CC BY-NC-ND 4.0).
                }\\
    
                Eres libre de compartir y redistribuir el contenido de esta obra en cualquier medio o formato, siempre y cuando des el crédito adecuado a los autores originales y no persigas fines comerciales. 
            \end{minipage}
        };
    \end{tikzpicture}
    
    
    
    % 1. Foto de fondo
    % 2. Título
    % 3. Encabezado Izquierdo
    % 4. Color de fondo
    % 5. Coord x del titulo
    % 6. Coord y del titulo
    % 7. Fecha


}


\newcommand{\portadaBook}[7]{

    % 1. Foto de fondo
    % 2. Título
    % 3. Encabezado Izquierdo
    % 4. Color de fondo
    % 5. Coord x del titulo
    % 6. Coord y del titulo
    % 7. Fecha

    % Personaliza el formato del título
    \pretitle{\begin{center}\bfseries\fontsize{42}{56}\selectfont}
    \posttitle{\par\end{center}\vspace{2em}}
    
    % Personaliza el formato del autor
    \preauthor{\begin{center}\Large}
    \postauthor{\par\end{center}\vfill}
    
    % Personaliza el formato de la fecha
    \predate{\begin{center}\huge}
    \postdate{\par\end{center}\vspace{2em}}
    
    \title{#2}
    \author{\href{https://losdeldgiim.github.io/}{Los Del DGIIM}}
    \date{Granada, #7}
    \maketitle
    
    \tableofcontents
}




\newcommand{\portadaArticle}[7]{

    % 1. Foto de fondo
    % 2. Título
    % 3. Encabezado Izquierdo
    % 4. Color de fondo
    % 5. Coord x del titulo
    % 6. Coord y del titulo
    % 7. Fecha

    % Personaliza el formato del título
    \pretitle{\begin{center}\bfseries\fontsize{42}{56}\selectfont}
    \posttitle{\par\end{center}\vspace{2em}}
    
    % Personaliza el formato del autor
    \preauthor{\begin{center}\Large}
    \postauthor{\par\end{center}\vspace{3em}}
    
    % Personaliza el formato de la fecha
    \predate{\begin{center}\huge}
    \postdate{\par\end{center}\vspace{5em}}
    
    \title{#2}
    \author{\href{https://losdeldgiim.github.io/}{Los Del DGIIM}}
    \date{Granada, #7}
    \thispagestyle{empty}               % Sin encabezado ni pie de página
    \maketitle
    \vfill
}
    \portadaExamen{ffccA4.jpg}{Álgebra II\\Examen I}{Álgebra II. Examen I}{MidnightBlue}{-8}{28}{2025}{Arturo Olivares Martos}

    \begin{description}
        \item[Asignatura] Álgebra II.
        \item[Curso Académico] 2017-18.
        \item[Grado] Doble Grado en Ingeniería Informática y Matemáticas.
        \item[Grupo] Único.
        %\item[Profesor] José María Espinar García.
        \item[Descripción] Parcial I.
        \item[Fecha] Octubre de 2017.
        % \item[Duración] 60 minutos.
    
    \end{description}
    \newpage

    \begin{ejercicio}~
        \begin{enumerate}
            \item Demostrar que en un grupo de orden par el número de elementos de orden 2 es impar.
            \item Describe dos grupos de orden 6 que sean isomorfos y otros dos que no lo sean. Razona la respuesta.
        \end{enumerate}
    \end{ejercicio}

    \begin{ejercicio}
        Razona cual es la respuesta correcta en cada una de las siguientes cuestiones:
        \begin{enumerate}
            \item Dados grupos $G$ y $H$:
            \begin{enumerate}
                \item Si tienen el mismo orden son isomorfos.
                \item Si son isomorfos tienen el mismo orden.
                \item Si se pueden generar por el mismo número de elementos son isomorfos.
            \end{enumerate}
            \item Elije la opción correcta:
            \begin{enumerate}
                \item En $D_4$ todos los elementos tienen orden par.
                \item $D_4$ y $S_4$ son grupos isomorfos.
                \item Salvo isomorfismo, $D_4$ es el único grupo no abeliano de orden 8.
            \end{enumerate}
            \item Si $f : G \to H$ es un homomorfismo de grupos, entonces:
            \begin{enumerate}
                \item $O(x)$ divide a $O(f(x))$ $\forall x \in G$.
                \item $O(f(x))$ divide a $O(x)$ $\forall x \in G$.
                \item $O(x) = O(f(x))$ $\forall x \in G$.
            \end{enumerate}
            \item Dadas las permutaciones $\alpha = (2\ 3\ 6)(6\ 5\ 7\ 1\ 3\ 4)$, $\beta = (2\ 4\ 7\ 3)$ $\in S_{10}$ se tiene que $\beta\alpha\beta^{-1}$:
            \begin{enumerate}
                \item Es par.
                \item Su orden es 12.
                \item Es un ciclo de longitud 7.
            \end{enumerate}
            \item Si $\mu_6$ denota el grupo de las raíces sextas de la unidad, entonces:
            \begin{enumerate}
                \item $\mu_6 \cong C_6$.
                \item $\mu_6 \cong S_3$.
                \item $\mu_6 \cong D_6$.
            \end{enumerate}
            \item En $S_4$ se tiene que:
            \begin{enumerate}
                \item $\{(1\ 2),(3\ 4)\}$ es un conjunto de generadores.
                \item $\{(1\ 2\ 3\ 4)\}$ es un conjunto de generadores.
                \item $\{(1\ 2),(2\ 3),(3\ 4)\}$ es un conjunto de generadores.
            \end{enumerate}
            \item Sea $G$ un grupo y $f : G \to G$ la aplicación dada por $f(x) = x^{-1}$. Entonces:
            \begin{enumerate}
                \item $f$ es un homomorfismo de grupos.
                \item $f$ es un automorfismo.
                \item Si $f$ es un homomorfismo entonces $G$ es abeliano.
            \end{enumerate}
            \item Para cualquier permutación $\sigma \in S_n$, si $\veps(\sigma)$ denota su signo o paridad, se tiene:
            \begin{enumerate}
                \item $\veps(\sigma) = \veps(\sigma^{-1})$.
                \item $\veps(\sigma) = -\veps(\sigma^{-1})$.
                \item Ninguna de las anteriores.
            \end{enumerate}
            \item Cualquier permutación $\sigma \in S_n$:
            \begin{enumerate}
                \item Se descompone de forma única como producto de trasposiciones.
                \item Es producto de trasposiciones.
                \item Se descompone de forma única como producto de trasposiciones disjuntas.
            \end{enumerate}
            \item El grupo $\GL_2(\bb{Z}_2)$ de matrices invertibles $2 \times 2$ con entradas en $\bb{Z}_2$:
            \begin{enumerate}
                \item Es un grupo no abeliano de orden 8.
                \item Es un grupo isomorfo a $Z_6$.
                \item Es un grupo isomorfo a $S_3$.
            \end{enumerate}
                
        \end{enumerate}
    \end{ejercicio}


    \newpage
    \setcounter{ejercicio}{0}
    \begin{ejercicio}~
        \begin{enumerate}
            \item Demostrar que en un grupo de orden par el número de elementos de orden 2 es impar.\\
            
            En primer lugar, dado un grupo arbitrario $G$ y fijado $k \in \bb{N}$, se define el conjunto siguiente:
            \begin{equation*}
                G_k = \{x \in G \mid O(x) = k\}
            \end{equation*}

            Sabemos que $G_1=\{1\}$. Ahora, vamos a ver que el orden de $G_k$ para todo $k\geq 3$ es par. Dado $x\in G$ con $O(x)=k$, entonces $O(x^{-1})=k$ y $x^{-1}=x^{k-1}$. Para $k\geq 3$, se tiene además que $x\neq x^{-1}$. Por tanto, para cada $x\in G_k$ con $k\geq 3$, se tiene que $x\neq x^{-1}$ y $x^{-1}\in G_k$, por lo que los elementos de $G_k$ van por pares y, por tanto, $|G_k|$ es par.\\

            Supongamos ahora nuestra hipótesis, $G$ un grupo de orden par (en particular, finito). Por tanto, todo elemento de $G$ tiene orden finito y $G$ se descompone en grupos disjuntos como sigue:
            \begin{align*}
                G=\bigcup_{k=1}^{\infty}G_k = \{1\}\cup G_2\cup \left(\bigcup_{k=3}^{\infty}G_k\right)
            \end{align*}

            Considerando cardinales, puesto que son disjuntos, se tiene que:
            \begin{align*}
                |G_2| = |G|-1-\sum_{k=3}^{\infty}|G_k|
            \end{align*}

            Como $|G|$ es par y $|G_k|$ es par para todo $k\geq 3$, se tiene que $|G_2|$ es impar. Por tanto, el número de elementos de orden 2 en un grupo de orden par es impar.
            \item Describe dos grupos de orden 6 que sean isomorfos y otros dos que no lo sean. Razona la respuesta.\\
            
            En un ejercicio, vimos que todo grupo de orden $6$ o es cíclico o es isomorfo a $D_3$. Consideramos por tanto los grupos siguientes:
            \begin{equation*}
                C_6\ncong D_3\cong S_3
            \end{equation*}

            Sabemos que $C_6$ es conmutativo y $S_3$ no, por lo que $C_6\ncong S_3$ y por tanto $D_3\cong S_3$.
        \end{enumerate}
    \end{ejercicio}

    \begin{ejercicio}
        Razona cual es la respuesta correcta en cada una de las siguientes cuestiones:
        \begin{enumerate}
            \item Dados grupos $G$ y $H$:
            \begin{enumerate}
                \item Si tienen el mismo orden son isomorfos.
                
                No es correcta, pues $D_3\ncong C_6$ y ambos tienen orden 6.
                \item Si son isomorfos tienen el mismo orden.
                
                Correcta, pues si todo isomorfismo en particular es una biyección. Por tanto, si $G\cong H$ entonces $|G|=|H|$.
                \item Si se pueden generar por el mismo número de elementos son isomorfos.
                
                No es correcta, pues $D_3\ncong D_4$ y ambos se generan por dos elementos.
            \end{enumerate}
            
            Por tanto, la opción correcta es la \textbf{b)}.
            \item Elije la opción correcta:
            \begin{enumerate}
                \item En $D_4$ todos los elementos tienen orden par.
                
                Sabemos que $O(1)=1$, luego es incorrecta.
                \item $D_4$ y $S_4$ son grupos isomorfos.
                
                Falso, pues $|D_4|=8\neq 24=|S_4|$.
                \item Salvo isomorfismo, $D_4$ es el único grupo no abeliano de orden 8.
                
                Consideramos el grupo de los cuaternios $Q_2$. Tenemos que:
                \begin{equation*}
                    ij=k\neq -k=ji
                \end{equation*}
                Por tanto, $Q_2$ no es abeliano, y $|Q_2|=8$. Veamos que no es isomorfo a $D_4$. Los órdenes de los elementos de $Q_2$ son:
                \begin{equation*}
                    O(1)=1,\quad O(-1)=2,\qquad O(\pm i)=O(\pm j)=O(\pm k)=4
                \end{equation*}

                Los órdenes de los elementos de $D_4$ son:
                \begin{gather*}
                    O(1)=1,\quad O(r)=O(r^3)=4\\
                    O(r^2)=O(s)=O(sr)=O(sr^2)=O(sr^3)=2
                \end{gather*}

                Por tanto, $Q_2\ncong D_4$ y ambos son no abelianos y de orden 8. Por tanto, es incorrecta.
                
                \begin{comment}
                Veámoslo. Sea $G$ no abeliano, con $|G|=8$, sabemos que todo elemento de $G$ tiene orden 1, 2, 4 u 8. Caben las siguientes posibilidades:
                \begin{itemize}
                    \item \ul{$\exists x\in G$ con $O(x)=8$}.
                    
                    En este caso, $\langle x\rangle = G$ y $G$ es cíclico y por tanto abeliano, lo que contradice la hipótesis.

                    \item \ul{$\nexists x\in G$ con $O(x)=8$}.
                    
                    Entonces, aparte del $1$, todo elemento de $G$ cumplirá $O(x)\in \{2,4\}$.
                    \begin{itemize}
                        \item \ul{$\nexists x\in G$ con $O(x)=4$}.
                        
                        En este caso, todos los elementos de $G$ tendrán orden 2. Sea $x\in G$ con $O(x)=2$, entonces $\langle x\rangle = \{1,x\}$. Como $|G|=8$, elegimos $y\in G\setminus \langle x\rangle$ y $O(y)=2$. Comprobemos que $xy\notin \{1,x,y\}$:
                        \begin{itemize}
                            \item $xy=1\Rightarrow y=x$, contradicción.
                            \item $xy=x\Rightarrow y=1$, contradicción.
                            \item $xy=y\Rightarrow x=1$, contradicción.
                        \end{itemize}

                        Por tanto, $\{1,x,y,xy\}$ son cuatro elementos distintos de $G$. Como $|G|=8$, elegimos $z\notin \langle x,y\rangle$ y $O(z)=2$. Comprobemos que $xz\notin \{1,x,y,xy,z\}$:
                        \begin{itemize}
                            \item $xz=1\Rightarrow z=x$, contradicción.
                            \item $xz=x\Rightarrow z=1$, contradicción.
                            \item $xz=y\Rightarrow z=y$, contradicción.
                            \item $xz=xy\Rightarrow z=y$, contradicción.
                            \item $xz=z\Rightarrow x=1$, contradicción.
                        \end{itemize}

                        Comprobemos que $zy\notin \{1,x,y,xy,z, xz\}$:
                        \begin{itemize}
                            \item $zy=1\Rightarrow z=y$, contradicción.
                            \item $zy=x\Rightarrow z=xy$, contradicción.
                            \item $zy=y\Rightarrow z=1$, contradicción.
                            \item $zy=xy\Rightarrow z=x$, contradicción.
                            \item $zy=z\Rightarrow y=1$, contradicción.
                        \end{itemize}
                    \end{itemize}


                \end{itemize}
                \end{comment}
            \end{enumerate}

            Por tanto, no hay ninguna opción correcta.
            \item Si $f : G \to H$ es un homomorfismo de grupos, entonces:
            \begin{enumerate}
                \item $O(x)$ divide a $O(f(x))$ $\forall x \in G$.
                
                Consideramos el homomorfismo trivial:
                \Func{f}{G}{H}{x}{1}

                Tenemos que $O(f(x))=O(1)=1$ para todo $x\in G$. Por tanto, tomando $x\in G\setminus \{1\}$, tenemos que $O(x)\nmid 1$, por lo que no es cierta.
                \item $O(f(x))$ divide a $O(x)$ $\forall x \in G$.
                
                Supongamos $O(x)$ finito (puesto que si no, no tiene sentido hablar de división). Entonces:
                \begin{equation*}
                    1 = f(1)=f\left(x^{O(x)}\right)=f(x)^{O(x)}
                    \Longrightarrow O(f(x))\mid O(x)\qquad \forall x\in G
                \end{equation*}
                \item $O(x) = O(f(x))$ $\forall x \in G$.
                
                Esto sabemos que es cierto si $f$ es un monomorfismo, pero no de forma general. De hecho, el homomorfismo trivial es un contraejemplo.
            \end{enumerate}

            Por tanto, la opción correcta es la \textbf{b)}.
            \item Dadas las permutaciones $\alpha = (2\ 3\ 6)(6\ 5\ 7\ 1\ 3\ 4)$, $\beta = (2\ 4\ 7\ 3)$ $\in S_{10}$ se tiene que $\beta\alpha\beta^{-1}$:
            \begin{enumerate}
                \item Es par.
                \item Su orden es 12.
                \item Es un ciclo de longitud 7.
            \end{enumerate}

            Calculamos en primer lugar $\alpha$ como producto de ciclos disjuntos:
            \begin{equation*}
                \alpha=(1\ 6\ 5\ 7)(2\ 3\ 4)
            \end{equation*}

            Por tanto, tenemos que:
            \begin{align*}
                \beta\alpha\beta^{-1}&=\beta(1\ 6\ 5\ 7)\beta^{-1}\ \beta(2\ 3\ 4)\beta^{-1}
                =\\&= (1\ 6\ 5\ 3)(4\ 2\ 7)
            \end{align*}

            Por tanto, sabemos que $\veps(\beta\alpha\beta^{-1})=-1$, que no es un ciclo, y que:
            \begin{equation*}
                O(\beta\alpha\beta^{-1})=mcm(4,3)=12
            \end{equation*}

            Por tanto, la opción correcta es la \textbf{b)}.
            \item Si $\mu_6$ denota el grupo de las raíces sextas de la unidad, entonces:
            \begin{enumerate}
                \item $\mu_6 \cong C_6$.
                
                Es cierta, pues $\mu_6=\langle \xi\mid \xi^6=1\rangle$. El isomorfismo se obtiene gracias al Teorema de Dyck.
                \item $\mu_6 \cong S_3$.
                
                No es correcta, pues $\mu_6$ es abeliano y $S_3$ no.
                \item $\mu_6 \cong D_6$.
                
                No es correcta, pues $|D_6|=12\neq 6=|\mu_6|$.
            \end{enumerate}
            \item En $S_4$ se tiene que:
            \begin{enumerate}
                \item $\{(1\ 2),(3\ 4)\}$ es un conjunto de generadores.
                
                Falso, puesto que no se podría generar la trasposición $(2\ 3)$.
                \item $\{(1\ 2\ 3\ 4)\}$ es un conjunto de generadores.
                
                De serlo, $S_4$ sería cíclico y, por tanto, abeliano, lo cual no es cierto.
                \item $\{(1\ 2),(2\ 3),(3\ 4)\}$ es un conjunto de generadores.
                
                Cierto, puesto que se vió que:
                \begin{equation*}
                    S_n=\langle (1\ 2),(2\ 3),\ldots,(n-1\ n)\rangle
                \end{equation*}
            \end{enumerate}
            \item Sea $G$ un grupo y $f : G \to G$ la aplicación dada por $f(x) = x^{-1}$. Entonces:
            \begin{enumerate}
                \item $f$ es un homomorfismo de grupos.
                \item $f$ es un automorfismo.
                \item Si $f$ es un homomorfismo entonces $G$ es abeliano.
            \end{enumerate}

            En la relación se ha visto que:
            \begin{equation*}
                f\ \text{es un homomorfismo}\Longleftrightarrow G\ \text{es abeliano}
            \end{equation*}

            Por tanto, en el caso de que $G$ no sea abeliano, la opción $a)$ es incorrecta, luego $b)$ también lo es. De hecho, la opción correcta es la \textbf{c)}.
            \item Para cualquier permutación $\sigma \in S_n$, si $\veps(\sigma)$ denota su signo o paridad, se tiene:
            \begin{enumerate}
                \item $\veps(\sigma) = \veps(\sigma^{-1})$.
                \item $\veps(\sigma) = -\veps(\sigma^{-1})$.
                \item Ninguna de las anteriores.
            \end{enumerate}

            Pues que la signatura depende del número de trasposiciones de longitud par y esta es invariante al tomar la inversa de una permutación, la opción correcta es la \textbf{a)}.
            \item Cualquier permutación $\sigma \in S_n$:
            \begin{enumerate}
                \item Se descompone de forma única como producto de trasposiciones.
                \item Es producto de trasposiciones.
                \item Se descompone de forma única como producto de trasposiciones disjuntas.
            \end{enumerate}

            Sabemos que toda permutación se descompone de forma única como producto de \emph{ciclos} disjuntos \emph{salvo el orden}. No obstante, respecto a las trasposiciones tan solo sabemos que toda permutación se descompone como producto de trasposiciones, pero no de forma única. Por tanto, la opción correcta es la \textbf{b)}.
            \item El grupo $\GL_2(\bb{Z}_2)$ de matrices invertibles $2 \times 2$ con entradas en $\bb{Z}_2$:
            \begin{enumerate}
                \item Es un grupo no abeliano de orden 8.
                \item Es un grupo isomorfo a $\bb{Z}_6$.
                \item Es un grupo isomorfo a $S_3$.
            \end{enumerate}

            Calculemos el orden:
            \begin{equation*}
                |\GL_2(\bb{Z}_2)|=(2^2-1)(2^2-2)=3\cdot 2=6
            \end{equation*}

            Veamos ahora que no es abeliano:
            \begin{align*}
                \begin{pmatrix}
                    1 & 1\\
                    0 & 1
                \end{pmatrix}
                \begin{pmatrix}
                    1 & 0\\
                    1 & 1
                \end{pmatrix}=
                \begin{pmatrix}
                    0 & 1\\
                    1 & 1
                \end{pmatrix}\neq
                \begin{pmatrix}
                    1 & 1\\
                    1 & 0
                \end{pmatrix}=
                \begin{pmatrix}
                    1 & 0\\
                    1 & 1
                \end{pmatrix}
                \begin{pmatrix}
                    1 & 1\\
                    0 & 1
                \end{pmatrix}
            \end{align*}

            Por tanto, sabemos que no es abeliano. Por ser de orden $6$, sabemos que, o bien es cíclico (que no puede serlo por no ser abeliano), o es isomorfo a $D_3$. Por tanto:
            \begin{equation*}
                \GL_2(\bb{Z}_2)\cong D_3\cong S_3
            \end{equation*}

            Por tanto, la opción correcta es la \textbf{c)}.
                
        \end{enumerate}
    \end{ejercicio}

\end{document}