\documentclass[12pt]{article}

% Idioma y codificación
\usepackage[spanish, es-tabla, es-notilde]{babel}       %es-tabla para que se titule "Tabla"
\usepackage[utf8]{inputenc}

% Márgenes
\usepackage[a4paper,top=3cm,bottom=2.5cm,left=3cm,right=3cm]{geometry}

% Comentarios de bloque
\usepackage{verbatim}

% Paquetes de links
\usepackage[hidelinks]{hyperref}    % Permite enlaces
\usepackage{url}                    % redirecciona a la web

% Más opciones para enumeraciones
\usepackage{enumitem}

% Personalizar la portada
\usepackage{titling}

% Paquetes de tablas
\usepackage{multirow}

% Para añadir el símbolo de euro
\usepackage{eurosym}


%------------------------------------------------------------------------

%Paquetes de figuras
\usepackage{caption}
\usepackage{subcaption} % Figuras al lado de otras
\usepackage{float}      % Poner figuras en el sitio indicado H.


% Paquetes de imágenes
\usepackage{graphicx}       % Paquete para añadir imágenes
\usepackage{transparent}    % Para manejar la opacidad de las figuras

% Paquete para usar colores
\usepackage[dvipsnames, table, xcdraw]{xcolor}
\usepackage{pagecolor}      % Para cambiar el color de la página

% Habilita tamaños de fuente mayores
\usepackage{fix-cm}

% Para los gráficos
\usepackage{tikz}
\usepackage{forest}

% Para poder situar los nodos en los grafos
\usetikzlibrary{positioning}


%------------------------------------------------------------------------

% Paquetes de matemáticas
\usepackage{mathtools, amsfonts, amssymb, mathrsfs}
\usepackage[makeroom]{cancel}     % Simplificar tachando
\usepackage{polynom}    % Divisiones y Ruffini
\usepackage{units} % Para poner fracciones diagonales con \nicefrac

\usepackage{pgfplots}   %Representar funciones
\pgfplotsset{compat=1.18}  % Versión 1.18

\usepackage{tikz-cd}    % Para usar diagramas de composiciones
\usetikzlibrary{calc}   % Para usar cálculo de coordenadas en tikz

%Definición de teoremas, etc.
\usepackage{amsthm}
%\swapnumbers   % Intercambia la posición del texto y de la numeración

\theoremstyle{plain}

\makeatletter
\@ifclassloaded{article}{
  \newtheorem{teo}{Teorema}[section]
}{
  \newtheorem{teo}{Teorema}[chapter]  % Se resetea en cada chapter
}
\makeatother

\newtheorem{coro}{Corolario}[teo]           % Se resetea en cada teorema
\newtheorem{prop}[teo]{Proposición}         % Usa el mismo contador que teorema
\newtheorem{lema}[teo]{Lema}                % Usa el mismo contador que teorema
\newtheorem*{lema*}{Lema}

\theoremstyle{remark}
\newtheorem*{observacion}{Observación}

\theoremstyle{definition}

\makeatletter
\@ifclassloaded{article}{
  \newtheorem{definicion}{Definición} [section]     % Se resetea en cada chapter
}{
  \newtheorem{definicion}{Definición} [chapter]     % Se resetea en cada chapter
}
\makeatother

\newtheorem*{notacion}{Notación}
\newtheorem*{ejemplo}{Ejemplo}
\newtheorem*{ejercicio*}{Ejercicio}             % No numerado
\newtheorem{ejercicio}{Ejercicio} [section]     % Se resetea en cada section


% Modificar el formato de la numeración del teorema "ejercicio"
\renewcommand{\theejercicio}{%
  \ifnum\value{section}=0 % Si no se ha iniciado ninguna sección
    \arabic{ejercicio}% Solo mostrar el número de ejercicio
  \else
    \thesection.\arabic{ejercicio}% Mostrar número de sección y número de ejercicio
  \fi
}


% \renewcommand\qedsymbol{$\blacksquare$}         % Cambiar símbolo QED
%------------------------------------------------------------------------

% Paquetes para encabezados
\usepackage{fancyhdr}
\pagestyle{fancy}
\fancyhf{}

\newcommand{\helv}{ % Modificación tamaño de letra
\fontfamily{}\fontsize{12}{12}\selectfont}
\setlength{\headheight}{15pt} % Amplía el tamaño del índice


%\usepackage{lastpage}   % Referenciar última pag   \pageref{LastPage}
%\fancyfoot[C]{%
%  \begin{minipage}{\textwidth}
%    \centering
%    ~\\
%    \thepage\\
%    \href{https://losdeldgiim.github.io/}{\texttt{\footnotesize losdeldgiim.github.io}}
%  \end{minipage}
%}
\fancyfoot[C]{\thepage}
\fancyfoot[R]{\href{https://losdeldgiim.github.io/}{\texttt{\footnotesize losdeldgiim.github.io}}}

%------------------------------------------------------------------------

% Conseguir que no ponga "Capítulo 1". Sino solo "1."
\makeatletter
\@ifclassloaded{book}{
  \renewcommand{\chaptermark}[1]{\markboth{\thechapter.\ #1}{}} % En el encabezado
    
  \renewcommand{\@makechapterhead}[1]{%
  \vspace*{50\p@}%
  {\parindent \z@ \raggedright \normalfont
    \ifnum \c@secnumdepth >\m@ne
      \huge\bfseries \thechapter.\hspace{1em}\ignorespaces
    \fi
    \interlinepenalty\@M
    \Huge \bfseries #1\par\nobreak
    \vskip 40\p@
  }}
}
\makeatother

%------------------------------------------------------------------------
% Paquetes de cógido
\usepackage{minted}
\renewcommand\listingscaption{Código fuente}

\usepackage{fancyvrb}
% Personaliza el tamaño de los números de línea
\renewcommand{\theFancyVerbLine}{\small\arabic{FancyVerbLine}}

% Estilo para C++
\newminted{cpp}{
    frame=lines,
    framesep=2mm,
    baselinestretch=1.2,
    linenos,
    escapeinside=||
}

% para minted
\definecolor{LightGray}{rgb}{0.95,0.95,0.92}
\setminted{
    linenos=true,
    stepnumber=5,
    numberfirstline=true,
    autogobble,
    breaklines=true,
    breakautoindent=true,
    breaksymbolleft=,
    breaksymbolright=,
    breaksymbolindentleft=0pt,
    breaksymbolindentright=0pt,
    breaksymbolsepleft=0pt,
    breaksymbolsepright=0pt,
    fontsize=\footnotesize,
    bgcolor=LightGray,
    numbersep=10pt
}


\usepackage{listings} % Para incluir código desde un archivo

\renewcommand\lstlistingname{Código Fuente}
\renewcommand\lstlistlistingname{Índice de Códigos Fuente}

% Definir colores
\definecolor{vscodepurple}{rgb}{0.5,0,0.5}
\definecolor{vscodeblue}{rgb}{0,0,0.8}
\definecolor{vscodegreen}{rgb}{0,0.5,0}
\definecolor{vscodegray}{rgb}{0.5,0.5,0.5}
\definecolor{vscodebackground}{rgb}{0.97,0.97,0.97}
\definecolor{vscodelightgray}{rgb}{0.9,0.9,0.9}

% Configuración para el estilo de C similar a VSCode
\lstdefinestyle{vscode_C}{
  backgroundcolor=\color{vscodebackground},
  commentstyle=\color{vscodegreen},
  keywordstyle=\color{vscodeblue},
  numberstyle=\tiny\color{vscodegray},
  stringstyle=\color{vscodepurple},
  basicstyle=\scriptsize\ttfamily,
  breakatwhitespace=false,
  breaklines=true,
  captionpos=b,
  keepspaces=true,
  numbers=left,
  numbersep=5pt,
  showspaces=false,
  showstringspaces=false,
  showtabs=false,
  tabsize=2,
  frame=tb,
  framerule=0pt,
  aboveskip=10pt,
  belowskip=10pt,
  xleftmargin=10pt,
  xrightmargin=10pt,
  framexleftmargin=10pt,
  framexrightmargin=10pt,
  framesep=0pt,
  rulecolor=\color{vscodelightgray},
  backgroundcolor=\color{vscodebackground},
}

%------------------------------------------------------------------------

% Comandos definidos
\newcommand{\bb}[1]{\mathbb{#1}}
\newcommand{\cc}[1]{\mathcal{#1}}

% I prefer the slanted \leq
\let\oldleq\leq % save them in case they're every wanted
\let\oldgeq\geq
\renewcommand{\leq}{\leqslant}
\renewcommand{\geq}{\geqslant}

% Si y solo si
\newcommand{\sii}{\iff}

% MCD y MCM
\DeclareMathOperator{\mcd}{mcd}
\DeclareMathOperator{\mcm}{mcm}

% Signo
\DeclareMathOperator{\sgn}{sgn}

% Letras griegas
\newcommand{\eps}{\epsilon}
\newcommand{\veps}{\varepsilon}
\newcommand{\lm}{\lambda}

\newcommand{\ol}{\overline}
\newcommand{\ul}{\underline}
\newcommand{\wt}{\widetilde}
\newcommand{\wh}{\widehat}

\let\oldvec\vec
\renewcommand{\vec}{\overrightarrow}

% Derivadas parciales
\newcommand{\del}[2]{\frac{\partial #1}{\partial #2}}
\newcommand{\Del}[3]{\frac{\partial^{#1} #2}{\partial #3^{#1}}}
\newcommand{\deld}[2]{\dfrac{\partial #1}{\partial #2}}
\newcommand{\Deld}[3]{\dfrac{\partial^{#1} #2}{\partial #3^{#1}}}


\newcommand{\AstIg}{\stackrel{(\ast)}{=}}
\newcommand{\Hop}{\stackrel{L'H\hat{o}pital}{=}}

\newcommand{\red}[1]{{\color{red}#1}} % Para integrales, destacar los cambios.

% Método de integración
\newcommand{\MetInt}[2]{
    \left[\begin{array}{c}
        #1 \\ #2
    \end{array}\right]
}

% Declarar aplicaciones
% 1. Nombre aplicación
% 2. Dominio
% 3. Codominio
% 4. Variable
% 5. Imagen de la variable
\newcommand{\Func}[5]{
    \begin{equation*}
        \begin{array}{rrll}
            \displaystyle #1:& \displaystyle  #2 & \longrightarrow & \displaystyle  #3\\
               & \displaystyle  #4 & \longmapsto & \displaystyle  #5
        \end{array}
    \end{equation*}
}

%------------------------------------------------------------------------


\DeclareMathOperator{\GL}{GL}
\DeclareMathOperator{\fact}{fact}
\DeclareMathOperator{\Aut}{Aut}
\DeclareMathOperator{\Syl}{Syl}
\begin{document}

    % 1. Foto de fondo
    % 2. Título
    % 3. Encabezado Izquierdo
    % 4. Color de fondo
    % 5. Coord x del titulo
    % 6. Coord y del titulo
    % 7. Fecha

    
    % 1. Foto de fondo
% 2. Título
% 3. Encabezado Izquierdo
% 4. Color de fondo
% 5. Coord x del titulo
% 6. Coord y del titulo
% 7. Fecha
% 8. Autor

\newcommand{\portada}[8]{
    \portadaBase{#1}{#2}{#3}{#4}{#5}{#6}{#7}{#8}
    \portadaBook{#1}{#2}{#3}{#4}{#5}{#6}{#7}{#8}
}

\newcommand{\portadaFotoDif}[8]{
    \portadaBaseFotoDif{#1}{#2}{#3}{#4}{#5}{#6}{#7}{#8}
    \portadaBook{#1}{#2}{#3}{#4}{#5}{#6}{#7}{#8}
}

\newcommand{\portadaExamen}[8]{
    \portadaBase{#1}{#2}{#3}{#4}{#5}{#6}{#7}{#8}
    \portadaArticle{#1}{#2}{#3}{#4}{#5}{#6}{#7}{#8}
}

\newcommand{\portadaExamenFotoDif}[8]{
    \portadaBaseFotoDif{#1}{#2}{#3}{#4}{#5}{#6}{#7}{#8}
    \portadaArticle{#1}{#2}{#3}{#4}{#5}{#6}{#7}{#8}
}




\newcommand{\portadaBase}[8]{

    % Tiene la portada principal y la licencia Creative Commons
    
    % 1. Foto de fondo
    % 2. Título
    % 3. Encabezado Izquierdo
    % 4. Color de fondo
    % 5. Coord x del titulo
    % 6. Coord y del titulo
    % 7. Fecha
    % 8. Autor    
    
    \thispagestyle{empty}               % Sin encabezado ni pie de página
    \newgeometry{margin=0cm}        % Márgenes nulos para la primera página
    
    
    % Encabezado
    \fancyhead[L]{\helv #3}
    \fancyhead[R]{\helv \nouppercase{\leftmark}}
    
    
    \pagecolor{#4}        % Color de fondo para la portada
    
    \begin{figure}[p]
        \centering
        \transparent{0.3}           % Opacidad del 30% para la imagen
        
        \includegraphics[width=\paperwidth, keepaspectratio]{../../_assets/#1}
    
        \begin{tikzpicture}[remember picture, overlay]
            \node[anchor=north west, text=white, opacity=1, font=\fontsize{60}{90}\selectfont\bfseries\sffamily, align=left] at (#5, #6) {#2};
            
            \node[anchor=south east, text=white, opacity=1, font=\fontsize{12}{18}\selectfont\sffamily, align=right] at (9.7, 3) {\href{https://losdeldgiim.github.io/}{\textbf{Los Del DGIIM}, \texttt{\footnotesize losdeldgiim.github.io}}};
            
            \node[anchor=south east, text=white, opacity=1, font=\fontsize{12}{15}\selectfont\sffamily, align=right] at (9.7, 1.8) {Doble Grado en Ingeniería Informática y Matemáticas\\Universidad de Granada};
        \end{tikzpicture}
    \end{figure}
    
    
    \restoregeometry        % Restaurar márgenes normales para las páginas subsiguientes
    \nopagecolor      % Restaurar el color de página
    
    
    \newpage
    \thispagestyle{empty}               % Sin encabezado ni pie de página
    \begin{tikzpicture}[remember picture, overlay]
        \node[anchor=south west, inner sep=3cm] at (current page.south west) {
            \begin{minipage}{0.5\paperwidth}
                \href{https://creativecommons.org/licenses/by-nc-nd/4.0/}{
                    \includegraphics[height=2cm]{../../_assets/Licencia.png}
                }\vspace{1cm}\\
                Esta obra está bajo una
                \href{https://creativecommons.org/licenses/by-nc-nd/4.0/}{
                    Licencia Creative Commons Atribución-NoComercial-SinDerivadas 4.0 Internacional (CC BY-NC-ND 4.0).
                }\\
    
                Eres libre de compartir y redistribuir el contenido de esta obra en cualquier medio o formato, siempre y cuando des el crédito adecuado a los autores originales y no persigas fines comerciales. 
            \end{minipage}
        };
    \end{tikzpicture}
    
    
    
    % 1. Foto de fondo
    % 2. Título
    % 3. Encabezado Izquierdo
    % 4. Color de fondo
    % 5. Coord x del titulo
    % 6. Coord y del titulo
    % 7. Fecha
    % 8. Autor


}


\newcommand{\portadaBaseFotoDif}[8]{

    % Tiene la portada principal y la licencia Creative Commons
    
    % 1. Foto de fondo
    % 2. Título
    % 3. Encabezado Izquierdo
    % 4. Color de fondo
    % 5. Coord x del titulo
    % 6. Coord y del titulo
    % 7. Fecha
    % 8. Autor    
    
    \thispagestyle{empty}               % Sin encabezado ni pie de página
    \newgeometry{margin=0cm}        % Márgenes nulos para la primera página
    
    
    % Encabezado
    \fancyhead[L]{\helv #3}
    \fancyhead[R]{\helv \nouppercase{\leftmark}}
    
    
    \pagecolor{#4}        % Color de fondo para la portada
    
    \begin{figure}[p]
        \centering
        \transparent{0.3}           % Opacidad del 30% para la imagen
        
        \includegraphics[width=\paperwidth, keepaspectratio]{#1}
    
        \begin{tikzpicture}[remember picture, overlay]
            \node[anchor=north west, text=white, opacity=1, font=\fontsize{60}{90}\selectfont\bfseries\sffamily, align=left] at (#5, #6) {#2};
            
            \node[anchor=south east, text=white, opacity=1, font=\fontsize{12}{18}\selectfont\sffamily, align=right] at (9.7, 3) {\href{https://losdeldgiim.github.io/}{\textbf{Los Del DGIIM}, \texttt{\footnotesize losdeldgiim.github.io}}};
            
            \node[anchor=south east, text=white, opacity=1, font=\fontsize{12}{15}\selectfont\sffamily, align=right] at (9.7, 1.8) {Doble Grado en Ingeniería Informática y Matemáticas\\Universidad de Granada};
        \end{tikzpicture}
    \end{figure}
    
    
    \restoregeometry        % Restaurar márgenes normales para las páginas subsiguientes
    \nopagecolor      % Restaurar el color de página
    
    
    \newpage
    \thispagestyle{empty}               % Sin encabezado ni pie de página
    \begin{tikzpicture}[remember picture, overlay]
        \node[anchor=south west, inner sep=3cm] at (current page.south west) {
            \begin{minipage}{0.5\paperwidth}
                %\href{https://creativecommons.org/licenses/by-nc-nd/4.0/}{
                %    \includegraphics[height=2cm]{../../_assets/Licencia.png}
                %}\vspace{1cm}\\
                Esta obra está bajo una
                \href{https://creativecommons.org/licenses/by-nc-nd/4.0/}{
                    Licencia Creative Commons Atribución-NoComercial-SinDerivadas 4.0 Internacional (CC BY-NC-ND 4.0).
                }\\
    
                Eres libre de compartir y redistribuir el contenido de esta obra en cualquier medio o formato, siempre y cuando des el crédito adecuado a los autores originales y no persigas fines comerciales. 
            \end{minipage}
        };
    \end{tikzpicture}
    
    
    
    % 1. Foto de fondo
    % 2. Título
    % 3. Encabezado Izquierdo
    % 4. Color de fondo
    % 5. Coord x del titulo
    % 6. Coord y del titulo
    % 7. Fecha
    % 8. Autor


}


\newcommand{\portadaBook}[8]{

    % 1. Foto de fondo
    % 2. Título
    % 3. Encabezado Izquierdo
    % 4. Color de fondo
    % 5. Coord x del titulo
    % 6. Coord y del titulo
    % 7. Fecha
    % 8. Autor

    % Personaliza el formato del título
    \pretitle{\begin{center}\bfseries\fontsize{42}{56}\selectfont}
    \posttitle{\par\end{center}\vspace{2em}}
    
    % Personaliza el formato del autor
    \preauthor{\begin{center}\Large}
    \postauthor{\par\end{center}\vfill}
    
    % Personaliza el formato de la fecha
    \predate{\begin{center}\huge}
    \postdate{\par\end{center}\vspace{2em}}
    
    \title{#2}
    \author{\href{https://losdeldgiim.github.io/}{Los Del DGIIM, \texttt{\large losdeldgiim.github.io}}
    \\ \vspace{0.5cm}#8}
    \date{Granada, #7}
    \maketitle
    
    \tableofcontents
}




\newcommand{\portadaArticle}[8]{

    % 1. Foto de fondo
    % 2. Título
    % 3. Encabezado Izquierdo
    % 4. Color de fondo
    % 5. Coord x del titulo
    % 6. Coord y del titulo
    % 7. Fecha
    % 8. Autor

    % Personaliza el formato del título
    \pretitle{\begin{center}\bfseries\fontsize{42}{56}\selectfont}
    \posttitle{\par\end{center}\vspace{2em}}
    
    % Personaliza el formato del autor
    \preauthor{\begin{center}\Large}
    \postauthor{\par\end{center}\vspace{3em}}
    
    % Personaliza el formato de la fecha
    \predate{\begin{center}\huge}
    \postdate{\par\end{center}\vspace{5em}}
    
    \title{#2}
    \author{\href{https://losdeldgiim.github.io/}{Los Del DGIIM, \texttt{\large losdeldgiim.github.io}}
    \\ \vspace{0.5cm}#8}
    \date{Granada, #7}
    \thispagestyle{empty}               % Sin encabezado ni pie de página
    \maketitle
    \vfill
}
    \portadaExamen{ffccA4.jpg}{Álgebra II\\Examen V}{Álgebra II. Examen V}{MidnightBlue}{-8}{28}{2025}{Arturo Olivares Martos}

    \begin{description}
        \item[Asignatura] Álgebra II.
        \item[Curso Académico] 2022-23.
        \item[Grado] Doble Grado en Ingeniería Informática y Matemáticas.
        \item[Grupo] Único.
        \item[Profesor] Manuel Bullejos Lorenzo.
        \item[Descripción] Convocatoria Ordinaria.
        %\item[Fecha] 21 de mayo del 2025.
        %\item[Duración] 2 horas.
    
    \end{description}
    \newpage


    \begin{ejercicio}
        Sea $A$ el grupo abeliano siguiente. Indique las descomposiciones Cíclica Primaria y Cíclica de $A$, así como su orden y el rango de su parte libre. Clasifique, indicando sus descomposiciones Cíclicas Primaria y Cíclicas todos los grupos abelianos del mismo orden que $A$.
        \begin{align*}
            A & = \left\langle x, y, z\left|\begin{array}{rcl}
                10x + 12y + 4z & = & 0 \\
                8x + 11y + 6z & = & 0 \\
                4x + 6y + 8z & = & 0
            \end{array}\right.\right\rangle
        \end{align*}
    \end{ejercicio}

    \begin{ejercicio}
        Sea $G=\langle (1\ 2\ 3\ 4)\rangle \leq S_5$.
        \begin{enumerate}
            \item Calcula el número de conjugados de $(1\ 2\ 3\ 4)$ y demuestra que $G$ no es normal en $S_5$.
            \item Demuestra que $G$ no es un $2$-subgrupo de Sylow de $S_5$.
            \item Construye un $2$-subgrupo de Sylow de $S_5$ que contenga a $G$.
        \end{enumerate}
    \end{ejercicio}

    \begin{ejercicio}
        Sea $G$ un grupo de orden $125$.
        \begin{enumerate}
            \item Sea $x$ un elemento de $G$ de orden $25$, y sea $K=\langle x\rangle$. Demuestra que $K$ es normal en $G$.
            \item Sea $y$ un elemento de $G$ que no está en $K$ y que tiene orden $5$. Sea ahora $H=\langle y\rangle$. Demuestra que $G=K\rtimes H$.
            \item Prueba que $\prescript{y}{}{x}=x^6$ es una acción de grupos de $H$ en $K$.
            \item Si se cumple $yxy^{-1}=x^6$, demuestra que $\langle a, b\mid a^{25}=b^5=1, ba=a^6b\rangle$ es una presentación de $G$.
        \end{enumerate}
    \end{ejercicio}

    \begin{ejercicio}
        Demuestra que:
        \begin{enumerate}
            \item Ningún grupo de orden $390$ es simple.
            \item Ningún grupo de orden $30$ es simple.
            \item Todo grupo de orden $390$ es resoluble.
        \end{enumerate}
    \end{ejercicio}




    \newpage
    \setcounter{ejercicio}{0}



    \begin{ejercicio}
        Sea $A$ el grupo abeliano siguiente. Indique las descomposiciones Cíclica Primaria y Cíclica de $A$, así como su orden y el rango de su parte libre. Clasifique, indicando sus descomposiciones Cíclicas Primaria y Cíclicas todos los grupos abelianos del mismo orden que $A$.
        \begin{align*}
            A & = \left\langle x, y, z\left.\begin{array}{rcl}
                10x + 12y + 4z & = & 0 \\
                8x + 11y + 6z & = & 0 \\
                4x + 6y + 8z & = & 0
            \end{array}\right|\right\rangle
        \end{align*}

        La matriz de relaciones de $A$ es:
        \begin{align*}
            M & = \begin{pmatrix}
                10 & 12 & 4 \\
                8 & 11 & 6 \\
                4 & 6 & 8
            \end{pmatrix}
        \end{align*}

        Calculamos su forma normal de Smith:
        \begin{multline*}
            M=\begin{pmatrix}
                10 & 12 & 4 \\
                8 & 11 & 6 \\
                4 & 6 & 8
            \end{pmatrix}
            \xrightarrow{F_1'=F_1-F_2}
            \begin{pmatrix}
                2 & 1 & -2 \\
                8 & 11 & 6 \\
                4 & 6 & 8
            \end{pmatrix}
            \xrightarrow{C_1 \leftrightarrow C_2}
            \begin{pmatrix}
                1 & 2 & -2 \\
                11 & 8 & 6 \\
                6 & 4 & 8
            \end{pmatrix}
            \xrightarrow[F_3'=F_3-6F_1]{F_2'=F_2-11F_1}\\
            \begin{pmatrix}
                1 & 2 & -2 \\
                0 & -14 & 28 \\
                0 & -8 & 20
            \end{pmatrix}
            \xrightarrow[C_2'=C_2-2C_1]{C_3'=C_3+2C_1}
            \begin{pmatrix}
                1 & 0 & 0\\
                0 & -14 & 28 \\
                0 & -8 & 20
            \end{pmatrix}
            \xrightarrow{F_2'=F_2-2F_3}
            \begin{pmatrix}
                1 & 0 & 0\\
                0 & 2 & -12 \\
                0 & -8 & 20
            \end{pmatrix}
            \xrightarrow{F_3'=-(F_3+4F_2)}\\
            \begin{pmatrix}
                1 & 0 & 0\\
                0 & 2 & -12 \\
                0 & 0 & 28
            \end{pmatrix}
            \xrightarrow{C_3'=C_3+6C_2}
            \begin{pmatrix}
                1 & 0 & 0\\
                0 & 2 & 0 \\
                0 & 0 & 28
            \end{pmatrix}
        \end{multline*}

        Por tanto, la descomposición cíclica de $A$ es:
        \begin{align*}
            A & \cong C_2 \oplus C_{28}
        \end{align*}

        La descomposición cíclica primaria de $A$ es:
        \begin{align*}
            A & \cong C_2 \oplus C_4 \oplus C_{7}
        \end{align*}

        El orden de $A$ es:
        \begin{align*}
            |A| & = 2\cdot 28 = 56
        \end{align*}

        El rango de la parte libre de $A$ es:
        \begin{align*}
            3-3 & = 0
        \end{align*}

        Clasificamos ahora los grupos abelianos de orden $56=2^3\cdot 7$.
        Estos se muestran en la Tabla~\ref{tab:grupos_abelianos_orden_56}.
        \begin{table}[h]
            \centering
            \begin{tabular}{c|c|c|c|c}
                & \textbf{Fact. Inv.} & \textbf{Div. element.} & \textbf{Desc. cíclica primaria} & \textbf{Desc. cíclica} \\
                \hline
                $\begin{pmatrix}
                    2^3\\
                    7
                \end{pmatrix}
                $ & $d_1=56$ & $\{2^3; 7\}$ & $C_8 \oplus C_{7}$ & $C_{56}$ \\ \hline
                $\begin{pmatrix}
                    2^2 & 2\\
                    7 & 1
                \end{pmatrix}
                $ & $\begin{array}{c}
                    d_1=28\\
                    d_2=2
                \end{array}$ & $\{2^2; 2; 7\}$ & $C_4 \oplus C_2 \oplus C_{7}$ & $C_{28} \times C_2$ \\ \hline
                $\begin{pmatrix}
                    2 & 2 & 2\\
                    7 & 1 & 1
                \end{pmatrix}
                $ & $\begin{array}{c}
                    d_1=14\\
                    d_2=2\\
                    d_3=2
                \end{array}$ & $\{2; 2; 2; 7\}$ & $C_2 \oplus C_2 \oplus C_2 \oplus C_{7}$ & $C_{14} \times C_2 \times C_2$
            \end{tabular}
            \caption{Grupos abelianos de orden $56$.}
            \label{tab:grupos_abelianos_orden_56}
        \end{table}

        
    \end{ejercicio}

    \begin{ejercicio}
        Sea $G=\langle (1\ 2\ 3\ 4)\rangle \leq S_5$.
        \begin{enumerate}
            \item Calcula el número de conjugados de $(1\ 2\ 3\ 4)$ y demuestra que $G$ no es normal en $S_5$.
            
            Puesto que la conjugación mantiene la estructura de las permutaciones, sabemos que los conjugados de $(1\ 2\ 3\ 4)$ son los $4-$ ciclos de $S_5$. En total:
            \begin{align*}
                \frac{5\cdot 4\cdot 3\cdot 2}{4} = 30
            \end{align*}

            Por tanto, el número de conjugados de $(1\ 2\ 3\ 4)$ es $30$.\\

            Veamos ahora que $G$ no es normal en $S_5$. Como $|G|=4$, sea $\gamma$ un $4-$ciclo de $S_5\setminus G$. Como $\gamma$ y $(1\ 2\ 3\ 4)$ son conjugados, sea $\tau\in S_5$ tal que $\tau(1\ 2\ 3\ 4)\tau^{-1}=\gamma$. Entonces:
            \begin{align*}
                \tau (1\ 2\ 3\ 4)\tau^{-1} & = \gamma \notin G\Longrightarrow G\cancel{\lhd} S_5
            \end{align*}


            \item Demuestra que $G$ no es un $2$-subgrupo de Sylow de $S_5$.
            
            Como $|S_5|=120=2^3\cdot 3\cdot 5$, los $2$-subgrupos de Sylow de $S_5$ tienen orden $8$. Como $|G|=4$, tenemos que $G$ no es un $2$-subgrupo de Sylow de $S_5$.
            \item Construye un $2$-subgrupo de Sylow de $S_5$ que contenga a $G$.
            
            Simplemente, tenemos que construir un grupo de orden $8$ que contenga a $G$. Para no tener que realizar pruebas, buscaremos tomarlo isomorfo a $D_4$. Como $(1\ 2\ 3\ 4)^4=(1)$, buscamos una transposición $\tau$ tal que los siguientes resultados coincidan:
            \begin{align*}
                (1\ 2\ 3\ 4)^3 &= (1\ 4\ 3\ 2) \\
                \tau(1\ 2\ 3\ 4)\tau^{-1} &= (\tau(1)\ \tau(2)\ \tau(3)\ \tau(4))
            \end{align*}

            Consideramos por tanto $\tau=(2\ 4)$. De esta forma, por el Teorema de Dyck, tenemos que:
            \begin{align*}
                Q=\langle (1\ 2\ 3\ 4), (2\ 4)\rangle \cong D_4
            \end{align*}
            Tenemos por tanto que $Q$ es un $2$-subgrupo de Sylow de $S_5$ que contiene a $G$.
        \end{enumerate}
    \end{ejercicio}

    \begin{ejercicio}
        Sea $G$ un grupo de orden $125$.
        \begin{enumerate}
            \item Sea $x$ un elemento de $G$ de orden $25$, y sea $K=\langle x\rangle$. Demuestra que $K$ es normal en $G$.
            
            Como $[G:K]=5$ y $5$ es el menor primo que divide a $|G|$, tenemos que $K$ es normal en $G$.
            \item Sea $y$ un elemento de $G$ que no está en $K$ y que tiene orden $5$. Sea ahora $H=\langle y\rangle$. Demuestra que $G=K\rtimes H$.
            
            Supongamos que $H\cap K\neq \{1\}$. Por tanto, $\exists i\in \{1,2,3,4\}$ tal que $y^i\in K$. Como $\varphi(5)=4$, tenemos que $H=\langle y^i\rangle$, luego $y\in K$, lo cual es una contradicción. Por tanto, $H\cap K=\{1\}$.\\

            Por el Segundo Teorema de Isomorfía, tenemos que:
            \begin{equation*}
                \dfrac{H}{H\cap K}\cong \dfrac{HK}{K}
                \Longrightarrow |HK|=|H|\cdot |K|=5\cdot 25=125=|G|
                \Longrightarrow HK=G
            \end{equation*}

            Por tanto, $G=K\rtimes H$.
            
            \item Prueba que $\prescript{y}{}{x}=x^6$ es una acción de grupos de $H$ en $K$.
            
            Esto equivale a probar que:
            \Func{\theta}{H}{\Aut(k)}{y}{\theta(y)}
            \Func{\theta(y)}{K}{K}{x}{\prescript{y}{}{x}=x^6}
            es un homomorfismo de grupos. Por el Teorema de Dyck, puesto que $H=\langle y\rangle$, bastará con comprobar que $\theta^5(y)=Id_K$.
            \begin{align*}
                \theta^5(y)(x) & = x^{6^5} = x^{7776}\qquad \forall x\in K
            \end{align*}
            Dado $x\in K$, tenemos que $O(x)\in \{1,5,25\}$. En cualquier caso:
            \begin{align*}
                x^{7776} & = x^1 = x
            \end{align*}

            Por tanto, $\theta^5(y)=Id_K$, luego $\theta$ es un homomorfismo de grupos, y por tanto $\prescript{y}{}{x}=x^6$ es una acción de grupos de $H$ en $K$.


            \item Si se cumple $yxy^{-1}=x^6$, demuestra que $\langle a, b\mid a^{25}=b^5=1, ba=a^6b\rangle$ es una presentación de $G$.
            
            Sea $Q=\langle a, b\mid a^{25}=b^5=1, ba=a^6b\rangle$. Sea $f:Q\to G$ el homomorfismo de grupos definido por:
            \begin{align*}
                f(a) & = x \\
                f(b) & = y
            \end{align*}

            Comprobamos que $x,y$ cumplen las relaciones de $Q$:
            \begin{align*}
                x^{25} = 1\qquad y^5 = 1\qquad yx=x^6y\iff yxy^{-1}=x^6
            \end{align*}

            Por tanto, $f$ es un homomorfismo de grupos. Además, como $G\cong K\rtimes H$, tenemos que $x,y$ generan $G$. Por tanto, $f$ es sobreyectivo. Por tanto, $|Q|\geq |G|=125$.

            Por otro lado, como $ba=a^6b$, tenemos que todo elemento de $Q$ es de la forma $a^i b^j$, con $0\leq i < 25$ y $0\leq j < 5$. Por tanto, $|Q|\leq 25\cdot 5=125$. Por tanto, $|Q|=|G|=125$. Por el Teorema de Dyck, tenemos que $f$ es un isomorfismo de grupos, luego $G\cong Q$.
            \begin{equation*}
                G\cong \langle a, b\mid a^{25}=b^5=1, ba=a^6b\rangle
            \end{equation*}
        \end{enumerate}
    \end{ejercicio}


    \begin{ejercicio}
        Demuestra que:
        \begin{enumerate}
            \item Ningún grupo de orden $390$ es simple.
            
            Sea $G$ un grupo de orden $390=2\cdot 3\cdot 5\cdot 13$. Por el Segundo Teorema de Sylow, tenemos que:
            \begin{align*}
                \left.\begin{array}{l}
                    n_{13} \equiv 1 \mod 13 \\
                    n_{13} \mid 30
                \end{array}
                \right\}
                \Longrightarrow n_{13} \in \left\{1,\cancel{3},\cancel{6},\cancel{10},\cancel{15},\cancel{30}\right\}
            \end{align*}

            Por tanto, $n_{13}=1$, luego hay un único $13$-subgrupo de Sylow de $G$, que por ser único es normal. Además, como su orden es $13$, es un subgrupo normal propio de $G$. Por tanto, $G$ no es simple.
            \item Ningún grupo de orden $30$ es simple.
            

            Sea $G$ un grupo de orden $30=2\cdot 3\cdot 5$. Por el Segundo Teorema de Sylow, tenemos que:
            \begin{align*}
                \left.\begin{array}{l}
                    n_{5} \equiv 1 \mod 5 \\
                    n_{5} \mid 6
                \end{array}
                \right\}
                \Longrightarrow n_{5} \in \left\{1,\cancel{2},\cancel{3},6\right\}
            \end{align*}
            
            Por otro lado, tenemos que:
            \begin{align*}
                \left.\begin{array}{l}
                    n_{3} \equiv 1 \mod 3 \\
                    n_{3} \mid 10
                \end{array}
                \right\}
                \Longrightarrow n_{3} \in \left\{1,\cancel{2},\cancel{5},10\right\}
            \end{align*}

            Supongamos que $n_{5}=6$ y $n_{3}=10$.
            \begin{itemize}
                \item Como $n_{5}=6$, tenemos que hay $6$ $5$-subgrupos de Sylow, cada uno de orden $5$ (luego cíclicos). Además, la intersección de dos $5$-subgrupos de Sylow es trivial, luego hay $6\cdot 4=24$ elementos de orden $5$.
                \item Como $n_{3}=10$, tenemos que hay $10$ $3$-subgrupos de Sylow, cada uno de orden $3$ (luego cíclicos). Además, la intersección de dos $3$-subgrupos de Sylow es trivial, luego hay $10\cdot 2=20$ elementos de orden $3$.
                \item Por tanto, hay $24+20=44$ elementos de orden $5$ o $3$, pero $|G|=30$.
            \end{itemize}

            Llegamos a una contradicción, luego $n_{5}=1$ o $n_{3}=1$. Por tanto, hay un único $5$-subgrupo de Sylow o un único $3$-subgrupo de Sylow, que por ser único es normal. Además, como su orden es $5$ o $3$, es un subgrupo normal propio de $G$. Por tanto, $G$ no es simple.
            \item Todo grupo de orden $390$ es resoluble.
            
            Sea $G$ un grupo de orden $390$. Como hemos visto en el primer apartado, hay un único $13$-subgrupo de Sylow de $G$ (llamémoslo $P_{13}$), que es normal.\\

            Consideramos por tanto $G/P_{13}$. Sabemos que $P_{13}$ es abeliano, luego resoluble. Estudiamos ahora el cociente:
            \begin{equation*}
                |G/P_{13}| = \frac{|G|}{|P_{13}|} = \frac{390}{13} = 30
            \end{equation*}

            Por tanto, por el apartado anterior, $G/P_{13}$ tiene un grupo (llamémoslo $H$) normal propio de orden $3$ o $5$. En cualquier caso, $H$ es abeliano, luego resoluble. Además:
            \begin{equation*}
                \left|\dfrac{G/P_{13}}{H}\right| = \frac{|G/P_{13}|}{|H|} =\frac{30}{|H|} \in 2\cdot \{3,5\}
            \end{equation*}
            Por tanto, $\frac{G/P_{13}}{H}$ es de la forma $pq$, con $p,q$ primos distintos, es resoluble. Como $H$ también es resoluble, tenemos que $G/P_{13}$ es resoluble.\\

            Como $P_{13}$ es resoluble y $G/P_{13}$ es resoluble, tenemos que $G$ es resoluble.
        \end{enumerate}
    \end{ejercicio}
\end{document}
