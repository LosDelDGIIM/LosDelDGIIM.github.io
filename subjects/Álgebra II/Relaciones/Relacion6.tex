\section{$G-$conjuntos y $p$-grupos}

\begin{ejercicio}\label{ej:6.1}
    Si $X$ es un $G-$conjunto, demostrar que $x^g = \prescript{g^{-1}}{}{x},~ x \in X, g \in G$, define una acción por la derecha de $G$ sobre $X$.\\

    En primer lugar, vemos que se trata de una aplicación de $G \times X$ en $X$. Veamos ahora que cumple las condiciones necesarias para ser una acción por la derecha:
    \begin{itemize}
        \item $x^1 = x$ para todo $x \in X$.
        \begin{equation*}
            x^1 = \prescript{1^{-1}}{}{x} = \prescript{1}{}{x} = x
        \end{equation*}

        \item $(x^g)^h = x^{gh}$ para todo $x \in X$ y $g, h \in G$.
        \begin{equation*}
            (x^g)^h = \prescript{h^{-1}}{}{(x^g)} = \prescript{h^{-1}}{}{(\prescript{g^{-1}}{}{x})} = \prescript{h^{-1}g^{-1}}{}{x} =  \prescript{(gh)^{-1}}{}{x} = x^{gh}
        \end{equation*}
    \end{itemize}

    Por tanto, se trata de una acción por la derecha de $G$ sobre $X$.
\end{ejercicio}

\begin{ejercicio}\label{ej:6.2}
    Sea $G$ un grupo y $N$ un subgrupo normal abeliano de $G$. Demostrar que $G/N$ actúa sobre $N$ por conjugación y obtener entonces un homomorfismo $\varphi: G/N \to \Aut(N)$.\\

    Veamos en primer lugar que $G/N$ actúa sobre $N$ por conjugación. Es decir, que la siguiente aplicación es una acción de $G/N$ sobre $N$:
    \Func{ac}{G/N \times N}{N}{(gN, n)}{\prescript{gN}{}{n} = gng^{-1}}

    Veamos en primer lugar que está bien definida. Sean $g_1, g_2 \in G$ de forma que $g_1N = g_2N$. Entonces $\exists n'\in N$ tal que $g_1 = g_2n'$. Entonces:
    \begin{align*}
        \prescript{g_1N}{}{n} &= g_1ng_1^{-1} = g_2n' n (g_2n')^{-1} = g_2n' n (n')^{-1}g_2^{-1} \AstIg g_2 n'(n')^{-1} n g_2^{-1} = g_2 n g_2^{-1}
        = \prescript{g_2N}{}{n}
    \end{align*}
    donde en $(\ast)$ hemos usado que $N$ es abeliano. Por tanto, la acción está bien definida. Veamos ahora que se trata de una acción.
    \begin{itemize}
        \item $\prescript{1N}{}{n} = 1n1^{-1} = n$ para todo $n \in N$.
        \item Comprobemos la segunda propiedad:
        \begin{align*}
            \prescript{(g_1N)(g_2N)}{}{n} &= \prescript{g_1g_2N}{}{n} = g_1g_2ng_2^{-1}g_1^{-1} = g_1 \left(\prescript{g_2N}{}{n}\right) g_1^{-1}
            = \prescript{g_1N}{}{\left(\prescript{g_2N}{}{n}\right)}.
        \end{align*}
    \end{itemize}

    Buscamos ahora el homomorfismo $\varphi: G/N \to \Aut(N)$. En primer lugar, consideramos el siguiente homomorfismo:
    \Func{\Phi}{G/N}{\Perm(N)}{gN}{\prescript{gN}{}{(\cdot)}=ac(gN, \cdot)}


    Es necesario ver que, fijado $gN \in G/N$, la aplicación siguiente, además de pertenecer a $\Perm(N)$, pertenece a $\Aut(N)$:
    \Func{f}{N}{N}{n}{\prescript{gN}{}{n} = gng^{-1}}

    Sabemos que es biyectiva, por lo que tan solo nos queda probar que es un homomorfismo. Sean $n_1, n_2 \in N$:
    \begin{align*}
        f(n_1n_2) &= \prescript{gN}{}{(n_1n_2)} = g(n_1n_2)g^{-1} = g n_1 g^{-1} g n_2 g^{-1} = f(n_1)f(n_2).
    \end{align*}

    Por tanto, $f$ es un homomorfismo. La aplicación $\varphi$ pedida entonces es:
    \Func{\varphi}{G/N}{\Aut(N)}{gN}{f = \prescript{gN}{}{(\cdot)}}
\end{ejercicio}

\begin{ejercicio}\label{ej:6.3}
    Sean $S$ y $T$ dos $G-$conjuntos. Se define la \emph{acción diagonal} de $G$ sobre el producto cartesiano $S \times T$ mediante $\prescript{x}{}{(s,t)} = (\prescript{x}{}{s},\prescript{x}{}{t})$. Demostrar que, para la acción diagonal, el estabilizador de $(s, t)$ es la intersección de los estabilizadores de $s$ y $t$ en las acciones dadas.\\

    Fijados $s \in S$ y $t \in T$, el estabilizador de $(s,t)$ es:
    \begin{align*}
        \Stab_{G}(s,t) &= \{g \in G \mid \prescript{g}{}{(s,t)} = (s,t)\} = \{g \in G \mid (\prescript{g}{}{s},\prescript{g}{}{t}) = (s,t)\}\\
        &= \{g \in G \mid \prescript{g}{}{s} = s \land \prescript{g}{}{t} = t\} = \{g \in G \mid \prescript{g}{}{s} = s\} \cap \{g \in G \mid \prescript{g}{}{t} = t\}\\
        &= \Stab_{G}(s) \cap \Stab_{G}(t).
    \end{align*}
\end{ejercicio}

\begin{ejercicio}\label{ej:6.4}
    Demostrar que si $G$ contiene un elemento $x$ que tiene exactamente dos conjugados, entonces $G$ tiene un subgrupo normal propio.
    \begin{observacion}
        Considerar el centralizador de $x$.
    \end{observacion}

    Consideramos la acción por conjugación de $G$ sobre sí mismo:
    \Func{ac}{G \times G}{G}{(g,h)}{\prescript{g}{}{h} = ghg^{-1}}

    Calculamos el centralizador de $x$:
    \begin{align*}
        C_G(\{x\}) &= \{g \in G \mid gx = xg\} = \{g \in G \mid gxg^{-1} = x\} = \{g \in G \mid \prescript{g}{}{x} = x\} = \Stab_G(x)
    \end{align*}

    Por tanto, $C_G(\{x\}) = \Stab_G(x)<G$. Veamos ahora que es normal en $G$. Calculemos la órbita de $x$:
    \begin{align*}
        \Orb(x) &= \{y\in G \mid \exists g \in G \text{ tal que } y = \prescript{g}{}{x}\} = \{y\in G \mid \exists g \in G \text{ tal que } y = gxg^{-1}\}
        = \Cl_G(x)
    \end{align*}

    Como $x$ tiene exactamente dos conjugados (él mismo y otro elemento $y\in G$), tenemos que $|\Orb(x)| = 2$. Por tanto:
    \begin{align*}
        [G:C_G(\{x\})] &= |\Orb(x)| = 2 \implies C_G(\{x\})\lhd G
    \end{align*}
    \begin{observacion}
        Notemos que, aun sin saber si $G$ es finito, la igualdad anterior tiene perfecto sentido, puesto que $|\Orb(x)|=2$ y $[G:C_G(\{x\})]$ indica el número de clases en el conjunto cociente, que sabemos que es biyectivo con $\Orb(x)$, luego es $2$.
    \end{observacion}

    Por tanto, $C_G(\{x\})$ es un subgrupo normal de $G$. Tan solo falta por comprobar que es propio.
    \begin{itemize}
        \item Si $C_G(\{x\}) = G$, entonces:
        \begin{equation*}
            2 = |\Orb(x)| = [G:\Stab_G(x)] = [G:C_G(\{x\})] = 1 \implies \text{Contradicción.}
        \end{equation*}

        \item Si $C_G(\{x\}) = \{1\}$, entonces:
        \begin{equation*}
            2 = |\Orb(x)| = [G:\Stab_G(x)] = [G:C_G(\{x\})] = [G:\{1\}] = |G|
        \end{equation*}
        Por tanto, $G=\{1,x\}$. Calculemos el número de conjugados de $1$ y de $x$:
        \begin{align*}
            \Cl_G(1) &= \{g1g^{-1} \mid g \in G\} = \{1\} \\
            \Cl_G(x) &= \{gxg^{-1} \mid g \in G\} = \{1x1, xxx^{-1}\} = \{x\}
        \end{align*}
        Por tanto, ambos tienen un único conjugado. Por tanto, no se puede dar este caso.
    \end{itemize}
    Por tanto, $C_G(\{x\})$ es un subgrupo normal propio de $G$.
\end{ejercicio}

\begin{ejercicio}\label{ej:6.5}
    Encontrar todos los grupos finitos que tienen exactamente dos clases de conjugación.\\

    Sea $G$ un grupo finito con $|G| = n$ que tiene exactamente dos clases de conjugación; a saber, $\exists x_1, x_2 \in G$ tales que $\Cl_G(x_1) \neq \Cl_G(x_2)$. Considerando la acción de $G$ sobre sí mismo por conjugación, tenemos que:
    \begin{equation*}
        \Orb(x) = \Cl_G(x) \qquad \forall x \in G
    \end{equation*}

    Como las órbitas forman una partición de $G$, tenemos que:
    \begin{equation*}
        |G| = |\Orb(x_1)| + |\Orb(x_2)| = |\Cl_G(x_1)| + |\Cl_G(x_2)|
    \end{equation*}

    Calculamos no obstante la clase de conjugación del $1\in G$:
    \begin{align*}
        \Cl_G(1) &= \{g1g^{-1} \mid g \in G\} = \{g g^{-1} \mid g \in G\} = \{1\}
    \end{align*}
    Por tanto, $|\Cl_G(1)| = 1$. Supongamos sin pérdida de generalidad que $1\in \Cl_G(x_1)$. Entonces:
    \begin{equation*}
        n = |\Cl_G(x_1)| + |\Cl_G(x_2)| = 1 + |\Cl_G(x_2)|
        \Longrightarrow |\Cl_G(x_2)| = n - 1
    \end{equation*}

    Por otro lado, como $|\Cl_G(x_2)| = [G:\Stab_G(x_2)]$, tenemos que $|\Cl_G(x_2)|$ divide a $|G|$; es decir, $(n-1) \mid n$. Por tanto, $n=2$, y tenemos por tanto que:
    \begin{equation*}
        G\cong \bb{Z}_2
    \end{equation*}
\end{ejercicio}

\begin{ejercicio}\label{ej:6.6}
    Describir explícitamente las clases de conjugación del grupo $D_4$.\\

    Consideramos el grupo $D_4$:
    \begin{align*}
        D_4 &= \{1, r, r^2, r^3, s, sr, sr^2, sr^3\} \\
        &= \{s^ir^j \mid i = 0, 1, j = 0, 1, 2, 3\}
    \end{align*}

    Tenemos que:
    \begin{align*}
        \Cl_{D_4}(1) &= \{(s^i r^j)1(s^i r^j)^{-1} \mid i = 0, 1, j = 0, 1, 2, 3\} = \{1\} \\
        \Cl_{D_4}(r) &= \{(s^i r^j)r(s^i r^j)^{-1} \mid i = 0, 1, j = 0, 1, 2, 3\} = \{s^ir^j\ r\ r^{-j}s^{-i} \mid i = 0, 1, j = 0, 1, 2, 3\} \\
        &= \{s^i r s^i \mid i = 0, 1\} = \{r, r^3\} = \Cl_{D_4}(r^3) \\
        \Cl_{D_4}(r^2) &= \{(s^i r^j)r^2(s^i r^j)^{-1} \mid i = 0, 1, j = 0, 1, 2, 3\} = \{s^i r^j\ r^2\ r^{-j}s^{-i} \mid i = 0, 1, j = 0, 1, 2, 3\} \\
        &= \{s^i r^2 s^{-i} \mid i = 0, 1\} = \{r^2\}\\
        \Cl_{D_4}(s) &= \{(s^i r^j)s(s^i r^j)^{-1} \mid i = 0, 1, j = 0, 1, 2, 3\}
        = \{s^ir^j\ s\ r^{-j}s^{-i} \mid i = 0, 1, j = 0, 1, 2, 3\} 
    \end{align*}

    Este último no es tan sencillo, puesto que $r$ y $s$ no conmutan. Calculamos en primer lugar para $s=0$, sabiendo que las clases de conjugación son cerradas para inversos.
    \begin{align*}
        r\ s\ r^{-1} &= r\ s\ r^3 = sr^6 = sr^2 \in \Cl_{D_4}(s) \\
        r^2\ s\ r^{-2} &= r^2\ s\ r^2 = sr^6r^2=s\in \Cl_{D_4}(s) \\
        r^3\ s\ r^{-3} &= r^3\ s\ r = sr^9r = sr^2 \in \Cl_{D_4}(s)
    \end{align*}

    Por otro lado, para $s=1$, tenemos que:
    \begin{equation*}
        s\ s\ s = s\qquad \text{ y } s\ sr^2\ s = r^2s = sr^6 = sr^2
    \end{equation*}

    Por tanto, $\Cl_{D_4}(s) = \{s, sr^2\} = \Cl_{D_4}(sr^2)$. Tan solo queda por tanto calcular la clase de conjugación de $sr$ y de $sr^3$.
    \begin{equation*}
        r\ sr\ r^{-1} = r\ sr\ r^3 = rs = sr^3 \in \Cl_{D_4}(sr)
    \end{equation*}

    Por tanto, tenemos que $\Cl_{D_4}(sr) = \Cl_{D_4}(sr^3)$. Como las clases de conjugación forman una partición de $D_4$, tenemos que:
    \begin{align*}
        \Cl_{D_4}(1) &= \{1\} \\
        \Cl_{D_4}(r) &= \{r, r^3\} \\
        \Cl_{D_4}(r^2) &= \{r^2\} \\
        \Cl_{D_4}(s) &= \{s, sr^2\} \\
        \Cl_{D_4}(sr) &= \{sr, sr^3\}
    \end{align*}
\end{ejercicio}

\begin{ejercicio}\label{ej:6.7}
    Se dice que la acción de un grupo finito $G$ sobre un conjunto $X$ es \emph{transitiva} si hay una sola órbita para esta acción (es decir, si para cada $x, y \in X$ existe algún $g \in G$ tal que $\prescript{g}{}{x} = y$). Demostrar que si $G$ actúa transitivamente sobre un conjunto $X$ con $n$ elementos, entonces $|G|$ es un múltiplo de $n$.\\

    Sea $x\in X$. Como las órbitas forman una partición de $X$ y hay una única órbita, tenemos que:
    \begin{equation*}
        n = |X| = |\Orb(x)|
    \end{equation*}

    Como además tenemos que $|\Orb(x)|=[G:\Stab_G(x)]$, tenemos que:
    \begin{equation*}
        |G| = n\cdot |\Stab_G(x)|
    \end{equation*}
    Por tanto, $|G|$ es un múltiplo de $n$.
\end{ejercicio}

\begin{ejercicio}\label{ej:6.8}
    Un subgrupo $G \leq S_n$ se dice \emph{transitivo} si la acción de $G$ sobre $\{1, 2, \ldots, n\}$ es transitiva. Encontrar todos los subgrupos transitivos de $S_3$ y $S_4$.
    \begin{enumerate}
        \item $S_3$.
        
        Consideramos la acción natural de $S_3$ sobre $\{1, 2, 3\}$ dada por:
        \Func{ac}{S_3 \times \{1, 2, 3\}}{\{1, 2, 3\}}{(\sigma, i)}{\prescript{\sigma}{}{i} = \sigma(i)}

        Consideramos ahora la restricción de la acción a $G\leq S_3$, que sigue siendo una acción.
        Buscamos ahora los subgrupos transitivos $G\leq S_3$. En primer lugar, por el Ejercicio anterior, sabemos que $|G|$ es un múltiplo de $3$. Además, como $|S_3| = 6$, tenemos que $|G|$ divide a $6$. Por tanto, $|G|\in \{3,6\}$. Es decir, $G\in \{A_3, S_3\}$. Comprobemos si estos son transitivos.

        Dados $x,y\in \{1, 2, 3\}$ distintos, consideramos el tercer elemento $z\in \{1, 2, 3\}$. Sea ahora $\sigma=(x\ y\ z)\in S_3\cap A_3$. Entonces:
        \begin{align*}
            \prescript{\sigma}{}{x} &= y
        \end{align*}

        Entonces, $S_3$ y $A_3$ son transitivos. Por tanto, los únicos subgrupos transitivos de $S_3$ son $S_3$ y $A_3$.

        \item $S_4$.
        
        Consideramos la acción natural de $S_4$ sobre $\{1, 2, 3, 4\}$ dada por:
        \Func{ac}{S_4 \times \{1, 2, 3, 4\}}{\{1, 2, 3, 4\}}{(\sigma, i)}{\prescript{\sigma}{}{i} = \sigma(i)}

        Consideramos ahora la restricción de la acción a $G\leq S_4$, que sigue siendo una acción.
        Buscamos ahora los subgrupos transitivos $G\leq S_4$. En primer lugar, por el Ejercicio anterior, sabemos que $|G|$ es un múltiplo de $4$. Además, como $|S_4| = 24$, tenemos que $|G|$ divide a $24$. Por tanto, $|G|\in \{4,8,12,24\}$.
        \begin{itemize}
            \item Si $|G|=24$, entonces $G=S_4$. Dados por tanto $x,y\in \{1, 2, 3, 4\}$ distintos, consideramos un tercer elemento $z\in \{1, 2, 3, 4\}\setminus \{x,y\}$. Entonces, tomando $\sigma=(x\ y\ z)\in S_4$:
            \begin{align*}
                \prescript{\sigma}{}{x} &= y
            \end{align*}
            Entonces, $S_4$ es transitivo.

            \item Si $|G|=12$, entonces $G=A_4$. Empleando el mismo razonamiento que en el caso anterior, tenemos que $\sigma\in A_4$ y, por tanto, $\prescript{\sigma}{}{x} = y$. Entonces, $A_4$ es transitivo.
            
            \item Si $|G|=8$, entonces es un $2-$subgrupo de Sylow de $S_4$. Calculemos cuántos $2-$subgrupos de Sylow de $S_4$ hay. Como $|S_4|=24=2^3\cdot 3$, notando por $n_2$ al número de $2-$subgrupos de Sylow de $S_4$, por el Segundo Teorema de Sylow tenemos que:
            \begin{equation*}
                n_2 \equiv 1 \mod 2 \qquad\land \qquad n_2 \mid 3
            \end{equation*}

            Por tanto, puede ser $n_2=1$ o $n_2=3$. Puesto que $S_4$ no contiene subgrupos de orden $8$ normales, tenemos que $n_2=3$. Por tanto, hay tres subgrupos de orden $8$ en $S_4$. Probando, llegamos a que estos son:
            \begin{itemize}
                \item $\langle (1\ 2\ 3\ 4), (1\ 3)\rangle$.
                
                Sea $a=(1\ 2\ 3\ 4)$ y $b=(1\ 3)$. Entonces, tenemos que:
                \begin{align*}
                    ab &= (1\ 2\ 3\ 4)(1\ 3) = (1\ 4)(2\ 3)\\
                    ba^3 &= (1\ 3)(1\ 2\ 3\ 4)^3
                    = (1\ 3)(1\ 4\ 3\ 2) = (1\ 4)(2\ 3)
                \end{align*}
                Por tanto, $ab=ba^3$. Por el Teorema de Dyck, este grupo es isomorfo a $D_4$, luego es de orden $8$. Veamos si es transitivo. Para ello, vemos que:
                \begin{align*}
                    a^0(1) &= 1 \qquad
                    a^1(1) = 2 \qquad
                    a^2(1) = 3 \qquad
                    a^3(1) = 4
                \end{align*}

                Por tanto, $\Orb(1)=\{1, 2, 3, 4\}$. Por tanto, como $\Orb(x)$ es una partición de $\{1, 2, 3, 4\}$, tenemos que la única órbita es $\{1, 2, 3, 4\}$. Por tanto, es transitivo.

                \item $\langle (1\ 3\ 2\ 4), (1\ 2)\rangle$.
                
                Sea $a=(1\ 2\ 3\ 4)$ y $b=(1\ 2)$. Entonces, tenemos que:
                \begin{align*}
                    ab &= (1\ 3\ 2\ 4)(1\ 2) = (1\ 4)(2\ 3)\\
                    ba^3 &= (1\ 2)(1\ 3\ 2\ 4)^3
                    = (1\ 2)(1\ 4\ 2\ 3) = (1\ 4)(2\ 3)
                \end{align*}
                Por tanto, $ab=ba^3$. Por el Teorema de Dyck, este grupo es isomorfo a $D_4$, luego es de orden $8$. Veamos si es transitivo. Para ello, vemos que:
                \begin{align*}
                    a^0(1) &= 1 \qquad
                    a^1(1) = 3 \qquad
                    a^2(1) = 2 \qquad
                    a^3(1) = 4
                \end{align*}

                Por tanto, $\Orb(1)=\{1, 2, 3, 4\}$. Por tanto, como $\Orb(x)$ es una partición de $\{1, 2, 3, 4\}$, tenemos que la única órbita es $\{1, 2, 3, 4\}$. Por tanto, es transitivo.

                \item $\langle (1\ 2\ 3\ 4), (2\ 4)\rangle$.
                
                Sea $a=(1\ 2\ 3\ 4)$ y $b=(2\ 4)$. Entonces, tenemos que:
                \begin{align*}
                    ab &= (1\ 2\ 3\ 4)(2\ 4) = (1\ 2)(3\ 4)\\
                    ba^3 &= (2\ 4)(1\ 2\ 3\ 4)^3
                    = (2\ 4)(1\ 4\ 3\ 2) = (1\ 2)(3\ 4)
                \end{align*}

                Por tanto, $ab=ba^3$. Por el Teorema de Dyck, este grupo es isomorfo a $D_4$, luego es de orden $8$. Veamos si es transitivo. Para ello, vemos que:
                \begin{align*}
                    a^0(1) &= 1 \qquad
                    a^1(1) = 2 \qquad
                    a^2(1) = 3 \qquad
                    a^3(1) = 4
                \end{align*}

                Por tanto, $\Orb(1)=\{1, 2, 3, 4\}$. Por tanto, como $\Orb(x)$ es una partición de $\{1, 2, 3, 4\}$, tenemos que la única órbita es $\{1, 2, 3, 4\}$. Por tanto, es transitivo.
            \end{itemize}

            \item Si $|G|=4$, entonces es cíclico o isomorfo a $V$.
            \begin{itemize}
                \item Sea $G\cong \bb{Z}_4$, y sea $a\in G$ un generador. Entonces, por ser $a$ una biyección en $\{1, 2, 3, 4\}$, tenemos que:
                \begin{equation*}
                    \{a^0(1), a^1(1), a^2(1), a^3(1)\} = \{1,2,3,4\}
                \end{equation*}
                Por tanto, $\Orb(1)=\{1, 2, 3, 4\}$. Por tanto, como $\Orb(x)$ es una partición de $\{1, 2, 3, 4\}$, tenemos que la única órbita es $\{1, 2, 3, 4\}$. Por tanto, es transitivo.

                \item Sea $G\cong V$. Entonces, está formado por la identidad y $3$ elementos de orden $2$ de forma que el producto de dos de ellos es el tercero. Los elementos de orden $2$ son transposiciones o productos de transposiciones. Como es de orden $4$, ha de generarse con dos elementos.
                \begin{itemize}
                    \item Si $G$ está generado por dos productos de transposiciones disjuntas, entonces $G=V$. Veamos sí es transitivo. Sean $i,j\in \{1, 2, 3, 4\}$ distintos. Entonces, sean $k,l\in \{1, 2, 3, 4\}\setminus \{i,j\}$ distintos, de forma que $(i\ j)(k\ l)\in G$. Entonces:
                    \begin{align*}
                        \prescript{(i\ j)(k\ l)}{}{i} &= j
                    \end{align*}
                    Por tanto, como $j$ era arbitrario, tenemos que:
                    \begin{equation*}
                        \Orb(i) = \{1, 2, 3, 4\}
                    \end{equation*}
                    Por tanto, como $\Orb(x)$ es una partición de $\{1, 2, 3, 4\}$, tenemos que la única órbita es $\{1, 2, 3, 4\}$. Por tanto, es transitivo.

                    \item Si $G$ está generado por dos transposiciones no disjuntas, entonces $\exists i,j,k\in \{1, 2, 3, 4\}$ distintos tales que:
                    \begin{equation*}
                        G = \langle (i\ j), (i\ k) \rangle
                    \end{equation*}
                    Entonces:
                    \begin{equation*}
                        (i\ j)(i\ k) = (i\ k\ j)
                    \end{equation*}
                    Por tanto, contendría un elemento de orden $3$, que no puede ser, puesto que $G$ es de orden $4$. Por tanto, no se puede dar este caso.

                    \item Si $G$ está generado por dos transposiciones disjuntas, entonces $\exists i,j,k,l\in \{1, 2, 3, 4\}$ distintos tales que:
                    \begin{equation*}
                        G = \langle (i\ j), (k\ l) \rangle
                    \end{equation*}

                    Entonces, tenemos que:
                    \begin{equation*}
                        G = \langle (i\ j), (k\ l) \rangle = \{1, (i\ j), (k\ l), (i\ j)(k\ l)\}
                    \end{equation*}

                    En este caso, $\Orb(i) = \{1,i,j\}\neq \{1, 2, 3, 4\}$. Por tanto, no es transitivo.

                    \item Si $G$ está generado por una transposición y un producto de transposiciones disjuntas, caben dos casos:
                    \begin{itemize}
                        \item $G=\langle (i\ j), (i\ j)(k\ l) \rangle$, donde $i,j,k,l\in \{1, 2, 3, 4\}$ son distintos. Entonces:
                        \begin{equation*}
                            (i\ j)(i\ j)(k\ l) = (k\ l)
                        \end{equation*}
                        Por tanto, $G=\langle (i\ j), (k\ l) \rangle$, que es un caso ya visto.
                        \item $G=\langle (i\ j), (i\ k)(j\ l) \rangle$, donde $i,j,k,l\in \{1, 2, 3, 4\}$ son distintos. Entonces:
                        \begin{equation*}
                            (i\ j)(i\ k)(j\ l) = (i\ k\ j\ l)
                        \end{equation*}
                        No obstante, este elemento es de orden $4$, por lo que no puede ser, puesto que $G$ sería cíclico. Por tanto, no se puede dar este caso.
                    \end{itemize}
                \end{itemize}
            \end{itemize}
        \end{itemize}
        Como hemos visto, los únicos subgrupos transitivos de $S_4$ son:
        \begin{itemize}
            \item $S_4$.
            \item $A_4$.
            \item Los tres subgrupos de orden $8$ isomorfos a $D_4$:
            \begin{itemize}
                \item $\langle (1\ 2\ 3\ 4), (1\ 3)\rangle$.
                \item $\langle (1\ 3\ 2\ 4), (1\ 2)\rangle$.
                \item $\langle (1\ 2\ 3\ 4), (2\ 4)\rangle$.
            \end{itemize}
            \item Los grupos cíclicos de orden $4$ y $V$.
        \end{itemize}
    \end{enumerate}
\end{ejercicio}

\begin{ejercicio}\label{ej:6.9}
    Sea $n\in \bb{N}$. Una \emph{partición} de $n$ es una sucesión no decreciente de enteros positivos cuya suma es $n$. Dada una permutación $\sigma \in S_n$, la descomposición en ciclos disjuntos (incluyendo los ciclos de longitud 1) de $\sigma = \gamma_1 \gamma_2 \cdots \gamma_r$ determina una partición $n_1, n_2, \ldots, n_r$ de $n$ donde cada $n_i$ es la longitud del ciclo $\gamma_i$. Dos permutaciones en $S_n$ se dice que son del mismo tipo si determinan la misma partición de $n$. Demostrar:
    \begin{enumerate}
        \item Dos elementos de $S_n$ son conjugados si y solo si son del mismo tipo.
        \begin{description}
            \item[$\Longrightarrow$)] Sean $\sigma,\tau\in S_n$ dos elementos conjugados; es decir, $\exists \gamma\in S_n$ tal que $\gamma\sigma\gamma^{-1}=\tau$. Consideramos ahora la descomposición en ciclos disjuntos (incluyendo los de longitud $1$) de $\sigma$:
            \begin{equation*}
                \sigma = \sigma_1\cdots\sigma_r
            \end{equation*}
            Por tanto, tenemos que:
            \begin{equation*}
                \tau=\gamma\sigma\gamma^{-1} = (\gamma\sigma_1\gamma^{-1})\cdots(\gamma\sigma_r\gamma^{-1})
            \end{equation*}

            Además, sabemos que la longitud de $\sigma_i$ coincide con la de $\gamma\sigma_i\gamma^{-1}$ para todo $i\in \{1,\ldots,r\}$. Por tanto, $\tau$ y $\gamma$ determinan la misma partición y por tanto son del mismo tipo.

            \item[$\Longleftarrow$)] Sean $\sigma,\tau\in S_n$ dos elementos del mismo tipo; y sea $n_1,\dots,n_r$ la partición de $n$ que determinan. Consideramos por tanto ambas particiones en ciclos disjuntos (incluyendo los ciclos de longitud uno):
            \begin{align*}
                \sigma &= \sigma_1\cdots\sigma_r
                = (a_{11} \ a_{12} \cdots a_{1n_1})(a_{21} \ a_{22} \cdots a_{2n_2})\cdots(a_{r1} \ a_{r2} \cdots a_{rn_r})\\
                \tau &= \tau_1\cdots\tau_r
                = (b_{11} \ b_{12} \cdots b_{1n_1})(b_{21} \ b_{22} \cdots b_{2n_2})\cdots(b_{r1} \ b_{r2} \cdots b_{rn_r})
            \end{align*}

            Consideramos ahora $\gamma\in S_n$ cuya representación matricial es:
            \begin{equation*}
                \gamma = \begin{pmatrix}
                    a_{11} & a_{12} & \cdots & a_{1n_1} & \cdots & a_{r1} & a_{r2} & \cdots & a_{rn_r}\\
                    b_{11} & b_{12} & \cdots & b_{1n_1} & \cdots & b_{r1} & b_{r2} & \cdots & b_{rn_r}
                \end{pmatrix}
            \end{equation*}

            Entonces, tenemos que:
            \begin{align*}
                \gamma\sigma\gamma^{-1} &= \gamma(\sigma_1\cdots\sigma_r)\gamma^{-1} = \gamma\sigma_1\cdots\gamma\sigma_r\gamma^{-1}\\
                &= (\gamma\sigma_1\gamma^{-1})(\gamma\sigma_2\gamma^{-1})\cdots(\gamma\sigma_r\gamma^{-1}) = \tau_1\tau_2\cdots\tau_r = \tau
            \end{align*}
            Por tanto, $\sigma$ y $\tau$ son conjugados.
        \end{description}

        \item El número de clases de conjugación de $S_n$ es igual al número de particiones de $n$.
        
        \begin{description}
            \item[$\leq$)] Sean $\sigma,\tau\in S_n$ dos elementos tal que $\Cl_{S_n}(\sigma)\neq \Cl_{S_n}(\tau)$. Entonces, no son conjugados. Por tanto, no son del mismo tipo y determinan distintas particiones de $n$. Por tanto, el número de clases de conjugación de $S_n$ es menor o igual al número de particiones de $n$.
            
            \item[$\geq$)] Veamos en primer lugar que, dada una partición de $n$, existe al menos un elemento de $S_n$ que determina dicha partición. Sea $n_1, n_2, \ldots, n_r$ la partición de $n$. Consideramos el siguiente elemento de $S_n$:
            \begin{equation*}
                \sigma = (1\ 2\ \cdots\ n_1)(n_1+1\ n_1+2\ \cdots\ n_1+n_2)\cdots(n_1+\cdots+n_{r-1}+1\ n_1+\cdots+n_{r-1}+2\ \cdots\ n)
            \end{equation*}
            Entonces, $\sigma$ es un elemento de $S_n$ que determina la partición $n_1, n_2, \ldots, n_r$. Por tanto, existe al menos un elemento de $S_n$ que determina cada partición de $n$.

            Ahora, dados dos elementos $\sigma,\tau\in S_n$ que determinan particiones distintas de $n$, tenemos que $\sigma$ y $\tau$ no son del mismo tipo. Por tanto, no son conjugados. Por tanto, $\Cl_{S_n}(\sigma)\neq \Cl_{S_n}(\tau)$. Por tanto, el número de clases de conjugación de $S_n$ es mayor o igual al número de particiones de $n$.
        \end{description}

        Como conclusión, tenemos que el número de clases de conjugación de $S_n$ es igual al número de particiones de $n$.
    \end{enumerate}
\end{ejercicio}

\begin{ejercicio}\label{ej:6.10}
    Calcular el número de clases de conjugación de $S_5$. Dar un representante de cada una y encontrar el orden de cada clase. Calcular el estabilizador de $(1\ 2\ 3)$ bajo la acción de conjugación de $S_5$ sobre sí mismo.\\


    Por el ejercicio anterior, sabemos que hay tantas clases de conjugación como particiones de $n$:
    \begin{equation*}
        \begin{array}{l|c|c}
            \text{Partición} & \text{Representante} & \text{Orden}\\ \hline
            1\ 1\ 1\ 1\ 1 & id_{5} & 1\\
            1\ 1\ 1\ 2 & (1\ 2) & 10\\
            1\ 2\ 2 & (1\ 2)(3\ 4) & 15\\
            1\ 1\ 3 & (1\ 2\ 3) & 20\\
            2\ 3 & (1\ 2)(3\ 4\ 5) & 20\\
            1\ 4 & (1\ 2\ 3\ 4) & 30\\
            5 & (1\ 2\ 3\ 4\ 5) & 24
        \end{array}
    \end{equation*}
    
    Para calcular el orden de cada clase, usamos que:
    \begin{equation*}
        |\Cl_{S_5}(\sigma)| = \dfrac{5!}{\prod\limits_{i=1}^5 m_i! \cdot i^{m_i}}
    \end{equation*}
    donde $m_i$ es el número de ciclos de longitud $i$ en la descomposición en ciclos disjuntos de $\sigma$. Por tanto, tenemos que:
    \begin{align*}
        |\Cl_{S_5}(id_5)| &= \dfrac{5!}{5! \cdot 1^5} = 1\\
        |\Cl_{S_5}((1\ 2))| &= \dfrac{5!}{3!\cdot 1^3 \cdot 1!\cdot 2^1} = \dfrac{120}{6\cdot 2} = 10\\
        |\Cl_{S_5}((1\ 2)(3\ 4))| &= \dfrac{5!}{2!\cdot 2^2} = \dfrac{120}{2\cdot 4} = 15\\
        |\Cl_{S_5}((1\ 2\ 3))| &= \dfrac{5!}{2!\cdot 1^2 \cdot 3^1} = \dfrac{120}{2\cdot 3} = 20\\
        |\Cl_{S_5}((1\ 2)(3\ 4\ 5))| &= \dfrac{5!}{1!\cdot 1^1 \cdot 3^1 \cdot 2^1} = \dfrac{120}{1\cdot 3\cdot 2} = 20\\
        |\Cl_{S_5}((1\ 2\ 3\ 4))| &= \dfrac{5!}{1!\cdot 1^1 \cdot 4^1} = \dfrac{120}{1\cdot 4} = 30\\
        |\Cl_{S_5}((1\ 2\ 3\ 4\ 5))| &= \dfrac{5!}{1!\cdot 5^1} = \dfrac{120}{1\cdot 5} = 24
    \end{align*}


    Calculamos ahora el estabilizador de $(1\ 2\ 3)$:
    \begin{align*}
        \Stab_{S_5}((1\ 2\ 3)) &= \{\gamma\in S_5\mid \gamma(1\ 2\ 3)\gamma^{-1} = (1\ 2\ 3)\}
        =\\&= \{\gamma\in S_5\mid (\gamma(1)\ \gamma(2)\ \gamma(3)) = (1\ 2\ 3) = (2\ 3\ 1) = (3\ 1\ 2)\}
    \end{align*}

    A simple vista, vemos que:
    \begin{equation*}
        id_5,\ (4\ 5),\ (1\ 2\ 3),\ (1\ 3\ 2),\ (1\ 2\ 3)(4\ 5),\ (1\ 3\ 2)(4\ 5)\in \Stab_{S_5}((1\ 2\ 3))
    \end{equation*}

    No obstante, podría haber más. Comprobemos que no:
    \begin{equation*}
        |\Stab_{S_5}((1\ 2\ 3))| = \dfrac{|S_5|}{|\Orb((1\ 2\ 3))|}= \dfrac{|S_5|}{|\Cl_{S_5}((1\ 2\ 3))|}
        = \dfrac{120}{20} = 6
    \end{equation*}

    Por tanto:
    \begin{equation*}
        \Stab_{S_5}((1\ 2\ 3))=\{id_5,\ (4\ 5),\ (1\ 2\ 3),\ (1\ 3\ 2),\ (1\ 2\ 3)(4\ 5),\ (1\ 3\ 2)(4\ 5)\}
    \end{equation*}
\end{ejercicio}

\begin{ejercicio}
    Sea $G$ un grupo finito y $\Phi: G \to \Perm(G)$ la representación regular izquierda (que corresponde a la acción de $G$ sobre sí mismo por traslación por la izquierda).
    \begin{enumerate}
        \item Demostrar que si $x$ es un elemento de $G$ de orden $n$ y $|G| = nm$, entonces $\Phi(x)$ es un producto de $n-$ciclos. Deducir que $\Phi(x)$ es una permutación impar si y solo si el orden de $x$ es par y el cociente del orden de $G$ y el de $x$ es impar.
        
        Sea $x\in G$ con $\ord(x)=n$. Entonces, $\Phi(x)\in \Perm(G)$, y como $|G|=nm$, tenemos que $\Phi(x)\in S_{nm}$. Sea ahora $k\in G$. Con vistas de estudiar la descomposición de $\Phi(x)$ en ciclos disjuntos, veamos las imágenes sucesivas de $k$ bajo $\Phi(x)$:
        \begin{align*}
            k &\mapsto xk \mapsto x^2k \mapsto \cdots \mapsto x^{n-1}k \mapsto x^n k = k
        \end{align*}
        Por tanto, $k$ pertenece al ciclo $(k\ xk\ x^2k\ \cdots\ x^{n-1}k)$ de $\Phi(x)$. Como $k$ era arbitrario, hemos visto que $\Phi(x)$ es producto de $n-$ciclos. Además, todos estos son trivialmente disjuntos.
        \begin{comment}
        hemos visto que $\Phi(x)$ tiene un ciclo de longitud $n$ que contiene a $k$. Por tanto, para cada $k\in G$, $\Phi(x)$ tiene un ciclo de longitud $n$ que contiene a $k$.
        \begin{itemize}
            \item Supongamos que $\Phi(x)$ tiene un ciclo de longitud $j<n$. Entonces, fijado $g\in G$ perteneciente a dicho ciclo, como el orden de un ciclo es su longitud, tenemos que:
            \begin{equation*}
                g = \Phi^j(g) = x^j g\Longrightarrow x^j = 1
                \Longrightarrow n\mid j
                \Longrightarrow n\leq j
            \end{equation*}
            Por tanto, hemos llegado a una contradicción, luego $\Phi(x)$ no tiene ciclos de longitud $j<n$.
            \item Supongamos que $\Phi(x)$ tiene un ciclo de longitud $j>n$. Entonces:
            \begin{equation*}
                g \neq \Phi^n(g) = x^n g\Longrightarrow x^n \neq 1
            \end{equation*}
            Por tanto, hemos llegado a una contradicción, luego $\Phi(x)$ no tiene ciclos de longitud $j>n$.
        \end{itemize}

        Por tanto, la descomposición de $\Phi(x)$ en ciclos disjuntos está formada por $n-$ciclos. Veamos ahora por cuántos $n-$ciclos está formada. Veamos en primer lugar que dos $n-$ciclos distintos han de ser disjuntos. Supongamos que $\Phi(x)$ tiene dos $n-$ciclos $C_1$ y $C_2$ tales que $\exists k\in C_1\cap C_2$. Entonces, por cómo actúa $\Phi(x)$, tenemos que:
        \begin{align*}
            k &\mapsto xk \mapsto x^2k \mapsto \cdots \mapsto x^{n-1}k \mapsto x^n k = k
        \end{align*}
        Por tanto, el único ciclo que contiene a $k$ es $(k\ xk\ x^2k\ \cdots\ x^{n-1}k)$, luego $C_1=C_2$.\\

        Por tanto, $\Phi(x)$ está formada por ciclos de longitud $n$ disjuntos. 
        \end{comment}
        
        Como en la representación de $\Phi(x)$ deben aparecer los $nm$ elementos de $G$, tenemos que el número de ciclos de longitud $n$ en la descomposición de $\Phi(x)$ es:
        \begin{equation*}
            \dfrac{|G|}{n} = \dfrac{nm}{n} = m
        \end{equation*}

        Por tanto, $\Phi(x)$ está formada por $m$ ciclos de longitud $n$ disjuntos. Como una permutación es par si y solo si el número de ciclos de longitud par es par, tenemos que:
        \begin{equation*}
            \veps(\Phi(x)) = -1 \iff \text{el número de ciclos de longitud par es impar}
        \end{equation*}

        Como el $0$ es par, al menos uno de los ciclos de longitud $n$ ha de ser par, luego $n$ ha de ser par. Además, como el número de ciclos de longitud $n$ es $m$, tenemos que $m$ ha de ser impar. Por tanto:
        \begin{equation*}
            \veps(\Phi(x)) = -1 \iff n\text{ es par y }m\text{ es impar}
        \end{equation*}
        Como $m=\dfrac{|G|}{n}$, se tiene lo pedido.

        \item Demostrar que si $Im(\Phi)$ contiene una permutación impar entonces $G$ tiene un subgrupo de índice 2.
        
        Supongamos que $Im(\Phi)$ contiene una permutación impar. Entonces, existe $x\in G$ tal que $\veps(\Phi(x))=-1$. Como tanto $\Phi$ como $\veps$ son homomorfismos, consideramos el homomorfismo composición:
        \begin{equation*}
            \veps\circ\Phi: G \to \{-1, 1\}
        \end{equation*}

        Veamos si ese homomorfismo es sobreyectivo. Como $\veps(\Phi(x))=-1$, tenemos que:
        \begin{equation*}
            (\veps\circ\Phi)(x) = -1\Longrightarrow -1\in Im(\veps\circ\Phi)
        \end{equation*}

        Por otro lado, considerando $1\in G$, tenemos que:
        \begin{equation*}
            (\veps\circ\Phi)(1) = \veps(\Phi(1)) = \veps(id_{G}) = 1
        \end{equation*}

        Por tanto, $Im(\veps\circ\Phi)=\{-1, 1\}$. Por el Primero Teorema de Isomorfía, tenemos que:
        \begin{equation*}
            \dfrac{G}{\ker(\veps\circ\Phi)}\cong Im(\veps\circ\Phi) = \{-1, 1\}
        \end{equation*}

        Como $G$ es finito, tenemos que:
        \begin{equation*}
            [G:\ker(\veps\circ\Phi)] = |Im(\veps\circ\Phi)| = 2
        \end{equation*}

        Por tanto, $\ker(\veps\circ\Phi)$ es un subgrupo de $G$ de índice $2$.       

        \item Demostrar que si $|G| = 2k$ con $k$ impar, entonces $G$ tiene un subgrupo de índice 2.
        \begin{observacion}
            Usar el Teorema de Cauchy para obtener un elemento de orden 2 y entonces usar los dos apartados anteriores.
        \end{observacion}

        Por el Teorema de Cauchy, como $2\mid |G|$, existe $x\in G$ tal que $\ord(x)=2$. Además, como $|G|=2k$ con $k$ impar; y $\ord(x)=2$ par, por el primer apartado, tenemos que $\Phi(x)$ es una permutación impar. Por el segundo apartado, tenemos que $G$ tiene un subgrupo de índice $2$.
    \end{enumerate}
\end{ejercicio}

\begin{ejercicio}\label{ej:6.12}
    Sea $G$ un $p-$grupo actuando sobre un conjunto finito $X$. Demostrar que
    \[
        |X| \equiv |\Fix(X)| \mod p.
    \]

    Sea $G$ un $p-$grupo finito y sea $X$ un conjunto finito sobre el que actúa. Como $G$ es finito, $\exists n\in \bb{N}$ tal que $|G|=p^n$.
    En vistas de aplicar la fórmula de clases, definimos $\Gamma$ como un conjunto que tiene un representante de cada órbita no unitaria de la acción de $G$ sobre $X$. Entonces, tenemos que:
    \begin{equation*}
        |X| = |\Fix(X)| + \sum_{x\in \Gamma} |\Orb(x)|
    \end{equation*}

    Demostrar lo pedido equivale a demostrar que:
    \begin{equation*}
        X-|\Fix(X)| = \sum_{x\in \Gamma} |\Orb(x)| \equiv 0 \mod p
    \end{equation*}

    Fijado $x\in \Gamma$, tenemos que $|\Orb(x)|>1$. Asímismo, como $|\Orb(x)|=[G:\Stab_G(x)]$, tenemos que $|\Orb(x)|\mid |G|=p^n$. Uniendo ambas afirmaciones, tenemos que $|\Orb(x)|=p^{k_x}$ con $k_x\in \{1,\ldots,n\}$.\\

    Por tanto, $|\Orb(x)|\equiv 0 \mod p$ para todo $x\in \Gamma$. Por tanto, la suma de los términos de la suma es congruente a $0$ módulo $p$. Por tanto:
    \begin{equation*}
        |X| - |\Fix(X)| = \sum_{x\in \Gamma} |\Orb(x)| \equiv 0 \mod p
    \end{equation*}
    como queríamos demostrar.
\end{ejercicio}

\begin{ejercicio}
    Sea $G$ un $2-$grupo finito que actúa sobre un conjunto finito $X$ cuya cardinalidad es un número impar. ¿Podemos afirmar que existe al menos un punto de $X$ que queda fijo bajo la acción de $G$? ¿Podemos decir lo mismo si $|X|$ es par?

    Por la fórmula de clases, definiendo $\Gamma$ como un conjunto que tiene un representante de cada órbita no unitaria de la acción de $G$ sobre $X$, tenemos que:
    \begin{equation*}
        |X| = |\Fix(X)| + \sum_{x\in \Gamma} |\Orb(x)|
    \end{equation*}

    Dado $x\in \Gamma$, tenemos que $|\Orb(x)|=[G:\Stab_G(x)]$. Como $G$ es un $2-$grupo, tenemos que $|G|=2^k$ con $k\in \bb{N}$. Por tanto, $|\Orb(x)|$ es una potencia de $2$. Además, como $x\in \Gamma$, tenemos que $|\Orb(x)|>1$. Por tanto, $|\Orb(x)|$ es un número par para todo $x\in \Gamma$. Por tanto, la suma de los términos de la suma es par. Como $|X|$ es impar, tenemos que $|\Fix(X)|$ es impar. Por tanto, existe al menos un punto de $X$ que queda fijo bajo la acción de $G$.\\

    Si $|X|$ es par, entonces $|\Fix(X)|$ es par, pero podría ser $0$. Por tanto, no podemos afirmar que existe al menos un punto de $X$ que queda fijo bajo la acción de $G$.
\end{ejercicio}

\begin{ejercicio}\label{ej:6.14}
    Sea $C_n = \langle a \mid a^n = 1 \rangle$ un grupo cíclico de orden $n$. Describir sus subgrupos de Sylow.\\


    Sea $n=p_1^{k_1}p_2^{k_2}\cdots p_m^{k_m}$ la factorización de $n$ en primos. Entonces, para cada $i\in \{1,\ldots,m\}$, tenemos que el $p_i-$subgrupo de Sylow de $C_n$ es único, puesto que $C_n$ es cíclico luego abeliano, y por tanto sus subgrupos son normales. Por tanto, el $p_i-$subgrupo de Sylow de $C_n$ es el único subgrupo de orden $p_i^{k_i}$ de $C_n$.
    \begin{equation*}
        P_{p_i} = \left\langle a^{\left(\frac{n}{(p_i^{k_i})}\right)} \right\rangle
    \end{equation*}
\end{ejercicio}

\begin{ejercicio}\label{ej:6.15}
    Sea $G$ un grupo finito y $|G| = pn$ con $p$ primo y $p > n$. Demostrar que $G$ contiene un subgrupo normal de orden $p$ y que todo subgrupo de $G$ de orden $p$ es normal en $G$.\\

    Buscamos obtener $n_p$. Como $p>n$, sabemos que $\mcd(p,n)=1$. Por tanto, por el Teorema de Sylow, tenemos que:
    \begin{align*}
        n_p &\equiv 1 \mod p \\
        n_p &\mid n
    \end{align*}

    Por tanto, $n_p\leq n<p$, luego $n_p=1$. Por tanto, existe un único $p-$subgrupo de Sylow de $G$ (de orden $p$), que es normal. Llamémoslo $P_p$, y tendrá orden $|P_p|=p$.\\

    Sea ahora $H$ un subgrupo de $G$ de orden $p$. Entonces, este es un $p-$subgrupo de Sylow de $G$, luego $H=P_p$.
\end{ejercicio}

\begin{ejercicio}\label{ej:6.16}
    Sea $H$ un subgrupo de un grupo finito $G$ con $[G : H] = p$ primo y $p$ el menor primo que divide a $|G|$. Demostrar que entonces $H$ es normal en $G$.\\

    Como no sabemos si $H$ es normal en $G$, no podemos considerar el grupo cociente pero sí el conjunto de las clases laterales por la izquierda de $H$ en $G$, que denotamos por $G/\sim_H$. Consideramos la acción de $G$ sobre $G/\sim_H$ por traslación por la izquierda, y consideramos su representación por permutaciones
    \begin{equation*}
        \Phi: G \to \Perm(G/\sim_H)
    \end{equation*}

    Como $|G/\sim_H| = [G:H] = p$, tenemos que $\Phi$ es un homomorfismo de grupos entre $G$ y $S_p$:
    \begin{equation*}
        \Phi: G \to S_p
    \end{equation*}

    En vistas de aplicar el Primer Teorema de Isomorfía, calculamos el núcleo de $\Phi$:
    \begin{align*}
        \ker(\Phi) &= \{g\in G\mid \Phi(g) = id_{S_p}\}\\
        &= \{g\in G\mid g\cdot (aH) = (aH)\text{ para todo }a\in G\} \subset\\
        &\subset \{g\in G\mid gH = H\} = \{g\in G\mid g\in H\} = H
    \end{align*}

    Por el Primer Teorema de Isomorfía, tenemos que:
    \begin{equation*}
        G/\ker(\Phi) \cong Im(\Phi) \leq S_p
    \end{equation*}

    Por tanto, como $|S_p| = p!$ con $p$ primo, tenemos que $|G/\ker(\Phi)|$ divide a $p!$.
    Por otro lado, $|G/\ker(\Phi)|$ divide a $|G|$. Como $p$ es el menor primo que divide a $|G|$, tenemos que $|G/\ker(\Phi)|=p$.
    Por último, como $\ker(\Phi)\lhd G$ y $\ker(\Phi)\leq H$, tenemos que $\ker(\Phi)\lhd H$. Por tanto:
    \begin{equation*}
        p = [G:\ker(\Phi)] = [G:H]\cdot [H:\ker(\Phi)] = p\cdot [H:\ker(\Phi)]
        \Longrightarrow [H:\ker(\Phi)]=1
    \end{equation*}

    Por tanto, $|\ker(\Phi)|=|H|$. Como $\ker(\Phi)\leq H$, tenemos que $\ker(\Phi)=H$. Por tanto, $\ker(\Phi)=H$ es un subgrupo normal de $G$, concluyendo así la demostración.
\end{ejercicio}

\begin{ejercicio}\label{ej:6.17}
    Sea $p$ un número primo. Demostrar:
    \begin{enumerate}
        \item Todo grupo no abeliano de orden $p^3$ tiene un centro de orden $p$.
        
        Sea $G$ un grupo no abeliano de orden $p^3$. Entonces, $G$ es un $p-$grupo. Por ser $Z(G)<G$, tenemos que $|Z(G)|=p^k$ con $k\in \{0,1,2,3\}$.
        \begin{itemize}
            \item Por ser un $p-$grupo, $Z(G)$ es no trivial, luego $k>0$.
            \item Por no ser abeliano, $Z(G)\neq G$, luego $k<3$.
            \item Por ser un $p-$grupo, $|Z(G)|\neq p^{3-1}$, luego $k<2$.
        \end{itemize}
        Por tanto, $k=1$. Por tanto, $|Z(G)|=p$.
        \item Existen únicamente dos grupos no isomorfos de orden $p^2$.
        
        Todo grupo de orden $p^2$ es abeliano. Consideramos el grupo cíclico $C_{p^2}$ y el grupo directo $C_p\times C_p$. Estos son de orden $p^2$ y no son isomorfos entre sí, puesto que uno es cíclico y el otro no ($\mcd(p,p)=p\neq 1$).

        % // TODO: Hay más?
        \item Todo subgrupo normal de orden $p$ de un $p-$grupo finito está contenido en el centro.
        
        Sea $G$ un $p-$grupo finito y sea $H\lhd G$ un subgrupo normal de orden $p$. Como es de orden $p$, $\exists h\in H$ con $\ord(h)=p$, de forma que:
        \begin{equation*}
            H = \langle h \rangle
        \end{equation*}

        Por tanto, para ver que $H\leq Z(G)$, bastará con ver que $h\in Z(G)$. Fijado $g\in G$, buscamos ver que $gh=hg$. Como $H\lhd G$, tenemos que:
        \begin{equation*}
            ghg^{-1} \in H
            \Longrightarrow
            \exists k\in \{0,\ldots,p-1\}\text{ tal que }ghg^{-1} = h^k
        \end{equation*}

        Si demostramos que $k=1$, lo tendremos.
        \begin{itemize}
            \item Supongamos $k=0$. Entonces, $ghg^{-1}=1$, luego $gh=g$ y $h=1$, luego $\ord(h)=1\neq p$, lo cual es una contradicción.
            \item Supongamos $k>1$. Como $g\in G$ y $G$ es un $p-$grupo, $\exists m\in \bb{N}$ tal que $\ord(g)=p^m$. Entonces, tenemos que:
            % // TODO: Hacer
        \end{itemize}

    \end{enumerate}
\end{ejercicio}

\begin{ejercicio}\label{ej:6.18}
    Demostrar que si $N\lhd G$ y $N$ y $G/N$ son $p-$grupos entonces $G$ es un $p-$grupo.\\

    Para demostrar que $G$ es un $p-$grupo, puesto que $G$ no tiene por qué ser finito, hemos de comprobar que el orden de todo elemento de $G\setminus \{1\}$ es una potencia de $p$. Sea $g\in G\setminus \{1\}$ un elemento cualquiera. Distinguimos dos casos:
    \begin{itemize}
        \item Si $g\in N$, entonces como $N$ es un $p-$grupo y $g\neq 1$, tenemos que $\ord(g)$ es una potencia de $p$.
        \item Si $g\notin N$, entonces como $G/N$ es un $p-$grupo y $gN\neq N$, tenemos que $\ord(gN)$ es una potencia de $p$. Por tanto, existe $k\in \bb{N}$ tal que:
        \begin{equation*}
            (gN)^{p^k} = N
            \Longrightarrow
            g^{p^k}N = N
            \Longrightarrow
            g^{p^k} \in N
        \end{equation*}
        Si $g^{p^k}=1$, entonces $\ord(g)$ divide a $p^k$, y como $g\neq 1$, tenemos que $\ord(g)$ es una potencia de $p$. Si $g^{p^k}\neq 1$, entonces como $N$ es un $p-$grupo, tenemos que $\ord(g^{p^k})$ es una potencia de $p$. Por tanto, $\exists k'\in \bb{N}$ tal que:
        \begin{equation*}
            \left(g^{p^k}\right)^{p^{k'}} = 1
            \Longrightarrow
            g^{p^{k+k'}} = 1
        \end{equation*}
        Por tanto, $\ord(g)$ divide a $p^{k+k'}$, luego $\ord(g)$ es una potencia de $p$.
    \end{itemize}
\end{ejercicio}

\begin{ejercicio}\label{ej:6.19}
    Si $G$ es un grupo de orden $p^n$, $p$ primo, demostrar que para todo $k$, $0 \leq k \leq n$, existe un subgrupo normal de $G$ de orden $p^k$.\\

    Demostramos por inducción sobre $n$.
    \begin{itemize}
        \item Para $n=1$, tenemos que $|G|=p$, luego $G$ es cíclico, luego abeliano, luego todo subgrupo de $G$ es normal.
        \item Supongamos que para todo $p-$grupo de orden $p^m$, con $m<n$, se cumple que para todo $k$, $0\leq k\leq m$, existe un subgrupo normal de orden $p^k$.\\
        
        Sea $G$ un $p-$grupo de orden $p^n$, y consideramos su centro $Z(G)$. Como $Z(G)\neq \{1\}$ y $Z(G)<G$, tenemos que $|Z(G)|=p^k$ con $k\in \{1,\ldots,n\}$. Como $p\mid |Z(G)|$, por el Teorema de Cauchy, existe $N<Z(G)$ tal que $|N|=p$. Como $Z(G)\lhd G$, se tiene que $N\lhd G$, lo que nos permite considerar:
        \begin{equation*}
            |G/N| = \dfrac{|G|}{|N|} = \dfrac{p^n}{p} = p^{n-1}
        \end{equation*}
        Por tanto, $G/N$ es un $p-$grupo de orden $p^{n-1}$. Por la hipótesis de inducción, tenemos que para todo $k$, $0\leq k\leq n-1$, existe $L_k\lhd G/N$ tal que $|L_k|=p^k$. Por el Tercer Teorema de Isomorfía, tenemos que, para cada $k\in \{0,\ldots,n-1\}$, existe $H_k<G$, con $N\lhd H_k\lhd G$, tal que $L_k=H_k/N$. Por tanto, tenemos que:
        \begin{equation*}
            |H_k| = |H_k/N| \cdot |N| = |L_k| \cdot |N| = p^k\cdot p = p^{k+1}
        \end{equation*}

        Por tanto, para cada $k'\in \{1,\ldots,n\}$, existe $W_{k'}=H_{k'-1}$ tal que $|W_{k'}|=p^{k'}$ y $W_{k'}\lhd G$. Falta ver el resultado para $k=0$, pero esto es directo tomando $W_0 = \{1\}$, que es un subgrupo normal de $G$ de orden $1=p^0$.
        Por tanto, hemos visto que para todo $k\in \{0,\ldots,n\}$, existe un subgrupo normal de $G$ de orden $p^k$.
    \end{itemize}
    Por tanto, hemos demostrado que para todo $p-$grupo $G$ de orden $p^n$, con $p$ primo, y para todo $k\in \{0,\ldots,n\}$, existe un subgrupo normal de $G$ de orden $p^k$.
\end{ejercicio}

\begin{ejercicio}\label{ej:6.20}
    Hallar todos los subgrupos de Sylow de los grupos $S_3$ y $S_4$.
    \begin{observacion}
        Para los $2-$subgrupos de Sylow de $S_4$, observar primero que todos deben contener al subgrupo de Klein $V$, y, al menos, una trasposición $\tau$, y que como consecuencia se pueden obtener como producto de $V$ por el grupo cíclico generado por $\tau$.
    \end{observacion}
    \begin{enumerate}
        \item $S_3$.
        
        Sabemos que $|S_3|=6=2\cdot 3$. Calculamos los $p-$subgrupos de Sylow, con $p\in \{2,3\}$.
        \begin{itemize}
            \item $2-$subgrupo de Sylow.
            
            Por el Segundo Teorema de Sylow, tenemos que:
            \begin{equation*}
                n_2 \equiv 1 \mod 2 \qquad n_2 \mid 3
            \end{equation*}
            Por tanto, $n_2\in \{1,3\}$.
        \end{itemize}
        Como hay más de un grupo de orden $2$ en $S_3$, tenemos que $n_2=3$. Estos grupos son:
        \begin{align*}
            \langle (1\ 2) \rangle, \quad \langle (1\ 3) \rangle, \quad \langle (2\ 3) \rangle
        \end{align*}
        \begin{itemize}
            \item $3-$subgrupo de Sylow.
            
            Por el Segundo Teorema de Sylow, tenemos que:
            \begin{equation*}
                n_3 \equiv 1 \mod 3 \qquad n_3 \mid 2
            \end{equation*}
            Por tanto, $n_3=1$. El único $3-$subgrupo de Sylow de $S_3$ es:
            \begin{equation*}
                P_3 = \langle (1\ 2\ 3) \rangle
            \end{equation*}
        \end{itemize}

        \item $S_4$.
        
        Sabemos que $|S_4|=24=2^3\cdot 3$. Calculamos los $p-$subgrupos de Sylow, con $p\in \{2,3\}$.
        \begin{itemize}
            \item $3-$subgrupo de Sylow.
            
            Por el Segundo Teorema de Sylow, tenemos que:
            \begin{equation*}
                n_3 \equiv 1 \mod 3 \qquad n_3 \mid 8
            \end{equation*}
            Por tanto, $n_3\in \{1,4\}$. Como hay más de un grupo de orden $3$ en $S_4$, tenemos que $n_3=4$. Estos grupos son:
            \begin{align*}
                \langle (1\ 2\ 3) \rangle, \quad \langle (1\ 2\ 4) \rangle, \quad \langle (1\ 3\ 4) \rangle, \quad \langle (2\ 3\ 4) \rangle
            \end{align*}

            \item $2-$subgrupo de Sylow.
            
            Por el Segundo Teorema de Sylow, tenemos que:
            \begin{equation*}
                n_2 \equiv 1 \mod 2 \qquad n_2 \mid 3
            \end{equation*}
            Por tanto, $n_2\in \{1,3\}$. En el Ejercicio~\ref{ej:6.8}, vimos que estos grupos son:
            \begin{itemize}
                \item $\langle (1\ 2\ 3\ 4), (1\ 3)\rangle$.
                \item $\langle (1\ 3\ 2\ 4), (1\ 2)\rangle$.
                \item $\langle (1\ 2\ 3\ 4), (2\ 4)\rangle$.
            \end{itemize}
            
        \end{itemize}
    \end{enumerate}
\end{ejercicio}

\begin{ejercicio}\label{ej:6.21}
    Hallar todos los subgrupos de Sylow de los grupos $\bb{Z}_{600}$, $Q_2$, $D_5$, $D_6$, $A_4$, $A_5$, $S_5$.
    \begin{enumerate}
        \item $\bb{Z}_{600}$.
        
        Tenemos $600=2^3\cdot 3\cdot 5^2$. Calculamos los $p-$subgrupos de Sylow, para valores de $p\in \{2,3,5\}$. Como $\bb{Z}_{600}$ es cíclico, en particular es abeliano, y por tanto sus subgrupos son normales, luego son únicos.
        \begin{itemize}
            \item $2-$subgrupo de Sylow.
            
            Es un grupo cíclico de orden $8$, luego es isomorfo a $\bb{Z}_8$. De hecho:
            \begin{equation*}
                P_2 = \langle 3\cdot 5^2 \rangle = \langle 75 \rangle \cong \bb{Z}_8
            \end{equation*}

            \item $3-$subgrupo de Sylow.
            
            Es un grupo cíclico de orden $3$, luego es isomorfo a $\bb{Z}_3$. De hecho:
            \begin{equation*}
                P_3 = \langle 2^3\cdot 5^2 \rangle = \langle 200 \rangle \cong \bb{Z}_3
            \end{equation*}
            \item $5-$subgrupo de Sylow.
            
            Es un grupo cíclico de orden $25$, luego es isomorfo a $\bb{Z}_{25}$. De hecho:
            \begin{equation*}
                P_5 = \langle 2^3\cdot 3 \rangle = \langle 24 \rangle \cong \bb{Z}_{25}
            \end{equation*}
        \end{itemize}

        \item $Q_2$.
        
        Sabemos que $|Q_2|=8=2^3$. Además, como $Q_2\lhd Q_2$, el único $2-$subgrupo de Sylow de $Q_2$ es $Q_2$ mismo. Por tanto, el único subgrupo de Sylow de $Q_2$ es $Q_2$.

        \item $D_5$.
        
        Sabemos que $|D_5|=10=2\cdot 5$. Calculamos los $p-$subgrupos de Sylow, con $p\in \{2,5\}$.
        \begin{itemize}
            \item $2-$subgrupos de Sylow.
            
            Por el Segundo Teorema de Sylow, tenemos que:
            \begin{equation*}
                n_2 \equiv 1 \mod 2 \qquad n_2 \mid 5
            \end{equation*}
            Por tanto, $n_2\in \{1,5\}$. Se tiene que $n_2=5$, puesto que hay $5$ elementos de orden $2$ en $D_5$. Estos grupos son:
            \begin{align*}
                \langle sr^i \rangle \qquad \forall i\in \{0,1,2,3,4\}
            \end{align*}

            \item $5-$subgrupo de Sylow.
            
            Sea $H$ un $5-$subgrupo de Sylow de $D_5$. Como $|D_5|=10$ y $|H|=5$, tenemos que $[D_5:H]=2$, luego $H$ es normal en $D_5$, luego es el único $5-$subgrupo de Sylow de $D_5$. Como además $5$ es primo, $H$ es cíclico. Por tanto:
            \begin{equation*}
                H = \langle r \rangle
            \end{equation*}
        \end{itemize}

        \item $D_6$.
        
        Sabemos que $|D_6|=12=2^2\cdot 3$. Calculamos los $p-$subgrupos de Sylow, con $p\in \{2,3\}$.
        \begin{itemize}
            \item $2-$subgrupos de Sylow.
            
            Por el Segundo Teorema de Sylow, tenemos que:
            \begin{equation*}
                n_2 \equiv 1 \mod 2 \qquad n_2 \mid 3
            \end{equation*}
            Por tanto, $n_2\in \{1,3\}$. Como no hay elementos de orden $4$ en $D_6$, no puede ser cíclico. Por tanto, ha de estar generado por más de un elemento de orden 2:
            \begin{align*}
                \langle r^3,s\rangle &= \{1,r^3,s,sr^3\}\\
                \langle r^3,sr\rangle &= \{1,r^3,sr,sr^4\}\\
                \langle r^3,sr^2\rangle &= \{1,r^3,sr,sr^5\}
            \end{align*}

            Como estos son tres $2-$subgrupos de Sylow de $D_6$, estos son los únicos.


            \item $3-$subgrupos de Sylow.
            
            Por el Segundo Teorema de Sylow, tenemos que:
            \begin{equation*}
                n_3 \equiv 1 \mod 3 \qquad n_3 \mid 4
            \end{equation*}
            Por tanto, $n_3\in \{1,4\}$. Como además los subgrupos son de orden $3$, son cíclicos, luego buscamos elementos de orden $3$ en $D_6$. Todos los elementos de la forma $sr^i$ con $i\in \{0,\dots,5\}$ tienen orden 2. Por tanto, el único $3-$subgrupo es:
            \begin{equation*}
                \langle r^2\rangle
            \end{equation*}
        \end{itemize}

        \item $A_4$.
        
        Sabemos que $|A_4|=12=2^2\cdot 3$. Calculamos los $p-$subgrupos de Sylow, con $p\in \{2,3\}$.
        \begin{itemize}
            \item $2-$subgrupos de Sylow.
            
            Como $V$ es un $2-$subgrupo de Sylow de $A_4$ y $V\lhd A_4$, tenemos que $V$ es el único $2-$subgrupo de Sylow de $A_4$.
            \item $3-$subgrupos de Sylow.
            
            Por el Segundo Teorema de Sylow, tenemos que:
            \begin{equation*}
                n_3 \equiv 1 \mod 3 \qquad n_3 \mid 4
            \end{equation*}
            Por tanto, $n_3\in \{1,4\}$. Como $A_4$ tiene $8$ elementos de orden $3$, tenemos que $n_3=4$. Por tanto, los $3-$subgrupos de Sylow son:
            \begin{align*}
                \langle (1\ 2\ 3) \rangle &= \{1,(1\ 2\ 3),(1\ 3\ 2)\}\\
                \langle (1\ 2\ 4) \rangle &= \{1,(1\ 2\ 4),(1\ 4\ 2)\}\\
                \langle (1\ 3\ 4) \rangle &= \{1,(1\ 3\ 4),(1\ 4\ 3)\}\\
                \langle (2\ 3\ 4) \rangle &= \{1,(2\ 3\ 4),(2\ 4\ 3)\}
            \end{align*}
        \end{itemize}
        \item $A_5$.
        
        Sabemos que $|A_5|=60=2^2\cdot 3\cdot 5$. Calculamos los $p-$subgrupos de Sylow, con $p\in \{2,3,5\}$.
        \begin{itemize}
            \item $2-$subgrupos de Sylow.
            
            Por el Segundo Teorema de Sylow, tenemos que:
            \begin{equation*}
                n_2 \equiv 1 \mod 2 \qquad n_2 \mid 15
            \end{equation*}
            Por tanto, $n_2\in \{1,3,5,15\}$.
            En $A_5$ no hay elementos de orden $4$, y los únicos de orden $2$ son lo productos de transposiciones. Veamos cuántas hay:
            \begin{equation*}
                |\Cl_{A_5}((1\ 2)(3\ 4))| = \frac{5!}{2^2\cdot 2!} = \frac{120}{8} = 15
            \end{equation*}
            
            Por tanto, como mínimo habrá estos $5$ $2-$subgrupos de Sylow:
            \begin{align*}
                \langle (1\ 2)(3\ 4),\ (1\ 3)(2\ 4) \rangle
                &= \{1,(1\ 2)(3\ 4),(1\ 3)(2\ 4),(1\ 4)(2\ 3)\}\\
                \langle (1\ 2)(3\ 5),\ (1\ 3)(2\ 5) \rangle
                &= \{1,(1\ 2)(3\ 5),(1\ 3)(2\ 5),(1\ 5)(2\ 3)\}\\
                \langle (1\ 2)(4\ 5),\ (1\ 4)(2\ 5) \rangle
                &= \{1,(1\ 2)(4\ 5),(1\ 4)(2\ 5),(1\ 5)(2\ 4)\}\\
                \langle (1\ 3)(4\ 5),\ (1\ 4)(3\ 5) \rangle
                &= \{1,(1\ 3)(4\ 5),(1\ 4)(3\ 5),(1\ 5)(2\ 3)\}\\
                \langle (2\ 3)(4\ 5),\ (2\ 4)(3\ 5) \rangle
                &= \{1,(2\ 3)(4\ 5),(2\ 4)(3\ 5),(2\ 5)(3\ 4)\}
            \end{align*}

            Se comprueba que, efectivamente, estos son $5$ $2-$subgrupos de Sylow de $A_5$. Por tanto, $n_2=5$.

            \item $3-$subgrupos de Sylow.
            Por el Segundo Teorema de Sylow, tenemos que:
            \begin{equation*}
                n_3 \equiv 1 \mod 3 \qquad n_3 \mid 20
            \end{equation*}
            Por tanto, $n_3\in \{1,4,10\}$. Veamos cuántos elementos de orden $3$ hay en $A_5$:
            \begin{equation*}
                |\Cl_{A_5}((1\ 2\ 3))| = \frac{5!}{3!\cdot 2} = 20
            \end{equation*}

            Por tanto, hay $10$ $3-$subgrupos de Sylow, que son:
            \begin{align*}
                \langle (1\ 2\ 3) \rangle &= \{1,(1\ 2\ 3),(1\ 3\ 2)\}\\
                \langle (1\ 2\ 4) \rangle &= \{1,(1\ 2\ 4),(1\ 4\ 2)\}\\
                \langle (1\ 2\ 5) \rangle &= \{1,(1\ 2\ 5),(1\ 5\ 2)\}\\
                \langle (1\ 3\ 4) \rangle &= \{1,(1\ 3\ 4),(1\ 4\ 3)\}\\
                \langle (1\ 3\ 5) \rangle &= \{1,(1\ 3\ 5),(1\ 5\ 3)\}\\
                \langle (1\ 4\ 5) \rangle &= \{1,(1\ 4\ 5),(1\ 5\ 4)\}\\
                \langle (2\ 3\ 4) \rangle &= \{1,(2\ 3\ 4),(2\ 4\ 3)\}\\
                \langle (2\ 3\ 5) \rangle &= \{1,(2\ 3\ 5),(2\ 5\ 3)\}\\
                \langle (2\ 4\ 5) \rangle &= \{1,(2\ 4\ 5),(2\ 5\ 4)\}\\
                \langle (3\ 4\ 5) \rangle &= \{1,(3\ 4\ 5),(3\ 5\ 4)\}
            \end{align*}

            \item $5-$subgrupos de Sylow.
            
            Por el Segundo Teorema de Sylow, tenemos que:
            \begin{equation*}
                n_5 \equiv 1 \mod 5 \qquad n_5 \mid 12
            \end{equation*}
            Por tanto, $n_5\in \{1,6\}$. Veamos cuántos elementos de orden $5$ hay en $A_5$:
            \begin{equation*}
                |\Cl_{A_5}((1\ 2\ 3\ 4\ 5))| = \frac{5!}{5} = 24
            \end{equation*}
            Por tanto, hay $6$ $5-$subgrupos de Sylow, que son:
            \begin{align*}
                \langle (1\ 2\ 3\ 4\ 5)\rangle\\
                \langle (1\ 2\ 3\ 5\ 4)\rangle\\
                \langle (1\ 2\ 5\ 3\ 4)\rangle\\
                \langle (1\ 5\ 2\ 3\ 4)\rangle\\
                \langle (1\ 2\ 4\ 3\ 5)\rangle\\
                \langle (1\ 2\ 4\ 5\ 3)\rangle
            \end{align*}


        \end{itemize}


        \item $S_5$.
        
        % // TODO: Hacer
    \end{enumerate}
\end{ejercicio}

\begin{ejercicio}\label{ej:6.22}
    Demostrar que $D_4$ es isomorfo a los $2-$subgrupos de Sylow de $S_4$.
    \begin{observacion}
        Considerar la representación asociada a la acción de $D_4$ sobre los vértices del cuadrado.
    \end{observacion}
    
    
    Este ejercicio ya se resolvió en el Ejercicio~\ref{ej:6.8}, donde vimos que los $2-$subgrupos de Sylow de $S_4$ son isomorfos a $D_4$ aplicando el Teorema de Dyck.
\end{ejercicio}

\begin{ejercicio}\label{ej:6.23}
    Demostrar que todo grupo de orden $12$ con más de un $3-$subgrupo de Sylow es isomorfo al grupo alternado $A_4$.
    \begin{observacion}
        Considerar la acción por traslación de un tal grupo sobre el conjunto de clases módulo $P$, siendo $P$ un $3-$subgrupo de Sylow. Probar que dicha acción es fiel.
    \end{observacion}
    
    Sea $G$ un grupo de orden $12$ con más de un $3-$subgrupo de Sylow. Por el Segundo Teorema de Sylow, tenemos que:
    \begin{equation*}
        n_3 \equiv 1 \mod 3 \qquad n_3 \mid 4
    \end{equation*}
    Por tanto, $n_3\in \{1,4\}$. Como $G$ tiene más de un $3-$subgrupo de Sylow, tenemos que $n_3=4$. Sean por tanto:
    \begin{align*}
        \Syl_3 = \{P_1,P_2,P_3,P_4\}
    \end{align*}

    Como cada $P_i$ es un grupo de orden $3$, es cíclico. De esta forma, supongamos que $\exists x\neq 1$ tal que $x\in P_i\cap P_j$ con $i\neq j$. Entonces, tenemos que:
    \begin{equation*}
        P_i = \langle x \rangle = \{1,x,x^2\} = P_j
    \end{equation*}
    Por tanto, $P_i=P_j$, lo cual es una contradicción. Por tanto, tenemos que:
    \begin{equation*}
        P_1\cap P_2 = P_1\cap P_3 = P_1\cap P_4 = P_2\cap P_3 = P_2\cap P_4 = P_3\cap P_4 = \{1\}
    \end{equation*}
    Por tanto, los $3-$subgrupos de Sylow son disjuntos dos a dos.\\

    Consideramos la acción de $G$ sobre el conjunto de clases módulo $P_1$. Como $P_1$ no es normal en $G$, no podemos considerar el grupo cociente, pero consideramos el conjunto de las clases por la izquierda:
    \begin{equation*}
        G/{\sim}_{P_1} = \{gP_1\mid g\in G\}
    \end{equation*}
    
    Sea por tanto la siguiente acción:
    \Func{ac}{G\times G/{\sim}_{P_1}}{G/{\sim}_{P_1}}{(g,hP_1)}{(gh)P_1}

    Veamos que está bien definida. Sea $g\in G$ y $h_1P_1,h_2P_1\in G/{\sim}_{P_1}$ tales que $h_1P_1=h_2P_1$. Entonces:
    \begin{equation*}
        (gh_1)P_1 = (gh_2)P_1
        \iff (gh_1)^{-1}(gh_2) \in P_1
        \iff h_1^{-1}h_2 \in P_1
        \iff h_1P_1 = h_2P_1
    \end{equation*}

    Por tanto, la acción está bien definida. Veamos que efectivamente es una acción:
    \begin{align*}
        \prescript{1}{}{hP_1} &= (1h)P_1 = hP_1\qquad \forall hP_1\in G/{\sim}_{P_1}\\
        \prescript{g_1}{}{\left(\prescript{g_2}{}{hP_1}\right)} &= \prescript{g_1}{}{(g_2h)P_1} = (g_1g_2h)P_1 = \prescript{g_1g_2}{}{hP_1}\qquad \forall g_1,g_2\in G,\ hP_1\in G/{\sim}_{P_1}
    \end{align*}

    Por tanto, consideramos su representación por permutaciones asociada a la acción:
    \Func{\Phi}{G}{\Perm(G/{\sim}_{P_1})}{g}{\prescript{g}{}{\left(\cdot P_1\right)}}


    Veamos el cardinal del conjunto de clases:
    \begin{equation*}
        |G/{\sim}_{P_1}| = [G:P_1] = \frac{|G|}{|P_1|} = \frac{12}{3} = 4
    \end{equation*}

    Por tanto, tenemos que:
    \Func{\Phi}{G}{S_4}{g}{\left(g(\cdot)\right)P_1}

    Calculamos que se trata de una acción fiel:
    \begin{align*}
        \ker(\Phi) &= \{g\in G\mid \prescript{g}{}{\left(\cdot P_1\right)} = Id_{G/{\sim}_{P_1}}\}\\
        &= \{g\in G\mid \prescript{g}{}{\left(hP_1\right)} = hP_1\ \forall hP_1\in G/{\sim}_{P_1}\}\\
        &= \{g\in G\mid (gh)P_1 = hP_1\ \forall hP_1\in G/{\sim}_{P_1}\}\\
        &= \{g\in G\mid h^{-1}gh \in P_1\ \forall h\in G\}\\
        &= \{g\in G\mid g\in hP_1h^{-1}\ \forall h\in G\}\\
    \end{align*}

    Por el Segundo Teorema de Sylow, todos los $3-$subgrupos de Sylow son conjugados entre sí, luego $hP_1h^{-1}$ es un $3-$subgrupo de Sylow de $G$ para todo $h\in G$. Por tanto, tenemos que:
    \begin{align*}
        \ker(\Phi) &\subset \bigcap_{h\in G} hP_1h^{-1}
        \subset \bigcap_{i=1}^4 P_i = \{1\}
    \end{align*}
    Por tanto, $\ker(\Phi)=\{1\}$, luego la acción es fiel.\\

    Por el Primer Teorema de Isomorfía, tenemos que:
    \begin{equation*}
        G\cong G/\{1\} = G/{\ker(\Phi)} \cong \Im(\Phi) \subset S_4
    \end{equation*}

    Por tanto, hemos visto que $G$ es isomorfo a un subgrupo de $S_4$. Como $|G|=12$ y el único subgrupo de orden $12$ de $S_4$ es $A_4$, tenemos que:
    \begin{equation*}
        G\cong A_4
    \end{equation*}
    Por tanto, hemos demostrado que todo grupo de orden $12$ con más de un $3-$subgrupo de Sylow es isomorfo al grupo alternado $A_4$.
\end{ejercicio}

\begin{ejercicio}\label{ej:6.24}~
    \begin{enumerate}
        \item Demostrar que no existen grupos simples de orden $12$. Más concretamente, demostrar que todo grupo de orden $12$ admite un subgrupo normal de orden $3$ o de orden $4$.
        
        Sea $G$ un grupo de orden $12=3\cdot 2^2$. Por el Segundo Teorema de Sylow, tenemos que:
        \begin{equation*}
            n_3 \mid 4 \qquad n_3 \equiv 1 \mod 3
        \end{equation*}

        Por tanto, $n_3\in \{1,4\}$.
        \begin{itemize}
            \item Si $n_3=1$, entonces el $3-$subgrupo de Sylow es normal con cardinal $3$ (luego no es propio), por lo que $G$ no es simple.
            
            \item Si $n_3=4$, aplicamos de nuevo el Segundo Teorema de Sylow:
            \begin{equation*}
                n_2 \mid 3 \qquad n_2 \equiv 1 \mod 2
            \end{equation*}
            Por tanto, $n_2\in \{1,3\}$.
            \begin{itemize}
                \item Si $n_2=1$, entonces el $2-$subgrupo de Sylow es normal con cardinal $4$ (luego no es propio), por lo que $G$ no es simple.
                \item Si $n_2=3$, $n_3=4$.

                Estudiamos la situación.
                \begin{itemize}
                    \item Como $n_3=4$, tenemos $4$ $3-$subgrupos de Sylow (todos ellos disjuntos), por lo que tenemos $4\cdot 2=8$ elementos de orden $3$.
                    \item Como $n_2=3$, tenemos $3$ $2-$subgrupos de Sylow, pero no podemos garantizar que sean disjuntos. Fijado $P\in \Syl_2(G)$, este tendrá $3$ elementos de orden $2$ o $4$. Además, puesto que los $2-$subgrupos de Sylow son distintos, al menos habrá otro elemento de orden $2$ o $4$ distinto. Por tanto, tenemos al menos $4$ elementos de orden $2$ o $4$.
                \end{itemize}
                
                Por tanto, tenemos:
                \begin{itemize}
                    \item $1$ elemento de orden $1$.
                    \item $8$ elementos de orden $3$.
                    \item Al menos $4$ elementos de orden $2$ o $4$.
                \end{itemize}
                Esto implica que el grupo tiene al menos $13$ elementos, lo cual es una contradicción. Por tanto, este caso no puede darse.
            \end{itemize}
        \end{itemize}

        Por tanto, hemos visto que $n_3=1$ (en cuyo caso $G$ tiene un subgrupo normal de orden $3$) o $n_2=1$ (en cuyo caso $G$ tiene un subgrupo normal de orden $4$). Por tanto, todo grupo de orden $12$ admite un subgrupo normal de orden $3$ o de orden $4$.


        \item Demostrar que no existen grupos simples de orden $28$. Más concretamente, probar que todo grupo de orden $28$ contiene un subgrupo normal de orden $7$.
        
        Sea $G$ un grupo de orden $28=7\cdot 2^2$. Por el Segundo Teorema de Sylow, tenemos que:
        \begin{equation*}
            n_7 \mid 4 \qquad n_7 \equiv 1 \mod 7
        \end{equation*}
        Por tanto, $n_7=1$. Por tanto el $7-$subgrupo de Sylow es normal con cardinal $7$ (luego no es propio), por lo que $G$ no es simple.
        \item Demostrar que no existen grupos simples de orden $56$. Más concretamente, probar que todo grupo de orden $56$ contiene un subgrupo normal de orden $7$ o de orden $8$.
        
        Sea $G$ un grupo de orden $56=7\cdot 2^3$. Por el Segundo Teorema de Sylow, tenemos que:
        \begin{equation*}
            n_7 \mid 8 \qquad n_7 \equiv 1 \mod 7
        \end{equation*}
        Por tanto, $n_7\in \{1,8\}$.
        \begin{itemize}
            \item Si $n_7=1$, entonces el $7-$subgrupo de Sylow es normal con cardinal $7$ (luego no es propio), por lo que $G$ no es simple.
            \item Si $n_7=8$, aplicamos de nuevo el Segundo Teorema de Sylow:
            \begin{equation*}
                n_2 \mid 7 \qquad n_2 \equiv 1 \mod 2
            \end{equation*}
            Por tanto, $n_2\in \{1,7\}$.
            \begin{itemize}
                \item Si $n_2=1$, entonces el $2-$subgrupo de Sylow es normal con cardinal $8$ (luego no es propio), por lo que $G$ no es simple.
                \item Si $n_2=7$, $n_7=8$.
                \begin{itemize}
                    \item Como $n_7=8$, hay $8$ $7-$subgrupos de Sylow (todos ellos distintos), por lo que tenemos $8\cdot 6=48$ elementos de orden $7$. Como $n_2=7$, tenemos $7$ $2-$subgrupos de Sylow, pero no podemos garantizar que sean disjuntos. Fijado $P\in \Syl_2(G)$, este contendrá $7$ elementos de orden $2$, $4$ o $8$. Además, puesto que  los $2-$subgrupos de Sylow son distintos, al menos habrá otro elemento de orden $2$, $4$ o $8$ distinto. Por tanto, tenemos al menos $8$ elementos de orden $2$, $4$ o $8$. 
                \end{itemize}
                
                Por tanto, tenemos:
                \begin{itemize}
                    \item $1$ elemento de orden $1$.
                    \item $48$ elementos de orden $7$.
                    \item $8$ elementos de orden $2$, $4$ o $8$.
                \end{itemize}
                Esto implica que el grupo tiene al menos $57$ elementos, lo cual es una contradicción. Por tanto, este caso no puede darse.
            \end{itemize}
        \end{itemize}

        Por tanto, hemos visto que $n_7=1$ (en cuyo caso $G$ tiene un subgrupo normal de orden $7$) o $n_2=1$ (en cuyo caso $G$ tiene un subgrupo normal de orden $8$).
        \item Demostrar que no existen grupos simples de orden $148$.
        
        Sea $G$ un grupo de orden $148=37\cdot 2^2$. Por el Segundo Teorema de Sylow, tenemos que:
        \begin{equation*}
            n_{37} \mid 4 \qquad n_{37} \equiv 1 \mod 37
        \end{equation*}
        Por tanto, $n_{37}=1$. Por tanto el $37-$subgrupo de Sylow es normal con cardinal $37$ (luego no es propio), por lo que $G$ no es simple.
        \item Demostrar que no existen grupos simples de orden $200$.
        Sea $G$ un grupo de orden $200=5^2\cdot 2^3$. Por el Segundo Teorema de Sylow, tenemos que:
        \begin{equation*}
            n_5 \mid 8 \qquad n_5 \equiv 1 \mod 5
        \end{equation*}
        Por tanto, $n_5=1$. Por tanto el $5-$subgrupo de Sylow es normal con cardinal $5^2$ (luego no es propio), por lo que $G$ no es simple.
        \item Demostrar que no existen grupos simples de orden $351$.
        Sea $G$ un grupo de orden $351=3^3\cdot 13$. Por el Segundo Teorema de Sylow, tenemos que:
        \begin{equation*}
            n_{13} \mid 27 \qquad n_{13} \equiv 1 \mod 13
        \end{equation*}
        Por tanto, $n_{13}\in \{1,27\}$.
        \begin{itemize}
            \item Si $n_{13}=1$, entonces el $13-$subgrupo de Sylow es normal con cardinal $13$ (luego no es propio), por lo que $G$ no es simple.
            \item Si $n_{13}=27$, aplicamos de nuevo el Segundo Teorema de Sylow:
            \begin{equation*}
                n_3 \mid 13 \qquad n_3 \equiv 1 \mod 3
            \end{equation*}
            Por tanto, $n_3\in \{1,13\}$.
            \begin{itemize}
                \item Si $n_3=1$, entonces el $3-$subgrupo de Sylow es normal con cardinal $27$ (luego no es propio), por lo que $G$ no es simple.
                \item Si $n_3=13$, $n_{13}=27$.
                \begin{itemize}
                    \item Como $n_{13}=27$, tenemos $27$ $13-$subgrupos de Sylow (todos ellos distintos), por lo que tenemos $27\cdot 12=324$ elementos de orden $13$.
                    \item Como $n_3=13$, tenemos $13$ $3-$subgrupos de Sylow, pero no podemos garantizar que sean disjuntos. Fijado $P\in \Syl_3(G)$, este contendrá $26$ elementos de orden $3$, $9$ o $27$. Además, puesto que los $3-$subgrupos de Sylow son distintos, al menos habrá otro elemento de orden $3$, $9$ o $27$ distinto. Por tanto, tenemos al menos $27$ elementos de orden $3$, $9$ o $27$.
                \end{itemize}
                Por tanto, tenemos:
                \begin{itemize}
                    \item $1$ elemento de orden $1$.
                    \item $324$ elementos de orden $13$.
                    \item $27$ elementos de orden $3,9$ o $27$.
                \end{itemize}
                Esto implica que el grupo tiene más de $351$ elementos, lo cual es una contradicción. Por tanto, este caso no puede darse.
            \end{itemize}
        \end{itemize}

        Por tanto, hemos visto que $n_{13}=1$ (en cuyo caso $G$ tiene un subgrupo normal de orden $13$) o $n_3=1$ (en cuyo caso $G$ tiene un subgrupo normal de orden $27$).
    \end{enumerate}
\end{ejercicio}

\begin{ejercicio}\label{ej:6.25}
    Calcular el número de elementos de orden $7$ que tiene un grupo simple de orden $168$.\\


    Sabemos que $168=2^3\cdot 3\cdot 7$. Como cada elemento de orden $7$ va a generar un grupo cíclico de orden $7$, buscamos el número de subgrupos de Sylow de orden $7$. Por la descomposición de $168$, sabemos que dichos grupos serán $7-$subgrupos de Sylow. Por el Segundo Teorema de Sylow, tenemos que:
    \begin{equation*}
        n_7\equiv 1\mod 7\qquad n_7\mid 24
    \end{equation*}
    Por tanto, $n_7\in \{1, 8\}$.
    \begin{itemize}
        \item Si $n_7=1$, entonces el $7-$subgrupo de Sylow es normal con cardinal $7$ (luego no es propio), por lo que $G$ no es simple.
    \end{itemize}

    Por tanto, $n_7=8$. Por tanto, hay exactamente $8$ subgrupos de orden $7$. Cada uno de los elementos de orden $7$ del grupo pertenece a un único subgrupo de orden $7$, y será un generador de estos. Además, sabemos que el número de elementos de orden $7$ de un grupo cíclico de orden $7$ viene dado por la función $\varphi(7) = 6$. Por tanto, el número de elementos de orden $7$ de un grupo simple de orden $168$ es:
    \begin{equation*}
        8\cdot 6 = 48
    \end{equation*}
\end{ejercicio}