\section{Grupos cociente. Teoremas de isomorfismo. Productos}

\begin{ejercicio}
    Demostrar que si $G\leq S_n$, entonces $G\subseteq A_n$ o bien se tiene que $[G:G\cap A_n]=2$. Concluir que un subgrupo de $S_n$ consiste sólo en permutaciones pares, o bien contiene el mismo número de permutaciones pares que de impares.\\

    Sea $G\leq S_n$ un subgrupo de $S_n$. O bien $G\subseteq A_n$ (en cuyo caso consiste sólo de permutaciones pares); o bien $\exists \sigma\in G$ tal que $\sigma\notin A_n$, es decir, $\veps(\sigma)=-1$.
    Para ver que $[G:G\cap A_n]=2$, hay varias posibilidades.
    \begin{description}
        \item[Opción 1:] Consideramos el homomorfismo $\veps:G\to \{-1,1\}$ dado por la aplicación signatura. Calculemos su núcleo y su imagen:
        \begin{align*}
            \ker(\veps) &= \{\sigma\in G\mid \veps(\sigma)=1\} = A_n\cap G,\\
            Im(\veps) &= \{\veps(\sigma)\mid \sigma\in G\}\AstIg \{-1,1\}
        \end{align*}
        donde vamos a razonar el por qué de $(\ast)$.
        Como $1\in G$ por ser este un grupo, entonces $1\in Im(\veps)$, y como $\exists \sigma\in G$ tal que $\veps(\sigma)=-1$, entonces $-1\in Im(\veps)$. Por lo tanto, $Im(\veps)=\{-1,1\}$. Por el Primer Teorema de Isomorfía, se tiene que:
        \begin{equation*}
            \frac{G}{\ker(\veps)}\cong Im(\veps) \implies \frac{G}{A_n\cap G}\cong \{-1,1\}\cong \bb{Z}_2.
        \end{equation*}

        Por definición de índice, se tiene que:
        \begin{equation*}
            [G:A_n\cap G] = \left|\dfrac{G}{A_n\cap G}\right| = |\bb{Z}_2| = 2
        \end{equation*}

        \item[Opción 2:] Por el Teorema de Lagrange, sabemos que:
        \begin{equation*}
            [S_n : A_n] = \dfrac{|S_n|}{|A_n|} = 2\Longrightarrow A_n \lhd S_n
        \end{equation*}

        Aplicando el Segundo Teorema de Isomorfía a $G$ y $A_n$, tenemos que:
        \begin{equation*}
            \dfrac{G}{G\cap A_n} \cong \dfrac{GA_n}{A_n}
        \end{equation*}

        Veamos qué grupo es $GA_n$. En primer lugar, para $1\in G$ vemos que $A_n\subset GA_n$. No obstante, como $\exists \sigma\in G$ con $\veps(\sigma)=-1$, entonces $A_n\neq GA_n$, por lo que $|GA_n|>|A_n|=\nicefrac{|S_n|}{2}$. Como $|GA_n| \mid |S_n|$, ha de ser $|GA_n|=|S_n|$, por lo que $GA_n=S_n$. Por tanto:
        \begin{equation*}
            \dfrac{G}{G\cap A_n} \cong \dfrac{S_n}{A_n}
        \end{equation*}

        Por definición de índice, se tiene que:
        \begin{equation*}
            [G:A_n\cap G] = \left|\dfrac{G}{A_n\cap G}\right| = \left|\dfrac{S_n}{A_n}\right| = \dfrac{|S_n|}{|A_n|} = 2
        \end{equation*}
    \end{description}

    En cualquier caso, hemos visto que $[G:A_n\cap G]=2$. Por el Teorema de Lagrange, se tiene que $|G|=2\cdot |G\cap A_n|$, por lo que la mitad de las permutaciones de $G$ son pares. Como una permutación o bien es par o es impar, entonces la otra mitad ha de tener signatura impar. Por tanto, contiene el mismo número de permutaciones pares que de impares.
\end{ejercicio}

\begin{ejercicio}
    Sea $\bb{K}$ un cuerpo.
    \begin{enumerate}
        \item Se considera la siguiente aplicación:
        \Func{\det}{\GL_n(\bb{K})}{\bb{K}^\times}{G}{\det(G)}
        Demostrar que dicha aplicación es un epimorfismo de grupos. ¿Cuál es el núcleo de este homomorfismo?

        Para comprobar que se trata de un homomorfismo, tomamos $A,B\in \GL_n(\bb{K})$ y por las propiedades de la determinante, se tiene que:
        \begin{equation*}
            \det(AB) = \det(A)\cdot\det(B)
        \end{equation*}

        Por otro lado, para cada $a\in \bb{K}^\times$, se considera la siguiente matriz:
        \begin{equation*}
            A_a = \begin{pmatrix}
                a & 0 & \cdots & 0 \\
                0 & 1 & \cdots & 0 \\
                \vdots & \vdots & \ddots & \vdots \\
                0 & 0 & \cdots & 1
            \end{pmatrix}
        \end{equation*}

        Como $\det(A_a)=a\neq 0$, entonces $A_a\in \GL_n(\bb{K})$. Como $\det(A_a)=a$, se tiene que $\det$ es sobreyectiva. Por lo tanto, $\det$ es un epimorfismo de grupos. Su núcleo es:
        \begin{equation*}
            \ker(\det) = \{A\in \GL_n(\bb{K})\mid \det(A)=1\} = \SL_n(\bb{K})
        \end{equation*}
        \item Si $\bb{K}$ es un cuerpo finito con $q$ elementos, determinar el orden del grupo $\SL_n(\bb{K})$.\\
        
        Por el Primer Teorema de Isomorfía, se tiene que:
        \begin{equation*}
            \dfrac{\GL_n(\bb{K})}{\SL_n(\bb{K})}\cong \bb{K}^\times
        \end{equation*}

        Por el Teorema de Lagrange, se tiene que:
        \begin{equation*}
            |\SL_n(\bb{K})| = \dfrac{|\GL_n(\bb{K})|}{|\bb{K}^\times|} = \dfrac{|\GL_n(\bb{K})|}{q-1} = \dfrac{(q^n-1)(q^n-q)\cdots(q^n-q^{n-1})}{q-1}
        \end{equation*}
    \end{enumerate}
\end{ejercicio}

\begin{ejercicio}
    Sea $n\in \bb{N}\setminus \{0\}$, y sea $G$ un grupo verificando que para todo par de elementos $x,y\in G$ se tiene que $(xy)^n=x^ny^n$. Se definen:
    \begin{align*}
        H &= \{x\in G\mid x^n=1\},\\
        K &= \{x^n\mid x\in G\}.
    \end{align*}
    Demostrar que $H,K\lhd G$, y que $|K|=[G:H]$.\\

    Definimos en primer lugar la siguiente aplicación:
    \Func{f}{G}{G}{x}{x^n}

    Para demostrar que se trata de un homomorfismo emplearemos la propiedad dada en el enunciado $(\ast)$:
    \begin{align*}
        f(xy) &= (xy)^n \AstIg x^ny^n = f(x)f(y)\qquad \forall x,y\in G
    \end{align*}

    Por tanto, $f$ es un homomorfismo. Como $\{1\},G<G$, entonces los siguientes grupos son subgrupos de $G$:
    \begin{align*}
        f^*(\{1\}) &= \{x\in G\mid f(x)=1\} = \{x\in G\mid x^n=1\} = H = \ker(f),\\
        f_*(G) &= \{f(x)\mid x\in G\} = \{x^n\mid x\in G\} = K = Im(f).
    \end{align*}

    Por tanto, tenemos que $H,K<G$. Probamos ahora que son grupos normales en $G$. Como $H=\ker f$ y $f$ es un homomorfismo, se tiene que $H\lhd G$. Ahora probamos que $K$ es normal en $G$. Tomamos $x\in G$ y $k\in K$ (por lo que $\exists y\in G$ tal que $k=y^n$). Entonces, consideramos $xyx^{-1}\in G$ y calculamos:
    \begin{align*}
        xkx^{-1} &= x(y^n)x^{-1} = (xyx^{-1})^n \in K
    \end{align*}
    Por tanto, $K\lhd G$. Para probar que $|K|=[G:H]$, tomamos el homomorfismo $f$ anteriormente descrito. Por el Primer Teorema de Isomorfía, se tiene que:
    \begin{equation*}
        \dfrac{G}{H}\cong K \implies |K| = \left|\dfrac{G}{H}\right| = [G:H]
    \end{equation*}
\end{ejercicio}

\begin{ejercicio}\label{ej:4.4}
    Para un grupo $G$ se define su \emph{centro} como
    \[
        Z(G) = \{a\in G\mid ax=xa\ \forall x\in G\}.
    \]
    \begin{enumerate}
        \item Demostrar que $Z(G)< G$.
        
        Como $Z(G)\subset G$, hay dos principales posibilidades, ambas equivalentes:
        \begin{description}
            \item[Opción 1:]
            Comprobamos las tres condiciones que caracterizan a los subgrupos:
            \begin{itemize}
                \item $1\in Z(G)$: Para todo $x\in G$, se tiene que $1x=x1=x$.
                \item $a,b\in Z(G)\implies ab\in Z(G)$: Para todo $x\in G$, se tiene que:
                \begin{equation*}
                    (ab)x = a(bx) = a(xb) = (ax)b = (xa)b = x(ab).
                \end{equation*}
                \item $a\in Z(G)\implies a^{-1}\in Z(G)$: Para todo $x\in G$, se tiene que:
                \begin{equation*}
                    a^{-1}x = (x^{-1}a)^{-1} = (ax^{-1})^{-1} = xa^{-1}.
                \end{equation*}
            \end{itemize}

            \item[Opción 2:]
            
            Dados $a,b\in Z(G)$, comprobemos que $ab^{-1}\in Z(G)$:
            \begin{align*}
                \hspace{-1cm}(ab^{-1})x &= a(b^{-1}x) = a(x^{-1}b)^{-1} = a(bx^{-1})^{-1} = a(xb^{-1}) = (ax)b^{-1} = (xa)b^{-1} = x(ab^{-1}).
            \end{align*}
        \end{description}

        En cualquier caso, $Z(G)$ es un subgrupo de $G$.
        \item Demostrar que $Z(G)\lhd G$.
        
        De nuevo, hay dos posibilidades:
        \begin{description}
            \item[Opción 1:]
            Para $x\in G$. Entonces:
            \begin{equation*}
                xZ(G) = \{xz\mid z\in Z(G)\} = \{zx\mid z\in Z(G)\} = Z(G)x.
            \end{equation*}

            \item[Opción 2:]
            Empleamos la caracterización de subgrupo normal. Para $x\in G$ y $z\in Z(G)$, buscamos ver que $xzx^{-1}\in Z(G)$:
            \begin{align*}
                xzx^{-1}y &= zxx^{-1}y = zy = yz = yzxx^{-1} = yxzx^{-1}\qquad \forall y\in G.
            \end{align*}
        \end{description}
        En ambos casos, se tiene que $Z(G)\lhd G$.
        \item Demostrar que $G$ es abeliano si, y sólo si, $G=Z(G)$.
        \begin{equation*}
            \hspace{-1cm}G=Z(G) \iff G=\{a\in G\mid ax=xa\ \forall x\in G\} \iff ax=xa\ \forall a,x\in G \iff G\text{ es abeliano.}
        \end{equation*}
        \item\label{ej:4.4.4} Demostrar que $G/Z(G)$ es abeliano si, y sólo si, $G$ es abeliano.
        Demostrar que si $G/Z(G)$ es cíclico, entonces $G$ es abeliano.\\
        
        Como $G/Z(G)$ es cíclico, entonces existe $x\in G$ tal que $G/Z(G)=\langle xZ(G)\rangle$. 
        Como las clases de equivalencia forman una partición disjunta de $G$, para cada $y\in G$ existe $k\in \bb{Z}$ tal que $y\in x^kZ(G)$. Buscamos ahora demostrar que $G$ es abeliano. Dados $a,b\in G$, entonces existen $p,q\in \bb{Z}$ tales que $a\in x^pZ(G)$ y $b\in x^qZ(G)$. Entonces, existen $z_a,z_b\in Z(G)$ tales que $a=x^pz_a$ y $b=x^qz_b$. Por tanto:
        \begin{align*}
            ab &= (x^pz_a)(x^qz_b) = x^{p+q}z_az_b = x^qz_b x^pz_a = (x^qz_b)(x^pz_a) = ba
        \end{align*}

        Por tanto, $ab=ba$ para todo $a,b\in G$, por lo que $G$ es abeliano.        
    \end{enumerate}
\end{ejercicio}

\begin{ejercicio}\label{ej:4.5}
    Determinar el centro del grupo diédrico $D_4$. Observar que el cociente $D_4/Z(D_4)$ es abeliano, aunque $D_4$ no lo sea (compárese este hecho con el tercer apartado del ejercicio anterior).\\

    El grupo $D_4$ está formado por las siguientes rotaciones y reflexiones:
    \begin{align*}
        D_4 &= \{1,r,r^2,r^3,s,sr,sr^2,sr^3\}
    \end{align*}

    Sabemos que $Z(D_4)$ es un subgrupo normal de $D_4$. En primer lugar, sabemos que $1\in Z(D_4)$. Veamos que $r^2\in Z(D_4)$. En primer lugar, vemos que conmuta con todas las potencias de $r$. Veamos ahora que conmuta con el resto de elementos:
    \begin{align*}
        r^2s &= rsr^3 = sr^6= sr^2\\
        r^2\ sr &= sr^2r = sr\ r^2\\
        r^2\ sr^2 &= srr^2r = s = sr^4 = sr^2\ r^2\\
        r^2\ sr^3 &= sr =sr^5 = sr^3\ r^2
    \end{align*}

    Por tanto, $\{1,r^2\}\subset Z(D_4)$. Además, tenemos que:
    \begin{align*}
        sr&=r^3s\neq rs \Longrightarrow s,r\notin Z(D_4)\\
        sr^3&=r^9s = rs\neq r^3s \Longrightarrow r^3\notin Z(D_4)\\
        sr\ s &= r^3\neq r\Longrightarrow sr \notin Z(D_4)\\
        sr^3\ s &= r \neq r^3 \Longrightarrow sr^3\notin Z(D_4)        
    \end{align*}

    Como $|Z(D_4)|$ divide a $|D_4|=8$ y $2\leq |Z(D_4)|\leq 3$, se tiene que $|Z(D_4)|=2$. Por tanto, $$Z(D_4)=\{1,r^2\}$$

    Veamos que $D_4/Z(D_4)$ es abeliano. Sabemos que:
    \begin{equation*}
        \left|D_4/Z(D_4)\right| = \dfrac{|D_4|}{|Z(D_4)|} = \dfrac{8}{2} = 4
    \end{equation*}

    Por tanto, $D_4/Z(D_4)$ es isomorfo a $V$ o a $C_4$. Por el último apartado del ejercicio anterior, si fuese $D_4/Z(D_4)$ isomorfo a $C_4$, entonces $D_4$ sería abeliano. Como no lo es, se tiene que $D_4/Z(D_4)\cong V$. Como $V$ es abeliano, se tiene que $D_4/Z(D_4)$ también lo es.
\end{ejercicio}

\begin{ejercicio}
    Determinar el centro de los grupos $S_n$ y $A_n$ para $n\geq 2$.\\

    Para $n=2$, tenemos que:
    \begin{align*}
        S_2 &= \{1,(1\ 2)\} \implies Z(S_2) = S_2\\
        A_2 &= \{1\} \implies Z(A_2) = A_2
    \end{align*}

    Calculamos ahora $Z(S_n)$ para $n\geq 3$. Sea $\sigma\neq 1\in S_n$. Entonces,$\exists i,j\in \{1,\ldots,n\}$ tales que $i\neq j$ y $\sigma(i)=j$. Consideramos ahora $k\in \{1,\ldots,n\}$ tal que $k\neq i,j$, y sea la permutación $\tau = (j\ k)$. Entonces:
    \begin{align*}
        \sigma\tau(i) &= \sigma(i) = j\\
        \tau\sigma(i) &= \tau(j) = k
    \end{align*}

    Por tanto, $\sigma\tau\neq \tau\sigma$. Como $\sigma$ es arbitrario, se tiene que $Z(S_n)=\{1\}$ para $n\geq 3$.\\

    Trabajamos ahora con $A_n$. En primer lugar, para $n=3$ tenemos que $|A_3|=3$ primo, por lo que $A_3$ es cíclico y en particular abeliano. Por tanto, $Z(A_3)=A_3$.\\

    Para $n\geq 4$, sea $\sigma\in A_n$ tal que $\sigma\neq 1$. Entonces, $\exists i,j\in \{1,\ldots,n\}$ tales que $i\neq j$ y $\sigma(i)=j$. Consideramos ahora $k,l\in \{1,\ldots,n\}\setminus \{i,j\}$, tal que $k\neq l$. Consideramos ahora $\tau=(j\ k\ l)\in A_n$. Entonces:
    \begin{align*}
        \sigma\tau(i) &= \sigma(i) = j\\
        \tau\sigma(i) &= \tau(j) = k
    \end{align*}

    Por tanto, $\sigma\tau\neq \tau\sigma$. Como $\sigma$ es arbitrario, se tiene que $Z(A_n)=\{1\}$ para $n\geq 4$. Notemos que hemos tenido que imponer que $n\geq 4$ para que podamos elegir 4 elementos distintos.

\end{ejercicio}

\begin{ejercicio}
    Determinar el centro del grupo $D_n$ para $n\geq 3$.\\

    Dado $z\in D_n$, como $D_n=\langle r,s\rangle$ entonces $z\in Z(D_n)$ si y sólo si $z$ conmuta con $r$ y con $s$.
    Distinguimos según los elementos de $D_n$:
    \begin{itemize}
        \item $1\in Z(D_n)$ para todo $n\geq 3$.
        \item Fijado $m\in\{1,\ldots,n-1\}$, veamos si $r^m\in Z(D_n)$.
        \begin{align*}
            r^m\ r &= r^{m+1} = r\ r^m\\
            r^m\ s &= sr^{m(n-1)} = sr^{-m} = sr^{n-m} = s\ r^m\iff r^{n-m}=r^m
            \stackrel{(\ast)}{\iff} n-m=m\iff n=2m
        \end{align*}
        donde en $(\ast)$ hemos usado que $n-m,m\in\{1,\ldots,n-1\}$. Por tanto, hay que distinguir según la paridad de $n$:
        \begin{itemize}
            \item Si $n$ es impar, entonces $r^m \notin Z(D_n)$ para todo $m\in\{1,\ldots,n-1\}$.
            \item Si $n$ es par, entonces $r^{\nicefrac{n}{2}}\in Z(D_n)$, y $r^m\notin Z(D_n)$ para todo $m\in\{1,\ldots,n-1\}\setminus \{\nicefrac{n}{2}\}$.
        \end{itemize}

        \item Fijado $m\in\{0,\ldots,n-1\}$, veamos si $sr^m\in Z(D_n)$.
        \begin{align*}
            sr^m\ r &= sr^{m+1} = r^{(m+1)(n-1)}s = r^{mn -m + n -1}s = r^{n-m-1}s\\
            r\ sr^m &= r\ r^{n-m}s = r^{n-m+1}s
        \end{align*}

        Por tanto, se necesita que $r^{n-m-1}=r^{n-m+1}$, y equivalentemente se necesita que $r^{-1}=r$, es decir, que $r^2=1$. Esto es cierto para $n=2$, pero no para $n\geq 3$. Por tanto, $sr^m\notin Z(D_n)$ para todo $m\in\{0,\ldots,n-1\}$.
    \end{itemize}

    En resumen, tenemos que:
    \begin{itemize}
        \item Si $n$ es impar, entonces $Z(D_n)=\{1\}$.
        \item Si $n$ es par, entonces $Z(D_n)=\{1,r^{\nicefrac{n}{2}}\}$.
    \end{itemize}
\end{ejercicio}

\begin{ejercicio}
    Sean $H$ y $K$ dos subgrupos finitos de un grupo $G$, uno de ellos normal. Demostrar que
    \[
        |H||K| = |HK||H\cap K|.
    \]

    Sean $H,K<G$, y $K\lhd G$. Entonces, por el Segundo Teorema de Isomorfía, se tiene que:
    \begin{equation*}
        \dfrac{HK}{K} \cong \dfrac{H}{H\cap K}
    \end{equation*}

    Tomando índices y usando el Teorema de Lagrange, se tiene que:
    \begin{align*}
        \left|\dfrac{HK}{K}\right| &= \dfrac{|HK|}{|K|} = \left|\dfrac{H}{H\cap K}\right| = \dfrac{|H|}{|H\cap K|}\Longrightarrow |H||K| = |HK||H\cap K|.
    \end{align*}

    Si por el contrario $H\lhd G$, entonces:
    \begin{equation*}
        \dfrac{HK}{H} \cong \dfrac{K}{H\cap K}
    \end{equation*}
    y de igual forma se llega a la misma conclusión.
\end{ejercicio}

\begin{ejercicio}\label{ej:9_relacion4}
    Sea $G$ finito y $N\lhd G$. Probar que $G/N\cong G$ si, y sólo si, $N=\{1\}$, y que $G/N\cong \{1\}$ si, y sólo si, $N=G$.\\

    Veamos que $G/N\cong G$ si, y sólo si, $N=\{1\}$.
    \begin{description}
        \item[$\Longrightarrow)$] Supongamos que $G/N\cong G$. Entonces, por el Teorema de Lagrange, se tiene que:
        \begin{equation*}
            |G| = |G/N| = \dfrac{|G|}{|N|} \implies |N|=1\Longrightarrow N=\{1\}.
        \end{equation*}

        \item[$\Longleftarrow)$] Supongamos que $N=\{1\}$. Definimos ahora el homomorfismo siguiente:
        \Func{f}{G}{G}{x}{x}

        Tenemos que:
        \begin{align*}
            \ker(f) &= \{x\in G\mid f(x)=1\} = \{x\in G\mid x=1\} = \{1\} = N,\\
            Im(f) &= \{f(x)\mid x\in G\} = \{x\mid x\in G\} = G
        \end{align*}

        Por tanto, por el Primer Teorema de Isomorfía, se tiene que:
        \begin{equation*}
            \dfrac{G}{N}\cong G
        \end{equation*}
    \end{description}

    Veamos ahora que $G/N\cong \{1\}$ si, y sólo si, $N=G$.
    \begin{description}
        \item[$\Longrightarrow)$] Supongamos que $G/N\cong \{1\}$. Entonces, por el Teorema de Lagrange, se tiene que:
        \begin{equation*}
            1 = |G/N| = \dfrac{|G|}{|N|} \implies |G|=|N|
        \end{equation*}
        Como $N\leq G$, se tiene que $N\subset G$, y por tanto $N=G$.

        \item[$\Longleftarrow)$] Supongamos que $N=G$. Entonces, definimos el homomorfismo siguiente:
        \Func{f}{G}{\{1\}}{x}{1}

        Entonces, tenemos que:
        \begin{align*}
            \ker(f) &= \{x\in G\mid f(x)=1\} = \{x\in G\mid 1=1\} = G = N,\\
            Im(f) &= \{f(x)\mid x\in G\} = \{1\mid x\in G\} = \{1\}
        \end{align*}

        Por tanto, por el Primer Teorema de Isomorfía, se tiene que:
        \begin{equation*}
            \dfrac{G}{N}\cong \{1\}
        \end{equation*}
    \end{description}
    \begin{observacion}
        Notemos que en las implicaciones hacia la izquierda no es necesario suponer que $G$ es finito. Lo usaremos por tanto sin esta restricción.
    \end{observacion}
\end{ejercicio}

\begin{ejercicio}
    Sean $G$ y $H$ dos grupos cuyos órdenes sean primos relativos. Probar que si $f:G\to H$ es un homomorfismo, entonces necesariamente $f(x)=1$ para todo $x\in G$, es decir, que el único homomorfismo entre ellos es el trivial.\\

    \begin{description}
        \item[Opción 1]~
        
        Como $|G|$ y $|H|$ son primos relativos, en particular son grupos finitos.    Por ser $f$ un homomorfismo, se tiene que $f(G)<H$, luego $|f(G)|\mid |H|$. Por el Primer Teorema de Isomorfía, se tiene que:
    \begin{equation*}
        \dfrac{G}{\ker(f)}\cong f(G) \implies |G| = |\ker(f)|\cdot |f(G)|\Longrightarrow |f(G)|\mid |G|
    \end{equation*}

    Como $\mcd(|G|,|H|)=1$, se tiene que $|f(G)|=1$. Como además $f(G)$ es un grupo, se tiene que $f(G)=\{1\}$. Por tanto, $f$ es el homomorfismo trivial.

        \item[Opción 2]~
        
        Sea $y\in f(G)$. Entonces, $\exists x\in G$, $x\neq 1$, tal que $f(x)=y$. Como $G$ es finito, $\exists n\in \bb{N}$ tal que $\ord(x)=n$, luego $n\mid |G|$. Por otro lado:
        \begin{equation*}
            1=f(1)=f(x^n)=f(x)^n=y^n\in H\Longrightarrow n\mid \ord(y)
        \end{equation*}

        Como $\ord(y)\mid |H|$, se tiene que $n\mid |H|$. Por tanto, $n\mid \mcd(|G|,|H|)=1$, luego $n=1$. Por tanto, $\ord(x)=1$, por lo que $x=1$. Por tanto, $y=f(x)=f(1)=1$ para todo $y\in f(G)$. Por tanto, $f$ es el homomorfismo trivial.
    \end{description}
\end{ejercicio}

\begin{ejercicio}
    Sean $H,K\leq G$, y sea $N\lhd G$ un subgrupo normal de $G$ tal que $HN=KN$. Demostrar que
    \[
        \frac{H}{H\cap N}\cong \frac{K}{K\cap N}.
    \]

    Aplicamos dos veces el Segundo Teorema de Isomorfía:
    \begin{itemize}
        \item Consideramos $H,N\leq G$, con $N\lhd G$. Entonces, por el Segundo Teorema de Isomorfía, se tiene que:
        \begin{equation*}
            \dfrac{HN}{N} \cong \dfrac{H}{H\cap N}
        \end{equation*}
        \item Consideramos $K,N\leq G$, con $N\lhd G$. Entonces, por el Segundo Teorema de Isomorfía, se tiene que:
        \begin{equation*}
            \dfrac{KN}{N} \cong \dfrac{K}{K\cap N}
        \end{equation*}
    \end{itemize}

    Como $KN=HN$ y ``ser isomorfo'' es una relación de equivalencia, se tiene que:
    \begin{align*}
        \dfrac{H}{H\cap N} \cong \dfrac{HN}{N} = \dfrac{KN}{N} \cong \dfrac{K}{K\cap N}
    \end{align*}
\end{ejercicio}

\begin{ejercicio}
    Sea $N\lhd G$ tal que $N$ y $G/N$ son abelianos. Sea $H$ un subgrupo cualquiera de $G$. Demostrar que existe un subgrupo normal $K\lhd H$ tal que $K$ y $H/K$ son abelianos.\\

    Por el Segundo Teorema de Isomorfía, se tiene que $K=H\cap N\lhd H$. Como $K\subset N$ y $K$ es un grupo, entonces $K\leq N$. Como $N$ es abeliano, se tiene que $K$ también lo es. Nos falta por ver que $H/K$ es abeliano. Por el Segundo Teorema de Isomorfía, se tiene que:
    \begin{equation*}
        \dfrac{H}{K} \cong \dfrac{HN}{N}
    \end{equation*}

    Veamos ahora que $HN/N\leq G/N$. Como $HN\subset G$, por definición de conjunto cociente se tiene que $HN/N \ \subset\ G/N$. Como $HN/N$ es un grupo, se tiene que $HN/N\leq G/N$. Como $G/N$ es abeliano, se tiene que $HN/N$ también lo es. Por tanto, por el Segundo Teorema de Isomorfía, se tiene que $H/K$ es abeliano.
\end{ejercicio}

\begin{ejercicio}
    Sea $G$ un grupo finito, y sean $H,K\leq G$, con $K\lhd G$ y tales que $|H|$ y $[G:K]$ son primos relativos. Demostrar que $H\subseteq K$.\\

    Por el Segundo Teorema de Isomorfía, se tiene que:
    \begin{equation*}
        \dfrac{H}{H\cap K} \cong \dfrac{HK}{K}
    \end{equation*}

    Como $HK\leq G$, entonces $HK/K \leq G/K$. Por tanto $\left|\frac{H}{H\cap K}\right| = \left|\frac{HK}{K}\right|$ divide a $\left|\frac{G}{K}\right| = [G:K]$. Por otro lado:
    \begin{equation*}
        \left|\frac{H}{H\cap K}\right| = \dfrac{|H|}{|H\cap K|} \implies |H| = |H\cap K|\cdot \left|\frac{H}{H\cap K}\right|\Longrightarrow \left|\frac{H}{H\cap K}\right| \mid |H|
    \end{equation*}

    Como $\mcd(|H|,[G:K])=1$, se tiene que $\left|\frac{H}{H\cap K}\right|=1$. Por tanto, $|H|=|H\cap K|$, y como $H\cap K\subset H$, se tiene que $H\cap K=H$. Por tanto, $H\subset K$.
\end{ejercicio}

\begin{ejercicio}
    Sea $G$ un grupo.
    \begin{enumerate}
        \item Demostrar que para cada $a\in G$ la aplicación $\varphi_a:G\to G$ definida por $\varphi_a(x)=axa^{-1}$, es un automorfismo de $G$. $\varphi_a$ se llama automorfismo interno o de conjugación de $G$ definido por $a$.\\
        
        Vemos en primer lugar que está bien definida, puesto que un grupo es cerrado para inversos y productos. Por tanto, $\varphi_a(x)\in G$ para todo $x\in G$. Veamos ahora que es un isomorfismo:
        \begin{itemize}
            \item Para ver que es un homomorfismo:
                \begin{equation*}
                    \varphi_a(xy) = a(xy)a^{-1} = axa^{-1}aya^{-1} = \varphi_a(x)\varphi_a(y) \qquad \forall x,y\in G
                \end{equation*}

            \item Para ver que es inyectiva, sean $x,y\in G$ de forma que:
                \begin{equation*}
                    \varphi_a(x) = axa^{-1} = aya^{-1} = \varphi_a(y)
                \end{equation*}
                Entonces, aplicando dos veces la propiedad cancelativa, tenemos que:
                \begin{equation*}
                    axa^{-1} = aya^{-1} \Longrightarrow ax = ay \Longrightarrow x = y
                \end{equation*}

            \item Para ver que es sobreyectiva, sea $y\in G$, tomamos $x=a^{-1}ya$. Entonces:
                \begin{equation*}
                    \varphi_a(x) = \varphi_a(a^{-1}ya) = a(a^{-1}ya)a^{-1} = aa^{-1}yaa^{-1} = y
                \end{equation*}
        \end{itemize}
        Concluimos que $\varphi_a$ es un automorfismo de $G$.
        \item Demostrar que la siguiente aplicación es un homomorfismo:
        \Func{\varphi}{G}{\Aut(G)}{a}{\varphi_a}

        Para esto, es necesario probar que, fijados $a,b\in G$, se tiene que:
        \begin{equation*}
            \varphi_{ab} = \varphi_a\circ\varphi_b
        \end{equation*}

        Tenemos que:
        \begin{align*}
            \varphi_{ab}(x) &= (ab)x(ab)^{-1} = (ab)x(b^{-1}a^{-1}) = a(bxb^{-1})a^{-1} = a\varphi_b(x)a^{-1} = \varphi_a(\varphi_b(x))\qquad \forall x\in G
        \end{align*}

        Por tanto, $\varphi_{ab} = \varphi_a\circ\varphi_b$. Por tanto, $\varphi$ es un homomorfismo de grupos.


        \item Demostrar que el conjunto de automorfismos internos de $G$, que se denota $\Int(G)$, es un subgrupo normal de $\Aut(G)$.
        
        Para ello, en primer lugar es necesario ver que $\Int(G)<\Aut(G)$. Considerados $a,b\in G$, tenemos que:
        \begin{align*}
            (\varphi_a\circ \varphi^{-1}_b)(x) &= \varphi_a\left(b^{-1}xb\right) = a\left(b^{-1}xb\right)a^{-1} = ab^{-1}xba^{-1}
            =\\&= (ab^{-1})x(ab^{-1})^{-1} = \varphi_{ab^{-1}}(x)\qquad \forall x\in G
        \end{align*}
        
        Veamos ahora que $\Int(G)\lhd \Aut(G)$. Para ello es necesario ver que se tiene $f\circ \varphi \circ f^{-1}\in \Int(G)$ para todo $f\in \Aut(G)$ y $\varphi\in \Int(G)$.
        Sea $f\in \Aut(G)$ y $\varphi_a\in \Int(G)$. Entonces, tenemos que:
        \begin{align*}
            (f\circ \varphi_a \circ f^{-1})(x) &= f\left(\varphi_a(f^{-1}(x))\right) = f\left(a(f^{-1}(x))a^{-1}\right)\\
            &= f(a)f(f^{-1}(x))f(a^{-1}) = f(a)xf(a^{-1}) = \varphi_{f(a)}(x)\in \Int(G)
        \end{align*}

        Por tanto, $f\circ \varphi_a \circ f^{-1}\in \Int(G)$, y por tanto $\Int(G)\lhd \Aut(G)$.
        \item Demostrar que $G/Z(G)\cong \Int(G)$.
        
        Buscamos aplicar el Primer Teorema de Isomorfía al homomorfismo $\varphi$ del apartado 2. Tenemos que:
        \begin{align*}
            \ker(\varphi) &= \{a\in G\mid \varphi_a = \Id_{G}\} = \{a\in G\mid \varphi_a(x) = x\ \forall x\in G\}\\
            &= \{a\in G\mid axa^{-1} = x\ \forall x\in G\}
            = \{a\in G\mid ax=xa\ \forall x\in G\} = Z(G)\\
        Im(\varphi) &= \{\varphi_a\mid a\in G\} = \{f\in \Aut(G)\mid f(x) = axa^{-1}\ \forall x\in G\} = \Int(G)
        \end{align*}

        Por tanto, por el Primer Teorema de Isomorfía, se tiene que:
        \begin{equation*}
            \dfrac{G}{Z(G)}\cong \Int(G)
        \end{equation*}
        \item Demostrar que $\Int(G)=\{\Id_G\}$ si y sólo si $G$ es abeliano.
            \begin{description}
                \item [Opción 1.]~\\
                \begin{description}
                    \item[$\Longrightarrow)$] Supongamos que $\Int(G)=\{\Id_G\}$. Entonces, para todo $a\in G$, se tiene que:
                    \begin{equation*}
                        \varphi_a = \Id_G \implies axa^{-1} = x\ \forall x\in G
                    \end{equation*}
                    Por tanto, $ax=xa\ \forall x\in G$. Como $a$ es arbitrario, se tiene que $G$ es abeliano.
                    \item[$\Longleftarrow)$] Supongamos que $G$ es abeliano. Entonces, para todo $a\in G$, se tiene que:
                    \begin{equation*}
                        \varphi_a(x) = axa^{-1} = aa^{-1}x = x\ \forall x\in G
                    \end{equation*}
                    Por tanto, $\varphi_a = \Id_G$. Como $a$ es arbitrario, se tiene que $\Int(G)=\{\Id_G\}$.
                \end{description}
                \item [Opción 2.]~\\
                    Como $G/Z(G)\cong Int(G)$, aplicando el Ejercicio~\ref{ej:9_relacion4}, tenemos que:
                    \begin{equation*}
                        Int(G) = \{1\} \Longleftrightarrow G = Z(G) \Longleftrightarrow G \text{\ es abeliano}
                    \end{equation*}
            \end{description}
    \end{enumerate}
\end{ejercicio}

\begin{ejercicio}
    Demostrar que el grupo de automorfismos de un grupo no abeliano no puede ser cíclico.\\

    Sea $G$ un grupo no abeliano. Por el recíproco del Ejercicio~\ref{ej:4.4}.\ref{ej:4.4.4}, sabemos que $G/Z(G)$ no es cíclico. Como $G/Z(G)\cong \Int(G)$, se tiene que $\Int(G)$ no es cíclico. Como $\Int(G)< \Aut(G)$ y todo subgrupo de un grupo cíclico es cíclico, se tiene que $\Aut(G)$ no es cíclico.
\end{ejercicio}

\begin{ejercicio}
    Demostrar que $\Aut(\bb{Z}_2\times \bb{Z}_2)\cong S_3$.\\

    Sabemos que:
    \begin{equation*}
        \bb{Z}_2\times \bb{Z}_2 \cong V^{\text{abs}} = \langle x,y\mid x^2=y^2=1,xy=yx\rangle
    \end{equation*}

    Construimos ahora automorfismos de $\bb{Z}_2\times \bb{Z}_2$. Sabemos que todos los elementos de $\bb{Z}_2\times \bb{Z}_2$ (a excepción del neutro) son de orden $2$, y además este grupo es conmutativo. Por tanto, con el Teorema de Dyck podemos constuir automorfismos de forma sencilla. Haciendo uso de que $\bb{Z}_2\times \bb{Z}_2=\langle(1,0),(0,1)\rangle$, tenemos los siguientes automorfismos:
    \begin{equation*}
        \begin{array}{c||c|c|c|c|c|c}
            & f_1 & f_2 & f_3 & f_4 & f_5 & f_6\\ \hline
            f_i(1,0) & (1,0) & (1,0) & (0,1) & (0,1) & (1,1) & (1,1)\\
            f_i(0,1) & (0,1) & (1,1) & (1,0) & (1,1) & (0,1) & (1,0)
        \end{array}
    \end{equation*}
    
    
    
    Además, no puede haber más automorfismos, puesto que fijada la imagen de un elemento generador, la imagen del otro elemento generador tan solo tiene dos posibilidades, puesto que no puede ser el neutro. Por tanto, tenemos que:
    \begin{equation*}
        |\Aut(\bb{Z}_2\times \bb{Z}_2)| = 6
    \end{equation*}

    Por tanto, $\Aut(\bb{Z}_2\times \bb{Z}_2)$ es un grupo de orden $6$, por lo que es cíclico o es isomorfo a $D_3\cong S_3$. Tenemos que:
    \begin{align*}
        (f_2\circ f_3)(1,0) &= f_2(f_3(1,0)) = f_2(0,1) = (1,1)\\
        (f_3\circ f_2)(1,0) &= f_3(f_2(1,0)) = f_3(1,0) = (0,1)
    \end{align*}

    Por tanto, $f_2\circ f_3\neq f_3\circ f_2$, por lo que $\Aut(\bb{Z}_2\times \bb{Z}_2)$ no es abeliano y por tanto no es cíclico. Por tanto, $\Aut(\bb{Z}_2\times \bb{Z}_2)\cong S_3$.
\end{ejercicio}

\begin{ejercicio}
    Demostrar que los grupos $S_3$, $\bb{Z}_{p^n}$ (con $p$ primo) y $\bb{Z}$ no son producto directo internos de subgrupos propios.
    \begin{enumerate}
        \item $S_3$.
        
        Como $|S_3|=6$, entonces sus subgrupos propios son de orden $2$ y $3$. 
        Por reducción al absurdo, supongamos que $S_3\cong K\times H$ con $K,H<S_3$. Entonces, $|S_3|=|K||H|$, y como $|K|$ y $|H|$ son divisores de $6$, se tiene que $|K|=2$ y $|H|=3$. Por tanto, $K\cong C_2$ y $H\cong C_3$. Por tanto, $S_3=K\times H\cong C_2\times C_3$. Como $C_2$ y $C_3$ son cíclicos con $\mcd(|C_2|,|C_3|)=1$, se tiene que $C_2\times C_3$ es cíclico. Por tanto, $S_3$ es cíclico, lo cual es una contradicción.

        \item $\bb{Z}_{p^n}$.
        
        Por ser $\bb{Z}_{p^n}$ un grupo cíclico de orden $p^n$, todos sus subgrupos propios son cíclicos de orden $p^k$ con $k\in \{1,\ldots,n-1\}$. Por tanto, por reducción al absurdo, supongamos que $\exists p,q\in \{1,\ldots,n-1\}$ tales que $\bb{Z}_{p^n}\cong \bb{Z}_{p^k}\times \bb{Z}_{p^q}$. Por ser $\bb{Z}_{p^k}$ y $\bb{Z}_{p^q}$ cíclicos, se tiene que $\bb{Z}_{p^k}\times \bb{Z}_{p^q}$ es cíclico si y solo si $\mcd(p^k,p^q)=1$. Como $p^k$ y $p^q$ son potencias de un mismo primo, se tiene que $\mcd(p^k,p^q)\geq p$. Por tanto, $\bb{Z}_{p^k}\times \bb{Z}_{p^q}$ no es cíclico. Por tanto, $\bb{Z}_{p^n}$ no es producto directo interno de subgrupos propios.

        \item $\bb{Z}$.
        
        Sabemos que todos los subgrupos de $\bb{Z}$ son de la forma $k\bb{Z}$, con $k\in \bb{N}$. Por tanto, por reducción al absurdo, supongamos que $\exists p,q\in \bb{N}$ tales que se tiene $\bb{Z}\cong p\bb{Z}\times q\bb{Z}\stackrel{(\ast)}{\cong} \bb{Z}\times \bb{Z}$. No obstante, $\bb{Z}\times \bb{Z}$ no es cíclico, y por tanto no es isomorfo a $\bb{Z}$. Por tanto, $\bb{Z}$ no es producto directo interno de subgrupos propios.

        En $(\ast)$, el isomorfismo es el siguiente (que se prueba fácilmente es es un isomorfismo):
        \Func{f}{n\bb{Z}}{\bb{Z}}{x}{\nicefrac{x}{n}}      
    \end{enumerate}
\end{ejercicio}



\begin{ejercicio}
    En cada uno de los siguientes casos, decidir si el grupo $G$ es o no producto directo interno de los subgrupos $H$ y $K$.
    \begin{enumerate}
        \item $G=\bb{R}^\times$, $H=\{\pm 1\}$, $K=\{x\in \bb{R}\mid x>0\}$.
        
        Es directo que que $G=HK$ y $H\cap K=\{1\}$. Como $H,K<G$ y $G$ es abeliano, se tiene que $H,K\lhd G$. Por tanto, $G\cong H\times K$.

        \item $G=\left\{\begin{pmatrix} a & b \\ 0 & c \end{pmatrix}\in \GL_2(\bb{R})\right\}$, $H=\left\{\begin{pmatrix} a & 0 \\ 0 & c \end{pmatrix}\in \GL_2(\bb{R})\right\}$, $K=\left\{\begin{pmatrix} 1 & b \\ 0 & 1 \end{pmatrix}\in \GL_2(\bb{R})\right\}$.
        
        Fijados $a,b,c\in \bb{R}$, sean las siguientes matrices:
        \begin{align*}
            A &= \begin{pmatrix} a & 0 \\ 0 & c \end{pmatrix}\in H\\
            B &= \begin{pmatrix} 1 & b \\ 0 & 1 \end{pmatrix}\in K
        \end{align*}

        Entonces:
        \begin{align*}
            AB = \begin{pmatrix} a & 0 \\ 0 & c \end{pmatrix}\begin{pmatrix} 1 & b \\ 0 & 1 \end{pmatrix} &= \begin{pmatrix} a & ab \\ 0 & c \end{pmatrix}\\
            BA = \begin{pmatrix} 1 & b \\ 0 & 1 \end{pmatrix}\begin{pmatrix} a & 0 \\ 0 & c \end{pmatrix} &= \begin{pmatrix} a & bc \\ 0 & c \end{pmatrix}
        \end{align*}

        Como $ab\neq bc$ en general, se tiene que $AB\neq BA$. Por tanto, por la caracterización del producto directo interno, se tiene que $G$ no es producto directo interno de $H$ y $K$.
        \item $G=\bb{C}^\times$, $H=\left\{z\in \bb{C}\mid |z|=1\right\}$, $K=\left\{x\in \bb{R}\mid x>0\right\}$.
        
        Dado $z\in G$, podemos escribir:
        \begin{equation*}
            z = \dfrac{z}{|z|}\cdot |z|\qquad \dfrac{z}{|z|}\in H,\ |z|\in K
        \end{equation*}

        Por tanto, $G=HK$. Además, $H\cap K=\{1\}$, y como $H,K<G$ y $G$ es abeliano, se tiene que $H,K\lhd G$. Por tanto, $G\cong H\times K$.
    \end{enumerate}
\end{ejercicio}

\begin{ejercicio}
    Sean $G,H$ y $K$ grupos. Demostrar que:
    \begin{enumerate}
        \item $H\times K\cong K\times H$.
        
        Definimos el isomorfismo:
        \Func{f}{H\times K}{K\times H}{(h,k)}{(k,h)}

        Vemos que $f$ es un isomorfismo:
        \begin{itemize}
            \item $f$ es un homomorfismo:
                \begin{align*}
                    f((h_1,k_1)(h_2,k_2)) &= f(h_1h_2,k_1k_2) = (k_1k_2,h_1h_2)\\
                    &= (k_1,h_1)(k_2,h_2) = f(h_1,k_1)f(h_2,k_2)
                \end{align*}
            \item $f$ es inyectiva:
                \begin{align*}
                    \ker (f) &= \{(h,k)\in H\times K\mid f(h,k)=(1,1)\}\\
                    &= \{(h,k)\in H\times K\mid (k,h)=(1,1)\} = \{(1,1)\}
                \end{align*}
            \item $f$ es sobreyectiva:
                
            Dado $(k,h)\in K\times H$, tomamos $(h,k)\in H\times K$. Entonces:
                \begin{align*}
                    f(h,k) &= (k,h) \qquad \forall (h,k)\in H\times K
                \end{align*}
        \end{itemize}

        Por tanto, $f$ es un isomorfismo.
        \item $G\times (H\times K)\cong (G\times H)\times K$.
        
        Definimos el isomorfismo:
        \Func{f}{G\times (H\times K)}{(G\times H)\times K}{(g,(h,k))}{((g,h),k)}

        Vemos que $f$ es un isomorfismo:
        \begin{itemize}
            \item $f$ es un homomorfismo:
                \begin{align*}
                    f((g_1,(h_1,k_1))(g_2,(h_2,k_2))) &= f(g_1g_2,(h_1h_2,k_1k_2))\\
                    &= ((g_1g_2,h_1h_2),k_1k_2) =\\&= ((g_1,h_1),k_1)((g_2,h_2),k_2) =\\&= f(g_1,(h_1,k_1))f(g_2,(h_2,k_2))
                \end{align*}
            \item $f$ es inyectiva:
                \begin{align*}
                    \ker (f) &= \{(g,(h,k))\in G\times (H\times K)\mid f(g,(h,k))=((1,1),1)\}\\
                    &= \{(g,(h,k))\in G\times (H\times K)\mid ((g,h),k)=((1,1),1)\} = \{(1,(1,1))\}
                \end{align*}
            \item $f$ es sobreyectiva:
                
            Dado $((g,h),k)\in (G\times H)\times K$, tomamos $(g,(h,k))\in G\times (H\times K)$. Entonces:
                \begin{align*}
                    f(g,(h,k)) &= ((g,h),k) \qquad \forall (g,(h,k))\in G\times (H\times K)
                \end{align*}
        \end{itemize}
        Por tanto, $f$ es un isomorfismo.
    \end{enumerate}
\end{ejercicio}

\begin{ejercicio}
    Dados isomorfismos de grupos $A\cong K$ y $B\cong H$, demostrar que $A \times B\cong K\times H$.\\

    Sea $f:A\to K$ y $g:B\to H$ los isomorfismos. Definimos la siguiente aplicación:
    \Func{h}{A\times B}{K\times H}{(a,b)}{(f(a),g(b))}

    Veamos que $h$ es un homomorfismo. Fijados $(a_1,b_1),(a_2,b_2)\in A\times B$, tenemos que:
    \begin{align*}
        h((a_1,b_1)(a_2,b_2)) &= h(a_1a_2,b_1b_2) = (f(a_1a_2),g(b_1b_2))\\
        &= (f(a_1)f(a_2),g(b_1)g(b_2)) = (f(a_1),g(b_1))(f(a_2),g(b_2))\\
        &= h(a_1,b_1)h(a_2,b_2)\qquad \forall (a_1,b_1),(a_2,b_2)\in A\times B
    \end{align*}
    Por tanto, $h$ es un homomorfismo. Veamos ahora que es inyectiva. Para ello, tenemos que:
    \begin{align*}
        \ker(h) &= \{(a,b)\in A\times B\mid h(a,b)=(1,1)\}\\
        &= \{(a,b)\in A\times B\mid (f(a),g(b))=(1,1)\} =\\&= \{(a,b)\in A\times B\mid f(a)=1\wedge g(b)=1\} =\\&= \{(1,1)\}
    \end{align*}
    Por tanto, $h$ es inyectiva. Veamos ahora que es sobreyectiva. Dado $(k,h)\in K\times H$, tomamos $(a,b)\in A\times B$ tales que $f(a)=k$ y $g(b)=h$ (que existe por ser $f,g$ biyectivas). Entonces:
    \begin{align*}
        h(a,b) &= (f(a),g(b)) = (k,h) \qquad \forall (a,b)\in A\times B
    \end{align*}
    Por tanto, $h$ es sobreyectiva. Concluimos que $h$ es un isomorfismo.

\end{ejercicio}

\begin{ejercicio}
    Sean $H,K,L$ y $M$ grupos tales que $H\times K\cong L\times M$. ¿Se verifica necesariamente que $H\cong L$ y $K\cong M$?

    Sabemos que $C_2\times C_3$ es cíclico de orden $6$, luego $C_2\times C_3\cong C_6$. Además, sabemos que para todo grupo $G$, se tiene que $G\cong G\times \{1\}$ usando como isomorfismo la aplicación:
    \Func{f}{G}{G\times \{1\}}{x}{(x,1)}

    Por tanto, $C_2\times C_3\cong C_6\cong C_6\times \{1\}$.\\

    No obstante, $|C_2|\neq |C_6|$ y $|C_3|\neq |1|$. Por tanto, tomando $H=C_2$, $K=C_3$, $L=C_6$ y $M=\{1\}$, se tiene que $H\times K\cong L\times M$. Sin embargo, no se verifica que $H\cong L$ y $K\cong M$.
\end{ejercicio}

\begin{ejercicio}
    Demostrar que no todo subgrupo de un producto directo $H\times K$ es de la forma $H_1\times K_1$, con $H_1\leq H$ y $K_1\leq K$.\\

    Trabajamos con el grupo directo $\bb{Z}_2\times \bb{Z}_2$. El grupo $\bb{Z}_2$ no tiene subgrupos propios, y por tanto los subgrupos suyos son $\{0\}$ y $\bb{Z}_2$. Por tanto, los subgrupos de $\bb{Z}_2\times \bb{Z}_2$ que se descomponen como producto directo de subgrupos de $\bb{Z}_2$ son:
    \begin{itemize}
        \item $\{0\}\times \{0\}=\{(0,0)\}$
        \item $\{0\}\times \bb{Z}_2=\{(0,0),(0,1)\}$
        \item $\bb{Z}_2\times \{0\}=\{(0,0),(1,0)\}$
        \item $\bb{Z}_2\times \bb{Z}_2=\{(0,0),(1,0),(0,1),(1,1)\}$
    \end{itemize}

    No obstante, consideramos el subgrupo de $\bb{Z}_2\times \bb{Z}_2$ dado por:
    \begin{equation*}
        \langle (1,1)\rangle = \{(0,0),(1,1)\}\qquad O((1,1))=\mcm(2,2)=2
    \end{equation*}

    Este subgrupo no es de la forma $H_1\times K_1$, con $H_1\leq H$ y $K_1\leq K$. Por tanto, no todo subgrupo de un producto directo $H\times K$ es de la forma $H_1\times K_1$, con $H_1\leq H$ y $K_1\leq K$.
\end{ejercicio}

\begin{ejercicio}
    Sean $H,K$ dos grupos y sean $H_1\lhd H$, $K_1\lhd K$. Demostrar que $H_1\times K_1\lhd H\times K$ y que
    \[
        \frac{H\times K}{H_1\times K_1}\cong \frac{H}{H_1}\times \frac{K}{K_1}.
    \]

    Como $H_1<H$ y $K_1<K$, se tiene que $H_1\times K_1<H\times K$. Veamos ahora que $H_1\times K_1\lhd H\times K$. Para cada $(h,k)\in H\times K$ y $(h_1,k_1)\in H_1\times K_1$, tenemos que:
    \begin{align*}
        (h,k)(h_1,k_1)(h,k)^{-1} &= (hh_1h^{-1},kk_1k^{-1})\in H_1\times K_1
    \end{align*}
    por ser $H_1\lhd H$ y $K_1\lhd K$. Por tanto, $H_1\times K_1\lhd H\times K$. Para ver el isomorfismo, consideramos la siguiente aplicación:
    \Func{f}{H\times K}{\frac{H}{H_1}\times \frac{K}{K_1}}{(h,k)}{(hH_1,kK_1)}

    Vemos que $f$ es un homomorfismo:
    \begin{align*}
        f((h_1,k_1)(h_2,k_2)) &= f(h_1h_2,k_1k_2) = (h_1h_2H_1,k_1k_2K_1)\\
        &= (h_1H_1\cdot h_2H_1,k_1K_1\cdot k_2K_1) = (h_1H_1,k_1K_1)(h_2H_1,k_2K_1)\\
        &= f(h_1,k_1)f(h_2,k_2)\qquad \forall (h_1,k_1),(h_2,k_2)\in H\times K
    \end{align*}

    Calculamos ahora su núcleo e imagen:
    \begin{align*}
        \ker(f) &= \{(h,k)\in H\times K\mid f(h,k)=(H_1,K_1)\}\\
        &= \{(h,k)\in H\times K\mid (hH_1,kK_1)=(H_1,K_1)\} =\\&= \{(h,k)\in H\times K\mid hH_1=H_1\wedge kK_1=K_1\} =  H_1\times K_1\\
        Im(f) &= \left\{(hH_1,kK_1)\in \frac{H}{H_1}\times \frac{K}{K_1}\mid (h,k)\in H\times K\right\} =\\&= \left\{(hH_1,kK_1)\in \frac{H}{H_1}\times \frac{K}{K_1}\mid h\in H\wedge k\in K\right\} = \frac{H}{H_1}\times \frac{K}{K_1}
    \end{align*}

    Por tanto, por el Primer Teorema de Isomorfía, se tiene que:
    \begin{equation*}
        \frac{H\times K}{H_1\times K_1}\cong \frac{H}{H_1}\times \frac{K}{K_1}
    \end{equation*}
\end{ejercicio}

\begin{ejercicio}
    Sean $H,K\lhd G$ tales que $H\cap K=\{1\}$. Demostrar que $G$ es isomorfo a un subgrupo de $G/H\times G/K$.\\

    Definimos la aplicación:
    \Func{f}{G}{G/H\times G/K}{g}{(gH,gK)}

    Vemos que $f$ es un homomorfismo:
    \begin{align*}
        f(gh) &= (ghH,ghK) = (gH,gK)(hH,hK) =\\&= f(g)f(h)\qquad \forall g,h\in G
    \end{align*}

    Calculamos su núcleo:
    \begin{align*}
        \ker(f) &= \{g\in G\mid f(g)=(H,K)\}\\
        &= \{g\in G\mid (gH,gK)=(H,K)\} =\\&= \{g\in G\mid gH=H\wedge gK=K\} = H\cap K = \{1\}
    \end{align*}

    Por tanto, tenemos que:
    \begin{equation*}
        \dfrac{G}{\ker(f)}=\dfrac{G}{\{1\}} \cong G
    \end{equation*}

    Por tanto, por el Primer Teorema de Isomorfía, se tiene que:
    \begin{equation*}
        G\cong \dfrac{G}{\ker(f)}\cong Im(f)
    \end{equation*}

    Como $G<G$ y $f$ es un homomorfismo, se tiene que $Im(f)<G/H\times G/K$. Por tanto, $G$ es isomorfo a $Im(f)$, que es un subgrupo de $G/H\times G/K$.
\end{ejercicio}

\begin{ejercicio}
    Sean $H,K\lhd G$ tales que $HK=G$. Demostrar que
    \[
        \frac{G}{H\cap K}\cong \frac{H}{H\cap K}\times \frac{K}{H\cap K}\cong \frac{G}{H}\times \frac{G}{K}.
    \]

    La demostración no es directa, y la dividiremos en dos partes. Por un lado, demostraremos la primera relación de isomofía, y posteriormente veremos la segunda.\\

    Por el Segundo Teorema de Isomorfía aplicado dos veces, tenemos que $H\cap K\lhd H,K$. Veamos ahora que $H\cap K\lhd G$. Para ello, tomamos $g\in G$ y $x\in H\cap K$. Como $G=HK$, tenemos que $g=hk$ con $h\in H$ y $k\in K$. Entonces:
    \begin{align*}
        gxg^{-1} &= (hk)x(hk)^{-1} = (hk)x(k^{-1}h^{-1}) =\\&= h(kxk^{-1})h^{-1} \stackrel{(\ast)}{=} h\wt{k}h^{-1}\stackrel{(\ast\ast)}{\in} H\cap K
    \end{align*}
    donde $(\ast)$ se cumple porque $H\cap K\lhd K$, por lo que $kxk^{-1}=\wt{k}\in K$ y $(\ast\ast)$ se cumple porque $H\cap K\lhd H$, por lo que $h\wt{k}h^{-1}\in H$. Por tanto, $H\cap K\lhd G$, y el primer cociente tiene sentido.\\

    Definimos la aplicación:
    \Func{f}{G}{\frac{H}{H\cap K}\times \frac{K}{H\cap K}}{g}{(h(H\cap K),k(H\cap K))}

    Veamos que está bien definida, puesto que la descomposición no tiene por qué ser única. Sean $k_1,k_2\in K$ y $h_1,h_2\in H$ tales que $g=k_1h_1=k_2h_2$. Entonces, en primer lugar vemos que $h_2^{-1}h_1 = k_2k_1^{-1}\in H\cap K$. Por tanto:
    \begin{itemize}
        \item $h_1=k_1^{-1}k_2h_2$, con $k_1^{-1}k_2\in K\cap K$ y $h_2\in H$, por lo que $h_1(H\cap K)=h_2(H\cap K)$.
        \item $k_1=k_2h_1^{-1}h_2$, con $h_1^{-1}h_2\in H\cap H$ y $k_2\in K$, por lo que $k_1(H\cap K)=k_2(H\cap K)$.
    \end{itemize}
    Por tanto, $f$ está bien definida. Veamos que es un homomorfismo.
    Dados $g_1,g_2\in G$, $\exists h_1,h_2\in H$ y $k_1,k_2\in K$ tales que $g_1=h_1k_1$ y $g_2=h_2k_2$. Entonces:
    \begin{align*}
        f(g_1g_2) &= f(h_1k_1h_2k_2) =\\&= (h_1h_2(H\cap K),k_1k_2(H\cap K)) =\\&= (h_1(H\cap K),k_1(H\cap K))(h_2(H\cap K),k_2(H\cap K))\\
        &= f(g_1)f(g_2)\qquad \forall g_1,g_2\in G
    \end{align*}

    Calculamos su núcleo:
    \begin{align*}
        \ker(f) &= \{g\in G\mid f(g)=(H\cap K,H\cap K)\}\\
        &= \{g=hk\in G\mid (h(H\cap K),k(H\cap K))=(H\cap K,H\cap K)\} =\\&= \{g=hk\in G\mid h(H\cap K)=H\cap K\wedge k(H\cap K)=H\cap K\} =\\&=
        \{g=hk\in G\mid h,k\in H\cap K\} \AstIg H\cap K
    \end{align*}
    donde $(\ast)$ se cumple porque la inclusión $\ker f \subseteq H\cap K$ se cumple por ser $H\cap K$ cerrado para el producto; mientras que la inclusión $H\cap K \subseteq \ker f$ se cumple tomando $g=g\cdot 1$.\\

    Veamos ahora que es sobreyectiva. Dado $(h(H\cap K),k(H\cap K))\in \frac{H}{H\cap K}\times \frac{K}{H\cap K}$, tomamos $g=hk\in G$. Entonces:
    \begin{align*}
        f(g) &= f(hk) = (h(H\cap K),k(H\cap K)) \qquad \forall g\in G
    \end{align*}

    Por tanto, $f$ es sobreyectiva. Por el Primer Teorema de Isomorfía, se tiene que:
    \begin{equation*}
        \frac{G}{H\cap K}\cong \frac{H}{H\cap K}\times \frac{K}{H\cap K}
    \end{equation*}~\\

    Llegamos a este punto, tan solo nos falta probar que:
    \begin{equation*}
        \frac{H}{H\cap K}\times \frac{K}{H\cap K}\cong \frac{G}{H}\times \frac{G}{K}
    \end{equation*}

    Para ello, aplicamos en primer lugar dos veces el Segundo Teorema de Isomorfía.
    \begin{itemize}
        \item $H,K<G$ y $K\lhd G$:
        \begin{equation*}
            \frac{G}{K}\cong \frac{H}{H\cap K}
        \end{equation*}

        \item $H,K<G$ y $H\lhd G$:
        \begin{equation*}
            \frac{G}{H}\cong \frac{K}{H\cap K}
        \end{equation*}
    \end{itemize}

    Por tanto, tenemos que:
    \begin{align*}
        \frac{H}{H\cap K}\times \frac{K}{H\cap K} &\cong \frac{G}{K}\times \frac{G}{H}
    \end{align*}

    Como $G/H\times G/K\cong G/K\times G/H$, tenemos que:
    \begin{align*}
        \frac{H}{H\cap K}\times \frac{K}{H\cap K} &\cong \frac{G}{K}\times \frac{G}{H} \cong \frac{G}{H}\times \frac{G}{K}
    \end{align*}
    Por tanto, se cumple la segunda relación de isomorfía. Concluimos que:
    \begin{equation*}
        \frac{G}{H\cap K}\cong \frac{H}{H\cap K}\times \frac{K}{H\cap K}\cong \frac{G}{H}\times \frac{G}{K}
    \end{equation*}
\end{ejercicio}

\begin{ejercicio}
    Demostrar que si $G$ es un grupo que es producto directo interno de subgrupos $H$ y $K$, y $N\lhd G$ tal que $N\cap H=\{1\}=N\cap K$, entonces $N$ es abeliano.\\

    En primer lugar, veamos que $N$ conmuta con $H$; es decir, que dado $n\in N$ y $h\in H$, se tiene que $nh=hn$. Equivalentemente, veremos que $nhn^{-1}h^{-1}=1$.
    \begin{itemize}
        \item Como $N$ es un grupo, $n^{-1}\in N$. Como $H<G$, $h\in G$. Como $N\lhd G$, tenemos que $hn^{-1}h^{-1}\in N$. Como $N$ es cerrado para el producto, tenemos que:
        \begin{equation*}
            n\ hn^{-1}h^{-1} \in N
        \end{equation*}

        \item Como $N<G$, $n\in G$. Como $H\lhd G$ por la caracterización del producto directo interno, tenemos que $nhn^{-1}\in H$. Como $H$ es cerrado para el producto, tenemos que:
        \begin{equation*}
            nhn^{-1}h^{-1} \in H
        \end{equation*}
    \end{itemize}

    Por tanto, $nhn^{-1}h^{-1}\in N\cap H$. Como $N\cap H=\{1\}$, tenemos que $nhn^{-1}h^{-1}=1$. Por tanto, $nh=hn$ para todo $n\in N$ y $h\in H$. De forma análoga, como $K\lhd G$, tenemos que $nk=kn$ para todo $n\in N$ y $k\in K$. Sea ahora $g\in G$ y $n\in N$. Entonces, como $G=HK$, tenemos que $g=hk$ con $h\in H$ y $k\in K$. Entonces:
    \begin{align*}
        gn &= (hk)n = h(kn) = h(nk) = (hn)k = (nh)k = n(hk) = ng
    \end{align*}
    Por tanto, $gn=ng$ para todo $g\in G$ y $n\in N$. Como $N<G$, tenemos que $N$ es abeliano.
\end{ejercicio}

\begin{ejercicio}
    Dar un ejemplo de un grupo $G$ que sea producto directo interno de dos subgrupos propios $H$ y $K$, y que contenga a un subgrupo normal no trivial $N$ tal que $N\cap H=\{1\}=N\cap K$. Concluir que para $N\lhd H\times K$ es posible que se tenga
    \[
        N\neq (N\cap (H\times \{1\}))\times (N\cap (\{1\}\times K)).
    \]

    Sea $G=\bb{Z}_2\times \bb{Z}_2$, $H=\langle(1,0)\rangle$ y $K=\langle(0,1)\rangle$.
    Sabemos que $G\cong V$ abeliano, luego $H,K\lhd G$. Además, $H\cap K=\{(0,0)\}$. Por último, se tiene que $G=HK$. Por tanto, $G$ es producto directo interno de $H$ y $K$. Sea ahora:
    \begin{equation*}
        N=\langle(1,1)\rangle = \{(0,0),(1,1)\}
    \end{equation*}

    Sabemos que $N<G$, luego $N\lhd G$. Además, $N\cap H=\{(0,0)\}=N\cap K$. Por tanto, se cumple la condición del ejercicio.\\

    Sea ahora $\varphi:G\to H\times K$ el isomorfismo correspondiente. Sea ahora el subgrupo $\varphi(N)<H\times K$. Como el ser normal se mantiene por isomorfismos, $\varphi(N)\lhd H\times K$. Veamos ahora que:
    \begin{align*}
        \varphi((0,0))&=((0,0),(0,0))\\
        \varphi((1,1))&=((1,0),(0,1))\\
        \varphi(N) &= \{((0,0),(0,0)),((1,0),(0,1))\}
    \end{align*}
    Por tanto, $\varphi(N)$ no puede ser producto cartesiano, puesto que $((0,0),(0,1))$ debería pertenecer también. Por tanto:
    \begin{align*}
        \varphi(N)\neq (\varphi(N)\cap (H\times \{1\}))\times (\varphi(N)\cap (\{1\}\times K)).
    \end{align*}
    Y se tiene demostrado lo pedido.
\end{ejercicio}

\begin{ejercicio}
    Sea $G$ un grupo finito que sea producto directo interno de dos subgrupos $H$ y $K$ tales que $\mcd(|H|,|K|)=1$. Demostrar que para todo subgrupo $N\leq G$ se verifica que $N=(N\cap H)\times (N\cap K)$.\\

    \begin{comment}

    Como $G$ es producto directo interno de $H$ y $K$, consideramos el siguiente isomorfismo:
    \Func{f}{G}{H\times K}{g=hk}{(h,k)}

    Dado $N\leq G$, consideramos $f(N)<H\times K$. Como $\mcd(|H|,|K|)=1$, se tiene que $\exists_1 H_1\leq H$ y $K_1\leq K$ tales que $f(N)=H_1\times K_1$. De hecho, se tiene que:
    \begin{align*}
        H_1 = \pi_1(f(N)) &= \{h\in H\mid \exists k\in K, n\in N\text{ tal que }f(n)=(h,k)\}\\
        &= \{h\in H\mid \exists k\in K\ \text{ tal que }hk\in N\}\\
        K_1 = \pi_2(f(N)) &= \{k\in K\mid \exists h\in H, n\in N\text{ tal que }f(n)=(h,k)\}\\
        &= \{k\in K\mid \exists h\in H\ \text{ tal que }hk\in N\}
    \end{align*}
    \end{comment}
\end{ejercicio}

\begin{ejercicio}
    Sea $G$ un grupo y sea $f:G\to G$ un endomorfismo idempotente (esto es, verificando que $f^2=f$) y tal que $Im(f)\lhd G$. Demostrar que
    \[
        G\cong Im(f)\times \ker(f).
    \]

    \begin{description}
        \item[Opción 1. Método Directo]~
        
        Sea la siguiente aplicación:
        \Func{\varphi}{Im(f)\times \ker(f)}{G}{(x,y)}{xy}

        Veamos que $\varphi$ es un isomorfismo. Para ello, previamente veremos que para todo $x\in Im(f)$, se tiene que $f(x)=x$. Dado $x\in Im(f)$, existe $y\in G$ tal que $f(y)=x$. Entonces:
        \begin{align*}
            f(x) &= f(f(y)) = f^2(y) \AstIg f(y) = x
        \end{align*}
        donde en $(\ast)$ se cumple porque $f$ es idempotente.

        \begin{description}
            \item [Homomorfismo]~\\
            Veamos que $\varphi$ es homomorfismo. Para cada $(x_1,y_1),(x_2,y_2)\in Im(f)\times~\ker(f)$, tenemos que:
            \begin{align*}
                \varphi((x_1,y_1)(x_2,y_2)) &= \varphi(x_1x_2,y_1y_2) = x_1x_2y_1y_2\\
                \varphi(x_1,y_1)\varphi(x_2,y_2) &= (x_1y_1)(x_2y_2) = x_1y_1x_2y_2
            \end{align*}

            Veamos que $x_2y_1 = y_1x_2$. Es decir, que dados $x\in Im(f)$ y $y\in \ker(f)$, tenemos que $xy=yx$. Como $Im(f)\lhd G$, y $\ker f \subset G$, tenemos que $yxy^{-1}\in Im(f)$. Por lo visto anteriormente, $f(yxy^{-1})=yxy^{-1}$. Aplicando que $f$ es un homomorfismo (estamos demostrando que $\varphi$ lo es, pero sabemos que $f$ lo es), tenemos que:
            \begin{align*}
                yxy^{-1} &= f(yxy^{-1}) = f(y)f(x)f(y^{-1}) = f(y) x f(y)^{-1} \AstIg 1x1 = x\Longrightarrow yx=xy
            \end{align*}
            donde en $(\ast)$ se cumple porque $y\in \ker(f)$ y $x\in Im(f)$.\\

            Por tanto, $\varphi$ es un homomorfismo.
            
            \item [Inyectividad]~\\
            Veamos que $\varphi$ es inyectiva. Para ello, sean $(x_1,y_1),(x_2,y_2)\in Im(f)\times~\ker(f)$ tales que $\varphi(x_1,y_1)=\varphi(x_2,y_2)$. Entonces:
            \begin{align*}
                x_1y_1 &= x_2y_2 \Longrightarrow x_2^{-1}x_1=y_2y_1^{-1}\in \ker(f)\cap Im(f)
            \end{align*}

            Veamos ahora que $\ker(f)\cap Im(f)=\{1\}$. Sea $x\in \ker(f)\cap Im(f)$. Como $x\in \ker(f)$, tenemos que $f(x)=1$, y como $x\in Im(f)$, $f(x)=x$. Entonces $1=f(x)=x$, por lo que $x=1$. Por tanto, $\ker(f)\cap Im(f)=\{1\}$, por lo que:
            \begin{align*}
                x_2^{-1}x_1 &= 1\Longrightarrow x_1=x_2\\
                y_2y_1^{-1} &= 1\Longrightarrow y_1=y_2
            \end{align*}

            Por tanto, $\varphi$ es inyectiva.
            
            \item [Sobreyectividad]~\\
            Veamos que es sobreyectiva. Dado $g\in G$, tomamos $x=f(g)\in Im(f)$ e $y=f(g^{-1})g$. Veamos en primer lugar que $y\in \ker(f)$:
            \begin{align*}
                f(y) &= f(f(g^{-1})g) = f(f(g^{-1}))f(g) = f^2(g^{-1})f(g)
                = f(g^{-1})f(g) = f(g^{-1}g) = f(1) = 1
            \end{align*}

            Por tanto, $y\in \ker(f)$. Veamos ahora que $\varphi(x,y)=g$. Entonces:
            \begin{align*}
                \varphi(x,y) &= \varphi(f(g),f(g^{-1})g) = f(g)f(g^{-1})g = f(gg^{-1})g = f(1)g = 1g = g
            \end{align*}

            Por tanto, $\varphi$ es sobreyectiva.
        \end{description}

        Concluimos que $\varphi$ es un isomorfismo. Por tanto, $G\cong Im(f)\times \ker(f)$.

        \item[Opción 2.]~
        
        En primer lugar, sabemos que $\ker f \lhd G$, y por hipótesis $Im(f)\lhd G$. Veamos ahora que $\ker (f) \cap Im(f)=\{1\}$. Previamente veremos que para todo $x\in Im(f)$, se tiene que $f(x)=x$. Dado $x\in Im(f)$, existe $y\in G$ tal que $f(y)=x$. Entonces:
        \begin{align*}
            f(x) &= f(f(y)) = f^2(y) \AstIg f(y) = x
        \end{align*}
        donde en $(\ast)$ se cumple porque $f$ es idempotente. Sea ahora $x\in \ker(f)\cap Im(f)$. Como $x\in \ker(f)$, tenemos que $f(x)=1$, y como $x\in Im(f)$, $f(x)=x$. Entonces $1=f(x)=x$, por lo que $x=1$. Por tanto, $\ker(f)\cap Im(f)=\{1\}$.
        Veamos ahora que $G=\ker(f)Im(f)$:
        \begin{description}
            \item[$\supseteq$)] Como $\ker(f),Im(f)<G$, y $G$ es un grupo, tenemos que $\ker(f)Im(f)\subset G$.
            \item[$\subseteq$)] Sea $g\in G$. Consideramos ahora $x=f(g)\in Im(f)$ e $y=f(g^{-1})g$. Veamos en primer lugar que $y\in \ker(f)$:
            \begin{align*}
                f(y) &= f(f(g^{-1})g) = f(f(g^{-1}))f(g) = f^2(g^{-1})f(g)
                = f(g^{-1})f(g) = f(g^{-1}g) = f(1) = 1
            \end{align*}
            Por tanto, $y\in \ker(f)$. Veamos ahora que $g=xy$. Entonces:
            \begin{align*}
                xy = f(g)f(g^{-1})g = f(gg^{-1})g = f(1)g = 1g = g
            \end{align*}
            Por tanto, $g=xy\in \ker(f)Im(f)$.
        \end{description}

        Por tanto, $G=\ker(f)Im(f)$. Por la condición suficiente de producto directo interno, tenemos que $G\cong \ker(f)\times Im(f)$.
        

    \end{description}
\end{ejercicio}

\begin{ejercicio}
    Sea $S$ un subconjunto de un grupo $G$. Se llama \emph{centralizador} de $S$ en $G$ al conjunto
    \[
        C_G(S) = \{x\in G\mid xs=sx\ \forall s\in S\}
    \]
    y se llama \emph{normalizador} de $S$ en $G$ al conjunto
    \[
        N_G(S) = \{x\in G\mid xS=Sx\}.
    \]
    \begin{enumerate}
        \item Demostrar que $N_G(S)\leq G$.
        
        Comprobamos las tres condiciones:
        \begin{itemize}
            \item $1\in N_G(S)$, ya que $1S=S1=S$.
            \item Sea $x,y\in N_G(S)$. Entonces $xS=Sx$ y $yS=Sy$. Entonces:
                \begin{align*}
                    (xy)S &= x(yS) = x(Sy) = (xS)y = (Sx)y = S(xy)
                \end{align*}
                Por tanto, es cerrado para el producto.
            \item Sea $x\in N_G(S)$, $xS=Sx$. Entonces:
                \begin{align*}
                    xS &= Sx \Longrightarrow S = x^{-1}Sx \Longrightarrow Sx^{-1} = x^{-1}S \Longrightarrow x^{-1}\in N_G(S)
                \end{align*}
                Por tanto, es cerrado para inversos.
        \end{itemize}
        \item Demostrar que $C_G(S)\lhd N_G(S)$.
        
        Veamos en primer lugar que $C_G(S)\leq N_G(S)$. Para ello, vemos en primer lugar que $C_G(S)\subseteq N_G(S)$. Fijado $x\in C_G(S)$, tenemos que $xs=sx$ para todo $s\in S$, por lo que:
        \begin{equation*}
            xS = \{xs\mid s\in S\} = \{sx\mid s\in S\} = Sx
        \end{equation*}
        
        Comprobamos ahora las tres condiciones para ser subgrupo:
        \begin{itemize}
            \item $1\in C_G(S)$, ya que $1s=s=s1$.
            \item Sea $x,y\in C_G(S)$. Entonces $xs=sx$ y $ys=sy$ para todo $s\in S$. Entonces:
                \begin{align*}
                    (xy)s &= x(ys) = x(sy) = (xs)y = (sx)y = s(xy)
                \end{align*}
                Por tanto, es cerrado para el producto.
            \item Sea $x\in C_G(S)$, $xs=sx$. Entonces:
                \begin{align*}
                    xs &= sx \Longrightarrow s = x^{-1}sx \Longrightarrow sx^{-1} = x^{-1}s \Longrightarrow x^{-1}\in C_G(S)
                \end{align*}
                Por tanto, es cerrado para inversos.
        \end{itemize}

        Comprobamos ahora que $C_G(S)\lhd N_G(S)$. Para ello, tomamos $x\in N_G(S)$ y $y\in C_G(S)$, y veamos que $xyx^{-1}\in C_G(S)$. Para ello, tomamos $s\in S$, y como $x\in N_G(S)$, tenemos que $xS=Sx$, por lo que $sx = xs'$ con $s'\in S$ y por tanto $x^{-1}sx \in S$. Entonces:
        \begin{equation*}
            yx^{-1}sx = x^{-1}sxy
            \Longrightarrow xyx^{-1}s = sxyx^{-1}
            \Longrightarrow xyx^{-1}\in C_G(S)
        \end{equation*}
        \item Demostrar que si $S\leq G$ entonces $S\lhd N_G(S)$.
        
        Sea $s\in S$. Veamos que $s\in N_G(S)$, es decir que $sS=Ss$.
        \begin{description}
            \item[$\subset)$] Sea $x\in sS$. Entonces, $\exists s'\in S$ tal que $x=ss'$. Entonces, como $ss'(s^{-1})\in S$, tenemos que:
            \begin{equation*}
                x=ss'=ss'(s^{-1})s\in Ss
            \end{equation*}

            \item[$\supset)$] Sea $x\in Ss$. Entonces, $\exists s'\in S$ tal que $x=s's$. Entonces, como $(s^{-1})s's\in S$, tenemos que:
            \begin{equation*}
                x=s's=ss^{-1}s's\in sS
            \end{equation*}
        \end{description}

        Por tanto, $S\subset N_G(S)$. Como $S\subset N_G(S)$ y $S,N_G(S)< G$, tenemos que $S< N_G(S)$. Veamos ahora que $S\lhd N_G(S)$. Para ello, tomamos $x\in N_G(S)$ y $s\in S$, y veamos que $xsx^{-1}\in S$. Entonces:
        \begin{equation*}
            xsx^{-1} \AstIg s'xx^{-1}=s'\in S
        \end{equation*}
        donde en $(\ast)$ hemos empleado que $x\in N_G(S)$, por lo que $xS=Sx$.
    \end{enumerate}
\end{ejercicio}

\begin{ejercicio}
    Sea $G$ un grupo y $H$ y $K$ subgrupos suyos con $H\subseteq K$. Entonces demostrar que $H$ es normal en $K$ si y sólo si $K< N_G(H)$. (Así, el normalizador $N_G(H)$ queda caracterizado como el mayor subgrupo de $G$ en el que $H$ es normal.)
    \begin{description}
        \item[$\Longrightarrow)$]
        Sea $H\lhd K$, y veamos que $K\subset N_G(H)$. Como $H\lhd K$, tenemos que $kH=Hk$ para todo $k\in K$, luego $K\subset N_G(H)$. Como además $K,N_G(H)<G$, tenemos que $K<N_G(H)$.

        \item[$\Longleftarrow)$]
        Sea $K<N_G(H)$, luego $K\subset N_G(H)$. Entonces, para todo $k\in K$, tenemos que $kH=Hk$. Por tanto, $H\lhd K$.        
    \end{description}
\end{ejercicio}

\begin{ejercicio} Sea $G$ un grupo.
    \begin{enumerate}
        \item Demostrar que $C_G(Z(G))=G$ y que $N_G(Z(G))=G$.\\
        
        Veamos en primer lugar que $C_G(Z(G))=G$.
        \begin{description}
            \item[$\subset)$] Sea $x\in C_G(Z(G))$, luego trivialmente $x\in G$.
            \item[$\supset)$] Sea $x\in G$, luego $xz=zx$ para todo $z\in Z(G)$. Por tanto, $x\in C_G(Z(G))$. 
        \end{description}

        Veamos ahora que $N_G(Z(G))=G$.
        \begin{description}
            \item[$\subset)$] Sea $x\in N_G(Z(G))$, luego trivialmente $x\in G$.
            \item[$\supset)$] Sea $x\in G$, luego $xz=zx$ para todo $z\in Z(G)$. Por tanto, $xZ(G)=Z(G)x$, luego $x\in N_G(Z(G))$.
        \end{description}
        Análogamente, también podríamos haber usando que $G=C_G(Z(G))$ y $C_G(Z(G))\lhd N_G(Z(G))$.
        \item Si $G$ es un grupo y $H<G$ ¿Cuándo es $G=N_G(H)$? ¿Y cuándo es $G=C_G(H)$?
        
        Por el ejercicio anterior, que nos caracteriza el normalizador, sabemos que:
        \begin{equation*}
            G=N_G(H) \Longleftrightarrow H\lhd G
        \end{equation*}

        Por otro lado, sabemos que $C_G(H)=G$ si y sólo si $H\subset Z(G)$.
        \item Si $H\leq G$ con $|H|=2$, demostrar que $N_G(H)=C_G(H)$. Deducir que $H\lhd G$ si y sólo si $H\subset Z(G)$.
        
        Si $|H|=2$, entonces $H=\{1,h\}$ con $h\in G\setminus \{1\}$, luego $h=h^{-1}$. Veamos que $N_G(H)=C_G(H)$.
        \begin{description}
            \item[$\subset)$] Sea $x\in N_G(H)$, luego $xH=Hx$. Como $xh\in Hx$, entonces hay dos opciones:
            \begin{itemize}
                \item $xh=1$, luego $x=h^{-1}$. Por tanto, $xh=h^{-1}h=1=hh^{-1}=hx$.
                \item $xh=h$, luego $x=1$. Por tanto, $xh=1h=h=h1=hx$.
            \end{itemize}

            En cual caso, $xh=hx$, luego $x\in C_G(H)$.

            \item[$\supset)$] Como $C_G(H)\lhd N_G(H)$, se tiene.
        \end{description}

        Demostramos ahora que $H\lhd G$ si y sólo si $H\subset Z(G)$.
        \begin{description}
            \item[$\Longrightarrow)$] Como $H\lhd G$, por el apartado anterior tenemos que $G=N_G(H)$. Como acacaba de ver que $N_G(H)=C_G(H)$, tenemos que $G=C_G(H)$. Por tanto, para todo $x\in G$, tenemos que $xh=hx$ para todo $h\in H$. Por tanto, $H\subset Z(G)$.
            
            \item[$\Longleftarrow)$] Como $H\subset Z(G)$, tenemos que $C_G(H)=G$. Por tanto, se tiene que $N_G(H)=C_G(H)=G$, luego $H\lhd G$.
        \end{description}
    \end{enumerate}
\end{ejercicio}

\begin{ejercicio}
    Sea $G$ un grupo arbitrario. Para dos elementos $x,y\in G$ se define su \emph{conmutador} como el elemento
    \[
        [x,y] = xyx^{-1}y^{-1}.
    \]
    \begin{observacion}
        (El conmutador recibe tal nombre porque $[x,y]yx=xy$.)
    \end{observacion}
    
    Como $[x,y]^{-1}=[y,x]$, el inverso de un conmutador es un conmutador. Sin embargo el producto de dos conmutadores no tiene porqué ser un conmutador. Entonces se define el \emph{subgrupo conmutador} o (primer) \emph{subgrupo derivado} de $G$, denotado $[G,G]$, como el subgrupo generado por todos los conmutadores de $G$.
    \begin{enumerate}
        \item Demostrar que, $\forall a,x,y\in G$, se tiene que $a[x,y]a^{-1}=[axa^{-1},aya^{-1}]$.
        \begin{align*}
            [axa^{-1},aya^{-1}] &= axa^{-1}aya^{-1}ax^{-1}a^{-1}ay^{-1}a^{-1} = axyx^{-1}y^{-1}a^{-1} = a[x,y]a^{-1}
        \end{align*}
        \item Demostrar que $[G,G]\lhd G$.
        
        Tenemos que:
        \begin{equation*}
            [G,G] = \langle [x,y]\mid x,y\in G\rangle
        \end{equation*}

        Como $[x,y]\in G$ para todo $x,y\in G$ y $[G,G]$ es un grupo, tenemos que $[G,G]<~G$. Veamos que $[G,G]\lhd G$. Para ello, tomamos $g\in G$ y $z\in [G,G]$. Entonces, como $z\in [G,G]$, tenemos que:
        \begin{equation*}
            z = [x_1,y_1]\cdots[x_n,y_n]\qquad x_i,y_i\in G\ \forall i=1,\ldots,n
        \end{equation*}

        Entonces:
        \begin{align*}
            gzg^{-1} &= g[x_1,y_1]\cdots[x_n,y_n]g^{-1} =\\&=
            g[x_1,y_1]g^{-1}g[x_2,y_2]g^{-1}\cdots g[x_n,y_n]g^{-1} =\\&=
            [g x_1 g^{-1},g y_1 g^{-1}]\cdots[g x_n g^{-1},g y_n g^{-1}] \in [G,G]
        \end{align*}
        \item Demostrar que el grupo cociente $G/[G,G]$, que se representa por $G^{ab}$, es un grupo abeliano (que se llama el abelianizado de $G$).
        
        Dados $x,y\in G$, hemos de ver que $x[G,G]y[G,G]=y[G,G]x[G,G]$. Usando el producto en el cociente, hemos de ver que:
        \begin{align*}
            xy[G,G] &= yx[G,G]
        \end{align*}

        Consideramos ahora $[y^{-1},x^{-1}]\in [G,G]$. Entonces:
        \begin{align*}
            xy[y^{-1},x^{-1}] &= xyy^{-1}x^{-1}yx = yx
        \end{align*}

        Por tanto, $xy[G,G]=yx[G,G]$, luego $G/[G,G]$ es abeliano.
        \item Demostrar que $G$ es abeliano si y sólo si $[G,G]=1$.
        \begin{description}
            \item[$\Longrightarrow)$] Si $G$ es abeliano, entonces:
            \begin{align*}
                [G,G] &= \langle [x,y]\mid x,y\in G\rangle = \langle xyx^{-1}y^{-1}\mid x,y\in G\rangle =\\&= \langle xx^{-1}yy^{-1}\mid x,y\in G\rangle = \langle 1\mid x,y\in G\rangle = \{1\}
            \end{align*}

            \item[$\Longleftarrow)$] Si $[G,G]=1$, entonces dados $x,y\in G$, tenemos que $[x,y]=1$, luego:
            \begin{align*}
                1=[x,y]=xyx^{-1}y^{-1}=xy(yx)^{-1} \Longrightarrow xy=yx
            \end{align*}
            Por tanto, $G$ es abeliano.
        \end{description}
        \item Sea $N\lhd G$. Demostrar que el grupo cociente $G/N$ es abeliano si y sólo si $[G,G]<N$ (así que el grupo $[G,G]$ es el menor subgrupo normal de $G$ tal que el cociente es abeliano).
        \begin{description}
            \item[$\Longrightarrow)$] Sea $G/N$ abeliano, y consideramos la proyección en el cociente:
            \Func{p}{G}{G/N}{g}{gN}

            Sabemos que $p$ es un homomorfismo. Dados $x,y\in G$, tenemos que:
            \begin{align*}
                p([x,y]) &= p(xyx^{-1}y^{-1}) = p(x)p(y)p(x)^{-1}p(y)^{-1} = 1
            \end{align*}

            Por tanto, por ser $p$ un homomorfismo, tenemos que $p([G,G])=\{1\}$, luego $[G,G]\subseteq \ker p = N$. Por tanto, como además $[G,G]$ es un grupo, se tiene que $[G,G]<N$.

            \item[$\Longleftarrow)$] Sea $[G,G]<N$. Sean ahora $x,y\in G$, y buscamos ver que $xyN=yxN$. Tenemos que:
            \begin{equation*}
                xy[y^{-1},x^{-1}] = xyy^{-1}x^{-1}yx = yx
            \end{equation*}
            Como $[y^{-1},x^{-1}]\in [G,G]<N$, tenemos que $xyN = yxN$. Por tanto:
            \begin{align*}
                (xN)(yN) &= xyN = yxN = (yN)(xN)
            \end{align*}

            Por tanto, $G/N$ es abeliano.
        \end{description}
    \end{enumerate}
\end{ejercicio}

\begin{ejercicio}~\label{ej:4.34}
    \begin{enumerate}
        \item Calcular el subgrupo conmutador de los grupos $S_3$, $A_4$, $D_4$ y $Q_2$.
        \begin{enumerate}
            \item $S_3$:
            
            Recordemos que el subgrupo conmutador de $G$ es el menor subgrupo normal de $G$ tal que el cociente es abeliano. El diagrama de Hasse de $S_3$ conviene tenerlo presente, y se encuentra en la Figura~\ref{fig:ej11_S3}. Comprobemos cada subgrupo de $S_3$, de menor a mayor orden:
            \begin{itemize}
                \item $\{1\}$ efectivamente cumple que $\{1\}\lhd S_3$, pero:
                \begin{align*}
                    S_3/\{1\} &\cong S_3
                \end{align*}
                Por tanto, el cociente no es abeliano, luego $\{1\}\neq [S_3,S_3]$.

                \item $\langle (1\ 2)\rangle$.
                \begin{equation*}
                    (2\ 3)(1\ 2)(2\ 3)^{-1} = (2\ 3)(1\ 2)(2\ 3) = (1\ 3)\notin \langle (1,2)\rangle
                \end{equation*}
                Por tanto, $\langle (1\ 2)\rangle \cancel{\lhd} S_3$, luego $\langle (1\ 2)\rangle \neq [S_3,S_3]$.

                \item $\langle (1 3)\rangle$.
                \begin{equation*}
                    (2\ 3)(1\ 3)(2\ 3)^{-1} = (2\ 3)(1\ 3)(2\ 3) = (1\ 2)\notin \langle (1,3)\rangle
                \end{equation*}
                Por tanto, $\langle (1\ 3)\rangle \cancel{\lhd} S_3$, luego $\langle (1\ 3)\rangle \neq [S_3,S_3]$.

                \item $\langle (2\ 3)\rangle$.
                \begin{equation*}
                    (1\ 2)(2\ 3)(1\ 2)^{-1} = (1\ 2)(2\ 3)(1\ 2) = (1\ 3)\notin \langle (2,3)\rangle
                \end{equation*}
                Por tanto, $\langle (2\ 3)\rangle \cancel{\lhd} S_3$, luego $\langle (2\ 3)\rangle \neq [S_3,S_3]$.

                \item $\langle (1\ 2\ 3)\rangle$.
                
                Sabemos que $|\langle (1\ 2\ 3)\rangle|=3$, por lo que $[S_3,\langle (1\ 2\ 3)\rangle]=2$, luego $\langle (1\ 2\ 3)\rangle \lhd S_3$. Por tanto, es un candidato a ser el subgrupo conmutador. Veamos si $S_3/\langle (1\ 2\ 3)\rangle$ es abeliano:
                \begin{equation*}
                    \left|\dfrac{S_3}{\langle (1\ 2\ 3)\rangle}\right| = \frac{|S_3|}{|\langle (1\ 2\ 3)\rangle|} = \frac{6}{3} = 2
                    \Longrightarrow \dfrac{S_3}{\langle (1\ 2\ 3)\rangle} \cong C_2
                \end{equation*}

                Por tanto, $S_3/\langle (1\ 2\ 3)\rangle$ es abeliano, luego:
                \begin{equation*}
                    [S_3,S_3] = \langle (1\ 2\ 3)\rangle = A_3
                \end{equation*}
            \end{itemize}

            \item $A_4$:
            
            El diagrama de Hasse de $A_4$ conviene tenerlo presente, y se encuentra en la Figura~\ref{fig:ej11_A4}. Comprobemos cada subgrupo de $A_4$, de menor a mayor orden:
            \begin{itemize}
                \item $\{1\}$ efectivamente cumple que $\{1\}\lhd A_4$, pero:
                \begin{align*}
                    A_4/\{1\} &\cong A_4
                \end{align*}
                Por tanto, el cociente no es abeliano, luego $\{1\}\neq [A_4,A_4]$.

                \item $\langle (1\ 2)(3\ 4)\rangle$.
                \begin{align*}
                    [(1\ 2\ 3)(1\ 2)(3\ 4)(1\ 2\ 3)^{-1}](1) &= [(1\ 2\ 3)(1\ 2)(3\ 4)(3\ 2\ 1)](1) = 4
                \end{align*}
                Por tanto, $(1\ 2\ 3)(1\ 2)(3\ 4)(1\ 2\ 3)^{-1}\notin \langle (1\ 2)(3\ 4)\rangle$, luego $\langle (1\ 2)(3\ 4)\rangle \cancel{\lhd} A_4$, luego $\langle (1\ 2)(3\ 4)\rangle \neq [A_4,A_4]$.

                \item $\langle (1\ 3)(2\ 4)\rangle$.
                \begin{align*}
                    [(1\ 2\ 3)(1\ 3)(2\ 4)(1\ 2\ 3)^{-1}](3) &= [(1\ 2\ 3)(1\ 3)(2\ 4)(3\ 2\ 1)](3) = 4
                \end{align*}
                Por tanto, $(1\ 2\ 3)(1\ 3)(2\ 4)(1\ 2\ 3)^{-1}\notin \langle (1\ 3)(2\ 4)\rangle$, luego $\langle (1\ 3)(2\ 4)\rangle \cancel{\lhd} A_4$, luego $\langle (1\ 3)(2\ 4)\rangle \neq [A_4,A_4]$.

                \item $\langle (1\ 4)(2\ 3)\rangle$.
                \begin{align*}
                    [(1\ 2\ 3)(1\ 4)(2\ 3)(1\ 2\ 3)^{-1}](2) &= [(1\ 2\ 3)(1\ 4)(2\ 3)(3\ 2\ 1)](2) = 4
                \end{align*}
                Por tanto, $(1\ 2\ 3)(1\ 4)(2\ 3)(1\ 2\ 3)^{-1}\notin \langle (1\ 4)(2\ 3)\rangle$, luego $\langle (1\ 4)(2\ 3)\rangle \cancel{\lhd} A_4$, luego $\langle (1\ 4)(2\ 3)\rangle \neq [A_4,A_4]$.

                \item $\langle (1\ 2\ 3)\rangle$.
                \begin{align*}
                    [(1\ 2)(3\ 4)(1\ 2\ 3)(1\ 2)^{-1}(3\ 4)^{-1}](1) &= [(1\ 2)(3\ 4)(1\ 2\ 3)(1\ 2)(3\ 4)](1) = 4
                \end{align*}
                Por tanto, $(1\ 2)(3\ 4)(1\ 2\ 3)(1\ 2)(3\ 4)\notin \langle (1\ 2\ 3)\rangle$, luego $\langle (1\ 2\ 3)\rangle \cancel{\lhd} A_4$, luego $\langle (1\ 2\ 3)\rangle \neq [A_4,A_4]$.

                \item $\langle (2\ 3\ 4)\rangle$.
                \begin{align*}
                    [(1\ 2)(3\ 4)(2\ 3\ 4)(1\ 2)^{-1}(3\ 4)^{-1}](1) &= [(1\ 2)(3\ 4)(2\ 3\ 4)(1\ 2)(3\ 4)](1) = 4
                \end{align*}
                Por tanto, $(1\ 2)(3\ 4)(2\ 3\ 4)(1\ 2)(3\ 4)\notin \langle (2\ 3\ 4)\rangle$, luego $\langle (2\ 3\ 4)\rangle \cancel{\lhd} A_4$, luego $\langle (2\ 3\ 4)\rangle \neq [A_4,A_4]$.

                \item $\langle (1\ 3\ 4)\rangle$.
                \begin{align*}
                    [(1\ 2)(3\ 4)(1\ 3\ 4)(1\ 2)^{-1}(3\ 4)^{-1}](2) &= [(1\ 2)(3\ 4)(1\ 3\ 4)(1\ 2)(3\ 4)](2) = 4
                \end{align*}
                Por tanto, $(1\ 2)(3\ 4)(1\ 3\ 4)(1\ 2)(3\ 4)\notin \langle (1\ 3\ 4)\rangle$, luego $\langle (1\ 3\ 4)\rangle \cancel{\lhd} A_4$, luego $\langle (1\ 3\ 4)\rangle \neq [A_4,A_4]$.

                \item $\langle (1\ 2\ 4)\rangle$.
                \begin{align*}
                    [(1\ 3)(2\ 4)(1\ 2\ 4)(1\ 3)^{-1}(2\ 4)^{-1}](3) &= [(1\ 3)(2\ 4)(1\ 2\ 4)(1\ 3)(2\ 4)](3) = 4
                \end{align*}
                Por tanto, $(1\ 3)(2\ 4)(1\ 2\ 4)(1\ 3)(2\ 4)\notin \langle (1\ 2\ 4)\rangle$, luego $\langle (1\ 2\ 4)\rangle \cancel{\lhd} A_4$, luego $\langle (1\ 2\ 4)\rangle \neq [A_4,A_4]$.

                \item $V$.
                
                En Teoría vimos que el subgrupo de Klein $V$ es normal en $A_4$. Veamos que $A_4/V$ es abeliano:
                \begin{align*}
                    \left|\dfrac{A_4}{V}\right| &= \frac{|A_4|}{|V|} = \frac{12}{4} = 3
                    \Longrightarrow \dfrac{A_4}{V} \cong C_3
                \end{align*}
                Por tanto, $A_4/V$ es abeliano, luego:
                \begin{equation*}
                    [A_4,A_4] = V
                \end{equation*}               
            \end{itemize}

            \item $D_4$:
            
            El diagrama de Hasse de $D_4$ conviene tenerlo presente, y se encuentra en la Figura~\ref{fig:ej11_D4}. Comprobemos cada subgrupo de $D_4$, de menor a mayor orden:
            \begin{itemize}
                \item $\{1\}$ efectivamente cumple que $\{1\}\lhd D_4$, pero:
                \begin{align*}
                    D_4/\{1\} &\cong D_4
                \end{align*}
                Por tanto, el cociente no es abeliano, luego $\{1\}\neq [D_4,D_4]$.

                \item $\langle r^2\rangle$.
                
                En el Ejercicio~\ref{ej:4.5} vimos que:
                \begin{equation*}
                    Z(D_4) = \langle r^2\rangle
                \end{equation*}

                Como además $Z(D_4)\lhd D_4$, tenemos que $\langle r^2\rangle \lhd D_4$. Veamos que $D_4/\langle r^2\rangle$ es abeliano:
                \begin{align*}
                    \left|\dfrac{D_4}{\langle r^2\rangle}\right| &= \frac{|D_4|}{|\langle r^2\rangle|} = \frac{8}{2} = 4
                \end{align*}
                Por tanto, es isomorfo a $C_4$ o a $V$, ambos abelianos. Por tanto, $D_4/\langle r^2\rangle$ es abeliano, luego:
                \begin{equation*}
                    [D_4,D_4] = \langle r^2\rangle
                \end{equation*}
            \end{itemize}

            \item $Q_2$:
            
            El diagrama de Hasse de $Q_2$ conviene tenerlo presente, y se encuentra en la Figura~\ref{fig:ej11_Q2}. Comprobemos cada subgrupo de $Q_2$, de menor a mayor orden:
            \begin{itemize}
                \item $\{1\}$ efectivamente cumple que $\{1\}\lhd Q_2$, pero:
                \begin{align*}
                    Q_2/\{1\} &\cong Q_2
                \end{align*}
                Por tanto, el cociente no es abeliano, luego $\{1\}\neq [Q_2,Q_2]$.

                \item $\langle -1\rangle=\{1,-1\}$.
                
                Sea $g\in Q_2$. Entonces:
                \begin{equation*}
                    g(-1)g^{-1} = -(g g^{-1}) = -1\in \langle -1\rangle
                \end{equation*}
                Por tanto, $\langle -1\rangle \lhd Q_2$. Veamos que $Q_2/\langle -1\rangle$ es abeliano:
                \begin{align*}
                    \left|\dfrac{Q_2}{\langle -1\rangle}\right| &= \frac{|Q_2|}{|\langle -1\rangle|} = \frac{8}{2} = 4
                \end{align*}
                Por tanto, es isomorfo a $C_4$ o a $V$, ambos abelianos. Por tanto, $Q_2/\langle -1\rangle$ es abeliano, luego:
                \begin{equation*}
                    [Q_2,Q_2] = \langle -1\rangle
                \end{equation*}
            \end{itemize}
        \end{enumerate}
        \item Demostrar que, para $n\geq 3$, el subgrupo conmutador de $S_n$ es $A_n$ y que éste es el único subgrupo de $S_n$ de orden $\nicefrac{n!}{2}$.\\
        
        Veamos que $[S_n,S_n]=A_n$.
        \begin{description}
            \item[$\subset)$] Tenemos que $[S_n : A_n] = 2$, luego $A_n\lhd S_n$. Por tanto, $[S_n,S_n]\subseteq A_n$.
            
            \item[$\supset)$] Sean $i,j,k\in \{1,\ldots,n\}$ tres naturales distintos. Entonces:
            \begin{equation*}
                (i\ j\ k) = (i\ j)(j\ k) = (i\ j)(i\ k)(i\ j)(i\ k)
                = (i\ j)(i\ k)(i\ j)^{-1}(i\ k)^{-1} = [(i\ j),(i\ k)]
            \end{equation*}

            Por tanto, como los $3-$ciclos son generadores de $A_n$, entonces $A_n\subseteq [S_n,S_n]$.
        \end{description}

        Por tanto, $[S_n,S_n]=A_n$. Supongamos ahora que $H$ es un subgrupo de $S_n$ tal que $|H|=\nicefrac{n!}{2}$. Entonces:
        \begin{equation*}
            [S_n : H] = \frac{|S_n|}{|H|} = \frac{n!}{\nicefrac{n!}{2}} = 2
            \Longrightarrow H\lhd S_n
        \end{equation*}

        Además, el cociente $S_n/H$ es un grupo de orden $2$, luego es cíclico, y por tanto abeliano. Por tanto, $A_n=[S_n,S_n]<H$ Y $|H|=|[S_n,S_n]|=|A_n|$, luego $H=A_n$. Por tanto, $A_n$ es el único subgrupo de $S_n$ de orden $\nicefrac{n!}{2}$.
    \end{enumerate}
\end{ejercicio}
