\chapter{Clasificación de grupos de orden bajo}
\noindent
Clasificar grupos es una tarea dura y difícil, por lo que nos centraremos en aprender a clasificar grupos de orden bajo. En concreto, nuestro objetivo será saber clasificar todos los grupos de orden menor o igual que 15.

\subsubsection{Grupos abelianos}
\noindent
En el Capítulo anterior aprendimos ya a clasificar todos los grupos abelianos finitos. En particular, sabemos ya clasificar todos los grupos abelianos finitos de orden menor o igual que 15:
\begin{equation*}
    \begin{array}{c|ccc}
        \text{Orden}  & \text{Grupos} & &  \\
        \hline
        1 & \{1\} &  &\\
        2 & C_2 & & \\
        3 & C_3 & & \\
        4 & C_4, &  C_2\oplus C_2& \\
        5 & C_5 & & \\
        6 & C_6 & & \\
        7 & C_7 & & \\
        8 & C_8, &  C_2\oplus C_4,  & C_2\oplus C_2 \oplus C_2\\
        9 & C_9, &  C_3\oplus C_3 & \\
        10 & C_{10} & & \\
        11 & C_{11} & & \\
        12 & C_{12}, &  C_2\oplus C_6 & \\
        13 & C_{13} & & \\
        14 & C_{14} & & \\
        15 & C_{15} & & \\
        \vdots & \vdots 
    \end{array}
\end{equation*}
\noindent
Por tanto, nos centraremos ahora en tratar de describir todos los grupos finitos no abelianos de orden menor o igual que 15.

\section{Producto semidirecto}
\noindent
Con el fin de conseguir nuestro objetivo, definiremos el producto semidirecto, herramienta que nos permitirá escribir muchos grupos no abelianos (aunque no todos).

\begin{ejemplo}
    En el Capítulo~\ref{cap:1} vimos que $Q_2 = \{\pm 1, \pm i, \pm j, \pm k\}$ es isomorfo a:
    \begin{equation*}
        Q_2^{\text{abs}} = \langle x,y\mid x^4=1, x^2=y^2, yxy^{-1} = x^{-1} \rangle 
    \end{equation*}
    Es decir, teníamos una aplicación (gracias a Teorema de Dyck) $f:Q_2\to Q_2^{\text{abs}}$ dada por:
    \begin{equation*}
        f(x) = i \qquad f(y) = j
    \end{equation*}
    Que además era un epimorfismo, porque $Q_2^{\text{abs}} = \langle i,j \rangle $. Veamos que $|Q_2^{\text{abs}}| = 8$, de una forma distinta que contar elementos:
    \begin{proof} % // TODO: Revisar esta demostración
        Como $x^4 = 1$, si consideramos $H = \langle x \rangle $, tendremos que $|H| \leq 4$. Ahora, como:
        \begin{equation*}
            yxy^{-1} = x^{-1}\in H 
        \end{equation*}
        Tenemos que $H\lhd Q_2^{\text{abs}}$. Si escribimos $Q_2^{\text{abs}}$ en su partición de clases (como $Q_2^{\text{abs}} = \langle x,y \rangle $, si un elemento no está en $H$ es porque es producto de $y$ por algo más):
        \begin{equation*}
            Q_2^{\text{abs}} \cong H\cup yH
        \end{equation*}
        Ya que $y\notin H$, de donde al tomar cocientes:
        \begin{equation*}
            Q_2^{\text{abs}}/H = \langle yH \rangle 
        \end{equation*}
        Ahora, como:
        \begin{equation*}
            {(yH)}^{2} = y^2H = x^2H = H
        \end{equation*}
        Entonces $O(yH) = 2$, por lo que:
        \begin{equation*}
            |Q_2^{\text{abs}}/H| = 2
        \end{equation*}
        Si aplicamos el Primer Teorema de Isomorfía sobre $f$:
        \begin{equation*}
            Q_2^{\text{abs}}/\ker(f) \cong Im(f) = Q_2
        \end{equation*}
        De donde $|Q_2^{\text{abs}}| = |Q_2||\ker(f)| \geq 8$. Concluimos que $|Q_2^{\text{abs}}| = 8$.
    \end{proof}
\end{ejemplo}

\noindent
El hecho de introducir $Q_2^{\text{abs}}$ en el Capítulo~\ref{cap:1} fue para ahora generalizar lo que hacíamos con $Q_2$ a todo grupo, con el producto semidirecto.

\begin{definicion}[Grupos dicíclicos]
    Para cada $k\in \mathbb{N}\setminus \{0\}$, definimos el $k-$ésimo grupo dicíclico como el grupo:
    \begin{equation*}
        Q_k = \left\langle x,y \mid x^{2k} = 1, y^2 = x^k, yxy^{-1} = x^{-1}  \right\rangle 
    \end{equation*}
\end{definicion}

\begin{ejemplo}
    Estudiemos los grupos dicíclicos:
    \begin{itemize}
        \item Para $k=1$:
            \begin{equation*}
                Q_1 = \langle x,y\mid x^2 = 1, y^2 = x, yxy^{-1} = x \rangle 
            \end{equation*}
            Nos preguntamos qué grupo es. Si tratamos de describir los elementos, obtenemos:
            \begin{equation*}
                \{1,x,y,xy\} = \{1,y,y^2,y^3\}
            \end{equation*}
            Es decir, $Q_1\cong C_4$.
        \item Observemos que si\footnote{Aquí cometemos un pequeño abuso de notación, ya que $Q_2$ no es el grupo de los cuaternios, sino el segundo grupo dicíclico.} $k=2$, obtenemos $Q_2^\text{abs}$:
            \begin{equation*}
                Q_2 = \langle x,y \mid x^4 = 1, y^2 = x^2, yx = x^{-1}y \rangle  = Q_2^{\text{abs}}
            \end{equation*}
        \item Para $k\geq 3$, veamos que:
            \begin{equation*}
                2k \leq |Q_k| \leq 4k \qquad \forall k\geq 3
            \end{equation*}
            Y si $k$ es impar, entonces $|Q_k| = 4k$.
            \begin{proof}
                Si recordamos al grupo diédrico de orden $k$:
                \begin{equation*}
                    D_k = \langle r,s \mid r^k = s^2 = 1, sr = r^{-1}s \rangle 
                \end{equation*}
                Recordamos que:
                \begin{equation*}
                    Q_k = \langle x,y \mid x^{2k} = 1, y^2 = x^k, yx=x^{-1}y \rangle 
                \end{equation*}
                Y observamos que:
                \begin{itemize}
                    \item $r^{2k} = {(r^k)}^{2} = 1$.
                    \item $s^2 = 1 = r^k$.
                    \item $sr = r^{-1}s$.
                \end{itemize}
                El Teorema de Dyck nos da un homomorfismo $f:Q_k\to D_k$ de forma que:
                \begin{equation*}
                    f(x) = r \qquad f(y) = s
                \end{equation*}
                Como además $D_k = \langle r,s \rangle $, tenemos que $f$ es un epimorfismo. Si aplicamos el Primer Teorema de Isomorfía sobre $f$, obtenemos que:
                \begin{equation*}
                    Q_k / \ker(f) \cong D_k
                \end{equation*}
                En primer lugar, observamos que $Q_k$ no es abeliano, ya que $D_k$ no lo es y cualquier cociente de un grupo abeliano es abeliano. Veamos que: 
                \begin{equation*}
                    2k \leq |Q_k| \leq 4k
                \end{equation*}
                \begin{itemize}
                    \item Para la desigualdad de la izquierda, sabemos que $|D_k| = 2k$, y por ser $[Q_k : \ker(f)] = |D_k|$, sabemos que $2k$ divide a $|Q_k|$, de donde $|Q_k| \geq 2k$.
                    \item Para la otra, si tomamos $H = \langle x \rangle $, como $x^{2k} = 1$, tendremos que $\red{|H| \leq 2k}$. Como además también tenemos que:
                        \begin{align*}
                            yxy^{-1} &= x^{-1} \in  H \\
                            y^{-1}xy &= x^{-1} \in H
                        \end{align*}
                        Tendremos que $H\lhd Q_k$, de donde al considerar el cociente, tendremos que:
                        \begin{equation*}
                            Q_k/H = \langle yH \rangle 
                        \end{equation*}
                        De esta forma, como:
                        \begin{equation*}
                            {(yH)}^{2} = y^2H = x^k H = H
                        \end{equation*}
                        Deducimos que $O(yH) \leq 2$, por lo que:
                        \begin{equation*}
                            \red{|Q_k/H| \leq 2}
                        \end{equation*}
                        De donde tenemos que:
                        \begin{equation*}
                            |Q_k| = |Q_k/H| |H| \leq 2k + 2k = 4k
                        \end{equation*}
                \end{itemize}

                Si suponemos ahora que $k = 2t + 1$ para cierto $t\in \mathbb{N}$, si consideramos el cíclico de orden 4:
                \begin{equation*}
                    C_4 = \langle a \mid a^4 = 1 \rangle 
                \end{equation*}
                Como:
                \begin{itemize}
                    \item ${(a^2)}^{2k} = {(a^4)}^{k} = 1$.
                    \item $a^2 = a^{4t+2} = a^{2k} = {(a^2)}^{k}$.
                    \item $aa^2 = a^3 = a^2a = {(a^2)}^{-1}a$.
                \end{itemize}
                Podemos aplicar el Teorema de Dyck, obteniendo un homomorfismo $f:Q_k\to C_4$ de forma que:
                \begin{equation*}
                    f(x) = a^2 \qquad f(y) = a
                \end{equation*}
                Que de hecho es un epimorfismo, ya que $C_4 = \langle a \rangle $. Al igual que antes, tendremos que:
                \begin{equation*}
                    Q_k/\ker(f)\cong C_4
                \end{equation*}
                De donde 4 divide a $|Q_k|$. Como también teníamos antes que $2k$ dividía a $|Q_k|$, tenemos que $\mcm(2k,4)$ divide a $|Q_k|$ y como $k$ era impar, tenemos que:
                \begin{equation*}
                    \mcm(2k,4) = 4k
                \end{equation*}
                De donde $4k$ divide a $|Q_k|$, por lo que $4k\leq |Q_k|$, y la otra desigualdad que teníamos ya probada nos da la igualdad.
            \end{proof}
    \end{itemize}
\end{ejemplo}

\begin{ejemplo}
    Grupos no abelianos de orden 12 conocíamos:
    \begin{itemize}
        \item $A_4$.
        \item $D_6$.
    \end{itemize}
    Y ahora conocemos $Q_3$. Próximamente veremos que estos grupos son los únicos grupos que existen de orden 12, salvo isomorfismo.
\end{ejemplo}

\noindent
Antes de proceder con la definición del producto semidirecto, daremos una útil Proposición que nos será de ayuda a la hora de construir productos semidirectos:
\begin{prop}\label{prop:aut_ciclico}
    Sea $p$ un primo, entonces:
    \begin{equation*}
        Aut(C_p) \cong C_{p-1}
    \end{equation*} % // TODO:
\end{prop}

\begin{definicion}[Producto semidirecto]
    Sean $K$, $H$ dos grupos y dado un homomorfismo $\theta:H\to Aut(K)$, sobre el producto cartesiano de $K$ por $H$ podemos definir la operación:
    \begin{equation*}
        (k_1,h_1)(k_2,h_2) = (k_1\theta(h_1)(k_2), h_1h_2) \qquad \forall (k_1,h_1),(k_2,h_2) \in K\times H
    \end{equation*}
    Se verifia que $K\times H$ con esta operación tiene estructura de grupo (como ponemos de manifiesto en la siguiente Proposición), al que llamaremos \textbf{producto semidirecto de $K$ por $H$ relativo a $\theta$}, y que denotaremos por:
    \begin{equation*}
        K\rtimes_\theta H
    \end{equation*}
    Observemos que, en particular, $\theta$ es un homomorfismo de $H$ sobre $\Perm(K)$, por lo que $\theta$ nos define una acción:
    \begin{equation*}
        \theta(h)(k) = \prescript{h}{}{k} \qquad \forall h\in H, k\in K
    \end{equation*}
    Por lo que será habitual escribir:
    \begin{equation*}
        (k_1,h_1)(k_2,h_2) = (k_1 \prescript{h_1}{}{k_2}, h_1h_2)
    \end{equation*}
    Y no se nos debe olvidar que $\theta(h) \in Aut(K)$ para todo $h\in H$, ya que esta propeidad es importante a la hora de ver que $K\rtimes_\theta H$ es un grupo.
\end{definicion}

\begin{prop}
    Se verifica que $K\times H$ con la operación definida en la definición anterior es un grupo.
    \begin{proof}
        Veamos que efectivamente cumple con todas las conddiciones de ser un grupo:
        \begin{itemize}
            \item Para la propiedad asociativa, si $a,b,c\in K$ y $x,y,z\in H$:
                \begin{align*}
                    ((a,x)(b,y))(c,z) &= (a\prescript{x}{}{b},xy)(c,z) = (a\prescript{x}{}{b}\prescript{xy}{}{c}, xyz) \\
                    (a,x)((b,y)(c,z)) &= (a,x)(b\prescript{y}{}{c}, yz) = (a\prescript{x}{}{(b\prescript{y}{}{c})}, xyz) = (a\prescript{x}{}{b}\prescript{xy}{}{c},xyz)
                \end{align*}
            \item El elemento $(1,1)$ es el neutro:
                \begin{align*}
                    (k,h)(1,1) &= (k\prescript{h}{}{1},h) = (k,h) \\
                    (1,1)(k,h) &= (1 \prescript{1}{}{k},h) = (k,h) \\
                               &\forall (k,h)\in K\times H
                \end{align*}
            \item Para el inverso, dado $(k,h)\in K\times H$, el inverso será:
                \begin{equation*}
                    {(k,h)}^{-1} = \left(\prescript{h^{-1}}{}{k^{-1}},h^{-1}\right)
                \end{equation*}
                Ya que:
                \begin{equation*}
                    (k,h)\left(\prescript{h^{-1}}{}{k^{-1}},h^{-1}\right) = \left(k\prescript{h}{}{\left(\prescript{h^{-1}}{}{k^{-1}}\right)}, hh^{-1}\right) = \left(k\prescript{hh^{-1}}{}{k^{-1}}, 1\right) = (kk^{-1},1) = (1,1)
                \end{equation*}
        \end{itemize}
    \end{proof}
\end{prop}

\begin{ejemplo}
    Veamos:
    \begin{itemize}
        \item Si $\theta(k) = id_H$ para todo $k\in K$, entonces $K\rtimes_\theta H = K\times H$, ya que:
            \begin{equation*}
                (k_1,h_1)(k_2,h_2) = (k_1 \prescript{h_1}{}{k_2}, h_1h_2) = (k_1k_2, h_1h_1) \qquad \forall (k_1,h_1), (k_2,h_2) \in K\times H
            \end{equation*}
        \item Como ejemplo útil del producto semidirecto, veamos si somos capaces de escribir $S_3$ como producto semidirecto de $C_3$ por $C_2$ relativo a algún homomorfismo $\theta$:
            \begin{equation*}
                S_3\cong C_3\rtimes_{\theta} C_2
            \end{equation*}
        \item Veamos cómo escribir $S_3$ como producto semidirecto:
            \begin{equation*}
                S_3 \cong C_3 \rtimes_{\theta} C_2
            \end{equation*}
            Tenemos los elementos:
            \begin{equation*}
                C_3\times C_2 = \{(x,y) \mid x\in C_3, y\in C_2\}
            \end{equation*}
            Buscamos qué homomorfismo $\theta:C_2\to Aut(C_3)$ hemos de coger. Como $\theta$ ha de ser un homomorfismo, sabemos que $\theta(0) = id_{C_3}$, por lo que solo hemos de determinar la imagen de $1\in C_2$. Además, como $Aut(C_3)\cong C_2$, solo hay dos isomorfismos de $C_3$ en $C_3$:
            \begin{equation*}
                \begin{array}{c}
                    id_{C_3} \\
                    0 \longmapsto 0 \\
                    1 \longmapsto 1 \\
                    2 \longmapsto 2
                \end{array} \qquad 
                \begin{array}{c}
                    x^{-1} \\
                    0 \longmapsto 0 \\
                    1 \longmapsto 2 \\
                    2 \longmapsto 1
                \end{array} 
            \end{equation*}
            Si cogemos $\theta(1) = id_{C_3}$, en el ejemplo superior vimos que entonces tendremos que $C_3\rtimes_\theta C_2 = C_3\times C_2$, y sabemos que $C_3\times C_2 \cong C_6$. Sin embargo, $S_3$ no es cíclico, luego no vamos a tener que $S_3 \cong C_3\rtimes_\theta C_2$. Nos decantamos por tanto a coger $\theta(1)(x) = x^{-1}$ $\forall x\in C_3$.\\

            De esta forma, tendremos que $|C_3\rtimes_\theta C_2| = 6$ y como podemos clasificar todos los grupos de orden 6 en $S_3$ (no abeliano) y en $C_6$ (abeliano), si vemos que $C_3\rtimes_\theta C_2$ no es abeliano, tendremos que $C_3\rtimes_\theta C_2 \cong S_3$, que es lo que queríamos probar. Para ello (usamos notación aditiva por representar $C_3$ y $C_2$ con notación aditiva):
            \begin{align*}
                (1, 0) + (0, 1) &= (1 + \theta(0)(0), 0+1) = (1, 1) \\
                (0, 1) + (1, 0) &= (0 + \theta(1)(1), 1+0) = (2, 1)
            \end{align*}
            Como $(1,1) \neq (2,1)$, tenemos que $C_3\rtimes_\theta C_2$ no es abeliano, por lo que ha de ser $C_3\rtimes_\theta C_2 \cong S_3$.
        % \item Ahora, veamos que $Q_3 \cong C_3\rtimes_\theta C_4$, usando un homomorfismo muy similar al anterior, $\theta:C_4\to Aut(C_3)$ dado por:
        %     \begin{equation*}
        %         \theta(y)(x) = x^{-1} \qquad \forall y\in C_4, \forall x\in C_3
        %     \end{equation*} % // TODO: Arreglar ejemplo

% // TODO: Arreglar este ejemplo, no sé ni qué dice ni como y hay que estructurarlo



        \item Veamos que $Q_3 = C_3\rtimes_\theta C_4$. De nuevo, el homomorfismo a considerar será:
            \Func{\theta}{C_4}{Aut(C_3)}{y}{\theta(y)(x)=x^{-1}}
            Tendremos:
            \begin{equation*}
                C_3\rtimes_\theta C_4 = \langle x,y\mid x^3=1, y^4 = 1, \prescript{y}{}{x}= x^{-1} \rangle 
            \end{equation*}
            Y queremos ver el isomorfismo con:
            \begin{equation*}
                Q_3 = \langle c,d\mid c^6=1, d^2 = c^3, dc=c^{-1}d \rangle 
            \end{equation*}
            Si cogemos:
            \begin{equation*}
                c = (x^2,y) \qquad d = (1,y)
            \end{equation*}
            Vemos que:
            \begin{itemize}
                \item $c^6 = 1$.
                \item $d^2 = c^3$.
                \item $dc = c^{-1}d$, que equivale a ver que $cdc = d$. Para ello:
                    \begin{equation*}
                        (x^2,y)(1,y)(x^2,y) = (x^2,y)\left(1\prescript{y}{}{x^2}, y^2\right) = (x^2,y)(x,y^2) = (x^2\prescript{y}{}{x},y^3) = (x,y^3)
                    \end{equation*}
                    \begin{equation*}
                        (x^2,1)(1,y)(x^2,1) = (x^2,y)(x^2,1) = (x^2\prescript{y}{}{x^2}, y) = (x^3,y) = (1,y)
                    \end{equation*}
            \end{itemize}
        \item Si $n\geq 3$, si consideramos $\theta:C_2\to Aut(C_n)$ dado por:
            \begin{equation*}
                \theta(y)(x) = x^{-1} \qquad \forall x\in C_2, y\in C_n
            \end{equation*}
            Tendremos que $C_n\rtimes_\theta C_2 \cong D_n$.
    \end{itemize}
\end{ejemplo}

\subsection{Propiedades}
\begin{definicion}
    Dado un homomorfismo $\theta:H\to Aut(K)$ que nos da un producto semidirecto $K\rtimes_\theta H$, sobre este producto definimos:
    \begin{itemize}
        \item La inyección en primera coordenada, $\lm_1:K\to K\rtimes_\theta H$ dada por:
            \begin{equation*}
                \lm_1(k) = (k,1) \qquad \forall k\in K
            \end{equation*}
        \item La inyección en segunda coordenada, $\lm_2:H\to K\rtimes_\theta H$ dada por:
            \begin{equation*}
                \lm_2(h) = (1,h) \qquad \forall h\in H
            \end{equation*}
        \item La proyección $\pi:K\rtimes_\theta H \to H$ dada por:
            \begin{equation*}
                \pi(k,h) = h \qquad \forall (k,h)\in K\rtimes_\theta H
            \end{equation*}
    \end{itemize}
    \begin{figure}[H]
        \centering
        \shorthandoff{""}
        \begin{tikzcd}
            K \arrow[r, "\lambda_1"] & K\rtimes_\theta H \arrow[d, "\pi"] & H \arrow[l, "\lambda_2"'] \\
                                     & H                           &                          
        \end{tikzcd}
        \shorthandon{""}
    \end{figure}
\end{definicion}

\begin{prop}
    Si $K\rtimes_\theta H$ es un producto semidirecto, se verifica que:
    \begin{enumerate}
        \item $\lm_1, \lm_2, \pi$ son homomorfismos de grupos.
        \item $\pi \lm_1$ es trivial.
        \item $\pi\lm_2 = id_H$.
        \item $(k,h) = (k,1)(1,h)$ $\forall (k,h) \in K\rtimes_\theta H$.
    \end{enumerate}
    \begin{proof}
        Veamos cada propiedad:
        \begin{enumerate}
            \item Vemos que:
                \begin{gather*}
                    \lm_1(k_1)\lm_1(k_2) = (k_1,1)(k_2,1) = (k_1 \prescript{1}{}{k_2},1) = (k_1k_2, 1) = \lm_1(k_1k_2) \qquad \forall k_1,k_2 \in K \\
                    \lm_2(h_1)\lm_2(h_2) = (1,h_1)(1,h_2) = (1\prescript{1}{}{1},h_1h_2) = (1,h_1h_2) = \lm_2(h_1h_2) \qquad \forall h_1,h_2 \in H 
                \end{gather*}
                Y para la proyección:
                \begin{multline*}
                    \pi(k_1,h_1)\pi(k_2,h_2) = h_1h_2 = \pi(k_1\prescript{h_1}{}{k_2},h_1h_2) = \pi((k_1,h_1)(k_2,h_2)) \\ \forall (k_1,h_1),(k_2,h_2) \in K\rtimes_\theta H
                \end{multline*}
            \item $(\pi\lm_1)(k) = \pi(k,1) = 1$ $\forall k\in K$.
            \item $(\pi\lm_2)(h) = \pi(k,h) = h$ $\forall h\in H$.
            \item $(k,1)(1,h) = (k\prescript{1}{}{1},h) = (k,h)$ $\forall (k,h)\in K\rtimes_\theta H$.
        \end{enumerate}
    \end{proof}
\end{prop}

\noindent
De forma análoga a la propiedad universal del producto directo, podemos tener la propiedad universal para el producto semidirecto.
\begin{teo}[Propiedad Universal del Producto Semidirecto]
    Sean $K, H$ dos grupos y $\theta:H\to Aut(K)$ un homomorfismo, consideramos $G = K\rtimes_\theta H$. Sea $T$ un grupo, para cada par de homomorfismos $f:K\to T$ y $g:H\to T$ que verifiquen:
    \begin{equation*}
        f(\theta(h)(k)) = g(h)f(k){(g(h))}^{-1} \qquad \forall k\in K, \forall h\in H
    \end{equation*}
    Entonces, existe un único homomorfismo $\varphi:G\to T$ de forma que:
    \begin{equation*}
        \varphi(k,1) = f(k) \qquad \varphi(1,h) = g(h) \qquad \forall k\in K, \forall h\in H
    \end{equation*}
    Y por tanto, $\varphi(k,h) = f(k)g(h)$ $\forall (k,h) \in G$.
    \begin{figure}[H]
        \centering
        \shorthandoff{""}
        \begin{tikzcd}
            K \arrow[rd, "f"'] & K\rtimes_\theta H \arrow[d, "\varphi", dashed] & H \arrow[ld, "g"] \\
                               & T                                              &                  
        \end{tikzcd}
        \shorthandon{""}
    \end{figure}
    \begin{proof}
        Definiendo $\varphi:G\to T$ por:
        \begin{equation*}
            \varphi(k,h) = f(k)g(h) \qquad \forall (k,h)\in G
        \end{equation*}
        Veamos que $\varphi$ es un homomorfismo:
        \begin{align*}
            \varphi((k_1,h_1)(k_2,h_2)) &= \varphi(k_1\theta(h_1)(k_2),h_1h_2) = f(k_1\theta(h_1)(k_2))g(h_1h_2) \\
                                        &= f(k_1)f(\theta(h_1)(k_2))g(h_1)g(h_2) = f(k_1)g(h_1)f(k_2){g(h_1)}^{-1}g(h_1)g(h_2) \\
                                        &= f(k_1) g(h_1)f(k_2) g(h_2) = \varphi(k_1,h_1)\varphi(k_2,h_2) \\
                                        & \forall (k_1,h_1),(k_2,h_2) \in K\rtimes_\theta H
        \end{align*}
        En particular, tendremos que:
        \begin{align*}
            \varphi(k,1) &= f(k)g(1) = f(k) \qquad \forall k\in K \\
            \varphi(1,h) &= f(1)g(h) = g(h) \qquad \forall h\in H
        \end{align*}
        Ahora, si $\phi:G\to T$ es otro homomorfismo que verifica que:
        \begin{equation*}
            \phi(k,1) = f(k) \qquad \phi(1,h) = g(h) \qquad \forall k\in K, \forall h\in H
        \end{equation*}
        Entonces:
        \begin{equation*}
            \phi(k,h) = \phi(k,1)\phi(1,h) = f(k)g(h) = \varphi(k,h) \qquad \forall (k,h)\in K\rtimes_\theta H
        \end{equation*}
        Por lo que $\phi = \varphi$.
    \end{proof}
\end{teo}

\noindent
La siguiente Proposición nos será de utilidad para clasificar grupos haciéndolos isomorfos a un producto semidirecto, a partir del orden.

\begin{teo}
    Sea $G$ un grupo y $K,H<G$ con $K\lhd G$ que verifican:
    \begin{itemize}
        \item $KH = G$.
        \item $K\cap H = \{1\}$.
    \end{itemize}
    Sea $\theta:H\to Aut(K)$ un homomorfismo que nos da la acción por conjugación\footnote{La condición $K\lhd G$ nos dice que $\theta$ está bien definida} $ac:H\times K\to K$:
    \begin{equation*}
        \theta(h)(k) = hkh^{-1} \qquad \forall (k,h)\in K\rtimes_\theta H
    \end{equation*}
    Entonces, $K\rtimes_\theta H \cong G$.
    \begin{proof}
        Definiremos la aplicación $f:K\rtimes_\theta H \to G$ dada por:
        \begin{equation*}
            f(k,h) = kh \qquad \forall k\in K, \forall h\in H
        \end{equation*}
        Veamos que es un isomorfismo:
        \begin{itemize}
            \item $f$ es sobreyectiva, ya que $G = KH$, de donde cualquier elemento $g\in G$ se escribe como $g = kh$, para cierto $(k,h)\in K\rtimes_\theta H$.
            \item Para la inyectividad, si $f(k_1,h_1) = f(k_2,h_2)$, entonces $k_1h_1 = k_2h_2$, de donde $k_2^{-1}k_1=h_2h_1^{-1}$:
                \begin{itemize}
                    \item $k_2^{-1}k_1\in K$.
                    \item $h_2h_1^{-1}\in H$.
                \end{itemize}
                Y como $H\cap K = \{1\}$, concluimos que $k_1 = k_2$ y $h_1 = h_2$, de donde $f$ es inyectiva.
            \item Para ver que $f$ es un homomorfismo, si $(k_1,h_1),(k_2,h_2)\in K\rtimes_\theta H$:
                \begin{multline*}
                    f((k_1,h_1)(k_2,h_2)) = f(k_1\prescript{h_1}{}{k_2},h_1h_2) = f(k_1h_1k_2h_1^{-1},h_1h_2) \\ = k_1h_1k_2h_1^{-1}h_1h_2 = k_1h_1k_2h_2 = f(k_1,h_1)f(k_2,h_2)
                \end{multline*}
        \end{itemize}
    \end{proof}
\end{teo}

\begin{definicion}\ 
    \begin{itemize}
        \item Si $G$ verifica las condiciones del Teorema anterior, decimos que $G$ es \underline{producto} \underline{semidirecto interno} de $K$ y $H$.
        \item Si $K < G$, un subgrupo $H<G$ se llama \underline{complemento para $K$ en $G$} si $G~=~KH$ con $K\cap H = \{1\}$.
    \end{itemize}
    De esta forma, tendremos que $G$ será un producto semidirecto interno de dos subgrupos propios suyos si y solo si algún subgrupo propio normal tiene un complemento.
\end{definicion}

\begin{ejemplo}
    Mostraremos ahora que dado un grupo $G$, no siempre será posible encontrar dos subgrupos propios suyos de forma que $G$ sea producto semidirecto interno de ellos. Como vismo en la definición superior, este motivo puede ser por dos razones:
    \begin{itemize}
        \item Que $G$ no tenga subgrupos normales propios, como puede ser cualquier grupo simple.
        \item Que $G$ tenga subgrupos normales propios, pero que no seamos capaces de encontrar ningún complemento para algún subgrupo.

            Por ejemplo, si recordamos el Diagrama de Hasse de $Q_2$:
            \begin{figure}[H]
                \centering
                \begin{tikzpicture}[node distance=2cm]
                    \node (Q2) {$Q_2$};
                    \node (1) [below of=Q2, xshift=-2cm] {$\left\langle i\right\rangle$};
                    \node (2) [below of=Q2] {$\left\langle j\right\rangle$};
                    \node (3) [below of=Q2, xshift=2cm] {$\left\langle k\right\rangle$};
                    \node (4) [below of=2] {$\left\langle -1\right\rangle$};
                    \node (5) [below of=4] {$\{1\}$};

                    \draw (Q2) -- node[above left] {$2$} (1);
                    \draw (Q2) -- node[left] {$2$} (2);
                    \draw (Q2) -- node[above right] {$2$} (3);
                    \draw (1) -- node[below left] {$2$} (4);
                    \draw (2) -- node[left] {$2$} (4);
                    \draw (3) -- node[below right] {$2$} (4);
                    \draw (4) -- node[left] {$2$} (5);
                \end{tikzpicture}
                \caption{Diagrama de Hasse para los subgrupos del grupo de los cuaternios.}
            \end{figure}
            Vemos que todos los subgrupo de $Q_2$ son normales, pero sin embargo:
            \begin{equation*}
                \langle i \rangle \cap \langle j \rangle  \cap \langle k \rangle  \cap \langle -1 \rangle  = \{1, -1\}
            \end{equation*}
            Por lo que dado un subgrupo normal propio de $Q_2$ no vamos a poder encontrar un complemento para él en $Q_2$.
    \end{itemize}
\end{ejemplo}

\begin{ejemplo}
    Como ejemplos de aplicaciones inmediantas del último Teorema, tenemos (no especificaremos bajo qué homomorfismo se realiza el producto semidirecto, entendiendo que es la acción por conjugación):
    \begin{itemize}
        \item Sea $G = S_n$, $K = A_n \lhd S_n$ y $H = \langle (1\ 2) \rangle \cong \mathbb{Z}_2 $, entonces:
            \begin{equation*}
                A_n H = S_n \qquad A_n\cap H = \{1\}
            \end{equation*}
            Por lo que estamos en las condinciones de aplicar el Teorema, con lo que:
            \begin{equation*}
                S_n\cong A_n \rtimes \mathbb{Z}_2
            \end{equation*}
        \item En $G = S_4$, si tomamos $K = V\lhd S_4$, $H = S_3 = Stab_{S_4}(V)$, tenemos que:
            \begin{equation*}
                VH = S_4 \qquad V\cap H = \{1\}
            \end{equation*}
            Por lo que:
            \begin{equation*}
                S_4\cong V\rtimes S_3
            \end{equation*}
        \item Sea $G = A_4$, $K = V\lhd A_4$, $H = \langle (1\ 2\ 3) \rangle $, tenemos que:
            \begin{equation*}
                A_4\cong V\rtimes H
            \end{equation*}
    \end{itemize}
\end{ejemplo}

\begin{definicion}
    Sea $G$ un grupo, si tomamos $\theta:G\to Aut(G)$ dado por:
    \begin{equation*}
        \theta(\phi)(x) = \phi(x) \qquad \forall \phi \in Aut(G), \forall x\in G
    \end{equation*}
    Definimos el grupo holomorfo de $G$ por:
    \begin{equation*}
        Hol(G) = G\rtimes_\theta Aut(G)
    \end{equation*}
\end{definicion}

\section{Clasificación de grupos de orden menor o igual que 15}
\noindent
Sabemos ya los únicos grupos de orden menor o igual a 15 que hay salvo isomorfismos, los cuales vamos a listar a continuación en una tabla resumen:
\begin{itemize}
    \item De orden 1: $\{1\}$.
    \item De orden 2: $C_2$.
    \item De orden 3: $C_3$.
    \item De orden 4: $C_4, C_2\oplus C_2$.
    \item De orden 5: $C_5$.
    \item De orden 6: $C_6, D_3$.
    \item De orden 7: $C_7$.
    \item De orden 8: $C_8, C_4\oplus C_2, C_2\oplus C_2\oplus C_2, D_4, Q_2$.
    \item De orden 9: $C_9, C_3\oplus C_3$.
    \item De orden 10: $C_{10}, D_5$.
    \item De orden 11: $C_{11}$.
    \item De orden 12: $C_{12}, C_6\oplus C_2, D_6, Q_3, A_4$.
    \item De orden 13: $C_{13}$.
    \item De orden 14: $C_{14}, D_7$.
    \item De orden 15: $C_{15}$.
\end{itemize}
Ya que:
\begin{itemize}
    \item De orden 1 solo está el grupo trivial, como vimos en los primeros capítulos.
    \item Como 2 es primo, solo está el grupo cíclico.
    \item Como 3 es primo, solo está el grupo cíclico.
    \item Como 4 es un primo al cuadrado, por un corolario del Teorema de Burnside sabemos que tiene que ser abeliano, luego no hay más.
    \item Como 5 es primo, solo está el grupo cíclico.
    \item $6 = 2\cdot 3$, de la forma $pq$ con $p=2$, luego solo está el cíclico y $D_3$.
    \item Como 7 es primo, solo está el grupo cíclico.
    \item Para 8 hemos realizado un análisis especial, obteniendo que hay 5 grupos, 3 abelianos y 2 no abelianos.
    \item Como 9 es un primo al cuadrado, sabemos que tiene que ser un grupo abeliano.
    \item $10 = 2\cdot 5$, de la forma $pq$ con $p =2$, luego solo está el cíclico y $D_5$.
    \item Como 11 es primo, solo está el grupo cíclico.
    \item Para 12 hemos realizado un análisis especial, obteniendo que hay 5 grupos de orden 12, 2 abelianos y 3 no abelianos.
    \item Como 13 es primo, solo está el grupo cíclico.
    \item $14 = 2\cdot 7$, de la forma $pq$ con $p=2$, luego solo está el cíclico y $D_7$.
    \item $15 = 3\cdot 5$, de la forma $pq$ y anteriormente discutimos este caso.
\end{itemize}
Si ha llegado hasta aquí y desea saber cómo se clasifican los grupos de orden $pq$, 12 y 8, le recomendamos que consulte \url{https://github.com/Joshoccas/3-DGIIM/tree/main/SegundoCuatrimestre/}.
