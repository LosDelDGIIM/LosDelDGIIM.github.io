\section{Grupos cociente. Teoremas de isomorfismo. Productos}

\begin{ejercicio}
    Demostrar que si $G\leq S_n$, entonces $G\subseteq A_n$ o bien se tiene que $[G:G\cap A_n]=2$. Concluir que un subgrupo de $S_n$ consiste sólo en permutaciones pares, o bien contiene el mismo número de permutaciones pares que de impares.
\end{ejercicio}

\begin{ejercicio}
    Sea $\bb{K}$ un cuerpo.
    \begin{enumerate}
        \item Se considera la siguiente aplicación:
        \Func{\det}{\GL_n(\bb{K})}{\bb{K}^\times}{G}{\det(G)}
        Demostrar que dicha aplicación es un epimorfismo de grupos. ¿Cuál es el núcleo de este homomorfismo?

        \item Si $\bb{K}$ es un cuerpo finito con $q$ elementos, determinar el orden del grupo $\SL_n(\bb{K})$.
    \end{enumerate}
\end{ejercicio}

\begin{ejercicio}
    Sea $n\in \bb{N}\setminus \{0\}$, y sea $G$ un grupo verificando que para todo par de elementos $x,y\in G$ se tiene que $(xy)^n=x^ny^n$. Se definen:
    \begin{align*}
        H &= \{x\in G\mid x^n=1\},\\
        K &= \{x^n\mid x\in G\}.
    \end{align*}
    Demostrar que $H,K\lhd G$, y que $|K|=[G:H]$.
\end{ejercicio}

\begin{ejercicio}
    Para un grupo $G$ se define su centro como
    \[
        Z(G) = \{a\in G\mid ax=xa\ \forall x\in G\}.
    \]
    \begin{enumerate}
        \item Demostrar que $Z(G)\leq G$.
        \item Demostrar que $Z(G)\lhd G$.
        \item Demostrar que $G$ es abeliano si, y sólo si, $G=Z(G)$.
        \item Demostrar que si $G/Z(G)$ es cíclico, entonces $G$ es abeliano.
    \end{enumerate}
\end{ejercicio}

\begin{ejercicio}
    Determinar el centro del grupo diédrico $D_4$. Observar que el cociente $D_4/Z(D_4)$ es abeliano, aunque $D_4$ no lo sea (compárese este hecho con el tercer apartado del ejercicio anterior).
\end{ejercicio}

\begin{ejercicio}
    Determinar el centro de los grupos $S_n$ y $A_n$ para $n\geq 2$.
\end{ejercicio}

\begin{ejercicio}
    Determinar el centro del grupo $D_n$ para $n\geq 3$.
\end{ejercicio}

\begin{ejercicio}
    Sean $H$ y $K$ dos subgrupos finitos de un grupo $G$, uno de ellos normal. Demostrar que
    \[
        |H||K| = |HK||H\cap K|.
    \]
\end{ejercicio}

\begin{ejercicio}
    Sea $G$ finito y $N\lhd G$. Probar que $G/N\cong G$ si, y sólo si, $N=\{1\}$, y que $G/N\cong \{1\}$ si, y sólo si, $N=G$.
\end{ejercicio}

\begin{ejercicio}
    Sean $G$ y $H$ dos grupos cuyos órdenes sean primos relativos. Probar que si $f:G\to H$ es un homomorfismo, entonces necesariamente $f(x)=1$ para todo $x\in G$, es decir, que el único homomorfismo entre ellos es el trivial.
\end{ejercicio}

\begin{ejercicio}
    Sean $H,K\leq G$, y sea $N\lhd G$ un subgrupo normal de $G$ tal que $HN=KN$. Demostrar que
    \[
        \frac{H}{H\cap N}\cong \frac{K}{K\cap N}.
    \]
\end{ejercicio}

\begin{ejercicio}
    Sea $N\lhd G$ tal que $N$ y $G/N$ son abelianos. Sea $H$ un subgrupo cualquiera de $G$. Demostrar que existe un subgrupo normal $K\lhd H$ tal que $K$ y $H/K$ son abelianos.
\end{ejercicio}

\begin{ejercicio}
    Sea $G$ un grupo finito, y sean $H,K\leq G$, con $K\lhd G$ y tales que $|H|$ y $[G:K]$ son primos relativos. Demostrar que $H\subseteq K$.
\end{ejercicio}

\begin{ejercicio}
    Sea $G$ un grupo.
    \begin{enumerate}
        \item Demostrar que para cada $a\in G$ la aplicación $\varphi_a:G\to G$ definida por $\varphi_a(x)=axa^{-1}$, es un automorfismo de $G$. $\varphi_a$ se llama automorfismo interno o de conjugación de $G$ definido por $a$.
        \item Demostrar que la siguiente aplicación es un homomorfismo:
        \Func{\varphi}{G}{\Aut(G)}{a}{\varphi_a}
        \item Demostrar que el conjunto de automorfismos internos de $G$, que se denota $\Int(G)$, es un subgrupo normal de $\Aut(G)$.
        \item Demostrar que $G/Z(G)\cong \Int(G)$.
        \item Demostrar que $\Int(G)=1$ si y sólo si $G$ es abeliano.
    \end{enumerate}
\end{ejercicio}

\begin{ejercicio}
    Demostrar que el grupo de automorfismos de un grupo no abeliano no puede ser cíclico.
\end{ejercicio}

\begin{ejercicio}
    Demostrar que $\Aut(\bb{Z}_2\times \bb{Z}_2)\cong S_3$.
\end{ejercicio}

\begin{ejercicio}
    Demostrar que los grupos $S_3$, $\bb{Z}_{p^n}$ (con $p$ primo) y $\bb{Z}$ no son producto directo internos de subgrupos propios.
\end{ejercicio}

\begin{ejercicio}
    En cada uno de los siguientes casos, decidir si el grupo $G$ es o no producto directo de los subgrupos $H$ y $K$.
    \begin{enumerate}
        \item $G=\bb{R}^\times$, $H=\{\pm 1\}$, $K=\{x\in \bb{R}\mid x>0\}$.
        \item $G=\left\{\begin{pmatrix} a & b \\ 0 & c \end{pmatrix}\in \GL_2(\bb{R})\right\}$, $H=\left\{\begin{pmatrix} a & 0 \\ 0 & c \end{pmatrix}\in \GL_2(\bb{R})\right\}$, $K=\left\{\begin{pmatrix} 1 & b \\ 0 & 1 \end{pmatrix}\in \GL_2(\bb{R})\right\}$.
        \item $G=\bb{C}^\times$, $H=\left\{z\in \bb{C}\mid |z|=1\right\}$, $K=\left\{x\in \bb{R}\mid x>0\right\}$.
    \end{enumerate}
\end{ejercicio}

\begin{ejercicio}
    Sean $G,H$ y $K$ grupos. Demostrar que:
    \begin{enumerate}
        \item $H\times K\cong K\times H$.
        \item $G\times (H\times K)\cong (G\times H)\times K$.
    \end{enumerate}
\end{ejercicio}

\begin{ejercicio}
    Dados isomorfismos de grupos $H\cong J$ y $K\cong L$, demostrar que $H\times K\cong J\times L$.
\end{ejercicio}

\begin{ejercicio}
    Sean $H,K,L$ y $M$ grupos tales que $H\times K\cong L\times M$. ¿Se verifica necesariamente que $H\cong L$ y $K\cong M$?
\end{ejercicio}

\begin{ejercicio}
    Demostrar que no todo subgrupo de un producto directo $H\times K$ es de la forma $H_1\times K_1$, con $H_1\leq H$ y $K_1\leq K$.
\end{ejercicio}

\begin{ejercicio}
    Sean $H,K$ dos grupos y sean $H_1\lhd H$, $K_1\lhd K$. Demostrar que $H_1\times K_1\lhd H\times K$ y que
    \[
        \frac{H\times K}{H_1\times K_1}\cong \frac{H}{H_1}\times \frac{K}{K_1}.
    \]
\end{ejercicio}

\begin{ejercicio}
    Sean $H,K\lhd G$ tales que $H\cap K=\{1\}$. Demostrar que $G$ es isomorfo a un subgrupo de $G/H\times G/K$.
\end{ejercicio}

\begin{ejercicio}
    Sean $H,K\lhd G$ tales que $HK=G$. Demostrar que
    \[
        \frac{G}{H\cap K}\cong \frac{H}{H\cap K}\times \frac{K}{H\cap K}\cong \frac{G}{H}\times \frac{G}{K}.
    \]
\end{ejercicio}

\begin{ejercicio}
    Demostrar que si $G$ es un grupo que es producto directo interno de subgrupos $H$ y $K$, y $N\lhd G$ tal que $N\cap H=\{1\}=N\cap K$, entonces $N$ es abeliano.
\end{ejercicio}

\begin{ejercicio}
    Dar un ejemplo de un grupo $G$ que sea producto directo interno de dos subgrupos propios $H$ y $K$, y que contenga a un subgrupo normal no trivial $N$ tal que $N\cap H=\{1\}=N\cap K$. Concluir que para $N\lhd H\times K$ es posible que se tenga
    \[
        N\neq (N\cap (H\times \{1\}))\times (N\cap (\{1\}\times K)).
    \]
\end{ejercicio}

\begin{ejercicio}
    Sea $G$ un grupo finito que sea producto directo interno de dos subgrupos $H$ y $K$ tales que $\mcd(|H|,|K|)=1$. Demostrar que para todo subgrupo $N\leq G$ se verifica que $N=(N\cap H)\times (N\cap K)$.
\end{ejercicio}

\begin{ejercicio}
    Sea $G$ un grupo y sea $f:G\to G$ un endomorfismo idempotente (esto es, verificando que $f^2=f$) y tal que $Im(f)\lhd G$. Demostrar que
    \[
        G\cong Im(f)\times \ker(f).
    \]
\end{ejercicio}

\begin{ejercicio}
    Sea $S$ un subconjunto de un grupo $G$. Se llama \emph{centralizador} de $S$ en $G$ al conjunto
    \[
        C_G(S) = \{x\in G\mid xs=sx\ \forall s\in S\}
    \]
    y se llama \emph{normalizador} de $S$ en $G$ al conjunto
    \[
        N_G(S) = \{x\in G\mid xS=Sx\}.
    \]
    \begin{enumerate}
        \item Demostrar que $N_G(S)\leq G$.
        \item Demostrar que $C_G(S)\lhd N_G(S)$.
        \item Demostrar que si $S\leq G$ entonces $S\lhd N_G(S)$.
    \end{enumerate}
\end{ejercicio}

\begin{ejercicio}
    Sea $G$ un grupo y $H$ y $K$ subgrupos suyos con $H\subseteq K$. Entonces demostrar que $H$ es normal en $K$ si y sólo si $K< N_G(H)$. (Así, el normalizador $N_G(H)$ queda caracterizado como el mayor subgrupo de $G$ en el que $H$ es normal.)
\end{ejercicio}

\begin{ejercicio} Sea $G$ un grupo.
    \begin{enumerate}
        \item Demostrar que $C_G(Z(G))=G$ y que $N_G(Z(G))=G$.
        \item Si $G$ es un grupo y $H<G$ ¿Cuándo es $G=N_G(H)$? ¿Y cuándo es $G=C_G(H)$?
        \item Si $H\leq G$ con $|H|=2$, demostrar que $N_G(H)=C_G(H)$. Deducir que $H\lhd G$ si y sólo si $H\subset Z(G)$.
    \end{enumerate}
\end{ejercicio}

\begin{ejercicio}
    Sea $G$ un grupo arbitrario. Para dos elementos $x,y\in G$ se define su \emph{conmutador} como el elemento
    \[
        [x,y] = xyx^{-1}y^{-1}.
    \]
    \begin{observacion}
        (El conmutador recibe tal nombre porque $[x,y]yx=xy$.)
    \end{observacion}
    
    Como $[x,y]^{-1}=[y,x]$, el inverso de un conmutador es un conmutador. Sin embargo el producto de dos conmutadores no tiene porqué ser un conmutador. Entonces se define el \emph{subgrupo conmutador} o (primer) \emph{subgrupo derivado} de $G$, denotado $[G,G]$, como el subgrupo generado por todos los conmutadores de $G$.
    \begin{enumerate}
        \item Demostrar que, $\forall a,x,y\in G$, se tiene que $a[x,y]a^{-1}=[axa^{-1},aya^{-1}]$.
        \item Demostrar que $[G,G]\lhd G$.
        \item demostrar que el grupo cociente $G/[G,G]$, que se representa por $G^{ab}$, es un grupo abeliano (que se llama el abelianizado de $G$).
        \item Demostrar que $G$ es abeliano si y sólo si $[G,G]=1$.
        \item Sea $N\lhd G$. Demostrar que el grupo cociente $G/N$ es abeliano si y sólo si $N>[G,G]$ (así que el grupo $[G,G]$ es el menor subgrupo normal de $G$ tal que el cociente es abeliano).
    \end{enumerate}
\end{ejercicio}

\begin{ejercicio}~
    \begin{enumerate}
        \item Calcular el subgrupo conmutador de los grupos $S_3$, $A_4$, $D_4$ y $Q_2$.
        \item Demostrar que, para $n\geq 3$, el subgrupo conmutador de $S_n$ es $A_n$ y que éste es el único subgrupo de $S_n$ de orden $\nicefrac{n!}{2}$.
    \end{enumerate}
\end{ejercicio}



\begin{comment}
Ejercicio 1. Demostrar que si G ≤ Sn, entonces G ⊆ An o bien se tiene
que [G : G ∩ An] = 2. Concluir que un subgrupo de Sn consiste s´olo en
permutaciones pares, o bien contiene el mismo n´umero de permutaciones
pares que de impares.
Ejercicio 2. Dado un cuerpo K, el grupo lineal especial de orden n sobre
K, SLn(K), (tambi´en llamado el grupo unimodular de orden n sobre K) es
SLn(K) = {G ∈ GLn(K)| det(G) = 1}
1. Se considera la aplicaci´on det : GLn(K) → F
× que aplica cada matriz
en su determinante. Demostrar que dicha aplicaci´on es un epimorfismo
de grupos. ¿Cu´al es el n´ucleo de este homomorfismo?
2. Si K es un cuerpo finito con q elementos, determinar el orden del grupo
SLn(K).
Ejercicio 3. Sea n > 1 un n´umero natural, y sea G un grupo verificando
que para todo par de elementos x, y ∈ G se tiene que (xy)
n = x
ny
n
. Se
definen H = {x ∈ G| x
n = 1}, y K = {x
n
| x ∈ G}. Demostrar que H y K
son subgrupos normales de G, y que |K| = [G : H].
Ejercicio 4. Para un grupo G se define su centro como
Z(G) = {a ∈ G| ∀ x ∈ G xa = ax}.
1. Demostrar que Z(G) es un subgrupo de G.
2. Demostrar que Z(G) es normal en G.
3. Demostrar que G es abeliano si, y s´olo si, G = Z(G).
4. Demostrar que si G/Z(G) es c´ıclico, entonces G es abeliano.
Ejercicio 5. Determinar el centro del grupo di´edrico D4. Observar que el
cociente D4/Z(D4) es abeliano, aunque D4 no lo sea (comp´arese este hecho
con el tercer apartado del ejercicio anterior).
Ejercicio 6. Determinar el centro de los grupos Sn y An para n ≥ 2.
Ejercicio 7. Determinar el centro del grupo Dn para n ≥ 3-
Ejercicio 8. Sean H y K dos subgrupos finitos de un grupo G, uno de ellos
normal. Demostrar que
|H||K| = |HK||H ∩ K|.
Ejercicio 9. Sea G finito y N ⊴ G. Probar que G/N ∼= G si, y s´olo si,
N = {1}, y que G/N ∼= {1} si, y s´olo si, N = G.
Ejercicio 10. Sean G y H dos grupos cuyos ´ordenes sean primos relativos.
Probar que si f : G → H es un homomorfismo, entonces necesariamente
f(x) = 1 para todo x ∈ G, es decir, que el ´unico homomorfismo entre ellos
es el trivial.
Ejercicio 11. Sean H y K subgrupos de G, y sea N ⊴ G un subgrupo
normal de G tal que HN = KN. Demostrar que
H
H ∩ N
∼=
K
K ∩ N
.
Ejercicio 12. Sea N un subgrupo normal de G tal que N y G/N son
abelianos. Sea H un subgrupo cualquiera de G. Demostrar que existe un
subgrupo normal K ⊴ H tal que K y H/K son abelianos.
Ejercicio 13. Sea G un grupo finito, y sean H, K subgrupos de G, con K
normal y tales que |H| y [G : K] son primos relativos. Demostrar que H
est´a contenido en K.
Ejercicio 14. Sea G un grupo
1. Demostrar que para cada a ∈ G la aplicaci´on φa : G → G definida por
φa(x) = axa−1
, es un automorfismo de G. φa se llama automorfismo
interno o de conjugaci´on de G definido por a.
2. Demostrar que la aplicaci´on G → Aut(G), a 7→ φa es un homomorfismo.
3. Demostrar que el conjunto de automorfismos internos de G, que se
denota Int(G), es un subgrupo normal de Aut(G).
4. Demostrar que G/Z(G) ∼= Int(G).
5. Demostrar que Int(G)=1 si y s´olo si G es abeliano.
Ejercicio 15. Demostrar que el grupo de automorfismos de un grupo no
abeliano no puede ser c´ıclico.



Ejercicio 16. Demostrar que el grupo Aut(Z2 × Z2) es isomorfo a S3.
Ejercicio 17. Demostrar que los grupos S3, Zpn (con p primo) y Z no son
producto directo internos de subgrupos propios.
Ejercicio 18. En cada uno de los siguientes casos, decidir si el grupo G es
o no producto directo de los subgrupos H y K.
1. G = R
×, H = {±1}, K = {x ∈ R| x > 0}.
2. G =
 a b
0 c

∈ GL2(R)
	
, H = {(
a 0
0 c
) ∈ GL2(R)}, K =
 1 b
0 1 
∈ GL2(R)
	
.
3. G = C
×, H = z ∈ C| |z| = 1}, K = x ∈ R| x > 0}.
Ejercicio 19. Sean G, H y K grupos. Demostrar que:
1. H × K ∼= K × H,
2. G × (H × K) ∼= (G × H) × K.
Ejercicio 20. Dados isomorfismos de grupos H ∼= J y K ∼= L, demostrar
que H × K ∼= J × L.
Ejercicio 21. Sean H, K, L y M grupos tales que H × K ∼= L × M. ¿Se
verifica necesariamente que H ∼= L y K ∼= M?
Ejercicio 22. Demostrar que no todo subgrupo de un producto directo
H × K es de la forma H1 × K1, con H1 subgrupo de H y K1 subgrupo de
K.
Ejercicio 23. Sean H, K dos grupos y sean H1 ◁ H, K1 ◁ K. Demostrar
que H1 × K1 ◁ H × K y que
H × K
H1 × K1
∼=
H
H1
×
K
K1
.
Ejercicio 24. Sean H, K ◁ G tales que H ∩ K = 1. Demostrar que G es
isomorfo a un subgrupo de G/H × G/K.
Ejercicio 25. Sean H y K subgrupos normales de G tales que HK = G.
Demostrar que
G/(H ∩ K) ∼= H/(H ∩ K) × K/(H ∩ K) ∼= (G/H) × (G/K).
Ejercicio 26. Demostrar que si G es un grupo que es producto directo
interno de subgrupos H y K, y N ⊴ G tal que N ∩ H = {1} = N ∩ K,
entonces N es abeliano




Ejercicio 27. Dar un ejemplo de un grupo G que sea producto directo
interno de dos subgrupos propios H y K, y que contenga a un subgrupo
normal no trivial N tal que N ∩ H = {1} = N ∩ K. Concluir que para
N ⊴ H × K es posible que se tenga
N ̸= (N ∩ (H × 1)) × (N ∩ (1 × K)).
Ejercicio 28. Sea G un grupo finito que sea producto directo interno de
dos subgrupos H y K tales que mcd(|H|, |K|) = 1. Demostrar que para todo
subgrupo N ≤ G verifica que N = (N ∩ H) × (N ∩ K).
Ejercicio 29. Sea G un grupo y sea f : G → G un endomorfismo idempotente (esto es, verificando que f
2 = f) y tal que Im(f) ⊴ G. Demostrar que
G ∼= Im(f) × Ker(f).
Ejercicio 30. Sea S un subconjunto de un grupo G. Se llama centralizador
de S en G al conjunto
CG(S) = {x ∈ G | xs = sx ∀s ∈ S}
y se llama normalizador de S en G al conjunto
NG(S) = {x ∈ G | xS = Sx }
1. Demostrar que el normalizador NG(S) es un subgrupo de G.
2. Demostrar que el centralizador CG(S) es un subgrupo normal de NG(S).
3. Demostrar que si S es un subgrupo de G entonces S es un subgrupo
normal de NG(S).
Ejercicio 31. Sea G un grupo y H y K subgrupos suyos con H ⊂ K.
Entonces demostrar que H es normal en K si y s´olo si K < NG(H). (As´ı,
el normalizador NG(H) queda caracterizado como el mayor subgrupo de G
en el que H es normal.)
Ejercicio 32. 1. Demostrar que CG(Z(G)) = G y que NG(Z(G)) = G.
2. Si G es un grupo y H < G ¿Cuando es G = NG(H)? ¿Y cuando es
G = CG(H)?
3. Si H es un subgrupo de orden 2 de un grupo G, demostrar que
NG(H) = CG(H). Deducir que H es normal en G si y solo si est´a
contenido en Z(G).
Ejercicio 33. Sea G un grupo arbitrario. Para dos elementos x, y ∈ G se
define su conmutador como el elemento [x, y] = xyx−1y
−1
. (El conmutador
recibe tal nombre porque [x, y]yx = xy.)





Como [x, y]
−1 = [y, x], el inverso de un conmutador es un conmutador.
Sin embargo el producto de dos conmutadores no tiene porqu´e ser un conmutador. Entonces se define el subgrupo conmutador o (primer) subgrupo
derivado de G, denotado [G, G], como el subgrupo generado por todos los
conmutadores de G.
1. Demostrar que, ∀a, x, y ∈ G, se tiene que a[x, y]a
−1 = [axa−1
, aya−1
].
2. Demostrar que [G, G] es un subgrupo normal de G.
3. demostrar que el grupo cociente G/[G, G], que se representa por Gab
,
es un grupo abeliano (que se llama el abelianizado de G).
4. Demostrar que G es abeliano si y s´olo si [G, G] = 1.
5. Sea N un subgrupo normal de G. Demostrar que el grupo cociente
G/N es abeliano si y s´olo si N > [G, G] (as´ı que el grupo [G, G] es el
menor subgrupo normal de G tal que el cociente es abeliano).
Ejercicio 34. 1. Calcular el subgrupo conmutador de los grupos S3, A4,
D4 y Q2.
2. Demostrar que, para n ≥ 3, el subgrupo conmutador de Sn es An y
que ´este es el ´unico subgrupo de Sn de orden n!/2.
\end{comment}