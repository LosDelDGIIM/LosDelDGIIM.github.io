\chapter{Subgrupos, Generadores, Retículos y Grupos cíclicos}
\begin{definicion}[Subgrupo]
    Dados dos grupos $G$ y $H$, se dice que $H$ es un subgrupo de $G$, denotado por $H < G$ si $H\subseteq G$ y la aplicación de inclusión\footnote{Viene dada por $i(x) = x$, para todo $x\in G$.} $i:H\to G$ es un homomorfismo de grupos.
\end{definicion}

\begin{observacion}
    Dado un grupo $(G,\ast,e)$, este tendrá siempre dos subgrupos: $(G,\ast,e)$ y $(\{e\},\ast,e)$, que llamaremos \underline{subgrupos impropios}. Si $H\subseteq G$ es un subgrupo de $G$ que no es ninguno de esos, diremos que es un \underline{subgrupo propio}.
\end{observacion}

\begin{ejemplo}
    Vemos claramente que:
    \begin{enumerate}
        \item $(\mathbb{Z},+) < (\mathbb{Q},+) < (\mathbb{R}, +)$
        \item $\{r^k \mid k\leq n, r\in D_n\} < D_n$ 
        \item $n\mathbb{Z} < \mathbb{Z}$ 
        \item $\SL_n(\mathbb{F}) < \GL_n(\mathbb{F})$
        \item $(\mathbb{Q}^\ast, \cdot) \not< (\mathbb{R}, +)$ No es un subgrupo, ya que $i(1) = 1 \neq 0$.
        \item $(\mathbb{Z}^+, +) \not< (\mathbb{Z}, +)$, ya que $(\mathbb{Z}^+, +)$ no es un grupo.
        \item $D_6 \not< D_8$, ya que $D_6\nsubseteq D_8$.
    \end{enumerate}
\end{ejemplo}

\begin{observacion}
    Si $G$, $H$ y $T$ son grupos de forma que $G < H < T$, entonces $G < T$.
    \begin{proof}
        La transitividad de $\subseteq $ nos da que $G\subseteq H\subseteq T$. Por otra parte, como $j:G\to H$ y $k:H\to T$ son homomorfismos, tendremos que $i=k\circ j:G\to T$ es un homomorfismo.
    \end{proof}
\end{observacion}

\begin{prop}
    Sea $G$ un grupo y $\emptyset \neq H \subseteq G$, entonces son equivalentes:
    \begin{enumerate}
        \item[$i)$] $H < G$
        \item[$ii)$] Se verifican:
            \begin{enumerate}[label=(\alph*)]
                \item Si $x,y\in H$ entonces $xy\in H$.
                \item $1\in H$.
                \item Si $x\in H$, entonces $x^{-1}\in H$.
            \end{enumerate}
        \item[$iii)$] Si $x,y\in H$, entonces $xy^{-1}\in H$.
    \end{enumerate}
    \begin{proof}
        Veamos las implicaciones de forma cíclica:
        \begin{description}
            \item [$i)\Longrightarrow ii)$] Como $H$ es un grupo, por su propia definición se han de cumplir $(a)$, $(b)$ y $(c)$.
            \item [$ii)\Longrightarrow iii)$] Si $x,y\in H$, entonces (por $(c)$) $y^{-1}\in H$, por lo que (por $(a)$) tendremos que $xy^{-1}\in H$.
            \item [$iii)\Longrightarrow i)$] Como $\emptyset \neq H$, existirá al menos un $x\in H$, por lo que $xx^{-1} = 1 \in H$. % // TODO: Terminar
        \end{description}
    \end{proof}
\end{prop}

\begin{prop}
    Sea $G$ un grupo finito y $\emptyset \neq H \subseteq G$, entonces son equivalentes:
    \begin{enumerate}
        \item[$i)$] $H < G$
        \item[$ii)$] Si $x,y\in H$, entonces $xy\in H$
    \end{enumerate}
    \begin{proof}
        Veamos las dos implicaciones:
        \begin{description}
            \item [$ii)\Longrightarrow i)$] Como $G$ es finito, para todo $x\in G$, existirá $n>0$ de forma que $x^n = 1$, por lo que $x^{-1} = x^{n-1}$% // TODO: Terminar
            \item [] 
        \end{description}
    \end{proof}
\end{prop}

\begin{ejemplo}
    Veamos que:
    \begin{enumerate}
        \item $A_n < S_n$
        \item Si $n>0$, $n\mathbb{Z} < \mathbb{Z}$ Todo subgrupo de $\mathbb{Z}$ es así. % // TODO: Probarlo
        \item $V<S_4$ % // TODO: Definir grupo de klein, klein abs, \ldots en el tema anterior
        \item Si $n\mid m$, entonces $D_n < D_m$
    \end{enumerate}
\end{ejemplo}

\begin{definicion}
    Sea $G$ un grupo y $f$ una aplicación, llamamos imagen directa de $G$ por $f$ a:
    \begin{equation*}
        f_\ast(G) = \{f(x) \mid x \in G\}
    \end{equation*}
    Respectivamente, imagen inversa a:
    \begin{equation*}
        f^\ast(H') = \{x\in G \mid f(x) \in  G\}
    \end{equation*}
\end{definicion}

\begin{prop}
    Sea $f:G\to G'$ un homomorfismo de grupos: 
    \begin{enumerate}
        \item[$i)$] Si $H<G$, entonces $f_\ast(H) = \{f(x) \mid x\in H\}< G'$ 
        \item[$ii)$] Si $H'<G'$, entonces $f^\ast(H') = \{x\in G\mid f(x) \in H'\}< G$
    \end{enumerate}
    \begin{equation*}
        Im(f) = f_\ast(G)\subseteq G' \\
        \ker(f) = f^\ast(1) < G
    \end{equation*}
\end{prop}

\begin{prop}
    Sea $\{H_i\}_{i \in I}$ una familia de subgrupos de $G$, entonces:
    \begin{equation*}
        \bigcap_{i \in I} H_i < G
    \end{equation*}
    \begin{proof}
        Si $1\in H_i$ para todo $i \in I$, entonces:
        \begin{equation*}
            1 \in \bigcap_{i \in I} H_i \neq \emptyset 
        \end{equation*}
        Si ahora $x,y\in \bigcap_{i \in I} H_i$, entonces $x,y \in H_i$ para todo $i \in I$, por lo que $xy^{-1}\in H_i$ para todo $i \in I$, por lo que:
        \begin{equation*}
            xy^{-1} \in \bigcap_{i \in I}H_i
        \end{equation*}
    \end{proof}
\end{prop}

\begin{ejemplo}
    En general, la unión de subgrupos no es un subgrupo:
    \begin{equation*}
        2\mathbb{Z} \cup 3\mathbb{Z}  \not< \mathbb{Z}
    \end{equation*}
\end{ejemplo}

\begin{definicion}
    Sea $G$ un grupo y $S\subseteq G$, el subgrupo generado por $S$, $\langle S \rangle $ es la intersección todos los subgrupos de $G$ que contienen a $S$.

    Es decir, $\langle S \rangle $ es el menor subgrupo de $G$ que contiene a $S$.
\end{definicion}

\begin{prop}
    Sea $G$ un grupo, $S\subseteq G$, $\langle S \rangle $, entonces:
    \begin{itemize}
        \item Si $S = \emptyset $, entonces $\langle S \rangle = \{e\}$, el grupo trivial.
        \item Si $S\neq \emptyset $, entonces $\langle S \rangle = \{x_1^{\gamma_1}x_2^{\gamma_2}\ldots x_m^{\gamma_m} \mid m \geq 1, x_i \in S, \gamma_i \in \mathbb{Z}\}$
        \item Si $G$ es finito y $\emptyset \neq S$, entonces $\langle S \rangle = \{productos finitos de elementos de S\}$
    \end{itemize} % // TODO: Revisar diferencia entre 2 y 3, poner 2 como producto infinito
\end{prop}

\begin{ejemplo}
    Veamos:
    \begin{enumerate}
        \item Si $S= \{r\}\subseteq D_n$, entonces $\langle S \rangle = \{1,r,r^2, \ldots, r^{n-1}\}$
        \item Si $S = \{s\}\subseteq D_n$, entonces $\langle S \rangle = \{1, s\}$
        \item Si $S = \{(1\ 2)(3\ 4), (1\ 3)(2\ 4)\}\subseteq S_4$, entonces $\langle S \rangle = V$
        \item Si $S=\{(x_1\ x_2\ x_3) \mid x_1<x_2<x_3\}\subseteq S_n$, entonces $\langle S \rangle = A_n$
        \item Si $S = \left\{i =\left(\begin{array}{cc}
            i & 0 \\
            0 & -i 
        \end{array}\right), j = \left(\begin{array}{cc}
            0 & 1 \\
            -1 & 0 
        \end{array}\right)\right\}\subseteq \GL_2(\mathbb{C})$, entonces $\langle i,j \rangle < \GL_2(\mathbb{C})$.
    \end{enumerate}
\end{ejemplo}

\begin{prop}
    Si $S\subseteq G$ de forma que $\langle S \rangle =G$, entonces $S$ es un sistema de generadores de $G$.

    En el caso en el que $S$ sea finito, diremos que $G$ es finitamente generado. % // TODO: Meter esto en T1.
\end{prop}

