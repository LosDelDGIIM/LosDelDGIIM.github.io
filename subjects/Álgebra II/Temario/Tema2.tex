\chapter{Subgrupos, Generadores, Retículos y Grupos cíclicos}
\begin{definicion}[Subgrupo]
    Dados dos grupos $G$ y $H$, decimos que $H$ es un subgrupo de $G$, denotado por $H < G$ si $H\subseteq G$ y la aplicación de inclusión\footnote{Viene dada por $i(x) = x$, para todo $x\in G$.} $i:H\to G$ es un homomorfismo de grupos.
\end{definicion}

\begin{observacion}
    Dado un grupo $(G,\ast,e)$, este tendrá siempre dos subgrupos:
    \begin{itemize}
        \item $(\{e\},\ast,e)$, al que llamaremos \underline{subgrupo trivial}.
        \item El propio $(G, \ast, e)$
    \end{itemize}
\end{observacion}

\begin{definicion}
    Sea $H$ un subgrupo de otro $G$, diremos que $H$ es un \underline{subgrupo impropio} de $G$ si $H$ es el grupo trivial o el propio $G$. En otro caso, diremos que $H$ es un \underline{subgrupo propio} de $G$.
\end{definicion}

\begin{notacion}
    Recordamos la notación que ya usábamos en Álgebra I para, fijado $n\in \mathbb{N}\setminus\{0\}$, denotar a todos los múltiplos de $n$ en $\mathbb{Z}$:
    \begin{equation*}
        n\mathbb{Z} = \{nm \mid m\in \mathbb{Z}\}
    \end{equation*}
\end{notacion}

\begin{ejemplo}
    Vemos claramente que:
    \begin{enumerate}
        \item $(\mathbb{Z},+) < (\mathbb{Q},+) < (\mathbb{R}, +)$
        \item $\{r^k \mid k\leq n, r\in D_n\} < D_n$ 
        \item $n\mathbb{Z} < \mathbb{Z}$ para todo $n\in \mathbb{N}$.
        \item $\SL_n(\mathbb{F}) < \GL_n(\mathbb{F})$
        \item $(\mathbb{Q}^\ast, \cdot) \not< (\mathbb{R}, +)$ No es un subgrupo, ya que $i(1) = 1 \neq 0$.
        \item $(\mathbb{Z}^+, +) \not< (\mathbb{Z}, +)$, ya que $(\mathbb{Z}^+, +)$ no es un grupo.
        \item $D_6 \not< D_8$, ya que $D_6\nsubseteq D_8$.
    \end{enumerate}
\end{ejemplo}

\begin{observacion}
    Si $G$, $H$ y $T$ son grupos de forma que $G < H < T$, entonces $G < T$.
    \begin{proof}
        La transitividad de $\subseteq $ nos da que $G\subseteq H\subseteq T$. Por otra parte, como las inclusiones $j:G\to H$ y $k:H\to T$ son homomorfismos, tendremos que $i=k\circ j:G\to T$ es un homomorfismo.
    \end{proof}
\end{observacion}

\begin{prop}\label{prop:carac_subgrupo}
    Sea $G$ un grupo y $\emptyset \neq H \subseteq G$, entonces son equivalentes:
    \begin{enumerate}
        \item[$i)$] $H < G$
        \item[$ii)$] Se verifican:
            \begin{enumerate}[label=(\alph*)]
                \item Si $x,y\in H$ entonces $xy\in H$.
                \item $1\in H$.
                \item Si $x\in H$, entonces $x^{-1}\in H$.
            \end{enumerate}
        \item[$iii)$] Si $x,y\in H$, entonces $xy^{-1}\in H$.
    \end{enumerate}
    \begin{proof}
        Veamos las implicaciones de forma cíclica:
        \begin{description}
            \item [$i)\Longrightarrow ii)$] Como $H$ es un grupo, por su definición se han de cumplir $(a)$, $(b)$ y $(c)$.
            \item [$ii)\Longrightarrow iii)$] Si $x,y\in H$, entonces $y^{-1}\in H$, por lo que tendremos que $xy^{-1}\in H$.
            \item [$iii)\Longrightarrow i)$] Como $\emptyset \neq H$, existirá al menos un $x\in H$, por lo que $xx^{-1} = 1 \in H$.

                Además, si $x\in H$ también tendremos que $1x^{-1} = x^{-1}\in H$. Con esto tenemos ya que $H$ es un grupo. Al considerar en $H$ la misma operación que en $G$, tenemos directamente que $i:H\to G$ es un homomorfismo, ya que $id:H\to H$ es un homomorfismo y al extender el codominio para considerar la aplicación inclusión $i$, seguirá siendo un homomorfismo\footnote{Notemos que si en $H$ tenemos una operación distinta que en $G$ esto no siempre será cierto y habrá que comprobar que $i:H\to G$ es un homomorfismo.}.
        \end{description}
    \end{proof}
\end{prop}

\begin{prop}
    Sea $G$ un grupo finito y $\emptyset \neq H \subseteq G$, entonces son equivalentes:
    \begin{enumerate}
        \item[$i)$] $H < G$
        \item[$ii)$] Si $x,y\in H$, entonces $xy\in H$
    \end{enumerate}
    \begin{proof}
        Veamos las dos implicaciones:
        \begin{description}
            \item [$i)\Longrightarrow ii)$] Se verifica por ser $H$ un grupo.
            \item [$ii)\Longrightarrow i)$] Como $G$ es finito, por la Proposición~\ref{prop:orden_grupo}, para todo $x\in G$ existirá $n>0$ de forma que $x^n = 1$, por lo que $x^{-1} = x^{n-1}$. De esto deducimos que $x^{-1}\in H$ y que $1=xx^{-1}\in H$. Finalmente, como en $H$ consideramos la misma operación que en $G$, tenemos que $i:H\to G$ es un homomorfismo.
        \end{description}
    \end{proof}
\end{prop}

\begin{ejemplo}
    Se deja como ejercicio comprobar que:
    \begin{enumerate}
        \item $A_n < S_n$
        \item Todo subgrupo de $\mathbb{Z}$ es de la forma $n\mathbb{Z}$ con $n\in \mathbb{N}$.
        \item $V<S_4$
        \item Si $n\mid m$, entonces $D_n < D_m$
    \end{enumerate}
\end{ejemplo}

\begin{definicion}
    Sea $G$ un grupo, $f:G\to G'$ una aplicación, y $H\subseteq G$, $H'\subseteq G'$, definimos:
    \begin{itemize}
        \item El conjunto imagen directa de $H$ por $f$ como el conjunto:
            \begin{equation*}
                f_\ast(H) = \{f(x) \mid x\in H\}\subseteq G'
            \end{equation*}
        \item El conjunto imagen inversa de $H'$ por $f$ como el conjunto:
            \begin{equation*}
                f^\ast(H') = \{x\in H \mid f(x) \in H'\}\subseteq G
            \end{equation*}
    \end{itemize}
\end{definicion}

\begin{prop}
    Sea $f:G\to G'$ un homomorfismo de grupos, entonces: 
    \begin{enumerate}
        \item[$i)$] Si $H<G$, entonces $f_\ast(H)< G'$ 
        \item[$ii)$] Si $H'<G'$, entonces $f^\ast(H')< G$
    \end{enumerate}
    \begin{proof}
        Demostramos las dos implicaciones:
        \begin{enumerate}
            \item[$i)$] Sean $x,y\in f_\ast(H)$, entonces $\exists a,b\in H$ de forma que $x=f(a), y=f(b)$. Como $H$ es un subgrupo de $G$, tendremos que $ab^{-1}\in H$, por lo que $f(ab^{-1}) = f(a){f(b)}^{-1} = xy^{-1}\in f_\ast(H)$. Concluimos que $f_\ast(H)$ es un subgrupo de $G'$.
            \item[$ii)$] Sean $x,y\in f^\ast(H')$, entonces $a=f(x), b=f(y)\in H'$. Por ser $H'$ un subgrupo de $G'$, tendremos que $ab^{-1} = f(x){f(y)}^{-1}=f(xy^{-1})\in H'$, por lo que ${xy^{-1}\in f^\ast(H')}$. Concluimos que $f^\ast(H')$ es un subgrupo de $G$.
        \end{enumerate}
    \end{proof}
\end{prop}

\begin{prop}
    Sea $\{H_i\}_{i \in I}$ una familia de subgrupos de $G$, entonces la intersección de todo ellos sigue siendo un subgrupo de $G$:
    \begin{equation*}
        \bigcap_{i \in I} H_i < G
    \end{equation*}
    \begin{proof}
        En primer lugar, como $H_i < G$ para todo $i \in I$, se ha de verificar que $1\in H_i$ $\forall i \in I$, por lo que $1\in \cap_{i \in I} H_i \neq \emptyset $. Como la intersección es no vacía, podemos pensar aplicar el tercer punto de la Proposición~\ref{prop:carac_subgrupo}, para comprobar que es un subgrupo de $G$.\\

        \noindent
        Para ello, sean $x,y\in \cap_{i \in I}H_i$, entonces $x,y\in H_i$ para todo $i \in I$, por lo que por ser $H_i < G$, tendremos que $xy^{-1}\in H_i$ $\forall i \in I$, luego:
        \begin{equation*}
            xy^{-1} \in \bigcap_{i \in I}H_i
        \end{equation*}
        Concluimos que $\cap_{i \in I}H_i$ es un subgrupo de $G$.
    \end{proof}
\end{prop}

\begin{ejemplo}
    En general, la unión de subgrupos no es un subgrupo:
    \begin{equation*}
        2\mathbb{Z} \cup 3\mathbb{Z}  \not< \mathbb{Z}
    \end{equation*}
    Ya que $2,3\in 2\mathbb{Z}\cup 3\mathbb{Z}$ y $2+3 = 5\notin 2\mathbb{Z}\cup 3\mathbb{Z}$.
\end{ejemplo}

\begin{definicion}[Subgrupo generado]
    Sea $G$ un grupo y $S\subseteq G$, definimos el subgrupo generador por $S$ como el menor subgrupo de $G$ que contiene a $S$, es decir:
    \begin{equation*}
        \langle S \rangle  = \bigcap \{H < G \mid S\subseteq H\}
    \end{equation*}
\end{definicion}

\begin{prop}\label{prop:subgrupos_generados}
    Sea $(G,\cdot ,e)$ un grupo, $S\subseteq G$, entonces:
    \begin{itemize}
        \item Si $S = \emptyset $, entonces $\langle S \rangle = \{e\}$, el grupo trivial.
        \item Si $S\neq \emptyset $, entonces $\langle S \rangle = \{x_1^{\gamma_1}x_2^{\gamma_2}\ldots x_m^{\gamma_m} \mid m \geq 1, x_i \in S, \gamma_i \in \mathbb{Z}\}$
    \end{itemize} 
    \begin{proof}
        Distinguimos casos:
        \begin{itemize}
            \item Si $S=\emptyset $, entonces $\{e\} < G$ con $S\subseteq \{e\}$. Como $\{e\}$ solo tiene un elemento y todo subgrupo de $G$ contiene a $e$, concluimos que:
                \begin{equation*}
                    \langle S \rangle = \bigcap \{H < G \mid S \subseteq H\} = \{e\}
                \end{equation*}
            \item Si $S\neq \emptyset $, si notamos $\cc{S} = \bigcap \{H < G \mid S \subseteq H\}$, queremos ver que:
                \begin{equation*}
                    \cc{S} = \{x_1^{\gamma_1}x_2^{\gamma_2}\ldots x_m^{\gamma_m} \mid m \geq 1, x_i \in S, \gamma_i \in \mathbb{Z}\}
                \end{equation*}
                \begin{description}
                    \item [$\supseteq)$] Como $S\subseteq \cc{S}$ y $\cc{S}$ es un grupo, tendremos que:
                        \begin{equation*}
                            x_1^{\gamma_1}x_2^{\gamma_2}\ldots x_m^{\gamma_m} \in \cc{S} \qquad x_i \in S, \gamma_i \in \mathbb{Z}\quad \forall 1\leq i \leq m
                        \end{equation*}
                    \item [$\subseteq)$] Si llamamos $A$ al conjunto de la derecha, $A$ es un grupo, ya que si tomamos $a,b\in A$, existirán $x_1,\ldots,x_p$ y $y_1,\ldots,y_q$ en $S$ y $\gamma_1,\ldots,\gamma_p,\alpha_1,\ldots,\alpha_q\in \mathbb{Z}$ de forma que:
                        \begin{equation*}
                            a = x_1^{\gamma_1} \ldots x_p^{\gamma_p} \qquad b = y_1^{\alpha_1} \ldots y_q^{\alpha_q
}                        \end{equation*}
                        Por lo que
                        \begin{equation*}
                            ab^{-1} = x_1^{\gamma_1} \ldots x_p^{\gamma_p} y_q^{-\alpha_q} \ldots y_1^{-\alpha_1} \in A
                        \end{equation*}
                        Lo que demuestra que $A$ es un subgrupo de $G$. Además, como es claro que $S\subseteq A$, tenemos un grupo del que $S$ es subconjunto, por lo que por ser $\cc{S}$ el menor subgrupo que contiene a $S$, está claro que $\cc{S}\subseteq A$.
                \end{description}
        \end{itemize}
    \end{proof}
\end{prop}

\begin{coro}
    Si $S\subseteq G$ de forma que $\langle S \rangle =G$, entonces $S$ es un conjuntos de generadores de $G$.
    \begin{proof}
        Por la Proposición~\ref{prop:subgrupos_generados}, sabemos que si $\langle S \rangle = G$, entonces cualquier elemento $x\in G$ se puede expresar de la forma:
        \begin{equation*}
            x = x_1^{\gamma_1}x_2^{\gamma_2}\ldots x_m^{\gamma_m} \qquad x_i \in S, \gamma_i \in \mathbb{Z}, \quad \forall 1\leq i \leq m
        \end{equation*}
        Por lo que $S$ es un conjunto de generadores de $G$.
    \end{proof}
\end{coro}

\begin{ejemplo}
    Ejemplos interesantes de subgrupos generados por ciertos conjuntos son:
    \begin{enumerate}
        \item Si $S= \{r\}\subseteq D_n$, entonces $\langle S \rangle = \{1,r,r^2, \ldots, r^{n-1}\}$
        \item Si $S = \{s\}\subseteq D_n$, entonces $\langle S \rangle = \{1, s\}$
        \item Si $S = \{(1\ 2)(3\ 4), (1\ 3)(2\ 4)\}\subseteq S_4$, entonces $\langle S \rangle = V$
        \item Si $S=\{(x_1\ x_2\ x_3) \mid x_1<x_2<x_3\}\subseteq S_n$, entonces $\langle S \rangle = A_n$
        \item Si $S = \left\{\left(\begin{array}{cc}
            i & 0 \\
            0 & -i 
        \end{array}\right),  \left(\begin{array}{cc}
            0 & 1 \\
            -1 & 0 
        \end{array}\right)\right\}\subseteq \GL_2(\mathbb{C})$, entonces $\langle S \rangle < \GL_2(\mathbb{C})$.
    \end{enumerate}
\end{ejemplo}

