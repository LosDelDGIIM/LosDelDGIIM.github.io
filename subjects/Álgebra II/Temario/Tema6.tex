\chapter{Clasificación de grupos abelianos finitos}
\noindent
El objetivo final del tema es demostrar los teoremas de estructura de los grupos abelianos finitos, que permiten clasificar todos los grupos de este tipo según su orden. De esta forma, dado un grupo abeliano finito, la clasificación que realizaremos en este tema nos permitirá encontrar un grupo abeliano finito bien conocido al que el grupo dado sea isomorfo.

\section{Descomposiciones como producto de grupos cíclicos}
\noindent
Como toma de contacto, serán de especial relevancia dos resultados que ya vimos en Capítulos anteriores, como:

\begin{enumerate}
    \item En la Proposición~\ref{prop:carac_prod_finito_ciclicos} vimos que:
        \begin{equation*}
            C_n\oplus C_m \cong C_{nm} \Longleftrightarrow \mcd(n,m) = 1
        \end{equation*}
    \item En el Teorema~\ref{teo:prod_grupos_sylow} vimos que si $G$ es un grupo finito en el que todos sus subgrupos de Sylow son únicos, entonces $G$ es producto directo interno de todos ellos:
        \begin{equation*}
            G \cong P_1\oplus P_2 \oplus \ldots \oplus P_k
        \end{equation*}
\end{enumerate}
Como trabajaremos con subgrupos abelianos, será usual usar la notación de $\oplus$ en lugar de la de $\times$.

\begin{teo}[Estructura de los $p-$grupos abelianos finitos]\label{teo:1_tema6}\ \\
    Sea $A$ un $p-$grupo abeliano finito con orden $|A| = p^n$ para $n\geq 1$, entonces existen enteros $\beta_1\geq \beta_2 \geq \ldots \geq \beta_t \geq 1$ de forma que:
    \begin{equation*}
        \beta_1 + \beta_2 + \ldots + \beta_t = n \quad \text{y} \quad A\cong C_{p^{\beta_1}} \oplus C_{p^{\beta_2}} \oplus \ldots \oplus C_{p^{\beta_t}}
    \end{equation*}
    Además, esta expresión es única, es decir, si existen $\alpha_1\geq \alpha_2\geq \ldots \geq \alpha_s \geq 1$ de forma que:
    \begin{equation*}
        \alpha_1 + \alpha_2 + \ldots + \alpha_s = n \quad \text{y} \quad A\cong C_{p^{\alpha_1}} \oplus C_{p^{\alpha_2}} \oplus \ldots \oplus C_{p^{\alpha_s}}
    \end{equation*}
    entonces $s = t$ y $\alpha_k = \beta_k$, para todo $k \in \{1,\ldots,t\}$.
    % \begin{proof} % // TODO: Copiar de Prado
    % \end{proof}
\end{teo}

\begin{observacion}
    Notemos que lo que estamos haciendo es tomar particiones de $n$ de la forma $\beta_i$, y este Teorema nos dice que el $p-$grupo puede escribirse de forma única salvo isomorfismos como producto de ciertos grupos cíclicos.\newline 
    Es decir, existen tantos $p-$grupos abelianos de orden $p^n$ como particiones tengamos del número $n$, salvo isomorfismos. Por tanto, conocemos ya cómo son todos los $p-$grupos abelianos finitos.
\end{observacion}

\begin{ejemplo}
    Por ejemplo:
    \begin{itemize}
        \item Para saber los grupos abelianos finitos de orden $8 = 2^3$ que hay (salvo isomorfismos), calculamos cada una de las posibles particiones del número 3 (el exponente del $2$):
            \begin{align*}
                3& \longrightarrow A\cong C_8 \\
                2, 1& \longrightarrow A\cong C_4\oplus C_2 \\
                1, 1, 1& \longrightarrow A\cong C_2 \oplus C_2 \oplus C_2
            \end{align*}
        \item Para saber los grupos abelianos finitos de orden $81 = 3^4$, calculamos cada una de las particiones de $4$:
            \begin{align*}
                4 &\longrightarrow A\cong C_{81} \\
                3, 1 &\longrightarrow A\cong C_{27} \oplus C_3 \\
                2, 2 &\longrightarrow A \cong C_{9} \oplus C_9 \\
                2, 1, 1 &\longrightarrow A \cong C_9 \oplus C_3 \oplus C_3 \\
                1, 1, 1, 1 &\longrightarrow A \cong C_3 \oplus C_3 \oplus C_3 \oplus C_3 
            \end{align*}
    \end{itemize}
\end{ejemplo}

\subsection{Descomposición cíclica primaria}

\begin{teo}[Estructura de los grupos abelianos finitos]\label{teo:2_tema6}\ \\
    Sea $A$ un grupo abeliano finito con $|A| = p_1^{\gamma_1}\ldots p_k^{\gamma_k}$ siendo $p_i$ primo $\forall i \in \{1,\ldots,k\}$, entonces existen $t_1,t_2,\ldots, t_k\in \mathbb{N}$ de forma que para el $i-$ésimo entero $t_i$ existen
    \begin{equation*}
        n_{i1} \geq n_{i2} \geq \ldots \geq n_{it_{i}} \geq 1 \\
    \end{equation*}
    Con:
    \begin{equation*}
        n_{i1} + n_{i2} + \ldots + n_{it_{i}} = \gamma_i
    \end{equation*}
    Para dichos $n_{ij}$ con $j \in \{1,\ldots, t_i\}$ y $i \in \{1,\ldots,k\}$ podremos escribir:
    \begin{equation*}
        A \cong \bigoplus_{i=1}^k \left(\bigoplus_{j=1}^{t_i} C_{p_i^{n_{ij}}}\right)
    \end{equation*}
    Y la descomposición es única.
    \begin{proof}
        Si $A$ es abeliano y finito, entonces todos sus $p-$subgrupos de Sylow son normales, luego podemos escribir:
        \begin{equation*}
            A = P_1 \oplus P_2 \oplus \ldots \oplus P_k
        \end{equation*}
        Siendo $\{P_1, P_2, \ldots, P_k\}$ el conjunto de todos sus $p-$subgrupos de Sylow, de forma que $|P_i| = p_i^{r_i}$, para todo $i \in \{1,\ldots,k\}$. Como cada $P_i$ es un $p_i-$subgrupo abeliano finito, aplicando el Teorema~\ref{teo:1_tema6}, podemos encontrar:
        \begin{equation*}
            n_{i1} \geq n_{i2} \geq \ldots \geq n_{it_{i}} \geq 1 \qquad 
            n_{i1} + n_{i2} + \ldots + n_{it_{i}} = \gamma_i
        \end{equation*}
        De forma que podamos escribir:
        \begin{equation*}
            P_i = \bigoplus_{j=1}^{t_i} C_{p_i^{n_{ij}}} \qquad \forall i \in \{1,\ldots,k\}
        \end{equation*}
        De donde tenemos la expresión de la tesis.
    \end{proof}
\end{teo}

\begin{definicion}
    Sea $A$ un grupo abeliano finito, el Teorema~\ref{teo:2_tema6} motiva las siguientes definiciones:
    \begin{itemize}
        \item La única descomposición obtenida para $A$ en dicho teorema recibirá el nombre de \underline{descomposición cíclica primaria de $A$}.
        \item A las potencias $p_i^{n_{ij}}$ obtenidas (usando la notación del Teorema), las llamaremos \underline{divisores elementales de $A$}.
        \item A cada $p-$subgrupo de Sylow de $A$ lo llamaremos componente $p-$primaria de $A$.
    \end{itemize}
\end{definicion}

\begin{ejemplo}
    Si tenemos un grupo finito abeliano $A$ con $|A| = 360 = 2^3\cdot 3^2\cdot 5$, buscamos las posibles descomposiciones cíclicas primarias de $A$, que obtenemos fácilmente tras combinar todas las particiones posibles de los exponentes de los primos que aparecen en la descomposición de $|A|$, es decir, las particiones de $3$, $2$ y $1$:
    \begin{equation*}
        \begin{array}{r|l}
            \text{Divisores elementales} & \text{Descomposición cíclica primaria} \\
            \hline
            2^3\ 3^2\ 5 & C_8\oplus C_9 \oplus C_5 \\
            2^2\ 2\ 3^2\ 5 & C_4\oplus C_2 \oplus C_9 \oplus C_5\\
            2\ 2\ 2\ 3^2\ 5 & C_2 \oplus C_2 \oplus C_2 \oplus C_9 \oplus C_5\\
            2^3\ 3\ 3\ 5 & C_8\oplus C_3 \oplus C_3 \oplus C_5 \\
            2\ 2^2\ 3\ 3\ 5 & C_2 \oplus C_4 \oplus C_3 \oplus C_3 \oplus C_5\\
            2\ 2\ 2\ 3\ 3\ 5 & C_2 \oplus C_2 \oplus C_2 \oplus C_3 \oplus C_3 \oplus C_5
        \end{array}
    \end{equation*}
    Estas sería todas las descomposiciones cíclicas primarias de $A$. Es decir, dado cualquier grupo de orden $360$, sabemos que será isomorfo a alguno de los grupos que aparecen a la derecha de la tabla.
\end{ejemplo}

\noindent
Sin embargo, si recordamos la Proposición~\ref{prop:carac_prod_finito_ciclicos}, podemos escribir (multiplicando aquellos cíclicos de mayor orden que sean primos relativos):
\begin{align*}
    C_8\oplus C_9 \oplus C_5 &\cong C_{360} \\
    C_4\oplus C_2 \oplus C_9 \oplus C_5 &\cong C_{180} \oplus C_2 \\
    C_2 \oplus C_2 \oplus C_2 \oplus C_9 \oplus C_5 &\cong C_{90} \oplus C_2 \oplus C_2 \\
    C_8\oplus C_3 \oplus C_3 \oplus C_5 \oplus C_8 &\cong C_{120} \oplus C_3 \\
    C_2 \oplus C_4 \oplus C_3 \oplus C_3 \oplus C_5 &\cong C_{60} \oplus C_6 \\
    C_2 \oplus C_2 \oplus C_2 \oplus C_3 \oplus C_3 \oplus C_5 &\cong C_{30} \oplus C_6 \oplus C_2
\end{align*}

\begin{coro}
    Si $A$ es un grupo abeliano finito con $|A| = p_1p_2 \ldots p_k = n$, entonces salvo isomorfismo, el único grupo abeliano de orden $n$ es el cíclico $C_n$.
    \begin{proof}
        Utilizando el Teorema~\ref{teo:2_tema6}, podemos escribir:
        \begin{equation*}
            A \cong C_{p_1} \oplus C_{p_2} \oplus \ldots \oplus C_{p_k}
        \end{equation*}
        Y como $\mcd(p_i, p_j) = 1$ para cada $i,j\in \{1,\ldots,k\}$ con $i\neq j$, tenemos que:
        \begin{equation*}
            C_{p_1} \oplus C_{p_2} \oplus \ldots \oplus C_{p_k} = C_{p_1p_2\ldots p_k} = C_n
        \end{equation*}
    \end{proof}
\end{coro}

\subsection{Descomposición cíclica}

\begin{teo}[Descomposición cíclica de un grupo abeliano finito]\label{teo:3_tema6}\ \\
    Si $A$ es un grupo abeliano finito, entonces existen unos únicos $d_1,d_2,\ldots,d_t \in \mathbb{N}$ de forma que:
    \begin{equation*}
        d_1d_2\ldots d_t = |A| \qquad \text{\ y\ } \qquad d_i \mid d_j, \quad \forall j\leq i
    \end{equation*}
    Para los que se tiene que:
    \begin{equation*}
        A\cong C_{d_1} \oplus C_{d_2} \oplus \ldots \oplus C_{d_t}
    \end{equation*}
    \begin{proof}
        Supuesto que $|A| = p_1^{r_1}\ldots p_k^{r_k}$ es la descomposición de $|A|$ en primos, si usamos la descomposición que nos da el Teorema~\ref{teo:2_tema6}, existen $t_1,t_2,\ldots,t_k \in \mathbb{N}$ y
        \begin{align*}
            m_{i1} \geq m_{i2} \geq \ldots \geq m_{it_{i}} \geq 1 \\
            m_{i1} + m_{i2} + \ldots + m_{it_{i}} = r_i \\
            \qquad \forall i \in \{1,\ldots,k\}
        \end{align*}
        De forma que:
        \begin{equation*}
            A \cong \bigoplus_{i=1}^k \left(\bigoplus_{j=1}^{t_i} C_{p_i^{m_{ij}}}\right)
        \end{equation*}
        Sea $t = \max\{t_1, t_2, \ldots, t_k\}$, definimos:
        \begin{equation*}
            n_{ij} = \left\{\begin{array}{ll}
                    m_{ij} & \text{si\ } j \leq t_i \\
                    0 & \text{si\ } j > t_i
            \end{array}\right. \qquad \forall j\in \{1,\ldots,t\}, \in i\{1,\ldots,k\}
        \end{equation*}
        Observemos que no hemos hecho mas que extender la anterior tabla dentada $(m_{ij})_{\substack{j \in \{1,\ldots,t_i\} \\ i \in \{1,\ldots,k\}}}$ a la tabla $k\times t$ $(n_{ij})_{\substack{j \in \{1,\ldots,t\} \\ i \in \{1,\ldots,k\}}}$, rellenando con ceros los huecos que no teníamos. De esta forma, si consideramos la matriz que en la entrada $(i,j)$ tiene $p_i^{n_{ij}}$:
        \begin{equation*}
            \left(\begin{array}{cccc}
                p_1^{n_{11}} & p_1^{n_{12}} & \ldots & p_1^{n_{1t}} \\
                p_2^{n_{21}} & p_2^{n_{22}} & \ldots  & p_2^{n_{2t}}  \\
                 \vdots & \vdots & \ddots & \vdots \\
                 p_k^{n_{k1}} & p_k^{n_{k2}} & \ldots & p_k^{n_{kt}}  
            \end{array}\right)
        \end{equation*}
        Tenemos que $A$ es el producto directo de los grupos cíclicos de órdenes las entradas de la tabla anterior (ya que $A \cong A \oplus C_1 = A\oplus \{1\}$). Si tomamos el producto de los elementos de cada columna:
        \begin{align*}
            d_1 &= p_1^{n_{11}}p_2^{n_{21}} \ldots p_k^{n_{k1}} \\
            d_2 &= p_1^{n_{12}}p_2^{n_{22}} \ldots p_k^{n_{k2}} \\
                &\vdots \\
            d_t &= p_1^{n_{1t}}p_2^{n_{2t}} \ldots p_k^{n_{kt}} 
        \end{align*}
        Efectivamente, tendremos que:
        \begin{equation*}
            d_1d_2 \ldots d_t = p_1^{r_1} p_2^{r_2} \ldots p_k^{r_k} = |A|
        \end{equation*}
        Fijado $i \in \{1,\ldots,k\}$, como $n_{ij}\geq n_{i(j+1)}$ (por la construcción realizada) para todo $j \in \{1,\ldots,t-1\}$, tendremos entonces que si $u,v\in \{1,\ldots,t\}$ con $u\leq v$, los exponentes de los primos en $d_u$ serán mayores que los exponentes de los primos en $d_v$, por lo que $d_v \mid d_u$, lo que se verifica para todo $u\leq v$. Además, tendremos que:
        \begin{align*}
            C_{d_1} &\cong C_{p_1^{n_{11}}} \oplus C_{p_2^{n_{21}}} \oplus \ldots \oplus C_{p_k^{n_{k1}}} \\
                    &\vdots \\
            C_{d_t} &\cong C_{p_1^{n_{1t}}} \oplus C_{p_2^{n_{2t}}} \oplus \ldots \oplus C_{p_k^{n_{kt}}} 
        \end{align*}
        De donde $A \cong C_{d_1}\oplus C_{d_2}\oplus \ldots \oplus C_{d_t}$. La unicidad de la descomposición viene de la unicidad de la descomposición del Teorema~\ref{teo:2_tema6} más la construcción de los $d_j$ realizada.
    \end{proof} 
\end{teo}

\begin{definicion}
    Sea $A$ un grupo abeliano finito, el Teorema~\ref{teo:3_tema6} motiva las siguientes definiciones:
    \begin{itemize}
        \item La única descomposición obtenida para $A$ en dicho teorema recibirá el nombre de \underline{descomposición cíclica de $A$}.
        \item Los enteros $d_j$ obtenidos recibirán el nombre de \underline{factores invariantes}.
    \end{itemize}
\end{definicion}

\begin{ejemplo}
    Recuperando el ejemplo anterior, si tenemos $A$, un grupo abeliano finito con $|A| = 360 = 2^3\cdot 3^2\cdot 5$, buscaremos escribir para cada conjunto de divisores elementales las respectivas descomposiciones cíclicas:
    \begin{itemize}
        \item Para la partición $\{2^3, 3^2, 5\}$, teníamos la descomposición cíclica primaria:
            \begin{equation*}
                A\cong C_8\oplus C_9\oplus C_5 
            \end{equation*}
            Que siguiendo con la construcción realizada en la demostración anterior, nos da la tabla:
            \begin{equation*}
                \left(\begin{array}{ccc}
                    2^3 \\
                    3^2 \\
                    5 
                \end{array}\right)
            \end{equation*}
            Por tanto, obtenemos el factor invariante:
            \begin{align*}
                d_1 = 2^3\cdot 3^2\cdot 5
            \end{align*}
            Por lo que la descomposición cíclica de $A$ será $A\cong C_{360}$.
        \item Para la partición $\{2^2, 2, 3^2, 5\}$, la descomposición cíclica primaria fue:
            \begin{equation*}
                A \cong C_4 \oplus C_2 \oplus C_9 \oplus C_5
            \end{equation*}
            En este caso, tendremos $t = \max\{2, 1, 1\} = 2$, por lo que tendremos dos factores invariantes, que podemos calcular de forma fácil a partir de la tabla:
            \begin{equation*}
                \left(\begin{array}{ccc}
                    2^2 & 2 \\
                    3^2 & 1 \\
                    5 & 1  
                \end{array}\right)
            \end{equation*}
            Por lo que tendremos (los productos de las columnas):
            \begin{align*}
                d_1 &= 2^2 \cdot 3^2 \cdot 5 = 180 \\
                d_2 &= 2\cdot 1\cdot 1 = 2
            \end{align*}
            Y la descomposición cíclica es:
            \begin{equation*}
                A\cong C_{180}\oplus C_2
            \end{equation*}
        \item Para la descomposición $\{2,2,2,3^2,5\}$, teníamos:
            \begin{equation*}
                A\cong C_2 \oplus C_2 \oplus C_2 \oplus C_9 \oplus C_5
            \end{equation*}
            Y tendremos $t=3$, con:
            \begin{equation*}
                \left(\begin{array}{ccc}
                    2 & 2 & 2 \\
                    3^2 & 1 & 1 \\
                    5 & 1 & 1
                \end{array}\right)
            \end{equation*}
            Por lo que:
            \begin{equation*}
                A\cong C_{90} \oplus C_2 \oplus C_2
            \end{equation*}
        \item Para $\{2^3, 3,3,5\}$, teníamos:
            \begin{equation*}
                A\cong C_8\oplus C_3\oplus C_3\oplus C_5
            \end{equation*}
            Y tenemos:
            \begin{equation*}
                \left(\begin{array}{ccc}
                    2^3 & 1 \\
                    3 & 3 \\
                    5 & 1  
                \end{array}\right)
            \end{equation*}
            La descomposición cíclica será:
            \begin{equation*}
                A\cong C_{120} \oplus C_3
            \end{equation*}
        \item Para $\{2^2,2,3,3,5\}$, teníamos:
            \begin{equation*}
                A\cong C_4\oplus C_2\oplus C_3 \oplus C_3 \oplus C_5
            \end{equation*}
            Y tenemos:
            \begin{equation*}
                \left(\begin{array}{ccc}
                    2^2 & 2 \\
                    3 & 3 \\
                    5 & 1  
                \end{array}\right)
            \end{equation*}
            Por lo que tenemos la descomposición cíclica:
            \begin{equation*}
                A\cong C_{60} \oplus C_6
            \end{equation*}
        \item Para $\{2, 2, 2, 3, 3, 5\}$ teníamos:
            \begin{equation*}
                A\cong C_2\oplus C_2\oplus C_2\oplus C_3\oplus C_3\oplus C_5
            \end{equation*}
            Y tenemos:
            \begin{equation*}
                \left(\begin{array}{ccc}
                    2 & 2 & 2 \\
                    3 & 3 & 1 \\
                    5 & 1 & 1
                \end{array}\right)
            \end{equation*}
            Por lo que la descomposición cíclica será:
            \begin{equation*}
                A\cong C_{30} \oplus C_6 \oplus C_2
            \end{equation*}
    \end{itemize}
\end{ejemplo}

\begin{ejemplo}
    Sea $A$ un grupo abeliano finito con $|A| = 180 = 2^2\cdot 3^2\cdot 5$, buscamos hayar sus posibles descomposiciones cíclicas y descomposiciones cíclicas primarias:
    \begin{equation*}
        \begin{array}{c|c|c|c}
            \text{Divisores elementales} & \text{desc. cíclica primaria} & \text{factores invariantes} & \text{desc. cíclica} \\
            \hline
            \{2^2, 3^2, 5\} & C_4\oplus C_9 \oplus C_5 & d_1= 2^2\cdot 3^2\cdot 5 = 180 & C_{180} \\
            \hline
            \{2,2,3^2,5\} & C_2\oplus C_2\oplus C_9\oplus C_5 & \begin{array}{c}
                    d_1 = 2\cdot 3^2\cdot 5 = 90 \\
                    d_2 = 2
            \end{array}& C_{90}\oplus C_2 \\
            \hline
            \{2^2, 3, 3, 5\} & C_4\oplus C_3\oplus C_3\oplus C_5 & \begin{array}{c}
                    d_1 = 2^2\cdot 3\cdot 5 = 60 \\
                    d_2 = 3
            \end{array}& C_{60} \oplus C_3 \\
            \hline
                    \{2,2,3,3,5\} & C_2\oplus C_2\oplus C_3\oplus C_3\oplus C_5 & \begin{array}{c}
                            d_1 = 2\cdot 3\cdot 5 = 30 \\
                            d_2 = 2\cdot 3 = 6
                        \end{array} & C_{30} \oplus C_6
        \end{array}
    \end{equation*}
\end{ejemplo}

\begin{ejemplo} % // TODO: EJercicio 1
    Listar los órdenes de todos los elementos de un grupo abeliano de orden 8.\\

    \noindent
    Sea $A$ un grupo abeliano finito de orden $8 = 2^3$, entonces lo podemos clasificar en:
    \begin{itemize}
        \item $C_8$, donde usaremos la Proposición~\ref{prop:orden_zn} y el Corolario~\ref{coro:primos_relativos_generan_todo}:
            \begin{itemize}
                \item $O(0) = 1$.
                \item Los elementos $1,3,5$ y $7$ tienen orden 8.
                \item $O(2) = \nicefrac{8}{\mcd(2,8)} = 4$.
                \item $O(4) = \nicefrac{8}{\mcd(4,8)} = 2$.
                \item $O(6) = \nicefrac{8}{\mcd(6,8)} = 4$.
            \end{itemize}
        \item $C_4\oplus C_2$, aplicamos que $O(a,b) = \mcm(O(a), O(b))$:
            Como los órdenes de los elementos en $C_4$ son $\{1,2,4\}$ y en $C_2$ son $\{1,2\}$, las posibilidades que tenemos son: $\{1,2,4\}$. Si primero listamos los órdenes de los elementos en $C_4$:
            \begin{itemize}
                \item $O(0) = 1$.
                \item $O(1) = 4$.
                \item $O(3) = 4$.
                \item $O(2) = 2$.
            \end{itemize}
            Podemos ver de forma fácil que:
            \begin{itemize}
                \item $O(0,0) = 1$.
                \item $O(0,1) = 2$.
                \item $O(1,b) = 4 = O(3,b)$, $\forall b\in C_2$
                \item $O(2,b) = 2 $, $\forall b\in C_2$.
            \end{itemize}
        \item $C_2\oplus C_2\oplus C_2$, los órdenes son $\{1,2\}$ y todos tienen orden 2 salvo el elemento $(0,0,0)$, que tiene orden 1.
    \end{itemize}
\end{ejemplo}

\begin{ejemplo}
    Listar los órdenes de todos los elementos de un grupo abeliano de orden 12.\\

    \noindent
    Sea $A$ con $|A| = 12 = 2^2\cdot 3$, tenemos entonces que $A\cong \mathbb{Z}_{12}$ o $A\cong \mathbb{Z}_6\oplus \mathbb{Z}_2$.
    \begin{itemize}
        \item En $\mathbb{Z}_{12}$:
            \begin{itemize}
                \item $O(0) = 1$.
                \item $1, 5, 7$ y $11$ tienen orden 12.
                \item $O(2) = \nicefrac{12}{\mcd(2,12)} = 6$.
                \item $O(3) = \nicefrac{12}{\mcd(3,12)} = 4$.
                \item $O(4) = \nicefrac{12}{\mcd(4,12)} = 3$.
                \item $O(6) = \nicefrac{12}{\mcd(6,12)} = 2$.
                \item $O(8) = \nicefrac{12}{\mcd(8,12)} = 3$.
                \item $O(9) = \nicefrac{12}{\mcd(9,12)} = 4$.
                \item $O(10) = \nicefrac{12}{\mcd(12,10)} = 6$.
            \end{itemize}
        \item En $\mathbb{Z}_6\oplus\mathbb{Z}_2$:
            \begin{multline*}
                O(a,b) \in  \mcm(Div(6), Div(2)) = \mcm(\{1,2,3,6\}, \{1,2\}) = \{1,2,3,6\} \\ \forall (a,b)\in \mathbb{Z}_6\oplus\mathbb{Z}_2
            \end{multline*}
            El orden de los elementos de $\mathbb{Z}_6$ son:
            \begin{itemize}
                \item $O(0) = 1$.
                \item 1 y 5 tienen orden 6.
                \item $O(2) = \nicefrac{6}{\mcd(2,6)} = 3$.
                \item $O(3) = \nicefrac{6}{\mcd(3,6)} = 2$.
                \item $O(4) = \nicefrac{6}{\mcd(4,6)} = 3$.
            \end{itemize}
            Ahora:
            \begin{itemize}
                \item $O(0,0) = 1$.
                \item $O(0,1) = 2$, ya que ${(0,1)}^{2} = (0,0)$.
                \item $O(1,b) = O(5,b) = 6$ $\forall b\in \mathbb{Z}_2$.
                \item $O(3, b) = 2$ $\forall b\in \mathbb{Z}_2$.
                \item $O(2, 0) = O(4, 0) = 3$.
                \item $O(2, 1) = O(4, 1) = 6$.
            \end{itemize}
    \end{itemize}
\end{ejemplo}

\section{Clasificación de grupos abelianos no finitos}
\noindent
Buscamos ahora tratar de clasificar los grupos abelianos no finitos. Para ello, recordaremos lo que es un grupo finitamente generado, e introduciremos nuevos conceptos.

\begin{definicion}
    Un grupo abeliano $A$ se dice que es finitamente generado si existe un conjunto:
    \begin{equation*}
        X = \{x_1,\ldots,x_r\} \subseteq A
    \end{equation*}
    De forma que para todo $a\in A$, existen $\lm_1, \ldots, \lm_r \in \mathbb{Z}$ de forma que:
    \begin{equation*}
        a = \sum_{k=1}^{r} \lm_k x_k
    \end{equation*}
    En dicho caso, diremos que $X$ es un \underline{sistema de generadores de A}, y notaremos:
    \begin{equation*}
        A = \langle x_1, \ldots, x_r \rangle 
    \end{equation*}
\end{definicion}

\begin{definicion}[Base]
Sea $A$ un grupo abeliano, un conjunto de generadores $X = \{x_1,\ldots,x_r\}$ de $A$ es una \underline{base} si los elementos de $X$ son $\mathbb{Z}-$linealmente independientes. Es decir, que si $\lm_1,\ldots,\lm_r\in \mathbb{Z}$ con:
\begin{equation*}
    \sum_{k=1}^{r}\lm_k x_k = 0
\end{equation*}
Entonces, ha de ser $\lm_k = 0$ para todo $k\in \{1,\ldots,r\}$. En dicho caso, diremos que $A$ es un \underline{grupo abeliano libre de rango $r$}.
\end{definicion}

% Un grupo finito no puede tener bases por la independencia !!!

\begin{prop}
    Si $A$ es un grupo abeliano libre de rango $r$, entonces:
    \begin{equation*}
        A\cong \mathbb{Z}^r
    \end{equation*}
    \begin{proof}
        Como $A$ es un grupo abeliano libre de rango $r$, para dar un homomorfismo de $A$ en cualquier otro grupo basta dar las imágenes de los elementos de la base de $A$.\\

        \noindent
        De esta forma, si $X = \{x_1,\ldots,x_r\}$ es una base de $A$, definimos el homomorfismo $\phi:A\to \mathbb{Z}^r$ de la forma más canónica posible sobre los elementos de la base de $A$:
        \begin{align*}
            \phi(x_1) &= (1, 0, \ldots, 0) \\
            \phi(x_2) &= (0, 1, \ldots, 0) \\
                      &\vdots \\
            \phi(x_r) &= (0, 0, \ldots, 1)
        \end{align*}
        Dado $a\in A$, como $X$ es una base de $A$, existirán $\lm_1,\ldots,\lm_r\in \mathbb{Z}$ de forma que:
        \begin{equation*}
            a = \sum_{k=1}^{r}\lm_kx_k
        \end{equation*}
        Por lo que:
        \begin{equation*}
            \phi(a) = \phi\left(\sum_{k=1}^{r}\lm_kx_k\right) = \sum_{k=1}^{r}\phi(\lm_kx_k) = \sum_{k=1}^{r}\lm_k \phi(x_k)
        \end{equation*}
        Es fácil ver que $\phi$ es biyectiva, por lo que $\phi$ nos da un isomorfismo entre $A$ y $\mathbb{Z}^r$.
    \end{proof}
\end{prop}

\subsection{Proceso de clasificación}
\noindent
Una vez entendidas las definiciones básicas necesarias para comenzar el estudio de los grupos abelianos no finitos procederemos ahora a explicar el procedimiento por el cual somos capaces de clasificar cualquier grupo abeliano no finito. Esto es, dar un isomorfismo estándar para cualquier grupo abeliano no finito dado.\\

\noindent
Este procedimiento requiere de una gran cantidad de resultados que tienen que ver con cómo son los subrupos y los cocientes de grupos como $\mathbb{Z}^r$ para $r\in \mathbb{N}$, que ya hemos visto que es el único grupo libre de rango $r$, salvo isomorfismo. Como esta intención escapa al interés de la asignatura y como seremos capaces de clasificar los grupos abelianos no finitos mediante un procedimiento algorítmico, mostraremos ahora los resultados que nos permiten realizarlo, la mayoría de ellos sin demostración. 

Animamos al lector a profundizar más en estos teoremas de clasificación, que seguro se encuentran en algún libro de la bibliografía de la asignatura.\\

\noindent
El primer problema con el que nos encontramos es con el de cómo conocer un grupo abeliano no finito, ya que al tener infinitos elementos no nos es posible listar todos sus elementos para conocerlo bien. Como puede adivinarse, lo que haremos será trabajar con grupos abeliano finitamente generados, y las relaciones entre los elementos del grupo las deduciremos a partir de las relaciones entre los generadores del grupo. Esto nos conlleva a pensar que la forma en la que describiremos un grupo abeliano no finito será mediante su \textbf{presentación}. 

Como a lo largo de este capítulo siempre conoceremos un grupo por su presentación, impondremos ahora varias reglas para tratar de estandarizar la forma en la que nos den las presentaciones, con el fin también de hacer los razonamientos abstractos y genéricos con una notación más fácil y cómoda. Estas reglas las crearemos a partir de la clasificación de un ejemplo de grupo abeliano no finito.

\begin{ejemplo}
    Se pide clasificar el grupo:
    \begin{equation*}
        G = \langle x,y,z \mid x^3=y^4, x^2z = z^{-1}y, xy = yx, xz = zx, yz=zy \rangle 
    \end{equation*}
    Este grupo nos viene dado con la notación multiplicativa, algo habitual en grupos y que venimos haciendo durante toda la asignatura, pero podemos tratar de escribir el grupo con notación aditiva, algo que nos será más cómodo en estos casos:
    \begin{equation*}
        G = \langle x,y,z \mid 3x=4y, 2x+z = -z+y, x+y=y+x, x+z=z+x, y+z=z+y \rangle 
    \end{equation*}
    Además, como nuestro objetivo es trabajar con grupos abelianos esta notación estará más que justificada, aprovechando la intuición de que es mucho más natural que una suma sea abeliana antes que un producto lo sea (podemos pensar en las matrices, por ejemplo). De esta forma, convenimos en eliminar de la presentación del grupo todas las relaciones que nos indiquen la conmutatividad entre los generadores del grupo, por simplicidad:
    \begin{equation*}
        G = \langle x,y,z \mid 3x=4y, 2x+z=-z+y \rangle 
    \end{equation*}
    Finalmente, convenimos estandarizar la forma en la que damos las ecuaciones, tratando de expresar estas siempre como una combinación lineal de los generadores igualadas a ceros:
    \begin{equation*}
        G = \langle x,y,z \mid 3x-4y = 0, 2x+2z-y = 0 \rangle 
    \end{equation*}
\end{ejemplo}

Con las tres reglas de notación introducidas en el ejemplo superior, cualquier grupo abeliano finitamente generado vendrá dado a nosotros como una presentación del estilo:
\begin{equation*}
    G = \left\langle x_1,x_2,\ldots,x_n \mid \begin{array}{c}
        a_{11}x_1 + a_{12}x_2 + \ldots + a_{1n}x_n = 0 \\
        a_{21}x_1 + a_{22}x_2 + \ldots + a_{2n}x_n = 0 \\
        \vdots \\
        a_{m1}x_1 + a_{m2}x_2 + \ldots + a_{mn}x_n = 0
    \end{array}\right\rangle 
\end{equation*}

Notemos que, de esta forma, dar un grupo es equivalente a dar una matriz. Es decir, dada una matriz $m\times n$, podemos pensar que hay un grupo asociado a dicha matriz que tendrá $n$ elementos que generen el grupo y que dichos elementos cumplan $m$ relaciones entre sí. Así, la presentación superior nos da la matriz:
\begin{equation*}
    \left(\begin{array}{cccc}
        a_{11} & a_{12} & \cdots & a_{1n} \\
        a_{21} & a_{22} & \cdots & a_{2n} \\
        \vdots & \vdots & \ddots & \vdots \\
        a_{m1} & a_{m2} & \cdots & a_{mn} 
    \end{array}\right)
\end{equation*}
Esta matriz recibirá el nombre de \underline{matriz de relaciones} del grupo.\\

\noindent
Una vez introducida la matriz de relaciones de un grupo a partir de su presentación, si volvemos a la presentación del grupo, podemos observar que dar una presentación de un grupo $G$ generado por los elementos $\{x_1,x_2,\ldots,x_n\}$ es equivalente a dar un epimorfismo $\phi: \mathbb{Z}^n \to G$ de forma que la base canónica\footnote{Podemos trasladar el concepto de ``base canónica de $\mathbb{R}^n$'' que teníamos en Álgebra Lineal a este ámbito de clasificación de grupos.} de $\mathbb{Z}^n$ $\{e_1,e_2,\ldots, e_n\}$ tenga como imágenes:
\begin{equation*}
    \phi(e_k) = x_k \qquad \forall k\in \{1,\ldots,n\}
\end{equation*}
Y que además el conjunto:
\begin{equation*}
    \left\{\begin{array}{c}
        a_{11}e_1+a_{12}e_2 + \cdots + a_{1n}e_n,\\
        a_{21}e_1 + a_{22}e_2+\cdots+a_{2n}e_n, \\ 
        \vdots\\
        a_{m1}e_1+a_{m2}e_2 + \cdots + a_{mn}e_n
    \end{array}\right\}
\end{equation*}
Sea un sistema de generadores de $\ker(\phi)$. Aplicando el Primer Teorema de Isomorfía sobre $\phi$ obtenemos que:
\begin{equation*}
    \mathbb{Z}^n/\ker(\phi) \cong G
\end{equation*}
Por lo que parece que vamos por buen camino si queremos clasificar todos los grupos no abelianos finitamente generados, nos falta estudiar cómo son los grupos cocientes de $\mathbb{Z}^n$. Como dijimos anteriormente, no vamos a hacerlo, por lo que mostraremos ahora una serie de resultados sin demostración que nos ayudarán a seguir en nuestra tarea.\\

\noindent
Observemos ahora que podemos hacer los siguientes cambios en una base de un grupo libre y que tras ellos seguiremos teniendo una base del mismo:
\begin{enumerate}
    \item Sustituir un elemento de una base por su opuesto.
    \item Reordenar los elementos de la base.
    \item Sumarle a un elemento de la base otro elemento de la base distinto a él.
\end{enumerate}
Estos cambios en la base de un grupo libre dan lugar a las siguientes transformaciones elementales sobre las columnas de una matriz e relaciones:
\begin{enumerate}
    \item Cambiar una columna por su opuesta.
    \item Reordenar las columnas de la matriz.
    \item Sumar a todos los elementos de la columna $i-$ésima un múltiplo de los elementos de la columna $j-$ésima, con $j\neq i$.
\end{enumerate}
Como las columnas de una matriz no tienen nada de especial, de forma análoga pueden justificarse estas operaciones sobre las filas de una matriz, sustituyendo la palabra ``columna'' por ``fila''. Estas transformaciones dan lugar al siguiente resultado:

\begin{prop}
    Si $M$ es la matriz de relaciones de una presentación de un grupo abeliano $G$, es decir:
    \begin{equation*}
        G = \left\langle x_1,\ldots,x_n \mid MX = 0  \right\rangle 
    \end{equation*}
    Donde:
    \begin{equation*}
        X = \left(\begin{array}{c}
            x_1 \\
            \vdots \\
            x_n
        \end{array}\right)
    \end{equation*}
    Y $M'$ es una matriz obtenida mediante transformaciones elementales del tipo 1, 2 o 3 sobre filas o columnas de $M$, entonces $M'$ también es una matriz de relaciones de una presentación de $G$.
\end{prop}

\begin{teo}
    Dada $M\in \cc{M}_{m,n}(\mathbb{Z})$, podemos realizar transformaciones elementales en $M$ del tipo 1, 2 o 3 en las filas y/o columnas de $M$ hasta llegar a una matriz diagonal de la forma\footnote{Puede que la matriz no tenga filas de ceros y que en su lugar tenga columnas de ceros, o que no tenga ni filas ni columnas de ceros, todo dependerá del orden y del rango de la matriz.}:
    \begin{equation*}
        M' = \left(\begin{array}{cccc}
            d_1 & 0 & \cdots & 0 \\
            0 & d_2 & \cdots & 0 \\
            \vdots & \vdots & \ddots & \vdots \\
            0 & 0 & \cdots &  d_r \\
            0 & 0 & \cdots & 0 
        \end{array}\right)
    \end{equation*}
    Donde $d_i \mid d_{i+1}$ para $i \in \{1,\ldots,r-1\}$ y $r$ es el rango de $M$. Además, si $M'$ es la matriz de relaciones de un grupo $G$ generado por $n$ generadores, entonces:
    \begin{equation*}
        G \cong \mathbb{Z}^{n-r} \oplus \mathbb{Z}_{d_1} \oplus \mathbb{Z}_{d_2} \oplus \cdots \oplus \mathbb{Z}_{d_r}
    \end{equation*}
\end{teo}

\begin{definicion}
    Si $G$ es un grupo abeliano finitamente generado, este tendrá su matriz de relaciones $M\in \cc{M}_{m,n}(\mathbb{Z})$, sobre la que podemos aplicar las transformaciones pertinentes para conseguir la matriz $M'$ del Teorema anterior, obteniendo que:
    \begin{equation*}
        G \cong \mathbb{Z}^{n-r} \oplus \mathbb{Z}_{d_1} \oplus \mathbb{Z}_{d_2} \oplus \cdots \oplus \mathbb{Z}_{d_r}
    \end{equation*}
    Para ciertos enteros $n,r,d_1,d_2,\ldots,d_r\in \mathbb{Z}$, con $d_i \mid d_{i+1}$ $\forall i \in \{1,\ldots,r-1\}$
    \begin{itemize}
    \item La matriz $M'$ obtenida a partir de $M$ recibirá el nombre de \underline{forma normal de} \underline{Smith de $M$}.
        \item A los elementos $d_i$ obtenidos en $M'$ los llamaremos \underline{factores invariantes de $M$}.
        \item Diremos que $G$ tiene rango $n-r$.
        \item Diremos que $\mathbb{Z}^{n-r}$ es la \underline{parte libre de $G$}.
        \item Diremos que $\mathbb{Z}_{d_1}\oplus \mathbb{Z}_{d_2}\oplus \cdots \oplus \mathbb{Z}_{d_r}$ es la \underline{parte de torsión de $G$}, o \underline{grupo de} \underline{torsión de $G$}, denotado por $T(G)$.
    \end{itemize}
\end{definicion}

\subsection{Ejemplos}
\noindent
Una vez explicado el procedimiento teórico, mostraremos varios ejemplos de cómo conseguir la forma normal de Smith de una matriz dada, así como ejemplos sobre cómo podemos clasificar los grupos abelianos no finitos finitamente generados.

\begin{ejemplo}
    Se pide calcular la forma normal de Smith de:
    \begin{equation*}
        \left(\begin{array}{cccc}
            0 & 2 & 0 \\
            -6 & -4 & -6 \\
            6 & 6 & 6 \\
            7 & 10 & 6  
        \end{array}\right)
    \end{equation*}
    Aplicaremos a continuación un algoritmo similar al que usábamos en Geometría I para calcular la forma normal de Hermite de una matriz, por lo que los pasos mediante los que ``intentamos tener un número en una cierta posición de una matriz'' no los explicaremos, confiando en que el lector es suficientemente habilidoso como para conseguirlo por él mismo. Sin embargo, explicaremos algunos pasos clave en el algoritmo a aplicar que sí debemos tener en cuenta.\\

    \noindent
    En primer lugar, calcularemos el máximo común divisor de los elementos que aparecen en las entradas de la matriz, en este caso, tenemos que es 1, por lo que buscamos escribir un 1 en la posición\footnote{La esquina superior izquierda.} $(1,1)$ de la matriz. Como consejo, diremos que es recomendable no hacer ceros en los elementos hasta no tener algún 1 disponible. Una forma de conseguir un 1 en la posición $(1,1)$ es (usaremos una notación informal para describir las operaciones, $F_4-F_3$ debe entenderse como ``a la fila 4 le restamos la 3''):
    \begin{equation*}
        \left(\begin{array}{cccc}
            0 & 2 & 0 \\
            -6 & -4 & -6 \\
            6 & 6 & 6 \\
            7 & 10 & 6  
        \end{array}\right) \xrightarrow{F_4 - F_3}
        \left(\begin{array}{cccc}
            0 & 2 & 0  \\
            -6 & -4 & -6  \\
            6 & 6 & 6  \\
            1 & 4 & 0  
        \end{array}\right) \xrightarrow{F_1 \leftrightarrow F_4}
        \left(\begin{array}{cccc}
            1 & 4 & 0  \\
            -6 & -4 & -6  \\
            6 & 6 & 6  \\
            0 & 2 & 0  
        \end{array}\right) 
    \end{equation*}
    Una vez tenemos el 1 en la posición deseada, tratamos de rellenar la primera fila y la primera columna entera con ceros (salvo el 1 que acabamos de colocar), algo que ya sabíamos hacer de otras asignaturas:
    \begin{equation*}
        \left(\begin{array}{cccc}
            1 & 4 & 0  \\
            -6 & -4 & -6  \\
            6 & 6 & 6  \\
            0 & 2 & 0  
        \end{array}\right) 
        \underset{C_1-C_3}{\xrightarrow{F_1-2F_4}}
        \left(\begin{array}{cccc}
            1 & 0 & 0  \\
            0 & -4 & -6  \\
            0 & 6 & 6  \\
            0 & 2 & 0  
        \end{array}\right) 
    \end{equation*}
    Ahora el máximo común divisor de todos los elementos es 2, por lo que tratamos de poner un 2 en la siguiente posición de la diagonal de la matriz, tras el 1 de antes:
    \begin{equation*}
        \left(\begin{array}{cccc}
            1 & 0 & 0  \\
            0 & -4 & -6  \\
            0 & 6 & 6  \\
            0 & 2 & 0  
        \end{array}\right) 
        \xrightarrow{F_2\leftrightarrow F_4}
        \left(\begin{array}{cccc}
            1 & 0 & 0  \\
            0 & 2 & 0  \\
            0 & 6 & 6  \\
            0 & -4 & -6  
        \end{array}\right) 
    \end{equation*}
    Ahora, hacemos ceros debajo de este 2:
    \begin{equation*}
        \left(\begin{array}{cccc}
            1 & 0 & 0  \\
            0 & 2 & 0  \\
            0 & 6 & 6  \\
            0 & -4 & -6  
        \end{array}\right) 
        \underset{F_4+2F_2}{\xrightarrow{F_3-3F_2}}
        \left(\begin{array}{cccc}
            1 & 0 & 0  \\
            0 & 2 & 0  \\
            0 & 0 & 6  \\
            0 & 0 & -6  
        \end{array}\right) 
    \end{equation*}
    El máximo común divisor de los elementos que nos quedan es 6, que ya está en la posición deseada, por lo que solo nos queda hacer ceros en la última fila:
    \begin{equation*}
        \left(\begin{array}{cccc}
            1 & 0 & 0  \\
            0 & 2 & 0  \\
            0 & 0 & 6  \\
            0 & 0 & -6  
        \end{array}\right) 
        \xrightarrow{F_4+F_3}
        \left(\begin{array}{cccc}
            1 & 0 & 0  \\
            0 & 2 & 0  \\
            0 & 0 & 6  \\
            0 & 0 & 0  
        \end{array}\right) 
    \end{equation*}
    Tenemos ya la forma normal de Smith de la matriz original.
\end{ejemplo}

\begin{ejemplo}
    Calcular la forma normal de Smith de:
    \begin{equation*}
        \left(\begin{array}{ccc}
            4 & 0 & 0 \\
            0 & 6 & 0 \\
            0 & 0 & 8
        \end{array}\right)
    \end{equation*}
    Como no está en forma normal de Smith porque el primer elemento no se corresponde con el máximo común divisor de todos los demás (que es 2), veamos cómo podemos añadir este 2 a la posición $(1,1)$ de la matriz. Mostraremos solo las operaciones a realizar sobre $M$, entendiendo que el algoritmo que explicamos en el ejemplo anterior está ya claro 
    \begin{align*}
        &\left(\begin{array}{ccc}
            4 & 0 & 0 \\
            0 & 6 & 0 \\
            0 & 0 & 8
        \end{array}\right)
        \xrightarrow{F_2 + F_1}
        \left(\begin{array}{ccc}
            4 & 0 & 0 \\
            4 & 6 & 0 \\
            0 & 0 & 8
        \end{array}\right)
        \xrightarrow{C_2 - C_1}
        \left(\begin{array}{ccc}
            4 & -4 & 0 \\
            4 & 2 & 0 \\
            0 & 0 & 8
        \end{array}\right)
        \xrightarrow{-F_1} \\
        &\left(\begin{array}{ccc}
            -4 & 4 & 0 \\
            4 & 2 & 0 \\
            0 & 0 & 8
        \end{array}\right)
        \xrightarrow{F_2\leftrightarrow F_1}
        \left(\begin{array}{ccc}
            4 & 2 & 0 \\
            -4 & 4 & 0 \\
            0 & 0 & 8
        \end{array}\right)
        \xrightarrow{C_2\leftrightarrow C_1}
        \left(\begin{array}{ccc}
            2 & 4 & 0 \\
            4 & -4 & 0 \\
            0 & 0 & 8
        \end{array}\right) \\
        &\xrightarrow{F_2-2F_1}
        \left(\begin{array}{ccc}
            2 & 4 & 0 \\
            0 & -12 & 0 \\
            0 & 0 & 8
        \end{array}\right) 
        \xrightarrow{C_2-2C_1}
        \left(\begin{array}{ccc}
            2 & 0 & 0 \\
            0 & -12 & 0 \\
            0 & 0 & 8
        \end{array}\right) 
        \xrightarrow{F_2+2F_3}
        \left(\begin{array}{ccc}
            2 & 0 & 0 \\
            0 & -12 & 16 \\
            0 & 0 & 8
        \end{array}\right)  \\
        &\xrightarrow{C_2+C_3}
        \left(\begin{array}{ccc}
            2 & 0 & 0 \\
            0 & 4 & 16 \\
            0 & 8 & 8
        \end{array}\right)  
        \xrightarrow{F_3-2F_2}
        \left(\begin{array}{ccc}
            2 & 0 & 0 \\
            0 & 4 & 16 \\
            0 & 0 & -24
        \end{array}\right)  
        \underset{-C_3}{\xrightarrow{C_3-4C_2}}
        \left(\begin{array}{ccc}
            2 & 0 & 0 \\
            0 & 4 & 0 \\
            0 & 0 & 24
        \end{array}\right)  
    \end{align*}
\end{ejemplo}

\begin{ejemplo}
    Sea $A$ el grupo:
    \begin{equation*}
        A = \left\langle x,y,z,t \mid \begin{array}{r}
            14x + 4y + 4z + 14t = 0 \\
            -6x + 4y + 4z + 10t = 0 \\
            -16x- 4y - 4z - 20t = 0
        \end{array}\right\rangle 
    \end{equation*}
    Se pide calcular el rango de $A$ y todos los grupos abelianos del mismo orden que el grupo de torsión de $A$ que no sean isomorfos al grupo de torsión de $A$.\\

    \noindent
    Sea:
    \begin{equation*}
        M = \left(\begin{array}{cccc}
            14 & 4 & 4 & 14 \\
            -6 & 4 & 4 & 10 \\
            -16 & -4 & -4 & -20 
        \end{array}\right)
    \end{equation*}
    Vamos a calcular la forma normal de Smith de $M$, con el fin de clasificar $A$ para conocer su rango y grupo de torsión ($-(F_1+F_3)$ significa que primero a la fila 1 le sumamos la 3 y que luego consideramos los opuestos de los elementos de la fila 1 como la nueva fila 1).
    \begin{align*}
        &\left(\begin{array}{cccc}
            14 & 4 & 4 & 14 \\
            -6 & 4 & 4 & 10 \\
            -16 & -4 & -4 & -20 
        \end{array}\right) 
        \xrightarrow{-(F_1+F_3)}
        \left(\begin{array}{cccc}
            2 & 0 & 0 & 6 \\
            -6 & 4 & 4 & 10 \\
            -16 & -4 & -4 & -20 
        \end{array}\right)  
        \xrightarrow{C_4-3C_1} \\
        &\left(\begin{array}{cccc}
            2 & 0 & 0 & 0 \\
            -6 & 4 & 4 & 28 \\
            -16 & -4 & -4 & 28
        \end{array}\right) 
        \xrightarrow{\substack{F_2 + 3F_1\\F_3 + 8F_1}}
        \left(\begin{array}{cccc}
            2 & 0 & 0 & 0 \\
            0 & 4 & 4 & 28 \\
            0 & -4 & -4 & 28
        \end{array}\right)  
        \xrightarrow{F_2 + F_3} \\
        &\left(\begin{array}{cccc}
            2 & 0 & 0 & 0 \\
            0 & 0 & 0 & 56 \\
            0 & -4 & -4 & 28
        \end{array}\right) 
        \xrightarrow{F_2\leftrightarrow F_3}
        \left(\begin{array}{cccc}
            2 & 0 & 0 & 0 \\
            0 & -4 & -4 & 28 \\
            0 & 0 & 0 & 56 
        \end{array}\right)  
        \xrightarrow{F_2' = -F_2} 
        \left(\begin{array}{cccc}
            2 & 0 & 0 & 0 \\
            0 & 4 & 4 & -28 \\
            0 & 0 & 0 & 56 
        \end{array}\right)  
        \xrightarrow{C_3 + 7C_2} \\
        & \left(\begin{array}{cccc}
            2 & 0 & 0 & 0 \\
            0 & 4 & 4 & 0 \\
            0 & 0 & 0 & 56 
        \end{array}\right)   
        \xrightarrow{C_3 - C_2} 
        \left(\begin{array}{cccc}
            2 & 0 & 0 & 0 \\
            0 & 4 & 0 & 0 \\
            0 & 0 & 0 & 56 
        \end{array}\right)   
        \xrightarrow{C_3 \leftrightarrow C_4}
        \left(\begin{array}{cccc}
            2 & 0 & 0 & 0 \\
            0 & 4 & 0 & 0 \\
            0 & 0 & 56 & 0 
        \end{array}\right)   
    \end{align*}
    De esta forma, tendremos que:
    \begin{equation*}
        A \cong \mathbb{Z} \oplus \mathbb{Z}_2 \oplus \mathbb{Z}_4 \oplus \mathbb{Z}_{56}
    \end{equation*}
    Por lo que el rango de $A$ es 1 y su grupo de torsión es:
    \begin{equation*}
        T(A) \cong \mathbb{Z}_2 \oplus \mathbb{Z}_4 \oplus \mathbb{Z}_{56}
    \end{equation*}
    Un grupo de orden $2\cdot 4\cdot 56 = 448 = 2^6\cdot 7$. Esta de arriba es su descomposición cíclica, de la que podemos sacar fácilmente su descomposición cíclica primaria:
    \begin{equation*}
        T(A) \cong \mathbb{Z}_2 \oplus \mathbb{Z}_4 \oplus \mathbb{Z}_8 \oplus \mathbb{Z}_7
    \end{equation*}
    Que como vemos, corresponde a la partición $\{2, 2^2, 2^3, 7\}$. Para calcular todos los grupos abelianos de orden 448 no isomorfos a $T(A)$, calculamos las distintas particiones de 6 (el exponente del 2):
    \begin{gather*}
        6 \\
        5, 1 \\
        4, 1, 1 \\
        4, 2 \\
        3, 1, 1, 1 \\
        3, 2, 1 \\
        3, 3 \\
        2, 1, 1, 1, 1 \\
        2, 2, 1, 1 \\
        2, 2, 2 \\
        1, 1, 1, 1, 1, 1
    \end{gather*}
    Calculamos para cada una de ellas el grupo correspondiente en descomposición cíclica primaria (no nos especifican una o la otra, luego elegimos la que queramos):
    \begin{equation*}
        \begin{array}{c|c}
            \text{Divisores elementales} & \text{Descomposición cíclica primaria}\\
            \hline
        2^6, 7 & C_{64} \oplus C_7 \\
        2^5, 2, 7 & C_{32} \oplus C_2 \oplus C_7 \\
        2^4, 2, 2, 7 & C_{16} \oplus C_2 \oplus C_2 \oplus C_7 \\
        2^4, 2^2, 7 & C_{16} \oplus C_4 \oplus C_7 \\
        2^3, 2, 2, 2, 7 & C_8 \oplus C_2 \oplus C_2 \oplus C_2 \oplus C_7 \\
        2^3, 2^2, 2, 7 & C_8 \oplus C_4 \oplus C_2 \oplus C_7 \\
        2^3, 2^3, 7 & C_8 \oplus C_8 \oplus C_7 \\
        2^2, 2, 2, 2, 2, 7 & C_4 \oplus C_2 \oplus C_2 \oplus C_2 \oplus C_2 \oplus C_7 \\
        2^2, 2^2, 2, 2, 7 & C_4 \oplus C_4 \oplus C_2 \oplus C_2 \oplus C_7 \\
        2^2, 2^2, 2^2, 7 & C_4 \oplus C_4 \oplus C_4 \oplus C_7 \\
        2, 2, 2, 2, 2, 2, 7 & C_2 \oplus C_2 \oplus C_2 \oplus C_2 \oplus C_2 \oplus C_2 \oplus C_7
        \end{array}
    \end{equation*}
    Si quitamos el grupo correspondiente a $\{2^3, 2^2, 2, 7\}$, tenemos todos los grupos no isomorfos a $T(A)$ de orden $|T(A)|$.

    Podemos hacernos más preguntas que sabemos responder sobre $A$, como:
    \begin{itemize}
        \item ¿Hay algún elemento de orden infinito en $A$?

            Sí, $(1, 0, 0, 0)$.
        \item ¿Hay algún elemento de orden 56?

            Sí, $(0, 0, 0, 1)$.
        \item ¿Hay algún elemento de orden 8?

            Sí, $(0, 0, 0, 7)$, o también $(0, 1, 1, 7)$.
    \end{itemize}
\end{ejemplo}



% // TODO: Ejercicios para ARTURO
% \begin{ejemplo} % // TODO: Ejeciccio 5
%     Sea $G$ un grupo abeliano de orden $n$ y $l(G)$ su longitud (la longitud de su serie de composición). Si la descomposición en factores primos de $n$ es:
%     \begin{equation*}
%         n = p_1^{r_1} \ldots p_r^{e_r}
%     \end{equation*}
%     Entonces:
%     \begin{equation*}
%         l(G) = e_1 + \ldots e_r
%     \end{equation*}
%     Y los factores de composición son:
%     \begin{equation*}
%         fact(G) = (C_{p_1}, \stackrel{e_1}{\ldots}, C_{p_1}, C_{p_2}\stackrel{e_2}{\ldots}, C_{p_2}, \ldots, C_{p_r} \stackrel{e_r}{\ldots}, C_{p_r})
%     \end{equation*}
%     Como:
%     \begin{equation*}
%         G \cong (C_{p_1^{\alpha_{11}}} \oplus \ldots \oplus C_{p_1^{\alpha_{1n_1}}}) \oplus \ldots \oplus (C_{p_r^{\alpha_{r1}}} \oplus \ldots \oplus C_{p_r^{\alpha_{rn_1}}})
%     \end{equation*}
%     Para:
%     \begin{gather*}
%         \begin{array}{c}
%         \alpha_{11}\geq \ldots \geq \alpha_{1n_1} \geq 1 \\
%         \vdots \\
%         \alpha_{r1}\geq \ldots \geq \alpha_{rn_1} \geq 1 
%         \end{array} \qquad 
%         \begin{array}{c}
%         \alpha_{11} + \ldots + \alpha_{1n_1} = e_1 \\
%         \vdots \\
%         \alpha_{r1} + \ldots + \alpha_{rn_1} = e_r 
%         \end{array}
%     \end{gather*}
%     Como $G$ es abeliano, los factores de composición son cíclicos.\\

%     \noindent
%     Para conseguir la serie de composición, lo que haremos será considerar la descomposición de $G$ en suma de grupos cíclicos y en cada paso, iremos quitando un grupo cíclico:
%     \begin{align*}
%         G_1 &= (C_{p_1^{\alpha_{12}}} \oplus \ldots \oplus C_{p_1^{\alpha_{1n_1}}}) \oplus \ldots \oplus (C_{p_r^{\alpha_{r1}}} \oplus \ldots \oplus C_{p_r^{\alpha_{rn_1}}}) \\
%         G_2 &= (C_{p_1^{\alpha_{13}}} \oplus \ldots \oplus C_{p_1^{\alpha_{1n_1}}}) \oplus \ldots \oplus (C_{p_r^{\alpha_{r1}}} \oplus \ldots \oplus C_{p_r^{\alpha_{rn_1}}}) \\
%             &\vdots \\
%         G_{n_1} &= (C_{p_2^{\alpha_{21}}} \oplus \ldots \oplus C_{p_2^{\alpha_{2n_2}}}) \oplus \ldots \oplus (C_{p_r^{\alpha_{r1}}} \oplus \ldots \oplus C_{p_r^{\alpha_{rn_1}}}) \\
%                 &\vdots
%     \end{align*}
% \end{ejemplo}

% \begin{ejemplo} % // TODO: EJercicio 6
%     Sea $A$ un grupo con $|A| = 40 = 2^3\cdot 5$:
%     \begin{equation*}
%         \begin{array}{c|c|c|c}
%                 \text{Descomposición} & \text{Descomp. cíclica primaria} & \text{Factores invariantes} & \text{Descomp. cíclica} \\
%                 \hline
%                 2^3, 5 & C_8 \oplus C_5 & 40 & C_{40} \\
%                 2, 2^2, 5 & C_2 \oplus C_4 \oplus C_5 & 2, 20 & C_2 \oplus C_{20} \\
%                 2, 2, 2, 5 & C_2 \oplus C_2 \oplus C_2 \oplus C_5 & 2, 2, 10 & C_2 \oplus C_2\oplus C_{10}
%         \end{array}
%     \end{equation*}
%     Como $l(A) = 4$ por el ejercicio anterior, series de composición serán:
%     \begin{align*}
%         &C_{40} \rhd C_{20} \rhd C_{10} \rhd C_5 \rhd \{1\} \\
%         &C_{40} \rhd C_{8} \rhd C_{4} \rhd C_2 \rhd \{1\} 
%     \end{align*}
%     Los factores de composición de la primera son:
%     \begin{equation*}
%         C_{40}/C_{20} \cong C_2 \qquad C_{20}/C_{10}\cong C_2 \qquad C_{10}/C_5\cong C_2 \qquad C_5/\{1\}\cong C_5
%     \end{equation*}
% \end{ejemplo}

% \begin{ejemplo} % // TODO: Ejecicio 11
%     Sea:    
%     \begin{equation*}
%         G = \langle a,b,c\mid \begin{array}{c}
%             2a - 6b + 18c = 0 \\
%             6a + 6c = 0
%         \end{array}\rangle 
%     \end{equation*}
%     Y sea:
%     \begin{equation*}
%         H = \mathbb{Z}^3/\langle (1,-9,3), (1,-7,1), (1,-1,1) \rangle 
%     \end{equation*}
%     Tenemos la matriz:
%     \begin{align*}
%         &\left(\begin{array}{ccc}
%             2 & -6 & 18 \\
%             6 & 0 & 6 
%         \end{array}\right)
%         \stackrel{}{\longrightarrow}
%         \left(\begin{array}{ccc}
%             2 & -6 & 18 \\
%             0 & 18 & -48 
%         \end{array}\right)
%         \stackrel{}{\longrightarrow}
%         \left(\begin{array}{ccc}
%             2 & 0 & 0 \\
%             0 & 18 & 48 
%         \end{array}\right)
%         \stackrel{}{\longrightarrow} \\
%         &\left(\begin{array}{ccc}
%             2 & 0 & 0 \\
%             0 & 18 & 6 
%         \end{array}\right)
%         \stackrel{}{\longrightarrow} 
%         \left(\begin{array}{ccc}
%             2 & 0 & 0 \\
%             0 & 6 & 0 
%         \end{array}\right)
%         \stackrel{}{\longrightarrow} \\
%     \end{align*}
%     Por lo que $G$ tiene rango 1 y sus descommposiciones cíclica y cíclica primaria son:
%     \begin{equation*}
%         G\cong \mathbb{Z}\oplus\mathbb{Z}_2\oplus\mathbb{Z}_6 \cong \mathbb{Z}\oplus\mathbb{Z}_2\oplus\mathbb{Z}_2\oplus\mathbb{Z}_3
%     \end{equation*}
%     Con $H$ tenemos lo mismo:
%     \begin{align*}
%         &\left(\begin{array}{ccc}
%             1 & 1 & 1 \\
%             -9 & -7 &-1 \\
%             3 & 1 & 1
%         \end{array}\right)
%         \stackrel{}{\longrightarrow}
%         \left(\begin{array}{ccc}
%             1 & 0 & 0 \\
%             -9 & 2 & 8 \\
%             3 & -2 & -2
%         \end{array}\right)
%         \stackrel{}{\longrightarrow}
%         \left(\begin{array}{ccc}
%             1 & 0 & 0 \\
%             0 & 2 & 8 \\
%             0 & -2 & -2
%         \end{array}\right)
%         \stackrel{}{\longrightarrow} \\
%         &\left(\begin{array}{ccc}
%             1 & 0 & 0 \\
%             0 & 2 & 8 \\
%             0 & 0 & 6
%         \end{array}\right)
%         \stackrel{}{\longrightarrow} 
%         \left(\begin{array}{ccc}
%             1 & 0 & 0 \\
%             0 & 2 & 0 \\
%             0 & 0 & 6
%         \end{array}\right)
%     \end{align*}
%     Por lo que:
%     \begin{equation*}
%         H\cong \mathbb{Z}_2 \oplus \mathbb{Z}_6
%     \end{equation*}
%     y $H$ no tiene parte libre. Tendremos:
%     \begin{equation*}
%         l(H) = 3
%     \end{equation*}
%     Los factores de composición serán $\mathbb{Z}_2, \mathbb{Z}_2, \mathbb{Z}_3$.
%     \begin{align*}
%         G&\not\cong H \\
%         T(G)&\cong T(H) = H
%     \end{align*}
%     ¿Cuáles son los elementos de orden 6 de $H$? Tiene al menos:
%     \begin{equation*}
%         O(a,1) = O(a,5) = 6 \qquad \forall a\in \mathbb{Z}_2
%     \end{equation*}
%     También tendremos:
%     \begin{equation*}
%         O(1,2) = \mcm(O(1), O(2)) = \mcm(2, 3) = 6
%     \end{equation*}
%     ¿$G$ tiene elementos de orden 6? Sí, los mismos pero con un 0 en primera coordenada.
% \end{ejemplo}
