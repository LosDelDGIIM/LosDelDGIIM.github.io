\chapter{Clasificación de grupos abelianos finitos}
\noindent
El objetivo final del tema es demostrar los teoremas de estructura de los grupos abelianos finitos, que permiten clasificar todos los grupos de este tipo según su orden. De esta forma, dado un grupo abeliano finito, la clsificación que realizaremos en este tema nos permitirá encontrar un grupo abeliano finito bien conocido al que el grupo dado sea isomorfo.\\

\noindent
Como toma de contacto, serán de especial relevancia dos resultados que ya vimos en Capítulos anteriores, como:

\begin{enumerate}
    \item En la Proposición~\ref{prop:carac_prod_finito_ciclicos} vimos que:
        \begin{equation*}
            C_n\oplus C_m \cong C_{nm} \Longleftrightarrow \mcd(n,m) = 1
        \end{equation*}
    \item En el Teorema~\ref{teo:prod_grupos_sylow} vimos que si $G$ es un grupo finito en el que todos sus subgrupos de Sylow son únicos, entonces $G$ es producto directo de todos ellos:
        \begin{equation*}
            G \cong P_1\oplus P_2 \oplus \ldots \oplus P_k
        \end{equation*}
\end{enumerate}
Como trabajaremos con subgrupos abelianos, recordamos que la notación que usábamos para el producto directo de grupos abelianos era $\oplus$.

\begin{teo}[Estructura de los $p-$grupos abelianos finitos]\label{teo:1_tema6}\ \\
    Sea $A$ un $p-$grupo abeliano finito con orden $|A| = p^n$ para $n\geq 1$, entonces existen enteros $\beta_1\geq \beta_2 \geq \ldots \geq \beta_t \geq 1$ de forma que:
    \begin{equation*}
        \beta_1 + \beta_2 + \ldots + \beta_t = n \quad \text{y} \quad A\cong C_{p^{\beta_1}} \oplus C_{p^{\beta_2}} \oplus \ldots \oplus C_{p^{\beta_t}}
    \end{equation*}
    Además, esta expresión es única, es decir, si existen $\alpha_1\geq \alpha_2\geq \ldots \geq \alpha_s \geq 1$ de forma que:
    \begin{equation*}
        \alpha_1 + \alpha_2 + \ldots + \alpha_s = n \quad \text{y} \quad A\cong C_{p^{\alpha_1}} \oplus C_{p^{\alpha_2}} \oplus \ldots \oplus C_{p^{\alpha_s}}
    \end{equation*}
    entonces $s = t$ y $\alpha_k = \beta_k$, para todo $k \in \{1,\ldots,t\}$.
    % \begin{proof} % // TODO: Copiar de Prado
    % \end{proof}
\end{teo}

\begin{observacion}
    Notemos que lo que estamos haciendo es tomar particiones de $n$ de la forma $\beta_i$, y este Teorema nos dice que el $p-$grupo puede escribirse de forma única salvo isomorfismos como producto de ciertos subgrupos cíclicos.\newline
    Es decir, existen tantos $p-$grupos abelianos de orden $p^n$ como particiones tengamos del número $n$, salvo isomorfismos.
\end{observacion}

\begin{ejemplo}
    Por ejemplo:
    \begin{itemize}
        \item Para saber los grupos abelianos finitos de orden $8 = 2^3$ que hay (salvo isomorfismos), calculamos cada una de las posibles particiones del número 3 (el exponente de $2$):
            \begin{align*}
                3& \longrightarrow A\cong C_8 \\
                2, 1& \longrightarrow A\cong C_4\oplus C_2 \\
                1, 1, 1& \longrightarrow A\cong C_2 \oplus C_2 \oplus C_2
            \end{align*}
        \item Para saber los grupos abelianos finitos de orden $81 = 3^4$, calculamos cada una de las particiones de $4$:
            \begin{align*}
                4 &\longrightarrow A\cong C_{81} \\
                3, 1 &\longrightarrow A\cong C_{27} \oplus C_3 \\
                2, 2 &\longrightarrow A \cong C_{9} \oplus C_9 \\
                2, 1, 1 &\longrightarrow A \cong C_9 \oplus C_3 \oplus C_3 \\
                1, 1, 1, 1 &\longrightarrow A \cong C_3 \oplus C_3 \oplus C_3 \oplus C_3 
            \end{align*}
    \end{itemize}
\end{ejemplo}

\begin{teo}[Estructura de los grupos abelianos finitos]\label{teo:2_tema6}\ \\
    Sea $A$ un grupo abeliano finito con $|A| = p_1^{\gamma_1}\ldots p_k^{\gamma_k}$ siendo $p_i$ primo $\forall i \in \{1,\ldots,k\}$, entonces:
    \begin{equation*}
        A \cong \bigoplus_{i=1}^k \left(\bigoplus_{j=1}^{t_i} C_{p_i^{n_{ij}}}\right)
    \end{equation*}
    Donde para cada $i \in \{1,\ldots,k\}$ tenemos:
    \begin{align*}
        n_{i1} \geq n_{i2} \geq \ldots \geq n_{it_{i}} \geq 1 \\
        n_{i1} + n_{i2} + \ldots + n_{it_{i}} = r_i
    \end{align*}
    Y la descomposición es única salvo el orden.\\

    \noindent
    Esta última recibe el nombre de descomposición cíclica primaria, y a los $p_i^{n_{ij}}$ con $i \in \{1,\ldots,k\}$ y $j \in \{1,\ldots t_i\}$ se les llama divisores elementales de $A$.\newline
    \begin{proof}
        Si $A$ es abeliano y finito, entonces todos sus $p-$subgrupos de Sylow son normales, luego podemos escribir:
        \begin{equation*}
            A = P_1 \oplus P_2 \oplus \ldots \oplus P_k
        \end{equation*}
        Siendo ${P_1, P_2, \ldots, P_k}$ el conjunto de todos sus $p-$subgrupos de Sylow, de forma que $|P_i| = p_i^{r_i}$, para todo $i \in \{1,\ldots,k\}$. Como cada $P_i$ es un $p_i-$subgrupo abeliano finito, aplicando el Teorema~\ref{teo:1_tema6}, podemos escribir:
        \begin{equation*}
            P_i = \bigoplus_{j=1}^{t_i} C_{p_i^{n_{ij}}} \qquad \forall i \in \{1,\ldots,k\}
        \end{equation*}
        De donde tenemos la expresión de la tesis.\newline
        A cada $P_i$ con $i \in \{1,\ldots,k\}$ lo llamaremos componente $p_i-$primaria de $A$.
    \end{proof}
\end{teo}

\begin{ejemplo}
    Si tenemos un subgrupo finito abeliano $A$ con $|A| = 360 = 2^3\cdot 3^2\cdot 5$, veamos los divisores elementales:
    \begin{equation*}
        \begin{array}{r|l}
            \text{Div. elementales} & \text{Descomp. cíclica primaria} \\
            \hline
            2^3\ 3^2\ 5 & C_8\oplus C_9 \oplus C_5 \\
            2^2\ 2\ 3^2\ 5 & C_4\oplus C_2 \oplus C_9 \oplus C_5\\
            2\ 2\ 2\ 3^2\ 5 & C_2 \oplus C_2 \oplus C_2 \oplus C_9 \oplus C_5\\
            2^3\ 3\ 3\ 5 & C_8\oplus C_3 \oplus C_3 \oplus C_5 \oplus C_8\\
            2\ 2^2\ 3\ 3\ 5 & C_2 \oplus C_4 \oplus C_3 \oplus C_3 \oplus C_5\\
            2\ 2\ 2\ 3\ 3\ 5 & C_2 \oplus C_2 \oplus C_2 \oplus C_3 \oplus C_3 \oplus C_5
        \end{array}
    \end{equation*}
    Serían todas las descomposiciones cíclicas primarias de $A$. Es decir, $A$ será isomorfo a cualquiera de esos.
\end{ejemplo}

\noindent
Sin embargo, si recordamos la Proposición~\ref{prop:carac_prod_finito_ciclicos}, esto nos llevará a la descomposición cíclica, donde observaremos por ejemplo que:
\begin{align*}
    C_8\oplus C_9 \oplus C_5 &\cong C_{360} \\
    C_4\oplus C_2 \oplus C_9 \oplus C_5 &\cong C_{180} \oplus C_2 \\
    C_2 \oplus C_2 \oplus C_2 \oplus C_9 \oplus C_5 &\cong C_{90} \oplus C_2 \oplus C_2 \\
    C_8\oplus C_3 \oplus C_3 \oplus C_5 \oplus C_8 &\cong C_{120} \oplus C_3 \\
    C_2 \oplus C_4 \oplus C_3 \oplus C_3 \oplus C_5 &\cong C_{60} \oplus C_6 \\
    C_2 \oplus C_2 \oplus C_2 \oplus C_3 \oplus C_3 \oplus C_5 &\cong C_{30} \oplus C_6 \oplus C_2
\end{align*}

Ahora, usaremos que:
\begin{equation*}
    C_n\times C_m\cong C_{nm} \Longleftrightarrow \mcd(n,m) = 1
\end{equation*}

\begin{teo}[Descomposición cíclica de un grupo abeliano finito]\label{teo:3_tema6}\ \\
    Si $A$ es un grupo abeliano finito, entonces:
    \begin{equation*}
        A\cong C_{d_1} \oplus C_{d_2} \oplus \ldots \oplus C_{d_t}
    \end{equation*}
    Donde los $d_i$ son enteros positivos de forma que:
    \begin{equation*}
        d_1d_2\ldots d_t = |A|
    \end{equation*}
    Y $d_i \mid d_j$ para cada $j \leq i$. Además, la descomposición es única salvo el orden, para cada partición.
    \begin{proof}
        Supuesto que $|A| = p_1^{r_1}\ldots p_k^{r_k}$, si usamos la descomposición que nos da el Teorema~\ref{teo:2_tema6}:
        \begin{equation*}
            A \cong \bigoplus_{i=1}^k \left(\bigoplus_{j=1}^{t_i} C_{p_i^{n_{ij}}}\right)
        \end{equation*}
        Para ciertos:
        \begin{align*}
            n_{i1} \geq n_{i2} \geq \ldots \geq n_{it_{i}} \geq 1 \\
            n_{i1} + n_{i2} + \ldots + n_{it_{i}} = r_i
        \end{align*}
        Sea $t = \max{t_1, t_2, \ldots, t_k}$, si $t_i < l \leq t$, tendremos entonces que $n_{il} = 0$.

        Lo que estamos haciendo es dada una partición, como por ejemplo la $\{2, 2^2, 3^2, 5\}$, denotar por $t_i$ al número de particiones de cada número y por $n_{ij}$ a los exponentes de cada una de las particiones, construyendo la tabla:
        \begin{equation*}
            \left\{\begin{array}{c|cc}
                    t_1 & n_{11} & n_{12} \\
                    t_2 & n_{21} & n_{22} \\
                    t_3 & n_{31} & n_{32} 
            \end{array}\right.
        \end{equation*}
        De esta forma, tenemos:
        \begin{equation*}
            \left(\begin{array}{cccc}
                p_1^{n_{11}} & p_2^{n_{21}} & \ldots & p_k^{n_{k1}} \\
                p_1^{n_{12}} & p_2^{n_{22}} & \ldots  & p_k^{n_{k2}}  \\
                 \vdots & \vdots & \ddots & \vdots \\
                 p_1^{n_{1k}} & p_2^{n_{2k}} & \ldots & p_k^{n_{kk}}  
            \end{array}\right)
        \end{equation*}
        Y $A$ es la suma directa de los cíclicos con órdenas las entradas de las columnas. 
        Si tomamos el producto por columnas obtenemos la cíclica primaria y si la hacemos por filas la que estamos interesados:
        \begin{align*}
            d_1 &= p_1^{n_{11}}p_2^{n_{21}} \ldots p_k^{n_{k1}} \\
                &\vdots \\
            d_t &= p_1^{n_{1t}}p_2^{n_{2t}} \ldots p_k^{n_{kt}} 
        \end{align*}
        Efectivamente, tendremos que:
        \begin{equation*}
            d_1d_2 \ldots d_t = p_1^{r_1} p_2^{r_2} \ldots p_k^{r_k} = |A|
        \end{equation*}
        Como $n_{ij}\geq n_{ij+1}$, tendremos entonces que $d_i \mid d_j$, para todo $j\leq i$. Además, tendremos que:
        \begin{align*}
            C_{d_1} &\cong C_{p_1^{n_{11}}} \oplus C_{p_2^{n_{21}}} \oplus \ldots \oplus C_{p_k^{n_{k1}}} \\
                    &\vdots \\
            C_{d_t} &\cong C_{p_1^{n_{1t}}} \oplus C_{p_2^{n_{2t}}} \oplus \ldots \oplus C_{p_k^{n_{kt}}} 
        \end{align*}
        De donde $A \cong C_{d_1}\oplus C_{d_2}\oplus \ldots \oplus C_{d_t}$. La unicidad viene de la unicidad dada por la descomposición del Teorema~\ref{teo:2_tema6}.
    \end{proof} 
    Los $d_i$ reciben el nombre de \underline{factores invariantes}.
\end{teo}

\begin{ejemplo}
    Sea $A$ un grupo abeliano finito con $|A| = 360 = 2^3\cdot 3^2\cdot 5$:
    \begin{itemize}
        \item Para la partición $\{2^3, 3^2, 5\}$, tenemos que:
            \begin{equation*}
                A\cong C_8\oplus C_9\oplus C_5 
            \end{equation*}
            Los factores invariantes serán:
            \begin{align*}
                d_1 = 2^3\cdot 3^2\cdot 5
            \end{align*}
            Por lo que la descomposición cíclica será $A\cong C_{360}$.
        \item Para la partición $\{2^2, 2, 3^2, 5\}$, la descomposición cíclica primaria fue:
            \begin{equation*}
                A \cong C_4 \oplus C_2 \oplus C_9 \oplus C_5
            \end{equation*}
            En este caso, tendremos $t = \max\{2, 1, 1\} = 2$, por lo que tendremos dos factores invariantes:
            \begin{equation*}
                \left(\begin{array}{ccc}
                    2^2 & 3^2 & 5 \\
                     2 & 1 & 1
                \end{array}\right)
            \end{equation*}
            Por lo que tendremos (los productos de las filas):
            \begin{align*}
                d_1 &= 2^2 \cdot 3^2 \cdot 5 = 180 \\
                d_2 &= 2\cdot 1\cdot 1 = 2
            \end{align*}
            Y la descomposición cíclica es:
            \begin{equation*}
                A\cong C_{180}\oplus C_2
            \end{equation*}
        \item Para la descomposición $\{2,2,2,3^2,5\}$, tenemos:
            \begin{equation*}
                A\cong C_2 \oplus C_2 \oplus C_2 \oplus C_9 \oplus C_5
            \end{equation*}
            Y tendremos $t=3$:
            \begin{equation*}
                \left(\begin{array}{cccc}
                        d_1 =& 2 & 3^2 & 5 \\
                        d_2 =&2 & 1 & 1 \\
                        d_3 =&2 & 1 & 1
                \end{array}\right)
            \end{equation*}
            Por lo que:
            \begin{equation*}
                A\cong C_{90} \oplus C_2 \oplus C_2
            \end{equation*}
        \item Para $\{2^3, 3,3,5\}$:
            \begin{equation*}
                A\cong C_8\oplus C_3\oplus C_3\oplus C_5
            \end{equation*}
            Y tenemos:
            \begin{equation*}
                \left(\begin{array}{ccc}
                    2^3 & 3 & 5 \\
                    1 & 3 & 1
                \end{array}\right)
            \end{equation*}
            Y la descomposición cíclica será:
            \begin{equation*}
                A\cong C_{120} \oplus C_3
            \end{equation*}
        \item Para $\{2^2,2,3,3,5\}$:
            \begin{equation*}
                A\cong C_4\oplus C_2\oplus C_3 \oplus C_3 \oplus C_5
            \end{equation*}
            Y tenemos:
            \begin{equation*}
                \left(\begin{array}{ccc}
                    2^2 & 3 & 5 \\
                    2 & 3 & 1
                \end{array}\right)
            \end{equation*}
            Por lo que tenemos la descomposición cíclica:
            \begin{equation*}
                A\cong C_{60} \oplus C_6
            \end{equation*}
        \item Para $\{2, 2, 2, 3, 3, 5\}$:
            \begin{equation*}
                A\cong C_2\oplus C_2\oplus C_2\oplus C_3\oplus C_3\oplus C_5
            \end{equation*}
            Tenemos:
            \begin{equation*}
                \left(\begin{array}{ccc}
                    2 & 3 & 5 \\
                    2 & 3 & 1 \\
                    2 & 1 & 1
                \end{array}\right)
            \end{equation*}
            Y:
            \begin{equation*}
                A\cong C_{30} \oplus C_6 \oplus C_2
            \end{equation*}
    \end{itemize}
\end{ejemplo}

En el caso particular de que todos los primos tengan exponente 1:

\begin{coro}
    Si $A$ es un grupo abeliano finito con $|A| = p_1p_2 \ldots p_k = n$, entonces salvo isomorfismo, el único grupo abeliano de orden $n$ es el cíclico $C_n$.
    \begin{proof}
        Utilizando el Teorema~\ref{teo:2_tema6}, podemos escribir:
        \begin{equation*}
            A \cong C_{p_1} \oplus C_{p_2} \oplus \ldots \oplus C_{p_k}
        \end{equation*}
        Y como $\mcd(p_i, p_j) = 1$ para cada $i,j\in \{1,\ldots,k\}$ con $i\neq j$, tenemos que:
        \begin{equation*}
            C_{p_1} \oplus C_{p_2} \oplus \ldots \oplus C_{p_k} = C_{p_1p_2\ldots p_k} = C_n
        \end{equation*}
    \end{proof}
\end{coro}

\begin{ejemplo}
    Sea $A$ un grupo abeliano finito con $|A| = 180 = 2^2\cdot 3^2\cdot 5$, buscamos clasificarlo según la descomposición cíclica:
    \begin{equation*}
        \begin{array}{c|c|c|c}
            \text{Descomposición} & \text{desc. cíclica primaria} & \text{factores invariantes} & \text{desc. cíclica} \\
            \hline
            \{2^2, 3^2, 5\} & C_4\oplus C_9 \oplus C_5 & 2^2\cdot 3^2\cdot 5 = 180 & C_{180} \\
            \hline
            \{2,2,3^2,5\} & C_2\oplus C_2\oplus C_9\oplus C_5 & \begin{array}{c}
                    d_1 = 2\cdot 9\cdot 5 = 90 \\
                    d_2 = 2
            \end{array}& C_{90}\oplus C_2 \\
            \hline
            \{2^3, 3, 3, 5\} & C_4\oplus C_3\oplus C_3\oplus C_5 & \begin{array}{c}
                    d_1 = 2^2\cdot 3\cdot 5 = 60 \\
                    d_2 = 3
            \end{array}& C_{60} \oplus C_3 \\
            \hline
                    \{2,2,3,3,5\} & C_2\oplus C_2\oplus C_3\oplus C_3\oplus C_5 & \begin{array}{c}
                            d_1 = 2\cdot 3\cdot 5 = 30 \\
                            d_2 = 2\cdot 3 = 6
                        \end{array} & C_{30} \oplus C_6
        \end{array}
    \end{equation*}
\end{ejemplo}

\begin{ejemplo} % // TODO: EJercicio 1
    Listar los órdenes de todos los elementos de un grupo de orden 8.
    Sea $A$ un grupo abeliano finito de orden $8$, entonces lo podemos clasificar en:
    \begin{itemize}
        \item $C_8$:
            \begin{itemize}
                \item Los elementos $\{1,3,5,7\}$ tienen orden 8.
                \item $O(0) = 1$.
                \item $O(2) = \nicefrac{8}{\mcd(2,8)} = 4 = O(6)$.
                \item $O(4) = 2$.
            \end{itemize}
        \item $C_4\oplus C_2$, aplicamos que $O(a,b) = \mcm(O(a), O(b))$:
            Como los órdenes de los elementos en $C_4$ son: $\{1,2,4\}$ y en $C_2$ son $\{1,2\}$, las posibilidades son: $\{1,2,4\}$:
            \begin{itemize}
                \item $O(0,0) = 1$.
                \item $O(0,1) = 2$.
                \item $O(1,b) = 4 = O(3,b)$, $\forall b\in C_2$
                \item $O(2,b) = 2 $, $\forall b\in C_2$.
            \end{itemize}
        \item $C_2\oplus C_2\oplus C_2$, los órdenes son $\{1,2\}$ y todos tienen orden 2 salvo el elemento $(0,0,0)$, que tiene orden 1.
    \end{itemize}
\end{ejemplo}

\begin{ejemplo}
    Listar los ódenes de todos los elementos de un grupo abeliano $A$ de orden 12.\\

    \noindent
    Sea $A$ con $|A| = 12 = 2^2\cdot 3$, tenemos $A\cong \mathbb{Z}_{12}$ o $A\cong \mathbb{Z}_6\oplus \mathbb{Z}_2$.
    \begin{itemize}
        \item En $\mathbb{Z}_{12}$:
            \begin{itemize}
                \item $U(\mathbb{Z}_{12}) = \{1,5,7,11\}$.
                \item $O(2) = 6$.
                \item $O(3) = 4 = O(9)$.
                \item $O(4) = 3 = O(8)$.
                \item $O(6) = 2$.
            \end{itemize}
        \item En $\mathbb{Z}_6\oplus\mathbb{Z}_2$:
            \begin{equation*}
                O(a,b) = \mcm(Div(6), Div(2)) = \mcm(\{1,2,3,6\}, \{1,2\}) = \{1,2,3,6\}
            \end{equation*}
            El orden de los elementos de $\mathbb{Z}_6$ son:
            \begin{itemize}
                \item $U(\mathbb{Z}_6) = \{1,5\}$, luego $O(1) = O(5) = 6$.
                \item $O(2) = 3 = O(4)$.
                \item $O(3) = 2$.
                \item $O(0) = 1$.
            \end{itemize}
            Ahora:
            \begin{itemize}
                \item $O(0,0) = 1$.
                \item $O(1,b) = O(5,b) = 6$ $\forall b\in \mathbb{Z}_2$.
                \item $O(3, b) = 2$ $\forall b\in \mathbb{Z}_2$.
                \item $O(2, 0) = O(4, 0) = 3$.
                \item $O(2, 1) = O(4, 1) = 6$.
            \end{itemize}
    \end{itemize}
\end{ejemplo}

\section{Clasificación de grupos abelianos no finitos}
\noindent
Buscamos hayar la descomposición cíclica y la descomposición cíclica primaria de dos grupos cualesquiera. Para ello, recordamos varias definiciones que ya vimos.

\begin{notacion}
    Como trabajaremos con grupos abelianos finitos, usaremos la notación aditiva.
\end{notacion}

\begin{definicion}
    Un grupo abeliano $A$ se dice que es finitamente generado si existe un conjunto:
    \begin{equation*}
        X = \{x_1,\ldots,x_r\} \subseteq A
    \end{equation*}
    De forma que para todo $a\in A$, existirán $\lm_1, \ldots, \lm_r \in \mathbb{Z}$ de forma que:
    \begin{equation*}
        a = \sum_{k=1}^{r} \lm_k x_k
    \end{equation*}
    En dicho caso, diremos que $X$ es un \underline{sistema de generadores de A}, y notaremos:
    \begin{equation*}
        A = \langle x_1, \ldots, x_r \rangle 
    \end{equation*}
\end{definicion}

\begin{definicion}[Base]
Sea $A$ un grupo abeliano, un conjunto de generadores $X = \{x_1,\ldots,x_r\}$ de $A$ es una \underline{base} si son $\mathbb{Z}-$linealmente independientes.\\ % // TODO: Meter def

\noindent
En dicho caso $A$ es un \underline{grupo abeliano libre de rango $r$}.
\end{definicion}

% Un grupo finito no puede tener bases por la independencia !!!

\begin{observacion}
    Observemos que si $A$ es un grupo abeliano libre de rango $r$, entonces tendremos que:
    \begin{equation*}
        A \cong \mathbb{Z}^r
    \end{equation*}
    Además, si $H < A$, tendremos entonces que $H\cong \mathbb{Z}^s$, para cierta $s\leq r$.
\end{observacion}

\noindent
De esta forma, si $A$ es un grupo finitamente generado, podemos descomponerlo en:
\begin{equation*}
    A\cong F \oplus T(A)
\end{equation*}
Que será la \underline{descomposición estándar} de $A$. $F$ será un grupo abeliano libre de rango finito y:
\begin{equation*}
    T(A) = \{a\in A \mid O(a) < +\infty\}
\end{equation*}
Que recibe el nombre de \underline{subgrupo de torsión de $A$}. % // TODO: Meter defs y props

\begin{prop}
    El subgrupo de torsión de un grupo es un grupo abeliano finito.
\end{prop}
De esta forma, existirán $r\geq 0$ y $d_1,\ldots,d_s$ con $d_i\mid d_j$ con $j\leq i$ de forma que:
\begin{equation*}
    d_1d_2\ldots d_s = |T(A)|
\end{equation*}
Por lo que:
\begin{equation*}
    A \cong \mathbb{Z}^r \oplus \mathbb{Z}_{d_1} \oplus \mathbb{Z}_{d_2} \oplus \ldots \oplus \mathbb{Z}_{d_s}
\end{equation*}

\begin{itemize}
    \item Llamaremos $r$ al rango de $A$.
    \item A los $d_i$ los llamaremos factores invariantes de $A$.
\end{itemize}

\begin{ejemplo}
    Si tomamos:
    \begin{equation*}
        A = \langle x,y,z \mid x^3=y^4, x^2z = z^{-1}y, xy=yx, xz=zx, yz=zy \rangle 
    \end{equation*}
    Si lo escribimos en notación aditiva:
    \begin{equation*}
        A = \langle x,y,z \mid 3x=4y, 2x+z = y-z, x+y=y+x, x+z = z+x, y+z=z+y\rangle 
    \end{equation*}
    Si nos olvidamos de las últimas y pensamos que el grupo es abeliano, así como despejando:
    \begin{equation*}
        A = \langle x,y,z \mid 3x-4y = 0, 2x-y+2z = 0 \rangle 
    \end{equation*}
    Y tenemos el sistema:
    \begin{equation*}
        M = \left(\begin{array}{ccc}
            3 & -4 & 0 \\
            2 & -1 & 2
        \end{array}\right)
    \end{equation*}
    Tenemos 3 incógnitas y $rg(M) = 2$, un Sistema Compatible Indeterminado, con un parámetro libre. Veremos que transformaremos $M$ en:
    \begin{equation*}
        \left(\begin{array}{ccc}
            3 & -4 & 0 \\
            2 & -1 & 2
        \end{array}\right) \sim \left(\begin{array}{ccc}
            1 & 0 & 0 \\
            0 & 2 & 0
        \end{array}\right)
    \end{equation*}
    Que es la \underline{forma normal de Smith} (parecido a Hermite pero en $\mathbb{Z}$). De esta forma, tendremos que:
    \begin{equation*}
        A\cong \mathbb{Z} \oplus \{0\} \oplus \mathbb{Z}_2 \cong \mathbb{Z} \oplus \mathbb{Z}_2
    \end{equation*} % // TODO: Se explicará la absorción del Z
\end{ejemplo}

% // TODO: Esto es otra clase

Sea:
\begin{equation*}
    A = \left\langle x_1,x_2,\ldots, x_n \mid \begin{array}{c}
        a_{11}x_1 + a_{12}x_2 + \ldots + a_{1n} x_n = 0 \\
        \vdots \\
        a_{m1}x_1 + a_{m2}x_2 + \ldots + a_{mn} x_n = 0 
    \end{array}\right\rangle 
\end{equation*}
con $n$ generadores y $m\leq n$ relaciones siendo. Sea:
\begin{equation*}
    X = \{e_1, \ldots, e_n\}
\end{equation*}
Consideramos $F = \langle X \rangle $, que será $F\cong \mathbb{Z}^r$. La descomposición estándar de $A$ será:
\begin{equation*}
    A = F + T(A)
\end{equation*}
Si definimos:
\Func{\varphi}{F}{A}{e_i}{x_i}
Tenemos que $\varphi$ está bien definida, así como que es sobreyectiva. Tendremos:
\begin{equation*}
    \ker(\varphi) < F
\end{equation*}
Por lo que:
\begin{equation*}
    \ker(\varphi) \cong \mathbb{Z}^m
\end{equation*}
De esta forma, si $\{y_1,\ldots,y_m\}$ es una base de $\ker(\varphi)$, cumplirá que (basta aplicar $\varphi$):
\begin{equation*}
    \left\{\begin{array}{rl}
        a_{11}e_1 + a_{12}e_2 + \ldots + a_{1n} e_n &= y_1 \\
        \vdots& \\
        a_{m1}e_1 + a_{m2}e_2 + \ldots + a_{mn} e_n &= y_m 
    \end{array}\right.
\end{equation*}
De esta forma, tenemos:
\begin{equation*}
    \ker(\varphi) \stackrel{i}{\longrightarrow} F \rightarrow A
\end{equation*}
De forma que la matriz:
\begin{equation*}
    \left(\begin{array}{cccc}
        a_{11} & a_{12} & \ldots & a_{1n} \\
        \vdots & \vdots & \ddots & \vdots \\
        a_{m1} & a_{m2} & \ldots & a_{mn} 
    \end{array}\right)
\end{equation*}
A la que llamaremos matrices de relaciones del grupo, que nos lleva el vector $(y_1, \ldots, y_m)$ en $(x_1, \ldots, x_n)$, tras multiplicar por $(e_1, \ldots, e_n)$, y tendremos aplicando Teoremas de Isomorfía que:
\begin{equation*}
    A\cong F/\ker(\varphi)
\end{equation*}
Esta matriz la convertiremos en la forma normal de Smith.\\

\noindent
Como los factores invariantes eran productos de primos, no nos podrá salir ningún 1, por lo que esos unos los eliminaremos, ya que como factores invariantes han de ser mayor que 1.

\begin{ejemplo}
    En $A = \mathbb{Z} \oplus \mathbb{Z}_2$, una base para $\mathbb{Z}$ es:
    \begin{equation*}
        X = \{1\}
    \end{equation*}
    Y un sistema de generadores para $\mathbb{Z}\oplus\mathbb{Z}_2$ es:
    \begin{equation*}
        \{(1,1)\}
    \end{equation*}
\end{ejemplo}

\subsection{Forma Normal de Smith de una matriz}
% // TODO: Recordar qué era la forma normal de Hermite
\begin{ejemplo}
    Una forma de Hermite por filas es:
    \begin{equation*}
        \left(\begin{array}{ccc}
            1 & 0 \\
            0 & 1 \\
            0 & 0  
        \end{array}\right)
    \end{equation*}
    \begin{equation*}
        \left(\begin{array}{cccc}
            1 & 0 & 0 & 0 \\
            0 & 0 & 1 & 0 \\
            0 & 0 & 0 & 0
        \end{array}\right)
    \end{equation*}
\end{ejemplo}

Las operaciones elementamos sobre matrices eran:
\begin{itemize}
    \item Intercambiar filas.
    \item Mutiplicar una fila por un número.
    \item Sumar un múltiplo de una fila a otra.
\end{itemize}
Al hacer la forma normal de Smith podemos encontrar dos matrices $P$ y $Q$ regulares de forma que:
\begin{equation*}
    PAQ = \left(\begin{array}{ccccc}
            d_1 &  &  & & \\
                & d_2 &  & & \\
                &  & \ddots & & \\
                &  &  &  d_s & \\
                &  &  &   &  0\\
    \end{array}\right)
\end{equation*}
De forma que $d_i \mid d_{i+1}$. $P$ contenía las transformaciones elementales por filas y $Q$ por columnas.

\begin{ejemplo} % // TODO: Ejercicio 7 a)
    Si consideramos:
    \begin{equation*}
        \left(\begin{array}{ccc}
            0 & 2 & 0 \\
            -6 & -4 & -6 \\
            6 & 6 & 6 \\
            7 & 10 & 6 
        \end{array}\right)
    \end{equation*}
    Como $\mcd$ de todos los elementos es 1, tenemos que poner un 1 arriba (consejo: no hacer ceros hasta poner un 1). Para ello, hacemos la cuarta fila más la segunda:
    \begin{equation*}
        \left(\begin{array}{ccc}
            0 & 2 & 0 \\
            -6 & -4 & -6 \\
            6 & 6 & 6 \\
            1 & 6 & 0 
        \end{array}\right)
    \end{equation*}
    Si nos la llevamos a la primera posición:
    \begin{equation*}
        \left(\begin{array}{ccc}
            1 & 6 & 0 \\
            -6 & -4 & -6 \\
            6 & 6 & 6 \\
            0 & 2 & 0 
        \end{array}\right)
    \end{equation*}
    Ahora, hacemos ceros en la primera fila, salvo el 1. Restamos a la primera la cuarta multiplicada por 3:
    \begin{equation*}
        \left(\begin{array}{ccc}
            1 & 0 & 0 \\
            -6 & -4 & -6 \\
            6 & 6 & 6  \\
            0 & 2 & 0 
        \end{array}\right)
    \end{equation*}
    \begin{equation*}
        \left(\begin{array}{ccc}
            1 & 0 & 0 \\
            0 & -4 & -6 \\
            0 & 6 & 6  \\
            0 & 2 & 0 
        \end{array}\right)
    \end{equation*}
    Como el $\mcd$ es 2, hay que sacar un 2 en la posición $2,2$. Para ello, intercambiamos las filas segunda y cuarta:
    \begin{equation*}
        \left(\begin{array}{ccc}
            1 & 0 & 0 \\
            0 & 2 & 0 \\
            0 & 6 & 6  \\
            0 & -4 & -6 
        \end{array}\right)
    \end{equation*}
    Hacemos ceros:
    \begin{equation*}
        \left(\begin{array}{ccc}
            1 & 0 & 0 \\
            0 & 2 & 0 \\
            0 & 0 & 6  \\
            0 & 0 & 0
        \end{array}\right)
    \end{equation*}
    Como el $\mcd$ es 6 y tenemos un 6, hemos terminado. Hemos conseguido la forma normal de Smith.
\end{ejemplo}

\begin{ejemplo}
    Calcular el rango de $A$ y todos los grupos abelianos no isomorfos de orden igual que la torsión.
    \begin{equation*}
        A = \left\langle x,y,z,t \mid \begin{array}{c}
            14x + 4y + 4z + 14t = 0 \\
            -6x + 4y + 4z + 10t = 0 \\
            -16x -4y -4z -20t = 0
        \end{array}\right\rangle 
    \end{equation*}
    Calculamos la forma normal de Smith de la matriz:
    \begin{align*}
        &\left(\begin{array}{cccc}
            14 & 4 & 4 & 14 \\
            -6 & 4 & 4 & 10 \\
            -16 & -4 & -4 & -20 
        \end{array}\right) 
        \stackrel{F_1' = -(F_1 + F_3)}{\longrightarrow}
        \left(\begin{array}{cccc}
            2 & 0 & 0 & 6 \\
            -6 & 4 & 4 & 10 \\
            -16 & -4 & -4 & -20 
        \end{array}\right)  
        \stackrel{C_4 - 3C_1}{\longrightarrow} \\
        &\left(\begin{array}{cccc}
            2 & 0 & 0 & 0 \\
            -6 & 4 & 4 & 28 \\
            -16 & -4 & -4 & 28
        \end{array}\right) 
        \stackrel{\substack{F_2 + 3F_1\\F_3 + 8F_1}}{\longrightarrow}
        \left(\begin{array}{cccc}
            2 & 0 & 0 & 0 \\
            0 & 4 & 4 & 28 \\
            0 & -4 & -4 & 28
        \end{array}\right)  
        \stackrel{F_2 + F_3}{\longrightarrow} \\
        &\left(\begin{array}{cccc}
            2 & 0 & 0 & 0 \\
            0 & 0 & 0 & 56 \\
            0 & -4 & -4 & 28
        \end{array}\right) 
        \stackrel{F_2\leftrightarrow F_3}{\longrightarrow}
        \left(\begin{array}{cccc}
            2 & 0 & 0 & 0 \\
            0 & -4 & -4 & 28 \\
            0 & 0 & 0 & 56 
        \end{array}\right)  
        \stackrel{F_2' = -F_2}{\longrightarrow} 
        \left(\begin{array}{cccc}
            2 & 0 & 0 & 0 \\
            0 & 4 & 4 & -28 \\
            0 & 0 & 0 & 56 
        \end{array}\right)  
        \stackrel{C_3 + 7C_2}{\longrightarrow} \\
        & \left(\begin{array}{cccc}
            2 & 0 & 0 & 0 \\
            0 & 4 & 4 & 0 \\
            0 & 0 & 0 & 56 
        \end{array}\right)   
        \stackrel{C_3 - C_2}{\longrightarrow} 
        \left(\begin{array}{cccc}
            2 & 0 & 0 & 0 \\
            0 & 4 & 0 & 0 \\
            0 & 0 & 0 & 56 
        \end{array}\right)   
        \stackrel{C_3 \leftrightarrow C_4}{\longrightarrow}
        \left(\begin{array}{cccc}
            2 & 0 & 0 & 0 \\
            0 & 4 & 0 & 0 \\
            0 & 0 & 56 & 0 
        \end{array}\right)   
    \end{align*}
    Por tanto, el rango de $A$ es (el número de incógnitas menos el rango de la matriz) 3. Ahora, la descomposición cíclica sería:
    \begin{equation*}
        T(A) \cong \mathbb{Z}_2\oplus \mathbb{Z}_4 \oplus \mathbb{Z}_{56}
    \end{equation*}
    Y la descomposición cíclica primaria:
    \begin{equation*}
        T(A) \cong \mathbb{Z}_2 \oplus \mathbb{Z}_4 \oplus \mathbb{Z}_8 \oplus \mathbb{Z}_7
    \end{equation*}
    Y tendremos:
    \begin{equation*}
        A \cong \mathbb{Z} \oplus \mathbb{Z}_2\oplus \mathbb{Z}_4 \oplus \mathbb{Z}_{56} \cong \mathbb{Z} \oplus \mathbb{Z}_2 \oplus \mathbb{Z}_4 \oplus \mathbb{Z}_8 \oplus \mathbb{Z}_7
    \end{equation*}
    Para ello, buscamos los grupos $G$ con orden $2\cdot 4\cdot 8\cdot 7 = 448 = 2^{6}\cdot 7$ y descartamos el isomorfo a $T(A)$:
    \begin{equation*}
        \begin{array}{c|c}
            \text{Divisores elementales} & \text{Factores invariantes} \\
            \hline
            2^6, 7 & 448 \\
            2, 2^5, 7 & 2, 224 \\
            2, 2, 2^4, 7 & 2, 2, 112\\
            2, 2, 2, 2, 2^3, 7 & 2, 2, 2, 56\\
            2, 2, 2, 2, 2, 2^2, 7 & 2, 2, 2, 2, 28 \\
            2, 2, 2, 2, 2, 2, 2, 7 & 2, 2, 2, 2, 2, 14 \\
            2, 2^2, 2^3, 7 & 2, 4, 56\\
            2^2, 2^2, 2^2, 7 & 4, 4, 28 \\
            2^3, 2^3, 7 & 8, 56 \\
            2^2, 2^4, 7 & 4, 112\\
            2, 2, 2^2, 2^2  & 2, 2, 4, 28
        \end{array}
    \end{equation*}
    ¿Hay algún elemento de orden infinito en $A$? Sí:
    \begin{equation*}
        (1, 0, 0, 0)
    \end{equation*}
    ¿Hay algún elemento de orden 56? Sí:
    \begin{equation*}
        (0, 0, 0, 1)
    \end{equation*}
    ¿Hay algún elemento de orden 8? Sí:
    \begin{equation*}
        (0, 0, 0, 7)
    \end{equation*}
    O también:
    \begin{equation*}
        (0, 1, 1, 7)
    \end{equation*}
\end{ejemplo}

\begin{ejemplo}
    Forma normal de Smith de:
    \begin{equation*}
        \left(\begin{array}{ccc}
            4 & 0 & 0 \\
            0 & 6 & 0 \\
            0 & 0 & 8
        \end{array}\right)
    \end{equation*}
    Como no está en forma normal, hemos de añadir elementos para poder hayar el 2:
    \begin{equation*}
        2 = \mcd(4,6,8)
    \end{equation*}
    \begin{align*}
        &\left(\begin{array}{ccc}
            4 & 0 & 0 \\
            0 & 6 & 0 \\
            0 & 0 & 8
        \end{array}\right)
        \stackrel{F_2 + F_1}{\longrightarrow}
        \left(\begin{array}{ccc}
            4 & 0 & 0 \\
            4 & 6 & 0 \\
            0 & 0 & 8
        \end{array}\right)
        \stackrel{C_2 - C_1}{\longrightarrow}
        \left(\begin{array}{ccc}
            4 & -4 & 0 \\
            4 & 2 & 0 \\
            0 & 0 & 8
        \end{array}\right)
        \stackrel{}{\longrightarrow} \\
        &\left(\begin{array}{ccc}
            -4 & 4 & 0 \\
            4 & 2 & 0 \\
            0 & 0 & 8
        \end{array}\right)
        \stackrel{}{\longrightarrow} 
        \left(\begin{array}{ccc}
            2 & 4 & 0 \\
            -4 & 4 & 0 \\
            0 & 0 & 8
        \end{array}\right)
        \stackrel{}{\longrightarrow} 
        \left(\begin{array}{ccc}
            2 & 4 & 0 \\
            0 & 12 & 0 \\
            0 & 0 & 8
        \end{array}\right) 
        \stackrel{}{\longrightarrow}  \\
        &\left(\begin{array}{ccc}
            2 & 0 & 0 \\
            0 & 12 & 0 \\
            0 & 0 & 8
        \end{array}\right) 
        \stackrel{}{\longrightarrow}  
        \left(\begin{array}{ccc}
            2 & 0 & 0 \\
            0 & 12 & 8 \\
            0 & 0 & 8
        \end{array}\right) 
        \stackrel{}{\longrightarrow}  
        \left(\begin{array}{ccc}
            2 & 0 & 0 \\
            0 & 4 & 8 \\
            0 & -8 & 8
        \end{array}\right) 
        \stackrel{}{\longrightarrow}   \\
        &\left(\begin{array}{ccc}
            2 & 0 & 0 \\
            0 & 4 & 8 \\
            0 & 0 & 24
        \end{array}\right) 
        \stackrel{}{\longrightarrow}   
        \left(\begin{array}{ccc}
            2 & 0 & 0 \\
            0 & 4 & 0 \\
            0 & 0 & 24
        \end{array}\right) 
    \end{align*}
\end{ejemplo}

\begin{ejemplo} % // TODO: Ejeciccio 5
    Sea $G$ un grupo abeliano de orden $n$ y $l(G)$ su longitud (la longitud de su serie de composición). Si la descomposición en factores primos de $n$ es:
    \begin{equation*}
        n = p_1^{r_1} \ldots p_r^{e_r}
    \end{equation*}
    Entonces:
    \begin{equation*}
        l(G) = e_1 + \ldots e_r
    \end{equation*}
    Y los factores de composición son:
    \begin{equation*}
        fact(G) = (C_{p_1}, \stackrel{e_1}{\ldots}, C_{p_1}, C_{p_2}\stackrel{e_2}{\ldots}, C_{p_2}, \ldots, C_{p_r} \stackrel{e_r}{\ldots}, C_{p_r})
    \end{equation*}
    Como:
    \begin{equation*}
        G \cong (C_{p_1^{\alpha_{11}}} \oplus \ldots \oplus C_{p_1^{\alpha_{1n_1}}}) \oplus \ldots \oplus (C_{p_r^{\alpha_{r1}}} \oplus \ldots \oplus C_{p_r^{\alpha_{rn_1}}})
    \end{equation*}
    Para:
    \begin{gather*}
        \begin{array}{c}
        \alpha_{11}\geq \ldots \geq \alpha_{1n_1} \geq 1 \\
        \vdots \\
        \alpha_{r1}\geq \ldots \geq \alpha_{rn_1} \geq 1 
        \end{array} \qquad 
        \begin{array}{c}
        \alpha_{11} + \ldots + \alpha_{1n_1} = e_1 \\
        \vdots \\
        \alpha_{r1} + \ldots + \alpha_{rn_1} = e_r 
        \end{array}
    \end{gather*}
    Como $G$ es abeliano, los factores de composición son cíclicos.\\

    \noindent
    Para conseguir la serie de composición, lo que haremos será considerar la descomposición de $G$ en suma de grupos cíclicos y en cada paso, iremos quitando un grupo cíclico:
    \begin{align*}
        G_1 &= (C_{p_1^{\alpha_{12}}} \oplus \ldots \oplus C_{p_1^{\alpha_{1n_1}}}) \oplus \ldots \oplus (C_{p_r^{\alpha_{r1}}} \oplus \ldots \oplus C_{p_r^{\alpha_{rn_1}}}) \\
        G_2 &= (C_{p_1^{\alpha_{13}}} \oplus \ldots \oplus C_{p_1^{\alpha_{1n_1}}}) \oplus \ldots \oplus (C_{p_r^{\alpha_{r1}}} \oplus \ldots \oplus C_{p_r^{\alpha_{rn_1}}}) \\
            &\vdots \\
        G_{n_1} &= (C_{p_2^{\alpha_{21}}} \oplus \ldots \oplus C_{p_2^{\alpha_{2n_2}}}) \oplus \ldots \oplus (C_{p_r^{\alpha_{r1}}} \oplus \ldots \oplus C_{p_r^{\alpha_{rn_1}}}) \\
                &\vdots
    \end{align*}
\end{ejemplo}

\begin{ejemplo} % // TODO: EJercicio 6
    Sea $A$ un grupo con $|A| = 40 = 2^3\cdot 5$:
    \begin{equation*}
        \left\{\begin{array}{c|c|c|c}
                \text{Descomposición} & \text{Descomp. cíclica primaria} & \text{Factores invariantes} & \text{Descomp. cíclica} \\
                \hline
                2^3, 5 & C_8 \oplus C_5 & 40 & C_{40} \\
                2, 2^2, 5 & C_2 \oplus C_4 \oplus C_5 & 2, 20 & C_2 \oplus C_{20} \\
                2, 2, 2, 5 & C_2 \oplus C_2 \oplus C_2 \oplus C_5 & 2, 2, 10 & C_2 \oplus C_2\oplus C_{10}
        \end{array}\right.
    \end{equation*}
    Como $l(A) = 4$ por el ejercicio anterior, series de composición serán:
    \begin{align*}
        &C_{40} \rhd C_{20} \rhd C_{10} \rhd C_5 \rhd \{1\} \\
        &C_{40} \rhd C_{8} \rhd C_{4} \rhd C_2 \rhd \{1\} 
    \end{align*}
    Los factores de composición de la primera son:
    \begin{equation*}
        C_{40}/C_{20} \cong C_2 \qquad C_{20}/C_{10}\cong C_2 \qquad C_{10}/C_5\cong C_2 \qquad C_5/\{1\}\cong C_5
    \end{equation*}
\end{ejemplo}

\begin{ejemplo} % // TODO: Ejecicio 11
    Sea:    
    \begin{equation*}
        G = \langle a,b,c\mid \begin{array}{c}
            2a - 6b + 18c = 0 \\
            6a + 6c = 0
        \end{array}\rangle 
    \end{equation*}
    Y sea:
    \begin{equation*}
        H = \mathbb{Z}^3/\langle (1,-9,3), (1,-7,1), (1,-1,1) \rangle 
    \end{equation*}
    Tenemos la matriz:
    \begin{align*}
        &\left(\begin{array}{ccc}
            2 & -6 & 18 \\
            6 & 0 & 6 
        \end{array}\right)
        \stackrel{}{\longrightarrow}
        \left(\begin{array}{ccc}
            2 & -6 & 18 \\
            0 & 18 & -48 
        \end{array}\right)
        \stackrel{}{\longrightarrow}
        \left(\begin{array}{ccc}
            2 & 0 & 0 \\
            0 & 18 & 48 
        \end{array}\right)
        \stackrel{}{\longrightarrow} \\
        &\left(\begin{array}{ccc}
            2 & 0 & 0 \\
            0 & 18 & 6 
        \end{array}\right)
        \stackrel{}{\longrightarrow} 
        \left(\begin{array}{ccc}
            2 & 0 & 0 \\
            0 & 6 & 0 
        \end{array}\right)
        \stackrel{}{\longrightarrow} \\
    \end{align*}
    Por lo que $G$ tiene rango 1 y sus descommposiciones cíclica y cíclica primaria son:
    \begin{equation*}
        G\cong \mathbb{Z}\oplus\mathbb{Z}_2\oplus\mathbb{Z}_6 \cong \mathbb{Z}\oplus\mathbb{Z}_2\oplus\mathbb{Z}_2\oplus\mathbb{Z}_3
    \end{equation*}
    Con $H$ tenemos lo mismo:
    \begin{align*}
        &\left(\begin{array}{ccc}
            1 & 1 & 1 \\
            -9 & -7 &-1 \\
            3 & 1 & 1
        \end{array}\right)
        \stackrel{}{\longrightarrow}
        \left(\begin{array}{ccc}
            1 & 0 & 0 \\
            -9 & 2 & 8 \\
            3 & -2 & -2
        \end{array}\right)
        \stackrel{}{\longrightarrow}
        \left(\begin{array}{ccc}
            1 & 0 & 0 \\
            0 & 2 & 8 \\
            0 & -2 & -2
        \end{array}\right)
        \stackrel{}{\longrightarrow} \\
        &\left(\begin{array}{ccc}
            1 & 0 & 0 \\
            0 & 2 & 8 \\
            0 & 0 & 6
        \end{array}\right)
        \stackrel{}{\longrightarrow} 
        \left(\begin{array}{ccc}
            1 & 0 & 0 \\
            0 & 2 & 0 \\
            0 & 0 & 6
        \end{array}\right)
    \end{align*}
    Por lo que:
    \begin{equation*}
        H\cong \mathbb{Z}_2 \oplus \mathbb{Z}_6
    \end{equation*}
    y $H$ no tiene parte libre. Tendremos:
    \begin{equation*}
        l(H) = 3
    \end{equation*}
    Los factores de composición serán $\mathbb{Z}_2, \mathbb{Z}_2, \mathbb{Z}_3$.
    \begin{align*}
        G&\not\cong H \\
        T(G)&\cong T(H) = H
    \end{align*}
    ¿Cuáles son los elementos de orden 6 de $H$? Tiene al menos:
    \begin{equation*}
        O(a,1) = O(a,5) = 6 \qquad \forall a\in \mathbb{Z}_2
    \end{equation*}
    También tendremos:
    \begin{equation*}
        O(1,2) = \mcm(O(1), O(2)) = \mcm(2, 3) = 6
    \end{equation*}
    ¿$G$ tiene elementos de orden 6? Sí, los mismos pero con un 0 en primera coordenada.
\end{ejemplo}
