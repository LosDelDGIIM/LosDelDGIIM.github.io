\chapter{Clasificación de grupos abelianos finitos}
\noindent
El objetivo final del tema es demostrar los teoremas de estructura de los grupos abelianos finitos, que permiten clasificar todos estos grupos según su orden (para cada orden tendremos una clasificación), salvo isomorfismos.\\

Serán de especial relevancia varios resultados que ya hemos visto: % // TODO: Buscarlos para referenciarlos
\begin{itemize}
    \item $C_n\times C_m \cong C_{nm} \Longleftrightarrow \mcd(n,m) = 1$, en la Proposición~\ref{prop:carac_prod_finito_ciclicos}.
    \item Si $|G| = p_1^{n_1} \ldots p_k^{n_k}$ y $G$ tenía un único $P_i$ $p_i-$subgrupo de Sylow para cada $i \in \{1,\ldots,k\}$, entonces $G\cong P_1\times P_2\times \ldots \times P_k$.
\end{itemize}
Como trabajaremos con subgrupos abelianos, recordamos que la notación que usábamos para el producto directo de grupos abelianos era $\oplus$.

% // TODO: Cambiar posiblemente el enunciado:
% - Para cada particion de n hay una forma unica
\begin{teo}[Estructura de los $p-$grupos abelianos finitos]\label{teo:1_tema6}\ \\
    Sea $A$ un $p-$grupo abeliano finito con orden $|A| = p^n$ para $n\geq 1$, entonces existen enteros $\beta_1\geq \beta_2 \geq \ldots \geq \beta_t \geq 1$ de forma que:
    \begin{equation*}
        \beta_1 + \beta_2 + \ldots + \beta_t = n
    \end{equation*}
    Y $A\cong C_{p^{\beta_1}} \oplus C_{p^{\beta_2}} \oplus \ldots \oplus C_{p^{\beta_t}}$.\newline Además, esta expresión es única, es decir, si existen $\alpha_1\geq \alpha_2\geq \ldots \geq \alpha_s \geq 1$ de forma que:
    \begin{equation*}
        \alpha_1 + \alpha_2 + \ldots + \alpha_s = n
    \end{equation*}
    Y $A\cong C_{p^{\alpha_1}} \oplus C_{p^{\alpha_2}} \oplus \ldots \oplus C_{p^{\alpha_s}}$, entonces $s = t$ y $\alpha_i = \beta_i$ para todo $i \in \{1,\ldots,t\}$.
    % \begin{proof} % // TODO: Se hará en clase, es larga
    % \end{proof}
\end{teo}

\begin{observacion}
    Notemos que lo que estamos haciendo es tomar particiones de $n$ de la forma $\beta_i$, y este Teorema nos dice que el $p-$grupo puede escribirse de forma única salvo isomorfismos como producto de ciertos subgrupos cíclicos.\newline

    Es decir, existen tantos $p-$grupos abelianos de orden $p^n$ como particiones tengamos del número $n$.
\end{observacion}

\begin{ejemplo}
    Por ejemplo:
    \begin{itemize}
        \item Grupos abelianos finitos de orden $8 = 2^3$, tenemos como particiones:
            \begin{align*}
                3& \Longrightarrow A\cong C_8 \\
                1, 2& \Longrightarrow A\cong C_4\oplus C_2 \\
                1, 1, 1& \Longrightarrow A\cong C_2 \oplus C_2 \oplus C_2
            \end{align*}
        \item Los grupos abelianos finitos de orden $81 = 3^4$, tenemos como particiones:
            \begin{align*}
                A &\cong C_{81} \\
                A &\cong C_{27} \oplus C_3 \\
                A &\cong C_9 \oplus C_9 \\
                A &\cong C_9 \oplus C_3 \oplus C_3 \\
                A &\cong C_3 \oplus C_3 \oplus C_3 \oplus C_3 
            \end{align*}
    \end{itemize}
\end{ejemplo}

\begin{teo}[Estructura de los grupos abelianos finitos]\label{teo:2_tema6}\ \\
    Sea $A$ un grupo abeliano finito con $|A| = p_1^{\gamma_1}\ldots p_k^{\gamma_k}$ siendo $p_i$ primo para todo $i \in \{1,\ldots,k\}$, entonces:
    \begin{equation*}
        A \cong \bigoplus_{i=1}^k \left(\bigoplus_{j=1}^{t_i} C_{p_i^{n_{ij}}}\right)
    \end{equation*}
    Donde para cada $i \in \{1,\ldots,k\}$ tenemos:
    \begin{align*}
        n_{i1} \geq n_{i2} \geq \ldots \geq n_{it_{i}} \geq 1 \\
        n_{i1} + n_{i2} + \ldots + n_{it_{i}} = r_i
    \end{align*}
    Y la descomposición es única salvo el orden.\\

    \noindent
    Esta última recibe el nombre de descomposición cíclica primaria, y a los $p_i^{n_{ij}}$ con $i \in \{1,\ldots,k\}$ y $j \in \{1,\ldots t_i\}$ se les llama divisores elementales de $A$.\newline
    \begin{proof}
        Si $A$ es abeliano y finito, entonces todos sus $p-$subgrupos de Sylow son normales, luego podemos escribir:
        \begin{equation*}
            A = P_1 \oplus P_2 \oplus \ldots \oplus P_k
        \end{equation*}
        Siendo ${P_1, P_2, \ldots, P_k}$ el conjunto de todos sus $p-$subgrupos de Sylow, de forma que $|P_i| = p_i^{r_i}$, para todo $i \in \{1,\ldots,k\}$. Como cada $P_i$ es un $p_i-$subgrupo abeliano finito, aplicando el Teorema~\ref{teo:1_tema6}, podemos escribir:
        \begin{equation*}
            P_i = \bigoplus_{j=1}^{t_i} C_{p_i^{n_{ij}}} \qquad \forall i \in \{1,\ldots,k\}
        \end{equation*}
        De donde tenemos la expresión de la tesis.\newline
        A cada $P_i$ con $i \in \{1,\ldots,k\}$ lo llamaremos componente $p_i-$primaria de $A$.
    \end{proof}
\end{teo}

\begin{ejemplo}
    Si tenemos un subgrupo finito abeliano $A$ con $|A| = 360 = 2^3\cdot 3^2\cdot 5$, veamos los divisores elementales:
    \begin{equation*}
        \begin{array}{r|l}
            \text{Div. elementales} & \text{Descomp. cíclica primaria} \\
            \hline
            2^3\ 3^2\ 5 & C_8\oplus C_9 \oplus C_5 \\
            2^2\ 2\ 3^2\ 5 & C_4\oplus C_2 \oplus C_9 \oplus C_5\\
            2\ 2\ 2\ 3^2\ 5 & C_2 \oplus C_2 \oplus C_2 \oplus C_9 \oplus C_5\\
            2^3\ 3\ 3\ 5 & C_8\oplus C_3 \oplus C_3 \oplus C_5 \oplus C_8\\
            2\ 2^2\ 3\ 3\ 5 & C_2 \oplus C_4 \oplus C_3 \oplus C_3 \oplus C_5\\
            2\ 2\ 2\ 3\ 3\ 5 & C_2 \oplus C_2 \oplus C_2 \oplus C_3 \oplus C_3 \oplus C_5
        \end{array}
    \end{equation*}
    Serían todas las descomposiciones cíclicas primarias de $A$. Es decir, $A$ será isomorfo a cualquiera de esos.
\end{ejemplo}

\noindent
Sin embargo, si recordamos la Proposición~\ref{prop:carac_prod_finito_ciclicos}, esto nos llevará a la descomposición cíclica, donde observaremos por ejemplo que:
\begin{align*}
    C_8\oplus C_9 \oplus C_5 &\cong C_{360} \\
    C_4\oplus C_2 \oplus C_9 \oplus C_5 &\cong C_{180} \oplus C_2 \\
    C_2 \oplus C_2 \oplus C_2 \oplus C_9 \oplus C_5 &\cong C_{90} \oplus C_2 \oplus C_2 \\
    C_8\oplus C_3 \oplus C_3 \oplus C_5 \oplus C_8 &\cong C_{120} \oplus C_3 \\
    C_2 \oplus C_4 \oplus C_3 \oplus C_3 \oplus C_5 &\cong C_{60} \oplus C_6 \\
    C_2 \oplus C_2 \oplus C_2 \oplus C_3 \oplus C_3 \oplus C_5 &\cong C_{30} \oplus C_6 \oplus C_7
\end{align*}
