\chapter{$G-$conjuntos y $p$-grupos}
\begin{definicion}
    Sea $G$ un grupo y $X$ un conjunto no vacío, una acción\footnote{En realidad esta es la definición de acción por la izquierda, pero no vamos a trabajar con las acciones por la derecha, por lo que hablaremos simplemente de acciones.} de $G$ sobre $X$ es una aplicación: 
    \Func{ac}{G\times X}{X}{(g,x)}{ac(g,x)}
    Que verifica:
    \begin{enumerate}
        \item[$i)$] $ac(1,x) = x$ $\quad \forall x\in X$.
        \item[$ii)$] $ac(g, ac(h, x)) = ac(gh, x)$ $\quad \forall x\in X, \quad \forall g,h\in G$.
    \end{enumerate}
    En dicho caso, diremos que $G$ actúa\footnote{En realidad deberíamos decir que ``$G$ actúa por la izquierda sobre $X$''.} (o que opera) sobre $X$.\\

    \noindent
    Si $G$ actúa sobre $X$, diremos que este conjunto $X$ es el $G-$conjunto a izquierda. A la aplicación $ac$ se le llama aplicación de la $G-$estructura.
\end{definicion}

\begin{notacion}
    Si $ac:G\times X\to X$ es una acción de $G$ sobre $X$, es común denotar:
    \begin{equation*}
        ac(g,x) = \prescript{g}{}{x} = g\cdot x = g\ast x
    \end{equation*}
    En este documento, usaremos la notación $ac(g,x) = \prescript{g}{}{x}$.
\end{notacion}

\begin{ejemplo}
    Si $G$ es un grupo y $X$ es un conjunto no vacío, ejemplos de acciones de $G$ sobre $X$ son:
    \begin{enumerate}
        \item La \underline{acción trivial}:
            \Func{ac}{G\times X}{X}{(g,x)}{x}
        \item Si tenemos una acción $ac:G\times X\to X$ y $H<G$, podemos considerar la \underline{acción por restricción} $ac:H\times X\to X$, dada por:
            \begin{equation*}
                ac(h,x) = ac(i(h),x) \qquad \forall h\in H, x\in X
            \end{equation*}
            Donde consideramos la aplicación inclusión $i:H\to G$ dada por $i(h) = h$, para todo $h\in H$.
        \item Dado $n\in \mathbb{N}$, si $X = \{1,\ldots,n\}$ y $G = S_n$, la \underline{acción natural} de $S_n$ sobre $X$ será la acción $ac:S_n\times X\to X$ dada por:
            \begin{equation*}
                ac(\sigma,k) = \prescript{\sigma}{}{k} = \sigma(k) \qquad \forall \sigma\in S_n, k\in X
            \end{equation*}
    \end{enumerate}
\end{ejemplo}

\begin{prop}\label{prop:accion_homomorfismo}
    Sea $G$ un grupo y $X$ un conjunto no vacío, dar una acción de $G$ sobre $X$ equivale a dar un homomorfismo de grupos de $G$ en $\Perm(X)$.
    \begin{proof}
        Veamos que es posible:
        \begin{itemize}
            \item Por una parte, dada una acción de $G$ sobre $X$, $ac:G\times X \to X$, podemos definir la aplicación: 
                \Func{\phi}{G}{\Perm(X)}{g}{\phi(g)}
                Donde $\phi(g)$ es una aplicación $\phi(g):X\longrightarrow X$ dada por:
                \begin{equation*}
                    \phi(g)(x) = \prescript{g}{}{x} \qquad \forall x\in X
                \end{equation*}
                Veamos en primer lugar que $\phi$ está bien definida, es decir, que $\phi(g) \in \Perm(X)$ para cada $g\in G$. Para ello, veamos antes que $\phi$ cumple:
                \begin{itemize}
                    \item $\phi(1) = id_X$, ya que la aplicación $x\mapsto ac(1,x)$ es la aplicación identidad en $X$, por ser $ac$ una acción de $G$ sobre $X$.
                    \item $\phi(g)\phi(h) = \phi(gh)$, ya que al evaluar en cualquier $x\in G$:
                        \begin{equation*}
                            (\phi(g)\phi(h))(x) = \phi(g)(\phi(h)(x)) = \phi(g)(\prescript{h}{}{x}) = \prescript{g}{}{(\prescript{h}{}{x})} \AstIg \prescript{gh}{}{x} = \phi(gh)(x)
                        \end{equation*}
                        Donde en $(\ast)$ hemos usado que $ac$ es una acción de $G$ sobre $X$.
                \end{itemize}
                Ahora, veamos que dado $g\in G$, la aplicación $\phi(g)$ es biyectiva (es decir, está en $\Perm(X)$), ya que su aplicación inversa es $\phi(g^{-1})$:
                \begin{equation*}
                    \phi(g^{-1})\phi(g) = \phi(g^{-1}g) = \phi(1) = \phi(gg^{-1}) = \phi(g) \phi(g^{-1})
                \end{equation*}
                Y anteriormente vimos que $\phi(1) = id_X$, por lo que $\phi(g) \in \Perm(X)$, para todo $g\in G$ y la aplicación $\phi$ está bien definida.\\

                \noindent
                Además, por las dos propiedades anteriores, tenemos que $\phi$ es un homomorfismo de grupos.
            \item Sea $\phi:G\to \Perm(X)$ un homomorfismo de grupos, definimos la aplicación $ac:G\times X \to X$ dada por:
                \begin{equation*}
                    ac(g,x) = \phi(g)(x) \qquad \forall g\in G, x\in X
                \end{equation*}
                Veamos que es una acción:
                \begin{align*}
                    ac(1,x) &= \phi(1)(x) = x \qquad \forall x\in X \\
                    ac(g,ac(h,x)) &= \phi(g)(\phi(h)(x)) = \phi(gh)(x) = ac(gh,x)  \qquad \forall x\in X, \quad \forall g,h\in G
                \end{align*} \qedhere
        \end{itemize}
    \end{proof}
\end{prop}

\begin{definicion}[Representación por permutaciones]
    Sea $G$ un grupo y $X$ un conjunto no vacío, si tenemos una acción de $G$ sobre $X$, el homomorfismo $\phi$ dado por esta acción según la Proposición~\ref{prop:accion_homomorfismo} recibirá el nombre de \underline{representación de $G$ por} \underline{permutaciones}.\\

    \noindent
    Además, llamaremos a $\ker(\phi)$ \underline{núcleo de la acción}, ya que:
    \begin{equation*}
        \ker(\phi) = \{g\in G \mid \phi(g) = id_X\} = \{g\in G \mid \prescript{g}{}{x} = x \quad \forall x\in X\}
    \end{equation*}
    En el caso de que $\ker(\phi) = \{1\}$, diremos que la acción es \underline{fiel}.
\end{definicion}

\begin{ejemplo}
    A continuación, dadas varios ejemplos de acciones, consideraremos en cada caso su representación por permutaciones:
    \begin{enumerate}
        \item La representación por permutaciones de la acción trivial es el homomorfismo $\phi:G\to Perm(X)$ dado por:
            \begin{equation*}
                \phi(g) = id_X \qquad \forall g\in G
            \end{equation*}
        \item Si tenemos una acción $ac:G\times X\to X$ sobre un grupo $G$ y un conjunto no vacío $X$ que tiene asociada una representación por permutaciones $\phi$, entonces la acción por restricción $ac:H\times X\to X$ tendrá asociada como representación por permutaciones el homomorfismo $\phi_H:H\to X$ dado por:
            \begin{equation*}
                \phi_H = \phi \circ i 
            \end{equation*}
            Siendo $i:H\to G$ la aplicación inclusión.
        \item En el caso de la acción natural de $S_n$ sobre $X =\{1,\ldots,n\} $, tenemos que la representación por permutaciones es el homomorfismo $\phi:S_n\to S_n$ dado por:
            \begin{equation*}
                \phi(\sigma) = \sigma \qquad \forall \sigma\in S_n
            \end{equation*}
            Es decir, $\phi = id_{S_n}$.
        \item Sea $G$ un grupo, podemos definir la \underline{acción por traslación} como:
            \Func{ac}{G\times G}{G}{(g,h)}{gh}
            Y el homomorfismo asociado a la acción como representación por permutaciones será $\phi:G\to \Perm(G)$ dado por:
            \begin{equation*}
                \phi(g)(h) = gh \qquad \forall g,h\in G
            \end{equation*}
            Como además:
            \begin{equation*}
                \ker(\phi) = \{g\in G\mid gh = h \quad \forall h\in G\} = \{1\}
            \end{equation*}
            Tenemos que es una acción fiel.
    \end{enumerate}
\end{ejemplo}

\begin{teo}[Cayley]
    Todo grupo finito es isomorfo a un subgrupo de un grupo de permutaciones.
    \begin{proof}
        Sea $G$ un grupo finito, consideramos la acción por traslación:
        \Func{ac}{G\times G}{G}{(g,h)}{gh}
        Y su representación por permutaciones, $\phi:G\to \Perm(G)$ dado por:
        \begin{equation*}
            \phi(g)(h) = gh \qquad \forall g\in G, \forall h\in G
        \end{equation*}
        Como la acción por traslación es una acción fiel, tendremos que $\ker(\phi) = \{1\}$ y aplicando el Primer Teorema de Isomorfía sobre $\phi$, obtenemos que:
        \begin{equation*}
            G \cong G/\{1\} \cong Im(\phi)
        \end{equation*}
        Donde $Im(\phi) = \phi_\ast(G)$, que en la Proposición~\ref{prop:imagen_directa} vimos que es un subgrupo de $\Perm(G)$.
    \end{proof}
\end{teo}

\begin{ejemplo}
    Podemos considerar las traslaciones de $G$ sobre conjuntos especiales:
    \begin{itemize}
        \item La acción por traslación de $G$ sobre $\cc{P}(G)$ será $ac:G\times \cc{P}(G) \to \cc{P}(G)$ dada por:
            \begin{equation*}
                ac(g,A) = gA = \{ga \mid a\in A\}  \subseteq G \qquad \forall A\in \cc{P}(G)
            \end{equation*}
        \item Podemos también considerar la acción por traslación en el cociente por las clases laterales por la izquierda\footnote{No es necesario considerar $H\lhd G$, ya que solo consideramos conjuntos no vacíos, por lo que no es necesario que el cociente tenga estructura de grupo.}: si $H<G$, consideramos el cociente de $G$ sobre $H$ por la izquierda y la acción $ac:G\times G/\prescript{}{H}{\sim} \to G/\prescript{}{H}{\sim}$ dada por:
            \begin{equation*}
                ac(g,xH) = \prescript{g}{}{(xH)} = gxH  = \{gxh \mid h\in H\}
            \end{equation*}
    \end{itemize}
    \begin{enumerate}
        \item[6.] La \underline{acción por conjugación} se define como $ac:G\times G\to G$ dada por:
            \begin{equation*}
                ac(g,h) = \prescript{g}{}{h} = ghg^{-1}
            \end{equation*}
            Que es una acción, ya que:
            \begin{align*}
                \prescript{1}{}{h} &= 1h1^{-1} = h \qquad \forall h\in G\\
                \prescript{g}{}{(\prescript{h}{}{l})} &= g\prescript{h}{}{l}g^{-1} = ghlh^{-1}g^{-1} ghl{(gh)}^{-1} = \prescript{gh}{}{l} \qquad \forall g,h,l \in G
            \end{align*}
            El homomorfismo asociado es:
            \begin{gather*}
                \phi:G\to\Perm(X) \\
                \phi(g)(h) = ghg^{-1} \qquad \forall g,h\in G
            \end{gather*}
            Y a $\phi(g)$ lo llamábamos (en el ej 14 de la relación 4, $\varphi_g$) \newline \underline{automorfismo interior definido por $G$}, con imagen:
            \begin{equation*}
                Im(\phi) = Int(G)
            \end{equation*}
            El núcleo en este caso es:
            \begin{equation*}
                \ker(\phi) = \{g\in G \mid ghg^{-1} = h\} = \{g\in G \mid gh = hg \quad \forall h\in G\} = Z(G)
            \end{equation*}
        \item[7.] La \underline{acción por conjugación en partes de $G$} se define como la aplicación \newline $ac:G\times \cc{P}(G) \to \cc{P}(G)$ dada por:
            \begin{equation*}
                ac(g,A) = \prescript{g}{}{A} = gAg^{-1} = \{gag^{-1} \mid a\in A\} \subseteq G \qquad \forall A\in \cc{P}(G)
            \end{equation*}
        \item[8.] Podemos definir la acción por conjugación de $G$ también sobre $Subg(G)$:
            \begin{equation*}
                Subg(G) = \{H\subseteq G \mid H<G\}
            \end{equation*}
            Como la aplicación $ac:G\times Subg(G) \to Subg(G)$ dada por:
            \begin{equation*}
                ac(g,H) = \prescript{g}{}{H} = gHg^{-1} < G
            \end{equation*}
            Ya que en la Proposición~\ref{prop:xHx_subgrupo} vimos que $gHg^{-1}$ era un subgrupo de $G$, al que llamaremos subgrupos conjugado de $G$.
    \end{enumerate}
\end{ejemplo}


% // TODO: Hacer un resumen de todas las acciones, al igual que en el T1?
