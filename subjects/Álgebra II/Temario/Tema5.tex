\chapter{$G-$conjuntos y $p$-grupos}
\begin{definicion}
    Sea $G$ un grupo y $X$ un conjunto no vacío, una acción\footnote{En realidad esta es la definición de acción por la izquierda, pero no vamos a trabajar con las acciones por la derecha, por lo que hablaremos simplemente de acciones.} de $G$ sobre $X$ es una aplicación: 
    \Func{ac}{G\times X}{X}{(g,x)}{ac(g,x)}
    Que verifica:
    \begin{enumerate}
        \item[$i)$] $ac(1,x) = x$ $\quad \forall x\in X$.
        \item[$ii)$] $ac(g, ac(h, x)) = ac(gh, x)$ $\quad \forall x\in X, \quad \forall g,h\in G$.
    \end{enumerate}
    En dicho caso, diremos que $G$ actúa\footnote{En realidad deberíamos decir que ``$G$ actúa por la izquierda sobre $X$''.} (o que opera) sobre $X$.\\

    \noindent
    Si $G$ actúa sobre $X$, diremos que este conjunto $X$ es el $G-$conjunto a izquierda. A la aplicación $ac$ se le llama aplicación de la $G-$estructura.
\end{definicion}

\begin{notacion}
    Si $ac:G\times X\to X$ es una acción de $G$ sobre $X$, es común denotar:
    \begin{equation*}
        ac(g,x) = \prescript{g}{}{x} = g\cdot x = g\ast x
    \end{equation*}
    En este documento, usaremos la notación $ac(g,x) = \prescript{g}{}{x}$.
\end{notacion}

\begin{ejemplo}
    Ejemplos de acciones son:
    \begin{enumerate}
        \item Tenemos la \underline{acción trivial}: $ac:G\times X \to X$ dada por $ac(g,x) = x$ $\forall g\in G, x\in X$.
        \item La \underline{acción por restricción} de un subgrupo $H<G$ sobre $X$
            \begin{figure}[H]
                \centering
                \shorthandoff{""}
                \begin{tikzcd}
                ac:H\times X \arrow[rd, "i_{X1}"] \arrow[rr] &                              & X \\
                                                             & G\times X \arrow[ru, "ac_G"] &  
                \end{tikzcd}
                \shorthandon{""}
            \end{figure}
        \item La \underline{acción natural} de $S_n$ sobre $X = \{1,\ldots,n\}$ será:
            \begin{gather*}
                ac:S_n\times X \to X \\
                (\sigma,i) \longmapsto ac(\sigma,i) = \prescript{\sigma}{}{i} = \sigma(i)
            \end{gather*}
    \end{enumerate}
\end{ejemplo}

\begin{prop}
    Sea $G$ un grupo y $X$ un conjunto no vacío, dar una acción de $G$ sobre $X$ equivale a dar un homomorfismo de grupos de $G$ en $\Perm(X)$.
    \begin{proof}
        Veamos que es posible:
        \begin{itemize}
            \item Por una parte, dada una acción de $G$ sobre $X$, $ac:G\times X \to X$, podemos definir la aplicación: 
                \Func{\phi}{G}{\Perm(X)}{g}{\phi(g)}
                Donde $\phi(g)$ es una aplicación $\phi(g):X\longrightarrow X$ dada por:
                \begin{equation*}
                    \phi(g)(x) = \prescript{g}{}{x} \qquad \forall x\in X
                \end{equation*}
                Veamos en primer lugar que $\phi$ está bien definida, es decir, que $\phi(g) \in \Perm(X)$ para cada $g\in G$. Para ello, veamos antes que $\phi$ cumple:
                \begin{itemize}
                    \item $\phi(1) = id_X$, ya que la aplicación $x\mapsto ac(1,x)$ es la aplicación identidad en $X$, por ser $ac$ una acción de $G$ sobre $X$.
                    \item $\phi(g)\phi(h) = \phi(gh)$, ya que al evaluar en cualquier $x\in G$:
                        \begin{equation*}
                            (\phi(g)\phi(h))(x) = \phi(g)(\phi(h)(x)) = \phi(g)(\prescript{h}{}{x}) = \prescript{g}{}{(\prescript{h}{}{x})} \AstIg \prescript{gh}{}{x} = \phi(gh)(x)
                        \end{equation*}
                        Donde en $(\ast)$ hemos usado que $ac$ es una acción de $G$ sobre $X$.
                \end{itemize}
                Ahora, veamos que dado $g\in G$, la aplicación $\phi(g)$ es biyectiva (es decir, está en $\Perm(X)$), ya que su aplicación inversa es $\phi(g^{-1})$:
                \begin{equation*}
                    \phi(g^{-1})\phi(g) = \phi(g^{-1}g) = \phi(1) = \phi(gg^{-1}) = \phi(g) \phi(g^{-1})
                \end{equation*}
                Y anteriormente vimos que $\phi(1) = id_X$, por lo que $\phi(g) \in \Perm(X)$, para todo $g\in G$ y la aplicación $\phi$ está bien definida.\\

                \noindent
                Además, por las dos propiedades anteriores, tenemos que $\phi$ es un homomorfismo de grupos.
                % // TODO: Demsotrar el siguiente item
            \item Sea $\phi:G\to \Perm(X)$ un homomorfismo de grupos, definimos la acción $ac:G\times X \to X$ dada por:
                \begin{equation*}
                    ac(g,x) = \phi(g)(x) = \prescript{g}{}{x} \qquad \forall g\in G, x\in X
                \end{equation*}
                Veamos que es una acción:
                \begin{align*}
                    \prescript{1}{}{x} &= \phi(1)(x) = x \\
                    \prescript{g}{}{(\prescript{h}{}{x})} &= \phi(g)(\phi(h)x) = \phi(gh)(x) = \prescript{(gh)}{}{x}
                \end{align*}
        \end{itemize}
        Este homomorfismo $\phi$ se conoce como la \underline{representación de $G$ por permutaciones}.
    \end{proof}
\end{prop}

Podemos calcular el núcleo de $\phi$, que recibirá el nombre \underline{núcleo de la acción}.
\begin{equation*}
    \ker(\phi) = \{g\in G\mid \phi(g) = id_X\} = \{g\in G \mid \prescript{g}{}{x} = x \quad \forall x\in X\}
\end{equation*}

\begin{definicion}
    Si $\ker(\phi) = \{1\}$, decimos que la acción es \underline{fiel}.
\end{definicion}

\begin{ejemplo}
    Ejemplos de representaciones por permutaciones de acciones son:
    \begin{enumerate}
        \item Si consideramos la acción trivial, la representación de $G$ por permutaciones será:
            \begin{equation*}
                \phi(g) = id_X
            \end{equation*}
        \item En la acción por restricción, sea $H\subseteq G$:
            \begin{figure}[H]
                \centering
                \shorthandoff{""}
                \begin{tikzcd}
                H \arrow[r, "i"] \arrow[rr, "\phi_H"', bend right, shift right] & G \arrow[r, "\phi"] & Perm(X)
                \end{tikzcd}
                \shorthandon{""}
            \end{figure}
        \item En el caso de la acción natural, $\phi(g) = id_X$.
        \item En el caso de la acción de $D_4$ sobre $X = \{1,2,3,4\}$:
            \begin{gather*}
                ac:G\times X \to X \\
                ac(g,x) = \prescript{g}{}{x} = \phi(g)(x)
            \end{gather*}
            Tendremos:
            \begin{gather*}
                \phi:G\to\Perm(X) = S_4 \\
                \phi(r) = (1\ 2\ 3\ 4) \\
                \phi(s) = (2\ 4)
            \end{gather*}
            $\phi$ es inyectiva, por lo que será una acción fiel.

            A partir de ahora, consideramos $X = G$.
        \item La \underline{acción por traslación} se define como:
            \begin{gather*}
                ac:G\times G \to G \\
                ac(g,h) = \prescript{g}{}{h} = gh
            \end{gather*}
            La representación asociada será:
            \begin{gather*}
                \phi:G\to\Perm(X) \\
                \phi(g)(h) = gh
            \end{gather*}
            $\{g\in G\mid gh = h \quad \forall h\in G\} = \ker(\phi) = \{1\}$, por lo que se trata de una acción fiel.
    \end{enumerate}
\end{ejemplo}

\begin{teo}[Cayley]
    Todo grupo finito es isomorfo a un subgrupo de un grupo de permutaciones.
    \begin{proof}
        \begin{equation*}
            \phi/\ker(\phi) \cong Im(\phi)
        \end{equation*}
        De donde $\phi \cong Im(\phi)$, por ser una acción fiel.
    \end{proof}
\end{teo}

\begin{ejemplo}
    Podemos considerar las traslaciones de $G$ sobre conjuntos especiales:
    \begin{itemize}
        \item La acción por traslación de $G$ sobre $\cc{P}(G)$ es:
            \begin{gather*}
                ac:G\times\cc{P}(G) \to \cc{P}(G) \\
                (g,A) \longmapsto ac(g,A) = \prescript{g}{}{A} = gA = \{ga \mid a\in A\} \subseteq G
            \end{gather*}
        \item Podemos también hacer la acción por traslación en el cociente por las clases laterales por la izquierda: si $H<G$, consideramos el cociente de $G$ sobre $G/\prescript{}{H}{\sim}$:
            \begin{gather*}
                ac:G\times G/\prescript{}{H}{\sim} \to G/\prescript{}{H}{\sim} \\
                ac(g,xH) = \prescript{g}{}{xH} = gxH
            \end{gather*}
    \end{itemize}
    \begin{enumerate}
        \item[6.] La \underline{acción por conjugación} se define como la aplicación
            \begin{gather*}
                ac:G\times G\to G \\
                ac(g,h) = \prescript{g}{}{h} = ghg^{-1}
            \end{gather*}
            Que es una acción ya que:
            \begin{align*}
                \prescript{1}{}{h} &= 1h1^{-1} = h \\
                \prescript{g}{}{(\prescript{h}{}{l})} &= ghl{(gh)}^{-1} = g(hlh^{-1})g^{-1}
            \end{align*}
            El homomorfismo asociado es:
            \begin{gather*}
                \phi:G\to\Perm(X) \\
                \phi(g)(h) = ghg^{-1} \qquad \forall g,h\in G
            \end{gather*}
            Y a $\phi(g)$ lo llamábamos (en el ej 14 de la relación 4, $\varphi_g$) \newline \underline{automorfismo interior definido por $G$}, con imagen:
            \begin{equation*}
                Im(\phi) = Int(G)
            \end{equation*}
            El núcleo en este caso es:
            \begin{equation*}
                \ker(\phi) = \{g\in G \mid ghg^{-1} = h\} = \{g\in G \mid gh = hg \quad \forall h\in G\} = Z(G)
            \end{equation*}
        \item[7.] La \underline{acción por conjugación en partes de $G$} se define como:
            \begin{gather*}
                ac:G\times \cc{P}(G) \to \cc{P}(G) \\
                ac(g,A) = \prescript{g}{}{A} = gAg^{-1} = \{gag^{-1} \mid a\in A\} \subseteq G
            \end{gather*}
        \item[8.] Podemos definir la acción por conjugación también de $G$ sobre $Subg(G)$:
            \begin{equation*}
                Subg(G) = \{H\subseteq G \mid H<G\}
            \end{equation*}
            \begin{gather*}
                ac:G\times Subg(G) \to Subg(G) \\
                ac(g,H) = \prescript{g}{}{H} = gHg^{-1} < G
            \end{gather*}
            A $gHg^{-1}$ lo llamaremos subgrupo conjugado de $G$. % // TODO: Poner esta definición cuando se demostró que gHg-1 era un subgrupo.
    \end{enumerate}
\end{ejemplo}
