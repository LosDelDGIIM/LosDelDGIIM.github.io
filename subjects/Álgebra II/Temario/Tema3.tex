\chapter{Grupos cociente y Teoremas de isomorfía}
Recordamos las clases laterales $x\prescript{}{H}{\sim\ } y$ y $x\sim_H y$ que vimos en el Tema anterior, junto con los conjutnos cociente que obteníamos: $G/\prescript{}{H}{\sim\ } $ y $G/\sim_H$. Vamos a tratar de considerar simplemente el conjunto $G/H$.

\begin{definicion}[Subgrupos normales]
    Sea $G$ un grupo y $H<G$, se dice que es un subgrupo normal de $G$, denotado por $H \lhd G$ si las clases laterales coinciden.
    \begin{equation*}
        xH = Hx \qquad \forall x\in G
    \end{equation*}
    En cuyo caso, tendremos que $G/\prescript{}{H}{\sim\ } = G/\sim_H$, por lo que notaremos estos conjuntos como $G/H$, y lo llamaremos \underline{conjunto de las clases laterales de $H$ en $G$}. 
\end{definicion}

\begin{definicion}[Conjugado]
    Al conjunto $xHx^{-1}$ lo llamaremos conjugado de $H$:
    \begin{equation*}
        xHx^{-1} = \{xhx^{-1} \mid h\in H\}
    \end{equation*}
\end{definicion}

\begin{prop}[Caracterízación de subgrupos normales]
    Sea $G$ un grupo y $H< G$, son equivalentes:
    \begin{enumerate}
        \item[$i)$] $H\lhd G$.
        \item[$ii)$] $\forall x\in G, \forall h\in H$, $xhx^{-1}\in H$.
        \item[$iii)$] $\forall x\in G$, $xHx^{-1}\subseteq H$.
        \item[$iv)$] $\forall x\in G$, $xHx^{-1}= H$. Es decir, $H$ coincide con todos sus conjugados
    \end{enumerate}
    \begin{proof}
        Veamos las implicaciones:
        \begin{description}
            \item [$i)\Longrightarrow ii)$] Por ser $H\lhd G$, tenemos que $xH = Hx$, luego $xh\in Hx$, con lo que $\exists h'\in H$ de forma que $xh = h'x$. Si multiplicamos por $x^{-1}$ a la derecha:
                \begin{equation*}
                    xhx^{-1} = h' \in H
                \end{equation*}
            \item [$ii)\Longleftrightarrow iii)$] Es claro
            \item [$iii)\Longrightarrow iv)$] $H\subseteq xHx^{-1}$ (la otra ya se sabe). $\forall x\in G$, en particular, $x^{-1}\in G$, tenemos que:
                \begin{equation*}
                    x^{-1}H{(x^{-1})}^{-1} = x^{-1}Hx \subseteq H
                \end{equation*}
                Y tenemos que:
                \begin{equation*}
                    H = x(x^{-1}Hx)x^{-1} \subseteq xHx^{-1}
                \end{equation*}
            \item [$iv)\Longrightarrow i)$] $\forall h\in H$, $xhx^{-1}\in xHx^{-1} = H$, luego $\exists h'\in H$ de forma que $xhx^{-1} = h'$. Si multiplicamos por $x$ a la derecha:
                \begin{equation*}
                    xh = h'x \in Hx
                \end{equation*}
                Hemos probado que $xH\subseteq Hx$, y la otra inclusión es análoga. Concluimos que $xH = Hx$, de donde $H\lhd G$.
        \end{description}
    \end{proof}
\end{prop}

\begin{ejemplo}
    Veamos que subgrupos normales no hay precisamente pocos:
    \begin{enumerate}
        \item Dado un grupo $G$, los dos subgrupos impropios de $G$ siempre son normales (se demuestra fácil por $ii)$ de la proposición anterior).
        \item En un grupo abeliano, todos sus subgrupos son normales:
            \begin{equation*}
                xH = \{xh \mid h \in H\} = \{hx \mid h \in H\} = Hx \qquad \forall x\in G
            \end{equation*}
        \item Todo subgrupo de índice 2 es normal, es decir, si $H<G$ con $[G:H] = 2$, entonces $H$ es normal.
            \begin{proof}
                Sea $x\in G\setminus H$, como $[G:H] = 2$, tenemos que:
                \begin{equation*}
                    G = H\cup xH = H\cup Hx
                \end{equation*}
                En ambos casos, como son particiones disjuntas, tenemos que $xH = Hx$, con lo que $H\lhd G$.
            \end{proof}
        \item En $S_3$, si consideramos $H = \langle (1\ 2) \rangle $, no tenemos $H\lhd S_3$, como se vio en el correspondiente ejemplo del tema anterior, y podemos volver a comprobar con la caracterizacion, ya que:
            \begin{equation*}
                (2\ 3)(1\ 2){(2\ 3)}^{-1} = (1\ 3)\notin H
            \end{equation*}
            Igual les pasa a los subgrupos $\langle (2\ 3) \rangle $ y $\langle (1\ 3) \rangle $. Sea ahora $K = \{1, (1\ 2\ 3), (1\ 3\ 2)\}$, tenemos que $K\lhd S_3$:
            \begin{equation*}
                S_3 / K \cong \{H, H(1\ 2)\} = \{K, (1\ 2)H\}
            \end{equation*}
            Ya que $[S_3 : K] = 2$, con lo que es normal.
        \item La relación de ``ser un subgrupo normal de'' no es transitiva, es decir, si $G$ es un grupo con $K<H<G$, $K\lhd H$ y $H\lhd G$, entonces no necesariamente se tiene que $K\lhd G$. La situación es la descrita en la Figura~\ref{fig:situacion}
            \begin{figure}[H]
                \centering
                \begin{tikzpicture}
                    \node (G) {$G$};
                    \node[below right=of G] (H) {$H$};
                    \node[below left=of H] (K) {$K$};

                    \draw (G) -- (H);
                    \draw (H) -- (K);
                    \draw (K) -- (G);
                \end{tikzpicture}
                \caption{Situación descrita.}
                \label{fig:situacion}
            \end{figure}

            Por ejemplo, en $A_4$ consideramos el grupo de Klein $V$ y $\langle (1\ 2)(3\ 4) \rangle $. Vamos a ver que $\langle (1\ 2)(3\ 4) \rangle \lhd V \lhd A_4 $ pero no se cumple que $\langle (1\ 2)(3\ 4) \rangle \lhd A_4 $:
            \begin{figure}[H]
                \centering
                \begin{tikzpicture}
                    \node (G) {$A_4$};
                    \node[below right=of G] (H) {$V$};
                    \node[below=of G, yshift=-2cm] (K) {$\langle (1\ 2)(3\ 4) \rangle $};

                    \draw (G) -- (H);
                    \draw (H) -- (K);
                    \draw (K) -- (G);
                \end{tikzpicture}
            \end{figure}
            \begin{itemize}
                \item En primer lugar, $\langle (1\ 2)(3\ 4) \rangle \lhd V $, por ser un subgrupo de índice 2.
                \item Veamos ahora que $V\lhd A_4$ (por índices no podemos, $[A_4:V] = 3$). Consideramos:
                    \begin{equation*}
                        A_4 = \langle (1\ 2\ 3), (1\ 2\ 4) \rangle 
                    \end{equation*}
                    Ya que $(1\ 3\ 4) = (1\ 2\ 4)(1\ 2\ 3)$. Basta comprobar que los generadores de $A_4$ están en $V$: % // TODO: Quizas meter esto en prop
                    \begin{align*}
                        &(1\ 2\ 3)(1\ 2)(3\ 4){(1\ 2\ 3)}^{-1} \in  V \\
                        &(1\ 2\ 3)(1\ 3)(2\ 4){(1\ 2\ 3)}^{-1} \in  V \\
                        &(1\ 2\ 3)(1\ 4)(2\ 3){(1\ 2\ 3)}^{-1} \in  V \\
                        &\vdots
                    \end{align*}
                \item Veremos ahora que no se tiene que $\langle (1\ 2)(3\ 4) \rangle\lhd A_4 $, ya que:
                    \begin{equation*}
                        (1\ 2\ 3)(1\ 2)(3\ 4){(1\ 2\ 3)}^{-1} = (1\ 4)(2\ 3)\notin H
                    \end{equation*}
            \end{itemize}
    \end{enumerate}
\end{ejemplo}

\begin{definicion}[Centro]
    Sea $G$ un grupo, definimos el \underline{centro de $G$} como el conjunto:
    \begin{equation*}
        Z(G) = \{a\in G \mid ax = xa, \forall x\in G\}
    \end{equation*}
\end{definicion}
Podemos entener $Z(G)$ como ``la parte abeliana del grupo'' $G$.

\begin{prop}
    Sea $G$ un grupo, se verifica:
    \begin{enumerate}
        \item[$i)$] $Z(G)$ es un subgrupo.
        \item[$ii)$] $Z(G)\lhd G$.
        \item[$iii)$] Si $G$ es abeliano, entonces $L(G) = G$.
    \end{enumerate}
    % \begin{proof} % // TODO: Hacer
    % \end{proof}
\end{prop}

\begin{ejemplo} % // TODO: Es el ejercicio 6 ARTURITO
    Ejemplos interesantes:
    \begin{itemize}
        \item Veamos que $Z(S_n) = 1$ cuando $n\geq 3$. Para ello, supongamos que $n\geq 3$ y consideremos $1\neq \sigma\in S_n$, con lo que existirán $i,j\in \{1,\ldots,n\}$ con $i\neq j$ de forma que $\sigma(i) = j$.

            En dicho caso, $\exists i\neq k \neq j$ en $\{1,\ldots,n\}$. Si consideramos $\tau = (j\ k)$:
            \begin{equation*}
                \left.\begin{array}{rl}
                    \sigma\tau(i) &= \sigma(i) = j \\
                    \tau\sigma(i) &= \tau(j) = k
                \end{array}\right\} \Longrightarrow \sigma\tau \neq \tau \sigma
            \end{equation*}
            Por tanto, $\sigma\notin Z(S_n)$, para todo $\sigma\in S_n$.
        \item Veamos que  que $Z(A_n) = 1$ cuando $n\geq 4$. Para $n\geq 4$, $\exists i\neq j$ de forma que $\sigma(i) = j$, con lo que puedo encontrar $k,l$, distintos entre sí y distintos de $i$ y $j$. Consideramos:
            \begin{equation*}
                \tau = (j\ k\ l) \in A_4
            \end{equation*}
            \begin{equation*}
                \left.\begin{array}{rl}
                        \sigma\tau(i) &= k \\
                        \tau\sigma(i) &= j
                \end{array}\right\} \Longrightarrow Z(A_n) = 1
            \end{equation*}
    \end{itemize}
\end{ejemplo}

Veamos ahora la última caracterización que vamos a considerar para los subgrupos normados:

\begin{prop}
    Sea $G$ un grupo, $H<G$, entonces, equivalen:
    \begin{enumerate}
        \item[$i)$] $H\lhd G$.
        \item[$ii)$] $\forall x,y\in G \mid xy \in H$, entonces $yx \in H$
    \end{enumerate}
    \begin{proof}
        Veamos las dos implicaciones:
        \begin{description}
            \item [$i)\Longrightarrow ii)$] Si $xy \in H$, entonces:
                \begin{equation*}
                    x\sim_H y^{-1} \Longrightarrow Hx = Hy^{-1}
                \end{equation*}
                Por lo que podemos encontrar $h,h'\in H$ de forma que $hx = h'y^{-1}$. Al ser $H\lhd G$, tenemos que:
                \begin{equation*}
                    \left.\begin{array}{rl}
                            xH &= Hx \\
                            y^{-1}H &= Hy^{-1}
                        \end{array}\right\} \Longrightarrow \left\{\begin{array}{rll}
                        hx &= xh'' & h'' \in H \\
                        h'y^{-1} &= y^{-1} h ''' &h''' \in H
                    \end{array}\right.
                \end{equation*}
                En conclusión:
                \begin{equation*}
                    xh'' = y^{-1}h'''\Longrightarrow yx = h''' h''^{-1} \in H
                \end{equation*}
            \item [$ii)\Longrightarrow i)$] $\forall x\in G, \forall h\in H$, tenemos $x^{-1}xh \in H$. Cogemos $x$ como $x^{-1}$ e $y = xh$. Por hipótesis, tenemos que:
                \begin{equation*}
                    xhx^{-1} \in H \Longrightarrow H\lhd G
                \end{equation*}
        \end{description}
    \end{proof}
\end{prop}

Veamos ahora que el hecho de que $H$ sea normal nos da que $G/H$ es un grupo.

\begin{teo}
    Sea $G$ un grupo y $H\lhd G$, entonces en el conjunto $G/H$ podemos definir una operación binaria $G/H\times G/H\longrightarrow G/H$ que dota a $G/H$ de estructura de grupo, de forma que la proyección canónica $p:G\to G/H$ sea un homomorfismo de grupos.\\

    \noindent
    De esta forma, llamaremos a $G/H$ \underline{grupo cociente}.
    \begin{proof}
        Definimos la operación binaria $\cdot G/H\times G/H\longrightarrow G/H$ dada por:
        \begin{equation*}
            xH\cdot yH = xyH \qquad \forall x,y\in G
        \end{equation*}
        A esta operación la denotaremos a partir de ahora por yuxtaposición.
        \begin{itemize}
            \item En primer lugar, comprobemos que está bien definida, es decir, si $xH = x'H$ y $yH=y'H$, entonces $xyH = x'y'H$. Para ello:
                \begin{equation*}
                    \left.\begin{array}{l}
                        xH = x'H \\
                        yH = y'H
                    \end{array}\right\} \Longrightarrow \left\{\begin{array}{l}
                        x'= xh_1 \\
                        y' = yh_2 \\
                        h_1,h_2\in H
                    \end{array}\right.
                \end{equation*}
                Vemos ahora que:
                \begin{equation*}
                    x'y'h = xh_1yh_2h \stackrel{H\lhd\ G}{=} xyh_1'h_2h \in xyH
                \end{equation*}
                Para cierto $h_1'\in H$, por lo que tenemos $\subseteq$.
            \item Que la operación está definida es clara, ya que la operación de $G$ es asociativa.
            \item En neutro es $1H = H$.
            \item Fijado un elemento $xH \in G/H$, tendremos que ${(xH)}^{-1} = x^{-1}H$.
        \end{itemize}
        Concluimos que $G/H$ es un grupo.\\

        \noindent
        Ahora, consideramos $p:G\rightarrow G/H$ que viene definida por $p(x) = xH$ para todo $x\in G$, gracias a la definición de la operación de $G/H$, tenemos que:
        \begin{equation*}
            p(xy) = xyH = xHyH = p(x)p(y)
        \end{equation*}
    \end{proof}
\end{teo}
