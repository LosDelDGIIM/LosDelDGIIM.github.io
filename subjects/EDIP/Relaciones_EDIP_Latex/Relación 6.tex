\section{Modelos de Distribuciones Discretas}



\begin{ejercicio}
    La probabilidad de que cada enfermo de cierto hospital reaccione favorablemente después de aplicarle un calmante es $0.01$. Si se aplica el calmante a 200 enfermos, determinar:
    \begin{enumerate}
        \item La distribución de probabilidad del número de enfermos que reaccionan favorablemente, la media y la varianza.

        Sea $X$ la variable aleatoria que determina el número de enfermos que reaccionan favorablemente tras aplicarle el calmante. Tenemos que $X\leadsto B(200,\;0.01)$. Por tanto:
        \begin{equation*}
            P(x)=\binom{200}{x} 0.01^x (0.99)^{200-x}
        \end{equation*}

        Por ser una distribución binomial, tenemos que:
        \begin{equation*}
            E[X]=np = 200\cdot 0.01=2
        \end{equation*}
        \begin{equation*}
            Var[X]=np(1-p) = 200\cdot 0.01\cdot 0.99=1.98
        \end{equation*}
        
        \item Probabilidad de que a lo sumo 2 enfermos reaccionen favorablemente.
        \begin{equation*}
            P[X\leq 2]=P[X=0] + P[X=1] + P[X=2] = 0.6766 
        \end{equation*}
        
        \item Probabilidad de que más de 3 enfermos reaccionen favorablemente.
        \begin{equation*}
            P[X>3]=1-P[X\leq 2] -P[X=3] =0.3233 -0.18136=0.14196
        \end{equation*}
    \end{enumerate}
\end{ejercicio}


\begin{ejercicio}
    Cada vez que una máquina dedicada a la fabricación de comprimidos produce uno, la probabilidad de que sea defectuoso es $0.01$.
    \begin{enumerate}
        \item  Si los comprimidos se colocan en tubos de 25, ¿cuál es la probabilidad de que en un tubo todos los comprimidos sean buenos?

        Sea $X$ la variable aleatoria que determina el número de comprimidos defectuosos en tubos de 25 unidades. Tenemos que $X\leadsto B(25,\;0.01)$. Por tanto, si todos los comprimidos son buenos, tenemos que no hay ninguno defectuoso. Por tanto, la probabilidad de que en un tubo todos sean buenos es:
        \begin{equation*}
            P[X=0]=\binom{25}{0}0.01^0\cdot 0.99^{25} = 0.99^{25}
        \end{equation*}
        
        \item  Si los tubos se colocan en cajas de 10, ¿cuál es la probabilidad de que en una determinada caja haya exactamente 5 tubos con un comprimido defectuoso?

        La probabilidad de que un tubo tenga un comprimido defectuoso es:
        \begin{equation*}
            P[X=1]=\binom{25}{1}0.01^1\cdot 0.99^{24} = 0.1964
        \end{equation*}

        Definimos $p=0.9164$. Sea $Y$ una variable aleatoria que determina el número de tubos con un comprimido defectuoso en una caja de 10. Tenemos que $Y\leadsto B(10,p)$. Por tanto, la probabilidad de que haya 5 tubos con un comprimido defectuoso es:
        \begin{equation*}
            P[Y=5]=\binom{10}{5}p^5\cdot (1-p)^{5} = 0.02468
        \end{equation*}
    \end{enumerate}
\end{ejercicio}


\begin{ejercicio}
     Se capturan 100 peces de un estanque que contiene 10000. Se les marca con una anilla y se devuelven al agua. Transcurridos unos días se capturan de nuevo 100 peces y se cuentan los anillados.
     \begin{enumerate}
         \item Calcular la probabilidad de que en la segunda captura se encuentre al menos un pez anillado.

        Tenemos que de una población total de $N=10000$ se encuentran divididos en dos poblaciones, la primera de $N_1=100$ y la segunda de $N-N_1$.
        
        Sea $X$ la variable aleatoria que contabiliza la cantidad de indiviuos de $N_1$ en una muestra de $n=100$. Tenemos que $X\leadsto H(10^4, 10^2, 10^2)$.

        Para simplificar los cálculos, como $N_1\leq 0.1N$, aproximamos la distribución hipergeométrica a una binomial con el mismo valor de $n$ y $p=\frac{N_1}{N}=0.01$. Por tanto, $X\leadsto H(10^4, 10^2, 10^2)\cong B(100, 0.01)$.

        Por tanto,
        \begin{equation*}
            P[X\geq 1]=1-P[X=0] = 1-\binom{100}{0}0.01^0\cdot 0.99^{100} \approx 1-0.3660 = 0.63397
        \end{equation*}
         
         \item Calcular el número esperado de peces anillados en la segunda captura.

         Para la binomial, tenemos que:
         \begin{equation*}
             E[X]=np= 100\cdot 0.01 = 1
         \end{equation*}

     \end{enumerate}
\end{ejercicio}




\begin{ejercicio}
     Un comerciante de bombillas las recibe en lotes de 20 unidades. Solo acepta un lote si, al seleccionar aleatoriamente 5 bombillas del mismo, no encuentra ninguna defectuosa.

     Si un determinado lote tiene dos bombillas defectuosas, calcular la probabilidad de que el comerciante lo acepte, y el número esperado de bombillas defectuosas entre las seleccionadas, en cada uno de los siguientes casos:
     \begin{itemize}
         \item Las bombillas se seleccionan con reemplazamiento.

        En este caso, tenemos la población de $N=20$ bombillas dividida en 2 poblaciones. En primer lugar, tenemos $N_1=2$ bombillas defectuosas, y $N-N_1$ bombillas correctas.

        Sea $X$ la variable aleatoria que determina el número de bombillas defectuosas que hay en una muestra de 5 elegida sin reemplazamiento. Tenemos que $X\leadsto~H(20, 2, 5)$. Entonces, tenemos que:
        \begin{equation*}
            p[X=0]=\frac{\binom{2}{0}\binom{18}{5}}{\binom{20}{5}} = \frac{18!\cdot 5!\cdot 15!}{20!\cdot 13!\cdot 5!} = \frac{15\cdot 14}{20\cdot 19} = \frac{21}{38} \approx 0.5526
        \end{equation*}
        \begin{equation*}
            E[X]=n\cdot \frac{N_1}{N} = 5\cdot \frac{2}{20} = 0.5
        \end{equation*}

        Por tanto, tenemos que la probabilidad de que no haya ninguna bombilla defectuosa en el lote y, por tanto, se acepte, es de $p[X=0]=0.5526$. Además, el número esperado de bombillas defectuosas es $0.5$; es decir, 1 o 2 bombillas defectuosas.

         
         \item Las bombillas se seleccionan sin reemplazamiento.

         La probabilidad de elegir una bombilla defectuosa de las 20 viene dada por la Regla de Laplace, y es:
         \begin{equation*}
             p=\frac{2}{20} = 0.1
         \end{equation*}

         Sea $X$ la variable aleatoria que determina el número de bombillas defectuosas que hay en una muestra de 5 elegida con reemplazamiento, donde la probabilidad de elegir una defectuosa es $p=0.1$. Tenemos que $X\leadsto~B(5, 0.1)$. Entonces, tenemos que:
        \begin{equation*}
            p[X=0]=0.5905
        \end{equation*}
        \begin{equation*}
            E[X]=np=5\cdot 0.1=0.5
        \end{equation*}

        Por tanto, tenemos que la probabilidad de que no haya ninguna bombilla defectuosa en el lote y, por tanto, se acepte, es de $p[X=0]=0.5905$. Además, el número esperado de bombillas defectuosas es $0.5$; es decir, 1 o 2 bombillas defectuosas.

        \begin{observacion}
            También se podría haber hecho con la variable $X'$ que determinase el número de bombillas correctas hasta la primera defectuosa. Entonces, tendríamos que $X'\leadsto BN(1,\;0.1)$. La probabilidad pedida sería $P[X'=5]$.
        \end{observacion}
     \end{itemize}
\end{ejercicio}



\begin{ejercicio}
    Se estudian las plantas de una determinada zona donde ha atacado un virus. La probabilidad de que cada planta esté contaminada es $0.35$.
    \begin{enumerate}
        \item  Definir la variable que modeliza el experimento de elegir una planta al azar y comprobar si está contaminada. Dar su ley de probabilidad.

        Tenemos que $X$ es la variable aleatoria que determina si la planta está contaminada o no, de forma que:
        \begin{equation*}
            X:\left\{\begin{array}{cc}
                \text{``Está contaminada''} & \longmapsto 1 \\
                \text{``No está contaminada''} & \longmapsto 0 \\
            \end{array}\right.    
        \end{equation*}

        Por tanto, tenemos que se trata de un experimento de Bernouilli con $p=~0.35$. Dar su ley de probabilidad implica describir la distribución de Bernouilli. Tenemos que:
        \begin{equation*}
            P(x)=p^x(1-p)^{1-x}
        \end{equation*}
        \begin{equation*}
            m_k=E[X^k] = \sum_{x=0}^1 x^k P[X=x] = 0^k\cdot (1-p) + 1^k\cdot p = p
        \end{equation*}
        \begin{multline*}
            \mu_k=E[(X - E[X])^k] =
            E[(X - p)^k]
            = \sum_{x=0}^1 (x-p)^k P[X=x] = (-p)^k\cdot (1-p) + (1-p)^k\cdot p
            =\\= p(1-p)^k +(-p)^k(1-p)
        \end{multline*}
        \begin{equation*}
            Var[X]=E[X^2]-E[X]^2 = p-p^2 = p(1-p)
        \end{equation*}
        
        \item ¿Cuál es el número medio de plantas contaminadas que se pueden esperar en 5 plantas analizadas?

        Sea $X$ la variable aleatoria que determina el número de plantas contaminadas en 5 plantas analizadas. Tenemos que la probabilidad se mantiene constante, por lo que $X\leadsto B(5,0.35)$. En este caso, la esperanza es:
        \begin{equation*}
            E[X]=np = 5\cdot 0.35 = \frac{7}{4}
        \end{equation*}
        
        \item Calcular la probabilidad de encontrar 8 plantas contaminadas en 10 exámenes.

        Sea $X$ la variable aleatoria que determina el número de plantas contaminadas en 10 plantas analizadas. Tenemos que la probabilidad se mantiene constante, por lo que $X\leadsto B(10,0.35)$. Por tanto,
        \begin{equation*}
            P[X=8]=0.0043
        \end{equation*}
        
        \item Calcular la probabilidad de encontrar entre 2 y 5 plantas contaminadas en 9 exámenes.

        Sea $X$ la variable aleatoria que determina el número de plantas contaminadas en 9 plantas analizadas. Tenemos que la probabilidad se mantiene constante, por lo que $X\leadsto B(9,0.35)$.
        \begin{multline*}
            P[2\leq X\leq 5] = P[X=2] + P[X=3] + P[X=4] + P[X=5]
            =\\= 0.2162 + 0.2716 + 0.2194 + 0.1181 = 0.8253
        \end{multline*}

        
        \item Hallar la probabilidad de que en 6 análisis se encuentren 4 plantas no contaminadas.

        Sea $X$ la variable aleatoria que determina el número de plantas contaminadas en 6 plantas analizadas. Tenemos que la probabilidad se mantiene constante, por lo que $X\leadsto B(6,0.35)$.
        
        Si hay 4 no contaminadas, tenemos que hay 2 contaminadas. Por tanto,
        \begin{equation*}
            P[X=2]=0.3280
        \end{equation*}
        
    \end{enumerate}
\end{ejercicio}



\begin{ejercicio}
 Cada página impresa de un libro contiene 40 líneas, y cada línea contiene 75 posiciones de impresión. Se supone que la probabilidad de que en cada posición haya error es $1/6000$.
    \begin{enumerate}
        \item ¿Cuál es la distribución del número de errores por página?

        Determinemos cuántas posiciones de impresión hay en una página:
        \begin{equation*}
            1 \text{ página } \cdot \frac{40 \text{ líneas }}{1 \text{ página }} \cdot \frac{75 \text{ posiciones de impresión }}{1 \text{ línea }} = 3\cdot 10^3 \text{ posiciones de impresión.}
        \end{equation*}

        Sea $X$ la variable aleatoria que determina el número de errores en una página, sabiendo que la probabilidad de que haya un error en una posición es de $1/6000$. Tenemos que $X\leadsto B(3\cdot 10^3, 1/6000)$.


        \item Calcular la probabilidad de que una página no contenga errores y de que contenga como mínimo 5 errores.
        \begin{equation*}
            P[X=0]=\binom{3\cdot 10^3}{0}\cdot \left(\frac{1}{6000}\right)^0 \left(\frac{5999}{6000}\right)^{3\cdot 10^3} \approx 0.6065
        \end{equation*}
        \begin{equation*}\begin{split}
            P[X\geq 5]
            &=1-P[X\leq 4]
            =1-P[X=0]-P[X=1]-\\ &\hspace{2cm}- P[X=2]-P[X=3]-P[X=4] =\\
            &= 1-P[X=0]
            -
            \binom{3\cdot 10^3}{1}\cdot \left(\frac{1}{6000}\right)^1 \left(\frac{5999}{6000}\right)^{2999} -
            \\&\hspace{1cm} -
            \binom{3\cdot 10^3}{2}\cdot \left(\frac{1}{6000}\right)^2 \left(\frac{5999}{6000}\right)^{2998}
            -
            \binom{3\cdot 10^3}{3}\cdot \left(\frac{1}{6000}\right)^3 \left(\frac{5999}{6000}\right)^{2997} -
            \\&\hspace{1cm}
            -
            \binom{3\cdot 10^3}{4}\cdot \left(\frac{1}{6000}\right)^4 \left(\frac{5999}{6000}\right)^{2996} \approx\\
            &\approx 1-0.6065 - 0.3033-0.0758 - 0.0126-0.00158 = 0.1716\cdot 10^{-3}
        \end{split}\end{equation*}

        En este caso, lo resolvemos también mediante una aproximación a la Poisson. Como $np=0.5\leq 5$, podemos aproximarlo a una Poisson de $\lambda=np=0.5$. $X\leadsto B(np)\cong \mathcal{P}(0.5)$.
        \begin{equation*}
            P[X=0]=0.6065
        \end{equation*}
        \begin{equation*}\begin{split}
            P[X\geq 5]
            &=1-P[X\leq 4]
            =1-P[X=0]-P[X=1]-\\ &\hspace{2cm}- P[X=2]-P[X=3]-P[X=4] =\\
            & = 1-\sum_{x=0}^4 \frac{e^{-0.5}\cdot 0.5^x}{x!} \approx 0.1721\cdot 10^{-3}
        \end{split}\end{equation*}
        
        \item ¿Cuál es la probabilidad de que un capítulo de 20 páginas no contenga ningún error?

        Sea $X$ la variable aleatoria que determina el número de errores en 20 páginas, sabiendo que la probabilidad de que haya un error en una posición es de $1/6000$. Tenemos que $X\leadsto B(6\cdot 10^4, 1/6000)$.

        Entonces,
        \begin{equation*}
            P[X=0] = \binom{6\cdot 10^4}{0}\cdot \left(\frac{1}{6000}\right)^0 \cdot \left(\frac{5999}{6000}\right)^{6\cdot 10^4}=45.3621\cdot 10^{-6}
        \end{equation*}
    \end{enumerate}
\end{ejercicio}



\begin{ejercicio}
    Se lanzan cuatro monedas 48 veces. ¿Cuál es la probabilidad de obtener exactamente 4 caras cinco veces?

    Sea el experimento de Bernouilli lanzar cuatro monedas, y consideramos como éxito obtener las 4 caras. Tenemos que esa probabilidad, por ser los 4 lanzamientos independientes, es $p=\frac{1}{2^4}$.
    
    Sea ahora $X$ la variable aleatoria que determina el número de éxitos en dicho experimento de Bernouilli en 48 repeticiones. Tenemos que $X\leadsto B(48, p)$. Por tanto:
    \begin{equation*}
        P[X=5]=\binom{48}{5}p^5(1-p)^{43}=0.1018
    \end{equation*}
\end{ejercicio}


\begin{ejercicio}
     Un pescador desea capturar un ejemplar de sardina que se encuentra siempre en una determinada zona del mar con probabilidad 0.15. Hallar la probabilidad de que tenga que pescar 10 peces de especies distintas de la deseada antes de:
     \begin{enumerate}
         \item pescar la sardina buscada,

         Sea $X$ el número de peces de distintas especies distintas de la deseada que ha de pescar antes de pescar la sardina buscada. La probabilidad de pescar la sardina buscada es de $0.15$. Tenemos que $X\leadsto BN(1,\;0.15)$. Tenemos que:
         \begin{equation*}
            P[X=10]=\binom{10}{10}0.15^1 \cdot 0.85^{10}\approx 0.0295
         \end{equation*}
         
         \item pescar tres ejemplares de la sardina buscada.

         Sea $X$ el número de peces de distintas especies distintas de la deseada que ha de pescar antes de pescar tres ejemplares de la sardina buscada. La probabilidad de pescar la sardina buscada es de $0.15$. Tenemos que $X\leadsto BN(3,\;0.15)$. Tenemos que:
         \begin{equation*}
            P[X=10]=\binom{12}{10}0.15^3 \cdot 0.85^{10}\approx 0.043854
         \end{equation*}
     \end{enumerate}
\end{ejercicio}


\begin{ejercicio}
    Un científico necesita 5 monos afectados por cierta enfermedad para realizar un experimento. La incidencia de la enfermedad en la población de monos es siempre del 30 \%. El científico examinará uno a uno los monos de un gran colectivo, hasta encontrar 5 afectados por la enfermedad.
    \begin{enumerate}
        \item Calcular el número medio de exámenes requeridos.

        Como consideramos que es un gran colectivo y afirma explícitamente confirma que la probabilidad de que un mono esté infectado siempre es de $0.3$, podemos suponer que no se trata de una dispersión hipergeométrica.

        Sea $X$ el número de monos examinados sanos antes de encontrar el 5º mono afectado por la enfermedad. Tenemos que $X\leadsto BN(5,\;0.3)$. Tenemos que:
        \begin{equation*}
            E[X]=\frac{r(1-p)}{p}=\frac{5\cdot 0.7}{0.3} = 11.\bar{6}
        \end{equation*}

        Por tanto, el número de monos sanos examinados son, de media, $11.\bar{6}$.

        Por tanto, el número medio de exámenes requeridos será:
        \begin{equation*}
            E[X]+5 = 16.\bar{6}
        \end{equation*}

        
        \item Calcular la probabilidad de que tenga que examinar por lo menos 20 monos.

        Como 5 monos serán siempre sanos, tenemos que buscamos la probabilidad de $X\geq 15$.
        \begin{equation*}
            P[X\geq 15] = 1-P[X\leq 14] = 1-\sum_{k=0}^{14} \binom{k+4}{k}0.7^{x}\cdot 0.3^5 \approx 0.2822
        \end{equation*}

        Alternativamente, podemos definir una variable aleatoria $Y$ que determine el número de monos afectados en los 19 primeros. Para que haya como mínimo 20 exámenes, necesitamos que entre esos 19 monos haya menos de 5 monos afectados. Por tanto, como $Y\leadsto (19, 0.3)$, tenemos:
        \begin{equation*}
            P[Y<5] = P[Y\leq 4] = \sum_{k=0}^4 \binom{19}{k}0.3^{k}\cdot 0.7^{19-k} \approx0.2822
        \end{equation*}
        
        \item Calcular la probabilidad de que encuentre 10 monos sanos antes de encontrar los 5 afectados.

        Sea $X$ el número de monos examinados sanos antes de encontrar el 5º mono afectado por la enfermedad. Tenemos que $X\leadsto BN(5,\;0.3)$.
        \begin{equation*}
            P[X=10] = \binom{14}{10}0.3^{5}\cdot 0.7^{10} \approx0.06871
        \end{equation*}
    \end{enumerate}
\end{ejercicio}


\begin{ejercicio}
    Para controlar la calidad de un determinado artículo que se fabrica en serie, se inspecciona diariamente el 5 \% de la producción. Un día la máquina sufre una avería y, de los 1000 artículos fabricados ese día, produce $k$ defectuosos.
    \begin{enumerate}
        \item Dar la expresión de la probabilidad de no obtener más de un artículo defectuoso en la inspección de ese día.

        

        Tenemos que la población total es $N=1000$, y se han producido $N_1=k$ defectuosos. Además, tenemos que la muestra examinada es $n=1000\cdot 5\% = 50$.

        Sea $X$ una variable aleatoria que determina el número de productos defectuosos de la muestra de 50. Tenemos que $X\leadsto H(N,N_1,n)$. Tenemos que la probabilidad de no obtener más de un artículo defectuoso es:
        \begin{equation*}
            P[X\leq 1]=\sum_{x=0}^1 P[X=x] = \sum_{x=0}^1 \frac{\binom{k}{x}\binom{1000-k}{50-x}}{\binom{1000}{50}}
        \end{equation*}
        
        \item Si $k = 90$, calcular la probabilidad de obtener menos de 6 artículos defectuosos en la inspección.

        Como $N_1=k=90\leq 0.1N=100$, podemos aproximar $X$ como una binomial de parámetro $p=\frac{N_1}{N}=0.09$. Tenemos que $X\leadsto (N,N_1,n)\cong B(n,p)$. Tenemos que:
        \begin{equation*}
            P[X< 6]=\sum_{x=0}^5 \binom{50}{x}0.09^x\cdot 0.91^{50-x} \approx 0.7072
        \end{equation*}
    \end{enumerate}
\end{ejercicio}


\begin{ejercicio}
    En una central telefónica de una ciudad se recibe un promedio de 480 llamadas por hora. Se sabe que el número de llamadas se distribuye según una ley de Poisson. Si la central sólo tiene capacidad para atender a lo sumo doce llamadas por minuto, ¿cuál es la probabilidad de que en un minuto determinado no sea posible dar línea a todos los clientes?

    Calculamos en primer lugar cuántas llamadas hay por minuto:
    \begin{equation*}
        480 \;\frac{\text{ llamadas}}{\text{ hora}} \cdot \frac{1 \text{ hora}}{60 \text{ minutos}} = 8\;\frac{\text{ llamadas}}{\text{ minuto}}
    \end{equation*}

    Por tanto, sea $X$ la variable que determina las llamadas que se reciben en un minuto. Tenemos que $X\leadsto \cc{P}(8)$.

    Para que no se pueda dar línea a todos los clientes, se han de recibir 13 o más llamadas. Por tanto,
    \begin{equation*}
        P[X\geq 13] = 1-P[X\leq 12] = 1-e^{-8}\sum_{k=0}^{12} \frac{8^k}{k!} \approx 0.063797
    \end{equation*}
\end{ejercicio}

\begin{ejercicio}
    Cierta compañía de seguros ha determinado que una de cada 5000 personas fallecen al año por accidente laboral. La compañía tiene hechos 50000 seguros de vida en toda la nación y, en caso de accidente, debe abonar 3000 euros por póliza. ¿Cuál es la probabilidad de que la compañía tenga que pagar en un año por lo menos 36000 euros en concepto de primas?

    Pagar 36000 euros en primas corresponde a $\frac{36000}{3000}=12$ fallecimientos de asegurados al año. Sea $X$ la variable aleatoria que determina el número de fallecimientos de personas aseguradas al año. Tenemos que $X\leadsto B(5\cdot 10^4, 1/5000)$. Tenemos que:
    \begin{equation*}
        P[X\geq 12]=1-\sum_{k=0}^{11} \binom{5\cdot 10^4}{x}\cdot \left(\frac{1}{5000}\right)^{x} \left(\frac{4999}{5000}\right)^{50000-x}=0.3032
    \end{equation*}
\end{ejercicio}


\begin{ejercicio}
    Suponiendo que, en cada parto, la probabilidad de que nazca una niña es 0.51, y prescindiendo de nacimientos múltiples, calcular:
    \begin{enumerate}
        \item Probabilidad de que un matrimonio tenga tres hijos varones antes de tener una niña.

        Sea $X$ la variable aleatoria que determina el número de niños antes de tener una niña. Tenemos que $X\leadsto BN(1, 0.51)$.
        \begin{equation*}
            P[X=3]=\binom{3}{3}\cdot 0.51^1 \cdot 0.49^{3} \approx 0.06   
        \end{equation*}
        
        \item Probabilidad de que tenga tres hijos varones antes de tener la segunda niña.

        Sea $X$ la variable aleatoria que determina el número de niños antes de tener dos niñas. Tenemos que $X\leadsto BN(2, 0.51)$.
        \begin{equation*}
            P[X=3]=\binom{4}{3}\cdot 0.51^2 \cdot 0.49^{3} \approx 0.1224   
        \end{equation*}
        
        \item ¿Cuál es el número medio de hijos que debe tener un matrimonio para conseguir dos niñas?

        Sea $X$ la variable aleatoria que determina el número de niños antes de tener dos niñas. Tenemos que $X\leadsto BN(2, 0.51)$.
        \begin{equation*}
            E[X]=\frac{r(1-p)}{p} = \frac{2\cdot 0.49}{0.51} \approx 1.9216  
        \end{equation*}

        Por tanto, de media se requieren $1.9216$ hijos de varones antes de conseguir 2 niñas.
        
    \end{enumerate}
\end{ejercicio}



\begin{ejercicio}
    El 60\% de los clientes de un almacén paga con dinero, el 30\% con tarjeta y el 10\% con cheque.
    \begin{enumerate}
        \item Calcular la probabilidad de que, de diez clientes, cuatro paguen con dinero.

        Sea $X$ una variable aleatoria que determina el número de personas que pagan con dinero de un total de 10 clientes. Tenemos que $X\leadsto B(10, 0.6)$.
        \begin{equation*}
            P[X=4]=\binom{10}{4}0.6^4\cdot 0.4^6 \approx 0.1115
        \end{equation*}
        
        \item Calcular la probabilidad de que el décimo cliente sea el cuarto en pagar con dinero.

        Si el décimo cliente es el cuarto en pagar con dinero, previamente ha habido 6 que no han pagado con dinero.

        Sea $X$ la variable aleatoria que determinan el número de personas que no pagan con dinero antes de que 4 paguen con dinero. Tenemos que $X\leadsto BN(4, 0.6)$.
        \begin{equation*}
            P[X=6]=\binom{9}{6}\cdot 0.6^4\cdot 0.4^6 \approx 0.04459
        \end{equation*}

        
    \end{enumerate}
\end{ejercicio}



\begin{ejercicio}
    En un departamento de control de calidad se inspeccionan las unidades terminadas que provienen de una línea de ensamble. La probabilidad de que cada unidad sea defectuosa es $0.05$.
    \begin{enumerate}
        \item  ¿Cuál es la probabilidad de que la vigésima unidad inspeccionada sea la segunda que se encuentra defectuosa?

        Sea $X$ la variable aleatoria que determina el número de unidades adecuadas antes de la segunda defectuosa. Tenemos que $X\leadsto BN(2, 0.05)$.

        Como la vigésima unidad es la segunda defectuosa, previamente han llegado 18 correctas.
        \begin{equation*}
            P[X=18]=\binom{19}{18}\cdot 0.05^2 \cdot 0.95^{18} \approx 0.018868
        \end{equation*}
        
        \item  ¿Cuántas unidades deben inspeccionarse por término medio hasta encontrar cuatro defectuosas?

        Sea $X$ la variable aleatoria que determina el número de unidades adecuadas antes de la cuarta defectuosa. Tenemos que $X\leadsto BN(4, 0.05)$.
        \begin{equation*}
            E[X]=\frac{4\cdot 0.95}{0.05}=76
        \end{equation*}

        Por tanto, de media se examinarán 76 unidades adecuadas, por lo que en total 80 unidades serán aproximadas.
        
        \item  Calcular la desviación típica del número de unidades inspeccionadas hasta encontrar cuatro defectuosas.
        \begin{equation*}
            Var[X+4]=Var[X]=\frac{4\cdot 0.95}{0.05^2}=1520 \Longrightarrow \sigma_{X+4}=\sqrt{1520}\approx 38.987
        \end{equation*}
    \end{enumerate}
\end{ejercicio}


\begin{ejercicio}
     Se supone que la demanda de un cierto fármaco en una farmacia sigue una ley de Poisson con una demanda diaria media de 8 unidades. ¿Qué stock debe tener el farmacéutico al comienzo del día para tener, como mínimo, probabilidad $0.99$ de satisfacer la demanda durante el día?

     Sea $X$ la variable aleatoria que determina la demanda en la tienda en un día. Tenemos que $X\leadsto \cc{P}(8)$. Se pide el percentil 99.
     
     \begin{equation*}
         P[X\leq P_{0.99}]\geq 0.99
         \Longleftrightarrow e^{-8}\sum_{k=0}^{P_{0.99}} \frac{8^k}{k!}\geq 0.99
         \Longleftrightarrow \sum_{k=0}^{P_{0.99}} \frac{8^k}{k!}\geq 0.99\cdot e^8 \approx 2951.148
     \end{equation*}

     Tenemos que $P_{0.99}=15$ cumple dicha condición (se ha determinando probando con valores naturales, a ``fuerza bruta''). Comrpobemos la siguiente condición:
     \begin{equation*}
         P[X\geq 15]\geq 0.01
         \Longleftrightarrow 
         1-P[X< 15]\geq 0.01
         \Longleftrightarrow 
        P[X< 15]\leq 0.99
        \Longleftrightarrow 
        e^{-8}\sum_{k=0}^{14}\frac{8^k}{k!}\leq 0.99
     \end{equation*}

     Tenemos que es cierto, por lo que confirmamos que el valor pedido para poder satisfacer la demanda con una probabilidad de $0.99$ es $P_{99}=15$ fármacos.
\end{ejercicio}


\begin{ejercicio}
     Los números $1,\dots,10$ se escriben en diez tarjetas y se colocan en una urna. Las tarjetas se extraen una a una y sin devolución. Calcular las probabilidades de los siguientes sucesos:
     \begin{enumerate}
         \item Hay exactamente tres números pares en cinco extracciones.

         Sea $X$ una variable aleatoria que determina cuántos números pares hay en cinco extracciones. Tenemos que $X\leadsto H(10, 5, 5)$, donde la población total son los 10 números, $N_1$ son los 5 números pares y $5$ son la muestra escogida. Por tanto,
         \begin{equation*}
             P[X=3]=\frac{\binom{5}{3}\binom{5}{2}}{\binom{10}{5}} = \frac{25}{63}\approx 0.3968
         \end{equation*}
         
         \item Se necesitan cinco extracciones para obtener tres números pares.

         Sean los siguientes sucesos:
         \begin{itemize}
             \item $A\longrightarrow $ Se extraen 4 pares y dos impares en las 4 primeras extracciones.

             \item $B\longrightarrow $ Sale par en la última extracción.
         \end{itemize}
         
         Tenemos que:
         \begin{equation*}
             P(A\cap B)=P(A)\cdot P(B|A)
         \end{equation*}

         Por la Ley de Laplace, tenemos que:
         \begin{equation*}
             P(B|A)=\frac{(10-4)/2}{10-4} = \frac{1}{2}
         \end{equation*}

         Para calcular la probabilidad de que se obtengan 2 números pares y 2 impares en las primeras 4 extracciones, trabajamos con la siguiente variable aleatoria. Sea $X$ una variable aleatoria que determina cuántos números pares hay en cuatro extracciones. Tenemos que $X\leadsto H(10, 5, 4)$, donde la población total son los 10 números, $N_1$ son los 5 números pares y $4$ son la muestra escogida. Por tanto, la probabilidad buscada es:
         \begin{equation*}
             P(A)=P[X=2]= \frac{\binom{5}{2}\binom{5}{2}}{\binom{10}{4}} = \frac{10}{21}\approx 0.4762
         \end{equation*}

         Por tanto, tenemos que:
         \begin{equation*}
             P(A\cap B)=P(A)\cdot P(B|A) = \frac{5}{21}\approx 0.238
         \end{equation*}

         
         \item Obtener el número 7 en la cuarta extracción.

         Sean los siguientes sucesos:
         \begin{itemize}
             \item $A\longrightarrow $ No extraer el 7 hasta la 4 extracción.

             \item $B\longrightarrow $ Extraer el 7 en la cuarta extracción.
         \end{itemize}
         
         Tenemos que:
         \begin{equation*}
             P(A\cap B)=P(A)\cdot P(B|A)
         \end{equation*}

         Por la Ley de Laplace, tenemos que:
         \begin{equation*}
             P(B|A)=\frac{1}{7}
         \end{equation*}

         Para calcular la probabilidad de que no se extraiga el 7 en las primeras 3 extracciones, trabajamos con la siguiente variable aleatoria. Sea $X$ una variable aleatoria que determina cuántos 7 hay en las primeras tres extracciones. Tenemos que $X\leadsto H(10, 1, 3)$, donde la población total son los 10 números, $N_1$ es el 7 y $3$ son la muestra escogida. Por tanto, la probabilidad buscada es:
         \begin{equation*}
             P(A)=P[X=0]= \frac{\binom{1}{0}\binom{9}{3}}{\binom{10}{3}} = \frac{7}{10}
         \end{equation*}

         Por tanto, tenemos que:
         \begin{equation*}
             P(A\cap B)=P(A)\cdot P(B|A) = \frac{1}{10} = 0.1
         \end{equation*}
     \end{enumerate}
\end{ejercicio}

\begin{ejercicio}
    Supongamos que el número de televisores vendidos en un comercio durante un mes se distribuye según una Poisson de parámetro 10, y que el beneficio neto por unidad es 30 euros.
    \begin{enumerate}
        \item ¿Cuál es la probabilidad de que el beneficio neto obtenido por un comerciante durante un mes sea al menos de 360 euros?

        Beneficio neto de 360 euros equivale a $\frac{360}{30}=12$ televisores.

        Sea $X$ una variable aleatoria que determina el número de televisores vendidos en un mes. Tenemos que $X\leadsto \mathcal{P}(10)$. Entonces,
        \begin{equation*}
            P[X\geq 12] = 1-P[X\leq 11] = 1-e^{-10}\sum_{k=0}^{11}\frac{10^k}{k!} = 0.3032
        \end{equation*}
        
        \item ¿Cuántos televisores debe tener el comerciante a principio de mes para tener al menos probabilidad $0.95$ de satisfacer toda la demanda?

        Nos pide calcular el percentil 95.
        \begin{equation*}
            P[X\leq P_{95}]\geq 0.95 \Longleftrightarrow e^{-10}\sum_{k=0}^{P_{99}}\frac{10^k}{k!} \geq 0.95
        \end{equation*}

        Tenemos que $P_{95}=15$.
        \begin{equation*}
            P[X\geq 15]\geq 0.05 \Longleftrightarrow 1 - e^{-10}\sum_{k=0}^{14}\frac{10^k}{k!} \geq 0.05
        \end{equation*}

        Por tanto, se confirma que el percentil 95 es $P_{95}=15$. Deberá tener 15 televisores para cubrir adecuadamente la demanda con la probabilidad deseada.
    \end{enumerate}
\end{ejercicio}

\begin{ejercicio}
    El número de accidentes que se producen semanalmente en una fábrica sigue una ley de Poisson, y se sabe que la probabilidad de que ocurran cinco accidentes en una semana es $16/15$ de la probabilidad de que ocurran dos. Calcular:
    \begin{enumerate}
        \item Media del número de accidentes por semana.

        Sea $X$ una variable aleatoria que determina el número de accidentes que ocurren en dicha fábrica en una semana. Tenemos que $X\leadsto \cc{P}(\lambda)$. Para calcular $\lambda$, sabemos que:
        \begin{equation*}
            P[X=5]=\frac{16}{15}P[X=2] \Longleftrightarrow e^{-\lambda}\cdot \frac{\lambda^5}{5!} = \frac{16}{15} e^{-\lambda}\cdot \frac{\lambda^2}{2!}
            \Longleftrightarrow \frac{1}{64}\cdot \lambda^5 = \lambda^2
        \end{equation*}

        Como $\lambda>0$, tenemos que esto se da si y solo si $\lambda^3 = 64\Longleftrightarrow \lambda = 4$.

        Por tanto, tenemos que $X\leadsto \cc{P}(4)$. Por tanto, tenemos que:
        \begin{equation*}
            E[X]=\lambda=4
        \end{equation*}
        
        \item Número máximo de accidentes semanales que pueden ocurrir con probabilidad no menor que $0.9$.

        En este caso, nos piden el percentil 90.
        \begin{equation*}
            P[X\leq P_{90}]\geq 0.9 \Longleftrightarrow
            e^{-4}\sum_{k=0}^{P_{90}}\frac{4^k}{4!}\geq 0.9
        \end{equation*}

        Tenemos que el valor buscado es $P_{90}=7$. Comprobemos que cumple la segunda condición:
        \begin{equation*}
            P[X\geq 7]\geq 0.1 \Longleftrightarrow
            1 - e^{-4}\sum_{k=0}^{6}\frac{4^k}{4!}\geq 0.1
        \end{equation*}

        Tenemos que es cierto, por lo que $P_{90}=7$. 
    \end{enumerate}
\end{ejercicio}