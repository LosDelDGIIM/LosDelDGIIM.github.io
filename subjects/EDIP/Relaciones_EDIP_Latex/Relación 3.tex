\section{Espacios de Probabilidad}


\begin{ejercicio} \label{ej:3.Ejercicio1}
    Durante un año, las personas de una ciudad utilizan 3 tipos de transportes: metro (M), autobús (A), y coche particular (C). Las probabilidades de que durante el año hayan usado unos u otros transportes son:
    
    M: 0.3; \;A: 0.2; \;C: 0.15; \;M y A: 0.1; \;M y C: 0.05; \;A y C: 0.06; \;M, A y C: 0.01

    Calcular las probabilidades siguientes:
    \begin{enumerate}
        \item Que una persona viaje en metro y no en autobús.
        \begin{equation*}
            P(M\cap \bar{A}) = P(M-A) = P(M) - P(A\cap M) = 0.3 - 0.1 = 0.2
        \end{equation*}
        
        \item Que una persona tome al menos dos medios de transporte.

        El suceso descrito es $S = (M\cap A) \cup (M\cap C) \cup (A\cap C) \cup (A\cap M \cap C)$. Como son sucesos incompatibles, tenemos que:
        \begin{equation*}\begin{split}
            P(S) &= P[(M\cap A) \cup (M\cap C) \cup (A\cap C) \cup (A\cap M \cap C)] =\\
            &= P[(M\cap A) \cup (M\cap C)] + P[(A\cap C) \cup (A\cap M \cap C)] 
            \\& \qquad \qquad - P[[(M\cap A) \cup (M\cap C)] \cap [(A\cap C) \cup (A\cap M \cap C)]] = \\
            &= P(M\cap A) + P(M\cap C) - P(M\cap A \cap C) + P(A\cap C) - P(A\cap C \cap M)  = \\
            &= 0.1 +0.05 - 0.01 +0.06 -0.01 = 0.19
        \end{split}\end{equation*}

        donde he usado que:
        \begin{equation*}
            (A\cap C) \cup (A\cap M \cap C) = A\cap C,\text{ ya que } (A\cap M \cap C) \subset (A\cap C)
        \end{equation*}
        \begin{equation*}
            [(M\cap A) \cup (M\cap C)] \cap [(A\cap C) \cup (A\cap M \cap C)] = [M\cap (A\cup C)] \cap (A\cap C) = A\cap C \cap M
        \end{equation*}
        
        \item Que una persona viaje en metro o en coche, pero no en autobús.
        \begin{equation*}\begin{split}
            P[(M\cup C) \cap \bar{A}]&= P[(M\cup C) - A] = P(M\cup C) - P[(M\cup C) \cap A] =\\&=
            P(M) + P(C) - P(M\cap C) - P[(M\cup C) \cap A] =
            \\&= 0.3 + 0.15 - 0.05 -0.15 = 0.25
        \end{split}\end{equation*}

        donde he tenido que usar que:
        \begin{equation*}\begin{split}
            P[(M\cup C) \cap A] &= P[(M\cap A) \cup (C\cap A)] =
            \\&=  P(M\cap A) + P(C\cap A) - P[(M\cap A) \cap (C\cap A)] =
            \\&=  P(M\cap A) + P(C\cap A) - P(A\cap M \cap C) =
            \\&= 0.1 + 0.06 - 0.01 = 0.15
        \end{split}\end{equation*}
        
        \item Que viaje en metro, o bien en autobús y en coche.
        \begin{equation*}
            \begin{split}
                P[M\cup (A\cap C)] &= P(M) + P(A\cap C) - P(A\cap C \cap M) =\\
                &= 0.3 + 0.06 - 0.01 = 0.35
            \end{split}
        \end{equation*}
        
        \item Que una persona vaya a pie.

        Se presupone que ir a pie es la única alternativa a los tres medios de transporte descritos. En ese caso,
        \begin{equation*}\begin{split}
            P&(\bar{A} \cap \bar{M} \cap \bar{C}) = P\left(\overline{A\cup M\cup C}\right) = 1-P(A\cup M\cup C) =\\
            &= 1-[P(A) + P(M\cup C) - P[A\cap (M\cup C)]] =\\
            &= 1-[P(A) + P(M) + P(C) - P(M\cap C) - P[A\cap (M\cup C)]] =\\
            &= 1-[0.2 + 0.3 + 0.15 - 0.05 - 0.15] = 1-0.45 = 0.55
        \end{split}\end{equation*}
    \end{enumerate}
\end{ejercicio}

\begin{ejercicio} \label{ej:3.Ejercicio2}
    Sean $A,B$ y $C$ tres sucesos de un espacio probabilístico $(\Omega,\cc{A}, P)$, tales que $P(A) = 0.4$, $P(B) = 0.2$, $P(C) = 0.3$, $P(A\cap B) = 0.1$ y $(A\cup B)\cap C = \emptyset$. Calcular las probabilidades de los siguientes sucesos:
    \begin{enumerate}
        \item Sólo ocurre $A$,
        \begin{multline*}
            P((A-B)-C)=P(A-B)-\cancelto{0}{P((A-B)\cap C)}=P(A-B)=\\=P(A)-P(A\cap B) = 0.4-0.1=0.3
        \end{multline*}
        donde he empleado que $(A-B)\cap C = \emptyset$. Esto se debe a que:
        \begin{equation*}
            \emptyset = (A\cup B)\cap C = (A\cap C) \cup (B\cap C) \Longrightarrow A\cap C=B\cap C=\emptyset
        \end{equation*}
        Como $A\cap C=\emptyset$ y $A-B\subset A$, entonces $[(A-B)\cap C]\subset [A\cap C] = \emptyset$.
        
        \item Ocurren los tres sucesos,

        Teniendo en cuenta que $B\cap C=\emptyset$,
        \begin{equation*}
            P(A\cap B\cap C)=P(A\cap \emptyset)=P(\emptyset)=0
        \end{equation*}
        
        \item Ocurren A y B pero no C,
        \begin{equation*}
        P((A\cap B)-C)=P(A\cap B)-\cancelto{0}{P(A\cap B \cap C)} = P(A\cap B)     = 0.1
        \end{equation*}
        
        \item Por lo menos dos ocurren,
        
        Teniendo en cuenta que $A\cap C = B\cap C = \emptyset$, tenemos que:
        \begin{equation*}
            P[(A\cap B) \cup (A\cap C) \cup (B\cap C) \cup (A\cap B\cap C)] = P(A\cap B) = 0.1 
        \end{equation*}
        
        \item Ocurren dos y no más,
        
        Sea $S$ el suceso,
        \begin{equation*}\begin{split}
            P(S)&= P\left[((A\cap B) \cup \cancelto{\emptyset}{(A\cap C)} \cup \cancelto{\emptyset}{(B\cap C)}) \cap (\overline{A\cap B \cap C})\right] \\
            &= P[A\cap B \cap (\overline{A\cap B \cap C})] = P(A\cap B \cap \overline{\emptyset}) =\\
            &= P(A\cap B \cap \Omega) = P(A\cap B) = 0.1
        \end{split}\end{equation*}
        
        \item No ocurren más de dos,
        
        Esto suceso equivale a que no ocurran los tres, es decir, $\overline{A\cap B \cap C}$.
        \begin{equation*}
            P\left(\overline{A\cap B \cap C}\right) = 1-P({A\cap B \cap C}) = 1-P(\emptyset) = 1
        \end{equation*}
        
        \item Ocurre por lo menos uno,

        Cabe destacar que este suceso es:
        \begin{equation*}
            A\cup B \cup C \cup (A\cap B) \cup (A\cap C) \cup (B\cap C) \cup (A\cap B \cap C) = A\cup B \cup C
        \end{equation*}
        \begin{equation}\label{Ej2.g}\begin{split}
            P(A\cup B \cup C) &= P(A\cup B) + P(C) - \cancelto{0}{P[(A\cup B) \cap C]}
            =\\&
            = P(A) + P(B) + P(C) - P(A\cap B)
            =\\&
            = 0.4 + 0.2 + 0.3 - 0.1 = 0.8
        \end{split}\end{equation}
        
        \item Ocurre sólo uno,

        Sea $S_i$ el suceso en el que sólo ocurre el suceso $i$, es decir,
        \begin{equation*}
            S_a = A\cap \bar{B} \cap \bar{C}
            \qquad
            S_b = \bar{A} \cap {B} \cap \bar{C}
            \qquad
            S_c = \bar{A} \cap \bar{B} \cap {C}
            \qquad
        \end{equation*}

        Además, como $A\cap C = B\cap C = \emptyset$, tenemos que:
        \begin{equation*}
            S_a = A\cap \bar{B}
            \qquad
            S_b = \bar{A} \cap {B}
            \qquad
            S_c = {C}
            \qquad
        \end{equation*}

        Como además tenemos que dichos sucesos son incompatibles, ya que no pueden ocurrir a la vez, tenemos que:
        \begin{equation*}\begin{split}
            P(S_a \cup S_b \cup S_c) &= P(S_a) + P(S_b) + P(S_c) 
            =\\&= P(A\cap \bar{B}) + P(\bar{A}\cap B) + P(C) 
            = P(A-B) + P(B-A) + P(C)
            =\\&= P(A) + P(B) + P(C) - P(A\cap B) - P(B\cap A) =\\&= 0.4+0.2+0.3 -2\cdot 0.1 = 0.7
        \end{split}\end{equation*}
        
        \item No ocurre ninguno.
        \begin{equation*}
            P(\bar{A} \cap \bar{B} \cap \bar{C}) = P(\overline{A\cup B \cup C}) = 1-P(A\cup B \cup C) \stackrel{Ec.\;\ref{Ej2.g}}{=} 1-0.8 = 0.2
        \end{equation*}
    \end{enumerate}
\end{ejercicio}

\begin{ejercicio} \label{ej:3.Ejercicio3}
    Se sacan dos bolas sucesivamente sin devolución de una urna que contiene 3 bolas rojas distinguibles y 2 blancas distinguibles.
    \begin{enumerate}
        \item Describir el espacio de probabilidad asociado a este experimento.

        Sea el experimento aleatorio sacar dos bolas sucesivamente sin devolución de una urna que contiene 3 bolas rojas distinguibles y 2 blancas distinguibles.

        Sean $r_{i},\;(i=1,2,3)$ sacar una de las tres bolas rojas y $b_{j},\;(j=1,2)$ sacar una de las dos bolas blancas. El espacio de probabilidad viene dado por:
        \begin{multline*}
            \Omega =
            \lbrace r_{i}r_{j}\mid i, j = 1, 2, 3; i \neq j \rbrace
            \cup
            \lbrace r_{i} b_{j}\mid i=1, 2, 3 ; j = 1, 2 \rbrace
            \cup\\\cup
            \lbrace b_{i} r_{j}\mid i = 1, 2 ; j = 1, 2, 3 \rbrace
            \cup
            \lbrace b_{i} b_{j}\mid i, j = 1, 2; i \neq j \rbrace
        \end{multline*}
        
        \item Descomponer en sucesos elementales los sucesos: \textit{la primera bola es roja}, \textit{la segunda bola es blanca} y calcular la probabilidad de cada uno de ellos.

        El suceso \textit{la primera bola es roja} se descompone como:
        \begin{equation*}
            1^a\text{roja}=\lbrace r_{i}r_{j}\mid i, j = 1, 2, 3; i \neq j \rbrace
            \cup
            \lbrace r_{i} b_{j}\mid i=1, 2, 3 ; j = 1, 2 \rbrace
        \end{equation*}

        La cantidad de sucesos totales de mi experimento son:
        \begin{equation*}
            V_{5,2}=\frac{5!}{(5-2)!}=5\cdot 4 = 20
        \end{equation*}

        La cantidad de sucesos posibles en los que las dos bolas son rojas son:
        \begin{equation*}
            V_{3,2}=\frac{3!}{(3-2)!}=6
        \end{equation*}

        La cantidad de sucesos posibles de la forma $\{r_ib_j\}\mid i=1,2,3,\;j=1,2$ son:
        \begin{equation*}
            6 \Longrightarrow  (r_1b_1,r_2b_1,r_3b_1, r_1b_2, r_2b_2,r_3b_2)
        \end{equation*}

        Por tanto, usando la Ley de Laplace y sabiendo que todos los sucesos elementales son incompatibles, tenemos que:
        \begin{multline*}
            P(1^a\text{roja})=P[\lbrace r_{i}r_{j}\mid i, j = 1, 2, 3; i \neq j \rbrace
            \cup
            \lbrace r_{i} b_{j}\mid i=1, 2, 3 ; j = 1, 2 \rbrace]
            =\\=
            P[\lbrace r_{i}r_{j}\mid i, j = 1, 2, 3; i \neq j \rbrace]
            +P[
            \lbrace r_{i} b_{j}\mid i=1, 2, 3 ; j = 1, 2 \rbrace]
            =\\=
            \frac{6}{20} + \frac{6}{20} = \frac{6}{10}=\frac{3}{5} = 0.6
        \end{multline*}

        \vspace{1cm}
        Por otro lado, el suceso \textit{la segunda bola es blanca} se descompone como:
        \begin{equation*}
            2^a\text{blanca}=\lbrace b_{i}b_{j}\mid i, j = 1, 2; i \neq j \rbrace
            \cup
            \lbrace r_{i} b_{j}\mid i=1, 2, 3 ; j = 1, 2 \rbrace
        \end{equation*}

        La cantidad de sucesos posibles en los que las dos bolas son blancas son:
        \begin{equation*}
            V_{2,2}=\frac{2!}{(2-2)!}=2
        \end{equation*}
        
        Por tanto, usando la Ley de Laplace y sabiendo que todos los sucesos elementales son incompatibles, tenemos que:
        \begin{multline*}
            P(2^a\text{blanca})=P[\lbrace b_{i}b_{j}\mid i, j = 1, 2; i \neq j \rbrace
            \cup
            \lbrace r_{i} b_{j}\mid i=1, 2, 3 ; j = 1, 2 \rbrace]
            =\\=
            P[\lbrace b_{i}b_{j}\mid i, j = 1, 2; i \neq j \rbrace]
            +P[
            \lbrace r_{i} b_{j}\mid i=1, 2, 3 ; j = 1, 2 \rbrace]
            =\\=
            \frac{2}{20} + \frac{6}{20} = \frac{8}{20}=\frac{2}{5} = 0.4
        \end{multline*}

        \begin{comment}
        La probabilidad de que la primera bola sea roja es la suma los resultados: \textit{la primera bola es roja y la segunda bola es blanca} y \textit{la primera bola es roja y la segunda bola es roja}:

        \begin{center}
            $P(1^{a} \text{ roja}) = P(1^{a}\text{ roja y } 2^{a} \text{ blanca}) + P(1^{a} \text{ roja y } 2^{a} \text{ roja}) = \frac{3}{5} \cdot \frac{2}{4} + \frac{3}{5} \cdot \frac{2}{4} = 0.3 + 0.3 = 0.6$
        \end{center}
        

        Por otra parte, a probabilidad de que la segunda bola sea blanca es la suma los resultados: \textit{la primera bola es roja y la segunda bola es blanca} y \textit{la primera bola es blanca y la segunda bola es blanca}:

        \begin{center}
            $P(2^{a} \text{ blanca}) = P(1^{a}\text{ roja y } 2^{a} \text{ blanca}) + P(1^{a} \text{ blanca y } 2^{a} \text{ blanca}) = \frac{3}{5} \cdot \frac{2}{4} + \frac{2}{5} \cdot \frac{1}{4} = 0.3 + 0.1 = 0.4$
        \end{center}
        
        \item ¿Cuál es la probabilidad de que ocurra alguno de los sucesos considerados en el apartado anterior?

        Sabiendo que:

        \begin{equation*}
            P(A\cup B) = P(A) + P(B) - P(A\cap B)
        \end{equation*}

        tenemos que:

        \begin{center}
            $P(1^{a} \text{ roja ó } 2^{a} \text{ blanca}) = P(1^{a} \text{ roja}) + P(2^{a} \text{ blanca}) - P(1^{a} \text{ roja y } 2^{a} \text{ blanca}) = 0.6 + 0.4 - P(r_{i}b_{j}) = 0.6 + 0.4 - \frac{3}{5} \cdot \frac{2}{4} = 0.7$
        \end{center}
        \end{comment}
        
    \end{enumerate}
\end{ejercicio}

\begin{ejercicio} \label{ej:3.Ejercicio4}
    Una urna contiene $a$ bolas blancas y $b$ bolas negras. ¿Cuál es la probabilidad de que al extraer dos bolas simultáneamente sean de distinto color?

    Sea $N$ sacar una bola negra y $B$ sacar una bola blanca. Tenemos que:
    \begin{equation*}
        \Omega = \{NN,BB,NB\}
    \end{equation*}

    En primer lugar, tenemos en cuenta no influye el orden en el que sacan las bolas. Además, como una misma bola no se puede sacar dos veces, estamos ante combinaciones sin repetición.   
    
    Los casos totales, sabiendo que tengo que hacer grupos de 2 de un total de ($a+b$) elementos, son:
    \begin{equation*}
        C_{2,a+b} = \binom{a+b}{2} = \frac{(a+b)!}{2!(a+b-2)!} = \frac{(a+b)(a+b-1)}{2}
    \end{equation*}

    Calculamos en primer lugar la probabilidad de los siguientes sucesos:
    \begin{enumerate}
        \item dos bolas negras,

        En este caso, el suceso es $\{NN\}\in \cc{A}$.

        Los casos favorables son las combinaciones de 2 elementos entre un total de $b$ bolas negras, es decir,
        \begin{equation*}
            C_{2,b} = \binom{b}{2} = \frac{b!}{2!\cdot (b-2)!} = \frac{b(b-1)}{2}
        \end{equation*}

        Por tanto, usando la regla de Laplace, tenemos que:
        \begin{equation*}
            P(NN)=\frac{C_{2,b}}{C_{2,a+b}} = \frac{b(b-1)}{(a+b)(a+b-1)}
        \end{equation*}
        
        \item dos bolas blancas,

        En este caso, el suceso es $\{BB\}\in \cc{A}$.
        
        Los casos favorables son las combinaciones de 2 elementos entre un total de $a$ bolas blancas, es decir,
        \begin{equation*}
            C_{2,a} = \binom{a}{2} = \frac{a!}{2!\cdot (a-2)!} = \frac{a(a-1)}{2}
        \end{equation*}

        Por tanto, usando la regla de Laplace, tenemos que:
        \begin{equation*}
            P(BB)=\frac{C_{2,a}}{C_{2,a+b}} = \frac{a(a-1)}{(a+b)(a+b-1)}
        \end{equation*}
    \end{enumerate}

    Por tanto, calculamos ahora la probabilidad de que las bolas extraídas sean de distinto color, es decir, $P(NB)$. Como los sucesos del espacio muestral son mutuamente excluyentes, tenemos que:
    \begin{multline*}
        1=P(\Omega) = P(NN\cup BB \cup NB) =P(NN)+P(BB)+P(NB)
        \Longrightarrow \\ \Longrightarrow
        P(NB)=1-\frac{a(a-1)}{(a+b)(a+b-1)}-\frac{b(b-1)}{(a+b)(a+b-1)} =\\= \frac{(a+b)(a+b-1)-a(a-1)-b(b-1)}{(a+b)(a+b-1)} = \frac{2ab}{(a+b)(a+b-1)}
    \end{multline*}

    Por tanto, tengo que:
    \begin{equation*}
        P(NB)=\frac{2ab}{(a+b)(a+b-1)}
    \end{equation*}

    \begin{observacion}
        Como podemos ver, las probabilidades coinciden con las calculadas en el ejercicio \ref{ej:3.Ejercicio5}.
    \end{observacion}
\end{ejercicio}

\begin{ejercicio} \label{ej:3.Ejercicio5}
    Una urna contiene 5 bolas blancas y 3 rojas. Se extraen 2 bolas simultáneamente. Calcular la probabilidad de obtener los siguientes sucesos.\\

    Sea $R$ sacar una bola roja y $B$ sacar una bola blanca. Tenemos que:
    \begin{equation*}
        \Omega = \{RR,BB,RB\}
    \end{equation*}

    En primer lugar, tenemos en cuenta no influye el orden en el que sacan las bolas. Además, como una misma bola no se puede sacar dos veces, estamos ante combinaciones sin repetición.   
    
    Los casos totales, sabiendo que tengo que hacer grupos de 2 de un total de 8 elementos, son:
    \begin{equation*}
        C_{2,8} = \binom{8}{2} = \frac{8!}{2!(8-2)!} = \frac{8\cdot 7 \cdot 6!}{2\cdot 6!} = 4\cdot 7 = 28
    \end{equation*}
    
    \begin{enumerate}
        \item dos bolas rojas,

        En este caso, el suceso es $\{RR\}\in \cc{A}$.

        Los casos favorables son las combinaciones de 2 elementos entre un total de 3 bolas rojas, es decir,
        \begin{equation*}
            C_{2,3} = \binom{3}{2} = \frac{3!}{2!\cdot 1!} = 3    
        \end{equation*}

        Por tanto, usando la regla de Laplace, tenemos que:
        \begin{equation*}
            P(RR)=\frac{C_{2,3}}{C_{2,8}} = \frac{3}{28} \approx 0.1071
        \end{equation*}
        
        \item dos bolas blancas,

        En este caso, el suceso es $\{BB\}\in \cc{A}$.
        
        Los casos favorables son las combinaciones de 2 elementos entre un total de 5 bolas blancas, es decir,
        \begin{equation*}
            C_{2,5} = \binom{5}{2} = \frac{5!}{2!\cdot 3!} = \frac{5\cdot 4}{2} = 10    
        \end{equation*}

        Por tanto, usando la regla de Laplace, tenemos que:
        \begin{equation*}
            P(BB)=\frac{C_{2,5}}{C_{2,8}} = \frac{10}{28} = \frac{5}{14} \approx 0.357
        \end{equation*}
        
        \item una blanca y otra roja.

        En este caso, se pide la probabilidad de $P(RB)$.  Como los sucesos del espacio muestral son mutuamente excluyentes, tenemos que:
        \begin{multline*}
            1=P(\Omega) = P(RR\cup BB \cup RB) =P(RR)+P(BB)+P(RB) \Longrightarrow \\ \Longrightarrow
            P(RB)=1-P(RR)-P(BB) = 1-\frac{3}{28}-\frac{5}{14}=\frac{15}{28}
        \end{multline*}

        Por tanto, tengo que:
        \begin{equation*}
            P(RB)=\frac{15}{28}\approx 0.5357
        \end{equation*}
    \end{enumerate}
\end{ejercicio}

\begin{ejercicio} \label{ej:3.Ejercicio6}
    En una lotería de 100 billetes hay 2 que tienen premio.
    \begin{enumerate}
        \item ¿Cuál es la probabilidad de ganar al menos un premio si se compran 12 billetes?

        Estamos ante una situación en la que no importa el orden de los billetes y es sin repetición, ya que a comprar un billete ya hay uno menos disponible. Por tanto, estamos ante combinaciones.

        El número total de combinaciones posibles es:
        \begin{equation*}
            C_{100, 12} = \binom{100}{12} = \frac{100!}{12! \cdot 88!}
        \end{equation*}

        Sea $A$ el suceso de ganar al menos con un billete. Por tanto, $\bar{A}$ es el suceso de no haber comprado ningún billete premiado.
        
        Las combinaciones no premiadas son:
        \begin{equation*}
            C_{98,12} = \binom{98}{12} = \frac{98!}{12!\cdot 86!}
        \end{equation*}
        
        Por tanto,
        \begin{multline*}
            P(A)=1-P(\bar{A}) = 1-\frac{C_{98,12}}{C_{100,12}} = 1-\frac{98! \cdot 12!\cdot 88!}{100!\cdot 12!\cdot 86!}
            = 1-\frac{88\cdot 87}{100\cdot 99} =\\= 1-\frac{58}{75} = \frac{17}{75} = 0.22\Bar{6}
        \end{multline*}

        
        \item ¿Cuantos billetes habría que comprar para que la probabilidad de ganar al menos un premio sea mayor que $4/5$?

        Sea $x$ el número de billetes comprado.
        Las combinaciones totales son:
        \begin{equation*}
            C_{100,x}=\frac{100!}{x!(100-x)!}
        \end{equation*}

        El número de combinaciones no premiadas son:
        \begin{equation*}
            C_{98,x}=\frac{98!}{x!(98-x)!}
        \end{equation*}

        Sea $A$ el suceso de ganar al menos un premio, por lo que sea $\bar{A}$ no ganar ningún premio. Como $P(A)\geq \frac{4}{5}=0.8$, tenemos que:
        \begin{equation*}
            P(A)\geq 0.8 \Longleftrightarrow 1-P(\bar{A}) \geq 0.8 \Longleftrightarrow P(\Bar{A})\leq 0.2
        \end{equation*}

        Además, por la Regla de Laplace, tenemos que:
        \begin{multline*}
            P(\bar{A})=\frac{C_{98,x}}{C_{100,x}} = \frac{98! \cdot x!(100-x)!}{100!\cdot x!(98-x)!}
            = \frac{(100-x)(99-x)}{100\cdot 99} \leq 0.2
            \Longleftrightarrow \\ \Longleftrightarrow
            100\cdot 99 -100x-99x+x^2 \leq 0.2(100\cdot 99)
            \Longleftrightarrow x^2-199x+7920\leq 0
        \end{multline*}

        Las raíces de esa parábola son $x_1=144$, $x_2=55$. Además, como se trata de una parábola cóncava hacia arriba, tenemos que es $\leq 0 \Longleftrightarrow x\in [55,144]$.

        No obstante, como tenemos que el número máximo de billetes disponible es de 100, tenemos que:
        \begin{equation*}
            P(A)\geq 0.8 \Longleftrightarrow x\geq 55
        \end{equation*}

        Por tanto, habría que comprar, como mínimo, 55 billetes.
    \end{enumerate}
\end{ejercicio}

\begin{ejercicio} \label{ej:3.Ejercicio7}
    Se consideran los 100 primeros números naturales. Se sacan 3 al azar.
    \begin{enumerate}
        \item Calcular la probabilidad de que en los 3 números obtenidos no exista ningún cuadrado perfecto.

        Llamemos a este suceso el \textit{suceso A}. Además, tenemos en cuenta que hay 10 cuadrados perfectos en $\bb{N}\cap [1,100]$.
        
        Como el orden en el que se sacan no importa, estamos ante combinaciones. Es necesario distinguir entre los casos en los que hay repetición y los que no:
        \begin{itemize}
            \item \underline{Si no hay repetición}:

            Los casos posibles son:
            \begin{equation*}
                C_{100,3}=\binom{100}{3}=\frac{100!}{3!\cdot 97!}
            \end{equation*}

            Los casos en los que no hay cuadrados perfectos son las combinaciones posibles de 3 elementos de un conjunto de 90 elementos $(100-10)$ sin repetición:
            \begin{equation*}
                C_{90,3}=\binom{90}{3}=\frac{90!}{3!\cdot 87!}
            \end{equation*}

            Por tanto, usando la Ley de Laplace, tenemos que:
            \begin{multline*}
                P(A)=\frac{C_{90,3}}{C_{100,3}}=\frac{90!\cdot 3!\cdot 97!}{100!\cdot 3!\cdot 87!}
                =\frac{90\cdot 89\cdot 88 \cdot 87!\cdot 97!}{100\cdot 99\cdot 98 \cdot 97!\cdot 87!}
                =\\=
                \frac{90\cdot 89\cdot 88}{100\cdot 99\cdot 98} = \frac{178}{245} \approx 0.72652
            \end{multline*}



            \item \underline{Si sí hay repetición}:

            Los casos posibles son:
            \begin{equation*}
                CR_{100,3}=\binom{100+3-1}{3}=\binom{102}{3}=\frac{102!}{3!\cdot 99!}
            \end{equation*}

            Los casos en los que no hay cuadrados perfectos son las combinaciones posibles de 3 elementos de un conjunto de 90 elementos $(100-10)$ con repetición:
            \begin{equation*}
                CR_{90,3}=\binom{90+3-1}{3}=\binom{92}{3}=\frac{92!}{3!\cdot 89!}
            \end{equation*}

            Por tanto, usando la Ley de Laplace, tenemos que:
            \begin{multline*}
                P(A)=\frac{CR_{90,3}}{CR_{100,3}}=\frac{92!\cdot 3!\cdot 99!}{102!\cdot 3!\cdot 89!}
                =\frac{92\cdot 91\cdot 90 \cdot 89!\cdot 99!}{102\cdot 101\cdot 100 \cdot 99!\cdot 89!}
                =\\=
                \frac{92\cdot 91\cdot 90}{102\cdot 101\cdot 100} = \frac{6279}{8585} \approx 0.73139
            \end{multline*}
        \end{itemize}

        \begin{comment}
        Si hay repetición:

        \begin{equation*}
            P(A) = \frac{90^{3}}{100^{3}} = 0.729
        \end{equation*}
        \end{comment}
        
        \item Calcular la probabilidad de que exista al menos un cuadrado perfecto.

        Llamemos a este suceso el \textit{suceso B}. Como tenemos que $B=\bar{A}$, tenemos que:
        \begin{equation*}
            P(B) = P(\bar{A}) = 1 - P(A)
        \end{equation*}

        Por tanto, deberemos distinguir de nuevo si hay o no repetición. Esto es, deberemos usar el resultado del apartado anterior calculado con o sin repetición.

        \begin{itemize}
            \item \underline{Si no hay repetición}:
            \begin{equation*}
                P(B) = P(\bar{A}) = 1 - P(A)=0.27348
            \end{equation*}

            \item \underline{Si sí hay repetición}:
            \begin{equation*}
                P(B) = P(\bar{A}) = 1 - P(A)=0.26860
            \end{equation*}
        \end{itemize}
        
        \item Calcular la probabilidad de que exista un sólo cuadrado perfecto, de que existan dos, y la de que los tres lo sean.

        Denotemos ahora por 1C, 2C y 3C el resultado de obtener 1, 2 ó 3 cuadrados perfectos, respectivamente. Nuevamente debemos distinguir si hay o no repetición.

        \begin{itemize}
            \item \underline{Si no hay repetición}:

            Los casos posibles son:
            \begin{equation*}
                C_{100,3}=\binom{100}{3}=\frac{100!}{3!\cdot 97!} = \frac{100\cdot 99\cdot 98}{6}
            \end{equation*}

            Los casos en los que sólo hay un cuadrado perfecto son las combinaciones posibles de 2 elementos de un conjunto de 90 elementos $(100-10)$ y de 1 elemento de 10 elementos sin repetición:
            \begin{equation*}
                C_{90,2}\cdot C_{10,1}=\binom{90}{2}\cdot \binom{10}{1}=\frac{90!}{2!\cdot 88!}\cdot 10
            \end{equation*}

            Por tanto, usando la Ley de Laplace, tenemos que:
            \begin{multline*}
                P(1C)=\frac{C_{90,2}\cdot C_{10,1}}{C_{100,3}}=\frac{90!\cdot 3!\cdot 97!\cdot 10}{100!\cdot 2!\cdot 88!}
                =\frac{90\cdot 89\cdot 88!\cdot 3\cdot 2!\cdot 97!\cdot 10}{100\cdot 99\cdot 98 \cdot 97!\cdot 2!\cdot 88!}
                =\\=
                \frac{90\cdot 89\cdot 3\cdot 10}{100\cdot 99\cdot 98} =
                \frac{90\cdot 89\cdot 3}{10\cdot 99\cdot 98} =\frac{267}{1078} \approx 0.24768
            \end{multline*}


            Los casos en los que sólo hay dos cuadrados perfectos son las combinaciones posibles de 1 elemento de un conjunto de 90 elementos $(100-10)$ y de 2 elemento de 10 elementos sin repetición:
            \begin{equation*}
                C_{90,1}\cdot C_{10,2}=\binom{90}{1}\cdot \binom{10}{2}=90\cdot \frac{10!}{2!\cdot 8!} = 90 \cdot \frac{10\cdot 9}{2}=4050
            \end{equation*}

            Por tanto, usando la Ley de Laplace, tenemos que:
            \begin{equation*}
                P(2C)=\frac{C_{90,1}\cdot C_{10,2}}{C_{100,3}}=\frac{4050\cdot 6}{100 \cdot 99 \cdot 98}
                \approx 0.025046
            \end{equation*}


            Los casos en los que hay tres cuadrados perfectos son las combinaciones posibles de 3 elemento de un conjunto de 10 elementos sin repetición:
            \begin{equation*}
                C_{10,3}=\binom{10}{3}=\frac{10!}{3!\cdot 7!} = \frac{10\cdot 9\cdot 8}{6}=120
            \end{equation*}

            Por tanto, usando la Ley de Laplace, tenemos que:
            \begin{equation*}
                P(3C)=\frac{C_{10,3}}{C_{100,3}}=\frac{120\cdot 6}{100 \cdot 99 \cdot 98}
                \approx 0.000742
            \end{equation*}



            \item \underline{Si sí hay repetición}:

            Los casos posibles son:
            \begin{equation*}
                CR_{100,3}=\binom{100+3-1}{3}=\binom{102}{3}=\frac{102!}{3!\cdot 99!} = 171700
            \end{equation*}

            Los casos en los que sólo hay un cuadrado perfecto son las combinaciones posibles de 2 elementos de un conjunto de 90 elementos $(100-10)$ y de 1 elemento de 10 elementos con repetición:
            \begin{equation*}
                CR_{90,2}\cdot CR_{10,1}=\binom{90+2-1}{2}\cdot \binom{10}{1}=\binom{91}{2}\cdot 10=\frac{91!}{2!\cdot 89!}\cdot 10
            \end{equation*}

            Por tanto, usando la Ley de Laplace, tenemos que:
            \begin{multline*}
                P(1C)=\frac{CR_{90,2}\cdot CR_{10,1}}{CR_{100,3}}=\frac{91!\cdot 3!\cdot 99!\cdot 10}{102!\cdot 2!\cdot 89!}
                =\frac{91\cdot 90\cdot 89!\cdot 3\cdot 2!\cdot 99!\cdot 10}{102\cdot 101\cdot 100 \cdot 99!\cdot 2!\cdot 89!}
                =\\=
                \frac{91\cdot 90\cdot 3\cdot 10}{102\cdot 101\cdot 100} =
                \frac{91\cdot 90\cdot 3}{102\cdot 101\cdot 10} =\frac{819}{3434} \approx 0.238497
            \end{multline*}



            Los casos en los que hay dos cuadrados perfectos son las combinaciones posibles de 1 elemento de un conjunto de 90 elementos $(100-10)$ y de 2 elemento de 10 elementos con repetición:
            \begin{equation*}
                CR_{90,1}\cdot CR_{10,2}=90\cdot \binom{11}{2}=90\cdot \frac{11!}{2!\cdot 9!} = 4950
            \end{equation*}

            Por tanto, usando la Ley de Laplace, tenemos que:
            \begin{equation*}
                P(2C)=\frac{CR_{90,1}\cdot CR_{10,2}}{CR_{100,3}} = \frac{4950}{171700} \approx 0.028829
            \end{equation*}


            Los casos en los que hay tres cuadrados perfectos son las combinaciones posibles de 3 elementos de un conjunto de 10 elementos con repetición:
            \begin{equation*}
                CR_{10,3}=\binom{12}{3}=\frac{12!}{3!\cdot 9!} = 220
            \end{equation*}

            Por tanto, usando la Ley de Laplace, tenemos que:
            \begin{equation*}
                P(3C)=\frac{CR_{10,3}}{CR_{100,3}} = \frac{220}{171700} \approx 0.0012813
            \end{equation*}
        \end{itemize}
        
    \end{enumerate}
\end{ejercicio}

\begin{ejercicio} \label{ej:3.Ejercicio8}
    En una carrera de relevos cada equipo se compone de 4 atletas. La sociedad deportiva de un colegio cuenta con 10 corredores y su entrenador debe formar un equipo de relevos que disputará el campeonato, y el orden en que participarán los seleccionados.
    \begin{enumerate}
        \item ¿Entre cuántos equipos distintos habría de elegir el entrenador si los 10 corredores son de igual valía? (Dos equipos con los mismos atletas en orden distinto se consideran diferentes)

        En este caso, el orden sí afecta a la hora de hacer los grupos. Además, como un atleta no puede ser seleccionado dos veces, tenemos que no hay repetición. Por tanto, estamos ante variaciones sin repetición.
        \begin{equation*}
            V_{10,4} = \frac{10!}{(10-4)!} = 10\cdot 9\cdot8 \cdot 7 = 5040
        \end{equation*}

        Por tanto, habría 5040 equipos distintos.
        
        \item Calcular la probabilidad de que un alumno cualquiera sea seleccionado.

        Dado un alumno, la cantidad de grupos distintos que se pueden hacer sin él son:
        \begin{equation*}
            V_{9,4} = \frac{9!}{(9-4)!} = 9\cdot8 \cdot 7\cdot 6 = 3024
        \end{equation*}

        Por tanto, la cantidad de grupos que se pueden hacer con él son:
        \begin{equation*}
            V_{10,4} - V_{9,4} = 2016
        \end{equation*}

        Por tanto, por la regla de Laplace, tenemos que la probabilidad de que un alumno cualquiera sea elegido es:
        \begin{equation*}
            \frac{2016}{V_{10,4}} = \frac{2}{5} = 0.4
        \end{equation*}
    \end{enumerate}
\end{ejercicio}

\begin{ejercicio} \label{ej:3.Ejercicio9}
    Una tienda compra bombillas en lotes de 300 unidades. Cuando un lote llega, se comprueban 60 unidades elegidas al azar, rechazándose el envío si se supera la cifra de 5 defectuosas. ¿Cuál es la probabilidad de aceptar un lote en el que haya 10 defectuosas?\\

    Analizamos en primer lugar el problema:
    \begin{itemize}
        \item No se repiten las bombillas, pues no se pueden revisar 2 bombillas iguales.
        \item El orden es indiferente, ya que solo importa el número de bombillas defectuosas.
    \end{itemize}
    Por tanto, se trata de combinaciones sin repeticiones. La cantidad de lotes posibles de $60$ bombillas a analizar son:
    \begin{equation*}
        C_{300,60} = \binom{300}{60} = \frac{300!}{60!\cdot 240!}
    \end{equation*}

    Sea $D_n$ el suceso de que, en el lote a revisar, haya exactamente $n$ bombillas defectuosas. Como los sucesos $D_i,D_j\; (i\neq j)$ son incompatibles, ya que no puede haber a la vez dos números distintos de bombillas, tenemos que la probabilidad de aceptar el envío es:
    \begin{equation*}
        P\left(\bigcup_{n=0}^5 D_n\right) = \sum_{n=0}^5 P(D_n)
    \end{equation*}

    Sabiendo que de las 300 bombillas hay 10 defectuosas y que se hacen lotes de $60$ bombillas, vemos cuántos lotes se pueden hacer con $n$ bombillas defectuosas.
    
    De un total de 10 bombillas defectuosas, se han de elegir $n$. Por tanto, se pueden hacer $C_{10,n}$ combinaciones para las $n$ defectuosas.
    
    Además, de un total de $300-10=290$ bombillas correctas, se han de elegir $60-n$, por lo que se pueden hacer $C_{290,60-n}$ combinaciones para las $60-n$ defectuosas.

    Por tanto, como cada grupo de bombillas defectuosas puede estar con un grupo de bombillas correctas distinto, tenemos que el número total de lotes con $n$ bombillas defectuosas es:
    \begin{equation*}
        C_{10,n}\cdot C_{290,60-n}
    \end{equation*}
    
    Por tanto, por la regla de Laplace tenemos que la probabilidad de hacer un lote de $n$ bombillas defectuosas es:
    \begin{equation*}
        P(D_n) = \frac{C_{10,n}\cdot C_{290,60-n}}{C_{300,60}} \qquad \qquad n=0,1,\dots,10
    \end{equation*}

    Calculamos por tanto la probabilidad de encontrar $n=0,1,\dots, 5$ bombillas defectuosas:
    \begin{equation*}
        P(D_0) = \frac{C_{10,0}\cdot C_{290,60}}{C_{300,60}} = \frac{\binom{10}{0}\cdot \binom{290}{60}}{\binom{300}{60}} = \frac{10!\cdot 290! \cdot 60! \cdot 240!}{300!\cdot 10!\cdot 0! \cdot 60! \cdot 230!} = \frac{240\cdot_{\; \dots} \cdot 231}{300\cdot_{\; \dots} \cdot 291} \approx 0.1033
    \end{equation*}
    \begin{equation*}
        P(D_1) = \frac{C_{10,1}\cdot C_{290,59}}{C_{300,60}} = \frac{\binom{10}{1}\cdot \binom{290}{59}}{\binom{300}{60}} = \frac{10!\cdot 290! \cdot 60! \cdot 240!}{300!\cdot 9!\cdot 1! \cdot 59! \cdot 231!} = \frac{240\cdot_{\; \dots} \cdot 232\cdot 10 \cdot 60}{300\cdot_{\; \dots} \cdot 291} \approx 0.26838
    \end{equation*}
    \begin{multline*}
        P(D_2) = \frac{C_{10,2}\cdot C_{290,58}}{C_{300,60}} = \frac{\binom{10}{2}\cdot \binom{290}{58}}{\binom{300}{60}} = \frac{10!\cdot 290! \cdot 60! \cdot 240!}{300!\cdot 8!\cdot 2! \cdot 58! \cdot 232!} =\\= \frac{240\cdot_{\; \dots} \cdot 233\cdot 10\cdot 9 \cdot 60\cdot 59}{300\cdot_{\; \dots} \cdot 291\cdot 2!} \approx 0.3071
    \end{multline*}
    \begin{multline*}
        P(D_3) = \frac{C_{10,3}\cdot C_{290,57}}{C_{300,60}} = \frac{\binom{10}{3}\cdot \binom{290}{57}}{\binom{300}{60}} = \frac{10!\cdot 290! \cdot 60! \cdot 240!}{300!\cdot 7!\cdot 3! \cdot 57! \cdot 233!} =\\= \frac{240\cdot_{\; \dots} \cdot 234\cdot 10\cdot_{\; \dots} \cdot 8 \cdot 60\cdot_{\; \dots} \cdot 58}{300\cdot_{\; \dots} \cdot 291\cdot 3!} \approx 0.2039
    \end{multline*}
    \begin{multline*}
        P(D_4) = \frac{C_{10,4}\cdot C_{290,56}}{C_{300,60}} = \frac{\binom{10}{4}\cdot \binom{290}{56}}{\binom{300}{60}} = \frac{10!\cdot 290! \cdot 60! \cdot 240!}{300!\cdot 6!\cdot 4! \cdot 56! \cdot 234!} =\\= \frac{240\cdot_{\; \dots} \cdot 235\cdot 10\cdot_{\; \dots} \cdot 7 \cdot 60\cdot_{\; \dots} \cdot 57}{300\cdot_{\; \dots} \cdot 291\cdot 4!} \approx 0.08691
    \end{multline*}
    \begin{multline*}
        P(D_5) = \frac{C_{10,5}\cdot C_{290,55}}{C_{300,60}} = \frac{\binom{10}{5}\cdot \binom{290}{55}}{\binom{300}{60}} = \frac{10!\cdot 290! \cdot 60! \cdot 240!}{300!\cdot 5!\cdot 5! \cdot 55! \cdot 235!} =\\= \frac{240\cdot_{\; \dots} \cdot 234\cdot 10\cdot_{\; \dots} \cdot 6 \cdot 60\cdot_{\; \dots} \cdot 56}{300\cdot_{\; \dots} \cdot 291\cdot 5!} \approx 0.02485
    \end{multline*}


    Por tanto, la probabilidad de que se acepte el envío (es decir, que tenga entre 0 y 5 bombillas defectuosas) es:
    \begin{equation*}
        P\left(\bigcup_{n=0}^5 D_n\right) = \sum_{n=0}^5 P(D_n) \approx 0.99444
    \end{equation*}
    
\end{ejercicio}

\begin{ejercicio} \label{ej:3.Ejercicio10}
    Una secretaria debe echar al correo 3 cartas; para ello, introduce cada carta en un sobre y escribe las direcciones al azar. ¿Cuál es la probabilidad de que al menos una carta llegue a su destino?\\
    
    En este caso, tenemos que considerar 2 conjuntos, el conjunto $C$ de las cartas y conjunto $D$ de los destinos. Sean los siguientes sucesos:
    \begin{itemize}
        \item $A_1$: la carta 1 $(C_1)$ llega a su destino $(D_1)$.
        \item $A_2$: la carta 2 $(C_2)$ llega a su destino $(D_2)$.
        \item $A_3$: la carta 3 $(C_3)$ llega a su destino $(D_3)$.
    \end{itemize}

    Tenemos que la probabilidad del suceso pedido es:
    \begin{equation*}
        P(A_1 \cup A_2 \cup A_3) = P(A_1) + P(A_2) + P(A_3) - P(A_1\cap A_2) - P(A_1\cap A_3) - P(A_2\cap A_3) + P(A_1 \cap A_2 \cap A_3)
    \end{equation*}

    Para calcular las probabilidades, empleamos combinatoria. Como buscamos las formas en las que se pueden organizar las direcciones, buscamos las distintas permutaciones. Como no se pueden repetir, son \textbf{permutaciones sin repetición}.
    
    Para los casos posibles, como hay 3 direcciones, tenemos que son:
    \begin{equation*}
        P_3 = 3!
    \end{equation*}

    En el suceso $A_i$ de que la carta $C_i$ llegue a su destino $D_i$, tenemos que se ha fijado dicha dirección ($C_i$ con $D_i$), por lo que quedan 2 direcciones que se pueden ordenar como se desee. Las permutaciones favorables son, por tanto,
    \begin{equation*}
        P_2 = 2! 
    \end{equation*}
    Por la Regla de Laplace, tengo que:
    \begin{equation*}
        P(A_i) = \frac{2!}{3!} = \frac{1}{3} \qquad i=1,2,3
    \end{equation*}


    En el suceso $A_i\cap A_j$ de que las carta $C_i,C_j$ lleguen a su destino $D_i,D_j$, tenemos que se han fijado dos direcciones, por lo que queda una dirección que se pueden ordenar como se desee. Las permutaciones favorables son, por tanto,
    \begin{equation*}
        P_1 = 1! \Longrightarrow P(A_i\cap A_j) = \frac{1!}{3!} = \frac{1}{6} \qquad i,j=1,2,3 \quad i\neq j
    \end{equation*}

    Además, en el suceso $A_1\cap A_2 \cap A_3$ las tres direcciones están fijadas, por lo que solo hay una posibilidad. Es decir,
    \begin{equation*}
        P(A_1\cap A_2 \cap A_3) = \frac{1}{6}
    \end{equation*}

    Por tanto, usando las probabilidades ya obtenidas, tengo que:
    \begin{equation*}\begin{split}
        P(A_1 \cup A_2 \cup A_3) &= P(A_1) + P(A_2) + P(A_3) - P(A_1\cap A_2) - P(A_1\cap A_3) -\\
        &\qquad \qquad \qquad- P(A_2\cap A_3) + P(A_1 \cap A_2 \cap A_3) =\\
        &= 3\cdot \frac{1}{3} -3\cdot \frac{1}{6} + \frac{1}{6} = \frac{2}{3}
    \end{split}\end{equation*}
    
\end{ejercicio}
