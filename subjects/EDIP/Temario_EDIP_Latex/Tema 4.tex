\chapter{Axiomática probabilística}

La Probabilidad es la ciencia que estudia cómo debe emplearse la información y cómo dar una
guía de acción en situación prácticas que envuelven incertidumbre.

\begin{itemize}
  \item \textbf{Fenómenos determinísticos:} Los que dan lugar al mismo resultado si se
        hacen bajo condiciones idénticas.
  \item \textbf{Fenómenos aleatorios:} los resultados pueden variar incluso si el estudio
        se realiza con las mismas condiciones iniciales.
\end{itemize}

Características de los fenómenos aleatorios:
\begin{itemize}
  \item El experimento se puede repetir indefinidamente bajo idénticas condiciones.
  \item Cualquier modificación mínima en las condiciones iniciales de la repetición
        puede modificar completamente el resultado final del experimento.
  \item Se puede determinar el conjunto de posibles resultados del experimento, pero
        no se puede predecir previamente un resultado particular.
  \item Si el experimento se repite un número grande de veces, entonces aparece algún
        modelo de regularidad estadística en los resultados obtenidos.
\end{itemize}

\section{Espacio Muestral}

Si consideramos un experimento aleatorio arbitrario, cada uno de los posibles resultados
que no puedan descomponerse en otros más simples recibirán el nombre de
\underline{suceso elemental}. En el ejemplo de tirar un dado, los posibles sucesos
elementales son: 1, 2, 3, 4, 5 y 6.\\


El conjunto que contiene a todos los sucesos elementales asociados a un experimento
aleatorio recibe el nombre de \underline{espacio muestral}, $\Omega$. En el ejemplo anterior:
$\Omega = \{1, 2, 3, 4, 5, 6\}$. Los espacios muestrales pueden ser finitos, infinitos numerables
o continuos.\\


Llamaremos \underline{suceso} (o suceso aleatorio) a cualquier subconjunto del espacio muestral, es
decir, un conjunto de sucesos elementales cuya aparición da lugar a un suceso. En el ejemplo del dado,
podemos considerar el suceso A de sacar un número par: $A = \{2,4,6\} \subseteq \Omega$

\subsection{Sucesos}

Cabe destacar que existen diversos tipos de sucesos:
\begin{itemize}
  \item \textbf{Suceso elemental:} Definido anteriormente, cada uno de los resultados posibles de
        nuestro experimento aleatorio, consta de un único elemento del espacio muestral.
  \item \textbf{Suceso compuesto:} Aquel que consta de dos o más sucesos.
  \item \textbf{Suceso seguro o universal:} Aquel que ocurre siempre. Consta con todos los sucesos
        elementales del espacio muestral y, por tanto, se identifica con él. En el ejemplo anterior, un
        suceso seguro es sacar en un dado un número del 1 al 6.
  \item \textbf{Suceso imposible:} Aquel que no puede ocurrir nunca. No contiene ningún elemento
        del espacio muestral, lo representamos con $\emptyset$. En el ejemplo del dado, un suceso imposible
        es sacar un 7 al tirar el dado.
\end{itemize}

\subsection{Relaciones y Operaciones de sucesos}

Dados dos sucesos $A$ y $B$ de un experimento aleatorio, diremos que el suceso $A$ \underline{está contenido}
en el suceso $B$, notado $A \subseteq B$ si siempre que ocurre el suceso $A$ ocurre el suceso $B$, dicho en
lenguaje de conjuntos, si $\forall a \in A \ a \in B$.\\


Dados dos sucesos $A$ y $B$ de un experimento aleatorio, diremos que el suceso $A$ \underline{es igual} al
suceso $B$ si siempre que ocurre el suceso $A$ ocurre el suceso $B$ y siempre que ocurre el suceso $B$
ocurre el suceso $A$, es decir: $A \subseteq B \ \land \ B \subseteq A$.\\


Dados dos sucesos $A$ y $B$ de un experimento aleatorio, definimos la \underline{unión} de $A$ y de $B$,
notado $A \cup B$ como aquel suceso que ocurre siempre que ocurre $A$ o que ocurre $B$, es decir:
$\forall a \in A \cup B \Rightarrow a \in A \ \lor \ a \in B $\\


Dados dos sucesos $A$ y $B$ de un experimento aleatorio, definimos la \underline{intersección} de $A$ y
de $B$, notado $A \cap B$ como aquel suceso que si ocurre implica que ocurre $A$ y que ocure $B$, es decir:
$\forall a \in A \cap B \Rightarrow a \in A \ \land \ a \in B $\\


Dados dos sucesos $A$ y $B$ de un experimento aleatorio, definimos el \underline{complementario} de $A$
en $B$, notado $B-A$ como aquel suceso que ocurre siempre que ocurre $B$ y no ocurre $A$, es decir:
$\forall a \in B - A \Rightarrow a \in B \ \land \ a \notin A$.


Podemos hablar solamente del complementario de un suceso. En este caso, entenderems que el complementario
de un suceso $A$ de un experimento aleatorio, notado $\overline{A}$ es igual al complementario de $A$
en el espacio muestral $\Omega$: $\overline{A} = \Omega - A$.\\

\bigskip


Diremos que dos sucesos $A$ y $B$ son \underline{disjuntos o incompatibles} si no pueden ocurrir
simultáneamente, es decir, que su intersección sea vacía: $A \cap B = \emptyset$. En el ejemplo de
tirar un dado, el suceso de que salga un número par es incompatible con el suceso de que salga impar.\\


Diremos que un conjunto de sucesos $\{A_1, A_2, \ldots, A_n\}$ es un \newline \underline{sistema exhaustivo
  de sucesos} si la unión de todos ellos es igual al espacio muestral:
$A_1 \cup A_2 \cup \ldots \cup A_n = \Omega$ \\


Diremos que un conjunto de sucesos $\{A_1, A_2, \ldots, A_n\}$ es un \underline{sistema completo de sucesos} o \underline{una partición del espacio muestral} si dicho conjunto constituye un sistema exhaustivo
de sucesos y además son mutuamente excluyentes, es decir, son disjuntos dos a dos:
$$A_1 \mathop{\cup}^{\cdot} A_2 \mathop{\cup}^{\cdot} \ldots \mathop{\cup}^{\cdot} A_n = \Omega$$

\section{Estructuras álgebra y sigma-álgebra}

Un conjunto de subconjuntos de $\Omega$ no trivial $\mathcal{A}$ se dice que tiene estructura de
\underline{álgebra de sucesos o álgebra de Boole} si verifica que:
\begin{itemize}
  \item $\forall A \in \mathcal{A} \Rightarrow \overline{A} \in \mathcal{A}$
  \item $\forall A_1, A_2 \in \mathcal{A} \Rightarrow A_1 \cup A_2 \in \mathcal{A}$
\end{itemize}

Notemos que $\forall \mathcal{A} \Rightarrow \Omega \in \mathcal{A} \ \land \ \emptyset \in \mathcal{A}$.

Un ejemplo de álgebra de Boole es, para un cierto $\Omega$, $\mathcal{A} = P(\Omega)$.\\

Un conjunto de subconjuntos de $\Omega$ no trivial $\mathcal{A}$ se dice que tiene estructura de
\underline{sigma-álgebra} si verifica que:
\begin{itemize}
  \item $\forall A \in \mathcal{A} \Rightarrow \overline{A} \in \mathcal{A}$
  \item $\forall A_1, A_2, \ldots \in \mathcal{A} \Rightarrow \bigcup\limits_{i=1}^\infty A_i \in \mathcal{A}$
\end{itemize}

Notemos que $\forall \mathcal{A} \Rightarrow \Omega \in \mathcal{A} \ \land \ \emptyset \in \mathcal{A}$.

Y que cualquier sigma-álgebra es un álgebra.

A lo largo de los apuntes, cada vez que aparezca $\mathcal{A}$ haremos referencia a una sigma-álgebra.

\section{Diferentes concepciones de Probabilidad}

Definimos una \underline{probabilidad} como una fución $P: \mathcal{A} \rightarrow \R$, que verifica
unos ciertos axiomas. A continuación veremos algunas concepciones distintas de cómo calcular la imagen
de cualquier elemento de la sigma-álgebra.

\subsection{Concepción clásica}

Sea $A$ un suceso arbitrario asociado a un experimento aleatorio, se define la probabilidad del suceso $A$ mediante la \textbf{Regla de Laplace} como: $$P(A) = \dfrac{|A|}{|\Omega|}$$

\subsection{Concepción frecuentista}

Si se realizan $N$ repeticiones de un experimento y un determinado suceso $A$ se ha presentado en $N_A$
ocasiones, se define la \underline{frecuencia relativa} de $A$ en las $N$ pruebas como:
$$f_N(A) = \dfrac{N_A}{N}$$


Supongamos que el número de realizaciones del experimento crece indefinidamente y consideramos la sucesión
de frecuencias relativas de $A$. Se define la probabilidad de $A$ como:
$$P(A) = \lim_{N \to \infty} f_N(A)$$

\section{Definición axiomática de Kolmogorov}

Sea ($\Omega, \mathcal{A}$) un espacio medible asociado a un experimento aleatorio, se define una
\underline{probabilidad} como una función
$$P: \mathcal{A} \rightarrow [0,1]$$
que verifica los siguiente axiomas:
\begin{itemize}
  \item Axioma de la no negatividad: $P(A) \geq 0~~\forall A \in \mathcal{A}$
  \item Axioma del suceso seguro: $P(\Omega) = 1$
  \item Axioma de $\sigma$-aditividad o aditividad numerable:
        Sea $A_1, A_2, \ldots \in \mathcal{A}$ una sucesión de sucesos incompatibles ($A_i \cap A_j = \emptyset$
        $\forall i \neq j$) entonces:
        $$P\left(\bigcup_{i=1}^\infty A_i \right) = \sum_{i=1}^\infty P(A_i)$$
\end{itemize}

Algunas consecuencias de la definición axiomática de una probabilidad son:
\begin{prop}
    La probabilidad del suceso imposible es nula: $P(\emptyset)=0$.
\end{prop}
\begin{proof}
  $$1 = P(\Omega) = P(\Omega \cup \emptyset) = P(\Omega) + P(\emptyset) = 1 + P(\emptyset) \Rightarrow
    P(\emptyset) = 0$$
\end{proof}

\begin{prop}
    $\forall A \in \mathcal{A} \Rightarrow P(\overline{A}) = 1-P(A)$.
\end{prop}
\begin{proof}
  $$P(A \cup \overline{A}) = P(\Omega) = 1 = P(A) + P(\overline{A}) \Rightarrow P(\overline{A}) = 1 - P(A)$$
\end{proof}

\begin{prop}
    $P$ es monótona no decreciente. Es decir,
    $$\forall A,B \in \mathcal{A} \mid A \subseteq B \Rightarrow P(A) \leq P(B)$$
    
    Además, $P(B-A)=P(B)-P(A)$
\end{prop}
\begin{proof}
  $$P(B) = P(A \cup (B-A)) = P(A) + P(B-A) \Rightarrow P(B) \geq P(A)$$

    Además, se tiene que:
  $$P(B) = P(A) + P(B-A) \Rightarrow P(B-A) = P(B) - P(A)$$
\end{proof}

\begin{prop}
    $\forall A \in \mathcal{A} \Rightarrow P(A) \leq 1$
\end{prop}
\begin{proof}
  $$\forall A \in \mathcal{A} \Rightarrow A \subseteq \Omega \Rightarrow 1 = P(\Omega) \geq P(A)$$
\end{proof}

\begin{prop}
    $\forall A,B \in \mathcal{A} \Rightarrow P(B-A) = P(B) - P(A \cap B)$
\end{prop}
\begin{proof}
  $$P(B) = P((B \cap A) \cup (B-A)) = P(B \cap A) + P(B-A) \Rightarrow P(B-A) = P(B) - P(B \cap A)$$
\end{proof}

\begin{prop}
    $\forall A,B \in \mathcal{A} \Rightarrow P(A\cup B) = P(A) + P(B) -P(A \cap B)$
\end{prop}
\begin{proof}
    \begin{multline*}
        P(B \cup A)
        = P[(B \cup A) \cap \Omega]
        = P[(B \cup A) \cap (A \cup \overline{A})]
        = P[(B \cap \overline{A}) \cup A]
        =\\
        = P(B \cap \overline{A}) + P(A)
        = P(B-A) + P(A)
        = P(B) - P(A\cap B) + P(A)
    \end{multline*}
\end{proof}

\begin{prop}[Subaditividad finita] $\forall A,B \in \mathcal{A}$
  $$P(A \cup B) \leq P(A) + P(B)$$

  En general, dados $A_1, A_2, \ldots \in \mathcal{A}$ se verifica que:
  $$P\left( \bigcup_{i=1}^nA_i \right) \leq \sum_{i=1}^nP(A_i)$$
\end{prop}
\begin{proof}
  $$P(A \cup B) = P(A) + P(B) - P(A \cap B) \Rightarrow P(A \cup B) \leq P(A) + P(B)$$

    Para demostrarlo de forma general, hacemos inducción sobre $n$.
    \begin{itemize}
        \item \underline{Para $n=1$}: Trivialmente es cierto.

        \item \underline{Para $n=2$}: Se acaba de probar previamente, en el caso particular.

        \item \underline{Supuesto cierto para $n-1$, comprobamos para $n$}:
        \begin{multline*}
            P\left( \bigcup_{i=1}^n A_i \right)
            = P\left( \bigcup_{i=1}^{n-1} A_i \cup A_n \right)
            = P\left( \bigcup_{i=1}^{n-1} A_i \right) + P(A_n) -
            P\left( \bigcup_{i=1}^{n-1} A_i \cap A_n \right)
            \stackrel{(\ast)}{\leq} \\ \stackrel{(\ast)}{\leq}
            \sum_{i=1}^{n-1}P(A_i) + P(A_n) - P\left( \bigcup_{i=1}^{n-1} A_i \cap A_n \right)
            =\\=
            \sum_{i=1}^n P(A_i) - P\left( \bigcup_{i=1}^{n-1} A_i \cap A_n \right)
            \leq \sum_{i=1}^n P(A_i)
        \end{multline*}
        donde en $(\ast)$ hemos aplicado la hipótesis de inducción.
    \end{itemize}
\end{proof}

\begin{coro}[Subaditividad numerable]
    Dada una colección de sucesos $A_1, A_2, \ldots \in~\mathcal{A}$, se verifica: 
    $$P\left( \bigcup_{i=1}^\infty A_i \right) \leq \sum_{i=1}^\infty P(A_i)$$
\end{coro}

\begin{prop}[Principio de inclusión-exclusión]
    Dada una colección de sucesos $A_1, A_2, \ldots A_n\in
    \mathcal{A}$, entonces:
  $$P\left(\bigcup_{i=1}^n A_i\right) = \sum_{i=1}^n P(A_i) - \sum_{i<j}^n P(A_i \cap A_j) + $$$$ + \sum_{i<j<k}^n
    P(A_i \cap A_j \cap A_k) + \ldots + (-1)^{n-1} P\left(\bigcap_{i=1}^n A_i\right)$$
\end{prop}
\begin{proof}
  Hacemos inducción sobre $n$:
  \begin{itemize}
      \item \underline{Para $n=2$}: $P(A_1 \cup A_2) = P(A_1) + P(A_2) - P(A_1 \cap A_2)$, cierto.

      \item \underline{Supuesto cierto para $n-1$, comprobamos para $n$}:
      \begin{equation*}\begin{split}
          P\left(\bigcup_{i=1}^nA_i\right)
          &= P\left(\bigcup_{i=1}^{n-1}A_i \cup A_n\right)
          = P\left(\bigcup_{i=1}^{n-1}A_i\right) + P(A_n) - P\left(\bigcup_{i=1}^{n-1}A_i \cap A_n\right)
          =\\&\mathop{=}^{HI}
          \sum_{i=1}^{n} P(A_i) - \sum_{i<j}^{n-1} P(A_i \cap A_j) + \sum_{i<j<k}^{n-1} P(A_i \cap A_j \cap A_k) + \\ &\hspace{1cm}
          + \ldots + (-1)^{n-2} P\left(\bigcap_{i=1}^{n-1} A_i\right) - P\left(\bigcup_{i=1}^{n-1}(A_i \cap A_n)\right) 
          =\\&\mathop{=}^{HI}
          \sum_{i=1}^{n} P(A_i) - \sum_{i<j}^{n-1} P(A_i \cap A_j) + \sum_{i<j<k}^{n-1} P(A_i \cap A_j \cap A_k) + \\ &\hspace{1cm}
          + \ldots + (-1)^{n-2} P\left(\bigcap_{i=1}^{n-1} A_i\right) - \left[\sum_{i=1}^{n-1} P(A_i\cap A_n) - \sum_{i<j}^{n-1} P(A_i \cap A_j \cap A_n) + \right. \\ &\hspace{1cm} \left.
          + \sum_{i<j<k}^{n-1} P(A_i \cap A_j \cap A_k\cap A_n) + \ldots + (-1)^{n-2} P\left(\bigcap_{i=1}^{n-1} A_i\cap A_n\right) \right]
      \end{split}\end{equation*}

      Lo que encontramos entre corchetes es la última iteración (la $n$) de los bucles anteriores, luego:
      \begin{equation*}\begin{split}
          P\left(\bigcup_{i=1}^nA_i\right)
          &= P\left(\bigcup_{i=1}^{n-1}A_i \cup A_n\right)
          = P\left(\bigcup_{i=1}^{n-1}A_i\right) + P(A_n) - P\left(\bigcup_{i=1}^{n-1}A_i \cap A_n\right)
          =\\& =
          \sum_{i=1}^{n} P(A_i) - \sum_{i<j}^{n} P(A_i \cap A_j) + \sum_{i<j<k}^{n} P(A_i \cap A_j \cap A_k) + \\ &\hspace{1cm}
          + \ldots + (-1)^{n-1} P\left(\bigcap_{i=1}^{n} A_i\right)
      \end{split}\end{equation*}
  \end{itemize}
\end{proof}

\begin{prop}[Desigualdad de Bonferroni]
    Dada una colección de sucesos $A_1, A_2, \ldots A_n\in~\mathcal{A}$, entonces:
      $$\sum_{i=1}^n P(A_i) \geq P\left(\bigcup_{i=1}^n A_i\right) \geq \sum_{i=1}^n P(A_i) - \sum_{i<j}^n P(A_i \cap A_j)$$
\end{prop}
\begin{proof}
    Hacemos inducción sobre $n$:
    \begin{itemize}
        \item \underline{Para $n=2$}:
        $P(A_1 \cup A_2) = P(A_1) + P(A_2) - P(A_1 \cap A_2)$, siendo ciertas por tanto las dos desigualdades.

        \item \underline{Supuesto cierto para $n-1$, comprobamos para $n$}:
        \begin{equation*}\begin{split}
            P\left(\bigcup_{i=1}^n A_i\right) 
            &= P\left(\bigcup_{i=1}^{n-1}A_i \cup A_n\right) = P\left(\bigcup_{i=1}^{n-1}A_i\right) + P(A_n) - P\left(\bigcup_{i=1}^{n-1} A_i \cap A_n\right)
            \geq \\ &
            \mathop{\geq}^{HI} \sum_{i=1}^n P(A_i) - \sum_{i<j}^{n-1}P(A_i \cap A_j) - P\left(\bigcup_{i=1}^{n-1} A_i \cap A_n\right)
            \geq \\ &
            \mathop{\geq}^{HI} \sum_{i=1}^n P(A_i) - \sum_{i<j}^{n-1}P(A_i \cap A_j) - \sum_{i=1}^{n-1}P(A_i \cap A_n)
            = \\ &=
            \sum_{i=1}^n P(A_i) - \sum_{i<j}^n P(A_i \cap A_j)
        \end{split}\end{equation*}

        La primera desigualdad se demuestra con la primera línea, mientras que es necesario el resto para demostrar la segunda desigualdad.
    \end{itemize}
\end{proof}

\begin{prop}[Desigualdad de Boole]
    $\forall A,B \in \mathcal{A}$, entonces: $$P(A \cap B) \geq 1 - P(\overline{A}) - P(\overline{B})$$
\end{prop}
\begin{proof}
    \begin{equation*}\begin{split}
        P(A \cap B) &
        = P(\overline{\overline{A \cap B}})
        = P(\overline{\overline{A} \cup \overline{B}})
        = 1- P(\overline{A} \cup \overline{B}) 
        = 1 - P(\overline{A}) - P(\overline{B}) + P(\overline{A}
    \cap \overline{B})
    \end{split}\end{equation*}
    
    Por tanto, tenemos que
    $$P(A \cap B) \geq 1 - P(\overline{A}) - P(\overline{B})$$
\end{proof}