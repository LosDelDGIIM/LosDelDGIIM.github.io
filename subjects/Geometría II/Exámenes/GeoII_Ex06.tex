\documentclass[12pt]{article}

% Idioma y codificación
\usepackage[spanish, es-tabla]{babel}       %es-tabla para que se titule "Tabla"
\usepackage[utf8]{inputenc}

% Márgenes
\usepackage[a4paper,top=3cm,bottom=2.5cm,left=3cm,right=3cm]{geometry}

% Comentarios de bloque
\usepackage{verbatim}

% Paquetes de links
\usepackage[hidelinks]{hyperref}    % Permite enlaces
\usepackage{url}                    % redirecciona a la web

% Más opciones para enumeraciones
\usepackage{enumitem}

% Personalizar la portada
\usepackage{titling}

% Paquetes de tablas
\usepackage{multirow}


%------------------------------------------------------------------------

%Paquetes de figuras
\usepackage{caption}
\usepackage{subcaption} % Figuras al lado de otras
\usepackage{float}      % Poner figuras en el sitio indicado H.


% Paquetes de imágenes
\usepackage{graphicx}       % Paquete para añadir imágenes
\usepackage{transparent}    % Para manejar la opacidad de las figuras

% Paquete para usar colores
\usepackage[dvipsnames]{xcolor}
\usepackage{pagecolor}      % Para cambiar el color de la página

% Habilita tamaños de fuente mayores
\usepackage{fix-cm}

% Para los gráficos
\usepackage{tikz}

% Para poder situar los nodos en los grafos
\usetikzlibrary{positioning}


%------------------------------------------------------------------------

% Paquetes de matemáticas
\usepackage{mathtools, amsfonts, amssymb, mathrsfs}
\usepackage[makeroom]{cancel}     % Simplificar tachando
\usepackage{polynom}    % Divisiones y Ruffini
\usepackage{units} % Para poner fracciones diagonales con \nicefrac

\usepackage{pgfplots}   %Representar funciones
\pgfplotsset{compat=1.18}  % Versión 1.18

\usepackage{tikz-cd}    % Para usar diagramas de composiciones
\usetikzlibrary{calc}   % Para usar cálculo de coordenadas en tikz

%Definición de teoremas, etc.
\usepackage{amsthm}
%\swapnumbers   % Intercambia la posición del texto y de la numeración

\theoremstyle{plain}

\makeatletter
\@ifclassloaded{article}{
  \newtheorem{teo}{Teorema}[section]
}{
  \newtheorem{teo}{Teorema}[chapter]  % Se resetea en cada chapter
}
\makeatother

\newtheorem{coro}{Corolario}[teo]           % Se resetea en cada teorema
\newtheorem{prop}[teo]{Proposición}         % Usa el mismo contador que teorema
\newtheorem{lema}[teo]{Lema}                % Usa el mismo contador que teorema

\theoremstyle{remark}
\newtheorem*{observacion}{Observación}

\theoremstyle{definition}

\makeatletter
\@ifclassloaded{article}{
  \newtheorem{definicion}{Definición} [section]     % Se resetea en cada chapter
}{
  \newtheorem{definicion}{Definición} [chapter]     % Se resetea en cada chapter
}
\makeatother

\newtheorem*{notacion}{Notación}
\newtheorem*{ejemplo}{Ejemplo}
\newtheorem*{ejercicio*}{Ejercicio}             % No numerado
\newtheorem{ejercicio}{Ejercicio} [section]     % Se resetea en cada section


% Modificar el formato de la numeración del teorema "ejercicio"
\renewcommand{\theejercicio}{%
  \ifnum\value{section}=0 % Si no se ha iniciado ninguna sección
    \arabic{ejercicio}% Solo mostrar el número de ejercicio
  \else
    \thesection.\arabic{ejercicio}% Mostrar número de sección y número de ejercicio
  \fi
}


% \renewcommand\qedsymbol{$\blacksquare$}         % Cambiar símbolo QED
%------------------------------------------------------------------------

% Paquetes para encabezados
\usepackage{fancyhdr}
\pagestyle{fancy}
\fancyhf{}

\newcommand{\helv}{ % Modificación tamaño de letra
\fontfamily{}\fontsize{12}{12}\selectfont}
\setlength{\headheight}{15pt} % Amplía el tamaño del índice


%\usepackage{lastpage}   % Referenciar última pag   \pageref{LastPage}
\fancyfoot[C]{\thepage}

%------------------------------------------------------------------------

% Conseguir que no ponga "Capítulo 1". Sino solo "1."
\makeatletter
\@ifclassloaded{book}{
  \renewcommand{\chaptermark}[1]{\markboth{\thechapter.\ #1}{}} % En el encabezado
    
  \renewcommand{\@makechapterhead}[1]{%
  \vspace*{50\p@}%
  {\parindent \z@ \raggedright \normalfont
    \ifnum \c@secnumdepth >\m@ne
      \huge\bfseries \thechapter.\hspace{1em}\ignorespaces
    \fi
    \interlinepenalty\@M
    \Huge \bfseries #1\par\nobreak
    \vskip 40\p@
  }}
}
\makeatother

%------------------------------------------------------------------------
% Paquetes de cógido
\usepackage{minted}
\renewcommand\listingscaption{Código fuente}

\usepackage{fancyvrb}
% Personaliza el tamaño de los números de línea
\renewcommand{\theFancyVerbLine}{\small\arabic{FancyVerbLine}}

% Estilo para C++
\newminted{cpp}{
    frame=lines,
    framesep=2mm,
    baselinestretch=1.2,
    linenos,
    escapeinside=||
}

% para minted
\definecolor{LightGray}{rgb}{0.95,0.95,0.92}
\setminted{
    linenos=true,
    stepnumber=5,
    numberfirstline=true,
    autogobble,
    breaklines=true,
    breakautoindent=true,
    breaksymbolleft=,
    breaksymbolright=,
    breaksymbolindentleft=0pt,
    breaksymbolindentright=0pt,
    breaksymbolsepleft=0pt,
    breaksymbolsepright=0pt,
    fontsize=\footnotesize,
    bgcolor=LightGray,
    numbersep=10pt
}


\usepackage{listings} % Para incluir código desde un archivo

\renewcommand\lstlistingname{Código Fuente}
\renewcommand\lstlistlistingname{Índice de Códigos Fuente}

% Definir colores
\definecolor{vscodepurple}{rgb}{0.5,0,0.5}
\definecolor{vscodeblue}{rgb}{0,0,0.8}
\definecolor{vscodegreen}{rgb}{0,0.5,0}
\definecolor{vscodegray}{rgb}{0.5,0.5,0.5}
\definecolor{vscodebackground}{rgb}{0.97,0.97,0.97}
\definecolor{vscodelightgray}{rgb}{0.9,0.9,0.9}

% Configuración para el estilo de C similar a VSCode
\lstdefinestyle{vscode_C}{
  backgroundcolor=\color{vscodebackground},
  commentstyle=\color{vscodegreen},
  keywordstyle=\color{vscodeblue},
  numberstyle=\tiny\color{vscodegray},
  stringstyle=\color{vscodepurple},
  basicstyle=\scriptsize\ttfamily,
  breakatwhitespace=false,
  breaklines=true,
  captionpos=b,
  keepspaces=true,
  numbers=left,
  numbersep=5pt,
  showspaces=false,
  showstringspaces=false,
  showtabs=false,
  tabsize=2,
  frame=tb,
  framerule=0pt,
  aboveskip=10pt,
  belowskip=10pt,
  xleftmargin=10pt,
  xrightmargin=10pt,
  framexleftmargin=10pt,
  framexrightmargin=10pt,
  framesep=0pt,
  rulecolor=\color{vscodelightgray},
  backgroundcolor=\color{vscodebackground},
}

%------------------------------------------------------------------------

% Comandos definidos
\newcommand{\bb}[1]{\mathbb{#1}}
\newcommand{\cc}[1]{\mathcal{#1}}

% I prefer the slanted \leq
\let\oldleq\leq % save them in case they're every wanted
\let\oldgeq\geq
\renewcommand{\leq}{\leqslant}
\renewcommand{\geq}{\geqslant}

% Si y solo si
\newcommand{\sii}{\iff}

% Letras griegas
\newcommand{\eps}{\epsilon}
\newcommand{\veps}{\varepsilon}
\newcommand{\lm}{\lambda}

\newcommand{\ol}{\overline}
\newcommand{\ul}{\underline}
\newcommand{\wt}{\widetilde}
\newcommand{\wh}{\widehat}

\let\oldvec\vec
\renewcommand{\vec}{\overrightarrow}

% Derivadas parciales
\newcommand{\del}[2]{\frac{\partial #1}{\partial #2}}
\newcommand{\Del}[3]{\frac{\partial^{#1} #2}{\partial #3^{#1}}}
\newcommand{\deld}[2]{\dfrac{\partial #1}{\partial #2}}
\newcommand{\Deld}[3]{\dfrac{\partial^{#1} #2}{\partial #3^{#1}}}


\newcommand{\AstIg}{\stackrel{(\ast)}{=}}
\newcommand{\Hop}{\stackrel{L'H\hat{o}pital}{=}}

\newcommand{\red}[1]{{\color{red}#1}} % Para integrales, destacar los cambios.

% Método de integración
\newcommand{\MetInt}[2]{
    \left[\begin{array}{c}
        #1 \\ #2
    \end{array}\right]
}

% Declarar aplicaciones
% 1. Nombre aplicación
% 2. Dominio
% 3. Codominio
% 4. Variable
% 5. Imagen de la variable
\newcommand{\Func}[5]{
    \begin{equation*}
        \begin{array}{rrll}
            #1:& #2 & \longrightarrow & #3\\
               & #4 & \longmapsto & #5
        \end{array}
    \end{equation*}
}

%------------------------------------------------------------------------



\begin{document}

    % 1. Foto de fondo
    % 2. Título
    % 3. Encabezado Izquierdo
    % 4. Color de fondo
    % 5. Coord x del titulo
    % 6. Coord y del titulo
    % 7. Fecha

    
    % 1. Foto de fondo
% 2. Título
% 3. Encabezado Izquierdo
% 4. Color de fondo
% 5. Coord x del titulo
% 6. Coord y del titulo
% 7. Fecha

\newcommand{\portada}[7]{

    \portadaBase{#1}{#2}{#3}{#4}{#5}{#6}{#7}
    \portadaBook{#1}{#2}{#3}{#4}{#5}{#6}{#7}
}

\newcommand{\portadaExamen}[7]{

    \portadaBase{#1}{#2}{#3}{#4}{#5}{#6}{#7}
    \portadaArticle{#1}{#2}{#3}{#4}{#5}{#6}{#7}
}




\newcommand{\portadaBase}[7]{

    % Tiene la portada principal y la licencia Creative Commons
    
    % 1. Foto de fondo
    % 2. Título
    % 3. Encabezado Izquierdo
    % 4. Color de fondo
    % 5. Coord x del titulo
    % 6. Coord y del titulo
    % 7. Fecha
    
    
    \thispagestyle{empty}               % Sin encabezado ni pie de página
    \newgeometry{margin=0cm}        % Márgenes nulos para la primera página
    
    
    % Encabezado
    \fancyhead[L]{\helv #3}
    \fancyhead[R]{\helv \nouppercase{\leftmark}}
    
    
    \pagecolor{#4}        % Color de fondo para la portada
    
    \begin{figure}[p]
        \centering
        \transparent{0.3}           % Opacidad del 30% para la imagen
        
        \includegraphics[width=\paperwidth, keepaspectratio]{assets/#1}
    
        \begin{tikzpicture}[remember picture, overlay]
            \node[anchor=north west, text=white, opacity=1, font=\fontsize{60}{90}\selectfont\bfseries\sffamily, align=left] at (#5, #6) {#2};
            
            \node[anchor=south east, text=white, opacity=1, font=\fontsize{12}{18}\selectfont\sffamily, align=right] at (9.7, 3) {\textbf{\href{https://losdeldgiim.github.io/}{Los Del DGIIM}}};
            
            \node[anchor=south east, text=white, opacity=1, font=\fontsize{12}{15}\selectfont\sffamily, align=right] at (9.7, 1.8) {Doble Grado en Ingeniería Informática y Matemáticas\\Universidad de Granada};
        \end{tikzpicture}
    \end{figure}
    
    
    \restoregeometry        % Restaurar márgenes normales para las páginas subsiguientes
    \pagecolor{white}       % Restaurar el color de página
    
    
    \newpage
    \thispagestyle{empty}               % Sin encabezado ni pie de página
    \begin{tikzpicture}[remember picture, overlay]
        \node[anchor=south west, inner sep=3cm] at (current page.south west) {
            \begin{minipage}{0.5\paperwidth}
                \href{https://creativecommons.org/licenses/by-nc-nd/4.0/}{
                    \includegraphics[height=2cm]{assets/Licencia.png}
                }\vspace{1cm}\\
                Esta obra está bajo una
                \href{https://creativecommons.org/licenses/by-nc-nd/4.0/}{
                    Licencia Creative Commons Atribución-NoComercial-SinDerivadas 4.0 Internacional (CC BY-NC-ND 4.0).
                }\\
    
                Eres libre de compartir y redistribuir el contenido de esta obra en cualquier medio o formato, siempre y cuando des el crédito adecuado a los autores originales y no persigas fines comerciales. 
            \end{minipage}
        };
    \end{tikzpicture}
    
    
    
    % 1. Foto de fondo
    % 2. Título
    % 3. Encabezado Izquierdo
    % 4. Color de fondo
    % 5. Coord x del titulo
    % 6. Coord y del titulo
    % 7. Fecha


}


\newcommand{\portadaBook}[7]{

    % 1. Foto de fondo
    % 2. Título
    % 3. Encabezado Izquierdo
    % 4. Color de fondo
    % 5. Coord x del titulo
    % 6. Coord y del titulo
    % 7. Fecha

    % Personaliza el formato del título
    \pretitle{\begin{center}\bfseries\fontsize{42}{56}\selectfont}
    \posttitle{\par\end{center}\vspace{2em}}
    
    % Personaliza el formato del autor
    \preauthor{\begin{center}\Large}
    \postauthor{\par\end{center}\vfill}
    
    % Personaliza el formato de la fecha
    \predate{\begin{center}\huge}
    \postdate{\par\end{center}\vspace{2em}}
    
    \title{#2}
    \author{\href{https://losdeldgiim.github.io/}{Los Del DGIIM}}
    \date{Granada, #7}
    \maketitle
    
    \tableofcontents
}




\newcommand{\portadaArticle}[7]{

    % 1. Foto de fondo
    % 2. Título
    % 3. Encabezado Izquierdo
    % 4. Color de fondo
    % 5. Coord x del titulo
    % 6. Coord y del titulo
    % 7. Fecha

    % Personaliza el formato del título
    \pretitle{\begin{center}\bfseries\fontsize{42}{56}\selectfont}
    \posttitle{\par\end{center}\vspace{2em}}
    
    % Personaliza el formato del autor
    \preauthor{\begin{center}\Large}
    \postauthor{\par\end{center}\vspace{3em}}
    
    % Personaliza el formato de la fecha
    \predate{\begin{center}\huge}
    \postdate{\par\end{center}\vspace{5em}}
    
    \title{#2}
    \author{\href{https://losdeldgiim.github.io/}{Los Del DGIIM}}
    \date{Granada, #7}
    \thispagestyle{empty}               % Sin encabezado ni pie de página
    \maketitle
    \vfill
}
    \portadaExamen{ffccA4.jpg}{Geometría II\\Examen VI}{Geometría II. Examen VI}{MidnightBlue}{-8}{28}{2023}{Arturo Olivares Martos}

    \begin{description}
        \item[Asignatura] Geometría II.
        \item[Curso Académico] 2021-22.
        \item[Grado] Matemáticas.
        %\item[Grupo] A.
        \item[Profesor] Francisco Milán López\footnote{El examen lo pone el departamento.}.
        \item[Descripción] Convocatoria Ordinaria.
        \item[Fecha] 15 de junio de 2022.
        %\item[Duración] 60 minutos.
    
    \end{description}
    \newpage
    
    \begin{ejercicio}
Se considera la matriz real
\begin{equation*}
    A=\left(\begin{array}{ccc}
        0 & a & a \\
        a & -1 & -1 \\
        0 & 1 & 1
    \end{array}\right), \qquad a\in \bb{R}
\end{equation*}
con polinomio característico $P_A(\lambda)=-\lambda^3+a^2\lambda$.

\begin{enumerate}
    \item Encontrar los valores de $a$ para los que $A$ es diagonalizable.

    Tenemos que su polinomio característico es:
    \begin{equation*}
        P_A(\lambda)=\lambda(-\lambda^2 +a^2)=\lambda(a+\lambda)(a-\lambda)
    \end{equation*}

    Por tanto, los valores propios son $\lambda=\{0,a,-a\}$. Por tanto,
    \begin{itemize}
        \item \underline{Si $a\neq0$}: Tenemos que los tres valores propios son distintos, por lo que son diagonalizables.

        \item \underline{Si $a=0$}: Tenemos que la multiplicidad algebraica de $\lambda=0$ es tres, pero $\dim V_0=\dim Ker (f)=3-1=2$. Por tanto, no es diagonalizable.
    \end{itemize}

    \item Diagonalizar para $a=1$.

    \item Para $a=0$, estudiar si $\exists B\in M_3(\bb{R})$ diagonalizable tal que $$B^2+A=0$$

    Supongamos que $\exists B\in \cc{M}_3(\bb{R})$ diagonalizable. Entonces:
    \begin{equation*}
        B \text{ diagonalizable} \Longrightarrow
        -B^2 \text{ diagonalizable}
    \end{equation*}

    No obstante, tenemos que $-B^2=A$ no es diagonalizable para $a=0$, por lo que llegamos a una contradicción. No existe la matriz buscada.

    
\end{enumerate}

\end{ejercicio}


\begin{ejercicio}
    Sea $g_a$ la métrica en $\bb{R}^3$ cuya forma cuadrática está dada por:
    \begin{equation*}
        F_a(x_1, x_2, x_3) = ax_1^2 + x_2^2 + 2(1 - a)x_1x_3 + ax_3^2
    \end{equation*}

    \begin{enumerate}

        \item Clasificar $g_a$ según los valores de $a\in \bb{R}$:

        En primer lugar, calculamos la matriz asociada a $g_a$:
        \begin{equation*}
            G_a = M(g_a, \cc{B}_u) = \left(\begin{array}{ccc}
                a & 0 & 1-a \\
                0 & 1 & 0 \\
                1-a & 0 & a
            \end{array}\right)
        \end{equation*}

        Calculamos el determinante:
        \begin{equation*}
            |G_a| = a^2 -(1-a)^2 = a^2 -a^2 -1 +2a = 2a-1 = 0\Longleftrightarrow a=\frac{1}{2}
        \end{equation*}
        
        \begin{itemize}
            \item \underline{Para $a=\frac{1}{2}$}: Tenemos que $Nul(g_a)=1$. Además, para $U=\cc{L}\{e_2, e_3\}$, tenemos que la restricción es definida positiva. Por tanto, tenemos que:
            \begin{equation*}
                Nul(g_a)=1 \qquad Ind(g_a)=0
            \end{equation*}
            En este caso $g_a$ es semidefinida positiva.

            \item \underline{Para $a>\frac{1}{2}$}: Tenemos que $|G_a|>0$. Además, tenemos $G_a$ es definida positiva al ser todos sus menores principales positivos, por lo que $g_a$ también es definida positiva.

            \item \underline{Para $a <\frac{1}{2}$}: Tenemos que $|G_a|<0$. Además, $g_a(e_2, e_2)=1>0$, tenemos que $g_a$ tiene al menos un 1 en la matriz asociada a la base de Sylvester. Por tanto, como $Nul(g_a)=0$ y $|G_a|<0$, es necesario que:
            \begin{equation*}
                G_a\sim_c \left(\begin{array}{ccc}
                    1 &  \\
                     & 1 &  \\
                    &  & -1
                \end{array}\right)
            \end{equation*}

            Por tanto, $g_a$ es indefinida y $Nul(g_a)=0$, $Ind(g_a)=1$.
        \end{itemize}


        \item Calcular una base ortogonal de $g_{-1}$.

        \begin{comment}
        Tenemos que:
        \begin{equation*}
            G_{-1} = M(g_{-1}, \cc{B}_u) = \left(\begin{array}{ccc}
                -1 & 0 & 2 \\
                0 & 1 & 0 \\
                2 & 0 & -1
            \end{array}\right)
        \end{equation*}

        Calculamos su polinomio característico:
        \begin{equation*}
            P_{G_{-1}}(\lambda)
        \end{equation*}
        \end{comment}
        
        \item Calcular el núcleo de $g_a$.

        Usando lo calculado en el primer apartado,
        \begin{itemize}
            \item \underline{Para $a\neq \frac{1}{2}$}:

            Tenemos que $g_a$ es no degenerada, por lo que $Ker(g_a)=\{0\}$.

            \item \underline{Para $a = \frac{1}{2}$}:

            Tenemos que $rg(G_a)=2$, por lo que $\dim Ker(g_a) = 1$.
            \begin{equation*}
                \begin{split}
                    Ker(g_a) &= \left\{\left(\begin{array}{c}
                        x \\ y \\ z
                    \end{array}\right)\in \bb{R}^3 \left|
                    \left(\begin{array}{ccc}
                        \frac{1}{2} & 0 & \frac{1}{2} \\
                        0 & 1 & 0 \\
                        \frac{1}{2} & 0 & \frac{1}{2}
                    \end{array}\right)
                    \left(\begin{array}{c}
                        x \\ y \\ z
                    \end{array}\right) = 0
                    \right.\right\} \\
                    &= \cc{L}\left\{\left(\begin{array}{c}
                        1 \\ 0 \\ -1
                    \end{array}\right)\right\}
                \end{split}
            \end{equation*}
        \end{itemize}

        \item Resolver $F_3(x_1, x_2, x_3) = 0$.


        Tenemos que:
        \begin{equation*}
            F_3\left(\begin{array}{c}
                        x \\ y \\ z
                    \end{array}\right)
            =\left(\begin{array}{cccc}
                        x & y & z
                    \end{array}\right) 
            G_3
            \left(\begin{array}{c}
                        x \\ y \\ z
                    \end{array}\right) = 0
        \end{equation*}

        Por tanto,
        \begin{equation*}
            F_3\left(\begin{array}{c}
                        x \\ y \\ z
                    \end{array}\right)
            = 0 \equiv \left\{v\in \bb{R}^3 \mid g_3(v,v)=0\right\}
        \end{equation*}

        Para $a=3$, tenemos que $g_a$ es definida positiva. Por tanto, el único vector con cuadrado nulo es $v=0$. Por tanto, la solución es un punto, el origen.        
    \end{enumerate}
\end{ejercicio}


\begin{ejercicio}
    Sea $(\bb{R}^3, \langle,\rangle)$ el espacio vectorial euclídeo dado por el producto escalar y $U_1,U_2\subset \bb{R}^3$ dos planos vectoriales distintos.
    \begin{enumerate}
        \item Demostrar que existe una simetría axial $s\in End(\bb{R}^3)$ verificando:
        \begin{equation*}
            s(U_1)=U_1 \hspace{2cm} s(U_2)=U_2
        \end{equation*}

        Consideramos el subespacio vectorial $L=U_1\cap U_2$. Como los planos son distintos, se cortan en una recta. Sea la recta $L=\cc{L}\{e\}$. Como $L=U_1\cap U_2$, tenemos que:
        \begin{gather*}
            L\subset U_1 \Longrightarrow U_1=\cc{L}\{e, e_1\} \\
            L\subset U_2 \Longrightarrow U_2=\cc{L}\{e, e_2\}
        \end{gather*}

        Suponemos sin pérdida de generalidad que $e_1,e_2\perp e$, por lo que $e_1,e_2\in L^\perp$. Por tanto,
        \begin{gather*}
            \forall u_1=ae+be_1\in U_1,\quad s(u_1)=ae-be_1\in U_1 \Longrightarrow s(U_1)\subset U_1 \\
            \forall u_2=ae+be_2\in U_2,\quad s(u_2)=ae-be_2\in U_2 \Longrightarrow s(U_2)\subset U_2
        \end{gather*}

        Por tanto, la simetría respecto de $L$ cumple lo pedido.


        \item Si consideramos los planos vectoriales
        \begin{equation*}
            U_1 = \{(x, y, z) \in \bb{R}^3 \mid x + y = 0\}
            \qquad
            U_2 = \{(x, y, z) \in \bb{R}^3 \mid x - z = 0\}
        \end{equation*}

        encontrar la matriz de $s$ respecto de la base usual.

        Sea $\cc{B}_u=\{e_1, e_2, e_3\}$. Tenemos que, en este caso, $U=\cc{L}\left\{\left(\begin{array}{c}
            1 \\ -1 \\ 1
        \end{array}\right)\right\} = \{e_1-e_2+e_3\}$. Por tanto, tenemos que:
        \begin{equation*}
            U^\perp = \cc{L}\left\{
            \left(\begin{array}{c}
                1 \\ 1 \\ 0
            \end{array}\right),
            \left(\begin{array}{c}
                1 \\ 0 \\ -1
            \end{array}\right)
            \right\} = \{e_1+e_2, e_1-e_3\}
        \end{equation*}

        Por tanto, sabiendo que $U=V_1, U^\perp = V_{-1}$, tenemos que:
        \begin{equation*}
            \left\{\begin{array}{rl}
                s_U(e_1-e_2+e_3) &= e_1-e_2+e_3 = s_U(e_1) -s_U(e_2) + s_U(e_3) \\
                s_U(e_1+e_2) &= -e_1-e_2 = s_U(e_1) +s_U(e_2) \Longrightarrow s_U(e_2) = -s_U(e_1)-e_1-e_2\\
                s_U(e_1-e_3) &= -e_1+e_3 = s_U(e_1) -s_U(e_3) \Longrightarrow s_U(e_3)=s_U(e_1)+e_1-e_3\\
            \end{array}\right.
        \end{equation*}
        Sustituyendo en la primera ecuación, tenemos:
        \begin{equation*}
            e_1-e_2+e_3 = s_U(e_1) +s_U(e_1)+e_1+e_2 + s_U(e_1)+e_1-e_3
        \end{equation*}

        Por tanto, tenemos:
        \begin{equation*}
            \left\{\begin{array}{l}
                3s_U(e_1) = -e_1-2e_2+2e_3 \\
                3s_U(e_2) = -3s_U(e_1)-3e_1-3e_2 = e_1+2e_2-2e_3 -3e_1-3e_2 = -2e_1-e_2-2e_3\\
                3s_U(e_3) = 3s_U(e_1)+3e_1-3e_3 = -e_1-2e_2+2e_3+3e_1-3e_3 = 2e_1-2e_2-e_3
            \end{array}\right.
        \end{equation*}

        Por tanto, la matriz buscada es:
        \begin{equation*}
            M(s_U, \cc{B}_u) = \frac{1}{3}\left(\begin{array}{ccc}
                -1 & -2 & 2 \\
                -2 & -1 & -2\\
                2 & -2 & -1
            \end{array}\right)
        \end{equation*}
    \end{enumerate}
\end{ejercicio}

\end{document}