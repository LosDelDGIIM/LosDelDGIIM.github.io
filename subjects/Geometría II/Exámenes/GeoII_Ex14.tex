\documentclass[12pt]{article}

% Idioma y codificación
\usepackage[spanish, es-tabla, es-notilde]{babel}       %es-tabla para que se titule "Tabla"
\usepackage[utf8]{inputenc}

% Márgenes
\usepackage[a4paper,top=3cm,bottom=2.5cm,left=3cm,right=3cm]{geometry}

% Comentarios de bloque
\usepackage{verbatim}

% Paquetes de links
\usepackage[hidelinks]{hyperref}    % Permite enlaces
\usepackage{url}                    % redirecciona a la web

% Más opciones para enumeraciones
\usepackage{enumitem}

% Personalizar la portada
\usepackage{titling}

% Paquetes de tablas
\usepackage{multirow}

% Para añadir el símbolo de euro
\usepackage{eurosym}


%------------------------------------------------------------------------

%Paquetes de figuras
\usepackage{caption}
\usepackage{subcaption} % Figuras al lado de otras
\usepackage{float}      % Poner figuras en el sitio indicado H.


% Paquetes de imágenes
\usepackage{graphicx}       % Paquete para añadir imágenes
\usepackage{transparent}    % Para manejar la opacidad de las figuras

% Paquete para usar colores
\usepackage[dvipsnames, table, xcdraw]{xcolor}
\usepackage{pagecolor}      % Para cambiar el color de la página

% Habilita tamaños de fuente mayores
\usepackage{fix-cm}

% Para los gráficos
\usepackage{tikz}
\usepackage{forest}

% Para poder situar los nodos en los grafos
\usetikzlibrary{positioning}


%------------------------------------------------------------------------

% Paquetes de matemáticas
\usepackage{mathtools, amsfonts, amssymb, mathrsfs}
\usepackage[makeroom]{cancel}     % Simplificar tachando
\usepackage{polynom}    % Divisiones y Ruffini
\usepackage{units} % Para poner fracciones diagonales con \nicefrac

\usepackage{pgfplots}   %Representar funciones
\pgfplotsset{compat=1.18}  % Versión 1.18

\usepackage{tikz-cd}    % Para usar diagramas de composiciones
\usetikzlibrary{calc}   % Para usar cálculo de coordenadas en tikz

%Definición de teoremas, etc.
\usepackage{amsthm}
%\swapnumbers   % Intercambia la posición del texto y de la numeración

\theoremstyle{plain}

\makeatletter
\@ifclassloaded{article}{
  \newtheorem{teo}{Teorema}[section]
}{
  \newtheorem{teo}{Teorema}[chapter]  % Se resetea en cada chapter
}
\makeatother

\newtheorem{coro}{Corolario}[teo]           % Se resetea en cada teorema
\newtheorem{prop}[teo]{Proposición}         % Usa el mismo contador que teorema
\newtheorem{lema}[teo]{Lema}                % Usa el mismo contador que teorema
\newtheorem*{lema*}{Lema}

\theoremstyle{remark}
\newtheorem*{observacion}{Observación}

\theoremstyle{definition}

\makeatletter
\@ifclassloaded{article}{
  \newtheorem{definicion}{Definición} [section]     % Se resetea en cada chapter
}{
  \newtheorem{definicion}{Definición} [chapter]     % Se resetea en cada chapter
}
\makeatother

\newtheorem*{notacion}{Notación}
\newtheorem*{ejemplo}{Ejemplo}
\newtheorem*{ejercicio*}{Ejercicio}             % No numerado
\newtheorem{ejercicio}{Ejercicio} [section]     % Se resetea en cada section


% Modificar el formato de la numeración del teorema "ejercicio"
\renewcommand{\theejercicio}{%
  \ifnum\value{section}=0 % Si no se ha iniciado ninguna sección
    \arabic{ejercicio}% Solo mostrar el número de ejercicio
  \else
    \thesection.\arabic{ejercicio}% Mostrar número de sección y número de ejercicio
  \fi
}


% \renewcommand\qedsymbol{$\blacksquare$}         % Cambiar símbolo QED
%------------------------------------------------------------------------

% Paquetes para encabezados
\usepackage{fancyhdr}
\pagestyle{fancy}
\fancyhf{}

\newcommand{\helv}{ % Modificación tamaño de letra
\fontfamily{}\fontsize{12}{12}\selectfont}
\setlength{\headheight}{15pt} % Amplía el tamaño del índice


%\usepackage{lastpage}   % Referenciar última pag   \pageref{LastPage}
%\fancyfoot[C]{%
%  \begin{minipage}{\textwidth}
%    \centering
%    ~\\
%    \thepage\\
%    \href{https://losdeldgiim.github.io/}{\texttt{\footnotesize losdeldgiim.github.io}}
%  \end{minipage}
%}
\fancyfoot[C]{\thepage}
\fancyfoot[R]{\href{https://losdeldgiim.github.io/}{\texttt{\footnotesize losdeldgiim.github.io}}}

%------------------------------------------------------------------------

% Conseguir que no ponga "Capítulo 1". Sino solo "1."
\makeatletter
\@ifclassloaded{book}{
  \renewcommand{\chaptermark}[1]{\markboth{\thechapter.\ #1}{}} % En el encabezado
    
  \renewcommand{\@makechapterhead}[1]{%
  \vspace*{50\p@}%
  {\parindent \z@ \raggedright \normalfont
    \ifnum \c@secnumdepth >\m@ne
      \huge\bfseries \thechapter.\hspace{1em}\ignorespaces
    \fi
    \interlinepenalty\@M
    \Huge \bfseries #1\par\nobreak
    \vskip 40\p@
  }}
}
\makeatother

%------------------------------------------------------------------------
% Paquetes de cógido
\usepackage{minted}
\renewcommand\listingscaption{Código fuente}

\usepackage{fancyvrb}
% Personaliza el tamaño de los números de línea
\renewcommand{\theFancyVerbLine}{\small\arabic{FancyVerbLine}}

% Estilo para C++
\newminted{cpp}{
    frame=lines,
    framesep=2mm,
    baselinestretch=1.2,
    linenos,
    escapeinside=||
}

% para minted
\definecolor{LightGray}{rgb}{0.95,0.95,0.92}
\setminted{
    linenos=true,
    stepnumber=5,
    numberfirstline=true,
    autogobble,
    breaklines=true,
    breakautoindent=true,
    breaksymbolleft=,
    breaksymbolright=,
    breaksymbolindentleft=0pt,
    breaksymbolindentright=0pt,
    breaksymbolsepleft=0pt,
    breaksymbolsepright=0pt,
    fontsize=\footnotesize,
    bgcolor=LightGray,
    numbersep=10pt
}


\usepackage{listings} % Para incluir código desde un archivo

\renewcommand\lstlistingname{Código Fuente}
\renewcommand\lstlistlistingname{Índice de Códigos Fuente}

% Definir colores
\definecolor{vscodepurple}{rgb}{0.5,0,0.5}
\definecolor{vscodeblue}{rgb}{0,0,0.8}
\definecolor{vscodegreen}{rgb}{0,0.5,0}
\definecolor{vscodegray}{rgb}{0.5,0.5,0.5}
\definecolor{vscodebackground}{rgb}{0.97,0.97,0.97}
\definecolor{vscodelightgray}{rgb}{0.9,0.9,0.9}

% Configuración para el estilo de C similar a VSCode
\lstdefinestyle{vscode_C}{
  backgroundcolor=\color{vscodebackground},
  commentstyle=\color{vscodegreen},
  keywordstyle=\color{vscodeblue},
  numberstyle=\tiny\color{vscodegray},
  stringstyle=\color{vscodepurple},
  basicstyle=\scriptsize\ttfamily,
  breakatwhitespace=false,
  breaklines=true,
  captionpos=b,
  keepspaces=true,
  numbers=left,
  numbersep=5pt,
  showspaces=false,
  showstringspaces=false,
  showtabs=false,
  tabsize=2,
  frame=tb,
  framerule=0pt,
  aboveskip=10pt,
  belowskip=10pt,
  xleftmargin=10pt,
  xrightmargin=10pt,
  framexleftmargin=10pt,
  framexrightmargin=10pt,
  framesep=0pt,
  rulecolor=\color{vscodelightgray},
  backgroundcolor=\color{vscodebackground},
}

%------------------------------------------------------------------------

% Comandos definidos
\newcommand{\bb}[1]{\mathbb{#1}}
\newcommand{\cc}[1]{\mathcal{#1}}

% I prefer the slanted \leq
\let\oldleq\leq % save them in case they're every wanted
\let\oldgeq\geq
\renewcommand{\leq}{\leqslant}
\renewcommand{\geq}{\geqslant}

% Si y solo si
\newcommand{\sii}{\iff}

% MCD y MCM
\DeclareMathOperator{\mcd}{mcd}
\DeclareMathOperator{\mcm}{mcm}

% Signo
\DeclareMathOperator{\sgn}{sgn}

% Letras griegas
\newcommand{\eps}{\epsilon}
\newcommand{\veps}{\varepsilon}
\newcommand{\lm}{\lambda}

\newcommand{\ol}{\overline}
\newcommand{\ul}{\underline}
\newcommand{\wt}{\widetilde}
\newcommand{\wh}{\widehat}

\let\oldvec\vec
\renewcommand{\vec}{\overrightarrow}

% Derivadas parciales
\newcommand{\del}[2]{\frac{\partial #1}{\partial #2}}
\newcommand{\Del}[3]{\frac{\partial^{#1} #2}{\partial #3^{#1}}}
\newcommand{\deld}[2]{\dfrac{\partial #1}{\partial #2}}
\newcommand{\Deld}[3]{\dfrac{\partial^{#1} #2}{\partial #3^{#1}}}


\newcommand{\AstIg}{\stackrel{(\ast)}{=}}
\newcommand{\Hop}{\stackrel{L'H\hat{o}pital}{=}}

\newcommand{\red}[1]{{\color{red}#1}} % Para integrales, destacar los cambios.

% Método de integración
\newcommand{\MetInt}[2]{
    \left[\begin{array}{c}
        #1 \\ #2
    \end{array}\right]
}

% Declarar aplicaciones
% 1. Nombre aplicación
% 2. Dominio
% 3. Codominio
% 4. Variable
% 5. Imagen de la variable
\newcommand{\Func}[5]{
    \begin{equation*}
        \begin{array}{rrll}
            \displaystyle #1:& \displaystyle  #2 & \longrightarrow & \displaystyle  #3\\
               & \displaystyle  #4 & \longmapsto & \displaystyle  #5
        \end{array}
    \end{equation*}
}

%------------------------------------------------------------------------


\begin{document}
	
	% 1. Foto de fondo
	% 2. Título
	% 3. Encabezado Izquierdo
	% 4. Color de fondo
	% 5. Coord x del titulo
	% 6. Coord y del titulo
	% 7. Fecha
	
	
	% 1. Foto de fondo
% 2. Título
% 3. Encabezado Izquierdo
% 4. Color de fondo
% 5. Coord x del titulo
% 6. Coord y del titulo
% 7. Fecha
% 8. Autor

\newcommand{\portada}[8]{
    \portadaBase{#1}{#2}{#3}{#4}{#5}{#6}{#7}{#8}
    \portadaBook{#1}{#2}{#3}{#4}{#5}{#6}{#7}{#8}
}

\newcommand{\portadaFotoDif}[8]{
    \portadaBaseFotoDif{#1}{#2}{#3}{#4}{#5}{#6}{#7}{#8}
    \portadaBook{#1}{#2}{#3}{#4}{#5}{#6}{#7}{#8}
}

\newcommand{\portadaExamen}[8]{
    \portadaBase{#1}{#2}{#3}{#4}{#5}{#6}{#7}{#8}
    \portadaArticle{#1}{#2}{#3}{#4}{#5}{#6}{#7}{#8}
}

\newcommand{\portadaExamenFotoDif}[8]{
    \portadaBaseFotoDif{#1}{#2}{#3}{#4}{#5}{#6}{#7}{#8}
    \portadaArticle{#1}{#2}{#3}{#4}{#5}{#6}{#7}{#8}
}




\newcommand{\portadaBase}[8]{

    % Tiene la portada principal y la licencia Creative Commons
    
    % 1. Foto de fondo
    % 2. Título
    % 3. Encabezado Izquierdo
    % 4. Color de fondo
    % 5. Coord x del titulo
    % 6. Coord y del titulo
    % 7. Fecha
    % 8. Autor    
    
    \thispagestyle{empty}               % Sin encabezado ni pie de página
    \newgeometry{margin=0cm}        % Márgenes nulos para la primera página
    
    
    % Encabezado
    \fancyhead[L]{\helv #3}
    \fancyhead[R]{\helv \nouppercase{\leftmark}}
    
    
    \pagecolor{#4}        % Color de fondo para la portada
    
    \begin{figure}[p]
        \centering
        \transparent{0.3}           % Opacidad del 30% para la imagen
        
        \includegraphics[width=\paperwidth, keepaspectratio]{../../_assets/#1}
    
        \begin{tikzpicture}[remember picture, overlay]
            \node[anchor=north west, text=white, opacity=1, font=\fontsize{60}{90}\selectfont\bfseries\sffamily, align=left] at (#5, #6) {#2};
            
            \node[anchor=south east, text=white, opacity=1, font=\fontsize{12}{18}\selectfont\sffamily, align=right] at (9.7, 3) {\href{https://losdeldgiim.github.io/}{\textbf{Los Del DGIIM}, \texttt{\footnotesize losdeldgiim.github.io}}};
            
            \node[anchor=south east, text=white, opacity=1, font=\fontsize{12}{15}\selectfont\sffamily, align=right] at (9.7, 1.8) {Doble Grado en Ingeniería Informática y Matemáticas\\Universidad de Granada};
        \end{tikzpicture}
    \end{figure}
    
    
    \restoregeometry        % Restaurar márgenes normales para las páginas subsiguientes
    \nopagecolor      % Restaurar el color de página
    
    
    \newpage
    \thispagestyle{empty}               % Sin encabezado ni pie de página
    \begin{tikzpicture}[remember picture, overlay]
        \node[anchor=south west, inner sep=3cm] at (current page.south west) {
            \begin{minipage}{0.5\paperwidth}
                \href{https://creativecommons.org/licenses/by-nc-nd/4.0/}{
                    \includegraphics[height=2cm]{../../_assets/Licencia.png}
                }\vspace{1cm}\\
                Esta obra está bajo una
                \href{https://creativecommons.org/licenses/by-nc-nd/4.0/}{
                    Licencia Creative Commons Atribución-NoComercial-SinDerivadas 4.0 Internacional (CC BY-NC-ND 4.0).
                }\\
    
                Eres libre de compartir y redistribuir el contenido de esta obra en cualquier medio o formato, siempre y cuando des el crédito adecuado a los autores originales y no persigas fines comerciales. 
            \end{minipage}
        };
    \end{tikzpicture}
    
    
    
    % 1. Foto de fondo
    % 2. Título
    % 3. Encabezado Izquierdo
    % 4. Color de fondo
    % 5. Coord x del titulo
    % 6. Coord y del titulo
    % 7. Fecha
    % 8. Autor


}


\newcommand{\portadaBaseFotoDif}[8]{

    % Tiene la portada principal y la licencia Creative Commons
    
    % 1. Foto de fondo
    % 2. Título
    % 3. Encabezado Izquierdo
    % 4. Color de fondo
    % 5. Coord x del titulo
    % 6. Coord y del titulo
    % 7. Fecha
    % 8. Autor    
    
    \thispagestyle{empty}               % Sin encabezado ni pie de página
    \newgeometry{margin=0cm}        % Márgenes nulos para la primera página
    
    
    % Encabezado
    \fancyhead[L]{\helv #3}
    \fancyhead[R]{\helv \nouppercase{\leftmark}}
    
    
    \pagecolor{#4}        % Color de fondo para la portada
    
    \begin{figure}[p]
        \centering
        \transparent{0.3}           % Opacidad del 30% para la imagen
        
        \includegraphics[width=\paperwidth, keepaspectratio]{#1}
    
        \begin{tikzpicture}[remember picture, overlay]
            \node[anchor=north west, text=white, opacity=1, font=\fontsize{60}{90}\selectfont\bfseries\sffamily, align=left] at (#5, #6) {#2};
            
            \node[anchor=south east, text=white, opacity=1, font=\fontsize{12}{18}\selectfont\sffamily, align=right] at (9.7, 3) {\href{https://losdeldgiim.github.io/}{\textbf{Los Del DGIIM}, \texttt{\footnotesize losdeldgiim.github.io}}};
            
            \node[anchor=south east, text=white, opacity=1, font=\fontsize{12}{15}\selectfont\sffamily, align=right] at (9.7, 1.8) {Doble Grado en Ingeniería Informática y Matemáticas\\Universidad de Granada};
        \end{tikzpicture}
    \end{figure}
    
    
    \restoregeometry        % Restaurar márgenes normales para las páginas subsiguientes
    \nopagecolor      % Restaurar el color de página
    
    
    \newpage
    \thispagestyle{empty}               % Sin encabezado ni pie de página
    \begin{tikzpicture}[remember picture, overlay]
        \node[anchor=south west, inner sep=3cm] at (current page.south west) {
            \begin{minipage}{0.5\paperwidth}
                %\href{https://creativecommons.org/licenses/by-nc-nd/4.0/}{
                %    \includegraphics[height=2cm]{../../_assets/Licencia.png}
                %}\vspace{1cm}\\
                Esta obra está bajo una
                \href{https://creativecommons.org/licenses/by-nc-nd/4.0/}{
                    Licencia Creative Commons Atribución-NoComercial-SinDerivadas 4.0 Internacional (CC BY-NC-ND 4.0).
                }\\
    
                Eres libre de compartir y redistribuir el contenido de esta obra en cualquier medio o formato, siempre y cuando des el crédito adecuado a los autores originales y no persigas fines comerciales. 
            \end{minipage}
        };
    \end{tikzpicture}
    
    
    
    % 1. Foto de fondo
    % 2. Título
    % 3. Encabezado Izquierdo
    % 4. Color de fondo
    % 5. Coord x del titulo
    % 6. Coord y del titulo
    % 7. Fecha
    % 8. Autor


}


\newcommand{\portadaBook}[8]{

    % 1. Foto de fondo
    % 2. Título
    % 3. Encabezado Izquierdo
    % 4. Color de fondo
    % 5. Coord x del titulo
    % 6. Coord y del titulo
    % 7. Fecha
    % 8. Autor

    % Personaliza el formato del título
    \pretitle{\begin{center}\bfseries\fontsize{42}{56}\selectfont}
    \posttitle{\par\end{center}\vspace{2em}}
    
    % Personaliza el formato del autor
    \preauthor{\begin{center}\Large}
    \postauthor{\par\end{center}\vfill}
    
    % Personaliza el formato de la fecha
    \predate{\begin{center}\huge}
    \postdate{\par\end{center}\vspace{2em}}
    
    \title{#2}
    \author{\href{https://losdeldgiim.github.io/}{Los Del DGIIM, \texttt{\large losdeldgiim.github.io}}
    \\ \vspace{0.5cm}#8}
    \date{Granada, #7}
    \maketitle
    
    \tableofcontents
}




\newcommand{\portadaArticle}[8]{

    % 1. Foto de fondo
    % 2. Título
    % 3. Encabezado Izquierdo
    % 4. Color de fondo
    % 5. Coord x del titulo
    % 6. Coord y del titulo
    % 7. Fecha
    % 8. Autor

    % Personaliza el formato del título
    \pretitle{\begin{center}\bfseries\fontsize{42}{56}\selectfont}
    \posttitle{\par\end{center}\vspace{2em}}
    
    % Personaliza el formato del autor
    \preauthor{\begin{center}\Large}
    \postauthor{\par\end{center}\vspace{3em}}
    
    % Personaliza el formato de la fecha
    \predate{\begin{center}\huge}
    \postdate{\par\end{center}\vspace{5em}}
    
    \title{#2}
    \author{\href{https://losdeldgiim.github.io/}{Los Del DGIIM, \texttt{\large losdeldgiim.github.io}}
    \\ \vspace{0.5cm}#8}
    \date{Granada, #7}
    \thispagestyle{empty}               % Sin encabezado ni pie de página
    \maketitle
    \vfill
}
	\portadaExamen{ffccA4.jpg}{Geometría II\\Examen XIV}{Geometría II. Examen XIV}{MidnightBlue}{-8}{28}{2025}{Daniel Arias Calero}
	
	\begin{description}
		\item[Asignatura] Geometría II.
		\item[Curso Académico] 2024-2025.
		\item[Grado] Doble Grado en Ingeniería Informática y Matemáticas.
		\item[Grupo] Único.
		\item[Profesor] Antonio Ros Mulero.
		\item[Descripción] Convocatoria Extraordinaria.
		\item[Fecha] 9 de Julio de 2025.
		\item[Duración] 3 horas.
	\end{description}
	\newpage
	
	
	% ------------------------------------
	\begin{ejercicio}[2 puntos]
		Demuestra que las matrices $A$ y $B$ tienen el mismo polinomio característico, pero una de ellas diagonaliza y la otra no.
		
		\[
		A = \begin{pmatrix}
			3 & 1 & 0 \\
			1 & 3 & 2 \\
			0 & 0 & 2
		\end{pmatrix}, \quad
		B = \begin{pmatrix}
			3 & 1 & -1 \\
			0 & 2 & 0 \\
			-1 & -1 & 3
		\end{pmatrix}
		\]
		
	\end{ejercicio}

	\begin{ejercicio}[4 puntos]
		Sea $g$ una métrica euclídea sobre $\mathbb{R}^3$ tal que el endomorfismo $f$ dado por
		\[
		f(x, y, z) = (y,\ y - z,\ -x + y)
		\]
		es una isometría de $(\mathbb{R}^3, g)$. Se pide:
		\begin{enumerate}
			\item[(a)] Razonar que $f^2$ es la simetría respecto de un subespacio $U$.
			\item[(b)] Calcular el subespacio ortogonal $U^\perp$.
			\item[(c)] Razonar que $f$ es un giro de ángulo $\pi/2$ y encontrar una base ortogonal de $g$.
			\item[(d)] Dar una matriz, respecto de la base usual de $(\mathbb{R}^3)$, de una de las métricas de $g$.
		\end{enumerate}
	\end{ejercicio}

    \begin{ejercicio}[4 puntos]
    Se considera la siguiente matriz con coeficientes reales:
    \[
    A = \begin{pmatrix}
        1 & 0 & 0 \\
        0 & a & 1 \\
        0 & 1 & a
    \end{pmatrix}
    \]
    
    Sea $g_a$ la métrica de $\mathbb{R}^3$ cuya matriz en la base usual es $A$. Sea $f$ el endomorfismo de $\mathbb{R}^3$ cuya matriz en la base usual es también $A$.
    
    \begin{enumerate}
        \item[(a)] Calcular la signatura y clasificar la métrica $g_a$ según los valores de $a$.
        \item[(b)] Para $a = 1$ obtener una base conjugada (es decir, ortogonal) para la métrica $g_1$.
        \item[(c)] ¿Para qué valores de $a$ es $g_a$ una métrica euclídea y $f$ autoadjunto en $(\mathbb{R}^3, g_a)$?
        \item[(d)] Para $a = 2$ calcular, si es posible, una base ortogonal de $(\mathbb{R}^3, g_2)$ formada por vectores propios de $f$.
    \end{enumerate}
    \end{ejercicio}
    
    \newpage
    \setcounter{ejercicio}{0}
	
	\begin{ejercicio}[2 puntos]
		Demuestra que las matrices $A$ y $B$ tienen el mismo polinomio característico, pero una de ellas diagonaliza y la otra no.
		
		\[
		A = \begin{pmatrix}
			3 & 1 & 0 \\
			1 & 3 & 2 \\
			0 & 0 & 2
		\end{pmatrix}, \quad
		B = \begin{pmatrix}
			3 & 1 & -1 \\
			0 & 2 & 0 \\
			-1 & -1 & 3
		\end{pmatrix}
		\]
		
		\text{1. Cálculo del polinomio característico de \( A \)}
		
		\[
		P_A(\lambda) = \det(A - \lambda I) =
		\begin{vmatrix}
			3 - \lambda & 1 & 0 \\
			1 & 3 - \lambda & 2 \\
			0 & 0 & 2 - \lambda
		\end{vmatrix}
		\]
		
		\[
		= (2 - \lambda) \cdot \begin{vmatrix}
			3 - \lambda & 1 \\
			1 & 3 - \lambda
		\end{vmatrix}
		= (2 - \lambda) \left((3 - \lambda)^2 - 1\right)
		\]
		
		\[
		= (2 - \lambda)\left(\lambda^2 - 6\lambda + 8\right)
		= -(\lambda - 2)^2(\lambda - 4)
		\]
		
		\text{2. Cálculo del polinomio característico de \( B \)}
		
		\[
		P_B(\lambda) = \det(B - \lambda I) =
		\begin{vmatrix}
			3 - \lambda & 1 & -1 \\
			0 & 2 - \lambda & 0 \\
			-1 & -1 & 3 - \lambda
		\end{vmatrix}
		\]
		
		\[
		= (2 - \lambda)
		\begin{vmatrix}
			3 - \lambda & -1 \\
			-1 & 3 - \lambda
		\end{vmatrix}
		= (2 - \lambda)\left((3 - \lambda)^2 - 1\right)
		\]
		
		\[
		= (2 - \lambda)(\lambda^2 - 6\lambda + 8)
		= -(\lambda - 2)^2(\lambda - 4)
		\]
		
		\text{Ambas matrices tienen el mismo polinomio característico, veamos la diagonalización.}
		\medskip
		
		
		\text{3. Subespacio propio de \( A \) }
		\medskip
		
		\text{\( \lambda = 2 \):}
		
		\[
		V_2^A = \left\{
		\begin{pmatrix}
			x_1 \\
			x_2 \\
			x_3
		\end{pmatrix}
		\in \mathbb{R}^3 \ \middle| \ 
		(A - 2I)
		\begin{pmatrix}
			x_1 \\
			x_2 \\
			x_3
		\end{pmatrix}
		= 0
		\right\}
		\]
		
		\[
		= \left\{
		\begin{pmatrix}
			x_1 \\
			x_2 \\
			x_3
		\end{pmatrix}
		\in \mathbb{R}^3 \ \middle| \
		\begin{pmatrix}
			1 & 1 & 0 \\
			1 & 1 & 2 \\
			0 & 0 & 0
		\end{pmatrix}
		\begin{pmatrix}
			x_1 \\
			x_2 \\
			x_3
		\end{pmatrix}
		= 0
		\right\}
		\]
		
		\[
		= \left\{
		\begin{pmatrix}
			x_1 \\
			x_2 \\
			x_3
		\end{pmatrix}
		\in \mathbb{R}^3 \ \middle| \
		\begin{array}{l}
			x_1 + x_2 = 0 \\
			x_1 + x_2 + 2x_3 = 0
		\end{array}
		\right\}
		= \mathcal{L} \left( \begin{pmatrix} -1 \\ 1 \\ 0 \end{pmatrix} \right)
		\]
		
		\[
		\dim V_2^A = 1 < 2 \quad \Rightarrow \quad A \text{ no es diagonalizable.}
		\]
		
		\bigskip
		
		\text{4. Subespacio propio de \( B \)}
		
		\medskip
		
		\text{\( \lambda = 4 \):}
		
		\[
		V_4^B = \left\{
		\begin{pmatrix}
			x_1 \\
			x_2 \\
			x_3
		\end{pmatrix}
		\in \mathbb{R}^3 \ \middle| \
		(B - 4I)
		\begin{pmatrix}
			x_1 \\
			x_2 \\
			x_3
		\end{pmatrix}
		= 0
		\right\}
		\]
		
		\[
		= \left\{
		\begin{pmatrix}
			x_1 \\
			x_2 \\
			x_3
		\end{pmatrix}
		\in \mathbb{R}^3 \ \middle| \
		\begin{pmatrix}
			-1 & 1 & -1 \\
			0 & -2 & 0 \\
			-1 & -1 & -1
		\end{pmatrix}
		\begin{pmatrix}
			x_1 \\
			x_2 \\
			x_3
		\end{pmatrix}
		= 0
		\right\}
		\]
		
		\[
		= \left\{
		\begin{pmatrix}
			x_1 \\
			x_2 \\
			x_3
		\end{pmatrix}
		\in \mathbb{R}^3 \ \middle| \
		\begin{array}{l}
			x_2 = 0 \\
			x_1 = -x_3
		\end{array}
		\right\}
		= \mathcal{L} \left( \begin{pmatrix} -1 \\ 0 \\ 1 \end{pmatrix} \right)
		\]
		
		\[
		\Rightarrow \dim V_4^B = 1 = \text{multiplicidad algebráica}
		\]
		\medskip
		
		\text{\( \lambda = 2 \):}
		
		\[
		V_2^B = \left\{
		\begin{pmatrix}
			x_1 \\
			x_2 \\
			x_3
		\end{pmatrix}
		\in \mathbb{R}^3 \ \middle| \ 
		(B - 2I)
		\begin{pmatrix}
			x_1 \\
			x_2 \\
			x_3
		\end{pmatrix}
		= 0
		\right\}
		\]
		
		\[
		= \left\{
		\begin{pmatrix}
			x_1 \\
			x_2 \\
			x_3
		\end{pmatrix}
		\in \mathbb{R}^3 \ \middle| \
		\begin{pmatrix}
			1 & 1 & -1 \\
			0 & 0 & 0 \\
			-1 & -1 & 1
		\end{pmatrix}
		\begin{pmatrix}
			x_1 \\
			x_2 \\
			x_3
		\end{pmatrix}
		= 0
		\right\}
		\]
		
		\[
		= \left\{
		\begin{pmatrix}
			x_1 \\
			x_2 \\
			x_3
		\end{pmatrix}
		\in \mathbb{R}^3 \ \middle| \
		\begin{array}{l}
			x_1 + x_2 - x_3 = 0 \\
		\end{array}
		\right\}
		\]
		
		\[
		\Rightarrow \quad V_2^B = \mathcal{L} \left(
		\begin{pmatrix}
			1 \\ 0 \\ 1
		\end{pmatrix},
		\begin{pmatrix}
			1 \\ -1 \\ 0
		\end{pmatrix}
		\right)
		\quad \Rightarrow \quad \dim V_2^B = 2 = \text{multiplicidad algebráica}
		\]
		
		\bigskip
		
		\textbf{Conclusión:}
		
		Ambas matrices tienen el mismo polinomio característico, pero \( B \) es diagonalizable y \( A \) no.
	\end{ejercicio}
	
	\begin{ejercicio}[4 puntos]
		Sea $g$ una métrica euclídea sobre $\mathbb{R}^3$ tal que el endomorfismo $f$ dado por
		\[
		f(x, y, z) = (y,\ y - z,\ -x + y)
		\]
		es una isometría de $(\mathbb{R}^3, g)$. Se pide:
		\begin{enumerate}
			\item[(a)] Razonar que $f^2$ es la simetría respecto de un subespacio $U$.
			
					Para analizar la transformación compuesta \( f \circ f \), calculamos la matriz asociada a dicha composición en la base canónica de \( \mathbb{R}^3 \). Si llamamos
					
					\[
					A = M(f, \mathcal{B}_u) = \begin{pmatrix}
						0 & 1 & 0 \\
						0 & 1 & -1 \\
						-1 & 1 & 0
					\end{pmatrix},
					\]
					
					entonces la matriz de \( f \circ f \) se obtiene como:
					
					\[
					B = M(f \circ f, \mathcal{B}_u) = A \cdot A =
					\begin{pmatrix}
						0 & 1 & 0 \\
						0 & 1 & -1 \\
						-1 & 1 & 0
					\end{pmatrix}
					\cdot
					\begin{pmatrix}
						0 & 1 & 0 \\
						0 & 1 & -1 \\
						-1 & 1 & 0
					\end{pmatrix}
					=
					\begin{pmatrix}
						0 & 1 & -1 \\
						1 & 0 & -1 \\
						0 & 0 & -1
					\end{pmatrix}.
					\]
					
					\medskip
					
					Procedemos ahora a calcular el polinomio característico de \( B \):
					
					\[
					p_{f \circ f}(\lambda) = \det(B - \lambda I) =
					\left|
					\begin{array}{ccc}
						-\lambda & 1 & -1 \\
						1 & -\lambda & -1 \\
						0 & 0 & -1 - \lambda
					\end{array}
					\right|
					= (-1 - \lambda)
					\left|
					\begin{array}{cc}
						-\lambda & 1 \\
						1 & -\lambda
					\end{array}
					\right|
					= (-1 - \lambda)(\lambda^2 - 1).
					\]
					
					\[
					\Rightarrow p_{f \circ f}(\lambda) = -(\lambda - 1)(\lambda + 1)^2.
					\]
					
					Así, los valores propios de \( f \circ f \) son \( \{1, -1\} \), siendo \( -1 \) con multiplicidad doble.
					
					\medskip
					
					Comprobamos ahora si la dimensión del espacio propio asociado a \( -1 \) coincide con su multiplicidad algebraica. Para ello, estudiamos el núcleo de \( B + I \):
					
					\[
					B + I = 
					\begin{pmatrix}
						1 & 1 & -1 \\
						1 & 1 & -1 \\
						0 & 0 & 0
					\end{pmatrix} \Rightarrow \operatorname{rang}(B + I) = 1.
					\]
					
					\[
					\Rightarrow \dim(\ker(B + I)) = 3 - 1 = 2.
					\]
					
					\medskip
					
					Dado que la dimensión del subespacio propio coincide con la multiplicidad algebraica del valor propio \( -1 \), y que el valor propio \( 1 \) es simple, concluimos que \( f \circ f \) es diagonalizable.
					
					\medskip
					
					Por tanto, \( f \circ f \) es semejante a una matriz diagonal de la forma:
					
					\[
					\begin{pmatrix}
						1 & 0 & 0 \\
						0 & -1 & 0 \\
						0 & 0 & -1
					\end{pmatrix},
					\]
					
					que representa una simetría respecto a una recta del subespacio $U$.
				
			\item[(b)] Calcular el subespacio ortogonal $U^\perp$.
			
			Como la transformación \( f \circ f \) actúa como una simetría respecto a una recta, podemos descomponer el espacio en la suma directa \( \mathbb{R}^3 = U \oplus U^\perp \), donde \( U \) es el subespacio fijo por \( f \circ f \), es decir:
			
			\[
			U = \left\{ 
			(x, y, z) \in \mathbb{R}^3 \;\middle|\; 
			\begin{pmatrix}
				0 & 1 & -1 \\
				1 & 0 & -1 \\
				0 & 0 & -1
			\end{pmatrix}
			\begin{pmatrix}
				x \\ y \\ z
			\end{pmatrix}
			=
			\begin{pmatrix}
				x \\ y \\ z
			\end{pmatrix}
			\right\}.
			\]
			
			Equivalente a resolver el sistema \( (B - I)\vec{x} = 0 \):
			
			\[
			\left(
			\begin{array}{ccc}
				-1 & 1 & -1 \\
				1 & -1 & -1 \\
				0 & 0 & -2
			\end{array}
			\right)
			\begin{pmatrix}
				x \\ y \\ z
			\end{pmatrix}
			= 
			\begin{pmatrix}
				0 \\ 0 \\ 0
			\end{pmatrix}
			\Rightarrow
			\left\{
			\begin{array}{l}
				-x + y - z = 0 \\
				x - y - z = 0 \\
				-2z = 0
			\end{array}
			\right.
			\Rightarrow
			\mathcal{L}\left\{ (1,1,0) \right\}.
			\]
			
			Por tanto, el subespacio \( U \) es generado por el vector \( (1,1,0) \).
			
			\medskip
			
			Para hallar su ortogonal \( U^\perp \), basta con calcular el subespacio propio asociado al valor propio \( -1 \). Este viene dado por:
			
			\[
			U^\perp = \left\{ 
			(x, y, z) \in \mathbb{R}^3 \;\middle|\; 
			(B + I)
			\begin{pmatrix}
				x \\ y \\ z
			\end{pmatrix}
			=
			\vec{0}
			\right\},
			\]
			
			\[
			B + I = 
			\begin{pmatrix}
				1 & 1 & -1 \\
				1 & 1 & -1 \\
				0 & 0 & 0
			\end{pmatrix}
			\Rightarrow x + y - z = 0.
			\]
			
			Una base del subespacio solución es \( \mathcal{L}\left\{(1,0,1), (1,-1,0)\right\} \).
			\item[(c)] Razonar que $f$ es un giro de ángulo $\pi/2$ y encontrar una base ortogonal de $g$.
			
			Calculamos la traza de la matriz \( A = M(f, \mathcal{B}_u) \):
			\[
			A =
			\begin{pmatrix}
				0 & 1 & 0 \\
				0 & 1 & -1 \\
				-1 & 1 & 0
			\end{pmatrix}
			\Rightarrow \operatorname{tr}(A) = 1
			\]
			En \( \mathbb{R}^3 \), si \( f \) es una isometría ortogonal con \( \det(f) = 1 \) y tiene traza \( 1 \), entonces \( f \) es un giro de ángulo \( \theta \) tal que la matriz asociada:
			
			\[
			G =
			\begin{pmatrix}
				1 & 0 & 0 \\
				0 & \cos\theta & -\sin\theta \\
				0 & \sin\theta & \cos\theta
			\end{pmatrix}
			\]
			
			La traza de esta matriz es:
			\[
			\operatorname{tr}(G) = 1 + 2\cos\theta
			\]
			
			\[
			\text{tr}(f) = 1 + 2\cos\theta
			\quad \Rightarrow \quad
			1 = 1 + 2\cos\theta \Rightarrow \cos\theta = 0 \Rightarrow \theta = \dfrac{\pi}{2}
			\]
			
			Vamos ahora a determinar el eje del giro, que corresponde al subespacio propio asociado a valor propio \( \lambda = 1 \). Calculamos:
			\[
			V_1 = \left\lbrace
			\begin{pmatrix}
				x \\
				y \\
				z
			\end{pmatrix}
			\in \mathbb{R}^3 \; \middle| \;
			(A - I)
			\begin{pmatrix}
				x \\
				y \\
				z
			\end{pmatrix}
			= 0
			\right\rbrace
			\]
			\[
			A - I =
			\begin{pmatrix}
				-1 & 1 & 0 \\
				0 & 0 & -1 \\
				-1 & 1 & -1
			\end{pmatrix}
			\Rightarrow
			\left\lbrace
			\begin{array}{l}
				-x + y = 0 \\
				-z = 0 \\
				-x + y - z = 0
			\end{array}
			\right. \Rightarrow
			\mathcal{L}\left( \begin{pmatrix} 1 \\ 1 \\ 0 \end{pmatrix} \right)
			\]
			El eje del giro es el subespacio propio asociado al valor propio \( \lambda = 1 \), generado por el vector:
			\[
			\begin{pmatrix}
				1 \\ 1 \\ 0
			\end{pmatrix}
			\]
			
			Buscamos ahora una base ortogonal del plano perpendicular a este eje. Para ello, consideramos los vectores \( (x, y, z) \in \mathbb{R}^3 \) ortogonales a \( (1, 1, 0) \), es decir:
			
			\[
			\begin{pmatrix}
				1 & 1 & 0
			\end{pmatrix}
			\cdot
			\begin{pmatrix}
				x \\ y \\ z
			\end{pmatrix}
			= x + y = 0
			\quad \Rightarrow \quad y = -x
			\]
			
			De esta ecuación obtenemos dos vectores linealmente independientes del plano y por tanto, una base ortogonal asociada al giro es:
			\[
			\left\lbrace
			\begin{pmatrix} 1 \\ 1 \\ 0 \end{pmatrix},\;
			\begin{pmatrix} 1 \\ -1 \\ 0 \end{pmatrix},\;
			\begin{pmatrix} 0 \\ 0 \\ 1 \end{pmatrix}
			\right\rbrace
			\]
			
			\item[(d)] Dar una matriz, respecto de la base usual de $(\mathbb{R}^3)$, de una de las métricas de $g$.
			
			Sea la matriz de f en la base usual:
			\[
			A = M(f, \mathcal{B}_u) = \begin{pmatrix}
				0 & 1 & 0 \\
				0 & 1 & -1 \\
				-1 & 1 & 0
			\end{pmatrix}
			\]
			
			Queremos determinar una métrica simétrica definida positiva \( G \in \mathcal{S}_3(\mathbb{R}) \) tal que \( f \) sea una isometría respecto a \( g \), es decir, que se cumpla:
			\[
			A^t G A = G
			\]
			
			\textbf{Cálculo:} Sea \( G = \begin{pmatrix} a & b & c \\ b & d & e \\ c & e & h \end{pmatrix} \), y calculemos:
			\[
			A^t = \begin{pmatrix}
				0 & 0 & -1 \\
				1 & 1 & 1 \\
				0 & -1 & 0
			\end{pmatrix}
			\]
			
			Entonces:
			\[
			A^t G = \begin{pmatrix}
				0 & 0 & -1 \\
				1 & 1 & 1 \\
				0 & -1 & 0
			\end{pmatrix}
			\begin{pmatrix}
				a & b & c \\
				b & d & e \\
				c & e & h
			\end{pmatrix}
			= \begin{pmatrix}
				c & e & h \\
				a + b + c & b + d + e & c + e + h \\
				-b & -d & -e
			\end{pmatrix}
			\]
			
			Y finalmente:
			\[
			A^t G A = \begin{pmatrix}
				c & e & h \\
				a + b + c & b + d + e & c + e + h \\
				-b & -d & -e
			\end{pmatrix}
			\begin{pmatrix}
				0 & 1 & 0 \\
				0 & 1 & -1 \\
				-1 & 1 & 0
			\end{pmatrix}
			 =
			 \]
			 
			 \[
			\begin{pmatrix}
				h & -c - e - h & e \\
				-c - e - h & a + 2b + 2c + d + 2e + h & -b - d - e \\
				e & -b - d - e & d
			\end{pmatrix}
			\]
			
			Igualamos con \( G \), y se obtiene un sistema de ecuaciones:
			\[
			\begin{cases}
				h = a \\
				-c - e - h = b \\
				e = c \\
				a + 2b + 2c + d + 2e + h = d \\
				-b - d - e = e 
			\end{cases}
			\]
			
			Resolviendo el sistema, una solución posible es:
			\[
			G = \begin{pmatrix} 2 & 0 & -1 \\ 0 & 2 & -1 \\ -1 & -1 & 2 \end{pmatrix}
			\]
			
			\text{Definida positiva por el criterio de Sylvester:}
			\[
			|2| > 0, \quad \left|\begin{matrix}2 & 0 \\ 0 & 2\end{matrix}\right| = 4 > 0,
			\quad \left|\begin{matrix}2 & 0 & -1 \\ 0 & 2 & -1 \\ -1 & -1 & 2\end{matrix}\right| = 4 > 0
			\Rightarrow G \text{ es definida positiva.}
			\]
			
			\text{Por tanto, } \( f \text{ es una isometría respecto a la métrica } g \text{ de matriz } G.\)
			
		\end{enumerate}
	\end{ejercicio}
	
		\begin{ejercicio}[4 puntos]
		Se considera la siguiente matriz con coeficientes reales:
		\[
		A = \begin{pmatrix}
			1 & 0 & 0 \\
			0 & a & 1 \\
			0 & 1 & a
		\end{pmatrix}
		\]
		
		Sea $g_a$ la métrica de $\mathbb{R}^3$ cuya matriz en la base usual es $A$. Sea $f$ el endomorfismo de $\mathbb{R}^3$ cuya matriz en la base usual es también $A$.
		
		\begin{enumerate}
			\item[(a)] Calcular la signatura y clasificar la métrica $g_a$ según los valores de $a$.
			
			Calculamos el determinante de la métrica \( g_a \):
			
			\[
			\det(g_a) = \left|
			\begin{array}{ccc}
				1 & 0 & 0 \\
				0 & a & 1 \\
				0 & 1 & a \\
			\end{array}
			\right|
			= 1 \cdot \left|
			\begin{array}{cc}
				a & 1 \\
				1 & a \\
			\end{array}
			\right|
			= a^2 - 1
			\]
			
			Por tanto, \( g_a \) es no degenerada si \( a \neq \pm 1 \). Estudiamos la signatura en función de los valores de \( a \):
			
			\begin{itemize}
				\item \textbf{Caso \( a > 1 \) (por ejemplo \( a = 2 \)):}
				
				\[
				g = \begin{pmatrix}
					1 & 0 & 0 \\
					0 & 2 & 1 \\
					0 & 1 & 2 \\
				\end{pmatrix}
				\]
				Menores principales:
				\[
				\Delta_1 = 1 > 0, \quad
				\Delta_2 = \begin{vmatrix} 1 & 0 \\ 0 & 2 \end{vmatrix} = 2 > 0, \quad
				\Delta_3 = \det(g) = a^2 - 1 = 4 - 1 = 3 > 0
				\]
				Por el criterio de Sylvester, la métrica es definida positiva. Signatura: \( (3, 0) \)
				
				\[
				\begin{pmatrix}
					+ &   &   \\
					& + &   \\
					&   & +
				\end{pmatrix}
				\]
				
				\item \textbf{Caso \( -1 < a < 1 \) (por ejemplo \( a = 1/2 \)):}
				
				\[
				g = \begin{pmatrix}
					1 & 0 & 0 \\
					0 & 1/2 & 1 \\
					0 & 1 & 1/2 \\
				\end{pmatrix}
				\]
				\[
				\Delta_1 = 1 > 0, \quad \Delta_2 = \begin{vmatrix} 1 & 0 \\ 0 & 1/2 \end{vmatrix} = 1/2 > 0, \quad
				\Delta_3 = -3/4 < 0
				\]
				Como el determinante es negativo, la signatura es \( (2,1) \): métrica indefinida.
				
				\[
				\begin{pmatrix}
					+ &   &   \\
					& + &   \\
					&   & -
				\end{pmatrix}
				\]
				
				\item \textbf{Caso \( a < -1 \) (por ejemplo \( a = -2 \)):}
				
				\[
				g = \begin{pmatrix}
					1 & 0 & 0 \\
					0 & -2 & 1 \\
					0 & 1 & -2 \\
				\end{pmatrix}
				\]
				\[
				\det(g) = a^2 - 1 = 4 - 1 = 3 > 0
				\]
				Pero el segundo menor:
				\[
				\Delta_2 = \begin{vmatrix} 1 & 0 \\ 0 & -2 \end{vmatrix} = -2 < 0
				\]
				Por tanto, la signatura es también \( (2,1) \): métrica indefinida.
				
				\[
				\begin{pmatrix}
					+ &   &   \\
					& + &   \\
					&   & -
				\end{pmatrix}
				\]
				
				\item \textbf{Caso \( a = 1 \):} La métrica es:
				
				\[
				G_1 = 
				\begin{pmatrix}
					1 & 0 & 0 \\
					0 & 1 & 1 \\
					0 & 1 & 1
				\end{pmatrix}
				\]
				
				Como el determinante es cero, la métrica es \textbf{degenerada}.
				
				Buscamos ahora un vector \( \vec{e}_1 \in \mathbb{R}^3 \) tal que \( g(\vec{e}_1, \vec{e}_1) \neq 0 \;\; \forall \vec{x} \in \mathbb{R}^3 \). Por simplicidad, probamos con \( \vec{e}_1 = (1,0,0) \).
				
				Buscamos ahora su perpendicular tal que \( g(\vec{e}_1, \vec{e}_2) = 0 \;\; \forall \vec{x} \in \mathbb{R}^3 \) y que \( g(\vec{e}_2, \vec{e}_2) \neq 0 \).
				
				\[
				g_1(\vec{e}_1, \vec{e}_2) = 
				\begin{pmatrix} 1 & 0 & 0 \end{pmatrix}
				\begin{pmatrix}
					1 & 0 & 0 \\
					0 & 1 & 1 \\
					0 & 1 & 1
				\end{pmatrix}
				\begin{pmatrix} x_1 \\ x_2 \\ x_3 \end{pmatrix}
				= 
				\begin{pmatrix} 1 & 0 & 0 \end{pmatrix}
				\begin{pmatrix} x_1 \\ x_2 \\ x_3 \end{pmatrix}
				= x_1 = 0
				\]
				
				Entonces, cualquier vector ortogonal a \( \vec{e}_1 = (1,0,0) \) en la métrica \( g_1 \) debe cumplir \( x_1 = 0 \), por lo que:
				
				\[
				\vec{e}_2 \in \left\lbrace 
				\begin{pmatrix} 0 \\ 1 \\ 0 \end{pmatrix},\;
				\begin{pmatrix} 0 \\ 0 \\ 1 \end{pmatrix} 
				\right\rbrace
				\]
				
				Elijamos \( \vec{e}_2 = \begin{pmatrix} 0 & 1 & 0 \end{pmatrix}\) y que \(g_{1}(\vec{e}_2, \vec{e}_2) = 1\)
				
				Por lo que obtenemos la matriz de Sylvester:
				
				\[
				S_1 = \begin{pmatrix}
					1 &   &   \\
					& 1 &   \\
					&   & 0
				\end{pmatrix}
				\]
				\item \textbf{Caso \( a = -1 \):}
				\[
				G_{-1} = 
				\begin{pmatrix}
					1 & 0 & 0 \\
					0 & -1 & 1 \\
					0 & 1 & -1
				\end{pmatrix}
				\]
				
				La métrica también es \textbf{degenerada}.
				
				Probamos \( \vec{e}_1 = (1,0,0) \), tal que
				\[g_{-1}(\vec{e}_1, \vec{e}_1) \neq 0\]
				
				Buscamos ahora su perpendicular tal que \( g_{-1}(\vec{e}_1, \vec{e}_2) = 0 \;\; \forall \vec{x} \in \mathbb{R}^3 \) y que \( g_{-1}(\vec{e}_2, \vec{e}_2) \neq 0 \).
				\[
				g_{-1}(\vec{e}_1, \vec{e}_2) =
				\begin{pmatrix} 1 & 0 & 0 \end{pmatrix}
				\begin{pmatrix}
					1 & 0 & 0 \\
					0 & -1 & 1 \\
					0 & 1 & -1
				\end{pmatrix}
				\begin{pmatrix} x_1 \\ x_2 \\ x_3 \end{pmatrix}
				= \begin{pmatrix} 1 & 0 & 0 \end{pmatrix}
				\begin{pmatrix} x_1 \\ x_2 \\ x_3 \end{pmatrix}
				= x_1 = 0
				\]
				
				Entonces obtenemos:
				\[
				\left\lbrace \begin{pmatrix} 0 \\ 1 \\ 0 \end{pmatrix}, \begin{pmatrix} 0 \\ 0 \\ 1 \end{pmatrix} \right\rbrace
				\quad 
				\]
				
				Elijamos \( \vec{e}_2 = \begin{pmatrix} 0 & 1 & 0 \end{pmatrix}\) y que \(g_{-1}(\vec{e}_2, \vec{e}_2) = -1\)
				Por lo que obtenemos la matriz de Sylvester:
				
				\[
				S_1 = \begin{pmatrix}
					1 &   &   \\
					& -1 &   \\
					&   & 0
				\end{pmatrix}
				\]
				
			\end{itemize}
			
			\item[(b)] Para $a = 1$ obtener una base conjugada (es decir, ortogonal) para la métrica $g_1$.
			
			Ya en el apartado anterior obtuvimos una base ortogonal para el plano no degenerado de \( g_1 \), formada por los vectores:
			
			\[
			\vec{e}_1 = \begin{pmatrix} 1 \\ 0 \\ 0 \end{pmatrix}, \quad
			\vec{e}_2 = \begin{pmatrix} 0 \\ 1 \\ 0 \end{pmatrix}
			\]
			
			Como la métrica es degenerada, el tercer vector ortogonal debe tomarse en el núcleo de \( g_1 \). Calculamos el núcleo resolviendo \( G \vec{x} = 0 \), con:
			
			\[
			G = 
			\begin{pmatrix}
				1 & 0 & 0 \\
				0 & 1 & 1 \\
				0 & 1 & 1
			\end{pmatrix}
			\Rightarrow
			\left\{
			\begin{array}{l}
				x_1 = 0 \\
				x_2 + x_3 = 0
			\end{array}
			\right.
			\quad \Rightarrow \quad
			\ker(g_1) = \mathcal{L} \left\lbrace \begin{pmatrix} 0 \\ 1 \\ -1 \end{pmatrix} \right\rbrace
			\]
			
			Por tanto, una base conjugada ortogonal para \( g_1 \) es:
			
			\[
			\left\lbrace
			\begin{pmatrix} 1 \\ 0 \\ 0 \end{pmatrix},\;
			\begin{pmatrix} 0 \\ 1 \\ 0 \end{pmatrix},\;
			\begin{pmatrix} 0 \\ 1 \\ -1 \end{pmatrix}
			\right\rbrace
			\]
			
			\item[(c)] ¿Para qué valores de $a$ es $g_a$ una métrica euclídea y $f$ autoadjunto en $(\mathbb{R}^3, g_a)$?
			
			En el apartado (a) calculamos que la métrica \( g_a \) es definida positiva (y por tanto, euclídea) si y solo si \( a > 1 \). Así que, en lo que respecta a la euclideidad, esto solo ocurre para valores de \( a > 1 \).
			
			Ahora, veamos cuándo \( f \) es autoadjunto respecto de \( g_a \). Para ello, recordamos que se cumple:
			
			\[
			M(f, \mathcal{B})^t \cdot G = G \cdot M(f, \mathcal{B})
			\]
			
			Donde la matriz resultante debe ser \textbf{simétrica} y que la matriz \( G \) de la métrica \( g_a \) en la base usual es:
			
			\[
			G = A = 
			\begin{pmatrix}
				1 & 0 & 0 \\
				0 & a & 1 \\
				0 & 1 & a
			\end{pmatrix} =\quad M(f, \mathcal{B})
			\]
			
			Entonces:
			
			\[
			A^t G = 
			\begin{pmatrix}
				1 & 0 & 0 \\
				0 & a & 1 \\
				0 & 1 & a
			\end{pmatrix}
			\begin{pmatrix}
				1 & 0 & 0 \\
				0 & a & 1 \\
				0 & 1 & a
			\end{pmatrix}
			=
			\begin{pmatrix}
				1 & 0 & 0 \\
				0 & a^2 + 1 & 2a \\
				0 & 2a & a^2 + 1
			\end{pmatrix}
			\]
			
			y por otro lado:
			
			\[
			G A =
			\begin{pmatrix}
				1 & 0 & 0 \\
				0 & a & 1 \\
				0 & 1 & a
			\end{pmatrix}
			\begin{pmatrix}
				1 & 0 & 0 \\
				0 & a & 1 \\
				0 & 1 & a
			\end{pmatrix}
			=
			\begin{pmatrix}
				1 & 0 & 0 \\
				0 & a^2 + 1 & 2a \\
				0 & 2a & a^2 + 1
			\end{pmatrix}
			\]
			
			Por tanto, se cumple que \( A^t G = G A \) y que es simétrica para todo valor de \( a \), lo que implica que \( f \) es autoadjunto respecto de \( g_a \) para cualquier \( a \in \mathbb{R} \).
			
			\bigskip
			
			\noindent
			\textbf{Conclusión:} \( f \) es autoadjunto para todo \( a \in \mathbb{R} \), pero \( g_a \) es euclídea solo si \( a > 1 \).
			
			\item[(d)] Para $a = 2$ calcular, si es posible, una base ortogonal de $(\mathbb{R}^3, g_2)$ formada por vectores propios de $f$.
			
			 Calculamos el polinomio característico de \( f \) y sus subespacios propios.
			\[
			p_f(\lambda) = 
			\left| \begin{array}{ccc}
				1 - \lambda & 0 & 0 \\
				0 & 2 - \lambda & 1 \\
				0 & 1 & 2 - \lambda \\
			\end{array} \right|
			= (1 - \lambda)\left| \begin{array}{cc}
				2 - \lambda & 1 \\
				1 & 2 - \lambda \\
			\end{array} \right|
			= \]
			\[(1 - \lambda)\left( (2 - \lambda)^2 - 1 \right)
			= (1 - \lambda)\left( \lambda^2 - 4\lambda + 3 \right)
			= -(\lambda - 1)^2(\lambda - 3)
			\]
			
			\vspace{0.3cm}
			
			Para \( \lambda = 1 \), resolvemos \( (A - I)\vec{x} = 0 \):
			
			\[
			(A - I) = 
			\begin{pmatrix}
				0 & 0 & 0 \\
				0 & 1 & 1 \\
				0 & 1 & 1 \\
			\end{pmatrix}
			\Rightarrow x_2 + x_3 = 0
			\Rightarrow V_1 = \mathcal{L}\left( \begin{pmatrix} 1 \\ 0 \\ 0 \end{pmatrix}, \begin{pmatrix} 0 \\ 1 \\ -1 \end{pmatrix} \right)
			\]
			
			Para \( \lambda = 3 \), resolvemos \( (A - 3I)\vec{x} = 0 \):
			
			\[
			(A - 3I) = 
			\begin{pmatrix}
				-2 & 0 & 0 \\
				0 & -1 & 1 \\
				0 & 1 & -1 \\
			\end{pmatrix}
			\Rightarrow 
			\left\{
			\begin{array}{l}
				x = 0 \\
				-y + z = 0
			\end{array}
			\right.
			\Rightarrow V_3 = \mathcal{L}\left( \begin{pmatrix} 0 \\ 1 \\ 1 \end{pmatrix} \right)
			\]
			
			\vspace{0.3cm}
			
			Estos vectores forman una base ortogonal ya que \( V_1 \perp V_3 \), y los dos vectores que generan \( V_1 \) son perpendiculares entre sí:
			
			\[
			g_2\left( \begin{pmatrix} 1 \\ 0 \\ 0 \end{pmatrix}, \begin{pmatrix} 0 \\ 1 \\ -1 \end{pmatrix} \right) = 0
			\]
			
			Por tanto, la base pedida es:
			\[
				\mathcal{B} = \left\lbrace 
				\begin{pmatrix} 1 \\ 0 \\ 0 \end{pmatrix}, 
				\begin{pmatrix} 0 \\ 1 \\ -1 \end{pmatrix}, 
				\begin{pmatrix} 0 \\ 1 \\ 1 \end{pmatrix}
				\right\rbrace
			\]
		\end{enumerate}
		\end{ejercicio}
\end{document}
