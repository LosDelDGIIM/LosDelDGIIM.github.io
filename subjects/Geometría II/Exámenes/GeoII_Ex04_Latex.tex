\documentclass[12pt]{article}

% Idioma y codificación
\usepackage[spanish, es-tabla]{babel}       %es-tabla para que se titule "Tabla"
\usepackage[utf8]{inputenc}

% Márgenes
\usepackage[a4paper,top=3cm,bottom=2.5cm,left=3cm,right=3cm]{geometry}

% Comentarios de bloque
\usepackage{verbatim}

% Paquetes de links
\usepackage[hidelinks]{hyperref}    % Permite enlaces
\usepackage{url}                    % redirecciona a la web

% Más opciones para enumeraciones
\usepackage{enumitem}

% Personalizar la portada
\usepackage{titling}

% Paquetes de tablas
\usepackage{multirow}


%------------------------------------------------------------------------

%Paquetes de figuras
\usepackage{caption}
\usepackage{subcaption} % Figuras al lado de otras
\usepackage{float}      % Poner figuras en el sitio indicado H.


% Paquetes de imágenes
\usepackage{graphicx}       % Paquete para añadir imágenes
\usepackage{transparent}    % Para manejar la opacidad de las figuras

% Paquete para usar colores
\usepackage[dvipsnames]{xcolor}
\usepackage{pagecolor}      % Para cambiar el color de la página

% Habilita tamaños de fuente mayores
\usepackage{fix-cm}

% Para los gráficos
\usepackage{tikz}

% Para poder situar los nodos en los grafos
\usetikzlibrary{positioning}


%------------------------------------------------------------------------

% Paquetes de matemáticas
\usepackage{mathtools, amsfonts, amssymb, mathrsfs}
\usepackage[makeroom]{cancel}     % Simplificar tachando
\usepackage{polynom}    % Divisiones y Ruffini
\usepackage{units} % Para poner fracciones diagonales con \nicefrac

\usepackage{pgfplots}   %Representar funciones
\pgfplotsset{compat=1.18}  % Versión 1.18

\usepackage{tikz-cd}    % Para usar diagramas de composiciones
\usetikzlibrary{calc}   % Para usar cálculo de coordenadas en tikz

%Definición de teoremas, etc.
\usepackage{amsthm}
%\swapnumbers   % Intercambia la posición del texto y de la numeración

\theoremstyle{plain}

\makeatletter
\@ifclassloaded{article}{
  \newtheorem{teo}{Teorema}[section]
}{
  \newtheorem{teo}{Teorema}[chapter]  % Se resetea en cada chapter
}
\makeatother

\newtheorem{coro}{Corolario}[teo]           % Se resetea en cada teorema
\newtheorem{prop}[teo]{Proposición}         % Usa el mismo contador que teorema
\newtheorem{lema}[teo]{Lema}                % Usa el mismo contador que teorema

\theoremstyle{remark}
\newtheorem*{observacion}{Observación}

\theoremstyle{definition}

\makeatletter
\@ifclassloaded{article}{
  \newtheorem{definicion}{Definición} [section]     % Se resetea en cada chapter
}{
  \newtheorem{definicion}{Definición} [chapter]     % Se resetea en cada chapter
}
\makeatother

\newtheorem*{notacion}{Notación}
\newtheorem*{ejemplo}{Ejemplo}
\newtheorem*{ejercicio*}{Ejercicio}             % No numerado
\newtheorem{ejercicio}{Ejercicio} [section]     % Se resetea en cada section


% Modificar el formato de la numeración del teorema "ejercicio"
\renewcommand{\theejercicio}{%
  \ifnum\value{section}=0 % Si no se ha iniciado ninguna sección
    \arabic{ejercicio}% Solo mostrar el número de ejercicio
  \else
    \thesection.\arabic{ejercicio}% Mostrar número de sección y número de ejercicio
  \fi
}


% \renewcommand\qedsymbol{$\blacksquare$}         % Cambiar símbolo QED
%------------------------------------------------------------------------

% Paquetes para encabezados
\usepackage{fancyhdr}
\pagestyle{fancy}
\fancyhf{}

\newcommand{\helv}{ % Modificación tamaño de letra
\fontfamily{}\fontsize{12}{12}\selectfont}
\setlength{\headheight}{15pt} % Amplía el tamaño del índice


%\usepackage{lastpage}   % Referenciar última pag   \pageref{LastPage}
\fancyfoot[C]{\thepage}

%------------------------------------------------------------------------

% Conseguir que no ponga "Capítulo 1". Sino solo "1."
\makeatletter
\@ifclassloaded{book}{
  \renewcommand{\chaptermark}[1]{\markboth{\thechapter.\ #1}{}} % En el encabezado
    
  \renewcommand{\@makechapterhead}[1]{%
  \vspace*{50\p@}%
  {\parindent \z@ \raggedright \normalfont
    \ifnum \c@secnumdepth >\m@ne
      \huge\bfseries \thechapter.\hspace{1em}\ignorespaces
    \fi
    \interlinepenalty\@M
    \Huge \bfseries #1\par\nobreak
    \vskip 40\p@
  }}
}
\makeatother

%------------------------------------------------------------------------
% Paquetes de cógido
\usepackage{minted}
\renewcommand\listingscaption{Código fuente}

\usepackage{fancyvrb}
% Personaliza el tamaño de los números de línea
\renewcommand{\theFancyVerbLine}{\small\arabic{FancyVerbLine}}

% Estilo para C++
\newminted{cpp}{
    frame=lines,
    framesep=2mm,
    baselinestretch=1.2,
    linenos,
    escapeinside=||
}

% para minted
\definecolor{LightGray}{rgb}{0.95,0.95,0.92}
\setminted{
    linenos=true,
    stepnumber=5,
    numberfirstline=true,
    autogobble,
    breaklines=true,
    breakautoindent=true,
    breaksymbolleft=,
    breaksymbolright=,
    breaksymbolindentleft=0pt,
    breaksymbolindentright=0pt,
    breaksymbolsepleft=0pt,
    breaksymbolsepright=0pt,
    fontsize=\footnotesize,
    bgcolor=LightGray,
    numbersep=10pt
}


\usepackage{listings} % Para incluir código desde un archivo

\renewcommand\lstlistingname{Código Fuente}
\renewcommand\lstlistlistingname{Índice de Códigos Fuente}

% Definir colores
\definecolor{vscodepurple}{rgb}{0.5,0,0.5}
\definecolor{vscodeblue}{rgb}{0,0,0.8}
\definecolor{vscodegreen}{rgb}{0,0.5,0}
\definecolor{vscodegray}{rgb}{0.5,0.5,0.5}
\definecolor{vscodebackground}{rgb}{0.97,0.97,0.97}
\definecolor{vscodelightgray}{rgb}{0.9,0.9,0.9}

% Configuración para el estilo de C similar a VSCode
\lstdefinestyle{vscode_C}{
  backgroundcolor=\color{vscodebackground},
  commentstyle=\color{vscodegreen},
  keywordstyle=\color{vscodeblue},
  numberstyle=\tiny\color{vscodegray},
  stringstyle=\color{vscodepurple},
  basicstyle=\scriptsize\ttfamily,
  breakatwhitespace=false,
  breaklines=true,
  captionpos=b,
  keepspaces=true,
  numbers=left,
  numbersep=5pt,
  showspaces=false,
  showstringspaces=false,
  showtabs=false,
  tabsize=2,
  frame=tb,
  framerule=0pt,
  aboveskip=10pt,
  belowskip=10pt,
  xleftmargin=10pt,
  xrightmargin=10pt,
  framexleftmargin=10pt,
  framexrightmargin=10pt,
  framesep=0pt,
  rulecolor=\color{vscodelightgray},
  backgroundcolor=\color{vscodebackground},
}

%------------------------------------------------------------------------

% Comandos definidos
\newcommand{\bb}[1]{\mathbb{#1}}
\newcommand{\cc}[1]{\mathcal{#1}}

% I prefer the slanted \leq
\let\oldleq\leq % save them in case they're every wanted
\let\oldgeq\geq
\renewcommand{\leq}{\leqslant}
\renewcommand{\geq}{\geqslant}

% Si y solo si
\newcommand{\sii}{\iff}

% Letras griegas
\newcommand{\eps}{\epsilon}
\newcommand{\veps}{\varepsilon}
\newcommand{\lm}{\lambda}

\newcommand{\ol}{\overline}
\newcommand{\ul}{\underline}
\newcommand{\wt}{\widetilde}
\newcommand{\wh}{\widehat}

\let\oldvec\vec
\renewcommand{\vec}{\overrightarrow}

% Derivadas parciales
\newcommand{\del}[2]{\frac{\partial #1}{\partial #2}}
\newcommand{\Del}[3]{\frac{\partial^{#1} #2}{\partial #3^{#1}}}
\newcommand{\deld}[2]{\dfrac{\partial #1}{\partial #2}}
\newcommand{\Deld}[3]{\dfrac{\partial^{#1} #2}{\partial #3^{#1}}}


\newcommand{\AstIg}{\stackrel{(\ast)}{=}}
\newcommand{\Hop}{\stackrel{L'H\hat{o}pital}{=}}

\newcommand{\red}[1]{{\color{red}#1}} % Para integrales, destacar los cambios.

% Método de integración
\newcommand{\MetInt}[2]{
    \left[\begin{array}{c}
        #1 \\ #2
    \end{array}\right]
}

% Declarar aplicaciones
% 1. Nombre aplicación
% 2. Dominio
% 3. Codominio
% 4. Variable
% 5. Imagen de la variable
\newcommand{\Func}[5]{
    \begin{equation*}
        \begin{array}{rrll}
            #1:& #2 & \longrightarrow & #3\\
               & #4 & \longmapsto & #5
        \end{array}
    \end{equation*}
}

%------------------------------------------------------------------------



\begin{document}

    % 1. Foto de fondo
    % 2. Título
    % 3. Encabezado Izquierdo
    % 4. Color de fondo
    % 5. Coord x del titulo
    % 6. Coord y del titulo
    % 7. Fecha

    
    % 1. Foto de fondo
% 2. Título
% 3. Encabezado Izquierdo
% 4. Color de fondo
% 5. Coord x del titulo
% 6. Coord y del titulo
% 7. Fecha

\newcommand{\portada}[7]{

    \portadaBase{#1}{#2}{#3}{#4}{#5}{#6}{#7}
    \portadaBook{#1}{#2}{#3}{#4}{#5}{#6}{#7}
}

\newcommand{\portadaExamen}[7]{

    \portadaBase{#1}{#2}{#3}{#4}{#5}{#6}{#7}
    \portadaArticle{#1}{#2}{#3}{#4}{#5}{#6}{#7}
}




\newcommand{\portadaBase}[7]{

    % Tiene la portada principal y la licencia Creative Commons
    
    % 1. Foto de fondo
    % 2. Título
    % 3. Encabezado Izquierdo
    % 4. Color de fondo
    % 5. Coord x del titulo
    % 6. Coord y del titulo
    % 7. Fecha
    
    
    \thispagestyle{empty}               % Sin encabezado ni pie de página
    \newgeometry{margin=0cm}        % Márgenes nulos para la primera página
    
    
    % Encabezado
    \fancyhead[L]{\helv #3}
    \fancyhead[R]{\helv \nouppercase{\leftmark}}
    
    
    \pagecolor{#4}        % Color de fondo para la portada
    
    \begin{figure}[p]
        \centering
        \transparent{0.3}           % Opacidad del 30% para la imagen
        
        \includegraphics[width=\paperwidth, keepaspectratio]{assets/#1}
    
        \begin{tikzpicture}[remember picture, overlay]
            \node[anchor=north west, text=white, opacity=1, font=\fontsize{60}{90}\selectfont\bfseries\sffamily, align=left] at (#5, #6) {#2};
            
            \node[anchor=south east, text=white, opacity=1, font=\fontsize{12}{18}\selectfont\sffamily, align=right] at (9.7, 3) {\textbf{\href{https://losdeldgiim.github.io/}{Los Del DGIIM}}};
            
            \node[anchor=south east, text=white, opacity=1, font=\fontsize{12}{15}\selectfont\sffamily, align=right] at (9.7, 1.8) {Doble Grado en Ingeniería Informática y Matemáticas\\Universidad de Granada};
        \end{tikzpicture}
    \end{figure}
    
    
    \restoregeometry        % Restaurar márgenes normales para las páginas subsiguientes
    \pagecolor{white}       % Restaurar el color de página
    
    
    \newpage
    \thispagestyle{empty}               % Sin encabezado ni pie de página
    \begin{tikzpicture}[remember picture, overlay]
        \node[anchor=south west, inner sep=3cm] at (current page.south west) {
            \begin{minipage}{0.5\paperwidth}
                \href{https://creativecommons.org/licenses/by-nc-nd/4.0/}{
                    \includegraphics[height=2cm]{assets/Licencia.png}
                }\vspace{1cm}\\
                Esta obra está bajo una
                \href{https://creativecommons.org/licenses/by-nc-nd/4.0/}{
                    Licencia Creative Commons Atribución-NoComercial-SinDerivadas 4.0 Internacional (CC BY-NC-ND 4.0).
                }\\
    
                Eres libre de compartir y redistribuir el contenido de esta obra en cualquier medio o formato, siempre y cuando des el crédito adecuado a los autores originales y no persigas fines comerciales. 
            \end{minipage}
        };
    \end{tikzpicture}
    
    
    
    % 1. Foto de fondo
    % 2. Título
    % 3. Encabezado Izquierdo
    % 4. Color de fondo
    % 5. Coord x del titulo
    % 6. Coord y del titulo
    % 7. Fecha


}


\newcommand{\portadaBook}[7]{

    % 1. Foto de fondo
    % 2. Título
    % 3. Encabezado Izquierdo
    % 4. Color de fondo
    % 5. Coord x del titulo
    % 6. Coord y del titulo
    % 7. Fecha

    % Personaliza el formato del título
    \pretitle{\begin{center}\bfseries\fontsize{42}{56}\selectfont}
    \posttitle{\par\end{center}\vspace{2em}}
    
    % Personaliza el formato del autor
    \preauthor{\begin{center}\Large}
    \postauthor{\par\end{center}\vfill}
    
    % Personaliza el formato de la fecha
    \predate{\begin{center}\huge}
    \postdate{\par\end{center}\vspace{2em}}
    
    \title{#2}
    \author{\href{https://losdeldgiim.github.io/}{Los Del DGIIM}}
    \date{Granada, #7}
    \maketitle
    
    \tableofcontents
}




\newcommand{\portadaArticle}[7]{

    % 1. Foto de fondo
    % 2. Título
    % 3. Encabezado Izquierdo
    % 4. Color de fondo
    % 5. Coord x del titulo
    % 6. Coord y del titulo
    % 7. Fecha

    % Personaliza el formato del título
    \pretitle{\begin{center}\bfseries\fontsize{42}{56}\selectfont}
    \posttitle{\par\end{center}\vspace{2em}}
    
    % Personaliza el formato del autor
    \preauthor{\begin{center}\Large}
    \postauthor{\par\end{center}\vspace{3em}}
    
    % Personaliza el formato de la fecha
    \predate{\begin{center}\huge}
    \postdate{\par\end{center}\vspace{5em}}
    
    \title{#2}
    \author{\href{https://losdeldgiim.github.io/}{Los Del DGIIM}}
    \date{Granada, #7}
    \thispagestyle{empty}               % Sin encabezado ni pie de página
    \maketitle
    \vfill
}
    \portadaExamen{ffccA4.jpg}{Geometría II\\Examen IV}{Geometría II. Examen IV}{MidnightBlue}{-8}{28}{2023}

    \begin{description}
        \item[Asignatura] Geometría II.
        \item[Curso Académico] 2022-23.
        \item[Grado] Matemáticas.
        \item[Grupo] A.
        \item[Profesor] Francisco Milán López.
        \item[Descripción] Parcial del Tema 2. Formas Bilineales Simétricas.
        \item[Fecha] 16 de mayo de 2023.
        %\item[Duración] 60 minutos.
    
    \end{description}
    \newpage
    
    \begin{ejercicio}\textbf{[7 puntos]}
    En $\bb{R}^3$ consideramos para cada $a\in \bb{R}$ la métrica $g_a$, cuya matriz en la base usual viene dada por:
    \begin{equation*}
        A_a = M(g_a, \cc{B}_u) = \left(\begin{array}{ccc}
            1 & -a & 1 \\
            -a & a^2-1 & -2a \\
            1 & -2a & 1-a^2
        \end{array}\right)
    \end{equation*}

    Calcular una base de Sylvester para $g_a$ y su signatura.

    En primer lugar, clasifico la métrica.
    \begin{equation*}
        |A_a| = -(a^2-1)^2+2a^2+2a^2-(a^2-1)-4a^2-a^2(1-a^2) = -a^4-1+2a^2 -a^2+1-a^2+a^4 = 0
    \end{equation*}
    \begin{equation*}
        \left|\begin{array}{ccc}
            1 & -a \\
            -a & a^2-1 \\
        \end{array}\right| = a^2-1-a^2=-1
    \end{equation*}
    Por tanto, tengo que $Nul(g_a)=1\;\forall a$. Además, $g(e_1, e_1)=1$. Por tanto, tenemos que:
    \begin{equation*}
        A_a\sim \left(\begin{array}{ccc}
             1 \\
              & 1 \\
              && 0
        \end{array}\right)
        \qquad \text{o} \qquad
        A_a\sim \left(\begin{array}{ccc}
             1 \\
              & -1 \\
              && 0
        \end{array}\right)
    \end{equation*}

    \begin{itemize}
        \item \underline{Para $a\neq \pm1$}:
        \begin{equation*}
            g(e_2, e_2) = a^2-1 = -(1-a^2) = -g(e_3, e_3)
        \end{equation*}
        Como $a\neq 1\Longrightarrow a^2-1\neq 0 \Longrightarrow g(e_2, e_2)=-g(e_3, e_3)\neq 0$. Por tanto, al menos uno de los dos cuadrados es negativo, por lo que:
        \begin{equation*}
            A_a\sim \left(\begin{array}{ccc}
             1 \\
              & -1 \\
              && 0
            \end{array}\right)
        \end{equation*}

        \item \underline{Para $a=1$}:
        \begin{equation*}
            A_1 = M(g_1, \cc{B}_u) = \left(\begin{array}{ccc}
                1 & -1 & 1 \\
                -1 & 0 & -2 \\
                1 & -2 & 0
            \end{array}\right)
        \end{equation*}
        \begin{equation*}
            g_1(e_1+e_2, e_1+e_2) = 1 +0 +2g(e_1, e_2) = 1+2(-1) = -1 <0
        \end{equation*}
        Por tanto, tenemos que $g$ no es semidefinida positiva. Por tanto, estamos ante el segundo caso.
        \begin{equation*}
            A_a\sim \left(\begin{array}{ccc}
             1 \\
              & -1 \\
              && 0
            \end{array}\right)
        \end{equation*}

        \item \underline{Para $a=-1$}:
        \begin{equation*}
            A_1 = M(g_1, \cc{B}_u) = \left(\begin{array}{ccc}
                1 & 1 & 1 \\
                1 & 0 & 2 \\
                1 & 2 & 0
            \end{array}\right)
        \end{equation*}
        \begin{equation*}
            g_1(e_1-e_2, e_1-e_2) = 1 +0 -2g(e_1, e_2) = 1-2(1) = -1 <0
        \end{equation*}
        Por tanto, tenemos que $g$ no es semidefinida negativa. Por tanto, estamos ante el segundo caso.
        \begin{equation*}
            A_a\sim \left(\begin{array}{ccc}
             1 \\
              & -1 \\
              && 0
            \end{array}\right)
        \end{equation*}
    \end{itemize}
    Por tanto, en los tres casos tenemos que:
    \begin{equation*}
        \text{Signatura}=(1,1)\qquad Nul(g_a)=1 \qquad Ind(g_a)=1
    \end{equation*}

    Para hallar la base de Sylvester, calculo en primer lugar $Ker(g_a)$:
    \begin{equation*}\begin{split}
        Ker(g_a) &= \{v \in \bb{R}^3 \mid g_a(v,u) = 0 \quad \forall u\in V\} 
        = \left\{ \left(\begin{array}{c}
             x_1 \\ x_2 \\ x_3
        \end{array} \right) \in \bb{R}^3 \mid A_a
        \left(\begin{array}{c}
             x_1 \\ x_2 \\ x_3
        \end{array} \right) = 0\right\} \\
        &= \left\{ \left(\begin{array}{c}
             x_1 \\ x_2 \\ x_3
        \end{array} \right) \in \bb{R}^3 \mid
        \left(\begin{array}{ccc}
            1 & -a & 1 \\
            -a & a^2-1 & -2a \\
            1 & -2a & 1-a^2
        \end{array} \right) 
        \left(\begin{array}{c}
             x_1 \\ x_2 \\ x_3
        \end{array} \right) = 0\right\} \\
        &= \left\{ \left(\begin{array}{c}
             x_1 \\ x_2 \\ x_3
        \end{array} \right) \in \bb{R}^3 \left|\begin{array}{c}
            x_1-ax_2+x_3 = 0\\
            -ax_1 +(a^2-1)x_2-2ax_3 = 0\\
            x_1-2ax_2+(1-a^2)x_3=0
        \end{array}
        \right.\right\}
        = \cc{L} \left\{ \left(\begin{array}{c}
             -1-a^2 \\ -a \\ 1
        \end{array} \right)\right\}
    \end{split}\end{equation*}

    Sea $\cc{B}_s = \{\bar{e_1}, \bar{e_2}, \bar{e_3}\}$ la base de Sylvester. Sea $\bar{e_1}=e_1$, $\bar{e_3}=(-1-a^2, -a, 1)^t$. Para obtener $\bar{e_2}$, necesito que $\bar{e_2}\perp \bar{e_1}$:
    \begin{equation*}\begin{split}
        <\bar{e_1}>^\perp &= \{v \in \bb{R}^3 \mid g(\bar{e_1},v) = 0\} 
        = \left\{ \left(\begin{array}{c}
             x_1 \\ x_2 \\ x_3
        \end{array} \right) \in \bb{R}^3 \mid \bar{e_1}^t A_a
        \left(\begin{array}{c}
             x_1 \\ x_2 \\ x_3
        \end{array} \right) = 0\right\} \\
        &= \left\{ \left(\begin{array}{c}
             x_1 \\ x_2 \\ x_3
        \end{array} \right) \in \bb{R}^3 \mid (1,0,0)\left(\begin{array}{ccc}
            1 & -a & 1 \\
            -a & a^2-1 & -2a \\
            1 & -2a & 1-a^2
        \end{array} \right) 
        \left(\begin{array}{c}
             x_1 \\ x_2 \\ x_3
        \end{array} \right) = 0\right\} \\
        &= \left\{ \left(\begin{array}{c}
             x_1 \\ x_2 \\ x_3
        \end{array} \right) \in \bb{R}^3 \mid (1,-a,1)
        \left(\begin{array}{c}
             x_1 \\ x_2 \\ x_3
        \end{array} \right) = 0\right\} \\
        &= \left\{ \left(\begin{array}{c}
             x_1 \\ x_2 \\ x_3
        \end{array} \right) \in \bb{R}^3 \mid x_1-ax_2+x_3 = 0\right\}
        = \cc{L} \left\{ \left(\begin{array}{c}
             1 \\  0 \\ -1
        \end{array} \right),
        \left(\begin{array}{c}
             a \\ 1 \\ 0
        \end{array} \right)\right\}
    \end{split}\end{equation*}

    Sea $\bar{e_2}=(a,1,0)^t$. Tengo que:
    \begin{equation*}
        g(\bar{e_2}, \bar{e_2}) = a^2g(e_1, e_1) + g(e_2, e_2) +2ag(e_1, e_2) = a^2+a^2-1+2a(-a) = -1
    \end{equation*}

    Por tanto, la base de Sylvester es:
    \begin{equation*}
        \cc{B}_S = \{\bar{e_1}, \bar{e_2}, \bar{e_3}\}=\left\{ \left(\begin{array}{c}
             1 \\  0 \\ 0
        \end{array} \right),
        \left(\begin{array}{c}
             a \\ 1 \\ 0
        \end{array} \right),
        \left(\begin{array}{c}
             -1-a^2 \\ -a \\ 1
        \end{array} \right)\right\}
    \end{equation*}
    \begin{equation*}
        M(g_a, \cc{B}_S) = \left(\begin{array}{ccc}
            1 &  \\
             & -1 \\
             && 0
        \end{array}\right)
    \end{equation*}
\end{ejercicio}

\begin{ejercicio}\textbf{[3 puntos]} Sean $(V,g)$ un espacio vectorial métrico con $\dim V=2$ y $\cc{B}$ una base de $V$ para la cual
\begin{equation*}
    |M(g;\cc{B})|=-3
\end{equation*}
Razona si son verdaderas o falsas las siguientes afirmaciones:
\begin{enumerate}
    \item $\exists\; \cc{B}_1$ base de $V$ tal que $|M(g;\cc{B}_1)|=3$.

    Esto es \textbf{falso}, ya que el signo del determinante es un invariante para la misma métrica independientemente de la base.

    Como ambas matrices representan la misma base respecto de distintas métricas, tenemos que son congruentes, es decir, $\exists\; P$ regular tal que:
    \begin{equation*}
        M(g;\cc{B}_1) = P^t M(g;\cc{B}) P
    \end{equation*}

    Por tanto, tomando determinante,
    \begin{equation*}
        |M(g;\cc{B}_1)| = |P^t M(g;\cc{B}) P| = |P|^2 |M(g;\cc{B})|
    \end{equation*}

    Por tanto, tenemos que:
    \begin{equation*}
        |M(g;\cc{B}_1)| \cdot |M(g;\cc{B})| > 0
    \end{equation*}

    No obstante, tenemos que $-3\cdot 3<0$, por lo que esto no es posible.

    \item $\exists\; \cc{B}_2$ base de $V$ tal que $|M(g;\cc{B}_2)|=-7$.

    Como ambas matrices representan la misma base respecto de distintas métricas, tenemos que son congruentes, es decir, $\exists\; P$ regular tal que:
    \begin{equation*}
        M(g;\cc{B}_2) = P^t M(g;\cc{B}) P
    \end{equation*}

    Por tanto, tomando determinante,
    \begin{equation*}
        |M(g;\cc{B}_2)| = |P^t M(g;\cc{B}) P| = |P|^2 |M(g;\cc{B})| \Longrightarrow |P|^2 = \frac{7}{3} \Longrightarrow |P|=\pm \sqrt{\frac{7}{3}}
    \end{equation*}

    Un ejemplo, por tanto, de matriz $P$ podría ser:
    \begin{equation*}
        P=\left(\begin{array}{cc}
                \sqrt{\frac{7}{3}} & 0 \\
                0 & 1 \\
            \end{array}\right)
    \end{equation*}

    Como $P=M(\cc{B}_2;\cc{B})$, tenemos que, dado $\cc{B}=\{e_1, e_2\}$, tendríamos que:
    \begin{equation*}
        \cc{B}_2 = \left\{\sqrt{\frac{7}{3}}e_1, e_2\right\}
    \end{equation*}

    Por tanto, es \textbf{cierto}, y la base dada es un ejemplo.

    \item $\exists\; \cc{B}_3$ base de Sylvester de $V$ tal que:
    \begin{equation*}
        M(g;\cc{B}_3)=\left(\begin{array}{cc}
                1 & 0 \\
                0 & -1 \\
            \end{array}\right)
    \end{equation*}

    Como tenemos que $|M(g;\cc{B})|=-3\neq 0$, tenemos que $rg(M(g;\cc{B}))=2$, por lo que $Nul(g)=0$. Además, como el signo del determinante es un invariante, tenemos que es necesario que $|M(g;\cc{B})|<0$. Por tanto, el Teorema de Sylvester afirma que este enunciado es \textbf{cierto}.
\end{enumerate}
    
\end{ejercicio}

\end{document}