\documentclass[12pt]{article}

% Idioma y codificación
\usepackage[spanish, es-tabla]{babel}       %es-tabla para que se titule "Tabla"
\usepackage[utf8]{inputenc}

% Márgenes
\usepackage[a4paper,top=3cm,bottom=2.5cm,left=3cm,right=3cm]{geometry}

% Comentarios de bloque
\usepackage{verbatim}

% Paquetes de links
\usepackage[hidelinks]{hyperref}    % Permite enlaces
\usepackage{url}                    % redirecciona a la web

% Más opciones para enumeraciones
\usepackage{enumitem}

% Personalizar la portada
\usepackage{titling}

% Paquetes de tablas
\usepackage{multirow}


%------------------------------------------------------------------------

%Paquetes de figuras
\usepackage{caption}
\usepackage{subcaption} % Figuras al lado de otras
\usepackage{float}      % Poner figuras en el sitio indicado H.


% Paquetes de imágenes
\usepackage{graphicx}       % Paquete para añadir imágenes
\usepackage{transparent}    % Para manejar la opacidad de las figuras

% Paquete para usar colores
\usepackage[dvipsnames]{xcolor}
\usepackage{pagecolor}      % Para cambiar el color de la página

% Habilita tamaños de fuente mayores
\usepackage{fix-cm}

% Para los gráficos
\usepackage{tikz}

% Para poder situar los nodos en los grafos
\usetikzlibrary{positioning}


%------------------------------------------------------------------------

% Paquetes de matemáticas
\usepackage{mathtools, amsfonts, amssymb, mathrsfs}
\usepackage[makeroom]{cancel}     % Simplificar tachando
\usepackage{polynom}    % Divisiones y Ruffini
\usepackage{units} % Para poner fracciones diagonales con \nicefrac

\usepackage{pgfplots}   %Representar funciones
\pgfplotsset{compat=1.18}  % Versión 1.18

\usepackage{tikz-cd}    % Para usar diagramas de composiciones
\usetikzlibrary{calc}   % Para usar cálculo de coordenadas en tikz

%Definición de teoremas, etc.
\usepackage{amsthm}
%\swapnumbers   % Intercambia la posición del texto y de la numeración

\theoremstyle{plain}

\makeatletter
\@ifclassloaded{article}{
  \newtheorem{teo}{Teorema}[section]
}{
  \newtheorem{teo}{Teorema}[chapter]  % Se resetea en cada chapter
}
\makeatother

\newtheorem{coro}{Corolario}[teo]           % Se resetea en cada teorema
\newtheorem{prop}[teo]{Proposición}         % Usa el mismo contador que teorema
\newtheorem{lema}[teo]{Lema}                % Usa el mismo contador que teorema

\theoremstyle{remark}
\newtheorem*{observacion}{Observación}

\theoremstyle{definition}

\makeatletter
\@ifclassloaded{article}{
  \newtheorem{definicion}{Definición} [section]     % Se resetea en cada chapter
}{
  \newtheorem{definicion}{Definición} [chapter]     % Se resetea en cada chapter
}
\makeatother

\newtheorem*{notacion}{Notación}
\newtheorem*{ejemplo}{Ejemplo}
\newtheorem*{ejercicio*}{Ejercicio}             % No numerado
\newtheorem{ejercicio}{Ejercicio} [section]     % Se resetea en cada section


% Modificar el formato de la numeración del teorema "ejercicio"
\renewcommand{\theejercicio}{%
  \ifnum\value{section}=0 % Si no se ha iniciado ninguna sección
    \arabic{ejercicio}% Solo mostrar el número de ejercicio
  \else
    \thesection.\arabic{ejercicio}% Mostrar número de sección y número de ejercicio
  \fi
}


% \renewcommand\qedsymbol{$\blacksquare$}         % Cambiar símbolo QED
%------------------------------------------------------------------------

% Paquetes para encabezados
\usepackage{fancyhdr}
\pagestyle{fancy}
\fancyhf{}

\newcommand{\helv}{ % Modificación tamaño de letra
\fontfamily{}\fontsize{12}{12}\selectfont}
\setlength{\headheight}{15pt} % Amplía el tamaño del índice


%\usepackage{lastpage}   % Referenciar última pag   \pageref{LastPage}
\fancyfoot[C]{\thepage}

%------------------------------------------------------------------------

% Conseguir que no ponga "Capítulo 1". Sino solo "1."
\makeatletter
\@ifclassloaded{book}{
  \renewcommand{\chaptermark}[1]{\markboth{\thechapter.\ #1}{}} % En el encabezado
    
  \renewcommand{\@makechapterhead}[1]{%
  \vspace*{50\p@}%
  {\parindent \z@ \raggedright \normalfont
    \ifnum \c@secnumdepth >\m@ne
      \huge\bfseries \thechapter.\hspace{1em}\ignorespaces
    \fi
    \interlinepenalty\@M
    \Huge \bfseries #1\par\nobreak
    \vskip 40\p@
  }}
}
\makeatother

%------------------------------------------------------------------------
% Paquetes de cógido
\usepackage{minted}
\renewcommand\listingscaption{Código fuente}

\usepackage{fancyvrb}
% Personaliza el tamaño de los números de línea
\renewcommand{\theFancyVerbLine}{\small\arabic{FancyVerbLine}}

% Estilo para C++
\newminted{cpp}{
    frame=lines,
    framesep=2mm,
    baselinestretch=1.2,
    linenos,
    escapeinside=||
}

% para minted
\definecolor{LightGray}{rgb}{0.95,0.95,0.92}
\setminted{
    linenos=true,
    stepnumber=5,
    numberfirstline=true,
    autogobble,
    breaklines=true,
    breakautoindent=true,
    breaksymbolleft=,
    breaksymbolright=,
    breaksymbolindentleft=0pt,
    breaksymbolindentright=0pt,
    breaksymbolsepleft=0pt,
    breaksymbolsepright=0pt,
    fontsize=\footnotesize,
    bgcolor=LightGray,
    numbersep=10pt
}


\usepackage{listings} % Para incluir código desde un archivo

\renewcommand\lstlistingname{Código Fuente}
\renewcommand\lstlistlistingname{Índice de Códigos Fuente}

% Definir colores
\definecolor{vscodepurple}{rgb}{0.5,0,0.5}
\definecolor{vscodeblue}{rgb}{0,0,0.8}
\definecolor{vscodegreen}{rgb}{0,0.5,0}
\definecolor{vscodegray}{rgb}{0.5,0.5,0.5}
\definecolor{vscodebackground}{rgb}{0.97,0.97,0.97}
\definecolor{vscodelightgray}{rgb}{0.9,0.9,0.9}

% Configuración para el estilo de C similar a VSCode
\lstdefinestyle{vscode_C}{
  backgroundcolor=\color{vscodebackground},
  commentstyle=\color{vscodegreen},
  keywordstyle=\color{vscodeblue},
  numberstyle=\tiny\color{vscodegray},
  stringstyle=\color{vscodepurple},
  basicstyle=\scriptsize\ttfamily,
  breakatwhitespace=false,
  breaklines=true,
  captionpos=b,
  keepspaces=true,
  numbers=left,
  numbersep=5pt,
  showspaces=false,
  showstringspaces=false,
  showtabs=false,
  tabsize=2,
  frame=tb,
  framerule=0pt,
  aboveskip=10pt,
  belowskip=10pt,
  xleftmargin=10pt,
  xrightmargin=10pt,
  framexleftmargin=10pt,
  framexrightmargin=10pt,
  framesep=0pt,
  rulecolor=\color{vscodelightgray},
  backgroundcolor=\color{vscodebackground},
}

%------------------------------------------------------------------------

% Comandos definidos
\newcommand{\bb}[1]{\mathbb{#1}}
\newcommand{\cc}[1]{\mathcal{#1}}

% I prefer the slanted \leq
\let\oldleq\leq % save them in case they're every wanted
\let\oldgeq\geq
\renewcommand{\leq}{\leqslant}
\renewcommand{\geq}{\geqslant}

% Si y solo si
\newcommand{\sii}{\iff}

% Letras griegas
\newcommand{\eps}{\epsilon}
\newcommand{\veps}{\varepsilon}
\newcommand{\lm}{\lambda}

\newcommand{\ol}{\overline}
\newcommand{\ul}{\underline}
\newcommand{\wt}{\widetilde}
\newcommand{\wh}{\widehat}

\let\oldvec\vec
\renewcommand{\vec}{\overrightarrow}

% Derivadas parciales
\newcommand{\del}[2]{\frac{\partial #1}{\partial #2}}
\newcommand{\Del}[3]{\frac{\partial^{#1} #2}{\partial #3^{#1}}}
\newcommand{\deld}[2]{\dfrac{\partial #1}{\partial #2}}
\newcommand{\Deld}[3]{\dfrac{\partial^{#1} #2}{\partial #3^{#1}}}


\newcommand{\AstIg}{\stackrel{(\ast)}{=}}
\newcommand{\Hop}{\stackrel{L'H\hat{o}pital}{=}}

\newcommand{\red}[1]{{\color{red}#1}} % Para integrales, destacar los cambios.

% Método de integración
\newcommand{\MetInt}[2]{
    \left[\begin{array}{c}
        #1 \\ #2
    \end{array}\right]
}

% Declarar aplicaciones
% 1. Nombre aplicación
% 2. Dominio
% 3. Codominio
% 4. Variable
% 5. Imagen de la variable
\newcommand{\Func}[5]{
    \begin{equation*}
        \begin{array}{rrll}
            #1:& #2 & \longrightarrow & #3\\
               & #4 & \longmapsto & #5
        \end{array}
    \end{equation*}
}

%------------------------------------------------------------------------



\begin{document}

    % 1. Foto de fondo
    % 2. Título
    % 3. Encabezado Izquierdo
    % 4. Color de fondo
    % 5. Coord x del titulo
    % 6. Coord y del titulo
    % 7. Fecha

    
    % 1. Foto de fondo
% 2. Título
% 3. Encabezado Izquierdo
% 4. Color de fondo
% 5. Coord x del titulo
% 6. Coord y del titulo
% 7. Fecha

\newcommand{\portada}[7]{

    \portadaBase{#1}{#2}{#3}{#4}{#5}{#6}{#7}
    \portadaBook{#1}{#2}{#3}{#4}{#5}{#6}{#7}
}

\newcommand{\portadaExamen}[7]{

    \portadaBase{#1}{#2}{#3}{#4}{#5}{#6}{#7}
    \portadaArticle{#1}{#2}{#3}{#4}{#5}{#6}{#7}
}




\newcommand{\portadaBase}[7]{

    % Tiene la portada principal y la licencia Creative Commons
    
    % 1. Foto de fondo
    % 2. Título
    % 3. Encabezado Izquierdo
    % 4. Color de fondo
    % 5. Coord x del titulo
    % 6. Coord y del titulo
    % 7. Fecha
    
    
    \thispagestyle{empty}               % Sin encabezado ni pie de página
    \newgeometry{margin=0cm}        % Márgenes nulos para la primera página
    
    
    % Encabezado
    \fancyhead[L]{\helv #3}
    \fancyhead[R]{\helv \nouppercase{\leftmark}}
    
    
    \pagecolor{#4}        % Color de fondo para la portada
    
    \begin{figure}[p]
        \centering
        \transparent{0.3}           % Opacidad del 30% para la imagen
        
        \includegraphics[width=\paperwidth, keepaspectratio]{assets/#1}
    
        \begin{tikzpicture}[remember picture, overlay]
            \node[anchor=north west, text=white, opacity=1, font=\fontsize{60}{90}\selectfont\bfseries\sffamily, align=left] at (#5, #6) {#2};
            
            \node[anchor=south east, text=white, opacity=1, font=\fontsize{12}{18}\selectfont\sffamily, align=right] at (9.7, 3) {\textbf{\href{https://losdeldgiim.github.io/}{Los Del DGIIM}}};
            
            \node[anchor=south east, text=white, opacity=1, font=\fontsize{12}{15}\selectfont\sffamily, align=right] at (9.7, 1.8) {Doble Grado en Ingeniería Informática y Matemáticas\\Universidad de Granada};
        \end{tikzpicture}
    \end{figure}
    
    
    \restoregeometry        % Restaurar márgenes normales para las páginas subsiguientes
    \pagecolor{white}       % Restaurar el color de página
    
    
    \newpage
    \thispagestyle{empty}               % Sin encabezado ni pie de página
    \begin{tikzpicture}[remember picture, overlay]
        \node[anchor=south west, inner sep=3cm] at (current page.south west) {
            \begin{minipage}{0.5\paperwidth}
                \href{https://creativecommons.org/licenses/by-nc-nd/4.0/}{
                    \includegraphics[height=2cm]{assets/Licencia.png}
                }\vspace{1cm}\\
                Esta obra está bajo una
                \href{https://creativecommons.org/licenses/by-nc-nd/4.0/}{
                    Licencia Creative Commons Atribución-NoComercial-SinDerivadas 4.0 Internacional (CC BY-NC-ND 4.0).
                }\\
    
                Eres libre de compartir y redistribuir el contenido de esta obra en cualquier medio o formato, siempre y cuando des el crédito adecuado a los autores originales y no persigas fines comerciales. 
            \end{minipage}
        };
    \end{tikzpicture}
    
    
    
    % 1. Foto de fondo
    % 2. Título
    % 3. Encabezado Izquierdo
    % 4. Color de fondo
    % 5. Coord x del titulo
    % 6. Coord y del titulo
    % 7. Fecha


}


\newcommand{\portadaBook}[7]{

    % 1. Foto de fondo
    % 2. Título
    % 3. Encabezado Izquierdo
    % 4. Color de fondo
    % 5. Coord x del titulo
    % 6. Coord y del titulo
    % 7. Fecha

    % Personaliza el formato del título
    \pretitle{\begin{center}\bfseries\fontsize{42}{56}\selectfont}
    \posttitle{\par\end{center}\vspace{2em}}
    
    % Personaliza el formato del autor
    \preauthor{\begin{center}\Large}
    \postauthor{\par\end{center}\vfill}
    
    % Personaliza el formato de la fecha
    \predate{\begin{center}\huge}
    \postdate{\par\end{center}\vspace{2em}}
    
    \title{#2}
    \author{\href{https://losdeldgiim.github.io/}{Los Del DGIIM}}
    \date{Granada, #7}
    \maketitle
    
    \tableofcontents
}




\newcommand{\portadaArticle}[7]{

    % 1. Foto de fondo
    % 2. Título
    % 3. Encabezado Izquierdo
    % 4. Color de fondo
    % 5. Coord x del titulo
    % 6. Coord y del titulo
    % 7. Fecha

    % Personaliza el formato del título
    \pretitle{\begin{center}\bfseries\fontsize{42}{56}\selectfont}
    \posttitle{\par\end{center}\vspace{2em}}
    
    % Personaliza el formato del autor
    \preauthor{\begin{center}\Large}
    \postauthor{\par\end{center}\vspace{3em}}
    
    % Personaliza el formato de la fecha
    \predate{\begin{center}\huge}
    \postdate{\par\end{center}\vspace{5em}}
    
    \title{#2}
    \author{\href{https://losdeldgiim.github.io/}{Los Del DGIIM}}
    \date{Granada, #7}
    \thispagestyle{empty}               % Sin encabezado ni pie de página
    \maketitle
    \vfill
}
    \portadaExamen{ffccA4.jpg}{Geometría II\\Examen V}{Geometría II. Examen V}{MidnightBlue}{-8}{28}{2023}{Arturo Olivares Martos}

    \begin{description}
        \item[Asignatura] Geometría II.
        \item[Curso Académico] 2019-20.
        \item[Grado] Matemáticas.
        %\item[Grupo] A.
        %\item[Profesor] Francisco Milán López.
        \item[Descripción] Parcial del Tema 2. Formas Bilineales Simétricas.
        \item[Fecha] 5 de mayo de 2020.
        %\item[Duración] 60 minutos.
    
    \end{description}
    \newpage
    
    \begin{ejercicio}
    Consideramos sobre $\bb{R}^3$ la métrica $g$ cuya forma cuadrática asociada es:
    \begin{equation*}
        \omega(x,y,z) = ax^2 + 2xy + y^2 + 2yz + az^2,\qquad a\in \bb{R}
    \end{equation*}

    \begin{enumerate}
        \item Encontrar los valores de $a$ para los que $g$ es definida positiva.

        Tenemos que la matriz asociada a $g$ en la base usual es:
        \begin{equation*}
            A = M(g;\cc{B}_u) = \left(\begin{array}{ccc}
                a & 1 & 0\\
                1 & 1 & 1\\
                0 & 1 & a
            \end{array}\right)
        \end{equation*}

        Para que $A$ sea definida positiva, es necesario que todos los menores principales sean positivos. Por tanto,
        \begin{equation*}
            |a|=a \qquad
            \left|\begin{array}{cc}
                a & 1\\
                1 & 1\\
            \end{array}\right| = a-1 \qquad
            |A| = a^2-2a = a(a-2)
        \end{equation*}

        Por tanto, para que $g$ sea definida positiva es necesario que $a>2$.

        \item Para $a=0$, obtener la base que nos da el Teorema de Sylvester.

        
        Sea $\cc{B}_S=\{\bar{e_1}, \bar{e_2}, \bar{e_3}\}$
        Tengo que $rg(A)=2$, por lo que $Nul(g)=1$. Obtengo $\bar{e_3}\in Ker(g)$.
        \begin{equation*}\begin{split}
            Ker(g) &= \{v \in \bb{R}^3 \mid g(u,v) = 0 \qquad \forall u\in \bb{R}^3\} 
            =\\
            &= \left\{ \left(\begin{array}{c}
                 x_1 \\ x_2 \\ x_3
            \end{array} \right) \in \bb{R}^3 \mid
            A
            \left(\begin{array}{c}
                 x_1 \\ x_2 \\ x_3
            \end{array} \right) = 0\right\} \\
            &= \left\{ \left(\begin{array}{c}
                 x_1 \\ x_2 \\ x_3
            \end{array} \right) \in \bb{R}^3 \mid \left(\begin{array}{ccc}
                0 & 1 & 0\\
                1 & 1 & 1 \\
                0 & 1 & 0
            \end{array} \right) 
            \left(\begin{array}{c}
                 x_1 \\ x_2 \\ x_3
            \end{array} \right) = 0\right\} \\
            &= \left\{ \left(\begin{array}{c}
                 x_1 \\ x_2 \\ x_3
            \end{array} \right) \in \bb{R}^3  \left|
            \begin{array}{c}
                x_2 = 0\\
                x_1 + x_2 + x_3 = 0 
            \end{array}\right.        
            \right\}
            = \cc{L} \left\{ \left(\begin{array}{c}
                 1 \\ 0 \\ -1 
            \end{array} \right) \right\}
        \end{split}\end{equation*}
        Por tanto, $\bar{e_3} = (1, 0, -1)^t_{\cc{B}_u} = e_1-e_3$.
        \begin{equation*}
            g(\bar{e_3}, \bar{e_3}) = 0
        \end{equation*}

        Sea $\bar{e_1}=e_2$.
        \begin{equation*}
            g(\bar{e_1}, \bar{e_1}) = g(e_2, e_2) = 1
        \end{equation*}

        Busco ahora $\bar{e_2}\in <\bar{e_1}>^\perp$:
        \begin{equation*}\begin{split}
                <\bar{e_1}>^\perp &= \{v \in \bb{R}^3 \mid g(\bar{e_1},v) = 0\} \\
                &= \left\{ \left(\begin{array}{c}
                     x_1 \\ x_2 \\ x_3
                \end{array} \right) \in \bb{R}^3 \mid (0, 1, 0) \left(\begin{array}{ccc}
                    0 & 1 & 0 \\
                    1 & 1 & 1 \\
                    0 & 1 & 0 \\
                \end{array} \right) 
                \left(\begin{array}{c}
                     x_1 \\ x_2 \\ x_3
                \end{array} \right) = 0\right\} \\
                &= \left\{ \left(\begin{array}{c}
                     x_1 \\ x_2 \\ x_3 
                \end{array} \right) \in \bb{R}^3 \mid (1,1,1)
                \left(\begin{array}{c}
                     x_1 \\ x_2 \\ x_3
                \end{array} \right) = 0\right\} \\
                &= \left\{ \left(\begin{array}{c}
                     x_1 \\ x_2 \\ x_3
                \end{array} \right) \in \bb{R}^3 \mid x_1+x_2+x_3=0\right\}
            \end{split}\end{equation*}

            Sea $\bar{e_2} = (1, -1, 0)^t$. Tenemos que:
            \begin{equation*}
                g(\bar{e_2}, \bar{e_2}) = g(e_1, e_1) + g(e_2, e_2) - 2g(e_1, e_2) = 0 + 1 -2 = -1
            \end{equation*}

            Por tanto, dado $\cc{B}_S = \left\{\left(\begin{array}{c}
                 0 \\ 1 \\ 0
            \end{array} \right),
            \left(\begin{array}{c}
                 1 \\ -1 \\ 0
            \end{array} \right),
            \left(\begin{array}{c}
                 1 \\ 0 \\ -1
            \end{array} \right)\right\}$, tenemos que:
            \begin{equation*}
            M(g;\cc{B}_S) = \left(\begin{array}{ccc}
                1 & \\
                 & -1 \\
                && 0
            \end{array}\right)
        \end{equation*}


        \item Para $a=1$, encontrar la ecuación reducida de la cuádrica
        \begin{equation*}
            ax^2 +2xy +y^2+2yz+az^2 = m-5
        \end{equation*}
        donde $m$ es el último dígito de vuestro DNI.

        Como $m-5$ es una constante, obtengo en primer lugar la expresión reducida de $\omega$ para $a=1$. Es decir, clasifico la matriz siguiente:
        \begin{equation*}
            A = M(g;\cc{B}_u) = \left(\begin{array}{ccc}
                1 & 1 & 0\\
                1 & 1 & 1\\
                0 & 1 & 1
            \end{array}\right)
        \end{equation*}
        Tengo que $|A|=1-1-1=-1$. Además, $g$ no es definida negativa, ya que $g(e_1, e_1)=1$. Por tanto, tenemos que:
        \begin{equation*}
            A = M(g;\cc{B}_u) = \left(\begin{array}{ccc}
                1 & 1 & 0\\
                1 & 1 & 1\\
                0 & 1 & 1
            \end{array}\right) \sim_c
            \left(\begin{array}{ccc}
                1 & \\
                 & 1 \\
                && -1
            \end{array}\right)
        \end{equation*}

        Por tanto, en la Base de Sylvester, tenemos que:
        \begin{equation*}
            \omega(\bar{x}, \bar{y}, \bar{z}) = \bar{x}^2 + \bar{y}^2 - \Bar{z}^2
        \end{equation*}

        Por tanto, la expresión reducida de la cuádrica dada es:
        \begin{equation*}
            \bar{x}^2 + \bar{y}^2 - \Bar{z}^2 = m-5
        \end{equation*}
    \end{enumerate}
\end{ejercicio}

\begin{ejercicio}
    Razonar si las siguientes afirmaciones son verdaderas o falsas.
    \begin{enumerate}
        \item Sean $A,B$ dos matrices simétricas reales de orden 2 tales que $A^2$ y $B^2$ son congruentes sobre $\bb{R}$. Entonces, $A$ y $B$ son congruentes sobre los complejos.

        Veamos en primer lugar que, dado $A\in S_2(\bb{R})$, se cumple lo siguiente:
        \begin{equation} \label{Ej2.1.A0}
            A^2=0 \Longrightarrow A=0
        \end{equation}
        
        Sea $A=\left(\begin{array}{cc}
            a & b \\
            b & c
        \end{array}\right)$. Entonces:
        \begin{equation*}
            A^2=0=\left(\begin{array}{cc}
            a & b \\
            b & c
        \end{array}\right)
        \left(\begin{array}{cc}
            a & b \\
            b & c
        \end{array}\right)=
        \left(\begin{array}{cc}
            a^2+b^2 & b(a+c) \\
            b(a+b) & b^2+c^2
        \end{array}\right) = 0
        \end{equation*}

        Por tanto, tenemos que:
        \begin{gather*}
            a^2+b^2=0 \Longrightarrow a=b=0 \\
            b^2+c^2=0 \Longrightarrow b=c=0
        \end{gather*}
        Por tanto, $A=0$.

        Veamos ahora que, dada $A\in \cc{S}_2(\bb{R})$, se tiene que:
        \begin{equation}\label{Ej2.1.rg}
            rg(A)=rg(A^2)
        \end{equation}

        Tenemos que $0\leq rg(A)\leq 2$:
        \begin{itemize}
            \item Supongamos que $rg(A)=2$.

            Tenemos que $|A|\neq 0 \Longrightarrow |A^2| = |A|^2 \neq 0 \Longrightarrow rg(A^2)=2$.

            \item Supongamos que $rg(A)=1$.

            Tenemos que $|A|=0$ pero $A\neq 0 \Longrightarrow |A^2| = |A|^2 = 0 \Longrightarrow rg(A^2)<2$.

            Como $A\neq 0$, por la E. \ref{Ej2.1.A0}, tenemos que $ A^2 \neq 0$. Por tanto, $rg(A^2)>0$.

            En conclusión, tenemos que $rg(A^2) = 1$.

            \item Supongamos que $rg(A)=0$.

            Entonces, $A=0\Longrightarrow A^2=0 \Longrightarrow rg(A^2)=0$.
        \end{itemize}
        Por tanto, $rg(A)=rg(A)^2$.

        Procedemos ahora a demostrar lo pedido. Como $A^2\sim_c B^2$, se tiene que:
        \begin{equation*}
            rg(A)\stackrel{Ec.\;\ref{Ej2.1.rg}}{=} rg(A^2) \stackrel{A^2\sim_c B^2}{=} rg(B^2) \stackrel{Ec.\;\ref{Ej2.1.rg}}{=} rg(B)
        \end{equation*}
        Por tanto, tenemos que $rg(A)=rg(B)$.

        Por el Teorema de Sylvester en el caso de $\bb{K}=\bb{C}$, $rg(A)=rg(B)\Longrightarrow A\sim_c B$ sobre $\bb{C}$.

        Por tanto, es \textbf{cierto}.

        \item Sea $g$ una métrica sobre $\bb{R}^3$ y $U_1, U_2 \subset \bb{R}^3$ dos planos vectoriales distintos. Si $g$ es definida positiva sobre $U_1$ y $U_2$, entonces $g$ es definida positiva en $\bb{R}^3$.

        Sea $\cc{B}=\{e_1, e_2, e_3\}$, y sea:
        \begin{equation*}
            A=M(g; \cc{B}) = \left(\begin{array}{ccc}
                8 & 0 & 10 \\
                0 & 1 & 0 \\
                10 & 0 & 1
            \end{array}\right)
        \end{equation*}

        Como $A\in \cc{S}_3(\bb{R})$, tenemos que es una métrica.

        Sea $U_1 = \cc{L}(\{e_1, e_2\})$, $U_2 = \cc{L}(\{e_2, e_3\})$. Tenemos que $g$ es definida positiva sobre ambos subespacios, pero $$|A|=8-10^2<0$$
        Por tanto, tenemos que $g$ no puede ser definida positiva. Por tanto, es \textbf{falso}.

        \item Sea $(V^2, g)$ un plano vectorial euclídeo, $\cc{B}=\{e_1, e_2\}$ una base con $g(e_1, e_1)=g(e_2, e_2)=1$, $g(e_1, e_2)\neq 0$ y $f:V\to V$ un endomorfismo dado por:
        \begin{equation*}
            f(e_1)=e_2 \qquad f(e_2)=e_1.
        \end{equation*}
        Entonces $f$ es un endomorfismo autoadjunto.
    \end{enumerate}
\end{ejercicio}

\end{document}