\documentclass[12pt]{article}

% Idioma y codificación
\usepackage[spanish, es-tabla]{babel}       %es-tabla para que se titule "Tabla"
\usepackage[utf8]{inputenc}

% Márgenes
\usepackage[a4paper,top=3cm,bottom=2.5cm,left=3cm,right=3cm]{geometry}

% Comentarios de bloque
\usepackage{verbatim}

% Paquetes de links
\usepackage[hidelinks]{hyperref}    % Permite enlaces
\usepackage{url}                    % redirecciona a la web

% Más opciones para enumeraciones
\usepackage{enumitem}

% Personalizar la portada
\usepackage{titling}

% Paquetes de tablas
\usepackage{multirow}


%------------------------------------------------------------------------

%Paquetes de figuras
\usepackage{caption}
\usepackage{subcaption} % Figuras al lado de otras
\usepackage{float}      % Poner figuras en el sitio indicado H.


% Paquetes de imágenes
\usepackage{graphicx}       % Paquete para añadir imágenes
\usepackage{transparent}    % Para manejar la opacidad de las figuras

% Paquete para usar colores
\usepackage[dvipsnames]{xcolor}
\usepackage{pagecolor}      % Para cambiar el color de la página

% Habilita tamaños de fuente mayores
\usepackage{fix-cm}

% Para los gráficos
\usepackage{tikz}

% Para poder situar los nodos en los grafos
\usetikzlibrary{positioning}


%------------------------------------------------------------------------

% Paquetes de matemáticas
\usepackage{mathtools, amsfonts, amssymb, mathrsfs}
\usepackage[makeroom]{cancel}     % Simplificar tachando
\usepackage{polynom}    % Divisiones y Ruffini
\usepackage{units} % Para poner fracciones diagonales con \nicefrac

\usepackage{pgfplots}   %Representar funciones
\pgfplotsset{compat=1.18}  % Versión 1.18

\usepackage{tikz-cd}    % Para usar diagramas de composiciones
\usetikzlibrary{calc}   % Para usar cálculo de coordenadas en tikz

%Definición de teoremas, etc.
\usepackage{amsthm}
%\swapnumbers   % Intercambia la posición del texto y de la numeración

\theoremstyle{plain}

\makeatletter
\@ifclassloaded{article}{
  \newtheorem{teo}{Teorema}[section]
}{
  \newtheorem{teo}{Teorema}[chapter]  % Se resetea en cada chapter
}
\makeatother

\newtheorem{coro}{Corolario}[teo]           % Se resetea en cada teorema
\newtheorem{prop}[teo]{Proposición}         % Usa el mismo contador que teorema
\newtheorem{lema}[teo]{Lema}                % Usa el mismo contador que teorema

\theoremstyle{remark}
\newtheorem*{observacion}{Observación}

\theoremstyle{definition}

\makeatletter
\@ifclassloaded{article}{
  \newtheorem{definicion}{Definición} [section]     % Se resetea en cada chapter
}{
  \newtheorem{definicion}{Definición} [chapter]     % Se resetea en cada chapter
}
\makeatother

\newtheorem*{notacion}{Notación}
\newtheorem*{ejemplo}{Ejemplo}
\newtheorem*{ejercicio*}{Ejercicio}             % No numerado
\newtheorem{ejercicio}{Ejercicio} [section]     % Se resetea en cada section


% Modificar el formato de la numeración del teorema "ejercicio"
\renewcommand{\theejercicio}{%
  \ifnum\value{section}=0 % Si no se ha iniciado ninguna sección
    \arabic{ejercicio}% Solo mostrar el número de ejercicio
  \else
    \thesection.\arabic{ejercicio}% Mostrar número de sección y número de ejercicio
  \fi
}


% \renewcommand\qedsymbol{$\blacksquare$}         % Cambiar símbolo QED
%------------------------------------------------------------------------

% Paquetes para encabezados
\usepackage{fancyhdr}
\pagestyle{fancy}
\fancyhf{}

\newcommand{\helv}{ % Modificación tamaño de letra
\fontfamily{}\fontsize{12}{12}\selectfont}
\setlength{\headheight}{15pt} % Amplía el tamaño del índice


%\usepackage{lastpage}   % Referenciar última pag   \pageref{LastPage}
\fancyfoot[C]{\thepage}

%------------------------------------------------------------------------

% Conseguir que no ponga "Capítulo 1". Sino solo "1."
\makeatletter
\@ifclassloaded{book}{
  \renewcommand{\chaptermark}[1]{\markboth{\thechapter.\ #1}{}} % En el encabezado
    
  \renewcommand{\@makechapterhead}[1]{%
  \vspace*{50\p@}%
  {\parindent \z@ \raggedright \normalfont
    \ifnum \c@secnumdepth >\m@ne
      \huge\bfseries \thechapter.\hspace{1em}\ignorespaces
    \fi
    \interlinepenalty\@M
    \Huge \bfseries #1\par\nobreak
    \vskip 40\p@
  }}
}
\makeatother

%------------------------------------------------------------------------
% Paquetes de cógido
\usepackage{minted}
\renewcommand\listingscaption{Código fuente}

\usepackage{fancyvrb}
% Personaliza el tamaño de los números de línea
\renewcommand{\theFancyVerbLine}{\small\arabic{FancyVerbLine}}

% Estilo para C++
\newminted{cpp}{
    frame=lines,
    framesep=2mm,
    baselinestretch=1.2,
    linenos,
    escapeinside=||
}

% para minted
\definecolor{LightGray}{rgb}{0.95,0.95,0.92}
\setminted{
    linenos=true,
    stepnumber=5,
    numberfirstline=true,
    autogobble,
    breaklines=true,
    breakautoindent=true,
    breaksymbolleft=,
    breaksymbolright=,
    breaksymbolindentleft=0pt,
    breaksymbolindentright=0pt,
    breaksymbolsepleft=0pt,
    breaksymbolsepright=0pt,
    fontsize=\footnotesize,
    bgcolor=LightGray,
    numbersep=10pt
}


\usepackage{listings} % Para incluir código desde un archivo

\renewcommand\lstlistingname{Código Fuente}
\renewcommand\lstlistlistingname{Índice de Códigos Fuente}

% Definir colores
\definecolor{vscodepurple}{rgb}{0.5,0,0.5}
\definecolor{vscodeblue}{rgb}{0,0,0.8}
\definecolor{vscodegreen}{rgb}{0,0.5,0}
\definecolor{vscodegray}{rgb}{0.5,0.5,0.5}
\definecolor{vscodebackground}{rgb}{0.97,0.97,0.97}
\definecolor{vscodelightgray}{rgb}{0.9,0.9,0.9}

% Configuración para el estilo de C similar a VSCode
\lstdefinestyle{vscode_C}{
  backgroundcolor=\color{vscodebackground},
  commentstyle=\color{vscodegreen},
  keywordstyle=\color{vscodeblue},
  numberstyle=\tiny\color{vscodegray},
  stringstyle=\color{vscodepurple},
  basicstyle=\scriptsize\ttfamily,
  breakatwhitespace=false,
  breaklines=true,
  captionpos=b,
  keepspaces=true,
  numbers=left,
  numbersep=5pt,
  showspaces=false,
  showstringspaces=false,
  showtabs=false,
  tabsize=2,
  frame=tb,
  framerule=0pt,
  aboveskip=10pt,
  belowskip=10pt,
  xleftmargin=10pt,
  xrightmargin=10pt,
  framexleftmargin=10pt,
  framexrightmargin=10pt,
  framesep=0pt,
  rulecolor=\color{vscodelightgray},
  backgroundcolor=\color{vscodebackground},
}

%------------------------------------------------------------------------

% Comandos definidos
\newcommand{\bb}[1]{\mathbb{#1}}
\newcommand{\cc}[1]{\mathcal{#1}}

% I prefer the slanted \leq
\let\oldleq\leq % save them in case they're every wanted
\let\oldgeq\geq
\renewcommand{\leq}{\leqslant}
\renewcommand{\geq}{\geqslant}

% Si y solo si
\newcommand{\sii}{\iff}

% Letras griegas
\newcommand{\eps}{\epsilon}
\newcommand{\veps}{\varepsilon}
\newcommand{\lm}{\lambda}

\newcommand{\ol}{\overline}
\newcommand{\ul}{\underline}
\newcommand{\wt}{\widetilde}
\newcommand{\wh}{\widehat}

\let\oldvec\vec
\renewcommand{\vec}{\overrightarrow}

% Derivadas parciales
\newcommand{\del}[2]{\frac{\partial #1}{\partial #2}}
\newcommand{\Del}[3]{\frac{\partial^{#1} #2}{\partial #3^{#1}}}
\newcommand{\deld}[2]{\dfrac{\partial #1}{\partial #2}}
\newcommand{\Deld}[3]{\dfrac{\partial^{#1} #2}{\partial #3^{#1}}}


\newcommand{\AstIg}{\stackrel{(\ast)}{=}}
\newcommand{\Hop}{\stackrel{L'H\hat{o}pital}{=}}

\newcommand{\red}[1]{{\color{red}#1}} % Para integrales, destacar los cambios.

% Método de integración
\newcommand{\MetInt}[2]{
    \left[\begin{array}{c}
        #1 \\ #2
    \end{array}\right]
}

% Declarar aplicaciones
% 1. Nombre aplicación
% 2. Dominio
% 3. Codominio
% 4. Variable
% 5. Imagen de la variable
\newcommand{\Func}[5]{
    \begin{equation*}
        \begin{array}{rrll}
            #1:& #2 & \longrightarrow & #3\\
               & #4 & \longmapsto & #5
        \end{array}
    \end{equation*}
}

%------------------------------------------------------------------------



\begin{document}

    % 1. Foto de fondo
    % 2. Título
    % 3. Encabezado Izquierdo
    % 4. Color de fondo
    % 5. Coord x del titulo
    % 6. Coord y del titulo
    % 7. Fecha

    
    % 1. Foto de fondo
% 2. Título
% 3. Encabezado Izquierdo
% 4. Color de fondo
% 5. Coord x del titulo
% 6. Coord y del titulo
% 7. Fecha

\newcommand{\portada}[7]{

    \portadaBase{#1}{#2}{#3}{#4}{#5}{#6}{#7}
    \portadaBook{#1}{#2}{#3}{#4}{#5}{#6}{#7}
}

\newcommand{\portadaExamen}[7]{

    \portadaBase{#1}{#2}{#3}{#4}{#5}{#6}{#7}
    \portadaArticle{#1}{#2}{#3}{#4}{#5}{#6}{#7}
}




\newcommand{\portadaBase}[7]{

    % Tiene la portada principal y la licencia Creative Commons
    
    % 1. Foto de fondo
    % 2. Título
    % 3. Encabezado Izquierdo
    % 4. Color de fondo
    % 5. Coord x del titulo
    % 6. Coord y del titulo
    % 7. Fecha
    
    
    \thispagestyle{empty}               % Sin encabezado ni pie de página
    \newgeometry{margin=0cm}        % Márgenes nulos para la primera página
    
    
    % Encabezado
    \fancyhead[L]{\helv #3}
    \fancyhead[R]{\helv \nouppercase{\leftmark}}
    
    
    \pagecolor{#4}        % Color de fondo para la portada
    
    \begin{figure}[p]
        \centering
        \transparent{0.3}           % Opacidad del 30% para la imagen
        
        \includegraphics[width=\paperwidth, keepaspectratio]{assets/#1}
    
        \begin{tikzpicture}[remember picture, overlay]
            \node[anchor=north west, text=white, opacity=1, font=\fontsize{60}{90}\selectfont\bfseries\sffamily, align=left] at (#5, #6) {#2};
            
            \node[anchor=south east, text=white, opacity=1, font=\fontsize{12}{18}\selectfont\sffamily, align=right] at (9.7, 3) {\textbf{\href{https://losdeldgiim.github.io/}{Los Del DGIIM}}};
            
            \node[anchor=south east, text=white, opacity=1, font=\fontsize{12}{15}\selectfont\sffamily, align=right] at (9.7, 1.8) {Doble Grado en Ingeniería Informática y Matemáticas\\Universidad de Granada};
        \end{tikzpicture}
    \end{figure}
    
    
    \restoregeometry        % Restaurar márgenes normales para las páginas subsiguientes
    \pagecolor{white}       % Restaurar el color de página
    
    
    \newpage
    \thispagestyle{empty}               % Sin encabezado ni pie de página
    \begin{tikzpicture}[remember picture, overlay]
        \node[anchor=south west, inner sep=3cm] at (current page.south west) {
            \begin{minipage}{0.5\paperwidth}
                \href{https://creativecommons.org/licenses/by-nc-nd/4.0/}{
                    \includegraphics[height=2cm]{assets/Licencia.png}
                }\vspace{1cm}\\
                Esta obra está bajo una
                \href{https://creativecommons.org/licenses/by-nc-nd/4.0/}{
                    Licencia Creative Commons Atribución-NoComercial-SinDerivadas 4.0 Internacional (CC BY-NC-ND 4.0).
                }\\
    
                Eres libre de compartir y redistribuir el contenido de esta obra en cualquier medio o formato, siempre y cuando des el crédito adecuado a los autores originales y no persigas fines comerciales. 
            \end{minipage}
        };
    \end{tikzpicture}
    
    
    
    % 1. Foto de fondo
    % 2. Título
    % 3. Encabezado Izquierdo
    % 4. Color de fondo
    % 5. Coord x del titulo
    % 6. Coord y del titulo
    % 7. Fecha


}


\newcommand{\portadaBook}[7]{

    % 1. Foto de fondo
    % 2. Título
    % 3. Encabezado Izquierdo
    % 4. Color de fondo
    % 5. Coord x del titulo
    % 6. Coord y del titulo
    % 7. Fecha

    % Personaliza el formato del título
    \pretitle{\begin{center}\bfseries\fontsize{42}{56}\selectfont}
    \posttitle{\par\end{center}\vspace{2em}}
    
    % Personaliza el formato del autor
    \preauthor{\begin{center}\Large}
    \postauthor{\par\end{center}\vfill}
    
    % Personaliza el formato de la fecha
    \predate{\begin{center}\huge}
    \postdate{\par\end{center}\vspace{2em}}
    
    \title{#2}
    \author{\href{https://losdeldgiim.github.io/}{Los Del DGIIM}}
    \date{Granada, #7}
    \maketitle
    
    \tableofcontents
}




\newcommand{\portadaArticle}[7]{

    % 1. Foto de fondo
    % 2. Título
    % 3. Encabezado Izquierdo
    % 4. Color de fondo
    % 5. Coord x del titulo
    % 6. Coord y del titulo
    % 7. Fecha

    % Personaliza el formato del título
    \pretitle{\begin{center}\bfseries\fontsize{42}{56}\selectfont}
    \posttitle{\par\end{center}\vspace{2em}}
    
    % Personaliza el formato del autor
    \preauthor{\begin{center}\Large}
    \postauthor{\par\end{center}\vspace{3em}}
    
    % Personaliza el formato de la fecha
    \predate{\begin{center}\huge}
    \postdate{\par\end{center}\vspace{5em}}
    
    \title{#2}
    \author{\href{https://losdeldgiim.github.io/}{Los Del DGIIM}}
    \date{Granada, #7}
    \thispagestyle{empty}               % Sin encabezado ni pie de página
    \maketitle
    \vfill
}
    \portadaExamen{ffccA4.jpg}{Geometría II\\Examen VII}{Geometría II. Examen VII}{MidnightBlue}{-8}{28}{2023}{Arturo Olivares Martos}

    \begin{description}
        \item[Asignatura] Geometría II.
        \item[Curso Académico] 2016-17.
        \item[Grado] Matemáticas.
        %\item[Grupo] A.
        \item[Profesor] Desconocido\footnote{El examen lo pone el departamento.}.
        \item[Descripción] Convocatoria Ordinaria.
        \item[Fecha] 6 de junio de 2017.
        %\item[Duración] 60 minutos.
    
    \end{description}
    \newpage
    
    \begin{ejercicio} [\textbf{2.5 puntos}]
Enuncia y demuestra el Teorema de Cayley Hamilton.
\end{ejercicio}

\begin{ejercicio} [\textbf{2.5 puntos}]
    En $\bb{R}^3$ se considera la métrica $g$ cuya matriz en la base usual viene dada por:
    \begin{equation*}
        G=M(g,\cc{B}_u)=\left(\begin{array}{ccc}
            3 & 1 & 0 \\
            1 & 1 & 1 \\
            0 & 1 & 2
        \end{array}\right)
    \end{equation*}

    y el endomorfismo $f\in End(\bb{R}^3)$ dado por:
    \begin{equation*}
        f(x,y,z)=(2x+y+z, -x-y-2z, x+y+2z)
    \end{equation*}

    \begin{enumerate}
        \item Comprueba que $g$ es una métrica euclídea.

        El enunciado ya nos afirma que es una métrica, por lo que solo es necesario que sea euclídea (definida positiva). Para ello, compruebo que todos sus menores principales son positivos:
        \begin{equation*}
            |3|=3>0
            \qquad
            \left|\begin{array}{cc}
                3 & 1 \\
                1 & 1 \\
            \end{array}\right| = 3-1 = 2>0
            \qquad
            |G|=6-3-2=1>0
        \end{equation*}

        Por tanto, tenemos que es definida positiva y, por tanto, euclídea.

        \item ¿Es $f$ autoadjunto respecto de la métrica $g$?

        Tenemos que:
        \begin{equation*}
            F=M(f,\cc{B}_u)=\left(\begin{array}{ccc}
                2 & 1 & 1 \\
                -1 & -1 & -2 \\
                1 & 1 & 2
            \end{array}\right)
        \end{equation*}
    
        Tenemos que $F$ es autoadjunto respecto de $(\bb{R}^3, g) \Longleftrightarrow F^tG=GF$:
        \begin{equation*}
            F^tG = \left(\begin{array}{ccc}
                2 & -1 & 1 \\
                1 & -1 & 1 \\
                1 & -2 & 2
            \end{array}\right)
            \left(\begin{array}{ccc}
                3 & 1 & 0 \\
                1 & 1 & 1 \\
                0 & 1 & 2
            \end{array}\right)
            =
            \left(\begin{array}{ccc}
                5 & 2 & 1 \\
                2 & 1 & 1 \\
                1 & 1 & 2
            \end{array}\right)
        \end{equation*}
        \begin{equation*}
            GF = \left(\begin{array}{ccc}
                3 & 1 & 0 \\
                1 & 1 & 1 \\
                0 & 1 & 2
            \end{array}\right)
            \left(\begin{array}{ccc}
                2 & 1 & 1 \\
                -1 & -1 & -2 \\
                1 & 1 & 2
            \end{array}\right)
            =
            \left(\begin{array}{ccc}
                5 & 2 & 1 \\
                2 & 1 & 1 \\
                1 & 1 & 2
            \end{array}\right)
        \end{equation*}
    
        Por tanto, tenemos que sí es autoadjunto respecto de $g$.

        \item En caso afirmativo, encuentra una base ortonormal de vectores propios de $f$.

        Calculamos en primer lugar el polinomio característico:
        \begin{equation*}\begin{split}
            P_f(\lambda) = |F-\lambda I| &= \left|\begin{array}{ccc}
                2-\lambda & 1 & 1 \\
                -1 & -1-\lambda & -2 \\
                1 & 1 & 2-\lambda
            \end{array}\right|
            =
            \left|\begin{array}{ccc}
                3-\lambda & 1 & 1 \\
                -3 & -1-\lambda & -2 \\
                3-\lambda & 1 & 2-\lambda
            \end{array}\right|=\\
            &=
            \left|\begin{array}{ccc}
                3-\lambda & 1 & 1 \\
                -3 & -1-\lambda & -2 \\
                0 & 0 & 1-\lambda
            \end{array}\right|
            = (1-\lambda)\left|\begin{array}{cc}
                3-\lambda & 1 \\
                -3 & -1-\lambda \\
            \end{array}\right|
            =\\&= (1-\lambda)\left|\begin{array}{cc}
                2-\lambda & 1 \\
                -2-\lambda & -1-\lambda \\
            \end{array}\right|
            = (1-\lambda)(2-\lambda)\left|\begin{array}{cc}
                1 & 1 \\
                -1 & -1-\lambda \\
            \end{array}\right|
            =\\&= -\lambda(1-\lambda)(2-\lambda)
        \end{split}\end{equation*}

        Por tanto, los valores propios son $\{0,1,2\}$. Calculamos los subespacios propios:
        \begin{equation*}
            \begin{split}
                V_0 = \left\{\left(\begin{array}{c}
                    x\\y\\z
                \end{array}\right)\in \bb{R}^3 \left|
                \begin{array}{c}
                    2x+y+z=0 \\
                    -x-y-2z=0 \\
                    x+y+2z=0
                \end{array}
                \right.\right\}
                = \cc{L}\left\{\left(\begin{array}{c}
                    1\\-3\\1
                \end{array}\right)\right\}
            \end{split}
        \end{equation*}
        \begin{equation*}
            \begin{split}
                V_1 = \left\{\left(\begin{array}{c}
                    x\\y\\z
                \end{array}\right)\in \bb{R}^3 \left|
                \begin{array}{c}
                    x+y+z=0 \\
                    -x-2y-2z=0 \\
                    x+y+z=0
                \end{array}
                \right.\right\}
                = \cc{L}\left\{\left(\begin{array}{c}
                    0\\1\\-1
                \end{array}\right)\right\}
            \end{split}
        \end{equation*}
        \begin{equation*}
            \begin{split}
                V_2 = \left\{\left(\begin{array}{c}
                    x\\y\\z
                \end{array}\right)\in \bb{R}^3 \left|
                \begin{array}{c}
                    y+z=0 \\
                    -x-3y-2z=0 \\
                    x+y=0
                \end{array}
                \right.\right\}
                = \cc{L}\left\{\left(\begin{array}{c}
                    1\\-1\\1
                \end{array}\right)\right\}
            \end{split}
        \end{equation*}

        Por tanto, tenemos que una base ortogonal de vectores propios es:
        \begin{equation*}
            {B}=\left\{
            \left(\begin{array}{c}
                    1\\-3\\1
                \end{array}\right),
            \left(\begin{array}{c}
                    0\\1\\-1
                \end{array}\right),
            \left(\begin{array}{c}
                    1\\-1\\1
                \end{array}\right)
            \right\} = \{e_1, e_2, e_3\}
        \end{equation*}

        Para que sea ortonormal, calculamos la norma de cada uno de los vectores. Al no ser la base usual ortonormal con esta métrica, hemos de calcular la norma de cada uno de los vectores:
        \begin{equation*}
            g(e_1, e_1) = \left(\begin{array}{ccc}
                    1&-3&1
                \end{array}\right)
                \left(\begin{array}{ccc}
                    3 & 1 & 0 \\
                    1 & 1 & 1 \\
                    0 & 1 & 2
                \end{array}\right)
                \left(\begin{array}{c}
                    1\\-3\\1
                \end{array}\right)
                =
                \left(\begin{array}{ccc}
                    0&-1&-1
                \end{array}\right)
                \left(\begin{array}{c}
                    1\\-3\\1
                \end{array}\right) = 3-1=2
        \end{equation*}
        \begin{equation*}
            g(e_2, e_2) = \left(\begin{array}{ccc}
                    0&1&-1
                \end{array}\right)
                \left(\begin{array}{ccc}
                    3 & 1 & 0 \\
                    1 & 1 & 1 \\
                    0 & 1 & 2
                \end{array}\right)
                \left(\begin{array}{c}
                    0\\1\\-1
                \end{array}\right)
                =
                \left(\begin{array}{ccc}
                    1 & 0 & -1
                \end{array}\right)
                \left(\begin{array}{c}
                    0\\1\\-1
                \end{array}\right) = 1
        \end{equation*}
        \begin{equation*}
            g(e_3, e_3) = \left(\begin{array}{ccc}
                    1&-1&1
                \end{array}\right)
                \left(\begin{array}{ccc}
                    3 & 1 & 0 \\
                    1 & 1 & 1 \\
                    0 & 1 & 2
                \end{array}\right)
                \left(\begin{array}{c}
                    1\\-1\\1
                \end{array}\right)
                =
                \left(\begin{array}{ccc}
                    2 & 1 & 1
                \end{array}\right)
                \left(\begin{array}{c}
                     1\\-1\\1
                \end{array}\right) = 2
        \end{equation*}

        Por tanto, tenemos que la base ortonormal de vectores propios de $f$ es:
        \begin{equation*}
            \bar{B}=\left\{
            \frac{1}{\sqrt{2}} 
            \left(\begin{array}{c}
                    1\\-3\\1
                \end{array}\right),
            \left(\begin{array}{c}
                    0\\1\\-1
                \end{array}\right),
            \frac{1}{\sqrt{2}}
            \left(\begin{array}{c}
                    1\\-1\\1
                \end{array}\right)
            \right\}
        \end{equation*}
    \end{enumerate}
\end{ejercicio}

\begin{ejercicio}[\textbf{3 puntos}] Responde a las siguientes cuestiones:

\begin{enumerate}
    \item Encuentra, si es posible, un endomorfismo diagonalizable de $\bb{R}^3$ que verifique que:
    \begin{equation*}
        Im(f)=\{(x,y,z)\in \bb{R}^3\mid x+y+z=0\}
    \end{equation*}
    
    y otro endomorfismo no diagonalizable de $\bb{R}^3$ que verifique que:
    \begin{equation*}
        Im(g)=\{(x,y,z)\in \bb{R}^3\mid x+y=0,\quad y+z=0\}
    \end{equation*}
    y da su matriz en la base usual de $\bb{R}^3$.\\


    Busco primero $f$. Como tengo que todos los endomorfismos autoadjuntos son diagonalizables, busco que $f$ sea autoadjunto. Como $\cc{B}_u=\{e_1,e_2,e_3\}$ es ortonormal para $(\bb{R}^3, \langle, \rangle)$, si su matriz asociada en dicha base es simétrica, entonces será diagonalizable. Por tanto, a la hora de calcular la base de la imagen de $f$, busco dos vectores que permitan que la matriz sea simétrica.
    \begin{equation*}
        Im(f)=\cc{L}\left\{
        \left(\begin{array}{c}
            1 \\ -1 \\ 0
        \end{array}\right),
        \left(\begin{array}{c}
            -1 \\ 0 \\ 1
        \end{array}\right)
        \right\} :=\cc{L}\{f(e_1), f(e_2)\}
    \end{equation*}

    Para elegir $f(e_3)$, quiero que sea combinación lineal de los dos vectores que generan la base. Por tanto, elijo:
    \begin{equation*}
        f(e_3)=-f(e_2)-f(e_1)
    \end{equation*}

    Por tanto, tengo que:
    \begin{equation*}
        M(f,\cc{B}_u)=
        \left(\begin{array}{ccc}
            1 & -1 & 0 \\
            -1 & 0 & 1 \\
            0 & 1 & -1
        \end{array}\right)
    \end{equation*}
    Tenemos que la imagen de $f$ es la dada y, por ser su matriz respecto de una base ortonormal simétrica es autoadjunto y, por tanto, diagonalizable.

    \vspace{1cm}

    Trabajamos ahora con $g$. Tenemos que:
    \begin{equation*}
        Im(g)=\cc{L}\left\{
        \left(\begin{array}{c}
            1 \\ -1 \\ 1
        \end{array}\right)
        \right\}
    \end{equation*}

    Por tanto, sea $g$ el endomorfismo con la siguiente matriz asociada:
    \begin{equation*}
        G=M(g,\cc{B}_u)=
        \left(\begin{array}{ccc}
            1 & 0 & 0 \\
            -1 & 0 & 0 \\
            1 & 0 & 0
        \end{array}\right)
    \end{equation*}

    Tenemos que la imagen es la dada. Veamos que no es diagonalizable. Su polinomio característico es:
    \begin{equation*}
        P_g(\lambda)=\left|\begin{array}{ccc}
            1-\lambda & 0 & 0 \\
            -1 & -\lambda & 0 \\
            1 & 0 & -\lambda
        \end{array}\right| = \lambda^2(1-\lambda)
    \end{equation*}
    \begin{equation*}
        \dim V_0 = \dim Ker(g) = 3-\dim Im(g)=1
    \end{equation*}

    Por tanto, como tenemos que las multiplicidades algebraicas y geométricas del 0 no coinciden, tenemos que no es diagonalizable.


    \item Sean $(V,g)$ un EVM de dimensión 3 y $\cc{B}$ una base de $V$. Prueba que $g$ es una métrica euclídea si y solo si:
    \begin{equation*}
        a_{33}>0 \qquad \left|\begin{array}{cc}
            a_{22} & a_{23} \\
            a_{32} & a_{33}
        \end{array}\right| >0
        \qquad |A|>0
    \end{equation*}
    donde
    \begin{equation*}
        A=M(g,\cc{B})=\left(\begin{array}{ccc}
            a_{11} & a_{12} & a_{13} \\
            a_{21} & a_{22} & a_{23} \\
            a_{31} & a_{32} & a_{33}
        \end{array}\right)
    \end{equation*}



    \begin{description}
        \item [$\Longrightarrow$)] Suponemos $g$ métrica euclídea. Entonces, $A$ es definida positiva, por lo que sus menores principales son positivos. Por tanto, $|A|>0$. Además, $a_{33}$ es el cuadrado del tercer vector de la base, por  lo que también es positivo. Nos falta ver que:
        \begin{equation*}
            \left|\begin{array}{cc}
                a_{22} & a_{23} \\
                a_{32} & a_{33}
            \end{array}\right| >0
        \end{equation*}


        Sea $\bar{B}=\{e_1,e_2,e_3\}$. Definiendo $\bar{\cc{B}}=\{e_3,e_2,e_1\}$, tenemos que:
        \begin{equation*}
            \bar{A}=M(g,\bar{\cc{B}})=\left(\begin{array}{ccc}
                a_{33} & a_{32} & a_{31} \\
                a_{32} & a_{22} & a_{21} \\
                a_{31} & a_{21} & a_{11}
            \end{array}\right)
        \end{equation*}

        Como $g$ es definida positiva, tengo que los menores principales de $\bar{A}$ también. Por tanto,
        \begin{equation*}
            0<\left|\begin{array}{cc}
                a_{33} & a_{32} \\
                a_{32} & a_{22}
            \end{array}\right|
            \stackrel{F_2\leftrightarrow F_1}{=}-
            \left|\begin{array}{cc}
                a_{32} & a_{22} \\
                a_{33} & a_{32}
            \end{array}\right|
            \stackrel{C_2\leftrightarrow C_1}{=}
            \left|\begin{array}{cc}
                a_{22} & a_{32} \\
                a_{32} & a_{33}
            \end{array}\right|
        \end{equation*}
        Por tanto, se tiene lo pedido.

        \item [$\Longleftarrow$)] Suponemos $g$ que métrica que cumple lo establecido en el enunciado. Entonces, por ser una métrica y supuesto $\cc{B}=\{e_1, e_2, e_3\}$, tenemos que:
        \begin{equation*}
            a_{ij}=a_{ji}=g(e_i,e_j)=g(e_j,e_i)
        \end{equation*}

        Por tanto, definiendo $\bar{\cc{B}}=\{e_3,e_2,e_1\}$, tenemos que:
        \begin{equation*}
            \bar{A}=M(g,\bar{\cc{B}})=\left(\begin{array}{ccc}
                a_{33} & a_{32} & a_{31} \\
                a_{32} & a_{22} & a_{21} \\
                a_{31} & a_{21} & a_{11}
            \end{array}\right)
        \end{equation*}

        Veamos si $\bar{A}$ es definida positiva.
        \begin{equation*}
            |a_{33}|=a_{33}>0
        \end{equation*}
        \begin{equation*}\left|\begin{array}{cc}
                a_{33} & a_{32} \\
                a_{32} & a_{22}
            \end{array}\right|
            \stackrel{F_2\leftrightarrow F_1}{=}-\left|\begin{array}{cc}
                a_{32} & a_{22} \\
                a_{33} & a_{32}
            \end{array}\right|
            \stackrel{C_2\leftrightarrow C_1}{=}
            \left|\begin{array}{cc}
                a_{22} & a_{32} \\
                a_{32} & a_{33}
            \end{array}\right|
            =\left|\begin{array}{cc}
                a_{22} & a_{23} \\
                a_{32} & a_{33}
            \end{array}\right| >0
        \end{equation*}
        \begin{multline*}
            |\bar{A}|=\left|\begin{array}{ccc}
                a_{33} & a_{32} & a_{31} \\
                a_{32} & a_{22} & a_{21} \\
                a_{31} & a_{21} & a_{11}
            \end{array}\right|
            \stackrel{F_3\leftrightarrow F_1}{=}
            -\left|\begin{array}{ccc}
                a_{31} & a_{21} & a_{11} \\
                a_{32} & a_{22} & a_{21} \\
                a_{33} & a_{32} & a_{31}
            \end{array}\right|
            \stackrel{C_3\leftrightarrow C_1}{=}
            \left|\begin{array}{ccc}
                a_{11} & a_{21} & a_{31} \\
                a_{21} & a_{22} & a_{32} \\
                a_{31} & a_{32} & a_{33}
            \end{array}\right|
            =\\= 
            \left|\begin{array}{ccc}
                a_{11} & a_{12} & a_{13} \\
                a_{21} & a_{22} & a_{23} \\
                a_{31} & a_{32} & a_{33}
            \end{array}\right|>0
        \end{multline*}

        Por tanto, tenemos que $\bar{A}$ es definida positiva y, como $\bar{A}\sim_c A$, tenemos que $A$ también es definida positiva, por lo que $g$ es una métrica euclídea.
    \end{description}

    
\end{enumerate}
    
\end{ejercicio}



\begin{ejercicio}
    En $(\bb{R}^3, g)$ encuentra, si es posible, una isometría $f$ que lleve el subespacio $U$ en el subespacio $W$, donde:
    \begin{equation*}
        U=\{x,y,z\in \bb{R}^3 \mid x-z=0\} \hspace{2cm}
        W=\{x,y,z\in \bb{R}^3 \mid z=0\}
    \end{equation*}

    Si es posible, da $M(f,\cc{B}_u)$, clasifica y describe la isometría $f$.\\

    
    Sea $\cc{B}_u=\{e_1, e_2, e_3\}$.
    \begin{observacion}
    Gráficamente, deducimos que se trata de un giro de ángulo $\theta=\frac{\pi}{4}$ sobre la recta $L=\cc{L}\{e_2\}$. Esto nos ayuda en las suposiciones que hacemos.
    \end{observacion}
    
    \vspace{1cm}


    Tenemos que:
    \begin{equation*}
        U=\cc{L}\left\{
        \left(
        \begin{array}{c}
            0 \\ 1 \\ 0 
        \end{array}
        \right),
        \frac{1}{\sqrt{2}}
        \left(
        \begin{array}{c}
            1 \\ 0 \\ 1
        \end{array}
        \right)
        \right\} = \left\{e_2, \frac{1}{\sqrt{2}}(e_1+e_3)\right\}
    \end{equation*}
    \begin{equation*}
        W = \cc{L}\left\{e_2, e_1\right\}
    \end{equation*}

    Definimos las condiciones para que $f(U)=W$ de la siguiente forma:
    \begin{equation*}
        \left\{\begin{array}{l}
             \displaystyle f(e_2)=e_2  \\
             \\
             \displaystyle \frac{1}{\sqrt{2}}[f(e_1)+f(e_3)]=e_1
        \end{array}\right.
    \end{equation*}

    Además, imponemos como condición que $f(e_3)=\frac{1}{\sqrt{2}}[e_1+e_3]$. Por tanto, tenemos que:
    \begin{equation*}
        \left\{\begin{array}{l}
             \displaystyle f(e_1)=\sqrt{2}e_1-\frac{1}{\sqrt{2}}[e_1+e_3]
             = \frac{1}{\sqrt{2}}[e_1-e_3]
             \\ \\
             \displaystyle f(e_2)=e_2  \\
             \\
             \displaystyle f(e_3)=\frac{1}{\sqrt{2}}[e_1+e_3]
        \end{array}\right.
    \end{equation*}

    Por tanto, tenemos que isometría buscada tiene como matriz en la base usual:
    \begin{equation*}
        M(f,\cc{B}_u)=\frac{1}{\sqrt{2}}\left(\begin{array}{ccc}
            1 & 0 & 1 \\
            0 & \sqrt{2} & 0 \\
            -1 & 0 & 1 \\
        \end{array}\right) \in O(3)
    \end{equation*}

    Tenemos que se trata de una isometría, ya que su matriz respecto de la base usual (que es ortonormal) es ortogonal. Esto se debe a que sus filas son base ortonormal de $\bb{R}^3$.

    Como además lleva una base de $U$ en una base de $W$, tenemos que $f(U)=W$.
\end{ejercicio}

\end{document}