\documentclass[12pt]{article}

% Idioma y codificación
\usepackage[spanish, es-tabla]{babel}       %es-tabla para que se titule "Tabla"
\usepackage[utf8]{inputenc}

% Márgenes
\usepackage[a4paper,top=3cm,bottom=2.5cm,left=3cm,right=3cm]{geometry}

% Comentarios de bloque
\usepackage{verbatim}

% Paquetes de links
\usepackage[hidelinks]{hyperref}    % Permite enlaces
\usepackage{url}                    % redirecciona a la web

% Más opciones para enumeraciones
\usepackage{enumitem}

% Personalizar la portada
\usepackage{titling}

% Paquetes de tablas
\usepackage{multirow}


%------------------------------------------------------------------------

%Paquetes de figuras
\usepackage{caption}
\usepackage{subcaption} % Figuras al lado de otras
\usepackage{float}      % Poner figuras en el sitio indicado H.


% Paquetes de imágenes
\usepackage{graphicx}       % Paquete para añadir imágenes
\usepackage{transparent}    % Para manejar la opacidad de las figuras

% Paquete para usar colores
\usepackage[dvipsnames]{xcolor}
\usepackage{pagecolor}      % Para cambiar el color de la página

% Habilita tamaños de fuente mayores
\usepackage{fix-cm}

% Para los gráficos
\usepackage{tikz}

% Para poder situar los nodos en los grafos
\usetikzlibrary{positioning}


%------------------------------------------------------------------------

% Paquetes de matemáticas
\usepackage{mathtools, amsfonts, amssymb, mathrsfs}
\usepackage[makeroom]{cancel}     % Simplificar tachando
\usepackage{polynom}    % Divisiones y Ruffini
\usepackage{units} % Para poner fracciones diagonales con \nicefrac

\usepackage{pgfplots}   %Representar funciones
\pgfplotsset{compat=1.18}  % Versión 1.18

\usepackage{tikz-cd}    % Para usar diagramas de composiciones
\usetikzlibrary{calc}   % Para usar cálculo de coordenadas en tikz

%Definición de teoremas, etc.
\usepackage{amsthm}
%\swapnumbers   % Intercambia la posición del texto y de la numeración

\theoremstyle{plain}

\makeatletter
\@ifclassloaded{article}{
  \newtheorem{teo}{Teorema}[section]
}{
  \newtheorem{teo}{Teorema}[chapter]  % Se resetea en cada chapter
}
\makeatother

\newtheorem{coro}{Corolario}[teo]           % Se resetea en cada teorema
\newtheorem{prop}[teo]{Proposición}         % Usa el mismo contador que teorema
\newtheorem{lema}[teo]{Lema}                % Usa el mismo contador que teorema

\theoremstyle{remark}
\newtheorem*{observacion}{Observación}

\theoremstyle{definition}

\makeatletter
\@ifclassloaded{article}{
  \newtheorem{definicion}{Definición} [section]     % Se resetea en cada chapter
}{
  \newtheorem{definicion}{Definición} [chapter]     % Se resetea en cada chapter
}
\makeatother

\newtheorem*{notacion}{Notación}
\newtheorem*{ejemplo}{Ejemplo}
\newtheorem*{ejercicio*}{Ejercicio}             % No numerado
\newtheorem{ejercicio}{Ejercicio} [section]     % Se resetea en cada section


% Modificar el formato de la numeración del teorema "ejercicio"
\renewcommand{\theejercicio}{%
  \ifnum\value{section}=0 % Si no se ha iniciado ninguna sección
    \arabic{ejercicio}% Solo mostrar el número de ejercicio
  \else
    \thesection.\arabic{ejercicio}% Mostrar número de sección y número de ejercicio
  \fi
}


% \renewcommand\qedsymbol{$\blacksquare$}         % Cambiar símbolo QED
%------------------------------------------------------------------------

% Paquetes para encabezados
\usepackage{fancyhdr}
\pagestyle{fancy}
\fancyhf{}

\newcommand{\helv}{ % Modificación tamaño de letra
\fontfamily{}\fontsize{12}{12}\selectfont}
\setlength{\headheight}{15pt} % Amplía el tamaño del índice


%\usepackage{lastpage}   % Referenciar última pag   \pageref{LastPage}
\fancyfoot[C]{\thepage}

%------------------------------------------------------------------------

% Conseguir que no ponga "Capítulo 1". Sino solo "1."
\makeatletter
\@ifclassloaded{book}{
  \renewcommand{\chaptermark}[1]{\markboth{\thechapter.\ #1}{}} % En el encabezado
    
  \renewcommand{\@makechapterhead}[1]{%
  \vspace*{50\p@}%
  {\parindent \z@ \raggedright \normalfont
    \ifnum \c@secnumdepth >\m@ne
      \huge\bfseries \thechapter.\hspace{1em}\ignorespaces
    \fi
    \interlinepenalty\@M
    \Huge \bfseries #1\par\nobreak
    \vskip 40\p@
  }}
}
\makeatother

%------------------------------------------------------------------------
% Paquetes de cógido
\usepackage{minted}
\renewcommand\listingscaption{Código fuente}

\usepackage{fancyvrb}
% Personaliza el tamaño de los números de línea
\renewcommand{\theFancyVerbLine}{\small\arabic{FancyVerbLine}}

% Estilo para C++
\newminted{cpp}{
    frame=lines,
    framesep=2mm,
    baselinestretch=1.2,
    linenos,
    escapeinside=||
}

% para minted
\definecolor{LightGray}{rgb}{0.95,0.95,0.92}
\setminted{
    linenos=true,
    stepnumber=5,
    numberfirstline=true,
    autogobble,
    breaklines=true,
    breakautoindent=true,
    breaksymbolleft=,
    breaksymbolright=,
    breaksymbolindentleft=0pt,
    breaksymbolindentright=0pt,
    breaksymbolsepleft=0pt,
    breaksymbolsepright=0pt,
    fontsize=\footnotesize,
    bgcolor=LightGray,
    numbersep=10pt
}


\usepackage{listings} % Para incluir código desde un archivo

\renewcommand\lstlistingname{Código Fuente}
\renewcommand\lstlistlistingname{Índice de Códigos Fuente}

% Definir colores
\definecolor{vscodepurple}{rgb}{0.5,0,0.5}
\definecolor{vscodeblue}{rgb}{0,0,0.8}
\definecolor{vscodegreen}{rgb}{0,0.5,0}
\definecolor{vscodegray}{rgb}{0.5,0.5,0.5}
\definecolor{vscodebackground}{rgb}{0.97,0.97,0.97}
\definecolor{vscodelightgray}{rgb}{0.9,0.9,0.9}

% Configuración para el estilo de C similar a VSCode
\lstdefinestyle{vscode_C}{
  backgroundcolor=\color{vscodebackground},
  commentstyle=\color{vscodegreen},
  keywordstyle=\color{vscodeblue},
  numberstyle=\tiny\color{vscodegray},
  stringstyle=\color{vscodepurple},
  basicstyle=\scriptsize\ttfamily,
  breakatwhitespace=false,
  breaklines=true,
  captionpos=b,
  keepspaces=true,
  numbers=left,
  numbersep=5pt,
  showspaces=false,
  showstringspaces=false,
  showtabs=false,
  tabsize=2,
  frame=tb,
  framerule=0pt,
  aboveskip=10pt,
  belowskip=10pt,
  xleftmargin=10pt,
  xrightmargin=10pt,
  framexleftmargin=10pt,
  framexrightmargin=10pt,
  framesep=0pt,
  rulecolor=\color{vscodelightgray},
  backgroundcolor=\color{vscodebackground},
}

%------------------------------------------------------------------------

% Comandos definidos
\newcommand{\bb}[1]{\mathbb{#1}}
\newcommand{\cc}[1]{\mathcal{#1}}

% I prefer the slanted \leq
\let\oldleq\leq % save them in case they're every wanted
\let\oldgeq\geq
\renewcommand{\leq}{\leqslant}
\renewcommand{\geq}{\geqslant}

% Si y solo si
\newcommand{\sii}{\iff}

% Letras griegas
\newcommand{\eps}{\epsilon}
\newcommand{\veps}{\varepsilon}
\newcommand{\lm}{\lambda}

\newcommand{\ol}{\overline}
\newcommand{\ul}{\underline}
\newcommand{\wt}{\widetilde}
\newcommand{\wh}{\widehat}

\let\oldvec\vec
\renewcommand{\vec}{\overrightarrow}

% Derivadas parciales
\newcommand{\del}[2]{\frac{\partial #1}{\partial #2}}
\newcommand{\Del}[3]{\frac{\partial^{#1} #2}{\partial #3^{#1}}}
\newcommand{\deld}[2]{\dfrac{\partial #1}{\partial #2}}
\newcommand{\Deld}[3]{\dfrac{\partial^{#1} #2}{\partial #3^{#1}}}


\newcommand{\AstIg}{\stackrel{(\ast)}{=}}
\newcommand{\Hop}{\stackrel{L'H\hat{o}pital}{=}}

\newcommand{\red}[1]{{\color{red}#1}} % Para integrales, destacar los cambios.

% Método de integración
\newcommand{\MetInt}[2]{
    \left[\begin{array}{c}
        #1 \\ #2
    \end{array}\right]
}

% Declarar aplicaciones
% 1. Nombre aplicación
% 2. Dominio
% 3. Codominio
% 4. Variable
% 5. Imagen de la variable
\newcommand{\Func}[5]{
    \begin{equation*}
        \begin{array}{rrll}
            #1:& #2 & \longrightarrow & #3\\
               & #4 & \longmapsto & #5
        \end{array}
    \end{equation*}
}

%------------------------------------------------------------------------

\let\oldRe\Re % save them in case they're every wanted
\let\oldIm\Im
\renewcommand{\Re}{\operatorname{Re}} % redefine them
\renewcommand{\Im}{\operatorname{Im}}
\DeclareMathOperator{\Log}{Log}
\DeclareMathOperator{\Arg}{Arg}
\DeclareMathOperator{\ord}{ord}
\DeclareMathOperator{\Ind}{Ind}
\DeclareMathOperator{\Fr}{Fr}
\DeclareMathOperator{\Res}{Res}
\begin{document}

    % 1. Foto de fondo
    % 2. Título
    % 3. Encabezado Izquierdo
    % 4. Color de fondo
    % 5. Coord x del titulo
    % 6. Coord y del titulo
    % 7. Fecha

    
    % 1. Foto de fondo
% 2. Título
% 3. Encabezado Izquierdo
% 4. Color de fondo
% 5. Coord x del titulo
% 6. Coord y del titulo
% 7. Fecha

\newcommand{\portada}[7]{

    \portadaBase{#1}{#2}{#3}{#4}{#5}{#6}{#7}
    \portadaBook{#1}{#2}{#3}{#4}{#5}{#6}{#7}
}

\newcommand{\portadaExamen}[7]{

    \portadaBase{#1}{#2}{#3}{#4}{#5}{#6}{#7}
    \portadaArticle{#1}{#2}{#3}{#4}{#5}{#6}{#7}
}




\newcommand{\portadaBase}[7]{

    % Tiene la portada principal y la licencia Creative Commons
    
    % 1. Foto de fondo
    % 2. Título
    % 3. Encabezado Izquierdo
    % 4. Color de fondo
    % 5. Coord x del titulo
    % 6. Coord y del titulo
    % 7. Fecha
    
    
    \thispagestyle{empty}               % Sin encabezado ni pie de página
    \newgeometry{margin=0cm}        % Márgenes nulos para la primera página
    
    
    % Encabezado
    \fancyhead[L]{\helv #3}
    \fancyhead[R]{\helv \nouppercase{\leftmark}}
    
    
    \pagecolor{#4}        % Color de fondo para la portada
    
    \begin{figure}[p]
        \centering
        \transparent{0.3}           % Opacidad del 30% para la imagen
        
        \includegraphics[width=\paperwidth, keepaspectratio]{assets/#1}
    
        \begin{tikzpicture}[remember picture, overlay]
            \node[anchor=north west, text=white, opacity=1, font=\fontsize{60}{90}\selectfont\bfseries\sffamily, align=left] at (#5, #6) {#2};
            
            \node[anchor=south east, text=white, opacity=1, font=\fontsize{12}{18}\selectfont\sffamily, align=right] at (9.7, 3) {\textbf{\href{https://losdeldgiim.github.io/}{Los Del DGIIM}}};
            
            \node[anchor=south east, text=white, opacity=1, font=\fontsize{12}{15}\selectfont\sffamily, align=right] at (9.7, 1.8) {Doble Grado en Ingeniería Informática y Matemáticas\\Universidad de Granada};
        \end{tikzpicture}
    \end{figure}
    
    
    \restoregeometry        % Restaurar márgenes normales para las páginas subsiguientes
    \pagecolor{white}       % Restaurar el color de página
    
    
    \newpage
    \thispagestyle{empty}               % Sin encabezado ni pie de página
    \begin{tikzpicture}[remember picture, overlay]
        \node[anchor=south west, inner sep=3cm] at (current page.south west) {
            \begin{minipage}{0.5\paperwidth}
                \href{https://creativecommons.org/licenses/by-nc-nd/4.0/}{
                    \includegraphics[height=2cm]{assets/Licencia.png}
                }\vspace{1cm}\\
                Esta obra está bajo una
                \href{https://creativecommons.org/licenses/by-nc-nd/4.0/}{
                    Licencia Creative Commons Atribución-NoComercial-SinDerivadas 4.0 Internacional (CC BY-NC-ND 4.0).
                }\\
    
                Eres libre de compartir y redistribuir el contenido de esta obra en cualquier medio o formato, siempre y cuando des el crédito adecuado a los autores originales y no persigas fines comerciales. 
            \end{minipage}
        };
    \end{tikzpicture}
    
    
    
    % 1. Foto de fondo
    % 2. Título
    % 3. Encabezado Izquierdo
    % 4. Color de fondo
    % 5. Coord x del titulo
    % 6. Coord y del titulo
    % 7. Fecha


}


\newcommand{\portadaBook}[7]{

    % 1. Foto de fondo
    % 2. Título
    % 3. Encabezado Izquierdo
    % 4. Color de fondo
    % 5. Coord x del titulo
    % 6. Coord y del titulo
    % 7. Fecha

    % Personaliza el formato del título
    \pretitle{\begin{center}\bfseries\fontsize{42}{56}\selectfont}
    \posttitle{\par\end{center}\vspace{2em}}
    
    % Personaliza el formato del autor
    \preauthor{\begin{center}\Large}
    \postauthor{\par\end{center}\vfill}
    
    % Personaliza el formato de la fecha
    \predate{\begin{center}\huge}
    \postdate{\par\end{center}\vspace{2em}}
    
    \title{#2}
    \author{\href{https://losdeldgiim.github.io/}{Los Del DGIIM}}
    \date{Granada, #7}
    \maketitle
    
    \tableofcontents
}




\newcommand{\portadaArticle}[7]{

    % 1. Foto de fondo
    % 2. Título
    % 3. Encabezado Izquierdo
    % 4. Color de fondo
    % 5. Coord x del titulo
    % 6. Coord y del titulo
    % 7. Fecha

    % Personaliza el formato del título
    \pretitle{\begin{center}\bfseries\fontsize{42}{56}\selectfont}
    \posttitle{\par\end{center}\vspace{2em}}
    
    % Personaliza el formato del autor
    \preauthor{\begin{center}\Large}
    \postauthor{\par\end{center}\vspace{3em}}
    
    % Personaliza el formato de la fecha
    \predate{\begin{center}\huge}
    \postdate{\par\end{center}\vspace{5em}}
    
    \title{#2}
    \author{\href{https://losdeldgiim.github.io/}{Los Del DGIIM}}
    \date{Granada, #7}
    \thispagestyle{empty}               % Sin encabezado ni pie de página
    \maketitle
    \vfill
}
    \portadaExamen{ffccA4.jpg}{Variable Compleja I\\Examen XIII}{Variable Compleja I. Examen XIII}{MidnightBlue}{-9.5}{28}{2024-2025}{Arturo Olivares Martos}

    \begin{description}
        \item[Asignatura] Variable Compleja I.
        \item[Curso Académico] 2024-25.
        \item[Grado] Grado en Matemáticas y Doble Grado en Matemáticas y Física.
        \item[Grupo] Único.
        \item[Profesor] Javier Merí de la Maza.
        \item[Descripción] Convocatoria Extraordinaria.
        \item[Fecha] 10 de Febrero de 2025.
        \item[Duración] 3.5 horas.
    \end{description}
    \newpage

    \begin{ejercicio}[2.5 puntos]
         Sean $S$ un conjunto finito de puntos en un dominio $\Omega$ homológicamente conexo y $f$ una función holomorfa en $\Omega \setminus S$. Prueba que $f$ tiene una primitiva en $\Omega \setminus S$ si y solo si
        \[
            \Res(f, w) = 0, \quad \forall w \in S.
        \]
    \end{ejercicio}

    \begin{ejercicio}[2.5 puntos]
        Sea $f : \ol{D}(0, 1) \to \bb{C}$ continua en $\ol{D}(0, 1)$ y holomorfa en $D(0, 1)$ de modo que $f(z) \in \bb{R}$ para cada $z \in \bb{C}$ con $|z| = 1$. Prueba que $f$ es constante.
    \end{ejercicio}

    \begin{ejercicio}[2.5 puntos]
        Demuestra que no puede existir una función $f$ entera verificando
        \[
            |f(z)| > |z| + |\sen(z)|, \quad \forall z \in \bb{C}.
        \]
    \end{ejercicio}

    \begin{ejercicio}[2.5 puntos]
        Sean $f, g$ holomorfas en $\bb{C} \setminus \{0\}$ verificando $f(n) = n^2 g(n)$ para cada $n \in \bb{N}$. Supongamos que existen $\lim\limits_{z \to \infty} f(z) \in \bb{C}$ y $\lim\limits_{z \to \infty} z^2 g(z) \in \bb{C}$. Prueba que
        \[
            f(z) = z^2 g(z) \quad \text{para cada } z \in \bb{C} \setminus \{0\}.
        \]
    \end{ejercicio}




    \newpage
    \setcounter{ejercicio}{0}


    \begin{ejercicio}[2.5 puntos]
         Sean $S$ un conjunto finito de puntos en un dominio $\Omega$ homológicamente conexo y $f$ una función holomorfa en $\Omega \setminus S$. Prueba que $f$ tiene una primitiva en $\Omega \setminus S$ si y solo si
        \[
            \Res(f, w) = 0, \quad \forall w \in S.
        \]
        \begin{description}
            \item[$\Longrightarrow$)] Supongamos que $f$ tiene una primitiva $F$ en $\Omega \setminus S$. Fijado $w\in S$, queremos calcular:
            \[
                \Res(f, w) = \frac{1}{2\pi i} \int_{C(w, r)} f(z) \, dz,
            \]
            para cualquier $r\in \bb{R}^+$ tal que $\ol{D}(w,r)\cap S=\{w\}$. Consideramos por tanto $r\in \bb{R}^+$ suficientemente pequeño de manera que $\ol{D}(w,r)\cap S=\{w\}$. Como $C(w, r)$ es un camino cerrado en $\Omega \setminus S$ y $f$ admite una primitiva en $\Omega \setminus S$, tenemos que
            \[
                \int_{C(w, r)} f(z) \, dz = 0.
            \]
            Por tanto, $\Res(f, w) = 0$.

            \item[$\Longleftarrow$)] Supongamos que $\Res(f, w) = 0$ para cada $w \in S$. Sea $\gamma$ un camino cerrado en $\Omega \setminus S$. Como $S\subset \Omega$ es finito, tenemos que $S'\cap \Omega=\emptyset$. Como $\Omega$ es homológicamente conexo, podemos aplicar el Teorema de los Residuos sobre el camino $\gamma$, obteniendo:
            \[
                \int_{\gamma} f(z) \, dz = 2\pi i \sum_{w\in S} \Res(f, w) = 0.
            \]
            Por la caracterización de las funciones que admiten primitivas, como $\gamma$ era arbitrario, tenemos que $f$ admite una primitiva en $\Omega \setminus S$.
        \end{description}
    \end{ejercicio}

    \begin{ejercicio}[2.5 puntos]
        Sea $f : \ol{D}(0, 1) \to \bb{C}$ continua en $\ol{D}(0, 1)$ y holomorfa en $D(0, 1)$ de modo que $f(z) \in \bb{R}$ para cada $z \in \bb{C}$ con $|z| = 1$. Prueba que $f$ es constante.\\

        Supongamos que $f$ no es constante, y llegaremos a una contradicción. Consideramos la parte imaginaria de $f$, que es una aplicación continua. Como $\ol{D}(0,1)$ es compacto, existen $z_1,z_2\in \ol{D}(0,1)$ tales que:
        \[
            \Im f(z_1) = \min\{\Im f(z) : z \in \ol{D}(0, 1)\}\leq \max\{\Im f(z) : z \in \ol{D}(0, 1)\} = \Im f(z_2).
        \]

        Como $\bb{T}\subset \ol{D}(0,1)$ y $\Im f(\bb{T})=\{0\}$ por hipótesis, tenemos que:
        \begin{equation*}
            \Im f(z_1) = \min\{\Im f(z) : z \in \ol{D}(0, 1)\}\leq 0\leq \max\{\Im f(z) : z \in \ol{D}(0, 1)\} = \Im f(z_2).
        \end{equation*}

        Por otro lado, como $f\in \cc{H}(D(0,1))$ no es constante, por el Teorema de la Aplicación Abierta tenemos que $f(D(0,1))$ es un conjunto abierto de $\bb{C}$. En particular, dado $z_0\in f(D(0,1))$, existe $r\in \bb{R}^+$ tal que $D(z_0,r)\subset f(D(0,1))$. Por tanto, deducimos que $\Im f(z_1)<\Im f(z_2)$.\\
        
        Como $\Im f(z_1)\leq 0\leq \Im f(z_2)$, $\exists j\in \{1,2\}$ tal que $\Im f(z_j) \neq 0$, y por la hipótesis dada se tiene que $z_j\in D(0,1)$.\\

        Aplicamos ahora de nuevo el Teorema de la Aplicación Abierta, obteniendo que $f(D(0,1))$ es un conjunto abierto de $\bb{C}$ con $f(z_j)\in f(D(0,1))$ y $\Im f(z_j) \neq 0$. Por tanto, existe $r\in \bb{R}^+$ tal que:
        \[
            D(f(z_j), r) \subset f(D(0,1)).
        \]
        De aquí, deducimos que:
        \begin{align*}
            \Im f(z_j) &\neq \max\{\Im f(z) : z \in \ol{D}(0, 1)\} \\
            \Im f(z_j) &\neq \min\{\Im f(z) : z \in \ol{D}(0, 1)\}
        \end{align*}

        Por tanto, hemos llegado a una contradicción, y concluimos que $f$ es constante.
    \end{ejercicio}

    \begin{ejercicio}[2.5 puntos]
        Demuestra que no puede existir una función $f$ entera verificando
        \[
            |f(z)| > |z| + |\sen(z)|, \quad \forall z \in \bb{C}.
        \]

        Supongamos que existe una función $f$ entera verificando la desigualdad dada. Por tanto:
        \begin{equation*}
            |f(z)| > |z|\qquad \forall z \in \bb{C}.
        \end{equation*}

        Por tanto:
        \begin{equation*}
            \lim_{z \to \infty} f(z) = \infty.
        \end{equation*}

        Por tanto, por el Corolario del Corolario del Teorema de Casorati-Weierstrass, tenemos que $f$ es un polinomio.
        Por otro lado, tenemos que:
        \begin{equation*}
            |\sen(z)| < |f(z)|\qquad \forall z \in \bb{C}.
        \end{equation*}

        Por tanto, como $\sen(z)$ es una función entera con crecimiento subpolinómico, tenemos que $\sen(z)$ es un polinomio (clara contradicción). Por tanto, no puede existir una función $f$ entera verificando la desigualdad dada.
    \end{ejercicio}

    \begin{ejercicio}[2.5 puntos]
        Sean $f, g$ holomorfas en $\bb{C} \setminus \{0\}$ verificando $f(n) = n^2 g(n)$ para cada $n \in \bb{N}$. Supongamos que existen $\lim\limits_{z \to \infty} f(z) \in \bb{C}$ y $\lim\limits_{z \to \infty} z^2 g(z) \in \bb{C}$. Prueba que
        \[
            f(z) = z^2 g(z) \quad \text{para cada } z \in \bb{C} \setminus \{0\}.
        \]

        Trabajar con los límites en $+\infty$ no es tan sencillo, puesto que $\{n\in \bb{N}\}$ no tiene puntos de acumulación en $\bb{C}$. Por tanto, consideramos el siguiente conjunto:
        \begin{equation*}
            A=\left\{\dfrac{1}{n} : n \in \bb{N}\right\}
        \end{equation*}

        Definimos las siguientes funciones:
        \Func{\wt{h_1}}{\bb{C}^*}{\bb{C}}{z}{f\left(\frac{1}{z}\right)}
        \Func{\wt{h_2}}{\bb{C}^*}{\bb{C}}{z}{\dfrac{1}{z^2}\cdot g\left(\frac{1}{z}\right)}

        Tenemos que:
        \begin{align*}
            \lim_{z \to 0} \wt{h_1}(z) &= \lim_{z \to 0} f\left(\frac{1}{z}\right) = \lim_{z \to \infty} f(z) = L_1 \in \bb{C},\\
            \lim_{z \to 0} \wt{h_2}(z) &= \lim_{z \to 0} \dfrac{1}{z^2}\cdot g\left(\frac{1}{z}\right) = \lim_{z \to \infty} z^2 g(z) = L_2 \in \bb{C}.
        \end{align*}

        Definimos por tanto las extensiones $h_1, h_2 : \bb{C} \to \bb{C}$ de $\wt{h_1}, \wt{h_2}$ respectivamente, de modo que:
        \begin{align*}
            h_1(0) &= L_1,\\
            h_2(0) &= L_2.
        \end{align*}

        De esta forma, tenemos que $h_1,h_2\in \cc{H}(\bb{C}^*)$ y continuas en $\bb{C}$. Por tanto, por Teorema de Extensión de Riemman, tenemos que $h_1,h_2\in \cc{H}(\bb{C})$.\\

        Tenemos que $A'=\{0\}\subset \bb{C}$, y:
        \begin{equation*}
            h_1\left(\frac{1}{n}\right) = f(n) = n^2 g(n) = h_2\left(\frac{1}{n}\right), \quad \forall n \in \bb{N}.
        \end{equation*}

        Por tanto, por el Principio de Identidad de funciones holomorfas, tenemos que:
        \begin{equation*}
            h_1(z) = h_2(z), \quad \forall z \in \bb{C}.
        \end{equation*}

        Para cada $z \in \bb{C}^*$, evaluamos dicha igualdad en $\nicefrac{1}{z}$, obteniendo:
        \begin{equation*}
            f(z) = h_1\left(\frac{1}{z}\right) = h_2\left(\frac{1}{z}\right) = z^2g(z), \quad \forall z \in \bb{C}^*.
        \end{equation*}
        como queríamos demostrar.
    \end{ejercicio}
\end{document}