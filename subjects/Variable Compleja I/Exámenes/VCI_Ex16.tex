\documentclass[12pt]{article}

% Idioma y codificación
\usepackage[spanish, es-tabla]{babel}       %es-tabla para que se titule "Tabla"
\usepackage[utf8]{inputenc}

% Márgenes
\usepackage[a4paper,top=3cm,bottom=2.5cm,left=3cm,right=3cm]{geometry}

% Comentarios de bloque
\usepackage{verbatim}

% Paquetes de links
\usepackage[hidelinks]{hyperref}    % Permite enlaces
\usepackage{url}                    % redirecciona a la web

% Más opciones para enumeraciones
\usepackage{enumitem}

% Personalizar la portada
\usepackage{titling}

% Paquetes de tablas
\usepackage{multirow}


%------------------------------------------------------------------------

%Paquetes de figuras
\usepackage{caption}
\usepackage{subcaption} % Figuras al lado de otras
\usepackage{float}      % Poner figuras en el sitio indicado H.


% Paquetes de imágenes
\usepackage{graphicx}       % Paquete para añadir imágenes
\usepackage{transparent}    % Para manejar la opacidad de las figuras

% Paquete para usar colores
\usepackage[dvipsnames]{xcolor}
\usepackage{pagecolor}      % Para cambiar el color de la página

% Habilita tamaños de fuente mayores
\usepackage{fix-cm}

% Para los gráficos
\usepackage{tikz}

% Para poder situar los nodos en los grafos
\usetikzlibrary{positioning}


%------------------------------------------------------------------------

% Paquetes de matemáticas
\usepackage{mathtools, amsfonts, amssymb, mathrsfs}
\usepackage[makeroom]{cancel}     % Simplificar tachando
\usepackage{polynom}    % Divisiones y Ruffini
\usepackage{units} % Para poner fracciones diagonales con \nicefrac

\usepackage{pgfplots}   %Representar funciones
\pgfplotsset{compat=1.18}  % Versión 1.18

\usepackage{tikz-cd}    % Para usar diagramas de composiciones
\usetikzlibrary{calc}   % Para usar cálculo de coordenadas en tikz

%Definición de teoremas, etc.
\usepackage{amsthm}
%\swapnumbers   % Intercambia la posición del texto y de la numeración

\theoremstyle{plain}

\makeatletter
\@ifclassloaded{article}{
  \newtheorem{teo}{Teorema}[section]
}{
  \newtheorem{teo}{Teorema}[chapter]  % Se resetea en cada chapter
}
\makeatother

\newtheorem{coro}{Corolario}[teo]           % Se resetea en cada teorema
\newtheorem{prop}[teo]{Proposición}         % Usa el mismo contador que teorema
\newtheorem{lema}[teo]{Lema}                % Usa el mismo contador que teorema

\theoremstyle{remark}
\newtheorem*{observacion}{Observación}

\theoremstyle{definition}

\makeatletter
\@ifclassloaded{article}{
  \newtheorem{definicion}{Definición} [section]     % Se resetea en cada chapter
}{
  \newtheorem{definicion}{Definición} [chapter]     % Se resetea en cada chapter
}
\makeatother

\newtheorem*{notacion}{Notación}
\newtheorem*{ejemplo}{Ejemplo}
\newtheorem*{ejercicio*}{Ejercicio}             % No numerado
\newtheorem{ejercicio}{Ejercicio} [section]     % Se resetea en cada section


% Modificar el formato de la numeración del teorema "ejercicio"
\renewcommand{\theejercicio}{%
  \ifnum\value{section}=0 % Si no se ha iniciado ninguna sección
    \arabic{ejercicio}% Solo mostrar el número de ejercicio
  \else
    \thesection.\arabic{ejercicio}% Mostrar número de sección y número de ejercicio
  \fi
}


% \renewcommand\qedsymbol{$\blacksquare$}         % Cambiar símbolo QED
%------------------------------------------------------------------------

% Paquetes para encabezados
\usepackage{fancyhdr}
\pagestyle{fancy}
\fancyhf{}

\newcommand{\helv}{ % Modificación tamaño de letra
\fontfamily{}\fontsize{12}{12}\selectfont}
\setlength{\headheight}{15pt} % Amplía el tamaño del índice


%\usepackage{lastpage}   % Referenciar última pag   \pageref{LastPage}
\fancyfoot[C]{\thepage}

%------------------------------------------------------------------------

% Conseguir que no ponga "Capítulo 1". Sino solo "1."
\makeatletter
\@ifclassloaded{book}{
  \renewcommand{\chaptermark}[1]{\markboth{\thechapter.\ #1}{}} % En el encabezado
    
  \renewcommand{\@makechapterhead}[1]{%
  \vspace*{50\p@}%
  {\parindent \z@ \raggedright \normalfont
    \ifnum \c@secnumdepth >\m@ne
      \huge\bfseries \thechapter.\hspace{1em}\ignorespaces
    \fi
    \interlinepenalty\@M
    \Huge \bfseries #1\par\nobreak
    \vskip 40\p@
  }}
}
\makeatother

%------------------------------------------------------------------------
% Paquetes de cógido
\usepackage{minted}
\renewcommand\listingscaption{Código fuente}

\usepackage{fancyvrb}
% Personaliza el tamaño de los números de línea
\renewcommand{\theFancyVerbLine}{\small\arabic{FancyVerbLine}}

% Estilo para C++
\newminted{cpp}{
    frame=lines,
    framesep=2mm,
    baselinestretch=1.2,
    linenos,
    escapeinside=||
}

% para minted
\definecolor{LightGray}{rgb}{0.95,0.95,0.92}
\setminted{
    linenos=true,
    stepnumber=5,
    numberfirstline=true,
    autogobble,
    breaklines=true,
    breakautoindent=true,
    breaksymbolleft=,
    breaksymbolright=,
    breaksymbolindentleft=0pt,
    breaksymbolindentright=0pt,
    breaksymbolsepleft=0pt,
    breaksymbolsepright=0pt,
    fontsize=\footnotesize,
    bgcolor=LightGray,
    numbersep=10pt
}


\usepackage{listings} % Para incluir código desde un archivo

\renewcommand\lstlistingname{Código Fuente}
\renewcommand\lstlistlistingname{Índice de Códigos Fuente}

% Definir colores
\definecolor{vscodepurple}{rgb}{0.5,0,0.5}
\definecolor{vscodeblue}{rgb}{0,0,0.8}
\definecolor{vscodegreen}{rgb}{0,0.5,0}
\definecolor{vscodegray}{rgb}{0.5,0.5,0.5}
\definecolor{vscodebackground}{rgb}{0.97,0.97,0.97}
\definecolor{vscodelightgray}{rgb}{0.9,0.9,0.9}

% Configuración para el estilo de C similar a VSCode
\lstdefinestyle{vscode_C}{
  backgroundcolor=\color{vscodebackground},
  commentstyle=\color{vscodegreen},
  keywordstyle=\color{vscodeblue},
  numberstyle=\tiny\color{vscodegray},
  stringstyle=\color{vscodepurple},
  basicstyle=\scriptsize\ttfamily,
  breakatwhitespace=false,
  breaklines=true,
  captionpos=b,
  keepspaces=true,
  numbers=left,
  numbersep=5pt,
  showspaces=false,
  showstringspaces=false,
  showtabs=false,
  tabsize=2,
  frame=tb,
  framerule=0pt,
  aboveskip=10pt,
  belowskip=10pt,
  xleftmargin=10pt,
  xrightmargin=10pt,
  framexleftmargin=10pt,
  framexrightmargin=10pt,
  framesep=0pt,
  rulecolor=\color{vscodelightgray},
  backgroundcolor=\color{vscodebackground},
}

%------------------------------------------------------------------------

% Comandos definidos
\newcommand{\bb}[1]{\mathbb{#1}}
\newcommand{\cc}[1]{\mathcal{#1}}

% I prefer the slanted \leq
\let\oldleq\leq % save them in case they're every wanted
\let\oldgeq\geq
\renewcommand{\leq}{\leqslant}
\renewcommand{\geq}{\geqslant}

% Si y solo si
\newcommand{\sii}{\iff}

% Letras griegas
\newcommand{\eps}{\epsilon}
\newcommand{\veps}{\varepsilon}
\newcommand{\lm}{\lambda}

\newcommand{\ol}{\overline}
\newcommand{\ul}{\underline}
\newcommand{\wt}{\widetilde}
\newcommand{\wh}{\widehat}

\let\oldvec\vec
\renewcommand{\vec}{\overrightarrow}

% Derivadas parciales
\newcommand{\del}[2]{\frac{\partial #1}{\partial #2}}
\newcommand{\Del}[3]{\frac{\partial^{#1} #2}{\partial #3^{#1}}}
\newcommand{\deld}[2]{\dfrac{\partial #1}{\partial #2}}
\newcommand{\Deld}[3]{\dfrac{\partial^{#1} #2}{\partial #3^{#1}}}


\newcommand{\AstIg}{\stackrel{(\ast)}{=}}
\newcommand{\Hop}{\stackrel{L'H\hat{o}pital}{=}}

\newcommand{\red}[1]{{\color{red}#1}} % Para integrales, destacar los cambios.

% Método de integración
\newcommand{\MetInt}[2]{
    \left[\begin{array}{c}
        #1 \\ #2
    \end{array}\right]
}

% Declarar aplicaciones
% 1. Nombre aplicación
% 2. Dominio
% 3. Codominio
% 4. Variable
% 5. Imagen de la variable
\newcommand{\Func}[5]{
    \begin{equation*}
        \begin{array}{rrll}
            #1:& #2 & \longrightarrow & #3\\
               & #4 & \longmapsto & #5
        \end{array}
    \end{equation*}
}

%------------------------------------------------------------------------

\let\oldRe\Re % save them in case they're every wanted
\let\oldIm\Im
\renewcommand{\Re}{\operatorname{Re}} % redefine them
\renewcommand{\Im}{\operatorname{Im}}
\DeclareMathOperator{\Log}{Log}
\DeclareMathOperator{\Arg}{Arg}
\DeclareMathOperator{\ord}{ord}
\DeclareMathOperator{\Ind}{Ind}
\DeclareMathOperator{\Fr}{Fr}
\DeclareMathOperator{\Res}{Res}


\usetikzlibrary{arrows.meta, decorations.markings} % Cargar las bibliotecas necesarias

% Configuración para las flechas
\tikzset{
    arrow at 1/3/.style={postaction={decorate},
        decoration={markings, mark=at position 0.33 with {\arrow{Stealth}}}},
    arrow at 2/3/.style={postaction={decorate},
        decoration={markings, mark=at position 0.66 with {\arrow{Stealth}}}}
}


\begin{document}

    % 1. Foto de fondo
    % 2. Título
    % 3. Encabezado Izquierdo
    % 4. Color de fondo
    % 5. Coord x del titulo
    % 6. Coord y del titulo
    % 7. Fecha

    
    % 1. Foto de fondo
% 2. Título
% 3. Encabezado Izquierdo
% 4. Color de fondo
% 5. Coord x del titulo
% 6. Coord y del titulo
% 7. Fecha

\newcommand{\portada}[7]{

    \portadaBase{#1}{#2}{#3}{#4}{#5}{#6}{#7}
    \portadaBook{#1}{#2}{#3}{#4}{#5}{#6}{#7}
}

\newcommand{\portadaExamen}[7]{

    \portadaBase{#1}{#2}{#3}{#4}{#5}{#6}{#7}
    \portadaArticle{#1}{#2}{#3}{#4}{#5}{#6}{#7}
}




\newcommand{\portadaBase}[7]{

    % Tiene la portada principal y la licencia Creative Commons
    
    % 1. Foto de fondo
    % 2. Título
    % 3. Encabezado Izquierdo
    % 4. Color de fondo
    % 5. Coord x del titulo
    % 6. Coord y del titulo
    % 7. Fecha
    
    
    \thispagestyle{empty}               % Sin encabezado ni pie de página
    \newgeometry{margin=0cm}        % Márgenes nulos para la primera página
    
    
    % Encabezado
    \fancyhead[L]{\helv #3}
    \fancyhead[R]{\helv \nouppercase{\leftmark}}
    
    
    \pagecolor{#4}        % Color de fondo para la portada
    
    \begin{figure}[p]
        \centering
        \transparent{0.3}           % Opacidad del 30% para la imagen
        
        \includegraphics[width=\paperwidth, keepaspectratio]{assets/#1}
    
        \begin{tikzpicture}[remember picture, overlay]
            \node[anchor=north west, text=white, opacity=1, font=\fontsize{60}{90}\selectfont\bfseries\sffamily, align=left] at (#5, #6) {#2};
            
            \node[anchor=south east, text=white, opacity=1, font=\fontsize{12}{18}\selectfont\sffamily, align=right] at (9.7, 3) {\textbf{\href{https://losdeldgiim.github.io/}{Los Del DGIIM}}};
            
            \node[anchor=south east, text=white, opacity=1, font=\fontsize{12}{15}\selectfont\sffamily, align=right] at (9.7, 1.8) {Doble Grado en Ingeniería Informática y Matemáticas\\Universidad de Granada};
        \end{tikzpicture}
    \end{figure}
    
    
    \restoregeometry        % Restaurar márgenes normales para las páginas subsiguientes
    \pagecolor{white}       % Restaurar el color de página
    
    
    \newpage
    \thispagestyle{empty}               % Sin encabezado ni pie de página
    \begin{tikzpicture}[remember picture, overlay]
        \node[anchor=south west, inner sep=3cm] at (current page.south west) {
            \begin{minipage}{0.5\paperwidth}
                \href{https://creativecommons.org/licenses/by-nc-nd/4.0/}{
                    \includegraphics[height=2cm]{assets/Licencia.png}
                }\vspace{1cm}\\
                Esta obra está bajo una
                \href{https://creativecommons.org/licenses/by-nc-nd/4.0/}{
                    Licencia Creative Commons Atribución-NoComercial-SinDerivadas 4.0 Internacional (CC BY-NC-ND 4.0).
                }\\
    
                Eres libre de compartir y redistribuir el contenido de esta obra en cualquier medio o formato, siempre y cuando des el crédito adecuado a los autores originales y no persigas fines comerciales. 
            \end{minipage}
        };
    \end{tikzpicture}
    
    
    
    % 1. Foto de fondo
    % 2. Título
    % 3. Encabezado Izquierdo
    % 4. Color de fondo
    % 5. Coord x del titulo
    % 6. Coord y del titulo
    % 7. Fecha


}


\newcommand{\portadaBook}[7]{

    % 1. Foto de fondo
    % 2. Título
    % 3. Encabezado Izquierdo
    % 4. Color de fondo
    % 5. Coord x del titulo
    % 6. Coord y del titulo
    % 7. Fecha

    % Personaliza el formato del título
    \pretitle{\begin{center}\bfseries\fontsize{42}{56}\selectfont}
    \posttitle{\par\end{center}\vspace{2em}}
    
    % Personaliza el formato del autor
    \preauthor{\begin{center}\Large}
    \postauthor{\par\end{center}\vfill}
    
    % Personaliza el formato de la fecha
    \predate{\begin{center}\huge}
    \postdate{\par\end{center}\vspace{2em}}
    
    \title{#2}
    \author{\href{https://losdeldgiim.github.io/}{Los Del DGIIM}}
    \date{Granada, #7}
    \maketitle
    
    \tableofcontents
}




\newcommand{\portadaArticle}[7]{

    % 1. Foto de fondo
    % 2. Título
    % 3. Encabezado Izquierdo
    % 4. Color de fondo
    % 5. Coord x del titulo
    % 6. Coord y del titulo
    % 7. Fecha

    % Personaliza el formato del título
    \pretitle{\begin{center}\bfseries\fontsize{42}{56}\selectfont}
    \posttitle{\par\end{center}\vspace{2em}}
    
    % Personaliza el formato del autor
    \preauthor{\begin{center}\Large}
    \postauthor{\par\end{center}\vspace{3em}}
    
    % Personaliza el formato de la fecha
    \predate{\begin{center}\huge}
    \postdate{\par\end{center}\vspace{5em}}
    
    \title{#2}
    \author{\href{https://losdeldgiim.github.io/}{Los Del DGIIM}}
    \date{Granada, #7}
    \thispagestyle{empty}               % Sin encabezado ni pie de página
    \maketitle
    \vfill
}
    \portadaExamen{ffccA4.jpg}{Variable Compleja I\\Examen XVI}{Variable Compleja I. Examen XVI}{MidnightBlue}{-9.5}{28}{2024-2025}{Arturo Olivares Martos}

    \begin{description}
        \item[Asignatura] Variable Compleja I.
        \item[Curso Académico] 2021-22.
        \item[Grado] Doble Grado en Ingeniería Informática y Matemáticas.
        \item[Grupo] Único.
        \item[Profesor] Javier Merí de la Maza.
        \item[Descripción] Convocatoria Ordinaria.
        \item[Fecha] 15 de Junio de 2022.
        \item[Duración] 3.5 horas.
    \end{description}
    \newpage

    \begin{ejercicio}[2.5 puntos]
        Integrando una conveniente función sobre un camino cerrado que recorra la frontera del conjunto $\{z \in \bb{C} : \varepsilon < |z| < R, 0 < \arg z < \nicefrac{\pi}{2}\}$, con $0 < \varepsilon < 1 < R$, calcular la integral:
        \begin{equation*}
            \int_0^{+\infty} \frac{\log(x)}{1 + x^4} \, dx.
        \end{equation*}
    \end{ejercicio}

    \begin{ejercicio}[2.5 puntos]
        Para cada $n \in \bb{N}$, sea $f_n : \bb{C} \to \bb{C}$ la función dada por
        \begin{equation*}
            f_n(z) = \int_n^{n+1} e^{z-t} \sen(tn + z^2) \, dt \quad \forall z \in \bb{C}.
        \end{equation*}
        Demostrar que:
        \begin{enumerate}
            \item $f_n$ es holomorfa en $\bb{C}$.
            \item La serie de funciones $\sum\limits_{n\geq 1} f_n$ converge uniformemente en $\bb{C}$ y su suma es una función entera.
        \end{enumerate}
    \end{ejercicio}

    \begin{ejercicio}[2.5 puntos]
        Sea $\Omega$ un dominio y $f \in \cc{H}(\Omega)$. Demostrar que, si la función $\Im f$ tiene un extremo relativo en un punto de $\Omega$, entonces $f$ es constante.
    \end{ejercicio}

    \begin{ejercicio}[2.5 puntos]
        Sean $f, g \in \cc{H}(\bb{C})$ de modo que
        $$f\left(g\left(\frac{1}{n}\right)\right) = \frac{1}{n^3}$$
        para todo $n \in \bb{N}$.
        Probar que una de las funciones es un polinomio de grado uno y que la otra es un polinomio de grado tres.
    \end{ejercicio}

    \newpage
    \setcounter{ejercicio}{0} % Reiniciar el contador de secciones
    

    \begin{ejercicio}[2.5 puntos]\label{ej:14.15}
        Integrando una conveniente función sobre un camino cerrado que recorra la frontera del conjunto $\{z \in \bb{C} : \varepsilon < |z| < R, 0 < \arg z < \nicefrac{\pi}{2}\}$, con $0 < \varepsilon < 1 < R$, calcular la integral:
        \begin{equation*}
            \int_0^{+\infty} \frac{\log(x)}{1 + x^4} \, dx.
        \end{equation*}

        Calculamos primero los puntos donde se anula el denominador de la función a integrar:
    \begin{align*}
        1 + z^4 &= 0 \implies z^4 = -1 \implies z\in \left\{e^{i\left(\frac{\pi}{4} + k\frac{\pi}{2}\right)} : k\in \{0, 1, 2, 3\}\right\}
        = \left\{e^{i\frac{\pi}{4}}, e^{i\frac{3\pi}{4}}, e^{i\frac{5\pi}{4}}, e^{i\frac{7\pi}{4}}\right\}.
    \end{align*}

    Para cada $k\in \{0, 1, 2, 3\}$, definimos por simplicidad:
    \begin{align*}
        z_k &= e^{i\left(\frac{\pi}{4} + k\frac{\pi}{2}\right)}
    \end{align*}

    Sea por tanto $A = \{z_k : k\in \{0, 1, 2, 3\}\}$. Definimos la función:
    \Func{f}{\bb{C}\setminus A}{\bb{C}}{z}{\frac{\log(z)}{1 + z^4}}

    Notemos que $f\in \cc{H}(\bb{C}\setminus A)$, y que $A'=\emptyset$, por lo que podemos aplicar el Teorema de los Residuos. Como $\bb{C}$ es homológicamente conexo, podemos aplicar el Teorema de los Residuos para cualquier ciclo $\Sigma$ en $\bb{C}\setminus A$. Para todo $R>1$ y $\veps\in \left]0, 1\right[$, consideramos el siguiente ciclo:
    \begin{align*}
        \Sigma_{\veps, R} &= -\gamma_{\veps} + [\veps, R] + \sigma_R - [i\veps, iR]
    \end{align*}
    representada en la Figura~\ref{fig:ej:14.15}, donde:
    \Func{\gamma_{\veps}}{[0, \nicefrac{\pi}{2}]}{\bb{C}}{t}{\veps e^{it}}
    \Func{[\veps, R]}{[\veps, R]}{\bb{C}}{t}{t}
    \Func{\sigma_R}{[0, \nicefrac{\pi}{2}]}{\bb{C}}{t}{Re^{it}}
    \Func{[i\veps, iR]}{[\veps, R]}{\bb{C}}{t}{i t}
    \begin{figure}
        \centering
        \begin{tikzpicture}
            \begin{axis}[
                axis lines=middle,
                xlabel={$x$},
                ylabel={$y$},
                xtick=\empty,
                ytick=\empty,
                xmin=-2, xmax=2,
                ymin=-1.5, ymax=2.5,
                axis equal image,
                clip=false,
            ]
                \def\R{1.5}
                \def\eps{0.5}
                \def\n{4}

                \def\x{0}
                \def\y{0}

                % Polos
                % Representamos los polos en bucle
                \foreach \k in {0,...,\numexpr\n-1\relax} {
                    \pgfmathsetmacro{\angle}{180*(1+2*\k)/\n}
                    \pgfmathsetmacro{\x}{cos(\angle)}
                    \pgfmathsetmacro{\y}{sin(\angle)}
                    \pgfmathsetmacro{\labelx}{\x + 0.2*cos(\angle)} % Offset for label
                    \pgfmathsetmacro{\labely}{\y + 0.2*sin(\angle)} % Offset for label
                    \edef\temp{\noexpand\node[font=\noexpand\footnotesize] at (axis cs:\labelx,\labely) {$e^{i\frac{\pi + 2 \cdot \k \cdot \pi}{\n}}$};}
                    \edef\temp2{\noexpand\node[font=\noexpand\footnotesize] at (axis cs:\labelx,\labely) {$z_{\k}$};}
                    \addplot[
                        only marks,
                        mark=*,
                        mark options={fill=red},
                    ] coordinates {(\x, \y)};
                    %\temp
                    \temp2
                }



                % [-eps, R]
                \draw[thick, blue, arrow at 1/3, arrow at 2/3] (\eps, 0) -- (\R, 0)
                    node[pos=0.7, below] {$[\varepsilon, R]$};

                % sigma_r
                \draw[thick, blue, arrow at 1/3, arrow at 2/3] (\R, 0) arc[start angle=0, end angle=90, radius=\R]
                    node[midway, below left, yshift=-1em, xshift=-1em] {$\sigma_R$};

                % [iR, i eps]
                \draw[thick, blue, arrow at 1/3, arrow at 2/3] (0, \R) -- (0,\eps)
                    node[pos=0, left] {$-[i \veps, i R]$};

                % gamma_eps
                \draw[thick, blue, arrow at 2/3] (0, \eps) arc[start angle=90, end angle=0, radius=\eps];

                % Marca los puntos de unión
                \addplot[only marks, mark=*, mark options={fill=blue}, samples=1] coordinates {(\eps, 0) (\R, 0) (0, \R) (0, \eps)};

                % Radio veps
                \draw[-Stealth, teal, dashed] (0, 0) -- (\eps*0.7071, \eps*0.7071)
                    node[pos=1, above right] {$\varepsilon$};
            \end{axis}
        \end{tikzpicture}
        \caption{Ciclo de integración $\Sigma_{\veps, R}$ del Ejercicio~\ref{ej:14.15}.}
        \label{fig:ej:14.15}
    \end{figure}

    Por el Teorema de los Residuos, tenemos que:
    \begin{align*}
        \int_{\Sigma_{\veps, R}} f(z) \, dz &= 2\pi i\sum_{w_0\in A}\Res(f,w_0)\Ind_{\Sigma_{\veps, R}}(w_0).
    \end{align*}
    Calculemos ahora los índices de los polos. Por cómo hemos definido el ciclo, tenemos que:
    \begin{align*}
        \Ind_{\Sigma_{\veps, R}}(z_0) &= 1\\
        \Ind_{\Sigma_{\veps, R}}(z_k) &= 0 \quad \text{para todo } k\in \{1, 2, 3\}.
    \end{align*}

    Por tanto, tenemos que:
    \begin{align*}
        \int_{\Sigma_{\veps, R}} f(z) \, dz &= 2\pi i\Res(f, z_0).
    \end{align*}

    Antes de calcular el residuo, calculemos las integrales resultantes. Tenemos que:
    \begin{align*}
        \int_{[\veps, R]} f(z) \, dz &= \int_{\veps}^{R} \frac{\log(z)}{1 + z^4} \, dz
    \end{align*}
    Tomando límite con $\veps\to 0^+$, y $R\to +\infty$, tenemos lo buscado.\\

    Veamos ahora qué ocurre con la integral sobre el segmento $[i\veps, iR]$:
    \begin{align*}
        \int_{[i\veps, iR]} f(z) \, dz &= i\int_{\veps}^{R} f(it) \, dt
        = i\int_{\veps}^{R} \frac{\log(it)}{1 + (it)^4} \, dt
        = i\int_{\veps}^{R} \frac{\ln t + i\arg(it)}{1 + t^4} \, dt
        =\\&= i\int_{\veps}^{R} \frac{\ln t + i\cdot \nicefrac{\pi}{2}}{1 + t^4} \, dt
        = -\frac{\pi}{2}\cdot \int_{\veps}^{R} \frac{1}{1 + t^4} \, dt + i\int_{\veps}^{R} \frac{\ln t}{1 + t^4} \, dt.
    \end{align*}

    Veamos ahora qué ocurre con la integral sobre la curva $\gamma_{\veps}$:
    \begin{align*}
        \left|\int_{\gamma_{\veps}} f(z) \, dz\right| &\leq \frac{\pi}{2}\cdot \veps \cdot \sup\left\{\left|\dfrac{\log(z)}{1 + z^4}\right| : z\in \gamma_{\veps}^*\right\}
    \end{align*}
    Para todo $z\in \gamma_{\veps}$, tenemos que:
    \begin{align*}
        |1+z^4| &\geq \left||1| - |z^4|\right| = \left|1 - \veps^4\right| = 1 - \veps^4\\
        |\log(z)| &= |\ln|z| + |\arg(z)|\leq \ln\veps + \frac{\pi}{2}
    \end{align*}

    Por tanto, tenemos que:
    \begin{align*}
        \left|\int_{\gamma_{\veps}} f(z) \, dz\right| &\leq \frac{\pi}{2}\cdot \veps \cdot \frac{\ln\veps + \frac{\pi}{2}}{1 - \veps^4}.
    \end{align*}

    Como esta expresión es válida para cualquier $\veps\in \left]0, 1\right[$, podemos hacer $\veps \to 0^+$ y tenemos que:
    \begin{align*}
        \lim_{\veps\to 0^+} \int_{\gamma_{\veps}} f(z) \, dz &= 0.
    \end{align*}

    Veamos ahora qué ocurre con la integral sobre $\sigma_R$:
    \begin{align*}
        \left|\int_{\sigma_R} f(z) \, dz\right| &\leq \frac{\pi}{2}\cdot R \cdot \sup\left\{\left|\dfrac{\log(z)}{1 + z^4}\right| : z\in \sigma_R^*\right\}
    \end{align*}
    Para todo $z\in \sigma_R$, tenemos que:
    \begin{align*}
        |1+z^4| &\geq \left||1| - |z^4|\right| = \left|1 - R^4\right| = R^4 - 1\\
        |\log(z)| &= |\ln|z| + |\arg(z)|\leq \ln R + \frac{\pi}{2}
    \end{align*}
    Por tanto, tenemos que:
    \begin{align*}
        \left|\int_{\sigma_R} f(z) \, dz\right| &\leq \frac{\pi}{2}\cdot R \cdot \frac{\ln R + \frac{\pi}{2}}{R^4 - 1}.
    \end{align*}
    Como esta expresión es válida para cualquier $R > 1$, podemos hacer $R \to +\infty$ y tenemos que:
    \begin{align*}
        \lim_{R\to+\infty} \int_{\sigma_R} f(z) \, dz &= 0.
    \end{align*}

    Uniendo todas las integrales que hemos calculado, tenemos que:
    \begin{align*}
        2\pi i\Res(f, z_0) &= \int_0^{\infty} \frac{\log(t)}{1 + t^4} \, dt + \frac{\pi}{2}\cdot \int_{0}^{\infty} \frac{1}{1 + t^4} \, dt - i\int_0^{\infty} \frac{\log(t)}{1 + t^4} \, dt
    \end{align*}

    Calculemos ahora el residuo en el punto $z_0 = e^{i\frac{\pi}{4}}$ aplicando la Regla de L'Hôpital:
    \begin{align*}
        \lim_{z\to e^{i\frac{\pi}{4}}} \left(z - e^{i\frac{\pi}{4}}\right)f(z) &= \log\left(e^{i\frac{\pi}{4}}\right) \lim_{z\to e^{i\frac{\pi}{4}}} \dfrac{1}{4z^3}
        = \dfrac{\log\left(e^{i\frac{\pi}{4}}\right)}{4\left(e^{i\frac{\pi}{4}}\right)^3}
        = \dfrac{i\cdot \nicefrac{\pi}{4}}{4 e^{i\frac{3\pi}{4}}}
        = \dfrac{i\pi e^{i\frac{7\pi}{4}}}{16}
        =\\&= \dfrac{i\pi}{16}\left(\dfrac{\sqrt{2}}{2} - i\frac{\sqrt{2}}{2}\right)
        = \dfrac{i\pi\sqrt{2}}{32}(1-i)
    \end{align*}

    Por tanto, sabemos que $f$ tiene un polo simple en $z_0 = e^{i\frac{\pi}{4}}$, y que:
    \begin{align*}
        \Res\left(f, e^{i\frac{\pi}{4}}\right) &= \dfrac{i\pi\sqrt{2}}{32}(1-i)
    \end{align*}

    Por tanto, tenemos que:
    \begin{align*}
        2\pi i\left(\dfrac{i\pi\sqrt{2}}{32}(1-i)\right) &= \int_0^{+\infty} \frac{\log(t)}{1 + t^4} \, dt + \frac{\pi}{2}\cdot \int_{0}^{+\infty} \frac{1}{1 + t^4} \, dt - i\int_0^{+\infty} \frac{\log(t)}{1 + t^4} \, dt\\
        -\dfrac{\pi^2\sqrt{2}}{16}(1-i) &= \int_0^{+\infty} \frac{\log(t)}{1 + t^4} \, dt + \frac{\pi}{2}\cdot \int_{0}^{+\infty} \frac{1}{1 + t^4} \, dt - i\int_0^{+\infty} \frac{\log(t)}{1 + t^4} \, dt
    \end{align*}

    Igualando las partes imaginarias, tenemos que:
    \begin{align*}
        \int_0^{+\infty} \frac{\log(t)}{1 + t^4} \, dt
        = -\dfrac{\pi^2\sqrt{2}}{16}
    \end{align*}
    \end{ejercicio}

    \begin{ejercicio}[2.5 puntos]
        Para cada $n \in \bb{N}$, sea $f_n : \bb{C} \to \bb{C}$ la función dada por
        \begin{equation*}
            f_n(z) = \int_n^{n+1} e^{z-t} \sen(tn + z^2) \, dt \quad \forall z \in \bb{C}.
        \end{equation*}
        Demostrar que:
        \begin{enumerate}
            \item $f_n$ es holomorfa en $\bb{C}$.
            
            Definimos $\Phi$ como sigue:
            \Func{\Phi}{[n, n+1]\times \bb{C}}{\bb{C}}{(t, z)}{e^{z-t} \sen(tn + z^2)}

            Tenemos claramente que $\Phi$ es continua en $[n, n+1]\times \bb{C}$, y para cada $t\in [n, n+1]$, la función $z\mapsto \Phi(t, z)$ es holomorfa en $\bb{C}$. Por tanto, por el Teorema de Holomorfía de Integrales dependientes de un parámetro, se concluye que $f_n\in \cc{H}(\bb{C})$.
            \item La serie de funciones $\sum\limits_{n\geq 1} f_n$ converge uniformemente en $\bb{C}$ y su suma es una función entera.
            
            Sea $K\subset \bb{C}$ compacto. Para todo $z\in K$ y $n\in \bb{N}$, tenemos que:
            \begin{align*}
                |f_n(z)| &= \left|\int_n^{n+1} e^{z-t} \sen(tn + z^2) \, dt\right|\\
                &\leq \sup\left\{\left|e^{z-t} \sen(tn + z^2)\right| : t\in [n, n+1]\right\}
            \end{align*}
            Hacemos uso de que, para cada $t\in [n, n+1]$, tenemos que:
            \begin{align*}
                |e^{z-t}| &= e^{\Re(z) - t} \leq e^{\Re(z) - n}\\
                |\sen(tn + z^2)| &\leq |\sen(tn)\cos(z^2)| + |\cos(tn)\sen(z^2)| \leq |\cos(z^2)| + |\sen(z^2)|
            \end{align*}

            Además, como $K$ es compacto y las funciones parte real, seno y coseno son continuas, tenemos que $\exists M_1, M_2\in \bb{R}$ tales que:
            \begin{align*}
                M_1 &= \max\left\{\Re(z) : z\in K\right\}\\
                M_2 &= \max\left\{|\cos(z^2)| + |\sen(z^2)| : z\in K\right\}
            \end{align*}

            Por tanto, tenemos que:
            \begin{align*}
                |f_n(z)| &\leq e^{M_1 - n} M_2
            \end{align*}

            Veamos ahora que la serie de las cotas converge. Para ello, previamente vemos que la siguiente serie converge:
            \begin{equation*}
                \sum_{n\geq 1} e^{-n} = \sum_{n\geq 1} \left(\frac{1}{e}\right)^n
            \end{equation*}
            Como $\nicefrac{1}{e} < 1$, la serie anterior converge. Por tanto, tenemos que la serie de las cotas converge, y por el Test de Weierstrass, la serie de funciones $\sum\limits_{n\geq 1} f_n$ converge uniformemente en $K$.\\

            Por el Teorema de Convergencia de Weierstrass, la suma de la serie de funciones es una función holomorfa en $\bb{C}$.
        \end{enumerate}
    \end{ejercicio}

    \begin{ejercicio}[2.5 puntos]
        Sea $\Omega$ un dominio y $f \in \cc{H}(\Omega)$. Demostrar que, si la función $\Im f$ tiene un extremo relativo en un punto de $\Omega$, entonces $f$ es constante.\\

        Definimos la siguiente función:
        \Func{g}{\Omega}{\bb{R}}{z}{e^{-if(z)}}.

        Como $f$ es holomorfa, $g$ es holomorfa en $\Omega$. Calculemos su módulo:
        \begin{align*}
            |g(z)| &= |e^{-if(z)}| = e^{\Im f(z)}.
        \end{align*}

        Como $\Im f$ tiene un extremo relativo en un punto $z_0\in \Omega$, como la exponencial real es estrictamente creciente, entonces $|g|$ tiene un extremo relativo en $z_0$.
        \begin{itemize}
            \item Si $\Im f$ tiene un máximo relativo en $z_0$, entonces $|g|$ tiene un máximo relativo en $z_0$. Por el principio del módulo máximo, $g$ es constante en $\Omega$.
            \item Si $\Im f$ tiene un mínimo relativo en $z_0$, entonces $|g|$ tiene un mínimo relativo en $z_0$. Por el principio del módulo mínimo, como la exponencial compleja no se anula, $g$ es constante en $\Omega$.
        \end{itemize}

        En cualquier caso, $g$ es constante en $\Omega$. Sea por tanto $\alpha\in \bb{C}^*$ tal que:
        \begin{equation*}
            g(z) = e^{-if(z)} = \alpha \quad \forall z\in \Omega.
        \end{equation*}

        Por tanto, se tiene que:
        \begin{equation*}
            f(z) \in i\Log(\alpha)\qquad \forall z\in \Omega.
        \end{equation*}

        Como además $f$ es continua y dicho conjunto es discreto, se tiene que $\exists \beta\in i\Log(\alpha)$ tal que:
        \begin{equation*}
            f(z) = \beta \quad \forall z\in \Omega.
        \end{equation*}
        Por tanto, $f$ es constante en $\Omega$.
    \end{ejercicio}

    \begin{ejercicio}[2.5 puntos]
        Sean $f, g \in \cc{H}(\bb{C})$ de modo que
        $$f\left(g\left(\frac{1}{n}\right)\right) = \frac{1}{n^3}$$
        para todo $n \in \bb{N}$.
        Probar que una de las funciones es un polinomio de grado uno y que la otra es un polinomio de grado tres.\\

        Definimos el siguiente conjunto:
        \begin{equation*}
            A = \left\{\dfrac{1}{n} : n\in \bb{N}\right\}
        \end{equation*}

        Como $A'=\{0\}\subset \bb{C}$, podemos aplicar el Pincipio de Identidad, y deducir que:
        \begin{equation*}
            f\left(g(z)\right) = z^3\qquad \forall z\in \bb{C}
        \end{equation*}

        Supongamos que $g$ es una función entera no polinómica. Por el Corolario del Teorema de Casorati, $\exists \{z_n\}_{n\in \bb{N}}\subset \bb{C}$ con $\{z_n\}\to \infty$ tal que:
        \begin{equation*}
            \{g(z_n)\} \to 0.
        \end{equation*}

        Ese hecho, junto con la continuidad de $f$, nos permite deducir que:
        \begin{align*}
            \{f(g(z_n))\}\to f(0).
        \end{align*}

        Por otro lado, $\{z_n\}\to \infty$, junto con la continuidad de $f, g$ y que $f(g(z))=z^3$, nos permite deducir que:
        \begin{align*}
            \{f(g(z_n))\}\to \infty.
        \end{align*}

        Por tanto, llegamos a que la sueción $\{f(g(z_n))\}$ es a la vez convergente y divergente, lo que es una contradicción. Por tanto, $g$ es un polinomio.\\

        Suponemos ahora que $f$ no es un polinomio. Por el Corolario del Teorema de Casorati, $\exists \{w_n\}_{n\in \bb{N}}\subset \bb{C}$ con $\{w_n\}\to \infty$ tal que:
        \begin{equation*}
            \{f(w_n)\} \to 0.
        \end{equation*}

        Ahora, haciendo uso de que $g$ es sobreyectiva por ser un polinomio (gracias al Teorema Fundamental del Álgebra), podemos encontrar una sucesión $\{z_n\}_{n\in \bb{N}}\subset \bb{C}$ tal que:
        \begin{equation*}
            g(z_n) = w_n\qquad \forall n\in \bb{N}.
        \end{equation*}
        Por tanto, tenemos que:
        \begin{align*}
            \{f(g(z_n))\} &= \{f(w_n)\} \to 0.
        \end{align*}

        Por otro lado, supongamos que $\{z_n\}\to \alpha\in \bb{C}$. Entonces, por la continuidad de $g$ tenemos que:
        \begin{align*}
            \{g(z_n)\} &= \{w_n\} \to g(\alpha)
        \end{align*}
        En contradicción con que $\{w_n\}\to \infty$. Por tanto, $\{z_n\}\to \infty$. Por la continuidad de $f, g$ y que $f(g(z))=z^3$, tenemos que:
        \begin{align*}
            \{f(g(z_n))\} &= \{z_n^3\} \to \infty.
        \end{align*}

        Por tanto, llegamos a que la sueción $\{f(g(z_n))\}$ es a la vez convergente y divergente, lo que es una contradicción. Por tanto, $f$ es un polinomio.\\

        Por tanto, $f$ y $g$ son polinomios. Como $f(g(z))=z^3$, tenemos que:
        \begin{align*}
            \deg(f)\cdot \deg(g) &= 3
            \Longrightarrow
            \{\deg(f), \deg(g)\} = \{1, 3\}.
        \end{align*}
        Por tanto, una de las funciones es un polinomio de grado uno y la otra es un polinomio de grado tres.
    \end{ejercicio}


\end{document}