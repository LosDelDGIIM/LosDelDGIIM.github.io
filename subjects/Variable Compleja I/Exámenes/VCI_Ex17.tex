\documentclass[12pt]{article}

% Idioma y codificación
\usepackage[spanish, es-tabla]{babel}       %es-tabla para que se titule "Tabla"
\usepackage[utf8]{inputenc}

% Márgenes
\usepackage[a4paper,top=3cm,bottom=2.5cm,left=3cm,right=3cm]{geometry}

% Comentarios de bloque
\usepackage{verbatim}

% Paquetes de links
\usepackage[hidelinks]{hyperref}    % Permite enlaces
\usepackage{url}                    % redirecciona a la web

% Más opciones para enumeraciones
\usepackage{enumitem}

% Personalizar la portada
\usepackage{titling}

% Paquetes de tablas
\usepackage{multirow}


%------------------------------------------------------------------------

%Paquetes de figuras
\usepackage{caption}
\usepackage{subcaption} % Figuras al lado de otras
\usepackage{float}      % Poner figuras en el sitio indicado H.


% Paquetes de imágenes
\usepackage{graphicx}       % Paquete para añadir imágenes
\usepackage{transparent}    % Para manejar la opacidad de las figuras

% Paquete para usar colores
\usepackage[dvipsnames]{xcolor}
\usepackage{pagecolor}      % Para cambiar el color de la página

% Habilita tamaños de fuente mayores
\usepackage{fix-cm}

% Para los gráficos
\usepackage{tikz}

% Para poder situar los nodos en los grafos
\usetikzlibrary{positioning}


%------------------------------------------------------------------------

% Paquetes de matemáticas
\usepackage{mathtools, amsfonts, amssymb, mathrsfs}
\usepackage[makeroom]{cancel}     % Simplificar tachando
\usepackage{polynom}    % Divisiones y Ruffini
\usepackage{units} % Para poner fracciones diagonales con \nicefrac

\usepackage{pgfplots}   %Representar funciones
\pgfplotsset{compat=1.18}  % Versión 1.18

\usepackage{tikz-cd}    % Para usar diagramas de composiciones
\usetikzlibrary{calc}   % Para usar cálculo de coordenadas en tikz

%Definición de teoremas, etc.
\usepackage{amsthm}
%\swapnumbers   % Intercambia la posición del texto y de la numeración

\theoremstyle{plain}

\makeatletter
\@ifclassloaded{article}{
  \newtheorem{teo}{Teorema}[section]
}{
  \newtheorem{teo}{Teorema}[chapter]  % Se resetea en cada chapter
}
\makeatother

\newtheorem{coro}{Corolario}[teo]           % Se resetea en cada teorema
\newtheorem{prop}[teo]{Proposición}         % Usa el mismo contador que teorema
\newtheorem{lema}[teo]{Lema}                % Usa el mismo contador que teorema

\theoremstyle{remark}
\newtheorem*{observacion}{Observación}

\theoremstyle{definition}

\makeatletter
\@ifclassloaded{article}{
  \newtheorem{definicion}{Definición} [section]     % Se resetea en cada chapter
}{
  \newtheorem{definicion}{Definición} [chapter]     % Se resetea en cada chapter
}
\makeatother

\newtheorem*{notacion}{Notación}
\newtheorem*{ejemplo}{Ejemplo}
\newtheorem*{ejercicio*}{Ejercicio}             % No numerado
\newtheorem{ejercicio}{Ejercicio} [section]     % Se resetea en cada section


% Modificar el formato de la numeración del teorema "ejercicio"
\renewcommand{\theejercicio}{%
  \ifnum\value{section}=0 % Si no se ha iniciado ninguna sección
    \arabic{ejercicio}% Solo mostrar el número de ejercicio
  \else
    \thesection.\arabic{ejercicio}% Mostrar número de sección y número de ejercicio
  \fi
}


% \renewcommand\qedsymbol{$\blacksquare$}         % Cambiar símbolo QED
%------------------------------------------------------------------------

% Paquetes para encabezados
\usepackage{fancyhdr}
\pagestyle{fancy}
\fancyhf{}

\newcommand{\helv}{ % Modificación tamaño de letra
\fontfamily{}\fontsize{12}{12}\selectfont}
\setlength{\headheight}{15pt} % Amplía el tamaño del índice


%\usepackage{lastpage}   % Referenciar última pag   \pageref{LastPage}
\fancyfoot[C]{\thepage}

%------------------------------------------------------------------------

% Conseguir que no ponga "Capítulo 1". Sino solo "1."
\makeatletter
\@ifclassloaded{book}{
  \renewcommand{\chaptermark}[1]{\markboth{\thechapter.\ #1}{}} % En el encabezado
    
  \renewcommand{\@makechapterhead}[1]{%
  \vspace*{50\p@}%
  {\parindent \z@ \raggedright \normalfont
    \ifnum \c@secnumdepth >\m@ne
      \huge\bfseries \thechapter.\hspace{1em}\ignorespaces
    \fi
    \interlinepenalty\@M
    \Huge \bfseries #1\par\nobreak
    \vskip 40\p@
  }}
}
\makeatother

%------------------------------------------------------------------------
% Paquetes de cógido
\usepackage{minted}
\renewcommand\listingscaption{Código fuente}

\usepackage{fancyvrb}
% Personaliza el tamaño de los números de línea
\renewcommand{\theFancyVerbLine}{\small\arabic{FancyVerbLine}}

% Estilo para C++
\newminted{cpp}{
    frame=lines,
    framesep=2mm,
    baselinestretch=1.2,
    linenos,
    escapeinside=||
}

% para minted
\definecolor{LightGray}{rgb}{0.95,0.95,0.92}
\setminted{
    linenos=true,
    stepnumber=5,
    numberfirstline=true,
    autogobble,
    breaklines=true,
    breakautoindent=true,
    breaksymbolleft=,
    breaksymbolright=,
    breaksymbolindentleft=0pt,
    breaksymbolindentright=0pt,
    breaksymbolsepleft=0pt,
    breaksymbolsepright=0pt,
    fontsize=\footnotesize,
    bgcolor=LightGray,
    numbersep=10pt
}


\usepackage{listings} % Para incluir código desde un archivo

\renewcommand\lstlistingname{Código Fuente}
\renewcommand\lstlistlistingname{Índice de Códigos Fuente}

% Definir colores
\definecolor{vscodepurple}{rgb}{0.5,0,0.5}
\definecolor{vscodeblue}{rgb}{0,0,0.8}
\definecolor{vscodegreen}{rgb}{0,0.5,0}
\definecolor{vscodegray}{rgb}{0.5,0.5,0.5}
\definecolor{vscodebackground}{rgb}{0.97,0.97,0.97}
\definecolor{vscodelightgray}{rgb}{0.9,0.9,0.9}

% Configuración para el estilo de C similar a VSCode
\lstdefinestyle{vscode_C}{
  backgroundcolor=\color{vscodebackground},
  commentstyle=\color{vscodegreen},
  keywordstyle=\color{vscodeblue},
  numberstyle=\tiny\color{vscodegray},
  stringstyle=\color{vscodepurple},
  basicstyle=\scriptsize\ttfamily,
  breakatwhitespace=false,
  breaklines=true,
  captionpos=b,
  keepspaces=true,
  numbers=left,
  numbersep=5pt,
  showspaces=false,
  showstringspaces=false,
  showtabs=false,
  tabsize=2,
  frame=tb,
  framerule=0pt,
  aboveskip=10pt,
  belowskip=10pt,
  xleftmargin=10pt,
  xrightmargin=10pt,
  framexleftmargin=10pt,
  framexrightmargin=10pt,
  framesep=0pt,
  rulecolor=\color{vscodelightgray},
  backgroundcolor=\color{vscodebackground},
}

%------------------------------------------------------------------------

% Comandos definidos
\newcommand{\bb}[1]{\mathbb{#1}}
\newcommand{\cc}[1]{\mathcal{#1}}

% I prefer the slanted \leq
\let\oldleq\leq % save them in case they're every wanted
\let\oldgeq\geq
\renewcommand{\leq}{\leqslant}
\renewcommand{\geq}{\geqslant}

% Si y solo si
\newcommand{\sii}{\iff}

% Letras griegas
\newcommand{\eps}{\epsilon}
\newcommand{\veps}{\varepsilon}
\newcommand{\lm}{\lambda}

\newcommand{\ol}{\overline}
\newcommand{\ul}{\underline}
\newcommand{\wt}{\widetilde}
\newcommand{\wh}{\widehat}

\let\oldvec\vec
\renewcommand{\vec}{\overrightarrow}

% Derivadas parciales
\newcommand{\del}[2]{\frac{\partial #1}{\partial #2}}
\newcommand{\Del}[3]{\frac{\partial^{#1} #2}{\partial #3^{#1}}}
\newcommand{\deld}[2]{\dfrac{\partial #1}{\partial #2}}
\newcommand{\Deld}[3]{\dfrac{\partial^{#1} #2}{\partial #3^{#1}}}


\newcommand{\AstIg}{\stackrel{(\ast)}{=}}
\newcommand{\Hop}{\stackrel{L'H\hat{o}pital}{=}}

\newcommand{\red}[1]{{\color{red}#1}} % Para integrales, destacar los cambios.

% Método de integración
\newcommand{\MetInt}[2]{
    \left[\begin{array}{c}
        #1 \\ #2
    \end{array}\right]
}

% Declarar aplicaciones
% 1. Nombre aplicación
% 2. Dominio
% 3. Codominio
% 4. Variable
% 5. Imagen de la variable
\newcommand{\Func}[5]{
    \begin{equation*}
        \begin{array}{rrll}
            #1:& #2 & \longrightarrow & #3\\
               & #4 & \longmapsto & #5
        \end{array}
    \end{equation*}
}

%------------------------------------------------------------------------

\let\oldRe\Re % save them in case they're every wanted
\let\oldIm\Im
\renewcommand{\Re}{\operatorname{Re}} % redefine them
\renewcommand{\Im}{\operatorname{Im}}
\DeclareMathOperator{\Log}{Log}
\DeclareMathOperator{\Arg}{Arg}
\DeclareMathOperator{\ord}{ord}
\DeclareMathOperator{\Ind}{Ind}
\DeclareMathOperator{\Fr}{Fr}
\DeclareMathOperator{\Res}{Res}


\usetikzlibrary{arrows.meta, decorations.markings} % Cargar las bibliotecas necesarias

% Configuración para las flechas
\tikzset{
    arrow at 1/3/.style={postaction={decorate},
        decoration={markings, mark=at position 0.33 with {\arrow{Stealth}}}},
    arrow at 2/3/.style={postaction={decorate},
        decoration={markings, mark=at position 0.66 with {\arrow{Stealth}}}}
}


\begin{document}

    % 1. Foto de fondo
    % 2. Título
    % 3. Encabezado Izquierdo
    % 4. Color de fondo
    % 5. Coord x del titulo
    % 6. Coord y del titulo
    % 7. Fecha

    
    % 1. Foto de fondo
% 2. Título
% 3. Encabezado Izquierdo
% 4. Color de fondo
% 5. Coord x del titulo
% 6. Coord y del titulo
% 7. Fecha

\newcommand{\portada}[7]{

    \portadaBase{#1}{#2}{#3}{#4}{#5}{#6}{#7}
    \portadaBook{#1}{#2}{#3}{#4}{#5}{#6}{#7}
}

\newcommand{\portadaExamen}[7]{

    \portadaBase{#1}{#2}{#3}{#4}{#5}{#6}{#7}
    \portadaArticle{#1}{#2}{#3}{#4}{#5}{#6}{#7}
}




\newcommand{\portadaBase}[7]{

    % Tiene la portada principal y la licencia Creative Commons
    
    % 1. Foto de fondo
    % 2. Título
    % 3. Encabezado Izquierdo
    % 4. Color de fondo
    % 5. Coord x del titulo
    % 6. Coord y del titulo
    % 7. Fecha
    
    
    \thispagestyle{empty}               % Sin encabezado ni pie de página
    \newgeometry{margin=0cm}        % Márgenes nulos para la primera página
    
    
    % Encabezado
    \fancyhead[L]{\helv #3}
    \fancyhead[R]{\helv \nouppercase{\leftmark}}
    
    
    \pagecolor{#4}        % Color de fondo para la portada
    
    \begin{figure}[p]
        \centering
        \transparent{0.3}           % Opacidad del 30% para la imagen
        
        \includegraphics[width=\paperwidth, keepaspectratio]{assets/#1}
    
        \begin{tikzpicture}[remember picture, overlay]
            \node[anchor=north west, text=white, opacity=1, font=\fontsize{60}{90}\selectfont\bfseries\sffamily, align=left] at (#5, #6) {#2};
            
            \node[anchor=south east, text=white, opacity=1, font=\fontsize{12}{18}\selectfont\sffamily, align=right] at (9.7, 3) {\textbf{\href{https://losdeldgiim.github.io/}{Los Del DGIIM}}};
            
            \node[anchor=south east, text=white, opacity=1, font=\fontsize{12}{15}\selectfont\sffamily, align=right] at (9.7, 1.8) {Doble Grado en Ingeniería Informática y Matemáticas\\Universidad de Granada};
        \end{tikzpicture}
    \end{figure}
    
    
    \restoregeometry        % Restaurar márgenes normales para las páginas subsiguientes
    \pagecolor{white}       % Restaurar el color de página
    
    
    \newpage
    \thispagestyle{empty}               % Sin encabezado ni pie de página
    \begin{tikzpicture}[remember picture, overlay]
        \node[anchor=south west, inner sep=3cm] at (current page.south west) {
            \begin{minipage}{0.5\paperwidth}
                \href{https://creativecommons.org/licenses/by-nc-nd/4.0/}{
                    \includegraphics[height=2cm]{assets/Licencia.png}
                }\vspace{1cm}\\
                Esta obra está bajo una
                \href{https://creativecommons.org/licenses/by-nc-nd/4.0/}{
                    Licencia Creative Commons Atribución-NoComercial-SinDerivadas 4.0 Internacional (CC BY-NC-ND 4.0).
                }\\
    
                Eres libre de compartir y redistribuir el contenido de esta obra en cualquier medio o formato, siempre y cuando des el crédito adecuado a los autores originales y no persigas fines comerciales. 
            \end{minipage}
        };
    \end{tikzpicture}
    
    
    
    % 1. Foto de fondo
    % 2. Título
    % 3. Encabezado Izquierdo
    % 4. Color de fondo
    % 5. Coord x del titulo
    % 6. Coord y del titulo
    % 7. Fecha


}


\newcommand{\portadaBook}[7]{

    % 1. Foto de fondo
    % 2. Título
    % 3. Encabezado Izquierdo
    % 4. Color de fondo
    % 5. Coord x del titulo
    % 6. Coord y del titulo
    % 7. Fecha

    % Personaliza el formato del título
    \pretitle{\begin{center}\bfseries\fontsize{42}{56}\selectfont}
    \posttitle{\par\end{center}\vspace{2em}}
    
    % Personaliza el formato del autor
    \preauthor{\begin{center}\Large}
    \postauthor{\par\end{center}\vfill}
    
    % Personaliza el formato de la fecha
    \predate{\begin{center}\huge}
    \postdate{\par\end{center}\vspace{2em}}
    
    \title{#2}
    \author{\href{https://losdeldgiim.github.io/}{Los Del DGIIM}}
    \date{Granada, #7}
    \maketitle
    
    \tableofcontents
}




\newcommand{\portadaArticle}[7]{

    % 1. Foto de fondo
    % 2. Título
    % 3. Encabezado Izquierdo
    % 4. Color de fondo
    % 5. Coord x del titulo
    % 6. Coord y del titulo
    % 7. Fecha

    % Personaliza el formato del título
    \pretitle{\begin{center}\bfseries\fontsize{42}{56}\selectfont}
    \posttitle{\par\end{center}\vspace{2em}}
    
    % Personaliza el formato del autor
    \preauthor{\begin{center}\Large}
    \postauthor{\par\end{center}\vspace{3em}}
    
    % Personaliza el formato de la fecha
    \predate{\begin{center}\huge}
    \postdate{\par\end{center}\vspace{5em}}
    
    \title{#2}
    \author{\href{https://losdeldgiim.github.io/}{Los Del DGIIM}}
    \date{Granada, #7}
    \thispagestyle{empty}               % Sin encabezado ni pie de página
    \maketitle
    \vfill
}
    \portadaExamen{ffccA4.jpg}{Variable Compleja I\\Examen XVII}{Variable Compleja I. Examen XVII}{MidnightBlue}{-9.5}{28}{2024-2025}{Arturo Olivares Martos}

    \begin{description}
        \item[Asignatura] Variable Compleja I.
        \item[Curso Académico] 2020-21.
        \item[Grado] Doble Grado en Ingeniería Informática y Matemáticas.
        \item[Grupo] Único.
        \item[Profesor] Javier Merí de la Maza.
        \item[Descripción] Convocatoria Ordinaria.
        \item[Fecha] 16 de Junio de 2022.
        \item[Duración] 3.5 horas.
    \end{description}
    \newpage

    \begin{ejercicio}[2.5 puntos]
        Probar que, para $a, t \in \mathbb{R}^+$, se tiene:
        \[
            \int_{-\infty}^{+\infty} \frac{\cos(tx)}{(x^2 + a^2)^2} \, dx = \frac{2a^3 (1 + at)}{e^{at}}.
        \]
    \end{ejercicio}

    \begin{ejercicio}[2.5 puntos]
        Para cada $n \in \mathbb{N}$, sea $f_n : \mathbb{C} \to \mathbb{C}$ la función dada por
        \[
            f_n(z) = \int_n^{n+1} \frac{e^{\frac{z^3}{1+t}}}{1 + t^2} \, dt \quad \forall z \in \mathbb{C}.
        \]
        \begin{enumerate}
            \item Probar que $f_n \in \cc{H}(\mathbb{C})$.
            \item Probar que la serie de funciones $\sum\limits_{n\geq 1} f_n$ converge en $\mathbb{C}$ y que su suma es una función entera.
        \end{enumerate}
    \end{ejercicio}

    \begin{ejercicio}[2.5 puntos]
        Probar que no hay más funciones enteras e inyectivas que los polinomios de grado uno.
    \end{ejercicio}

    \begin{ejercicio}[2.5 puntos]
        Sea $f \in \cc{H}(D(0, 1))$ y supongamos que existe $n \in \mathbb{N}$ tal que, para cada $r \in (0, 1)$ se verifica
        \[
            \max\left\{ |f(z)| : |z| = r \right\} = r^n.
        \]
        Probar que existe $\alpha \in \mathbb{T}$ tal que $f(z) = \alpha z^n$ para todo $z \in D(0, 1)$.
    \end{ejercicio}

    \newpage
    \setcounter{ejercicio}{0}


    \begin{ejercicio}[2.5 puntos]\label{ej:14.7}
        Probar que, para $a, t \in \mathbb{R}^+$, se tiene:
        \[
            \int_{-\infty}^{+\infty} \frac{\cos(tx)}{(x^2 + a^2)^2} \, dx = \frac{2a^3 (1 + at)}{e^{at}}.
        \]




        Calculamos las raíces del denominador:
    \begin{align*}
        x^2 + a^2 &= 0 \implies x^2 = -a^2 \implies x\in A:=\left\{-ai, ai\right\}.
    \end{align*}

    Definimos la función:
    \Func{f}{\bb{C}\setminus A}{\bb{C}}{z}{\frac{e^{i t z}}{(z^2 + a^2)^2}}
    Notemos que $f\in \cc{H}(\bb{C}\setminus A)$, y que $A'=\emptyset$, por lo que podemos aplicar el Teorema de los Residuos. Como $\bb{C}$ es homológicamente conexo, podemos aplicar el Teorema de los Residuos para cualquier ciclo $\Sigma$ en $\bb{C}\setminus A$.

    Para todo $R > a$, consideramos el siguiente ciclo $\Sigma_R = \gamma_R + \sigma_R$, representado en la Figura~\ref{fig:ej:14.7}, donde:
    \Func{\gamma_R}{[-R, R]}{\bb{C}}{t}{t}
    \Func{\sigma_R}{[0, \pi]}{\bb{C}}{t}{Re^{it}}
    \begin{figure}
        \centering
        \begin{tikzpicture}
            \begin{axis}[
                axis lines=middle,
                xlabel={$x$},
                ylabel={$y$},
                xtick=\empty,
                ytick=\empty,
                xmin=-3.5, xmax=3.5,
                ymin=-1, ymax=1,
                axis equal,
            ]
                \def\R{2.4}
                \def\a{1}

                % Polos
                \draw[fill=red] (0, -\a) circle (2pt) node[right] {$-ai$};
                \draw[fill=red] (0, \a) circle (2pt) node[right] {$ai$};

                % sigma_r
                \draw[thick, blue, arrow at 1/3, arrow at 2/3] (\R, 0) arc[start angle=0, end angle=180, radius=\R]
                    node[midway, below left, yshift=-1em, xshift=-1em] {$\sigma_r$};

                % gamma_r
                \draw[thick, blue, arrow at 1/3, arrow at 2/3] (-\R, 0) -- (\R, 0)
                    node[pos=0.25, below] {$\gamma_r$};

                % Puntos de unión
                \draw[fill=blue] (-\R, 0) circle (2pt);
                \draw[fill=blue] (\R, 0) circle (2pt);

            \end{axis}
        \end{tikzpicture}
        \caption{Ciclo de integración $\Sigma_R$ del Ejercicio~\ref{ej:14.7}.}
        \label{fig:ej:14.7}
    \end{figure}

    De esta forma, tenemos que:
    \begin{align*}
        \int_{\Sigma_R} f(z) \, dz &= \int_{\gamma_R} f(z) \, dz + \int_{\sigma_R} f(z) \, dz
        = 2\pi i\sum_{z_0\in A}\Res(f,z_0)\Ind_{\Sigma_R}(z_0)
    \end{align*}

    Calculemos la primera integral que nos ha resultado:
    \begin{align*}
        \int_{\gamma_R} f(z) \, dz &= \int_{-R}^{R} \frac{e^{i t z}}{(z^2 + a^2)^2} \, dz
        = \int_{-R}^{R} \frac{\cos(t z)}{(z^2 + a^2)^2} \, dz
        + i\int_{-R}^{R} \frac{\sen(t z)}{(z^2 + a^2)^2} \, dz
    \end{align*}

    Notemos que la integral pedida es la parte real de la integral. Veamos la siguiente integral:
    \begin{align*}
        \int_{\sigma_R} f(z) \, dz &\leq \pi R\cdot \sup\left\{\left|\dfrac{e^{i t z}}{(z^2 + a^2)^2}\right| : z\in \sigma_R^*\right\}
        \leq \frac{\pi R}{(R^2 - a^2)^2}
    \end{align*}
    donde hemos usado que, si $z\in \sigma_R^*$, entonces $|z|=R$ y, como $R>a>0$, tenemos que $R^2>a^2$, por lo que:
    \begin{align*}
        |z^2 + a^2| &\geq \left||z^2| - |a^2|\right| = \left|R^2 - a^2\right| = R^2 - a^2\\
        |e^{i t z}| &=e^{-t\Im(z)}\leq e^0 = 1.
    \end{align*}
    Por tanto, como la expresión anterior es válida para cualquier $R > a$, podemos hacer $R \to +\infty$ y tenemos que:
    \begin{align*}
        \lim_{R\to+\infty} \int_{\sigma_R} f(z) \, dz &= 0.
    \end{align*}

    Calculamos ahora los índices. Por la forma en la que se ha definido el ciclo $\Sigma_R$, para todo $R > a$, tenemos que:
    \begin{align*}
        \Ind_{\Sigma_R}(-ai) &= 0\\
        \Ind_{\Sigma_R}(ai) &= 1.
    \end{align*}

    Por tanto, tan solo hemos de calcular el residuo en el polo $ai$.
    \begin{align*}
        \lim_{z\to ai} (z - ai)f(z) &= \lim_{z\to ai} (z - ai)\cdot \frac{e^{i t z}}{[(z - ai)(z + ai)]^2}
        = \lim_{z\to ai} \frac{e^{i t z}}{(z + ai)^2(z - ai)} = +\infty.\\
        \lim_{z\to ai} (z - ai)^2f(z) &= \lim_{z\to ai} \frac{e^{i t z}}{(z + ai)^2}
        = \frac{e^{i t ai}}{(2ai)^2} = \frac{e^{-at}}{-4a^2} = -\frac{e^{-at}}{4a^2}\in \bb{C}^*
    \end{align*}

    Por tanto, deducimos que el orden del polo $ai$ es $2$, y que el residuo es:
    \begin{align*}
        \Res(f, ai) &= \lim_{z\to ai} \dfrac{d}{dz}\left((z - ai)^2f(z)\right)
        = \lim_{z\to ai} \dfrac{d}{dz}\left(\frac{e^{i t z}}{(z + ai)^2}\right)
        =\\&= \lim_{z\to ai} \frac{i t e^{i t z}(z + ai)^2 - e^{i t z}\cdot 2(z + ai)}{(z + ai)^4}
        = \lim_{z\to ai} \frac{i t e^{i t z}(z + ai) - 2e^{i t z}}{(z + ai)^3}
        =\\&=  \lim_{z\to ai} e^{i t z} \frac{i t (z + ai) - 2}{(z + ai)^3}
        = e^{-at}\cdot \dfrac{i t (2ai) - 2}{(2ai)^3}
        = e^{-at}\cdot \dfrac{-at -1}{-4a^3i}
        = e^{-at}\cdot \dfrac{at+1}{4a^3i}
    \end{align*}

    Por tanto, tenemos que:
    \begin{align*}
        \int_{\Sigma_R} f(z) \, dz &= 2\pi i\left(e^{-at}\cdot \frac{at + 1}{4a^3i} \cdot 1\right)
        = \frac{\pi\cdot e^{-at}(at + 1)}{2a^3}.
    \end{align*}

    Por tanto, tenemos que:
    \begin{align*}
        \int_{-R}^{R} \frac{\cos(t z)}{(z^2 + a^2)^2} \, dz + i\int_{-R}^{R} \frac{\sen(t z)}{(z^2 + a^2)^2} \, dz + \int_{\sigma_R} f(z) \, dz &= \frac{\pi\cdot e^{-at}(at + 1)}{2a^3}.
    \end{align*}

    Como esta expresión es válida para cualquier $R > a$, podemos hacer $R \to +\infty$ y tenemos que:
    \begin{align*}
        \int_{-\infty}^{+\infty} \frac{\cos(t z)}{(z^2 + a^2)^2} \, dz + i\int_{-\infty}^{+\infty} \frac{\sen(t z)}{(z^2 + a^2)^2} \, dz &= \frac{\pi\cdot e^{-at}(at + 1)}{2a^3}.
    \end{align*}

    Igualando las partes reales, tenemos que:
    \begin{align*}
        \int_{-\infty}^{+\infty} \frac{\cos(t z)}{(z^2 + a^2)^2} \, dz &= \frac{\pi\cdot e^{-at}(at + 1)}{2a^3}.
    \end{align*}
    como queríamos demostrar.
    \end{ejercicio}

    \begin{ejercicio}[2.5 puntos]
        Para cada $n \in \mathbb{N}$, sea $f_n : \mathbb{C} \to \mathbb{C}$ la función dada por
        \[
            f_n(z) = \int_n^{n+1} \frac{e^{\frac{z^3}{1+t}}}{1 + t^2} \, dt \quad \forall z \in \mathbb{C}.
        \]
        \begin{enumerate}
            \item Probar que $f_n \in \cc{H}(\mathbb{C})$.
            
            Definimos la función:
            \Func{\Phi}{[n, n+1]\times \bb{C}}{\bb{C}}{(t, z)}{\frac{e^{\frac{z^3}{1+t}}}{1 + t^2}}

            Como $1 + t^2 > 0$, $1+t> 0$ para todo $t\in [n, n+1]$, tenemos que $\Phi$ está bien definida. $\Phi$ es continua en $[n, n+1]\times \bb{C}$, y para cada $t\in [n, n+1]$, la función $z\mapsto \Phi(t, z)$ es holomorfa en $\bb{C}$. Por tanto, por el Teorema de Holomorfía de integrales dependientes de un parámetro, tenemos que $f_n$ es holomorfa en $\bb{C}$.
            \item Probar que la serie de funciones $\sum\limits_{n\geq 1} f_n$ converge en $\mathbb{C}$ y que su suma es una función entera.
            
            Sea $K\subset \bb{C}$ compacto. Para todo $z\in K$, $n\in \mathbb{N}$, tenemos que:
            \begin{align*}
                |f_n(z)| &= \left|\int_n^{n+1} \frac{e^{\frac{z^3}{1+t}}}{1 + t^2} \, dt\right|
                \leq \sup\left\{\left|\frac{e^{\frac{z^3}{1+t}}}{1 + t^2}\right| : t\in [n, n+1]\right\}
            \end{align*}
            Hacemos uso de que, para todo $t\in [n, n+1]$, tenemos que:
            \begin{align*}
                |1+ t^2| &\geq 1+t^2\geq 1+n^2\geq n^2\\
                \left|e^{\frac{z^3}{1+t}}\right| &= e^{\frac{\Re(z^3)}{1+t}} \leq
                e^{\frac{\Re(z^3)}{n+1}}
                \leq e^{\frac{\Re(z^3)}{n}}
            \end{align*}

            Por ser $K$ compacto y $\Re$ continua, existe $M\in \bb{R}$ tal que:
            \begin{align*}
                M=\max\left\{\Re(z^3) : z\in K\right\}.
            \end{align*}

            Por tanto, para todo $z\in K$ y $n\in \mathbb{N}$, tenemos que:
            \begin{align*}
                |f_n(z)| &\leq \frac{e^{\frac{M}{n}}}{n^2} \leq \frac{e^{\frac{M}{1}}}{n^2}
                = e^M\cdot \frac{1}{n^2}.
            \end{align*}
            Como dicha serie es convergente, por el Test de Weierstrass, tenemos que la serie de funciones $\sum\limits_{n\geq 1} f_n$ converge uniformemente en $K$.\\

            Por el Teorema de Convergencia de Weierstrass, tenemos que la suma de la serie de funciones es una función holomorfa en $\bb{C}$.
        \end{enumerate}
    \end{ejercicio}

    \begin{ejercicio}[2.5 puntos]
        Probar que no hay más funciones enteras e inyectivas que los polinomios de grado uno.\\

        Sea $f$ una función entera e inyectiva. Supongamos que $f$ no es polinómica. Entonces, fijado $R\in \bb{R}^+$, por el Corolario del Teorema de Casorati:
        \begin{align*}
            \ol{f(\bb{C}\setminus D(0, R))} &= \bb{C}.
        \end{align*}

        Por otro lado, como $\in \cc{H}(\bb{C})$ no es constante, por el Teorema de la Aplicación Abierta, tenemos que $f(D(0, R))$ es un abierto. Como el conjunto $f(\bb{C}\setminus D(0, R))$ es denso en $\bb{C}$, interseca a todos los abiertos no vacíos de $\bb{C}$, y por tanto, también interseca a $f(D(0, R))$. Por tanto, existe $w_0\in \bb{R}$ tal que:
        \begin{align*}
            w_0\in f(\bb{C}\setminus D(0, R))\cap f(D(0, R))\neq \emptyset.
        \end{align*}

        Por tanto, existe $z_1\in \bb{C}\setminus D(0, R)$ y $z_2\in D(0, R)$ tales que $f(z_1) = f(z_2) = w_0$. Como $|z_1| > R$ y $|z_2| < R$, tenemos que $z_1\neq z_2$. Por tanto, $f$ no es inyectiva, lo cual contradice la hipótesis. Por tanto, $f$ es un polinomio.\\

        Supongamos ahora que $f$ es un polinomio de grado $n\in \mathbb{N}$. Entonces:
        \begin{equation*}
            f(z) = \sum_{k=0}^n a_k z^k \quad \forall z\in \bb{C},
        \end{equation*}
        donde $a_n\neq 0$. Como $f$ es inyectiva, entonces $f$ no es constante, luego $n\geq 1$. Por la caracterización de la inyectividad local, tenemos que:
        \begin{equation*}
            0\neq f'(z) \quad \forall z\in \bb{C}.
        \end{equation*}
        Por tanto, $f'$ no tiene raíces. Como $f'$ es un polinomio y no tiene raíces, por el recíproco del Teorema Fundamental del Álgebra, tenemos que $f'$ es constante y no nulo, por lo que $f'$ es un polinomio de grado $0$. Por tanto, $f$ es un polinomio de grado $1$.
    \end{ejercicio}

    \begin{ejercicio}[2.5 puntos]
        Sea $f \in \cc{H}(D(0, 1))$ y supongamos que existe $n \in \mathbb{N}$ tal que, para cada $r \in \left]0,1\right[$ se verifica
        \[
            \max\left\{ |f(z)| : |z| = r \right\} = r^n.
        \]
        Probar que existe $\alpha \in \mathbb{T}$ tal que $f(z) = \alpha z^n$ para todo $z \in D(0, 1)$.\\

        Definimos la función:
        \Func{\wt{g}}{D(0, 1)\setminus \{0\}}{\bb{C}}{z}{\frac{f(z)}{z^n}}

        Como $\wt{g}\in \cc{H}(D(0, 1)\setminus \{0\})$, por el Teorema de Extensión de Riemman, $\exists g\in \cc{H}(D(0, 1))$ tal que:
        \begin{equation*}
            g(z) = \wt{g}(z) = \frac{f(z)}{z^n} \quad \forall z\in D(0, 1)\setminus \{0\}.
        \end{equation*}

        Fijado ahora $r\in \left]0, 1\right[$, consideramos la restricción de $g$ a $\ol{D}(0, r)$. Como $g$ es continua en dicho conjunto y holomorfa en su interior, por el Principio del Módulo Máximo, tenemos que:
        \begin{align*}
            \max\left\{ |g(z)| : |z| \leq r \right\} &= \max\left\{ |g(z)| : |z| = r \right\}
            = \max\left\{ \left|\frac{f(z)}{z^n}\right| : |z| = r \right\} 
            =\\&= \max\left\{ \frac{|f(z)|}{r^n} : |z| = r \right\}
            = \frac{1}{r^n}\max\left\{ |f(z)| : |z| = r \right\} = 1.
        \end{align*}

        Como esta expresión es válida para todo $r\in \left]0, 1\right[$, tenemos que:
        \begin{align*}
            \max\left\{ |g(z)| : |z| < 1 \right\} &= 1.
        \end{align*}



        Sea ahora $r\in \left]0, 1\right[$, y consideramos $z_r\in C(0,r)^*$ tal que:
        \begin{align*}
            |g(z_r)| &= \max\left\{ |g(z)| : |z| = r \right\} = 1.
        \end{align*}

        Por tanto, tenemos que $z_r\in D(0, 1)$ y:
        \begin{equation*}
            1 = |g(z_r)| \geq \left|g(z)\right|\forall z\in D(0, 1).
        \end{equation*}

        Por tanto, $z_r$ es un máximo de $g$ en $D(0, 1)$, y por el Teorema del Módulo Máximo, tenemos que $g$ es constante. Por tanto, existe $\alpha\in \bb{C}$ tal que:
        \begin{equation*}
            g(z) = \alpha \quad \forall z\in D(0, 1).
        \end{equation*}

        Por tanto, tenemos que:
        \begin{align*}
            f(z) &= g(z)z^n = \alpha z^n \quad \forall z\in D(0, 1).
        \end{align*}

        Reescribimos por tanto la ecuación dada. Para todo $r\in \left]0, 1\right[$, tenemos que:
        \begin{align*}
            r^n &= \max\left\{ |f(z)| : |z| = r \right\} = \max\left\{ |\alpha z^n| : |z| = r \right\}
            = |\alpha| r^n.
        \end{align*}

        Despejando, obtenemos $|\alpha| = 1$, por lo que $\alpha \in \bb{T}$.
    \end{ejercicio}

\end{document}