\documentclass[12pt]{article}

% Idioma y codificación
\usepackage[spanish, es-tabla]{babel}       %es-tabla para que se titule "Tabla"
\usepackage[utf8]{inputenc}

% Márgenes
\usepackage[a4paper,top=3cm,bottom=2.5cm,left=3cm,right=3cm]{geometry}

% Comentarios de bloque
\usepackage{verbatim}

% Paquetes de links
\usepackage[hidelinks]{hyperref}    % Permite enlaces
\usepackage{url}                    % redirecciona a la web

% Más opciones para enumeraciones
\usepackage{enumitem}

% Personalizar la portada
\usepackage{titling}

% Paquetes de tablas
\usepackage{multirow}


%------------------------------------------------------------------------

%Paquetes de figuras
\usepackage{caption}
\usepackage{subcaption} % Figuras al lado de otras
\usepackage{float}      % Poner figuras en el sitio indicado H.


% Paquetes de imágenes
\usepackage{graphicx}       % Paquete para añadir imágenes
\usepackage{transparent}    % Para manejar la opacidad de las figuras

% Paquete para usar colores
\usepackage[dvipsnames]{xcolor}
\usepackage{pagecolor}      % Para cambiar el color de la página

% Habilita tamaños de fuente mayores
\usepackage{fix-cm}

% Para los gráficos
\usepackage{tikz}

% Para poder situar los nodos en los grafos
\usetikzlibrary{positioning}


%------------------------------------------------------------------------

% Paquetes de matemáticas
\usepackage{mathtools, amsfonts, amssymb, mathrsfs}
\usepackage[makeroom]{cancel}     % Simplificar tachando
\usepackage{polynom}    % Divisiones y Ruffini
\usepackage{units} % Para poner fracciones diagonales con \nicefrac

\usepackage{pgfplots}   %Representar funciones
\pgfplotsset{compat=1.18}  % Versión 1.18

\usepackage{tikz-cd}    % Para usar diagramas de composiciones
\usetikzlibrary{calc}   % Para usar cálculo de coordenadas en tikz

%Definición de teoremas, etc.
\usepackage{amsthm}
%\swapnumbers   % Intercambia la posición del texto y de la numeración

\theoremstyle{plain}

\makeatletter
\@ifclassloaded{article}{
  \newtheorem{teo}{Teorema}[section]
}{
  \newtheorem{teo}{Teorema}[chapter]  % Se resetea en cada chapter
}
\makeatother

\newtheorem{coro}{Corolario}[teo]           % Se resetea en cada teorema
\newtheorem{prop}[teo]{Proposición}         % Usa el mismo contador que teorema
\newtheorem{lema}[teo]{Lema}                % Usa el mismo contador que teorema

\theoremstyle{remark}
\newtheorem*{observacion}{Observación}

\theoremstyle{definition}

\makeatletter
\@ifclassloaded{article}{
  \newtheorem{definicion}{Definición} [section]     % Se resetea en cada chapter
}{
  \newtheorem{definicion}{Definición} [chapter]     % Se resetea en cada chapter
}
\makeatother

\newtheorem*{notacion}{Notación}
\newtheorem*{ejemplo}{Ejemplo}
\newtheorem*{ejercicio*}{Ejercicio}             % No numerado
\newtheorem{ejercicio}{Ejercicio} [section]     % Se resetea en cada section


% Modificar el formato de la numeración del teorema "ejercicio"
\renewcommand{\theejercicio}{%
  \ifnum\value{section}=0 % Si no se ha iniciado ninguna sección
    \arabic{ejercicio}% Solo mostrar el número de ejercicio
  \else
    \thesection.\arabic{ejercicio}% Mostrar número de sección y número de ejercicio
  \fi
}


% \renewcommand\qedsymbol{$\blacksquare$}         % Cambiar símbolo QED
%------------------------------------------------------------------------

% Paquetes para encabezados
\usepackage{fancyhdr}
\pagestyle{fancy}
\fancyhf{}

\newcommand{\helv}{ % Modificación tamaño de letra
\fontfamily{}\fontsize{12}{12}\selectfont}
\setlength{\headheight}{15pt} % Amplía el tamaño del índice


%\usepackage{lastpage}   % Referenciar última pag   \pageref{LastPage}
\fancyfoot[C]{\thepage}

%------------------------------------------------------------------------

% Conseguir que no ponga "Capítulo 1". Sino solo "1."
\makeatletter
\@ifclassloaded{book}{
  \renewcommand{\chaptermark}[1]{\markboth{\thechapter.\ #1}{}} % En el encabezado
    
  \renewcommand{\@makechapterhead}[1]{%
  \vspace*{50\p@}%
  {\parindent \z@ \raggedright \normalfont
    \ifnum \c@secnumdepth >\m@ne
      \huge\bfseries \thechapter.\hspace{1em}\ignorespaces
    \fi
    \interlinepenalty\@M
    \Huge \bfseries #1\par\nobreak
    \vskip 40\p@
  }}
}
\makeatother

%------------------------------------------------------------------------
% Paquetes de cógido
\usepackage{minted}
\renewcommand\listingscaption{Código fuente}

\usepackage{fancyvrb}
% Personaliza el tamaño de los números de línea
\renewcommand{\theFancyVerbLine}{\small\arabic{FancyVerbLine}}

% Estilo para C++
\newminted{cpp}{
    frame=lines,
    framesep=2mm,
    baselinestretch=1.2,
    linenos,
    escapeinside=||
}

% para minted
\definecolor{LightGray}{rgb}{0.95,0.95,0.92}
\setminted{
    linenos=true,
    stepnumber=5,
    numberfirstline=true,
    autogobble,
    breaklines=true,
    breakautoindent=true,
    breaksymbolleft=,
    breaksymbolright=,
    breaksymbolindentleft=0pt,
    breaksymbolindentright=0pt,
    breaksymbolsepleft=0pt,
    breaksymbolsepright=0pt,
    fontsize=\footnotesize,
    bgcolor=LightGray,
    numbersep=10pt
}


\usepackage{listings} % Para incluir código desde un archivo

\renewcommand\lstlistingname{Código Fuente}
\renewcommand\lstlistlistingname{Índice de Códigos Fuente}

% Definir colores
\definecolor{vscodepurple}{rgb}{0.5,0,0.5}
\definecolor{vscodeblue}{rgb}{0,0,0.8}
\definecolor{vscodegreen}{rgb}{0,0.5,0}
\definecolor{vscodegray}{rgb}{0.5,0.5,0.5}
\definecolor{vscodebackground}{rgb}{0.97,0.97,0.97}
\definecolor{vscodelightgray}{rgb}{0.9,0.9,0.9}

% Configuración para el estilo de C similar a VSCode
\lstdefinestyle{vscode_C}{
  backgroundcolor=\color{vscodebackground},
  commentstyle=\color{vscodegreen},
  keywordstyle=\color{vscodeblue},
  numberstyle=\tiny\color{vscodegray},
  stringstyle=\color{vscodepurple},
  basicstyle=\scriptsize\ttfamily,
  breakatwhitespace=false,
  breaklines=true,
  captionpos=b,
  keepspaces=true,
  numbers=left,
  numbersep=5pt,
  showspaces=false,
  showstringspaces=false,
  showtabs=false,
  tabsize=2,
  frame=tb,
  framerule=0pt,
  aboveskip=10pt,
  belowskip=10pt,
  xleftmargin=10pt,
  xrightmargin=10pt,
  framexleftmargin=10pt,
  framexrightmargin=10pt,
  framesep=0pt,
  rulecolor=\color{vscodelightgray},
  backgroundcolor=\color{vscodebackground},
}

%------------------------------------------------------------------------

% Comandos definidos
\newcommand{\bb}[1]{\mathbb{#1}}
\newcommand{\cc}[1]{\mathcal{#1}}

% I prefer the slanted \leq
\let\oldleq\leq % save them in case they're every wanted
\let\oldgeq\geq
\renewcommand{\leq}{\leqslant}
\renewcommand{\geq}{\geqslant}

% Si y solo si
\newcommand{\sii}{\iff}

% Letras griegas
\newcommand{\eps}{\epsilon}
\newcommand{\veps}{\varepsilon}
\newcommand{\lm}{\lambda}

\newcommand{\ol}{\overline}
\newcommand{\ul}{\underline}
\newcommand{\wt}{\widetilde}
\newcommand{\wh}{\widehat}

\let\oldvec\vec
\renewcommand{\vec}{\overrightarrow}

% Derivadas parciales
\newcommand{\del}[2]{\frac{\partial #1}{\partial #2}}
\newcommand{\Del}[3]{\frac{\partial^{#1} #2}{\partial #3^{#1}}}
\newcommand{\deld}[2]{\dfrac{\partial #1}{\partial #2}}
\newcommand{\Deld}[3]{\dfrac{\partial^{#1} #2}{\partial #3^{#1}}}


\newcommand{\AstIg}{\stackrel{(\ast)}{=}}
\newcommand{\Hop}{\stackrel{L'H\hat{o}pital}{=}}

\newcommand{\red}[1]{{\color{red}#1}} % Para integrales, destacar los cambios.

% Método de integración
\newcommand{\MetInt}[2]{
    \left[\begin{array}{c}
        #1 \\ #2
    \end{array}\right]
}

% Declarar aplicaciones
% 1. Nombre aplicación
% 2. Dominio
% 3. Codominio
% 4. Variable
% 5. Imagen de la variable
\newcommand{\Func}[5]{
    \begin{equation*}
        \begin{array}{rrll}
            #1:& #2 & \longrightarrow & #3\\
               & #4 & \longmapsto & #5
        \end{array}
    \end{equation*}
}

%------------------------------------------------------------------------

\let\oldRe\Re % save them in case they're every wanted
\let\oldIm\Im
\renewcommand{\Re}{\operatorname{Re}} % redefine them
\renewcommand{\Im}{\operatorname{Im}}
\DeclareMathOperator{\Log}{Log}
\DeclareMathOperator{\Arg}{Arg}
\DeclareMathOperator{\ord}{ord}
\DeclareMathOperator{\Ind}{Ind}
\DeclareMathOperator{\Fr}{Fr}
\DeclareMathOperator{\Res}{Res}


\usetikzlibrary{arrows.meta, decorations.markings} % Cargar las bibliotecas necesarias

% Configuración para las flechas
\tikzset{
    arrow at 1/3/.style={postaction={decorate},
        decoration={markings, mark=at position 0.33 with {\arrow{Stealth}}}},
    arrow at 2/3/.style={postaction={decorate},
        decoration={markings, mark=at position 0.66 with {\arrow{Stealth}}}}
}


\begin{document}

    % 1. Foto de fondo
    % 2. Título
    % 3. Encabezado Izquierdo
    % 4. Color de fondo
    % 5. Coord x del titulo
    % 6. Coord y del titulo
    % 7. Fecha

    
    % 1. Foto de fondo
% 2. Título
% 3. Encabezado Izquierdo
% 4. Color de fondo
% 5. Coord x del titulo
% 6. Coord y del titulo
% 7. Fecha

\newcommand{\portada}[7]{

    \portadaBase{#1}{#2}{#3}{#4}{#5}{#6}{#7}
    \portadaBook{#1}{#2}{#3}{#4}{#5}{#6}{#7}
}

\newcommand{\portadaExamen}[7]{

    \portadaBase{#1}{#2}{#3}{#4}{#5}{#6}{#7}
    \portadaArticle{#1}{#2}{#3}{#4}{#5}{#6}{#7}
}




\newcommand{\portadaBase}[7]{

    % Tiene la portada principal y la licencia Creative Commons
    
    % 1. Foto de fondo
    % 2. Título
    % 3. Encabezado Izquierdo
    % 4. Color de fondo
    % 5. Coord x del titulo
    % 6. Coord y del titulo
    % 7. Fecha
    
    
    \thispagestyle{empty}               % Sin encabezado ni pie de página
    \newgeometry{margin=0cm}        % Márgenes nulos para la primera página
    
    
    % Encabezado
    \fancyhead[L]{\helv #3}
    \fancyhead[R]{\helv \nouppercase{\leftmark}}
    
    
    \pagecolor{#4}        % Color de fondo para la portada
    
    \begin{figure}[p]
        \centering
        \transparent{0.3}           % Opacidad del 30% para la imagen
        
        \includegraphics[width=\paperwidth, keepaspectratio]{assets/#1}
    
        \begin{tikzpicture}[remember picture, overlay]
            \node[anchor=north west, text=white, opacity=1, font=\fontsize{60}{90}\selectfont\bfseries\sffamily, align=left] at (#5, #6) {#2};
            
            \node[anchor=south east, text=white, opacity=1, font=\fontsize{12}{18}\selectfont\sffamily, align=right] at (9.7, 3) {\textbf{\href{https://losdeldgiim.github.io/}{Los Del DGIIM}}};
            
            \node[anchor=south east, text=white, opacity=1, font=\fontsize{12}{15}\selectfont\sffamily, align=right] at (9.7, 1.8) {Doble Grado en Ingeniería Informática y Matemáticas\\Universidad de Granada};
        \end{tikzpicture}
    \end{figure}
    
    
    \restoregeometry        % Restaurar márgenes normales para las páginas subsiguientes
    \pagecolor{white}       % Restaurar el color de página
    
    
    \newpage
    \thispagestyle{empty}               % Sin encabezado ni pie de página
    \begin{tikzpicture}[remember picture, overlay]
        \node[anchor=south west, inner sep=3cm] at (current page.south west) {
            \begin{minipage}{0.5\paperwidth}
                \href{https://creativecommons.org/licenses/by-nc-nd/4.0/}{
                    \includegraphics[height=2cm]{assets/Licencia.png}
                }\vspace{1cm}\\
                Esta obra está bajo una
                \href{https://creativecommons.org/licenses/by-nc-nd/4.0/}{
                    Licencia Creative Commons Atribución-NoComercial-SinDerivadas 4.0 Internacional (CC BY-NC-ND 4.0).
                }\\
    
                Eres libre de compartir y redistribuir el contenido de esta obra en cualquier medio o formato, siempre y cuando des el crédito adecuado a los autores originales y no persigas fines comerciales. 
            \end{minipage}
        };
    \end{tikzpicture}
    
    
    
    % 1. Foto de fondo
    % 2. Título
    % 3. Encabezado Izquierdo
    % 4. Color de fondo
    % 5. Coord x del titulo
    % 6. Coord y del titulo
    % 7. Fecha


}


\newcommand{\portadaBook}[7]{

    % 1. Foto de fondo
    % 2. Título
    % 3. Encabezado Izquierdo
    % 4. Color de fondo
    % 5. Coord x del titulo
    % 6. Coord y del titulo
    % 7. Fecha

    % Personaliza el formato del título
    \pretitle{\begin{center}\bfseries\fontsize{42}{56}\selectfont}
    \posttitle{\par\end{center}\vspace{2em}}
    
    % Personaliza el formato del autor
    \preauthor{\begin{center}\Large}
    \postauthor{\par\end{center}\vfill}
    
    % Personaliza el formato de la fecha
    \predate{\begin{center}\huge}
    \postdate{\par\end{center}\vspace{2em}}
    
    \title{#2}
    \author{\href{https://losdeldgiim.github.io/}{Los Del DGIIM}}
    \date{Granada, #7}
    \maketitle
    
    \tableofcontents
}




\newcommand{\portadaArticle}[7]{

    % 1. Foto de fondo
    % 2. Título
    % 3. Encabezado Izquierdo
    % 4. Color de fondo
    % 5. Coord x del titulo
    % 6. Coord y del titulo
    % 7. Fecha

    % Personaliza el formato del título
    \pretitle{\begin{center}\bfseries\fontsize{42}{56}\selectfont}
    \posttitle{\par\end{center}\vspace{2em}}
    
    % Personaliza el formato del autor
    \preauthor{\begin{center}\Large}
    \postauthor{\par\end{center}\vspace{3em}}
    
    % Personaliza el formato de la fecha
    \predate{\begin{center}\huge}
    \postdate{\par\end{center}\vspace{5em}}
    
    \title{#2}
    \author{\href{https://losdeldgiim.github.io/}{Los Del DGIIM}}
    \date{Granada, #7}
    \thispagestyle{empty}               % Sin encabezado ni pie de página
    \maketitle
    \vfill
}
    \portadaExamen{ffccA4.jpg}{Variable Compleja I\\Examen XV}{Variable Compleja I. Examen XV}{MidnightBlue}{-9.5}{28}{2024-2025}{Arturo Olivares Martos}

    \begin{description}
        \item[Asignatura] Variable Compleja I.
        \item[Curso Académico] 2022-23.
        \item[Grado] Doble Grado en Ingeniería Informática y Matemáticas.
        \item[Grupo] Único.
        \item[Profesor] Javier Merí de la Maza.
        \item[Descripción] Convocatoria Ordinaria.
        \item[Fecha] 22 de Junio de 2023.
        \item[Duración] 3.5 horas.
    \end{description}
    \newpage

    \begin{ejercicio}[2.5 puntos]
        Integrando una conveniente función sobre la poligonal $\Gamma_R$ dada por
        $$[-R, R, R + \pi i, -R + \pi i, -R]$$
        con $R \in \bb{R}^+$, calcular la integral:
        \begin{equation*}
            \int_{-\infty}^{+\infty} \frac{\cos(x)}{e^x + e^{-x}} \, dx.
        \end{equation*}
    \end{ejercicio}

    \begin{ejercicio}[2.5 + 1.5 puntos]
        Sea $f \in \cc{H}(\bb{C}^*)$ y supongamos que $f$ diverge en $0$ y en $\infty$.
        Probar que $f$ se anula en algún punto de $\bb{C}^*$.
        \begin{itemize}
            \item \ul{Extra} Demostrar que, de hecho, $f$ se anula al menos dos veces (contando multiplicidad) y que tiene un número finito de ceros. 
        \end{itemize}
    \end{ejercicio}

    \begin{ejercicio}[2.5 puntos]
        Para cada $n \in \bb{N} \cup \{0\}$, sea $f_n : \bb{C} \to \bb{C}$ la función dada por
        $$f_n(z) = \int_n^{n+1} \frac{\cos(t + z^2) + \sen(t^2 - z)}{1 + t^4} \, dt$$
        para todo $z \in \bb{C}$.
        \begin{enumerate}
            \item Probar que $f_n \in \cc{H}(\bb{C})$.
            \item Probar que la serie de funciones entera $\sum\limits_{n\geq 0} f_n$ converge en $\bb{C}$ y que su suma es una función entera.
        \end{enumerate}
    \end{ejercicio}

    \begin{ejercicio}[2.5 puntos]
        Sean $f, g \in \cc{H}(\bb{C})$ de modo que
        $$f\left(g\left(\frac{1}{n}\right)\right) = \frac{1}{n^3}$$
        para todo $n \in \bb{N}$.
        Probar que una de las funciones es un polinomio de grado uno y que la otra es un polinomio de grado tres.
    \end{ejercicio}



    \newpage
    \setcounter{ejercicio}{0}




    \begin{ejercicio}[2.5 puntos]\label{ej:14.14}
        Integrando una conveniente función sobre la poligonal $\Gamma_R$ dada por
        $$[-R, R, R + \pi i, -R + \pi i, -R]$$
        con $R \in \bb{R}^+$, calcular la integral:
        \begin{equation*}
            \int_{-\infty}^{+\infty} \frac{\cos(x)}{e^x + e^{-x}} \, dx.
        \end{equation*}



        Veamos en primer lugar en qué puntos se anula el denominador de mi función a integrar:
    \begin{align*}
        e^z + e^{-z} &= 0 \implies e^{2z} = -1 \implies 2z \in \Log(-1) = i\Arg(-1)=i\left(\pi + 2\pi \bb{Z}\right)
        \implies z \in i\pi \left(\nicefrac{1}{2} + \bb{Z}\right).
    \end{align*}

    Sea por tanto $A= i\pi \left(\nicefrac{1}{2} + \bb{Z}\right)$. Definimos la función:
    \Func{f}{\bb{C}\setminus A}{\bb{C}}{z}{\frac{e^{iz}}{e^z + e^{-z}}}

    Notemos que $f\in \cc{H}(\bb{C}\setminus A)$, y que $A'=\emptyset$, por lo que podemos aplicar el Teorema de los Residuos. Como $\bb{C}$ es homológicamente conexo, podemos aplicar el Teorema de los Residuos para cualquier ciclo $\Sigma$ en $\bb{C}\setminus A$. Para todo $R\in \bb{R}^+$, consideramos la poligonal siguiente:
    \begin{align*}
        \Gamma_R &= [-R, R] + [R, R + \pi i] - [-R + \pi i, R + \pi i] - [-R, -R + \pi i]
    \end{align*}
    representada en la Figura~\ref{fig:ej:14.14}, donde:
    \Func{[-R, R]}{[-R, R]}{\bb{C}}{t}{t}
    \Func{[R, R + \pi i]}{[0, \pi]}{\bb{C}}{t}{R + i t}
    \Func{[-R + \pi i, R + \pi i]}{[-R, R]}{\bb{C}}{t}{t + i\pi}
    \Func{[-R, -R + \pi i]}{[0, \pi]}{\bb{C}}{t}{-R + i t}
    \begin{figure}
        \centering
        \begin{tikzpicture}
            \begin{axis}[
                axis lines=middle,
                xlabel={$x$},
                ylabel={$y$},
                xtick=\empty,
                ytick=\empty,
                xmin=-4, xmax=4,
                ymin=-1, ymax=4,
                axis equal image,
                clip=false,
            ]
                \def\B{pi}
                \def\A{-\B}
                \def\R{3}

                % Polos
                \foreach \k in {0} {
                    \addplot[only marks, mark=*, mark options={fill=red}, samples=1] coordinates {(0, \k*\B + \B*0.5)} 
                        node[right, font=\footnotesize] {$i\left(\nicefrac{1}{2} + 0\right)\pi$};
                }

                % [-R, R]
                \draw[thick, blue, arrow at 1/3, arrow at 2/3] (\A, 0) -- (\B, 0)
                    node[pos=0.7, below] {$[-R, R]$};

                % [R, R + pi i]
                \draw[thick, blue, arrow at 1/3, arrow at 2/3] (\B, 0) -- (\B, \B)
                    node[pos=0.5, right] {$[R, R + \pi i]$};

                % [-R + pi i, R + pi i]
                \draw[thick, blue, arrow at 1/3, arrow at 2/3] (\B, \B) -- (\A, \B)
                    node[pos=1.0, above] {$-[-R + \pi i, R + \pi i]$};

                % [-R, -R + pi i]
                \draw[thick, blue, arrow at 1/3, arrow at 2/3] (\A, \B) -- (\A, 0)
                    node[pos=0.5, left] {$-[-R,-R+\pi i]$};

                % Puntos de unión. Coordenadas:
                \addplot[only marks, mark=*, mark options={fill=blue}, samples=1] coordinates {(\B, 0) (\B, \B) (\A, \B) (\A, 0)};
            \end{axis}
        \end{tikzpicture}
        \caption{Poligonal de integración $\Gamma_R$ del Ejercicio~\ref{ej:14.14}.}
        \label{fig:ej:14.14}
    \end{figure}

    Por el Teorema de los Residuos, tenemos que:
    \begin{align*}
        \int_{\Gamma_R} f(z) \, dz &= 2\pi i\sum_{z_0\in A}\Res(f,z_0)\Ind_{\Gamma_R}(z_0).
    \end{align*}

    Calculemos ahora los índices de los polos. Para cada $k\in \bb{Z}^*$, tenemos que:
    \begin{align*}
        \left|\Im\left(i\left(\nicefrac{1}{2} + k\right)\pi\right)\right| = \left|\left(\nicefrac{1}{2} + k\right)\pi\right| > \pi
    \end{align*}
    Por tanto, para todo $R\in \bb{R}^+$, tenemos que:
    \begin{align*}
        \Ind_{\Gamma_R}\left(i\left(\nicefrac{1}{2} + k\right)\pi\right) &= 0 \quad \text{para todo } k\in \bb{Z}^*\\
        \Ind_{\Gamma_R}\left(i\left(\nicefrac{1}{2}\right)\pi\right) &= 1.
    \end{align*}

    Por tanto, tenemos que:
    \begin{align*}
        \int_{\Gamma_R} f(z) \, dz &= 2\pi i\Res(f, i\left(\nicefrac{1}{2}\right)\pi).
    \end{align*}

    Antes de calcular el residuo, calculemos las integrales resultantes. Tenemos que:
    \begin{align*}
        \int_{[-R, R]} f(z) \, dz &= \int_{-R}^{R} \frac{e^{iz}}{e^z + e^{-z}} \, dz
    \end{align*}
    Tomando límite con $R\to +\infty$, la parte que nos interesa es la parte real. Por tanto, vamos por buen camino. Calculamos el resto de las integrales:
    \begin{align*}
        \left|\int_{[R, R + \pi i]} f(z) \, dz\right| &\leq \pi\cdot \sup\left\{\left|\dfrac{e^{iz}}{e^z + e^{-z}}\right| : z\in [R, R + \pi i]\right\}
    \end{align*}
    donde, para todo $z\in [R, R + \pi i]^*$, tenemos que:
    \begin{align*}
        |e^{iz}| &= e^{-\Im z}\leq e^0=1\\
        |e^z + e^{-z}| &\geq \left||e^z| - |e^{-z}|\right| = \left|e^{\Re z} - e^{-\Re z}\right| = \left|e^{R} - e^{-R}\right| = e^{R} - e^{-R}.
    \end{align*}

    Por tanto, tenemos que:
    \begin{align*}
        \left|\int_{[R, R + \pi i]} f(z) \, dz\right| &\leq \pi\cdot \frac{1}{e^{R} - e^{-R}}.
    \end{align*}
    Como esta expresión es válida para cualquier $R > 0$, podemos hacer $R \to +\infty$ y tenemos que:
    \begin{align*}
        \lim_{R\to+\infty} \int_{[R, R + \pi i]} f(z) \, dz &= 0.
    \end{align*}

    Veamos ahora qué ocurre con la integral sobre el segmento $[-R, -R + \pi i]$:
    \begin{align*}
        \left|\int_{[-R, -R + \pi i]} f(z) \, dz\right| &\leq \pi\cdot \sup\left\{\left|\dfrac{e^{iz}}{e^z + e^{-z}}\right| : z\in [-R, -R + \pi i]\right\}
    \end{align*}
    donde, para todo $z\in [-R, -R + \pi i]^*$, tenemos que:
    \begin{align*}
        |e^{iz}| &= e^{-\Im z}\leq e^0=1\\
        |e^z + e^{-z}| &\geq \left||e^z| - |e^{-z}|\right| = \left|e^{\Re z} - e^{-\Re z}\right| = \left|e^{-R} - e^{R}\right| = e^{R} - e^{-R}.
    \end{align*}
    Por tanto, tenemos que:
    \begin{align*}
        \left|\int_{[-R, -R + \pi i]} f(z) \, dz\right| &\leq \pi\cdot \frac{1}{e^{R} - e^{-R}}.
    \end{align*}
    Como esta expresión es válida para cualquier $R > 0$, podemos hacer $R \to +\infty$ y tenemos que:
    \begin{align*}
        \lim_{R\to+\infty} \int_{[-R, -R + \pi i]} f(z) \, dz &= 0.
    \end{align*}




    Veamos ahora qué ocurre con la integral sobre el segmento $[-R + \pi i, R + \pi i]$:
    \begin{align*}
        \int_{[-R + \pi i, R + \pi i]} f(z) \, dz &= \int_{-R}^{R} f(t + i\pi) \, dt
        = \int_{-R}^{R} \frac{e^{i(t + i\pi)}}{e^{t + i\pi} + e^{-(t + i\pi)}} \, dt
        = \int_{-R}^{R} \frac{e^{it}e^{-\pi}}{e^{t}e^{i\pi} + e^{-t}e^{-i\pi}} \, dt
        =\\&= e^{-\pi}\int_{-R}^{R} \frac{e^{it}}{- e^t - e^{-t}} \, dt
        = -e^{-\pi}\int_{-R}^{R} \frac{e^{it}}{e^t + e^{-t}} \, dt
        = -e^{-\pi}\int_{-R}^{R} f(t) \, dt.
    \end{align*}

    Por tanto, uniendo todas las integrales que hemos calculado, y tomando el límite con $R\to +\infty$, tenemos que:
    \begin{align*}
        2\pi i\Res(f, i\left(\nicefrac{1}{2}\right)\pi) &= \left(1+e^{-\pi}\right)\int_{-\infty}^{+\infty} f(z) \, dz
    \end{align*}

    Ahora calculemos el residuo en el punto $i\left(\nicefrac{1}{2}\right)\pi$:
    \begin{align*}
        \lim_{z\to i\cdot \frac{\pi}{2}} \left(z - i\cdot \frac{\pi}{2}\right)f(z) &= \lim_{z\to i\cdot \frac{\pi}{2}} \left(z - i\cdot \frac{\pi}{2}\right)\cdot \frac{e^{iz}}{e^z + e^{-z}}
    \end{align*}

    Por el Teorema de la Regla de L'Hôpital, tenemos que:
    \begin{align*}
        \lim_{z\to i\cdot \frac{\pi}{2}} \left(z - i\cdot \frac{\pi}{2}\right)f(z) &= e^{-\frac{\pi}{2}}\cdot \lim_{z\to i\cdot \frac{\pi}{2}} \frac{1}{e^z - e^{-z}}
        = e^{-\frac{\pi}{2}}\cdot \dfrac{1}{e^{i\cdot \frac{\pi}{2}} - e^{-i\cdot \frac{\pi}{2}}}
        = e^{-\frac{\pi}{2}}\cdot \dfrac{1}{i - (-i)} = e^{-\frac{\pi}{2}}\cdot \dfrac{1}{2i}
    \end{align*}

    Por tanto, sabemos que $f$ tiene un polo simple en $i\cdot \frac{\pi}{2}$, y que:
    \begin{align*}
        \Res\left(f, i\cdot \frac{\pi}{2}\right) &= e^{-\frac{\pi}{2}}\cdot \dfrac{1}{2i}.
    \end{align*}

    Por tanto, tenemos que:
    \begin{align*}
        \int_{-\infty}^{+\infty} f(z) \, dz &= \dfrac{2\pi i\Res\left(f, i\cdot \frac{\pi}{2}\right)}{1 + e^{-\pi}}
        = \dfrac{\pi e^{-\frac{\pi}{2}}}{1 + e^{-\pi}}.
    \end{align*}

    Por tanto, como buscamos la parte real de la integral anterior, tenemos que:
    \begin{align*}
        \int_{-\infty}^{+\infty} \frac{\cos(x)}{e^x + e^{-x}} \, dx &= \Re\left(\int_{-\infty}^{+\infty} f(z) \, dz\right) = \Re\left(\dfrac{\pi e^{-\frac{\pi}{2}}}{1 + e^{-\pi}}\right)
        = \dfrac{\pi e^{-\frac{\pi}{2}}}{1 + e^{-\pi}}
    \end{align*}
    \end{ejercicio}

    \begin{ejercicio}[2.5 + 1.5 puntos]
        Sea $f \in \cc{H}(\bb{C}^*)$ y supongamos que $f$ diverge en $0$ y en $\infty$.
        Probar que $f$ se anula en algún punto de $\bb{C}^*$.
        \begin{itemize}
            \item \ul{Extra} Demostrar que, de hecho, $f$ se anula al menos dos veces (contando multiplicidad) y que tiene un número finito de ceros. 
        \end{itemize}


        \begin{description}
            \item[Opción Complicada y sin el Extra]~\\
            

                Sea $f \in \cc{H}(\bb{C}^*)$ tal que:
                \begin{equation*}
                    \lim_{z\to 0} f(z) = +\infty \qquad \text{y} \qquad \lim_{z\to +\infty} f(z) = +\infty.
                \end{equation*}
        
                Supongamos por reducción al absurdo que $f$ no tiene ceros. Entonces, podemos definir:
                \Func{g}{\bb{C}^*}{\bb{C}}{z}{\frac{1}{f(z)}}
        
                Notemos que $g$ es holomorfa en $\bb{C}^*$. Veamos cómo definirla en el origen:
                \begin{equation*}
                    \lim_{z\to 0} g(z) = \lim_{z\to 0} \frac{1}{f(z)} = 0
                \end{equation*}
                donde hemos usado que $f(z) \to +\infty$ cuando $z\to 0$. Por tanto, podemos definir (notemos el abuso de notación):
                \Func{g}{\bb{C}}{\bb{C}}{z}{\begin{cases}
                    0 & \text{si } z = 0 \\
                    \frac{1}{f(z)} & \text{si } z \in \bb{C}^*.
                \end{cases}}
        
                Como $g\in \cc{H}(\bb{C}^*)$ y $g$ continua en el origen, por el Teorema de Extensión de Riemann, tenemos que $g\in \cc{H}(\bb{C})$. Veamos ahora que $g$ es acotada en $\bb{C}$. Para todo $R\in \bb{R}^+$, como $g$ es continua y $\ol{D}(0, R)$ es compacto, tenemos que $g\left(\ol{D}(0, R)\right)$ es acotado. Además, veamos el comportamiento de $g$ en el infinito:
                \begin{equation*}
                    \lim_{z\to +\infty} g(z) = \lim_{z\to +\infty} \frac{1}{f(z)} = 0
                \end{equation*}
                donde hemos usado que $f(z) \to +\infty$ cuando $z\to +\infty$. Por tanto, tenemos que $g$ es acotada en $\bb{C}$. Por el Teorema de Liouville, tenemos que $g$ es constante, pero esto es una contradicción, puesto que $f$ no es constante. Por tanto, $f$ tiene al menos un cero. Por tanto, $\exists z_0\in \bb{C}^*$ y $g\in \cc{H}(\bb{C})$ con $g(z_0)\neq 0$ tal que:
                \begin{equation*}
                    f(z)=(z-z_0)g(z)
                \end{equation*}
        
                Veamos ahora que el número de ceros de $f$ es finito y mayor o igual que 2 (contando multiplicidad). Supongamos que $f$ tiene un número infinito de ceros (que sabemos que será numerable). Consideramos la siguiente sucesión de ceros:
                \begin{equation*}
                    \left\{z_n\right\}_{n\in\bb{N}}\subset Z(f)
                \end{equation*}
        
                Como $f$ diverge en $+\infty$, sabemos que $\{z_n\}_{n\in\bb{N}}$ está acotada, por lo que admite una parcial $\{z_{n_k}\}_{k\in\bb{N}}$ que converge a un punto $z_0\in \bb{C}$:
                \begin{equation*}
                    \{z_{n_k}\}_{k\in\bb{N}}\to z_0\in \bb{C}.
                \end{equation*}
        
                Supongamos que $z_0=0$. Consideramos la sucesión de imágenes de los ceros:
                \begin{equation*}
                    \left\{f(z_{n_k})\right\}_{k\in\bb{N}} = \left\{f(0)\right\}_{k\in\bb{N}} = \left\{0\right\}_{k\in\bb{N}}\to 0
                \end{equation*}
        
                No obstante, esto contradice el hecho de que $f$ diverge en el origen. Por tanto, $z_0\neq 0$. Como hemos encontrado un punto de acumulación de $Z(f)$ en $\bb{C}^*$, tenemos que $f$ es idénticamente nula, lo que contradice el hecho de que $f$ diverge en el origen. Por tanto, $f$ tiene un número finito de ceros.\\
            \item[Opción Directa y que, además, incluye el apartado extra]~\\
            
            Como $f$ diverge en el origen, sabemos que el $0$ es un polo de orden $k\in \bb{N}$ de $f$. Por tanto, $\exists \Psi\in \cc{H}(\bb{C})$ tal que:
            \begin{equation*}
                f(z) = \dfrac{\Psi(z)}{z^k}\qquad \forall z\in \bb{C}^*
            \end{equation*}
            donde $\Psi(0)\neq 0$. De esta forma:
            \begin{equation*}
                \Psi(z) = z^k f(z)\qquad \forall z\in \bb{C}^*.
            \end{equation*}
    
            Puesto que conocemos el comportamiento de $f$ en el infinito, sabemos que $\Psi(z)$ diverge en el infinito. Por tanto, como $\Psi\in \cc{H}(\bb{C})$ y $\Psi$ diverge en el infinito, por el Corolario del Corolario del Teorema de Casorati, tenemos que $\Psi$ es un polinomio. Estudiemos ahora el grado de $\Psi$. Haciendo uso de que $f$ diverge en el infinito, tenemos que:
            \begin{equation*}
                \lim_{z\to +\infty} f(z) = \lim_{z\to +\infty} \dfrac{\Psi(z)}{z^k} = +\infty
            \end{equation*}
            Este es un límite de un cociente de polinomios que diverge, por lo que el grado del numerador es mayor que el grado del denominador. Por tanto, se tiene $\deg \Psi = m\in \bb{N}$, donde $m>k$. Por el Teorema Fundamental del Álgebra, sabemos que $\Psi$ tiene $m$ raíces. Como sabemos que:
            \begin{equation*}
                f(z) = \dfrac{\Psi(z)}{z^k}\qquad \forall z\in \bb{C}^*,
            \end{equation*}
    
            Sabemos que $Z(f) = Z(\Psi)$, y por tanto $f$ tiene $m$ ceros.
        \end{description}
    \end{ejercicio}

    \begin{ejercicio}[2.5 puntos]
        Para cada $n \in \bb{N} \cup \{0\}$, sea $f_n : \bb{C} \to \bb{C}$ la función dada por
        $$f_n(z) = \int_n^{n+1} \frac{\cos(t + z^2) + \sen(t^2 - z)}{1 + t^4} \, dt$$
        para todo $z \in \bb{C}$.
        \begin{enumerate}
            \item Probar que $f_n \in \cc{H}(\bb{C})$.
            
            Definimos la siguiente función:
            \Func{\Phi}{[n,n+1]\times \bb{C}}{\bb{C}}{(t,z)}{\dfrac{\cos(t + z^2) + \sen(t^2 - z)}{1 + t^4}}

            Veamos que $\Phi$ está bien definida; es decir, que no se anula el denominador. Como $t\geq n\geq 0$, tenemos que $1 + t^4 \geq 1 > 0$. Por tanto, $\Phi$ está bien definida.
            Por tanto, $\Phi$ es continua y, fijado $t\in [n,n+1]$, tenemos que la aplicación $z\mapsto \Phi(t,z)$ es holomorfa en $\bb{C}$.
            Por tanto, por el Teorema de Holomorfía de Integrales dependientes de Parámetros, tenemos que $f_n\in \cc{H}(\bb{C})$.
            \item Probar que la serie de funciones entera $\sum\limits_{n\geq 0} f_n$ converge en $\bb{C}$ y que su suma es una función entera.
            
            Sea $K\subset \bb{C}$ compacto. Veamos que la serie de funciones converge uniformemente en $K$. Para cada $n\in \bb{N}$, $z\in K$, tenemos que:
            \begin{align*}
                \left|f_n(z)\right|
                &= \left|\int_n^{n+1} \frac{\cos(t + z^2) + \sen(t^2 - z)}{1 + t^4} \, dt\right|\\
                &\leq \sup\left\{\left|\frac{\cos(t + z^2) + \sen(t^2 - z)}{1 + t^4}\right| : t\in [n,n+1]\right\} 
            \end{align*}
            donde hemos hecho uso de que:
            \begin{align*}
                |1+ t^4| &= 1+t^4\geq 1+n^4\\
                |\cos(t + z^2) &+ \sen(t^2 - z)|\leq |\cos(t + z^2)| + |\sen(t^2 - z)|=\\&=|\cos(t)\cos(z^2) - \sen(t)\sen(z^2)| + |\sen(t^2)\cos(z) + \cos(t^2)\sen(z)|
                \leq\\&\leq |\cos(t)\cos(z^2)| + |\sen(t)\sen(z^2)| + |\sen(t^2)\cos(z)| + |\cos(t^2)\sen(z)|
                \leq\\&\leq |\cos(z^2)| + |\sen(z^2)| + |\cos(z)| + |\sen(z)|
            \end{align*}
            En importante destacar que no podemos acotar el seno y el coseno complejos.

            Como $K$ es compacto y las funciones seno, coseno y módulo son continuas, tenemos que $\exists M\in \bb{R}^+$ tal que:
            \begin{align*}
                M = \max\left\{|\cos(z^2)| + |\sen(z^2)| + |\cos(z)| + |\sen(z)| : z\in K\right\}.
            \end{align*}
            Por tanto, tenemos que:
            \begin{align*}
                \left|f_n(z)\right| &\leq \frac{M}{1+n^4}\leq \frac{M}{n^4}\qquad \forall n\in \bb{N} \text{ y } z\in K.
            \end{align*}

            Por tanto, por el Test de Weierstrass, tenemos que la serie de funciones $\sum\limits_{n\geq 0} f_n$ converge uniformemente en $K$.\\

            Como $f_n\in \cc{H}(\bb{C})$ para todo $n\in \bb{N}$, por el Teorema de la Convergencia de Weierstrass, tenemos que la suma es una función entera.
        \end{enumerate}
    \end{ejercicio}

    \begin{ejercicio}[2.5 puntos]
        Sean $f, g \in \cc{H}(\bb{C})$ de modo que
        $$f\left(g\left(\frac{1}{n}\right)\right) = \frac{1}{n^3}$$
        para todo $n \in \bb{N}$.
        Probar que una de las funciones es un polinomio de grado uno y que la otra es un polinomio de grado tres.\\

        Definimos el siguiente conjunto:
        \begin{equation*}
            A = \left\{\dfrac{1}{n} : n\in \bb{N}\right\}
        \end{equation*}

        Como $A'=\{0\}\subset \bb{C}$, podemos aplicar el Pincipio de Identidad, y deducir que:
        \begin{equation*}
            f\left(g(z)\right) = z^3\qquad \forall z\in \bb{C}
        \end{equation*}

        Supongamos que $g$ es una función entera no polinómica. Por el Corolario del Teorema de Casorati, $\exists \{z_n\}_{n\in \bb{N}}\subset \bb{C}$ con $\{z_n\}\to \infty$ tal que:
        \begin{equation*}
            \{g(z_n)\} \to 0.
        \end{equation*}

        Ese hecho, junto con la continuidad de $f$, nos permite deducir que:
        \begin{align*}
            \{f(g(z_n))\}\to f(0).
        \end{align*}

        Por otro lado, $\{z_n\}\to \infty$, junto con la continuidad de $f, g$ y que $f(g(z))=z^3$, nos permite deducir que:
        \begin{align*}
            \{f(g(z_n))\}\to \infty.
        \end{align*}

        Por tanto, llegamos a que la sueción $\{f(g(z_n))\}$ es a la vez convergente y divergente, lo que es una contradicción. Por tanto, $g$ es un polinomio.\\

        Suponemos ahora que $f$ no es un polinomio. Por el Corolario del Teorema de Casorati, $\exists \{w_n\}_{n\in \bb{N}}\subset \bb{C}$ con $\{w_n\}\to \infty$ tal que:
        \begin{equation*}
            \{f(w_n)\} \to 0.
        \end{equation*}

        Ahora, haciendo uso de que $g$ es sobreyectiva por ser un polinomio (gracias al Teorema Fundamental del Álgebra), podemos encontrar una sucesión $\{z_n\}_{n\in \bb{N}}\subset \bb{C}$ tal que:
        \begin{equation*}
            g(z_n) = w_n\qquad \forall n\in \bb{N}.
        \end{equation*}
        Por tanto, tenemos que:
        \begin{align*}
            \{f(g(z_n))\} &= \{f(w_n)\} \to 0.
        \end{align*}

        Por otro lado, supongamos que $\{z_n\}\to \alpha\in \bb{C}$. Entonces, por la continuidad de $g$ tenemos que:
        \begin{align*}
            \{g(z_n)\} &= \{w_n\} \to g(\alpha)
        \end{align*}
        En contradicción con que $\{w_n\}\to \infty$. Por tanto, $\{z_n\}\to \infty$. Por la continuidad de $f, g$ y que $f(g(z))=z^3$, tenemos que:
        \begin{align*}
            \{f(g(z_n))\} &= \{z_n^3\} \to \infty.
        \end{align*}

        Por tanto, llegamos a que la sueción $\{f(g(z_n))\}$ es a la vez convergente y divergente, lo que es una contradicción. Por tanto, $f$ es un polinomio.\\

        Por tanto, $f$ y $g$ son polinomios. Como $f(g(z))=z^3$, tenemos que:
        \begin{align*}
            \deg(f)\cdot \deg(g) &= 3
            \Longrightarrow
            \{\deg(f), \deg(g)\} = \{1, 3\}.
        \end{align*}
        Por tanto, una de las funciones es un polinomio de grado uno y la otra es un polinomio de grado tres.
    \end{ejercicio}
    
    
\end{document}