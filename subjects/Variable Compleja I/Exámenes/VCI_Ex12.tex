\documentclass[12pt]{article}

% Idioma y codificación
\usepackage[spanish, es-tabla]{babel}       %es-tabla para que se titule "Tabla"
\usepackage[utf8]{inputenc}

% Márgenes
\usepackage[a4paper,top=3cm,bottom=2.5cm,left=3cm,right=3cm]{geometry}

% Comentarios de bloque
\usepackage{verbatim}

% Paquetes de links
\usepackage[hidelinks]{hyperref}    % Permite enlaces
\usepackage{url}                    % redirecciona a la web

% Más opciones para enumeraciones
\usepackage{enumitem}

% Personalizar la portada
\usepackage{titling}

% Paquetes de tablas
\usepackage{multirow}


%------------------------------------------------------------------------

%Paquetes de figuras
\usepackage{caption}
\usepackage{subcaption} % Figuras al lado de otras
\usepackage{float}      % Poner figuras en el sitio indicado H.


% Paquetes de imágenes
\usepackage{graphicx}       % Paquete para añadir imágenes
\usepackage{transparent}    % Para manejar la opacidad de las figuras

% Paquete para usar colores
\usepackage[dvipsnames]{xcolor}
\usepackage{pagecolor}      % Para cambiar el color de la página

% Habilita tamaños de fuente mayores
\usepackage{fix-cm}

% Para los gráficos
\usepackage{tikz}

% Para poder situar los nodos en los grafos
\usetikzlibrary{positioning}


%------------------------------------------------------------------------

% Paquetes de matemáticas
\usepackage{mathtools, amsfonts, amssymb, mathrsfs}
\usepackage[makeroom]{cancel}     % Simplificar tachando
\usepackage{polynom}    % Divisiones y Ruffini
\usepackage{units} % Para poner fracciones diagonales con \nicefrac

\usepackage{pgfplots}   %Representar funciones
\pgfplotsset{compat=1.18}  % Versión 1.18

\usepackage{tikz-cd}    % Para usar diagramas de composiciones
\usetikzlibrary{calc}   % Para usar cálculo de coordenadas en tikz

%Definición de teoremas, etc.
\usepackage{amsthm}
%\swapnumbers   % Intercambia la posición del texto y de la numeración

\theoremstyle{plain}

\makeatletter
\@ifclassloaded{article}{
  \newtheorem{teo}{Teorema}[section]
}{
  \newtheorem{teo}{Teorema}[chapter]  % Se resetea en cada chapter
}
\makeatother

\newtheorem{coro}{Corolario}[teo]           % Se resetea en cada teorema
\newtheorem{prop}[teo]{Proposición}         % Usa el mismo contador que teorema
\newtheorem{lema}[teo]{Lema}                % Usa el mismo contador que teorema

\theoremstyle{remark}
\newtheorem*{observacion}{Observación}

\theoremstyle{definition}

\makeatletter
\@ifclassloaded{article}{
  \newtheorem{definicion}{Definición} [section]     % Se resetea en cada chapter
}{
  \newtheorem{definicion}{Definición} [chapter]     % Se resetea en cada chapter
}
\makeatother

\newtheorem*{notacion}{Notación}
\newtheorem*{ejemplo}{Ejemplo}
\newtheorem*{ejercicio*}{Ejercicio}             % No numerado
\newtheorem{ejercicio}{Ejercicio} [section]     % Se resetea en cada section


% Modificar el formato de la numeración del teorema "ejercicio"
\renewcommand{\theejercicio}{%
  \ifnum\value{section}=0 % Si no se ha iniciado ninguna sección
    \arabic{ejercicio}% Solo mostrar el número de ejercicio
  \else
    \thesection.\arabic{ejercicio}% Mostrar número de sección y número de ejercicio
  \fi
}


% \renewcommand\qedsymbol{$\blacksquare$}         % Cambiar símbolo QED
%------------------------------------------------------------------------

% Paquetes para encabezados
\usepackage{fancyhdr}
\pagestyle{fancy}
\fancyhf{}

\newcommand{\helv}{ % Modificación tamaño de letra
\fontfamily{}\fontsize{12}{12}\selectfont}
\setlength{\headheight}{15pt} % Amplía el tamaño del índice


%\usepackage{lastpage}   % Referenciar última pag   \pageref{LastPage}
\fancyfoot[C]{\thepage}

%------------------------------------------------------------------------

% Conseguir que no ponga "Capítulo 1". Sino solo "1."
\makeatletter
\@ifclassloaded{book}{
  \renewcommand{\chaptermark}[1]{\markboth{\thechapter.\ #1}{}} % En el encabezado
    
  \renewcommand{\@makechapterhead}[1]{%
  \vspace*{50\p@}%
  {\parindent \z@ \raggedright \normalfont
    \ifnum \c@secnumdepth >\m@ne
      \huge\bfseries \thechapter.\hspace{1em}\ignorespaces
    \fi
    \interlinepenalty\@M
    \Huge \bfseries #1\par\nobreak
    \vskip 40\p@
  }}
}
\makeatother

%------------------------------------------------------------------------
% Paquetes de cógido
\usepackage{minted}
\renewcommand\listingscaption{Código fuente}

\usepackage{fancyvrb}
% Personaliza el tamaño de los números de línea
\renewcommand{\theFancyVerbLine}{\small\arabic{FancyVerbLine}}

% Estilo para C++
\newminted{cpp}{
    frame=lines,
    framesep=2mm,
    baselinestretch=1.2,
    linenos,
    escapeinside=||
}

% para minted
\definecolor{LightGray}{rgb}{0.95,0.95,0.92}
\setminted{
    linenos=true,
    stepnumber=5,
    numberfirstline=true,
    autogobble,
    breaklines=true,
    breakautoindent=true,
    breaksymbolleft=,
    breaksymbolright=,
    breaksymbolindentleft=0pt,
    breaksymbolindentright=0pt,
    breaksymbolsepleft=0pt,
    breaksymbolsepright=0pt,
    fontsize=\footnotesize,
    bgcolor=LightGray,
    numbersep=10pt
}


\usepackage{listings} % Para incluir código desde un archivo

\renewcommand\lstlistingname{Código Fuente}
\renewcommand\lstlistlistingname{Índice de Códigos Fuente}

% Definir colores
\definecolor{vscodepurple}{rgb}{0.5,0,0.5}
\definecolor{vscodeblue}{rgb}{0,0,0.8}
\definecolor{vscodegreen}{rgb}{0,0.5,0}
\definecolor{vscodegray}{rgb}{0.5,0.5,0.5}
\definecolor{vscodebackground}{rgb}{0.97,0.97,0.97}
\definecolor{vscodelightgray}{rgb}{0.9,0.9,0.9}

% Configuración para el estilo de C similar a VSCode
\lstdefinestyle{vscode_C}{
  backgroundcolor=\color{vscodebackground},
  commentstyle=\color{vscodegreen},
  keywordstyle=\color{vscodeblue},
  numberstyle=\tiny\color{vscodegray},
  stringstyle=\color{vscodepurple},
  basicstyle=\scriptsize\ttfamily,
  breakatwhitespace=false,
  breaklines=true,
  captionpos=b,
  keepspaces=true,
  numbers=left,
  numbersep=5pt,
  showspaces=false,
  showstringspaces=false,
  showtabs=false,
  tabsize=2,
  frame=tb,
  framerule=0pt,
  aboveskip=10pt,
  belowskip=10pt,
  xleftmargin=10pt,
  xrightmargin=10pt,
  framexleftmargin=10pt,
  framexrightmargin=10pt,
  framesep=0pt,
  rulecolor=\color{vscodelightgray},
  backgroundcolor=\color{vscodebackground},
}

%------------------------------------------------------------------------

% Comandos definidos
\newcommand{\bb}[1]{\mathbb{#1}}
\newcommand{\cc}[1]{\mathcal{#1}}

% I prefer the slanted \leq
\let\oldleq\leq % save them in case they're every wanted
\let\oldgeq\geq
\renewcommand{\leq}{\leqslant}
\renewcommand{\geq}{\geqslant}

% Si y solo si
\newcommand{\sii}{\iff}

% Letras griegas
\newcommand{\eps}{\epsilon}
\newcommand{\veps}{\varepsilon}
\newcommand{\lm}{\lambda}

\newcommand{\ol}{\overline}
\newcommand{\ul}{\underline}
\newcommand{\wt}{\widetilde}
\newcommand{\wh}{\widehat}

\let\oldvec\vec
\renewcommand{\vec}{\overrightarrow}

% Derivadas parciales
\newcommand{\del}[2]{\frac{\partial #1}{\partial #2}}
\newcommand{\Del}[3]{\frac{\partial^{#1} #2}{\partial #3^{#1}}}
\newcommand{\deld}[2]{\dfrac{\partial #1}{\partial #2}}
\newcommand{\Deld}[3]{\dfrac{\partial^{#1} #2}{\partial #3^{#1}}}


\newcommand{\AstIg}{\stackrel{(\ast)}{=}}
\newcommand{\Hop}{\stackrel{L'H\hat{o}pital}{=}}

\newcommand{\red}[1]{{\color{red}#1}} % Para integrales, destacar los cambios.

% Método de integración
\newcommand{\MetInt}[2]{
    \left[\begin{array}{c}
        #1 \\ #2
    \end{array}\right]
}

% Declarar aplicaciones
% 1. Nombre aplicación
% 2. Dominio
% 3. Codominio
% 4. Variable
% 5. Imagen de la variable
\newcommand{\Func}[5]{
    \begin{equation*}
        \begin{array}{rrll}
            #1:& #2 & \longrightarrow & #3\\
               & #4 & \longmapsto & #5
        \end{array}
    \end{equation*}
}

%------------------------------------------------------------------------

\let\oldRe\Re % save them in case they're every wanted
\let\oldIm\Im
\renewcommand{\Re}{\operatorname{Re}} % redefine them
\renewcommand{\Im}{\operatorname{Im}}
\DeclareMathOperator{\Log}{Log}
\DeclareMathOperator{\Arg}{Arg}
\DeclareMathOperator{\ord}{ord}
\DeclareMathOperator{\Ind}{Ind}
\DeclareMathOperator{\Fr}{Fr}
\DeclareMathOperator{\Res}{Res}
\begin{document}

    % 1. Foto de fondo
    % 2. Título
    % 3. Encabezado Izquierdo
    % 4. Color de fondo
    % 5. Coord x del titulo
    % 6. Coord y del titulo
    % 7. Fecha

    
    % 1. Foto de fondo
% 2. Título
% 3. Encabezado Izquierdo
% 4. Color de fondo
% 5. Coord x del titulo
% 6. Coord y del titulo
% 7. Fecha

\newcommand{\portada}[7]{

    \portadaBase{#1}{#2}{#3}{#4}{#5}{#6}{#7}
    \portadaBook{#1}{#2}{#3}{#4}{#5}{#6}{#7}
}

\newcommand{\portadaExamen}[7]{

    \portadaBase{#1}{#2}{#3}{#4}{#5}{#6}{#7}
    \portadaArticle{#1}{#2}{#3}{#4}{#5}{#6}{#7}
}




\newcommand{\portadaBase}[7]{

    % Tiene la portada principal y la licencia Creative Commons
    
    % 1. Foto de fondo
    % 2. Título
    % 3. Encabezado Izquierdo
    % 4. Color de fondo
    % 5. Coord x del titulo
    % 6. Coord y del titulo
    % 7. Fecha
    
    
    \thispagestyle{empty}               % Sin encabezado ni pie de página
    \newgeometry{margin=0cm}        % Márgenes nulos para la primera página
    
    
    % Encabezado
    \fancyhead[L]{\helv #3}
    \fancyhead[R]{\helv \nouppercase{\leftmark}}
    
    
    \pagecolor{#4}        % Color de fondo para la portada
    
    \begin{figure}[p]
        \centering
        \transparent{0.3}           % Opacidad del 30% para la imagen
        
        \includegraphics[width=\paperwidth, keepaspectratio]{assets/#1}
    
        \begin{tikzpicture}[remember picture, overlay]
            \node[anchor=north west, text=white, opacity=1, font=\fontsize{60}{90}\selectfont\bfseries\sffamily, align=left] at (#5, #6) {#2};
            
            \node[anchor=south east, text=white, opacity=1, font=\fontsize{12}{18}\selectfont\sffamily, align=right] at (9.7, 3) {\textbf{\href{https://losdeldgiim.github.io/}{Los Del DGIIM}}};
            
            \node[anchor=south east, text=white, opacity=1, font=\fontsize{12}{15}\selectfont\sffamily, align=right] at (9.7, 1.8) {Doble Grado en Ingeniería Informática y Matemáticas\\Universidad de Granada};
        \end{tikzpicture}
    \end{figure}
    
    
    \restoregeometry        % Restaurar márgenes normales para las páginas subsiguientes
    \pagecolor{white}       % Restaurar el color de página
    
    
    \newpage
    \thispagestyle{empty}               % Sin encabezado ni pie de página
    \begin{tikzpicture}[remember picture, overlay]
        \node[anchor=south west, inner sep=3cm] at (current page.south west) {
            \begin{minipage}{0.5\paperwidth}
                \href{https://creativecommons.org/licenses/by-nc-nd/4.0/}{
                    \includegraphics[height=2cm]{assets/Licencia.png}
                }\vspace{1cm}\\
                Esta obra está bajo una
                \href{https://creativecommons.org/licenses/by-nc-nd/4.0/}{
                    Licencia Creative Commons Atribución-NoComercial-SinDerivadas 4.0 Internacional (CC BY-NC-ND 4.0).
                }\\
    
                Eres libre de compartir y redistribuir el contenido de esta obra en cualquier medio o formato, siempre y cuando des el crédito adecuado a los autores originales y no persigas fines comerciales. 
            \end{minipage}
        };
    \end{tikzpicture}
    
    
    
    % 1. Foto de fondo
    % 2. Título
    % 3. Encabezado Izquierdo
    % 4. Color de fondo
    % 5. Coord x del titulo
    % 6. Coord y del titulo
    % 7. Fecha


}


\newcommand{\portadaBook}[7]{

    % 1. Foto de fondo
    % 2. Título
    % 3. Encabezado Izquierdo
    % 4. Color de fondo
    % 5. Coord x del titulo
    % 6. Coord y del titulo
    % 7. Fecha

    % Personaliza el formato del título
    \pretitle{\begin{center}\bfseries\fontsize{42}{56}\selectfont}
    \posttitle{\par\end{center}\vspace{2em}}
    
    % Personaliza el formato del autor
    \preauthor{\begin{center}\Large}
    \postauthor{\par\end{center}\vfill}
    
    % Personaliza el formato de la fecha
    \predate{\begin{center}\huge}
    \postdate{\par\end{center}\vspace{2em}}
    
    \title{#2}
    \author{\href{https://losdeldgiim.github.io/}{Los Del DGIIM}}
    \date{Granada, #7}
    \maketitle
    
    \tableofcontents
}




\newcommand{\portadaArticle}[7]{

    % 1. Foto de fondo
    % 2. Título
    % 3. Encabezado Izquierdo
    % 4. Color de fondo
    % 5. Coord x del titulo
    % 6. Coord y del titulo
    % 7. Fecha

    % Personaliza el formato del título
    \pretitle{\begin{center}\bfseries\fontsize{42}{56}\selectfont}
    \posttitle{\par\end{center}\vspace{2em}}
    
    % Personaliza el formato del autor
    \preauthor{\begin{center}\Large}
    \postauthor{\par\end{center}\vspace{3em}}
    
    % Personaliza el formato de la fecha
    \predate{\begin{center}\huge}
    \postdate{\par\end{center}\vspace{5em}}
    
    \title{#2}
    \author{\href{https://losdeldgiim.github.io/}{Los Del DGIIM}}
    \date{Granada, #7}
    \thispagestyle{empty}               % Sin encabezado ni pie de página
    \maketitle
    \vfill
}
    \portadaExamen{ffccA4.jpg}{Variable Compleja I\\Examen XII}{Variable Compleja I. Examen XII}{MidnightBlue}{-9.5}{28}{2024-2025}{Arturo Olivares Martos}

    \begin{description}
        \item[Asignatura] Variable Compleja I.
        \item[Curso Académico] 2024-25.
        \item[Grado] Grado en Matemáticas y Doble Grado en Matemáticas y Física.
        \item[Grupo] Único.
        \item[Profesor] Javier Merí de la Maza.
        \item[Descripción] Convocatoria Ordinaria.
        \item[Fecha] 17 de Enero de 2025.
        \item[Duración] 3.5 horas.
    \end{description}
    \newpage

    \begin{ejercicio}[2.5 puntos]
        Sea $\Omega$ un dominio y sean $f, g \in \cc{H}(\Omega)$ de modo que $\ol{f}g\in \cc{H}(\Omega)$.
        Probar que $g \equiv 0$ en $\Omega$ o $f$ es constante en $\Omega$.        
    \end{ejercicio}

    \begin{ejercicio}[2.5 puntos]
        Sea $f \in \cc{H}(D(0, 1))$ no constante, continua en $\ol{D}(0, 1)$ y verificando que $|f(z)| = 1$ para cada $z \in \cc{C}$ con $|z| = 1$.
        \begin{enumerate}
            \item Probar que $f$ tiene un número finito (no nulo) de ceros en $D(0, 1)$.
            \item Probar que $f\left(\ol{D}(0, 1)\right) = \ol{D}(0, 1)$.
        \end{enumerate}
    \end{ejercicio}

    \begin{ejercicio}[2.5 puntos]
        Para cada $n \in \bb{N}$ tomamos $a_n = \nicefrac{1}{n}$ y consideramos la función:
        \Func{f_n}{\bb{C}\setminus \{a_n\}}{\bb{C}}{z}{\frac{1}{z - a_n}}
        \begin{enumerate}
            \item Si $A = \ol{\{a_n : n \in \bb{N}\}}$, probar que la serie de funciones
            \[
                \sum_{n \geq 1} \frac{f_n(z)}{n^n}
            \]
            converge absolutamente en todo punto del dominio $\Omega = \bb{C}\setminus A$ y uniformemente en cada subconjunto compacto contenido en $\Omega$.
            \item Deducir que la función dada por
            \[
                f(z) = \sum_{n=1}^{\infty} \frac{f_n(z)}{n^n}
            \]
            es holomorfa en $\Omega$ y estudiar sus singularidades aisladas.

            \item (Extra) Probar que para cada $\delta\in \bb{R}^+$ el conjunto $f\left(D(0,\delta)\setminus A\right)$ es denso en $\bb{C}$.
        \end{enumerate}
    \end{ejercicio}

    \begin{ejercicio}[2.5 puntos]
        Sea $f$ holomorfa en $\bb{C}\setminus\{-1, 1\}$. Supongamos que $-1$ y $1$ son polos de $f$ y que
        \[
            \Res(f, 1) = -\Res(f, -1).
        \]
        Probar que $f$ admite primitiva en $\bb{C} \setminus [-1, 1]$.
    \end{ejercicio}



    \newpage
    \setcounter{ejercicio}{0}

    \begin{ejercicio}[2.5 puntos]
        Sea $\Omega$ un dominio y sean $f, g \in \cc{H}(\Omega)$ de modo que $\ol{f}g\in \cc{H}(\Omega)$.
        Probar que $g \equiv 0$ en $\Omega$ o $f$ es constante en $\Omega$.\\

        Supongamos que $g \not\equiv 0$ en $\Omega$. Entonces, $\exists z_0 \in \Omega$ tal que $g(z_0) \neq 0$. Por continuidad de $g$, $\exists\delta\in \bb{R}^+$ tal que $g(z) \neq 0$ para todo $z \in D(z_0, \delta)$, por lo que consideramos:
        \Func{\nicefrac{1}{g}}{D(z_0, \delta)}{\bb{C}}{z}{\frac{1}{g(z)}}

        Como $g$ es holomorfa en $D(z_0, \delta)$, $\nicefrac{1}{g}\in \cc{H}(D(z_0, \delta))$. Por tanto, restringuiéndonos a $D(z_0, \delta)$, tenemos que:
        \begin{equation*}
            \ol{f}g\cdot \frac{1}{g} = \ol{f} \in \cc{H}(D(z_0, \delta)).
        \end{equation*}

        Por tanto, $\ol{f}\in \cc{H}(D(z_0, \delta))$ y, por tanto:
        \begin{equation*}
            \Re f = \dfrac{f + \ol{f}}{2} \in \cc{H}(D(z_0, \delta)).
        \end{equation*}

        Como $\Re f\in \cc{H}(D(z_0, \delta))$ y tiene parte imaginaria constante (en particular, nula), tenemos que $\Re f$ es constante en $D(z_0, \delta)$. De ahí, concluimos que $f$ es constante en $D(z_0, \delta)$. Puesto que $\Omega$ es un dominio y $D(z_0, \delta) \subseteq \Omega$, como $D(z_0, \delta)$ no es numerable, por el Principio de Identidad, tenemos que $f$ es constante en $\Omega$.


    \end{ejercicio}

    \begin{ejercicio}[2.5 puntos]
        Sea $f \in \cc{H}(D(0, 1))$ no constante, continua en $\ol{D}(0, 1)$ y verificando que $|f(z)| = 1$ para cada $z \in \bb{C}$ con $|z| = 1$.
        \begin{enumerate}
            \item Probar que $f$ tiene un número finito (no nulo) de ceros en $D(0, 1)$.\\
            
            En primer lugar, queremos ver que $Z(f)\neq \emptyset$. Por el principio del módulo máximo, como $f$ es continua en $\ol{D}(0, 1)$ y holomorfa en $D(0, 1)$, tenemos que:
            \begin{equation*}
                \max\{|f(z)| : z \in \ol{D}(0, 1)\} = \max\{|f(z)| : |z|=1\} = 1.
            \end{equation*}

            Por otro lado, por por ser $\ol{D}(0, 1)$ compacto y $f$ y el módulo de $f$ continuo, $\exists z_0 \in \ol{D}(0, 1)$ tal que:
            \begin{equation*}
                |f(z_0)| = \min\{|f(z)| : z \in \ol{D}(0, 1)\}
            \end{equation*}

            Buscamos ahora aplicar el Principio del Módulo Mínimo en $D(0, 1)$, para lo que hemos de demostrar que $|z_0| < 1$. Supongamos que $|z_0| = 1$. Entonces, por hipótesis, tenemos que $|f(z_0)| = 1$, por lo que:
            \begin{equation*}
                1 = |f(z_0)| = \min\{|f(z)| : z \in \ol{D}(0, 1)\} = \max\{|f(z)| : z \in \ol{D}(0, 1)\} = 1.
            \end{equation*}

            Por tanto, $|f(z)| = 1$ para todo $z \in \ol{D}(0, 1)$, por lo que $|f|$ es constante en $\ol{D}(0, 1)$. Por las consecuencias de Cauchy-Riemann, tenemos que $f$ es constante en $D(0, 1)$, lo que contradice la hipótesis de que $f$ no es constante. Por tanto, $|z_0| < 1$ y, por tanto, $z_0 \in D(0, 1)$. Por el Principio del Módulo Mínimo, como $f$ no es constante, tenemos que $f(z_0) = 0$, por lo que $Z(f) \neq \emptyset$.\\

            Veamos ahora que $Z(f)$ es finito. Supongamos por el contrario que $Z(f)$ es infinito, y consideramos una sucesión $\{z_n\}_{n \in \bb{N}}$ con $z_n\in D(0, 1)$ y $f(z_n) = 0$ para todo $n \in \bb{N}$. Por ser una sucesión acotada, admite una parcial convergente, digamos $\{z_{n_k}\}_{k \in \bb{N}}$, que converge a un punto $w\in \ol{D}(0,1)$. Por continuidad de $f$ en $\ol{D}(0, 1)$, tenemos que:
            \begin{equation*}
                f(w) = \lim_{k\to\infty} f(z_{n_k}) = \lim_{k\to\infty} 0 = 0.
            \end{equation*}
            Por tanto, $w\in Z(f)$ y, por tanto, $w\in D(0, 1)$. No obstante, hemos probado que $Z(f)$ tiene un punto de acumulación $w\in D(0, 1)$, algo que contradice el principio de los ceros aislados. Por tanto, $Z(f)$ es finito y no nulo.
            \item Probar que $f\left(\ol{D}(0, 1)\right) = \ol{D}(0, 1)$.
            
            Demostraremos mediante doble inclusión.
            \begin{description}
                \item[$\subseteq$)] Consideramos la siguiente igualdad, ya mencionada anteriormente:
                \begin{equation*}
                    \max\{|f(z)| : z \in \ol{D}(0, 1)\} = \max\{|f(z)| : |z|=1\} = 1.
                \end{equation*}
                Por tanto, $|f(z)| \leq 1$ para todo $z \in \ol{D}(0, 1)$, por lo que $f\left(\ol{D}(0, 1)\right) \subseteq \ol{D}(0, 1)$.

                \item[$\supseteq$)] Sea $w_0 \in D(0, 1)$, y por reducción al absurdo supongamos que $w_0 \notin f\left(\ol{D}(0, 1)\right)$. Consideramos el siguiente conjunto:
                \begin{equation*}
                    A = \left\{\lm \leq 1\mid \lm\ w_0\in f\left(\ol{D}(0, 1)\right)\right\}.
                \end{equation*}

                Por el apartado anterior, como $0\in f\left(\ol{D}(0, 1)\right)$, tenemos que $A\neq \emptyset$, y por el Axioma del Supremo podemos considerar $\lm_0\in \bb{R}$ tal que:
                \begin{equation*}
                    \lm_0 = \sup A = \sup\{\lm \leq 1\mid \lm\ w_0\in f\left(\ol{D}(0, 1)\right)\}.
                \end{equation*}

                Como $w_0 \notin f\left(\ol{D}(0, 1)\right)$, tenemos que $1\notin A$, luego $\lm_0 \leq 1$. Por la continuidad de $f$ en el compacto, tenemos que $\lm_0\in A$ con $\lm_0 < 1$. Por tanto, $|\lm_0 w_0| < 1$, y $\lm_0 w_0 \in D(0, 1)$. Por tanto, y uniéndolo a que $\lm_0 w_0 \in f\left(\ol{D}(0, 1)\right)$, tenemos que $\exists z_0 \in D(0, 1)$ tal que:
                \begin{equation*}
                    f(z_0) = \lm_0 w_0.
                \end{equation*}

                Como $f$ no es constante, por el Teorema de la Aplicación Abierta tenemos que $f\left(D(0, 1)\right)$ es abierto en $\bb{C}$. Por tanto, como $f(z_0)\in f\left(D(0, 1)\right)$, existe $\delta\in \bb{R}^+$ tal que:
                \begin{equation*}
                    D(f(z_0), \delta) \subseteq f\left(D(0, 1)\right).
                \end{equation*}

                No obstante, entonces veamos que $w_0\left(\lm_0 + \delta\right)\in D(f(z_0), \delta)$, ya que:
                \begin{equation*}
                    \left|f(z_0) - w_0\left(\lm_0 + \delta\right)\right| = \left|\lm_0 w_0 - w_0\left(\lm_0 + \delta\right)\right| = |w_0\delta| < \delta.
                \end{equation*}
                donde la última igualdad se cumple porque $|w_0|<1$. Por tanto, $\lm_0+\delta \in A$, lo que contradice el hecho de que $\lm_0 = \sup A$. Por tanto, hemos llegado a una contradicción, por lo que:
                \begin{equation*}
                    D(0, 1) \subseteq f\left(\ol{D}(0, 1)\right).
                \end{equation*}

                Tomando cerrados, teniendo en cuenta que el conjunto $f\left(\ol{D}(0, 1)\right)$ es cerrado por ser compacto, tenemos que:
                \begin{equation*}
                    \ol{D}(0, 1) \subseteq f\left(\ol{D}(0, 1)\right).
                \end{equation*}
            \end{description}
        \end{enumerate}
    \end{ejercicio}

    \begin{ejercicio}[2.5 puntos]
        Para cada $n \in \bb{N}$ tomamos $a_n = \nicefrac{1}{n}$ y consideramos la función:
        \Func{f_n}{\bb{C}\setminus \{a_n\}}{\bb{C}}{z}{\frac{1}{z - a_n}}
        \begin{enumerate}
            \item Si $A = \ol{\{a_n : n \in \bb{N}\}}$, probar que la serie de funciones
            \[
                \sum_{n \geq 1} \frac{f_n(z)}{n^n}
            \]
            converge absolutamente en todo punto del dominio $\Omega = \bb{C}\setminus A$ y uniformemente en cada subconjunto compacto contenido en $\Omega$.\\

            Sea $K \subseteq \Omega=\bb{C}\setminus A$ compacto. como $K,A$ son conjuntos compactos disjuntos, definimos $d(K,A)>0$. Por tanto:
            \begin{equation*}
                \left|\dfrac{f_n(z)}{n^n}\right| = \frac{1}{n^n}\cdot \dfrac{1}{|z - a_n|} \leq \frac{1}{n^n\cdot d(K,A)}\qquad \forall z\in K, n\in \bb{N}.
            \end{equation*}

            Veamos que podemos aplicar el Test de Weierstrass a la serie de término general $n^{-n}$. Por el Criterio de la raíz, tenemos que:
            \begin{align*}
                \left\{\frac{n^n}{{(n+1)}^{n+1}}\right\} &= \left\{\frac{1}{n+1}\left(\frac{n}{n+1}\right)^n\right\}\to 0
            \end{align*}

            Por tanto, la serie $\sum\limits_{n=1}^{\infty} \frac{1}{n^n}$ converge. Por tanto, por el Test de Weierstrass, tenemos que la serie de funciones $\sum\limits_{n=1}^{\infty} \frac{f_n(z)}{n^n}$ converge uniformemente en $K$.

            Generalizando, tenemos que la serie de funciones $\sum\limits_{n=1}^{\infty} \frac{f_n(z)}{n^n}$ converge uniformemente en cada compacto $K\subseteq \Omega$, y por tanto, converge absolutamente en todo punto de $\Omega$.


            \item Deducir que la función dada por
            \[
                f(z) = \sum_{n=1}^{\infty} \frac{f_n(z)}{n^n}
            \]
            es holomorfa en $\Omega$ y estudiar sus singularidades aisladas.\\

            Por el Teorema de Convergencia de Weierstrass, gracias al apartado anterior deducimos que $f\in \cc{H}(\Omega)$. Sus posibles singularidades aisladas son los puntos de $A$. Comencemos por estudiar el punto $0\in A$.
            \begin{equation*}
                \lim_{z\to 0} f(z) = \lim_{z\to 0} \sum_{n=1}^{\infty} \frac{f_n(z)}{n^n} = \sum_{n=1}^{\infty} \frac{f_n(0)}{n^n} = \sum_{n=1}^{\infty} \dfrac{1}{ - n^n\cdot a_n} = \sum_{n=1}^{\infty} \dfrac{-n}{n^{n}} = -\sum_{n=1}^{\infty} \dfrac{1}{n^{n-1}}.
            \end{equation*}

            Por tanto, tenemos que $0$ es un punto regular de $f$, ya que:
            \begin{equation*}
                \lim_{z\to 0} zf(z) = 0
            \end{equation*}

            Fijado ahora $k\in \bb{N}$, consideramos el punto $a_k\in A$. Consideramos la serie $\sum\limits_{\substack{n\geq 1\\n\neq k}} \frac{f_n(z)}{n^n}$, que converge uniformemente en cada compacto de $\Omega\cup\{a_k\}$. Por tanto, podemos aplicar el Teorema de lA Convergencia de Weierstrass para deducir que dicha serie es holomorfa en $\Omega\cup\{a_k\}$. Por tanto, tenemos que:
            \begin{align*}
                \lim_{z\to a_k} f(z) &= \lim_{z\to a_k} (z-a_k) \dfrac{f_k(z)}{k^k} + (z-a_k) \sum_{\substack{n\geq 1\\n\neq k}} \frac{f_n(z)}{n^n} = \lim_{z\to a_k} \dfrac{1}{k^k} + (z-a_k) \sum_{\substack{n\geq 1\\n\neq k}} \frac{f_n(z)}{n^n} =\\&= \dfrac{1}{k^k} + 0 = \dfrac{1}{k^k}\neq 0.
            \end{align*}

            Por tanto, $a_k$ es un polo de orden $1$ de $f$.

            \item (Extra) Probar que para cada $\delta\in \bb{R}^+$ el conjunto $f\left(D(0,\delta)\setminus A\right)$ es denso en $\bb{C}$.
            
            Suponemos por reducción al absurdo que existe $\delta > 0$ de modo que $f\left(D(0, \delta) \setminus A\right)$ no es denso en $\bb{C}$. Entonces existen $w \in \bb{C}$ y $r > 0$ tales que $D(w, r)\cap f\left(D(0, \delta)\setminus A\right) = \emptyset$, es decir, para cada $z \in D(0, \delta) \setminus A$ se cumple que $|f(z) - w| > r$. Definimos la función:
            \Func{g}{D(0, \delta) \setminus \{0\}}{\bb{C}}{z}{\begin{cases}
                \frac{1}{f(z) - w} & \text{si } z \in D(0, \delta) \setminus A\\
                \lim\limits_{z\to a_n} g(z) = 0 & \text{si } z = a_n\in A
            \end{cases}}

            Notemos que el límite anterior vale cero porque $f$ diverge en cada $a_n$ por tener un polo. Como $g$ es holomorfa en $D(0, \delta) \setminus A$ y es continua en cada $a_n$, el Teorema de Extensión de Riemann nos dice que $g$ es holomorfa en $D(0, \delta)\setminus \{0\}$. Además, se tiene que $|g(z)| \leq \nicefrac{1}{r}$ para cada $z \in D(0, \delta)\setminus \{0\}$, es decir, $g$ está acotada en un entorno reducido de $0$. Aplicando de nuevo el Teorema de Extensión de Riemann, obtenemos que $g$ es derivable en cero y, como $g(a_n) = 0$ para cada $n \in \bb{N}$, deducimos que $g(0) = 0$. Entonces el conjunto de los ceros de $g$ tiene un punto de acumulación en $D(0, 1)$ y el Principio de Identidad nos dice que $g \equiv 0$ en $D(0, 1)$ pero esto es imposible por la definición de $\frac{1}{f(z) - w}$.
        \end{enumerate}
    \end{ejercicio}

    \begin{ejercicio}[2.5 puntos]
        Sea $f$ holomorfa en $\bb{C}\setminus\{-1, 1\}$. Supongamos que $-1$ y $1$ son polos de $f$ y que
        \[
            \Res(f, 1) = -\Res(f, -1).
        \]
        Probar que $f$ admite primitiva en $\bb{C} \setminus [-1, 1]$.\\

        Por la caracterización de las funciones primitivas, queremos ver que, para todo camino cerrado $\gamma$ contenido en $\bb{C}\setminus [-1, 1]$, se cumple que:
        \begin{equation*}
            \int_{\gamma} f(z) \, dz = 0.
        \end{equation*}

        Sea pues $\gamma$ un camino cerrado contenido en $\bb{C}\setminus [-1, 1]\subset \bb{C}\setminus \{-1, 1\}$. Como $f\in \cc{H}(\bb{C}\setminus \{-1, 1\})$, $\{-1,1\}'=\emptyset$ y $\bb{C}$ es homológicamente conexo, por el Teorema de los Residuos, tenemos que:
        \begin{align*}
            \int_{\gamma} f(z) \, dz &= 2\pi i \sum_{z_k\in \{-1, 1\}} \Ind_{\gamma}(z_k) \Res(f, z_k)
        \end{align*}

        Como $\gamma*\subset \bb{C}\setminus [-1, 1]$ y $[-1,1]$ es un conjunto conexo, existe $V$ componente conexa de $\bb{C}\setminus [-1, 1]$ tal que $\{-1,1\}\subset [-1,1]\subset V$. Por tanto, como el índice en una componente conexa es constante, tenemos que:
        \begin{equation*}
            \Ind_{\gamma}(-1) = \Ind_{\gamma}(1)
        \end{equation*}

        Por tanto:
        \begin{align*}
            \int_{\gamma} f(z) \, dz &= 2\pi i \Ind_{\gamma}(1) \sum_{z_k\in \{-1, 1\}}\Res(f, z_k)
            = 2\pi i \Ind_{\gamma}(1) \left(\Res(f, 1) + \Res(f, -1)\right)\\
            &= 2\pi i \Ind_{\gamma}(1) \left(\Res(f, 1) - \Res(f, 1)\right) = 0.
        \end{align*}

        Como $\gamma$ era un camino cerrado arbitrario contenido en $\bb{C}\setminus [-1, 1]$, hemos probado que $f$ admite primitiva en $\bb{C}\setminus [-1, 1]$.
    \end{ejercicio}



\end{document}