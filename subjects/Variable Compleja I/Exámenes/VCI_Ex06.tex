\documentclass[12pt]{article}

% Idioma y codificación
\usepackage[spanish, es-tabla]{babel}       %es-tabla para que se titule "Tabla"
\usepackage[utf8]{inputenc}

% Márgenes
\usepackage[a4paper,top=3cm,bottom=2.5cm,left=3cm,right=3cm]{geometry}

% Comentarios de bloque
\usepackage{verbatim}

% Paquetes de links
\usepackage[hidelinks]{hyperref}    % Permite enlaces
\usepackage{url}                    % redirecciona a la web

% Más opciones para enumeraciones
\usepackage{enumitem}

% Personalizar la portada
\usepackage{titling}

% Paquetes de tablas
\usepackage{multirow}


%------------------------------------------------------------------------

%Paquetes de figuras
\usepackage{caption}
\usepackage{subcaption} % Figuras al lado de otras
\usepackage{float}      % Poner figuras en el sitio indicado H.


% Paquetes de imágenes
\usepackage{graphicx}       % Paquete para añadir imágenes
\usepackage{transparent}    % Para manejar la opacidad de las figuras

% Paquete para usar colores
\usepackage[dvipsnames]{xcolor}
\usepackage{pagecolor}      % Para cambiar el color de la página

% Habilita tamaños de fuente mayores
\usepackage{fix-cm}

% Para los gráficos
\usepackage{tikz}

% Para poder situar los nodos en los grafos
\usetikzlibrary{positioning}


%------------------------------------------------------------------------

% Paquetes de matemáticas
\usepackage{mathtools, amsfonts, amssymb, mathrsfs}
\usepackage[makeroom]{cancel}     % Simplificar tachando
\usepackage{polynom}    % Divisiones y Ruffini
\usepackage{units} % Para poner fracciones diagonales con \nicefrac

\usepackage{pgfplots}   %Representar funciones
\pgfplotsset{compat=1.18}  % Versión 1.18

\usepackage{tikz-cd}    % Para usar diagramas de composiciones
\usetikzlibrary{calc}   % Para usar cálculo de coordenadas en tikz

%Definición de teoremas, etc.
\usepackage{amsthm}
%\swapnumbers   % Intercambia la posición del texto y de la numeración

\theoremstyle{plain}

\makeatletter
\@ifclassloaded{article}{
  \newtheorem{teo}{Teorema}[section]
}{
  \newtheorem{teo}{Teorema}[chapter]  % Se resetea en cada chapter
}
\makeatother

\newtheorem{coro}{Corolario}[teo]           % Se resetea en cada teorema
\newtheorem{prop}[teo]{Proposición}         % Usa el mismo contador que teorema
\newtheorem{lema}[teo]{Lema}                % Usa el mismo contador que teorema

\theoremstyle{remark}
\newtheorem*{observacion}{Observación}

\theoremstyle{definition}

\makeatletter
\@ifclassloaded{article}{
  \newtheorem{definicion}{Definición} [section]     % Se resetea en cada chapter
}{
  \newtheorem{definicion}{Definición} [chapter]     % Se resetea en cada chapter
}
\makeatother

\newtheorem*{notacion}{Notación}
\newtheorem*{ejemplo}{Ejemplo}
\newtheorem*{ejercicio*}{Ejercicio}             % No numerado
\newtheorem{ejercicio}{Ejercicio} [section]     % Se resetea en cada section


% Modificar el formato de la numeración del teorema "ejercicio"
\renewcommand{\theejercicio}{%
  \ifnum\value{section}=0 % Si no se ha iniciado ninguna sección
    \arabic{ejercicio}% Solo mostrar el número de ejercicio
  \else
    \thesection.\arabic{ejercicio}% Mostrar número de sección y número de ejercicio
  \fi
}


% \renewcommand\qedsymbol{$\blacksquare$}         % Cambiar símbolo QED
%------------------------------------------------------------------------

% Paquetes para encabezados
\usepackage{fancyhdr}
\pagestyle{fancy}
\fancyhf{}

\newcommand{\helv}{ % Modificación tamaño de letra
\fontfamily{}\fontsize{12}{12}\selectfont}
\setlength{\headheight}{15pt} % Amplía el tamaño del índice


%\usepackage{lastpage}   % Referenciar última pag   \pageref{LastPage}
\fancyfoot[C]{\thepage}

%------------------------------------------------------------------------

% Conseguir que no ponga "Capítulo 1". Sino solo "1."
\makeatletter
\@ifclassloaded{book}{
  \renewcommand{\chaptermark}[1]{\markboth{\thechapter.\ #1}{}} % En el encabezado
    
  \renewcommand{\@makechapterhead}[1]{%
  \vspace*{50\p@}%
  {\parindent \z@ \raggedright \normalfont
    \ifnum \c@secnumdepth >\m@ne
      \huge\bfseries \thechapter.\hspace{1em}\ignorespaces
    \fi
    \interlinepenalty\@M
    \Huge \bfseries #1\par\nobreak
    \vskip 40\p@
  }}
}
\makeatother

%------------------------------------------------------------------------
% Paquetes de cógido
\usepackage{minted}
\renewcommand\listingscaption{Código fuente}

\usepackage{fancyvrb}
% Personaliza el tamaño de los números de línea
\renewcommand{\theFancyVerbLine}{\small\arabic{FancyVerbLine}}

% Estilo para C++
\newminted{cpp}{
    frame=lines,
    framesep=2mm,
    baselinestretch=1.2,
    linenos,
    escapeinside=||
}

% para minted
\definecolor{LightGray}{rgb}{0.95,0.95,0.92}
\setminted{
    linenos=true,
    stepnumber=5,
    numberfirstline=true,
    autogobble,
    breaklines=true,
    breakautoindent=true,
    breaksymbolleft=,
    breaksymbolright=,
    breaksymbolindentleft=0pt,
    breaksymbolindentright=0pt,
    breaksymbolsepleft=0pt,
    breaksymbolsepright=0pt,
    fontsize=\footnotesize,
    bgcolor=LightGray,
    numbersep=10pt
}


\usepackage{listings} % Para incluir código desde un archivo

\renewcommand\lstlistingname{Código Fuente}
\renewcommand\lstlistlistingname{Índice de Códigos Fuente}

% Definir colores
\definecolor{vscodepurple}{rgb}{0.5,0,0.5}
\definecolor{vscodeblue}{rgb}{0,0,0.8}
\definecolor{vscodegreen}{rgb}{0,0.5,0}
\definecolor{vscodegray}{rgb}{0.5,0.5,0.5}
\definecolor{vscodebackground}{rgb}{0.97,0.97,0.97}
\definecolor{vscodelightgray}{rgb}{0.9,0.9,0.9}

% Configuración para el estilo de C similar a VSCode
\lstdefinestyle{vscode_C}{
  backgroundcolor=\color{vscodebackground},
  commentstyle=\color{vscodegreen},
  keywordstyle=\color{vscodeblue},
  numberstyle=\tiny\color{vscodegray},
  stringstyle=\color{vscodepurple},
  basicstyle=\scriptsize\ttfamily,
  breakatwhitespace=false,
  breaklines=true,
  captionpos=b,
  keepspaces=true,
  numbers=left,
  numbersep=5pt,
  showspaces=false,
  showstringspaces=false,
  showtabs=false,
  tabsize=2,
  frame=tb,
  framerule=0pt,
  aboveskip=10pt,
  belowskip=10pt,
  xleftmargin=10pt,
  xrightmargin=10pt,
  framexleftmargin=10pt,
  framexrightmargin=10pt,
  framesep=0pt,
  rulecolor=\color{vscodelightgray},
  backgroundcolor=\color{vscodebackground},
}

%------------------------------------------------------------------------

% Comandos definidos
\newcommand{\bb}[1]{\mathbb{#1}}
\newcommand{\cc}[1]{\mathcal{#1}}

% I prefer the slanted \leq
\let\oldleq\leq % save them in case they're every wanted
\let\oldgeq\geq
\renewcommand{\leq}{\leqslant}
\renewcommand{\geq}{\geqslant}

% Si y solo si
\newcommand{\sii}{\iff}

% Letras griegas
\newcommand{\eps}{\epsilon}
\newcommand{\veps}{\varepsilon}
\newcommand{\lm}{\lambda}

\newcommand{\ol}{\overline}
\newcommand{\ul}{\underline}
\newcommand{\wt}{\widetilde}
\newcommand{\wh}{\widehat}

\let\oldvec\vec
\renewcommand{\vec}{\overrightarrow}

% Derivadas parciales
\newcommand{\del}[2]{\frac{\partial #1}{\partial #2}}
\newcommand{\Del}[3]{\frac{\partial^{#1} #2}{\partial #3^{#1}}}
\newcommand{\deld}[2]{\dfrac{\partial #1}{\partial #2}}
\newcommand{\Deld}[3]{\dfrac{\partial^{#1} #2}{\partial #3^{#1}}}


\newcommand{\AstIg}{\stackrel{(\ast)}{=}}
\newcommand{\Hop}{\stackrel{L'H\hat{o}pital}{=}}

\newcommand{\red}[1]{{\color{red}#1}} % Para integrales, destacar los cambios.

% Método de integración
\newcommand{\MetInt}[2]{
    \left[\begin{array}{c}
        #1 \\ #2
    \end{array}\right]
}

% Declarar aplicaciones
% 1. Nombre aplicación
% 2. Dominio
% 3. Codominio
% 4. Variable
% 5. Imagen de la variable
\newcommand{\Func}[5]{
    \begin{equation*}
        \begin{array}{rrll}
            #1:& #2 & \longrightarrow & #3\\
               & #4 & \longmapsto & #5
        \end{array}
    \end{equation*}
}

%------------------------------------------------------------------------

\let\oldRe\Re % save them in case they're every wanted
\let\oldIm\Im
\renewcommand{\Re}{\operatorname{Re}} % redefine them
\renewcommand{\Im}{\operatorname{Im}}
\DeclareMathOperator{\Log}{Log}
\DeclareMathOperator{\Arg}{Arg}

\begin{document}

    % 1. Foto de fondo
    % 2. Título
    % 3. Encabezado Izquierdo
    % 4. Color de fondo
    % 5. Coord x del titulo
    % 6. Coord y del titulo
    % 7. Fecha

    
    % 1. Foto de fondo
% 2. Título
% 3. Encabezado Izquierdo
% 4. Color de fondo
% 5. Coord x del titulo
% 6. Coord y del titulo
% 7. Fecha

\newcommand{\portada}[7]{

    \portadaBase{#1}{#2}{#3}{#4}{#5}{#6}{#7}
    \portadaBook{#1}{#2}{#3}{#4}{#5}{#6}{#7}
}

\newcommand{\portadaExamen}[7]{

    \portadaBase{#1}{#2}{#3}{#4}{#5}{#6}{#7}
    \portadaArticle{#1}{#2}{#3}{#4}{#5}{#6}{#7}
}




\newcommand{\portadaBase}[7]{

    % Tiene la portada principal y la licencia Creative Commons
    
    % 1. Foto de fondo
    % 2. Título
    % 3. Encabezado Izquierdo
    % 4. Color de fondo
    % 5. Coord x del titulo
    % 6. Coord y del titulo
    % 7. Fecha
    
    
    \thispagestyle{empty}               % Sin encabezado ni pie de página
    \newgeometry{margin=0cm}        % Márgenes nulos para la primera página
    
    
    % Encabezado
    \fancyhead[L]{\helv #3}
    \fancyhead[R]{\helv \nouppercase{\leftmark}}
    
    
    \pagecolor{#4}        % Color de fondo para la portada
    
    \begin{figure}[p]
        \centering
        \transparent{0.3}           % Opacidad del 30% para la imagen
        
        \includegraphics[width=\paperwidth, keepaspectratio]{assets/#1}
    
        \begin{tikzpicture}[remember picture, overlay]
            \node[anchor=north west, text=white, opacity=1, font=\fontsize{60}{90}\selectfont\bfseries\sffamily, align=left] at (#5, #6) {#2};
            
            \node[anchor=south east, text=white, opacity=1, font=\fontsize{12}{18}\selectfont\sffamily, align=right] at (9.7, 3) {\textbf{\href{https://losdeldgiim.github.io/}{Los Del DGIIM}}};
            
            \node[anchor=south east, text=white, opacity=1, font=\fontsize{12}{15}\selectfont\sffamily, align=right] at (9.7, 1.8) {Doble Grado en Ingeniería Informática y Matemáticas\\Universidad de Granada};
        \end{tikzpicture}
    \end{figure}
    
    
    \restoregeometry        % Restaurar márgenes normales para las páginas subsiguientes
    \pagecolor{white}       % Restaurar el color de página
    
    
    \newpage
    \thispagestyle{empty}               % Sin encabezado ni pie de página
    \begin{tikzpicture}[remember picture, overlay]
        \node[anchor=south west, inner sep=3cm] at (current page.south west) {
            \begin{minipage}{0.5\paperwidth}
                \href{https://creativecommons.org/licenses/by-nc-nd/4.0/}{
                    \includegraphics[height=2cm]{assets/Licencia.png}
                }\vspace{1cm}\\
                Esta obra está bajo una
                \href{https://creativecommons.org/licenses/by-nc-nd/4.0/}{
                    Licencia Creative Commons Atribución-NoComercial-SinDerivadas 4.0 Internacional (CC BY-NC-ND 4.0).
                }\\
    
                Eres libre de compartir y redistribuir el contenido de esta obra en cualquier medio o formato, siempre y cuando des el crédito adecuado a los autores originales y no persigas fines comerciales. 
            \end{minipage}
        };
    \end{tikzpicture}
    
    
    
    % 1. Foto de fondo
    % 2. Título
    % 3. Encabezado Izquierdo
    % 4. Color de fondo
    % 5. Coord x del titulo
    % 6. Coord y del titulo
    % 7. Fecha


}


\newcommand{\portadaBook}[7]{

    % 1. Foto de fondo
    % 2. Título
    % 3. Encabezado Izquierdo
    % 4. Color de fondo
    % 5. Coord x del titulo
    % 6. Coord y del titulo
    % 7. Fecha

    % Personaliza el formato del título
    \pretitle{\begin{center}\bfseries\fontsize{42}{56}\selectfont}
    \posttitle{\par\end{center}\vspace{2em}}
    
    % Personaliza el formato del autor
    \preauthor{\begin{center}\Large}
    \postauthor{\par\end{center}\vfill}
    
    % Personaliza el formato de la fecha
    \predate{\begin{center}\huge}
    \postdate{\par\end{center}\vspace{2em}}
    
    \title{#2}
    \author{\href{https://losdeldgiim.github.io/}{Los Del DGIIM}}
    \date{Granada, #7}
    \maketitle
    
    \tableofcontents
}




\newcommand{\portadaArticle}[7]{

    % 1. Foto de fondo
    % 2. Título
    % 3. Encabezado Izquierdo
    % 4. Color de fondo
    % 5. Coord x del titulo
    % 6. Coord y del titulo
    % 7. Fecha

    % Personaliza el formato del título
    \pretitle{\begin{center}\bfseries\fontsize{42}{56}\selectfont}
    \posttitle{\par\end{center}\vspace{2em}}
    
    % Personaliza el formato del autor
    \preauthor{\begin{center}\Large}
    \postauthor{\par\end{center}\vspace{3em}}
    
    % Personaliza el formato de la fecha
    \predate{\begin{center}\huge}
    \postdate{\par\end{center}\vspace{5em}}
    
    \title{#2}
    \author{\href{https://losdeldgiim.github.io/}{Los Del DGIIM}}
    \date{Granada, #7}
    \thispagestyle{empty}               % Sin encabezado ni pie de página
    \maketitle
    \vfill
}
    \portadaExamen{ffccA4.jpg}{Variable Compleja I\\Examen VI}{Variable Compleja I. Examen VI}{MidnightBlue}{-8}{28}{2024-2025}{Arturo Olivares Martos}

    \begin{description}
        \item[Asignatura] Variable Compleja I.
        \item[Curso Académico] 2019-20.
        \item[Grado] Doble Grado en Ingeniería Informática y Matemáticas.
        \item[Grupo] Único.
        \item[Profesor] Javier Merí de la Maza.
        \item[Descripción] Prueba Intermedia.
        \item[Fecha] 20 de Abril de 2020.
        \item[Duración] 120 minutos.
    \end{description}
    \newpage

    \begin{ejercicio}[3 puntos]
        Estudiar la convergencia puntual, absoluta y uniforme de la serie $\sum\limits_{n \geq 0} f_n$ donde:
        \[
            f_n(z) = \left(\dfrac{z^2-1}{z^2+1}\right)^n\qquad \forall z \in \mathbb{C} \setminus \left\{\pm \dfrac{-1+i}{\sqrt{2}}\right\}.
        \]
    \end{ejercicio}

    \begin{ejercicio}[3 puntos]
        Estudiar la derivabilidad de las funciones $f,g : \mathbb{C} \to \mathbb{C}$ dadas por:
        \[
            f(z) = z^2 e^{\ol{z}}
            \qquad g(z) = \sen(z) f(z) \qquad \forall z \in \mathbb{C}.
        \]
    \end{ejercicio}

    \begin{ejercicio}[1 punto]
        Calcular
        \[
            \int_{C(0,1)} \dfrac{\cos(z)}{z(z-2)^2} \, dz.
        \]
    \end{ejercicio}

    \begin{ejercicio}[3 puntos]
        Sean $a,b \in \mathbb{C}$ con $a \neq b$ y sea $R > 0$ de modo que $R > \max\{|a|,|b|\}$. Probar que, si $f$ es una función entera, se tiene que:
        \[
            \int_{C(0,R)} \dfrac{f(z)}{(z-a)(z-b)} \, dz = 2\pi i \cdot \dfrac{f(b) - f(a)}{b-a}.
        \]
        Deducir que toda función entera y acotada es constante (Teorema de Liouville).
    \end{ejercicio}

    \begin{ejercicio}[Extra: 1.5 puntos]
        Sea $\emptyset\neq \Omega=\Omega^\circ \subset \mathbb{C}$ y sean $g, g_n : \Omega \to \mathbb{C}$ para cada $n \in \mathbb{N}$. Probar que $\{g_n\}$ converge uniformemente a $g$ en cada compacto de $\Omega$ si, y sólo si, para cada $a \in \Omega$ existe un entorno de $a$ en el que $\{g_n\}$ converge uniformemente a $g$.
    \end{ejercicio}


    \newpage
    \setcounter{ejercicio}{0}

    \begin{ejercicio}[3 puntos]
        Estudiar la convergencia puntual, absoluta y uniforme de la serie $\sum\limits_{n \geq 0} f_n$ donde:
        \[
            f_n(z) = \left(\dfrac{z^2-i}{z^2+i}\right)^n\qquad \forall z \in \mathbb{C} \setminus \left\{\pm \dfrac{-1+i}{\sqrt{2}}\right\}.
        \]

        Se trata de una serie geométrica. Sabemos que esta converge puntualmente si la razón tiene módulo menor que uno. Fijado $z \in \mathbb{C} \setminus \left\{\pm \dfrac{-1+i}{\sqrt{2}}\right\}$, tenemos que:
        \begin{align*}
            \hspace{-1cm}\left|\dfrac{z^2-i}{z^2+i}\right| < 1 &\iff |z^2-i|^2 < |z^2+i|^2 \\
            \hspace{-1cm}&\iff \cancel{\left(\Re^2 z - \Im^ 2 z\right)^2} + \left(2 \Re z \Im z - 1\right)^2 < \cancel{\left(\Re^2 z - \Im^ 2 z\right)^2} + \left(2 \Re z \Im z + 1\right)^2 \iff\\
            \hspace{-1cm}&\iff 4\Re^2 z \Im^2 z - 4\Re z \Im z +1 < 4\Re^2 z \Im^2 z + 4\Re z \Im z +1 \iff\\
            &\iff \Re z\Im z > 0
        \end{align*}

        Definimos por tanto tanto el siguiente conjunto:
        \begin{equation*}
            \Omega = \left\{z \in \mathbb{C} : \Re z \Im z > 0\right\}.
        \end{equation*}

        Tenemos por tanto que converge puntualmente y absolutamente en $\Omega$ y no converge en $\mathbb{C} \setminus \Omega$.\\

        Respecto a la convergencia uniforme, dado $A\subset \mathbb{C}$, veamos que la serie converge uniformemente en $A$ si y solo si:
        \begin{equation*}
            \sup\left\{\left|\dfrac{z^2-i}{z^2+i}\right| : z \in A\right\} < 1.
        \end{equation*}
        % // TODO: Suficiente?
        \begin{comment}
        \begin{equation*}
            \inf\{\Re z \Im z : z \in A\} > 0.
        \end{equation*}
        
        \begin{description}
            \item[$\Longleftarrow)$] Sea $r>0$ dicho ínfimo. Por tanto, tengo que:
            \begin{equation*}
                \Re z \Im z \geq r > 0 \qquad \forall z \in A.
            \end{equation*}

            Por las dobles implicaciones anteriores, tenemos que:
            \begin{equation*}
                \left|\dfrac{z^2-i}{z^2+i}\right| < 1 \qquad \forall z \in A.
            \end{equation*}

            Por la densidad de $\bb{R}$ en $\bb{R}$, $\exists M>0$ tal que:
            \begin{equation*}
                \left|\dfrac{z^2-i}{z^2+i}\right| < M<1 \qquad \forall z \in A.
            \end{equation*}

            Por tanto, y como la serie geométrica de razón $M$ converge, usando el Test de Weierstrass, tenemos que:
            \begin{equation*}
                \sum_{n=0}^\infty f_n(z) \text{ converge uniformemente en } A.
            \end{equation*}

            \item[$\Longrightarrow)$] Demostraremos el recíproco. Como $A\subset \Omega$, no puede darse el caso de que dicho ínfimo sea negativo. Por tanto, supongamos que dicho ínfimo es cero. Por la caracterización del ínfimo con sucesiones, $\exists \{z_n\}$ sucesión de puntos de $A$ tal que $\{\Re z_n \Im z_n\} \to 0$.
            
        
        \end{description}
    \end{comment}
    \end{ejercicio}

    \begin{ejercicio}[3 puntos]
        Estudiar la derivabilidad de las funciones $f,g : \mathbb{C} \to \mathbb{C}$ dadas por:
        \[
            f(z) = z^2 e^{\ol{z}}
            \qquad g(z) = \sen(z) f(z) \qquad \forall z \in \mathbb{C}.
        \]

        Distinguimos en función del valor de $z\in \bb{C}$:
        \begin{itemize}
            \item Si $z\neq 0$:
            
            Definimos la siguiente función:
            \Func{h}{\bb{C}}{\bb{C}}{z}{e^{\ol{z}}}
            
            Además, sabemos que:
            \begin{equation*}
                h(z) = \dfrac{f(z)}{z^2}
            \end{equation*}

            Supuesto que $f$ es derivable en $z$, entonces $h$ también lo es. Pero se ha visto que $h$ no es derivable en ningún punto de $\bb{C}$. Por tanto, $f$ no es derivable en $z$.

            \item Si $z=0$:
            \begin{equation*}
                f'(0) = \lim_{z\to 0} \dfrac{f(z)-f(0)}{z-0} = \lim_{z\to 0} \dfrac{z^2 e^{\ol{z}}}{z} = \lim_{z\to 0} z e^{\ol{z}} = 0.
            \end{equation*}
        \end{itemize}

        Por tanto, $f$ es derivable en $0$ y no lo es en ningún otro punto de $\bb{C}$.\\

        Para $g$, distinguimos también en función del valor de $z\in \bb{C}$. Para ello, veamos antes dónde se anula el seno:
        \begin{equation*}
            \sen(x+iy)=0\iff \left\{
                \begin{array}{l}
                    \sen x \cosh y = 0\iff \sen x =0\\
                    \cos x \senh y = 0
                \end{array}
            \right\}\iff
            \left\{
                \begin{array}{l}
                    \sen x =0\\
                    \senh y = 0
                \end{array}
            \right\}\iff x+iy\in \pi\bb{Z}
        \end{equation*}

        Distinguimos por tanto en función de si $z\in \pi\bb{Z}$ o no:
        \begin{itemize}

            \item Si $z\notin \pi\bb{Z}$:
            \begin{equation*}
                f(z)=\dfrac{g(z)}{\sen z}
            \end{equation*}

            Si $g$ es derivable en $z$, entonces $f$ también lo es. Pero se ha visto que $f$ no es derivable en ningún punto de $\bb{C}^*$. Por tanto, $g$ no es derivable en $z$.

            \item Si $z=0$:
            
            Como $f$ es derivable en $0$, también lo es $g$ en $0$, con:
            \begin{equation*}
                g'(0) = \cos(0)f(0) + \sen(0)f'(0) = f(0)
            \end{equation*}
            \begin{observacion}
            Este caso no habría por qué distinguirlo (puesto que estña incluido en el siguiente), pero se incluye por ser el más directo.
            \end{observacion}

            \item Si $\exists k\in \bb{Z}$ tal que $z=2k\pi$:
            
            \begin{equation*}
                \dfrac{g(z)-g(2k\pi)}{z-2k\pi} = \dfrac{\sen(z)f(z)}{z-2k\pi}= \qquad \forall z\in \bb{C}\setminus \{2k\pi\}
            \end{equation*}

            Por la definición formal de derivada del seno en $2\pi k$, se tiene que:
            \begin{equation*}
                1 = \cos(2\pi k) = \lim_{z\to 2\pi k}\dfrac{\sen(z)-\sen(2\pi k)}{z-2\pi k}
                \lim_{z\to 2\pi k}\dfrac{\sen(z)}{z-2\pi k}
            \end{equation*}

            Por tanto:
            \begin{equation*}
                g'(2\pi k) = \lim_{z\to 2\pi k}\dfrac{\sen(z)f(z)}{z-2k\pi}
                = \lim_{z\to 2\pi k}\dfrac{\sen(z)}{z-2k\pi}\cdot \lim_{z\to 2\pi k} f(z)
                = f(2\pi k)
            \end{equation*}

            \item Si $\exists k\in \bb{Z}$ tal que $z=(2k+1)\pi$:
            \begin{equation*}
                \dfrac{g(z)-g((2k+1)\pi)}{z-(2k+1)\pi} = \dfrac{\sen(z)f(z)}{z-(2k+1)\pi}\qquad \forall z\in \bb{C}\setminus \{(2k+1)\pi\}
            \end{equation*}

            Por la definición formal de derivada del seno en $(2k+1)\pi$, se tiene que:
            \begin{equation*}
                -1 = \cos((2k+1)\pi) = \lim_{z\to (2k+1)\pi}\dfrac{\sen(z)-\sen((2k+1)\pi)}{z-(2k+1)\pi}
                \lim_{z\to (2k+1)\pi}\dfrac{\sen(z)}{z-(2k+1)\pi}
            \end{equation*}

            Por tanto:
            \begin{equation*}
                g'((2k+1)\pi) = \lim_{z\to (2k+1)\pi}\dfrac{\sen(z)f(z)}{z-2k\pi}
                = \lim_{z\to (2k+1)\pi}\dfrac{\sen(z)}{z-2k\pi}\cdot \lim_{z\to (2k+1)\pi} f(z)
                = -f((2k+1)\pi)
            \end{equation*}
        \end{itemize}
        Por tanto, $g$ es derivable en $z$ si y solo si $z\in \pi \bb{Z}$.
    \end{ejercicio}

    \begin{ejercicio}[1 punto]
        Calcular
        \[
            \int_{C(0,1)} \dfrac{\cos(z)}{z(z-2)^2} \, dz.
        \]

        Definimos la función $f$ como:
        \Func{f}{D(0,\nicefrac{1}{2})}{\mathbb{C}}{z}{\dfrac{\cos(z)}{(z-2)^2}}

        Como $f$ es racional, $f\in \cc{H}(D(0,\nicefrac{1}{2}))$. Por tanto, por la fórmula de Cauchy para la circunferencia, tenemos que:
        \begin{align*}
            \int_{C(0,1)} \dfrac{f(z)}{z}\ dz &= 2\pi i \cdot f(0)=\dfrac{2\pi}{4}\cdot i = \dfrac{\pi}{2}\cdot i
        \end{align*}
    \end{ejercicio}

    \begin{ejercicio}[3 puntos]
        Sean $a,b \in \mathbb{C}$ con $a \neq b$ y sea $R > 0$ de modo que $R > \max\{|a|,|b|\}$. Probar que, si $f$ es una función entera, se tiene que:
        \[
            \int_{C(0,R)} \dfrac{f(z)}{(z-a)(z-b)} \, dz = 2\pi i \cdot \dfrac{f(b) - f(a)}{b-a}.
        \]
        Deducir que toda función entera y acotada es constante (Teorema de Liouville).\\

        Descomponemos el integrando en fracciones simples:
    \begin{equation*}
        \frac{1}{(z-a)(z-b)} = \frac{A}{z-a} + \frac{B}{z-b} = \frac{A(z-b)+B(z-a)}{(z-a)(z-b)}
    \end{equation*}
    \begin{itemize}
        \item Para $z=a$: $1=A(a-b)\Longrightarrow A=\frac{1}{a-b}$.
        \item Para $z=b$: $1=B(b-a)\Longrightarrow B=\frac{1}{b-a}=-\frac{1}{a-b}$.
    \end{itemize}

    Por tanto, la integral queda:
    \begin{align*}
        \int_{C(0,R)} \frac{f(z)}{(z-a)(z-b)}dz &= \frac{1}{a-b}\left(\int_{C(0,R)} \frac{f(z)}{z-a}dz - \int_{C(0,R)} \frac{f(z)}{z-b}dz\right)\\
        &\AstIg \frac{1}{a-b}\left(2\pi i f(a) - 2\pi i f(b)\right)\\
        &= 2\pi i \cdot \frac{f(b)-f(a)}{b-a}
    \end{align*}
    donde $(\ast)$ se debe a que la función $f(z)$ es entera y que $a,b\in D(0,R)$, por lo que se puede aplicar la Fórmula de Cauchy para la circunferencia considerando como función $f(z)$.\\

    Sea ahora $f$ entera y acotada. Entonces, $\exists M\in \bb{R}^+$ tal que $|f(z)|\leq M$ para todo $z\in \bb{C}$. Por tanto, se tiene que:
    \begin{align*}
        \left|\dfrac{f(z)}{(z-a)(z-b)}\right| &\leq \frac{M}{|z-a||z-b|}
        \leq \frac{M}{\left||z| - |a|\right|\left||z| - |b|\right|}
        \leq \frac{M}{\left|R - |a|\right|\left|R - |b|\right|}
        \leq\\&\leq  \frac{M}{(R-|a|)(R-|b|)}\qquad \forall z\in C(0,R)^*
    \end{align*}
    donde hemos usado que $z\in C(0,R)^*$, por lo que $|z|=R$; y que $R>\max\{|a|,|b|\}$, por lo que $R-|a|>0$ y $R-|b|>0$. Por tanto, se tiene que:
    \begin{align*}
        \left|\int_{C(0,R)} \frac{f(z)}{(z-a)(z-b)}dz\right| &\leq \int_{C(0,R)} \left|\frac{f(z)}{(z-a)(z-b)}\right|dz\\
        &\leq 2\pi R \cdot \frac{M}{(R-|a|)(R-|b|)} = \frac{2\pi M R}{(R-|a|)(R-|b|)}
    \end{align*}

    Como la anterior expresión es válida para todo $R\in \bb{R}^+$ tal que $R>\max\{|a|,|b|\}$, podemos hacer tender $R\to \infty$. Por el Lema del Sándwich, se tiene que:
    \begin{equation*}
        \lim_{R\to \infty} \int_{C(0,R)} \frac{f(z)}{(z-a)(z-b)}dz = 0
    \end{equation*}

    Por la expresión anterior a la que habíamos llegado, se tiene que:
    \begin{equation*}
        \lim_{R\to \infty} \int_{C(0,R)} \frac{f(z)}{(z-a)(z-b)}dz = \lim_{R\to \infty} 2\pi i \cdot \frac{f(b)-f(a)}{b-a} = 2\pi i \cdot \frac{f(b)-f(a)}{b-a}
    \end{equation*}

    Por la unicidad del límite, se tiene que:
    \begin{equation*}
        f(b)=f(a) \qquad \forall a,b\in \bb{C}
    \end{equation*}

    Por tanto, $f$ es constante.
    \end{ejercicio}

    \begin{ejercicio}[Extra: 1.5 puntos]
        Sea $\emptyset\neq \Omega=\Omega^\circ \subset \mathbb{C}$ y sean $g, g_n : \Omega \to \mathbb{C}$ para cada $n \in \mathbb{N}$. Probar que $\{g_n\}$ converge uniformemente a $g$ en cada compacto de $\Omega$ si, y sólo si, para cada $a \in \Omega$ existe un entorno de $a$ en el que $\{g_n\}$ converge uniformemente a $g$.

        % // TODO: Hacer
    \end{ejercicio}
\end{document}