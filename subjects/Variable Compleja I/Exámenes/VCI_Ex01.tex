\documentclass[12pt]{article}

% Idioma y codificación
\usepackage[spanish, es-tabla]{babel}       %es-tabla para que se titule "Tabla"
\usepackage[utf8]{inputenc}

% Márgenes
\usepackage[a4paper,top=3cm,bottom=2.5cm,left=3cm,right=3cm]{geometry}

% Comentarios de bloque
\usepackage{verbatim}

% Paquetes de links
\usepackage[hidelinks]{hyperref}    % Permite enlaces
\usepackage{url}                    % redirecciona a la web

% Más opciones para enumeraciones
\usepackage{enumitem}

% Personalizar la portada
\usepackage{titling}

% Paquetes de tablas
\usepackage{multirow}


%------------------------------------------------------------------------

%Paquetes de figuras
\usepackage{caption}
\usepackage{subcaption} % Figuras al lado de otras
\usepackage{float}      % Poner figuras en el sitio indicado H.


% Paquetes de imágenes
\usepackage{graphicx}       % Paquete para añadir imágenes
\usepackage{transparent}    % Para manejar la opacidad de las figuras

% Paquete para usar colores
\usepackage[dvipsnames]{xcolor}
\usepackage{pagecolor}      % Para cambiar el color de la página

% Habilita tamaños de fuente mayores
\usepackage{fix-cm}

% Para los gráficos
\usepackage{tikz}

% Para poder situar los nodos en los grafos
\usetikzlibrary{positioning}


%------------------------------------------------------------------------

% Paquetes de matemáticas
\usepackage{mathtools, amsfonts, amssymb, mathrsfs}
\usepackage[makeroom]{cancel}     % Simplificar tachando
\usepackage{polynom}    % Divisiones y Ruffini
\usepackage{units} % Para poner fracciones diagonales con \nicefrac

\usepackage{pgfplots}   %Representar funciones
\pgfplotsset{compat=1.18}  % Versión 1.18

\usepackage{tikz-cd}    % Para usar diagramas de composiciones
\usetikzlibrary{calc}   % Para usar cálculo de coordenadas en tikz

%Definición de teoremas, etc.
\usepackage{amsthm}
%\swapnumbers   % Intercambia la posición del texto y de la numeración

\theoremstyle{plain}

\makeatletter
\@ifclassloaded{article}{
  \newtheorem{teo}{Teorema}[section]
}{
  \newtheorem{teo}{Teorema}[chapter]  % Se resetea en cada chapter
}
\makeatother

\newtheorem{coro}{Corolario}[teo]           % Se resetea en cada teorema
\newtheorem{prop}[teo]{Proposición}         % Usa el mismo contador que teorema
\newtheorem{lema}[teo]{Lema}                % Usa el mismo contador que teorema

\theoremstyle{remark}
\newtheorem*{observacion}{Observación}

\theoremstyle{definition}

\makeatletter
\@ifclassloaded{article}{
  \newtheorem{definicion}{Definición} [section]     % Se resetea en cada chapter
}{
  \newtheorem{definicion}{Definición} [chapter]     % Se resetea en cada chapter
}
\makeatother

\newtheorem*{notacion}{Notación}
\newtheorem*{ejemplo}{Ejemplo}
\newtheorem*{ejercicio*}{Ejercicio}             % No numerado
\newtheorem{ejercicio}{Ejercicio} [section]     % Se resetea en cada section


% Modificar el formato de la numeración del teorema "ejercicio"
\renewcommand{\theejercicio}{%
  \ifnum\value{section}=0 % Si no se ha iniciado ninguna sección
    \arabic{ejercicio}% Solo mostrar el número de ejercicio
  \else
    \thesection.\arabic{ejercicio}% Mostrar número de sección y número de ejercicio
  \fi
}


% \renewcommand\qedsymbol{$\blacksquare$}         % Cambiar símbolo QED
%------------------------------------------------------------------------

% Paquetes para encabezados
\usepackage{fancyhdr}
\pagestyle{fancy}
\fancyhf{}

\newcommand{\helv}{ % Modificación tamaño de letra
\fontfamily{}\fontsize{12}{12}\selectfont}
\setlength{\headheight}{15pt} % Amplía el tamaño del índice


%\usepackage{lastpage}   % Referenciar última pag   \pageref{LastPage}
\fancyfoot[C]{\thepage}

%------------------------------------------------------------------------

% Conseguir que no ponga "Capítulo 1". Sino solo "1."
\makeatletter
\@ifclassloaded{book}{
  \renewcommand{\chaptermark}[1]{\markboth{\thechapter.\ #1}{}} % En el encabezado
    
  \renewcommand{\@makechapterhead}[1]{%
  \vspace*{50\p@}%
  {\parindent \z@ \raggedright \normalfont
    \ifnum \c@secnumdepth >\m@ne
      \huge\bfseries \thechapter.\hspace{1em}\ignorespaces
    \fi
    \interlinepenalty\@M
    \Huge \bfseries #1\par\nobreak
    \vskip 40\p@
  }}
}
\makeatother

%------------------------------------------------------------------------
% Paquetes de cógido
\usepackage{minted}
\renewcommand\listingscaption{Código fuente}

\usepackage{fancyvrb}
% Personaliza el tamaño de los números de línea
\renewcommand{\theFancyVerbLine}{\small\arabic{FancyVerbLine}}

% Estilo para C++
\newminted{cpp}{
    frame=lines,
    framesep=2mm,
    baselinestretch=1.2,
    linenos,
    escapeinside=||
}

% para minted
\definecolor{LightGray}{rgb}{0.95,0.95,0.92}
\setminted{
    linenos=true,
    stepnumber=5,
    numberfirstline=true,
    autogobble,
    breaklines=true,
    breakautoindent=true,
    breaksymbolleft=,
    breaksymbolright=,
    breaksymbolindentleft=0pt,
    breaksymbolindentright=0pt,
    breaksymbolsepleft=0pt,
    breaksymbolsepright=0pt,
    fontsize=\footnotesize,
    bgcolor=LightGray,
    numbersep=10pt
}


\usepackage{listings} % Para incluir código desde un archivo

\renewcommand\lstlistingname{Código Fuente}
\renewcommand\lstlistlistingname{Índice de Códigos Fuente}

% Definir colores
\definecolor{vscodepurple}{rgb}{0.5,0,0.5}
\definecolor{vscodeblue}{rgb}{0,0,0.8}
\definecolor{vscodegreen}{rgb}{0,0.5,0}
\definecolor{vscodegray}{rgb}{0.5,0.5,0.5}
\definecolor{vscodebackground}{rgb}{0.97,0.97,0.97}
\definecolor{vscodelightgray}{rgb}{0.9,0.9,0.9}

% Configuración para el estilo de C similar a VSCode
\lstdefinestyle{vscode_C}{
  backgroundcolor=\color{vscodebackground},
  commentstyle=\color{vscodegreen},
  keywordstyle=\color{vscodeblue},
  numberstyle=\tiny\color{vscodegray},
  stringstyle=\color{vscodepurple},
  basicstyle=\scriptsize\ttfamily,
  breakatwhitespace=false,
  breaklines=true,
  captionpos=b,
  keepspaces=true,
  numbers=left,
  numbersep=5pt,
  showspaces=false,
  showstringspaces=false,
  showtabs=false,
  tabsize=2,
  frame=tb,
  framerule=0pt,
  aboveskip=10pt,
  belowskip=10pt,
  xleftmargin=10pt,
  xrightmargin=10pt,
  framexleftmargin=10pt,
  framexrightmargin=10pt,
  framesep=0pt,
  rulecolor=\color{vscodelightgray},
  backgroundcolor=\color{vscodebackground},
}

%------------------------------------------------------------------------

% Comandos definidos
\newcommand{\bb}[1]{\mathbb{#1}}
\newcommand{\cc}[1]{\mathcal{#1}}

% I prefer the slanted \leq
\let\oldleq\leq % save them in case they're every wanted
\let\oldgeq\geq
\renewcommand{\leq}{\leqslant}
\renewcommand{\geq}{\geqslant}

% Si y solo si
\newcommand{\sii}{\iff}

% Letras griegas
\newcommand{\eps}{\epsilon}
\newcommand{\veps}{\varepsilon}
\newcommand{\lm}{\lambda}

\newcommand{\ol}{\overline}
\newcommand{\ul}{\underline}
\newcommand{\wt}{\widetilde}
\newcommand{\wh}{\widehat}

\let\oldvec\vec
\renewcommand{\vec}{\overrightarrow}

% Derivadas parciales
\newcommand{\del}[2]{\frac{\partial #1}{\partial #2}}
\newcommand{\Del}[3]{\frac{\partial^{#1} #2}{\partial #3^{#1}}}
\newcommand{\deld}[2]{\dfrac{\partial #1}{\partial #2}}
\newcommand{\Deld}[3]{\dfrac{\partial^{#1} #2}{\partial #3^{#1}}}


\newcommand{\AstIg}{\stackrel{(\ast)}{=}}
\newcommand{\Hop}{\stackrel{L'H\hat{o}pital}{=}}

\newcommand{\red}[1]{{\color{red}#1}} % Para integrales, destacar los cambios.

% Método de integración
\newcommand{\MetInt}[2]{
    \left[\begin{array}{c}
        #1 \\ #2
    \end{array}\right]
}

% Declarar aplicaciones
% 1. Nombre aplicación
% 2. Dominio
% 3. Codominio
% 4. Variable
% 5. Imagen de la variable
\newcommand{\Func}[5]{
    \begin{equation*}
        \begin{array}{rrll}
            #1:& #2 & \longrightarrow & #3\\
               & #4 & \longmapsto & #5
        \end{array}
    \end{equation*}
}

%------------------------------------------------------------------------

\let\oldRe\Re % save them in case they're every wanted
\let\oldIm\Im
\renewcommand{\Re}{\operatorname{Re}} % redefine them
\renewcommand{\Im}{\operatorname{Im}}
\DeclareMathOperator{\Log}{Log}
\DeclareMathOperator{\Arg}{Arg}

\begin{document}

    % 1. Foto de fondo
    % 2. Título
    % 3. Encabezado Izquierdo
    % 4. Color de fondo
    % 5. Coord x del titulo
    % 6. Coord y del titulo
    % 7. Fecha

    
    % 1. Foto de fondo
% 2. Título
% 3. Encabezado Izquierdo
% 4. Color de fondo
% 5. Coord x del titulo
% 6. Coord y del titulo
% 7. Fecha

\newcommand{\portada}[7]{

    \portadaBase{#1}{#2}{#3}{#4}{#5}{#6}{#7}
    \portadaBook{#1}{#2}{#3}{#4}{#5}{#6}{#7}
}

\newcommand{\portadaExamen}[7]{

    \portadaBase{#1}{#2}{#3}{#4}{#5}{#6}{#7}
    \portadaArticle{#1}{#2}{#3}{#4}{#5}{#6}{#7}
}




\newcommand{\portadaBase}[7]{

    % Tiene la portada principal y la licencia Creative Commons
    
    % 1. Foto de fondo
    % 2. Título
    % 3. Encabezado Izquierdo
    % 4. Color de fondo
    % 5. Coord x del titulo
    % 6. Coord y del titulo
    % 7. Fecha
    
    
    \thispagestyle{empty}               % Sin encabezado ni pie de página
    \newgeometry{margin=0cm}        % Márgenes nulos para la primera página
    
    
    % Encabezado
    \fancyhead[L]{\helv #3}
    \fancyhead[R]{\helv \nouppercase{\leftmark}}
    
    
    \pagecolor{#4}        % Color de fondo para la portada
    
    \begin{figure}[p]
        \centering
        \transparent{0.3}           % Opacidad del 30% para la imagen
        
        \includegraphics[width=\paperwidth, keepaspectratio]{assets/#1}
    
        \begin{tikzpicture}[remember picture, overlay]
            \node[anchor=north west, text=white, opacity=1, font=\fontsize{60}{90}\selectfont\bfseries\sffamily, align=left] at (#5, #6) {#2};
            
            \node[anchor=south east, text=white, opacity=1, font=\fontsize{12}{18}\selectfont\sffamily, align=right] at (9.7, 3) {\textbf{\href{https://losdeldgiim.github.io/}{Los Del DGIIM}}};
            
            \node[anchor=south east, text=white, opacity=1, font=\fontsize{12}{15}\selectfont\sffamily, align=right] at (9.7, 1.8) {Doble Grado en Ingeniería Informática y Matemáticas\\Universidad de Granada};
        \end{tikzpicture}
    \end{figure}
    
    
    \restoregeometry        % Restaurar márgenes normales para las páginas subsiguientes
    \pagecolor{white}       % Restaurar el color de página
    
    
    \newpage
    \thispagestyle{empty}               % Sin encabezado ni pie de página
    \begin{tikzpicture}[remember picture, overlay]
        \node[anchor=south west, inner sep=3cm] at (current page.south west) {
            \begin{minipage}{0.5\paperwidth}
                \href{https://creativecommons.org/licenses/by-nc-nd/4.0/}{
                    \includegraphics[height=2cm]{assets/Licencia.png}
                }\vspace{1cm}\\
                Esta obra está bajo una
                \href{https://creativecommons.org/licenses/by-nc-nd/4.0/}{
                    Licencia Creative Commons Atribución-NoComercial-SinDerivadas 4.0 Internacional (CC BY-NC-ND 4.0).
                }\\
    
                Eres libre de compartir y redistribuir el contenido de esta obra en cualquier medio o formato, siempre y cuando des el crédito adecuado a los autores originales y no persigas fines comerciales. 
            \end{minipage}
        };
    \end{tikzpicture}
    
    
    
    % 1. Foto de fondo
    % 2. Título
    % 3. Encabezado Izquierdo
    % 4. Color de fondo
    % 5. Coord x del titulo
    % 6. Coord y del titulo
    % 7. Fecha


}


\newcommand{\portadaBook}[7]{

    % 1. Foto de fondo
    % 2. Título
    % 3. Encabezado Izquierdo
    % 4. Color de fondo
    % 5. Coord x del titulo
    % 6. Coord y del titulo
    % 7. Fecha

    % Personaliza el formato del título
    \pretitle{\begin{center}\bfseries\fontsize{42}{56}\selectfont}
    \posttitle{\par\end{center}\vspace{2em}}
    
    % Personaliza el formato del autor
    \preauthor{\begin{center}\Large}
    \postauthor{\par\end{center}\vfill}
    
    % Personaliza el formato de la fecha
    \predate{\begin{center}\huge}
    \postdate{\par\end{center}\vspace{2em}}
    
    \title{#2}
    \author{\href{https://losdeldgiim.github.io/}{Los Del DGIIM}}
    \date{Granada, #7}
    \maketitle
    
    \tableofcontents
}




\newcommand{\portadaArticle}[7]{

    % 1. Foto de fondo
    % 2. Título
    % 3. Encabezado Izquierdo
    % 4. Color de fondo
    % 5. Coord x del titulo
    % 6. Coord y del titulo
    % 7. Fecha

    % Personaliza el formato del título
    \pretitle{\begin{center}\bfseries\fontsize{42}{56}\selectfont}
    \posttitle{\par\end{center}\vspace{2em}}
    
    % Personaliza el formato del autor
    \preauthor{\begin{center}\Large}
    \postauthor{\par\end{center}\vspace{3em}}
    
    % Personaliza el formato de la fecha
    \predate{\begin{center}\huge}
    \postdate{\par\end{center}\vspace{5em}}
    
    \title{#2}
    \author{\href{https://losdeldgiim.github.io/}{Los Del DGIIM}}
    \date{Granada, #7}
    \thispagestyle{empty}               % Sin encabezado ni pie de página
    \maketitle
    \vfill
}
    \portadaExamen{ffccA4.jpg}{Variable Compleja I\\Examen I}{Variable Compleja I. Examen I}{MidnightBlue}{-8}{28}{2024-2025}{Arturo Olivares Martos}

    \begin{description}
        \item[Asignatura] Variable Compleja I.
        \item[Curso Académico] 2024-25.
        \item[Grado] Grado en Matemáticas y Doble Grado en Matemáticas y Física.
        %\item[Grupo] Único.
        \item[Profesor] Javier Merí de la Maza.
        \item[Descripción] Prueba Intermedia.
        \item[Fecha] 28 de noviembre de 2024.
        \item[Duración] 120 minutos.
    
    \end{description}
    \newpage

    \begin{ejercicio}[2.5 puntos] Probar que la serie $\displaystyle \sum\limits_{n\geq 1} \dfrac{z^n}{1-z^{n+1}}$ converge absolutamente en todo punto de $D(0, 1)$ y uniformemente en cada subconjunto compacto contenido en $D(0, 1)$.
    \end{ejercicio}

    \begin{ejercicio}[2.5 puntos] Dado $\beta \in \mathbb{C} \setminus \mathbb{Z}$ probar que existe una única función $f \in \cc{H}(D(0, 1))$ verificando
        $$z^2 f'(z) + \beta z f(z) = \log(1 + z)\qquad \forall z \in D(0, 1).$$
    \end{ejercicio}

    \begin{ejercicio}[2 puntos] Dado $R > 0$ con $R \neq 1$ y $R \neq 2$, calcular la integral
        $$\int_{C(0,R)} \frac{e^z}{(z + 1)^2(z - 2)}\ dz.$$
    \end{ejercicio}

    \begin{ejercicio}[3 puntos] Sea $\Omega \subset \mathbb{R}^2 \equiv \mathbb{C}$ un abierto. Decimos que una función $\varphi : \Omega \to \mathbb{R}$ es armónica en $\Omega$ si $\varphi \in C^2(\Omega)$ y:
    \begin{equation*}
        \frac{\partial^2 \varphi}{\partial x^2} + \frac{\partial^2 \varphi}{\partial y^2} = 0\qquad \forall (x, y) \in \Omega.
    \end{equation*}

    \begin{enumerate}
        \item \textbf{[1.5 puntos]} Sea $\varphi : \Omega \to \mathbb{R}$ armónica en $\Omega$. Probar que la función $g : \Omega \to \mathbb{C}$ dada por
        $$g(x + iy) = \frac{\partial \varphi}{\partial x}(x, y) - i\cdot  \frac{\partial \varphi}{\partial y}(x, y)$$
        es holomorfa en $\Omega$.

        \item \textbf{[1.5 puntos]} Suponiendo que $\Omega$ es un dominio estrellado, probar que existe $f \in \cc{H}(\Omega)$ de modo que $\Re f = \varphi$ y que $f$ es única salvo una constante.
        
    \end{enumerate}
    \end{ejercicio}



    \newpage
    \setcounter{ejercicio}{0}

    \begin{ejercicio}[2.5 puntos] Probar que la serie $\displaystyle \sum\limits_{n\geq 1} \dfrac{z^n}{1-z^{n+1}}$ converge absolutamente en todo punto de $D(0, 1)$ y uniformemente en cada subconjunto compacto contenido en $D(0, 1)$.\\

        Estudiamos en primer la convergencia uniforme. Sea $K\subset D(0, 1)$ no vacío y compacto. Entonces existe $r \in \left[0,1\right[$ tal que $|z|\leq r<1$ para todo $z \in K$. Entonces:
        \begin{align*}
            \left|\dfrac{z^n}{1-z^{n+1}}\right| &\leq \frac{|z|^n}{\left|1 - |z|^{n+1}\right|} = \frac{|z|^n}{1 - |z|^{n+1}}\leq \frac{r^n}{1 - r^{n+1}} \qquad \forall n \in \mathbb{N},~\forall z \in K
        \end{align*}

        Aplicamos ahora el criterio del cociente para series, usando que $r<1$:
        \begin{align*}
            \left\{\dfrac{r^{n+1}}{1 - r^{n+2}} \cdot \dfrac{1 - r^{n+1}}{r^n} \right\}&= \left\{r \cdot \dfrac{1 - r^{n+1}}{1 - r^{n+2}}\right\} \to r\cdot 1 = r<1
        \end{align*}
        
        Por el Criterio del Cociente, la serie siguiente es convergente.
        \begin{equation*}
            \sum_{n \geq 1} \frac{r^n}{1 - r^{n+1}}
        \end{equation*}
        
        Por tanto, por el Test de Weierstrass, la serie del enunciado converge uniformemente en $K$.\\

        Para la convergencia absoluta, consideramos $z\in D(0,1)$ fijo. Como $\{z\}$ es compacto, por el Test de Weierstrass tenemos que converge absolutamente en $\{z\}$. Como $z$ es arbitrario, tenemos que la serie converge absolutamente en todo punto de $D(0, 1)$.
    \end{ejercicio}

    \begin{ejercicio}[2.5 puntos] Dado $\beta \in \mathbb{C} \setminus \mathbb{Z}$ probar que existe una única función $f \in \cc{H}(D(0, 1))$ verificando
        $$z^2 f'(z) + \beta z f(z) = \log(1 + z)\qquad \forall z \in D(0, 1).$$

        Supongamos la existencia. Entonces, como $f\in \cc{H}(D(0, 1))$, $f$ es analítica en $D(0, 1)$ y por tanto:
        \begin{equation*}
            f(z) = \sum_{n=0}^{\infty} \alpha_n z^n\qquad \forall z \in D(0, 1)
        \end{equation*}

        Por el Teorema de Derivación de funciones dadas como Sumas de Series de Potencias, tenemos que:
        \begin{equation*}
            f'(z) = \sum_{n=1}^{\infty} n \alpha_n z^{n-1}\qquad \forall z \in D(0, 1)
        \end{equation*}

        Por tanto, podemos reescribir la ecuación dada en el enunciado para cada $z \in D(0, 1)$ como:
        \begin{align*}
            \sum_{n=1}^{\infty} n \alpha_n z^{n+1} + \sum_{n=0}^{\infty} \beta \alpha_n z^{n+1} &= \log(1 + z)\\
            \sum_{n=0}^{\infty} \left(n + \beta\right) \alpha_n z^{n+1} &= \sum_{n=0}^{\infty} (-1)^{n}\dfrac{z^{n+1}}{n+1}
        \end{align*}

        Por el Principio de Identidad, tenemos que:
        \begin{equation*}
            \alpha_n = \frac{(-1)^{n}}{(n+1)(n + \beta)}\qquad \forall n \in \mathbb{N}\cup \{0\}
        \end{equation*}

        Por tanto, la función $f$ queda dada por:
        \begin{equation*}
            f(z) = \sum_{n=0}^{\infty} \frac{(-1)^{n}}{(n+1)(n + \beta)} z^n\qquad \forall z \in D(0, 1)
        \end{equation*}

        Veamos no obstante que $f$ está bien definida. Para ello, vamos a ver que la serie converge puntualmente en $D(0, 1)$. Para ello, clculamos su radio de convergencia:
        \begin{align*}
            \lim_{n \to \infty} \sqrt{|\alpha_n|} &= \lim_{n \to \infty} \dfrac{1}{\sqrt{(n+1)(n + \beta)}} = 1
        \end{align*}

        Por tanto, por la fórmula de Cauchy-Hadamard, el radio de convergencia de la serie es $R = 1$. Por tanto, la serie converge puntualmente en $D(0, 1)$, y esto demuestra la existencia de $f$.
    \end{ejercicio}

    \begin{ejercicio}[2 puntos] Dado $R > 0$ con $R \neq 1$ y $R \neq 2$, calcular la integral
        $$\int_{C(0,R)} \frac{e^z}{(z + 1)^2(z - 2)}\ dz.$$

        Distinguimos en función del valor de $R$.
        \begin{enumerate}
            \item \ul{Caso $R\in \left]0,1\right[$}: En este caso, definimos:
            \Func{f}{D(0, 1)}{\bb{C}}{z}{\dfrac{e^z}{(z + 1)^2(z - 2)}}

            Tenemos que $f\in \cc{H}(D(0, 1))$ y $D(0,1)$ es estrellado. Por el Teorema Local de Cauchy, $f$ admite una primitiva en $D(0,1)$. Como $C(0,R)$ es un camino cerrado en $D(0,1)$, tenemos que:
            \begin{equation*}
                \int_{C(0,R)} \frac{e^z}{(z + 1)^2(z - 2)}\ dz
                = \int_{C(0,R)} f(z)\ dz
                = 0
            \end{equation*}

            \item \ul{Caso $R\in \left]1,2\right[$}: En este caso, definimos:
            \Func{f}{D(0, 2)}{\bb{C}}{z}{\dfrac{e^z}{z - 2}}

            Entonces, $f\in \cc{H}(D(0, 2))$ y $\ol{D}(0,R)\subset D(0, 2)$. Por la Fórmula de las Derivadas de Cauchy, tenemos que:
            \begin{align*}
                \int_{C(0,R)} \frac{e^z}{(z + 1)^2(z - 2)}\ dz &= \int_{C(0,R)} \dfrac{f(z)}{(z + 1)^2}\ dz = 2\pi i \cdot f'(-1)
            \end{align*}

            Calcularemos por tanto la derivada de $f$:
            \begin{align*}
                f'(z) &= \dfrac{(z - 2)e^z - e^z}{(z - 2)^2} = \dfrac{(z - 3)e^z}{(z - 2)^2}\qquad \forall z \in D(0, 2)
            \end{align*}

            Por tanto, tenemos que:
            \begin{align*}
                f'(-1) &= -\dfrac{4}{9e}
            \end{align*}

            Por tanto, la integral queda dada por:
            \begin{align*}
                \int_{C(0,R)} \frac{e^z}{(z + 1)^2(z - 2)}\ dz &= 2\pi i \cdot f'(-1) = -\dfrac{8\pi i}{9e}
            \end{align*}

            \item \ul{Caso $R\in \left]2,+\infty\right[$}: En este caso, descomponemos en fracciones simples:
            \begin{align*}
                \dfrac{1}{(z + 1)^2(z - 2)} &= \dfrac{A}{z + 1} + \dfrac{B}{(z + 1)^2} + \dfrac{C}{z - 2}
                = \dfrac{A(z - 2)(z + 1) + B(z - 2) + C(z + 1)^2}{(z + 1)^2(z - 2)}
            \end{align*}
            Igualando numeradores, tenemos que:
            \begin{itemize}
                \item Para $z = -1$: $1=-3B\implies B=\nicefrac{-1}{3}$.
                \item Para $z = 2$: $1=9C\implies C=\nicefrac{1}{9}$.
                \item Igualando los coeficientes de $z^2$: $0=A+C\implies A=\nicefrac{-1}{9}$.
            \end{itemize}

            Resolvemos ahora cada una de las integrales por separado. Como la exponencial es entera, podemos usar la Fórmula de Cauchy para la circunferencia en dos de las integrales:
            \begin{align*}
                \int_{C(0,R)} \frac{e^z}{z + 1}\ dz &= 2\pi i \cdot e^{-1}\\
                \int_{C(0,R)} \frac{e^z}{z-2}\ dz &= 2\pi i \cdot e^{2}
            \end{align*}

            Para la tercera integral, empleamos la Fórmula de Cauchy para la derivada, sabiendo que la derivada de la exponencial es la exponencial:
            \begin{align*}
                \int_{C(0,R)} \frac{e^z}{(z + 1)^2}\ dz &= 2\pi i \cdot e^{-1}
            \end{align*}

            Por tanto, la integral queda dada por:
            \begin{align*}
                \hspace{-2cm}\int_{C(0,R)} \frac{e^z}{(z + 1)^2(z - 2)}\ dz &= A\cdot \int_{C(0,R)} \frac{e^z}{z + 1}\ dz + B\cdot \int_{C(0,R)} \frac{e^z}{(z + 1)^2}\ dz + C\cdot \int_{C(0,R)} \frac{e^z}{z - 2}\ dz\\
                \hspace{-2cm}&= \left(-\frac{1}{9}\cdot 2\pi i \cdot e^{-1}\right) + \left(-\frac{1}{3}\cdot 2\pi i \cdot e^{-1}\right) + \left(\frac{1}{9}\cdot 2\pi i \cdot e^{2}\right)\\
                \hspace{-2cm}&= -\dfrac{4}{9}\cdot 2\pi i\cdot e^{-1} + \dfrac{1}{9}\cdot 2\pi i\cdot e^{2}
                = \dfrac{2\pi i}{9}\left(e^{2} - 4e^{-1}\right)
            \end{align*}
        \end{enumerate}
    \end{ejercicio}

    \begin{ejercicio}[3 puntos] Sea $\Omega \subset \mathbb{R}^2 \equiv \mathbb{C}$ un abierto. Decimos que una función $\varphi : \Omega \to \mathbb{R}$ es armónica en $\Omega$ si $\varphi \in C^2(\Omega)$ y:
    \begin{equation*}
        \frac{\partial^2 \varphi}{\partial x^2} + \frac{\partial^2 \varphi}{\partial y^2} = 0\qquad \forall (x, y) \in \Omega.
    \end{equation*}

    \begin{enumerate}
        \item \textbf{[1.5 puntos]} Sea $\varphi : \Omega \to \mathbb{R}$ armónica en $\Omega$. Probar que la función $g : \Omega \to \mathbb{C}$ dada por
        $$g(x + iy) = \frac{\partial \varphi}{\partial x}(x, y) - i\cdot  \frac{\partial \varphi}{\partial y}(x, y)$$
        es holomorfa en $\Omega$.

        Definimos las funciones $u,v:\Omega\to \bb{R}$, con vistas a aplicar las Ecuaciones de Cauchy-Riemann:
        \begin{align*}
            u(x,y) &= \Re g(x+iy) = \frac{\partial \varphi}{\partial x}(x, y)\\
            v(x,y) &= \Im g(x+iy) = -\frac{\partial \varphi}{\partial y}(x, y)
        \end{align*}

        Calculamos cada una de las derivadas parciales de $u$ y $v$:
        \begin{align*}
            \frac{\partial u}{\partial x} &= \frac{\partial^2 \varphi}{\partial x^2}&
            \frac{\partial u}{\partial y} = \frac{\partial^2 \varphi}{\partial x \partial y}\\
            \frac{\partial v}{\partial x} &= -\frac{\partial^2 \varphi}{\partial y \partial x}&
            \frac{\partial v}{\partial y} = -\frac{\partial^2 \varphi}{\partial y^2}
        \end{align*}

        Comprobemos que se cumplen las Ecuaciones de Cauchy-Riemann:
        \begin{align*}
            \frac{\partial u}{\partial x} &= \frac{\partial^2 \varphi}{\partial x^2} = -\frac{\partial^2 \varphi}{\partial y^2} = \frac{\partial v}{\partial y}\iff \frac{\partial^2 \varphi}{\partial x^2} + \frac{\partial^2 \varphi}{\partial y^2} = 0\\
            \frac{\partial u}{\partial y} &= \frac{\partial^2 \varphi}{\partial x \partial y} = -\left(-\frac{\partial^2 \varphi}{\partial y \partial x}\right) = -\frac{\partial v}{\partial x}
        \end{align*}
        Notemos que la primera ecuación se cumple por hipótesis, y la segunda se cumple por el Teorema de Clairaut. Por tanto, se cumplen las Ecuaciones de Cauchy-Riemann, y por tanto $g$ es holomorfa en $\Omega$.

        \item \textbf{[1.5 puntos]} Suponiendo que $\Omega$ es un dominio estrellado, probar que existe $f \in \cc{H}(\Omega)$ de modo que $\Re f = \varphi$ y que $f$ es única salvo una constante.
        
        Como $\Omega$ es un dominio estrellado y $g\in \cc{H}(\Omega)$, por el Teorema Local de Cauchy, $\exists f\in \cc{H}(\Omega)$ una primitiva de $g$ en $\Omega$. Por tanto, tenemos que:
        \begin{align*}
            f'(x+iy) &= g(x + iy) = \frac{\partial \varphi}{\partial x}(x, y) - i\cdot  \frac{\partial \varphi}{\partial y}(x, y)\qquad \forall (x, y) \in \Omega
        \end{align*}

        Definimos ahora de nuevo $u,v:\Omega\to \bb{R}$ como:
        \begin{align*}
            u(x,y) &= \Re f(x+iy)\\
            v(x,y) &= \Im f(x+iy)
        \end{align*}

        Por las Ecuaciones de Cauchy-Riemann, tenemos que:
        \begin{align*}
            \frac{\partial u}{\partial x} &= \frac{\partial v}{\partial y}= \frac{\partial \varphi}{\partial x}\\
            \frac{\partial v}{\partial x} &= -\frac{\partial \varphi}{\partial y}=-\frac{\partial u}{\partial y}
        \end{align*}

        Para calcular $u=\Re f$, integramos la primera ecuación respecto de $x$:
        \begin{align*}
            u(x,y) &= \int \frac{\partial \varphi}{\partial x}\ dx = \varphi(x,y) + h(y)\qquad \forall (x,y) \in \Omega
        \end{align*}
        donde $h$ es una función de $y$ que no depende de $x$ y representa la constante de integración. Derivando respecto de $y$, tenemos que:
        \begin{align*}
            \frac{\partial u}{\partial y}(x,y) &= \frac{\partial \varphi}{\partial y}(x,y) + h'(y)\Longrightarrow
            h'(y) = 0\qquad \forall (x,y) \in \Omega
        \end{align*}

        Por tanto, $h$ es constante, por lo que $\exists C\in \bb{R}$ tal que:
        \begin{align*}
            u(x,y) &= \varphi(x,y) + C\qquad \forall (x,y) \in \Omega
        \end{align*}  
        
        Como $u=\Re f$, y por hipótesis buscamos que $\Re f = \varphi$, tenemos que $C=0$. Por tanto, tenemos que:
        \begin{align*}
            \Re f(x,y) &= \varphi(x,y)\qquad \forall (x,y) \in \Omega
        \end{align*}


        Por tanto, la existencia está probada. Supongamos ahora que existe otra función $g\in \cc{H}(\Omega)$ tal que $\Re g = \varphi$. Definimos $h = f - g\in \cc{H}(\Omega)$. Entonces, tenemos que:
        \begin{align*}
            \Re h &= \Re f - \Re g = \varphi - \varphi = 0
        \end{align*}

        Como $h$ es holomorfa, está definida en un dominio, y su parte real es nula, entonces $h$ es constante, luego $\exists \lambda \in \bb{C}$ tal que:
        \begin{align*}
            f(z) &= g(z) + \lambda\qquad \forall z \in \Omega
        \end{align*}

        Por tanto, $f$ es única salvo una constante.
    \end{enumerate}
    \end{ejercicio}
\end{document}