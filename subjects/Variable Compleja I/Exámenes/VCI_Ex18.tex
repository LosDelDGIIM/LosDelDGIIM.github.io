\documentclass[12pt]{article}

% Idioma y codificación
\usepackage[spanish, es-tabla]{babel}       %es-tabla para que se titule "Tabla"
\usepackage[utf8]{inputenc}

% Márgenes
\usepackage[a4paper,top=3cm,bottom=2.5cm,left=3cm,right=3cm]{geometry}

% Comentarios de bloque
\usepackage{verbatim}

% Paquetes de links
\usepackage[hidelinks]{hyperref}    % Permite enlaces
\usepackage{url}                    % redirecciona a la web

% Más opciones para enumeraciones
\usepackage{enumitem}

% Personalizar la portada
\usepackage{titling}

% Paquetes de tablas
\usepackage{multirow}


%------------------------------------------------------------------------

%Paquetes de figuras
\usepackage{caption}
\usepackage{subcaption} % Figuras al lado de otras
\usepackage{float}      % Poner figuras en el sitio indicado H.


% Paquetes de imágenes
\usepackage{graphicx}       % Paquete para añadir imágenes
\usepackage{transparent}    % Para manejar la opacidad de las figuras

% Paquete para usar colores
\usepackage[dvipsnames]{xcolor}
\usepackage{pagecolor}      % Para cambiar el color de la página

% Habilita tamaños de fuente mayores
\usepackage{fix-cm}

% Para los gráficos
\usepackage{tikz}

% Para poder situar los nodos en los grafos
\usetikzlibrary{positioning}


%------------------------------------------------------------------------

% Paquetes de matemáticas
\usepackage{mathtools, amsfonts, amssymb, mathrsfs}
\usepackage[makeroom]{cancel}     % Simplificar tachando
\usepackage{polynom}    % Divisiones y Ruffini
\usepackage{units} % Para poner fracciones diagonales con \nicefrac

\usepackage{pgfplots}   %Representar funciones
\pgfplotsset{compat=1.18}  % Versión 1.18

\usepackage{tikz-cd}    % Para usar diagramas de composiciones
\usetikzlibrary{calc}   % Para usar cálculo de coordenadas en tikz

%Definición de teoremas, etc.
\usepackage{amsthm}
%\swapnumbers   % Intercambia la posición del texto y de la numeración

\theoremstyle{plain}

\makeatletter
\@ifclassloaded{article}{
  \newtheorem{teo}{Teorema}[section]
}{
  \newtheorem{teo}{Teorema}[chapter]  % Se resetea en cada chapter
}
\makeatother

\newtheorem{coro}{Corolario}[teo]           % Se resetea en cada teorema
\newtheorem{prop}[teo]{Proposición}         % Usa el mismo contador que teorema
\newtheorem{lema}[teo]{Lema}                % Usa el mismo contador que teorema

\theoremstyle{remark}
\newtheorem*{observacion}{Observación}

\theoremstyle{definition}

\makeatletter
\@ifclassloaded{article}{
  \newtheorem{definicion}{Definición} [section]     % Se resetea en cada chapter
}{
  \newtheorem{definicion}{Definición} [chapter]     % Se resetea en cada chapter
}
\makeatother

\newtheorem*{notacion}{Notación}
\newtheorem*{ejemplo}{Ejemplo}
\newtheorem*{ejercicio*}{Ejercicio}             % No numerado
\newtheorem{ejercicio}{Ejercicio} [section]     % Se resetea en cada section


% Modificar el formato de la numeración del teorema "ejercicio"
\renewcommand{\theejercicio}{%
  \ifnum\value{section}=0 % Si no se ha iniciado ninguna sección
    \arabic{ejercicio}% Solo mostrar el número de ejercicio
  \else
    \thesection.\arabic{ejercicio}% Mostrar número de sección y número de ejercicio
  \fi
}


% \renewcommand\qedsymbol{$\blacksquare$}         % Cambiar símbolo QED
%------------------------------------------------------------------------

% Paquetes para encabezados
\usepackage{fancyhdr}
\pagestyle{fancy}
\fancyhf{}

\newcommand{\helv}{ % Modificación tamaño de letra
\fontfamily{}\fontsize{12}{12}\selectfont}
\setlength{\headheight}{15pt} % Amplía el tamaño del índice


%\usepackage{lastpage}   % Referenciar última pag   \pageref{LastPage}
\fancyfoot[C]{\thepage}

%------------------------------------------------------------------------

% Conseguir que no ponga "Capítulo 1". Sino solo "1."
\makeatletter
\@ifclassloaded{book}{
  \renewcommand{\chaptermark}[1]{\markboth{\thechapter.\ #1}{}} % En el encabezado
    
  \renewcommand{\@makechapterhead}[1]{%
  \vspace*{50\p@}%
  {\parindent \z@ \raggedright \normalfont
    \ifnum \c@secnumdepth >\m@ne
      \huge\bfseries \thechapter.\hspace{1em}\ignorespaces
    \fi
    \interlinepenalty\@M
    \Huge \bfseries #1\par\nobreak
    \vskip 40\p@
  }}
}
\makeatother

%------------------------------------------------------------------------
% Paquetes de cógido
\usepackage{minted}
\renewcommand\listingscaption{Código fuente}

\usepackage{fancyvrb}
% Personaliza el tamaño de los números de línea
\renewcommand{\theFancyVerbLine}{\small\arabic{FancyVerbLine}}

% Estilo para C++
\newminted{cpp}{
    frame=lines,
    framesep=2mm,
    baselinestretch=1.2,
    linenos,
    escapeinside=||
}

% para minted
\definecolor{LightGray}{rgb}{0.95,0.95,0.92}
\setminted{
    linenos=true,
    stepnumber=5,
    numberfirstline=true,
    autogobble,
    breaklines=true,
    breakautoindent=true,
    breaksymbolleft=,
    breaksymbolright=,
    breaksymbolindentleft=0pt,
    breaksymbolindentright=0pt,
    breaksymbolsepleft=0pt,
    breaksymbolsepright=0pt,
    fontsize=\footnotesize,
    bgcolor=LightGray,
    numbersep=10pt
}


\usepackage{listings} % Para incluir código desde un archivo

\renewcommand\lstlistingname{Código Fuente}
\renewcommand\lstlistlistingname{Índice de Códigos Fuente}

% Definir colores
\definecolor{vscodepurple}{rgb}{0.5,0,0.5}
\definecolor{vscodeblue}{rgb}{0,0,0.8}
\definecolor{vscodegreen}{rgb}{0,0.5,0}
\definecolor{vscodegray}{rgb}{0.5,0.5,0.5}
\definecolor{vscodebackground}{rgb}{0.97,0.97,0.97}
\definecolor{vscodelightgray}{rgb}{0.9,0.9,0.9}

% Configuración para el estilo de C similar a VSCode
\lstdefinestyle{vscode_C}{
  backgroundcolor=\color{vscodebackground},
  commentstyle=\color{vscodegreen},
  keywordstyle=\color{vscodeblue},
  numberstyle=\tiny\color{vscodegray},
  stringstyle=\color{vscodepurple},
  basicstyle=\scriptsize\ttfamily,
  breakatwhitespace=false,
  breaklines=true,
  captionpos=b,
  keepspaces=true,
  numbers=left,
  numbersep=5pt,
  showspaces=false,
  showstringspaces=false,
  showtabs=false,
  tabsize=2,
  frame=tb,
  framerule=0pt,
  aboveskip=10pt,
  belowskip=10pt,
  xleftmargin=10pt,
  xrightmargin=10pt,
  framexleftmargin=10pt,
  framexrightmargin=10pt,
  framesep=0pt,
  rulecolor=\color{vscodelightgray},
  backgroundcolor=\color{vscodebackground},
}

%------------------------------------------------------------------------

% Comandos definidos
\newcommand{\bb}[1]{\mathbb{#1}}
\newcommand{\cc}[1]{\mathcal{#1}}

% I prefer the slanted \leq
\let\oldleq\leq % save them in case they're every wanted
\let\oldgeq\geq
\renewcommand{\leq}{\leqslant}
\renewcommand{\geq}{\geqslant}

% Si y solo si
\newcommand{\sii}{\iff}

% Letras griegas
\newcommand{\eps}{\epsilon}
\newcommand{\veps}{\varepsilon}
\newcommand{\lm}{\lambda}

\newcommand{\ol}{\overline}
\newcommand{\ul}{\underline}
\newcommand{\wt}{\widetilde}
\newcommand{\wh}{\widehat}

\let\oldvec\vec
\renewcommand{\vec}{\overrightarrow}

% Derivadas parciales
\newcommand{\del}[2]{\frac{\partial #1}{\partial #2}}
\newcommand{\Del}[3]{\frac{\partial^{#1} #2}{\partial #3^{#1}}}
\newcommand{\deld}[2]{\dfrac{\partial #1}{\partial #2}}
\newcommand{\Deld}[3]{\dfrac{\partial^{#1} #2}{\partial #3^{#1}}}


\newcommand{\AstIg}{\stackrel{(\ast)}{=}}
\newcommand{\Hop}{\stackrel{L'H\hat{o}pital}{=}}

\newcommand{\red}[1]{{\color{red}#1}} % Para integrales, destacar los cambios.

% Método de integración
\newcommand{\MetInt}[2]{
    \left[\begin{array}{c}
        #1 \\ #2
    \end{array}\right]
}

% Declarar aplicaciones
% 1. Nombre aplicación
% 2. Dominio
% 3. Codominio
% 4. Variable
% 5. Imagen de la variable
\newcommand{\Func}[5]{
    \begin{equation*}
        \begin{array}{rrll}
            #1:& #2 & \longrightarrow & #3\\
               & #4 & \longmapsto & #5
        \end{array}
    \end{equation*}
}

%------------------------------------------------------------------------

\let\oldRe\Re % save them in case they're every wanted
\let\oldIm\Im
\renewcommand{\Re}{\operatorname{Re}} % redefine them
\renewcommand{\Im}{\operatorname{Im}}
\DeclareMathOperator{\Log}{Log}
\DeclareMathOperator{\Arg}{Arg}
\DeclareMathOperator{\ord}{ord}
\DeclareMathOperator{\Ind}{Ind}
\DeclareMathOperator{\Fr}{Fr}
\DeclareMathOperator{\Res}{Res}


\usetikzlibrary{arrows.meta, decorations.markings} % Cargar las bibliotecas necesarias

% Configuración para las flechas
\tikzset{
    arrow at 1/3/.style={postaction={decorate},
        decoration={markings, mark=at position 0.33 with {\arrow{Stealth}}}},
    arrow at 2/3/.style={postaction={decorate},
        decoration={markings, mark=at position 0.66 with {\arrow{Stealth}}}}
}


\begin{document}

    % 1. Foto de fondo
    % 2. Título
    % 3. Encabezado Izquierdo
    % 4. Color de fondo
    % 5. Coord x del titulo
    % 6. Coord y del titulo
    % 7. Fecha

    
    % 1. Foto de fondo
% 2. Título
% 3. Encabezado Izquierdo
% 4. Color de fondo
% 5. Coord x del titulo
% 6. Coord y del titulo
% 7. Fecha

\newcommand{\portada}[7]{

    \portadaBase{#1}{#2}{#3}{#4}{#5}{#6}{#7}
    \portadaBook{#1}{#2}{#3}{#4}{#5}{#6}{#7}
}

\newcommand{\portadaExamen}[7]{

    \portadaBase{#1}{#2}{#3}{#4}{#5}{#6}{#7}
    \portadaArticle{#1}{#2}{#3}{#4}{#5}{#6}{#7}
}




\newcommand{\portadaBase}[7]{

    % Tiene la portada principal y la licencia Creative Commons
    
    % 1. Foto de fondo
    % 2. Título
    % 3. Encabezado Izquierdo
    % 4. Color de fondo
    % 5. Coord x del titulo
    % 6. Coord y del titulo
    % 7. Fecha
    
    
    \thispagestyle{empty}               % Sin encabezado ni pie de página
    \newgeometry{margin=0cm}        % Márgenes nulos para la primera página
    
    
    % Encabezado
    \fancyhead[L]{\helv #3}
    \fancyhead[R]{\helv \nouppercase{\leftmark}}
    
    
    \pagecolor{#4}        % Color de fondo para la portada
    
    \begin{figure}[p]
        \centering
        \transparent{0.3}           % Opacidad del 30% para la imagen
        
        \includegraphics[width=\paperwidth, keepaspectratio]{assets/#1}
    
        \begin{tikzpicture}[remember picture, overlay]
            \node[anchor=north west, text=white, opacity=1, font=\fontsize{60}{90}\selectfont\bfseries\sffamily, align=left] at (#5, #6) {#2};
            
            \node[anchor=south east, text=white, opacity=1, font=\fontsize{12}{18}\selectfont\sffamily, align=right] at (9.7, 3) {\textbf{\href{https://losdeldgiim.github.io/}{Los Del DGIIM}}};
            
            \node[anchor=south east, text=white, opacity=1, font=\fontsize{12}{15}\selectfont\sffamily, align=right] at (9.7, 1.8) {Doble Grado en Ingeniería Informática y Matemáticas\\Universidad de Granada};
        \end{tikzpicture}
    \end{figure}
    
    
    \restoregeometry        % Restaurar márgenes normales para las páginas subsiguientes
    \pagecolor{white}       % Restaurar el color de página
    
    
    \newpage
    \thispagestyle{empty}               % Sin encabezado ni pie de página
    \begin{tikzpicture}[remember picture, overlay]
        \node[anchor=south west, inner sep=3cm] at (current page.south west) {
            \begin{minipage}{0.5\paperwidth}
                \href{https://creativecommons.org/licenses/by-nc-nd/4.0/}{
                    \includegraphics[height=2cm]{assets/Licencia.png}
                }\vspace{1cm}\\
                Esta obra está bajo una
                \href{https://creativecommons.org/licenses/by-nc-nd/4.0/}{
                    Licencia Creative Commons Atribución-NoComercial-SinDerivadas 4.0 Internacional (CC BY-NC-ND 4.0).
                }\\
    
                Eres libre de compartir y redistribuir el contenido de esta obra en cualquier medio o formato, siempre y cuando des el crédito adecuado a los autores originales y no persigas fines comerciales. 
            \end{minipage}
        };
    \end{tikzpicture}
    
    
    
    % 1. Foto de fondo
    % 2. Título
    % 3. Encabezado Izquierdo
    % 4. Color de fondo
    % 5. Coord x del titulo
    % 6. Coord y del titulo
    % 7. Fecha


}


\newcommand{\portadaBook}[7]{

    % 1. Foto de fondo
    % 2. Título
    % 3. Encabezado Izquierdo
    % 4. Color de fondo
    % 5. Coord x del titulo
    % 6. Coord y del titulo
    % 7. Fecha

    % Personaliza el formato del título
    \pretitle{\begin{center}\bfseries\fontsize{42}{56}\selectfont}
    \posttitle{\par\end{center}\vspace{2em}}
    
    % Personaliza el formato del autor
    \preauthor{\begin{center}\Large}
    \postauthor{\par\end{center}\vfill}
    
    % Personaliza el formato de la fecha
    \predate{\begin{center}\huge}
    \postdate{\par\end{center}\vspace{2em}}
    
    \title{#2}
    \author{\href{https://losdeldgiim.github.io/}{Los Del DGIIM}}
    \date{Granada, #7}
    \maketitle
    
    \tableofcontents
}




\newcommand{\portadaArticle}[7]{

    % 1. Foto de fondo
    % 2. Título
    % 3. Encabezado Izquierdo
    % 4. Color de fondo
    % 5. Coord x del titulo
    % 6. Coord y del titulo
    % 7. Fecha

    % Personaliza el formato del título
    \pretitle{\begin{center}\bfseries\fontsize{42}{56}\selectfont}
    \posttitle{\par\end{center}\vspace{2em}}
    
    % Personaliza el formato del autor
    \preauthor{\begin{center}\Large}
    \postauthor{\par\end{center}\vspace{3em}}
    
    % Personaliza el formato de la fecha
    \predate{\begin{center}\huge}
    \postdate{\par\end{center}\vspace{5em}}
    
    \title{#2}
    \author{\href{https://losdeldgiim.github.io/}{Los Del DGIIM}}
    \date{Granada, #7}
    \thispagestyle{empty}               % Sin encabezado ni pie de página
    \maketitle
    \vfill
}
    \portadaExamen{ffccA4.jpg}{Variable Compleja I\\Examen XVIII}{Variable Compleja I. Examen XVIII}{MidnightBlue}{-9.5}{28}{2024-2025}{José Juan Urrutia Milán}

    \begin{description}
        \item[Asignatura] Variable Compleja I.
        \item[Curso Académico] 2024-25.
        \item[Grado] Doble Grado en Ingeniería Informática y Matemáticas.
        \item[Grupo] Único.
        \item[Profesor] Javier Merí de la Maza.
        \item[Descripción] Convocatoria Ordinaria.
        \item[Fecha] 6 de Junio de 2025.
        \item[Duración] 3.5 horas.
    \end{description}
    \newpage

    \begin{ejercicio}[2.5 puntos]
        Para cada $n\in \mathbb{N}\cup \{0\}$, sea $f_n:\mathbb{C}\to\mathbb{C}$ la función dada por
        \begin{equation*}
            f_n(z) = \int_{n}^{n+1} \dfrac{\cos(t^n+e^z)}{1+t^n}~dt  \qquad \forall z\in \mathbb{C}
        \end{equation*}
        \begin{enumerate}[label=\alph*)]
            \item Probar que $f_n\in \cc{H}(\mathbb{C})$.
            \item Probar que la serie de funciones $\sum\limits_{n\geq 0} f_n$ converge en $\mathbb{C}$ y que su suma es una función entera.
        \end{enumerate}
    \end{ejercicio}

    \begin{ejercicio}[2.5 puntos]
        Integrando la función $z\longmapsto \dfrac{\log(z+1)}{1+z^2}$ sobre un camino cerrado que recorra la frontera del conjunto $\{z\in \mathbb{C} : |z| < R, \text{Im}z > 0\}$, con $R\in \mathbb{R}$ y $R>1$, evaluar la integral
        \begin{equation*}
            \int_{-\infty}^{+\infty} \dfrac{\log(1+x^2)}{1+x^2}~dx 
        \end{equation*}
    \end{ejercicio}

    \begin{ejercicio}[2.5 puntos]
        Sea $f\in \cc{H}(D(0,1))$.
        \begin{enumerate}[label=\alph*)]
            \item Probar que la función $f^\ast:D(0,1)\to \mathbb{C}$ dada por $f^\ast(z) = \overline{f(\overline{z})}$ es holomorfa en $D(0,1)$.
            \item Supongamos que se cumple $|f(z)| \leq |f(\ol{z})|$ para cada $z\in D(0,1)$. Probar que existe $\lm\in \bb{T}$ tal que $f(z)=\lm f^\ast(z)$ para cada $z\in D(0,1)$.
        \end{enumerate}
    \end{ejercicio}

    \begin{ejercicio}[2.5 puntos]
        Sea $S\subset \mathbb{C}$ un conjunto sin puntos de acumulación y $f\in \cc{H(\mathbb{C}\setminus S)}$ verificando que tiene un polo en cada punto de $S$ (en tal caso se dice que $f$ es una función meromorfa). Supongamos a demás que $f$ diverge en infinito. Se pide:
        \begin{enumerate}[label=\alph*)]
            \item Probar que $f$ tiene una cantidad finita de ceros.
            \item Probar que $f$ es una función racional.
        \end{enumerate}
    \end{ejercicio}

    \newpage
    \setcounter{ejercicio}{0}
    \begin{ejercicio}[2.5 puntos]
        Para cada $n\in \mathbb{N}\cup \{0\}$, sea $f_n:\mathbb{C}\to\mathbb{C}$ la función dada por
        \begin{equation*}
            f_n(z) = \int_{n}^{n+1} \dfrac{\cos(t^n+e^z)}{1+t^n}~dt  \qquad \forall z\in \mathbb{C}
        \end{equation*}
        \begin{enumerate}[label=\alph*)]
            \item Probar que $f_n\in \cc{H}(\mathbb{C})$.

                Dado $n\in \mathbb{N}\cup \{0\}$, como la función 
                \Func{\varphi_{n,z}}{[n,n+1]}{\bb{C}}{t}{\dfrac{\cos(t^n+e^z)}{1+t^n}}
                es continua para cada $z\in \mathbb{C}$, entonces $f_n$ está bien definida. Ahora, como
                \Func{\phi_{n,t}}{\bb{C}}{\bb{C}}{z}{\dfrac{\cos(t^n+e^z)}{1+t^n}}
                es holomorfa para todo $t\in [n,n+1]$, tenemos por el Teorema de Funciones definidas como una Integral Dependiente de parámetro que $f_n$ es holomorfa, para cada $n\in \mathbb{N}$.
            \item Probar que la serie de funciones $\sum\limits_{n\geq 0} f_n$ converge en $\mathbb{C}$ y que su suma es una función entera.

                Sea $K\subset \mathbb{C}$ un conjunto compacto, como la función $z\longmapsto |\cos(e^z)|+|\sen(e^z)|$ es continua como composición de funciones continuas, podemos considerar:
                \begin{equation*}
                    \beta = \max\{|\cos(e^z)|+|\sen(e^z)| : z\in K\}
                \end{equation*}

                Para $n\in \mathbb{N}\cup \{0\}$ y $z\in K$, tenemos que:
                \begin{equation*}
                    |f_n(z)| = \left|\int_{n}^{n+1} \dfrac{\cos(t^n+e^x)}{1+t^n}~dt \right| \leq \sup\left\{\left|\dfrac{\cos(t^n+e^x)}{1+t^n}\right| : t\in [n,n+1]\right\}
                \end{equation*}
                Y como:
                \begin{equation*}
                    |1+t^n| = 1+t^n \geq 1+n^n \qquad \forall t\in [n,n+1]
                \end{equation*}
                Y además:
                \begin{multline*}
                    |\cos(t^n+e^z)| \leq |\cos(t^n)\cos(e^z)| + |\sen(t^n)\sen(e^z)| \leq |\cos(e^z)| + |\sen(e^z)|\leq \beta \\ \forall t\in [n,n+1], \forall z\in K
                \end{multline*}
                Tomando $M_n = \nicefrac{\beta}{(1+n^n)}$ para cada $n\in \mathbb{N}\cup \{0\}$, concluimos que:
                \begin{equation*}
                    |f_n(z)| \leq \sup\left\{\left|\dfrac{\cos(t^n+e^x)}{1+t^n}\right| : t\in [n,n+1]\right\} \leq M_n \qquad \forall z\in \mathbb{C}, \forall n\in \mathbb{N}\cup \{0\}
                \end{equation*}
                Como:
                \begin{equation*}
                    \dfrac{M_{n+1}}{M_n} = \dfrac{1+n^n}{1+{(n+1)}^{n+1}} \to 0
                \end{equation*}
                Por el Criterio del Cociente, tenemos que $\sum\limits_{n\geq 0}M_n$ es convergente. Por el Test de Weierstrass, deducimos que $\sum\limits_{n\geq 0}f_n$ converge absoluta y uniformemente en cada compacto $K\subset \mathbb{C}$. Como la convergencia absoluta implica la puntual, también tendremos que $\sum\limits_{n\geq 0}f_n$ converge puntualmente en cada compacto $K\subset \mathbb{C}$.

                Además, como tenemos que:
                \begin{equation*}
                    \mathbb{C} = \bigcup_{z\in \mathbb{C}} \{z\}
                \end{equation*}
                Y cada $\{z\}$ es compacto, como la convergencia puntual es una propiedad puntual, concluimos que $\sum\limits_{n\geq 0}f_n$ converge uniformemente en $\mathbb{C}$, a la función $f:\mathbb{C}\to \mathbb{C}$ dada por:
                \begin{equation*}
                    f(z) = \lim_{n\to\infty} f_n(z) = \int_{-\infty}^{+\infty} \dfrac{\cos(t^n+e^z)}{1+t^n}~dt  \qquad \forall z\in \mathbb{C}
                \end{equation*} 
                Como $f_n\in \cc{H}(\mathbb{C})$ para cada $n\in \mathbb{N}\cup\{0\}$, aplicando el Teorema de Convergencia de Weierstrass, deducimos que $f\in \cc{H}(\mathbb{C})$.
        \end{enumerate}
    \end{ejercicio}

    \begin{ejercicio}[2.5 puntos]
        Integrando la función $z\longmapsto \dfrac{\log(z+1)}{1+z^2}$ sobre un camino cerrado que recorra la frontera del conjunto $\{z\in \mathbb{C} : |z| < R, \text{Im}z > 0\}$, con $R\in \mathbb{R}$ y $R>1$, evaluar la integral
        \begin{equation*}
            \int_{-\infty}^{+\infty} \dfrac{\log(1+x^2)}{1+x^2}~dx 
        \end{equation*}
    \end{ejercicio}

    \begin{ejercicio}[2.5 puntos]
        Sea $f\in \cc{H}(D(0,1))$.
        \begin{enumerate}[label=\alph*)]
            \item Probar que la función $f^\ast:D(0,1)\to \mathbb{C}$ dada por $f^\ast(z) = \overline{f(\overline{z})}$ es holomorfa en $D(0,1)$.
            \item Supongamos que se cumple $|f(z)| \leq |f(\ol{z})|$ para cada $z\in D(0,1)$. Probar que existe $\lm\in \bb{T}$ tal que $f(z)=\lm f^\ast(z)$ para cada $z\in D(0,1)$.
        \end{enumerate}
    \end{ejercicio}

    \begin{ejercicio}[2.5 puntos]
        Sea $S\subset \mathbb{C}$ un conjunto sin puntos de acumulación y $f\in \cc{H(\mathbb{C}\setminus S)}$ verificando que tiene un polo en cada punto de $S$ (en tal caso se dice que $f$ es una función meromorfa). Supongamos a demás que $f$ diverge en infinito. Se pide:
        \begin{enumerate}[label=\alph*)]
            \item Probar que $f$ tiene una cantidad finita de ceros.

                Por comodidad, a lo largo del ejercicio escribiremos $\Omega=\mathbb{C}\setminus S$.

                Es bien sabido que si $f$ es una función continua, entonces:
                \begin{equation*}
                    Z(f) = \{z\in Domf : f(z) = 0\}
                \end{equation*}
                Es un conjunto cerrado, ya que si $\{z_n\}\to z$ con $z_n\in Z(f)$ para cada $n\in \mathbb{N}$, entonces la continuidad de $f$ nos dice que:
                \begin{equation*}
                    0 \leftarrow \{0\} = \{f(z_n)\} \to f(z)
                \end{equation*}
                Por lo que $f(z) = 0$, luego $z\in Z(f)$. Una vez recordada esta propiedad, continuamos con el ejercicio:

                \begin{itemize}
                    \item En primer lugar, veamos que $Z(f)\cap \overline{D}(0,r)$ es finito para cada $r\in \mathbb{R}^+$: Por reducción al absurdo, supongamos que existe $R\in \mathbb{R}^+$ de forma que $Z(f)\cap \overline{D}(0,R)$ es infinito. En dicho caso, tenemos un conjunto cerrado (como intersección de cerrados) y acotado (contenido en $D(0,R)$), luego un conjunto compacto e infinito, luego (por lo que vimos en el Tema 9 antes del Principio de Identidad) tendrá puntos de acumulación. De esta forma:
                        \begin{equation*}
                            f(z) = 0 \qquad \forall z\in Z(f)\cap \overline{D}(0,R)
                        \end{equation*}
                        Como $(Z(f)\cap \overline{D}(0,1))' \cap \Omega \neq \emptyset $, por lo que $f(z) = 0$ para todo $z\in \Omega$. Sin embargo, entonces $f$ no diverge en infinito (tendería a 0), \underline{contradicción}, luego $Z(f)\cap \overline{D}(0,r)$ ha de ser finito, para todo $r\in \mathbb{R}^+$.
                    \item Por el apartado anterior, $Z(f)\cap \overline{D}(0,r)$ ha de ser finito $\forall r\in \mathbb{R}^+$, por lo que, intuitivamente, la única forma de que $Z(f)$ sea infinito es que podamos encontrarnos ceros tan lejos como queramos. Veremos a continuación que esta idea contradice que $f$ diverja:

                        Por reducción al absurdo, supongamos que $Z(f)$ es infinito. Sea $n\in \mathbb{N}$, como $Z(f)\cap \overline{D}(0,n)$ es finito, podemos encontrar $z_n\in Z(f)\setminus \overline{D}(0,n)$; es decir, podemos encontrar $z_n\in \Omega$ con $|z_n| > n$ y $f(z_n) = 0$. Tenemos por tanto que $z_n\to \infty$ y como $\lim\limits_{z\to \infty} f(z) = \infty$, tendremos que:
                        \begin{equation*}
                            \{f(z_n)\} \to \infty
                        \end{equation*}
                        Sin embargo, $f(z_n) = 0$ para todo $n\in \mathbb{N}$, \underline{contradicción}, que venía de suponer que $Z(f)$ es infinito.
                \end{itemize}
            \item Probar que $f$ es una función racional.

                Como $Z(f)$ es finito, suponemos que tiene $n$ elementos:
                \begin{equation*}
                    Z(f) = \{x_1,x_2,\ldots,x_n\}
                \end{equation*}
                Por tanto, por el Teorema de Caracterización de los ceros de una función holomorfa aplicado $n$ veces, tenemos que $\exists k_1,k_2,\ldots,k_n \in \mathbb{N}$ y una función $\varphi:\Omega\to \mathbb{C}$ que no se anula en $Z(f)$, de forma que:
                \begin{equation*}
                    f(z) = \prod_{j=1}^{n}{(z-x_j)}^{k_j} \varphi(z) \qquad \forall z\in \Omega
                \end{equation*}
                Como $\varphi$ no se anula en $Z(f)$ y tampoco se anula en ningún punto de $\Omega\setminus Z(f)$, concluimos que $\varphi$ no se anula en ningún punto de $\Omega$, lo que nos permite definir $\phi:\mathbb{C}\to \mathbb{C}$ dada por:
                \begin{equation*}
                    \phi(z) = \dfrac{1}{\varphi(z)} \qquad \forall z\in \mathbb{C}\setminus S, \quad \qquad \phi(z) = 0 \qquad \forall z\in S
                \end{equation*}
                Que es holomorfa en $\mathbb{C}\setminus S$. Sea $s\in S$, como:
                \begin{equation*}
                    \lim_{z\to s} \phi(z) = \lim_{z\to s} \dfrac{1}{f(z)} = 0 = \phi(s)
                \end{equation*}
                Tendremos que $\phi$ es continua en $s$. Aplicando el Teorema de Extensión de Riemann, tenemso que $\phi$ será holomorfa en $s$, para todo $s\in S$, luego $\phi$ es una función entera. Además, podemos escribir:
                \begin{equation*}
                    f(z) = \prod_{j=1}^{n}{(z-x_j)}^{k_j} \varphi(z) = \dfrac{\prod\limits_{j=1}^{n}{(z-x_j)}^{k_j}}{\phi(z)} \qquad \forall z\in \Omega
                \end{equation*}
                De esta forma, si probamos que $\phi$ es una función polinómica ya tenemos que $f$ es racional. Para ello, como:
                \begin{equation*}
                    \lim_{z\to\infty} \dfrac{\prod\limits_{j=1}^{n}{(z-x_j)}^{k_j}}{\phi(z)} = \lim_{z\to\infty} f(z) =  \infty
                \end{equation*}
                Tendremos que $\forall M\in \mathbb{R}^+$ $\exists R\in \mathbb{R}^+$ de forma que:
                \begin{equation*}
                    \dfrac{\left|\prod\limits_{j=1}^{n}{(z-x_j)}^{k_j}\right|}{|\phi(z)|}  = |f(z)| > M \qquad \forall z\in \mathbb{C}\setminus \overline{D}(0,R)
                \end{equation*}
                Por tanto:
                \begin{equation*}
                    \left|\prod\limits_{j=1}^{n}{(z-x_j)}^{k_j}\right| > M|\phi(z)| \qquad \forall z\in \mathbb{C}\setminus\overline{D}(0,R)
                \end{equation*}
                Es decir, $\phi$ es una función entera con crecimiento subpolinómico en el exterior de un disco. Vimos en un ejercicio del Tema 9 que entonces $\phi$ es una función polinómica, como queríamos probar.
        \end{enumerate}
    \end{ejercicio}

\end{document}
