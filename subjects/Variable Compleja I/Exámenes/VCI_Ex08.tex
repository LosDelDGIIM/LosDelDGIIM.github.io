\documentclass[12pt]{article}

% Idioma y codificación
\usepackage[spanish, es-tabla]{babel}       %es-tabla para que se titule "Tabla"
\usepackage[utf8]{inputenc}

% Márgenes
\usepackage[a4paper,top=3cm,bottom=2.5cm,left=3cm,right=3cm]{geometry}

% Comentarios de bloque
\usepackage{verbatim}

% Paquetes de links
\usepackage[hidelinks]{hyperref}    % Permite enlaces
\usepackage{url}                    % redirecciona a la web

% Más opciones para enumeraciones
\usepackage{enumitem}

% Personalizar la portada
\usepackage{titling}

% Paquetes de tablas
\usepackage{multirow}


%------------------------------------------------------------------------

%Paquetes de figuras
\usepackage{caption}
\usepackage{subcaption} % Figuras al lado de otras
\usepackage{float}      % Poner figuras en el sitio indicado H.


% Paquetes de imágenes
\usepackage{graphicx}       % Paquete para añadir imágenes
\usepackage{transparent}    % Para manejar la opacidad de las figuras

% Paquete para usar colores
\usepackage[dvipsnames]{xcolor}
\usepackage{pagecolor}      % Para cambiar el color de la página

% Habilita tamaños de fuente mayores
\usepackage{fix-cm}

% Para los gráficos
\usepackage{tikz}

% Para poder situar los nodos en los grafos
\usetikzlibrary{positioning}


%------------------------------------------------------------------------

% Paquetes de matemáticas
\usepackage{mathtools, amsfonts, amssymb, mathrsfs}
\usepackage[makeroom]{cancel}     % Simplificar tachando
\usepackage{polynom}    % Divisiones y Ruffini
\usepackage{units} % Para poner fracciones diagonales con \nicefrac

\usepackage{pgfplots}   %Representar funciones
\pgfplotsset{compat=1.18}  % Versión 1.18

\usepackage{tikz-cd}    % Para usar diagramas de composiciones
\usetikzlibrary{calc}   % Para usar cálculo de coordenadas en tikz

%Definición de teoremas, etc.
\usepackage{amsthm}
%\swapnumbers   % Intercambia la posición del texto y de la numeración

\theoremstyle{plain}

\makeatletter
\@ifclassloaded{article}{
  \newtheorem{teo}{Teorema}[section]
}{
  \newtheorem{teo}{Teorema}[chapter]  % Se resetea en cada chapter
}
\makeatother

\newtheorem{coro}{Corolario}[teo]           % Se resetea en cada teorema
\newtheorem{prop}[teo]{Proposición}         % Usa el mismo contador que teorema
\newtheorem{lema}[teo]{Lema}                % Usa el mismo contador que teorema

\theoremstyle{remark}
\newtheorem*{observacion}{Observación}

\theoremstyle{definition}

\makeatletter
\@ifclassloaded{article}{
  \newtheorem{definicion}{Definición} [section]     % Se resetea en cada chapter
}{
  \newtheorem{definicion}{Definición} [chapter]     % Se resetea en cada chapter
}
\makeatother

\newtheorem*{notacion}{Notación}
\newtheorem*{ejemplo}{Ejemplo}
\newtheorem*{ejercicio*}{Ejercicio}             % No numerado
\newtheorem{ejercicio}{Ejercicio} [section]     % Se resetea en cada section


% Modificar el formato de la numeración del teorema "ejercicio"
\renewcommand{\theejercicio}{%
  \ifnum\value{section}=0 % Si no se ha iniciado ninguna sección
    \arabic{ejercicio}% Solo mostrar el número de ejercicio
  \else
    \thesection.\arabic{ejercicio}% Mostrar número de sección y número de ejercicio
  \fi
}


% \renewcommand\qedsymbol{$\blacksquare$}         % Cambiar símbolo QED
%------------------------------------------------------------------------

% Paquetes para encabezados
\usepackage{fancyhdr}
\pagestyle{fancy}
\fancyhf{}

\newcommand{\helv}{ % Modificación tamaño de letra
\fontfamily{}\fontsize{12}{12}\selectfont}
\setlength{\headheight}{15pt} % Amplía el tamaño del índice


%\usepackage{lastpage}   % Referenciar última pag   \pageref{LastPage}
\fancyfoot[C]{\thepage}

%------------------------------------------------------------------------

% Conseguir que no ponga "Capítulo 1". Sino solo "1."
\makeatletter
\@ifclassloaded{book}{
  \renewcommand{\chaptermark}[1]{\markboth{\thechapter.\ #1}{}} % En el encabezado
    
  \renewcommand{\@makechapterhead}[1]{%
  \vspace*{50\p@}%
  {\parindent \z@ \raggedright \normalfont
    \ifnum \c@secnumdepth >\m@ne
      \huge\bfseries \thechapter.\hspace{1em}\ignorespaces
    \fi
    \interlinepenalty\@M
    \Huge \bfseries #1\par\nobreak
    \vskip 40\p@
  }}
}
\makeatother

%------------------------------------------------------------------------
% Paquetes de cógido
\usepackage{minted}
\renewcommand\listingscaption{Código fuente}

\usepackage{fancyvrb}
% Personaliza el tamaño de los números de línea
\renewcommand{\theFancyVerbLine}{\small\arabic{FancyVerbLine}}

% Estilo para C++
\newminted{cpp}{
    frame=lines,
    framesep=2mm,
    baselinestretch=1.2,
    linenos,
    escapeinside=||
}

% para minted
\definecolor{LightGray}{rgb}{0.95,0.95,0.92}
\setminted{
    linenos=true,
    stepnumber=5,
    numberfirstline=true,
    autogobble,
    breaklines=true,
    breakautoindent=true,
    breaksymbolleft=,
    breaksymbolright=,
    breaksymbolindentleft=0pt,
    breaksymbolindentright=0pt,
    breaksymbolsepleft=0pt,
    breaksymbolsepright=0pt,
    fontsize=\footnotesize,
    bgcolor=LightGray,
    numbersep=10pt
}


\usepackage{listings} % Para incluir código desde un archivo

\renewcommand\lstlistingname{Código Fuente}
\renewcommand\lstlistlistingname{Índice de Códigos Fuente}

% Definir colores
\definecolor{vscodepurple}{rgb}{0.5,0,0.5}
\definecolor{vscodeblue}{rgb}{0,0,0.8}
\definecolor{vscodegreen}{rgb}{0,0.5,0}
\definecolor{vscodegray}{rgb}{0.5,0.5,0.5}
\definecolor{vscodebackground}{rgb}{0.97,0.97,0.97}
\definecolor{vscodelightgray}{rgb}{0.9,0.9,0.9}

% Configuración para el estilo de C similar a VSCode
\lstdefinestyle{vscode_C}{
  backgroundcolor=\color{vscodebackground},
  commentstyle=\color{vscodegreen},
  keywordstyle=\color{vscodeblue},
  numberstyle=\tiny\color{vscodegray},
  stringstyle=\color{vscodepurple},
  basicstyle=\scriptsize\ttfamily,
  breakatwhitespace=false,
  breaklines=true,
  captionpos=b,
  keepspaces=true,
  numbers=left,
  numbersep=5pt,
  showspaces=false,
  showstringspaces=false,
  showtabs=false,
  tabsize=2,
  frame=tb,
  framerule=0pt,
  aboveskip=10pt,
  belowskip=10pt,
  xleftmargin=10pt,
  xrightmargin=10pt,
  framexleftmargin=10pt,
  framexrightmargin=10pt,
  framesep=0pt,
  rulecolor=\color{vscodelightgray},
  backgroundcolor=\color{vscodebackground},
}

%------------------------------------------------------------------------

% Comandos definidos
\newcommand{\bb}[1]{\mathbb{#1}}
\newcommand{\cc}[1]{\mathcal{#1}}

% I prefer the slanted \leq
\let\oldleq\leq % save them in case they're every wanted
\let\oldgeq\geq
\renewcommand{\leq}{\leqslant}
\renewcommand{\geq}{\geqslant}

% Si y solo si
\newcommand{\sii}{\iff}

% Letras griegas
\newcommand{\eps}{\epsilon}
\newcommand{\veps}{\varepsilon}
\newcommand{\lm}{\lambda}

\newcommand{\ol}{\overline}
\newcommand{\ul}{\underline}
\newcommand{\wt}{\widetilde}
\newcommand{\wh}{\widehat}

\let\oldvec\vec
\renewcommand{\vec}{\overrightarrow}

% Derivadas parciales
\newcommand{\del}[2]{\frac{\partial #1}{\partial #2}}
\newcommand{\Del}[3]{\frac{\partial^{#1} #2}{\partial #3^{#1}}}
\newcommand{\deld}[2]{\dfrac{\partial #1}{\partial #2}}
\newcommand{\Deld}[3]{\dfrac{\partial^{#1} #2}{\partial #3^{#1}}}


\newcommand{\AstIg}{\stackrel{(\ast)}{=}}
\newcommand{\Hop}{\stackrel{L'H\hat{o}pital}{=}}

\newcommand{\red}[1]{{\color{red}#1}} % Para integrales, destacar los cambios.

% Método de integración
\newcommand{\MetInt}[2]{
    \left[\begin{array}{c}
        #1 \\ #2
    \end{array}\right]
}

% Declarar aplicaciones
% 1. Nombre aplicación
% 2. Dominio
% 3. Codominio
% 4. Variable
% 5. Imagen de la variable
\newcommand{\Func}[5]{
    \begin{equation*}
        \begin{array}{rrll}
            #1:& #2 & \longrightarrow & #3\\
               & #4 & \longmapsto & #5
        \end{array}
    \end{equation*}
}

%------------------------------------------------------------------------

\let\oldRe\Re % save them in case they're every wanted
\let\oldIm\Im
\renewcommand{\Re}{\operatorname{Re}} % redefine them
\renewcommand{\Im}{\operatorname{Im}}
\DeclareMathOperator{\Log}{Log}
\DeclareMathOperator{\Arg}{Arg}

\begin{document}

    % 1. Foto de fondo
    % 2. Título
    % 3. Encabezado Izquierdo
    % 4. Color de fondo
    % 5. Coord x del titulo
    % 6. Coord y del titulo
    % 7. Fecha

    
    % 1. Foto de fondo
% 2. Título
% 3. Encabezado Izquierdo
% 4. Color de fondo
% 5. Coord x del titulo
% 6. Coord y del titulo
% 7. Fecha

\newcommand{\portada}[7]{

    \portadaBase{#1}{#2}{#3}{#4}{#5}{#6}{#7}
    \portadaBook{#1}{#2}{#3}{#4}{#5}{#6}{#7}
}

\newcommand{\portadaExamen}[7]{

    \portadaBase{#1}{#2}{#3}{#4}{#5}{#6}{#7}
    \portadaArticle{#1}{#2}{#3}{#4}{#5}{#6}{#7}
}




\newcommand{\portadaBase}[7]{

    % Tiene la portada principal y la licencia Creative Commons
    
    % 1. Foto de fondo
    % 2. Título
    % 3. Encabezado Izquierdo
    % 4. Color de fondo
    % 5. Coord x del titulo
    % 6. Coord y del titulo
    % 7. Fecha
    
    
    \thispagestyle{empty}               % Sin encabezado ni pie de página
    \newgeometry{margin=0cm}        % Márgenes nulos para la primera página
    
    
    % Encabezado
    \fancyhead[L]{\helv #3}
    \fancyhead[R]{\helv \nouppercase{\leftmark}}
    
    
    \pagecolor{#4}        % Color de fondo para la portada
    
    \begin{figure}[p]
        \centering
        \transparent{0.3}           % Opacidad del 30% para la imagen
        
        \includegraphics[width=\paperwidth, keepaspectratio]{assets/#1}
    
        \begin{tikzpicture}[remember picture, overlay]
            \node[anchor=north west, text=white, opacity=1, font=\fontsize{60}{90}\selectfont\bfseries\sffamily, align=left] at (#5, #6) {#2};
            
            \node[anchor=south east, text=white, opacity=1, font=\fontsize{12}{18}\selectfont\sffamily, align=right] at (9.7, 3) {\textbf{\href{https://losdeldgiim.github.io/}{Los Del DGIIM}}};
            
            \node[anchor=south east, text=white, opacity=1, font=\fontsize{12}{15}\selectfont\sffamily, align=right] at (9.7, 1.8) {Doble Grado en Ingeniería Informática y Matemáticas\\Universidad de Granada};
        \end{tikzpicture}
    \end{figure}
    
    
    \restoregeometry        % Restaurar márgenes normales para las páginas subsiguientes
    \pagecolor{white}       % Restaurar el color de página
    
    
    \newpage
    \thispagestyle{empty}               % Sin encabezado ni pie de página
    \begin{tikzpicture}[remember picture, overlay]
        \node[anchor=south west, inner sep=3cm] at (current page.south west) {
            \begin{minipage}{0.5\paperwidth}
                \href{https://creativecommons.org/licenses/by-nc-nd/4.0/}{
                    \includegraphics[height=2cm]{assets/Licencia.png}
                }\vspace{1cm}\\
                Esta obra está bajo una
                \href{https://creativecommons.org/licenses/by-nc-nd/4.0/}{
                    Licencia Creative Commons Atribución-NoComercial-SinDerivadas 4.0 Internacional (CC BY-NC-ND 4.0).
                }\\
    
                Eres libre de compartir y redistribuir el contenido de esta obra en cualquier medio o formato, siempre y cuando des el crédito adecuado a los autores originales y no persigas fines comerciales. 
            \end{minipage}
        };
    \end{tikzpicture}
    
    
    
    % 1. Foto de fondo
    % 2. Título
    % 3. Encabezado Izquierdo
    % 4. Color de fondo
    % 5. Coord x del titulo
    % 6. Coord y del titulo
    % 7. Fecha


}


\newcommand{\portadaBook}[7]{

    % 1. Foto de fondo
    % 2. Título
    % 3. Encabezado Izquierdo
    % 4. Color de fondo
    % 5. Coord x del titulo
    % 6. Coord y del titulo
    % 7. Fecha

    % Personaliza el formato del título
    \pretitle{\begin{center}\bfseries\fontsize{42}{56}\selectfont}
    \posttitle{\par\end{center}\vspace{2em}}
    
    % Personaliza el formato del autor
    \preauthor{\begin{center}\Large}
    \postauthor{\par\end{center}\vfill}
    
    % Personaliza el formato de la fecha
    \predate{\begin{center}\huge}
    \postdate{\par\end{center}\vspace{2em}}
    
    \title{#2}
    \author{\href{https://losdeldgiim.github.io/}{Los Del DGIIM}}
    \date{Granada, #7}
    \maketitle
    
    \tableofcontents
}




\newcommand{\portadaArticle}[7]{

    % 1. Foto de fondo
    % 2. Título
    % 3. Encabezado Izquierdo
    % 4. Color de fondo
    % 5. Coord x del titulo
    % 6. Coord y del titulo
    % 7. Fecha

    % Personaliza el formato del título
    \pretitle{\begin{center}\bfseries\fontsize{42}{56}\selectfont}
    \posttitle{\par\end{center}\vspace{2em}}
    
    % Personaliza el formato del autor
    \preauthor{\begin{center}\Large}
    \postauthor{\par\end{center}\vspace{3em}}
    
    % Personaliza el formato de la fecha
    \predate{\begin{center}\huge}
    \postdate{\par\end{center}\vspace{5em}}
    
    \title{#2}
    \author{\href{https://losdeldgiim.github.io/}{Los Del DGIIM}}
    \date{Granada, #7}
    \thispagestyle{empty}               % Sin encabezado ni pie de página
    \maketitle
    \vfill
}
    \portadaExamen{ffccA4.jpg}{Variable Compleja I\\Examen VIII}{Variable Compleja I. Examen VIII}{MidnightBlue}{-9.5}{28}{2024-2025}{Arturo Olivares Martos}

    \begin{description}
        \item[Asignatura] Variable Compleja I.
        \item[Curso Académico] 2017-18.
        \item[Grado] Doble Grado en Ingeniería Informática y Matemáticas.
        \item[Grupo] Único.
        \item[Profesor] Javier Merí de la Maza.
        \item[Descripción] Prueba Intermedia.
        \item[Fecha] 25 de Abril de 2018.
        \item[Duración] 120 minutos.
    \end{description}
    \newpage

    \begin{ejercicio}[3.5 puntos]
        Probar que la serie $\sum\limits_{n \geq 0} e^{-zn}$ converge absolutamente en todo punto del dominio $\Omega = \{z \in \mathbb{C} : \Re z > 0\}$ y uniformemente en cada subconjunto compacto contenido en $\Omega$. Deducir que la función $g : \Omega \to \mathbb{C}$ dada por
        \[
            g(z) = \sum\limits_{n=0}^{\infty} e^{-zn}
        \]
        es continua en $\Omega$ y calcular $\displaystyle \int_{C(2,1)} g(z) \, dz$.
    \end{ejercicio}

    \begin{ejercicio}[3.5 puntos]
        Estudiar la derivabilidad de las funciones $f , g : \mathbb{C} \to \mathbb{C}$ dadas por
        \[
            f(z) = \cos\left(\ol{z}\right)\qquad g(z) = (z - 1) f(z)\qquad \forall z \in \mathbb{C}.
        \]
    \end{ejercicio}

    \begin{ejercicio}[3 puntos]
        Sea $\Omega$ un abierto de $\mathbb{C}$ y $f \in \cc{H} (\Omega)$. Probar que la función $|f|$ no puede tener ningún máximo relativo estricto. Es decir, no pueden existir $z_0 \in \Omega$ y $r\in \bb{R}^+$ con $\ol{D}(z_0 , r) \subset \Omega$ de modo que $|f(z_0 )| > |f(z)|$ para cada $z \in \ol{D}(z_0 , r) \setminus \{z_0\}$.
    \end{ejercicio}



    \newpage
    \setcounter{ejercicio}{0}


    \begin{ejercicio}[3.5 puntos]
        Probar que la serie $\sum\limits_{n \geq 0} e^{-zn}$ converge absolutamente en todo punto del dominio $\Omega = \{z \in \mathbb{C} : \Re z > 0\}$ y uniformemente en cada subconjunto compacto contenido en $\Omega$. Deducir que la función $g : \Omega \to \mathbb{C}$ dada por
        \[
            g(z) = \sum\limits_{n=0}^{\infty} e^{-zn}
        \]
        es continua en $\Omega$ y calcular $\displaystyle \int_{C(2,1)} g(z) \, dz$.\\

        Estudiamos en primer lugar la convergencia sobre compactos. Sea $K$ un compacto de $\Omega$. Como $K$ es compacto y la parte real es continua, tenemos que $\exists M\in \bb{R}^+$ tal que:
        \begin{equation*}
            M=\min\{\Re(z) : z \in K\}
        \end{equation*}
        
        Por lo tanto, para todo $z \in K$ se tiene que:
        \begin{equation*}
            \left|e^{-zn}\right| = e^{-\Re(zn)} = e^{-n\Re(z)} \leq e^{-nM} = (e^{-M})^n \qquad \forall n \in \bb{N}.
        \end{equation*}

        Como $e^{-M} < 1$, la serie $\sum\limits_{n=0}^{\infty} (e^{-M})^n$ converge. Por el Test de Weierstrass, la serie de partida converge uniformemente en $K$.\\

        Para la convergencia absoluta, consideramos $z\in \Omega$ fijo. Como $\{z\}$ es compacto, por el Test de Weierstrass tenemos que converge absolutamente en $\{z\}$. Como $z$ es arbitrario, tenemos que la serie converge absolutamente en todo punto de $\Omega$.\\

        Para probar que $g$ es continua, consideramos $z\in \Omega$. Como $\Omega$ es abierto, existe $R\in \bb{R}^+$ tal que $D(z,R)\subset \Omega$. Considerando $\nicefrac{R}{2}$, tenemos que $z\in \ol{D}(z,\nicefrac{R}{2})\subset \Omega$. Por ser este compacto, la serie converge uniformemente en $\ol{D}(z,\nicefrac{R}{2})$. Como el término general de la serie es continuo para todo $n\in \bb{N}$, entonces $g$ es continua en $z$. Como $z$ es arbitrario, $g$ es continua en $\Omega$.\\

        Como el integrando converge uniformemente en el compacto $C(2,1)\subset \Omega$, podemos intercambiar la sumatoria y la integral, algo que nos será de utilidad a la hora de calcular la integral:
        \begin{align*}
            \int_{C(2,1)} g(z)\ dz &= \sum_{n=0}^{\infty} \int_{C(2,1)} e^{-zn}\ dz \AstIg \sum_{n=0}^{\infty} 0 = 0
        \end{align*}
        donde en $(\ast)$ hemos aplicado que dicha exponencial es entera y está definida en un dominio extrellado, luego por el Teorema Local de Cauchy admite primitiva en dicho dominio. Por lo tanto, la integral es nula.
    \end{ejercicio}

    \begin{ejercicio}[3.5 puntos]
        Estudiar la derivabilidad de las funciones $f , g : \mathbb{C} \to \mathbb{C}$ dadas por
        \[
            f(z) = \cos\left(\ol{z}\right)\qquad g(z) = (z - 1) f(z)\qquad \forall z \in \mathbb{C}.
        \]

        Estudiamos primero la función $f$. En vistas a aplicar el Teorema de Cauchy-Riemann, definimos $u, v : \bb{R}^2 \to \bb{R}$ como:
        \begin{align*}
            u(x,y) &= \Re f(x+iy) = \cos(x)\cosh(-y) \\
            v(x,y) &= \Im f(x+iy) = -\sen(x)\senh(-y)
        \end{align*}

        Calculamos las derivadas parciales de $u$ y $v$:
        \begin{align*}
            \frac{\partial u}{\partial x}(x,y) &= -\sen(x)\cosh(-y) \\
            \frac{\partial u}{\partial y}(x,y) &= -\cos(x)\senh(-y) \\
            \frac{\partial v}{\partial x}(x,y) &= -\cos(x)\senh(-y) \\
            \frac{\partial v}{\partial y}(x,y) &= \sen(x)\cosh(-y)
        \end{align*}

        La primera condición de Cauchy-Riemann se cumple si y sólo si:
        \begin{align*}
            \frac{\partial u}{\partial x}(x,y) = \frac{\partial v}{\partial y}(x,y) \iff \sen(x)\cosh(-y) =  0 \iff \sen(x) = 0
        \end{align*}

        La segunda condición de Cauchy-Riemann se cumple si y sólo si:
        \begin{align*}
            \frac{\partial u}{\partial y}(x,y) = -\frac{\partial v}{\partial x}(x,y) \iff \cos(x)\senh(-y) = 0 \iff \left\{
            \begin{array}{c}
                \cos(x) = 0 \\
                \lor
                \senh(-y) = 0
            \end{array}
            \right.
        \end{align*}

        Para que se cumplan ambas condiciones, es necesario que se cumpla que $y=0$ y $\sen(x) = 0$; es decir, $z\in \pi\bb{Z}$. Por tanto, $f$ es holomorfa en $\pi\bb{Z}$ y no es holomorfa en ningún otro punto de $\bb{C}$.\\

        Ahora, estudiamos la función $g$. Fijado $z\in \bb{C}$, distinguimos en función del valor de $z$:
        \begin{itemize}
            \item \ul{Si $z\in \pi\bb{Z}$:} 
            
            En este caso, $f$ es derivable en $z$ y por tanto $g$ también lo es.
            
            \item \ul{Si $z\notin \pi\bb{Z}$:}
            
            Distinguimos dos casos:
            \begin{itemize}
                \item \ul{Si $z=1$:} 
                \begin{equation*}
                    g'(1) = \lim_{z\to 1} \frac{g(z) - g(1)}{z - 1} = \lim_{z\to 1} \frac{(z-1)f(z)}{z-1} = \lim_{z\to 1} f(z)
                    = f(1) = \cos 1
                \end{equation*}

                \item \ul{Si $z\neq 1$:}
                
                Entonces:
                \begin{equation*}
                    f(z) = \dfrac{g(z)}{z-1}
                \end{equation*}

                Supuesto que $g$ es derivable en $z$, entonces $f$ también lo es. Pero como $z\notin \pi\bb{Z}$, $f$ no es derivable en $z$. Por tanto, $g$ no es derivable en $z$.
            \end{itemize}
        \end{itemize}

        Por tanto, $g$ es derivable en $\pi\bb{Z}\cup \{1\}$ y no lo es en ningún otro punto de $\bb{C}$.
    \end{ejercicio}

    \begin{ejercicio}[3 puntos]
        Sea $\Omega$ un abierto de $\mathbb{C}$ y $f \in \cc{H} (\Omega)$. Probar que la función $|f|$ no puede tener ningún máximo relativo estricto. Es decir, no pueden existir $z_0 \in \Omega$ y $r\in \bb{R}^+$ con $\ol{D}(z_0 , r) \subset \Omega$ de modo que $|f(z_0 )| > |f(z)|$ para cada $z \in \ol{D}(z_0 , r) \setminus \{z_0\}$.\\


        Por reducción al absurdo, supongamos que existen $z_0 \in \Omega$ y $r\in \bb{R}^+$ tales que $\ol{D}(z_0 , r) \subset \Omega$ y $|f(z_0 )| > |f(z)|$ para cada $z \in \ol{D}(z_0 , r) \setminus \{z_0\}$.\\

        Entonces, para cada $z\in D(z_0 , r)$, por la fórmul de Cauchy para la circunferencia se tiene que:
        \begin{align*}
            f(z) = \frac{1}{2\pi i} \int_{C(z_0 , r)} \frac{f(w)}{w - z}\ dw
        \end{align*}

        En particular, para $z_0\in D(z_0 , r)$ se tiene que:
        \begin{align*}
            |f(z_0)| &= \frac{1}{2\pi |i|} \left|\int_{C(z_0 , r)} \frac{f(w)}{w - z_0}\ dw\right| \leq \frac{2\pi r}{2\pi} \sup\left\{\dfrac{|f(w)|}{|w - z_0|} : w \in C(z_0 , r)^*\right\} \\
            &= r\cdot \sup\left\{\dfrac{|f(w)|}{r} : w \in C(z_0 , r)^*\right\} = \sup\left\{|f(w)| : w \in C(z_0 , r)^*\right\}
            \\ &\AstIg \max \left\{|f(w)| : w \in C(z_0 , r)^*\right\} < |f(z_0)
        \end{align*}
        donde en $(\ast)$ hemos aplicado que $f$ y el módulo son funciones continuas y $C(z_0 , r)^*$ es compacto, luego dicho máximo se alcanza. La última desigualdad estricta se debe a la hipótesis, puesto que $C(z_0 , r)^*\subset \ol{D}(z_0 , r) \setminus \{z_0\}$. Por tanto, hemos llegado a una contradicción.
    \end{ejercicio}
\end{document}