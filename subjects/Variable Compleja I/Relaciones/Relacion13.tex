\section{Singularidades}

\begin{ejercicio}
    Dados $a, b \in \bb{C}$ con $0 < |a| < |b|$, obtener el desarrollo en serie de Laurent de la función $f$ definida por:
    \Func{f}{\bb{C} \setminus \{a, b\}}{\bb{C}}{z}{\frac{1}{(z - a)(z - b)}}
    en cada uno de los anillos siguientes:
    \begin{enumerate}
        \item $A(0; |a|, |b|)$
        \item $A(0; |b|, \infty)$
        \item $A(a; 0, |b - a|)$
        \item $A(a; |b - a|, \infty)$
    \end{enumerate}
\end{ejercicio}

\begin{ejercicio}
    Obtener el desarrollo en serie de Laurent de la función
    \Func{f}{\bb{C} \setminus \{1, -1\}}{\bb{C}}{z}{\frac{1}{(z^2 - 1)^2}}
    en los anillos:
    \begin{enumerate}
        \item $A(1; 0, 2)$
        \item $A(1; 2, \infty)$
    \end{enumerate}
\end{ejercicio}

\begin{ejercicio}
    En cada uno de los siguientes casos, clasificar las singularidades de la función $f$ y determinar la parte singular de $f$ en cada una de sus singularidades:
    \begin{enumerate}
        \item $f(z) = \dfrac{1 - \cos z}{z^n}$, $\forall z \in \bb{C}^*\qquad (n \in \bb{N})$
        \item $f(z) = z^n \sen(\nicefrac{1}{z})$, $\forall z \in \bb{C}^*\qquad (n \in \bb{N})$
        \item $f(z) = \dfrac{\log(1 + z)}{z^2}$, $\forall z \in \bb{C} \setminus \{0, -1\}$
        \item $f(z) = \dfrac{1}{z(1 - e^{2\pi i z})}$, $\forall z \in \bb{C} \setminus \bb{Z}$
        \item $f(z) = z \tg\left(\dfrac{2\pi z}{2}\right)$, $\forall z \in \bb{C} \setminus \bb{Z}$
    \end{enumerate}
\end{ejercicio}

\begin{ejercicio}
    Sea $\Omega$ un abierto del plano, $a \in \Omega$ y $f \in \cc{H}(\Omega \setminus \{a\})$. ¿Qué relación existe entre las posibles singularidades en el punto $a$ de las funciones $f$ y $f'$?
\end{ejercicio}

\begin{ejercicio}
    Sea $\Omega$ un dominio, $a \in \Omega$ y $f \in \cc{H}(\Omega \setminus \{a\})$ tal que $f(z) \neq 0$ para todo $z \in \Omega \setminus \{a\}$. ¿Qué relación existe entre las posibles singularidades en $a$ de las funciones $f$ y $\nicefrac{1}{f}$?
\end{ejercicio}

\begin{ejercicio}
    Sea $\Omega$ un abierto del plano, $a \in \Omega$ y $f, g \in \cc{H}(\Omega \setminus \{a\})$. Estudiar el comportamiento en el punto $a$ de las funciones $f + g$ y $f g$, suponiendo conocido el de $f$ y $g$.
\end{ejercicio}

\begin{ejercicio}
    La función $f$ es holomorfa en un entorno del punto $a$ y otra función $g$ tiene un polo de orden $m$ en el punto $f(a)$. ¿Cómo se comporta en $a$ la composición $g \circ f$? ¿Qué ocurre si $a$ es una singularidad esencial en $g$?
\end{ejercicio}

\begin{ejercicio}
    Sea $U$ un entorno reducido de un punto $a \in \bb{C}$ y supongamos que $f \in \cc{H}(U)$ tiene un polo en $a$. Probar que existe $R > 0$ tal que $\bb{C} \setminus D(0,R) \subseteq f(U)$.
\end{ejercicio}

\begin{ejercicio}
    Sea $a$ una singularidad de una función $f$. Probar que la función $\Re f$ no puede estar acotada en un entorno reducido de $a$.
\end{ejercicio}

\begin{ejercicio}
    Sea $\Omega$ un abierto del plano, $a \in \Omega$ y $\{a_n\}$ una sucesión de puntos de $\Omega \setminus \{a\}$ tal que $\{a_n\} \to a$. Consideremos el conjunto $K = \{a_n : n \in \bb{N}\} \cup \{a\}$, sea $f \in \cc{H}(\Omega \setminus K)$ y supongamos que $f$ tiene un polo en $a_n$ para todo $n \in \bb{N}$. Probar que, para todo $\varepsilon > 0$ que verifique $D(a, \varepsilon) \subseteq \Omega$, el conjunto $f^{-1}(D(a, \varepsilon) \setminus K)$ es denso en $\bb{C}$.
\end{ejercicio}