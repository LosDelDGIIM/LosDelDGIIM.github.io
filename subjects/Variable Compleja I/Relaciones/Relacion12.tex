\section{El teorema general de Cauchy}

\begin{ejercicio}
    Enunciar con detalle y demostrar que el índice de un punto respecto a un camino cerrado se conserva por giros, homotecias y traslaciones.
\end{ejercicio}

\begin{ejercicio}
    Sea $\rho : [-\pi,\pi] \to \bb{R}^+$ una función de clase $C^1$, con $\rho(-\pi)=\rho(\pi)$, y sea $\sigma$ el siguiente arco:
    \Func{\sigma}{[-\pi,\pi]}{\bb{C}}{t}{\rho(t)e^{it}}
    Calcular $\Ind_\sigma(z)$ para todo $z \in \bb{C}\setminus\sigma^*$.
\end{ejercicio}

\begin{ejercicio}
    Sean $\gamma_1, \gamma_2 : [a,b] \to \bb{C}$ caminos cerrados y $z \in \bb{C}$ verificando que
    \begin{equation*}
        | \gamma_2(t) - \gamma_1(t) | < | \gamma_1(t) - z | \quad \forall t \in [a,b].
    \end{equation*}
    Probar que $\Ind_{\gamma_1}(z) = \Ind_{\gamma_2}(z)$.
\end{ejercicio}

\begin{ejercicio}
    Sea $\alpha : \bb{R}^+_0 \to \bb{C}$ una función continua, tal que:
    \begin{equation*}
        \alpha(0) = 0 \quad \text{y} \quad \lim_{t \to +\infty} \alpha(t) = +\infty
    \end{equation*}
    y sea $\Omega = \bb{C} \setminus \alpha(\bb{R}^+_0)$. Probar que $\Omega$ es abierto y que existe $f \in \cc{H}(\Omega)$ tal que $e^{f(z)} = z$ para todo $z \in \Omega$.
\end{ejercicio}

\begin{ejercicio}
    Sea $\Omega$ un abierto homológicamente conexo del plano tal que $\bb{R}^+ \subseteq \Omega \subseteq \bb{C}^*$. Probar que existe $f \in \cc{H}(\Omega)$ tal que
    \begin{equation*}
        f(x) = x^x \quad \forall x \in \bb{R}^+.
    \end{equation*}
\end{ejercicio}