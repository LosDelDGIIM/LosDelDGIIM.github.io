\section{Funciones holomorfas}

\begin{ejercicio}
    En cada uno de los siguientes casos, estudiar la derivabilidad de la función $f : \mathbb{C} \to \mathbb{C}$ definida como se indica:
    \begin{enumerate}
        \item $f(z) = z(\Re z)^2$ para todo $z \in \mathbb{C}$.
        
        Sea $z = x + iy\in \bb{C}$, entonces:
        \[
            f(z) = z(\Re z)^2 = (x + iy)x^2 = x^3 + ix^2y.
        \]

        Consideramos ahora las funciones $u,v:\bb{R}^2\to\bb{R}$ dadas por:
        \begin{align*}
            u(x,y) &= \Re f(x+iy) = x^3,&\forall (x,y)\in\bb{R}^2,\\
            v(x,y) &= \Im f(x+iy) = x^2y,&\forall (x,y)\in\bb{R}^2.
        \end{align*}

        Puesto que son polinómicas, es directo ver que $u,v$ son diferenciables en $\bb{R}^2$, por lo que $f$ será derivable en $z=x+iy$ si y solo si se verifican las ecuaciones de Cauchy-Riemann en el punto $(x,y)$, es decir:
        \begin{align*}
            \frac{\partial u}{\partial x}(x,y) &= \frac{\partial v}{\partial y}(x,y),\\
            \frac{\partial u}{\partial y}(x,y) &= -\frac{\partial v}{\partial x}(x,y).
        \end{align*}

        Sustituyendo los valores de dichas derivadas parciales, las ecuaciones de Cauchy-Riemann quedan:
        \begin{equation*}
            \left\{
                \begin{array}{rcl}
                    3x^2 &=& x^2,\\
                    0 &=& -2xy.
                \end{array}
            \right\}
            \iff
            \left\{
                \begin{array}{rcl}
                    x &=& 0,\\
                    xy &=& 0.
                \end{array}
            \right\}
        \end{equation*}

        Por tanto, las ecuaciones de Cauchy-Riemann solo se verifican en el siguiente conjunto $A\subset \bb{R}^2$:
        \[
            A = \{(0,a)\in\bb{R}^2\mid a\in\bb{R}\}\equiv
            \{ai\in\bb{C}\mid a\in\bb{R}\}\subset \bb{C}.
        \]
        
        Por tanto, $f$ es derivable en $A$, mientras que no lo es en ningún punto de $\bb{C}\setminus A$. Es decir, $f$ es derivable en los números imaginarios puros, pero no en ningún otro punto del plano complejo.
        Podemos además definir la función derivada $f':A\to\bb{C}$ como:
        \[
            f'(ai) = \dfrac{\partial u}{\partial x}(0,a) + i\dfrac{\partial v}{\partial x}(0,a) = 0+i\cdot 2\cdot 0 \cdot a = 0\qquad \forall ai\in A.
        \]

        Por tanto, $f$ es constante en $A$. De hecho, se tiene que:
        \[
            f(ai) = 0\qquad \forall ai\in A.
        \]



        \item $f(x + iy) = x^3 - y + i\left(y^3 + \dfrac{x^2}{2}\right)$ para todo $x, y \in \mathbb{R}$.
        
        Definimos las funciones $u,v:\bb{R}^2\to\bb{R}$ dadas por:
        \begin{align*}
            u(x,y) &= \Re f(x+iy) = x^3 - y,&\forall (x,y)\in\bb{R}^2,\\
            v(x,y) &= \Im f(x+iy) = y^3 + \frac{x^2}{2},&\forall (x,y)\in\bb{R}^2.
        \end{align*}

        Puesto que son polinómicas, es directo ver que $u,v$ son diferenciables en $\bb{R}^2$, por lo que $f$ será derivable en $z=x+iy$ si y solo si se verifican las ecuaciones de Cauchy-Riemann en el punto $(x,y)$, es decir:
        \begin{align*}
            \frac{\partial u}{\partial x}(x,y) &= \frac{\partial v}{\partial y}(x,y),\\
            \frac{\partial u}{\partial y}(x,y) &= -\frac{\partial v}{\partial x}(x,y).
        \end{align*}

        Sustituyendo los valores de dichas derivadas parciales, las ecuaciones de Cauchy-Riemann quedan:
        \begin{equation*}
            \left\{
                \begin{array}{rcl}
                    3x^2 &=& 3y^2,\\
                    -1 &=& -x.
                \end{array}
            \right\}
            \iff
            \left\{
                \begin{array}{rcl}
                    x &=& 1,\\
                    y &\in& \{-1,1\}.
                \end{array}
            \right\}
        \end{equation*}

        Por tanto, fijado $z_0=1+i\in\bb{C}$, tenemos que $f$ es derivable en $\{z_0,\ol{z_0}\}$, mientras que no lo es en ningún otro punto del plano complejo. En estos puntos, tenemos que:
        \begin{align*}
            f'(z_0) &= f'(1+i) = \dfrac{\partial u}{\partial x}(1,1) + i\dfrac{\partial v}{\partial x}(1,1) = 3+i,\\
            f'(\ol{z_0}) &= f'(1-i) = \dfrac{\partial u}{\partial x}(1,-1) + i\dfrac{\partial v}{\partial x}(1,-1) = 3+i.
        \end{align*}
        \item $f(x + iy) = \dfrac{x^3 + iy^3}{x^2 + y^2}$ para todo $(x, y) \in \mathbb{R}^2 \setminus \{(0, 0)\}$, con $f(0) = 0$.
        
        Definimos las funciones $u,v:\bb{R}^2\to\bb{R}$ dadas por:
        \begin{align*}
            u(x,y) &= \Re f(x+iy) = \dfrac{x^3}{x^2+y^2},&\forall (x,y)\in\bb{R}^2\setminus\{(0,0)\},\\
            v(x,y) &= \Im f(x+iy) = \dfrac{y^3}{x^2+y^2},&\forall (x,y)\in\bb{R}^2\setminus\{(0,0)\}.
        \end{align*}
        donde, además, $u(0,0)=v(0,0)=0$. Estudiamos la derivabilidad por partes:
        \begin{itemize}
            \item \ul{Estudiamos en $A=\bb{R}^2\setminus\{(0,0)\}$}:
            
            Por el carácter local de la diferenciabilidad, sabemos que $u,v$ con diferenciables en $A$, por lo que $f$ será derivable en $z=x+iy\in A$ si y solo si se verifican las ecuaciones de Cauchy-Riemann en el punto $(x,y)$, es decir:
            \begin{align*}
                \frac{\partial u}{\partial x}(x,y) &= \frac{\partial v}{\partial y}(x,y),\\
                \frac{\partial u}{\partial y}(x,y) &= -\frac{\partial v}{\partial x}(x,y).
            \end{align*}

            Sustituyendo los valores de dichas derivadas parciales, las ecuaciones de Cauchy-Riemann quedan:
            \begin{align*}
                &\left\{
                    \begin{array}{rcl}
                        \dfrac{3x^2(x^2+y^2)-2x^4}{(x^2+y^2)^2} &=& \dfrac{3y^2(x^2+y^2)-2y^4}{(x^2+y^2)^2},\\
                        \dfrac{3y^2(x^2+y^2)-2y^4}{(x^2+y^2)^2} &=& -\dfrac{3x^2(x^2+y^2)-2x^4}{(x^2+y^2)^2}.
                    \end{array}
                \right\}
                \iff\\&\iff
                \left\{
                    \begin{array}{rcl}
                        3x^2(x^2+y^2)-2x^4 &=& 3y^2(x^2+y^2)-2y^4,\\
                        3y^2(x^2+y^2)-2y^4 &=& -3x^2(x^2+y^2)+2x^4.
                    \end{array}
                \right\}
                \iff\\&\iff
                \left\{
                    \begin{array}{rcl}
                        3(x^2+y^2)(x^2-y^2)&=& 2(x^4-y^4),\\
                        3(x^2+y^2)^2&=& 2(x^4+y^4).
                    \end{array}
                \right\}
                \iff\\&\iff
                \left\{
                    \begin{array}{rcl}
                        3(x^4-y^4)&=& 2(x^4-y^4),\\
                        3(x^2+y^2)^2&=& 2(x^4+y^4).
                    \end{array}
                \right\}
                \iff\\&\iff
                \left\{
                    \begin{array}{rcl}
                        x^4=y^4,\\
                        x^4+y^4+6x^2y^2=0.
                    \end{array}
                \right\}
            \end{align*}
            Debido a que la segunda ecuación tan solo se cumple si $x=y=0$ (valor que no pertenece a $A$), tenemos que no se verifican las ecuaciones de Cauchy-Riemann en ningún punto de $A$. Por tanto, $f$ no es derivable en ningún punto de $A$.

            \item \ul{Estudiamos en el origen, $z=0=(0,0)$}:
            
            Lo estudiaremos a partir de la definición de derivada en un punto. Consiste en ver si el siguiente límite existe:
            \begin{equation*}
                \lim_{z\to 0}\dfrac{f(z)}{z} = \lim_{(x,y)\to (0,0)}\dfrac{\dfrac{x^3}{x^2+y^2}+i\dfrac{y^3}{x^2+y^2}}{x+iy} = \lim_{(x,y)\to (0,0)}\dfrac{x^3+iy^3}{(x^2+y^2)(x+iy)}
            \end{equation*}

            Como sabemos, la existencia de este límite equivale a que exista el límite de las partes reales e imaginarias. Por tanto, trabajamos en primer lugar con la parte real:
            \begin{align*}
                \lim_{(x,y)\to (0,0)}\dfrac{x^3}{x^3+xy^2}
                &= \lim_{(x,y)\to (0,0)}\dfrac{x^2}{x^2+y^2} 
            \end{align*}

            Para ver si dicho límite existe, calculamos los límites parciales:
            \begin{align*}
                \lim_{t\to 0} \dfrac{0^2}{0^2+t^2} &= \lim_{t\to 0} 0 = 0,\\
                \lim_{t\to 0} \dfrac{t^2}{t^2+0^2} &= \lim_{t\to 0} 1 = 1.
            \end{align*}
            Como ambos límites parciales no coinciden, el límite no existe. Por tanto, como la parte real no tiene límite, dicho límite no existe y; por tanto, $f$ no es derivable en el origen.
        \end{itemize}

        Por tanto, $f$ no es derivable en ningún punto del plano complejo.
    \end{enumerate}
\end{ejercicio}

\begin{ejercicio}
    Probar que existe una función entera $f$ tal que:
    \[
        \Re f(x + iy) = x^4 - 6x^2y^2 + y^4 \quad \text{para todo } x, y \in \mathbb{R}.
    \]
    Si se exige además que $f(0) = 0$, entonces $f$ es única.\\

    Supongamos que existe una función entera $f$ cumpliendo las condiciones dadas.
    Definimos las funciones $u,v:\bb{R}^2\to\bb{R}$ dadas por:
    \begin{align*}
        u(x,y) &= \Re f(x+iy) = x^4 - 6x^2y^2 + y^4,&\forall (x,y)\in\bb{R}^2,\\
        v(x,y) &= \Im f(x+iy),&\forall (x,y)\in\bb{R}^2.
    \end{align*}

    Por ser una función entera, $f$ es derivable en todo $\bb{C}$, por lo que $u,v$ son diferenciables en $\bb{R}^2$ y, además, se verifican las ecuaciones de Cauchy-Riemann en todo $\bb{R}^2$. Esto es:
    \begin{align*}
        \frac{\partial u}{\partial x}(x,y) &= \frac{\partial v}{\partial y}(x,y),\\
        \frac{\partial u}{\partial y}(x,y) &= -\frac{\partial v}{\partial x}(x,y).
    \end{align*}

    Sustituyendo los valores de las derivadas parciales de $u$, las ecuaciones de Cauchy-Riemann quedan:
    \begin{align*}
        \left\{
            \begin{array}{rcl}
                \dfrac{\partial v}{\partial y}(x,y) &=& 4x^3-12xy^2,\\ \\
                \dfrac{\partial v}{\partial x}(x,y) &=& 12x^2y-4y^3.
            \end{array}
        \right\}
    \end{align*}

    Integrando con respecto a $y$ la primera ecuación, tenemos que:
    \begin{align*}
        v(x,y) &= 4x^3y -4xy^3 + \varphi(x)\qquad \forall (x,y)\in\bb{R}^2,
    \end{align*}
    donde $\varphi:\bb{R}\to\bb{R}$ es una función derivable que depende solo de $x$ y representa la constante de integración. Derivando con respecto a $x$ la expresión anterior, obtenemos:
    \begin{align*}
        \dfrac{\partial v}{\partial x}(x,y) &= 12x^2y - 4y^3 + \varphi'(x)
    \end{align*}
    Por tanto, como también tenemos las ecuaciones de Cauchy-Riemann, deducimos que $\varphi'(x)=0$ para todo $x\in\bb{R}$. Por tanto, $\varphi(x)=C\in\bb{R}$ y, por tanto:
    \begin{align*}
        v(x,y) &= 4x^3y -4xy^3 + C\qquad \forall (x,y)\in\bb{R}^2.
    \end{align*}

    Por tanto, la función $f$ es de la forma:
    \[
        f(x+iy) = u(x,y) + iv(x,y) = x^4 - 6x^2y^2 + y^4 + i(4x^3y -4xy^3 + C)\qquad \forall (x,y)\in\bb{R}^2,\qquad C\in \bb{R}
    \]

    Si imponemos la condición adicional $f(0)=0$, tenemos que:
    \[
        f(0) = 0 = 0+ Ci \iff C=0.
    \]

    Por tanto, la función $f$ es única y viene dada por:
    \[
        f(x+iy) = x^4 - 6x^2y^2 + y^4 + i(4x^3y -4xy^3)\qquad \forall (x,y)\in\bb{R}^2.
    \]
\end{ejercicio}

\begin{ejercicio}
    Encontrar la condición necesaria y suficiente que deben cumplir $a, b, c \in \mathbb{R}$ para que exista una función entera $f$ tal que:
    \[
        \Re f(x + iy) = ax^2 + bxy + cy^2 \quad \text{para todo } x, y \in \mathbb{R}.
    \]

    Supongamos que existe una función entera $f$ cumpliendo las condiciones dadas.
    Definimos las funciones $u,v:\bb{R}^2\to\bb{R}$ dadas por:
    \begin{align*}
        u(x,y) &= \Re f(x+iy) = ax^2 + bxy + cy^2,&\forall (x,y)\in\bb{R}^2,\\
        v(x,y) &= \Im f(x+iy),&\forall (x,y)\in\bb{R}^2.
    \end{align*}

    Por ser una función entera, $f$ es derivable en todo $\bb{C}$, por lo que $u,v$ son diferenciables en $\bb{R}^2$ y, además, se verifican las ecuaciones de Cauchy-Riemann en todo $\bb{R}^2$. Esto es:
    \begin{align*}
        \frac{\partial u}{\partial x}(x,y) &= \frac{\partial v}{\partial y}(x,y),\\
        \frac{\partial u}{\partial y}(x,y) &= -\frac{\partial v}{\partial x}(x,y).
    \end{align*}
    
    Sustituyendo los valores de las derivadas parciales de $u$, las ecuaciones de Cauchy-Riemann quedan:
    \begin{align*}
        \left\{
            \begin{array}{rcl}
                \dfrac{\partial v}{\partial y}(x,y) &=& 2ax + by,\\ \\
                \dfrac{\partial v}{\partial x}(x,y) &=& -bx - 2cy.
            \end{array}
        \right\}
    \end{align*}

    Integrando con respecto a $y$ la primera ecuación, tenemos que:
    \begin{align*}
        v(x,y) &= 2ax y + b\cdot \dfrac{y^2}{2} + \varphi(x)\qquad \forall (x,y)\in\bb{R}^2,
    \end{align*}
    donde $\varphi:\bb{R}\to\bb{R}$ es una función derivable que depende solo de $x$ y representa la constante de integración. Derivando con respecto a $x$ la expresión anterior, obtenemos:
    \begin{align*}
        \dfrac{\partial v}{\partial x}(x,y) &= 2ay + \varphi'(x)
    \end{align*}

    Por tanto, como también tenemos las ecuaciones de Cauchy-Riemann, tenemos que:
    \begin{align*}
        2ay + \varphi'(x) &= -bx - 2cy\\
        \varphi'(x) &= -bx - 2y(a+c)\qquad \forall (x,y)\in\bb{R}^2.
    \end{align*}

    Como $\varphi$ tan solo depende de $x$, la ecuación anterior se cumplirá si y solo si $a+c=0$; en cuyo caso:
    \[
        \varphi(x) = -\dfrac{bx^2}{2} + C\qquad \forall x\in\bb{R},\qquad C\in\bb{R}.
    \]

    Por tanto, la función $f$ será de la forma:
    \begin{align*}
        f(x+iy) &= u(x,y) + iv(x,y) = ax^2 + bxy + cy^2 + i\left(2axy + C\right)\\
        &= a(x^2-y^2) + bxy + i\left(2axy + C\right)\qquad \forall (x,y)\in\bb{R}^2,\quad C\in\bb{R}.
    \end{align*}~\\

    Por tanto, y a modo de resumen, tenemos que:
    \[
        \exists f \in \cc{H}(\bb{C}) \ \text{tal que }\ \Re f(x + iy) = ax^2 + bxy + cy^2 \ \forall x, y \in \mathbb{R} \iff a+c=0.
    \]
    \begin{description}
        \item[$\Rightarrow$)] Si $f$ cumple las condiciones dadas, hemos probado anteriormente que $a+c=0$.
        \item[$\Leftarrow$)] Si $a+c=0$, La función $f$ descrita anteriormente cumple las condiciones dadas.
    \end{description}

\end{ejercicio}

\begin{ejercicio}
    Sea $\Omega$ un dominio y $f \in \cc{H}(\Omega)$. Supongamos que existen $a, b, c \in \mathbb{R}$ con $a^2 + b^2 > 0$, tales que:
    \[
        a\Re f(z) + b\Im f(z) = c \quad \text{para todo } z \in \Omega.
    \]
    Probar que $f$ es constante.
\end{ejercicio}

\begin{ejercicio}
    Sea $\Omega$ un dominio y $f \in \cc{H}(\Omega)$. Probar que si $\ol{f} \in \cc{H}(\Omega)$, entonces $f$ es constante.
\end{ejercicio}

\begin{ejercicio}
    Sea $\Omega$ un dominio y $f \in \cc{H}(\Omega)$. Sea $\Omega^* = \{\ol{z} \mid z \in \Omega\}$ y $f^* : \Omega^* \to \mathbb{C}$ la función definida por:
    \[
        f^*(z) = \ol{f\left(\ol{z}\right)} \quad \text{para todo } z \in \Omega^*.
    \]
    Probar que $f^* \in \cc{H}(\Omega^*)$.
\end{ejercicio}

\begin{ejercicio}
    Probar que la restricción de la función exponencial a un subconjunto abierto no vacío del plano, nunca es una función racional.
\end{ejercicio}