\section{Teorema local de Cauchy}

\begin{ejercicio}
    Sean $a \in \bb{C}$ y $r \in \bb{R}^+$. Probar que, para cada $z \in \bb{C}$ con $|z-a| > r$, se tiene:
    \[
        \int_{C(a,r)} \frac{dw}{w-z} = 0
    \]
    \begin{description}
        \item[Opción 1] Como se trata de una integral sobre un camino cerrado igualada a $0$, podríamos buscar una primitiva del integrando holomorfa en un abierto que contenga a $C(a,r)^*$. Sea dicho abierto el conjunto $\Omega=D(a,|z-a|)$, de forma que $C(a,r)^* \subset \Omega$, y veamos que $z-w$ admite un argumento continuo en $\Omega$. Veamos en primer lugar que $z-w$ no se anula en $\Omega$. Para ello, supongamos que $w \in \Omega$, entonces:
        \begin{equation*}
            |w-a| < |z-a|=|z-w+w-a| \leq |z-w|+|w-a| \Longrightarrow 0<|z-w|\Longrightarrow z-w\neq 0.
        \end{equation*}

        Por tanto, podemos considerar un argumento de $z-w$ en $\Omega$. Veamos que $z-w$ admite un argumento continuo en $\Omega$. Para ello, y en vistas de usar el Ehercicio~\ref{ej:2.2}, veamos que o bien nunca toma valores en $\bb{R}^+$, o bien nunca toma valores en $\bb{R}^-$. Supongamos que toma valores en ambos conjuntos; es decir, supongamos que $\exists w,w'\in \Omega$ tal que $w-z\in \bb{R}^+$ y $w'-z\in \bb{R}^-$. Entonces:
        \begin{align*}
            w-z\in \bb{R}^+ & \Longrightarrow \left\{\begin{array}{l}
                \Re w > \Re z\\
                \Im w = \Im z
            \end{array}\right.\\
            w'-z\in \bb{R}^- & \Longrightarrow \left\{\begin{array}{l}
                \Re w' < \Re z\\
                \Im w' = \Im z
            \end{array}\right.
        \end{align*}

        Por tanto, se tiene que:
        \begin{align*}
            \Re w' < \Re z < \Re w\\
            \Im w' = \Im z = \Im w
        \end{align*}

        Llegados a este punto, uno ya puede ver gráficamente que hemos llegado a una contradicción, puesto que $w$ y $w'$ no pueden estar ammbos en $D(a,|z-a|)$, pero veámoslo formalmente.
        Como $w\in \Omega$, se tiene que:
        \begin{align*}
            |w-a| < |z-a|\Longrightarrow \left(\Re w-\Re a\right)^2+\left(\Im w-\Im a\right)^2 < \left(\Re z-\Re a\right)^2+\left(\Im z-\Im a\right)^2
        \end{align*}

        Como $\Im w=\Im z$, se tiene que:
        \begin{multline*}
            \left(\Re w-\Re a\right)^2<\left(\Re z-\Re a\right)^2
            \Longrightarrow |\Re w-\Re a|<|\Re z-\Re a|\Longrightarrow \\ \Longrightarrow
            \Re a - |\Re z -\Re a| < \Re w < \Re a + |\Re z - \Re a|
        \end{multline*}

        Realizamos el procedimiento análogo para $w'$, llegando por tanto a que:
        \begin{align*}
            \Re a - |\Re z -\Re a| < \Re w,\Re w' < \Re a + |\Re z - \Re a|
        \end{align*}
        \begin{itemize}
            \item Si $|\Re z -\Re a| \geq 0$, se tiene que $\Re w<\Re z$.
            \item Si $|\Re z -\Re a| < 0$, se tiene que $\Re z<\Re w'$.
        \end{itemize}

        En cualquier caso, se llega a una contradicción, por lo que $z-w$ no puede tomar valores en ambos conjuntos. Por tanto, $z-w$ admite un argumento continuo en $\Omega$, por lo que se puede considerar un logaritmo continuo en $\Omega$ de $z-w$ (función holomorfa en $\Omega$). Por tanto, existe un logaritmo holomorfo de $z-w$ en $\Omega$ (llamémosle $F:\Omega \to \bb{C}$), de forma que:
        \begin{equation*}
            F'(w) = \frac{1}{w-z} \qquad \forall w \in \Omega.
        \end{equation*}

        Como $F\in \cc{H}(\Omega)$ es una primitiva del integrando (siendo este una función continua en $\Omega$ por no anularse el denominador) y $C(a,r)$ es un camino cerrado contenido en $\Omega$, se tiene que:
        \begin{equation*}
            \int_{C(a,r)} \frac{dw}{w-z} = 0
        \end{equation*}

        \item[Opción 2] Como $z\neq a$, tenemos que:
        \begin{equation*}
            \dfrac{1}{w-z} = \dfrac{1}{w-a+a-z}\cdot \dfrac{a-z}{a-z}
            = \dfrac{1}{a-z}\cdot \dfrac{1}{1-\frac{w-a}{a-z}}\qquad \forall w\neq z
        \end{equation*}

        % // TODO: Cont
    \end{description}
\end{ejercicio}

\begin{ejercicio}[Versión más general de la fórmula de Cauchy]
    Sean $a \in \bb{C}$, $R \in \bb{R}^+$ y $f : \ol{D}(a,R) \to \bb{C}$ una función continua en $\ol{D}(a,R)$ y holomorfa en $D(a,R)$. Se tiene entonces:
    \[
        f(z) = \frac{1}{2\pi i} \int_{C(a,R)} \frac{f(w)}{w-z}dw \qquad \forall z \in D(a,R).
    \]
\end{ejercicio}

\begin{ejercicio}
    Dados $a \in \bb{C}$, $r \in \bb{R}^+$ y $b,c \in \bb{C}\setminus C(a,r)^\ast$, calcular todos los posibles valores de la integral
    \[
        \int_{C(a,r)} \frac{dz}{(z-b)(z-c)}
    \]
    dependiendo de la posición relativa de $b,c$ respecto de la circunferencia $C(a,r)^\ast$.
\end{ejercicio}

\begin{ejercicio}
    Calcular las siguientes integrales:
    \begin{enumerate}
        \item $\displaystyle\int_{C(0,r)} \frac{z+1}{z(z^2+4)}\ dz$ \qquad ($r \in \bb{R}^+$, $r \neq 2$)
        \item $\displaystyle\int_{C(0,1)} \frac{\cos z}{(a^2+1)z - a(z^2+1)}\ dz$ \qquad ($a \in \bb{C}$, $|a| \neq 1$)
    \end{enumerate}
\end{ejercicio}

\begin{ejercicio}
    Dados $a,b \in \bb{C}$ con $a \neq b$, sea $R \in \bb{R}^+$ tal que $R > \max\{|a|,|b|\}$. Probar que, si $f$ es una función entera, se tiene:
    \[
        \int_{C(0,R)} \frac{f(z)}{(z-a)(z-b)}dz = 2\pi i \cdot \frac{f(b)-f(a)}{b-a}.
    \]
    Deducir que toda función entera y acotada es constante.
\end{ejercicio}