\section{Residuos}

\begin{ejercicio}
    Probar que, para $a \in \left]0,1\right[$, se tiene:
    \begin{equation*}
        \int_0^{2\pi} \frac{\cos^2(3t)}{1 + a^2 - 2a \cos(2t)} \, dt = \pi\cdot \frac{a^2 - a + 1}{1 - a}.
    \end{equation*}
\end{ejercicio}

\begin{ejercicio}
    Probar que, para $n \in \bb{N}$, se tiene:
    \begin{equation*}
        \int_0^{2\pi} \frac{(1 + 2\cos t)^n \cos(nt)}{3 + 2\cos t} \, dt = \dfrac{2\pi}{\sqrt{5}} \left(3 - \sqrt{5}\right)^n.
    \end{equation*}
\end{ejercicio}

\begin{ejercicio}
    Probar que, para $n \in \bb{N}$, se tiene:
    \begin{equation*}
        \int_0^{2\pi} e^{\cos t} \cos\left(n t - \sin t\right) \, dt = \frac{2\pi}{n!}.
    \end{equation*}
\end{ejercicio}

\begin{ejercicio}
    Probar que, para cualesquiera $a, b \in \bb{R}^+$, se tiene:
    \begin{equation*}
        \int_{-\infty}^{+\infty} \frac{dx}{(x^2 + a^2)(x^2 + b^2)^2} = \pi\cdot \frac{(a + 2b)^2}{2ab^3(a + b)^2}.
    \end{equation*}
\end{ejercicio}

\begin{ejercicio}
    Probar que, para $a \in \bb{R}^+$, se tiene:
    \begin{equation*}
        \int_{-\infty}^{+\infty} \frac{x^6}{(x^4 + a^4)^2} \, dx = \dfrac{3\pi\sqrt{2}}{8a}.
    \end{equation*}
\end{ejercicio}

\begin{ejercicio}
    Sea $n \in \bb{N}$ con $n > 2$, integrar una conveniente función sobre un camino cerrado que recorra la frontera del sector $D(0,R) \cap \{z \in \bb{C}^* : 0 < \arg z < \nicefrac{2\pi}{n}\}$ con $R \in \bb{R}^+$, para probar que:
    \begin{equation*}
        \int_0^{+\infty} \frac{dx}{1 + x^n} = \frac{\pi}{n}\csc\left(\frac{\pi}{n}\right).
    \end{equation*}
\end{ejercicio}

\begin{ejercicio}
    Probar que, para $a, t \in \bb{R}^+$, se tiene:
    \begin{equation*}
        \int_{-\infty}^{+\infty} \frac{\cos(tx)}{(x^2 + a^2)^2} \, dx = \frac{\pi}{2a^3}(1 + a t)e^{-a t}.
    \end{equation*}
\end{ejercicio}

\begin{ejercicio}
    Probar que:
    \begin{equation*}
        \int_{-\infty}^{+\infty} \frac{x\sen(\pi x)}{x^2-5x+6} \, dx = -5\pi.
    \end{equation*}
    \begin{observacion}
        Para resolver este ejercicio, se puede integrar la función $\frac{ze^{i\pi z}}{z^2-5z+6}$ en un semicírculo retocado (que rodee los polos de la función).
    \end{observacion}
\end{ejercicio}

\begin{ejercicio}
    Integrando la función $z \mapsto \frac{1-e^{2i z}}{z^2}$ sobre un camino cerrado que recorra la frontera de la mitad superior del anillo $A(0; \varepsilon, R)$, probar que:
    \begin{equation*}
        \int_0^{+\infty} \frac{\sen^2(x)}{x^2} \, dx = \frac{\pi}{2}.
    \end{equation*}
\end{ejercicio}

\begin{ejercicio}
    Dado $a \in \bb{R}$ con $a > 1$, integrar la función $z \mapsto \frac{z}{a - e^{-i z}}$ sobre la poligonal $\left[-\pi, \pi, \pi + i n, -\pi + i n, -\pi\right]$, con $n \in \bb{N}$, para probar que:
    \begin{equation*}
        \int_{-\pi}^{\pi} \frac{x\sen(x)}{1 + a^2 - 2a\cos(x)} \, dx = \frac{2\pi}{a}\log\left(\frac{1 + a}{a}\right).
    \end{equation*}
\end{ejercicio}

\begin{ejercicio}
    Integrando una conveniente función compleja a lo largo de la frontera de la mitad superior del anillo $A(0; \varepsilon, R)$, probar que, para $\alpha \in \left]-1, 3\right[$, se tiene:
    \begin{equation*}
        \int_0^{+\infty} \frac{x^\alpha}{(1 + x^2)^2} \, dx = \frac{\pi}{4}(1 - \alpha)\sec\left(\frac{\pi \alpha}{2}\right).
    \end{equation*}
    \begin{observacion}
        Para resolver este ejercicio, se puede hacer el cambio de variable $x = e^t$ y luego integrar $\frac{e^{t(\alpha + 1)}}{(1 + e^{2t})^2}$ en un rectángulo.
    \end{observacion}
\end{ejercicio}

\begin{ejercicio}
    Probar que, para $\alpha \in \left]0, 2\right[$, se tiene:
    \begin{equation*}
        \int_{-\infty}^{+\infty} \frac{e^{\alpha x}}{1 + e^x + e^{2x}} \, dx = \int_0^{+\infty} \frac{t^{\alpha - 1}}{1 + t + t^2} \, dt = \dfrac{2\pi}{\sqrt{3}}\cdot \dfrac{\sen\left(\frac{\pi(1 - \alpha)}{3}\right)}{\sen\left(\pi \alpha\right)}.
    \end{equation*}
    \begin{observacion}
        Para resolver este ejercicio, se puede integrar $\frac{e^x \alpha}{1 + e^x + e^{2x}}$ en un rectángulo.
    \end{observacion}
\end{ejercicio}

\begin{ejercicio}
    Integrando la función $z \mapsto \frac{\log(z + i)}{1 + z^2}$ sobre un camino cerrado que recorra la frontera del conjunto $\{z \in \bb{C} : |z| < R, \Im z > 0\}$, con $R \in \bb{R}$ y $R > 1$, calcular la integral:
    \begin{equation*}
        \int_{-\infty}^{+\infty} \frac{\log(1 + x^2)}{1 + x^2} \, dx.
    \end{equation*}
\end{ejercicio}

\begin{ejercicio}
    Integrando una conveniente función sobre la poligonal $[-R, R, R + \pi i, -R + \pi i, -R]$, con $R \in \bb{R}^+$, calcular la integral:
    \begin{equation*}
        \int_{-\infty}^{+\infty} \frac{\cos(x)}{e^x + e^{-x}} \, dx.
    \end{equation*}
\end{ejercicio}

\begin{ejercicio}
    Integrando una conveniente función sobre un camino cerrado que recorra la frontera del conjunto $\{z \in \bb{C} : \varepsilon < |z| < R, 0 < \arg z < \nicefrac{\pi}{2}\}$, con $0 < \varepsilon < 1 < R$, calcular la integral:
    \begin{equation*}
        \int_0^{+\infty} \frac{\log(x)}{1 + x^4} \, dx.
    \end{equation*}
\end{ejercicio}

\begin{ejercicio}
    Integrando una conveniente función sobre la poligonal $[-R, R, R + 2\pi i, -R + 2\pi i, -R]$, con $R \in \bb{R}^+$, calcular la integral:
    \begin{equation*}
        \int_{-\infty}^{+\infty} \frac{e^{\nicefrac{x}{2}}}{e^x + 1} \, dx.
    \end{equation*}
\end{ejercicio}