\section{Residuos}

\begin{ejercicio}
    Probar que, para $a \in \left]0,1\right[$, se tiene:
    \begin{equation*}
        \int_0^{2\pi} \frac{\cos^2(3t)}{1 + a^2 - 2a \cos(2t)} \, dt = \pi\cdot \frac{a^2 - a + 1}{1 - a}.
    \end{equation*}
\end{ejercicio}

\begin{ejercicio}
    Probar que, para $n \in \bb{N}$, se tiene:
    \begin{equation*}
        \int_0^{2\pi} \frac{(1 + 2\cos t)^n \cos(nt)}{3 + 2\cos t} \, dt = \dfrac{2\pi}{\sqrt{5}} \left(3 - \sqrt{5}\right)^n.
    \end{equation*}
\end{ejercicio}

\begin{ejercicio}
    Probar que, para $n \in \bb{N}$, se tiene:
    \begin{equation*}
        \int_0^{2\pi} e^{\cos t} \cos\left(n t - \sin t\right) \, dt = \frac{2\pi}{n!}.
    \end{equation*}
\end{ejercicio}

\begin{ejercicio}\label{ej:14.4}
    Probar que, para cualesquiera $a, b \in \bb{R}^+$, se tiene:
    \begin{equation*}
        \int_{-\infty}^{+\infty} \frac{dx}{(x^2 + a^2)(x^2 + b^2)^2} = \pi\cdot \frac{(a + 2b)}{2ab^3(a + b)^2}.
    \end{equation*}

    Calculamos las raíces del denominador:
    \begin{align*}
        x^2 + a^2 &= 0 \implies x\in \{ai, -ai\}, \\
        (x^2 + b^2)^2 &= 0 \implies x^2 + b^2 = 0 \implies x\in \{bi, -bi\}.
    \end{align*}

    Por tanto, definimos $A=\{ai, -ai, bi, -bi\}$, y sea la siguiente función:
    \Func{f}{\bb{C}\setminus A}{\bb{C}}{z}{\frac{1}{(z^2 + a^2)(z^2 + b^2)^2}}
    
    Tenemos que $A'=\emptyset$, y $f\in \cc{H}(\bb{C}\setminus A)$, por lo que podemos aplicar el teorema de los residuos. Como $\bb{C}$ es homológicamente conexo puesto que $\bb{C}\setminus \bb{C}=\emptyset$, podemos aplicar el Teormea de los Residuos para cualquier ciclo $\Sigma$ en $\bb{C}\setminus A$. 
    
    Para cada $r>\max\{a,b\}$, consideramos el ciclo $\Sigma_r = \gamma_r + \sigma_r$, representado en la Figura~\ref{fig:ej:14.4}, donde:
    \Func{\gamma_r}{[-r, r]}{\bb{C}}{t}{t}
    \Func{\sigma_r}{[0, \pi]}{\bb{C}}{t}{re^{it}}
    \begin{figure}
        \centering
        \begin{tikzpicture}
            \begin{axis}[
                axis lines=middle,
                xlabel={$x$},
                ylabel={$y$},
                xtick=\empty,
                ytick=\empty,
                xmin=-3.5, xmax=3.5,
                ymin=-2, ymax=2,
                axis equal,
            ]
                \def\R{2}
                \def\A{1.5}
                \def\B{0.5}

                % Polos
                \draw[fill=red] (0, \A) circle (2pt) node[right] {$ai$};
                \draw[fill=red] (0, -\A) circle (2pt) node[right] {$-ai$};
                \draw[fill=teal] (0, \B) circle (2pt) node[right] {$bi$};
                \draw[fill=teal] (0, -\B) circle (2pt) node[right] {$-bi$};

                
                % sigma_r
                \draw[thick, blue, arrow at 1/3, arrow at 2/3] (\R, 0) arc[start angle=0, end angle=180, radius=\R]
                    node[midway, below left, yshift=-1em, xshift=-1em] {$\sigma_r$};

                % gamma_r
                \draw[thick, blue, arrow at 1/3, arrow at 2/3] (-\R, 0) -- (\R, 0)
                    node[pos=0.25, below] {$\gamma_r$};

                % Puntos de unión
                \draw[fill=blue] (-\R, 0) circle (2pt);
                \draw[fill=blue] (\R, 0) circle (2pt);

            \end{axis}
        \end{tikzpicture}
        \caption{Ciclo de integración $\Sigma_r$ del Ejercicio~\ref{ej:14.4}.}
        \label{fig:ej:14.4}
    \end{figure}

    De esta forma, tenemos que:
    \begin{align*}
        \int_{\Sigma_r} f(z) \, dz &= \int_{\gamma_r} f(z) \, dz + \int_{\sigma_r} f(z) \, dz \\
        &= 2\pi i\sum_{z_0\in A}\Res(f,z_0)\Ind_{\Sigma_r}(z_0)
    \end{align*}

    Calculemos las dos integrales que nos han resultado:
    \begin{align*}
        \int_{\gamma_r} f(z) \, dz &= \int_{-r}^{r} \frac{1}{(t^2 + a^2)(t^2 + b^2)^2} \, dt, \\
        \int_{\sigma_r} f(z) \, dz &= \int_0^{\pi} \frac{1}{(r^2 e^{2it} + a^2)(r^2 e^{2it} + b^2)^2} i r e^{it} \, dt.
    \end{align*}

    Notemos que la primera integral es la integral pedida, por lo que buscamos que la integral resultante de la segunda integral sea cero.
    \begin{align*}
        \left| \int_0^{\pi} \frac{1}{(r^2 e^{2it} + a^2)(r^2 e^{2it} + b^2)^2} i r e^{it} \, dt \right|
        \leq \pi\cdot \frac{r}{(r^2 - a^2)(r^2 - b^2)^2}
    \end{align*}
    donde hemos usado que $|e^{it}|=1$ y que, como $r>\max\{a,b\}>0$, tenemos que $r^2>\max\{a^2,b^2\}$, por lo que:
    \begin{align*}
        \left| r^2 e^{2it} + a^2 \right| &\geq \left||r^2 e^{2it}| - |a^2|\right| = \left| r^2 - a^2 \right| = r^2 - a^2
    \end{align*}

    Como la expresión anterior es válida para cualquier $r > \max\{a,b\}$, podemos hacer $r \to +\infty$ y tenemos que:
    \begin{align*}
        \lim_{r\to+\infty} \int_{\gamma_r} f(z) \, dz &= \int_{-\infty}^{+\infty} \frac{1}{(x^2 + a^2)(x^2 + b^2)^2} \, dx, \\
        \lim_{r\to+\infty} \int_{\sigma_r} f(z) \, dz &= 0.
    \end{align*}

    Calculamos en primer lugar los índices. Por la forma en la que se ha definido el ciclo $\Sigma_r$, para todo $r>\max\{a,b\}$, tenemos que:
    \begin{align*}
        \Ind_{\Sigma_r}(ai) &= \Ind_{\Sigma_r}(bi) = 1\\
        \Ind_{\Sigma_r}(-ai) &= \Ind_{\Sigma_r}(-bi) = 0.
    \end{align*}

    Por tanto, tan solo hemos de calcular los residuos en los polos $ai$ y $bi$. Comenzamos por calcular el residuo en $ai$:
    \begin{align*}
        \lim_{z\to ai} (z - ai)f(z) &= \lim_{z\to ai} \frac{z - ai}{(z^2 + a^2)(z^2 + b^2)^2} \\
        &= \lim_{z\to ai} \dfrac{1}{(z + ai)(z^2 + b^2)^2} \\
        &= \dfrac{1}{(2ai)(-a^2 + b^2)^2} = \dfrac{1}{2ai(b^2 - a^2)^2}.
    \end{align*}

    Por tanto, deducimos que el orden del polo $ai$ es $1$, y que el residuo es:
    \begin{align*}
        \Res(f, ai) &= \dfrac{1}{2ai(b^2 - a^2)^2}
    \end{align*}

    Ahora, calculamos el residuo en $bi$:
    \begin{align*}
        \lim_{z\to bi} (z - bi)f(z) &= \lim_{z\to bi} \frac{z - bi}{(z^2 + a^2)(z^2 + b^2)^2} \\
        &= \lim_{z\to bi} \dfrac{1}{(z^2 + a^2)(z+bi)^2(z - bi)} = +\infty \\
        \lim_{z\to bi} (z-bi)^2 f(z) &= \lim_{z\to bi} \frac{1}{(z^2 + a^2)(z+bi)^2}\in \bb{C}^*
    \end{align*}
    Por tanto, deducimos que el orden del polo $bi$ es $2$. El residuo se calcula como:
    \begin{align*}
        \Res(f, bi) &= \dfrac{1}{1!} \lim_{z\to bi} \frac{d}{dz}\left((z - bi)^2 f(z)\right)
        = \lim_{z\to bi} \frac{d}{dz}\left(\frac{1}{(z^2 + a^2)(z+bi)^2}\right)
        =\\&= \lim_{z\to bi} -\frac{2z(z+bi)^2+2(z^2+a^2)(z+bi)}{(z^2 + a^2)^2(z+bi)^4}
        =\\&= \lim_{z\to bi} -\frac{2z(z+bi)+2(z^2+a^2)}{(z^2 + a^2)^2(z+bi)^3}
        = -\dfrac{2bi(2bi) + (a^2-b^2)}{(a^2-b^2)^2(2bi)^3}
        =\\&= -\dfrac{-4b^2 + 2(a^2-b^2)}{(a^2-b^2)^2\cdot 8bi(-b^2)}
        = \dfrac{-2b^2 + a^2-b^2}{(a^2-b^2)^2\cdot 4b^3i}
        = \dfrac{a^2-3b^2}{(b^2-a^2)^2\cdot 4b^3i}
    \end{align*}

    Por tanto, deducimos que el residuo en $bi$ es:
    \begin{align*}
        \Res(f, bi) &= \dfrac{a^2-3b^2}{(b^2-a^2)^2\cdot 4b^3i}
    \end{align*}

    Por tanto, tenemos que:
    \begin{align*}
        \lim_{r\to+\infty} \int_{\Sigma_r} f(z) \, dz &= \int_{-\infty}^{+\infty} \frac{1}{(x^2 + a^2)(x^2 + b^2)^2} \, dx
        = 2\pi i\left(\Res(f, ai) + \Res(f, bi)\right)
        =\\&= 2\pi i\left(\dfrac{1}{2ai(b^2 - a^2)^2} + \dfrac{a^2-3b^2}{(b^2-a^2)^2\cdot 4b^3i}\right)
        =\\&= \dfrac{\pi}{a(b^2 - a^2)^2} + \dfrac{\pi(a^2-3b^2)}{(b^2-a^2)^2\cdot 2b^3}
        = \dfrac{\pi(2b^3 + a^3-3b^2a)}{2ab^3(b^2-a^2)^2}
        =\\&= \dfrac{\pi(b-a)^2(a+2b)}{2ab^3(b+a)^2(b-a)^2}
        = \dfrac{\pi(a+2b)}{2ab^3(b+a)^2}
    \end{align*}
    como queríamos demostrar.
\end{ejercicio}

\begin{ejercicio}\label{ej:14.5}
    Probar que, para $a \in \bb{R}^+$, se tiene:
    \begin{equation*}
        \int_{-\infty}^{+\infty} \frac{x^6}{(x^4 + a^4)^2} \, dx = \dfrac{3\pi\sqrt{2}}{8a}.
    \end{equation*}

    Calculamos las raíces del denominador:
    \begin{align*}
        (x^4 + a^4)^2 &= 0 \implies x^4 + a^4 = 0 \implies x\in A:=\left[\sqrt[4]{-1}\right]\cdot a = a\cdot \left\{e^{i\frac{\pi}{4}}, e^{i\frac{3\pi}{4}}, e^{i\frac{5\pi}{4}}, e^{i\frac{7\pi}{4}}\right\}
    \end{align*}

    En este tipo de ejercicios, buscamos que los polos del integrando tengan el menor orden posible, para que así a la hora de calcular los residuos, estos sean más sencillos. Por ello, en este caso no vamos a realizar el cálculo directamente, sino que antes vamos a aplicar esta idea. Es por ello que vamos a aplicar el método de Integración por Partes:
    \begin{equation*}
        \MetInt{u(x)=x^3\qquad u'(x)=3x^2}{v'(x)=\dfrac{x^3}{(x^4 + a^4)^2}\qquad v(x)=-\dfrac{1}{4}\cdot \dfrac{1}{x^4 + a^4}}
    \end{equation*}

    Por tanto, tenemos que:
    \begin{align*}
        \int_{-\infty}^{+\infty} \frac{x^6}{(x^4 + a^4)^2} \, dx &= -\frac{1}{4}\left[\dfrac{x^3}{x^4 + a^4}\right]_{-\infty}^{+\infty} + \frac{3}{4}\int_{-\infty}^{+\infty} \frac{x^2}{x^4 + a^4} \, dx
        = \frac{3}{4}\int_{-\infty}^{+\infty} \frac{x^2}{x^4 + a^4} \, dx.
    \end{align*}

    Por tanto, hemos reducido el orden de los polos del integrando. Ahora sí, definimos la función:
    \Func{f}{\bb{C}\setminus A}{\bb{C}}{z}{\frac{z^2}{z^4 + a^4}}
    donde $A$ es el conjunto de los cuatro polos que hemos calculado anteriormente. Notemos que $f\in \cc{H}(\bb{C}\setminus A)$, y que $A'=\emptyset$, por lo que podemos aplicar el Teorema de los Residuos. Como $\bb{C}$ es homológicamente conexo, podemos aplicar el Teorema de los Residuos para cualquier ciclo $\Sigma$ en $\bb{C}\setminus A$.

    Para cada $r>a$, consideramos el ciclo $\Sigma_r = \gamma_r + \sigma_r$, representado en la Figura~\ref{fig:ej:14.5}, donde:
    \Func{\gamma_r}{[-r, r]}{\bb{C}}{t}{t}
    \Func{\sigma_r}{[0,\pi]}{\bb{C}}{t}{re^{it}}
    \begin{figure}
        \centering
        \begin{tikzpicture}
            \begin{axis}[
                axis lines=middle,
                xlabel={$x$},
                ylabel={$y$},
                xtick=\empty,
                ytick=\empty,
                xmin=-3.5, xmax=3.5,
                ymin=-1, ymax=1,
                axis equal,
            ]
                \def\R{2}
                \def\A{1}

                % Polos
                \draw[fill=red] (45:\A) circle (2pt) node[right] {$a e^{i\frac{\pi}{4}}$};
                \draw[fill=red] (135:\A) circle (2pt) node[left] {$a e^{i\frac{3\pi}{4}}$};
                \draw[fill=red] (225:\A) circle (2pt) node[below] {$a e^{i\frac{5\pi}{4}}$};
                \draw[fill=red] (315:\A) circle (2pt) node[below] {$a e^{i\frac{7\pi}{4}}$};


                % sigma_r
                \draw[thick, blue, arrow at 1/3, arrow at 2/3] (\R, 0) arc[start angle=0, end angle=180, radius=\R]
                    node[midway, below left, yshift=-1em, xshift=-1em] {$\sigma_r$};

                % gamma_r
                \draw[thick, blue, arrow at 1/3, arrow at 2/3] (-\R, 0) -- (\R, 0)
                    node[pos=0.25, below] {$\gamma_r$};

                % Puntos de unión
                \draw[fill=blue] (-\R, 0) circle (2pt);
                \draw[fill=blue] (\R, 0) circle (2pt);

            \end{axis}
        \end{tikzpicture}
        \caption{Ciclo de integración $\Sigma_r$ del Ejercicio~\ref{ej:14.5}.}
        \label{fig:ej:14.5}
    \end{figure}

    De esta forma, tenemos que:
    \begin{align*}
        \int_{\Sigma_r} f(z) \, dz &= \int_{\gamma_r} f(z) \, dz + \int_{\sigma_r} f(z) \, dz \\
        &= 2\pi i\sum_{z_0\in A}\Res(f,z_0)\Ind_{\Sigma_r}(z_0)
    \end{align*}

    Calculemos la primera integral que nos ha resultado:
    \begin{align*}
        \int_{\gamma_r} f(z) \, dz &= \int_{-r}^{r} \frac{t^2}{t^4 + a^4} \, dt
    \end{align*}

    Notemos que es la integral pedida, por lo que buscamos que la integral resultante de la segunda integral sea cero.
    \begin{align*}
        \left| \int_{\sigma_r} f(z) \, dz \right|
        &\leq \pi r\cdot \sup\left\{\left|\dfrac{z^2}{z^4 + a^4}\right| : z\in \sigma_r^*\right\}
        \leq \pi r\cdot \frac{r^2}{r^4 - a^4}
        = \pi \cdot \frac{r^3}{r^4 - a^4}
    \end{align*}
    donde hemos usado que, si $z\in \sigma_r^*$, entonces $|z|=r$ y, como $r>a>0$, tenemos que $r^4>a^4$, por lo que:
    \begin{align*}
        |z^4 + a^4| &\geq \left||z^4| - |a^4|\right| = \left|r^4 - a^4\right| = r^4 - a^4.
    \end{align*}

    Como la expresión anterior es válida para cualquier $r > a$, podemos hacer $r \to +\infty$ y tenemos que:
    \begin{align*}
        \lim_{r\to+\infty} \int_{\gamma_r} f(z) \, dz &= \int_{-\infty}^{+\infty} \frac{x^2}{x^4 + a^4} \, dx, \\
        \lim_{r\to+\infty} \int_{\sigma_r} f(z) \, dz &= 0.
    \end{align*}

    Calculamos en primer lugar los índices. Por la forma en la que se ha definido el ciclo $\Sigma_r$, para todo $r>a$, tenemos que:
    \begin{align*}
        \Ind_{\Sigma_r}(a e^{i\frac{\pi}{4}}) &= \Ind_{\Sigma_r}(a e^{i\frac{3\pi}{4}}) = 1\\
        \Ind_{\Sigma_r}(a e^{i\frac{5\pi}{4}}) &= \Ind_{\Sigma_r}(a e^{i\frac{7\pi}{4}}) = 0.
    \end{align*}

    Por tanto, tan solo hemos de calcular los residuos en los polos $a e^{i\frac{\pi}{4}}$ y $a e^{i\frac{3\pi}{4}}$. Comenzamos por calcular el residuo en $a e^{i\frac{\pi}{4}}$:
    \begin{align*}
        \lim_{z\to a e^{i\frac{\pi}{4}}} (z - a e^{i\frac{\pi}{4}})f(z) &= \lim_{z\to a e^{i\frac{\pi}{4}}} \frac{z^2}{(z-a e^{i\frac{3\pi}{4}})(z-a e^{i\frac{5\pi}{4}})(z-a e^{i\frac{7\pi}{4}})}
        =\\&= \dfrac{a^2e^{i\frac{\pi}{2}}}{(a e^{i\frac{\pi}{4}} - a e^{i\frac{3\pi}{4}})(a e^{i\frac{\pi}{4}} - a e^{i\frac{5\pi}{4}})(a e^{i\frac{\pi}{4}} - a e^{i\frac{7\pi}{4}})}
        =\\&= \dfrac{i}{a\left(e^{i\frac{\pi}{4}} - e^{i\frac{3\pi}{4}}\right)\left(e^{i\frac{\pi}{4}} - e^{i\frac{5\pi}{4}}\right)\left(e^{i\frac{\pi}{4}} - e^{i\frac{7\pi}{4}}\right)}
        =\\&= \dfrac{i}{a\left(\sqrt{2}\right)\left(\sqrt{2} + \sqrt{2}i\right)\left(\sqrt{2}i\right)}
        = \dfrac{1}{2\sqrt{2}a\left(1 + i\right)}\in \bb{C}^*
    \end{align*}

    Por tanto, deducimos que el orden del polo $a e^{i\frac{\pi}{4}}$ es $1$, y que el residuo es:
    \begin{align*}
        \Res(f, a e^{i\frac{\pi}{4}}) &= \dfrac{1}{2\sqrt{2}a(1 + i)}
        = \dfrac{1-i}{4\sqrt{2}a}
    \end{align*}

    Ahora, calculamos el residuo en $a e^{i\frac{3\pi}{4}}$:
    \begin{align*}
        \lim_{z\to a e^{i\frac{3\pi}{4}}} (z - a e^{i\frac{3\pi}{4}})f(z) &= \lim_{z\to a e^{i\frac{3\pi}{4}}} \frac{z^2}{(z-a e^{i\frac{\pi}{4}})(z-a e^{i\frac{5\pi}{4}})(z-a e^{i\frac{7\pi}{4}})}
        =\\&= \dfrac{a^2e^{i\frac{3\pi}{2}}}{(a e^{i\frac{3\pi}{4}} - a e^{i\frac{\pi}{4}})(a e^{i\frac{3\pi}{4}} - a e^{i\frac{5\pi}{4}})(a e^{i\frac{3\pi}{4}} - a e^{i\frac{7\pi}{4}})}
        =\\&= \dfrac{-ia^2}{a^3\left(e^{i\frac{3\pi}{4}} - e^{i\frac{\pi}{4}}\right)\left(e^{i\frac{3\pi}{4}} - e^{i\frac{5\pi}{4}}\right)\left(e^{i\frac{3\pi}{4}} - e^{i\frac{7\pi}{4}}\right)}
        =\\&= \dfrac{-i}{a\left(-\sqrt{2}\right)\left(\sqrt{2}i\right)\left(-\sqrt{2} + \sqrt{2}i\right)}
        = \dfrac{1}{2\sqrt{2}a\left(-1 + i\right)}\in \bb{C}^*
    \end{align*}

    Por tanto, deducimos que el orden del polo $a e^{i\frac{3\pi}{4}}$ es $1$, y que el residuo es:
    \begin{align*}
        \Res(f, a e^{i\frac{3\pi}{4}}) &= \dfrac{1}{2\sqrt{2}a(-1 + i)}
        = -\dfrac{1+i}{4\sqrt{2}a}
    \end{align*}

    Por tanto, tenemos que:
    \begin{align*}
        \lim_{r\to+\infty} \int_{\Sigma_r} f(z) \, dz &= \int_{-\infty}^{+\infty} \frac{x^2}{x^4 + a^4} \, dx
        = 2\pi i\left(\Res(f, a e^{i\frac{\pi}{4}}) + \Res(f, a e^{i\frac{3\pi}{4}})\right)
        =\\&= 2\pi i\left(\dfrac{1-i}{4\sqrt{2}a} - \dfrac{1+i}{4\sqrt{2}a}\right)
        = 2\pi i\left(\dfrac{-2i}{4\sqrt{2}a}\right)
        = \dfrac{\pi}{\sqrt{2}a} = \dfrac{\pi\sqrt{2}}{2a}.
    \end{align*}

    Por tanto, tenemos que:
    \begin{align*}
        \int_{-\infty}^{+\infty} \frac{x^6}{(x^4 + a^4)^2} \, dx &= \frac{3}{4}\int_{-\infty}^{+\infty} \frac{x^2}{x^4 + a^4} \, dx
        = \frac{3}{4}\cdot \dfrac{\pi\sqrt{2}}{2a}
        = \dfrac{3\pi\sqrt{2}}{8a},
    \end{align*}
    como queríamos demostrar.
\end{ejercicio}

\begin{ejercicio}\label{ej:14.6}
    Sea $n \in \bb{N}$ con $n > 2$, integrar una conveniente función sobre un camino cerrado que recorra la frontera del siguiente sector circular $$D(0,R) \cap \{z \in \bb{C}^* : 0 < \arg z < \nicefrac{2\pi}{n}\}$$
    con $R \in \bb{R}^+$, para probar que:
    \begin{equation*}
        \int_0^{+\infty} \frac{dx}{1 + x^n} = \frac{\pi}{n}\csc\left(\frac{\pi}{n}\right).
    \end{equation*}

    Calculamos las raíces del denominador:
    \begin{align*}
        1 + x^n &= 0 \implies x^n = -1 \implies x\in A:=\left\{e^{i\frac{\pi + 2k\pi}{n}} : k=0, 1, \ldots, n-1\right\}.
    \end{align*}

    Por simplicidad, notaremos por $x_i = e^{i\frac{\pi + 2i\pi}{n}}$ los polos, para $i=0, 1, \ldots, n-1$. De esta forma, tenemos que:
    \begin{equation*}
        A=\left\{x_0, x_1, \ldots, x_{n-1}\right\}
    \end{equation*}

    Definimos la función:
    \Func{f}{\bb{C}\setminus A}{\bb{C}}{z}{\frac{1}{1 + z^n}}

    Notemos que $f\in \cc{H}(\bb{C}\setminus A)$, y que $A'=\emptyset$, por lo que podemos aplicar el Teorema de los Residuos. Como $\bb{C}$ es homológicamente conexo, podemos aplicar el Teorema de los Residuos para cualquier ciclo $\Sigma$ en $\bb{C}\setminus A$.

    Para todo $R > 1$, consideramos el siguiente ciclo $\Sigma_R = \gamma_R + \sigma_R + \tau_R$, representado en la Figura~\ref{fig:ej:14.6}, donde:
    \Func{\gamma_R}{[0, R]}{\bb{C}}{t}{t}
    \Func{\sigma_R}{[0, \nicefrac{2\pi}{n}]}{\bb{C}}{t}{Re^{it}}
    \Func{\tau_R}{[0,R]}{\bb{C}}{t}{(R-t)e^{i\nicefrac{2\pi}{n}}}
    \begin{figure}
        \centering
        \begin{tikzpicture}
            \begin{axis}[
                axis lines=middle,
                xlabel={$x$},
                ylabel={$y$},
                xtick=\empty,
                ytick=\empty,
                xmin=-3.5, xmax=3.5,
                ymin=-1, ymax=1,
                axis equal,
            ]
                \def\R{2.4}
                \def\n{6}
                \def\x{0}
                \def\y{0}

                % Polos
                % Representamos los polos en bucle
                \foreach \k in {0,...,\numexpr\n-1\relax} {
                    \pgfmathsetmacro{\angle}{180*(1+2*\k)/\n}
                    \pgfmathsetmacro{\x}{cos(\angle)}
                    \pgfmathsetmacro{\y}{sin(\angle)}
                    \pgfmathsetmacro{\labelx}{\x + 0.5*cos(\angle)} % Offset for label
                    \pgfmathsetmacro{\labely}{\y + 0.5*sin(\angle)} % Offset for label
                    \edef\temp{\noexpand\node[font=\noexpand\footnotesize] at (axis cs:\labelx,\labely) {$e^{i\frac{\pi + 2 \cdot \k \cdot \pi}{\n}}$};}
                    \edef\temp2{\noexpand\node[font=\noexpand\footnotesize] at (axis cs:\labelx,\labely) {$x_{\k}$};}
                    \addplot[
                        only marks,
                        mark=*,
                        mark options={fill=red},
                    ] coordinates {(\x, \y)};
                    %\temp
                    \temp2
                }

                \draw[dashed] (1, 0) arc[start angle=0, end angle=360, radius=1]
                    node[font=\footnotesize, pos=0.3, yshift=-0.8em] {$C(0,1)$};
                \draw[dashed] (\R, 0) arc[start angle=0, end angle=360, radius=\R]
                    node[pos=0.28, below left, yshift=-10em, xshift=-1em] {$C(0,R)$};
                
                
                \draw[thick, blue, arrow at 1/3, arrow at 2/3] (\R, 0) arc[start angle=0, end angle={360/\n}, radius=\R]
                    node[midway, below left, yshift=-1em, xshift=-1em] {$\sigma_r$};

                % gamma_r
                \draw[thick, blue, arrow at 1/3, arrow at 2/3] (0, 0) -- (\R, 0)
                    node[pos=0.25, below] {$\gamma_r$};


                % tau_r
                \draw[thick, blue, arrow at 1/3, arrow at 2/3] ({\R*cos(360/\n)}, {\R*sin(360/\n)}) -- (0, 0) node[pos=0.25, below, yshift=-9.2em, xshift=0.3em] {$\tau_r$};

            \end{axis}
        \end{tikzpicture}
        \caption{Ciclo de integración $\Sigma_R$ del Ejercicio~\ref{ej:14.6}.}
        \label{fig:ej:14.6}
    \end{figure}

    De esta forma, tenemos que:
    \begin{align*}
        \int_{\Sigma_R} f(z) \, dz &= \int_{\gamma_R} f(z) \, dz + \int_{\sigma_R} f(z) \, dz + \int_{\tau_R} f(z) \, dz \\
        &= 2\pi i\sum_{z_0\in A}\Res(f,z_0)\Ind_{\Sigma_R}(z_0)
    \end{align*}

    Calculemos la primera integral que nos ha resultado:
    \begin{align*}
        \int_{\gamma_R} f(z) \, dz &= \int_{0}^{R} \frac{1}{1 + t^n} \, dt
    \end{align*}
    Notemos que es la integral pedida, por lo que buscamos que las integrales resultantes de las otras dos integrales sean cero.
    \begin{align*}
        \left| \int_{\sigma_R} f(z) \, dz \right|
        &\leq \frac{2\pi R}{n}\cdot \sup\left\{\left|\dfrac{1}{1 + z^n}\right| : z\in \sigma_R^*\right\}
        = \frac{2\pi R}{n}\cdot \frac{1}{R^n - 1}
        = \frac{2\pi}{n(R^{n-1} - \nicefrac{1}{R})}
    \end{align*}
    donde hemos usado que, si $z\in \sigma_R^*$, entonces $|z|=R$ y, como $R>1$, tenemos que $R^n>1$, por lo que:
    \begin{align*}
        |1 + z^n| &\geq \left|1- |z|^n\right| = \left|1 - R^n\right| = R^n - 1.
    \end{align*}
    Como la expresión anterior es válida para cualquier $R > 1$, podemos hacer $R \to +\infty$ y tenemos que (usando que $n > 2$):
    \begin{align*}
        \lim_{R\to+\infty} \int_{\sigma_R} f(z) \, dz &= 0.
    \end{align*}

    Calculamos ahora la tercera integral:
    \begin{align*}
        \int_{\tau_R} f(z) \, dz &= \int_{0}^{R} \frac{1}{1 + (R-t)^n \left(e^{i\frac{2\pi}{n}}\right)^n} \cdot \left(-e^{i\frac{2\pi}{n}}\right) \, dt
        =\\&= -e^{i\frac{2\pi}{n}}\int_{0}^{R} \frac{1}{1 + (R-t)^n} \, dt
    \end{align*}

\end{ejercicio}

\begin{ejercicio}
    Probar que, para $a, t \in \bb{R}^+$, se tiene:
    \begin{equation*}
        \int_{-\infty}^{+\infty} \frac{\cos(tx)}{(x^2 + a^2)^2} \, dx = \frac{\pi}{2a^3}(1 + a t)e^{-a t}.
    \end{equation*}
\end{ejercicio}

\begin{ejercicio}
    Probar que:
    \begin{equation*}
        \int_{-\infty}^{+\infty} \frac{x\sen(\pi x)}{x^2-5x+6} \, dx = -5\pi.
    \end{equation*}
    \begin{observacion}
        Para resolver este ejercicio, se puede integrar la función $\frac{ze^{i\pi z}}{z^2-5z+6}$ en un semicírculo retocado (que rodee los polos de la función).
    \end{observacion}
\end{ejercicio}

\begin{ejercicio}
    Integrando la función $z \mapsto \frac{1-e^{2i z}}{z^2}$ sobre un camino cerrado que recorra la frontera de la mitad superior del anillo $A(0; \varepsilon, R)$, probar que:
    \begin{equation*}
        \int_0^{+\infty} \frac{\sen^2(x)}{x^2} \, dx = \frac{\pi}{2}.
    \end{equation*}
\end{ejercicio}

\begin{ejercicio}
    Dado $a \in \bb{R}$ con $a > 1$, integrar la función $z \mapsto \frac{z}{a - e^{-i z}}$ sobre la poligonal $\left[-\pi, \pi, \pi + i n, -\pi + i n, -\pi\right]$, con $n \in \bb{N}$, para probar que:
    \begin{equation*}
        \int_{-\pi}^{\pi} \frac{x\sen(x)}{1 + a^2 - 2a\cos(x)} \, dx = \frac{2\pi}{a}\log\left(\frac{1 + a}{a}\right).
    \end{equation*}
\end{ejercicio}

\begin{ejercicio}
    Integrando una conveniente función compleja a lo largo de la frontera de la mitad superior del anillo $A(0; \varepsilon, R)$, probar que, para $\alpha \in \left]-1, 3\right[$, se tiene:
    \begin{equation*}
        \int_0^{+\infty} \frac{x^\alpha}{(1 + x^2)^2} \, dx = \frac{\pi}{4}(1 - \alpha)\sec\left(\frac{\pi \alpha}{2}\right).
    \end{equation*}
    \begin{observacion}
        Para resolver este ejercicio, se puede hacer el cambio de variable $x = e^t$ y luego integrar $\frac{e^{t(\alpha + 1)}}{(1 + e^{2t})^2}$ en un rectángulo.
    \end{observacion}
\end{ejercicio}

\begin{ejercicio}
    Probar que, para $\alpha \in \left]0, 2\right[$, se tiene:
    \begin{equation*}
        \int_{-\infty}^{+\infty} \frac{e^{\alpha x}}{1 + e^x + e^{2x}} \, dx = \int_0^{+\infty} \frac{t^{\alpha - 1}}{1 + t + t^2} \, dt = \dfrac{2\pi}{\sqrt{3}}\cdot \dfrac{\sen\left(\frac{\pi(1 - \alpha)}{3}\right)}{\sen\left(\pi \alpha\right)}.
    \end{equation*}
    \begin{observacion}
        Para resolver este ejercicio, se puede integrar $\frac{e^x \alpha}{1 + e^x + e^{2x}}$ en un rectángulo.
    \end{observacion}
\end{ejercicio}

\begin{ejercicio}
    Integrando la función $z \mapsto \frac{\log(z + i)}{1 + z^2}$ sobre un camino cerrado que recorra la frontera del conjunto $\{z \in \bb{C} : |z| < R, \Im z > 0\}$, con $R \in \bb{R}$ y $R > 1$, calcular la integral:
    \begin{equation*}
        \int_{-\infty}^{+\infty} \frac{\log(1 + x^2)}{1 + x^2} \, dx.
    \end{equation*}
\end{ejercicio}

\begin{ejercicio}
    Integrando una conveniente función sobre la poligonal $[-R, R, R + \pi i, -R + \pi i, -R]$, con $R \in \bb{R}^+$, calcular la integral:
    \begin{equation*}
        \int_{-\infty}^{+\infty} \frac{\cos(x)}{e^x + e^{-x}} \, dx.
    \end{equation*}
\end{ejercicio}

\begin{ejercicio}
    Integrando una conveniente función sobre un camino cerrado que recorra la frontera del conjunto $\{z \in \bb{C} : \varepsilon < |z| < R, 0 < \arg z < \nicefrac{\pi}{2}\}$, con $0 < \varepsilon < 1 < R$, calcular la integral:
    \begin{equation*}
        \int_0^{+\infty} \frac{\log(x)}{1 + x^4} \, dx.
    \end{equation*}
\end{ejercicio}

\begin{ejercicio}
    Integrando una conveniente función sobre la poligonal $[-R, R, R + 2\pi i, -R + 2\pi i, -R]$, con $R \in \bb{R}^+$, calcular la integral:
    \begin{equation*}
        \int_{-\infty}^{+\infty} \frac{e^{\nicefrac{x}{2}}}{e^x + 1} \, dx.
    \end{equation*}
\end{ejercicio}