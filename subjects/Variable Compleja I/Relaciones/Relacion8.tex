\section{Equivalencia entre analiticidad y holomorfía}

\begin{ejercicio}
    Sea $\gamma$ un camino y $\varphi : \gamma^* \to \bb{C}$ una función continua. Se define $f : \bb{C}\setminus \gamma^* \to \bb{C}$ por:
    \Func{f}{\bb{C}\setminus \gamma^*}{\bb{C}}{z}{\displaystyle\int_{\gamma} \frac{\varphi(w)}{w-z}\ dw}

    Probar que $f$ es una función analítica en $\bb{C}\setminus \gamma^*$ y que:
    \begin{equation*}
        f^{(k)}(z) = k! \int_{\gamma} \frac{\varphi(w)}{(w-z)^{k+1}}dw \quad \forall z \in \bb{C}\setminus \gamma^*, \forall k \in \bb{N}
    \end{equation*}~

    Realizaremos un desarrollo similar al de la demostración del desarrollo en Serie de Taylor. Como $\gamma$ es continua y está definida en un compacto, $\gamma*$ es cerrado. Por tanto, sea le abierto $\Omega=\bb{C}\setminus \gamma^*$. Fijamos ahora $a\in \Omega$, y sea $R_a=d(a,\gamma^*)>0$. Entonces, para cada $w\in \gamma^*$ y $z\in D(a,R_a)$, se tiene que:
    \begin{align*}
        \sum_{n=0}^{\infty} \left(\frac{z-a}{w-a}\right)^n &= \frac{1}{1-\frac{z-a}{w-a}} = \frac{w-a}{w-z}
    \end{align*}

    Por lo tanto, para cada $z\in D(a,R_a)$, se tiene que:
    \begin{align*}
        f(z) &= \int_{\gamma} \frac{\varphi(w)}{w-z}\ dw = \int_{\gamma} \varphi(w)\cdot \dfrac{1}{w-a}\cdot \frac{w-a}{w-z}\ dw \\
        &= \int_{\gamma} \varphi(w)\cdot \dfrac{1}{w-a}\cdot \sum_{n=0}^{\infty} \left(\frac{z-a}{w-a}\right)^n\ dw \\
        &= \int_{\gamma}\ \sum_{n=0}^{\infty} \left(\varphi(w)\cdot \dfrac{1}{(w-a)^{n+1}}\cdot (z-a)^n\right)\ dw 
    \end{align*}

    Aplicamos ahora el Test de Weierstrass para demostrar la convergencia uniforme de la serie. Como $\varphi$ es continua y está definida en un compacto, $\exists M\in \bb{R}^+$ tal que $|\varphi(w)|\leq M$ para todo $w\in \gamma^*$. Entonces, para cada $n\in \bb{N}$ y $z\in D(a,R_a)$, se tienw que:
    \begin{align*}
        \left|\varphi(w)\cdot \dfrac{1}{(w-a)^{n+1}}\cdot (z-a)^n\right| &\leq M\cdot \frac{|z-a|^n}{|w-a|^{n+1}}
        \leq \dfrac{M}{R_a} \cdot \left(\frac{|z-a|}{R_a}\right)^n \qquad \forall w\in \gamma^*
    \end{align*}

    Como $|z-a|<R_a$, se tiene que la serie de término general la cota dada es convergente. Por lo tanto, por el Test de Weierstrass, la serie converge uniformemente en $\gamma^*$, y por lo tanto podemos cambiar el orden de integración y suma. Entonces, se tiene que:
    \begin{align*}
        f(z) &= \sum_{n=0}^{\infty} (z-a)^n \int_{\gamma} \dfrac{\varphi(w)}{(w-a)^{n+1}}\ dw
        \qquad \forall z\in D(a,R_a)
    \end{align*}

    Por tanto, como este razonamiento es válido para cualquier $a\in \Omega$, se tiene la analiticidad de $f$ en $\Omega$. Para la segunda parte, por el Teorema de Holomorfía de funciones dadas como suma de series de potencias, se tiene que:
    \begin{align*}
        f^{(k)}(z) &= \sum_{n=k}^{\infty} (z-a)^{n-k}\cdot \dfrac{n!}{(n-k)!}\cdot \int_{\gamma} \dfrac{\varphi(w)}{(w-a)^{n+1}}\ dw
        \qquad \forall z\in D(a,R_a),\ \forall k\in \bb{N}
    \end{align*}

    En particular, evaluando en $z=a$, se tiene que:
    \begin{align*}
        f^{(k)}(a) &= k!\cdot \int_{\gamma} \varphi(w)\cdot \dfrac{\varphi(w)}{(w-a)^{k+1}}\ dw\qquad \forall k\in \bb{N},\ \forall a\in \Omega
    \end{align*}
\end{ejercicio}

\begin{ejercicio}\label{ej:8.2}
    Para $\alpha \in \bb{C}$ se define:
    \begin{equation*}
        \binom{\alpha}{0} = 1 \quad \text{y} \quad \binom{\alpha}{n} = \frac{1}{n!} \prod_{j=0}^{n-1} (\alpha-j) = \frac{\alpha(\alpha-1)\cdots(\alpha-n+1)}{n!} \quad \forall n \in \bb{N}
    \end{equation*}
    Probar que:
    \begin{equation*}
        (1+z)^{\alpha} = \sum_{k=0}^{\infty} \binom{\alpha}{n} z^n \quad \forall z \in D(0,1)
    \end{equation*}~

    Definimos la función $f$ siguiente:
    \Func{f}{D(0,1)}{\bb{C}}{z}{(1+z)^{\alpha}}

    Por la definición de potencia principal, se tiene que:
    \begin{equation*}
        f(z) = e^{\alpha \log(1+z)}\qquad \forall z \in D(0,1)
    \end{equation*}

    Como $\log(1+z)$ es holomorfa en $D(0,1)$, se tiene que $f\in \cc{H}(D(0,1))$. Por lo tanto, por el Desarrollo en Serie de Taylor de $f$ centrado en el origen, se tiene que:
    \begin{equation*}
        f(z) = \sum_{n=0}^{\infty} \frac{f^{(n)}(0)}{n!} z^n \quad \forall z \in D(0,1)
    \end{equation*}

    Veamos por inducción sobre $n$ que:
    \begin{equation*}
        f^{(n)}(z) = \binom{\alpha}{n}\cdot n!\cdot (1+z)^{\alpha-n} \qquad \forall n \in \bb{N}, \forall z \in D(0,1)
    \end{equation*}
    \begin{itemize}
        \item Para $n=0$, se tiene que:
        \begin{equation*}
            f(z) = (1+z)^{\alpha} = \binom{\alpha}{0}\cdot 0!\cdot (1+z)^{\alpha-0}
        \end{equation*}

        \item Para $n=1$, se tiene que:
        \begin{equation*}
            f'(z) = \alpha(1+z)^{\alpha-1} = \binom{\alpha}{1}\cdot 1!\cdot (1+z)^{\alpha-1}
        \end{equation*}

        \item Supongamos que es cierto para $n$, y veamos que es cierto para $n+1$. Entonces, se tiene que:
        \begin{align*}
            f^{(n+1)}(z) &= \binom{\alpha}{n}\cdot n!\cdot (\alpha-n)(1+z)^{\alpha-n-1} \\
            &= \dfrac{1}{\cancel{n!}}\prod_{j=0}^{n-1} (\alpha-j)\cdot \cancel{n!}\cdot (\alpha-n)(1+z)^{\alpha-n-1}\cdot \dfrac{(n+1)!}{(n+1)!} \\
            &= \dfrac{1}{(n+1)!}\prod_{j=0}^{n} (\alpha-j)\cdot (n+1)!\cdot (1+z)^{\alpha-n-1} \\
            &= \binom{\alpha}{n+1}\cdot (n+1)!\cdot (1+z)^{\alpha-(n+1)} \qquad \forall z \in D(0,1)
        \end{align*}
        Por lo tanto, se ha probado por inducción. Evaluando en $z=0$, se tiene que:
        \begin{equation*}
            f^{(n)}(0) = \binom{\alpha}{n}\cdot n!\qquad \forall n \in \bb{N}
        \end{equation*}

        Por lo tanto, se tiene que:
        \begin{align*}
            (1+z)^{\alpha}=f(z) &= \sum_{n=0}^{\infty} \frac{f^{(n)}(0)}{n!} z^n \\
            &= \sum_{n=0}^{\infty} \binom{\alpha}{n}\cdot z^n\qquad \forall z \in D(0,1)
        \end{align*}
    \end{itemize}
\end{ejercicio}

\begin{ejercicio}
    Obtener el desarrollo en serie de Taylor de la función $f$, centrado en el origen, en cada uno de los siguientes casos:
    \begin{enumerate}
        \item $f(z) = \log(z^2 - 3z + 2) \quad \forall z \in D(0,1)$
        
        Fijado $z\in D(0,1)$, se tiene que:
        \begin{align*}
            \Re(z^2 - 3z + 2) &= \Re(z^2) - 3\Re(z) + 2
            = \Re(z)^2-\Im(z)^2 - 3\Re(z) + 2\\
            \Im(z^2 - 3z + 2) &= \Im(z^2) - 3\Im(z) = 2\Re(z)\Im(z) - 3\Im(z)
        \end{align*}

        Supongamos que $z^2 - 3z + 2 \in \bb{R}^-_0$. Entonces, se tiene que:
        \begin{align*}
            0 = \Im(z^2 - 3z + 2) \iff 2\Re(z)\Im(z) = 3\Im(z)\iff \left\{
            \begin{array}{c}
                \Im(z) = 0\\
                \lor\\
                \Re(z) = \frac{3}{2}
            \end{array}\right.
        \end{align*}
        
        Distinguimos dos casos:
        \begin{itemize}
            \item Supongamos $\Im(z) = 0$. Como la parte real es negativa, resolvemos la inecuación:
            \begin{equation*}
                \Re(z)^2-3\Re(z)+2<0
            \end{equation*}

            Las raíces de la ecuación son $1$ y $2$, y evaluando en $\Re(z)=0$ vemos que:
            \begin{equation*}
                \Re(z)^2-3\Re(z)+2>0\qquad \forall z\in D(0,1)
            \end{equation*}

            Por lo tanto, llegamos a una contradicción.

            \item Supongamos $\Re(z) = \frac{3}{2}$. En este caso, $z\notin D(0,1)$, y por lo tanto no es posible.
        \end{itemize}

        Por tanto, como el logaritmo principal es holomorfo en $C^*\setminus \bb{R}^-$, se tiene que $f\in \cc{H}(D(0,1))$. Por el Desarollo en Serie de Taylor, se tiene que:
        \begin{equation*}
            f(z) = \sum_{n=0}^{\infty} \frac{f^{(n)}(0)}{n!} z^n \qquad \forall z \in D(0,1)
        \end{equation*}

        Obtener la derivada $n-$ésima de $f$ en el origen no es sencillo, por lo que optaremos por otro método para obtener el desarrollo. Tenemos que:
        \begin{equation*}
            f'(z) = \frac{2z-3}{z^2-3z+2}\qquad \forall z \in D(0,1)
        \end{equation*}

        Descomponemos la función en fracciones parciales:
        \begin{equation*}
            \frac{2z-3}{z^2-3z+2} = \frac{A}{z-1} + \frac{B}{z-2}= \frac{A(z-2)+B(z-1)}{(z-1)(z-2)}
        \qquad \forall z \in D(0,1)
        \end{equation*}
        \begin{itemize}
            \item Para $z=1$, se tiene que $-1=-A$, y por lo tanto $A=1$.
            \item Para $z=2$, se tiene que $1=B$, y por lo tanto $B=1$.
        \end{itemize}

        Por lo tanto, para cada $z\in D(0,1)$, se tiene que:
        \begin{align*}
            f'(z) &= \frac{1}{z-1} + \frac{1}{z-2}
            = -\left(\frac{1}{1-z} + \frac{1}{2}\cdot \frac{1}{1-\frac{z}{2}}\right) \\
            &= -\left(\sum_{n=0}^{\infty} z^n + \frac{1}{2}\cdot \sum_{n=0}^{\infty} \left(\frac{z}{2}\right)^n\right)
            = -\left(\sum_{n=0}^{\infty}\left(1+\frac{1}{2^{n+1}}\right)z^n\right)
        \end{align*}

        Integrando término a término, se tiene que:
        \begin{align*}
            f(z) &= -\left(\sum_{n=0}^{\infty}\left(1+\frac{1}{2^{n+1}}\right)\cdot \frac{z^{n+1}}{n+1}\right)+C =
            -\left(\sum_{n=0}^{\infty}\dfrac{2^{n+1}+1}{2^{n+1}(n+1)}\cdot z^{n+1}\right)+C \\
            &= -\left(\sum_{n=1}^{\infty}\dfrac{2^{n}+1}{n2^n}\cdot z^{n}\right)+C\qquad \forall z \in D(0,1)
        \end{align*}

        Como $f(0)=\log 2 = \ln 2$, se tiene que $C=\ln 2$ y el desarrollo en Serie de Taylor buscado es:
        \begin{align*}
            f(z) &= -\left(\sum_{n=1}^{\infty}\dfrac{2^{n}+1}{n2^n}\cdot z^{n}\right)+\ln 2 \qquad \forall z \in D(0,1)
        \end{align*} 
        \item $f(z) = \dfrac{z^2}{(z+1)^2} \quad \forall z \in \bb{C}\setminus\{-1\}$
        
        Por ser racional, sabemos que $f\in \cc{H}(\bb{C}\setminus\{-1\})$, por lo que solo podremos aspirar a un desarrollo en serie de Taylor en $D(0,1)$. Hay dos opciones:
        \begin{description}
            \item[Fracciones Simples] 
            
            Para evitar el cálculo de la derivada $n-$ésima, descomponemos en fracciones simples, pero antes hemos de realizar la división de polinomios:
            \begin{equation*}
                \polyset{vars=z}
                \polylongdiv[style=D]{z^2}{z^2+2z+1}
            \end{equation*}

            Por lo tanto, se tiene que:
            \begin{equation*}
                f(z) = 1 - \frac{2z+1}{(z+1)^2} = 1-\left(\dfrac{A}{z+1}+\dfrac{B}{(z+1)^2}\right)
                = 1-\left(\dfrac{A(z+1)+B}{(z+1)^2}\right)
            \end{equation*}
            \begin{itemize}
                \item Para $z=-1$, se tiene que $-1=B$.
                \item Para $z=0$, se tiene que $1=A+B$, y por lo tanto $A=2$.
            \end{itemize}

            Por tanto:
            \begin{equation*}
                f(z) = 1 - \frac{2}{z+1} + \frac{1}{(z+1)^2} \qquad \forall z \in D(0,1)
            \end{equation*}

            Viendo el tercer sumando como la derivada de una función que es la suma de una serie geométrica, se tiene que:
            \begin{align*}
                f(z) &= 1 - 2\cdot \left(\sum_{n=0}^{\infty} (-z)^n\right) + \left(-\sum_{n=0}^{\infty} (-z)^n\right)'
                =\\&= 1-2\cdot \left(\sum_{n=0}^{\infty} (-z)^n\right) + \left(-\sum_{n=1}^{\infty} -n(-z)^{n-1}\right) =\\
                &= 1-2\cdot \left(\sum_{n=0}^{\infty} (-z)^n\right) + \left(\sum_{n=0}^{\infty} (n+1)(-z)^{n}\right)= \\
                &= 1-\left(\sum_{n=0}^{\infty} \left(n-1\right)(-z)^{n}\right)
                = 1-\left(\sum_{n=0}^{\infty} \left(n-1\right)(-1)^{n}z^{n}\right) 
                =\\&= 1+\left(\sum_{n=0}^{\infty} \left(n-1\right)(-1)^{n+1}z^{n}\right) \qquad \forall z \in D(0,1)
            \end{align*}

            \item[Usando el Ejercicio~\ref{ej:8.2}]
            
            Ya hemos visto que solo podemos aspirar a un desarrollo en serie de Taylor en $D(0,1)$. En ese conjunto, tenemos que:
            \begin{align*}
                f(z) &= z^2(1+z)^{-2} = z^2\cdot \left(\sum_{n=0}^{\infty} \binom{-2}{n}\cdot z^n\right)
                =\\&= \sum_{n=0}^{\infty} \binom{-2}{n}\cdot z^{n+2}
                = \sum_{n=2}^{\infty} \binom{-2}{n-2}\cdot z^{n} \qquad \forall z \in D(0,1)
            \end{align*}
        \end{description}
        
        
        
        \item $f(z) = \arcsen z \quad \forall z \in D(0,1)$
        
        \item $f(z) = \cos^2 z \quad \forall z \in \bb{C}$
    \end{enumerate}
\end{ejercicio}

\begin{ejercicio}
    Dado $\alpha \in \bb{C}^*\setminus \bb{N}$, probar que existe una única función $f$ tal que $f \in \cc{H}(D(0,1))$ verificando que:
    \begin{equation*}
        z f'(z) - \alpha f(z) = \frac{1}{1+z} \quad \forall z \in D(0,1)
    \end{equation*}
\end{ejercicio}

\begin{ejercicio}
    Probar que existe una única función $f$ tal que $f \in \cc{H}(D(0,1))$ verificando que $f(0) = 0$ y:
    \begin{equation*}
        \exp\left({-z} f'(z)\right) = 1-z \quad \forall z \in D(0,1)
    \end{equation*}
\end{ejercicio}

\begin{ejercicio}
    Para $z \in \bb{C}$ con $1-z-z^2 \neq 0$ se define $f(z) = (1-z-z^2)^{-1}$. Sea $\sum\limits_{n>0} \alpha_n z^n$ la serie de Taylor de $f$ centrada en el origen. Probar que $\{\alpha_n\}$ es la sucesión de Fibonacci:
    \begin{equation*}
        \alpha_0 = \alpha_1 = 1 \quad \text{y} \quad \alpha_{n+2} = \alpha_n + \alpha_{n+1} \quad \forall n \in \bb{N}\cup\{0\}
    \end{equation*}
    Calcular en forma explícita dicha sucesión.
\end{ejercicio}

\begin{ejercicio}
    En cada uno de los siguientes casos, decidir si existe una función $f$ tal que $f \in \cc{H}(\Omega)$ verificando que $f^{(n)}(0) = a_n$ para todo $n \in \bb{N}$:
    \begin{enumerate}
        \item $\Omega = \bb{C}$,\qquad $a_n = n$
        \item $\Omega = \bb{C}$,\qquad $a_n = (n+1)!$
        \item $\Omega = D(0,1)$,\qquad $a_n = 2^n n!$
        \item $\Omega = D(0,\nicefrac{1}{2})$,\qquad $a_n = n^n$
    \end{enumerate}
\end{ejercicio}

\begin{ejercicio}
    Dados $r \in \bb{R}^+$, $k \in \bb{N}$, y $a,b \in \bb{C}$ con $|b| < r < |a|$, calcular la siguiente integral:
    \begin{equation*}
        \int_{C(0,r)} \frac{dz}{(z-a)(z-b)^k}
    \end{equation*}
\end{ejercicio}

\begin{ejercicio}
    Calcular la integral para cada una de las siguientes curvas:
    \begin{equation*}
        \int_{\gamma} \frac{e^z}{z^2(z-1)}dz
    \end{equation*}
    \begin{enumerate}
        \item $\gamma = C(\nicefrac{1}{4},\nicefrac{1}{2})$
        \item $\gamma = C(1,\nicefrac{1}{2})$
        \item $\gamma = C(0,2)$
    \end{enumerate}
\end{ejercicio}

\begin{ejercicio}
    Dado $n \in \bb{N}$, calcular las siguientes integrales:
    \begin{enumerate}
        \item $\displaystyle\int_{C(0,1)} \frac{\sen z}{z^n}dz$
        \item $\displaystyle\int_{C(0,1)} \frac{e^z - e^{-z}}{z^n}dz$
        \item $\displaystyle\int_{C(0,\nicefrac{1}{2})} \frac{\log(1+z)}{z^n}dz$
    \end{enumerate}
\end{ejercicio}

\begin{ejercicio}[Fórmula de cambio de variable]
    Si $\Omega$ es un abierto del plano, $\gamma$ un camino en $\Omega$ y $\varphi \in \cc{H}(\Omega)$, entonces $\varphi \circ \gamma$ es un camino y, para cualquier función $f$ que sea continua en $(\varphi\circ \gamma)^*$ se tiene:
    \begin{equation*}
        \int_{\varphi\circ\gamma} f(z)dz = \int_{\gamma} f(\varphi(w)) \varphi'(w)dw
    \end{equation*}
\end{ejercicio}

\begin{ejercicio}
    Usar el resultado del ejercicio anterior para calcular las siguientes integrales:
    \begin{enumerate}
        \item $\displaystyle\int_{C(0,2)} \frac{dz}{z^2(z-1)^2}$
        \item $\displaystyle\int_{C(0,2)} \frac{dz}{(z-1)^2(z+1)^2(z-3)}$
    \end{enumerate}
\end{ejercicio}