\newpage
\section{Estimación por intervalos de confianza}

\begin{ejercicio}
    Sea $\overline{X}$ la media de una muestra aleatoria de tamaño $n$ de una población $\cc{N}(\mu, 16)$ Encontrar el menor valor de $n$ para que $\left]\overline{X}-1,\overline{X}+1\right[$ sea un intervalo de confianza para $\mu$ al nivel de confianza $0.9$.\\

    \noindent
    Si $\overline{X}$ es la media de una m.a.s. de tamaño $n$ de una población $\cc{N}(\mu,16)$, tenemos entonces que $\overline{X}\rightsquigarrow\cc{N}(\mu, \frac{16}{n})$. Calculamos $n$ para que el intervalo $\left]\overline{X}-1,\overline{X}+1\right[$ sea un intervalo de confianza para $\mu$ a nivel $0.9$, buscando:
    \begin{align*}
        P_\mu\left[\overline{X}-1 \leq \mu \leq \overline{X}+1\right] \geq 0.9 &\Longleftrightarrow P_\mu[\mu-1\leq \overline{X}\leq \mu+1] \geq 0.9 \\
                                                                    &\stackrel{\text{tipif.}}{\Longleftrightarrow} P_\mu\left[\frac{-\sqrt{n}}{4} \leq Z \leq \frac{\sqrt{n}}{4}\right] \geq 0.9 \\
                                                                    &\Longleftrightarrow 2P_\mu\left[Z\leq \frac{\sqrt{n}}{4}\right] - 1 \geq 0.9 \\
                                                                    &\Longleftrightarrow 2P_\mu\left[Z\leq \frac{\sqrt{n}}{4}\right] \geq 1.9 \\
                                                                    &\Longleftrightarrow P_\mu\left[Z\leq \frac{\sqrt{n}}{4}\right] \geq 0.95 
    \end{align*}
    Consultando la tabla de la normal $\cc{N}(0,1)$, tenemos que la primera abscisa en la que se alcanza una probabilidad superior a $0.95$ es $1.65$, por lo que:
    \begin{equation*}
        \frac{\sqrt{n}}{4} = 1.65 \Longleftrightarrow \sqrt{n} = 6.6 \Longleftrightarrow n = 43.56
    \end{equation*}
    Por tanto, el menor valor de $n$ para el cual el intervalo $\left]\overline{X}-1,\overline{X}+1\right[$ es un intervalo de confianza para $\mu$ a nivel de confianza $0.9$ es $44$.
\end{ejercicio}

\begin{ejercicio}
    La altura en cm. de los individuos varones de una población sigue una distribución $\cc{N}(\mu,\ 56.25)$. Si en una muestra aleatoria simple de tamaño $12$ de dicha población se obtiene una altura media de $175$ cm., determinar un intervalo de confianza para $\mu$ al nivel de confianza $0.95$. ¿Qué tamaño de muestra es necesario para que el intervalo de confianza a dicho nivel tenga longitud menor que 1 cm?\\

    \noindent
    Buscamos un intervalo de confianza para $\mu$ en una población normal con varianza $\sigma_0$ conocida a nivel de confianza $1-\alpha = 0.95$ (por lo que $\alpha= 0.05$). Tenemos pues una muestra aleatoria simple $(X_1, \ldots, X_n)$ de $X\rightsquigarrow\cc{N}(\mu,56.25)$. Usaremos el método de la cantidad pivotal, usando para ello la función pivote (donde $n=12$):
    \begin{equation*}
        T(X_1, \ldots, X_n; \mu) = \frac{\overline{X}-\mu}{\nicefrac{\sigma_0}{\sqrt{n}}} \rightsquigarrow\cc{N}(0,1)
    \end{equation*}
    Que:
    \begin{itemize}
        \item Es estrictamente decreciente respecto $\mu$.
        \item Si tomamos $\lm$ de forma que:
            \begin{equation*}
                \lm = \frac{\overline{X}-\mu}{\nicefrac{\sigma_0}{\sqrt{n}}} \Longrightarrow \mu = \overline{X} - \lm \frac{\sigma_0}{\sqrt{n}}
            \end{equation*}
    \end{itemize}
    Por lo que el intervalo de confianza que consideraremos viene dado por:
    \begin{equation*}
        \left]\overline{X}-\lm_2 \frac{\sigma_0}{\sqrt{n}}, \overline{X}-\lm_1 \frac{\sigma_0}{\sqrt{n}}\right[
    \end{equation*}
    donde $\lm_1$ y $\lm_2$ están sujetos a la restricción $P_\mu[\lm_1<T<\lm_2]=1-\alpha$. Obtenemos un intervalo de longitud esperada:
    \begin{equation*}
        E[L] = E\left[(\lm_2-\lm_1)\frac{\sigma_0}{\sqrt{n}}\right] = (\lm_2-\lm_1)\frac{\sigma_0}{\sqrt{n}}
    \end{equation*}
    Como $\nicefrac{\sigma_0}{\sqrt{n}}$ es una constante positiva, bastará minimizar la cantidad $\lm_2-\lm_1$ (sujeta a la restricción $P_\mu[\lm_1<T<\lm_2]=1-\alpha$) para minimizar la longitud del intervalo. La restricción mencionada puede reescribirse como: 
    \begin{equation*}
        1-\alpha = P_\mu[\lm_1<T<\lm_2] = F_Z(\lm_2) - F_Z(\lm_1)
    \end{equation*}
    Consideramos por el método de los multiplicadores de Lagrange:
    \begin{equation*}
        F(\lm_1,\lm_2) = (\lm_2-\lm_1) - \lm(F_Z(\lm_2) - F_Z(\lm_1) - (1-\alpha))
    \end{equation*}
    y buscamos los valores de $\lm_1$ y $\lm_2$ que minimicen la expresión, por lo que calculamos sus derivadas parciales:
    \begin{equation*}
        \dfrac{\partial F}{\partial \lm_1} = -1+\lm f_Z(\lm_1) \qquad \dfrac{\partial F}{\partial \lm_2} = 1-\lm f_Z(\lm_2)
    \end{equation*}
    Si igualamos ambas a cero y tratamos de despejar $\lm$:
    \begin{equation*}
        \left.\begin{array}{l}
            0 = \dfrac{\partial F}{\partial \lm_1} = -1+\lm f_Z(\lm_1) \\
            0 = \dfrac{\partial F}{\partial \lm_2} = 1-\lm f_Z(\lm_2)
        \end{array}\right\} \Longrightarrow \left.\begin{array}{l}
            \lm = \frac{1}{f_Z(\lm_1)} \\
            \lm = \frac{1}{f_Z(\lm_2)}
        \end{array}\right\} \Longrightarrow f_Z(\lm_1) = f_Z(\lm_2)
    \end{equation*}
    Como $f_Z$ es una función simétrica respecto al origen, las únicas posibilidades son bien $\lm_1 = \lm_2$ bien $\lm_1 = -\lm_2$. Como estos valores han de cumplir la restricción $P_\mu[\lm_1<T<\lm_2]=1-\alpha > 0$, la primera opción es imposible, por lo que tiene que ser $\lm_1 = -\lm_2$. Como han de dejar entre ellos un valor de $1-\alpha$, han de ser:
    \begin{equation*}
        \lm_2 = Z_{\nicefrac{\alpha}{2}}, \qquad \lm_1 = -\lm_2 = -Z_{\nicefrac{\alpha}{2}}
    \end{equation*}
    Por lo que el intervalo a considerar es:
    \begin{equation*}
        \left]\overline{X}-Z_{\nicefrac{\alpha}{2}}\frac{\sigma_0}{\sqrt{n}}, \overline{X}+Z_{\nicefrac{\alpha}{2}}\frac{\sigma_0}{\sqrt{n}}\right[
    \end{equation*}
    A partir de los datos proporcionados, tenemos que:
    \begin{itemize}
        \item $\alpha = 0.05$.
        \item $\sigma_0 = \sqrt{56.25} = 7.5$.
        \item $\sqrt{n} = \sqrt{12} = 3.4641$.
        \item $\overline{x} = 175$.
        \item $Z_{\nicefrac{\alpha}{2}}$ = $Z_{0.025} = 1.96$.
    \end{itemize}
    Por lo que el intervalo a considerar es:
    \begin{equation*}
        \left]175-1.96\cdot \frac{7.5}{3.4641},175+1.96\cdot \frac{7.5}{3.4641}\right[ = \left]170.7564755, 179.2435245\right[
    \end{equation*}
    Buscamos ahora el tamaño de la muestra para que la longitud del intervalo sea menor que $1$cm. Para ello, vemos que nuestro intervalo tiene longitud:
    \begin{equation*}
        2Z_{\nicefrac{\alpha}{2}}\frac{\sigma_0}{\sqrt{n}} = 2\cdot 1.96 \cdot \frac{7.5}{\sqrt{n}} = \frac{29.4}{\sqrt{n}}  \leq 1 \Longleftrightarrow 29.4 \leq \sqrt{n} \Longleftrightarrow 864.36 \leq n
    \end{equation*}
    Por lo que nicesitaremos al menos una nuestra de tamaño $865$ para que el intervalo de confianza a dicho nivel tenga longitud menor que 1 cm.
\end{ejercicio}

\begin{ejercicio}\label{ej:3_rel6}
    Una fábrica produce tornillos cuyo diámetro medio es $3$ mm. Se seleccionan aleatoria e independientemente 12 de estos tornillos y se miden sus diámetros, que resultan ser $3.01,\ 3.05,\ 2.99,\ 2.99,\ 3.00,\ 3.02,\ 2.98,\ 2.99,\ 2.97,\ 2.97,\ 3.02$ y $3.01$. Suponiendo que el diámetro es una variable aleatoria con distribución normal, determinar un intervalo de confianza para la varianza al nivel de confianza $0.99$, y una cota superior de confianza al mismo nivel. Interpretar los resultados en términos de la desviación típica del diámetro de los tornillos.\\

    \noindent
    Sea $(X_1, \ldots, X_n)$ una m.a.s. de $X\rightsquigarrow\cc{N}(3,\sigma^2)$, vamos a calcular un intervalo de confianza para $\sigma^2$ en una población normal con media $\mu_0$ conocida a nivel de confianza $1-\alpha$ mediante el método de la cantidad pivotal, usando la función pivote:
    \begin{equation*}
        T(X_1, \ldots, X_n;\sigma^2) = \frac{\sum\limits_{i=1}^n{(X_i-\mu_0)}^{2}}{\sigma^2} \rightsquigarrow\chi^2(n)
    \end{equation*}
    Que:
    \begin{itemize}
        \item Es estrictamente decreciente en $\sigma^2$.
        \item Si tenemos $\lm$ con:
            \begin{equation*}
                \lm = \frac{\sum\limits_{i=1}^n{(X_i-\mu_0)}^{2}}{\sigma^2} \rightsquigarrow\chi^2(n) \Longrightarrow \sigma^2 = \frac{\sum\limits_{i=1}^n{(X_i-\mu_0)}^{2}}{\lm}
            \end{equation*}
    \end{itemize}
    Por lo que el intervalo a considerar será:
    \begin{equation*}
        \left]\frac{1}{\lm_2}\sum\limits_{i=1}^n{(X_i-\mu_0)}^{2},\frac{1}{\lm_1}\sum\limits_{i=1}^n{(X_i-\mu_0)}^{2}\right[
    \end{equation*}
    donde $\lm_1$ y $\lm_2$ están sujetos a la restricción:
    \begin{equation*}
        1-\alpha \leq P_{\sigma^2}[\lm_1<T<\lm_2] = F_T(\lm_2) - F_T(\lm_1) \qquad (T\rightsquigarrow\chi^2(n))
    \end{equation*}
    Tenemos que la longitud esperada del intervalo es:
    \begin{equation*}
        E[L] = E\left[\left(\frac{1}{\lm_1}-\frac{1}{\lm_2}\right)\sum\limits_{i=1}^n{(X_i-\mu_0)}^{2}\right] = \left(\frac{1}{\lm_1}-\frac{1}{\lm_2}\right)E\left[\sum\limits_{i=1}^n{(X_i-\mu_0)}^{2}\right]
    \end{equation*}
    Donde la última esperanza es una cantidad constante y positiva, por lo que trataremos de minimizar simplemente $\nicefrac{1}{\lm_1}-\nicefrac{1}{\lm_2}$. Mediante el método de los multiplicadores de Lagrange:
    \begin{equation*}
        F(\lm_1,\lm_2) = \left(\frac{1}{\lm_1}-\frac{1}{\lm_2}\right) - \lm(F_T(\lm_2)-F_T(\lm_1)-(1-\alpha))
    \end{equation*}
    Calculando las derivadas parciales e igualando a cero llegamos a:
    \begin{equation*}
        \frac{f_T(\lm_1)}{f_T(\lm_2)} = \frac{\lm_2^2}{\lm_1^2}
    \end{equation*}
    Se trata de un despeje difícil, por lo que en la práctica se toman los valores:
    \begin{equation*}
        \lm_1 = \chi^2_{n,1-\nicefrac{\alpha}{2}}, \qquad \lm_2 = \chi^2_{n,\nicefrac{\alpha}{2}}
    \end{equation*}
    En definitiva, consideraremos el intervalo:
    \begin{equation*}
        \left]\frac{1}{\chi^2_{n,\nicefrac{\alpha}{2}}}\sum\limits_{i=1}^n{(X_i-\mu_0)}^{2},\frac{1}{\chi^2_{n,1-\nicefrac{\alpha}{2}}}\sum\limits_{i=1}^n{(X_i-\mu_0)}^{2} \right[
    \end{equation*}
    Para determinar una cota superior, lo que hacemos es tomar $\lm_2 = \infty$, por lo que:
    \begin{equation*}
        1-\alpha = F_T(\lm_2)-F_T(\lm_1) = 1-F_T(\lm_1) \Longrightarrow F_T(\lm_1) = \alpha
    \end{equation*}
    Luego podemos tomar $\lm_1 = \chi^2_{n,1-\alpha}$, obteniendo la cota superior:
    \begin{equation*}
        \frac{1}{\chi^2_{n,1-\alpha}}\sum\limits_{i=1}^n{(X_i-\mu_0)}^{2} 
    \end{equation*}
    Si situimos ahora en los parámetros que tenemos:
    \begin{itemize}
        \item $\mu_0 = 3$.
        \item $n = 12$.
        \item $\alpha = 0.01$.
        \item $\chi^2_{n,\nicefrac{\alpha}{2}} = \chi^2_{12,\ 0.005} = 28.2997$.
        \item $\chi^2_{n,1-\nicefrac{\alpha}{2}} = \chi^2_{12,\ 0.995} = 3.0738$.
        \item $\chi^2_{n,1-\alpha}= \chi^2_{12,0.99} = 3.5706$.
        \item Finalmente:
            \begin{equation*}
                \sum_{i=1}^{n}{(x_i - \mu_0)}^{2} = 0.0059
            \end{equation*}
    \end{itemize}
    El intervalo de confianza a nivel de confianza $0.99$ será:
    \begin{equation*}
        \left]\frac{1}{28.2997} 0.0059, \frac{1}{3.0738} 0.0059\right[ = \left]0.0002084,0.00191945\right[
    \end{equation*}
    Y la cota superior será:
    \begin{equation*}
        \frac{1}{3.5706} 0.0059 = 0.00165238
    \end{equation*}
    La desviación típica de la muestra dada es:
    \begin{equation*}
        \frac{1}{n}\sum_{i=1}^{n}{(x_i-\mu_0)}^{2} = \frac{1}{12}\cdot 0.0059 = 0.000491667
    \end{equation*}
    Que como veos pertence al intervalo y es menor que la cota dada.
\end{ejercicio}

\begin{ejercicio} % // TODO: ESTA MAL
    Las notas en cierta asignatura de 7 alumnos de una clase, elegidos de forma aleatoria e independiente son: $4.5$, $3, 6$, $7$, $1.5$, $5.2$ y $3.6$. Suponiendo que las notas tienen distribución normal, dar un intervalo de confianza para la varianza de las mismas al nivel de confianza $0.95$.\\

    \noindent
    Sea $(X_1, \ldots, X_n)$ una m.a.s. de $X\rightsquigarrow\cc{N}(\mu_0, \sigma^2)$. Tenemos que el intervalo de confianza para $\sigma^2$ a nivel de confianza $1-\alpha$ es:
    \begin{equation*}
        \left]\frac{(n-1)S^2}{\chi^2_{n-1,\nicefrac{\alpha}{2}}},\frac{(n-1)S^2}{\chi^2_{n-1,1-\nicefrac{\alpha}{2}}}\right[
    \end{equation*}
    Pero ahora tenemos los valores:
    \begin{itemize}
        \item $n = 7$.
        \item $\alpha = 0.05$.
        \item $\chi^2_{n,\nicefrac{\alpha}{2}} = \chi^2_{7,0.025} = 16.0128$.
        \item $\chi^2_{n,1-\nicefrac{\alpha}{2}} = \chi^2_{7,0.975} = 1.6899$.
    \end{itemize}
    Por lo que el intervalo será:
    \begin{equation*}
        \left]1.451936, 16.959525\right[
    \end{equation*}
\end{ejercicio}

\begin{ejercicio}
    Dos muestras independientes, cada una de tamaño 7, de poblaciones normales con igual varianza, producen medias $4.8$ y $5.4$ y cuasivarianzas muestrales $8.38$ y $7.62$, respectivamente. Encontrar un intervalo de confianza para la diferencia de medias al nivel de confianza $0.95$.\\

    \noindent
    Si tenemos $(X_1, \ldots, X_{n_1})$ m.a.s.  de $X\rightsquigarrow\cc{N}(\mu_1,\sigma^2_1)$ y $(Y_1, \ldots, Y_{n_2})$ m.a.s. de $Y\rightsquigarrow\cc{N}(\mu_2,\sigma_2^2)$. Si queremos calcular el intervalo de confianza de $\mu_1-\mu_2$ a nivel de confianza $1-\alpha$, hemos visto en teoría que se obtiene:
    \begin{equation*}
        \left]\overline{X}-\overline{Y}-t_{n_1+n_2-2,\nicefrac{\alpha}{2}}S_p\sqrt{\frac{1}{n_1}+\frac{1}{n_2}},\overline{X}-\overline{Y}+t_{n_1+n_2-2,\nicefrac{\alpha}{2}}S_p\sqrt{\frac{1}{n_1}+\frac{1}{n_2}}\right[
    \end{equation*}
    donde:
    \begin{equation*}
        S_p^2 = \frac{(n_1-1)S_1^2 + (n_2-1)S_2^2}{n_1+n_2-2}
    \end{equation*}
    Calculamos el intervalo, usando para ello que:
    \begin{itemize}
        \item $n_1 = 7 = n_2$.
        \item $\overline{x} = 4.8$, $\overline{y} = 5.4$.
        \item $s_1^2 = 8.38$, $s_2^2 = 7.62$.
        \item $\alpha = 0.05$.
    \end{itemize}
    Por lo que:
    \begin{equation*}
        S_p = \sqrt{\frac{6\cdot 8.38+6\cdot 7.62}{7 + 7 -2}} = 2.828427
    \end{equation*}
    De donde el intervalo será, usando que:
    \begin{equation*}
        t_{n_1+n_2-2,\nicefrac{\alpha}{2}} = t_{12,\ 0.025} = 2.1788
    \end{equation*}
    \begin{align*}
        &\left]4.8-5.4-2.1788\cdot 2.828427\cdot \sqrt{\frac{1}{7}+\frac{1}{7}}, 4.8-5.4+2.1788\cdot 2.828427\cdot \sqrt{\frac{1}{7}+\frac{1}{7}}\right[ \\
        &= \left]-3.894036, 2.694036\right[    
    \end{align*}
\end{ejercicio}

\begin{ejercicio}
    La siguiente tabla presenta los salarios anuales (en miles de euros) de dos grupos de recién graduados de dos carreras diferentes. Suponiendo normalidad en los salarios de ambos grupos, determinar un intervalo de confianza para el cociente de las varianzas al nivel de confianza $0.90$.
    \begin{table}[H]
    \centering
    \begin{tabular}{|c|cccccccccc|}
        \hline 
        GRUPO 1 & 16.3 & 18.2 & 17.5 & 16.1 & 15.9 & 15.4 & 15.8 & 17.3 & 14.9 & 15.1 \\
        \hline
        GRUPO 2 & 13.2 & 15.1 & 13.9 & 14.7 & 15.6 & 15.8 & 14.9 & 18.1 & 15.6 & 15.3 \\ 
                & 16.2 & 15.2 & 15.4 & 16.6 & & & & & &  \\
        \hline
    \end{tabular}
    \end{table}
    \noindent
    Si tenemos $(X_1, \ldots, X_{n_1})$ m.a.s.  de $X\rightsquigarrow\cc{N}(\mu_1,\sigma^2_1)$ (que se corresponderá con el grupo 1) y $(Y_1, \ldots, Y_{n_2})$ m.a.s. de $Y\rightsquigarrow\cc{N}(\mu_2,\sigma_2^2)$ (que se corresponderá con el grupo 2). Si queremos calcular el intervalo de confianza de $\frac{\sigma_1^2}{\sigma_2^2}$ a nivel de confianza $1-\alpha$, hemos visto en teoría que este será:
    \begin{equation*}
        \left]F_{n_1-1,n_2-1,1-\nicefrac{\alpha}{2}}\frac{S_1^2}{S_2^2},F_{n_1-1,n_2-1,\nicefrac{\alpha}{2}}\frac{S_1^2}{S_2^2} \right[
    \end{equation*}
    Con los datos:
    \begin{itemize}
        \item $\alpha = 0.1$.
        \item $n_1 = 10$.
        \item $n_2 = 14$.
        \item $S_1^2 = 1.187222$.
        \item $S_2^2 = 1.352308$.
    \end{itemize}
    Tenemos ($F_{9,13}$ no sale en la tabla, tomo los valores de $F_{9,12}$):
    \begin{itemize}
        \item $F_{n_1-1,n_2-1,1-\nicefrac{\alpha}{2}} = F_{9,13,0.95} = 0.325$.
        \item $F_{n_1-1,n_2-1,\nicefrac{\alpha}{2}} = F_{9,13,0.05} = 2.8$.
    \end{itemize}
    Por lo que el intervalo es:
    \begin{equation*}
        \left]0.325\cdot \frac{1.189222}{1.352308}, 2.8\cdot \frac{1.189222}{1.352308}\right[ = \left]0.285325, 2.458184\right[
    \end{equation*}
\end{ejercicio}

\begin{ejercicio}
    Con objeto de estudiar la efectividad de un agente diurético, se eligen al azar 11 pacientes, aplicando dicho fármaco a seis de ellos y un placebo a los cinco restantes. La variable observada en esta experiencia fue la concentración de sodio en la orina a las 24 horas, que se supone tiene una distribución normal en ambos casos. Los resultados observados fueron:
    \begin{itemize}
        \item Diurético: $20.4$, $62.5$, $61.3$, $44.2$, $11.1$, $23.7$.
        \item Placebo: $1.2$, $6.9$, $38.7$, $20.4$, $17.2$.
    \end{itemize}
    \begin{enumerate}[label=\alph*)]
        \item Calcular un intervalo de confianza para el cociente de las varianzas al nivel de confianza $0.95$.
        \item Suponiendo que las varianzas son iguales, calcular un intervalo de confianza para la diferencia de las medias al nivel de confianza $0.9$, y una cota inferior de confianza al mismo nivel. Interpretar los resultados.
    \end{enumerate}
    
    \noindent
    \textbf{Solución.} Tenemos:
    \begin{align*}
        X&\equiv \text{``Concentración de \texttt{Na} en la orina a las 24h tomando diurético''} \rightsquigarrow \cc{N}(\mu_1,\sigma_1^2) \\
        Y&\equiv \text{``Concentración de \texttt{Na} en la orina a las 24h tomando placebo''} \rightsquigarrow \cc{N}(\mu_2, \sigma_2^2)
    \end{align*}
    Y tenemos dos m.a.s. de ellas: $(X_1, \ldots, X_{n_1})$ de $X$ y $(Y_1, \ldots, Y_{n_2})$ de $Y$.
    \begin{enumerate}[label=\alph*)]
        \item Calcular un intervalo de confianza para el cociente de las varianzas al nivel de confianza $0.95$.

            Buscamos un intervalo de confianza para $\frac{\sigma_1^2}{\sigma_2^2}$ a nivel de confianza $1-\alpha = 0.95\Longrightarrow \alpha = 0.05$. La función pivote en este caso es:
            \begin{equation*}
                T\left(X_1, \ldots, X_{n_1},Y_1, \ldots, Y_{n_2};\frac{\sigma_1^2}{\sigma_2^2}\right) = \frac{S_2^2}{S_1^2} \frac{\sigma_1^2}{\sigma_2^2} \rightsquigarrow F(n_2-1,n_1-1)
            \end{equation*}
            Que:
            \begin{itemize}
                \item Sale creciente respecto el parámetro.
                \item Si tomamos $\lm$ con:
                    \begin{equation*}
                        \frac{S_2^2}{S_1^2} \frac{\sigma_1^2}{\sigma_2^2} \rightsquigarrow F(n_2-1,n_1-1) = \lm \Longrightarrow \frac{\sigma_1^2}{\sigma_2^2} = \frac{S_1^2}{S_1^2}\lm
                    \end{equation*}
            \end{itemize}
            Por lo que el intervalo general a considerar será:
            \begin{equation*}
                \left]\frac{S_1^2}{S_2^2}\lm_1, \frac{S_1^2}{S_2^2}\lm_2\right[
            \end{equation*}
            Por lo que vimos en teoría, tendremos:
            \begin{equation*}
                \lm_1 = F_{n_2-1,n_1-1,1-\nicefrac{\alpha}{2}}, \qquad \lm_2 = F_{n_2-1,n_1-1,\nicefrac{\alpha}{2}}
            \end{equation*}
            Por lo que el intervalo a considerar será:
            \begin{equation*}
                \left]\frac{S_1^2}{S_2^2}F_{n_2-1,n_1-1,1-\nicefrac{\alpha}{2}}, \frac{S_1^2}{S_2^2}F_{n_2-1,n_1-1,\nicefrac{\alpha}{2}}\right[
            \end{equation*}
            Calculamos las cuasivarianzas:
            \begin{equation*}
                S_1^2 = \frac{\sum\limits_{i=1}^n{(x_i-\overline{x})}^{2}}{n_1-1} = \frac{(n-1)}{n}Var(X_1, \ldots, X_n)
            \end{equation*}
            Usando la calculadora, obtenemos:
            \begin{equation*}
                S_1^2 = 483.12 \qquad S_2^2 = 208.517
            \end{equation*}
            Y consultando la tabla de la $F$ de Snedecor:
            \begin{equation*}
                F_{n_2-1,n_1-1,0.975}= 0.107, \qquad F_{n_2-1,n_1-1,0.025} = 7.39
            \end{equation*}
            Por lo que el intervalo obtenido es:
            \begin{equation*}
                \left]0.2479, 17.1219\right[
            \end{equation*}
            Si lo hubiéramos hecho para $\frac{\sigma_2^2}{\sigma_1^2}$ habríamos obtenido:
            \begin{equation*}
                \left]0.0583, 4.0398\right[
            \end{equation*}
        \item Suponiendo que las varianzas son iguales, calcular un intervalo de confianza para la diferencia de las medias al nivel de confianza $0.9$, y una cota inferior de confianza al mismo nivel. Interpretar los resultados.

            Como el valor $1$ para $\frac{\sigma_1^2}{\sigma_2^2}$ está en los intervalos, no podemos descartar que las varianzas sean iguales, con lo que podemos asumirlo. En teoría vimos que el intervalo a obtener es:
            \begin{equation*}
                \left]\overline{X}-\overline{Y}-t_{n_1+n_2-2,\nicefrac{\alpha}{2}}S_p\sqrt{\frac{1}{n_1}+\frac{1}{n_2}},\overline{X}-\overline{Y}+t_{n_1+n_2-2,\nicefrac{\alpha}{2}}S_p\sqrt{\frac{1}{n_1}+\frac{1}{n_2}}\right[
            \end{equation*}
            Y la cota inferior es:
            \begin{equation*}
                \overline{X}-\overline{Y}-t_{n_1+n_2-2,\alpha}S_p\sqrt{\frac{1}{n_1}+\frac{1}{n_2}}
            \end{equation*}
            donde:
            \begin{equation*}
                S_p^2 = \frac{(n_1-1)S_1^2 + (n_2-1)S_2^2}{n_1+n_2-2}
            \end{equation*}
            Como tenemos los datos:
            \begin{itemize}
                \item $n_1 = 6$.
                \item $n_2 = 5$.
                \item $\alpha = 0.05$.
                \item $S_1^2 = 483.12$.
                \item $S_2^2 = 208.517$.
                \item $\overline{x} = 37.2$.
                \item $\overline{y} = 16.88$.
            \end{itemize}
            Podemos también consultar:
            \begin{itemize}
                \item $t_{n_1+n_2-2,\nicefrac{\alpha}{2}} = t_{9,0.025} = 2.2622$.
                \item $t_{n_1+n_2-2,\alpha} = t_{9,0.05} = 1.8331$.
            \end{itemize}
            Por lo que:
            \begin{equation*}
                S_p = \sqrt{\frac{5\cdot 483.12+4\cdot 16.88}{6+5-2}} = 16.610305
            \end{equation*}
            y el intervalo y la cota inferior son:
            \begin{align*}
                &\left]37.2-16.88-2.2622\cdot 16.610305\cdot \sqrt{\frac{1}{6}+\frac{1}{5}},37.2-16.88+\right.\\&\qquad \qquad \qquad \qquad \qquad \qquad \qquad \qquad \qquad \qquad \left. +2.2622\cdot 16.610305\cdot \sqrt{\frac{1}{6}+\frac{1}{5}} \right[ \\
                &=\left]-2.433296, 43.073269\right[
            \end{align*}
            Vemos que $0$ está en dicho intervalo, por lo que no podemos descartar que las dos medias sean iguales.
    \end{enumerate}
\end{ejercicio}

\begin{ejercicio}
    Sea $(X_1, \ldots, X_n)$ una muestra aleatoria simple de una variable aleatoria con distribución $U(0,\theta)$. Dado un nivel de confianza arbitrario, calcular el intervalo de confianza para $\theta$ de menor longitud media uniformemente basado en un estadistico suficiente.\\

    \noindent
    Si la muestra es de $X\rightsquigarrow U(0,\theta)$ y consideramos un nivel de confianza $1-\alpha$ para $\alpha \in \left]0,1\right[$, calculamos un estadístico suficiente para $\theta$:
    \begin{equation*}
        f_\theta^n(x_1, \ldots, x_n) \stackrel{\text{iid.}}{=} \prod_{i=1}^{n}f_\theta(x_i) = \prod_{i=1}^{n}\frac{1}{\theta} = \frac{1}{\theta^n} \qquad x_i \in \left]0,\theta\right[
    \end{equation*}
    Por lo que para $0<X_{(1)}<X_{(n)}<\theta$ tenemos:
    \begin{equation*}
        f_\theta^n(x_1, \ldots, x_n) = I_{\mathbb{R}^+}(X_{(1)}) I_{\mathbb{R}^-}(X_{(n)}-\theta)\frac{1}{\theta^n}
    \end{equation*}
    Tomando:
    \begin{gather*}
        h(X_1, \ldots, X_n) = I_{\mathbb{R}^+}(X_{(1)}), \qquad T(X_1, \ldots, X_n) = X_{(n)} \\
        g_\theta(t) = I_{\mathbb{R}^-}(t-\theta)\frac{1}{\theta^n}
    \end{gather*}
    Tenemos por el Teorema de factorización de Neymann-Fisher que el estadístico $T(X_1, \ldots, X_n) = X_{(n)}$ es suficiente para $\theta$. Calculamos su función de distribución:
    \begin{equation*}
        F_T(t) = {(F_X(t))}^{n} = {\left(\frac{t}{\theta}\right)}^{n} \qquad t\in [0,\theta]
    \end{equation*}
    Por lo que una función pivote para aplicar el método de la cantidad pivotal es:
    \begin{equation*}
        T(X_1, \ldots, X_n;\theta) = F_T(T) = {\left(\frac{X_{(n)}}{\theta}\right)}^{n} \rightsquigarrow U(0,1)
    \end{equation*}
    Tenemos que:
    \begin{itemize}
        \item La función es estrictamente decreciente en $\theta$.
        \item Si tenemos $\lm$ de forma que:
            \begin{equation*}
                \lm = {\left(\frac{X_{(n)}}{\theta}\right)}^{n} \Longrightarrow \theta = \frac{X_{(n)}}{\sqrt[n]{\lm}}
            \end{equation*}
    \end{itemize}
    Por lo que el intervalo de confianza para $\theta$ a nivel de confianza $1-\alpha$ será:
    \begin{equation*}
        \left]\frac{X_{(n)}}{\sqrt[n]{\lm_2}},\frac{X_{(n)}}{\sqrt[n]{\lm_1}} \right[
    \end{equation*}
    donde $\lm_1$ y $\lm_2$ verifican:
    \begin{equation*} % // TODO: A PARTIR DE AQUI
        1-\alpha\leq P_\theta[\lm_1<T<\lm_2] = \lm_2 - \lm_1
    \end{equation*}
    La longitud esperada del intervalo es:
    \begin{equation*}
        E[L] = E\left[\left(\frac{1}{\sqrt[n]{\lm_1}} - \frac{1}{\sqrt[n]{\lm_2}}\right) X_{(n)}\right] = \left(\frac{1}{\sqrt[n]{\lm_1}} - \frac{1}{\sqrt[n]{\lm_2}}\right)E[X_{(n)}]
    \end{equation*}
    Con $E[X_{(n)}]\geq 0$, por lo que será suficiente con minimizar el primer término. Por el método de los multiplicadores de Lagrange:
    \begin{equation*}
        F(\lm_1,\lm_2) = \left(\frac{1}{\sqrt[n]{\lm_1}} - \frac{1}{\sqrt[n]{\lm_2}}\right) - \lm\left(\lm_2-\lm_1 - (1-\alpha)\right)
    \end{equation*}
    Si derivamos e igualamos a cero despejando $\lm$:
    \begin{align*}
        \dfrac{\partial F}{\partial \lm_1} = \frac{-1}{n\sqrt[n]{\lm_1^{n+1}}} + \lm = 0 \quad &\Longrightarrow \quad  \lm = \frac{1}{n\sqrt[n]{\lm_1^{n+1}}} = \frac{1}{n\lm_1\sqrt[n]{\lm_1}} \\
        \dfrac{\partial F}{\partial \lm_2} = \frac{1}{n\sqrt[n]{\lm_2^{n+1}}} -\lm= 0 \quad &\Longrightarrow \quad \lm = \frac{1}{n\sqrt[n]{\lm_2^{n+1}}} = \frac{1}{n\lm_2\sqrt[n]{\lm_2}}
    \end{align*} % // TODO: TERMINAR, pedir a Irina
\end{ejercicio}

\begin{ejercicio}
    Utilizando la desigualdad de Chebychev, dar un intervalo de confianza para $p$ a nivel de confianza arbitrario, basado en una muestra de tamaño arbitrario de una variable aleatoria con distribución $B(1,p)$.\\

    \noindent
    Sea $(X_1, \ldots, X_n)$ una m.a.s. de $X\rightsquigarrow B(1,p)$, buscamos por el método de Chebyshev un intervalo de confianza para $p$ a nivel de confianza $1-\alpha$. Para ello, es necesario tener un estimador insesgado de $p$. Si consideramos:
    \begin{equation*}
        T(X_1, \ldots, X_n) = \overline{X}
    \end{equation*}
    tenemos que $\overline{X}\in [0,1]$, así como que $\overline{X}$ es insesgado, puesto que:
    \begin{equation*}
        E[\overline{X}] = E\left[\frac{1}{n}\sum_{i=1}^{n}X_i\right] = \frac{1}{n}\sum_{i=1}^{n}E[X_i] = \frac{1}{n}\sum_{i=1}^{n}p = \frac{np}{n} = p
    \end{equation*}
    Usando la reproductividad de la binomial vemos que:
    \begin{equation*}
        Var(\overline{X}) = Var\left(\frac{1}{n}\sum_{i=1}^{n}X_i\right) = \frac{1}{n^2}Var\left(\sum_{i=1}^{n}X_i\right) = \frac{np(1-p)}{n^2} = \frac{p(1-p)}{n}
    \end{equation*}
    Y si tomamos $c$ una constante de forma que:
    \begin{equation*} % // TODO: tomar c = 1/n o c = 1/4n
        \frac{p(1-p)}{n} \leq c \qquad \forall p\in \left]0,1\right[
    \end{equation*}
    tenemos entonces que el intervalo:
    \begin{equation*}
        \left]\overline{X}-\sqrt{\frac{c}{\alpha}},\overline{X}+\sqrt{\frac{c}{\alpha}}\right[
    \end{equation*}
    Es un intervalo de confianza para $p$ a nivel de confianza $1-\alpha$. % // TODO: dar la constante c?
\end{ejercicio}

\begin{ejercicio}
    Para una muestra de tamaño $n$ de una variable aleatoria con función de densidad
    \begin{equation*}
        f_\theta(x) = \frac{2x}{\theta^2}, \qquad 0<x<\theta
    \end{equation*}
    encontrar el intervalo de confianza para $\theta$ de menor longitud media uniformemente a nivel de confianza $1-\alpha$, basado en un estadístico suficiente.\\

    \noindent
    Buscamos un estadístico suficiente de $\theta$:
    \begin{equation*}
        f_\theta^n(X_1, \ldots, X_n) \stackrel{\text{iid.}}{=} \prod_{i=1}^{n}f_\theta(X_i) = \prod_{i=1}^{n}\frac{2X_i}{\theta^2} = \frac{2^n}{\theta^{2n}}\prod_{i=1}^{n}X_i \qquad 0<X_i<\theta
    \end{equation*}
    Por lo que tomando $0<X_{(1)}<X_{(n)}<\theta$ tenemos:
    \begin{equation*}
        f_\theta^n(X_1, \ldots, X_n) = \frac{2^n}{\theta^{2n}}I_{\mathbb{R}^+}(X_{(1)})I_{\mathbb{R} ^-}(X_{(n)}-\theta)\prod_{i=1}^n X_i
    \end{equation*}
    Si tomamos:
    \begin{gather*}
        h(X_1, \ldots, X_n) = I_{\mathbb{R}^+}(X_{(1)})\prod_{i=1}^n X_i, \qquad T(X_1, \ldots, X_n) = X_{(n)} \\
        g_\theta(t) = \frac{2^n}{\theta^{2n}}I_{\mathbb{R}^-}(t-\theta)
    \end{gather*}
    Tenemos por el Teorema de factorización de Neymann-Fisher que $X_{(n)}$ es un estadístico suficiente para $\theta$. Calculamos su función de distribución:
    \begin{equation*}
        F_T(t) = {(F_X(t))}^{n}
    \end{equation*}
    Y para ello vemos que  tenemos que calcular $F_X$:
    \begin{equation*}
        F_X(t) = \int_{0}^{t} \frac{2x}{\theta^2}~dx = \left[\frac{x^2}{\theta^2}\right]_0^t = \frac{t^2}{\theta^2}
    \end{equation*}
    Por lo que:
    \begin{equation*}
        F_T(t) = {(F_X(t))}^{n} = {\left(\frac{t}{\theta}\right)}^{2n}
    \end{equation*}
    La función pivote que consideramos para aplicar el método de la cantidad pivotal es:
    \begin{equation*}
        T(X_1, \ldots, X_n;\theta) = F_T(T) = {\left(\frac{X_{(n)}}{\theta}\right)}^{2n} \rightsquigarrow U(0,1)
    \end{equation*}
    Tenemos que:
    \begin{itemize}
        \item Es estrictamente decreciente respecto $\theta$.
        \item Si tenemos $\lm$ de forma que:
            \begin{equation*}
                \lm = {\left(\frac{X_{(n)}}{\theta}\right)}^{2n} \quad \Longrightarrow\quad  \theta = \frac{X_{(n)}}{\sqrt[2n]{\lm}}
            \end{equation*}
    \end{itemize}
    Tenemos por tanto que el intervalo es:
    \begin{equation*}
        \left]\frac{X_{(n)}}{\sqrt[2n]{\lm_2}}, \frac{X_{(n)}}{\sqrt[2n]{\lm_1}}\right[
    \end{equation*}
    donde $\lm_1$ y $\lm_2$ verifican:
    \begin{equation*}
        1-\alpha\leq P_\theta[\lm_1<T<\lm_2] = F_T(\lm_2) - F_T(\lm_1) = \lm_2 - \lm_1
    \end{equation*}
    La longitud esperada del intervalo es:
    \begin{equation*}
        E[L] = E\left[\left(\frac{1}{\sqrt[2n]{\lm_1}} - \frac{1}{\sqrt[2n]{\lm_2}}\right) X_{(n)}\right] = \left(\frac{1}{\sqrt[2n]{\lm_1}} - \frac{1}{\sqrt[2n]{\lm_2}}\right)E[X_{(n)}]
    \end{equation*}
    Con $E[X_{(n)}]\geq 0$, por lo que será suficiente con minimizar el primer término. Si despejamos $\lm_2$:
    \begin{equation*}
        \lm_2 = 1 - \alpha + \lm_1
    \end{equation*}
    Tenemos que la función a minimizar es:
    \begin{equation*}
        F(\lm_1) 0 \lm_1^\frac{-1}{2n} - {(1-\alpha+\lm_1)}^{\frac{-1}{2n}}
    \end{equation*}
    Calculamos su derivada e igualdando a cero:
    \begin{equation*}
        F'(\lm_1) = \frac{-1}{2n}\lm_1^{\frac{-1}{2n}-1} + \frac{1}{2n}{(1-\alpha+\lm_1)}^{\frac{-1}{2n}-1} = 0
    \end{equation*}
    Tenemos que $F'(\lm_1) = 0 \Longleftrightarrow \lm_1 = 1-\alpha+\lm_1 \Longleftrightarrow \alpha = 1$. Por lo que no podemos encontrar una solución, luego $F$ es estrictamente monótona:
    \begin{itemize}
        \item Supuesto que $F$ es estrictamente crecientes, tenemos entonces que $\lm_1 = 0$ y $\lm_2 = 1-\alpha$, por lo que el intervalo es:
            \begin{equation*}
                \left]\frac{X_{(n)}}{\sqrt[2n]{1-\alpha}},+\infty\right[
            \end{equation*}
        \item Supuesto que $F$ es estrictamente decreciente, tenemos que $\lm_2 = 1$ y $\lm_1 = \alpha$, por lo que el intervalo es:
            \begin{equation*}
                \left]X_{(n)}, \frac{X_{(n)}}{\sqrt[2n]{\alpha}}\right[
            \end{equation*}
    \end{itemize}
    Como en la segunda opción nos sale un intervalo acotado, este es el de menor longitud esperada, por lo que nos quedamos con:
    \begin{equation*}
        \left]X_{(n)}, \frac{X_{(n)}}{\sqrt[2n]{\alpha}}\right[
    \end{equation*}
\end{ejercicio}

\begin{ejercicio} 
    Para una muestra de tamaño $n$ de una variable aleatoria con función de densidad
    \begin{equation*}
        f_\theta(x) = \frac{\theta}{x^2}, \qquad x>\theta
    \end{equation*}
    encontrar el intervalo de confianza para $\theta$ de menor longitud media uniformemente a nivel de confianza $1 - \alpha$, basado en el estimador máximo verosímil de $\theta$.\\

    \noindent
    La función de verosimilitud es:
    \begin{equation*}
        L_{x_1,\ldots,x_n}(\theta) = \prod_{i=1}^{n} \frac{\theta}{x_i^2} \qquad \forall x_i > \theta > 0
    \end{equation*}
    Por lo que:
    \begin{equation*}
        L_{x_1,\ldots,x_n}(\theta) = \left\{\begin{array}{ll}
            \prod\limits_{i=1}^{n}\frac{\theta}{x_i^2} & \text{si\ } \theta < x_{(1)}  \\
             0 & \text{en otro caso} 
        \end{array}\right. 
    \end{equation*}
    Y vemos que $L_{x_1,\ldots,x_n}(\theta)$ es creciente, con lo que alcanza su máximo en $\theta = x_{(1)}$. En definitiva, el EMV es $\hat{\theta} = X_{(1)}$, por lo que buscamos un intervalo de confianza basado en $X_{(1)}$.\\

    \noindent
    La función de distribución del mínimo es:
    \begin{equation*}
        F_{X_{(1)}}(t) = 1-{(1-F_X(t))}^{n}
    \end{equation*}
    Por lo que calculamos:
    \begin{equation*}
        F_X(t) = \int_{\theta}^{t} \frac{\theta}{x^2}~dx  = \left[\frac{-\theta}{x}\right]_\theta^t = \frac{-\theta}{t} + 1 \qquad t>\theta
    \end{equation*}

    de donde:
    \begin{equation*}
        F_{X_{(1)}}(t) = 1-{(1-F_X(t))}^{n} = 1 - {\left(1-\left(1-\frac{\theta}{t}\right)\right)}^{n} = 1-{\left(\frac{\theta}{t}\right)}^{n} \qquad t> \theta
    \end{equation*}
    Tomamos por tanto como función pivote:
    \begin{equation*}
        T(X_1, \ldots, X_n;\theta) = 1-{\left(\frac{\theta}{X_{(1)}}\right)}^{n} \rightsquigarrow U(0,1)
    \end{equation*}
    Comprobamos las condiciones del método de la cantidad pivotal:
    \begin{itemize}
        \item Si derivamos:
            \begin{equation*}
                \dfrac{\partial T}{\partial \theta} = -n{\left(\frac{\theta}{X_{(1)}}\right)}^{n-1} \frac{1}{X_{(1)}} = \frac{-n\theta^{n-1}}{X_{(1)}^n} < 0 \qquad \forall \theta
            \end{equation*}
            Por lo que $T(X_1, \ldots, X_n;\theta)$ es estrictamente decreciente en función de $\theta$.
        \item Si tratamos de despejar el parámetro para cierto $\lm$:
            \begin{equation*}
                1-{\left(\frac{\theta}{X_{(1)}}\right)}^{n} = \lm \Longrightarrow \theta = \sqrt[n]{(1-\lm)X_{(1)}^n}  = X_{(1)}\sqrt[n]{1-\lm}
            \end{equation*}
    \end{itemize}
    Por lo que hemos obtenido como intervalo de confianza (donde los índices de $\lm$ son debidos a que es estrictamente decreciente):
    \begin{equation*}
        \left]X_{(1)}\sqrt[n]{1-\lm_2}, X_{(1)}\sqrt[n]{1-\lm_2}\right[
    \end{equation*}
    Tratamos ahora de minimizar la longitud del intervalo, que es:
    \begin{equation*}
        L = X_{(1)}\left(\sqrt[n]{1-\lm_1} - \sqrt[n]{1-\lm_2}\right)
    \end{equation*}

    de donde:
    \begin{equation*}
        E_\theta[L] = E_\theta[X_{(1)}]\left(\sqrt[n]{1-\lm_1} - \sqrt[n]{1-\lm_2}\right)
    \end{equation*}
    Y como $E_\theta(X_{(1)})$ es una constante positiva, podemos obviarla a la hora de minimizar, por lo que buscamos minimizar el segundo trozo, sujeto a la restricción:
    \begin{equation*}
        P_\theta[\lm_1 < T < \lm_2] = 1-\alpha
    \end{equation*}
    Y como tenemos:
    \begin{equation*}
        P[\lm_1 < T < \lm_2] = F_T(\lm_2) - F_T(\lm_1) = \lm_2 - \lm_1
    \end{equation*}
    Por lo que tenemos la restricción $\lm_2-\lm_1 = 1-\alpha$. Para meterla en la función a minimizar:
    \begin{itemize}
        \item Bien despejamos $\lm_1$ o $\lm_2$, sustituimos en la función a minimizar y obtenemos una función de una variable, que sabemos minimizar.
        \item Usamos los multiplicadores de Lagrange, sumamos a la función la restricción (igualdada a 0) multiplicada por un cierto parámetro, que llamaremos $\lm$, por lo que buscamos minimizar:
            \begin{equation*}
                {(1-\lm_1)}^{\nicefrac{1}{n}}- {(1-\lm_2)}^{\nicefrac{1}{n}} + \lm (\lm_2 - \lm_1 - 1+\alpha)
            \end{equation*}
    \end{itemize}
    Cuando las restricciones son tan sencillas el método de despejar y sustituir es más sencillo, por lo que optamos por dicho método:
    \begin{equation*}
        \lm_2 = 1-\alpha+\lm_1
    \end{equation*}
    Sustituyendo en la función a minimizar, buscamos la constante  $\lm_1$ que minimiza la expresión:
    \begin{equation*}
        f(\lm_1) = {(1-\lm_1)}^{\nicefrac{1}{n}}-{(1-(a-\alpha+\lm_1))}^{\nicefrac{1}{n}} = {(1-\lm_1)}^{\nicefrac{1}{n}} - {(\alpha-\lm_1)}^{\nicefrac{1}{n}}
    \end{equation*}
    Calculamos la derivada y la igualamos a $0$, para despejar $\lm_1$:
    \begin{equation*}
        f'(\lm_1) = \frac{1}{n}{(1-\lm_1)}^{\nicefrac{1}{n}-1}(-1) - \frac{1}{n}{(\alpha-\lm_1)}^{\nicefrac{1}{n}-1}(-1) = 0
    \end{equation*}
    Haciendo cálculos llegamos a que $f'(\lm_1)=0$ si y solo si $\alpha=1$, pero $\alpha\in \left]0,1\right[$, por lo que $f'(\lm_1)\neq 0$, con lo que $f$ es estrictamente monónotona. Sabemos también que $0<\lm_1<\lm_2<1$, buscamos razonar si $f$ es estrictamente creciente o estrictamente decreciente. Para ello, podemos estudiar el signo de la derivada o bien razonarlo de la siguiente forma:
    \begin{itemize}
        \item Si la función es estrictamente creciente, esta alcanza su mínimo en $\lm_1=0$, con lo que usando que $\lm_2-\lm_1 = 1-\alpha$ tenemos que $\lm_2 = 1-\alpha$. Sustituyendo, obtenemos el intervalo:
            \begin{equation*}
                \left]X_{(1)}\sqrt[n]{1-(1-\alpha)},X_{(1)}\sqrt[n]{1-0}\right[ = \left]X_{(1)}\sqrt[n]{\alpha},X_{(1)}\right[
            \end{equation*}
        \item Si la función es estrictamente decreciente, entonces esta alcana su mínimo cuando $\lm_2=1$, con lo que $\lm_1 = \alpha$, de donde obtenemos el intervalo:
            \begin{equation*}
                \left]X_{(1)}\sqrt[n]{0},X_{(1)}\sqrt[n]{1-\alpha}\right[ = \left]0,X_{(1)}\sqrt[n]{1-\alpha}\right[
            \end{equation*}
    \end{itemize}
    El segundo intervalo no está acotado por debajo (ya que el mínimo a considerar en el espacio paramétrico es $0$), pero en el primero sí que tenemos un valor acotado por debajo y otro por encima.  Preferimos elegir el primer intervalo, puesto que restringe más la longitud del intervalo (es decir, lo acota por ambos lados), aunque puede ocurrir que en función de los valores de $X_{(1)}$ y $\alpha$ el intervalo mínimo sea a veces el primero y otras veces el segundo. % // TODO: Quizas mirar esto
\end{ejercicio}

