\section{Estadísticos muestrales}

\begin{ejercicio}
    Sea $(X_1, \ldots, X_n)$ una muestra aleatoria simple de una variable aleatoria $X$. Dar el espacio muestral y calcular la función masa de probabilidad de $(X_1, \ldots, X_n)$ en cada uno de los siguientes casos:
    \begin{enumerate}[label=\alph*)]
        \item $X\rightsquigarrow \{B(k_0,p) : p\in (0,1)\}$ Binomial.

            El espacio muestral en este caso es $\cc{X}^n$, donde:
            \begin{equation*}
                \cc{X} = \{0, 1, ..., k_0\}
            \end{equation*}
            Recordamos que si $X\rightsquigarrow B(k_0,p)$, entonces:
            \begin{equation*}
                P[X=x] = \binom{k_0}{x} p^x {(1-p)}^{k_0-x} \qquad \forall x\in \cc{X}
            \end{equation*}
            Por tanto, para nuestra m.a.s. tendremos la función masa de probabilidad:
            \begin{align*}
                P[X_1 &= x_1, \ldots, X_n = x_n] \stackrel{\text{indep.}}{=} \prod_{i=1}^{n}P[X_i = x_i]\stackrel{\text{id. d.}}{=} \prod_{i=1}^{n} P[X=x_i] \\
                      &= \prod_{i=1}^{n} \binom{k_0}{x_i} p^{x_i} {(1-p)}^{k_0-x_i} = p^{\sum\limits_{i=1}^{n}x_i} {(1-p)}^{nk_0 - \sum\limits_{i=1}^{n}x_i} \prod_{i=1}^{n}\binom{k_0}{x_i} \\
                      & \qquad \forall (x_1, \ldots, x_n) \in \cc{X}^n
            \end{align*}
        \item $X\rightsquigarrow\{\cc{P}(\lm) : \lm \in \mathbb{R}^+\}$ Poisson.

            El espacio muestral de $X$ es:
            \begin{equation*}
                \cc{X} = \mathbb{N} \cup \{0\}
            \end{equation*} 
            Recordamos que si $X\rightsquigarrow \cc{P}(\lm)$, entonces:
            \begin{equation*}
                P[X=x] = e^{-\lm} \dfrac{\lm^x}{x!} \qquad \forall x\in \cc{X}
            \end{equation*}
            Por tanto:
            \begin{align*}
                P[X_1 &= x_1, \ldots, X_n = x_n] \stackrel{\text{indep.}}{=} \prod_{i=1}^{n}P[X_i = x_i]\stackrel{\text{id. d.}}{=} \prod_{i=1}^{n} P[X=x_i] \\
                      &= \prod_{i=1}^{n} e^{-\lm} \dfrac{\lm^{x_i}}{x_i!} = e^{-n\lm} \prod_{i=1}^{n} \dfrac{\lm^{x_i}}{x_i!} = e^{-n\lm} \cdot \dfrac{\lm^{\sum\limits_{i=1}^n x_i}}{\prod\limits_{i=1}^{n}x_i}  \qquad \forall (x_1, \ldots, x_n) \in \cc{X}^n
            \end{align*}
        \item $X\rightsquigarrow\{BN(k_0,p) : p\in (0,1)\}$ Binomial Negativa.

            El espacio muestral de $X$ es:
            \begin{equation*}
                \cc{X} = \mathbb{N}\cup \{0\}
            \end{equation*}
            Recordamos que si $X\rightsquigarrow BN(k_0,p)$, entonces:
            \begin{equation*}
                P[X=x] = \binom{x+k_0-1}{x} {(1-p)}^{x}p^{k_0} \qquad \forall x\in \cc{X}
            \end{equation*}
            Por tanto:
            \begin{align*}
                P[X_1 &= x_1, \ldots, X_n = x_n] \stackrel{\text{indep.}}{=} \prod_{i=1}^{n}P[X_i = x_i]\stackrel{\text{id. d.}}{=} \prod_{i=1}^{n} P[X=x_i] \\
                      &= \prod_{i=1}^{n} \binom{x_i+k_0-1}{x_i}{(1-p)}^{x_i}p^{k_0} = p^{nk_0}{(1-p)}^{\sum\limits_{i=1}^n x_i} \prod_{i=1}^{n}\binom{x_i+k_0-1}{x_i} \\
                      &\forall (x_1,\ldots,x_n)\in \cc{X}^n
            \end{align*}
        \item $X\rightsquigarrow\{G(p) : p\in (0,1)\}$ Geométrica.

            El espacio muestral de $X$ es:
            \begin{equation*}
                \cc{X} = \mathbb{N}\cup \{0\}
            \end{equation*}
            Recordamos que $G(p)\equiv BN(1,p)$, por lo que si sustituimos en la fórmula obtenida en la Binomial Negativa $k_0 = 1$:
            \begin{equation*}
                P[X_1=x_1, \ldots, X_n = x_n] = p^{n}{(1-p)}^{\sum\limits_{i=1}^n x_i} \qquad \forall (x_1,\ldots,x_n)\in \cc{X}^n
            \end{equation*}
        \item $X\rightsquigarrow\{P_N : N\in \mathbb{N}\}$, $\quad P_N(X=x) = \dfrac{1}{N}$, $\quad x=1,\ldots,N$.

            El espacio muestral ya nos lo dan: $\cc{X} = \{1,\ldots,N\}$. Calculemos la masa de probabilidad:
            \begin{align*}
                P[X_1 &= x_1, \ldots, X_n = x_n] \stackrel{\text{indep.}}{=} \prod_{i=1}^{n}P[X_i = x_i]\stackrel{\text{id. d.}}{=} \prod_{i=1}^{n} P[X=x_i] \\
                      &= \prod_{i=1}^{n} \dfrac{1}{N} = {\left(\dfrac{1}{N}\right)}^{n} \qquad \forall (x_1,\ldots,x_n) \in \cc{X}^n
            \end{align*}
    \end{enumerate}
\end{ejercicio}

\begin{ejercicio}
    Sea $(X_1, \ldots, X_n)$ una muestra aleatoria simple de una variable aleatoria $X$. Dar el espacio muestral y calcular la función de densidad de $(X_1, \ldots, X_n)$ en cada uno de los siguientes casos:
    \begin{enumerate}[label=\alph*)]
        \item $X\rightsquigarrow\{U(a,b) : a,b\in \mathbb{R}, a<b\}$ Uniforme.

            El espcio muestral en este caso es $\cc{X}^n$, donde:
            \begin{equation*}
                \cc{X} = [a,b]
            \end{equation*}
            Recordamos que si $X\rightsquigarrow U(a,b)$, entonces:
            \begin{equation*}
                f_X(x) = \dfrac{1}{b-a} \qquad \forall x\in [a,b]
            \end{equation*}
            Por lo que:
            \begin{align*}
                f_{(X_1, \ldots, X_n)}(x_1, \ldots, x_n) &\stackrel{\text{indep.}}{=} \prod_{i=1}^{n} f_{X_i}(x_i) \stackrel{\text{id. d.}}{=} \prod_{i=1}^{n} f_X(x_i) = \prod_{i=1}^{n} \dfrac{1}{b-a} \\ &= {\left(\dfrac{1}{b-a}\right)}^{n} \qquad \forall (x_1, \ldots, x_n) \in \cc{X}^n
            \end{align*}

        \item $X\rightsquigarrow\{\cc{N}(\mu, \sigma^2) : \mu \in \mathbb{R}, \sigma^2 \in \mathbb{R}^+\}$ Normal.

            El espacio muestral de $X$ es $\cc{X} = \mathbb{R}$. Recordamos que si $X\rightsquigarrow \cc{N}(\mu, \sigma^2)$, entonces:
            \begin{equation*}
                f_X(x) = \dfrac{1}{\sqrt{2\pi} \sigma} e^{-\dfrac{{(x-\mu)}^{2}}{2\sigma^2}} \qquad \forall x\in \mathbb{R}
            \end{equation*}
            Por lo que:
            \begin{align*}
                f_{(X_1, \ldots, X_n)}(x_1, \ldots, x_n) &\stackrel{\text{indep.}}{=} \prod_{i=1}^{n} f_{X_i}(x_i) \stackrel{\text{id. d.}}{=} \prod_{i=1}^{n} f_X(x_i) 
                = \prod_{i=1}^{n} \dfrac{1}{\sqrt{2\pi} \sigma} e^{-\dfrac{{(x_i-\mu)}^{2}}{2\sigma^2}}  \\
                &= {\left(\dfrac{1}{\sqrt{2\pi}\sigma}\right)}^{n} \prod_{i=1}^{n} e^{-\dfrac{{(x_i-\mu)}^{2}}{2\sigma^2}}  = {\left(\dfrac{1}{\sqrt{2\pi}\sigma}\right)}^{n} e^{-\sum\limits_{i=1}^n\dfrac{{(x_i-\mu)}^{2}}{2\sigma^2}}   \\
                &= {\left(\dfrac{1}{\sqrt{2\pi}\sigma}\right)}^{n} e^{\frac{-1}{2\sigma^2}\sum\limits_{i=1}^n {(x_i-\mu)}^{2}} \qquad \forall (x_1,\ldots,x_n) \in \mathbb{R}^n
            \end{align*}
        \item $X\rightsquigarrow\{\Gamma(p,a) : p,a\in \mathbb{R}^+\}$ Gamma.

            El espacio muestral de $X$ es $\cc{X}=\mathbb{R}^+_0$. Recordamos que si $X\rightsquigarrow \Gamma(p,a)$, entonces:
            \begin{equation*}
                f_X(x) = \dfrac{a^p}{\Gamma(p)} x^{p-1} e^{-ax} \qquad \forall x\in \mathbb{R}^+_0
            \end{equation*}
            Por lo que:
            \begin{align*}
                f_{(X_1, \ldots, X_n)}(x_1, \ldots, x_n) &\stackrel{\text{indep.}}{=} \prod_{i=1}^{n} f_{X_i}(x_i) \stackrel{\text{id. d.}}{=} \prod_{i=1}^{n} f_X(x_i) 
                = \prod_{i=1}^{n} \dfrac{a^p}{\Gamma(p)} x_i^{p-1} e^{-ax_i} \\
                                                         &= {\left(\dfrac{a^p}{\Gamma(p)}\right)}^{n} \cdot e^{-a\sum\limits_{i=1}^n x_i} \cdot \prod_{i=1}^{n} x_i^{p-1} \qquad \forall (x_1,\ldots,x_n)\in \cc{X}^n
            \end{align*}
        \item $X\rightsquigarrow\{\beta(p,q) : p,q\in \mathbb{R}^+\}$ Beta.

            El espacio muestral de $X$ es $\cc{X} = [0,1]$. Recordamos que si $X\rightsquigarrow \beta(p,q)$, entonces:
            \begin{equation*}
                f_X(x) = \dfrac{1}{\beta(p,q)} x^{p-1} {(1-x)}^{q-1} \qquad \forall x\in [0,1]
            \end{equation*}
            Donde:
            \begin{equation*}
                \beta(p,q) = \dfrac{\Gamma(p)\Gamma(q)}{\Gamma(p+q)}
            \end{equation*}
            Por tanto:
            \begin{align*}
                f_{(X_1, \ldots, X_n)}(x_1, \ldots, x_n) &\stackrel{\text{indep.}}{=} \prod_{i=1}^{n} f_{X_i}(x_i) \stackrel{\text{id. d.}}{=} \prod_{i=1}^{n} f_X(x_i) 
                = \prod_{i=1}^{n} \dfrac{1}{\beta(p,q)} x_i^{p-1} {(1-x_i)}^{q-1} \\
                                                         &= \dfrac{1}{{\beta(p,q)}^{n}} \prod_{i=1}^{n}x_i^{p-1} {(1-x_i)}^{q-1} \qquad \forall (x_1,\ldots,x_n) \in \cc{X}^n
            \end{align*}
        \item $X\rightsquigarrow\{P_\theta : \theta \in \mathbb{R}^+\}$, $\quad f_\theta(x) = \dfrac{1}{2\sqrt{x\theta}}$, $\quad 0<x<\theta$.

            Se nos dice que $\cc{X} = \left]0,\theta\right[$. Calculamos la función de densidad conjunta:
            \begin{align*}
                f_{(X_1, \ldots, X_n)}(x_1, \ldots, x_n) &\stackrel{\text{indep.}}{=} \prod_{i=1}^{n} f_{X_i}(x_i) \stackrel{\text{id. d.}}{=} \prod_{i=1}^{n} f_X(x_i) 
                = \prod_{i=1}^{n} \dfrac{1}{2\sqrt{x_i \theta}} \\
                                                         &= \dfrac{1}{{\left(2\sqrt{\theta}\right)}^{n}} \prod_{i=1}^{n} \dfrac{1}{\sqrt{x_i}} \qquad \forall (x_1, \ldots, x_n) \in  \cc{X}^n
            \end{align*}
    \end{enumerate}
\end{ejercicio}

\begin{ejercicio}
    Se miden los tiempos de sedimentación de una muestra de partículas flotando en un líquido. Los tiempos observados son: 
    \begin{gather*}
        11.5; 1.8; 7.3; 12.1; 1.8; 21.3; 7.3; 15.2; 7.3; 12.1; 15.2;\\ 7.3; 12.1; 1.8; 10.5; 15.2; 21.3; 10.5; 15.2; 11.5
    \end{gather*}
    \begin{itemize}
        \item Construir la función de distribución muestral asociada a a dichas observaciones.

            Si aplicamos la definición de función de distribución muestral obtenemos que esta viene dada por:
            \begin{equation*}
                F_n^\ast(x) =
                \begin{cases} 
                0 & \text{si\ } x < 1.8 \\[6pt]
                \nicefrac{3}{20} &\text{si\ } 1.8 \leq x < 7.3 \\[6pt]
                \nicefrac{7}{20} &\text{si\ } 7.3 \leq x < 10.5 \\[6pt]
                \nicefrac{9}{20} &\text{si\ } 10.5 \leq x < 11.5 \\[6pt]
                \nicefrac{11}{20} &\text{si\ } 11.5 \leq x < 12.1 \\[6pt]
                \nicefrac{14}{20} &\text{si\ } 12.1 \leq x < 15.2 \\[6pt]
                \nicefrac{18}{20} &\text{si\ } 15.2 \leq x < 21.3 \\[6pt]
                \nicefrac{20}{20} &\text{si\ } x \geq 21.3
                \end{cases}
            \end{equation*}

            \begin{figure}[H]
                \centering
                \begin{tikzpicture}
                \begin{axis}[
                    width=12cm, height=7cm, xlabel={$x$}, ylabel={$F_n^\ast(x)$}, ymin=0, ymax=1.1,
                    xmin=0, xmax=23, xtick={0,1.8,7.3,10.5,12.1,15.2,21.3,23},
                    ytick={0,0.1,0.2,0.3,0.4,0.5,0.6,0.7,0.8,0.9,1},
                    grid=both, domain=0:23, samples=200,
                ]

                % Graficamos la función escalonada
                \addplot[
                    thick, blue, const left
                ] coordinates {
                    (0,0) (1.8,0) (1.8,3/20) (7.3,3/20) (7.3,7/20) (10.5,7/20)
                    (10.5,9/20) (11.5,9/20) (11.5,11/20) (12.1,11/20) (12.1,14/20)
                    (15.2,14/20) (15.2,18/20) (21.3,18/20) (21.3,20/20) (23,20/20)
                };
                \end{axis}
                \end{tikzpicture}
                \caption{Gráfica de la función de distribución muestral.}
            \end{figure}
        \item Hallar los valores de los tres primeros momentos muestrales respecto al origen y respecto a la media.

            Calculamos primero los tres primeros momentos respecto al origen para luego calcular los centrados respecto a la media a partir de ellos:
            \begin{align*}
                a_1 &= \sum_{i=1}^{n} f_i x_i = 10.915 \qquad 
                a_2 = \sum_{i=1}^{n} f_i x_i^2 = 148.9325 \\
                a_3 &= \sum_{i=1}^{n} f_i x_i^3 = 2280.98365 \\
                b_1 &= \sum_{i=1}^{n} f_i {(x_i - \overline{x})}= 0\qquad  \\
                b_2 &= \frac{1}{n}\sum_{i=1}^{n}{(x_i-\overline{x})}^{2} = \frac{1}{n}\sum_{i=1}^{n}(x_i^2 - 2x_i \overline{x}+\overline{x}^2)  \\ &= \frac{1}{n}\sum_{i=1}^{n}x_i^2 - \frac{2\overline{x}}{n}\sum_{i=1}^{n}x_i + \overline{x}^2 = a_2 -2a_1^2 + a_1^2 = a_2 - a_1^2 \\
                    &= 148.9325 - 10.915  = 29.795275\\
                b_3 &= \frac{1}{n}\sum_{i=1}^{n} {(x_i-\overline{x})}^{3} = \frac{1}{n}\sum_{i=1}^{n}\left(x_i^3 - 3x_i^2 \overline{x} + 3x_i\overline{x}^2 -\overline{x}^3\right) \\
                    &= \frac{1}{n}\sum_{i=1}^{n}x_i^3 - \frac{3\overline{x}}{n}\sum_{i=1}^{n}x_i^2 + \frac{3\overline{x}^2}{n}\sum_{i=1}^{n}x_i - \overline{x}^3 = a_3 - 3a_1a_2 + 3a_1^3 - a_1^3 \\
                    &= a_3 - 3a_1a_2 + 2a_1^3 = 4.95455925
            \end{align*}
        \item Determinar los valores de los cuartiles muestrales y el percentil 70.

            Para ello, primero ordenamos los datos de menor a mayor y los agrupamos en grupos de $\nicefrac{20}{4} = 5$ en 5:
            \begin{gather*}
                1.8;\ 1.8;\ 1.8;\ 7.3;\ 7.3;\ \red{7.3;\ 7.3;\ 10.5;\ 10.5;\ 11.5};\ 11.5;\ 12.1;\ 12.1;\\ 12.1;\ 15.2;\ \red{15.2;\ 15.2;\ 15.2;\ 21.3;\ 21.3}
            \end{gather*}
            Como en los cambios de agrupaciones de números estos se repiten, hemos obtenido el valor de los cuartiles:
            \begin{equation*}
                q_1 = 7.3 \qquad q_2 = 11.5 \qquad q_3 = 15.2 \qquad q_4 = 21.3
            \end{equation*}
            Para el percentil $70$, calculamos:
            \begin{equation*}
                0.7\cdot 20 = 14
            \end{equation*}
            Como hemos obtenido un número entero, el percentil 70 será:
            \begin{equation*}
                c_{70} = \dfrac{X_{(14)} + X_{(15)}}{2} = \dfrac{12.1 + 15.2}{2} = 13.65
            \end{equation*}
            En el caso de haber obtenido un número no entero (por ejemplo, $14.2$), sería $X_{(15)}$.
    \end{itemize}
\end{ejercicio}

\begin{ejercicio}
    Se dispone de una muestra aleatoria simple de tamaño 40 de una distribución exponencial de media 3, ¿cuál es la probabilidad de que los valores de la función de distribución muestral y la teórica, en $x=1$, difieran menos de $0.01$? Aproximadamente, ¿cuál debe ser el tamaño muestral para que dicha probabilidad sea como mínimo $0.98$?
\end{ejercicio}

\begin{ejercicio}
    Se dispone de una muestra aleatoria simple de tamaño 50 de una distribución de Poisson de media 2, ¿cuál es la probabilidad de que los valores de la función de distribución muestral y la teórica, en $x=2$, difieran menos de $0.02$? Aproximadamente, ¿qué tamaño muestral hay que tomar para que dicha probabilidad sea como mínimo $0.99$?
\end{ejercicio}

\begin{ejercicio}
   Sea $X\rightsquigarrow B(1,p)$ y $(X_1, X_2, X_3)$ una muestra aleatoria simple de $X$. Calcular la función masa de probabilidad de los estadísticos $\overline{X}$, $S^2$, $\min X_i$ y $\max X_i$.
\end{ejercicio}

\begin{ejercicio}
    Obtener la función masa de probabilidad o función de densidad de $\overline{X}$ en el muestreo de una variable de Bernoulli, de una Poisson y de una exponencial.
\end{ejercicio}

\begin{ejercicio}
    Calcular las funciones de densidad de los estadísticos $\max X_i$ y $\min X_i$ en el muestreo de una variable $X$ con funcion de densidad:
    \begin{equation*}
        f_\theta(x) = e^{\theta-x}, \qquad x>\theta.
    \end{equation*}
\end{ejercicio}

\begin{ejercicio}
    El número de pacientes que visitan diariamente una determinada consulta médica es una variable aleatoria con varianza de 16 personas. Se supone que el número de visitas de cada día es independiente de cualquier otro. Si se observa el número de visitas diarias durante 64 días, calcular aproximadamente la probabilidad de que la media muestral no difiera en más de una persona del valor medio verdadero de visitas diarias.
\end{ejercicio}

\begin{ejercicio}
    Una máquina de refrescos está arreglada para que la cantidad de bebida que sirve sea una variable aleatoria con media 200 ml. y desviación típica 15 ml. Calcular de forma aproximada la probabilidad de que la cantidad media servida en una muestra aleatoria de tamaño 36 sea al menos 204 ml.
\end{ejercicio}
