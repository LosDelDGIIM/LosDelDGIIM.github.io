\newpage
\section{Estimación puntual. Insesgadez y mínima varianza}

\begin{ejercicio}
    Sea $(X_1, \ldots, X_n)$ una muestra de una variable $X\rightsquigarrow\cc{N}(\mu, \sigma^2)$ con $\mu\in \mathbb{R}$, $\sigma\in \mathbb{R}^+$. Probar que
    \begin{equation*}
        T(X_1, \ldots, X_n) = \left\{\begin{array}{ll}
            1 & \text{si\ } \overline{X}\leq 0 \\
            0 & \text{si\ } \overline{X} > 0
        \end{array}\right. 
    \end{equation*}
    es un estimador insesgado de la función paramétrica $\Phi\left(\frac{-\mu \sqrt{n}}{\sigma}\right)$, siendo $\Phi$ la función de distribución de la $\cc{N}(0,1)$.
\end{ejercicio}

\begin{ejercicio}
    Sea $(X_1, \ldots, X_n)$ una muestra aleatoria simple de $X\rightsquigarrow B(1,p)$ con $p\in \left]0,1\right[$ y sea $T=\sum\limits_{i=1}^{n}X_i$.
    \begin{enumerate}[label=\alph*)]
        \item Probar que si $k\in \mathbb{N}$ y $k\leq n$, el estadístico
            \begin{equation*}
                \dfrac{T(T_1)\cdot \ldots\cdot (T-k+1)}{n(n-1)\cdot \ldots\cdot (n-k+1)}
            \end{equation*}
            es un estimador insesgado de $p^k$. ¿Es este estimador el UMVUE?.
        \item Probar que si $k>n$, no existe ningún estimador insesgado para $p^k$.
        \item ¿Puede afirmarse que $\frac{T}{n}{(1-\frac{T}{n})}^{2}$ es insesgado para $p{(1-p)}^{2}$? 
    \end{enumerate}
\end{ejercicio}

\begin{ejercicio}
    Sea $(X_1, \ldots, X_n)$ una muestra aleatoria simple de una variable $X\rightsquigarrow \cc{P}(\lm)$ con $\lm\in \mathbb{R}^+$. Encontrar, si existe, el UMVUE para $\lm^s$, siendo $s\in \mathbb{N}$ arbitrario.
\end{ejercicio}

\begin{ejercicio}
    Sea $(X_1, \ldots, X_n)$ una muestra aleatoria simple de una variable con distribución uniforme discreta en los puntos $\{1,\ldots,N\}$, siendo $N$ un número natural arbitrario. Encontrar el UMVUE para $N$.
\end{ejercicio}

\begin{ejercicio}
    Sea $(X_1, \ldots, X_n)$ una muestra aleatoria simple de una variable aleatoria $X$ cuya función de densidad es de la forma
    \begin{equation*}
        f_\theta(x) = \dfrac{1}{2\sqrt{x\theta}}, \quad 0<x<\theta
    \end{equation*}
    Calcular, si existe, el UMVUE para $\theta$.
\end{ejercicio}

\begin{ejercicio}
    Sea $(X_1, \ldots, X_n)$ una muestra aleatoria simple de una variable aleatoria $X$ con función de densidad
    \begin{equation*}
        f_\theta(x) = \dfrac{\theta}{x^2}, \quad x>\theta
    \end{equation*}
    Calcular, si existen, los UMVUE para $\theta$ y para $\nicefrac{1}{\theta}$.
\end{ejercicio}

\begin{ejercicio}
    Sea $X\rightsquigarrow P_\theta$ siendo $P_\theta$ una distribución con función de densidad
    \begin{equation*}
        f_\theta(x) = e^{\theta-x}, \quad x\geq \theta
    \end{equation*}
    Dada una muestra aleatoria simple de tamaño arbitrario, encontrar los UMVUE de $\theta$ y de $e^{\theta}$.
\end{ejercicio}

\begin{ejercicio}
    Sea $X$ la variable que describe el número de fracasos antes del primer éxito en una sucesión de pruebas de Bernoulli con probabilidad de éxito $\theta\in \left]0,1\right[$, y sea $(X_1, \ldots, X_n)$ una muestra aleatoria simple de $X$.
    \begin{enumerate}[label=\alph*)]
        \item Probar que la familia de distribuciones de $X$ es regular y calcular la función de infor- mación asociada a la muestra.
        \item Especificar la clase de funciones paramétricas que admiten estimadores eficientes y los correspondientes estimadores.
        \item Calcular la varianza de cada estimador eficiente y comprobar que coincide con las correspondiente cota de Fréchet-Cramér-Rao.
        \item Calcular, si existen, los UMVUE para $P_\theta[X=0]$ y para $E_\theta[X]$ y decir si son eficientes.
    \end{enumerate}
\end{ejercicio}

\begin{ejercicio}
    Sea $(X_1, \ldots, X_n)$ una muestra aleatoria simple de una variable aleatoria $X$ con distribución exponencial.
    \begin{enumerate}[label=\alph*)]
        \item Probar que la familia de distribuciones de $X$ es regular.
        \item Encontrar la clase de funciones paramétricas que admiten estimador eficiente y el estimador correspondiente. Calcular la varianza de estos estimadores.
        \item Basándose en el apartado anterior, encontrar el UMVUE para la media de $X$.
        \item Dar la cota de Fréchet-Cramér-Rao para la varianza de estimadores insesgados y regulares de $\lm^3$. ¿Es alcanzable dicha cota?
    \end{enumerate}
\end{ejercicio}

\begin{ejercicio}
    Sea $X$ una variable aleatoria con función de densidad de la forma
    \begin{equation*}
        f_\theta(x) = \theta x^{\theta-1}, \quad 0<x<1
    \end{equation*}
    \begin{enumerate}[label=\alph*)]
        \item Sabiendo que $E_\theta[\ln X] = -\frac{1}{\theta}$ y $Var_\theta[\ln X] = \frac{1}{\theta^2}$, comprobar que esta familia de distribuciones es regular.
        \item Basándose en una muestra aleatoria simple de $X$, dar la clase de funciones paramétricas con estimador eficiente, los estimadores y su varianza.
    \end{enumerate}
\end{ejercicio}
