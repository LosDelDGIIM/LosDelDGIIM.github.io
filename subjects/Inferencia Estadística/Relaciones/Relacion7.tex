\newpage
\section{Contraste de hipótesis}

\begin{ejercicio}
    Se toma una observación de una variable con distribución de Poisson para contrastar que la media vale 1 frente a que vale 2.
    \begin{enumerate}[label=\alph*)]
        \item Construir un test no aleatorizado con nivel de significación $0.05$ para el contraste planteado. Calcular las probabilidades de cometer error de tipo 1 y de tipo 2, el tamaño y la potencia del test frente a la hipótesis alternativa.
        \item ¿Cómo debe aleatorizarse el test para alcanzar el tamaño $0.05$? ¿Cuál es la potencia de este test?
    \end{enumerate}
    
    \textbf{Solución}
    \begin{enumerate}[label=\alph*)]
        \item Sea $X\rightsquigarrow \cc{P}(\lm)$, queremos resolver el contraste:
            \begin{equation*}
                \left\{\begin{array}{l}
                    H_0 : \lm = 1 \\
                    H_1 : \lm = 2
                \end{array}\right.
            \end{equation*}
            Trabajaremos por tanto con:
            \begin{equation*}
                \Theta_0 = \{1\}, \qquad \Theta_1 = \{2\}, \qquad \Theta = \{1,2\}
            \end{equation*}
            y escribiremos $\lm_0 = 1$, $\lm_1 = 2$. Pensamos la forma que ha de tener el test: el espacio muestral de $X$ es $\mathbb{R}^+_0$:
            \begin{figure}[H]
                \centering
                \begin{tikzpicture}
                % Dibuja la recta real
                \draw[->] (-1,0) -- (3,0) node[right] {};

                % Marca los puntos 1 y 2
                \foreach \x in {0,1,2} {
                    \draw (\x,0.1) -- (\x,-0.1) node[below] {$\x$};
                }
                \end{tikzpicture}
            \end{figure}
            Es claro que si la observación es mayor o igual que 2 hemos de rechazar la hipótesis nula, así como que si la observación está entre 0 o 1 no podemos rechazar la hipótesis nula. No está claro lo que haremos en el intervalo $\left]1,2\right[$, por lo que estableceremos un cierto punto $c$ que delimite la región crítica del test, y calcularemos el valor de $c$ imponiendo nivel de significancia $\alpha$ ($\alpha = 0.05$). Por tanto, el test será de la forma:
            \begin{equation*}
                \varphi(X) = \left\{\begin{array}{ll}
                    1 & \text{si\ } X> c \\
                    0 & \text{si\ } X \leq c
                \end{array}\right. 
            \end{equation*}
            Imponemos nivel de significancia $\alpha$:
            \begin{equation*}
                \alpha \geq \sup_{\lm \in \Theta_0}\beta_\varphi(\lm) = \sup_{\lm\in \{\lm_0\}} \beta_\varphi(\lm) = \beta_\varphi(\lm_0) = E_{\lm_0}[\varphi]  = P_{\lm_0}[X>c] 
            \end{equation*}
            Como $c\in [1,2]$ y la distribución es discreta, ha de ser $c\in \{1,2\}$, por lo que miramos qué valor nos viene mejor:
            \begin{equation*}
                0.05 = \alpha \geq P_{\lm_0}[X>c] = 1-P_{\lm_0}[X\leq c] = 1-P_{1}[X\leq c]
            \end{equation*}
            \begin{itemize}
                \item Si $c = 1$, tenemos:
                    \begin{equation*}
                        P_1[X\leq 1] = P_1[X=0] + P_1[X=1] = 0.3679 + 0.3679 = 0.7358
                    \end{equation*}
                    Por lo que:
                    \begin{equation*}
                        1-P_1[X\leq 1] = 1- 0.7358 = 0.2642
                    \end{equation*}
                \item Si $c=2$, tenemos:
                    \begin{align*}
                        P_1[X\leq 2] &= P_1[X=0] + P_1[X=1] + P_1[X=2] \\ &= 0.3679 + 0.3679  + 0.1839 = 0.9197
                    \end{align*}
                    Por lo que:
                    \begin{equation*}
                        1 - P_1[X\leq 2] = 1-0.9197 = 0.0803
                    \end{equation*}
                    % // TODO: SEGUIR
            \end{itemize}
        \item 
    \end{enumerate}
\end{ejercicio}

\begin{ejercicio}
    Una urna contiene 10 bolas, blancas y negras. Para contrastar que el número de bolas blancas es 5 frente a que dicho número es 6 o 7, se extraen tres bolas con reemplazamiento y se rechaza $H_0$ sólo si se obtienen 2 o 3 bolas blancas. Calcular el tamaño de este test y la potencia frente a las alternativas.\\

    \noindent % // TODO: Plantear constraste de hipotesis
    Se quiere resolver el contraste de hipótesis:
    \begin{equation*}
        \left\{\begin{array}{l}
            H_0 : p = p_0 =  \nicefrac{5}{10} \\
            H_1 : p \in \{\nicefrac{6}{10}, \nicefrac{7}{10}\}
        \end{array}\right.
    \end{equation*}
    Por lo que consideraremos:
    \begin{equation*}
        \Theta_0 = \{\nicefrac{5}{10}\}, \qquad \Theta_1 = \{\nicefrac{6}{10}, \nicefrac{7}{10}\}, \qquad \Theta = \{\nicefrac{5}{10}, \nicefrac{6}{10}, \nicefrac{7}{10}\}
    \end{equation*}
    Sea:
    \begin{equation*}
        X\equiv \text{``Número de bolas blancas en una extracción''} \rightsquigarrow B(1,p)
    \end{equation*}
    Y tenemos $(X_1, X_2, X_3)$ una m.a.s. de $X$. Planteamos el test:
    \begin{equation*}
        \varphi(X_1, X_2, X_3) = \left\{\begin{array}{ll}
            1 & \text{si\ } X_1+X_2+X_3 \in \{2,3\} \\
            0 & \text{en otro caso\ } 
        \end{array}\right. 
    \end{equation*}
    Por la reproductividad de la binomial tenemos que $X_1+X_2+X_3\rightsquigarrow B(3,p)$. Calculamos el tamaño del test:
    \begin{align*}
        \sup_{p\in \Theta_0} \beta_\varphi(p) &= \beta_\varphi(p_0) = E_{p_0}[\varphi] = P_{p_0}[X_1+X_2+X_3 = 2] + P_{p_0}[X_1 + X_2 + X_3 = 3] \\
                                              &= P_{\nicefrac{1}{2}}[X_1+X_2+X_3 = 2] + P_{\nicefrac{1}{2}}[X_1 + X_2 + X_3 = 3]  \\
                                              &= 0.375 + 0.125 = 0.5
    \end{align*}
    Y ahora su potencia frente las alternativas:
    \begin{equation*}
        \sup_{p\in \Theta_1}\beta_\varphi(p) = \sup_{p \in \{\nicefrac{6}{10},\nicefrac{7}{10}\}} \beta_\varphi(p)
    \end{equation*}
    Calculamos cada una de ellas:
    \begin{itemize}
        \item Para $p = \nicefrac{6}{10}$:
            \begin{align*}
                \beta_\varphi(p) &= E_p[\varphi] = P_p[X_1 + X_2 + X_3 = 2] + P_p[X_1+X_2+X_3] % // TODO: TERMINAR, p = 0.6
            \end{align*}
        \item 
    \end{itemize}
\end{ejercicio}

\begin{ejercicio}
    Sea $(X_1, \ldots, X_n)$ una muestra aleatoria simple de una variable aleatoria con distribución de Poisson de parámetro $\lm$. Encontrar el test más potente de tamaño $\alpha$ para resolver el problema de contraste
    \begin{align*}
        H_0 &: \lm = \lm_0 \\
        H_1 &: \lm = \lm_1
    \end{align*}
    \textit{Aplicación:} En una centralita telefónica el número de llamadas por minuto sigue una distribución de Poisson. Si en cinco minutos se han recibido 12 llamadas, ¿puede aceptarse que el número medio de llamadas por minuto es $1.5$, frente a que dicho número es 2, al nivel de significación $0.05$? Calcular la potencia del test obtenido.
\end{ejercicio}

\begin{ejercicio}
    Dada una muestra de tamaño $n$ de una variable con distribución $\cc{N}(\mu, \sigma_0^2)$, deducir el test más potente de tamaño arbitrario para contrastar hipótesis simples sobre $\mu$.
\end{ejercicio}

\begin{ejercicio}
    Dada una muestra de tamaño $n$ de una variable aleatoria con distribución $U(-\theta,\theta)$, deducir el test más potente de tamaño $\alpha$ para contrastar $H_0 : \theta = \theta_0$ frente a $H_1 : \theta = \theta_1$ y calcular su potencia. ¿Cuál es el test óptimo fijado un nivel de significación arbitrario?
\end{ejercicio}

\begin{ejercicio}
    Deducir el test más potente de tamaño arbitrario para contrastar $h_0 : \theta = \theta_0$ frente a $H_1 : \theta = \theta_1$, basándose en una muestra de tamaño $n$ de una variable aleatoria con función de densidad
    \begin{equation*}
        f_\theta(x) = \frac{\theta}{x^2}, \qquad x>\theta
    \end{equation*}
    Deducir el test óptimo para un nivel de significación arbitrario.
\end{ejercicio}

\begin{ejercicio}
    Construir el test de Neyman-Pearson de tamaño $\alpha$ para contrastar $H_0:\theta = \theta_0$ frente a $H_1 : \theta = \theta_1$, basándose en una muestra de tamaño $n$ de una variable aleatoria con función de densidad
    \begin{equation*}
        f_\theta(x) = \frac{1}{x\ln \theta}, \qquad 1<x<\theta
    \end{equation*}
    Deducir el test óptimo para un nivel de significación arbitrario.
\end{ejercicio}

\begin{ejercicio}
    Sea $X$ una observación de una variable aleatoria con función de densidad
    \begin{equation*}
        f_\theta(x) = \frac{1}{\theta} e^{\nicefrac{-x}{\theta}}, \qquad x>\theta
    \end{equation*}
    Construir el test de razón de verosimilitudes de tamaño $\alpha$ arbitrario para contrastar
    \begin{align*}
        H_0 &: \theta = \theta_0 \\
        H_1 &: \theta = \theta_1
    \end{align*}
\end{ejercicio}

\begin{ejercicio}
    En base a una observación de $X\rightsquigarrow \{B(n,p) : p\in \left]0,1\right[\}$, deducir el test de razón de verosimilitudes para contrastar la hipótesis de que el parámetro $p$ no supera un determinado valor, $p_0$.
\end{ejercicio}

\begin{ejercicio}
    Sea $X$ una variable con función de densidad
    \begin{equation*}
        f_\theta(x) = \theta x^{\theta-1}, \qquad 0<x<1
    \end{equation*}
    Basándose en una observación de $X$, deducir el test de razón de verosimilitudes de tamaño arbitrario para contrastar
    \begin{align*}
        H_0 &: \theta \leq \theta_0 \\
        H_1 &: \theta > \theta_1
    \end{align*}
\end{ejercicio}

\begin{ejercicio}
    Un fabricante de coches asegura que la distancia media recorrida con un galón de gasolina es al menos 30 millas. Probados 9 coches de esta fábrica, la distancia media recorrida con un galón de gasolina ha sido 26 millas, y la suma de los cuadrados 6106 millas al cuadrado.
    \begin{enumerate}[label=\alph*)]
        \item Suponiendo que la distancia recorrida por estos coches con un galón de gasolina tiene distribución normal, contrastar la hipótesis del fabricante a partir de estos datos, a nivel de significación $0.01$.
        \item ¿Qué conclusión se obtendría de estos mismos datos, al mismo nivel de significación, si se sabe que la desviación típica de la variable considerada es $5.5$?
    \end{enumerate}
\end{ejercicio}

\begin{ejercicio}
    Un fabricante de baterías asegura que la desviación típica del tiempo de vida de las mismas es, a lo sumo, 70 horas. Una muestra de 26 baterías tomadas al azar ha dado una cuasidesviación típica de 84 horas. Haciendo las hipótesis adecuadas de normalidad, ¿proporcionan los datos evidencia para rechazar la hipótesis del fabricante al nivel $0.02$?
\end{ejercicio}

\begin{ejercicio}
    Un profesor asegura que tiene un nuevo método de enseñanza mejor que el usado tradicionalmente. Para comprobar si tiene razón se selecciona de forma aleatoria e independiente dos grupos de alumnos, $A$ y $B$, utilizándose el nuevo método con el grupo $A$ y el tradicional con el $B$. $A$ final de curso se hace un examen a los alumnos, obteniéndose las siguientes puntuaciones:\newline
    Grupo $A$: 6, 5, 4, 7, 3, $5.5$, 6, 7, 6.\newline
    Grupo $B$: 5, 4, 5, 6, 4, 6, 5, 3, 7.\newline
    Supuesto que las puntuaciones de cada grupo tienen distribución normal, ¿proporcionan estos datos evidencia para rechazar el nuevo método a nivel de significación $0.05$?
\end{ejercicio}

\begin{ejercicio}
    A partir de las siguientes observaciones de muestras independientes de dos poblaciones normales, contrastar, al nivel de significación $0.01$, si la media de la primera población supera en al menos una unidad la media de la segunda.\newline
    Muestra 1: 132, 139, 126, 114, 122, 132, 141, 126.\newline
    Muestra 2: 124, 141, 118, 116, 114, 132, 145, 123, 121.
\end{ejercicio}

\begin{ejercicio}
    Una central lechera recibe diariamente leche de dos granjas $A$ y $B$. Para comparar la calidad de los productos recibidos se ha medido el contenido en grasa en muestras de leche tomadas al azar de cada una de las granjas, con los siguientes resultados:
    \begin{table}[H]
    \centering
    \begin{tabular}{c|cccccc}
     & \multicolumn{6}{c}{Contenido en grasa $(\%)$} \\
     \hline
        Granja A & 14 & 12 & 15 & 15 & 11 & 16 \\
        Granja B & 20 & 18 & 18 & 19 & 15 &
    \end{tabular}
    \end{table}
    \begin{enumerate}[label=\alph*)]
        \item ¿Puede suponerse, a nivel de significación $0.05$, que el contenido medio en grasa de la leche de las dos granjas es el mismo? Especificar las hipótesis bajo las que se resuelve este problema.
        \item Calcular un intervalo de confianza, a nivel de confianza $0.9$, para la varianza del contenido en grasa de la leche de la granja B. A partir de dicho intervalo, deducir si puede aceptarse que la varianza de esta población es igual a 3. Especificar el problema de contraste y el test utilizado; calcular su tamaño.
    \end{enumerate}
\end{ejercicio}
