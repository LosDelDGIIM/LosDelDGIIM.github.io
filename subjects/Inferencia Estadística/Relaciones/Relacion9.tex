\newpage
\section{Constrastes de hipótesis no paramétricos}

\begin{ejercicio}
    A partir de los siguientes datos, que muestran el número de accidentes en un determinado regimiento del ejército durante 200 días elegidos al azar, contrastar si el número de accidentes diarios sigue una distribución de Poisson de parámetro 2.    
    \begin{table}[H]
    \centering
    \begin{tabular}{c|cccccccc}
        \hline
        Nº de accidentes & 0 & 1 & 2 & 3 & 4 & 5 & 6 & 7 \\
        \hline
        Nº de días & 22 & 53 & 58 & 39 & 20 & 5 & 2 & 1 \\
        \hline
    \end{tabular}
    \end{table}~\\

    \noindent
    Sea la variable aleatoria:
    \begin{equation*}
        X \equiv \text{``Número de accidentes en un día.''}
    \end{equation*}
    que se distribuye según una cierta función de distribución $F$, se plantea el contraste de hipótesis:
    \begin{equation*}
        \left\{\begin{array}{l}
                H_0 : F = F_{\cc{P}(2)}\\
                H_1 : F\neq F_{\cc{P}(2)}
        \end{array}\right.
    \end{equation*}
    En teoría hemos visto cómo resolver este contraste usando dos tests distintos:
    \begin{itemize}
        \item Test $\chi^2$ de Pearson.
        \item Test de Kolmogorov-Smirnov.
    \end{itemize}
    El segundo solo se puede aplicar para distribuciones continuas, por lo que no podemos usarlo en este caso. Usaremos por tanto el test $\chi^2$ de Pearson, que usa el estadístico:
        \begin{equation*}
            \chi^2(N_0,\ldots,N_k) = \sum_{i=0}^{k} \frac{{(N_i - np_i^0)}^{2}}{np_i^0} = -n + \sum_{i=0}^{k} \frac{N_i^2}{np_i^0} 
        \end{equation*}
        donde tenemos $n=200$, $k=8$ y los $N_i$ dados por la tabla anterior. Calculamos cada una de las $p_i^0 = P_{H_0}[X=i]$:
        \begin{table}[H]
        \centering
        \begin{tabular}{cccccccc}
            $p_0^0$ & $p_1^0$ & $p_2^0$ & $p_3^0$ & $p_4^0$ & $p_5^0$ & $p_6^0$ & $p_7^0$ \\
            \hline
            $0.1353$ & $0.2707$ & $0.2707$ & $0.1804$ & $0.0902$ & $0.0361$ & $0.0120$ & $0.0034$ \\
        \end{tabular}
        \end{table}
        \noindent
        de donde:
        \begin{equation*}
            \chi^2_{\text{exp}} = -n + \sum_{i=0}^{k} \frac{N_i^2}{np_i^0} = -n + \frac{1}{n}\sum_{i=0}^{k} \frac{N_i^2}{p_i^0} \approx 2.834515
        \end{equation*}
        Si calculamos ahora el $p-$valor obtenemos que:
        \begin{equation*}
            P[\chi^2(k-1)\geq \chi^2_{\text{exp}}] = P[\chi^2(7)\geq 2.834515] \approx P[\chi^2(7) \geq 2.8331] = 0.9
        \end{equation*}
        Por lo que no podemos rechazar $H_0$.
\end{ejercicio}

\begin{ejercicio}
   Una tela cuadrada tiene 60 defectos de fabricación. Con objeto de analizar la distribución de los defectos en la superficie de la tela, se ha dividido ésta en 9 zonas cuadradas exactamente iguales, observándose los siguientes defectos en cada zona: 
   \begin{table}[H]
   \centering
   \begin{tabular}{|c|c|c|}
       \hline
       8 & 7 & 3 \\
       \hline
       5 & 9 & 11 \\
       \hline
       6 & 4 & 7 \\
       \hline
   \end{tabular}
   \end{table}
   \noindent
   Contrastar, a partir de estos datos, si los defectos se distribuyen uniformemente en toda la superficie o, por el contrario, siguen algún patrón de ocurrencia.\\

   \noindent
   Sea la variable aleatoria:
   \begin{equation*}
       X\equiv \text{``Región en la que se localiza un defecto.''} 
   \end{equation*}
   que se distribuye según una función de distribución $F$, planteamos el contraste:
   \begin{equation*}
       \left\{\begin{array}{l}
           H_0: F=F_{U(1,2,\ldots,9)} \\
           H_1: F\neq F_{U(1,2,\ldots,9)}
       \end{array}\right.
   \end{equation*}
   Usaremos el test $\chi^2$ de Pearson, pues la distribución es discreta. Para ello, calculamos primero $p_i^0 = P_{H_0}[X = i]$, obteniendo en este caso que:
   \begin{equation*}
       p_i^0 = \frac{1}{9} \qquad \forall i \in \{1,\ldots,9\}
   \end{equation*}
   Por lo que ($k = 9$, $n=60$):
   \begin{equation*}
       \chi^2_{\text{exp}} = -n + \sum_{i=1}^{k} \frac{N_i^2}{np_i^0} = -n + \frac{1}{np_1^0}\sum_{i=1}^{k} N_i^2 = 7.5
   \end{equation*}
   Calculamos ahora el valor del $p-$valor:
   \begin{equation*}
       P_{H_0}[\chi^2(k-1)\geq \chi^2_{\text{exp}}] = P[\chi^2(8)\geq 7.5] \approx 0.475
   \end{equation*}
   Por lo que los datos no aportan evidencia suficiente para rechazar la hipótesis nula.
\end{ejercicio}

\begin{ejercicio}
    Un modelo genético indica que la distribución de una población de hombres y mujeres, daltónicos o no, se ajusta a probabilidades de la forma:
    \begin{table}[H]
    \centering
    \begin{tabular}{|c|c|c|}
        \hline
        & Hombres & Mujeres \\
        \hline
        No daltónicos & $\nicefrac{(1-p)}{2}$ & $\nicefrac{(1-p^2)}{2}$ \\
        \hline
        Daltónicos & $\nicefrac{p}{2}$ & $\nicefrac{p^2}{2}$ \\
        \hline
    \end{tabular}
    \end{table}
   \noindent
    Para comprobar esta teoría se examinaron 2000 individuos de la población, elegidos al azar, obteniéndose los siguientes resultados:
    \begin{table}[H]
    \centering
    \begin{tabular}{|c|c|c|}
        \hline
        & Hombres & Mujeres \\
        \hline
        No daltónicos & 894 & 1015 \\
        \hline
        Daltónicos & 81 & 10 \\
        \hline
    \end{tabular}
    \end{table}
   \noindent
    Contrastar, mediante el test de la $\chi^2$, si esta muestra concuerda con el modelo teórico.
\end{ejercicio}

\begin{ejercicio}
    Un laboratorio farmacéutico afirma que uno de sus productos confiere inmunidad a la picadura de insectos durante un tiempo exponencial de media $2.5$ horas. Probado el producto en 20 sujetos, en un ambiente con gran número de mosquitos, los instantes (en horas) en que recibieron la primera picadura fueron:
    \begin{table}[H]
    \centering
    \begin{tabular}{cccccccccc}
        $0.01$ & $0.02$ & $0.03$ & $0.23$ & $0.51$ & $0.74$ & $0.96$ & $1.17$ & $1.46$ & $1.62$ \\
        $2.18$ & $2.25$ & $2.79$ & $3.45$ & $3.82$ & $3.92$ & $4.27$ & $5.43$ & $5.79$ & $6.34$
    \end{tabular}
    \end{table}
    \noindent
    Usando el test de Kolmogorov-Smirnov, contrastar, a partir de estos datos, si puede aceptarse la afirmación del laboratorio.\\

    \noindent
    Sea la variable aleatoria:
    \begin{equation*}
        X\equiv \text{``Tiempo (en horas) en recibir la primera picadura''}
    \end{equation*}
    que se distribuye según una función de distribución $F$, planteamos el contraste:
    \begin{equation*}
        \left\{\begin{array}{l}
            H_0:F=F_0 \\
            H_1:F\neq F_0
        \end{array}\right.
    \end{equation*}
    donde $F_0 = F_{\text{exp}(2.5)}$. Como la distribución exponencial es continua podemos aplicar el Test de Kolmogorov-Smirnov, por lo que nos disponemos a calcular primero $D_{\text{exp}}$, mediante la fórmula (como las $n=20$ observaciones son distintas):
    \begin{equation*}
        D_{\text{exp}} = \max\{\max_{x_i}\{F_{X_1,\ldots,X_n}^\ast(x_i)-F_0(x_i)\}, \max_{x_i}\{F_0(x_i)-F_{X_1,\ldots,X_n}^\ast(x_i^-)\}\}
    \end{equation*}
    Para ello, usamos la tabla (abreviando con $F^\ast = F_{X_1,\ldots,X_n}^\ast$):
    \begin{table}[H]
    \centering
    \begin{tabular}{|c|c|c|c|c|c|c|}
        \hline
        $x_i$ & $n F^\ast(x_i)$ & $F^\ast(x_i)$ & $F^\ast(x_i^-)$ & $F_0(x_i)$ & $F^\ast(x_i)-F_0(x_i)$ & $F_0(x_i) - F^\ast(x_i^-)$\\
        \hline
        $0.01$ & 1 & 0.05& 0    & 0.02469 & 0.02530 & 0.02469\\
        $0.02$ & 2 & 0.1 & 0.05 & 0.04877 & 0.05122 & -0.0012\\
        $0.03$ & 3 & 0.15& 0.1  & 0.07225 & 0.07774 & -0.0277\\
        $0.23$ & 4 & 0.2 & 0.15 & 0.43729 & -0.2372 & 0.28729\\
        $0.51$ & 5 & 0.25& 0.2  & 0.72056 & -0.4705 & 0.52056\\
        $0.74$ & 6 & 0.3 & 0.25 & 0.84276 & -0.5427 & 0.59276\\
        $0.96$ & 7 & 0.35& 0.3  & 0.90928 & -0.5592 & 0.60928\\
        $1.17$ & 8 & 0.4 & 0.35 & 0.94633 & -0.5463 & 0.59633\\
        $1.46$ & 9 & 0.45& 0.4  & 0.98772 & -0.5377 & 0.58772\\
        $1.62$ & 10& 0.5 & 0.45 & 0.98257 & -0.4825 & 0.53257\\
        $2.18$ & 11& 0.55& 0.5  & 0.99570 & -0.4457 & 0.49570\\
        $2.25$ & 12& 0.6 & 0.55 & 0.99639 & -0.3963 & 0.44639\\
        $2.79$ & 13& 0.65& 0.6  & 0.99906 & -0.3490 & 0.39906\\
        $3.45$ & 14& 0.7 & 0.65 & 0.99982 & -0.2998 & 0.34982\\
        $3.82$ & 15& 0.75& 0.7  & 0.99993 & -0.2499 & 0.29993\\
        $3.92$ & 16& 0.8 & 0.75 & 0.99994 & -0.1999 & 0.24994\\
        $4.27$ & 17& 0.85& 0.8  & 0.99997 & -0.1499 & 0.19997\\
        $5.43$ & 18& 0.9 & 0.85 & 0.99999 & -0.0999 & 0.14999\\
        $5.79$ & 19& 0.95& 0.9  & 0.99999 & -0.0499 & 0.09999\\
        $6.34$ & 20& 1   & 0.95 & 0.99999 & 0       & 0.04999\\
        \hline
    \end{tabular}
    \end{table}
    \noindent
    de donde obtenemos:
    \begin{equation*}
        \left.\begin{array}{l}
            \max\limits_{x_i}\{F_{X_1,\ldots,X_n}^\ast(x_i)-F_0(x_i)\} = 0.07774 \\
            \max\limits_{x_i}\{F_0(x_i)-F_{X_1,\ldots,X_n}^\ast(x_i^-)\} = 0.60928 
        \end{array}\right\} \quad\Longrightarrow\quad D_{\text{exp}} = 0.60928
    \end{equation*}
    Calculamos el $p-$valor, usando para ello la distribución $Z$ de Kolmogorov, donde $D(X_1,\ldots,X_{20})\rightsquigarrow Z(20)$:
    \begin{equation*}
        P_{H_0}[D(X_1,\ldots,X_n)\geq D_{\text{exp}}] = P_{H_0}[D(X_1,\ldots,X_n)\geq 0.60928] < 0.001
    \end{equation*}
    obtenemos un dato mucho menor que $0.001$, por lo que podemos rechazar la afirmación del laboratorio.
\end{ejercicio}

\begin{ejercicio}
    Cierta comunidad ha modificado la procedencia del agua destinada al consumo doméstico. Se sabe que, con el antiguo suministro, la distribución de la cantidad de sodio por unidad de volumen de sangre de sus habitantes es simétrica alrededor de $3.24$ gr. Tras cierto tiempo, se quiere comprobar si la modificación ha afectado a la concentración de sodio, en el sentido de que su distribución se haya trasladado o no. Para ello, se han realizado 15 análisis con los siguientes resultados (en gr. por unidad):
    \begin{multline*}
        2.37 \quad 2.95 \quad 3.4 \quad 2.64 \quad 3.66 \quad 3.18 \quad 2.72 \quad 3.61 \quad 3.87 \quad 1.97 \quad 1.66 \quad 3.72 \quad 2.10 \\ 1.83 \quad 3.03
    \end{multline*}
    ¿Se puede afirmar, al nivel de significación $0.1$, que la distribución de la cantidad de sodio no ha variado?
\end{ejercicio}

\begin{ejercicio}
    La siguiente tabla presenta las presiones sanguíneas sistólicas de 10 individuos antes y después de haber dejado la bebida.
    \begin{table}[H]
    \centering
    \begin{tabular}{c|cccccccccc}
        \hline
        $A$ & 140 & 165 & 160 & 160 & 175 & 190 & 170 & 175 & 155 & 160 \\
        \hline
        $D$ & 145 & 150 & 150 & 160 & 170 & 175 & 160 & 165 & 145 & 170 \\
        \hline
    \end{tabular}
    \end{table}
    \noindent
    ¿Se puede afirmar a partir de los datos que el abandono de la bebida no disminuye la presión sanguínea? ¿Bajo qué hipótesis?
\end{ejercicio}

\begin{ejercicio}
    En cierta comunidad de E.E.U.U. se realizó un estudio para investigar si el sueldo anual de las familias influía en los hijos para la elección de los diferentes cursos de enseñanza secundaria (Preparatorio, General y Comercial). Para ello, se hizo una clasificación de los sueldos en cuatro niveles (I, II, III y IV), y se tomó una muestra aleatoria simple de 390 estudiantes, obteniéndose la siguiente tabla de frecuencias:
    \begin{table}[H]
    \centering
    \begin{tabular}{|c|c|c|c|c|}
        \hline
        Sueldo & I & II & III & IV \\
        \hline
        Preparatorio & 23 & 40 & 16 & 2 \\
        \hline
        General & 11 & 75 & 107 & 14 \\
        \hline
        Comercial & 1 & 31 & 60 & 10 \\
        \hline
    \end{tabular}
    \end{table}
    \noindent
    A la vista de los datos, decidir, al nivel de significación $0.01$, si se acepta que el nivel económico familiar no influye en la decisión de los estudiantes a la hora de elegir curso.
\end{ejercicio}

\begin{ejercicio}
    En un estudio sociológico sobre la polución atmosférica se entrevistó a 40 residentes de cada una de tres zonas residenciales en Gran Bretaña. La siguiente tabla muestra las respuestas a la pregunta: ¿Hay problema de polución en su barrio?
    \begin{table}[H]
    \centering
    \begin{tabular}{|c|c|c|c|c|}
        \hline
        Zona residencial & No & Sí & No sabe & No contesta \\
        \hline
        1 & 5 & 31 & 2 & 2 \\
        \hline
        2 & 10 & 21 & 4 & 5 \\
        \hline
        3 & 11 & 20 & 7 & 2 \\
        \hline
    \end{tabular}
    \end{table}
    \noindent
    Contrastar si las tres poblaciones de residentes pueden considerarse homogéneas con respecto a su opinión sobre la polución.
\end{ejercicio}

\begin{ejercicio}
    Para determinar si diferentes tipos de profesiones de los individuos activos de cierto colectivo afectan a la tensión arterial, se clasificó a los individuos en cuatro grupos, atendiendo a su profesión, y se midió la tensión a una muestra de individuos elegidos de forma aleatoria. Clasificando la tensión en los niveles ``Bajo'', ``Normal'' y ``Alto'', se obtuvo los siguientes resultados, que muestran el número de individuos de cada tipo de profesión con los distintos niveles de tensión:
    \begin{table}[H]
    \centering
    \begin{tabular}{|c|c|c|c|}
        \hline
        Profesión & Bajo & Normal & Alto \\
        \hline
        I & 8 & 4 & 3 \\ 
        \hline
        II & 5 & 7 & 7 \\
        \hline
        III & 4 & 8 & 8 \\
        \hline
        IV & 5 & 7 & 8 \\
        \hline
    \end{tabular}
    \end{table}
    \noindent
    ¿Qué conclusión acerca del problema planteado se obtiene a la vista de estos datos?  Especificar las hipótesis nula y alternativa que se contrastan.
\end{ejercicio}

\begin{ejercicio}
    Para determinar si las calificaciones de los alumnos en selectividad son independientes de las calificaciones en bachiller, se eligió de forma aleatoria una muestra de alumnos, a los que se preguntó ambas calificaciones, obteniendo los siguientes resultados:
    \begin{table}[H]
    \centering
    \begin{tabular}{|c|c|c|c|c|}
        \hline
        Bachiller & Suspenso & Aprobado & Notable & Sobresaliente \\
        \hline
        Aprobado & 10 & 6 & 4 & 5 \\
        \hline
        Notable & 7 & 9 & 9 & 4 \\
        \hline
        Sobresaliente & 6 & 10 & 10 & 6 \\
        \hline
    \end{tabular}
    \end{table}
    \noindent
    ¿Qué conclusión acerca del problema planteado se obtiene a la vista de estos datos?  Especificar las hipótesis nula y alternativa que se contrastan.
\end{ejercicio}
