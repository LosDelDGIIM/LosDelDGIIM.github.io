\newpage
\section{Estimación por intervalos de confianza}

\begin{ejercicio}
    Sea $\overline{X}$ la media de una muestra aleatoria de tamaño $n$ de una población $\cc{N}(\mu, 16)$ Encontrar el menor valor de $n$ para que $\left]\overline{X}-1,\overline{X}+1\right[$ sea un intervalo de confianza para $\mu$ al nivel de confianza $0.9$.\\

    \noindent
    Si $\overline{X}$ es la media de una m.a.s. de tamaño $n$ de una población $\cc{N}(\mu,16)$, tenemos entonces que $\overline{X}\rightsquigarrow\cc{N}(\mu, \frac{16}{n})$. Calculamos $n$ para que el intervalo $\left]\overline{X}-1,\overline{X}+1\right[$ sea un intervalo de confianza para $\mu$ a nivel $0.9$, buscando:
    \begin{align*}
        P_\mu\left[\overline{X}-1 \leq \mu \leq \overline{X}+1\right] \geq 0.9 &\Longleftrightarrow P_\mu[\mu-1\leq \overline{X}\leq \mu+1] \geq 0.9 \\
                                                                    &\stackrel{\text{tipif.}}{\Longleftrightarrow} P_\mu\left[\frac{-\sqrt{n}}{4} \leq Z \leq \frac{\sqrt{n}}{4}\right] \geq 0.9 \\
                                                                    &\Longleftrightarrow 2P_\mu\left[Z\leq \frac{\sqrt{n}}{4}\right] - 1 \geq 0.9 \\
                                                                    &\Longleftrightarrow 2P_\mu\left[Z\leq \frac{\sqrt{n}}{4}\right] \geq 1.9 \\
                                                                    &\Longleftrightarrow P_\mu\left[Z\leq \frac{\sqrt{n}}{4}\right] \geq 0.95 
    \end{align*}
    Consultando la tabla de la normal $\cc{N}(0,1)$, tenemos que la primera abscisa en la que se alcanza una probabilidad superior a $0.95$ es $1.65$, por lo que:
    \begin{equation*}
        \frac{\sqrt{n}}{4} = 1.65 \Longleftrightarrow \sqrt{n} = 6.6 \Longleftrightarrow n = 43.56
    \end{equation*}
    Por tanto, el menor valor de $n$ para el cual el intervalo $\left]\overline{X}-1,\overline{X}+1\right[$ es un intervalo de confianza para $\mu$ a nivel de confianza $0.9$ es $44$.
\end{ejercicio}

\begin{ejercicio}
    La altura en cm. de los individuos varones de una población sigue una distribución $\cc{N}(\mu,\ 56.25)$. Si en una muestra aleatoria simple de tamaño $12$ de dicha población se obtiene una altura media de $175$ cm., determinar un intervalo de confianza para $\mu$ al nivel de confianza $0.95$. ¿Qué tamaño de muestra es necesario para que el intervalo de confianza a dicho nivel tenga longitud menor que 1 cm?
\end{ejercicio}

\begin{ejercicio}
    Una fábrica produce tornillos cuyo diámetro medio es $3$ mm. Se seleccionan aleatoria e independientemente 12 de estos tornillos y se miden sus diámetros, que resultan ser $3.01,\ 3.05,\ 2.99,\ 2.99,\ 3.00,\ 3.02,\ 2.98,\ 2.99,\ 2.97,\ 2.97,\ 3.02$ y $3.01$. Suponiendo que el diámetro es una variable aleatoria con distribución normal, determinar un intervalo de confianza para la varianza al nivel de confianza $0.99$, y una cota superior de confianza al mismo nivel. Interpretar los resultados en términos de la desviación típica del diámetro de los tornillos.
\end{ejercicio}

\begin{ejercicio}
    Las notas en cierta asignatura de 7 alumnos de una clase, elegidos de forma aleatoria e independiente son: $4.5$, $3, 6$, $7$, $1.5$, $5.2$ y $3.6$. Suponiendo que las notas tienen distribución normal, dar un intervalo de confianza para la varianza de las mismas al nivel de confianza $0.95$.
\end{ejercicio}

\begin{ejercicio}
    La siguiente tabla presenta los salarios anuales (en miles de euros) de dos grupos de recién graduados de dos carreras diferentes. Suponiendo normalidad en los salarios de ambos grupos, determinar un intervalo de confianza para el cociente de las varianzas al nivel de confianza $0.90$.
    \begin{table}[H]
    \centering
    \begin{tabular}{|c|cccccccccc|}
        \hline 
        GRUPO 1 & 16.3 & 18.2 & 17.5 & 16.1 & 15.9 & 15.4 & 15.8 & 17.3 & 14.9 & 15.1 \\
        \hline
        GRUPO 2 & 13.2 & 15.1 & 13.9 & 14.7 & 15.6 & 15.8 & 14.9 & 18.1 & 15.6 & 15.3 \\ 
                & 16.2 & 15.2 & 15.4 & 16.6 & & & & & &  \\
        \hline
    \end{tabular}
    \end{table}
\end{ejercicio}

\begin{ejercicio}
    Con objeto de estudiar la efectividad de un agente diurético, se eligen al azar 11 pacientes, aplicando dicho fármaco a seis de ellos y un placebo a los cinco restantes. La variable observada en esta experiencia fue la concentración de sodio en la orina a las 24 horas, que se supone tiene una distribución normal en ambos casos. Los resultados observados fueron:
    \begin{itemize}
        \item Diurético: $20.4$, $62.5$, $61.3$, $44.2$, $11.1$, $23.7$.
        \item Placebo: $1.2$, $6.9$, $38.7$, $20.4$, $17.2$.
    \end{itemize}
    \begin{enumerate}[label=\alph*)]
        \item Calcular un intervalo de confianza para el cociente de las varianzas al nivel de confianza $0.95$.
        \item Suponiendo que las varianzas son iguales, calcular un intervalo de confianza para la diferencia de las medias al nivel de confianza $0.9$, y una cota inferior de confianza al mismo nivel. Interpretar los resultados.
    \end{enumerate}
\end{ejercicio}

\begin{ejercicio}
    Sea $(X_1, \ldots, X_n)$ una muestra aleatoria simple de una variable aleatoria con distribución $U(0,\theta)$. Dado un nivel de confianza arbitrario, calcular el intervalo de confianza para $\theta$ de menor longitud media uniformemente basado en un estadistico suficiente.
\end{ejercicio}

\begin{ejercicio}
    Utilizando la desigualdad de Chebychev, dar un intervalo de confianza para p a nivel de confianza arbitrario, basado en una muestra de tamaño arbitrario de una variable aleatoria con distribución $B(1,p)$.
\end{ejercicio}

\begin{ejercicio}
    Para una muestra de tamaño $n$ de una variable aleatoria con función de densidad
    \begin{equation*}
        f_\theta(x) = \frac{2x}{\theta^2}, \qquad 0<x<\theta
    \end{equation*}
    encontrar el intervalo de confianza para $\theta$ de menor longitud media uniformemente a nivel de confianza $1-\alpha$, basado en un estadístico suficiente.
\end{ejercicio}

\begin{ejercicio} % // TODO: CAMBIAR
    Para una muestra de tamaño $n$ de una variable aleatoria con función de densidad
    \begin{equation*}
        f_\theta(x) = \frac{\theta}{x^2}, \qquad x>\theta
    \end{equation*}
    encontrar el intervalo de confianza para $\theta$ de menor longitud media uniformemente a nivel de confianza $1 - \alpha$, basado en el estimador máximo verosímil de $\theta$.\\

    \noindent
    La función de verosimilitud es:
    \begin{equation*}
        L_{x_1,\ldots,x_n}(\theta) = \prod_{i=1}^{n} \frac{\theta}{x_i^2} \qquad \forall x_i > \theta > 0
    \end{equation*}
    Por lo que:
    \begin{equation*}
        L_{x_1,\ldots,x_n}(\theta) = \left\{\begin{array}{ll}
            \prod\limits_{i=1}^{n}\frac{\theta}{x_i^2} & \text{si\ } \theta < x_{(1)}  \\
             0 & \text{en otro caso} 
        \end{array}\right. 
    \end{equation*}
    Y vemos que $L_{x_1,\ldots,x_n}(\theta)$ es creciente, con lo que alcanza su máximo en $\theta = x_{(1)}$. En definitiva, el EMV es $\hat{\theta} = X_{(1)}$, por lo que buscamos un intervalo de confianza basado en $X_{(1)}$.\\

    \noindent
    La función de distribución del mínimo es:
    \begin{equation*}
        F_{X_{(1)}}(t) = 1-{(1-F_X(t))}^{n}
    \end{equation*}
    Por lo que calculamos:
    \begin{equation*}
        F_X(t) = \int_{\theta}^{t} \frac{\theta}{x^2}~dx  = \left[\frac{-\theta}{x}\right]_\theta^t = \frac{-\theta}{t} + 1 \qquad t>\theta
    \end{equation*}

    de donde:
    \begin{equation*}
        F_{X_{(1)}}(t) = 1-{(1-F_X(t))}^{n} = 1 - {\left(1-\left(1-\frac{\theta}{t}\right)\right)}^{n} = 1-{\left(\frac{\theta}{t}\right)}^{n} \qquad t> \theta
    \end{equation*}
    Tomamos por tanto como función pivote:
    \begin{equation*}
        T(X_1, \ldots, X_n;\theta) = 1-{\left(\frac{\theta}{X_{(1)}}\right)}^{n} \rightsquigarrow U(0,1)
    \end{equation*}
    Comprobamos las condiciones del método de la cantidad pivotal:
    \begin{itemize}
        \item Si derivamos:
            \begin{equation*}
                \dfrac{\partial T}{\partial \theta} = -n{\left(\frac{\theta}{X_{(1)}}\right)}^{n-1} \frac{1}{X_{(1)}} = \frac{-n\theta^{n-1}}{X_{(1)}^n} < 0 \qquad \forall \theta
            \end{equation*}
            Por lo que $T(X_1, \ldots, X_n;\theta)$ es estrictamente decreciente en función de $\theta$.
        \item Si tratamos de despejar el parámetro para cierto $\lm$:
            \begin{equation*}
                1-{\left(\frac{\theta}{X_{(1)}}\right)}^{n} = \lm \Longrightarrow \theta = \sqrt[n]{(1-\lm)X_{(1)}^n}  = X_{(1)}\sqrt[n]{1-\lm}
            \end{equation*}
    \end{itemize}
    Por lo que hemos obtenido como intervalo de confianza (donde los índices de $\lm$ son debidos a que es estrictamente decreciente):
    \begin{equation*}
        \left]X_{(1)}\sqrt[n]{1-\lm_2}, X_{(1)}\sqrt[n]{1-\lm_2}\right[
    \end{equation*}
    Tratamos ahora de minimizar la longitud del intervalo, que es:
    \begin{equation*}
        L = X_{(1)}\left(\sqrt[n]{1-\lm_1} - \sqrt[n]{1-\lm_2}\right)
    \end{equation*}

    de donde:
    \begin{equation*}
        E_\theta[L] = E_\theta[X_{(1)}]\left(\sqrt[n]{1-\lm_1} - \sqrt[n]{1-\lm_2}\right)
    \end{equation*}
    Y como $E_\theta(X_{(1)})$ es una constante positiva, podemos obviarla a la hora de minimizar, por lo que buscamos minimizar el segundo trozo, sujeto a la restricción:
    \begin{equation*}
        P_\theta[\lm_1 < T < \lm_2] = 1-\alpha
    \end{equation*}
    Y como tenemos:
    \begin{equation*}
        P[\lm_1 < T < \lm_2] = F_T(\lm_2) - F_T(\lm_1) = \lm_2 - \lm_1
    \end{equation*}
    Por lo que tenemos la restricción $\lm_2-\lm_1 = 1-\alpha$. Para meterla en la función a minimizar:
    \begin{itemize}
        \item Bien despejamos $\lm_1$ o $\lm_2$, sustituimos en la función a minimizar y obtenemos una función de una variable, que sabemos minimizar.
        \item Usamos los multiplicadores de Lagrange, sumamos a la función la restricción (igualdada a 0) multiplicada por un cierto parámetro, que llamaremos $\lm$, por lo que buscamos minimizar:
            \begin{equation*}
                {(1-\lm_1)}^{\nicefrac{1}{n}}- {(1-\lm_2)}^{\nicefrac{1}{n}} + \lm (\lm_2 - \lm_1 - 1+\alpha)
            \end{equation*}
    \end{itemize}
    Por ninguna de las dos formas hayaremos solución, por lo que la cantidad siempre crece o decrece. % // TODO: TERMINAR
\end{ejercicio}

% // TODO: FALTA UN EJERCICIO POR COPIAR
