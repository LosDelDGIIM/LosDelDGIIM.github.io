\section{Estadísticos muestrales}

\begin{ejercicio}
    Sea $(X_1, \ldots, X_n)$ una muestra aleatoria simple de una variable aleatoria $X$. Dar el espacio muestral y calcular la función masa de probabilidad de $(X_1, \ldots, X_n)$ en cada uno de los siguientes casos:
    \begin{enumerate}[label=\alph*)]
        \item $X\rightsquigarrow \{B(k_0,p) : p\in (0,1)\}$ Binomial.
        \item $X\rightsquigarrow\{\cc{P}(\lm) : \lm \in \mathbb{R}^+\}$ Poisson.
        \item $X\rightsquigarrow\{BN(k_0,p) : p\in (0,1)\}$ Binomial Negativa.
        \item $X\rightsquigarrow\{G(p) : p\in (0,1)\}$ Geométrica.
        \item $X\rightsquigarrow\{P_N : N\in \mathbb{N}\}$, $\quad P_N(X=x) = \dfrac{1}{N}$, $\quad x=1,\ldots,N$.
    \end{enumerate}
\end{ejercicio}

\begin{ejercicio}
    Sea $(X_1, \ldots, X_n)$ una muestra aleatoria simple de una variable aleatoria $X$. Dar el espacio muestral y calcular la función de densidad de $(X_1, \ldots, X_n)$ en cada uno de los siguientes casos:
    \begin{enumerate}[label=\alph*)]
        \item $X\rightsquigarrow\{U(a,b) : a,b\in \mathbb{R}, a<b\}$ Uniforme.
        \item $X\rightsquigarrow\{\cc{N}(\mu, \sigma^2) : \mu \in \mathbb{R}, \sigma^2 \in \mathbb{R}^+\}$ Normal.
        \item $X\rightsquigarrow\{\Gamma(p,a) : p,a\in \mathbb{R}^+\}$ Gamma.
        \item $X\rightsquigarrow\{\beta(p,q) : p,q\in \mathbb{R}^+\}$ Beta.
        \item $X\rightsquigarrow\{P_\theta : \theta \in \mathbb{R}^+\}$, $\quad f_\theta(x) = \dfrac{1}{2\sqrt{x\theta}}$, $\quad 0<x<\theta$.
    \end{enumerate}
\end{ejercicio}

\begin{ejercicio}
    Se miden los tiempos de sedimentación de una muestra de partículas flotando en un líquido. Los tiempos observados son: 
    \begin{equation*}
        11.5; 1.8; 7.3; 12.1 1.8; 21.3; 7.3; 15.2; 7.3; 12.1; 15.2; 7.3; 12.1; 1.8; 10.5; 15.2; 21.3; 10.5; 15.2; 11.5
    \end{equation*}
    \begin{itemize}
        \item Construir la función de distribución muestral asociada a a dichas observaciones.
        \item Hallar los valores de los tres primeros momentos muestrales respecto al origen y respecto a la media.
        \item Determinar los valores de los cuartiles muestrales y el percentil 70.
    \end{itemize}
\end{ejercicio}

\begin{ejercicio}
    Se dispone de una muestra aleatoria simple de tamaño 40 de una distribución exponencial de media 3, ¿cuál es la probabilidad de que los valores de la función de distribución muestral y la teórica, en $x=1$, difieran menos de $0.01$? Aproximadamente, ¿cuál debe ser el tamaño muestral para que dicha probabilidad sea como mı́nimo $0.98$?
\end{ejercicio}

\begin{ejercicio}
    Se dispone de una muestra aleatoria simple de tamaño 50 de una distribución de Poisson de media 2, ¿cuál es la probabilidad de que los valores de la función de distribución muestral y la teórica, en $x=2$, difieran menos de $0.02$? Aproximadamente, ¿qué tamaño muestral hay que tomar para que dicha probabilidad sea como mı́nimo $0.99$?
\end{ejercicio}

\begin{ejercicio}
   Sea $X\rightsquigarrow B(1,p)$ y $(X_1, X_2, X_3)$ una muestra aleatoria simple de $X$. Calcular la función masa de probabilidad de los estadísticos $\overline{X}$, $S^2$, $\min X_i$ y $\max X_i$.
\end{ejercicio}

\begin{ejercicio}
    Obtener la función masa de probabilidad o función de densidad de $\overline{X}$ en el muestreo de una variable de Bernoulli, de una Poisson y de una exponencial.
\end{ejercicio}

\begin{ejercicio}
    Calcular las funciones de densidad de los estadísticos $\max X_i$ y $\min X_i$ en el muestreo de una variable $X$ con funcion de densidad:
    \begin{equation*}
        f_\theta(x) = e^{\theta-x}, \qquad x>\theta.
    \end{equation*}
\end{ejercicio}

\begin{ejercicio}
    El número de pacientes que visitan diariamente una determinada consulta médica es una variable aleatoria con varianza de 16 personas. Se supone que el número de visitas de cada dı́a es independiente de cualquier otro. Si se observa el número de visitas diarias durante 64 dı́as, calcular aproximadamente la probabilidad de que la media muestral no difiera en más de una persona del valor medio verdadero de visitas diarias.
\end{ejercicio}

\begin{ejercicio}
    Una máquina de refrescos está arreglada para que la cantidad de bebida que sirve sea una variable aleatoria con media 200 ml. y desviación tı́pica 15 ml. Calcular de forma aproximada la probabilidad de que la cantidad media servida en una muestra aleatoria de tamaño 36 sea al menos 204 ml.
\end{ejercicio}
