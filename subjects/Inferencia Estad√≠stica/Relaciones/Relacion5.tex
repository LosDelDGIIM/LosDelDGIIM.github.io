\newpage
\section{Estimación de máxima verosimilitud y otros métodos}
\begin{ejercicio}
    Sea $X\rightsquigarrow\{P_\theta : \theta \in \mathbb{R}\}$ siendo $P_\theta$ una distribución con función de densidad
    \begin{equation*}
        f_\theta(x) = e^{\theta-x}, \qquad x\geq \theta
    \end{equation*}
    Dada una muestra aleatoria simple de tamaño $n$, encontrar los estimadores máximo verosímiles de $\theta$ y de $e^\theta$. Basándose en los resultados del Ejercicio~\ref{ej:7_rel4}, decir si estos estimadores son insesgados.
\end{ejercicio}

\begin{ejercicio}
    Sea $(X_1, \ldots, X_n)$ una muestra aleatoria simple de una variable aleatoria $X$ con distribución exponencial. Basándose en los resultados del Ejercicio~\ref{ej:9_rel4}, encontrar los estimadores máximo verosímiles de la media y de la varianza de $X$.
\end{ejercicio}

\begin{ejercicio}
    Sea $X$ una variable aleatoria con función de densidad de la forma
    \begin{equation*}
        f_\theta(x) = \theta x^{\theta -1}, \qquad 0<x<1
    \end{equation*}
    \begin{enumerate}[label=\alph*)]
        \item Calcular un estimador máximo verosímil para $\theta$.
        \item Deducir dicho estimador a partir de los resultados del Ejercicio~\ref{ej:10_rel4}.
    \end{enumerate}
\end{ejercicio}

\begin{ejercicio}
    Sea $(X_1, \ldots, X_n)$ una muestra de una variable $X\rightsquigarrow\{B(k_0,p): p\in \left]0,1\right[\}$. Estimar, por máxima verosimilitud y por el método de los momentos, el parámetro $p$ y la varianza de $X$.\\

    \noindent
    \textit{Aplicación}: Se lanza $10$ veces un dado cargado y se cuenta el número de veces que sale un $4$. Este experimento se realiza $100$ veces de forma independiente, obteniéndose los siguientes resultados:
    \begin{table}[H]
    \centering
    \begin{tabular}{l|cccc}
        nº de 4 & 0 & 1 & 2 & 3 \\
        \hline
        frecuencia & 84 & 15 & 1 & 0
    \end{tabular}
    \end{table}
    Estimar, a partir de estos datos, la probabilidad de salir un cuatro.
\end{ejercicio}

\begin{ejercicio}
    Se lanza un dado hasta que salga un $4$ y se anota el número de lanzamientos necesarios; este experimento se efectúa veinte veces de forma independiente. A partir de los resultados obtenidos, estimar la probabilidad de sacar un $4$ por máxima verosimilitud.
\end{ejercicio}

\begin{ejercicio}
    En $20$ días muy fríos, una granjera pudo arrancar su tractor en el primer, tercer, quinto, primer, segundo, primer, tercer, séptimo, segundo, cuarto, cuarto, octavo, primer, tercer, sexto, quinto, segundo, primer, sexto y segundo intento. Suponiendo que la probabilidad de arrancar en cada intento es constante, y que las observaciones se han obtenido de forma independiente, dar la estimación más verosímil de la probabilidad de que el tractor arranque en el segundo intento.
\end{ejercicio}

\begin{ejercicio}
    Una variable aleatoria discreta toma los valores $0.1$ y $2$ con las siguientes probabilidades
    \begin{equation*}
        P_p[X=0] = p^2, \qquad P_p[X=1] = 2p(1-p), \qquad P_p[X=2] = {(1-p)}^{2}
    \end{equation*}
    siendo $p$ un parámetro desconocido. En una muestra aleatoria simple de tamaño $100$, se ha presentado $22$ veces el $0$, $53$ veces el $1$ y $25$ veces el $2$. Calcular la función de verosimilitud asociada a dicha muestra y dar la estimación más verosímil de $p$.
\end{ejercicio}

\begin{ejercicio}
    En el muestreo de una variable aleatoria con distribución $\cc{N}(\mu,1)$, $\mu \in \mathbb{R}$, se observa que no se obtiene un valor menor que $-1$ hasta la quinta observación. Dar una estimación máximo verosímil de $\mu$.
\end{ejercicio}

\begin{ejercicio}
    En la producción de filamentos eléctricos la medida de interés, $X$, es el tiempo de vida de cada filamento, que tiene una distribución exponencial de parámetro $\theta$. Se eligen $n$ de tales filamentos de forma aleatoria e independiente, pero, por razones de economía, no conviene esperar a que todos se quemen y la observación acaba en el tiempo $T$. Dar el estimador máximo verosímil para la media de $X$ a partir del número de filamentos quemados durante el tiempo de observación.
\end{ejercicio}

\begin{ejercicio}
    Sean $X_1, \ldots, X_n$ observaciones independientes de una variable $X\rightsquigarrow\{\Gamma(p,a) : p,a>0\}$. Estimar ambos parámetros mediante el método de los momentos.\\

    \noindent
    \textit{Aplicación}: Ciertos neumáticos radiales tuvieron vidas útiles de $35200$, $41000$, $44700$, $38600$ y $41500$ kilómetros. Suponiendo que estos datos son observaciones independientes de una variable con distribución exponencial de parámetro $\theta$, dar una estimación de dicho parámetro por el método de los momentos.
\end{ejercicio}
