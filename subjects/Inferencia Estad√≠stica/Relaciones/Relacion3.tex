\newpage
\section{Suficiencia y completitud}

\begin{ejercicio}
    Sea $(X_1, \ldots, X_n)$ una muestra aleatoria simple de una variable $X\rightsquigarrow\{B(k,p) : p\in \left]0,1\right[\}$ y sea $T(X_1, \ldots, X_n) = \sum\limits_{i=1}^n X_i$. Probar
    \begin{enumerate}[label=\alph*)]
        \item usando la definición
        \item aplicando el teorema de factorización
    \end{enumerate}
    que $T$ es suficiente para $p$.
\end{ejercicio}

\begin{ejercicio}
    Sea $(X_1, \ldots, X_n)$ una muestra aleatoria simple de una variable $X\rightsquigarrow\{\cc{P}(\lm) : \lm\in \mathbb{R}^+\}$ y sea $T(X_1, \ldots, X_n) = \sum\limits_{i=1}^n X_i$. Probar
    \begin{enumerate}[label=\alph*)]
        \item usando la definición
        \item aplicando el teorema de factorización
    \end{enumerate}
    que $T$ es suficiente para $\lm$.
\end{ejercicio}

\begin{ejercicio}
    Sea $(X_1, X_2, X_3)$ una muestra aleatoria simple de una variable $X\rightsquigarrow\{B(1,p) : p \in \left]0,1\right[\}$. Probar que el estadístico $X_1 + 2X_2 + 3_X3$ no es suficiente.
\end{ejercicio}

\begin{ejercicio}
    Aplicando el teorema de factorización, y basándose en una muestra de tamaño arbitrario, encontrar un estadístico suficiente para cada una de las siguientes familias de distribuciones (en las familias biparamétricas, suponer los casos de sólo un parámetro desconocido y de los dos desconocidos).
    \begin{enumerate}[label=\alph*)]
        \item $X\rightsquigarrow\{U(-\nicefrac{\theta}{2},\nicefrac{\theta}{2}) : \theta>0\}$
        \item $X\rightsquigarrow\{\Gamma(p,a) : p,a > 0\}$
        \item $X\rightsquigarrow\{\beta(p,q) : p,q>0\}$
        \item $X\rightsquigarrow\{P_{N_1,N_2} : N_1, N_2 \in \mathbb{N}, N_1 \leq N_2\}$ y la masa de probabilidad viene dada por:
            \begin{equation*}
                P_{N_1, N_2}[X=x] = \dfrac{1}{N_2 - N_1 + 1} \qquad x\in \{N_1, \ldots, N_2\}
            \end{equation*}
    \end{enumerate}
\end{ejercicio}

\begin{ejercicio}
    Sea $X\rightsquigarrow\{P_N : N\in \mathbb{N}\}$, siendo $P_N$ la distribución uniforme en los puntos $\{1,\ldots,N\}$, y sea $(X_1, \ldots, X_n)$ una muestra aleatoria simple de $X$. Probar que $\max(X_1, \ldots, X_n)$ es un estadístico suficiente y completo.
\end{ejercicio}

\begin{ejercicio}
    Basándose en una muestra de tamaño arbitrario, obtener un estadístico suficiente y completo para la familia de distribuciones definidas por todas las densidades de la forma
    \begin{equation*}
        f_\theta(x) = e^{\theta-x}, \qquad x>\theta
    \end{equation*}
\end{ejercicio}

\begin{ejercicio}
    Comprobar que las siguientes familias de distribuciones son exponenciales uniparamétricas y, considerando una muestra aleatoria simple de una variable con distribución en dicha familia, obtener, si existe, un estadístico suficiente y completo.
    \begin{enumerate}[label=\alph*)]
        \item $\{B(k_0,p) : 0<p<1\}$
        \item $\{\cc{P}(\lm) : \lm>0\}$
        \item $\{BN(k_0,p) : 0<p<1\}$
        \item $\{exp(\lm) : \lm>0\}$
    \end{enumerate}
\end{ejercicio}

\begin{ejercicio}
    Estudiar si las siguientes familias de distribuciones son exponenciales biparamétricas.  En caso afirmativo, considerando una muestra aleatoria simple de una variable con distribución en dicha familia, obtener, si existe, un estadístico suficiente y completo.
    \begin{enumerate}[label=\alph*)]
        \item $\{\Gamma(p,a) : p,a>0\}$
        \item $\{\beta(p,1) : p,q>0\}$
    \end{enumerate}
\end{ejercicio}
