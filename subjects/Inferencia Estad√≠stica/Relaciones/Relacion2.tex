\newpage
\section{Distribuciones en el muestreo de poblaciones normales}

\begin{ejercicio}
    Se toma una muestra aleatoria simple de tamaño 5 de una variable aleatoria con distribución $\cc{N}(2.5,\ 3.6)$. Calcular:
    \begin{enumerate}[label=\alph*)]
        \item Probabilidad de que la cuasivarianza muestral esté comprendida entre $1.863$ y $2.674$.
        \item Probabilidad de que la media muestral esté comprendida entre $1.3$ y $3.5$, supuesto que la cuasivarianza muestral está entre $30$ y $40$.
    \end{enumerate}
\end{ejercicio}

\begin{ejercicio}
    La longitud craneal en una determinada población humana es una variable aleatoria que sigue una distribución normal con media $185.6$ mm. y desviación típica $12.78$ mm. ¿Cuál es la probabilidad de que una muestra aleatoria simple de tamaño 20 de esa población tenga media mayor que $190$ mm.?
\end{ejercicio}

\begin{ejercicio}
    ¿De que tamaño mínimo habría que seleccionar una muestra de una variable con distribución normal $\cc{N}(\mu, 4)$ para poder afirmar, con probabilidad mayor que $0.9$, que la media muestral diferirá de la poblacional menos de $0.1$?
\end{ejercicio}

\begin{ejercicio}
    Sea $(X_1, \ldots, X_n)$ una muestra aleatoria simple de una variable con distribución normal.  Calcular la probabilidad de que la cuasivarianza muestral sea menor que un $50\%$ de la varianza poblacional para $n = 16$ y para $n = 1000$.
\end{ejercicio}

\begin{ejercicio}
    Sean $S_1^2$ y $S_2^2$ las cuasivarianzas muestrales de dos muestras independientes de tamaños $n_1 = 5$ y $n_2 = 4$ de dos poblaciones normales con la misma varianza. Calcular la probabilidad de que $\nicefrac{S_1^2}{S_2^2}$ sea menor que $5.34$ o mayor que $9.12$.
\end{ejercicio}

\begin{ejercicio}
    Se consideran dos poblaciones de bombillas cuyas longitudes de vida siguen una ley normal con la misma media y desviaciones típicas $425$ y $375$ horas, respectivamente.  Con objeto de realizar un estudio comparativo de ambas poblaciones, se considera una muestra aleatoria simple de $10$ bombillas en la primera población y una de tamaño 6 en la segunda. ¿Cuál es la probabilidad de que la media muestral del primer grupo menos la del segundo sea menor que la observada en dos realizaciones muestrales que dieron $1325$ horas y $1215$ horas, respectivamente?
\end{ejercicio}

\begin{ejercicio}
    Sean $X_1, \ldots, X_n, X_{n+1}$ variables aleatorias independientes e idénticamente distribuidas según una $\cc{N}(\mu, \sigma^2)$, y sean $\overline{X}$ y $S^2$ la media y la cuasivarianza muestral de $(X_1, \ldots, X_n)$. Calcular la distribución de
    \begin{equation*}
        \dfrac{X_{n+1}- \overline{X}}{S}\sqrt{\dfrac{n}{n+1}}
    \end{equation*}
\end{ejercicio}

\begin{ejercicio}
    Sean $(X_1, \ldots, X_n)$, $(Y_1, \ldots, Y_m)$ muestras aleatorias simples independientes de poblaciones $\cc{N}(\mu_1,\sigma^2)$ y $\cc{N}(\mu_2, \sigma^2)$, respectivamente. Sean $\alpha,\beta\in \mathbb{R}$ y $\overline{X},\overline{Y},S_1^2, S_2^2$ las medias y cuasivarianzas de las dos muestras. Calcular la distribución de
    \begin{equation*}
        \dfrac{\alpha\left(\overline{X}-\mu_1\right)+\beta\left(\overline{Y}-\mu_2\right)}{\sqrt{\dfrac{(n-1)S_1^2 + (m-1)S_2^2}{n+m-2}} \sqrt{\dfrac{\alpha^2}{n} + \dfrac{\beta^2}{m}}}
    \end{equation*}
\end{ejercicio}
