\documentclass[12pt]{book}

% Idioma y codificación
\usepackage[spanish, es-tabla, es-notilde]{babel}       %es-tabla para que se titule "Tabla"
\usepackage[utf8]{inputenc}

% Márgenes
\usepackage[a4paper,top=3cm,bottom=2.5cm,left=3cm,right=3cm]{geometry}

% Comentarios de bloque
\usepackage{verbatim}

% Paquetes de links
\usepackage[hidelinks]{hyperref}    % Permite enlaces
\usepackage{url}                    % redirecciona a la web

% Más opciones para enumeraciones
\usepackage{enumitem}

% Personalizar la portada
\usepackage{titling}

% Paquetes de tablas
\usepackage{multirow}

% Para añadir el símbolo de euro
\usepackage{eurosym}


%------------------------------------------------------------------------

%Paquetes de figuras
\usepackage{caption}
\usepackage{subcaption} % Figuras al lado de otras
\usepackage{float}      % Poner figuras en el sitio indicado H.


% Paquetes de imágenes
\usepackage{graphicx}       % Paquete para añadir imágenes
\usepackage{transparent}    % Para manejar la opacidad de las figuras

% Paquete para usar colores
\usepackage[dvipsnames, table, xcdraw]{xcolor}
\usepackage{pagecolor}      % Para cambiar el color de la página

% Habilita tamaños de fuente mayores
\usepackage{fix-cm}

% Para los gráficos
\usepackage{tikz}
\usepackage{forest}

% Para poder situar los nodos en los grafos
\usetikzlibrary{positioning}


%------------------------------------------------------------------------

% Paquetes de matemáticas
\usepackage{mathtools, amsfonts, amssymb, mathrsfs}
\usepackage[makeroom]{cancel}     % Simplificar tachando
\usepackage{polynom}    % Divisiones y Ruffini
\usepackage{units} % Para poner fracciones diagonales con \nicefrac

\usepackage{pgfplots}   %Representar funciones
\pgfplotsset{compat=1.18}  % Versión 1.18

\usepackage{tikz-cd}    % Para usar diagramas de composiciones
\usetikzlibrary{calc}   % Para usar cálculo de coordenadas en tikz

%Definición de teoremas, etc.
\usepackage{amsthm}
%\swapnumbers   % Intercambia la posición del texto y de la numeración

\theoremstyle{plain}

\makeatletter
\@ifclassloaded{article}{
  \newtheorem{teo}{Teorema}[section]
}{
  \newtheorem{teo}{Teorema}[chapter]  % Se resetea en cada chapter
}
\makeatother

\newtheorem{coro}{Corolario}[teo]           % Se resetea en cada teorema
\newtheorem{prop}[teo]{Proposición}         % Usa el mismo contador que teorema
\newtheorem{lema}[teo]{Lema}                % Usa el mismo contador que teorema
\newtheorem*{lema*}{Lema}

\theoremstyle{remark}
\newtheorem*{observacion}{Observación}

\theoremstyle{definition}

\makeatletter
\@ifclassloaded{article}{
  \newtheorem{definicion}{Definición} [section]     % Se resetea en cada chapter
}{
  \newtheorem{definicion}{Definición} [chapter]     % Se resetea en cada chapter
}
\makeatother

\newtheorem*{notacion}{Notación}
\newtheorem*{ejemplo}{Ejemplo}
\newtheorem*{ejercicio*}{Ejercicio}             % No numerado
\newtheorem{ejercicio}{Ejercicio} [section]     % Se resetea en cada section


% Modificar el formato de la numeración del teorema "ejercicio"
\renewcommand{\theejercicio}{%
  \ifnum\value{section}=0 % Si no se ha iniciado ninguna sección
    \arabic{ejercicio}% Solo mostrar el número de ejercicio
  \else
    \thesection.\arabic{ejercicio}% Mostrar número de sección y número de ejercicio
  \fi
}


% \renewcommand\qedsymbol{$\blacksquare$}         % Cambiar símbolo QED
%------------------------------------------------------------------------

% Paquetes para encabezados
\usepackage{fancyhdr}
\pagestyle{fancy}
\fancyhf{}

\newcommand{\helv}{ % Modificación tamaño de letra
\fontfamily{}\fontsize{12}{12}\selectfont}
\setlength{\headheight}{15pt} % Amplía el tamaño del índice


%\usepackage{lastpage}   % Referenciar última pag   \pageref{LastPage}
%\fancyfoot[C]{%
%  \begin{minipage}{\textwidth}
%    \centering
%    ~\\
%    \thepage\\
%    \href{https://losdeldgiim.github.io/}{\texttt{\footnotesize losdeldgiim.github.io}}
%  \end{minipage}
%}
\fancyfoot[C]{\thepage}
\fancyfoot[R]{\href{https://losdeldgiim.github.io/}{\texttt{\footnotesize losdeldgiim.github.io}}}

%------------------------------------------------------------------------

% Conseguir que no ponga "Capítulo 1". Sino solo "1."
\makeatletter
\@ifclassloaded{book}{
  \renewcommand{\chaptermark}[1]{\markboth{\thechapter.\ #1}{}} % En el encabezado
    
  \renewcommand{\@makechapterhead}[1]{%
  \vspace*{50\p@}%
  {\parindent \z@ \raggedright \normalfont
    \ifnum \c@secnumdepth >\m@ne
      \huge\bfseries \thechapter.\hspace{1em}\ignorespaces
    \fi
    \interlinepenalty\@M
    \Huge \bfseries #1\par\nobreak
    \vskip 40\p@
  }}
}
\makeatother

%------------------------------------------------------------------------
% Paquetes de cógido
\usepackage{minted}
\renewcommand\listingscaption{Código fuente}

\usepackage{fancyvrb}
% Personaliza el tamaño de los números de línea
\renewcommand{\theFancyVerbLine}{\small\arabic{FancyVerbLine}}

% Estilo para C++
\newminted{cpp}{
    frame=lines,
    framesep=2mm,
    baselinestretch=1.2,
    linenos,
    escapeinside=||
}

% para minted
\definecolor{LightGray}{rgb}{0.95,0.95,0.92}
\setminted{
    linenos=true,
    stepnumber=5,
    numberfirstline=true,
    autogobble,
    breaklines=true,
    breakautoindent=true,
    breaksymbolleft=,
    breaksymbolright=,
    breaksymbolindentleft=0pt,
    breaksymbolindentright=0pt,
    breaksymbolsepleft=0pt,
    breaksymbolsepright=0pt,
    fontsize=\footnotesize,
    bgcolor=LightGray,
    numbersep=10pt
}


\usepackage{listings} % Para incluir código desde un archivo

\renewcommand\lstlistingname{Código Fuente}
\renewcommand\lstlistlistingname{Índice de Códigos Fuente}

% Definir colores
\definecolor{vscodepurple}{rgb}{0.5,0,0.5}
\definecolor{vscodeblue}{rgb}{0,0,0.8}
\definecolor{vscodegreen}{rgb}{0,0.5,0}
\definecolor{vscodegray}{rgb}{0.5,0.5,0.5}
\definecolor{vscodebackground}{rgb}{0.97,0.97,0.97}
\definecolor{vscodelightgray}{rgb}{0.9,0.9,0.9}

% Configuración para el estilo de C similar a VSCode
\lstdefinestyle{vscode_C}{
  backgroundcolor=\color{vscodebackground},
  commentstyle=\color{vscodegreen},
  keywordstyle=\color{vscodeblue},
  numberstyle=\tiny\color{vscodegray},
  stringstyle=\color{vscodepurple},
  basicstyle=\scriptsize\ttfamily,
  breakatwhitespace=false,
  breaklines=true,
  captionpos=b,
  keepspaces=true,
  numbers=left,
  numbersep=5pt,
  showspaces=false,
  showstringspaces=false,
  showtabs=false,
  tabsize=2,
  frame=tb,
  framerule=0pt,
  aboveskip=10pt,
  belowskip=10pt,
  xleftmargin=10pt,
  xrightmargin=10pt,
  framexleftmargin=10pt,
  framexrightmargin=10pt,
  framesep=0pt,
  rulecolor=\color{vscodelightgray},
  backgroundcolor=\color{vscodebackground},
}

%------------------------------------------------------------------------

% Comandos definidos
\newcommand{\bb}[1]{\mathbb{#1}}
\newcommand{\cc}[1]{\mathcal{#1}}

% I prefer the slanted \leq
\let\oldleq\leq % save them in case they're every wanted
\let\oldgeq\geq
\renewcommand{\leq}{\leqslant}
\renewcommand{\geq}{\geqslant}

% Si y solo si
\newcommand{\sii}{\iff}

% MCD y MCM
\DeclareMathOperator{\mcd}{mcd}
\DeclareMathOperator{\mcm}{mcm}

% Signo
\DeclareMathOperator{\sgn}{sgn}

% Letras griegas
\newcommand{\eps}{\epsilon}
\newcommand{\veps}{\varepsilon}
\newcommand{\lm}{\lambda}

\newcommand{\ol}{\overline}
\newcommand{\ul}{\underline}
\newcommand{\wt}{\widetilde}
\newcommand{\wh}{\widehat}

\let\oldvec\vec
\renewcommand{\vec}{\overrightarrow}

% Derivadas parciales
\newcommand{\del}[2]{\frac{\partial #1}{\partial #2}}
\newcommand{\Del}[3]{\frac{\partial^{#1} #2}{\partial #3^{#1}}}
\newcommand{\deld}[2]{\dfrac{\partial #1}{\partial #2}}
\newcommand{\Deld}[3]{\dfrac{\partial^{#1} #2}{\partial #3^{#1}}}


\newcommand{\AstIg}{\stackrel{(\ast)}{=}}
\newcommand{\Hop}{\stackrel{L'H\hat{o}pital}{=}}

\newcommand{\red}[1]{{\color{red}#1}} % Para integrales, destacar los cambios.

% Método de integración
\newcommand{\MetInt}[2]{
    \left[\begin{array}{c}
        #1 \\ #2
    \end{array}\right]
}

% Declarar aplicaciones
% 1. Nombre aplicación
% 2. Dominio
% 3. Codominio
% 4. Variable
% 5. Imagen de la variable
\newcommand{\Func}[5]{
    \begin{equation*}
        \begin{array}{rrll}
            \displaystyle #1:& \displaystyle  #2 & \longrightarrow & \displaystyle  #3\\
               & \displaystyle  #4 & \longmapsto & \displaystyle  #5
        \end{array}
    \end{equation*}
}

%------------------------------------------------------------------------


\let\oldRe\Re % save them in case they're every wanted
\let\oldIm\Im
\renewcommand{\Re}{\operatorname{Re}} % redefine them
\renewcommand{\Im}{\operatorname{Im}}
\DeclareMathOperator{\Log}{Log}
\DeclareMathOperator{\Arg}{Arg}
\DeclareMathOperator{\ord}{ord}
\DeclareMathOperator{\Ind}{Ind}
\DeclareMathOperator{\Fr}{Fr}
\DeclareMathOperator{\Res}{Res}

\usetikzlibrary{arrows.meta, decorations.markings} % Cargar las bibliotecas necesarias

% Configuración para las flechas
\tikzset{
    arrow at 1/3/.style={postaction={decorate},
        decoration={markings, mark=at position 0.33 with {\arrow{Stealth}}}},
    arrow at 2/3/.style={postaction={decorate},
        decoration={markings, mark=at position 0.66 with {\arrow{Stealth}}}}
}

\begin{document}

    % 1. Foto de fondo
    % 2. Título
    % 3. Encabezado Izquierdo
    % 4. Color de fondo
    % 5. Coord x del titulo
    % 6. Coord y del titulo
    % 7. Fecha
    % 8. Autor

    % 1. Foto de fondo
% 2. Título
% 3. Encabezado Izquierdo
% 4. Color de fondo
% 5. Coord x del titulo
% 6. Coord y del titulo
% 7. Fecha
% 8. Autor

\newcommand{\portada}[8]{
    \portadaBase{#1}{#2}{#3}{#4}{#5}{#6}{#7}{#8}
    \portadaBook{#1}{#2}{#3}{#4}{#5}{#6}{#7}{#8}
}

\newcommand{\portadaFotoDif}[8]{
    \portadaBaseFotoDif{#1}{#2}{#3}{#4}{#5}{#6}{#7}{#8}
    \portadaBook{#1}{#2}{#3}{#4}{#5}{#6}{#7}{#8}
}

\newcommand{\portadaExamen}[8]{
    \portadaBase{#1}{#2}{#3}{#4}{#5}{#6}{#7}{#8}
    \portadaArticle{#1}{#2}{#3}{#4}{#5}{#6}{#7}{#8}
}

\newcommand{\portadaExamenFotoDif}[8]{
    \portadaBaseFotoDif{#1}{#2}{#3}{#4}{#5}{#6}{#7}{#8}
    \portadaArticle{#1}{#2}{#3}{#4}{#5}{#6}{#7}{#8}
}




\newcommand{\portadaBase}[8]{

    % Tiene la portada principal y la licencia Creative Commons
    
    % 1. Foto de fondo
    % 2. Título
    % 3. Encabezado Izquierdo
    % 4. Color de fondo
    % 5. Coord x del titulo
    % 6. Coord y del titulo
    % 7. Fecha
    % 8. Autor    
    
    \thispagestyle{empty}               % Sin encabezado ni pie de página
    \newgeometry{margin=0cm}        % Márgenes nulos para la primera página
    
    
    % Encabezado
    \fancyhead[L]{\helv #3}
    \fancyhead[R]{\helv \nouppercase{\leftmark}}
    
    
    \pagecolor{#4}        % Color de fondo para la portada
    
    \begin{figure}[p]
        \centering
        \transparent{0.3}           % Opacidad del 30% para la imagen
        
        \includegraphics[width=\paperwidth, keepaspectratio]{../../_assets/#1}
    
        \begin{tikzpicture}[remember picture, overlay]
            \node[anchor=north west, text=white, opacity=1, font=\fontsize{60}{90}\selectfont\bfseries\sffamily, align=left] at (#5, #6) {#2};
            
            \node[anchor=south east, text=white, opacity=1, font=\fontsize{12}{18}\selectfont\sffamily, align=right] at (9.7, 3) {\href{https://losdeldgiim.github.io/}{\textbf{Los Del DGIIM}, \texttt{\footnotesize losdeldgiim.github.io}}};
            
            \node[anchor=south east, text=white, opacity=1, font=\fontsize{12}{15}\selectfont\sffamily, align=right] at (9.7, 1.8) {Doble Grado en Ingeniería Informática y Matemáticas\\Universidad de Granada};
        \end{tikzpicture}
    \end{figure}
    
    
    \restoregeometry        % Restaurar márgenes normales para las páginas subsiguientes
    \nopagecolor      % Restaurar el color de página
    
    
    \newpage
    \thispagestyle{empty}               % Sin encabezado ni pie de página
    \begin{tikzpicture}[remember picture, overlay]
        \node[anchor=south west, inner sep=3cm] at (current page.south west) {
            \begin{minipage}{0.5\paperwidth}
                \href{https://creativecommons.org/licenses/by-nc-nd/4.0/}{
                    \includegraphics[height=2cm]{../../_assets/Licencia.png}
                }\vspace{1cm}\\
                Esta obra está bajo una
                \href{https://creativecommons.org/licenses/by-nc-nd/4.0/}{
                    Licencia Creative Commons Atribución-NoComercial-SinDerivadas 4.0 Internacional (CC BY-NC-ND 4.0).
                }\\
    
                Eres libre de compartir y redistribuir el contenido de esta obra en cualquier medio o formato, siempre y cuando des el crédito adecuado a los autores originales y no persigas fines comerciales. 
            \end{minipage}
        };
    \end{tikzpicture}
    
    
    
    % 1. Foto de fondo
    % 2. Título
    % 3. Encabezado Izquierdo
    % 4. Color de fondo
    % 5. Coord x del titulo
    % 6. Coord y del titulo
    % 7. Fecha
    % 8. Autor


}


\newcommand{\portadaBaseFotoDif}[8]{

    % Tiene la portada principal y la licencia Creative Commons
    
    % 1. Foto de fondo
    % 2. Título
    % 3. Encabezado Izquierdo
    % 4. Color de fondo
    % 5. Coord x del titulo
    % 6. Coord y del titulo
    % 7. Fecha
    % 8. Autor    
    
    \thispagestyle{empty}               % Sin encabezado ni pie de página
    \newgeometry{margin=0cm}        % Márgenes nulos para la primera página
    
    
    % Encabezado
    \fancyhead[L]{\helv #3}
    \fancyhead[R]{\helv \nouppercase{\leftmark}}
    
    
    \pagecolor{#4}        % Color de fondo para la portada
    
    \begin{figure}[p]
        \centering
        \transparent{0.3}           % Opacidad del 30% para la imagen
        
        \includegraphics[width=\paperwidth, keepaspectratio]{#1}
    
        \begin{tikzpicture}[remember picture, overlay]
            \node[anchor=north west, text=white, opacity=1, font=\fontsize{60}{90}\selectfont\bfseries\sffamily, align=left] at (#5, #6) {#2};
            
            \node[anchor=south east, text=white, opacity=1, font=\fontsize{12}{18}\selectfont\sffamily, align=right] at (9.7, 3) {\href{https://losdeldgiim.github.io/}{\textbf{Los Del DGIIM}, \texttt{\footnotesize losdeldgiim.github.io}}};
            
            \node[anchor=south east, text=white, opacity=1, font=\fontsize{12}{15}\selectfont\sffamily, align=right] at (9.7, 1.8) {Doble Grado en Ingeniería Informática y Matemáticas\\Universidad de Granada};
        \end{tikzpicture}
    \end{figure}
    
    
    \restoregeometry        % Restaurar márgenes normales para las páginas subsiguientes
    \nopagecolor      % Restaurar el color de página
    
    
    \newpage
    \thispagestyle{empty}               % Sin encabezado ni pie de página
    \begin{tikzpicture}[remember picture, overlay]
        \node[anchor=south west, inner sep=3cm] at (current page.south west) {
            \begin{minipage}{0.5\paperwidth}
                %\href{https://creativecommons.org/licenses/by-nc-nd/4.0/}{
                %    \includegraphics[height=2cm]{../../_assets/Licencia.png}
                %}\vspace{1cm}\\
                Esta obra está bajo una
                \href{https://creativecommons.org/licenses/by-nc-nd/4.0/}{
                    Licencia Creative Commons Atribución-NoComercial-SinDerivadas 4.0 Internacional (CC BY-NC-ND 4.0).
                }\\
    
                Eres libre de compartir y redistribuir el contenido de esta obra en cualquier medio o formato, siempre y cuando des el crédito adecuado a los autores originales y no persigas fines comerciales. 
            \end{minipage}
        };
    \end{tikzpicture}
    
    
    
    % 1. Foto de fondo
    % 2. Título
    % 3. Encabezado Izquierdo
    % 4. Color de fondo
    % 5. Coord x del titulo
    % 6. Coord y del titulo
    % 7. Fecha
    % 8. Autor


}


\newcommand{\portadaBook}[8]{

    % 1. Foto de fondo
    % 2. Título
    % 3. Encabezado Izquierdo
    % 4. Color de fondo
    % 5. Coord x del titulo
    % 6. Coord y del titulo
    % 7. Fecha
    % 8. Autor

    % Personaliza el formato del título
    \pretitle{\begin{center}\bfseries\fontsize{42}{56}\selectfont}
    \posttitle{\par\end{center}\vspace{2em}}
    
    % Personaliza el formato del autor
    \preauthor{\begin{center}\Large}
    \postauthor{\par\end{center}\vfill}
    
    % Personaliza el formato de la fecha
    \predate{\begin{center}\huge}
    \postdate{\par\end{center}\vspace{2em}}
    
    \title{#2}
    \author{\href{https://losdeldgiim.github.io/}{Los Del DGIIM, \texttt{\large losdeldgiim.github.io}}
    \\ \vspace{0.5cm}#8}
    \date{Granada, #7}
    \maketitle
    
    \tableofcontents
}




\newcommand{\portadaArticle}[8]{

    % 1. Foto de fondo
    % 2. Título
    % 3. Encabezado Izquierdo
    % 4. Color de fondo
    % 5. Coord x del titulo
    % 6. Coord y del titulo
    % 7. Fecha
    % 8. Autor

    % Personaliza el formato del título
    \pretitle{\begin{center}\bfseries\fontsize{42}{56}\selectfont}
    \posttitle{\par\end{center}\vspace{2em}}
    
    % Personaliza el formato del autor
    \preauthor{\begin{center}\Large}
    \postauthor{\par\end{center}\vspace{3em}}
    
    % Personaliza el formato de la fecha
    \predate{\begin{center}\huge}
    \postdate{\par\end{center}\vspace{5em}}
    
    \title{#2}
    \author{\href{https://losdeldgiim.github.io/}{Los Del DGIIM, \texttt{\large losdeldgiim.github.io}}
    \\ \vspace{0.5cm}#8}
    \date{Granada, #7}
    \thispagestyle{empty}               % Sin encabezado ni pie de página
    \maketitle
    \vfill
}
    \portada{ffccA4.jpg}{Inferencia\\Estadística}{Inferencia Estadística}{MidnightBlue}{-8}{28}{2025}{José Juan Urrutia Milán}
    
    %\section{Estadísticos muestrales}

\begin{ejercicio}
    Sea $(X_1, \ldots, X_n)$ una muestra aleatoria simple de una variable aleatoria $X$. Dar el espacio muestral y calcular la función masa de probabilidad de $(X_1, \ldots, X_n)$ en cada uno de los siguientes casos:
    \begin{enumerate}[label=\alph*)]
        \item $X\rightsquigarrow \{B(k_0,p) : p\in (0,1)\}$ Binomial.

            El espacio muestral en este caso es $\cc{X}^n$, donde:
            \begin{equation*}
                \cc{X} = \{0, 1, ..., k_0\}
            \end{equation*}
            Recordamos que si $X\rightsquigarrow B(k_0,p)$, entonces:
            \begin{equation*}
                P[X=x] = \binom{k_0}{x} p^x {(1-p)}^{k_0-x} \qquad \forall x\in \cc{X}
            \end{equation*}
            Por tanto, para nuestra m.a.s. tendremos la función masa de probabilidad:
            \begin{align*}
                P[X_1 &= x_1, \ldots, X_n = x_n] \stackrel{\text{indep.}}{=} \prod_{i=1}^{n}P[X_i = x_i]\stackrel{\text{id. d.}}{=} \prod_{i=1}^{n} P[X=x_i] \\
                      &= \prod_{i=1}^{n} \binom{k_0}{x_i} p^{x_i} {(1-p)}^{k_0-x_i} = p^{\sum\limits_{i=1}^{n}x_i} {(1-p)}^{nk_0 - \sum\limits_{i=1}^{n}x_i} \prod_{i=1}^{n}\binom{k_0}{x_i} \\
                      & \qquad \forall (x_1, \ldots, x_n) \in \cc{X}^n
            \end{align*}
        \item $X\rightsquigarrow\{\cc{P}(\lm) : \lm \in \mathbb{R}^+\}$ Poisson.

            El espacio muestral de $X$ es:
            \begin{equation*}
                \cc{X} = \mathbb{N} \cup \{0\}
            \end{equation*} 
            Recordamos que si $X\rightsquigarrow \cc{P}(\lm)$, entonces:
            \begin{equation*}
                P[X=x] = e^{-\lm} \dfrac{\lm^x}{x!} \qquad \forall x\in \cc{X}
            \end{equation*}
            Por tanto:
            \begin{align*}
                P[X_1 &= x_1, \ldots, X_n = x_n] \stackrel{\text{indep.}}{=} \prod_{i=1}^{n}P[X_i = x_i]\stackrel{\text{id. d.}}{=} \prod_{i=1}^{n} P[X=x_i] \\
                      &= \prod_{i=1}^{n} e^{-\lm} \dfrac{\lm^{x_i}}{x_i!} = e^{-n\lm} \prod_{i=1}^{n} \dfrac{\lm^{x_i}}{x_i!} = e^{-n\lm} \cdot \dfrac{\lm^{\sum\limits_{i=1}^n x_i}}{\prod\limits_{i=1}^{n}x_i}  \qquad \forall (x_1, \ldots, x_n) \in \cc{X}^n
            \end{align*}
        \item $X\rightsquigarrow\{BN(k_0,p) : p\in (0,1)\}$ Binomial Negativa.

            El espacio muestral de $X$ es:
            \begin{equation*}
                \cc{X} = \mathbb{N}\cup \{0\}
            \end{equation*}
            Recordamos que si $X\rightsquigarrow BN(k_0,p)$, entonces:
            \begin{equation*}
                P[X=x] = \binom{x+k_0-1}{x} {(1-p)}^{x}p^{k_0} \qquad \forall x\in \cc{X}
            \end{equation*}
            Por tanto:
            \begin{align*}
                P[X_1 &= x_1, \ldots, X_n = x_n] \stackrel{\text{indep.}}{=} \prod_{i=1}^{n}P[X_i = x_i]\stackrel{\text{id. d.}}{=} \prod_{i=1}^{n} P[X=x_i] \\
                      &= \prod_{i=1}^{n} \binom{x_i+k_0-1}{x_i}{(1-p)}^{x_i}p^{k_0} = p^{nk_0}{(1-p)}^{\sum\limits_{i=1}^n x_i} \prod_{i=1}^{n}\binom{x_i+k_0-1}{x_i} \\
                      &\forall (x_1,\ldots,x_n)\in \cc{X}^n
            \end{align*}
        \item $X\rightsquigarrow\{G(p) : p\in (0,1)\}$ Geométrica.

            El espacio muestral de $X$ es:
            \begin{equation*}
                \cc{X} = \mathbb{N}\cup \{0\}
            \end{equation*}
            Recordamos que $G(p)\equiv BN(1,p)$, por lo que si sustituimos en la fórmula obtenida en la Binomial Negativa $k_0 = 1$:
            \begin{equation*}
                P[X_1=x_1, \ldots, X_n = x_n] = p^{n}{(1-p)}^{\sum\limits_{i=1}^n x_i} \qquad \forall (x_1,\ldots,x_n)\in \cc{X}^n
            \end{equation*}
        \item $X\rightsquigarrow\{P_N : N\in \mathbb{N}\}$, $\quad P_N(X=x) = \dfrac{1}{N}$, $\quad x=1,\ldots,N$.

            El espacio muestral ya nos lo dan: $\cc{X} = \{1,\ldots,N\}$. Calculemos la masa de probabilidad:
            \begin{align*}
                P[X_1 &= x_1, \ldots, X_n = x_n] \stackrel{\text{indep.}}{=} \prod_{i=1}^{n}P[X_i = x_i]\stackrel{\text{id. d.}}{=} \prod_{i=1}^{n} P[X=x_i] \\
                      &= \prod_{i=1}^{n} \dfrac{1}{N} = {\left(\dfrac{1}{N}\right)}^{n} \qquad \forall (x_1,\ldots,x_n) \in \cc{X}^n
            \end{align*}
    \end{enumerate}
\end{ejercicio}

\begin{ejercicio}
    Sea $(X_1, \ldots, X_n)$ una muestra aleatoria simple de una variable aleatoria $X$. Dar el espacio muestral y calcular la función de densidad de $(X_1, \ldots, X_n)$ en cada uno de los siguientes casos:
    \begin{enumerate}[label=\alph*)]
        \item $X\rightsquigarrow\{U(a,b) : a,b\in \mathbb{R}, a<b\}$ Uniforme.

            El espcio muestral en este caso es $\cc{X}^n$, donde:
            \begin{equation*}
                \cc{X} = [a,b]
            \end{equation*}
            Recordamos que si $X\rightsquigarrow U(a,b)$, entonces:
            \begin{equation*}
                f_X(x) = \dfrac{1}{b-a} \qquad \forall x\in [a,b]
            \end{equation*}
            Por lo que:
            \begin{align*}
                f_{(X_1, \ldots, X_n)}(x_1, \ldots, x_n) &\stackrel{\text{indep.}}{=} \prod_{i=1}^{n} f_{X_i}(x_i) \stackrel{\text{id. d.}}{=} \prod_{i=1}^{n} f_X(x_i) = \prod_{i=1}^{n} \dfrac{1}{b-a} \\ &= {\left(\dfrac{1}{b-a}\right)}^{n} \qquad \forall (x_1, \ldots, x_n) \in \cc{X}^n
            \end{align*}

        \item $X\rightsquigarrow\{\cc{N}(\mu, \sigma^2) : \mu \in \mathbb{R}, \sigma^2 \in \mathbb{R}^+\}$ Normal.

            El espacio muestral de $X$ es $\cc{X} = \mathbb{R}$. Recordamos que si $X\rightsquigarrow \cc{N}(\mu, \sigma^2)$, entonces:
            \begin{equation*}
                f_X(x) = \dfrac{1}{\sqrt{2\pi} \sigma} e^{-\dfrac{{(x-\mu)}^{2}}{2\sigma^2}} \qquad \forall x\in \mathbb{R}
            \end{equation*}
            Por lo que:
            \begin{align*}
                f_{(X_1, \ldots, X_n)}(x_1, \ldots, x_n) &\stackrel{\text{indep.}}{=} \prod_{i=1}^{n} f_{X_i}(x_i) \stackrel{\text{id. d.}}{=} \prod_{i=1}^{n} f_X(x_i) 
                = \prod_{i=1}^{n} \dfrac{1}{\sqrt{2\pi} \sigma} e^{-\dfrac{{(x_i-\mu)}^{2}}{2\sigma^2}}  \\
                &= {\left(\dfrac{1}{\sqrt{2\pi}\sigma}\right)}^{n} \prod_{i=1}^{n} e^{-\dfrac{{(x_i-\mu)}^{2}}{2\sigma^2}}  = {\left(\dfrac{1}{\sqrt{2\pi}\sigma}\right)}^{n} e^{-\sum\limits_{i=1}^n\dfrac{{(x_i-\mu)}^{2}}{2\sigma^2}}   \\
                &= {\left(\dfrac{1}{\sqrt{2\pi}\sigma}\right)}^{n} e^{\frac{-1}{2\sigma^2}\sum\limits_{i=1}^n {(x_i-\mu)}^{2}} \qquad \forall (x_1,\ldots,x_n) \in \mathbb{R}^n
            \end{align*}
        \item $X\rightsquigarrow\{\Gamma(p,a) : p,a\in \mathbb{R}^+\}$ Gamma.

            El espacio muestral de $X$ es $\cc{X}=\mathbb{R}^+_0$. Recordamos que si $X\rightsquigarrow \Gamma(p,a)$, entonces:
            \begin{equation*}
                f_X(x) = \dfrac{a^p}{\Gamma(p)} x^{p-1} e^{-ax} \qquad \forall x\in \mathbb{R}^+_0
            \end{equation*}
            Por lo que:
            \begin{align*}
                f_{(X_1, \ldots, X_n)}(x_1, \ldots, x_n) &\stackrel{\text{indep.}}{=} \prod_{i=1}^{n} f_{X_i}(x_i) \stackrel{\text{id. d.}}{=} \prod_{i=1}^{n} f_X(x_i) 
                = \prod_{i=1}^{n} \dfrac{a^p}{\Gamma(p)} x_i^{p-1} e^{-ax_i} \\
                                                         &= {\left(\dfrac{a^p}{\Gamma(p)}\right)}^{n} \cdot e^{-a\sum\limits_{i=1}^n x_i} \cdot \prod_{i=1}^{n} x_i^{p-1} \qquad \forall (x_1,\ldots,x_n)\in \cc{X}^n
            \end{align*}
        \item $X\rightsquigarrow\{\beta(p,q) : p,q\in \mathbb{R}^+\}$ Beta.

            El espacio muestral de $X$ es $\cc{X} = [0,1]$. Recordamos que si $X\rightsquigarrow \beta(p,q)$, entonces:
            \begin{equation*}
                f_X(x) = \dfrac{1}{\beta(p,q)} x^{p-1} {(1-x)}^{q-1} \qquad \forall x\in [0,1]
            \end{equation*}
            Donde:
            \begin{equation*}
                \beta(p,q) = \dfrac{\Gamma(p)\Gamma(q)}{\Gamma(p+q)}
            \end{equation*}
            Por tanto:
            \begin{align*}
                f_{(X_1, \ldots, X_n)}(x_1, \ldots, x_n) &\stackrel{\text{indep.}}{=} \prod_{i=1}^{n} f_{X_i}(x_i) \stackrel{\text{id. d.}}{=} \prod_{i=1}^{n} f_X(x_i) 
                = \prod_{i=1}^{n} \dfrac{1}{\beta(p,q)} x_i^{p-1} {(1-x_i)}^{q-1} \\
                                                         &= \dfrac{1}{{\beta(p,q)}^{n}} \prod_{i=1}^{n}x_i^{p-1} {(1-x_i)}^{q-1} \qquad \forall (x_1,\ldots,x_n) \in \cc{X}^n
            \end{align*}
        \item $X\rightsquigarrow\{P_\theta : \theta \in \mathbb{R}^+\}$, $\quad f_\theta(x) = \dfrac{1}{2\sqrt{x\theta}}$, $\quad 0<x<\theta$.

            Se nos dice que $\cc{X} = \left]0,\theta\right[$. Calculamos la función de densidad conjunta:
            \begin{align*}
                f_{(X_1, \ldots, X_n)}(x_1, \ldots, x_n) &\stackrel{\text{indep.}}{=} \prod_{i=1}^{n} f_{X_i}(x_i) \stackrel{\text{id. d.}}{=} \prod_{i=1}^{n} f_X(x_i) 
                = \prod_{i=1}^{n} \dfrac{1}{2\sqrt{x_i \theta}} \\
                                                         &= \dfrac{1}{{\left(2\sqrt{\theta}\right)}^{n}} \prod_{i=1}^{n} \dfrac{1}{\sqrt{x_i}} \qquad \forall (x_1, \ldots, x_n) \in  \cc{X}^n
            \end{align*}
    \end{enumerate}
\end{ejercicio}

\begin{ejercicio}
    Se miden los tiempos de sedimentación de una muestra de partículas flotando en un líquido. Los tiempos observados son: 
    \begin{gather*}
        11.5; 1.8; 7.3; 12.1; 1.8; 21.3; 7.3; 15.2; 7.3; 12.1; 15.2;\\ 7.3; 12.1; 1.8; 10.5; 15.2; 21.3; 10.5; 15.2; 11.5
    \end{gather*}
    \begin{itemize}
        \item Construir la función de distribución muestral asociada a a dichas observaciones.

            Si aplicamos la definición de función de distribución muestral obtenemos que esta viene dada por:
            \begin{equation*}
                F_n^\ast(x) =
                \begin{cases} 
                0 & \text{si\ } x < 1.8 \\[6pt]
                \nicefrac{3}{20} &\text{si\ } 1.8 \leq x < 7.3 \\[6pt]
                \nicefrac{7}{20} &\text{si\ } 7.3 \leq x < 10.5 \\[6pt]
                \nicefrac{9}{20} &\text{si\ } 10.5 \leq x < 11.5 \\[6pt]
                \nicefrac{11}{20} &\text{si\ } 11.5 \leq x < 12.1 \\[6pt]
                \nicefrac{14}{20} &\text{si\ } 12.1 \leq x < 15.2 \\[6pt]
                \nicefrac{18}{20} &\text{si\ } 15.2 \leq x < 21.3 \\[6pt]
                \nicefrac{20}{20} &\text{si\ } x \geq 21.3
                \end{cases}
            \end{equation*}

            \begin{figure}[H]
                \centering
                \begin{tikzpicture}
                \begin{axis}[
                    width=12cm, height=7cm, xlabel={$x$}, ylabel={$F_n^\ast(x)$}, ymin=0, ymax=1.1,
                    xmin=0, xmax=23, xtick={0,1.8,7.3,10.5,12.1,15.2,21.3,23},
                    ytick={0,0.1,0.2,0.3,0.4,0.5,0.6,0.7,0.8,0.9,1},
                    grid=both, domain=0:23, samples=200,
                ]

                % Graficamos la función escalonada
                \addplot[
                    thick, blue
                ] coordinates {
                    (0,0) (1.8,0) (1.8,3/20) (7.3,3/20) (7.3,7/20) (10.5,7/20)
                    (10.5,9/20) (11.5,9/20) (11.5,11/20) (12.1,11/20) (12.1,14/20)
                    (15.2,14/20) (15.2,18/20) (21.3,18/20) (21.3,20/20) (23,20/20)
                };
                \end{axis}
                \end{tikzpicture}
                \caption{Gráfica de la función de distribución muestral.}
            \end{figure}
        \item Hallar los valores de los tres primeros momentos muestrales respecto al origen y respecto a la media.

            Calculamos primero los tres primeros momentos respecto al origen para luego calcular los centrados respecto a la media a partir de ellos:
            \begin{align*}
                a_1 &= \sum_{i=1}^{n} f_i x_i = 10.915 \qquad 
                a_2 = \sum_{i=1}^{n} f_i x_i^2 = 148.9325 \\
                a_3 &= \sum_{i=1}^{n} f_i x_i^3 = 2280.98365 \\
                b_1 &= \sum_{i=1}^{n} f_i {(x_i - \overline{x})}= 0\qquad  \\
                b_2 &= \frac{1}{n}\sum_{i=1}^{n}{(x_i-\overline{x})}^{2} = \frac{1}{n}\sum_{i=1}^{n}(x_i^2 - 2x_i \overline{x}+\overline{x}^2)  \\ &= \frac{1}{n}\sum_{i=1}^{n}x_i^2 - \frac{2\overline{x}}{n}\sum_{i=1}^{n}x_i + \overline{x}^2 = a_2 -2a_1^2 + a_1^2 = a_2 - a_1^2 \\
                    &= 148.9325 - 10.915  = 29.795275\\
                b_3 &= \frac{1}{n}\sum_{i=1}^{n} {(x_i-\overline{x})}^{3} = \frac{1}{n}\sum_{i=1}^{n}\left(x_i^3 - 3x_i^2 \overline{x} + 3x_i\overline{x}^2 -\overline{x}^3\right) \\
                    &= \frac{1}{n}\sum_{i=1}^{n}x_i^3 - \frac{3\overline{x}}{n}\sum_{i=1}^{n}x_i^2 + \frac{3\overline{x}^2}{n}\sum_{i=1}^{n}x_i - \overline{x}^3 = a_3 - 3a_1a_2 + 3a_1^3 - a_1^3 \\
                    &= a_3 - 3a_1a_2 + 2a_1^3 = 4.95455925
            \end{align*}
        \item Determinar los valores de los cuartiles muestrales y el percentil 70.

            Para ello, primero ordenamos los datos de menor a mayor y los agrupamos en grupos de $\nicefrac{20}{4} = 5$ en 5:
            \begin{gather*}
                1.8;\ 1.8;\ 1.8;\ 7.3;\ 7.3;\ \red{7.3;\ 7.3;\ 10.5;\ 10.5;\ 11.5};\ 11.5;\ 12.1;\ 12.1;\\ 12.1;\ 15.2;\ \red{15.2;\ 15.2;\ 15.2;\ 21.3;\ 21.3}
            \end{gather*}
            Como en los cambios de agrupaciones de números estos se repiten, hemos obtenido el valor de los cuartiles:
            \begin{equation*}
                q_1 = 7.3 \qquad q_2 = 11.5 \qquad q_3 = 15.2 \qquad q_4 = 21.3
            \end{equation*}
            Para el percentil $70$, calculamos:
            \begin{equation*}
                0.7\cdot 20 = 14
            \end{equation*}
            Como hemos obtenido un número entero, el percentil 70 será:
            \begin{equation*}
                c_{70} = \dfrac{X_{(14)} + X_{(15)}}{2} = \dfrac{12.1 + 15.2}{2} = 13.65
            \end{equation*}
            En el caso de haber obtenido un número no entero (por ejemplo, $14.2$), sería $X_{(15)}$.
    \end{itemize}
\end{ejercicio}

\begin{ejercicio}
    Se dispone de una muestra aleatoria simple de tamaño 40 de una distribución exponencial de media 3, ¿cuál es la probabilidad de que los valores de la función de distribución muestral y la teórica, en $x=1$, difieran menos de $0.01$? Aproximadamente, ¿cuál debe ser el tamaño muestral para que dicha probabilidad sea como mínimo $0.98$?\\

    \noindent
    Como dice el enunciado, tenemos una m.a.s. $(X_1, \ldots, X_n)$ con $n=40$, todas ellas idénticamente distribuidas a $X\rightsquigarrow exp(\lm)$. Sabemos de la asignatura de Probabilidad que:
    \begin{equation*}
        E[X] = \dfrac{1}{\lm} = 3 \Longrightarrow \lm = \dfrac{1}{3}
    \end{equation*}
    Por lo que $X\rightsquigarrow exp\left(\frac{1}{3}\right)$. Denotaremos por comodidad:
    \begin{equation*}
        F_n^\ast(x) = F_{(X_1, \ldots, X_n)}^\ast(x) 
    \end{equation*}
    Y el enunciado nos pregunta por:
    \begin{equation*}
        P[|F_n^\ast(1) - F_X(1)| < 0.01]
    \end{equation*}
    Para ello, primero calculamos $F_X(1)$:
    \begin{equation*}
        F_X(1) = 1 - e^{-\lm\cdot  1} = 1-e^{-\lm} = 1-e^{-\nicefrac{1}{3}} = \alpha
    \end{equation*}
    Por lo que nos disponemos ya a calcular la probabilidad:
    \begin{align*}
        P[|F_n^\ast(1) - F_X(1)| < 0.01] &= P[|F_n^\ast(1) - \alpha| < 0.01] = P[-0.01 < F_n^\ast(1) - \alpha < 0.01] \\
                                         &= P[-0.01+\alpha < F_n^\ast(1) < 0.01+\alpha] \\
                                         &= P[40(-0.01+\alpha) < 40F_n^\ast(1) < 40(0.01+\alpha)]
    \end{align*}
    Y como sabemos que $Y = 40F_n^\ast(1) \rightsquigarrow B(40, F_X(1)) \equiv B(40, \alpha)$:
    \begin{equation*}
        P[|F_n^\ast(1) - F_X(1)| < 0.01] = P[40(-0.01+\alpha) < Y< 40(0.01+\alpha)] 
    \end{equation*}
    Si ahora tomamos:
    \begin{equation*}
        \alpha = 1-e^{-\nicefrac{1}{3}}\approx 0.283469
    \end{equation*}
    Entonces:
    \begin{align*}
        40(0.01 + \alpha) &\approx 40(0.01+0.283469)= 11.73876 \\
        40(-0.01 + \alpha) &\approx 40(-0.01 + 0.283469)  = 10.93876
    \end{align*}
    Por lo que:
    \begin{align*}
        P[|F_n^\ast(1) - F_X(1)| < 0.01] &\approx P[10.93876 < Y < 11.73876] = P[Y=11]
    \end{align*}
    De donde usando la masa de probabilidad de la Binomial:
    \begin{equation*}
        P[Y=11] = \binom{40}{11} {(0.283469)}^{11} {(1-0.283469)}^{40-11} \approx 0.139
    \end{equation*}

    \noindent
    Para el segundo apartado, como para $n=40$ obtenemos una probabilidad de $0.139$, podemos intuir que para que dicha probabilidad sea como mínimo $0.98$, nos es necesario un valor de $n$ grande, por lo que podemos suponer que:
    \begin{equation*}
        F_n^\ast(1) \rightsquigarrow \cc{N}\left(\alpha, \dfrac{\alpha(1-\alpha)}{n}\right)
    \end{equation*}
    De donde:
    \begin{equation*}
        Z = \dfrac{\sqrt{n}(F_n^\ast(1)-\alpha)}{\sqrt{\alpha(1-\alpha)}} \rightsquigarrow \cc{N}(0,1)
    \end{equation*}
    Buscamos el valor de $n$ que verifica:
    \begin{equation*}
        0.98 \leq P[|F_n^\ast(1) - F_X(1)| < 0.01] = P\left[|Z| < \dfrac{\sqrt{n}0.01}{\sqrt{\alpha(1-\alpha)}}\right]
    \end{equation*}
    Si aplicamos propiedades conocidas de la Normal, si $a\in \mathbb{R}$, entonces:
    \begin{equation*}
        P[|Z| < a] = P[-a < Z < a] = P[Z<a] - P[Z< -a]
    \end{equation*}
    Pero:
    \begin{equation*}
        P[Z< -a] = P[Z>a] = 1-P[Z-a]
    \end{equation*}
    Por lo que:
    \begin{equation*}
        P[|Z| < a] = P[Z<a] - P[Z < -a] = 2P[Z<a] - 1
    \end{equation*}
    Volviendo al caso que nos interesa:
    \begin{equation*}
        0.98 \leq P\left[|Z| < \dfrac{\sqrt{n}0.01}{\sqrt{\alpha(1-\alpha)}}\right] = 2P\left[Z<\dfrac{\sqrt{n}0.01}{\sqrt{\alpha(1-\alpha)}}\right] - 1
    \end{equation*}
    Luego:
    \begin{equation*}
        0.99 = \dfrac{0.98+1}{2} \leq P\left[Z < \dfrac{\sqrt{n}0.01}{\sqrt{\alpha(1-\alpha)}}\right]
    \end{equation*}
    Si consultamos la tabla de la normal $\cc{N}(0,1)$, observamos que el primer valor que supera la probbailidad de $0.99$ es $2.33$, por lo que:
    \begin{equation*}
        2.33 = \dfrac{\sqrt{n}0.01}{\sqrt{\alpha(1-\alpha)}} = \dfrac{\sqrt{n}0.01}{\sqrt{0.283469(1-0.283469)}} \approx 0.0221886 \sqrt{n}
    \end{equation*}
    De donde:
    \begin{equation*}
        5.4289 = {(2.33)}^{2} = {(0.0221886\sqrt{n})}^{2} = 0.00049233 n \Longrightarrow n = \dfrac{5.4289}{0.00049233} = 11026.95347
    \end{equation*}
    Por lo que para $n \geq 11027$ podemos asegurar que la probabilidad es como mínimo $0.98$.
\end{ejercicio}

\begin{ejercicio}
    Se dispone de una muestra aleatoria simple de tamaño 50 de una distribución de Poisson de media 2, ¿cuál es la probabilidad de que los valores de la función de distribución muestral y la teórica, en $x=2$, difieran menos de $0.02$? Aproximadamente, ¿qué tamaño muestral hay que tomar para que dicha probabilidad sea como mínimo $0.99$?\\

    \noindent
    Tenemos una m.a.s. $(X_1, \ldots, X_n)$ con $n=50$ idénticamente distribuidas a $X\rightsquigarrow \cc{P}(2)$. Notamos por comodidad:
    \begin{equation*}
        F_{(X_1, \ldots, X_n)}^\ast(x) = F_n^\ast(x)
    \end{equation*}
    Nos preguntan por:
    \begin{equation*}
        P[|F_n^\ast(2) - F_X(2)| < 0.02]
    \end{equation*}
    Para ello primero calculamos:
    \begin{equation*}
        F_X(2) = \sum_{k=0}^{2} e^{-2}\dfrac{2^k}{k!} =  e^{-2}\left(\dfrac{2^0}{0!} + \dfrac{2^1}{1!} + \dfrac{2^2}{2!}\right) = e^{-2} (1+2+2) \approx 0.6767
    \end{equation*}
    Por lo que:
    \begin{equation*}
        P[|F_n^\ast(2) - 0.6767| < 0.02] = P[-0.02 < F_n^\ast(2) - 0.6767 < 0.02] = P[0.6567 < F_n^\ast(2) < 0.6967]
    \end{equation*}
    Como sabemos por lo visto en teoría que:
    \begin{equation*}
        Y = 50F_n^\ast(2) \rightsquigarrow B(50, F_X(2)) \equiv B(50,\ \ 0.6767)
    \end{equation*}
    Multiplicamos por $50$ la última expresión:
    \begin{align*}
        P[|F_n^\ast(2) - 0.6767| < 0.02] = P[0.6567 < F_n^\ast(2) < 0.6967] &= P[32.835 < Y< 34.835] \\
                                         &= P[Y = 33] + P[Y=34]
    \end{align*}
    Y calculamos estas dos probabilidades:
    \begin{align*}
        P[Y=33] &= \binom{50}{33} {(0.6767)}^{33} {(1-0.6767)}^{50-33} \approx 0.114734\\
        P[Y=34] &= \binom{50}{34} {(0.6767)}^{34} {(1-0.6767)}^{50-34} \approx 0.120075
    \end{align*}
    Por lo que:
    \begin{equation*}
        P[|F_n^\ast(2) - 0.6767| < 0.02] \approx 0.114734 + 0.120075 = 0.234809
    \end{equation*}

    \noindent
    Para el segundo apartado, como para $n=50$ obtenemos una probabilidad de $0.234809$, podemos intuir que para que dicha probabilidad sea como mínimo $0.99$, nos es necesario un valor de $n$ grande, por lo que podemos suponer que:
    \begin{equation*}
        F_n^\ast(2) \rightsquigarrow \cc{N}\left(0.6767, \dfrac{0.6767(1-0.6767)}{n}\right) \equiv \cc{N}\left(0.6767, \dfrac{0.218777}{n}\right)
    \end{equation*}
    Por lo que:
    \begin{equation*}
        Z = \dfrac{\sqrt{n}(F_n^\ast(2) - 0.6767)}{\sqrt{0.218777}} \rightsquigarrow\cc{N}(0,1)
    \end{equation*}
    En dicho caso, buscamos $n$ de forma que:
    \begin{equation*}
        0.99 \leq P[|F_n^\ast(2) - F_X(2)| < 0.02] = P\left[|Z| < \dfrac{\sqrt{n}0.02}{\sqrt{0.218777}}\right]
    \end{equation*}
    De forma análoga al ejercicio anterior:
    \begin{equation*}
        P\left[|Z| < \dfrac{\sqrt{n}0.02}{\sqrt{0.218777}}\right] = 2P\left[Z < \dfrac{\sqrt{n}0.02}{\sqrt{0.218777}}\right] - 1
    \end{equation*}
    Luego:
    \begin{equation*}
        0.995 = \dfrac{0.99 + 1}{2} \leq P\left[Z < \dfrac{\sqrt{n}0.02}{\sqrt{0.218777}}\right] 
    \end{equation*}
    Y si miramos la tabla de la Normal observamos que el primer valor que supera la probabilidad de $0.995$ es $2.58$, luego:
    \begin{equation*}
        2.58 = \dfrac{\sqrt{n}0.02}{\sqrt{0.218777}} = 0.042759 \sqrt{n}
    \end{equation*}
    Por lo que:
    \begin{equation*}
        6.6564 = {(2.58)}^{2} = {(0.042759 \sqrt{n})}^{2} = 0.00182833 n
    \end{equation*}
    Luego:
    \begin{equation*}
        n = \dfrac{6.6564}{0.00182833} \approx 3640.7
    \end{equation*}
    Por lo que para $n\geq 3641$ podemos asegurar que la probabilidad es como mínimo $0.99$.
\end{ejercicio}

\begin{ejercicio}
   Sea $X\rightsquigarrow B(1,p)$ y $(X_1, X_2, X_3)$ una muestra aleatoria simple de $X$. Calcular la función masa de probabilidad de los estadísticos $\overline{X}$, $S^2$, $\min X_i$ y $\max X_i$.\\

   \noindent
   Para resolver este ejercicio, como $X$ sigue una distribución discreta, buscamos aplicar el teorema de cambio de variable de discreta a discreta. Para ello, la forma más cómoda será analizar cada uno de los valores que puede tomar la muestra aleatoria simple $(X_1, X_2, X_3)$ y determinar en consecuencia cada uno de los valores que toman $\overline{X}$, $S^2$, $\min X_i$ y $\max X_i$. Acompañaremos la tabla junto con la probabilidad de que la muestra tome dicho valor, es decir, en la fila correpondiente a $(x_1, x_2, x_3)$ incluiremos $P[X_1 = x_1, X_2 = x_2, X_3 = x_3]$:
   \begin{equation*}
   \begin{array}{c|c|c|c|c|c}
       P & (X_1, X_2, X_3) & \overline{X} & S^2 & \min X_i & \max X_i  \\
       \hline
       {(1-p)}^{3} & (0,0,0) & 0 & 0 & 0 & 0 \\
       p{(1-p)}^{2} & (0,0,1) & \nicefrac{1}{3} & \nicefrac{1}{3} & 0 & 1 \\
       p{(1-p)}^{2} & (0,1,0) & \nicefrac{1}{3} & \nicefrac{1}{3} & 0 & 1 \\
       p^2(1-p) & (0,1,1) & \nicefrac{2}{3} & \nicefrac{1}{3} & 0 & 1 \\
       p{(1-p)}^{2} & (1,0,0) & \nicefrac{1}{3} & \nicefrac{1}{3} & 0 & 1 \\
       p^2(1-p) & (1,0,1) & \nicefrac{2}{3} & \nicefrac{1}{3} & 0 & 1 \\
       p^2(1-p) & (1,1,0) & \nicefrac{2}{3} & \nicefrac{1}{3} & 0 & 1 \\
       p^3 & (1,1,1) & 1 & 0 & 1 & 1 
   \end{array}
   \end{equation*}
   Podemos ya calcular la función masa de probabilidad de cada uno de los estadísticos, simplemente sumando las probabilidades de la tabla que corresponden a cada valor del espacio muestral de cada estadístico:
   \begin{itemize}
       \item Para $\overline{X}$:
           \begin{align*}
               P[\overline{X} = 0] &= P[X_1 = 0, X_2 = 0, X_3=0] = {(1-p)}^{3}  \\
               P[\overline{X} = \nicefrac{1}{3}] &= \sum_{i=1}^{3}p{(1-p)}^{2} = 3p{(1-p)}^{2} \\
               P[\overline{X} = \nicefrac{2}{3}] &= \sum_{i=1}^{3}p^2(1-p) = 3p^2(1-p) \\
               P[\overline{X} = 1] &= P[X_1 = 1, X_2 = 1, X_3 = 1] = p^3
           \end{align*}
       \item Para $S^2$:
           \begin{align*}
               P[S^2 = 0] &= P[X_1 = 0, X_2 = 0, X_3 = 0] + P[X_1 = 1, X_2 = 1, X_3 = 1] = p^3 + {(1-p)}^{3} \\
               P[S^2 = \nicefrac{1}{3}] &= \sum_{i=1}^{3} p{(1-p)}^{2} + \sum_{i=1}^{3}p^2(1-p) = 3p(1-p)(p+1-p) = 3p(1-p)
           \end{align*}
       \item Para $\min X_i$:
           \begin{align*}
               P[\min X_i = 1] &= P[X_1 = 1, X_2 = 1, X_3 = 1] = p^3 \\
               P[\min X_i = 0] &= 1-P[\min X_i = 1] = 1-p^3
           \end{align*}
       \item Para $\max X_i$:
           \begin{align*}
               P[\max X_i = 0] &= P[X_1 = 0, X_2 = 0, X_3 = 0] = {(1-p)}^{3}\\
               P[\max X_i = 1] &= 1-P[\max X_i = 0] = 1-{(1-p)}^{3}
           \end{align*}
   \end{itemize}
\end{ejercicio}

\begin{ejercicio}
    Obtener la función masa de probabilidad o función de densidad de $\overline{X}$ en el muestreo de una variable de Bernoulli, de una Poisson y de una exponencial.\\

    \noindent
    Calculamos la masa de probabilidad o función de densidad en cada caso, suponiendo que tenemos $(X_1, \ldots, X_n)$ una muestra aleatoria simple con variables aleatorias idénticamente distribuidas a $X$, que sigue una distribución distinta en cada caso y estaremos interesados en calcular la masa de:
    \begin{equation*}
        \overline{X} = \sum_{i=1}^{n}
    \end{equation*}
    \begin{description}
        \item [Bernoulli.] Supuesto que $X\rightsquigarrow B(1,p)$ para cierto $p\in \left]0,1\right[$, si tomamos:
            \begin{equation*}
                Y = \sum_{i=1}^{n}X_i
            \end{equation*}
            Por la propiedad reproductiva de la Bernoulli, tenemos que $Y\rightsquigarrow B(n,p)$. En dicho caso:
            \begin{equation*}
                P[Y=k] = \binom{n}{k} p^k {(1-p)}^{n-k} \qquad \forall k\in \{0,\ldots,n\}
            \end{equation*}
            Por tanto, tendremos que:
            \begin{equation*}
                P\left[\overline{X} = \frac{k}{n}\right] = P[Y=k] = \binom{n}{k} p^k {(1-p)}^{n-k} \qquad \forall k\in \{0,\ldots,n\}
            \end{equation*}
        \item [Poisson.] Supuesto que $X\rightsquigarrow \cc{P}(\lm)$ para cierto $\lm\in \mathbb{R}^+$, si tomamos:
            \begin{equation*}
                Y = \sum_{i=1}^{n}X_i
            \end{equation*}
            Por la propiedad reproductiva de la Poisson, tendremos que:
            \begin{equation*}
                Y\rightsquigarrow\cc{P}\left(\sum_{i=1}^{n}\lm\right) \equiv \cc{P}(n\lm)
            \end{equation*}
            En dicho caso:
            \begin{equation*}
                P[Y=x] = e^{-n\lm} \dfrac{{(n\lm)}^{x}}{x!} \qquad \forall x\in \mathbb{N}
            \end{equation*}
            Por lo que:
            \begin{equation*}
                P[\overline{X}=\nicefrac{x}{n}] = P[Y=x] = e^{-n\lm} \dfrac{{(n\lm)}^{x}}{x!} \qquad \forall x\in \mathbb{N}
            \end{equation*}
        \item [Exponencial.] Supuesto ahora que $X\rightsquigarrow exp(\lm)$ para cierto $\lm\in \mathbb{R}^+$, tendremos entonces que:
            \begin{equation*}
                M_X(t) = \dfrac{\lm}{\lm - t} \qquad t<\lm
            \end{equation*}
            Si aplicamos la igualdad $(\ast)$ vista en teoría:
            \begin{align*}
                M_{\overline{X}}(t) &\AstIg {(M_X(\nicefrac{t}{n}))}^{n} = {\left(\dfrac{\lm}{\lm - \nicefrac{t}{n}}\right)}^{n} = {\left(\dfrac{n\lm}{n\lm - t}\right)}^{n}
            \end{align*}
            Observamos que obtenemos una función generatriz de momentos para $\overline{X}$ igual que para una variable aleatoria de distribución $\Gamma(n,n\lm)$. Como la función generatriz de momentos de una variable aleatoria caracteriza su distribución, concluimos que $\overline{X}\rightsquigarrow\Gamma(n,n\lm)$.
    \end{description}
\end{ejercicio}

\begin{ejercicio}
    Calcular las funciones de densidad de los estadísticos $\max X_i$ y $\min X_i$ en el muestreo de una variable $X$ con funcion de densidad:
    \begin{equation*}
        f_\theta(x) = e^{\theta-x}, \qquad x>\theta.
    \end{equation*}

    \noindent
    Calculamos primero la función de distribución, para calcular con mayor comodidad las funciones de distribución de $X_{(n)}$ y $X_{(1)}$:
    \begin{equation*}
        F_\theta(x) = \int_{\theta}^{x} f_\theta(t)~dt = \int_{\theta}^{x} e^{\theta-t}~dt  = \left[-e^{\theta-t}\right]_\theta^x = 1 - e^{\theta - x} \qquad \forall x\in \mathbb{R}^+
    \end{equation*}
    Supuesto ahora que disponemos de una m.a.s. $(X_1, \ldots, X_n)$ idénticamente distribuidas a $X$ cuya función de densidad es la anteriormente dicha, podemos aplicar las fórmulas obtenidas en teoría para calcular las funciones de distribución del mínimo y del máximo. Para el máximo:
    \begin{equation*}
        F_{X_{(n)}}(x) = {(F_X(x))}^{n} = {(1-e^{\theta-x})}^{n} \Longrightarrow f_{X_{(n)}} = n{(1-e^{\theta - x})}^{n-1}e^{\theta -x} \qquad \forall x\in \mathbb{R}^+
    \end{equation*}
    Para el mínimo:
    \begin{equation*}
        F_{X_{(1)}}(x) = 1 - {(1-F_X(n))}^{n} = 1-{(1-1+e^{\theta-x})}^{n} = 1-e^{n(\theta-x)} \qquad \forall x\in \mathbb{R}^+
    \end{equation*}
    de donde:
    \begin{equation*}
        f_{X_{(1)}}(x) = ne^{n(\theta-x)-1} \qquad \forall x\in \mathbb{R}^+
    \end{equation*}

\end{ejercicio}

\begin{ejercicio}
    El número de pacientes que visitan diariamente una determinada consulta médica es una variable aleatoria con varianza de 16 personas. Se supone que el número de visitas de cada día es independiente de cualquier otro. Si se observa el número de visitas diarias durante 64 días, calcular aproximadamente la probabilidad de que la media muestral no difiera en más de una persona del valor medio verdadero de visitas diarias.\\

    \noindent
    Sea $X$ una variable aleatoria que indica el número de pacientes que visitan diariamente dicha consulta médica, por cómo nos definen $X$ sabemos que $X\rightsquigarrow\cc{P}(\lm)$. Como además nos dicen que la varianza de dicha variable aleatoria es $16$, tenemos que $Var(X) = \lm = 16$. Si tenemos ahora una muestra aleatoria simple $(X_1, \ldots, X_n)$ con $n=64$, nos preguntan por:
    \begin{equation*}
        P[|\overline{X} - E[X]| < 1]
    \end{equation*}
    Donde $E[X] = \lm = 16$, ya que $X\rightsquigarrow\cc{P}(16)$. Calculamos:
    \begin{align*}
        P[|\overline{X} - E[X]| < 1] &= P[-1 < \overline{X}-16 < 1] = P[15 < \overline{X} < 17]
    \end{align*}
    Aplicamos ahora lo visto en el ejercicio 7, ya que si $X\rightsquigarrow\cc{P}(\lm)$, entonces tendremos que $n\overline{X}\rightsquigarrow \cc{P}(n\lm)$, gracias a la propiedad reproductiva de la Poisson:
    \begin{align*}
        P[|\overline{X} - E[X]| < 1] &= P[15 < \overline{X} < 17] = P[64\cdot 15 < 64\overline{X}<64\cdot 17] \\
                                     &= P[960 < 64\overline{X} < 1088]
    \end{align*}
    Donde $64\overline{X}\rightsquigarrow\cc{P}(64\cdot 16) \equiv \cc{P}(1024)$. Para calcular dicha probabilidad, aproximaremos la Poisson a una distribución normal:
    \begin{equation*}
        \cc{P}(1024) \approx \cc{N}(1024, 1024)
    \end{equation*}
    Por lo que:
    \begin{align*}
        P[|\overline{X} - E[X]| < 1] &= P[960 < 64\overline{X} < 1088] \approx P\left[\dfrac{960-1024}{\sqrt{1024}} < Z < \dfrac{1088 - 1024}{\sqrt{1024}}\right] \\
                                     &= P[-2 < Z < 2] = 2P[Z<2] -1  \\
                                     &= 2\cdot 0.97725 - 1 = 0.9545
    \end{align*}
\end{ejercicio}

\begin{ejercicio}   % // TODO: HACER
    Una máquina de refrescos está arreglada para que la cantidad de bebida que sirve sea una variable aleatoria con media 200 ml. y desviación típica 15 ml. Calcular de forma aproximada la probabilidad de que la cantidad media servida en una muestra aleatoria de tamaño 36 sea al menos 204 ml.
\end{ejercicio}


    \fancyhead[R]{\helv \nouppercase{\rightmark}}
    \chapter{Ejercicios de clase}
\noindent
Esta sección tiene el propósito de recoger todos los ejercicios propuestos en clase por parte de la profesora y que fueron resueltos por los alumnos en pizarra.

\section{Estadísticos muestrales}
\begin{ejercicio}
    Obtener la función masa de probabilidad conjunta de una m.a.s. de $X\rightsquigarrow B(k_0,p)$ y la función de densidad de una m.a.s. de $X\rightsquigarrow U(a,b)$.\\

    \noindent
    Recordamos que si $X\rightsquigarrow B(k_0, p)$, entonces:
    \begin{equation*}
        P[X = x] = \binom{k_0}{x} p^x{(1-p)}^{n-x} \qquad \forall x\in \{0,\ldots,k_0\}
    \end{equation*}
    Por lo que si tenemos una m.a.s. de $n$ variables independientes e idénticamente distribuidas a $X$, $(X_1, \ldots, X_n)$, su función de densidad vendrá dada por:
    \begin{align*}
        P[X_1 &= x_1, \ldots, X_n = x_n] \stackrel{\text{indep.}}{=} \prod_{i=1}^{n}P[X_i = x_i]\stackrel{\text{id. d.}}{=} \prod_{i=1}^{n} P[X=x_i] \\
              &= \prod_{i=1}^{n} \binom{k_0}{x_i} p^{x_i} {(1-p)}^{k_0-x_i} = p^{\sum\limits_{i=1}^{n}x_i} {(1-p)}^{nk_0 - \sum\limits_{i=1}^{n}x_i} \prod_{i=1}^{n}\binom{k_0}{x_i} \\
              & \qquad \forall x_i \in \{0,\ldots,k_0\}
    \end{align*}

    \noindent
    Si ahora $X\rightsquigarrow U(a,b)$ para ciertos $a,b\in \mathbb{R}$ con $a<b$, entonces:
    \begin{equation*}
        f_X(x) = \dfrac{1}{b-a} \qquad \forall x\in [a,b]
    \end{equation*}

    de donde:
    \begin{equation*}
        f_{(X_1, \ldots, X_n)}(x_1, \ldots, x_n) \stackrel{\text{indep.}}{=} \prod_{i=1}^{n} f_{X_i}(x_i)\stackrel{\text{id. d.}}{=} \prod_{i=1}^{n} f_X(x_i) = \prod_{i=1}^{n} \dfrac{1}{b-a} = \dfrac{1}{{(b-a)}^{n}} \qquad \forall x\in [a,b]
    \end{equation*}
\end{ejercicio}

\begin{ejercicio}
    Para cada realización muestral, $(x_1, \ldots, x_n)\in \cc{X}^n$, $F_{x_1,\ldots,x_n}^\ast$ es una función de distribución en $\mathbb{R}$. En particular es una función a saltos, con saltos de amplitud $\nicefrac{1}{n}$ en los sucesivos valores muestrales ordenados de menor a mayor, supuestos que sean distintos, y de saltos múltiplos en el caso de que varios valores muestrales coincidieran.\\

    \noindent
    En las condiciones del enunciado, es decir, suponiendo que $x_1, \ldots, x_n$ están ordenados de menor a mayor y son distintos, entonces es fácil ver que:
    \begin{equation*}
        F_{x_1, \ldots, x_n}^\ast (x) = \left\{\begin{array}{ll}
                0 & \text{si\ } x < x_1 \\
                \nicefrac{1}{n} & \text{si\ } x_1 \leq x < x_2 \\
                                &\vdots \\
                            1 & \text{si\ } x > x_n
        \end{array}\right. \qquad \forall x\in \mathbb{R}
    \end{equation*}
    Por lo que es claro que $F_{x_1, \ldots, x_n}^\ast $ es no decreciente, continua por la derecha, con límite 0 en $-\infty$ y con límite 1 en $+\infty$.
\end{ejercicio}

\begin{ejercicio}
    $\forall x\in \mathbb{R}$, $F_{X_1, \ldots, X_n}^\ast(x)$ es una variable aleatoria tal que \newline $nF_{X_1, \ldots, X_n}^\ast(x) \rightsquigarrow B(n,F(x))$ y:
    \begin{equation*}
        E[F_{X_1, \ldots, X_n}^\ast (x)] = F(x), \qquad Var[F_{X_1, \ldots, X_n}^\ast (x)] = \dfrac{F(x)(1-F(x))}{n}
    \end{equation*}
    donde $F(x)$ es la función de distribución de $X$.\\

    \noindent
    Recordamos que:
    \begin{equation*}
    F_{X_1, \ldots, X_n}^\ast (x) = \dfrac{1}{n}\sum_{i=1}^{n}I_{\left]-\infty,x\right]}(X_i) \qquad \forall x\in \mathbb{R}
    \end{equation*}
Fijado $x\in \mathbb{R}$, tenemos que $I_{\left]-\infty,x\right]}(X)\rightsquigarrow B(1,P[X\leq x]) \equiv B(1,F(x))$, por lo que por la propiedad reproductiva de la binomial tenemos que:
    \begin{equation*}
        nF_{X_1, \ldots, X_n}^\ast (x) \rightsquigarrow B(n,F(x))
    \end{equation*}
    Por lo que:
    \begin{equation*}
        nE[F_{X_1, \ldots, X_n}^\ast (x)] = E[nF_{X_1, \ldots, X_n}^\ast (x)] = nF(x)
    \end{equation*}

    de donde:
    \begin{equation*}
        E[F_{X_1, \ldots, X_n}^\ast (x)] = F(x)
    \end{equation*}
    Para la varianza:
    \begin{equation*}
        n^2Var[F_{X_1, \ldots, X_n}^\ast (x)] = Var[nF_{X_1, \ldots, X_n}^\ast (x)] = nF(x)(1-F(x))
    \end{equation*}
    
    de donde:
    \begin{equation*}
        Var[F_{X_1, \ldots, X_n}^\ast (x)] = \dfrac{F(x)(1-F(x))}{n}
    \end{equation*}
\end{ejercicio}

\begin{ejercicio}
    Para valores grandes de $n$, en virtual del Teorema Central del Límite:
    \begin{equation*}
        F_{X_1, \ldots, X_n}^\ast (x) \rightsquigarrow \cc{N}\left(F(x), \dfrac{F(x)(1-F(x))}{n}\right)
    \end{equation*}

    \noindent
    Sea $(X_1, \ldots, X_n)$ una m.a.s. de $n$ muestras, sea:
    \begin{equation*}
        S_n = \sum_{i=1}^{n} I_{\left]-\infty,x\right]}(X_i) \qquad \forall n\in \mathbb{N}
    \end{equation*}
    Por el Teorema Central del Límite tenemos que:
    \begin{equation*}
        \dfrac{S_n - E[S_n]}{\sqrt{Var[S_n]}} \stackrel{n\to \infty}{\rightsquigarrow} \cc{N}(0,1) \Longrightarrow S_n \stackrel{n\to \infty}{\rightsquigarrow} \cc{N}\left(F(x), \dfrac{F(x)(1-F(x))}{n}\right)
    \end{equation*}
    Como $S_n \rightsquigarrow B(n,F(x))$, entonces tenemos que:
    \begin{align*}
        E[S_n] &= nF(x) \\
        Var[S_n] &= nF(x)(1-F(x))
    \end{align*}
    Por lo que:
    \begin{equation*}
        F_{X_1, \ldots, X_n}^\ast (x) = \dfrac{1}{n}S_n \stackrel{n\to \infty}{\rightsquigarrow} \cc{N}\left(F(x), \dfrac{F(x)(1-F(x))}{n}\right)
    \end{equation*}
\end{ejercicio}

\begin{ejercicio}
    Dada una muestra aleatoria simple formada por las observaciones $(3, 8, 5, 4, 5)$, obtener su función de distribución muestral y realizar la representación gráfica.\\

    \noindent
    Aplicando la definición de la función de distribución muestral obtenemos que:
    \begin{equation*}
        F_{(3, 8, 5, 4, 5)}^\ast(x) = \left\{\begin{array}{ll}
                0 & \text{si\ } x < 3 \\
                1 & \text{si\ } 3 \leq x < 4 \\
                2 & \text{si\ } 4 \leq x < 5 \\
                4 & \text{si\ } 5 \leq x < 8 \\
                5 & \text{si\ } x \geq 8 
        \end{array}\right.
    \end{equation*}

    \begin{figure}[H]
        \centering
        \begin{tikzpicture}
        \begin{axis}[
            axis lines=middle,
            xlabel={$x$},
            ylabel={$F_{(3,8,5,4,5)}^\ast (x)$},
            ymin=-0.5, ymax=6,
            xmin=0, xmax=10,
            xtick={3,4,5,8},
            ytick={0,1,2,4,5},
            grid=both,
            width=10cm,
            height=6cm
        ]

        % Tramos de la función
        \addplot[domain=0:3, thick] {0};

        \addplot[domain=3:4, thick] {1};
        \addplot[domain=4:5, thick] {2};
        \addplot[domain=5:8, thick] {4};
        \addplot[domain=8:10, thick] {5};

        % Puntos abiertos y cerrados
        \addplot[only marks, mark=o] coordinates {(3,0) (4,1) (5,2) (8,4)};
        \addplot[only marks, mark=*] coordinates {(3,1) (4,2) (5,4) (8,5)};

        \end{axis}
        \end{tikzpicture}
        \caption{Representación gráfica de $F_{(3,8,5,4,5)}^\ast (x)$.}
    \end{figure}
\end{ejercicio}

\begin{ejercicio}
    Sea $X$ una variable aleatoria con distribución $B(1,p)$ con $p\in (0,1)$. Se toma una muestra de tamaño 5, $(X_1, X_2, X_3, X_4, X_5)$, y se obtiene la siguiente observación $(0,1,1,0,0)$. Determinar el valor de los estadísticos estudiados en la observación.\\

    \noindent
    Aplicando las fórmulas vistas en clase obtenemos:
    \begin{itemize}
        \item Media: $0.4$.
        \item Varianza: $0.24$.
        \item Cuasivarianza: $0.3$.
        \item $x_{(1)} = 0$, $x_{(2)} = 0$, $x_{(3)} = 0$, $x_{(4)} = 1$, $x_{(5)} = 1$.
    \end{itemize}
\end{ejercicio}

\begin{ejercicio}
    Sea $(X_1, \ldots, X_n)$ una m.a.s. y $\overline{X} = \dfrac{1}{n}\sum\limits_{i=1}^{n}X_i$, entonces:
    \begin{equation*}
        M_{\overline{X}}(t) = {(M_X(\nicefrac{t}{n}))}^{n}
    \end{equation*}

    \begin{equation*}
        M_{\overline{X}}(t) = E\left[e^{t\overline{X}}\right] = E\left[e^{\frac{t}{n}\sum\limits_{i=1}^{n}X_i}\right] = M_{\sum\limits_{i=1}^{n}X_i}\left(\frac{t}{n}\right) \stackrel{\text{indep.}}{=} \prod_{i=1}^{n} M_{X_i}\left(\frac{t}{n}\right) \stackrel{\text{id. d.}}{=} {\left(M_X\left(\frac{t}{n}\right)\right)}^{n}
    \end{equation*}
\end{ejercicio}

\begin{ejercicio}
    Obtener la distribución muestral de $\overline{X}$ para $(X_1, \ldots, X_n)$ una m.a.s. de $X\rightsquigarrow \cc{N}(\mu, \sigma^2)$.

    \begin{equation*}
        M_{\overline{X}}(t) = {\left(M_X\left(\frac{t}{n}\right)\right)}^{n} = {\left(e^{\mu t + \frac{\sigma^2 t^2}{2n^2}}\right)}^{n} = e^{\mu t + \frac{\sigma^2 t^2}{2n}}
    \end{equation*}
    Luego $\overline{X}\rightsquigarrow \cc{N}\left(\mu, \frac{\sigma^2}{n}\right)$, ya que la función generatriz de momentos caracteriza la distribución.
\end{ejercicio}

\begin{prop}
    Si tenemos una m.a.s. $(X_1, \ldots, X_n)$, entonces:
    \begin{align*}
        F_{X_{(n)}}(x) &= {(F_X(x))}^{n} \qquad \forall x\in \mathbb{R} \\
        F_{X_{(1)}}(x) &= 1-{(1-F_X(x))}^{n}
    \end{align*}
    \begin{proof}
        Para la distribución del máximo:
        \begin{align*}
            F_{X_{(n)}}(x) &= P[X_{(n)} \leq x] = P[X_1 \leq x, \ldots, X_n \leq x] \stackrel{\text{indep.}}{=} \prod_{i=1}^{n} P[X_i \leq x] \\ &\stackrel{\text{id. d.}}{=} \prod_{i=1}^{n} P[X\leq x] = {(F_X(x))}^{n}
        \end{align*}
        Para la del mínimo:
        \begin{align*}
            F_{X_{(1)}}(x) &= P[X_{(1)}\leq x] = 1-P[X_{(1)} > x] = 1-P[X_1 > x, \ldots, X_n > x] \\ &\stackrel{\text{indep.}}{=} 1 - \prod_{i=1}^{n}P[X_i > x] \stackrel{\text{id. d.}}{=} 1-{(P[X>x])}^{n} = 1-{(1-F_X(x))}^{n}
        \end{align*}
    \end{proof}
\end{prop}

\begin{ejercicio}
    Obtener las distribuciones muestrales de $X_{(1)}$ y  $X_{(n)}$ para \newline $X\rightsquigarrow U(a,b)$.\\

    \noindent
    Si $X\rightsquigarrow U(a,b)$, entonces:
    \begin{equation*}
        F_X(x) = \dfrac{x-a}{b-a} \qquad \forall x\in [a,b]
    \end{equation*}
    Por lo que aplicando la Proposición superior:
    \begin{align*}
        F_{X_{(n)}}(x) &= {(F_X(x))}^{n} = {\left(\dfrac{x-a}{b-a}\right)}^{n} \qquad \forall x\in [a,b] \\
        F_{X_{(1)}}(x) &= 1 - {(1-F_X(x))}^{n} = 1-{(1-F_X(x))}^{n} = 1-{\left(1-\dfrac{x-a}{b-a}\right)}^{n} \\
                       &= 1-{\left(\dfrac{b-x}{b-a}\right)}^{n} \qquad \forall x\in [a,b]
    \end{align*}
\end{ejercicio}

\section{Distribuciones en el muestreo de poblaciones normales}
\begin{prop}
    Sea $X\rightsquigarrow\cc{N}(0,1)$, entonces $X^2\rightsquigarrow \chi^2(1)$.
    \begin{proof}
        Sea $Y = X^2 = h(X)$, entonces $X = \pm \sqrt{Y} = h^{-1}(y)$, por lo que:
        \begin{equation*}
            f_Y(y) = f_X(h_1^{-1}(y)) \left|\dfrac{dh_1^{-1}(y)}{dy}\right| + f_X(h_2^{-1}(y)) + \left|\dfrac{dh_2^{-1}(y)}{dy}\right| 
        \end{equation*}
        Como $X\rightsquigarrow \cc{N}(0,1)$, entonces:
        \begin{equation*}
            f_X(x) = \dfrac{1}{\sqrt{2\pi}} e^{\frac{-x^2}{2}} \qquad \forall x\in \mathbb{R}
        \end{equation*}
        De donde:
        \begin{equation*}
            f_Y(y) = \dfrac{1}{\sqrt{2\pi}} e^{\frac{-{(\sqrt{y})}^{2}}{2}} \left|\dfrac{1}{2\sqrt{y}}\right| + \dfrac{1}{\sqrt{2\pi}} e^{\frac{-{(-\sqrt{y})}^{2}}{2}} \left|\dfrac{-1}{2\sqrt{y}}\right|  = \dfrac{1}{\sqrt{2\pi y}} e^{\frac{-y}{2}} \qquad \forall y>0
        \end{equation*}
        Por lo que $Y\rightsquigarrow\chi^2(1)$.
    \end{proof}
\end{prop}

% // TODO: Pasar a limpio

\begin{ejercicio}
    Calcula el valor de $k$ o la probabilidad inducida:
    \begin{enumerate}[label=\alph*)]
        \item $P[\chi^2(10)\geq k] = 0.005$.

            $k = 25.1881$.
        \item $P[\chi^2(45) \leq k] = 0.005$.

            \begin{equation*}
                P[\chi^2(45) \geq k] = 0.995 \Longrightarrow k = 24.3110
            \end{equation*}
        \item $P[\chi^2(14) \geq 21.06]$

            $0.1$
        \item $P[\chi^2(20) \leq 12.44]$

            \begin{equation*}
                P[\chi^2(20) \leq 12.44] = 1-P[\chi^2(20) \geq 12.44] = 1-0.9 = 0.1
            \end{equation*}
    \end{enumerate}
\end{ejercicio}

\begin{ejercicio}
    Calcula el valor de $k$ o la probabilidad inducida:
    \begin{enumerate}[label=\alph*)]
        \item $P[t(26)\geq k] = 0.05$

            $k = 1.7056$
        \item $P[t(20)\leq k] = 0.25$

            $k = -0.6870$
        \item $P[t(26) \geq k] = 0.9$

            $k = -1.3150$
        \item $P[t(21) \geq 1.721]$

            $0.05$
        \item $P[t(11) \leq 0.697]$

            $0.75$
        \item $P[t(8) \leq -2.306]$

            $0.025$
    \end{enumerate}
\end{ejercicio}

\begin{ejercicio}
    Calcula el valor de $k$ o la probabilidad inducida:
    \begin{enumerate}[label=\alph*)]
        \item $P[F(7,3) \leq k] = 0.95$

            $k = 8.89$
        \item $P[F(8,4) \geq k] = 0.01$

            \begin{equation*}
                0.01 = 1 - P[F(8,4) \leq k] \Longrightarrow P[F(8,4) \leq k] = 0.99 \Longrightarrow k = 14.8
            \end{equation*}
        \item $P[F(2,2) \leq 19]$

            $0.95$
        \item $P[F(3,5) \geq 12.1]$

            \begin{equation*}
                P[F(3,5) \geq 12.1] = 1-P[F(3,5) \leq 12.1] = 1-0.99 = 0.01
            \end{equation*}
        \item $P[F(60,40) \leq k] = 0.05$

            $k = 0.627$
    \end{enumerate}
\end{ejercicio}

\subsection{Varias demostraciones}
Tenemos una $(X_1, \ldots, X_n)$ m.a.s. con $X\rightsquigarrow \cc{N}(\mu, \sigma^2)$, si tomamos:
\begin{equation*}
    \overline{X} = \dfrac{\sum_{i=1}^{n}X_i}{n}
\end{equation*}
Demostraciones importantes que pueden caer.

\begin{prop}
    En dichas condiciones, veamos que:
    \begin{equation*}
        \overline{X}, (X_1 - \overline{X}, \ldots, X_n - \overline{X}) \text{\ son independientes}
    \end{equation*}
    \begin{proof}
        Para ellos, usaremos la caracterización por la función generatriz de momentos conjunta:
        \begin{equation*}
            M_{\overline{X},X_1 - \overline{X}, \ldots, X_n - \overline{X}}(t,t_1, \ldots, t_n) \stackrel{\text{?}}{=} M_{\overline{X}}(t) M_{X_1 - \overline{X}, \ldots, X_n - \overline{X}}(t_1, \ldots, t_n)
        \end{equation*}

        \begin{align*}
            M_{\overline{X},X_1 - \overline{X}, \ldots, X_n - \overline{X}}(t, t_1, \ldots, t_n) &= E[e^{(t, t_1, \ldots, t_n) \cdot (\overline{X},X_1 - \overline{X}, \ldots, X_n - \overline{X})}] \\
                                                                                                 &= E\left[e^{t\overline{X} + \sum\limits_{i=1}^{n}(X_i - \overline{X})t_i}\right] \\
                                                                                                 &= E\left[e^{\frac{t}{n}\sum\limits_{i=1}^n X_i + \sum\limits_{i=1}^n X_it_i - \sum\limits_{i=1}^n \overline{X}t_i}\right] \\
                                                                                                 &= E\left[e^{\frac{t}{n}\sum\limits_{i=1}^n X_i + \sum\limits_{i=1}^n X_it_i - \overline{X}\sum\limits_{i=1}^n t_i}\right] \\
                                                                                                 &= E\left[e^{\frac{t}{n}\sum\limits_{i=1}^n X_i + \sum\limits_{i=1}^n X_it_i - \frac{1}{n}\sum\limits_{i=1}^n X_i\sum\limits_{i=1}^n t_i}\right] \\
                                                                                                 &= E\left[e^{\frac{t}{n}\sum\limits_{i=1}^n X_i + \sum\limits_{i=1}^n X_it_i - \sum\limits_{i=1}^n X_i\frac{1}{n}\sum\limits_{i=1}^n t_i}\right] \\
                                                                                                 &= E\left[e^{\frac{t}{n}\sum\limits_{i=1}^n X_i + \sum\limits_{i=1}^n X_it_i - \sum\limits_{i=1}^n X_i\overline{t}}\right] \\
                                                                                                 &= E\left[e^{\sum\limits_{i=1}^n X_i\frac{t}{n} + \sum\limits_{i=1}^n X_it_i - \sum\limits_{i=1}^n X_i\overline{t}}\right] \\
                                                                                                 &= E\left[e^{\sum\limits_{i=1}^n X_i \left(\frac{t}{n} + t_i - \overline{t}\right)}\right] \\
                                                                                                 &= E\left[\prod_{i=1}^{n}e^{X_i\left(\frac{t}{n}+t_i - \overline{t}\right)}\right] \\
                                                                                                 &\stackrel{\text{indep.}}{=} \prod_{i=1}^{n} E\left[e^{X_i\left(\frac{t}{n}+t_i - \overline{t}\right)}\right] = \prod_{i=1}^{n} M_{X_i}\left(\frac{t}{n}+t_i - \overline{t}\right) \\
                                                                                                 &\AstIg \prod_{i=1}^{n} e^{\left(\frac{t}{n}+t_i-\overline{t}\right)\mu + {\left(\frac{t}{n}+t_i - t\right)}^{2}\frac{\sigma^2}{2}} \\
                                                                                                 &=e ^{\sum\limits_{i=1}^n \left[\left(\frac{t}{n}+t_i - \overline{t}\right)\mu + \frac{\sigma^2}{2} \left(\frac{t^2}{n^2} + {\left(t_i - \overline{t}\right)}^{2} + 2\frac{t}{n}(t_i - \overline{t})\right)\right]}  \\
                                                                                                 &= e^{\sum\limits_{i=1}^n \frac{\mu t}{n} + \sum\limits_{i=1}^n \mu (t_i-\overline{t}) + \frac{\sigma^2}{2}\left(\sum\limits_{i=1}^n \frac{t^2}{n^2} + \sum\limits_{i=1}^n {(t_i - \overline{t})}^{2} + 2\sum\limits_{i=1}^n \frac{t}{n}(t_i - \overline{t}) \right)} \\
                                                                                                 &= e^{\frac{\cancel{n}\mu t}{\cancel{n}} \mu \cancel{\sum\limites{t_i - \overline{t}}} + \frac{\sigma^2}{n^2} \left(\frac{nt^2}{n^2} + \sum\limits_{i=1}^n {(t_i - \overline{t})}^{2} + 2\frac{t}{n}\sum\limits_{i=1}^n (t_i - \overline{t})\right)} \\
                                                                                                 &= e^{\mu t + \frac{\sigma^2t^2}{2n} + \frac{\sigma^2}{2}\sum\limits_{i=1}^n {(t_i - \overline{t})}^2{}}
        \end{align*}
        Sabemos que:
        \begin{align*}
            M_{\overline{X}}(t) &= M_{\left(\overline{X},X_1 - \overline{X}, \ldots, X_n - \overline{X}\right)}(t,0, ..., 0) = e^{\mu t + \frac{\sigma^2t^2}{2n}} \\
            M_{(X_1 - \overline{X}, \ldots, X_n - \overline{X})}(t_1, ..., t_n) &= M_{(\overline{X},X_1 - \overline{X}, \ldots, X_n - \overline{X})}(0,t_1, \ldots, t_n) \\
                                &= e^{\frac{\sigma^2}{n}\sum\limits_{i=1}^n {(t_i - \overline{t})}^{2}}
        \end{align*}
        Por lo que es cierto que el producto de las funciones generatrices de momentos es la generatriz de mmoentos conjunta, luego las variables son independientes.
    \end{proof}
\end{prop}

\begin{coro}
    Como corolario de la Proposición anterior, tenemos que:
    \begin{itemize}
        \item Se vio ya, y se saca de la demostración de arriba.
        \item Lema de Fisher: $\overline{X}$ y $S^2$ son independientes.

            Como $S^2$ es función del vector de la Proposición anterior, tenemos que es independiente con $\overline{X}$, ya que las funciones de variables independientes son independientes.
        \item $\dfrac{(n-1)S^2}{\sigma^2}\rightsquigarrow\chi^{2}(n-1)$

            Para demostrarlo:
            \begin{equation*}
                \sum\limits_{i=1}^n{\left(\dfrac{X_i - \mu}{\sigma}\right)}^{2} \rightsquigarrow\chi^2(n)
            \end{equation*}
            Ahora, queremos ver que:
            \begin{equation*}
                \dfrac{(n-1)S^2}{\sigma^2} = \dfrac{\sum\limits_{i=1}^n {(X_i - \overline{X})}^{2}}{\sigma^2} \rightsquigarrow\chi^2(n-1)
            \end{equation*}
            Para ello:
            \begin{align*}
                \dfrac{\sum\limits_{i=1}^n {(X_i - \mu)}^{2}}{\sigma^2} &= \dfrac{\sum\limits_{i=1}^n{(X_i - \overline{X} + \overline{X} - \mu)}^{2}}{\sigma^2} \\ 
                &= \dfrac{\sum\limits_{i=1}^n {(X_i - \overline{X})}^{2} + \sum\limits_{i=1}^n {(\overline{X}-\mu)}^{2} + 2\sum\limits_{i=1}^n (X_i - \overline{X})(\overline{X} - \mu)}{\sigma^2} \\
                &= \dfrac{\sum\limits_{i=1}^n {(X_i - \overline{X})}^{2} }{\sigma^2} + n \dfrac{{(\overline{X}-\mu)}^{2}}{\sigma^2}
            \end{align*}
            Y como:
            \begin{equation*}
                n \dfrac{{(\overline{X}-\mu)}^{2}}{\sigma^2} \rightsquigarrow\chi^2(1)
            \end{equation*}
            Buscamos ver lo que sigue lo de la derecha ($A=B+C$). Para ello, usaremos la función generatriz de momentos. tenemos que $B = f(S^2)$ y $C = f(\overline{X})$, luego $B$ y $C$ son independientes, por lo que:
            \begin{equation*}
                M_{A=B+C}(t) \stackrel{\text{indep.}}{=} M_B(t) M_C(t) = M_B(t) \dfrac{1}{{(1-2t)}^{\frac{1}{2}}} \qquad t < \frac{1}{2}
            \end{equation*}
            Y sabemos que:
            \begin{equation*}
                M_A(t) = \dfrac{1}{{(1-2t)}^{\frac{n}{2}}}
            \end{equation*}
            De donde:
            \begin{equation*}
                M_B(t) = \dfrac{M_A(t)}{M_C(t)} = \dfrac{\dfrac{1}{{(1-2t)}^{\frac{n}{2}}}}{\dfrac{1}{{(1-2t)}^{\frac{1}{2}}}} = \dfrac{1}{{(1-2t)}^{\frac{n-1}{2}}} \qquad  t<\frac{1}{2}
            \end{equation*}
            Por lo que $B\rightsquigarrow \chi^2(n-1)$
        \item $\dfrac{\overline{X}-\mu}{\nicefrac{S}{\sqrt{n}}}\rightsquigarrow t(n-1)$

            Para ello, al igual que la $\chi^2$, lo más sencillo es ir a la construcción de $t$:
            \begin{equation*}
                \left.\begin{array}{l}
                        X\rightsquigarrow \cc{N}(0,1) \\
                        Y \rightsquigarrow\chi^2(n) \\
                        indep
                \end{array}\right\} \dfrac{X}{\sqrt{\nicefrac{Y}{n}}} \rightsquigarrow t(n)
            \end{equation*}
            Como:
            \begin{gather*}
                \overline{X}\rightsquigarrow\cc{N}(\mu,\sigma^2)\Longrightarrow \dfrac{\overline{X}-\mu}{\nicefrac{\sigma}{\sqrt{n}}} \rightsquigarrow \cc{N}(0,1) \\
                \dfrac{(n-1)S^2}{\sigma^2} \rightsquigarrow \chi^2(n-1)
            \end{gather*}
            que son independientes por el Lema de Fisher. Si aplicamos la construcción:
            \begin{equation*}
                \dfrac{\dfrac{\overline{X}-\mu}{\nicefrac{\sigma}{\sqrt{n}}}}{\sqrt{\dfrac{(n-1)S^2}{\sigma^2(n-1)}}} = \dfrac{\dfrac{\overline{X}-\mu}{\nicefrac{\sigma}{\sqrt{n}}}}{\dfrac{S}{\sigma}} = \dfrac{\overline{X}-\mu}{\nicefrac{S}{\sqrt{n}}} \rightsquigarrow t(n-1)
            \end{equation*}
    \end{itemize}
\end{coro} 

    \chapter{Relaciones de Ejercicios}
    \section{Estadísticos muestrales}

\begin{ejercicio}
    Sea $(X_1, \ldots, X_n)$ una muestra aleatoria simple de una variable aleatoria $X$. Dar el espacio muestral y calcular la función masa de probabilidad de $(X_1, \ldots, X_n)$ en cada uno de los siguientes casos:
    \begin{enumerate}[label=\alph*)]
        \item $X\rightsquigarrow \{B(k_0,p) : p\in (0,1)\}$ Binomial.

            El espacio muestral en este caso es $\cc{X}^n$, donde:
            \begin{equation*}
                \cc{X} = \{0, 1, ..., k_0\}
            \end{equation*}
            Recordamos que si $X\rightsquigarrow B(k_0,p)$, entonces:
            \begin{equation*}
                P[X=x] = \binom{k_0}{x} p^x {(1-p)}^{k_0-x} \qquad \forall x\in \cc{X}
            \end{equation*}
            Por tanto, para nuestra m.a.s. tendremos la función masa de probabilidad:
            \begin{align*}
                P[X_1 &= x_1, \ldots, X_n = x_n] \stackrel{\text{indep.}}{=} \prod_{i=1}^{n}P[X_i = x_i]\stackrel{\text{id. d.}}{=} \prod_{i=1}^{n} P[X=x_i] \\
                      &= \prod_{i=1}^{n} \binom{k_0}{x_i} p^{x_i} {(1-p)}^{k_0-x_i} = p^{\sum\limits_{i=1}^{n}x_i} {(1-p)}^{nk_0 - \sum\limits_{i=1}^{n}x_i} \prod_{i=1}^{n}\binom{k_0}{x_i} \\
                      & \qquad \forall (x_1, \ldots, x_n) \in \cc{X}^n
            \end{align*}
        \item $X\rightsquigarrow\{\cc{P}(\lm) : \lm \in \mathbb{R}^+\}$ Poisson.

            El espacio muestral de $X$ es:
            \begin{equation*}
                \cc{X} = \mathbb{N} \cup \{0\}
            \end{equation*} 
            Recordamos que si $X\rightsquigarrow \cc{P}(\lm)$, entonces:
            \begin{equation*}
                P[X=x] = e^{-\lm} \dfrac{\lm^x}{x!} \qquad \forall x\in \cc{X}
            \end{equation*}
            Por tanto:
            \begin{align*}
                P[X_1 &= x_1, \ldots, X_n = x_n] \stackrel{\text{indep.}}{=} \prod_{i=1}^{n}P[X_i = x_i]\stackrel{\text{id. d.}}{=} \prod_{i=1}^{n} P[X=x_i] \\
                      &= \prod_{i=1}^{n} e^{-\lm} \dfrac{\lm^{x_i}}{x_i!} = e^{-n\lm} \prod_{i=1}^{n} \dfrac{\lm^{x_i}}{x_i!} = e^{-n\lm} \cdot \dfrac{\lm^{\sum\limits_{i=1}^n x_i}}{\prod\limits_{i=1}^{n}x_i}  \qquad \forall (x_1, \ldots, x_n) \in \cc{X}^n
            \end{align*}
        \item $X\rightsquigarrow\{BN(k_0,p) : p\in (0,1)\}$ Binomial Negativa.

            El espacio muestral de $X$ es:
            \begin{equation*}
                \cc{X} = \mathbb{N}\cup \{0\}
            \end{equation*}
            Recordamos que si $X\rightsquigarrow BN(k_0,p)$, entonces:
            \begin{equation*}
                P[X=x] = \binom{x+k_0-1}{x} {(1-p)}^{x}p^{k_0} \qquad \forall x\in \cc{X}
            \end{equation*}
            Por tanto:
            \begin{align*}
                P[X_1 &= x_1, \ldots, X_n = x_n] \stackrel{\text{indep.}}{=} \prod_{i=1}^{n}P[X_i = x_i]\stackrel{\text{id. d.}}{=} \prod_{i=1}^{n} P[X=x_i] \\
                      &= \prod_{i=1}^{n} \binom{x_i+k_0-1}{x_i}{(1-p)}^{x_i}p^{k_0} = p^{nk_0}{(1-p)}^{\sum\limits_{i=1}^n x_i} \prod_{i=1}^{n}\binom{x_i+k_0-1}{x_i} \\
                      &\forall (x_1,\ldots,x_n)\in \cc{X}^n
            \end{align*}
        \item $X\rightsquigarrow\{G(p) : p\in (0,1)\}$ Geométrica.

            El espacio muestral de $X$ es:
            \begin{equation*}
                \cc{X} = \mathbb{N}\cup \{0\}
            \end{equation*}
            Recordamos que $G(p)\equiv BN(1,p)$, por lo que si sustituimos en la fórmula obtenida en la Binomial Negativa $k_0 = 1$:
            \begin{equation*}
                P[X_1=x_1, \ldots, X_n = x_n] = p^{n}{(1-p)}^{\sum\limits_{i=1}^n x_i} \qquad \forall (x_1,\ldots,x_n)\in \cc{X}^n
            \end{equation*}
        \item $X\rightsquigarrow\{P_N : N\in \mathbb{N}\}$, $\quad P_N(X=x) = \dfrac{1}{N}$, $\quad x=1,\ldots,N$.

            El espacio muestral ya nos lo dan: $\cc{X} = \{1,\ldots,N\}$. Calculemos la masa de probabilidad:
            \begin{align*}
                P[X_1 &= x_1, \ldots, X_n = x_n] \stackrel{\text{indep.}}{=} \prod_{i=1}^{n}P[X_i = x_i]\stackrel{\text{id. d.}}{=} \prod_{i=1}^{n} P[X=x_i] \\
                      &= \prod_{i=1}^{n} \dfrac{1}{N} = {\left(\dfrac{1}{N}\right)}^{n} \qquad \forall (x_1,\ldots,x_n) \in \cc{X}^n
            \end{align*}
    \end{enumerate}
\end{ejercicio}

\begin{ejercicio}
    Sea $(X_1, \ldots, X_n)$ una muestra aleatoria simple de una variable aleatoria $X$. Dar el espacio muestral y calcular la función de densidad de $(X_1, \ldots, X_n)$ en cada uno de los siguientes casos:
    \begin{enumerate}[label=\alph*)]
        \item $X\rightsquigarrow\{U(a,b) : a,b\in \mathbb{R}, a<b\}$ Uniforme.

            El espcio muestral en este caso es $\cc{X}^n$, donde:
            \begin{equation*}
                \cc{X} = [a,b]
            \end{equation*}
            Recordamos que si $X\rightsquigarrow U(a,b)$, entonces:
            \begin{equation*}
                f_X(x) = \dfrac{1}{b-a} \qquad \forall x\in [a,b]
            \end{equation*}
            Por lo que:
            \begin{align*}
                f_{(X_1, \ldots, X_n)}(x_1, \ldots, x_n) &\stackrel{\text{indep.}}{=} \prod_{i=1}^{n} f_{X_i}(x_i) \stackrel{\text{id. d.}}{=} \prod_{i=1}^{n} f_X(x_i) = \prod_{i=1}^{n} \dfrac{1}{b-a} \\ &= {\left(\dfrac{1}{b-a}\right)}^{n} \qquad \forall (x_1, \ldots, x_n) \in \cc{X}^n
            \end{align*}

        \item $X\rightsquigarrow\{\cc{N}(\mu, \sigma^2) : \mu \in \mathbb{R}, \sigma^2 \in \mathbb{R}^+\}$ Normal.

            El espacio muestral de $X$ es $\cc{X} = \mathbb{R}$. Recordamos que si $X\rightsquigarrow \cc{N}(\mu, \sigma^2)$, entonces:
            \begin{equation*}
                f_X(x) = \dfrac{1}{\sqrt{2\pi} \sigma} e^{-\dfrac{{(x-\mu)}^{2}}{2\sigma^2}} \qquad \forall x\in \mathbb{R}
            \end{equation*}
            Por lo que:
            \begin{align*}
                f_{(X_1, \ldots, X_n)}(x_1, \ldots, x_n) &\stackrel{\text{indep.}}{=} \prod_{i=1}^{n} f_{X_i}(x_i) \stackrel{\text{id. d.}}{=} \prod_{i=1}^{n} f_X(x_i) 
                = \prod_{i=1}^{n} \dfrac{1}{\sqrt{2\pi} \sigma} e^{-\dfrac{{(x_i-\mu)}^{2}}{2\sigma^2}}  \\
                &= {\left(\dfrac{1}{\sqrt{2\pi}\sigma}\right)}^{n} \prod_{i=1}^{n} e^{-\dfrac{{(x_i-\mu)}^{2}}{2\sigma^2}}  = {\left(\dfrac{1}{\sqrt{2\pi}\sigma}\right)}^{n} e^{-\sum\limits_{i=1}^n\dfrac{{(x_i-\mu)}^{2}}{2\sigma^2}}   \\
                &= {\left(\dfrac{1}{\sqrt{2\pi}\sigma}\right)}^{n} e^{\frac{-1}{2\sigma^2}\sum\limits_{i=1}^n {(x_i-\mu)}^{2}} \qquad \forall (x_1,\ldots,x_n) \in \mathbb{R}^n
            \end{align*}
        \item $X\rightsquigarrow\{\Gamma(p,a) : p,a\in \mathbb{R}^+\}$ Gamma.

            El espacio muestral de $X$ es $\cc{X}=\mathbb{R}^+_0$. Recordamos que si $X\rightsquigarrow \Gamma(p,a)$, entonces:
            \begin{equation*}
                f_X(x) = \dfrac{a^p}{\Gamma(p)} x^{p-1} e^{-ax} \qquad \forall x\in \mathbb{R}^+_0
            \end{equation*}
            Por lo que:
            \begin{align*}
                f_{(X_1, \ldots, X_n)}(x_1, \ldots, x_n) &\stackrel{\text{indep.}}{=} \prod_{i=1}^{n} f_{X_i}(x_i) \stackrel{\text{id. d.}}{=} \prod_{i=1}^{n} f_X(x_i) 
                = \prod_{i=1}^{n} \dfrac{a^p}{\Gamma(p)} x_i^{p-1} e^{-ax_i} \\
                                                         &= {\left(\dfrac{a^p}{\Gamma(p)}\right)}^{n} \cdot e^{-a\sum\limits_{i=1}^n x_i} \cdot \prod_{i=1}^{n} x_i^{p-1} \qquad \forall (x_1,\ldots,x_n)\in \cc{X}^n
            \end{align*}
        \item $X\rightsquigarrow\{\beta(p,q) : p,q\in \mathbb{R}^+\}$ Beta.

            El espacio muestral de $X$ es $\cc{X} = [0,1]$. Recordamos que si $X\rightsquigarrow \beta(p,q)$, entonces:
            \begin{equation*}
                f_X(x) = \dfrac{1}{\beta(p,q)} x^{p-1} {(1-x)}^{q-1} \qquad \forall x\in [0,1]
            \end{equation*}
            Donde:
            \begin{equation*}
                \beta(p,q) = \dfrac{\Gamma(p)\Gamma(q)}{\Gamma(p+q)}
            \end{equation*}
            Por tanto:
            \begin{align*}
                f_{(X_1, \ldots, X_n)}(x_1, \ldots, x_n) &\stackrel{\text{indep.}}{=} \prod_{i=1}^{n} f_{X_i}(x_i) \stackrel{\text{id. d.}}{=} \prod_{i=1}^{n} f_X(x_i) 
                = \prod_{i=1}^{n} \dfrac{1}{\beta(p,q)} x_i^{p-1} {(1-x_i)}^{q-1} \\
                                                         &= \dfrac{1}{{\beta(p,q)}^{n}} \prod_{i=1}^{n}x_i^{p-1} {(1-x_i)}^{q-1} \qquad \forall (x_1,\ldots,x_n) \in \cc{X}^n
            \end{align*}
        \item $X\rightsquigarrow\{P_\theta : \theta \in \mathbb{R}^+\}$, $\quad f_\theta(x) = \dfrac{1}{2\sqrt{x\theta}}$, $\quad 0<x<\theta$.

            Se nos dice que $\cc{X} = \left]0,\theta\right[$. Calculamos la función de densidad conjunta:
            \begin{align*}
                f_{(X_1, \ldots, X_n)}(x_1, \ldots, x_n) &\stackrel{\text{indep.}}{=} \prod_{i=1}^{n} f_{X_i}(x_i) \stackrel{\text{id. d.}}{=} \prod_{i=1}^{n} f_X(x_i) 
                = \prod_{i=1}^{n} \dfrac{1}{2\sqrt{x_i \theta}} \\
                                                         &= \dfrac{1}{{\left(2\sqrt{\theta}\right)}^{n}} \prod_{i=1}^{n} \dfrac{1}{\sqrt{x_i}} \qquad \forall (x_1, \ldots, x_n) \in  \cc{X}^n
            \end{align*}
    \end{enumerate}
\end{ejercicio}

\begin{ejercicio}
    Se miden los tiempos de sedimentación de una muestra de partículas flotando en un líquido. Los tiempos observados son: 
    \begin{gather*}
        11.5; 1.8; 7.3; 12.1; 1.8; 21.3; 7.3; 15.2; 7.3; 12.1; 15.2;\\ 7.3; 12.1; 1.8; 10.5; 15.2; 21.3; 10.5; 15.2; 11.5
    \end{gather*}
    \begin{itemize}
        \item Construir la función de distribución muestral asociada a a dichas observaciones.

            Si aplicamos la definición de función de distribución muestral obtenemos que esta viene dada por:
            \begin{equation*}
                F_n^\ast(x) =
                \begin{cases} 
                0 & \text{si\ } x < 1.8 \\[6pt]
                \nicefrac{3}{20} &\text{si\ } 1.8 \leq x < 7.3 \\[6pt]
                \nicefrac{7}{20} &\text{si\ } 7.3 \leq x < 10.5 \\[6pt]
                \nicefrac{9}{20} &\text{si\ } 10.5 \leq x < 11.5 \\[6pt]
                \nicefrac{11}{20} &\text{si\ } 11.5 \leq x < 12.1 \\[6pt]
                \nicefrac{14}{20} &\text{si\ } 12.1 \leq x < 15.2 \\[6pt]
                \nicefrac{18}{20} &\text{si\ } 15.2 \leq x < 21.3 \\[6pt]
                \nicefrac{20}{20} &\text{si\ } x \geq 21.3
                \end{cases}
            \end{equation*}

            \begin{figure}[H]
                \centering
                \begin{tikzpicture}
                \begin{axis}[
                    width=12cm, height=7cm, xlabel={$x$}, ylabel={$F_n^\ast(x)$}, ymin=0, ymax=1.1,
                    xmin=0, xmax=23, xtick={0,1.8,7.3,10.5,12.1,15.2,21.3,23},
                    ytick={0,0.1,0.2,0.3,0.4,0.5,0.6,0.7,0.8,0.9,1},
                    grid=both, domain=0:23, samples=200,
                ]

                % Graficamos la función escalonada
                \addplot[
                    thick, blue
                ] coordinates {
                    (0,0) (1.8,0) (1.8,3/20) (7.3,3/20) (7.3,7/20) (10.5,7/20)
                    (10.5,9/20) (11.5,9/20) (11.5,11/20) (12.1,11/20) (12.1,14/20)
                    (15.2,14/20) (15.2,18/20) (21.3,18/20) (21.3,20/20) (23,20/20)
                };
                \end{axis}
                \end{tikzpicture}
                \caption{Gráfica de la función de distribución muestral.}
            \end{figure}
        \item Hallar los valores de los tres primeros momentos muestrales respecto al origen y respecto a la media.

            Calculamos primero los tres primeros momentos respecto al origen para luego calcular los centrados respecto a la media a partir de ellos:
            \begin{align*}
                a_1 &= \sum_{i=1}^{n} f_i x_i = 10.915 \qquad 
                a_2 = \sum_{i=1}^{n} f_i x_i^2 = 148.9325 \\
                a_3 &= \sum_{i=1}^{n} f_i x_i^3 = 2280.98365 \\
                b_1 &= \sum_{i=1}^{n} f_i {(x_i - \overline{x})}= 0\qquad  \\
                b_2 &= \frac{1}{n}\sum_{i=1}^{n}{(x_i-\overline{x})}^{2} = \frac{1}{n}\sum_{i=1}^{n}(x_i^2 - 2x_i \overline{x}+\overline{x}^2)  \\ &= \frac{1}{n}\sum_{i=1}^{n}x_i^2 - \frac{2\overline{x}}{n}\sum_{i=1}^{n}x_i + \overline{x}^2 = a_2 -2a_1^2 + a_1^2 = a_2 - a_1^2 \\
                    &= 148.9325 - 10.915  = 29.795275\\
                b_3 &= \frac{1}{n}\sum_{i=1}^{n} {(x_i-\overline{x})}^{3} = \frac{1}{n}\sum_{i=1}^{n}\left(x_i^3 - 3x_i^2 \overline{x} + 3x_i\overline{x}^2 -\overline{x}^3\right) \\
                    &= \frac{1}{n}\sum_{i=1}^{n}x_i^3 - \frac{3\overline{x}}{n}\sum_{i=1}^{n}x_i^2 + \frac{3\overline{x}^2}{n}\sum_{i=1}^{n}x_i - \overline{x}^3 = a_3 - 3a_1a_2 + 3a_1^3 - a_1^3 \\
                    &= a_3 - 3a_1a_2 + 2a_1^3 = 4.95455925
            \end{align*}
        \item Determinar los valores de los cuartiles muestrales y el percentil 70.

            Para ello, primero ordenamos los datos de menor a mayor y los agrupamos en grupos de $\nicefrac{20}{4} = 5$ en 5:
            \begin{gather*}
                1.8;\ 1.8;\ 1.8;\ 7.3;\ 7.3;\ \red{7.3;\ 7.3;\ 10.5;\ 10.5;\ 11.5};\ 11.5;\ 12.1;\ 12.1;\\ 12.1;\ 15.2;\ \red{15.2;\ 15.2;\ 15.2;\ 21.3;\ 21.3}
            \end{gather*}
            Como en los cambios de agrupaciones de números estos se repiten, hemos obtenido el valor de los cuartiles:
            \begin{equation*}
                q_1 = 7.3 \qquad q_2 = 11.5 \qquad q_3 = 15.2 \qquad q_4 = 21.3
            \end{equation*}
            Para el percentil $70$, calculamos:
            \begin{equation*}
                0.7\cdot 20 = 14
            \end{equation*}
            Como hemos obtenido un número entero, el percentil 70 será:
            \begin{equation*}
                c_{70} = \dfrac{X_{(14)} + X_{(15)}}{2} = \dfrac{12.1 + 15.2}{2} = 13.65
            \end{equation*}
            En el caso de haber obtenido un número no entero (por ejemplo, $14.2$), sería $X_{(15)}$.
    \end{itemize}
\end{ejercicio}

\begin{ejercicio}
    Se dispone de una muestra aleatoria simple de tamaño 40 de una distribución exponencial de media 3, ¿cuál es la probabilidad de que los valores de la función de distribución muestral y la teórica, en $x=1$, difieran menos de $0.01$? Aproximadamente, ¿cuál debe ser el tamaño muestral para que dicha probabilidad sea como mínimo $0.98$?\\

    \noindent
    Como dice el enunciado, tenemos una m.a.s. $(X_1, \ldots, X_n)$ con $n=40$, todas ellas idénticamente distribuidas a $X\rightsquigarrow exp(\lm)$. Sabemos de la asignatura de Probabilidad que:
    \begin{equation*}
        E[X] = \dfrac{1}{\lm} = 3 \Longrightarrow \lm = \dfrac{1}{3}
    \end{equation*}
    Por lo que $X\rightsquigarrow exp\left(\frac{1}{3}\right)$. Denotaremos por comodidad:
    \begin{equation*}
        F_n^\ast(x) = F_{(X_1, \ldots, X_n)}^\ast(x) 
    \end{equation*}
    Y el enunciado nos pregunta por:
    \begin{equation*}
        P[|F_n^\ast(1) - F_X(1)| < 0.01]
    \end{equation*}
    Para ello, primero calculamos $F_X(1)$:
    \begin{equation*}
        F_X(1) = 1 - e^{-\lm\cdot  1} = 1-e^{-\lm} = 1-e^{-\nicefrac{1}{3}} = \alpha
    \end{equation*}
    Por lo que nos disponemos ya a calcular la probabilidad:
    \begin{align*}
        P[|F_n^\ast(1) - F_X(1)| < 0.01] &= P[|F_n^\ast(1) - \alpha| < 0.01] = P[-0.01 < F_n^\ast(1) - \alpha < 0.01] \\
                                         &= P[-0.01+\alpha < F_n^\ast(1) < 0.01+\alpha] \\
                                         &= P[40(-0.01+\alpha) < 40F_n^\ast(1) < 40(0.01+\alpha)]
    \end{align*}
    Y como sabemos que $Y = 40F_n^\ast(1) \rightsquigarrow B(40, F_X(1)) \equiv B(40, \alpha)$:
    \begin{equation*}
        P[|F_n^\ast(1) - F_X(1)| < 0.01] = P[40(-0.01+\alpha) < Y< 40(0.01+\alpha)] 
    \end{equation*}
    Si ahora tomamos:
    \begin{equation*}
        \alpha = 1-e^{-\nicefrac{1}{3}}\approx 0.283469
    \end{equation*}
    Entonces:
    \begin{align*}
        40(0.01 + \alpha) &\approx 40(0.01+0.283469)= 11.73876 \\
        40(-0.01 + \alpha) &\approx 40(-0.01 + 0.283469)  = 10.93876
    \end{align*}
    Por lo que:
    \begin{align*}
        P[|F_n^\ast(1) - F_X(1)| < 0.01] &\approx P[10.93876 < Y < 11.73876] = P[Y=11]
    \end{align*}
    De donde usando la masa de probabilidad de la Binomial:
    \begin{equation*}
        P[Y=11] = \binom{40}{11} {(0.283469)}^{11} {(1-0.283469)}^{40-11} \approx 0.139
    \end{equation*}

    \noindent
    Para el segundo apartado, como para $n=40$ obtenemos una probabilidad de $0.139$, podemos intuir que para que dicha probabilidad sea como mínimo $0.98$, nos es necesario un valor de $n$ grande, por lo que podemos suponer que:
    \begin{equation*}
        F_n^\ast(1) \rightsquigarrow \cc{N}\left(\alpha, \dfrac{\alpha(1-\alpha)}{n}\right)
    \end{equation*}
    De donde:
    \begin{equation*}
        Z = \dfrac{\sqrt{n}(F_n^\ast(1)-\alpha)}{\sqrt{\alpha(1-\alpha)}} \rightsquigarrow \cc{N}(0,1)
    \end{equation*}
    Buscamos el valor de $n$ que verifica:
    \begin{equation*}
        0.98 \leq P[|F_n^\ast(1) - F_X(1)| < 0.01] = P\left[|Z| < \dfrac{\sqrt{n}0.01}{\sqrt{\alpha(1-\alpha)}}\right]
    \end{equation*}
    Si aplicamos propiedades conocidas de la Normal, si $a\in \mathbb{R}$, entonces:
    \begin{equation*}
        P[|Z| < a] = P[-a < Z < a] = P[Z<a] - P[Z< -a]
    \end{equation*}
    Pero:
    \begin{equation*}
        P[Z< -a] = P[Z>a] = 1-P[Z-a]
    \end{equation*}
    Por lo que:
    \begin{equation*}
        P[|Z| < a] = P[Z<a] - P[Z < -a] = 2P[Z<a] - 1
    \end{equation*}
    Volviendo al caso que nos interesa:
    \begin{equation*}
        0.98 \leq P\left[|Z| < \dfrac{\sqrt{n}0.01}{\sqrt{\alpha(1-\alpha)}}\right] = 2P\left[Z<\dfrac{\sqrt{n}0.01}{\sqrt{\alpha(1-\alpha)}}\right] - 1
    \end{equation*}
    Luego:
    \begin{equation*}
        0.99 = \dfrac{0.98+1}{2} \leq P\left[Z < \dfrac{\sqrt{n}0.01}{\sqrt{\alpha(1-\alpha)}}\right]
    \end{equation*}
    Si consultamos la tabla de la normal $\cc{N}(0,1)$, observamos que el primer valor que supera la probbailidad de $0.99$ es $2.33$, por lo que:
    \begin{equation*}
        2.33 = \dfrac{\sqrt{n}0.01}{\sqrt{\alpha(1-\alpha)}} = \dfrac{\sqrt{n}0.01}{\sqrt{0.283469(1-0.283469)}} \approx 0.0221886 \sqrt{n}
    \end{equation*}
    De donde:
    \begin{equation*}
        5.4289 = {(2.33)}^{2} = {(0.0221886\sqrt{n})}^{2} = 0.00049233 n \Longrightarrow n = \dfrac{5.4289}{0.00049233} = 11026.95347
    \end{equation*}
    Por lo que para $n \geq 11027$ podemos asegurar que la probabilidad es como mínimo $0.98$.
\end{ejercicio}

\begin{ejercicio}
    Se dispone de una muestra aleatoria simple de tamaño 50 de una distribución de Poisson de media 2, ¿cuál es la probabilidad de que los valores de la función de distribución muestral y la teórica, en $x=2$, difieran menos de $0.02$? Aproximadamente, ¿qué tamaño muestral hay que tomar para que dicha probabilidad sea como mínimo $0.99$?\\

    \noindent
    Tenemos una m.a.s. $(X_1, \ldots, X_n)$ con $n=50$ idénticamente distribuidas a $X\rightsquigarrow \cc{P}(2)$. Notamos por comodidad:
    \begin{equation*}
        F_{(X_1, \ldots, X_n)}^\ast(x) = F_n^\ast(x)
    \end{equation*}
    Nos preguntan por:
    \begin{equation*}
        P[|F_n^\ast(2) - F_X(2)| < 0.02]
    \end{equation*}
    Para ello primero calculamos:
    \begin{equation*}
        F_X(2) = \sum_{k=0}^{2} e^{-2}\dfrac{2^k}{k!} =  e^{-2}\left(\dfrac{2^0}{0!} + \dfrac{2^1}{1!} + \dfrac{2^2}{2!}\right) = e^{-2} (1+2+2) \approx 0.6767
    \end{equation*}
    Por lo que:
    \begin{equation*}
        P[|F_n^\ast(2) - 0.6767| < 0.02] = P[-0.02 < F_n^\ast(2) - 0.6767 < 0.02] = P[0.6567 < F_n^\ast(2) < 0.6967]
    \end{equation*}
    Como sabemos por lo visto en teoría que:
    \begin{equation*}
        Y = 50F_n^\ast(2) \rightsquigarrow B(50, F_X(2)) \equiv B(50,\ \ 0.6767)
    \end{equation*}
    Multiplicamos por $50$ la última expresión:
    \begin{align*}
        P[|F_n^\ast(2) - 0.6767| < 0.02] = P[0.6567 < F_n^\ast(2) < 0.6967] &= P[32.835 < Y< 34.835] \\
                                         &= P[Y = 33] + P[Y=34]
    \end{align*}
    Y calculamos estas dos probabilidades:
    \begin{align*}
        P[Y=33] &= \binom{50}{33} {(0.6767)}^{33} {(1-0.6767)}^{50-33} \approx 0.114734\\
        P[Y=34] &= \binom{50}{34} {(0.6767)}^{34} {(1-0.6767)}^{50-34} \approx 0.120075
    \end{align*}
    Por lo que:
    \begin{equation*}
        P[|F_n^\ast(2) - 0.6767| < 0.02] \approx 0.114734 + 0.120075 = 0.234809
    \end{equation*}

    \noindent
    Para el segundo apartado, como para $n=50$ obtenemos una probabilidad de $0.234809$, podemos intuir que para que dicha probabilidad sea como mínimo $0.99$, nos es necesario un valor de $n$ grande, por lo que podemos suponer que:
    \begin{equation*}
        F_n^\ast(2) \rightsquigarrow \cc{N}\left(0.6767, \dfrac{0.6767(1-0.6767)}{n}\right) \equiv \cc{N}\left(0.6767, \dfrac{0.218777}{n}\right)
    \end{equation*}
    Por lo que:
    \begin{equation*}
        Z = \dfrac{\sqrt{n}(F_n^\ast(2) - 0.6767)}{\sqrt{0.218777}} \rightsquigarrow\cc{N}(0,1)
    \end{equation*}
    En dicho caso, buscamos $n$ de forma que:
    \begin{equation*}
        0.99 \leq P[|F_n^\ast(2) - F_X(2)| < 0.02] = P\left[|Z| < \dfrac{\sqrt{n}0.02}{\sqrt{0.218777}}\right]
    \end{equation*}
    De forma análoga al ejercicio anterior:
    \begin{equation*}
        P\left[|Z| < \dfrac{\sqrt{n}0.02}{\sqrt{0.218777}}\right] = 2P\left[Z < \dfrac{\sqrt{n}0.02}{\sqrt{0.218777}}\right] - 1
    \end{equation*}
    Luego:
    \begin{equation*}
        0.995 = \dfrac{0.99 + 1}{2} \leq P\left[Z < \dfrac{\sqrt{n}0.02}{\sqrt{0.218777}}\right] 
    \end{equation*}
    Y si miramos la tabla de la Normal observamos que el primer valor que supera la probabilidad de $0.995$ es $2.58$, luego:
    \begin{equation*}
        2.58 = \dfrac{\sqrt{n}0.02}{\sqrt{0.218777}} = 0.042759 \sqrt{n}
    \end{equation*}
    Por lo que:
    \begin{equation*}
        6.6564 = {(2.58)}^{2} = {(0.042759 \sqrt{n})}^{2} = 0.00182833 n
    \end{equation*}
    Luego:
    \begin{equation*}
        n = \dfrac{6.6564}{0.00182833} \approx 3640.7
    \end{equation*}
    Por lo que para $n\geq 3641$ podemos asegurar que la probabilidad es como mínimo $0.99$.
\end{ejercicio}

\begin{ejercicio}
   Sea $X\rightsquigarrow B(1,p)$ y $(X_1, X_2, X_3)$ una muestra aleatoria simple de $X$. Calcular la función masa de probabilidad de los estadísticos $\overline{X}$, $S^2$, $\min X_i$ y $\max X_i$.\\

   \noindent
   Para resolver este ejercicio, como $X$ sigue una distribución discreta, buscamos aplicar el teorema de cambio de variable de discreta a discreta. Para ello, la forma más cómoda será analizar cada uno de los valores que puede tomar la muestra aleatoria simple $(X_1, X_2, X_3)$ y determinar en consecuencia cada uno de los valores que toman $\overline{X}$, $S^2$, $\min X_i$ y $\max X_i$. Acompañaremos la tabla junto con la probabilidad de que la muestra tome dicho valor, es decir, en la fila correpondiente a $(x_1, x_2, x_3)$ incluiremos $P[X_1 = x_1, X_2 = x_2, X_3 = x_3]$:
   \begin{equation*}
   \begin{array}{c|c|c|c|c|c}
       P & (X_1, X_2, X_3) & \overline{X} & S^2 & \min X_i & \max X_i  \\
       \hline
       {(1-p)}^{3} & (0,0,0) & 0 & 0 & 0 & 0 \\
       p{(1-p)}^{2} & (0,0,1) & \nicefrac{1}{3} & \nicefrac{1}{3} & 0 & 1 \\
       p{(1-p)}^{2} & (0,1,0) & \nicefrac{1}{3} & \nicefrac{1}{3} & 0 & 1 \\
       p^2(1-p) & (0,1,1) & \nicefrac{2}{3} & \nicefrac{1}{3} & 0 & 1 \\
       p{(1-p)}^{2} & (1,0,0) & \nicefrac{1}{3} & \nicefrac{1}{3} & 0 & 1 \\
       p^2(1-p) & (1,0,1) & \nicefrac{2}{3} & \nicefrac{1}{3} & 0 & 1 \\
       p^2(1-p) & (1,1,0) & \nicefrac{2}{3} & \nicefrac{1}{3} & 0 & 1 \\
       p^3 & (1,1,1) & 1 & 0 & 1 & 1 
   \end{array}
   \end{equation*}
   Podemos ya calcular la función masa de probabilidad de cada uno de los estadísticos, simplemente sumando las probabilidades de la tabla que corresponden a cada valor del espacio muestral de cada estadístico:
   \begin{itemize}
       \item Para $\overline{X}$:
           \begin{align*}
               P[\overline{X} = 0] &= P[X_1 = 0, X_2 = 0, X_3=0] = {(1-p)}^{3}  \\
               P[\overline{X} = \nicefrac{1}{3}] &= \sum_{i=1}^{3}p{(1-p)}^{2} = 3p{(1-p)}^{2} \\
               P[\overline{X} = \nicefrac{2}{3}] &= \sum_{i=1}^{3}p^2(1-p) = 3p^2(1-p) \\
               P[\overline{X} = 1] &= P[X_1 = 1, X_2 = 1, X_3 = 1] = p^3
           \end{align*}
       \item Para $S^2$:
           \begin{align*}
               P[S^2 = 0] &= P[X_1 = 0, X_2 = 0, X_3 = 0] + P[X_1 = 1, X_2 = 1, X_3 = 1] = p^3 + {(1-p)}^{3} \\
               P[S^2 = \nicefrac{1}{3}] &= \sum_{i=1}^{3} p{(1-p)}^{2} + \sum_{i=1}^{3}p^2(1-p) = 3p(1-p)(p+1-p) = 3p(1-p)
           \end{align*}
       \item Para $\min X_i$:
           \begin{align*}
               P[\min X_i = 1] &= P[X_1 = 1, X_2 = 1, X_3 = 1] = p^3 \\
               P[\min X_i = 0] &= 1-P[\min X_i = 1] = 1-p^3
           \end{align*}
       \item Para $\max X_i$:
           \begin{align*}
               P[\max X_i = 0] &= P[X_1 = 0, X_2 = 0, X_3 = 0] = {(1-p)}^{3}\\
               P[\max X_i = 1] &= 1-P[\max X_i = 0] = 1-{(1-p)}^{3}
           \end{align*}
   \end{itemize}
\end{ejercicio}

\begin{ejercicio}
    Obtener la función masa de probabilidad o función de densidad de $\overline{X}$ en el muestreo de una variable de Bernoulli, de una Poisson y de una exponencial.\\

    \noindent
    Calculamos la masa de probabilidad o función de densidad en cada caso, suponiendo que tenemos $(X_1, \ldots, X_n)$ una muestra aleatoria simple con variables aleatorias idénticamente distribuidas a $X$, que sigue una distribución distinta en cada caso y estaremos interesados en calcular la masa de:
    \begin{equation*}
        \overline{X} = \sum_{i=1}^{n}
    \end{equation*}
    \begin{description}
        \item [Bernoulli.] Supuesto que $X\rightsquigarrow B(1,p)$ para cierto $p\in \left]0,1\right[$, si tomamos:
            \begin{equation*}
                Y = \sum_{i=1}^{n}X_i
            \end{equation*}
            Por la propiedad reproductiva de la Bernoulli, tenemos que $Y\rightsquigarrow B(n,p)$. En dicho caso:
            \begin{equation*}
                P[Y=k] = \binom{n}{k} p^k {(1-p)}^{n-k} \qquad \forall k\in \{0,\ldots,n\}
            \end{equation*}
            Por tanto, tendremos que:
            \begin{equation*}
                P\left[\overline{X} = \frac{k}{n}\right] = P[Y=k] = \binom{n}{k} p^k {(1-p)}^{n-k} \qquad \forall k\in \{0,\ldots,n\}
            \end{equation*}
        \item [Poisson.] Supuesto que $X\rightsquigarrow \cc{P}(\lm)$ para cierto $\lm\in \mathbb{R}^+$, si tomamos:
            \begin{equation*}
                Y = \sum_{i=1}^{n}X_i
            \end{equation*}
            Por la propiedad reproductiva de la Poisson, tendremos que:
            \begin{equation*}
                Y\rightsquigarrow\cc{P}\left(\sum_{i=1}^{n}\lm\right) \equiv \cc{P}(n\lm)
            \end{equation*}
            En dicho caso:
            \begin{equation*}
                P[Y=x] = e^{-n\lm} \dfrac{{(n\lm)}^{x}}{x!} \qquad \forall x\in \mathbb{N}
            \end{equation*}
            Por lo que:
            \begin{equation*}
                P[\overline{X}=\nicefrac{x}{n}] = P[Y=x] = e^{-n\lm} \dfrac{{(n\lm)}^{x}}{x!} \qquad \forall x\in \mathbb{N}
            \end{equation*}
        \item [Exponencial.] Supuesto ahora que $X\rightsquigarrow exp(\lm)$ para cierto $\lm\in \mathbb{R}^+$, tendremos entonces que:
            \begin{equation*}
                M_X(t) = \dfrac{\lm}{\lm - t} \qquad t<\lm
            \end{equation*}
            Si aplicamos la igualdad $(\ast)$ vista en teoría:
            \begin{align*}
                M_{\overline{X}}(t) &\AstIg {(M_X(\nicefrac{t}{n}))}^{n} = {\left(\dfrac{\lm}{\lm - \nicefrac{t}{n}}\right)}^{n} = {\left(\dfrac{n\lm}{n\lm - t}\right)}^{n}
            \end{align*}
            Observamos que obtenemos una función generatriz de momentos para $\overline{X}$ igual que para una variable aleatoria de distribución $\Gamma(n,n\lm)$. Como la función generatriz de momentos de una variable aleatoria caracteriza su distribución, concluimos que $\overline{X}\rightsquigarrow\Gamma(n,n\lm)$.
    \end{description}
\end{ejercicio}

\begin{ejercicio}
    Calcular las funciones de densidad de los estadísticos $\max X_i$ y $\min X_i$ en el muestreo de una variable $X$ con funcion de densidad:
    \begin{equation*}
        f_\theta(x) = e^{\theta-x}, \qquad x>\theta.
    \end{equation*}

    \noindent
    Calculamos primero la función de distribución, para calcular con mayor comodidad las funciones de distribución de $X_{(n)}$ y $X_{(1)}$:
    \begin{equation*}
        F_\theta(x) = \int_{\theta}^{x} f_\theta(t)~dt = \int_{\theta}^{x} e^{\theta-t}~dt  = \left[-e^{\theta-t}\right]_\theta^x = 1 - e^{\theta - x} \qquad \forall x\in \mathbb{R}^+
    \end{equation*}
    Supuesto ahora que disponemos de una m.a.s. $(X_1, \ldots, X_n)$ idénticamente distribuidas a $X$ cuya función de densidad es la anteriormente dicha, podemos aplicar las fórmulas obtenidas en teoría para calcular las funciones de distribución del mínimo y del máximo. Para el máximo:
    \begin{equation*}
        F_{X_{(n)}}(x) = {(F_X(x))}^{n} = {(1-e^{\theta-x})}^{n} \Longrightarrow f_{X_{(n)}} = n{(1-e^{\theta - x})}^{n-1}e^{\theta -x} \qquad \forall x\in \mathbb{R}^+
    \end{equation*}
    Para el mínimo:
    \begin{equation*}
        F_{X_{(1)}}(x) = 1 - {(1-F_X(n))}^{n} = 1-{(1-1+e^{\theta-x})}^{n} = 1-e^{n(\theta-x)} \qquad \forall x\in \mathbb{R}^+
    \end{equation*}
    de donde:
    \begin{equation*}
        f_{X_{(1)}}(x) = ne^{n(\theta-x)-1} \qquad \forall x\in \mathbb{R}^+
    \end{equation*}

\end{ejercicio}

\begin{ejercicio}
    El número de pacientes que visitan diariamente una determinada consulta médica es una variable aleatoria con varianza de 16 personas. Se supone que el número de visitas de cada día es independiente de cualquier otro. Si se observa el número de visitas diarias durante 64 días, calcular aproximadamente la probabilidad de que la media muestral no difiera en más de una persona del valor medio verdadero de visitas diarias.\\

    \noindent
    Sea $X$ una variable aleatoria que indica el número de pacientes que visitan diariamente dicha consulta médica, por cómo nos definen $X$ sabemos que $X\rightsquigarrow\cc{P}(\lm)$. Como además nos dicen que la varianza de dicha variable aleatoria es $16$, tenemos que $Var(X) = \lm = 16$. Si tenemos ahora una muestra aleatoria simple $(X_1, \ldots, X_n)$ con $n=64$, nos preguntan por:
    \begin{equation*}
        P[|\overline{X} - E[X]| < 1]
    \end{equation*}
    Donde $E[X] = \lm = 16$, ya que $X\rightsquigarrow\cc{P}(16)$. Calculamos:
    \begin{align*}
        P[|\overline{X} - E[X]| < 1] &= P[-1 < \overline{X}-16 < 1] = P[15 < \overline{X} < 17]
    \end{align*}
    Aplicamos ahora lo visto en el ejercicio 7, ya que si $X\rightsquigarrow\cc{P}(\lm)$, entonces tendremos que $n\overline{X}\rightsquigarrow \cc{P}(n\lm)$, gracias a la propiedad reproductiva de la Poisson:
    \begin{align*}
        P[|\overline{X} - E[X]| < 1] &= P[15 < \overline{X} < 17] = P[64\cdot 15 < 64\overline{X}<64\cdot 17] \\
                                     &= P[960 < 64\overline{X} < 1088]
    \end{align*}
    Donde $64\overline{X}\rightsquigarrow\cc{P}(64\cdot 16) \equiv \cc{P}(1024)$. Para calcular dicha probabilidad, aproximaremos la Poisson a una distribución normal:
    \begin{equation*}
        \cc{P}(1024) \approx \cc{N}(1024, 1024)
    \end{equation*}
    Por lo que:
    \begin{align*}
        P[|\overline{X} - E[X]| < 1] &= P[960 < 64\overline{X} < 1088] \approx P\left[\dfrac{960-1024}{\sqrt{1024}} < Z < \dfrac{1088 - 1024}{\sqrt{1024}}\right] \\
                                     &= P[-2 < Z < 2] = 2P[Z<2] -1  \\
                                     &= 2\cdot 0.97725 - 1 = 0.9545
    \end{align*}
\end{ejercicio}

\begin{ejercicio}   % // TODO: HACER
    Una máquina de refrescos está arreglada para que la cantidad de bebida que sirve sea una variable aleatoria con media 200 ml. y desviación típica 15 ml. Calcular de forma aproximada la probabilidad de que la cantidad media servida en una muestra aleatoria de tamaño 36 sea al menos 204 ml.
\end{ejercicio}

    \section{El grupo fundamental}
\begin{ejercicio}
    Prueba que en un espacio topológico simplemente conexo $X$, dos arcos cualesquiera $\alpha,\beta\in \Omega(X,x,y)$ son homotópicos por arcos.\\

    \noindent
    Sean $\alpha,\beta\in \Omega(X,x,y)$, tenemos que $\alpha\ast\tilde{\beta}$ es un lazo basado en $x$, y por ser $X$ simplemente conexo tenemos que $\left[\alpha\ast\tilde{\beta}\right] = [\varepsilon_x]$, de donde:
    \begin{equation*}
        [\alpha]\ast\left[\tilde{\beta}\right] = \left[\alpha\ast\tilde{\beta}\right] = [\varepsilon_x] \Longrightarrow [\alpha] = [\beta]
    \end{equation*}
\end{ejercicio}

\begin{ejercicio}
    Sean $X$ un subconjunto de $\mathbb{R}^n$ y $f:X\to Y$ una aplicación. Demuestra que si $f$ se puede extender a una aplicación continua $F:\mathbb{R}^n\to Y$, entonces $f_\ast$ es el homomorfismo trivial, es decir, el homomorfismo que lleva todo elemento en el neutro.\\

    \noindent
    Como $F$ es una extensión de $f$, tenemos que $f\circ i = F$, es decir:
    \begin{figure}[H]
        \centering
        \shorthandoff{""}
        \begin{tikzcd}
            {\pi_1(X,x_0)} \arrow[r, "i_\ast", hook] \arrow[rr, "f_\ast", bend right] & {\pi_1(\mathbb{R}^n,x_0)} \arrow[r, "F_\ast"] & {\pi_1(Y,f(x_0))}
        \end{tikzcd}
        \shorthandon{""}
    \end{figure}
    \noindent
    Y como cada una de ellas es continua ($f$ es continua por ser $f=F\big|_X$), podemos inducir el diagrama a grupos fundamentales, obteniendo para cada $x_0\in X$:
    \begin{figure}[H]
        \centering
        \shorthandoff{""}
        \begin{tikzcd}
            {\pi_1(X,x_0)} \arrow[r, "i_\ast", hook] \arrow[rr, "f_\ast", bend right=49] & {\pi_1(\mathbb{R}^n,x_0)} \arrow[r, "F_\ast"] & {\pi_1(Y,f(x_0))}
        \end{tikzcd}
        \shorthandon{""}
    \end{figure}
    de donde:
    \begin{equation*}
        f_\ast([\alpha]_X) = F_\ast(i_\ast([\alpha]_{X})) = F_\ast({[\alpha]}_{\bb{R}^n}) \AstIg F_\ast([\varepsilon_{x_0}]_{\mathbb{R}^n}) = [\varepsilon_{f(x_0)}]_Y \qquad \forall [\alpha]_X \in \pi_1(X,x_0)
    \end{equation*}
    donde en $(\ast)$ hemos usado que $\mathbb{R}^n$ es simplemente conexo.
\end{ejercicio}

\begin{ejercicio}
    Se dice que un grupo $G$ con operación $\cdot $ es un grupo topológico si $G$ tiene una topología de forma que las aplicaciones producto e inversión
    \Func{}{G\times G}{G}{(x,y)}{x\cdot y} \Func{}{G}{G}{x}{x^{-1}}
    son continuas. Sea e el elemento neutro en $G$:
    \begin{enumerate}[label=\alph*)]
        \item Dados $\alpha,\beta\in \Omega(G,e)$, se define $\alpha\cdot \beta:[0,1]\to G$ como $(\alpha\cdot \beta)(t) = \alpha(t)\cdot \beta(t)$. Demuestra que $\alpha\cdot \beta\in \Omega(G,e)$.

            Hemos de probar que $\alpha\cdot \beta$ es un lazo basado en $e$. Para ello:
            \begin{itemize}
                \item $\alpha\cdot \beta$ es continua, puesto que si consideramos:
                    \Func{\Phi}{G\times G}{G}{(x,y)}{x\cdot y}            
                    \Func{\Psi}{[0,1]}{G\times G}{t}{(\alpha(t), \beta(t))}
                    Tenemos que $\alpha\cdot \beta = \Phi \circ \Psi$:
                    \begin{equation*}
                        (\alpha\cdot \beta)(t) = \alpha(t)\cdot \beta(t) = \Phi(\alpha(t),\beta(t)) = \Phi(\Psi(t)) \qquad \forall t\in [0,1]
                    \end{equation*}
                    con $\Phi$ continua por hipótesis y $\Psi$ continua por ser $\Psi=(\alpha,\beta)$, con sus dos componentes funciones continuas, por ser arcos.
                \item Observemos que:
                    \begin{align*}
                        (\alpha\cdot \beta)(0) &= \alpha(0) \cdot \beta(0) = e\cdot e = e \\
                        (\alpha\cdot \beta)(1) &= \alpha(1) \cdot \beta(1) = e\cdot e = e \\
                    \end{align*}
            \end{itemize}
            Con lo que $\alpha\cdot \beta\in \Omega(G,e)$.
        \item Comprueba que $(\alpha\ast \varepsilon_e)\cdot (\varepsilon_e\ast \beta)=\alpha\ast \beta$ para cualesquiera $\alpha,\beta\in \Omega(G,e)$.

            Sean $\alpha,\beta\in \Omega(G,e)$, tenemos que:
            \begin{align*}
                ((\alpha\ast \varepsilon_e)&\cdot (\varepsilon_e\ast \beta))(t) = (\alpha\ast \varepsilon_e)(t)\cdot (\varepsilon_e\ast \beta)(t)  \\ &= \left\{\begin{array}{ll}
                    \alpha(2t)\cdot \varepsilon_e(2t) & \text{si\ } 0\leq t \leq \nicefrac{1}{2} \\
                    \varepsilon_e(2t-1)\cdot \beta(2t-1) & \text{si\ } \nicefrac{1}{2}\leq t \leq 1
                \end{array}\right.  \\
               &= \left\{\begin{array}{ll}
                   \alpha(2t)\cdot e & \text{si\ } 0\leq t \leq \nicefrac{1}{2} \\
                   e\cdot \beta(2t-1) & \text{si\ } \nicefrac{1}{2}\leq t\leq 1
           \end{array}\right\} = \left\{\begin{array}{ll}
               \alpha(2t) & \text{si\ } 0\leq t \leq \nicefrac{1}{2} \\
               \beta(2t-1) & \text{si\ } \nicefrac{1}{2}\leq t \leq 1
           \end{array}\right.\\ &= (\alpha\ast \beta)(t) \qquad \forall t\in [0,1]
            \end{align*}

        \item Sean $[\alpha],[\beta]\in \pi_1(G,e)$. Prueba que la operación $[\alpha]\cdot [\beta]=[\alpha\cdot \beta]$ está bien definida.

            Sean $\alpha,\alpha',\gamma,\gamma'\in \Omega(G,e)$ de forma que:
            \begin{equation}\label{eq:igualdades_ej3rel2}
                [\alpha] = [\alpha'], \qquad [\gamma] = [\gamma']
            \end{equation}
            hemos de probar que $[\alpha\cdot \gamma] = [\alpha'\cdot \gamma']$. De las igualdades~(\ref{eq:igualdades_ej3rel2}) sabemos que existen $H_1,H_2:[0,1]\times [0,1]\to G$ aplicaciones continuas con:
            \begin{gather*}
                H_1(s,0) = \alpha(s),\qquad  H_1(s,1) = \alpha'(s),\qquad  H_1(0,t) = e = H_1(1,t) \\
                H_2(s,0) = \gamma(s),\qquad  H_2(s,1) = \gamma'(s),\qquad  H_2(0,t) = e = H_2(1,t) 
            \end{gather*}
            Si definimos $H:[0,1]\times[0,1]\to G$ dada por:
            \begin{equation*}
                H(s,t) = H_1(s,t)\cdot H_2(s,t) \qquad \forall (s,t)\in [0,1]\times[0,1]
            \end{equation*}
            tenemos que $H$ es continua, ya que podemos verla como $H = \Phi\circ (H_1,H_2)$, al igual que inicmos en el apartado a), así como que:
            \begin{gather*}
                H(s,0) = H_1(s,0)\cdot H_2(s,0) = \alpha(s)\cdot \gamma(s) = (\alpha\cdot \gamma)(s) \\
                H(s,1) = H_1(s,1)\cdot H_2(s,1) = \alpha'(s)\cdot \gamma'(s) = (\alpha'\cdot \gamma')(s) \\
                H(0,t) = H_1(0,t)\cdot H_2(0,t) = e\cdot e = e = e\cdot e = H_1(1,t)\cdot H_2(1,t) = H(1,t)
            \end{gather*}
            Con lo que $H$ es una homotopía, lo que nos dice que $[\alpha\cdot \gamma] = [\alpha'\cdot \gamma']$, luego la operación está bien definida.
        \item Muestra que $[\alpha]\cdot [\beta]=[\alpha]\ast[\beta]$, para cada $[\alpha],[\beta]\in \pi_1(G,e)$.
            \begin{equation*}
                [\alpha]\cdot [\beta] = [\alpha\ast \varepsilon_e] \cdot [\varepsilon_e\ast \beta] = [(\alpha\ast\varepsilon_e)\cdot (\varepsilon_e\ast\beta)] \stackrel{b)}{=} [\alpha\ast\beta] = [\alpha]\ast [\beta]
            \end{equation*}
        \item Demuestra que $\pi_1(G,e)$ es abeliano.

            Sean $[\alpha],[\beta]\in \pi_1(G,e)$, tenemos que:
            \begin{align*}
                [\alpha]\ast[\beta] &= [\alpha\ast \beta] = [(\alpha\ast \varepsilon_e)\cdot (\varepsilon_e\ast \beta)] = [\alpha\ast \varepsilon_e] \cdot [\varepsilon_e\ast \beta]=  [\varepsilon_e\ast \alpha] \cdot [\beta\ast\varepsilon_e]  \\
                                    &= \left[ \left\{\begin{array}{ll}
                                         e\cdot \beta(2t)& \text{si\ } 0\leq t \leq \nicefrac{1}{2} \\
                                         \alpha(2t-1)\cdot e& \text{si\ } \nicefrac{1}{2}\leq t \leq 1
                             \end{array}\right.  \right] = [\beta\ast \alpha] = [\beta]\ast [\alpha]
            \end{align*}
    \end{enumerate}
\end{ejercicio}

\begin{ejercicio}
    Sean $X$ un espacio topológico y $f,g:X\to \bb{S}^n$ aplicaciones continuas con $g(x)\neq -f(x)$ para cada $x\in X$. Prueba que $f$ y $g$ son homotópicas. Deduce que si $f:\bb{S}^n\to \bb{S}^n$ es continua y carece de puntos fijos, entonces $f$ es homotópica a $-Id_{\bb{S}^n}$.\\

    \noindent
    Definimos $H:X\times [0,1]\to Y$ dada por:
    \begin{equation*}
        H(x,t) = \dfrac{(1-t)f(x) + tg(x)}{\|(1-t)f(x) + tg(x)\|_2} \qquad \forall (x,t)\in X\times [0,1]
    \end{equation*}
    \begin{itemize}
        \item $H$ está bien definida (es decir, el denominador no se anula), ya que si tenemos $x\in X$ y $t\in [0,1]$ de forma que $(1-t)f(x)+tg(x) = 0 $, entonces:
            \begin{equation*}
                (1-t)f(x) = -tg(x)
            \end{equation*}
            de donde:
            \begin{equation*}
                1-t = (1-t)\|f(x)\|_2 = \|(1-t)f(x)\|_2 = \|-tg(x)\|_2 = t\|g(x)\|_2 = t
            \end{equation*}
            por lo que ha de ser $t = \nicefrac{1}{2}$. Sin embargo, la condición $g(x)\neq -f(x)$ implica que $f(x) \neq -\nicefrac{1}{2}g(x)$, por lo que es imposible que el denominador se anule.
        \item $H$ es continua.
        \item Observamos que:
            \begin{equation*}
                H(x,0) = \dfrac{f(x)}{\|f(x)\|_2} = f(x) \qquad 
                H(x,1) = \dfrac{g(x)}{\|g(x)\|_2} = g(x) 
            \end{equation*}
    \end{itemize}
    con lo que $H$ es una homotopía entre $f$ y $g$.\\

    \noindent
    Sea ahora $f:\bb{S}^n\to \bb{S}^n$ una aplicación sin puntos fijos, entonces:
    \begin{equation*}
        f(x) \neq x = -(-x) = -(-Id_{\bb{S}^n})(x) \qquad \forall x\in \bb{S}^n
    \end{equation*}
    y si aplicamos la parte del ejercicio que acabamos de probar, obtenemos que $f$ es homotópica a $-Id_{\bb{S}^n}$.
\end{ejercicio}

\begin{ejercicio}
    Sea $p:R\to B$ una aplicación recubridora y $b\in B$. Demuestra que el subespacio topológico $p^{-1}(\{b\})\subset R$ tiene la topología discreta.\\

    \noindent
    Sea $X = p^{-1}(\{b\})$, como $p$ es una aplicación recubridora, podemos tomar un abierto $O_b$ que contiene a $b$ y está regularmente recubierto, con lo que existen $\{A_i\}_{i \in I}$ conjuntos abiertos de $R$ de forma que:
    \begin{equation*}
        p^{-1}(O_b) = \biguplus_{i \in I}A_i
    \end{equation*}
    Sea $r\in X\subseteq p^{-1}(O_b)$, tenemos entonces que existe un índice $j\in I$ de forma que $r\in A_j$. Veamos que $A_j$ no puede contener dos elementos distintos de $X$, pues si $r,r'\in A_j\cap X$, tenemos que:
    \begin{equation*}
        p\big|_{A_j}(r) = b = p\big|_{A_j}(r') 
    \end{equation*}
    y como $p\big|_{A_j}:A_j\to O_b$ es un homeomorfismo por ser $p$ una aplicación recubridora, tenemos en particular que es inyectiva, luego $r = r'$. En definitiva, hemos probado que si $r\in X$, entonces existe un índice $j\in I$ de forma que $\{r\} = X\cap A_j$, con $A_j$ un abierto de $R$, por lo que $\{r\}$ es un abierto de $X$, $\forall r\in X$, con lo que $X$ tiene la topología discreta.
\end{ejercicio}

\begin{ejercicio}
    Demuestra que toda aplicación recubridora es una aplicación abierta.\\

    \noindent
    Sea $p:R\to B$ una aplicación recubridora y $U$ un abierto de $R$, queremos probar que $p(U)$ es un abierto de $B$. Para ello, sea $y\in p(U)$, existirá $x\in R$ de forma que $p(x) = y$. Como $p$ es una aplicación recubridora, existirá $O_y$ abierto de $B$ con $y\in O_y$ y de forma que $O_y$ está regularmente recubierto, es decir, existe una familia $\{A_i\}_{i \in I}$ de abiertos disjuntos de $R$ de forma que:
    \begin{equation*}
        x\in p^{-1}(O_y) = \biguplus_{i \in I}A_i
    \end{equation*}
    con lo que tenemos un cierto índice $j \in I$ de modo que $x\in A_j$, luego $x\in U\cap A_j$, siendo $U\cap A_j$ un conjunto abierto, como intersección de conjuntos abiertos. Como $p\big|_{A_j}:A_j\to O_y$ es un homeomorfismo por ser $p$ una aplicación recubridora, en particular $p\big|_{A_j}$ es abierta, luego $p\big|_{A_j}(U\cap A_j)$ es un abierto contenido en $p(U)\cap O_y$, que contiene a $y$. Como este procedimiento podemos repetirlo para todo $y\in p(U)$, concluimos que $p(U)$ es un abierto de $B$.
\end{ejercicio}

\begin{ejercicio}
    Sea $p:R\to B$ una aplicación recubridora, con $B$ conexo. Demuestra que si $p^{-1}(b_0)$ tiene $k$ elementos para algún $b_0\in B$, entonces $p^{-1}(b)$ tiene $k$ elementos para todo $b\in B$. En tal caso, se dice que $R$ es un recubridor de $k$ hojas de $B$.\\

    \noindent
    Sea:
    \begin{equation*}
        A = \{x\in B : p^{-1}(x) \text{\ tiene\ } k \text{\ elementos}\}
    \end{equation*}
    Como $b_0 \in A$, tenemos que $A\neq \emptyset $. Veamos que $A$ es abierto y cerrado:
    \begin{itemize}
        \item Si $a\in A$, existe un abierto regularmente recubierto $O_a$ de $B$ que contiene a $a$, con lo que existe una familia de abiertos $\{A_i\}_{i \in I}$ de $R$ de forma que:
            \begin{equation*}
                p^{-1}(O_a) = \biguplus_{i \in I}A_i
            \end{equation*}
            y tal que $p\big|_{A_i}:A_i\to O_a$ es un homeomorfismo, $\forall i \in I$. Para cada $x\in O_a$ podemos construir la aplicación $\Phi_x:I\to p^{-1}(x)$ de forma que a cada índice $i$ le hace corresponder aquel elemento de $A_i$ cuya imagen por $p$ es $x$.
            \begin{itemize}
                \item La aplicación $\Phi_x$ está bien definida, pues si $i \in I$, como la aplicación $p\big|_{A_i}:A_i\to O_a$ es un homeomorfismo, ha de existir un único elemento $y \in A_i$ tal que $p(y) = x$.
                \item $\Phi_x$ es inyectiva, pues si $i,j\in I$ con $\Phi_x(i) = \Phi_x(j)$, entonces tenemos $y\in A_i\cap A_j$ de forma que $p(y) = x$. Como la familia $\{A_i\}_{i \in I}$ es disjunta, ha de ser $i = j$.
                \item $\Phi_x$ es sobreyectiva, pues si:
                    \begin{equation*}
                        y \in p^{-1}(x) \subseteq p^{-1}(O_a) = \biguplus_{i \in I}A_i
                    \end{equation*}
                    entonces ha de existir $y \in A_i$ para cierto índice $i$ de forma que $p(y) = x$, con lo que $\Phi_x(i) = y$.
            \end{itemize}
            En definitiva, $\Phi_x$ es biyectiva para cada $x\in O_a$. Como en particular $p^{-1}(a)$ tiene $k$ elementos, tendremos entonces que $I$ tiene $k$ elementos, de donde $p^{-1}(x)$ tiene $k$ elementos, $\forall x\in O_a$, con lo que $O_a\subseteq A$. De donde deducimos que $A$ es abierto.
        \item Sea $x\in \overline{A}$, como $p$ es recubridora, existe un abierto regularmente recubierto $O_x$ de $B$ que contiene a $x$. Como $x\in \overline{A}$, se verifica que $\exists a\in O_x\cap A$. Como $O_x$ está regularmente recubierto, existe una familia de abiertos $\{A_i\}_{i \in I}$ de $R$ de forma que:
            \begin{equation*}
                p^{-1}(O_x) = \biguplus_{i \in I}A_i
            \end{equation*}
            tal que $p\big|_{A_i}:A_i\to O_a$ es un homeomorfismo $\forall i \in I$. Al igual que antes, para cada $y \in O_x$ podemos construir la aplicación $\Phi_y:I\to p^{-1}(y)$ de forma que a cada índice $i$ le hace corresponder aquel elemento de $A_i$ cuya imagen por $x$ es $y$, obteniendo una aplicación biyectiva. Como $a\in O_x\cap A$, tenemos que $p^{-1}(a)$ tiene $k$ elementos, por lo que $I$ ha de tener $k$ elementos, de donde deducimos que $p^{-1}(y)$ tiene $k$ elementos, para todo $y \in O_x$. En particular, $x\in O_x$, de donde $p^{-1}(x)$ tiene $k$ elementos, es decir, $x\in A$.
    \end{itemize}
\end{ejercicio}

\begin{ejercicio}
    Sean $p_1:X\to Y$ y $p_2:Y\to Z$ dos aplicaciones recubridoras. Prueba que si $p_2^{-1}(z)$ es finito para todo $z\in Z$, entonces $p_2\circ p_1:X\to Z$ es una aplicación recubridora.\\

    % \noindent
    % Como $p_1$ y $p_2$ son continuas y sobreyectivas, es claro que $p_2\circ p_1$ es continua y sobreyectiva, por lo que para probar que $p_2\circ p_1$ es una aplicación recubridora falta ver que todo elemento de $Z$ está contenido en un abierto regularmente recubierto.

    % Para ello, sea $z\in Z$, como $p_2$ es una aplicación recubridora, existe $O_z$ un abierto regularmente recubierto de $Z$ que contiene a $Z$, con lo que existe una familia de abiertos disjuntos $\{B_i\}_{i \in I}$ de forma que:
    % \begin{equation*}
    %     p_2^{-1}(O_z) = \biguplus_{i \in I}B_i
    % \end{equation*}
    % tal que $p_2\big|_{B_i}:B_i\to O_z$ es un homeomorfismo $\forall i \in I$. Como $p_2^{-1}(z)$ es finito (supongamos que tiene cardinal $k$), por un procedimiento similar al que hicimos en el ejercicio anterior podemos justificar que entonces $I$ tiene $k$ elementos, y que en cada uno de los abiertos $\{B_i\}_{i \in I}$ podemos encontrar una única preimagen de $z$, $\Phi_z(i)$. 

    % Para cada $i \in I$ (que es finito), tenemos que $y=\Phi_z(i)\in Y$ y como $p_1$ es recubridora, existe un abierto $O_i$ regularmente recubierto de $Y$ que contiene a $y$, por lo que existe una familia de abiertos disjuntos $\{A_j\}_{j \in J}$ de forma que:
    % \begin{equation*}
    %     p_1^{-1}(O_i) = \biguplus_{j \in J}A_j
    % \end{equation*}
    % tal que $p_1\big|_{A_j}:A_j\to O_i$ es un homeomorfismo $\forall j\in J$. Si tomamos:
    % \begin{equation*}
    %     C_i = B_i \cap O_i \qquad \forall  i \in I
    % \end{equation*}% // TODO: TERMINAR





    % Repasando lo que tenemos, para $z\in Z$ hemos encontrado un abierto $O_z$ de $Z$ que contiene a $z$ verificando que:
    % \begin{equation*}
    %     {(p_2\circ p_1)}^{-1}(O_z) = p_1^{-1}(p_2^{-1}(O_z)) = p_1^{-1}\left(\biguplus_{i \in I}B_i\right) = \biguplus_{i \in I}p^{-1}(B_i) = 
    % \end{equation*}
\end{ejercicio}

\begin{ejercicio}
    Consideremos una aplicación recubridora $p:R\to B$ y la relación de equivalencia $\cc{R}_p$ en $R$ dada por
    \begin{equation*}
        r_1 \cc{R}_p r_2 \Longleftrightarrow p(r_1) = p(r_2)
    \end{equation*}
    Demuestra que $R/\cc{R}_p$ es homeomorfo a $B$:
\end{ejercicio}

\begin{ejercicio}
   Sea $p:R\to B$ una aplicación recubridora, con $R$ arcoconexo y $B$ simplemente conexo. Prueba que p es un homeomorfismo.
\end{ejercicio}

\begin{ejercicio}
    Dado un espacio topológico $Y$, prueba que estas afirmaciones son equivalentes:
    \begin{enumerate}[label=\alph*)]
        \item $Y$ es contráctil.
        \item Para cualesquiera $f,g:X\to Y$ continuas se tiene que $f$ y $g$ son homotópicas.
        \item Cada aplicación continua $f:X\to Y$ es nulhomótopa.
        \item La identidad $Id_Y$ es nulhomótopa.
        \item Cada conjunto $\{y_0\}$ con $y_0\in Y$ es un retracto de deformación de $Y$.
    \end{enumerate}
\end{ejercicio}

\begin{ejercicio}
    Prueba que $X=([0,1]\times \{0\})\cup ((K\cup \{0\})\times [0,1])\subset \mathbb{R}^2$ es contráctil, donde $K = \{\frac{1}{m}:m\in \mathbb{N}\}$.
\end{ejercicio}

\begin{ejercicio}
    Sea $f:\mathbb{R}\to \mathbb{R}^+$ una función continua. Definimos el conjunto:
    \begin{equation*}
        S_f = \{(x,y,z)\in \mathbb{R}^3: x^2+y^2 = {(f(z))}^{2}\}
    \end{equation*}
    \begin{enumerate}[label=\alph*)]
        \item Estudia el conjunto $S_f\cap \{z=z_0\}$ con $z_0\in \mathbb{R}$.
        \item Demuestra que cualesquiera dos conjuntos $S_f$ son homeomorfos entre sí.
        \item Calcula el grupo fundamental de $S_f$.
    \end{enumerate}
\end{ejercicio}

\begin{ejercicio}
    Prueba que $\mathbb{R}\times \left[0,+\infty\right[$ no es homeomorfo a $\mathbb{R}^2$. ¿Son del mismo tipo de homotopía?
\end{ejercicio}

\begin{ejercicio}
    Sea $S$ un subespacio afín de $\mathbb{R}^n$ de dimensión $k\leq n-2$. Calcula $\pi_1(\mathbb{R}^n\setminus S)$.
\end{ejercicio}

\begin{ejercicio}
    Prueba que si $X$ es de Hausdorff y $A\subseteq X$ es un rectracto de $X$, entonces $A$ es cerrado en $X$. Deduce que una bola abierta en $\mathbb{R}^n$ no es un retracto de $\mathbb{R}^n$. ¿Lo es una bola cerrada?
\end{ejercicio}

\begin{ejercicio}
    En este ejercicio demostraremos que \textit{un abierto de $\mathbb{R}^2$ no puede ser homeomorfo a un abierto de $\mathbb{R}^n$ si} $n\geq 3$. Supongamos que $f:U\to V$ es un homeomorfismo entre abiertos no vacíos $U\subset \mathbb{R}^n$ y $V\subset \mathbb{R}^2$ con $n\geq 3$.
    \begin{enumerate}[label=\alph*)]
        \item Prueba que existen bolas abiertas $B_1\subset U$ y $B_2,B_2'\subset V$ (estas últimas con el mismo centro $y_0\in \mathbb{R}^2$) tales que $\overline{B_2'}\subset f(\overline{B_1})\subset \overline{B_2}$.
        \item Si $i:\overline{B_2'}\setminus \{y_0\}\to \overline{B_2}\setminus \{y_0\}$ es la inclusión, deduce de $a)$ que el homomorfismo inducido en cualquier punto $i_\ast$ es trivial.
        \item Prueba que $\overline{B_2'}\setminus \{y_0\}$ es un retracto de deformación de $\overline{B_2}\setminus \{y_0\}$. Concluye que $i_\ast$ es un isomorfismo no trivial, lo que contradice $b)$.
    \end{enumerate}
\end{ejercicio}

\begin{ejercicio}
    Demuestra que el sistema de ecuaciones:
    \begin{equation*}
        \left\{\begin{array}{l}
            x-\arctg(x^2-y^3) = 2 \\
            \cos(x) + \sen(xy^3) + e^x + e^{y^2} + \dfrac{1}{y} = -5
        \end{array}\right.
    \end{equation*}
    tiene al menos una solución en $\mathbb{R}^2$.
\end{ejercicio}

\begin{ejercicio}
    Sean $M$ una matriz cuadrada real de orden 3 por 3 cuyas entradas son números reales positivos
    y $f:\mathbb{R}^3\to \mathbb{R}^3$ la aplicación lineal $f(v)=Mv$. Demuéstrese que:
    \begin{enumerate}[label=\alph*)]
        \item El conjunto $A=\{(x,y,z)\in \bb{S}^2:x_1,x_2,x_3\geq 0\}$ es homeomorfo al disco cerrado $\overline{\bb{D}}$.
        \item La aplicación $g:A\to \bb{S}^2$ dada por $g(v) = \frac{f(v)}{|f(v)|}$ está bien definida y $g(A)\subset A$.
        \item $f$ tiene un valor propio real y positivo.
    \end{enumerate}
\end{ejercicio}

\begin{ejercicio}
    Teorema de Lusternik-Schnirelmann. Demuestra que si $\bb{S}^2$ es la unión de tres subconjuntos cerrados $C_1,C_2,C_3$, entonces alguno de ellos contiene dos puntos antípodas. Para ello prueba que la función $f:\bb{S}^2\to \bb{R}^2$ dada por
    \begin{equation*}
        f(x) = (dist(x,C_1), dist(x,C_2))
    \end{equation*}
    tiene un putno $x_0\in \bb{S}^2$ tal que $f(x_0) = f(-x_0)$, donde $dist(\cdot ,\cdot )$ denota la función distancia en $\mathbb{R}^3$.
\end{ejercicio}

\begin{ejercicio}
    Calcula $\pi_1(X)$ en los siguientes casos:
    \begin{enumerate}[label=\alph*)]
        \item $X=\bb{S}^2 \cup (\bb{D}\times \{0\})$.
        \item $X=(\bb{S}^1 \times [-1,1])\cup (\bb{D}\times \{-1,1\})$.
        \item $X = \{(x,y,z)\in \mathbb{R}^3 : x^2+y^2 = {(z+1)}^{2}, -1\leq z\leq 0\}\cup (\bb{S}^2 \cap \{z\geq 0\})$.
        \item $X = S_1 \cup S_2 \cup L$, donde $S_1,S_2$ son cerrados disjuntos simplemente conexos de $\mathbb{R}^n$ y $L\subset \mathbb{R}^n$ es un segmento tal que $L\cap S_i =\{x_i\}$, $i = 1,2$.
        \item $X\subset \mathbb{R}^3$ es la unión de una circunferencia y de una esfera que se tocan en un único punto.
        \item $X = \bb{S}^2\cup \{(x,y,z)\in \mathbb{R}^3 : y^2 + {(z-2)}^{2} = 1\}\cup \{(x,y,z)\in \mathbb{R}^3 : y^2 + {(z+2)}^{2}=1\}$.
        \item $X = S_1 \cup (\bb{S}^1 \times \{0\})\cup S_2$, donde $S_1$ y $S_2$ son, respectivamente, las esferas de radio 1 centradas en el $(0,-2,0)$ y en el $(0,2,0)$.
        \item $X\subset \mathbb{R}^2$ es la unión de las tres circunferencias de radio 1 centradas en los puntos $(-2,0)$, $(0,0)$ y $(2,0)$.
        \item $X = \bb{S}^1\cup [(-1,0), (1,0)]$.
    \end{enumerate}
\end{ejercicio}

\begin{ejercicio}
    Razona si son verdaderas o falsas las siguientes afirmaciones:
    \begin{enumerate}[label=\alph*)]
        \item Sean $\alpha_1,\alpha_2,\beta_1,\beta_2\in \Omega(X,x_0)$  con $[\alpha_1\ast \beta_1] = [\alpha_2\ast \beta_2]$. Entonces $[\alpha_1] = [\alpha_2]$ y $[\beta_1] = [\beta_2]$.
        \item Sean $f:A\to Y$ una aplicación continua con $A\subset X$ y $X$ simplemente conexo. Si existe $F:X\to Y$ continua con $F\big|_A = f$, entonces $f_\ast$ es trivial.
        \item Si $f:X\to Y$ es continua y nulhomótopa, entonces $f_\ast$ es trivial.
        \item Si $f:X\to \bb{S}^n$ es continua y no sobreyectiva, entonces es nulhomótopa.
        \item Si $X$ es simplemente conexo y $A\subset X$ un rectracto de $X$, entonces $A$ es simplemente conexo.
        \item Si $f:X\to Y$ es continua e inyectiva con $f(x_0)=y_0$ entonces $f_\ast:\pi_1(X,x_0)\to \pi_1(Y,y_0)$ es un monomorfismo.
        \item Sea $f:X\to Y$ una equivalencia homotópica y $A\subset X$. La restricción $f\big|_A:A\to f(A)$ es una equivalencia homotópica.
        \item $\bb{S}^1$ no tiene ningún retracto de deformación $A\neq \bb{S}^1$.
        \item Existe un homeomorfismo de $\mathbb{R}^2$ que intercambia las componentes conexas de $\mathbb{R}^2\setminus \bb{S}^1$.
        \item Si $A$ es un retracto del disco unidad cerrado de $\mathbb{R}^2$, entonces toda aplicación continua $f:A\to A$ tiene al menos un punto fijo.
    \end{enumerate}
\end{ejercicio}

    \newpage
\section{Suficiencia y completitud}

\begin{ejercicio}
    Sea $(X_1, \ldots, X_n)$ una muestra aleatoria simple de una variable $X\rightsquigarrow\{B(k,p) : p\in \left]0,1\right[\}$ y sea $T(X_1, \ldots, X_n) = \sum\limits_{i=1}^n X_i$. Probar
    \begin{enumerate}[label=\alph*)]
        \item usando la definición
        \item aplicando el teorema de factorización
    \end{enumerate}
    que $T$ es suficiente para $p$.

    \begin{enumerate}[label=\alph*)]
        \item Si notamos para abreviar $T=T(X_1, \ldots, X_n)$, tenemos que probar que la distribución de la muestra condicionada a cualquier valor del estadístico no depende del parámetro $p$ para probar que $T$ es suficiente para $p$. Para ello:
            \begin{align*}
                P_p[X_1 = x_1, \ldots, X_n = x_n \ &|\ T=t] = \dfrac{P_p[X_1=x_1, \ldots, X_n = x_n, T=t]}{P_p[T=t]} \\
                                                           &= \left\{\begin{array}{ll}
                                                               0 & \text{si\ } T(x_1,\ldots,x_n)\neq t \\
                                                               \dfrac{P_p[X_1=x_1, \ldots, X_n=x_n]}{P_p[T=t]}& \text{si\ } T(x_1,\ldots,x_n) = t
                                                           \end{array}\right. 
            \end{align*}
            Como $0$ obviamente no depende de $p$, el caso $T(x_1,\ldots,x_2)\neq t$ se encuentra ya estudiado, por lo que nos centramos en el caso $T(x_1,\ldots,x_2)=t$:
            \begin{align*}
                \dfrac{P_p[X_1=x_1, \ldots, X_n = x_n]}{P_p[T=t]} &\stackrel{\text{iid.}}{=} \dfrac{\prod\limits_{i=1}^{n}P_p[X=x_i]}{P_p[T=t]} \AstIg \dfrac{\prod\limits_{i=1}^n \binom{k}{x_i} p^{x_i} {(1-p)}^{k-x_i}}{\binom{nk}{t}p^t{(1-p)}^{nk-t}} \\
                                                                  &= \dfrac{p^{\sum\limits_{i=1}^n x_i} {(1-p)}^{nk - \sum\limits_{i=1}^n x_i}\prod\limits_{i=1}^n \binom{k}{x_i}}{\binom{nk}{t}p^t{(1-p)}^{nk-t}} \stackrel{(\ast\ast)}{=} \dfrac{\prod\limits_{i=1}^{n}\binom{k}{x_i}}{\binom{nk}{t}}
            \end{align*}
            Donde en $(\ast)$ hemos usado que $(X_1, \ldots, X_n)$ es una m.a.s. (variables independientes) y la reproductividad de la Binomial, por lo que $T\rightsquigarrow B(nk,p)$; y en $(\ast\ast)$ hemos usado que $t=T(x_1, \ldots, x_n) = \sum\limits_{i=1}^{n}x_i$. En definitiva, hemos obtenido que la distribución de la muestra condicionada a cualquier valor del estadístico no dependende del parámetro $p$, por lo que $T$ es suficiente para $p$.
        \item Si podemos usar el Teorema de factorización, escribimos la función masa de probabilidad de la distribución conjunta de la muestra aleatoria simple:
            \begin{align*}
                P[X_1=x_1, \ldots, X_n = x_n] &\stackrel{\text{iid.}}{=} \prod_{i=1}^{n} P[X=x_i] = \prod_{i=1}^{n} \binom{k}{x_i} p^{x_i} {(1-p)}^{k-x_i} \\
                                              &= p^{\sum\limits_{i=1}^n x_i} {(1-p)}^{nk - \sum\limits_{i=1}^n x_i} \prod_{i=1}^{n} \binom{k}{x_i} 
            \end{align*}
            Si tomamos:
            \begin{equation*}
                h(x_1, \ldots, x_n) = \prod_{i=1}^{n}\binom{k}{x_i}, \qquad 
                T(X_1, \ldots, X_n) = \sum_{i=1}^{n} X_i, \qquad 
                g_p(t) = p^t {(1-p)}^{nk - t}
            \end{equation*}
            Podemos aplicar el Teorema de Factorización de Neymann-Fisher, obteniendo que $T$ es un estadístico suficiente para $p$.
    \end{enumerate}
\end{ejercicio}

\begin{ejercicio}
    Sea $(X_1, \ldots, X_n)$ una muestra aleatoria simple de una variable $X\rightsquigarrow\{\cc{P}(\lm) : \lm\in \mathbb{R}^+\}$ y sea $T(X_1, \ldots, X_n) = \sum\limits_{i=1}^n X_i$. Probar
    \begin{enumerate}[label=\alph*)]
        \item usando la definición
        \item aplicando el teorema de factorización
    \end{enumerate}
    que $T$ es suficiente para $\lm$.

    \begin{enumerate}[label=\alph*)]
        \item Notando $T=T(X_1, \ldots, X_n)$, seguimos los mismos pasos que en el ejercicio anterior:
            \begin{align*}
                P_\lm[X_1 = x_1, \ldots, X_n = x_n \ &|\ T=t] = \dfrac{P_\lm[X_1=x_1, \ldots, X_n = x_n, T=t]}{P_\lm[T=t]} \\
                                                           &= \left\{\begin{array}{ll}
                                                               0 & \text{si\ } T(x_1,\ldots,x_n)\neq t \\
                                                               \dfrac{P_\lm[X_1=x_1, \ldots, X_n=x_n]}{P_\lm[T=t]}& \text{si\ } T(x_1,\ldots,x_n) = t
                                                           \end{array}\right. 
            \end{align*}
            Y ahora nos interesamos por el segundo término, que es el que puede depender de $\lm$:
            \begin{align*}
                \dfrac{P_\lm[X_1 = x_1, \ldots, X_n = x_n]}{P_\lm[T=t]} &\stackrel{\text{iid.}}{=} \dfrac{\prod\limits_{i=1}^{n} P_\lm[X=x_i]}{P_\lm[T=t]} \AstIg \dfrac{\prod\limits_{i=1}^n e^{-\lm} \dfrac{\lm^{x_i}}{x_i!}}{e^{-n\lm} \dfrac{{(n\lm)}^{t}}{t!}} = \dfrac{e^{-n\lm} \cdot \dfrac{\lm^{\sum\limits_{i=1}^n x_i}}{\prod\limits_{i=1}^n x_i!}}{e^{-n\lm}\cdot  \dfrac{n^t \lm^t}{t!}} \\
                                                                        &\stackrel{(\ast\ast)}{=} \dfrac{t!}{n^t \prod\limits_{i=1}^{n}x_i!}
            \end{align*}
            Donde en $(\ast)$ usamos que $T\rightsquigarrow P(n\lm)$, por la reproductividad de la Poisson y en $(\ast\ast)$ usamos que $t = T(x_1, \ldots, x_n) = \sum_{i=1}^{n} x_i$. Obtenemos una cantidad que no depende de $\lm$, por lo que $T$ es suficiente para $\lm$.
        \item Si podemos aplicar el Teorema de factorización, escribimos:
            \begin{equation*}
                P_\lm[X_1=x_1, \ldots, X_n = x_n] \stackrel{\text{iid.}}{=} \prod_{i=1}^{n} P_\lm[X=x_i] = \prod_{i=1}^{n}e^{-\lm}\dfrac{\lm^{x_i}}{x_i!} = e^{-n\lm} \cdot \dfrac{\lm^{\sum\limits_{i=1}^{n} x_i}}{\prod\limits_{i=1}^{n}x_i!}
            \end{equation*}
            Si tomamos:
            \begin{equation*}
                h(x_1, \ldots, x_n) = \dfrac{1}{\prod\limits_{i=1}^{n}x_i!}, \qquad T(X_1, \ldots, X_n) = \sum_{i=1}^{n}X_i, \qquad g_\lm(t) = e^{-n\lm} \cdot \lm^t
            \end{equation*}
            Por el Teorema de Factorización de Neymann-Fisher obtenemos que $T$ es suficiente para $\lm$.
    \end{enumerate}
\end{ejercicio}

\begin{ejercicio}
    Sea $(X_1, X_2, X_3)$ una muestra aleatoria simple de una variable $X\rightsquigarrow\{B(1,p) : p \in \left]0,1\right[\}$. Probar que el estadístico $X_1 + 2X_2 + 3X_3$ no es suficiente.\\

    \noindent
    Consideramos el estadístico $T(X_1, X_2, X_3) = X_1 + 2X_2 + 3X_3$. El espacio muestral de $X$ es $\cc{X}=\{0,1\}$, por lo que el espacio muestral de $T$ es $\cc{T} = \{0, 1, 2, 3, 4, 5, 6\}$. Sabemos por un ejemplo visto en teoría que el ``truco'' para demostrar que $T(X_1, X_2,X_3)$ no es suficiente para $p$ es buscar un valor del espacio muestral $\cc{T}$ que provenga de varias combinaciones de estados del espacio muestral $\cc{X}^3$. Como $0,1,2$ solo provienen de una combinación del espacio muestral $\cc{X}^3$ (son $(0,0,0)$, $(1,0,0)$ y $(0,1,0)$ respectivamente), probamos buscar el contraejemplo con $t=3$, que proviene de considerar las observaciones de la muestra $(1,1,0)$ y $(0,0,1)$.

    Una vez explicado el procedimiento para buscar cuál es el valor de $t$ que funciona, procedemos a probar que la distribución de la muestra condicionada a dicho valor de $t$ depende del parámetro $p$. Para ello:
            \begin{align*}
                P_p[X_1 = x_1, X_2=x_2, X_3 = x_3 \ &|\ T=3] = \dfrac{P_p[X_1=x_1, X_2=x_2, X_3 = x_3, T=3]}{P_p[T=3]} \\
                                                           &= \left\{\begin{array}{ll}
                                                               0 & \text{si\ } T(x_1,x_2,x_2)\neq 3 \\
                                                               \dfrac{P_p[X_1=x_1, X_2=x_2, X_3=x_3]}{P_p[T=3]}& \text{si\ } T(x_1,x_2,x_3) = 3
                                                           \end{array}\right. 
            \end{align*}
            Vemos que el primer caso no puede depender nunca de $p$, por lo que buscamos probar que el segundo caso sí que depende de $p$. Para ello:
            \begin{align*}
                \dfrac{P_p[X_1=x_1, X_2=x_2, X_3=x_3]}{P_p[T=3]} &\stackrel{\text{iid.}}{=} \dfrac{P_p[X=x_1]P_p[X=x_2]P_p[X=x_3]}{P_p[T=3]} \\
                                                                 &\AstIg \dfrac{p^{x_1}{(1-p)}^{1-x_1} p^{x_2}{(1-p)}^{1-x_2}p^{x_3}{(1-p)}^{1-x_3}}{P_p[X_1=1,X_2=1,X_3=0]+P_p[X_1=0,X_2=0,X_3=1]} \\
                                                                 &= \dfrac{p^{x_1}{(1-p)}^{1-x_1} p^{x_2}{(1-p)}^{1-x_2}p^{x_3}{(1-p)}^{1-x_3}}{p\cdot p\cdot (1-p) + (1-p)(1-p)p} \\
            \end{align*}
            Donde en $(\ast)$ he usado la propiedad reproductiva de la Binomial, por lo que $T\rightsquigarrow B(3,p)$, así como que la condición $t=3$ provenía de los valores $(1,1,0)$ y $(0,0,1)$.
            Ahora, si tomamos $(x_1,x_2,x_3) = (1,1,0)$, tenemos que:
            \begin{align*}
                \dfrac{P_p[X_1=1, X_2=1, X_3=0]}{P_p[T=3]} &= \dfrac{p^{1}{(1-p)}^{0} p^{1}{(1-p)}^{0}p^{0}{(1-p)}^{1}}{p\cdot p\cdot (1-p) + (1-p)(1-p)p} \\
                                                                 &= \dfrac{pp(1-p)}{pp(1-p)+(1-p)(1-p)p} \\
                                                                 &= \dfrac{p^2(1-p)}{p(1-p)\cancelto{1}{(p+1-p)}} = p
            \end{align*}
            Que claramente depende de $p$, por lo que $T$ no es suficiente para $p$.
\end{ejercicio}

\begin{ejercicio}
    Aplicando el teorema de factorización, y basándose en una muestra de tamaño arbitrario, encontrar un estadístico suficiente para cada una de las siguientes familias de distribuciones (en las familias biparamétricas, suponer los casos de sólo un parámetro desconocido y de los dos desconocidos).
    \begin{enumerate}[label=\alph*)]
        \item $X\rightsquigarrow\{U(-\nicefrac{\theta}{2},\nicefrac{\theta}{2}) : \theta>0\}$
        \item $X\rightsquigarrow\{\Gamma(p,a) : p,a > 0\}$
        \item $X\rightsquigarrow\{\beta(p,q) : p,q>0\}$
        \item $X\rightsquigarrow\{P_{N_1,N_2} : N_1, N_2 \in \mathbb{N}, N_1 \leq N_2\}$ y la masa de probabilidad viene dada por:
            \begin{equation*}
                P_{N_1, N_2}[X=x] = \dfrac{1}{N_2 - N_1 + 1} \qquad x\in \{N_1, \ldots, N_2\}
            \end{equation*}
    \end{enumerate}

    \noindent
    En lo que sigue, supondremos que tenemos una m.a.s. $(X_1, \ldots, X_n)$ de variables idénticamente distribuidas a la respectiva variable $X$:
    \begin{enumerate}[label=\alph*)]
        \item Si $X\rightsquigarrow U(-\nicefrac{\theta}{2},\nicefrac{\theta}{2})$ con $\theta>0$:
            \begin{equation*}
                f(x_1, \ldots, x_n) \stackrel{\text{iid.}}{=}\prod_{i=1}^{n}f(x_i) = \prod_{i=1}^{n} \dfrac{1}{\theta} = \dfrac{1}{\theta^n} \qquad x_i \in \left]-\nicefrac{\theta}{2},\nicefrac{\theta}{2}\right[, \quad \forall i \in \{1,\ldots,n\}
            \end{equation*}
            Por lo que tendremos $-\nicefrac{\theta}{2}<X_{(1)}\leq X_{(n)} < \nicefrac{\theta}{2}$:
            \begin{equation*}
                f(x_1, \ldots, x_n) = \dfrac{1}{\theta^n} I_{\left]0,+\infty\right[}(X_{(1)}+\nicefrac{\theta}{2}) I_{\left]-\infty,0\right[}(X_{(n)}-\nicefrac{\theta}{n})
            \end{equation*}
            Si tomamos:
            \begin{gather*}
                h(x_1,\ldots,x_n)=1,\qquad T(X_1, \ldots, X_n) = (X_{(1)}, X_{(n)}) \\ g_{\theta}(t_1,t_2) = \dfrac{1}{\theta^n} I_{\left]0,+\infty\right[}(t_1 + \nicefrac{\theta}{2}) I_{\left]-\infty,0\right[}(t_2-\nicefrac{\theta}{2})
            \end{gather*}
            Por el Teorema de factorización, tenemos que $T(X_1, \ldots, X_n)$ es un estadístico suficiente para $\theta$.
        \item Si $X\rightsquigarrow\Gamma(p,a)$ con $p,a>0$:
            \begin{align*}
                f(x_1, \ldots, x_n) &\stackrel{\text{iid.}}{=} \prod_{i=1}^{n}f(x_i) = \prod_{i=1}^{n} \dfrac{a^p}{\Gamma(p)} x_i^{p-1} e^{-ax_i} = {\left(\dfrac{a^p}{\Gamma(p)}\right)}^{n} e^{-a\sum\limits_{i=1}^{n}x_i} \prod_{i=1}^{n}x_i^{p-1} \\ 
                                    &= {\left(\dfrac{a^p}{\Gamma(p)}\right)}^{n} e^{-a\sum\limits_{i=1}^{n}x_i} {\left(\prod_{i=1}^{n}x_i\right)}^{p-1} \qquad x_i \geq 0 \qquad \forall i \in \{1,\ldots,n\}
            \end{align*}
            \begin{itemize}
                \item Suponiendo que $p$ es conocida, podemos tomar:
                    \begin{gather*}
                        h(x_1, \ldots, x_n) = {\left(\prod_{i=1}^{n}x_i\right)}^{p-1}, \qquad  T(X_1, \ldots, X_n) = \sum_{i=1}^{n} X_i \\
                        g_a(t) = {\left(\dfrac{a^p}{\Gamma(p)}\right)}^{n}e^{-at}
                    \end{gather*}
                \item Suponiendo ahora que $a$ es conocida:
                    \begin{gather*}
                        h(x_1, \ldots, x_n) = e^{-a\sum\limits_{i=1}^n x_i}, \qquad T(X_1, \ldots, X_n) = \prod_{i=1}^{n}x_i \\
                        g_p(t) = {\left(\dfrac{a^p}{\Gamma(p)}\right)}^{n} t^{p-1}
                    \end{gather*}
                \item Si ahora tanto $p$ como $a$ son desconocidas, podemos tomar:
                    \begin{gather*}
                        h(x_1, \ldots, x_n) = 1, \qquad T(X_1, \ldots, X_n) = \left(\sum_{i=1}^{n}X_i, \prod_{i=1}^{n} X_i\right) \\
                        g_{(a,p)}(t_1,t_2) = {\left(\dfrac{a^p}{\Gamma(p)}\right)}^{n}e^{-at_1}t_2^{p-1}
                    \end{gather*}
            \end{itemize}
        \item Si $X\rightsquigarrow\beta(p,q)$ con $p,q>0$:
            \begin{align*}
                f(x_1, \ldots, x_n) &\stackrel{\text{iid.}}{=} \prod_{i=1}^{n}f(x_i) = \prod_{i=1}^{n} \dfrac{1}{\beta(p,q)} x_i^{p-1} {(1-x_i)}^{q-1} \\
                &= \dfrac{1}{{\beta(p,q)}^{n}} {\left(\prod_{i=1}^{n}x_i\right)}^{p-1} {\left(\prod_{i=1}^{n}(1-x_i)\right)}^{q-1} \quad x_i \in [0,1], \forall i \in \{1,\ldots,n\}
            \end{align*}
            \begin{itemize}
                \item Si $p$ es conocida, tomamos:
                    \begin{gather*}
                        h(x_1, \ldots, x_n) = {\left(\prod_{i=1}^{n}x_i\right)}^{p-1}, \qquad T(X_1, \ldots, X_n) = \prod_{i=1}^{n}(1-X_i) \\
                        g_q(t) = \dfrac{1}{{\beta(p,q)}^{n}} t^{q-1}
                    \end{gather*}
                \item Si $q$ es conocida:
                    \begin{gather*}
                        h(x_1, \ldots, x_n) = {\left(\prod_{i=1}^{n}(1-x_i)\right)}^{q-1}, \qquad T(X_1, \ldots, X_n) = \prod_{i=1}^{n}X_i \\
                        g_p(t) = \dfrac{1}{{\beta(p,q)}^{n}} t^{p-1}
                    \end{gather*}
                \item Si tanto $p$ como $q$ son parámetros:
                    \begin{gather*}
                        h(x_1, \ldots, x_n) = 1, \qquad T(X_1, \ldots, X_n) = \left(\prod_{i=1}^{n}X_i, \prod_{i=1}^{n}(1-X_i)\right) \\
                        g_{(p,q)}(t_1, t_2) = \dfrac{1}{{\beta(p,q)}^{n}}t_1^{p-1}t_2^{q-1}
                    \end{gather*}
            \end{itemize}
        \item Si $X\rightsquigarrow P_{N_1,N_2}$ con $N_1,N_2\in \mathbb{N}$, $N_1\leq N_2$, entonces:
            \begin{align*}
                P[X_1=x_1, \ldots, X_n=x_n]&\stackrel{\text{iid.}}{=} \prod_{i=1}^{n} P[X=x_i] =\prod_{i=1}^{n} \dfrac{1}{N_2-N_1+1} \\
                                           &= \dfrac{1}{{(N_2-N_1+1)}^{n}}, \quad x_i \in \{N_1, \ldots N_2\} 
            \end{align*}
            Como se tiene $N_1\leq X_{(1)}\leq X_{(n)}\leq N_2$, entonces:
            \begin{equation*}
                P[X_1=x_1, \ldots, X_n=x_n] = \dfrac{I_{-\bb{N}_0}(X_{(1)}-N_1)I_{\bb{N}_0}(X_{(n)}-N_2)}{{(N_2-N_1+1)}^{n}}
            \end{equation*}
            \begin{itemize}
                \item Si $N_1$ es conocido:
                    \begin{gather*}
                        h(x_1, \ldots, x_n) = I_{-\bb{N}_0}(X_{(1)}-N_1), \qquad T(X_1, \ldots, X_n) = X_{(n)} \\
                        g_{N_2}(t) = \dfrac{I_{\bb{N}_0}(t-N_2)}{{(N_2-N_1+1)}^{n}}
                    \end{gather*}
                \item Si $N_2$ es conocida:
                    \begin{gather*}
                        h(x_1, \ldots, x_n) = I_{\bb{N}_0}(X_{(n)}-N_2), \qquad T(X_1, \ldots, X_n) = X_{(1)} \\
                        g_{N_1}(t) = \dfrac{I_{-\bb{N}_0}(t-N_1)}{{(N_2-N_1+1)}^{n}}
                    \end{gather*}
                \item Si tanto $N_1$ como $N_2$ son parámetros:
                    \begin{gather*}
                        h(x_1,\ldots,x_n) = 1, \qquad T(X_1, \ldots, X_n) = \left(X_{(1)}, X_{(n)}\right) \\
                        g_{(N_1,N_2)}(t_1,t_2) = \dfrac{I_{-\bb{N}_0}(t_1-N_1)I_{\bb{N}_0}(t_2-N_2)}{{(N_2-N_1+1)}^{n}}
                    \end{gather*}
            \end{itemize}
            Por lo que en cualquier caso obtenemos un estadístico suficiente, por el Teorema de Factorización.
    \end{enumerate}
\end{ejercicio}

\begin{ejercicio}
    Sea $X\rightsquigarrow\{P_N : N\in \mathbb{N}\}$, siendo $P_N$ la distribución uniforme en los puntos $\{1,\ldots,N\}$, y sea $(X_1, \ldots, X_n)$ una muestra aleatoria simple de $X$. Probar que $\max(X_1, \ldots, X_n)$ es un estadístico suficiente y completo.\\

    \noindent
    Buscamos aplicar el Teorema de factorización de Neymann-Fisher:
    \begin{equation*}
        P[X_1 = x_1, \ldots, X_n = x_n] \stackrel{\text{iid.}}{=} \prod_{i=1}^{n}P[X=x_i] =  \prod_{i=1}^{n} \dfrac{1}{N}, \quad x_i \in \{1,\ldots, N\}, \forall i \in \{1,\ldots,n\}
    \end{equation*}
    Por lo que $1\leq X_{(1)} \leq X_{(n)} \leq N$:
    \begin{equation*}
        P[X_1 = x_1, \ldots, X_n=x_n] = \dfrac{I_{-\bb{N}_0}(X_{(1)}-1)I_{\bb{N}_0}(X_{(n)}-N)}{N^n}
    \end{equation*}
    De donde podemos tomar:
    \begin{gather*}
        h(x_1, \ldots, x_n) = I_{-\bb{N}_0}(X_{(1)}-1), \qquad T(X_1, \ldots, X_n) = X_{(n)} \\ 
        g_N(t) = \dfrac{I_{\bb{N}_0}(t-N)}{N^n}
    \end{gather*}
    Por el Teorema de factorización de Neymann-Fisher, tenemos que el estadístico $T(X_1, \ldots, X_n) = X_{(n)}$ es suficiente. Comprobamos que también es completo: sea $g$ cualquier función medible, supongamos que (abreviaremos $T(X_1, \ldots, X_n) = T$):

    \begin{equation*}
        0 = E[g(T)] = \sum_{t=1}^{N}g(t)P[T=t] \qquad \forall N\in \mathbb{N}
    \end{equation*}
    Calculamos la función masa de probabilidad de $T$:
    \begin{equation*}
        F_T(t) = {(F_X(t))}^{n} \Longrightarrow P[T=t] = P[T\leq t] - P[T\leq t-1] = {(F_X(t))}^{n} - {(F_X(t-1))}^{n}
    \end{equation*}
    Como $F_X(t) = \frac{t}{N}$, tenemos entonces que:
    \begin{equation*}
        P[T=t] = {(F_X(t))}^{n} - {(F_X(t-1))}^{n} = \dfrac{t^n}{N^n} - \dfrac{{(t-1)}^{n}}{N^n} = \dfrac{t^n - {(t-1)}^{n}}{N^n}
    \end{equation*}
    Por lo que:
    \begin{equation*}
        0 = E[g(T)] = \sum_{t=1}^{N} g(t) \dfrac{t^n -{(t-1)}^{n}}{N^n} = \dfrac{1}{N^n} \sum_{t=1}^{N} g(t) (t^n - {(t-1)}^{n}) \qquad \forall N\in \mathbb{N}
    \end{equation*}

    de donde:
    \begin{equation*}
        \sum_{t=1}^{N}g(t)(t^n -{(t-1)}^{n}) = 0 \qquad \forall N\in \mathbb{N}
    \end{equation*}
    Probemos por inducción sobre $N$ que $g(t) = 0\quad \forall t\in \mathbb{N}$:
    \begin{itemize}
        \item \underline{Para $N=1$:} tenemos que:
            \begin{equation*}
                g(1) = g(1)(1^n-{(1-1)}^{n}) = \sum_{t=1}^{1}g(t) (t^n - {(t-1)}^{n}) = 0
            \end{equation*}
        \item \underline{Supuesto que $g(t) = 0$ para $t<N$:}
            \begin{equation*}
                g(N)(N^n - {(N-1)}^{n}) = \sum_{t=1}^{N}g(t)(t^n-{(t-1)}^{n}) = 0
            \end{equation*}
            Por lo que:
            \begin{itemize}
                \item $g(N) = 0$.
                \item $N^n - {(N-1)}^{n}=0$, que es imposible, puesto que la potencia $n-$ésima es una función estrictamente creciente en el intervalo $\left[0,+\infty\right[$.
            \end{itemize}
            En definitiva, tenemos que:
            \begin{equation*}
                \mathbb{N} \subseteq \{t:g(t) = 0\}
            \end{equation*}
            Por lo que:
            \begin{equation*}
                1 \geq P[g(T) = 0] \geq P[T \in \mathbb{N}] = 1 \Longrightarrow P[g(T)= 0] = 1
            \end{equation*}
            Lo que demuestra que $X_{(n)}$ es completo.
    \end{itemize}
\end{ejercicio}

\begin{ejercicio}
    Basándose en una muestra de tamaño arbitrario, obtener un estadístico suficiente y completo para la familia de distribuciones definidas por todas las densidades de la forma
    \begin{equation*}
        f_\theta(x) = e^{\theta-x}, \qquad x>\theta
    \end{equation*}

    \noindent
    Sea $(X_1, \ldots, X_n)$ una m.a.s. de tamaño $n\in \mathbb{N}$: 
    \begin{equation*}
        f_{\theta}(x_1, \ldots, x_n)  \stackrel{\text{iid.}}{=} \prod_{i=1}^{n}f_\theta(x_i) = \prod_{i=1}^{n}e^{\theta - x_i} = e^{n\theta - \sum\limits_{i=1}^{n}x_i} = \dfrac{e^{n\theta}}{e^{\sum\limits_{i=1}^n x_i}}, \quad x_i > \theta\ \forall i \in \{1,\ldots,n\}
    \end{equation*}
    Por lo que ha de ser $\theta < X_{(1)}$:
    \begin{equation*}
        f_{\theta}(x_1, \ldots, x_n) = \dfrac{e^{n\theta}\cdot I_{\left]-\infty,0\right[}(X_{(1)}-\theta)}{e^{\sum\limits_{i=1}^n x_i}}
    \end{equation*}
    Si tomamos:
    \begin{gather*}
        h(x_1, \ldots, x_n) = e^{-\sum\limits_{i=1}^n x_i}, \qquad T(X_1, \ldots, X_n) = X_{(1)} \\
        g_\theta(t) = e^{n\theta}\cdot I_{\left]-\infty,0\right[}(t-\theta)
    \end{gather*}
    Por el Teorema de factorización de Neymann-Fisher tenemos que el estadístico $X_{(1)}$ es suficiente para $\theta$. Comprobemos si es completo: sea $g$ cualquier función medible, suponemos que (y escribimos $T=T(X_1, \ldots, X_n)$ para abreviar):
    \begin{equation*}
        0 = E[g(T)] = \int_{\theta}^{+\infty} g(t)f_T(t) ~dt  \qquad \forall \theta \in \mathbb{R}
    \end{equation*}
    Como $T=X_{(1)}$, tenemos que:
    \begin{equation*}
        F_T(t) = 1-{(1-F_\theta(t))}^{n} \Longrightarrow f_T(t) = n{(1-F_\theta(t))}^{n-1}f_\theta(t)
    \end{equation*}

    donde:
    \begin{equation*}
        F_\theta(t) = \int_{\theta}^{t} e^{\theta-x} ~dx  = \left[-e^{\theta-x}\right]_{\theta}^t = 1-e^{\theta-t}, \qquad t>\theta
    \end{equation*}

    por lo que:
    \begin{equation*}
        f_T(t) = n{\left(e^{\theta-t}\right)}^{n-1}e^{\theta-t} = n{\left(e^{\theta-t}\right)}^{n}, \qquad t>\theta
    \end{equation*}

    volviendo al caso que nos interesa:
    \begin{equation*}
        0 = E[g(T)] = \int_{\theta}^{+\infty} g(t)n{\left(e^{\theta-t}\right)}^{n} ~dt  = ne^{n\theta} \int_{\theta}^{+\infty} g(t)e^{-nt} ~dt  \qquad \forall \theta \in \mathbb{R}
    \end{equation*}

    por lo que:
    \begin{equation*}
        \int_{\theta}^{+\infty} g(t)e^{-nt}~dt  = 0 \qquad \forall \theta\in \mathbb{R}
    \end{equation*}
    Por simplicidad de cálculos, \underline{supondremos que $g$ es continua}, con lo que si $G(t)$ es una primitiva de $g(t)e^{-nt}$, entonces:
    \begin{equation*}
        \int_{\theta}^{+\infty} g(t)e^{-nt}~dt = \lim_{n\to\infty}\int_{\theta}^{n}g(t)e^{-nt} ~dt  = \lim_{n\to\infty}G(n)-G(\theta) =  0 \qquad \forall \theta\in \mathbb{R}
    \end{equation*}
    Si derivamos ahora respecto a $\theta$, tenemos que:
    \begin{equation*}
        -g(\theta)e^{-n\theta} = 0\qquad \forall \theta\in \mathbb{R}
    \end{equation*}

    con lo que $g(\theta) = 0 \quad \forall \theta\in \mathbb{R}$. Es decir:
    \begin{equation*}
        \mathbb{R}\subseteq \{t:g(t) = 0\}
    \end{equation*}

    luego:
    \begin{equation*}
        1\geq P[g(T)=0] \geq P[T\in \mathbb{R}] = 1 \Longrightarrow P[g(T)=0]=1
    \end{equation*}

    por lo que $T$ es completo.
\end{ejercicio}

\begin{ejercicio}
    Comprobar que las siguientes familias de distribuciones son exponenciales uniparamétricas y, considerando una muestra aleatoria simple de una variable con distribución en dicha familia, obtener, si existe, un estadístico suficiente y completo.
    \begin{enumerate}[label=\alph*)]
        \item $\{B(k_0,p) : 0<p<1\}$

            Comprobamos todas las condiciones:
            \begin{enumerate}[label=\arabic*.]
                \item El espacio paramétrico es: $\left]0,1\right[\subseteq \mathbb{R}$.
                \item El espacio muestral es $\cc{X} = \{0,\ldots,k_0\}$, que no depende de $p$.
                \item Para la tercera condición:
                    \begin{align*}
                        P[X=x]_p &= \binom{k_0}{p} p^{x}{(1-p)}^{k_0-x} = exp\left[\ln\left(\binom{k_0}{p}p^{x}{(1-p)}^{k_0-x} \right)\right]\\
                               &= exp\left[\ln\binom{k_0}{p} + x\ln p + (k_0-x)\ln(1-p)\right] \\
                               &= exp\left[\ln\binom{k_0}{p} + x\ln p + k_0 \ln (1-p) - x\ln(1-p)\right] \\
                               &= exp\left[\ln\binom{k_0}{p} + k_0 \ln (1-p) + x\ln\left(\frac{p}{1-p}\right)\right] 
                    \end{align*}
                    Tomando:
                    \begin{align*}
                        T(x) = x, &\qquad S(x) = 0 \\
                        Q(p) = \ln\left(\dfrac{p}{1-p}\right), &\qquad D(p) = \ln\binom{k_0}{p} + k_0\ln(1-p)
                    \end{align*}
                    obtenemos la tercera condición. 
            \end{enumerate}
            Sea $(X_1, \ldots, X_n)$ una m.a.s. de variables aleatorias idénticamente distribuidas a $X\rightsquigarrow B(k_0,p)$ con $p\in \left]0,1\right[$, el Teorema visto en teoría para las familias de distribuciones exponenciales nos dice que el estadístico:
            \begin{equation*}
                T = T(X_1, \ldots, X_n) = \sum_{i=1}^{n}T(X_i) = \sum_{i=1}^{n}X_i
            \end{equation*}
            es suficiente para $p$. Para ver que $T$ es también completo, hemos de ver que $Im Q$ contiene un abierto de $\mathbb{R}$:
            \begin{description}
                \item [Opción 1.] Como $Q:\left]0,1\right[\to \mathbb{R}$ es continua, no constante y definida sobre un intervalo, por el Teorema del Valor Intermedio su imagen ha de ser un intervalo, que es un abierto de $\mathbb{R}$, por lo que $T$ es completo.
                \item [Opción 2.] 
                \begin{align*}
                    Im Q = \left\{\ln\left(\frac{p}{1-p}\right) : p \in \left]0,1\right[\right\}
                \end{align*}
                Si definimos la función $f:\left]0,1\right[\to \mathbb{R}^+$, tenemos que $f$ es sobreyectiva (de hecho es biyectiva):
                \begin{itemize}
                    \item Está bien definida, puesto que si $x\in \left]0,1\right[$, entonces $1-x>0$, con lo que $f(x)\in \mathbb{R}^+$.
                    \item Sea $y\in \mathbb{R}^+$, tenemos que:
                        \begin{equation*}
                            \dfrac{t}{1-t} = y \Longleftrightarrow t = y(1-t) \Longleftrightarrow t=y-yt \Longleftrightarrow t(1+y) = y \Longleftrightarrow t = \dfrac{y}{y+1}<1
                        \end{equation*}
                        Por lo que $f(t) = y$ con $t\in \left]0,1\right[$, con lo que $f$ es sobreyectiva.
                \end{itemize}
                Por tanto, $Im f = \mathbb{R}^+$, de donde deducimos que:
                \begin{equation*}
                    Im Q = \left\{\ln\left(\frac{p}{1-p}\right) : p \in \left]0,1\right[\right\} = \{\ln(f(p)) : p\in \left]0,1\right[\} = \ln(Im f) = \ln(\mathbb{R}^+) = \mathbb{R}
                \end{equation*}
                Como obviamente $\mathbb{R}$ contiene algún abierto de $\mathbb{R}$, deducimos que $T$ era un estadístico completo.
            \end{description}
        \item $\{\cc{P}(\lm) : \lm>0\}$

            Comprobamos las condiciones:
            \begin{enumerate}[label=\arabic*.]
                \item El espacio paramétrico es $\mathbb{R}^+ \subseteq \mathbb{R}$.
                \item El espacio muestral es $\cc{X} = \mathbb{N}\cup \{0\}$, que no depende de $\lm$.
                \item Para la tercera condición:
                    \begin{align*}
                        P_\lm[X=x] &= e^{-\lm} \dfrac{\lm^x}{x!} = exp\left[\ln\left(e^{-\lm} \dfrac{\lm^x}{x!}\right)\right] = exp[-\lm + x\ln \lm - \ln(x!)]
                    \end{align*}
                    Tomando:
                    \begin{align*}
                        T(x) = x, &\qquad S(x) = -\ln(x!) \\
                        Q(\lm) = \ln\lm, &\qquad D(\lm) = -\lm
                    \end{align*}
                    obtenemos la tercera condición.
            \end{enumerate}
            Sea $(X_1, \ldots, X_n)$ una m.a.s. de variables aleatorias idénticamente distribuidas a $X\rightsquigarrow \cc{P}(\lm)$ con $\lm\in \mathbb{R}^+$, el Teorema visto en teoría para las familias de distribuciones exponenciales nos dice que el estadístico:
            \begin{equation*}
                T = T(X_1, \ldots, X_n) = \sum_{i=1}^{n}T(X_i) = \sum_{i=1}^{n}X_i
            \end{equation*}
            es suficiente para $\lm$. Para ver que $T$ es también completo, hemos de ver que $Im Q$ contiene un abierto de $\mathbb{R}$. Como:
            \begin{equation*}
                Im Q = \{\ln(\lm) : \lm \in \mathbb{R}^+\} = \ln(\mathbb{R}^+) = \mathbb{R}
            \end{equation*}
            Tenemos que $Im Q = \mathbb{R}$ claramente contiene un abierto de $\mathbb{R}$, por lo que $T$ es completo.
        \item $\{BN(k_0,p) : 0<p<1\}$

            Comprobamos las condiciones:
            \begin{enumerate}[label=\arabic*.]
                \item El espacio paramétrico es $\left]0,1\right[\subseteq \mathbb{R}$.
                \item El espacio muestral es $\cc{X} = \mathbb{N}\cup \{0\}$, que no depende de $p$.
                \item Para la tercera condición:
                    \begin{align*}
                        P_p[X=x] &= \binom{x+k_0-1}{x} {(1-p)}^{x}p^{k_0 }= exp\left[\ln\left(\binom{x+k_0-1}{x} {(1-p)}^{x}p^{k_0} \right)\right] \\
                                 &= exp\left[\ln\binom{x+k_0-1}{x} + x\ln(1-p) + k_0\ln p\right]
                    \end{align*}
                    Tomando:
                    \begin{align*}
                        T(x) = x, &\qquad S(x) = \ln\binom{x+k_0-1}{x} \\
                        Q(p) = \ln(1-p), &\qquad D(p) = k_0\ln p
                    \end{align*}
                    Obtenemos la tercera condición.
            \end{enumerate}
            Sea $(X_1, \ldots, X_n)$ una m.a.s. de variables aleatorias idénticamente distribuidas a $X\rightsquigarrow BN(k_0,p)$ con $p\in \left]0,1\right[$, el Teorema visto en teoría para las familias de distribuciones exponenciales nos dice que el estadístico:
            \begin{equation*}
                T = T(X_1, \ldots, X_n) = \sum_{i=1}^{n}T(X_i) = \sum_{i=1}^{n}X_i
            \end{equation*}
            es suficiente para $p$. Para ver que $T$ es también completo, hemos de ver que $Im Q$ contiene un abierto de $\mathbb{R}$. Como $Q:\left]0,1\right[\to\mathbb{R}$ es una función continua, no constante y definida en un intervalo, tenemos que su imagen es un intervalo, por lo que contiene abiertos de $\mathbb{R}$, de donde $T$ es completo.
        \item $\{exp(\lm) : \lm>0\}$

            Comprobamos las condiciones:
            \begin{enumerate}[label=\arabic*.]
                \item El espacio paramétrico es $\mathbb{R}^+ \subseteq \mathbb{R}$.
                \item El espacio muestral es $\cc{X} = \mathbb{R}^+$, que no depende de $\lm$.
                \item Para la tercera condición:
                    \begin{equation*}
                        f_\lm(x) = \lm e^{-\lm x} = exp\left[\ln\left(\lm e^{-\lm x} \right)\right] = exp\left[\ln(\lm) -\lm x\right]
                    \end{equation*}
                    Tomando:
                    \begin{align*}
                        T(x) = x, &\qquad S(x) = 0 \\
                        Q(\lm) = -\lm, &\qquad D(\lm) = \ln(\lm)
                    \end{align*}
                    tenemos la tercera condición.
            \end{enumerate}
            Sea $(X_1, \ldots, X_n)$ una m.a.s. de variables aleatorias idénticamente distribuidas a $X\rightsquigarrow exp(\lm)$ con $\lm \in \mathbb{R}^+$, el Teorema visto en teoría para las familias de distribuciones exponenciales nos dice que el estadístico:
            \begin{equation*}
                T = T(X_1, \ldots, X_n) = \sum_{i=1}^{n}T(X_i) = \sum_{i=1}^{n}X_i
            \end{equation*}
            es suficiente para $\lm$. Para ver que $T$ es también completo, hemos de ver que $Im Q$ contiene un abierto de $\mathbb{R}$. Como $Q:\mathbb{R}^+\to\mathbb{R}$ es una función continua, no constante y definida en un intervalo, tenemos que su imagen es un intervalo, por lo que contiene abiertos de $\mathbb{R}$, de donde $T$ es completo.
    \end{enumerate}
\end{ejercicio}

\begin{ejercicio}
    Estudiar si las siguientes familias de distribuciones son exponenciales biparamétricas.  En caso afirmativo, considerando una muestra aleatoria simple de una variable con distribución en dicha familia, obtener, si existe, un estadístico suficiente y completo.
    \begin{enumerate}[label=\alph*)]
        \item $\{\Gamma(p,a) : p,a>0\}$

            Comprobamos las condiciones:
            \begin{enumerate}[label=\arabic*.]
                \item El espacio paramétrico es $\mathbb{R}^+\times \mathbb{R}^+\subseteq \mathbb{R}^2$.
                \item El espacio muestral es $\mathbb{R}^+$, que no depende de $p$ ni de $a$.
                \item Para la tercera condición:
                    \begin{align*}
                        f_{(p,a)}(x) &= \dfrac{a^p}{\Gamma(p)} x^{p-1}e^{-ax} = exp\left[\ln\left(\dfrac{a^p}{\Gamma(p)} x^{p-1}e^{-ax} \right)\right] \\ &= exp\left[\ln\left(\dfrac{a^p}{\Gamma(p)}\right) + (p-1)\ln x - ax\right]
                    \end{align*}
                Tomando:
                \begin{align*}
                    T_1(x) = \ln x, &\qquad  T_2(x) = x, \qquad \qquad   S(x) = 0 \\
                    Q_1(p,a) = (p-1), &\qquad Q_2(p,a) = -a, \qquad D(p,a) = \ln\left(\dfrac{a^p}{\Gamma(p)}\right) 
                \end{align*}
                Tenemos la tercera condición.
            \end{enumerate}
            Sea $(X_1, \ldots, X_n)$ una m.a.s. de variables aleatorias idénticamente distribuidas a $X\rightsquigarrow \Gamma(p,a)$ con $p,a\in \mathbb{R}^+$, el Teorema visto en teoría para las familias de distribuciones exponenciales multiparamétricas nos dice que el estadístico:
            \begin{equation*}
                T = T(X_1, \ldots, X_n) = \left(\sum_{i=1}^{n}T_1(X_i), \sum_{i=1}^{n}T_2(X_i)\right) = \left(\sum_{i=1}^{n}\ln(X_i), \sum_{i=1}^{n}X_i\right)
            \end{equation*}
            es suficiente para $(p,a)$. Para ver que $T$ es también completo, hemos de ver que $Im Q$ contiene un abierto de $\mathbb{R}^2$, donde $Q:{(\mathbb{R}^+)}^{2}\to \mathbb{R}^2$, con $Q=(Q_1,Q_2)$. Para ello:
            \begin{align*}
                Im Q = \left\{(p-1,-a) : (p,a)\in {(\mathbb{R}^+)}^{2}\right\} &= \{(x-1,y) : x\in \mathbb{R}^+, y\in \mathbb{R}^-\} \\ &= \left]-1,+\infty\right[\times \mathbb{R}^-
            \end{align*}
            Como claramente $\left]-1,+\infty\right[\times\mathbb{R}^-$ contiene un abierto de $\mathbb{R}^2$, tenemos que $T$ es completo.

            Observemos que también podríamos haber tomado:
            \begin{equation*}
                T(X_1, \ldots, X_n) = \left(\prod_{i=1}^n X_i, \sum_{i=1}^n X_i\right)
            \end{equation*}
        \item $\{\beta(p,q) : p,q>0\}$

            Comprobamos las condiciones:
            \begin{enumerate}[label=\arabic*.]
                \item El espacio paramétrico es $\mathbb{R}^+\times \mathbb{R}^+\subseteq \mathbb{R}^2$.
                \item El espacio muestral es $[0,1]$, que no depende de $p$ ni de $q$.
                \item Para la tercera condición:
                    \begin{align*}
                        f_{(p,q)}(x) &= \dfrac{1}{\beta(p,q)}x^{p-1}{(1-x)}^{q-1} = exp\left[\ln\left(\dfrac{1}{\beta(p,q)}x^{p-1}{(1-x)}^{q-1} \right)\right] \\
                                     &= exp\left[\ln\left(\dfrac{1}{\beta(p,q)}\right) + (p-1)\ln x + (q-1)\ln(1-x)\right]
                    \end{align*}
                Tomando:
                \begin{align*}
                    T_1(x) = \ln x, &\qquad T_2(x) = \ln(1-x), \qquad S(x) = 0 \\
                    Q_1(p,q) = p-1, &\qquad Q_2(p,q) = q-1, \qquad D(p,q) = \ln\left(\dfrac{1}{\beta(p,q)}\right) 
                \end{align*}
                Tenemos la tercera condición.
            \end{enumerate}
            Sea $(X_1, \ldots, X_n)$ una m.a.s. de variables aleatorias idénticamente distribuidas a $X\rightsquigarrow \beta(p,q)$ con $p,q\in \mathbb{R}^+$, el Teorema visto en teoría para las familias de distribuciones exponenciales multiparamétricas nos dice que el estadístico:
            \begin{equation*}
                T = T(X_1, \ldots, X_n) = \left(\sum_{i=1}^{n}T_1(X_i), \sum_{i=1}^{n}T_2(X_i)\right) = \left(\sum_{i=1}^{n}\ln(X_i), \sum_{i=1}^{n}\ln(1-X_i)\right)
            \end{equation*}
            es suficiente para $(p,q)$. Para ver que $T$ es también completo, hemos de ver que $Im Q$ contiene un abierto de $\mathbb{R}^2$, donde $Q:{(\mathbb{R}^+)}^{2}\to \mathbb{R}^2$, con $Q=(Q_1,Q_2)$. Para ello:
            \begin{equation*}
                Im Q = \{(p-1,q-1):p,q\in \mathbb{R}^+\} = \left]-1,+\infty\right[\times \left]-1,+\infty\right[
            \end{equation*}
            Como claramente este conjunto contiene un abierto de $\mathbb{R}^2$, tenemos que $T$ es completo.
    \end{enumerate}
\end{ejercicio}

\end{document}
