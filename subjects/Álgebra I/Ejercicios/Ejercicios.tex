\documentclass[12pt]{article}

% Idioma y codificación
\usepackage[spanish, es-tabla]{babel}       %es-tabla para que se titule "Tabla"
\usepackage[utf8]{inputenc}

% Márgenes
\usepackage[a4paper,top=3cm,bottom=2.5cm,left=3cm,right=3cm]{geometry}

% Comentarios de bloque
\usepackage{verbatim}

% Paquetes de links
\usepackage[hidelinks]{hyperref}    % Permite enlaces
\usepackage{url}                    % redirecciona a la web

% Más opciones para enumeraciones
\usepackage{enumitem}

% Personalizar la portada
\usepackage{titling}

% Paquetes de tablas
\usepackage{multirow}


%------------------------------------------------------------------------

%Paquetes de figuras
\usepackage{caption}
\usepackage{subcaption} % Figuras al lado de otras
\usepackage{float}      % Poner figuras en el sitio indicado H.


% Paquetes de imágenes
\usepackage{graphicx}       % Paquete para añadir imágenes
\usepackage{transparent}    % Para manejar la opacidad de las figuras

% Paquete para usar colores
\usepackage[dvipsnames]{xcolor}
\usepackage{pagecolor}      % Para cambiar el color de la página

% Habilita tamaños de fuente mayores
\usepackage{fix-cm}

% Para los gráficos
\usepackage{tikz}

% Para poder situar los nodos en los grafos
\usetikzlibrary{positioning}


%------------------------------------------------------------------------

% Paquetes de matemáticas
\usepackage{mathtools, amsfonts, amssymb, mathrsfs}
\usepackage[makeroom]{cancel}     % Simplificar tachando
\usepackage{polynom}    % Divisiones y Ruffini
\usepackage{units} % Para poner fracciones diagonales con \nicefrac

\usepackage{pgfplots}   %Representar funciones
\pgfplotsset{compat=1.18}  % Versión 1.18

\usepackage{tikz-cd}    % Para usar diagramas de composiciones
\usetikzlibrary{calc}   % Para usar cálculo de coordenadas en tikz

%Definición de teoremas, etc.
\usepackage{amsthm}
%\swapnumbers   % Intercambia la posición del texto y de la numeración

\theoremstyle{plain}

\makeatletter
\@ifclassloaded{article}{
  \newtheorem{teo}{Teorema}[section]
}{
  \newtheorem{teo}{Teorema}[chapter]  % Se resetea en cada chapter
}
\makeatother

\newtheorem{coro}{Corolario}[teo]           % Se resetea en cada teorema
\newtheorem{prop}[teo]{Proposición}         % Usa el mismo contador que teorema
\newtheorem{lema}[teo]{Lema}                % Usa el mismo contador que teorema

\theoremstyle{remark}
\newtheorem*{observacion}{Observación}

\theoremstyle{definition}

\makeatletter
\@ifclassloaded{article}{
  \newtheorem{definicion}{Definición} [section]     % Se resetea en cada chapter
}{
  \newtheorem{definicion}{Definición} [chapter]     % Se resetea en cada chapter
}
\makeatother

\newtheorem*{notacion}{Notación}
\newtheorem*{ejemplo}{Ejemplo}
\newtheorem*{ejercicio*}{Ejercicio}             % No numerado
\newtheorem{ejercicio}{Ejercicio} [section]     % Se resetea en cada section


% Modificar el formato de la numeración del teorema "ejercicio"
\renewcommand{\theejercicio}{%
  \ifnum\value{section}=0 % Si no se ha iniciado ninguna sección
    \arabic{ejercicio}% Solo mostrar el número de ejercicio
  \else
    \thesection.\arabic{ejercicio}% Mostrar número de sección y número de ejercicio
  \fi
}


% \renewcommand\qedsymbol{$\blacksquare$}         % Cambiar símbolo QED
%------------------------------------------------------------------------

% Paquetes para encabezados
\usepackage{fancyhdr}
\pagestyle{fancy}
\fancyhf{}

\newcommand{\helv}{ % Modificación tamaño de letra
\fontfamily{}\fontsize{12}{12}\selectfont}
\setlength{\headheight}{15pt} % Amplía el tamaño del índice


%\usepackage{lastpage}   % Referenciar última pag   \pageref{LastPage}
\fancyfoot[C]{\thepage}

%------------------------------------------------------------------------

% Conseguir que no ponga "Capítulo 1". Sino solo "1."
\makeatletter
\@ifclassloaded{book}{
  \renewcommand{\chaptermark}[1]{\markboth{\thechapter.\ #1}{}} % En el encabezado
    
  \renewcommand{\@makechapterhead}[1]{%
  \vspace*{50\p@}%
  {\parindent \z@ \raggedright \normalfont
    \ifnum \c@secnumdepth >\m@ne
      \huge\bfseries \thechapter.\hspace{1em}\ignorespaces
    \fi
    \interlinepenalty\@M
    \Huge \bfseries #1\par\nobreak
    \vskip 40\p@
  }}
}
\makeatother

%------------------------------------------------------------------------
% Paquetes de cógido
\usepackage{minted}
\renewcommand\listingscaption{Código fuente}

\usepackage{fancyvrb}
% Personaliza el tamaño de los números de línea
\renewcommand{\theFancyVerbLine}{\small\arabic{FancyVerbLine}}

% Estilo para C++
\newminted{cpp}{
    frame=lines,
    framesep=2mm,
    baselinestretch=1.2,
    linenos,
    escapeinside=||
}

% para minted
\definecolor{LightGray}{rgb}{0.95,0.95,0.92}
\setminted{
    linenos=true,
    stepnumber=5,
    numberfirstline=true,
    autogobble,
    breaklines=true,
    breakautoindent=true,
    breaksymbolleft=,
    breaksymbolright=,
    breaksymbolindentleft=0pt,
    breaksymbolindentright=0pt,
    breaksymbolsepleft=0pt,
    breaksymbolsepright=0pt,
    fontsize=\footnotesize,
    bgcolor=LightGray,
    numbersep=10pt
}


\usepackage{listings} % Para incluir código desde un archivo

\renewcommand\lstlistingname{Código Fuente}
\renewcommand\lstlistlistingname{Índice de Códigos Fuente}

% Definir colores
\definecolor{vscodepurple}{rgb}{0.5,0,0.5}
\definecolor{vscodeblue}{rgb}{0,0,0.8}
\definecolor{vscodegreen}{rgb}{0,0.5,0}
\definecolor{vscodegray}{rgb}{0.5,0.5,0.5}
\definecolor{vscodebackground}{rgb}{0.97,0.97,0.97}
\definecolor{vscodelightgray}{rgb}{0.9,0.9,0.9}

% Configuración para el estilo de C similar a VSCode
\lstdefinestyle{vscode_C}{
  backgroundcolor=\color{vscodebackground},
  commentstyle=\color{vscodegreen},
  keywordstyle=\color{vscodeblue},
  numberstyle=\tiny\color{vscodegray},
  stringstyle=\color{vscodepurple},
  basicstyle=\scriptsize\ttfamily,
  breakatwhitespace=false,
  breaklines=true,
  captionpos=b,
  keepspaces=true,
  numbers=left,
  numbersep=5pt,
  showspaces=false,
  showstringspaces=false,
  showtabs=false,
  tabsize=2,
  frame=tb,
  framerule=0pt,
  aboveskip=10pt,
  belowskip=10pt,
  xleftmargin=10pt,
  xrightmargin=10pt,
  framexleftmargin=10pt,
  framexrightmargin=10pt,
  framesep=0pt,
  rulecolor=\color{vscodelightgray},
  backgroundcolor=\color{vscodebackground},
}

%------------------------------------------------------------------------

% Comandos definidos
\newcommand{\bb}[1]{\mathbb{#1}}
\newcommand{\cc}[1]{\mathcal{#1}}

% I prefer the slanted \leq
\let\oldleq\leq % save them in case they're every wanted
\let\oldgeq\geq
\renewcommand{\leq}{\leqslant}
\renewcommand{\geq}{\geqslant}

% Si y solo si
\newcommand{\sii}{\iff}

% Letras griegas
\newcommand{\eps}{\epsilon}
\newcommand{\veps}{\varepsilon}
\newcommand{\lm}{\lambda}

\newcommand{\ol}{\overline}
\newcommand{\ul}{\underline}
\newcommand{\wt}{\widetilde}
\newcommand{\wh}{\widehat}

\let\oldvec\vec
\renewcommand{\vec}{\overrightarrow}

% Derivadas parciales
\newcommand{\del}[2]{\frac{\partial #1}{\partial #2}}
\newcommand{\Del}[3]{\frac{\partial^{#1} #2}{\partial #3^{#1}}}
\newcommand{\deld}[2]{\dfrac{\partial #1}{\partial #2}}
\newcommand{\Deld}[3]{\dfrac{\partial^{#1} #2}{\partial #3^{#1}}}


\newcommand{\AstIg}{\stackrel{(\ast)}{=}}
\newcommand{\Hop}{\stackrel{L'H\hat{o}pital}{=}}

\newcommand{\red}[1]{{\color{red}#1}} % Para integrales, destacar los cambios.

% Método de integración
\newcommand{\MetInt}[2]{
    \left[\begin{array}{c}
        #1 \\ #2
    \end{array}\right]
}

% Declarar aplicaciones
% 1. Nombre aplicación
% 2. Dominio
% 3. Codominio
% 4. Variable
% 5. Imagen de la variable
\newcommand{\Func}[5]{
    \begin{equation*}
        \begin{array}{rrll}
            #1:& #2 & \longrightarrow & #3\\
               & #4 & \longmapsto & #5
        \end{array}
    \end{equation*}
}

%------------------------------------------------------------------------

\usepackage{extarrows}
\usepackage{stackrel}
\usetikzlibrary{matrix} % Para divisiones de polinomios.

% En el preámbulo del documento o en tu archivo .sty
\newcounter{ejercicio}[section] % Define el contador de ejercicio y lo reinicia con cada sección

\newcounter{ejercicio}
\newcommand{\resetearcontador}{%
  \setcounter{ejercicio}{0}% Resetea el contador de ejercicios a 0
}

\renewcommand{\theejercicio}{\arabic{ejercicio}}

% COMPILAR CON: pdflatex
\begin{document}

    % 1. Foto de fondo
    % 2. Título
    % 3. Encabezado Izquierdo
    % 4. Color de fondo
    % 5. Coord x del titulo
    % 6. Coord y del titulo
    % 7. Fecha

    
    % 1. Foto de fondo
% 2. Título
% 3. Encabezado Izquierdo
% 4. Color de fondo
% 5. Coord x del titulo
% 6. Coord y del titulo
% 7. Fecha

\newcommand{\portada}[7]{

    \portadaBase{#1}{#2}{#3}{#4}{#5}{#6}{#7}
    \portadaBook{#1}{#2}{#3}{#4}{#5}{#6}{#7}
}

\newcommand{\portadaExamen}[7]{

    \portadaBase{#1}{#2}{#3}{#4}{#5}{#6}{#7}
    \portadaArticle{#1}{#2}{#3}{#4}{#5}{#6}{#7}
}




\newcommand{\portadaBase}[7]{

    % Tiene la portada principal y la licencia Creative Commons
    
    % 1. Foto de fondo
    % 2. Título
    % 3. Encabezado Izquierdo
    % 4. Color de fondo
    % 5. Coord x del titulo
    % 6. Coord y del titulo
    % 7. Fecha
    
    
    \thispagestyle{empty}               % Sin encabezado ni pie de página
    \newgeometry{margin=0cm}        % Márgenes nulos para la primera página
    
    
    % Encabezado
    \fancyhead[L]{\helv #3}
    \fancyhead[R]{\helv \nouppercase{\leftmark}}
    
    
    \pagecolor{#4}        % Color de fondo para la portada
    
    \begin{figure}[p]
        \centering
        \transparent{0.3}           % Opacidad del 30% para la imagen
        
        \includegraphics[width=\paperwidth, keepaspectratio]{assets/#1}
    
        \begin{tikzpicture}[remember picture, overlay]
            \node[anchor=north west, text=white, opacity=1, font=\fontsize{60}{90}\selectfont\bfseries\sffamily, align=left] at (#5, #6) {#2};
            
            \node[anchor=south east, text=white, opacity=1, font=\fontsize{12}{18}\selectfont\sffamily, align=right] at (9.7, 3) {\textbf{\href{https://losdeldgiim.github.io/}{Los Del DGIIM}}};
            
            \node[anchor=south east, text=white, opacity=1, font=\fontsize{12}{15}\selectfont\sffamily, align=right] at (9.7, 1.8) {Doble Grado en Ingeniería Informática y Matemáticas\\Universidad de Granada};
        \end{tikzpicture}
    \end{figure}
    
    
    \restoregeometry        % Restaurar márgenes normales para las páginas subsiguientes
    \pagecolor{white}       % Restaurar el color de página
    
    
    \newpage
    \thispagestyle{empty}               % Sin encabezado ni pie de página
    \begin{tikzpicture}[remember picture, overlay]
        \node[anchor=south west, inner sep=3cm] at (current page.south west) {
            \begin{minipage}{0.5\paperwidth}
                \href{https://creativecommons.org/licenses/by-nc-nd/4.0/}{
                    \includegraphics[height=2cm]{assets/Licencia.png}
                }\vspace{1cm}\\
                Esta obra está bajo una
                \href{https://creativecommons.org/licenses/by-nc-nd/4.0/}{
                    Licencia Creative Commons Atribución-NoComercial-SinDerivadas 4.0 Internacional (CC BY-NC-ND 4.0).
                }\\
    
                Eres libre de compartir y redistribuir el contenido de esta obra en cualquier medio o formato, siempre y cuando des el crédito adecuado a los autores originales y no persigas fines comerciales. 
            \end{minipage}
        };
    \end{tikzpicture}
    
    
    
    % 1. Foto de fondo
    % 2. Título
    % 3. Encabezado Izquierdo
    % 4. Color de fondo
    % 5. Coord x del titulo
    % 6. Coord y del titulo
    % 7. Fecha


}


\newcommand{\portadaBook}[7]{

    % 1. Foto de fondo
    % 2. Título
    % 3. Encabezado Izquierdo
    % 4. Color de fondo
    % 5. Coord x del titulo
    % 6. Coord y del titulo
    % 7. Fecha

    % Personaliza el formato del título
    \pretitle{\begin{center}\bfseries\fontsize{42}{56}\selectfont}
    \posttitle{\par\end{center}\vspace{2em}}
    
    % Personaliza el formato del autor
    \preauthor{\begin{center}\Large}
    \postauthor{\par\end{center}\vfill}
    
    % Personaliza el formato de la fecha
    \predate{\begin{center}\huge}
    \postdate{\par\end{center}\vspace{2em}}
    
    \title{#2}
    \author{\href{https://losdeldgiim.github.io/}{Los Del DGIIM}}
    \date{Granada, #7}
    \maketitle
    
    \tableofcontents
}




\newcommand{\portadaArticle}[7]{

    % 1. Foto de fondo
    % 2. Título
    % 3. Encabezado Izquierdo
    % 4. Color de fondo
    % 5. Coord x del titulo
    % 6. Coord y del titulo
    % 7. Fecha

    % Personaliza el formato del título
    \pretitle{\begin{center}\bfseries\fontsize{42}{56}\selectfont}
    \posttitle{\par\end{center}\vspace{2em}}
    
    % Personaliza el formato del autor
    \preauthor{\begin{center}\Large}
    \postauthor{\par\end{center}\vspace{3em}}
    
    % Personaliza el formato de la fecha
    \predate{\begin{center}\huge}
    \postdate{\par\end{center}\vspace{5em}}
    
    \title{#2}
    \author{\href{https://losdeldgiim.github.io/}{Los Del DGIIM}}
    \date{Granada, #7}
    \thispagestyle{empty}               % Sin encabezado ni pie de página
    \maketitle
    \vfill
}
    \portada{ffccA4.jpg}{Álgebra I}{Álgebra I}{MidnightBlue}{-8}{28}{2023-2024}{José Juan Urrutia Milán\\Arturo Olivares Martos}

    \newpage
    \fancyhead[R]{\helv \nouppercase{\rightmark}}

    \section{Relaciones de ejercicios}
    \section{Conexión por arcos}

\begin{ejercicio}
    Muestra que cualquier esfera de $\mathbb{R}^n$, $n\geq 2$ es arcoconexa con la topología usual.\\

    \noindent
    Es decir, queremos ver que $\bb{S}^n$ es arcoconexa para $n\geq 1$. \newline (notemos que $\bb{S}^0 = \{x\in \mathbb{R} : \|x\| = 1\} = \{-1,1\}$ no es un conjunto arcoconexo).\\

    \noindent
    Para ello, sea $n\geq 2$, sabemos que $\bb{S}^n\setminus\{p\}$ (con $p\in \bb{S}^n$) es homeomorfa a $\mathbb{R}^{n-1}$, que es un conjunto arcoconexo por ser convexo (es una espacio vectorial). Como la arcoconexión es una propiedad topológica, esta se conserva por homeomorfismo, luego $\bb{S}^n\setminus \{p\}$ es un conjunto arcoconexo, $\forall p\in \bb{S}^n$.

    Tomando $N = (0,\ldots,0,1), S =(0,\ldots,0,-1) \in \bb{S}^n$, podemos ver $\bb{S}^n$ como unión de dos conjuntos arcoconexos:
    \begin{equation*}
        \bb{S}^n = (\bb{S}^n\setminus\{N\}) \cup (\bb{S}^n\setminus\{S\})
    \end{equation*}

    no disjuntos:
    \begin{equation*}
        (\bb{S}^n\setminus\{N\}) \cap (\bb{S}^n\setminus\{S\}) = \bb{S}^n\setminus\{N,S\}
    \end{equation*}
    Por lo que $\bb{S}^n$ es un conjunto arcoconexo, $\forall n\geq 2$.
\end{ejercicio}

\begin{ejercicio}
    Demuestra que si $\{A_i\}_{i \in I}$ es una familia de arcoconexos de $X$ tales que todos intersecan a uno de ellos, es decir,
    \begin{equation*}
        A_i\cap A_{i_0} \neq \emptyset, \qquad \forall i \in I,
    \end{equation*}
    entonces $\bigcup\limits_{i \in I}A_i$ es arcoconexo.\\

    \noindent
    Sean $x,y\in \bigcup\limits_{i \in I}A_i$, entonces existen $i,j\in I$ de forma que $x\in A_i$ y $y\in A_j$. Como $A_i \cap A_{i_0}, A_j\cap A_{i_0}\neq \emptyset $, podemos tomar $a\in A_i\cap A_{i_0}$ y $b\in A_j\cap A_{i_0}$.
    \begin{itemize}
        \item $A_i$ es un conjunto arcoconexo con $x,a\in A_i$, por lo que existe un camino, $\alpha$, que une $x$ con $a$.
        \item $A_j$ también es un conjunto arcoconexo con $y,b\in A_j$, por lo que existe un camino, $\beta$, que une $y$ con $b$.
        \item Además, $A_{i_0}$ es un conjunto arcoconexo con $a,b\in A_{i_0}$, por lo que existe un tercer camino, $\gamma$, que une $a$ con $b$.
    \end{itemize}
    De esta forma, podemos tomar:
    \begin{equation*}
        \sigma = \alpha \ast \left(\gamma \ast \tilde{\beta}\right)
    \end{equation*}
    Que es un camino que une $x$ con $y$. Como $x$ e $y$ eran arbitrarios, podemos unir cualesquiera dos puntos de $\bigcup\limits_{i \in I}A_i$, por lo que dicho conjunto es arcoconexo.

    \begin{figure}[H]
        \centering
        \begin{tikzpicture}[scale=1.2]
        % Dibujar elipses (conjuntos)
        \draw[thick] (0,1.5) ellipse (2 and 0.8); % elipse superior horizontal
        \draw[thick] (0,-1.5) ellipse (2 and 0.8); % elipse inferior horizontal
        \draw[thick] (1.2,0) ellipse (0.8 and 2);  % elipse vertical

        % Puntos
        \node[circle,fill=black,inner sep=1pt,label=left:$x$] (x) at (-1,1.5) {};
        \node[circle,fill=black,inner sep=1pt,label=right:$a$] (a) at (1,1.5) {};
        \node[circle,fill=black,inner sep=1pt,label=right:$b$] (b) at (1,-1.5) {};
        \node[circle,fill=black,inner sep=1pt,label=left:$y$] (y) at (-1,-1.5) {};

        % Arcos dirigidos
        \draw[-stealth,thick,red,bend left=50] (x) to (a);
        \draw[-stealth,thick,red,bend left=10] (a) to (b);
        \draw[-stealth,thick,red,bend right=15] (y) to (b);
        \end{tikzpicture}
        \caption{Forma de unir dos puntos cualesquiera.}
    \end{figure}
\end{ejercicio}

\begin{ejercicio}
    Sea $X$ un conjunto, $x_0\in X$, y consideramos la topología (del punto incluido) dada por
    \begin{equation*}
        T = \{U\subset X : x_0 \in U\} \cup \{\emptyset \}
    \end{equation*}
    ¿Es $(X,T)$ arcoconexo?\\

    \noindent
    Sí: sea $x\in X$, veamos que la aplicación $\alpha:[0,1]\to X$ dada por
    \begin{equation*}
        \alpha(t) = \left\{\begin{array}{ll}
                x & \text{si } t\in [0,\nicefrac{1}{2}] \\
                x_0 & \text{si } t\in \left]\nicefrac{1}{2},1\right]
        \end{array}\right. \qquad \forall t\in [0,1]
    \end{equation*}
    es continua. Sea $U\in T$:
    \begin{itemize}
        \item Si $U = \emptyset $, entonces $\alpha^{-1}(U) = \emptyset \in \cc{T}_u\big|_{[0,1]}$.
        \item Si $x_0\in U$ y $x\notin U$, entonces $\alpha^{-1}(U) = \left]\nicefrac{1}{2},1\right]\in \cc{T}_u\big|_{[0,1]}$.
        \item Si $x_0,x\in U$, entonces $\alpha^{-1}(U) = [0,1] \in \cc{T}_u\big|_{[0,1]}$.
    \end{itemize}
    Como la preimagen de cualquier conjunto abierto es abierta, tenemos que $\alpha$ es continua, luego es un arco que une $x$ con $x_0$.\\

    \noindent
    Ahora, si $x,y\in X$, tenemos que existen $\alpha,\beta:[0,1]\to X$ de forma que $\alpha$ une $x$ con $x_0$ y $\beta$ une $y$ con $x_0$; por lo que $\alpha\ast\tilde{\beta}$ es un arco que une $x$ con $y$. Como $x$ e $y$ eran arbitrarios, concluimos que $X$ es arcoconexo.
\end{ejercicio}

\begin{ejercicio}
   Demustra que en $\mathbb{R}^n$  con la topología usual, todo abierto conexo es arcoconexo. ¿Es cierto que todo cerrado conexo de $\mathbb{R}^n$ es arcoconexo?\\

   \noindent
   En teoría vimos que:
   \begin{equation*}
       \text{Un conjunto es arcoconexo} \Longleftrightarrow \left\{\begin{array}{l}
           \text{Es conexo} \\
           \text{Todo punto admite un entorno arcoconexo}
       \end{array}\right.
   \end{equation*}
   Sea $U$ un abierto conexo de $(\mathbb{R}^n, \cc{T}_u)$, falta ver que todo punto suyo admite un entorno arcoconexo en la topología inducida en $U$ para ver que $U$ es arcoconexo. Para ello, sea $x\in U$, como $U$ es abierto existe $r\in \mathbb{R}^+$ de forma que $B(x,r)\subset U$. $B(x,r)$ es un conjunto arcoconexo por ser convexo, luego es un entorno arcoconexo de $x$ en $U$. Como $x$ era un punto arbitrario de $U$, todo punto suyo admite un entorno arcoconexo, y como $U$ era conexo, tenemos que $U$ es arcoconexo.\\

   \noindent
   Ahora, no es cierto que todo cerrado conexo de $\mathbb{R}^n$ es arcoconexo, ya que si consideramos $f:\mathbb{R}^+\to \mathbb{R}$ dada por:
   \begin{equation*}
       f(x) = \sen\left(\dfrac{1}{x}\right) \qquad \forall x\in \mathbb{R}^+
   \end{equation*}
   Tenemos que
   \begin{equation*}
       C = \overline{Gr(f)} = \overline{\{(x,f(x)) : x\in \mathbb{R}^+\}} = Gr(f) \cup (\{0\}\times [-1,1])
   \end{equation*}
   es un conjunto cerrado y conexo (se vio en Topología I) pero que no es arcoconexo, puede probarse por un razonamiento similar a un ejemplo visto en teoría.

   \begin{figure}[H]
       \centering
        \begin{tikzpicture}
          \begin{axis}[
            axis lines=middle,
            xlabel={$x$}, ylabel={$y$},
            xmin=0, xmax=1,
            ymin=-1.2, ymax=1.2,
            samples=200 % menos puntos = compila más rápido
          ]
            \addplot[blue, thick, domain=0.0005:1] {sin(deg(1/x))};
            \addplot[blue, ultra thick] coordinates {(0,-1) (0,1)};
          \end{axis}
        \end{tikzpicture}
        \caption{Dibujo de la adherencia de la gráfica de $f(x)$.}
   \end{figure}
\end{ejercicio}

\begin{ejercicio}
    Prueba que la componente arcoconexa de un punto $x_0$ está contenida en la componente conexa de $x_0$.\\

    \noindent
    Sea $(X,T)$ un espacio topológico, $x_0\in X$ y $C$ la componente arcoconexa de $x_0$ en $X$, en particular tenemos que $C$ es un conjunto arcoconexo, luego es conexo, por lo que está contenida en la componente conexa de $x$, al ser esta el mayor conjunto conexo que contiene a $x$.
\end{ejercicio}

\begin{ejercicio}
    En $\mathbb{R}$ con la topología de Sorgenfrey, esto es, la topología que tiene como base
    \begin{equation*}
        \cc{B}_S = \{[a,b) \subset \mathbb{R} : a<b\},
    \end{equation*}
    determina sus componentes arcoconexas.\\

    \noindent
    En Topología I vimos que las componentes conexas de la topología de Sorgenfrey eran los conjuntos de puntos unitarios $\{x\}$, ya que si tenemos un conjunto $A\subset \mathbb{R}$ con al menos dos puntos distintos $x$ e $y$ (suponemos $x<y$), entonces en la topología inducida en $A$ podemos considerar los abiertos:
    \begin{equation*}
        U = [-\infty,y)\cap A, \qquad V = [y,+\infty)\cap A
    \end{equation*}
    de forma que $U,V\neq \emptyset $, $U\cup V = A$ y $U\cap V = \emptyset $, por lo que $A$ (cualquier conjunto con al menos dos puntos distintos) es disconexo, luego las componentes conexas han de ser los conjuntos unitarios, ya que los conjuntos unitarios son conexos en cualquier topología.

    Como las componentes arcoconexas se encuentran contenidas en las componentes conexas, no queda más salida que las componentes arcoconexas de la topología de Sorgenfrey sean los conjuntos unitarios.
\end{ejercicio}

\begin{ejercicio}
    Sea $f:X\to Y$ un homeomorfismo entre espacios topológicos. Demuestra que $A\subset X$ es una componente arcoconexa de $X$ si y solo si $f(X)$ es una componente arcoconexa de $Y$. Deduce que el número de componentes arcoconexas es invariante por homeomorfismos.\\

    \noindent
    Sea $A\subset X$ una componente arcoconexa de $X$, veamos que $f(A)$ es una componente arcoconexa de $Y$. Para ello, por reducción al absurdo, si $f(A)$ no fuera una componente arcoconexa de $Y$ podría ser por dos razones:
    \begin{itemize}
        \item $f(A)$ no es un conjunto arcoconexo, algo que llevaría a una contradicción, ya que se vio que la imagen por una función continua de un conjunto arcoconexo era arcoconexa.
        \item Porque existe $B\subset Y$ un conjunto arcoconexo distinto de $f(A)$ de forma que $f(A)\subset B\subset Y$. En dicho caso, si aplicamos $f^{-1}$ en la anterior inclusión tenemos que:
            \begin{equation*}
                f^{-1}(f(A)) = A \subset f^{-1}(B) \subset X
            \end{equation*}
            Por lo que tenemos $f^{-1}(B)$, un conjunto arcoconexo\footnote{por ser imagen por una función continua de un conjunto arcoconexo.} distinto de $A$ que contiene a $A$, luego $A$ no era una componentes arcoconexa de $X$, contradicción.
    \end{itemize}
    En definitiva, si $A\subset X$ es una componente arcoconexa entonces $f(A)$ también lo es de $Y$. Ahora, si $f(A)$ es una componente arcoconexa de $Y$, basta aplicar que $f^{-1}$ también es un homeomorfismo para concluir que $f^{-1}(f(A)) = A$ es una componente arcoconexa de $X$.\\

    \noindent
    Sea $Z$ un espacio topológico, notaremos en este ejercicio:
    \begin{equation*}
        \Gamma_Z = \{U\subset Z : U \text{\ es una componente arcoconexa de\ } Z\}
    \end{equation*}
    Recuperando el homeomorfismo $f:X\to Y$, definimos
    \Func{\Phi}{\Gamma_X}{\Gamma_Y}{U}{f(U)}
    \begin{itemize}
        \item $\Phi$ está bien definida (es decir, $f(U)\in \Gamma_Y$ para $U\in \Gamma_X$), ya que hemos visto que la imagen de una componente arcoconexa de $X$ es una componente arcoconexa de $Y$.
        \item $\Phi$ es inyectiva, ya que si $U,V\in \Gamma_X$ con $f(U)=f(V)$, entonces por ser $f$ inyectiva tenemos que $U=V$.
        \item $\Phi$ es sobreyectiva, ya que si $W\in \Gamma_Y$, entonces $f^{-1}(W)\in \Gamma_X$, con:
            \begin{equation*}
                \Phi(f^{-1}(W)) = f(f^{-1}(W)) = W
            \end{equation*}
    \end{itemize}
    Por ser $\Phi$ biyectiva concluimos que $|\Gamma_X| = |\Gamma_Y|$; es decir, el número de componentes arcoconexas es invariante por homeomorfismos.
\end{ejercicio}

\begin{ejercicio}
    En $X=\mathbb{R}\times \{0,1\}$ se considera la topología que tiene por base
    \begin{equation*}
        \cc{B} = \{\left]a,b\right[\times \{0,1\} : a<b\}.
    \end{equation*}
    Demuestra que $X$ es arcoconexo. ¿Es $X$ homeomorfo a $\mathbb{R}$ con la topología usual?\\

    \noindent
    Sean $\alpha=(x,a),\beta=(y,b)\in X$, vamos a tratar de crear un arco que una $\alpha$ con $\beta$:
    \begin{itemize}
        \item Si $a=b$, entonces $\gamma:[0,1]\to X$ dada por:
            \begin{equation*}
                \gamma(t) = ((1-t)x + ty, a) \qquad \forall t\in [0,1]
            \end{equation*}
            Es una aplicación continua, ya que si tomamos $B = \left]a,b\right[\times \{0,1\}\in \cc{B}$, tenemos:
            \begin{equation*}
                \gamma^{-1}(B) = \gamma^{-1}(\left]a,b\right[\times \{0\}) \text{\ abierto de\ } [0,1]
            \end{equation*}
            Ya que el conjunto $\left]a,b\right[\times \{0\}$ es un abierto para la topología usual y $\alpha$ es una aplicación continua para la topología usual.
        \item Si $\alpha = (0,0)$ y $\beta = (0,1)$, entonces si tomamos $\gamma:[0,1]\to X$ dada por:
            \begin{equation*}
                \gamma(t) = \left\{\begin{array}{ll}
                        \alpha & \text{si\ } 0 \leq t \leq \nicefrac{1}{2}\\
                        \beta & \text{si\ } \nicefrac{1}{2}<t \leq 1
                \end{array}\right.
            \end{equation*}
            tenemos que $\gamma$ es continua, ya que si $B = \left]a,b\right[\times \{0,1\}\in \cc{B}$, tenemos que:
            \begin{equation*}
                \gamma^{-1}(B) = \left\{\begin{array}{cl}
                        \emptyset & \text{si\ } 0 \notin \left]a,b\right[ \\
                        \left[0,1\right] & \text{si\ } 0 \in \left]a,b\right[
                \end{array}\right.
            \end{equation*}
        \item Una vez discutidos dichos casos, suponemos ahora que $\alpha=(x,0)$ y $\beta = (y,1)$ (en caso contrario, sustituimos los papeles de $\alpha$ y $\beta$), en cuyo caso:
            \begin{itemize}
                \item Sabemos de la existencia de un arco $\gamma$ que une $\alpha$ con $(0,0)$.
                \item Sabemos de la existencia de un arco $\tau$ que une $(0,0)$ con $(0,1)$.
                \item Sabemos de la existencia de un arco $\pi$ que une $\beta$ con $(0,1)$.
            \end{itemize}
            Si consideramos el arco $\gamma \ast (\tau \ast \tilde{\pi})$ obtenemos un arco que une $\alpha$ con $\beta$.
    \end{itemize} 
    Por tanto, $X$ es arcoconexo, ya que somos capaces de unir cualesquiera dos puntos distintos de $X$ por un arco.\\

    \noindent
    Ahora, para responder a la pregunta de si $(\mathbb{R},\cc{T}_u)$ es homeomorfo a $X$, la respuesta es que no, y tenemos dos formas de justificar la respuesta:
    \begin{description}
        \item [Opción 1.] Sabemos que $(\mathbb{R},\cc{T}_u)$ es $T2$ por ser un espacio topológico metrizable, mientras que podemos probar que $X$ no es $T2$, ya que no existen ningún par de abiertos disjuntos uno conteniendo a $(0,0)$ y otro conteniendo a $(0,1)$, puesto que si $U$ es un abierto de $X$ que contiene a $(0,0)$, entonces como $\cc{B}$ es una base, existen $a,b\in \mathbb{R}$ de forma que:
            \begin{equation*}
                (0,0) \in \left]a,b\right[\times \{0,1\} \subset U
            \end{equation*}
            Sin embargo, tendríamos entonces que $(0,1)\in \left]a,b\right[\times \{0,1\}$, de donde $(0,1)\in U$, por lo que $X$ no es T2 y como ser T2 es una propiedad topológica, dichos espacios no pueden ser homeomorfos.
        \item [Opción 2.] Otra forma sería suponer que son homeomorfos, con lo que existe un homeomorfismo $f:\mathbb{R}\to X$. Sea $p\in \mathbb{R}$, resulta entonces que $\mathbb{R}\setminus \{p\}$ es homeomorfo a $X\setminus \{(p,0)\}$, pero:
            \begin{itemize}
                \item $\mathbb{R}\setminus\{p\}$ no es arcoconexo.
                \item $X\setminus \{(p,0)\}$ sí es arcoconexo, ya que podemos hacer que cualquier curva ``salte'' a $(p,1)$ sin perder su continuidad, con lo que podemos seguir conectando dos puntos cualesquiera.
            \end{itemize}
    \end{description}
\end{ejercicio}

\begin{ejercicio}
    En $\mathbb{R}^3$ con la topología usual, calcula las componentes arcoconexas de
    \begin{equation*}
        X = \{x,y,z) \in \mathbb{R}^3 : xyz = 1\}
    \end{equation*}

    \noindent
    Notemos que como $xyz = 1$, ninguno de ellos puede ser igual a 0, por lo que:
    \begin{equation*}
        X = \left\{(x,y,z)\in \mathbb{R}^3 : z=\dfrac{1}{xy} ,\quad  xy\neq 0\right\}
    \end{equation*}
    Si tomamos:
    \begin{equation*}
        \Gamma = \left\{(x,y) \in \mathbb{R}^2 : xy \neq 0\right\} = \mathbb{R}^\ast \times \mathbb{R}^\ast
    \end{equation*}
    y definimos $f:\Gamma\to \mathbb{R}$ dada por:
    \begin{equation*}
        f(x,y) = \dfrac{1}{xy} \qquad \forall (x,y)\in \Gamma
    \end{equation*}
    Tenemos que $X = Gr(f)$. Por tanto, definiendo $h:\Gamma\to X$ por:
    \begin{equation*}
        h(x,y) = (x,y,f(x,y)) \qquad \forall (x,y)\in \Gamma
    \end{equation*}
    Obtenemos (como vimos en Topología I) un homeomorfismo entre $\Gamma$ y $X$. Como $\Gamma$ tiene 4 componentes arcoconexas:
    \begin{equation*}
        \mathbb{R}^+\times \mathbb{R}^+, \qquad \mathbb{R}^+\times \mathbb{R}^-, \qquad \mathbb{R}^-\times\mathbb{R}^-, \qquad \mathbb{R}^-\times\mathbb{R}^-
    \end{equation*}
    y las componentes arcoconexas se convervan por homeomorfismos tal y como acabamos de ver en el ejercicio 7, tenemos que:
    \begin{equation*}
        h(\mathbb{R}^+\times \mathbb{R}^+), \qquad h(\mathbb{R}^+\times \mathbb{R}^-), \qquad h(\mathbb{R}^-\times\mathbb{R}^-), \qquad h(\mathbb{R}^-\times\mathbb{R}^-)
    \end{equation*}
    son las componentes arcoconexas de $X$.
\end{ejercicio}

\begin{ejercicio} % // TODO: HACER
    En $\mathbb{R}^2$ con la topología usual consideremos las rectas horizontales $A_n = \mathbb{R}\times \{\nicefrac{1}{n}\}$, $B_n = \mathbb{R}\times \{\nicefrac{-1}{n}\}$ y el eje de ordenadas menos el origen, esto es, $C=\{0\}\times (\mathbb{R}\setminus \{0\})$. Calcula las componentes conexas y arcoconexas de
    \begin{equation*}
        X = \left(\bigcup_{n\in \mathbb{N}}A_n\right) \cup \left(\bigcup_{n\in \mathbb{N}}B_n\right) \cup C \cup \{(1,0)\}.
    \end{equation*}
\end{ejercicio}


    \section{Topología del plano complejo}

\begin{ejercicio}
    Estudiar la continuidad de la función argumento principal; esta es, $\arg : \mathbb{C}^\ast \to \mathbb{R}$.\\

    Por el Ejercicio~\ref{ej:1.6}, sabemos que:
    \begin{equation*}
        \arg z = 2\arctan\left(\frac{\Im z}{\Re z + |z|}\right) \quad \forall z \in \mathbb{C}^\ast\setminus \bb{R}^-
    \end{equation*}

    Consideramos $\Omega = \mathbb{C}^\ast\setminus \bb{R}^-$. Como la función $Id$ es continua, tenemos que $\Re z,\Im z, |z|$ son continuas en $\cc{C}$. Además, como el denominador tan solo se anula en $\bb{R}^-_0$, el argumento de la arcotangente restringido a $\Omega$ es una función continua. Por ser la arcotangente continua en $\bb{R}$ y serlo el producto de funciones continuas, concluimos que $\arg_{\big| \Omega}$ es continua. Como $\Omega$ es abierto, por el carácter local de la continuidad, $\arg$ es continua en $\Omega=\mathbb{C}^\ast\setminus \bb{R}^-$.\\

    Tan falta por estudiar la continuidad en $\bb{R}^-$. Para ello, sea $z\in \bb{R}^-$, del que sabemos que $\arg z = \pi$. Sea la sucesión $\{\theta_n\}$ que recorre los ángulos desde $0$ en sentido horario hasta $-\pi$, límite de la sucesión:
    \begin{equation*}
        \{\theta_n\} = \left\{-\pi\left(1+\frac{1}{n}\right)\right\}\to -\pi
    \end{equation*}

    A partir de dicha sucesión, definimos $\{z_n\}$ como los números complejos de módulo $|z|$ y argumento $\theta_n$; que recorren los puntos de la circunferencia unitaria desde el eje positivo en sentido horario hasta el eje negativo.
    \begin{equation*}
        \{z_n\} = \left\{|z|\left(\cos\left(\theta_n\right)+i\sin\left(\theta_n\right)\right)\right\}\to |z|\left(\cos(-\pi)+i\sin(-\pi)\right) = -|z| = z
    \end{equation*}

    Por último, tenemos que:
    \begin{equation*}
        \{\arg z_n\} = \{\theta_n\} \to -\pi\neq \pi = \arg z
    \end{equation*}

    Por tanto, hemos encontrado una sucesión $\{z_n\}$ con $z_n\in \bb{C}^*~\forall n\in \mathbb{N}$, con $\{z_n\}\to z$ pero $\{\arg z_n\}\nrightarrow \arg z$. Por tanto, $\arg$ no es continua en $z$. Como $z$ era arbitrario, concluimos que $\arg$ no es continua en $\bb{R}^-$.\\

    Por tanto, concluimos que $\arg$ es continua en $\mathbb{C}^\ast\setminus \bb{R}^-$, pero no lo es en $\bb{R}^-$.
\end{ejercicio}

\begin{ejercicio}\label{ej:2.2}
    Dado $\theta \in \mathbb{R}$, se considera el conjunto $S_\theta = \{z \in \mathbb{C}^\ast \mid \theta \notin \Arg z\}$. Probar que existe una función $\varphi \in \cc{C}(S_\theta)$ que verifica $\varphi(z) \in \Arg (z)$ para todo $z \in S_\theta$.\\

    La elección del argumento principal de un número complejo realizada provoca que haya una discontinuidad en $\bb{R}^-=S_{\pi}$. Este ejercicio nos pide encontrar una función que, dado un argumento $\theta$, sea continua en $\bb{C}^*$ excepto en los puntos $z$ para los cuales $\theta\in \Arg z$.\\

    Dado $z\in S_{\theta}$, como $\arg$ es continua en $\mathbb{C}^\ast\setminus \bb{R}^-$, en primer lugar definiremos una función $g_{\theta}:S_{\theta}\to C^{\ast}\setminus \bb{R}^-$ que nos lleve $z$ a un punto $w\notin \bb{R}^-$ (esto lo haremos girando $z$ un ángulo de $\pi-\theta$); para poder aplicar luego $\arg$ y modificar el valor de forma que $\varphi(z)\in \Arg z$ (esto lo haremos restando $\pi-\theta$). Vamos a ello.\\

    Definimos en primer lugar $w_\theta=\cos(\pi-\theta) + i\sin(\pi-\theta)\in \mathbb{C}$, de forma que $|w_\theta|=1$ y $\pi-\theta\in \Arg w_\theta$. Definimos $g_{\theta}$ como:
    \Func{g_{\theta}}{S_{\theta}}{\mathbb{C}^{\ast}\setminus \bb{R}^-}{z}{zw_{\theta}}

    En primer lugar, como $g_{\theta}$ es polinómica, tenemos que $g_\theta\in \cc{C}(S_{\theta})$. Además, dado $z\in S_{\theta}$, tenemos que:
    \begin{align*}
        \Arg g_{\theta}(z) &= \Arg(zw_\theta) = \Arg z + \Arg w_{\theta} = (\arg z + \pi-\theta)+2\pi\bb{Z}
    \end{align*}

    Veamos que $g_{\theta}(z)\notin \bb{R}^-$. Supongamos que $g_{\theta}(z)\in \bb{R}^-$. Entonces, $\exists k\in \bb{Z}$ tal que $\arg z + \pi-\theta = 2k\pi$. Por tanto, $\arg z = 2k\pi - \pi + \theta = (2k-1)\pi + \theta$. Por tanto, $\theta\in \Arg z$, lo cual es una contradicción. Por tanto, $g_{\theta}(z)\notin \bb{R}^-$.\\

    A continuación, definimos $\varphi$ como sigue:
    \Func{\varphi}{S_{\theta}}{\bb{R}}{z}{\arg (g_{\theta}(z)) - (\pi-\theta)}

    De esta forma, tenemos que $\varphi$ es continua en $S_{\theta}$, puesto que $\arg$ es continua en $\mathbb{C}^{\ast}\setminus \bb{R}^-$ y $g_{\theta}$ es continua en $S_{\theta}$. Además, dado $z\in S_{\theta}$, tenemos que:
    \begin{equation*}
        \varphi(z) \in \Arg g_{\theta}(z) - \Arg w_{\theta} = \Arg g_{\theta}(z) + \Arg\frac{1}{w_{\theta}} = \Arg\left(\frac{g_{\theta}(z)}{w_{\theta}}\right) = \Arg\left(\frac{zw_{\theta}}{w_{\theta}}\right) = \Arg z
    \end{equation*}
\end{ejercicio}

\begin{ejercicio}
    Probar que no existe ninguna función $\varphi \in \cc{C}(\mathbb{C}^\ast)$ de forma que $\varphi(z) \in \Arg z$ para todo $z \in \mathbb{C}^\ast$, y que el mismo resultado es cierto, sustituyendo $\mathbb{C}^\ast$ por $\bb{T} = \{z \in \mathbb{C} \mid |z| = 1\}$.\\

    Por reducción al absurdo, supongamos que existe una función $\varphi\in \cc{C}(\mathbb{C}^{\ast})$ tal que $\varphi(z)\in \Arg z~\forall z\in \mathbb{C}^{\ast}$. Definimos la siguiente función auxiliar:
    \Func{f}{\mathbb{C}^{\ast}}{\bb{R}}{z}{\varphi(z)-\varphi(-z)}

    Por ser $\varphi$ continua, $f$ es continua. Además, dado $z\in \mathbb{C}^{\ast}$, tenemos que:
    \begin{align*}
        f(z) &= \varphi(z)-\varphi(-z)\\
        f(-z) &= \varphi(-z)-\varphi(z) = -(\varphi(z)-\varphi(-z)) = -f(z)
    \end{align*}

    Por tanto, fijado $w\in \mathbb{C}^{\ast}$, hay dos opciones:
    \begin{itemize}
        \item Si $f(w)=0$, entonces sea $z_0=w$, y se tiene que $f(z_0)=0$.
        \item Si $f(w)\neq 0$, entonces $f(w)f(-w)<0$. Como $\bb{C}^{\ast}$ es conexo, por el Teorema del Valor Intermedio $\exists z_0\in \mathbb{C}^{\ast}$ tal que $f(z_0)=0$.
    \end{itemize}
    En cualquier caso, $\exists z_0\in \mathbb{C}^{\ast}$ tal que $f(z_0)=0$. Por tanto, $\varphi(z_0)=\varphi(-z_0)$. Esto implica que $\Arg z_0 = \Arg (-z_0)$, lo cual es una contradicción ya que:
    \begin{align*}
        \Arg -z_0 &= (\arg z_0 + \pi) + 2\pi\bb{Z}
    \end{align*}

    Por tanto, no puede existir una función $\varphi\in \cc{C}(\mathbb{C}^{\ast})$ tal que $\varphi(z)\in \Arg z~\forall z\in \mathbb{C}^{\ast}$.\\

    Por otro lado, consideramos el caso para $\bb{T}$. Hay diversas formas de probarlo:
    \begin{itemize}
        \item De forma análoga, haciendo uso ahora de que $\bb{T}$ es conexo.
        \item Aplicando de forma directa el Teorema de Borsuk-Ulam a $\varphi$ (esto es lo que en realidad hacemos en la opción anterior).
        \item Haciendo uso de lo anteriormente demostrado.
    \end{itemize}

    Desarrollaremos la tercera opción, por ser aquella que difiere de lo anterior. De nuevo, supongamos por reducción al absurdo que existe una función $\varphi\in \cc{C}(\bb{T})$ tal que $\varphi(z)\in \Arg z~\forall z\in \bb{T}$. Definimos la siguiente función auxiliar:
    \Func{f}{\bb{C}^{\ast}}{\bb{R}}{z}{\varphi\left(\dfrac{z}{|z|}\right)}

    Tenemos que $f$ es continua, y verifica que:
    \begin{equation*}
        f(z) = \varphi\left(\frac{z}{|z|}\right) \in \Arg\left(\frac{z}{|z|}\right) = \Arg z - \Arg (|z|) = \Arg z - 2\pi\bb{Z} = \Arg z
    \end{equation*}
    No obstante, hemos demostrado que no puede existir una función $f\in \cc{C}(\mathbb{C}^{\ast})$ tal que $f(z)\in \Arg z~\forall z\in \mathbb{C}^{\ast}$. Por tanto, hemos llegado a una contradicción, y concluimos que no puede existir una función $\varphi\in \cc{C}(\bb{T})$ tal que $\varphi(z)\in \Arg z~\forall z\in \bb{T}$.
\end{ejercicio}

\begin{ejercicio}
    Probar que la función $\Arg : \mathbb{C}^\ast \to \mathbb{R}/2\pi\mathbb{Z}$ es continua, considerando en $\mathbb{R}/2\pi\mathbb{Z}$ la topología cociente. Más concretamente, se trata de probar que, si $\{z_n\}$ es una sucesión de números complejos no nulos, tal que $\{z_n\} \to z \in \mathbb{C}^\ast$ y $\theta \in \Arg z$, se puede elegir $\theta_n \in \Arg z_n$ para todo $n \in \mathbb{N}$, de forma que $\{\theta_n\} \to \theta$.

    \begin{description}
        \item[Usando sucesiones:] Usaremos la caracterización que en el mismo enunciado describen. Dada una sucesión $\{z_n\}$ de números complejos no nulos, tal que $\{z_n\}\to z\in \mathbb{C}^{\ast}$ y $\theta\in \Arg z$, definimos $\theta_n$ como sigue:
        \begin{itemize}
            \item \ul{Si $z\notin \bb{R}^-$}:
            
            Como $\arg z\in \Arg z$, tenemos que $\exists k\in \bb{Z}$ tal que $\theta=2k\pi+\arg z$. Por tanto, definimos $\theta_n$ como:
            \begin{equation*}
                \theta_n = \arg z_n + 2k\pi \in \Arg z_n\qquad \forall n\in \mathbb{N}
            \end{equation*}

            Además, tenemos que:
            \begin{equation*}
                \left\{\theta_n\right\} = \left\{\arg z_n + 2k\pi\right\}\to \arg z + 2k\pi = \theta
            \end{equation*}
            donde hemos usado que, al ser $\arg$ continua en $z\in \mathbb{C}^{\ast}\setminus \bb{R}^-$, como se tiene que $\{z_n\}\to z$, entonces $\{\arg z_n\}\to \arg z$.
            
            \item \ul{Si $z\in \bb{R}^-$}:
            
            Por el Ejercicio~\ref{ej:2.2}, $\exists \varphi\in \cc{C}(S_{0})$ tal que $\varphi(w)\in \Arg w~\forall w\in S_{0}$. En particular, $\varphi(z)\in \Arg z$, por lo que $\exists k\in \bb{Z}$ tal que $\varphi(z)=\theta+2k\pi$.\\

            Como $\{z_n\}\to z\in S_0=S_0^\circ$ abierto, $\exists N\in \mathbb{N}$ tal que $\forall n\geq N$ se tiene que $z_n\in S_{0}$. Por tanto, definimos $\theta_n$ como:
            \begin{equation*}
                \begin{cases}
                    \theta_n = \arg z_n & \text{si } n<N\\
                    \theta_n = \varphi(z_n)-2k\pi & \text{si } n\geq N
                \end{cases}
            \end{equation*}

            De esta forma, tenemos que $\theta_n\in \Arg z_n~\forall n\in \mathbb{N}$, y además:
            \begin{equation*}
                \left\{\theta_n\right\} \to \varphi(z)-2k\pi = \theta
            \end{equation*}
            donde hemos usado que, al ser $\varphi$ continua en $z\in S_0$, como $\{z_n\}\to z$, se tiene que $\{\varphi(z_n)\}\to \varphi(z)$.
        \end{itemize}
        \begin{observacion}
            Notemos que podríamos haber generalizado todo en el segundo caso, considerando $S_{\theta+\pi}$. No obstante, se ha optado por hacerlo de forma más explícita para facilitar la comprensión, ya que el primer caso seguramente sea más intuitivo.
        \end{observacion}
        
        \item[Usando el punto de vista topoógico:]~\\
        
        Definimos la función proyección:
        \Func{\pi}{\mathbb{R}}{\mathbb{R}/2\pi\mathbb{Z}}{x}{x+2\pi\mathbb{Z}}

        Tenemos la siguiente descomposición de $\bb{C}^*$:
        \begin{equation*}
            \bb{C}^* = \left(\bb{C}^*\setminus \bb{R}^-\right)\cup \left(\bb{C}^*\setminus \bb{R}^+\right)
        \end{equation*}
        Tenemos que:
        \begin{itemize}
            \item \ul{En $\bb{C}^*\setminus \bb{R}^-$}:
            \begin{equation*}
                \Arg\left(z\right)=\left(\pi\circ \arg\right)(z)\qquad \forall z\in \bb{C}^*\setminus \bb{R}^-
            \end{equation*}

            Por tanto, $\Arg$ es continua en $\bb{C}^*\setminus \bb{R}^-$.

            \item \ul{En $\bb{C}^*\setminus \bb{R}^+$}:
            
            Por el Ejercicio~\ref{ej:2.2}, sabemos que $\exists \varphi\in \cc{C}(S_{0})$ tal que:
            \begin{equation*}
                \Arg\left(z\right)=\left(\pi\circ \varphi\right)(z)\qquad \forall z\in S_0=\bb{C}^*\setminus \bb{R}^+
            \end{equation*}

            Por tanto, $\Arg$ es continua en $\bb{C}^*\setminus \bb{R}^+$.
        \end{itemize}

        Por el carácter local de la continuidad, $\Arg$ es continua en $\bb{C}^*$.
    \end{description}
\end{ejercicio}

\begin{ejercicio}
    Dado $z \in \mathbb{C}$, probar que la sucesión $\left\{\left(1 + \dfrac{z}{n}\right)^n\right\}$ es convergente y calcular su límite.\\

    Para facilitar la notación, sea:
    \begin{equation*}
        z_n = \left(1 + \dfrac{z}{n}\right)^n\qquad \forall n\in \mathbb{N}
    \end{equation*}

    En primer lugar, vamos a estudiar el límite de la sucesión $\{|z_n|\}$:
    \begin{align*}
        |z_n| &= \left|\left(1 + \dfrac{z}{n}\right)^n\right| = \left|1 + \dfrac{z}{n}\right|^n = \left(\sqrt{\left(1+\dfrac{\Re z}{n}\right)^2 + \left(\dfrac{\Im z}{n}\right)^2}\right)^n
        =\\&= \sqrt{\left(1 + \dfrac{\Re^2z}{n^2} + \dfrac{2\Re z}{n} + \dfrac{\Im^2 z}{n^2}\right)^{n}}
        = \sqrt{\left(1 + \dfrac{\dfrac{\Re^2z + \Im^2z}{n}+2\Re z}{n}\right)^{n}}
    \end{align*}

    Por tanto, tenemos que:
    \begin{align*}
        \lim_{n\to \infty}|z_n| &= \sqrt{\lim_{n\to \infty}\left(1 + \dfrac{\dfrac{\Re^2z + \Im^2z}{n}+2\Re z}{n}\right)^{n}}
        =\\&= \sqrt{\exp\left(\lim_{n\to \infty}\dfrac{\Re^2z + \Im^2z + 2n\Re z}{n}+2\Re z\right)}
        = \sqrt{\exp(2\Re z)} = e^{\Re z}
    \end{align*}
    donde en la primera igualdad hemos usado que la raíz es una función continua, y en la segunda igualdad hemos usado el Criterio de Euler. A continuación, estudiamos los argumentos de $z_n$. Para ello, definimos:
    \begin{equation*}
        w_n = 1 + \dfrac{z}{n}\qquad \forall n\in \mathbb{N}
    \end{equation*}

    Como $\{w_n\}\to 1$, $\exists N\in \mathbb{N}$ tal que $\forall n\geq N$ se tiene que $\Re w_n>0$. Por tanto, $\forall n\geq N$ se tiene que:
    \begin{equation*}
        \arg w_n = \arctan\left(\dfrac{\Im w_n}{\Re w_n}\right) = \arctan\left(\dfrac{\Im z}{n+\Re z}\right)
    \end{equation*}

    Como $\Arg(zw)=\Arg z + \Arg w$ para todo $z,w\in \mathbb{C}^{\ast}$, tenemos que:
    \begin{equation*}
        \Arg z_n = \Arg \left((w_n)^n\right) = n\Arg w_n \Longrightarrow
        n\arctan\left(\dfrac{\Im z}{n+\Re z}\right)\in \Arg z_n\qquad \forall n\geq N
    \end{equation*}

    Por tanto, definimos la sucesión $\{\theta_n\}$ como sigue:
    \begin{equation*}
        \theta_n = \begin{cases}
            \arg z_n & \text{si } n<N\\
            n\arctan\left(\dfrac{\Im z}{n+\Re z}\right) & \text{si } n\geq N
        \end{cases}
    \end{equation*}

    Por tanto, para todo $n\in \mathbb{N}$, tenemos que $\theta_n\in \Arg z_n$. 
    Calculemos el límite de la sucesión $\{\theta_n\}$:
    \begin{align*}
        \lim_{n\to \infty} \theta_n &= \lim_{n\to \infty}n\arctan\left(\dfrac{\Im z}{n+\Re z}\right) = \lim_{n\to \infty}\dfrac{\arctan\left(\dfrac{\Im z}{n+\Re z}\right)}{\dfrac{1}{n}}
        =\\&= \lim_{n\to \infty}\dfrac{-n^2}{1+\left(\dfrac{\Im z}{n+\Re z}\right)^2}\cdot \dfrac{-\Im z}{(n+\Re z)^2}
        = \lim_{n\to \infty}\dfrac{n^2\Im z}{(n+\Re z)^2+\Im^2 z} = \Im z
    \end{align*}
    
    
    Uniendo ambos resultados, tenemos que:
    \begin{align*}
        z_n = |z_n|\left(\cos(\theta_n) + i\sen(\theta_n)\right) \qquad \forall n\in \mathbb{N}
    \end{align*}

    Tomando límite, y como las funciones seno y coseno son continuas, tenemos que:
    \begin{equation*}
        \lim_{n\to \infty}z_n = \lim_{n\to \infty}|z_n|\left(\cos\left(\lim_{n\to \infty}\theta_n\right) + i\sen\left(\lim_{n\to \infty}\theta_n\right)\right) = e^{\Re z}\left(\cos(\Im z) + i\sen(\Im z)\right)
    \end{equation*}
\end{ejercicio}
    \section{Espacios recubridores}

\begin{ejercicio}
    Sea $R=\left]-1,1\right[\subset \mathbb{R}$. Demuestra que existe una aplicación recubridora $p:R\to \mathbb{S}^1$. ¿Se puede levantar al recubridor la aplicación $f:\mathbb{S}^1\to\mathbb{S}^1$ dada por $f(x,y) = (y,x)$?\\

    \noindent
    Sabemos que $R$ es homeomorfo a $\mathbb{R}$ y que la aplicación $p_0:\mathbb{R}\to\mathbb{S}^1$ dada por:
    \begin{equation*}
        p_0(x) = (\cos(2\pi x), \sen(2\pi x))
    \end{equation*}
    es una aplicación recubridora. Si tomamos $h:R\to \mathbb{R}$ cualquier homeomofismo tendremos entonces que $p=p_0\circ h:R\to\mathbb{S}^1$ es una aplicación recubridora (como vimos en el Tema 1). En vistas del segundo apartado daremos $h$ de forma explícita, como por ejemplo:
    \begin{equation*}
        h(t) = \tg\left(\frac{\pi}{2}t\right)
    \end{equation*}
    Si consideramos la aplicación $f$ enunciada, estamos en la siguiente situación:
    \begin{figure}[H]
        \centering
        \shorthandoff{""}
        \begin{tikzcd}
                                        & R \arrow[d, "p"] \\
            \mathbb{S}^1 \arrow[r, "f"] & \mathbb{S}^1    
        \end{tikzcd}
        \shorthandon{""}
    \end{figure}
    \noindent
    Fijado $x_0 = (0,1)\in \mathbb{S}^1$, tomamos $b_0 = f(x_0) = (1,0)$ y $r_0 = 0\in p^{-1}(b_0)$. Tenemos que existe un levantamiento $\hat{f}:\mathbb{S}^1\to R$ de $f$ con $\hat{f}(x_0) = r_0$ si y solo si:
    \begin{equation}\label{eq:ej1_rel3}
        f_\ast(\pi_1(\mathbb{S}^1,(0,1))) \subseteq p_\ast(\pi_1(R,0))
    \end{equation}
    Pero tenemos que:
    \begin{equation*}
        \pi_1(R,0) = \{[\varepsilon_0]\} \quad\Longrightarrow\quad p_\ast(\pi_1(R,0)) = \{[\varepsilon_{(1,0)}]\}
    \end{equation*}
    $f_\ast$ es un isomorfismo por ser $f$ un homeomorfismo, por lo que:
    \begin{equation*}
        \pi_1(\mathbb{S}^1,(0,1))\cong \mathbb{Z} \quad\Longrightarrow\quad f_\ast(\pi_1(\mathbb{S}^1,(0,1))) \cong \mathbb{Z}
    \end{equation*}
    como no se puede cumplir la ecuación~\eqref{eq:ej1_rel3} tenemos que no existe dicho levantamiento. El mismo razonamiento puede repetirse para todo punto $x_0\in \mathbb{S}^1$.
\end{ejercicio}

\begin{ejercicio} 
    Dada la aplicación recubridora estándar $p:\mathbb{R}\to\mathbb{S}^1$ definida por
    \begin{equation*}
        p(x) = (\cos(2\pi x), \sen(2\pi x)),
    \end{equation*}
    determina si la aplicación $f:\mathbb{S}^1\to\mathbb{S}^1$ dada por $f(x,y) = (x,|y|)$ puede ser levantada al recubridor. Y, en tal caso, calcula sus levantamientos.\\

    \noindent
    Fijado $x_0=(1,0)\in \mathbb{S}^1$, tomamos $b_0 = f(x_0) = (1,0)$ y consideramos el punto $r_0=0\in p^{-1}(b_0)$. Existirá un levantamiento $\hat{f}:\mathbb{S}^1\to \mathbb{R}$ de $f$ con $\hat{f}(x_0) = r_0$ si y solo si:
    \begin{equation*}
        f_\ast(\pi_1(\mathbb{S}^1,x_0)) \subseteq p_\ast(\pi_1(\mathbb{R},r_0))
    \end{equation*}
    Tenemos que:
    \begin{equation*}
        \pi_1(\mathbb{R},r_0) = \{[\varepsilon_{r_0}]\} \quad\Longrightarrow\quad p_\ast(\pi_1(\mathbb{R},r_0)) = \{[\varepsilon_{b_0}]\}
    \end{equation*}
    Por otra parte, $\pi_1(\mathbb{S}^1,x_0)$ está generado por $[\alpha]$, con $\alpha:[0,1]\to \mathbb{S}^1$ dado por:
    \begin{equation*}
        \alpha(t) = (\cos(2\pi t), \sen(2\pi t))
    \end{equation*}

    y tenemos que:
    \begin{equation*}
        f_\ast([\alpha]) = [f\circ \alpha] 
    \end{equation*}

    donde:
    \begin{equation*}
        (f\circ \alpha)(t) = (\cos(2\pi t), |\sen(2\pi t)|)
    \end{equation*}
    que claramente es un arco homotópico a $\varepsilon_{b_0}$, por lo que tenemos:
    \begin{equation*}
        f_\ast(\pi_1(\mathbb{S}^1,x_0)) = \{[\varepsilon_{b_0}]\}
    \end{equation*}
    De donde existe una única $\hat{f}:\mathbb{S}^1\to \mathbb{R}$ aplicación continua con $\hat{f}(x_0) = r_0$. % // TODO: Calcular levantamiento
\end{ejercicio}

\begin{ejercicio}
    ¿Existe una aplicación recubridora desde $\mathbb{S}^1\times \mathbb{S}^1$ en $\mathbb{S}^1$?\\

    \noindent
    No: por reducción al absurdo, si existiera una aplicación recubridora $p:\mathbb{S}^1\times \mathbb{S}^1\to \mathbb{S}^1$:
    \begin{description}
        \item [Opción 1.] Tendríamos entonces por el Teorema de Monodromía que la aplicación $p$ induce un homomorfismo inyectivo $p_\ast:\pi_1(\mathbb{S}^1\times \mathbb{S}^1,x_0)\to \pi_1(\mathbb{S}^1,p(x_0))$ con $\pi_1(\mathbb{S}^1\times \mathbb{S}^1,x_0)\cong \mathbb{Z}\times \mathbb{Z}$ y $\pi_1(\mathbb{S}^1,p(x_0))\cong \mathbb{Z}$. Sin embargo, ningún subgrupo de $\mathbb{Z}$ es isomorfo a $\mathbb{Z}\times \mathbb{Z}$, por lo que llegamos a una contradicción.
        \item [Opción 2.] $\mathbb{R}$ es el recubridor universal de $\mathbb{S}^1$, por lo que entonces este también recubre a $\mathbb{S}^1\times \mathbb{S}^1$, y $\mathbb{R}^2$ también es recubridor universal de $\mathbb{S}^1\times \mathbb{S}^1$, por lo que ha de existir un isomorfismo de recubridores (y por tanto, un homeomorfismo) entre $\mathbb{R}$ y $\mathbb{R}^2$, algo imposible, pues $\mathbb{R}$ menos un punto no es conexo y $\mathbb{R}^2$ menos un punto sí lo es.
    \end{description}
\end{ejercicio}

\begin{ejercicio}
    Determina, salvo isomorfismos, todos los recubridores del cilindro $\mathbb{S}^1\times \mathbb{R}$.\\

    \noindent
    Fijado $b_0 = (1,0,0)\in \mathbb{S}^1\times \mathbb{R}$, tenemos que $\pi_1(\mathbb{S}^1\times \mathbb{R},b_0)\cong \mathbb{Z}$, y los subgrupos de $\mathbb{Z}$ son:
    \begin{equation*}
        H_k = k\mathbb{Z} \qquad k \in \mathbb{N}\cup \{0\}
    \end{equation*}
    \begin{itemize}
        \item Asociado a $H_0 = \{0\}$ tenemos el recubridor $(\mathbb{R}^2,p)$, donde $p = p_0\times Id_\mathbb{R}$, con $p_0:\mathbb{R}\to \mathbb{S}^1$ la aplicación recubridora estándar.
        \item Asociado a $H_k$ con $k\geq 1$ tenemos el recubridor $(\mathbb{S}^1\times \mathbb{R}, p_k)$, con la aplicación recubridora $p_k:\mathbb{S}^1\times \mathbb{R}\to \mathbb{S}^1\times \mathbb{R}$ dada por:
            \begin{equation*}
                p_k(\cos\theta, \sen\theta, y) = (\cos(k\theta), \sen(k\theta), y)
            \end{equation*}
    \end{itemize}
\end{ejercicio}

\begin{ejercicio} % // TODO: HACER
    Demuestra que si $p:R\to B$ es un homeomorfismo local con $R$ compacto y $B$ Hausdorff (y conexo), entonces $p$ es una aplicación recubridora.\\

    \noindent
    Recordamos que si $p$ es un homeomorfismo local entonces es una aplicación continua de forma que para cada punto $x\in R$ existe un abierto $U_x$ entorno de $x$ de forma que $p(U_x)$ es abierto en $B$ y $p\big|_{U_x}:U_x\to p(U_x)$ es un homeomorfismo.

    \begin{itemize}
        \item Tenemos que $p$ es continua por ser un homeomorfismo local.
        \item Como $R$ es compacto y $B$ es Hausdorff tenemos que $p$ es cerrada.
        \item Sea $U\subseteq R$ un abierto, podemos escribir:
            \begin{equation*}
                U = \bigcup_{x\in U}U_x
            \end{equation*}
            de donde $U_x\cap U$ es un abierto para cada $x\in U$, por lo que
            \begin{equation*}
                U = \bigcup_{x\in U} U_x \cap U
            \end{equation*}
            Para cada $x\in X$ tenemos que $p\big|_{U_x}:U_x\to p(U_x)$ es un homeomorfismo, por lo que $p(U_x\cap U)$ es un abierto de $B$. Finalmente, vemos:
            \begin{equation*}
                p(U) = p\left(\bigcup_{x\in U} U_x\cap U\right) = \bigcup_{x\in U} p(U_x\cap U)
            \end{equation*}
            Por lo que $p(U)$ es abierto, de donde $p$ es una aplicación abierta.
    \end{itemize}
    Como $p$ es abierta y cerrada tenemos que $p(R)\subseteq B$ es abierto y cerrado, como $B$ es conexo ha de ser $p(R) = B$, por lo que $p$ es sobreyectiva. Falta probar que todo punto $b\in B$ tiene un entorno abierto regularmente recubierto. % // TODO:
\end{ejercicio}

\begin{ejercicio} % // TODO: HACER
    Sea $p:\mathbb{R}^n\to \mathbb{R}^n$ un homeomorfismo local tal que para todo $r>0$ se tiene que $p^{-1}(\overline{B}(0,r))$ es compacto. Demuestra que $p$ es un homeomorfismo.
\end{ejercicio}

\begin{ejercicio}
    Sea $X$ conexo y localmente arcoconexo con grupo fundamental finito. Si $f,g:X\to \mathbb{R}$ son aplicaciones continuas cumpliendo que $f(x)^2 + g(x)^2 = 1$ para todo $x\in X$. Prueba que existe $h:X\to\mathbb{R}$ continua tal que $\cos(h(x)) = f(x)$ y $\sen(h(x)) = g(x)$ para cada $x\in X$.\\

    \noindent
    Observando la condición que cumplen $f$ y $g$ así como de la presencia de senos y cosenos pensamos en que el problema está relacionado con una circunferencia. Definimos por tanto $F:X\to \mathbb{S}^1$ dada por:
    \begin{equation*}
        F(x) = (f(x),g(x))
    \end{equation*}
    que está bien definida, pues $f(x)^2+g(x)^2 = 1$, por lo que $F(x)\in \mathbb{S}^1$ para todo $x\in X$. Si consideramos la aplicación recubridora estándar $p_0:\mathbb{R}\to \mathbb{S}^1$, fijado $x_0\in X$, $b_0 = F(x_0)$ y tomando $r_0\in p^{-1}(b_0)$ tenemos que existe un levantamiento $\hat{F}:X\to \mathbb{R}$ de $F$ con $\hat{F}(x_0) = r_0$ si y solo si:
    \begin{equation*}
        F_\ast(\pi_1(X,x_0)) \subseteq p_\ast(\pi_1(\mathbb{R},r_0))
    \end{equation*}
    Por una parte:
    \begin{equation*}
        \pi_1(\mathbb{R},r_0) = \{[\varepsilon_{r_0}]\} \quad\Longrightarrow\quad p_\ast(\pi_1k(\mathbb{R},r_0)) = \{[\varepsilon_{b_0}]\}
    \end{equation*}
    Por otra tenemos que como $\pi_1(X,x_0)$ es un grupo finito ha de ser por tanto $F_\ast(\pi_1(X,x_0))$ un subgrupo finito de $\pi_1(\mathbb{S}^1,b_0)\cong \mathbb{Z}$, que solo tiene un subgrupo finito, $\{[\varepsilon_{b_0}]\}$, de donde deducimos que ha de ser $F_\ast(\pi_1(X,x_0)) = \{[\varepsilon_{b_0}]\}$. Sea por tanto $\hat{F}:X\to \mathbb{R}$ el único levantamiento de $F$ con $\hat{F}(x_0) = r_0$, tenemos que:
    \begin{equation*}
        (\cos(2\pi \hat{F}(x)), \sen(2\pi \hat{F}(x))) = p(\hat{F}(x)) = F(x) = (f(x),g(x)) \qquad \forall x\in X
    \end{equation*}
    Por lo que si definimos $h:X\to \mathbb{R}$ dada por $h(x) = 2\pi \hat{F}(x)$, tenemos que:
    \begin{equation*}
        (\cos(h(x)), \sen(h(x))) = F(x) = (f(x),g(x)) \qquad \forall x\in X
    \end{equation*}
\end{ejercicio}

\begin{ejercicio} 
    Sean $p:X\to Y$ y $f:Y\to Z$ dos aplicaciones continuas tales que $p$ y $f\circ p$ son aplicaciones recubridoras. Prueba que $f$ es también una aplicación recubridora.\\

    \noindent
    Como $f\circ p$ es una aplicación recubridora es en particular sobreyectiva, por lo que $f$ también es sobreyectiva. Basta ver que para todo $z\in Z$ existe un entorno abierto de $z$ regularmente recubierto. Fijado $z\in Z$, tomamos $O_z$ un entorno abierto y arcoconexo de $z$ regularmente recubierto por la aplicación recubridora $f\circ p$, por lo que:
    \begin{equation*}
        p^{-1}(f^{-1}(O_z)) = (f\circ p)^{-1}(O_z) = \biguplus_{i \in I}A_i
    \end{equation*}
    con $A_i\subseteq X$ abierto y $(f\circ p)\big|_{A_i}:A_i\to O_z$ homeomorfismo para cada $i \in I$. Si aplicamos $p$ a dicha igualdad, obtenemos que:
    \begin{equation*}
        \bigcup_{i \in I}p(A_i) = p\left(\bigcup_{i \in I}A_i\right) = p(p^{-1}(f^{-1}(O_z))) \AstIg f^{-1}(O_z)
    \end{equation*}
    donde en $(\ast)$ hemos usado que $p$ es sobreyectiva, y tenemos además que $p(A_i)$ es abierto para cada $i \in I$, ya que $p$ es una aplicación abierta por ser una aplicación recubridora.\\

    \noindent
    Veamos ahora que si $p(A_i)\cap p(A_j)\neq \emptyset  \Longrightarrow p(A_i) = p(A_j)$: sea $y \in p(A_i)\cap p(A_j)$ existen entonces $x_i \in A_i$, $x_j \in A_j$ de forma que $p(x_i) = y = p(x_j)$, basta demostrar que $p(A_i)\subseteq p(A_j)$ y la otra inclusión será análoga. Para ello, sea $y'\in p(A_i)$, tenemos que existe $x'\in A_i$ de forma que $p(x') = y'$. Tomamos ahora $w=f(y), w'=f(y')$ y tendremos que $w,w'\in O_z$. Como $O_z$ es arcoconexo, existirá $\gamma:[0,1]\to Z$ de forma que $\gamma(0) = w$, $\gamma(1) = w'$. Si tomamos como $\hat{\gamma}$ el único levantamiento de $\gamma$ por $f\circ p$ con $\hat{\gamma}(0) = x_i$, como $Im \hat{\gamma}$ es un conjunto conexo ha de ser $Im \hat{\gamma}\subseteq A_i$ y como $(f\circ p)\big|_{A_i}:A_i\to O_z$ es un homeomorfismo, ha de ser $\hat{\gamma}(1) = x'$, ya que $(f\circ p)^{-1}(\{w'\})\cap A_i = \{x'\}$.\\

    \noindent
    Si consideramos ahora $\delta = p\circ \hat{\gamma}$ tenemos que:
    \begin{equation*}
        \delta(0) = p(x_i) = y, \qquad \delta(1) = p(x') = y'
    \end{equation*}
    y consideramos como $\hat{\delta}$ el único levantamiento de $\delta$ por $p$ con $\hat{\delta}(0) = x_j$. Tendremos:
    \begin{equation*}
        (f\circ p)\circ \hat{\delta} = f\circ \delta = f\circ p \circ \hat{\gamma} = \gamma
    \end{equation*}
    Por lo que $\hat{\delta}$ es el único levantamiento de $\gamma$ por $f\circ p$ con $\hat{\delta}(0) = x_j \in A_j$. Tendrá que ser $Im \hat{\delta}\subseteq A_j$, luego $\hat{\delta}(1)\in A_j$ y tenemos que:
    \begin{equation*}
        p(\hat{\delta}(1)) = \delta(1) = y'
    \end{equation*}
    por lo que $y'\in p(A_j)$, como queríamos probar. De esta forma, si eliminamos de la unión:
    \begin{equation*}
        \bigcup_{i \in I}p(A_i)
    \end{equation*}
    los conjuntos repetidos, obtendremos una unión disjunta, luego será:
    \begin{equation*}
        f^{-1}(O_z) = \biguplus_{j \in J}p(A_j)
    \end{equation*}
    y como $(f\circ p)\big|_{A_j}:A_j\to O_z$ es un homeomorfismo para cada $j \in J$ tenemos que:
    \begin{equation*}
        (f\circ p)\big|_{A_j} = f\big|_{p(A_j)} \circ p\big|_{A_j}
    \end{equation*}
    Por lo que $f\big|_{p(A_j)}:A_j\to O_z$ es inyectiva y con todas las demás propiedades que hemos probado de $f$ deducimos que $f\big|_{p(A_j)}:p(A_j)\to O_z$ es un homeomorfismo para cada $j\in J$. En definitiva, hemos probado que la aplicación $f$ es una aplicación recubridora.
\end{ejercicio}

\begin{ejercicio}
    Sean $p_1:X\to Y$ y $p_2:Y\to Z$ dos aplicaciones recubridoras. Prueba que si $Z$ tiene recubridor universal, entonces $p_2\circ p_1:X\to Z$ es una aplicación recubridora.\\

    \noindent
    Si $Z$ tiene recubridor universal $(R,p)$ tenemos que como $(Y,p_2)$ también recubre a $Z$ existirá entonces un homomorfismo de recubridores $\phi_1$ de $(R,p)$ en $(Y,p_2)$, por lo que $R$ es recubridor universal de $Y$. Repetiendo el razonamiento con el recubridor $(X,p_1)$ de $Y$ tenemos que existe un homomorfismo de recubridores $\phi_2$ de $(R,p)$ en $(X,p_1)$, obteniendo que el siguiente diagrama es conmutativo:
    \begin{figure}[H]
        \centering
        \shorthandoff{""}
        \begin{tikzcd}
                               &                    & R \arrow[d, "p"] \arrow[ld, "\phi_1", bend right] \arrow[lld, "\phi_2", bend right] \\
            X \arrow[r, "p_1"] & Y \arrow[r, "p_2"] & Z                                                                                  
        \end{tikzcd}
        \shorthandon{""}
    \end{figure}
    \noindent
    De esta forma, tenemos que:
    \begin{equation*}
        (p_2\circ p_1)\circ \phi_2 = p
    \end{equation*}
    con $\phi_2$ y $p$ aplicaciones recubridoras, por lo que por el ejercicio anterior tenemos que $p_2\circ p_1$ es una aplicación recubridora.
\end{ejercicio}

\begin{ejercicio}
    Sean $p:R\to B$ una aplicación recubridora y $b_0\in B$. Definimos la aplicación (correspondencia del levantamiento generalizada)
    \Func{\phi}{p^{-1}(\{b_0\})\times \pi_1(B,b_0)}{p^{-1}(\{b_0\})}{(r,[\alpha])}{\hat{\alpha}_r(1)}
    donde $\hat{\alpha}_r(s)$ es el único levantamiento de $\alpha(s)$ con condición inicial $\hat{\alpha}_r(0)=r$. Demuestra que:
    \begin{enumerate}[label=\alph*)]
        \item $\phi$ está bien definida.

            Fijado $r\in p^{-1}(b_0)$ tenemos que la aplicación correspondencia del levantamiento 
            \Func{\phi_r}{\pi_1(B,b_0)}{p^{-1}(\{b_0\})}{[\alpha]}{\hat{\alpha}(1)}
            donde $\hat{\alpha}$ es el único levantamiento de $\alpha$ con condición inicial $\hat{\alpha}(0) = r$ está bien definida, por lo que $\phi$ estará bien definida.
        \item $\phi(r,[\varepsilon_{b_0}])=r$, para cualquier $r\in p^{-1}(\{b_0\})$.

            Sea $r\in p^{-1}(\{b_0\})$, si consideramos $\varepsilon_{r}$ tenemos que:
            \begin{equation*}
                p\circ \varepsilon_r = \varepsilon_{b_0}, \qquad \varepsilon_r(0) = r
            \end{equation*}
            Por lo que $\varepsilon_r$ es el único levantamiento de $\varepsilon_{b_0}$ con $\varepsilon_r(0) = r$, de donde ha de ser:
            \begin{equation*}
                \phi_r(r,[\varepsilon_{b_0}]) = \varepsilon_r(1) = r
            \end{equation*}
        \item $\phi(\phi(r,[\alpha]),[\beta]) = \phi(r,[\alpha]\ast[\beta])$, para cualesquiera $r\in p^{-1}(\{b_0\})$ y $[\alpha],[\beta]\in \pi_1(B,b_0)$.

            Sea $\hat{\alpha}_r$ el único levantamiento de $\alpha$ con $\hat{\alpha}_r(0) = r$ tenemos entonces que $\phi(r,[\alpha]) = \hat{\alpha}_r(1) = r_1\in p^{-1}(\{b_0\})$. Sea $\hat{\beta}_{r_1}$ el único levantamiento de $\beta$ con $\hat{\beta}_{r_1}(0) = r_1$, tenemos que $\phi(r_1,[\beta]) = \hat{\beta}_{r_1}(1)$.

            Veamos finalmente que $\hat{\alpha}_r\ast \hat{\beta}_{r_1}$ es un levantamiento de $\alpha\ast\beta$, que además cumple:
            \begin{equation*}
                (\hat{\alpha}_r\ast\hat{\beta}_{r_1})(0) = \hat{\alpha}_r(0) = r
            \end{equation*}
            Para ello, vemos que:
            \begin{align*}
                p((\hat{\alpha}_r\ast\hat{\beta}_{r_1})(t)) &= p\left(
                \left\{\begin{array}{ll}
                        \hat{\alpha}_r(2t) & \text{si\ } 0\leq t\leq \nicefrac{1}{2} \\ 
                        \hat{\beta}_{r_1}(2t-1) & \text{si\ } \nicefrac{1}{2}\leq t\leq 1
                \end{array}\right. \right) \\
                            &= \left\{\begin{array}{ll}
                                    p(\hat{\alpha}_r(2t)) & \text{si\ } 0\leq t\leq \nicefrac{1}{2} \\
                                    p(\hat{\beta}_{r_1}(2t-1)) & \text{si\ } \nicefrac{1}{2}\leq t\leq 1
                                    \end{array}\right.  \\ &= \left\{\begin{array}{ll}
                                \alpha(2t) & \text{si\ } 0\leq t\leq\nicefrac{1}{2} \\
                                \beta(2t-1) & \text{si\ } \nicefrac{1}{2}\leq t\leq 1
                            \end{array}\right.  = (\alpha\ast \beta)(t)
            \end{align*}
            Por lo que ha de ser:
            \begin{equation*}
                \phi(r,[\alpha]\ast [\beta]) = \phi(r,[\alpha\ast \beta]) = (\hat{\alpha}_r\ast\hat{\beta}_{r_1})(1) = \hat{\beta}_{r_1}(1) = \phi(\phi(r,[\alpha]),[\beta])
            \end{equation*}
        \item $\phi$ es sobreyectiva.

            En efecto, sea $r\in p^{-1}(b_0)$, tenemos que $\phi(r,[\varepsilon_{b_0}]) = r$.
        \item $\phi(r,[\alpha]) = r$ si y solo si $[\alpha] \in p_\ast(\pi_1(R,r))$.
            Por doble implicación:
            \begin{description}
                \item [$\Longleftarrow )$] Si $[\alpha]\in p_\ast(\pi_1(R,r))$ tenemos entonces que existe $[\beta]\in \pi_1(R,r)$ de forma que $[p\circ \beta] = p_\ast([\beta]) = [\alpha]$, por lo que $\beta$ es un levantamiento de $\alpha$, que además cumple $\beta(0) = r$, por lo que:
                    \begin{equation*}
                        \phi(r,[\alpha]) = \hat{\alpha}_r(0) = \beta(0) = r
                    \end{equation*}
                \item [$\Longrightarrow )$] Si $\phi(r,[\alpha]) = r$ tenemos entonces que el único levantamiento $\hat{\alpha}_r$ de $\alpha$ con $\hat{\alpha}_r(0) = r$ es un lazo en $R$, $[\hat{\alpha}_r]\in \pi_1(R,r)$, y tenemos que $p_\ast([\hat{\alpha}_r]) = [\alpha]$ por ser $\hat{\alpha}_r$ levantamiento de $\alpha$.
            \end{description}
        \item El cardinal de $p^{-1}(\{b_0\})$ coincide con el cardinal de $\pi_1(B,b_0)/p_\ast(\pi_1(R,r))$ (es decir, el índice de $p_\ast(\pi_1(R,r))$ como subgrupo de $\pi_1(B,b_0)$). % // TODO: TERMINAR
    \end{enumerate}
\end{ejercicio}

\begin{ejercicio}\label{ej:11_rel3}
    Sea $X$ un espacio topológico (conexo y localmente arcoconexo), $G$ un grupo de homeomorfismos de $X$ y $X/\cc{R}_G$ el espacio topológico cociente dado por la relación de equivalencia:
    \begin{equation*}
        x\cc{R}_G y \quad\Longleftrightarrow\quad\exists \varphi \in G : y = \varphi(x)
    \end{equation*}
    para cualesquiera $x,y\in X$.\newline
    Demuestra que la aplicación proyección $\pi:X\to X/\cc{R}_G$ es recubridora si y solo si para cada $x\in X$ existe un entorno suyo $U_x$ tal que $\varphi(U_x)\cap U_x=\emptyset $ para todo $\varphi\in G\setminus \{Id_X\}$.\newline
    Deduce que, además, $\varphi:(X,\pi)\to (X,\pi)$ es un isomorfismo de recubridores si y solo si $\varphi \in G$.
\end{ejercicio}

\begin{ejercicio}
    Para cada $n\in \mathbb{Z}$ se define $f_n:\mathbb{R}^2\to\mathbb{R}^2$ como $f_n(x,y) = (x+2n,{(-1)}^{n}y)$. Utiliza el ejercicio anterior para demostrar que:
    \begin{enumerate}[label=\alph*)]
        \item $G = \{f_n:n\in \mathbb{Z}\}$ es un grupo de homeomorfismos de $\mathbb{R}^2$ y para cada $x\in \mathbb{R}^2$ existe un entorno suyo $U_x$ tal que $f_n(U_x)\cap U_x = \emptyset $ para todo $n\in \mathbb{Z}\setminus \{0\}$.
        \item La proyección $p:\mathbb{R}^2\to\mathbb{R}^2/\cc{R}_G$ es una aplicación recubridora de $\mathbb{R}^2$ en la cinta de Moebius $\mathbb{R}^2/\cc{R}_G$.
    \end{enumerate}
\end{ejercicio}

\begin{ejercicio}
    Para cada $n,m\in \mathbb{Z}$ se define $f_{n,m}:\mathbb{R}^2\to\mathbb{R}^2$ como:
    \begin{equation*}
        f_{n,m}(x,y) = (x,{(-1)}^{n}y) + 2(n,{(-1)}^{n}m).
    \end{equation*}
    Utiliza el ejercicio~\ref{ej:11_rel3} para demostrar que:
    \begin{enumerate}[label=\alph*)]
        \item $G=\{f_{n,m}:n,m\in \mathbb{Z}\}$ es un grupo de homeomorfismos de $\mathbb{R}^2$ y para cada $x\in \mathbb{R}^2$ existe un entorno suyo $U_x$ tal que $f_{n,m}(U_x)\cap U_x=\emptyset $ para todo $n,m\in \mathbb{Z}$ con $(n,m)\neq (0,0)$.
        \item La proyección $p:\mathbb{R}^2\to\mathbb{R}^2/\cc{R}_G$ es una aplicación recubridora de $\mathbb{R}^2$ en la botella de Klein $\mathbb{R}^2/\cc{R}_G$.
    \end{enumerate}
\end{ejercicio}

\begin{ejercicio}
    Razona si son verdaderas o falsas las siguientes afirmaciones:
    \begin{enumerate}[label=\alph*)]
        \item Existe una aplicación recubridora $p:\mathbb{S}^1\to [0,1]$.

            Es falsa, ya que si existiera una aplicación recubridora $p:\mathbb{S}^1\to [0,1]$ el Teorema de Monodromía nos diría que el homomorfismo inducido por la aplicación $p$ $p_\ast:\pi_1(\mathbb{S}^1,x_0)\to \pi_1([0,1],p(x_0))$ es inyectivo, con $\pi_1(\mathbb{S}^1,x_0)\cong \mathbb{Z}$ y $\pi_1([0,1],p(x_0)) = \{[\varepsilon_{x_0}]\}$, lo que llevaría a una contradicción.
        \item Existe una aplicación recubridora $p:\mathbb{R}\bb{P}^2\to \mathbb{S}^1$.

            Es falsa, ya que si existiera una aplicación recubridora $p:\mathbb{R}\bb{P}^2\to \mathbb{S}^1$:
            \begin{description}
                \item [Opción 1.] el Teorema de Monodromía nos diría que el homomorfismo inducido $p_\ast:\pi_1(\mathbb{R}\bb{P}^2,x_0)\to \pi_1(\mathbb{S}^1,p(x_0))$ es inyectivo, con $\pi_1(\mathbb{R}\bb{P}^2,x_0)\cong \mathbb{Z}_2$, pero entonces tenemos que $p_\ast(\pi_1(\mathbb{R}\bb{P}^2,x_0)) \cong \mathbb{Z}_2$ y subgrupo de $\pi_1(\mathbb{S}^1,p(x_0))\cong \mathbb{Z}$, que es una contradicción.
                \item [Opción 2.] Como $\mathbb{R}$ es el recubridor universal de $\mathbb{S}^1$ tendríamos entonces que $\mathbb{R}$ recubriría a $\mathbb{R}\bb{P}^2$, y el recubridor universal de $\mathbb{R}\bb{P}^2$ es $\mathbb{S}^2$, por lo que tendríamos entonces un homeomorfismo entre $\mathbb{R}$ y $\mathbb{S}^2$, algo imposible, pues $\mathbb{S}^2$ es compacto y $\mathbb{R}$ no.
            \end{description}
        \item El semiplano $X=\{(x,y)\in \mathbb{R}^2 : y\geq 0\}$ es el recubridor universal de la bola cerrada punteada $Y = \{(x,y)\in \mathbb{R}^2 : 0<x^2+y^2 \leq 1\}$.

            Es verdadera, es claro que $X$ es simplemente conexo, por lo que basta ver que existe alguna aplicación recubridora $p:X\to Y$.
            \begin{figure}[H]
                \centering
                \begin{tikzpicture}
                    \filldraw[fill=blue!40, draw=blue, opacity=0.3] (0,0) rectangle (3,2);
                    \draw[thick] (0,0) -- (3,0);
                    \fill(1.5,1) node[] {$X$};
                    \draw[thick](6,1) circle(1cm);
                    \draw[fill = red!40, draw=red, opacity=0.3](6,1) circle(1cm);
                    \fill(5.5,1) node[] {$Y$};
                    \draw[fill=white] (6,1) circle (2pt);
                \end{tikzpicture}
            \end{figure}
            Sabemos ya de la existencia de una aplicación recubridora $p_0:\mathbb{R}\to \mathbb{S}^1$, por lo que sabemos llevarnos el ``borde'' de $X$ al ``borde'' de $Y$. Si repetimos el proceso subiendo la altura en $X$ y achicando el radio en $Y$ conseguiremos una aplicación recubridora $p:X\to Y$. La idea es buscar una aplicación que en $0$ valga $1$ y que su imagen tienda a $0$ en infinito. Tomamos por tanto $p:X\to Y$ dada por:
            \begin{equation*}
                p(x,y) = e^{-y}p_0(x)
            \end{equation*}
            \begin{figure}[H]
                \centering
                \begin{tikzpicture}
                    \filldraw[fill=blue!40, draw=blue, opacity=0.3] (0,0) rectangle (3,2);
                    \draw[thick] (0,0) -- (3,0);
                    \fill(1.5,1) node[] {$X$};
                    \draw[thick](6,1) circle(1cm);
                    \draw[fill = red!40, draw=red, opacity=0.3](6,1) circle(1cm);
                    \fill(5.5,1) node[] {$Y$};
                    \draw[fill=white] (6,1) circle (2pt);

                    \draw[thick, draw=green] (0,0.3) -- (3,0.3);
                    \draw[thick, draw=green] (6,1) circle(0.8cm);
                    \draw[thick, draw=purple] (0,1.7) -- (3,1.7);
                    \draw[thick, draw=purple] (6,1) circle(0.3cm);
                \end{tikzpicture}
            \end{figure}
            Es claro que $p$ es continua y sobreyectiva. Se comprueba además que $p$ es una aplicación recubridora.

        \item Si $X$ es un espacio topológico (conexo y localmente arcoconexo) con grupo fundamental finito y $p:X\to X$ es una aplicación recubridora, entonces $p$ es un homeomorfismo.

            Es verdadera. Para probar que $p$ es un homeomorfismo basta probar que $p$ es inyectiva, pues al ser una aplicación recubridora tenemos ya que es continua, sobreyectiva y abierta. Para ver que es inyectiva, tomamos dos puntos $x,y\in X$ con $p(x) = p(y)$, como $X$  es conexo y localmente arcoconexo vimos en el Tema 1 que entonces $X$ es arcoconexo, por lo que en particular ha de existir un arco $\alpha:[0,1]\to X$ de forma que $\alpha(0) = x$ y $\alpha(1) = y$. Si consideramos el arco $p\circ \alpha$ tenemos que:
            \begin{equation*}
                (p\circ \alpha)(0) = p(x) = p(y) = (p\circ \alpha)(1)
            \end{equation*}
            por lo que $p\circ \alpha$ es un lazo en $X$. Por otra parte, el Teorema de Monodromía nos dice que el homomorfismo $p_\ast:\pi_1(X,x) \to \pi_1(X,p(x))$ es inyectivo, por lo que será un isomorfismo de grupos, al ser $\pi_1(X,x)$ finito. De esta forma, como $[p\circ \alpha] \in \pi_1(X,p(x))$, ha de existir $\beta\in \Omega(X,x)$ de forma que $p_\ast([\beta]) = [p\circ \alpha]$. En este momento tenemos que tanto $\beta$ como $\alpha$ son dos levantamientos de $p\circ \alpha$ con:
            \begin{equation*}
                \beta(0) = x = \alpha(0)
            \end{equation*}
            por lo que han de ser iguales, de donde:
            \begin{equation*}
                y = \alpha(1) = \beta(1) = x
            \end{equation*}
    \end{enumerate}
\end{ejercicio}

    \newpage
\section{Estimación puntual. Insesgadez y mínima varianza}

\begin{ejercicio}
    Sea $(X_1, \ldots, X_n)$ una muestra de una variable $X\rightsquigarrow\cc{N}(\mu, \sigma^2)$ con $\mu\in \mathbb{R}$, $\sigma\in \mathbb{R}^+$. Probar que
    \begin{equation*}
        T(X_1, \ldots, X_n) = \left\{\begin{array}{ll}
            1 & \text{si\ } \overline{X}\leq 0 \\
            0 & \text{si\ } \overline{X} > 0
        \end{array}\right. 
    \end{equation*}
    es un estimador insesgado de la función paramétrica $\Phi\left(\frac{-\mu \sqrt{n}}{\sigma}\right)$, siendo $\Phi$ la función de distribución de la $\cc{N}(0,1)$.\\

    \noindent
    Tenemos $T(X_1, \ldots, X_n) = I_{\left]-\infty,0\right]}(\overline{X})$. Como $X\rightsquigarrow \cc{N}(\mu, \sigma^2)$, sabemos por lo visto en el Tema 1 que entonces:
    \begin{equation*}
        \overline{X} \rightsquigarrow \cc{N}\left(\mu, \frac{\sigma^2}{n}\right)
    \end{equation*}

    de donde (escribiendo $T=T(X_1,\ldots,X_n)$):
    \begin{equation*}
    T = I_{\left]-\infty,0\right]}(\overline{X}) \rightsquigarrow B(1,P[\overline{X}\leq 0])
    \end{equation*}

    estamos ya en condiciones de ver que $T$ es insesgado para dicha función:
    \begin{equation*}
        E[T] \AstIg P[\overline{X}\leq 0] \stackrel{\text{tipif.}}{=} P\left[Z \leq \dfrac{-\mu\sqrt{n}}{\sigma}\right] = \Phi\left(\dfrac{-\mu\sqrt{n}}{\sigma}\right) 
    \end{equation*}
    donde en $(\ast)$ usamos que conocemos bien la esperanza de una distribución Bernoulli.
\end{ejercicio}

\begin{ejercicio}
    Sea $(X_1, \ldots, X_n)$ una muestra aleatoria simple de $X\rightsquigarrow B(1,p)$ con $p\in \left]0,1\right[$ y sea $T=\sum\limits_{i=1}^{n}X_i$.
    \begin{enumerate}[label=\alph*)]
        \item Probar que si $k\in \mathbb{N}$ y $k\leq n$, el estadístico
            \begin{equation*}
                \dfrac{T(T-1)\cdot \ldots\cdot (T-k+1)}{n(n-1)\cdot \ldots\cdot (n-k+1)}
            \end{equation*}
            es un estimador insesgado de $p^k$. ¿Es este estimador el UMVUE?.
        \item Probar que si $k>n$, no existe ningún estimador insesgado para $p^k$.
        \item ¿Puede afirmarse que $\frac{T}{n}{\left(1-\frac{T}{n}\right)}^{2}$ es insesgado para $p{(1-p)}^{2}$? 
    \end{enumerate}
\end{ejercicio}

\begin{ejercicio}
    Sea $(X_1, \ldots, X_n)$ una muestra aleatoria simple de una variable $X\rightsquigarrow \cc{P}(\lm)$ con $\lm\in \mathbb{R}^+$. Encontrar, si existe, el UMVUE para $\lm^s$, siendo $s\in \mathbb{N}$ arbitrario.\\
    
    \noindent
    Veamos que $T(X_1, \ldots, X_n) = \sum\limits_{i=1}^{n}X_i$ es un estadístico suficiente y completo. Para ello, recordemos que $\{\cc{P}(\lm) : \lm>0\}$ es una familia exponencial:
    \begin{enumerate}
        \item El espacio paramétrico es $\mathbb{R}^+ \subseteq \mathbb{R}$.
        \item El espacio muestral es $\cc{X}=\mathbb{N}\cup \{0\}$, que no depende de $\lm$.
        \item Observamos que:
            \begin{align*}
                P_\lm[X=x] &= e^{-\lm}\dfrac{\lm^x}{x!} = exp\left[\ln\left(e^{-\lm}\dfrac{\lm^x}{x!}\right)\right] = exp\left(-\lm + x\ln\lm -\ln(x!)\right)
            \end{align*}
            por lo que basta tomar:
            \begin{equation*}
                Q(\lm) = \ln\lm, \qquad T(x) = x \qquad D(\lm) = -\lm, \qquad S(x) = -\ln(x!)
            \end{equation*}
    \end{enumerate}
    En consecuencia, por un Teorema visto en teoría, tenemos que el estadístico:
    \begin{equation*}
        T(X_1, \ldots, X_n) = \sum_{i=1}^{n}T(X_i) = \sum_{i=1}^{n}X_i
    \end{equation*}
    es suficiente y completo para $\lm$. Observemos que por la reproductividad de la Poisson tenemos que (notando $T=T(X_1, \ldots, X_n)$): 
    \begin{equation*}
        T\rightsquigarrow \cc{P}\left(\sum_{i=1}^{n}\lm\right) \equiv \cc{P}(n\lm)
    \end{equation*}
    Ahora, para buscar el UMVUE, buscamos una función $h$ medible de forma que:
    \begin{equation*}
        \lm^s = E[h(T)] = \sum_{t\in \mathbb{N}\cup \{0\}}h(t)P[T=t] = \sum_{t\in \mathbb{N}\cup \{0\}} h(t) e^{-n\lm} \dfrac{{(n\lm)}^{t}}{t!} 
    \end{equation*}

    por lo que:
    \begin{equation*}
        \lm^s e^{n\lm} = \sum_{t\in \mathbb{N}\cup \{0\}} h(t)\dfrac{{(n\lm)}^{t}}{t!}
    \end{equation*}

    y si aplicamos el desarrollo en serie de la exponencial, obtenemos:
    \begin{equation*}
        \lm^s \sum_{t\in \mathbb{N}\cup \{0\}} \dfrac{{(n\lm)}^{t}}{t!} = \lm^s e^{n\lm} = \sum_{t\in \mathbb{N}\cup\{0\}} h(t) \dfrac{{(n\lm)}^{t}}{t!} 
    \end{equation*}

    si desarrollamos cada uno de los términos:
    \begin{equation*}
        \lm^s + \lm^{s+1}n + \frac{\lm^{s+2}n^2}{2!} + \ldots = h(0) + h(1)(n\lm) + \ldots + h(s)\dfrac{{(n\lm)}^{s}}{s!} + \ldots
    \end{equation*}

    observamos que tomando:
    \begin{gather*}
        h(0) = \ldots = h(s-1) = 0,\quad  h(s) = \dfrac{s!}{n^s}\\h(s+1) = \dfrac{(s+1)!}{n^s}, \quad \ldots \quad  h(s+k) = \dfrac{(s+k)!}{n^s k!}
    \end{gather*}

    es decir:
    \begin{equation*}
        h(T) = \left\{\begin{array}{ll}
            0 & \text{si\ } T<s \\
            \dfrac{T!}{n^s(T-s)!}& \text{si\ } T\geq s
        \end{array}\right. 
    \end{equation*}
    tenemos que $h(T)$ es insesgado para $\lm^s$. Es claro además que $h(t)\in \mathbb{R}^+$ para cualquier valor de $t$, con lo que $h(T)$ es un estimador de $\lm^s$. Finalmente, observemos que:
    \begin{align*}
        E\left[{(h(T))}^{2}\right] &= \sum_{t\in \mathbb{N}\cup \{0\}} {(h(t))}^{2}P[T=t] = \sum_{t\geq s} {\left(\dfrac{t!}{n^s(t-s)!}\right)}^{2} e^{-n\lm} \dfrac{{(n\lm)}^{t}}{t!} \\ &= \dfrac{1}{n^s e^{n\lm}} \sum_{t\geq s} \dfrac{{(n\lm)}^{t}t!}{{((t-s)!)}^{2}}
    \end{align*}

    como:
    \begin{equation*}
        \dfrac{\dfrac{{(n\lm)}^{t+1}(t+1)!}{{((t+1-s)!)}^{2}}}{\dfrac{{(n\lm)}^{t}t!}{{((t-s)!)}^{2}}} = \dfrac{{(n\lm)}^{t+1}(t+1)!{((t-s)!)}^{2}}{{(n\lm)}^{t}t!{((t+1-s)!)}^{2}} = \dfrac{n\lm(t+1)}{{(t+1-s)}^{2}} \to 0 < 1
    \end{equation*}
    por el Criterio del cociente, tenemos que:
    \begin{equation*}
        \sum_{t\geq s} \dfrac{{(n\lm)}^{t}t!}{{((t-s)!)}^{2}} < \infty \Longrightarrow E\left[{(h(T))}^{2}\right]  =\dfrac{1}{n^s e^{n\lm}}\sum_{t\geq s} \dfrac{{(n\lm)}^{t}t!}{{((t-s)!)}^{2}}   < \infty
    \end{equation*}
    en consecuencia, tenemos que $h(T)$ es un estimador insesgado para $\lm^s$ y de momento de segundo orden finito y es función de un estadístico suficiente y completo, con lo que el Teorema de Lehmann-Scheffé nos dice que:
    \begin{equation*}
        E[h(T)/T] = h(T)
    \end{equation*}
    es un UMVUE para $\lm^s$.
\end{ejercicio}

\begin{ejercicio}
    Sea $(X_1, \ldots, X_n)$ una muestra aleatoria simple de una variable con distribución uniforme discreta en los puntos $\{1,\ldots,N\}$, siendo $N$ un número natural arbitrario. Encontrar el UMVUE para $N$.\\

    \noindent
    En el Ejercicio~\ref{ej:3.5} vimos que $T(X_1, \ldots, X_n) = X_{(n)}$ era un estadístico suficiente y completo. Si notamos $T = T(X_1,\ldots, X_n)$, tenemos que:
    \begin{equation*}
        F_T(t) = {(F_X(t))}^{n} \Longrightarrow P[T=t] = P[T\leq t] - P[T\leq t-1] = {(F_X(t))}^{n}-{(F_X(t-1))}^{n}
    \end{equation*}
    como $F_X(t) = \frac{t}{N}$, tenemos:
    \begin{equation*}
        P[T=t] = {(F_X(t))}^{n}-{(F_X(t-1))}^{n} = \dfrac{t^n - {(t-1)}^{n}}{N^n}
    \end{equation*} % // TODO: TERMINAR
\end{ejercicio}

\begin{ejercicio}
    Sea $(X_1, \ldots, X_n)$ una muestra aleatoria simple de una variable aleatoria $X$ cuya función de densidad es de la forma
    \begin{equation*}
        f_\theta(x) = \dfrac{1}{2\sqrt{x\theta}}, \quad 0<x<\theta
    \end{equation*}
    Calcular, si existe, el UMVUE para $\theta$.
\end{ejercicio}

\begin{ejercicio}
    Sea $(X_1, \ldots, X_n)$ una muestra aleatoria simple de una variable aleatoria $X$ con función de densidad
    \begin{equation*}
        f_\theta(x) = \dfrac{\theta}{x^2}, \quad x>\theta
    \end{equation*}
    Calcular, si existen, los UMVUE para $\theta$ y para $\nicefrac{1}{\theta}$.
\end{ejercicio}

\begin{ejercicio}
    Sea $X\rightsquigarrow P_\theta$ siendo $P_\theta$ una distribución con función de densidad
    \begin{equation*}
        f_\theta(x) = e^{\theta-x}, \quad x\geq \theta
    \end{equation*}
    Dada una muestra aleatoria simple de tamaño arbitrario, encontrar los UMVUE de $\theta$ y de $e^{\theta}$.
\end{ejercicio}

\begin{ejercicio}
    Sea $X$ la variable que describe el número de fracasos antes del primer éxito en una sucesión de pruebas de Bernoulli con probabilidad de éxito $\theta\in \left]0,1\right[$, y sea $(X_1, \ldots, X_n)$ una muestra aleatoria simple de $X$.
    \begin{enumerate}[label=\alph*)]
        \item Probar que la familia de distribuciones de $X$ es regular y calcular la función de infor- mación asociada a la muestra.
        \item Especificar la clase de funciones paramétricas que admiten estimadores eficientes y los correspondientes estimadores.
        \item Calcular la varianza de cada estimador eficiente y comprobar que coincide con las correspondiente cota de Fréchet-Cramér-Rao.
        \item Calcular, si existen, los UMVUE para $P_\theta[X=0]$ y para $E_\theta[X]$ y decir si son eficientes.
    \end{enumerate}
\end{ejercicio}

\begin{ejercicio}
    Sea $(X_1, \ldots, X_n)$ una muestra aleatoria simple de una variable aleatoria $X$ con distribución exponencial.
    \begin{enumerate}[label=\alph*)]
        \item Probar que la familia de distribuciones de $X$ es regular.
        \item Encontrar la clase de funciones paramétricas que admiten estimador eficiente y el estimador correspondiente. Calcular la varianza de estos estimadores.
        \item Basándose en el apartado anterior, encontrar el UMVUE para la media de $X$.
        \item Dar la cota de Fréchet-Cramér-Rao para la varianza de estimadores insesgados y regulares de $\lm^3$. ¿Es alcanzable dicha cota?
    \end{enumerate}
\end{ejercicio}

\begin{ejercicio}
    Sea $X$ una variable aleatoria con función de densidad de la forma
    \begin{equation*}
        f_\theta(x) = \theta x^{\theta-1}, \quad 0<x<1
    \end{equation*}
    \begin{enumerate}[label=\alph*)]
        \item Sabiendo que $E_\theta[\ln X] = -\frac{1}{\theta}$ y $Var_\theta[\ln X] = \frac{1}{\theta^2}$, comprobar que esta familia de distribuciones es regular.
        \item Basándose en una muestra aleatoria simple de $X$, dar la clase de funciones paramétricas con estimador eficiente, los estimadores y su varianza.
    \end{enumerate}
\end{ejercicio}

    \section{Funciones Elementales}

\begin{ejercicio}
    Sea $f : \mathbb{C} \to \mathbb{C}$ una función verificando que
    \[
        f(z + w) = f(z)f(w) \quad \forall z,w \in \mathbb{C}
    \]
    Probar que, si $f$ es derivable en algún punto del plano, entonces $f$ es entera. Encontrar todas las funciones enteras que verifiquen la condición anterior. Dar un ejemplo de una función que verifique dicha condición y no sea entera.
\end{ejercicio}

\begin{ejercicio}
    Calcular la imagen por la función exponencial de una banda horizontal o vertical y del dominio cuya frontera es un rectángulo de lados paralelos a los ejes.
\end{ejercicio}

\begin{ejercicio}
    Dado $\theta\in \left] -\pi, \pi \right]$, estudiar la existencia del límite en $+\infty$ de la función siguiente:
    \Func{\varphi}{\mathbb{R}^+}{\mathbb{C}}{r}{e^{re^{i\theta}}}
\end{ejercicio}

\begin{ejercicio}
    Probar que si $\{z_n\}$ y $\{w_n\}$ son sucesiones de números complejos, con $z_n \neq 0$ para todo $n \in \mathbb{N}$ y $\{z_n\} \to 1$, entonces
    \[
        \left\{w_n(z_n - 1)\right\} \to \lm \in \mathbb{C} \implies \left\{{z_n}^{w_n}\right\} \to e^{\lm}
    \]
\end{ejercicio}

\begin{ejercicio}
    Estudiar la convergencia puntual, absoluta y uniforme de la serie de funciones
    \[
        \sum_{n\geq 0} e^{-nz^2}
    \]
\end{ejercicio}

\begin{ejercicio}
    Probar que si $a,b,c \in \mathbb{T}$ son vértices de un triángulo equilátero si, y sólo si, $a+b+c = 0$.
\end{ejercicio}

\begin{ejercicio}
    Sea $\Omega$ un subconjunto abierto no vacío de $\mathbb{C}^*$ y $\varphi \in \mathcal{C}(\Omega)$ tal que $\varphi(z)^2 = z$ para todo $z \in \Omega$. Probar que $\varphi \in \mathcal{H}(\Omega)$ y calcular su derivada.
\end{ejercicio}

\begin{ejercicio}
    Probar que, para todo $z \in D(0,1)$ se tiene:
    \begin{enumerate}
        \item $\sum\limits_{n= 1}^\infty \dfrac{(-1)^{n+1}}{n}z^n = \log(1+z)$
        \item $\sum\limits_{n= 1}^\infty \dfrac{z^{2n+1}}{n(2n+1)} = 2z - (1+z)\log(1+z) + (1-z)\log(1-z)$
    \end{enumerate}
\end{ejercicio}

\begin{ejercicio}
    Sea la siguiente función:
    \Func{f}{\mathbb{C}\setminus\{1,-1\}}{\mathbb{C}}{z}{\log\left(\frac{1+z}{1-z}\right)}
    Probar que $f$ es holomorfa en el dominio $W = \mathbb{C} \setminus \{x \in \mathbb{R} : |x| \geq 1\}$ y calcular su derivada. Probar también que
    \[
        f(z) = 2\sum_{n=0}^\infty \frac{z^{2n+1}}{2n+1} \quad \forall z \in D(0,1)
    \]
\end{ejercicio}

\begin{ejercicio}
    Sean $\alpha,\beta \in \left[ -\pi, \pi \right]$ con $\alpha < \beta$, y $\rho \in \mathbb{R}^+$ tal que $\rho\alpha,\rho\beta \in \left[ -\pi, \pi \right]$. Consideramos los siguientes dominios:
    \begin{align*}
        \Omega &= \{z \in \mathbb{C}^* : \alpha < \arg z < \beta\} \\
        \Omega_\rho &= \{z \in \mathbb{C}^* : \rho\alpha < \arg z < \rho\beta\}
    \end{align*}
    Probar que la siguiente función define una biyección de $\Omega$ sobre el dominio $\Omega_\rho$:
    \Func{f}{\Omega}{\Omega_\rho}{z}{z^\rho}
\end{ejercicio}

\begin{ejercicio}
    Probar que el seno, el coseno y la tangente son funciones simplemente periódicas.
\end{ejercicio}

\begin{ejercicio}
    Estudiar la convergencia de la serie
    \[
        \sum_{n\geq 0} \frac{\sen(nz)}{2^n}
    \]
\end{ejercicio}

\begin{ejercicio}
    Sea $\Omega = \mathbb{C} \setminus \{x \in \mathbb{R} : |x| \geq 1\}$. Probar que existe $f \in \mathcal{H}(\Omega)$ tal que $\cos f(z) = z$ para todo $z \in \Omega$ y $f(x) = \arccos x$ para todo $x \in \left] -1, 1 \right[$. Calcular la derivada de $f$.
\end{ejercicio}

\begin{ejercicio}
    Para $z \in D(0,1)$ con $\Re z \neq 0$, probar que
    \[
        \arctan\left(\frac{1}{z}\right) + \sum_{n=0}^\infty \frac{(-1)^n}{2n+1}z^{2n+1} = \begin{cases}
            \nicefrac{\pi}{2} & \text{si } \Re z > 0 \\
            \nicefrac{-\pi}{2} & \text{si } \Re z < 0
        \end{cases}
    \]
\end{ejercicio}
    \section{$G-$conjuntos y $p$-grupos}

\begin{ejercicio}\label{ej:6.1}
    Si $X$ es un $G-$conjunto, demostrar que $x^g = \prescript{g^{-1}}{}{x},~ x \in X, g \in G$, define una acción por la derecha de $G$ sobre $X$.\\

    En primer lugar, vemos que se trata de una aplicación de $G \times X$ en $X$. Veamos ahora que cumple las condiciones necesarias para ser una acción por la derecha:
    \begin{itemize}
        \item $x^1 = x$ para todo $x \in X$.
        \begin{equation*}
            x^1 = \prescript{1^{-1}}{}{x} = \prescript{1}{}{x} = x
        \end{equation*}

        \item $(x^g)^h = x^{gh}$ para todo $x \in X$ y $g, h \in G$.
        \begin{equation*}
            (x^g)^h = \prescript{h^{-1}}{}{(x^g)} = \prescript{h^{-1}}{}{(\prescript{g^{-1}}{}{x})} = \prescript{h^{-1}g^{-1}}{}{x} =  \prescript{(gh)^{-1}}{}{x} = x^{gh}
        \end{equation*}
    \end{itemize}

    Por tanto, se trata de una acción por la derecha de $G$ sobre $X$.
\end{ejercicio}

\begin{ejercicio}\label{ej:6.2}
    Sea $G$ un grupo y $N$ un subgrupo normal abeliano de $G$. Demostrar que $G/N$ actúa sobre $N$ por conjugación y obtener entonces un homomorfismo $\varphi: G/N \to \Aut(N)$.\\

    Veamos en primer lugar que $G/N$ actúa sobre $N$ por conjugación. Es decir, que la siguiente aplicación es una acción de $G/N$ sobre $N$:
    \Func{ac}{G/N \times N}{N}{(gN, n)}{\prescript{gN}{}{n} = gng^{-1}}

    Veamos en primer lugar que está bien definida. Sean $g_1, g_2 \in G$ de forma que $g_1N = g_2N$. Entonces $\exists n'\in N$ tal que $g_1 = g_2n'$. Entonces:
    \begin{align*}
        \prescript{g_1N}{}{n} &= g_1ng_1^{-1} = g_2n' n (g_2n')^{-1} = g_2n' n (n')^{-1}g_2^{-1} \AstIg g_2 n'(n')^{-1} n g_2^{-1} = g_2 n g_2^{-1}
        = \prescript{g_2N}{}{n}
    \end{align*}
    donde en $(\ast)$ hemos usado que $N$ es abeliano. Por tanto, la acción está bien definida. Veamos ahora que se trata de una acción.
    \begin{itemize}
        \item $\prescript{1N}{}{n} = 1n1^{-1} = n$ para todo $n \in N$.
        \item Comprobemos la segunda propiedad:
        \begin{align*}
            \prescript{(g_1N)(g_2N)}{}{n} &= \prescript{g_1g_2N}{}{n} = g_1g_2ng_2^{-1}g_1^{-1} = g_1 \left(\prescript{g_2N}{}{n}\right) g_1^{-1}
            = \prescript{g_1N}{}{\left(\prescript{g_2N}{}{n}\right)}.
        \end{align*}
    \end{itemize}

    Buscamos ahora el homomorfismo $\varphi: G/N \to \Aut(N)$. En primer lugar, consideramos el siguiente homomorfismo:
    \Func{\Phi}{G/N}{\Perm(N)}{gN}{\prescript{gN}{}{(\cdot)}=ac(gN, \cdot)}


    Es necesario ver que, fijado $gN \in G/N$, la aplicación siguiente, además de pertenecer a $\Perm(N)$, pertenece a $\Aut(N)$:
    \Func{f}{N}{N}{n}{\prescript{gN}{}{n} = gng^{-1}}

    Sabemos que es biyectiva, por lo que tan solo nos queda probar que es un homomorfismo. Sean $n_1, n_2 \in N$:
    \begin{align*}
        f(n_1n_2) &= \prescript{gN}{}{(n_1n_2)} = g(n_1n_2)g^{-1} = g n_1 g^{-1} g n_2 g^{-1} = f(n_1)f(n_2).
    \end{align*}

    Por tanto, $f$ es un homomorfismo. La aplicación $\varphi$ pedida entonces es:
    \Func{\varphi}{G/N}{\Aut(N)}{gN}{f = \prescript{gN}{}{(\cdot)}}
\end{ejercicio}

\begin{ejercicio}\label{ej:6.3}
    Sean $S$ y $T$ dos $G-$conjuntos. Se define la \emph{acción diagonal} de $G$ sobre el producto cartesiano $S \times T$ mediante $\prescript{x}{}{(s,t)} = (\prescript{x}{}{s},\prescript{x}{}{t})$. Demostrar que, para la acción diagonal, el estabilizador de $(s, t)$ es la intersección de los estabilizadores de $s$ y $t$ en las acciones dadas.\\

    Fijados $s \in S$ y $t \in T$, el estabilizador de $(s,t)$ es:
    \begin{align*}
        \Stab_{G}(s,t) &= \{g \in G \mid \prescript{g}{}{(s,t)} = (s,t)\} = \{g \in G \mid (\prescript{g}{}{s},\prescript{g}{}{t}) = (s,t)\}\\
        &= \{g \in G \mid \prescript{g}{}{s} = s \land \prescript{g}{}{t} = t\} = \{g \in G \mid \prescript{g}{}{s} = s\} \cap \{g \in G \mid \prescript{g}{}{t} = t\}\\
        &= \Stab_{G}(s) \cap \Stab_{G}(t).
    \end{align*}
\end{ejercicio}

\begin{ejercicio}\label{ej:6.4}
    Demostrar que si $G$ contiene un elemento $x$ que tiene exactamente dos conjugados, entonces $G$ tiene un subgrupo normal propio.
    \begin{observacion}
        Considerar el centralizador de $x$.
    \end{observacion}

    Consideramos la acción por conjugación de $G$ sobre sí mismo:
    \Func{ac}{G \times G}{G}{(g,h)}{\prescript{g}{}{h} = ghg^{-1}}

    Calculamos el centralizador de $x$:
    \begin{align*}
        C_G(\{x\}) &= \{g \in G \mid gx = xg\} = \{g \in G \mid gxg^{-1} = x\} = \{g \in G \mid \prescript{g}{}{x} = x\} = \Stab_G(x)
    \end{align*}

    Por tanto, $C_G(\{x\}) = \Stab_G(x)<G$. Veamos ahora que es normal en $G$. Calculemos la órbita de $x$:
    \begin{align*}
        \Orb(x) &= \{y\in G \mid \exists g \in G \text{ tal que } y = \prescript{g}{}{x}\} = \{y\in G \mid \exists g \in G \text{ tal que } y = gxg^{-1}\}
        = \Cl_G(x)
    \end{align*}

    Como $x$ tiene exactamente dos conjugados (él mismo y otro elemento $y\in G$), tenemos que $|\Orb(x)| = 2$. Por tanto:
    \begin{align*}
        [G:C_G(\{x\})] &= |\Orb(x)| = 2 \implies C_G(\{x\})\lhd G
    \end{align*}
    \begin{observacion}
        Notemos que, aun sin saber si $G$ es finito, la igualdad anterior tiene perfecto sentido, puesto que $|\Orb(x)|=2$ y $[G:C_G(\{x\})]$ indica el número de clases en el conjunto cociente, que sabemos que es biyectivo con $\Orb(x)$, luego es $2$.
    \end{observacion}

    Por tanto, $C_G(\{x\})$ es un subgrupo normal de $G$. Tan solo falta por comprobar que es propio.
    \begin{itemize}
        \item Si $C_G(\{x\}) = G$, entonces:
        \begin{equation*}
            2 = |\Orb(x)| = [G:\Stab_G(x)] = [G:C_G(\{x\})] = 1 \implies \text{Contradicción.}
        \end{equation*}

        \item Si $C_G(\{x\}) = \{1\}$, entonces:
        \begin{equation*}
            2 = |\Orb(x)| = [G:\Stab_G(x)] = [G:C_G(\{x\})] = [G:\{1\}] = |G|
        \end{equation*}
        Por tanto, $G=\{1,x\}$. Calculemos el número de conjugados de $1$ y de $x$:
        \begin{align*}
            \Cl_G(1) &= \{g1g^{-1} \mid g \in G\} = \{1\} \\
            \Cl_G(x) &= \{gxg^{-1} \mid g \in G\} = \{1x1, xxx^{-1}\} = \{x\}
        \end{align*}
        Por tanto, ambos tienen un único conjugado. Por tanto, no se puede dar este caso.
    \end{itemize}
    Por tanto, $C_G(\{x\})$ es un subgrupo normal propio de $G$.
\end{ejercicio}

\begin{ejercicio}\label{ej:6.5}
    Encontrar todos los grupos finitos que tienen exactamente dos clases de conjugación.\\

    Sea $G$ un grupo finito con $|G| = n$ que tiene exactamente dos clases de conjugación; a saber, $\exists x_1, x_2 \in G$ tales que $\Cl_G(x_1) \neq \Cl_G(x_2)$. Considerando la acción de $G$ sobre sí mismo por conjugación, tenemos que:
    \begin{equation*}
        \Orb(x) = \Cl_G(x) \qquad \forall x \in G
    \end{equation*}

    Como las órbitas forman una partición de $G$, tenemos que:
    \begin{equation*}
        |G| = |\Orb(x_1)| + |\Orb(x_2)| = |\Cl_G(x_1)| + |\Cl_G(x_2)|
    \end{equation*}

    Calculamos no obstante la clase de conjugación del $1\in G$:
    \begin{align*}
        \Cl_G(1) &= \{g1g^{-1} \mid g \in G\} = \{g g^{-1} \mid g \in G\} = \{1\}
    \end{align*}
    Por tanto, $|\Cl_G(1)| = 1$. Supongamos sin pérdida de generalidad que $1\in \Cl_G(x_1)$. Entonces:
    \begin{equation*}
        n = |\Cl_G(x_1)| + |\Cl_G(x_2)| = 1 + |\Cl_G(x_2)|
        \Longrightarrow |\Cl_G(x_2)| = n - 1
    \end{equation*}

    Por otro lado, como $|\Cl_G(x_2)| = [G:\Stab_G(x_2)]$, tenemos que $|\Cl_G(x_2)|$ divide a $|G|$; es decir, $(n-1) \mid n$. Por tanto, $n=2$, y tenemos por tanto que:
    \begin{equation*}
        G\cong \bb{Z}_2
    \end{equation*}
\end{ejercicio}

\begin{ejercicio}\label{ej:6.6}
    Describir explícitamente las clases de conjugación del grupo $D_4$.\\

    Consideramos el grupo $D_4$:
    \begin{align*}
        D_4 &= \{1, r, r^2, r^3, s, sr, sr^2, sr^3\} \\
        &= \{s^ir^j \mid i = 0, 1, j = 0, 1, 2, 3\}
    \end{align*}

    Tenemos que:
    \begin{align*}
        \Cl_{D_4}(1) &= \{(s^i r^j)1(s^i r^j)^{-1} \mid i = 0, 1, j = 0, 1, 2, 3\} = \{1\} \\
        \Cl_{D_4}(r) &= \{(s^i r^j)r(s^i r^j)^{-1} \mid i = 0, 1, j = 0, 1, 2, 3\} = \{s^ir^j\ r\ r^{-j}s^{-i} \mid i = 0, 1, j = 0, 1, 2, 3\} \\
        &= \{s^i r s^i \mid i = 0, 1\} = \{r, r^3\} = \Cl_{D_4}(r^3) \\
        \Cl_{D_4}(r^2) &= \{(s^i r^j)r^2(s^i r^j)^{-1} \mid i = 0, 1, j = 0, 1, 2, 3\} = \{s^i r^j\ r^2\ r^{-j}s^{-i} \mid i = 0, 1, j = 0, 1, 2, 3\} \\
        &= \{s^i r^2 s^{-i} \mid i = 0, 1\} = \{r^2\}\\
        \Cl_{D_4}(s) &= \{(s^i r^j)s(s^i r^j)^{-1} \mid i = 0, 1, j = 0, 1, 2, 3\}
        = \{s^ir^j\ s\ r^{-j}s^{-i} \mid i = 0, 1, j = 0, 1, 2, 3\} 
    \end{align*}

    Este último no es tan sencillo, puesto que $r$ y $s$ no conmutan. Calculamos en primer lugar para $s=0$, sabiendo que las clases de conjugación son cerradas para inversos.
    \begin{align*}
        r\ s\ r^{-1} &= r\ s\ r^3 = sr^6 = sr^2 \in \Cl_{D_4}(s) \\
        r^2\ s\ r^{-2} &= r^2\ s\ r^2 = sr^6r^2=s\in \Cl_{D_4}(s) \\
        r^3\ s\ r^{-3} &= r^3\ s\ r = sr^9r = sr^2 \in \Cl_{D_4}(s)
    \end{align*}

    Por otro lado, para $s=1$, tenemos que:
    \begin{equation*}
        s\ s\ s = s\qquad \text{ y } s\ sr^2\ s = r^2s = sr^6 = sr^2
    \end{equation*}

    Por tanto, $\Cl_{D_4}(s) = \{s, sr^2\} = \Cl_{D_4}(sr^2)$. Tan solo queda por tanto calcular la clase de conjugación de $sr$ y de $sr^3$.
    \begin{equation*}
        r\ sr\ r^{-1} = r\ sr\ r^3 = rs = sr^3 \in \Cl_{D_4}(sr)
    \end{equation*}

    Por tanto, tenemos que $\Cl_{D_4}(sr) = \Cl_{D_4}(sr^3)$. Como las clases de conjugación forman una partición de $D_4$, tenemos que:
    \begin{align*}
        \Cl_{D_4}(1) &= \{1\} \\
        \Cl_{D_4}(r) &= \{r, r^3\} \\
        \Cl_{D_4}(r^2) &= \{r^2\} \\
        \Cl_{D_4}(s) &= \{s, sr^2\} \\
        \Cl_{D_4}(sr) &= \{sr, sr^3\}
    \end{align*}
\end{ejercicio}

\begin{ejercicio}\label{ej:6.7}
    Se dice que la acción de un grupo finito $G$ sobre un conjunto $X$ es \emph{transitiva} si hay una sola órbita para esta acción (es decir, si para cada $x, y \in X$ existe algún $g \in G$ tal que $\prescript{g}{}{x} = y$). Demostrar que si $G$ actúa transitivamente sobre un conjunto $X$ con $n$ elementos, entonces $|G|$ es un múltiplo de $n$.\\

    Sea $x\in X$. Como las órbitas forman una partición de $X$ y hay una única órbita, tenemos que:
    \begin{equation*}
        n = |X| = |\Orb(x)|
    \end{equation*}

    Como además tenemos que $|\Orb(x)|=[G:\Stab_G(x)]$, tenemos que:
    \begin{equation*}
        |G| = n\cdot |\Stab_G(x)|
    \end{equation*}
    Por tanto, $|G|$ es un múltiplo de $n$.
\end{ejercicio}

\begin{ejercicio}\label{ej:6.8}
    Un subgrupo $G \leq S_n$ se dice \emph{transitivo} si la acción de $G$ sobre $\{1, 2, \ldots, n\}$ es transitiva. Encontrar todos los subgrupos transitivos de $S_3$ y $S_4$.
    \begin{enumerate}
        \item $S_3$.
        
        Consideramos la acción natural de $S_3$ sobre $\{1, 2, 3\}$ dada por:
        \Func{ac}{S_3 \times \{1, 2, 3\}}{\{1, 2, 3\}}{(\sigma, i)}{\prescript{\sigma}{}{i} = \sigma(i)}

        Consideramos ahora la restricción de la acción a $G\leq S_3$, que sigue siendo una acción.
        Buscamos ahora los subgrupos transitivos $G\leq S_3$. En primer lugar, por el Ejercicio anterior, sabemos que $|G|$ es un múltiplo de $3$. Además, como $|S_3| = 6$, tenemos que $|G|$ divide a $6$. Por tanto, $|G|\in \{3,6\}$. Es decir, $G\in \{A_3, S_3\}$. Comprobemos si estos son transitivos.

        Dados $x,y\in \{1, 2, 3\}$ distintos, consideramos el tercer elemento $z\in \{1, 2, 3\}$. Sea ahora $\sigma=(x\ y\ z)\in S_3\cap A_3$. Entonces:
        \begin{align*}
            \prescript{\sigma}{}{x} &= y
        \end{align*}

        Entonces, $S_3$ y $A_3$ son transitivos. Por tanto, los únicos subgrupos transitivos de $S_3$ son $S_3$ y $A_3$.

        \item $S_4$.
        
        Consideramos la acción natural de $S_4$ sobre $\{1, 2, 3, 4\}$ dada por:
        \Func{ac}{S_4 \times \{1, 2, 3, 4\}}{\{1, 2, 3, 4\}}{(\sigma, i)}{\prescript{\sigma}{}{i} = \sigma(i)}

        Consideramos ahora la restricción de la acción a $G\leq S_4$, que sigue siendo una acción.
        Buscamos ahora los subgrupos transitivos $G\leq S_4$. En primer lugar, por el Ejercicio anterior, sabemos que $|G|$ es un múltiplo de $4$. Además, como $|S_4| = 24$, tenemos que $|G|$ divide a $24$. Por tanto, $|G|\in \{4,8,12,24\}$.
        \begin{itemize}
            \item Si $|G|=24$, entonces $G=S_4$. Dados por tanto $x,y\in \{1, 2, 3, 4\}$ distintos, consideramos un tercer elemento $z\in \{1, 2, 3, 4\}\setminus \{x,y\}$. Entonces, tomando $\sigma=(x\ y\ z)\in S_4$:
            \begin{align*}
                \prescript{\sigma}{}{x} &= y
            \end{align*}
            Entonces, $S_4$ es transitivo.

            \item Si $|G|=12$, entonces $G=A_4$. Empleando el mismo razonamiento que en el caso anterior, tenemos que $\sigma\in A_4$ y, por tanto, $\prescript{\sigma}{}{x} = y$. Entonces, $A_4$ es transitivo.
            
            \item Si $|G|=8$, entonces es un $2-$subgrupo de Sylow de $S_4$. Calculemos cuántos $2-$subgrupos de Sylow de $S_4$ hay. Como $|S_4|=24=2^3\cdot 3$, notando por $n_2$ al número de $2-$subgrupos de Sylow de $S_4$, por el Segundo Teorema de Sylow tenemos que:
            \begin{equation*}
                n_2 \equiv 1 \mod 2 \qquad\land \qquad n_2 \mid 3
            \end{equation*}

            Por tanto, puede ser $n_2=1$ o $n_2=3$. Puesto que $S_4$ no contiene subgrupos de orden $8$ normales, tenemos que $n_2=3$. Por tanto, hay tres subgrupos de orden $8$ en $S_4$. Probando, llegamos a que estos son:
            \begin{itemize}
                \item $\langle (1\ 2\ 3\ 4), (1\ 3)\rangle$.
                
                Sea $a=(1\ 2\ 3\ 4)$ y $b=(1\ 3)$. Entonces, tenemos que:
                \begin{align*}
                    ab &= (1\ 2\ 3\ 4)(1\ 3) = (1\ 4)(2\ 3)\\
                    ba^3 &= (1\ 3)(1\ 2\ 3\ 4)^3
                    = (1\ 3)(1\ 4\ 3\ 2) = (1\ 4)(2\ 3)
                \end{align*}
                Por tanto, $ab=ba^3$. Por el Teorema de Dyck, este grupo es isomorfo a $D_4$, luego es de orden $8$. Veamos si es transitivo. Para ello, vemos que:
                \begin{align*}
                    a^0(1) &= 1 \qquad
                    a^1(1) = 2 \qquad
                    a^2(1) = 3 \qquad
                    a^3(1) = 4
                \end{align*}

                Por tanto, $\Orb(1)=\{1, 2, 3, 4\}$. Por tanto, como $\Orb(x)$ es una partición de $\{1, 2, 3, 4\}$, tenemos que la única órbita es $\{1, 2, 3, 4\}$. Por tanto, es transitivo.

                \item $\langle (1\ 3\ 2\ 4), (1\ 2)\rangle$.
                
                Sea $a=(1\ 2\ 3\ 4)$ y $b=(1\ 2)$. Entonces, tenemos que:
                \begin{align*}
                    ab &= (1\ 3\ 2\ 4)(1\ 2) = (1\ 4)(2\ 3)\\
                    ba^3 &= (1\ 2)(1\ 3\ 2\ 4)^3
                    = (1\ 2)(1\ 4\ 2\ 3) = (1\ 4)(2\ 3)
                \end{align*}
                Por tanto, $ab=ba^3$. Por el Teorema de Dyck, este grupo es isomorfo a $D_4$, luego es de orden $8$. Veamos si es transitivo. Para ello, vemos que:
                \begin{align*}
                    a^0(1) &= 1 \qquad
                    a^1(1) = 3 \qquad
                    a^2(1) = 2 \qquad
                    a^3(1) = 4
                \end{align*}

                Por tanto, $\Orb(1)=\{1, 2, 3, 4\}$. Por tanto, como $\Orb(x)$ es una partición de $\{1, 2, 3, 4\}$, tenemos que la única órbita es $\{1, 2, 3, 4\}$. Por tanto, es transitivo.

                \item $\langle (1\ 2\ 3\ 4), (2\ 4)\rangle$.
                
                Sea $a=(1\ 2\ 3\ 4)$ y $b=(2\ 4)$. Entonces, tenemos que:
                \begin{align*}
                    ab &= (1\ 2\ 3\ 4)(2\ 4) = (1\ 2)(3\ 4)\\
                    ba^3 &= (2\ 4)(1\ 2\ 3\ 4)^3
                    = (2\ 4)(1\ 4\ 3\ 2) = (1\ 2)(3\ 4)
                \end{align*}

                Por tanto, $ab=ba^3$. Por el Teorema de Dyck, este grupo es isomorfo a $D_4$, luego es de orden $8$. Veamos si es transitivo. Para ello, vemos que:
                \begin{align*}
                    a^0(1) &= 1 \qquad
                    a^1(1) = 2 \qquad
                    a^2(1) = 3 \qquad
                    a^3(1) = 4
                \end{align*}

                Por tanto, $\Orb(1)=\{1, 2, 3, 4\}$. Por tanto, como $\Orb(x)$ es una partición de $\{1, 2, 3, 4\}$, tenemos que la única órbita es $\{1, 2, 3, 4\}$. Por tanto, es transitivo.
            \end{itemize}

            \item Si $|G|=4$, entonces es cíclico o isomorfo a $V$.
            \begin{itemize}
                \item Sea $G\cong \bb{Z}_4$, y sea $a\in G$ un generador. Entonces, por ser $a$ una biyección en $\{1, 2, 3, 4\}$, tenemos que:
                \begin{equation*}
                    \{a^0(1), a^1(1), a^2(1), a^3(1)\} = \{1,2,3,4\}
                \end{equation*}
                Por tanto, $\Orb(1)=\{1, 2, 3, 4\}$. Por tanto, como $\Orb(x)$ es una partición de $\{1, 2, 3, 4\}$, tenemos que la única órbita es $\{1, 2, 3, 4\}$. Por tanto, es transitivo.

                \item Sea $G\cong V$. Entonces, está formado por la identidad y $3$ elementos de orden $2$ de forma que el producto de dos de ellos es el tercero. Los elementos de orden $2$ son transposiciones o productos de transposiciones. Como es de orden $4$, ha de generarse con dos elementos.
                \begin{itemize}
                    \item Si $G$ está generado por dos productos de transposiciones disjuntas, entonces $G=V$. Veamos sí es transitivo. Sean $i,j\in \{1, 2, 3, 4\}$ distintos. Entonces, sean $k,l\in \{1, 2, 3, 4\}\setminus \{i,j\}$ distintos, de forma que $(i\ j)(k\ l)\in G$. Entonces:
                    \begin{align*}
                        \prescript{(i\ j)(k\ l)}{}{i} &= j
                    \end{align*}
                    Por tanto, como $j$ era arbitrario, tenemos que:
                    \begin{equation*}
                        \Orb(i) = \{1, 2, 3, 4\}
                    \end{equation*}
                    Por tanto, como $\Orb(x)$ es una partición de $\{1, 2, 3, 4\}$, tenemos que la única órbita es $\{1, 2, 3, 4\}$. Por tanto, es transitivo.

                    \item Si $G$ está generado por dos transposiciones no disjuntas, entonces $\exists i,j,k\in \{1, 2, 3, 4\}$ distintos tales que:
                    \begin{equation*}
                        G = \langle (i\ j), (i\ k) \rangle
                    \end{equation*}
                    Entonces:
                    \begin{equation*}
                        (i\ j)(i\ k) = (i\ k\ j)
                    \end{equation*}
                    Por tanto, contendría un elemento de orden $3$, que no puede ser, puesto que $G$ es de orden $4$. Por tanto, no se puede dar este caso.

                    \item Si $G$ está generado por dos transposiciones disjuntas, entonces $\exists i,j,k,l\in \{1, 2, 3, 4\}$ distintos tales que:
                    \begin{equation*}
                        G = \langle (i\ j), (k\ l) \rangle
                    \end{equation*}

                    Entonces, tenemos que:
                    \begin{equation*}
                        G = \langle (i\ j), (k\ l) \rangle = \{1, (i\ j), (k\ l), (i\ j)(k\ l)\}
                    \end{equation*}

                    En este caso, $\Orb(i) = \{1,i,j\}\neq \{1, 2, 3, 4\}$. Por tanto, no es transitivo.

                    \item Si $G$ está generado por una transposición y un producto de transposiciones disjuntas, caben dos casos:
                    \begin{itemize}
                        \item $G=\langle (i\ j), (i\ j)(k\ l) \rangle$, donde $i,j,k,l\in \{1, 2, 3, 4\}$ son distintos. Entonces:
                        \begin{equation*}
                            (i\ j)(i\ j)(k\ l) = (k\ l)
                        \end{equation*}
                        Por tanto, $G=\langle (i\ j), (k\ l) \rangle$, que es un caso ya visto.
                        \item $G=\langle (i\ j), (i\ k)(j\ l) \rangle$, donde $i,j,k,l\in \{1, 2, 3, 4\}$ son distintos. Entonces:
                        \begin{equation*}
                            (i\ j)(i\ k)(j\ l) = (i\ k\ j\ l)
                        \end{equation*}
                        No obstante, este elemento es de orden $4$, por lo que no puede ser, puesto que $G$ sería cíclico. Por tanto, no se puede dar este caso.
                    \end{itemize}
                \end{itemize}
            \end{itemize}
        \end{itemize}
        Como hemos visto, los únicos subgrupos transitivos de $S_4$ son:
        \begin{itemize}
            \item $S_4$.
            \item $A_4$.
            \item Los tres subgrupos de orden $8$ isomorfos a $D_4$:
            \begin{itemize}
                \item $\langle (1\ 2\ 3\ 4), (1\ 3)\rangle$.
                \item $\langle (1\ 3\ 2\ 4), (1\ 2)\rangle$.
                \item $\langle (1\ 2\ 3\ 4), (2\ 4)\rangle$.
            \end{itemize}
            \item Los grupos cíclicos de orden $4$ y $V$.
        \end{itemize}
    \end{enumerate}
\end{ejercicio}

\begin{ejercicio}\label{ej:6.9}
    Sea $n\in \bb{N}$. Una \emph{partición} de $n$ es una sucesión no decreciente de enteros positivos cuya suma es $n$. Dada una permutación $\sigma \in S_n$, la descomposición en ciclos disjuntos (incluyendo los ciclos de longitud 1) de $\sigma = \gamma_1 \gamma_2 \cdots \gamma_r$ determina una partición $n_1, n_2, \ldots, n_r$ de $n$ donde cada $n_i$ es la longitud del ciclo $\gamma_i$. Dos permutaciones en $S_n$ se dice que son del mismo tipo si determinan la misma partición de $n$. Demostrar:
    \begin{enumerate}
        \item Dos elementos de $S_n$ son conjugados si y solo si son del mismo tipo.
        \begin{description}
            \item[$\Longrightarrow$)] Sean $\sigma,\tau\in S_n$ dos elementos conjugados; es decir, $\exists \gamma\in S_n$ tal que $\gamma\sigma\gamma^{-1}=\tau$. Consideramos ahora la descomposición en ciclos disjuntos (incluyendo los de longitud $1$) de $\sigma$:
            \begin{equation*}
                \sigma = \sigma_1\cdots\sigma_r
            \end{equation*}
            Por tanto, tenemos que:
            \begin{equation*}
                \tau=\gamma\sigma\gamma^{-1} = (\gamma\sigma_1\gamma^{-1})\cdots(\gamma\sigma_r\gamma^{-1})
            \end{equation*}

            Además, sabemos que la longitud de $\sigma_i$ coincide con la de $\gamma\sigma_i\gamma^{-1}$ para todo $i\in \{1,\ldots,r\}$. Por tanto, $\tau$ y $\gamma$ determinan la misma partición y por tanto son del mismo tipo.

            \item[$\Longleftarrow$)] Sean $\sigma,\tau\in S_n$ dos elementos del mismo tipo; y sea $n_1,\dots,n_r$ la partición de $n$ que determinan. Consideramos por tanto ambas particiones en ciclos disjuntos (incluyendo los ciclos de longitud uno):
            \begin{align*}
                \sigma &= \sigma_1\cdots\sigma_r
                = (a_{11} \ a_{12} \cdots a_{1n_1})(a_{21} \ a_{22} \cdots a_{2n_2})\cdots(a_{r1} \ a_{r2} \cdots a_{rn_r})\\
                \tau &= \tau_1\cdots\tau_r
                = (b_{11} \ b_{12} \cdots b_{1n_1})(b_{21} \ b_{22} \cdots b_{2n_2})\cdots(b_{r1} \ b_{r2} \cdots b_{rn_r})
            \end{align*}

            Consideramos ahora $\gamma\in S_n$ cuya representación matricial es:
            \begin{equation*}
                \gamma = \begin{pmatrix}
                    a_{11} & a_{12} & \cdots & a_{1n_1} & \cdots & a_{r1} & a_{r2} & \cdots & a_{rn_r}\\
                    b_{11} & b_{12} & \cdots & b_{1n_1} & \cdots & b_{r1} & b_{r2} & \cdots & b_{rn_r}
                \end{pmatrix}
            \end{equation*}

            Entonces, tenemos que:
            \begin{align*}
                \gamma\sigma\gamma^{-1} &= \gamma(\sigma_1\cdots\sigma_r)\gamma^{-1} = \gamma\sigma_1\cdots\gamma\sigma_r\gamma^{-1}\\
                &= (\gamma\sigma_1\gamma^{-1})(\gamma\sigma_2\gamma^{-1})\cdots(\gamma\sigma_r\gamma^{-1}) = \tau_1\tau_2\cdots\tau_r = \tau
            \end{align*}
            Por tanto, $\sigma$ y $\tau$ son conjugados.
        \end{description}

        \item El número de clases de conjugación de $S_n$ es igual al número de particiones de $n$.
        
        \begin{description}
            \item[$\leq$)] Sean $\sigma,\tau\in S_n$ dos elementos tal que $\Cl_{S_n}(\sigma)\neq \Cl_{S_n}(\tau)$. Entonces, no son conjugados. Por tanto, no son del mismo tipo y determinan distintas particiones de $n$. Por tanto, el número de clases de conjugación de $S_n$ es menor o igual al número de particiones de $n$.
            
            \item[$\geq$)] Veamos en primer lugar que, dada una partición de $n$, existe al menos un elemento de $S_n$ que determina dicha partición. Sea $n_1, n_2, \ldots, n_r$ la partición de $n$. Consideramos el siguiente elemento de $S_n$:
            \begin{equation*}
                \sigma = (1\ 2\ \cdots\ n_1)(n_1+1\ n_1+2\ \cdots\ n_1+n_2)\cdots(n_1+\cdots+n_{r-1}+1\ n_1+\cdots+n_{r-1}+2\ \cdots\ n)
            \end{equation*}
            Entonces, $\sigma$ es un elemento de $S_n$ que determina la partición $n_1, n_2, \ldots, n_r$. Por tanto, existe al menos un elemento de $S_n$ que determina cada partición de $n$.

            Ahora, dados dos elementos $\sigma,\tau\in S_n$ que determinan particiones distintas de $n$, tenemos que $\sigma$ y $\tau$ no son del mismo tipo. Por tanto, no son conjugados. Por tanto, $\Cl_{S_n}(\sigma)\neq \Cl_{S_n}(\tau)$. Por tanto, el número de clases de conjugación de $S_n$ es mayor o igual al número de particiones de $n$.
        \end{description}

        Como conclusión, tenemos que el número de clases de conjugación de $S_n$ es igual al número de particiones de $n$.
    \end{enumerate}
\end{ejercicio}

\begin{ejercicio}\label{ej:6.10}
    Calcular el número de clases de conjugación de $S_5$. Dar un representante de cada una y encontrar el orden de cada clase. Calcular el estabilizador de $(1\ 2\ 3)$ bajo la acción de conjugación de $S_5$ sobre sí mismo.\\


    Por el ejercicio anterior, sabemos que hay tantas clases de conjugación como particiones de $n$:
    \begin{equation*}
        \begin{array}{l|c|c}
            \text{Partición} & \text{Representante} & \text{Orden}\\ \hline
            1\ 1\ 1\ 1\ 1 & id_{5} & 1\\
            1\ 1\ 1\ 2 & (1\ 2) & 10\\
            1\ 2\ 2 & (1\ 2)(3\ 4) & 15\\
            1\ 1\ 3 & (1\ 2\ 3) & 20\\
            2\ 3 & (1\ 2)(3\ 4\ 5) & 20\\
            1\ 4 & (1\ 2\ 3\ 4) & 30\\
            5 & (1\ 2\ 3\ 4\ 5) & 24
        \end{array}
    \end{equation*}
    
    Para calcular el orden de cada clase, usamos que:
    \begin{equation*}
        |\Cl_{S_5}(\sigma)| = \dfrac{5!}{\prod\limits_{i=1}^5 m_i! \cdot i^{m_i}}
    \end{equation*}
    donde $m_i$ es el número de ciclos de longitud $i$ en la descomposición en ciclos disjuntos de $\sigma$. Por tanto, tenemos que:
    \begin{align*}
        |\Cl_{S_5}(id_5)| &= \dfrac{5!}{5! \cdot 1^5} = 1\\
        |\Cl_{S_5}((1\ 2))| &= \dfrac{5!}{3!\cdot 1^3 \cdot 1!\cdot 2^1} = \dfrac{120}{6\cdot 2} = 10\\
        |\Cl_{S_5}((1\ 2)(3\ 4))| &= \dfrac{5!}{2!\cdot 2^2} = \dfrac{120}{2\cdot 4} = 15\\
        |\Cl_{S_5}((1\ 2\ 3))| &= \dfrac{5!}{2!\cdot 1^2 \cdot 3^1} = \dfrac{120}{2\cdot 3} = 20\\
        |\Cl_{S_5}((1\ 2)(3\ 4\ 5))| &= \dfrac{5!}{1!\cdot 1^1 \cdot 3^1 \cdot 2^1} = \dfrac{120}{1\cdot 3\cdot 2} = 20\\
        |\Cl_{S_5}((1\ 2\ 3\ 4))| &= \dfrac{5!}{1!\cdot 1^1 \cdot 4^1} = \dfrac{120}{1\cdot 4} = 30\\
        |\Cl_{S_5}((1\ 2\ 3\ 4\ 5))| &= \dfrac{5!}{1!\cdot 5^1} = \dfrac{120}{1\cdot 5} = 24
    \end{align*}


    Calculamos ahora el estabilizador de $(1\ 2\ 3)$:
    \begin{align*}
        \Stab_{S_5}((1\ 2\ 3)) &= \{\gamma\in S_5\mid \gamma(1\ 2\ 3)\gamma^{-1} = (1\ 2\ 3)\}
        =\\&= \{\gamma\in S_5\mid (\gamma(1)\ \gamma(2)\ \gamma(3)) = (1\ 2\ 3) = (2\ 3\ 1) = (3\ 1\ 2)\}
    \end{align*}

    A simple vista, vemos que:
    \begin{equation*}
        id_5,\ (4\ 5),\ (1\ 2\ 3),\ (1\ 3\ 2),\ (1\ 2\ 3)(4\ 5),\ (1\ 3\ 2)(4\ 5)\in \Stab_{S_5}((1\ 2\ 3))
    \end{equation*}

    No obstante, podría haber más. Comprobemos que no:
    \begin{equation*}
        |\Stab_{S_5}((1\ 2\ 3))| = \dfrac{|S_5|}{|\Orb((1\ 2\ 3))|}= \dfrac{|S_5|}{|\Cl_{S_5}((1\ 2\ 3))|}
        = \dfrac{120}{20} = 6
    \end{equation*}

    Por tanto:
    \begin{equation*}
        \Stab_{S_5}((1\ 2\ 3))=\{id_5,\ (4\ 5),\ (1\ 2\ 3),\ (1\ 3\ 2),\ (1\ 2\ 3)(4\ 5),\ (1\ 3\ 2)(4\ 5)\}
    \end{equation*}
\end{ejercicio}

\begin{ejercicio}
    Sea $G$ un grupo finito y $\Phi: G \to \Perm(G)$ la representación regular izquierda (que corresponde a la acción de $G$ sobre sí mismo por traslación por la izquierda).
    \begin{enumerate}
        \item Demostrar que si $x$ es un elemento de $G$ de orden $n$ y $|G| = nm$, entonces $\Phi(x)$ es un producto de $n-$ciclos. Deducir que $\Phi(x)$ es una permutación impar si y solo si el orden de $x$ es par y el cociente del orden de $G$ y el de $x$ es impar.
        
        Sea $x\in G$ con $\ord(x)=n$. Entonces, $\Phi(x)\in \Perm(G)$, y como $|G|=nm$, tenemos que $\Phi(x)\in S_{nm}$. Sea ahora $k\in G$. Con vistas de estudiar la descomposición de $\Phi(x)$ en ciclos disjuntos, veamos las imágenes sucesivas de $k$ bajo $\Phi(x)$:
        \begin{align*}
            k &\mapsto xk \mapsto x^2k \mapsto \cdots \mapsto x^{n-1}k \mapsto x^n k = k
        \end{align*}
        Por tanto, $k$ pertenece al ciclo $(k\ xk\ x^2k\ \cdots\ x^{n-1}k)$ de $\Phi(x)$. Como $k$ era arbitrario, hemos visto que $\Phi(x)$ es producto de $n-$ciclos. Además, todos estos son trivialmente disjuntos.
        \begin{comment}
        hemos visto que $\Phi(x)$ tiene un ciclo de longitud $n$ que contiene a $k$. Por tanto, para cada $k\in G$, $\Phi(x)$ tiene un ciclo de longitud $n$ que contiene a $k$.
        \begin{itemize}
            \item Supongamos que $\Phi(x)$ tiene un ciclo de longitud $j<n$. Entonces, fijado $g\in G$ perteneciente a dicho ciclo, como el orden de un ciclo es su longitud, tenemos que:
            \begin{equation*}
                g = \Phi^j(g) = x^j g\Longrightarrow x^j = 1
                \Longrightarrow n\mid j
                \Longrightarrow n\leq j
            \end{equation*}
            Por tanto, hemos llegado a una contradicción, luego $\Phi(x)$ no tiene ciclos de longitud $j<n$.
            \item Supongamos que $\Phi(x)$ tiene un ciclo de longitud $j>n$. Entonces:
            \begin{equation*}
                g \neq \Phi^n(g) = x^n g\Longrightarrow x^n \neq 1
            \end{equation*}
            Por tanto, hemos llegado a una contradicción, luego $\Phi(x)$ no tiene ciclos de longitud $j>n$.
        \end{itemize}

        Por tanto, la descomposición de $\Phi(x)$ en ciclos disjuntos está formada por $n-$ciclos. Veamos ahora por cuántos $n-$ciclos está formada. Veamos en primer lugar que dos $n-$ciclos distintos han de ser disjuntos. Supongamos que $\Phi(x)$ tiene dos $n-$ciclos $C_1$ y $C_2$ tales que $\exists k\in C_1\cap C_2$. Entonces, por cómo actúa $\Phi(x)$, tenemos que:
        \begin{align*}
            k &\mapsto xk \mapsto x^2k \mapsto \cdots \mapsto x^{n-1}k \mapsto x^n k = k
        \end{align*}
        Por tanto, el único ciclo que contiene a $k$ es $(k\ xk\ x^2k\ \cdots\ x^{n-1}k)$, luego $C_1=C_2$.\\

        Por tanto, $\Phi(x)$ está formada por ciclos de longitud $n$ disjuntos. 
        \end{comment}
        
        Como en la representación de $\Phi(x)$ deben aparecer los $nm$ elementos de $G$, tenemos que el número de ciclos de longitud $n$ en la descomposición de $\Phi(x)$ es:
        \begin{equation*}
            \dfrac{|G|}{n} = \dfrac{nm}{n} = m
        \end{equation*}

        Por tanto, $\Phi(x)$ está formada por $m$ ciclos de longitud $n$ disjuntos. Como una permutación es par si y solo si el número de ciclos de longitud par es par, tenemos que:
        \begin{equation*}
            \veps(\Phi(x)) = -1 \iff \text{el número de ciclos de longitud par es impar}
        \end{equation*}

        Como el $0$ es par, al menos uno de los ciclos de longitud $n$ ha de ser par, luego $n$ ha de ser par. Además, como el número de ciclos de longitud $n$ es $m$, tenemos que $m$ ha de ser impar. Por tanto:
        \begin{equation*}
            \veps(\Phi(x)) = -1 \iff n\text{ es par y }m\text{ es impar}
        \end{equation*}
        Como $m=\dfrac{|G|}{n}$, se tiene lo pedido.

        \item Demostrar que si $Im(\Phi)$ contiene una permutación impar entonces $G$ tiene un subgrupo de índice 2.
        
        Supongamos que $Im(\Phi)$ contiene una permutación impar. Entonces, existe $x\in G$ tal que $\veps(\Phi(x))=-1$. Como tanto $\Phi$ como $\veps$ son homomorfismos, consideramos el homomorfismo composición:
        \begin{equation*}
            \veps\circ\Phi: G \to \{-1, 1\}
        \end{equation*}

        Veamos si ese homomorfismo es sobreyectivo. Como $\veps(\Phi(x))=-1$, tenemos que:
        \begin{equation*}
            (\veps\circ\Phi)(x) = -1\Longrightarrow -1\in Im(\veps\circ\Phi)
        \end{equation*}

        Por otro lado, considerando $1\in G$, tenemos que:
        \begin{equation*}
            (\veps\circ\Phi)(1) = \veps(\Phi(1)) = \veps(id_{G}) = 1
        \end{equation*}

        Por tanto, $Im(\veps\circ\Phi)=\{-1, 1\}$. Por el Primero Teorema de Isomorfía, tenemos que:
        \begin{equation*}
            \dfrac{G}{\ker(\veps\circ\Phi)}\cong Im(\veps\circ\Phi) = \{-1, 1\}
        \end{equation*}

        Como $G$ es finito, tenemos que:
        \begin{equation*}
            [G:\ker(\veps\circ\Phi)] = |Im(\veps\circ\Phi)| = 2
        \end{equation*}

        Por tanto, $\ker(\veps\circ\Phi)$ es un subgrupo de $G$ de índice $2$.       

        \item Demostrar que si $|G| = 2k$ con $k$ impar, entonces $G$ tiene un subgrupo de índice 2.
        \begin{observacion}
            Usar el Teorema de Cauchy para obtener un elemento de orden 2 y entonces usar los dos apartados anteriores.
        \end{observacion}

        Por el Teorema de Cauchy, como $2\mid |G|$, existe $x\in G$ tal que $\ord(x)=2$. Además, como $|G|=2k$ con $k$ impar; y $\ord(x)=2$ par, por el primer apartado, tenemos que $\Phi(x)$ es una permutación impar. Por el segundo apartado, tenemos que $G$ tiene un subgrupo de índice $2$.
    \end{enumerate}
\end{ejercicio}

\begin{ejercicio}\label{ej:6.12}
    Sea $G$ un $p-$grupo actuando sobre un conjunto finito $X$. Demostrar que
    \[
        |X| \equiv |\Fix(X)| \mod p.
    \]

    Sea $G$ un $p-$grupo finito y sea $X$ un conjunto finito sobre el que actúa. Como $G$ es finito, $\exists n\in \bb{N}$ tal que $|G|=p^n$.
    En vistas de aplicar la fórmula de clases, definimos $\Gamma$ como un conjunto que tiene un representante de cada órbita no unitaria de la acción de $G$ sobre $X$. Entonces, tenemos que:
    \begin{equation*}
        |X| = |\Fix(X)| + \sum_{x\in \Gamma} |\Orb(x)|
    \end{equation*}

    Demostrar lo pedido equivale a demostrar que:
    \begin{equation*}
        X-|\Fix(X)| = \sum_{x\in \Gamma} |\Orb(x)| \equiv 0 \mod p
    \end{equation*}

    Fijado $x\in \Gamma$, tenemos que $|\Orb(x)|>1$. Asímismo, como $|\Orb(x)|=[G:\Stab_G(x)]$, tenemos que $|\Orb(x)|\mid |G|=p^n$. Uniendo ambas afirmaciones, tenemos que $|\Orb(x)|=p^{k_x}$ con $k_x\in \{1,\ldots,n\}$.\\

    Por tanto, $|\Orb(x)|\equiv 0 \mod p$ para todo $x\in \Gamma$. Por tanto, la suma de los términos de la suma es congruente a $0$ módulo $p$. Por tanto:
    \begin{equation*}
        |X| - |\Fix(X)| = \sum_{x\in \Gamma} |\Orb(x)| \equiv 0 \mod p
    \end{equation*}
    como queríamos demostrar.
\end{ejercicio}

\begin{ejercicio}
    Sea $G$ un $2-$grupo finito que actúa sobre un conjunto finito $X$ cuya cardinalidad es un número impar. ¿Podemos afirmar que existe al menos un punto de $X$ que queda fijo bajo la acción de $G$? ¿Podemos decir lo mismo si $|X|$ es par?

    Por la fórmula de clases, definiendo $\Gamma$ como un conjunto que tiene un representante de cada órbita no unitaria de la acción de $G$ sobre $X$, tenemos que:
    \begin{equation*}
        |X| = |\Fix(X)| + \sum_{x\in \Gamma} |\Orb(x)|
    \end{equation*}

    Dado $x\in \Gamma$, tenemos que $|\Orb(x)|=[G:\Stab_G(x)]$. Como $G$ es un $2-$grupo, tenemos que $|G|=2^k$ con $k\in \bb{N}$. Por tanto, $|\Orb(x)|$ es una potencia de $2$. Además, como $x\in \Gamma$, tenemos que $|\Orb(x)|>1$. Por tanto, $|\Orb(x)|$ es un número par para todo $x\in \Gamma$. Por tanto, la suma de los términos de la suma es par. Como $|X|$ es impar, tenemos que $|\Fix(X)|$ es impar. Por tanto, existe al menos un punto de $X$ que queda fijo bajo la acción de $G$.\\

    Si $|X|$ es par, entonces $|\Fix(X)|$ es par, pero podría ser $0$. Por tanto, no podemos afirmar que existe al menos un punto de $X$ que queda fijo bajo la acción de $G$.
\end{ejercicio}

\begin{ejercicio}\label{ej:6.14}
    Sea $C_n = \langle a \mid a^n = 1 \rangle$ un grupo cíclico de orden $n$. Describir sus subgrupos de Sylow.\\


    Sea $n=p_1^{k_1}p_2^{k_2}\cdots p_m^{k_m}$ la factorización de $n$ en primos. Entonces, para cada $i\in \{1,\ldots,m\}$, tenemos que el $p_i-$subgrupo de Sylow de $C_n$ es único, puesto que $C_n$ es cíclico luego abeliano, y por tanto sus subgrupos son normales. Por tanto, el $p_i-$subgrupo de Sylow de $C_n$ es el único subgrupo de orden $p_i^{k_i}$ de $C_n$.
    \begin{equation*}
        P_{p_i} = \left\langle a^{\left(\frac{n}{(p_i^{k_i})}\right)} \right\rangle
    \end{equation*}
\end{ejercicio}

\begin{ejercicio}\label{ej:6.15}
    Sea $G$ un grupo finito y $|G| = pn$ con $p$ primo y $p > n$. Demostrar que $G$ contiene un subgrupo normal de orden $p$ y que todo subgrupo de $G$ de orden $p$ es normal en $G$.\\

    Buscamos obtener $n_p$. Como $p>n$, sabemos que $\mcd(p,n)=1$. Por tanto, por el Teorema de Sylow, tenemos que:
    \begin{align*}
        n_p &\equiv 1 \mod p \\
        n_p &\mid n
    \end{align*}

    Por tanto, $n_p\leq n<p$, luego $n_p=1$. Por tanto, existe un único $p-$subgrupo de Sylow de $G$ (de orden $p$), que es normal. Llamémoslo $P_p$, y tendrá orden $|P_p|=p$.\\

    Sea ahora $H$ un subgrupo de $G$ de orden $p$. Entonces, este es un $p-$subgrupo de Sylow de $G$, luego $H=P_p$.
\end{ejercicio}

\begin{ejercicio}\label{ej:6.16}
    Sea $H$ un subgrupo de un grupo finito $G$ con $[G : H] = p$ primo y $p$ el menor primo que divide a $|G|$. Demostrar que entonces $H$ es normal en $G$.\\

    Como no sabemos si $H$ es normal en $G$, no podemos considerar el grupo cociente pero sí el conjunto de las clases laterales por la izquierda de $H$ en $G$, que denotamos por $G/\sim_H$. Consideramos la acción de $G$ sobre $G/\sim_H$ por traslación por la izquierda, y consideramos su representación por permutaciones
    \begin{equation*}
        \Phi: G \to \Perm(G/\sim_H)
    \end{equation*}

    Como $|G/\sim_H| = [G:H] = p$, tenemos que $\Phi$ es un homomorfismo de grupos entre $G$ y $S_p$:
    \begin{equation*}
        \Phi: G \to S_p
    \end{equation*}

    En vistas de aplicar el Primer Teorema de Isomorfía, calculamos el núcleo de $\Phi$:
    \begin{align*}
        \ker(\Phi) &= \{g\in G\mid \Phi(g) = id_{S_p}\}\\
        &= \{g\in G\mid g\cdot (aH) = (aH)\text{ para todo }a\in G\} \subset\\
        &\subset \{g\in G\mid gH = H\} = \{g\in G\mid g\in H\} = H
    \end{align*}

    Por el Primer Teorema de Isomorfía, tenemos que:
    \begin{equation*}
        G/\ker(\Phi) \cong Im(\Phi) \leq S_p
    \end{equation*}

    Por tanto, como $|S_p| = p!$ con $p$ primo, tenemos que $|G/\ker(\Phi)|$ divide a $p!$.
    Por otro lado, $|G/\ker(\Phi)|$ divide a $|G|$. Como $p$ es el menor primo que divide a $|G|$, tenemos que $|G/\ker(\Phi)|=p$.
    Por último, como $\ker(\Phi)\lhd G$ y $\ker(\Phi)\leq H$, tenemos que $\ker(\Phi)\lhd H$. Por tanto:
    \begin{equation*}
        p = [G:\ker(\Phi)] = [G:H]\cdot [H:\ker(\Phi)] = p\cdot [H:\ker(\Phi)]
        \Longrightarrow [H:\ker(\Phi)]=1
    \end{equation*}

    Por tanto, $|\ker(\Phi)|=|H|$. Como $\ker(\Phi)\leq H$, tenemos que $\ker(\Phi)=H$. Por tanto, $\ker(\Phi)=H$ es un subgrupo normal de $G$, concluyendo así la demostración.
\end{ejercicio}

\begin{ejercicio}\label{ej:6.17}
    Sea $p$ un número primo. Demostrar:
    \begin{enumerate}
        \item Todo grupo no abeliano de orden $p^3$ tiene un centro de orden $p$.
        
        Sea $G$ un grupo no abeliano de orden $p^3$. Entonces, $G$ es un $p-$grupo. Por ser $Z(G)<G$, tenemos que $|Z(G)|=p^k$ con $k\in \{0,1,2,3\}$.
        \begin{itemize}
            \item Por ser un $p-$grupo, $Z(G)$ es no trivial, luego $k>0$.
            \item Por no ser abeliano, $Z(G)\neq G$, luego $k<3$.
            \item Por ser un $p-$grupo, $|Z(G)|\neq p^{3-1}$, luego $k<2$.
        \end{itemize}
        Por tanto, $k=1$. Por tanto, $|Z(G)|=p$.
        \item Existen únicamente dos grupos no isomorfos de orden $p^2$.
        
        Todo grupo de orden $p^2$ es abeliano. Consideramos el grupo cíclico $C_{p^2}$ y el grupo directo $C_p\times C_p$. Estos son de orden $p^2$ y no son isomorfos entre sí, puesto que uno es cíclico y el otro no ($\mcd(p,p)=p\neq 1$).

        % // TODO: Hay más?
        \item Todo subgrupo normal de orden $p$ de un $p-$grupo finito está contenido en el centro.
        
        Sea $G$ un $p-$grupo finito y sea $H\lhd G$ un subgrupo normal de orden $p$. Como es de orden $p$, $\exists h\in H$ con $\ord(h)=p$, de forma que:
        \begin{equation*}
            H = \langle h \rangle
        \end{equation*}

        Por tanto, para ver que $H\leq Z(G)$, bastará con ver que $h\in Z(G)$. Fijado $g\in G$, buscamos ver que $gh=hg$. Como $H\lhd G$, tenemos que:
        \begin{equation*}
            ghg^{-1} \in H
            \Longrightarrow
            \exists k\in \{0,\ldots,p-1\}\text{ tal que }ghg^{-1} = h^k
        \end{equation*}

        Si demostramos que $k=1$, lo tendremos.
        \begin{itemize}
            \item Supongamos $k=0$. Entonces, $ghg^{-1}=1$, luego $gh=g$ y $h=1$, luego $\ord(h)=1\neq p$, lo cual es una contradicción.
            \item Supongamos $k>1$. Como $g\in G$ y $G$ es un $p-$grupo, $\exists m\in \bb{N}$ tal que $\ord(g)=p^m$. Entonces, tenemos que:
            % // TODO: Hacer
        \end{itemize}

    \end{enumerate}
\end{ejercicio}

\begin{ejercicio}\label{ej:6.18}
    Demostrar que si $N\lhd G$ y $N$ y $G/N$ son $p-$grupos entonces $G$ es un $p-$grupo.\\

    Para demostrar que $G$ es un $p-$grupo, puesto que $G$ no tiene por qué ser finito, hemos de comprobar que el orden de todo elemento de $G\setminus \{1\}$ es una potencia de $p$. Sea $g\in G\setminus \{1\}$ un elemento cualquiera. Distinguimos dos casos:
    \begin{itemize}
        \item Si $g\in N$, entonces como $N$ es un $p-$grupo y $g\neq 1$, tenemos que $\ord(g)$ es una potencia de $p$.
        \item Si $g\notin N$, entonces como $G/N$ es un $p-$grupo y $gN\neq N$, tenemos que $\ord(gN)$ es una potencia de $p$. Por tanto, existe $k\in \bb{N}$ tal que:
        \begin{equation*}
            (gN)^{p^k} = N
            \Longrightarrow
            g^{p^k}N = N
            \Longrightarrow
            g^{p^k} \in N
        \end{equation*}
        Si $g^{p^k}=1$, entonces $\ord(g)$ divide a $p^k$, y como $g\neq 1$, tenemos que $\ord(g)$ es una potencia de $p$. Si $g^{p^k}\neq 1$, entonces como $N$ es un $p-$grupo, tenemos que $\ord(g^{p^k})$ es una potencia de $p$. Por tanto, $\exists k'\in \bb{N}$ tal que:
        \begin{equation*}
            \left(g^{p^k}\right)^{p^{k'}} = 1
            \Longrightarrow
            g^{p^{k+k'}} = 1
        \end{equation*}
        Por tanto, $\ord(g)$ divide a $p^{k+k'}$, luego $\ord(g)$ es una potencia de $p$.
    \end{itemize}
\end{ejercicio}

\begin{ejercicio}\label{ej:6.19}
    Si $G$ es un grupo de orden $p^n$, $p$ primo, demostrar que para todo $k$, $0 \leq k \leq n$, existe un subgrupo normal de $G$ de orden $p^k$.\\

    Demostramos por inducción sobre $n$.
    \begin{itemize}
        \item Para $n=1$, tenemos que $|G|=p$, luego $G$ es cíclico, luego abeliano, luego todo subgrupo de $G$ es normal.
        \item Supongamos que para todo $p-$grupo de orden $p^m$, con $m<n$, se cumple que para todo $k$, $0\leq k\leq m$, existe un subgrupo normal de orden $p^k$.\\
        
        Sea $G$ un $p-$grupo de orden $p^n$, y consideramos su centro $Z(G)$. Como $Z(G)\neq \{1\}$ y $Z(G)<G$, tenemos que $|Z(G)|=p^k$ con $k\in \{1,\ldots,n\}$. Como $p\mid |Z(G)|$, por el Teorema de Cauchy, existe $N<Z(G)$ tal que $|N|=p$. Como $Z(G)\lhd G$, se tiene que $N\lhd G$, lo que nos permite considerar:
        \begin{equation*}
            |G/N| = \dfrac{|G|}{|N|} = \dfrac{p^n}{p} = p^{n-1}
        \end{equation*}
        Por tanto, $G/N$ es un $p-$grupo de orden $p^{n-1}$. Por la hipótesis de inducción, tenemos que para todo $k$, $0\leq k\leq n-1$, existe $L_k\lhd G/N$ tal que $|L_k|=p^k$. Por el Tercer Teorema de Isomorfía, tenemos que, para cada $k\in \{0,\ldots,n-1\}$, existe $H_k<G$, con $N\lhd H_k\lhd G$, tal que $L_k=H_k/N$. Por tanto, tenemos que:
        \begin{equation*}
            |H_k| = |H_k/N| \cdot |N| = |L_k| \cdot |N| = p^k\cdot p = p^{k+1}
        \end{equation*}

        Por tanto, para cada $k'\in \{1,\ldots,n\}$, existe $W_{k'}=H_{k'-1}$ tal que $|W_{k'}|=p^{k'}$ y $W_{k'}\lhd G$. Falta ver el resultado para $k=0$, pero esto es directo tomando $W_0 = \{1\}$, que es un subgrupo normal de $G$ de orden $1=p^0$.
        Por tanto, hemos visto que para todo $k\in \{0,\ldots,n\}$, existe un subgrupo normal de $G$ de orden $p^k$.
    \end{itemize}
    Por tanto, hemos demostrado que para todo $p-$grupo $G$ de orden $p^n$, con $p$ primo, y para todo $k\in \{0,\ldots,n\}$, existe un subgrupo normal de $G$ de orden $p^k$.
\end{ejercicio}

\begin{ejercicio}\label{ej:6.20}
    Hallar todos los subgrupos de Sylow de los grupos $S_3$ y $S_4$.
    \begin{observacion}
        Para los $2-$subgrupos de Sylow de $S_4$, observar primero que todos deben contener al subgrupo de Klein $V$, y, al menos, una trasposición $\tau$, y que como consecuencia se pueden obtener como producto de $V$ por el grupo cíclico generado por $\tau$.
    \end{observacion}
    \begin{enumerate}
        \item $S_3$.
        
        Sabemos que $|S_3|=6=2\cdot 3$. Calculamos los $p-$subgrupos de Sylow, con $p\in \{2,3\}$.
        \begin{itemize}
            \item $2-$subgrupo de Sylow.
            
            Por el Segundo Teorema de Sylow, tenemos que:
            \begin{equation*}
                n_2 \equiv 1 \mod 2 \qquad n_2 \mid 3
            \end{equation*}
            Por tanto, $n_2\in \{1,3\}$.
        \end{itemize}
        Como hay más de un grupo de orden $2$ en $S_3$, tenemos que $n_2=3$. Estos grupos son:
        \begin{align*}
            \langle (1\ 2) \rangle, \quad \langle (1\ 3) \rangle, \quad \langle (2\ 3) \rangle
        \end{align*}
        \begin{itemize}
            \item $3-$subgrupo de Sylow.
            
            Por el Segundo Teorema de Sylow, tenemos que:
            \begin{equation*}
                n_3 \equiv 1 \mod 3 \qquad n_3 \mid 2
            \end{equation*}
            Por tanto, $n_3=1$. El único $3-$subgrupo de Sylow de $S_3$ es:
            \begin{equation*}
                P_3 = \langle (1\ 2\ 3) \rangle
            \end{equation*}
        \end{itemize}

        \item $S_4$.
        
        Sabemos que $|S_4|=24=2^3\cdot 3$. Calculamos los $p-$subgrupos de Sylow, con $p\in \{2,3\}$.
        \begin{itemize}
            \item $3-$subgrupo de Sylow.
            
            Por el Segundo Teorema de Sylow, tenemos que:
            \begin{equation*}
                n_3 \equiv 1 \mod 3 \qquad n_3 \mid 8
            \end{equation*}
            Por tanto, $n_3\in \{1,4\}$. Como hay más de un grupo de orden $3$ en $S_4$, tenemos que $n_3=4$. Estos grupos son:
            \begin{align*}
                \langle (1\ 2\ 3) \rangle, \quad \langle (1\ 2\ 4) \rangle, \quad \langle (1\ 3\ 4) \rangle, \quad \langle (2\ 3\ 4) \rangle
            \end{align*}

            \item $2-$subgrupo de Sylow.
            
            Por el Segundo Teorema de Sylow, tenemos que:
            \begin{equation*}
                n_2 \equiv 1 \mod 2 \qquad n_2 \mid 3
            \end{equation*}
            Por tanto, $n_2\in \{1,3\}$. En el Ejercicio~\ref{ej:6.8}, vimos que estos grupos son:
            \begin{itemize}
                \item $\langle (1\ 2\ 3\ 4), (1\ 3)\rangle$.
                \item $\langle (1\ 3\ 2\ 4), (1\ 2)\rangle$.
                \item $\langle (1\ 2\ 3\ 4), (2\ 4)\rangle$.
            \end{itemize}
            
        \end{itemize}
    \end{enumerate}
\end{ejercicio}

\begin{ejercicio}\label{ej:6.21}
    Hallar todos los subgrupos de Sylow de los grupos $\bb{Z}_{600}$, $Q_2$, $D_5$, $D_6$, $A_4$, $A_5$, $S_5$.
    \begin{enumerate}
        \item $\bb{Z}_{600}$.
        
        Tenemos $600=2^3\cdot 3\cdot 5^2$. Calculamos los $p-$subgrupos de Sylow, para valores de $p\in \{2,3,5\}$. Como $\bb{Z}_{600}$ es cíclico, en particular es abeliano, y por tanto sus subgrupos son normales, luego son únicos.
        \begin{itemize}
            \item $2-$subgrupo de Sylow.
            
            Es un grupo cíclico de orden $8$, luego es isomorfo a $\bb{Z}_8$. De hecho:
            \begin{equation*}
                P_2 = \langle 3\cdot 5^2 \rangle = \langle 75 \rangle \cong \bb{Z}_8
            \end{equation*}

            \item $3-$subgrupo de Sylow.
            
            Es un grupo cíclico de orden $3$, luego es isomorfo a $\bb{Z}_3$. De hecho:
            \begin{equation*}
                P_3 = \langle 2^3\cdot 5^2 \rangle = \langle 200 \rangle \cong \bb{Z}_3
            \end{equation*}
            \item $5-$subgrupo de Sylow.
            
            Es un grupo cíclico de orden $25$, luego es isomorfo a $\bb{Z}_{25}$. De hecho:
            \begin{equation*}
                P_5 = \langle 2^3\cdot 3 \rangle = \langle 24 \rangle \cong \bb{Z}_{25}
            \end{equation*}
        \end{itemize}

        \item $Q_2$.
        
        Sabemos que $|Q_2|=8=2^3$. Además, como $Q_2\lhd Q_2$, el único $2-$subgrupo de Sylow de $Q_2$ es $Q_2$ mismo. Por tanto, el único subgrupo de Sylow de $Q_2$ es $Q_2$.

        \item $D_5$.
        
        Sabemos que $|D_5|=10=2\cdot 5$. Calculamos los $p-$subgrupos de Sylow, con $p\in \{2,5\}$.
        \begin{itemize}
            \item $2-$subgrupos de Sylow.
            
            Por el Segundo Teorema de Sylow, tenemos que:
            \begin{equation*}
                n_2 \equiv 1 \mod 2 \qquad n_2 \mid 5
            \end{equation*}
            Por tanto, $n_2\in \{1,5\}$. Se tiene que $n_2=5$, puesto que hay $5$ elementos de orden $2$ en $D_5$. Estos grupos son:
            \begin{align*}
                \langle sr^i \rangle \qquad \forall i\in \{0,1,2,3,4\}
            \end{align*}

            \item $5-$subgrupo de Sylow.
            
            Sea $H$ un $5-$subgrupo de Sylow de $D_5$. Como $|D_5|=10$ y $|H|=5$, tenemos que $[D_5:H]=2$, luego $H$ es normal en $D_5$, luego es el único $5-$subgrupo de Sylow de $D_5$. Como además $5$ es primo, $H$ es cíclico. Por tanto:
            \begin{equation*}
                H = \langle r \rangle
            \end{equation*}
        \end{itemize}

        \item $D_6$.
        
        Sabemos que $|D_6|=12=2^2\cdot 3$. Calculamos los $p-$subgrupos de Sylow, con $p\in \{2,3\}$.
        \begin{itemize}
            \item $2-$subgrupos de Sylow.
            
            Por el Segundo Teorema de Sylow, tenemos que:
            \begin{equation*}
                n_2 \equiv 1 \mod 2 \qquad n_2 \mid 3
            \end{equation*}
            Por tanto, $n_2\in \{1,3\}$. Como no hay elementos de orden $4$ en $D_6$, no puede ser cíclico. Por tanto, ha de estar generado por más de un elemento de orden 2:
            \begin{align*}
                \langle r^3,s\rangle &= \{1,r^3,s,sr^3\}\\
                \langle r^3,sr\rangle &= \{1,r^3,sr,sr^4\}\\
                \langle r^3,sr^2\rangle &= \{1,r^3,sr,sr^5\}
            \end{align*}

            Como estos son tres $2-$subgrupos de Sylow de $D_6$, estos son los únicos.


            \item $3-$subgrupos de Sylow.
            
            Por el Segundo Teorema de Sylow, tenemos que:
            \begin{equation*}
                n_3 \equiv 1 \mod 3 \qquad n_3 \mid 4
            \end{equation*}
            Por tanto, $n_3\in \{1,4\}$. Como además los subgrupos son de orden $3$, son cíclicos, luego buscamos elementos de orden $3$ en $D_6$. Todos los elementos de la forma $sr^i$ con $i\in \{0,\dots,5\}$ tienen orden 2. Por tanto, el único $3-$subgrupo es:
            \begin{equation*}
                \langle r^2\rangle
            \end{equation*}
        \end{itemize}

        \item $A_4$.
        
        Sabemos que $|A_4|=12=2^2\cdot 3$. Calculamos los $p-$subgrupos de Sylow, con $p\in \{2,3\}$.
        \begin{itemize}
            \item $2-$subgrupos de Sylow.
            
            Como $V$ es un $2-$subgrupo de Sylow de $A_4$ y $V\lhd A_4$, tenemos que $V$ es el único $2-$subgrupo de Sylow de $A_4$.
            \item $3-$subgrupos de Sylow.
            
            Por el Segundo Teorema de Sylow, tenemos que:
            \begin{equation*}
                n_3 \equiv 1 \mod 3 \qquad n_3 \mid 4
            \end{equation*}
            Por tanto, $n_3\in \{1,4\}$. Como $A_4$ tiene $8$ elementos de orden $3$, tenemos que $n_3=4$. Por tanto, los $3-$subgrupos de Sylow son:
            \begin{align*}
                \langle (1\ 2\ 3) \rangle &= \{1,(1\ 2\ 3),(1\ 3\ 2)\}\\
                \langle (1\ 2\ 4) \rangle &= \{1,(1\ 2\ 4),(1\ 4\ 2)\}\\
                \langle (1\ 3\ 4) \rangle &= \{1,(1\ 3\ 4),(1\ 4\ 3)\}\\
                \langle (2\ 3\ 4) \rangle &= \{1,(2\ 3\ 4),(2\ 4\ 3)\}
            \end{align*}
        \end{itemize}
        \item $A_5$.
        
        Sabemos que $|A_5|=60=2^2\cdot 3\cdot 5$. Calculamos los $p-$subgrupos de Sylow, con $p\in \{2,3,5\}$.
        \begin{itemize}
            \item $2-$subgrupos de Sylow.
            
            Por el Segundo Teorema de Sylow, tenemos que:
            \begin{equation*}
                n_2 \equiv 1 \mod 2 \qquad n_2 \mid 15
            \end{equation*}
            Por tanto, $n_2\in \{1,3,5,15\}$.
            En $A_5$ no hay elementos de orden $4$, y los únicos de orden $2$ son lo productos de transposiciones. Veamos cuántas hay:
            \begin{equation*}
                |\Cl_{A_5}((1\ 2)(3\ 4))| = \frac{5!}{2^2\cdot 2!} = \frac{120}{8} = 15
            \end{equation*}
            
            Por tanto, como mínimo habrá estos $5$ $2-$subgrupos de Sylow:
            \begin{align*}
                \langle (1\ 2)(3\ 4),\ (1\ 3)(2\ 4) \rangle
                &= \{1,(1\ 2)(3\ 4),(1\ 3)(2\ 4),(1\ 4)(2\ 3)\}\\
                \langle (1\ 2)(3\ 5),\ (1\ 3)(2\ 5) \rangle
                &= \{1,(1\ 2)(3\ 5),(1\ 3)(2\ 5),(1\ 5)(2\ 3)\}\\
                \langle (1\ 2)(4\ 5),\ (1\ 4)(2\ 5) \rangle
                &= \{1,(1\ 2)(4\ 5),(1\ 4)(2\ 5),(1\ 5)(2\ 4)\}\\
                \langle (1\ 3)(4\ 5),\ (1\ 4)(3\ 5) \rangle
                &= \{1,(1\ 3)(4\ 5),(1\ 4)(3\ 5),(1\ 5)(2\ 3)\}\\
                \langle (2\ 3)(4\ 5),\ (2\ 4)(3\ 5) \rangle
                &= \{1,(2\ 3)(4\ 5),(2\ 4)(3\ 5),(2\ 5)(3\ 4)\}
            \end{align*}

            Se comprueba que, efectivamente, estos son $5$ $2-$subgrupos de Sylow de $A_5$. Por tanto, $n_2=5$.

            \item $3-$subgrupos de Sylow.
            Por el Segundo Teorema de Sylow, tenemos que:
            \begin{equation*}
                n_3 \equiv 1 \mod 3 \qquad n_3 \mid 20
            \end{equation*}
            Por tanto, $n_3\in \{1,4,10\}$. Veamos cuántos elementos de orden $3$ hay en $A_5$:
            \begin{equation*}
                |\Cl_{A_5}((1\ 2\ 3))| = \frac{5!}{3!\cdot 2} = 20
            \end{equation*}

            Por tanto, hay $10$ $3-$subgrupos de Sylow, que son:
            \begin{align*}
                \langle (1\ 2\ 3) \rangle &= \{1,(1\ 2\ 3),(1\ 3\ 2)\}\\
                \langle (1\ 2\ 4) \rangle &= \{1,(1\ 2\ 4),(1\ 4\ 2)\}\\
                \langle (1\ 2\ 5) \rangle &= \{1,(1\ 2\ 5),(1\ 5\ 2)\}\\
                \langle (1\ 3\ 4) \rangle &= \{1,(1\ 3\ 4),(1\ 4\ 3)\}\\
                \langle (1\ 3\ 5) \rangle &= \{1,(1\ 3\ 5),(1\ 5\ 3)\}\\
                \langle (1\ 4\ 5) \rangle &= \{1,(1\ 4\ 5),(1\ 5\ 4)\}\\
                \langle (2\ 3\ 4) \rangle &= \{1,(2\ 3\ 4),(2\ 4\ 3)\}\\
                \langle (2\ 3\ 5) \rangle &= \{1,(2\ 3\ 5),(2\ 5\ 3)\}\\
                \langle (2\ 4\ 5) \rangle &= \{1,(2\ 4\ 5),(2\ 5\ 4)\}\\
                \langle (3\ 4\ 5) \rangle &= \{1,(3\ 4\ 5),(3\ 5\ 4)\}
            \end{align*}

            \item $5-$subgrupos de Sylow.
            
            Por el Segundo Teorema de Sylow, tenemos que:
            \begin{equation*}
                n_5 \equiv 1 \mod 5 \qquad n_5 \mid 12
            \end{equation*}
            Por tanto, $n_5\in \{1,6\}$. Veamos cuántos elementos de orden $5$ hay en $A_5$:
            \begin{equation*}
                |\Cl_{A_5}((1\ 2\ 3\ 4\ 5))| = \frac{5!}{5} = 24
            \end{equation*}
            Por tanto, hay $6$ $5-$subgrupos de Sylow, que son:
            \begin{align*}
                \langle (1\ 2\ 3\ 4\ 5)\rangle\\
                \langle (1\ 2\ 3\ 5\ 4)\rangle\\
                \langle (1\ 2\ 5\ 3\ 4)\rangle\\
                \langle (1\ 5\ 2\ 3\ 4)\rangle\\
                \langle (1\ 2\ 4\ 3\ 5)\rangle\\
                \langle (1\ 2\ 4\ 5\ 3)\rangle
            \end{align*}


        \end{itemize}


        \item $S_5$.
        
        % // TODO: Hacer
    \end{enumerate}
\end{ejercicio}

\begin{ejercicio}\label{ej:6.22}
    Demostrar que $D_4$ es isomorfo a los $2-$subgrupos de Sylow de $S_4$.
    \begin{observacion}
        Considerar la representación asociada a la acción de $D_4$ sobre los vértices del cuadrado.
    \end{observacion}
    
    
    Este ejercicio ya se resolvió en el Ejercicio~\ref{ej:6.8}, donde vimos que los $2-$subgrupos de Sylow de $S_4$ son isomorfos a $D_4$ aplicando el Teorema de Dyck.
\end{ejercicio}

\begin{ejercicio}\label{ej:6.23}
    Demostrar que todo grupo de orden $12$ con más de un $3-$subgrupo de Sylow es isomorfo al grupo alternado $A_4$.
    \begin{observacion}
        Considerar la acción por traslación de un tal grupo sobre el conjunto de clases módulo $P$, siendo $P$ un $3-$subgrupo de Sylow. Probar que dicha acción es fiel.
    \end{observacion}
    
    Sea $G$ un grupo de orden $12$ con más de un $3-$subgrupo de Sylow. Por el Segundo Teorema de Sylow, tenemos que:
    \begin{equation*}
        n_3 \equiv 1 \mod 3 \qquad n_3 \mid 4
    \end{equation*}
    Por tanto, $n_3\in \{1,4\}$. Como $G$ tiene más de un $3-$subgrupo de Sylow, tenemos que $n_3=4$. Sean por tanto:
    \begin{align*}
        \Syl_3 = \{P_1,P_2,P_3,P_4\}
    \end{align*}

    Como cada $P_i$ es un grupo de orden $3$, es cíclico. De esta forma, supongamos que $\exists x\neq 1$ tal que $x\in P_i\cap P_j$ con $i\neq j$. Entonces, tenemos que:
    \begin{equation*}
        P_i = \langle x \rangle = \{1,x,x^2\} = P_j
    \end{equation*}
    Por tanto, $P_i=P_j$, lo cual es una contradicción. Por tanto, tenemos que:
    \begin{equation*}
        P_1\cap P_2 = P_1\cap P_3 = P_1\cap P_4 = P_2\cap P_3 = P_2\cap P_4 = P_3\cap P_4 = \{1\}
    \end{equation*}
    Por tanto, los $3-$subgrupos de Sylow son disjuntos dos a dos.\\

    Consideramos la acción de $G$ sobre el conjunto de clases módulo $P_1$. Como $P_1$ no es normal en $G$, no podemos considerar el grupo cociente, pero consideramos el conjunto de las clases por la izquierda:
    \begin{equation*}
        G/{\sim}_{P_1} = \{gP_1\mid g\in G\}
    \end{equation*}
    
    Sea por tanto la siguiente acción:
    \Func{ac}{G\times G/{\sim}_{P_1}}{G/{\sim}_{P_1}}{(g,hP_1)}{(gh)P_1}

    Veamos que está bien definida. Sea $g\in G$ y $h_1P_1,h_2P_1\in G/{\sim}_{P_1}$ tales que $h_1P_1=h_2P_1$. Entonces:
    \begin{equation*}
        (gh_1)P_1 = (gh_2)P_1
        \iff (gh_1)^{-1}(gh_2) \in P_1
        \iff h_1^{-1}h_2 \in P_1
        \iff h_1P_1 = h_2P_1
    \end{equation*}

    Por tanto, la acción está bien definida. Veamos que efectivamente es una acción:
    \begin{align*}
        \prescript{1}{}{hP_1} &= (1h)P_1 = hP_1\qquad \forall hP_1\in G/{\sim}_{P_1}\\
        \prescript{g_1}{}{\left(\prescript{g_2}{}{hP_1}\right)} &= \prescript{g_1}{}{(g_2h)P_1} = (g_1g_2h)P_1 = \prescript{g_1g_2}{}{hP_1}\qquad \forall g_1,g_2\in G,\ hP_1\in G/{\sim}_{P_1}
    \end{align*}

    Por tanto, consideramos su representación por permutaciones asociada a la acción:
    \Func{\Phi}{G}{\Perm(G/{\sim}_{P_1})}{g}{\prescript{g}{}{\left(\cdot P_1\right)}}


    Veamos el cardinal del conjunto de clases:
    \begin{equation*}
        |G/{\sim}_{P_1}| = [G:P_1] = \frac{|G|}{|P_1|} = \frac{12}{3} = 4
    \end{equation*}

    Por tanto, tenemos que:
    \Func{\Phi}{G}{S_4}{g}{\left(g(\cdot)\right)P_1}

    Calculamos que se trata de una acción fiel:
    \begin{align*}
        \ker(\Phi) &= \{g\in G\mid \prescript{g}{}{\left(\cdot P_1\right)} = Id_{G/{\sim}_{P_1}}\}\\
        &= \{g\in G\mid \prescript{g}{}{\left(hP_1\right)} = hP_1\ \forall hP_1\in G/{\sim}_{P_1}\}\\
        &= \{g\in G\mid (gh)P_1 = hP_1\ \forall hP_1\in G/{\sim}_{P_1}\}\\
        &= \{g\in G\mid h^{-1}gh \in P_1\ \forall h\in G\}\\
        &= \{g\in G\mid g\in hP_1h^{-1}\ \forall h\in G\}\\
    \end{align*}

    Por el Segundo Teorema de Sylow, todos los $3-$subgrupos de Sylow son conjugados entre sí, luego $hP_1h^{-1}$ es un $3-$subgrupo de Sylow de $G$ para todo $h\in G$. Por tanto, tenemos que:
    \begin{align*}
        \ker(\Phi) &\subset \bigcap_{h\in G} hP_1h^{-1}
        \subset \bigcap_{i=1}^4 P_i = \{1\}
    \end{align*}
    Por tanto, $\ker(\Phi)=\{1\}$, luego la acción es fiel.\\

    Por el Primer Teorema de Isomorfía, tenemos que:
    \begin{equation*}
        G\cong G/\{1\} = G/{\ker(\Phi)} \cong \Im(\Phi) \subset S_4
    \end{equation*}

    Por tanto, hemos visto que $G$ es isomorfo a un subgrupo de $S_4$. Como $|G|=12$ y el único subgrupo de orden $12$ de $S_4$ es $A_4$, tenemos que:
    \begin{equation*}
        G\cong A_4
    \end{equation*}
    Por tanto, hemos demostrado que todo grupo de orden $12$ con más de un $3-$subgrupo de Sylow es isomorfo al grupo alternado $A_4$.
\end{ejercicio}

\begin{ejercicio}\label{ej:6.24}~
    \begin{enumerate}
        \item Demostrar que no existen grupos simples de orden $12$. Más concretamente, demostrar que todo grupo de orden $12$ admite un subgrupo normal de orden $3$ o de orden $4$.
        
        Sea $G$ un grupo de orden $12=3\cdot 2^2$. Por el Segundo Teorema de Sylow, tenemos que:
        \begin{equation*}
            n_3 \mid 4 \qquad n_3 \equiv 1 \mod 3
        \end{equation*}

        Por tanto, $n_3\in \{1,4\}$.
        \begin{itemize}
            \item Si $n_3=1$, entonces el $3-$subgrupo de Sylow es normal con cardinal $3$ (luego no es propio), por lo que $G$ no es simple.
            
            \item Si $n_3=4$, aplicamos de nuevo el Segundo Teorema de Sylow:
            \begin{equation*}
                n_2 \mid 3 \qquad n_2 \equiv 1 \mod 2
            \end{equation*}
            Por tanto, $n_2\in \{1,3\}$.
            \begin{itemize}
                \item Si $n_2=1$, entonces el $2-$subgrupo de Sylow es normal con cardinal $4$ (luego no es propio), por lo que $G$ no es simple.
                \item Si $n_2=3$, $n_3=4$.

                Estudiamos la situación.
                \begin{itemize}
                    \item Como $n_3=4$, tenemos $4$ $3-$subgrupos de Sylow (todos ellos disjuntos), por lo que tenemos $4\cdot 2=8$ elementos de orden $3$.
                    \item Como $n_2=3$, tenemos $3$ $2-$subgrupos de Sylow, pero no podemos garantizar que sean disjuntos. Fijado $P\in \Syl_2(G)$, este tendrá $3$ elementos de orden $2$ o $4$. Además, puesto que los $2-$subgrupos de Sylow son distintos, al menos habrá otro elemento de orden $2$ o $4$ distinto. Por tanto, tenemos al menos $4$ elementos de orden $2$ o $4$.
                \end{itemize}
                
                Por tanto, tenemos:
                \begin{itemize}
                    \item $1$ elemento de orden $1$.
                    \item $8$ elementos de orden $3$.
                    \item Al menos $4$ elementos de orden $2$ o $4$.
                \end{itemize}
                Esto implica que el grupo tiene al menos $13$ elementos, lo cual es una contradicción. Por tanto, este caso no puede darse.
            \end{itemize}
        \end{itemize}

        Por tanto, hemos visto que $n_3=1$ (en cuyo caso $G$ tiene un subgrupo normal de orden $3$) o $n_2=1$ (en cuyo caso $G$ tiene un subgrupo normal de orden $4$). Por tanto, todo grupo de orden $12$ admite un subgrupo normal de orden $3$ o de orden $4$.


        \item Demostrar que no existen grupos simples de orden $28$. Más concretamente, probar que todo grupo de orden $28$ contiene un subgrupo normal de orden $7$.
        
        Sea $G$ un grupo de orden $28=7\cdot 2^2$. Por el Segundo Teorema de Sylow, tenemos que:
        \begin{equation*}
            n_7 \mid 4 \qquad n_7 \equiv 1 \mod 7
        \end{equation*}
        Por tanto, $n_7=1$. Por tanto el $7-$subgrupo de Sylow es normal con cardinal $7$ (luego no es propio), por lo que $G$ no es simple.
        \item Demostrar que no existen grupos simples de orden $56$. Más concretamente, probar que todo grupo de orden $56$ contiene un subgrupo normal de orden $7$ o de orden $8$.
        
        Sea $G$ un grupo de orden $56=7\cdot 2^3$. Por el Segundo Teorema de Sylow, tenemos que:
        \begin{equation*}
            n_7 \mid 8 \qquad n_7 \equiv 1 \mod 7
        \end{equation*}
        Por tanto, $n_7\in \{1,8\}$.
        \begin{itemize}
            \item Si $n_7=1$, entonces el $7-$subgrupo de Sylow es normal con cardinal $7$ (luego no es propio), por lo que $G$ no es simple.
            \item Si $n_7=8$, aplicamos de nuevo el Segundo Teorema de Sylow:
            \begin{equation*}
                n_2 \mid 7 \qquad n_2 \equiv 1 \mod 2
            \end{equation*}
            Por tanto, $n_2\in \{1,7\}$.
            \begin{itemize}
                \item Si $n_2=1$, entonces el $2-$subgrupo de Sylow es normal con cardinal $8$ (luego no es propio), por lo que $G$ no es simple.
                \item Si $n_2=7$, $n_7=8$.
                \begin{itemize}
                    \item Como $n_7=8$, hay $8$ $7-$subgrupos de Sylow (todos ellos distintos), por lo que tenemos $8\cdot 6=48$ elementos de orden $7$. Como $n_2=7$, tenemos $7$ $2-$subgrupos de Sylow, pero no podemos garantizar que sean disjuntos. Fijado $P\in \Syl_2(G)$, este contendrá $7$ elementos de orden $2$, $4$ o $8$. Además, puesto que  los $2-$subgrupos de Sylow son distintos, al menos habrá otro elemento de orden $2$, $4$ o $8$ distinto. Por tanto, tenemos al menos $8$ elementos de orden $2$, $4$ o $8$. 
                \end{itemize}
                
                Por tanto, tenemos:
                \begin{itemize}
                    \item $1$ elemento de orden $1$.
                    \item $48$ elementos de orden $7$.
                    \item $8$ elementos de orden $2$, $4$ o $8$.
                \end{itemize}
                Esto implica que el grupo tiene al menos $57$ elementos, lo cual es una contradicción. Por tanto, este caso no puede darse.
            \end{itemize}
        \end{itemize}

        Por tanto, hemos visto que $n_7=1$ (en cuyo caso $G$ tiene un subgrupo normal de orden $7$) o $n_2=1$ (en cuyo caso $G$ tiene un subgrupo normal de orden $8$).
        \item Demostrar que no existen grupos simples de orden $148$.
        
        Sea $G$ un grupo de orden $148=37\cdot 2^2$. Por el Segundo Teorema de Sylow, tenemos que:
        \begin{equation*}
            n_{37} \mid 4 \qquad n_{37} \equiv 1 \mod 37
        \end{equation*}
        Por tanto, $n_{37}=1$. Por tanto el $37-$subgrupo de Sylow es normal con cardinal $37$ (luego no es propio), por lo que $G$ no es simple.
        \item Demostrar que no existen grupos simples de orden $200$.
        Sea $G$ un grupo de orden $200=5^2\cdot 2^3$. Por el Segundo Teorema de Sylow, tenemos que:
        \begin{equation*}
            n_5 \mid 8 \qquad n_5 \equiv 1 \mod 5
        \end{equation*}
        Por tanto, $n_5=1$. Por tanto el $5-$subgrupo de Sylow es normal con cardinal $5^2$ (luego no es propio), por lo que $G$ no es simple.
        \item Demostrar que no existen grupos simples de orden $351$.
        Sea $G$ un grupo de orden $351=3^3\cdot 13$. Por el Segundo Teorema de Sylow, tenemos que:
        \begin{equation*}
            n_{13} \mid 27 \qquad n_{13} \equiv 1 \mod 13
        \end{equation*}
        Por tanto, $n_{13}\in \{1,27\}$.
        \begin{itemize}
            \item Si $n_{13}=1$, entonces el $13-$subgrupo de Sylow es normal con cardinal $13$ (luego no es propio), por lo que $G$ no es simple.
            \item Si $n_{13}=27$, aplicamos de nuevo el Segundo Teorema de Sylow:
            \begin{equation*}
                n_3 \mid 13 \qquad n_3 \equiv 1 \mod 3
            \end{equation*}
            Por tanto, $n_3\in \{1,13\}$.
            \begin{itemize}
                \item Si $n_3=1$, entonces el $3-$subgrupo de Sylow es normal con cardinal $27$ (luego no es propio), por lo que $G$ no es simple.
                \item Si $n_3=13$, $n_{13}=27$.
                \begin{itemize}
                    \item Como $n_{13}=27$, tenemos $27$ $13-$subgrupos de Sylow (todos ellos distintos), por lo que tenemos $27\cdot 12=324$ elementos de orden $13$.
                    \item Como $n_3=13$, tenemos $13$ $3-$subgrupos de Sylow, pero no podemos garantizar que sean disjuntos. Fijado $P\in \Syl_3(G)$, este contendrá $26$ elementos de orden $3$, $9$ o $27$. Además, puesto que los $3-$subgrupos de Sylow son distintos, al menos habrá otro elemento de orden $3$, $9$ o $27$ distinto. Por tanto, tenemos al menos $27$ elementos de orden $3$, $9$ o $27$.
                \end{itemize}
                Por tanto, tenemos:
                \begin{itemize}
                    \item $1$ elemento de orden $1$.
                    \item $324$ elementos de orden $13$.
                    \item $27$ elementos de orden $3,9$ o $27$.
                \end{itemize}
                Esto implica que el grupo tiene más de $351$ elementos, lo cual es una contradicción. Por tanto, este caso no puede darse.
            \end{itemize}
        \end{itemize}

        Por tanto, hemos visto que $n_{13}=1$ (en cuyo caso $G$ tiene un subgrupo normal de orden $13$) o $n_3=1$ (en cuyo caso $G$ tiene un subgrupo normal de orden $27$).
    \end{enumerate}
\end{ejercicio}

\begin{ejercicio}\label{ej:6.25}
    Calcular el número de elementos de orden $7$ que tiene un grupo simple de orden $168$.\\


    Sabemos que $168=2^3\cdot 3\cdot 7$. Como cada elemento de orden $7$ va a generar un grupo cíclico de orden $7$, buscamos el número de subgrupos de Sylow de orden $7$. Por la descomposición de $168$, sabemos que dichos grupos serán $7-$subgrupos de Sylow. Por el Segundo Teorema de Sylow, tenemos que:
    \begin{equation*}
        n_7\equiv 1\mod 7\qquad n_7\mid 24
    \end{equation*}
    Por tanto, $n_7\in \{1, 8\}$.
    \begin{itemize}
        \item Si $n_7=1$, entonces el $7-$subgrupo de Sylow es normal con cardinal $7$ (luego no es propio), por lo que $G$ no es simple.
    \end{itemize}

    Por tanto, $n_7=8$. Por tanto, hay exactamente $8$ subgrupos de orden $7$. Cada uno de los elementos de orden $7$ del grupo pertenece a un único subgrupo de orden $7$, y será un generador de estos. Además, sabemos que el número de elementos de orden $7$ de un grupo cíclico de orden $7$ viene dado por la función $\varphi(7) = 6$. Por tanto, el número de elementos de orden $7$ de un grupo simple de orden $168$ es:
    \begin{equation*}
        8\cdot 6 = 48
    \end{equation*}
\end{ejercicio}

    \section{Cuestionarios} 
    \subsection{Cuestionario I}
\begin{ejercicio}
    Si $A$ es un conjunto finito arbitrario, la afirmación ``$|P(A)| > |A|$'' es:
    \begin{itemize}
        \item Siempre verdadera.
        \item Verdadera o falsa, depende de $A$.
        \item Siempre falsa.
    \end{itemize}
\end{ejercicio}

\begin{ejercicio}
    Si $A$, $B$, $C$ son conjuntos cualesquira con $B$ y $C$ disjuntos, selecciona la afirmación verdadera:
    \begin{itemize}
        \item $(A \cup B)\cap C = A$.
        \item $(A \cup B)\cap (A \cup C)=A$.
        \item $(A\cap B)\cup(A \cap C)=A$.
    \end{itemize}
\end{ejercicio}

\begin{ejercicio}
    Si $A$ y $B$ son subconjuntos de un conjunto, la afirmación \newline ``$c(A) \cap c(B) = c(A \cap B)$'' es:
    \begin{itemize}
        \item Siempre cierta.
        \item Siempre falsa.
        \item A veces verdadera y a veces falsa, depende de $A$ y $B$.
    \end{itemize}
\end{ejercicio}

\begin{ejercicio}
    Sean $P$ y $Q$ las propiedades referidas a los elementos de un conjunto. Las proposiciones $P \Rightarrow \neg Q$ y $Q \Rightarrow \neg P$ son:
    \begin{itemize}
        \item Siempre equivalentes.
        \item Nunca equivalentes.
        \item A veces equivalentes y a veces no, depende de $P$ y de $Q$.
    \end{itemize}
\end{ejercicio}

\begin{ejercicio}
    Sean $P$, $Q$ y $R$ propiedades referidas a los elementos de un conjunto tal que $P \Rightarrow Q \lor R$, entonces (seleccionar la afirmación correcta):
    \begin{itemize}
        \item $P \Rightarrow Q$ y $P \Rightarrow R$.
        \item $P \Rightarrow Q$ o $P \Rightarrow R$.
        \item $P \Rightarrow Q$ siempre que $R \Rightarrow Q$.
    \end{itemize}
\end{ejercicio}

\newpage
\ % --------------------------------------------------------------------------------
\resetearcontador

\begin{ejercicio}
    Si $A$ es un conjunto finito arbitrario, la afirmación ``$|P(A)| > |A|$'' es:
    \begin{itemize}
        \item \underline{Siempre verdadera.}
        \item Verdadera o falsa, depende de $A$.
        \item Siempre falsa.
    \end{itemize}

    \noindent
    \textbf{Justificación}:
    Si $A = \emptyset$, entonces $P(A) = \{\emptyset\}$ y $|P(A)|=1>0=|A|$.\newline
    Si $A \neq \emptyset$, entonces $P(A)$ contiene a todos los subconjuntos unitarios $\{a\}$, con $a \in A$ (luego, el cardinal de $P(A)$ es, como mínimo, igual al de $|A|$) y, además, contiene el subconjunto vacío, luego tiene al menos tantos elementos como $A$ más uno.\\

    \noindent
    Otra alternativa es usar la fórmula vista para el cardinal del conjunto potencia de un conjunto finito vista en teoría:\newline
    Sea $A$ un conjunto finito arbitrario con $|A| = n \in \bb{N}$, entonces $|\mathcal{P}(A)| = 2^n$.\newline
    Notemos que $2^n > n\quad\forall n \in \bb{N}$.
\end{ejercicio}

\begin{ejercicio}
    Si $A$, $B$, $C$ son conjuntos cualesquira con $B$ y $C$ disjuntos, selecciona la afirmación verdadera:
    \begin{itemize}
        \item $(A \cup B)\cap C = A$.
        \item \underline{$(A \cup B)\cap (A \cup C)=A$.}
        \item $(A\cap B)\cup(A \cap C)=A$.
    \end{itemize}

    \noindent
    \textbf{Justificación}:
    \begin{equation*}
        (A \cup B) \cap (A \cup C) = A \cup (B \cap C) = A \cup \emptyset = A    
    \end{equation*}
\end{ejercicio}

\begin{ejercicio}
    Si $A$ y $B$ son subconjuntos de un conjunto, la afirmación \newline ``$c(A) \cap c(B) = c(A \cap B)$'' es:
    \begin{itemize}
        \item Siempre cierta.
        \item Siempre falsa.
        \item \underline{A veces verdadera y a veces falsa, depende de $A$ y $B$.}
    \end{itemize}

    \noindent
    \textbf{Justificación}:
    Por las Leyes de Morgan: $c(A \cap B) = c(A) \cup c(B)$, por lo que podemos intuir que la afirmación no siempre es cierta. Podemos dar un contraejemplo para ilustrarlo:\newline
    Sea $X = \{1,2,3,4,5\}$, sean $A = \{1,2,3\}$, $B = \{4,5\} \subseteq X$:
    \begin{gather*}
        c(A) = B\qquad c(B) = A
        c(A \cap B) = c(\emptyset) = X \neq c(A) \cap c(B) = \emptyset
    \end{gather*}
    Además, como no impone nada sobre los conjuntos, podemos ver que si $A = B$, es cierta la afirmación. Supongamos que $A = B$:
    \begin{equation*}
        c(A \cap B) = c(A \cap A) = c(A) = c(A) \cup c(A) = c(A) \cup c(B)
    \end{equation*}
\end{ejercicio}

\newpage
\begin{ejercicio}
    Sean $P$ y $Q$ las propiedades referidas a los elementos de un conjunto. Las proposiciones $P \Rightarrow \neg Q$ y $Q \Rightarrow \neg P$ son:
    \begin{itemize}
        \item \underline{Siempre equivalentes.}
        \item Nunca equivalentes.
        \item A veces equivalentes y a veces no, depende de $P$ y de $Q$.
    \end{itemize}

    \noindent
    \textbf{Justificación}:
    $Q \Rightarrow \neg P$ es el contrarrecíproco de $P \Rightarrow \neg Q$.\newline
    Demostremos que $(Q \Rightarrow \neg P) \Leftrightarrow (P \Rightarrow \neg Q)$:\newline
    O, equivalentemente, que $X_Q \subseteq c(X_P) \Leftrightarrow X_P \subseteq c(X_Q)$.
    \begin{description}
        \item [$\Rightarrow)$]
            Sea $ x \in X_P \Rightarrow x \notin c(X_P) \Rightarrow x \notin X_Q \Rightarrow x \in c(X_Q)$\newline
            Para todo $x \in X_P$, luego $X_P \subseteq c(X_Q)$.
        \item [$\Leftarrow)$]
            Sea $ x \in X_Q \Rightarrow x \notin c(X_Q) \Rightarrow x \notin X_P \Rightarrow x \in c(X_P)$\newline
            Para todo $x \in X_Q$, luego $X_Q \subseteq c(X_P)$.
    \end{description}
\end{ejercicio}

\begin{ejercicio}
    Sean $P$, $Q$ y $R$ propiedades referidas a los elementos de un conjunto tal que $P \Rightarrow Q \lor R$, entonces (seleccionar la afirmación correcta):
    \begin{itemize}
        \item $P \Rightarrow Q$ y $P \Rightarrow R$.
        \item $P \Rightarrow Q$ o $P \Rightarrow R$.
        \item \underline{$P \Rightarrow Q$ siempre que $R \Rightarrow Q$.}
    \end{itemize}

    \noindent
    \textbf{Justificación}:
    Por hipótesis, $X_P \subseteq X_Q \cup X_R$.\newline
    Si $X_R \subseteq X_Q \Rightarrow X_P \subseteq X_Q = X_Q \cup X_R$.

\end{ejercicio}

\newpage
\resetearcontador

    \section{Cuestionario II}
\begin{ejercicio}
    Sean $X$ e $Y$ dos conjuntos finitos con $|X| = |Y|$ y $f:X \rightarrow Y$ una aplicación. La afirmación ``Si $f$ es inyectiva o sobreyectiva, entonces $f$ es biyectiva'' es:
    \begin{itemize}
        \item Verdadera o falsa, depende de $f$.
        \item Siempre verdadera.
        \item Siempre falsa.
    \end{itemize}
\end{ejercicio}

\begin{ejercicio}
    Sea $f:X \rightarrow Y$ una aplicación inyectiva y sean $A, B \subseteq X$. Selecciona la afirmación verdadera:
    \begin{itemize}
        \item $f_{*}(A) - f_{*}(B)$ es un subconjunto propio de $f_{*}(A-B)$.
        \item $f_{*}(A-B)$ es un subconjunto propio de $f_{*}(A) - f_{*}(B)$.
        \item $f_{*}(A-B) = f_{*}(A) - f_{*}(B)$.
    \end{itemize}
\end{ejercicio}

\begin{ejercicio}
    Sea $f:X \rightarrow X$ una aplicación tal que $f_{*}(c(A)) = c(f_{*}(A))$, para todo $A \in \mathcal{P}(X)$. Entonces:
    \begin{itemize}
        \item $f$ es inyectiva, pero no necesariamente sobreyectiva.
        \item $f$ es sobreyectiva, pero no necesariamente inyectiva.
        \item $f$ es biyectiva.
    \end{itemize}
\end{ejercicio}

\begin{ejercicio}
Sea $X$ un conjunto con $|X|\geq 2$. La afirmación ``Todo subconjunto de $X \times X$ es de la forma $A \times B$ para ciertos subconjuntos $A, B \subseteq X$'' es:
    \begin{itemize}
        \item Verdadera o falsa, depende de $X$.
        \item Siempre verdadera.
        \item Siempre falsa.
    \end{itemize}
\end{ejercicio}

\begin{ejercicio}
    Sea $R$ una relación simétrica y transitiva en un conjunto $X \neq \emptyset$ ¿Prueba el siguiente razonamiento que $R$ es reflexiva?:\newline
    ``Por simetría, $aRb$ implica $bRa$ y entonces, por transitividad, concluimos que $aRa$''.
    \begin{itemize}
        \item Sí.
        \item No.
    \end{itemize}
\end{ejercicio}

\newpage
\ % --------------------------------------------------------------------------------
\resetearcontador

\begin{ejercicio}
    Sean $X$ e $Y$ dos conjuntos finitos con $|X| = |Y|$ y $f:X \rightarrow Y$ una aplicación. La afirmación ``Si $f$ es inyectiva o sobreyectiva, entonces $f$ es biyectiva'' es:
    \begin{itemize}
        \item Verdadera o falsa, depende de $f$.
        \item \underline{Siempre verdadera.}
        \item Siempre falsa.
    \end{itemize}

    \noindent
    \textbf{Justificación}:
    Si $f$ es inyectiva, entonces $|X| = |Img(f)|$, luego $|Img(f)| = |Y|$ y por tanto, $Img(f) = Y$ y $f$ es sobreyectiva luego biyectiva.\newline
    Si $f$ es sobreyectiva, entonces $|Y|=|Img(f)|$, luego $|Img(f)| = |X|$ y por tanto, $f$ es necesariamente inyectiva luego biyectiva.
\end{ejercicio}

\begin{ejercicio}
    Sea $f:X \rightarrow Y$ una aplicación inyectiva y sean $A, B \subseteq X$. Selecciona la afirmación verdadera:
    \begin{itemize}
        \item $f_{*}(A) - f_{*}(B)$ es un subconjunto propio de $f_{*}(A-B)$.
        \item $f_{*}(A-B)$ es un subconjunto propio de $f_{*}(A) - f_{*}(B)$.
        \item \underline{$f_{*}(A-B) = f_{*}(A) - f_{*}(B)$.}
    \end{itemize}

    \noindent
    \textbf{Justificación}:
    Empezamos recordando la definición de $f_{*}(A)$ para $A \subseteq X$:
    \begin{equation*}
        f_{*}(A) = \{y \in X \mid \exists x \in X \mbox{\ con\ } f(x) = y \}
    \end{equation*}
    \begin{description}
        \item [$\subseteq)$]
            Sea $y \in f_{*}(A-B) \Rightarrow \exists x \in A -B \mid y = f(x)$.\newline
            Esto es, $\exists x \in A \land x \notin B \mid y = f(x)$.\newline
            Como $x \in A \Rightarrow y = f(x) \in f_{*}(A)$. Además, por ser $f$ inyectiva, se tiene que $y \notin f_{*}(B)$, ya que si suponemos que $y \in f_{*}(B)$:

            $y \in f_{*}(B) \Rightarrow \exists b \in B \mid y = f(b) \Rightarrow f(x) = f(b)$ con lo que $x = b \in B$, en contradicción con que $x \notin B$.

            \noindent
            Así, $y \in f_{*}(A) - f_{*}(B)$ para todo $y \in f_{*}(A-B)$. Luego:
            \begin{equation*}
                f_{*}(A-B) \subseteq f_{*}(A) - f_{*}(B)
            \end{equation*}
        \item [$\supseteq)$]
            Sea $y \in f_{*}(A) - f_{*}(B) \Rightarrow y \in f_{*}(A) \land y \notin f_{*}(B)$.\newline
            Como $y \in f_{*}(A) \Rightarrow \exists x \in A \mid y = f(x)$.\newline
            Como $y \notin f_{*}(B) \Rightarrow x \notin B$.\newline
            Luego $x \in A -B \Rightarrow y = f(x) \in f_{*}(A-B)$ para todo $y \in f_{*}(A) - f_{*}(B)$. Luego:
            \begin{equation*}
                f_{*}(A)-f_{*}(B) \subseteq f_{*}(A-B)
            \end{equation*}
    \end{description}
\end{ejercicio}

\begin{ejercicio}
    Sea $f:X \rightarrow X$ una aplicación tal que $f_{*}(c(A)) = c(f_{*}(A))$, para todo $A \in \mathcal{P}(X)$. Entonces:
    \begin{itemize}
        \item $f$ es inyectiva, pero no necesariamente sobreyectiva.
        \item $f$ es sobreyectiva, pero no necesariamente inyectiva.
        \item \underline{$f$ es biyectiva.}
    \end{itemize}

    \noindent
    \textbf{Justificación}:
    Procedemos a demostrar la inyectividad y sobreyectividad de la aplicación.\newline
    Para la sobreyectividad, consideramos $\emptyset \in \mathcal{P}(X)$:
    \begin{equation*}
        f_{*}(c(\emptyset)) = f_{*}(X) = Img(f) = c(f_{*}(\emptyset)) = c(\emptyset) = X
    \end{equation*}
    Para la inyectividad, podemos suponer sin perder generalidad que $|X| \geq 2$ (si no lo fuera, la aplicación sería automáticamente inyectiva).\newline
    Sean $x, x' \in X \mid x \neq x'$. Entonces, $x' \in c(\{x\})$ luego:
    \begin{equation*}
        f(x') \in f_{*}(c(\{x\})) = c(\{f(x)\})
    \end{equation*}
    Luego $f(x') \neq f(x)$.
\end{ejercicio}

\begin{ejercicio}
Sea $X$ un conjunto con $|X|\geq 2$. La afirmación ``Todo subconjunto de $X \times X$ es de la forma $A \times B$ para ciertos subconjuntos $A, B \subseteq X$'' es:
    \begin{itemize}
        \item Verdadera o falsa, depende de $X$.
        \item Siempre verdadera.
        \item \underline{Siempre falsa.}
    \end{itemize}

    \noindent
    \textbf{Justificación}: Supongamos que sí y consideremos el siguiente conjunto:\newline
    Sea $D = \{(x,x) \mid x \in X\} \subseteq X \times X$.\newline
    Si $D = A \times B$ para ciertos $A, B \subseteq X$, entonces para todo $x \in X$, $(x,x) \in A \times B$ y, por tanto, $x \in A$ y $x \in B$.\newline
    Así que $A = X = B$ y, necesariamente, $D = X \times X$. Pero $|X| \geq 2$, luego existen $a,b \in X$ con $a \neq b$, esto es, $(a,b) \notin D$ y $D \neq X \times X$.\newline
    Lo que nos lleva a contradicción.
\end{ejercicio}

\begin{ejercicio}
    Sea $R$ una relación simétrica y transitiva en un conjunto $X \neq \emptyset$ ¿Prueba el siguiente razonamiento que $R$ es reflexiva?:\newline
    ``Por simetría, $aRb$ implica $bRa$ y entonces, por transitividad, concluimos que $aRa$''.
    \begin{itemize}
        \item Sí.
        \item \underline{No.}
    \end{itemize}

    \noindent
    \textbf{Justificación}:
    Dado un $a \in X$, no tiene por qué existir a priori un elemento $b \in X$ tal que $aRb$. Por tanto, buscamos un contraejemplo para desmentir la afirmación:\\

    \noindent
    Dado $X = \{ a,b,c \} \neq \emptyset$ y la relación $R = \{ (a,b), (b,a), (b,b),(a,a) \} \subseteq X \times X$. \newline Observemos que $R$ es simétrica y transitiva pero no reflexiva:

    \noindent
    Es simétrica ya que para todos $\alpha, \beta \in X \mid \alpha R \beta \Rightarrow \beta R \alpha$:
    \begin{center}
        Ya que $aRb$, ¿se cumple que $bRa$?. Sí.\\
        Ya que $bRa$, ¿se cumple que $aRb$?. Sí.\\
        Ya que $bRb$, ¿se cumple que $bRb$?. Sí.\\
        Ya que $aRa$, ¿se cumple que $aRa$?. Sí.
    \end{center}
    Es transitiva ya que para todos $\alpha, \beta, \gamma \in X \mid \alpha R \beta \land \beta R \gamma \Rightarrow \alpha R \gamma$:
    \begin{center}
        Ya que $aRb$ y $bRa$, ¿se cumple que $aRa$?. Sí.\\
        Ya que $bRa$ y $aRb$, ¿se cumple que $bRb$?. Sí.\\
        Ya que $bRb$ y $bRb$, ¿se cumple que $bRb$?. Sí.\\
        Ya que $aRa$ y $aRa$, ¿se cumple que $aRa$?. Sí.
    \end{center}
    No es reflexiva, ya que $\exists c \in X \mid c\cancel{R}c$.
\end{ejercicio}

\newpage
\resetearcontador


    \subsection{Cuestionario III}
\begin{ejercicio}
    Sea $X$ un conjunto no vacío. Definimos en $\mathcal{P}(X)$ operaciones de suma y producto por $A+B = A \cup B$ y $A \cdot B = A \cap B$. Entonces (selecciona la respuesta correcta).
    \begin{itemize}
        \item $\mathcal{P}(X)$ es un anillo conmutativo.
        \item $\mathcal{P}(X)$ no es un anillo conmutativo, falla un axioma.
        \item $\mathcal{P}(X)$ no es un anillo conmutativo, fallan dos axiomas.
    \end{itemize}
\end{ejercicio}

\begin{ejercicio}
    Para enteros $m$ y $n$ tales que $2 \leq m < n$, la afirmación ``$\bb{Z}_m$ es un subanillo de $\bb{Z}_n$'' es:
    \begin{itemize}
        \item Verdadera o falsa, dependiendo de $m$ y de $n$.
        \item Siempre verdadera.
        \item Siempre falsa.
    \end{itemize}
\end{ejercicio}

\begin{ejercicio}
    En el anillo $\bb{Z}_8$ (seleccion la afirmación verdadera).
    \begin{itemize}
        \item $3$ es una unidad y $4 \cdot 3^{-1} = 4$.
        \item $3$ es una unidad, pero $4 \cdot 3^{-1} \neq 4$.
        \item $3$ no es una unidad.
    \end{itemize}
\end{ejercicio}

\begin{ejercicio}
    En el anillo $\bb{Z}[\sqrt{3}]$, la afirmación ``${(7+4\sqrt{3})}^n$ es una unidad para todo natural $n \geq 1$'' es:
    \begin{itemize}
        \item Verdadera o falsa, dependiendo de $n$.
        \item Siempre verdadera.
        \item Siempre falsa.
    \end{itemize}
\end{ejercicio}

\begin{ejercicio}
    Sea $A \subseteq \bb{R}$ un subanillo. La afirmación ``\bb{Z} es un subanillo de A'' es:
    \begin{itemize}
        \item Siempre verdadera.
        \item Siempre falsa.
        \item Verdadera o falsa, dependiendo de $A$.
    \end{itemize}
\end{ejercicio}

\newpage
\ % --------------------------------------------------------------------------------
\resetearcontador

\begin{ejercicio}
    Sea $X$ un conjunto no vacío. Definimos en $\mathcal{P}(X)$ operaciones de suma y producto por $A+B = A \cup B$ y $A \cdot B = A \cap B$. Entonces (selecciona la respuesta correcta).
    \begin{itemize}
        \item $\mathcal{P}(X)$ es un anillo conmutativo.
        \item \underline{$\mathcal{P}(X)$ no es un anillo conmutativo, falla un axioma.}
        \item $\mathcal{P}(X)$ no es un anillo conmutativo, fallan dos axiomas.
    \end{itemize}

    \noindent
    \textbf{Justificación}:
    En este caso, $0 = \emptyset$, ya que:
    \begin{equation*}
        \emptyset + A = \emptyset \cup A = A\quad\forall A \in \mathcal{P}(X)
    \end{equation*}
    Y no hay opuestos, sea $A\neq \emptyset \in \mathcal{P}(X)$:
    \begin{equation*}
        A + B = A \cup B \supseteq A \neq \emptyset\quad\forall B \in \mathcal{P}(X)
    \end{equation*}
    Podemos ver que el resto de axiomas se cumplen:
    
    \begin{itemize}
        \item Conmutativa de la suma:
        \begin{equation*}
            A + B = A \cup B = B \cup A = B + A\quad\forall A,B \in \mathcal{P}(X)
        \end{equation*}
        \item Asociativa de la suma:
        \begin{equation*}
            A + (B + C) = A \cup (B \cup C) = (A \cup B) \cup C = (A+B)+C\quad\forall A,B,C \in \mathcal{P}(X)
        \end{equation*}
        \item Elemento neutro de la suma (ya demostrado).
        \item Existencia de opuestos (ya se ha visto que no se cumple).
        \item Conmutativa del producto:
        \begin{equation*}
            A \cdot B = A \cap B = B \cap A = B \cdot A\quad\forall A,B \in \mathcal{P}(X)
        \end{equation*}
        \item Asociativa del producto:
        \begin{equation*}
            A \cdot (B \cdot C) = A \cap (B \cap C) = (A \cap B) \cap C = (A\cdot B)\cdot C\quad\forall A,B,C \in \mathcal{P}(X)
        \end{equation*}
        \item Elemento neutro del producto:
            \begin{equation*}
                A \cdot X = A\quad\forall A \in \mathcal{P}(X)
            \end{equation*}
        \item Distributiva del producto respecto de la suma:
            \begin{equation*}
                A \cdot (B + C) = A \cap (B \cup C) = (A \cap B) \cup (A \cap C) = (A \cdot B) +(A\cdot C)\quad\forall A,B,C \in \mathcal{P}(X)
            \end{equation*}
    \end{itemize}
\end{ejercicio}

\begin{ejercicio}
    Para enteros $m$ y $n$ tales que $2 \leq m < n$, la afirmación ``$\bb{Z}_m$ es un subanillo de $\bb{Z}_n$'' es:
    \begin{itemize}
        \item Verdadera o falsa, dependiendo de $m$ y de $n$.
        \item Siempre verdadera.
        \item \underline{Siempre falsa.}
    \end{itemize}

    \noindent
    \textbf{Justificación}:
    En $\bb{Z}_m$, se tiene que $m = 0$.\newline
    Sin embargo, por ser $2 \leq m < n$, tenemos que $m \neq 0$ en $\bb{Z}_n$.
\end{ejercicio}

\begin{ejercicio}
    En el anillo $\bb{Z}_8$ (seleccion la afirmación verdadera).
    \begin{itemize}
        \item \underline{$3$ es una unidad y $4 \cdot 3^{-1} = 4$.}
        \item $3$ es una unidad, pero $4 \cdot 3^{-1} \neq 4$.
        \item $3$ no es una unidad.
    \end{itemize}

    \noindent
    \textbf{Justificación}:
    $3$ es una unidad ya que $3 \cdot 3 = 9 = 1$, luego $3^{-1} = 3$.\newline
    Entonces, $4 \cdot 3^{-1} = 4 \cdot 3 = 12 = 4$.
\end{ejercicio}

\begin{ejercicio}
    En el anillo $\bb{Z}[\sqrt{3}]$, la afirmación ``${(7+4\sqrt{3})}^n$ es una unidad para todo natural $n \geq 1$'' es:
    \begin{itemize}
        \item Verdadera o falsa, dependiendo de $n$.
        \item Siempre falsa.
        \item \underline{Siempre verdadera.}
    \end{itemize}

    \noindent
    \textbf{Justificación}:
    Tenemos que $7 + 4\sqrt{3}$ es invertible, puesto que:
    \begin{equation*}
        N(7+4\sqrt{3}) = 7^2 - 3 \cdot 16 = 49 - 48 = 1
    \end{equation*}
    Como el producto de unidades es una unidad, cualquier potencia de una unidad también lo es.
\end{ejercicio}

\begin{ejercicio}
    Sea $A \subseteq \bb{R}$ un subanillo. La afirmación ``\bb{Z} es un subanillo de A'' es:
    \begin{itemize}
        \item \underline{Siempre verdadera.}
        \item Siempre falsa.
        \item Verdadera o falsa, dependiendo de $A$.
    \end{itemize}

    \noindent
    \textbf{Justificación}:
    Por inducción, veamos primero que $\bb{N} = \bb{Z}^{+} \subseteq A$.\newline
    Esto es, que $n \in A\quad\forall n \in \bb{N}$.
    \begin{enumerate}
        \item [$n=0$:]
            Por ser $A$ subanillo de $\bb{R}$, se tiene que $0 \in A$.
        \item [$n=1$:]
            Por ser $A$ subanillo de $\bb{R}$, se tiene que $1 \in A$.
        \item [$n>1$:]
            Como hipótesis de inducción, supongamos que $n \in A$ y veamos que $n+1 \in A$.\newline
            Por ser $A$ cerrado para la suma, tenemos que $1 \in A$ y que $n \in A$ por hipótesis de inducción, luego $n+1 \in A$.
    \end{enumerate}
    Por tanto, $\bb{N} = \bb{Z}^{+} \subseteq A$.\newline
    Ahora, para $n \in \bb{Z}$ con $n \geq 0$, $A$ es cerrado para opuestos, luego $-n \in A$.\newline
    Por tanto, $\bb{Z} \subseteq A$.\\

    \noindent
    Por ser $\bb{Z}$ cerrado para la suma, producto, opuestos y contiene al $0$ y al $1$, $\bb{Z}$ es subanillo de $A$. Por tanto, $\bb{Z}$ es el menor subanillo de $\bb{R}$.
\end{ejercicio}

\newpage
\resetearcontador


    \subsection{Cuestionario IV}
\begin{ejercicio}
    En el anillo $\bb{Z}_{10}$, la afirmación ``$3^{4k+3} = -3$, para cualquier $k \in \bb{Z}$'' es:
    \begin{itemize}
        \item Siempre falsa.
        \item Siempre cierta.
        \item A veces cierta y a veces falsa, depende de $k$.
    \end{itemize}
\end{ejercicio}

\begin{ejercicio}
    En el anillo $\bb{Z}_n[x]$, la afirmación ``la suma reiterada $n$ veces de cualquier polinomio es $0$'', es:
    \begin{itemize}
        \item Verdera o falsa, depende de $n$.
        \item Siempre falsa.
        \item Siempre verdadera.
    \end{itemize}
\end{ejercicio}

\begin{ejercicio}
    Un subanillo $A$ de un anillo $B$ se dice propio si $A \subsetneq B$. Seleccion el enunciado correcto:
    \begin{itemize}
        \item En anillo $\bb{Z}$ no tiene subanillos propios.
        \item El conjunto $A = \{ 5k \mid k \in \bb{Z} \}$ es un subanillo propio de $\bb{Z}$.
        \item El cuerpo $\bb{Q}$ no tiene subanillos propios.
    \end{itemize}
\end{ejercicio}

\begin{ejercicio}
    Homomorifismos $\phi : \bb{Z}_2 \rightarrow \bb{Z}$,
    \begin{itemize}
        \item Hay exactamente uno.
        \item Hay al menos dos.
        \item No hay ninguno.
    \end{itemize}
\end{ejercicio}

\begin{ejercicio}
    Sea $A$ un anillo comutativo, la afirmación ``Para cualesquiera indeterminadas $x$ e $y$, los anillos de polinomios $A[x]$ y $A[y]$ son isomorifos''. Es:
    \begin{itemize}
        \item Verdadera o falsa, depende de $A$.
        \item Siempre verdadera.
        \item Siempre falsa.
    \end{itemize}
\end{ejercicio}

\newpage
\ % --------------------------------------------------------------------------------
\resetearcontador

\begin{ejercicio}
    En el anillo $\bb{Z}_{10}$, la afirmación ``$3^{4k+3} = -3$, para cualquier $k \in \bb{Z}$'' es:
    \begin{itemize}
        \item Siempre falsa.
        \item \underline{Siempre cierta.}
        \item A veces cierta y a veces falsa, depende de $k$.
    \end{itemize}

    \noindent
    \textbf{Justificación}:
    \begin{equation*}
    3^{4k+3}={(3^4)}^k \cdot 3^3 = {(9 \cdot  9)}^k \cdot 9 \cdot 3 = 1^k \cdot 7 = 7\quad \forall k \in \mathbb{Z}
    \end{equation*}
\end{ejercicio}

\begin{ejercicio}
    En el anillo $\bb{Z}_n[x]$, la afirmación ``la suma reiterada $n$ veces de cualquier polinomio es $0$'', es:
    \begin{itemize}
        \item Verdera o falsa, depende de $n$.
        \item Siempre falsa.
        \item \underline{Siempre verdadera.}
    \end{itemize}

    \noindent
    \textbf{Justificación}:
    Sea $R_n:\mathbb{Z}[x]\to \mathbb{Z}_n[x]$ el homomorfismo de reducción módulo $n$. Para cualquier $f \in \mathbb{Z}_n[x]$:
    \begin{equation*}
        nf = nR_n(f) = R_n(nf) = R_n(n)R_n(f) = 0 \cdot f = 0
    \end{equation*}
\end{ejercicio}

\begin{ejercicio}
    Un subanillo $A$ de un anillo $B$ se dice propio si $A \subsetneq B$. Seleccion el enunciado correcto:
    \begin{itemize}
        \item \underline{En anillo $\bb{Z}$ no tiene subanillos propios.}
        \item El conjunto $A = \{ 5k \mid k \in \bb{Z} \}$ es un subanillo propio de $\bb{Z}$.
        \item El cuerpo $\bb{Q}$ no tiene subanillos propios.
    \end{itemize}

    \noindent
    \textbf{Justificación}:
    Si $A$ es un subanillo de $\mathbb{Z}$, entonces $1 \in A$ con lo que para todo $n \geq 0$, $\overbrace{1+\cdots+1}^{n \text{\ veces}}=n \in A$ y, como $A$ contiene a sus opuestos, entonces $\mathbb{Z}\subseteq A$. Por lo que $A = \mathbb{Z}$.
\end{ejercicio}

\begin{ejercicio}
    Homomorifismos $\phi : \bb{Z}_2 \rightarrow \bb{Z}$,
    \begin{itemize}
        \item Hay exactamente uno.
        \item Hay al menos dos.
        \item \underline{No hay ninguno.}
    \end{itemize}

    \noindent
    \textbf{Justificación}:
    Si $\phi:\mathbb{Z}_2\to \mathbb{Z}$ fuese un homomorfismo, tendríamos que:
    \begin{equation*}
        \phi(1+1) = \phi(1) + \phi(1) = 1+1 = 2
    \end{equation*}
    Pero en $\mathbb{Z}_2$, $1+1=0$ y por tanto, $\phi(1+1)=\phi(0)=0$, así que sería $0 = 2$ en $\mathbb{Z}$, lo que es una contradicción.
\end{ejercicio}

\begin{ejercicio}
    Sea $A$ un anillo comutativo, la afirmación ``Para cualesquiera indeterminadas $x$ e $y$, los anillos de polinomios $A[x]$ y $A[y]$ son isomorifos''. Es:
    \begin{itemize}
        \item Verdadera o falsa, depende de $A$.
        \item \underline{Siempre verdadera.}
        \item Siempre falsa.
    \end{itemize}

    \noindent
    \textbf{Justificación}:
    El automorfismo identidad $id_A:A \cong A$ extiende a un único homomorfismo $\phi:A[x]\to A[y]$ tal que $\phi(x)=y$. Explícitamente:
    \begin{equation*}
        \phi\left(\sum_{i=0}^{n} a_i x^i\right) = \sum_{i=0}^{n} a_i y^i
    \end{equation*}
    Claramente $\phi$ es biyectiva.
\end{ejercicio}

\newpage
\resetearcontador


    \subsection{Cuestionario V}

\begin{ejercicio}
    En relación con los anillos $\bb{Z}_6$ y $\bb{Z} \times \bb{Z}$, selecciona la afirmación correcta:
    \begin{itemize}
        \item Ambos son DI.
        \item Uno de ellos es DI, pero el otro no.
        \item Ninguno es DI.
    \end{itemize}
\end{ejercicio}

\begin{ejercicio}
    En relación a las siguientes proposiciones, referidas a los elementos de un Dominio de Integridad:
    \begin{enumerate}
        \item [(a)] $a\mid b \land a \nmid c \Rightarrow b \nmid b+c$.
        \item [(b)] $a\mid b \land a \nmid c \Rightarrow a \nmid b+c$.
    \end{enumerate}
    Selecciona la afirmación correcta:
    \begin{itemize}
        \item Ambas son verdad.
        \item Una es verdad y la otra es falsa.
        \item Ambas son falsas.
    \end{itemize}
\end{ejercicio}

\begin{ejercicio}
    Polinomios de grado uno que son unidades en el anillo de polinomios $\bb{Z}_4[x]$:
    \begin{itemize}
        \item No hay.
        \item Hay dos.
        \item Hay infinitos.
    \end{itemize}
\end{ejercicio}

\begin{ejercicio}
    En el anillo $\bb{Z}[i]$:
    \begin{itemize}
        \item $3$ es unidad.
        \item $3$ es irreducible.
        \item $3$ no es irreducible.
    \end{itemize}
\end{ejercicio}

\begin{ejercicio}
    En el anillo $\bb{Z}[i]$:
    \begin{itemize}
        \item $2$ es unidad.
        \item $2$ es irreducible.
        \item $2$ no es irreducible.
    \end{itemize}
\end{ejercicio}

\newpage
\ % --------------------------------------------------------------------------------
\resetearcontador

\begin{ejercicio}
    En relación con los anillos $\bb{Z}_6$ y $\bb{Z} \times \bb{Z}$, selecciona la afirmación correcta:
    \begin{itemize}
        \item Ambos son DI.
        \item Uno de ellos es DI, pero el otro no.
        \item \underline{Ninguno es DI.}
    \end{itemize}

    \noindent
    \textbf{Justificación}:
    \begin{itemize}
        \item En $\mathbb{Z}_6$, $2\cdot 3=0$.
        \item En $\mathbb{Z}\times \mathbb{Z}$, $(1,0)\cdot (0,1)=(0,0)$.
    \end{itemize}
\end{ejercicio}

\begin{ejercicio}
    En relación a las siguientes proposiciones, referidas a los elementos de un Dominio de Integridad:
    \begin{enumerate}
        \item [(a)] $a\mid b \land a \nmid c \Rightarrow b \nmid b+c$.
        \item [(b)] $a\mid b \land a \nmid c \Rightarrow a \nmid b+c$.
    \end{enumerate}
    Selecciona la afirmación correcta:
    \begin{itemize}
        \item Ambas son verdad.
        \item \underline{Una es verdad y la otra es falsa.}
        \item Ambas son falsas.
    \end{itemize}

    \noindent
    \textbf{Justificación}:
    \begin{itemize}
        \item La primera es cierta: si $b=ax$ y fuese $b+c=ay$, tendríamos que $c=ay-ax=a(x-y)$, así que $a\mid c$, lo que es contradictorio.
        \item La segunda es falsa: por ejemplo, en $\mathbb{Z}$, $2\nmid 1$ y $2\nmid 3$, pero $2\mid 1+3=4$.
    \end{itemize}
\end{ejercicio}

\begin{ejercicio}
    Polinomios de grado uno que son unidades en el anillo de polinomios $\bb{Z}_4[x]$:
    \begin{itemize}
        \item No hay.
        \item \underline{Hay dos.}
        \item Hay infinitos.
    \end{itemize}

    \noindent
    \textbf{Justificación}:
    La tabla de multiplicar en $\mathbb{Z}_4$ es:
    \begin{equation*}
       \begin{array}{c|cccc}
           (\mathbb{Z}_4, \cdot) & 0 & 1 & 2 & 3 \\
           \hline
           0 & 0 & 0 & 0 & 0 \\
           1 & 0 & 1 & 2 & 3 \\
           2 & 0 & 2 & 0 & 2 \\
           3 & 0 & 3 & 2 & 1
       \end{array} 
    \end{equation*}
    Buscamos estudiar el cardinal del conjunto:
    \begin{equation*}
        \left\{p \in U(\mathbb{Z}_4[x]) \mid \deg(p) =1\right\}
    \end{equation*}
    Sea $ax+b \in U(\mathbb{Z}_4[x])$ con $a\neq 0$:
    \begin{align*}
        (ax+b)(ax+b) &= 1 \Longrightarrow {(ax+b)}^{2}=1 \Longrightarrow a^2x + 2abx + b^2 = 1 \\
                     &\Longrightarrow a^2 = 0 \quad\land\quad 2ab = 0 \quad\land\quad b^2 = 1
    \end{align*}
    \begin{equation*}
        \left\{\begin{array}{lll}
                a^2 = 0 & \Longrightarrow & a = 2 \\
                2ab = 0 & \Longrightarrow & 4b = 0 \Longrightarrow 0b = 0 \Longrightarrow 0=0 \\
                b^2 = 1 & \Longrightarrow & b = 1 \quad\lor\quad b = 3
        \end{array}\right.
    \end{equation*}
    Luego:
    \begin{gather*}
        2x+1 \in U(\mathbb{Z}_4[x]) \\
        2x+3 \in U(\mathbb{Z}_4[x])
    \end{gather*}
    Tenemos dos polinomios que verifican la segunda opción. Además, la última no puede ser por ser $\mathbb{Z}_4[x]$ finito.
\end{ejercicio}

\begin{ejercicio}
    En el anillo $\bb{Z}[i]$:
    \begin{itemize}
        \item $3$ es unidad.
        \item \underline{$3$ es irreducible.}
        \item $3$ no es irreducible.
    \end{itemize}

    \noindent
    \textbf{Justificación}:
    \begin{equation*}
        N(3) = 9 \neq \pm 1 \Longrightarrow 3 \notin U(\mathbb{Z}[i])
    \end{equation*}
    Para probar que $3$ es irreducible, supongamos una factorización $3=\alpha \cdot \beta$ con $\alpha, \beta \in \mathbb{Z}[i]\setminus U(\mathbb{Z}[i])$. Entonces:
    \begin{equation*}
        N(3) = N(\alpha)N(\beta) \Longrightarrow 9 = N(\alpha)N(\beta) \quad N(\alpha), N(\beta) \in \mathbb{Z}
    \end{equation*}
    Como $\alpha, \beta \notin U(\mathbb{Z}[i]) \Longrightarrow N(\alpha), N(\beta)\neq \pm 1$
    Como $\alpha, \beta \in \mathbb{Z}[i]$, se tiene que:
    \begin{align*}
        N(\alpha) &= a^2 + b^2 \geq 1 \\
        N(\beta) &= {(a')}^{2} + {(b')}^{2} \geq 1
    \end{align*}
    Por tanto, $N(\alpha), N(\beta) \in \bb{N}$. Además, $9=N(\alpha)N(\beta)\Longrightarrow N(\alpha)=N(\beta)=3$.
    \begin{equation*}
        N(\alpha) = 3 \Longrightarrow a^2 + b^2 = 3
    \end{equation*}
    Pero $\nexists a,b \in \mathbb{Z} \mid a^2 + b^2 = 3$, por lo que 3 es irreducible.
\end{ejercicio}

\begin{ejercicio}
    En el anillo $\bb{Z}[i]$:
    \begin{itemize}
        \item $2$ es unidad.
        \item $2$ es irreducible.
        \item \underline{$2$ no es irreducible.}
    \end{itemize}

    \noindent
    \textbf{Justificación}:
    \begin{equation*}
        N(2) = 4 \neq 1 \Longrightarrow 2 \notin U(\mathbb{Z}[i])
    \end{equation*}
    Para ver que 2 no es irreducible, supongamos una factorización: $2=\alpha \cdot \beta \mid \alpha, \beta \in \mathbb{Z}[i]\setminus U(\mathbb{Z}[i])$.
    \begin{equation*}
        N(2) = N(\alpha \beta) \Longrightarrow 4=N(\alpha)N(\beta) \Longrightarrow N(\alpha) = N(\beta) = 2
    \end{equation*}
    Por ejemplo, $\alpha = \beta = 1+i$
    \begin{equation*}
        -i{(1+i)}^{2} = (1+i^2 + 2i)(-i) = (-i)(1-1+2i) = (-i)2i = -2i^2 = 2
    \end{equation*}
    Luego $2 = -i{(1+i)}^{2}$ es la factorización esencialmente única de 2 $\Longrightarrow$ es reducible.
\end{ejercicio}
\newpage
\resetearcontador

    \subsection{Cuestionario VI}

\begin{ejercicio}
    En relación a las siguientes proposiciones, referidas a elementos cualesquiera de un DI, selecciona las verdaderas:
    \begin{itemize}
        \item $c\mid ab \Longrightarrow c\mid a \lor c\mid b$.
        \item $a\mid c \land b\mid c \Longrightarrow ab\mid c$. 
        \item $c\mid a \lor c\mid b \Longrightarrow c\mid ab$.
    \end{itemize}
\end{ejercicio}

\begin{ejercicio}
    Entre los siguientes DE, selecciona aquellos en los que el máximo común divisor y el mínimo común múltiplo son únicos salvo signo:
    \begin{itemize}
        \item $\mathbb{Z}\left[\sqrt{-2}\right]$.
        \item $\mathbb{Z}\left[\sqrt{3}\right]$. 
        \item $\mathbb{Z}_3[x]$.
    \end{itemize}
\end{ejercicio}

\begin{ejercicio}
    En un DE, tenemos la ecuación diofántica $px+by=1$, donde $p$ es irreducible. Entre las siguientes afirmaciones, selecciona la que es verdad.
    \begin{itemize}
        \item Nunca tiene solución.
        \item Puede tener solución o no, depende de $b$. 
        \item Siempre tiene solución.
    \end{itemize}
\end{ejercicio}

\begin{ejercicio}
    En un DE, tenemos la ecuación diofántica $px+qy=c$, donde $p$ y $q$ son irreducibles no asociados entre sí. Entre las siguientes afirmaciones, selecciona la que es verdad.
    \begin{itemize}
        \item Nunca tiene solución.
        \item Puede tener solución o no, depende de $p$ y de $q$. 
        \item Siempre tiene solución.
    \end{itemize}
\end{ejercicio}

\begin{ejercicio}
    Entre las siguientes proposiciones, referidas a un DE, selecciona las verdaderas.
    \begin{itemize}
        \item Si la ecuación $ax+by=1$ tiene solución, entonces la ecuación $ax+by=c$ tiene solución para todo $c$.
        \item Si la ecuacióin $ax+bb'y=1$ tiene solución, entonces las ecuaciones $ax+by=1$ y $ax+b'y=1$ tienen solución. 
        \item Si las ecuaciones $ax+by=1$ y $ax+b'y=1$ tienen solución, entonces la ecuación $ax+bb'y=1$ tiene solución.
    \end{itemize}
\end{ejercicio}

\newpage
\ % --------------------------------------------------------------------------------
\resetearcontador

\begin{ejercicio}
    En relación a las siguientes proposiciones, referidas a elementos cualesquiera de un DI, selecciona las verdaderas:
    \begin{itemize}
        \item $c\mid ab \Longrightarrow c\mid a \lor c\mid b$.
        \item $a\mid c \land b\mid c \Longrightarrow ab\mid c$. 
        \item \underline{$c\mid a \lor c\mid b \Longrightarrow c\mid ab$.}
    \end{itemize}

    \noindent
    \textbf{Justificación}:
    \begin{itemize}
        \item La primera es falsa, en $\mathbb{Z}$, $6\mid 12 = 4 \cdot 3$ pero $6\nmid 4$.
        \item La segunda es falsa, en $\mathbb{Z}$, $2\mid 6$ pero $2 \cdot 2 \nmid 6$.
        \item La tercera es verdadera. De hecho, basta con que $c$ divida a uno de ellos para que divida al producto:
            \begin{equation*}
                a = ca' \Longrightarrow ab=c(a'b)
            \end{equation*}
    \end{itemize}
\end{ejercicio}

\begin{ejercicio}
    Entre los siguientes DE, selecciona aquellos en los que el máximo común divisor y el mínimo común múltiplo son únicos salvo signo:
    \begin{itemize}
        \item \underline{$\mathbb{Z}\left[\sqrt{-2}\right]$.}
        \item $\mathbb{Z}\left[\sqrt{3}\right]$. 
        \item \underline{$\mathbb{Z}_3[x]$.}
    \end{itemize}

    \noindent
    \textbf{Justificación}:
    Serán aquellos cuyas unidades sean $\pm 1$:
    \begin{itemize}
        \item En $\mathbb{Z}\left[\sqrt{-2}\right]$, $a+b\sqrt{-2}$ es unidad si y sólo si $a^2 + 2b^2 =1$, lo que sólo se verifica si $a=1$ y $b=0$.
        \item En $\mathbb{Z}\left[\sqrt{3}\right]$, $a+b\sqrt{3}$ es unidad si y sólo si $a^2 - 3b^2 =\pm 1$, lo que verifica por ejemplo $2+\sqrt{3}\neq \pm 1$, luego aquí el mcd y el mcm no son únicos salvo signo.
        \item En $\mathbb{Z}_3[x]$:
            \begin{equation*}
                U\left(\mathbb{Z}_3[x]\right) = U\left(\mathbb{Z}_3\right) = \{1,2\} = \{ 1, -1 \} = \{ \pm 1\}
            \end{equation*}
    \end{itemize}
\end{ejercicio}

\begin{ejercicio}
    En un DE, tenemos la ecuación diofántica $px+by=1$, donde $p$ es irreducible. Entre las siguientes afirmaciones, selecciona la que es verdad.
    \begin{itemize}
        \item Nunca tiene solución.
        \item \underline{Puede tener solución o no, depende de $b$.} 
        \item Siempre tiene solución.
    \end{itemize}

    \noindent
    \textbf{Justificación}:
    La ecuación tendrá solución $\Longleftrightarrow \text{mcd}(p,b)\mid 1 \Longleftrightarrow \text{mcd}(p,b)=1$. Como $p$ es irreducible, equivale a que $p \nmid b$, luego puede tener solución o no, dependiendo de $b$:
    \begin{itemize}
        \item Para $b=1$ sí tiene solución.
        \item Pero para $b=2p \Longrightarrow \text{mcd}(p,2p)=p\neq 1$ no tiene solución.
    \end{itemize}
\end{ejercicio}

\begin{ejercicio}
    En un DE, tenemos la ecuación diofántica $px+qy=c$, donde $p$ y $q$ son irreducibles no asociados entre sí. Entre las siguientes afirmaciones, selecciona la que es verdad.
    \begin{itemize}
        \item Nunca tiene solución.
        \item Puede tener solución o no, depende de $p$ y de $q$. 
        \item \underline{Siempre tiene solución.}
    \end{itemize}

    \noindent
    \textbf{Justificación}:
    La ecuación tendrá solución $\Longleftrightarrow \text{mcd}(p,q)\mid c$. Como $p$ y $q$ son irreducibles no asociados, tenemos que $\text{mcd}(p,q)=1$ y como $1\mid c$ $\forall c \in A$, la ecuación siempre tendrá solución.
\end{ejercicio}

\begin{ejercicio}
    Entre las siguientes proposiciones, referidas a un DE, selecciona las verdaderas.
    \begin{itemize}
    \item \underline{Si la ecuación $ax+by=1$ tiene solución, entonces la ecuación $ax+by=c$} \newline
        \underline{ tiene solución para todo $c$.}
\item \underline{Si la ecuacióin $ax+bb'y=1$ tiene solución, entonces las ecuaciones $ax+by=1$}
    \underline{ y $ax+b'y=1$ tienen solución.} 
\item \underline{Si las ecuaciones $ax+by=1$ y $ax+b'y=1$ tienen solución, entonces la }\newline
    \underline{ecuación $ax+bb'y=1$ tiene solución.}
    \end{itemize}

    \noindent
    \textbf{Justificación}:
    \begin{itemize}
        \item Sea $(x_0,y_0)$ solución de $ax+by=1 \Longrightarrow (cx_0, cy_0)$ es solución de $ax+by=c$.
        \item Sea $(x_0,y_0)$ solución de $ax+bb'y=1 \Longrightarrow (x_0, y_0b')$ es solución de $ax+by=1$ y $(x_0, y_0b)$ es solución de $ax+b'y=1$.
        \item 
            \begin{equation*}
                \left.
                    \begin{array}{lcr}
                        ax+by=1 \text{\ tiene\ solución} & \Longrightarrow & \text{mcd}(a,b)=1 \\
                        ax+b'y=1 \text{\ tiene\ solución} & \Longrightarrow & \text{mcd}(a,b')=1
                    \end{array}
                \right\} \Longrightarrow \text{mcd}(a,bb')=1
            \end{equation*}
            Luego $ax+bb'y=1$ tiene solución. 
    \end{itemize}
\end{ejercicio}

\newpage
\resetearcontador

    \section{Cuestionario VII}

\begin{ejercicio}
    En relación a las siguientes proposiciones sobre elementos de un DE, selecciona las verdaderas:
    \begin{itemize}
        \item Si $\text{mcd}(a,b)=1$, entonces $\text{mcd}(a,b^n)=1$ para todo $n \in \mathbb{N}$.
        \item Si $a \equiv a'\mod(b)$, entonces $\text{mcd}(a,b)=\text{mcd}(a',b)$.
        \item Si $a\equiv a'\mod(b)$, entonces $\text{mcm}(a,b)=\text{mcm}(a',b)$.
    \end{itemize}
\end{ejercicio}

\begin{ejercicio}
    Entre las siguientes ecuaciones en congruencias, selecciona las que tienen solución.
    \begin{itemize}
        \item En $\mathbb{Z}$, $6x\equiv 10 \mod (45)$.
        \item En $\mathbb{Z}$, $100x\equiv 20\mod (15)$.
        \item En $\mathbb{Z}[i]$, $(2+2i)x\equiv 5\mod(3-i)$.
    \end{itemize}
\end{ejercicio}

\begin{ejercicio}
    Entre las siguientes afirmaciones relativas a ecuaciones en el anillo $\mathbb{Z}_{64}$, selecciona las que son verdad.
    \begin{itemize}
        \item $12x=28$ tiene $4$ soluciones.
        \item $14x=28$ tiene $4$ soluciones.
        \item $12x=30$ tiene $4$ soluciones.
    \end{itemize}
\end{ejercicio}

\begin{ejercicio}
    Entre las siguientes proposiciones, selecciona las verdaderas.
    \begin{itemize}
        \item El anillo $\mathbb{Z}_{900}$ tiene 240 unidades.
        \item $14^{20}\equiv 1\mod (33)$.
        \item $3^{16}=3$ en $\mathbb{Z}_{16}$.
    \end{itemize}
\end{ejercicio}

\begin{ejercicio}
    Sea $p$ un número primo y considérese la congruencia $ax\equiv 1\mod (p^2)$. En relación a las siguientes proposiciones, selecciona las verdaderas:
    \begin{itemize}
        \item No tiene solución, pues $p^2$ no es primo.
        \item Tiene solución si y sólo si la congruencia $ax\equiv 1\mod (p)$ tiene solución.
        \item Tiene solución salvo que $a$ sea múltiplo de $p^2$.
    \end{itemize}
\end{ejercicio}

\newpage
\ % --------------------------------------------------------------------------------
\resetearcontador

\begin{ejercicio}
    En relación a las siguientes proposiciones sobre elementos de un DE, selecciona las verdaderas:
    \begin{itemize}
        \item \underline{Si $\text{mcd}(a,b)=1$, entonces $\text{mcd}(a,b^n)=1$ para todo $n \in \mathbb{N}$.}
        \item \underline{Si $a \equiv a'\mod(b)$, entonces $\text{mcd}(a,b)=\text{mcd}(a',b)$.}
        \item Si $a\equiv a'\mod(b)$, entonces $\text{mcm}(a,b)=\text{mcm}(a',b)$.
    \end{itemize}

    \noindent
    \textbf{Justificación}:
    \begin{itemize}
        \item Es cierto, lo probamos por inducción:
            \begin{description}
                \item [Para $n=0$:] 
                    $\text{mcd}(a,b^0) = \text{mcd}(a,1)=1$, cierto.
                \item [Para $n=1$:] 
                    $\text{mcd}(a,b)=1$, cierto.
                \item [Supuesto cierto para $n-1$, lo vemos para $n$:] 
                    \begin{equation*}
                        \left.\begin{array}{r}
                            \text{mcd}(a,b)=1 \\
                            \text{mcd}(a,b^{n-1}) = 1
                    \end{array}\right\} \text{mcd}(a,b^n) = \text{mcd}(a,b^{n-1}b) = 1
                    \end{equation*}
            \end{description}
        \item Es cierto, sea $A$ el DE:
            \begin{align*}
                a\equiv a'\mod(b) &\Longrightarrow \exists q\in A \mid a-a' = qb \\
                                  &\Longrightarrow  a'=a-qb
            \end{align*}
            \begin{equation*}
                \text{mcd}(a,b) = \text{mcd}(a-qb,b) = \text{mcd}(a',b)
            \end{equation*}
        \item Es falso, por ejemplo en $\mathbb{Z}$, sean $a=6$, $a' = 2$, $b = 4$
            \begin{gather*}
                6\equiv 2\mod (4) \\
                \text{mcm}(6,4) = 12 \neq 4 = \text{mcm}(2,4)
            \end{gather*}
    \end{itemize}
\end{ejercicio}

\begin{ejercicio}
    Entre las siguientes ecuaciones en congruencias, selecciona las que tienen solución.
    \begin{itemize}
        \item En $\mathbb{Z}$, $6x\equiv 10 \mod (45)$.
        \item \underline{En $\mathbb{Z}$, $100x\equiv 20\mod (15)$.}
        \item En $\mathbb{Z}[i]$, $(2+2i)x\equiv 5\mod(3-i)$.
    \end{itemize}

    \noindent
    \textbf{Justificación}:
    \begin{itemize}
        \item $\text{mcd}(6,45)=3$, como $3\nmid 10 \Longrightarrow$ no tiene solución.
        \item $\text{mcd}(100,15)=5$, como $5\mid 20 \Longrightarrow $ tiene solución:
            \begin{equation*}
                20x\equiv 4\mod (3) \qquad \text{mcd}(20,3)=1
            \end{equation*}
            \begin{align*}
                1 = 20(-1)+7\cdot 3 &\Longrightarrow 20\cdot 1=-1\mod (3) \\
                                    &\Longrightarrow 20(-4)\equiv 4\mod (3)
            \end{align*}
            \begin{gather*}
                x_0 = -4 \text{\ es\ solución\ particular} \\
                x_0 = 2 \text{\ es\ solución\ óptima} \\
                x_0 = 2+3k\quad k\in \mathbb{Z}
            \end{gather*}
        \item Calculamos $\text{mcd}(2+2i, 3-i)$ en $\mathbb{Q}[i]$:
            \begin{equation*}
                \dfrac{3-i}{2+2i} = \dfrac{(2-2i)(3-i)}{8} = \dfrac{6-2i-6i-2}{8}=\dfrac{4}{8}-\dfrac{8i}{8} = \dfrac{1}{2}-i
            \end{equation*}
            Tenemos $q=i$, $r = 1+i$
            \begin{equation*}
                \begin{array}{rcl}
                    r_i & u_i & v_i \\
                    3-i & 1 & 0 \\
                    2+2i & 0 & 1 \\
                    1+i & 1 & -i 
                \end{array}
            \end{equation*}
            Existe solución $\Longleftrightarrow 1+i\mid 5$, pero como $1+i\nmid 5$, no existe solución.
    \end{itemize}
\end{ejercicio}

\begin{ejercicio}
    Entre las siguientes afirmaciones relativas a ecuaciones en el anillo $\mathbb{Z}_{64}$, selecciona las que son verdad.
    \begin{itemize}
        \item \underline{$12x=28$ tiene $4$ soluciones.}
        \item $14x=28$ tiene $4$ soluciones.
        \item $12x=30$ tiene $4$ soluciones.
    \end{itemize}

    \noindent
    \textbf{Justificación}:
    \begin{itemize}
        \item 
            \begin{align*}
                12x &\equiv 28\mod(64)\\
                6x &\equiv 14\mod(32) \\
                3x &\equiv 7\mod(16)
            \end{align*}
            Como $\text{mcd}(16,3)=1$, tiene solución.
            \begin{align*}
                1 = 16\cdot 1 + 3(-5) &\Longrightarrow 3\cdot 5\equiv -1\mod(16) \\
                                      &\Longrightarrow 3\cdot 5(-7)\equiv 7\mod(16)
            \end{align*}
            \begin{gather*}
                5(-7) = -35 \text{\ es\ solución\ particular} \\
                x_0 = 13 \text{\ es\ solución\ óptima} \\
                x = 13 + 16k\quad k\in \mathbb{Z}
            \end{gather*}
            Por tanto:
            \begin{equation*}
                \begin{array}{ll}
                    x_1 = 13 & x_2 = 29 \\
                    x_3 = 45 & x_4 = 61
                \end{array}
            \end{equation*}
            Tiene 4 soluciones.
        \item 
            \begin{align*}
                14x &\equiv 28\mod(64) \\
                7x &\equiv 14\mod(32)
            \end{align*}
            $\text{mcd}(7,32)=1$, tiene solución.
            \begin{align*}
                1 = 32\cdot 2+7(-9) &\Longrightarrow 7\cdot 9\equiv -1\mod (32) \\
                                    &\Longrightarrow 7\cdot 9(-14)\equiv 14\mod(32)
            \end{align*}
            \begin{gather*}
                x_0 = 9(-14) = -126 \text{\ es\ solución\ particular} \\
                y_0 = 2 \text{\ es\ solución\ óptima} \\
                x = 2+23k\quad k\in \mathbb{Z}
            \end{gather*}
            Por tanto:
            \begin{gather*}
                x_1 = 2 \\
                x_2 = 34
            \end{gather*}
            No tiene 4 soluciones, es falso.
            
        \item 
            \begin{align*}
                12x &\equiv 30\mod(64) \\
                6x &\equiv 15\mod(32)
            \end{align*}
            \begin{equation*}
                \text{mcd}(6,32) = 2 \nmid 15 \Longrightarrow \text{\ no\ tiene\ solución}
            \end{equation*}
            Es falso.
    \end{itemize}
\end{ejercicio}

\begin{ejercicio}
    Entre las siguientes proposiciones, selecciona las verdaderas.
    \begin{itemize}
        \item \underline{El anillo $\mathbb{Z}_{900}$ tiene 240 unidades.}
        \item \underline{$14^{20}\equiv 1\mod (33)$.}
        \item $3^{16}=3$ en $\mathbb{Z}_{16}$.
    \end{itemize}

    \noindent
    \textbf{Justificación}:
    \begin{itemize}
        \item 
            \begin{equation*}
                |U(\mathbb{Z}_{900})| = \varphi(900) = \varphi(3^2 \cdot 2^2 \cdot 5^2) = 3\cdot 2\cdot 5\cdot 2\cdot 1\cdot 4 = 240
            \end{equation*}
        \item 
            \begin{equation*}
                \left.\begin{array}{r}
                    \varphi(33) = \varphi(3\cdot 11) = 2\cdot 10 = 20 \\
                    \text{mcd}(14,33) = 1
            \end{array}\right\} \mathop{\Longrightarrow}^{\text{Fermat}} 14^{20}\equiv 1\mod(33)
            \end{equation*}
        \item 
            \begin{align*}
                \left.\begin{array}{r}
                    \varphi(16) = \varphi(2^4) = 2^3 \cdot 1 =8 \\
                    \text{mcd}(3,16) = 1
            \end{array}\right\} &\Longrightarrow 3^8\equiv 1\mod (16) \Longrightarrow 3^{16}\equiv 1\mod(16) \\
            &\Longrightarrow 3^{16}\not\equiv 3\mod(16)
            \end{align*}
    \end{itemize}
\end{ejercicio}

\newpage
\begin{ejercicio}
    Sea $p$ un número primo y considérese la congruencia $ax\equiv 1\mod (p^2)$. En relación a las siguientes proposiciones, selecciona las verdaderas:
    \begin{itemize}
        \item No tiene solución, pues $p^2$ no es primo.
        \item \underline{Tiene solución si y sólo si la congruencia $ax\equiv 1\mod (p)$ tiene solución.}
        \item Tiene solución salvo que $a$ sea múltiplo de $p^2$.
    \end{itemize}

    \noindent
    \textbf{Justificación}:
    \begin{align*}
        \text{La\ equación\ tiene\ solución\ } &\Longleftrightarrow \text{mcd}(a,p^2) \mid 1 \Longleftrightarrow \text{mcd}(a,p^2) = 1 \\
                                              &\Longleftrightarrow \text{mcd}(a,p)= 1 \Longleftrightarrow ax\equiv 1\mod (p) \text{\ tiene\ solución}
    \end{align*}
    Luego la segunda opción es verdadera. Estudiamos ahora la tercera, si $a = kp^2$ con $k \in A \Longrightarrow \text{mcd}(a,p^2) = p^2$ por lo que es cierto que no tiene solución. Sin embargo, si $p^2$ es múltiplo de $a \Longrightarrow \text{mcd}(a,p^2) = a$, por lo que tampoco tiene solución.
    Luego la tercera es falsa, al existir más casos en los que no tiene solución.
\end{ejercicio}

\newpage
\resetearcontador

    \subsection{Cuestionario VIII}

\begin{ejercicio}
    En el anillo $\mathbb{Z}[i]$, selecciona las afirmaciones verdaderas:
    \begin{itemize}
        \item $2+ i$ y $2-i$ son unidades.
        \item $2+i$ y $2-i$ son asociados.
        \item $2+i$ y $2-i$ son irreducibles.
    \end{itemize}
\end{ejercicio}

\begin{ejercicio}
    Entre las siguientes afirmaciones, selecciona las afirmaciones verdaderas:
    \begin{itemize}
        \item En el anillo $\mathbb{Z}\left[\sqrt{2}\right]$, los número $2+\sqrt{2}$ y $2-\sqrt{2}$ son asociados.
        \item En el anillo $\mathbb{Z}\left[\sqrt{2}\right]$, los número $2+\sqrt{2}$ y $2-\sqrt{2}$ son primos.
        \item En el anillo $\mathbb{Z}\left[\sqrt{2}\right]$, el número 2 no es primo.
    \end{itemize}
\end{ejercicio}

\begin{ejercicio}
    Entre las siguientes afirmaciones, selecciona las correctas.
    \begin{itemize}
        \item En $\mathbb{Z}[x]$, todo polinomio de grado 1 es irreducible.
        \item En $\mathbb{Z}[x]$, todo polinomio mónico de grado menor o igual que 3 y sin raíces en $\mathbb{Z}$ es irreducible.
        \item Todo polinomio de grado mayor o igual que 1 en $\mathbb{Q}[x]$ es asociado a un primitivo de $\mathbb{Z}[x]$.
    \end{itemize}
\end{ejercicio}

\begin{ejercicio}
    Entre las siguientes afirmaciones relativas a un polinomio $f\in \mathbb{Z}[x]$, selecciona las que son verdad:
    \begin{itemize}
        \item Si el reducido $R_p(f)$ es irreducible en $\mathbb{Z}_p[x]$, entonces $f$ es irreducible.
        \item Si $f$ es mónico y el reducido $R_p(f)$ es irreducible en $\mathbb{Z}_p[x]$, entonces $f$ es irreducible.
        \item Si $f$ es primitivo y el reducido $R_p(f)$ es irreducible en $\mathbb{Z}_p[x]$, entonces $f$ es irreducible.
    \end{itemize}
\end{ejercicio}

\begin{ejercicio*}
    Entre las siguientes afirmaciones relativas a un polinomo mónimo $f\in \mathbb{Z}[x]$, selecciona las que son verdad:
    \begin{itemize}
        \item Si $f$ no tiene raíces en $\mathbb{Z}$ y para un primo entero $p\geq 2$, el reducido $R_p(f)$ factoriza en irreducibles $\mathbb{Z}_p[x]$ en la forma $R_p(f) = f_1 \cdot f_2$ con $\deg(f_1)=1$, entonces $f$ es irreducible en $\mathbb{Z}[x]$.
        \item Si para un entero primo $p\geq 2$, el reducido $R_p(f)$ factoriza en irreducibles $\mathbb{Z}_p[x]$ en la forma $R_p(f) = f_1^2$ con $\deg(f_1)=3$ y para un entero primo $q\geq 2$, el reducido $R_q(f)$ factoriza en irredcuibles $\mathbb{Z}_q[x]$ en la forma $R_q(f)=g_1g_2g_3$ con $\deg(g_1)=1=\deg(g_2)$ y $\deg(g_3)=4$, entonces $f$ es irreducible.
        \item Si para un entero primo $p\geq 2$, el reducido $R_p(f)$ factoriza en irreducibles $\mathbb{Z}_p[x]$ en la forma $R_p(f)=f_1^2$ con $\deg(f_1)=2$ y para un entero primo $q\geq 2$, el reducido $R_q(f)$ factoriza en irreducibles $\mathbb{Z}_q[x]$ en la forma $R_q(f)=g_1g_2g_3g_4$ con $\deg(g_1)=1$, entonces $f$ es irreducible.
    \end{itemize}
\end{ejercicio*}

\newpage
\ % --------------------------------------------------------------------------------
\resetearcontador


\newpage
\resetearcontador

    

\end{document}

