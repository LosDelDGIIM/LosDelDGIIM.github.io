\documentclass[12pt]{article}

% Idioma y codificación
\usepackage[spanish, es-tabla]{babel}       %es-tabla para que se titule "Tabla"
\usepackage[utf8]{inputenc}

% Márgenes
\usepackage[a4paper,top=3cm,bottom=2.5cm,left=3cm,right=3cm]{geometry}

% Comentarios de bloque
\usepackage{verbatim}

% Paquetes de links
\usepackage[hidelinks]{hyperref}    % Permite enlaces
\usepackage{url}                    % redirecciona a la web

% Más opciones para enumeraciones
\usepackage{enumitem}

% Personalizar la portada
\usepackage{titling}

% Paquetes de tablas
\usepackage{multirow}


%------------------------------------------------------------------------

%Paquetes de figuras
\usepackage{caption}
\usepackage{subcaption} % Figuras al lado de otras
\usepackage{float}      % Poner figuras en el sitio indicado H.


% Paquetes de imágenes
\usepackage{graphicx}       % Paquete para añadir imágenes
\usepackage{transparent}    % Para manejar la opacidad de las figuras

% Paquete para usar colores
\usepackage[dvipsnames]{xcolor}
\usepackage{pagecolor}      % Para cambiar el color de la página

% Habilita tamaños de fuente mayores
\usepackage{fix-cm}

% Para los gráficos
\usepackage{tikz}

% Para poder situar los nodos en los grafos
\usetikzlibrary{positioning}


%------------------------------------------------------------------------

% Paquetes de matemáticas
\usepackage{mathtools, amsfonts, amssymb, mathrsfs}
\usepackage[makeroom]{cancel}     % Simplificar tachando
\usepackage{polynom}    % Divisiones y Ruffini
\usepackage{units} % Para poner fracciones diagonales con \nicefrac

\usepackage{pgfplots}   %Representar funciones
\pgfplotsset{compat=1.18}  % Versión 1.18

\usepackage{tikz-cd}    % Para usar diagramas de composiciones
\usetikzlibrary{calc}   % Para usar cálculo de coordenadas en tikz

%Definición de teoremas, etc.
\usepackage{amsthm}
%\swapnumbers   % Intercambia la posición del texto y de la numeración

\theoremstyle{plain}

\makeatletter
\@ifclassloaded{article}{
  \newtheorem{teo}{Teorema}[section]
}{
  \newtheorem{teo}{Teorema}[chapter]  % Se resetea en cada chapter
}
\makeatother

\newtheorem{coro}{Corolario}[teo]           % Se resetea en cada teorema
\newtheorem{prop}[teo]{Proposición}         % Usa el mismo contador que teorema
\newtheorem{lema}[teo]{Lema}                % Usa el mismo contador que teorema

\theoremstyle{remark}
\newtheorem*{observacion}{Observación}

\theoremstyle{definition}

\makeatletter
\@ifclassloaded{article}{
  \newtheorem{definicion}{Definición} [section]     % Se resetea en cada chapter
}{
  \newtheorem{definicion}{Definición} [chapter]     % Se resetea en cada chapter
}
\makeatother

\newtheorem*{notacion}{Notación}
\newtheorem*{ejemplo}{Ejemplo}
\newtheorem*{ejercicio*}{Ejercicio}             % No numerado
\newtheorem{ejercicio}{Ejercicio} [section]     % Se resetea en cada section


% Modificar el formato de la numeración del teorema "ejercicio"
\renewcommand{\theejercicio}{%
  \ifnum\value{section}=0 % Si no se ha iniciado ninguna sección
    \arabic{ejercicio}% Solo mostrar el número de ejercicio
  \else
    \thesection.\arabic{ejercicio}% Mostrar número de sección y número de ejercicio
  \fi
}


% \renewcommand\qedsymbol{$\blacksquare$}         % Cambiar símbolo QED
%------------------------------------------------------------------------

% Paquetes para encabezados
\usepackage{fancyhdr}
\pagestyle{fancy}
\fancyhf{}

\newcommand{\helv}{ % Modificación tamaño de letra
\fontfamily{}\fontsize{12}{12}\selectfont}
\setlength{\headheight}{15pt} % Amplía el tamaño del índice


%\usepackage{lastpage}   % Referenciar última pag   \pageref{LastPage}
\fancyfoot[C]{\thepage}

%------------------------------------------------------------------------

% Conseguir que no ponga "Capítulo 1". Sino solo "1."
\makeatletter
\@ifclassloaded{book}{
  \renewcommand{\chaptermark}[1]{\markboth{\thechapter.\ #1}{}} % En el encabezado
    
  \renewcommand{\@makechapterhead}[1]{%
  \vspace*{50\p@}%
  {\parindent \z@ \raggedright \normalfont
    \ifnum \c@secnumdepth >\m@ne
      \huge\bfseries \thechapter.\hspace{1em}\ignorespaces
    \fi
    \interlinepenalty\@M
    \Huge \bfseries #1\par\nobreak
    \vskip 40\p@
  }}
}
\makeatother

%------------------------------------------------------------------------
% Paquetes de cógido
\usepackage{minted}
\renewcommand\listingscaption{Código fuente}

\usepackage{fancyvrb}
% Personaliza el tamaño de los números de línea
\renewcommand{\theFancyVerbLine}{\small\arabic{FancyVerbLine}}

% Estilo para C++
\newminted{cpp}{
    frame=lines,
    framesep=2mm,
    baselinestretch=1.2,
    linenos,
    escapeinside=||
}

% para minted
\definecolor{LightGray}{rgb}{0.95,0.95,0.92}
\setminted{
    linenos=true,
    stepnumber=5,
    numberfirstline=true,
    autogobble,
    breaklines=true,
    breakautoindent=true,
    breaksymbolleft=,
    breaksymbolright=,
    breaksymbolindentleft=0pt,
    breaksymbolindentright=0pt,
    breaksymbolsepleft=0pt,
    breaksymbolsepright=0pt,
    fontsize=\footnotesize,
    bgcolor=LightGray,
    numbersep=10pt
}


\usepackage{listings} % Para incluir código desde un archivo

\renewcommand\lstlistingname{Código Fuente}
\renewcommand\lstlistlistingname{Índice de Códigos Fuente}

% Definir colores
\definecolor{vscodepurple}{rgb}{0.5,0,0.5}
\definecolor{vscodeblue}{rgb}{0,0,0.8}
\definecolor{vscodegreen}{rgb}{0,0.5,0}
\definecolor{vscodegray}{rgb}{0.5,0.5,0.5}
\definecolor{vscodebackground}{rgb}{0.97,0.97,0.97}
\definecolor{vscodelightgray}{rgb}{0.9,0.9,0.9}

% Configuración para el estilo de C similar a VSCode
\lstdefinestyle{vscode_C}{
  backgroundcolor=\color{vscodebackground},
  commentstyle=\color{vscodegreen},
  keywordstyle=\color{vscodeblue},
  numberstyle=\tiny\color{vscodegray},
  stringstyle=\color{vscodepurple},
  basicstyle=\scriptsize\ttfamily,
  breakatwhitespace=false,
  breaklines=true,
  captionpos=b,
  keepspaces=true,
  numbers=left,
  numbersep=5pt,
  showspaces=false,
  showstringspaces=false,
  showtabs=false,
  tabsize=2,
  frame=tb,
  framerule=0pt,
  aboveskip=10pt,
  belowskip=10pt,
  xleftmargin=10pt,
  xrightmargin=10pt,
  framexleftmargin=10pt,
  framexrightmargin=10pt,
  framesep=0pt,
  rulecolor=\color{vscodelightgray},
  backgroundcolor=\color{vscodebackground},
}

%------------------------------------------------------------------------

% Comandos definidos
\newcommand{\bb}[1]{\mathbb{#1}}
\newcommand{\cc}[1]{\mathcal{#1}}

% I prefer the slanted \leq
\let\oldleq\leq % save them in case they're every wanted
\let\oldgeq\geq
\renewcommand{\leq}{\leqslant}
\renewcommand{\geq}{\geqslant}

% Si y solo si
\newcommand{\sii}{\iff}

% Letras griegas
\newcommand{\eps}{\epsilon}
\newcommand{\veps}{\varepsilon}
\newcommand{\lm}{\lambda}

\newcommand{\ol}{\overline}
\newcommand{\ul}{\underline}
\newcommand{\wt}{\widetilde}
\newcommand{\wh}{\widehat}

\let\oldvec\vec
\renewcommand{\vec}{\overrightarrow}

% Derivadas parciales
\newcommand{\del}[2]{\frac{\partial #1}{\partial #2}}
\newcommand{\Del}[3]{\frac{\partial^{#1} #2}{\partial #3^{#1}}}
\newcommand{\deld}[2]{\dfrac{\partial #1}{\partial #2}}
\newcommand{\Deld}[3]{\dfrac{\partial^{#1} #2}{\partial #3^{#1}}}


\newcommand{\AstIg}{\stackrel{(\ast)}{=}}
\newcommand{\Hop}{\stackrel{L'H\hat{o}pital}{=}}

\newcommand{\red}[1]{{\color{red}#1}} % Para integrales, destacar los cambios.

% Método de integración
\newcommand{\MetInt}[2]{
    \left[\begin{array}{c}
        #1 \\ #2
    \end{array}\right]
}

% Declarar aplicaciones
% 1. Nombre aplicación
% 2. Dominio
% 3. Codominio
% 4. Variable
% 5. Imagen de la variable
\newcommand{\Func}[5]{
    \begin{equation*}
        \begin{array}{rrll}
            #1:& #2 & \longrightarrow & #3\\
               & #4 & \longmapsto & #5
        \end{array}
    \end{equation*}
}

%------------------------------------------------------------------------

\usepackage{extarrows}
\usepackage{stackrel}
\usetikzlibrary{matrix} % Para divisiones de polinomios.

% En el preámbulo del documento o en tu archivo .sty
\newcounter{ejercicio}[section] % Define el contador de ejercicio y lo reinicia con cada sección

\newcounter{ejercicio}
\newcommand{\resetearcontador}{%
  \setcounter{ejercicio}{0} % Resetea el contador de ejercicios a 0
}

\renewcommand{\theejercicio}{\arabic{ejercicio}}

% COMPILAR CON: pdflatex
\begin{document}

    % 1. Foto de fondo
    % 2. Título
    % 3. Encabezado Izquierdo
    % 4. Color de fondo
    % 5. Coord x del titulo
    % 6. Coord y del titulo
    % 7. Fecha

    
    % 1. Foto de fondo
% 2. Título
% 3. Encabezado Izquierdo
% 4. Color de fondo
% 5. Coord x del titulo
% 6. Coord y del titulo
% 7. Fecha

\newcommand{\portada}[7]{

    \portadaBase{#1}{#2}{#3}{#4}{#5}{#6}{#7}
    \portadaBook{#1}{#2}{#3}{#4}{#5}{#6}{#7}
}

\newcommand{\portadaExamen}[7]{

    \portadaBase{#1}{#2}{#3}{#4}{#5}{#6}{#7}
    \portadaArticle{#1}{#2}{#3}{#4}{#5}{#6}{#7}
}




\newcommand{\portadaBase}[7]{

    % Tiene la portada principal y la licencia Creative Commons
    
    % 1. Foto de fondo
    % 2. Título
    % 3. Encabezado Izquierdo
    % 4. Color de fondo
    % 5. Coord x del titulo
    % 6. Coord y del titulo
    % 7. Fecha
    
    
    \thispagestyle{empty}               % Sin encabezado ni pie de página
    \newgeometry{margin=0cm}        % Márgenes nulos para la primera página
    
    
    % Encabezado
    \fancyhead[L]{\helv #3}
    \fancyhead[R]{\helv \nouppercase{\leftmark}}
    
    
    \pagecolor{#4}        % Color de fondo para la portada
    
    \begin{figure}[p]
        \centering
        \transparent{0.3}           % Opacidad del 30% para la imagen
        
        \includegraphics[width=\paperwidth, keepaspectratio]{assets/#1}
    
        \begin{tikzpicture}[remember picture, overlay]
            \node[anchor=north west, text=white, opacity=1, font=\fontsize{60}{90}\selectfont\bfseries\sffamily, align=left] at (#5, #6) {#2};
            
            \node[anchor=south east, text=white, opacity=1, font=\fontsize{12}{18}\selectfont\sffamily, align=right] at (9.7, 3) {\textbf{\href{https://losdeldgiim.github.io/}{Los Del DGIIM}}};
            
            \node[anchor=south east, text=white, opacity=1, font=\fontsize{12}{15}\selectfont\sffamily, align=right] at (9.7, 1.8) {Doble Grado en Ingeniería Informática y Matemáticas\\Universidad de Granada};
        \end{tikzpicture}
    \end{figure}
    
    
    \restoregeometry        % Restaurar márgenes normales para las páginas subsiguientes
    \pagecolor{white}       % Restaurar el color de página
    
    
    \newpage
    \thispagestyle{empty}               % Sin encabezado ni pie de página
    \begin{tikzpicture}[remember picture, overlay]
        \node[anchor=south west, inner sep=3cm] at (current page.south west) {
            \begin{minipage}{0.5\paperwidth}
                \href{https://creativecommons.org/licenses/by-nc-nd/4.0/}{
                    \includegraphics[height=2cm]{assets/Licencia.png}
                }\vspace{1cm}\\
                Esta obra está bajo una
                \href{https://creativecommons.org/licenses/by-nc-nd/4.0/}{
                    Licencia Creative Commons Atribución-NoComercial-SinDerivadas 4.0 Internacional (CC BY-NC-ND 4.0).
                }\\
    
                Eres libre de compartir y redistribuir el contenido de esta obra en cualquier medio o formato, siempre y cuando des el crédito adecuado a los autores originales y no persigas fines comerciales. 
            \end{minipage}
        };
    \end{tikzpicture}
    
    
    
    % 1. Foto de fondo
    % 2. Título
    % 3. Encabezado Izquierdo
    % 4. Color de fondo
    % 5. Coord x del titulo
    % 6. Coord y del titulo
    % 7. Fecha


}


\newcommand{\portadaBook}[7]{

    % 1. Foto de fondo
    % 2. Título
    % 3. Encabezado Izquierdo
    % 4. Color de fondo
    % 5. Coord x del titulo
    % 6. Coord y del titulo
    % 7. Fecha

    % Personaliza el formato del título
    \pretitle{\begin{center}\bfseries\fontsize{42}{56}\selectfont}
    \posttitle{\par\end{center}\vspace{2em}}
    
    % Personaliza el formato del autor
    \preauthor{\begin{center}\Large}
    \postauthor{\par\end{center}\vfill}
    
    % Personaliza el formato de la fecha
    \predate{\begin{center}\huge}
    \postdate{\par\end{center}\vspace{2em}}
    
    \title{#2}
    \author{\href{https://losdeldgiim.github.io/}{Los Del DGIIM}}
    \date{Granada, #7}
    \maketitle
    
    \tableofcontents
}




\newcommand{\portadaArticle}[7]{

    % 1. Foto de fondo
    % 2. Título
    % 3. Encabezado Izquierdo
    % 4. Color de fondo
    % 5. Coord x del titulo
    % 6. Coord y del titulo
    % 7. Fecha

    % Personaliza el formato del título
    \pretitle{\begin{center}\bfseries\fontsize{42}{56}\selectfont}
    \posttitle{\par\end{center}\vspace{2em}}
    
    % Personaliza el formato del autor
    \preauthor{\begin{center}\Large}
    \postauthor{\par\end{center}\vspace{3em}}
    
    % Personaliza el formato de la fecha
    \predate{\begin{center}\huge}
    \postdate{\par\end{center}\vspace{5em}}
    
    \title{#2}
    \author{\href{https://losdeldgiim.github.io/}{Los Del DGIIM}}
    \date{Granada, #7}
    \thispagestyle{empty}               % Sin encabezado ni pie de página
    \maketitle
    \vfill
}
    \portada{ffccA4.jpg}{Álgebra I}{Álgebra I}{MidnightBlue}{-8}{28}{2023-2024}{José Juan Urrutia Milán\\Arturo OlivaresMartos}

    \newpage
    \fancyhead[R]{\helv \nouppercase{\rightmark}}
    \section{Cuestionarios} 
    \subsection{Cuestionario I}
\begin{ejercicio}
    Si $A$ es un conjunto finito arbitrario, la afirmación ``$|P(A)| > |A|$'' es:
    \begin{itemize}
        \item Siempre verdadera.
        \item Verdadera o falsa, depende de $A$.
        \item Siempre falsa.
    \end{itemize}
\end{ejercicio}

\begin{ejercicio}
    Si $A$, $B$, $C$ son conjuntos cualesquira con $B$ y $C$ disjuntos, selecciona la afirmación verdadera:
    \begin{itemize}
        \item $(A \cup B)\cap C = A$.
        \item $(A \cup B)\cap (A \cup C)=A$.
        \item $(A\cap B)\cup(A \cap C)=A$.
    \end{itemize}
\end{ejercicio}

\begin{ejercicio}
    Si $A$ y $B$ son subconjuntos de un conjunto, la afirmación \newline ``$c(A) \cap c(B) = c(A \cap B)$'' es:
    \begin{itemize}
        \item Siempre cierta.
        \item Siempre falsa.
        \item A veces verdadera y a veces falsa, depende de $A$ y $B$.
    \end{itemize}
\end{ejercicio}

\begin{ejercicio}
    Sean $P$ y $Q$ las propiedades referidas a los elementos de un conjunto. Las proposiciones $P \Rightarrow \neg Q$ y $Q \Rightarrow \neg P$ son:
    \begin{itemize}
        \item Siempre equivalentes.
        \item Nunca equivalentes.
        \item A veces equivalentes y a veces no, depende de $P$ y de $Q$.
    \end{itemize}
\end{ejercicio}

\begin{ejercicio}
    Sean $P$, $Q$ y $R$ propiedades referidas a los elementos de un conjunto tal que $P \Rightarrow Q \lor R$, entonces (seleccionar la afirmación correcta):
    \begin{itemize}
        \item $P \Rightarrow Q$ y $P \Rightarrow R$.
        \item $P \Rightarrow Q$ o $P \Rightarrow R$.
        \item $P \Rightarrow Q$ siempre que $R \Rightarrow Q$.
    \end{itemize}
\end{ejercicio}

\newpage
\ % --------------------------------------------------------------------------------
\resetearcontador

\begin{ejercicio}
    Si $A$ es un conjunto finito arbitrario, la afirmación ``$|P(A)| > |A|$'' es:
    \begin{itemize}
        \item \underline{Siempre verdadera.}
        \item Verdadera o falsa, depende de $A$.
        \item Siempre falsa.
    \end{itemize}

    \noindent
    \textbf{Justificación}:
    Si $A = \emptyset$, entonces $P(A) = \{\emptyset\}$ y $|P(A)|=1>0=|A|$.\newline
    Si $A \neq \emptyset$, entonces $P(A)$ contiene a todos los subconjuntos unitarios $\{a\}$, con $a \in A$ (luego, el cardinal de $P(A)$ es, como mínimo, igual al de $|A|$) y, además, contiene el subconjunto vacío, luego tiene al menos tantos elementos como $A$ más uno.\\

    \noindent
    Otra alternativa es usar la fórmula vista para el cardinal del conjunto potencia de un conjunto finito vista en teoría:\newline
    Sea $A$ un conjunto finito arbitrario con $|A| = n \in \bb{N}$, entonces $|\mathcal{P}(A)| = 2^n$.\newline
    Notemos que $2^n > n\quad\forall n \in \bb{N}$.
\end{ejercicio}

\begin{ejercicio}
    Si $A$, $B$, $C$ son conjuntos cualesquira con $B$ y $C$ disjuntos, selecciona la afirmación verdadera:
    \begin{itemize}
        \item $(A \cup B)\cap C = A$.
        \item \underline{$(A \cup B)\cap (A \cup C)=A$.}
        \item $(A\cap B)\cup(A \cap C)=A$.
    \end{itemize}

    \noindent
    \textbf{Justificación}:
    \begin{equation*}
        (A \cup B) \cap (A \cup C) = A \cup (B \cap C) = A \cup \emptyset = A    
    \end{equation*}
\end{ejercicio}

\begin{ejercicio}
    Si $A$ y $B$ son subconjuntos de un conjunto, la afirmación \newline ``$c(A) \cap c(B) = c(A \cap B)$'' es:
    \begin{itemize}
        \item Siempre cierta.
        \item Siempre falsa.
        \item \underline{A veces verdadera y a veces falsa, depende de $A$ y $B$.}
    \end{itemize}

    \noindent
    \textbf{Justificación}:
    Por las Leyes de Morgan: $c(A \cap B) = c(A) \cup c(B)$, por lo que podemos intuir que la afirmación no siempre es cierta. Podemos dar un contraejemplo para ilustrarlo:\newline
    Sea $X = \{1,2,3,4,5\}$, sean $A = \{1,2,3\}$, $B = \{4,5\} \subseteq X$:
    \begin{gather*}
        c(A) = B\qquad c(B) = A
        c(A \cap B) = c(\emptyset) = X \neq c(A) \cap c(B) = \emptyset
    \end{gather*}
    Además, como no impone nada sobre los conjuntos, podemos ver que si $A = B$, es cierta la afirmación. Supongamos que $A = B$:
    \begin{equation*}
        c(A \cap B) = c(A \cap A) = c(A) = c(A) \cup c(A) = c(A) \cup c(B)
    \end{equation*}
\end{ejercicio}

\newpage
\begin{ejercicio}
    Sean $P$ y $Q$ las propiedades referidas a los elementos de un conjunto. Las proposiciones $P \Rightarrow \neg Q$ y $Q \Rightarrow \neg P$ son:
    \begin{itemize}
        \item \underline{Siempre equivalentes.}
        \item Nunca equivalentes.
        \item A veces equivalentes y a veces no, depende de $P$ y de $Q$.
    \end{itemize}

    \noindent
    \textbf{Justificación}:
    $Q \Rightarrow \neg P$ es el contrarrecíproco de $P \Rightarrow \neg Q$.\newline
    Demostremos que $(Q \Rightarrow \neg P) \Leftrightarrow (P \Rightarrow \neg Q)$:\newline
    O, equivalentemente, que $X_Q \subseteq c(X_P) \Leftrightarrow X_P \subseteq c(X_Q)$.
    \begin{description}
        \item [$\Rightarrow)$]
            Sea $ x \in X_P \Rightarrow x \notin c(X_P) \Rightarrow x \notin X_Q \Rightarrow x \in c(X_Q)$\newline
            Para todo $x \in X_P$, luego $X_P \subseteq c(X_Q)$.
        \item [$\Leftarrow)$]
            Sea $ x \in X_Q \Rightarrow x \notin c(X_Q) \Rightarrow x \notin X_P \Rightarrow x \in c(X_P)$\newline
            Para todo $x \in X_Q$, luego $X_Q \subseteq c(X_P)$.
    \end{description}
\end{ejercicio}

\begin{ejercicio}
    Sean $P$, $Q$ y $R$ propiedades referidas a los elementos de un conjunto tal que $P \Rightarrow Q \lor R$, entonces (seleccionar la afirmación correcta):
    \begin{itemize}
        \item $P \Rightarrow Q$ y $P \Rightarrow R$.
        \item $P \Rightarrow Q$ o $P \Rightarrow R$.
        \item \underline{$P \Rightarrow Q$ siempre que $R \Rightarrow Q$.}
    \end{itemize}

    \noindent
    \textbf{Justificación}:
    Por hipótesis, $X_P \subseteq X_Q \cup X_R$.\newline
    Si $X_R \subseteq X_Q \Rightarrow X_P \subseteq X_Q = X_Q \cup X_R$.

\end{ejercicio}

\newpage
\resetearcontador

    \section{Cuestionario II}
\begin{ejercicio}
    Sean $X$ e $Y$ dos conjuntos finitos con $|X| = |Y|$ y $f:X \rightarrow Y$ una aplicación. La afirmación ``Si $f$ es inyectiva o sobreyectiva, entonces $f$ es biyectiva'' es:
    \begin{itemize}
        \item Verdadera o falsa, depende de $f$.
        \item Siempre verdadera.
        \item Siempre falsa.
    \end{itemize}
\end{ejercicio}

\begin{ejercicio}
    Sea $f:X \rightarrow Y$ una aplicación inyectiva y sean $A, B \subseteq X$. Selecciona la afirmación verdadera:
    \begin{itemize}
        \item $f_{*}(A) - f_{*}(B)$ es un subconjunto propio de $f_{*}(A-B)$.
        \item $f_{*}(A-B)$ es un subconjunto propio de $f_{*}(A) - f_{*}(B)$.
        \item $f_{*}(A-B) = f_{*}(A) - f_{*}(B)$.
    \end{itemize}
\end{ejercicio}

\begin{ejercicio}
    Sea $f:X \rightarrow X$ una aplicación tal que $f_{*}(c(A)) = c(f_{*}(A))$, para todo $A \in \mathcal{P}(X)$. Entonces:
    \begin{itemize}
        \item $f$ es inyectiva, pero no necesariamente sobreyectiva.
        \item $f$ es sobreyectiva, pero no necesariamente inyectiva.
        \item $f$ es biyectiva.
    \end{itemize}
\end{ejercicio}

\begin{ejercicio}
Sea $X$ un conjunto con $|X|\geq 2$. La afirmación ``Todo subconjunto de $X \times X$ es de la forma $A \times B$ para ciertos subconjuntos $A, B \subseteq X$'' es:
    \begin{itemize}
        \item Verdadera o falsa, depende de $X$.
        \item Siempre verdadera.
        \item Siempre falsa.
    \end{itemize}
\end{ejercicio}

\begin{ejercicio}
    Sea $R$ una relación simétrica y transitiva en un conjunto $X \neq \emptyset$ ¿Prueba el siguiente razonamiento que $R$ es reflexiva?:\newline
    ``Por simetría, $aRb$ implica $bRa$ y entonces, por transitividad, concluimos que $aRa$''.
    \begin{itemize}
        \item Sí.
        \item No.
    \end{itemize}
\end{ejercicio}

\newpage
\ % --------------------------------------------------------------------------------
\resetearcontador

\begin{ejercicio}
    Sean $X$ e $Y$ dos conjuntos finitos con $|X| = |Y|$ y $f:X \rightarrow Y$ una aplicación. La afirmación ``Si $f$ es inyectiva o sobreyectiva, entonces $f$ es biyectiva'' es:
    \begin{itemize}
        \item Verdadera o falsa, depende de $f$.
        \item \underline{Siempre verdadera.}
        \item Siempre falsa.
    \end{itemize}

    \noindent
    \textbf{Justificación}:
    Si $f$ es inyectiva, entonces $|X| = |Img(f)|$, luego $|Img(f)| = |Y|$ y por tanto, $Img(f) = Y$ y $f$ es sobreyectiva luego biyectiva.\newline
    Si $f$ es sobreyectiva, entonces $|Y|=|Img(f)|$, luego $|Img(f)| = |X|$ y por tanto, $f$ es necesariamente inyectiva luego biyectiva.
\end{ejercicio}

\begin{ejercicio}
    Sea $f:X \rightarrow Y$ una aplicación inyectiva y sean $A, B \subseteq X$. Selecciona la afirmación verdadera:
    \begin{itemize}
        \item $f_{*}(A) - f_{*}(B)$ es un subconjunto propio de $f_{*}(A-B)$.
        \item $f_{*}(A-B)$ es un subconjunto propio de $f_{*}(A) - f_{*}(B)$.
        \item \underline{$f_{*}(A-B) = f_{*}(A) - f_{*}(B)$.}
    \end{itemize}

    \noindent
    \textbf{Justificación}:
    Empezamos recordando la definición de $f_{*}(A)$ para $A \subseteq X$:
    \begin{equation*}
        f_{*}(A) = \{y \in X \mid \exists x \in X \mbox{\ con\ } f(x) = y \}
    \end{equation*}
    \begin{description}
        \item [$\subseteq)$]
            Sea $y \in f_{*}(A-B) \Rightarrow \exists x \in A -B \mid y = f(x)$.\newline
            Esto es, $\exists x \in A \land x \notin B \mid y = f(x)$.\newline
            Como $x \in A \Rightarrow y = f(x) \in f_{*}(A)$. Además, por ser $f$ inyectiva, se tiene que $y \notin f_{*}(B)$, ya que si suponemos que $y \in f_{*}(B)$:

            $y \in f_{*}(B) \Rightarrow \exists b \in B \mid y = f(b) \Rightarrow f(x) = f(b)$ con lo que $x = b \in B$, en contradicción con que $x \notin B$.

            \noindent
            Así, $y \in f_{*}(A) - f_{*}(B)$ para todo $y \in f_{*}(A-B)$. Luego:
            \begin{equation*}
                f_{*}(A-B) \subseteq f_{*}(A) - f_{*}(B)
            \end{equation*}
        \item [$\supseteq)$]
            Sea $y \in f_{*}(A) - f_{*}(B) \Rightarrow y \in f_{*}(A) \land y \notin f_{*}(B)$.\newline
            Como $y \in f_{*}(A) \Rightarrow \exists x \in A \mid y = f(x)$.\newline
            Como $y \notin f_{*}(B) \Rightarrow x \notin B$.\newline
            Luego $x \in A -B \Rightarrow y = f(x) \in f_{*}(A-B)$ para todo $y \in f_{*}(A) - f_{*}(B)$. Luego:
            \begin{equation*}
                f_{*}(A)-f_{*}(B) \subseteq f_{*}(A-B)
            \end{equation*}
    \end{description}
\end{ejercicio}

\begin{ejercicio}
    Sea $f:X \rightarrow X$ una aplicación tal que $f_{*}(c(A)) = c(f_{*}(A))$, para todo $A \in \mathcal{P}(X)$. Entonces:
    \begin{itemize}
        \item $f$ es inyectiva, pero no necesariamente sobreyectiva.
        \item $f$ es sobreyectiva, pero no necesariamente inyectiva.
        \item \underline{$f$ es biyectiva.}
    \end{itemize}

    \noindent
    \textbf{Justificación}:
    Procedemos a demostrar la inyectividad y sobreyectividad de la aplicación.\newline
    Para la sobreyectividad, consideramos $\emptyset \in \mathcal{P}(X)$:
    \begin{equation*}
        f_{*}(c(\emptyset)) = f_{*}(X) = Img(f) = c(f_{*}(\emptyset)) = c(\emptyset) = X
    \end{equation*}
    Para la inyectividad, podemos suponer sin perder generalidad que $|X| \geq 2$ (si no lo fuera, la aplicación sería automáticamente inyectiva).\newline
    Sean $x, x' \in X \mid x \neq x'$. Entonces, $x' \in c(\{x\})$ luego:
    \begin{equation*}
        f(x') \in f_{*}(c(\{x\})) = c(\{f(x)\})
    \end{equation*}
    Luego $f(x') \neq f(x)$.
\end{ejercicio}

\begin{ejercicio}
Sea $X$ un conjunto con $|X|\geq 2$. La afirmación ``Todo subconjunto de $X \times X$ es de la forma $A \times B$ para ciertos subconjuntos $A, B \subseteq X$'' es:
    \begin{itemize}
        \item Verdadera o falsa, depende de $X$.
        \item Siempre verdadera.
        \item \underline{Siempre falsa.}
    \end{itemize}

    \noindent
    \textbf{Justificación}: Supongamos que sí y consideremos el siguiente conjunto:\newline
    Sea $D = \{(x,x) \mid x \in X\} \subseteq X \times X$.\newline
    Si $D = A \times B$ para ciertos $A, B \subseteq X$, entonces para todo $x \in X$, $(x,x) \in A \times B$ y, por tanto, $x \in A$ y $x \in B$.\newline
    Así que $A = X = B$ y, necesariamente, $D = X \times X$. Pero $|X| \geq 2$, luego existen $a,b \in X$ con $a \neq b$, esto es, $(a,b) \notin D$ y $D \neq X \times X$.\newline
    Lo que nos lleva a contradicción.
\end{ejercicio}

\begin{ejercicio}
    Sea $R$ una relación simétrica y transitiva en un conjunto $X \neq \emptyset$ ¿Prueba el siguiente razonamiento que $R$ es reflexiva?:\newline
    ``Por simetría, $aRb$ implica $bRa$ y entonces, por transitividad, concluimos que $aRa$''.
    \begin{itemize}
        \item Sí.
        \item \underline{No.}
    \end{itemize}

    \noindent
    \textbf{Justificación}:
    Dado un $a \in X$, no tiene por qué existir a priori un elemento $b \in X$ tal que $aRb$. Por tanto, buscamos un contraejemplo para desmentir la afirmación:\\

    \noindent
    Dado $X = \{ a,b,c \} \neq \emptyset$ y la relación $R = \{ (a,b), (b,a), (b,b),(a,a) \} \subseteq X \times X$. \newline Observemos que $R$ es simétrica y transitiva pero no reflexiva:

    \noindent
    Es simétrica ya que para todos $\alpha, \beta \in X \mid \alpha R \beta \Rightarrow \beta R \alpha$:
    \begin{center}
        Ya que $aRb$, ¿se cumple que $bRa$?. Sí.\\
        Ya que $bRa$, ¿se cumple que $aRb$?. Sí.\\
        Ya que $bRb$, ¿se cumple que $bRb$?. Sí.\\
        Ya que $aRa$, ¿se cumple que $aRa$?. Sí.
    \end{center}
    Es transitiva ya que para todos $\alpha, \beta, \gamma \in X \mid \alpha R \beta \land \beta R \gamma \Rightarrow \alpha R \gamma$:
    \begin{center}
        Ya que $aRb$ y $bRa$, ¿se cumple que $aRa$?. Sí.\\
        Ya que $bRa$ y $aRb$, ¿se cumple que $bRb$?. Sí.\\
        Ya que $bRb$ y $bRb$, ¿se cumple que $bRb$?. Sí.\\
        Ya que $aRa$ y $aRa$, ¿se cumple que $aRa$?. Sí.
    \end{center}
    No es reflexiva, ya que $\exists c \in X \mid c\cancel{R}c$.
\end{ejercicio}

\newpage
\resetearcontador


    \subsection{Cuestionario III}
\begin{ejercicio}
    Sea $X$ un conjunto no vacío. Definimos en $\mathcal{P}(X)$ operaciones de suma y producto por $A+B = A \cup B$ y $A \cdot B = A \cap B$. Entonces (selecciona la respuesta correcta).
    \begin{itemize}
        \item $\mathcal{P}(X)$ es un anillo conmutativo.
        \item $\mathcal{P}(X)$ no es un anillo conmutativo, falla un axioma.
        \item $\mathcal{P}(X)$ no es un anillo conmutativo, fallan dos axiomas.
    \end{itemize}
\end{ejercicio}

\begin{ejercicio}
    Para enteros $m$ y $n$ tales que $2 \leq m < n$, la afirmación ``$\bb{Z}_m$ es un subanillo de $\bb{Z}_n$'' es:
    \begin{itemize}
        \item Verdadera o falsa, dependiendo de $m$ y de $n$.
        \item Siempre verdadera.
        \item Siempre falsa.
    \end{itemize}
\end{ejercicio}

\begin{ejercicio}
    En el anillo $\bb{Z}_8$ (seleccion la afirmación verdadera).
    \begin{itemize}
        \item $3$ es una unidad y $4 \cdot 3^{-1} = 4$.
        \item $3$ es una unidad, pero $4 \cdot 3^{-1} \neq 4$.
        \item $3$ no es una unidad.
    \end{itemize}
\end{ejercicio}

\begin{ejercicio}
    En el anillo $\bb{Z}[\sqrt{3}]$, la afirmación ``${(7+4\sqrt{3})}^n$ es una unidad para todo natural $n \geq 1$'' es:
    \begin{itemize}
        \item Verdadera o falsa, dependiendo de $n$.
        \item Siempre verdadera.
        \item Siempre falsa.
    \end{itemize}
\end{ejercicio}

\begin{ejercicio}
    Sea $A \subseteq \bb{R}$ un subanillo. La afirmación ``\bb{Z} es un subanillo de A'' es:
    \begin{itemize}
        \item Siempre verdadera.
        \item Siempre falsa.
        \item Verdadera o falsa, dependiendo de $A$.
    \end{itemize}
\end{ejercicio}

\newpage
\ % --------------------------------------------------------------------------------
\resetearcontador

\begin{ejercicio}
    Sea $X$ un conjunto no vacío. Definimos en $\mathcal{P}(X)$ operaciones de suma y producto por $A+B = A \cup B$ y $A \cdot B = A \cap B$. Entonces (selecciona la respuesta correcta).
    \begin{itemize}
        \item $\mathcal{P}(X)$ es un anillo conmutativo.
        \item \underline{$\mathcal{P}(X)$ no es un anillo conmutativo, falla un axioma.}
        \item $\mathcal{P}(X)$ no es un anillo conmutativo, fallan dos axiomas.
    \end{itemize}

    \noindent
    \textbf{Justificación}:
    En este caso, $0 = \emptyset$, ya que:
    \begin{equation*}
        \emptyset + A = \emptyset \cup A = A\quad\forall A \in \mathcal{P}(X)
    \end{equation*}
    Y no hay opuestos, sea $A\neq \emptyset \in \mathcal{P}(X)$:
    \begin{equation*}
        A + B = A \cup B \supseteq A \neq \emptyset\quad\forall B \in \mathcal{P}(X)
    \end{equation*}
    Podemos ver que el resto de axiomas se cumplen:
    
    \begin{itemize}
        \item Conmutativa de la suma:
        \begin{equation*}
            A + B = A \cup B = B \cup A = B + A\quad\forall A,B \in \mathcal{P}(X)
        \end{equation*}
        \item Asociativa de la suma:
        \begin{equation*}
            A + (B + C) = A \cup (B \cup C) = (A \cup B) \cup C = (A+B)+C\quad\forall A,B,C \in \mathcal{P}(X)
        \end{equation*}
        \item Elemento neutro de la suma (ya demostrado).
        \item Existencia de opuestos (ya se ha visto que no se cumple).
        \item Conmutativa del producto:
        \begin{equation*}
            A \cdot B = A \cap B = B \cap A = B \cdot A\quad\forall A,B \in \mathcal{P}(X)
        \end{equation*}
        \item Asociativa del producto:
        \begin{equation*}
            A \cdot (B \cdot C) = A \cap (B \cap C) = (A \cap B) \cap C = (A\cdot B)\cdot C\quad\forall A,B,C \in \mathcal{P}(X)
        \end{equation*}
        \item Elemento neutro del producto:
            \begin{equation*}
                A \cdot X = A\quad\forall A \in \mathcal{P}(X)
            \end{equation*}
        \item Distributiva del producto respecto de la suma:
            \begin{equation*}
                A \cdot (B + C) = A \cap (B \cup C) = (A \cap B) \cup (A \cap C) = (A \cdot B) +(A\cdot C)\quad\forall A,B,C \in \mathcal{P}(X)
            \end{equation*}
    \end{itemize}
\end{ejercicio}

\begin{ejercicio}
    Para enteros $m$ y $n$ tales que $2 \leq m < n$, la afirmación ``$\bb{Z}_m$ es un subanillo de $\bb{Z}_n$'' es:
    \begin{itemize}
        \item Verdadera o falsa, dependiendo de $m$ y de $n$.
        \item Siempre verdadera.
        \item \underline{Siempre falsa.}
    \end{itemize}

    \noindent
    \textbf{Justificación}:
    En $\bb{Z}_m$, se tiene que $m = 0$.\newline
    Sin embargo, por ser $2 \leq m < n$, tenemos que $m \neq 0$ en $\bb{Z}_n$.
\end{ejercicio}

\begin{ejercicio}
    En el anillo $\bb{Z}_8$ (seleccion la afirmación verdadera).
    \begin{itemize}
        \item \underline{$3$ es una unidad y $4 \cdot 3^{-1} = 4$.}
        \item $3$ es una unidad, pero $4 \cdot 3^{-1} \neq 4$.
        \item $3$ no es una unidad.
    \end{itemize}

    \noindent
    \textbf{Justificación}:
    $3$ es una unidad ya que $3 \cdot 3 = 9 = 1$, luego $3^{-1} = 3$.\newline
    Entonces, $4 \cdot 3^{-1} = 4 \cdot 3 = 12 = 4$.
\end{ejercicio}

\begin{ejercicio}
    En el anillo $\bb{Z}[\sqrt{3}]$, la afirmación ``${(7+4\sqrt{3})}^n$ es una unidad para todo natural $n \geq 1$'' es:
    \begin{itemize}
        \item Verdadera o falsa, dependiendo de $n$.
        \item Siempre falsa.
        \item \underline{Siempre verdadera.}
    \end{itemize}

    \noindent
    \textbf{Justificación}:
    Tenemos que $7 + 4\sqrt{3}$ es invertible, puesto que:
    \begin{equation*}
        N(7+4\sqrt{3}) = 7^2 - 3 \cdot 16 = 49 - 48 = 1
    \end{equation*}
    Como el producto de unidades es una unidad, cualquier potencia de una unidad también lo es.
\end{ejercicio}

\begin{ejercicio}
    Sea $A \subseteq \bb{R}$ un subanillo. La afirmación ``\bb{Z} es un subanillo de A'' es:
    \begin{itemize}
        \item \underline{Siempre verdadera.}
        \item Siempre falsa.
        \item Verdadera o falsa, dependiendo de $A$.
    \end{itemize}

    \noindent
    \textbf{Justificación}:
    Por inducción, veamos primero que $\bb{N} = \bb{Z}^{+} \subseteq A$.\newline
    Esto es, que $n \in A\quad\forall n \in \bb{N}$.
    \begin{enumerate}
        \item [$n=0$:]
            Por ser $A$ subanillo de $\bb{R}$, se tiene que $0 \in A$.
        \item [$n=1$:]
            Por ser $A$ subanillo de $\bb{R}$, se tiene que $1 \in A$.
        \item [$n>1$:]
            Como hipótesis de inducción, supongamos que $n \in A$ y veamos que $n+1 \in A$.\newline
            Por ser $A$ cerrado para la suma, tenemos que $1 \in A$ y que $n \in A$ por hipótesis de inducción, luego $n+1 \in A$.
    \end{enumerate}
    Por tanto, $\bb{N} = \bb{Z}^{+} \subseteq A$.\newline
    Ahora, para $n \in \bb{Z}$ con $n \geq 0$, $A$ es cerrado para opuestos, luego $-n \in A$.\newline
    Por tanto, $\bb{Z} \subseteq A$.\\

    \noindent
    Por ser $\bb{Z}$ cerrado para la suma, producto, opuestos y contiene al $0$ y al $1$, $\bb{Z}$ es subanillo de $A$. Por tanto, $\bb{Z}$ es el menor subanillo de $\bb{R}$.
\end{ejercicio}

\newpage
\resetearcontador


    \subsection{Cuestionario IV}
\begin{ejercicio}
    En el anillo $\bb{Z}_{10}$, la afirmación ``$3^{4k+3} = -3$, para cualquier $k \in \bb{Z}$'' es:
    \begin{itemize}
        \item Siempre falsa.
        \item Siempre cierta.
        \item A veces cierta y a veces falsa, depende de $k$.
    \end{itemize}
\end{ejercicio}

\begin{ejercicio}
    En el anillo $\bb{Z}_n[x]$, la afirmación ``la suma reiterada $n$ veces de cualquier polinomio es $0$'', es:
    \begin{itemize}
        \item Verdera o falsa, depende de $n$.
        \item Siempre falsa.
        \item Siempre verdadera.
    \end{itemize}
\end{ejercicio}

\begin{ejercicio}
    Un subanillo $A$ de un anillo $B$ se dice propio si $A \subsetneq B$. Seleccion el enunciado correcto:
    \begin{itemize}
        \item En anillo $\bb{Z}$ no tiene subanillos propios.
        \item El conjunto $A = \{ 5k \mid k \in \bb{Z} \}$ es un subanillo propio de $\bb{Z}$.
        \item El cuerpo $\bb{Q}$ no tiene subanillos propios.
    \end{itemize}
\end{ejercicio}

\begin{ejercicio}
    Homomorifismos $\phi : \bb{Z}_2 \rightarrow \bb{Z}$,
    \begin{itemize}
        \item Hay exactamente uno.
        \item Hay al menos dos.
        \item No hay ninguno.
    \end{itemize}
\end{ejercicio}

\begin{ejercicio}
    Sea $A$ un anillo comutativo, la afirmación ``Para cualesquiera indeterminadas $x$ e $y$, los anillos de polinomios $A[x]$ y $A[y]$ son isomorifos''. Es:
    \begin{itemize}
        \item Verdadera o falsa, depende de $A$.
        \item Siempre verdadera.
        \item Siempre falsa.
    \end{itemize}
\end{ejercicio}

\newpage
\ % --------------------------------------------------------------------------------
\resetearcontador

\begin{ejercicio}
    En el anillo $\bb{Z}_{10}$, la afirmación ``$3^{4k+3} = -3$, para cualquier $k \in \bb{Z}$'' es:
    \begin{itemize}
        \item Siempre falsa.
        \item \underline{Siempre cierta.}
        \item A veces cierta y a veces falsa, depende de $k$.
    \end{itemize}

    \noindent
    \textbf{Justificación}:
    \begin{equation*}
    3^{4k+3}={(3^4)}^k \cdot 3^3 = {(9 \cdot  9)}^k \cdot 9 \cdot 3 = 1^k \cdot 7 = 7\quad \forall k \in \mathbb{Z}
    \end{equation*}
\end{ejercicio}

\begin{ejercicio}
    En el anillo $\bb{Z}_n[x]$, la afirmación ``la suma reiterada $n$ veces de cualquier polinomio es $0$'', es:
    \begin{itemize}
        \item Verdera o falsa, depende de $n$.
        \item Siempre falsa.
        \item \underline{Siempre verdadera.}
    \end{itemize}

    \noindent
    \textbf{Justificación}:
    Sea $R_n:\mathbb{Z}[x]\to \mathbb{Z}_n[x]$ el homomorfismo de reducción módulo $n$. Para cualquier $f \in \mathbb{Z}_n[x]$:
    \begin{equation*}
        nf = nR_n(f) = R_n(nf) = R_n(n)R_n(f) = 0 \cdot f = 0
    \end{equation*}
\end{ejercicio}

\begin{ejercicio}
    Un subanillo $A$ de un anillo $B$ se dice propio si $A \subsetneq B$. Seleccion el enunciado correcto:
    \begin{itemize}
        \item \underline{En anillo $\bb{Z}$ no tiene subanillos propios.}
        \item El conjunto $A = \{ 5k \mid k \in \bb{Z} \}$ es un subanillo propio de $\bb{Z}$.
        \item El cuerpo $\bb{Q}$ no tiene subanillos propios.
    \end{itemize}

    \noindent
    \textbf{Justificación}:
    Si $A$ es un subanillo de $\mathbb{Z}$, entonces $1 \in A$ con lo que para todo $n \geq 0$, $\overbrace{1+\cdots+1}^{n \text{\ veces}}=n \in A$ y, como $A$ contiene a sus opuestos, entonces $\mathbb{Z}\subseteq A$. Por lo que $A = \mathbb{Z}$.
\end{ejercicio}

\begin{ejercicio}
    Homomorifismos $\phi : \bb{Z}_2 \rightarrow \bb{Z}$,
    \begin{itemize}
        \item Hay exactamente uno.
        \item Hay al menos dos.
        \item \underline{No hay ninguno.}
    \end{itemize}

    \noindent
    \textbf{Justificación}:
    Si $\phi:\mathbb{Z}_2\to \mathbb{Z}$ fuese un homomorfismo, tendríamos que:
    \begin{equation*}
        \phi(1+1) = \phi(1) + \phi(1) = 1+1 = 2
    \end{equation*}
    Pero en $\mathbb{Z}_2$, $1+1=0$ y por tanto, $\phi(1+1)=\phi(0)=0$, así que sería $0 = 2$ en $\mathbb{Z}$, lo que es una contradicción.
\end{ejercicio}

\begin{ejercicio}
    Sea $A$ un anillo comutativo, la afirmación ``Para cualesquiera indeterminadas $x$ e $y$, los anillos de polinomios $A[x]$ y $A[y]$ son isomorifos''. Es:
    \begin{itemize}
        \item Verdadera o falsa, depende de $A$.
        \item \underline{Siempre verdadera.}
        \item Siempre falsa.
    \end{itemize}

    \noindent
    \textbf{Justificación}:
    El automorfismo identidad $id_A:A \cong A$ extiende a un único homomorfismo $\phi:A[x]\to A[y]$ tal que $\phi(x)=y$. Explícitamente:
    \begin{equation*}
        \phi\left(\sum_{i=0}^{n} a_i x^i\right) = \sum_{i=0}^{n} a_i y^i
    \end{equation*}
    Claramente $\phi$ es biyectiva.
\end{ejercicio}

\newpage
\resetearcontador


    \subsection{Cuestionario V}

\begin{ejercicio}
    En relación con los anillos $\bb{Z}_6$ y $\bb{Z} \times \bb{Z}$, selecciona la afirmación correcta:
    \begin{itemize}
        \item Ambos son DI.
        \item Uno de ellos es DI, pero el otro no.
        \item Ninguno es DI.
    \end{itemize}
\end{ejercicio}

\begin{ejercicio}
    En relación a las siguientes proposiciones, referidas a los elementos de un Dominio de Integridad:
    \begin{enumerate}
        \item [(a)] $a\mid b \land a \nmid c \Rightarrow b \nmid b+c$.
        \item [(b)] $a\mid b \land a \nmid c \Rightarrow a \nmid b+c$.
    \end{enumerate}
    Selecciona la afirmación correcta:
    \begin{itemize}
        \item Ambas son verdad.
        \item Una es verdad y la otra es falsa.
        \item Ambas son falsas.
    \end{itemize}
\end{ejercicio}

\begin{ejercicio}
    Polinomios de grado uno que son unidades en el anillo de polinomios $\bb{Z}_4[x]$:
    \begin{itemize}
        \item No hay.
        \item Hay dos.
        \item Hay infinitos.
    \end{itemize}
\end{ejercicio}

\begin{ejercicio}
    En el anillo $\bb{Z}[i]$:
    \begin{itemize}
        \item $3$ es unidad.
        \item $3$ es irreducible.
        \item $3$ no es irreducible.
    \end{itemize}
\end{ejercicio}

\begin{ejercicio}
    En el anillo $\bb{Z}[i]$:
    \begin{itemize}
        \item $2$ es unidad.
        \item $2$ es irreducible.
        \item $2$ no es irreducible.
    \end{itemize}
\end{ejercicio}

\newpage
\ % --------------------------------------------------------------------------------
\resetearcontador

\begin{ejercicio}
    En relación con los anillos $\bb{Z}_6$ y $\bb{Z} \times \bb{Z}$, selecciona la afirmación correcta:
    \begin{itemize}
        \item Ambos son DI.
        \item Uno de ellos es DI, pero el otro no.
        \item \underline{Ninguno es DI.}
    \end{itemize}

    \noindent
    \textbf{Justificación}:
    \begin{itemize}
        \item En $\mathbb{Z}_6$, $2\cdot 3=0$.
        \item En $\mathbb{Z}\times \mathbb{Z}$, $(1,0)\cdot (0,1)=(0,0)$.
    \end{itemize}
\end{ejercicio}

\begin{ejercicio}
    En relación a las siguientes proposiciones, referidas a los elementos de un Dominio de Integridad:
    \begin{enumerate}
        \item [(a)] $a\mid b \land a \nmid c \Rightarrow b \nmid b+c$.
        \item [(b)] $a\mid b \land a \nmid c \Rightarrow a \nmid b+c$.
    \end{enumerate}
    Selecciona la afirmación correcta:
    \begin{itemize}
        \item Ambas son verdad.
        \item \underline{Una es verdad y la otra es falsa.}
        \item Ambas son falsas.
    \end{itemize}

    \noindent
    \textbf{Justificación}:
    \begin{itemize}
        \item La primera es cierta: si $b=ax$ y fuese $b+c=ay$, tendríamos que $c=ay-ax=a(x-y)$, así que $a\mid c$, lo que es contradictorio.
        \item La segunda es falsa: por ejemplo, en $\mathbb{Z}$, $2\nmid 1$ y $2\nmid 3$, pero $2\mid 1+3=4$.
    \end{itemize}
\end{ejercicio}

\begin{ejercicio}
    Polinomios de grado uno que son unidades en el anillo de polinomios $\bb{Z}_4[x]$:
    \begin{itemize}
        \item No hay.
        \item \underline{Hay dos.}
        \item Hay infinitos.
    \end{itemize}

    \noindent
    \textbf{Justificación}:
    La tabla de multiplicar en $\mathbb{Z}_4$ es:
    \begin{equation*}
       \begin{array}{c|cccc}
           (\mathbb{Z}_4, \cdot) & 0 & 1 & 2 & 3 \\
           \hline
           0 & 0 & 0 & 0 & 0 \\
           1 & 0 & 1 & 2 & 3 \\
           2 & 0 & 2 & 0 & 2 \\
           3 & 0 & 3 & 2 & 1
       \end{array} 
    \end{equation*}
    Buscamos estudiar el cardinal del conjunto:
    \begin{equation*}
        \left\{p \in U(\mathbb{Z}_4[x]) \mid \deg(p) =1\right\}
    \end{equation*}
    Sea $ax+b \in U(\mathbb{Z}_4[x])$ con $a\neq 0$:
    \begin{align*}
        (ax+b)(ax+b) &= 1 \Longrightarrow {(ax+b)}^{2}=1 \Longrightarrow a^2x + 2abx + b^2 = 1 \\
                     &\Longrightarrow a^2 = 0 \quad\land\quad 2ab = 0 \quad\land\quad b^2 = 1
    \end{align*}
    \begin{equation*}
        \left\{\begin{array}{lll}
                a^2 = 0 & \Longrightarrow & a = 2 \\
                2ab = 0 & \Longrightarrow & 4b = 0 \Longrightarrow 0b = 0 \Longrightarrow 0=0 \\
                b^2 = 1 & \Longrightarrow & b = 1 \quad\lor\quad b = 3
        \end{array}\right.
    \end{equation*}
    Luego:
    \begin{gather*}
        2x+1 \in U(\mathbb{Z}_4[x]) \\
        2x+3 \in U(\mathbb{Z}_4[x])
    \end{gather*}
    Tenemos dos polinomios que verifican la segunda opción. Además, la última no puede ser por ser $\mathbb{Z}_4[x]$ finito.
\end{ejercicio}

\begin{ejercicio}
    En el anillo $\bb{Z}[i]$:
    \begin{itemize}
        \item $3$ es unidad.
        \item \underline{$3$ es irreducible.}
        \item $3$ no es irreducible.
    \end{itemize}

    \noindent
    \textbf{Justificación}:
    \begin{equation*}
        N(3) = 9 \neq \pm 1 \Longrightarrow 3 \notin U(\mathbb{Z}[i])
    \end{equation*}
    Para probar que $3$ es irreducible, supongamos una factorización $3=\alpha \cdot \beta$ con $\alpha, \beta \in \mathbb{Z}[i]\setminus U(\mathbb{Z}[i])$. Entonces:
    \begin{equation*}
        N(3) = N(\alpha)N(\beta) \Longrightarrow 9 = N(\alpha)N(\beta) \quad N(\alpha), N(\beta) \in \mathbb{Z}
    \end{equation*}
    Como $\alpha, \beta \notin U(\mathbb{Z}[i]) \Longrightarrow N(\alpha), N(\beta)\neq \pm 1$
    Como $\alpha, \beta \in \mathbb{Z}[i]$, se tiene que:
    \begin{align*}
        N(\alpha) &= a^2 + b^2 \geq 1 \\
        N(\beta) &= {(a')}^{2} + {(b')}^{2} \geq 1
    \end{align*}
    Por tanto, $N(\alpha), N(\beta) \in \bb{N}$. Además, $9=N(\alpha)N(\beta)\Longrightarrow N(\alpha)=N(\beta)=3$.
    \begin{equation*}
        N(\alpha) = 3 \Longrightarrow a^2 + b^2 = 3
    \end{equation*}
    Pero $\nexists a,b \in \mathbb{Z} \mid a^2 + b^2 = 3$, por lo que 3 es irreducible.
\end{ejercicio}

\begin{ejercicio}
    En el anillo $\bb{Z}[i]$:
    \begin{itemize}
        \item $2$ es unidad.
        \item $2$ es irreducible.
        \item \underline{$2$ no es irreducible.}
    \end{itemize}

    \noindent
    \textbf{Justificación}:
    \begin{equation*}
        N(2) = 4 \neq 1 \Longrightarrow 2 \notin U(\mathbb{Z}[i])
    \end{equation*}
    Para ver que 2 no es irreducible, supongamos una factorización: $2=\alpha \cdot \beta \mid \alpha, \beta \in \mathbb{Z}[i]\setminus U(\mathbb{Z}[i])$.
    \begin{equation*}
        N(2) = N(\alpha \beta) \Longrightarrow 4=N(\alpha)N(\beta) \Longrightarrow N(\alpha) = N(\beta) = 2
    \end{equation*}
    Por ejemplo, $\alpha = \beta = 1+i$
    \begin{equation*}
        -i{(1+i)}^{2} = (1+i^2 + 2i)(-i) = (-i)(1-1+2i) = (-i)2i = -2i^2 = 2
    \end{equation*}
    Luego $2 = -i{(1+i)}^{2}$ es la factorización esencialmente única de 2 $\Longrightarrow$ es reducible.
\end{ejercicio}
\newpage
\resetearcontador

    \subsection{Cuestionario VI}

\begin{ejercicio}
    En relación a las siguientes proposiciones, referidas a elementos cualesquiera de un DI, selecciona las verdaderas:
    \begin{itemize}
        \item $c\mid ab \Longrightarrow c\mid a \lor c\mid b$.
        \item $a\mid c \land b\mid c \Longrightarrow ab\mid c$. 
        \item $c\mid a \lor c\mid b \Longrightarrow c\mid ab$.
    \end{itemize}
\end{ejercicio}

\begin{ejercicio}
    Entre los siguientes DE, selecciona aquellos en los que el máximo común divisor y el mínimo común múltiplo son únicos salvo signo:
    \begin{itemize}
        \item $\mathbb{Z}\left[\sqrt{-2}\right]$.
        \item $\mathbb{Z}\left[\sqrt{3}\right]$. 
        \item $\mathbb{Z}_3[x]$.
    \end{itemize}
\end{ejercicio}

\begin{ejercicio}
    En un DE, tenemos la ecuación diofántica $px+by=1$, donde $p$ es irreducible. Entre las siguientes afirmaciones, selecciona la que es verdad.
    \begin{itemize}
        \item Nunca tiene solución.
        \item Puede tener solución o no, depende de $b$. 
        \item Siempre tiene solución.
    \end{itemize}
\end{ejercicio}

\begin{ejercicio}
    En un DE, tenemos la ecuación diofántica $px+qy=c$, donde $p$ y $q$ son irreducibles no asociados entre sí. Entre las siguientes afirmaciones, selecciona la que es verdad.
    \begin{itemize}
        \item Nunca tiene solución.
        \item Puede tener solución o no, depende de $p$ y de $q$. 
        \item Siempre tiene solución.
    \end{itemize}
\end{ejercicio}

\begin{ejercicio}
    Entre las siguientes proposiciones, referidas a un DE, selecciona las verdaderas.
    \begin{itemize}
        \item Si la ecuación $ax+by=1$ tiene solución, entonces la ecuación $ax+by=c$ tiene solución para todo $c$.
        \item Si la ecuacióin $ax+bb'y=1$ tiene solución, entonces las ecuaciones $ax+by=1$ y $ax+b'y=1$ tienen solución. 
        \item Si las ecuaciones $ax+by=1$ y $ax+b'y=1$ tienen solución, entonces la ecuación $ax+bb'y=1$ tiene solución.
    \end{itemize}
\end{ejercicio}

\newpage
\ % --------------------------------------------------------------------------------
\resetearcontador

\begin{ejercicio}
    En relación a las siguientes proposiciones, referidas a elementos cualesquiera de un DI, selecciona las verdaderas:
    \begin{itemize}
        \item $c\mid ab \Longrightarrow c\mid a \lor c\mid b$.
        \item $a\mid c \land b\mid c \Longrightarrow ab\mid c$. 
        \item \underline{$c\mid a \lor c\mid b \Longrightarrow c\mid ab$.}
    \end{itemize}

    \noindent
    \textbf{Justificación}:
    \begin{itemize}
        \item La primera es falsa, en $\mathbb{Z}$, $6\mid 12 = 4 \cdot 3$ pero $6\nmid 4$.
        \item La segunda es falsa, en $\mathbb{Z}$, $2\mid 6$ pero $2 \cdot 2 \nmid 6$.
        \item La tercera es verdadera. De hecho, basta con que $c$ divida a uno de ellos para que divida al producto:
            \begin{equation*}
                a = ca' \Longrightarrow ab=c(a'b)
            \end{equation*}
    \end{itemize}
\end{ejercicio}

\begin{ejercicio}
    Entre los siguientes DE, selecciona aquellos en los que el máximo común divisor y el mínimo común múltiplo son únicos salvo signo:
    \begin{itemize}
        \item \underline{$\mathbb{Z}\left[\sqrt{-2}\right]$.}
        \item $\mathbb{Z}\left[\sqrt{3}\right]$. 
        \item \underline{$\mathbb{Z}_3[x]$.}
    \end{itemize}

    \noindent
    \textbf{Justificación}:
    Serán aquellos cuyas unidades sean $\pm 1$:
    \begin{itemize}
        \item En $\mathbb{Z}\left[\sqrt{-2}\right]$, $a+b\sqrt{-2}$ es unidad si y sólo si $a^2 + 2b^2 =1$, lo que sólo se verifica si $a=1$ y $b=0$.
        \item En $\mathbb{Z}\left[\sqrt{3}\right]$, $a+b\sqrt{3}$ es unidad si y sólo si $a^2 - 3b^2 =\pm 1$, lo que verifica por ejemplo $2+\sqrt{3}\neq \pm 1$, luego aquí el mcd y el mcm no son únicos salvo signo.
        \item En $\mathbb{Z}_3[x]$:
            \begin{equation*}
                U\left(\mathbb{Z}_3[x]\right) = U\left(\mathbb{Z}_3\right) = \{1,2\} = \{ 1, -1 \} = \{ \pm 1\}
            \end{equation*}
    \end{itemize}
\end{ejercicio}

\begin{ejercicio}
    En un DE, tenemos la ecuación diofántica $px+by=1$, donde $p$ es irreducible. Entre las siguientes afirmaciones, selecciona la que es verdad.
    \begin{itemize}
        \item Nunca tiene solución.
        \item \underline{Puede tener solución o no, depende de $b$.} 
        \item Siempre tiene solución.
    \end{itemize}

    \noindent
    \textbf{Justificación}:
    La ecuación tendrá solución $\Longleftrightarrow \text{mcd}(p,b)\mid 1 \Longleftrightarrow \text{mcd}(p,b)=1$. Como $p$ es irreducible, equivale a que $p \nmid b$, luego puede tener solución o no, dependiendo de $b$:
    \begin{itemize}
        \item Para $b=1$ sí tiene solución.
        \item Pero para $b=2p \Longrightarrow \text{mcd}(p,2p)=p\neq 1$ no tiene solución.
    \end{itemize}
\end{ejercicio}

\begin{ejercicio}
    En un DE, tenemos la ecuación diofántica $px+qy=c$, donde $p$ y $q$ son irreducibles no asociados entre sí. Entre las siguientes afirmaciones, selecciona la que es verdad.
    \begin{itemize}
        \item Nunca tiene solución.
        \item Puede tener solución o no, depende de $p$ y de $q$. 
        \item \underline{Siempre tiene solución.}
    \end{itemize}

    \noindent
    \textbf{Justificación}:
    La ecuación tendrá solución $\Longleftrightarrow \text{mcd}(p,q)\mid c$. Como $p$ y $q$ son irreducibles no asociados, tenemos que $\text{mcd}(p,q)=1$ y como $1\mid c$ $\forall c \in A$, la ecuación siempre tendrá solución.
\end{ejercicio}

\begin{ejercicio}
    Entre las siguientes proposiciones, referidas a un DE, selecciona las verdaderas.
    \begin{itemize}
    \item \underline{Si la ecuación $ax+by=1$ tiene solución, entonces la ecuación $ax+by=c$} \newline
        \underline{ tiene solución para todo $c$.}
\item \underline{Si la ecuacióin $ax+bb'y=1$ tiene solución, entonces las ecuaciones $ax+by=1$}
    \underline{ y $ax+b'y=1$ tienen solución.} 
\item \underline{Si las ecuaciones $ax+by=1$ y $ax+b'y=1$ tienen solución, entonces la }\newline
    \underline{ecuación $ax+bb'y=1$ tiene solución.}
    \end{itemize}

    \noindent
    \textbf{Justificación}:
    \begin{itemize}
        \item Sea $(x_0,y_0)$ solución de $ax+by=1 \Longrightarrow (cx_0, cy_0)$ es solución de $ax+by=c$.
        \item Sea $(x_0,y_0)$ solución de $ax+bb'y=1 \Longrightarrow (x_0, y_0b')$ es solución de $ax+by=1$ y $(x_0, y_0b)$ es solución de $ax+b'y=1$.
        \item 
            \begin{equation*}
                \left.
                    \begin{array}{lcr}
                        ax+by=1 \text{\ tiene\ solución} & \Longrightarrow & \text{mcd}(a,b)=1 \\
                        ax+b'y=1 \text{\ tiene\ solución} & \Longrightarrow & \text{mcd}(a,b')=1
                    \end{array}
                \right\} \Longrightarrow \text{mcd}(a,bb')=1
            \end{equation*}
            Luego $ax+bb'y=1$ tiene solución. 
    \end{itemize}
\end{ejercicio}

\newpage
\resetearcontador

    \section{Cuestionario VII}

\begin{ejercicio}
    En relación a las siguientes proposiciones sobre elementos de un DE, selecciona las verdaderas:
    \begin{itemize}
        \item Si $\text{mcd}(a,b)=1$, entonces $\text{mcd}(a,b^n)=1$ para todo $n \in \mathbb{N}$.
        \item Si $a \equiv a'\mod(b)$, entonces $\text{mcd}(a,b)=\text{mcd}(a',b)$.
        \item Si $a\equiv a'\mod(b)$, entonces $\text{mcm}(a,b)=\text{mcm}(a',b)$.
    \end{itemize}
\end{ejercicio}

\begin{ejercicio}
    Entre las siguientes ecuaciones en congruencias, selecciona las que tienen solución.
    \begin{itemize}
        \item En $\mathbb{Z}$, $6x\equiv 10 \mod (45)$.
        \item En $\mathbb{Z}$, $100x\equiv 20\mod (15)$.
        \item En $\mathbb{Z}[i]$, $(2+2i)x\equiv 5\mod(3-i)$.
    \end{itemize}
\end{ejercicio}

\begin{ejercicio}
    Entre las siguientes afirmaciones relativas a ecuaciones en el anillo $\mathbb{Z}_{64}$, selecciona las que son verdad.
    \begin{itemize}
        \item $12x=28$ tiene $4$ soluciones.
        \item $14x=28$ tiene $4$ soluciones.
        \item $12x=30$ tiene $4$ soluciones.
    \end{itemize}
\end{ejercicio}

\begin{ejercicio}
    Entre las siguientes proposiciones, selecciona las verdaderas.
    \begin{itemize}
        \item El anillo $\mathbb{Z}_{900}$ tiene 240 unidades.
        \item $14^{20}\equiv 1\mod (33)$.
        \item $3^{16}=3$ en $\mathbb{Z}_{16}$.
    \end{itemize}
\end{ejercicio}

\begin{ejercicio}
    Sea $p$ un número primo y considérese la congruencia $ax\equiv 1\mod (p^2)$. En relación a las siguientes proposiciones, selecciona las verdaderas:
    \begin{itemize}
        \item No tiene solución, pues $p^2$ no es primo.
        \item Tiene solución si y sólo si la congruencia $ax\equiv 1\mod (p)$ tiene solución.
        \item Tiene solución salvo que $a$ sea múltiplo de $p^2$.
    \end{itemize}
\end{ejercicio}

\newpage
\ % --------------------------------------------------------------------------------
\resetearcontador

\begin{ejercicio}
    En relación a las siguientes proposiciones sobre elementos de un DE, selecciona las verdaderas:
    \begin{itemize}
        \item \underline{Si $\text{mcd}(a,b)=1$, entonces $\text{mcd}(a,b^n)=1$ para todo $n \in \mathbb{N}$.}
        \item \underline{Si $a \equiv a'\mod(b)$, entonces $\text{mcd}(a,b)=\text{mcd}(a',b)$.}
        \item Si $a\equiv a'\mod(b)$, entonces $\text{mcm}(a,b)=\text{mcm}(a',b)$.
    \end{itemize}

    \noindent
    \textbf{Justificación}:
    \begin{itemize}
        \item Es cierto, lo probamos por inducción:
            \begin{description}
                \item [Para $n=0$:] 
                    $\text{mcd}(a,b^0) = \text{mcd}(a,1)=1$, cierto.
                \item [Para $n=1$:] 
                    $\text{mcd}(a,b)=1$, cierto.
                \item [Supuesto cierto para $n-1$, lo vemos para $n$:] 
                    \begin{equation*}
                        \left.\begin{array}{r}
                            \text{mcd}(a,b)=1 \\
                            \text{mcd}(a,b^{n-1}) = 1
                    \end{array}\right\} \text{mcd}(a,b^n) = \text{mcd}(a,b^{n-1}b) = 1
                    \end{equation*}
            \end{description}
        \item Es cierto, sea $A$ el DE:
            \begin{align*}
                a\equiv a'\mod(b) &\Longrightarrow \exists q\in A \mid a-a' = qb \\
                                  &\Longrightarrow  a'=a-qb
            \end{align*}
            \begin{equation*}
                \text{mcd}(a,b) = \text{mcd}(a-qb,b) = \text{mcd}(a',b)
            \end{equation*}
        \item Es falso, por ejemplo en $\mathbb{Z}$, sean $a=6$, $a' = 2$, $b = 4$
            \begin{gather*}
                6\equiv 2\mod (4) \\
                \text{mcm}(6,4) = 12 \neq 4 = \text{mcm}(2,4)
            \end{gather*}
    \end{itemize}
\end{ejercicio}

\begin{ejercicio}
    Entre las siguientes ecuaciones en congruencias, selecciona las que tienen solución.
    \begin{itemize}
        \item En $\mathbb{Z}$, $6x\equiv 10 \mod (45)$.
        \item \underline{En $\mathbb{Z}$, $100x\equiv 20\mod (15)$.}
        \item En $\mathbb{Z}[i]$, $(2+2i)x\equiv 5\mod(3-i)$.
    \end{itemize}

    \noindent
    \textbf{Justificación}:
    \begin{itemize}
        \item $\text{mcd}(6,45)=3$, como $3\nmid 10 \Longrightarrow$ no tiene solución.
        \item $\text{mcd}(100,15)=5$, como $5\mid 20 \Longrightarrow $ tiene solución:
            \begin{equation*}
                20x\equiv 4\mod (3) \qquad \text{mcd}(20,3)=1
            \end{equation*}
            \begin{align*}
                1 = 20(-1)+7\cdot 3 &\Longrightarrow 20\cdot 1=-1\mod (3) \\
                                    &\Longrightarrow 20(-4)\equiv 4\mod (3)
            \end{align*}
            \begin{gather*}
                x_0 = -4 \text{\ es\ solución\ particular} \\
                x_0 = 2 \text{\ es\ solución\ óptima} \\
                x_0 = 2+3k\quad k\in \mathbb{Z}
            \end{gather*}
        \item Calculamos $\text{mcd}(2+2i, 3-i)$ en $\mathbb{Q}[i]$:
            \begin{equation*}
                \dfrac{3-i}{2+2i} = \dfrac{(2-2i)(3-i)}{8} = \dfrac{6-2i-6i-2}{8}=\dfrac{4}{8}-\dfrac{8i}{8} = \dfrac{1}{2}-i
            \end{equation*}
            Tenemos $q=i$, $r = 1+i$
            \begin{equation*}
                \begin{array}{rcl}
                    r_i & u_i & v_i \\
                    3-i & 1 & 0 \\
                    2+2i & 0 & 1 \\
                    1+i & 1 & -i 
                \end{array}
            \end{equation*}
            Existe solución $\Longleftrightarrow 1+i\mid 5$, pero como $1+i\nmid 5$, no existe solución.
    \end{itemize}
\end{ejercicio}

\begin{ejercicio}
    Entre las siguientes afirmaciones relativas a ecuaciones en el anillo $\mathbb{Z}_{64}$, selecciona las que son verdad.
    \begin{itemize}
        \item \underline{$12x=28$ tiene $4$ soluciones.}
        \item $14x=28$ tiene $4$ soluciones.
        \item $12x=30$ tiene $4$ soluciones.
    \end{itemize}

    \noindent
    \textbf{Justificación}:
    \begin{itemize}
        \item 
            \begin{align*}
                12x &\equiv 28\mod(64)\\
                6x &\equiv 14\mod(32) \\
                3x &\equiv 7\mod(16)
            \end{align*}
            Como $\text{mcd}(16,3)=1$, tiene solución.
            \begin{align*}
                1 = 16\cdot 1 + 3(-5) &\Longrightarrow 3\cdot 5\equiv -1\mod(16) \\
                                      &\Longrightarrow 3\cdot 5(-7)\equiv 7\mod(16)
            \end{align*}
            \begin{gather*}
                5(-7) = -35 \text{\ es\ solución\ particular} \\
                x_0 = 13 \text{\ es\ solución\ óptima} \\
                x = 13 + 16k\quad k\in \mathbb{Z}
            \end{gather*}
            Por tanto:
            \begin{equation*}
                \begin{array}{ll}
                    x_1 = 13 & x_2 = 29 \\
                    x_3 = 45 & x_4 = 61
                \end{array}
            \end{equation*}
            Tiene 4 soluciones.
        \item 
            \begin{align*}
                14x &\equiv 28\mod(64) \\
                7x &\equiv 14\mod(32)
            \end{align*}
            $\text{mcd}(7,32)=1$, tiene solución.
            \begin{align*}
                1 = 32\cdot 2+7(-9) &\Longrightarrow 7\cdot 9\equiv -1\mod (32) \\
                                    &\Longrightarrow 7\cdot 9(-14)\equiv 14\mod(32)
            \end{align*}
            \begin{gather*}
                x_0 = 9(-14) = -126 \text{\ es\ solución\ particular} \\
                y_0 = 2 \text{\ es\ solución\ óptima} \\
                x = 2+23k\quad k\in \mathbb{Z}
            \end{gather*}
            Por tanto:
            \begin{gather*}
                x_1 = 2 \\
                x_2 = 34
            \end{gather*}
            No tiene 4 soluciones, es falso.
            
        \item 
            \begin{align*}
                12x &\equiv 30\mod(64) \\
                6x &\equiv 15\mod(32)
            \end{align*}
            \begin{equation*}
                \text{mcd}(6,32) = 2 \nmid 15 \Longrightarrow \text{\ no\ tiene\ solución}
            \end{equation*}
            Es falso.
    \end{itemize}
\end{ejercicio}

\begin{ejercicio}
    Entre las siguientes proposiciones, selecciona las verdaderas.
    \begin{itemize}
        \item \underline{El anillo $\mathbb{Z}_{900}$ tiene 240 unidades.}
        \item \underline{$14^{20}\equiv 1\mod (33)$.}
        \item $3^{16}=3$ en $\mathbb{Z}_{16}$.
    \end{itemize}

    \noindent
    \textbf{Justificación}:
    \begin{itemize}
        \item 
            \begin{equation*}
                |U(\mathbb{Z}_{900})| = \varphi(900) = \varphi(3^2 \cdot 2^2 \cdot 5^2) = 3\cdot 2\cdot 5\cdot 2\cdot 1\cdot 4 = 240
            \end{equation*}
        \item 
            \begin{equation*}
                \left.\begin{array}{r}
                    \varphi(33) = \varphi(3\cdot 11) = 2\cdot 10 = 20 \\
                    \text{mcd}(14,33) = 1
            \end{array}\right\} \mathop{\Longrightarrow}^{\text{Fermat}} 14^{20}\equiv 1\mod(33)
            \end{equation*}
        \item 
            \begin{align*}
                \left.\begin{array}{r}
                    \varphi(16) = \varphi(2^4) = 2^3 \cdot 1 =8 \\
                    \text{mcd}(3,16) = 1
            \end{array}\right\} &\Longrightarrow 3^8\equiv 1\mod (16) \Longrightarrow 3^{16}\equiv 1\mod(16) \\
            &\Longrightarrow 3^{16}\not\equiv 3\mod(16)
            \end{align*}
    \end{itemize}
\end{ejercicio}

\newpage
\begin{ejercicio}
    Sea $p$ un número primo y considérese la congruencia $ax\equiv 1\mod (p^2)$. En relación a las siguientes proposiciones, selecciona las verdaderas:
    \begin{itemize}
        \item No tiene solución, pues $p^2$ no es primo.
        \item \underline{Tiene solución si y sólo si la congruencia $ax\equiv 1\mod (p)$ tiene solución.}
        \item Tiene solución salvo que $a$ sea múltiplo de $p^2$.
    \end{itemize}

    \noindent
    \textbf{Justificación}:
    \begin{align*}
        \text{La\ equación\ tiene\ solución\ } &\Longleftrightarrow \text{mcd}(a,p^2) \mid 1 \Longleftrightarrow \text{mcd}(a,p^2) = 1 \\
                                              &\Longleftrightarrow \text{mcd}(a,p)= 1 \Longleftrightarrow ax\equiv 1\mod (p) \text{\ tiene\ solución}
    \end{align*}
    Luego la segunda opción es verdadera. Estudiamos ahora la tercera, si $a = kp^2$ con $k \in A \Longrightarrow \text{mcd}(a,p^2) = p^2$ por lo que es cierto que no tiene solución. Sin embargo, si $p^2$ es múltiplo de $a \Longrightarrow \text{mcd}(a,p^2) = a$, por lo que tampoco tiene solución.
    Luego la tercera es falsa, al existir más casos en los que no tiene solución.
\end{ejercicio}

\newpage
\resetearcontador

    \subsection{Cuestionario VIII}

\begin{ejercicio}
    En el anillo $\mathbb{Z}[i]$, selecciona las afirmaciones verdaderas:
    \begin{itemize}
        \item $2+ i$ y $2-i$ son unidades.
        \item $2+i$ y $2-i$ son asociados.
        \item $2+i$ y $2-i$ son irreducibles.
    \end{itemize}
\end{ejercicio}

\begin{ejercicio}
    Entre las siguientes afirmaciones, selecciona las afirmaciones verdaderas:
    \begin{itemize}
        \item En el anillo $\mathbb{Z}\left[\sqrt{2}\right]$, los número $2+\sqrt{2}$ y $2-\sqrt{2}$ son asociados.
        \item En el anillo $\mathbb{Z}\left[\sqrt{2}\right]$, los número $2+\sqrt{2}$ y $2-\sqrt{2}$ son primos.
        \item En el anillo $\mathbb{Z}\left[\sqrt{2}\right]$, el número 2 no es primo.
    \end{itemize}
\end{ejercicio}

\begin{ejercicio}
    Entre las siguientes afirmaciones, selecciona las correctas.
    \begin{itemize}
        \item En $\mathbb{Z}[x]$, todo polinomio de grado 1 es irreducible.
        \item En $\mathbb{Z}[x]$, todo polinomio mónico de grado menor o igual que 3 y sin raíces en $\mathbb{Z}$ es irreducible.
        \item Todo polinomio de grado mayor o igual que 1 en $\mathbb{Q}[x]$ es asociado a un primitivo de $\mathbb{Z}[x]$.
    \end{itemize}
\end{ejercicio}

\begin{ejercicio}
    Entre las siguientes afirmaciones relativas a un polinomio $f\in \mathbb{Z}[x]$, selecciona las que son verdad:
    \begin{itemize}
        \item Si el reducido $R_p(f)$ es irreducible en $\mathbb{Z}_p[x]$, entonces $f$ es irreducible.
        \item Si $f$ es mónico y el reducido $R_p(f)$ es irreducible en $\mathbb{Z}_p[x]$, entonces $f$ es irreducible.
        \item Si $f$ es primitivo y el reducido $R_p(f)$ es irreducible en $\mathbb{Z}_p[x]$, entonces $f$ es irreducible.
    \end{itemize}
\end{ejercicio}

\begin{ejercicio*}
    Entre las siguientes afirmaciones relativas a un polinomo mónimo $f\in \mathbb{Z}[x]$, selecciona las que son verdad:
    \begin{itemize}
        \item Si $f$ no tiene raíces en $\mathbb{Z}$ y para un primo entero $p\geq 2$, el reducido $R_p(f)$ factoriza en irreducibles $\mathbb{Z}_p[x]$ en la forma $R_p(f) = f_1 \cdot f_2$ con $\deg(f_1)=1$, entonces $f$ es irreducible en $\mathbb{Z}[x]$.
        \item Si para un entero primo $p\geq 2$, el reducido $R_p(f)$ factoriza en irreducibles $\mathbb{Z}_p[x]$ en la forma $R_p(f) = f_1^2$ con $\deg(f_1)=3$ y para un entero primo $q\geq 2$, el reducido $R_q(f)$ factoriza en irredcuibles $\mathbb{Z}_q[x]$ en la forma $R_q(f)=g_1g_2g_3$ con $\deg(g_1)=1=\deg(g_2)$ y $\deg(g_3)=4$, entonces $f$ es irreducible.
        \item Si para un entero primo $p\geq 2$, el reducido $R_p(f)$ factoriza en irreducibles $\mathbb{Z}_p[x]$ en la forma $R_p(f)=f_1^2$ con $\deg(f_1)=2$ y para un entero primo $q\geq 2$, el reducido $R_q(f)$ factoriza en irreducibles $\mathbb{Z}_q[x]$ en la forma $R_q(f)=g_1g_2g_3g_4$ con $\deg(g_1)=1$, entonces $f$ es irreducible.
    \end{itemize}
\end{ejercicio*}

\newpage
\ % --------------------------------------------------------------------------------
\resetearcontador


\newpage
\resetearcontador


\end{document}

