\subsection{Cuestionario VIII}

\begin{ejercicio}
    En el anillo $\mathbb{Z}[i]$, selecciona las afirmaciones verdaderas:
    \begin{itemize}
        \item $2+ i$ y $2-i$ son unidades.
        \item $2+i$ y $2-i$ son asociados.
        \item $2+i$ y $2-i$ son irreducibles.
    \end{itemize}
\end{ejercicio}

\begin{ejercicio}
    Entre las siguientes afirmaciones, selecciona las afirmaciones verdaderas:
    \begin{itemize}
        \item En el anillo $\mathbb{Z}\left[\sqrt{2}\right]$, los número $2+\sqrt{2}$ y $2-\sqrt{2}$ son asociados.
        \item En el anillo $\mathbb{Z}\left[\sqrt{2}\right]$, los número $2+\sqrt{2}$ y $2-\sqrt{2}$ son primos.
        \item En el anillo $\mathbb{Z}\left[\sqrt{2}\right]$, el número 2 no es primo.
    \end{itemize}
\end{ejercicio}

\begin{ejercicio}
    Entre las siguientes afirmaciones, selecciona las correctas.
    \begin{itemize}
        \item En $\mathbb{Z}[x]$, todo polinomio de grado 1 es irreducible.
        \item En $\mathbb{Z}[x]$, todo polinomio mónico de grado menor o igual que 3 y sin raíces en $\mathbb{Z}$ es irreducible.
        \item Todo polinomio de grado mayor o igual que 1 en $\mathbb{Q}[x]$ es asociado a un primitivo de $\mathbb{Z}[x]$.
    \end{itemize}
\end{ejercicio}

\begin{ejercicio}
    Entre las siguientes afirmaciones relativas a un polinomio $f\in \mathbb{Z}[x]$, selecciona las que son verdad:
    \begin{itemize}
        \item Si el reducido $R_p(f)$ es irreducible en $\mathbb{Z}_p[x]$, entonces $f$ es irreducible.
        \item Si $f$ es mónico y el reducido $R_p(f)$ es irreducible en $\mathbb{Z}_p[x]$, entonces $f$ es irreducible.
        \item Si $f$ es primitivo y el reducido $R_p(f)$ es irreducible en $\mathbb{Z}_p[x]$, entonces $f$ es irreducible.
    \end{itemize}
\end{ejercicio}

\begin{ejercicio*}
    Entre las siguientes afirmaciones relativas a un polinomo mónimo $f\in \mathbb{Z}[x]$, selecciona las que son verdad:
    \begin{itemize}
        \item Si $f$ no tiene raíces en $\mathbb{Z}$ y para un primo entero $p\geq 2$, el reducido $R_p(f)$ factoriza en irreducibles $\mathbb{Z}_p[x]$ en la forma $R_p(f) = f_1 \cdot f_2$ con $\deg(f_1)=1$, entonces $f$ es irreducible en $\mathbb{Z}[x]$.
        \item Si para un entero primo $p\geq 2$, el reducido $R_p(f)$ factoriza en irreducibles $\mathbb{Z}_p[x]$ en la forma $R_p(f) = f_1^2$ con $\deg(f_1)=3$ y para un entero primo $q\geq 2$, el reducido $R_q(f)$ factoriza en irredcuibles $\mathbb{Z}_q[x]$ en la forma $R_q(f)=g_1g_2g_3$ con $\deg(g_1)=1=\deg(g_2)$ y $\deg(g_3)=4$, entonces $f$ es irreducible.
        \item Si para un entero primo $p\geq 2$, el reducido $R_p(f)$ factoriza en irreducibles $\mathbb{Z}_p[x]$ en la forma $R_p(f)=f_1^2$ con $\deg(f_1)=2$ y para un entero primo $q\geq 2$, el reducido $R_q(f)$ factoriza en irreducibles $\mathbb{Z}_q[x]$ en la forma $R_q(f)=g_1g_2g_3g_4$ con $\deg(g_1)=1$, entonces $f$ es irreducible.
    \end{itemize}
\end{ejercicio*}

\newpage
\ % --------------------------------------------------------------------------------
\resetearcontador


\newpage
\resetearcontador
