\subsection{Cuestionario I}
\begin{ejercicio}
    Si $A$ es un conjunto finito arbitrario, la afirmación ``$|P(A)| > |A|$'' es:
    \begin{itemize}
        \item Siempre verdadera.
        \item Verdadera o falsa, depende de $A$.
        \item Siempre falsa.
    \end{itemize}
\end{ejercicio}

\begin{ejercicio}
    Si $A$, $B$, $C$ son conjuntos cualesquira con $B$ y $C$ disjuntos, selecciona la afirmación verdadera:
    \begin{itemize}
        \item $(A \cup B)\cap C = A$.
        \item $(A \cup B)\cap (A \cup C)=A$.
        \item $(A\cap B)\cup(A \cap C)=A$.
    \end{itemize}
\end{ejercicio}

\begin{ejercicio}
    Si $A$ y $B$ son subconjuntos de un conjunto, la afirmación \newline ``$c(A) \cap c(B) = c(A \cap B)$'' es:
    \begin{itemize}
        \item Siempre cierta.
        \item Siempre falsa.
        \item A veces verdadera y a veces falsa, depende de $A$ y $B$.
    \end{itemize}
\end{ejercicio}

\begin{ejercicio}
    Sean $P$ y $Q$ las propiedades referidas a los elementos de un conjunto. Las proposiciones $P \Rightarrow \neg Q$ y $Q \Rightarrow \neg P$ son:
    \begin{itemize}
        \item Siempre equivalentes.
        \item Nunca equivalentes.
        \item A veces equivalentes y a veces no, depende de $P$ y de $Q$.
    \end{itemize}
\end{ejercicio}

\begin{ejercicio}
    Sean $P$, $Q$ y $R$ propiedades referidas a los elementos de un conjunto tal que $P \Rightarrow Q \lor R$, entonces (seleccionar la afirmación correcta):
    \begin{itemize}
        \item $P \Rightarrow Q$ y $P \Rightarrow R$.
        \item $P \Rightarrow Q$ o $P \Rightarrow R$.
        \item $P \Rightarrow Q$ siempre que $R \Rightarrow Q$.
    \end{itemize}
\end{ejercicio}

\newpage
\ % --------------------------------------------------------------------------------
\resetearcontador

\begin{ejercicio}
    Si $A$ es un conjunto finito arbitrario, la afirmación ``$|P(A)| > |A|$'' es:
    \begin{itemize}
        \item \underline{Siempre verdadera.}
        \item Verdadera o falsa, depende de $A$.
        \item Siempre falsa.
    \end{itemize}

    \noindent
    \textbf{Justificación}:
    Si $A = \emptyset$, entonces $P(A) = \{\emptyset\}$ y $|P(A)|=1>0=|A|$.\newline
    Si $A \neq \emptyset$, entonces $P(A)$ contiene a todos los subconjuntos unitarios $\{a\}$, con $a \in A$ (luego, el cardinal de $P(A)$ es, como mínimo, igual al de $|A|$) y, además, contiene el subconjunto vacío, luego tiene al menos tantos elementos como $A$ más uno.\\

    \noindent
    Otra alternativa es usar la fórmula vista para el cardinal del conjunto potencia de un conjunto finito vista en teoría:\newline
    Sea $A$ un conjunto finito arbitrario con $|A| = n \in \bb{N}$, entonces $|\mathcal{P}(A)| = 2^n$.\newline
    Notemos que $2^n > n\quad\forall n \in \bb{N}$.
\end{ejercicio}

\begin{ejercicio}
    Si $A$, $B$, $C$ son conjuntos cualesquira con $B$ y $C$ disjuntos, selecciona la afirmación verdadera:
    \begin{itemize}
        \item $(A \cup B)\cap C = A$.
        \item \underline{$(A \cup B)\cap (A \cup C)=A$.}
        \item $(A\cap B)\cup(A \cap C)=A$.
    \end{itemize}

    \noindent
    \textbf{Justificación}:
    \begin{equation*}
        (A \cup B) \cap (A \cup C) = A \cup (B \cap C) = A \cup \emptyset = A    
    \end{equation*}
\end{ejercicio}

\begin{ejercicio}
    Si $A$ y $B$ son subconjuntos de un conjunto, la afirmación \newline ``$c(A) \cap c(B) = c(A \cap B)$'' es:
    \begin{itemize}
        \item Siempre cierta.
        \item Siempre falsa.
        \item \underline{A veces verdadera y a veces falsa, depende de $A$ y $B$.}
    \end{itemize}

    \noindent
    \textbf{Justificación}:
    Por las Leyes de Morgan: $c(A \cap B) = c(A) \cup c(B)$, por lo que podemos intuir que la afirmación no siempre es cierta. Podemos dar un contraejemplo para ilustrarlo:\newline
    Sea $X = \{1,2,3,4,5\}$, sean $A = \{1,2,3\}$, $B = \{4,5\} \subseteq X$:
    \begin{gather*}
        c(A) = B\qquad c(B) = A
        c(A \cap B) = c(\emptyset) = X \neq c(A) \cap c(B) = \emptyset
    \end{gather*}
    Además, como no impone nada sobre los conjuntos, podemos ver que si $A = B$, es cierta la afirmación. Supongamos que $A = B$:
    \begin{equation*}
        c(A \cap B) = c(A \cap A) = c(A) = c(A) \cup c(A) = c(A) \cup c(B)
    \end{equation*}
\end{ejercicio}

\newpage
\begin{ejercicio}
    Sean $P$ y $Q$ las propiedades referidas a los elementos de un conjunto. Las proposiciones $P \Rightarrow \neg Q$ y $Q \Rightarrow \neg P$ son:
    \begin{itemize}
        \item \underline{Siempre equivalentes.}
        \item Nunca equivalentes.
        \item A veces equivalentes y a veces no, depende de $P$ y de $Q$.
    \end{itemize}

    \noindent
    \textbf{Justificación}:
    $Q \Rightarrow \neg P$ es el contrarrecíproco de $P \Rightarrow \neg Q$.\newline
    Demostremos que $(Q \Rightarrow \neg P) \Leftrightarrow (P \Rightarrow \neg Q)$:\newline
    O, equivalentemente, que $X_Q \subseteq c(X_P) \Leftrightarrow X_P \subseteq c(X_Q)$.
    \begin{description}
        \item [$\Rightarrow)$]
            Sea $ x \in X_P \Rightarrow x \notin c(X_P) \Rightarrow x \notin X_Q \Rightarrow x \in c(X_Q)$\newline
            Para todo $x \in X_P$, luego $X_P \subseteq c(X_Q)$.
        \item [$\Leftarrow)$]
            Sea $ x \in X_Q \Rightarrow x \notin c(X_Q) \Rightarrow x \notin X_P \Rightarrow x \in c(X_P)$\newline
            Para todo $x \in X_Q$, luego $X_Q \subseteq c(X_P)$.
    \end{description}
\end{ejercicio}

\begin{ejercicio}
    Sean $P$, $Q$ y $R$ propiedades referidas a los elementos de un conjunto tal que $P \Rightarrow Q \lor R$, entonces (seleccionar la afirmación correcta):
    \begin{itemize}
        \item $P \Rightarrow Q$ y $P \Rightarrow R$.
        \item $P \Rightarrow Q$ o $P \Rightarrow R$.
        \item \underline{$P \Rightarrow Q$ siempre que $R \Rightarrow Q$.}
    \end{itemize}

    \noindent
    \textbf{Justificación}:
    Por hipótesis, $X_P \subseteq X_Q \cup X_R$.\newline
    Si $X_R \subseteq X_Q \Rightarrow X_P \subseteq X_Q = X_Q \cup X_R$.

\end{ejercicio}

\newpage
\resetearcontador
