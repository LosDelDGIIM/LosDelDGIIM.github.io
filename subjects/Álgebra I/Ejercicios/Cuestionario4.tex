\subsection{Cuestionario IV}
\begin{ejercicio}
    En el anillo $\bb{Z}_{10}$, la afirmación ``$3^{4k+3} = -3$, para cualquier $k \in \bb{Z}$'' es:
    \begin{itemize}
        \item Siempre falsa.
        \item Siempre cierta.
        \item A veces cierta y a veces falsa, depende de $k$.
    \end{itemize}
\end{ejercicio}

\begin{ejercicio}
    En el anillo $\bb{Z}_n[x]$, la afirmación ``la suma reiterada $n$ veces de cualquier polinomio es $0$'', es:
    \begin{itemize}
        \item Verdera o falsa, depende de $n$.
        \item Siempre falsa.
        \item Siempre verdadera.
    \end{itemize}
\end{ejercicio}

\begin{ejercicio}
    Un subanillo $A$ de un anillo $B$ se dice propio si $A \subsetneq B$. Seleccion el enunciado correcto:
    \begin{itemize}
        \item En anillo $\bb{Z}$ no tiene subanillos propios.
        \item El conjunto $A = \{ 5k \mid k \in \bb{Z} \}$ es un subanillo propio de $\bb{Z}$.
        \item El cuerpo $\bb{Q}$ no tiene subanillos propios.
    \end{itemize}
\end{ejercicio}

\begin{ejercicio}
    Homomorifismos $\phi : \bb{Z}_2 \rightarrow \bb{Z}$,
    \begin{itemize}
        \item Hay exactamente uno.
        \item Hay al menos dos.
        \item No hay ninguno.
    \end{itemize}
\end{ejercicio}

\begin{ejercicio}
    Sea $A$ un anillo comutativo, la afirmación ``Para cualesquiera indeterminadas $x$ e $y$, los anillos de polinomios $A[x]$ y $A[y]$ son isomorifos''. Es:
    \begin{itemize}
        \item Verdadera o falsa, depende de $A$.
        \item Siempre verdadera.
        \item Siempre falsa.
    \end{itemize}
\end{ejercicio}

\newpage
\ % --------------------------------------------------------------------------------
\resetearcontador

% // TODO: poner soluciones
\begin{ejercicio}
    En el anillo $\bb{Z}_{10}$, la afirmación ``$3^{4k+3} = -3$, para cualquier $k \in \bb{Z}$'' es:
    \begin{itemize}
        \item Siempre falsa.
        \item Siempre cierta.
        \item A veces cierta y a veces falsa, depende de $k$.
    \end{itemize}
\end{ejercicio}

\begin{ejercicio}
    En el anillo $\bb{Z}_n[x]$, la afirmación ``la suma reiterada $n$ veces de cualquier polinomio es $0$'', es:
    \begin{itemize}
        \item Verdera o falsa, depende de $n$.
        \item Siempre falsa.
        \item Siempre verdadera.
    \end{itemize}
\end{ejercicio}

\begin{ejercicio}
    Un subanillo $A$ de un anillo $B$ se dice propio si $A \subsetneq B$. Seleccion el enunciado correcto:
    \begin{itemize}
        \item En anillo $\bb{Z}$ no tiene subanillos propios.
        \item El conjunto $A = \{ 5k \mid k \in \bb{Z} \}$ es un subanillo propio de $\bb{Z}$.
        \item El cuerpo $\bb{Q}$ no tiene subanillos propios.
    \end{itemize}
\end{ejercicio}

\begin{ejercicio}
    Homomorifismos $\phi : \bb{Z}_2 \rightarrow \bb{Z}$,
    \begin{itemize}
        \item Hay exactamente uno.
        \item Hay al menos dos.
        \item No hay ninguno.
    \end{itemize}
\end{ejercicio}

\begin{ejercicio}
    Sea $A$ un anillo comutativo, la afirmación ``Para cualesquiera indeterminadas $x$ e $y$, los anillos de polinomios $A[x]$ y $A[y]$ son isomorifos''. Es:
    \begin{itemize}
        \item Verdadera o falsa, depende de $A$.
        \item Siempre verdadera.
        \item Siempre falsa.
    \end{itemize}
\end{ejercicio}

\newpage
\resetearcontador

