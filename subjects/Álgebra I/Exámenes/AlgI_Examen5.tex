\documentclass[12pt]{article}

% Idioma y codificación
\usepackage[spanish, es-tabla]{babel}       %es-tabla para que se titule "Tabla"
\usepackage[utf8]{inputenc}

% Márgenes
\usepackage[a4paper,top=3cm,bottom=2.5cm,left=3cm,right=3cm]{geometry}

% Comentarios de bloque
\usepackage{verbatim}

% Paquetes de links
\usepackage[hidelinks]{hyperref}    % Permite enlaces
\usepackage{url}                    % redirecciona a la web

% Más opciones para enumeraciones
\usepackage{enumitem}

% Personalizar la portada
\usepackage{titling}

% Paquetes de tablas
\usepackage{multirow}


%------------------------------------------------------------------------

%Paquetes de figuras
\usepackage{caption}
\usepackage{subcaption} % Figuras al lado de otras
\usepackage{float}      % Poner figuras en el sitio indicado H.


% Paquetes de imágenes
\usepackage{graphicx}       % Paquete para añadir imágenes
\usepackage{transparent}    % Para manejar la opacidad de las figuras

% Paquete para usar colores
\usepackage[dvipsnames]{xcolor}
\usepackage{pagecolor}      % Para cambiar el color de la página

% Habilita tamaños de fuente mayores
\usepackage{fix-cm}

% Para los gráficos
\usepackage{tikz}

% Para poder situar los nodos en los grafos
\usetikzlibrary{positioning}


%------------------------------------------------------------------------

% Paquetes de matemáticas
\usepackage{mathtools, amsfonts, amssymb, mathrsfs}
\usepackage[makeroom]{cancel}     % Simplificar tachando
\usepackage{polynom}    % Divisiones y Ruffini
\usepackage{units} % Para poner fracciones diagonales con \nicefrac

\usepackage{pgfplots}   %Representar funciones
\pgfplotsset{compat=1.18}  % Versión 1.18

\usepackage{tikz-cd}    % Para usar diagramas de composiciones
\usetikzlibrary{calc}   % Para usar cálculo de coordenadas en tikz

%Definición de teoremas, etc.
\usepackage{amsthm}
%\swapnumbers   % Intercambia la posición del texto y de la numeración

\theoremstyle{plain}

\makeatletter
\@ifclassloaded{article}{
  \newtheorem{teo}{Teorema}[section]
}{
  \newtheorem{teo}{Teorema}[chapter]  % Se resetea en cada chapter
}
\makeatother

\newtheorem{coro}{Corolario}[teo]           % Se resetea en cada teorema
\newtheorem{prop}[teo]{Proposición}         % Usa el mismo contador que teorema
\newtheorem{lema}[teo]{Lema}                % Usa el mismo contador que teorema

\theoremstyle{remark}
\newtheorem*{observacion}{Observación}

\theoremstyle{definition}

\makeatletter
\@ifclassloaded{article}{
  \newtheorem{definicion}{Definición} [section]     % Se resetea en cada chapter
}{
  \newtheorem{definicion}{Definición} [chapter]     % Se resetea en cada chapter
}
\makeatother

\newtheorem*{notacion}{Notación}
\newtheorem*{ejemplo}{Ejemplo}
\newtheorem*{ejercicio*}{Ejercicio}             % No numerado
\newtheorem{ejercicio}{Ejercicio} [section]     % Se resetea en cada section


% Modificar el formato de la numeración del teorema "ejercicio"
\renewcommand{\theejercicio}{%
  \ifnum\value{section}=0 % Si no se ha iniciado ninguna sección
    \arabic{ejercicio}% Solo mostrar el número de ejercicio
  \else
    \thesection.\arabic{ejercicio}% Mostrar número de sección y número de ejercicio
  \fi
}


% \renewcommand\qedsymbol{$\blacksquare$}         % Cambiar símbolo QED
%------------------------------------------------------------------------

% Paquetes para encabezados
\usepackage{fancyhdr}
\pagestyle{fancy}
\fancyhf{}

\newcommand{\helv}{ % Modificación tamaño de letra
\fontfamily{}\fontsize{12}{12}\selectfont}
\setlength{\headheight}{15pt} % Amplía el tamaño del índice


%\usepackage{lastpage}   % Referenciar última pag   \pageref{LastPage}
\fancyfoot[C]{\thepage}

%------------------------------------------------------------------------

% Conseguir que no ponga "Capítulo 1". Sino solo "1."
\makeatletter
\@ifclassloaded{book}{
  \renewcommand{\chaptermark}[1]{\markboth{\thechapter.\ #1}{}} % En el encabezado
    
  \renewcommand{\@makechapterhead}[1]{%
  \vspace*{50\p@}%
  {\parindent \z@ \raggedright \normalfont
    \ifnum \c@secnumdepth >\m@ne
      \huge\bfseries \thechapter.\hspace{1em}\ignorespaces
    \fi
    \interlinepenalty\@M
    \Huge \bfseries #1\par\nobreak
    \vskip 40\p@
  }}
}
\makeatother

%------------------------------------------------------------------------
% Paquetes de cógido
\usepackage{minted}
\renewcommand\listingscaption{Código fuente}

\usepackage{fancyvrb}
% Personaliza el tamaño de los números de línea
\renewcommand{\theFancyVerbLine}{\small\arabic{FancyVerbLine}}

% Estilo para C++
\newminted{cpp}{
    frame=lines,
    framesep=2mm,
    baselinestretch=1.2,
    linenos,
    escapeinside=||
}

% para minted
\definecolor{LightGray}{rgb}{0.95,0.95,0.92}
\setminted{
    linenos=true,
    stepnumber=5,
    numberfirstline=true,
    autogobble,
    breaklines=true,
    breakautoindent=true,
    breaksymbolleft=,
    breaksymbolright=,
    breaksymbolindentleft=0pt,
    breaksymbolindentright=0pt,
    breaksymbolsepleft=0pt,
    breaksymbolsepright=0pt,
    fontsize=\footnotesize,
    bgcolor=LightGray,
    numbersep=10pt
}


\usepackage{listings} % Para incluir código desde un archivo

\renewcommand\lstlistingname{Código Fuente}
\renewcommand\lstlistlistingname{Índice de Códigos Fuente}

% Definir colores
\definecolor{vscodepurple}{rgb}{0.5,0,0.5}
\definecolor{vscodeblue}{rgb}{0,0,0.8}
\definecolor{vscodegreen}{rgb}{0,0.5,0}
\definecolor{vscodegray}{rgb}{0.5,0.5,0.5}
\definecolor{vscodebackground}{rgb}{0.97,0.97,0.97}
\definecolor{vscodelightgray}{rgb}{0.9,0.9,0.9}

% Configuración para el estilo de C similar a VSCode
\lstdefinestyle{vscode_C}{
  backgroundcolor=\color{vscodebackground},
  commentstyle=\color{vscodegreen},
  keywordstyle=\color{vscodeblue},
  numberstyle=\tiny\color{vscodegray},
  stringstyle=\color{vscodepurple},
  basicstyle=\scriptsize\ttfamily,
  breakatwhitespace=false,
  breaklines=true,
  captionpos=b,
  keepspaces=true,
  numbers=left,
  numbersep=5pt,
  showspaces=false,
  showstringspaces=false,
  showtabs=false,
  tabsize=2,
  frame=tb,
  framerule=0pt,
  aboveskip=10pt,
  belowskip=10pt,
  xleftmargin=10pt,
  xrightmargin=10pt,
  framexleftmargin=10pt,
  framexrightmargin=10pt,
  framesep=0pt,
  rulecolor=\color{vscodelightgray},
  backgroundcolor=\color{vscodebackground},
}

%------------------------------------------------------------------------

% Comandos definidos
\newcommand{\bb}[1]{\mathbb{#1}}
\newcommand{\cc}[1]{\mathcal{#1}}

% I prefer the slanted \leq
\let\oldleq\leq % save them in case they're every wanted
\let\oldgeq\geq
\renewcommand{\leq}{\leqslant}
\renewcommand{\geq}{\geqslant}

% Si y solo si
\newcommand{\sii}{\iff}

% Letras griegas
\newcommand{\eps}{\epsilon}
\newcommand{\veps}{\varepsilon}
\newcommand{\lm}{\lambda}

\newcommand{\ol}{\overline}
\newcommand{\ul}{\underline}
\newcommand{\wt}{\widetilde}
\newcommand{\wh}{\widehat}

\let\oldvec\vec
\renewcommand{\vec}{\overrightarrow}

% Derivadas parciales
\newcommand{\del}[2]{\frac{\partial #1}{\partial #2}}
\newcommand{\Del}[3]{\frac{\partial^{#1} #2}{\partial #3^{#1}}}
\newcommand{\deld}[2]{\dfrac{\partial #1}{\partial #2}}
\newcommand{\Deld}[3]{\dfrac{\partial^{#1} #2}{\partial #3^{#1}}}


\newcommand{\AstIg}{\stackrel{(\ast)}{=}}
\newcommand{\Hop}{\stackrel{L'H\hat{o}pital}{=}}

\newcommand{\red}[1]{{\color{red}#1}} % Para integrales, destacar los cambios.

% Método de integración
\newcommand{\MetInt}[2]{
    \left[\begin{array}{c}
        #1 \\ #2
    \end{array}\right]
}

% Declarar aplicaciones
% 1. Nombre aplicación
% 2. Dominio
% 3. Codominio
% 4. Variable
% 5. Imagen de la variable
\newcommand{\Func}[5]{
    \begin{equation*}
        \begin{array}{rrll}
            #1:& #2 & \longrightarrow & #3\\
               & #4 & \longmapsto & #5
        \end{array}
    \end{equation*}
}

%------------------------------------------------------------------------

% Operador \Res
\DeclareMathOperator{\Res}{Res}

\begin{document}

    % 1. Foto de fondo
    % 2. Título
    % 3. Encabezado Izquierdo
    % 4. Color de fondo
    % 5. Coord x del titulo
    % 6. Coord y del titulo
    % 7. Fecha
    \newcommand{\Z}{{\mathbb{Z}}} % Enteros
    \newcommand{\Q}{{\mathbb{Q}}} % Racionales
    \newcommand{\R}{{\mathbb{R}}} % Reales
    \newcommand{\C}{{\mathbb{C}}} % Complejos
    
    
    % 1. Foto de fondo
% 2. Título
% 3. Encabezado Izquierdo
% 4. Color de fondo
% 5. Coord x del titulo
% 6. Coord y del titulo
% 7. Fecha

\newcommand{\portada}[7]{

    \portadaBase{#1}{#2}{#3}{#4}{#5}{#6}{#7}
    \portadaBook{#1}{#2}{#3}{#4}{#5}{#6}{#7}
}

\newcommand{\portadaExamen}[7]{

    \portadaBase{#1}{#2}{#3}{#4}{#5}{#6}{#7}
    \portadaArticle{#1}{#2}{#3}{#4}{#5}{#6}{#7}
}




\newcommand{\portadaBase}[7]{

    % Tiene la portada principal y la licencia Creative Commons
    
    % 1. Foto de fondo
    % 2. Título
    % 3. Encabezado Izquierdo
    % 4. Color de fondo
    % 5. Coord x del titulo
    % 6. Coord y del titulo
    % 7. Fecha
    
    
    \thispagestyle{empty}               % Sin encabezado ni pie de página
    \newgeometry{margin=0cm}        % Márgenes nulos para la primera página
    
    
    % Encabezado
    \fancyhead[L]{\helv #3}
    \fancyhead[R]{\helv \nouppercase{\leftmark}}
    
    
    \pagecolor{#4}        % Color de fondo para la portada
    
    \begin{figure}[p]
        \centering
        \transparent{0.3}           % Opacidad del 30% para la imagen
        
        \includegraphics[width=\paperwidth, keepaspectratio]{assets/#1}
    
        \begin{tikzpicture}[remember picture, overlay]
            \node[anchor=north west, text=white, opacity=1, font=\fontsize{60}{90}\selectfont\bfseries\sffamily, align=left] at (#5, #6) {#2};
            
            \node[anchor=south east, text=white, opacity=1, font=\fontsize{12}{18}\selectfont\sffamily, align=right] at (9.7, 3) {\textbf{\href{https://losdeldgiim.github.io/}{Los Del DGIIM}}};
            
            \node[anchor=south east, text=white, opacity=1, font=\fontsize{12}{15}\selectfont\sffamily, align=right] at (9.7, 1.8) {Doble Grado en Ingeniería Informática y Matemáticas\\Universidad de Granada};
        \end{tikzpicture}
    \end{figure}
    
    
    \restoregeometry        % Restaurar márgenes normales para las páginas subsiguientes
    \pagecolor{white}       % Restaurar el color de página
    
    
    \newpage
    \thispagestyle{empty}               % Sin encabezado ni pie de página
    \begin{tikzpicture}[remember picture, overlay]
        \node[anchor=south west, inner sep=3cm] at (current page.south west) {
            \begin{minipage}{0.5\paperwidth}
                \href{https://creativecommons.org/licenses/by-nc-nd/4.0/}{
                    \includegraphics[height=2cm]{assets/Licencia.png}
                }\vspace{1cm}\\
                Esta obra está bajo una
                \href{https://creativecommons.org/licenses/by-nc-nd/4.0/}{
                    Licencia Creative Commons Atribución-NoComercial-SinDerivadas 4.0 Internacional (CC BY-NC-ND 4.0).
                }\\
    
                Eres libre de compartir y redistribuir el contenido de esta obra en cualquier medio o formato, siempre y cuando des el crédito adecuado a los autores originales y no persigas fines comerciales. 
            \end{minipage}
        };
    \end{tikzpicture}
    
    
    
    % 1. Foto de fondo
    % 2. Título
    % 3. Encabezado Izquierdo
    % 4. Color de fondo
    % 5. Coord x del titulo
    % 6. Coord y del titulo
    % 7. Fecha


}


\newcommand{\portadaBook}[7]{

    % 1. Foto de fondo
    % 2. Título
    % 3. Encabezado Izquierdo
    % 4. Color de fondo
    % 5. Coord x del titulo
    % 6. Coord y del titulo
    % 7. Fecha

    % Personaliza el formato del título
    \pretitle{\begin{center}\bfseries\fontsize{42}{56}\selectfont}
    \posttitle{\par\end{center}\vspace{2em}}
    
    % Personaliza el formato del autor
    \preauthor{\begin{center}\Large}
    \postauthor{\par\end{center}\vfill}
    
    % Personaliza el formato de la fecha
    \predate{\begin{center}\huge}
    \postdate{\par\end{center}\vspace{2em}}
    
    \title{#2}
    \author{\href{https://losdeldgiim.github.io/}{Los Del DGIIM}}
    \date{Granada, #7}
    \maketitle
    
    \tableofcontents
}




\newcommand{\portadaArticle}[7]{

    % 1. Foto de fondo
    % 2. Título
    % 3. Encabezado Izquierdo
    % 4. Color de fondo
    % 5. Coord x del titulo
    % 6. Coord y del titulo
    % 7. Fecha

    % Personaliza el formato del título
    \pretitle{\begin{center}\bfseries\fontsize{42}{56}\selectfont}
    \posttitle{\par\end{center}\vspace{2em}}
    
    % Personaliza el formato del autor
    \preauthor{\begin{center}\Large}
    \postauthor{\par\end{center}\vspace{3em}}
    
    % Personaliza el formato de la fecha
    \predate{\begin{center}\huge}
    \postdate{\par\end{center}\vspace{5em}}
    
    \title{#2}
    \author{\href{https://losdeldgiim.github.io/}{Los Del DGIIM}}
    \date{Granada, #7}
    \thispagestyle{empty}               % Sin encabezado ni pie de página
    \maketitle
    \vfill
}
    \portadaExamen{ffccA4.jpg}{Álgebra I\\Examen V}{Álgebra I. Examen V}{MidnightBlue}{-8}{28}{2023-2024}{Víctor Naranjo}

    \begin{description}
        \item[Asignatura] Álgebra I.
        \item[Curso Académico] 2023-2024.
        \item[Grado] Doble Grado en Ingeniería Informática y Matemáticas.
        \item[Grupo] Único.
        \item[Profesor] María del Pilar Carrasco Carrasco.
        \item[Descripción] Parcial I.
        \item[Fecha] 15 de noviembre de 2023.
        %\item[Duración] 2 horas.
    
    \end{description}
    \newpage


    % ------------------------------------
    
    \begin{ejercicio}[1.25 puntos]
        Sean $P$, $Q$ propiedades que pueden ser satisfechas, o no, por los elementos 
        de un conjunto $X$. Demostrar que se tiene la siguiente equivalencia:
        \begin{equation*}
            (P\lor\lnot Q) \land (Q \lor \lnot P) \Leftrightarrow (P \land Q) \lor \lnot(P \lor Q).
        \end{equation*}
        Si $X_P = \{x \in X \mid x$ verifica la propiedad $P\}$ y $X_Q = \{x \in X \mid x$ verifica la propiedad $Q\}$. Se trata de demosstrar que
        \begin{equation*}
            [X_P \cup c(X_Q)] \cap [X_Q \cup c(X_P)] = [X_P \cap X_Q] \cup c(X_P \cup X_Q).
        \end{equation*}
        En efecto, empezando por el miembro de la derecha
        \begin{align*}
            & [X_P \cap X_Q] \cup c(X_P \cup X_Q) = [(X_P \cap X_Q) \cup c(X_P)] \cap [(X_P \cap X_Q) \cup c(X_Q)] \\
            & = [(X_P \cup c(X_P)) \cap (X_Q \cup c(X_P))] \cap [(X_P \cup c(X_Q)) \cap (X_Q \cup c(X_Q))] \\
            & = [X \cap (X_Q \cup c(X_P))] \cap [(X_P \cup c(X_Q)) \cap X] \\
            & = [X_Q \cup c(X_P)] \cap [X_P \cup c(X_Q)] = [X_P \cup c(X_Q)] \cap [X_Q \cup c(X_P)]. \\
        \end{align*}
        Como queríamos demostrar.
    \end{ejercicio}
    \begin{ejercicio}[1.25 puntos]~
        \begin{enumerate}[label=(\roman*)]
            \item Determinar si la asignación $\nicefrac{a}{b} \mapsto a$ define una aplicación $f: \Q \rightarrow \Z$
            \item Determinar si la asignación $\nicefrac{a}{b} \mapsto \nicefrac{a^2}{b^2}$ define una aplicación $g: \Q \rightarrow \Q$
        \end{enumerate}
        \begin{enumerate}[label=(\roman*)]
            \item La asignación $\nicefrac{a}{b} \mapsto a$ no define una aplicación de $\Q$ en $\Z$ pues por ejemplo el elemento $\nicefrac{1}{2} = \nicefrac{2}{4}$ tendría dos imágenes distintas.
            \item La asignación $\nicefrac{a}{b} \mapsto \nicefrac{a^2}{b^2}$ sí define una aplicación de $\Q$ en $\Q$. En efecto, si
            \begin{equation*}
                \frac{a}{b} = \frac{c}{d} \Rightarrow ad = bc \Rightarrow a^2d^2 = (ad)^2 = (bc)^2 = b^2c^2 \Rightarrow \frac{a^2}{b^2} = \frac{c^2}{d^2}
            \end{equation*}
            Esto es, la imagen de $\nicefrac{a}{b}$ no depende del representante elegido.
        \end{enumerate}
    \end{ejercicio}
    \begin{ejercicio}[1.25 puntos]
        Sea $f : S \rightarrow T$ una aplicación. Probar que, para cualesquiera subconjuntos $A \subseteq S$ y
        $B \subseteq T$, se verifica que $f_*(A \cap f^*(B)) = f_*(A) \cap B$. $_{\text{(Recordad   que, si} \ X \ \subseteq \ S \ e \ Y \ \subseteq \ T , \ entonces \
        f_*(X) \ = \ \{f(x) \ | \ x \ \in \ X\} \ y \ f^*(Y) \ = \ \{x \ \in \  S | \ f(x) \ \in \ Y\}.)}$ \\ \\
        
        Demostraremos por doble inclusión.
        \begin{description}
            \item[$\subseteq$)] Sea $y \in f_*(A \cap f^*(B)) \Rightarrow \exists \ x \in A \cap f^*(B) \mid y =f(x)$.
            \begin{equation*}
                \text{Si } x \in A \cap f^*(B) \Rightarrow \left\{\begin{array}{l}
                    x \in A \Rightarrow y = f(x) \in f_*(A) \\
                    x \in f^*(B) \Rightarrow y = f(x) \in B\\
                \end{array}\right\} \Rightarrow y \in f_*(A) \cap B
            \end{equation*}

            \item[$\supseteq$)] Sea $y \in f_*(A) \cap B$. Entonces:
            \begin{equation*}
                \left.\begin{array}{l}
                    y \in f_*(A) \Rightarrow \exists \ x \in A \mid y = f(x) \\
                    y = f(x) \in B \Rightarrow x \in f^*(B)
                \end{array}\right\} \Rightarrow y=f(x) \text{ con } x\in A \cap f^*(B) \Rightarrow y \in f_*(A \cap f^*(B)) 
            \end{equation*}
        \end{description}
    \end{ejercicio}
    \begin{ejercicio}[1.25 puntos]
        Sea $V$ el conjunto de los vértices de un cuadrado y $A = \{f : V \rightarrow \{1, 2, 3\} \mid f $ es aplicación\}. Definimos en A la siguiente relación binaria
        \begin{equation*}
            f_1 \sim f_2 \Longleftrightarrow \text{existe una biyección } \phi : V \rightarrow V \text{ tal que } f_1 = f_2 \circ \phi 
        \end{equation*}
        Demostrar que $\sim$ es una relación de equivalencia. Describir las clases de equivalencia.
        \begin{description}
            \item[Reflexiva] Para todo $f \in A$, sabemos que $f = f \circ id_V \Rightarrow f \sim f$.
            \item[Simétrica] Sea $f_1 \sim f_2 \Rightarrow \exists \ \phi : V \rightarrow V$ biyectiva tal que $f_1 = f_2 \circ \phi$. Considerando $\phi^{-1}$ (que existe por ser $\phi$ biyectiva), se tendrá:
            \begin{equation*}
                f_1 \circ \phi^{-1} = (f_2 \circ \phi) \circ \phi^{-1} = f_2 \circ (\phi \circ \phi^{-1}) = f_2 \circ id_V = f_2 \Rightarrow f_2 \sim f_1
            \end{equation*}
            \item[Transitiva] Sean $f_1, f_2, f_3 \in A$ tales que $f_1 \sim f_2$ y $f_2 \sim f_3$. Entonces:
            \begin{equation*}
                \left.\begin{array}{l}
                     f_1 \sim f_2 \\
                     f_2 \sim f_3 
                \end{array}\right\} \Rightarrow
                \begin{array}{l}
                     \exists \ \phi: V \rightarrow V \text{ biyectiva tq } f_1 = f_2 \circ \phi \\
                     \exists \ \psi :V \rightarrow V \text{ biyectiva tq } f_2 = f_3 \circ \psi
                \end{array}
             \end{equation*}
            Entonces
            \begin{equation*}
                f_1 = f_2 \circ \psi = (f_3 \circ \psi) \circ \phi = f_3 \circ (\psi \circ \phi) \Rightarrow f_1 \sim f_3
            \end{equation*}
            pues $\psi \circ \phi:V \rightarrow V$ es biyectiva por ser composición de biyectivas.
        \end{description}~\\

        Finalmente, para $f \in A$, se tiene que su clase de equivalencia es:
        \begin{equation*}
            [f] = \{g \in A \mid g \sim f\} = \{f \circ \phi \mid \phi:V \rightarrow V \text{ es biyectiva}\}
        \end{equation*}
    \end{ejercicio}
    \begin{ejercicio}[1.25 puntos]
        Calcula el cociente y el resto de dividir el entero $-2120$ entre $19$. Calcula el resto de dividir
        por $19$ el resultado de multiplicar $4825 \cdot (-2120)$. $_{\text{(Deja constancia del procedimiento y cálculos que has usado)}}$ \\ \\
        Puesto que 
        \begin{equation*}
            2120 = 19 \cdot 111 + 11 \Rightarrow -2120 = 19(-111) - 11 = 19 \cdot (-111) -19 + 19 -11 = 19(-112) + 8
        \end{equation*}
        Por tanto el cociente es $-112$ y el resto es $8$. Como $4825 = 19 \cdot 253 + 18 \Rightarrow \Res(4825; 19) = 18$. Entonces
        \begin{equation*}
            \Res(4825 \cdot (-2120); 19) = \Res(18 \cdot 8; 19) = \Res(144; 19) = 11, \text{ pues } 144 = 19 \cdot 7 +11
        \end{equation*}
    \end{ejercicio}
    \begin{ejercicio}[1.25 puntos]
        Determina todas las unidades del anillo $\Z[\sqrt{-5}]$. Calcula también el inverso de $\frac{1}{2} + \frac{3}{5}\sqrt{-5}$ en el cuerpo $\Q[\sqrt{-5}]$. $_{\text{(Deja constancia del procedimiento y }}$ \\  $_{\text{cálculos que has usado)}}$ \\ \\
        En $\Z[\sqrt{-5}]$, la norma de $\alpha = a + b \sqrt{-5}$ es $N(\alpha) = a^2 + 5b^2$. Entonces
        \begin{equation*}
            a+b\sqrt{-5} \in U(\Z[\sqrt{-5}]) \Leftrightarrow N(\alpha) = a^2 + 5b^2 = 1 \Leftrightarrow a \pm 1 \land b=0
        \end{equation*}
        Por tanto $U(\Z[\sqrt{-5}]) = \{1, -1\}$. \\ \\
        Sabemos que $\Q[\sqrt{-5}]$ es un cuerpo. Para $\alpha \in \Q[\sqrt{-5}], \alpha \neq 0,$ su inverso es $\alpha^{-1} = \frac{1}{N(\alpha)}\bar{\alpha}$. Para $\alpha = \frac{1}{2} + \frac{3}{5}\sqrt{-5} \neq 0, N(\alpha) = \frac{1}{4} + 5\frac{9}{25} = \frac{1}{4} + \frac{9}{5} = \frac{41}{20}$. Entonces
        \begin{equation*}
            \alpha^{-1} = \frac{20}{41}(\frac{1}{2} - \frac{3}{5}\sqrt{-5}) = \frac{10}{41} - \frac{12}{41}\sqrt{-5}
        \end{equation*}
    \end{ejercicio}
    \begin{ejercicio}[1.25 puntos]
        Sean $n, m \geq 2$ y consíderese el conjunto $U = U(\Z_n \times \Z_m)$ de las unidades del anillo producto cartesiano. La afirmación "$|U|= (n-1)(m-1)$" \ es
        \begin{enumerate}[label=$\square$]
            \item siempre cierta,
            \item siempre falsa
            \item a veces verdad y a veces falsa, dependiendo de $n$, $m$.
        \end{enumerate}
        $_{\text{Justificar la respuesta.}}$ \\ \\
        Para $n = 2$ y $m = 3$
        \begin{equation*}
            U = U(\Z_2 \times \Z_3) = U(\Z_2) \times U(\Z_3) = \{1\} \times \{1, 2\} = \{(1, 1), (1, 2)\}
        \end{equation*}
        Por tanto, $|U| = 2 = (2-1) \cdot (3-1)$ y la igualdad se tiene. \\
        Para $n = 3$ y $m = 4$
        \begin{equation*}
            U = U(\Z_3 \times \Z_4) = U(\Z_3) \times U(\Z_4) = \{1, 2\} \times \{1, 3\} = \{(1, 1), (1, 3), (2, 1), (2, 3)\}
        \end{equation*}
        Por tanto $|U| = 4 \neq (3-1)(4-1)$ y la igualdad no se tiene.
    \end{ejercicio}
    \begin{ejercicio}[1.25 puntos]
        Sea $\phi: R \rightarrow \C \times  \C $ la aplicación definida por $\phi(r) = (r, r).$ Demostrar que $\phi$ es un homomorfismo de anillos. \\ \\
        Para $b = (1, i) \in \C \times \C$ y haciendo uso de la propiedad universal, considérese $\phi_b: \R[X] \rightarrow \C \times \C$ el homomorfismo inducido por $\phi$. Determinar el valor de $\phi_b(f)$ para $f = 3 + 2x^2 - 3x^3$. \\ \\
        Puesto que, dados $r, s \in \R$ se tiene que
        \begin{align*}
            & \phi(r+s) = (r+s, r+s) = (r, r)+ (s,s) = \phi(r) + \phi(s) \\
            & \phi(rs) = (rs, rs) = (r, r) \cdot (s,s) = \phi(r) \cdot  \phi(s)
        \end{align*}
        y además, $\phi(1) = (1, 1)$, el uno de $\C \times \C$. Tenemos que $\phi$ es un homomorfismo. \\ \\
        Dado $b = (1, i)$, se tiene por la propiedad universal, el homomorfismo dado por $\phi_b : \R[X] \rightarrow \C \times \C$ que está definido por
        \begin{equation*}
            \phi_b\left(\sum_{k=0}^{n} a_{k}x^{k}\right) = \sum_{k=0}^{n} \phi(a_k) \cdot b^k = \sum_{k=0}^{n} \phi(a_k) (1, i)^k.
        \end{equation*}
        Entonces
        \begin{align*}
            & \phi_b(3+2x^2 - 3x^3) = \phi(3) + \phi(2)(1, i)^2 - \phi(3)(1, i)^3 \\
            & = (3, 3) + (2, 2)(1, -1) - (3, 3)(1, -i) = (3, 3) + (2, -2) - (3, -3i) \\
            & = (3 + 2 - 3, 3 -2 + 3i) = (2, 1 +3i).
        \end{align*}
    \end{ejercicio}
    
\end{document}