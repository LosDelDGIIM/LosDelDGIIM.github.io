\documentclass[12pt]{article}

% Idioma y codificación
\usepackage[spanish, es-tabla]{babel}       %es-tabla para que se titule "Tabla"
\usepackage[utf8]{inputenc}

% Márgenes
\usepackage[a4paper,top=3cm,bottom=2.5cm,left=3cm,right=3cm]{geometry}

% Comentarios de bloque
\usepackage{verbatim}

% Paquetes de links
\usepackage[hidelinks]{hyperref}    % Permite enlaces
\usepackage{url}                    % redirecciona a la web

% Más opciones para enumeraciones
\usepackage{enumitem}

% Personalizar la portada
\usepackage{titling}

% Paquetes de tablas
\usepackage{multirow}


%------------------------------------------------------------------------

%Paquetes de figuras
\usepackage{caption}
\usepackage{subcaption} % Figuras al lado de otras
\usepackage{float}      % Poner figuras en el sitio indicado H.


% Paquetes de imágenes
\usepackage{graphicx}       % Paquete para añadir imágenes
\usepackage{transparent}    % Para manejar la opacidad de las figuras

% Paquete para usar colores
\usepackage[dvipsnames]{xcolor}
\usepackage{pagecolor}      % Para cambiar el color de la página

% Habilita tamaños de fuente mayores
\usepackage{fix-cm}

% Para los gráficos
\usepackage{tikz}

% Para poder situar los nodos en los grafos
\usetikzlibrary{positioning}


%------------------------------------------------------------------------

% Paquetes de matemáticas
\usepackage{mathtools, amsfonts, amssymb, mathrsfs}
\usepackage[makeroom]{cancel}     % Simplificar tachando
\usepackage{polynom}    % Divisiones y Ruffini
\usepackage{units} % Para poner fracciones diagonales con \nicefrac

\usepackage{pgfplots}   %Representar funciones
\pgfplotsset{compat=1.18}  % Versión 1.18

\usepackage{tikz-cd}    % Para usar diagramas de composiciones
\usetikzlibrary{calc}   % Para usar cálculo de coordenadas en tikz

%Definición de teoremas, etc.
\usepackage{amsthm}
%\swapnumbers   % Intercambia la posición del texto y de la numeración

\theoremstyle{plain}

\makeatletter
\@ifclassloaded{article}{
  \newtheorem{teo}{Teorema}[section]
}{
  \newtheorem{teo}{Teorema}[chapter]  % Se resetea en cada chapter
}
\makeatother

\newtheorem{coro}{Corolario}[teo]           % Se resetea en cada teorema
\newtheorem{prop}[teo]{Proposición}         % Usa el mismo contador que teorema
\newtheorem{lema}[teo]{Lema}                % Usa el mismo contador que teorema

\theoremstyle{remark}
\newtheorem*{observacion}{Observación}

\theoremstyle{definition}

\makeatletter
\@ifclassloaded{article}{
  \newtheorem{definicion}{Definición} [section]     % Se resetea en cada chapter
}{
  \newtheorem{definicion}{Definición} [chapter]     % Se resetea en cada chapter
}
\makeatother

\newtheorem*{notacion}{Notación}
\newtheorem*{ejemplo}{Ejemplo}
\newtheorem*{ejercicio*}{Ejercicio}             % No numerado
\newtheorem{ejercicio}{Ejercicio} [section]     % Se resetea en cada section


% Modificar el formato de la numeración del teorema "ejercicio"
\renewcommand{\theejercicio}{%
  \ifnum\value{section}=0 % Si no se ha iniciado ninguna sección
    \arabic{ejercicio}% Solo mostrar el número de ejercicio
  \else
    \thesection.\arabic{ejercicio}% Mostrar número de sección y número de ejercicio
  \fi
}


% \renewcommand\qedsymbol{$\blacksquare$}         % Cambiar símbolo QED
%------------------------------------------------------------------------

% Paquetes para encabezados
\usepackage{fancyhdr}
\pagestyle{fancy}
\fancyhf{}

\newcommand{\helv}{ % Modificación tamaño de letra
\fontfamily{}\fontsize{12}{12}\selectfont}
\setlength{\headheight}{15pt} % Amplía el tamaño del índice


%\usepackage{lastpage}   % Referenciar última pag   \pageref{LastPage}
\fancyfoot[C]{\thepage}

%------------------------------------------------------------------------

% Conseguir que no ponga "Capítulo 1". Sino solo "1."
\makeatletter
\@ifclassloaded{book}{
  \renewcommand{\chaptermark}[1]{\markboth{\thechapter.\ #1}{}} % En el encabezado
    
  \renewcommand{\@makechapterhead}[1]{%
  \vspace*{50\p@}%
  {\parindent \z@ \raggedright \normalfont
    \ifnum \c@secnumdepth >\m@ne
      \huge\bfseries \thechapter.\hspace{1em}\ignorespaces
    \fi
    \interlinepenalty\@M
    \Huge \bfseries #1\par\nobreak
    \vskip 40\p@
  }}
}
\makeatother

%------------------------------------------------------------------------
% Paquetes de cógido
\usepackage{minted}
\renewcommand\listingscaption{Código fuente}

\usepackage{fancyvrb}
% Personaliza el tamaño de los números de línea
\renewcommand{\theFancyVerbLine}{\small\arabic{FancyVerbLine}}

% Estilo para C++
\newminted{cpp}{
    frame=lines,
    framesep=2mm,
    baselinestretch=1.2,
    linenos,
    escapeinside=||
}

% para minted
\definecolor{LightGray}{rgb}{0.95,0.95,0.92}
\setminted{
    linenos=true,
    stepnumber=5,
    numberfirstline=true,
    autogobble,
    breaklines=true,
    breakautoindent=true,
    breaksymbolleft=,
    breaksymbolright=,
    breaksymbolindentleft=0pt,
    breaksymbolindentright=0pt,
    breaksymbolsepleft=0pt,
    breaksymbolsepright=0pt,
    fontsize=\footnotesize,
    bgcolor=LightGray,
    numbersep=10pt
}


\usepackage{listings} % Para incluir código desde un archivo

\renewcommand\lstlistingname{Código Fuente}
\renewcommand\lstlistlistingname{Índice de Códigos Fuente}

% Definir colores
\definecolor{vscodepurple}{rgb}{0.5,0,0.5}
\definecolor{vscodeblue}{rgb}{0,0,0.8}
\definecolor{vscodegreen}{rgb}{0,0.5,0}
\definecolor{vscodegray}{rgb}{0.5,0.5,0.5}
\definecolor{vscodebackground}{rgb}{0.97,0.97,0.97}
\definecolor{vscodelightgray}{rgb}{0.9,0.9,0.9}

% Configuración para el estilo de C similar a VSCode
\lstdefinestyle{vscode_C}{
  backgroundcolor=\color{vscodebackground},
  commentstyle=\color{vscodegreen},
  keywordstyle=\color{vscodeblue},
  numberstyle=\tiny\color{vscodegray},
  stringstyle=\color{vscodepurple},
  basicstyle=\scriptsize\ttfamily,
  breakatwhitespace=false,
  breaklines=true,
  captionpos=b,
  keepspaces=true,
  numbers=left,
  numbersep=5pt,
  showspaces=false,
  showstringspaces=false,
  showtabs=false,
  tabsize=2,
  frame=tb,
  framerule=0pt,
  aboveskip=10pt,
  belowskip=10pt,
  xleftmargin=10pt,
  xrightmargin=10pt,
  framexleftmargin=10pt,
  framexrightmargin=10pt,
  framesep=0pt,
  rulecolor=\color{vscodelightgray},
  backgroundcolor=\color{vscodebackground},
}

%------------------------------------------------------------------------

% Comandos definidos
\newcommand{\bb}[1]{\mathbb{#1}}
\newcommand{\cc}[1]{\mathcal{#1}}

% I prefer the slanted \leq
\let\oldleq\leq % save them in case they're every wanted
\let\oldgeq\geq
\renewcommand{\leq}{\leqslant}
\renewcommand{\geq}{\geqslant}

% Si y solo si
\newcommand{\sii}{\iff}

% Letras griegas
\newcommand{\eps}{\epsilon}
\newcommand{\veps}{\varepsilon}
\newcommand{\lm}{\lambda}

\newcommand{\ol}{\overline}
\newcommand{\ul}{\underline}
\newcommand{\wt}{\widetilde}
\newcommand{\wh}{\widehat}

\let\oldvec\vec
\renewcommand{\vec}{\overrightarrow}

% Derivadas parciales
\newcommand{\del}[2]{\frac{\partial #1}{\partial #2}}
\newcommand{\Del}[3]{\frac{\partial^{#1} #2}{\partial #3^{#1}}}
\newcommand{\deld}[2]{\dfrac{\partial #1}{\partial #2}}
\newcommand{\Deld}[3]{\dfrac{\partial^{#1} #2}{\partial #3^{#1}}}


\newcommand{\AstIg}{\stackrel{(\ast)}{=}}
\newcommand{\Hop}{\stackrel{L'H\hat{o}pital}{=}}

\newcommand{\red}[1]{{\color{red}#1}} % Para integrales, destacar los cambios.

% Método de integración
\newcommand{\MetInt}[2]{
    \left[\begin{array}{c}
        #1 \\ #2
    \end{array}\right]
}

% Declarar aplicaciones
% 1. Nombre aplicación
% 2. Dominio
% 3. Codominio
% 4. Variable
% 5. Imagen de la variable
\newcommand{\Func}[5]{
    \begin{equation*}
        \begin{array}{rrll}
            #1:& #2 & \longrightarrow & #3\\
               & #4 & \longmapsto & #5
        \end{array}
    \end{equation*}
}

%------------------------------------------------------------------------

\newcommand{\im}{\mathit}

\begin{document}

    % 1. Foto de fondo
    % 2. Título
    % 3. Encabezado Izquierdo
    % 4. Color de fondo
    % 5. Coord x del titulo
    % 6. Coord y del titulo
    % 7. Fecha

    
    % 1. Foto de fondo
% 2. Título
% 3. Encabezado Izquierdo
% 4. Color de fondo
% 5. Coord x del titulo
% 6. Coord y del titulo
% 7. Fecha

\newcommand{\portada}[7]{

    \portadaBase{#1}{#2}{#3}{#4}{#5}{#6}{#7}
    \portadaBook{#1}{#2}{#3}{#4}{#5}{#6}{#7}
}

\newcommand{\portadaExamen}[7]{

    \portadaBase{#1}{#2}{#3}{#4}{#5}{#6}{#7}
    \portadaArticle{#1}{#2}{#3}{#4}{#5}{#6}{#7}
}




\newcommand{\portadaBase}[7]{

    % Tiene la portada principal y la licencia Creative Commons
    
    % 1. Foto de fondo
    % 2. Título
    % 3. Encabezado Izquierdo
    % 4. Color de fondo
    % 5. Coord x del titulo
    % 6. Coord y del titulo
    % 7. Fecha
    
    
    \thispagestyle{empty}               % Sin encabezado ni pie de página
    \newgeometry{margin=0cm}        % Márgenes nulos para la primera página
    
    
    % Encabezado
    \fancyhead[L]{\helv #3}
    \fancyhead[R]{\helv \nouppercase{\leftmark}}
    
    
    \pagecolor{#4}        % Color de fondo para la portada
    
    \begin{figure}[p]
        \centering
        \transparent{0.3}           % Opacidad del 30% para la imagen
        
        \includegraphics[width=\paperwidth, keepaspectratio]{assets/#1}
    
        \begin{tikzpicture}[remember picture, overlay]
            \node[anchor=north west, text=white, opacity=1, font=\fontsize{60}{90}\selectfont\bfseries\sffamily, align=left] at (#5, #6) {#2};
            
            \node[anchor=south east, text=white, opacity=1, font=\fontsize{12}{18}\selectfont\sffamily, align=right] at (9.7, 3) {\textbf{\href{https://losdeldgiim.github.io/}{Los Del DGIIM}}};
            
            \node[anchor=south east, text=white, opacity=1, font=\fontsize{12}{15}\selectfont\sffamily, align=right] at (9.7, 1.8) {Doble Grado en Ingeniería Informática y Matemáticas\\Universidad de Granada};
        \end{tikzpicture}
    \end{figure}
    
    
    \restoregeometry        % Restaurar márgenes normales para las páginas subsiguientes
    \pagecolor{white}       % Restaurar el color de página
    
    
    \newpage
    \thispagestyle{empty}               % Sin encabezado ni pie de página
    \begin{tikzpicture}[remember picture, overlay]
        \node[anchor=south west, inner sep=3cm] at (current page.south west) {
            \begin{minipage}{0.5\paperwidth}
                \href{https://creativecommons.org/licenses/by-nc-nd/4.0/}{
                    \includegraphics[height=2cm]{assets/Licencia.png}
                }\vspace{1cm}\\
                Esta obra está bajo una
                \href{https://creativecommons.org/licenses/by-nc-nd/4.0/}{
                    Licencia Creative Commons Atribución-NoComercial-SinDerivadas 4.0 Internacional (CC BY-NC-ND 4.0).
                }\\
    
                Eres libre de compartir y redistribuir el contenido de esta obra en cualquier medio o formato, siempre y cuando des el crédito adecuado a los autores originales y no persigas fines comerciales. 
            \end{minipage}
        };
    \end{tikzpicture}
    
    
    
    % 1. Foto de fondo
    % 2. Título
    % 3. Encabezado Izquierdo
    % 4. Color de fondo
    % 5. Coord x del titulo
    % 6. Coord y del titulo
    % 7. Fecha


}


\newcommand{\portadaBook}[7]{

    % 1. Foto de fondo
    % 2. Título
    % 3. Encabezado Izquierdo
    % 4. Color de fondo
    % 5. Coord x del titulo
    % 6. Coord y del titulo
    % 7. Fecha

    % Personaliza el formato del título
    \pretitle{\begin{center}\bfseries\fontsize{42}{56}\selectfont}
    \posttitle{\par\end{center}\vspace{2em}}
    
    % Personaliza el formato del autor
    \preauthor{\begin{center}\Large}
    \postauthor{\par\end{center}\vfill}
    
    % Personaliza el formato de la fecha
    \predate{\begin{center}\huge}
    \postdate{\par\end{center}\vspace{2em}}
    
    \title{#2}
    \author{\href{https://losdeldgiim.github.io/}{Los Del DGIIM}}
    \date{Granada, #7}
    \maketitle
    
    \tableofcontents
}




\newcommand{\portadaArticle}[7]{

    % 1. Foto de fondo
    % 2. Título
    % 3. Encabezado Izquierdo
    % 4. Color de fondo
    % 5. Coord x del titulo
    % 6. Coord y del titulo
    % 7. Fecha

    % Personaliza el formato del título
    \pretitle{\begin{center}\bfseries\fontsize{42}{56}\selectfont}
    \posttitle{\par\end{center}\vspace{2em}}
    
    % Personaliza el formato del autor
    \preauthor{\begin{center}\Large}
    \postauthor{\par\end{center}\vspace{3em}}
    
    % Personaliza el formato de la fecha
    \predate{\begin{center}\huge}
    \postdate{\par\end{center}\vspace{5em}}
    
    \title{#2}
    \author{\href{https://losdeldgiim.github.io/}{Los Del DGIIM}}
    \date{Granada, #7}
    \thispagestyle{empty}               % Sin encabezado ni pie de página
    \maketitle
    \vfill
}
    \portadaExamen{ffccA4.jpg}{Álgebra I\\Parcial VI}{Álgebra I. Parcial VI}{MidnightBlue}{-8}{28}{2023-2024}{José Juan Urrutia Milán}

    \begin{description}
        \item[Asignatura] Álgebra I.
        \item[Curso Académico] 2024-25.
        \item[Grado] Grado en Matemáticas.
        \item[Grupo] Único.
        \item[Profesor] María del Pilar Carrasco Carrasco.
        \item[Descripción] Parcial I
        \item[Fecha] 14 de noviembre de 2024.
        %\item[Duración] -
    \end{description}
    \newpage

    Los puntos de los ejercicios se reparten de forma equitativa entre los apartados.
    \begin{ejercicio}[3 puntos]
        \
        \begin{enumerate}[label=(\alph*)]
            \item Sean $P$, $Q$, $R$ propiedades referidas a los elementos de un conjunto $X$. Supongamos que $P\Longrightarrow \lnot R$. Demostrar la siguiente equivalencia:
                \begin{equation*}
                    (P\lor Q) \land \lnot R \Longleftrightarrow P\lor (Q \land \lnot R)
                \end{equation*}
            \item Sean $f:X\rightarrow Y$ y $g:Y\rightarrow Z$ aplicaciones componibles. Demsotrar que si $f$ y $g$ son biyectivas entonces $g\circ f:X\rightarrow Z$ es biyectiva. Demostrar que, en tal caso, ${(g\circ f)}^{-1} = f^{-1}\circ g^{-1}$.
            \item Sea $X=\{0,2,4\}$. En el conjunto $X\times X$ definimos la relación binaria $\sim$:
                \begin{equation*}
                    (a,b)\sim (c,d) \Longleftrightarrow a+d = b+c
                \end{equation*}
                Demostrar que $\sim$ es una relación de equivalencia y calcular (describiendo todas las clases de equivalencia) el conjunto cociente $X\times X/\sim$.
        \end{enumerate}~\\

        \begin{enumerate}[label=(\alph*)]
            \item Sean $X_P$, $X_Q$ y $X_R$ los subconjuntos de $X$ conformados por los elementos que verifican la propiedad $P$, $Q$ y $R$ respectivamente. Puesto que $P\Longrightarrow \lnot R$, se tiene que $X_P\subseteq X_{\lnot R} = c(X_R)$. Se trata de demostrar que:
                \begin{equation*}
                    (X_P \cup X_Q) \cap c(X_R) = X_P \cup (X_Q \cap c(X_R))
                \end{equation*}
                aplicando la propiedad distributiva de la intersección:
                \begin{equation*}
                    (X_P \cup X_Q) \cap c(X_R) = (X_P \cap c(X_R)) \cup (X_Q \cap c(X_R)) \AstIg X_P \cup (X_Q \cap c(X_R))
                \end{equation*}
                Donde en $(\ast)$ aplicamos que $X_P\subseteq c(X_R)$.
            \item Partimos de que $f:X\rightarrow Y$ y $g:Y\rightarrow Z$ son biyectivas. Entonces:
                \begin{align*}
                    &\exists f^{-1}:Y\rightarrow X \text{\ única tal que\ } f\circ f^{-1} = id_Y \land f^{-1} \circ f = id_X \\
                    &\exists g^{-1}:Z\rightarrow Y \text{\ única tal que\ } g\circ g^{-1} = id_Z \land g^{-1} \circ g = id_Y 
                \end{align*}
                Se trata de probar que $g\circ f:X\rightarrow Z$ es biyectiva. Para ello, consideramos la composición $f^{-1}\circ g^{-1}:Z\rightarrow X$. Se tiene entonces que:
                \begin{align*}
                    (g\circ f) \circ (f^{-1}\circ g^{-1}) &\AstIg g\circ (f\circ f^{-1})\circ g^{-1} = g\circ (id_Y \circ g^{-1}) = g\circ g^{-1} = id_Z \\
                    (f^{-1}\circ g^{-1})\circ (g\circ f)&\AstIg f^{-1}\circ (g^{-1}\circ g) \circ f = f^{-1} \circ  (id_Y \circ f) = f^{-1} \circ f = id_X
                \end{align*}
                Donde en $(\ast)$ hemos aplicado la propiedad asociativa de la composición. Por tanto, $g \circ f:X\rightarrow Z$ tiene inversa y por consiguiente es biyectiva.

                Además, como la inversa de una aplicación biyectiva es única, será
                \begin{equation*}
                    {(g\circ f)}^{-1} = f^{-1} \circ g^{-1}
                \end{equation*}
            \item Tenemos que:
                \begin{gather*}
                    X=\{0,2,4\} \qquad X\times X=\{(a,b) \mid a,b\in X\} \\
                    (a,b)\sim (c,d) \Longleftrightarrow a+d = b+c
                \end{gather*}
                Para ver que $\sim$ es una relación de equivalencia, hemos de ver:
                \begin{description}
                    \item [Propiedad reflexiva.]~\\
                        Puesto que $a+b=b+a$, entonces:
                        \begin{equation*}
                            (a,b)\sim (a,b) \qquad \forall (a,b)\in X\times X
                        \end{equation*}
                    \item [Propiedad simétrica.]~\\
                        Sean $(a,b),(c,d)\in X\times X$. Entonces:
                        \begin{equation*}
                            (a,b)\sim (c,d) \Longleftrightarrow a+d=b+c \Longleftrightarrow d+a = c+b \Longleftrightarrow (c,d)\sim (a,b)
                        \end{equation*}
                    \item [Propiedad transitiva.]~\\
                        Supongamos que $(a,b),(c,d),(e,f)\in X\times X$ de forma que $(a,b)\sim(c,d)$ y $(c,d)\sim (e,f)$. Entonces:
                        \begin{align*}
                            &\left.\begin{array}{c}
                                    (a,b) \sim (c,d) \\
                                    \land \\
                                    (c,d) \sim (e,f)
                            \end{array}\right\} \Longrightarrow 
                            \left\{\begin{array}{c}
                                a+d = b + c \\
                                \land \\
                                c + f = d + e
                            \end{array}\right\} \Longrightarrow 
                            \left\{\begin{array}{c}
                                a + d + f = b + c + f \\
                                \land \\
                                c + f + b = d + e + b
                        \end{array}\right. \\
                        &\Longrightarrow a+d+f = d+e+b \Longrightarrow a + f = e + b \Longrightarrow (a,b)\sim (e,f)
                        \end{align*}
                \end{description}
                Demostrado que $\sim$ es una relación de equivalencia, calculamos $X\times X/\sim$:
                \begin{gather*}
                    X\times X = \{(0,0),(0,2),(0,4),(2,0),(2,2),(2,4),(4,0),(4,2),(4,4)\} \\
                    X\times X/\sim = \{[(a,b)] \mid (a,b)\in X\times X\}
                \end{gather*}
                Calculamos las diferentes clases:
                \begin{align*}
                    [(0,0)] &= \{(a,b)\in X\times X \mid (a,b)\sim(0,0)\} = \{(a,b)\in X\times X \mid a=b\} \\
                            &= \{(0,0),(2,2),(4,4)\} = [(2,2)] = [(4,4)] \\
                    [(0,2)] &= \{(a,b)\in X\times X \mid (a,b)\sim (0,2)\} = \{(a,b)\in X\times X \mid a+2 = b\} \\
                            &= \{(0,2),(2,4)\} = [(0,2)] = [(2,4)] \\
                    [(0,4)] &= \{(a,b)\in X\times X \mid (a,b)\sim (0,4)\} = \{(a,b)\in X\times X \mid a+4 = b\} \\
                            &= \{(0,4)\} \\
                    [(2,0)] &= \{(a,b)\in X\times X \mid (a,b)\sim (0,2)\} =\{(a,b)\in X\times X \mid a = b +2\} \\
                            &= \{(2,0),(4,2)\} = [(4,2)] \\
                    [(4,0)] &= \{(a,b)\in X\times X \mid (a,b)\sim(4,0)\} = \{(a,b)\in X\times X \mid a = b+4\} \\
                            &= \{(4,0)\} 
                \end{align*}
                Con lo que:
                \begin{equation*}
                    X\times X/\sim\ = \{[(0,0)], [(0,2)], [(0,4)], [(2,0)], [(4,0)]\}
                \end{equation*}
        \end{enumerate}
    \end{ejercicio}

    \begin{ejercicio}[4 puntos]
        Efectuar los siguientes cálculos:
        \begin{enumerate}[label=(\alph*)]
            \item El resto de dividir $18\cdot 15-561\cdot 15^2$ entre 13.
            \item ${[2\cdot (3^5 - 5^2)]}^{-1}$ en $\mathbb{Z}_7$.
            \item $(2x^3-3x+5)(3x-2)$ en $\mathbb{Z}_6[x]$.
            \item ${\left(7-4\sqrt{3}\right)}^{-1}$ en $\mathbb{Z}\left[\sqrt{3}\right]$.
            \item ${\left(\frac{2}{3}-\frac{1}{9}\sqrt{-3}\right)}^{-1}$ en $\mathbb{Q}\left[\sqrt{-3}\right]$.
        \end{enumerate}~\\

        \begin{enumerate}[label=(\alph*)]
            \item Puesto que la aplicación 
                \Func{R}{\bb{Z}}{\bb{Z}_{13}}{a}{R(a):=Res(a;13)}
                Es un homomorfismo de anillos, será:
                \begin{align*}
                    Res(18\cdot 15-561\cdot 15^2; 13) &= Res(18;13)\cdot Res(15;13) - Res(561;13)\cdot Res(15;13)^2 \\
                                                      &\AstIg 5\cdot 2-2\cdot 2^2 = 10 -8 = 2
                \end{align*}
                Donde en $(\ast)$ hemos aplicado que $561=13\cdot 43+2$
            \item Para ello, primero calcularemos $2\cdot (3^5 - 5^2)$ en $\mathbb{Z}_7$:
                \begin{itemize}
                    \item $3^5 = 5$ en $\mathbb{Z}_7$ pues $3^5 = 243 = 7\cdot 34+5$ 
                    \item $5^2 = 5$ en $\mathbb{Z}_7$ pues $5^2 = 25 = 7\cdot 3 + 4$
                \end{itemize}
                Entonces:
                \begin{equation*}
                    2(3^5 - 5^2) = 2(5-4) = 2
                \end{equation*}
                Y solo queda calcular el inverso de 2 en $\mathbb{Z}_7$:
                \begin{equation*}
                    {[2\cdot (3^5-5^2)]}^{-1} = 2^{-1} = 4
                \end{equation*}
                Ya que $2\cdot 4 = 8 = 1$ en $\mathbb{Z}_7$.
            \item $(2x^3-3x+5)(3x-2) = 6x^4-4x^3-9x^2+6x+15x-10 = 2x^3+3x^2+3x+2$
            \item Puesto que
                \begin{equation*}
                    N(7-4\sqrt{3}) = (7-4\sqrt{3})(7+4\sqrt{3}) = 49 - 3\cdot 16 = 49-48 = 1
                \end{equation*}
                Entonces tenemos que $7-4\sqrt{3}\in U(\mathbb{Z}[\sqrt{3}])$ y:
                \begin{equation*}
                    {(7-4\sqrt{3})}^{-1} = 7+4\sqrt{3}
                \end{equation*}
            \item Sabemos que si $0\neq \alpha\in \mathbb{Q}[\sqrt{-3}]$ entonces tenemos que $N(\alpha)\neq 0$ y\newline $\alpha^{-1} = \frac{1}{N(\alpha)}\overline{\alpha}$.

                Para $\alpha = \frac{2}{3}-\frac{1}{9}\sqrt{-3}$, será:
                \begin{equation*}
                    N(\alpha) = \dfrac{4}{9}+\dfrac{3}{81} = \dfrac{13}{27}
                \end{equation*}
                y entonces:
                \begin{equation*}
                    {\left(\dfrac{2}{3}-\dfrac{1}{9}\sqrt{-3}\right)}^{-1} = \dfrac{27}{13} \left(\dfrac{2}{3}+\dfrac{1}{9}\sqrt{-3}\right) = \dfrac{18}{13} + \dfrac{3}{13}\sqrt{-3}
                \end{equation*}
        \end{enumerate}
    \end{ejercicio}

    \begin{ejercicio}[3 puntos]
        Sea $f:A\rightarrow B$ un homomorfismo de anillos. Demostrar:
        \begin{enumerate}[label=(\alph*)]
            \item $Img(f)$ es un subanillo de $B$.
            \item $f(n\cdot a) = n\cdot f(a)$, para todo $n\in \mathbb{Z}$ y todo $a\in A$.
            \item $f(u^n) = f(u)^n$, para todo $n\in \mathbb{Z}$ y todo $u\in U(A)$.
        \end{enumerate}~\\
        \begin{enumerate}[label=(\alph*)]
            \item Sabemos que $Img(f)=\{f(a)\mid a\in A\}\subseteq B$.

                Para demostrar que es un subanillo de $B$ hemos de ver que es cerrado para sumas, productos, opuestos y que contiene al 1 de $B$.

                Sean $b_1,b_2\in Img(f)$, entonces $\exists a_1,a_2 \in A$ tales que:
                \begin{equation*}
                    b_1 = f(a_1) \qquad b_2 = f(a_2)
                \end{equation*}
                Entonces:
                \begin{align*}
                    b_1 + b_2 &= f(a_1) + f(a_2) = f(a_1+a_2) \in Img(f) \\
                    b_1\cdot b_2 &= f(a_1)f(a_2) = f(a_1\cdot a_2) \in Img(f)
                \end{align*}
                por lo que $Img(f)$ es cerrado para sumas y productos. Para ver que es cerrado para opuestos, utilizamos que todo homomorfismo verifica que $f(-a) = -f(a)$ $\forall a\in A$. Entonces:
                \begin{equation*}
                    \text{Si\ } b\in Img(f) \Longrightarrow \exists a\in A \mid f(a) = b \Longrightarrow -b = -f(a) = f(-a) \in Img(f)
                \end{equation*}
                Finalmente, como $f(1)=1 \in Img(f)$, tenemos que $Img(f)$ es un subanillo de $B$.
            \item Distinguimos casos:
                \begin{itemize}
                    \item Para $n\geq 1$ y $a\in A$: $n\cdot a = \overbrace{a+ \ldots +a }^{n \text{\ veces}} $, con lo que:
                        \begin{equation*}
                            f(n\cdot a) = f( \overbrace{a+\ldots+a}^{n \text{\ veces}} ) =  \overbrace{f(a)+\ldots+f(a)}^{n \text{\ veces}}  = n\cdot f(a)
                        \end{equation*}
                    \item Para $n=0$ y $a\in A$, tenemos que $0\cdot a = 0$, con lo que:
                        \begin{equation*}
                            f(0\cdot a) = f(0) = 0 = 0\cdot f(a)
                        \end{equation*}
                    \item Para $n<0$ y $a\in A$, tenemos que $n\cdot a = (-n)(-a)$, con lo que:
                        \begin{equation*}
                            f(n\cdot a) = f((-n)(-a)) \AstIg (-n)f(-a) = (-n)(-f(a)) = n\cdot f(a)
                        \end{equation*}
                        Donde en $(\ast)$ usamos que $-n>0$, con lo que podemos aplicar el primer apartado.
                \end{itemize}
            \item Distinguimos casos:
                \begin{itemize}
                    \item Para $n\geq 1$ y $a\in A$: $a^n = \overbrace{a\cdot \ldots\cdot a}^{n \text{\ veces}} $, con lo que:
                        \begin{equation*}
                            f(u^n) = f(\overbrace{u\cdot \ldots\cdot u}^{n \text{\ veces}} ) =  \overbrace{f(u)\cdot \ldots\cdot f(u)}^{n \text{\ veces}} = {[f(u)]}^{n}
                        \end{equation*}
                    \item Para $n=0$ y $a\in A$, $a^0 = 1$, con lo que:
                        \begin{equation*}
                            f(u^0) = f(1) = 1 = {[f(u)]}^0
                        \end{equation*}
                    \item Para $n<0$ y $u\in U(A)$, $u^n = {(u^{-1})}^{-n}$.

                        Además, si $u\in U(A) \Longrightarrow \exists u^{-1}\in A \mid uu^{-1} = 1$. Entonces:
                        \begin{equation*}
                            1 = f(1) = f(uu^{-1}) = f(u)f(u^{-1})
                        \end{equation*}
                        y por tanto $f(u)\in U(B)$ y $f(u)^{-1} = f(u^{-1})$. Entonces:
                        \begin{equation*}
                            f(u^n) = f({(u^{-1})}^{-n}) \AstIg {[f(u^{-1})]}^{-n} = {[f(u)^{-1}]}^{-n} = f(u)^n
                        \end{equation*}
                        Donde en $(\ast)$ hemos usado que $-n>0$ y el primer apartado.
                \end{itemize}
        \end{enumerate}
    \end{ejercicio}

\end{document}
