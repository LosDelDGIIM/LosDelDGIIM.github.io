\documentclass[12pt]{article}

% Idioma y codificación
\usepackage[spanish, es-tabla, es-notilde]{babel}       %es-tabla para que se titule "Tabla"
\usepackage[utf8]{inputenc}

% Márgenes
\usepackage[a4paper,top=3cm,bottom=2.5cm,left=3cm,right=3cm]{geometry}

% Comentarios de bloque
\usepackage{verbatim}

% Paquetes de links
\usepackage[hidelinks]{hyperref}    % Permite enlaces
\usepackage{url}                    % redirecciona a la web

% Más opciones para enumeraciones
\usepackage{enumitem}

% Personalizar la portada
\usepackage{titling}

% Paquetes de tablas
\usepackage{multirow}

% Para añadir el símbolo de euro
\usepackage{eurosym}


%------------------------------------------------------------------------

%Paquetes de figuras
\usepackage{caption}
\usepackage{subcaption} % Figuras al lado de otras
\usepackage{float}      % Poner figuras en el sitio indicado H.


% Paquetes de imágenes
\usepackage{graphicx}       % Paquete para añadir imágenes
\usepackage{transparent}    % Para manejar la opacidad de las figuras

% Paquete para usar colores
\usepackage[dvipsnames, table, xcdraw]{xcolor}
\usepackage{pagecolor}      % Para cambiar el color de la página

% Habilita tamaños de fuente mayores
\usepackage{fix-cm}

% Para los gráficos
\usepackage{tikz}
\usepackage{forest}

% Para poder situar los nodos en los grafos
\usetikzlibrary{positioning}


%------------------------------------------------------------------------

% Paquetes de matemáticas
\usepackage{mathtools, amsfonts, amssymb, mathrsfs}
\usepackage[makeroom]{cancel}     % Simplificar tachando
\usepackage{polynom}    % Divisiones y Ruffini
\usepackage{units} % Para poner fracciones diagonales con \nicefrac

\usepackage{pgfplots}   %Representar funciones
\pgfplotsset{compat=1.18}  % Versión 1.18

\usepackage{tikz-cd}    % Para usar diagramas de composiciones
\usetikzlibrary{calc}   % Para usar cálculo de coordenadas en tikz

%Definición de teoremas, etc.
\usepackage{amsthm}
%\swapnumbers   % Intercambia la posición del texto y de la numeración

\theoremstyle{plain}

\makeatletter
\@ifclassloaded{article}{
  \newtheorem{teo}{Teorema}[section]
}{
  \newtheorem{teo}{Teorema}[chapter]  % Se resetea en cada chapter
}
\makeatother

\newtheorem{coro}{Corolario}[teo]           % Se resetea en cada teorema
\newtheorem{prop}[teo]{Proposición}         % Usa el mismo contador que teorema
\newtheorem{lema}[teo]{Lema}                % Usa el mismo contador que teorema
\newtheorem*{lema*}{Lema}

\theoremstyle{remark}
\newtheorem*{observacion}{Observación}

\theoremstyle{definition}

\makeatletter
\@ifclassloaded{article}{
  \newtheorem{definicion}{Definición} [section]     % Se resetea en cada chapter
}{
  \newtheorem{definicion}{Definición} [chapter]     % Se resetea en cada chapter
}
\makeatother

\newtheorem*{notacion}{Notación}
\newtheorem*{ejemplo}{Ejemplo}
\newtheorem*{ejercicio*}{Ejercicio}             % No numerado
\newtheorem{ejercicio}{Ejercicio} [section]     % Se resetea en cada section


% Modificar el formato de la numeración del teorema "ejercicio"
\renewcommand{\theejercicio}{%
  \ifnum\value{section}=0 % Si no se ha iniciado ninguna sección
    \arabic{ejercicio}% Solo mostrar el número de ejercicio
  \else
    \thesection.\arabic{ejercicio}% Mostrar número de sección y número de ejercicio
  \fi
}


% \renewcommand\qedsymbol{$\blacksquare$}         % Cambiar símbolo QED
%------------------------------------------------------------------------

% Paquetes para encabezados
\usepackage{fancyhdr}
\pagestyle{fancy}
\fancyhf{}

\newcommand{\helv}{ % Modificación tamaño de letra
\fontfamily{}\fontsize{12}{12}\selectfont}
\setlength{\headheight}{15pt} % Amplía el tamaño del índice


%\usepackage{lastpage}   % Referenciar última pag   \pageref{LastPage}
%\fancyfoot[C]{%
%  \begin{minipage}{\textwidth}
%    \centering
%    ~\\
%    \thepage\\
%    \href{https://losdeldgiim.github.io/}{\texttt{\footnotesize losdeldgiim.github.io}}
%  \end{minipage}
%}
\fancyfoot[C]{\thepage}
\fancyfoot[R]{\href{https://losdeldgiim.github.io/}{\texttt{\footnotesize losdeldgiim.github.io}}}

%------------------------------------------------------------------------

% Conseguir que no ponga "Capítulo 1". Sino solo "1."
\makeatletter
\@ifclassloaded{book}{
  \renewcommand{\chaptermark}[1]{\markboth{\thechapter.\ #1}{}} % En el encabezado
    
  \renewcommand{\@makechapterhead}[1]{%
  \vspace*{50\p@}%
  {\parindent \z@ \raggedright \normalfont
    \ifnum \c@secnumdepth >\m@ne
      \huge\bfseries \thechapter.\hspace{1em}\ignorespaces
    \fi
    \interlinepenalty\@M
    \Huge \bfseries #1\par\nobreak
    \vskip 40\p@
  }}
}
\makeatother

%------------------------------------------------------------------------
% Paquetes de cógido
\usepackage{minted}
\renewcommand\listingscaption{Código fuente}

\usepackage{fancyvrb}
% Personaliza el tamaño de los números de línea
\renewcommand{\theFancyVerbLine}{\small\arabic{FancyVerbLine}}

% Estilo para C++
\newminted{cpp}{
    frame=lines,
    framesep=2mm,
    baselinestretch=1.2,
    linenos,
    escapeinside=||
}

% para minted
\definecolor{LightGray}{rgb}{0.95,0.95,0.92}
\setminted{
    linenos=true,
    stepnumber=5,
    numberfirstline=true,
    autogobble,
    breaklines=true,
    breakautoindent=true,
    breaksymbolleft=,
    breaksymbolright=,
    breaksymbolindentleft=0pt,
    breaksymbolindentright=0pt,
    breaksymbolsepleft=0pt,
    breaksymbolsepright=0pt,
    fontsize=\footnotesize,
    bgcolor=LightGray,
    numbersep=10pt
}


\usepackage{listings} % Para incluir código desde un archivo

\renewcommand\lstlistingname{Código Fuente}
\renewcommand\lstlistlistingname{Índice de Códigos Fuente}

% Definir colores
\definecolor{vscodepurple}{rgb}{0.5,0,0.5}
\definecolor{vscodeblue}{rgb}{0,0,0.8}
\definecolor{vscodegreen}{rgb}{0,0.5,0}
\definecolor{vscodegray}{rgb}{0.5,0.5,0.5}
\definecolor{vscodebackground}{rgb}{0.97,0.97,0.97}
\definecolor{vscodelightgray}{rgb}{0.9,0.9,0.9}

% Configuración para el estilo de C similar a VSCode
\lstdefinestyle{vscode_C}{
  backgroundcolor=\color{vscodebackground},
  commentstyle=\color{vscodegreen},
  keywordstyle=\color{vscodeblue},
  numberstyle=\tiny\color{vscodegray},
  stringstyle=\color{vscodepurple},
  basicstyle=\scriptsize\ttfamily,
  breakatwhitespace=false,
  breaklines=true,
  captionpos=b,
  keepspaces=true,
  numbers=left,
  numbersep=5pt,
  showspaces=false,
  showstringspaces=false,
  showtabs=false,
  tabsize=2,
  frame=tb,
  framerule=0pt,
  aboveskip=10pt,
  belowskip=10pt,
  xleftmargin=10pt,
  xrightmargin=10pt,
  framexleftmargin=10pt,
  framexrightmargin=10pt,
  framesep=0pt,
  rulecolor=\color{vscodelightgray},
  backgroundcolor=\color{vscodebackground},
}

%------------------------------------------------------------------------

% Comandos definidos
\newcommand{\bb}[1]{\mathbb{#1}}
\newcommand{\cc}[1]{\mathcal{#1}}

% I prefer the slanted \leq
\let\oldleq\leq % save them in case they're every wanted
\let\oldgeq\geq
\renewcommand{\leq}{\leqslant}
\renewcommand{\geq}{\geqslant}

% Si y solo si
\newcommand{\sii}{\iff}

% MCD y MCM
\DeclareMathOperator{\mcd}{mcd}
\DeclareMathOperator{\mcm}{mcm}

% Signo
\DeclareMathOperator{\sgn}{sgn}

% Letras griegas
\newcommand{\eps}{\epsilon}
\newcommand{\veps}{\varepsilon}
\newcommand{\lm}{\lambda}

\newcommand{\ol}{\overline}
\newcommand{\ul}{\underline}
\newcommand{\wt}{\widetilde}
\newcommand{\wh}{\widehat}

\let\oldvec\vec
\renewcommand{\vec}{\overrightarrow}

% Derivadas parciales
\newcommand{\del}[2]{\frac{\partial #1}{\partial #2}}
\newcommand{\Del}[3]{\frac{\partial^{#1} #2}{\partial #3^{#1}}}
\newcommand{\deld}[2]{\dfrac{\partial #1}{\partial #2}}
\newcommand{\Deld}[3]{\dfrac{\partial^{#1} #2}{\partial #3^{#1}}}


\newcommand{\AstIg}{\stackrel{(\ast)}{=}}
\newcommand{\Hop}{\stackrel{L'H\hat{o}pital}{=}}

\newcommand{\red}[1]{{\color{red}#1}} % Para integrales, destacar los cambios.

% Método de integración
\newcommand{\MetInt}[2]{
    \left[\begin{array}{c}
        #1 \\ #2
    \end{array}\right]
}

% Declarar aplicaciones
% 1. Nombre aplicación
% 2. Dominio
% 3. Codominio
% 4. Variable
% 5. Imagen de la variable
\newcommand{\Func}[5]{
    \begin{equation*}
        \begin{array}{rrll}
            \displaystyle #1:& \displaystyle  #2 & \longrightarrow & \displaystyle  #3\\
               & \displaystyle  #4 & \longmapsto & \displaystyle  #5
        \end{array}
    \end{equation*}
}

%------------------------------------------------------------------------

\usepackage{extarrows}
\usepackage{stackrel}
\usetikzlibrary{matrix} % Para divisiones de polinomios.

\newcommand{\resetearcontador}{%
  \setcounter{ejercicio}{0} % Resetea el contador de ejercicios a 0
}
\begin{document}

    % 1. Foto de fondo
    % 2. Título
    % 3. Encabezado Izquierdo
    % 4. Color de fondo
    % 5. Coord x del titulo
    % 6. Coord y del titulo
    % 7. Fecha

    
    % 1. Foto de fondo
% 2. Título
% 3. Encabezado Izquierdo
% 4. Color de fondo
% 5. Coord x del titulo
% 6. Coord y del titulo
% 7. Fecha
% 8. Autor

\newcommand{\portada}[8]{
    \portadaBase{#1}{#2}{#3}{#4}{#5}{#6}{#7}{#8}
    \portadaBook{#1}{#2}{#3}{#4}{#5}{#6}{#7}{#8}
}

\newcommand{\portadaFotoDif}[8]{
    \portadaBaseFotoDif{#1}{#2}{#3}{#4}{#5}{#6}{#7}{#8}
    \portadaBook{#1}{#2}{#3}{#4}{#5}{#6}{#7}{#8}
}

\newcommand{\portadaExamen}[8]{
    \portadaBase{#1}{#2}{#3}{#4}{#5}{#6}{#7}{#8}
    \portadaArticle{#1}{#2}{#3}{#4}{#5}{#6}{#7}{#8}
}

\newcommand{\portadaExamenFotoDif}[8]{
    \portadaBaseFotoDif{#1}{#2}{#3}{#4}{#5}{#6}{#7}{#8}
    \portadaArticle{#1}{#2}{#3}{#4}{#5}{#6}{#7}{#8}
}




\newcommand{\portadaBase}[8]{

    % Tiene la portada principal y la licencia Creative Commons
    
    % 1. Foto de fondo
    % 2. Título
    % 3. Encabezado Izquierdo
    % 4. Color de fondo
    % 5. Coord x del titulo
    % 6. Coord y del titulo
    % 7. Fecha
    % 8. Autor    
    
    \thispagestyle{empty}               % Sin encabezado ni pie de página
    \newgeometry{margin=0cm}        % Márgenes nulos para la primera página
    
    
    % Encabezado
    \fancyhead[L]{\helv #3}
    \fancyhead[R]{\helv \nouppercase{\leftmark}}
    
    
    \pagecolor{#4}        % Color de fondo para la portada
    
    \begin{figure}[p]
        \centering
        \transparent{0.3}           % Opacidad del 30% para la imagen
        
        \includegraphics[width=\paperwidth, keepaspectratio]{../../_assets/#1}
    
        \begin{tikzpicture}[remember picture, overlay]
            \node[anchor=north west, text=white, opacity=1, font=\fontsize{60}{90}\selectfont\bfseries\sffamily, align=left] at (#5, #6) {#2};
            
            \node[anchor=south east, text=white, opacity=1, font=\fontsize{12}{18}\selectfont\sffamily, align=right] at (9.7, 3) {\href{https://losdeldgiim.github.io/}{\textbf{Los Del DGIIM}, \texttt{\footnotesize losdeldgiim.github.io}}};
            
            \node[anchor=south east, text=white, opacity=1, font=\fontsize{12}{15}\selectfont\sffamily, align=right] at (9.7, 1.8) {Doble Grado en Ingeniería Informática y Matemáticas\\Universidad de Granada};
        \end{tikzpicture}
    \end{figure}
    
    
    \restoregeometry        % Restaurar márgenes normales para las páginas subsiguientes
    \nopagecolor      % Restaurar el color de página
    
    
    \newpage
    \thispagestyle{empty}               % Sin encabezado ni pie de página
    \begin{tikzpicture}[remember picture, overlay]
        \node[anchor=south west, inner sep=3cm] at (current page.south west) {
            \begin{minipage}{0.5\paperwidth}
                \href{https://creativecommons.org/licenses/by-nc-nd/4.0/}{
                    \includegraphics[height=2cm]{../../_assets/Licencia.png}
                }\vspace{1cm}\\
                Esta obra está bajo una
                \href{https://creativecommons.org/licenses/by-nc-nd/4.0/}{
                    Licencia Creative Commons Atribución-NoComercial-SinDerivadas 4.0 Internacional (CC BY-NC-ND 4.0).
                }\\
    
                Eres libre de compartir y redistribuir el contenido de esta obra en cualquier medio o formato, siempre y cuando des el crédito adecuado a los autores originales y no persigas fines comerciales. 
            \end{minipage}
        };
    \end{tikzpicture}
    
    
    
    % 1. Foto de fondo
    % 2. Título
    % 3. Encabezado Izquierdo
    % 4. Color de fondo
    % 5. Coord x del titulo
    % 6. Coord y del titulo
    % 7. Fecha
    % 8. Autor


}


\newcommand{\portadaBaseFotoDif}[8]{

    % Tiene la portada principal y la licencia Creative Commons
    
    % 1. Foto de fondo
    % 2. Título
    % 3. Encabezado Izquierdo
    % 4. Color de fondo
    % 5. Coord x del titulo
    % 6. Coord y del titulo
    % 7. Fecha
    % 8. Autor    
    
    \thispagestyle{empty}               % Sin encabezado ni pie de página
    \newgeometry{margin=0cm}        % Márgenes nulos para la primera página
    
    
    % Encabezado
    \fancyhead[L]{\helv #3}
    \fancyhead[R]{\helv \nouppercase{\leftmark}}
    
    
    \pagecolor{#4}        % Color de fondo para la portada
    
    \begin{figure}[p]
        \centering
        \transparent{0.3}           % Opacidad del 30% para la imagen
        
        \includegraphics[width=\paperwidth, keepaspectratio]{#1}
    
        \begin{tikzpicture}[remember picture, overlay]
            \node[anchor=north west, text=white, opacity=1, font=\fontsize{60}{90}\selectfont\bfseries\sffamily, align=left] at (#5, #6) {#2};
            
            \node[anchor=south east, text=white, opacity=1, font=\fontsize{12}{18}\selectfont\sffamily, align=right] at (9.7, 3) {\href{https://losdeldgiim.github.io/}{\textbf{Los Del DGIIM}, \texttt{\footnotesize losdeldgiim.github.io}}};
            
            \node[anchor=south east, text=white, opacity=1, font=\fontsize{12}{15}\selectfont\sffamily, align=right] at (9.7, 1.8) {Doble Grado en Ingeniería Informática y Matemáticas\\Universidad de Granada};
        \end{tikzpicture}
    \end{figure}
    
    
    \restoregeometry        % Restaurar márgenes normales para las páginas subsiguientes
    \nopagecolor      % Restaurar el color de página
    
    
    \newpage
    \thispagestyle{empty}               % Sin encabezado ni pie de página
    \begin{tikzpicture}[remember picture, overlay]
        \node[anchor=south west, inner sep=3cm] at (current page.south west) {
            \begin{minipage}{0.5\paperwidth}
                %\href{https://creativecommons.org/licenses/by-nc-nd/4.0/}{
                %    \includegraphics[height=2cm]{../../_assets/Licencia.png}
                %}\vspace{1cm}\\
                Esta obra está bajo una
                \href{https://creativecommons.org/licenses/by-nc-nd/4.0/}{
                    Licencia Creative Commons Atribución-NoComercial-SinDerivadas 4.0 Internacional (CC BY-NC-ND 4.0).
                }\\
    
                Eres libre de compartir y redistribuir el contenido de esta obra en cualquier medio o formato, siempre y cuando des el crédito adecuado a los autores originales y no persigas fines comerciales. 
            \end{minipage}
        };
    \end{tikzpicture}
    
    
    
    % 1. Foto de fondo
    % 2. Título
    % 3. Encabezado Izquierdo
    % 4. Color de fondo
    % 5. Coord x del titulo
    % 6. Coord y del titulo
    % 7. Fecha
    % 8. Autor


}


\newcommand{\portadaBook}[8]{

    % 1. Foto de fondo
    % 2. Título
    % 3. Encabezado Izquierdo
    % 4. Color de fondo
    % 5. Coord x del titulo
    % 6. Coord y del titulo
    % 7. Fecha
    % 8. Autor

    % Personaliza el formato del título
    \pretitle{\begin{center}\bfseries\fontsize{42}{56}\selectfont}
    \posttitle{\par\end{center}\vspace{2em}}
    
    % Personaliza el formato del autor
    \preauthor{\begin{center}\Large}
    \postauthor{\par\end{center}\vfill}
    
    % Personaliza el formato de la fecha
    \predate{\begin{center}\huge}
    \postdate{\par\end{center}\vspace{2em}}
    
    \title{#2}
    \author{\href{https://losdeldgiim.github.io/}{Los Del DGIIM, \texttt{\large losdeldgiim.github.io}}
    \\ \vspace{0.5cm}#8}
    \date{Granada, #7}
    \maketitle
    
    \tableofcontents
}




\newcommand{\portadaArticle}[8]{

    % 1. Foto de fondo
    % 2. Título
    % 3. Encabezado Izquierdo
    % 4. Color de fondo
    % 5. Coord x del titulo
    % 6. Coord y del titulo
    % 7. Fecha
    % 8. Autor

    % Personaliza el formato del título
    \pretitle{\begin{center}\bfseries\fontsize{42}{56}\selectfont}
    \posttitle{\par\end{center}\vspace{2em}}
    
    % Personaliza el formato del autor
    \preauthor{\begin{center}\Large}
    \postauthor{\par\end{center}\vspace{3em}}
    
    % Personaliza el formato de la fecha
    \predate{\begin{center}\huge}
    \postdate{\par\end{center}\vspace{5em}}
    
    \title{#2}
    \author{\href{https://losdeldgiim.github.io/}{Los Del DGIIM, \texttt{\large losdeldgiim.github.io}}
    \\ \vspace{0.5cm}#8}
    \date{Granada, #7}
    \thispagestyle{empty}               % Sin encabezado ni pie de página
    \maketitle
    \vfill
}
    \portadaExamen{ffccA4.jpg}{Álgebra I\\Examen II}{Álgebra I. Examen II}{MidnightBlue}{-8}{28}{2023-2024}{José Juan Urrutia Milán\\Arturo Olivares Martos}

    
    \begin{description}
        \item[Asignatura] Álgebra I.
        \item[Curso Académico] 2021-22.
        \item[Grado] Doble Grado en Ingeniería Informática y Matemáticas.
        \item[Grupo] Único.
        \item[Profesor] María Pilar Carrasco Carrasco.
        \item[Descripción] Examen Ordinario.
        \item[Fecha] 14 de enero de 2022.
        \item[Duración] 3 horas.
    
    \end{description}
    \newpage
    
    \begin{ejercicio}
        Verdadero o falso:
        \begin{itemize}
            \item Sea $A$ un DE con $\phi:A\setminus\{0\} \rightarrow \bb{N}$ su función euclídea. Sean $a,b \in A$ elementos no nulos tales que $a\mid b$ y $b\nmid a$. Entonces: $\phi(a)<\phi(b)$.
            \item El anillo $\bb{Z}_{120} \times \bb{Z}_{60}$ tiene 48 unidades.
            \item En $\bb{Z}[\sqrt{2}]$, los elementos $\sqrt{2}$ y $4+3\sqrt{2}$ son unidades.
            \item Sea $A$ un anillo conmutativo, entonces $U(A) = U(A[x])$.
            \item Sea $\alpha = a+bi \in \bb{Z}[i]$ con $a,b \neq 0$. Entonces, $\alpha$ es irreducible si, y sólo si $a^2 + b^2$ es primo en $\bb{Z}$.
        \end{itemize}
    \end{ejercicio}

    \begin{ejercicio}
        Encuentre un polinomio $f(x)\in \bb{Z}_7[x]$ de grado $8$ tal que:\newline
        $f(x)\equiv x+2 \mod (3x^2 + 4x + 3)$ y $f(x)\equiv 5x^2 + 2x + 3 \mod (2x^2+3)$.
    \end{ejercicio}

    \begin{ejercicio}
        \ 
        \begin{itemize}
            \item Factoriza en $\bb{Z}[i]$ el elemento $\alpha = 31 + 12i$.
            \item Estudie si es o no irreducible en $\bb{Q}[x]$ el polinomio $f(x) = x^6 + x^5 + 2x^4 - x^3 + 4x^2 + 3x +1$.
        \end{itemize}
    \end{ejercicio}

    \begin{ejercicio}
        En el anillo $\bb{Z}_2[x]$, sea $f(x) = x^3 + 1$ e $I = f(x)\bb{Z}_2[x]$, el ideal principal generado por $f(x)$. Describa el anillo cociente $\bb{Z}_2/I$, listando todos sus elementos y calculando el inverso de aquellos que lo tengan.
    \end{ejercicio}

    \newpage
    \       % -------------------------------------------------------------------------------------------------------------------------------- 
    \newpage
    \resetearcontador

    \begin{ejercicio}
        Verdadero o falso:
        \begin{itemize}
            \item Sea $A$ un DE con $\phi:A\setminus\{0\} \rightarrow \bb{N}$ su función euclídea. Sean $a,b \in A$ elementos no nulos tales que $a\mid b$ y $b\nmid a$. Entonces: $\phi(a)<\phi(b)$.\\

                \noindent
                \textbf{Verdadero:}\newline
                Como $a\mid b \Rightarrow \exists k \in A$ tal que $b = ka \Rightarrow \phi(b) = \phi(ka) \geq \phi(a)$.

                \noindent
                Supongamos que $\phi(a) = \phi(b)$:\newline
                Dividimos $a$ entre $b$, $\exists q,r \in A$ tales que:
                $$a = bq + r~~~\land~~~\left\{ \begin{tabular}{l}
                    $r = 0$ \\
                    $\lor$    \\
                    $\phi(r) < \phi(b)$
                \end{tabular}\right.$$
                Como $b \nmid a \Rightarrow r \neq 0 \Rightarrow \phi(r) < \phi(b) \Rightarrow \phi(r) < \phi(a)$
                $$r = a-bq = a -kaq = a(1-kq) \Rightarrow \phi(r) = \phi(a(1-kq)) \geq \phi(a) = \phi(b)$$
                Pero también tenemos que $\phi(r) < \phi(b)$. \underline{Contradicción}, luego $\phi(a) \neq \phi(b)$.\\

                \noindent
                En definitiva, $\phi(a) < \phi(b)$.

            \item El anillo $\bb{Z}_{120} \times \bb{Z}_{60}$ tiene 48 unidades.\\

                \noindent
                \textbf{Falso:}\newline
                $$|U(\bb{Z}_{120})| = \varphi(120) = \varphi(2^3 \cdot 3 \cdot 5) = 4 \cdot 1 \cdot 2 \cdot 4 = 32$$
                $$|U(\bb{Z}_{60})| = \varphi(60) = \varphi(2^2 \cdot 3 \cdot 5) = 2 \cdot 1 \cdot 2 \cdot 4 = 16$$

                Sea $(a,b) \in U(\bb{Z}_{120} \times \bb{Z}_{60}) \Rightarrow \exists (c,d) \in (\bb{Z}_{120} \times \bb{Z}_{60})$ tales que:
                $$(a,b)(c,d) = 1 = (1,1) \Rightarrow (ac, bd) = (1,1) \Rightarrow $$
                $$\Rightarrow \left\{\begin{tabular}{lcl}
                        $ac = 1 \mbox{ en  } \bb{Z}_{120}$ & $\Rightarrow$ & $a,c \in U(\bb{Z}_{120})$ \\
                        $\land$ & & \\
                        $bd = 1 \mbox{ en  } \bb{Z}_{60}$ & $\Rightarrow$ & $b,d \in U(\bb{Z}_{60})$ \\
                \end{tabular}\right\} \Rightarrow$$
                $$\Rightarrow U(\bb{Z}_{120} \times \bb{Z}_{60}) = \left\{ \begin{tabular}{lcl}
                        & & $a \in U(\bb{Z}_{120})$ \\
                        $(a,b)$ & $\mid$ & \\
                        & & $b \in U(\bb{Z}_{60})$
                \end{tabular} \right\}$$
                Luego: 
                $$|U(\bb{Z}_{120}\times \bb{Z}_{60})| = |U(\bb{Z}_{120})||U(\bb{Z}_{60})| = \varphi(120)\varphi(60) = 32 \cdot 16 = 2^5 \cdot 2^4 = 2^9 = 512$$

            \item En $\bb{Z}[\sqrt{2}]$, los elementos $\sqrt{2}$ y $4+3\sqrt{2}$ son unidades.\\

                \noindent
                \textbf{Falso:}\newline
                Sean $\alpha = \sqrt{2}$ y $\beta = 4 + 3\sqrt{2}$:
                $$N(\sqrt{2}) = -2 \neq \pm 1 \Rightarrow \alpha \notin U(\bb{Z}[\sqrt{2}])$$
                $$N(4 + 3\sqrt{2}) = 16 - 2\cdot 9 = -2 \neq \pm 1 \Rightarrow \beta \notin U(\bb{Z}[\sqrt{2}])$$

            \item Sea $A$ un anillo conmutativo, entonces $U(A) = U(A[x])$.\\

                \noindent
                \textbf{Falso:}\newline
                Sea $A = \bb{Z}_4$ un anillo conmutativo:\newline
                Por ser $A \subseteq A[x]$, es claro que $U(A) \subseteq U(A[x])$.\newline
                Sea $f = 2x+1 \in \bb{Z}_4[x]$:
                $$f^2 = (2x+1)^2 = (4x^2 + 4x + 1) = 1$$ 
                Luego $f^{-1} = f \Rightarrow f \in U(A[x])$. Pero $f \notin A \Rightarrow f \notin U(A)$.\newline
                Por tanto, $U(\bb{Z}_4[x]) \nsubseteq U(\bb{Z}_4) \Rightarrow U(\bb{Z}_4[x]) \neq U(\bb{Z}_4)$.


            \item Sea $\alpha = a+bi \in \bb{Z}[i]$ con $a,b \neq 0$. Entonces, $\alpha$ es irreducible si, y sólo si $a^2 + b^2$ es primo en $\bb{Z}$.\\

                \noindent
                \textbf{Verdadero:}\newline
                $\Rightarrow)$ Supongamos que $\alpha$ es irreducible:\newline
                Como $\bb{Z}[i]$ es un DFU, primo equivale a irreducible, luego:\newline
                $N(\alpha) \in \{\pm p, \pm p^2\}$ con $p \in \bb{Z}$ primo.\newline
                Por ser $N(\alpha) = a^2 + b^2 > 0 \Rightarrow N(\alpha) \in \{p, p^2\}$, con $p$ primo en $\bb{Z}$.\\

                \noindent
                Supongamos que $N(\alpha) = p^2$ con $p \in \bb{Z}$ primo:\newline
                $$N(\alpha) = p^2 \Rightarrow \alpha \sim p \Rightarrow \exists u\in U(\bb{Z}[i]) \mid p = \alpha u$$
                $$U(\bb{Z}[i]) = \{1, -1, i, -i\} \Rightarrow p \in \{\alpha, -\alpha, i\alpha, -i\alpha\}$$
                Sin embargo, como $\alpha = a+bi$ con $a,b \neq 0 \Rightarrow p \notin \bb{Z}$, \underline{contradicción}.\newline
                Luego $N(\alpha) = p$ con $p \in \bb{Z}$ primo.\\

                \noindent
                $\Leftarrow)$ Supongamos que $N(\alpha) = p \in \bb{Z}$ primo:\newline
                Supongamos que $\alpha$ es reducible $\Rightarrow \exists \gamma, \beta \notin U(\bb{Z}[i]) \mid \alpha = \beta \gamma$.
                $$p = N(\alpha) = N(\beta)N(\gamma) \Rightarrow \left\{ \begin{tabular}{l}
                    $N(\beta) \mid p $\\
                    $\land$ \\
                    $N(\gamma) \mid p$
                \end{tabular}\right\} \Rightarrow \left\{ \begin{tabular}{l}
                    $N(\beta) = \pm 1$ \\
                    $\lor$ \\
                    $N(\gamma) = \pm 1$
                \end{tabular}\right\} \Rightarrow$$
                $$\Rightarrow \left\{ \begin{tabular}{l}
                    $ \beta \in U(\bb{Z}[i])$ \\
                    $ \lor$ \\
                    $ \gamma \in U(\bb{Z}[i])$
            \end{tabular}\right.~~~~\mbox{\underline{Contradicción}}$$
            Luego $\alpha$ es irreducible.
        \end{itemize}
    \end{ejercicio}

    \begin{ejercicio}
        Encuentre un polinomio $f(x)\in \bb{Z}_7[x]$ de grado $8$ tal que:\newline
        $f(x)\equiv x+2 \mod (3x^2 + 4x + 3)$ y $f(x)\equiv 5x^2 + 2x + 3 \mod (2x^2+3)$.\\

        \noindent
        $$\left\{ \begin{tabular}{l}
                $f \equiv x+2 \mod (3x^2 + 4x + 3)$ \\
                $f \equiv 5x^2 + 2x + 3 \mod (2x^2 + 3)$ \\
        \end{tabular}\right.$$
        De la primera congruencia:
        $$f = x+2 +g\cdot(3x^2 +4x + 3) \mbox{ con  } g \in \bb{Z}_7[x]$$
        De la segunda:
        $$f = x+2 +g\cdot(3x^2 +4x + 3) \equiv 5x^2 +2x + 3 \mod (2x^2 + 3) \Rightarrow$$
        $$\Rightarrow g\cdot(3x^2 +4x + 3) \equiv 5x^2 + x + 1 \mod (2x^2+ 3)$$
        Calculamos mcd$(3x^2 + 4x + 3, 2x^2 + 3)$:
        
        \begin{center}
        \begin{tikzpicture}
            \matrix (a) [matrix of math nodes, column sep=0pt]
            {
                  3x^2 &   +4x &  +3 &    & 2x^2 + 3 \\
                -10x^2 &       & -15 &    & 5\hspace{1cm} \\
                       &   4x &  +2 &    &  \\
            };
            \draw (a-1-5.south west) -- (a-1-5.south east);
            \draw (a-1-5.north west) -- (a-1-5.south west);
            \draw (a-2-1.south west) -- (a-2-3.south east);
        \end{tikzpicture}
        \end{center}

        \begin{center}
        \begin{tikzpicture}
            \matrix (a) [matrix of math nodes, column sep=0pt]
            {
                  2x^2 &       &  +3 &    & 4x + 2 \\
                -16x^2 &  -8x  &  \  &    & 4x + 5 \\
                       &   6x &  +3 &    &  \\
                       & -20x & -10 &    &  \\
                       &      &   0 &    &  \\
            };
            \draw (a-1-5.south west) -- (a-1-5.south east);
            \draw (a-2-1.south west) -- (a-2-3.south east);
            \draw (a-4-2.south west) -- (a-4-3.south east);
            \draw (a-1-5.north west) -- (a-1-5.south west);
        \end{tikzpicture}
        \end{center}

        \noindent
        Luego mcd$(3x^2 + 4x + 3, 2x^2 + 3) = 4x+2$
        %$$\begin{tabular}{rcr}
        %    r_i & u_i & v_i \\
        %    3x^2 + 4x + 3 & 1 & 0 \\
        %    2x^2 + 3 & 0 & 1 \\
        %    4x + 2 & 1 & -5
        %\end{tabular}$$

        \begin{center}
        \begin{tikzpicture}
            \matrix (a) [matrix of math nodes, column sep=0pt]
            {
                  5x^2 &   +x  &  +1 &    & 4x + 2 \\
                -12x^2 &  -6x  &  \  &    & 3x + 4 \\
                       &   2x &  +1 &    &  \\
                       & -16x &  -8 &    &  \\
                       &      &   0 &    &  \\
            };
            \draw (a-1-5.south west) -- (a-1-5.south east);
            \draw (a-2-1.south west) -- (a-2-3.south east);
            \draw (a-4-2.south west) -- (a-4-3.south east);
            \draw (a-1-5.north west) -- (a-1-5.south west);
        \end{tikzpicture}
        \end{center}
        $$4x+2\mid 5x^2+x+1 \Rightarrow \mbox{ el sistema tiene solución }$$

        \begin{center}
        \begin{tikzpicture}
            \matrix (a) [matrix of math nodes, column sep=0pt]
            {
                  3x^2 &  +4x  &  +3 &    & 4x + 2 \\
                -24x^2 & -12x  &  \  &    & 6x + 5 \\
                       &   6x &  +3 &    &  \\
                       & -20x & -10 &    &  \\
                       &      &   0 &    &  \\
            };
            \draw (a-1-5.south west) -- (a-1-5.south east);
            \draw (a-2-1.south west) -- (a-2-3.south east);
            \draw (a-4-2.south west) -- (a-4-3.south east);
            \draw (a-1-5.north west) -- (a-1-5.south west);
        \end{tikzpicture}
        \end{center}

        $$g\cdot(3x^2 +4x + 3) \equiv 5x^2 + x + 1 \mod (2x^2+ 3) \Rightarrow$$
        $$\Rightarrow g\cdot (6x+5) \equiv 3x+4 \mod (4x+5)$$

        Calculamos mcd$(6x+5, 4x+5)$:

        \begin{center}
        \begin{tikzpicture}
            \matrix (a) [matrix of math nodes, column sep=0pt]
            {
               6x  &  +5 &    & 4x + 5 \\
              -20x & -25 &    & 5     \\
                   &   1 &    &  \\
            };
            \draw (a-1-4.south west) -- (a-1-4.south east);
            \draw (a-2-1.south west) -- (a-2-2.south east);
            \draw (a-1-4.north west) -- (a-1-4.south west);
        \end{tikzpicture}
        \end{center}
        $$\begin{tabular}{rcr}
            $r_i$ & $u_i$ & $v_i$ \\
            $6x+5$ & 1 & 0 \\
            $4x+5 $& 0 & 1 \\
            1 & 1 & -5
        \end{tabular}$$
        $$\mbox{mcd}(6x+5, 4x+5) = 1 = (6x+5) + (-5)(4x+5)$$
        $$5(4x+5) = (6x+5)-1 \Rightarrow 6x+5 \equiv 1 \mod (4x+5) \Rightarrow (3x+4)(6x+5) \equiv 3x+4\mod (4x+5)$$
        Luego $g_0= 3x+4$ es una solución particular del sistema.
        \begin{center}
        \begin{tikzpicture}
            \matrix (a) [matrix of math nodes, column sep=0pt]
            {
               3x  &  +4 &    & 4x + 5 \\
              -24x & -30 &    & 6     \\
                   &   2 &    &  \\
            };
            \draw (a-1-4.south west) -- (a-1-4.south east);
            \draw (a-2-1.south west) -- (a-2-2.south east);
            \draw (a-1-4.north west) -- (a-1-4.south west);
        \end{tikzpicture}
        \end{center}
        $g_0' = 2$ es la solución óptima del sistema.\\

        \noindent
        Luego las soluciones son:
        $$g = 2+(4x+5)h\mid h \in \bb{Z}_7[x]$$
        Buscamos ahora un polinomio $f$ de grado 8 que sea solución del sistema.\newline
        Como $f = x+2+g \cdot (3x^2 +4x+ 3)$, sea $h \in \bb{Z}_7[x]$:
        $$x+2+[2+(4x+5)h](3x^2+4x+3) = x+2+6x^2+x+6 + (4x+5)(3x^2+4x+3)h =$$
        $$=6x^2+2x+1+(4x+5)(3x^2+4x+3)h$$
        Como grd$(f) = 8 \Rightarrow$ grd$[(4x+5)(3x^2+4x+3)h] = 8$
        $$\mbox{grd}(4x+5)+\mbox{grd}(3x^2+4x+3)+\mbox{grd}(h) = 8 \Rightarrow $$
        $$\Rightarrow 1+2+\mbox{grd}(h) = 8 \Rightarrow \mbox{grd}(h) = 5$$

        $h_0 = x^5$ es una solución particular:
        $$f_0 = 6x^2 +2x+1 + (4x+5)(3x^2+4x+3)x^5 = 6x^2 +2x+1 + (4x^6+5x^5)(3x^2+4x+3) =$$
        $$= 5x^8 +3x^7 + 4x^6 + x^5 + 6x^2 + 2x+1 \mbox{ es una solución particular. }$$

        \noindent
        En general, sirve cualquier:
        $$f = 6x^2 + 2x+1 + (4x+5)(3x^2+4x+3)h \mid h \in \bb{Z}_7[x] \land \mbox{grd}(h)=5$$

    \end{ejercicio}

    \newpage

    \begin{ejercicio}
        \ 
        \begin{itemize}
            \item Factoriza en $\bb{Z}[i]$ el elemento $\alpha = 31 + 12i$.\\

                \noindent
                $$N(\alpha) = 31^2 + 12^2 = 1105 = 5 \cdot 13 \cdot 17$$
                Busco irreducibles en $\bb{Z}[i]$ cuya norma sea 5. Sea $\beta = a+bi$:
                $$N(\beta) = 5 = a^2 + b^2 \Rightarrow \left\{ \begin{tabular}{lcl}
                    $a = \pm 2$ &$ \land$ & $b = \pm 1$ \\
                    $\lor$ & & \\
                    $a = \pm 1$ & $\land$ & $b = \pm 2 $
                \end{tabular}\right\} \Rightarrow \left\{ \begin{tabular}{ll}
                    $2+i$ & \mbox{y asociados} \\ 
                    $1+2i$ & \mbox{y asociados} \\ 
                \end{tabular}\right.$$
                Como sus normas son 5, primo, son irreducibles.\\

                \noindent
                Divido $\alpha$ entre $2+i$:
                $$\dfrac{31+12i}{2+i} = \dfrac{(31+12i)(2-i)}{5} = \dfrac{62-31i+24i+12}{5} = \dfrac{74-7i}{5} \notin \bb{Z}[i]$$
                Luego $2+i \nmid \alpha$.\\

                \noindent
                Divido $\alpha$ entre $1+2i$:
                $$\dfrac{31+12i}{1+2i} = \dfrac{(31+12i)(1-2i)}{5} = \dfrac{31+12i-62i+24}{5} = \dfrac{55-50i}{5} = 11 -10i$$
                Luego $\alpha = (1+2i)\beta$ con $\beta = 11-10i$.\\

                \noindent
                Factorizo ahora $\beta$:
                $$N(\beta) = 11^2 + 10^2 = 13 \cdot 17$$
                Busco irreducibles en $\bb{Z}[i]$ cuya norma sea 13. Sea $\gamma = a+bi$:
                $$N(\gamma) = a^2 + b^2 = 13 \Rightarrow \left\{ \begin{tabular}{lcl}
                    $a = \pm 2$ & $\land$ & $b = \pm 3$ \\
                    $\lor$ & & \\
                    $a = \pm 3$ & $\land$ & $b = \pm 2 $
                \end{tabular}\right\} \Rightarrow \left\{ \begin{tabular}{ll}
                    $2+3i$ & \mbox{y asociados} \\ 
                    $3+2i$ & \mbox{y asociados} \\ 
                \end{tabular}\right.$$
                Como sus normas son 13, primo, son irreducibles.\\

                \noindent
                Divido $\beta$ entre $3+2i$:
                $$\dfrac{11-10i}{3+2i} = \dfrac{(11-10i)(3-2i)}{13} = \dfrac{33-30i-22i-20}{13} = 1-4i \Rightarrow$$
                $$\Rightarrow \alpha = (1+2i)(3+2i)\gamma \mbox{ con } \gamma = 1-4i \in \bb{Z}[i]$$
                Como $N(\gamma) = 1^2 + 4^2 = 17$, con 17 primo, $\gamma$ es irreducible.
                $$\alpha = (1+2i)(3+2i)(1-4i)$$
                Como las normas de los tres factores son distintas, no son asociados.\\
 

            \item Estudie si es o no irreducible en $\bb{Q}[x]$ el polinomio $f(x) = x^6 + x^5 + 2x^4 - x^3 + 4x^2 + 3x +1$.\\
                \noindent
                Error de enunciado, el ejercicio no puede resolverse (reducir módulo 2 o 3 no aporta ninguna información).
        \end{itemize}
    \end{ejercicio}

    \begin{ejercicio}
        En el anillo $\bb{Z}_2[x]$, sea $f(x) = x^3 + 1$ e $I = f(x)\bb{Z}_2[x]$, el ideal principal generado por $f(x)$. Describa el anillo cociente $\bb{Z}_2/I$, listando todos sus elementos y calculando el inverso de aquellos que lo tengan.\\

        \noindent
        Como $grd(f) = 3$:
        $$\bb{Z}_2/I = \{g+I \mid grd(g) < 3 \land g \in \bb{Z}_2[x]\} = $$
        $$ \{0+I, 1+I, x+I, (x+1)+I, x^2+I, (x^2+1)+I, (x^2+x)+I, (x^2+x+1)+I\}$$
        Calculamos el inverso de los elementos que lo tengan:
        \begin{itemize}
            \item $(0+I)$ no tiene inverso.
            \item $(1+I)^{-1} = (1+I)$.
            \item $(x+I)^{-1} = (x^2+I)$:

                $$\mbox{mcd}(x,f) = \mbox{mcd}(x,x^3+1) = 1 \Rightarrow \exists (x+I)^{-1}$$
                \begin{center}
                \begin{tikzpicture}
                    \matrix (a) [matrix of math nodes, column sep=0pt]
                    {
                          x^3&  + 1 &    & x\\
                         -x^3&  \  &    & x^2\\
                             &   1 &    &  \\
                    };
                    \draw (a-2-1.south west) -- (a-2-2.south east);
                    \draw (a-1-4.south west) -- (a-1-4.south east);
                    \draw (a-1-4.north west) -- (a-1-4.south west);
                \end{tikzpicture}
                \end{center}
                $$\begin{tabular}{rcr}
                    $r_i$ & $u_i$ & $v_i$ \\
                    $x^3$ +1 & 1 & 0 \\
                    x & 0 & 1 \\
                    1 & 1 & $-x^2$
                \end{tabular}$$

                $$1 = (x^3 + 1) + x(-x^2) = (x^3 + 1) + x(x^2) \mbox{ en } \bb{Z}_2[i]$$

            \item $(x+1)+I$ no tiene inverso:
                $$\mbox{mcd}(x+1,x^3+1)=x+1 \nsim 1 \Rightarrow \nexists[(x+1)+I]^{-1}$$
            \item $(x^2+I)^{-1} = (x+I)$.
            \item $(x^2+1)+I$ no tiene inverso:
                $$\mbox{mcd}(x^2+1, x^3+1) = x+1 \nsim 1 \Rightarrow \nexists [(x^2+1)+I]^{-1}$$
            \item $(x^2+x)+I$ no tiene inverso:
                $$\mbox{mcd}(x^2+x, x^3+1) = x+1 \nsim 1 \Rightarrow \nexists [(x^2+x)+I]^{-1}$$
            \item $(x^2+x+1)+I$ no tiene inverso:
                $$\mbox{mcd}(x^2+x+1, x^3+1) = x^2+x+1 \nsim 1 \Rightarrow \nexists [(x^2+x+1)+I]^{-1}$$


        \end{itemize}
    \end{ejercicio}


\end{document}
