\documentclass[12pt]{article}

% Idioma y codificación
\usepackage[spanish, es-tabla]{babel}       %es-tabla para que se titule "Tabla"
\usepackage[utf8]{inputenc}

% Márgenes
\usepackage[a4paper,top=3cm,bottom=2.5cm,left=3cm,right=3cm]{geometry}

% Comentarios de bloque
\usepackage{verbatim}

% Paquetes de links
\usepackage[hidelinks]{hyperref}    % Permite enlaces
\usepackage{url}                    % redirecciona a la web

% Más opciones para enumeraciones
\usepackage{enumitem}

% Personalizar la portada
\usepackage{titling}

% Paquetes de tablas
\usepackage{multirow}


%------------------------------------------------------------------------

%Paquetes de figuras
\usepackage{caption}
\usepackage{subcaption} % Figuras al lado de otras
\usepackage{float}      % Poner figuras en el sitio indicado H.


% Paquetes de imágenes
\usepackage{graphicx}       % Paquete para añadir imágenes
\usepackage{transparent}    % Para manejar la opacidad de las figuras

% Paquete para usar colores
\usepackage[dvipsnames]{xcolor}
\usepackage{pagecolor}      % Para cambiar el color de la página

% Habilita tamaños de fuente mayores
\usepackage{fix-cm}

% Para los gráficos
\usepackage{tikz}

% Para poder situar los nodos en los grafos
\usetikzlibrary{positioning}


%------------------------------------------------------------------------

% Paquetes de matemáticas
\usepackage{mathtools, amsfonts, amssymb, mathrsfs}
\usepackage[makeroom]{cancel}     % Simplificar tachando
\usepackage{polynom}    % Divisiones y Ruffini
\usepackage{units} % Para poner fracciones diagonales con \nicefrac

\usepackage{pgfplots}   %Representar funciones
\pgfplotsset{compat=1.18}  % Versión 1.18

\usepackage{tikz-cd}    % Para usar diagramas de composiciones
\usetikzlibrary{calc}   % Para usar cálculo de coordenadas en tikz

%Definición de teoremas, etc.
\usepackage{amsthm}
%\swapnumbers   % Intercambia la posición del texto y de la numeración

\theoremstyle{plain}

\makeatletter
\@ifclassloaded{article}{
  \newtheorem{teo}{Teorema}[section]
}{
  \newtheorem{teo}{Teorema}[chapter]  % Se resetea en cada chapter
}
\makeatother

\newtheorem{coro}{Corolario}[teo]           % Se resetea en cada teorema
\newtheorem{prop}[teo]{Proposición}         % Usa el mismo contador que teorema
\newtheorem{lema}[teo]{Lema}                % Usa el mismo contador que teorema

\theoremstyle{remark}
\newtheorem*{observacion}{Observación}

\theoremstyle{definition}

\makeatletter
\@ifclassloaded{article}{
  \newtheorem{definicion}{Definición} [section]     % Se resetea en cada chapter
}{
  \newtheorem{definicion}{Definición} [chapter]     % Se resetea en cada chapter
}
\makeatother

\newtheorem*{notacion}{Notación}
\newtheorem*{ejemplo}{Ejemplo}
\newtheorem*{ejercicio*}{Ejercicio}             % No numerado
\newtheorem{ejercicio}{Ejercicio} [section]     % Se resetea en cada section


% Modificar el formato de la numeración del teorema "ejercicio"
\renewcommand{\theejercicio}{%
  \ifnum\value{section}=0 % Si no se ha iniciado ninguna sección
    \arabic{ejercicio}% Solo mostrar el número de ejercicio
  \else
    \thesection.\arabic{ejercicio}% Mostrar número de sección y número de ejercicio
  \fi
}


% \renewcommand\qedsymbol{$\blacksquare$}         % Cambiar símbolo QED
%------------------------------------------------------------------------

% Paquetes para encabezados
\usepackage{fancyhdr}
\pagestyle{fancy}
\fancyhf{}

\newcommand{\helv}{ % Modificación tamaño de letra
\fontfamily{}\fontsize{12}{12}\selectfont}
\setlength{\headheight}{15pt} % Amplía el tamaño del índice


%\usepackage{lastpage}   % Referenciar última pag   \pageref{LastPage}
\fancyfoot[C]{\thepage}

%------------------------------------------------------------------------

% Conseguir que no ponga "Capítulo 1". Sino solo "1."
\makeatletter
\@ifclassloaded{book}{
  \renewcommand{\chaptermark}[1]{\markboth{\thechapter.\ #1}{}} % En el encabezado
    
  \renewcommand{\@makechapterhead}[1]{%
  \vspace*{50\p@}%
  {\parindent \z@ \raggedright \normalfont
    \ifnum \c@secnumdepth >\m@ne
      \huge\bfseries \thechapter.\hspace{1em}\ignorespaces
    \fi
    \interlinepenalty\@M
    \Huge \bfseries #1\par\nobreak
    \vskip 40\p@
  }}
}
\makeatother

%------------------------------------------------------------------------
% Paquetes de cógido
\usepackage{minted}
\renewcommand\listingscaption{Código fuente}

\usepackage{fancyvrb}
% Personaliza el tamaño de los números de línea
\renewcommand{\theFancyVerbLine}{\small\arabic{FancyVerbLine}}

% Estilo para C++
\newminted{cpp}{
    frame=lines,
    framesep=2mm,
    baselinestretch=1.2,
    linenos,
    escapeinside=||
}

% para minted
\definecolor{LightGray}{rgb}{0.95,0.95,0.92}
\setminted{
    linenos=true,
    stepnumber=5,
    numberfirstline=true,
    autogobble,
    breaklines=true,
    breakautoindent=true,
    breaksymbolleft=,
    breaksymbolright=,
    breaksymbolindentleft=0pt,
    breaksymbolindentright=0pt,
    breaksymbolsepleft=0pt,
    breaksymbolsepright=0pt,
    fontsize=\footnotesize,
    bgcolor=LightGray,
    numbersep=10pt
}


\usepackage{listings} % Para incluir código desde un archivo

\renewcommand\lstlistingname{Código Fuente}
\renewcommand\lstlistlistingname{Índice de Códigos Fuente}

% Definir colores
\definecolor{vscodepurple}{rgb}{0.5,0,0.5}
\definecolor{vscodeblue}{rgb}{0,0,0.8}
\definecolor{vscodegreen}{rgb}{0,0.5,0}
\definecolor{vscodegray}{rgb}{0.5,0.5,0.5}
\definecolor{vscodebackground}{rgb}{0.97,0.97,0.97}
\definecolor{vscodelightgray}{rgb}{0.9,0.9,0.9}

% Configuración para el estilo de C similar a VSCode
\lstdefinestyle{vscode_C}{
  backgroundcolor=\color{vscodebackground},
  commentstyle=\color{vscodegreen},
  keywordstyle=\color{vscodeblue},
  numberstyle=\tiny\color{vscodegray},
  stringstyle=\color{vscodepurple},
  basicstyle=\scriptsize\ttfamily,
  breakatwhitespace=false,
  breaklines=true,
  captionpos=b,
  keepspaces=true,
  numbers=left,
  numbersep=5pt,
  showspaces=false,
  showstringspaces=false,
  showtabs=false,
  tabsize=2,
  frame=tb,
  framerule=0pt,
  aboveskip=10pt,
  belowskip=10pt,
  xleftmargin=10pt,
  xrightmargin=10pt,
  framexleftmargin=10pt,
  framexrightmargin=10pt,
  framesep=0pt,
  rulecolor=\color{vscodelightgray},
  backgroundcolor=\color{vscodebackground},
}

%------------------------------------------------------------------------

% Comandos definidos
\newcommand{\bb}[1]{\mathbb{#1}}
\newcommand{\cc}[1]{\mathcal{#1}}

% I prefer the slanted \leq
\let\oldleq\leq % save them in case they're every wanted
\let\oldgeq\geq
\renewcommand{\leq}{\leqslant}
\renewcommand{\geq}{\geqslant}

% Si y solo si
\newcommand{\sii}{\iff}

% Letras griegas
\newcommand{\eps}{\epsilon}
\newcommand{\veps}{\varepsilon}
\newcommand{\lm}{\lambda}

\newcommand{\ol}{\overline}
\newcommand{\ul}{\underline}
\newcommand{\wt}{\widetilde}
\newcommand{\wh}{\widehat}

\let\oldvec\vec
\renewcommand{\vec}{\overrightarrow}

% Derivadas parciales
\newcommand{\del}[2]{\frac{\partial #1}{\partial #2}}
\newcommand{\Del}[3]{\frac{\partial^{#1} #2}{\partial #3^{#1}}}
\newcommand{\deld}[2]{\dfrac{\partial #1}{\partial #2}}
\newcommand{\Deld}[3]{\dfrac{\partial^{#1} #2}{\partial #3^{#1}}}


\newcommand{\AstIg}{\stackrel{(\ast)}{=}}
\newcommand{\Hop}{\stackrel{L'H\hat{o}pital}{=}}

\newcommand{\red}[1]{{\color{red}#1}} % Para integrales, destacar los cambios.

% Método de integración
\newcommand{\MetInt}[2]{
    \left[\begin{array}{c}
        #1 \\ #2
    \end{array}\right]
}

% Declarar aplicaciones
% 1. Nombre aplicación
% 2. Dominio
% 3. Codominio
% 4. Variable
% 5. Imagen de la variable
\newcommand{\Func}[5]{
    \begin{equation*}
        \begin{array}{rrll}
            #1:& #2 & \longrightarrow & #3\\
               & #4 & \longmapsto & #5
        \end{array}
    \end{equation*}
}

%------------------------------------------------------------------------

\newcommand{\im}{\mathit}

\begin{document}

    % 1. Foto de fondo
    % 2. Título
    % 3. Encabezado Izquierdo
    % 4. Color de fondo
    % 5. Coord x del titulo
    % 6. Coord y del titulo
    % 7. Fecha

    
    % 1. Foto de fondo
% 2. Título
% 3. Encabezado Izquierdo
% 4. Color de fondo
% 5. Coord x del titulo
% 6. Coord y del titulo
% 7. Fecha

\newcommand{\portada}[7]{

    \portadaBase{#1}{#2}{#3}{#4}{#5}{#6}{#7}
    \portadaBook{#1}{#2}{#3}{#4}{#5}{#6}{#7}
}

\newcommand{\portadaExamen}[7]{

    \portadaBase{#1}{#2}{#3}{#4}{#5}{#6}{#7}
    \portadaArticle{#1}{#2}{#3}{#4}{#5}{#6}{#7}
}




\newcommand{\portadaBase}[7]{

    % Tiene la portada principal y la licencia Creative Commons
    
    % 1. Foto de fondo
    % 2. Título
    % 3. Encabezado Izquierdo
    % 4. Color de fondo
    % 5. Coord x del titulo
    % 6. Coord y del titulo
    % 7. Fecha
    
    
    \thispagestyle{empty}               % Sin encabezado ni pie de página
    \newgeometry{margin=0cm}        % Márgenes nulos para la primera página
    
    
    % Encabezado
    \fancyhead[L]{\helv #3}
    \fancyhead[R]{\helv \nouppercase{\leftmark}}
    
    
    \pagecolor{#4}        % Color de fondo para la portada
    
    \begin{figure}[p]
        \centering
        \transparent{0.3}           % Opacidad del 30% para la imagen
        
        \includegraphics[width=\paperwidth, keepaspectratio]{assets/#1}
    
        \begin{tikzpicture}[remember picture, overlay]
            \node[anchor=north west, text=white, opacity=1, font=\fontsize{60}{90}\selectfont\bfseries\sffamily, align=left] at (#5, #6) {#2};
            
            \node[anchor=south east, text=white, opacity=1, font=\fontsize{12}{18}\selectfont\sffamily, align=right] at (9.7, 3) {\textbf{\href{https://losdeldgiim.github.io/}{Los Del DGIIM}}};
            
            \node[anchor=south east, text=white, opacity=1, font=\fontsize{12}{15}\selectfont\sffamily, align=right] at (9.7, 1.8) {Doble Grado en Ingeniería Informática y Matemáticas\\Universidad de Granada};
        \end{tikzpicture}
    \end{figure}
    
    
    \restoregeometry        % Restaurar márgenes normales para las páginas subsiguientes
    \pagecolor{white}       % Restaurar el color de página
    
    
    \newpage
    \thispagestyle{empty}               % Sin encabezado ni pie de página
    \begin{tikzpicture}[remember picture, overlay]
        \node[anchor=south west, inner sep=3cm] at (current page.south west) {
            \begin{minipage}{0.5\paperwidth}
                \href{https://creativecommons.org/licenses/by-nc-nd/4.0/}{
                    \includegraphics[height=2cm]{assets/Licencia.png}
                }\vspace{1cm}\\
                Esta obra está bajo una
                \href{https://creativecommons.org/licenses/by-nc-nd/4.0/}{
                    Licencia Creative Commons Atribución-NoComercial-SinDerivadas 4.0 Internacional (CC BY-NC-ND 4.0).
                }\\
    
                Eres libre de compartir y redistribuir el contenido de esta obra en cualquier medio o formato, siempre y cuando des el crédito adecuado a los autores originales y no persigas fines comerciales. 
            \end{minipage}
        };
    \end{tikzpicture}
    
    
    
    % 1. Foto de fondo
    % 2. Título
    % 3. Encabezado Izquierdo
    % 4. Color de fondo
    % 5. Coord x del titulo
    % 6. Coord y del titulo
    % 7. Fecha


}


\newcommand{\portadaBook}[7]{

    % 1. Foto de fondo
    % 2. Título
    % 3. Encabezado Izquierdo
    % 4. Color de fondo
    % 5. Coord x del titulo
    % 6. Coord y del titulo
    % 7. Fecha

    % Personaliza el formato del título
    \pretitle{\begin{center}\bfseries\fontsize{42}{56}\selectfont}
    \posttitle{\par\end{center}\vspace{2em}}
    
    % Personaliza el formato del autor
    \preauthor{\begin{center}\Large}
    \postauthor{\par\end{center}\vfill}
    
    % Personaliza el formato de la fecha
    \predate{\begin{center}\huge}
    \postdate{\par\end{center}\vspace{2em}}
    
    \title{#2}
    \author{\href{https://losdeldgiim.github.io/}{Los Del DGIIM}}
    \date{Granada, #7}
    \maketitle
    
    \tableofcontents
}




\newcommand{\portadaArticle}[7]{

    % 1. Foto de fondo
    % 2. Título
    % 3. Encabezado Izquierdo
    % 4. Color de fondo
    % 5. Coord x del titulo
    % 6. Coord y del titulo
    % 7. Fecha

    % Personaliza el formato del título
    \pretitle{\begin{center}\bfseries\fontsize{42}{56}\selectfont}
    \posttitle{\par\end{center}\vspace{2em}}
    
    % Personaliza el formato del autor
    \preauthor{\begin{center}\Large}
    \postauthor{\par\end{center}\vspace{3em}}
    
    % Personaliza el formato de la fecha
    \predate{\begin{center}\huge}
    \postdate{\par\end{center}\vspace{5em}}
    
    \title{#2}
    \author{\href{https://losdeldgiim.github.io/}{Los Del DGIIM}}
    \date{Granada, #7}
    \thispagestyle{empty}               % Sin encabezado ni pie de página
    \maketitle
    \vfill
}
    \portadaExamen{ffccA4.jpg}{Álgebra I\\Parcial IV}{Álgebra I. Parcial IV}{MidnightBlue}{-8}{28}{2023-2024}{Arturo Olivares Martos\\Joaquín Avilés de la Fuente}

    \begin{description}
        \item[Asignatura] Álgebra I.
        \item[Curso Académico] 2022-23.
        \item[Grado] Grado en Matemáticas.
        %\item[Grupo] -
        %\item[Profesor] -
        \item[Descripción] Parcial I
        \item[Fecha] 21 de diciembre de 2022.
        %\item[Duración] -
    \end{description}
    \newpage
    \subsubsection*{Parte 1: Cuestionario}

    \begin{ejercicio}[1 punto]
        Selecciona la afirmación correcta relativa al número $13\in \bb{Z}$.
        \begin{enumerate}[label=$\square$]
            \item No es irreducible en $\bb{Z}[i]$ pero sí en $\bb{Z}[\sqrt{-3}]$.
            \item[{$\blacksquare$}] No es irreducible en $\bb{Z}[i]$ ni en $\bb{Z}[\sqrt{-3}]$.
            \item Es irreducible en $\bb{Z}[i]$ pero no en $\bb{Z}[\sqrt{-3}]$.
        \end{enumerate}~\\
        Estudiemos la irreducibilidad de $13$ en $\bb{Z}[i]$ y $\bb{Z}[\sqrt{-3}]$. Comencemos por $\bb{Z}[i]$. \\ \\
        Supongamos $13$ reducible, entonces $13=\beta\gamma$, $\beta,\gamma\notin \bb{Z}[i]$ y tenemos $\mbox{$N(13)=N(\beta)N(\gamma)$}$, de donde podemos deducir:
        \begin{align*}
            13^2&=N(\beta)N(\gamma) \Longrightarrow \left\{
            \begin{array}{c}
                N(\beta)=13 \Longrightarrow \beta=2+3i \text{ y asociados} \\ 
                \wedge \\
                N(\gamma)=13 \Longrightarrow \gamma=3+2i \text{ y asociados}
            \end{array}\right. 
            \Longrightarrow 13=(2+3i)(3+2i)(-i)
        \end{align*}
        Tenemos finalmente que $13$ es reducible en $\bb{Z}[i]$. Veamos a continuación el caso para $\bb{Z}[\sqrt{-3}]$. \\ \\
        Supongamos $13$ reducible, entonces $13=\beta\gamma$, $\beta,\gamma\notin \bb{Z}[\sqrt{-3}]$ con $\beta=a+b\sqrt{-3}$ y $\gamma=c+d\sqrt{-3}$. Sabemos que entonces
        $N(13)=N(\beta)N(\gamma)$, de donde podemos deducir
        \begin{align*}
            13^2&=N(\beta)N(\gamma) \Longrightarrow \left\{
            \begin{array}{c}
                N(\beta)=13 \Longrightarrow a^2+3b^2=13\Longrightarrow \beta=\pm1\pm2\sqrt{-3} \text{ y asociado} \\
                \wedge \\
                N(\gamma)=13 \Longrightarrow c^2+3d^2=13\Longrightarrow \gamma=\pm1\pm2\sqrt{-3} \text{ y asociado}
            \end{array} \right.  \\ \\
            & \overset{(*)}\Longrightarrow 13=(1-2\sqrt{-3})(1+2\sqrt{-3})
        \end{align*}
        donde en ($\ast$) se ha usado lo siguiente:
        \begin{equation*}
            \frac{13}{1+2\sqrt{-3}}=\frac{13(1-2\sqrt{-3})}{13}=1-2\sqrt{-3}
        \end{equation*}
        Sabiendo que $\dfrac{1-2\sqrt{-3}}{1+2\sqrt{-3}}=\dfrac{(1-2\sqrt{-3})^2}{13}\notin \bb{Z}[\sqrt{-3}]$ tenemos por tanto que $13$ es reducible en $\bb{Z}[\sqrt{-3}]$.
    \end{ejercicio}

    \begin{ejercicio}[1 punto]
        Selecciona la afirmación correcta relativa al polinomio \\ 
        $f=25x^6+125x^3+5 \in \bb{Z}[x]$
        \begin{enumerate}[label=$\square$]
            \item $f$ es irreducible en $\bb{Z}[x]$ y en $\bb{Q}[x]$.
            \item $f$ no es irreducible en $\bb{Z}[x]$ ni en $\bb{Q}[x]$.
            \item[$\blacksquare$] $f$ no es irreducible en $\bb{Z}[x]$, pero sí en $\bb{Q}[x]$.
        \end{enumerate}
        Sea $f=5g$ donde $g=5x^6+25x^3+1\in \bb{Z}[x]$, por Eisenstein para $p=5$, $g$ es irreducible y tenemos 
        \begin{equation*}
            f=5(5x^6+25x^3+1)\in \bb{Z}[x]
        \end{equation*}
        Como $5$ es irreducible en $\bb{Z}[x]$, $f$ es reducible en $\bb{Z}[x]$ y como $5$ es unidad en $\bb{Q}[x]$, $f$ es irreducible en $\bb{Q}$.
    \end{ejercicio}

    \begin{ejercicio}[1 punto]
        Entre las siguientes proposiciones, seleccione las verdaderas:
        \begin{enumerate}[label=$\blacksquare$]
            \item El anillo $\bb{Z}_{441}$ tiene 252 unidades
            \item $47^{22}\equiv 31 \bmod{(33)}$
            \item $3^{18}\equiv 9$ en $\bb{Z}_{16}$
        \end{enumerate}
        Estudiemos la veracidad de la primera afirmación
        \begin{equation*}
            |\mathcal{U}(\bb{Z}_{441})|=\varphi(441)=\varphi(3^2\cdot 7^2)=3\cdot7\cdot2\cdot6=252\Longrightarrow \textbf{afirmación cierta}
        \end{equation*}
        Estudiemos la veracidad de la segunda afirmación.
        \begin{equation*}
            \left.\begin{array}{c}
                \text{mcd}{(47,33)}=1 \\
                \varphi(33)=\varphi(3\cdot11)=2\cdot10=20
            \end{array} \right\} \Longrightarrow 47^{20}\equiv1\bmod{(33)}
        \end{equation*}
        Tenemos entonces 
        \begin{align*}
            &\left. \begin{array}{c}
                47^{20}\equiv1\bmod{(33)} \\
                47^2=2209=66\cdot33+31\Longrightarrow 47^2\equiv31\bmod{(33)}
            \end{array} \right\} \Longrightarrow 47^{20}\cdot 47^2\equiv1\bmod{(33)} \\ \\
            &\Longrightarrow 47^{22}\equiv31\bmod{(33)}\Longrightarrow \textbf{afirmación cierta}
        \end{align*}
        Estudiemos por último la tercera afirmación. Es claro que dicha afirmación equivale a $3^{18}=9\bmod{(16)}$, veámos entonces como podemos analizarla
        \begin{equation*}
            \left.\begin{array}{c}
                \text{mcd}{(3,16)}=1 \\
                \varphi(16)=\varphi(2^4)=2^3=8
            \end{array} \right\} \Longrightarrow 3^8\equiv1\bmod{(16)} \Longrightarrow 3^{16}\equiv1\bmod{(16)}
        \end{equation*}
        Tenemos por tanto
        \begin{equation*}
            \left.\begin{array}{c}
                3^2=9\equiv9\bmod{(16)}\\
                3^{16}\equiv1\bmod{(16)}
            \end{array} \right\} \Longrightarrow 3^{18}\equiv9\bmod{(16)}\Longrightarrow 3^{18}=9 \text{ en } \bb{Z}_{16}\Longrightarrow \textbf{afirmación cierta}
        \end{equation*}
    \end{ejercicio}

    \begin{ejercicio}[1 punto]
        Si $n\geq 1$ es un entero, la afirmación \textbf{``la ecuación diofántica $34x+51y=5^{2n}-2^{3n}$ tiene solución''} es:
        \begin{itemize}[label=$\square$]
            \item [$\blacksquare$]siempre verdad
            \item siempre falsa
            \item verdad o falsa, depende de n
        \end{itemize}
        Sabemos que la ecuación tiene solución sii $\text{mcd}(34,51)=17 \mid 5^{2n}-2^{3n}$, por lo que demostrémoslo por inducción. \\ \\
        Para $n=1$ tenemos $17\mid 5^{2}-2^3=17$ que es cierto.
        Supuesto cierto para $n$, comprobémoslo para $n+1$, donde comenzaremos usando la hipótesis de inducción $ 17 \mid 5^{2n}-2^{3n}$.
        \begin{align*}
            &5^{2n}\equiv 2^{3n}\bmod(17)\Longrightarrow 5^{2n}\cdot 5^2\equiv 2^{3n}\cdot5^2\bmod(17) \\
            &\Longrightarrow 5^{2(n+1)}-2^ {3(n+1)}\equiv 2^{3n}\cdot5^2-2^ {3(n+1)}\bmod(17) \\
            &\Longrightarrow 5^{2(n+1)}-2^ {3(n+1)}\equiv 2^{3n}\cdot(5^2-2^3)\bmod(17) \\
            &\Longrightarrow 5^{2(n+1)}-2^ {3(n+1)}\equiv 2^{3n}\cdot17\bmod(17)\Longrightarrow 17\mid 5^{2(n+1)}-2^ {3(n+1)}
        \end{align*}
        entonces $\exists$ sol $\forall n\in \bb{N}, n\neq 0$.
    \end{ejercicio}
    \begin{ejercicio}[1 punto]
        Dados los anillos cocientes $\bb{Z}_3[x]/x^3 +x +1, \bb{Z}_3[x]/x^3 -x +1$ y $\bb{Z}_3[x]/x^3 +x^2 -1$, selecciona las afirmaciones correctas.
        \begin{itemize}[label=$\square$]
            \item Sólo uno de ellos es cuerpo.
            \item Dos de ellos son cuerpos y uno no lo es.
            \item Ninguno de ellos es cuerpo.
        \end{itemize}  
        Estudiemos si los anillos cocientes son cuerpos uno a uno.\\ \\
        Para $\bb{Z}_3[x]/x^3 +x +1$ veamos si $f =x^3+x+1 \in \bb{Z}_3[x]$ es irreducible
        \begin{equation*}
            \begin{array}{l}
                f(0)=1\neq0 \\
                f(1)=3=0\Longrightarrow (x-1)=(x-2)\mid f\Longrightarrow f \text{ es reducible}
            \end{array}
        \end{equation*}
        entonces $\bb{Z}_3[x]/x^3 +x +1$ no es un cuerpo. \\ \\
        Para $\bb{Z}_3[x]/x^3 -x +1$ veamos si $f =x^3-x+1 \in \bb{Z}_3[x]$ es irreducible
        \begin{equation*}
            \left.\begin{array}{l}
                f(0)=1\neq0 \\
                f(1)=1-1+1=1\neq0\\
                f(2)=8-2+1=7=1\neq0
            \end{array}\right\}  \Longrightarrow f \text{ es irreducible por el criterio de la raíz}
        \end{equation*}
        entonces $\bb{Z}_3[x]/x^3 +x^2 -1$ sí es un cuerpo. \\ \\
        Para $\bb{Z}_3[x]/x^3 +x^2 -1$ veamos si $f =x^3+x^2-1 \in \bb{Z}_3[x]$ es irreducible
        \begin{equation*}
            \left.\begin{array}{l}
                f(0)=-1=2\neq0 \\
                f(1)=1+1-1=1\neq0\\
                f(2)=8+4-1=2\neq0
            \end{array}\right\}  \Longrightarrow f \text{ es irreducible por el criterio de la raíz}
        \end{equation*}
        entonces $\bb{Z}_3[x]/x^3 +x^2 -1$ sí es un cuerpo
    \end{ejercicio}
    \subsubsection*{Parte 2: Ejercicios}
    \setcounter{ejercicio}{0} % Reinicia el contador de ejercicios

    \begin{ejercicio}[1,25 puntos]
        Factorizar en producto de irreducibles en $\bb{Z}[x]$ el polinomio
        \begin{equation*}
            f=20x^5-10x^4+60x^2-10x-10
        \end{equation*}
        Sea $f=10g$ y $g=2x^5-x^4+6x^2-x-1\in \bb{Z}[x]$. Las posibles raíces de $g$ en $\bb{Q}$ son:$\{\pm1,\nicefrac{\pm1}{2}\}$, tenemos entonces
        \begin{align*}
            \left.\begin{array}{l}
            g(1)=2-1+6-1-1=5\neq0\\
            g(-1)=-2-1+6+1-1\neq0 \\
            g(\frac{1}{2})=0\Longrightarrow(x-\nicefrac{1}{2})\mid g \text{ en } \bb{Q}[x] \Longrightarrow (2x-1)\mid g \text{ en } \bb{Z}[x]
            \end{array}\right\} \Longrightarrow f=5\cdot2\cdot(2x-1)h 
        \end{align*}
        donde $h=x^4+3x+1\in\bb{Z}[x]$ viene dado del siguiente cálculo:\\
        \[
        \polylongdiv[style=C]{2x^5-x^4+6x^2-x-1}{2x - 1}
        \]
        Como las posibles raíces de $h$ en $\bb{Q}$ son: $\{\pm1\}$, tenemos 
        \begin{align*}
            \left.\begin{array}{l}
                h(1)=1+3+1\neq0 \\
                h(-1)=1-3+1=-1\neq0
            \end{array} \right\} \Longrightarrow h\text{ no tiene factores de grado }1 \text{ ni }3
        \end{align*}
        Reduciendo módulo $2$ obtenemos $R_2(h)=x^4+x+1\in\bb{Z}_2[x]$, entonces
        \begin{align*}
            \left.\begin{array}{l}
                R_2(h)(0)=1\neq0 \\
                R_2(h)(1)=3=1\neq0
            \end{array} \right\} \Longrightarrow R_2(h) \text{ no tiene factores de grado }1 \text{ ni }3
        \end{align*}
        Veamos ahora si es que $R_2(h)$ tiene factores de grado $2$ mediante la siguiente división
        \begin{center}
            \begin{tikzpicture}
                \matrix (a) [matrix of math nodes, column sep=0pt]
                {
                      x^4 &      &      & +x &  +1 &    & x^2+x+1 \\
                     -x^4 & -x^3 & -x^2 &    &     &    & x^2+x\hspace{1cm} \\
                          &  x^3 & +x^2 & +x & +1  &    & \\
                          & -x^3 & -x^2 & -x &     &    & \\
                          &      &      &    & +1  &   & \\
                };
                \draw (a-1-7.south west) -- (a-1-7.south east);
                \draw (a-1-7.north west) -- (a-1-7.south west);
                \draw (a-2-1.south west) -- (a-2-3.south east);
                \draw (a-4-2.south west) -- (a-4-4.south east);
            \end{tikzpicture}
        \end{center}
        Como el resto es $1\neq0$ tenemos que $x^2+x+1 \nmid R_2(h)$, por lo que $R_2(h)$ no tiene factores de grado 2.
        Tenemos por tanto que $R_2(h)$ es irreducible y  por consecuencia $h$ también es irreducible, obteniendo así la factorización del polinomio pedido como
        \begin{equation*}
            f=5\cdot 2\cdot (2x-1)(x^4+3x+1)
        \end{equation*}
    \end{ejercicio}
    \begin{ejercicio} [1,25 puntos]
        Encuentra una solución al siguiente sistema de congruencia en $\bb{Z}$ que esté entre $4000$ y $6000$.
        \begin{align*}
            \left\{\begin{array}{l}
                4x\equiv 4\bmod{(8)}\\ 
                x\equiv86\bmod{(121)}\\
                x\equiv2\bmod{(7)}
            \end{array}\right.
        \end{align*}
        Resolvemos primero el siguiente sistema
        \begin{align*}
            \left\{
            \begin{array}{l}
                x \equiv 1 \bmod{2} \Longrightarrow x = 1 + 2k, \quad k \in \mathbb{Z} \\
                x \equiv 86 \bmod{121}
            \end{array}
            \right.
            \Longrightarrow x = 1 + 2k \equiv 86 \bmod{121}
            \Longrightarrow 2k \equiv 85 \bmod{121}
        \end{align*}
        Como $\text{mcd}(2,121)=1=61\cdot2+121\cdot(-1)$, entonces tenemos
        \begin{multline*}
            61\cdot2\equiv1\bmod{(121)}\Longrightarrow61\cdot2\cdot85\equiv85\bmod{(121)}
            \Longrightarrow k_0'=61\cdot85 \text{ es solución particular} \\
            \Longrightarrow k_0'=61\cdot85=5185=42\cdot121+103\Longrightarrow k_0=103 \text{ es solución óptima} \\
            \Longrightarrow k=103+121t, t\in\bb{Z}
        \end{multline*}
        Dada $k_0$ la solución óptima veamos ahora sustituyendo en $x$ la solución obtenida del sistema
        \begin{equation*}
            x=1+2k=1+2(103+121t)=1+206+242t\Longrightarrow x=207+242t, t\in\bb{Z}
        \end{equation*}
        Por tanto el sistema a resolver es
        \begin{align*}
            \left\{\begin{array}{l}
                x\equiv207\bmod{(242)}\\
                x\equiv2\bmod{(7)}\Longrightarrow x=2+7k, k\in\bb{Z}
            \end{array}\right. &\Longrightarrow x=2+7k\equiv207\bmod{(242)}\\
            &\Longrightarrow7k\equiv205\bmod{(242)}
        \end{align*}
        Por el algoritmo extendido de Euclídes, desarrollado a continuación, tenemos que $\text{mcd}(7.242)=1=2\cdot242-69\cdot7$
        \begin{equation*}
            \begin{array}{c|c|c}
                r_i & u_i & v_i\\ \hline
                242 & 1 & 0\\
                7 & 0 & 1 \\
                4 & 1 & -34 \\
                3 & -1 & 35 \\
                1 & 2 & -69 \\
                0 & 0 & 0
            \end{array}
        \end{equation*}
        Por tanto usando el mcd dado y mediante el siguiente desarrollo tenemos la solución del sistema pedido
        \begin{multline*}
            -69\cdot7\equiv 1\bmod{(242)}\Longrightarrow -205\cdot69\cdot7\equiv205\bmod{(242)} \\
            \Longrightarrow k_0'=-205\cdot69 \text{ es solución particular} \Longrightarrow k_0=133 \text{ es solución óptima} \\
            \Longrightarrow k=133+242t, t\in\bb{Z}\Longrightarrow x=2+7k=2+7(133+242t), t\in\bb{Z}\\
            \Longrightarrow x=933+1694t, t\in\bb{Z}
        \end{multline*}
    \end{ejercicio}

    \begin{ejercicio} [1,25 puntos]
        Factoriza en irreducibles $11+3\mathit{i}$ en productos irreducibles en $\bb{Z}[i]$ \\ \\
        Supongamos $\alpha=11+3\mathit{i}=\beta\gamma$, entonces tenemos
        \begin{equation*}
            N(\alpha)=N(\beta)N(\gamma)\Longrightarrow11^2+3^2=13\cdot5\cdot2=N(\beta)N(\gamma)
        \end{equation*}
        Buscamos irreducibles de norma $2$
        \begin{multline*}
            N(\beta)=a^2+b^2\Longrightarrow a=\pm 1 \wedge b=\pm 1\Longrightarrow (1+i)\text{ y asociados}\\
            \overset{(*)}\Longrightarrow \alpha=(1+\mathit{i})\beta, \beta=7-4\mathit{i}
        \end{multline*}
        donde en (*) se ha usado que
        \begin{equation*}
            \frac{11+3\im{i}}{1+i}=\frac{(11+3\im{i})(1-\im{(1-i)})}{2}=\frac{11+3\im{i}-11\im{i}+3}{2}=7-4\im{i}
        \end{equation*}
        Busquemos ahora irreducibles de norma 5
        \begin{align*}
            N(\gamma)=a^2+b^2\Longrightarrow &\left\{\begin{array}{c}
                a=\pm2 \wedge b=\pm1\Longrightarrow(2+\im{i}) \text{ y asociados} \\
                \vee \\
                a=\pm1 \wedge b=\pm2\Longrightarrow(1+2\im{i}) \text{ y asociados} \\
            \end{array}\right. \Longrightarrow \\
            \\
            &\overset{(*)}\Longrightarrow \alpha=(1+\im{i})(2+\im{i})(2-3\im{i})
        \end{align*}
        donde en (*) se ha usado que
        \begin{equation*}
            \frac{7-4\im{i}}{2+\im{i}}=\frac{(2-\im{i})(7-4\im{i})}{5}=\frac{4-8\im{i}-7\im{i}-4}{5}=2-3\im{i}
        \end{equation*}
        Como $N(2-3\im{i})=4+9=13$ y $13$ es primo, entonces $2-3\im{i}$ es primo. Además, como sus  normas son distintas y no son asociados, tenemos finalmente la factorización pedida
        \begin{equation*}
            11+3\im{i}=\alpha=(1+\im{i})(2+\im{i})(2-3\im{i})
        \end{equation*}
    \end{ejercicio}
    \begin{ejercicio} [1,25 puntos]
        Encuentra todos los polinomios (si los hay) $f,g\in\bb{R}[x]$, con $f$ mónico y de grado 2, tales que:
        \begin{equation*}
            (x^5+x^4+2x^3+2x^2+x+1)\cdot f+(x^4+x^3+2x^2+x+1)\cdot g=x^3+2x^2+x+2
        \end{equation*}
        Desarrolando el algoritmo extendido de Euclídes tenemos:
        \begin{equation*}
            \begin{array}{c|c|cl}
                r_i & u_i & v_i & \\ \hline
                x^5+x^4+2x^3+2x^2+x+1 & 1 & 0  &\\
                x^4+x^3+2x^2+x+1 & 1 & 1 & \\
                x^2+1 & 1 & -x & \quad  \quad(\ast) \\
                0 & 0 & 0
            \end{array}
        \end{equation*}
        donde ($\ast$) viene dado por 
        \begin{center}
        \begin{tikzpicture}
            \matrix (a) [matrix of math nodes, column sep=1pt]{
                x^5  & +x^4 & +2x^3 & +2x^2 & +x & +1 &  & x^4 +x^3 +2x^2 +x +1 \\
                -x^5 & -x^4 & -2x^3 & -x^2 & -x &  &  & x \hspace{3.5cm} \\
                        &      &       &  x^2 &    &  +1 &   & \\
            };
            \draw (a-1-8.north west) -- (a-1-8.south west);
            \draw (a-1-8.south west) -- (a-1-8.south east);
            \draw (a-2-1.south west) -- (a-2-5.south east);
        \end{tikzpicture}
        \end{center}
        Se tiene entonces que $\text{mcd}(x^5+x^4+2x^3+2x^2+x+1,x^4+x^3+2x^2+x+1)=x^2+1$, comprobemos a continuación
        si $x^2+1$ divide a $x^3+2x^2+x+2$
        \begin{center}
        \begin{tikzpicture}
            \matrix (a) [matrix of math nodes, column sep=1pt]{
                x^3  & +2x^2 & +x & +2 &  & x^2 +1 \\
                -x^3 &       & -x &    &  & x +2 \\
                        &  2x^2 &    & +2 & \\
                        & -2x^2 &    & -2 & \\ 
                        &       &    &  0 & \\
            };
            \draw (a-1-6.north west) -- (a-1-6.south west);
            \draw (a-1-6.south west) -- (a-1-7.south east);
            \draw (a-2-1.south west) -- (a-2-3.south east);
            \draw (a-4-2.south west) -- (a-4-4.south east);
        \end{tikzpicture}
        \end{center}
        Como efectivamente hemos demostrado que $x^2+1 \mid x^3+2x^2+x+2$, tenemos a continuación la ecuación diofántica reducida 
        al dividir entre $x^2+1$
        \begin{equation*}
            (x^3+x^2+x+1)\cdot f +(x^2+x+1)\cdot g=x+2
        \end{equation*}
        ecuación obtenida a partir de estas divisiones rutinarias de polinomios
        \begin{center}
        \begin{minipage}{0.45\textwidth}
        \fontsize{10}{10}        
        \begin{tikzpicture}
            \matrix (a) [matrix of math nodes, column sep=-4pt]{
                x^5  & +x^4 & +2x^3 & +2x^2 & +x & +1 & x^2+1 \hspace{0.75cm} \\
                -x^5 &      & -x^3  &       &    &    & \hspace{0.5cm}x^3+x^2+x+1 \\
                     & x^4  & +x^3  & +2x^2 & +x & +1 \\
                     & -x^4 &       & -x^2 \\ 
                     & &  x^3  & +x^2  & +x & +1 \\
                     & & -x^3  &       & -x \\
                     & & & x^2  & & +1 \\
                     & & & -x^2 & & -1 \\
                     & & & & & 0 \\
            };
            \draw (a-1-7.south west) -- (a-1-7.south east);
            \draw (a-1-7.south west) -- (a-1-7.north west);
            \draw (a-2-1.south west) -- (a-2-3.south east);
            \draw (a-4-2.south west) -- (a-4-4.south east);
            \draw (a-6-3.south west) -- (a-6-5.south east);
            \draw (a-8-4.south west) -- (a-8-6.south east);
        \end{tikzpicture}
        \end{minipage}
        \hfill
        \begin{minipage}{0.45\textwidth}
        \fontsize{10}{10}        
        \begin{tikzpicture}
            \matrix (a) [matrix of math nodes, column sep=-4pt]{
                x^4  & +x^3 & +2x^2 & +x & +1 & x^2+1\hspace{0.15cm} \\
                -x^4 &      & -x^2  &    &    & \hspace{0.5cm}x^2+x+1 \\
                     & x^3  & +x^2  & +x & +1 \\
                     & -x^3 &       & -x \\
                     & & x^2 & & +1 \\
                     & & -x^2 & & -1 \\
                     & & & & 0 \\
            };
            \draw (a-1-6.south west) -- (a-1-6.south east);
            \draw (a-1-6.north west) -- (a-1-6.south west);
            \draw (a-2-1.south west) -- (a-2-3.south east);
            \draw (a-4-2.south west) -- (a-4-4.south east);
            \draw (a-6-3.south west) -- (a-6-5.south east);
        \end{tikzpicture}
        \end{minipage}
        \end{center}
        Como sabemos que $\text{mcd}(x^3+x^2+x+1,x^2+x+1)=1=1\cdot(x^3+x^2+x+1)-x(x^2+x+1)$, tenemos entonces que
        \begin{align*}
            &(x+2)=(x+2)\cdot(x^3+x^2+x+1)-x(x+2)(x^2+x+1)\Longrightarrow \\
            &\Longrightarrow\left\{\begin{array}{l}
                f_0=x+2 \\
                g_0=-x(x+2)=-x^2-2x\\
            \end{array}\right. \text{ solución particular} \Longrightarrow\\
            &\Longrightarrow\left\{\begin{array}{l}
                f=(x+2)+k(x^2+x+1) \\
                g=-x^2-2x+k(x^3+x^2+x+1)\\
            \end{array}\right. k\in\bb{R}[x]
        \end{align*}
        Como se pide $grd(f)=2\Longrightarrow grd(k)=1\Longrightarrow k\in\bb{R}$ y se pide $f$ mónico entonces tenemos $k=1$. Obtenemos
        finalmente los dos polinomios requeridos
        \begin{align*}
            \left\{\begin{array}{l}
                f=x+2+x^2+x+1 \\
                g=-x^2-2x-x^3-x^2-1\\
            \end{array}\right. \Longrightarrow 
            \left\{\begin{array}{l}
                f=x^2+2x+3 \\
                g=-x^2-2x^2-3x-1\\
            \end{array}\right. 
        \end{align*}
    \end{ejercicio}
\end{document}