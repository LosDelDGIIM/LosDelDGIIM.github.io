\documentclass[12pt]{article}

% Idioma y codificación
\usepackage[spanish, es-tabla, es-notilde]{babel}       %es-tabla para que se titule "Tabla"
\usepackage[utf8]{inputenc}

% Márgenes
\usepackage[a4paper,top=3cm,bottom=2.5cm,left=3cm,right=3cm]{geometry}

% Comentarios de bloque
\usepackage{verbatim}

% Paquetes de links
\usepackage[hidelinks]{hyperref}    % Permite enlaces
\usepackage{url}                    % redirecciona a la web

% Más opciones para enumeraciones
\usepackage{enumitem}

% Personalizar la portada
\usepackage{titling}

% Paquetes de tablas
\usepackage{multirow}

% Para añadir el símbolo de euro
\usepackage{eurosym}


%------------------------------------------------------------------------

%Paquetes de figuras
\usepackage{caption}
\usepackage{subcaption} % Figuras al lado de otras
\usepackage{float}      % Poner figuras en el sitio indicado H.


% Paquetes de imágenes
\usepackage{graphicx}       % Paquete para añadir imágenes
\usepackage{transparent}    % Para manejar la opacidad de las figuras

% Paquete para usar colores
\usepackage[dvipsnames, table, xcdraw]{xcolor}
\usepackage{pagecolor}      % Para cambiar el color de la página

% Habilita tamaños de fuente mayores
\usepackage{fix-cm}

% Para los gráficos
\usepackage{tikz}
\usepackage{forest}

% Para poder situar los nodos en los grafos
\usetikzlibrary{positioning}


%------------------------------------------------------------------------

% Paquetes de matemáticas
\usepackage{mathtools, amsfonts, amssymb, mathrsfs}
\usepackage[makeroom]{cancel}     % Simplificar tachando
\usepackage{polynom}    % Divisiones y Ruffini
\usepackage{units} % Para poner fracciones diagonales con \nicefrac

\usepackage{pgfplots}   %Representar funciones
\pgfplotsset{compat=1.18}  % Versión 1.18

\usepackage{tikz-cd}    % Para usar diagramas de composiciones
\usetikzlibrary{calc}   % Para usar cálculo de coordenadas en tikz

%Definición de teoremas, etc.
\usepackage{amsthm}
%\swapnumbers   % Intercambia la posición del texto y de la numeración

\theoremstyle{plain}

\makeatletter
\@ifclassloaded{article}{
  \newtheorem{teo}{Teorema}[section]
}{
  \newtheorem{teo}{Teorema}[chapter]  % Se resetea en cada chapter
}
\makeatother

\newtheorem{coro}{Corolario}[teo]           % Se resetea en cada teorema
\newtheorem{prop}[teo]{Proposición}         % Usa el mismo contador que teorema
\newtheorem{lema}[teo]{Lema}                % Usa el mismo contador que teorema
\newtheorem*{lema*}{Lema}

\theoremstyle{remark}
\newtheorem*{observacion}{Observación}

\theoremstyle{definition}

\makeatletter
\@ifclassloaded{article}{
  \newtheorem{definicion}{Definición} [section]     % Se resetea en cada chapter
}{
  \newtheorem{definicion}{Definición} [chapter]     % Se resetea en cada chapter
}
\makeatother

\newtheorem*{notacion}{Notación}
\newtheorem*{ejemplo}{Ejemplo}
\newtheorem*{ejercicio*}{Ejercicio}             % No numerado
\newtheorem{ejercicio}{Ejercicio} [section]     % Se resetea en cada section


% Modificar el formato de la numeración del teorema "ejercicio"
\renewcommand{\theejercicio}{%
  \ifnum\value{section}=0 % Si no se ha iniciado ninguna sección
    \arabic{ejercicio}% Solo mostrar el número de ejercicio
  \else
    \thesection.\arabic{ejercicio}% Mostrar número de sección y número de ejercicio
  \fi
}


% \renewcommand\qedsymbol{$\blacksquare$}         % Cambiar símbolo QED
%------------------------------------------------------------------------

% Paquetes para encabezados
\usepackage{fancyhdr}
\pagestyle{fancy}
\fancyhf{}

\newcommand{\helv}{ % Modificación tamaño de letra
\fontfamily{}\fontsize{12}{12}\selectfont}
\setlength{\headheight}{15pt} % Amplía el tamaño del índice


%\usepackage{lastpage}   % Referenciar última pag   \pageref{LastPage}
%\fancyfoot[C]{%
%  \begin{minipage}{\textwidth}
%    \centering
%    ~\\
%    \thepage\\
%    \href{https://losdeldgiim.github.io/}{\texttt{\footnotesize losdeldgiim.github.io}}
%  \end{minipage}
%}
\fancyfoot[C]{\thepage}
\fancyfoot[R]{\href{https://losdeldgiim.github.io/}{\texttt{\footnotesize losdeldgiim.github.io}}}

%------------------------------------------------------------------------

% Conseguir que no ponga "Capítulo 1". Sino solo "1."
\makeatletter
\@ifclassloaded{book}{
  \renewcommand{\chaptermark}[1]{\markboth{\thechapter.\ #1}{}} % En el encabezado
    
  \renewcommand{\@makechapterhead}[1]{%
  \vspace*{50\p@}%
  {\parindent \z@ \raggedright \normalfont
    \ifnum \c@secnumdepth >\m@ne
      \huge\bfseries \thechapter.\hspace{1em}\ignorespaces
    \fi
    \interlinepenalty\@M
    \Huge \bfseries #1\par\nobreak
    \vskip 40\p@
  }}
}
\makeatother

%------------------------------------------------------------------------
% Paquetes de cógido
\usepackage{minted}
\renewcommand\listingscaption{Código fuente}

\usepackage{fancyvrb}
% Personaliza el tamaño de los números de línea
\renewcommand{\theFancyVerbLine}{\small\arabic{FancyVerbLine}}

% Estilo para C++
\newminted{cpp}{
    frame=lines,
    framesep=2mm,
    baselinestretch=1.2,
    linenos,
    escapeinside=||
}

% para minted
\definecolor{LightGray}{rgb}{0.95,0.95,0.92}
\setminted{
    linenos=true,
    stepnumber=5,
    numberfirstline=true,
    autogobble,
    breaklines=true,
    breakautoindent=true,
    breaksymbolleft=,
    breaksymbolright=,
    breaksymbolindentleft=0pt,
    breaksymbolindentright=0pt,
    breaksymbolsepleft=0pt,
    breaksymbolsepright=0pt,
    fontsize=\footnotesize,
    bgcolor=LightGray,
    numbersep=10pt
}


\usepackage{listings} % Para incluir código desde un archivo

\renewcommand\lstlistingname{Código Fuente}
\renewcommand\lstlistlistingname{Índice de Códigos Fuente}

% Definir colores
\definecolor{vscodepurple}{rgb}{0.5,0,0.5}
\definecolor{vscodeblue}{rgb}{0,0,0.8}
\definecolor{vscodegreen}{rgb}{0,0.5,0}
\definecolor{vscodegray}{rgb}{0.5,0.5,0.5}
\definecolor{vscodebackground}{rgb}{0.97,0.97,0.97}
\definecolor{vscodelightgray}{rgb}{0.9,0.9,0.9}

% Configuración para el estilo de C similar a VSCode
\lstdefinestyle{vscode_C}{
  backgroundcolor=\color{vscodebackground},
  commentstyle=\color{vscodegreen},
  keywordstyle=\color{vscodeblue},
  numberstyle=\tiny\color{vscodegray},
  stringstyle=\color{vscodepurple},
  basicstyle=\scriptsize\ttfamily,
  breakatwhitespace=false,
  breaklines=true,
  captionpos=b,
  keepspaces=true,
  numbers=left,
  numbersep=5pt,
  showspaces=false,
  showstringspaces=false,
  showtabs=false,
  tabsize=2,
  frame=tb,
  framerule=0pt,
  aboveskip=10pt,
  belowskip=10pt,
  xleftmargin=10pt,
  xrightmargin=10pt,
  framexleftmargin=10pt,
  framexrightmargin=10pt,
  framesep=0pt,
  rulecolor=\color{vscodelightgray},
  backgroundcolor=\color{vscodebackground},
}

%------------------------------------------------------------------------

% Comandos definidos
\newcommand{\bb}[1]{\mathbb{#1}}
\newcommand{\cc}[1]{\mathcal{#1}}

% I prefer the slanted \leq
\let\oldleq\leq % save them in case they're every wanted
\let\oldgeq\geq
\renewcommand{\leq}{\leqslant}
\renewcommand{\geq}{\geqslant}

% Si y solo si
\newcommand{\sii}{\iff}

% MCD y MCM
\DeclareMathOperator{\mcd}{mcd}
\DeclareMathOperator{\mcm}{mcm}

% Signo
\DeclareMathOperator{\sgn}{sgn}

% Letras griegas
\newcommand{\eps}{\epsilon}
\newcommand{\veps}{\varepsilon}
\newcommand{\lm}{\lambda}

\newcommand{\ol}{\overline}
\newcommand{\ul}{\underline}
\newcommand{\wt}{\widetilde}
\newcommand{\wh}{\widehat}

\let\oldvec\vec
\renewcommand{\vec}{\overrightarrow}

% Derivadas parciales
\newcommand{\del}[2]{\frac{\partial #1}{\partial #2}}
\newcommand{\Del}[3]{\frac{\partial^{#1} #2}{\partial #3^{#1}}}
\newcommand{\deld}[2]{\dfrac{\partial #1}{\partial #2}}
\newcommand{\Deld}[3]{\dfrac{\partial^{#1} #2}{\partial #3^{#1}}}


\newcommand{\AstIg}{\stackrel{(\ast)}{=}}
\newcommand{\Hop}{\stackrel{L'H\hat{o}pital}{=}}

\newcommand{\red}[1]{{\color{red}#1}} % Para integrales, destacar los cambios.

% Método de integración
\newcommand{\MetInt}[2]{
    \left[\begin{array}{c}
        #1 \\ #2
    \end{array}\right]
}

% Declarar aplicaciones
% 1. Nombre aplicación
% 2. Dominio
% 3. Codominio
% 4. Variable
% 5. Imagen de la variable
\newcommand{\Func}[5]{
    \begin{equation*}
        \begin{array}{rrll}
            \displaystyle #1:& \displaystyle  #2 & \longrightarrow & \displaystyle  #3\\
               & \displaystyle  #4 & \longmapsto & \displaystyle  #5
        \end{array}
    \end{equation*}
}

%------------------------------------------------------------------------

\usepackage{extarrows}
\usepackage{stackrel}
\usetikzlibrary{matrix} % Para divisiones de polinomios.

\newcommand{\resetearcontador}{%
  \setcounter{ejercicio}{0} % Resetea el contador de ejercicios a 0
}

\begin{document}

    % 1. Foto de fondo
    % 2. Título
    % 3. Encabezado Izquierdo
    % 4. Color de fondo
    % 5. Coord x del titulo
    % 6. Coord y del titulo
    % 7. Fecha

    
    % 1. Foto de fondo
% 2. Título
% 3. Encabezado Izquierdo
% 4. Color de fondo
% 5. Coord x del titulo
% 6. Coord y del titulo
% 7. Fecha
% 8. Autor

\newcommand{\portada}[8]{
    \portadaBase{#1}{#2}{#3}{#4}{#5}{#6}{#7}{#8}
    \portadaBook{#1}{#2}{#3}{#4}{#5}{#6}{#7}{#8}
}

\newcommand{\portadaFotoDif}[8]{
    \portadaBaseFotoDif{#1}{#2}{#3}{#4}{#5}{#6}{#7}{#8}
    \portadaBook{#1}{#2}{#3}{#4}{#5}{#6}{#7}{#8}
}

\newcommand{\portadaExamen}[8]{
    \portadaBase{#1}{#2}{#3}{#4}{#5}{#6}{#7}{#8}
    \portadaArticle{#1}{#2}{#3}{#4}{#5}{#6}{#7}{#8}
}

\newcommand{\portadaExamenFotoDif}[8]{
    \portadaBaseFotoDif{#1}{#2}{#3}{#4}{#5}{#6}{#7}{#8}
    \portadaArticle{#1}{#2}{#3}{#4}{#5}{#6}{#7}{#8}
}




\newcommand{\portadaBase}[8]{

    % Tiene la portada principal y la licencia Creative Commons
    
    % 1. Foto de fondo
    % 2. Título
    % 3. Encabezado Izquierdo
    % 4. Color de fondo
    % 5. Coord x del titulo
    % 6. Coord y del titulo
    % 7. Fecha
    % 8. Autor    
    
    \thispagestyle{empty}               % Sin encabezado ni pie de página
    \newgeometry{margin=0cm}        % Márgenes nulos para la primera página
    
    
    % Encabezado
    \fancyhead[L]{\helv #3}
    \fancyhead[R]{\helv \nouppercase{\leftmark}}
    
    
    \pagecolor{#4}        % Color de fondo para la portada
    
    \begin{figure}[p]
        \centering
        \transparent{0.3}           % Opacidad del 30% para la imagen
        
        \includegraphics[width=\paperwidth, keepaspectratio]{../../_assets/#1}
    
        \begin{tikzpicture}[remember picture, overlay]
            \node[anchor=north west, text=white, opacity=1, font=\fontsize{60}{90}\selectfont\bfseries\sffamily, align=left] at (#5, #6) {#2};
            
            \node[anchor=south east, text=white, opacity=1, font=\fontsize{12}{18}\selectfont\sffamily, align=right] at (9.7, 3) {\href{https://losdeldgiim.github.io/}{\textbf{Los Del DGIIM}, \texttt{\footnotesize losdeldgiim.github.io}}};
            
            \node[anchor=south east, text=white, opacity=1, font=\fontsize{12}{15}\selectfont\sffamily, align=right] at (9.7, 1.8) {Doble Grado en Ingeniería Informática y Matemáticas\\Universidad de Granada};
        \end{tikzpicture}
    \end{figure}
    
    
    \restoregeometry        % Restaurar márgenes normales para las páginas subsiguientes
    \nopagecolor      % Restaurar el color de página
    
    
    \newpage
    \thispagestyle{empty}               % Sin encabezado ni pie de página
    \begin{tikzpicture}[remember picture, overlay]
        \node[anchor=south west, inner sep=3cm] at (current page.south west) {
            \begin{minipage}{0.5\paperwidth}
                \href{https://creativecommons.org/licenses/by-nc-nd/4.0/}{
                    \includegraphics[height=2cm]{../../_assets/Licencia.png}
                }\vspace{1cm}\\
                Esta obra está bajo una
                \href{https://creativecommons.org/licenses/by-nc-nd/4.0/}{
                    Licencia Creative Commons Atribución-NoComercial-SinDerivadas 4.0 Internacional (CC BY-NC-ND 4.0).
                }\\
    
                Eres libre de compartir y redistribuir el contenido de esta obra en cualquier medio o formato, siempre y cuando des el crédito adecuado a los autores originales y no persigas fines comerciales. 
            \end{minipage}
        };
    \end{tikzpicture}
    
    
    
    % 1. Foto de fondo
    % 2. Título
    % 3. Encabezado Izquierdo
    % 4. Color de fondo
    % 5. Coord x del titulo
    % 6. Coord y del titulo
    % 7. Fecha
    % 8. Autor


}


\newcommand{\portadaBaseFotoDif}[8]{

    % Tiene la portada principal y la licencia Creative Commons
    
    % 1. Foto de fondo
    % 2. Título
    % 3. Encabezado Izquierdo
    % 4. Color de fondo
    % 5. Coord x del titulo
    % 6. Coord y del titulo
    % 7. Fecha
    % 8. Autor    
    
    \thispagestyle{empty}               % Sin encabezado ni pie de página
    \newgeometry{margin=0cm}        % Márgenes nulos para la primera página
    
    
    % Encabezado
    \fancyhead[L]{\helv #3}
    \fancyhead[R]{\helv \nouppercase{\leftmark}}
    
    
    \pagecolor{#4}        % Color de fondo para la portada
    
    \begin{figure}[p]
        \centering
        \transparent{0.3}           % Opacidad del 30% para la imagen
        
        \includegraphics[width=\paperwidth, keepaspectratio]{#1}
    
        \begin{tikzpicture}[remember picture, overlay]
            \node[anchor=north west, text=white, opacity=1, font=\fontsize{60}{90}\selectfont\bfseries\sffamily, align=left] at (#5, #6) {#2};
            
            \node[anchor=south east, text=white, opacity=1, font=\fontsize{12}{18}\selectfont\sffamily, align=right] at (9.7, 3) {\href{https://losdeldgiim.github.io/}{\textbf{Los Del DGIIM}, \texttt{\footnotesize losdeldgiim.github.io}}};
            
            \node[anchor=south east, text=white, opacity=1, font=\fontsize{12}{15}\selectfont\sffamily, align=right] at (9.7, 1.8) {Doble Grado en Ingeniería Informática y Matemáticas\\Universidad de Granada};
        \end{tikzpicture}
    \end{figure}
    
    
    \restoregeometry        % Restaurar márgenes normales para las páginas subsiguientes
    \nopagecolor      % Restaurar el color de página
    
    
    \newpage
    \thispagestyle{empty}               % Sin encabezado ni pie de página
    \begin{tikzpicture}[remember picture, overlay]
        \node[anchor=south west, inner sep=3cm] at (current page.south west) {
            \begin{minipage}{0.5\paperwidth}
                %\href{https://creativecommons.org/licenses/by-nc-nd/4.0/}{
                %    \includegraphics[height=2cm]{../../_assets/Licencia.png}
                %}\vspace{1cm}\\
                Esta obra está bajo una
                \href{https://creativecommons.org/licenses/by-nc-nd/4.0/}{
                    Licencia Creative Commons Atribución-NoComercial-SinDerivadas 4.0 Internacional (CC BY-NC-ND 4.0).
                }\\
    
                Eres libre de compartir y redistribuir el contenido de esta obra en cualquier medio o formato, siempre y cuando des el crédito adecuado a los autores originales y no persigas fines comerciales. 
            \end{minipage}
        };
    \end{tikzpicture}
    
    
    
    % 1. Foto de fondo
    % 2. Título
    % 3. Encabezado Izquierdo
    % 4. Color de fondo
    % 5. Coord x del titulo
    % 6. Coord y del titulo
    % 7. Fecha
    % 8. Autor


}


\newcommand{\portadaBook}[8]{

    % 1. Foto de fondo
    % 2. Título
    % 3. Encabezado Izquierdo
    % 4. Color de fondo
    % 5. Coord x del titulo
    % 6. Coord y del titulo
    % 7. Fecha
    % 8. Autor

    % Personaliza el formato del título
    \pretitle{\begin{center}\bfseries\fontsize{42}{56}\selectfont}
    \posttitle{\par\end{center}\vspace{2em}}
    
    % Personaliza el formato del autor
    \preauthor{\begin{center}\Large}
    \postauthor{\par\end{center}\vfill}
    
    % Personaliza el formato de la fecha
    \predate{\begin{center}\huge}
    \postdate{\par\end{center}\vspace{2em}}
    
    \title{#2}
    \author{\href{https://losdeldgiim.github.io/}{Los Del DGIIM, \texttt{\large losdeldgiim.github.io}}
    \\ \vspace{0.5cm}#8}
    \date{Granada, #7}
    \maketitle
    
    \tableofcontents
}




\newcommand{\portadaArticle}[8]{

    % 1. Foto de fondo
    % 2. Título
    % 3. Encabezado Izquierdo
    % 4. Color de fondo
    % 5. Coord x del titulo
    % 6. Coord y del titulo
    % 7. Fecha
    % 8. Autor

    % Personaliza el formato del título
    \pretitle{\begin{center}\bfseries\fontsize{42}{56}\selectfont}
    \posttitle{\par\end{center}\vspace{2em}}
    
    % Personaliza el formato del autor
    \preauthor{\begin{center}\Large}
    \postauthor{\par\end{center}\vspace{3em}}
    
    % Personaliza el formato de la fecha
    \predate{\begin{center}\huge}
    \postdate{\par\end{center}\vspace{5em}}
    
    \title{#2}
    \author{\href{https://losdeldgiim.github.io/}{Los Del DGIIM, \texttt{\large losdeldgiim.github.io}}
    \\ \vspace{0.5cm}#8}
    \date{Granada, #7}
    \thispagestyle{empty}               % Sin encabezado ni pie de página
    \maketitle
    \vfill
}
    \portadaExamen{ffccA4.jpg}{Álgebra I\\Examen III}{Álgebra I. Examen III}{MidnightBlue}{-8}{28}{2023-2024}{José Juan Urrutia Milán\\Arturo Olivares Martos}

    
    \begin{description}
        \item[Asignatura] Álgebra I.
        \item[Curso Académico] 2021-22.
        \item[Grado] Doble Grado en Ingeniería Informática y Matemáticas.
        \item[Grupo] Único.
        \item[Profesor] María Pilar Carrasco Carrasco.
        \item[Descripción] Examen Extraordinario.
        \item[Fecha] 7 de febrero de 2022.
        \item[Duración] 3 horas.
    
    \end{description}
    \newpage
    
    \begin{ejercicio}[4 puntos]
        \ 
        \begin{enumerate} 
            \item (0.5 puntos) Sea $X$ un conjunto y $A$, $B$ y $C$ subconjuntos de $X$. Suponiendo que $A \cap C = \emptyset$, probad que $(A \cup B)-C=A\cup (B-C)$.
            \item (1 punto) Sea $f:\bb{R}\rightarrow \bb{R}$ la aplicación definida para cada $x \in \bb{R}$, por $f(x)~=~x^2-3x+1$. Sea $R_f$ la relación de equivalencia en $\bb{R}$ definida por la aplicación $f$. Describid el conjunto cociente. ¿Cuántos elementos tiene cada clase?
            \item (0.5 puntos) En el anillo $\bb{Z}_8$, sea $I = 4\bb{Z}_8$, el ideal principal generado por 4. Describid $\bb{Z}_8/I$ listando todos sus elementos.
            \item (1 punto) Sea $A$ un dominio euclídeo con función euclídea $\phi:A \setminus \{0\}\rightarrow \bb{N}$. Demostrad que para todo $a \neq 0$, se tiene que $\phi(a) \geq \phi(1)$ y que se da la igualdad si, y sólo si $a \in U(A)$.
            \item (0.5 puntos) Demostrad que $n^{13}-n$ es divisible por 2 y 5 para todo $n \in \bb{Z}$.
            \item (0.5 puntos) Demostrad que en $\bb{Z}[i]$ se tiene que mcd$(n,n+i) = 1$, para todo $n \in \bb{Z}$, $n \neq 0$.
        \end{enumerate} 
    \end{ejercicio}
    
    \begin{ejercicio}[3.5 puntos]
        \ 
        \begin{enumerate}
            \item (2.5 puntos) Factorizar en $\bb{Z}[x]$ y en $\bb{Q}[x]$ los siguientes polinomios:
            \begin{enumerate}
                \item[a.] $48x^4 + 24x^3 -72x +80$
                \item[b.] $x^6 + 8x^4 + 4x^3 + 5x^2 + 4x+1$
                \item[c.] $10x^5 + 23x^4 -20x^3 + 5x^2 + 5x-3$
            \end{enumerate}
            \item Sea $f(x) \in \bb{Z}[x]$ un polinomio primitivo de grado $n>0$ y tal que existe un primo $p \in \bb{Z}$ verificando:
                \begin{enumerate}
                    \item[(i)] Su reducido módulo $p$ es de la forma $R_p(f(x)) = \alpha x^n$, con $\alpha \in \bb{Z}$ no nulo.
                    \item[(ii)] $p^2 \nmid f(p)$. 
                \end{enumerate}
                Demostrad que $f(x)$ es irreducible.
        \end{enumerate}
    \end{ejercicio}

    \begin{ejercicio}[2.5 puntos]
        Dado el sistema de congruencias en $\bb{Z}[\sqrt{-2}]$
        $$\left\{ \begin{array}{l}
            x \equiv 1 \mod (1+\sqrt{-2}) \\
            x \equiv 2 \mod (3+\sqrt{-2}) \\
            x \equiv 4 \mod (3+2\sqrt{-2}) 
        \end{array}\right.$$
        Demostrad que tiene solución sin resolverlo. Resolverlo, dando una solución óptima y la solución general. \newline
        ¿Es posible encontrar una solución a $a + b\sqrt{-2}$ tal que $30 < a < 56$?
    \end{ejercicio}

    \newpage
    \ % --------------------------------------------------------------------------------
    \newpage
    \resetearcontador
    
    \begin{ejercicio}[4 puntos]
        \ 
        \begin{enumerate} 
            \item (0.5 puntos) Sea $X$ un conjunto y $A$, $B$ y $C$ subconjuntos de $X$. Suponiendo que $A \cap C = \emptyset$, probad que $(A \cup B)-C=A\cup (B-C)$.
            \item (1 punto) Sea $f:\bb{R}\rightarrow \bb{R}$ la aplicación definida para cada $x \in \bb{R}$, por $f(x)~=~x^2-3x+1$. Sea $R_f$ la relación de equivalencia en $\bb{R}$ definida por la aplicación $f$. Describid el conjunto cociente. ¿Cuántos elementos tiene cada clase?
            \item (0.5 puntos) En el anillo $\bb{Z}_8$, sea $I = 4\bb{Z}_8$, el ideal principal generado por 4. Describid $\bb{Z}_8/I$ listando todos sus elementos.\\

                \noindent
                $$\bb{Z}_8/I = \{[a] \mid a \in \bb{Z}_8\} = \{a+I \mid a \in \bb{Z}_8\}$$
                $$[0] = \{ b \in \bb{Z}_8 \mid b \equiv 0 \mod (4) \} = \{ b \in \bb{Z}_8 \mid b = 0 \mbox{ en } \bb{Z}_4 \} = \{ 0,4 \}$$
                $$[1] = \{ b \in \bb{Z}_8 \mid b \equiv 1 \mod (4) \} = \{ b \in \bb{Z}_8 \mid b = 1 \mbox{ en } \bb{Z}_4 \} = \{ 1,5 \}$$
                $$[2] = \{ b \in \bb{Z}_8 \mid b \equiv 2 \mod (4) \} = \{ b \in \bb{Z}_8 \mid b = 2 \mbox{ en } \bb{Z}_4 \} = \{ 2,6 \}$$
                $$[3] = \{ b \in \bb{Z}_8 \mid b \equiv 3 \mod (4) \} = \{ b \in \bb{Z}_8 \mid b = 3 \mbox{ en } \bb{Z}_4 \} = \{ 3,7 \}$$
                Luego:
                $$\bb{Z}_8/I = \{ 0+I, 1+I, 2+I, 3+I \}$$

            \item (1 punto) Sea $A$ un dominio euclídeo con función euclídea $\phi:A \setminus \{0\}\rightarrow \bb{N}$. Demostrad que para todo $a \neq 0$, se tiene que $\phi(a) \geq \phi(1)$ y que se da la igualdad si, y sólo si $a \in U(A)$.\\

                $$\phi(a) = \phi(a \cdot 1) \geq \phi(1)~~\forall a \in A \setminus \{0\}$$

                \noindent
                $\Rightarrow)$ Supongamos que $\phi(a) = \phi(1)$:\newline
                Dividimos 1 entre $a$, $1 = qa + r$ con:
                $$\left\{ \begin{array}{l}
                    r = 0 \\
                    \lor \\
                    \phi(r) < \phi(a) = \phi(1)
                \end{array}\right.$$
                $\bullet$ Supongamos que $r \neq 0$, luego $\phi(r) < \phi(1)$, pero:
                $$\phi(b) \geq \phi(1)~~\forall b \in A \setminus\{0\} \Rightarrow \phi(r) \geq \phi(1)$$
                \underline{Contradicción}. Luego $r = 0$.\newline
                Por tanto: $1 = qa \Rightarrow a \in U(A)$.\\

                \noindent
                $\Leftarrow)$ Supongamos que $a \in U(A) \Rightarrow \exists a^{-1} \in A \mid aa^{-1} = 1 \Rightarrow \phi(aa^{-1}) = \phi(1)$
                $$\left. \begin{array}{r}
                    \phi(1) = \phi(aa^{-1}) \geq \phi(a) \\
                    \phi(a) \geq \phi(1)~~\forall a \in A \setminus\{0\}
                \end{array}\right\}\Rightarrow \phi(a) = \phi(1)$$

            \item (0.5 puntos) Demostrad que $n^{13}-n$ es divisible por 2 y 5 para todo $n \in \bb{Z}$.\\

                \noindent
                Demostremos primero que $2 \mid n^{13}-n~~\forall n \in \bb{Z}$:\newline
                $\bullet$ Supongamos que $2 \mid n \Rightarrow \exists k \in \bb{Z}$ tal que $n = 2k$.
                $$n^{13}-n = n(n^{12}-1) = 2k(n^{12}-1)=2k'$$
                Con $k' = k(n^{12}-1) \Rightarrow 2 \mid n^{13}-n$.

                \noindent
                $\bullet$ Supongamos que $2 \nmid n$:\newline
                Como 2 es primo, mcd$(2,n) = 1$. Por el Teorema de Fermat:
                $$n^{\varphi(2)} = n \equiv 1 \mod (2) \Rightarrow n^{12} \equiv 1 \mod (2) \Rightarrow n^{13} \equiv n \mod (2) \Rightarrow$$
                $$\Rightarrow n^{13}-n \equiv 0 \mod (2) \Rightarrow 2 \mid n^{13}-n$$\\

                \noindent
                Veamos que $5 \mid n^{13}-n~~\forall n \in \bb{Z}$:\newline
                $\bullet$ Supongamos que $5 \mid n \Rightarrow \exists k \in \bb{Z}$ tal que $n = 5k$.
                $$n^{13}-n = n(n^{12}-1) = 5k(n^{12}-1)=5k'$$
                Con $k' = k(n^{12}-1) \Rightarrow 5 \mid n^{13}-n$.

                \noindent
                $\bullet$ Supongamos que $5 \nmid n$:\newline
                Como 5 es primo, mcd$(5,n) = 1$. Por el Teorema de Fermat:
                $$n^{\varphi(5)} = n^4 \equiv 1 \mod (5) \Rightarrow n^{12} \equiv 1 \mod (5) \Rightarrow n^{13} \equiv n \mod (5) \Rightarrow$$
                $$\Rightarrow n^{13}-n \equiv 0 \mod (5) \Rightarrow 5 \mid n^{13}-n$$


            \item (0.5 puntos) Demostrad que en $\bb{Z}[i]$ se tiene que mcd$(n,n+i) = 1$, para todo $n \in \bb{Z}$, $n \neq 0$.\\

                \noindent
                Supongamos que $\exists n \in \bb{Z}^{+} \mid$ mcd$(n, n+i) \neq 1$.\newline
                Como el máximo común divisor es único salvo asociados y:\newline $1\sim a$ con $a \in U(\bb{Z}[i])$:
                $$\mbox{mcd}(n, n+i) = \alpha \mid \alpha \notin U(A)$$
                $$\left\{ \begin{array}{lclcl}
                    \alpha \mid n & \Rightarrow & N(\alpha) \mid N(n) & \Rightarrow & N(\alpha) \mid n^2 \\
                    \land & & & & \\
                    \alpha \mid n+i & \Rightarrow & N(\alpha) \mid N(n+i) & \Rightarrow & N(\alpha) \mid n^2 +1
                \end{array}\right.$$

                Veamos ahora el siguiente resultado:\newline
                Sean $a$, $b \in \bb{Z}^{+}$: $a \mid b \land a\mid b+1 \Leftrightarrow a = \pm 1$

                \noindent
                $\Leftarrow)$ Supongamos que $a = \pm 1 \Rightarrow a \in U(A) \Rightarrow a \mid b \land a \mid b+1$\newline
                $\Rightarrow)$ Supongamos que $a \mid b \land a \mid b + 1$:
                $$\left. \begin{array}{lcr}
                    a \mid b & \Rightarrow & \exists k \in \bb{Z} \mid b = ka \\
                    a \mid b +1 & \Rightarrow & \exists k' \in \bb{Z} \mid b+1 = k'a
                \end{array}\right\} \Rightarrow ka+1 = k'a \Rightarrow$$
                $$\Rightarrow 1 = a(k'-k) \Rightarrow a \mid 1 \Rightarrow a \in U(\bb{Z}) \Rightarrow a = \pm 1$$

                $$\left. \begin{array}{l}
                    N(\alpha) \mid n^2 \\
                    \land \\
                    N(\alpha) \mid n^2 + 1
                \end{array}\right\} \Rightarrow N(\alpha) = \pm 1 \Rightarrow \alpha \in U(A)$$
                Lo que es una \underline{contradicción}.

                \noindent
                Luego $mcd(n, n+i) = 1~~\forall n \in \bb{Z}^{+}$.
        \end{enumerate} 
    \end{ejercicio}
    
    \begin{ejercicio}[3.5 puntos]
        \ 
        \begin{enumerate}
            \item (2.5 puntos) Factorizar en $\bb{Z}[x]$ y en $\bb{Q}[x]$ los siguientes polinomios:
            \begin{enumerate}
                \item[a.] $48x^4 + 24x^3 -72x +80$\\

                    \noindent
                    Sea $f =48x^4 + 24x^3 -72x +80 $, $f=8g$ con $g = 6x^4+3x^3-9x+10 \in \bb{Z}[x]$.\newline
                    $g$ es irreducible en $\bb{Z}[x]$ por Eisenstein para $p = 3$. Por la misma razón, también lo es en $\bb{Q}[x]$.
                    $$f = 2^3 (6x^4+3x^3-9x+10 ) \mbox{ en } \bb{Z}[x]$$
                    $$f = 8 (6x^4+3x^3-9x+10 ) \mbox{ en } \bb{Q}[x]$$

                \item[b.] $x^6 + 8x^4 + 4x^3 + 5x^2 + 4x+1$\\

                    \noindent
                    Sea $f =x^6 + 8x^4 + 4x^3 + 5x^2 + 4x+1 $.\newline
                    Las posibles raíces de $f$ en $\bb{Q}$ son $\{\pm 1\}$:
                    $$\left.\begin{array}{l}
                            $f$(1) = 1 + 8 + 4 + 5 + 4 + 1 = 23 \neq 0 \\
                            $f$(-1) = 1 + 8 -4 + 5 -4 + 1 = 7 \neq 0
                        \end{array}\right\} \Rightarrow \begin{array}{l}
                        $f$ \mbox{ no tiene factores } \\
                        \mbox{ de grado 1 ni 5 }
                    \end{array}$$
                    $\bullet$ Reducimos módulo 2:
                    $$R_2(f) = x^6 +x^2 + 1 = (x^3 +x +1 )^3 \Rightarrow R_2(f) \mbox{ no tiene factores de grado } 2 \mbox{ ni } 4$$
                    Luego $f$ tampoco tiene factores de grado 2 ni 4.\\

                    \noindent
                    $\bullet$ Reducimos módulo 3:
                    $$R_3(f) = x^6 + 2x^4 + x^3 + 2x^2 + x + 1 \in \bb{Z}_3[x]$$
                    $$\left. \begin{array}{l}
                        R_3(f)(0) = 1\neq 0 \\
                        R_3(f)(1) = 8 \neq 0 \\
                        R_3(f)(2) = 115 = 1 \neq 0 \\
                    \end{array}\right\} \Rightarrow \begin{array}{l}
                        R_3(f) \text{ no tiene factores } \\
                        \mbox{ de grado 1 ni 5 }
                    \end{array}$$
                    
                    Al dividir $R_3(f)$ entre $x^2+1$ obtenemos que:
                    $$x^6 + 2x^4 + x^3 + 2x^2 + x + 1 = (x^2+1)(x^4+x^2+x+1) \Rightarrow (x^2+1) \mid R_3(f)$$
                    $$R_3(f) = (x^2+1)g \mbox{ con } g=x^4 +x^2+x+1 \in \bb{Z}_3[x]$$
                    Como $R_3(f)$ no tiene factores de grado 1 $\Rightarrow g$ no tiene de grado 1 ni 3.\newline
                    $R_3(f)$ no tiene factores de grado 3 $\Rightarrow f$ tampoco.\\

                    \noindent
                    Concluimos que $f$ es irreducible en $\bb{Z}[x]$ y en $\bb{Q}[x]$.


                \item[c.] $10x^5 + 23x^4 -20x^3 + 5x^2 + 5x-3$\\

                    \noindent
                    Sea $f = 10x^5 + 23x^4 -20x^3 + 5x^2 + 5x-3$. Es primitivo.
                    $$Div(-3) = \{ \pm 1, \pm 3 \}$$
                    $$Div(10) = \{ \pm 1, \pm 2, \pm 5, \pm 10 \}$$
                    Por lo que las posibles raíces de $f$ en $\bb{Q}$ son:
                $$\left\{ \pm 1, \pm 3, \pm \dfrac{1}{2}, \pm \dfrac{3}{2}, \pm \dfrac{1}{5}, \pm \dfrac{3}{5}, \pm \dfrac{1}{10}, \pm \dfrac{3}{10} \right\}$$
                    $$f(1) \neq 0~~~~f(-1)\neq 0~~~~f(3) \neq 0$$
                    $$f(-3) = 0 \Rightarrow (x+3)\mid f$$
                    Al dividir $f$ entre $(x+3)$ obtenemos:
                    $$f = (x+3)g \mbox{ con } g=10x^4-7x^3+x^2+2x-1 \in \bb{Z}[x]$$
                    Con $g$ primitivo. Las posibles raíces de $g$ en $\bb{Q}$ son:
                    $$\left\{ \pm 1, \pm \dfrac{1}{2}, \pm \dfrac{1}{5}, \pm \dfrac{1}{10} \right\}$$
                    $$g(1) \neq 0~~~~g(-1)\neq 0$$
                    $$g\left(\dfrac{1}{2}\right) = 0 \Rightarrow \left(x-\dfrac{1}{2}\right) \mid g \Rightarrow (2x-1)\mid g$$
                    Al dividir $g$ entre $(2x-1)$ obtenemos:
                    $$g = (2x-1)(5x^3-x^2+1)$$
                    Luego:
                    $$f = (x+3)(2x-1)h \mbox{ con } h = 5x^3-x^2+1 \in \bb{Z}[x]$$
                    Con $h$ primitivo.

                    \noindent
                    Las posibles ríaces de $h$ en $\bb{Q}$ son: $\left\{ \pm 1, \pm \dfrac{1}{5} \right\}$
                    $$h(1) \neq 0~~~~h(-1)\neq 0~~~~h\left(\dfrac{1}{5}\right) \neq 0 ~~~~h\left(-\dfrac{1}{5}\right) \neq 0$$
                    Luego $h$ es irreducible por el criterio de la raíz.\\
                    $$f = (x+3)(2x-1)(5x^3-x^2+1) \mbox{ en } \bb{Z}[x]$$
                    $$f = \dfrac{5}{2}(x+3)(x-\dfrac{1}{2})(x^3-\dfrac{1}{2}x^2+\dfrac{1}{5}) \mbox{ en } \bb{Q}[x]$$

                    


            \end{enumerate}
            \item Sea $f(x) \in \bb{Z}[x]$ un polinomio primitivo de grado $n>0$ y tal que existe un primo $p \in \bb{Z}$ verificando:
                \begin{enumerate}
                    \item[(i)] Su reducido módulo $p$ es de la forma $R_p(f(x)) = \alpha x^n$, con $\alpha \in \bb{Z}$ no nulo.
                    \item[(ii)] $p^2 \nmid f(p)$. 
                \end{enumerate}
                Demostrad que $f(x)$ es irreducible.\\

                \noindent
                Sea $f = \displaystyle\sum_{i=0}^n a_i x^i \in \bb{Z}[x]$:

                \noindent
                Como $R_p(f) = \alpha x^n \Rightarrow p \nmid a_n \land p \mid a_i~~\forall i \in \{0, \ldots, n-1\}$\newline
                Para poder aplicar Eisenstein, es necesario demostrar que $p^2 \nmid a_0$:\\

                \noindent
                $$f(p) = \displaystyle \sum_{i=0}^n a_i p^i = \sum_{i=1}^n (a_i p^i) + a_0 = \sum_{i=2}^n (a_i p^i) + a_1 p + a_0 = p^2 \sum_{i=2}^n (a_i p^{i-2}) + a_1 p + a_0$$
                Como $p\mid a_1\Rightarrow \exists k \in \bb{Z} \mid a_1 = pk$
                $$f(p) = p^2 \displaystyle\sum_{i=2}^n (a_i p^{i-2}) + (pk)p+a_0 = p^2 \left[\sum_{i=2}^n (a_i p^{i-2}) + k \right] + a_0$$

                \noindent
                $\bullet$ Supongamos que $p^2 \mid a_0 \Rightarrow \exists k' \in \bb{Z} \mid a_0 = p^2 k'$ 
            $$f(p) = p^2 \left[\sum_{i=2}^n(a_ip^{i-2})+k\right] + p^2 k' = p^2 \left[\sum_{i=2}^n (a_i p^{i-2})+k+k'\right] \Rightarrow p^2 \mid f(p)$$
            Lo que es una \underline{contradicción}. Luego $p^2 \nmid a_0$.\\

            \noindent
            Como $f$ es primitivo y $\exists p \in \bb{Z}$ primo tal que:
            $$p^2 \nmid a_0 \land p \mid a_i~~\forall i \in \{0, \ldots, n-1\}$$
            Entonces, $f$ es irreducible por Eisenstein para $p$
                
        \end{enumerate}
    \end{ejercicio}

    \begin{ejercicio}[2.5 puntos]
        Dado el sistema de congruencias en $\bb{Z}[\sqrt{-2}]$
        $$\left\{ \begin{array}{l}
            x \equiv 1 \mod (1+\sqrt{-2}) \\
            x \equiv 2 \mod (3+\sqrt{-2}) \\
            x \equiv 4 \mod (3+2\sqrt{-2}) 
        \end{array}\right.$$
        Demostrad que tiene solución sin resolverlo. Resolverlo, dando una solución óptima y la solución general. \newline
        ¿Es posible encontrar una solución a $a + b\sqrt{-2}$ tal que $30 < a < 56$?\\

        \noindent
        Primero, calculamos:\newline mcd$(1+\sqrt{-2}, 3+\sqrt{-2})$, mcd$(1+\sqrt{-2}, 3+2\sqrt{-2})$ y mcd$(3+\sqrt{-2}, 3+2\sqrt{-2})$:\\

        \noindent
        $\bullet$ En $\bb{Q}[\sqrt{-2}]$:
        $$\dfrac{3+\sqrt{-2}}{1+\sqrt{-2}} = \dfrac{(3+\sqrt{-2})(1-\sqrt{-2})}{3} = \dfrac{3-3\sqrt{-2}+\sqrt{-2}+2}{3} = \dfrac{5}{3} - \dfrac{2}{3}\sqrt{-2} \Rightarrow$$
        $$\Rightarrow 3+\sqrt{-2}=(2-\sqrt{-2})(1+\sqrt{-2})-1$$
        \begin{equation*}
        \begin{array}{rcl}
            r_i & u_i & v_i \\
            3+\sqrt{-2} & 1 & 0 \\
            1+\sqrt{-2} & 0 & 1 \\
            -1 & 1 & -(2-\sqrt{-2})
        \end{array}
        \end{equation*}
        Luego mcd$(1+\sqrt{-2}, 3+\sqrt{-2}) = -1 \sim 1$\\

        \noindent
        $\bullet$ En $\bb{Q}[\sqrt{-2}]$:
        $$\dfrac{3+2\sqrt{-2}}{1+\sqrt{-2}} = \dfrac{(3+2\sqrt{-2})(1-\sqrt{-2})}{3} = \dfrac{3+2\sqrt{-2}+3\sqrt{-2}+4}{3} = \dfrac{7}{3} + \dfrac{1}{3}\sqrt{-2} \Rightarrow$$
        $$\Rightarrow 3+2\sqrt{-2} = (1+\sqrt{-2})(2)+1$$
        \begin{equation*}
        \begin{array}{rcr}
            r_i & u_i & v_i \\ \\
            3+2\sqrt{-2} & 1 & 0 \\
            1+\sqrt{-2} & 0 & 1 \\
            1 & 1 & -2
        \end{array}
        \end{equation*}
        Luego mcd$(1+\sqrt{-2}, 3+2\sqrt{-2}) = 1$\\

        \noindent
        $\bullet$ En $\bb{Q}[\sqrt{-2}]$:
        $$\dfrac{3+2\sqrt{-2}}{3+\sqrt{-2}} = \dfrac{(3+2\sqrt{-2})(3-\sqrt{-2})}{9+2} = \dfrac{9+6\sqrt{-2}-3\sqrt{-2}+4}{11} = \dfrac{13}{11}+\dfrac{3}{11}\sqrt{-2} \Rightarrow$$
        $$\Rightarrow (3+2\sqrt{-2})=(3+\sqrt{-2})(1)+\sqrt{-2}$$
        En $\bb{Q}[\sqrt{-2}]$:
        $$\dfrac{3+\sqrt{-2}}{\sqrt{-2}} = \dfrac{(3+\sqrt{-2})(\sqrt{-2})}{-2} = \dfrac{3\sqrt{-2}-2}{-2} = 1-\dfrac{3}{2}\sqrt{-2} \Rightarrow$$
        $$\Rightarrow (3+\sqrt{-2})=(\sqrt{-2})(1-\sqrt{-2})+1$$
        \begin{equation*}
        \begin{array}{rcl}
            r_i & u_i & v_i \\ \\
            3+2\sqrt{-2} & 1 & 0 \\
            3+\sqrt{-2} & 0 & 1 \\
            \sqrt{-2} & 1 & -1 \\
            1 & -(1-\sqrt{-2}) & 2-\sqrt{-2}
        \end{array}
        \end{equation*}
        Luego $\mcd(3+\sqrt{-2}, 3+2\sqrt{-2}) = 1$\\

        \noindent
        Como:
        \begin{equation*}
            \left.\begin{array}{l}
                \mcd(1+\sqrt{-2}, 3+\sqrt{-2}) = 1 \\
                \mcd(1+\sqrt{-2}, 3+2\sqrt{-2}) = 1\\
                \mcd(3+\sqrt{-2}, 3+2\sqrt{-2}) = 1
            \end{array}\right\}
        \end{equation*}

        \ \newline
        \noindent
        Por el Teorema Chino del resto generalizado, sabemos que el sistema tiene solución.\\
        
        \noindent
        Pasamos a resolver el sistema.\newline
        Resolvemos en primer lugar:
        $$\left\{\begin{array}{l}
            x \equiv 1 \mod (1+\sqrt{-2}) \\
            x \equiv 2 \mod (3+\sqrt{-2})
        \end{array}\right.$$
        De la primera ecución, tenemos que:
        $$x = 1 + (1+\sqrt{-2})\alpha \mid \alpha \in \bb{Z}[\sqrt{-2}]$$
        De la segunda:
        $$x = 1 + (1+\sqrt{-2})\alpha \equiv 2\mod (3+\sqrt{-2})\Leftrightarrow$$
        $$\Leftrightarrow(1+\sqrt{-2})\alpha \equiv 1 \mod (3+\sqrt{-2})$$
        Sabemos que $-1 = 3+\sqrt{-2}-(2-\sqrt{-2})(1+\sqrt{-2})$. Luego:
        $$(1+\sqrt{-2})(2-\sqrt{-2})\equiv 1 \mod (3+\sqrt{-2})$$
        Luego $\alpha_0 = 2-\sqrt{-2}$ es una solución particular.\newline
        De hecho, se trata de la solución óptima, luego las soluciones son:
        $$\alpha = (2-\sqrt{-2})+(3+\sqrt{-2})\beta \mid \beta \in \bb{Z}[\sqrt{-2}]$$
        Por lo que las soluciones del sistema son ($\forall \beta \in \bb{Z}[\sqrt{-2}]$):
        $$x = 1+(1+\sqrt{-2})\alpha = 1+(1+\sqrt{-2})[(2-\sqrt{-2})+(3+\sqrt{-2})\beta] = $$
        $$=1+(1+\sqrt{-2})(2-\sqrt{-2})+(1+\sqrt{-2})(3+\sqrt{-2})\beta = $$
        $$=1 + 2-\sqrt{-2} +2\sqrt{-2} +2+ (1+\sqrt{-2})(3+\sqrt{-2})\beta = $$
        $$=5+\sqrt{-2} + (1+\sqrt{-2})(3+\sqrt{-2})\beta$$
        Luego:
        $$x \equiv 5+\sqrt{-2} \mod [(1+\sqrt{-2})(3+\sqrt{-2})]$$
        Por lo que el sistema inicial es equivalente a:
        $$\left\{ \begin{array}{l}
            x \equiv 5+\sqrt{-2} \mod [(1+\sqrt{-2})(3+\sqrt{-2})] \\
            x \equiv 4 \mod (3+2\sqrt{-2}) 
        \end{array}\right.$$
        De la segunda ecución, obtenemos que:
        $$x = 4+(3+2\sqrt{-2})\phi \mid \phi \in \bb{Z}[\sqrt{-2}]$$
        Y de la segunda:
        $$x=4+(3+2\sqrt{-2})\phi \equiv 5+\sqrt{-2} \mod [(1+\sqrt{-2})(3+\sqrt{-2})]\Leftrightarrow$$
        $$\Leftrightarrow (3+2\sqrt{-2})\phi \equiv 1+\sqrt{-2} \mod [(1+\sqrt{-2})(3+\sqrt{-2})] \Leftrightarrow$$
        $$\Leftrightarrow (3+2\sqrt{-2})\phi \equiv 1+\sqrt{-2} \mod (1+4\sqrt{-2})$$
        
        \newpage
        \noindent
        Calculamos mcd$(3+2\sqrt{-2}, 1+4\sqrt{-2})$:

        \noindent
        En $\bb{Q}[\sqrt{-2}]$:
        $$\dfrac{1+4\sqrt{-2}}{3+2\sqrt{-2}} = \dfrac{19}{17} + \dfrac{10}{17}\sqrt{-2} \Rightarrow$$
        $$\Rightarrow (1+4\sqrt{-2}) = (3+2\sqrt{-2})(1+\sqrt{-2}) + (2-\sqrt{-2})$$
        En $\bb{Q}[\sqrt{-2}]$:
        $$\dfrac{3+2\sqrt{-2}}{2-\sqrt{-2}} = \dfrac{1}{3} + \dfrac{7}{6}\sqrt{-2} \Rightarrow$$
        $$\Rightarrow (3+2\sqrt{-2}) = (2-\sqrt{-2})(\sqrt{-2})+1$$
        \begin{equation*}
        \begin{array}{rcl}
            r_i & u_i & v_i \\ \\
            1+4\sqrt{-2} & 1 & 0 \\
            3+2\sqrt{-2} & 0 & 1 \\
            2-\sqrt{-2} & 1 & -(1+\sqrt{-2}) \\
            1 & -\sqrt{-2} & 1+\sqrt{-2}(1+\sqrt{-2})
        \end{array}
        \end{equation*}

        \noindent
        Luego mcd$(3+2\sqrt{-2}, 1+4\sqrt{-2})=1$ con identidad de Bezout:
        $$1 = -\sqrt{-2}(1+4\sqrt{-2}) + (3+2\sqrt{-2})(-1+\sqrt{-2}) \Rightarrow$$
        $$\Rightarrow 1+\sqrt{-2} = -\sqrt{-2}(1+\sqrt{-2})(1+4\sqrt{-2}) + (3+2\sqrt{-2})(-1+\sqrt{-2})(1+\sqrt{-2})$$
        Por lo que:
        $$(3+2\sqrt{-2})(-1+\sqrt{-2})(1+\sqrt{-2}) \equiv 1+\sqrt{-2} \mod (1+4\sqrt{-2})$$
        Luego $\phi_0 = (-1+\sqrt{-2})(1+\sqrt{-2}) = -3$ es una solución particular.\newline
        De hecho es la óptima, luego la solución general es:
        $$\phi = -3 + (1+4\sqrt{-2})\psi = -3 + (1+\sqrt{-2})(3+\sqrt{-2})\psi \mid \psi \in \bb{Z}[\sqrt{-2}]$$

        $$x = 4+(3+2\sqrt{-2})\phi = 4+(3+2\sqrt{-2})[-3+(1+\sqrt{-2})(3+\sqrt{-2})\psi] = $$
        $$ = 4+(3+2\sqrt{-2})(-3) + (3+2\sqrt{-2})(1+\sqrt{-2})(3+\sqrt{-2})\psi = $$
        $$ = 4-9-6\sqrt{-2} + (3+2\sqrt{-2})(1+\sqrt{-2})(3+\sqrt{-2})\psi = $$
        $$ = -5-6\sqrt{-2} + (3+2\sqrt{-2})(1+\sqrt{-2})(3+\sqrt{-2})\psi \mid \psi \in \bb{Z}[\sqrt{-2}]$$


        $$N(-5-6\sqrt{-2}) = 5^2 + 6^2 \cdot 2 = 97$$
        $$N[(3+2\sqrt{-2})(1+\sqrt{-2})(3+\sqrt{-2})] = N(3+2\sqrt{-2})N(1+\sqrt{-2})N(3+\sqrt{-2}) = $$
        $$= (3^2 + 2^2 \cdot 2)(1+2)(3^2 + 2) = 17 \cdot 3 \cdot 11 = 561$$
        Como $97 < 561 \Rightarrow x_0 = -5-6\sqrt{-2}$ es la solución óptima del sistema.\\

        \noindent
        Por tanto, la solución general del sistema es:
        $$x = -5-6\sqrt{-2} + (3+2\sqrt{-2})(1+\sqrt{-2})(3+\sqrt{-2})\psi \mid \psi \in \bb{Z}[\sqrt{-2}]$$
        \newpage
        \noindent
        Veamos si es posible encontrar una solución $a+b\sqrt{-2} \mid 30 < a < 56$:\\

        \noindent
        $$x = -5-6\sqrt{-2} + (-13+14\sqrt{-2})\psi \mid \psi \in \bb{Z}[\sqrt{-2}]$$
        Con $\psi = c+d\sqrt{-2}$ para ciertos $c$, $d \in \bb{Z}$.
        $$x = -5-6\sqrt{-2} + (-13+14\sqrt{-2})(c+d\sqrt{-2}) =$$ 
        $$ = -5-6\sqrt{-2} -13c + 14\sqrt{-2}c-13d\sqrt{-2}-28d =$$
        $$= (-5-13c-28d)+(-6+14c-13d)\sqrt{-2} \mid c,d \in \bb{Z}$$
        Como $30 < a < 56 \Rightarrow 30 < -5-13c-28d<56 \Rightarrow \psi = -1-\sqrt{-2}$ es una solución posible.\newline
        Luego $x = 36 -7\sqrt{-2}$ es una solución particular luego sí, es posible.

    \end{ejercicio}


\end{document}
