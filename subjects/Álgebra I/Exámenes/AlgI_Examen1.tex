\documentclass[12pt]{article}

% Idioma y codificación
\usepackage[spanish, es-tabla, es-notilde]{babel}       %es-tabla para que se titule "Tabla"
\usepackage[utf8]{inputenc}

% Márgenes
\usepackage[a4paper,top=3cm,bottom=2.5cm,left=3cm,right=3cm]{geometry}

% Comentarios de bloque
\usepackage{verbatim}

% Paquetes de links
\usepackage[hidelinks]{hyperref}    % Permite enlaces
\usepackage{url}                    % redirecciona a la web

% Más opciones para enumeraciones
\usepackage{enumitem}

% Personalizar la portada
\usepackage{titling}

% Paquetes de tablas
\usepackage{multirow}

% Para añadir el símbolo de euro
\usepackage{eurosym}


%------------------------------------------------------------------------

%Paquetes de figuras
\usepackage{caption}
\usepackage{subcaption} % Figuras al lado de otras
\usepackage{float}      % Poner figuras en el sitio indicado H.


% Paquetes de imágenes
\usepackage{graphicx}       % Paquete para añadir imágenes
\usepackage{transparent}    % Para manejar la opacidad de las figuras

% Paquete para usar colores
\usepackage[dvipsnames, table, xcdraw]{xcolor}
\usepackage{pagecolor}      % Para cambiar el color de la página

% Habilita tamaños de fuente mayores
\usepackage{fix-cm}

% Para los gráficos
\usepackage{tikz}
\usepackage{forest}

% Para poder situar los nodos en los grafos
\usetikzlibrary{positioning}


%------------------------------------------------------------------------

% Paquetes de matemáticas
\usepackage{mathtools, amsfonts, amssymb, mathrsfs}
\usepackage[makeroom]{cancel}     % Simplificar tachando
\usepackage{polynom}    % Divisiones y Ruffini
\usepackage{units} % Para poner fracciones diagonales con \nicefrac

\usepackage{pgfplots}   %Representar funciones
\pgfplotsset{compat=1.18}  % Versión 1.18

\usepackage{tikz-cd}    % Para usar diagramas de composiciones
\usetikzlibrary{calc}   % Para usar cálculo de coordenadas en tikz

%Definición de teoremas, etc.
\usepackage{amsthm}
%\swapnumbers   % Intercambia la posición del texto y de la numeración

\theoremstyle{plain}

\makeatletter
\@ifclassloaded{article}{
  \newtheorem{teo}{Teorema}[section]
}{
  \newtheorem{teo}{Teorema}[chapter]  % Se resetea en cada chapter
}
\makeatother

\newtheorem{coro}{Corolario}[teo]           % Se resetea en cada teorema
\newtheorem{prop}[teo]{Proposición}         % Usa el mismo contador que teorema
\newtheorem{lema}[teo]{Lema}                % Usa el mismo contador que teorema
\newtheorem*{lema*}{Lema}

\theoremstyle{remark}
\newtheorem*{observacion}{Observación}

\theoremstyle{definition}

\makeatletter
\@ifclassloaded{article}{
  \newtheorem{definicion}{Definición} [section]     % Se resetea en cada chapter
}{
  \newtheorem{definicion}{Definición} [chapter]     % Se resetea en cada chapter
}
\makeatother

\newtheorem*{notacion}{Notación}
\newtheorem*{ejemplo}{Ejemplo}
\newtheorem*{ejercicio*}{Ejercicio}             % No numerado
\newtheorem{ejercicio}{Ejercicio} [section]     % Se resetea en cada section


% Modificar el formato de la numeración del teorema "ejercicio"
\renewcommand{\theejercicio}{%
  \ifnum\value{section}=0 % Si no se ha iniciado ninguna sección
    \arabic{ejercicio}% Solo mostrar el número de ejercicio
  \else
    \thesection.\arabic{ejercicio}% Mostrar número de sección y número de ejercicio
  \fi
}


% \renewcommand\qedsymbol{$\blacksquare$}         % Cambiar símbolo QED
%------------------------------------------------------------------------

% Paquetes para encabezados
\usepackage{fancyhdr}
\pagestyle{fancy}
\fancyhf{}

\newcommand{\helv}{ % Modificación tamaño de letra
\fontfamily{}\fontsize{12}{12}\selectfont}
\setlength{\headheight}{15pt} % Amplía el tamaño del índice


%\usepackage{lastpage}   % Referenciar última pag   \pageref{LastPage}
%\fancyfoot[C]{%
%  \begin{minipage}{\textwidth}
%    \centering
%    ~\\
%    \thepage\\
%    \href{https://losdeldgiim.github.io/}{\texttt{\footnotesize losdeldgiim.github.io}}
%  \end{minipage}
%}
\fancyfoot[C]{\thepage}
\fancyfoot[R]{\href{https://losdeldgiim.github.io/}{\texttt{\footnotesize losdeldgiim.github.io}}}

%------------------------------------------------------------------------

% Conseguir que no ponga "Capítulo 1". Sino solo "1."
\makeatletter
\@ifclassloaded{book}{
  \renewcommand{\chaptermark}[1]{\markboth{\thechapter.\ #1}{}} % En el encabezado
    
  \renewcommand{\@makechapterhead}[1]{%
  \vspace*{50\p@}%
  {\parindent \z@ \raggedright \normalfont
    \ifnum \c@secnumdepth >\m@ne
      \huge\bfseries \thechapter.\hspace{1em}\ignorespaces
    \fi
    \interlinepenalty\@M
    \Huge \bfseries #1\par\nobreak
    \vskip 40\p@
  }}
}
\makeatother

%------------------------------------------------------------------------
% Paquetes de cógido
\usepackage{minted}
\renewcommand\listingscaption{Código fuente}

\usepackage{fancyvrb}
% Personaliza el tamaño de los números de línea
\renewcommand{\theFancyVerbLine}{\small\arabic{FancyVerbLine}}

% Estilo para C++
\newminted{cpp}{
    frame=lines,
    framesep=2mm,
    baselinestretch=1.2,
    linenos,
    escapeinside=||
}

% para minted
\definecolor{LightGray}{rgb}{0.95,0.95,0.92}
\setminted{
    linenos=true,
    stepnumber=5,
    numberfirstline=true,
    autogobble,
    breaklines=true,
    breakautoindent=true,
    breaksymbolleft=,
    breaksymbolright=,
    breaksymbolindentleft=0pt,
    breaksymbolindentright=0pt,
    breaksymbolsepleft=0pt,
    breaksymbolsepright=0pt,
    fontsize=\footnotesize,
    bgcolor=LightGray,
    numbersep=10pt
}


\usepackage{listings} % Para incluir código desde un archivo

\renewcommand\lstlistingname{Código Fuente}
\renewcommand\lstlistlistingname{Índice de Códigos Fuente}

% Definir colores
\definecolor{vscodepurple}{rgb}{0.5,0,0.5}
\definecolor{vscodeblue}{rgb}{0,0,0.8}
\definecolor{vscodegreen}{rgb}{0,0.5,0}
\definecolor{vscodegray}{rgb}{0.5,0.5,0.5}
\definecolor{vscodebackground}{rgb}{0.97,0.97,0.97}
\definecolor{vscodelightgray}{rgb}{0.9,0.9,0.9}

% Configuración para el estilo de C similar a VSCode
\lstdefinestyle{vscode_C}{
  backgroundcolor=\color{vscodebackground},
  commentstyle=\color{vscodegreen},
  keywordstyle=\color{vscodeblue},
  numberstyle=\tiny\color{vscodegray},
  stringstyle=\color{vscodepurple},
  basicstyle=\scriptsize\ttfamily,
  breakatwhitespace=false,
  breaklines=true,
  captionpos=b,
  keepspaces=true,
  numbers=left,
  numbersep=5pt,
  showspaces=false,
  showstringspaces=false,
  showtabs=false,
  tabsize=2,
  frame=tb,
  framerule=0pt,
  aboveskip=10pt,
  belowskip=10pt,
  xleftmargin=10pt,
  xrightmargin=10pt,
  framexleftmargin=10pt,
  framexrightmargin=10pt,
  framesep=0pt,
  rulecolor=\color{vscodelightgray},
  backgroundcolor=\color{vscodebackground},
}

%------------------------------------------------------------------------

% Comandos definidos
\newcommand{\bb}[1]{\mathbb{#1}}
\newcommand{\cc}[1]{\mathcal{#1}}

% I prefer the slanted \leq
\let\oldleq\leq % save them in case they're every wanted
\let\oldgeq\geq
\renewcommand{\leq}{\leqslant}
\renewcommand{\geq}{\geqslant}

% Si y solo si
\newcommand{\sii}{\iff}

% MCD y MCM
\DeclareMathOperator{\mcd}{mcd}
\DeclareMathOperator{\mcm}{mcm}

% Signo
\DeclareMathOperator{\sgn}{sgn}

% Letras griegas
\newcommand{\eps}{\epsilon}
\newcommand{\veps}{\varepsilon}
\newcommand{\lm}{\lambda}

\newcommand{\ol}{\overline}
\newcommand{\ul}{\underline}
\newcommand{\wt}{\widetilde}
\newcommand{\wh}{\widehat}

\let\oldvec\vec
\renewcommand{\vec}{\overrightarrow}

% Derivadas parciales
\newcommand{\del}[2]{\frac{\partial #1}{\partial #2}}
\newcommand{\Del}[3]{\frac{\partial^{#1} #2}{\partial #3^{#1}}}
\newcommand{\deld}[2]{\dfrac{\partial #1}{\partial #2}}
\newcommand{\Deld}[3]{\dfrac{\partial^{#1} #2}{\partial #3^{#1}}}


\newcommand{\AstIg}{\stackrel{(\ast)}{=}}
\newcommand{\Hop}{\stackrel{L'H\hat{o}pital}{=}}

\newcommand{\red}[1]{{\color{red}#1}} % Para integrales, destacar los cambios.

% Método de integración
\newcommand{\MetInt}[2]{
    \left[\begin{array}{c}
        #1 \\ #2
    \end{array}\right]
}

% Declarar aplicaciones
% 1. Nombre aplicación
% 2. Dominio
% 3. Codominio
% 4. Variable
% 5. Imagen de la variable
\newcommand{\Func}[5]{
    \begin{equation*}
        \begin{array}{rrll}
            \displaystyle #1:& \displaystyle  #2 & \longrightarrow & \displaystyle  #3\\
               & \displaystyle  #4 & \longmapsto & \displaystyle  #5
        \end{array}
    \end{equation*}
}

%------------------------------------------------------------------------

\usepackage{extarrows}
\usepackage{stackrel}

% En el preámbulo del documento o en tu archivo .sty
\newcounter{ejercicio}[section] % Define el contador de ejercicio y lo reinicia con cada sección

\newcounter{ejercicio}
\newcommand{\resetearcontador}{%
  \setcounter{ejercicio}{0} % Resetea el contador de ejercicios a 0
}

\renewcommand{\theejercicio}{\arabic{ejercicio}}

\begin{document}

    % 1. Foto de fondo
    % 2. Título
    % 3. Encabezado Izquierdo
    % 4. Color de fondo
    % 5. Coord x del titulo
    % 6. Coord y del titulo
    % 7. Fecha

    
    % 1. Foto de fondo
% 2. Título
% 3. Encabezado Izquierdo
% 4. Color de fondo
% 5. Coord x del titulo
% 6. Coord y del titulo
% 7. Fecha
% 8. Autor

\newcommand{\portada}[8]{
    \portadaBase{#1}{#2}{#3}{#4}{#5}{#6}{#7}{#8}
    \portadaBook{#1}{#2}{#3}{#4}{#5}{#6}{#7}{#8}
}

\newcommand{\portadaFotoDif}[8]{
    \portadaBaseFotoDif{#1}{#2}{#3}{#4}{#5}{#6}{#7}{#8}
    \portadaBook{#1}{#2}{#3}{#4}{#5}{#6}{#7}{#8}
}

\newcommand{\portadaExamen}[8]{
    \portadaBase{#1}{#2}{#3}{#4}{#5}{#6}{#7}{#8}
    \portadaArticle{#1}{#2}{#3}{#4}{#5}{#6}{#7}{#8}
}

\newcommand{\portadaExamenFotoDif}[8]{
    \portadaBaseFotoDif{#1}{#2}{#3}{#4}{#5}{#6}{#7}{#8}
    \portadaArticle{#1}{#2}{#3}{#4}{#5}{#6}{#7}{#8}
}




\newcommand{\portadaBase}[8]{

    % Tiene la portada principal y la licencia Creative Commons
    
    % 1. Foto de fondo
    % 2. Título
    % 3. Encabezado Izquierdo
    % 4. Color de fondo
    % 5. Coord x del titulo
    % 6. Coord y del titulo
    % 7. Fecha
    % 8. Autor    
    
    \thispagestyle{empty}               % Sin encabezado ni pie de página
    \newgeometry{margin=0cm}        % Márgenes nulos para la primera página
    
    
    % Encabezado
    \fancyhead[L]{\helv #3}
    \fancyhead[R]{\helv \nouppercase{\leftmark}}
    
    
    \pagecolor{#4}        % Color de fondo para la portada
    
    \begin{figure}[p]
        \centering
        \transparent{0.3}           % Opacidad del 30% para la imagen
        
        \includegraphics[width=\paperwidth, keepaspectratio]{../../_assets/#1}
    
        \begin{tikzpicture}[remember picture, overlay]
            \node[anchor=north west, text=white, opacity=1, font=\fontsize{60}{90}\selectfont\bfseries\sffamily, align=left] at (#5, #6) {#2};
            
            \node[anchor=south east, text=white, opacity=1, font=\fontsize{12}{18}\selectfont\sffamily, align=right] at (9.7, 3) {\href{https://losdeldgiim.github.io/}{\textbf{Los Del DGIIM}, \texttt{\footnotesize losdeldgiim.github.io}}};
            
            \node[anchor=south east, text=white, opacity=1, font=\fontsize{12}{15}\selectfont\sffamily, align=right] at (9.7, 1.8) {Doble Grado en Ingeniería Informática y Matemáticas\\Universidad de Granada};
        \end{tikzpicture}
    \end{figure}
    
    
    \restoregeometry        % Restaurar márgenes normales para las páginas subsiguientes
    \nopagecolor      % Restaurar el color de página
    
    
    \newpage
    \thispagestyle{empty}               % Sin encabezado ni pie de página
    \begin{tikzpicture}[remember picture, overlay]
        \node[anchor=south west, inner sep=3cm] at (current page.south west) {
            \begin{minipage}{0.5\paperwidth}
                \href{https://creativecommons.org/licenses/by-nc-nd/4.0/}{
                    \includegraphics[height=2cm]{../../_assets/Licencia.png}
                }\vspace{1cm}\\
                Esta obra está bajo una
                \href{https://creativecommons.org/licenses/by-nc-nd/4.0/}{
                    Licencia Creative Commons Atribución-NoComercial-SinDerivadas 4.0 Internacional (CC BY-NC-ND 4.0).
                }\\
    
                Eres libre de compartir y redistribuir el contenido de esta obra en cualquier medio o formato, siempre y cuando des el crédito adecuado a los autores originales y no persigas fines comerciales. 
            \end{minipage}
        };
    \end{tikzpicture}
    
    
    
    % 1. Foto de fondo
    % 2. Título
    % 3. Encabezado Izquierdo
    % 4. Color de fondo
    % 5. Coord x del titulo
    % 6. Coord y del titulo
    % 7. Fecha
    % 8. Autor


}


\newcommand{\portadaBaseFotoDif}[8]{

    % Tiene la portada principal y la licencia Creative Commons
    
    % 1. Foto de fondo
    % 2. Título
    % 3. Encabezado Izquierdo
    % 4. Color de fondo
    % 5. Coord x del titulo
    % 6. Coord y del titulo
    % 7. Fecha
    % 8. Autor    
    
    \thispagestyle{empty}               % Sin encabezado ni pie de página
    \newgeometry{margin=0cm}        % Márgenes nulos para la primera página
    
    
    % Encabezado
    \fancyhead[L]{\helv #3}
    \fancyhead[R]{\helv \nouppercase{\leftmark}}
    
    
    \pagecolor{#4}        % Color de fondo para la portada
    
    \begin{figure}[p]
        \centering
        \transparent{0.3}           % Opacidad del 30% para la imagen
        
        \includegraphics[width=\paperwidth, keepaspectratio]{#1}
    
        \begin{tikzpicture}[remember picture, overlay]
            \node[anchor=north west, text=white, opacity=1, font=\fontsize{60}{90}\selectfont\bfseries\sffamily, align=left] at (#5, #6) {#2};
            
            \node[anchor=south east, text=white, opacity=1, font=\fontsize{12}{18}\selectfont\sffamily, align=right] at (9.7, 3) {\href{https://losdeldgiim.github.io/}{\textbf{Los Del DGIIM}, \texttt{\footnotesize losdeldgiim.github.io}}};
            
            \node[anchor=south east, text=white, opacity=1, font=\fontsize{12}{15}\selectfont\sffamily, align=right] at (9.7, 1.8) {Doble Grado en Ingeniería Informática y Matemáticas\\Universidad de Granada};
        \end{tikzpicture}
    \end{figure}
    
    
    \restoregeometry        % Restaurar márgenes normales para las páginas subsiguientes
    \nopagecolor      % Restaurar el color de página
    
    
    \newpage
    \thispagestyle{empty}               % Sin encabezado ni pie de página
    \begin{tikzpicture}[remember picture, overlay]
        \node[anchor=south west, inner sep=3cm] at (current page.south west) {
            \begin{minipage}{0.5\paperwidth}
                %\href{https://creativecommons.org/licenses/by-nc-nd/4.0/}{
                %    \includegraphics[height=2cm]{../../_assets/Licencia.png}
                %}\vspace{1cm}\\
                Esta obra está bajo una
                \href{https://creativecommons.org/licenses/by-nc-nd/4.0/}{
                    Licencia Creative Commons Atribución-NoComercial-SinDerivadas 4.0 Internacional (CC BY-NC-ND 4.0).
                }\\
    
                Eres libre de compartir y redistribuir el contenido de esta obra en cualquier medio o formato, siempre y cuando des el crédito adecuado a los autores originales y no persigas fines comerciales. 
            \end{minipage}
        };
    \end{tikzpicture}
    
    
    
    % 1. Foto de fondo
    % 2. Título
    % 3. Encabezado Izquierdo
    % 4. Color de fondo
    % 5. Coord x del titulo
    % 6. Coord y del titulo
    % 7. Fecha
    % 8. Autor


}


\newcommand{\portadaBook}[8]{

    % 1. Foto de fondo
    % 2. Título
    % 3. Encabezado Izquierdo
    % 4. Color de fondo
    % 5. Coord x del titulo
    % 6. Coord y del titulo
    % 7. Fecha
    % 8. Autor

    % Personaliza el formato del título
    \pretitle{\begin{center}\bfseries\fontsize{42}{56}\selectfont}
    \posttitle{\par\end{center}\vspace{2em}}
    
    % Personaliza el formato del autor
    \preauthor{\begin{center}\Large}
    \postauthor{\par\end{center}\vfill}
    
    % Personaliza el formato de la fecha
    \predate{\begin{center}\huge}
    \postdate{\par\end{center}\vspace{2em}}
    
    \title{#2}
    \author{\href{https://losdeldgiim.github.io/}{Los Del DGIIM, \texttt{\large losdeldgiim.github.io}}
    \\ \vspace{0.5cm}#8}
    \date{Granada, #7}
    \maketitle
    
    \tableofcontents
}




\newcommand{\portadaArticle}[8]{

    % 1. Foto de fondo
    % 2. Título
    % 3. Encabezado Izquierdo
    % 4. Color de fondo
    % 5. Coord x del titulo
    % 6. Coord y del titulo
    % 7. Fecha
    % 8. Autor

    % Personaliza el formato del título
    \pretitle{\begin{center}\bfseries\fontsize{42}{56}\selectfont}
    \posttitle{\par\end{center}\vspace{2em}}
    
    % Personaliza el formato del autor
    \preauthor{\begin{center}\Large}
    \postauthor{\par\end{center}\vspace{3em}}
    
    % Personaliza el formato de la fecha
    \predate{\begin{center}\huge}
    \postdate{\par\end{center}\vspace{5em}}
    
    \title{#2}
    \author{\href{https://losdeldgiim.github.io/}{Los Del DGIIM, \texttt{\large losdeldgiim.github.io}}
    \\ \vspace{0.5cm}#8}
    \date{Granada, #7}
    \thispagestyle{empty}               % Sin encabezado ni pie de página
    \maketitle
    \vfill
}
    \portadaExamen{ffccA4.jpg}{Álgebra I\\Examen I}{Álgebra I. Examen I}{MidnightBlue}{-8}{28}{2023-2024}{José Juan Urrutia Milán}

    
    \begin{description}
        \item[Asignatura] Álgebra I.
        \item[Curso Académico] 2022-23.
        \item[Grado] Doble Grado en Ingeniería Informática y Matemáticas.
        \item[Grupo] Único.
        \item[Profesor] María Pilar Carrasco Carrasco.
        \item[Descripción] Parcial de temas 1 y 2.
        \item[Fecha] 7 de noviembre de 2022.
        \item[Duración] 2 horas.
    
    \end{description}
    \newpage
    
    \noindent
    La puntuación de cada ejercicio es de 1 punto.\newline
    Todas las respuestas deben estar justificadas.

    \begin{ejercicio}
        El polinomio $f = x² + x + 4 \in \mathbb{Z}_5[x]$ tiene:
        \begin{itemize}
            \item Dos raíces en $\mathbb{Z}_5[x]$.
            \item Una raíz en $\mathbb{Z}_5[x]$.
            \item No tiene raíces en $\mathbb{Z}_5[x]$.
        \end{itemize}
    \end{ejercicio}

    \begin{ejercicio}
        El anillo producto cartesiano $\mathbb{Z}_{10} \times \mathbb{Z}_3$ tiene:
        \begin{itemize}
            \item 13 Unidades.
            \item 18 Unidades.
            \item 8 Unidades.
        \end{itemize}
    \end{ejercicio}

    \begin{ejercicio}
        Sean $a_1 = 2120$, $a_2 = 4825$, $b = 19$. El resto de dividir $-a_1a_2$ entre $b$ es:
        \begin{itemize}
            \item 11.
            \item 8.
            \item 18.
        \end{itemize}
    \end{ejercicio}

    \begin{ejercicio}
       Sea $A$ un subanillo no trivial de un cuerpo $K$ ¿Cuál de las siguientes afirmaciones es verdadera? 
        \begin{itemize}
            \item $A$ es siempre un cuerpo.
            \item $A$ es nunca un cuerpo.
            \item $A$ es un cuerpo si, y sólamente si, es cerrado para inversos.
        \end{itemize}
    \end{ejercicio}

    \begin{ejercicio}
        Sea $X$ un conjunto con $n$ elementos y $R$ la relación de equivalencia sobre el conjunto $\mathcal{P}(X)$ definida por:
        $$A R B \Leftrightarrow |A| = |B|$$
        Selecciona la afirmación verdadera:
        \begin{itemize}
            \item $|\mathcal{P}(X)/R| = n$.
            \item $|\mathcal{P}(X)/R| = n+1$.
            \item $|\mathcal{P}(X)/R| = n-1$.
        \end{itemize}
    \end{ejercicio}

    \begin{ejercicio}
        Sea $f:\mathbb{R} \rightarrow \mathbb{R}$ definida por $f(x) = 5x -2$, para todo $x \in \mathbb{R}$. Selecciona la afirmación verdadera:
        \begin{itemize}
            \item $f$ es inyectiva no sobreyectiva.
            \item $f$ es sobreyectiva no inyectiva.
            \item $f$ tiene inversa.
        \end{itemize}
    \end{ejercicio}

    \begin{ejercicio}
        Sea $X = \{a, b, c\}$ e $Y = \{1, 2\}$. Selecciona la afirmación verdadera:
        \begin{itemize}
            \item No existe ninguna aplicación biyectiva de $X$ de $Y$ pero existe al menos una inyectiva.
            \item Hay exactamente 6 aplicaciones de $X$ en $Y$ que son sobreyectivas.
            \item Hay exactamente 3 aplicaciones de $X$ en $Y$ que no son sobreyectivas.
        \end{itemize}
    \end{ejercicio}

    \begin{ejercicio}
        Sea $X$ un conjunto no vacío y $A$, $B$, $C \in \mathcal{P}(X)$. Selecciona la afirmación verdadera:
        \begin{itemize}
            \item $(A-C) \cup (B-C) = (A\cap B)-C$
            \item $(A-C) \cap (B-C) = (A\cup B)-C$
            \item $(A-C) \cup (B-C) = (A\cup B)-C$
        \end{itemize}
    \end{ejercicio}

    \begin{ejercicio}
        Para $a \in \mathbb{Z}$ un número entero, denotemos por $[a]$ a su clase en el anillo $\mathbb{Z}_5$. Selecciona la respuesta correcta:
        \begin{itemize}
            \item Si $[a] \neq [0] \Rightarrow [a^4+4] = [0]$.
            \item Si $[a] \neq [0] \Rightarrow [a^4+4] \neq [0]$.
            \item Si $[a] = [0] \Rightarrow [a^4+4] = [0]$.
        \end{itemize}
    \end{ejercicio}

    \begin{ejercicio}
        Sea $X$ un conjunto finito no vacío e $Y$ un subconjunto de $X$. Sea $\sim$ la relación de equivalencia sobre el conjunto $\mathcal{P}(X)$ definida por:
        $$A \sim B \Leftrightarrow A \cup Y = B \cup Y$$
        La afirmación ''$|\mathcal{P}(X)/\sim|=1$'' es:
        \begin{itemize}
            \item Siempre cierta.
            \item Siempre falsa.
            \item A veces verdad y a veces falsa, dependiendo de $Y$.
        \end{itemize}
    \end{ejercicio}

    \resetearcontador

    \newpage
    \ 
    \newpage

    % ------------------------------------------------------------------------------------------------------------------------------------------------
    
    \begin{ejercicio}
        El polinomio $f = x² + x + 4 \in \mathbb{Z}_5[x]$ tiene:
        \begin{itemize}
            \item Dos raíces en $\mathbb{Z}_5[x]$.
            \item \underline{Una raíz en $\mathbb{Z}_5[x]$.}
            \item No tiene raíces en $\mathbb{Z}_5[x]$.
        \end{itemize}
        \textbf{Justificación:}\newline
        Evaluando $f$ en cada uno de los elementos de $\bb{Z}_5$ obtenemos:
        $$f(0) = 4~~~f(1)=1~~~f(2)=0~~~f(3)=1~~~f(4)=4$$
        Sólo una raíz (un 0).
    \end{ejercicio}

    \begin{ejercicio}
        El anillo producto cartesiano $\mathbb{Z}_{10} \times \mathbb{Z}_3$ tiene:
        \begin{itemize}
            \item 13 Unidades.
            \item 18 Unidades.
            \item \underline{8 Unidades.}
        \end{itemize}
        \textbf{Justificación:}\newline
        Sabemos que $U(\bb{Z}_{10} \times \bb{Z}_3) = U(\bb{Z}_{10})\times U(\bb{Z}_3)$.\newline
        Como: $U(\bb{Z}_{10}) = \{1, 3, 7, 9\}$ (siendo $1^{-1} = 1$, $3^{-1}=7$, $7^{-1}=3$ y $9^{-1}=9$).\newline
        $U(\bb{Z}_3) = \{1, 2\}$ (siendo $1^{-1}=1$, $2^{-1}=2$).\newline
        Entonces, $\mathbb{Z}_{10} \times \mathbb{Z}_3$ tiene $4\cdot 2 = 8$ unidades.
    \end{ejercicio}

    \begin{ejercicio}
        Sean $a_1 = 2120$, $a_2 = 4825$, $b = 19$. El resto de dividir $-a_1a_2$ entre $b$ es:
        \begin{itemize}
            \item \underline{11.}
            \item 8.
            \item 18.
        \end{itemize}
        \textbf{Justificación:}\newline
        Al dividir $a_1a_2$ entre 19 obtenemos: $a_1a_2 = 19\cdot q + r$ con $q = 538368$ y $r = 8$.\newline
        Entonces, $-a_1a_2 = 19(-q-1)+19-r$ y entonces el resto es $19 - r = 19 -8 = 11$.
    \end{ejercicio}

    \begin{ejercicio}
       Sea $A$ un subanillo no trivial de un cuerpo $K$ ¿Cuál de las siguientes afirmaciones es verdadera? 
        \begin{itemize}
            \item $A$ es siempre un cuerpo.
            \item $A$ es nunca un cuerpo.
            \item \underline{$A$ es un cuerpo si, y sólamente si, es cerrado para inversos.}
        \end{itemize}
        \textbf{Justificación:}\newline
        Supongamos que $A$ es un cuerpo y sea $u \in A\setminus \{0\}$ un elemento no nulo de $A$.\newline
        Si $u' \in A$ denota el inverso de $u$ en $A$, será $u\cdot u'=1$ en el cuerpo $K$.\newline
        Como el inverso es único, entonces $u'=u^{-1}$ y $A$ es cerrado para opuestos.\\

        \noindent
        Recíprocamente, si $A$ es cerrado para inversos, entonces todo elemento no nulo de $A$ tiene inverso en $A$ (el mismo que en $K$).\newline
        Es decir, $U(A) = A\setminus \{0\}$. Consecuentemente, $A$ es un cuerpo.
    \end{ejercicio}

    \begin{ejercicio}
        Sea $X$ un conjunto con $n$ elementos y $R$ la relación de equivalencia sobre el conjunto $\mathcal{P}(X)$ definida por:
        $$A R B \Leftrightarrow |A| = |B|$$
        Selecciona la afirmación verdadera:
        \begin{itemize}
            \item $|\mathcal{P}(X)/R| = n$.
            \item \underline{$|\mathcal{P}(X)/R| = n+1$.}
            \item $|\mathcal{P}(X)/R| = n-1$.
        \end{itemize}
        \textbf{Justificación:}\newline
        Para cada $0\leq k\leq n$, sea $A\in \mathcal{P}(X)$ con $|A|=k$.\newline
        Entonces, $[A] = \{B \in \mathcal{P}(X) \mid |B| = |A|\} = \{B \in \mathcal{P}(X) \mid |B| = k\}$.\newline
        Consecuentemente, si $X = \{x_1, \ldots, x_n\}$, las clases de equivalencia son:
        $$[\emptyset],~~~[\{x_1\}],~~~[\{x_1,x_2\}],~~~\ldots,~~~[X]$$
        Es decir, $|\mathcal{P}(X)/R| = n+1$.
    \end{ejercicio}

    \begin{ejercicio}
        Sea $f:\mathbb{R} \rightarrow \mathbb{R}$ definida por $f(x) = 5x -2$, para todo $x \in \mathbb{R}$. Selecciona la afirmación verdadera:
        \begin{itemize}
            \item $f$ es inyectiva no sobreyectiva.
            \item $f$ es sobreyectiva no inyectiva.
            \item \underline{$f$ tiene inversa.}
        \end{itemize}
        \textbf{Justificación:}\newline
        Es fácil ver que la aplicación $g:\bb{R}\rightarrow \bb{R}$ definida por:
        $$g(x) = \dfrac{x+2}{5}~~~~~~\forall x \in \bb{R}$$
        Es la inversa de $f$:
        $$(f \circ g)(x) = f(g(x)) = f\left(\dfrac{x+2}{5}\right) = 5\left(\dfrac{x+2}{5}\right)-2 = (x+2)-2 = x = Id(x)$$
        $$(g \circ f)(x) = g(f(x)) = g(5x-2) = \dfrac{5x-2+2}{5} = \dfrac{5x}{5} = x = Id(x)$$
    \end{ejercicio}

    \begin{ejercicio}
        Sea $X = \{a, b, c\}$ e $Y = \{1, 2\}$. Selecciona la afirmación verdadera:
        \begin{itemize}
            \item No existe ninguna aplicación biyectiva de $X$ de $Y$ pero existe al menos una inyectiva.
            \item \underline{Hay exactamente 6 aplicaciones de $X$ en $Y$ que son sobreyectivas.}
            \item Hay exactamente 3 aplicaciones de $X$ en $Y$ que no son sobreyectivas.
        \end{itemize}
        \textbf{Justificación:}\newline
        Son las siguientes:
        $$f_1:\left\{\begin{tabular}{ccc}
            a & \mapsto & 1 \\
            b & \mapsto & 1 \\
            c & \mapsto & 2 \\
        \end{tabular}\right.~~~~
        f_2:\left\{\begin{tabular}{ccc}
            a & \mapsto & 1 \\
            b & \mapsto & 2 \\
            c & \mapsto & 1 \\
        \end{tabular}\right.~~~~
        f_3:\left\{\begin{tabular}{ccc}
            a & \mapsto & 1 \\
            b & \mapsto & 2 \\
            c & \mapsto & 2 \\
        \end{tabular}\right.$$
        
        $$f_4:\left\{\begin{tabular}{ccc}
            a & \mapsto & 2 \\
            b & \mapsto & 1 \\
            c & \mapsto & 1 \\
        \end{tabular}\right.~~~~
        f_5:\left\{\begin{tabular}{ccc}
            a & \mapsto & 2 \\
            b & \mapsto & 1 \\
            c & \mapsto & 2 \\
        \end{tabular}\right.~~~~
        f_6:\left\{\begin{tabular}{ccc}
            a & \mapsto & 2 \\
            b & \mapsto & 2 \\
            c & \mapsto & 1 \\
        \end{tabular}\right.$$
        Es fácil ver que son sobreyectivas, ya que $f_i(X) = Y$ para $i \in \{1, 2, 3, 4, 5, 6\}$.
        
    \end{ejercicio}

    \begin{ejercicio}
        Sea $X$ un conjunto no vacío y $A$, $B$, $C \in \mathcal{P}(X)$. Selecciona la afirmación verdadera:
        \begin{itemize}
            \item $(A-C) \cup (B-C) = (A\cap B)-C$
            \item $(A-C) \cap (B-C) = (A\cup B)-C$
            \item \underline{$(A-C) \cup (B-C) = (A\cup B)-C$}
        \end{itemize}
        \textbf{Justificación:}\newline
        $$(A-C)\cup (B-C) = (A \cap c(C)) \cup (B \cap c(C)) = (A \cup B)\cap c(C) = (A \cup B) - C$$
    \end{ejercicio}

    \begin{ejercicio}
        Para $a \in \mathbb{Z}$ un número entero, denotemos por $[a]$ a su clase en el anillo $\mathbb{Z}_5$. Selecciona la respuesta correcta:
        \begin{itemize}
            \item \underline{Si $[a] \neq [0] \Rightarrow [a^4+4] = [0]$.}
            \item Si $[a] \neq [0] \Rightarrow [a^4+4] \neq [0]$.
            \item Si $[a] = [0] \Rightarrow [a^4+4] = [0]$.
        \end{itemize}
        \textbf{Justificación:}\newline
        Si $[a] \neq 0$, entonces $[a] = [r]$ con $1 \leq r \leq 4$.\newline
        Y entonces: $[a^4+1] = [a]^4 + [1] = [r]^4 +[1] = [r^4 +1]$, con lo que:
        $$\begin{tabular}{rl}
            \mbox{Para } r=1, & [a^4+4] = [1+4] = [5] = [0] \\
            \mbox{Para } r=2, & [a^4+4] = [16+4] = [20] = [0] \\
            \mbox{Para } r=3, & [a^4+4] = [81+4] = [85] = [0] \\
            \mbox{Para } r=4, & [a^4+4] = [256+4] = [260] = [0]
        \end{tabular}$$
    \end{ejercicio}

    \begin{ejercicio}
        Sea $X$ un conjunto finito no vacío e $Y$ un subconjunto de $X$. Sea $\sim$ la relación de equivalencia sobre el conjunto $\mathcal{P}(X)$ definida por:
        $$A \sim B \Leftrightarrow A \cup Y = B \cup Y$$
        La afirmación ''$|\mathcal{P}(X)/\sim|=1$'' es:
        \begin{itemize}
            \item Siempre cierta.
            \item Siempre falsa.
            \item \underline{A veces verdad y a veces falsa, dependiendo de $Y$.}
        \end{itemize}
        \textbf{Justificación:}\newline
        Para $Y = \emptyset$, la relación $\sim$ es:
        $$A \sim B \Leftrightarrow A \cup \emptyset = B \cup \emptyset \Leftrightarrow A = B$$
        Y entonces, para cada $A \in \mathcal{P}(X)$, su clase es $[A] = \{A\}$, con lo que:
        $$|\mathcal{P}(X)/\sim| = |\mathcal{P}(X)| = 2^{|X|} \geq 2$$
        Pues $|X| \geq 1$. Así que la afirmación es falsa en este caso.\\

        \noindent
        Por otro lado, para $Y=X$, la relación $\sim$ es:
        $$A \sim B \Leftrightarrow A \cup X = B \cup X \Leftrightarrow X = X$$
        Y entonces, todos los elemtos de $\mathcal{P}(X)$ están relacionados, con lo que hay únicamente una clase de equivalencia. Luego:
        $$|\mathcal{P}(X)/\sim| =1$$
        Así que la afirmación es verdadera en este caso.
    \end{ejercicio}

\end{document}
