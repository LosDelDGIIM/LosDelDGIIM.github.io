\documentclass[12pt]{book}

% Idioma y codificación
\usepackage[spanish, es-tabla]{babel}       %es-tabla para que se titule "Tabla"
\usepackage[utf8]{inputenc}

% Márgenes
\usepackage[a4paper,top=3cm,bottom=2.5cm,left=3cm,right=3cm]{geometry}

% Comentarios de bloque
\usepackage{verbatim}

% Paquetes de links
\usepackage[hidelinks]{hyperref}    % Permite enlaces
\usepackage{url}                    % redirecciona a la web

% Más opciones para enumeraciones
\usepackage{enumitem}

% Personalizar la portada
\usepackage{titling}

% Paquetes de tablas
\usepackage{multirow}


%------------------------------------------------------------------------

%Paquetes de figuras
\usepackage{caption}
\usepackage{subcaption} % Figuras al lado de otras
\usepackage{float}      % Poner figuras en el sitio indicado H.


% Paquetes de imágenes
\usepackage{graphicx}       % Paquete para añadir imágenes
\usepackage{transparent}    % Para manejar la opacidad de las figuras

% Paquete para usar colores
\usepackage[dvipsnames]{xcolor}
\usepackage{pagecolor}      % Para cambiar el color de la página

% Habilita tamaños de fuente mayores
\usepackage{fix-cm}

% Para los gráficos
\usepackage{tikz}

% Para poder situar los nodos en los grafos
\usetikzlibrary{positioning}


%------------------------------------------------------------------------

% Paquetes de matemáticas
\usepackage{mathtools, amsfonts, amssymb, mathrsfs}
\usepackage[makeroom]{cancel}     % Simplificar tachando
\usepackage{polynom}    % Divisiones y Ruffini
\usepackage{units} % Para poner fracciones diagonales con \nicefrac

\usepackage{pgfplots}   %Representar funciones
\pgfplotsset{compat=1.18}  % Versión 1.18

\usepackage{tikz-cd}    % Para usar diagramas de composiciones
\usetikzlibrary{calc}   % Para usar cálculo de coordenadas en tikz

%Definición de teoremas, etc.
\usepackage{amsthm}
%\swapnumbers   % Intercambia la posición del texto y de la numeración

\theoremstyle{plain}

\makeatletter
\@ifclassloaded{article}{
  \newtheorem{teo}{Teorema}[section]
}{
  \newtheorem{teo}{Teorema}[chapter]  % Se resetea en cada chapter
}
\makeatother

\newtheorem{coro}{Corolario}[teo]           % Se resetea en cada teorema
\newtheorem{prop}[teo]{Proposición}         % Usa el mismo contador que teorema
\newtheorem{lema}[teo]{Lema}                % Usa el mismo contador que teorema

\theoremstyle{remark}
\newtheorem*{observacion}{Observación}

\theoremstyle{definition}

\makeatletter
\@ifclassloaded{article}{
  \newtheorem{definicion}{Definición} [section]     % Se resetea en cada chapter
}{
  \newtheorem{definicion}{Definición} [chapter]     % Se resetea en cada chapter
}
\makeatother

\newtheorem*{notacion}{Notación}
\newtheorem*{ejemplo}{Ejemplo}
\newtheorem*{ejercicio*}{Ejercicio}             % No numerado
\newtheorem{ejercicio}{Ejercicio} [section]     % Se resetea en cada section


% Modificar el formato de la numeración del teorema "ejercicio"
\renewcommand{\theejercicio}{%
  \ifnum\value{section}=0 % Si no se ha iniciado ninguna sección
    \arabic{ejercicio}% Solo mostrar el número de ejercicio
  \else
    \thesection.\arabic{ejercicio}% Mostrar número de sección y número de ejercicio
  \fi
}


% \renewcommand\qedsymbol{$\blacksquare$}         % Cambiar símbolo QED
%------------------------------------------------------------------------

% Paquetes para encabezados
\usepackage{fancyhdr}
\pagestyle{fancy}
\fancyhf{}

\newcommand{\helv}{ % Modificación tamaño de letra
\fontfamily{}\fontsize{12}{12}\selectfont}
\setlength{\headheight}{15pt} % Amplía el tamaño del índice


%\usepackage{lastpage}   % Referenciar última pag   \pageref{LastPage}
\fancyfoot[C]{\thepage}

%------------------------------------------------------------------------

% Conseguir que no ponga "Capítulo 1". Sino solo "1."
\makeatletter
\@ifclassloaded{book}{
  \renewcommand{\chaptermark}[1]{\markboth{\thechapter.\ #1}{}} % En el encabezado
    
  \renewcommand{\@makechapterhead}[1]{%
  \vspace*{50\p@}%
  {\parindent \z@ \raggedright \normalfont
    \ifnum \c@secnumdepth >\m@ne
      \huge\bfseries \thechapter.\hspace{1em}\ignorespaces
    \fi
    \interlinepenalty\@M
    \Huge \bfseries #1\par\nobreak
    \vskip 40\p@
  }}
}
\makeatother

%------------------------------------------------------------------------
% Paquetes de cógido
\usepackage{minted}
\renewcommand\listingscaption{Código fuente}

\usepackage{fancyvrb}
% Personaliza el tamaño de los números de línea
\renewcommand{\theFancyVerbLine}{\small\arabic{FancyVerbLine}}

% Estilo para C++
\newminted{cpp}{
    frame=lines,
    framesep=2mm,
    baselinestretch=1.2,
    linenos,
    escapeinside=||
}

% para minted
\definecolor{LightGray}{rgb}{0.95,0.95,0.92}
\setminted{
    linenos=true,
    stepnumber=5,
    numberfirstline=true,
    autogobble,
    breaklines=true,
    breakautoindent=true,
    breaksymbolleft=,
    breaksymbolright=,
    breaksymbolindentleft=0pt,
    breaksymbolindentright=0pt,
    breaksymbolsepleft=0pt,
    breaksymbolsepright=0pt,
    fontsize=\footnotesize,
    bgcolor=LightGray,
    numbersep=10pt
}


\usepackage{listings} % Para incluir código desde un archivo

\renewcommand\lstlistingname{Código Fuente}
\renewcommand\lstlistlistingname{Índice de Códigos Fuente}

% Definir colores
\definecolor{vscodepurple}{rgb}{0.5,0,0.5}
\definecolor{vscodeblue}{rgb}{0,0,0.8}
\definecolor{vscodegreen}{rgb}{0,0.5,0}
\definecolor{vscodegray}{rgb}{0.5,0.5,0.5}
\definecolor{vscodebackground}{rgb}{0.97,0.97,0.97}
\definecolor{vscodelightgray}{rgb}{0.9,0.9,0.9}

% Configuración para el estilo de C similar a VSCode
\lstdefinestyle{vscode_C}{
  backgroundcolor=\color{vscodebackground},
  commentstyle=\color{vscodegreen},
  keywordstyle=\color{vscodeblue},
  numberstyle=\tiny\color{vscodegray},
  stringstyle=\color{vscodepurple},
  basicstyle=\scriptsize\ttfamily,
  breakatwhitespace=false,
  breaklines=true,
  captionpos=b,
  keepspaces=true,
  numbers=left,
  numbersep=5pt,
  showspaces=false,
  showstringspaces=false,
  showtabs=false,
  tabsize=2,
  frame=tb,
  framerule=0pt,
  aboveskip=10pt,
  belowskip=10pt,
  xleftmargin=10pt,
  xrightmargin=10pt,
  framexleftmargin=10pt,
  framexrightmargin=10pt,
  framesep=0pt,
  rulecolor=\color{vscodelightgray},
  backgroundcolor=\color{vscodebackground},
}

%------------------------------------------------------------------------

% Comandos definidos
\newcommand{\bb}[1]{\mathbb{#1}}
\newcommand{\cc}[1]{\mathcal{#1}}

% I prefer the slanted \leq
\let\oldleq\leq % save them in case they're every wanted
\let\oldgeq\geq
\renewcommand{\leq}{\leqslant}
\renewcommand{\geq}{\geqslant}

% Si y solo si
\newcommand{\sii}{\iff}

% Letras griegas
\newcommand{\eps}{\epsilon}
\newcommand{\veps}{\varepsilon}
\newcommand{\lm}{\lambda}

\newcommand{\ol}{\overline}
\newcommand{\ul}{\underline}
\newcommand{\wt}{\widetilde}
\newcommand{\wh}{\widehat}

\let\oldvec\vec
\renewcommand{\vec}{\overrightarrow}

% Derivadas parciales
\newcommand{\del}[2]{\frac{\partial #1}{\partial #2}}
\newcommand{\Del}[3]{\frac{\partial^{#1} #2}{\partial #3^{#1}}}
\newcommand{\deld}[2]{\dfrac{\partial #1}{\partial #2}}
\newcommand{\Deld}[3]{\dfrac{\partial^{#1} #2}{\partial #3^{#1}}}


\newcommand{\AstIg}{\stackrel{(\ast)}{=}}
\newcommand{\Hop}{\stackrel{L'H\hat{o}pital}{=}}

\newcommand{\red}[1]{{\color{red}#1}} % Para integrales, destacar los cambios.

% Método de integración
\newcommand{\MetInt}[2]{
    \left[\begin{array}{c}
        #1 \\ #2
    \end{array}\right]
}

% Declarar aplicaciones
% 1. Nombre aplicación
% 2. Dominio
% 3. Codominio
% 4. Variable
% 5. Imagen de la variable
\newcommand{\Func}[5]{
    \begin{equation*}
        \begin{array}{rrll}
            #1:& #2 & \longrightarrow & #3\\
               & #4 & \longmapsto & #5
        \end{array}
    \end{equation*}
}

%------------------------------------------------------------------------


%Definiciones para ahorrar trabajo
\def\R{\mathbb R}
\def\C{\mathbb C}
\def\N{\mathbb N}
\def\Z{\mathbb Z}
\def\Q{\mathbb Q}
\def\B{\mathcal{B}}
\def\A{\mathcal{A}}
\def\L{\mathcal{L}}
\def\y{\ \land \ }
\def\o{\ \lor \ }


\def\todoi{\forall i \in \{1, \ldots, n\}}
\def\todon{\ \forall n \in \N}

\usepackage{xlop}       % Para hacer cálculos básicos. Divisiones, etc.
\opset{mulsymbol={$\cdot$}}

\usetikzlibrary{matrix} % Para divisiones de polinomios.


\DeclareMathOperator{\mcd}{mcd}
\DeclareMathOperator{\mcm}{mcm}


\begin{document}

    % 1. Foto de fondo
    % 2. Título
    % 3. Encabezado Izquierdo
    % 4. Color de fondo
    % 5. Coord x del titulo
    % 6. Coord y del titulo
    % 7. Fecha
    % 8. Autor

    
    % 1. Foto de fondo
% 2. Título
% 3. Encabezado Izquierdo
% 4. Color de fondo
% 5. Coord x del titulo
% 6. Coord y del titulo
% 7. Fecha

\newcommand{\portada}[7]{

    \portadaBase{#1}{#2}{#3}{#4}{#5}{#6}{#7}
    \portadaBook{#1}{#2}{#3}{#4}{#5}{#6}{#7}
}

\newcommand{\portadaExamen}[7]{

    \portadaBase{#1}{#2}{#3}{#4}{#5}{#6}{#7}
    \portadaArticle{#1}{#2}{#3}{#4}{#5}{#6}{#7}
}




\newcommand{\portadaBase}[7]{

    % Tiene la portada principal y la licencia Creative Commons
    
    % 1. Foto de fondo
    % 2. Título
    % 3. Encabezado Izquierdo
    % 4. Color de fondo
    % 5. Coord x del titulo
    % 6. Coord y del titulo
    % 7. Fecha
    
    
    \thispagestyle{empty}               % Sin encabezado ni pie de página
    \newgeometry{margin=0cm}        % Márgenes nulos para la primera página
    
    
    % Encabezado
    \fancyhead[L]{\helv #3}
    \fancyhead[R]{\helv \nouppercase{\leftmark}}
    
    
    \pagecolor{#4}        % Color de fondo para la portada
    
    \begin{figure}[p]
        \centering
        \transparent{0.3}           % Opacidad del 30% para la imagen
        
        \includegraphics[width=\paperwidth, keepaspectratio]{assets/#1}
    
        \begin{tikzpicture}[remember picture, overlay]
            \node[anchor=north west, text=white, opacity=1, font=\fontsize{60}{90}\selectfont\bfseries\sffamily, align=left] at (#5, #6) {#2};
            
            \node[anchor=south east, text=white, opacity=1, font=\fontsize{12}{18}\selectfont\sffamily, align=right] at (9.7, 3) {\textbf{\href{https://losdeldgiim.github.io/}{Los Del DGIIM}}};
            
            \node[anchor=south east, text=white, opacity=1, font=\fontsize{12}{15}\selectfont\sffamily, align=right] at (9.7, 1.8) {Doble Grado en Ingeniería Informática y Matemáticas\\Universidad de Granada};
        \end{tikzpicture}
    \end{figure}
    
    
    \restoregeometry        % Restaurar márgenes normales para las páginas subsiguientes
    \pagecolor{white}       % Restaurar el color de página
    
    
    \newpage
    \thispagestyle{empty}               % Sin encabezado ni pie de página
    \begin{tikzpicture}[remember picture, overlay]
        \node[anchor=south west, inner sep=3cm] at (current page.south west) {
            \begin{minipage}{0.5\paperwidth}
                \href{https://creativecommons.org/licenses/by-nc-nd/4.0/}{
                    \includegraphics[height=2cm]{assets/Licencia.png}
                }\vspace{1cm}\\
                Esta obra está bajo una
                \href{https://creativecommons.org/licenses/by-nc-nd/4.0/}{
                    Licencia Creative Commons Atribución-NoComercial-SinDerivadas 4.0 Internacional (CC BY-NC-ND 4.0).
                }\\
    
                Eres libre de compartir y redistribuir el contenido de esta obra en cualquier medio o formato, siempre y cuando des el crédito adecuado a los autores originales y no persigas fines comerciales. 
            \end{minipage}
        };
    \end{tikzpicture}
    
    
    
    % 1. Foto de fondo
    % 2. Título
    % 3. Encabezado Izquierdo
    % 4. Color de fondo
    % 5. Coord x del titulo
    % 6. Coord y del titulo
    % 7. Fecha


}


\newcommand{\portadaBook}[7]{

    % 1. Foto de fondo
    % 2. Título
    % 3. Encabezado Izquierdo
    % 4. Color de fondo
    % 5. Coord x del titulo
    % 6. Coord y del titulo
    % 7. Fecha

    % Personaliza el formato del título
    \pretitle{\begin{center}\bfseries\fontsize{42}{56}\selectfont}
    \posttitle{\par\end{center}\vspace{2em}}
    
    % Personaliza el formato del autor
    \preauthor{\begin{center}\Large}
    \postauthor{\par\end{center}\vfill}
    
    % Personaliza el formato de la fecha
    \predate{\begin{center}\huge}
    \postdate{\par\end{center}\vspace{2em}}
    
    \title{#2}
    \author{\href{https://losdeldgiim.github.io/}{Los Del DGIIM}}
    \date{Granada, #7}
    \maketitle
    
    \tableofcontents
}




\newcommand{\portadaArticle}[7]{

    % 1. Foto de fondo
    % 2. Título
    % 3. Encabezado Izquierdo
    % 4. Color de fondo
    % 5. Coord x del titulo
    % 6. Coord y del titulo
    % 7. Fecha

    % Personaliza el formato del título
    \pretitle{\begin{center}\bfseries\fontsize{42}{56}\selectfont}
    \posttitle{\par\end{center}\vspace{2em}}
    
    % Personaliza el formato del autor
    \preauthor{\begin{center}\Large}
    \postauthor{\par\end{center}\vspace{3em}}
    
    % Personaliza el formato de la fecha
    \predate{\begin{center}\huge}
    \postdate{\par\end{center}\vspace{5em}}
    
    \title{#2}
    \author{\href{https://losdeldgiim.github.io/}{Los Del DGIIM}}
    \date{Granada, #7}
    \thispagestyle{empty}               % Sin encabezado ni pie de página
    \maketitle
    \vfill
}
    \portada{ffccA4.jpg}{Álgebra I}{Álgebra I}{MidnightBlue}{-8}{28}{2023}{José Juan Urrutia Milán ``JJ''\\Arturo Olivares Martos}
    
    \chapter{El lenguaje de los conjuntos}
En este primer tema, abordaremos un desarrollo sencillo de la teoría de conjuntos, basado en los axiomas de Zermelo-Fraenkel. El lector puede adentrarse en este campo gracias al libro \emph{Naive Set Theory}, de Paul Halmos, cuya lectura recomendamos. En este documento no haremos un desarrollo tan exhaustivo desde la axiomática por falta de tiempo. Es por tanto que daremos al principio algunas definiciones basadas en la intuición del matemático que inicia este curso.

\section{Conceptos básicos}

\begin{definicion}[Conjunto]
    Llamaremos \textbf{conjunto} a una colección de objetos (a los que también llamaremos \textbf{elementos})
    en la que no influye el orden.
\end{definicion}
\begin{notacion}
    Usualmente, notaremos a los conjuntos con letras mayúsculas y a los elementos con letras minúsculas, pudiendo haciendo uso incluso de letras griegas.
\end{notacion}

Si $A$ es un conjunto y $a$ es un elemento suyo, diremos que \textit{$a$ pertenece a $A$}, notado $a \in A$; mientras que si $a$ no es un elemento de $A$, diremos que \textit{$a$ no pertenece a $A$}, notado $a \notin A$.\\

A la hora de definir un conjunto, es posible hacerlo por \textit{extensión}, proporcionando todos sus elementos; o por \textit{comprensión}, proporcionando una regla que cumplan todos los elementos que pertenecen al conjunto. Por ejemplo, las siguientes definiciones son equivalentes:
\begin{gather*}
    X = \{0, 1, 2, 3, 4, 5\} \\
    X = \{x \mid x \in \N \y x < 6\}
\end{gather*}

Si $X$ es un conjunto finito con $n$ elementos $a_1, a_2, \ldots, a_n$, es habitual escribir: $$X = \{a_1, a_2, \ldots, a_n\} = \{a_i \mid 1 \leq i \leq n\} = \{a_i\}_{i = 1, \ldots, n}$$

\begin{definicion}[Cardinal]
    Al número $n$ de elementos de un conjunto le llamaremos \textbf{cardinal del conjunto}. Si $X$
    es un conjunto, notaremos a su cardinal por $|X|$ ó por $\#X$:
    $$|X| = \#X = n$$
\end{definicion}

Diremos que dos conjuntos $X$ e $Y$ son iguales (notado $X = Y$) si tienen los mismos elementos, ya que un conjunto está totalmente definido por sus elementos. Por otra parte, si $\exists \ x \in X$ tal que $x \notin Y$, o bien $\exists \ y \in Y$ tal que $y \notin X$, diremos que $X$ e $Y$ son distintos: $X \neq Y$.\\

Además, admitimos la existencia de un conjunto vacío (notado $\emptyset$), como aquel conjunto con cardinal 0 ($|\emptyset| = 0$). Es decir, $\emptyset$ no tiene elementos, luego nunca será posible encontrar un elemento $x$ de forma que\footnote{Esta observación no parece de mucha relevancia, pero gran número de demostraciones se basan en llegar a contradicción viendo que un elemento pertenece al conjunto vacío. Se dice que son demostraciones ``por vacuidad''.} $x\in \emptyset $.

\begin{definicion}[Subconjunto]
    Dados dos conjuntos $X$ e $Y$, diremos que \textbf{$X$ es un subconjunto de $Y$} si todo elemento de $X$ es también un elemento de $Y$. Es decir:
    $$\forall x \in X \Rightarrow x \in Y$$
    
    Lo notaremos como $X \subseteq Y$. En dicho caso, podremos decir también que \textbf{$X$ está contenido en $Y$}.
\end{definicion}

Algunas consecuencias inmediatas de fácil comprobación son:
\begin{itemize}
    \item $X = Y \Longleftrightarrow X \subseteq Y \y Y \subseteq X$.
    \item $\emptyset \subseteq X$ para todo conjunto $X$.
    \item $X \subseteq X$ para todo conjunto $X$.
\end{itemize}
\begin{notacion}
    La notación $X \subset Y$ es equivalente a la de $X \subseteq Y$.
\end{notacion}

\begin{definicion}[Subconjunto propio]
    Dado un conjunto $Y$, si $X \neq \emptyset$ es un conjunto tal que se tiene $X \subseteq Y \y X \neq Y$ diremos que $X$ es un subconjunto propio de $Y$.
    Es decir, $X$ es un subconjunto propio de $Y$ si:
    \begin{enumerate}
        \item $\forall x \in X \Rightarrow x \in Y$
        \item $\exists y \in Y \mid y \notin X$
    \end{enumerate}
    
    En dicho caso, lo notaremos por $X \subsetneq Y$.
\end{definicion}
Notemos que los únicos subconjuntos no propios de un conjunto $X$ son $X$ y $\emptyset $.

\begin{definicion}[Partes de un conjunto]
    Dado cualquier conjunto $X$, podremos formar un nuevo conjunto, que notaremos como $\cc{P}(X)$ y llamaremos \textbf{conjunto partes de $X$} ó \textbf{conjunto potencia de $X$} al conjunto cuyos elementos son cada uno de los posibles subconjuntos de $X$ que podamos formar:
    $$\cc{P}(X) = \{A \mid A \subseteq X\}$$
\end{definicion}
De la definición, se deduce que $\emptyset,X\in \cc{P}(X)$ para todo conjunto $X$.

\begin{ejemplo} Algunos ejemplos del conjunto de las partes de $X$ son:
\begin{enumerate}
    \item $X = \{1, 2, 3\}$
    $$\cc{P}(X) = \{\emptyset, \{1\}, \{2\}, \{3\}, \{1, 2\}, \{1, 3\}, \{2, 3\}, X\}$$

    \item $X=\emptyset$
    \begin{gather*}
        \cc{P}(\emptyset)=\{\emptyset\}\\
        \cc{P}[\cc{P}(\emptyset)] = \cc{P}[\{\emptyset\}] = \{\emptyset, \{\emptyset\}\}
    \end{gather*}
\end{enumerate}
\end{ejemplo}

Notemos que, dado un conjunto $X$, el conjunto $\mathcal{P}(X)$ es el primer ejemplo de conjunto que a su vez contiene a conjuntos. El alumno puede llegar a confundirse con qué notación usar en cada caso. El siguiente ejemplo muestra un caso básico de la notación que debemos usar al trabajar con distintos tipos de elementos matemáticos.

\begin{ejemplo}
    Si $X$ es un conjunto, $x$ es un elemento suyo ($x\in X$) y consideramos el conjunto partes de $X$, $\mathcal{P}(X)$. Podremos escribir:
    \begin{equation*}
        x\in X \qquad \{x\} \subseteq X \qquad \{x\} \in  \mathcal{P}(X) \qquad X \in \mathcal{P}(X)
    \end{equation*}
    Pero \textbf{no} podremos escribir:
    \begin{equation*}
        \{x\} \in X \qquad x \subseteq X \qquad \{x\} \subseteq \mathcal{P}(X) \qquad X \subseteq \mathcal{P}(X)
    \end{equation*}
\end{ejemplo}

Durante la carrera de matemáticas se verán numerosos ejemplos de conjuntos que a su vez contienen a conjuntos (y dichos conjuntos quizás contendrán otros conjuntos), basta considerar el conjunto $\mathcal{P}(\mathcal{P}(X))$ para cualquier conjunto $X$. 

Será usual denotar por ``familia'' a los conjuntos cuyos elementos son a su vez conjuntos. Como notación para estos, se suelen usar letras estilográficas ($\cc{A}, \cc{B}, \ldots$), o en ocasiones por letras griegas mayúsculas, aunque siempre podremos saber la naturaleza del conjunto gracias a cómo esté definido.

\begin{definicion}[Intersección]
    Sea $X$ un conjunto y sean $A, B \in \cc{P}(X)$, definimos la \textbf{intersección} de $A$ y de $B$, notado
    $A \cap B$ como el subconjunto de $X$ formado por aquellos elementos que pertenecen simultáneamente
    a $A$ y a $B$:
    $$A \cap B = \{x \in X \mid x \in A \y x \in B\}$$
\end{definicion}

\begin{definicion}[Unión]
    Sea $X$ un conjunto y sean $A, B \in \cc{P}(X)$, definimos la \textbf{unión} de $A$ y de $B$, notado
    $A \cup B$ como el subconjunto de $X$ formado por aquellos elementos que pertenecen a $A$ o a $B$:
    $$A \cup B = \{x \in X \mid x \in A \o x \in B\}$$
\end{definicion}

Cuando el conjunto $X$ esté claro por el contexto podremos mencionar simplemente la intersección o unión de dos conjuntos, sin determinar de forma explícita el conjunto $X$ del que ambos son subconjuntos.

\begin{definicion}[Disjuntos]
    Sea $X$ un conjunto y sean $A, B \in \cc{P}(X)$, diremos que \textbf{$A$ y $B$ son disjuntos} si
    $A \cap B = \emptyset$.
\end{definicion}

\begin{prop}
    Sea $X$ un conjunto y $A, B, C \in \cc{P}(X)$. Algunas de las propiedades que se verifican sobre conjuntos son:
    \begin{enumerate}
        \item Propiedad conmutativa:
        $$A \cap B = B \cap A \quad;\quad A \cup B = B \cup A$$
        \item Propiedad asociativa:
        $$A \cap (B \cap C) = (A \cap B) \cap C \quad;\quad A \cup (B \cup C) = (A \cup B) \cup C$$
        \item Propiedad de la idempotencia:
        $$A \cap A = A \quad;\quad A \cup A = A$$
        \item Propiedad distributiva:
        $$A \cap (B \cup C) = (A \cap B) \cup (A \cap C) \quad;\quad A \cup (B \cap C) = (A \cup B) \cap (A \cup C)$$
    \end{enumerate}
\end{prop}
\begin{proof}
    Demostramos cada una de las propiedades por separado:
    \begin{enumerate}
        \item Propiedad conmutativa:
        \begin{gather*}
            A \cap B = \{x \in X \mid x \in A \y x \in B \} = \{x \in X \mid x \in B \y x \in A \} = B \cap A \\
            A \cup B = \{x \in X \mid x \in A \o x \in B\} = \{x \in X \mid x \in B \o x \in A\} = B \cup A
        \end{gather*}
        \item Propiedad asociativa:
        \begin{equation*}
            \begin{split}
                A \cap (B \cap C) &= A \cap \{x \in X \mid x \in B \y x \in C \} =\\
                & =\{x \in X \mid x \in A \y x \in B \y x \in C\} = \\
                & =\{x \in X \mid x \in A \y x \in B\} \cap C =\\
                & = (A \cap B) \cap C
            \end{split}
        \end{equation*}
        \begin{equation*}
            \begin{split}
                A \cup (B \cup C) &= A \cup \{x \in X \mid x \in B \o x \in C \} =\\
                & =\{x \in X \mid x \in A \o x \in B \o x \in C\} =\\
                & =\{x \in X \mid x \in A \o x \in B\} \cup C =\\
                & =(A \cup B) \cup C
            \end{split}
        \end{equation*}
        
        \item Propiedad de la idempotencia:
        \begin{gather*}
            A \cap A = \{x \in X \mid x \in A \y x \in A\} = \{x \in X \mid x \in A\} = A \\
            A \cup A = \{x \in X \mid x \in A \o x \in A\} = \{x \in X \mid x \in A\} = A
        \end{gather*}
        \item Propiedad distributiva:
        \begin{equation*}
            \begin{split}
                A \cap (B \cup C) &= A \cap \{x \in X \mid x \in B \o x \in C\} =\\
                & =\{x \in X \mid x \in A \y (x \in B \o x \in C)\} =\\
                & =\{x \in X \mid (x \in A \y x \in B) \o (x \in A \y x \in C)\} =\\
                & = \{x \in X \mid x \in A \y x \in B\} \cup \{x \in X \mid x \in A \y x \in B\} =\\
                & = (A \cap B) \cup (A \cap C)
            \end{split}
        \end{equation*}
        \begin{equation*}
            \begin{split}
                A \cup (B \cap C) &= A \cup \{x \in X \mid x \in B \y x \in C\} =\\
                & =\{x \in X \mid x \in A \o (x \in B \y x \in C)\} =\\
                & =\{x \in X \mid (x \in A \o x \in B) \y (x \in A \o x \in C)\} =\\
                &= \{x \in X \mid x \in A \o x \in B\} \cap \{x \in X \mid x \in A \o x \in B\} =\\
                &= (A \cup B) \cap (A \cup C)
            \end{split}
        \end{equation*}
        \qedhere
    \end{enumerate}
\end{proof}

\begin{definicion}[Uniones e intersecciones generalizadas]
    Sea $X$ un conjunto, y consideramos $\Gamma \subseteq \cc{P}(X)$ una familia de subconjuntos de $X$. Definimos la unión y la intersección de todos los elementos de $\Gamma$ por:
    \begin{gather*}
        \bigcap_{A \in \Gamma} A = \{x \in X \mid x \in A~~\forall A \in \Gamma \}\\
        \bigcup_{A \in \Gamma} A = \{x \in X \mid \exists A \in \Gamma\ \text{tal que}\ x \in A \}
    \end{gather*}

    A veces, simplemente lo notaremos por:
    \begin{gather*}
        \bigcap \Gamma := \bigcap_{A\in \Gamma}A \\
        \bigcup \Gamma := \bigcup_{A\in \Gamma}A
    \end{gather*}

    Notemos que si $\Gamma$ es una familia finita: $\Gamma = \{A_1, A_2, \ldots, A_n\} \subseteq \cc{P}(X)$, entonces:
    \begin{equation*}
        \bigcap_{A \in \Gamma}A = A_1 \cap A_2 \cap \ldots \cap A_n
        \hspace{1cm}
        \bigcup_{A \in \Gamma}A = A_1 \cup A_2 \cup \ldots \cup A_n
    \end{equation*}
    
    En el caso anterior, podemos notar:
    \begin{equation*}
        \bigcap_{A \in \Gamma}A = \bigcap_{i=1}^n A_i
        \hspace{1cm}
        \bigcup_{A \in \Gamma}A = \bigcup_{i=1}^n A_i
    \end{equation*}
\end{definicion}

\begin{ejemplo}
Sea $X = \{0, 1, 2, 3, 4, 5\}$, y consideramos ${\Gamma = \{\{0, 1\}, \{1, 2\}, \{1, 3, 5\}\}\subseteq \mathcal{P}(X)}$:
    \begin{equation*}
        \bigcap_{A \in \Gamma}A = \{1\}
        \hspace{1cm}
        \bigcup_{A \in \Gamma}A = \{0, 1, 2, 3, 5\}
    \end{equation*}
\end{ejemplo}

\begin{definicion}[Complementario]
    Sea $X$ un conjunto y $A \in \cc{P}(X)$, definimos el \textbf{complementario de $A$ en $X$}, notado $X-A$ o $X\setminus A$, como el subconjunto de $X$ formado por aquellos elementos de $X$ que no pertenezcan a $A$:
    $$X-A = \{x \in X \mid x \notin A\}$$
\end{definicion}
\begin{notacion}
    Cuando el conjunto $X$ sea claro por el contexto (por ejemplo, cuando estemos trabajando continuamente con números reales), notaremos simplemente $\overline{A}$ o $C(A)$ (que será equivalente a escribir $X-A$).
\end{notacion}

\begin{ejemplo}
    Sea $A = \{x \in \N \mid x \geq 4\} \subseteq \N$:
    \begin{align*}
        \N - A &= \{0, 1, 2, 3\} = \{x \in \N \mid x < 4\}\\
        \Z - A &= \{x \in \Z \mid x < 4\}
    \end{align*}
\end{ejemplo}

\begin{prop} Sea $X$ un conjunto y $A \in \cc{P}(X)$. Algunas propiedades que se verifican sobre el complementario son:
    \begin{enumerate}
        \item $C(\emptyset) = X$.
        \item $C(X) = \emptyset$.
        \item $A \cup C(A) = X$.
        \item $A \cap C(A) = \emptyset$.
        \item $C(C(A)) = A$.
    \end{enumerate}
\end{prop}
\begin{proof}
    Demostramos cada una de las propiedades por separado:
    \begin{enumerate}
        \item $C(\emptyset) = \{x \in X \mid x \notin \emptyset\} = \{x \in X\} = X$.
        \item $C(X) = \{x \in X \mid x \notin X\} = \emptyset$.
        \item $A \cup C(A) = \{x \in X \mid x \in A \o x \notin A \} = \{x \in X\} = X$.
        \item $A \cap C(A) = \{x \in X \mid x \in A \y x \notin A\} = \emptyset$.
        \item $C(C(A)) = \{x \in X \mid x \notin C(A) \} = \{x \in X \mid x \in A\} = A$.
    \end{enumerate}
\end{proof}

\begin{prop}[Leyes de De Morgan]
    Sea $X$ un conjunto con $A, B \in \cc{P}(X)$, se verifica que:
    \begin{enumerate}
        \item $C(A \cup B) = C(A) \cap C(B)$
        \item $C(A \cap B) = C(A) \cup C(B)$
    \end{enumerate}
\end{prop}
\begin{proof} Demostramos cada una de las igualdades:
\begin{enumerate}
    \item $C(A \cup B) = C(A) \cap C(B)$:
    \begin{equation*}
        \begin{split}
            C(A \cup B) &= \{x \in X \mid x \notin (A \cup B)\} =\\
            & =\{x \in X \mid x \notin \{x \in X \mid x \in A \o x \in B\} \} =\\
            & = \{x \in X \mid x \notin A \y x \notin B \} =\\
            & =\{x \in X \mid x \notin A\} \cap \{x \in X \mid x \notin B\} = C(A) \cap C(B)
        \end{split}
    \end{equation*}

    \item $C(A \cap B) = C(A) \cup C(B)$:
    \begin{equation*}
        \begin{split}
            C(A \cap B) &= \{x \in X \mid x \notin (A \cap B)\} =\\
            & =\{x \in X \mid x \notin \{x \in X \mid x \in A \y x \in B\} \} =\\
            & = \{x \in X \mid x \notin A \o x \notin B \} =\\
            & =\{x \in X \mid x \notin A\} \cup \{x \in X \mid x \notin B\} = C(A) \cup C(B)
        \end{split}
    \end{equation*}
\end{enumerate}
\end{proof}

\begin{prop}[Leyes de De Morgan generalizadas]
    Sea $X$ un conjunto, $\Gamma \subseteq~\cc{P}(X)$, se verifica:
    \begin{enumerate}
        \item El complementario de la unión es la intersección de los complementarios.
        $$C\left( \bigcup_{A\in\Gamma}A \right) = \bigcap_{A\in\Gamma}C(A)$$

        \item El complementario de la intersección es la unión de los complementarios.
        $$C\left( \bigcap_{A\in\Gamma}A \right) = \bigcup_{A\in\Gamma}C(A)$$
    \end{enumerate}
\end{prop}
\begin{proof} Demostramos cada igualdad por separado:
\begin{enumerate}
    \item $C\left( \bigcup\limits_{A\in\Gamma}A \right) = \left\{x \in X \mid x \notin \bigcup\limits_{A\in\Gamma}A\right\} = \{x \in X \mid x \notin A ~\forall A \in \Gamma \} = \bigcap\limits_{A\in\Gamma}C(A)$

    \item $C\left( \bigcap\limits_{A\in\Gamma}A \right) = \left\{ x \in X \mid x \notin \bigcap\limits_{A\in\Gamma}A \right\} = \{x \in X \mid \exists A \in \Gamma \mid x \notin A\} = \bigcup\limits_{A\in\Gamma}C(A)$
\end{enumerate}
\end{proof}

\begin{definicion}[Complementario generalizado]
    Sea $X$ un conjunto y consideramos $A, B \in \cc{P}(X)$, definimos \textbf{el complementario de $A$ en $B$}, notado
    $B-A$ como el conjunto:
    $$B-A = \{x \in X \mid x \in B \y x \notin A\} = B \cap C(A)$$
\end{definicion}

\begin{prop}[Propiedad distributiva generalizada]
    Sea $X$ un conjunto con $B \in \cc{P}(X)$ y $\Gamma \subseteq \cc{P}(X)$, se tiene que:
    \begin{equation*}
        B \cap \left( \bigcup_{A \in \Gamma}A \right) = \bigcup_{A\in \Gamma} (B \cap A)
        \hspace{1cm}
        B \cup \left( \bigcap_{A \in \Gamma}A \right) = \bigcap_{A\in \Gamma} (B \cup A)
    \end{equation*}
\end{prop}
\begin{proof}
    \begin{equation*}
        \begin{split}
            B \cap \left( \bigcup_{A \in \Gamma}A \right) &= \left\{ x \in X \mid x \in B \y x \in \bigcup_{A \in \Gamma}A \right\} =\\
            & = \{x \in X \mid x \in B \y \exists A \in \Gamma \mid x \in A\} =\\
            & =\{x \in X \mid \exists A \in \Gamma \mid x \in B \y x \in A\}= \bigcup_{A\in \Gamma} (B \cap A)
        \end{split}
    \end{equation*}

    \begin{equation*}
        \begin{split}
            B \cup \left( \bigcap_{A \in \Gamma}A \right) &= \left\{ x \in X \mid x \in B \o x \in \bigcap_{A \in \Gamma}A \right\} =\\
            & = \{x \in X \mid x \in B \o x \in A~\forall A \in \Gamma\} =\\
            & = \{x \in X \mid \forall A \in \Gamma~x \in B \o x \in A\} = \bigcap_{A\in \Gamma} (B \cup A)
        \end{split}
    \end{equation*}
\end{proof}


\section{Álgebra de proposiciones}
\begin{definicion}[Conjunto que verifica una propiedad]
    Sea $X$ un conjunto y sea $P$ una propiedad referida a los elementos de dicho conjunto, definimos el conjunto de elementos de $X$ que verifica dicha propiedad, que usualmente notaremos por $X_P$, como:
    $$X_P = \{x \in X \mid x \mbox{ verifica } P \}$$
\end{definicion}
\begin{ejemplo}
    Sea $X=\bb{Z}$, y sea $P$ la propiedad de ser un número positivo. Entonces:
    \begin{equation*}
        X_P = \{x\in \bb{Z}\mid x\geq 0\} = \bb{N}
    \end{equation*}
\end{ejemplo}

\begin{prop}
    Sea $X$ un conjunto y sean $P$ y $Q$ dos propiedades referidas a dicho conjunto. Es posible calcular el conjunto de elementos de $X$ que verifican $P$ y $Q$ simultáneamente o el conjunto de elementos de $X$ que verifica al menos una propiedad a partir de la fórmula:
    \begin{gather*}
        X_{(P \y Q)} = X_P \cap X_Q \\
        X_{(P \o Q)} = X_P \cup X_Q
    \end{gather*}
\end{prop}
\begin{proof} Trivialmente, se verifica lo siguiente:
    \begin{gather*}
        X_P \cap X_Q = \{x \in X \mid x \mbox{ verifica } P \y x \mbox{ verifica } Q\} = X_{(P \y Q)} \\
        X_P \cup X_Q = \{x \in X \mid x \mbox{ verifica } P \o x \mbox{ verifica } Q\} = X_{(P \o Q)}
    \end{gather*}
\end{proof}

\begin{prop}
    Sea $X$ un conjunto y sea $P$ una propiedad referida a dicho conjunto, podemos calcular el conjunto de elementos de $X$ que no cumplen la propiedad $P$ a partir de $X_P$, de la forma:
    $$X_{\lnot P} = C(X_P)$$
\end{prop}
\begin{proof}
    $$C(X_P) = \{x \in X \mid x \notin X_P \} = \{x \in X \mid x \mbox{ no verifica } P\} = X_{\lnot P}$$
\end{proof}

\begin{definicion}[Proposición matemática]
    Una \textbf{proposición matemática} es una relación entre dos propiedades $P$ y $Q$ referidas a los elementos de un conjunto $X$ del tipo $$P \Longrightarrow Q$$
    
    Se lee ``$P$ implica $Q$'' o ``$P$ entonces $Q$'', y significa que si $x\in X$ verifica $P$, entonces también verifica $Q$. Equivalentemente, ha de ser $X_P\subseteq X_Q$.
\end{definicion}

Demostrar la falsedad de la proposición matemática $P \Longrightarrow Q$ es equivalente a demostrar que $X_p\not \subseteq X_Q$. Es decir, ver que $\exists x\in X_P$ tal que $x\notin X_Q$. A dicho elemento $x$ se le llama \textbf{contraejemplo}.

\begin{definicion}[Recíproco]
    Dada una proposición matemática $P\Longrightarrow Q$, definimos su \textbf{proposición matemática recíproca}, o \textbf{recíproco} como la proposición matemática:
    \begin{equation*}
        Q\Longrightarrow P
    \end{equation*}
\end{definicion}

\begin{observacion}
    Dada una proposición matemática, su recíproco no siempre es verdadero. Por ejemplo, es cierto que todo número natural es un número entero ($\mathbb{N} \subseteq \mathbb{Z}$) pero su recíproco, que todo número entero es un número natural, no es cierto ($\mathbb{Z} \nsubseteq \mathbb{N}$).
\end{observacion}

\begin{definicion}[Contrarrecíproco]
    Dada una proposición matemática $P\Longrightarrow Q$, definimos su \textbf{proposición matemática contrarrecíproca}, o \textbf{contrarrecíproco} como la proposición matemática:
    \begin{equation*}
        \lnot Q \Longrightarrow \lnot P
    \end{equation*}
\end{definicion}

\begin{prop}[Transitividad]
    Sean $P$, $Q$, $R$ propiedades referidas a los elementos de un conjunto $X$, tales que $P \Longrightarrow Q$
    y $Q \Longrightarrow R$. Entonces:
    $$P \Longrightarrow R$$
\end{prop}
\begin{proof}
    Se demuestra gracias a la transitividad de la inclusión de los subconjuntos, ya que $X_P\subseteq X_Q\subseteq X_R$, por lo que $X_P\subseteq X_R$.
\end{proof}

\begin{definicion}[Equivalencia]
    Sea $X$ un conjunto y $P$ y $Q$ propiedades referidas a sus elementos, diremos que \textbf{$P$ y $Q$ son equivalentes}, notado $P \Longleftrightarrow Q$ y leído ``$P$ si y solo si $Q$'', si:
    $$P \Longrightarrow Q \quad\land\quad Q \Longrightarrow P$$
    Notemos que la equivalencia se da cuando tanto una proposición matemática como su proposición recíproca son ciertas.
\end{definicion}

\begin{prop}[Equivalencia generalizada]
    Sea $X$ un conjunto y $P_1$, $P_2$, \ldots, $P_n$ propiedades referidas a elementos de $X$ tales que $P_i \Longrightarrow P_{i+1}~\forall i \in \{1, \ldots, n-1\}$ y que $P_n \Longrightarrow P_1$. Entonces:
    $$P_i \Longleftrightarrow P_j~~\forall i, j \in \{1, \ldots, n\}$$
\end{prop}
\begin{proof}
    $\forall i,j \in \{1, \ldots, n\}$:
    \begin{itemize}
        \item \underline{Si $i = j$}:

        Se tiene $P_i \Leftrightarrow P_j$ trivialmente, ya que:
        \begin{equation*}
            X_{P_i} = X_{P_j} \Rightarrow \left\{
            \begin{array}{ccc}
                X_{P_i} \subseteq X_{P_j} & \Longrightarrow & P_i \Rightarrow P_j \\
                \y && \y\\
                X_{P_j} \subseteq X_{P_i} & \Longrightarrow & P_j \Rightarrow P_i
            \end{array}
            \right.
        \end{equation*}
        Por tanto, $P_i \Leftrightarrow P_j$.

        \item \underline{Si $i < j$}: $(P_i \Rightarrow P_{i+1} \Rightarrow \ldots \Rightarrow P_j) \Rightarrow (P_i \Rightarrow P_j)$

        \item \underline{Si $i > j$}: $(P_i \Rightarrow P_n \Rightarrow P_1 \Rightarrow P_j) \Rightarrow (P_i \Rightarrow P_j)$
    \end{itemize}
    
    Para la implicación $P_j \Rightarrow P_i$ hágase un camino similar al especificado y se obtendrá $P_j \Leftrightarrow P_i$.
\end{proof}
De esta forma, siempre que queramos probar que un conjunto finito de propiedades matemáticas son equivalentes entre sí, bastará probar que la primera es equivalente a la segunda, la segunda a la tercera, y así hasta que la penúltima es equivalente a la última y finalmente que la última es equivalente a la primera.

\subsubsection{Demostración por reducción al absurdo}

Sea $X$ un conjunto, $P$ y $Q$ propiedades referidas a los elementos de dicho conjunto. Queremos demostrar que $P \Rightarrow Q$. El procedimiento es el siguiente:

\begin{itemize}
    \item Supongamos que $\exists x \in X \mid x \in X_P\cap C(X_Q)$, es decir, que $\exists x\in X$ que verifica $P$ pero no $Q$.
    
    \item Si llegamos a una resultado que es falso o que contradice nuestra hipótesis ($x\in C(X_P)$), habremos llegado a una contradicción y podemos concluir que $\forall x \in X_P \Longrightarrow x \in X_Q$. Es decir, queda demostrado que $P \Longrightarrow Q$.
\end{itemize}

\subsubsection{Demostración por contrarrecíproco}

\begin{lema}\label{lema:1.10}
    Sea $X$ un conjunto y $A, B \in \cc{P}(X)$. Entonces:
    $$A \subseteq B \Longleftrightarrow C(B) \subseteq C(A)$$
\end{lema}
\begin{proof} Procedemos mediante doble implicación:
\begin{description}
    \item [$\Longrightarrow)$] Sea $x \in C(B)$ y supongamos $x \notin C(A)$, luego $x \notin B \y x \in A$. Como $A\subseteq B$, tenemos que $x\in B$, por lo que llegamos a una \underline{contradicción}. Por tanto, se tiene que $x\in C(A)$ y, por tanto, $C(B)\subseteq C(A)$.

    \item [$\Longleftarrow)$] Usando la otra implicación (ya demostrada), tenemos que $$A = C(C(A)) \subseteq C(C(B)) = B$$
\end{description}
\end{proof}

\begin{prop}[Demostración por contrarrecíproco]
    Sea $X$ un conjunto, $P$ y $Q$ propiedades referidas a sus elementos, son equivalentes:
    \begin{enumerate}
        \item $P \Rightarrow Q$ (Demostración directa).
        \item $\neg Q \Rightarrow \neg P$ (Demostración por contrarrecíproco).
    \end{enumerate}
    Es decir, dada una proposición matemática, será verdadera si y solo si lo es su proposición matemática contrarrecíproca.
\end{prop}
\begin{proof}
\begin{equation*}
    (P \Rightarrow Q) \Leftrightarrow X_P \subseteq X_Q
    \stackrel{(\ast)}{\Longleftrightarrow}
    C(X_Q) \subseteq C(X_P) \Longleftrightarrow (\neg Q \Rightarrow \neg P)
\end{equation*}
donde en $(\ast)$ he aplicado el lema anterior, el Lema \ref{lema:1.10}.
\end{proof}

\section{Aplicaciones}

\begin{definicion}[Par ordenado]
    Un par ordenado es un conjunto que contiene a dos elementos $a$ y $b$, notado $(a,b)$ en el que importa el orden. Es decir, si $(a,b)$ y $(c,d)$ son dos pares ordenados:
    $$(a,b) = (c,d) \Longleftrightarrow a = c \y b = d$$
\end{definicion}

\begin{definicion}[Terna]
    Una terna es un conjunto de tres elementos $a$, $b$, $c$ en el que importa el orden, notado por:
    \begin{equation*}
        (a,b,c)
    \end{equation*}
\end{definicion}

\begin{definicion}[$n$-upla]
    Dado un número natural $n$, podemos generalizar el concepto de par ordenado o de terna a una $n$-upla, que es un conjunto de $n$ elementos $a_1$, $a_2$, \ldots, $a_n$ en el que importa el orden. A este lo notaremos por:
    \begin{equation*}
        (a_1,a_2,\ldots,a_n)
    \end{equation*}
\end{definicion}

\begin{definicion}[Producto Cartesiano]
    Sean $X$ e $Y$ dos conjuntos, definimos el \textbf{producto cartesiano de $X$ e $Y$} como el conjunto:
    $$X\times Y = \{(x,y) \mid x \in X \y y \in Y\}$$
\end{definicion}

Por lo general, se tiene que $X \times Y \neq Y \times X$ salvo que $X = Y$.

\begin{definicion}[Producto Cartesiano generalizado]
    Sean $X_1$, $X_2$, \ldots, $X_n$ conjuntos, definimos el \textbf{producto cartesiano de $X_1$, $X_2$,
        \ldots, $X_n$} como el conjunto:
    $$X_1 \times X_2 \times \ldots \times X_n = \{(x_1, x_2, \ldots, x_n) \mid x_i \in X_i ~~\forall i=1,\ldots,n \}$$
\end{definicion}
\begin{notacion}
    A veces notaremos: $\prod\limits_{i=1}^n X_i := X_1 \times X_2 \times \ldots \times X_n $.

    En el caso en el que $X=X_1 = X_2 = \ldots = X_n$, notaremos $\prod\limits_{i=1}^n X_i := X^n$.
\end{notacion}

\begin{ejemplo}
    Sea $X=\{a,b\}$ e $Y=\{1,2,3\}$. Entonces:
    \begin{gather*}
        X\times Y = \{(a,1), (a,2), (a,3), (b,1), (b,2), (b,3)\} \\
        Y\times X = \{(1,a), (1,b), (2,a), (2,b), (3,a), (3,b)\}
    \end{gather*}
\end{ejemplo}

\begin{prop}
    Si $X$ e $Y$ son dos conjuntos finitos, entonces $X\times Y$ es finito, con:
    \begin{equation*}
        |X\times Y|=|X|~|Y|
    \end{equation*}
\end{prop}
\begin{proof}
    Para cada elemento de $X$, tenemos que hay $|Y|$ opciones disponibles para completar el par ordenado. Como hay $|X|$ elementos en $X$, tenemos que en total hay $|X|~|Y|$ pares ordenados.
\end{proof}


\begin{definicion}[Aplicación]
    Una \textbf{aplicación} es una terna $(X, Y, f)$ donde $X$ es un conjunto llamado \textbf{dominio de la aplicación}, $Y$ es otro conjunto \textbf{llamado recorrido, rango o codominio de la aplicación} y $f\subseteq X\times Y$ es un conjunto llamado \textbf{grafo de la aplicación}. Esta terna ha de cumplir las siguientes propiedades:
    \begin{enumerate}
        \item $\forall x \in X,~\exists y \in Y \mid (x, y) \in f$.
        \item $\forall (x,y), (x',y') \in f$, si $x=x'\Longrightarrow y=y'$.
    \end{enumerate}
\end{definicion}

Las dos propiedades anteriores son equivalentes a que:
$$\forall x \in X~\exists_1 y \in Y \mid (x,y) \in f$$

Al único elemento $y \in Y$ que corresponde a un elemento $x \in X$ le llamaremos imagen por $f$ de $x$ (o simplemente $f$ de $x$), notado $y:=f(x)$. A veces, a dicho elemento $x$ tal que $f(x)=y$ lo llamaremos antiimagen de $y$.\\

Cuando tengamos una aplicación (es decir, una terna $(X, Y, f)$), hablaremos de una aplicación $f$ de $X$ en $Y$, notado de algunas de las siguientes formas:
\begin{equation*}
    f:X\longrightarrow Y
    \hspace{1cm}
    X \stackrel{f}{\longrightarrow} Y
\end{equation*}

Dar una aplicación es dar su dominio, su recorrido y el conjunto de pares ordenados; que es equivalente a dar el dominio, el recorrido y especificar a qué elemento del recorrido le corresponde cada elemento del dominio, que suele ser usual hacerlo mediante una fórmula. Por tanto, dos aplicaciones son iguales si tienen el mismo dominio, recorrido y grafo.

\begin{ejemplo} Algunos ejemplos de la definición anterior son:
\begin{enumerate}
    \item No existe la aplicación $f:\N \rightarrow \N$ dada por $f(x) = x-1$, ya que no cumple con la primera condición: $f(0)=-1\notin\N$.

    \item No existe la aplicación $g:\bb{N}\longrightarrow \bb{N}$ definida por la fórmula
    \begin{equation*}
        g(x)=\left\{
        \begin{array}{ccl}
            x & \text{si} & x \text{ no es múltiplo ni de $2$ ni de $3$} \\ \\
            \dfrac{x}{2} & \text{si} & x \text{ es múltiplo de $2$} \\ \\
            \dfrac{x}{3} & \text{si} & x \text{ es múltiplo de $3$}
        \end{array}
        \right.
    \end{equation*}
    Esto se debe a que $6$ podría tener dos imágenes, por lo que no cumpliría la segunda condición:
    \begin{equation*}
        g(6)=\frac{6}{2}=3
        \hspace{1cm}
        g(6)=\frac{6}{3}=2
    \end{equation*}

    \item La fórmula $f(x) = \dfrac{x^2+1}{x-1}$ define una aplicación $f:\left]0,1\right[ \rightarrow \R$ pero no puede definir una aplicación $f:[0,1] \rightarrow \R$, ya que $\nexists f(1)$.

    \item La suma de naturales $+:\bb{N}\times \bb{N}\longrightarrow \bb{N}$ dada por $+(x,y) = x+y$ es una aplicación.
\end{enumerate}
\end{ejemplo}


\begin{definicion}[Imagen de una aplicación]
    Si $f:X \rightarrow Y$ es una aplicación, al conjunto de las imágenes de los elementos de $X$ lo
    llamaremos \textbf{conjunto imagen de la aplicación}, notado $Img(f)$:
    $$Img(f) = \{f(x) \mid x \in X\} \subseteq Y$$
\end{definicion}

\begin{definicion}[Sobreyectividad]
    Dada una aplicación $f:X \rightarrow Y$, diremos que \textbf{$f$ es sobreyectiva} si $Img(f)~=~Y$. Es decir, se ha de cumplir que:
    $$\forall y \in Y ~\exists x \in X \mid f(x) = y$$
\end{definicion}

\begin{definicion}[Inyectividad]
    Dada una aplicación $f:X \rightarrow Y$, diremos que \textbf{$f$ es inyectiva} si elementos distintos tienen imágenes distintas. Es decir, se ha de cumplir:
    $$\forall x,z \in X \mid x \neq z \Longrightarrow f(x) \neq f(z)$$
    
    Por contrarrecíproco\footnote{Suele ser la forma más fácil de probar que una aplicación es inyectiva, mediante el contrarrecíproco de la definición.}, $f$ es inyectiva si $\forall x,z \in X \mid f(x) = f(z) \Longrightarrow x = z$.
\end{definicion}

\begin{definicion}[Biyectividad]
    Dada una aplicación $f:X \rightarrow Y$, diremos que \textbf{$f$ es biyectiva} si es a la vez inyectiva y sobreyectiva.
\end{definicion}

\begin{definicion}[Conjuntos biyectivos]
    Sean $X$ e $Y$ dos conjuntos, diremos que son biyectivos, notado $X \cong Y$ ó $\displaystyle X \mathop{\cong}^{f} Y$
    si existe una aplicación $f:X \rightarrow Y$ biyectiva.
\end{definicion}

\begin{ejemplo}
    Algunos ejemplos de inyectividad, sobreyectividad y biyectividad son:
    \begin{enumerate}
        \item $f:\bb{Z}\to\bb{Z}$, $f(x)=x^2$ no es sobreyectiva ni inyectiva.
        \item $g:\bb{Z}\to\bb{Z}$, $g(x)=2x$ es inyectiva pero no sobreyectiva.
        \item $h:\bb{Z}\to\bb{N}$, $h(x)=|x|$ es sobreyectiva pero no inyectiva.
        \item $t:\bb{Z}\to\bb{Z}$, $t(x)=x+2$ es biyectiva.
    \end{enumerate}
\end{ejemplo}

\begin{definicion}[Aplicación identidad]
    Sea $X$ un conjunto, definimos la aplicación \textbf{identidad en $X$}; notada como $id_X,~I_X,~Id_X$, o $1_X$; como la
    siguiente aplicación:
    \Func{id_X}{X}{X}{x}{id_X(x)=x}
\end{definicion}

\begin{definicion}[Composición]
    Sean $f:X\rightarrow Y$ y $g:Y \rightarrow Z$ dos aplicaciones, definimos la aplicación \textbf{$g$ compuesta con $f$}, notada $g \circ f$, como la siguiente aplicación:
    \Func{g\circ f}{X}{Z}{x}{g(f(x))}
\end{definicion}

Algunas propiedades de la composición de aplicaciones son:
\begin{prop}\label{prop:CompAsoc}
    La composición es asociativa. Es decir, dadas las aplicaciones $\displaystyle X \mathop{\longrightarrow}^{f} Y \mathop{\longrightarrow}^{g} Z \mathop{\longrightarrow}^{h} T$, se cumple:
    $$f \circ (g \circ h) = (f \circ g) \circ h$$
\end{prop}
\begin{proof}
    Los dominios de ambas aplicaciones son $X$ y los codominios $T$. Falta comprobar que los grafos coinciden. Para todo $x \in X$, se cumple que:
    \begin{gather*}
        (f \circ (g \circ h))(x) = f[(g\circ h)(x)] = f[g(h(x))]\\
        ((f \circ g) \circ h)(x) = (f\circ g)(h(x)) = f[g(h(x))]
    \end{gather*}
    Por tanto, se tiene $f \circ (g \circ h) = (f \circ g) \circ h$.
\end{proof}

\begin{prop}
    Dada una aplicación $f:X \rightarrow Y$ arbitraria, se verifica que la identidad es el elemento neutro de la composición. Es decir,
    \begin{gather*}
        f \circ id_X = f\\
        id_Y \circ f = f
    \end{gather*}
\end{prop}
\begin{proof}
    Los dominios y codominios de $f \circ id_X$, $f$, $id_Y \circ f$ coinciden. Falta ver que los grafos también lo hacen. $\forall x \in X$:
    \begin{gather*}
        (f \circ id_X)(x) = f(id_X(x)) = f(x) \\
        (id_Y \circ f)(x) = id_Y(f(x)) = f(x)
    \end{gather*}

    Por tanto, $f \circ id_X = f = id_Y \circ f$.
\end{proof}

\begin{lema}
    Sean $f:X\rightarrow Y$ y $g:Y \rightarrow X$ aplicaciones tales que $g \circ f = id_X$.
    
    Entonces, $f$ es inyectiva y $g$ es sobreyectiva.
\end{lema}
\begin{proof}
    Demostramos en primer lugar que $f$ es inyectiva:
    $$\forall x_1, x_2 \in X \mid f(x_1) = f(x_2) \Longrightarrow g(f(x_1)) = g(f(x_2)) \stackrel{(\ast)}{\Longrightarrow} x_1 = x_2$$
    donde en $(\ast)$ hemos usado la hipótesis de que $g \circ f = id_X$.
    Por tanto, como se tiene $f(x_1)=f(x_2)\Longrightarrow x_1=x_2$, se tiene que $f$ es inyectiva. Veamos ahora que $g$ es sobreyectiva:
    $$\forall x \in X ~ \exists y=f(x) \in Y \mid g(y) = x$$
    Por tanto, como todo elemento del codominio tiene su antiimagen correspondiente, tenemos que $g$ es sobreyectiva.
\end{proof}

\begin{teo}[Caracterización de la biyectividad] \label{teo:CarBiyect}
    Sea $f:X \rightarrow Y$ una aplicación. Entonces:
    $$f \mbox{ es biyectiva } \Longleftrightarrow \exists g:Y \rightarrow X \mid g \circ f = id_X \y f \circ g = id_Y$$
\end{teo}
\begin{proof} Demostremos por doble implicación:
    \begin{description}
        \item[$\Longrightarrow)$] Suponemos $f$ biyectiva. Por tanto, $\forall y \in Y~\exists_1 x \in X \mid f(x) = y$. Definimos la aplicación siguiente:
        \Func{g}{Y}{X}{y}{g(y)=x\mid f(x)=y}
        Esto es posible ya que, por ser $f$ biyectiva, dicho valor de $x\in X$ es único. Veamos que verifica que $f\circ g=id_Y$:
        \begin{equation*}
            (f \circ g)(y) = f(g(y)) = f(x) = y \qquad \forall y \in Y
        \end{equation*}
        Por tanto, se tiene que $f \circ g = id_Y$. 
        La otra igualdad se deduce directamente de la definición de $g$, ya que $f(x)=y$:
        \begin{equation*}
            (g\circ f)(x) = g(f(x)) = g(y) = x \qquad \forall x\in X
        \end{equation*}
        Por tanto, se tiene esta implicación.

        \item[$\Longleftarrow$)]  Supongamos que $\exists g:Y \rightarrow X \mid g \circ f = id_X \y f \circ g = id_Y$.
        Según el lema anterior, sabemos que:
            $$\left.\begin{array}{lll}
                g \circ f = id_X & \Rightarrow & f \mbox{ es inyectiva y } g \mbox{ es sobreyectiva} \\
                f \circ g = id_Y & \Rightarrow & g \mbox{ es inyectiva y } f \mbox{ es sobreyectiva}
            \end{array} \right.$$
        Por tanto, tenemos que $f$ es biyectiva.
    \end{description}
\end{proof}

\begin{lema}[Unicidad]\label{lema:Uni_Inv}
    Sea $f:X\to Y$. Si $f$ es biyectiva, se verifica que la función $g:Y \rightarrow X$ (cuya existencia ya está provada) es la única aplicación que verifica que $g \circ f = id_X \y f \circ g = id_Y$.
\end{lema}
\begin{proof}
    Supongamos que no es única, y sea $h:Y \rightarrow X$ otra aplicación tal que $h \circ f = id_X \y f \circ h = id_Y$ la otra opción. Entonces:
    $$h = h \circ id_Y = h \circ (f \circ g) = (h \circ f) \circ g = id_X \circ g = g$$
    Quedando así demostrada la unicidad de $g$.
\end{proof}

\begin{definicion}[Inversa]
    Sea $f:X \rightarrow Y$ una aplicación biyectiva. Por el lema anterior, sólo existe una aplicación $g:Y \rightarrow X \mid g \circ f = id_X \y f \circ g = id_Y$. Llamaremos a esta aplicación $g$ \textbf{aplicación
        inversa de $f$} y la notaremos como $f^{-1}$.
\end{definicion}


Notemos que, dada $f:X \rightarrow Y$, para comprobar que $g:Y \rightarrow X$ sea la inversa de $f$, gracias al Lema \ref{lema:Uni_Inv} nos basta con ver que $f \circ g = id_Y \y g \circ f = id_X$.

\begin{lema}
    Sea $f:X \rightarrow Y$ biyectiva. Entonces $f^{-1}$ es biyectiva, siendo su inversa $f$:
    $$(f^{-1})^{-1} = f$$
\end{lema}
\begin{proof}
    Como $f^{-1}$ es la inversa de $f$, se tiene de forma directa:
    \begin{gather*}
        f \circ f^{-1} = id_Y \\
        f^{-1} \circ f = id_X
    \end{gather*}
    
    Por el Teorema \ref{teo:CarBiyect}, $f^{-1}$ es biyectiva; y por el Lema \ref{lema:Uni_Inv}, $(f^{-1})^{-1} = f$.
\end{proof}

\begin{lema}[Inversa de una composición]
    Sean $f:X \rightarrow Y$, $g:Y \rightarrow Z$ funciones biyectivas. Entonces:
    $$(g \circ f)^{-1} = f^{-1} \circ g^{-1}$$
\end{lema}
\begin{proof}
Tenemos que el dominio de ambas es $Z$ y el codominio es $X$. Aplicamos el Lema \ref{lema:Uni_Inv} y la Proposición \ref{prop:CompAsoc}:
\begin{gather*}
    (g \circ f) \circ (f^{-1} \circ g^{-1}) = g \circ (f \circ f^{-1}) \circ g^{-1} = (g \circ id_Y) \circ g^{-1} = g \circ g^{-1} = id_Z \\
    (f^{-1} \circ g^{-1}) \circ (g \circ f) = f^{-1} \circ (g^{-1} \circ g) \circ f = (f^{-1} \circ id_Y) \circ f = f^{-1} \circ f = id_X
\end{gather*}
Por lo que: $$(g \circ f)^{-1} = f^{-1} \circ g^{-1}$$
\end{proof}

\begin{prop}\label{prop:ConjFinito_Equivalencias}
    Sea $X$ un conjunto \ul{finito} no vacío y $f:X \rightarrow X$ una aplicación, los siguientes enunciados
    son equivalentes:
    \begin{enumerate}[label=\roman*.]
        \item $f$ es biyectiva.
        \item $f$ es inyectiva.
        \item $f$ es sobreyectiva.
    \end{enumerate}
\end{prop}
\begin{proof} Demostramos la siguiente equivalencia:
\begin{description} 
    \item[I $\Longrightarrow$ II)] Trivial, a partir de la definición de aplicación biyectiva.
    
    \item[II $\Longrightarrow$ III)] Sea $f:X \rightarrow X$ inyectiva, supongamos que $|X| = n$, $(n \geq 1)$. Como $f$ es inyectiva, entonces $|Img(f)| = n$. Luego:
    $$Img(f) \subseteq X \y |Img(f)| = |X| \Rightarrow Img(f) = X$$
    Por tanto, tenemos que $f$ es sobreyectiva.

    \item[III $\Longrightarrow$ II)] Sea $f$ sobreyectiva, y demostraremos que $f$ es inyectiva. Para ello, por reducción al absurdo, supongamos que $f$ no es inyectiva. Por tanto, $|Img(f)|<~|X|$. Entonces, $Img(f) \subsetneq X$, llegando así a una \underline{contradicción}, ya que $f$ era sobreyectiva.
    
    Luego $f$ es inyectiva y como era sobreyectiva, tenemos que es biyectiva.
\end{description}
\end{proof}

\begin{definicion}[Conjunto imagen de un conjunto]
    Dada una aplicación ${f:X\rightarrow Y}$ y un conjunto $A\subseteq X$, definimos la imagen de $A$ mediante $f$, notado por $f_*(A)$ o $f(A)$ por:
    \begin{equation*}
        f(A) = f_*(A) = \{f(x) \mid x \in A\} \subseteq Y
    \end{equation*}
\end{definicion}

\begin{definicion}[Conjunto imagen inversa de un conjunto]
    Dada una aplicación $f:X\rightarrow Y$ y un conjunto $B\subseteq Y$, definimos la imagen inversa de $B$ mediante $f$, notado por $f^*(B)$ o $f^{-1}(B)$ por:
    \begin{equation*}
        f^{-1}(B) = f^*(B) = \{x \in X \mid f(x) \in B\} \subseteq X
    \end{equation*}
\end{definicion}
No debemos confundir la notación $f^{-1}(B)$ con la aplicación inversa de $f$, pues no es necesario suponer nada sobre $f$ para hablar de la imagen inversa del conjunto $B$.

\begin{prop}
    La imagen inversa es compatible con todas las operaciones con conjuntos. Sea $f:X\rightarrow Y$ una aplicación y $A,B\subseteq Y$, se verifica:
    \begin{enumerate}
        \item $f^*(A \cup B) = f^*(A) \cup f^*(B)$
        \item $f^*(A\cap B) = f^*(A)\cap f^*(B)$
        \item $f^*(A - B) = f^*(A) - f^*(B)$
        \item $f^*(Y - A) = X - f^*(A)$
    \end{enumerate}
\end{prop}
\begin{proof}
    Demostramos cada una de las propiedades:
    \begin{align*}
        f^*(A \cup B) &= \{x \in X \mid f(x) \in A \cup B\} = \{x \in X \mid f(x) \in A \vee f(x) \in B\} =\\&= \{x \in X \mid f(x) \in A\} \cup \{x \in X \mid f(x) \in B\} = f^*(A) \cup f^*(B)\\
        f^*(A \cap B) &= \{x \in X \mid f(x) \in A \cap B\} = \{x \in X \mid f(x) \in A \y f(x) \in B\} =\\&= \{x \in X \mid f(x) \in A\} \cap \{x \in X \mid f(x) \in B\} = f^*(A) \cap f^*(B)\\
        f^*(A - B) &= \{x \in X \mid f(x) \in A - B\} = \{x \in X \mid f(x) \in A \y f(x) \notin B\} =\\&= \{x \in X \mid f(x) \in A\} - \{x \in X \mid f(x) \in B\} = f^*(A) - f^*(B)
    \end{align*}
    El último apartado se obtiene a partir del tercero, haciendo uso además de que $f^*(Y)=X$.
\end{proof}

\begin{definicion}[Aplicación característica de un conjunto]
    Sea $X$ un conjunto y $A\subseteq X$, podemos definir la \textbf{aplicación característica de $A$}, notada por $\chi_A$ como la aplicación $\chi_A:X\rightarrow \{0,1\}$ dada por:
    \begin{equation*}
        \chi_A(x) = \left\{\begin{array}{lcl}
                1 & \text{si} & x\in A \\
                0 & \text{si} & x\notin A
        \end{array}\right.
    \end{equation*}
\end{definicion}


\section{Relaciones de equivalencia}
\begin{definicion}[Relación binaria]
    Sea $X$ un conjunto no vacío, una \textbf{relación binaria en $X$} es un subconjunto $R \subseteq X \times X$. Dados $a, b \in X \mid (a, b) \in R$, diremos que $a$ está relacionado con $b$ por $R$, notado
    $aRb$.
\end{definicion}

Dado un conjunto $X$ y una relación binaria $R$ en $X$, algunas propiedades que puede cumplir $R$ son:
\begin{itemize}
    \item \textbf{Reflexividad:}  $\forall a \in X \Rightarrow aRa$
    \item \textbf{Simetría:} si: $\forall a,b \in X \mid aRb \Rightarrow bRa$
    \item \textbf{Transitividad:} $\forall a,b,c \in X \mid aRb \y bRc \Rightarrow aRc$
\end{itemize}
En el caso de que una relación $R$ cumpla las tres propiedades mencionadas, diremos que \textbf{$R$ es una relación binaria de equivalencia sobre el conjunto $X$}.

\begin{ejemplo} Algunos ejemplos de relaciones binarias son:
\begin{enumerate}
    \item Sea $X=\{a,b,c\}$. Son relaciones binarias:
    \begin{equation*}
        \begin{array}{l|ccc}
            & \text{Reflexividad} & \text{Simetría} & \text{Transitividad} \\ \hline
            R_1 = \{(a,a), (a,b), (b,c)\} & \text{No} & \text{No} & \text{No} \\
            R_2 = \{(a,a), (b,b), (c,c), (a,b), (b,c)\} & \text{Sí} & \text{No} & \text{No} \\
            R_3 = \{(a,b), (b,a)\} & \text{No} & \text{Sí} & \text{No} \\
            R_4 = \{(a,a), (b,b), (c,c), (a,b), (b,a)\} & \text{Sí} & \text{Sí} & \text{Sí} \\
        \end{array}
    \end{equation*}

    \item Sea $X=\bb{N}$, y consideramos la relación binaria:
    \begin{equation*}
        R=\{(a,b)\in \bb{N}\times \bb{N} \mid a+b \text{ es un número par}\}
    \end{equation*}

    Veamos que es una relación de equivalencia:
    \begin{itemize}
        \item \textbf{Reflexividad:}  Sea $a\in X$. Entonces, $aRa \Longleftrightarrow a+a=2a$ es par, lo cual es cierto.
        \item \textbf{Simetría:} Sean $a,b \in X \mid aRb\Longrightarrow a+b=2k \Longrightarrow b+a=2k \Longrightarrow bRa$, para cierto $k\in \bb{N}$.
        \item \textbf{Transitividad:} $\forall a,b,c \in X \mid aRb \y bRc$, se tiene que $\exists k,k'\in \mathbb{N}$:
        \begin{equation*}
            \left.\begin{array}{ccc}
                aRb & \Longrightarrow & a+b=2k \\
                \land&&\land\\
                bRc & \Longrightarrow & b+c=2k'
            \end{array} \right\}
            \Longrightarrow a+b+b+c = a+2b+c = 2k + 2k' = 2(k+k')
        \end{equation*}
        Por tanto, se tiene que $a+c=2(k+k'-b)$, para ciertos $k,k'\in \bb{N}$. Por tanto, tenemos que $aRc$.
    \end{itemize}

    \item Sea $X=\bb{R}^2$ y definimos $O=(0,0)$ como el origen del plano cartesiano. Entonces, consideramos la relación binaria:
    \begin{equation*}
        pRq \Longleftrightarrow d(O,p) = d(O,q)
    \end{equation*}

    Es directo comprobar (dejamos la demostración al lector) que esta relación es de equivalencia.
\end{enumerate}
\end{ejemplo}

\begin{definicion}[Clase de equivalencia]
    Sea $X$ un conjunto no vacío y $R$ una relación binaria de equivalencia. Para cada $a \in X$, definimos \textbf{la clase de equivalencia de $a$}, notada por $\overline{a}$ ó por $[a]$ como el conjunto:
    $$[a] = \{x \in X \mid xRa\} \subseteq X$$
\end{definicion}

Esto es, $[a]$ contiene aquellos elementos de $X$ que estén relacionados o que son equivalentes con $a$.
Por la propiedad reflexiva, tenemos que $aRa \Rightarrow a \in [a]$, por lo que $ [a] \neq \emptyset~\forall a \in X$.\\


A cada uno de los elementos de $X$ que pertenezcan a $[a]$ para algún $a \in X$ se les llama \textbf{representantes de la clase de $a$}.

\begin{ejemplo} Veamos algunos ejemplos de clase de equivalencia respecto de las relaciones binarias anteriores:
\begin{enumerate}
    \item[2.] Veamos la clase de equivalencia con representante de clase $0$ de la relación de equivalencia de los pares:
    \begin{equation*}
        [0] = \{x\in \bb{N}\mid xR0\} = \{x\in \bb{N}\mid x+0 \text{ es par}\} = \{x\in \bb{N}\mid x \text{ es par}\}
    \end{equation*}

    \item[3.] Veamos la clase de equivalencia con representante de clase el punto $(2,3)$ de la relación de equivalencia de la distancia:
    \begin{multline*}
        [(2,3)] = \{p\in \bb{R}^2\mid pR(2,3)\} = \{p\in \bb{R}^2\mid d(O,p)=d(O,(2,3))\} =\\= \left\{p\in \bb{R}^2\mid d(O,p)=\sqrt{13}\right\}
    \end{multline*}
\end{enumerate}
\end{ejemplo}

\begin{prop}
    Sea $X$ un conjunto no vacío y $R$ una relación de equivalencia en $X$. Sean $a,b \in X$. Son
    equivalentes:
    \begin{enumerate}[label=\roman*.]
        \item $aRb$
        \item $a \in [b]$
        \item $b \in [a]$
        \item $[a] \cap [b] \neq \emptyset$
        \item $[a] = [b]$
    \end{enumerate}
\end{prop}
\begin{proof} Demostramos por implicaciones sucesivas:
\begin{description}
    \item [I $\Longrightarrow$ II)] Por la definición de $[b]$, tenemos que si $aRb \Rightarrow a \in [b]$.

    \item [II $\Longrightarrow$ III)] Suponemos $a \in [b]$, es decir, $ aRb$. Por ser una relación de equivalencia, es simétrica, luego $bRa$, por lo que $b \in [a]$.
    
    \item [III $\Longrightarrow$ IV)] Hemos supuesto que $b\in [a]$. Además, se ha visto que $\forall b\in X$, se tiene que $b\in [b]$. Por tanto, $b\in [a]\cap [b]$, por lo que este último no es vacío.
    
    \item [IV $\Longrightarrow$ V)]
    Como $[a] \cap [b] \neq \emptyset \Rightarrow \exists c \in X \mid c \in [a] \cap [b] \Rightarrow cRa \y cRb$.
    \begin{gather*}
        \forall x \in [a] \Rightarrow xRa \mathop{\Rightarrow}^{cRa} aRc \Rightarrow xRc \mathop{\Rightarrow}^{cRb} xRb \Rightarrow x \in [b] \Rightarrow [a] \subseteq [b] \\
        \forall x \in [b] \Rightarrow xRb \mathop{\Rightarrow}^{cRb} bRc \Rightarrow xRc
        \mathop{\Rightarrow}^{cRa} xRa \Rightarrow x \in [a] \Rightarrow [b] \subseteq [a]
    \end{gather*}

    Tenemos que $[a] \subseteq [b] \y [b] \subseteq [a] \Rightarrow [a] = [b]$.
    
    \item [V $\Longrightarrow$ I)] Como $a \in [a] = [b] \Rightarrow aRb$. \qedhere
\end{description}
\end{proof}

\begin{definicion}[Conjunto cociente]
    Dado un conjunto $X$ no vacío y una relación de equivalencia $R$ sobre $X$, se define el
    \textbf{conjunto cociente de $X$ por la relación de equivalencia $R$}, notado $X/R$ como el conjunto:
    $$X/R = \{[a] \mid a \in X\}$$
\end{definicion}

\begin{ejemplo} Veamos algunos ejemplos de conjuntos cocientes por las relaciones binarias anteriores:
\begin{enumerate}
    \item[2.] Veamos las distintas clases de equivalencia que hay en la relación de equivalencia de los pares:
    \begin{equation*}
        \begin{split}
            [0] &= \{x\in \bb{N} \mid xR0\} = \{x\in \bb{N}\mid x+0 \text{ es par}\} = \{x\in \bb{N}\mid x \text{ es par}\} \\&= \{0,2,4,\dots\} \Longrightarrow [0]=[2]=[4]=\dots
        \end{split}
    \end{equation*}
    \begin{equation*}
        \begin{split}
            [1] &= \{x\in \bb{N} \mid xR1\} = \{x\in \bb{N}\mid x+1 \text{ es par}\} = \{x\in \bb{N}\mid x \text{ es impar}\} \\&= \{1,3,5,\dots\} \Longrightarrow [1]=[3]=[5]=\dots
        \end{split}
    \end{equation*}

    Por tanto, $\displaystyle \bb{N}/R = \{[0],[1]\}$.

    \item[3.] Veamos las distintas clases de equivalencia de la relación de equivalencia de la distancia:
    \begin{multline*}
        [p] = \{x\in \bb{R}^2\mid xRp\} = \{x\in \bb{R}^2\mid d(0,x)=d(0,p)\} = \{x\in \bb{R}^2\mid d(0,x)=r\} =\\= C_r \quad \text{(circunferencia de radio $r$ y centro $O$)}.
    \end{multline*}

    Por tanto, se tiene que $\displaystyle \bb{R}^2/R = \{C_r\mid r\in \bb{R}^+_0\}$.
\end{enumerate}
\end{ejemplo}

\begin{prop}
    Sea $f:X \rightarrow Y$ una aplicación y $R$ una relación de equivalencia en $X$. Supongamos que $f$ verifica la siguiente propiedad:
    \begin{center}
        Dados $a,b \in X \mid aRb \Rightarrow f(a) = f(b)$.
    \end{center}
    
    Entonces, podemos definir la siguiente aplicación:
    \begin{equation*}
        \begin{array}{rll}
            \overline{f}: X/R & \longrightarrow & Y\\
                \left[ a \right] & \longmapsto & \overline{f}\left([a]\right)=f(a)
        \end{array}
    \end{equation*}
    
    Se verifica que:
    \begin{enumerate}
        \item $Img\left(\overline{f}\right) = Img(f)$.
        \item $\overline{f}$ es sobreyectiva $\Longleftrightarrow f$ es es sobreyectiva.
        \item $\overline{f}$ es inyectiva $\Longleftrightarrow~\forall a,b \in X \mid f(a) = f(b) \Rightarrow aRb$.
    \end{enumerate}
\end{prop}
\begin{proof}
    Veamos en primer lugar que $\overline{f}$ está bien definida, es decir, que dos elementos iguales tienen la misma imagen. Nuestra definición de $\overline{f}$ depende del representante de la clase escogida, por lo que debemos comprobar que al cambiar el representante no cambia la imagen de $\overline{f}$:
    $$\forall a, b \in X \mid [a] = [b] \Rightarrow aRb \Rightarrow f(a) = f(b) \Rightarrow \overline{f}\left([a]\right) = \overline{f}\left([b]\right)$$

    Por tanto, tenemos que $\overline{f}$ es una aplicación. Comprobemos las tres propiedades que se enuncian:
    \begin{enumerate}
        \item Comprobemos que $Im\left(\overline{f}\right)=Im(f)$:
        $$Img\left(\overline{f}\right) = \left\{\overline{f}\left([a]\right) \mid [a] \in X/R\right\} = \{f(a) \mid [a] \in X/R\} = \{f(a) \mid a \in X\} = Img(f)$$

        \item $\overline{f}$ es sobreyectiva $\Longleftrightarrow Img\left(\overline{f}\right) = Y \Longleftrightarrow Img(f) = Y \Longleftrightarrow f$ es sobreyectiva.

        \item Comprobemos que $\overline{f}$ es inyectiva $\Longleftrightarrow~\forall a,b \in X \mid f(a) = f(b) \Rightarrow aRb$:
        \begin{description}
            \item[$\Longrightarrow)$] Sean $a,b \in X \mid f(a) = f(b) \Rightarrow \overline{f}\left([a]\right) = \overline{f}\left([b]\right) \Rightarrow [a] = [b] \Rightarrow aRb$

            \item[$\Longleftarrow)$] $\forall [a], [b] \in X/R \mid \overline{f}\left([a]\right) = \overline{f}\left([b]\right) \Rightarrow f(a) = f(b) \Rightarrow aRb \Rightarrow [a] = [b] \Rightarrow \overline{f}$ es inyectiva.
        \end{description}
    \end{enumerate}
\end{proof}

A la función $\overline{f}$ de la proposición anterior la llamaremos \textbf{aplicación inducida por $f$ en el conjunto cociente}.

    \chapter{Estadística descriptiva bidimensional}

\section{Distribución conjunta de dos caracteres estadísticos}

Sea una población formada por $n$ individuos en la que se desea estudiar simultáneamente dos caracteres, $X$ e $Y$.
Dichos caracteres podrán ser ambos cualitativos, uno cualitativo y otro cuantitativo o ambos cuantitativos (los dos
discretos, los dos continuos o uno discreto y otro continuo).\\


Si designamos por $x_1, x_2, \ldots, x_k$ las $k$ modalidades posibles del carácter $X$ y por $y_1, y_2, \ldots, y_p$
las $p$ modalidades posibles del carácter $Y$, las observaciones correspondientes a cada individuo serán de la forma
$(x_i, y_j)$, par ordenado que representa las modalidades tomadas por dicho individuo en los caracteres $X$ e $Y$.

\begin{itemize}
    \item $n_{ij}$: Número total de individuos en la población que presentan simultáneamente la modalidad $x_i$ del carácter $X$
          y la modalidad $y_j$ del carácter $Y$. Le llamamos frecuencia absoluta del par $(x_i, y_j)$.
    \item $f_{ij}$: Proporción de individuos en la población que presentan simultáneamente la modalidad $x_i$ del
          carácter $X$ y la modalidad $y_j$ del carácter $Y$. Le llamamos frecuencia relativa del par $(x_i, y_j)$.
          Por la definición de proporción sobre el total, tenemos que:
          $$f_{ij} = \dfrac{n_{ij}}{n} \qquad i \in \{1, 2, \ldots, k\} ~ j \in \{1, 2, \ldots, p\}$$
\end{itemize}

Gracias al principio de incompatibilidad y exhaustividad de las modalidades, tenemos que:
$$\sum_{i=1}^k\sum_{j=1}^p n_{ij} = n ~~ \sum_{i=1}^k\sum_{j=1}^p f_{ij}=1$$

La distribución $\left\{ (x_i,y_j), n_{ij}\right\}_{\substack{i=1,\dots,k\\j=1,\dots,p}}$ recibe el nombre de distribución conjunta de los caracteres $X$ e $Y$.

\begin{itemize}
    \item $n_{i.}$: Número total de individuos que presentan la modalidad $x_i$ del carácter $X$ sin tener en cuenta
          las modalidades que puedan tomar para el carácter $Y$:
          $$n_{i.} = \sum_{j=1}^p n_{ij} \qquad i\in \{1, \ldots, k\}$$
    \item $f_{i.}$: Proporción total de individuos que presentan la modalidad $x_i$ del carácter $X$ sin tener
          en cuenta el carácter $Y$:
          $$f_{i.} = \sum_{j=1}^p f_{ij} = \dfrac{n_{i.}}{n} \qquad i \in \{1, \ldots, k\}$$
    \item $n_{.j}$: Número total de individuos que presentan la modalidad $y_j$ del carácter $Y$ sin tener en cuenta
          las modalidades que puedan tomar para el carácter $X$:
          $$n_{.j} = \sum_{i=1}^k n_{ij} \qquad j\in \{1, \ldots, p\}$$
    \item $f_{.j}$: Proporción total de individuos que presentan la modalidad $y_j$ del carácter $Y$ sin tener
          en cuenta el carácter $X$:
          $$f_{.j} = \sum_{i=1}^k f_{ij} = \dfrac{n_{.j}}{n} \qquad j \in \{1, \ldots, p\}$$
\end{itemize}

Se tiene que:
$$\sum_{i=1}^k n_{i.} = \sum_{j=1}^p n_{.j} = n \hspace{2cm} \sum_{i=1}^k f_{i.} = \sum_{j=1}^p f_{.j} = 1$$

\section{Tablas estadísticas bidimensionales}

Para agrupar nuestros datos estadísticos, usaremos una tabla de doble entrada como la siguiente:

\begin{center}
    \begin{tabular}{c|c|c|c|c|c|c|c}
        $X \backslash Y$ & $y_1$    & $y_2$    & $\ldots$ & $y_j$    & $\ldots$ & $y_p$    & $n_{i.}$ \\
        \hline
        $x_1$            & $n_{11}$ & $n_{12}$ & $\ldots$ & $n_{1j}$ & $\ldots$ & $n_{1p}$ & $n_{1.}$ \\
        \hline
        $x_2$            & $n_{21}$ & $n_{22}$ & $\ldots$ & $n_{2j}$ & $\ldots$ & $n_{2p}$ & $n_{2.}$ \\
        \hline
        $\vdots$         & $\vdots$ & $\vdots$ & $\ldots$ & $\vdots$ & $\ldots$ & $\vdots$ & $\vdots$ \\
        \hline
        $x_i$            & $n_{i1}$ & $n_{i1}$ & $\ldots$ & $n_{ij}$ & $\ldots$ & $n_{ip}$ & $n_{i.}$ \\
        \hline
        $\vdots$         & $\vdots$ & $\vdots$ & $\ldots$ & $\vdots$ & $\ldots$ & $\vdots$ & $\vdots$ \\
        \hline
        $x_k$            & $n_{k1}$ & $n_{k2}$ & $\ldots$ & $n_{kj}$ & $\ldots$ & $n_{kp}$ & $n_{k.}$ \\
        \hline
        $n_{.j}$         & $n_{\text{.}1}$ & $n_{\text{.}2}$ & $\ldots$ & $n_{.j}$ & $\ldots$ & $n_{.p}$ & $n$
    \end{tabular}
\end{center}

En el caso de que uno de los carácteres sea cualitativo, esta tabla recibirá el nombre de tabla de contingencia.

\section{Representaciones gráficas}
\subsection{Diagrama de dispersión o nube de puntos}

Consiste en representar cada par de observaciones $(x_i, y_j)$ por un punto en un plano bidimensional.
Para representar la frecuecia absoluta de cada punto, se suele incluir un número pequeño al lado de cada punto.
En caso de representar variables cuantitativas continuas, usaremos las marcas de clase.

\subsection{Estereogramas}

Los estereogramas son gráficos tridimensionales formados por barras colocadas en cada punto $(x_i, y_j)$ con altura
$n_{ij}$.

En el caso de las variables cuantitativas continuas, las barras pasarán a ser prismas, donde la base del prisma
tiene dimensiones $e_i - e_{i-1} x e_j - e_{j-1}$. La altura de cada prisma vendrá dada por la fórmula:
$$h_{ij} = \dfrac{n_{ij}}{(L_i - L_{i-1})(L_j - L_{j-1})}$$
De tal forma que el volumen de cada prisma es igual a la frecuencia absoluta de cada pareja de intervalos de clase.

\section{Distribuciones marginales}

Las modalidades del carácter $X$ junto con las frecuencias $n_{i.}$ forman la distribución marginal del carácter $X$:
$$\{x_i, n_{i.}\}_{i=1,\ldots,k}$$

Mientras que las modalidades del carácter $Y$ junto con las frecuencias $n_{.j}$
forman la distribución marginal del carácter $Y$: $$\{y_j, n_{.j}\}_{j=1,\ldots,p}$$

Ambas son distribuciones unidimensionales, a las que es posible dar el tratamiento visto en el tema anterior.

Como ejemplo, la distribución marginal del carácter $X$ es la siguiente:
\begin{center}
    \begin{tabular}{c|c|c}
        $X$      & $n_{i.}$ & $f_{i.}$ \\
        \hline
        $x_1$    & $n_{1.}$ & $f_{1.}$ \\
        \hline
        $x_2$    & $n_{2.}$ & $f_{2.}$ \\
        \hline
        $\vdots$ & $\vdots$ & $\vdots$ \\
        \hline
        $x_i$    & $n_{i.}$ & $f_{i.}$ \\
        \hline
        $\vdots$ & $\vdots$ & $\vdots$ \\
        \hline
        $x_k$    & $n_{k.}$ & $f_{k.}$ \\
        \hline
                 & $n$        & $1$
    \end{tabular}
\end{center}

\section{Distribuciones condicionadas}

En ocasiones es interesante el estudio de un carácter sólo sobre los individuos que presentan una modalidad (o varias)
del otro carácter. por ejemplo, podría ser interesante el estudio del carácter $X$ en la subpoblación formada por
los individuos que presentan la modalidad $y_j$ del carácter $Y$, una subpoblación de $n_{.j}$ individuos.\\


Decimos que la frecuencia relativa de la modalidad $x_i$ del carácter $X$ en aquellos individuos que presentan la
modalidad $y_j$ del carácter $Y$ es:
$$f_{i/j} \equiv f_i^j = \dfrac{n_{ij}}{n_{.j}} \qquad i \in \{1, \ldots, k\}$$


Análogamente, podemos considerar la frecuencia relativa de la modalidad $y_j$ del carácter $Y$ en aquellos individuos
que presentan la modalidad $x_i$ del carácter $X$:
$$f_{j/i} \equiv f_j^i = \dfrac{n_{ij}}{n_{i.}} \qquad j \in \{1, \ldots, p\}$$

De esta forma, existen $p$ distribuciones condicionadas para el carácter $X$ según una única modalidad del carácter
$Y$ y $k$ distribuciones condicionadas para el carácter $Y$ según una única modalidad del carácter $X$.\\


Ejemplo de la distribución condicionada del carácter $X$ respecto a la modalidad $y_j$ del carácter $Y$, que
denotaremos por $X/Y=y_j$:

\begin{center}
    \begin{tabular}{c|c|c}
        $X$      & $n_{ij}$ & $f_i^j$  \\
        \hline
        $x_1$    & $n_{1j}$ & $f_1^j$  \\
        \hline
        $x_2$    & $n_{2j}$ & $f_2^j$  \\
        \hline
        $\vdots$ & $\vdots$ & $\vdots$ \\
        \hline
        $x_i$    & $n_{ij}$ & $f_i^j$  \\
        \hline
        $\vdots$ & $\vdots$ & $\vdots$ \\
        \hline
        $x_k$    & $n_{kj}$ & $f_k^j$  \\
        \hline
                 & $n_{.j}$ & 1
    \end{tabular}
\end{center}

De las definiciones anteriores, tenemos que:
$$f_{ij} = \dfrac{n_{ij}}{n} = \dfrac{n_{i.}}{n}\dfrac{n_{ij}}{n_{i.}} = \dfrac{n_{.j}}{n}\dfrac{n_{ij}}{n_{.j}} $$
$$ f_{ij} = f_{i.} f_j^i = f_{.j}f_i^j $$

\section{Dependencia e Independencia estadística}

Dos caracteres $X$ e $Y$ serán \underline{estadísticamente dependientes} cuando la variación en uno de ellos influya en la distribución del otro.\\


Por otra parte, se dice que el carácter $X$ es \underline{estadísticamente independiente} del carácter $Y$ si las distribuciones
de $X$ condicionadas a cada valor $y_j$ de $Y$ ($X/Y=y_j$) son idénticas para cualquier valor de $j$. En este caso,
cada distribución condicionada es idéntica a la distribución marginal de $X$:
$$\dfrac{n_{i1}}{n_{\text{.}1}} = \dfrac{n_{i2}}{n_{\text{.}2}} = \ldots = \dfrac{n_{ij}}{n_{.j}} = \ldots = \dfrac{n_{ip}}{n_{.p}} \qquad \forall i = 1, \ldots, k$$

De donde tenemos que:
$$f_{i/j} \equiv f_i^j = \dfrac{n_{ij}}{n_{.j}} = \dfrac{n_{i1} + n_{i2} + \ldots + n_{ip}}{n_{.1} + n_{.2} + \ldots + n_{.p}}
    = \dfrac{n_{i.}}{n} = f_{i.}$$

Por lo que si $f_i^j = f_{i.} \ \forall i \in \{1, \ldots, k\}$, entonces el carácter $X$ será independiente del
carácter $Y$. (Análogamente, se define la independencia del carácter $Y$ con el carácter $X$).

\begin{prop}
Si $X$ es una variable independiente de $Y$ $\Longrightarrow f_{i\cdot} = f_{i/j}$
\end{prop}
\begin{proof}
Suponemos $X$ e $Y$ variables independientes.
$$f_{i\cdot} = \sum_{j=1}^p f_{ij} = \sum_{j=1}^p f_{i/j}f_{\cdot j} \stackrel{(\ast)}{=} f_{i/j}\sum_{j=1}^pf_{\cdot j} = f_{i/j} f_{\cdot \cdot} = f_{i/j}$$

donde en $(\ast)$ he usado que las variables son independientes, por lo que la frecuencia condicionada a $Y$ no depende de $j$.
\end{proof}

\begin{teo}[Teorema de Caracterización de la Independencia]
Sean $X$ e $Y$ dos variables estadísticas.

\centering
$X$ e $Y$ son independientes $\Longleftrightarrow$ $f_{ij} = f_{i\cdot} f_{\cdot j} \Longleftrightarrow n_{ij} = \frac{n_{i\cdot} n_{\cdot j}}{n} \quad \forall i,j$
\end{teo}
\begin{proof} Demostramos mediante la doble implicación:
    \begin{description}
    \item [$\Longrightarrow$)] Suponemos $X$ e $Y$ independientes, es decir, $f_{i/j} = f_{i \cdot}$

    Por tanto,
    $$f_{ij} = f_{i/j}f_{\cdot j} = f_{i \cdot }f_{\cdot j}$$
    
    \item [$\Longleftarrow$)] Suponemos $f_{ij} = f_{i\cdot} f_{\cdot j}$.

    Probemos que $X$ e $Y$ son linealmente independientes.
    $$f_{\cdot j} = \frac{f_{ij}}{f_{i \cdot}} = f_{j/i}$$
    \end{description}
\end{proof}

\begin{prop}
    Si el carácter $X$ es independiente del carácter $Y$, entonces $Y$ es independiente de $X$ (la independencia es una propiedad recíproca).
\end{prop}
\begin{proof}
    Supuesto $X$ independiente de $Y$, tenemos $f_{i\cdot} = f_{i/j} \quad \forall j=1,\dots,p$
    \begin{equation*}
        f_{ij} = f_{i\cdot} f_{\cdot j} = f_{i\cdot}f_{j/i} \Longrightarrow f_{\cdot j} = f_{j/i}
    \end{equation*}
    demostrando así que $X$ es independiente de $Y$.
\end{proof}


Se dice que el carácter $X$ \underline{depende funcionalmente} del carácter $Y$ si a cada modalidad de $y_j$ de $Y$
le corresponde una única modalidad posible de $X$ con frecuencia no nula. Es decir:
$$\forall j \in \{1, \ldots, p\}, n_{ij} = 0 \text{ excepto para un valor } i = \varphi(j) \mid n_{ij} = n_{.j}$$

\begin{ejemplo} Consideramos la siguiente tabla estadística bidimensional:
    \begin{center}
    \begin{tabular}{c|c|c|c|c|c|c}
        $X \backslash Y$ & $y_1$ & $y_2$ & $y_3$ & $y_4$ & $y_5$ &    \\
        \hline
        $x_1$            & 3     & 0     & 6     & 0     & 0     & 9  \\
        \hline
        $x_2$            & 0     & 4     & 0     & 0     & 2     & 6  \\
        \hline
        $x_3$            & 0     & 0     & 0     & 5     & 0     & 5  \\
        \hline
                         & 3     & 4     & 6     & 5     & 2     & 20 \\
    \end{tabular}
    \end{center}
    
    
    En dicha distribución conjunta, $X$ depende funcionalmente del carácter $Y$. Sin embargo, $Y$ no depende
    funcionalmente del carácter $X$.
\end{ejemplo}

Si sucede que la dependencia funcional es bidireccional, hablaremos de una \underline{dependencia funcional recíproca}.
Notemos que esta es de poco interés estadístico.

\section{Momentos bidimensionales}

Dada una variable estadística bidimensional ($X,Y$) con una distribución conjunta
$\left\{ (x_i,y_j), n_{ij}\right\}_{\substack{i=1,\dots,k\\j=1,\dots,p}}$, se definen los momentos conjunto central y no central de órdenes $r$ y $s$
($r,s \in \N \cup \{0\}$) como:
$$\mu_{rs} = \sum_{i=1}^k \sum_{j=1}^p f_{ij} (x_i - \overline{x})^r (y_j - \overline{y})^s$$
$$m_{rs}=\sum_{i=1}^k \sum_{j=1}^p f_{ij} x_i^r y_j^s$$


Los momentos centrales más utilizados son las varianzas marginales, $\mu_{20} = \sigma_x^2$ y $\mu_{02}=\sigma_y^2$
y el momento $\mu_{11}$, cuya importancia se describe a continuación:

\subsection{Varianza}
\begin{definicion} Dadas dos variables estadísticas unidimensionales, $X$ y $Y$, se define la covarianza de las variables $X$ e $Y$ como:
\begin{equation*}
    \sigma_{xy} = Cov(X,Y) = \mu_{11}
\end{equation*}
\end{definicion}

\begin{prop} Dadas dos variables estadísticas unidimensionales, $X$ y $Y$, se tiene:
    $$\sigma_{xy} = \mu_{11} = m_{11} - m_{10}m_{01}$$
\end{prop}
\begin{proof}
    \begin{equation*}\begin{split}
    \sigma_{xy} &=  Cov(X,Y) = \mu_{11} = \sum_{i=1}^k \sum_{j=1}^p f_{ij}(x_i - \bar{x})(y_j - \bar{y}) = \sum_{i=1}^k \sum_{j=1}^p f_{ij}(x_iy_j - x_i\bar{y} - \bar{x}y_j + \bar{x}\bar{y})
    =\\&=
    \sum_{i=1}^k \sum_{j=1}^px_iy_jf_{ij} - \sum_{i=1}^k \sum_{j=1}^p x_i\bar{y}f_{ij} - \sum_{i=1}^k \sum_{j=1}^p\bar{x}y_jf_{ij} + \sum_{i=1}^k \sum_{j=1}^p \bar{x}\bar{y} f_{ij}
    =\\&=
    \sum_{i=1}^k \sum_{j=1}^px_iy_jf_{ij} - \bar{y}\sum_{i=1}^k x_if_{i\cdot} - \bar{x}\sum_{j=1}^p y_jf_{\cdot j} + \bar{x}\bar{y}\sum_{i=1}^k \sum_{j=1}^pf_{ij}
    =\\&=
    \sum_{i=1}^k \sum_{j=1}^p x_iy_jf_{ij} - \bar{y}\bar{x} - \bar{x}\bar{y} + \bar{x}\bar{y}
    =\\&=
    \sum_{i=1}^k \sum_{j=1}^px_iy_jf_{ij} - \bar{x}\bar{y} = m_{11} - m_{10}m_{01}
\end{split}\end{equation*}
\end{proof}


\begin{prop}
    Si $X$ e $Y$ son independientes $\Longrightarrow \left\{ \begin{array}{c}
        m_{rs}=m_{r0}m_{0s}  \\
        \mu_{rs}=\mu_{r0}\mu_{0s} 
    \end{array}\right.$
\end{prop}
\begin{proof} Supongo $X$ e $Y$ independientes, por lo que $f_{ij}=f_{i.}f_{.j}$. Entonces:
    \begin{equation*}
        m_{rs} = \sum_{i=1}^k \sum_{j=1}^p f_{ij}x_i^r y_j^s
        = \sum_{i=1}^k \sum_{j=1}^p f_{i.}f_{.j}x_i^r y_j^s
        = \sum_{i=1}^k f_{i.}x_i^r \sum_{j=1}^p f_{.j} y_j^s
        = m_{r0}m_{0s}
    \end{equation*}
    \begin{multline*}
        \mu_{rs} = \sum_{i=1}^k \sum_{j=1}^p f_{ij}(x_i-\bar{x})^r (y_j-\bar{y})^s
        = \sum_{i=1}^k \sum_{j=1}^p f_{i.}f_{.j}(x_i-\bar{x})^r (y_j-\bar{y})^s
        =\\=
        \sum_{i=1}^k f_{i.}(x_i-\bar{x})^r \sum_{j=1}^p f_{.j} (y_j-\bar{y})^s
        = \mu_{r0}\mu_{0s}
    \end{multline*}
\end{proof}

\begin{coro}
    Si $X$ e $Y$ son independientes $\Longrightarrow \sigma_{xy} = 0$
\end{coro}
\begin{proof} Supongo $X$ e $Y$ independientes, por lo que $m_{rs}=m_{r0}m_{0s}$. Entonces:
    \begin{equation*}
        \sigma_{xy} = m_{11} - m_{10}m_{01} = m_{10}m_{01} - m_{10}m_{01} = 0
    \end{equation*}
\end{proof}

\begin{prop}
    Si se transforman los valores de $x_i$ e $y_j$ mediante transformaciones lineales dadas por:
    \begin{equation*}
        \left\{\begin{array}{c}
            x_i' = ax_i + b  \\
            y_j' = cy_j + d 
        \end{array} \right.
    \end{equation*}
    La covarianza queda como:
    $$\sigma_{x'y'} = ac \sigma_{xy}$$
\end{prop}
\begin{proof}
    \begin{multline*}
        \sigma_{x'y'} = \sum_{i=1}^k \sum_{j=1}^p f_{ij} (x_i' - \bar{x}')(y_j'-\bar{y}')
        = \sum_{i=1}^k \sum_{j=1}^p f_{ij} [(ax_i+b) - (a\bar{x}+b)][(cy_j+d)-(c\bar{y}+d)]
        =\\=
        ac \sum_{i=1}^k \sum_{j=1}^p (x_i - \bar{x})(y_j- \bar{y})f_{ij} = ac \sigma_{xy}
    \end{multline*}
\end{proof}


Si expresamos nuevas variables a partir de otras, podemos calcular su covarianza a partir de la otra:
$$x_i' = ax_i+b~~y_j'=cy_j+d$$
$$\sigma_{X'Y'}=\sum_{i=1}^k \sum_{j=1}^p f_{ij} (ax_i+b-\overline{x'})(cy_j+d-\overline{y'}) = $$
$$=\sum_{i=1}^k \sum_{j=1}^p f_{ij}(ax_i+b-(a\overline{x}+b))(cy_j+d-(c\overline{y}+d))=$$
$$=\sum_{i=1}^k \sum_{j=1}^p f_{ij}(ax_i-a\overline{x})(cy_j-c\overline{y}) =
    ac \sum_{i=1}^k \sum_{j=1}^p f_{ij}(x_i-\overline{x})(y_j-\overline{y}) = ac\sigma_{xy}$$

\section{Regresión}

Pretendemos buscar que un número de magnitudes $X_1, \ldots, X_n$ se relacionen con una variable $Y$ mediante la expresión:
$$Y = f(X_1, \ldots, X_n)$$

Podemos abordar el problema desde dos enfoques:
\begin{itemize}
    \item Regresión: La determinación de la estructura de dependencia que mejor expresa la relación de la variable $Y$
          con las demás.
    \item Correlación: El estudio del grado de dependencia que existe entre las variables.
\end{itemize}


Si dos variables presentan una dependencia estadística (es decir, una dependencia no funcional), no es posible encontrar una ecuación tal que los valores que puedan presentar dichas variables la satisfagan. Es decir, no es posible encontrar una función que pase por todos los puntos del diagrama de dispersión que representa esa distribución conjunta. Por tato, tendremos que ajustar lo mejor posible una función a una serie de valores observados, encontrando una curva
que, aunque no pase por todos los puntos de la nube, más se aproxime a ellos. Dicho método recibe el nombre de \underline{ajuste por mínimos cuadrados}.

\subsection{Método de mínimos cuadrados}
Sea $f(x_i, a_0,\dots,a_n)$ la función que aproxima la variable $Y$ en función de los valores de $X$.
\begin{notacion}
    A los valores ajustados se les notará de la siguiente manera:
    \begin{equation*}
        \hat{y_j} = f(x_i;a_0,\dots,a_n)
    \end{equation*}
\end{notacion}

\begin{definicion}[Residuo]
Se define el residuo de la modalidad $y_j$ de la variable $Y$ como:
\begin{equation*}
    e_{ij} = y_j - \hat{y_j}
    = y_j - f(x_i, a_0, a_1, \ldots, a_n)
\end{equation*}
\end{definicion}



El método de mínimos cuadrados consiste en encontrar una función $f$ que minimice la media de los cuadrados de los
residuos:
\begin{equation*}
        ECM(f(x_i, a_0, a_1, \ldots, a_n)) = \psi(a_0, a_1, \ldots, a_n)
        = \sum_{i=1}^k \sum_{j=1}^p f_{ij} e_{ij}^2
\end{equation*}


La función $\psi$ se denomina el \underline{error cuadrático medio de la función} $f$, denotada $ECM(f(x_i, a_0, a_1, \ldots, a_n))$.
Como los parámetros $(x_i, a_0, a_1, \ldots, a_n)$ sólo están sometidos a sumas, productos y cuadrados dentro de $\psi$,
dicha función es derivable respecto a cada $a_i \ \forall i \in \{0, \ldots, n\}$. Además, se puede asegurar que
el punto $(\hat{a_0}, \hat{a_1}, \ldots, \hat{a_n})$ donde se anulan las derivadas parciales primeras respecto
de cada $a_i$ corresponde a un mínimo de la función $\psi$.


El cálculo de los parámetros de la función de ajuste óptima según el método de los mínimos cuadrados consiste en
resolver el siguiente sistema, llamado \underline{sistema de ecuaciones normales}:
$$\dfrac{\partial \psi}{\partial a_r}=0 \Rightarrow \sum_{i=1}^k \sum_{j=1}^p f_{ij} e_{ij}
    \dfrac{\partial f}{\partial a_r} = 0 \qquad \forall r \in \{0, \ldots n\}$$


Una de las funciones de regresión más utilizadas para expresar el comportamiento de una variable en función de la otra
es un polinomio de grado $n$ (comenzaremos con $n=1$).

\subsubsection{Ajuste lineal (recta de regresión)}\vspace{-0.5cm}
\begin{equation*}
    Y=f(X;a,b) = a + bX
\end{equation*}

Supongamos que queremos ajustar por el método de mínimos cuadrados una recta que exprese $Y$ en función de $X$.
La función sería $Y=f(X;a,b) = a + bX$, por lo que tendremos que calcular el mínimo en $a$ y $b$ de la función:
$$\psi(a,b) = ECM(a,b) = \sum_{i=1}^k \sum_{j=1}^p f_{ij} [y_j - (a + bx_i)]^2 $$

Obtenemos el sistema de ecuaciones normales:
\begin{equation*}
    \left\{
    \begin{array}{l}
        \dfrac{\partial \psi}{\partial a} = 0 \Rightarrow \displaystyle\sum_{i=1}^k \sum_{j=1}^p f_{ij} [y_j - (a+bx_i)]=0\\ \\
        \dfrac{\partial \psi}{\partial b} = 0 \Rightarrow \displaystyle\sum_{i=1}^k \sum_{j=1}^p f_{ij} [y_j - (a+bx_i)]x_i=0
    \end{array}
    \right\} \Longrightarrow
    \left\{
    \begin{array}{l}
        m_{01} = a + b m_{10}\\ \\
        m_{11} = a m_{10} + bm_{20}
    \end{array}
    \right.
\end{equation*}

La resolución de dicho sistema nos proporciona los coeficientes buscados:
$$\hat a = m_{01} - \dfrac{m_{11} - m_{10}m_{01}}{m_{02}-m_{01}^2}m_{10} = \overline{y} - \dfrac{\sigma_{xy}}{\sigma_x^2}\overline{x}$$
$$\hat b = \dfrac{m_{11} - m_{10}m_{01}}{m_{02}- m_{01}^2} = \dfrac{\sigma_{xy}}{\sigma_x^2}$$

Por tanto, \textbf{la recta de regresión de $Y$ sobre $X$} tiene por expresión:
\begin{equation*}
    Y = \dfrac{\sigma_{xy}}{\sigma_x^2}X + \overline{y} - \dfrac{\sigma_{xy}}{\sigma_x^2}\overline{x}
    \hspace{1cm}
    \left(\text{Equivalentemente}, Y - \overline{y} = \dfrac{\sigma_{xy}}{\sigma_x^2} (X - \overline{x})\right)
\end{equation*}


\begin{definicion}[Coeficiente de regresión lineal]
    Al coeficiente $\dfrac{\sigma_{xy}}{\sigma_x^2}$ se le denomina coeficiente de regresión lineal de $Y$ sobre $X$.

    Análogamente, se define el coeficiente de regresión lineal de $X$ sobre $Y$.
\end{definicion}


Análogamente, la recta de regresión mínimo cuadrática de $X$ sobre $Y$ es la recta $X = h(Y; c, d) = c+dY$ que minimiza
la función
$$\phi(c,d) = \sum_{i=1}^k \sum_{j=1}^p f_{ij} (x_i - \hat x_j )^2$$

donde $\hat x_j = c+dy_j$. Siguiente el procedimiento anterior, llegamos a que \textbf{la recta de regresión de $X$ sobre $Y$} es:
$$X - \overline{x} = \dfrac{\sigma_{xy}}{\sigma_y^2} (Y - \overline{y})$$


Los coeficientes de regresión son las pendientes de las rectas de regresión. Los signos de dichos coeficientes son
los mismos para ambas rectas e igual al signo de la covarianza. Cuando exista dependencia funcional lineal, las dos
rectas de regresión coincidirán con la recta de dependencia.\\

Algunas propiedades de la recta de regresión son:
\begin{lema}
    Las rectas de regresión pasan por el punto $(\bar{x}, \bar{y})$.
\end{lema}
\begin{proof}
    La recta de regresión de $Y$ sobre $X$ tiene la forma de $y-\bar{y} = K(x-\bar{x})$. Para $x=\bar{x}$, vemos que $y=\bar{y}$.

    Análogamente, la recta de regresión de $X$ sobre $Y$ tiene la forma de $x-\bar{x} = K(y-\bar{y})$. Para $y=\bar{y}$, vemos que $x=\bar{x}$.
\end{proof}

\begin{lema}\label{lema:2.8}
    La media de los valores ajustados coincide con la de los valores observados de la variable.
\end{lema}
\begin{proof}
    \begin{equation*}
        \overline{\hat{y}} = \sum_{i=1}^k \sum_{j=1}^p f_{ij}\hat{y_j}
        = \sum_{i=1}^k \sum_{j=1}^p f_{ij} (ax_i + b)
        = a\bar{x} + b = \bar{y}
    \end{equation*}
\end{proof}

\begin{coro}
    La media de los residuos vale 0.
\end{coro}
\begin{proof}
    \begin{equation*}
        \sum_{i=1}^k \sum_{j=1}^p f_{ij}e_{ij}
        = \sum_{i=1}^k \sum_{j=1}^p f_{ij} (y_j - \hat{y_j})
        =  \bar{y} - \overline{\hat{y}} = 0
    \end{equation*}
\end{proof}

\begin{coro}
    La media de los productos de los residuos por los valores de la variable explicativa vale cero.
\end{coro}
\begin{proof}
    \begin{equation*}
        \sum_{i=1}^k \sum_{j=1}^p f_{ij} e_{ij}x_i = \bar{x} \sum_{i=1}^k \sum_{j=1}^p f_{ij}e_{ij} = 0
    \end{equation*}
\end{proof}

\begin{coro}
    La media de los productos de los residuos por los valores ajustados vale cero.
\end{coro}
\begin{proof}
    \begin{equation*}
        \sum_{i=1}^k \sum_{j=1}^p f_{ij} e_{ij}\hat{y_j} = \bar{\hat{y}} \sum_{i=1}^k \sum_{j=1}^p f_{ij}e_{ij} = 0
    \end{equation*}
    donde se ha aplicado la Proposición \ref{prop:1.3}.
\end{proof}


\subsubsection{Ajuste polinómico}\vspace{-0.5cm}
\begin{equation*}
    Y=f(X;a_0,a_1,\dots,a_n) = a_0 + a_1X + \dots + a_nX^n
\end{equation*}

Si queremos aproximar mediante un polinomio de grado superior o igual a dos, el método
de mínimos cuadrados nos conducirá al sistema de ecuaciones:
\begin{equation*}
    \left\{
    \begin{array}{cl}
        m_{01} &= a_0 + a_1 m_{10} + \ldots + a_n m_{n0}\\
        m_{11} &= a_0m_{10} + a_1 m_{20} + \ldots + a_n m_{n+1,0}\\
        m_{21} &= a_0m_{20} + a_1 m_{30} + \ldots + a_n m_{n+2,0}\\
        &\vdots\\
        m_{n1} &= a_0 m_{n0} + a_1 m_{n+1,0} + \ldots + a_n m_{2n,0}
    \end{array}
    \right.
\end{equation*}



Para ajustar a la nube otro tipo de función, intentaremos pasar a un ajuste polinómico. Ejemplos de esto son los siguientes ajustes:

\subsubsection{Ajuste hiperbólico}\vspace{-0.5cm}
\begin{equation*}
    Y=f(X;a,b) = a+b\frac{1}{X}
\end{equation*}


Si queremos realizar un ajuste hiperbólico mediante una hipérbola equilátera realizamos la transformación $Z = \dfrac{1}{X}$, y realizamos el ajusta de mínimos cuadrados a la recta $Y = a + bZ$ sobre
las variables $(Z,Y)$.

\subsubsection{Ajuste potencial}\vspace{-0.5cm}
\begin{equation*}
    Y=f(X;a,b) = aX^b
\end{equation*}

De otra forma, si queremos aplicar el ajuste potencial hemos de aplicar el logaritmo y obtenemos la siguiente expresión:
$$\ln Y = \ln a + b \ln X$$

Llamando a las variables $V = \ln Y$, $U = \ln X$, $A = \ln a$, quedándonos la siguiente expresión a calcular el ajuste lineal: $$V= A + bU$$

\subsubsection{Ajuste exponencial}\vspace{-0.5cm}
\begin{equation*}
    Y=f(X;a,b) = ab^{x}
\end{equation*}

De otra forma, si queremos aplicar el ajuste exponencial hemos de aplicar el logaritmo y obtenemos la siguiente expresión:
$$\ln Y = \ln a + X \ln b$$

Llamando a las variables $V = \ln Y$, $A = \ln a$ y $B = \ln b$, quedándonos la siguiente expresión a calcular el ajuste lineal: $$V= A + BX$$

\subsection{Regresión de tipo I}

Podemos además realizar regresiones de una variable dependiente $Y$ dado el valor $x_i$ de una variable independiente
asociada $X$. Es decir, predecir el comportamiento de la variable condicionada $Y/X=x_i$.\\


Teniendo en cuenta la representatividad de le media en lo que al comportamiento de una variable se refiere, se define la curva de regresión de tipo I de $Y/X$ como la curva que pasa por los puntos $(x_i, \overline{y_i}) \
    \forall i \in \{1, \ldots, k\}$. Análogamente, se defien la curva de regresión de tipo I de $X/Y$ como la curva
que pasa por los puntos $(\overline{x_j}, y_j) \ \forall j \in \{1, \ldots, p\}$.

Estas curvas tienen la propiedad de ser entre todas las funciones las que mejor se ajustan a los datos observados
según el método de mínimos cuadrados. Estas curvas no son de gran utilidad práctica, pues el hecho de conocerla
solamente en puntos aislados hace que sea inútil para le predicción en los demás casos.

\section{Correlación}

El grado de asociación entre las variables nos indicará en qué medida la expresión encontrada mediante la regresión explica una variable en función de la otra. El estudio de la correlación también equivale al estudio de la bondad del ajuste de una curva a una nube de puntos.

Para ello, en primer lugar es importante diferenciar entre los ajustes lineales en los parámetros y los no lineales en los parámetros.

Los que sí son lineales en los parámetros son aquellos a los que no se les ha aplicado ninguna transformación a los parámetros. Ejemplo de estos son los ajustes mediante rectas, parábolas o hipérbolas equiláteras.

Los no lineales en los parámetros implican que a alguno de los parámetros se le ha aplicado alguna transformación. Ejemplos son el ajuste potencial o el exponencial.

\subsection{Varianza residual. Coeficiente de determinación}

El método de mínimos cuadrados toma como medida del error que se comente al ajustar una curva la siguiente medida:
\begin{definicion}[Varianza Residual]

    Se define la varianza residual del ajuste de $Y$ en función de $X$ como:
    $$\sigma_{ry}^2 = \sum_{i=1}^k \sum_{j=1}^p f_{ij} e_{ij}^2 = \sum_{i=1}^k \sum_{j=1}^p f_{ij}(y_j - \hat y_i)^2 = \sum_{i=1}^k \sum_{j=1}^p
    f_{ij} [y_j - f(x_i)]^2$$
\end{definicion}

Dicha cantidad se usa como medida de la bondad del ajuste. Por tanto, cuanto menor sea la varianza resiudal, mejor será el ajuste.
\begin{observacion}
    En funciones lineales de los parámetros la media de los residuos es cero (generalización del lema \ref{lema:2.8}), por lo que la expresión anterior es precisamente la varianza de los residuos, o \underline{varianza residual}.
    
    En funciones no lineales en los
    parámetros, la media de los residuos no es nula, aunque se sigue denominando varianza residual. Por ello, es importante no confundir la varianza residual con la varianza de los residuos en los ajustes no lineales en los parámetros.
\end{observacion}\bigskip


También se define la siguiente medida:
\begin{definicion}[Varianza Explicada]

    Se define la varianza residual de $Y$ como:
    \begin{equation*}
        \sigma_{ey}^2=\sum_{i=1}^k \sum_{j=1}^p f_{ij}
    (\hat y_j - \overline{y})^2
    \end{equation*}
\end{definicion}


Por norma general, se toma como medida del grado de ajuste el coeficiente de determinación.
\begin{definicion}[Coeficiente de determinación]
El coeficiente de determinación, o razón de correlación, es la proporción de la varianza total de la variable $Y$ explicada por la regresión. Esto es, el cociente o razón entre la varianza explicada y la total.
\begin{equation*}
    \eta_{Y/X}^2 = \dfrac{\sigma_{ey}^2}{\sigma_y^2}
\end{equation*}    
\end{definicion}


Por tanto, para comparar todo tipo de ajustes se puede emplear la \textbf{varianza residual} o el \textbf{coeficiente de determinación}, aunque se suele emplear la primera medida.

\subsubsection{Caso concreto de ajustes lineales en los parámetros}

En este caso, como la varianza residual coincide con la varianza de los residuos, se tiene que es posible descomponer la varianza en una suma de la varianza residual y la varianza explicada por la regresión:
$$\sigma_y^2 = \sigma_{ey}^2 + \sigma_{ry}^2$$

En este caso, se tiene que:
$$\eta_{Y/X}^2 := \dfrac{\sigma_{ey}^2}{\sigma_y^2} = 1 - \dfrac{\sigma_{ry}^2}{\sigma_y^2}$$


\subsubsection{Interpretación del coeficiente de correlación}

De la misma expresión se deduce que: $0 \leq \eta_{Y/X}^2 \leq 1$.

\begin{itemize}
    \item $\eta_{Y/X}^2 = 0 \Leftrightarrow \dfrac{\sigma_{ey}^2}{\sigma_y^2}=0 \Leftrightarrow \sigma_{ey}^2 = 0$. Es decir, el modelo no explica nada de $Y$ a partir de $X$. El ajuste es el peor posible que se puede hacer por mínimos cuadrados.
    \item $\eta_{Y/x}^2 = 1 \Leftrightarrow \dfrac{\sigma_{ey}^2}{\sigma_y^2}=1$. Es decir, todos los residuos son nulos y s explica la variable totalmente. El ajuste es perfecto.
    \item Para valores intermedios entre 0 y 1, según estén más próximos a un extremo o a otro nos indicarán un peor o mejor ajuste: Un ajute del 60\% explica que el 60\% de la variabilidad total de $Y$ la explica el modelo propuesto mediante la variable independiente.
\end{itemize}

\subsection{Correlación en el caso lineal}

En este caso, tenemos el siguiente resultado, muy útil para calcular la bondad de los ajustes lineales:
\begin{teo} En el caso de un ajuste lineal, el ajuste de determinación viene dado por:
    \begin{equation*}
        \eta_{Y/X}^2 = \eta_{X/Y}^2 = r^2 = \dfrac{\sigma_{xy}^2}{\sigma_x^2 \sigma_y^2}
    \end{equation*}
\end{teo}
\begin{proof}
    Demostramos para la recta de $Y$ sobre $X$, ya que en el otro caso sería análogo. La recta mencionada tiene por expresión:
    \begin{equation*}
        Y-\overline{y} = \dfrac{\sigma_{xy}}{\sigma_x^2}(X - \overline{x}) 
        \Longrightarrow
        Y = \overline{y} + \dfrac{\sigma_{xy}}{\sigma_x^2}(X - \overline{x})
    \end{equation*}

    Por tanto, la varianza residual en el caso de la recta de regresión es:
    \begin{equation*}\begin{split}
        \sigma_{ry}^2 &
        = \sum_{i=1}^k\sum_{j=1}^p f_{ij}[y_j - f(x_i)]^2
        = \sum_{i=1}^k\sum_{j=1}^p f_{ij}\left[y_j - \left(\overline{y} + \dfrac{\sigma_{xy}}{\sigma_x^2}(x_i - \overline{x})\right)\right]^2 = \\
        & = \sum_{i=1}^k\sum_{j=1}^p f_{ij}\left[(y_j - \overline{y}) - \left(\dfrac{\sigma_{xy}}{\sigma_x^2}(x_i - \overline{x})\right)\right]^2 = \\
        & = \sum_{i=1}^k\sum_{j=1}^p f_{ij}\left[(y_j - \overline{y})^2 + \dfrac{\sigma_{xy}^2}{(\sigma_x^2)^2}(x_i - \overline{x})^2 - 2\frac{\sigma_{xy}}{\sigma_x^2}(y_j-\overline{y})(x_i-\overline{x})\right] = \\
        &= \sigma_y^2 + \frac{\sigma_{xy}^2}{\sigma_x} - 2\frac{\sigma_{xy}^2}{\sigma_x^2}
        = \sigma_y^2 - \frac{\sigma_{xy}^2}{\sigma_x^2}
    \end{split}\end{equation*}

    Por tanto, como en este caso estamos ante un ajuste lineal en los parámetros, tenemos que:
    \begin{equation*}
        \eta_{Y/X}^2 = 1-\frac{\sigma_{ry}^2}{\sigma_y^2}
        = 1-\frac{\sigma_y^2 - \frac{\sigma_{xy}^2}{\sigma_x^2}}{\sigma_y^2}
        = 1-1+\frac{\sigma_{xy}^2}{\sigma_x^2\sigma_y^2} = \frac{\sigma_{xy}^2}{\sigma_x^2\sigma_y^2}
        \qedhere
    \end{equation*}
\end{proof}

Este resultado es de gran ayuda, ya que nos permite calcular $\eta_{Y/X}^2$ de una forma mucho más cómoda.

Es útil calcular la varianza residual en función de $r^2$, ya que el cálculo del segundo es mucho más sencillo. No obstante, es necesario a veces conocer la varianza residual para comparar con modelos no lineales en los parámetros. Por eso, se tiene que:
\begin{equation*}
    \eta_{Y/X}^2 = 1 - \dfrac{\sigma_{ry}^2}{\sigma_y^2}
    = r^2 \Longrightarrow \sigma_{ry}^2 = (1-r^2)\sigma_y^2
\end{equation*}


Por último, se introduce un nuevo coeficiente. La raíz cuadrada del coeficiente de determinación lineal anterior (con el signo de la covarianza) recibe el nombre
de \underline{coeficiente de correlación lineal}:
$$r = \pm \sqrt{r^2} = \dfrac{\sigma_{xy}}{\sigma_x \sigma_y}$$


Dicho coeficiente se usa para determinar el grado de dependencia lineal de la variable dependiente ante los valores
de la variable independiente. Esta dependencia puede ser directa (o positiva) o indirecta (o negativa), según
el signo de la covarianza. Adopta valroes entre $\pm1$ y 0.

\begin{itemize}
    \item \underline{Para la covarianza positiva}:

    Si $r=1$, existirá una dependencia lineal funcional, mientras que si $r=0$ no existirá ninguna dependencia o asociación entre las variables de tipo lineal, aunque sí puede haberla de otra naturaleza, convirtiéndose las rectas de regresión paralelas a los ejes de coordenadas.

    \item \underline{Para la covarianza negativa}:
    Si $r=-1$, existirá una correlación perfecta, con una dependencia funcional lineal, coincidiendo las dos rectas en una sola.
\end{itemize}

Para resumir, diremos que $-1 \leq r \leq 1$. Cuando varía de $-1$ a $0$ estamos en una correlación negativa y la dependencia
será mayor cuanto más se aproxime a $-1$ mientras que si varía de $0$ a $1$, la correlación es positiva y el grado de
dependencia será mayor cuanto más se aproxime a $1$.

\section{Predicciones}
Uno de los objetivos principales de la regresión y correlación es hacer predicciones de la variable dependiente en función de los valores que toma la variable independiente. Las predicciones se efectúan utilizando la función estimada por el método de mínimos cuadrados, $f$. Con la que obtenemos los valores teóricos que ajustan a los observados. La predicción será más fiable cuanto mayor sean los coeficientes de determinación correspondientes o razones de correlación, ya que menor será la varianza de los residuos, que nos indica la cuantía de la separación entre lo observado y estimado.

Hay que tener presente que la fiabilidad de las predicciones disminuye a medida que los valores de la variable independiente se aleja de su recorrido, pues puede que el modelo ajustado no sea válido para dicho valores en la medida dada por $\eta^2$
    \chapter{Compilación y Enlazado de Programas}

\begin{center}
    Cód. fuente (Leng. alto nivel) $\xrightarrow{(*)}$
    Cód. objeto (Cód. máquina o ensamblador)
\end{center}

\begin{definicion} [Compilación] Proceso mediante el cual se pasa de código fuente a código objeto $(*)$. Se emplea para:
    \begin{itemize}
        \item Comprobar que no hay errores en el código fuente.
        \item Generar ficheros objeto.
    \end{itemize}
\end{definicion}

\begin{definicion}[Enlazado]
    Proceso mediante el cual, a partir de los ficheros objeto, se obtienen los ficheros ejecutables.
\end{definicion}
\section{Gramática}
\begin{definicion} [Gramática]
    La gramática $G=\{V_N, V_T, P, S\}$ está formada por:
    \begin{enumerate}
        \item \underline{$V_N$ o símbolos no terminales}:\\
        Aquellos símbolos auxiliares que podemos usar para operar con la gramática.

        \item \underline{$V_T$ o símbolos terminales}:\\
        Aquellos símbolos que podemos usar al programar.

        \item \underline{$P$ o reglas de producción}:\\
        Combinaciones válidas de los símbolos.

        \item \underline{$S$ o axioma}:\\
        Uno de los símbolos no terminales que se usa como símbolo inicial.
    \end{enumerate}
\end{definicion}

\begin{ejemplo}
    Sea $G=\left(\{0,1\},\{S\}, S, P \right)$, con $P$:
    \begin{equation*}
        P=\{S::=\footnote{Aquí se ha empleado la notación de Backus.} \;0 | 1 | 0S1\}
    \end{equation*}
    Por tanto, tengo tres reglas de producción.

    \begin{itemize}
        \item $S\longrightarrow 0$: Sí es válido, usando la primera regla.
        \item $S\longrightarrow 1$: Sí es válido, usando la segunda regla.
        \item $S\longrightarrow 101$: No es válido, ya que no es posible que empiece con el 1.
    \end{itemize}
\end{ejemplo}

\begin{ejercicio}
La gramática definida por $G=(\{0,1,2,3,4,5,6,7,8,9\},\{N,C\}, N, P)$, con:
    \begin{equation*}
        P = \{N::=NC | C,
        \quad C::=[0-9]\}
    \end{equation*}

Esta gramática puede generar los naturales, pero también admite los 0 no significativos. Modificar para que no admita los 0 no significativos.\\

Las reglas de producción serían:
\begin{equation*}
        P = \{N::=D | ND | N0,
        \quad D::=[1-9]\}
    \end{equation*}
\end{ejercicio}

\begin{definicion}[Gramática Ambigua] Una gramática es ambigua cuando admite más de un árbol sintáctico para una misma secuencia de símbolos de entrada.

Ejemplo de esto es la precedencia de los operadores en un lenguaje de programación, ya que se usan los paréntesis para evitar la ambigüedad.
\end{definicion}

\begin{definicion}[Gramática libre de contexto]
    Se dice que una gramática es libre de contexto cuando en el lado izquierdo de cada regla de producción solo puede haber un símbolo no terminar. Formalmente, cada producción es de la forma:
    \begin{equation*}
        A\longrightarrow \alpha \qquad A\in V_N \qquad \alpha \in (V_N\cup V_T)^\ast
    \end{equation*}
    donde $(V_N\cup V_T)^\ast$ representa todas las combinaciones posibles de dichos conjuntos.

    Se dice que es libre de contexto porque $A$ se puede sustituir por $\alpha$ independientemente del contexto en el que aparezca.
\end{definicion}





\section{Traducción}
\begin{definicion}[Traductor]
    Un traductor es un programa que recibe como entrada un texto en un lenguaje de programación concreto y produce, como salida, un texto en lenguaje máquina equivalente.
\end{definicion}

Existen dos tipos de traductores, los compiladores y los intérpretes.
\subsection{Compilador}
\begin{definicion}[Compilador]
    Un compilador traduce la especificación de entrada (archivos fuente) a lenguaje máquina incompleto (archivos objeto) y con instrucciones máquina incompletas. Por tanto, se necesita un complemento llamado enlazador.
\end{definicion}
\begin{definicion}[Enlazador/Linker]
    El linker completa los programas ligando las instrucciones máquina necesarias y genera un programa ejecutable para la máquina real.
\end{definicion}

\subsection{Intérprete}
\begin{definicion}[Intérprete]
    Un intérprete hace que un programa fuente escrito en un lenguaje vaya, sentencia a sentencia, traduciéndose y ejecutándose directamente por el computador. Cabe destacar que no se genera ningún archivo objeto ni equivalente al descrito en el compilador.
\end{definicion}

Algunas ventajas del intérprete son:
\begin{itemize}
    \item Es más fácil detectar errores, ya que suele ser posible detenerlo para conocer los valores de las variables. Esto solo sería posible en otro caso con un \textit{debugger}.

    \item Es más pedagógico.
\end{itemize}

No obstante, los inconvenientes son:
\begin{itemize}
    \item Cada vez que se ejecute, se ha de interpretar de nuevo, ya que no se genera archivo objeto. Con un compilador, aunque la traducción sea más lenta, solo ha de realizarse una vez.
    \item Una instrucción que se encuentre en un bucle se ha de interpretar tantas veces como se ejecute el bucle.
    \item La optimización solo se puede realizar línea a línea, no se puede realizar a nivel del programa completo.
\end{itemize}


Un ejemplo de intérprete es Bash.


\subsection{Fases en el proceso de la traducción}
En primer lugar, ocurre la fase de análisis del código. Este se realiza en tres pasos: léxico, sintáctico y semántico. Posteriormente, en la fase de síntesis, se optimiza y se genera el código objeto.
\begin{enumerate}
    \item \textbf{Fase de Análisis.}
    \renewcommand{\theenumii}{\theenumi.\arabic{enumii}}
    \begin{enumerate}
        \item \underline{Análisis Léxico}

        Para entender esta etapa, son necesarios los siguientes conceptos:
        
        \begin{definicion}[Lexema/Palabra] Es un conjunto de caracteres del alfabeto que tienen significado propio. 
        \end{definicion}

        \begin{definicion}[Token] Es el concepto asociado a un conjunto de lexemas o palabras que, según la gramática del lenguaje fuente, tienen la misma misión sintáctica.

        En el caso del lenguaje español, un token podría ser los determinantes artículos, ya que tienen la misma función sintáctica independientemente de la oración.

        En el caso de un lenguaje de programación, un token podría ser ``identificador'' asociado a los nombres de las variables o funciones u ``operador aritmético'' asociado a estos operadores.

        Los tokens asociados a más de una palabra (la mayoría) deben ir acompañados del lexema reconocido anteriormente. Este es el denominado \underline{atributo}, y es necesario para las fases posteriores de la traducción. Por ejemplo, el token que recoja los operadores es necesario que tenga el atributo del operador en sí, ya que a la hora de la traducción será necesario saber si se trata de una multiplicación o una división, por ejemplo.
        \end{definicion}

        \begin{definicion}[Patrón]
            Es una descripción de la forma que pueden tomar los lexemas de un token. Se suelen emplear expresiones regulares.
            
            En el caso de una palabra clave como token, el patrón es sólo la secuencia de caracteres que forman dicha palabra clave. Para los identificadores y algunos otros tokens, el patrón es una estructura más compleja (normalmente dada con una expresión regular).
        \end{definicion}
        

        \vspace{1cm}
        Por tanto, la \textbf{función} del analizador léxico es leer el texto de origen, identificar lexemas, asociarles el token al que pertenecen pero, además, debe eliminar los comentarios y caracteres superfluos existentes en el texto de entrada (espacios en blanco, tabuladores y retornos de carro). La equivalencia entre cada lexema con su token se guardan en la \textbf{tabla de símbolos}, que es donde su guarda la información.

        \begin{ejemplo} Ejemplos de tokens son:
        \begin{itemize}
            \item \textbf{IF}: lexema asociado \textit{if}.
            \item \textbf{ELSE}: lexema asociado \textit{else}.
            \item \textbf{IDENT}: lexemas asociados \textit{pi}, \textit{dato3} o \textit{i}, por ejemplo.
        \end{itemize}

        Se producirá un \textbf{error léxico} cuando el carácter de la entrada no tenga asociado a ninguno de los patrones disponibles en nuestra lista de tokens. Por ejemplo, un carácter extraño en la formación de una palabra reservada, como \textit{wh\textbf{ñ}le}.
            
        \end{ejemplo}



        \item \underline{Análisis Sintáctico}

        Tiene como objetivo analizar las secuencias de tokens y comprobar que son correctas sintácticamente.
        
        A partir de una secuencia de tokens el analizador sintáctico nos devuelve el orden en el que hay que aplicar las producciones de la gramática para obtener la secuencia de entrada; es decir, el árbol sintáctico abstracto en el que aparece el token con el atributo. Este se guarda también en la tabla de símbolos.

        Se produce un \textbf{error sintáctico} cuando no se puede llegar desde el axioma hasta la palabra buscada; es decir, no se puede construir el árbol sintáctico. Ejemplo de esto podría ser, por ejemplo, paréntesis mal balanceados.




        \item \underline{Análisis Semántico}

        La semántica de un lenguaje de programación es el significado dado a las distintas construcciones sintácticas. En los lenguajes de programación, el significado está ligado a la estructura sintáctica de las sentencias.

        En el caso de que no se produzcan errores, actualiza en la tabla de símbolos el árbol semántico abstracto resuelto; es decir, el árbol semántico con los lexemas y su significado.

        Se producen \textbf{errores semánticos} cuando se detectan construcciones sin un significado correcto (p.e. variable no declarada, tipos incompatibles en una asignación, llamada a un procedimiento con número de argumentos incorrectos, \dots).
    \end{enumerate}
    \item \textbf{Fase de Síntesis}
    \begin{enumerate}
        \item \underline{Generación de código}
        En esta fase se genera un archivo con un código en lenguaje objeto (generalmente lenguaje máquina) con el mismo significado que el texto fuente.

        Es posible la generación de código intermedio para facilitar el proceso.

        \item \underline{Optimización de código}

        Esta fase existe para mejorar el código mediante comprobaciones locales a un grupo de instrucciones (bloque básico) o a nivel global. Se suele realizar en el código intermedio.

        Ejemplo de esto es una asignación del tipo $b=7.3$ dentro de un bucle $for$. En esta fase se sacará dicha instrucción del bucle.
    \end{enumerate}
\end{enumerate}


\section{Modelos de Memoria de un Proceso}
\subsection{Tipos de Datos (desde el punto de vista de su implementación en memoria)}

\begin{itemize}
    \item Datos estáticos - existen a lo largo de toda la vida del programa.
    \begin{itemize}
        \item \underline{Según el ámbito de visibilidad}.

        Las globales a todo el programa se encuentran fuera del main y no dependen de ningún archivo.
        
        Las de módulo se especifican con \verb|static| (del módulo en cuestión) o \verb|extern| de otro módulo, siendo un módulo cada uno de los archivos.
        
        Las de bloque pertenecen a una parte del programa delimitada por las llaves.

        Los datos estáticos de clase o función pertenecen a cada clase o función en concreto.

        \item \underline{Constantes o variables}.

        Según si se pueden modificar o no. Las constantes se almacenan en un espacio de la memoria de solo lectura, mientras que las variables han de poder modificarse también.

        \item \underline{Con o sin valor inicial}.
    \end{itemize}

    \item Datos dinámicos asociados a la ejecución de una función:

    Se almacenan en la pila \textit{(stack)} y se crean al ser llamada la función y se destruyen cuando esta termina.

    Corresponde a los datos locales y a los parámetros de una función.


    \item Datos dinámicos controlados por el programa.

    Se almacenan en el \textit{heap} (zona de memoria usada tiempo de ejecución para albergar los datos no conocidos en tiempo de compilación). En C++, se controlan mediante los operadores \verb|new| y \verb|delete|. Generalmente se gestionan mediante punteros.

    Su tiempo de vida no esta vinculado a la activación de una función sino bajo el control directo del programa que los crea cuando los necesita.
\end{itemize}

\begin{definicion}[Código Independiente de la Posición \textit{(PIC, Position Independent Code)}]

Un fragmento de código cumple esta propiedad si puede ejecutarse en cualquier parte de la memoria.

Es necesario que todas sus referencias a instrucciones o datos no sean absolutas sino relativas en función del valor de un registro.

Por ejemplo, una instrucción puede ser \verb|MOV R4, 32+PC|. Dependiendo del valor del \verb|PC|, se almacenará el valor de $R4$ en posiciones distintas.
\end{definicion}



\section{Ciclo de Vida de un Programa}

Una vez que el programador ha finalizado la escritura del programa, éste debe pasar por varias fases antes de que pueda ejecutarse. Estas son:
\begin{enumerate}
    \item Preprocesado (archivos \verb|.i| en C).

    Se procesan los includes, las directivas de preprocesador, etc.

    \item Compilación (archivos \verb|.s| en C).

    Se genera el código en lenguaje ensamblador.

    \item Ensamblado (archivos \verb|.o| en C).

    El código ensamblador se traduce por el \textit{assembler} a código máquina. Las referencias a símbolos que no están definidos en el módulo quedan pendientes de resolver.

    \item Enlazado (archivos .exe y a.out).

    El enlazador/\textit{linker} se encarga de, a partir de los archivos objeto y las bibliotecas, resolver las referencias pendientes y generar el archivo ejecutable.

    Como diferencia entre los archivos ejecutables y los archivos objeto, tenemos principalmente que el ejecutable contiene una cabecera en la que se indica el punto de inicio del mismo, es decir, la primera dirección que se cargará en el PC.

    \item Carga y ejecución.

    Por tanto el cargador/\textit{loader} es el que ayuda a la asignación y carga del programa como un proceso en MP en estado nuevo.
\end{enumerate}

\subsection{Compilación}
En esta fase, el compilador procesa cada uno de los archivos de código fuente para generar el correspondiente archivo objeto. Se realizan las siguientes acciones:
\begin{enumerate}
    \item Genera el código objeto y determina cuanto espacio ocupan los diferentes tipos de datos.

    \item Asigna direcciones a los \underline{símbolos estáticos definidos en el módulo}.
    
    Estas son consecutivas: en primer lugar van las constantes, luego las variables con valor inicial, y por último las que no tiene valor inicial.

    Como son estáticas y permanecen durante todo el programa, se puede asignar la dirección directamente.

    \item Resuelve las referencias a los \underline{símbolos estáticos externos definidos en el módulo}.

    Al ser estáticos, en el proceso de compilación ya tienen una dirección asignada. Al ser del mismo módulo, no se requiere aún del enlazador.

    Las referencias pueden resolverse mediante un direccionamiento absoluto (será necesario reubicación) o relativo al $PC$ (estaremos ante un PIC).

    \item Las referencias a los símbolos estáticos externos definidos fuera del módulo se resolverán en el enlazado.

    \item Resuelve las referencias a los \underline{símbolos dinámicos almacenados en la pila}.

    Se resuelven mediante direccionamiento relativo a pila. Al no aparecer en el fichero objeto (se generan en tiempo de ejecución), no requieren reubicación.

    \item Resuelve las referencias a los \underline{símbolos dinámicos almacenados en el \textit{heap}}.

    Se resuelven mediante direccionamiento indirecto mediante punteros. Al no aparecer en el fichero objeto (se generan en tiempo de ejecución), no requieren reubicación.

    \item Genera la Tabla de símbolos e información de depuración.
\end{enumerate}


\subsection{Enlazado}
El enlazador \textit{(linker)} debe agrupar los archivos objetos de la aplicación y las bibliotecas, y resolver las referencias entre ellos. Concretamente, se llevan a cabo las siguientes tareas:
\begin{enumerate}
    \item Se completa la etapa de resolución de símbolos.

    \item Se agrupan en regiones las zonas de las mismas características de los módulos .

    Es decir, todas la parte de código de todos los módulos se agrupa en una región; y lo mismo ocurre con datos inicializados y los no inicializados.

    Esto reduce el número de regiones que ha de gestionar el Sistema Operativo.

    \item Se realiza la reubicación de módulos formando regiones, ya que hay que transformar las referencias dentro de un módulo a referencias dentro de las regiones.
\end{enumerate}

Hay distintos tipos de enlazado, que determinan el ámbito de cada dato.
\begin{itemize}
    \item \underline{Enlazado externo.}
    
    Cada vez que aparece un identificador con enlazado externo representa el mismo objeto o función a través del total de ficheros y librerías que componen el programa. Por tanto, esto equivale a tener visibilidad global.

    \item \underline{Enlazado interno.}
    
    Cada vez que aparece un identificador con enlazado interno representa el mismo objeto o función solo dentro del mismo fichero. Los objetos con el mismo nombre en otros ficheros son objetos distintos. Por tanto, esto equivale a tener visibilidad de fichero.

    \item \underline{Sin enlazado.}
    
    Cada identificador sin enlazar representa unidades únicas. Los objetos con el mismo nombre en otros bloques son objetos distintos. Por tanto, esto equivale a tener visibilidad de bloque. Identificadores sin enlazado son:
    \begin{itemize}
        \item Cualquier identificador distinto a un objeto o función.
        \item Parámetros de funciones.
        \item Objetos de ámbito de bloque (entre llaves) sin el especificador \verb|extern|. 
    \end{itemize}
\end{itemize}


\section{Bibliotecas}
\begin{definicion}[Biblioteca] Colección de objetos, normalmente relacionados entre sí. Favorecen modularidad y reusabilidad de código.
\end{definicion}

Las bibliotecas se pueden clasificar según la forma en la que se enlazan:
\begin{enumerate}
    \item \underline{Bibliotecas Estáticas}

    Tienen extensión \verb|.a|, y que se ligan con el programa en el proceso de enlazado. El archivo ejecutable las contiene.

    Algunos inconvenientes que tiene son:
    \begin{itemize}
        \item El código de la biblioteca está en todos los ejecutables que la usan, lo que desperdicia disco y memoria principal.

        \item Si actualizamos una biblioteca estática, debemos recompilar los programas que la usan para que se puedan beneficiar de la nueva versión.

        \item Producen ejecutables grandes.
    \end{itemize}

    \item \underline{Bibliotecas Dinámicas}

    Por norma general, tienen extensión \verb|.so| y se integran con los procesos en tiempo de ejecución. En el proceso de montaje, se incluye un módulo de montaje dinámico (\textit{enlazador dinámico}) que se encarga de cargar y montar las bibliotecas dinámicas usadas por el programa durante su ejecución.

    El archivo correspondiente a una biblioteca dinámica se diferencia de un archivo ejecutable en los siguientes aspectos:
    \begin{enumerate}
        \item Contiene información de reubicación.
        \item Contiene una tabla de símbolos propios de la biblioteca.
        \item En la cabecera no se almacena información de punto de entrada.
    \end{enumerate}
\end{enumerate}
    \chapter{Aproximación}
\section{Motivación}
\noindent
El objetivo de esta sección es el de, dada una función $f(x)$, aproximarla mediante otra función de forma
que minimicemos el área que forman las gráficas de ambas funciones. Para evitar el realizar integrales con
valores absolutos, lo sustituiremos por elevar el área al cuadrado.\\

\noindent
\textbf{Problema de aproximación.} Dados $N$ puntos $\{(x_0, y_0), (x_1, y_1), \ldots, (n_N, y_N)\}$, el problema de aproximar
(ajustar) dichos puntos consiste en encontrar una función $g(x)$, con unas ciertas condiciones, que esté lo más cerca posible
de los puntos.\newline
Esto es, se busca una función $g(x)$ que minimice la distancia a los puntos $(x_i,y_i)~~i=0,1,\ldots, N$.\\

\noindent
Al igual que pasaba con la interpolación, buscamos funciones $g(x)$ que posean propiedades deseables:
\begin{itemize}
    \item Fáciles de implementar.
    \item Fáciles de evaluar.
    \item Simples de calcular.
    \item Suficientemente regulares.
    \item \ldots
\end{itemize}

\noindent
En los siguientes apartados formalizaremos los conceptos de distancia y aproximación.

\section{Producto escalar}
\begin{definicion}[Producto escalar]
    Sea $V$ un espacio vectorial, definimos un producto escalar (o forma bilineal simétrica definida positiva)
    como una aplicación:
    $$\begin{array}{ccccc}
            \langle\cdot,\cdot\rangle & : & V \times V & \longrightarrow & \R                  \\
                                      &   & (u,v)      & \longmapsto     & \langle u,v \rangle
        \end{array}$$

    \noindent
    Que cumple las siguientes propiedades:
    \begin{enumerate}
        \item $\langle v,v\rangle \geq 0~~~~\forall v \in V$
        \item $\langle v,w\rangle = \langle w,v\rangle~~~~\forall v,w \in V$
        \item $\langle \alpha v + \beta w, z\rangle = \alpha \langle v,z\rangle + \beta \langle w,z \rangle~~~~\forall \alpha, \beta \in \R, v,w,z \in V$
        \item $\langle v,v\rangle = 0 \Leftrightarrow v=0 \  \forall v \in V$
    \end{enumerate}

    \noindent
    A un espacio con un producto escalar, $(V, \langle\cdot,\cdot\rangle)$ se le denomina \textbf{espacio con producto escalar}.
\end{definicion}


\begin{ejemplo}Es fácil ver que los siguientes son productos escalares:
\begin{enumerate}
    \item Producto escalar en $\R^n$:

    Sean $u = (u_1, u_2, \ldots, u_n), v=(v_1, v_2, \ldots, v_n) \in V$

    Definimos $\langle u,v\rangle = u_1 v_1 + u_2 v_2 + \ldots + u_n v_n$

    \item Sea $[a,b] \subset \R$ con $a \neq b$, $V=\mathcal{C}([a,b])$ el espacio vectorial de las funciones continuas en $[a,b]$.
    Definimos $\forall f,g \in V$:
    $$\langle f,g\rangle = \int_a^b f(x)g(x)~dx$$

    \item Sea $V=\mathcal{C}([a,b])$, tratamos de definir un producto escalar de la siguiente forma:
    Sean $x_i \in [a,b]~~\forall i \in \{1, \ldots, N\}$. Definimos $\forall f,g \in V$:
    $$\langle f,g\rangle = \sum_{i=1}^N f(x_i)g(x_i)$$
    
    \noindent
    Nos centraremos en comprobar si este producto escalar es válido para un espacio de polinomios.\newline
    Notemos que $\langle f,f \rangle = 0 \Leftrightarrow \sum\limits_{i=1}^N f(x_i)f(x_i)=0 \Leftrightarrow x_i$ es raíz de $f$
    $\forall i \in \{1, \ldots, N\}$\\
    
    \noindent
    Por lo que en $\bb{P}_k$ con $k \geq N \Rightarrow$ podemos tener:
    $$f(x) = \prod_{i=1}^N(x-x_i)$$
    Con $f\neq0 \ \land \ \langle f,f \rangle = 0$, que no define un producto escalar.\\
    
    \noindent
    Sin embargo, en los espacios $\bb{P}_k$ con $k < N$, nuestra aplicación sí define un producto escalar, ya que:
    $$\langle f,f \rangle = 0 \Leftrightarrow \sum_{i=1}^N (f(x_i))^2 = 0 \Leftrightarrow f=0$$
    (Es rutinario comprobar que el producto escalar así definido verifica además las propiedades 1), 2) y 3), por lo que se deja
    al lector a modo de ejercicio su demostración).
\end{enumerate}
\end{ejemplo}

\bigskip
\section{Norma}
\begin{definicion}[Norma]
    Una norma es una aplicación:
    $$\begin{array}{ccccc}
            \|\cdot\| & : & V & \longrightarrow & \R    \\
                      &   & v & \longmapsto     & \|v\|
        \end{array}$$
    Que cumple las siguientes propiedades:
    \begin{enumerate}
        \item $\|v\| \geq 0~~~~\forall v \in V$
        \item $\|v+w\| \leq \|v\|+\|w\|~~~~\forall v,w \in V$
        \item $\|\lambda v\| = |\lambda| \|v\|~~~~\forall \lambda \in \R~~\forall v \in V$
        \item $\|v\|=0 \Leftrightarrow v=0$
    \end{enumerate}

    \noindent
    Un espacio vectorial en el que hay definida una norma se denomina \textbf{espacio vectorial normado}.
\end{definicion}

\begin{teo}[Desigualdad de Cauchy-Schwarz]
    Sea $(V, \langle \cdot , \cdot \rangle)$ un espacio con producto escalar, se verifica:
    $$|\langle u,v \rangle| \leq \|u\|\|v\|~~~~\forall u,v \in V$$
\end{teo}
\begin{proof}
    Sea $\lambda \in \R$, $\forall u,v \in V \Rightarrow \lambda u + v \in V$.
    $$0 \leq \|\lambda u + v\|^2 = \langle \lambda u + v, \lambda u + v \rangle = \lambda^2 \langle u,u \rangle + 2\lambda
        \langle u,v \rangle + \langle v,v \rangle=$$
    $$= \lambda^2 \|u\|^2 + 2\lambda \langle u,v \rangle + \|v\|^2 \geq 0 ~~\forall \lambda \in \R \Leftrightarrow \Delta \leq 0 $$
    $$\Delta = 4\langle u,v \rangle^2 - 4\|u\|^2 \|v\|^2 \leq 0 \Leftrightarrow \langle u,v \rangle^2 \leq \|u\|^2\|v\|^2 \Leftrightarrow$$
    $$\Leftrightarrow \sqrt{\langle u,v \rangle^2} \leq \sqrt{\|u\|^2\|v\|^2} \Leftrightarrow |\langle u,v \rangle| \leq \|u\|\|v\|$$
\end{proof}

\begin{ejemplo} La aplicación de dicha desigualdad con diferentes productos escalares es:
\begin{enumerate}
    \item Producto escalar en $\R^n$:
$$\forall u=(u_1, \ldots, u_n), v=(v_1, \ldots, v_n) \in V~~~~\langle u,v \rangle^2 \leq \|u\|^2\|v\|^2$$
$$(u_1v_1 + u_2v_2 + \ldots + u_nv_n)^2 \leq (u_1^2 + u_2^2 + \ldots + u_n^2)(v_1^2 + v_2^2 + \ldots + v_n^2)$$

    \item \item Sean $f,g\in \cc{C}([a,b])$. Se define $\langle f,g\rangle = \int_a^b f(x)g(x)dx$.

    La desigualdad se escribiría:
    \begin{equation*}
        \left[\int_a^b f(x)g(x)dx\right]^2 \leq \left[\int_a^b f^2(x)dx\right]\left[\int_a^b g^2(x)dx\right]
    \end{equation*}

Tomando $f=1$:
$$\left[ \int_a^b g(x)~dx \right]^2 \leq \int_a^b 1~dx \int_a^b (g(x))^2~dx = (b-a) \int_a^b (g(x))^2~dx$$
\end{enumerate}
    
\end{ejemplo}

\begin{ejercicio}
    Demostrar la desigualdad triangular desde la desigualdad de Cauchy-Schwarz.
    \begin{multline*}
        ||u+v||^2 = \langle u+v,u+v \rangle = ||u||^2 + ||v||^2 +2\langle u,v\rangle \leq ||u||^2 + ||v||^2 +2|\langle u,v\rangle| {\leq}\\ \stackrel{C-S}{\leq} ||u||^2 + ||v||^2 +2||u||||v|| = (||u|| + ||v||)^2
    \end{multline*}

    Tomando raíces cuadradas, tenemos que:
    \begin{equation*}
        ||u+v|| \leq ||u|| + ||v||
    \end{equation*}
\end{ejercicio}

\begin{teo}[Norma inducida]
    En todo espacio vectorial con producto escalar $(V,\langle \cdot,\cdot \rangle)$, podemos definir una norma como sigue:\newline
    Sea $v \in V$, definimos su norma como:
    $$\|v\| = \sqrt{\langle v, v\rangle} \in \R_0^{+}$$
\end{teo}
\begin{proof}
    Claramente la norma así definida es una aplicación $\|\cdot\|:V \rightarrow~\R$.Veamos que cumple las propiedades
    mencionadas en la definición:
    \begin{enumerate}
        \item $$\|v\| = \sqrt{\langle v,v \rangle} \geq 0~~~~\forall v \in V$$
        \item $$\forall u,v \in V: \|u+v\|^2 = \langle u+v, u+v \rangle = \|u\|^2 + 2\langle u,v \rangle + \|v\|^2 \leq$$

        Luego: $$\|u+v\| \leq \|u\|+\|v\|$$
        \item $\forall \lambda \in \R$,$\forall v \in V$:
    $$\|\lambda v\| = \sqrt{\langle \lambda v,\lambda v \rangle} = \sqrt{\lambda^2 \langle v,v \rangle} = |\lambda| \sqrt{\langle v,v \rangle}
        = |\lambda| \|v\|$$

        \item $$\|v\| = 0 \Leftrightarrow \sqrt{\langle v,v \rangle} = 0 \Leftrightarrow \langle v,v \rangle = 0 \Leftrightarrow v=0$$
    \end{enumerate}
\end{proof}

\section{Distancia}
\begin{definicion}[Distancia]
    Una distancia es una aplicación:
    $$\begin{array}{ccccc}
            d & : & V \times V & \longrightarrow & \R     \\
              &   & (u,v)      & \longmapsto     & d(u,v)
        \end{array}$$
    Que cumple las siguientes propiedades:
    \begin{enumerate}
        \item $d(u,v) \geq 0~~~~\forall u,v \in V$
        \item $d(u,v) = d(v,u)~~~~\forall u,v \in V$
        \item $d(u,v) \leq d(u,w) + d(w,v)~~~~\forall u,v,w \in V$
        \item $d(u,v)=0 \Leftrightarrow u=v$
    \end{enumerate}
    
    \noindent
    Un espacio vectorial en el que hay definida una distancia se denomina \textbf{espacio vectorial métrico}.
\end{definicion}

\begin{teo}[Distancia inducida]
    Sea $(V,\langle \cdot,\cdot \rangle)$ un espcio vectorial con producto escalar, podemos definir una distancia como sigue:
    $$d(u,v) = \|u-v\| = \sqrt{\langle u-v,u-v \rangle}~~~~\forall u,v \in V$$
    Aplicando este teorema y el anterior deducimos que todo espacio vectorial con producto escalar $(V,\langle \cdot, \cdot \rangle)$
    es normado y, por tanto, métrico.
\end{teo}
\begin{proof}
    Claramente la distancia así definida es una aplicación $d:V\times V \rightarrow~\R$. Veamos que cumple las propiedades mencionadas en la definición:
    \begin{enumerate}
        \item $$d(u,v) = \|u-v\| \geq 0~~~~\forall u,v \in V$$
        \item $$d(u,v) = \|u-v\| = |-1| \|u-v\| = \|v-u\| = d(v,u)~~~~\forall u,v \in V$$
        \item \begin{multline*}
            d(u,v) = \|u-v\| = \|u-w+w-v\| \leq \|u-w\| + \|w-v\| = d(u,w) + d(w,v)\\\forall u,v,w \in V
        \end{multline*}
    \item $$d(u,v)=0 \Leftrightarrow \|u-v\| = 0 \Leftrightarrow u-v=0 \Leftrightarrow u=v$$
    \end{enumerate}
\end{proof}

\section{Mejor aproximación}
\begin{definicion}[Mejor aproximación]
    Sea $(V, \langle \cdot, \cdot \rangle)$ y $U \subset V$ un subconjunto de $V$.
    Sea $f \in V$. Se dice que $u\in U$ es una mejor aproximación (m.a.) de $f$ en $U$ sii:
    $$d(f,u) =: d(f,U) = \|f-u\| = \inf\{d(f,v) \mid v \in U\}$$
\end{definicion}

\bigskip
\noindent
Nos planteamos a continuación las siguientes cuestiones:
\begin{itemize}
    \item ¿Existe siempre la mejor aproximación?\newline
          No, en el caso de un círculo sin la circunferencia, no existe la mejor aproximación a un punto exterior:
          $$U = \{(x_1,x_2)\in \R^2 \mid x_1^2 + x_2^2 < 1\}~~~~f \in V\setminus U$$
    \item ¿Es única la mejor aproximación?\newline
          No, por ejemplo, si consideramos uan circunferencia, la mejor aproximación a su centro es cada uno de los puntos que
          componen la circunferencia y, por tanto, hay infinitas mejores aproximaciones:
          $$U = \{(x_1, x_2)\in \R^2 \mid x_1^2 + x_2^2 = 1 \}~~~~f = (0,0)$$
\end{itemize}

\noindent
Notemos que minimizar el conjunto (Sea $U$ un subconjunto de $V$)
$$\{d(f,v) \mid v \in U\} = \{\sqrt{\langle f-v, f-v \rangle} \mid v \in U\}$$
es equivalente a minimizar el conjunto:
$$\{d(f,v)^2 \mid v \in U\} = \{\langle f-v, f-v \rangle \mid v \in U\}$$

\noindent
Que es más sencillo de tratar ante la ausencia de la raíz. A este problema se le llama \textbf{aproximación por mínimos cuadrados}.

\begin{definicion}[Ortogonalidad]
    Sean $u,v \in V$, se dice que son ortogonales si:
    $$\langle u,v \rangle = 0$$
    Notado: $u \perp v$.
\end{definicion}

\begin{prop}[Teorema de Pitágoras]
    Sean $u,v\in (V,\langle,\rangle)$.
    \begin{equation*}
        \langle u,v\rangle = 0 \Longrightarrow ||u+v||^2 = ||u||^2 + ||v||^2
    \end{equation*}
\end{prop}
\begin{proof}
    Tenemos que:
    \begin{equation*}
        ||u+v||^2 = \langle u+v,u+v \rangle = ||u||^2 + ||v||^2 +2\cancelto{0}{\langle u,v\rangle} = ||u||^2 + ||v||^2
    \end{equation*}
\end{proof}

\begin{teo}[Caracterización de la mejor aproximación]
    Sea $V$ un espacio con producto escalar, y $U$ un subespacio de $V$. Dada $f\in V$, un elemento $u\in U$ es mejor aproximación de $f$ en $U$ si y solo si:
\begin{equation*}
    \langle f-u,w\rangle = 0\qquad \forall w\in V
\end{equation*}
\end{teo}
\begin{proof}
    Procedemos mediante doble implicación:
    \begin{description}
        \item [$\Longleftarrow$)] Para todo $v\in U$, se cumple:
        \begin{multline*}
            ||f-v||^2 = ||(f-u)+(u-v)||^2 = ||f-u||^2 + ||u-v||^2 +2\cancelto{0}{\langle f-u,u-v\rangle} = \\=
            ||f-u||^2 +||u-v||^2 \leq 0 \Longrightarrow ||f-v||^2 \geq ||u-v||^2 \qquad \forall v\in V
        \end{multline*}

        donde he aplicado que $U$ es un sucespacio vectorial, por lo que $u-v\in U$, y por tanto $\langle f-u,u-v\rangle = 0$ por hipótesis.

        Por tanto, tenemos que $||u-v||\leq ||f-v|| \qquad \forall v\in V$, por lo que $u$ es la mejor aproximación en $U$ de $f$.

        \item [$\Longrightarrow$)] Por ser $u$ la mejor aproximación de $f$, tenemos que:
        \begin{equation*}
            ||f-u||\leq ||f-w|| \Longrightarrow ||f-u||^2\leq ||f-w||^2 \qquad \forall w\in U
        \end{equation*}

        Tomamos $v\in U$, y sea $w=u+\lambda v \in U \mid \lambda \in \bb{R}$. Por tanto, como $w\in U$, tenemos que:
        \begin{equation*}
            ||f-u||^2\leq ||f-u-\lambda v||^2 = \langle f-u-\lambda v, f-u-\lambda v\rangle = ||f-u||^2 -2\lambda \langle f-u,v\rangle +\lambda^2 ||v||^2
        \end{equation*}

        Por tanto,
        \begin{equation*}
            0 \leq -2\lambda \langle f-u,v\rangle +\lambda^2 ||v||^2 
            = \lambda (\lambda ||v||^2 -2 \langle f-u,v\rangle) 
            \qquad \forall \lambda\in \bb{R}, \forall v\in U.
        \end{equation*}

        Las raíces de dicha parábola en la incógnita $\lambda \in \bb{R}$ son:
        \begin{equation*}
            \lambda_1 = 0 \qquad \lambda_2 = \frac{2\langle f-u,v\rangle}{||v||^2}
        \end{equation*}

        Por tanto, como es siempre $\geq 0$, tenemos que las dos raíces son iguales. Por tanto, $\langle f-u,v\rangle = 0$.
    \end{description}
\end{proof}


\subsection{Método para el cálculo de la mejor aproximación}
\begin{teo}[existencia y unicidad de la mejor aproximación]
    Sea $(V, \langle \cdot, \cdot \rangle)$ y $U \subseteq V$ subespacio vectorial de $V$ de dimensión finita, entonces la mejor
    aproximación existe y es única.
\end{teo}
\begin{proof}
    Buscamos $u \in U = \cc{L}\{\varphi_0, \varphi_1, \ldots, \varphi_m\} \Rightarrow \exists a_0, a_1, \ldots, a_m \in \R$ tal que:
    $u = a_0 \varphi_0 + a_1 \varphi_1 + \ldots + a_m \varphi_m$. Buscamos calcular $a_i~~\forall i \in \{0, \ldots, m\}$:\\

    \noindent
    Se tiene que $u$ es la mejor aproximación de $f \in V$ en $U$:
    $$\Leftrightarrow \langle f-u, v\rangle = 0~~\forall v \in U \Leftrightarrow \langle f-u, \varphi_k\rangle = 0~~\forall k \in \{0, \ldots, m\} \Leftrightarrow$$
    $$\Leftrightarrow \langle f-u, \varphi_k\rangle = \langle f-(a_0\varphi_0 + a_1\varphi_1 + \ldots + a_m\varphi_m),\varphi_k\rangle = $$
    $$ =\langle f,\varphi_k \rangle -a_0 \langle \varphi_0,\varphi_k \rangle - a_1\langle \varphi_1,\varphi_k \rangle - \ldots - a_m \langle
        \varphi_m,\varphi_k \rangle = 0 \Leftrightarrow$$
    $$\Leftrightarrow a_0 \langle \varphi_0,\varphi_k \rangle + a_1\langle \varphi_1,\varphi_k \rangle + \ldots - a_m \langle
        \varphi_m,\varphi_k \rangle = \langle f,\varphi_k \rangle~~\forall k \in \{0, \ldots, m\}$$
    Que nos da el siguiente sistema de ecuaciones lineales de $m+1$ ecuaciones y $m+1$ incógnitas:
    $$\left( \begin{array}{cccc}
                \langle \varphi_0,\varphi_0 \rangle & \langle \varphi_1,\varphi_0 \rangle & \ldots & \langle \varphi_m,\varphi_0 \rangle \\
                \langle \varphi_0,\varphi_1 \rangle & \langle \varphi_1,\varphi_1 \rangle & \ldots & \langle \varphi_m,\varphi_1 \rangle \\
                \vdots                              & \vdots                              & \ddots & \vdots                              \\
                \langle \varphi_0,\varphi_m \rangle & \langle \varphi_1,\varphi_m \rangle & \ldots & \langle \varphi_m,\varphi_m \rangle
            \end{array} \right) \left( \begin{array}{c}
                a_0    \\
                a_1    \\
                \vdots \\
                a_m
            \end{array} \right) = \left( \begin{array}{c}
                \langle f,\varphi_0 \rangle \\
                \langle f,\varphi_1 \rangle \\
                \vdots                      \\
                \langle f,\varphi_m \rangle
            \end{array} \right)$$
    A la matriz de coeficientes anterior se le llama matriz de Gram y verifica que es simétrica y definida positiva, por lo que
    el sistema anterior es compatible determinado, luego sabemos que los coeficientes $a_i~~i\in\{0, \ldots, m\}$ existen y que
    son únicos.
\end{proof}


\begin{ejemplo}
    Calcular la mejor aproximación de la función $f(x)=x^3$ en $\bb{P}_1$, utilizando el producto escalar definido como:
$$\langle v,u \rangle = \int_{-1}^1v(x)u(x)~dx~~~~\forall u,v \in V$$


\noindent
$\bb{P}_1 = \cc{L}\{1,x\}$, la mejor aproximación de $f$ será $u(x) = a_0 \cdot 1 + a_1x \in \bb{P}_1~~~a_0,a_1 \in \R$
$$\left( \begin{array}{cc}
            \langle 1,1 \rangle & \langle x,1 \rangle \\
            \langle 1,x \rangle & \langle x,x \rangle
        \end{array} \right) \left( \begin{array}{c}
            a_0 \\
            a_1
        \end{array} \right) = \left( \begin{array}{c}
            \langle x^3,1 \rangle \\
            \langle x^3,x \rangle
        \end{array} \right)$$
$$\langle 1,1 \rangle = 2$$
$$\langle 1,x \rangle = \langle x,1 \rangle = 0$$
$$\langle x,x \rangle = \dfrac{2}{3} $$
$$\langle x^3,1 \rangle = 0$$
$$\langle x^3,x \rangle = \dfrac{2}{5}$$

$$\left. \begin{array}{cccc}
        2a_0 &                 & = & 0            \\
             & \dfrac{2}{3}a_1 & = & \dfrac{2}{5}
    \end{array} \right\} \Rightarrow \left\{ \begin{array}{c}
        a_0 = 0 \\
        a_1 = \dfrac{3}{5}
    \end{array} \right\} \Rightarrow u(x) = \dfrac{3}{5}x$$
\end{ejemplo}



\begin{ejemplo}
    Calcular la recta que mejor aproxima por mínimos cuadrados los datos $\{(1,0),(2,1),(3,2),(4,3),(5,4)\}$.
Siendo $U=\cc{L}\{1,x\}$ y el producto escalar en $U$ se define como:
$$\langle v,w \rangle = v(1)w(1) + v(2)w(2) + v(3)w(3) + v(4)w(4) + v(5)w(5)~~~~\forall v,w \in V$$
La mejor aproximación de los puntos será $u(x) = a_0 \cdot 1 + a_1x \in \bb{P}_1~~~a_0,a_1 \in \R$
$$\left( \begin{array}{cc}
            \langle 1,1 \rangle & \langle x,1 \rangle \\
            \langle 1,x \rangle & \langle x,x \rangle
        \end{array} \right) \left( \begin{array}{c}
            a_0 \\
            a_1
        \end{array} \right) = \left( \begin{array}{c}
            \langle f,1 \rangle \\
            \langle f,x \rangle
        \end{array} \right)$$
$$\langle 1,1 \rangle = 5$$
$$\langle 1,x \rangle = \langle x,1 \rangle = 15$$
$$\langle x,x \rangle = 55 $$
$$\langle f,1 \rangle = 10$$
$$\langle f,x \rangle = 40$$

$$\left. \begin{array}{cccc}
        5a_0  & 15a_1 & = & 10 \\
        15a_0 & 55a_1 & = & 40
    \end{array} \right\} \Rightarrow \left\{ \begin{array}{c}
        a_0 = -1 \\
        a_1 = 1
    \end{array} \right\} \Rightarrow u(x) = -1+x$$
\end{ejemplo}



\subsection{Tipos de aproximación por mínimos cuadrados}
\noindent
\subsubsection{Aproximación por mínimos cuadrados continua}

Para la aproximación por mínimos cuadrados continua en el intervalo $[a,b]\subset \bb{R}$ se emplea el producto escalar siguiente:
\begin{equation*}
    \langle f,g\rangle = \int_{a}^b \omega(x)f(x)g(x)dx
\end{equation*}

donde $\omega$ es denominada \emph{función peso} y ha de ser integrable y $\omega\geq 0 \quad \forall x\in [a,b]$. Si no se especifica lo contrario, $\omega(x)=1$.

\begin{ejemplo}
    Sea el producto escalar definido como
    \begin{equation*}
        \langle f,g\rangle = \int_{-1}^1 f(x)g(x)dx
    \end{equation*}

    Encontrar la mejor aproximación de $f(x)=x^3$ en $\bb{P}_1$.

    Sea $u\in \bb{P}_1$ la mejor aproximación de $f$. Tomamos como base $\bb{P}_1=\cc{L}\{1,x\}$, por lo que sea $u(x)=a_0\cdot 1 + a_1x \quad a_0,a_1\in \bb{R}$. El sistema a resolver, por tanto, es:
    \begin{equation*}
        \left\{\begin{array}{c}
            a_0\langle 1,1\rangle + a_1 \langle x,1\rangle = \langle x^3, 1 \rangle \\
            a_0\langle 1,x\rangle + a_1 \langle x,1\rangle = \langle x^3, x \rangle \\
        \end{array}\right.
    \end{equation*}

    Tras calcular cada integral definida, tenemos que:
    \begin{equation*}
        \left\{\begin{array}{c}
            2a_0 + 0\cdot a_1 = 0 \\
            0\cdot a_0 + \frac{2}{3} a_1 = \frac{2}{5}
        \end{array}\right.
    \end{equation*}

    Por tanto, $u(x)=\frac{3}{5}x$.
    \begin{figure}[H]
        \centering
        \begin{tikzpicture}
        \begin{axis}[
            xlabel=$x$,
            ylabel=$y$,
            xmin=-1.5,
            xmax=1.5,
            ymin=-1.5,
            ymax=1.5,
            axis lines=middle,
            width=5cm,
            height=5cm,
            samples=90 % número de muestras para la función
        ]
        
        \addplot[red ,domain=-1.5:1.5] {3/5*x};
        \addplot[blue ,domain=-1.5:1.5] {x^3};
            
        \end{axis}
        \end{tikzpicture}
    \end{figure}
\end{ejemplo}


\subsubsection{Aproximación por mínimos cuadrados discreta}
Para la aproximación por mínimos cuadrados discreta en los nodos $x_i\subset \bb{R}$, con $i=0,\dots, N$; se emplea el producto escalar siguiente:
\begin{equation*}
    \langle f,g\rangle = \sum_{i=0}^N \omega(x_i)f(x_i)g(x_i)dx
\end{equation*}

donde $\omega$ es denominada \emph{función peso} y ha de ser $\omega(x_i) > 0 \quad \forall i=0,\dots, N$. Si no se especifica lo contrario, $\omega(x)=1$.

El problema de la mejor aproximación por mínimos cuadrados discreta consiste en lo siguiente:

Sea $f:[a,b]\to \bb{R}$ para la que conocemos $(x_i, f(x_i))\quad i=0,\dots, N$. Encontrar $p\in \bb{P}_m$ donde $m<N$ tal que:
\begin{equation*}
    ||f-p||^2 = \min_{q\in \bb{P}_m} ||f(x)-q(x)||^2 = \lim_{q\in \bb{P}_m} \sum_{k=0}^N [f(x_k)-q(x_k)]^2
\end{equation*}

donde he considerado el producto escalar definido como:
\begin{equation*}
    \langle f,g \rangle = \sum_{k=0}^N f(x_k)g(x_k)
\end{equation*}

Alternativamente, trabajamos de la siguiente manera. Sabemos que $\dim \bb{P}_m = m+1$. Consideramos 
\begin{equation*}
    \bb{P}_m = \cc{L}\{\varphi_0,\dots,\varphi_N\}
\end{equation*}

y la aplicación
\begin{equation*}
    \varphi_k \longmapsto \Phi=\left(\begin{array}{c}
        \varphi_k (x_0) \\ \varphi_k(x_1) \\ \vdots \\ \varphi_k(x_N)
    \end{array}\right)
\end{equation*}


Por la misma aplicación, tenemos que
\begin{equation*}
    f\longmapsto F=\left(\begin{array}{c}
        f(x_0) \\ f(x_1) \\ \vdots \\ f(x_N)
    \end{array}\right)
\end{equation*}


Consideramos $\nu = \cc{L}\{\Phi_0, \Phi_1, \dots, \Phi_m\}$, y demostremos que forman base.
\begin{equation*}
    a_0 \Phi_0 + a_1\Phi_1 + \dots a_m\Phi_m = 0 \Longrightarrow
    a_0\varphi_0(x_k) + \dots + a_m\varphi_m(x_k)=0 \qquad \forall k=0,\dots,N
\end{equation*}

Por tanto, como se anulan para todo $k$, tenemos que:
\begin{equation*}
    a_0\varphi_0 + \dots + a_m\varphi_m = 0 \Longrightarrow a_0=\dots=a_m=0
\end{equation*}

Por tanto, el problema se reduce a encontrar $P$ mejor aproximación de $F$ en $\nu$ con el producto escalar euclídeo.

No obstante, tenemos que:
\begin{equation*}
    \langle \Phi_i, \Phi_j\rangle = \sum_{k=0}^N \varphi_i(x_k)\varphi_j(x_k)
\end{equation*}
por tanto, tenemos que hemos llegado al producto escalar discreto definido en el primer caso.


\begin{ejemplo}
    Calcular la recta que mejor aproxima por mínimos cuadrados los siguientes datos:
    \begin{equation*}
        (1,0) \quad (2,1) \quad (3,2) \quad (4,3) \quad (5,4)
    \end{equation*}

    Sea el producto escalar definidio como:
    \begin{equation*}
        \langle u,v\rangle = \sum_{i=0}^N u(x_i)v(x_i)
    \end{equation*}

    Buscamos aproximar en $U=\bb{P}_1=\cc{L}\{1,x\}$. $f(x)$ está definida por:
    \begin{equation*}
        \left(\begin{array}{c}
            f(1) \\ f(2) \\ f(3) \\ f(4) \\ f(5)
        \end{array}\right)
        = \left(\begin{array}{c}
            0 \\ 1 \\ 2 \\ 3 \\ 4
        \end{array}\right)
    \end{equation*}

    Sea $u \in \bb{P}_1$ la mejor aproximación de $f$. Sea $u=a_0 \cdot 1 + a_1\cdot x$. El sistema a resolver, por tanto, es:
    \begin{equation*}
        \left\{\begin{array}{c}
            a_0\langle 1,1\rangle + a_1 \langle x,1\rangle = \langle f, 1 \rangle \\
            a_0\langle 1,x\rangle + a_1 \langle x,x\rangle = \langle f, x \rangle \\
        \end{array}\right.
    \end{equation*}

    Tenemos que:
    \begin{equation*}
        \begin{array}{c|c|c|c|c|c|c}
            x_i & f_i & x_1^0 & x_i^1 & x_i^2 & x_i^0f_i & x_i^1f_i \\ \hline
            1 & 0 & 1 & 1 & 1 & 0 & 0 \\
            2 & 1 & 1 & 2 & 4 & 1 & 2 \\
            3 & 2 & 1 & 3 & 9 & 2 & 6 \\
            4 & 3 & 1 & 4 & 16 & 3 & 12 \\
            5 & 4 & 1 & 5 & 25 & 4 & 20 \\ \hline
            && 5 & 15 & 55 & 10 & 40 \\
            && (\langle 1,1 \rangle) & (\langle 1,x \rangle) &
            (\langle x,x \rangle) & 
            (\langle f,1 \rangle) &
            (\langle f,x \rangle)
            
        \end{array}
    \end{equation*}

    Calculando cada producto escalar, tenemos que el sistema a resolver es:
    \begin{equation*}
        \left\{\begin{array}{c}
            5a_0 + 15a_1 = 10 \\
            15a_0 + 55a_1 = 40
        \end{array}\right.
    \end{equation*}

    Resolviendo, tenemos que $a_0=-1,\;a_1=1$.

    Por tanto, $u(x)=-1+x$.
\end{ejemplo}

\subsection{Ejemplos de Chebyshev}
\begin{itemize}
    \item \textbf{Ejemplo de Chebyshev de primera especie.}
En este caso, se toma:
$$w(x) = \dfrac{1}{\sqrt{1-x^2}}~~\forall x \in ]-1,1[$$
Esta función da un gran peso a los puntos que se encuentran en los extremos y un peso menor a los puntos centrales.

        \begin{figure}[H]
            \centering
            \begin{tikzpicture}
            \begin{axis}[
                xlabel=$x$,
                ylabel=$y$,
                xmin=-1.5,
                xmax=1.5,
                ymin=-0.5,
                ymax=7,
                axis lines=middle,
                width=5cm,
                height=5cm,
                samples=90 % número de muestras para la función
            ]
            
            \addplot[blue, thick, domain=-1:1] {1/sqrt(1-x^2)};
            \end{axis}
            \end{tikzpicture}
        \end{figure}

    \item \textbf{Ejemplo de Chebyshev de segunda especie.}
$$w(x) = \sqrt{1-x^2}~~\forall x \in [-1,1]$$
Esta función da un gran peso a los puntos centrales y un menor peso (casi inapreciable) a los puntos en lo extremos.
        \begin{figure}[H]
            \centering
            \begin{tikzpicture}
            \begin{axis}[
                xlabel=$x$,
                ylabel=$y$,
                xmin=-1.5,
                xmax=1.5,
                ymin=-1.5,
                ymax=1.5,
                axis lines=middle,
                width=5cm,
                height=5cm,
                samples=90 % número de muestras para la función
            ]
            
            \addplot[blue, thick, domain=-1:1] {(1-x^2)};
            \end{axis}
            \end{tikzpicture}
        \end{figure}
\end{itemize}


\section{Bases ortogonales}
\noindent
Si la base que cogemos del subespacio $U$ es ortogonal, esto es que:
$$\langle \tilde{\varphi_i}, \tilde{\varphi_j} \rangle = 0 ~~\forall i\neq j$$
Entonces, el sistema de ecuaciones se convierte en un sistema diagonal:
$$\langle \tilde{\varphi_k}, \tilde{\varphi_k} \rangle a_k = \langle f, \tilde{\varphi_k} \rangle~~k \in \{0, \ldots, m\}$$
Por lo que:
$$a_k = \dfrac{\langle f, \tilde{\varphi_k} \rangle}{\langle \tilde{\varphi_k}, \tilde{\varphi_k} \rangle}~~k \in \{0, \ldots, m\}$$
Y la mejor aproximación se calcula de la forma:
$$u = \sum_{k=0}^m \dfrac{\langle f, \tilde{\varphi_k} \rangle}{\langle \tilde{\varphi_k}, \tilde{\varphi_k} \rangle}\tilde{\varphi_k}$$

\bigskip
\begin{definicion}[Suma de Fourier]
    A la expresión:
    $$u = \sum_{k=0}^m \dfrac{\langle f, \tilde{\varphi_k} \rangle}{\langle \tilde{\varphi_k}, \tilde{\varphi_k} \rangle}\tilde{\varphi_k}$$
    Se le llama \textbf{$m$-ésima suma de Fourier} de $f$ asociada a la base ortogonal $\{\tilde{\varphi_0}, \tilde{\varphi_1},
        \ldots, \tilde{\varphi_m}\}$.\\

    \noindent
    A la cantidad:
    $$a_k = \dfrac{\langle f, \tilde{\varphi_k} \rangle}{\langle \tilde{\varphi_k}, \tilde{\varphi_k} \rangle}~~~~k \in \{0, \ldots, m\}$$
    Se le llama \textbf{$k$-ésimo coeficiente de Fourier} de $f$ asociado a la base ortongonal $\{\tilde{\varphi_0}, \tilde{\varphi_1},
        \ldots, \tilde{\varphi_m}\}$.\\
\end{definicion}

\subsection{Algoritmo de Gram-Schmidt}
\begin{teo}[Algoritmo de Gram-Schmidt]
    Sea $\{\varphi_0, \varphi_1, \ldots, \varphi_m\}$ una base de $U$, es posible obtener una base ortogonal
    $\{\tilde{\varphi_0}, \tilde{\varphi_1}, \ldots, \tilde{\varphi_m}\}$ de la forma:
    $$\tilde{\varphi_0} = \varphi_0$$
    $$\tilde{\varphi_k} = \varphi_k - \sum_{j=0}^{k-1} \dfrac{\langle \varphi_k, \tilde{\varphi_j} \rangle}
        {\langle \tilde{\varphi_j}, \tilde{\varphi_j} \rangle} \tilde{\varphi_j}~~~~k \in \{1,2, \ldots, m\}$$
    Como consecuencia, sabemos de la existencia de las bases ortogonales.
\end{teo}
\begin{proof}
    Realizamos inducción sobre $k = \dim \cc{L} \{\varphi_0, \varphi_1, \ldots, \varphi_k\}$:
    \begin{itemize}
        \item \underline{Para $k=2$:}

        Sea $\{\varphi_0, \varphi_1\}$ una base de $\cc{L}\{\varphi_0, \varphi_1\}$ con $\dim
        \cc{L}\{\varphi_0, \varphi_1\} = 2$:\par
    Construimos la base:\newline
    $$\tilde{\varphi_0} = \varphi_0$$
    $$\tilde{\varphi_1} = \varphi_1 - \dfrac{\langle \varphi_1, \tilde{\varphi_0} \rangle}
        {\langle \tilde{\varphi_0}, \tilde{\varphi_0} \rangle} \tilde{\varphi_0}$$
    $$\tilde{\varphi_1} \in \cc{L}\{\tilde{\varphi_0}, \varphi_1\} = \cc{L}\{\varphi_0, \varphi_1\}$$

    \noindent
    Comprobemos que $\tilde{\varphi_1}$ es ortogonal a $\tilde{\varphi_0}$ y que son linealmente independientes
    (vamos a demostrar que esto segundo es consecuencia de lo primero):
    $$\langle \tilde{\varphi_1}, \tilde{\varphi_0} \rangle = \langle \varphi_1, \tilde{\varphi_0} \rangle -
        \dfrac{\langle \varphi_1, \tilde{\varphi_0} \rangle}{\langle \tilde{\varphi_0}, \tilde{\varphi_0} \rangle}
        \langle \tilde{\varphi_0}, \tilde{\varphi_0} \rangle = 0 \Rightarrow \tilde{\varphi_1} \perp \tilde{\varphi_0}$$

    Supongamos que son linealmente dependientes: $\exists a,b \in \R \mid a\tilde{\varphi_0} + b \tilde{\varphi_1} = 0$:
    $$0 = \langle a\tilde{\varphi_0} + b\tilde{\varphi_1}, \tilde{\varphi_0} \rangle = a\langle \tilde{\varphi_0},
        \tilde{\varphi_0} \rangle + b \langle \tilde{\varphi_0}, \tilde{\varphi_1} \rangle \mathop{=}^{\tilde{\varphi_0}
        \perp \tilde{\varphi_1}} a\langle \tilde{\varphi_0}, \tilde{\varphi_0} \rangle = 0 \Leftrightarrow a = 0$$
    $$0 = \langle a\tilde{\varphi_0} + b\tilde{\varphi_1}, \tilde{\varphi_1} \rangle = a \langle \tilde{\varphi_0},
        \tilde{\varphi_1} \rangle + b\langle \tilde{\varphi_1}, \tilde{\varphi_1} \rangle \mathop{=}^{\tilde{\varphi_0}
        \perp \tilde{\varphi_1}} b\langle \tilde{\varphi_1}, \tilde{\varphi_1} \rangle = 0 \Leftrightarrow b = 0$$
    Luego $a = b = 0 \Rightarrow$ son linealmente independientes (consecuencia de ser ortogonales).

        \item \underline{Sea cierto para $k-1$:}

        $$\{\tilde{\varphi_0}, \tilde{\varphi_1}, \ldots, \tilde{\varphi}_{k-1}\} \mbox{ es una base de }
        \cc{L}\{\varphi_0, \varphi_1, \ldots, \varphi_{k-1}\} \mbox{ en la que:}$$
    $$\tilde{\varphi_i} \perp \tilde{\varphi_j}~~~~ i \neq j~~ i,j \in \{0, 1, \ldots, k-1\}$$
    Construimos:
    $$\tilde{\varphi_k} = \varphi_k - \sum_{j=0}^{k-1} \dfrac{\langle \varphi_k, \tilde{\varphi_j} \rangle}
        {\langle \tilde{\varphi_j}, \tilde{\varphi_j} \rangle} \tilde{\varphi_j}$$
    Comprobemos que sea ortogonal al resto (y por tanto, linealmente independientes):
    $$\forall j \in \{0, \ldots, k-1\}: \langle \tilde{\varphi_k}, \tilde{\varphi_j} \rangle = \langle \varphi_k,
        \tilde{\varphi_j} \rangle - \sum_{i=0}^{k-1} \dfrac{\langle \tilde{\varphi_k}, \tilde{\varphi_i} \rangle}
        {\langle \tilde{\varphi_i}, \tilde{\varphi_i} \rangle} \langle \tilde{\varphi_i}, \tilde{\varphi_j} \rangle=$$
    $$= \langle \varphi_k, \tilde{\varphi_j} \rangle - \sum_{i=0}^{k-1} \dfrac{\langle \tilde{\varphi_k},
            \tilde{\varphi_i} \rangle} {\langle \tilde{\varphi_i}, \tilde{\varphi_i} \rangle} \delta_{ij} \langle \tilde{\varphi_j},
        \tilde{\varphi_j} \rangle = \langle \varphi_k,
        \tilde{\varphi_j} \rangle - \dfrac{\langle \tilde{\varphi_k}, \tilde{\varphi_j} \rangle}
        {\langle \tilde{\varphi_j}, \tilde{\varphi_j} \rangle} \langle \tilde{\varphi_j}, \tilde{\varphi_j} \rangle = 0$$
    Por lo que $\tilde{\varphi_k} \perp \tilde{\varphi_j}~~~~\forall j \in \{0, \ldots, k-1\} \Rightarrow$ es linealmente
    independiente con todos ellos $\Rightarrow \{\tilde{\varphi_0}, \tilde{\varphi_1}, \ldots, \tilde{\varphi}_{k-1},
        \tilde{\varphi_k}\}$ es una base de $\cc{L}\{\varphi_0, \varphi_1, \ldots, \varphi_{k-1}, \varphi_k\}$.
    \end{itemize}
    
\end{proof}


\begin{ejemplo}
 Dado el siguiente producto escalar dentro de $\bb{P}_2$:
$$\langle f,g \rangle = \int_0^1 f(x)g(x)~dx~~~~\forall f,g \in \bb{P}_2$$
Buscar una base ortogonal a partir de la base $\{1,x,x^2\}$\\

\noindent
Aplicamos el algoritmo de Gram-Schmidt:\par
Dada $\{\varphi_i\} \mid \varphi_i = x^i~~i \in \{0,1,2\}$, construimos $\{\tilde{\varphi_i}\}$ como sigue:
$$\tilde{\varphi_0} = \varphi_0$$
$$\tilde{\varphi_1} = \varphi_1 - \dfrac{\langle \varphi_1, \tilde{\varphi_0} \rangle}{\langle \tilde{\varphi_0},
        \tilde{\varphi_0} \rangle} \tilde{\varphi_0}$$
$$\tilde{\varphi_2} = \varphi_2 - \dfrac{\langle \varphi_2, \tilde{\varphi_0} \rangle}{\langle \tilde{\varphi_0},
        \tilde{\varphi_0} \rangle} \tilde{\varphi_0} - \dfrac{\langle \varphi_2, \tilde{\varphi_1} \rangle}{\langle \tilde{\varphi_1},
        \tilde{\varphi_1} \rangle} \tilde{\varphi_1}$$

Calculamos los respectivos productos escalares:
$$\langle \varphi_1, \tilde{\varphi_0} \rangle = \int_0^1 x~dx = \dfrac{1}{2}
\qquad 
\langle \tilde{\varphi_0}, \tilde{\varphi_0} \rangle = \int_0^1 dx = 1$$
$$\tilde{\varphi_1} = \varphi_1 - \dfrac{\langle \varphi_1, \tilde{\varphi_0} \rangle}{\langle \tilde{\varphi_0},
        \tilde{\varphi_0} \rangle} \tilde{\varphi_0} = x - \dfrac{1}{2} \cdot 1 = x - \dfrac{1}{2}$$
\begin{equation*}
    \langle \varphi_2, \tilde{\varphi_0} \rangle = \int_0^1 x^2~dx = \dfrac{1}{3}
\qquad
\langle \varphi_2, \tilde{\varphi_1} \rangle = \int_0^1 x^2(x - \dfrac{1}{2})~dx = \dfrac{1}{12}
\qquad
\langle \tilde{\varphi_1}, \tilde{\varphi_1} \rangle = \int_0^1 (x-\dfrac{1}{2})^2~dx = \dfrac{1}{12}
\end{equation*}
$$\tilde{\varphi_2} = \varphi_2 - \dfrac{\langle \varphi_2, \tilde{\varphi_0} \rangle}{\langle \tilde{\varphi_0},
        \tilde{\varphi_0} \rangle} \tilde{\varphi_0} - \dfrac{\langle \varphi_2, \tilde{\varphi_1} \rangle}{\langle \tilde{\varphi_1},
        \tilde{\varphi_1} \rangle} \tilde{\varphi_1} = x^2 - \dfrac{1}{3} \cdot 1 - \left(x- \dfrac{1}{2}\right) = x^2 - x + \dfrac{1}{6}$$

\noindent
Por tanto, la base buscada es:
$$\left\{1, x-\dfrac{1}{2}, x^2-x+\dfrac{1}{6}\right\}$$
\end{ejemplo}

\subsection{Bases ortonormales}
\noindent Las bases ortonormales son bases ortogonales en las que la norma de cada elemento de la base es igual a 1.\\

\noindent
A la hora de obtener bases ortonormales podemos hacer dos procedimientos a partir del anterior algoritmo de Gram-Schmidt:
\begin{itemize}
    \item Aplicar el algoritmo de Gram-Schmidt y normalizar la base: esto es, aplicar el algoritmo sobre uan base para
          obtener una base ortogonal y luego escalar los vectores de la base para que tengan norma 1.
    \item O aplicar el algoritmo de Gram-Schmidt modificado, que consiste en obtener un vector de la base de Gram-Schmidt,
          normalizarlo y volver a aplicar el algoritmo con este vector normalizado y repetir sucesivamente.
\end{itemize}

\noindent
En la práctica, suele ser más cómodo hacerlo de la primera forma aunque esto tiene un inconveniente, ya que este
método es inestable numéricamente mientras que el segundo es estable numéricamente. Por tanto, cuando queramos evitar
errores al conseguir una base ortonormal a partir de una base para un subespacio de una dimensión considerable,
es conveniente aplicar el segundo método.\\

\noindent
\textbf{Normalización de vectores.} Dado un vector $v \in V$ que queremos normalizar, esto es $\|v\|=1 \Leftrightarrow \|v\|^2 = 1$.
Escalaremos $v$ de la siguiente forma:
$$\overline{v} = \dfrac{1}{\langle v,v \rangle}v = \dfrac{1}{\sqrt{\|v\|^2}}v = \dfrac{1}{\|v\|}v$$

De esta forma:
$$\|\overline{v}\| = \sqrt{\|\overline{v}\|^2} = \sqrt{\langle \overline{v},\overline{v} \rangle} =
    \sqrt{\left\langle \dfrac{1}{\|v\|}v,\dfrac{1}{\|v\|}v \right\rangle} = \sqrt{\dfrac{\langle v,v \rangle}{\|v\|^2}} =
    \sqrt{\dfrac{\|v\|^2}{\|v\|^2}} = 1$$

\subsection{Utilidad de bases ortogonales}
\noindent
Dado un producto escalar de la forma:
$$\langle x^i, x^j \rangle = \int_0^1 x^{i+j}~dx = \dfrac{1}{i+j+1}$$
Tenemos que la matriz de coeficientes del sistema que nos permite calcular la mejor aproximación respecto de la base
del estilo $\{x^i\}~~i \in \{0, 1, \ldots, n\}$ del subespacio $U=\bb{P}_n$ del espacio vectorial de polinomios es:
$$G=\left(\begin{array}{ccccc}
            \langle 1,1 \rangle   & \langle x,1 \rangle   & \langle x^2,1 \rangle & \ldots & \langle x^n,1 \rangle   \\
            \langle 1,x \rangle   & \langle x,x \rangle   & \ldots                & \ldots & \langle x^n, x \rangle  \\
            \langle 1,x^2 \rangle & \vdots                & \ddots                & \ddots & \vdots                  \\
            \vdots                & \vdots                & \ddots                & \ddots & \vdots                  \\
            \langle 1,x^n \rangle & \langle x,x^n \rangle & \ldots                & \ldots & \langle x^n,x^n \rangle
        \end{array}\right) = \left( \begin{array}{ccccc}
            1            & \dfrac{1}{2}   & \dfrac{1}{3} & \ldots & \dfrac{1}{n}   \\
                         &                &              &                         \\
            \dfrac{1}{2} & \dfrac{1}{3}   & \ldots       & \ldots & \dfrac{1}{n+1} \\
            \dfrac{1}{3} & \vdots         & \ddots       & \ddots & \vdots         \\
            \vdots       & \vdots         & \ddots       & \ddots & \vdots         \\
            \dfrac{1}{n} & \dfrac{1}{n+1} & \ldots       & \ldots & \dfrac{1}{2n}
        \end{array} \right)$$
Llamada matriz de Hilbert, donde todos los elementos de cada antidiagonal son iguales entre sí (matriz de Hankel):
$$\langle x^2,1 \rangle = \langle x,x \rangle = \langle 1,x^2 \rangle = \int_0^1 x^2~dx = \dfrac{1}{3}$$

\noindent
Dicha matriz está muy mal condicionada y para solventar el problema numérico al que nos enfrentamos, se suele aplicar
el algoritmo de Gram-Schmidt sobre la base anterior para obtener una base ortogonal y construir nuestro sistema
con dicha base, formando un sistema diagonal cuya resolución no nos supone ningún problema.\\

\noindent
Este procedimiento puede llegar a ser hasta más eficiente que el trabajar directamente con la matriz de Hilbert para
dimensiones grandes.

\section{Polinomios ortogonales}
\begin{definicion}[Polinomios Ortogonales]
    Dado un producto escalar $\langle \cdot,\cdot \rangle$, una familia de polinomios $\{P_n(x)\}_{n\geq 0}$ se dice que es
    una sucesión de polinomios ortogonales (SPO) si verifica:
    \begin{enumerate}
        \item $grd(P_n) = n$
        \item $\langle P_n, P_m \rangle = 0~~~~\forall n \neq m$
    \end{enumerate}
    \noindent
    Por tanto, tenemos que $\{P_0, P_1, \ldots, P_m\}$ es una base ortogonal de $\bb{P}_m$
\end{definicion}

\noindent
Una sucesión de polinomios ortogonales mónica (SPOM) (es decir, de coeficiente líder 1) puede obtenerse aplicando el algoritmo de
Gram-Schmidt a la base $\{1, x, x^2, \ldots, x^m\}$ de $\bb{P}_m$.

\noindent
Además, las sucesiones de polinomios ortogonales son únicas salvo constante multiplicativa.

\begin{teo}
    Dado un producto escalar $\langle \cdot,\cdot \rangle$ y sea $\{P_n\}_{n \geq 0}$ la correspondiente SPOM, entonces existen
    $\{c_n\}_{n \geq 0}$ y $\{\lambda_n\}_{n\geq 1}$ con $\lambda_n>0$ $\forall n$, tales que:
    $$P_{n+1}(x) = (x-c_{n+1})P_n(x) - \lambda_{n+1}P_{n-1}(x)~~~~\forall n \geq 0$$
    Con $P_{-1}(x) = 0$ y $P_0(x) = 1$\\

    \noindent
    En particular:
    $$c_{n+1} = \dfrac{\langle xP_n(x), P_n(x) \rangle}{\langle P_n(x),P_n(x) \rangle}$$
    $$\lambda_{n+1} = \dfrac{\langle P_n(x), P_n(x) \rangle}{\langle P_{n-1}(x),P_{n-1}(x) \rangle}$$
\end{teo}
\begin{proof}
    Dados los $n+1$ primeros polinomios ortogonales mónicos de una sucesión, nos disponemos a calcular el siguiente:
    $$xP_n(x) \in \bb{P}_{n+1} \Rightarrow xP_n(x) = \sum_{j=1}^{n+1} d_jP_j(x)$$
    $$\langle xP_n, P_k \rangle = \sum_{j=0}^{n+1} d_j \langle P_j, P_k \rangle = \sum_{j=0}^{n+1} d_j \delta_{kj} \langle P_j, P_k \rangle
        = d_k \langle P_k, P_k \rangle$$
    Luego:
    $$d_k = \dfrac{\langle xP_n, P_k \rangle}{\langle P_k, P_k \rangle} = \dfrac{\langle P_n, xP_k \rangle}
        {\langle P_k, P_k \rangle} = 0 \Leftrightarrow k+1<n \Leftrightarrow k<n-1$$
    Por lo que:
    $$d_j = 0~~~~\forall j \in \{0, 1, \ldots, n-2\}$$
    De donde concluimos:
    $$xP_n(x) = \sum_{j=1}^{n+1} d_jP_j(x) = \sum_{j=n-1}^{n+1} d_jP_j(x) =$$
    $$=d_{n-1}P_{n-1}(x) + d_nP_n(x) + d_{n+1}P_{n+1}(x)$$

    \noindent
    $xP_n(x)$ es de grado $n+1$ y su coeficiente líder es 1 por ser mónico, en el término de la derecha, el único
    monomio de grado $n+1$ es el de $d_{n+1}P_{n+1}(x)$ que es $d_{n+1}$, por lo que concluimos que $d_{n+1}=1$
    $$xP_n(x)=d_{n-1}P_{n-1}(x) + d_nP_n(x) + P_{n+1}(x)$$
    $$P_{n+1}(x) = (x-d_n)P_n(x) - d_{n-1}P_{n-1}(x)$$
    Y tenemos que:
    $$c_{n+1}=d_n~~~~~~\lambda_{n+1}=d_{n-1}$$
\end{proof}

\bigskip
\noindent
A continuación, la siguiente parte de los apuntes no es relevante en lo que se refiere a la asignatura de
Métodos Numéricos I, el lector puede leerlo para su disfrute.

\subsection{Polinomios de Jacobi}
\noindent
Se obtienen como la familia de polinomios ortogonales asociados a la función de peso dada por:
$$w(x) = (1-x)^\alpha(1+x)^\beta~~~~x \in [-1,1]~~\alpha, \beta > -1$$
Denotamos por $P_n^{(\alpha, \beta)}(x)$ a los polinomios de Jacobi, que verifican la condición de normalización:
$$P_n^{(\alpha, \beta)}(1) = \binom{n+\alpha}{n}$$

\noindent
Estos presentan como casos particulaes a las familias de polinomios de:
\begin{itemize}
    \item Chebyshev de primera especie, con $\alpha = \beta = \dfrac{-1}{2}$.
    \item Chebyshev de segunda espacie, con $\alpha = \beta = \dfrac{1}{2}$.
    \item Legendre, con $\alpha = \beta = 0$.
    \item Gegenbauer o ultraesféricos, con $\alpha = \beta$.
\end{itemize}

\noindent
Todos ellos de coeficiente líder:
$$P_n^{(\alpha, \beta)}(x) = \dfrac{(n + \alpha + \beta + 1)}{2n!}x^n + \ldots$$

\noindent
\textbf{Fórmula de Rodrigues.}
$$P_n^{(\alpha, \beta)}(x) = \dfrac{(-1)^n}{2^nn!}(1-x)^{-\alpha}(1+x)^{-\beta}D^n((1-x)^{n+\alpha}(1+x)^{n+\beta})~~\alpha, \beta>$$

\noindent
\textbf{Ortogonalidad.}
$$\int_{-1}^1 P_n^{(\alpha,\beta)}(x) P_m^{(\alpha, \beta)}(x) (1-x)^\alpha(1+x)^\beta~dx = \left\{ \begin{array}{ll}
        \tau_n \delta_{mn} & n=m      \\
        0                  & n \neq m
    \end{array} \right.$$
$$\tau_n = \dfrac{2^{\alpha + \beta + 1}}{2n + \alpha + \beta + 1} \dfrac{\Gamma(n+ \alpha + 1) \Gamma(n+\beta + 1)}
    {\Gamma(n+\alpha + \beta + 1)n!}$$

\noindent
\textbf{Fórmula de recurrencia.}
$$xP_n^{(\alpha, \beta)}(x) = \dfrac{2(n+1)(n+\alpha + \beta + 1)}{(2n + \alpha + \beta +1)(2n + \alpha + \beta + 2)}P_{n+1}^{(\alpha,\beta)}(x) + $$
$$+ \dfrac{\beta^2-\alpha^2}{(2n+\alpha+\beta)(2n+\alpha+\beta+2)}P_n^{(\alpha + \beta)}(x) + $$
$$+ \dfrac{2(n+\alpha)(n+\beta)}{(2n+\alpha+\beta)(2n+\alpha+\beta+1)}P_{n-1}^{(\alpha,\beta)}(x)$$

\subsection{Polinomios de Lagrere}
\noindent
Se obtienen como la familia de polinomio ortogonales asociados a la función peso dada por:
$$w(x) = x^\alpha e^{-x}~~~~x \in [0,+\infty[~~\alpha >1$$
Los correspondientes momentos existen y verifican:
$$\int_0^{+\infty}x^nx^\alpha e^{-x}~dx = \int_0^{+\infty}x^{n+\alpha}e^{-x}~dx = \Gamma(n+\alpha)$$
Notemos por $\{L_n^{(\alpha)}(x)\}_{n\geq 0}$ a los polinomios de Laguerre con coeficiente líder $L_n^{(\alpha)} =
    \dfrac{(-1)^n}{n!}x^n + \ldots$

\noindent
\textbf{Fórmula de Rodrigues.}
$$L_n^{(\alpha)}(x) = \dfrac{1}{n!}x^{\alpha}e^{-x}D^n(x^{n+\alpha e^{-x}})~~~~\alpha >-1$$

\noindent
\textbf{Ortogonalidad.}
$$\int_0^{+\infty}L_m^{(\alpha)}(x) L_n^{(\alpha)}(x) x^\alpha e^{-x}~dx = \left\{ \begin{array}{ll}
        \dfrac{\Gamma(n + \alpha + 1)}{n!}\gamma_{mn} & n = m    \\
        0                                             & n \neq m
    \end{array} \right.$$

\noindent
\textbf{Relación de recurrencia.}
$$(n+1)L_{n+1}^{(\alpha)}(x) = (2n + \alpha + 1 - x)L_n^{(\alpha)}(x) - (n+\alpha)L_{n-1}^{(\alpha)}(x)$$

\subsection{Polinomios de Hermite}
\noindent
Se obtienen como la familia de polinomios ortogonales a la función peso dada por:
$$w(x) = e^{-x^2}~~~~\forall x \in \R$$
$\todon$, notaremos por $H_n(x)$ al polinomio ortogonal con respecto a $w(x)$ y cuyo coeficiente líder es $H_n(x) = 2^nx^n + \ldots$.\\

\noindent
\textbf{Fórmula de Rodrigues.}
$$H_n(x) = (-1)^n e^{x^2} D^n(e^{-x^2})$$

\noindent
\textbf{Ortogonalidad.}
$$\int_{-\infty}^{+\infty}H_m(x) H_n(x) e^{-x^2}~dx = \left\{ \begin{array}{ll}
        2^n n! \sqrt{\pi}\delta_{nm} & n = m    \\
        0                            & n \neq m
    \end{array} \right.$$

\noindent
\textbf{Relación de recurrencia.}
$$H_n(x) = 2xH_{n-1}(x) -2nH_{n-2}(x)$$


\section{Ejercicios}
Los ejercicios relativos a este tema están disponbles en la sección \ref{sec:Rel4}.

    %\fancyhead[R]{\helv \nouppercase{\rightmark}}
    %\section{Diagonalización de Endomorfismos}\label{sec:EjerciciosTema1}


\begin{ejercicio}
    Dadas las siguientes matrices $A_i\in\mathcal{M}_3(\bb{R})$:
    \begin{equation*}
        A_1 = \left( \begin{array}{ccc}
            1 & 2 & 1 \\
            2 & 0 & 2 \\
            1 & 2 & 1 \\
        \end{array}\right) \qquad
        A_2 = \left( \begin{array}{ccc}
            1 & 0 & 1 \\
            1 & 2 & 3 \\
            1 & 1 & -1 \\
        \end{array}\right) \qquad
        A_3 = \left( \begin{array}{ccc}
            3 & -6 & 6 \\
            0 & 1 & 2 \\
            2 & -6 & 3 \\
        \end{array}\right)
    \end{equation*}
    \begin{equation*}
        A_4 = \left( \begin{array}{ccc}
            9 & -18 & 0 \\
            2 & -3 & 0 \\
            2 & -6 & 3 \\
        \end{array}\right) \qquad
        A_5 = \left( \begin{array}{ccc}
            6 & -9 & 3 \\
            1 & 0 & 1 \\
            2 & -6 & 3 \\
        \end{array}\right) \qquad
        A_6 = \left( \begin{array}{ccc}
            -3 & 18 & 3 \\
            -2 & 9 & 1 \\
            -1 & 3 & 3 \\
        \end{array}\right)
    \end{equation*}

    \begin{enumerate}
        \item Estudiar si $A_i$ son diagonalizables. En caso de serlo, encontrar una matriz diagonal $D_i$ y una matriz regular $P_i$ tal que $D_i=P_i^{-1}A_iP_i$.

        \begin{enumerate}
            \item Veamos si es diagonalizable $A_1$.\\
            \begin{equation*}\begin{split}
                P_{A_1}(\lambda) & = |A_1-\lambda_0 I| = \left| \begin{array}{ccc}
                1-\lambda_0 & 2 & 1 \\
                2 & -\lambda_0 & 2 \\
                1 & 2 & 1-\lambda_0 \\
                \end{array}\right| =
                \left| \begin{array}{ccc}
                -\lambda_0 & 2 & 1 \\
                0 & -\lambda_0 & 2 \\
                \lambda_0 & 2 & 1-\lambda_0 \\
                \end{array}\right| \\
                & = \lambda_0
                \left| \begin{array}{ccc}
                -1 & 2 & 1 \\
                0 & -\lambda_0 & 2 \\
                1 & 2 & 1-\lambda_0 \\
                \end{array}\right| = \lambda_0
                \left| \begin{array}{ccc}
                0 & 4 & 2-\lambda_0 \\
                0 & -\lambda_0 & 2 \\
                1 & 2 & 1-\lambda_0 \\
                \end{array}\right|\\
                & = \lambda_0(8+\lambda_0(2-\lambda_0)) = \lambda_0(-\lambda_0^2+2\lambda_0+8) = -\lambda_0(\lambda+2)(\lambda-4)
            \end{split}\end{equation*}

            Por tanto, los valores propios son: $\{0,-2,4\}$. Como los tres son distintos, sí es diagonalizable.
            \begin{table}[H]
                \centering
                \begin{tabular}{c|c|c}
                    Valores Propios & Mult. Alg. & Mult. Geom. \\ \hline 
                    0 & 1 & 1\\
                    $-$2 & 1 & 1\\
                    4 & 1 & 1\\
                \end{tabular}
                \caption{Valores propios con sus multiplicidades}
            \end{table}
            
            Calculemos las matrices $P_1$ y $D_1$.
    
            \begin{equation*}
            V_0 = \left\{x\in\bb{R}^3 \left| \begin{array}{c}
                 x_1+2x_2+x_3=0  \\
                 x_2+x_3 = 0
            \end{array}\right.\right\} = \mathcal{L}\left(\left\{ \left(\begin{array}{c}
                    1 \\
                    0 \\
                    -1 \\
               \end{array}\right)
               \right\}\right)
            \end{equation*}
    
            \begin{equation*}
            V_{-2} = \left\{x\in\bb{R}^3 \left| \begin{array}{c}
                 3x_1+2x_2+x_3=0  \\
                 x_1+x_2+x_3=0 \\
                 x_1+2x_2+3x_3 = 0
            \end{array}\right.\right\} = \mathcal{L}\left(\left\{ \left(\begin{array}{c}
                    1 \\
                    -2 \\
                    1 \\
               \end{array}\right)
               \right\}\right)
            \end{equation*}
    
            \begin{equation*}
            V_4 = \left\{x\in\bb{R}^3 \left| \begin{array}{c}
                 -3x_1+2x_2+x_3 = 0  \\
                 x_1-2x_2+x_3 =0 \\
                 x_1+2x_2-3x_3 = 0
            \end{array}\right.\right\} = \mathcal{L}\left(\left\{ \left(\begin{array}{c}
                    1 \\
                    1 \\
                    1 \\
               \end{array}\right)
               \right\}\right)
            \end{equation*}
    
            Por tanto, las matrices $P_1$ y $D_1$ son:
            \begin{equation*}
                D_1 = \left( \begin{array}{ccc}
                    0 & 0 & 0 \\
                    0 & -2 & 0 \\
                    0 & 0 & 4 \\
                \end{array}\right) \qquad 
                P_1 = \left( \begin{array}{ccc}
                    1 & 1 & 1 \\
                    0 & -2 & 1 \\
                    -1& 1 & 1 \\
                \end{array}\right)
            \end{equation*}

            \item Veamos si es diagonalizable $A_2$.\\
            \begin{equation*}\begin{split}
                P_{A_2}(\lambda) & = |A_2-\lambda_0 I| = \left| \begin{array}{ccc}
                1-\lambda_0 & 0 & 1 \\
                1 & 2-\lambda_0 & 3 \\
                1 & 1 & -1-\lambda_0 \\
                \end{array}\right| =\\
                & = -(1-\lambda_0)(1+\lambda_0)(2-\lambda_0) +1 -(2-\lambda_0)-3(1-\lambda_0) = \\
                &= -(1-\lambda_0^2)(2-\lambda_0) -4 +4\lambda_0 = -\lambda_0^3 +2\lambda_0^2+5\lambda_0-6
            \end{split}\end{equation*}

            \begin{figure}[H]
                \centering
                \polyhornerscheme[x=1]{-x^3+2x^2+5x-6} \hspace{1cm} 
                \polyhornerscheme[x=-2]{-x^2+x+6}
            \end{figure}

            Por tanto, $P_{A_2} = (\lambda_0-1)(\lambda_0+2)(-\lambda_0+3)$. Por ende, 
            los valores propios son: $\{1, -2, 3\}$. Como los tres son distintos, sí es diagonalizable.
            \begin{table}[H]
                \centering
                \begin{tabular}{c|c|c}
                    Valores Propios & Mult. Alg. & Mult. Geom. \\ \hline 
                    1 & 1 & 1\\
                    $-$2 & 1 & 1\\
                    3 & 1 & 1\\
                \end{tabular}
                \caption{Valores propios con sus multiplicidades}
            \end{table}
            
            Calculemos las matrices $P_2$ y $D_2$.
    
            \begin{equation*}
            V_1 = \left\{x\in\bb{R}^3 \left| \begin{array}{c}
                 x_3=0  \\
                 x_1+x_2+3x_3 = 0 \\
                 x_1+x_2-2x_3 = 0
            \end{array}\right.\right\} = \mathcal{L}\left(\left\{ \left(\begin{array}{c}
                    1 \\
                    -1 \\
                    0 \\
               \end{array}\right)
               \right\}\right)
            \end{equation*}
    
            \begin{equation*}
            V_{-2} = \left\{x\in\bb{R}^3 \left| \begin{array}{c}
                 3x_1+x_3=0  \\
                 x_1+4x_2+3x_3=0 \\
                 x_1+x_2+x_3 = 0
            \end{array}\right.\right\} = \mathcal{L}\left(\left\{ \left(\begin{array}{c}
                    1 \\
                    2 \\
                    -3 \\
               \end{array}\right)
               \right\}\right)
            \end{equation*}
    
            \begin{equation*}
            V_3 = \left\{x\in\bb{R}^3 \left| \begin{array}{c}
                 -2x_1+x_3 = 0  \\
                 x_1-x_2+3x_3 =0 \\
                 x_1+x_2-4x_3 = 0
            \end{array}\right.\right\} = \mathcal{L}\left(\left\{ \left(\begin{array}{c}
                    1 \\
                    7 \\
                    2 \\
               \end{array}\right)
               \right\}\right)
            \end{equation*}
    
            Por tanto, las matrices $P_2$ y $D_2$ son:
            \begin{equation*}
                D_2 = \left( \begin{array}{ccc}
                    1 & 0 & 0 \\
                    0 & -2 & 0 \\
                    0 & 0 & 3 \\
                \end{array}\right) \qquad 
                P_2 = \left( \begin{array}{ccc}
                    1 & 1 & 1 \\
                    -1 & 2 & 7 \\
                    0 & -3 & 2 \\
                \end{array}\right)
            \end{equation*}

            \item Veamos si es diagonalizable $A_3$.
            \begin{equation*}\begin{split}
                P_{A_3}(\lambda) & = |A_3-\lambda_0 I| = \left| \begin{array}{ccc}
                3-\lambda_0 & -6 & 6 \\
                0 & 1-\lambda_0 & 2 \\
                2 & -6 & 3-\lambda_0 \\
                \end{array}\right| =\\
                & = (3-\lambda_0)^2(1-\lambda_0)-24-12(1-\lambda_0)+12(3-\lambda_0) = \\
                &= 9+\lambda_0^2-6\lambda_0 -9\lambda_0-\lambda_0^3+6\lambda_0^2 -24 -12 +12\lambda_0 +36 -12\lambda_0 = \\
                &= -\lambda_0^3 + 7\lambda_0^2-15\lambda_0 + 9
            \end{split}\end{equation*}

            \begin{figure}[H]
                \centering
                \polyhornerscheme[x=1]{-x^3+7x^2-15x+9}
            \end{figure}

            Por tanto, $P_{A_3}  = -(\lambda_0-1)(\lambda_0-3)^2$. Los valores propios son: $\{1, 3\}$.

            Calculemos la multiplicidad geométrica de $3$.
            \begin{equation*}
            V_3 = \left\{x\in\bb{R}^3 \left| \begin{array}{c}
                 -x_2+x_3=0  \\
                 -x_2+x_3 = 0 \\
                 x_1-3x_2 = 0
            \end{array}\right.\right\} = \mathcal{L}\left(\left\{ \left(\begin{array}{c}
                    3 \\
                    1 \\
                    1 \\
               \end{array}\right)
               \right\}\right)
            \end{equation*}
            \begin{table}[H]
                \centering
                \begin{tabular}{c|c|c}
                    Valores Propios & Mult. Alg. & Mult. Geom. \\ \hline 
                    1 & 1 & 1\\
                    3 & 2 & 1\\
                \end{tabular}
                \caption{Valores propios con sus multiplicidades}
            \end{table}

            Por tanto, la multiplicidad geométrica de $3$ es $n_3=1 \neq 2 = m_3$. Por tanto, $A_3$ no es diagonalizable.

            \item Veamos si es diagonalizable $A_4$.\\
            \begin{equation*}\begin{split}
                P_{A_4}(\lambda) & = |A_4-\lambda_0 I| = \left| \begin{array}{ccc}
                9-\lambda_0 & -18 & 0 \\
                2 & -3-\lambda_0 & 0 \\
                2 & -6 & 3-\lambda_0 \\
                \end{array}\right| =\\
                & = (3-\lambda_0)\left(-(9-\lambda_0)(3+\lambda_0)+36\right) = (3-\lambda_0)(\lambda_0^2-6\lambda_0+9) \\
                &= (3-\lambda_0)(\lambda_0-3)^2 = -(\lambda_0-3)^3
            \end{split}\end{equation*}

            Por tanto, el único valor propio es $\{3\}$. Calculemos su multiplicidad geométrica.
            \begin{equation*}
            V_3 = \left\{x\in\bb{R}^3 \left| \begin{array}{c}
                 x_1-3x_2=0  \\
                 x_1-3x_2 = 0 \\
                 x_1-3x_2 = 0
            \end{array}\right.\right\} \neq \bb{R}^3
            \end{equation*}

            \begin{table}[H]
                \centering
                \begin{tabular}{c|c|c}
                    Valores Propios & Mult. Alg. & Mult. Geom. \\ \hline 
                    3 & 3 & 2\\
                \end{tabular}
                \caption{Valores propios con sus multiplicidades}
            \end{table}

            Por tanto, la multiplicidad geométrica de $3$ es $n_3=2 \neq 3 = m_3$. Por tanto, $A_4$ no es diagonalizable.

            \item Veamos si es diagonalizable $A_5$.\\
            \begin{equation*}\begin{split}
                P_{A_5}(\lambda) & = |A_5-\lambda_0 I| = \left| \begin{array}{ccc}
                6-\lambda_0 & -9 & 3 \\
                1 & -\lambda_0 & 1 \\
                2 & -6 & 3-\lambda_0 \\
                \end{array}\right| =\\
                & = -\lambda_0(3-\lambda_0)(6-\lambda_0)-18-18 +6\lambda_0+9(3-\lambda_0) +6(6-\lambda_0) =\\
                &= -18\lambda_0 +3\lambda_0^2 +6\lambda_0^2-\lambda_0^3 -36 +6\lambda_0+27-9\lambda_0 + 36-6\lambda_0 =\\
                &=-\lambda_0^3+9\lambda_0^2-27\lambda_0+27 = -(\lambda_0-3)(\lambda_0-3)^2 = -(\lambda_0-3)^3
            \end{split}\end{equation*}

            \begin{figure}[H]
                \centering
                \polyhornerscheme[x=3]{-x^3+9x^2-27x+27}
            \end{figure}

            Por tanto, el único valor propio es $\{3\}$. Calculemos su multiplicidad geométrica.
            \begin{equation*}
            V_3 = \left\{x\in\bb{R}^3 \left| \begin{array}{c}
                 x_1-3x_2+x_3=0  \\
                 x_1-3x_2+x_3=0  \\
                 x_1-2x_2 = 0
            \end{array}\right.\right\} \neq \bb{R}^3
            \end{equation*}

            \begin{table}[H]
                \centering
                \begin{tabular}{c|c|c}
                    Valores Propios & Mult. Alg. & Mult. Geom. \\ \hline 
                    3 & 3 & 1\\
                \end{tabular}
                \caption{Valores propios con sus multiplicidades}
            \end{table}

            Por tanto, la multiplicidad geométrica de $3$ es $n_3=1 \neq 3 = m_3$. Por tanto, $A_5$ no es diagonalizable.

            \item Veamos si es diagonalizable $A_6$.\\
            \begin{equation*}\begin{split}
                P_{A_6}(\lambda) & = |A_6-\lambda_0 I| = \left| \begin{array}{ccc}
                -3-\lambda_0 & 18 & 3 \\
                -2 & 9-\lambda_0 & 1 \\
                -1 & 3 & 3-\lambda_0 \\
                \end{array}\right| =\\
                & = -(9-\lambda_0^2)(9-\lambda_0)-18-18 +3(9-\lambda_0)+3(3+\lambda_0)+36(3-\lambda_0) =\\
                &= -81 +9\lambda_0 +9\lambda_0^2 -\lambda_0^3 -36 +27-3\lambda_0+9+3\lambda_0+108-36\lambda_0 =\\
                &= -\lambda_0^3 +9\lambda_0^2 -27\lambda_0 + 27 = -(\lambda_0-3)^3
            \end{split}\end{equation*}

            \begin{figure}[H]
                \centering
                \polyhornerscheme[x=3]{-x^3+9x^2-27x+27}
            \end{figure}

            Por tanto, el único valor propio es $\{3\}$. Calculemos su multiplicidad geométrica.
            \begin{equation*}
            V_3 = \left\{x\in\bb{R}^3 \left| \begin{array}{c}
                 -6x_1+18x_2+3x_3=0  \\
                 -2x_1+6x_2+x_3=0  \\
                 -x_1+3x_2 = 0
            \end{array}\right.\right\}
            = \left\{x\in\bb{R}^3 \left| \begin{array}{c}
                 -2x_1+6x_2+x_3=0  \\
                 -x_1+3x_2 = 0
            \end{array}\right.\right\}
            \end{equation*}

            \begin{table}[H]
                \centering
                \begin{tabular}{c|c|c}
                    Valores Propios & Mult. Alg. & Mult. Geom. \\ \hline 
                    3 & 3 & 1\\
                \end{tabular}
                \caption{Valores propios con sus multiplicidades}
            \end{table}

            Por tanto, la multiplicidad geométrica de $3$ es $n_3=1 \neq 3 = m_3$. Por tanto, $A_6$ no es diagonalizable.

        \end{enumerate}
        

         

        \item Estudiar cuáles de las matrices de la primera fila son semejantes entre sí.
        \begin{table}[H]
            \centering
            \begin{tabular}{r|ccc|l}
                 & $A_1$ & $A_2$ & $A_3$ & \\ \hline
                 $tr(A)$ & $2$ & $2$ & $7$ & $A_1 \nsim A_3 \quad \land \quad A_2 \nsim A_3$ \\
                 $det(A)$ & $0$ & $-6$ & $\times$ & $A_1 \nsim A_2$\\
                 $P_A(\lambda)$ & $\times$ & $\times$ & $\times$ &
            \end{tabular}
            \caption{Resolución usando propiedades de las matrices semejantes}
        \end{table}
        Además, como los tres polinomios característicos son distintos, no son semejantes.

        \item Estudiar si $A_4$ y $A_5$ son semejantes.\\
        
        Tienen el mismo rango, traza y polinomio característico. Por tanto, no podemos descartar que sean semejantes.

        \begin{table}[H]
            \centering
            \begin{tabular}{c|c|c|c}
                Matriz & Valores Propios & Mult. Alg. & Mult. Geom. \\ \hline 
                $A_4$ & 3 & 3 & 2\\
                $A_5$ & 3 & 3 & 1\\
            \end{tabular}
            \caption{Valores propios con sus multiplicidades para cada matriz}
        \end{table}
    
        Como tienen multiplicidades geométricas distintas para el mismo valor propio, entonces no representan el mismo endomorfismo. Por tanto, no son semejantes.
        $$A_4 \nsim A_5$$
        
    \end{enumerate}

\end{ejercicio}

\begin{ejercicio}
    Tomamos $\bb{K}=\bb{R}$. Para todo $a$ real consideramos la matriz
    \begin{equation*}
        A = \left( \begin{array}{ccc}
            -1+a & 1 & -1+a \\
            1-a & -a & 1-a \\
            -1 & -1 & -1 \\
        \end{array}\right)
    \end{equation*}
    \begin{enumerate}
        \item Estudiar los valores de $a\in \bb{R}$ para los que $A$ es diagonalizable. \\

        Obtengo en primer lugar su polinomio característico:
        \begin{equation*}
        \begin{split}
        P_A(\lambda) & = det(A-\lambda I) = \left| \begin{array}{ccc}
                -1+a-\lambda & 1 & -1+a \\
                1-a & -a-\lambda & 1-a \\
                -1 & -1 & -1-\lambda
            \end{array}\right| \stackrel{C'_1=C_1-C_3}{=} \\  
            & = \left| \begin{array}{ccc}
                -\lambda & 1 & -1+a \\
                0 & -a-\lambda & 1-a \\
                \lambda & -1 & -1-\lambda
            \end{array}\right| = \lambda
            \left| \begin{array}{ccc}
                -1 & 1 & -1+a \\
                0 & -a-\lambda & 1-a \\
                1 & -1 & -1-\lambda
            \end{array}\right| \stackrel{F'_1=F_1+F_3}{=} \\
            & =
            \lambda
            \left| \begin{array}{ccc}
                0 & 0 & -2+a-\lambda \\
                0 & -a-\lambda & 1-a \\
                1 & -1 & -1-\lambda
            \end{array}\right| = \lambda(a-2-\lambda)(-a-\lambda)
        \end{split}
        \end{equation*}
    
        Los valores propios son: $\{0,a-2,-a\}$. Los casos a tener en cuenta son:
       \begin{itemize}
           \item Dos valores propios son iguales.
           \begin{equation*} \begin{array}{ll}
               a-2=0 & \longrightarrow a=2 \\
               -a=0 & \longrightarrow a=0 \\
               a-2=-a & \longrightarrow a=1 \\
           \end{array}\end{equation*}
           Por tanto, si $a\neq0,1,2$, entonces los tres valores propios son distintos y, por tanto, $A$ es diagonalizable.
    
           \item Caso $a=1$.
           \begin{equation*}\begin{split}
               V_{-1} & = \left\{ \left(\begin{array}{c}
                    x_1 \\
                    x_2 \\
                    x_3
               \end{array}\right) \in \bb{R}^3 \mid (A+I)\left(\begin{array}{c}
                    x_1 \\
                    x_2 \\
                    x_3
               \end{array}\right) = 0 \right\} \\
               & = \left\{ \left(\begin{array}{c}
                    x_1 \\
                    x_2 \\
                    x_3
               \end{array}\right) \in \bb{R}^3 \mid \left( \begin{array}{ccc}
                1 & 1 & 0 \\
                0 & 0 & 0 \\
                -1 & -1 & 0 \\
            \end{array}\right) \left(\begin{array}{c}
                    x_1 \\
                    x_2 \\
                    x_3
               \end{array}\right) = 0 \right\} \\
               & = \left\{ \left(\begin{array}{c}
                    x_1 \\
                    x_2  \\
                    x_3
               \end{array}\right) \in \bb{R}^3 \mid x_1+x_2=0 \right\}
           \end{split}\end{equation*}
           \begin{table}[H]
                \centering
                \begin{tabular}{c|c|c}
                    Valores Propios & Mult. Alg. & Mult. Geom. \\ \hline 
                    0 & 1 & 1\\
                    $-$1 & 2 & 2\\
                \end{tabular}
                \caption{Valores propios con sus multiplicidades}
            \end{table}
            Por tanto, para $a=1$ la matriz es diagonalizable.

            \item Caso $a=0$.
           \begin{equation*}\begin{split}
               V_0 & = \left\{ \left(\begin{array}{c}
                    x_1 \\
                    x_2 \\
                    x_3
               \end{array}\right) \in \bb{R}^3 \mid A\left(\begin{array}{c}
                    x_1 \\
                    x_2 \\
                    x_3
               \end{array}\right) = 0 \right\} \\
               & = \left\{ \left(\begin{array}{c}
                    x_1 \\
                    x_2 \\
                    x_3
               \end{array}\right) \in \bb{R}^3 \mid \left( \begin{array}{ccc}
                -1 & 1 & -1 \\
                1 & 0 & 1 \\
                -1 & -1 & -1 \\
            \end{array}\right) \left(\begin{array}{c}
                    x_1 \\
                    x_2 \\
                    x_3
               \end{array}\right) = 0 \right\} \\
               & = \left\{ \left(\begin{array}{c}
                    x_1 \\
                    x_2  \\
                    x_3
               \end{array}\right) \in \bb{R}^3 \left|
               \begin{array}{c}
                   -x_1+x_2-x_3 = 0  \\
                    x_1+x_3 = 0\\
                    x_1+x_2+x_3 = 0
               \end{array}\right.\right\} =
               \mathcal{L}\left(\left\{ \left(\begin{array}{c}
                    1 \\
                    0 \\
                    -1 \\
               \end{array}\right)
               \right\}\right)
           \end{split}\end{equation*}
           \begin{table}[H]
                \centering
                \begin{tabular}{c|c|c}
                    Valores Propios & Mult. Alg. & Mult. Geom. \\ \hline 
                    0 & 2 & 1\\
                    $-$2 & 1 & 1\\
                \end{tabular}
                \caption{Valores propios con sus multiplicidades}
            \end{table}
            Por tanto, para $a=0$ la matriz no es diagonalizable.

            \item Caso $a=2$.
           \begin{equation*}\begin{split}
               V_0 & = \left\{ \left(\begin{array}{c}
                    x_1 \\
                    x_2 \\
                    x_3
               \end{array}\right) \in \bb{R}^3 \mid A\left(\begin{array}{c}
                    x_1 \\
                    x_2 \\
                    x_3
               \end{array}\right) = 0 \right\} \\
               & = \left\{ \left(\begin{array}{c}
                    x_1 \\
                    x_2 \\
                    x_3
               \end{array}\right) \in \bb{R}^3 \mid \left( \begin{array}{ccc}
                1 & 1 & 1 \\
                -1 & -2 & -1 \\
                -1 & -1 & -1 \\
            \end{array}\right) \left(\begin{array}{c}
                    x_1 \\
                    x_2 \\
                    x_3
               \end{array}\right) = 0 \right\} \\
               & = \left\{ \left(\begin{array}{c}
                    x_1 \\
                    x_2  \\
                    x_3
               \end{array}\right) \in \bb{R}^3 \left|
               \begin{array}{c}
                   x_1+x_2+x_3 = 0  \\
                    x_1+2x_2+x_3 = 0\\
                    x_1+x_2+x_3 = 0
               \end{array}\right.\right\} =
               \mathcal{L}\left(\left\{ \left(\begin{array}{c}
                    1 \\
                    0 \\
                    -1 \\
               \end{array}\right)
               \right\}\right)
           \end{split}\end{equation*}
           \begin{table}[H]
                \centering
                \begin{tabular}{c|c|c}
                    Valores Propios & Mult. Alg. & Mult. Geom. \\ \hline 
                    0 & 2 & 1\\
                    $-$2 & 1 & 1\\
                \end{tabular}
                \caption{Valores propios con sus multiplicidades}
            \end{table}
            Por tanto, para $a=2$ la matriz no es diagonalizable.
       \end{itemize} 

       Por tanto, $A$ es diagonalizable $\forall a \in \bb{R}-\{0,2\}$.
       

        \item Diagonalizar la matriz para $a=0$, $a=1$ y $a=-1$ (si ello es posible).
        \begin{itemize}
            \item Para $a=0$\\
            $A$ no es diagonalizable.

            \item Para $a=1$\\
            Los valores propios son: $\{0, -1\}$.
            \begin{equation*}\begin{split}
            V_0 & = \left\{ \left(\begin{array}{c}
                    x_1 \\
                    x_2 \\
                    x_3
               \end{array}\right) \in \bb{R}^3 \mid A\left(\begin{array}{c}
                    x_1 \\
                    x_2 \\
                    x_3
               \end{array}\right) = 0 \right\} \\
               & = \left\{ \left(\begin{array}{c}
                    x_1 \\
                    x_2 \\
                    x_3
               \end{array}\right) \in \bb{R}^3 \mid \left( \begin{array}{ccc}
                0 & 1 & 0 \\
                0 & -1 & 0 \\
                -1 & -1 & -1 \\
            \end{array}\right) \left(\begin{array}{c}
                    x_1 \\
                    x_2 \\
                    x_3
               \end{array}\right) = 0 \right\} \\
               & = \left\{ \left(\begin{array}{c}
                    x_1 \\
                    x_2  \\
                    x_3
               \end{array}\right) \in \bb{R}^3 \left|
               \begin{array}{c}
                    x_2=0  \\
                    x_1+x_2+x_3 = 0 
               \end{array}\right. \right\} =
               \mathcal{L}\left(\left\{ \left(\begin{array}{c}
                    1 \\
                    0 \\
                    -1 \\
               \end{array}\right)
               \right\}\right)
           \end{split}\end{equation*}
           \begin{equation*}\begin{split}
            V_{-1} & = \dots =
               \mathcal{L}\left(\left\{ \left(\begin{array}{c}
                    0 \\
                    0 \\
                    1 \\
               \end{array}\right),
               \left(\begin{array}{c}
                    1 \\
                    -1 \\
                    0 \\
               \end{array}\right)
               \right\}\right)
           \end{split}\end{equation*}
           Por tanto, las matrices $P_{1}$ y $D_{1}$ son:
            \begin{equation*}
                D_{1} = \left( \begin{array}{ccc}
                    0 & 0 & 0 \\
                    0 & -1 & 0 \\
                    0 & 0 & -1 \\
                \end{array}\right) \qquad 
                P_{1} = \left( \begin{array}{ccc}
                    1 & 0 & 1 \\
                    0 & 0 & -1 \\
                    -1& 1 & 0 \\
                \end{array}\right)
            \end{equation*}

            \item Para $a=-1$\\
            Los valores propios son: $\{0, -3, 1\}$.
            \begin{equation*}\begin{split}
            V_0 & = \left\{ \left(\begin{array}{c}
                    x_1 \\
                    x_2 \\
                    x_3
               \end{array}\right) \in \bb{R}^3 \mid A\left(\begin{array}{c}
                    x_1 \\
                    x_2 \\
                    x_3
               \end{array}\right) = 0 \right\} \\
               & = \left\{ \left(\begin{array}{c}
                    x_1 \\
                    x_2 \\
                    x_3
               \end{array}\right) \in \bb{R}^3 \mid \left( \begin{array}{ccc}
                -2 & 1 & -2 \\
                2 & 1 & 2 \\
                -1 & -1 & -1 \\
            \end{array}\right) \left(\begin{array}{c}
                    x_1 \\
                    x_2 \\
                    x_3
               \end{array}\right) = 0 \right\} \\
               & = \left\{ \left(\begin{array}{c}
                    x_1 \\
                    x_2  \\
                    x_3
               \end{array}\right) \in \bb{R}^3 \left|
               \begin{array}{c}
                    2x_1-x_2+2x_3=0  \\
                    x_1+x_2+x_3 = 0 
               \end{array}\right. \right\} =
               \mathcal{L}\left(\left\{ \left(\begin{array}{c}
                    1 \\
                    0 \\
                    -1 \\
               \end{array}\right)
               \right\}\right)
           \end{split}\end{equation*}
           \begin{equation*}\begin{split}
            V_1 & = \left\{ \left(\begin{array}{c}
                    x_1 \\
                    x_2 \\
                    x_3
               \end{array}\right) \in \bb{R}^3 \mid (A-I)\left(\begin{array}{c}
                    x_1 \\
                    x_2 \\
                    x_3
               \end{array}\right) = 0 \right\} \\
               & = \left\{ \left(\begin{array}{c}
                    x_1 \\
                    x_2 \\
                    x_3
               \end{array}\right) \in \bb{R}^3 \mid \left( \begin{array}{ccc}
                -3 & 1 & -2 \\
                2 & 0 & 2 \\
                -1 & -1 & -2 \\
            \end{array}\right) \left(\begin{array}{c}
                    x_1 \\
                    x_2 \\
                    x_3
               \end{array}\right) = 0 \right\} \\
               & = \left\{ \left(\begin{array}{c}
                    x_1 \\
                    x_2  \\
                    x_3
               \end{array}\right) \in \bb{R}^3 \left|
               \begin{array}{c}
                    3x_1-x_2+2x_3=0  \\
                    x_1+x_3=0 \\
                    x_1+x_2+2x_3 = 0 
               \end{array}\right. \right\} =
               \mathcal{L}\left(\left\{ \left(\begin{array}{c}
                    1 \\
                    1 \\
                    -1 \\
               \end{array}\right)
               \right\}\right)
           \end{split}\end{equation*}
           \begin{equation*}\begin{split}
            V_{-3} & = \left\{ \left(\begin{array}{c}
                    x_1 \\
                    x_2 \\
                    x_3
               \end{array}\right) \in \bb{R}^3 \mid (A+3I)\left(\begin{array}{c}
                    x_1 \\
                    x_2 \\
                    x_3
               \end{array}\right) = 0 \right\} \\
               & = \left\{ \left(\begin{array}{c}
                    x_1 \\
                    x_2 \\
                    x_3
               \end{array}\right) \in \bb{R}^3 \mid \left( \begin{array}{ccc}
                1 & 1 & -2 \\
                2 & 4 & 2 \\
                -1 & -1 & 2 \\
            \end{array}\right) \left(\begin{array}{c}
                    x_1 \\
                    x_2 \\
                    x_3
               \end{array}\right) = 0 \right\} \\
               & = \left\{ \left(\begin{array}{c}
                    x_1 \\
                    x_2  \\
                    x_3
               \end{array}\right) \in \bb{R}^3 \left|
               \begin{array}{c}
                    x_1+x_2-2x_3=0  \\
                    x_1+2x_2+x_3 = 0 
               \end{array}\right. \right\} =
               \mathcal{L}\left(\left\{ \left(\begin{array}{c}
                    5 \\
                    -3 \\
                    1 \\
               \end{array}\right)
               \right\}\right)
           \end{split}\end{equation*}
            
           Por tanto, las matrices $P_{-1}$ y $D_{-1}$ son:
            \begin{equation*}
                D_{-1} = \left( \begin{array}{ccc}
                    0 & 0 & 0 \\
                    0 & 1 & 0 \\
                    0 & 0 & -3 \\
                \end{array}\right) \qquad 
                P_{-1} = \left( \begin{array}{ccc}
                    1 & 1 & 5 \\
                    0 & 1 & -3 \\
                    -1& -1 & 1 \\
                \end{array}\right)
            \end{equation*}
            
        \end{itemize}
        

        \item Razonar si las matrices obtenidas para $a=-2$ y para $a=4$ son semejantes.

        Para $a=-2$, los valores propios son: $\{0,2,-4\}$.

        Para $a=4$, los valores propios son: $\{0,2,-4\}$.

        Por tanto, como tienen los mismos valores propios, saldrá la misma $D$ al diagonalizarla. Por tanto, $A_{-2}\sim D \land D\sim A_4$. Al ser $\sim$ una relación de equivalencia, $A_4 \sim A_{-2}$
        
    \end{enumerate}
\end{ejercicio}

\begin{ejercicio}
    Estudiar los valores de $a$ para los que la siguiente matriz es diagonalizable. Estudiar el caso real y el caso complejo.
    \begin{equation*}
        A = \left(\begin{array}{ccc}
            1 & -2 & -2 \\
            -2 & a & 8 \\
            2 & 8 & a
        \end{array}\right)
    \end{equation*}

    \begin{equation*}\begin{split}
        P_A(\lambda) & = \left|\begin{array}{ccc}
            1-\lambda & -2 & -2 \\
            -2 & a-\lambda & 8 \\
            2 & 8 & a-\lambda
        \end{array}\right| = \left|\begin{array}{ccc}
            1-\lambda & -2 & 0 \\
            -2 & a-\lambda & 8-a+\lambda \\
            2 & 8 & -8 + a-\lambda
        \end{array}\right| = \\
        &= \left|\begin{array}{ccc}
            1-\lambda & -2 & 0 \\
            0 & 8+a-\lambda & 0 \\
            2 & 8 & -8 + a-\lambda
        \end{array}\right| = (-8+a-\lambda)(8+a-\lambda)(1-\lambda)
    \end{split}\end{equation*}

    Por tanto, los valores propios son: $\{-8+a, 8+a, 1\}$. Los casos a tener en cuenta son:
    \begin{itemize}
       \item Dos valores propios son iguales.
       \begin{equation*} \begin{array}{ll}
           -8+a=1 & \longrightarrow a=9 \\
           8+a=1 & \longrightarrow a=-7 \\
           8+a=-8+a & \longrightarrow \nexists sol \\
       \end{array}\end{equation*}
       Por tanto, si $a\neq-7,9$, entonces los tres valores propios son distintos y, por tanto, $A$ es diagonalizable.

       \item Caso $a=-7$.
           \begin{equation*}\begin{split}
               V_{1} & = \left\{ \left(\begin{array}{c}
                    x_1 \\
                    x_2 \\
                    x_3
               \end{array}\right) \in \bb{R}^3 \mid (A-I)\left(\begin{array}{c}
                    x_1 \\
                    x_2 \\
                    x_3
               \end{array}\right) = 0 \right\} \\
               & = \left\{ \left(\begin{array}{c}
                    x_1 \\
                    x_2 \\
                    x_3
               \end{array}\right) \in \bb{R}^3 \mid \left( \begin{array}{ccc}
                0 & -2 & -2 \\
                -2 & -8 & 8 \\
                2 & 8 & -8 \\
            \end{array}\right) \left(\begin{array}{c}
                    x_1 \\
                    x_2 \\
                    x_3
               \end{array}\right) = 0 \right\} \\
               & = \left\{ \left(\begin{array}{c}
                    x_1 \\
                    x_2  \\
                    x_3
               \end{array}\right) \in \bb{R}^3 \left| \begin{array}{c}
                    x_2+x_3=0 \\
                    x_1+4x_2-4x_3 = 0
               \end{array}\right| \right\}
           \end{split}\end{equation*}
           \begin{table}[H]
                \centering
                \begin{tabular}{c|c|c}
                    Valores Propios & Mult. Alg. & Mult. Geom. \\ \hline 
                    1 & 2 & 1\\
                    $-$15 & 1 & 1\\
                \end{tabular}
                \caption{Valores propios con sus multiplicidades}
            \end{table}
            Por tanto, para $a=-7$ la matriz no es diagonalizable.
            
        \item Caso $a=9$.
           \begin{equation*}\begin{split}
               V_{1} & = \left\{ \left(\begin{array}{c}
                    x_1 \\
                    x_2 \\
                    x_3
               \end{array}\right) \in \bb{R}^3 \mid (A-I)\left(\begin{array}{c}
                    x_1 \\
                    x_2 \\
                    x_3
               \end{array}\right) = 0 \right\} \\
               & = \left\{ \left(\begin{array}{c}
                    x_1 \\
                    x_2 \\
                    x_3
               \end{array}\right) \in \bb{R}^3 \mid \left( \begin{array}{ccc}
                0 & -2 & -2 \\
                -2 & 8 & 8 \\
                2 & 8 & 8 \\
            \end{array}\right) \left(\begin{array}{c}
                    x_1 \\
                    x_2 \\
                    x_3
               \end{array}\right) = 0 \right\} \\
               & = \left\{ \left(\begin{array}{c}
                    x_1 \\
                    x_2  \\
                    x_3
               \end{array}\right) \in \bb{R}^3 \left| \begin{array}{c}
                    x_2+x_3=0 \\
                    x_1+4x_2+4x_3 = 0 \\
                    -x_1+4x_2+4x_3 = 0
               \end{array}\right. \right\}
           \end{split}\end{equation*}
           \begin{table}[H]
                \centering
                \begin{tabular}{c|c|c}
                    Valores Propios & Mult. Alg. & Mult. Geom. \\ \hline 
                    1 & 2 & 1\\
                    17 & 1 & 1\\
                \end{tabular}
                \caption{Valores propios con sus multiplicidades}
            \end{table}
            Por tanto, para $a=9$ la matriz no es diagonalizable.
    \end{itemize}

    Por tanto, $A$ es diagonalizable $\forall a \in \bb{C} \backslash \{-7,9\}$.
\end{ejercicio}

\begin{ejercicio}
    Sea $A\in \mathcal{M}_4(\bb{R})$. Estudiar los valores de $a\in\bb{R}$ para los que la matriz $A$ es diagonalizable.
    \begin{equation*}
        A = \left( \begin{array}{cccc}
            1 & 0 & 1 & 0 \\
            0 & a & 0 & a \\
            1 & 1 & 1 & 0 \\
            0 & a & 0 & 1 \\
        \end{array}\right)
    \end{equation*}

    \begin{equation*}\begin{split}
        P_A(\lambda) & = \left|\begin{array}{cccc}
            1-\lambda & 0 & 1 & 0 \\
            0 & a-\lambda & 0 & a \\
            1 & 1 & 1-\lambda & 0 \\
            0 & a & 0 & 1-\lambda \\
        \end{array} \right| = \left|\begin{array}{cccc}
            2-\lambda & 0 & 1 & 0 \\
            0 & a-\lambda & 0 & a \\
            2-\lambda & 1 & 1-\lambda & 0 \\
            0 & a & 0 & 1-\lambda \\
        \end{array} \right| = \\
        & = \left|\begin{array}{cccc}
            0 & -1 & \lambda & 0 \\
            0 & a-\lambda & 0 & a \\
            2-\lambda & 1 & 1-\lambda & 0 \\
            0 & a & 0 & 1-\lambda \\
        \end{array} \right|
        = (2-\lambda) \left|\begin{array}{ccc}
            -1 & \lambda & 0 \\
            a-\lambda & 0 & a \\
            a & 0 & 1-\lambda \\
        \end{array} \right| = \\
        & = -\lambda(2-\lambda)((a-\lambda)(1-\lambda)-a^2) = -\lambda(2-\lambda)(\lambda^2-(a+1)\lambda-a^2+a)
    \end{split}\end{equation*}

    Veo el número de soluciones de la ecuación $\lambda^2-(a+1)\lambda-a^2+a = 0$
    \begin{equation*}
        \Delta = (a+1)^2 +4a^2-4a = (a-1)^2+4a^2 > 0
    \end{equation*}
    
    Por tanto, la ecuación tiene dos soluciones. Para ver los casos en los que los valores propios se repiten, veamos si $$\exists a\in \bb{R} \mid \lambda^2-(a+1)\lambda-a^2+a = 0, \text{ con } \lambda=0,2$$
    \begin{itemize}
        \item \underline{$\lambda=0$}:
        $\Longrightarrow -a^2+a = a(-a+1) = 0 \Longrightarrow a = 0,1$

        \item \underline{$\lambda=2$}:
        $\Longrightarrow 4-2a-2-a^2+a = -a^2-a+2 = 0 \Longrightarrow a = 1,-2$
    \end{itemize}

    Por tanto,
    \begin{itemize}
        \item \underline{Si $a\neq \{-2,0,1\}$}:\\
        Hay $4$ valores propios distintos, por lo que $A$ es diagonalizable.

        \item \underline{Si $a=0$}:\\
        Hay dos valores propios con $m_i=1$, pero la multiplicidad algebraica del valor propio $0$ es doble ($m_0=2$). Veamos su multiplicidad geométrica:
        \begin{equation*}\begin{split}
               V_{0} & = \left\{ \left(\begin{array}{c}
                    x_1 \\
                    x_2 \\
                    x_3 \\
                    x_4
               \end{array}\right) \in \bb{R}^4 \mid A\left(\begin{array}{c}
                    x_1 \\
                    x_2 \\
                    x_3 \\
                    x_4
               \end{array}\right) = 0 \right\} \\
               & = \left\{ \left(\begin{array}{c}
                    x_1 \\
                    x_2 \\
                    x_3 \\
                    x_4
               \end{array}\right) \in \bb{R}^4 \mid \left( \begin{array}{cccc}
                    1 & 0 & 1 & 0 \\
                    0 & 0 & 0 & 0 \\
                    1 & 1 & 1 & 0 \\
                    0 & 0 & 0 & 1 \\
            \end{array}\right) \left(\begin{array}{c}
                    x_1 \\
                    x_2 \\
                    x_3 \\
                    x_4
               \end{array}\right) = 0 \right\} \\
               & = \left\{ \left(\begin{array}{c}
                    x_1 \\
                    x_2  \\
                    x_3 \\
                    x_4
               \end{array}\right) \in \bb{R}^4 \left| \begin{array}{c}
                    x_1+x_3=0 \\
                    x_1+x_2+x_3=0\\
                    x_4=0
               \end{array}\right. \right\}
       \end{split}\end{equation*}
        \begin{table}[H]
            \centering
            \begin{tabular}{c|c|c}
                Valores Propios & Mult. Alg. & Mult. Geom. \\ \hline 
                2 & 1 & 1\\
                0 & 2 & 1\\
                - & 1 & 1
            \end{tabular}
            \caption{Valores propios con sus multiplicidades}
        \end{table}
        Por tanto, para $a=0$, $A$ no es diagonalizable.

        \item \underline{Si $a=1$}:\\
        Hay dos valores propios con $m_i=2$. Veamos su multiplicidad geométrica:
        \begin{equation*}\begin{split}
               V_{0} & = \left\{ \left(\begin{array}{c}
                    x_1 \\
                    x_2 \\
                    x_3 \\
                    x_4
               \end{array}\right) \in \bb{R}^4 \mid A\left(\begin{array}{c}
                    x_1 \\
                    x_2 \\
                    x_3 \\
                    x_4
               \end{array}\right) = 0 \right\} \\
               & = \left\{ \left(\begin{array}{c}
                    x_1 \\
                    x_2 \\
                    x_3 \\
                    x_4
               \end{array}\right) \in \bb{R}^4 \mid \left( \begin{array}{cccc}
                    1 & 0 & 1 & 0 \\
                    0 & 1 & 0 & 1 \\
                    1 & 1 & 1 & 0 \\
                    0 & 1 & 0 & 1 \\
            \end{array}\right) \left(\begin{array}{c}
                    x_1 \\
                    x_2 \\
                    x_3 \\
                    x_4
               \end{array}\right) = 0 \right\} \\
               & = \left\{ \left(\begin{array}{c}
                    x_1 \\
                    x_2  \\
                    x_3 \\
                    x_4
               \end{array}\right) \in \bb{R}^4 \left| \begin{array}{c}
                    x_1+x_3=0 \\
                    x_1+x_2+x_3=0\\
                    x_2+x_4=0
               \end{array}\right. \right\}
       \end{split}\end{equation*}
        \begin{table}[H]
            \centering
            \begin{tabular}{c|c|c}
                Valores Propios & Mult. Alg. & Mult. Geom. \\ \hline 
                0 & 2 & 1\\
                2 & 2 & -\\
            \end{tabular}
            \caption{Valores propios con sus multiplicidades}
        \end{table}
        Por tanto, para $a=1$, $A$ no es diagonalizable.

        \item \underline{Si $a=-2$}:\\
        Hay dos valores propios con $m_i=1$, pero la multiplicidad algebraica del valor propio $2$ es doble ($m_2=2$). Veamos su multiplicidad geométrica:
        \begin{equation*}\begin{split}
               V_{2} & = \left\{ \left(\begin{array}{c}
                    x_1 \\
                    x_2 \\
                    x_3 \\
                    x_4
               \end{array}\right) \in \bb{R}^4 \mid (A-2I)\left(\begin{array}{c}
                    x_1 \\
                    x_2 \\
                    x_3 \\
                    x_4
               \end{array}\right) = 0 \right\} \\
               & = \left\{ \left(\begin{array}{c}
                    x_1 \\
                    x_2 \\
                    x_3 \\
                    x_4
               \end{array}\right) \in \bb{R}^4 \mid \left( \begin{array}{cccc}
                    -1 & 0 & 1 & 0 \\
                    0 & -4 & 0 & -2 \\
                    1 & 1 & -1 & 0 \\
                    0 & -2 & 0 & -1 \\
            \end{array}\right) \left(\begin{array}{c}
                    x_1 \\
                    x_2 \\
                    x_3 \\
                    x_4
               \end{array}\right) = 0 \right\} \\
               & = \left\{ \left(\begin{array}{c}
                    x_1 \\
                    x_2  \\
                    x_3 \\
                    x_4
               \end{array}\right) \in \bb{R}^4 \left| \begin{array}{c}
                    -x_1+x_3=0 \\
                    x_1+x_2-x_3=0\\
                    2x_2+x_4=0
               \end{array}\right. \right\}
       \end{split}\end{equation*}
        \begin{table}[H]
            \centering
            \begin{tabular}{c|c|c}
                Valores Propios & Mult. Alg. & Mult. Geom. \\ \hline 
                2 & 2 & 1\\
                0 & 1 & 1\\
                - & 1 & 1
            \end{tabular}
            \caption{Valores propios con sus multiplicidades}
        \end{table}
        Por tanto, para $a=-2$, $A$ no es diagonalizable.
        
        
    \end{itemize}
    
\end{ejercicio}

\begin{ejercicio}
    Sea $A\in \mathcal{M}_n(\bb{R})$ diagonalizable. Demostrar que su matriz transpuesta también lo es. Razonar que, en ese caso, $A\sim A^t$.
    \begin{multline*}
        A \text{ diagonalizable } \Longrightarrow D = P^{-1}AP \Longrightarrow D^t = (P^{-1}AP)^t = P^t A^t (P^{-1})^t = P^t A^t (P^t)^{-1}
    \end{multline*}
    
    Además, como $D$ es diagonal, $D^t=D$. Por tanto,
    \begin{equation*}
        D = P^t A^t ((P^t)^{-1}
    \end{equation*}

    Por tanto, como $D$ es diagonal y $P^t$ es regular, $A^t$ es diagonalizable.

    Como $A\sim D \land D\sim A^t$, al ser $\sim$ una relación de equivalencia, $A \sim A^t$. $\hfill \qed$
\end{ejercicio}

\begin{ejercicio}
    Demostrar que toda $A\in \mathcal{S}_2(\bb{R})$ es diagonalizable.
    \begin{proof}
        Sea $ A=\left(\begin{array}{cc}
            a & b \\
            b & c
        \end{array}\right)$.
    
        Calculamos su polinomio característico.
        \begin{equation*}
            P_A(\lambda) = \lambda^2 - tr(A)\lambda + det(A) = \lambda^2-(a+c)\lambda + (ac-b^2)
        \end{equation*}
    
        Su discriminante es $\Delta=(a+c)^2-4(ac-b^2) = a^2+c^2+2ac-4ac+b^2 = a^2+c^2-2ac+b^2 = (a-c)^2+4b^2 \geq 0$.
        \begin{itemize}
            \item Si $\Delta > 0 \Longrightarrow A$ tiene 2 soluciones $\Longrightarrow A$ es diagonalizable.
    
            \item Si $\Delta = 0 \Longrightarrow a=c,\quad b=0 \Longrightarrow A=\left(\begin{array}{cc}
            a & 0 \\
            0 & a
        \end{array}\right)$ es diagonalizable. 
        \end{itemize}
    \end{proof}
\end{ejercicio}

\begin{observacion} En el caso de $\bb{K}=\bb{C}$ esto no es cierto. Como contraejemplo, ver el ejercicio \ref{Ejercicio7}.\ref{Ejercicio7.1}.
\end{observacion}


\begin{ejercicio}
\label{Ejercicio7}
    Sean $A_1, A_2, A_3 \in \mathcal{M}_2(\bb{C})$. Estudiar cuáles son diagonalizables.
    \begin{equation*}
        A_1 = \left( \begin{array}{cc}
            1 & i \\
            i & -1 \\
        \end{array}\right) \qquad
        A_2 = \left( \begin{array}{cc}
            1 & i \\
            -i & 1 \\
        \end{array}\right) \qquad
        A_3 = \left( \begin{array}{cc}
            1 & i \\
            i & 1 \\
        \end{array}\right)
    \end{equation*}

    \begin{enumerate}
        \item Veamos si $A_1$ es diagonalizable.
        \label{Ejercicio7.1}
        \begin{equation*}
            P_{A_1}(\lambda) = |A_1-\lambda I| = \lambda^2 - tr(A_1)\lambda + det(A_1) = \lambda^2
        \end{equation*}
        Por tanto, el único valor propio es el $\{0\}$. Calculemos su multiplicidad geométrica.
        \begin{equation*}\begin{split}
               V_{0} & = \left\{ \left(\begin{array}{c}
                    x_1 \\
                    x_2
               \end{array}\right) \in \bb{C}^2 \mid A\left(\begin{array}{c}
                    x_1 \\
                    x_2
               \end{array}\right) = 0 \right\} \\
               & = \left\{ \left(\begin{array}{c}
                    x_1 \\
                    x_2
               \end{array}\right) \in \bb{C}^2 \mid \left( \begin{array}{cc}
                    1 & i \\
                    i & -1 \\
                \end{array}\right) \left(\begin{array}{c}
                    x_1 \\
                    x_2
               \end{array}\right) = 0 \right\} \\
               & = \left\{ \left(\begin{array}{c}
                    x_1 \\
                    x_2
               \end{array}\right) \in \bb{C}^2 \left| \begin{array}{c}
                    x_1+ix_2=0 \\
                    ix_1-x_2 = 0 \\
               \end{array}\right. \right\}
               = \left\{ \left(\begin{array}{c}
                    x_1 \\
                    x_2
               \end{array}\right) \in \bb{C}^2 \left| \begin{array}{c}
                    x_1+ix_2=0 \\
               \end{array}\right. \right\}
           \end{split}\end{equation*}
           \begin{table}[H]
                \centering
                \begin{tabular}{c|c|c}
                    Valores Propios & Mult. Alg. & Mult. Geom. \\ \hline 
                    0 & 2 & 1\\
                \end{tabular}
                \caption{Valores propios con sus multiplicidades}
            \end{table}

            Por tanto, $A_1$ no es diagonalizable.
        
        \item Veamos si $A_2$ es diagonalizable.
        \begin{equation*}
            P_{A_2}(\lambda) = |A_2-\lambda I| = \lambda^2 - tr(A_2)\lambda + det(A_2) = \lambda^2 -2\lambda = \lambda(\lambda-2)
        \end{equation*}
        Por tanto, los valores propios son $\{0,2\}$.
           \begin{table}[H]
                \centering
                \begin{tabular}{c|c|c}
                    Valores Propios & Mult. Alg. & Mult. Geom. \\ \hline 
                    0 & 1 & 1\\
                    2 & 1 & 1\\
                \end{tabular}
                \caption{Valores propios con sus multiplicidades}
            \end{table}
            
            Por tanto, $A_2$ sí es diagonalizable.
            
        \item Veamos si $A_3$ es diagonalizable.
        \begin{equation*}
            P_{A_3}(\lambda) = |A_3-\lambda I| = \lambda^2 - tr(A_3)\lambda + det(A_3) = \lambda^2 -2\lambda  +2 = (\lambda-1-i)(\lambda-1+i)
        \end{equation*}
        Por tanto, los valores propios son $\{1-i,1+i\}$.
           \begin{table}[H]
                \centering
                \begin{tabular}{c|c|c}
                    Valores Propios & Mult. Alg. & Mult. Geom. \\ \hline 
                    $1-i$ & 1 & 1\\
                    $1+i$ & 1 & 1\\
                \end{tabular}
                \caption{Valores propios con sus multiplicidades}
            \end{table}
            Por tanto, $A_3$ sí es diagonalizable.
    \end{enumerate}
\end{ejercicio}

\begin{ejercicio}
    Sea $A\in \mathcal{M}_2(\bb{C})$. Consideramos la matriz traspuesta $A^t$, la matriz conjugada $\bar{A}$ y la matriz traspuesta conjugada $\bar{A}^t$,
    \begin{equation*}
        A=\left(\begin{array}{cc}
            a+ib & c+id \\
            e+if & g+ih
        \end{array}\right) \qquad
        A^t=\left(\begin{array}{cc}
            a+ib & e+if \\
            c+id & g+ih
        \end{array}\right)
    \end{equation*}
    \begin{equation*}
        \bar{A}=\left(\begin{array}{cc}
            a-ib & c-id \\
            e-if & g-ih
        \end{array}\right) \qquad
        \bar{A}^t=\left(\begin{array}{cc}
            a-ib & e-if \\
            c-id & g-ih
        \end{array}\right)
    \end{equation*}
    Decimos que $A$ es \emph{simétrica} si $A=A^t$ y que es \emph{hermítica} si $A=\bar{A}^t$.

    \begin{enumerate}
        \item Demostrar que una matriz simétrica compleja no es necesariamente diagonalizable.

        Como contraejemplo, ver el ejercicio \ref{Ejercicio7}.\ref{Ejercicio7.1}.

        \item Demostrar que toda matriz hermítica es diagonalizable y que sus valores propios son reales.

        \begin{equation*}
            A=\bar{A}^t \Longrightarrow \left\{
            \begin{array}{ll}
                a+ib = a-ib & \longrightarrow b=0 \\
                c+id = e-if & \longrightarrow c=e-if-id \\
                e+if = c-id & \\
                g+ih = g-ih & \longrightarrow h=0
            \end{array}
            \right.
        \end{equation*}

        Por tanto, $e+if = e-if -id -id \Longrightarrow 2if = -2id \Longrightarrow f=-d \Longrightarrow c=e$.

        Por tanto, dado $a.b.c.d \in \bb{R}$, una matriz $A\in \mathcal{M}_2(\bb{C})$ hermítica es:
        \begin{equation*}
            A=\left(\begin{array}{cc}
            a & c+id \\
            c-id & b
        \end{array}\right)
        \end{equation*}

        Calculamos su polinomio característico:
        \begin{equation*}
            P_A(\lambda) = |A-\lambda I| = \lambda^2 - tr(A)\lambda + det(\lambda) = \lambda^2 -(a+b)\lambda + ab-c^2-d^2
        \end{equation*}
        $$\Delta = (a+b)^2 -4(ab-c^2 - d^2) = a^2+b^2+2ab -4ab +4c^2 +4d^2 = (a-b)^2 +4(c^2+d^2) \geq 0$$

        \begin{itemize}
            \item Si $\Delta > 0 \Longrightarrow$ 
            Hay dos soluciones distintas y, por tanto, $A$ es diagonalizable.

            \item Si $\Delta = 0 \Longrightarrow a=b, \quad  c=d=0 \Longrightarrow A=\left(\begin{array}{cc}
            a & 0 \\
            0 & a
        \end{array}\right)$ es diagonalizable.
        \end{itemize}
    \end{enumerate}
\end{ejercicio}


\begin{ejercicio}
    Sea $A\in \mathcal{M}_n(\bb{R})$.
    \begin{enumerate}
        \item Demostrar que  $A$ diagonalizable $\Longrightarrow A^2-2A+I$ diagonalizable.

        $A \text{ diagonalizable } \Longrightarrow D = P^{-1}AP \Longrightarrow A=PDP^{-1}$. Por tanto,
        $$A^2-2A+I = PD^2P^{-1} -2PDP^{-1} + PIP^{-1} = P[D^2-2D + I]P^{-1}$$

        Por tanto, como $D^2-2D + I$ es diagonal y $P$ es regular, $A^2-2A+I$ es semejante a una matriz diagonal, por lo que es diagonalizable.

        \item Razonar que  $A^2-2A+I$ diagonalizable $\nRightarrow A$ diagonalizable.\\

        Sea $A=\left( \begin{array}{cc}
            1 & 0 \\
            1 & 1
        \end{array}\right)$.

        $A^2-2A + I = (A-I)^2 =\left( \begin{array}{cc}
            0 & 0 \\
            0 & 0
        \end{array}\right) = 0$. Esta matriz es diagonalizable ya que $D=P^{-1}0P$, con $D=0$ y $P=I$.

        Sin embargo, el polinomio característico de $A$ es $P_A(\lambda) = \lambda^2-2\lambda + 1 = (\lambda-1)^2$.
        $$rg(A-I) = 1 \Longrightarrow \dim V_1 = 2-1 = 1$$
        Como $n_1 = 1 \neq 2 = m_1$, $A$ no es diagonalizable.
    \end{enumerate}
\end{ejercicio}

\begin{ejercicio}
    $A\in \mathcal{M}_2(\bb{R})$ no diagonalizable y con un valor propio $a\in\bb{R}$ de multiplicidad algebraica $m_a=2$. Demostrar que $A\sim B$, con
    \begin{equation*}
        B=\left( \begin{array}{cc}
            a & 0 \\
            1 & a
        \end{array}\right)
    \end{equation*}

    \begin{proof}
        Sea $f\in End(\bb{R}^2)$ con $A=M(f, \mathcal{B})$.
        
        Como $A$ no es diagonalizable, $n_a \neq 2$, por lo que $n_a = \dim V_a = 1$.

        Busco una base de $\bb{R}^2 \; \mathcal{B} = \{v_1, v_2\}\;\; v_2 \in V_a - \{0\}$. Como $\dim V_a = 1, \{v_2\}$ base de $V_a$. Por tanto,
        $$\begin{array}{rl}
            v_1 & \longrightarrow bv_1 + cv_2 \\
            v_2 & \longrightarrow av_2
        \end{array} \qquad \qquad 
        M(f, \mathcal{B}) = A = \left( \begin{array}{cc}
            b & 0 \\
            c & a
        \end{array} \right)$$

        Como la multiplicidad algebraica de $a$ es $2$, $$P_f(\lambda) = (a-\lambda)^2 = \left| \begin{array}{cc}
            b-\lambda & 0 \\
            c & a-\lambda
        \end{array} \right| = (b-\lambda)(a-\lambda) \Longrightarrow a = b$$
        Además, si fuese $c=0$ sería diagonalizable, por lo que $c\neq 0$.

        Sea ahora $\bar{\mathcal{B}} = \{\bar{v}_1,\bar{v}_2\}$, con $\bar{v}_1=v_1$ y $\bar{v}_2 = cv_2$. Forman base ya que $c\neq 0$ y $\mathcal{B}$ es una base.
        Sabemos que $$f(\bar{v}_2) = f(cv_2) = cf(v_2) = cav_2 = a\bar{v}_2$$
        $$\begin{array}{rl}
            \bar{v}_1 & \longrightarrow a\bar{v_1} + \bar{v_2} \\
            \bar{v_2} & \longrightarrow a\bar{v_2}
        \end{array} \qquad \qquad 
        M(f, \bar{\mathcal{B}}) = B = \left( \begin{array}{cc}
            a & 0 \\
            1 & a
        \end{array} \right)$$

        Por tanto, como la matriz $A$ y la matriz $B$ representan el mismo endormorfismo en dos bases distintas ($\mathcal{B}$ y $\bar{\mathcal{B}}$ respectivamente), las matrices son semejantes ($A\sim ~B$).
    \end{proof}
\end{ejercicio}


\begin{ejercicio}
    Sea $A\in \mathcal{M}_4(\bb{R})$. Estudiar si $A$ es diagonalizable.
    \begin{equation*}
        A = \left( \begin{array}{cccc}
            1 & 1 & 1 & 1 \\
            1 & 1 & 1 & 1 \\
            1 & 1 & 1 & 1 \\
            1 & 1 & 1 & 1 \\
        \end{array}\right)
    \end{equation*}

    $rg(A)=1 \Longrightarrow \dim Im(f) = 1 \Longrightarrow \dim Ker(f)=3$.
    
    Por tanto, como $Ker(f)\neq\{0\}\Longrightarrow 0$  es un valor propio. $V_0(f)=Ker(f)$

    Los valores propios de $A$ son $\{\lambda_0 \in \bb{R} \mid |A-\lambda_0 I| = 0\}$.
    \begin{equation*}
        |A-\lambda_0 I| = \left| \begin{array}{cccc}
            1-\lambda_0 & 1 & 1 & 1 \\
            1 & 1-\lambda_0 & 1 & 1 \\
            1 & 1 & 1-\lambda_0 & 1 \\
            1 & 1 & 1 & 1-\lambda_0 \\
        \end{array}\right| = 0
    \end{equation*}

    Como podemos ver, efectivamente $\lambda_0 = 0$ es un valor propio. Para $\lambda_0=4$, los vectores columna cumplen que la suma de sus componentes es nula, es decir, pertenecen al hiperplano de $\bb{R}$ con ecuación implícita $x_1+x_2+x_3+x_4 = 0$. Como la dimensión de ese hiperplano es $3$, uno de los vectores será linealmente dependiente y por tanto el determinante es nulo.

    Como $Ker(f) = V_0,\;\dim V_0 = 3$. Por tanto, la multiplicidad algebraica de $0\;m_0$ cumple que $3 \leq m_0 \leq 4$. Pero $m_0\neq 4$ porque sino sería el único valor propio. Por tanto, $m_0=3$ y $m_4=1$.
    
    
    \begin{table}[H]
        \centering
        \begin{tabular}{c|c|c}
            Valores Propios & Mult. Alg. & Mult. Geom. \\ \hline 
            0 & 3 & 3\\
            4 & 1 & 1\\
        \end{tabular}
        \caption{Valores propios con sus multiplicidades}
    \end{table}

    Por tanto, es diagonalizable. Busquemos las matrices $P$ y $D$. Calculamos en primer lugar cada subespacio propio.
    \begin{equation*}
        V_0 = \left\{x\in\bb{R}^4 \mid x_1+x_2+x_3+x_4 = 0\right\} = \mathcal{L}\left(\left\{ \left(\begin{array}{c}
                1 \\
                - 1 \\
                0 \\
                0 \\
           \end{array}\right),
           \left(\begin{array}{c}
                1 \\
                0 \\
                -1 \\
                0 \\
           \end{array}\right),
           \left(\begin{array}{c}
                1 \\
                0 \\
                0 \\
                -1 \\
           \end{array}\right)
           \right\}\right)
    \end{equation*}

    \begin{equation*}
        V_4 = \left\{x\in\bb{R}^4 \left| 
        \begin{array}{cl}
            -3x_1+x_2+x_3+x_4 & = 0  \\
            \dots & = 0  \\
            \dots & = 0  \\
        \end{array}\right.
        \right\} = \mathcal{L}\left(\left\{ \left(\begin{array}{c}
                1 \\
                1 \\
                1 \\
                1 \\
           \end{array}\right)
           \right\}\right)
    \end{equation*}

    Por tanto, la base de vectores propios es:
    \begin{equation*}
        \mathcal{B} = \left\{
        \left(\begin{array}{c}
                1 \\
                - 1 \\
                0 \\
                0 \\
           \end{array}\right),
           \left(\begin{array}{c}
                1 \\
                0 \\
                -1 \\
                0 \\
           \end{array}\right),
           \left(\begin{array}{c}
                1 \\
                0 \\
                0 \\
                -1 \\
           \end{array}\right),
           \left(\begin{array}{c}
                1 \\
                1 \\
                1 \\
                1 \\
           \end{array}\right)
        \right\}
    \end{equation*}

    Por tanto, las matrices $P$ y $D$ son:
    \begin{equation*}
        D = \left( \begin{array}{cccc}
            0 & 0 & 0 & 0 \\
            0 & 0 & 0 & 0 \\
            0 & 0 & 0 & 0 \\
            0 & 0 & 0 & 4 \\
        \end{array}\right) \qquad 
        P = \left( \begin{array}{cccc}
            1 & 1 & 1 & 1 \\
            -1 & 0 & 0 & 1 \\
            0& -1 & 0 & 1 \\
            0 & 0 & -1 & 1 \\
        \end{array}\right)
    \end{equation*}
\end{ejercicio}

\begin{ejercicio}
    Sea $A\in \mathcal{M}_3(\bb{R})$. Estudiar si $A$ es diagonalizable.
    \begin{equation*}
        A = \left( \begin{array}{ccc}
            1 & 3 & 0 \\
            3 & -2 & -1 \\
            0 & -1 & 1 \\
        \end{array}\right)
    \end{equation*}

    Calculamos su polinomio característico:
    \begin{equation*}
    P_A(\lambda) = det(A-\lambda I) = \left| \begin{array}{ccc}
            1-\lambda & 3 & 0 \\
            3 & -2-\lambda & -1 \\
            0 & -1 & 1-\lambda \\
        \end{array}\right|=  -\lambda^3 +13\lambda - 12    
    \end{equation*}

    Sus posibles raíces en $\bb{Q}$ son: $\{\pm1, \pm2, \pm3, \pm4, \pm6, \pm12\}$.

    \begin{figure}[H]
        \centering
        \polyhornerscheme[x=1]{-x^3+13x-12}
    \end{figure}
    Las raíces de $-\lambda^2 -\lambda +12 = 0$ son $\lambda_1=-4$ y $\lambda_2=3$.
    
    Por tanto, $P_A(\lambda)=-(\lambda-1)(\lambda+4)(\lambda-3)$.
    
    \begin{table}[H]
        \centering
        \begin{tabular}{c|c|c}
            Valores Propios & Mult. Alg. & Mult. Geom. \\ \hline 
            1 & 1 & 1\\
            $-$4 & 1 & 1\\
            3 & 1 & 1\\
        \end{tabular}
        \caption{Valores propios con sus multiplicidades}
    \end{table}

    Por tanto, $A$ es diagonalizable.
\end{ejercicio}

\begin{ejercicio}\label{Ej:TodosUnos}
    Sea $A=\mathcal{M}_n(\bb{R})$ con todos sus coeficientes $1$. ¿Es diagonalizable?

    $$A = (1)_{i,j} \qquad \forall i,j\mid 1 \leq i,j \leq n$$

    $rg(A)=1 \Longrightarrow \dim Im(f) = 1 \Longrightarrow \dim Ker(f)=n-1$.
    
    Por tanto, como $Ker(f)\neq\{0\}\Longrightarrow 0$  es un valor propio. $V_0(f)=Ker(f)$

    \begin{table}[H]
        \centering
        \begin{tabular}{c|c|c}
            Valores Propios & Mult. Alg. & Mult. Geom. \\ \hline 
            0 & $n-1 \leq m_0 \leq n$ & $n-1$\\
            - & - & -\\
        \end{tabular}
        \caption{Valores propios con lo que sabemos hasta el momento.}
    \end{table}
    
    Para completar las multiplicidades, es necesario saber si hay más valores propios.
    \begin{equation*}
        (A-\lambda_0 I) = \left( \begin{array}{cccc}
            1-\lambda_0 & 1 & \dots & 1 \\
            1 & \ddots &  & \vdots \\
            \vdots &  & \ddots & 1 \\
            1 & \dots & 1 & 1-\lambda_0 \\
        \end{array}\right) = \left\{
        \begin{array}{ccc}
            1 & \text{ si } & i\neq j\\
            1-\lambda_0 & \text{ si } & i= j
        \end{array}
        \right.
    \end{equation*}

    Los valores propios de $A$ son $\{\lambda_0 \in \bb{R} \mid |A-\lambda_0 I| = 0\}$.
    \begin{equation*}
        |A-\lambda_0 I| = \left| \begin{array}{cccc}
            1-\lambda_0 & 1 & \dots & 1 \\
            1 & \ddots &  & \vdots \\
            \vdots &  & \ddots & 1 \\
            1 & \dots & 1 & 1-\lambda_0 \\
        \end{array}\right| = 0
    \end{equation*}

    $\lambda_0 = n$ es un valor propio, ya que los vectores columna formarían parte del hiperplano con ecuación implícita $x_1+\dots +x_n=0$ y, por tanto, el determinante sería nulo.

    \begin{table}[H]
        \centering
        \begin{tabular}{c|c|c}
            Valores Propios & Mult. Alg. & Mult. Geom. \\ \hline 
            0 & $n-1$ & $n-1$\\
            $n$ & 1 & 1\\
        \end{tabular}
        \caption{Valores propios con sus multiplicidades}
    \end{table}

    Por tanto, $A$ es diagonalizable.   
\end{ejercicio}

\begin{ejercicio}
    Sea $A=\mathcal{M}_3(\bb{R})$, con
    \begin{equation*}
        A = \left(\begin{array}{ccc}
            3 & -1 & 1 \\
            -1 & 3 & 1 \\
            1 & 1 & 3
        \end{array} \right)
    \end{equation*}
    Estudiar si $\exists B \in \mathcal{M}_n(\bb{R}) \mid B^2=A$. Es decir, estudiar si $\exists \sqrt{A}$.\\

    En primer lugar, vemos si $A$ es diagonalizable y, en su caso, se diagonaliza.
    \begin{equation*}\begin{split}
        P_A(\lambda) & = |A-\lambda I| = \left|\begin{array}{ccc}
            3-\lambda & -1 & 1 \\
            -1 & 3-\lambda & 1 \\
            1 & 1 & 3-\lambda
        \end{array} \right| = \left|\begin{array}{ccc}
            4-\lambda & -1 & 1 \\
            0 & 3-\lambda & 1 \\
            4-\lambda & 1 & 3-\lambda
        \end{array} \right| = \\
        &= \left|\begin{array}{ccc}
            4-\lambda & -1 & 1 \\
            0 & 3-\lambda & 1 \\
            0 & 2 & 2-\lambda
        \end{array} \right| = (4-\lambda)((3-\lambda)(2-\lambda)-2) = (4-\lambda)(\lambda^2-5\lambda+4) =\\
        &=(4-\lambda)(\lambda-4)(\lambda-1) = (4-\lambda)^2(1-\lambda)
    \end{split}\end{equation*}

    Por tanto, los valores propios son $\{1,4\}$. Calculamos los subespacios propios.
    \begin{equation*}\begin{split}
           V_{1} & = \left\{ \left(\begin{array}{c}
                x_1 \\
                x_2 \\
                x_3
           \end{array}\right) \in \bb{R}^3 \mid (A-I)\left(\begin{array}{c}
                x_1 \\
                x_2 \\
                x_3
           \end{array}\right) = 0 \right\} \\
           & = \left\{ \left(\begin{array}{c}
                x_1 \\
                x_2 \\
                x_3
           \end{array}\right) \in \bb{R}^3 \mid \left( \begin{array}{ccc}
            2 & -1 & 1 \\
            -1 & 2 & 1 \\
            1 & 1 & 2
        \end{array}\right) \left(\begin{array}{c}
                x_1 \\
                x_2 \\
                x_3
           \end{array}\right) = 0 \right\} \\
           & = \left\{ \left(\begin{array}{c}
                x_1 \\
                x_2  \\
                x_3
           \end{array}\right) \in \bb{R}^3 \left| \begin{array}{c}
                2x_1 -x_2+x_3=0 \\
                -x_1+2x_2+x_3 = 0 \\
                x_1+x_2+2x_3 = 0
           \end{array}\right. \right\} =
           \mathcal{L}\left(\left\{ \left(\begin{array}{c}
                    1 \\
                    1 \\
                    -1 \\
               \end{array}\right)
               \right\}\right)
   \end{split}\end{equation*}
   \begin{equation*}\begin{split}
           V_{4} & = \left\{ \left(\begin{array}{c}
                x_1 \\
                x_2 \\
                x_3
           \end{array}\right) \in \bb{R}^3 \mid (A-I)\left(\begin{array}{c}
                x_1 \\
                x_2 \\
                x_3
           \end{array}\right) = 0 \right\} \\
           & = \left\{ \left(\begin{array}{c}
                x_1 \\
                x_2 \\
                x_3
           \end{array}\right) \in \bb{R}^3 \mid \left( \begin{array}{ccc}
            -1 & -1 & 1 \\
            -1 & -1 & 1 \\
            1 & 1 & -1
        \end{array}\right) \left(\begin{array}{c}
                x_1 \\
                x_2 \\
                x_3
           \end{array}\right) = 0 \right\} \\
           & = \left\{ \left(\begin{array}{c}
                x_1 \\
                x_2  \\
                x_3
           \end{array}\right) \in \bb{R}^3 \left| \begin{array}{c}
                x_1+x_2-x_3 = 0
           \end{array}\right. \right\} =
           \mathcal{L}\left(\left\{
                \left(\begin{array}{c}
                    1 \\
                    0 \\
                    1 \\
               \end{array}\right),
               \left(\begin{array}{c}
                    0 \\
                    1 \\
                    1 \\
               \end{array}\right)
               \right\}\right)
   \end{split}\end{equation*}
   \begin{table}[H]
        \centering
        \begin{tabular}{c|c|c}
            Valores Propios & Mult. Alg. & Mult. Geom. \\ \hline 
            1 & 1 & 1\\
            4 & 2 & 2\\
        \end{tabular}
        \caption{Valores propios con sus multiplicidades}
    \end{table}
    Por tanto, $A$ es diagonalizable de la forma $D=P^{-1}AP$, con
    \begin{equation*}
        D=\left( \begin{array}{ccc}
            1 & 0 & 0 \\
            0 & 4 & 0 \\
            0 & 0 & 4 
        \end{array}\right) \qquad
        P=\left( \begin{array}{ccc}
            1 & 1 & 0 \\
            1 & 0 & 1 \\
            -1 & 1 & 1 
        \end{array}\right)
    \end{equation*}

    Definimos $\sqrt{D}$ como:
    \begin{equation*}
        \sqrt{D}=\left( \begin{array}{ccc}
            1 & 0 & 0 \\
            0 & 2 & 0 \\
            0 & 0 & 2 
        \end{array}\right)
    \end{equation*}

    Para calcular $B$, despejamos $A$. $A=PDP^{-1}$
    \begin{equation*}
         P\sqrt{D}P^{-1} \cdot P\sqrt{D}P^{-1} = PDP^{-1} = A \Longrightarrow B = \sqrt{A} = P\sqrt{D}P^{-1}
    \end{equation*}

    Por tanto, $\exists B=\sqrt{A} \in \mathcal{M}_3(\bb{R}) \mid B^2 = A$, y se define como
    $$B = \sqrt{A} = P\sqrt{D}P^{-1}$$
\end{ejercicio}

\begin{ejercicio}\textbf{Prueba 2022}
\begin{enumerate}
    \item Estudiar $a\in\bb{R}$ para los que $A\in\mathcal{M}_3(\bb{R})$ se diagonaliza.
    \begin{equation*}
        A = \left( \begin{array}{ccc}
            1 & -a & 1 \\
            -a & 1 & -1 \\
            0 & a & 0
        \end{array} \right)
    \end{equation*}

    Calculo en primer lugar su polinomio característico.
    \begin{multline*}
        P_A(\lambda) = |A-\lambda I| = \left| \begin{array}{ccc}
            1-\lambda & -a & 1 \\
            -a & 1-\lambda & -1 \\
            0 & a & -\lambda
        \end{array} \right|
        = \left| \begin{array}{ccc}
            1-\lambda & -a & 1 \\
            -a & 1-\lambda & -1 \\
            1-\lambda & 0 & 1-\lambda
        \end{array} \right| = \\
        = \left| \begin{array}{ccc}
            1-\lambda & -a & \lambda \\
            -a & 1-\lambda & -1+a \\
            1-\lambda & 0 & 0
        \end{array} \right| 
        = (1-\lambda)\left| \begin{array}{cc}
            -a & \lambda \\
            1-\lambda & -1+a \\
        \end{array} \right| = \\
        = (1-\lambda)\left| \begin{array}{cc}
            -a+\lambda & \lambda \\
            a-\lambda & -1+a \\
        \end{array} \right|
        = (1-\lambda)\left| \begin{array}{cc}
            -a+\lambda & \lambda \\
            0 & \lambda-1+a \\
        \end{array} \right| = (1-\lambda)(\lambda-a)(\lambda-1+a)
    \end{multline*}
    Por tanto, los valores propios son: $\{1,a,1-a\}$. Hay que tener en cuenta el caso en que dos valores propios sean iguales.
    \begin{equation*} \begin{array}{ll}
       a=1 & \longrightarrow a=1 \\
       1-a=1 & \longrightarrow a=0 \\
       a=1-a & \longrightarrow a=\frac{1}{2} \\
   \end{array}\end{equation*}
    Para $a\neq \{0,\frac{1}{2},1\}$, los tres valores propios son distintos, por lo que sí es diagonalizable.
    \begin{itemize}
        \item[-] \underline{Para $a=0$}:
        \begin{equation*}\begin{split}
               V_{1} & = \left\{ \left(\begin{array}{c}
                    x_1 \\
                    x_2 \\
                    x_3
               \end{array}\right) \in \bb{R}^3 \mid (A-I)\left(\begin{array}{c}
                    x_1 \\
                    x_2 \\
                    x_3
               \end{array}\right) = 0 \right\} \\
               & = \left\{ \left(\begin{array}{c}
                    x_1 \\
                    x_2 \\
                    x_3
               \end{array}\right) \in \bb{R}^3 \mid \left( \begin{array}{ccc}
                0 & 0 & 1 \\
                0 & 0 & -1 \\
                0 & 0 & -1
            \end{array}\right) \left(\begin{array}{c}
                    x_1 \\
                    x_2 \\
                    x_3
               \end{array}\right) = 0 \right\} \\
               & = \left\{ \left(\begin{array}{c}
                    x_1 \\
                    x_2  \\
                    x_3
               \end{array}\right) \in \bb{R}^3 \left| \begin{array}{c}
                    x_3=0 \\
               \end{array}\right. \right\}
       \end{split}\end{equation*}
        \begin{table}[H]
            \centering
            \begin{tabular}{c|c|c}
                Valores Propios & Mult. Alg. & Mult. Geom. \\ \hline 
                1 & 2 & 2\\
                0 & 1 & 1\\
            \end{tabular}
            \caption{Valores propios con sus multiplicidades}
        \end{table}
        Por tanto, para $a=0$, $A$ sí es diagonalizable.

        \item[-] \underline{Para $a=\frac{1}{2}$}:
        \begin{equation*}\begin{split}
               V_{\frac{1}{2}} & = \left\{ \left(\begin{array}{c}
                    x_1 \\
                    x_2 \\
                    x_3
               \end{array}\right) \in \bb{R}^3 \mid \left(A-\frac{1}{2}I\right)\left(\begin{array}{c}
                    x_1 \\
                    x_2 \\
                    x_3
               \end{array}\right) = 0 \right\} \\
               & = \left\{ \left(\begin{array}{c}
                    x_1 \\
                    x_2 \\
                    x_3
               \end{array}\right) \in \bb{R}^3 \mid \left( \begin{array}{ccc}
                \frac{1}{2} & -\frac{1}{2} & 1 \\
                -\frac{1}{2} & \frac{1}{2} & -1 \\
                0 & \frac{1}{2} & -\frac{1}{2}
            \end{array}\right) \left(\begin{array}{c}
                    x_1 \\
                    x_2 \\
                    x_3
               \end{array}\right) = 0 \right\} \\
               & = \left\{ \left(\begin{array}{c}
                    x_1 \\
                    x_2  \\
                    x_3
               \end{array}\right) \in \bb{R}^3 \left| \begin{array}{c}
                    \frac{1}{2}x_1-\frac{1}{2}x_2+x_3=0 \\
                    -\frac{1}{2}x_1+\frac{1}{2}x_2-x_3=0\\
                    \frac{1}{2}x_2 - \frac{1}{2}x_3 = 0
               \end{array}\right. \right\} \\
               & = \left\{ \left(\begin{array}{c}
                    x_1 \\
                    x_2  \\
                    x_3
               \end{array}\right) \in \bb{R}^3 \left| \begin{array}{c}
                    \frac{1}{2}x_1-\frac{1}{2}x_2+x_3=0 \\
                    \frac{1}{2}x_2 - \frac{1}{2}x_3 = 0
               \end{array}\right. \right\}
       \end{split}\end{equation*}
        \begin{table}[H]
            \centering
            \begin{tabular}{c|c|c}
                Valores Propios & Mult. Alg. & Mult. Geom. \\ \hline 
                1 & 1 & 1\\
                $\frac{1}{2}$ & 2 & 1\\
            \end{tabular}
            \caption{Valores propios con sus multiplicidades}
        \end{table}
        Por tanto, para $a=\frac{1}{2}$, $A$ no es diagonalizable.

        \item[-] \underline{Para $a=1$}:
        \begin{equation*}\begin{split}
               V_{1} & = \left\{ \left(\begin{array}{c}
                    x_1 \\
                    x_2 \\
                    x_3
               \end{array}\right) \in \bb{R}^3 \mid (A-I)\left(\begin{array}{c}
                    x_1 \\
                    x_2 \\
                    x_3
               \end{array}\right) = 0 \right\} \\
               & = \left\{ \left(\begin{array}{c}
                    x_1 \\
                    x_2 \\
                    x_3
               \end{array}\right) \in \bb{R}^3 \mid \left( \begin{array}{ccc}
                0 & -1 & 1 \\
                -1 & 0 & -1 \\
                0 & 1 & -1
            \end{array}\right) \left(\begin{array}{c}
                    x_1 \\
                    x_2 \\
                    x_3
               \end{array}\right) = 0 \right\} \\
               & = \left\{ \left(\begin{array}{c}
                    x_1 \\
                    x_2  \\
                    x_3
               \end{array}\right) \in \bb{R}^3 \left| \begin{array}{c}
                    x_2-x_3=0 \\
                    x_1+x_3=0
               \end{array}\right. \right\}
       \end{split}\end{equation*}
        \begin{table}[H]
            \centering
            \begin{tabular}{c|c|c}
                Valores Propios & Mult. Alg. & Mult. Geom. \\ \hline 
                1 & 2 & 1\\
                0 & 1 & 1\\
            \end{tabular}
            \caption{Valores propios con sus multiplicidades}
        \end{table}
        Por tanto, para $a=1$, $A$ no es diagonalizable.
    \end{itemize}

    \begin{itemize}
        \item \textbf{Ver si, para $a=0$ y $a=-1$, las matrices resultantes son semejantes.}
        
        Para $a=0$, los valores propios son $\{0,1,1\}$.\\
        Para $a=-1$, los valores propios son $\{1,-1,2\}$.

        Por tanto, como tienen valores distintos, tienen polinomio característico distinto y, por tanto, no son semejantes.
    \end{itemize}

    \item Sea $A\in\mathcal{M}_2(\bb{R}) \mid |A|=-1$. Demostrar que $A$ es diagonalizable.

    Sea $A=\left( \begin{array}{cc}
        a & b \\
        c & d
    \end{array} \right)$, con $|A|=ad-bc = -1$.
    $$P_A(\lambda) = \lambda^2 - tr(A)\lambda + det(A) = \lambda^2 -(a+d)\lambda-1$$
    $$\Delta = (a+d)^2 +4 \geq 4 > 0 \qquad \forall a,d \in \bb{R}$$

    Por tanto, como el discriminante es positivo, tendrá dos valores propios distintos. Por tanto, $A$ sí es diagonalizable.

    \begin{itemize}
        \item \textbf{Encontrar $A\in\mathcal{M}_2(\bb{C}) \mid |A|=-1$ t.q. $A$ no sea diagonalizable.}\\
        En primer lugar, es necesario que solo tenga un valor propio, por lo que $$\Delta=0 \Longrightarrow (a+d)^2 = -4 \Longrightarrow a+d = 2i$$

        Además, como $|A|=-1 \Longrightarrow ad-bc=-1$. Por tanto, una posible $A\in\mathcal{M}_2(\bb{C})$ es: $$A=\left( \begin{array}{cc}
            i & 1 \\
            0 & i
        \end{array} \right)$$
        $$P_A(\lambda) = \lambda^2-2i\lambda-1 = (\lambda-i)^2$$

        Por tanto, el único valor propio es $\{i\}$.
        \begin{equation*}\begin{split}
               V_{i} & = \left\{ \left(\begin{array}{c}
                    x_1 \\
                    x_2 \\
               \end{array}\right) \in \bb{C}^2 \mid (A-iI)\left(\begin{array}{c}
                    x_1 \\
                    x_2 \\
               \end{array}\right) = 0 \right\} \\
               & = \left\{ \left(\begin{array}{c}
                    x_1 \\
                    x_2 \\
               \end{array}\right) \in \bb{C}^2 \mid \left( \begin{array}{cc}
                0 & 1 \\
                0 & 0
            \end{array}\right) \left(\begin{array}{c}
                    x_1 \\
                    x_2 \\
               \end{array}\right) = 0 \right\} \\
               & = \left\{ \left(\begin{array}{c}
                    x_1 \\
                    x_2  \\
               \end{array}\right) \in \bb{C}^2 \left| \begin{array}{c}
                    x_2=0 \\
               \end{array}\right. \right\}
       \end{split}\end{equation*}
       \begin{table}[H]
            \centering
            \begin{tabular}{c|c|c}
                Valores Propios & Mult. Alg. & Mult. Geom. \\ \hline 
                $i$ & 2 & 1\\
            \end{tabular}
            \caption{Valores propios con sus multiplicidades}
        \end{table}

        Por tanto, $A$ no es diagonalizable.
    \end{itemize}
\end{enumerate}
    
\end{ejercicio}

\begin{ejercicio}
    Sea $A \in \mathcal{M}_n(\bb{K})$ t.q. $A^2=I_n$. Demostrar que $A$ es diagonalizable.
    \begin{proof}
        Haciendo uso del isomorfismo natural entre $\mathcal{M}_n(\bb{K})$ y $End(V)$ con $\dim_\bb{K}(V)=n$, consideramos $A$ como la matriz a asociada a $f\in End(V^n)$ el cual cumple que $f\circ f = Id_\bb{K}$.
        $$f\circ f = Id_\bb{K} \hspace{2cm} f(f(x))=x\quad \forall x\in V$$

        Notemos ahora que, dado un $v\in V$ cualquiera, este se puede expresar como:
        $$v\in V\quad v = \underbrace{\frac{1}{2}(v+f(v))}_{t_1} + \underbrace{\frac{1}{2}(v-f(v))}_{t_2}$$
        
        Sea $t_1 = \frac{1}{2}(v+f(v))$ y $t_2$ = $\frac{1}{2}(v-f(v))$. Nótese también que, como $f\circ f = Id_\bb{K}$,
        \begin{equation}\label{RazonamientoErroneo}\tag{$\ast$}
        \begin{split}
            f(t_1) = \frac{1}{2}(v+f(v)) = t_1& \xrightarrow{(\ast)} t_1 \text{ es un vector propio.}\quad t_1\in V_1 \\
            f(t_2) = -\frac{1}{2}(v-f(v)) = -t_2& \xrightarrow{(\ast)} t_2 \text{ es un vector propio.}\quad t_2\in V_{-1}
        \end{split}\end{equation}


        Como todo $v \in V$ se expresa como la suma de un vector de $V_1$ y otro de $V_{-1} \Longrightarrow V=V_1 + V_{-1}$. Además, como los subespacios propios de valores propios distintos son disjuntos, $V_1 \cap V_{-1} = \{0\}$. Por tanto, $V_1 \oplus V_{-1} = V$.

        Sean $\mathcal{B}_1$ y $\mathcal{B}_{-1}$ bases de $V_1$ y $V_{-1}$ respectivamente. Como $V=V_1\oplus V_{-1} \Longrightarrow \mathcal{B}_{1} \cup \mathcal{B}_{-1}$ forman una base $\mathcal{B}$ de $V$. Así, $V$ tendrá una base de vectores propios y, por tanto, $f$ es diagonalizable, por lo que $A$ también lo es.

        (\ref{RazonamientoErroneo}) Este razonamiento es válido $\forall f \neq \{Id, -Id\}$, ya que en su caso $t_1$ o $t_2$ serían nulos. En el caso en el que sea $f=Id \lor f=-Id$, su matriz asociada $A$ es diagonal y por tanto será diagonalizable.
        
    \end{proof}
\end{ejercicio}

\begin{observacion}
    La aplicación transposición,
    $$\begin{array}{cccc}
        ^t: & \mathcal{M}_n(\bb{K}) & \to & \mathcal{M}_n(\bb{K}) \\
         & A & \to & A^t
    \end{array}$$
    Tiene como matriz asociada $M(^t, \mathcal{B}) \sim D = \left( \begin{array}{ccccc}
        1 & & & &\\
        & 1 & & &\\
        & & \ddots & &\\
        & & & -1 & \\
        & & & & -1 \\  
    \end{array} \right)$
\end{observacion}

\begin{ejercicio}
    Dadas $A_1,A_2,B_1,B_2 \in \mathcal{M}_n(\bb{K})$ tal que $A_1 \sim A_2$ y $B_1\sim B_2$. Demostrar que:
    \begin{equation*}
        \left.\begin{array}{c}
             A_1 \sim A_2  \\
             B_1\sim B_2 
        \end{array}\right\} \nRightarrow A_1+B_1 \sim A_2 + B_2
    \end{equation*}

    Se demuestra buscando un contraejemplo. Sean:
    \begin{equation*}
        A_1 = \left( \begin{array}{cc}
            0 & 0 \\
            1 & 0
        \end{array} \right) \qquad
        A_2 = \left( \begin{array}{cc}
            0 & 0 \\
            1 & 0
        \end{array} \right) \qquad
        B_1 = \left( \begin{array}{cc}
            0 & 0 \\
            1 & 0
        \end{array} \right) \qquad
        B_2 = \left( \begin{array}{cc}
            0 & 1 \\
            0 & 0
        \end{array} \right)
    \end{equation*}

    Es fácil ver que $A_1 \sim A_2$. Veamos ahora que $B_1 \sim B_2$.

    Sea $f\in End(\bb{R}^2)$ y sean $\mathcal{B}=\{e_1,e_2\}$ y $ \bar{\mathcal{B}} = \{\bar{e_1}, \bar{e_2}\}$ bases de $\bb{R}^2$ tales que $B_1 = M(f;\mathcal{B})$ y $B_2 = M(f;\bar{\mathcal{B}})$.
    \begin{equation*}
        f:\left| \begin{array}{c}
            e_1 \longmapsto e_2 \\
            e_2 \longmapsto 0 
        \end{array} \right. \hspace{2cm}
        f:\left| \begin{array}{cc}
            \bar{e_1} \longmapsto 0 \\
            \bar{e_2} \longmapsto \bar{e_1} 
        \end{array} \right.
    \end{equation*}
    Son semejantes, ya que si $\bar{e_1} = e_2$ y $\bar{e_2} = e_1$, las dos matrices representan el mismo endomorfismo respecto de distintas bases.

    Por tanto, sabiendo que se dan las hipótesis,
    \begin{equation*}
        A_1 + B_1 = \left( \begin{array}{cc}
            0 & 0 \\
            2 & 0
        \end{array} \right) \qquad
        A_2+B_2 = \left( \begin{array}{cc}
            0 & 1 \\
            1 & 0
        \end{array} \right)
    \end{equation*}
    Es fácil ver que $A_1+B_1 \nsim A_2 + B_2$, ya que tienen rango distinto.\footnote{También se puede ver que tienen determinante distinto, o que la segunda es diagonalizable (por ser simétrica) y la primera no.}
\end{ejercicio}

\begin{ejercicio}
    Sea $A,B \in \mathcal{M}_3(\bb{R})$. Demostrar que no son semejantes $(A\nsim B)$.

    \begin{equation*}
        A = \left( \begin{array}{ccc}
            1 & 0 & 0 \\
            0 & 1 & 0 \\
            0 & 1 & 1
        \end{array}\right) \qquad 
        B = \left( \begin{array}{ccc}
            1 & 0 & 0 \\
            1 & 1 & 0 \\
            0 & 1 & 1
        \end{array}\right) \qquad
    \end{equation*}

    En ambos casos, tienen el mismo polinomio característico
    $$P_A(\lambda) = P_B(\lambda) = (1-\lambda)^3$$
    Por tanto, ambas matrices tienen como valor propio $\{1\}$. Veamos ahora la dimensión del subespacio propio $V_1$ en cada caso.
    \begin{equation*}
        rg(A-I) = rg\left( \begin{array}{ccc}
            0 & 0 & 0 \\
            0 & 0 & 0 \\
            0 & 1 & 0
        \end{array}\right) = 1 \qquad
        rg(B-I) = rg\left( \begin{array}{ccc}
            0 & 0 & 0 \\
            1 & 0 & 0 \\
            0 & 1 & 0
        \end{array}\right) = 2
    \end{equation*}

    Por tanto,
    \begin{table}[H]
        \centering
        \begin{tabular}{c|c|c|c}
            Valores Propios & Matriz & Mult. Alg. & Mult. Geom. \\ \hline 
            $A$ & $1$ & 3 & $3-1=2$\\
            $B$ & $1$ & 3 & $3-2=1$\\
        \end{tabular}
        \caption{Valores propios con sus multiplicidades para cada matriz}
    \end{table}

    Como tienen multiplicidades geométricas distintas para el mismo valor propio, entonces no representan el mismo endomorfismo. Por tanto, no son semejantes.
\end{ejercicio}


\begin{ejercicio}
    Sea $A\in \mathcal{M}_4(\bb{R})$. Estudiar los valores de $a\in\bb{R}$ para los que la matriz $A$ es diagonalizable.
    \begin{equation*}
        A = \left( \begin{array}{cccc}
            1 & 0 & 1 & 0 \\
            0 & a & 0 & a \\
            1 & 0 & 1 & 0 \\
            0 & a & 0 & a \\
        \end{array}\right)
    \end{equation*}

    \begin{equation*}\begin{split}
        P_A(\lambda) & = \left|\begin{array}{cccc}
            1-\lambda & 0 & 1 & 0 \\
            0 & a-\lambda & 0 & a \\
            1 & 0 & 1-\lambda & 0 \\
            0 & a & 0 & a-\lambda \\
        \end{array} \right| \stackrel{F'_1 = F_1 - F_3}{=} \left|\begin{array}{cccc}
            -\lambda & 0 & \lambda & 0 \\
            0 & a-\lambda & 0 & a \\
            1 & 0 & 1-\lambda & 0 \\
            0 & a & 0 & a-\lambda \\
        \end{array} \right|
        \stackrel{F'_2 = F_2 - F_4} {=} \\
        & = \left|\begin{array}{cccc}
            -\lambda & 0 & \lambda & 0 \\
            0 & -\lambda & 0 & \lambda \\
            1 & 0 & 1-\lambda & 0 \\
            0 & a & 0 & a-\lambda \\
        \end{array} \right| = \lambda^2 \left|\begin{array}{cccc}
            -1 & 0 & 1 & 0 \\
            0 & -1 & 0 & 1 \\
            1 & 0 & 1-\lambda & 0 \\
            0 & a & 0 & a-\lambda \\
        \end{array} \right| 
        \stackrel{C'_3 = C_3 + C_1} {=} \\
        & = \lambda^2 \left|\begin{array}{cccc}
            -1 & 0 & 0 & 0 \\
            0 & -1 & 0 & 1 \\
            1 & 0 & 2-\lambda & 0 \\
            0 & a & 0 & a-\lambda \\
        \end{array} \right|
        \stackrel{C'_4 = C_4 + C_2} {=} \lambda^2 \left|\begin{array}{cccc}
            -1 & 0 & 0 & 0 \\
            0 & -1 & 0 & 0 \\
            1 & 0 & 2-\lambda & 0 \\
            0 & a & 0 & 2a-\lambda \\
        \end{array} \right| = \\
        & = \lambda^2 \left|\begin{array}{cc}
            -1 & 0 \\
            0 & -1
        \end{array} \right|
        \left|\begin{array}{cc}
            2-\lambda & 0 \\
            0 & 2a-\lambda
        \end{array} \right| = \lambda^2(2-\lambda)(2a-\lambda)
    \end{split}\end{equation*}

    Por tanto, los valores propios son $\{0, 2, 2a\}$.
    
    \begin{itemize}
        \item \underline{Si $a\neq \{0,1\}$}:\\
        \begin{equation*}
            rg(A) = rg(A-0\cdot I) = 2 \Longrightarrow \dim V_0 = 4- 2 = 2
        \end{equation*}
        \begin{table}[H]
            \centering
            \begin{tabular}{c|c|c}
                Valores Propios & Mult. Alg. & Mult. Geom. \\ \hline 
                0 & 2 & 2\\
                2 & 1 & 1\\
                $2a$ & 1 & 1\\
            \end{tabular}
            \caption{Valores propios con sus multiplicidades para $a\neq \{0,1\}$}
        \end{table}
        Por tanto, para $a\neq \{0,1\}, A$ sí es diagonalizable.

        \item \underline{Si $a=0$}:\\
        \begin{equation*}
            rg(A) = rg(A-0\cdot I) = rg\left( \begin{array}{cccc}
            1 & 0 & 1 & 0 \\
            0 & 0 & 0 & 0 \\
            1 & 0 & 1 & 0 \\
            0 & 0 & 0 & 0 \\
        \end{array}\right) = 1 \Longrightarrow \dim V_0 = 4- 1 = 3
        \end{equation*}
        \begin{table}[H]
            \centering
            \begin{tabular}{c|c|c}
                Valores Propios & Mult. Alg. & Mult. Geom. \\ \hline 
                0 & 3 & 3\\
                2 & 1 & 1\\
            \end{tabular}
            \caption{Valores propios con sus multiplicidades para $a=0$}
        \end{table}
        Por tanto, para $a=0, A$ sí es diagonalizable.

        \item \underline{Si $a=1$}:\\
        \begin{equation*}
            rg(A-2I) = rg\left( \begin{array}{cccc}
            -1 & 0 & 1 & 0 \\
            0 & -1 & 0 & 1 \\
            1 & 0 & -1 & 0 \\
            0 & 1 & 0 & -1 \\
        \end{array}\right) = 2 \Longrightarrow \dim V_2 = 4- 2 = 2
        \end{equation*}

        \begin{equation*}
            rg(A) = rg(A-0\cdot I) = rg\left( \begin{array}{cccc}
            1 & 0 & 1 & 0 \\
            0 & 1 & 0 & 1 \\
            1 & 0 & 1 & 0 \\
            0 & 1 & 0 & 1 \\
        \end{array}\right) = 2 \Longrightarrow \dim V_0 = 4- 2 = 2
        \end{equation*}
        
        \begin{table}[H]
            \centering
            \begin{tabular}{c|c|c}
                Valores Propios & Mult. Alg. & Mult. Geom. \\ \hline 
                0 & 2 & 2\\
                2 & 2 & 2\\
            \end{tabular}
            \caption{Valores propios con sus multiplicidades para $a=1$}
        \end{table}
        Por tanto, para $a=1, A$ sí es diagonalizable.
    \end{itemize}
    
\end{ejercicio}


\begin{ejercicio}
    Determinar los valores y vectores propios de las siguientes matrices reales. ¿Son diagonizables?

    \begin{enumerate}
        \item $A = \left( \begin{array}{ccc}
            -1 & 1 & 1 \\
            1  & -1 & 1 \\
            1 & 1 & -1
        \end{array}\right)$

        \begin{multline*}
            P_A(\lambda) = \left| \begin{array}{ccc}
            -1-\lambda & 1 & 1 \\
            1  & -1-\lambda & 1 \\
            1 & 1 & -1-\lambda
        \end{array}\right| \stackrel{C'_1 = C_1+C_3}{=}\left| \begin{array}{ccc}
            -\lambda & 1 & 1 \\
            2  & -1-\lambda & 1 \\
            -\lambda & 1 & -1-\lambda
        \end{array}\right| \stackrel{F'_1 = F_1-F_3}{=} \\
        = \left| \begin{array}{ccc}
            0 & 0 & 2+\lambda \\
            2  & -1-\lambda & 1 \\
            -\lambda & 1 & -1-\lambda
        \end{array}\right| = (2+\lambda)\left| \begin{array}{cc}
            2  & -1-\lambda \\
            -\lambda & 1
        \end{array}\right| = (2+\lambda)(2-\lambda(1+\lambda))= \\  = (2+\lambda)(-\lambda^2 - \lambda +2) = (2+\lambda)^2(1-\lambda)
        \end{multline*}

        Por tanto, $A$ tiene dos valores propios: $\{-2,1\}$. Veamos sus multiplicidades geométricas:
        \begin{equation*}
            rg(A+2I) = rg\left( \begin{array}{ccc}
                1 & 1 & 1 \\
                1  & 1 & 1 \\
                1 & 1 & 1
            \end{array}\right) = 1 \Longrightarrow \dim V_{-2} = 3-1 = 2
        \end{equation*}

        \begin{table}[H]
            \centering
            \begin{tabular}{c|c|c}
                Valores Propios & Mult. Alg. & Mult. Geom. \\ \hline 
               $ -2$ & 2 & 2\\
                1 & 1 & 1\\
            \end{tabular}
            \caption{Valores propios con sus multiplicidades}
        \end{table}
        Por tanto, $A$ sí es diagonalizable.
        Los vectores propios son:
        \begin{equation*}\begin{split}
           V_{1} & = \left\{ \left(\begin{array}{c}
                    x_1 \\
                    x_2 \\
                    x_3
               \end{array}\right) \in \bb{R}^3 \mid (A-I)\left(\begin{array}{c}
                    x_1 \\
                    x_2 \\
                    x_3
               \end{array}\right) = 0 \right\} \\
               & = \left\{ \left(\begin{array}{c}
                    x_1 \\
                    x_2 \\
                    x_3
               \end{array}\right) \in \bb{R}^3 \mid \left( \begin{array}{ccc}
                    -2 & 1 & 1 \\
                    1  & -2 & 1 \\
                    1 & 1 & -2
            \end{array}\right) \left(\begin{array}{c}
                    x_1 \\
                    x_2 \\
                    x_3
               \end{array}\right) = 0 \right\} \\
               & = \left\{ \left(\begin{array}{c}
                    x_1 \\
                    x_2  \\
                    x_3
               \end{array}\right) \in \bb{R}^3 \left| \begin{array}{c}
                    -2x_1 + x_2 +x_3= 0 \\
                    x_1 -2x_2 + x_3 = 0 \\
                    x_1 +x_2 -2x_3 = 0
               \end{array}\right. \right\} \\
               & = \left\{ \left(\begin{array}{c}
                    x_1 \\
                    x_2  \\
                    x_3
               \end{array}\right) \in \bb{R}^3 \left| \begin{array}{c}
                    -2x_1 + x_2 +x_3= 0 \\
                    x_1 -2x_2 + x_3 = 0 \\
               \end{array}\right. \right\} = \mathcal{L}\left(\left\{
                    \left(\begin{array}{c}
                        1 \\
                        1 \\
                        1 \\
                   \end{array}\right)
                   \right\}\right)
       \end{split}\end{equation*}

       \begin{equation*}\begin{split}
           V_{-2} & = \left\{ \left(\begin{array}{c}
                    x_1 \\
                    x_2 \\
                    x_3
               \end{array}\right) \in \bb{R}^3 \mid (A+2I)\left(\begin{array}{c}
                    x_1 \\
                    x_2 \\
                    x_3
               \end{array}\right) = 0 \right\} \\
               & = \left\{ \left(\begin{array}{c}
                    x_1 \\
                    x_2 \\
                    x_3
               \end{array}\right) \in \bb{R}^3 \mid \left( \begin{array}{ccc}
                    1 & 1 & 1 \\
                    1  & 1 & 1 \\
                    1 & 1 & 1
            \end{array}\right) \left(\begin{array}{c}
                    x_1 \\
                    x_2 \\
                    x_3
               \end{array}\right) = 0 \right\} \\
               & = \left\{ \left(\begin{array}{c}
                    x_1 \\
                    x_2  \\
                    x_3
               \end{array}\right) \in \bb{R}^3 \left| \begin{array}{c}
                    x_1 + x_2 + x_3 = 0
               \end{array}\right. \right\} = \mathcal{L}\left(\left\{
                    \left(\begin{array}{c}
                        0 \\
                        1 \\
                        -1 \\
                   \end{array}\right),
                   \left(\begin{array}{c}
                        1 \\
                        -1 \\
                        0 \\
                   \end{array}\right)
                   \right\}\right)
       \end{split}\end{equation*}
       

        \item $B = \left( \begin{array}{ccc}
            -1 & 1 & 0 \\
            0  & -1 & 1 \\
            1 & 0 & -1
        \end{array}\right)$

        \begin{multline*}
            P_A(\lambda) = \left| \begin{array}{ccc}
                -1-\lambda & 1 & 0 \\
                0  & -1-\lambda & 1 \\
                1 & 0 & -1-\lambda
            \end{array}\right| = -(1+\lambda)^3+1 =  \\ = -\lambda^3 -3\lambda^2 - 3\lambda -1 + 1 = -\lambda^3 -3\lambda^2 - 3\lambda = -\lambda(\lambda^2+3\lambda+3)
        \end{multline*}
        $$\Delta = 9 - 12 <0 \Longrightarrow \nexists \;\;sol \in \bb{R}$$

        Por tanto, el único valor propio es $\{0\}$ con multiplicidad simple. Por tanto, $B$ no es diagonalizable.

        Los vectores propios son:
        \begin{equation*}\begin{split}
           V_{0} & = \left\{ \left(\begin{array}{c}
                    x_1 \\
                    x_2 \\
                    x_3
               \end{array}\right) \in \bb{R}^3 \mid B\left(\begin{array}{c}
                    x_1 \\
                    x_2 \\
                    x_3
               \end{array}\right) = 0 \right\} \\
               & = \left\{ \left(\begin{array}{c}
                    x_1 \\
                    x_2 \\
                    x_3
               \end{array}\right) \in \bb{R}^3 \mid \left( \begin{array}{ccc}
                -1 & 1 & 0 \\
                0  & -1 & 1 \\
                1 & 0 & -1
            \end{array}\right) \left(\begin{array}{c}
                    x_1 \\
                    x_2 \\
                    x_3
               \end{array}\right) = 0 \right\} \\
               & = \left\{ \left(\begin{array}{c}
                    x_1 \\
                    x_2  \\
                    x_3
               \end{array}\right) \in \bb{R}^3 \left| \begin{array}{c}
                    -x_1 + x_2 = 0 \\
                    -x_2 + x_3 = 0 \\
                    x_1 - x_3 = 0
               \end{array}\right. \right\}
               = \mathcal{L}\left(\left\{
                    \left(\begin{array}{c}
                        1 \\
                        1 \\
                        1 \\
                   \end{array}\right)
                   \right\}\right)
       \end{split}\end{equation*}
    \end{enumerate}
\end{ejercicio}

\begin{ejercicio}
    Encontrar, si es posible, pares de endomorfismos de $\bb{R}^3(\bb{R})$, uno diagonalizable y otro no, que tengan los siguientes polinomios característicos:

    \begin{enumerate}
        \item $P_f(\lambda) = (1-\lambda)^3$\\
        El endomorfismo diagonalizable tiene por matriz asociada:
        $$M(f, \mathcal{B}) = \left( \begin{array}{ccc}
            1 & 0 & 0 \\
            0 & 1 & 0 \\
            0 & 0 & 1 \\
        \end{array}\right)$$
        En este caso, es diagonalizable, ya que $\dim V_1 = 3-0 = 3$.

        El endomorfismo no diagonalizable tiene por matriz asociada:
        $$M(f', \mathcal{B}) = \left( \begin{array}{ccc}
            1 & 0 & 1 \\
            0 & 1 & 0 \\
            0 & 0 & 1 \\
        \end{array}\right)$$
        En este caso, no es diagonalizable ya que $\dim V_1 = 3-1 = 2 \neq 3$.

        \item $P_f(\lambda) = -(1-\lambda)^2(1+\lambda)$\\
        El endomorfismo diagonalizable tiene por matriz asociada:
        $$M(f, \mathcal{B}) = \left( \begin{array}{ccc}
            1 & 0 & 0 \\
            0 & 1 & 0 \\
            0 & 0 & -1 \\
        \end{array}\right)$$
        En este caso, es diagonalizable, ya que $\dim V_1 = 3-1 = 2$.

        El endomorfismo no diagonalizable tiene por matriz asociada:
        $$M(f', \mathcal{B}) = \left( \begin{array}{ccc}
            1 & 1 & 0 \\
            1 & 1 & 0 \\
            0 & 0 & -1 \\
        \end{array}\right)$$
        En este caso, $f'$ no es diagonalizable ya que $\dim V_1 = 3-2 = 1 \neq 2$.

        \item $P_f(\lambda) = (1-\lambda)(\lambda^2+1)$\\
        En este caso, solo hay un valor propio real, el $\{1\}$, ya que el término $\lambda^2+1$ no tiene raíces reales. Por tanto, como además tiene multiplicidad simple, el endomorfismo solo tendrá un valor propio contado con multiplicidad, por lo que no podrá ser diagonalizable. Por tanto, no es posible encontrar un endomorfismo diagonalizable con ese polinomio característico.

        El endomorfismo no diagonalizable tiene por matriz asociada:
        $$M(f', \mathcal{B}) = \left( \begin{array}{ccc}
            1 & 0 & 0 \\
            0 & 0 & 1 \\
            0 & -1 & 0 \\
        \end{array}\right)$$
    \end{enumerate}
\end{ejercicio}

\begin{ejercicio}
    Probar que todo $f\in End_\bb{R}(\bb{R}^2) \mid det(f)<0$ es diagonalizable.\\

    Su polinomio característico será:
    $$P_f(\lambda) = \lambda^2 - tr(f)\lambda + det(f)$$
    $$\Delta = tr^2(f) -4det(f) > 0 \Longleftrightarrow tr^2(f) > 4det(f) \Longleftrightarrow \frac{tr^2(f)}{4} > det(f)$$
    lo cual es cierto ya que $\frac{tr^2(f)}{4} \geq 0 > det(f)$. Por tanto, el polinomio característico tiene dos raíces distintas, por lo que hay dos valores propios distintos. Como la multiplicidad geométrica es menor o igual que la algebraica, el endomorfismo es diagonalizable.    
\end{ejercicio}

\begin{ejercicio}
    Estudiar si las siguientes matrices son semejantes entre si.
    \begin{equation*}
        A_1 = \left( \begin{array}{ccc}
            1 & 2 & 1 \\
            2 & 0 & 2 \\
            1 & 2 & 1 \\
        \end{array}\right) \qquad
        A_2 = \left( \begin{array}{ccc}
            1 & 0 & 1 \\
            1 & 2 & 3 \\
            1 & 1 & -1 \\
        \end{array}\right) \qquad
        A_3 = \left( \begin{array}{ccc}
            -1 & 2 & -1 \\
            2 & 0 & 2 \\
            3 & 2 & 3 \\
        \end{array}\right)
    \end{equation*}
    \begin{table}[H]
        \centering
        \begin{tabular}{r|ccc|l}
             & $A_1$ & $A_2$ & $A_3$ & \\ \hline
             $tr(A)$ & $2$ & $2$ & $2$ &\\
             $det(A)$ & $0$ & $-6$ & $0$ & $A_1 \nsim A_2 \land A_2 \nsim A_3$\\
        \end{tabular}
        \caption{Resolución usando propiedades de las matrices semejantes}
    \end{table}

    Calculamos el polinomio característico de $A_1$:
    \begin{multline*}
        P_{A_1}(\lambda) = \left| \begin{array}{ccc}
            1-\lambda & 2 & 1 \\
            2 & -\lambda & 2 \\
            1 & 2 & 1-\lambda \\
        \end{array}\right|
        = \left| \begin{array}{ccc}
            -\lambda & 2 & 1 \\
            0 & -\lambda & 2 \\
            \lambda & 2 & 1-\lambda \\
        \end{array}\right|
        = \left| \begin{array}{ccc}
            0 & 4 & 2-\lambda \\
            0 & -\lambda & 2 \\
            \lambda & 2 & 1-\lambda \\
        \end{array}\right| =\\=
        \lambda(8+\lambda(2-\lambda)) = \lambda(-\lambda^2+2\lambda+8)=-\lambda(\lambda-4)(\lambda+2)
    \end{multline*}

    Calculamos el polinomio característico de $A_3$:
    \begin{multline*}
        P_{A_3}(\lambda) = \left| \begin{array}{ccc}
            -1-\lambda & 2 & -1 \\
            2 & -\lambda & 2 \\
            3 & 2 & 3-\lambda \\
        \end{array}\right|
        = \left| \begin{array}{ccc}
            -\lambda & 2 & -1 \\
            0 & -\lambda & 2 \\
            \lambda & 2 & 3-\lambda \\
        \end{array}\right|
        = \left| \begin{array}{ccc}
            0 & 4 & 2-\lambda \\
            0 & -\lambda & 2 \\
            \lambda & 2 & 3-\lambda \\
        \end{array}\right| =\\=
        \lambda(8+\lambda(2-\lambda)) = \lambda(-\lambda^2+2\lambda+8)=-\lambda(\lambda-4)(\lambda+2)
    \end{multline*}

    Por tanto, ambas matrices tienen como valores propios: $\{-2,0,4\}$, por lo que son diagonalizables y semejantes a
        $$D = \left( \begin{array}{ccc}
            -2 & 0 & 0 \\
            0 & 0 & 0 \\
            0 & 0 & 4 \\
        \end{array}\right)$$

    Por tanto, como $A_1 \sim D \land D\sim A_3$, por ser $\sim$ una relación de equivalencia, $A_1 \sim A_3$.
\end{ejercicio}

\begin{ejercicio}
    Calcular $A^{12}$ y $A^{-7}$ para la matriz
    $$A = \left( \begin{array}{ccc}
            1 & 3 & 0 \\
            3 & -2 & -1 \\
            0 & -1 & 1 \\
        \end{array}\right)$$
        ¿$\exists B\in \mathcal{M}_3(\bb{R}) \mid B^2 = A$?\\

        En primer lugar, vemos si $A$ es diagonalizable.
        \begin{multline*}
            P_A(\lambda) = \left| \begin{array}{ccc}
                1-\lambda & 3 & 0 \\
                3 & -2-\lambda & -1 \\
                0 & -1 & 1-\lambda \\
            \end{array}\right|
            = -(1-\lambda)^2(2+\lambda)-(1-\lambda)-9(1-\lambda) = \\
            = -(1-\lambda)((1-\lambda)(2+\lambda)+1+9) = -(1-\lambda)(-\lambda^2-\lambda+12) = (1-\lambda)(\lambda+4)(\lambda-3)
        \end{multline*}

        Por tanto, como tiene tres valores propios distintos $\{1,3,-4\}$, es diagonalizable. Calculamos los suespacios propios.
        \begin{equation*}\begin{split}
           V_{1} & = \left\{ \left(\begin{array}{c}
                    x_1 \\
                    x_2 \\
                    x_3
               \end{array}\right) \in \bb{R}^3 \mid (A-I)\left(\begin{array}{c}
                    x_1 \\
                    x_2 \\
                    x_3
               \end{array}\right) = 0 \right\} \\
               & = \left\{ \left(\begin{array}{c}
                    x_1 \\
                    x_2 \\
                    x_3
               \end{array}\right) \in \bb{R}^3 \mid \left( \begin{array}{ccc}
                    0 & 3 & 0 \\
                    3 & -3 & -1 \\
                    0 & -1 & 0 \\
            \end{array}\right) \left(\begin{array}{c}
                    x_1 \\
                    x_2 \\
                    x_3
               \end{array}\right) = 0 \right\} \\
               & = \left\{ \left(\begin{array}{c}
                    x_1 \\
                    x_2  \\
                    x_3
               \end{array}\right) \in \bb{R}^3 \left| \begin{array}{c}
                    x_2 = 0 \\
                    3x_1 - 3x_2 - x_3 = 0
               \end{array}\right. \right\} = \mathcal{L}\left(\left\{
                    \left(\begin{array}{c}
                        1 \\
                        0 \\
                        3 \\
                   \end{array}\right)
                   \right\}\right)
       \end{split}\end{equation*}
       \begin{equation*}\begin{split}
           V_{3} & = \left\{ \left(\begin{array}{c}
                    x_1 \\
                    x_2 \\
                    x_3
               \end{array}\right) \in \bb{R}^3 \mid (A-3I)\left(\begin{array}{c}
                    x_1 \\
                    x_2 \\
                    x_3
               \end{array}\right) = 0 \right\} \\
               & = \left\{ \left(\begin{array}{c}
                    x_1 \\
                    x_2 \\
                    x_3
               \end{array}\right) \in \bb{R}^3 \mid \left( \begin{array}{ccc}
                    -2 & 3 & 0 \\
                    3 & -5 & -1 \\
                    0 & -1 & -2 \\
            \end{array}\right) \left(\begin{array}{c}
                    x_1 \\
                    x_2 \\
                    x_3
               \end{array}\right) = 0 \right\} \\
               & = \left\{ \left(\begin{array}{c}
                    x_1 \\
                    x_2  \\
                    x_3
               \end{array}\right) \in \bb{R}^3 \left| \begin{array}{c}
                    -2x_1 + 3x_2 = 0 \\
                    x_2+2x_3 = 0
               \end{array}\right. \right\} = \mathcal{L}\left(\left\{
                    \left(\begin{array}{c}
                        3 \\
                        2 \\
                        -1 \\
                   \end{array}\right)
                   \right\}\right)
       \end{split}\end{equation*}
       \begin{equation*}\begin{split}
           V_{-4} & = \left\{ \left(\begin{array}{c}
                    x_1 \\
                    x_2 \\
                    x_3
               \end{array}\right) \in \bb{R}^3 \mid (A+4I)\left(\begin{array}{c}
                    x_1 \\
                    x_2 \\
                    x_3
               \end{array}\right) = 0 \right\} \\
               & = \left\{ \left(\begin{array}{c}
                    x_1 \\
                    x_2 \\
                    x_3
               \end{array}\right) \in \bb{R}^3 \mid \left( \begin{array}{ccc}
                    5 & 3 & 0 \\
                    3 & 2 & -1 \\
                    0 & -1 & 5 \\
            \end{array}\right) \left(\begin{array}{c}
                    x_1 \\
                    x_2 \\
                    x_3
               \end{array}\right) = 0 \right\} \\
               & = \left\{ \left(\begin{array}{c}
                    x_1 \\
                    x_2  \\
                    x_3
               \end{array}\right) \in \bb{R}^3 \left| \begin{array}{c}
                    5x_1 + 3x_2 = 0 \\
                    -x_2 + 5x_3 = 0
               \end{array}\right. \right\} = \mathcal{L}\left(\left\{
                    \left(\begin{array}{c}
                        -3 \\
                        5 \\
                        1 \\
                   \end{array}\right)
                   \right\}\right)
       \end{split}\end{equation*}
       Por tanto, $A$ es diagonalizable de la forma $D=P^{-1}AP$, con:
       \begin{equation*}
           D=\left( \begin{array}{ccc}
                1 & 0 & 0  \\
                0 & 3 & 0  \\
                0 & 0 & -4  \\
           \end{array}\right) \qquad \qquad
           P=\left( \begin{array}{ccc}
                1 & 3 & -3  \\
                0 & 2 & 5  \\
                3 & -1 & 1  \\
           \end{array}\right)
       \end{equation*}
       Por tanto, despejando $A$, $A=PDP^{-1}$. Por tanto,
       $$A^{12} = PD^{12}P^{-1},\text{ con }D^{12}=\left( \begin{array}{ccc}
                1 & 0 & 0  \\
                0 & 3^{12} & 0  \\
                0 & 0 & (-4)^{12}  \\
           \end{array}\right)$$
       $$A^{-7} = PD^{-7}P^{-1},\text{ con }D^{-7}=\left( \begin{array}{ccc}
                1 & 0 & 0  \\
                0 & \frac{1}{3^7} & 0  \\
                0 & 0 & \frac{-1}{4^7}  \\
           \end{array}\right)$$
       $$B=\sqrt{A} = P\sqrt{D}P^{-1},\text{ con }\sqrt{D}=\left( \begin{array}{ccc}
            1 & 0 & 0  \\
            0 & \sqrt{3} & 0  \\
            0 & 0 & \sqrt{-4}  \\
       \end{array}\right) \in \bb{C}$$
       Por tanto, $\nexists B\in \mathcal{M}_3(\bb{R}) \mid B^2 = A$, ya que $\sqrt{-4}\notin \bb{R}$.
\end{ejercicio}

\begin{ejercicio}
    Dada la ecuación $A^2 = 9I$:
    \begin{enumerate}
        \item Resolver en $\mathcal{M}_2(\bb{R})$.\\
        Transformamos la ecuación a $I = \left( \frac{1}{3}A\right)^2$. Por tanto, buscamos $B \in \mathcal{M}_2(\bb{R}) \mid B^2 = I$.

        \begin{equation*}
            B = I_2 \quad \text{ó} \quad 
            B = -I_2
            \quad \text{ó} \quad
            B\sim\footnote{En los dos primeros casos, no hay más matrices semejantes a ellas, ya que $B=\pm I$} \left(\begin{array}{cc}
                1 & 0\\
                0 &-1
            \end{array} \right)
        \end{equation*}
        Por tanto, las soluciones de $A^2 = 9I$ son $A=3B$, para las $B$ indicadas previamente.

        \item Resolver en $\mathcal{M}_3(\bb{R})$.\\
        Transformamos la ecuación a $I = \left( \frac{1}{3}A\right)^2$. Por tanto, buscamos $B \in \mathcal{M}_3(\bb{R}) \mid B^2 = I$.

        \begin{equation*}
            B = I_3 \quad \text{ó} \quad 
            B = -I_3
            \quad \text{ó} \quad
            B\sim \left(\begin{array}{ccc}
                1 & 0 & 0\\
                0 & 1 & 0\\
                0 & 0 & -1\\
            \end{array} \right)
            \quad \text{ó} \quad
            B\sim \left(\begin{array}{ccc}
                1 & 0 & 0\\
                0 & -1 & 0\\
                0 & 0 & -1\\
            \end{array} \right)
        \end{equation*}
        Por tanto, las soluciones de $A^2 = 9I$ son $A=3B$, para las $B$ indicadas previamente.
    \end{enumerate}
\end{ejercicio}

\begin{ejercicio}
    Sea la ecuación $A^2=0$:
    \begin{enumerate}
        \item Resolver en $\mathcal{M}_3(\bb{R})$:\\
        Sea $A=M(f;\mathcal{B})$, con $f\in End(\bb{R}^3)$ t.q. $f\circ f = 0$.

        Calculemos los valores propios. Si $\lambda_0$ es un valor propio de $A$, \begin{multline*}
            \exists v\in \bb{R}^3-\{0\} \mid f(v) = \lambda_0 v \Longrightarrow f(f(v)) = 0 = f(\lambda_0v) = \lambda_0f(v) = \lambda_0^2v\Longrightarrow \\ \Longrightarrow 0=\lambda_0^2v \Longrightarrow \lambda_0 = 0 \text{ es el único valor propio posible.}
        \end{multline*}
    
        Por tanto, $\lambda_0=0$ será una raíz del polinomio característico de $f$ con multiplicidad algebraica $m_0 \geq 1$. Por tanto, la multiplicidad geométrica debe ser $n_0\geq 1$.
        \begin{equation}
            n_0 = \dim Ker(f) \geq 1
        \end{equation}
        
        Además, como $f\circ f = 0 \Longrightarrow Im(f) \subset Ker(f)$, ya que $\forall  f(v) \in Im(f),\; f(f(v))=~0$ $ \Longrightarrow f(v) \in Ker(f)$.
        \begin{equation}\label{Ec:Subset}
            Im(f) \subset Ker(f) \Longrightarrow \dim Im(f) \leq \dim Ker(f) = n_0
        \end{equation}
        
       Además, sabemos que
       \begin{equation}\label{Ec:SumDim}
           \dim Ker(f) + \dim Im(f) = 3
       \end{equation}
    
       \begin{itemize}
           \item \underline{Supongamos $n_0 = 3$:}\\
           Como $\dim Ker(f)=3 \Longrightarrow Ker(f) = \bb{R}^3 \Longrightarrow f=0 \Longrightarrow A=0$
    
           \item \underline{Supongamos $n_0 = 1$:}\\
           Por la Ec. \ref{Ec:SumDim}, $\dim Im(f)=2$, pero por la Ec. \ref{Ec:Subset}, $2\leq 1$, lo que es una contradicción.
    
           \item \underline{Supongamos $n_0 = 2$:}\\
           Por la Ec. \ref{Ec:SumDim}, $\dim Im(f)=1$. Sea $\bar{\mathcal{B}}=\{e_1, e_2, e_3\}$ base de $\bb{R}^3$. Sea $\{e_2\}$ la base de $Im(f)$. Ampliamos a una base $\{e_2, e_3\}$ de $Ker(f)$. Además, definimos $f(e_1)=~e_2$.
           Por tanto,
           $$\bar{A} = M(f; \bar{\mathcal{B}}) = \left( \begin{array}{ccc}
               0 & 0 & 0 \\
               1 & 0 & 0 \\
               0 & 0 & 0 \\
           \end{array}\right)$$
       \end{itemize}
    
       Por tanto, las soluciones de la ecuación $A^2 = 0$ son\footnote{En este caso se dice que solo hay una clase de semejanza}:
       \begin{equation*}
           A = 0
           \qquad \text{ ó } \qquad
           A\sim \left( \begin{array}{ccc}
               0 & 0 & 0 \\
               1 & 0 & 0 \\
               0 & 0 & 0 \\
           \end{array}\right)
       \end{equation*}

        \item Resolver en $\mathcal{M}_4(\bb{R})$:\\
        Seguimos un razonamiento análogo al apartado anterior.
       \begin{itemize}
           \item \underline{Supongamos $n_0 = 4$:}\\
           Como $\dim Ker(f)=4 \Longrightarrow Ker(f) = \bb{R}^4 \Longrightarrow f=0 \Longrightarrow A=0$
    
           \item \underline{Supongamos $n_0 = 1$:}\\
           Por la análoga a la Ec. \ref{Ec:SumDim}, $\dim Im(f)=3$, pero por la Ec. \ref{Ec:Subset}, $3\leq 1$, lo que es una contradicción.
    
           \item \underline{Supongamos $n_0 = 3$:}\\
           Por la análoga a la Ec. \ref{Ec:SumDim}, $\dim Im(f)=1$. Sea $\bar{\mathcal{B}}=\{e_1, e_2, e_3, e_4\}$ base de $\bb{R}^4$. Sea $\{e_2\}$ la base de $Im(f)$. Ampliamos a una base $\{e_2, e_3, e_4\}$ de $Ker(f)$. Además, definimos $f(e_1)=~e_2$.
           Por tanto,
           $$\bar{A} = M(f; \bar{\mathcal{B}}) = \left( \begin{array}{cccc}
               0 & 0 & 0 & 0\\
               1 & 0 & 0 & 0\\
               0 & 0 & 0 & 0\\
               0 & 0 & 0 & 0\\
           \end{array}\right)$$

           \item \underline{Supongamos $n_0 = 2$:}\\
           Por la análoga a la Ec. \ref{Ec:SumDim}, $\dim Im(f)=2$. Sea $\bar{\mathcal{B}}=\{e_1, e_2, e_3, e_4\}$ base de $\bb{R}^4$. Sea $\{e_3, e_4\}$ la base de $Im(f)$ y de $Ker(f)$. Además, definimos $f(e_1)=~e_3,\;f(e_2)=~e_4$.
           Por tanto,
           $$\bar{A} = M(f; \bar{\mathcal{B}}) = \left( \begin{array}{cccc}
               0 & 0 & 0 & 0\\
               0 & 0 & 0 & 0\\
               1 & 0 & 0 & 0\\
               0 & 1 & 0 & 0\\
           \end{array}\right)$$
       \end{itemize}
    
       Por tanto, las soluciones de la ecuación $A^2 = 0$ son, con 2 clases de semejanza,
       \begin{equation*}
           A = 0
           \qquad \text{ ó } \qquad
           A\sim \left( \begin{array}{cccc}
               0 & 0 & 0 & 0\\
               1 & 0 & 0 & 0\\
               0 & 0 & 0 & 0\\
               0 & 0 & 0 & 0\\
           \end{array}\right)
           \qquad \text{ ó } \qquad
           A\sim \left( \begin{array}{cccc}
               0 & 0 & 0 & 0\\
               0 & 0 & 0 & 0\\
               1 & 0 & 0 & 0\\
               0 & 1 & 0 & 0\\
           \end{array}\right)
       \end{equation*}

       \item Resolver en $\mathcal{M}_2(\bb{R})$:\\
        Seguimos un razonamiento análogo al apartado anterior.
       \begin{itemize}
           \item \underline{Supongamos $n_0 = 2$:}\\
           Como $\dim Ker(f)=2 \Longrightarrow Ker(f) = \bb{R}^2 \Longrightarrow f=0 \Longrightarrow A=0$
    
           \item \underline{Supongamos $n_0 = 1$:}\\
           Por la análoga a la Ec. \ref{Ec:SumDim}, $\dim Im(f)=1$. Sea $\bar{\mathcal{B}}=\{e_1, e_2\}$ base de $\bb{R}^2$. Sea $\{e_2\}$ la base de $Im(f)$ y $Ker(f)$. Además, definimos $f(e_1)=~e_2$.
           Por tanto,
           $$\bar{A} = M(f; \bar{\mathcal{B}}) = \left( \begin{array}{cc}
               0 & 0\\
               1 & 0\\
           \end{array}\right)$$
       \end{itemize}
    
       Por tanto, las soluciones de la ecuación $A^2 = 0$, con 1 clase de semejanza, son:
       \begin{equation*}
           A = 0
           \qquad \text{ ó } \qquad
           A\sim \left( \begin{array}{cc}
               0 & 0\\
               1 & 0\\
           \end{array}\right)
       \end{equation*}
       
    \end{enumerate}
    
\end{ejercicio}

\begin{ejercicio}
    Resolver la ecuación $f^3=f \in End_\bb{R}\bb{R}^n$.\\
    
    Calculemos los posibles valores propios. Si $\lambda_0$ es un valor propio de $f$,
    \begin{multline*}
        \exists v\in \bb{R}^3-\{0\} \mid f(v) = \lambda_0 v \Longrightarrow f^3(v) = \lambda_0^3v\Longrightarrow \lambda_0^3v=\lambda_0v
        \Longrightarrow \\ \Longrightarrow
        \lambda_0^3 = \lambda_0 \Longrightarrow \lambda_0 = \{-1,0,1\} \text{ son los posibles valores propios.}
    \end{multline*}
    \begin{itemize}
        \item \underline{Caso 1:} $f\in Aut(\bb{R}^n)$\\
        Como $f^3 = f\Longrightarrow f(f^2-Id) = 0$. Además, como $f$ es biyectiva, $\exists f^{-1}$. Por tanto, las soluciones son $f^2=Id$.
        
        Por tanto, las soluciones tomando $A=M(f, \mathcal{B})$ son:
        \begin{equation*}
            A\sim \left(\begin{array}{cccccc}
                1&&&&&\\
                &\ddots^{r\;veces}&&&&\\
                &&1&&&\\
                &&&-1&&\\
                &&&&\ddots^{s\;veces}&\\
                &&&&&-1\\
            \end{array} \right), \text{ con $r$ y $s$ tal que $r+s=n$}
        \end{equation*}

        \item \underline{Caso 2:} $f$ no biyectiva. Es decir, $ker(f)\neq \{0\}$\\
        Sea $v\in Ker(f)\cap Im(f) -\{0\}$
        \begin{itemize}
            \item $\exists w \in V \mid f(w) = v$
            \item $f(v)=0$
        \end{itemize}
        Por tanto, $f^2(w)=0 \Longrightarrow f^3(w) = f(0) = 0 \Longrightarrow f^3(w) = f(w) = 0$. Por tanto, como $f(w)=v=0 \Longrightarrow v=0$. Pero $v\neq 0$, por lo que llegamos a una contradicción. Por tanto, $Ker(f)\cap Im(f)=\{0\}$. En conclusión,
        $$Ker(f)\oplus Im(f) = V$$

        La restricción $h:=f|_{Im(f)}:Im(f) \to Im(f)$ tiene núcleo trivial.
        $$Ker(h)=\{0\}$$
        Por tanto, nos referimos al apartado anterior, ya que la restricción $h$ es un automorfismo.
        
        Por tanto, existe una base de $Im(f)\;\{e_1, \dots, e_r\}$ t.q. $f(e_i)=\pm e_i\quad \forall i$.

        Sea $\{e_{r+1}, \dots, e_n\}$ base de $Ker(f)$.  Por tanto, $\{e_1, \dots, e_r, e_{r+1}, \dots, e_n\}$ es una base de $V$. Por tanto, $f$ es diagonalizable.
        
        Por tanto, las soluciones tomando $A=M(f, \mathcal{B})$ son:
        \begin{equation*}
            A\sim \left(\begin{array}{ccccccccc}
                1&&&&&&&&\\
                &\ddots^{x\;veces}&&&&&&&\\
                &&1&&&&&&\\
                &&&-1&&&&&\\
                &&&&\ddots^{y\;veces}&&&&\\
                &&&&&-1&&&\\
                &&&&&&0&&\\
                &&&&&&&\ddots^{t\;veces}&\\
                &&&&&&&&0\\
            \end{array} \right) \text{ con $x,y,t\mid x+y+t=n$}
        \end{equation*}
    \end{itemize}
\end{ejercicio}

\begin{ejercicio}
    Sea $f\in End(\bb{R}^3)$ que respecto a la base usual $\mathcal{B}_u$ tiene la matriz asociada
    \begin{equation*}
        M(f, \mathcal{B}_u) = A = \left(\begin{array}{ccc}
            1+\alpha & -\alpha & \alpha \\
            2+\alpha & -\alpha & \alpha-1 \\
            2 & -1 & 0
        \end{array} \right)
    \end{equation*}
    \begin{enumerate}
        \item Estudiar para qué valores de $\alpha \in \bb{R}$ la matriz es diagonalizable.
        \begin{multline*}
            P_f(\lambda) = \left|\begin{array}{ccc}
            1+\alpha-\lambda & -\alpha & \alpha \\
            2+\alpha & -\alpha-\lambda & \alpha-1 \\
            2 & -1 & -\lambda
        \end{array} \right| =
        \left|\begin{array}{ccc}
            1+\alpha-\lambda & -\alpha & 0 \\
            2+\alpha & -\alpha-\lambda & -1-\lambda \\
            2 & -1 & -1-\lambda
        \end{array} \right| = \\ =
        \left|\begin{array}{ccc}
            1+\alpha-\lambda & -\alpha & 0 \\
            \alpha & 1-\alpha-\lambda & 0 \\
            2 & -1 & -1-\lambda
        \end{array} \right|
        = -(1+\lambda)\left|\begin{array}{cc}
            1+\alpha-\lambda & -\alpha\\
            \alpha & 1-\alpha-\lambda\\
        \end{array} \right| = \\ = -(1+\lambda)\left[ (1-\lambda+\alpha)(1-\lambda-\alpha)+\alpha^2 \right]
        = -(1+\lambda)\left[ (1-\lambda)^2 - \alpha^2 +\alpha^2 \right] = \\=
        -(1+\lambda)(1-\lambda)^2
        \end{multline*}

        Por tanto, independientemente del valor de $\alpha$, los valores propios son: $\{1, -1\}$. Veamos la multiplicidad geométrica del $1$:
        \begin{equation*}
            rg(A-I) = rg\left(\begin{array}{ccc}
            \alpha & -\alpha & \alpha \\
            2+\alpha & -\alpha-1 & \alpha-1 \\
            2 & -1 & -1
        \end{array} \right)
        \end{equation*}
        Calculamos el rango mediante determinantes.
        \begin{multline*}
            det(A-I) = \left|\begin{array}{ccc}
            \alpha & -\alpha & \alpha \\
            2+\alpha & -\alpha-1 & \alpha-1 \\
            2 & -1 & -1
        \end{array} \right| = \alpha \left|\begin{array}{ccc}
            1 & -1 & 1 \\
            2+\alpha & -\alpha-1 & \alpha-1 \\
            2 & -1 & -1
        \end{array} \right| =\\=
        \alpha \left|\begin{array}{ccc}
            1 & -1 & 0 \\
            2+\alpha & -\alpha-1 & -2 \\
            2 & -1 & -2
        \end{array} \right|
        =\alpha \left|\begin{array}{ccc}
            1 & -1 & 0 \\
            \alpha & -\alpha & 0 \\
            2 & -1 & -2
        \end{array} \right| = \alpha \cdot 0 = 0
        \end{multline*}
        \begin{equation*}
            \left| \begin{array}{cc}
                \alpha & -\alpha \\
                2 & -1
            \end{array}\right| = -\alpha +2\alpha = \alpha = 0 \Longleftrightarrow \alpha = 0
        \end{equation*}

        \begin{itemize}
            \item \underline{Para $a\neq 0$:}\\
            Si $\alpha\neq 0\Longrightarrow rg(A-I)=2 \Longrightarrow \dim V_1 = 3-2 = 1$.
            \begin{table}[H]
                \centering
                \begin{tabular}{c|c|c}
                    Valores Propios & Mult. Alg. & Mult. Geom. \\ \hline 
                    1 & 2 & 1\\
                    $-$1 & 1 & 1\\
                \end{tabular}
                \caption{Valores propios con sus multiplicidades para $\alpha\neq 0$}
            \end{table}
            Por tanto, para $\alpha\neq 0, f$ no es diagonalizable.

            \item \underline{Para $\alpha=0$:}\\
            \begin{equation*}
                rg(A-I) = rg\left(\begin{array}{ccc}
                    0 & 0 & 0 \\
                    2 & -1 & -1 \\
                    2 & -1 & -1
                \end{array} \right) = 1
            \end{equation*}
            Como $rg(A-I)=1 \Longrightarrow \dim V_1 = 3-1 = 2$.
            \begin{table}[H]
                \centering
                \begin{tabular}{c|c|c}
                    Valores Propios & Mult. Alg. & Mult. Geom. \\ \hline 
                    1 & 2 & 2\\
                    $-$1 & 1 & 1\\
                \end{tabular}
                \caption{Valores propios con sus multiplicidades para $\alpha= 0$}
            \end{table}
            Por tanto, para $\alpha= 0, f$ sí es diagonalizable.
        \end{itemize}
        
        \item Diagonalizar $f$ dando la matriz de cambio de base.\\
        Diagonalizamos para para $\alpha=0$, ya que es el único valor para el cual $f$ es diagonalizable.
        \begin{equation*}
            M(f, \mathcal{B}_u) = A = \left(\begin{array}{ccc}
                1 & 0 & 0 \\
                2 & 0 & -1 \\
                2 & -1 & 0
            \end{array} \right)
        \end{equation*}
        \begin{equation*}\begin{split}
           V_{1} & = \left\{ \left(\begin{array}{c}
                    x_1 \\
                    x_2 \\
                    x_3
               \end{array}\right) \in \bb{R}^3 \mid (A-I)\left(\begin{array}{c}
                    x_1 \\
                    x_2 \\
                    x_3
               \end{array}\right) = 0 \right\} \\
               & = \left\{ \left(\begin{array}{c}
                    x_1 \\
                    x_2 \\
                    x_3
               \end{array}\right) \in \bb{R}^3 \mid \left( \begin{array}{ccc}
                    0 & 0 & 0 \\
                    2 & -1 & -1 \\
                    2 & -1 & -1
            \end{array}\right) \left(\begin{array}{c}
                    x_1 \\
                    x_2 \\
                    x_3
               \end{array}\right) = 0 \right\} \\
               & = \left\{ \left(\begin{array}{c}
                    x_1 \\
                    x_2  \\
                    x_3
               \end{array}\right) \in \bb{R}^3 \left| \begin{array}{c}
                    2x_1 -x_2 - x_3 = 0
               \end{array}\right. \right\} = \mathcal{L}\left(\left\{
                    \left(\begin{array}{c}
                        0 \\
                        -1 \\
                        1 \\
                   \end{array}\right),
                   \left(\begin{array}{c}
                        1 \\
                        1 \\
                        1 \\
                   \end{array}\right)
                   \right\}\right)
       \end{split}\end{equation*}
       \begin{equation*}\begin{split}
           V_{-1} & = \left\{ \left(\begin{array}{c}
                    x_1 \\
                    x_2 \\
                    x_3
               \end{array}\right) \in \bb{R}^3 \mid (A+I)\left(\begin{array}{c}
                    x_1 \\
                    x_2 \\
                    x_3
               \end{array}\right) = 0 \right\} \\
               & = \left\{ \left(\begin{array}{c}
                    x_1 \\
                    x_2 \\
                    x_3
               \end{array}\right) \in \bb{R}^3 \mid \left( \begin{array}{ccc}
                    2 & 0 & 0 \\
                    2 & 1 & -1 \\
                    2 & -1 & 1
            \end{array}\right) \left(\begin{array}{c}
                    x_1 \\
                    x_2 \\
                    x_3
               \end{array}\right) = 0 \right\} \\
               & = \left\{ \left(\begin{array}{c}
                    x_1 \\
                    x_2  \\
                    x_3
               \end{array}\right) \in \bb{R}^3 \left| \begin{array}{c}
                    x_1 = 0\\
                    2x_1 + x_2 - x_3 = 0
               \end{array}\right. \right\} = \mathcal{L}\left(\left\{
                    \left(\begin{array}{c}
                        0 \\
                        1 \\
                        1 \\
                   \end{array}\right)
                   \right\}\right)
       \end{split}\end{equation*}
       Por tanto, $D=P^{-1}AP$, con:
       \begin{equation*}
        D = \left(\begin{array}{ccc}
            1 & 0 & 0\\
            0 & 1 & 0\\
            0 & 0 & -1\\
        \end{array} \right) \qquad \qquad
        P = \left(\begin{array}{ccc}
            0 & 1 & 0\\
            -1 & 1 & 1\\
            1 & 1 & 1\\
        \end{array} \right)
    \end{equation*}
    siendo $D=M(f, \mathcal{B}')$ la matriz asociada a $f$ y $P=M(\mathcal{B}', \mathcal{B}_u)$ la matriz de cambio de base.   
    \end{enumerate}
\end{ejercicio}

\begin{ejercicio}
    Sea $f\in End(V)$. Probar:
    \begin{enumerate}
        \item $f(V_\lambda) = V_\lambda$, para todo valor propio $\lambda\neq 0$.
        \begin{proof}
            Veamos en primer lugar que $f(V_\lambda) \subset V_\lambda$:
            \begin{equation}\label{Ec:9.a.Subset}
                \forall f(v) \in f(V_\lambda), f(v) = \lambda v \in V_\lambda \Longrightarrow f(V_\lambda) \subset V_\lambda
            \end{equation}
            donde hemos usado que $\lambda$ es un escalar y que $V_\lambda$ es un subespacio vectorial, por lo que es cerrado para el producto por escalares.
    
            Veamos ahora que tienen la misma dimensión.
            $$Ker(V_\lambda) = \{v\in V_\lambda \mid \lambda v = 0\} = \{0\},\text{ ya que } \lambda \neq 0$$
            \begin{equation}\label{Ec:9.a.Dim}
                \dim Im(V_\lambda) = \dim V_\lambda - \dim Ker(V_\lambda) = \dim V_\lambda
            \end{equation}
    
            Por tanto, por las ecuaciones \ref{Ec:9.a.Subset} y \ref{Ec:9.a.Dim},
            $$f(V_\lambda) = V_\lambda \qquad \forall \lambda\neq 0$$
        \end{proof}

        \item $f \in Aut(V) \Longleftrightarrow 0$ no es un valor propio de $f$.
        \begin{proof} Procedemos mediante doble implicación:
            \begin{description}
                \item [$\Longrightarrow )$] Suponemos $f$ automorfismo.\\
                Por tanto, $f$ es homomorfismo, por lo que $Ker(f) = \{0\}$.
                Por tanto,
                $$\nexists v \in V-\{0\} \mid f(v) = 0v = 0$$
                Por tanto, el $0$ no es un valor propio de $f$.

                \item [$\Longleftarrow )$] Suponemos que el 0 no es un valor propio de $f$.\\
                Entonces, $Ker(f) = \{0\}$, por lo que $f$ es un homomorfismo.

                Además, como $\dim Im(f) = \dim V - \dim Ker(f) = \dim V$, también es un epimorfismo.

                Por tanto, $f$ es un isomorfismo, por lo que es un automorfismo.
            \end{description}
        \end{proof}

        \item Sea $f \in Aut(V)$. $\lambda$ es un valor propio de $f \Longleftrightarrow \lambda^{-1}$ es un valor propio de $f^{-1}$.
         \begin{proof} Procedemos mediante doble implicación:
            \begin{description}
                \item [$\Longrightarrow )$] Suponemos $\lambda$ valor propio de $f$.\\
                En primer lugar, hay que destacar que como $f$ es un automorfismo, y por lo demostrado en el apartado anterior, $\lambda\neq 0$.

                Como $\lambda$ es un valor propio, $\exists v \in V-\{0\} \mid f(v) = \lambda v$.
                \begin{multline*}
                    (f^{-1}\circ f)(v) = v \Longrightarrow f^{-1}(f(v)) = f^{-1}(\lambda v) = \lambda f^{-1}(v) =v
                    \Longrightarrow \\ \Longrightarrow
                    f^{-1}(v) = \lambda ^{-1} v \Longrightarrow \lambda^{-1} \text{ es un valor propio de } f^{-1}
                \end{multline*}

                \item [$\Longleftarrow )$] Suponemos $\lambda^{-1}$ valor propio de $f^{-1}$.\\
                En primer lugar, hay que destacar que como $f$ es un automorfismo, $f^{-1}$ también lo es y, por lo demostrado en el apartado anterior, $\lambda^{-1} \neq 0$.

                Como $\lambda^{-1}$ es un valor propio, $\exists v \in V-\{0\} \mid f^{-1}(v) = \lambda^{-1} v$.
                \begin{multline*}
                    (f\circ f^{-1})(v) = v \Longrightarrow f(f^{-1}(v)) = f(\lambda^{-1} v) = \lambda^{-1} f(v) =v
                    \Longrightarrow \\ \Longrightarrow
                    f(v) = \lambda v \Longrightarrow \lambda \text{ es un valor propio de } f
                \end{multline*}
            \end{description}
            
        \end{proof}
        
    \end{enumerate}
    
\end{ejercicio}

\begin{ejercicio}
    Sea $f\in End(\bb{R}^2) \mid nul(f) = 1$. Probar que $f$ es diagonalizable si y solo si $Ker(f) \cap Im(f) = \{0\}$.
    \begin{proof}Procedemos mediante doble implicación:
        \begin{description}
            \item [$\Longrightarrow )$]Suponemos $f$ diagonalizable.
            
            Como $nul(f) = \dim Ker(f) = 1$, sea $\{e_2\}$ base de $Ker(f)$.
            $$f(e_2) = 0$$
            Amplío a una base de $\bb{R}^2, \mathcal{B} = \{e_1,e_2\}$
            $$f(e_1) = ae_1 + be_2$$
            $$M(f;\mathcal{B}) = \left(\begin{array}{cc}
                a & 0 \\
                b & 0
            \end{array} \right)$$
            Su polinomio característico es: $P_f(\lambda) = \lambda(\lambda-a)$, por lo que tiene como valores propios $\{0,a\}$.
        
            Por el Teorema Fundamental de Diagonalización, como $f$ es diagonalizable y $\dim V_0 = \dim Ker(f) = 1$, la multiplicidad algebraica del 0 $m_0 = 1$. Por tanto, $$a\neq 0$$

            Veamos ahora el valor de $Ker(f) \cap Im(f)$. Supongamos $x \in Ker(f) \cap Im(f):$
            \begin{equation*}\begin{split}
                x \in Ker(f) & \Longrightarrow f(x) = 0 \Longrightarrow
                 \left(\begin{array}{cc}
                    a & 0 \\
                    b & 0
                \end{array} \right)
                 \left(\begin{array}{c}
                    x_1 \\
                    x_2
                \end{array} \right) = 
                \left(\begin{array}{c}
                    ax_1 \\
                    bx_1
                \end{array} \right) = 0 \Longrightarrow\\
               &  \qquad \Longrightarrow ax_1 = 0 \Longrightarrow x_1 = 0 \\
               x \in Im(f) & \Longrightarrow \exists y\in \bb{R}^2 \mid f(y) = x \Longrightarrow
                 \left(\begin{array}{cc}
                    a & 0 \\
                    b & 0
                \end{array} \right)
                 \left(\begin{array}{c}
                    y_1 \\
                    y_2
                \end{array} \right) = 
                \left(\begin{array}{c}
                    ay_1 \\
                    by_1
                \end{array} \right) =
                \left(\begin{array}{c}
                    0 \\
                    x_2
                \end{array} \right) \Longrightarrow\\
               &  \qquad \Longrightarrow \left\{\begin{array}{c}
                   ay_1 = 0 \Longrightarrow y_1 = 0   \\
                   by_1 = x_2 \Longrightarrow x_2 = 0 
               \end{array} \right. \\
            \end{split}\end{equation*}
            Por tanto, como $x_1=x_2 = 0$, $x=0$ y por tanto $Ker(f) \cap Im(f)=\{0\}$.

            \item [$\Longleftarrow )$]Suponemos $Ker(f) \cap Im(f)=\{0\}$. Veamos que $f$ es diagonalizable.

            Veamos que $V = Ker(f) \oplus~Im(f)$:
            \begin{multline*}
                \dim \bb{R}^2 = \dim Im(f) + \dim Ker(f) = 
                \dim(Im(f) + Ker(f)) + \dim(Im(f) \cap Ker(f)) = \\ = \dim(Im(f) + Ker(f))
            \end{multline*}
            Como $Im(f)\subset \bb{R}^2$ y $Ker(f)\subset \bb{R}^2 \Longrightarrow Im(f) + Ker(f) \subset \bb{R}^2$. Como además, tienen la misma dimensión, se da la igualdad, $Im(f) + Ker(f) = \bb{R}^2$. Por último, como $Ker(f) \cap Im(f)=\{0\}$, la suma es directa.
            $$Im(f) \oplus Ker(f) = \bb{R}^2$$

            Como $\dim Ker(f)=1$, $\dim Im(f) = 1$. Sea $\{e_2\}$ base de $Ker(f)$ y $\{e_1\}$ base de $Im(f)$. Como es suma directa, la unión de las bases también es una base. Por tanto, $\mathcal{B} = \{e_1, e_2\}$ es una base de $\bb{R}^2$.
            $$f(e_1) = ae_1,\; a\neq 0 \qquad f(e_2) = 0 \qquad M(f;\mathcal{B}) = \left(\begin{array}{cc}
                a & 0 \\
                0 & 0
            \end{array} \right)$$
            Por tanto, los valores propios del endomorfismo son $\{0,a\}$. Como $a \neq 0$, tiene dos valores propios distinos. Por tanto, $f$ es diagonalizable.\qedhere
        \end{description}
    \end{proof}
\end{ejercicio}

\begin{ejercicio}
    Sea $f\in End(V) \mid f^2=f$. Probar que
    $$V = Ker(f) \oplus Ker(f-1_V).$$
    Deducir que $f$ es diagonalizable.
    \begin{proof}
        Demostramos en primer lugar que $Ker(f) \cap Ker(f-1_V) = \{0\}$. Sea $v\in Ker(f) \cap Ker(f-1_V)$.
        \begin{equation*}
            \begin{split}
                v&\in Ker(f) \Longrightarrow f(v) = 0\\
                v&\in Ker(f-1_V) \Longrightarrow (f-1_V)(v) = 0 = f(v) - v \Longrightarrow f(v) = v
            \end{split}
        \end{equation*}
        Por tanto, $v=f(v) = 0\Longrightarrow Ker(f) \cap Ker(f-1_V) = \{0\}$.
    
        Veamos ahora que $V=Ker(f) + Ker(f-1_V)$. Sea $v\in V$:
        $$v=\underbrace{v-f(v)}_{t_1} + \underbrace{f(v)}_{t_2}$$
    
        Veamos que $t_1 = v-f(v) \in Ker(f)$:
        $$f(t_1) = f(v-f(v)) = f(v) - f^2(v) = f(v) - f(v) = 0 \Longrightarrow t_1 \in Ker(f)$$
        
         Veamos que $t_2 = f(v) \in Ker(f-1_V)$:
        $$(f-1_V)(t_2) =  (f-1_V)(f(v)) = f(v)-f(v) = 0\Longrightarrow t_2 \in Ker(f-1_V)$$
    
        Por tanto, como $V=Ker(f) + Ker(f-1_V)$ y, además, son conjuntos disjuntos, $$V=Ker(f) \oplus Ker(f-1_V)$$
    
        Veamos ahora que $f$ es diagonalizable:
        \begin{equation}\label{RazonamientoErroneo2}\tag{$\ast\ast$}
        \begin{split}
            f(t_1) = 0 & \xrightarrow{(\ast\ast)} t_1 \text{ es un vector propio.}\quad t_1\in V_0 \\
            f(t_2) = f(f(v)) = f(v) = t_2 & \xrightarrow{(\ast\ast)} t_2 \text{ es un vector propio.}\quad t_2\in V_{1}
        \end{split}\end{equation}
    
        Es fácil ver que $V_0 = Ker(f)$ y $V_1 = Ker(f-1_V)$. Por tanto, sea $\mathcal{B}_0=\{v_1, \dots, v_k\}$ base de $V_0 = Ker(f)$ y sea $\mathcal{B}_1=\{u_1, \dots, u_t\}$ base de $V_1 = Ker(f-1_V)$. $\mathcal{B}_0 \cup \mathcal{B}_1$ será base de $V$ por ser suma directa, y además todos sus elementos serán vectores propios, por lo que será diagonalizable.
    
        (\ref{RazonamientoErroneo2}) Este razonamiento es válido siempre que $f\neq \{Id, 0\}$, ya que $t_1$ o $t_2$ serían vectores nulos. Sin embargo, tanto $f=Id$ como $f=0$ son diagonalizables.
    \end{proof}
\end{ejercicio}

\begin{ejercicio}
    Sea $A \in \mathcal{M}_n(\bb{R})$ t.q. $|A|<0$. Estudiar si $\exists B\in \mathcal{M}_n(\bb{R}) \mid B^2=A$.\\

    Supongamos que sí existe. Tomando determinantes,
    $$|B^2| = |A| \Longrightarrow |B|^2 = |A| \Longrightarrow |B|^2 < 0$$
    Por tanto, llegamos a una contradicción, ya que no podemos encontrar $|B|\in \bb{R}$. Por tanto, $\nexists B \mid B^2=A$.
\end{ejercicio}

\begin{ejercicio}
    Dada $A\in \mathcal{M}_n(\bb{K})$, sea el espacio vectorial $\upsilon \subset \mathcal{M}_n(\bb{K})$ dado por:
    $$\upsilon = \left\{ p(A) \mid p\in \bb{K}[\lambda]\right\}$$
    Demostrar que $\dim \upsilon \leq n$.
    \begin{proof}
        Como $\dim \mathcal{M}_n(\bb{K}) = n^2 \Longrightarrow \dim \upsilon \leq n^2$.

        Aplicando el Teorema de Cayley-Hamilton,
        $$P_A(A) = 0 = \pm A^n + c_{n-1}A^{n-1} + \dots + c_1A + c_0\cdot I$$
        Por tanto, $A^n$ es combinación lineal de $\{I, A,\dots, A^{n-1}\}$.
    
        Divido $p(\lambda)$ un polinomio cualquiera entre $P_A(\lambda)$:
        $$p(\lambda) = P_A(\lambda)q(\lambda) + r(\lambda)\qquad grd(r)<n$$
    
        Por tanto, $p(A) = \cancelto{0}{P_A(A)q(A)} + r(A)$.
    
        Por tanto, el subespacio $\upsilon$ tiene como sistema de generadores $\{I, A, \dots, A^{n-1}\} \Longrightarrow$
        $\Longrightarrow \dim \upsilon \leq n$
    \end{proof}
\end{ejercicio}

\begin{ejercicio}
    Sea $A\in \mathcal{M}_n(\bb{R}) \subset \mathcal{M}_n(\bb{C})$.

    \begin{enumerate}
        \item ¿Puede ser que sea diagonalizable en $\bb{R}$ pero no en $\bb{C}$?\\
        No, ya que $\mathcal{M}_n(\bb{R}) \subset \mathcal{M}_n(\bb{C})$.
        
        \item ¿Puede ser que sea diagonalizable en $\bb{C}$ pero no en $\bb{R}$?\\
        Sí, y como ejemplo tenemos la matriz $A=\left(\begin{array}{cc}
            0 & 1 \\
            1 & 0
        \end{array} \right)$, con polinomio característico $P_A(\lambda) = \lambda^2 + 1$.

        \item ¿Puede ser que sea no sea diagonalizable en $\bb{R}$ ni en $\bb{C}$?\\
        Sí, y como ejemplo tenemos la matriz $A=\left(\begin{array}{cc}
            0 & 0 \\
            1 & 0
        \end{array} \right)$, con polinomio característico $P_A(\lambda) = \lambda^2$
    \end{enumerate}
\end{ejercicio}

\begin{ejercicio}
    Sea $A\in \mathcal{M}_n(\bb{K})$ cuyos valores propios son $\{-1,1,2,-2\}$ con multiplicidad algebraica cualquiera. Calcular $$rg(A+3I).$$

    $\lambda_0$ valor propio $\Longrightarrow |A-\lambda_0I| = 0 \Longrightarrow (A-\lambda_0 I)$ no es regular.
    
    Como el $-3$ no es un valor propio, entonces la matriz $(A+3I)$ es regular y, por tanto,
    $$rg(A+3I) = n$$
\end{ejercicio}

\begin{ejercicio}
    Sea $A\in \mathcal{M}_n(\bb{R})$.
    $$A=\left( \begin{array}{cccc}
        1+a & 1 & \dots & 1 \\
        1 & 1+a & \dots & 1 \\
        \vdots & \vdots & \ddots & \dots \\
        1 & 1 & \dots & 1+a \\
    \end{array}\right)$$
    Estudiar para qué valores de $a\in \bb{R}$ la matriz $A$ es diagonalizable.\\

    Sea $B=\left( \begin{array}{cccc}
        1 & 1 & \dots & 1 \\
        1 & 1 & \dots & 1 \\
        \vdots & \vdots & \ddots & \vdots \\
        1 & 1 & \dots & 1 \\
    \end{array}\right)$.
    
    Por tanto, sea $p(x)=x+a$. Como $A=B+aI=p(B)$, y $B$ es diagonalizable (ver Ej. \ref{Ej:TodosUnos}) $\Longrightarrow A$ es diagonalizable $\forall a\in \bb{R}$.
\end{ejercicio}

\begin{ejercicio}
    Sea $f\in End(\bb{V}^3(\bb{R}))$ que en cierta base tiene como matriz asociada
    $$A=\left(\begin{array}{ccc}
        1+a & 1+a & 1 \\
        -a & -a & -1 \\
        a & a-1 & 0
    \end{array}\right)$$
    Estudiar para qué valores de $a\in \bb{R}$, el endomorfismo $f$ es diagonalizable.
    \begin{multline*}
        P_f(\lambda) = |A-\lambda I|
        = \left|\begin{array}{ccc}
            1+a-\lambda & 1+a & 1 \\
            -a & -a-\lambda & -1 \\
            a & a-1 & -\lambda
        \end{array}\right|
        = \left|\begin{array}{ccc}
            1-\lambda & 1-\lambda & 0 \\
            -a & -a-\lambda & -1 \\
            a & a-1 & -\lambda
        \end{array}\right| =\\
        = \left|\begin{array}{ccc}
            1-\lambda & 0 & 0 \\
            -a & -\lambda & -1 \\
            a & -1 & -\lambda
        \end{array}\right| = (1-\lambda)(\lambda^2 - 1) = (1-\lambda)(\lambda+1)(\lambda-1) = -(\lambda+1)(\lambda-1)^2
    \end{multline*}
    Por tanto, los valores propios de $f$ son: $\{1, -1\}$.

    \begin{equation*}
        rg(A-I) = rg\left(\begin{array}{ccc}
            a & 1+a & 1 \\
            -a & -a-1 & -1 \\
            a & a-1 & -1
        \end{array}\right) = rg\left(\begin{array}{ccc}
            a & 1+a & 1 \\
            0 & 0 & 0 \\
            0 & 2 & 2
        \end{array}\right) = \left\{
        \begin{array}{cc}
            2 & \text{ si } a\neq 0 \\
            1 & \text{ si } a= 0
        \end{array}
        \right.
    \end{equation*}

    Por tanto, $\dim V_1 = 3 - rg(A-I) = \left\{
        \begin{array}{cc}
            1 & \text{ si } a\neq 0 \\
            2 & \text{ si } a= 0
        \end{array}
        \right.$
    \begin{table}[H]
        \centering
        \begin{tabular}{c|c|c}
            Valores Propios & Mult. Alg. & Mult. Geom. \\ \hline 
            $-1$ & 1 & 1\\
            1 & 2 & $\left\{
        \begin{array}{cc}
            1 & \text{ si } a\neq 0 \\
            2 & \text{ si } a= 0
        \end{array}
        \right.$\\
        \end{tabular}
        \caption{Valores propios con sus multiplicidades}
    \end{table}

    Por tanto, se puede ver que $f$ solo es diagonalizable si $a=0$.
\end{ejercicio}

\begin{ejercicio}
    Sea $A\in \mathcal{M}_n(\bb{K})$. Demostrar que $\exists p(\lambda)\in \bb{K}[\lambda] \mid p(A)=A^{-1}$.
    \begin{proof}
        Usando el Teorema de Cayley-Hamilton:
        \begin{equation*}
            P_A(A) = 0 = (-1)^n A^n + (-1)^{n-1}tr(A)A^{n-1} + \dots + c_1A + det(A)I
        \end{equation*}
    
        Como $A$ es regular, $\exists A^{-1}$. Multiplicando por $A^{-1}$,
        \begin{equation*}
            A^{-1}P_A(A) = 0 = (-1)^n A^{n-1} + (-1)^{n-1}tr(A)A^{n-2} + \dots + c_2A + c_1I + det(A)A^{-1}
        \end{equation*}
    
        Despejando $A^{-1}$,
        \begin{multline*}
            -det(A)A^{-1} = (-1)^n A^{n-1} + (-1)^{n-1}tr(A)A^{n-2} + \dots + c_2A + c_1I
            \Longrightarrow \\ \Longrightarrow
            A^{-1} = -\frac{(-1)^n A^{n-1} + (-1)^{n-1}tr(A)A^{n-2} + \dots + c_2A + c_1I}{det(A)}
        \end{multline*}
        donde he podido despejarla ya que $det(A)\neq 0$ por ser regular. Por tanto, he despejado el polinomio $A^{-1}$ en función de $A$, teniendo así el polinomio buscado.
    \end{proof}
\end{ejercicio}

\begin{ejercicio}
    Encontrar una matriz $A\in \mathcal{M}_n(\bb{R})$, siendo $n$ par, tal que no tenga ningún valor propio.\\

    La matriz $A_1=\left(\begin{array}{cc}
        0 & -1 \\
        1 & 0
    \end{array} \right)$ no tiene valores propios reales.

    La matriz $A_2=\left(\begin{array}{c|c}
        A_1 & 0 \\ \hline
        0 & A_1
    \end{array} \right)$ tampoco tiene valores propios reales, ya que su polinomio característico es $P_{A_2}(\lambda) = P_{A_1}(\lambda)P_{A_1}(\lambda)$, que tampoco tiene raíces reales. Por tanto, la solución es:
    \begin{equation*}
        A = \left(\begin{array}{c|c|c}
            A_1 & 0 & 0 \\ \hline
            0  & \ddots & 0 \\ \hline
            0 & 0 & A_1
        \end{array}\right)
    \end{equation*}
    ya que $P_{A}(\lambda) = P_{A_1}(\lambda)\dots P_{A_1}(\lambda)$.
\end{ejercicio}

\begin{comment}
\begin{ejercicio}
    Sea $V^3(\bb{R})$ un espacio vectorial. Denotemos por $\mathcal{B}=\{u_1, u_2, u_3\}$ una base de $V$. Sea $f\in End(V)$ del que se sabe:
    \begin{itemize}
        \item $f$ transforma el vector $6u_1 + 2u_2 + 5u_3$ en sí mismo.

        \item $U=\{(x_1, x_2, x_3)_\mathcal{B} \mid 2x_1 + 11x_2 -7x_3 = 0\}$ es un subespacio propio de $f$.

        \item La traza de $f$ es $5$.
    \end{itemize}
    Hallar los valores propios de $f$ y calcular $M(f; \mathcal{B})$.\\

    Como $f((6,2,5)_\mathcal{B}) = (6,2,5)_\mathcal{B} \Longrightarrow \lambda_0=1$ es un valor propio de $f$.

    \begin{equation*}
        U = \mathcal{L}\left(\left\{
        \left(\begin{array}{ccc}
            11 \\ -2 \\ 0
        \end{array} \right),
        \left(\begin{array}{ccc}
            0 \\ 7 \\ 11
        \end{array} \right)
        \right\}\right) \Longrightarrow
        \left\{ \begin{array}{c}
             f((11,-2,0)_\mathcal{B}) = \lambda_0(11,-2,0)_\mathcal{B} \\
             f((0,7,11)_\mathcal{B}) = \lambda_0(0,7,11)_\mathcal{B}
        \end{array}\right.
    \end{equation*}

    Por tanto, las ecuaciones quedan:
    \begin{equation*}
        \left\{\begin{array}{c}
            6f(u_1) + 2f(u_2) + 5f(u_3) = 6u_1 + 2u_2 + 5u_3 \\
            11f(u_1) - 2f(u_2) = \lambda_0(11u_1 - 2u_2)\\
            7f(u_2) + 11f(u_3) = \lambda_0(7u_2 + 11u_3)
        \end{array} \right.
    \end{equation*}
    
    \vspace{2cm} TERMINAR \vspace{2cm}
\end{ejercicio}
\end{comment}

\begin{ejercicio}
    Sea $A,C \in \mathcal{M}_2(\bb{\bb{K}})$ tal que $A=AC-CA$. Demostrar que $A^2=0_2$.\\

    Veamos en primer lugar que $tr(A)=0$:
    $$tr(A) = tr(AC-CA) = tr(AC) - tr(CA) = 0$$
    
    Por el Teorema de Cayley-Hamilton:
    $$P_A(A) = 0 = A^2 -\cancel{tr(A)A} + det(A)I = A^2 + det(A)I \Longrightarrow A^2 = -det(A)I$$

    Además, tenemos:
    \begin{gather}
        \label{Ej41_Ec1} A^2 = A(AC-CA) = A^2C - ACA \\
        \label{Ej41_Ec2} A^2 = (AC-CA)A = ACA - CA^2
    \end{gather}

    Sumando las ecuaciones \ref{Ej41_Ec1} y \ref{Ej41_Ec2},
    $$2A^2 = A^2C- CA^2 = -det(A)IC +Cdet(A)I = 0 \Longrightarrow A^2 = 0_2$$
\end{ejercicio}

\begin{ejercicio}
    Sea $A\in \mathcal{M}_2(\bb{R}) \mid A^3=0_2$. Demostrar que $A^2=0$.\\

    Por el Teorema de Cayley-Hamilton,
    \begin{equation*}
        P_A(A) = 0 = A^2 - tr(A)A + det(A)I \Longrightarrow \cancel{A^3} - tr(A)A^2 + det(A)A = 0 \Longrightarrow A^2 = A\frac{det(A)}{tr(A)}
    \end{equation*}

    Veamos el valor de $det(A):$
    \begin{equation*}
        0 = |A^3| = |A|^3 \Longrightarrow |A|=0
    \end{equation*}

    Por tanto, $A^2 = A\cdot 0 = 0$.
\end{ejercicio}


\begin{ejercicio}
    Sean $A,P,Q \in \mathcal{M}_2(\bb{R})$ tal que
    $$P^{-1}AP = Q^{-1}AQ = \left( \begin{array}{cc}
        2 & 0 \\
        0 & \sqrt{3}
    \end{array} \right)$$
    Demostrar que entonces las columnas de $P$ son proporcionales a las columnas de $Q$.
    \begin{comment}
    \begin{proof}
        Sea $X=P^{-1}AP$, $Y=Q^{-1}AQ$ y $D=\left( \begin{array}{cc}
            2 & 0 \\
            0 & \sqrt{3}
        \end{array} \right) = X=Y$
        
        Usando el isomorfismo entre endomorfismos y matrices cuadradas, sea $f\in~End(V^2)$ tal que $M(f, \mathcal{B}) = A$. Como $A\sim X$ y $A\sim  Y$, las tres matrices representan al mismo endomorfismo pero en distintas bases.
        $$X=M(f; \mathcal{B}_1) \qquad Y=M(f; \mathcal{B}_2) \qquad A=M(f; \mathcal{B}_3)$$

        Debido a la matriz $D$, sabemos que $V_2 \oplus V_{\sqrt{3}} = V^2(\bb{R})$. Los valores propios del endomorfismo con multiplicidad algebraica y geométrica 1 son: $\{2,  \sqrt{3}\}$. Como $X=Y=D$, sabemos que $\mathcal{B}_1$ y $\mathcal{B}_2$ son bases de vectores propios.

        Como la multiplicidad geométrica es $1$, sea $\{e_1\}$ base de $V_2$ y $\{e_2\}$ base de $V_{\sqrt{3}}$. Debido a la suma directa, sea $\mathcal{B}_1 =~\{e_1, e_2\}$.
        
        \vspace{5cm}
        Sea ahora $\mathcal{B}_2 = \{\bar{e_1}, \bar{e_1}\}$, con $\bar{e_1} \in V_2,\;\bar{e_1}\in V_{\sqrt{3}}$.
        \begin{itemize}
            \item Como $\bar{e_1} \in V_2 \Longrightarrow \bar{e_1} = ae_1$, con $a\neq 0$.
            \item Como $\bar{e_2} \in V_{\sqrt{3}} \Longrightarrow \bar{e_2} = be_2$, con $b\neq 0$.
        \end{itemize}
        Por tanto, los vectores de $\mathcal{B}_2$ son proporcionales a los vectores de $\mathcal{B}_1$. Veamos ahora que las columnas de $P$ y $Q$ también son proporcionales:
        \begin{itemize}
            \item $P=M(\mathcal{B}_1; \mathcal{B}_3) \Longrightarrow P$ tiene como columnas los vectores de $\mathcal{B}_1$ en la base $\mathcal{B}_3$.

            \item $Q=M(\mathcal{B}_2; \mathcal{B}_3) \Longrightarrow Q$ tiene como columnas los vectores de $\mathcal{B}_2$  en la base $\mathcal{B}_3$.
        \end{itemize}
        
        Por tanto, como los vectores de $\mathcal{B}_2$ son proporcionales a los vectores de $\mathcal{B}_1$, las columnas de $P$ también son proporcionales a las columnas de $Q$.
    \end{proof}
    \end{comment}

\end{ejercicio}

\end{document}
