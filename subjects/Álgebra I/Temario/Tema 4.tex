\chapter{Dominios de factorización única}
\begin{definicion}[Divisor propio]
    Sea $A$ un DI y $a \in A$. Un divisor no trivial de $a$ será un \textbf{divisor propio de $a$}. Es decir, es
    $$b \in A \mid b \notin U(A) \y b|a \y a\not|b$$
\end{definicion}

\begin{definicion}[Dominio de Factoriazación Única]
    \ \\
    Sea $A$ un DI. Diremos que es un \textbf{dominio de factorización única} $($abreviado DFU$)$, si $\forall a \in A \mid a \neq 0 \y
        a \notin U(A)$ podemos expresarlo como:
    $$a=p_1p_2\ldots p_s \mid p_i \mbox{ es irreducible } ~\forall i \in \{1, \ldots, s\}$$
    Además, tal factorización es esencialmente única en el sentido de que si:
    $$a=q_1q_2\ldots q_r \mid q_i \mbox{ es irreducible } ~\forall i \in \{1, \ldots, r\}$$
    Es otra factorización en irreducibles de $a$, entonces tenemos que $r=s$ y que, reordenando si fuera necesario:
    $$p_i \sim q_i~~\forall i \in \{1, \ldots, s\}$$
\end{definicion}

\begin{ejemplo}
    Un ejemplo de que dos factorizaciones sean esencialmente iguales es, por ejemplo:
    $$-6 = -2 \cdot 3$$
    $$-6 = 2 \cdot (-3)$$
    $$2\sim -2 \y 3 \sim -3$$
\end{ejemplo}~\\

En cualquier dominio de integridad $A$, podemos elegir un conjunto $P \neq 0$ representativo de sus elementos irreducibles,
en el siguiente sentido:
\begin{enumerate}
    \item[1)] $\forall p \in P \Rightarrow p$ es irreducible.
    \item[2)] Si $q\in A$ irreducible $\Rightarrow \exists p \in P \mid p\sim q$.
    \item[3)] Si $p,p'\in P \mid p\neq p' \Rightarrow p\not\sim p'$.
\end{enumerate}
\begin{ejemplo}
    Ejemplos del conjunto $P$ son:
    $$\mbox{En } \Z~~P=\{p \in \Z \mid p \mbox{ es irreducible } \y p > 0\}$$
    $$\mbox{En } K[x]~~P \mbox{ es el conjunto de polinomios irreducibles mónicos }$$
\end{ejemplo}

Notemos que si $A$ es un DFU, todo elemento $a \in A \mid a\neq 0$ se expresa de forma esencialmente única como:
$$a=up_1^{e_1}p_2^{e_2}\ldots p_r^{e_r}$$
Donde $u \in U(A)$, $r\geq 0$\newline
$p_i \in P~\forall i \in \{1, \ldots, r\}, p_i\neq p_j~\forall i,j \in \{1, \ldots, r\}~~i \neq j$\newline
$e_i \in \Z \mid e_i \geq 1$

\begin{notacion}
$\forall a \in A$, $a \neq 0$ y $\forall p \in P$, denotaremos por:
$$e(p,a)$$
Al exponente del irreducible $p$ en la factorización de $a$, entendiendo que si $p$ no aparece en la factorización de $a$,
entonces $e(p,a)=0$. Por tanto, sea $A$ un DFU, notaremos $\forall a \in A$:
$$a = u \prod_{p \in P} p^{e(p,a)}~~~~u \in U(A)$$
\end{notacion}

\begin{lema}
    Sea $A$ un DFU con $a, b \in A$ tales que $a\neq 0\neq b$ y $p \in P$:
    $$e(p,ab) = e(p,a) + e(p,b)$$
\begin{proof}
    Como $A$ es un DFU, existen $u,v\in U(A)$ de forma que:
    \begin{equation*}
        a = u \prod_{p\in P} p^{e(p,a)} \qquad 
        b = v \prod_{p\in P} p^{e(p,b)} 
    \end{equation*}
    Si realizamos el producto de ambos:
    \begin{align*}
        ab &= \left(u \prod_{p\in P} p^{e(p,a)}\right) \left(v \prod_{p\in P} p^{e(p,b)} \right) = uv \left(\prod_{p\in P} p^{e(p,a)}\right)  \left(\prod_{p\in P} p^{e(p,b)}\right) \\
           &= uv \prod_{p\in P} \left(p^{e(p,a)} p^{e(p,b)}\right) = uv \prod_{p\in P} p^{e(p,a) + e(p,b)}
    \end{align*}
    Como también $ab\in A$, existirá $w\in U(A)$ de forma que:
    \begin{equation*}
        ab = w\prod_{p\in P} p^{e(p,ab)}
    \end{equation*}
    Hemos encontrado dos factorizaciones de $ab$ en irreducibles, por lo que han de ser iguales, es decir, $w = uv$ y:
    \begin{equation*}
        e(p,a) + e(p,b) = e(p,ab) \qquad \forall p\in P
    \end{equation*}
\end{proof}
\end{lema}

\begin{lema}
    Sea $A$ un DFU con $a, c \in A$ tales que $a, b, c \neq 0$:
    $$a|c \Leftrightarrow e(p,a) \leq e(p,c)~~\forall p \in P$$
\begin{proof}
   \begin{description}
       \item [$\Longrightarrow)$] Si $a\mid c$, entonces $\exists c'\in A$ de forma que $c = ac'$. Aplicando el Lema anterior, tenemos que:
           \begin{equation*}
               e(p,c) = e(p,a) + e(p,c') \qquad \forall p\in P
           \end{equation*}
           Como $e(p,c') \geq 0$ para todo $p\in P$, tenemos que $e(p,c) \geq e(p,a)$.
       \item [$\Longleftarrow)$] Si consideramos las descomposiciones de $a$ y $c$, $\exists u,v\in U(A)$ de forma que:
           \begin{equation*}
               a = u\prod_{p\in P} p^{e(p,a)}, \qquad c = v\prod_{p\in P}p^{e(p,c)}
           \end{equation*}
           Y podemos escribir:
           \begin{align*}
               c &= v\prod_{p\in P}p^{e(p,c)}= v\prod_{p\in P}p^{e(p,c)-e(p,a)+e(p,a)} \\
                 &= v\prod_{p\in P}p^{e(p,c)-e(p,a)}p^{e(p,a)} = v\prod_{p\in P}\left(p^{e(p,c)-e(p,a)}\right) \prod_{p\in P}(p^{e(p,a)}) \\
                 &= vu^{-1} \prod_{p\in P}\left(p^{e(p,c)-e(p,a)}\right) u\prod_{p\in P}\left(p^{e(p,a)}\right) =\left[ vu^{-1} \prod_{p\in P}\left(p^{e(p,c)-e(p,a)}\right)\right] a
           \end{align*}
   \end{description}
\end{proof}
\end{lema}

\begin{prop}
    \label{propExistenMCDDFU}
    Sea $A$ un DFU y $a, b \in A$ tales que:
    $$a=u\prod_{p \in P} p^{e(p,a)}~~~~b=v\prod_{p\in P} p^{e(p,b)}~~u,v \in U(A)$$
    Entonces, $\exists mcd(a,b), mcm(a,b)$, dados por:
    $$mcd(a,b) = \prod_{p \in P} p^{\min\{e(p,a),e(p,b)\}}$$
    $$mcm(a,b) = \prod_{p \in P} p^{\max\{e(p,a),e(p,b)\}}$$
\begin{proof}
    Si llamamos:
    \begin{equation*}
        \alpha = \prod_{p \in P} p^{\min\{e(p,a),e(p,b)\}}, \qquad \beta = \prod_{p \in P} p^{\max\{e(p,a),e(p,b)\}}
    \end{equation*}
    Veamos que $\alpha = mcd(a,b)$  y que $\beta = mcm(a,b)$:
    \begin{itemize}
        \item Para el máximo común divisor, hemos de ver que $\alpha$ cumple dos propiedades:
            \begin{itemize}
                \item $\alpha \mid a$ y $\alpha \mid b$. Por la forma de definir $\alpha$, tenemos que:
                    \begin{equation*}
                        e(p,\alpha) = \min\{e(p,a),e(p,b)\} \leq e(p,a), e(p,b) \qquad \forall p\in P
                    \end{equation*}
                    Por lo que, por el Lema anterior, $\alpha \mid a $ y $\alpha \mid b$.
                \item Para la segunda propiedad, hemos de probar que si $t\in A$ verifica que $t\mid a$ y $t \mid b$, entonces $t\mid \alpha$. Para ello, si $t\mid a$ y $t\mid b$, se cumplirá por el Lema anterior que:
                    \begin{equation*}
                        e(p,t) \leq e(p,a), e(p,b) \qquad \forall p\in P
                    \end{equation*}
                    Por lo que tendremos que:
                    \begin{equation*}
                        e(p,t) \leq \min\{e(p,a), e(p,b)\} = e(p,\alpha) \qquad \forall p\in P
                    \end{equation*}
                    De donde $t \mid \alpha$.
            \end{itemize}
            De esta forma, $\alpha = mcd(a,b)$.
        \item Para el mínimo común múltiplo, la demostración es análoga:
            \begin{itemize}
                \item Hemos de ver que $a \mid \beta$ y que $b \mid \beta$. Por la forma de definir $\beta$:
                    \begin{equation*}
                        e(p,a), e(p,b) \leq \max\{e(p,a),e(p,b)\} = e(p,\beta)
                    \end{equation*}
                    Por lo que $a \mid \beta$ y $b\mid \beta$.
                \item Para la segunda propiedad, hemos de probar que si $t\in A$ verifica que $a\mid t$ y $b\mid t$, entonces $\beta \mid t$. Para ello, si $a \mid t$ y $b\mid t$, por el Lema anterior tenemos que:
                    \begin{equation*}
                        e(p,a),e(p,b) \leq e(p,t)\qquad \forall p\in P
                    \end{equation*}
                    Por lo que tendremos que:
                    \begin{equation*}
                        e(p,\beta) = \max\{e(p,a),e(p,b)\} \leq e(p,t) \qquad \forall p\in P
                    \end{equation*}
                    De donde $\beta \mid t$.
            \end{itemize}
            En definitiva, $\beta = mcm(a,b)$.
    \end{itemize}
\end{proof}
\end{prop}

Recordamos ahora la definición de número primo, Definición \ref{defPrimo}.

\begin{prop}
    \label{propPrimoIrreducible}
    Sea $A$ un DI:
    \begin{enumerate}
        \item[1)] $\forall p \in A$ primo $\Rightarrow p$ es irreducible.
        \item[2)] Si $A$ es además un DFU y $p \in A$ irreducible $\Rightarrow p$ es primo.
    \end{enumerate}
\begin{proof}
    \ \\
    1) Sea $p \in A$ primo.\newline
    Supongamos que $a \in A \mid a|p \Rightarrow \exists b \in A \mid p=ab \Rightarrow p|ab \Rightarrow p|a \o p|b$.\par
    Si $p|a \Rightarrow p\sim a$.\par
    Si $p|b \Rightarrow \exists c \in A \mid b=pc =abc \Rightarrow 1=ac \Rightarrow a \in U(A)$.\newline
    Por lo que los únicos divisores de $p$ son los triviales $\Rightarrow p$ es irreducible.\\

    
    2) Sea $A$ un DFU y $p \in A \mid p$ irreducible $($Luego $p\neq 0 \y p \notin U(A))$.\newline
    Sean $a,b \in A \mid p|ab \Rightarrow e(p,p)=1\leq e(p,ab) = e(p,a)+e(p,b)$. Entonces:
    $$\left. \begin{array}{lll}
            e(p,a) \geq 1 & \Rightarrow & p|a \\
            \o            &             &     \\
            e(p,b) \geq 1 & \Rightarrow & p|b
        \end{array} \right\} \Rightarrow p \mbox{ primo}$$
\end{proof}
\end{prop}


\begin{coro}
    de la Proposición \ref{propPrimoIrreducible}.\newline
    Sea $A$ un DFU y $p \in A$:
    $$p \mbox{ es primo } \Leftrightarrow p \mbox{ es irreducible}$$
\end{coro}

\begin{teo}[Caracterización de los DFUs]
    \ \\
    \label{teoCaracDFU}
    Sea $A$ un DI, son equivalentes:\newline
    i) $A$ es un DFU.\newline
    ii) Se verifica en $A$:
    $$\forall a \in A \mid a\neq 0 \y a \notin U(A) \mbox{ se expresa como producto de irreducibles}$$
    $$\exists mcd(a,b) \y \exists mcm(a,b)~~\forall a,b \in A$$
    iii) Se verifica en $A$:
    $$\forall a \in A \mid a\neq 0 \y a \notin U(A) \mbox{ se expresa como producto de irreducibles}$$
    $$\forall p \in A \mbox{ irreducible } \Rightarrow p \mbox{ primo}$$
\begin{proof}
    \begin{description}
        \item [$i) \Longrightarrow ii)$] Si $A$ es un DFU, entonces todo elemento se factoriza como producto de irreducibles. Además, anteriormente probamos que en un DFU siempre existen el máximo común divisor y mínimo común múltiplo de dos elementos.
        \item [$ii) \Longrightarrow iii)$] Basta probar que si siempre existe el máximo común divisor de dos elementos, entonces todo irreducible es primo. Para ello, supongamos que siempre existe el máximo común divisor de dos elementos y, por reducción al absurdo, supongamos que $p$ es un irreducible que no es primo, por lo que existen $a,b\in A$ de forma que $p\mid ab$, $p\nmid a$ y $p\nmid b$. En dicho caso, sea $\alpha = mcd(a,b)$, como $\alpha \mid a$ y $\alpha \mid b$, los divisores de $p$ son los triviales y ningún asociado a $p$ puede dividir a $a$ y a $b$, concluimos que:
            \begin{equation*}
                mcd(a,p) = 1 = mcd(b,p)
            \end{equation*}
            Como teníamos también que $p \mid ab$, tenemos que $mcd(ab,p) = p$. Utilizando la propiedad 7 del máximo común divisor, tenemos que $p\mid ab$ y $mcd(p,a) = 1$, luego tenemos que $p\mid b$, lo cual es una contradicción.

            Luego todo irreducible es primo.
        \item [$iii) \Longrightarrow i)$] Hemos de probar que si todo irreducible es primo, entonces si dos productos de irreducibles son iguales, entonces son asociados dos a dos; es decir, si:
            \begin{equation*}
                p_1p_2\ldots p_n = q_1q_2 \ldots q_m 
            \end{equation*}
            Siendo todos ellos irreducibles, entonces $n = m$ y $q_i \sim p_{\phi(i)}$ $\forall i \in \{1,\ldots,n\}$, siendo $\phi$ una permutación de $n$ elementos. Lo demostraremos por inducción sobre $n$:
            \begin{itemize}
                \item Si $n = 1$, tendremos que $p = q_1q_2 \ldots q_m$. En dicho caso, $p \mid q_1q_2 \ldots q_m$, y por ser $p$ primo, ha de existir $j \in \{1,\ldots,m\}$ de forma que $p \mid q_j$. En dicho caso, podemos dividir a ambos lados por $p$, obteniendo que\footnote{$\hat{q}_j$ denota que el elemento $j-$ésimo no aparece en el producto.}:
                    \begin{equation*}
                        1 = q_1q_2 \ldots \hat{q}_j \ldots q_m
                    \end{equation*}
                    Lo que implica que $m = 1 = n$, ya que ningún producto de irreducibles puede ser una unidad, tenemos que $p \sim q_j$.
                \item Supuesto cierto para $n-1$, si tenemos que:
                \begin{equation*}
                    p_1p_2\ldots p_n = q_1q_2 \ldots q_m 
                \end{equation*}
                Como $p_n$ es primo y $p_n \mid q_1q_2 \ldots q_m$, al igual que antes, debe existir $j \in \{1,\ldots,m\}$ de forma que $p_n \mid q_j$ y como $q_j$ es irreducible, $p_n \sim q_j$, con lo que $\exists u\in U(A)$ de forma que $p_n = uq_j$. Si dividimos a ambos lados por $p_n$:
                \begin{equation*}
                    {(p_n)}^{-1}p_1p_2 \ldots p_n = p_1p_2\ldots p_{n-1} = {(uq_j)}^{-1} (q_1 q_2 \ldots q_m) = u^{-1}(q_1 \ldots \hat{q}_j \ldots q_m)
                \end{equation*}
                Como multiplicar por una unidad no cambia la factorización en irreducibles, podemos aplicar la hipótesis de inducción, obteniendo que $m-1 = n-1$ y que $q_i \sim p_{\phi(i)}$ para $i \in \{1,\ldots,n-1\}$, con $\phi$ una permutación de $n-1$ elementos. En definitiva, tenemos que $m = n$ y definiendo $\phi(n) = j$, tenemos que $\phi$ es una permutación de $n$ elementos, con $q_i \sim p_{\phi(i)}$, $\forall i \in \{1,\ldots,n\}$.
            \end{itemize}
    \end{description}
\end{proof}
\end{teo}

\begin{lema}
    Sea $A$ un DE con función euclídea $\phi$, $a,b \in A$ tal que $a$ es un divisor propio de $b$. Entonces:
    $$\phi(a) < \phi(b)$$
\begin{proof}
    Sean $a,b\in A \setminus U(A)$ con $a \neq 0 \neq b$, por reducción al absurdo, supongamos que $\phi(a) = \phi(ab)$. Si dividimos $a$ entre $ab$, obtenemos que $\exists q,r\in A$ de forma que:
    \begin{equation*}
        a = abq + r
    \end{equation*}
    con $r = 0$ o $\phi(r) < \phi(ab)$.
    \begin{itemize}
        \item Si $r = 0$, entonces $a = abq$ y por la propiedad cancelativa, $1 = bq$, luego $b \in U(A)$, \underline{contradicción}.
        \item Tendremos por tanto que $\phi(r) < \phi(ab)$. En dicho caso, si despejamos $r$ de la igualdad superior: $r = a(1-bq)$. Como $1-bq\neq 0$ (ya que si no $b\in U(A)$), tenemos que $\phi(a(1-bq)) = \phi(r)$. En definitiva:
            \begin{equation*}
                \phi(r) = \phi(a(1-bq)) \geq \phi(a) = \phi(ab)
            \end{equation*}
            Y teníamos que $\phi(r) < \phi(ab)$. Hemos llegado a una \underline{contradicción}.
    \end{itemize}
    En definitiva, no puede ser $\phi(a) = \phi(ab)$, y como $\phi(ab) \geq \phi(a)$ por ser $\phi$ una función euclídea, tenemos que $\phi(a) < \phi(ab)$.
\end{proof}
\end{lema}

\begin{teo}
    \label{teoDEEntoncesDFU}
    Sea $A$ un DE $\Rightarrow A$ es un DFU.
\begin{proof}
    Sea $A$ un DE con función euclídea $\phi$:\newline
    Sabemos que por ser $A$ DE $\Rightarrow \exists mcd(a,b) \y \exists mcm(a,b)~~\forall a,b \in A$.\newline
    Veamos que $\forall a \in A \mid a \neq 0 \y a \notin U(A)$ podemos expresar $a$ como producto de irreducibles, por lo
    que según el Teorema \ref{teoCaracDFU}, $A$ sería un DFU:\\

    
    Supongamos que no: $\exists a \in A \mid a\neq 0 \y a \notin U(A)$ que no puede expresarse como producto de irreducibles,
    por lo que sabemos que no es irreducible $\Rightarrow \exists a_1 \in A \mid a_1$ es divisor propio de $a$.\newline
    Por tanto, $\exists b_1 \in A \mid a=a_1b_1$ y $a_1 \o b_1$ no puede expresarse como producto de irreducibles.\\

    
    Supongamos (lo que no nos hace perder generalidad) que es $a_1$ el que no puede expresarse como producto de irreducibles.
    Por lo que $a_1$ no es irreducible y sbemos que $\exists a_2 \in A$ divisor propio de $a_1$ que no puede expresarse como
    producto de irreducibles.\\

    
    Continuando con este razonamiento, llegamos a una sucesión no finita :
    $$a_0 = a, a_1, a_2, \ldots, a_n, \ldots \mid a_i \mbox{ es divisor propio de } a_{i-1}~~\forall i \geq 1$$
    Siendo ninguno de ellos producto de irreducibles y:
    $$\phi(a_0) > \phi(a_1) > \phi(a_2) > \ldots > \phi(a_n) > \ldots$$
    Una sucesión estrictamente decreciente de números naturales. \underline{Contradicción}, luego $\forall a \in A$, $a$
    puede expresarse como producto de irreducibles, haciendo que $A$ sea un DFU.
\end{proof}
\end{teo}

\begin{coro}[Teorema fundamental de la aritmética]
    El anillo $\Z$ es un DFU.
\begin{proof}
    Por la Proposición \ref{prop:ZDE}, sabemos que $\Z$ es un DE y ahora, por el Teorema \ref{teoDEEntoncesDFU}, sabemos
    que $\Z$ es un DFU.
\end{proof}
\end{coro}

\begin{coro}
    Sea $K$ un cuerpo. Entonces $K[x]$ es un DFU.
\begin{proof}
    Por el Teorema \ref{teo:DividirPolinomios}, sabemos que $K[x]$ es un DE y ahora, por el Teorema \ref{teoDEEntoncesDFU}, sabemos
    que $K[x]$ es un DFU.
\end{proof}
\end{coro}

\begin{coro}
    Sea $n \in \{-2, -1, 2, 3\}$. Entonces $\Z[\sqrt{n}]$ es un DFU.
\begin{proof}
    Por el Teorema \ref{teo:DEZRaizn}, sabemos que $\Z[\sqrt{n}]$ es un DU y ahora, por el Teorema \ref{teoDEEntoncesDFU}, sabemos
    que $\Z[\sqrt{n}]$ es un DFU.
\end{proof}
\end{coro}

Veremos también que hay anillos que son DFU y no son DE:
\begin{prop}
    $\Z[x]$ no es DE.
\begin{proof}
    \ \\
    Para ello, podemos ver que $\exists I \subseteq \Z[x]$ ideal $\mid I\neq m\Z[x]~~\forall m \in \Z[x]$:\newline
    Sea $\displaystyle I = \left\{p = \sum_{i=0}^n a_i x^i \in \Z[x] \mid a_0 \in 2\Z \right\}$ es un ideal de $\Z[x]$:
    $$\forall f = \sum_{i=0}^n a_i x^i, ~g=\sum_{j=0}^m b_j x^j \in I \Rightarrow a_0, b_0 \in 2\Z$$
    $$s=f+g = \sum_{i=0}^{\max\{n,m\}}(a_i+b_i) x^i \Rightarrow s_0 = a_0+b_0 \in 2\Z \Rightarrow s \in I$$
    $$p = fg = \sum_{k=0}^{n+m} k_i x^i \mid k_i = \sum_{i+j=k} a_i b_j \Rightarrow k_0 = a_0b_0 \in 2\Z \Rightarrow p \in I$$
    Veamos ahora que $I$ no es principal. Si lo fuera, $\exists f \in \Z[x] \mid I = f\Z[x]$:
    $$2 \in I=f\Z[x] \Rightarrow \exists g \in \Z[x] \mid fg = 2$$
    $$0 = grd(2) = grd(fg) = grd(f) + grd(g) \Rightarrow grd(f) = 0 = grd(g)$$
    Luego $f = a \y g=b$ con $a,b \in \Z$ y:
    $$2=ab \Rightarrow \left\{\begin{array}{lll}
            a = \pm 2 & \y & b = \pm 1 \\
            \o        &    &           \\
            a = \pm 1 & \y & b = \pm 2
        \end{array} \right.$$

    $$I = f\Z[x] = a\Z[x]$$
    $$\mbox{Si } a=\pm 2 \Rightarrow x \notin I \mbox{ \underline{Contradicción}}$$
    $$\mbox{Si } a=\pm 1 \Rightarrow I=\pm\Z[x] = \Z[x] \Rightarrow 1 \in I \mbox{ \underline{Contradicción}}$$
    Luego $\nexists f \in \Z[x] \mid I=f\Z[x] \Rightarrow I$ no es principal $\Rightarrow \Z[x]$ no es DE.
\end{proof}
\end{prop}

Veremos más tarde $($en el Teorema \ref{teoGauss}$)$ que $\Z[x]$ es un DFU.

\newpage
\section{Irreducibles y primos en el anillo de enteros cuadráticos}

\begin{lema}
    Sean $\alpha, \beta \in \Z[\sqrt{n}]$ tales que $\alpha$ es divisor propio de $\beta$ en $\Z[\sqrt{n}]$. Entonces:
    $$N(\alpha) \mbox{ es divisor propio de } N(\beta) \mbox{ en } \Z$$
\begin{proof}
    Sea $\alpha$ divisor propio de $\beta$:\newline
    Sabemos por tanto que $\alpha \notin U(\Z[\sqrt{n}]) \Rightarrow N(\alpha) \neq \pm 1 \Rightarrow N(\alpha) \notin U(\Z)$.
    \newline
    $$\exists \gamma \in \Z[\sqrt{n}] \mid \gamma \notin U(\Z[\sqrt{n}]) \y \beta = \alpha \cdot \gamma \Rightarrow
        N(\beta) = N(\alpha \cdot \gamma) = N(\alpha)N(\gamma)$$
    Como $\alpha \notin U(\Z[\sqrt{n}]) \Rightarrow N(\alpha) \notin U(\Z)$.\newline
    Luego $N(\alpha)|N(\beta)$ y es un divisor propio suyo.
\end{proof}
\end{lema}


\begin{teo}
    En $\Z$ hay infinitos números primos.
\begin{proof}
    Supongamos que en $\Z$ hay $n$ primos positivos: $\{p_1, p_2, \ldots, p_n\}$.\newline
    Sea $a=p_1p_2\ldots p_n +1 \Rightarrow a \in \Z \y \Z$ es DFU $\Rightarrow a$ tiene factorización en irreducibles:
    $$\exists i \in \{1, \ldots, n\} \mid p_i|a=p_1p_2\ldots p_n+1 \Rightarrow p_i|1 \Rightarrow p_i \in U(\Z) \Rightarrow
        p_i = \pm 1$$
    \underline{Contradicción} con que era irreducible. Luego en $\Z$ hay una cantidad no finita de números primos.
\end{proof}
\end{teo}

\begin{prop}
    Sea $n \in \Z \mid \sqrt{n} \notin \Z$ y $\alpha \in \Z[\sqrt{n}]$. Entonces:\newline
    1) Si $N(\alpha) = \pm p$ primo de $\Z \Rightarrow p$ es irreducible.\newline
    2) Si $\alpha$ es primo $\Rightarrow N(\alpha) \in \{\pm p, \pm p^2\}$ con $p \in \Z$ primo.\par
    Si $N(\alpha)=\pm p^2 \Rightarrow \alpha \sim p$, son asociados en $\Z[\sqrt{n}]$.
\begin{proof}
    \ \\
    % 1) %// TODO: this.\ \\

    
    2) $\alpha$ primo $\Rightarrow \alpha \neq 0 \y \alpha \notin U(\Z[\sqrt{n}]) \Rightarrow N(\alpha) \notin \{0, \pm 1\}$.
    $$\left. \begin{array}{l}
            N(\alpha) = \alpha \cdot \overline{\alpha} \in \Z \\
            \Z \mbox{ es un DFU}
        \end{array} \right\} \Rightarrow \alpha \cdot \overline{\alpha} = p_1\ldots p_k \mid p_i \in P~~\forall i
        \in \{1, \ldots, k\}$$
    En $\Z[\sqrt{n}]$:
    $$\alpha|\alpha \cdot \overline{\alpha} \Rightarrow \alpha|p_1\ldots p_k$$
    $\alpha$ es primo $\Rightarrow \exists i \in \{1, \ldots, k\} \mid \alpha|p_i$. Sea $p = p_i$:\\

    
    Hemos encontrado un primo $p \in \Z$ tal que $\alpha|p$ en $\Z[\sqrt{n}] \Rightarrow \exists \beta \in \Z[\sqrt{n}] \mid
        p=\alpha \beta$.\newline
    $$N(p) = p^2 = N(\alpha\cdot \beta) = N(\alpha)N(\beta) \Rightarrow N(\alpha)|p^2 \y N(\alpha)\neq \pm 1 \Rightarrow
        N(\alpha) \in \{\pm p, \pm p^2\}$$
    \ \\

    
    Supongamos que $N(\alpha) = \pm p^2$:
    $$p^2 = N(\alpha)N(\beta) = \pm p^2 N(\beta) \Rightarrow N(\beta) = \pm 1 \Rightarrow \beta \in U(\Z[\sqrt{n}]) \Rightarrow$$
    $$\Rightarrow p = \alpha \cdot \beta \mbox{ es asociado a } \alpha$$
\end{proof}
\end{prop}

\begin{lema}
    Sea $\alpha = a+bi \in \Z[i] \mid a,b \neq 0$. Entonces:
    $$\alpha \mbox{ es primo o irreducible } \Leftrightarrow N(\alpha) \mbox{ es un número primo en } \Z$$
% \begin{proof}
%     %// TODO: this
% \end{proof}
\end{lema}

Notemos que puede haber irreducibles cuya norma no sea un número primo o su opuesto:
En $\Z[\sqrt{-5}]$, $\alpha = 3$ es irreducible. Sin embargo, $N(3)=9$, que
no es primo en $\Z$. Además, notemos que:
$$\mbox{Si } N(\alpha) = p^2 \nRightarrow \alpha \mbox{ es irreducible}$$
En $Z[i]$, sea $\alpha = 2$, $N(\alpha)=4=2^2$
$$2=(1+i)(1-i) \Rightarrow \mbox{ no es irreducible}$$

\begin{ejemplo}
    \ \\
    1. Factorizar $\alpha = 11+7i$ en $\Z[i]$.\\

    
    $($Como $\Z[i]$ es DFU, podemos usar que $p \in \Z[i]$ es primo $\Leftrightarrow p$ es irreducible $)$.\newline
    $$N(\alpha) = 11^2 + 7^2 = 170 = 2 \cdot 5 \cdot 17$$
    $N(\alpha)$ no es primo ni cuadrado de primo $\Rightarrow \alpha$ no es irreducible.\newline
    Como ningún cuadrado de primo divide a $N(\alpha)$, buscamos elementos en $\Z[i]$ de norma $2$, $5$ ó $17$ que
    dividan a $\alpha$:
    $$a+bi \in \Z[i] \mid N(a+bi) = a^2 + b^2 = 2 \Leftrightarrow \left\{ \begin{array}{l}
            a = \pm 1 \\
            b = \pm 1
        \end{array} \right.$$
    Los elementos de norma $1$ son $1+i$ y sus asociados.
    $$\dfrac{11+7i}{1+i} = \dfrac{(11+7i)(1-i)}{2} = \dfrac{11-11i+7i+7}{2} = \dfrac{18-8i}{2} = 9-2i$$
    Luego $\alpha = (1+i)(9-2i)$ con $1+i$ irreducible por ser $N(1+i) = 2$, primo en $\Z$.\\

    
    $N(9-2i) = 5 \cdot 17$, buscamos elementos de norma 5:
    $$a+bi \in \Z[i] \mid N(a+bi) = a^2 + b^2 = 5 \Leftrightarrow \left\{ \begin{array}{lll}
            a = \pm 2 & \y & b = \pm 1 \\
            \o        &    &           \\
            a = \pm 1 & \y & b = \pm 2
        \end{array} \right.$$
    Los elementos de norma $5$ son $2+i$, $1+2i$ y sus asociados.
    $$\dfrac{9-2i}{2+i}= \dfrac{(9-2i)(2-i)}{5} = \dfrac{18-9i-4i-2}{5} = \dfrac{16}{5}-\dfrac{13}{5}i$$
    Luego $2+i\not|9-2i$ en $\Z[i]$.
    $$\dfrac{9-2i}{1+2i} = \dfrac{(9-2i)(1-2i)}{5} = \dfrac{5-20i}{5} = 1-4i$$
    Luego $9-2i = (1+2i)(1-4i)$ con $1+2i$, $1-4i$ irreducibles por ser $N(1+2i)=5$, $N(1-4i)=17$, primos.\\

    
    En definitiva, $\alpha = 11+7i = (1+i)(1+2i)(1-4i)$\\

    
    2) Sea $\beta = 2i \in \Z[i]$, calcular $mcd(\alpha, \beta), mcm(\alpha, \beta)$:
    $$\beta = 2i = (1+i)(1-i)i = (1+i)^2$$
    $$mcd(\alpha, \beta) = 1+i$$
    $$mcm(\alpha, \beta) = (1+i)^2(1+2i)(1-4i) = 18-4i$$
    Donde hemos aplicado la Proposición \ref{propExistenMCDDFU}.
\end{ejemplo}

\begin{ejemplo}
    Existen elementos irreducibles que no son primos:\newline
    Sea $\Z[\sqrt{-5}]$, consideramos $\alpha = 1+\sqrt{-5}$, $\alpha \neq 0 \y \alpha \notin U(\Z[\sqrt{-5}])$. Veamos
    que $\alpha$ es irreducible:\newline
    Sea $\beta \in \Z[\sqrt{-5}] \mid \beta|\alpha \Rightarrow N(\beta)|N(\alpha)=6=2\cdot 3$. Las opciones son:
    $$\left\{ \begin{array}{ll}
            N(\beta) = 1 & \Rightarrow \beta \in U(\Z[\sqrt{-5}]) \\
            N(\beta) = 2 & \exists a,b \in \Z \mid a^2+5b^2 = 2   \\
            N(\beta) = 3 & \exists a,b \in \Z \mid a^2+5b^2 = 3
        \end{array} \right.$$
    Luego $\beta \in U(\Z[\sqrt{-5}]) \Rightarrow \alpha$ es irreducible.\\

    
    Veamos que $\alpha$ no es primo:
    $$\left. \begin{array}{lll}
            N(\alpha) = 6 = \alpha \cdot \overline{\alpha} & \Rightarrow & \alpha|6 = 2 \cdot 3               \\
            \left. \begin{array}{l}
                       N(\alpha)\not|N(2) \\
                       N(\alpha)\not|N(3)
                   \end{array} \right\}                        & \Rightarrow & \alpha\not| 2 \y \alpha\not| 3
        \end{array} \right\} \Rightarrow \alpha \mbox{ no es primo}$$
\end{ejemplo}

\newpage
\section{Factoriazación en el anillo de polinomios}
Sabemos que si $K$ es un cuerpo $\Rightarrow K[x]$ es un DE $\Rightarrow K[x]$ es un DFU.\newline
Nuestro primer objetivo será demostrar que si $A$ es un DFU $\Rightarrow A[x]$ es un DFU.\\


Sea $A$ un DFU y $K = \Q(A)$, vamos a intentar relacionar a los irreducibles de $A[x]$ con los de $K[x]$.

\begin{prop}
    Sea $a \in A$. Entonces:
    $$a \mbox{ es irreducible en } A \Leftrightarrow a \mbox{ es irreducible en } A[x]$$
    Notemos que en $K[x]$ no hay irreducibles de grado 0, puesto que serían unidades.
\begin{proof}
    \ \\
    $\Longrightarrow)$ Sea $a \in A$ irreducible en $A$, luego $a \neq 0 \y a \notin U(A) = U(A[x])$.\par
    Si $f \in A[x]$ es un divisor de $a$:
    $$\exists g \in A[x] \mid a=fg \Rightarrow 0 = grd(a) = grd(fg) = grd(f) + grd(g)$$

    $ grd(f) = 0 \Rightarrow f \in A \Rightarrow f \in U(A) = U(A[x])$ ó $f$ asociado a $a$ en $A$\par
    Por lo que $a$ es también un irreducible en $A[x]$.\\

    
    $\Longleftarrow)$ Supongamos que $a$ es irreducible en $A[x]$ y que no lo es en $A$:\par
    Luego $\exists b \in A$ divisor propio de $a \Rightarrow b \in A[x]$ divisor propio de $a$. \par
    \underline{Contradicción}, luego $a$ es irreducible en $A$.
\end{proof}
\end{prop}

\begin{definicion}[Contenido de un polinomio]
    \ \\
    Sea $A$ un DFU y sea $\displaystyle f=\sum_{i=0}^n a_i x^i \in A[x]$ con $n \geq 1$. Definimos el \textbf{contenido
        de $f$}, notado $C(f)$ como:
    $$C(f):=mcd(a_0,a_1, \ldots, a_n)$$
    Si $C(f)=1$, diremos que $f$ es un \textbf{polinomio primitivo}.
\end{definicion}

\begin{lema}
    \label{lemaPropsContenido}
    Sea $A$ un DFU y $K = \Q(A)$. Se verifica:\newline
    i) $\forall a \in A \y ~\forall f \in A[x]$ con $grd(f) \geq 1 \Rightarrow C(af) = aC(f)$.\newline
    ii) $\forall f \in A[x] \mid grd(f) \geq 1$ se expresa de forma única como $f=af'$ tal que:
    $$a = C(f) \y f' \in A[x] \mid f' \mbox{ es primitivo}$$
    iii) $\forall \phi \in K[x] \mid grd(\phi) \geq 1$ se expresa de forma única como $\phi = \dfrac{a}{b}f$ tal que:
    $$\dfrac{a}{b} \in K \y f \in A[x] \mid f \mbox{ es primitivo}$$
\begin{proof}
    \ \\
    i) Sea $\displaystyle f=\sum_{i=0}^n a_i x^i$ con $\displaystyle n \geq 1 \Rightarrow af = \sum_{i=0}^n aa_i x^i$.
    $$C(af) = mcd(aa_0, aa_1, \ldots, aa_n) = mcd(a_0,a_1,\ldots, a_n)a = aC(f)$$
    ii) Sea $\displaystyle f=\sum_{i=0}^n a_i x^i$ con $n \geq 1$. Sea $a = C(f) = mcd(a_0,a_1,\ldots, a_n)$\newline
    Consideramos $a_i' = \dfrac{a_i}{a}~~\forall i \in \{1, \ldots, n\}$\newline
    Sea $\displaystyle f'=\sum_{i=0}^n a_i' x^i$. Se verifica que $af'=f$ y que:
    $$C(f) = mcd(a_0',a_1', \ldots, a_n')=1$$
    Luego $f'$ es primitivo.\\

    
    Falta ver la unicidad de $f'$ y $a$:\newline
    Sea $f=bg'$ con $b \in A$ y $g' \in A[x]$ primitivo:
    $$C(f) = C(bg') = bC(g') = b \Rightarrow a = b$$
    $$f=af'=ag' \mathop{\Rightarrow}^{DI} f' = g'$$
    iii) Sea $\displaystyle \phi = \sum_{i=0}^n \dfrac{a_i}{b_i}x^i \in K[x]$ con $n \geq 1$. Sea $b = b_1b_2\ldots b_n \in A$
    $$b\phi = \sum_{i=0}^n b\dfrac{a_i}{b_i}x^i~~~~c_i = b\dfrac{a_i}{b_i}\in A~~\forall i \in \{1, \ldots, n\}
        \Rightarrow b\phi \in A[x]$$
    Ya que $c_i = a_ib_1\ldots b_{i-1}b_{i+1}\ldots b_n$\\

    
    Aplicando ii):
    $$b\phi = af \mid a = C(b\phi) \y f\in A[x] \mbox{ primitivo}$$
    Por lo que tenemos que:
    $$\phi = \dfrac{a}{b} f = \dfrac{a}{b_1b_2\ldots b_n}f$$

    \ \\
    
    Falta ver la unicidad de $\dfrac{a}{b} \in K$ y de $f \in A[x]$:\newline
    Sea $\phi = \dfrac{a'}{b'}f' \mid f'$ primitivo:
    $$\dfrac{a}{b}f = \dfrac{a'}{b'}f' \Rightarrow b'af = ba'f'$$
    $$C(b'af) = C(ba'f') \Rightarrow b'aC(f) = ba'C(f') \Rightarrow b'a = ba' \Rightarrow \dfrac{a}{b} = \dfrac{a'}{b'}$$
    Y tenemos que:
    $$\phi = \dfrac{a}{b}f = \dfrac{a}{b}f' \mathop{\Rightarrow}^{DI} f = f'$$
\end{proof}
\end{lema}

\begin{lema}[de Gauss]
    \label{lemaGauss}
    El producto de polinomios primitivos es primitivo.
\begin{proof}
    Sean $f,g\in A[x]$ dos polinomios primitivos:
    \begin{itemize}
        \item Si $grd(f) = 1 = grd(g)$, tenemos que $f=ax +b$, $g = cx+d$ para ciertos elementos $a,b,c,d\in A$. Como $f$ y $g$ son primitivos, tenemos que $mcd(a,b) = 1 = mcd(c,d)$. Si multiplicamos ambos polinomios:
            \begin{equation*}
                fg = acx^2 + (ad + bc)x + bd
            \end{equation*}
            Hemos de probar que $mcd(ac, ad + bc, bd) = 1$. Para ello, por reducción al absurdo, supongamos que existe un elemento irreducible $p\in A$ que divide a todos estos coeficientes, es decir:
            \begin{equation*}
                p\mid ac \qquad p\mid ad + bc \qquad p \mid bd
            \end{equation*}
            Como nos encontramos en un DFU, tenemos que todo irreducible es primo, luego de $p\mid ac$ deducimos que $p\mid a$ o que $p \mid c$. Supongamos que $p \mid a$ y lleguemos a contradicción, con lo que el caso $p \mid c$ es totalmente análogo, aprovechando la simetría del problema (el polinomio $f$ es tan arbitrario como $g$).

            Como $p \mid a$ y $mcd(a,b) = 1$, tenemos que $p \nmid b$. Como $p \mid bd$ y $p \nmid b$, tendremos que $p\mid d$. Como $p \mid ad + bc$ y tenemos que $p\mid a$ y $p \mid d$, concluimos que $p \mid bc$. Como $p$ es primo:
            \begin{itemize}
                \item Si $p \mid b$, antes teníamos que $p\nmid b$, \underline{contradicción}.
                \item Si $p\mid c$, teníamos que $p\mid b$, por lo que $mcd(b,c) \neq 1$, \underline{contradicción}.
            \end{itemize}
        \item Supuesto que el producto de un polinomio de grado $n$ por otro de grado 1 es siempre primitivo, supuesto que $grd(f) = n+1$ y $grd(g) = 1$, demostremos que $fg$ es primitivo. %De esta forma, tenemos que:
            % \begin{equation*}
            %     f = a_{n+1}x^{n+1} + a_nx^n + \ldots + a_0, \qquad g = cx + d
            % \end{equation*}
            % con $mcd(a_0,\ldots,a_{n+1}) = 1 = mcd(c,d)$.
            % \begin{itemize}
            %     \item Si existen $i,j \in \{1,\ldots,n\}$ de forma que $mcd(a_i,a_j) = 1$ (observemos que estos coeficientes no son los del máximo grado de $f$), si tomamos:
            %         \begin{equation*}
            %             f' = a_nx^n + \ldots + a_0
            %         \end{equation*}
            %         Tenemos:
            %         \begin{equation*}
            %             fg = (f' + a_{n+1}x^{n+1}) = f'g + a_{n+1}x^{n+1}g
            %         \end{equation*}
            %         Con $f'$ un polinomio primitivo de grado $n$, por lo que $f'g$ es primitivo. Concluimos que $fg$ es primitivo. % // TODO: Por qué?
            %     \item 
            % \end{itemize}
            (la demostración está por completar) % // TODO: Completar
    \end{itemize}
\end{proof}
\end{lema}

\begin{coro}
    del Lema \ref{lemaGauss}.\newline
    Sean $f,g \in A[x] \mid grd(f),grd(g) \geq 1$. Entonces:
    $$C(fg)=C(f)C(g)$$
\begin{proof}
    Tenemos por el Lema \ref{lemaPropsContenido} que:
    $$f = C(f)f'~~~~g=C(g)g' \mid f',g' \mbox{ primitivos} \Rightarrow f'g' \mbox{ primitivo}$$
    $$fg = C(f)C(g)f'g' \Rightarrow C(fg) = C(f)C(g)C(f'g') = C(f)C(g)$$
\end{proof}
\end{coro}

\begin{teo}
    \label{teoEqIrreducibles}
    Sea $A$ un DFU y $K=\Q(A)$. Sea $\phi \in K[x]$ con $grd(\phi)\geq 1$ y sea $\phi=\dfrac{a}{b}f \mid f \in A[x]$
    primitivo. Son equivalentes:\par
    1) $\phi$ es irreducible en $K[x]$.\par
    2) $f$ es irreducible en $K[x]$.\par
    3) $f$ es irreducible en $A[x]$.
\begin{proof}
    \ \\
    $\phi$ y $f$ son asociados en $K[x] \Rightarrow Div(\phi)=Div(f) \Rightarrow (i) \Leftrightarrow ii))$\\

    
    2) $\Rightarrow$ 3) Supongamos que $f$ no es irreducible en $A[x]$:\newline
    $\exists f_1$ divisor propio de $f \Rightarrow \exists f_2 \in A[x] \mid f = f_1f_2$.\par
    $\bullet$ Si $f_1$ es constante $\Rightarrow f_1 \in A \Rightarrow C(f) = 1 = f_1C(f_2) \Rightarrow$
    \par $\Rightarrow f_1 \in U(A)=U(A[x]) \Rightarrow f_1$ no es divisor propio. Contradicción.\par
    $\bullet$ Si $f_2$ es constante $\Rightarrow f_2 \in A \Rightarrow C(f) = 1 = f_2C(f_1) \Rightarrow$
    \par $\Rightarrow f_2 \in U(A)=U(A[x])\Rightarrow f_1$ no es divisor propio. Contradicción.\par
    $\bullet$ Si $f_1$ y $f_2$ no son constantes:
    $$grad(f_1),grd(f_2)\geq 1$$
    Entonces:
    $f=f_1f_2$ es factorización en polinomios no unidades en $K[x]$. Contradicción.\newline
    Luego $f$ es irreducible en $A[x]$.\\

    
    3) $\Rightarrow$ 2) Supongamos que $f=\phi_1\phi_2\in K[x]$ con $grd(\phi_1),grd(\phi_2)\geq 1$.
    $$\phi_1 = \dfrac{a_1}{b_1}f_1~~~~\phi_2=\dfrac{a_2}{b_2}f_2 \mid f_1,f_2 \in A[x] \mbox{ primitivos}$$
    $$f=\dfrac{a_1a_2}{b_1b_2}f_1f_2 \Rightarrow b_1b_2f=a_1a_2f_1f_2 \Rightarrow C(b_1b_2f)=C(a_1a_2f_1f_2)$$
    $$C(b_1b_2f) = b_1b_2C(f) = b_1b_2$$
    $$C(a_1a_2f_1f_2)=a_1a_2C(f_1f_2)=a_1a_2C(f_1)C(f_2)=a_1a_2$$
    $$b_1b_2=a_1a_2 \Rightarrow \dfrac{a_1a_2}{b_1b_2}=1 \Rightarrow f=f_1f_2 \mbox{ \underline{Contradicción}}$$
    Ya que $f$ era irreducible en $A[x]$.\newline
    Por tanto, $f$ es irreducible en $K[x]$.
\end{proof}
\end{teo}

\begin{coro}
    del Teorema \ref{teoEqIrreducibles}.\newline
    Sea $A$ DFU, $f \in A[x] \mid grd(f) \geq 1$. Entonces:
    $$f \mbox{ es irreducible en } A[x] \Leftrightarrow f \mbox{ es primitivo e irreducible en } K[x]$$
\end{coro}

\begin{teo}[de Gauss]
    \label{teoGauss}
    Si $A$ es DFU $\Rightarrow A[x]$ es DFU.
\begin{proof}
    \ \\
    Sea $f \in A[x]$, $f \neq 0 \y f \notin U(A[x])$. Veamos que se expresa como producto de irreducibles.\\

    
    $\bullet$ Si $grd(f)=0 \Rightarrow f \in A \Rightarrow f=p_1\ldots p_k$ con $p_i \in A$ irreducible\newline
    $\forall i \in \{1, \ldots, k\} \Rightarrow p_i \in A[x] \Rightarrow f$ admite factorización en irreducibles en $A[x]$.\newline
    $\bullet$ Si $grd(f)\ge 1 \Rightarrow f=af' \mid f \in A[x]$ primitivo.
    $$\left. \begin{array}{l}
            f' \in K[x]=\Q(A) \\
            K[x] DFU
        \end{array} \right\} \Rightarrow f' = \phi_1 \ldots \phi_s$$
    Con $\phi_i \in K[x]$ irreducible $\forall i \in \{1, \ldots, s\}$\\

    
    $a \in A \Rightarrow a$ admite factorización en irreducibles: $a = p_1 \ldots p_k$.
    $$\phi_i = \dfrac{a_i}{b_i}f_i \mid f_i \in A[x] \mbox{ primitivo e irreducible } \forall i \in \{1, \ldots, s\}$$
    Luego:
    $$f' = \dfrac{a_1\ldots a_s}{b_1\ldots b_s}f_1\ldots f_s$$
    $$1= C(f') = C\left(\dfrac{a_1\ldots a_s}{b_1\ldots b_s}f_1\ldots f_s\right) = \dfrac{a_1\ldots a_s}{b_1\ldots b_s}C(f_1\ldots f_s)=$$
    $$= \dfrac{a_1\ldots a_s}{b_1\ldots b_s} C(f_1)\ldots C(f_s) = \dfrac{a_1\ldots a_s}{b_1\ldots b_s}$$
    Luego:
    $$f=af' = p_1\ldots p_k f_1\ldots f_s$$
    Es una factorización en irreducibles.\\

    
    Veamos ahora que todo irreducible de $A[x]$ es primo, para así tener que $A[x]$ es un DFU mediante el Teorema \ref{teoCaracDFU}.\\

    
    Sea $f \in A[x]$ irreducible con $grd(f)\geq 1$ y sean $g,h \in A[x] \mid f|gh$
    $$f \mbox{ es primitivo } \Rightarrow \mbox{ es irreducible en } K[x] \Rightarrow f \mbox{ es primo en } K[x]$$
    Luego si $f|gh \Rightarrow f|g \o f|h$ en $K[x]$.\newline
    Supongamos (lo cual no es restrictivo) que $f|g \Rightarrow \exists \phi \in K[x] \mid g=\phi f$
    $$\phi = \dfrac{a}{b}f' \mid f' \in A[x] \mbox{ primitivo}$$
    Luego $g=\dfrac{a}{b}ff' \Rightarrow bg=af'f \Rightarrow C(bg)=C(aff')$
    $$C(bg)=bC(g)=aC(ff')=aC(f)C(f')=a$$
    Luego $\dfrac{a}{b}=C(g) \in A \Rightarrow g = C(g)ff' \Rightarrow f|g$ en $A[x]$.
\end{proof}
\end{teo}

\newpage
\section{Criterios básicos de irreducibilidad de polinomios}
Sea $K$ un cuerpo:\newline
1) En $K[x]$, todo polinomio no nulo es asociado a un polinomio mónico:
$$\mbox{Sea } \phi=\sum_{i=0}^na_i x^i \mid a_n \neq 0 \Rightarrow a_n \in U(K[x])$$
$$\psi = a_n^{-1}\phi = \sum_{i=0}^n a_n^{-1}a_i x^i \mbox{ es mónico}$$

\begin{lema}
    Dos polinomios mónicos son asociados $\Leftrightarrow$ son iguales.
\begin{proof}
    Sean: $$\phi=\sum_{i=0}^n a_i x^i~~\psi = \sum_{i=0}^m b_i x^i \mid \phi, \psi \mbox{ mónicos }$$
    $\phi$ y $\psi$ son asociados $\Leftrightarrow \exists a \in K\setminus\{0\}\mid \phi = a\psi$\newline
    Que es decir que:
    $$\sum_{i=0}^n a_i x^i = \sum_{i=0}^m ab_i x^i \Leftrightarrow n=m \y a_i = ab_i ~\forall i \in \{0, \ldots, n\}$$
    En particular, como $a_n=b_m = 1 \Rightarrow 1=a \Leftrightarrow \phi = \psi$
\end{proof}
\end{lema}

Por tanto,
\underline{para conocer los polinomios irreducibles en $K[x]$ (salvo asociados),}\newline
\underline{basta con conocer los irreducibles mónicos}.\\

\noindent
2) En $K[x]$ no hay irreducibles de grado 0, ya que todos los de grado 0 son unidades.
\begin{prop}
    Todo polinomio de grado 1 en $K[x]$ es irreducible.
\begin{proof}
    Sea $\phi \in K[x] \mid grd(\phi)=1 \y \phi \notin U(K[x])$:\newline
    Sea $\phi'$ divisor de $\phi \Rightarrow \exists \phi'' \in K[x] \mid \phi = \phi'\phi''$ con $\phi',\phi''\neq 0$.\newline
    $$1 = grd(\phi) = grd(\phi') + grd(\phi'') \Rightarrow \left\{ \begin{array}{lll}
            grd(\phi') = 0                 & \Rightarrow & \phi' \in U(K[x])         \\
            \o                             &             &                           \\
            grd(\phi') = 1                 & \Rightarrow & grd(\phi'')=0 \Rightarrow \\
            \Rightarrow \phi'' \in U(K[x]) & \Rightarrow & \phi'\sim \phi
        \end{array}\right.$$
    Por lo que los únicos divisores de $\phi$ son los triviales $\Rightarrow \phi$ es irreducible.
\end{proof}
\end{prop}

Los polinomios irreducibles mónicos de grado 1 en $K[x]$ son los de la forma:
$$\{x+a \mid a \in K\}$$
Si $K=\Z_p$ con $p$ primo de $\Z$ mayor o igual que 2, son:
$$\{x, x+1, \ldots x+p-1\}$$

\begin{teo}[Teorema fundamental del Álgebra, Gauss]
    \ \\
    En $\C[x]$, los únicos polinomios irreducibles son los de grado 1.
\begin{proof}
    Excede el conocimiento del curso.
\end{proof}
\end{teo}

\begin{teo}[de Ruffini]
    \label{teoRuffini}
    Sea $\phi \in K[x]$, $a \in A$. Entonces:
    $$\mbox{El resto de dividir } \phi \mbox{ entre } (x-a) \mbox{ es } \phi(a)$$
\begin{proof}
    Dividimos $\phi$ entre $(x-a)$ y obtenemos que:
    $$\phi = (x-a)\psi + r \mid r = 0 \o grd(r)<grd(x-a)=1$$
    Por lo que sabemos que $r \in K$. Además:
    $$\phi(a) = ((x-a)\psi+r)(a) = (a-a)\psi(a) + r = 0 + r = r$$
\end{proof}
\end{teo}

\begin{prop}
    \label{propFactorGrd1Raiz}
    Sea $\phi \in K[x] \mid grd(\phi) \geq 2$:
    $$\phi \mbox{ tiene un factor de grado } 1 \mbox{ en } K[x] \Leftrightarrow \phi \mbox{ tiene una raíz en } K$$
\begin{proof}
    \ \\
    $\Longrightarrow)$
    $$\exists \phi_1 = a_0+a_1x \in K[x] \mid \phi_1 \neq 0 \y \phi_1|\phi \Rightarrow \exists \phi_2 \in K[x] \mid \phi = \phi_1\phi_2$$
    $$\phi = \phi_1\phi_2 = (a_0+a_1x)\phi_2$$
    Sea $a = -a_0a_1^{-1}$:
    $$\phi(a) = \phi_1(a)\phi_2(a) = (a_0+a_1a)\phi_2(a) = (a_0+a_1(-a_0a_1^{-1}))\phi_2(a) = 0$$

    
    $\Longleftarrow)$ $$\exists a \in K \mid \phi(a) = 0$$
    Por el Teorema de Ruffini $($Teorema \ref{teoRuffini}$)$, tenemos que:
    $$\phi = q(x-a)+\phi(a) = q(x-a) \Rightarrow x-a|\phi$$
\end{proof}
\end{prop}

\begin{coro}[Criterio de la raíz]
    \label{corCritRaiz}
    de la Proposición \ref{propFactorGrd1Raiz}.\newline
    Sea $\phi \in K[x] \mid grd(\phi) \in \{2,3\}$. Entonces:
    $$\phi \mbox{ es irreducible } \Leftrightarrow \phi \mbox{ no tiene raíces en } K$$
\begin{proof}
    \ \\
    Si $\phi$ es de grado 2, sus divisores propios serán de grado 1 o menos y sabemos por la Proposición \ref{teoRuffini}
    que no tiene de grado 1 y además, los polinomios de grado 0 son unidades, por lo que no serían factores propios.\\

    
    Si $\phi$ es de grado 3, sus divisores propios serían de grado 2 o menos y por lo anterior sabemos que no pueden ser
    de grado 1 o menos, luego podría tener de grado 2, pero para que el producto en el que se divide sea de grado 3 haría
    falta un polinoimo de grado 1, imposible.
\end{proof}
\end{coro}

\begin{teo}
    En $\R[x]$, los polinomios irreducibles son los de grado 1 y los de grado 2 de la forma:
    $$ax^2+bx+c \mid b^2-4ac<0$$
    Por tanto, los irreducibles mónicos en $\R[x]$ son:
    $$\{x+a \mid a \in \R\} \cup \{x^2+bx+c \mid b,c \in \R \y b^2-4c < 0\}$$
\end{teo}

\bigskip

Notemos que pondemos listar todos los polinomios mónicos irreducibles de grado 2 ó 3 en $\Z_p[x]$, considerando todos
los mónicos y quedándonos con los que no tengan raíces:\\


$\bullet$ Para $p=2$, listamos $x^2+bx+c \in \Z_2[x] \mid b,c \in \Z_2$:
$$a=0 \Rightarrow \left\{ \begin{array}{llll}
        b=0 & x^2   & \mbox{reducible} & (x^2)(0)=0   \\
        b=1 & x^2+1 & \mbox{reducible} & (x^2+1)(1)=0
    \end{array} \right.$$
$$a=1 \Rightarrow \left\{ \begin{array}{llll}
        b=0 & x^2+x   & \mbox{reducible}   & (x^2+x)(1)=0           \\
        b=1 & x^2+x+1 & \mbox{irreducible} & \mbox{no tiene raíces}
    \end{array} \right.$$

\ \\

$\bullet$ Para $p=3$, listamos $x^2+bx+c \in \Z_3[x] \mid b,c \in \Z_3$:
$$a=0 \Rightarrow \left\{ \begin{array}{llll}
        b=0 & x^2   & \mbox{reducible}   & (x^2)(0) = 0           \\
        b=1 & x^2+1 & \mbox{irreducible} & \mbox{no tiene raíces} \\
        b=2 & x^2+2 & \mbox{reducible}   & (x^2+2)(2)=0
    \end{array} \right.$$
$$a=1 \Rightarrow \left\{ \begin{array}{llll}
        b=0 & x^2+x   & \mbox{reducible}   & (x^2+x)(2) = 0         \\
        b=1 & x^2+x+1 & \mbox{reducible}   & (x^2+x+1)(1)=0         \\
        b=2 & x^2+2   & \mbox{irreducible} & \mbox{no tiene raíces}
    \end{array} \right.$$
$$a=2 \Rightarrow \left\{ \begin{array}{llll}
        b=0 & x^2+2x   & \mbox{reducible}   & (x^2+2x)(0) = 0        \\
        b=1 & x^2+2x+1 & \mbox{reducible}   & (x^2+2x+1)(2)=0        \\
        b=2 & x^2+2x+2 & \mbox{irreducible} & \mbox{no tiene raíces}
    \end{array} \right.$$
\bigskip

Para los polinomios de grado 4, los listamos y descartamos los que tengan raíces. Si no tienen raíces entonces no tienen
factores de grado 1 $\Rightarrow$ no tienen factores de grado 3.\newline
Puede tener de grado 2, luego cogemos el polinomio y lo dividimos entre irreducibles de grado 2.

\begin{ejemplo}
    \ \\
    1. Factorizar $f=x^4+x^3+x^2+x+1 \in \Z_2[x]$.
    $$f(0)=1=f(1) \Rightarrow f \mbox{ no tiene raíces } \Rightarrow f \mbox{ no tiene factores de grado } 1$$
    Luego $f$ tampoco tiene factores de grado $3$. Puede tener de grado $2$.\\

    
    Como el único polinomio irreducible de grado $2$ en $\Z_2[x]$ es $x^2+x+1$, dividimos entre él:
    $$f=(x^2+x+1)(x^2+x+1)+x+1 \Rightarrow x^2+x+1\not|f$$
    Luego $f$ no tiene factores de grado $2 \Rightarrow f$ es irreducible al no tener factores de grado $1$, $2$ ni $3$.\\

    
    2. Factorizar $x^5+x^4+x^2+1 \in \Z_3[x]$.
    $$f(0) = 1 = f(1)~~f(2) = 2 \Rightarrow f \mbox{ no tiene raíes } \Rightarrow f \mbox{ no tiene factores de grado } 1$$
    Luego $f$ tampoco tiene factores de grado $4$. Puede tener de grado $2$ ó $3$.\\

    
    Buscamos factores irreducibles de grado $2$, que pueden ser: $x^2+1$, $x^2+x+2$, $x^2+2x+2$.
    $$f=(x^2+1)(x^3+x^2+2x)+x+1$$
    $$f=(x^2+x+2)(x^3+x)+x+1$$
    $$f=(x^2+2x+2)(x^2+2x)+2$$
    Luego $f$ no tiene factores de grado $2 \Rightarrow f$ no tiene factores de grado $3$.\\

    
    Por tanto, $f$ es irreducible.\\

    
    3) Factorizar $f=x^7+2x^5+x^4+2x^3+x+2 \in \Z_3[x]$.
    $$f(1)=0 \Rightarrow x-1 = x+2|f \Rightarrow f=(x+2)g \mid g=x^6+x^5+x^3+1$$
    Factorizamos $g$:
    $$g(2)=0 \Rightarrow x-2=x+1|g \Rightarrow g=(x+1)h \mid h = x^5+x^2+2x+1$$
    Factorizamos $h$:
    $$h(0),h(1),h(2) \neq 0 \Rightarrow h \mbox{ no tiene raíces } \Rightarrow h \mbox{ no tiene factores de grado } 1$$
    Luego tampoco tiene factores de grado $4$. Buscamos factores de grado $2$ en $\Z_3[x]$, que pueden ser:\newline
    $x^2+1$, $x^2+x+2$ ó $x^2+2x+2$.
    $$h=(x^2+1)(x^3+2x+1)$$
    Factorizamos $x^3+2x+1$:\newline
    No tiene raíces en $\Z_3[x] \Rightarrow $ es irreducible por el Criterio de la Raíz $($Corolario \ref{corCritRaiz}$)$.\\

    
    En resumen:
    $$f=(x+2)(x+1)(x^2+1)(x^3+2x+1)$$
\end{ejemplo}

\newpage
\section{Polinomios en los anillos de enteros y de racionales cuadráticos}
Recordemos que:\newline
1) Los irreducibles de grado 0 en $\Z[x]$ son los irreducibles de $\Z$ (los números primos).\par
En $\Q[x]$ no hay irreducibles de grado 0 por ser $\Q$ un cuerpo.\newline
2) Sea $f \in \Z[x] \mid grd(f)>0$ y sea $f$ primitivo $(C(f)=1)$. Entonces:
$$f \mbox{ es irreducible en } \Z[x] \Leftrightarrow f \mbox{ es irreducible en } \Q[x]$$
3) Sea $\phi \in \Q[x] \mid grd(\phi)>0$ y sea $\phi=\dfrac{a}{b}f \mid a,b \in \Z$, $b \neq 0$ y $f$ primitivo. Entonces:
$$\phi \mbox{ es irreducible en } \Q[x] \Leftrightarrow f \mbox{ es irreducible en } \Z[x]$$
Como todo polinomio de grado 1 en $\Q[x]$ es irreducible, entonces:\newline
4) Un polinomio $f=a_0+a_1x \in \Z[x] \mid a_1 \neq 0$:
$$f \mbox{ es irreducible en } \Z[x] \Leftrightarrow mcd(a_0,a_1)=1$$
Para grado 2 ó 3 sabemso que en $\Q[x]$ son irreducibles aquellos polinomios que no tienen raíces en $\Q$.\newline
5) Un polinomio $f \in \Z[x]$ con $grd(f) \in \{2,3\}$:
$$f \mbox{ es irreducible en } \Z[x] \Leftrightarrow f \mbox{ primitivo y no tiene raíces en } \Q$$

\begin{prop}
    \label{propbxmenosA}
    Sea $f=a_0+a_1x+\ldots+a_nx^n \in \Z[x] \mid n\geq 1$. \newline Sea $\dfrac{a}{b} \in \Q \mid f\left(\dfrac{a}{b}\right)=0$
    y $mcd(a,b)=1$. Entonces:
    $$(bx-a) \mbox{ es un divisor propio de } f \mbox{ en } \Z[x]$$
    Es decir:
    $$\exists g \in \Z[x] \mid f=(bx-a)g$$
\begin{proof}
    \ \\
    Supongamos que $f\left(\dfrac{a}{b}\right) = 0$:\newline
    Por el Teorema de Ruffini $($Teorema \ref{teoRuffini}$)$:
    $$x-\dfrac{a}{b}|f \mbox{ en } \Q[x] \mathop{\Longrightarrow}^{b \in U(\Q)} bx-a=b\left(x-\dfrac{a}{b}\right)|f \mbox{ en } \Q[x]$$
    Por lo que $\exists g \in \Q[x] \mid f=(bx-a)g$\newline
    Sabemos que $g=\dfrac{c}{d}g' \mid c,d \in \Z \y d \neq 0 \y g' \mbox{ primitivo}$.
    $$f=(bx-a)g=(bx-a)\dfrac{c}{d}g' \Rightarrow df=c(bx-a)g'$$
    $$dC(f) = C(df) = C(c(bx-a)g')=cC((bx-a)g')=cC(bx-a)C(g')=c$$
    Luego $dC(f)=c \Rightarrow \Z \ni C(f) = \dfrac{c}{d}$.\\

    
    Por lo que $g=\dfrac{c}{d}g' = C(f)g' \in \Z[x]$.
\end{proof}
\end{prop}

\begin{coro}
    \label{corpropbxmenosA}
    de la Proposición \ref{propbxmenosA}.\newline
    Sea $f=a_0+a_1x+\ldots+a_nx^n \in \Z[x] \mid n\geq 1$. \newline Sea $\dfrac{a}{b} \in \Q \mid f\left(\dfrac{a}{b}\right)=0$
    y $mcd(a,b)=1$. Entonces:
    $$a|a_0 \y b|a_n \mbox{ en } \Z[x]$$
\begin{proof}
    $$f=(bx-a)g \in \Z[x]$$
    Sea $g = b_0+b_1x + \ldots + b_{n-1}x^{n-1}$:
    $$a_0+a_1x+\ldots+a_nx^n = (bx-a)(b_0+b_1x + \ldots + b_{n-1}x^{n-1})$$
    Luego: $$a_0=-ab_0=a(-b_0) \Rightarrow a|a_0$$
    $$a_n=bb_{n-1}\Rightarrow b|a_n$$
\end{proof}
\end{coro}

\begin{coro}
    del Corolario \ref{corpropbxmenosA}.\newline
    1) Sea $f=a_0+a_1x + \ldots +a_{n-1}x^{n-1}+x^n \in \Z[x] \mid n\geq 1$. Entonces:\par
    Las raíces de $f$ en $\Q$ están en $\Z$.\newline
    2) Sea $n \in \Z \mid \sqrt{n} \notin \Z \Rightarrow \sqrt{n} \notin \Q$.
\begin{proof}
    \ \\
    1) Si $\dfrac{a}{b} \in \Q \mid f\left(\dfrac{a}{b}\right)=0 \Rightarrow a|a_0 \y b|1 \Rightarrow b=\pm 1
        \Rightarrow \dfrac{a}{b} = \pm a \in \Z$.\\

    
    2) Sea $\displaystyle n \in \Z \mid \sqrt{n} \notin \Z \Rightarrow x^2-n$ no tiene raíces en $\Z \mathop{\Rightarrow}^{1)} $
    tampoco en $\Q \Rightarrow \sqrt{n} \notin \Q$.
\end{proof}
\end{coro}

\begin{ejemplo}
    Factorizar en $\Z[x]$ y en $\Q[x]$ $f=20x^4-10x^3-80x^2+80x-20$.
    $$C(f) = 10 \Rightarrow f=10g \mid g = 2x^4-x^3-8x^2+8x-2 \mbox{ primitivo}$$
    Factorizamos $g$:\newline
    Las posibles raíces en $\Q$ de $g$ son de la forma: $\dfrac{a}{b} \mid a|a_0=(-2) \y b|a_n=2$.\newline
    Luego las opciones son: $\pm1$, $\pm2$, $\pm 1/2$. Evaluamos en ellos y obtenemos que:
    $$g\left(\dfrac{1}{2}\right) = 0 \Rightarrow 2x-1|g \mbox{ en } \Z[x]$$
    Dividimos entre $2x-1$:
    $$g=(2x-1)(x^3-4x+2)$$
    $2x-1$ es irreducible en $\Z[x]$ por ser de grado 1 y primitivo.\newline
    Buscamos raíces de $x^3-4x+2$ que sabemos que son enteras, por ser mónico:\par
    Sus posibles raíces son de la forma $\dfrac{a}{b} \mid a|a_0=2 \y b|a_n=1$.\newline
    Luego las opcinoes son: $\pm1$ y $\pm2$. Evaluamos en ellos y ninguna es raíz.\\

    
    Por tanto, tenemos que $x^3-4x+2$ es irreducible por ser primitivo y no tener raíces en $\Q$.\\

    
    La factorización de $f$ en $Z[x]$ es:
    $$f=2 \cdot 5(2x-1)(x^3-4x+2)$$
    Mientras que en $\Q[x]$ es:
    $$f=10(2x-1)(x^3-4x+2)~~~~(10 \in U(\Q[x]))$$
    $$f=20\left(x-\dfrac{1}{2}\right)(x^3-4x+2)$$
\end{ejemplo}

\begin{ejemplo}
    Factorizar en $\Q[x]$: $\phi=x^3+\dfrac{1}{2}x^2-x-3$
    $$\phi = \dfrac{1}{2}(2x^3+x^2-2x-6)=\dfrac{1}{2}f \mid f \in \Z[x] \mbox{ primitivo}$$
    Las posibles raíces de $f$ son de la forma $\dfrac{a}{b} \mid a|a_0=-6 \y b|a_n=2$.\newline
    Luego las opciones son: $\pm1$, $\pm2$, $\pm3$, $\pm6$, $\pm1/2$ ó $\pm3/2$. Evaluando, llegamos a que:
    $$f\left(\dfrac{3}{2}\right)=0 \Rightarrow (2x-3)|f \mbox{ en } \Z[x] \Rightarrow f=(2x-3)(x^2+2x+2)$$
    $2x-3$ es irreducible en $\Q[x]$ por ser de grado 1.\newline
    $x^2+2x+2$, sus posibles raíces son de la forma $\dfrac{a}{b} \mid a|a_0=2 \y b|a_n=1$.\newline
    Luego las opciones son: $\pm1$, $\pm2$. Evaluando, ninguna es raíz, por lo que $x^2+2x+2$ no tiene raíces en $\Q$.
    Luego es irreducible.\\

    $$\phi = \dfrac{1}{2}(2x-3)(x^2+2x+2) = \left( x-\dfrac{3}{2} \right)(x^2+2x+2)$$
\end{ejemplo}

\subsection{Criterio de reducción}

Sea $p \in \Z \mid p\geq 2$ primo y consideramos el homomorfismo de anillos:
$$R_p:\Z\Rightarrow \Z_p~~~~~~R_p(a):=R(a;p)$$
Que induce un homomorfismo:
$$R_p:\Z[x]\Rightarrow \Z_p[x]~~~~~~R_p\left(\sum_{i=0}^n a_i x^i \right) = \sum_{i=0}^n R_p(a_i)x^i$$
Con $grd(R_p(f)) \leq grd(f)$

\begin{prop}[Criterio de reducción]
    \ \\
    Sea $f \in \Z[x] \mid grd(f)>0 \y grd(f)=grd(R_p(f))$\newline
    Supongamos que $R_p(f)$ no tiene factores de grado $r$ siendo $0<r<grd(f)$ en $\Z_p[x]$. Entonces:
    $$f \mbox{ no tiene factores de grado } r \mbox{ en } \Z[x]$$
    En particular, si $R_p(f)$ es irreducible en $\Z_p[x] \Rightarrow f$ es irreducible en $\Z[x]$.
\begin{proof}
    \ \\
    Supongamos que $f \in \Z[x] \mid grd(f)>0 \y grd(f)=grd(R_p(f))$ y que $R_p(f)$ no tiene factores de grado
    $r$ siendo $0<r<grd(f)$ en $\Z_p[x]$.\\

    
    Supongamos que $\exists g \in \Z[x] \mid grd(g)=r \y g|f \mbox{ en } \Z[x]$.\newline
    Entonces, $\exists h \in \Z[x] \mid f=gh \y s=grd(h) \Rightarrow grd(f)=grd(g)+grd(h)=r+s$
    $$R_p(f)=R_p(gh)=R_p(g)R_p(h) \mid grd(R_p(g))\leq r \y grd(R_p(h))\leq s$$ Entonces:
    $$r+s = grd(R_p(f)) = grd(R_p(g))+grd(R_p(h)) \leq r+s \Rightarrow$$
    $$\Rightarrow grd(R_p(g))=r \y grd(R_p(h))=s$$
    Luego $R_p(g)$ es un factor de grado $r$ de $R_p(f)$. \underline{Contradicción} con la hipótesis.\\

    
    Luego $f$ no tiene factores de grado $r$ en $\Z[x]$.
\end{proof}
\end{prop}

% \begin{ejemplo}
%     % // TODO: copiar ejemplo.
% \end{ejemplo}

% \begin{ejemplo}
%     % // TODO: copiar ejemplo.
% \end{ejemplo}

% \begin{ejemplo}
%     % // TODO: copiar ejemplo.
% \end{ejemplo}

\begin{prop}[Criterio de Eisenstein]
    Sea $f = a_0+a_1x + \ldots + a_nx^n \in \Z[x] \mid grd(f)>0$ y sea $f$ \underline{primitivo}.\newline
    Entonces, $f$ es irreducible en $\Z[x]$ $($y también en $\Q[x])$ si:
    $$\exists p \in \Z \mid p\geq 2 \mbox{ primo}$$
    Tal que verifica alguna de las siguientes condiciones:\par
    i) $p|a_i~~\forall i \in \{0, \ldots, n-1\} \y p^2\not|a_0$\par
    ii) $p|a_i~~\forall i \in \{1, \ldots, n\} \y p^2\not|a_n$
\begin{proof}
    \ \\
    i) Supongamos que se verifica i) y sin embargo, que $f$ es reducible:\newline
    Luego $f=gh \mid g,h \in \Z[x] \y grd(g)=r\geq 1 \y grd(g)=s\geq 1$.\newline
    Sean $g=b_0+b_1x+\ldots b_rx^r$ y $h=c_0+c_1x + \ldots + c_sx^s$.\newline
    Como $a_0=b_0c_0$ y $p|a_0 \Rightarrow p|b_0c_0 \Rightarrow p|b_0 \o p|c_0$.\newline
    Como $p^2\not|a_0 \Rightarrow p$ no puede dividir simultáneamente a los dos.\par
    Supongamos (lo cual no es resitrictivo) que $p|b_0 \y p\not|c_0$:\par
    Como $f$ es primitivo:
    $$\left. \begin{array}{l}
            mcd(a_0,a_1, \ldots, a_{n-1}, a_n)=1 \\
            p|a_i~~\forall i \in \{0,\ldots, n-1\}
        \end{array} \right\} \Rightarrow p\not| a_n$$

    Como $a_n=b_rc_s$, entonces $p\not|b_rc_s \Rightarrow p\not|b_r \y p\not|c_s$.\par
    Sea $b_i$ el primer coeficiente tal que $p\not|b_i$ con $0<i\leq r<n$.\par
    Consideramos el coeficiente $i$-ésimo de $f$: $a_i$:
    $$a_i=b_0c_i + b_ic_{i-1} + \ldots + b_{i-1}c_1 + b_ic_0$$
    $$p|a_i (i<n) \Rightarrow p(b_0c_i + b_ic_{i-1} + \ldots + b_{i-1}c_1 + b_ic_0) \Rightarrow p|b_ic_0$$

    Pero $p\not|b_i$ y $p\not|c_0$. \underline{Contradicción} con que $p$ es primo.\\

    
    Luego $f\neq gh \Rightarrow f$ es irreducible.\\

    
    % ii) % // TODO: si se encuentra una demostración, meterla.
\end{proof}
\end{prop}

% \begin{ejemplo}
%     % // TODO: copiar ejemplo.
% \end{ejemplo}

