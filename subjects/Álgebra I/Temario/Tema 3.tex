\chapter{Divisibilidad en Dominios Euclídeos}

Consideramos la ecuación: $ax=b$ con $a,b \in A$, $a \neq 0$.
\begin{itemize}
    \item Si $A$ es un cuerpo, la ecuación tiene solución única: $x=a^{-1}b$.
    \item Si $A$ no es un cuerpo, no tenemos asegurada la existencia de la solución en $A$:
        $$2x=3~~\mbox{ no tiene solución en }\Z$$
        Puede ocurrir además que la ecuación tenga más de una solución en $A$:
        $$2x=2\hspace{1cm}
        \left\{\begin{array}{c}
            x=1 \\ \y \\ x=4
        \end{array}\right\}
        \mbox{ en } \Z_6$$
\end{itemize}

\begin{definicion}[Dominio de Integridad]
    Un \textbf{dominio de integridad} (abreviado a partir de ahora como DI) es un anillo conmutativo $A$ no trivial en el que
    se verifica la propiedad cancelativa, que dice:
    $$\left. \begin{array}{c}
            ab = ac \\
            a \neq 0
        \end{array} \right\} \Longrightarrow b=c\qquad \forall a,b,c\in A$$
\end{definicion}

\begin{lema}
En un DI, la ecuación $ax=b$, con $a\neq 0$, si tiene solución necesariamente es única.
\end{lema}
\begin{proof}
    Supongamos que $\exists x,x' \in A \mid ax = b \y ax' =b$.
    $$b=b \Longrightarrow ax = ax' \mbox{ con } a\neq 0 \Longrightarrow x = x'$$
\end{proof}

\begin{prop}[Caracterización de un DI]
    Sea $A$ un anillo conmutativo no trivial. Entonces:
    $$A \mbox{ es DI } \Longleftrightarrow \forall a,b \in A \mid a,b \neq 0 \text{ se tiene que } ab \neq 0$$
    Equivalente, su contrarrecíproco es:
    $$A \mbox{ es DI } \Longleftrightarrow \forall a,b \in A \mid ab = 0 \text{ se tiene que } a = 0 \o b = 0$$
\end{prop}
\begin{proof}\
    \begin{description}
        \item[$\Longrightarrow)$]
        Sea $A$ un DI y sean $a,b \in A \mid ab = 0$. Supongamos $a\neq 0$ sin perder generalidad y sabemos que $a \cdot 0 = 0$ luego:
        $$\left. \begin{array}{c}
            ab = 0 = a \cdot 0 \\
            a \neq 0
        \end{array} \right\} \mathop{\Longrightarrow}^{A \mbox{ es DI}} b=0$$
            
        \item[$\Longleftarrow)$]
        Supongamos que en $A$ se verifica la propiedad enunciada. Entonces, $\forall a,b,c \in A$ con $a\neq 0$ y $ab=ac$, se tiene que:
        $$ ab = ac \Longrightarrow ab - ac = a(b - c) = 0$$
        Por la propiedad enunciada, como $a\neq 0$ se tiene que $b-c=0$, por lo que $b=c$ y se tiene que es un DI.
    \end{description}
\end{proof}

\begin{lema}
    \label{lema:CuerpoEntoncesDominio}
    Si $K$ es un cuerpo $\Longrightarrow$ $K$ es un dominio de integridad.
\end{lema}
\begin{proof}
    $K$ es un cuerpo $\Longrightarrow$ $K$ es un anillo conmutativo no trivial.

    Para todo $a,b,c \in k$ con $a\neq 0$ y $ab=ac$, se tiene que:
    $$ab=ac \Longrightarrow a^{-1}(ab) = a^{-1}(ac) \Longrightarrow b=c$$
    donde he aplicado que, como $a\neq 0$, es una unidad. Por tanto, se tiene que es un DI.
\end{proof}


Como corolario tenemos que $\Q$, $\R$, $\C$, $\Q[\sqrt{n}] \mid (n \in \Z~~\sqrt{n} \notin \Z)$, $\Z_2$, $\Z_3$, $\Z_5$, etc. son dominios de integridad.

\begin{lema}
    \label{lema:SubanilloDIEntoncesDI}
    Si $B$ es DI y $A$ es subanillo de $B \Longrightarrow A$ es DI.
\end{lema}
\begin{proof}
    Por reducción al absurdo. Como $A$ no es DI, tenemos que $\exists a,b,c \in A \mid a \neq 0$ tal que $ab=ac $ pero $b \neq c$.
    No obstante, como $A\subset B$, tenemos que lo escrito también se aplica a $B$ y por tanto $B$ no es un DI, por lo que llegamos a una contradicción.
\end{proof}
Por tanto, $\Z$, $\Z[\sqrt{n}] \mid (n \in \Z~~\sqrt{n} \notin \Z)$, etc. son dominios de integridad.

\begin{ejemplo}
    Algunos ejemplos de anillos conmutativos que no son dominios de integridad son: $\Z_4$, $\Z_6$, $\Z_8$:
    \begin{gather*}
        2 \in \Z_4 \mid 2 \neq 0 \mbox{ pero } 2\cdot 2 = 0 \Longrightarrow \Z_4 \mbox{ no es DI}\\
        2,3 \in \Z_6 \mid 2,3 \neq 0 \mbox{ pero } 2 \cdot 3 = 0 \Longrightarrow \Z_6 \mbox{ no es DI}\\
        2,4 \in \Z_8 \mid 2,4 \neq 0 \mbox{ pero } 2 \cdot 4 = 0 \Longrightarrow \Z_8 \mbox{ no es DI}
    \end{gather*}
\end{ejemplo}

\begin{teo}
    Sea $A$ un anillo conmutativo no trivial. Entonces:
    $$A \mbox{ es DI } \Longleftrightarrow A[x] \mbox{ es DI}$$
\end{teo}
\begin{proof}\
\begin{description}
    \item[$\Longrightarrow)$] Si $A[x]$ es DI, como $A$ es subanillo de $A[x] \Longrightarrow A$ es DI.
    \item[$\Longleftarrow)$]
    Supongamos que $A$ es DI. Sean $f,g \in A[x] \mid f,g \neq 0$:
    $$f= \sum_{i=0}^n a_i x^i~~g = \sum_{j=0}^m b_j x^j \mid grd(f) = n \y grd(g) = m \Longrightarrow a_n, b_m \neq 0$$

    Por el producto de polinomios, tenemos que:
    $$fg = \sum_{k=0}^{n+m} d_k x^k \mid d_k = \sum_{i+j=k}a_i b_j$$

    Sabemos que si $\left\{\begin{array}{ccc}
            i>n & \Longrightarrow & a_i=0\\
             & \y & \\
            j>m & \Longrightarrow & b_j=0
        \end{array}\right\}$.
    En particular:
    $$d_{n+m} = \sum\limits_{\mathclap{i+j=n+m}}a_ib_j = a_n b_m \neq 0$$ por ser $A$ un DI y  $a_n, b_m \neq 0$. Deducimos entonces que $fg \neq 0$, por lo que $A[x]$ es DI.
\end{description}
\end{proof}

\begin{lema}
    \label{lema:GradoProductoDI}
    Sea $A$ un DI y $f,g \in A[x] \mid f,g \neq 0$. Entonces:
    $$grd(fg) = grd(f) + grd(g)$$
\end{lema}
\begin{proof}
    Suponemos que:
    $$f= \sum_{i=0}^n a_i x^i~~g = \sum_{j=0}^m b_j x^j \mid grd(f) = n \y grd(g) = m \Longrightarrow a_n, b_m \neq 0$$
    Por el producto de polinomios, se tiene que:
    $$fg = \sum_{k=0}^{n+m} d_k x^k \mid d_k = \sum_{i+j=k} a_i b_j$$
    Tenemos que $\displaystyle d_{n+m} = \sum_{i+j=n+m} a_ib_j = a_n b_m \neq 0$, como hemos demostrado en el lema anterior. Se tiene por tanto que $grd(fg) = n+m = grd(f) + grd(g)$.
\end{proof}

\begin{prop}
    Sea $A$ un DI. Entonces:
    $$\cc{U}(A) = \cc{U}(A[x])$$
\end{prop}
\begin{proof}
    Como $A$ es subanillo de $A[x]$, es claro que $\cc{U}(A) \subseteq \cc{U}(A[x])$. Veamos la otra inclusión.
    
    Consideramos $f \in \cc{U}(A[x]) \Longrightarrow \exists g \in \cc{U}(A[x]) \mid fg = 1$.
    $$grd(f) + grd(g) = grd(fg) = grd(1) = 0 \Longrightarrow grd(f) = 0 \y grd(g) = 0$$
    
    Por tanto, $f \in A \y g \in A \mid fg = 1 \Longrightarrow f,g \in \cc{U}(A) \Longrightarrow \cc{U}(A[x]) \subseteq \cc{U}(A)$.\\
    
    Tenemos entonces que $\cc{U}(A) = \cc{U}(A[x])$ por la doble inclusión.
\end{proof}

\begin{prop}
    Sea $A$ un anillo conmutativo \ul{finito} no trivial:
    $$A \mbox{ es DI } \Longleftrightarrow A \mbox{ es un cuerpo}$$
\end{prop}
\begin{proof}\
\begin{description}
    \item[$\Longrightarrow)$] Vista en el Lema \ref{lema:CuerpoEntoncesDominio}.

    \item[$\Longleftarrow)$] Supongamos que $A$ es DI y cogemos $a \in A \mid a \neq 0$. Consideramos la aplicación $\lambda:A\rightarrow A \mid \lambda(x) = ax~~\forall x \in A$.
    
    Es claro que $\lambda$ es inyectiva: $\forall b,c \in A \mid \lambda(b) = \lambda(c) \Longrightarrow ab = ac \stackrel{\text{$A$ DI}}{\Longrightarrow} b=c$.
    Como $A$ es finito, por la Proposición \ref{prop:ConjFinito_Equivalencias} se tiene que $\lambda$ es sobreyectiva, por lo que $Img(\lambda)=A$.

    Como $1 \in A = Img(\lambda)$, se tiene que $\exists b \in A \mid \lambda(b) = ab = 1$. Por tanto, tenemos que $ a \in \cc{U}(A)~~\forall a \in A \setminus \{0\}$; es decir, $A$ es un cuerpo.
\end{description}
\end{proof}

\section{Cuerpo de fracciones de un dominio de integridad}

Sea $A$ un DI, consideramos $A \times (A \setminus \{0\}) = \{(a,s) \mid a,s \in A,~ s \neq 0\}$.\newline
Definimos la relación binaria notada $\sim$ definida por:
$$(a,s) \sim (b,t) \Longleftrightarrow at = bs\hspace{1cm}\forall (a,s), (b,t) \in A \times (A \setminus \{0\})$$
Notemos que $\sim$ es una relación de equivalencia:
\begin{enumerate}
    \item Reflexividad:
    $$\forall (a,s) \in A \times (A \setminus \{0\})\hspace{1cm}as=as \Longrightarrow (a,s)\sim(a,s)$$

    \item Simetría:
    $$\forall (a,s), (b,t) \in A \times (A \setminus \{0\}) \mid (a,s) \sim (b,t) \Longrightarrow at = bs \Longrightarrow bs = at
    \Longrightarrow (b,t) \sim (a,s)$$

    \item Transitividad: $\forall (a,s), (b,t), (c,u) \in A \times (A \setminus \{0\}) \mid (a,s) \sim (b,t) \sim (c,u)$
    \begin{multline*}
            \left. \begin{array}{l}
            (a,s) \sim (b,t) \Longrightarrow at = bs \\
            (b,t) \sim (c,u) \Longrightarrow bu = ct
        \end{array} \right\} \Longrightarrow atu = bsu = bus = cts \Longrightarrow tau = tcs \mathop{\Longrightarrow}^{\mbox{DI}}_{t\neq0} \\
        \Longrightarrow au = cs \Longrightarrow (a,s) \sim (c,u)
    \end{multline*}
\end{enumerate}

Consideramos el conjunto cociente $A \times (A \setminus \{0\})/\sim$, que denotaremos $\Q(A)$. Para todo
$(a,s) \in A \times (A \setminus \{0\})$, su clase de equivalencia la notaremos $\dfrac{a}{s} =: [(a,s)]$ y leeremos la fracción de numerador $a$ y denominador $s$.
$$\Q(A) = \left\{\dfrac{a}{s} \mid a,s \in A \y s \neq 0 \right\}$$

\noindent
Notemos que $\dfrac{a}{s} = \dfrac{b}{t} \Longleftrightarrow (a,s) \sim (b,t) \Longleftrightarrow at = bs,\hspace{1cm}\forall a,b,s,t \in A \mid s,t \neq 0$.\\

Definimos en $\Q(A)$ una suma y un producto a partir de la suma y el producto de $A$ como:
$$\dfrac{a}{s} + \dfrac{b}{t} := \dfrac{at + bs}{st}\hspace{1cm}~~\dfrac{a}{s}\dfrac{b}{t} := \dfrac{ab}{st}\hspace{1cm}~~\forall  ~\dfrac{a}{s},
    \dfrac{b}{t} \in \Q(A)$$
    
Tenemos que ver que la suma y el producto están bien definidos, es decir, que no dependen del representante de la clase:
$$\forall ~\dfrac{a_1}{s_1} = \dfrac{a_2}{s_2}, ~\dfrac{b_1}{t_1} = \dfrac{b_2}{t_2} \in \Q(A) \Longrightarrow a_1s_2 = a_2 s_1
    \y b_1 t_2 = b_2 t_1$$

Comprobamos en primer lugar que la suma está bien definida:
$$\dfrac{a_1}{s_1} + \dfrac{b_1}{t_1} = \dfrac{a_1t_1 + b_1s_1}{s_1t_1}\hspace{1cm}~~
    \dfrac{a_2}{s_2} + \dfrac{b_2}{t_2} = \dfrac{a_2t_2 + b_2s_2}{s_2t_2}$$
Veamos si son iguales ambas fracciones:
\begin{multline*}
    (a_1t_1 + b_1s_1)s_2t_2 = a_1t_1s_2t_2 + b_1s_1s_2t_2 = a_1s_2t_1t_2 + b_1t_2s_1s_2 = \\
    =a_2s_1t_1t_2 + b_2t_1s_1s_2 = a_2t_2s_1t_1 + b_2s_2s_1t_1 = (a_2t_2 + b_2s_2)s_1t_1
\end{multline*}
Luego ambas fracciones son iguales y, por tanto, la suma está bien definida. Veamos ahora el producto:
$$\dfrac{a_1}{s_1}\dfrac{b_1}{t_1} = \dfrac{a_1b_1}{s_1t_1}\hspace{1cm}~~
    \dfrac{a_2}{s_2}\dfrac{b_2}{t_2} = \dfrac{a_2b_2}{s_2t_2}$$
Veamos si ambas fracciones son iguales:
$$(a_1b_1)(s_2t_2) = (a_1s_2)(b_1t_2) = (a_2s_1)(b_2t_1) = (a_2b_2)s_1t_1$$

Por tanto, también tenemos que el producto está bien definido.

\begin{lema}
    $\Q(A)$ es un anillo conmutativo con la suma y el producto especificados.
\end{lema}
\begin{proof}
    Para todo $\dfrac{a}{s}, \dfrac{b}{t}, \dfrac{c}{u} \in \Q(A)$, veamos que se cumplen las condiciones necesarias:
    \begin{enumerate}
        \item Asociativa de la suma:
        $$\left(\dfrac{a}{s} + \dfrac{b}{t}\right) + \dfrac{c}{u} = \dfrac{at + bs}{st} + \dfrac{c}{u} = \dfrac{u(at+bs)+cst}{stu}
            = \dfrac{uat + ubs + cst}{stu}$$
        $$\dfrac{a}{s} + \left(\dfrac{b}{t} + \dfrac{c}{u} \right) = \dfrac{a}{s}+ \dfrac{bu + ct}{tu} = \dfrac{atu + s(bu+ct)}{stu}
            = \dfrac{uat + ubs + cst}{stu}$$

        \item Conmutativa de la suma:
        $$\dfrac{a}{s} + \dfrac{b}{t} = \dfrac{at + bs}{st} = \dfrac{bs + at}{ts} = \dfrac{b}{t} + \dfrac{a}{s}$$

        \item Neutro de la suma: $\dfrac{0}{1} \in \Q(A)$.
        $$\dfrac{0}{1} + \dfrac{a}{s} = \dfrac{a\cdot 1 + s\cdot 0}{s \cdot 1} = \dfrac{a}{s}$$

        \item Existencia de opuesto:
        $$\dfrac{-a}{s} \in \Q(A) \mid \dfrac{a}{s} + \dfrac{-a}{s} = \dfrac{as + (-a)s}{s^2} = \dfrac{s(a-a)}{s^2} =
            \dfrac{a-a}{s} = \dfrac{0}{s} = \dfrac{0}{1}$$

        \item Asociativa del producto:
        $$\left(\dfrac{a}{s} \dfrac{b}{t} \right) \dfrac{c}{u} = \dfrac{ab}{st} \dfrac{c}{u} = \dfrac{abc}{stu}
        \hspace{1cm}
        \dfrac{a}{s} \left(\dfrac{b}{t} \dfrac{c}{u} \right) = \dfrac{a}{s} \dfrac{bc}{tu} = \dfrac{abc}{stu}$$

        \item Conmutativa del producto:
        $$\dfrac{a}{s} \dfrac{b}{t} = \dfrac{ab}{st} = \dfrac{ba}{ts} = \dfrac{b}{t} \dfrac{a}{s}$$

        \item Neutro del producto: $\dfrac{1}{1}\in \Q(A).$
        $$\dfrac{a}{s} \dfrac{1}{1} = \dfrac{a \cdot 1}{s \cdot 1} = \dfrac{a}{s}$$

        \item Propiedad distributiva:
        $$\dfrac{a}{s} \left( \dfrac{b}{t} + \dfrac{c}{u} \right) = \dfrac{a}{s} \dfrac{bu + ct}{tu} = \dfrac{a(bu+ct)}{stu}
            = \dfrac{abu + act}{stu}$$
        $$\dfrac{a}{s}\dfrac{b}{t} + \dfrac{a}{s}\dfrac{c}{u} = \dfrac{ab}{st} + \dfrac{ac}{su} = \dfrac{absu + stac}{stsu}
            = \dfrac{s(abu + act)}{sstu} = \dfrac{abu + act}{stu}$$
    \end{enumerate}
\end{proof}

\begin{lema}
    $\Q(A)$ es un cuerpo.
\end{lema}
\begin{proof} Tenemos que es un anillo, por lo que para ver que es un cuerpo solo falta por demostrar que todos sus elementos no nulos son unidades. Entonces, $\forall ~\dfrac{a}{s} \in \Q(A) \mid \dfrac{a}{s} \neq \dfrac{0}{1}$, consideramos $\dfrac{s}{a}$. Veamos que $\left(\dfrac{a}{s}\right)^{-1} = \dfrac{s}{a}$:
    $$\dfrac{a}{s}\dfrac{s}{a} = \dfrac{as}{sa} = \dfrac{1}{1} \Longrightarrow \dfrac{a}{s} \in \cc{U}(\Q(A))~~\forall ~\dfrac{a}{s}
        \in A \setminus\left\{\dfrac{0}{1}\right\}$$
\end{proof}

\bigskip

\noindent
Llamamos a $\Q(A)$ el \textbf{cuerpo de fracciones del anillo $A$}. Notemos que $\Q(\Z) = \Q$.

\begin{lema}
    \label{lema:QAContieneCopiaIsomorfaA}
    $\Q(A)$ contiene una copia isomorfa de $A$.
\end{lema}
\begin{proof}
    Definimos $j:A\rightarrow \Q(A) \mid j(a) = \dfrac{a}{1}~~\forall a \in A$.
    Veamos que $j$ es un homomorfismo. $\forall a,b \in A$:
    \begin{gather*}
        j(a+b) = \dfrac{a+b}{1} = \dfrac{a}{1} + \dfrac{b}{1} = j(a) + j(b) \\
        j(ab) = \dfrac{ab}{1} = \dfrac{a}{1} \dfrac{b}{1} = j(a)j(b) \\
        j(1) = \dfrac{1}{1}
    \end{gather*}
    
    Veamos ahora que $j$ es un monomorfismo. $\forall a,b \in A \mid j(a) = j(b)$:
    $$\dfrac{a}{1} = j(a) = j(b) = \dfrac{b}{1} \Longleftrightarrow a = b$$
    Hemos podido definir un monomorfismo $j:A\rightarrow \Q(A)$ luego $\Q(A)$ contiene una copia isomorfa de $A$ y nos será usual identificar $A$ con $Img(j)$. Vemos así $A$ como subanillo de $\Q(A)$.
\end{proof}

\begin{lema}
    Si $K$ es un cuerpo. Entonces $\Q(K) = K$.
\end{lema}
\begin{proof}
    Es claro que $K \subseteq \Q(K)$ por el Lema \ref{lema:QAContieneCopiaIsomorfaA}. Veamos ahora la otra inclusión. Para todo $\dfrac{a}{s} \in \Q(K)$, por definición del cuerpo de fracciones se tiene que $a,s\in K$, $s\neq 0$. Además, como $K$ es un cuerpo, consideramos $s^{-1}$. Por tanto:
    $$\dfrac{a}{s} = \dfrac{as^{-1}}{ss^{-1}} = \dfrac{as^{-1}}{1} \in K$$
    Luego $\Q(K) \subseteq K$ y por doble inclusión tenemos que $\Q(K) = K$.
\end{proof}


Por tanto, podemos utilizar en cualquier cuerpo $K$ la notación de fracciones:
$$\dfrac{a}{s} =: as^{-1}\hspace{1cm}s \neq 0$$
\begin{ejemplo}
    En $\Z_5$, se tiene que:
    \begin{equation*}
        \frac{2}{3} = 2\cdot 3^{-1} = 2\cdot 2 = 4
    \end{equation*}
\end{ejemplo}

\begin{lema}
    Si $A$ y $B$ son DI con $A$ subanillo de $B$. Entonces $\Q(A)$ es subanillo de $\Q(B)$.
\end{lema}
\begin{proof} Para todo $\dfrac{a}{s} \in \Q(A)$, se tiene que:
    $$a,s \in A \y s \neq 0 \Longrightarrow a,s \in B \y s \neq 0 \Longrightarrow \dfrac{a}{s}
        \in \Q(B)$$
\end{proof}

\begin{lema}
    Si $A$ es DI y subanillo de un cuerpo $K$, entonces $\Q(A)$ es subcuerpo (subanillo y cuerpo) de $K$.

    Es decir, $\Q(A)$ es el menor cuerpo que contiene a $A$.
\end{lema}
\begin{proof}
    $\Q(A)$ es subanillo de $\Q(K) = K$ por ser $A$ subanillo de $K$ y sabemos que $\Q(A)$ es un cuerpo.
\end{proof}

\section{Divisibilidad en un dominio de integridad}
\begin{definicion}[Divisor]
    Sea $A$ un DI y sean $a,b \in A$. Diremos que \textbf{$a$ divide a $b$} o que \textbf{$a$ es un divisor de $b$} si:
    $$\exists c \in A \mid b = ca.$$
    En tal caso, escribiremos $a|b$. Diremos también que $b$ es un \textbf{múltiplo} de $a$.

    Es decir: $\forall a,b \in A$:
    $$a|b \Longleftrightarrow \exists c \in A \mid b = ca$$
\end{definicion}

Notemos que decir que $a|b$ es decir que, si $a \neq 0,$ entonces la ecuación $ax=b$ tiene solución única en $A$:
$$ax = b \y a|b,~a\neq 0 \Longrightarrow \exists c \in A \mid b = ca \Longrightarrow ax=ca \Longrightarrow x=c$$
Notemos además que $0|b \Longleftrightarrow b=0$:
$$0|b \Longleftrightarrow \exists c \in A \mid b = c \cdot 0 \Longleftrightarrow b=0$$

\begin{notacion}
    Dado $a \in A$ con $A$ DI, notaremos por $Div(a)$ al conjunto de todos los divisores de $a$:
    $$Div(a) = \{b \in A \mid b|a \}$$
\end{notacion}

\begin{notacion}
    Sea $A$ un DI, y sea $a,b\in A$. En el caso de que $a$ no divida a $b$, lo notaremos mediante $a\not|~b$.
\end{notacion}

\begin{lema}
    Sea $A$ un DI con $a,b \in A$, $a \neq 0$:
    $$a|b \Longleftrightarrow \dfrac{b}{a} \in A$$
\end{lema}
\begin{proof}\
\begin{description}
    \item[$\Longrightarrow)$] Se tiene que, por definición, $a|b \Longleftrightarrow \exists c \in A \mid b = ca$. Por tanto,
    $$\dfrac{b}{a} = \dfrac{ca}{a} = \dfrac{c}{1} \in A$$
    \item[$\Longleftarrow)$] Suponiendo que $\dfrac{b}{a} \in A$, se tiene que:
    $$\dfrac{b}{a} \in A \Longrightarrow \exists c \in A \mid \dfrac{b}{a} = \dfrac{c}{1} \Longrightarrow b = ca \Longrightarrow a|b$$
\end{description}
\end{proof}

\begin{prop}
    Algunas propiedades que cumple la divisibilidad son, para todo $a,b,c \in A \mid a,b,c \neq 0$,
    \begin{enumerate}
        \item Reflexividad. $a|a\hspace{1cm}\forall a \in A$,
        \item Transitividad. Si $a|b \y b|c \Longrightarrow a|c$,
        \item Si $a|b \y a|c \Longrightarrow a|bx+cy\hspace{1cm}\forall x,y \in A$,
        \item $a|b \Longleftrightarrow ac|bc$.
    \end{enumerate}
\end{prop}
\begin{proof}\
    \begin{enumerate}
        \item Reflexividad. $a|a\hspace{1cm}\forall a \in A$,
        $$\forall a \in A \mid a \neq 0 \Longrightarrow \exists \ 1 \in A \mid a = a \cdot 1 \Longrightarrow a|a$$
        
        \item Transitividad. Si $a|b \y b|c \Longrightarrow a|c$,
        \begin{gather*}
            a|b \Longrightarrow \exists d \in A \mid b = da\\
            b|c \Longrightarrow \exists e \in A \mid c = eb
        \end{gather*}

        Probamos por tanto el resultado buscado:
        $$c = eb = eda \Longrightarrow \exists ed \in A \mid c = eda \Longrightarrow a|c.$$
        
        \item Si $a|b \y a|c \Longrightarrow a|bx+cy\hspace{1cm}\forall x,y \in A$,

        Como $a|b \y a|c$, tenemos que:
        \begin{gather*}
            a|b \Longrightarrow \exists d \in A \mid b = da \\
            a|c \Longrightarrow \exists e \in A \mid c = ea
        \end{gather*}
        Por tanto, tenemos que:
        $$bx+cy = dax + eay = (dx+ey)a \Longrightarrow a|bx+cy$$
        
        \item $a|b \Longleftrightarrow ac|bc$.
        \begin{description}
            \item[$\Longrightarrow)$] Como $a|b$, tenemos que $\exists d \in A \mid b=da$. Por tanto,
                $$bc = dac \Longrightarrow ac|bc$$
            \item[$\Longleftarrow)$] Partimos desde que $ac|bc$, por lo que:
                \begin{multline*}
                    ac|bc \Longrightarrow \exists d \in A \mid bc = dac \Longleftrightarrow bc-dac=0 \Longleftrightarrow c(b-da)=0 \Longleftrightarrow \\
                    \Longleftrightarrow b-da=0 \Longleftrightarrow b = da \Longrightarrow a|b
                \end{multline*}
        \end{description}
    \end{enumerate}
\end{proof}

Estaremos interesados en conocer los divisores de cualquier elemento $b \in A$ siendo $A$ un DI. En el caso de que sea $b=0$, es trivial ya que todos los elementos del anillo son sus divisores:
\begin{equation*}
    a|0 \Longleftrightarrow \exists c \in A \mid 0 = ac \Longrightarrow a|0~~\forall a \in A\mbox{ siendo } c=0.
\end{equation*}

Es decir, $Div(0)=A$. Estudiemos por tanto los cosos en los que $b\neq 0$.
\begin{lema}
    Sea $A$ un DI con $b \in A \mid b \neq 0$. Si $u \in \cc{U}(A)$. Entonces $u|b$.
\end{lema}
\begin{proof} Partiendo de que $u|b \Longleftrightarrow \exists c\in A \mid b=uc$, tenemos que
    $$b=u(u^{-1}b) \Longrightarrow u|b$$
    por lo que es cierto, tomando $c=u^{-1}b$.
\end{proof}

Así pues, $\cc{U}(A)$ es un conjunto de divisores de $b$. Es decir, $\cc{U}(A)\subset Div(b)$.

\begin{lema} Sea $A$ un DI. Si $u \in \cc{U}(A)$:
    $$a|u \Longleftrightarrow a \in \cc{U}(A)$$
Es decir: $Div(u) = \cc{U}(A)$.
\end{lema}
\begin{proof}\
\begin{description}
    \item[$\Longleftarrow)$] Por el lema anterior, tenemos que $a|u$.

    \item[$\Longrightarrow)$] $a|u \Longrightarrow \exists c \in A \mid u=ca \Longrightarrow 1=u^{-1}ca \Longrightarrow a \in \cc{U}(A)$.
\end{description}
\end{proof}

\begin{definicion}[Asociados]
    Sea $A$ un DI y $a, b\in A \mid a,b \neq 0$. Diremos que $a$ y $b$ \textbf{son asociados}, notado $a\sim b$ si:
    $$a|b \y b|a$$
\end{definicion}

\begin{lema}
    La relación binaria de ser asociados, $\sim$, es una relación de equivalencia.
\end{lema}
\begin{proof} Demostramos las tres condiciones:
\begin{enumerate}
    \item Reflexividad: $\forall a \in A \mid a \neq 0 \Longrightarrow \exists \ 1 \in A \mid a = a \cdot 1 \Longrightarrow a|a \Longrightarrow a \sim a$.

    \item Simetría: $\forall a,b \in A \mid a,b \neq 0 \y a \sim b \Longrightarrow a|b \y b|a \Longrightarrow b|a \y a|b \Longrightarrow b \sim a$.

    \item Transitividad: $\forall a,b,c \in A \mid a,b,c \neq 0$, tenemos que:
    \begin{equation*}
        \left\{
        \begin{array}{ccc}
            a\sim b &\Longrightarrow & a|b \y b|a  \\ &\land& \\
            b\sim c &\Longrightarrow & b|c \y c|a
        \end{array}
        \right\}
        \Longrightarrow
        \left\{
        \begin{array}{ccc}
            a|b \land b|c &\Longrightarrow & a|c  \\ &\land& \\
            c|b \land b|a &\Longrightarrow &c|a
        \end{array}
        \right\} \Longrightarrow a\sim c.
    \end{equation*}
\end{enumerate}
\end{proof}

\begin{prop}[Caracterización de los asociados]
    Sea $A$ un DI con $a,b \in A \mid a,b$ no nulos. Entonces:
    $$a\sim b \Longleftrightarrow \exists u \in \cc{U}(A) \mid a=ub$$
    (De donde se deduce que que $b = u^{-1}a)$.
\end{prop}
\begin{proof} \
\begin{description}
    \item [$\Longrightarrow)$] Sea $a\sim b \Longrightarrow a|b \y b|a \Longrightarrow \exists u,v \in A \mid b=ua \y a=vb$. Entonces:
    $$b=ua=uvb \Longrightarrow b(1-uv)=0 \mathop{\Longrightarrow}^{\mbox{DI}}_{b\neq0} (1-uv)=0 \Longrightarrow uv = 1
        \Longrightarrow u,v \in \cc{U}(A)$$
        
    \item [$\Longleftarrow)$] Suponemos que $\exists u \in \cc{U}(A) \mid a=ub$ Entonces, por ser $u\in \cc{U}(A)$, tenemos que $b=u^{-1}a$. Por tanto,
        $$\left. \begin{array}{rcl}
            a=ub &\Longrightarrow& b|a \\ &\y & \\
            b=u^{-1}a &\Longrightarrow& a|b
        \end{array} \right\} \Longrightarrow a\sim b.$$
\end{description} 
\end{proof}

\begin{lema}
    Sea $A$ un DI, y sean $a,b\in A\mid a,b\neq 0$ tal que $a\sim b$. Entonces, $\forall c\in A$ se cumple que:
    $$c|a \Longleftrightarrow c|b$$
    
    Es decir, $a\sim b \Longrightarrow Div(a)=Div(b)$.
\end{lema}
\begin{proof}\
    \begin{description}
        \item[$\Longrightarrow)$] $c|a \y a\sim b \Longrightarrow c|a \y a|b \Longrightarrow c|b$.
        \item[$\Longleftarrow)$] $c|b \y a\sim b \Longrightarrow c|b \y b|a \Longrightarrow c|a$.
    \end{description}
\end{proof}

Por definición, el conjunto de todos los asociados a $b$, son siempre divisores de $b$. Es decir,
$$\{a\in A \mid a\sim b\} = \{ub \in A \mid u\in \cc{U}(A)\}\subset Div(b).$$


\begin{definicion}[Divisores triviales]
    Sea $A$ un DI, dado $b \in A$, diremos que los \textbf{divisores triviales} de $b$ son las unidades y sus asociados.
    Es decir, el conjunto de los divisores triviales de $b$ es el conjunto:
    $$\cc{U}(A) \cup \{ub \mid u \in \cc{U}(A) \} \subset Div(b)$$
\end{definicion}

\begin{ejemplo} Veamos algunos ejemplos de divisores triviales:
\begin{enumerate}
    \item En $\Z$, sabemos que $\cc{U}(\Z)=\{\pm 1\}$. Entonces, dado $b \in \Z$, sus divisores triviales son: $\{-1, 1, -b, b\}$.

    \item En $\Z[i]$, sabemos que $\cc{U}(\Z[i])=\{\pm 1, \pm i\}$. Entonces, dado $\alpha \in \Z[i]$, sus divisores triviales son: $\{\pm 1, \pm i, \pm \alpha, b-ai, -b+ai\}$.
\end{enumerate}
\end{ejemplo}

\begin{definicion}[Irreducible]
    Sea $A$ un DI. Un elemento $a \in A$ diremos que es \textbf{irreducible} si $a \neq 0$, $a \notin \cc{U}(A)$ y sus únicos
    divisores son los triviales:
    $$Div(a) = \cc{U}(A) \cup \{ua \mid u \in \cc{U}(A) \}$$
\end{definicion}

\begin{ejemplo} En $\Z$, tenemos que:
\begin{itemize}
    \item El 2 es irreducible: $Div(2) = \{-1, 1, -2, 2\}$.
    \item El 4 no lo es: $Div(4) = \{-1, 1, -2, 2, -4, 4\}$.
\end{itemize}
\end{ejemplo}

\begin{prop}[Caracterización de irreducibles]
    Sea $A$ un DI, $a \in A$ tal que $a~\neq~0 \y a \notin \cc{U}(A)$. Entonces:
    $$a \mbox{ es irreducible } \Longleftrightarrow \forall b,c \in A \mid a=bc \text{ se tiene que }
    \left\{\begin{array}{c}
        b \in \cc{U}(A) \\ \o \\ c \in \cc{U}(A)
    \end{array}\right.$$
\end{prop}
\begin{proof}\
\begin{description}
    \item[$\Longrightarrow)$]Demostramos mediante reducción al absurdo.
    
    Sea $a=bc$ con $b,c \in A$. Supongamos que $b,c \notin \cc{U}(A)$. Como $a=bc$, tenemos que $b|a \y c|a$ y, por ser $a$ irreducible, se tiene que $b,c$ son divisores triviales de $a$.
    
    Como $b,c \notin \cc{U}(A) \Longrightarrow b \sim a \y c \sim a \Longrightarrow \exists u,v \in \cc{U}(A) \mid b=ua \y c=va$. Por tanto,
    $$a=bc = ua \cdot va = a^2 uv \Longrightarrow 1=auv \Longrightarrow a \in \cc{U}(A) \mbox{ \underline{Contradicción.}}$$
    
    Luego $b \in \cc{U}(A) \o c \in \cc{U}(A)$.
    
    \item[$\Longleftarrow)$] Sea $b$ un divisor de $a$, es decir, $\exists c\in A\mid a=bc$. Entonces:
    \begin{itemize}
        \item Si $b \in \cc{U}(A) \Longrightarrow b$ es divisor trivial de $a$.
        \item Si $c \in \cc{U}(A) \Longrightarrow b=c^{-1}a \Longrightarrow b\sim a \Longrightarrow b$ es divisor trivial de $a$.
    \end{itemize}
    
    Por lo que $a$ es irreducible ya que todos sus divisores son triviales.
\end{description}
\end{proof}

\section{Dominios Euclídeos}

\begin{definicion}[Dominio Euclídeo]
    Un \textbf{Dominio Euclídeo} (abreviado DE) es un DI $A$ junto con una aplicación
    $$\phi:A\setminus\{0\}\rightarrow\N$$
    Llamada \textbf{función euclídea} de $A$ tal que:
    \begin{enumerate}
        \item $\phi(ab) \geq \phi(a)\hspace{1cm}\forall a,b \in A \mid a,b \neq 0$,
        \item  $\forall a,b \in A \mid b \neq 0\hspace{1cm}\exists q,r \in A \mid a =bq+r,\mbox{ con }
        \left\{\begin{array}{c}
            r=0 \\ \lor \\ \phi(r)<\phi(b)
        \end{array}\right.
        $
    \end{enumerate}
\end{definicion}

A dichos $q$ y $r$ los llamaremos cociente y resto de dividir $a$ entre $b$, respectivamente.
\begin{observacion}
    Notemos que en la definición de DE no se exige la unicidad de $q$ y $r$.
\end{observacion}

\begin{prop}
    \label{prop:ZDE}
    $\Z$ es un DE con función euclídea $|\cdot|:\Z\setminus\{0\}\rightarrow\N$ la función valor absoluto.
\end{prop}
\begin{proof}
    Sabemos que $\Z$ es un DI, visto como consecuencia del Lema \ref{lema:SubanilloDIEntoncesDI}.

    Veamos que el valor absoluto verifica la primera condición:
    $$\forall a,b \in \Z \mid a,b \neq 0 \Longrightarrow |ab| = |a||b| \geq |a| \hspace{1cm}(b \neq 0 \Longrightarrow |b|\geq1)$$
    Sabemos que se verifica la segunda condición gracias al Teorema \ref{teo:AlgDivEuclides}.
\end{proof}

\begin{prop}
    Sea $A$ DE y $a,b \in A \mid a,b \neq 0$, son equivalentes:
    \begin{enumerate}
        \item $b|a$.
        \item Todo resto de dividir $a$ entre $b$ es 0.
        \item 0 es un resto de dividir $a$ entre $b$.
    \end{enumerate}
\end{prop}
\begin{proof}\
\begin{description}
    \item [$1) \Longrightarrow 2)$]
    $b|a \Longrightarrow \exists c \in A \mid a=bc$.
    Supogamos que $q,r \in A$ son cociente y resto de dividir $a$ entre $b$. Entonces, por la definición de DE, tenemos que
    $$a=qb+r\hspace{1cm}r=0 \o \phi(r) < \phi(b)$$

    Supongamos que $r\neq0 \Longrightarrow \phi(r)<\phi(b)$.
    $$a=bq+r \Longrightarrow r=a-bq = bc-bq = b(c-q)$$
    $$\phi(r) = \phi(b(c-q)) \mathop{\geq}^{(1)} \phi(b) \mbox{ \underline{Contradicción.}}$$

    Luego $r=0$.

    \item [$2) \Longrightarrow 3)$] Trivial, ya que $2)$ es una generalización de $3).$

    \item [$3) \Longrightarrow 1)$] Al ser $0$ un resto de dividir $a$ entre $b$, tenemos que $\exists q \in A \mid a = bq \Longrightarrow b|a$.
\end{description}
\end{proof}

\begin{teo}
    \label{teo:DividirPolinomios}
    Sea $K$ un cuerpo y $f,g \in K[x] \mid g\neq 0$.
    
    Entonces, existen únicos $q,r \in K[x] \mid f=gq+r$ con $\left\{\begin{array}{c}
            r=0 \\ \lor \\ grd(r)<grd(g)
        \end{array}\right.$
\end{teo}
\begin{proof}
    Supuesta la existencia, comenzamos demostrando su unicidad:\newline
    Sea $\left\{\begin{array}{lllll}
        f=gq+r & \text{con} & r=0 & \lor & grd(r)<grd(g) \\
        f=gq'+r' & \text{con} & r'=0 & \lor & grd(r')<grd(g')
    \end{array}\right.$

    \begin{itemize}
        \item Si $r=r' \Longrightarrow gq+r = gq'+r' \Longrightarrow gq = gq' \Longrightarrow q=q'$.

        \item Si $r \neq r'$, tenemos que alguno de los dos no es nulo. Por tanto, tenemos que $grd(r')<grd(g) \lor grd(r)<grd(g)$. Independientemente, tenemos que $ r-r' \neq 0 \y grd(r-r') < grd(g)$. Entonces:
        \begin{multline*}
            gq+r = gq'+r' \Longrightarrow r-r' = g(q'-q) \Longrightarrow grd(r-r') = grd(g(q'-q)) =\\
            =grd(g)+grd(q'-q) \geq grd(g) \quad \mbox{ \underline{Contradicción.}}
        \end{multline*}
        donde he aplicado que el grado del producto es la suma de los grados, ya que $A$ es un DI por ser un cuerpo. Por tanto, estamos en el caso anterior.
    \end{itemize}
    
    Procedemos ahora a demostrar la existencia del cociente y del resto. Sean $f,g \in K[x] \mid g\neq 0$:
    \begin{itemize}
        \item Si $f=0 \Longrightarrow q=0 \y r= 0$.
        \item Si $f \neq 0 \y grd(f) < grd(g) \Longrightarrow q=0 \y r = f$.
        \item Si $f\neq 0\y grd(f) = n \geq m = grd(g)$:
        \begin{gather*}
            f(x) = \sum_{i=0}^n a_ix^i = a_nx^n + \ldots + a_1x + a_0~~\mbox{ con } a_n \neq 0 \\
            g(x) = \sum_{j=0}^m b_jx^j = a_mx^m + \ldots + b_1x + b_0~~\mbox{ con } b_m \neq 0
        \end{gather*}
        Hacemos inducción en $n = grd(f)$:
        \begin{itemize}
            \item \ul{Si $n=0$}: $f=a_0 \neq 0$ y como $n \geq m \Longrightarrow m = 0$ y $g = b_0 \neq 0$.
            Como $K$ es un cuerpo, $\exists b_0^{-1} \in K$. Luego:
            $$q=b_0^{-1}a_0 \y r=0 \Longrightarrow gq+r = b_0 b_0^{-1} a_0 = a_0 = f$$
    
            \item \ul{Supuesto cierto para $n-1$, lo probamos para $n>0$}:
            
            Consideramos el siguiente polinomio:
            \begin{equation*}
                \begin{split}
                    f_1 &= f-b_m^{-1}a_n x^{n-m}g(x) =\\
                    & = a_nx^n + \ldots + a_1x + a_0 - b_m^{-1} a_n x^{n-m} (b_mx^m + \ldots + b_1x + b_0) =\\
                    & = f-\underbrace{a_nx^n} + b_m^{-1} a_n b_{m-1} x^{n-1} + \ldots + b_m^{-1}a_nb_1x^{n-m+1} + b_m^{-1}a_nb_0x^{n-m}
                \end{split}
            \end{equation*}
            
            Tenemos que el término $n$-ésimo de $f$ se cancela con el destacado y por tanto $grd(f_1)~\leq~n-1$.
            Por hipótesis de inducción, tenemos que:
            $$\exists q_1,r_1 \in K[x] \mid f_1 = gq_1 + r_1 \mbox{ con } r_1=0 \o grd(r_1) < grd(g)$$
    
            Por tanto, como $f_1=f-b_m^{-1}a_nx^{n-m}g = gq_1+r_1$ se tiene que:
            \begin{equation*}
                f= gq_1+r_1 + b_m^{-1}a_nx^{n-m}g =g(\underbrace{b_m^{-1}a_nx^{n-m}+q_1}_q)+r_1
            \end{equation*}
            
            Por tanto, tenemos que el cociente y el resto buscados son: $$q = b_m^{-1}a_nx^{n-m} + q_1 \hspace{1cm} r=r_1$$
        \end{itemize}
    \end{itemize}
    Por tanto, queda demostrada la existencia también.
\end{proof}

\begin{coro}
    Si $K$ es un cuerpo. Entonces $K[x]$ es un DE con función euclídea:
    $$grd:K[x]\setminus\{0\}\rightarrow\N$$
\end{coro}
\begin{proof}
    Como $K$ es un cuerpo, tenemos que $K[x]$ es un DI. Veamos que efectivamente el grado es una función euclídea:
    
    \begin{enumerate}
        \item $\forall f,g \in K[x] \mid f,g \neq 0 \Longrightarrow grd(fg) = grd(f) + grd(g) \geq grd(f)$.
        \item Existen únicos $q,r \in K[x] \mid f=gq+r$ con $r=0 \o grd(r)<grd(g)$, como vimos en el teorema anterior.
    \end{enumerate}
\end{proof}

\begin{ejemplo} Veamos algunos ejemplos de divisiones de polinomios:
\begin{enumerate}
    \item En $\Q[x]$, comprobar si el polinomio $g=2x^4-3x^2+6x+10$ es un divisor del polinomio $f=6x^6-9x^5+2x^4+15x^3+30x^2+8x+10$.
    \begin{equation*} % Opcional: centra la división larga
        \hspace{-4.3cm}
        \polylongdiv[style=C]{6x^6-9x^5+2x^4+15x^3+30x^2+8x+10}{2x^4-3x^2+6x+10}
    \end{equation*}

    Por tanto, como $r\neq0 \Longrightarrow$ $g\not|f$.

    Notemos que $3x^2=\frac{1}{2}\cdot 6 \cdot c^{6-4} = b_m^{-1}a_nx^{n-m}$ es el factor que aparece en la demostración del teorema anterior.
    
    \item En $\Z_5[x]$, comprobar si $g=3x^2+1$ es divisor de $f=2x^4+4x^3+3x+2$.
    \begin{center}
    \begin{tikzpicture}
        \matrix (a) [matrix of math nodes, column sep=0pt]
        {
             2x^4 & +4x^3 &       &   +3x &  +2 &    & 3x^2+1 \hspace{1cm} \\
            -2x^4 &       & -4x^2 &       &     &    & 4x^2+3x+2\\
                  &  4x^3 &  +x^2 &   +3x &     &    &  \\
                  & -4x^3 &       & -  3x &     &    &  \\
                  &       &   x^2 &       &  +2 &    &  \\
                  &       &  -x^2 &       &  -2 &    &  \\
                  &       &       &       &   0 &    &  \\
        };
        \draw (a-1-7.south west) -- (a-1-7.south east);
        \draw (a-1-7.north west) -- (a-2-7.south west);
        \draw (a-2-1.south west) -- (a-2-3.south east);
        \draw (a-4-2.south west) -- (a-4-4.south east);
        \draw (a-6-3.south west) -- (a-6-5.south east);
    \end{tikzpicture}
    \end{center}
    
    Por tanto, tenemos que $r=0$ y que $g|f$.
\end{enumerate}
\end{ejemplo}

\begin{teo}
    \label{teo:DEZRaizn}
    Para $n=-2,-1,2,3$, el anillo $\Z[\sqrt{n}]$ es un DE con función euclídea:
    $$\phi:\Z[\sqrt{n}]\setminus\{0\}\rightarrow\N\hspace{1cm}~~ \phi(\alpha)=|N(\alpha)|~~\forall \alpha \in \Z[\sqrt{n}]$$
\end{teo}
\begin{proof}
    Para que sea un DE, su función euclídea ha de cumplir dos condiciones:
    \begin{enumerate}
        \item $\phi(\alpha\beta)\geq \phi(\alpha), \qquad \forall \alpha,\beta \in \Z[\sqrt{n}]\setminus \{0\}$.
        $$\phi(\alpha\beta) = |N(\alpha\beta)| = |N(\alpha)N(\beta)| = |N(\alpha)||N(\beta)| \geq |N(\alpha)| = \phi(\alpha)$$

        \item $\exists q,r\in \Z[\sqrt{n}]\mid \alpha=q\beta + r$, con $r=0\o \phi(r)<\phi(\beta)$. Realizamos la siguiente distinción de casos:
        \begin{itemize}
            \item Si $\phi(\alpha) = |N(\alpha)| < |N(\beta)| = \phi(\beta) \Longrightarrow q=0 \y r=\alpha$.
            \item Si $\phi(\alpha)\geq \phi(\beta)$:

            Sean $\alpha=a+b\sqrt{n}, \beta=c+d\sqrt{n}\neq 0$. Como $\beta\neq 0$, entonces $N(\beta)\neq 0$. Por tanto, en $\Q[\sqrt{n}]$ consideramos el sguiente elemento:
            \begin{equation*}\begin{split}
                \alpha \cdot \beta^{-1} &= \dfrac{\alpha}{\beta} = \dfrac{\alpha \cdot \overline{\beta}} {\beta \cdot \overline{\beta}} = \dfrac{(a+b\sqrt{n})(c-d\sqrt{n})}{N(\beta)} =\\
                & =\dfrac{(ac-nbd)+(cb-ad)\sqrt{n}}{N(\beta)} = \dfrac{ac-nbd}{N(\beta)} + \dfrac{cb-ad}{N(\beta)}\sqrt{n}
            \end{split}\end{equation*}

            Sea $a_1 := \dfrac{ac-nbd}{N(\beta)}$ y $a_2 := \dfrac{cb-ad}{N(\beta)}$; es decir, $\dfrac{\alpha}{\beta} = a_1 + a_2\sqrt{n} \in \Q[\sqrt{n}]$.

            Elegimos $q_1$, $q_2 \in \Z \mid |a_1-q_1| \leq \dfrac{1}{2}$ y $|a_2-q_2| \leq \dfrac{1}{2}$. Sea $q=q_1+q_2\sqrt{n} \in \Z[\sqrt{n}]$ y $r:=\alpha -q\beta \in \Z[\sqrt{n}]$.
            
            Se tiene entonces que hemos encontrado $q,r \in \Z[\sqrt{n}] \mid \alpha = q\beta + r$. Falta ver que $r=0 \o \phi(r) < \phi(\beta)$.\\

            Supongamos que $r\neq0$, y trabajamos en $\Q[\sqrt{n}]$:
            \begin{equation*}
                \begin{split}
                    \phi(r) &= |N(r)|
                    = |N(\alpha - q\beta)|
                    = \left|N\left(\beta\left( \dfrac{\alpha}{\beta} - q\right)\right)\right|
                    = \left|N(\beta) N\left( \dfrac{\alpha}{\beta}-q \right)\right| = \\
                    & = |N(\beta)|\cdot \left|N\left( \dfrac{\alpha}{\beta}-q \right)\right|
                    = |N(\beta)|\cdot |N[(a_1-q_1)+(a_2-q_2)\sqrt{n}]| =\\
                    &= |N(\beta)|\cdot |(a_1-q_1)^2-n(a_2-q_2)^2|
                \end{split}
            \end{equation*}
            
            Sea $A = (a_1-q_1)^2-n(a_2-q_2)^2$, y realizamos la distinción de casos siguiente:
            \begin{itemize}
                \item Si $n=-2$:
                $$A=(a_1-q_1)^2+2(a_2-q_2)^2 \leq \left(\dfrac{1}{2}\right)^2+2\left(\dfrac{1}{4}\right)^2 = \dfrac{3}{4} \Longrightarrow |A|<1$$

                \item Si $n=-1$:
                $$A=(a_1-q_1)^2+(a_2-q_2)^2 \leq \left(\dfrac{1}{2}\right)^2+ \left(\dfrac{1}{2}\right)^2 = \dfrac{1}{2} \Longrightarrow |A|<1$$

                \item Si $n=2$:
                $$A=(a_1-q_1)^2-2(a_2-q_2)^2 \leq \left(\dfrac{1}{2}\right)^2 - \dfrac{1}{2} \Longrightarrow \dfrac{-1}{2} \leq A \leq \dfrac{1}{4} \Longrightarrow |A|<1$$
            
                \item Si $n=3$:
                $$A=(a_1-q_1)^2-3(a_2-q_2)^2 \leq \left(\dfrac{1}{2}\right)^2 - 3\left(\dfrac{1}{2}\right)^2 \Longrightarrow \dfrac{-3}{4} \leq A \leq \dfrac{1}{4} \Longrightarrow |A|<1$$
            \end{itemize}

            Por tanto, en dichos casos tenemos que $|A|<1$. Por tanto,
            $$\phi(r) = |N(\beta)||A| \leq |N(\beta)| = \phi(\beta)$$
        \end{itemize}
    \end{enumerate}
\end{proof}

\begin{ejemplo}
    Dividir $\alpha=11+7i$ entre $\beta=2i$ en $\Z[i]$.\\
    
    Buscamos $q,r \in \Z[i] \mid \alpha = q\beta +r$ con $r=0 \o \phi(r) < \phi(\beta)$. Consideramos en $\Q[i]$ el siguiente elemento:
    $$\dfrac{\alpha}{\beta} = \dfrac{11+7i}{2i} = \dfrac{(11+7i)(-2i)}{4} = \dfrac{14-22i}{4} = \dfrac{7}{2} - \dfrac{11}{2}i$$
    Elegimos $q_1, q_2 \in \Z \mid \left|\dfrac{7}{2}-q_1\right|\leq \dfrac{1}{2} \y \left|\dfrac{11}{2}-q_2\right|\leq \dfrac{1}{2}$.
    \begin{enumerate}
        \item Tomamos, por ejemplo $q_1=3 \y q_2 = -5 \Longrightarrow q=3-5i$.
        \begin{gather*}
            r=\alpha-q\beta = (11-7i)-2i(3-5i) = 1+i \\
            11+7i = q\beta + r \y \phi(r) = N(1+i) = 2 < 4 = N(2i) = \phi(\beta)
        \end{gather*}

        \item Podemos elegir también $q_1'=4 \y q_2'=-6 \Longrightarrow q' = 4-6i$.
        \begin{gather*}
            r'=\alpha-q'\beta = (11-7i)-2i(4-6i) = -1-i \\
            11+7i = q'\beta + r' \y \phi(r) = N(-1-i) = 2 < 4 = N(2i) = \phi(\beta)
        \end{gather*}
    \end{enumerate}

    Por tanto, acabamos de ver que el cociente y el resto en $\Z[i]$ no son únicos.
\end{ejemplo}



\section{Máximo Común Divisor}
\begin{definicion}[Máximo común divisor]
    Sea $A$ un DI y $a,b \in A$. Un \textbf{máximo común divisor} de $a$ y $b$, notado $\mcd(a,b)$ es $d \in A$ tal que:
    \begin{enumerate}
        \item $d|a \y d|b$,
        \item Si $c\in A \mid c|a \y c|b \Longrightarrow c|d$.
    \end{enumerate}
\end{definicion}

De la definición se tiene por tanto que:
$$Div(d) = Div(a) \cap Div(b)$$

\begin{observacion}
    No siempre existe el máximo común divisor de dos elementos.
    
    Además, si existe no es único. Para verlo, sea $d=\mcd(a,b)$, y consideramos $du\mid u\in \cc{U}(A)$. Entonces:
    \begin{enumerate}
        \item Como $du|d$, y se tiene que $d|a\y d|b$, por transitividad se tiene que $du|a\y du|b$.
        \item Análogamente, si $c\in A$ cumple que $c|d$ por ser este el $\mcd(a,b)$, la transitividad afirma que $c|du$ al tener que $d|du$.
    \end{enumerate}

    Por tanto, hemos visto que $du=\mcd(a,b)$. De forma general, tenemos que el $\mcd$ de dos elementos, si existe, es único salvo asociados. Hablaremos simplemente del $\mcd$ de $a$ y $b$.
\end{observacion}

\begin{definicion}[Máximo común divisor generalizado]
    Sea $A$ un DI y consideramos $a_1, a_2, \ldots, a_n \in A (n \geq 2)$. Un máximo común divisor de $a_1, a_2, \ldots, a_n$, notado como $\mcd(a_1, a_2, \ldots, a_n)$ es $d \in A$ tal que verifica:
    \begin{enumerate}
        \item $d|a_i~~\todoi $,
        \item Si $c\in A \mid c|a_i~~\todoi \Longrightarrow c|d$.
    \end{enumerate}
\end{definicion}


\begin{definicion}[Primos relativos]
    Sean $a,b \in A$. Diremos que $a$ y $b$ son \textbf{primos relativos} si $\mcd(a,b)=1$.
\end{definicion}

\textbf{Propiedades del $\mcd$.} Sea $A$ DI, supuesta la existencia de los $\mcd$ que intervienen, tenemos que:
\begin{enumerate}
    \item $\mcd(a,b) = \mcd(b,a)$.
    En general: $$\mcd(a_1, \ldots, a_i, a_{i+1}, \ldots, a_n) = \mcd(a_1, \ldots, a_{i+1}, a_i, \ldots, a_n)~~~n\geq2,~i \in \{1, \ldots, n-1\}$$

    La demostración es trivial, ya que en la definición no se establece ningún orden.

    \item Si $a\sim a'$, entonces $\mcd(a,b)=\mcd(a',b)$.

    Sea $d = \mcd(a,b)$. Entonces, $d|a$. Además, por ser $a\sim a'$, se tiene que $a|a'$. Por la transitividad, $d|a'$. Además, se tiene también que $d|b$, por lo que la primera condición se tiene.

    Veamos la segunda. Como $a\sim a'$, tenemos que $c|a\Longleftrightarrow c|a'$, por lo que $\forall c\in A$ se tiene que si $c|a'\land c|b$, entonces $c|a\land c|b$; y por tanto $c|d$. Por tanto, se deduce que 
    $d=\mcd(a',b)$.

    \item $\mcd(a,b)=a \Longleftrightarrow a|b$. Particularmente, $\mcd(a,0) = a \y \mcd(a,1)=1$.
    \begin{description}
        \item[$\Longrightarrow)$] $\mcd(a,b) = a \Longrightarrow a|a \y a|b$.

        \item[$\Longleftarrow)$] Partimos de que $a|b$. Además, por la reflexividad se tiene que $a|a$. Entonces, se tiene la primera condición para que sea $a=\mcd(a,b)$.

        Además, como $\forall c \in A \mid c|a \y c|b$ se tiene que $c|a$, tenemos también la segunda condición, por lo que $a=\mcd(a,b)$.
    \end{description}


    \item $\mcd(\mcd(a,b),c) = \mcd(a,b,c) = \mcd(a, \mcd(b,c))$.

    Demostramos en primer lugar la primera igualdad. Sea $d=\mcd(\mcd(a,b),c)$. Por tanto, $d|\mcd(a,b)\land d|c$. Además, como $\mcd(a,b)|a$ y $\mcd(a,b)|b$, por la transitividad se tiene que $d|a\land d|b$, por lo que se tiene la primera condición para que $d=\mcd(a,b,c)$.

    Por la segunda condición, tenemos que $\forall z\in A$ tal que $z|\mcd(a,b)\land z|c$, se tiene que $z|d$. Como $z|\mcd(a,b)$, por la transitividad se tiene que $z|a$ y $z|b$, por lo que $\forall z\in A$ tal que $z|a\land z|b \land z|c$, se tiene que $z|d$. Por tanto, se tiene también la segunda condición y tenemos que $d=\mcd(a,b,c)$.\\

    Una vez demostrada la primera igualdad, para la segunda tenemos que:
    $$\mcd(a,b,c) = \mcd(b,c,a) = \mcd(\mcd(b,c), a) = \mcd(a, \mcd(b,c))$$

    \item $\mcd(ac,bc) = \mcd(a,b)c$.

    En el que caso de que alguno de los tres sea nulo es trivial, por lo que supones $a,b,c\neq 0$. Sea $d=\mcd(a,b)$ y $e=\mcd(ac,bc)$. Tenemos que:
    \begin{equation*}
        d=\mcd(a,b) \Longrightarrow \left\{
        \begin{array}{ccc}
            d|a & \Longrightarrow & dc|ac  \\
            &\land& \\
            d|b & \Longrightarrow & dc|bc  \\
        \end{array}
        \right\} \Longrightarrow dc|e
    \end{equation*}
    Como $dc|e$, supongamos $e=dcu$. Además, 
    \begin{equation*}
        e=dcu=\mcd(ac,bc) \Longrightarrow \left\{
        \begin{array}{ccc}
            dcu|ac & \Longrightarrow & du|a  \\
            &\land& \\
            dcu|bc & \Longrightarrow & du|b  \\
        \end{array}
        \right\} \Longrightarrow du|d
    \end{equation*}
    Como $du|d$, supongamos ahora $d=duv$. Por tanto, tenemos que $uv=1$ y, entonces, $u,v\in \cc{U}(A)$. Esto implica que $dc\sim e=\mcd(ac,bc)$, y como el $\mcd$ es único salvo asociados, tenemos la igualdad pedida.

    \item Si $\mcd(a,b)=d$ y $a=da',~ b=db'$, entonces $\mcd(a',b')=1$.
    $$d = \mcd(a,b) = \mcd(a'd, b'd) = \mcd(a',b')d \mathop{\Longrightarrow}^{\mbox{DI}} 1=\mcd(a',b')$$

    \item Si $a|bc \y \mcd(a,b)=1$, entonces $a|c$.

    Tenemos que $a|bc$ implica que $\exists t \in A \mid bc=at$. Entonces:
    $$c= c \cdot 1 = c \cdot \mcd(a,b) = \mcd(ac, bc) = \mcd(ac, at) = a \cdot \mcd(c,t) \Longrightarrow a|c$$

    \item Si $\mcd(a,b)=1 \y a|c \y b|c$, entonces $ab|c$.

    Como $b|c$, $\exists x\in A \mid c=bx$. Como $a|bx$ y $\mcd(a,b)$, por la propiedad anterior tenemos que $a|x$, es decir, $\exists y\in A\mid x=ay$. Por tanto, tenemos que $c=aby$, por lo que $ab|c$.

    \item $\mcd(a,b)=1 \y \mcd(a,c)=1 \Longleftrightarrow \mcd(a,bc)=1$.
    \begin{description}
        \item[$\Longrightarrow)$] Tenemos que $c=c\mcd(a,b)=\mcd(ac,bc)$. Entonces:
        \begin{equation*}
            1=\mcd(a,c)=\mcd(a,\mcd(ac,bc)) = \mcd(\mcd(a,ac), bc) = \mcd(a,bc)
        \end{equation*}

        \item[$\Longleftarrow)$] Partimos de que:
        \begin{equation*}
            1=\mcd(a,bc) = \mcd(\mcd(a,ac), bc) = \mcd(a,\mcd(ac,bc)) = \mcd(a,\mcd(a,b)c)
        \end{equation*}

        Como $\mcd(a,b)$ es  un divisor común a $a$ y a $\mcd(a,b)c$, tenemos que $\mcd(a,b)$ es un divisor de $1$; es decir, una unidad. Por tanto, $\mcd(a,b)=1$. De la primera igualdad deducimos entonces que $1=\mcd(a,\mcd(a,b)c) = \mcd(a,c)$.
    \end{description}

    \item $\mcd(a,b)=\mcd(a-qb, b)~~\forall q \in A$.

    Veamos que $Div(a)\cap Div(b)=Div(a-qb)\cap Div(b)$. Demostramos por doble inclusión:
    \begin{itemize}
        \item Sea $c\in A$ tal que $c|a\land c|b$. Entonces, por las propiedades de la divisibilidad tenemos que $c|a-qb\land c|b$. Por tanto, la primera inclusión se tiene.

        \item Sea $c\in A$ tal que $c|a-qb\land c|b$. Entonces, se tiene que $c|(a-qb)+qb$; es decir, $c|a$. Se tiene por tanto la otra inclusión.
    \end{itemize}
    Como tienen los mismos divisores, tienen el mismo $\mcd$.

    \item Sea $p\in A$ irreducible. Entonces, 
    $\mcd(p,a) = \left\{ \begin{array}{lll}
            p & \mbox{si} & p|a,      \\
            1 & \mbox{si} & p\not|~a.
        \end{array} \right.$

    Para demostrarlo, realizamos la siguiente división de casos:
    \begin{itemize}
        \item Si $\displaystyle p|a \mathop{\Longrightarrow}^{3)} \mcd(p,a)=p$.

        \item Si  $p\not|~a$, supongamos que $c=\mcd(p,a)$. Por las propiedades del $\mcd$, ha de cumplirse que $c|p$. No obstante, por ser $p$ irreducible tenemos que $c\in \cc{U}(A)\lor c=up,~u\in \cc{U}(A)$.
        \begin{itemize}
            \item Si $c\in \cc{U}(A)\Longrightarrow c=1$, ya que el $\mcd$ es único salvo asociados.
            \item Si $c=up\mid u\in \cc{U}(A)$, tenemos que $p|c$. Por ser $c=\mcd(p,a)$, también $c|a$. Por tanto, por la transitividad de la divisibilidad se tiene que $p|a$, por lo que llegamos a una \ul{contradicción}.
        \end{itemize}
        Por tanto, estamos en el primer caso y $\mcd(a,p)=1$.
    \end{itemize}
\end{enumerate}

\begin{ejemplo}
    Veamos en este ejemplo que en $\Z[\sqrt{-5}]$, $\nexists \mcd(2+2\sqrt{-5}, 6)$.\\

    Es importante recordar que $\cc{U}(\Z[\sqrt{-5}]) = \{-1, 1\}$. Para demostrar lo pedido, vamos a ver previamente algunos resultados:
    \begin{enumerate}
        \item $3$ es irreducible en $\Z[\sqrt{-5}]$:

        Sabemos que $3\neq0 \y 3 \notin \cc{U}(\Z[\sqrt{-5}])$, por lo que puede ser irreducible. Veamos cuáles son sus divisores. Sea $\alpha \in \Z[\sqrt{-5}] \mid \alpha|3 \Longrightarrow \exists \beta \in \Z[\sqrt{-5}] \mid 3=\alpha \cdot \beta$. Por tanto, aplicando la norma, tenemos que:
        $$N(3) = 9 =  N(\alpha)N(\beta)$$
        Veamos las distintas posibilidades que hay:
        \begin{itemize}
            \item Si $N(\alpha)=1 \y N(\beta) = 9$, entonces $\alpha \in \cc{U}(\Z[\sqrt{-5}])$.
            \item Si $N(\alpha) = 3 = N(\beta)$, tenemos que $a^2+5b^2=3$, con $a,b\in \Z$. Por tanto, como no es posible encontrar $a,b$, entonces es una contradicción, no es posible.
            \item Si $N(\alpha)=9 \y N(\beta) = 1$, entonces $\beta \in \cc{U}(\Z[\sqrt{-5}])$, y como $3=\alpha\beta$, entonces $\alpha\sim 3$
        \end{itemize}
        Por tanto, sus únicos divisores son los triviales y, por tanto, el $3$ es irreducible.

        \item Usando la propiedad 11), se tiene que $\exists \mcd(3, 1+\sqrt{-5})=1$ porque $3$ es irreducible y $3\not|~1+\sqrt{-5}$. Veamos esto último.

        Si $3$ fuese divisor de $1+\sqrt{-5}$, entonces $N(3)=9$ sería un divisor de $N(1+\sqrt{-5})=6$ en $\Z$, pero sabemos que, en $\Z$, $9\not|~6$, por lo que es una \ul{contradicción}.
    \end{enumerate}

    Demostramos ahora lo pedido mediante reducción al absurdo. Supongamos que $\exists \mcd(2+2\sqrt{-5}, 6)$:
    $$\mcd(2+2\sqrt{-5},6)=2\mcd(1+\sqrt{-5}, 3)=2 \text{ por ser $3$ irreducible.}$$

    No obstante, no verifica la segunda condición del $\mcd$, ya que $1+\sqrt{-5}|2+2\sqrt{-5}$ y $1+\sqrt{-5}|6$. Esto se debe a que:
    \begin{equation*}
        2+2\sqrt{-5} = 2(1+\sqrt{-5}) \hspace{1cm} 6=(1+\sqrt{-5})(1-\sqrt{-5})
    \end{equation*}
    
    Sin embargo, $1+\sqrt{-5}\not|2$, porque $N(1+\sqrt{-5})=6\not|~N(2)=4$.
    
    Por tanto, tenemos que $\nexists \mcd(2+2\sqrt{-5},6)$.
\end{ejemplo}

\begin{definicion}[Ideal]
    Sea $A$ un anillo conmutativo, un subconjunto $I \subseteq A$, con $I \neq \emptyset$, diremos que es un \textbf{ideal} de $A$ si verifica:
    \begin{enumerate}
        \item Es cerrado para la suma: $\forall x,y \in I \Longrightarrow x+y \in I$.
        \item Es cerrado para múltiplos: $\forall x \in I ~\forall a \in A \Longrightarrow ax \in I$.
    \end{enumerate}
\end{definicion}

\begin{teo}[Ideal principal]
    Sea $A$ un anillo conmutativo y $m \in A$. Definimos el \textbf{ideal principal generado por el elemento $m$} como el ideal:
    $$I = mA := \{ma \mid a \in A\}$$
\end{teo}
\begin{proof}
    Veamos que $mA$ es un ideal:
    \begin{enumerate}
        \item Sea $x,y\in mA$. Entonces, $\exists a,b \in A \mid x=ma \y y=mb$. Por tanto,
        \begin{equation*}
            x+y = ma+mb = m(a+b) \in mA
        \end{equation*}

        \item Sea $x\in mA$. Entonces, $\exists a\in A\mid x=am$. Veamos que es cerrado para el producto:
        \begin{equation*}
            bx = bam = m(ab)\in mA, \hspace{1cm} \forall b\in A
        \end{equation*}
    \end{enumerate}
\end{proof}

Notemos que:
\begin{itemize}
    \item Si $m=0 \Longrightarrow 0A = \{0\}$, es un ideal trivial de $A$.
    \item Si $m=1 \Longrightarrow 1A = A$, es un ideal.
    \item Si $I$ es cualquier otro ideal de $A$, entonces $\{0\} \subseteq I \subseteq A$.
\end{itemize}

\begin{teo}
    \label{teo:TodoIdealPrincipalDE}
    Si $A$ es un DE, entonces todo ideal de $A$ es principal.
\end{teo}
\begin{proof}
    Sea $\phi:A\setminus\{0\}\rightarrow\N$ la función euclídea de $A$ y $I$ un ideal de $A$. Si $I=\{0\}$ tenemos que $I = 0A$ y está demostrado. Consideramos por tanto $I \neq \{0\}$, y definimos $X = \{\phi(x) \mid x \in I \y x \neq 0\} \subseteq \N$.
    
    Como $\N$ es bien ordenado, sabemos que $\exists \min(X)$. Sea $m \in I \mid \phi(m) = \min(X)$. Veamos que $mA = I\hspace{1cm} (\mbox{sabemos que } m\neq 0$ porque $m\in X)$:\newline
    Como $m \in I$, entonces por ser $I$ un ideal tenemos $ma \in I ~\forall a \in A$. Entonces, $mA \subseteq I$. Veamos la otra inclusión:
    
    Sea $x \in I$, sabemos que $ \exists q,r \in A \mid x=mq+r$ con $r=0 \o \phi(r) < \phi(m)$. Entonces:
    \begin{itemize}
        \item Supongamos que $\displaystyle r\neq 0$. Entonces, como $\phi(r) < \phi(m)$ y $\phi(m) = \min(X)$, tenemos que $r\notin I$.

        Además, por ser $r\neq 0$, tenemos que $r=x-mq$, que por las propiedades de los ideales tenemos que $r\in I$. Por tanto, llegamos a una \ul{contradicción}.
        
        \item Supongamos $r=0$, es decir, $x=mq\in mA$.
    \end{itemize}

    Por tanto, al ser el primer caso una contradicción, tenemos el segundo caso. Por tanto, por doble inclusión, tenemos que $I=mA$.
\end{proof}

\begin{coro} \label{cor:DEExistemcdyBezout}
    Sea $A$ un DE y $a,b \in A$. Entonces $\exists \mcd(a,b)$.
    
    Además, si $d=\mcd(a,b)$, entonces $\exists u,v \in A$ tal que
    \begin{equation}\label{ec:IdBezout}
        \mcd(a,b) = d = au + bv
    \end{equation}
    
    A $u$, $v$ se les llama \textbf{coeficientes de Bezout}, siendo la Ecuación \ref{ec:IdBezout} la \textbf{identidad de Bezout}.
\end{coro}
\begin{proof}
    Consideramos $I = \{ax+by \mid x,y \in A\}$ y tenemos que $a, b\in I$, puesto que $\left\{\begin{array}{l}
         a = a\cdot 1 + b \cdot 0 \in I \\
         b = a\cdot 0 + b \cdot 1 \in I
    \end{array}\right\}$. Veamos que $I$ es un ideal de $A$.
    \begin{enumerate}
        \item Sean $w=ax_1 +by_1, z=ax_2+by_2\in A$. Entonces:
        $$ w+z = ax_1 + by_1 + ax_2 + by_2 = a(x_1 + x_2) + b(y_1+y_2) \in I$$

        \item Sea $w=ax+by$, $c \in A$. Entonces:
        $$w = c(ax+by) = cax + cby = a(cx) + b(cy) \in I$$
    \end{enumerate}
    
    Entonces, como $A$ es un Dominio Euclídeo, por el Teorema \ref{teo:TodoIdealPrincipalDE}, tenemos que $\exists d \in I \mid I = dA$.
    Notemos que $d\in I$, por lo que $\exists u,v \in A \mid d=au+bv$. Queda por tanto demostrada la Identidad de Bezout. Veamos ahora que $d=\mcd(a,b)$:
    \begin{enumerate}
        \item Como $a,b \in I =dA = \{dx \mid x \in A\}$, entonces $\exists a', b' \in A \mid a=da' \y b=db'$, por lo que $d|a \y d|b$.

        \item Sea $c \in A \mid c|a \y c|b \Longrightarrow \exists a'', b'' \in A \mid a=ca'' \y b=cb''$
        $$d=au+bv = ca''u + cb''v = c(a'' u + b''v) \Longrightarrow c|d$$  
    \end{enumerate}
    
    Por tanto, tenemos que $d = \mcd(a,b)$.
\end{proof}

\subsection{Algoritmo extendido de Euclídes}

Sea $A$ un DE con $a$, $b \in A$ siendo $\phi:A\setminus\{0\}\rightarrow\N$ la función euclídea de $A$. Nuestro objetivo es calcular $\mcd(a,b)$ y $u,v \in A \mid \mcd(a,b) = au+bv$, los coeficientes de Bezout de $a$ y $b$.

Realizamos la siguiente división por casos:
\begin{enumerate}
    \item Si ambos son cero: $a=b=0 \Longrightarrow \mcd(a,b)= 0 $ y $0 = 0\cdot u + 0 \cdot v~~\forall u,v \in A$.
    \item Si uno es cero (por ejemplo, $a=0)$, entonces $\mcd(0,b) = b$ y $b=a\cdot 0 + b \cdot 1$.

    \item Si $a \neq 0 \neq b$ y suponemos $($lo cual no es restrictivo ya que $\mcd(a,b)=\mcd(b,a))$ $\phi(a) \geq \phi(b)$:\\


    Construimos una sucesión $r_1, r_2, \ldots, r_i$ de elementos de $A$ de forma recurrente. Partimos de 
    $$r_1 = a \y r_2 = b$$
    Para el resto de valores, si $r_i \neq 0$, entonces:
    $$r_{i+1}= R(r_{i-1};r_i)~~\forall i \in \{2, \ldots, n\}$$

    Consideramos ahora la sucesión $\{\phi(r_i)\}$. Tenemos que es estrictamente decreciente, ya que por la definición de función euclídea:    
    $$\phi(r_1) \geq \phi(r_2) > \phi(r_3) > \ldots > \phi(r_i) > \ldots$$

    Por tanto, al ser una sucesión en $\bb{N}$ estrictamente decreciente y minorada por el $0$, tenemos que $\exists n\in \bb{N}\mid r_{n+1}=0$. Demostremos a continuación que $r_n = \mcd(a,b)$, es decir, que $\mcd(a,b)$ es el último resto no nulo. Para ello, demostramos por inducción que $\forall i \in \bb{N}$ se tiene que $\mcd(a,b) = \mcd(r_i, r_{i+1})$:
    \begin{itemize}
        \item Si $i=1 \Longrightarrow r_1 = a$, $r_2 = b \Longrightarrow \mcd(a,b) = \mcd(a,b)$, Cierto.
        \item Supongamos que $i>1$ y que $\mcd(a,b) = \mcd(r_{i-1}, r_i)$:
        $$\mcd(a,b) \stackrel{HI}{=} \mcd(r_{i-1}, r_i) \stackrel{10)}{=} \mcd(r_{i-1}-r_iq_i, r_i) = \mcd(r_{i+1}, r_i) = \mcd(r_i, r_{i+1})$$
    \end{itemize}

    Por tanto, para $i=n$ tenemos que: $$\mcd(a,b) = \mcd(r_n, r_{n+1}) = \mcd(r_n, 0) = r_n$$
    es decir, que el $\mcd$ es el último resto no nulo.\\
    
    A continuación, veamos que $\forall i \in \{1, \ldots, n\}$, $\exists u_i, v_i \mid r_i = au_i + bv_i$. Hacemos inducción en $i$:
    \begin{itemize}
        \item Si $i=1 \Longrightarrow r_1 = a \Longrightarrow u_1 = 1 \y v_1 = 0$
        \item Si $i=2 \Longrightarrow r_2 = b \Longrightarrow u_2 = 0 \y v_2 = 1$
        \item Supongamos que $i>1$ y que $r_{i-1} = au_{i-1}+bv_{i-1} \y r_i = au_i + bv_i$:
        $$r_{i+1} = r_{i-1}-q_ir_i = au_{i-1}+bv_{i-1} - q_i(au_i +bv_i) = a(v_{i-1}-qv_i)+b(v_{i-1}-q_iv_i)$$
        Por tanto, $u_{i+1} = u_{i-1}-q_iu_i \y v_{i+1} = v_{i-1}-q_iv_i$.
    \end{itemize}
\end{enumerate}

En particular, $\mcd(a,b) = r_n =au_n + bu_n$.

\begin{ejemplo}
    Trabajamos en $\Z$ y queremos hallar $\mcd(80,30)$ y su identidad de Bezout.
    $$\begin{array}{c|c|c|c||cl}
        i & r_i & u_i & v_i && \\
        \hline
        1 & 80  & 1   & 0 &&   \\
        2 & 30  & 0   & 1 &&  80=30\cdot 2 + 20\\
        3 & 20  & 1   & -2 && 30 = 20 \cdot 1 + 10\\
        4 & 10  & -1  & 3  &&  20 = 10\cdot 2 + 0\\
        5 & 0   &     &&& 
    \end{array}$$
    Por tanto, tenemos que $\mcd(80,30) = 10 = -1 \cdot 80 + 3 \cdot 30$.
    
    Las operaciones para hallar los coeficientes de bezout han sido:
    \begin{align*}
        & u_3 = u_1 - q_2u_2 = 1- 2 \cdot 0 = 1
        && u_4 = u_2 - q_3u_3 = 0 - 1 \cdot 1 = -1\\
        & v_3 = v_1 - q_2v_2 = 0 - 2\cdot 1 = 2 
        && v_4 = v_2 - q_3v_3 = 1 - 1\cdot(-2) = 3
    \end{align*}
\end{ejemplo}


\section{Ecuaciones Diofánticas}
\begin{definicion}[Ecuación diofántica]
    Llamamos \textbf{ecuación diofántica} en un DI $A$ a cualquier ecuación del tipo:
    $$ax + by = c, \qquad a,b,c\in A,~a,b\neq 0$$
    siendo $x$ e $y$ las incógnitas a buscar.\\

    
    Decimos que una solución a la ecuación diofántica anterior es $(x_0, y_0)\in A\times A$ si:
    $$ax_0 + by_0 = c$$
\end{definicion}

\begin{teo}
    Sea $A$ un DE y consideramos la ecuación diofántica:
    $$ax + by = c$$
    Con $a,b,c \in A \mid a,b\neq 0$ y con incógnitas $x$ e $y$. Sea $d = \mcd(a,b)$:
    \begin{enumerate}
        \item La ecuación tiene solución $\Longleftrightarrow d|c$.
        \item Si $(x_0, y_0)$ es una solución particular de la ecuación, entonces las demás soluciones son del tipo:
        $$x=x_0 + k\cdot\dfrac{b}{d}\hspace{1cm}y = y_0 - k\cdot \dfrac{a}{d}\hspace{1cm}\forall k \in A$$
    \end{enumerate}
\end{teo}
\begin{proof}
    Sea $d = \mcd(a,b)$, y sean $a = da'$, $b = db'$, con $a',b' \in A$ tal que $\mcd(a', b') = 1$. Entonces:
    \begin{enumerate} 
        \item Demostramos mediante doble implicación:
        \begin{description}
            \item[$\Longrightarrow)$] Supongamos que $\exists x_0, y_0 \in A \mid ax_0 + by_0 = c$. Entonces:
            $$c = da'x_0 + db'y_0 = d(a'x_0 + b'y_0) \Longrightarrow d|c$$

            \item[$\Longleftarrow)$] Supongamos que $d|c \Longrightarrow \exists c' \in A \mid c = dc'$. Entonces:
            $$ax+by=c \Longleftrightarrow da'x+db'y = dc' \Longleftrightarrow a'x+b'y=c'$$
            Como $\mcd(a',b')=1$, buscamos $u, v \in A \mid 1 = a'u + b'v$ con $u,v \in A$ los coeficientes de Bezout.
            $$1=a'u+b'v \Longleftrightarrow c' = a'uc' + b'vc' \Longrightarrow (x_0 = uc', y_0 = vc') \mbox{ es solución.}$$
        \end{description}

        \item Como la ecuación tiene solución $\Longrightarrow d|c \Longrightarrow \exists c' \mid c=dc'$:
        $$ax+by = c \Longleftrightarrow da'x + db'y = dc' \Longleftrightarrow a'x + b'y = c'$$
        Sea $(x_0, y_0)$ una solución, y consideramos $k \in A$:
        $$a'(x_0 + kb') + b'(y_0-ka') = a'x_0 + ka'b' + b' y_0 - ka'b' = a'x_0 + b'y_0 = c'$$
        Por tanto, $(x_0+kb', y_0-ka')$ es solución. Falta ver que no hay más soluciones.\\
            
        Sea $(x_1, y_1)$ otra solución y veamos que es de la forma anterior:
        $$a'x_1 + b'y_1 = c' = a'x_0 + b'y_0 \Longrightarrow a'(x_1-x_0) = b'(y_0-y_1)$$

        Entonces, sabiendo que $\mcd(a',b')=1$, tenemos que:
        \begin{gather*}
            b'|a'(x_1-x_0) \mathop{\Longrightarrow}^{\mcd(a',b')=1} b'|x_1-x_0 \Longrightarrow \exists k \in A \mid x_1 - x_0 = kb'
            \Longrightarrow x_1 = x_0 + kb' \\ a'|b'(y_0-y_1) \mathop{\Longrightarrow}^{\mcd(a',b')=1} a'|y_0-y_1 \Longrightarrow \exists h \in A \mid y_0 - y_1 = ha'
            \Longrightarrow y_1 = y_0 - ha'
        \end{gather*}
    
        Nos falta ver que $k=h$. Como $A$ es un DI, tenemos que: $$a'(x_1-x_0)=b'(y_0-y_1) \Longrightarrow a'kb' = b'ha' \Longleftrightarrow k = h$$

        Por tanto, las soluciones son del tipo:
        $$(x_0+kb', y_0-ka')=\left(x_0+k\cdot\dfrac{b}{d}, y_0-k\cdot\dfrac{a}{d}\right)$$
    \end{enumerate}
\end{proof}

\begin{ejemplo}
    Consideramos $A = \Z[i]$. Se pide resolver:
    $$(-2+3i)x + (1+i)y = 1+11i$$

    En primer lugar, calculamos el $\mcd$ para ver si la ecuación diofántica tiene solución. Para ello, aplicamos el algoritmo extendido de Euclídes. Para ello, en primer lugar tengo que dividir $r_1$ entre $r_2$ en $\Z[i]$. En $\Q[i]$:
    $$\dfrac{-2+3i}{1+i} = \dfrac{(-2+3i)(1-i)}{2} = \dfrac{1}{2} + \dfrac{5}{2}i$$
    
    Para dividir en $\Z[i]$, elegimos $q_1, q_2$ tal que $\left|\dfrac{1}{2}-q_1\right|\leq \dfrac{1}{2} \y \left|\dfrac{5}{2}-q_2\right|\leq \dfrac{1}{2}$.
    
    Sean $q_1 = 1 \y q_2 = 2 \Longrightarrow q = 1+2i \y r_3 = -2+3i - (1+2i)(1+i) = -1$. Por tanto, el algoritmo extendido de Euclídes queda de la siguiente forma:
    $$\begin{array}{c|c|c|c}
            i & r_i   & u_i & v_i    \\
            \hline
            1 & -2+3i & 1   & 0      \\
            2 & 1+i   & 0   & 1      \\
            3 & -1    & 1   & -1 -2i \\
            4 & 0     &     &
        \end{array}$$

    Entonces, tenemos el $\mcd$:
    $$\mcd(-2+3i, 1+i) = -1 \mathop{\Longleftrightarrow}^{-1 \sim 1} \mcd(-2+3i, 1+i) = 1$$
    Como $1|(1+11i) \Longrightarrow$ la ecuación tiene solución. Para calcularla, partimos de la identidad de Bezout:
    \begin{gather*}
        -1 = (-2+3i) \cdot 1 + (1+i)(-1-2i) \\
        1 = (-2+3i)(-1) + (1+i)(1+2i) \\
        (1+11i) = (-2+3i)(-1-11i) + (1+i)(1+2i)(1+11i) \\
        (1+11i) = (-2+3i)(-1-11i) + (1+i)(-21+13i)
    \end{gather*}
    
    Por tanto, una solución particular es $x_0 = -1-11i$, $y_0 = -21+13i$. La solución general es:
    \begin{equation*}
        \left\{\begin{array}{l}
            x=-1-11i + \alpha(1+i) \\
            y=-21+13i-\alpha(-2+3i)
        \end{array}\right.
        \hspace{2cm}
        \forall \alpha \in \Z[i]
    \end{equation*}
\end{ejemplo}

\section{Mínimo Común Múltiplo}
\begin{definicion}[Mínimo común múltiplo]
    Sea $A$ un DI y $a, b \in A$. Un elemento $m \in A$ diremos que es un \textbf{mínimo común múltiplo} (abreviado como $\mcm$) y notado $m = \mcm(a,b)$ si verifica:
    \begin{enumerate}
        \item $a|m \y b|m$,
        \item $\forall c \in A$ tal que $a|c \y b|c \Longrightarrow m|c$.
    \end{enumerate}
\end{definicion}

\begin{observacion}
    No siempre existe el mínimo común múltiplo de dos elementos.
    
    Además, si existe no es único (lo es cualquier asociado a él, únicamente). Para verlo, si $m = \mcm(a,b)$, consideramos $mu$, con $u\in \cc{U}(A)$. Entonces:
    \begin{enumerate}
        \item Como $m|mu$, y se tiene que $a|m\land b|m$, por la transitividad, se tiene que $a|mu$ y $b|mu$.
        \item Si $c\in A$ cumple que $a|c\land b|c$, entonces $m|c$; y la transitividad afirma que $mu|c$ al tener que $mu|m$.
    \end{enumerate}
\end{observacion}
Por tanto, hemos visto que $mu=\mcm(a,b)$. Tenemos que el $\mcm$ de dos elementos, si existe, es único salvo asociados. Hablaremos simplemente de $\mcm$ de $a$ y $b$.

\begin{definicion}[Mínimo común múltiplo generalizado]
    Sea $A$ un DI y $a_1, a_2, \ldots, a_n \in~A$, $(n\geq 2)$, un mínimo común múltiplo de $a_1, a_2, \ldots, a_n$, notado
    $\mcm(a_1, a_2, \ldots, a_n)$ es $m \in A$ tal que verifica:
    \begin{enumerate}
        \item $a_i|m~~\forall i \in \{1, \ldots, n\}$,
        \item Si $c\in A$ tal que $a_i|c~~\forall i \in \{1, \ldots, n\} \Longrightarrow m|c$.    \end{enumerate}
\end{definicion}

\textbf{Propiedades del $\mcm$}. Sea $A$ DI, supuesta la existencia de los $\mcm$ que intervienen, tenemos que:
\begin{enumerate}
    \item $\mcm(a,b)=\mcm(b,a)$. En general:
    $$\mcm(a_1, \ldots, a_i, a_{i+1}, a_n) = \mcm(a_1, \ldots, a_{i+1}, a_i, \ldots, a_n),~~n\geq2,~i\in \{1, \ldots, n\}$$

    La demostración es trivial, ya que en la definición no se establece ningún orden.

    \item Si $a\sim a' \Longrightarrow \mcm(a,b) = \mcm(a',b)$.
    
    Sea $m = \mcm(a,b)$. Entonces, $a|m$. Además, por ser $a\sim a'$, se tiene que $a'|a$. Por la transitividad, $a'|m$. Además, se tiene también que $b|m$, por lo que la primera condición se tiene.

    Veamos la segunda. Como $a\sim a'$, tenemos que $c|a\Longleftrightarrow c|a'$, por lo que $\forall c\in A$ se tiene que si $c|a'\land c|b$, entonces $c|a\land c|b$; y por tanto $m|c$. Por tanto, se deduce que 
    $m=\mcm(a',b)$.

    \item $\mcm(a,b)=a \Longleftrightarrow b|a$. En particular: $\mcm(a,0) = 0 \y \mcm(a,1) = 1$.
    \begin{description}
        \item[$\Longrightarrow)$] $\mcm(a,b)=a\Longrightarrow a|a\land b|a$.
        \item[$\Longleftarrow)$] Partimos de que $b|a$. Además, por reflexividad, tenemos que $a|a$. Entonces, se tiene la primera condición.

        Además, como $\forall c\in A\mid a|c\land b|c$ se tiene que $a|c$, tenemos la segunda condición. Por tanto, $a=\mcm(a,b)$.
    \end{description}


    \item $\mcm(\mcm(a,b), c)=\mcm(a,b,c) = \mcm(a,\mcm(b,c))$.

    Demostramos en primer lugar la primera igualdad. Sea $m=\mcm(\mcm(a,b),c)$. Por tanto, $\mcm(a,b)|m\land c|m$. Además, como $a|\mcm(a,b)\land b|\mcm(a,b)$; por la transitividad se tiene que $a|m\land b|m$. Por tanto, se cumple la primera condición para que $m=\mcm(a,b,c)$.

    Por la segunda condición, tenemos que $\forall z\in A$ tal que $\mcm(a,b)|z \land c|z$, se tiene que $m|z$. Como $\mcm(a,b)|z$, por la transitividad se tiene que $a|z\land b|z$. Por tanto, como $\forall z\in A$ tal que $a|z\land b|z\land c|z$ se tiene que $m|c$; entonces tenemos la segunda condición y $m=\mcm(a,b,c)$.\\

    Una vez demostrada la primera igualdad, para la segunda tenemos que:
    \begin{equation*}
        \mcm(a,b,c) = \mcm(b,c,a) = \mcm(\mcm(b,c), a) = \mcm(a,\mcm(b,c))
    \end{equation*}

    \item $\mcm(ac,bc) = \mcm(a,b)c$.
    
    Se deduce de forma directa de la definición. Se deja como \underline{ejercicio}.

    \item $\mcd(a,b)=1 \Longrightarrow \mcm(a,b)=ab$.

    Veamos las dos condiciones:
    \begin{enumerate}
        \item Claramente $a|ab \y b|ab$.
        \item Sea $c \in A$ tal que $\displaystyle a|c \y b|c \mathop{\Longrightarrow}_{\mcd(a,b)=1}^{8)} ab|c$.\\
    \end{enumerate}
\end{enumerate}



\begin{teo}
    \label{teo:Existenciamcm}
    Sea $A$ un DE y $a$, $b \in A$. Entonces $\exists \mcm(a,b)$. Se verifica además que 
    $$\mcm(a,b)\mcd(a,b) = ab$$
\end{teo}
\begin{proof}
    Sean $I_1 = aA$, $I_2=bA$. Entonces, $I=I_1 \cap I_2$ es un ideal de $A$ (veámoslo):
    \begin{enumerate}
        \item Cerrado para sumas:
        $$\forall w,z \in I \Longrightarrow \left\{ \begin{array}{lll}
            w,z \in I_1 & \Longrightarrow & w+z \in I_1 \\
            w,z \in I_2 & \Longrightarrow & w+z \in I_2
        \end{array} \right\} \Longrightarrow w+z \in I$$

        \item Cerrado para múltiplos:
        $$\forall w \in I ~\forall c \in A \Longrightarrow \left\{ \begin{array}{lll}
            w \in I_1 & \Longrightarrow & wc \in I_1 \\
            w \in I_2 & \Longrightarrow & wc \in I_2
        \end{array} \right\} \Longrightarrow wc \in I$$
    \end{enumerate}    
    
    Como $A$ es DE $\Longrightarrow \exists m \in A \mid I = mA$. Veamos que $m=\mcm(a,b)$:
    \begin{enumerate}
        \item Como $\displaystyle m \in I \Longrightarrow \left\{ \begin{array}{lll}
            m \in I_1 = aA & \Longrightarrow & a|m \\
            m \in I_2 = bA & \Longrightarrow & b|m
        \end{array} \right\}$

        \item Sea $c \in A$ tal que $a|c \y b|c \Longrightarrow \exists x,y \in A \mid c=ax \y c=bx$. Entonces, tenemos que $c \in aA=I_1 \y c\in bA=I_2 \Longrightarrow c \in I =mA \Longrightarrow m|c$. Por tanto, $m = \mcm(a,b)$.
    \end{enumerate}

    Veamos ahora la relación entre el $\mcd$ y el $\mcm$ dada. Si $d=\mcd(a,b)$, tenemos que $\exists a', b' \in A \mid a=da' \y b=db'$ con $\mcd(a',b')=1$. Por la propiedad 6), tenemos que $\mcm(a',b')=a'b'$. Por tanto:
    \begin{multline*}
        \mcm(a,b)\mcd(a,b) = \mcm(da',db')\mcd(da',db') = d\cdot \mcm(a',b') \cdot d \cdot \mcd(a',b') = \\
        =d^2 \cdot a'b' = da' \cdot db' = ab
    \end{multline*}
\end{proof}

\begin{coro}
    Consecuencia del Corolario \ref{cor:DEExistemcdyBezout} y del Teorema \ref{teo:Existenciamcm}.
    
    Sea $A$ un DE con $a$, $b \in A$. Consideramos los ideales $I_1 = aA$ y $I_2 = bA$. Entonces:
    \begin{gather*}
        I=I_1 + I_2 = \{ax+by \mid x,y \in A\} \\
        J=I_1 \cap I_2 = \{abx \mid x \in A\}
    \end{gather*}
    son ideales. Se verifica que:
    \begin{equation*}
        I=\mcd(a,b)A \hspace{2cm} J=\mcm(a,b)A
    \end{equation*}
    
    Notemos que para demostrar que $I$ y $J$ son ideales no es necesario imponer que $A$ sea DE (lo es para demostrar que son
    principales), nos es suficiente con que sea un anillo conmutativo.
\end{coro}

\section{Congruencias}
\begin{definicion}[Elementos congruentes]
    Sea $A$ un anillo conmutativo (trabajaremos con DI) y sea $I \subseteq A$ un ideal. Dos elementos $a,b \in A$ diremos que son \textbf{congruentes módulo $I$}, notado $a\equiv b\mod(I)$ ó $a\equiv_I b$ si:
    $$a-b \in I$$
\end{definicion}

\begin{notacion}
    Sea $I$ el ideal generado por $m \in A$; es decir, $I=mA$. Entonces si queremos decir que $a\equiv b\mod(mA)$, podremos notar:
    $$a\equiv b\mod(m)~~~\mbox{ ó }~~~a\equiv_m b$$
\end{notacion}


Por tanto:
\begin{multline*}
    a\equiv b\mod(m) \Longleftrightarrow a-b \in mA \Longleftrightarrow m|a-b \Longleftrightarrow \exists k \in A \mid a-b=km \Longleftrightarrow \\
    \Longleftrightarrow \exists k \in A \mid a = km + b ~~\mbox{ es decir, $b$ es un resto de dividir $a$ entre $m$.}
\end{multline*}

Notemos que si $A=\Z$ y $m\geq 2$. Entonces:
$$a\equiv b\mod(m) \Longleftrightarrow aR_mb $$


\textbf{Propiedades de las congruencias.}
Sean $a,b,c,d \in A$ con $A$ anillo conmutativo, y sea $I$ un ideal. Entonces:
\begin{enumerate}
    \item $\cdot\equiv\cdot\mod(I)$ es una relación de equivalencia.

    Demostramos las tres condiciones:
    \begin{itemize}
        \item \underline{Reflexividad}: $a\equiv a\mod (I)\Longleftrightarrow 0\in I$, lo cual es cierto.

        \item \underline{Simetría}: Si $a\equiv b\mod (I)$, entonces $a-b\in I$. Por tanto, como es cerrado para múltiplos, tenemos que $b-a\in I$, por lo que $b\equiv a\mod(I)$.

        \item \underline{Transitividad}: Si $a\equiv b\mod(I)\y \equiv c\mod (I)$, tenemos que $a-b,b-c\in I$. Entonces, $a-b+b-c=a-c\in I$, por ser el ideal cerrado para sumas. Por tanto, $a\equiv c \mod (I)$.
    \end{itemize}

    \item $a\equiv 0\mod(I) \Longleftrightarrow a\in I$.
    $$a\equiv 0\mod(I) \Longleftrightarrow a-0 \in I \Longleftrightarrow a \in I$$

    \item Si $a\equiv b\mod(I) \Longleftrightarrow \left\{ \begin{array}{l}
        a+c \equiv b+c\mod(I) \\
        ac \equiv bc\mod(I)
    \end{array} \right.\qquad \qquad \forall c \in A$.

    Veamos en primer lugar el resultado para la suma:
    \begin{multline*}
        a\equiv b\mod(I) \Longleftrightarrow a-b \in I \Longleftrightarrow a+c-c-b \in I ~~\forall c \in A \Longleftrightarrow \\
        \Longleftrightarrow (a+c)-(b+c) \in I ~~\forall c \in A \Longleftrightarrow a+c\equiv b+c\mod(I)~~\forall c \in A
    \end{multline*}

    Veamos ahora el resultado para el producto:
    \begin{multline*}
        a\equiv b\mod(I) \Longleftrightarrow a-b \in I \Longrightarrow c(a-b)\in I~~\forall c \in A \Longleftrightarrow \\
        \Longleftrightarrow ac - bc \in A ~~\forall c \in A \Longleftrightarrow ac \equiv bc\mod(I) ~~\forall c \in A 
    \end{multline*}

    \item Si $ a\equiv b\mod(I) \y c\equiv d\mod(I) \Longrightarrow \left\{ \begin{array}{l}
        a+c\equiv b+d\mod(I) \\
        ac\equiv bd\mod(I)
    \end{array} \right.$.

    Veamos en primer lugar el resultado para la suma:
    \begin{multline*}
        \left. \begin{array}{l}
            a\equiv b\mod(I) \\
            c\equiv d\mod(I)
        \end{array} \right\} \Longleftrightarrow \left\{ \begin{array}{l}
            a-b \in I \\
            c-d \in I
        \end{array} \right\} \Longrightarrow (a-b)+(c-d) \in I \Longrightarrow \\
        \Longrightarrow(a+c)-(b+d) \in I \Longleftrightarrow a+c \equiv b+d\mod(I)
    \end{multline*}

    Veamos ahora el resultado para el producto:
    \begin{multline*}
        \left. \begin{array}{l}
            a\equiv b\mod(I) \\
            c\equiv d\mod(I)
        \end{array} \right\} \Longleftrightarrow \left\{ \begin{array}{l}
            a-b \in I \\
            c-d \in I
        \end{array} \right\} \Longrightarrow \left\{ \begin{array}{l}
            c(a-b) \in I \\
            b(c-d) \in I
        \end{array} \right\} \Longrightarrow \\
        \Longrightarrow c(a-b) + b(c-d) \in I \Longleftrightarrow ac-bc+bc-bd \in I \Longleftrightarrow ac - bd \in I \Longleftrightarrow
        ac\equiv bd\mod(I)
    \end{multline*}\\

    \noindent A partir de ahora, sea $A$ un DE y sea $I=mA$ con $m \in A \mid m \neq 0$:\\

    \item Sea $r$ un resto de dividir $a$ por $m$ es decir, entonces $a\equiv r\mod (m)$.

    Dividimos $a$ entre $m$: $a=mq+r$ con $r= 0 \o \phi(r) < \phi(m)$.
    $$a-r = mq \Longrightarrow a-r \in mA \Longleftrightarrow a\equiv r\mod(m)$$

    \item $a\equiv b\mod(m) \Longleftrightarrow a \mbox{ y } b \mbox{ tienen el mismo resto al dividirlos entre } m.$
    \begin{description}
        \item[$\Longrightarrow)$] $a\equiv b\mod(m) \Longleftrightarrow a-b \in mA \Longleftrightarrow \exists k \in A \mid a-b =km$.
        
            Dividimos $b$ entre $m$: $b = mq + r$ con $r=0 \o \phi(r) < \phi(m)$, por lo que $r$ es un resto de dividir $b$ entre $m$.
            
            Como $a-b=km \Longrightarrow a=b+km \Longrightarrow a=mq+r+km = m(q+k)+r$ con $r=0 \o \phi(r)<\phi(m)$ por lo que $r$ es un resto de dividir $a$ entre $m$.
            
        \item[$\Longleftarrow)$]
            Supongamos que $a=mq+r$ con $r=0 \o \phi(r)<\phi(m)$ y $b=mq' +r$. Luego:
            $$a-b = mq +r -mq'-r = m(q-q') \Longrightarrow a-b \in mA \Longleftrightarrow a\equiv b\mod(m)$$
    \end{description}

    \item \label{conj_prop7}Si $\left\{ \begin{array}{c}
        ac \equiv bc\mod(m) \\ \y \\
        \mcd(c,m)=1
    \end{array} \right\} \Longrightarrow a\equiv b\mod(m)$.
    \begin{multline*}
        ac \equiv bc\mod(m) \Longleftrightarrow ac - bc \in mA \Longleftrightarrow c(a-b) \in mA \Longrightarrow m|c(a-b) \Longrightarrow \\
        \mathop{\Longrightarrow}^{7)}_{\mcd(c,m)=1} m|a-b \Longrightarrow a-b \in mA \Longleftrightarrow a\equiv b\mod(m)
    \end{multline*}

    Notemos que el recíproco es cierto $\forall c \in A$ sin tener que $\mcd(c,m)=1$.

    \item \label{conj_prop8} Sea $c\neq 0$. Entonces $ac\equiv bc\mod(mc) \Longleftrightarrow a\equiv b\mod(m)$.
    \begin{multline*}
        ac\equiv bc\mod(mc) \Longleftrightarrow ac-bc \in mcA \Longleftrightarrow \exists k \in A \mid ac-bc = mck \Longleftrightarrow \\
        \Longleftrightarrow a-b = mk \Longleftrightarrow a-b \in mA \Longleftrightarrow a\equiv b\mod(m)
    \end{multline*}
\end{enumerate}

Cuidado con la propiedad \ref{conj_prop7}: Si $A = \Z$ y queremos simplificar $30\equiv 6\mod(8)$:
$$\left. \begin{array}{l}
        30\equiv 6\mod(8) \\
        30 = 3\cdot 10    \\
        6 = 3 \cdot 2     \\
        \mcd(3,8)=1
    \end{array} \right\} \mathop{\Longrightarrow}^{7)} 10 \equiv 2\mod(8)$$

No obstante,
$$\left. \begin{array}{l}
        10 \equiv 2\mod(8) \\
        10 = 2 \cdot 5     \\
        2 = 2 \cdot 1      \\
        \mcd(2,8) = 2
    \end{array} \right\} \nRightarrow 5\equiv 1\mod(8)$$
ya que $5-1 = 4 \notin 8\Z = \{\ldots, -16, -8, 0, 8, 16, \ldots\}$\\


Lo que sí podemos hacer es, usando la propiedad \ref{conj_prop8},
$$\left. \begin{array}{l}
        10 \equiv 2\mod(8) \\
        10 = 2 \cdot 5     \\
        2 = 2 \cdot 1      \\
        8 = 2 \cdot 4      \\
        2 \neq 0
    \end{array} \right\} \mathop{\Longrightarrow}^{8)} 5 \equiv 1\mod(4)$$
que sí es cierto: $5-1 = 4 \in 4\Z$

\subsection{Ecuaciones de congruencias}

Sea $A$ un DE, estamos interesados en resolver en $A$ ecuaciones de la forma:
$$ax\equiv b\mod(m)$$
Con $a,b,m \in A$, $a,m \neq 0$ y $x$ incógnita.\\


Sabemos por ahora que, si $a=1$ (o unidad), las soluciones son:
$$x\equiv b\mod(m) \Longleftrightarrow x-b \in mA \Longleftrightarrow x \in \{b+km \mid k \in A\}$$

Veamos el caso general:
\begin{teo}
    \label{teo:ecCongruencia}
    Sea $A$ un DE, sea una ecuación del tipo:
    $$ax\equiv b\mod(m)$$
    Con $a,b,m \in A \mid a,m \neq 0$ y $x$ una incógnita. Sea $d = \mcd(a,m)$. Entonces:
    \begin{enumerate}
        \item La ecuación tiene solución $\Longleftrightarrow d|b$.
        \item Supongamos que tiene solución $\Longrightarrow d|b$. Consideramos $b=db'$, $a=da'$, $m=dm'$. Entonces, la ecuación anterior es equivalente (tiene las mismas soluciones) a la ecuación:
        $$a'x\equiv b'\mod(m')$$
        que llamaremos \textbf{ecuación reducida} de la ecuación primitiva.

        \item Supongamos que tiene solución y sea $x_0$ una solución particular. Entonces, la ecuación reducida y la primitiva son equivalentes a la ecuación:
        $$x\equiv x_0\mod(m')$$
        Por tanto, las soluciones de la ecuación primitiva son $\{x_0 + km' \mid k \in A\}$.

        \item Supongamos que tiene solución. Entonces, podemos encontrar una solución: $$y_0 \in A \mid y_0=0 \o \phi(y_0) < \phi(m')$$ que llamaremos \textbf{solución óptima} de la ecuación. y usaremos para dar el conjunto de soluciones del sistema:
        $$\{y_0 + km' \mid k \in A\}$$
    \end{enumerate}
\end{teo}
\begin{proof} Demostramos cada una de las partes:
\begin{enumerate}
    \item La ecuación primitiva tiene solución si y solo si:
    \begin{align*}
        & \exists x_0 \in A \mid ax_0 \equiv b\mod(m) \Longleftrightarrow \\
        &\hspace{1cm} \Longleftrightarrow \exists x_0 \in A \mid ax_0 -b\in mA \Longleftrightarrow \exists x_0 \in A \y \exists y_0 \in A \mid ax_0 -b = my_0 \Longleftrightarrow \\
        &\hspace{1cm} \Longleftrightarrow \exists x_0, y_0 \in A \mid ax_0 - my_0 = b \Longleftrightarrow \text{La diofántica } ax-my=b \text{ tiene solución} \Longleftrightarrow \\
        &\hspace{1cm} \Longleftrightarrow \mcd(a,-m) = \mcd(a,m)|b \Longleftrightarrow d|b
    \end{align*}

    \item Suponemos que tiene solución; es decir, que $d|b$. Entonces, por lo visto y por ser $d=\mcm(a,m)$, entonces $\exists a',b',m' \in A$ tales que $a=da', b=db', m=dm'$. Entonces:
    $$ax_0 \equiv b\mod(m) \Longleftrightarrow da'x_0 \equiv db'\mod(dm') \Longleftrightarrow a'x_0 \equiv b'\mod(m')$$

    \item Suponemos que tiene solución:
    $$ax\equiv b\mod(m) \Longleftrightarrow a'x\equiv b'\mod(m') \mbox{ con } \mcd(a',m')=1$$
    Sabemos por el Corolario \ref{cor:DEExistemcdyBezout} que $\exists u,v \in A \mid 1=a'u+m'v $. Entonces:
    $$1=a'u+m'v \Longleftrightarrow a'u-1=-m'v \Longleftrightarrow a'u\equiv 1\mod(m') \mathop{\Longrightarrow}^{3)}
        a'ub' \equiv b'\mod(m')$$
    Por tanto, $x_0 = ub'$ es solución de la ecuación reducida (y por tanto, de la primitiva).

    
    Es claro que $\forall k \in A$, $x_0+km'$ es una solución, puesto que:
    $$a'(x_0+km')-b' = a'x_0 + km'a' -b' = \underbrace{a'x_0-b'}_{k'm'}+km'a' = ka'm'+k'm' = (k'+ka')m'\in m'A$$

    Veamos que todas las soluciones son de dicha forma. Sea $y_0$ otra solución. Entonces:
    \begin{multline*}
        \left\{ \begin{array}{l}
            a'y_0 \equiv b'\mod(m') \\
            a'x_0 \equiv b'\mod(m')
        \end{array} \right\} \mathop{\Longrightarrow}^{1)} \left\{ \begin{array}{c}
            a'y_0 \equiv a'x_0 \mod(m') \\
            \mcd(a',m')=1
        \end{array} \right\} \mathop{\Longrightarrow}^{1)} \\
        \Longrightarrow y_0\equiv x_0\mod(m') \Longrightarrow y_0 \mbox{ es solucion de } x\equiv x_0\mod(m')
    \end{multline*}

    Por tanto, tenemos que la ecuación es equivalente a $x\equiv x_0\mod(m')$.

    \item Sea $x_0$ una solución. Dividimos $x_0$ entre $m'$ y obtenemos $x_0=m'q+y_0$ con $y_0=0 \o \phi(y_0)<\phi(m')$.

    Por la división, tenemos que $x_0\equiv y_0\mod(m')$; y sabiendo que $\mcd(a', m')=1$, tenemos que $a'x_0\equiv a'y_0 \mod(m')$. Por otro lado, tenemos que $x_0$ es una solución. Entonces, $a'x_0\equiv b'\mod(m')$.
    
    De ambas congruencias, por transitividad tenemos que $a'y_0\equiv b'\mod(m')$, de donde deducimos que $y_0$ es una solución, la denominada óptima.    
\end{enumerate}
\end{proof}

\begin{ejemplo} Resolver en $\Z$:
\begin{enumerate}
    \item $60x\equiv 90\mod(105)$.

    Veamos en primer lugar si tiene solución: 
    $$\mcd(60,105)=15 \y 90 = 15 \cdot 6 \Longrightarrow 15|90 \Longrightarrow \mbox{ Sí tiene solución.}$$

    Tenemos que la ecuación reducida es $ 4x\equiv 6\mod(7)$, con $\mcd(4,7) = 1$.

    Buscamos los coeficientes de Bezout: $u,v \in \Z \mid 1=4u + 7v$. Aplicando el algoritmo extendido de Euclídes, llegamos a que $1 = 4\cdot 2 + 7(-1)$. Luego:
    $$4\cdot 2\equiv 1\mod(7) \Longrightarrow 4 \cdot 2 \cdot 6\equiv 6\mod(7) \Longrightarrow 4\cdot 12\equiv 6\mod(7)$$
    Por lo que $x_0=12$ es solución particular del sistema.
    
    No obstante, no es la óptima, ya que  $12\neq 0$ y $|12| \not < |7|$. Dividimos $x_0=12$ entre $7$ para buscar la óptima: $12=7\cdot 1 + 5 \Longrightarrow y_0=5$ es la solución óptima.
        
    Por tanto, la ecuación es equivalente a $x\equiv 5\mod(7)$ y sus soluciones son: $\{5+7k \mid k \in A\}$.

    \item $1100x\equiv 660\mod(140)$.
    $$1100x\equiv 660\mod(140) \mathop{\Longleftrightarrow}^{8)} 110x\equiv 66\mod(14)
        \mathop{\Longleftrightarrow}^{8)} 55x\equiv 33\mod(7)$$
    $$\left. \begin{array}{l}
            55x\equiv 33\mod(7) \\
            55 = 11\cdot 5      \\
            33 = 11\cdot 3      \\
            \mcd(11,7) = 1
        \end{array} \right\} \mathop{\Longleftrightarrow}^{7)} 5x\equiv 3\mod(7)$$
    $$\left. \begin{array}{l}
            5x\equiv 3\mod(7)   \\
            5x\equiv -2x\mod(7) \\
            3\equiv -4\mod(7)
        \end{array} \right\} \mathop{\Longleftrightarrow}^{1)} -2x\equiv -4\mod(7)$$
    $$\left. \begin{array}{l}
            -2x\equiv -4\mod(7) \\
            -2 = -2 \cdot 1     \\
            -4 = -2 \cdot 2     \\
            \mcd(-2,7)=1
        \end{array}\right\} \mathop{\Longleftrightarrow}^{7)} x\equiv 2 \mod(7)$$
    $|2|<|7| \Longrightarrow y_0=2$ es solución óptima del sistema. Por tanto, el conjunto de soluciones de la ecuación es:
    $$\{2 + 7k \mid k \in A\}$$

    \noindent Una solución alternativa al ejercicio es, cuando tenemos $5x\equiv 3\mod(7)$, podemos darnos cuenta de que $5^{-1} = 3$ en $\Z_7$.
    $$5x\equiv 3\mod(7) \Leftrightarrow 5\cdot 5^{-1}x\equiv 3\cdot 3\mod(7) \Leftrightarrow x\equiv 2 \mod(7)$$
\end{enumerate}
\end{ejemplo}

\subsection{Sistemas de 2 ecuaciones de congruencias}
\noindent Sea $A$ un DE, queremos resolver sistemas de ecuaciones de congruencias con dos ecuaciones de la forma:
$$\left. \begin{array}{l}
        a_1x\equiv b_1\mod(m_1) \\
        a_2x\equiv b_2\mod(m_2)
    \end{array} \right\}$$
Con $a_i,b_i,m_i \in A \mid a_i\neq 0 \neq m_i$ con $i \in \{1,2\}$.\\


\noindent Supongamos que cada ecuación tiene solución (de forma independiente) y entonces $\exists a,b \in A$ tales que el sistema
anterior es equivalente a uno de la forma:
$$\left. \begin{array}{l}
        x\equiv a\mod(m) \\
        x\equiv b\mod(n)
    \end{array} \right\}$$
Con $a,b,m,n \in A \mid m\neq0\neq n$.

\begin{teo}
    Sea $A$ un DE y consideramos un sistema de la forma:
    $$\left. \begin{array}{l}
            x\equiv a\mod(m) \\
            x\equiv b\mod(n)
        \end{array} \right\}$$
    Con $a,b,m,n \in A \mid m\neq0\neq n$.\newline

    \noindent
    Sean $d=\mcd(m,n)$, $p=\mcm(m,n)$:
    \begin{enumerate}
        \item[i)] El sistema tiene solución $\Longleftrightarrow a\equiv b\mod(d)$.
        \item[ii)] Si el sistema tiene solución. Entonces, existe una solución: $$x_0 \in A \mid x_0=0 \o \phi(x_0)<\phi(p)$$
            Que llamaremos \textbf{solución óptima del sistema}. Por tanto, el resto de soluciones serán de la forma:
            $$\{x_0+kp \mid k \in A\}$$
            Y por tanto, el sistema será equivalente a la ecuación:
            $$x\equiv x_0\mod(p)$$
    \end{enumerate}
\begin{proof}
    Suponemos inicialmente que la primera ecuación tiene solución, ya que si no la tuviera, tenemos claro que el sistema tampoco.\newline
    \begin{enumerate}
        \item[i)] La solución general de la primera ecuación es, por el Teorema \ref{teo:ecCongruencia}: $$x=a+km \mid k \in A$$
                Sustituimos en la segunda ecuación y, entonces, el sistema tiene solución si la ecuación
                $$a+km\equiv b\mod(n)$$
                Con incógnita $k$ tiene solución:
                $$a+km\equiv b\mod(n) \Longleftrightarrow km\equiv b-a\mod(n) \Longleftrightarrow d|b-a \Longleftrightarrow a\equiv b\mod(d)$$
        \item[ii)] Suponemos que tiene solución $\Longrightarrow km\equiv b-a\mod(m)$.\newline
                Sea $k_0$ una solución particular, sabemos que la solución general es: $$k=k_0+t\dfrac{m}{d}~~\forall t \in A$$
                $$x=a+km \Longrightarrow x=a+\left( k_0 + t\dfrac{m}{d} \right)n = a+k_0n + t\dfrac{mn}{d} = a+k_0n + tp$$
                Por lo que $x_0 = a+k_0n$ es una solución particular del sistema y las demás son:
                $$\{x_0 + tp \mid t \in A\}$$
                Si dividimos $x_0$ entre $p$, obtenemos que $x_0=qp+y_0$ con $y_0=0 \o \phi(y_0)<\phi(p)$.
                $$y_0\equiv x_0\mod(p)$$
                Luego $y_0$ también es solución del sistema; es la óptima.
    \end{enumerate}
\end{proof}
\end{teo}

\begin{ejemplo}
    Calcular la menor capacidad posible de un depósito sabiendo que a un depóstio de doble capacidad le ha faltado 1 litro
    para llenarlo con garrafas de 5 litros y que a uno de quíntuple capacidad le ha faltado también 1 litro para llenarlo
    con garrafas de 7 litros.\\

    \noindent Sea $x$ la capacidad buscada:
    $$2x+1 \mbox{ es divisible por } 5 \mbox{ y } 5x+1 \mbox{ es divisible por } 7 \mbox{ luego:}$$
    $$\left. \begin{array}{rcl}
            2x+1 \equiv0\mod(5) & \Longleftrightarrow & 2x\equiv -1\mod(5) \\
            5x+1 \equiv0\mod(7) & \Longleftrightarrow & 5x\equiv -1\mod(7)
        \end{array} \right\}$$
    Buscamos la solución de cada sistema (si una no tiene solución, el sistema no tiene solución):
    $$\left. \begin{array}{l}
            2x \equiv -1 \mod (5) \\
            -1 \equiv 4 \mod (5)
    \end{array} \right\} \Longleftrightarrow 2x\equiv 4\mod(5)$$
    $$\left. \begin{array}{l}
            2x \equiv 4\mod(5) \\
            \mcd(2,5) = 1
        \end{array} \right\} \Longleftrightarrow x\equiv 2\mod(5)$$
    $$\left. \begin{array}{l}
            5x\equiv -1 \mod(7) \\
            -1\equiv 6\mod(7)
        \end{array} \right\} \Longleftrightarrow 5x\equiv 6\mod(7)$$
    $$\mcd(5,7)=1 \y 1|6 \Longrightarrow \mbox{ tiene solución.}$$
    Buscamos la solución de la segunda ecuación, buscando la identidad de Bezout para 5 y 7:
    $$1 = 5\cdot 3 + 7(-2) \Longrightarrow 5\cdot 3 \equiv 1\mod(7) \Longleftrightarrow 5\cdot 3\cdot 6\equiv 6 \mod(7)$$
    Por lo que $x_0=3\cdot 6=18$ es una solución particular.\newline
    Recordamos que en $\Z$, la función euclídea es el valor absoluto, antes de continuar.\newline
    $|18|\not <|7|$. $18=2\cdot 7 + 4 \Longrightarrow y_0=4$ es la solución óptima de la ecuación.\newline
    Por tanto, la ecuación es equivalente a: $x\equiv 4\mod(7)$ y el sistema es equivalente a:

    $$\left. \begin{array}{l}
            x \equiv 2 \mod(5) \\
            x \equiv 4 \mod(7)
        \end{array} \right\}$$
    Que tiene solución si: \newline
    $2\equiv 4\mod(\mcd(5,7)) \Longleftrightarrow 2\equiv 4\mod(1) \Longleftrightarrow 2-4 \in 1A = A$, Cierto.\\

    
    \noindent La solución de la primera ecuación es $x=2+5k \mid k \in A$. Sustituimos en la segunda:
    $$2+5k \equiv 4 \mod(7) \Longleftrightarrow 5k\equiv 2\mod(7)$$
    Buscamos la identidad de Bezout:
    $$1=5\cdot 3 + 7(-2) \Longrightarrow 5\cdot 3\equiv 1\mod(7) \Longleftrightarrow 5 \cdot 3 \cdot 2 \equiv 2 \mod(7)$$
    Luego $k_0=6$ es solución particular de esta. De hecho es la óptima, ya que $|6|<|7|$. La solucón general es
    $k=6+7t \mid t \in A$.\\

    
    Entonces:
    $$x=2+5k = 2+5(6+7t) = 32 + 35 t \mid t \in \Z$$
    Siendo $32$ la solución óptima del sistema.\\

    
    \noindent Buscamos la menor solución no negativa del sistema, que resulta ser la óptima. Por tanto, el depósito era de $32$ litros.
\end{ejemplo}

\subsection{Sistemas de $r$ ecuaciones de congruencias}

Sea $r\geq 2$, consideramos un sistema de la forma:
$$\left. \begin{array}{c}
        a_1x\equiv b_1\mod(m_1) \\
        a_2x\equiv b_2\mod(m_2) \\
        \vdots                  \\
        a_rx\equiv b_r\mod(m_r)
    \end{array} \right\}$$
\textbf{1.} Resolvemos cada una de forma independiente y si todas tienen solución, el sistema será equivalente a:
$$\left. \begin{array}{c}
        x \equiv c_1\mod(n_1) \\
        x\equiv c_2\mod(n_2)  \\
        \vdots                \\
        x\equiv c_r\mod(n_r)
    \end{array} \right\}$$
Para ciertos $c_1, c_2, \ldots, c_r, n_1, n_2, \ldots, n_r \in A$.\newline
\textbf{2.} Resolvemos el sistema de las dos primeras ecuaciones y si tiene solución, obtendremos una ecuación
$x\equiv c\mod(n)$ con $n=\mcm(n_1,n_2),c\in A$ equivalente a dicho sistema:
$$\left. \begin{array}{c}
        x \equiv c\mod(n)    \\
        x\equiv c_2\mod(n_2) \\
        \vdots               \\
        x\equiv c_r\mod(n_r)
    \end{array} \right\}$$
\textbf{3.} Repetimos el proceso $r-1$ veces, obteniendo que si todos los sistemas anteriores tenían solución, el
sistema de $r$ ecuaciones es equivalente a una ecuación del tipo:
$$x\equiv a\mod(m)$$
Para cierto $a \in A$ y $m=\mcm(n_1, n_2, \ldots, n_r)$, que tendrá como solución el conjunto:
$$\{a+km \mid k \in A\}$$

\newpage
\section{Anillos cocientes}
\noindent
Sea $A$ un anillo conmutativo e $I \subseteq A$ un ideal.\newline
Sabemos que $\equiv\mod(I)$ es una relación de equivalencia. Consideramos el conjunto cociente:
$$A/(\equiv\mod(I)) =: A/I = \{[a] \mid a \in A\}$$
$$[a] = \{b \in A \mid b\equiv a\mod(I)\} = \{b \in A \mid b-a \in I\} = \{a+y \mid y \in I\}$$
Notaremos a la clase de cada elemento $a \in A$ por $a+I$:
$$[a]=:a+I$$
A la que llamaremos \textbf{clase de $a$ módulo $I$}. Por tanto:
$$A/I = \{a+I \mid a \in A\}$$
Notemos que:
$$a+I = b+I \Longleftrightarrow a\equiv b\mod(I)$$

\bigskip

Definimos en $A/I$ las operaciones suma y producto $\forall a_1+I, a_2+I$ de la forma:
$$(a_1+I)+(a_2+I) := (a_1+a_2)+I$$
$$(a_1+I)(a_2+I) := (a_1a_2)+I$$
\begin{proof}
    \ \\
    Veamos que se encuentran bien definidas (que no dependen del representante):\newline
    $$\mbox{Suponemos } \left. \begin{array}{ccc}
            a_1+I=b_1+I & \Longrightarrow & a_1\equiv b_1\mod(I) \\
            a_2+I=b_2+I & \Longrightarrow & a_2\equiv b_2\mod(I)
        \end{array} \right\} \Longrightarrow$$ $$\Longrightarrow \left\{ \begin{array}{c}
            a_1+a_2\equiv b_1+b_2\mod(I) \\
            a_1a_2\equiv b_1b_2\mod(I)
        \end{array} \right\} \Longrightarrow \left\{ \begin{array}{c}
            (a_1+a_2)+I=(b_1+b_2)+I \\
            (a_1a_2)+I = (b_1b_2)+I
        \end{array} \right.$$
\end{proof}

\begin{prop}
    Se verifica que con estas operaciones, $A/I$ es un anillo conmutativo, que llamaremos \textbf{anillo cociente de $A$ sobre
        el ideal $I$} ó \textbf{anillo de restos módulo $I$}.
\begin{proof}
    $$\forall a+I, b+I, c+I \in A/I$$
    Asociativa de la suma:
    $$((a+I)+(b+I))+(c+I) = ((a+b)+I)+(c+I) = (a+b+c)+I =$$ $$=(a+I)+((b+c)+I) = (a+I)+((b+I)+(c+I))$$
    Conmutativa de la suma:
    $$(a+I)+(b+I) = (a+b)+I = (b+a)+I = (b+I)+(a+I)$$
    Existencia de elemento neutro de la suma:
    $$0+I \in A/I \mid (a+I)+(0+I) = (a+0)+I = a+I$$
    Existencia de opuestos:
    $$(-a)+I \in A/I \mid ((-a)+I)+(a+I) = (-a+a)+I = 0+I$$
    Asociativa del producto:
    $$((a+I)(b+I))(c+I) = ((ab)+I)(c+I)=(abc)+I=(a+I)((bc)+I) = (a+I)((b+I)(c+I))$$
    Conmutativa del producto:
    $$(a+I)(b+I) = (ab)+I = (ba)+I = (b+I)(a+I)$$
    Existencia de elemento neutro del producto:
    $$1+I \in A/I \mid (a+I)(1+I) = (a\cdot 1)+I=a+I$$
    Propiedad distributiva:
    $$(a+I)((b+I)+(c+I)) = (a+I)((b+c)+I) = (a(b+c))+I =$$ $$=(ab+ac)I = ((ab)+I)+((ac)+I) = (a+I)(b+I)+(a+I)(c+I)$$
    Por lo que $A/I$ es un anillo conmutativo.
\end{proof}
\end{prop}~\\

\noindent
Notemos que si $A=\Z$ e $I=n\Z$. \newline Entonces, como $a\equiv b\mod(n)\Longleftrightarrow aR_nb~~\forall a,b \in \Z$:
$$\Z/n\Z = \Z/R_n = \Z_n$$

\begin{definicion}[Núcleo de homomorfismo]
    Sea $f:A\rightarrow B$ un homomorfismo de anillos, definimos su \textbf{núcleo}, notado $Ker(f)$, como:
    $$Ker(f) = \{a \in A \mid f(a) = 0\}$$
\end{definicion}

\begin{prop}
    Sea $A$ un anillo conmutativo y consideramos un homomorfismo $f:A\Longrightarrow B$ de anillos. Entonces:
    $$Ker(f) \mbox{ es un ideal de } A$$
\begin{proof}
    $$\forall a,b \in Ker(f) \Longrightarrow f(a+b) = f(a)+f(b) = 0+0 = 0 \Longrightarrow a+b \in Ker(f)$$
    $$\forall a \in Ker(f)~~\forall c\in A \Longrightarrow f(ca) = f(c)f(a) = f(c)\cdot 0 = 0 \Longrightarrow ca \in Ker(f)$$
\end{proof}
\end{prop}



Antes de ver el siguiente teorema, recordamos la Proposición \ref{prop:ImagenHomomorfismo}.

\begin{teo}[Primer teorema de isomorfía]
    \ \\
    Todo homomorfismo de anillos conmutativos $f:A\rightarrow B$ induce un isomorfismo
    $$\overline{f}:A/Ker(f)\Longrightarrow Img(f)$$
    Definido por:
    $$\overline{f}(a+Ker(f)) := f(a)\hspace{1cm}\forall a+Ker(f) \in A/Ker(f)$$
\begin{proof}
    \ \\
    1) Veamos que $\overline{f}$ está bien definida y que no depende de representantes:
    $$\mbox{Sea } a+Ker(f) = b+Ker(f) \Longleftrightarrow a\equiv b\mod(Ker(f)) \Longleftrightarrow a-b \in Ker(f)$$
    $$0 = f(a-b) = f(a)-f(b) \Longleftrightarrow f(a) = f(b)$$
    2) Ahora, veamos que $\overline{f}$ es un homomorfismo. $\forall a,b \in A/Ker(f)$:
    $$\overline{f}((a+Ker(f))+(b+Ker(f))) = \overline{f}((a+b)+Ker(f)) = f(a+b) = f(a)+f(b)$$
    $$\overline{f}((a+Ker(f))(b+Ker(f))) = \overline{f}((ab)+Ker(f)) = f(ab) = f(a)f(b)$$
    $$\overline{f}(1+Ker(f)) = f(1) = 1$$
    3) Veamos que $\overline{f}$ es sobreyectiva:
    $$Img(\overline{f}) = \{\overline{f}(a) \mid a \in A/Ker(f)\} = \{f(a) \mid a \in A\} = Img(f)$$
    4) Veamos que $\overline{f}$ es inyectiva:
    $$\forall a+Ker(f),b+Ker(f) \in A/Ker(f) \mid \overline{f}(a+Ker(f)) = \overline{f}(b+Ker(f))$$
    $$\overline{f}(a+Ker(f)) = \overline{f}(b+Ker(f)) \Longrightarrow f(a)=f(b) \Longrightarrow 0=f(a)-f(b) = f(a-b)$$
    $$f(a-b)=0 \Longrightarrow a-b \in Ker(f) \Longrightarrow a\equiv b\mod(Ker(f)) \Longrightarrow a+Ker(f)=b+Ker(f)$$
\end{proof}
\end{teo}

\begin{teo}
    \label{teoEquivAnilloCociente}
    Sea $A$ un DE y $I=mA$ con $m\in A \mid m\neq 0 \y m\notin \cc{U}(A)$.
    Entonces:
    \begin{enumerate}
        \item[i)] $a+mA \in \cc{U}(A/mA) \Longleftrightarrow \mcd(a,m)=1$.
        \item[ii)]Los siguientes enunciados son equivalentes:
            \begin{enumerate}
                \item[1)] $m$ es irreducible.
                \item[2)] $A/mA$ es un cuerpo.
                \item[3)] $A/mA$ es un DI.
            \end{enumerate}
    \end{enumerate}
\begin{proof}
    \ \\
    i)\newline
    $\Longrightarrow)$ Sea $a+mA \in \cc{U}(A/mA)$. Entonces:
    $$\exists b+mA \in A/mA \mid 1+mA=(a+mA)(b+mA) = (ab)+mA \Longrightarrow$$
    $$\Longrightarrow ab\equiv 1\mod(m) \Longrightarrow m|ab-1 \Longrightarrow \exists q\in A \mid ab-1=qm$$
    $$\left. \begin{array}{c}
            1=ab-qm \\
            d=\mcd(a,b)
        \end{array} \right\} \Longrightarrow d|1 \Longrightarrow d\in \cc{U}(A) \Longrightarrow d\sim 1 \Longrightarrow \mcd(a,m) = \mcd(a,1) = 1$$
    $\Longleftarrow)$
    Sea $\mcd(a,m)=1$, elegimos los coeficientes de Bezout:
    $$u,v \in A \mid 1=au + mv \Longrightarrow au \equiv 1 \mod(m) \Longrightarrow$$
    $$ \Longrightarrow (a+mA)(u+mA)=(au)+mA = 1+mA \Longrightarrow a+mA \in \cc{U}(A/mA)$$

    \bigskip
    
    ii)\newline
    1) $\Longrightarrow$ 2) Suponemos que $m$ es irreducible:
    $$\forall a+mA \in A/mA \mid a+mA \neq 0+mA \Longrightarrow a\notin mA \Longrightarrow m\not|a$$
    Como $m$ es irreducible y $m\not|a \Longrightarrow \mcd(a,m)=1$.\newline
    Por i), tenemos que $a+mA \in \cc{U}(A/mA) \Longrightarrow \cc{U}(A/mA) = A/mA \setminus\{0\} \Longrightarrow A/mA$ es un cuerpo.\newline
    2) $\Longrightarrow$ 3) Cierto por el Lema \ref{lema:CuerpoEntoncesDominio}.\newline
    3) $\Longrightarrow$ 1) Suponemos que $A/mA$ es DI $\Longrightarrow m\neq 0 \y m\notin \cc{U}(A)$
    $$\forall a \in A \mid a|m \Longrightarrow \exists b \in A \mid m=ab \Longrightarrow $$
    $$\Longrightarrow 0+mA = m+mA = (ab)+mA = (a+mA)(b+mA)$$
    Como $A$ es DI:
    $$a+mA = 0+mA \Longrightarrow a\in mA \Longrightarrow a=ma'~~a' \in A$$
    $$m=ab \Longrightarrow m=ma'b \Longrightarrow a'b=1 \Longrightarrow b\in \cc{U}(A) \Longrightarrow a\sim m \Longrightarrow m \mbox{ irreducible}$$
    ó
    $$b+mA = 0+mA \Longrightarrow b\in mA \Longrightarrow b=mb'~~b'\in A$$
    $$m=ab \Longrightarrow m=mab' \Longrightarrow ab'=1 \Longrightarrow a\in \cc{U}(A) \Longrightarrow b\sim m \Longrightarrow m \mbox{ irreducible}$$
\end{proof}
\end{teo}

\begin{ejemplo}
    Sea $A=\Z$, consideramos $I=153\Z$ y $\Z/153\Z=\Z_{153}$.\newline
    Estudiar si $2 \in \cc{U}(\Z_{153})$ y encontrar $2^{-1}$.\\

    
    $$\mcd(2,153)=1 \Longrightarrow 2 \in \cc{U}(\Z_{153})$$
    Buscamos los coeficientes de Bezout: $1=2(-76)+153\cdot 1$:
    $$2(-76)\equiv 1\mod(153) \Longrightarrow 2^{-1} = -76 = 153-76 = 77$$
\end{ejemplo}

\begin{coro}
    del Teorema \ref{teoEquivAnilloCociente}.\newline
    1) Sea $n\geq 2$. Entonces $\Z_n$ es un cuerpo $\Longleftrightarrow n$ es irreducible $($primo en $\Z)$.\newline
    2) Sea $K$ un cuerpo y $f \in K[x] \mid f\neq 0 \y f\notin \cc{U}(K[x])$ es decir, $f$ no es constante. Entonces:
    $$K[x]/fK[x] \mbox{ es un cuerpo } \Longleftrightarrow f \mbox{ es irreducible}$$
    3) Sea $n\in \{-2, -1, 2, 3\}$ y $\alpha \in \Z[\sqrt{n}] \mid \alpha \neq 0 \y \alpha \notin \cc{U}(\Z[\sqrt{n}])$ es decir,
    $N(\alpha) \notin \{0, -1, 1\}$. Entonces:
    $$\Z[\sqrt{n}]/\alpha \Z[\sqrt{n}] \mbox{ es un cuerpo } \Longleftrightarrow \alpha \mbox{ es irreducible.}$$
\end{coro}~\\

Por tanto, sabemos que si $p \in \Z$ es primo $\Longrightarrow \Z_p$ es un cuerpo finito de $p$ elementos:
$$\Z_p = \{0, 1, \ldots, p-1\}$$

\begin{prop}
    \label{propAntesMoore}
    Sea $p \in \Z$ irreducible (primo), consideramos $\Z_p[x]$ y $f \in \Z_p[x] \mid f = a_0 + a_1x + \ldots + a_nx^n$ $n\geq1$.
    Entonces:
    $$\Z_p[x]/f\Z_p[x]\mbox{ es finito}$$
\begin{proof}
    \ \\
    $\forall g+f\Z_p[x] \in \Z_p[x]/f\Z_p[x]$ dividimos $g$ entre $f$:
    $$g=fq+r \mid r=0 \o grd(r)<grd(f)=n$$
    $$r=b_0+b_1x+\ldots+b_{n-1}x^{n-1}~~b_i \in \Z_p ~\forall i \in \{1, \ldots, n\}$$
    Luego $g-r=fq \Longrightarrow g\equiv r\mod(f) \Longrightarrow g+f\Z_p[x] = r+f\Z_p[x]$\newline
    Por lo que todos los elementos de $\Z_p[x]/f\Z_p[x]$ son de la forma:
    $$c_0 + c_1x + \ldots + c_{n-1}x^{n-1}~~\forall c_i \in \Z_p~\forall i \in \{1, \ldots, n\}$$
    Es decir, en $\Z_p[x]/f\Z_p[x]$ hay $p^n$ elementos.
\end{proof}
\end{prop}

\begin{teo}[Teorema de Moore]
    Sea $p \in \Z$ irreducible y $f \in \Z_p[x]$ irreducible. Entonces:
    $$\Z_p[x]/f\Z_p[x] \mbox{ es un cuerpo con } p^n \mbox{ elementos.}$$
    $$\Z_p[x]/f\Z_p[x] =: \mathbb{F}p^n$$
\begin{proof}
    Véase el Teorema \ref{teoEquivAnilloCociente} y la Proposición \ref{propAntesMoore}.
\end{proof}
\end{teo}

\begin{definicion}[Primos relativos]
    Sea $A$ un DI, dos elementos $a,b\in A$ diremos que son \textbf{primos relativos} si:
    $$\mcd(a,b) = 1$$
\end{definicion}

\begin{definicion}[Función de Euler]
    \textbf{La función de Euler} es una aplicación:
    $$\varphi:\N\setminus\{0\}\rightarrow\N\setminus\{0\}$$
    Definida por:
    $$\varphi(n) = |\{m \in \N \mid 1\leq m\leq n \y \mcd(m,n)=1\}|$$
\end{definicion}~\\

Es decir, la función de Euler cuenta el número de primos relativos que son menores a un cierto $n$:
$$\varphi(1)=1$$
$$\varphi(2)=1$$
$$\varphi(3)=2$$
$$\varphi(4)=2$$
$$\varphi(5)=4$$

\bigskip

Notemos que:
$$\varphi(n) = |\cc{U}(\Z_n)|$$
Ya que:
$$\cc{U}(\Z_n) = \{m \in \Z_n \mid \mcd(m,n) = 1\}$$

\begin{prop}
    Sean $A$ y $B$ dos anillos conmutativos. Entonces:
    $$A \times B$$
    Es un anillo conmutativo.
\begin{proof}
    Buscamos definir unas aplicaciones suma y producto que cumplan con las condiciones para que $A \times B$ sea un anillo
    conmutativo.\newline
    Definimos $+:A\times B\rightarrow A\times B$ como:
    $$(a,b)+(c,d) := (a+c,b+d)\hspace{1cm}\forall (a,b),(c,d) \in A\times B$$
    Y definimos también $\cdot:A\times B\rightarrow A\times B$ como:
    $$(a,b)\cdot(c,d) := (ac, bd)\hspace{1cm}\forall (a,b),(c,d) \in A\times B$$
    Veamos que con esta suma y producto $A\times B$ es un anillo conmutativo: $\forall (a,b),(c,d),(e,f) \in A \times B$:\\

    
    Asociativa de la suma:
    $$((a,b)+(c,d))+(e,f) = (a+c,b+d)+(e,f) = (a+c+e,b+d+f) =$$
    $$=(a,b) + (c+e,d+f) = (a,b) + ((c,d)+(e,f))$$
    Conmutativa de la suma:
    $$(a,b)+(c,d) = (a+c,b+d) = (c+a,d+b) = (c,d)+(a,b)$$
    Existencia de elemento neutro de la suma:
    $$(0,0) \in A\times B \mid (0,0)+(a,b) = (a+0,b+0) = (a,b)$$
    Existencia de opuesto:
    $$(-a,-b) \in A\times B \mid (-a,-b)+(a+b) = (-a+a, -b+b) = (0,0)$$
    Asociativa del producto:
    $$((a,b)(c,d))(e,f) = (ac,bd)(e,f) = (ace,bdf) =$$
    $$=(a,b)(ce,df) = (a,b) ((c,d)(e,f))$$
    Conmutativa del producto:
    $$(a,b)(c,d) = (ac,bd) = (ca,db) = (c,d)(a,b)$$
    Existencia de elemento netro del producto:
    $$(1,1) \in A\times B \mid (1,1)(a,b) = (a\cdot 1, b\cdot 1) = (a,b)$$
    Propiedad distributiva:
    $$(a,b)((c,d)+(e,f)) = (a,b)(c+e,d+f) = (a(c+e),b(d+f)) =$$
    $$= (ac+ae,bd+bf) = (ac,bd) + (ae,bf) = ((a,b)(c,d))+((a,b)(e,f))$$
\end{proof}
\end{prop}



Notemos que si tenemos $A$, $B$ anillos conmutativos y consideramos $A\times B$, este último conjunto puede ser también un anillo
conmutativo con suma y producto distintas a las especificadas en la proposición.

\begin{teo}[Teorema chino del resto]
    \ \\
    \label{teoChinoResto}
    Sea $A$ un DE y $m, n \in A \mid \mcd(m,n) =1$. Entonces:
    $$A/(mn)A \cong A/mA \times A/nA$$
\begin{proof}
    Definimos la aplicación $f:A\rightarrow A/mA \times A/nA$ por: $$f(a) = (a+mA, a+nA)~~\forall a \in A$$
    Que es un homomorfismo de anillos. $\forall a,b \in A$:
    $$f(a+b) = ((a+b)+mA, (a+b)+nA) = ((a+mA)+(b+mA), (a+nA)+(b+nA)) =$$
    $$=(a+mA, a+nA)+(b+mA,b+nA) = f(a) + f(b)$$
    $$f(ab) = ((ab)+mA, (ab)+nA) = ((a+mA)(b+mA), (a+nA)(b+nA)) =$$
    $$=(a+mA, a+nA)(b+mA,b+nA) = f(a)f(b)$$
    $$f(1) = (1+mA, 1+nA)$$
    Veamos además que $f$ es un epimorfismo (que es sobreyectiva):\newline
    Sea $(a+mA, b+nA) \in A/mA \times A/nA$, consideramos el sistema:
    $$\left\{ \begin{array}{l}
            x\equiv a\mod(m) \\
            x\equiv b\mod(n)
        \end{array} \right.$$
    Que tiene solución, ya que $a\equiv b\mod(\mcd(n,m)) \Longleftrightarrow a\equiv b\mod(1)$, cierto $\forall a,b \in A$\\

    
    Luego $\exists x_0 \in A$ tal que:
    $$\left. \begin{array}{llc}
            x_0\equiv a\mod(m) & \Longleftrightarrow & x_0+mA = a+mA \\
            x_0\equiv b\mod(n) & \Longleftrightarrow & x_0+nA = b+nA
        \end{array} \right\} \Longrightarrow f(x_0) = (a+mA, b+nA)$$

    
    Aplicando el Primer Teorema de Isomorfía, tenemos que $f$ induce un isomorfismo:
    $$\overline{f}:A/Ker(f)\rightarrow Img(f)=A/mA\times A/nA$$
    $$\overline{f}(a+Ker(f)A):=f(a) = (a+mA, a+nA)~~\forall a+Ker(f)A \in A/Ker(f)$$

    \bigskip
    
    Veamos que $Ker(f) = (mn)A$:
    $$Ker(f) = \{a \in A \mid f(a) = (0+mA, 0+nA)\} = \{a \in A \mid a+mA=0+mA \y a+nA = 0+nA\} = $$
    $$= \{a \in A \mid a\equiv 0\mod(m) \y a\equiv 0\mod(n)\} = \{a \in A \mid m|a \y n|a \}$$

    \bigskip
    
    Si $a \in Ker(f) \Longrightarrow m|a \y n|a \Longrightarrow \mcm(m,n)|a$.
    $$\left. \begin{array}{l}
            \mcm(m,n)|a \\
            \mcd(m,n) = 1
        \end{array} \right\} \Longrightarrow \mcm(m,n) = mn \Longrightarrow mn|a \Longrightarrow a\in (mn)A$$
    Luego $Ker(f) \subseteq (mn)A$.

    \bigskip
    
    Si $a \in (mn)A \Longrightarrow f(a) = (a+mA, a+nA)$. Pero:\newline
    $a \in (mn)A \Longrightarrow a \in mA \Longrightarrow a+mA = 0+mA$\newline
    $a \in (mn)A \Longrightarrow a\in nA \Longrightarrow a+nA = 0+nA$\newline
    Luego $a \in Ker(f) \Longrightarrow (mn)A \subseteq Ker(f)$\\

    
    Y por doble inclusión tenemos que $Ker(f) = (mn)A$.\\

    
    Por lo que tenemos un isomorfismo:
    $$\overline{f}:A/(mn)A\rightarrow A/mA\times A/nA$$
    Por lo que $A/(mn)A \cong A/mA\times A/nA$
\end{proof}
\end{teo}


\begin{coro}
    de \ref{teoChinoResto}.\newline
    Sean $m$, $n \in \Z \mid m,n \geq 2$ y $\mcd(m,n) = 1$. Entonces:
    $$\Z_{mn} \cong \Z_m \times \Z_n$$
\end{coro}~\\

Notemos que si $f:A\rightarrow B$ es un isomorfismo de anillos, la restricción de $f$ a $\cc{U}(A)$ nos da una biyección
$$f:\cc{U}(A)\rightarrow \cc{U}(B)$$
Ya que si $f$ es un homomorfismo y $a \in \cc{U}(A) \Longrightarrow f(a) \in \cc{U}(B)$.\newline
Por tanto, $|\cc{U}(A)| = |\cc{U}(B)|$.

\bigskip

Aunque no nos será necesario ahora, damos la definición de número primo:
\begin{definicion}[Primo]
    \label{defPrimo}
    Sea $A$ un DI. Un elemento $p \in A$ diremos que es \underline{primo} si $p \neq 0$, $p \notin \cc{U}(A)$ y siempre que:
    $$p|ab \Longrightarrow p|a \y p|b\hspace{1cm}\forall a,b \in A$$
    Considerando el contrarrecíproco, que si $a,b \in A \mid p\not|a \y p\not|b \Longrightarrow p\not|ab$
\end{definicion}

\begin{lema}
    \label{lemaParaPropEuler}
    Sea $p$ primo, $p\geq 2$, $m \in \N$ con $m \geq 1$ y $e \in \N \mid e \geq 1$. Entonces:
    $$\mcd(m,p^e) = 1 \Longleftrightarrow p\not|m$$
% \begin{proof}
%   %// TODO:
% \end{proof}
\end{lema}

\begin{prop}
    \label{propFuncionEuler}
    Sea $\varphi$ la función de Euler. Se verifica:\par
    1) Sean $m,n \in \N \mid m,n\geq 2 \y \mcd(m,n)=1 \Longrightarrow \varphi(mn)=\varphi(m)\varphi(n)$.\par
    2) Sea $p$ primo con $p\geq 2$ y $e\in \N \mid e\geq 1 \Longrightarrow \varphi(p^e) = p^{e-1}(p-1)$.\par
    3) Sea $n\in \N$ con $n\geq 2$ y $n=p_1^{e_1}p_2^{e_2}\ldots p_r^{e_r}$ su factorización en primos:
    $$\varphi(n) = p_1^{e_1-1}p_2^{e_2-1}\ldots p_r^{e_r-1}(p_1-1)(p_2-1)\ldots(p_r-1)$$
\begin{proof}
    \ \\
    1) \newline
    $$\varphi(mn) = |\cc{U}(\Z_{mn})|$$
    Como $\mcd(m,n)=1$, aplicando el Teorema Chino del Resto:
    $$\Z_{mn}\cong \Z_m\times \Z_n$$ y dicho isomorfismo induce una biyección:
    $$\cc{U}(\Z_{mn}) \cong \cc{U}(\Z_m \times \Z_n)$$
    $$\varphi(mn) = |\cc{U}(\Z_{mn})| = |\cc{U}(\Z_m \times Z_n)| = |\cc{U}(\Z_m) \times \cc{U}(\Z_n)| = |\cc{U}(\Z_m)||\cc{U}(\Z_n)| = \varphi(m)\varphi(n)$$

    \ \\
    
    2) \newline
    $$\varphi(p^e) = |\{m \in \N \mid 1\leq m\leq p^e \y \mcd(m,p^e) = 1\}| =$$
    $$= p^e - |\{m \in \Z \mid 1\leq m\leq p^e \y \mcd(m,p^e) \neq 1\}|$$
    Por el Lema \ref{lemaParaPropEuler} tenemos que $\mcd(m,p^e)=1 \Longleftrightarrow p\not|m$. Luego:\newline
    $$\mcd(m,p^e)\neq 1 \Longleftrightarrow p|m$$
    $$\varphi(p^e) = p^e - |\{m \in \N \mid 1 \leq m \leq p^e \y p|m\}| =$$
    $$=p^e - |\{pk \mid 1\leq k \leq p^{e-1}\}| = p^e - p^{e-1} = p^{e-1}(p-1)$$

    \ \\
    
    3) \newline
    $$\varphi(n) = \varphi(p_1^{e_1}p_2^{e_2}\ldots p_r^{e_r}) \mathop{=}^{1)}
        \varphi(p_1^{e_1})\varphi(p_2^{e_2})\ldots\varphi(p_r^{e_r}) \mathop{=}^{2)}$$
    $$=p_1^{e_1-1}(p_1-1)p_2^{e_2-1}(p_2-1)\ldots p_r^{e_r-1}(p_r-1) =$$
    $$=p_1^{e_1-1}p_2^{e_2-1}\ldots p_r^{e_r-1}(p_1-1)(p_2-1)\ldots(p_r-1)$$
\end{proof}
\end{prop}


\begin{prop}
    \label{propTeoFermat}
    Sea $A$ un anillo conmutativo con $|\cc{U}(A)|=r \geq 1$ y sea $u \in \cc{U}(A)$. Se verifica que:
    $$u^r = 1$$
\begin{proof}
    Sea $\cc{U}(A) = \{u_1, u_2, \ldots, u_r\}$. Tomamos $u \in \cc{U}(A)$.\newline
    Definimos $f:\cc{U}(A)\rightarrow \cc{U}(A)$ por:
    $$f(u_i):=uu_i\hspace{1cm}\forall u_i \in \cc{U}(A)$$
    Veamos que $f$ es inyectiva. Sean $u_i, u_j \in \cc{U}(A) \mid f(u_i)=f(u_j)$:
    $$f(u_i)=f(u_j) \Longrightarrow uu_i = uu_j \Longrightarrow u_i = u^{-1}uu_j = u_j$$
    Como $\cc{U}(A)$ es finito $\Longrightarrow f$ es biyectiva $\Longrightarrow$ es sobreyectiva $\Longrightarrow Img(f)=\cc{U}(A)$
    $$Img(f) = \{f(u_i) \mid u_i \in \cc{U}(A)\} = \{uu_1, uu_2, \ldots, uu_r\} = \cc{U}(A) = \{u_1, u_2, \ldots, u_r\}$$
    Luego:
    $$\prod_{i=1}^r uu_i = \prod_{i=1}^r u_i \Longrightarrow u^r\prod_{i=1}^r u_i = \prod_{i=1}^r u_i \Longrightarrow u^r = 1$$
\end{proof}
\end{prop}


\begin{teo}[de Euler]
    \label{teoEular}
    Sea $n \in \Z \mid n\geq 2$. $\forall a \in \Z \mid \mcd(a,b)=1$. Se verifica:
    $$a^{\varphi(n)}\equiv 1\mod(n)$$
\begin{proof}
    \ \\
    Sea $a \in \Z$ con $n \in \Z \mid n\geq 2 \mid \mcd(a,n)=1$. Entonces, por el Teorema \ref{teoEquivAnilloCociente}:
    $$\mcd(a,n)=1 \Longrightarrow a+n\Z \in \cc{U}(\Z/n\Z)$$
    Como $\varphi(n) = |\cc{U}(\Z/n\Z)|$, por la Proposición \ref{propTeoFermat}:
    $$\left. \begin{array}{l}
            (a+n\Z)^{\varphi(n)}=1+n\Z \\
            (a+n\Z)^{\varphi(n)}=a^{\varphi(n)}+n\Z
        \end{array} \right\} \Longrightarrow a^{\varphi(n)}\equiv 1\mod(n)$$
\end{proof}
\end{teo}

\begin{teo}[El pequeño Teorema de Fermat]
    \ \\
    \label{teoFermat}
    Sea $p \in \Z$, $p\geq 2$ primo. $\forall a \in \Z \mid p\not|a$. Se verifica que:
    $$a^{p-1}\equiv 1\mod(p) \Longleftrightarrow a^p\equiv a\mod(p)$$
\begin{proof}
    Sea $p \in \Z \mid p\geq 2$ primo $a \in \Z \mid p\not| a$. Por el Lema \ref{lemaParaPropEuler}:
    $$p\not|a \Longrightarrow \mcd(a,p)=1$$
    Luego por el Teorema \ref{teoEular}:
    $$\left. \begin{array}{l}
            a^{\varphi(p)}\equiv 1\mod(p) \\
            \varphi(p) = p-1
        \end{array} \right\} \Longrightarrow a^{p-1} \equiv 1\mod(p) \Longleftrightarrow a^p \equiv a\mod(p)$$
\end{proof}
\end{teo}

\begin{coro}
    del Teorema \ref{teoEular}.\newline
    Sea $n\geq2$. Entonces $\forall r \in \cc{U}(\Z_n)$, se verifica que:
    $$r^{\varphi(n)}=1\mbox{ en } \Z_n \Longleftrightarrow r^{-1} = r^{\varphi(n)-1}$$
\begin{proof}
    Por el Teorema \ref{teoEular}, tenemos que:
    $$r^{\varphi(n)}\equiv 1\mod(n) \Longleftrightarrow r^{\varphi(n)}=1\mbox{ en } \Z_n \Longleftrightarrow r^{-1} = r^{\varphi(n)-1}$$
    Donde el último paso se obtiene muliplicando por $r^{-1}$ en ambos lados.
\end{proof}
\end{coro}

\begin{coro}
    del Teorema \ref{teoFermat}.\newline
    Sea $p \in \Z$, $p\geq 2$ primo. Entonces $\forall r \in \Z_p$, $r\neq 0$. Se verifica que:
    $$r^{p-1}=1 \mbox{ en } \Z_p \Longleftrightarrow r^p = r \mbox{ en } \Z_p$$
\begin{proof}
    $$\Z_p = \{0, 1, \ldots, p-1\} \Longrightarrow \forall r \in \Z_p \mid r\neq 0 \Longrightarrow p\not|r$$
    Aplicando el Teorema \ref{teoFermat}, tenemos que:
    $$r^{p-1}\equiv 1\mod(p) \Longleftrightarrow r^{p-1}=1 \mbox{ en } \Z_p \Longleftrightarrow r^p = r \mbox{ en } \Z_p$$
    Donde el último paso se obtiene muliplicando por $r$ en ambos lados.
\end{proof}
\end{coro}


\begin{ejemplo}
    Calcular el resto de dividir $279^{323}$ entre $17$.\\

    
    Calculamos el resto de dividir $279$ entre $17$:
    Como $279=17 \cdot 16 + 7$:
    $$279\equiv 7\mod(17) \Longrightarrow 279^{323}\equiv 7^{323}\mod(17)$$
    Por tanto, el resto buscado es el mismo resto que al dividir $7^{323}$ entre $17$.\\

    
    Como $\mcd(7,17)=1 \Longrightarrow 7^{\varphi(17)}\equiv 1\mod(17)$, por el Teorema \ref{teoEular}.\newline
    Como $17$ es primo: $\varphi(17) = 17-1 = 16$, por la Proposición \ref{propFuncionEuler}. Por lo que:
    $$7^{16}\equiv 1\mod(17)$$
    Dividimos $323$ entre $16$:
    $$323= 20 \cdot 16 + 3$$
    $$7^{323}=7^{16\cdot 20+3} = (7^{16})^{20}\cdot 7^3 \equiv 1 \cdot 7^3 = 7^3\mod(17) $$
    Donde hemos aplicado que:
    $$7^{16}\equiv 1\mod(17) \Longrightarrow (7^{16})^{20}\equiv 1^{20}\mod(17)$$

    \ \\
    
    Finalmente, calculamos el resto de dividir $7^3$ entre $17$:
    $7^3 = 343 = 20\cdot 17 + 3 \Longrightarrow 7^3\equiv 3\mod(17)$. Luego:
    $$279^{323}\equiv 3\mod(17)$$
    Por lo que:
    $$R(279^{323};17)=3$$
\end{ejemplo}

\begin{teo}[Teorema chino del resto generalizado]
    \ \\
    Sea $A$ un DE y $m_1, m_2, \ldots, m_k \in A$ con $k \geq 2$ tales que $\mcd(m_i, m_j)=1$\newline$\forall i,j \in \{1, \ldots, k\} \mid
        i\neq j$. Entonces:
    $$A/(m_1m_2\ldots m_k)A \cong A/m_1A \times A/m_2A \times \ldots \times A/m_kA$$
% \begin{proof}
%     %// TODO:
% \end{proof}
\end{teo}

\begin{lema}
    \ \\
    Sea $A$ un DE y $m_1, m_2, \ldots, m_k \in A$ con $k \geq 2$ tales que $\mcd(m_i, m_j)=1$\newline$\forall i,j \in \{1, \ldots, k\} \mid
        i\neq j$. Entonces, $\forall a_1, a_2, \ldots, a_k \in A$, el sistema:
    $$\left\{ \begin{array}{l}
            x\equiv a_1\mod(m_1) \\
            x\equiv a_2\mod(m_2) \\
            \vdots               \\
            x\equiv a_k\mod(m_k)
        \end{array} \right.$$
    Tiene solución.\\

    
    Además, si $c \in A$ es una solución particular del sistema, entonces sl sistema es equivalente a:
    $$x\equiv c\mod(m_1m_2\ldots m_k)$$
\begin{proof}
    Realizamos inducción en $k$.\newline
    Si $k = 2$, tenemos un sistema de la forma:
    $$\left\{ \begin{array}{l}
            x\equiv a_1\mod(m_1) \\
            x\equiv a_2\mod(m_2) \\
        \end{array} \right.$$
    Que tiene solución $\Longleftrightarrow a_1\equiv a_2\mod(\mcd(m_1,m_2)) \Longleftrightarrow a_1\equiv a_2\mod(1)$, cierto siempre.\newline
    Además, si $c$ es una solución particular $\Longrightarrow$ sabemos que el sistema es equivalente a:
    $$x\equiv c\mod(\mcm(m_1,m_2))$$
    Pero como $\mcd(m_1,m_2)=1 \Longrightarrow \mcm(m_1,m_2) = m_1m_2$. Luego el sistema es equivlente a:
    $$x\equiv c\mod(m_1m_2)$$

    \ \\
    
    Sea $k>2$ y supongamos que:
    $$\left\{ \begin{array}{l}
            x\equiv a_1\mod(m_1) \\
            x\equiv a_2\mod(m_2) \\
            \vdots               \\
            x\equiv a_{k-1}\mod(m_{k-1})
        \end{array} \right.$$
    Tiene solución y que si $d\in A$ es una solución particular, entonces el sistema es equivalente a:
    $$x\equiv d\mod(m_1m_2\ldots m_{k-1})$$
    Consideramos el sistema:
    $$\left\{ \begin{array}{l}
            x\equiv a_1\mod(m_1) \\
            x\equiv a_2\mod(m_2) \\
            \vdots               \\
            x\equiv a_k\mod(m_k)
        \end{array} \right.$$
    Que es equivalente a:
    $$\left\{ \begin{array}{l}
            x\equiv d\mod(m_1m_2\ldots m_{k-1}) \\
            x\equiv a_k\mod(m_k)
        \end{array} \right.$$
    Puesto que $\mcd(m_k, m_i) = 1 ~~\forall i \in \{1, \ldots, k-1\}$.
    \newline Entonces: $\mcd(m_1m_2\ldots m_{k-1},m_k)=1$. Luego:
    $$d\equiv a_k\mod(\mcd(m_1m_2\ldots m_{k-1},m_k)) \Longleftrightarrow d\equiv a_k\mod(1) \Longleftrightarrow \mbox{ el sistema tiene solución}$$
    Además, como $\mcd(m_1m_2\ldots m_{k-1},m_k) = 1 \Longrightarrow \mcm(m_1m_2\ldots m_{k-1},m_k)=m_1m_2\ldots m_{k-1}m_k$.\\

    
    Por lo que el sistema será equivalente a (si $c$ es una solución particular):
    $$x\equiv c\mod(m_1m_2\ldots m_k)$$
\end{proof}
\end{lema}

