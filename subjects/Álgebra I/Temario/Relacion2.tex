\section{Relación II}

\begin{ejercicio}
    Dar ejemplos de relaciones binarias en un conjunto que verifiquen una sola de ls siguientes propiedades: reflexiva, simétrica, transitiva.
\end{ejercicio}

\begin{ejercicio}
    Sea $\mathbb{N} = \{0,1,2, \ldots\}$ el conjunto de los nńumeros naturales, sobre $\mathbb{N}^2 = \mathbb{N}\times \mathbb{N}$ definimos $(a,b)\sim (c,d)$ si $a+d=b+c$.
    \begin{description}
        \item [(a)] Verificar que $\sim$ es una relación de equivalencia.
        \item [(b)] Sea $f:\mathbb{N}^2\to \mathbb{Z}$ la aplicación definida por $f(a,b)=a-b$. Verificar que $f$ induce una biyección $\mathbb{N}^2 /\sim \cong \mathbb{Z}$.
    \end{description}
\end{ejercicio}

\begin{ejercicio}
    Sea $f:S\to T$ una aplicación.
    \begin{description}
        \item [(a)] Probar que $f$ define una relación de equivalencia $R_f$ en $S$, donde $aR_f b$ si $f(a) = f(b)$ (esta relación se llama \emph{la relación núcleo de} $f$).
        \item [(b)] Probar que, si $f$ es sobreyectiva, se inducie una biyección $S/R_f\cong T$.
    \end{description}
\end{ejercicio}

\begin{ejercicio}
    Sea $Y\subseteq X$ un subconjunto. Sea $f:\mathcal{P}(X)\to \mathcal{P}(Y)$ la aplicación tal que $f(A) = A\cap Y$, para cada $A\in \mathcal{P}(X)$.
    \begin{description}
        \item [(a)] Probar que $f$ es una sobreyección.
        \item [(b)] Describir la relación $R_f$, núcleo de $f$.
        \item [(c)] Probar que $f$ induce una biyección $\mathcal{P}(X)/R_f\cong \mathcal{P}(Y)$.
    \end{description}
\end{ejercicio}

\begin{ejercicio}
    Sea $R$ una relación de equivalencia sobre un conjunto $S$. La aplicación $p:S\to S/R$ definida por $p(A) = \overline{a}$ es la llamada \textbf{proyección canónica} de $S$ sobre el cociente. ¿Qué relación hay entre $R$ y $R_p$?
\end{ejercicio}

\begin{ejercicio}
    Un subconjunto $P\subseteq \mathcal{P}(S)$ es llamado una \textbf{partición del conjunto} $S$ si
    \begin{description}
        \item [(a)] $\forall A\in P$, $A\neq \emptyset $.
        \item [(b)] $\bigcup_{A\in P}A=S$.
        \item [(c)] Para cualesquiera $A, B\in P \mid A \neq B$, se verifica que $A \cap B = \emptyset $.
    \end{description}
    Así, por ejemplo, el conjunto cociente $S/R$, para $R$ una relación de equivalencia sobre $S$ es una partición.

    Sea $P$ una partición de $S$. Definimos la aplicación $p:S\to P$ por $p(a) = A$ si $a\in A$. ¿Qué relación hay entre $P$ y $S/R_p$?
\end{ejercicio}

\begin{ejercicio}
    Sea $X=\{ 1,2,3 \}$. Calcular todas las particiones de $X$.
\end{ejercicio}

\begin{ejercicio}
    Sea $X=\{ 0,1,2,3 \}$, $Y=\{ a,b,c \}$ y $f:X\to Y$ la aplicación dada por:
    \begin{equation*}
        f(0) = c \qquad f(1) = f(2) = a \qquad f(3) = b
    \end{equation*}
    Consideremos la aplicación $f^*:\mathcal{P}(Y)\to \mathcal{P}(X)$.
    \begin{description}
        \item [(a)] ¿Es $f^*$ inyectiva, sobreyectiva o biyectiva?
        \item [(b)] Describir la relación $R_{f^*}$ asociada a $f^*$ y el conjunto cociente $\mathcal{P}(Y)/R_{f^*}$.
    \end{description}
\end{ejercicio}

\begin{ejercicio}
    Sea $X$ un conjunto e $Y\subseteq X$ un subconjunto suyo. En el conjunto $\mathcal{P}(X)$ se define la siguiente relación binaria:
    \begin{equation*}
        A\sim B \Longleftrightarrow A\cap Y = B \cap Y   
    \end{equation*}
    Demostrad que dicha relación es de equivalencia. Para $X=\{ 1,2,3,4,5 \}$ e $Y=\{ 1,4 \}$, describir del conjunto cociente.
\end{ejercicio}

\begin{ejercicio}
    Sea $X$ un conjunto e $Y\in \mathcal{P}(X)$. Definimos la aplicación $f:X\to \mathcal{P}(X)$ por $f(x)=Y\cup \{x\}$, para $x\in X$ y consideramos en $X$ la relación de equivalencia $R_f$ (Ejercicio 3). Describir el conjunto cociente $X/R_f$. Si $X$ es un conjunto finito con $n$ elementos e $Y$ tiene $m$ elementos, calcular el cardinal de $X/R_f$.
\end{ejercicio}

\resetearcontador
\newpage
