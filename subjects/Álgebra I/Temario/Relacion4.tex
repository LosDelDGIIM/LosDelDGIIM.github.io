\section{Relación IV}

\begin{ejercicio}
    Argumenta si los siguientes anillos son, o no, Dominios de Integridad:
    \begin{equation*}
        \mathbb{Z}_8 \qquad \mathbb{Z}\left[\sqrt{2}\right]\qquad \mathbb{Z}_3 \qquad \mathbb{Z}\times \mathbb{Z}\qquad \mathbb{Z}_6[x] \qquad \mathbb{Z}[i] \qquad \mathbb{Z}_5[x]
    \end{equation*}
\end{ejercicio}

\begin{ejercicio}
    ¿Es el anillo definido por el conjunto $\mathbb{Z}\times \mathbb{Z}$ con las operaciones:
    \begin{equation*}
        (a,a')+(b,b') = (a+b, a'+b')\qquad (a,a')(b,b') = (ab,ab'+a'b)
    \end{equation*}
    un Dominio de Integridad? (ver Ejercicio 3 de la Relación III).
\end{ejercicio}

\begin{ejercicio}
    ¿Es el anillo definido por el conjunto $\mathbb{Z}$ de los números enteros con las operaciones:
    \begin{equation*}
        a\oplus b = a+b-1 \qquad a\otimes b = a+b-ab
    \end{equation*}
    un Dominio de Integridad? (ver Ejercicio 2 de la Relación III).
\end{ejercicio}

\begin{ejercicio}
    Demuestra que un Dominio de Integridad finito es un cuerpo.
\end{ejercicio}

\begin{ejercicio}
    Sea $n\in \mathbb{Z}$ un entero no cuadrado en $\mathbb{Z}$. Demuestra que el cuerpo de fracciones de $\mathbb{Z}\left[\sqrt{n}\right]$ es $\mathbb{Q}\left[\sqrt{n}\right]$.
\end{ejercicio}

\begin{ejercicio}
    Se define el cuerpo $\mathbb{Q}(x)$ como el cuerpo de fracciones del anillo $\mathbb{Z}[x]$, esto es, $\mathbb{Q}(x) := \mathbb{Q}(\mathbb{Z}[x])$. Demuestra que $\mathbb{Z}[x]$ y $\mathbb{Q}[x]$ tienen el mismo cuerpo de fracciones. Esto es:
    \begin{equation*}
        \mathbb{Q}(\mathbb{Q}[x]) = \mathbb{Q}(x)
    \end{equation*}
\end{ejercicio}

\begin{ejercicio}
    Sea $A=\{ \frac{m}{2^k} \in \mathbb{Q} \mid m \in \mathbb{Z} \land k \geq 0\}$. Argumentar que:
    \begin{description}
        \item [(a)] $A$ es subanillo de $\mathbb{Q}$.
        \item [(b)] $\mathbb{Z}\subsetneq A$.
        \item [(c)] El cuerpo de fracciones de $A$ es el mismo que el de $\mathbb{Z}$, o sea, $\mathbb{Q}$.
    \end{description}
\end{ejercicio}

\begin{ejercicio}
    Sea $A$ un DI y consideremos en $A$ la relación binaria $\sim$ de ser asociados. Esto es, $a\sim b$ si $a$ es asociado con $b$.
    \begin{description}
        \item [(a)] Probar que $\sim$ es una relación de equivalencia en $A$.
        \item [(b)] Sea $A/\sim = \{[a] \mid a\in A\}$, el correspondiente conjunto cociente. Establecemos entre sus elementos ls relación por la cual $[a]\leq [b]$ si $a$ es un divisor de $b$ en el anillo $A$. ¿Está bien definida esa relación en $A/\sim$? ¿Es una relación de orden?
    \end{description}
\end{ejercicio}

\begin{ejercicio}
    Para $n$ un número natural, calcular $\text{mcd}(n,n^2)$, $\text{mcd}(n,n+1)$ y $\text{mcd}(n,n+2)$.
\end{ejercicio}

\begin{ejercicio}
    ¿Podremos rellenar con precisión un depósito de 5388033 litros usando un recipiente de 371? En caso afirmativo, ¿cuántas veces usaremos el recipiente?
\end{ejercicio}

\begin{ejercicio}
    Determinar, si existe, un polinomio $p(x)\in \mathbb{Q}[x]$ tal que:
    \begin{equation*}
        \left(\dfrac{3}{5}x^3+\dfrac{1}{2}x+\dfrac{2}{3}\right)p(x) = \dfrac{9}{20}x^5 + \dfrac{147}{40}x^3+\dfrac{1}{2}x^2+\dfrac{11}{4}x+\dfrac{11}{3}
    \end{equation*}
\end{ejercicio}

\begin{ejercicio}
    Calcular el cociente y el resto de dividir, en el anillo $\mathbb{Q}[x]$, el polinomio $p$ entre el polinomio $q$:
    \begin{gather*}
        p = \dfrac{9}{20}x^5 + \dfrac{147}{40}x^3+\dfrac{1}{2}x^2+\dfrac{17}{4}x+\dfrac{17}{3} \\
        q = \dfrac{3}{5}x^3+\dfrac{1}{2}x+\dfrac{2}{3}
    \end{gather*}
\end{ejercicio}

\begin{ejercicio}
    Determinar, si existe, un polinomio $p(x)\in \mathbb{Z}_3[x]$ tal que
    \begin{equation*}
        (2x^2+x+2)p(x) = 2x^7 + x^6 + 2x^4 + 2
    \end{equation*}
\end{ejercicio}

\begin{ejercicio}
    En el anillo $\mathbb{Z}[i]$, calcular cociente y resto de dividir $1+15i$ entre $3+5i$.
\end{ejercicio}

\begin{ejercicio}
    ¿Es $2+5\sqrt{3}$ un divisor de $39-9\sqrt{3}$ en el anillo $\mathbb{Z}\left[\sqrt{3}\right]$?
\end{ejercicio}

\begin{ejercicio}
    Resolver en $\mathbb{Z}$ las ecuaciones diofánticas
    \begin{equation*}
        10x + 46y = 4050 \qquad 60x+36y = 12 \qquad 35x+6y=8 \qquad 12x+18y = 11
    \end{equation*}
\end{ejercicio}

\begin{ejercicio}
    ``Cuarenta y seis náufragos cansados arribaron a una bella isla. Allí encontraron ciento veintiséis montones de cocos, de no más de cincuenta cada uno, y catorce cocos sueltos, y se los repartiron equitativamente\ldots'' ¿Cuántos cocos había en cada montón?
\end{ejercicio}

\begin{ejercicio}
    Disponemos de 15 euros para comprar 40 sellos de correso, de 10, 40, y 60 céntimos y, al menos, necesitamos 2 de cata tipo ¿Cuántos sellos de cada clase podremos comprar?
\end{ejercicio}

\begin{ejercicio}
    En una torre eléctrica se nos ha roto una pata de 4 m de altura. Para equilibrarlo provisionalmente, disponemos de 7 discos de madera de 50 cm de grosor y de otros 12 de 30 cm. ¿Cuál de las siguientes afirmaciones es verdadera?
    \begin{itemize}
        \item No podremos equilibrar la torre.
        \item Podremos equilibrar la torre, y de una única manera.
        \item Podremos equilibrar la torre, y de dos únicas maneras.
        \item Podremos equilibrar la torre, y de más de 2 maneras distintas.
    \end{itemize}
\end{ejercicio}

\begin{ejercicio}
    Calcular el máximo común divisor y el mínimo común múltiplo, en el anillo $\mathbb{R}[x]$, de los polinomios $x^3-2x^2-5x+6$ y $x^3-3x^2-x+3$. Encontrar todos los polinomios $p(x)$ y $g(x)$ en $\mathbb{R}[x]$, ambos de grado 3, tales que
    \begin{equation*}
        (x^3-2x^2-5x+6)p(x)+(x^5+x^4-x-1)g(x) = x^4 + x^2+1
    \end{equation*}
\end{ejercicio}

\begin{ejercicio}
    En el anillo $\mathbb{Z}\left[\sqrt{2}\right]$, calcular
    \begin{equation*}
        \text{mcd}\left(2-3\sqrt{-2}, 1+\sqrt{-2}\right)\qquad \text{mcm}\left(2-3\sqrt{-2},2+\sqrt{-2}\right)
    \end{equation*}
\end{ejercicio}

\begin{ejercicio}
    En $\mathbb{Z}\left[\sqrt{3}\right]$, calcula $\text{mcd}(3+\sqrt{3},2)$ y $\text{mcm}(3+\sqrt{3},2)$.
\end{ejercicio}

\begin{ejercicio}
    Determina los enteros $x,y\in \mathbb{Z}$ tales que, en el anillo $\mathbb{Z}[i]$, se verifique la ecuación 
    \begin{equation*}
        (-2+3i)x + (1+i)y = 1 + 11i
    \end{equation*}
\end{ejercicio}

\begin{ejercicio}
    Resolver la siguiente ecuación en el anillo $\mathbb{Z}\left[\sqrt{2}\right]$:
    \begin{equation*}
        (4+\sqrt{2})x+(6+4\sqrt{2})y = \sqrt{2}
    \end{equation*}
\end{ejercicio}

\resetearcontador
\newpage
