\section{Cuestionario VII}

\begin{ejercicio}
    En relación a las siguientes proposiciones sobre elementos de un DE, selecciona las verdaderas:
    \begin{itemize}
        \item Si $\text{mcd}(a,b)=1$, entonces $\text{mcd}(a,b^n)=1$ para todo $n \in \mathbb{N}$.
        \item Si $a \equiv a'\mod(b)$, entonces $\text{mcd}(a,b)=\text{mcd}(a',b)$.
        \item Si $a\equiv a'\mod(b)$, entonces $\text{mcm}(a,b)=\text{mcm}(a',b)$.
    \end{itemize}
\end{ejercicio}

\begin{ejercicio}
    Entre las siguientes ecuaciones en congruencias, selecciona las que tienen solución.
    \begin{itemize}
        \item En $\mathbb{Z}$, $6x\equiv 10 \mod (45)$.
        \item En $\mathbb{Z}$, $100x\equiv 20\mod (15)$.
        \item En $\mathbb{Z}[i]$, $(2+2i)x\equiv 5\mod(3-i)$.
    \end{itemize}
\end{ejercicio}

\begin{ejercicio}
    Entre las siguientes afirmaciones relativas a ecuaciones en el anillo $\mathbb{Z}_{64}$, selecciona las que son verdad.
    \begin{itemize}
        \item $12x=28$ tiene $4$ soluciones.
        \item $14x=28$ tiene $4$ soluciones.
        \item $12x=30$ tiene $4$ soluciones.
    \end{itemize}
\end{ejercicio}

\begin{ejercicio}
    Entre las siguientes proposiciones, selecciona las verdaderas.
    \begin{itemize}
        \item El anillo $\mathbb{Z}_{900}$ tiene 240 unidades.
        \item $14^{20}\equiv 1\mod (33)$.
        \item $3^{16}=3$ en $\mathbb{Z}_{16}$.
    \end{itemize}
\end{ejercicio}

\begin{ejercicio}
    Sea $p$ un número primo y considérese la congruencia $ax\equiv 1\mod (p^2)$. En relación a las siguientes proposiciones, selecciona las verdaderas:
    \begin{itemize}
        \item No tiene solución, pues $p^2$ no es primo.
        \item Tiene solución si y sólo si la congruencia $ax\equiv 1\mod (p)$ tiene solución.
        \item Tiene solución salvo que $a$ sea múltiplo de $p^2$.
    \end{itemize}
\end{ejercicio}

\newpage
\ % --------------------------------------------------------------------------------
\resetearcontador

\begin{ejercicio}
    En relación a las siguientes proposiciones sobre elementos de un DE, selecciona las verdaderas:
    \begin{itemize}
        \item \underline{Si $\text{mcd}(a,b)=1$, entonces $\text{mcd}(a,b^n)=1$ para todo $n \in \mathbb{N}$.}
        \item \underline{Si $a \equiv a'\mod(b)$, entonces $\text{mcd}(a,b)=\text{mcd}(a',b)$.}
        \item Si $a\equiv a'\mod(b)$, entonces $\text{mcm}(a,b)=\text{mcm}(a',b)$.
    \end{itemize}

    \noindent
    \textbf{Justificación}:
    \begin{itemize}
        \item Es cierto, lo probamos por inducción:
            \begin{description}
                \item [Para $n=0$:] 
                    $\text{mcd}(a,b^0) = \text{mcd}(a,1)=1$, cierto.
                \item [Para $n=1$:] 
                    $\text{mcd}(a,b)=1$, cierto.
                \item [Supuesto cierto para $n-1$, lo vemos para $n$:] 
                    \begin{equation*}
                        \left.\begin{array}{r}
                            \text{mcd}(a,b)=1 \\
                            \text{mcd}(a,b^{n-1}) = 1
                    \end{array}\right\} \text{mcd}(a,b^n) = \text{mcd}(a,b^{n-1}b) = 1
                    \end{equation*}
            \end{description}
        \item Es cierto, sea $A$ el DE:
            \begin{align*}
                a\equiv a'\mod(b) &\Longrightarrow \exists q\in A \mid a-a' = qb \\
                                  &\Longrightarrow  a'=a-qb
            \end{align*}
            \begin{equation*}
                \text{mcd}(a,b) = \text{mcd}(a-qb,b) = \text{mcd}(a',b)
            \end{equation*}
        \item Es falso, por ejemplo en $\mathbb{Z}$, sean $a=6$, $a' = 2$, $b = 4$
            \begin{gather*}
                6\equiv 2\mod (4) \\
                \text{mcm}(6,4) = 12 \neq 4 = \text{mcm}(2,4)
            \end{gather*}
    \end{itemize}
\end{ejercicio}

\begin{ejercicio}
    Entre las siguientes ecuaciones en congruencias, selecciona las que tienen solución.
    \begin{itemize}
        \item En $\mathbb{Z}$, $6x\equiv 10 \mod (45)$.
        \item \underline{En $\mathbb{Z}$, $100x\equiv 20\mod (15)$.}
        \item En $\mathbb{Z}[i]$, $(2+2i)x\equiv 5\mod(3-i)$.
    \end{itemize}

    \noindent
    \textbf{Justificación}:
    \begin{itemize}
        \item $\text{mcd}(6,45)=3$, como $3\nmid 10 \Longrightarrow$ no tiene solución.
        \item $\text{mcd}(100,15)=5$, como $5\mid 20 \Longrightarrow $ tiene solución:
            \begin{equation*}
                20x\equiv 4\mod (3) \qquad \text{mcd}(20,3)=1
            \end{equation*}
            \begin{align*}
                1 = 20(-1)+7\cdot 3 &\Longrightarrow 20\cdot 1=-1\mod (3) \\
                                    &\Longrightarrow 20(-4)\equiv 4\mod (3)
            \end{align*}
            \begin{gather*}
                x_0 = -4 \text{\ es\ solución\ particular} \\
                x_0 = 2 \text{\ es\ solución\ óptima} \\
                x_0 = 2+3k\quad k\in \mathbb{Z}
            \end{gather*}
        \item Calculamos $\text{mcd}(2+2i, 3-i)$ en $\mathbb{Q}[i]$:
            \begin{equation*}
                \dfrac{3-i}{2+2i} = \dfrac{(2-2i)(3-i)}{8} = \dfrac{6-2i-6i-2}{8}=\dfrac{4}{8}-\dfrac{8i}{8} = \dfrac{1}{2}-i
            \end{equation*}
            Tenemos $q=i$, $r = 1+i$
            \begin{equation*}
                \begin{array}{rcl}
                    r_i & u_i & v_i \\
                    3-i & 1 & 0 \\
                    2+2i & 0 & 1 \\
                    1+i & 1 & -i 
                \end{array}
            \end{equation*}
            Existe solución $\Longleftrightarrow 1+i\mid 5$, pero como $1+i\nmid 5$, no existe solución.
    \end{itemize}
\end{ejercicio}

\begin{ejercicio}
    Entre las siguientes afirmaciones relativas a ecuaciones en el anillo $\mathbb{Z}_{64}$, selecciona las que son verdad.
    \begin{itemize}
        \item \underline{$12x=28$ tiene $4$ soluciones.}
        \item $14x=28$ tiene $4$ soluciones.
        \item $12x=30$ tiene $4$ soluciones.
    \end{itemize}

    \noindent
    \textbf{Justificación}:
    \begin{itemize}
        \item 
            \begin{align*}
                12x &\equiv 28\mod(64)\\
                6x &\equiv 14\mod(32) \\
                3x &\equiv 7\mod(16)
            \end{align*}
            Como $\text{mcd}(16,3)=1$, tiene solución.
            \begin{align*}
                1 = 16\cdot 1 + 3(-5) &\Longrightarrow 3\cdot 5\equiv -1\mod(16) \\
                                      &\Longrightarrow 3\cdot 5(-7)\equiv 7\mod(16)
            \end{align*}
            \begin{gather*}
                5(-7) = -35 \text{\ es\ solución\ particular} \\
                x_0 = 13 \text{\ es\ solución\ óptima} \\
                x = 13 + 16k\quad k\in \mathbb{Z}
            \end{gather*}
            Por tanto:
            \begin{equation*}
                \begin{array}{ll}
                    x_1 = 13 & x_2 = 29 \\
                    x_3 = 45 & x_4 = 61
                \end{array}
            \end{equation*}
            Tiene 4 soluciones.
        \item 
            \begin{align*}
                14x &\equiv 28\mod(64) \\
                7x &\equiv 14\mod(32)
            \end{align*}
            $\text{mcd}(7,32)=1$, tiene solución.
            \begin{align*}
                1 = 32\cdot 2+7(-9) &\Longrightarrow 7\cdot 9\equiv -1\mod (32) \\
                                    &\Longrightarrow 7\cdot 9(-14)\equiv 14\mod(32)
            \end{align*}
            \begin{gather*}
                x_0 = 9(-14) = -126 \text{\ es\ solución\ particular} \\
                y_0 = 2 \text{\ es\ solución\ óptima} \\
                x = 2+23k\quad k\in \mathbb{Z}
            \end{gather*}
            Por tanto:
            \begin{gather*}
                x_1 = 2 \\
                x_2 = 34
            \end{gather*}
            No tiene 4 soluciones, es falso.
            
        \item 
            \begin{align*}
                12x &\equiv 30\mod(64) \\
                6x &\equiv 15\mod(32)
            \end{align*}
            \begin{equation*}
                \text{mcd}(6,32) = 2 \nmid 15 \Longrightarrow \text{\ no\ tiene\ solución}
            \end{equation*}
            Es falso.
    \end{itemize}
\end{ejercicio}

\begin{ejercicio}
    Entre las siguientes proposiciones, selecciona las verdaderas.
    \begin{itemize}
        \item \underline{El anillo $\mathbb{Z}_{900}$ tiene 240 unidades.}
        \item \underline{$14^{20}\equiv 1\mod (33)$.}
        \item $3^{16}=3$ en $\mathbb{Z}_{16}$.
    \end{itemize}

    \noindent
    \textbf{Justificación}:
    \begin{itemize}
        \item 
            \begin{equation*}
                |U(\mathbb{Z}_{900})| = \varphi(900) = \varphi(3^2 \cdot 2^2 \cdot 5^2) = 3\cdot 2\cdot 5\cdot 2\cdot 1\cdot 4 = 240
            \end{equation*}
        \item 
            \begin{equation*}
                \left.\begin{array}{r}
                    \varphi(33) = \varphi(3\cdot 11) = 2\cdot 10 = 20 \\
                    \text{mcd}(14,33) = 1
            \end{array}\right\} \mathop{\Longrightarrow}^{\text{Fermat}} 14^{20}\equiv 1\mod(33)
            \end{equation*}
        \item 
            \begin{align*}
                \left.\begin{array}{r}
                    \varphi(16) = \varphi(2^4) = 2^3 \cdot 1 =8 \\
                    \text{mcd}(3,16) = 1
            \end{array}\right\} &\Longrightarrow 3^8\equiv 1\mod (16) \Longrightarrow 3^{16}\equiv 1\mod(16) \\
            &\Longrightarrow 3^{16}\not\equiv 3\mod(16)
            \end{align*}
    \end{itemize}
\end{ejercicio}

\newpage
\begin{ejercicio}
    Sea $p$ un número primo y considérese la congruencia $ax\equiv 1\mod (p^2)$. En relación a las siguientes proposiciones, selecciona las verdaderas:
    \begin{itemize}
        \item No tiene solución, pues $p^2$ no es primo.
        \item \underline{Tiene solución si y sólo si la congruencia $ax\equiv 1\mod (p)$ tiene solución.}
        \item Tiene solución salvo que $a$ sea múltiplo de $p^2$.
    \end{itemize}

    \noindent
    \textbf{Justificación}:
    \begin{align*}
        \text{La\ equación\ tiene\ solución\ } &\Longleftrightarrow \text{mcd}(a,p^2) \mid 1 \Longleftrightarrow \text{mcd}(a,p^2) = 1 \\
                                              &\Longleftrightarrow \text{mcd}(a,p)= 1 \Longleftrightarrow ax\equiv 1\mod (p) \text{\ tiene\ solución}
    \end{align*}
    Luego la segunda opción es verdadera. Estudiamos ahora la tercera, si $a = kp^2$ con $k \in A \Longrightarrow \text{mcd}(a,p^2) = p^2$ por lo que es cierto que no tiene solución. Sin embargo, si $p^2$ es múltiplo de $a \Longrightarrow \text{mcd}(a,p^2) = a$, por lo que tampoco tiene solución.
    Luego la tercera es falsa, al existir más casos en los que no tiene solución.
\end{ejercicio}

\newpage
\resetearcontador
