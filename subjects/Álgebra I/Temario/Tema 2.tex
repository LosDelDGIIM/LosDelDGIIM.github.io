\chapter{Anillos conmutativos}

\begin{definicion}[Operación Binaria]
    Sea $A$ un conjunto no vacío, una \textbf{operación binaria} en $A$ es una aplicación $$\ast:A\times A \longrightarrow A$$
\end{definicion}

Dada dicha aplicación, para cada $(a,b) \in A \times A$, la imagen de ese par por dicha aplicación suele denotarse como $a * b$.
Es decir:
$$\ast(a,b) =: a \ast b$$
Y se lee como el resultado de operar $a$ con $b$ mediante la operación $*$.

\begin{ejemplo}
    Sea $X$ un conjunto y $A = \cc{P}(X)$, entonces:
    \begin{gather*}
        \cap:\cc{P}(X)\times \cc{P}(X)\longrightarrow \cc{P}(X) \\
        \cup:\cc{P}(X)\times \cc{P}(X)\longrightarrow \cc{P}(X)
    \end{gather*}
    Son operaciones binarias en $A$.
\end{ejemplo}


Usaremos la notación aditiva o multiplicativa para las operaciones binarias:
\begin{itemize}
    \item \underline{Notación aditiva}: $+:A\times A\longrightarrow A$ donde $a + b$ se leerá $a$ más $b$.
    \item \underline{Notación multiplicativa}: $\cdot:A\times A\longrightarrow A$ donde $a \cdot b$ se leerá $a$ por $b$.
        En varias ocasiones, la notación multiplicativa será abreviada simplemente por yuxtaposición, entendiendo que $ab = a\cdot b$.
\end{itemize}

\begin{definicion}[Anillo Conmutativo, \emph{Emmy Noether, 1921}]
    Un \textbf{anillo conmutativo} es un conjunto no vacío $A$ junto con dos operaciones, una notada aditivamente y otra
    multiplicativamente, de la siguiente forma:
    \begin{equation*}
        +:A\times A\longrightarrow A
        \hspace{1cm}
        \cdot:A\times A\longrightarrow A
    \end{equation*}
    
    Tal que se verifica $\forall a,b,c \in A$:
    \begin{enumerate}
        \item Asociativa de la suma: $(a+b)+c = a+(b+c)$.
        \item Conmutativa de la suma: $a+b = b+a$.
        \item Existencia de un neutro de la suma (o cero): $\exists n \in A \mid a + n = a \quad \forall a\in A$.
        \item Existencia de opuesto: $\forall a\in A~\exists -a \in A \mid a + (-a) = n$.
        \item Asociativa del producto: $(ab)c = a(bc)$.
        \item Conmutativa del producto: $ab=ba$.
        \item Existencia de un neutro del producto (o uno): $\exists d \in A \mid a \cdot d = a\quad \forall a\in A$.
        \item Propiedad distributiva: $a(b+c) = ab + ac$.
    \end{enumerate}
\end{definicion}

Diremos que un conjunto no vacío $A$ es simplemente un \textbf{anillo} si verifica todas las propiedades anteriores sin necesidad de
verificar la $6$.

\begin{ejemplo}
    Algunos ejemplos de anillos conmutativos son:
    \begin{enumerate}
        \item $\Z$ es un anillo conmutativo con $+$ y $\cdot$, al igual que $\Q$, $\R$, $\C$.
        
        Notemos que $\N$ no lo es (no cumple $4$).

        \item Sea $A = \{f:[0,1] \longrightarrow \R \mid f \mbox{ es una aplicación}\}$ es anillo conmutativo con la suma y el producto definidos por $(\forall f,g \in A)$:
        \begin{gather*}
            f+g:[0,1] \longrightarrow \R \mid (f+g)(x) = f(x) + g(x)~~\forall x \in [0,1] \\
            f\cdot g:[0,1] \longrightarrow \R \mid (f\cdot g)(x) = f(x) g(x)~~\forall x \in [0,1]
        \end{gather*}
        El cero es la aplicación $0:[0,1] \longrightarrow \R$ definida por $0(x) = 0~~\forall x \in [0,1]$.
        
        El uno es la aplicación $1:[0,1] \longrightarrow \R$ definida por $1(x) = 1~~\forall x \in [0,1]$.
        
        La aplicación opuesta de $f:[0,1] \longrightarrow \R$ es la aplicación $-f:[0,1]\longrightarrow \R$ definida por $(-f)(x) = -f(x)~~ \forall x \in [0,1]$.

    \item Sea $n \geq 2$, entonces $M_n(\R)$, conjunto de matrices cuadradas de orden $n$ con entradas reales es un ejemplo de anillo no conmutativo\footnote{Ya que si $A$ y $B\in M_n(\mathbb{R})$, puede suceder que $A\cdot B\neq B\cdot A$.} con la suma y el producto de matrices.

        \item Otro ejemplo de anillo conmutativo es el conjunto $A = \{0\}$ con las operaciones:
        \Func{+}{A\times A}{A}{(0,0)}{0+0=0}
        \Func{\cdot}{A\times A}{A}{(0,0)}{0\cdot0=0}
        
        A este anillo lo llamaremos \textbf{anillo trivial}. Se trata del único anillo conmutativo que podemos formar con un conjunto unitario.
    \end{enumerate}
\end{ejemplo}

\subsubsection{Propiedades}
Deducidas de la definición de anillo conmutativo. Sea $A$ un anillo conmutativo:
\begin{enumerate}
    \item El cero y el 1 son únicos.

    Supongamos que $0, 0' \in A$ son dos ceros, luego $0 = 0+0' = 0'$.\\    
    Supongamos que $1, 1' \in A$ son dos unos, luego $1 = 1\cdot 1' = 1'$.

    \item $\forall a \in A$, $\exists_1 -a \in A \mid a + (-a) = 0$. (Podremos notar $a+(-a)$ como $a-a$).

    Supongamos que $-a, a' \in A$ son dos opuestos de $a$. Luego:
    $$a' = a' + 0 = a' + (a +(-a)) = (a' + a) + (-a) = 0 + (-a) = -a$$

    \item $\forall a \in A, -(-a) = a \y -0 = 0$.
    \begin{gather*}
        0 + 0 = 0 \Longrightarrow -0 = 0\\
        a + (-a) = 0 \Longrightarrow -(-a) = a
    \end{gather*}

    \item $\forall a \in A~~0 \cdot a = 0$.

    Notemos que $0 \cdot a = (0+0)a = 0\cdot a + 0\cdot a$. Entonces:
    \begin{equation*}
        0\cdot a - 0\cdot a = 0 = 0\cdot a + 0\cdot a - 0\cdot a = 0\cdot a \Longrightarrow 0 = 0\cdot a
    \end{equation*}

    \item $\forall a,b,c \in A$, se cumple que:
    \begin{enumerate}
        \item $(-a)b = -(ab) = a(-b)$. Esto es ya que:
        \begin{equation*}
            0 = 0\cdot b = (a-a)b = ab + (-a)b = 0 \Longrightarrow -(ab) = (-a)b
        \end{equation*}
        
        \item $(-a)(-b) = ab$
        \begin{equation*}
            (-a)(-b) = -(a(-b)) = -(-(ab)) = ab
        \end{equation*}
        \item $(-1)a=-a$
        \item $(-1)(-1)=1$
        \item $(a-b)c = ac - bc$
        \begin{equation*}
            (a-b)c = (a+(-b))c = ac + (-b)c = ac + (-bc) = ac - bc
        \end{equation*}
    \end{enumerate}
\end{enumerate}

Como consecuencia, se verifica el siguiente lema:
\begin{lema}
    Sea $A$ un anillo. Tenemos que:
    \begin{equation*}
        A \text{ es el anillo trivial } \Longleftrightarrow 0=1
    \end{equation*}
\end{lema}
\begin{proof}
    Procedemos mediante doble implicación:
    \begin{description}
        \item[$\Longrightarrow)$] Si $A$ es el anillo trivial, por definición tenemos que $0=1$.
        \item[$\Longleftarrow)$] $\forall a \in A$, $a = a\cdot 1 = a\cdot 0 = 0 \Longrightarrow a = 0 \Longrightarrow A = \{0\}$.
    \end{description}
\end{proof}


\section{Anillos de enteros módulo $n$}
\begin{definicion}
    Sea $n \geq 2$, definimos sobre $\Z$ la siguiente relación binaria, que notaremos como $R_n$:
    
    Dados $a,b \in  \Z$, $aR_nb \Longleftrightarrow \exists q \in \Z \mid a-b = qn$.
\end{definicion}

\begin{lema}
    Se verifica que $R_n$ es una relación de equivalencia.
\end{lema}
\begin{proof} Comprobemos las tres condiciones para que una relación binaria sea de equivalencia:
\begin{itemize}
    \item \underline{Reflexividad}: $\forall a \in \Z~~a-a = 0 = qn \mbox{ con } q = 0 \in \Z \Longrightarrow aR_na$.
    \item \underline{Simetría}: $\forall a,b \in \Z \mid bR_na$ se tiene que $\exists q \in \Z$ tal que $b-a = qn$. Por tanto, $a-b = -(b-a)=-qn$ con $-q \in \Z \Longrightarrow aR_nb$.
    \item \underline{Transitividad}: $\forall a,b,c \in \Z$ tal que $aR_nb \y bR_nc$ se tiene que $\exists q,p \in \Z $ con $ a-b = qn \y b-c = pn$. Entonces, $a-c = (a-b)+(b-c) = qn + pn = (q+p)n$, con $q+p\in \Z \Longrightarrow aR_nc$.
\end{itemize}
\end{proof}

Para cada $n \geq 2$ consideramos el conjunto cociente de $\Z$ por $R_n$, notado a partir de ahora como $\Z_n$:
$$\Z_n = \Z/R_n = \{[a] \mid a \in \Z \}$$
donde para cada $a \in \Z$:
$$[a] = \{x \in \Z \mid xR_na \} = \{x \in \Z \mid x-a = qn,~~q\in \Z \} = \{a+qn \mid q \in \Z\}$$

\begin{prop}
    \label{prop:SumaProductoRn}
    Sea $n\geq 2$ y $a,a',b,b' \in \Z \mid aR_na' \y bR_nb'$. Entonces:
    $$(a+b)R_n(a'+b') \hspace{1cm} (ab)R_n(a'b')$$
\end{prop}
\begin{proof}
    Por la definición de dicha relación de equivalencia:
    \begin{gather*}
        aR_na' \Longrightarrow \exists q\in \Z \mid a-a' = qn \\
        bR_nb' \Longrightarrow \exists p\in \Z \mid b-b' = pn
    \end{gather*}

    Demostramos en primer lugar la suma:
    \begin{equation*}
        (a+b)-(a'+b') = (a-a')+(b-b') = qn + pn = (q+p)n \mathop{\Longrightarrow}^{q+p \in \Z} (a+b)R_n(a'+b')
    \end{equation*}

    Respecto al producto:
    \begin{equation*}
        ab - a'b' = ab +a'b - a'b - a'b' = (a-a')b + (b-b')a' = qnb +pna' = (qb+pa')n \mathop{\Longrightarrow}^{qb+pa' \in \Z} (ab)R_n(a'b')
    \end{equation*}
\end{proof}

\begin{teo}[Anillo de enteros/restos módulo $n$]
    Para cada $n \geq 2$, $\Z_n$ es un anillo conmutativo con operaciones suma y producto definidas por:
    $$[a]+[b] = [a+b] \hspace{1cm} [a][b] = [ab],\qquad \forall a,b \in \Z$$

    Dicho anillo lo denominaremos el \textbf{anillo de enteros módulo $n$}, o también, el anillo de restos módulo $n$.
\end{teo}
\begin{proof}
    Por la Proposición \ref{prop:SumaProductoRn}, la suma y el producto no dependen del representante, por lo que están
    bien definidos.
    \begin{itemize}
        \item Las propiedades conmutativas, asociativas y distributiva son consecuencia inmediata de las propiedades de las operaciones en $\Z$.
    
        \item $[0]$ es el neutro para la suma.
    
        \item $[1]$ es el neutro para el producto.
    
        \item Dado $[a] \in \Z_n$, su opuesto es $-[a] = [-a] \in \Z_n$.
    \end{itemize}    
\end{proof}

\begin{teo}[Algoritmo de la división de Euclides]
    \label{teo:AlgDivEuclides}
    Sean $a,b \in \Z$ con $b \neq 0$. Entonces $\exists_1~q,r \in \Z$ tal que:
    \begin{enumerate}
        \item $a = bq + r$.
        \item $0 \leq r < |b|$.
    \end{enumerate}
    A $q$ y a $r$ se les llama cociente y resto de dividir $a$ entre $b$, respectivamente.
\end{teo}
\begin{proof}
    Supuesta la existencia de $q$ y $r$, nos disponemos primero a mostrar su unicidad. Sean $q,q',r,r' \in \Z \mid a=bq+r,~~a=bq'+r', \quad 0 \leq r,r' < |b|$. Entonces:
    \begin{equation*}
        bq+r = bq'+r' \Longrightarrow b(q-q')=r'-r
    \end{equation*}
    \begin{itemize}
        \item \underline{Si $q=q'$}: Entonces, se tiene que $r=r'$, por lo que queda demostrada la unicidad de $q,r$.

        \item \underline{Si $q\neq q'$}: Tenemos que $|q-q'|>0$. Además,
        \begin{equation*}
            |b|~|q-q'|=|r'-r|\Longrightarrow |r'-r|>|b|
        \end{equation*}
        en \underline{contradicción} con que $0 \leq |r'-r| < |b|$ o, equivalentemente, $0\leq r,r'<|b|$.
    \end{itemize}

    Por tanto, queda demostrado que $q=q',~r=r'$.\\
    
    Demostramos ahora la existencia, $\exists q,r \in \Z \mid a=bq+r \y 0\leq r < |b|$. Sean $a,b \in \Z \mid a \geq 0 \y b \geq 1$. Realizamos la siguiente distinción de casos:
    \begin{itemize}
        \item Si $a<b$:
        
            Entonces, considerando $q=0$ y $r = a$, se tiene que:
            \begin{equation*}
                a=0b+a \qquad 0\leq a < |b|=b
            \end{equation*}

        \item Si $a \geq b$:
        
        Sea $X = \{a-bq \mid q \in \N \}\cap \N$. Tenemos que $X\neq \emptyset$ por ser $a-b\in X$. Como $\emptyset \neq X \subseteq \N$, con $\N$ bien ordenado\footnote{Este teorema es materia de la asignatura de Cálculo I.}, $X$ tiene mínimo. Sea $r=\min X$. Como $r \in X \Longrightarrow r\geq 0 \y \exists q \in \N \mid r = a-bq \Longrightarrow a=bq+r$. Falta que $r<b$: \\
    
        Por reducción al absurdo, supongamos que $r \geq b$ y consideramos $r'=~r-~b\geq~0$.
        \begin{equation*}
            r'=r-b=a-bq-b = a-b(q+1) \mathop{\Longrightarrow}^{r'\geq 0} r' \in X \y r' < r
        \end{equation*}
        \underline{Contradicción} con que $r$ era el mínimo de $X$. Por tanto, se tiene que~$r<b$.

        \begin{ejemplo} Consideramos $a=3254,~b=17$. Tenemos que la división es:
            $$\opidiv{3254}{17}$$
            Por tanto, \opidiv[style=text]{3254}{17}. Es decir, $q=191,~r=7$.
        \end{ejemplo}

        \item Dividir $-a$ entre $b$:
        
        Dividimos $a$ entre $b$ y obtenemos $q, r \in \Z \mid a=bq+r \y 0\leq r < |b|$.
        \begin{itemize}
            \item Si $r=0 \Longrightarrow a=bq \Longrightarrow -a = -bq = b(-q)$.
            \item Si $r\neq 0$, como $a=bq+r$, se tiene que: $$ -a = -bq-r = -bq -b + b - r = b(-q-1)+b-r$$            
            Por lo que el cociente es $-q-1$ y el resto es $b-r$, siendo $0 \leq b-r < |b|$.
        \end{itemize}

        \begin{ejemplo}
            Sabemos que \opidiv[style=text]{3254}{17}. Por tanto, $-3254\div 17$ tiene por cociente $-192$ y resto $17-7=10$. Por tanto, \opidiv[style=text]{-3254}{17}.
        \end{ejemplo}

        \item Dividir $a$ entre $-b$:
        
        Dividimos $a$ entre $b$ y obtenemos $q, r \in \Z \mid a=bq+r \y 0\leq r < |b|$.
        Tenemos que $a = (-b)(-q)+r$, por lo que el cociente es $-q$ y el resto es $r$.

        \begin{ejemplo}
            Sabemos que \opidiv[style=text]{3254}{17}. Por tanto, $3254\div -17$ tiene por cociente $-191$ y resto $7$. Por tanto, \opidiv[style=text]{3254}{-17}.
        \end{ejemplo}

        \item Dividir $-a$ entre $-b$:
        
        Dividimos $a$ entre $b$ y obtenemos $q, r \in \Z \mid a=bq+r \y 0\leq r < |b|$.
        \begin{itemize}
            \item Si $r=0 \Longrightarrow a=bq \Longrightarrow -a = -bq = (-b)q$.
            \item Si $r\neq 0$, como $a=bq+r$, se tiene que: $$ -a = -bq-r = -bq -b + b - r = (-b)(q+1)+b-r$$            
            Por lo que el cociente es $q+1$ y el resto es $b-r$, siendo $0 \leq b-r < |b|$.
        \end{itemize}

        \begin{ejemplo}
            Sabemos que \opidiv[style=text]{3254}{17}. Por tanto, $(-3254)\div (-17)$ tiene por cociente $192$ y resto $17-7=10$. Por tanto, \opidiv[style=text]{-3254}{-17}.
        \end{ejemplo} 
    \end{itemize}
\end{proof}

\begin{teo}[Estructura del anillo de enteros módulo $n$]
    Sea $n \geq 2$:
    $$\Z_n = \Z/R_n = \{[0], [1], \ldots, [n-1]\}$$
\end{teo}
\begin{proof}
    Sabemos que $\{[0], [1], \ldots, [n-1]\} \subseteq \Z_n$.
    
    Sea $a\in \Z$. Consideramos $[a]\in \Z_n$ y dividimos $a$ entre $n$. Por el teorema anterior, $\exists~q,r \in \Z \mid a=nq+r \y 0 \leq r < n$ cumpliendo que:
    $$a-r = nq \Longrightarrow aR_nr \Longrightarrow [a]=[r]$$
    Por tanto, como $0\leq r<n  \Longleftrightarrow 0 \leq r \leq n-1$, tenemos que $$[a] = [r] \in \{[0], [1], \ldots, [n-1]\} \Longrightarrow \Z_n \subseteq \{[0], [1], \ldots, [n-1]\} $$
    Por tanto, se tiene la igualdad por la doble inclusión.
\end{proof}

\begin{notacion}
    Si para cada $a \in \Z$ notamos por $R(a;n)$ al resto $r$ de dividir $a$ entre $n$, entonces:
    $$[a] = [R(a;n)]$$
    
    Por ejemplo, en $\Z_3$, tenemos que $[11] = [2]$. Notemos que:
    \begin{gather*}
        [a]+[b] = [R(a+b;n)] \\
        [a]\cdot [b] = [R(ab;n)] \\
        -[a] = [-a] = [R(-a;n)] \mathop{=}^{(\ast)} [n-a]
    \end{gather*}
    donde la igualdad $(\ast)$ solo es cierta si $0\leq a \leq n-1$.
\end{notacion}
\begin{notacion}
    A partir de ahora, se omitirán los corchetes (por comodidad) a la hora de representar las clases de equivalencia, por lo que tendremos que $\Z_n$ es un anillo conmutativo de la forma (con $n \geq 2$):
    $$\Z_n = \{0, 1, \ldots, n-1\}$$

    Teniendo en cuenta que, para $a,b\in \Z_n$:
    \begin{gather*}
        a+b = R(a+b;n) \\
        ab = R(ab;n) \\
        -a = R(-a,n)
    \end{gather*}
\end{notacion}

\begin{ejemplo}
    En $\Z_8 = \{0, 1, \ldots, 7\}$:
    \begin{equation*}
        3 \cdot 7 = 5
        \qquad
        3 \cdot 3 = 1
        \qquad 
        -5 = 3
        \qquad 
        2 \cdot 4 = 0
    \end{equation*}
\end{ejemplo}

\section{Anillos de enteros cuadráticos y anillos de racionales cuadráticos}
\begin{definicion}[Subanillo]
    Sea $A$ un anillo conmutativo. Un subconjunto $B \subseteq A$, $B \neq \emptyset$ diremos que es un \textbf{subanillo} de $A$ si verifica:
    \begin{enumerate}
        \item $\forall a,b \in B~~a+b, ab \in B$. Cerrado para suma y producto.
        \item $0,1 \in B$. Contiene al $1$ y al $0$.
        \begin{samepage}
            \item $\forall a \in B.~-a \in B$. Cerrado para opuestos.
        \end{samepage}
    \end{enumerate}
\end{definicion}

Notemos que si $B$ es un subanillo de $A$, tenemos que $(B, +, \cdot)$ es un anillo conmutativo.

\begin{notacion}
     Sea $a \in \R^{+}$, $\exists b,c \in \R \mid b^2 = c^2 = a$ con $b \in \R^{+}$ y $c \in \R^{-}$.
     
     Notaremos: $\sqrt{a} = b \y -\sqrt{a}=c$. 
\end{notacion}
\begin{notacion}
    No existe ningún real cuyo cuadrado sea $-a$, con $a \in \R^{+}$. Sabemos de la existencia de dos números complejos. Uno es $i\sqrt{a}$ y otro es su opuesto, $-i\sqrt{a}$.
    
    A $i\sqrt{a}$ lo notaremos como $\sqrt{-a}$.
\end{notacion}

\begin{prop}[Anillo de enteros cuadráticos]
    \label{prop:enteros_cuadraticos}
    Sea $n \in \Z \mid \sqrt{n} \notin \Z$. Consideramos el siguiente subconjunto de $\C$:
    $$\Z[\sqrt{n}] := \{a+b\sqrt{n} \mid a,b \in \Z \} \subseteq \C$$
    Notemos que si $n >0 \Longrightarrow \Z[\sqrt{n}] \subseteq \R$.\\
    
    Se verifica que $\Z[\sqrt{n}]$ es subanillo de $\C$, que llamaremos \textbf{el anillo de enteros cuadráticos} definido por $n$.
\end{prop}
\begin{proof}
    Sean $\alpha, \beta \in \Z[\sqrt{n}] \Longrightarrow \exists a,b,a',b' \in \Z \mid \alpha = a+b\sqrt{n} \y \beta = a'+b'\sqrt{n}$.

    Veamos en primer lugar que es cerrado para la suma:
    \begin{gather*}
        \alpha + \beta = (a + b\sqrt{n})+(a'+b'\sqrt{n}) = (a+a') + (b+b')\sqrt{n} \\
        a+a' \in \Z \y b+b' \in \Z \Longrightarrow \alpha + \beta \in \Z[\sqrt{n}]
    \end{gather*}

    Veamos ahora si es cerrado para el producto:
    \begin{gather*}
        \alpha\beta = (a+b\sqrt{n})(a'+b'\sqrt{n}) = aa' + ab'\sqrt{n} + ba'\sqrt{n} + bb'n = (aa'+bb'n) + (ab'+ba') \sqrt{n} \\
        aa'+bb'n \in \Z \y ab'+ba' \in \Z \Longrightarrow \alpha\beta \in \Z[\sqrt{n}]
    \end{gather*}
    
    Tenemos que $\Z[\sqrt{n}]$ es cerrado para operaciones. Veamos si contiene al $1$ y al $0$.  
    $$0 = 0+0 \sqrt{n} \hspace{1.5cm} 1 = 1+0\sqrt{n}$$
    Como $0,1\in \Z$, tenemos que $0,1\in \Z[\sqrt{n}]$. Comprobemos ahora que es cerrado para opuestos:
    $$\alpha = a+b\sqrt{n} \Longrightarrow -\alpha = -a-b\sqrt{n} \hspace{1cm} -a,-b \in \Z \Longrightarrow -\alpha \in \Z[\sqrt{n}]$$
    Luego $\Z[\sqrt{n}]$ es cerrado para opuestos. Por tanto, queda demostrado que $\Z[\sqrt{n}]$ es un subanillo de $\C$
\end{proof}

Notemos además que $\Z$ es subanillo de $\Z[\sqrt{n}]$.

\begin{ejemplo} Algunos ejemplos de anillos de enteros cuadráticos son:
\begin{enumerate}
    \item $\Z[\sqrt{2}] = \{a+b\sqrt{2} \mid a,b \in \Z\}$.
    \item $\Z[i] = \{a+bi \mid a,b \in \Z\}$.

    Este segundo anillo se denomina el anillo de enteros de Gauss. Notemos que no coincide con $\bb{C}$, ya que en los complejos los coeficientes $a,b$ pueden ser reales, mientras que en este caso nos limitamos a enteros.
\end{enumerate}
\end{ejemplo}

\begin{prop}[Anillo de racionales cuadráticos]
    Sea $n \in \Z \mid \sqrt{n} \notin \Z$. Consideramos el siguiente subconjunto de $\C$:
    $$\Q[\sqrt{n}] := \{a+b \sqrt{n} \mid a,b \in \Q\}$$
    Se verifica que $\Q[\sqrt{n}]$ es un subanillo de $\C$, que llamaremos \textbf{el anillo de racionales cuadráticos} definido
    por $n$.
\end{prop}
\begin{proof}
    Análoga a la de la Proposición \ref{prop:enteros_cuadraticos}, basándose en que $\Q$ es cerrado para operaciones y opuestos y que contiene al 1 y al 0.
\end{proof}

Notemos que $\Z[\sqrt{n}]$ es un subanillo de $\Q[\sqrt{n}]$.

\begin{definicion}[Unidad]
    Sea $A$ un anillo conmutativo. Un elemento $u \in A$ diremos que $u$ es una \textbf{unidad} (o que es invertible)
    si existe $v \in A \mid uv = 1$.

    En tal caso, dicho $v$ es único, puesto que si $v' \in A \mid uv' = 1$, entonces:
    $$v' = v' \cdot 1 = (v'u)v = 1 \cdot v = v$$

    A este elemento único $v$ lo llamaremos \textbf{inverso de $u$} y lo notaremos por $u^{-1}$.
\end{definicion}

En cualquier anillo, el 1 y el $-1$ son unidades, con $1^{-1} = 1 \y (-1)^{-1} = -1$.\\


No todos los elementos de un anillo conmutativo son unidades. Si el anillo es no trivial; el 0, por ejemplo, no es una unidad\footnote{Ya que $a\cdot 0=0\neq 1$ $\forall a\in A$.}.\\


\begin{notacion}
Al conjunto de unidades de un anillo $A$ conmutativo lo notaremos $\cc{U}(A)$:
$$\cc{U}(A) = \{u \in A \mid u \mbox{ es una unidad }\}$$
\end{notacion}

\begin{ejemplo}
Algunos ejemplos de unidades en anillos ya conocidos son:
\begin{align*}
    \cc{U}(\Z) &= \{-1, 1\} \\
    \cc{U}(\Z_2) &= \{1\} \\
    \cc{U}(\Z_3) &= \{1,2\} \\
    \cc{U}(\Z_4) &= \{1, 3\} \\
    \cc{U}(\Z_5) &= \{1, 2, 3, 4\}
\end{align*}
\end{ejemplo}

\begin{definicion}[Cuerpo]
    Un anillo conmutativo $K$ diremos que es un \textbf{cuerpo} si $K$ es no trivial y todos los elementos no nulos
    de $K$ son unidades, es decir:
    $$\cc{U}(K) = K \setminus \{0\}$$
\end{definicion}

\begin{ejemplo}
    Ejemplos de cuerpos son: $\Q$, $\R$, $\C$, $\Z_2$, $\Z_3$, $\Z_5$. Veamos el caso de los números complejos, $\C$:
    $$\forall \alpha = a+bi \in \C \mid \alpha \neq 0 \hspace{1cm} \dfrac{1}{\alpha} = \dfrac{a-bi}{a^2+b^2} = \dfrac{a}{a^2+b^2} -
        \dfrac{b}{a^2+b^2}i$$
\end{ejemplo}

\begin{definicion}[Conjugado]
    Sea $\alpha= a+b \sqrt{n} \in \Q[\sqrt{n}]$ $(n \geq 2~~\sqrt{n}\notin \Z)$. Definimos el \textbf{conjugado de $\alpha$}, denotado por $\overline{\alpha}$, como el elemento:
    $$\overline{\alpha} = a - b\sqrt{n}$$
\end{definicion}

En el caso de $\Z[\sqrt{n}]$, al ser este subanillo de los racionales cuadráticos, se define el conjugado de forma análoga.\\

Algunas propiedades del conjugado de los racionales cuadráticos son, tomando $\alpha = a+b\sqrt{n},~~\beta = c+d\sqrt{n}~~ \in \Q[\sqrt{n}]$:
\begin{enumerate}
    \item $\overline{\alpha + \beta} = \overline{\alpha} + \overline{\beta} \hspace{1cm} \forall \alpha, \beta \in \Q[\sqrt{n}]$.
    \begin{gather*}
        \overline{\alpha + \beta} = \overline{(a+c) + (b+d)\sqrt{n}} = (a+c) - (b+d)\sqrt{n} \\
        \overline{\alpha} + \overline{\beta} = a-b\sqrt{n} + c-d\sqrt{n} = (a+c) - (b+d)\sqrt{n}
    \end{gather*}

    \item $\overline{\alpha\beta} = \overline{\alpha} \cdot \overline{\beta} \hspace{1cm} \forall \alpha, \beta \in \Q[\sqrt{n}]$.
    \begin{gather*}
        \overline{\alpha\beta} = \overline{(ac + bdn) + (bc + da)\sqrt{n}} = (ac + bdn) - (bc + da)\sqrt{n} \\
        \overline{\alpha} \cdot \overline{\beta} = (a-b\sqrt{n})(c-d\sqrt{n}) = (ac + bdn) - (bc + da)\sqrt{n}
    \end{gather*}

    \item $\overline{\overline{\alpha}} = \alpha \hspace{1cm} \forall \alpha \in \Q[\sqrt{n}]$.
    $$\overline{\overline{\alpha}} = \overline{\overline{a+b\sqrt{n}}} = \overline{a-b\sqrt{n}} = a+b\sqrt{n} = \alpha$$
\end{enumerate}

\begin{definicion}[Norma]
    Dado $\alpha=a+b\sqrt{n} ~~\in \Q[\sqrt{n}]$, definimos la \textbf{norma de $\alpha$}, notada $N(\alpha)$, por:
    $$N(\alpha) = \alpha \cdot \overline{\alpha} = (a+b\sqrt{n})(a-b\sqrt{n}) = a^2 - nb^2 \in \Q$$
\end{definicion}

Respecto a la norma, se verifica que, dados $\alpha, \beta \in \Q[\sqrt{n}]$, con $\alpha=a+b\sqrt{n}$, se tiene que:
\begin{enumerate}
    \item $N(\alpha \cdot \beta) = N(\alpha) \cdot N(\beta) \hspace{1cm} \forall \alpha, \beta \in \Q[\sqrt{n}]$
    \begin{equation*}
        N(\alpha \cdot \beta)
        = (\alpha \cdot \beta) \cdot \overline{\alpha \cdot \beta}
        = (\alpha \cdot \beta) \cdot (\overline{\alpha} \cdot \overline{\beta})
        = (\alpha \cdot \overline{\alpha}) \cdot (\beta \cdot \overline{\beta})
        = N(\alpha) \cdot N(\beta)
    \end{equation*}

    \item $N(\alpha) = 0 \Longleftrightarrow \alpha = 0$
    \begin{description}
        \item[$\Longrightarrow)$] $\alpha = 0 \Longleftrightarrow \alpha = 0 + 0\sqrt{n} \Longrightarrow N(\alpha) = 0$

        \item[$\Longleftarrow)$] Tenemos que $N(\alpha) = 0 \Longleftrightarrow \alpha \cdot \overline{\alpha} = 0$
        Por tanto,
        \begin{itemize}
            \item Si $\overline{\alpha} = 0 \Longrightarrow \overline{\alpha} = a-b\sqrt{n} = 0 \Longleftrightarrow a = 0 \y b = 0$. Por tanto, $\alpha=a+b\sqrt{n} = 0$

            \item Si $\alpha = 0$, se tiene lo que queríamos demostrar.
        \end{itemize}
    \end{description}
\end{enumerate}

\begin{prop}
    $\Q[\sqrt{n}]$ es un cuerpo $(\mbox{con } n \in \Z \mid \sqrt{n} \notin \Z)$.
\end{prop}
\begin{proof}
    Hemos de demostrar que todo elemento no nulo de $\Q[\sqrt{n}]$ es una unidad. Consideramos $\alpha\in \Q[\sqrt{n}]$, $\alpha\neq 0$. veamos si $\alpha$ es una unidad.

    En primer lugar, por lo visto anteriormente, como $\alpha\neq 0$ tenemos que $N(\alpha)\neq 0$. Consideramos ahora el siguiente elemento, $\beta = \dfrac{\overline{\alpha}}{N(\alpha)} \in \Q[\sqrt{n}]$. Veamos que:
    $$\alpha \cdot \beta = \alpha \cdot \dfrac{\overline{\alpha}}{N(\alpha)} = \frac{N(\alpha)}{N(\alpha)}= 1$$
    
    Por lo que tenemos que $\alpha=a+b\sqrt{n} \in \cc{U}(\Q[\sqrt{n}])$, siendo: $$\alpha^{-1} = \beta = \frac{\overline{\alpha}}{N(\alpha)} = \frac{a}{N(\alpha)} - \frac{b}{N(\alpha)}\sqrt{n}$$
\end{proof}

\begin{ejemplo}
    En $\Q[\sqrt{2}]$, consideramos $\alpha = 3-\sqrt{2}$. Veamos que es una unidad.
    
    Tenemos que $N(\alpha) = 3^2 - 2 = 7$, por lo que:
    $$\alpha^{-1} = \dfrac{\overline{\alpha}}{N(\alpha)} = \dfrac{3}{7} + \dfrac{\sqrt{2}}{7}$$
    Notemos que $\alpha \in \Z[\sqrt{n}] \subsetneq \Q[\sqrt{n}]$ mientras que $\alpha^{-1} \notin \Z[\sqrt{n}]$, $\alpha^{-1} \in \Q[\sqrt{n}]$
\end{ejemplo}

\begin{prop}
    Sea $\alpha = a+b\sqrt{n} ~~\in \Z[\sqrt{n}]$
    $$\alpha \in \cc{U}(\Z[\sqrt{n}]) \Longleftrightarrow N(\alpha) = \pm 1$$
\end{prop}
\begin{proof} Demostramos mediante doble implicación:
\begin{description}
    \item[$\Longrightarrow)$] Sea $\alpha \in \cc{U}(\Z[\sqrt{n}]) \Longrightarrow \exists \alpha^{-1} \in \Z[\sqrt{n}] \mid \alpha \cdot \alpha^{-1} = 1$. Tenemos que:
    $$N(\alpha) \cdot N(\alpha^{-1}) = N(\alpha \cdot \alpha^{-1}) = N(1) = 1$$

    Tenemos por tanto que $N(\alpha), N(\alpha^{-1}) \in \Z \mid N(\alpha)\cdot N(\alpha^{-1}) = 1$. Por tanto,
    $$N(\alpha) \in \cc{U}(\Z) = \{-1, 1\} \Longrightarrow N(\alpha) = \pm 1$$

    \item[$\Longleftarrow)$] Sea $\alpha \in \Z[\sqrt{n}] \mid N(\alpha) = \pm 1$:
    \begin{itemize}
        \item $N(\alpha) = 1 \Longrightarrow \alpha \cdot \overline{\alpha} = 1 \Longrightarrow \alpha \in \cc{U}(\Z[\sqrt{n}])$ con $\alpha^{-1} = \overline{\alpha}$.

        \item $N(\alpha) = -1 \Longrightarrow -N(\alpha) = -\alpha \cdot \overline{\alpha} = 1 \Longrightarrow \alpha \in \cc{U}(\Z[\sqrt{n}])$ con $\alpha^{-1} =-\overline{\alpha}$.
    \end{itemize}
\end{description}
\end{proof}

\begin{ejemplo}
    En el anillo de los enteros de Gauss, $\Z[i] = \{a+bi \mid a,b \in \Z\}$, calcular las unidades. Tenemos que $N(a+bi) = a^2 + b^2$, por lo que, usando la proposición anterior:
    \begin{equation*}
        \alpha \in \cc{U}(\Z[i]) \Longleftrightarrow a^2 + b^2 = 1 \Longleftrightarrow \left\{ \begin{array}{ccc}
            a^2 = 1 &  & a^2 = 0 \\
            \land    &  \lor  &  \land       \\
            b^2 = 0 &  & b^2 = 1
        \end{array} \right\}
        \Longleftrightarrow \left\{ \begin{array}{ccc}
            a = \pm 1 & & a = 0     \\
            \land    &  \lor  &  \land \\
            b = 0     & & b = \pm 1
        \end{array} \right\}
    \end{equation*}
    Por tanto, tenemos que $U[\Z[i]] = \{-1, 1, -i, i\}$.
\end{ejemplo}

\begin{ejemplo}
    Sea $n \geq 2$ y $\Z[\sqrt{-n}] = \{a+b\sqrt{-n} \mid a,b \in \Z\}$. Calcular las unidades de dicho anillo.

    Sea $\alpha = a+b\sqrt{-n}~~\in \Z[\sqrt{-n}]$. Tenemos que $N(\alpha)=a^2+nb^2\geq 0$. Por tanto,
    $$\alpha \in \cc{U}(\Z[\sqrt{-n}]) \Longleftrightarrow N(\alpha) = a^2 + nb^2 = 1 \Longleftrightarrow a=\pm 1 \y b = 0$$
    Luego $\cc{U}(\Z[\sqrt{-n}]) = \{-1, 1\}$.
\end{ejemplo}

\begin{ejemplo}
    Consideramos $\Z[\sqrt{2}] = \{a+b\sqrt{2} \mid a,b \in \Z\}$. Calcular las unidades de dicho anillo.

    Sea $\alpha=a+b\sqrt{2} \in \Z[\sqrt{2}]$. Tenemos que $N(a+b\sqrt{2}) = a^2-2b^2$. Por tanto,
    $$\alpha~\in \cc{U}(\Z[\sqrt{2}]) \Longleftrightarrow a^2-2b^2 = \pm 1$$
    Por lo que el conjunto de las unidades de $\Z[\sqrt{2}]$ es infinito:
    $$\{1+\sqrt{2}, -1, 1, 3+2\sqrt{2}\}\subsetneq \cc{U}(\Z[\sqrt{2}])$$
    
    Se verifica que\footnote{La resolución de dicha ecuación no entra en el contenido de este curso. Se conoce como Ecuación de Pell.}:
    $$\cc{U}(\Z[\sqrt{2}]) = \{\pm 1, \pm (1+\sqrt{2})^k, \pm (1-\sqrt{2})^k\mid k\geq 1\}$$
\end{ejemplo}

\section{Sumas y productos generalizados}
Lo que vamos a proceder a ver en este tema es de gran importancia. A pesar de que para el alumno parezca trivial, permite trabajar simultáneamente con todo tipo de anillos conmutativos junto con los enteros. Hemos de pensar que un anillo conmutativo $A$ no se limita a los conjuntos a los que el alumno está acostumbrado, como pueden ser los enteros.

\begin{definicion}
    Sea $A$ un anillo conmutativo y sea $n\in \N \mid n\geq 1$. Sea $(a_1, a_2, \ldots, a_n)\in~A^n$.
    Podemos definir la suma y el producto de $n$ elementos de forma inductiva:
    $$\sum_{i=1}^n a_i = \left\{ \begin{array}{cl}
            a_1                               & \mbox{ si } n=1   \\
            \sum\limits_{i=1}^{n-1} a_i + a_n & \mbox{ si } n > 1
        \end{array} \right.$$
    $$\prod_{i=1}^n a_i = \left\{ \begin{array}{cl}
            a_1                                                 & \mbox{ si } n=1   \\
            \left(\prod\limits_{i=1}^{n-1} a_i\right) \cdot a_n & \mbox{ si } n > 1
        \end{array} \right.$$
\end{definicion}

\begin{prop}[Propiedad asociativa generalizada]
    Sea $A$ un anillo conmutativo, y sean $m,n\in\N,\;\;m,n\geq 1$. Sea $(a_1, a_2, \ldots, a_m, a_{m+1}, \ldots, a_{m+n}) \in A^{m+n}$. Se verifica que:
    $$\sum_{i=1}^{m+n} a_i = \sum_{i=1}^m a_i + \sum_{i=m+1}^{m+n} a_i$$
    $$\prod_{i=1}^{m+n} a_i = \left(\prod_{i=1}^m a_i\right) \cdot \left(\prod_{i=m+1}^{m+n} a_i\right)$$
\end{prop}
\begin{proof}
    Demostramos para la suma. Para ello, fijamos el valor de $m$ y realizamos inducción sobre $n$:
    \begin{itemize}
        \item \underline{Para $n=1$}: aplicamos la definición:
        $$\sum_{i=1}^m a_i + \sum_{i=m+1}^{m+1} a_i = \sum_{i=1}^m a_i + a_{m+1} = \sum_{i=1}^{m+1} a_i$$

        \item \underline{Supuesto cierto para $n$, lo probamos para $n+1$}:
        \begin{multline*}
            \sum_{i=1}^m a_i + \sum_{i=m+1}^{m+n+1} a_i = \sum_{i=1}^m a_i + \left( \sum_{i=1}^{m+n} + a_{m+n+1} \right) = \left( \sum_{i=1}^m a_i + \sum_{i=1}^{m+n}\right) + a_{m+n+1} = \\
            =\sum_{i=1}^{m+n} a_i + a_{m+n+1} = \sum_{i=1}^{m+n+1} a_i
        \end{multline*}
    \end{itemize}
    Para el producto se realiza de una forma análoga.
\end{proof}

\begin{notacion}
    A veces, escribiremos:
    \begin{gather*}
        \sum_{i=1}^n a_i =: a_1 + a_2 + \ldots + a_n\\
        \prod_{i=1}^n a_i =: a_1 \cdot a_2 \cdot \ldots \cdot a_n
    \end{gather*}
\end{notacion}

\begin{prop}[Propiedad distributiva generalizada]
    Sea $A$ un anillo conmutativo, y sean $m,n \geq 1$ y consideramos
    $(a_1, a_2, \ldots, a_m) \in A^m,~~(b_1, b_2, \ldots, b_n) \in A^n$. Se verifica que:
    $$\left( \sum_{i=1}^n a_i \right) \left( \sum_{j=1}^m b_j \right) = \sum_{i=1}^n \sum_{j=1}^m a_i b_j$$
\end{prop}
\begin{proof}
    Demostramos mediante inducción en $m$:
    \begin{itemize}
        \item \underline{Para $m=1$}: Hacemos inducción en $n$:
        \begin{itemize}
            \item Para $n=1$, es obvio que $a_1b_1 = a_1b_1$.
            \item Supuesto cierto para un $n-1$, lo comprobamos para $n$:
            $$a_1 \sum_{j=1}^n b_j = a_1 \left( \sum_{j=1}^{n-1} b_j + b_n \right) = \left( a_1 \sum_{j=1}^{n-1} b_j \right) + a_1b_n = \sum_{j=1}^n a_1b_j$$
        \end{itemize}
        \item \underline{Supuesto cierto para un $m-1$, lo comprobamos para $m$}:
    \begin{multline*}
        \left(\sum_{i=1}^m a_i\right) \left(\sum_{j=1}^n b_j\right)
        = \left(\sum_{i=1}^{m-1}a_i + a_m\right) \left(\sum_{j=1}^n b_j\right)
        = \left(\sum_{i=1}^{m-1} a_i\right)\left(\sum_{j=1}^n b_j\right) + a_m \left(\sum_{j=1}^n b_j\right) =\\
        = \left(\sum_{i=1}^{m-1} \sum_{j=1}^n a_i b_j \right) + \sum_{j=1}^n a_mb_j = \sum_{i=1}^m \sum_{j=1}^n a_i b_j
    \end{multline*}
    \end{itemize}
\end{proof}

En el caso de tener una lista $(n \in \N)$ $(a_1, a_2, \ldots, a_n) \in A^n$ en la que todos sus elementos son iguales:
$$a_1 = a_2 = \ldots = a_n = a$$
Tenemos que:
\begin{gather*}
    \sum_{i=1}^n a_i = a_1 + a_2 + \ldots + a_n = n\cdot a\\
    \prod_{i=1}^n a_i = a_1 \cdot a_2 \cdot \ldots \cdot a_n = a^n
\end{gather*}
Con el convenio de que si $n = 0$, entonces: $\sum\limits_{i=1}^0 a_i = 0 \cdot a = 0$ y $\prod\limits_{i=1}^0 a_i = a^0 = 1$.

\begin{prop}\label{prop:2.13}
    Sea $A$ un anillo conmutativo y $n, m \in \N$, $a,b \in A$. Se verifica:
    \begin{enumerate}
        \item $(m+n)a = ma + na$
        \item $n(a+b) = na+nb$
        \item $m(na) = (mn)a$
        \item $(ma)(nb) = (mn)(ab)$
        \item $a^na^m = a^{n+m}$
        \item $(ab)^n = a^nb^n$
        \item $(a^m)^n = a^{mn}$
        \item $(a+b)^n = \sum\limits_{i=0}^n \binom{n}{i} a^i b^{n-i}$
        \item $(a-b) (a+b) = a^2-b^2$
    \end{enumerate}
\end{prop}
\begin{proof} \
    \begin{enumerate}
        \item $(m+n)a = ma + na$
        $$(m+n)a = \sum_{i=1}^{m+n} a = \sum_{i=1}^m a + \sum_{i=m+1}^{m+n} a = ma + na$$

        \item $n(a+b) = na+nb$
        
        Realizamos inducción sobre $n$:
        \begin{itemize}
            \item Para $n=0,1$: $0=0$ y $a+b = a+b$, Cierto.
            \item Cierto para $n$, lo probamos para $n+1$:
            $$(n+1)(a+b)=n(a+b)+a+b = na + nb +a+b=na+a+nb+b = (n+1)a+(n+1)b$$
        \end{itemize}

        \item $m(na) = (mn)a$
        Realizamos inducción sobre $m$:
        \begin{itemize}
            \item Para $m=0,1$: $0=0$ y $na = na$, Cierto.
            \item  Cierto para $m$, lo probamos para $m+1$:
            $$(m+1)(na) = m(na) + na = (mn)a + na = (mn+n)a = ((m+1)n)a$$
        \end{itemize}

        \item $(ma)(nb) = (mn)(ab)$
        $$(ma)(nb) = \left(\sum_{i=1}^m a\right) \left(\sum_{j=1}^n b_j\right) = \sum_{i=1}^m \sum_{j=1}^n a_ib_j = (mn)(ab)$$

        \item $a^na^m = a^{n+m}$
        $$a^na^m =  \prod_{i=1}^n a + \prod_{i=m+1}^{n+m} a = \prod_{i=1}^{n+m} a =a^{n+m}$$

        \item $(ab)^n = a^nb^n$
        
        Realizamos inducción sobre $n$:
        \begin{itemize}
            \item Para $n=0,1$: $1=1$ y $ab = ab$, Cierto.
            \item Cierto para $n$, lo probamos para $n+1$:
            $$(ab)^{n+1} = (ab)^n ab = a^nab^nb = a^{n+1}b^{n+1}$$
        \end{itemize}

        \item $(a^m)^n = a^{mn}$
        
        Realizamos inducción sobre $n$:
        \begin{itemize}
            \item Para $n=0,1$: $1=1$ y $a^m = a^m$, Cierto.
            \item Cierto para $n$, lo probamos para $n+1$:
            $$(a^m)^{n+1} = (a^m)^n a^m = a^{mn}a^m = a^{mn+m} = a^{m(n+1)}$$
        \end{itemize}

        \item $(a+b)^n = \sum\limits_{i=0}^n \binom{n}{i} a^i b^{n-i}$
        
        Recordamos la definición de número combinatorio: $$\binom{n}{i} = \dfrac{n!}{i!(n-i)!} = \dfrac{n(n-1)\ldots (n-i+1)}{i(i-1)\ldots 2 \cdot 1}$$
        
        Y tenemos en cuenta que:
        $$\binom{n}{j} + \binom{n}{j-1} = \dfrac{n!}{j!(n-j)!} + \dfrac{n!}{(j-1)!(n-j+1)!} =$$ $$=\dfrac{n!(n-j)!(j-1)!(n-j+1+j)} {j!(n-j)!(j-1)!(n-j+1)!} = \dfrac{n!(n+1)}{j!(n-j+1)!} = \binom{n+1}{j}$$

    
        Para hacer la demostración, realizamos inducción sobre $n$:
        \begin{itemize}
            \item Para $n=1$:
            $$\binom{1}{0}a^0b^1 + \binom{1}{1}a^1b^0 = b+a = a+b = (a+b)^1$$

            \item Supuesto cierto para un $n$, lo probamos para $n+1$:
            \begin{equation*}
                \begin{split}
                    (a+b)^{n+1} &= (a+b)(a+b)^n = (a+b)\sum_{i=0}^n\binom{n}{i}a^i b^{n-i} = \\
                    & = \sum_{i=0}^n \binom{n}{i} a^{i+1}b^{n-i} + \sum_{i=0}^n \binom{n}{i} a^i b^{n+1-i} = \\
                    & =\sum_{i=1}^{n+1} \binom{n}{i-1} a^i b^{n+1-i} + \sum_{i=0}^n \binom{n}{i} a^i b^{n+1-i} = \\
                    & = \sum_{i=1}^{n+1} \binom{n+1}{i} a^i b^{n+1-i} + b^{n+1} = \sum_{i=0}^{n+1}\binom{n+1}{i} a^i b^{n+1-i}
                \end{split}
            \end{equation*}
        \end{itemize}
        
        \item $(a-b) (a+b) = a^2-b^2$
        $$(a-b)(a+b) = a^2 -b^2 +ab - ab = a^2 - b^2$$
    \end{enumerate}   
\end{proof}

\begin{prop}[Cardinalidad del conjunto partes de un conjunto]
    Dado un conjunto $X \mid |X|=n\in \N$. Se verifica que:
    $$|\cc{P}(X)| = 2^n$$
\end{prop}
\begin{proof}
    El conjunto $\cc{P}(X)$ consiste en el conjunto $\emptyset$, los $n$ conjuntos de un sólo elemento de $X$, los $\binom{n}{2}$ subconjuntos de $X$ de dos elementos, \ldots, los $\binom{n}{i}$ subconjuntos de $i$ elementos de $X$,
    \ldots, y del conjunto $X$. De esta forma, tenemos que:
    $$|\cc{P}(X)| = \binom{n}{0} + \binom{n}{1} + \ldots + \binom{n}{i} + \ldots + \binom{n}{n} = (1+1)^n = 2^n$$
\end{proof}

\begin{prop}
    Sea $n \geq 1$, $(a_1, a_2, \ldots, a_n) \in A^n$. Entonces:
    $$-\left( \sum_{i=1}^n a_i \right) = \sum_{i=1}^n -a_i $$
    Además, si $a_1, a_2, \ldots, a_n \in \cc{U}(A) \Longrightarrow \prod\limits_{i=1}^n a_i \in \cc{U}(A)$, con:
    $$\left(\prod\limits_{i=1}^n a_i\right)^{-1} = \prod_{i=1}^n a_i^{-1}$$
\end{prop}
\begin{proof}
    Demostramos simultáneamente mediante inducción en $n$:
    \begin{itemize}
        \item \underline{Para $n=1$}: Por definición, es cierto.
        \item \underline{Supuesto cierto para $n-1$, lo probamos para $n$}:
        $$\sum_{i=1}^n a_i + \sum_{i=1}^n -a_i = \sum_{i=1}^{n-1} a_i + an + \sum_{i=1}^{n-1} -a_i -an = \sum_{i=1}^{n-1}a_i  + \sum_{i=1}^{n-1}-a_i = 0$$
        $$\left(\prod\limits_{i=1}^n a_i\right)\left(\prod\limits_{i=1}^n a_i^{-1}\right) =
        \left(\prod\limits_{i=1}^{n-1} a_i\right)a_n \left(\prod\limits_{i=1}^{n-1} a_i^{-1}\right)a_n^{-1} =
        \left(\prod\limits_{i=1}^{n-1} a_i\right) \left(\prod\limits_{i=1}^{n-1} a_i^{-1}\right) = 1$$
    \end{itemize}
\end{proof}


En el caso de tener una lista $(n\in \N)$ $(a_1, a_2, \ldots, a_n) \in A^n$ en la que todos sus elementos son iguales:
$$a_1 = a_2 = \ldots = a_n = a$$
Tenemos que:
$$-(na) := (-n)a = -\left(\sum_{i=1}^n a\right) = \left(\sum_{i=1}^n -a\right) = n(-a) =: -na$$
Y podemos definir, si $a \in \cc{U}(A)$:
$$a^{-n} := (a^n)^{-1} = (a^{-1})^n$$

\begin{prop}
    Sean $m,n \in \Z$, $a,b \in A$, $u,v \in \cc{U}(A)$. Se verifican:
    \begin{enumerate}
        \item $(m+n)a = ma+na$
        \item $n(a+b) = na + nb$
        \item $n(ma) = (nm)a$
        \item $(ma)(nb) = (mn)(ab)$
        \item $u^mu^n = u^{m+n}$
        \item $(uv)^n = u^nv^n$
        \item $(u^m)^n = u^{mn}$
    \end{enumerate}
\end{prop}
\begin{proof} Tan solo demostramos los casos en los que intervengan enteros negativos, ya que en caso contrario está demostrado en la proposición \ref{prop:2.13}. Consideramos $m,n>0$, y demostraremos por tanto para uno de los dos negativos ($-n$, sin perder generalidad), y ambos negativos.
    \begin{enumerate}
        \item $(m+n)a = ma+na$

        Demostramos en primer lugar si solo uno de los dos enteros es negativo. Trabajemos por tanto con $-n$:
        \begin{itemize}
            \item Si $m\geq n$, definimos $k=m-n\geq 0$. Entonces:
            \begin{equation*}
                ma-na = (n+k)a - na = na + ka -na = ka = (m-n)a
            \end{equation*}

            \item Si $m< n$, definimos $k=n-m\geq 0$. Entonces:
            \begin{equation*}
                ma-na = ma -(m+k)a = ma - ma -ka = -ka = -(n-m)a = (m-n)a
            \end{equation*}
        \end{itemize}

        Demostramos ahora con ambos negativos:
        \begin{equation*}
            (-m-n)a = [-(m+n)a] = -[(m+n)a] = -(ma+na) = -ma - na
        \end{equation*}
        
        \item $n(a+b) = na + nb$
        \begin{equation*}
            (-n)(a+b) = -n(a+b) = -(na+nb) = -na -nb
        \end{equation*}
        
        \item $n(ma) = (nm)a$
        
        Demostramos en primer lugar si solo uno de los dos enteros es negativo. Trabajemos por tanto con $-n$:
        \begin{equation*}
            (-n)(ma) = -[n(ma)] = -[(mn)a] = (-mn)a
        \end{equation*}

        Demostramos ahora con ambos negativos:
        \begin{equation*}
            (-n)[(-m)a] = -[n(-(ma))] = (mn)a = [(-n)(-m)a]
        \end{equation*}
        
        \item $(na)(mb) = (nm)(ab)$
        
        Demostramos en primer lugar si solo uno de los dos enteros es negativo. Trabajemos por tanto con $-n$:
        \begin{equation*}
            (-na)(mb) = -[(na)(mb)] = -[(nm)(ab)] = -(nm)(ab) = [(-n)m](ab)
        \end{equation*}

        Demostramos ahora con ambos negativos:
        \begin{equation*}
            (-na)(-mb) = (na)(mb) = (nm)(ab) = [(-n)(-m)](ab)
        \end{equation*}
        
        \item $u^mu^n = u^{m+n}$

        Demostramos en primer lugar si solo uno de los dos enteros es negativo. Trabajemos por tanto con $-n$:
        \begin{itemize}
            \item Si $m\geq n$, definimos $k=m-n\geq 0$. Entonces:
            \begin{equation*}
                u^mu^{-n} = u^{n+k}u^{-n} = u^ku^nu^{-n} = u^k\cdot 1 = u^k = u^{m-n}
            \end{equation*}

            \item Si $m< n$, definimos $k=n-m\geq 0$. Entonces:
            \begin{equation*}
                u^mu^{-n} = u^m u^{-(m+k)} = u^m(u^mu^k)^{-1} = u^m (u^m)^{-1} (u^k)^{-1} = u^{-k} = u^{m-n}
            \end{equation*}
        \end{itemize}

        Demostramos ahora con ambos negativos:
        \begin{equation*}
            u^{-m-n} = (u^{m+n})^{-1} = (u^mu^n)^{-1} = (u^m)^{-1}(u^n)^{-1} = u^{-m}u^{-n}
        \end{equation*}        
        
        \item $(uv)^n = u^nv^n$
        \begin{equation*}
            (uv)^{-n} = [(uv)^n]^{-1} = (u^n)^{-1}(v^n)^{-1} = u^{-n}v^{-n}
        \end{equation*}
        
        \item $(u^m)^n = u^{mn}$

        Demostramos en primer lugar si solo uno de los dos enteros es negativo. Trabajemos por tanto con $-n$:
        \begin{equation*}
            (u^m)^{-n} = [(u^m)^n]^{-1} = (u^{mn})^{-1} = u^{-mn}
        \end{equation*}

        Demostramos ahora con ambos negativos:
        \begin{equation*}
            (u^{-m})^{-n} = [((u^m)^{-1})^{-1}]^n = (u^m)^n = u^{mn} = u^{(-m)(-n)}
        \end{equation*}
    \end{enumerate}
\end{proof}


\section{Homomorfismos de anillos}
\begin{definicion}[Homomorfismo]
    Un \textbf{homomorfismo de anillos de $A$ en $A'$}, siendo $A$, $A'$ anillos conmutativos, es una aplicación
    $\phi:A \longrightarrow A'$ que verifica:
    \begin{enumerate}
        \item $\phi(a+b) = \phi(a) + \phi(b)$
        \item $\phi(ab) = \phi(a)\phi(b)$
        \item $\phi(1) = 1 \in A'$
    \end{enumerate}
\end{definicion}

\begin{ejemplo} Algunos ejemplos de homomorfismos son:
\begin{enumerate}
    \item Si $A$ es un anillo, la aplicación identidad
    \Func{id_A}{A}{A}{a}{a}
    es un homomorfismo de anillos.

    \item Si $B$ es un subanillo de $A$, la aplicación inclusión
    \Func{i}{B}{A}{b}{b}
    es un homomorfismo de anillos.

    \item $\forall n \geq 1$, la \textbf{proyección canónica}
    \Func{p}{\Z}{\Z_n}{a}{[a]}
    es un homomorfismo de anillos. Demostramos esta:
    \begin{enumerate}
        \item $p(a+b) = [a+b] = [a]+[b] = p(a)+p(b)$.
        \item $p(ab) = [ab] = [a][b] = p(a)p(b)$.
        \item $p(1) = [1]$.
    \end{enumerate}
\end{enumerate}
\end{ejemplo}

\begin{prop}
    Sean $A$, $A'$ anillos conmutativos, $\phi:A \longrightarrow A'$ un homomorfismo de anillos y $a \in A$, $u \in \cc{U}(A)$, $n \in \Z$.
    Se verifican:
    \begin{enumerate}
        \item $\phi\left( \sum\limits_{i=1}^n a_i \right) = \sum\limits_{i=1}^{n} \phi(a_i)$
        \item $\phi\left( \prod\limits_{i=1}^n a_i \right) = \prod\limits_{i=1}^n \phi(a_i)$
        \item $\phi(0)=0$
        \item $\phi(-a) = -\phi(a)$
        \item $\phi(na) = n\phi(a)$
        \item $\phi(a^n) = (\phi(a))^n$ si $n \in \N$
        \item $\phi(u) \in \cc{U}(A') \y \phi(u^{-1}) = (\phi(u))^{-1}$
        \item $\phi(u^n) = (\phi(u))^n$
    \end{enumerate}
\end{prop}
\begin{proof} \
\begin{enumerate}
    \item $\phi\left( \sum\limits_{i=1}^n a_i \right) = \sum\limits_{i=1}^{n} \phi(a_i)$

    Realizamos inducción en $n$:
    \begin{itemize}
        \item \underline{Para $n=1$}: $\phi(a_1) = \phi(a_1)$, Cierto.
        \item \underline{Supuesto cierto para $n-1$, lo probamos para $n$}:
        $$\phi\left( \sum\limits_{i=1}^n a_i \right)
        = \phi\left( \sum\limits_{i=1}^{n-1} a_i \right) + \phi(a_n)
        = \sum_{i=1}^{n-1} \phi(a_i) + \phi(a_n)
        = \sum\limits_{i=1}^{n} \phi(a_i)$$
    \end{itemize}
    
    \item $\phi\left( \prod\limits_{i=1}^n a_i \right) = \prod\limits_{i=1}^n \phi(a_i)$

    Realizamos inducción en $n$:
    \begin{itemize}
        \item \underline{Para $n=1$}: $\phi(a_1) = \phi(a_1)$, Cierto.
        \item \underline{Supuesto cierto para $n-1$, lo probamos para $n$}:
        $$\phi\left( \prod\limits_{i=1}^n a_i \right) = \phi\left( \prod\limits_{i=1}^{n-1} a_i \right)  \phi(a_n) =
        \left( \prod_{i=1}^{n-1} \phi(a_i) \right) \phi(a_n) = \prod\limits_{i=1}^{n} \phi(a_i)$$
    \end{itemize}
    
    \item $\phi(0)=0$

    Por ser $\phi$ un homomorfismo tenemos que $\phi(0) = \phi(0+0) = \phi(0) + \phi(0)$. Por tanto:
    $$0 = \phi(0) - \phi(0) = \phi(0) + \phi(0) - \phi(0) = \phi(0)$$
    
    \item $\phi(-a) = -\phi(a)$
    $$0 = \phi(0) = \phi(a+(-a)) = \phi(a) + \phi(-a) \Longrightarrow -\phi(a) = \phi(-a)$$
    
    \item $\phi(na) = n\phi(a)$
    $$\phi(na) = \phi\left( \sum_{i=1}^n a \right) = \sum_{i=1}^n \phi(a) = n\phi(a)$$
    
    \item $\phi(a^n) = (\phi(a))^n$ si $n \in \N$
    $$\phi(a^n) = \phi\left( \prod_{i=1}^n a \right) = \prod_{i=1}^n \phi(a) = (\phi(a))^n$$
    
    \item $\phi(u) \in \cc{U}(A') \y \phi(u^{-1}) = (\phi(u))^{-1}$

    Tenemos que $u \in \cc{U}(A)$, por lo que $\exists u^{-1} \mid u \cdot u^{-1} = 1$. Por tanto,
    $$1 = \phi(1) = \phi(u \cdot u^{-1}) = \phi(u) \phi(u^{-1})$$

    Por tanto, $\phi(u) \in \cc{U}(A')$, con $(\phi(u))^{-1} = \phi(u^{-1})$.
    
    \item $\phi(u^n) = (\phi(u))^n$

    Si el exponente es positivo, tenemos que se trata de un producto generalizado y, por ser un homomorfismo, se tiene de forma directa por la propiedad $2$. Por tanto, vemos el caso negativo. Sea $n \in \N$:
    $$\phi(u^{-n}) = \phi((u^{-1})^n) = (\phi(u)^{-1})^n = (\phi(u))^{-n}$$
    
\end{enumerate}
\end{proof}

\begin{prop}
    \label{prop:ImagenHomomorfismo}
    Sean $A$, $A'$ anillos conmutativos y $\phi:A \longrightarrow A'$ un homomorfismo de anillos. Entonces:
    $$Img(\phi) = \{\phi(a) \mid a \in A\} \mbox{ es un subanillo de } A'$$
\end{prop}
\begin{proof}
    Es demostrar que $Img(\phi)$ es cerrado para la suma, producto y opuesto de $A'$ y que contiene al 1 y al 0 de $A'$.
    
    Sean $a',b' \in Img(\phi) \Longrightarrow \exists a,b \in A \mid \phi(a) = a' \y \phi(b) = b'$. Por tanto,
    \begin{gather*}
        a'+b' = \phi(a)+\phi(b) = \phi(a+b) \in Img(\phi) \\
        a'b' = \phi(a)\phi(b) = \phi(ab) \in Img(\phi) \\
        -a' = -\phi(a) = \phi(-a) \in Img(\phi)
    \end{gather*}
    $$1 = \phi(1) \in Img(\phi) \hspace{1cm} 0 = \phi(0) \in Img(\phi)$$
\end{proof}

\begin{definicion}
    Sea $\phi:A\longrightarrow A'$ un homomorfismo de anillos, diremos que $\phi$ es un:
    \begin{itemize}
        \item \textbf{Epimorfismo} si $\phi$ es sobreyectiva (es decir, $Img(\phi)=A'$).
        \item \textbf{Monomorfismo} si $\phi$ es inyectiva.
        \item \textbf{Isomorfismo} si $\phi$ es biyectiva.
    \end{itemize}
\end{definicion}

En el caso de que $\phi$ sea un isomorfismo, diremos que $A$ y $A'$ son \textbf{isomorfos}, notado $\displaystyle A \mathop{\cong}^{\phi} A'$ o simplemente. Recordamos que si $\phi:A\longrightarrow A'$ es un isomorfismo, por ser biyectiva, tenemos que $$\exists \phi^{-1}:A' \longrightarrow A
    \mid \phi \circ \phi^{-1} = id_{A'} \y \phi^{-1} \circ \phi = id_A.$$
    
\begin{prop}
    Sea $\phi:A\longrightarrow A'$ un isomorfismo, se verifica que $\phi^{-1}:A'\longrightarrow A$ es también un isomorfismo.
\end{prop}
\begin{proof}
    Tenemos que $\phi^{-1}$ es biyectiva, por lo que solo nos falta demostrar que es un homomorfismo. Sean $a',b' \in A' \Longrightarrow \phi^{-1}(a'),\phi^{-1}(b') \in A$.\\

    Veamos en primer lugar que $\phi^{-1}(a'+b') = \phi^{-1}(a') + \phi^{-1}(b')$. Como $A$ es cerrado para sumas, $\phi^{-1}(a') + \phi^{-1}(b') \in A$. Luego:
    $$\phi(\phi^{-1}(a') + \phi^{-1}(b')) = \phi(\phi^{-1}(a')) + \phi(\phi^{-1}(b')) = a' + b' = \phi(\phi^{-1}(a'+b'))$$
    Como $\phi$ es inyectiva, tenemos que $\phi^{-1}(a')+\phi^{-1}(b') = \phi^{-1}(a'+b')$.\\
    
    Veamos ahora que $\phi^{-1}(a'b')=\phi^{-1}(a')\phi^{-1}(b')$. Como $A$ es cerrado para productos, $\phi^{-1}(a')\phi^{-1}(b') \in A$. Luego:
    $$\phi(\phi^{-1}(a')\phi^{-1}(b')) = \phi(\phi^{-1}(a'))\phi(\phi^{-1}(b')) = a'b' = \phi(\phi^{-1}(a'b'))$$
    Como $\phi$ es inyectiva, tenemos que $\phi^{-1}(a')\phi^{-1}(b') = \phi^{-1}(a'b')$.\\

    Veamos ahora que $\phi^{-1}(1)=1$. Como $1 = \phi(1) \Longrightarrow \phi^{-1}(1) = \phi^{-1}(\phi(1)) = 1$.
\end{proof}

\begin{ejemplo}
    Consideramos la proyección canónica; es decir,
    $$p:\Z \longrightarrow \Z_5 \mid p(a) = [a] \hspace{1cm} \forall a \in \Z$$
    $$p(12^3) = p(12)^3 = 2^3 = 3$$
\end{ejemplo}

\begin{prop}
    Sean $\phi:A\longrightarrow A'$, $\varphi:A'\longrightarrow A''$ homomorfismos de anillos, se verifica que $\varphi \circ \phi:A\longrightarrow A''$ es también un homomorfismo de anillos. Es decir, la composición de homomorfismos de anillos es un homomorfismo de anillos.
\end{prop}
\begin{proof}
    Sean $a,b \in A$. Veamos que cumple las tres condiciones para que sea un homomorfismo de anillos, donde para ello usamos que $\varphi, \phi$ son homomorfismos de anillos.
    \begin{enumerate}
        \item $(\varphi \circ \phi)(a+b) = (\varphi \circ \phi)(a) + (\varphi \circ \phi)(b)$.
        \begin{equation*}
            (\varphi \circ \phi)(a+b) = \varphi(\phi(a+b)) = \varphi(\phi(a) + \phi(b)) = \varphi(\phi(a)) + \varphi(\phi(b)) = (\varphi \circ \phi)(a) + (\varphi \circ \phi)(b)
        \end{equation*}

        \item $(\varphi \circ \phi)(ab) = (\varphi \circ \phi)(a)\cdot (\varphi \circ \phi)(b)$.
        \begin{equation*}
            (\varphi \circ \phi)(ab) = \varphi(\phi(ab)) = \varphi(\phi(a)\phi(b)) = \varphi(\phi(a))\varphi(\phi(b)) = (\varphi \circ \phi)(a)\cdot (\varphi \circ \phi)(b)
        \end{equation*}

        \item $(\varphi \circ \phi)(1)=1$.
        \begin{equation*}
            (\varphi \circ \phi)(1) = \varphi(\phi(1)) = \varphi(1) = 1
        \end{equation*}
    \end{enumerate}
\end{proof}

\section{El anillo de los polinomios}
\begin{definicion}[Copia isomorfa]
    Sea $\iota:A\longrightarrow B$ un monomorfismo de anillos conmutativos. Entonces, $\iota:A\longrightarrow Img(\iota)$ es un isomorfismo y diremos que el anillo $B$ contiene una \textbf{copia isomorfa} del anillo $A$ que no es otro que su imagen por $\iota$.
\end{definicion}


Siempre que el monomorfismo $\iota$ esté claro por el contexto no hay problema en identificar $A$ con $Img(\iota)$. Lo que
significa que si $a\in A$ y $b \in B$, escribiremos:
\begin{itemize}
    \item $ab$ para representar $\iota(a)b$.
    \item $a+b$ para representar $\iota(a)+b$.
\end{itemize}
Esto se debe a que, como no pertenecen al mismo anillo, por norma general no podemos operar entre ellos.

\begin{ejemplo}
    Sea $A = \{f:[0,1]\longrightarrow \R \mid f \mbox{ aplicación}\}$.
    
    Consideramos:
    \Func{\iota}{\R}{A}{r}{\iota(r)}

    Como $\iota(r)\in A$, tenemos que:
    \Func{\iota(r)}{[0,1]}{\R}{t}{\iota(r)(t)=r}
    
    Podemos ver que $\iota$ es un monomorfismo:\\

    
    Probamos primero que $\iota$ es un homomorfismo. Sean $a,b \in \R$, $\forall t \in [0,1]$:
    \begin{gather*}
        a+b = \iota(a+b)(t) = \iota(a)(t) + \iota(b)(t) = a+b \\
        ab = \iota(ab)(t) = \iota(a)(t) \iota(b)(t) = ab \\
        1 = \iota(1)(t)
    \end{gather*}
    
    A continuación, probamos que es inyectiva:
    $$\forall a,b \in \R \mid \iota(a) = \iota(b) \Longrightarrow \forall t \in [0,1] \hspace{1cm} a=\iota(a)(t)=\iota(b)(t) = b$$

    
    Por lo visto, podemos concluir que $\iota$ define un isomorfismo $\displaystyle \iota:\R \mathop{\longrightarrow}^{\cong}
        Img(\iota)$.\\
        
    Es usual identificar $\R$ con $Img(\iota)$ y de esta forma considerar $\R$ como un subanillo de $A$. Es decir, a cada número real $r \in \R$ lo identificaremos con la aplicación constantemente igual a $r$ en $[0,1]$.
\end{ejemplo}

\begin{teo}
    Sea $A$ un anillo conmutativo cualquiera. Entonces exite un anillo conmutativo $P$ que contiene una copia isomorfa de $A$
    y un elemento que llamaremos $x$ tal que cualquier elemento no nulo $f \in P$ se representa de forma única como:
    $$f = f_0 + f_1x + f_2 x^2 + \ldots + f_n x^n$$
    Donde $f_0, f_1, f_2, \ldots, f_n \in A$ con $f_n \neq 0$ y $n \geq 0$.
\end{teo}
\begin{proof}
    Definimos el anillo conmutativo $S$ que contendrá a $P$ como subanillo:
    $$S = \{f:\N \longrightarrow A \mid f \mbox{ es aplicación}\}$$
    
    Dado $f\in S$, escribiremos $f(n) = f_n~~\forall n \in \N$ y pondremos: $f=(f_n)_{n\geq 0}$ o bien $f = (f_0, f_1, \ldots, f_n, \ldots)$. Es decir, $f$ es una sucesión de elementos de $A$.\\
    
    Definimos la suma de $f=(f_n)_{n\geq0} $y $g=(g_n)_{n\geq 0}$ como:
    $$s = f+g = (f_n)_{n\geq0} + (g_n)_{n\geq0} = (f_n+g_n)_{n\geq0} = (s_n)_{n\geq0}$$
    $$s_i = f_i + g_i \hspace{1cm} \forall i \in \{0, 1, \ldots, n\}$$
    
    Y el producto de $f$ y $g$ por:
    $$p = f\cdot g = (p_n)_{n\geq0}$$
    $$p_k = \sum_{i+j=k} f_i g_j = f_n g_0 + f_{n-1}g_1 + \ldots + f_1 g_{n-1} + f_0 g_n~~\forall k \in \{0, \ldots, n\}$$

    
    Comprobamos a continuación que $S$ es un anillo conmutativo:
    \begin{itemize}
        \item La suma de $S$ es asociativa y conmutativa gracias a la suma de $A$ (compruébese).

        \item El cero de $S$ es $0 = (0, 0, \ldots, 0, \ldots) = (0_n)_{n\geq0}$, donde $0_n = 0~~\forall n \in\N$.

        \item El opuesto de $f=(f_n)_{n\geq0}$ es $-f:=(-f_n)_{n\geq0}$.

        \item Veamos la propiedad asociativa del producto. Sean $f=(f_n)_{n\geq0}$, $g=(g_n)_{n\geq0}$, $h=(h_n)_{n\geq0}$:
        
        El término $n$-ésimo de $(fg)h$ es:
        $$\sum\limits_{i+j=n} \left( \sum_{u+v=i} f_u g_v \right) h_j = \sum_{u+v+j=n} f_u g_v h_j$$
        El término $n$-ésimo de $f(gh)$ es:
        $$\sum\limits_{i+j=n} f_i \left( \sum_{u+v=j} g_u h_v \right) = \sum_{i+u+v=n} f_i g_u h_v$$

        Como podemos ver, coinciden. Por tanto, $(fg)h = f(gh)$.

        \item Veamos ahora la propiedad conmutativa del producto. Sean $f=(f_n)_{n\geq0}$, \newline $g=(g_n)_{n\geq0}$:
        $$\left. \begin{array}{l}
            f\cdot g = p \mid p_k = \sum\limits_{i+j=k} f_ig_j \\
            g \cdot f = p\mid p_k = \sum\limits_{j+i=k} g_if_j
        \end{array} \right\} \Longrightarrow f\cdot g = g\cdot f$$

        \item Veamos la propiedad distributiva del producto. Sean $f=(f_n)_{n\geq0}$, $g=(g_n)_{n\geq0}$, $h=(h_n)_{n\geq0}$:
        $$\left. \begin{array}{l}
            f(g+h) = p \mid p_k = \sum\limits_{i+j = k} f_i (g_j + h_j) \\
            fg + fh = p\mid p_k = \sum\limits_{i+j=k} f_ig_j + \sum\limits_{i+j=k} f_i h_j = \sum\limits_{i+j=k} f_i(g_j+h_j)
        \end{array}\right\} \Longrightarrow f(g+h) = fg + fh$$

        \item El uno del anillo es la sucesión $1=(1,0,0,\ldots, 0)$:
        $$(1\cdot f)_k = f_k \cdot 1 + f_{k-1} \cdot 0 + \ldots + f_1 \cdot 0 + f_0 \cdot 0 = f_k ~~ k \in \{0, \ldots, n\}
            \Longrightarrow 1 \cdot f = f$$
    \end{itemize}
    
    Por tanto, tenemos que $S$ es un anillo conmutativo.\\
    \bigskip
    
    Definimos ahora la aplicación $\iota:A\longrightarrow S$ por:
    $$\iota(a) = (a, 0, 0, \ldots, 0, \ldots) \hspace{1cm} \forall a \in A$$
    
    Veamos en primer lugar que $\iota$ es un homomorfismo de anillos. Sean $a,b \in A$:
    \begin{enumerate}
        \item $\iota(a+b) = (a+b, 0, 0, \ldots, 0, \ldots) = (a, 0, 0, \ldots, 0, \ldots) + (b, 0, 0, \ldots, 0, \ldots) = $ $=\iota(a)+\iota(b)$.

        \item $\iota(ab) = (ab, 0, 0, \ldots, 0, \ldots) \stackrel{(\ast)}{=} (a, 0, 0, \ldots, 0, \ldots) (b, 0, 0, \ldots, 0, \ldots) = \iota(a)\iota(b)$

        Donde en $(\ast)$ hemos aplicado que:
        \begin{itemize}
            \item Si $n \geq 1$, $\iota(ab)_n = \sum\limits_{i+j=n} \iota(a)_i \iota(b)_j$. Si $i+j=n$, como $n \geq 1$, tenemos que:
            \begin{equation*}
                \left\{\begin{array}{ccc}
                    i\geq 1 & \Longrightarrow & \iota(a)_i=0\\
                    \lor&&\lor\\
                    j\geq 1 & \Longrightarrow & \iota(b)_j=0
                \end{array}\right\}
            \end{equation*}

            \item Si $n=0$, $\iota(ab)_0 = \sum\limits_{i+j=0} \iota(a)_i \iota(b)_j=\iota(a)_0 \iota(b)_0 = ab$.
        \end{itemize}

        

        \item $\iota(1)=(1,0,0,\dots,0,\dots)=1$.
    \end{enumerate}

    Por tanto, tenemos que $\iota$ es un homomorfismo de anillos. Veamos que, en concreto, es un monomorfismo.
    Supuestos $a,b \in A \mid \iota(a) = \iota(b)$. Entonces:
    $$(a, 0, \ldots, 0, \ldots) = \iota(a) = \iota(b) = (b, 0, \ldots, 0, \ldots) \Longrightarrow a=b \Longrightarrow \iota \mbox{ inyectiva}$$

    
    $\iota:A \longrightarrow S$ es un monomorfismo por lo que $S$ contiene a $A$ como copia isomorfa.\newline
    Identificaremos $A$ con $Img(\iota)$ en $S$. Esto es, $a$ implicará $(a, 0, \ldots, 0, \ldots) \in S~~\forall a \in A$.\\

    
    Denotaremos por $x$ al elemento de $S$ definido por:
    $$x = (0, 1, 0, \ldots, 0, \ldots)$$
    Es decir, $x$ es la sucesión cuyos términos son $0$ excepto el término $n=1$ que es $1 \in A$. Tenemos por tanto, que $\forall a \in A$:
    $$ax = (a, 0, \ldots, 0, \ldots) (0, 1, 0, \ldots, 0, \ldots)$$

    Veamos el resultado de dicho producto:
    $$(ax)_n = \sum_{i+j=n} a_i x_j = a_n x_0 + a_{n-1} x_1 + \ldots + a_1 x_{n-1} + a_0 x_n0$$
    \begin{itemize}
        \item Para $n=0$:
        $$(ax)_0 = \sum_{i+j = 0} a_i x_j = a_0 x_0 = a \cdot 0 = 0$$

        \item Para $n=1$:
        $$(ax)_1 = \sum_{i+j=1} a_i x_j = a_1 x_0 + a_0 x_1 = 0 \cdot 0 + a \cdot 1 = a$$

        \item Para $n>1$:
        $$(ax)_{n > 1} = \sum_{i+j=n} a_i x_j = a_n x_0 + a_{n-1} x_1 + \ldots + a_1 x_{n-1} + a_0 x_n = 0 + 0 \cdot 1 + \ldots + 0+0 = 0$$
    \end{itemize}

    Por tanto, tenemos que:
    $$ax = (0, a, 0, \ldots, 0, \ldots)$$
    
    Veamos el efecto de multiplicar $x$ por $f=(f_n)_{n\geq 0} \in S$:
    $$xf = (0, 1, 0, \ldots, 0, \ldots) (f_0, f_1, \ldots, f_k, \ldots)$$
    \begin{itemize}
        \item Para $n=0$:
        $$(xf)_0 = x_0 f_0 = 0 \cdot f_0 = 0$$
        
        \item Para $n=1$:
        $$(xf)_1 = x_1 f_0 + x_0 f_1 = 1 \cdot f_0 + 0 \cdot f_1 = f_0$$
        
        \item Para $n=2$:
        $$(xf)_2 = x_2 f_0 + x_1 f_1 + x_0 f_2 = 0 + 1 \cdot f_1 + 0 = f_1$$

        \item Para $n=k\in \{1,2,\dots\}$:
        $$(xf)_k = x_k f_0 + x_{k-1} f_1 + \ldots + x_1 f_{k-1} + x_0 f_k = 0 + 0 + \ldots + f_{k-1} + 0 = f_{k-1}$$
    \end{itemize}
    
    Por tanto, tenemos que:
    $$xf = (0, f_0, f_1, \ldots, f_{k-1}, \ldots)$$

    De lo anterior, se deduce que
    \begin{gather*}
        \forall k \geq 1 \hspace{1cm} x^k = (0, 0, \ldots, \underbrace{1}_k, \ldots, 0, \ldots)\\
        \forall a \in A \hspace{1cm} ax^k = (0, 0, \ldots, \underbrace{a}_k, \ldots, 0, \ldots)
    \end{gather*}
    
    Definimos $P$ como el subconjunto de $S$ formado por aquellas sucesiones que tienen todos sus términos 0 salvo un número finito. Es decir, $f \in P$ si $\exists n \in \N$ tal que $f_k = 0~~\forall k \geq n+1$. Por tanto, $f \in P$ es de la forma:
    $$f = (f_0, f_1, \ldots, f_n, 0, 0, \ldots, 0, \ldots)$$

    
    Veamos que $P$ es un subanillo de $S$. Sean $n,m \in \N$ y $f=(f_0, f_1, \ldots, f_n, 0, \ldots)$, $g=(g_0, g_1, \ldots, g_m, 0, \ldots)~\in P$:
    \begin{enumerate}
        \item $P$ es cerrado para sumas:
        $$f+g = (f_0, \ldots, f_n, 0, \ldots) + (g_0, \ldots, g_m, 0, \ldots)$$
        \begin{itemize}
            \item Si $n = m$: $$f+g = (f_0 + g_0, f_1 + g_1, \ldots, f_n + g_n, 0, \ldots) \in P$$
            \item Si $n > m$: $$f+g = (f_0 + g_0, f_1 + g_1, \ldots, f_m + g_m, f_{m+1}, \ldots, f_n, 0, \ldots) \in P$$
            \item Si $m > n$: $$f+g = (f_0 + g_0, f_1 + g_1, \ldots, f_n + g_n, g_{n+1}, \ldots, g_m, 0, \ldots) \in P$$
        \end{itemize}
        Por lo que $P$ es cerrado para la suma de $S$.

        \item $P$ es cerrado para el producto:\\
        Para ello, observemos que $(fg)_k = 0 \hspace{1cm} \forall k > n+m$:
        $$(fg)_k = \sum_{i+j=k} f_i g_j$$
        Si $k = i+j > n+m \Longrightarrow 
        \left\{\begin{array}{ccc}
            i>n & \Longrightarrow & f_i=0\\
            \o && \o\\
            j>m & \Longrightarrow & g_j=0
        \end{array}\right\}
        \Longrightarrow f_i g_j = 0$.
        Por tanto, se tiene que $fg \in P$, por lo que es cerrado para el producto.

        \item $P$ es cerrado para opuestos:
        $$-f = (-f_0, -f_1, \ldots, -f_n, 0, \ldots, 0, \ldots) \in P$$

        \item $0,1\in P$:
        $$0 = (0, 0, \ldots, 0, \ldots) \in P \hspace{1cm} 1 = (1, 0, 0, \ldots, 0, \ldots) \in P$$
    \end{enumerate}
    
    Por lo que $P$ es un subanillo de $S$.

    
    Tenemos además que $P$ contiene una copia isomorfa de $A$ ya que anteriormente identificamos cualquier $a \in A$ como $(a, 0, 0, \ldots, 0, \ldots) \in P$ y tenemos además que $x = (0, 1, 0, \ldots, 0, \ldots) \in P$.

    Por tanto, tenemos ya demostrada la existencia del anillo $P$ anunciado en el teorema. Falta ver que cualquier elemento $f \in P$ no nulo puede expresarse de forma única como:
    $$f = f_0 + f_1x + f_2x^2 + \ldots + f_n x^n$$

    Para ello, aplicamos los resultados ya demostrados en este teorema:
    \begin{equation*}
        \begin{split}
            f
            &= f_0 + f_1x + f_2x^2 + \ldots + f_n x^n = \\
            & = (f_0, 0, \ldots, 0, \ldots) + (0, f_1, 0, \ldots, 0, \ldots) + \ldots + (0, \ldots, \underbrace{f_n}_n, 0, \ldots) = \\
            & = (f_0, f_1, f_2, \ldots, f_n, 0, \ldots, 0, \ldots)
        \end{split}
    \end{equation*}
    
    Como dos elementos de $P$ son iguales si y sólo si sus términos son iguales término a término, tenemos que podemos expresar de forma única cualquier elemento $f$ no nulo de $P$ de la forma buscada.
\end{proof}

\begin{definicion}[Anillo de polinomios]
    El anillo $P$ del teorema anterior se llama \textbf{anillo de polinomios en la indeterminada $x$ con coeficientes en el anillo conmutativo $A$}, denotado por $A[x]$. A sus elementos los llamaremos polinomios.
\end{definicion}

Respecto a los polinomios, tenemos que:
\begin{itemize}
    \item Si $f \in A[x] \mid f \neq 0 , \quad f = f_0 + f_1x + f_2 x^2 + \ldots + f_n x^n$ lo escribiremos también como:
    $$f = \sum_{i=0}^n f_i x^i$$

    \item Si $f \in A[x] \mid f\neq 0$ y sus términos son todos 0 a partir del término $n$-ésimo, diremos que \textbf{$f$ es de grado $n$} y lo notaremos $grd(f) = n$. Por convenio, al polinomio 0 se le asigna el grado $-\infty$.

    \item Cada $f_i x^i$ recibe el nombre de \textbf{monomio} o término de $f$.

    \item Al término $f_n x^n$ se le llama \textbf{término líder} y a $f_n$ le llamamos \textbf{coeficiente líder}.

    \item  Al término $f_0$ se le llama \textbf{término independiente}.

    \item Si $f_n = 1$, diremos que $f$ es un polinomio \textbf{mónico}.

    \item Los polinomios de grado 0 (los de la forma $f=a,~~a \neq 0$), junto con el polinomio 0 se les llama \textbf{polinomios constantes}.
\end{itemize}

\begin{notacion}
    A veces, dado $f=\sum\limits_{i=0}^n f_i x^i \in A[x]$, utilizaremos la notación $f(x)$ para referirnos al polinomio $f$. Es decir:
    $$f = f(x) = \sum_{i=0}^n f_i x^i$$
\end{notacion}

En el anillo de polinomios, tenemos que para $f = \sum\limits_{i=0}^n f_i x^i$, $g = \sum\limits_{j=0}^m g_j x^j ~\in A[x]$ con $n > m$, la suma y el producto vienen dados por:
\begin{equation*}
    f+g = \sum\limits_{i=0}^n f_i x^i + \sum\limits_{j=0}^m g_j x^j = \sum_{j=0}^m (f_j + g_j)x^j + \sum_{i=m}^n f_i x^i
\end{equation*}
\begin{equation*}
    fg = \left(\sum\limits_{i=0}^n f_i x^i \right) \left( \sum\limits_{j=0}^m g_j x^j\right) = \sum_{i,j} f_i x^i g_j x^j = \sum_{i,j} f_i g_j x^{i+j} = \sum_{k=0}^{n+m} \left( \sum_{i+j=k} f_i g_j \right) x^k
\end{equation*}

\begin{ejemplo}
    Sea $A = \Z_4$, $f = 1 + 3x + 3x^7$, $g=3+2x \in \Z_4[X]$:
    \begin{gather*}
        f+g = 0 + x + 3x^7 = x + 3x^7\\
        fg = 2 + 3x + 2x^2 + x^7 + 2x^8
    \end{gather*}
\end{ejemplo}

\begin{prop}
    Sean $f,g \in A[x]$. Entonces:
    \begin{gather*}
        grd(f+g) \leq \max\{grd(f), grd(g)\} \\
        grd(fg) \leq grd(f) + grd(g)
    \end{gather*}
\end{prop}
\begin{proof}
    Sean $f,g \in A[x] \mid grd(f) = n,~grd(g) = m$; es decir, $f = \sum\limits_{i=0}^n f_ix^i$ y $g=\sum\limits_{j=0}^m g_j x^j$. Sabemos que: 
        $\left\{\begin{array}{c}
            i>n \Longrightarrow f_i=0\\
            j>m \Longrightarrow g_j=0
        \end{array}\right.$. Por tanto:
    \begin{multline*}
        f+g = \sum_{i=0}^{\mathclap{\max\{m,n\}}} (f_i+g_i) \Longrightarrow grd(f+g) \leq \max\{n,m\}\\
        fg = \sum_{k=0}^{n+m} \left( \sum_{i+j=k} f_ig_j \right)x^k \Longrightarrow grd(fg) \leq n+m
    \end{multline*}
\end{proof}
También podemos afirmar que en $\Z$ se da la igualdad en el caso del producto, pero para ello hemos de introducir antes el concepto de Dominio de Integridad, que corresponde al siguiente tema.

\begin{ejemplo}\
\begin{enumerate}
    \item Sean $f=2+3x+x^2$, $g=1-x-x^2 \in \Z[x]$:
    \begin{gather*}
        f+g = 3 + 2x \Longrightarrow grd(f+g) < \max\{grd(f), grd(g)\} \\
        fg = 2+x-4x^2-4x^3-x^4 \Longrightarrow grd(fg) = grd(f) + grd(g)
    \end{gather*}

    \item Sean $f=1+2x$, $g=2x \in \Z_4[x]$:
    $$fg = 2x \Longrightarrow grd(fg) < grd(f) + grd(g)$$

    Vemos que, en este caso, no se da la igualdad en el producto.
\end{enumerate}
\end{ejemplo}

\begin{teo}[Propiedad universal del anillo de polinomios]
    Sean $A$, $B$ anillos conmutativos y $\phi:A\longrightarrow B$ un homomorfismo de anillos conmutativos. Entonces, $\forall b \in B~~\exists_1 \phi_b:A[x]\longrightarrow B$ tal que:
    \begin{enumerate}
        \item $\phi_b(a) = \phi(a), \quad \forall a\in A$,
        \item $\phi_b(x) = b$.
    \end{enumerate}
\end{teo}
\begin{proof}
    Demostramos primero la existencia del homomorfismo:

    
    Definimos $\phi_b:A[x]\longrightarrow B$ para cada $f(x) = \sum\limits_{i=0}^n a_i x^i \neq 0$ como:
    \begin{equation*}
        \phi_b(0) = 0 \hspace{1cm} \phi_b(f) = \sum_{i=0}^n \phi(a_i)b^i
    \end{equation*}
    
    Para demostrar el resultado, consideramos los siguientes polinomios: $$f(x) = \sum_{i=0}^n a_i x^i, ~g(x)=\sum_{i=0}^m c_i x^i \in A[x]$$

    Comprobemos en primer lugar que se trata de un homomorfismo:
    \begin{enumerate}
        \item $\phi_b(f+g) = \phi_b(f) + \phi_b(g)$
        \begin{equation*}
            \begin{split}
                \phi_b(f + g)
                &= \phi_b\left( \sum_{i=0}^{{\max\{m,n\}}} (f_i + g_i) x^i \right)
                =  \sum_{i=0}^{\mathclap{\max\{m,n\}}} \phi(f_i + g_i) b^i =\\
                &=\sum_{i=0}^{\mathclap{\max\{m,n\}}} \left( (\phi(f_i) + \phi(g_i)) b^i \right)
                = \sum_{i=0}^{\mathclap{\max\{m,n\}}} \left( \phi(f_i)b^i + \phi(g_i)b^i \right) =\\
                &= \sum_{i=0}^n \phi(f_i)b^i + \sum_{i=0}^m \phi(g_i)b^i
                = \phi_b(f) + \phi_b(g)
            \end{split}
        \end{equation*}

        \item $\phi_b(fg) = \phi_b(f)\phi_b(g)$
        \begin{equation*}
            \begin{split}
                \phi_b(fg)
                &= \phi_b\left( \sum_{k=0}^{n+m} \left( \sum_{i+j=k} a_i b_j \right) \right)
                = \sum_{k=0}^{n+m} \phi\left( \sum_{i+j=k} a_ib_j \right)b^k = \\
                &= \sum_{k=0}^{n+m}\left( \sum_{i+j=k} \phi(a_i)\phi(b_j) \right) b^k
                = \left( \sum_{i=0}^n \phi(a_i)b^i \right) \left( \sum_{j=0}^m \phi(b_j) b^j \right) =\\
                & = \phi_b(f)\phi_b(g)
            \end{split}
        \end{equation*}

        \item $\phi_b(1)=1$
        $$\phi_b(1) = \sum_{i=0}^0 \phi(1) b^i = \phi(1) b^0 = \phi(1) = 1$$
    \end{enumerate}
    
    Falta ver que cumple las dos condiciones pedidas para este homomorfismo en concreto:
    \begin{enumerate}
        \item $\phi_b(a) = \phi(a), \quad \forall a\in A$,
        $$\forall a \in A~~\phi_b(a) = \sum_{i=0}^0 \phi(a) b^i = \phi(a) b^0 = \phi(a)$$
        \item $\phi_b(x) = b$.
        $$\phi_b(x) = \sum_{i=0}^1 \phi(x_i) b^i = x_0 b^0 + x_1 b^1 = 0 + b = b$$
    \end{enumerate}
    
    Comprobemos ahora la unicidad del homomorfismo. Suponemos que $\exists \psi:A[x]\longrightarrow~B$ homomorfismo tal que:
    $$\psi(a) = \phi(a), \;\; \forall a \in A
    \hspace{2cm}
    \psi(x) = b
    $$
    Sea $f=\sum\limits_{i=0}^n a_ix^i \in A[x]$. Entonces, $\forall f\in A[X]$, se tiene que:
    \begin{equation*}
        \psi(f) = \psi\left( \sum_{i=0}^n a_i x^i \right) \stackrel{(\ast)}{=} \sum_{i=0}^n \psi(a_i x^i) \stackrel{(\ast)}{=} \sum_{i=0}^n \psi(a_i)\psi(x)^i =\sum_{i=0}^n \phi(a_i) b^i = \phi_b(f)
    \end{equation*}
    donde, en $(\ast)$, hemos empleado que $\psi$ es un homomorfismo. Por tanto, se tiene que $\psi = \phi$.
\end{proof}

\begin{definicion}[Homomorfismo de evaluación de polinomios]
    Sea $A\subset B$ un subanillo de un anillo conmutativo $B$. Podemos considerar el homomorfismo inclusión de $A$ en $B$. Es decir, el homomorfismo
    \Func{i}{A}{B}{a}{a}
    
    Entonces, por el teorema anterior tomando $\phi=i$, se tiene que $\forall b \in B,~~\exists_1$ homomorfismo $ev_b:A[x]\longrightarrow B$ tal que:
    \begin{enumerate}
        \item $ev_b(a) = a,~~\forall a \in A$,
        \item $ev_b(x) = b$.
    \end{enumerate}
    
    Llamaremos a este homomorfismo \textbf{homomorfismo de evaluación} en el elemento $b \in B$. Este cumple que:
    $$\forall f(x) = \sum_{i=0}^n a_ix^i \in A[x], \hspace{1cm} ev_b(f) = \sum_{i=0}^n a_i b^i$$
    Será usual notar $ev_b(f)$ por $f(b)$.
\end{definicion}

Para realizar $f(b)$ será tan fácil como sustituir el elemento $x$ por $b$ y operar dentro del anillo $B$. Es importante notar que evaluar es un homomorfismo, luego son triviales las propiedades:
$$(f+g)(b) = f(b) + g(b) \hspace{1cm} (fg) = f(b)g(b) \hspace{1cm} \forall b \in B$$

\begin{ejemplo}\
\begin{enumerate}
    \item Consideramos $A=\Z$, $B=\Q$, y sea el polinomio $f=2+x^2 \in \Z[x]$:
    $$f\left(\dfrac{1}{2}\right) = 2 + \left(\dfrac{1}{2}\right)^2 = 2 + \dfrac{1}{4} = \dfrac{9}{4}$$

    \item Consideramos $A=B=\Z$, y sea el polinomio $g=1+2x+x^2 \in \Z[x]$:
    $$g(-1) = 1 - 2 + 1 = 0$$

    \item Consideramos $A=\Z$, $B=\C$, y sea el polinomio $h=(x^2+1)^2 + (x-1)^2 \in \Z[i]$:
    $$h(i) = (i^2 + 1)^2 + (i-1)^2 = i^2 - 2i + 1 = -2i$$

    Notemos que no desarrollamos el producto para obtener un polinomio porque evaluar es un polinomio.
\end{enumerate}
\end{ejemplo}

\begin{definicion}[Raíz de un polinomio]
    Sea $A$ un subanillo de un anillo conmutativo $B$, sea $f \in A[x]$. Un elemento $b \in B$ decimos que es una \textbf{raíz} o un cero del polinomio $f$ si $f(b) = 0$.
\end{definicion}

\begin{teo}[Homomorfismo inducido en el anillo de polinomios]
    Sea $\phi:~A\longrightarrow~B$ un homomorfismo cualquiera de anillos conmutativos. Entonces, existe un único homomorfismo $\phi:A[x]\longrightarrow B[x]$ (el cual no debe confundirse con el homomorfismo $\phi:A\longrightarrow B$) tal que:
    \begin{enumerate}
        \item $a \longmapsto \phi(a) \hspace{1cm} \forall a \in A$
        \item $x \longmapsto x$
    \end{enumerate}

    Este está definido por:
    $$\phi\left( \sum_{i=0}^n a_i x^i \right) = \sum_{i=0}^n \phi(a_i) x^i$$
    
    Llamaremos a este homomorfismo $\phi:A[x]\longrightarrow B[x]$ el \textbf{homomorfismo inducido en los anillos de polinomios}.
\end{teo}
\begin{proof}
    Este resultado es un corolario directo de la Propiedad Universal del Anillo de Polinomios aplicado al homomorfismo $i\circ \phi$, donde $i$ es la inclusión. Veámoslo:
    $$A \mathop{\longrightarrow}^{\phi} B \mathop{\longrightarrow}^i B[x] \hspace{1cm} i \circ \phi:A\longrightarrow B[x]$$

    Como la composición de homomorfismos es un homomorfismo, tenemos que $i \circ \phi$ es un homomorfismo de anillos. Entonces, considerando $b = x \in B[x]$, la Propiedad Universal del Anillo de Polinomios aplicada a $i \circ \phi$ nos afirma que existe un único homomorfismo tal que:
    \begin{enumerate}
        \item $(i\circ \phi)_x(a) \AstIg (i\circ \phi)(a) = i(\phi(a))=\phi(a)$
        \item $(i\circ \phi)_x(x) \AstIg x$
    \end{enumerate}
    donde en $(\ast)$ he aplicado la Propiedad Universal del Anillo de Polinomios.

    Además, también afirma que dicho homomorfismo se define $\forall f=\sum_{i=0}^n a_ix^i$ por:
    \begin{equation*}
        (i\circ \phi)_x(0) = 0 \hspace{1cm}
        (i\circ \phi)_x(f) = \sum_{i=0}^n (i\circ \phi)(a_i)x^i = \sum_{i=0}^n \phi(a_i)x^i
    \end{equation*}
    por lo que demuestra directamente el resultado enunciado.
\end{proof}

\begin{ejemplo}\
    \begin{enumerate}
        \item Consideramos el homomorfismo $R_2:\Z\longrightarrow \Z_2 \mid R_2(a) = R(a;2)~~\forall a \in \Z$.
        
        $R_2:\Z[x]\longrightarrow \Z_2[x]$ es el homomorfismo inducido en los anillos de polinomios. Por ser un homomorfismo:
        \begin{equation*}
            \begin{split}
                &R_2(5+8x^2 + 3x^6) = 1 + x^6 \\
                &R_2((7x^2+3)(5x+1)+7) = R_2(7x^2+3)R_2(5x+1)+R_2(7) = \\
                & \hspace{2cm} = (x^2+1)(x+1)+1 = x^3+x^2+x
            \end{split}
        \end{equation*}

        \item Para $n \geq 2$, tenemos $R_n:\Z\longrightarrow \Z_n \mid R_n(a) = R(a;n) \hspace{1cm} \forall a \in \Z$.
        
        $R_n:\Z[x]\longrightarrow \Z_n[x]$ es el homomorfismo inducido por $R_n$ en los anillos de polinomios.
        $$R_n\left( \sum_{i=0}^k a_ix^i \right) = \sum_{i=0}^k R_n(a_i)x^i = \sum_{i=0}^k R(a;n)x^i$$
    \end{enumerate}
\end{ejemplo}
