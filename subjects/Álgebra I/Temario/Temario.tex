\documentclass[12pt]{book}

% Idioma y codificación
\usepackage[spanish, es-tabla]{babel}       %es-tabla para que se titule "Tabla"
\usepackage[utf8]{inputenc}

% Márgenes
\usepackage[a4paper,top=3cm,bottom=2.5cm,left=3cm,right=3cm]{geometry}

% Comentarios de bloque
\usepackage{verbatim}

% Paquetes de links
\usepackage[hidelinks]{hyperref}    % Permite enlaces
\usepackage{url}                    % redirecciona a la web

% Más opciones para enumeraciones
\usepackage{enumitem}

% Personalizar la portada
\usepackage{titling}

% Paquetes de tablas
\usepackage{multirow}


%------------------------------------------------------------------------

%Paquetes de figuras
\usepackage{caption}
\usepackage{subcaption} % Figuras al lado de otras
\usepackage{float}      % Poner figuras en el sitio indicado H.


% Paquetes de imágenes
\usepackage{graphicx}       % Paquete para añadir imágenes
\usepackage{transparent}    % Para manejar la opacidad de las figuras

% Paquete para usar colores
\usepackage[dvipsnames]{xcolor}
\usepackage{pagecolor}      % Para cambiar el color de la página

% Habilita tamaños de fuente mayores
\usepackage{fix-cm}

% Para los gráficos
\usepackage{tikz}

% Para poder situar los nodos en los grafos
\usetikzlibrary{positioning}


%------------------------------------------------------------------------

% Paquetes de matemáticas
\usepackage{mathtools, amsfonts, amssymb, mathrsfs}
\usepackage[makeroom]{cancel}     % Simplificar tachando
\usepackage{polynom}    % Divisiones y Ruffini
\usepackage{units} % Para poner fracciones diagonales con \nicefrac

\usepackage{pgfplots}   %Representar funciones
\pgfplotsset{compat=1.18}  % Versión 1.18

\usepackage{tikz-cd}    % Para usar diagramas de composiciones
\usetikzlibrary{calc}   % Para usar cálculo de coordenadas en tikz

%Definición de teoremas, etc.
\usepackage{amsthm}
%\swapnumbers   % Intercambia la posición del texto y de la numeración

\theoremstyle{plain}

\makeatletter
\@ifclassloaded{article}{
  \newtheorem{teo}{Teorema}[section]
}{
  \newtheorem{teo}{Teorema}[chapter]  % Se resetea en cada chapter
}
\makeatother

\newtheorem{coro}{Corolario}[teo]           % Se resetea en cada teorema
\newtheorem{prop}[teo]{Proposición}         % Usa el mismo contador que teorema
\newtheorem{lema}[teo]{Lema}                % Usa el mismo contador que teorema

\theoremstyle{remark}
\newtheorem*{observacion}{Observación}

\theoremstyle{definition}

\makeatletter
\@ifclassloaded{article}{
  \newtheorem{definicion}{Definición} [section]     % Se resetea en cada chapter
}{
  \newtheorem{definicion}{Definición} [chapter]     % Se resetea en cada chapter
}
\makeatother

\newtheorem*{notacion}{Notación}
\newtheorem*{ejemplo}{Ejemplo}
\newtheorem*{ejercicio*}{Ejercicio}             % No numerado
\newtheorem{ejercicio}{Ejercicio} [section]     % Se resetea en cada section


% Modificar el formato de la numeración del teorema "ejercicio"
\renewcommand{\theejercicio}{%
  \ifnum\value{section}=0 % Si no se ha iniciado ninguna sección
    \arabic{ejercicio}% Solo mostrar el número de ejercicio
  \else
    \thesection.\arabic{ejercicio}% Mostrar número de sección y número de ejercicio
  \fi
}


% \renewcommand\qedsymbol{$\blacksquare$}         % Cambiar símbolo QED
%------------------------------------------------------------------------

% Paquetes para encabezados
\usepackage{fancyhdr}
\pagestyle{fancy}
\fancyhf{}

\newcommand{\helv}{ % Modificación tamaño de letra
\fontfamily{}\fontsize{12}{12}\selectfont}
\setlength{\headheight}{15pt} % Amplía el tamaño del índice


%\usepackage{lastpage}   % Referenciar última pag   \pageref{LastPage}
\fancyfoot[C]{\thepage}

%------------------------------------------------------------------------

% Conseguir que no ponga "Capítulo 1". Sino solo "1."
\makeatletter
\@ifclassloaded{book}{
  \renewcommand{\chaptermark}[1]{\markboth{\thechapter.\ #1}{}} % En el encabezado
    
  \renewcommand{\@makechapterhead}[1]{%
  \vspace*{50\p@}%
  {\parindent \z@ \raggedright \normalfont
    \ifnum \c@secnumdepth >\m@ne
      \huge\bfseries \thechapter.\hspace{1em}\ignorespaces
    \fi
    \interlinepenalty\@M
    \Huge \bfseries #1\par\nobreak
    \vskip 40\p@
  }}
}
\makeatother

%------------------------------------------------------------------------
% Paquetes de cógido
\usepackage{minted}
\renewcommand\listingscaption{Código fuente}

\usepackage{fancyvrb}
% Personaliza el tamaño de los números de línea
\renewcommand{\theFancyVerbLine}{\small\arabic{FancyVerbLine}}

% Estilo para C++
\newminted{cpp}{
    frame=lines,
    framesep=2mm,
    baselinestretch=1.2,
    linenos,
    escapeinside=||
}

% para minted
\definecolor{LightGray}{rgb}{0.95,0.95,0.92}
\setminted{
    linenos=true,
    stepnumber=5,
    numberfirstline=true,
    autogobble,
    breaklines=true,
    breakautoindent=true,
    breaksymbolleft=,
    breaksymbolright=,
    breaksymbolindentleft=0pt,
    breaksymbolindentright=0pt,
    breaksymbolsepleft=0pt,
    breaksymbolsepright=0pt,
    fontsize=\footnotesize,
    bgcolor=LightGray,
    numbersep=10pt
}


\usepackage{listings} % Para incluir código desde un archivo

\renewcommand\lstlistingname{Código Fuente}
\renewcommand\lstlistlistingname{Índice de Códigos Fuente}

% Definir colores
\definecolor{vscodepurple}{rgb}{0.5,0,0.5}
\definecolor{vscodeblue}{rgb}{0,0,0.8}
\definecolor{vscodegreen}{rgb}{0,0.5,0}
\definecolor{vscodegray}{rgb}{0.5,0.5,0.5}
\definecolor{vscodebackground}{rgb}{0.97,0.97,0.97}
\definecolor{vscodelightgray}{rgb}{0.9,0.9,0.9}

% Configuración para el estilo de C similar a VSCode
\lstdefinestyle{vscode_C}{
  backgroundcolor=\color{vscodebackground},
  commentstyle=\color{vscodegreen},
  keywordstyle=\color{vscodeblue},
  numberstyle=\tiny\color{vscodegray},
  stringstyle=\color{vscodepurple},
  basicstyle=\scriptsize\ttfamily,
  breakatwhitespace=false,
  breaklines=true,
  captionpos=b,
  keepspaces=true,
  numbers=left,
  numbersep=5pt,
  showspaces=false,
  showstringspaces=false,
  showtabs=false,
  tabsize=2,
  frame=tb,
  framerule=0pt,
  aboveskip=10pt,
  belowskip=10pt,
  xleftmargin=10pt,
  xrightmargin=10pt,
  framexleftmargin=10pt,
  framexrightmargin=10pt,
  framesep=0pt,
  rulecolor=\color{vscodelightgray},
  backgroundcolor=\color{vscodebackground},
}

%------------------------------------------------------------------------

% Comandos definidos
\newcommand{\bb}[1]{\mathbb{#1}}
\newcommand{\cc}[1]{\mathcal{#1}}

% I prefer the slanted \leq
\let\oldleq\leq % save them in case they're every wanted
\let\oldgeq\geq
\renewcommand{\leq}{\leqslant}
\renewcommand{\geq}{\geqslant}

% Si y solo si
\newcommand{\sii}{\iff}

% Letras griegas
\newcommand{\eps}{\epsilon}
\newcommand{\veps}{\varepsilon}
\newcommand{\lm}{\lambda}

\newcommand{\ol}{\overline}
\newcommand{\ul}{\underline}
\newcommand{\wt}{\widetilde}
\newcommand{\wh}{\widehat}

\let\oldvec\vec
\renewcommand{\vec}{\overrightarrow}

% Derivadas parciales
\newcommand{\del}[2]{\frac{\partial #1}{\partial #2}}
\newcommand{\Del}[3]{\frac{\partial^{#1} #2}{\partial #3^{#1}}}
\newcommand{\deld}[2]{\dfrac{\partial #1}{\partial #2}}
\newcommand{\Deld}[3]{\dfrac{\partial^{#1} #2}{\partial #3^{#1}}}


\newcommand{\AstIg}{\stackrel{(\ast)}{=}}
\newcommand{\Hop}{\stackrel{L'H\hat{o}pital}{=}}

\newcommand{\red}[1]{{\color{red}#1}} % Para integrales, destacar los cambios.

% Método de integración
\newcommand{\MetInt}[2]{
    \left[\begin{array}{c}
        #1 \\ #2
    \end{array}\right]
}

% Declarar aplicaciones
% 1. Nombre aplicación
% 2. Dominio
% 3. Codominio
% 4. Variable
% 5. Imagen de la variable
\newcommand{\Func}[5]{
    \begin{equation*}
        \begin{array}{rrll}
            #1:& #2 & \longrightarrow & #3\\
               & #4 & \longmapsto & #5
        \end{array}
    \end{equation*}
}

%------------------------------------------------------------------------


%Definiciones para ahorrar trabajo
\def\R{\mathbb R}
\def\C{\mathbb C}
\def\N{\mathbb N}
\def\Z{\mathbb Z}
\def\Q{\mathbb Q}
\def\B{\mathcal{B}}
\def\A{\mathcal{A}}
\def\L{\mathcal{L}}
\def\y{\ \land \ }
\def\o{\ \lor \ }


\def\todoi{\forall i \in \{1, \ldots, n\}}
\def\todon{\ \forall n \in \N}

\usepackage{xlop}       % Para hacer cálculos básicos. Divisiones, etc.
\opset{mulsymbol={$\cdot$}}

\usepackage{extarrows}
\usepackage{stackrel}
\usetikzlibrary{matrix} % Para divisiones de polinomios.


\DeclareMathOperator{\mcd}{mcd}
\DeclareMathOperator{\mcm}{mcm}

\newcommand{\resetearcontador}{%
  \setcounter{ejercicio}{0}% Resetea el contador de ejercicios a 0
}
\begin{document}

    % 1. Foto de fondo
    % 2. Título
    % 3. Encabezado Izquierdo
    % 4. Color de fondo
    % 5. Coord x del titulo
    % 6. Coord y del titulo
    % 7. Fecha
    % 8. Autor

    
    % 1. Foto de fondo
% 2. Título
% 3. Encabezado Izquierdo
% 4. Color de fondo
% 5. Coord x del titulo
% 6. Coord y del titulo
% 7. Fecha

\newcommand{\portada}[7]{

    \portadaBase{#1}{#2}{#3}{#4}{#5}{#6}{#7}
    \portadaBook{#1}{#2}{#3}{#4}{#5}{#6}{#7}
}

\newcommand{\portadaExamen}[7]{

    \portadaBase{#1}{#2}{#3}{#4}{#5}{#6}{#7}
    \portadaArticle{#1}{#2}{#3}{#4}{#5}{#6}{#7}
}




\newcommand{\portadaBase}[7]{

    % Tiene la portada principal y la licencia Creative Commons
    
    % 1. Foto de fondo
    % 2. Título
    % 3. Encabezado Izquierdo
    % 4. Color de fondo
    % 5. Coord x del titulo
    % 6. Coord y del titulo
    % 7. Fecha
    
    
    \thispagestyle{empty}               % Sin encabezado ni pie de página
    \newgeometry{margin=0cm}        % Márgenes nulos para la primera página
    
    
    % Encabezado
    \fancyhead[L]{\helv #3}
    \fancyhead[R]{\helv \nouppercase{\leftmark}}
    
    
    \pagecolor{#4}        % Color de fondo para la portada
    
    \begin{figure}[p]
        \centering
        \transparent{0.3}           % Opacidad del 30% para la imagen
        
        \includegraphics[width=\paperwidth, keepaspectratio]{assets/#1}
    
        \begin{tikzpicture}[remember picture, overlay]
            \node[anchor=north west, text=white, opacity=1, font=\fontsize{60}{90}\selectfont\bfseries\sffamily, align=left] at (#5, #6) {#2};
            
            \node[anchor=south east, text=white, opacity=1, font=\fontsize{12}{18}\selectfont\sffamily, align=right] at (9.7, 3) {\textbf{\href{https://losdeldgiim.github.io/}{Los Del DGIIM}}};
            
            \node[anchor=south east, text=white, opacity=1, font=\fontsize{12}{15}\selectfont\sffamily, align=right] at (9.7, 1.8) {Doble Grado en Ingeniería Informática y Matemáticas\\Universidad de Granada};
        \end{tikzpicture}
    \end{figure}
    
    
    \restoregeometry        % Restaurar márgenes normales para las páginas subsiguientes
    \pagecolor{white}       % Restaurar el color de página
    
    
    \newpage
    \thispagestyle{empty}               % Sin encabezado ni pie de página
    \begin{tikzpicture}[remember picture, overlay]
        \node[anchor=south west, inner sep=3cm] at (current page.south west) {
            \begin{minipage}{0.5\paperwidth}
                \href{https://creativecommons.org/licenses/by-nc-nd/4.0/}{
                    \includegraphics[height=2cm]{assets/Licencia.png}
                }\vspace{1cm}\\
                Esta obra está bajo una
                \href{https://creativecommons.org/licenses/by-nc-nd/4.0/}{
                    Licencia Creative Commons Atribución-NoComercial-SinDerivadas 4.0 Internacional (CC BY-NC-ND 4.0).
                }\\
    
                Eres libre de compartir y redistribuir el contenido de esta obra en cualquier medio o formato, siempre y cuando des el crédito adecuado a los autores originales y no persigas fines comerciales. 
            \end{minipage}
        };
    \end{tikzpicture}
    
    
    
    % 1. Foto de fondo
    % 2. Título
    % 3. Encabezado Izquierdo
    % 4. Color de fondo
    % 5. Coord x del titulo
    % 6. Coord y del titulo
    % 7. Fecha


}


\newcommand{\portadaBook}[7]{

    % 1. Foto de fondo
    % 2. Título
    % 3. Encabezado Izquierdo
    % 4. Color de fondo
    % 5. Coord x del titulo
    % 6. Coord y del titulo
    % 7. Fecha

    % Personaliza el formato del título
    \pretitle{\begin{center}\bfseries\fontsize{42}{56}\selectfont}
    \posttitle{\par\end{center}\vspace{2em}}
    
    % Personaliza el formato del autor
    \preauthor{\begin{center}\Large}
    \postauthor{\par\end{center}\vfill}
    
    % Personaliza el formato de la fecha
    \predate{\begin{center}\huge}
    \postdate{\par\end{center}\vspace{2em}}
    
    \title{#2}
    \author{\href{https://losdeldgiim.github.io/}{Los Del DGIIM}}
    \date{Granada, #7}
    \maketitle
    
    \tableofcontents
}




\newcommand{\portadaArticle}[7]{

    % 1. Foto de fondo
    % 2. Título
    % 3. Encabezado Izquierdo
    % 4. Color de fondo
    % 5. Coord x del titulo
    % 6. Coord y del titulo
    % 7. Fecha

    % Personaliza el formato del título
    \pretitle{\begin{center}\bfseries\fontsize{42}{56}\selectfont}
    \posttitle{\par\end{center}\vspace{2em}}
    
    % Personaliza el formato del autor
    \preauthor{\begin{center}\Large}
    \postauthor{\par\end{center}\vspace{3em}}
    
    % Personaliza el formato de la fecha
    \predate{\begin{center}\huge}
    \postdate{\par\end{center}\vspace{5em}}
    
    \title{#2}
    \author{\href{https://losdeldgiim.github.io/}{Los Del DGIIM}}
    \date{Granada, #7}
    \thispagestyle{empty}               % Sin encabezado ni pie de página
    \maketitle
    \vfill
}
    \portada{ffccA4.jpg}{Álgebra I}{Álgebra I}{MidnightBlue}{-8}{28}{2023}{José Juan Urrutia Milán ``JJ''\\Arturo Olivares Martos}
    
    \chapter{El lenguaje de los conjuntos}
En este primer tema, abordaremos un desarrollo sencillo de la teoría de conjuntos, basado en los axiomas de Zermelo-Fraenkel. El lector puede adentrarse en este campo gracias al libro \emph{Naive Set Theory}, de Paul Halmos, cuya lectura recomendamos. En este documento no haremos un desarrollo tan exhaustivo desde la axiomática por falta de tiempo. Es por tanto que daremos al principio algunas definiciones basadas en la intuición del matemático que inicia este curso.

\section{Conceptos básicos}

\begin{definicion}[Conjunto]
    Llamaremos \textbf{conjunto} a una colección de objetos (a los que también llamaremos \textbf{elementos})
    en la que no influye el orden.
\end{definicion}
\begin{notacion}
    Usualmente, notaremos a los conjuntos con letras mayúsculas y a los elementos con letras minúsculas, pudiendo haciendo uso incluso de letras griegas.
\end{notacion}

Si $A$ es un conjunto y $a$ es un elemento suyo, diremos que \textit{$a$ pertenece a $A$}, notado $a \in A$; mientras que si $a$ no es un elemento de $A$, diremos que \textit{$a$ no pertenece a $A$}, notado $a \notin A$.\\

A la hora de definir un conjunto, es posible hacerlo por \textit{extensión}, proporcionando todos sus elementos; o por \textit{comprensión}, proporcionando una regla que cumplan todos los elementos que pertenecen al conjunto. Por ejemplo, las siguientes definiciones son equivalentes:
\begin{gather*}
    X = \{0, 1, 2, 3, 4, 5\} \\
    X = \{x \mid x \in \N \y x < 6\}
\end{gather*}

Si $X$ es un conjunto finito con $n$ elementos $a_1, a_2, \ldots, a_n$, es habitual escribir: $$X = \{a_1, a_2, \ldots, a_n\} = \{a_i \mid 1 \leq i \leq n\} = \{a_i\}_{i = 1, \ldots, n}$$

\begin{definicion}[Cardinal]
    Al número $n$ de elementos de un conjunto le llamaremos \textbf{cardinal del conjunto}. Si $X$
    es un conjunto, notaremos a su cardinal por $|X|$ ó por $\#X$:
    $$|X| = \#X = n$$
\end{definicion}

Diremos que dos conjuntos $X$ e $Y$ son iguales (notado $X = Y$) si tienen los mismos elementos, ya que un conjunto está totalmente definido por sus elementos. Por otra parte, si $\exists \ x \in X$ tal que $x \notin Y$, o bien $\exists \ y \in Y$ tal que $y \notin X$, diremos que $X$ e $Y$ son distintos: $X \neq Y$.\\

Además, admitimos la existencia de un conjunto vacío (notado $\emptyset$), como aquel conjunto con cardinal 0 ($|\emptyset| = 0$). Es decir, $\emptyset$ no tiene elementos, luego nunca será posible encontrar un elemento $x$ de forma que\footnote{Esta observación no parece de mucha relevancia, pero gran número de demostraciones se basan en llegar a contradicción viendo que un elemento pertenece al conjunto vacío. Se dice que son demostraciones ``por vacuidad''.} $x\in \emptyset $.

\begin{definicion}[Subconjunto]
    Dados dos conjuntos $X$ e $Y$, diremos que \textbf{$X$ es un subconjunto de $Y$} si todo elemento de $X$ es también un elemento de $Y$. Es decir:
    $$\forall x \in X \Rightarrow x \in Y$$
    
    Lo notaremos como $X \subseteq Y$. En dicho caso, podremos decir también que \textbf{$X$ está contenido en $Y$}.
\end{definicion}

Algunas consecuencias inmediatas de fácil comprobación son:
\begin{itemize}
    \item $X = Y \Longleftrightarrow X \subseteq Y \y Y \subseteq X$.
    \item $\emptyset \subseteq X$ para todo conjunto $X$.
    \item $X \subseteq X$ para todo conjunto $X$.
\end{itemize}
\begin{notacion}
    La notación $X \subset Y$ es equivalente a la de $X \subseteq Y$.
\end{notacion}

\begin{definicion}[Subconjunto propio]
    Dado un conjunto $Y$, si $X \neq \emptyset$ es un conjunto tal que se tiene $X \subseteq Y \y X \neq Y$ diremos que $X$ es un subconjunto propio de $Y$.
    Es decir, $X$ es un subconjunto propio de $Y$ si:
    \begin{enumerate}
        \item $\forall x \in X \Rightarrow x \in Y$
        \item $\exists y \in Y \mid y \notin X$
    \end{enumerate}
    
    En dicho caso, lo notaremos por $X \subsetneq Y$.
\end{definicion}
Notemos que los únicos subconjuntos no propios de un conjunto $X$ son $X$ y $\emptyset $.

\begin{definicion}[Partes de un conjunto]
    Dado cualquier conjunto $X$, podremos formar un nuevo conjunto, que notaremos como $\cc{P}(X)$ y llamaremos \textbf{conjunto partes de $X$} ó \textbf{conjunto potencia de $X$} al conjunto cuyos elementos son cada uno de los posibles subconjuntos de $X$ que podamos formar:
    $$\cc{P}(X) = \{A \mid A \subseteq X\}$$
\end{definicion}
De la definición, se deduce que $\emptyset,X\in \cc{P}(X)$ para todo conjunto $X$.

\begin{ejemplo} Algunos ejemplos del conjunto de las partes de $X$ son:
\begin{enumerate}
    \item $X = \{1, 2, 3\}$
    $$\cc{P}(X) = \{\emptyset, \{1\}, \{2\}, \{3\}, \{1, 2\}, \{1, 3\}, \{2, 3\}, X\}$$

    \item $X=\emptyset$
    \begin{gather*}
        \cc{P}(\emptyset)=\{\emptyset\}\\
        \cc{P}[\cc{P}(\emptyset)] = \cc{P}[\{\emptyset\}] = \{\emptyset, \{\emptyset\}\}
    \end{gather*}
\end{enumerate}
\end{ejemplo}

Notemos que, dado un conjunto $X$, el conjunto $\mathcal{P}(X)$ es el primer ejemplo de conjunto que a su vez contiene a conjuntos. El alumno puede llegar a confundirse con qué notación usar en cada caso. El siguiente ejemplo muestra un caso básico de la notación que debemos usar al trabajar con distintos tipos de elementos matemáticos.

\begin{ejemplo}
    Si $X$ es un conjunto, $x$ es un elemento suyo ($x\in X$) y consideramos el conjunto partes de $X$, $\mathcal{P}(X)$. Podremos escribir:
    \begin{equation*}
        x\in X \qquad \{x\} \subseteq X \qquad \{x\} \in  \mathcal{P}(X) \qquad X \in \mathcal{P}(X)
    \end{equation*}
    Pero \textbf{no} podremos escribir:
    \begin{equation*}
        \{x\} \in X \qquad x \subseteq X \qquad \{x\} \subseteq \mathcal{P}(X) \qquad X \subseteq \mathcal{P}(X)
    \end{equation*}
\end{ejemplo}

Durante la carrera de matemáticas se verán numerosos ejemplos de conjuntos que a su vez contienen a conjuntos (y dichos conjuntos quizás contendrán otros conjuntos), basta considerar el conjunto $\mathcal{P}(\mathcal{P}(X))$ para cualquier conjunto $X$. 

Será usual denotar por ``familia'' a los conjuntos cuyos elementos son a su vez conjuntos. Como notación para estos, se suelen usar letras estilográficas ($\cc{A}, \cc{B}, \ldots$), o en ocasiones por letras griegas mayúsculas, aunque siempre podremos saber la naturaleza del conjunto gracias a cómo esté definido.

\begin{definicion}[Intersección]
    Sea $X$ un conjunto y sean $A, B \in \cc{P}(X)$, definimos la \textbf{intersección} de $A$ y de $B$, notado
    $A \cap B$ como el subconjunto de $X$ formado por aquellos elementos que pertenecen simultáneamente
    a $A$ y a $B$:
    $$A \cap B = \{x \in X \mid x \in A \y x \in B\}$$
\end{definicion}

\begin{definicion}[Unión]
    Sea $X$ un conjunto y sean $A, B \in \cc{P}(X)$, definimos la \textbf{unión} de $A$ y de $B$, notado
    $A \cup B$ como el subconjunto de $X$ formado por aquellos elementos que pertenecen a $A$ o a $B$:
    $$A \cup B = \{x \in X \mid x \in A \o x \in B\}$$
\end{definicion}

Cuando el conjunto $X$ esté claro por el contexto podremos mencionar simplemente la intersección o unión de dos conjuntos, sin determinar de forma explícita el conjunto $X$ del que ambos son subconjuntos.

\begin{definicion}[Disjuntos]
    Sea $X$ un conjunto y sean $A, B \in \cc{P}(X)$, diremos que \textbf{$A$ y $B$ son disjuntos} si
    $A \cap B = \emptyset$.
\end{definicion}

\begin{prop}
    Sea $X$ un conjunto y $A, B, C \in \cc{P}(X)$. Algunas de las propiedades que se verifican sobre conjuntos son:
    \begin{enumerate}
        \item Propiedad conmutativa:
        $$A \cap B = B \cap A \quad;\quad A \cup B = B \cup A$$
        \item Propiedad asociativa:
        $$A \cap (B \cap C) = (A \cap B) \cap C \quad;\quad A \cup (B \cup C) = (A \cup B) \cup C$$
        \item Propiedad de la idempotencia:
        $$A \cap A = A \quad;\quad A \cup A = A$$
        \item Propiedad distributiva:
        $$A \cap (B \cup C) = (A \cap B) \cup (A \cap C) \quad;\quad A \cup (B \cap C) = (A \cup B) \cap (A \cup C)$$
    \end{enumerate}
\end{prop}
\begin{proof}
    Demostramos cada una de las propiedades por separado:
    \begin{enumerate}
        \item Propiedad conmutativa:
        \begin{gather*}
            A \cap B = \{x \in X \mid x \in A \y x \in B \} = \{x \in X \mid x \in B \y x \in A \} = B \cap A \\
            A \cup B = \{x \in X \mid x \in A \o x \in B\} = \{x \in X \mid x \in B \o x \in A\} = B \cup A
        \end{gather*}
        \item Propiedad asociativa:
        \begin{equation*}
            \begin{split}
                A \cap (B \cap C) &= A \cap \{x \in X \mid x \in B \y x \in C \} =\\
                & =\{x \in X \mid x \in A \y x \in B \y x \in C\} = \\
                & =\{x \in X \mid x \in A \y x \in B\} \cap C =\\
                & = (A \cap B) \cap C
            \end{split}
        \end{equation*}
        \begin{equation*}
            \begin{split}
                A \cup (B \cup C) &= A \cup \{x \in X \mid x \in B \o x \in C \} =\\
                & =\{x \in X \mid x \in A \o x \in B \o x \in C\} =\\
                & =\{x \in X \mid x \in A \o x \in B\} \cup C =\\
                & =(A \cup B) \cup C
            \end{split}
        \end{equation*}
        
        \item Propiedad de la idempotencia:
        \begin{gather*}
            A \cap A = \{x \in X \mid x \in A \y x \in A\} = \{x \in X \mid x \in A\} = A \\
            A \cup A = \{x \in X \mid x \in A \o x \in A\} = \{x \in X \mid x \in A\} = A
        \end{gather*}
        \item Propiedad distributiva:
        \begin{equation*}
            \begin{split}
                A \cap (B \cup C) &= A \cap \{x \in X \mid x \in B \o x \in C\} =\\
                & =\{x \in X \mid x \in A \y (x \in B \o x \in C)\} =\\
                & =\{x \in X \mid (x \in A \y x \in B) \o (x \in A \y x \in C)\} =\\
                & = \{x \in X \mid x \in A \y x \in B\} \cup \{x \in X \mid x \in A \y x \in B\} =\\
                & = (A \cap B) \cup (A \cap C)
            \end{split}
        \end{equation*}
        \begin{equation*}
            \begin{split}
                A \cup (B \cap C) &= A \cup \{x \in X \mid x \in B \y x \in C\} =\\
                & =\{x \in X \mid x \in A \o (x \in B \y x \in C)\} =\\
                & =\{x \in X \mid (x \in A \o x \in B) \y (x \in A \o x \in C)\} =\\
                &= \{x \in X \mid x \in A \o x \in B\} \cap \{x \in X \mid x \in A \o x \in B\} =\\
                &= (A \cup B) \cap (A \cup C)
            \end{split}
        \end{equation*}
        \qedhere
    \end{enumerate}
\end{proof}

\begin{definicion}[Uniones e intersecciones generalizadas]
    Sea $X$ un conjunto, y consideramos $\Gamma \subseteq \cc{P}(X)$ una familia de subconjuntos de $X$. Definimos la unión y la intersección de todos los elementos de $\Gamma$ por:
    \begin{gather*}
        \bigcap_{A \in \Gamma} A = \{x \in X \mid x \in A~~\forall A \in \Gamma \}\\
        \bigcup_{A \in \Gamma} A = \{x \in X \mid \exists A \in \Gamma\ \text{tal que}\ x \in A \}
    \end{gather*}

    A veces, simplemente lo notaremos por:
    \begin{gather*}
        \bigcap \Gamma := \bigcap_{A\in \Gamma}A \\
        \bigcup \Gamma := \bigcup_{A\in \Gamma}A
    \end{gather*}

    Notemos que si $\Gamma$ es una familia finita: $\Gamma = \{A_1, A_2, \ldots, A_n\} \subseteq \cc{P}(X)$, entonces:
    \begin{equation*}
        \bigcap_{A \in \Gamma}A = A_1 \cap A_2 \cap \ldots \cap A_n
        \hspace{1cm}
        \bigcup_{A \in \Gamma}A = A_1 \cup A_2 \cup \ldots \cup A_n
    \end{equation*}
    
    En el caso anterior, podemos notar:
    \begin{equation*}
        \bigcap_{A \in \Gamma}A = \bigcap_{i=1}^n A_i
        \hspace{1cm}
        \bigcup_{A \in \Gamma}A = \bigcup_{i=1}^n A_i
    \end{equation*}
\end{definicion}

\begin{ejemplo}
Sea $X = \{0, 1, 2, 3, 4, 5\}$, y consideramos ${\Gamma = \{\{0, 1\}, \{1, 2\}, \{1, 3, 5\}\}\subseteq \mathcal{P}(X)}$:
    \begin{equation*}
        \bigcap_{A \in \Gamma}A = \{1\}
        \hspace{1cm}
        \bigcup_{A \in \Gamma}A = \{0, 1, 2, 3, 5\}
    \end{equation*}
\end{ejemplo}

\begin{definicion}[Complementario]
    Sea $X$ un conjunto y $A \in \cc{P}(X)$, definimos el \textbf{complementario de $A$ en $X$}, notado $X-A$ o $X\setminus A$, como el subconjunto de $X$ formado por aquellos elementos de $X$ que no pertenezcan a $A$:
    $$X-A = \{x \in X \mid x \notin A\}$$
\end{definicion}
\begin{notacion}
    Cuando el conjunto $X$ sea claro por el contexto (por ejemplo, cuando estemos trabajando continuamente con números reales), notaremos simplemente $\overline{A}$ o $C(A)$ (que será equivalente a escribir $X-A$).
\end{notacion}

\begin{ejemplo}
    Sea $A = \{x \in \N \mid x \geq 4\} \subseteq \N$:
    \begin{align*}
        \N - A &= \{0, 1, 2, 3\} = \{x \in \N \mid x < 4\}\\
        \Z - A &= \{x \in \Z \mid x < 4\}
    \end{align*}
\end{ejemplo}

\begin{prop} Sea $X$ un conjunto y $A \in \cc{P}(X)$. Algunas propiedades que se verifican sobre el complementario son:
    \begin{enumerate}
        \item $C(\emptyset) = X$.
        \item $C(X) = \emptyset$.
        \item $A \cup C(A) = X$.
        \item $A \cap C(A) = \emptyset$.
        \item $C(C(A)) = A$.
    \end{enumerate}
\end{prop}
\begin{proof}
    Demostramos cada una de las propiedades por separado:
    \begin{enumerate}
        \item $C(\emptyset) = \{x \in X \mid x \notin \emptyset\} = \{x \in X\} = X$.
        \item $C(X) = \{x \in X \mid x \notin X\} = \emptyset$.
        \item $A \cup C(A) = \{x \in X \mid x \in A \o x \notin A \} = \{x \in X\} = X$.
        \item $A \cap C(A) = \{x \in X \mid x \in A \y x \notin A\} = \emptyset$.
        \item $C(C(A)) = \{x \in X \mid x \notin C(A) \} = \{x \in X \mid x \in A\} = A$.
    \end{enumerate}
\end{proof}

\begin{prop}[Leyes de De Morgan]
    Sea $X$ un conjunto con $A, B \in \cc{P}(X)$, se verifica que:
    \begin{enumerate}
        \item $C(A \cup B) = C(A) \cap C(B)$
        \item $C(A \cap B) = C(A) \cup C(B)$
    \end{enumerate}
\end{prop}
\begin{proof} Demostramos cada una de las igualdades:
\begin{enumerate}
    \item $C(A \cup B) = C(A) \cap C(B)$:
    \begin{equation*}
        \begin{split}
            C(A \cup B) &= \{x \in X \mid x \notin (A \cup B)\} =\\
            & =\{x \in X \mid x \notin \{x \in X \mid x \in A \o x \in B\} \} =\\
            & = \{x \in X \mid x \notin A \y x \notin B \} =\\
            & =\{x \in X \mid x \notin A\} \cap \{x \in X \mid x \notin B\} = C(A) \cap C(B)
        \end{split}
    \end{equation*}

    \item $C(A \cap B) = C(A) \cup C(B)$:
    \begin{equation*}
        \begin{split}
            C(A \cap B) &= \{x \in X \mid x \notin (A \cap B)\} =\\
            & =\{x \in X \mid x \notin \{x \in X \mid x \in A \y x \in B\} \} =\\
            & = \{x \in X \mid x \notin A \o x \notin B \} =\\
            & =\{x \in X \mid x \notin A\} \cup \{x \in X \mid x \notin B\} = C(A) \cup C(B)
        \end{split}
    \end{equation*}
\end{enumerate}
\end{proof}

\begin{prop}[Leyes de De Morgan generalizadas]
    Sea $X$ un conjunto, $\Gamma \subseteq~\cc{P}(X)$, se verifica:
    \begin{enumerate}
        \item El complementario de la unión es la intersección de los complementarios.
        $$C\left( \bigcup_{A\in\Gamma}A \right) = \bigcap_{A\in\Gamma}C(A)$$

        \item El complementario de la intersección es la unión de los complementarios.
        $$C\left( \bigcap_{A\in\Gamma}A \right) = \bigcup_{A\in\Gamma}C(A)$$
    \end{enumerate}
\end{prop}
\begin{proof} Demostramos cada igualdad por separado:
\begin{enumerate}
    \item $C\left( \bigcup\limits_{A\in\Gamma}A \right) = \left\{x \in X \mid x \notin \bigcup\limits_{A\in\Gamma}A\right\} = \{x \in X \mid x \notin A ~\forall A \in \Gamma \} = \bigcap\limits_{A\in\Gamma}C(A)$

    \item $C\left( \bigcap\limits_{A\in\Gamma}A \right) = \left\{ x \in X \mid x \notin \bigcap\limits_{A\in\Gamma}A \right\} = \{x \in X \mid \exists A \in \Gamma \mid x \notin A\} = \bigcup\limits_{A\in\Gamma}C(A)$
\end{enumerate}
\end{proof}

\begin{definicion}[Complementario generalizado]
    Sea $X$ un conjunto y consideramos $A, B \in \cc{P}(X)$, definimos \textbf{el complementario de $A$ en $B$}, notado
    $B-A$ como el conjunto:
    $$B-A = \{x \in X \mid x \in B \y x \notin A\} = B \cap C(A)$$
\end{definicion}

\begin{prop}[Propiedad distributiva generalizada]
    Sea $X$ un conjunto con $B \in \cc{P}(X)$ y $\Gamma \subseteq \cc{P}(X)$, se tiene que:
    \begin{equation*}
        B \cap \left( \bigcup_{A \in \Gamma}A \right) = \bigcup_{A\in \Gamma} (B \cap A)
        \hspace{1cm}
        B \cup \left( \bigcap_{A \in \Gamma}A \right) = \bigcap_{A\in \Gamma} (B \cup A)
    \end{equation*}
\end{prop}
\begin{proof}
    \begin{equation*}
        \begin{split}
            B \cap \left( \bigcup_{A \in \Gamma}A \right) &= \left\{ x \in X \mid x \in B \y x \in \bigcup_{A \in \Gamma}A \right\} =\\
            & = \{x \in X \mid x \in B \y \exists A \in \Gamma \mid x \in A\} =\\
            & =\{x \in X \mid \exists A \in \Gamma \mid x \in B \y x \in A\}= \bigcup_{A\in \Gamma} (B \cap A)
        \end{split}
    \end{equation*}

    \begin{equation*}
        \begin{split}
            B \cup \left( \bigcap_{A \in \Gamma}A \right) &= \left\{ x \in X \mid x \in B \o x \in \bigcap_{A \in \Gamma}A \right\} =\\
            & = \{x \in X \mid x \in B \o x \in A~\forall A \in \Gamma\} =\\
            & = \{x \in X \mid \forall A \in \Gamma~x \in B \o x \in A\} = \bigcap_{A\in \Gamma} (B \cup A)
        \end{split}
    \end{equation*}
\end{proof}


\section{Álgebra de proposiciones}
\begin{definicion}[Conjunto que verifica una propiedad]
    Sea $X$ un conjunto y sea $P$ una propiedad referida a los elementos de dicho conjunto, definimos el conjunto de elementos de $X$ que verifica dicha propiedad, que usualmente notaremos por $X_P$, como:
    $$X_P = \{x \in X \mid x \mbox{ verifica } P \}$$
\end{definicion}
\begin{ejemplo}
    Sea $X=\bb{Z}$, y sea $P$ la propiedad de ser un número positivo. Entonces:
    \begin{equation*}
        X_P = \{x\in \bb{Z}\mid x\geq 0\} = \bb{N}
    \end{equation*}
\end{ejemplo}

\begin{prop}
    Sea $X$ un conjunto y sean $P$ y $Q$ dos propiedades referidas a dicho conjunto. Es posible calcular el conjunto de elementos de $X$ que verifican $P$ y $Q$ simultáneamente o el conjunto de elementos de $X$ que verifica al menos una propiedad a partir de la fórmula:
    \begin{gather*}
        X_{(P \y Q)} = X_P \cap X_Q \\
        X_{(P \o Q)} = X_P \cup X_Q
    \end{gather*}
\end{prop}
\begin{proof} Trivialmente, se verifica lo siguiente:
    \begin{gather*}
        X_P \cap X_Q = \{x \in X \mid x \mbox{ verifica } P \y x \mbox{ verifica } Q\} = X_{(P \y Q)} \\
        X_P \cup X_Q = \{x \in X \mid x \mbox{ verifica } P \o x \mbox{ verifica } Q\} = X_{(P \o Q)}
    \end{gather*}
\end{proof}

\begin{prop}
    Sea $X$ un conjunto y sea $P$ una propiedad referida a dicho conjunto, podemos calcular el conjunto de elementos de $X$ que no cumplen la propiedad $P$ a partir de $X_P$, de la forma:
    $$X_{\lnot P} = C(X_P)$$
\end{prop}
\begin{proof}
    $$C(X_P) = \{x \in X \mid x \notin X_P \} = \{x \in X \mid x \mbox{ no verifica } P\} = X_{\lnot P}$$
\end{proof}

\begin{definicion}[Proposición matemática]
    Una \textbf{proposición matemática} es una relación entre dos propiedades $P$ y $Q$ referidas a los elementos de un conjunto $X$ del tipo $$P \Longrightarrow Q$$
    
    Se lee ``$P$ implica $Q$'' o ``$P$ entonces $Q$'', y significa que si $x\in X$ verifica $P$, entonces también verifica $Q$. Equivalentemente, ha de ser $X_P\subseteq X_Q$.
\end{definicion}

Demostrar la falsedad de la proposición matemática $P \Longrightarrow Q$ es equivalente a demostrar que $X_p\not \subseteq X_Q$. Es decir, ver que $\exists x\in X_P$ tal que $x\notin X_Q$. A dicho elemento $x$ se le llama \textbf{contraejemplo}.

\begin{definicion}[Recíproco]
    Dada una proposición matemática $P\Longrightarrow Q$, definimos su \textbf{proposición matemática recíproca}, o \textbf{recíproco} como la proposición matemática:
    \begin{equation*}
        Q\Longrightarrow P
    \end{equation*}
\end{definicion}

\begin{observacion}
    Dada una proposición matemática, su recíproco no siempre es verdadero. Por ejemplo, es cierto que todo número natural es un número entero ($\mathbb{N} \subseteq \mathbb{Z}$) pero su recíproco, que todo número entero es un número natural, no es cierto ($\mathbb{Z} \nsubseteq \mathbb{N}$).
\end{observacion}

\begin{definicion}[Contrarrecíproco]
    Dada una proposición matemática $P\Longrightarrow Q$, definimos su \textbf{proposición matemática contrarrecíproca}, o \textbf{contrarrecíproco} como la proposición matemática:
    \begin{equation*}
        \lnot Q \Longrightarrow \lnot P
    \end{equation*}
\end{definicion}

\begin{prop}[Transitividad]
    Sean $P$, $Q$, $R$ propiedades referidas a los elementos de un conjunto $X$, tales que $P \Longrightarrow Q$
    y $Q \Longrightarrow R$. Entonces:
    $$P \Longrightarrow R$$
\end{prop}
\begin{proof}
    Se demuestra gracias a la transitividad de la inclusión de los subconjuntos, ya que $X_P\subseteq X_Q\subseteq X_R$, por lo que $X_P\subseteq X_R$.
\end{proof}

\begin{definicion}[Equivalencia]
    Sea $X$ un conjunto y $P$ y $Q$ propiedades referidas a sus elementos, diremos que \textbf{$P$ y $Q$ son equivalentes}, notado $P \Longleftrightarrow Q$ y leído ``$P$ si y solo si $Q$'', si:
    $$P \Longrightarrow Q \quad\land\quad Q \Longrightarrow P$$
    Notemos que la equivalencia se da cuando tanto una proposición matemática como su proposición recíproca son ciertas.
\end{definicion}

\begin{prop}[Equivalencia generalizada]
    Sea $X$ un conjunto y $P_1$, $P_2$, \ldots, $P_n$ propiedades referidas a elementos de $X$ tales que $P_i \Longrightarrow P_{i+1}~\forall i \in \{1, \ldots, n-1\}$ y que $P_n \Longrightarrow P_1$. Entonces:
    $$P_i \Longleftrightarrow P_j~~\forall i, j \in \{1, \ldots, n\}$$
\end{prop}
\begin{proof}
    $\forall i,j \in \{1, \ldots, n\}$:
    \begin{itemize}
        \item \underline{Si $i = j$}:

        Se tiene $P_i \Leftrightarrow P_j$ trivialmente, ya que:
        \begin{equation*}
            X_{P_i} = X_{P_j} \Rightarrow \left\{
            \begin{array}{ccc}
                X_{P_i} \subseteq X_{P_j} & \Longrightarrow & P_i \Rightarrow P_j \\
                \y && \y\\
                X_{P_j} \subseteq X_{P_i} & \Longrightarrow & P_j \Rightarrow P_i
            \end{array}
            \right.
        \end{equation*}
        Por tanto, $P_i \Leftrightarrow P_j$.

        \item \underline{Si $i < j$}: $(P_i \Rightarrow P_{i+1} \Rightarrow \ldots \Rightarrow P_j) \Rightarrow (P_i \Rightarrow P_j)$

        \item \underline{Si $i > j$}: $(P_i \Rightarrow P_n \Rightarrow P_1 \Rightarrow P_j) \Rightarrow (P_i \Rightarrow P_j)$
    \end{itemize}
    
    Para la implicación $P_j \Rightarrow P_i$ hágase un camino similar al especificado y se obtendrá $P_j \Leftrightarrow P_i$.
\end{proof}
De esta forma, siempre que queramos probar que un conjunto finito de propiedades matemáticas son equivalentes entre sí, bastará probar que la primera es equivalente a la segunda, la segunda a la tercera, y así hasta que la penúltima es equivalente a la última y finalmente que la última es equivalente a la primera.

\subsubsection{Demostración por reducción al absurdo}

Sea $X$ un conjunto, $P$ y $Q$ propiedades referidas a los elementos de dicho conjunto. Queremos demostrar que $P \Rightarrow Q$. El procedimiento es el siguiente:

\begin{itemize}
    \item Supongamos que $\exists x \in X \mid x \in X_P\cap C(X_Q)$, es decir, que $\exists x\in X$ que verifica $P$ pero no $Q$.
    
    \item Si llegamos a una resultado que es falso o que contradice nuestra hipótesis ($x\in C(X_P)$), habremos llegado a una contradicción y podemos concluir que $\forall x \in X_P \Longrightarrow x \in X_Q$. Es decir, queda demostrado que $P \Longrightarrow Q$.
\end{itemize}

\subsubsection{Demostración por contrarrecíproco}

\begin{lema}\label{lema:1.10}
    Sea $X$ un conjunto y $A, B \in \cc{P}(X)$. Entonces:
    $$A \subseteq B \Longleftrightarrow C(B) \subseteq C(A)$$
\end{lema}
\begin{proof} Procedemos mediante doble implicación:
\begin{description}
    \item [$\Longrightarrow)$] Sea $x \in C(B)$ y supongamos $x \notin C(A)$, luego $x \notin B \y x \in A$. Como $A\subseteq B$, tenemos que $x\in B$, por lo que llegamos a una \underline{contradicción}. Por tanto, se tiene que $x\in C(A)$ y, por tanto, $C(B)\subseteq C(A)$.

    \item [$\Longleftarrow)$] Usando la otra implicación (ya demostrada), tenemos que $$A = C(C(A)) \subseteq C(C(B)) = B$$
\end{description}
\end{proof}

\begin{prop}[Demostración por contrarrecíproco]
    Sea $X$ un conjunto, $P$ y $Q$ propiedades referidas a sus elementos, son equivalentes:
    \begin{enumerate}
        \item $P \Rightarrow Q$ (Demostración directa).
        \item $\neg Q \Rightarrow \neg P$ (Demostración por contrarrecíproco).
    \end{enumerate}
    Es decir, dada una proposición matemática, será verdadera si y solo si lo es su proposición matemática contrarrecíproca.
\end{prop}
\begin{proof}
\begin{equation*}
    (P \Rightarrow Q) \Leftrightarrow X_P \subseteq X_Q
    \stackrel{(\ast)}{\Longleftrightarrow}
    C(X_Q) \subseteq C(X_P) \Longleftrightarrow (\neg Q \Rightarrow \neg P)
\end{equation*}
donde en $(\ast)$ he aplicado el lema anterior, el Lema \ref{lema:1.10}.
\end{proof}

\section{Aplicaciones}

\begin{definicion}[Par ordenado]
    Un par ordenado es un conjunto que contiene a dos elementos $a$ y $b$, notado $(a,b)$ en el que importa el orden. Es decir, si $(a,b)$ y $(c,d)$ son dos pares ordenados:
    $$(a,b) = (c,d) \Longleftrightarrow a = c \y b = d$$
\end{definicion}

\begin{definicion}[Terna]
    Una terna es un conjunto de tres elementos $a$, $b$, $c$ en el que importa el orden, notado por:
    \begin{equation*}
        (a,b,c)
    \end{equation*}
\end{definicion}

\begin{definicion}[$n$-upla]
    Dado un número natural $n$, podemos generalizar el concepto de par ordenado o de terna a una $n$-upla, que es un conjunto de $n$ elementos $a_1$, $a_2$, \ldots, $a_n$ en el que importa el orden. A este lo notaremos por:
    \begin{equation*}
        (a_1,a_2,\ldots,a_n)
    \end{equation*}
\end{definicion}

\begin{definicion}[Producto Cartesiano]
    Sean $X$ e $Y$ dos conjuntos, definimos el \textbf{producto cartesiano de $X$ e $Y$} como el conjunto:
    $$X\times Y = \{(x,y) \mid x \in X \y y \in Y\}$$
\end{definicion}

Por lo general, se tiene que $X \times Y \neq Y \times X$ salvo que $X = Y$.

\begin{definicion}[Producto Cartesiano generalizado]
    Sean $X_1$, $X_2$, \ldots, $X_n$ conjuntos, definimos el \textbf{producto cartesiano de $X_1$, $X_2$,
        \ldots, $X_n$} como el conjunto:
    $$X_1 \times X_2 \times \ldots \times X_n = \{(x_1, x_2, \ldots, x_n) \mid x_i \in X_i ~~\forall i=1,\ldots,n \}$$
\end{definicion}
\begin{notacion}
    A veces notaremos: $\prod\limits_{i=1}^n X_i := X_1 \times X_2 \times \ldots \times X_n $.

    En el caso en el que $X=X_1 = X_2 = \ldots = X_n$, notaremos $\prod\limits_{i=1}^n X_i := X^n$.
\end{notacion}

\begin{ejemplo}
    Sea $X=\{a,b\}$ e $Y=\{1,2,3\}$. Entonces:
    \begin{gather*}
        X\times Y = \{(a,1), (a,2), (a,3), (b,1), (b,2), (b,3)\} \\
        Y\times X = \{(1,a), (1,b), (2,a), (2,b), (3,a), (3,b)\}
    \end{gather*}
\end{ejemplo}

\begin{prop}
    Si $X$ e $Y$ son dos conjuntos finitos, entonces $X\times Y$ es finito, con:
    \begin{equation*}
        |X\times Y|=|X|~|Y|
    \end{equation*}
\end{prop}
\begin{proof}
    Para cada elemento de $X$, tenemos que hay $|Y|$ opciones disponibles para completar el par ordenado. Como hay $|X|$ elementos en $X$, tenemos que en total hay $|X|~|Y|$ pares ordenados.
\end{proof}


\begin{definicion}[Aplicación]
    Una \textbf{aplicación} es una terna $(X, Y, f)$ donde $X$ es un conjunto llamado \textbf{dominio de la aplicación}, $Y$ es otro conjunto \textbf{llamado recorrido, rango o codominio de la aplicación} y $f\subseteq X\times Y$ es un conjunto llamado \textbf{grafo de la aplicación}. Esta terna ha de cumplir las siguientes propiedades:
    \begin{enumerate}
        \item $\forall x \in X,~\exists y \in Y \mid (x, y) \in f$.
        \item $\forall (x,y), (x',y') \in f$, si $x=x'\Longrightarrow y=y'$.
    \end{enumerate}
\end{definicion}

Las dos propiedades anteriores son equivalentes a que:
$$\forall x \in X~\exists_1 y \in Y \mid (x,y) \in f$$

Al único elemento $y \in Y$ que corresponde a un elemento $x \in X$ le llamaremos imagen por $f$ de $x$ (o simplemente $f$ de $x$), notado $y:=f(x)$. A veces, a dicho elemento $x$ tal que $f(x)=y$ lo llamaremos antiimagen de $y$.\\

Cuando tengamos una aplicación (es decir, una terna $(X, Y, f)$), hablaremos de una aplicación $f$ de $X$ en $Y$, notado de algunas de las siguientes formas:
\begin{equation*}
    f:X\longrightarrow Y
    \hspace{1cm}
    X \stackrel{f}{\longrightarrow} Y
\end{equation*}

Dar una aplicación es dar su dominio, su recorrido y el conjunto de pares ordenados; que es equivalente a dar el dominio, el recorrido y especificar a qué elemento del recorrido le corresponde cada elemento del dominio, que suele ser usual hacerlo mediante una fórmula. Por tanto, dos aplicaciones son iguales si tienen el mismo dominio, recorrido y grafo.

\begin{ejemplo} Algunos ejemplos de la definición anterior son:
\begin{enumerate}
    \item No existe la aplicación $f:\N \rightarrow \N$ dada por $f(x) = x-1$, ya que no cumple con la primera condición: $f(0)=-1\notin\N$.

    \item No existe la aplicación $g:\bb{N}\longrightarrow \bb{N}$ definida por la fórmula
    \begin{equation*}
        g(x)=\left\{
        \begin{array}{ccl}
            x & \text{si} & x \text{ no es múltiplo ni de $2$ ni de $3$} \\ \\
            \dfrac{x}{2} & \text{si} & x \text{ es múltiplo de $2$} \\ \\
            \dfrac{x}{3} & \text{si} & x \text{ es múltiplo de $3$}
        \end{array}
        \right.
    \end{equation*}
    Esto se debe a que $6$ podría tener dos imágenes, por lo que no cumpliría la segunda condición:
    \begin{equation*}
        g(6)=\frac{6}{2}=3
        \hspace{1cm}
        g(6)=\frac{6}{3}=2
    \end{equation*}

    \item La fórmula $f(x) = \dfrac{x^2+1}{x-1}$ define una aplicación $f:\left]0,1\right[ \rightarrow \R$ pero no puede definir una aplicación $f:[0,1] \rightarrow \R$, ya que $\nexists f(1)$.

    \item La suma de naturales $+:\bb{N}\times \bb{N}\longrightarrow \bb{N}$ dada por $+(x,y) = x+y$ es una aplicación.
\end{enumerate}
\end{ejemplo}


\begin{definicion}[Imagen de una aplicación]
    Si $f:X \rightarrow Y$ es una aplicación, al conjunto de las imágenes de los elementos de $X$ lo
    llamaremos \textbf{conjunto imagen de la aplicación}, notado $Img(f)$:
    $$Img(f) = \{f(x) \mid x \in X\} \subseteq Y$$
\end{definicion}

\begin{definicion}[Sobreyectividad]
    Dada una aplicación $f:X \rightarrow Y$, diremos que \textbf{$f$ es sobreyectiva} si $Img(f)~=~Y$. Es decir, se ha de cumplir que:
    $$\forall y \in Y ~\exists x \in X \mid f(x) = y$$
\end{definicion}

\begin{definicion}[Inyectividad]
    Dada una aplicación $f:X \rightarrow Y$, diremos que \textbf{$f$ es inyectiva} si elementos distintos tienen imágenes distintas. Es decir, se ha de cumplir:
    $$\forall x,z \in X \mid x \neq z \Longrightarrow f(x) \neq f(z)$$
    
    Por contrarrecíproco\footnote{Suele ser la forma más fácil de probar que una aplicación es inyectiva, mediante el contrarrecíproco de la definición.}, $f$ es inyectiva si $\forall x,z \in X \mid f(x) = f(z) \Longrightarrow x = z$.
\end{definicion}

\begin{definicion}[Biyectividad]
    Dada una aplicación $f:X \rightarrow Y$, diremos que \textbf{$f$ es biyectiva} si es a la vez inyectiva y sobreyectiva.
\end{definicion}

\begin{definicion}[Conjuntos biyectivos]
    Sean $X$ e $Y$ dos conjuntos, diremos que son biyectivos, notado $X \cong Y$ ó $\displaystyle X \mathop{\cong}^{f} Y$
    si existe una aplicación $f:X \rightarrow Y$ biyectiva.
\end{definicion}

\begin{ejemplo}
    Algunos ejemplos de inyectividad, sobreyectividad y biyectividad son:
    \begin{enumerate}
        \item $f:\bb{Z}\to\bb{Z}$, $f(x)=x^2$ no es sobreyectiva ni inyectiva.
        \item $g:\bb{Z}\to\bb{Z}$, $g(x)=2x$ es inyectiva pero no sobreyectiva.
        \item $h:\bb{Z}\to\bb{N}$, $h(x)=|x|$ es sobreyectiva pero no inyectiva.
        \item $t:\bb{Z}\to\bb{Z}$, $t(x)=x+2$ es biyectiva.
    \end{enumerate}
\end{ejemplo}

\begin{definicion}[Aplicación identidad]
    Sea $X$ un conjunto, definimos la aplicación \textbf{identidad en $X$}; notada como $id_X,~I_X,~Id_X$, o $1_X$; como la
    siguiente aplicación:
    \Func{id_X}{X}{X}{x}{id_X(x)=x}
\end{definicion}

\begin{definicion}[Composición]
    Sean $f:X\rightarrow Y$ y $g:Y \rightarrow Z$ dos aplicaciones, definimos la aplicación \textbf{$g$ compuesta con $f$}, notada $g \circ f$, como la siguiente aplicación:
    \Func{g\circ f}{X}{Z}{x}{g(f(x))}
\end{definicion}

Algunas propiedades de la composición de aplicaciones son:
\begin{prop}\label{prop:CompAsoc}
    La composición es asociativa. Es decir, dadas las aplicaciones $\displaystyle X \mathop{\longrightarrow}^{f} Y \mathop{\longrightarrow}^{g} Z \mathop{\longrightarrow}^{h} T$, se cumple:
    $$f \circ (g \circ h) = (f \circ g) \circ h$$
\end{prop}
\begin{proof}
    Los dominios de ambas aplicaciones son $X$ y los codominios $T$. Falta comprobar que los grafos coinciden. Para todo $x \in X$, se cumple que:
    \begin{gather*}
        (f \circ (g \circ h))(x) = f[(g\circ h)(x)] = f[g(h(x))]\\
        ((f \circ g) \circ h)(x) = (f\circ g)(h(x)) = f[g(h(x))]
    \end{gather*}
    Por tanto, se tiene $f \circ (g \circ h) = (f \circ g) \circ h$.
\end{proof}

\begin{prop}
    Dada una aplicación $f:X \rightarrow Y$ arbitraria, se verifica que la identidad es el elemento neutro de la composición. Es decir,
    \begin{gather*}
        f \circ id_X = f\\
        id_Y \circ f = f
    \end{gather*}
\end{prop}
\begin{proof}
    Los dominios y codominios de $f \circ id_X$, $f$, $id_Y \circ f$ coinciden. Falta ver que los grafos también lo hacen. $\forall x \in X$:
    \begin{gather*}
        (f \circ id_X)(x) = f(id_X(x)) = f(x) \\
        (id_Y \circ f)(x) = id_Y(f(x)) = f(x)
    \end{gather*}

    Por tanto, $f \circ id_X = f = id_Y \circ f$.
\end{proof}

\begin{lema}
    Sean $f:X\rightarrow Y$ y $g:Y \rightarrow X$ aplicaciones tales que $g \circ f = id_X$.
    
    Entonces, $f$ es inyectiva y $g$ es sobreyectiva.
\end{lema}
\begin{proof}
    Demostramos en primer lugar que $f$ es inyectiva:
    $$\forall x_1, x_2 \in X \mid f(x_1) = f(x_2) \Longrightarrow g(f(x_1)) = g(f(x_2)) \stackrel{(\ast)}{\Longrightarrow} x_1 = x_2$$
    donde en $(\ast)$ hemos usado la hipótesis de que $g \circ f = id_X$.
    Por tanto, como se tiene $f(x_1)=f(x_2)\Longrightarrow x_1=x_2$, se tiene que $f$ es inyectiva. Veamos ahora que $g$ es sobreyectiva:
    $$\forall x \in X ~ \exists y=f(x) \in Y \mid g(y) = x$$
    Por tanto, como todo elemento del codominio tiene su antiimagen correspondiente, tenemos que $g$ es sobreyectiva.
\end{proof}

\begin{teo}[Caracterización de la biyectividad] \label{teo:CarBiyect}
    Sea $f:X \rightarrow Y$ una aplicación. Entonces:
    $$f \mbox{ es biyectiva } \Longleftrightarrow \exists g:Y \rightarrow X \mid g \circ f = id_X \y f \circ g = id_Y$$
\end{teo}
\begin{proof} Demostremos por doble implicación:
    \begin{description}
        \item[$\Longrightarrow)$] Suponemos $f$ biyectiva. Por tanto, $\forall y \in Y~\exists_1 x \in X \mid f(x) = y$. Definimos la aplicación siguiente:
        \Func{g}{Y}{X}{y}{g(y)=x\mid f(x)=y}
        Esto es posible ya que, por ser $f$ biyectiva, dicho valor de $x\in X$ es único. Veamos que verifica que $f\circ g=id_Y$:
        \begin{equation*}
            (f \circ g)(y) = f(g(y)) = f(x) = y \qquad \forall y \in Y
        \end{equation*}
        Por tanto, se tiene que $f \circ g = id_Y$. 
        La otra igualdad se deduce directamente de la definición de $g$, ya que $f(x)=y$:
        \begin{equation*}
            (g\circ f)(x) = g(f(x)) = g(y) = x \qquad \forall x\in X
        \end{equation*}
        Por tanto, se tiene esta implicación.

        \item[$\Longleftarrow$)]  Supongamos que $\exists g:Y \rightarrow X \mid g \circ f = id_X \y f \circ g = id_Y$.
        Según el lema anterior, sabemos que:
            $$\left.\begin{array}{lll}
                g \circ f = id_X & \Rightarrow & f \mbox{ es inyectiva y } g \mbox{ es sobreyectiva} \\
                f \circ g = id_Y & \Rightarrow & g \mbox{ es inyectiva y } f \mbox{ es sobreyectiva}
            \end{array} \right.$$
        Por tanto, tenemos que $f$ es biyectiva.
    \end{description}
\end{proof}

\begin{lema}[Unicidad]\label{lema:Uni_Inv}
    Sea $f:X\to Y$. Si $f$ es biyectiva, se verifica que la función $g:Y \rightarrow X$ (cuya existencia ya está provada) es la única aplicación que verifica que $g \circ f = id_X \y f \circ g = id_Y$.
\end{lema}
\begin{proof}
    Supongamos que no es única, y sea $h:Y \rightarrow X$ otra aplicación tal que $h \circ f = id_X \y f \circ h = id_Y$ la otra opción. Entonces:
    $$h = h \circ id_Y = h \circ (f \circ g) = (h \circ f) \circ g = id_X \circ g = g$$
    Quedando así demostrada la unicidad de $g$.
\end{proof}

\begin{definicion}[Inversa]
    Sea $f:X \rightarrow Y$ una aplicación biyectiva. Por el lema anterior, sólo existe una aplicación $g:Y \rightarrow X \mid g \circ f = id_X \y f \circ g = id_Y$. Llamaremos a esta aplicación $g$ \textbf{aplicación
        inversa de $f$} y la notaremos como $f^{-1}$.
\end{definicion}


Notemos que, dada $f:X \rightarrow Y$, para comprobar que $g:Y \rightarrow X$ sea la inversa de $f$, gracias al Lema \ref{lema:Uni_Inv} nos basta con ver que $f \circ g = id_Y \y g \circ f = id_X$.

\begin{lema}
    Sea $f:X \rightarrow Y$ biyectiva. Entonces $f^{-1}$ es biyectiva, siendo su inversa $f$:
    $$(f^{-1})^{-1} = f$$
\end{lema}
\begin{proof}
    Como $f^{-1}$ es la inversa de $f$, se tiene de forma directa:
    \begin{gather*}
        f \circ f^{-1} = id_Y \\
        f^{-1} \circ f = id_X
    \end{gather*}
    
    Por el Teorema \ref{teo:CarBiyect}, $f^{-1}$ es biyectiva; y por el Lema \ref{lema:Uni_Inv}, $(f^{-1})^{-1} = f$.
\end{proof}

\begin{lema}[Inversa de una composición]
    Sean $f:X \rightarrow Y$, $g:Y \rightarrow Z$ funciones biyectivas. Entonces:
    $$(g \circ f)^{-1} = f^{-1} \circ g^{-1}$$
\end{lema}
\begin{proof}
Tenemos que el dominio de ambas es $Z$ y el codominio es $X$. Aplicamos el Lema \ref{lema:Uni_Inv} y la Proposición \ref{prop:CompAsoc}:
\begin{gather*}
    (g \circ f) \circ (f^{-1} \circ g^{-1}) = g \circ (f \circ f^{-1}) \circ g^{-1} = (g \circ id_Y) \circ g^{-1} = g \circ g^{-1} = id_Z \\
    (f^{-1} \circ g^{-1}) \circ (g \circ f) = f^{-1} \circ (g^{-1} \circ g) \circ f = (f^{-1} \circ id_Y) \circ f = f^{-1} \circ f = id_X
\end{gather*}
Por lo que: $$(g \circ f)^{-1} = f^{-1} \circ g^{-1}$$
\end{proof}

\begin{prop}\label{prop:ConjFinito_Equivalencias}
    Sea $X$ un conjunto \ul{finito} no vacío y $f:X \rightarrow X$ una aplicación, los siguientes enunciados
    son equivalentes:
    \begin{enumerate}[label=\roman*.]
        \item $f$ es biyectiva.
        \item $f$ es inyectiva.
        \item $f$ es sobreyectiva.
    \end{enumerate}
\end{prop}
\begin{proof} Demostramos la siguiente equivalencia:
\begin{description} 
    \item[I $\Longrightarrow$ II)] Trivial, a partir de la definición de aplicación biyectiva.
    
    \item[II $\Longrightarrow$ III)] Sea $f:X \rightarrow X$ inyectiva, supongamos que $|X| = n$, $(n \geq 1)$. Como $f$ es inyectiva, entonces $|Img(f)| = n$. Luego:
    $$Img(f) \subseteq X \y |Img(f)| = |X| \Rightarrow Img(f) = X$$
    Por tanto, tenemos que $f$ es sobreyectiva.

    \item[III $\Longrightarrow$ II)] Sea $f$ sobreyectiva, y demostraremos que $f$ es inyectiva. Para ello, por reducción al absurdo, supongamos que $f$ no es inyectiva. Por tanto, $|Img(f)|<~|X|$. Entonces, $Img(f) \subsetneq X$, llegando así a una \underline{contradicción}, ya que $f$ era sobreyectiva.
    
    Luego $f$ es inyectiva y como era sobreyectiva, tenemos que es biyectiva.
\end{description}
\end{proof}

\begin{definicion}[Conjunto imagen de un conjunto]
    Dada una aplicación ${f:X\rightarrow Y}$ y un conjunto $A\subseteq X$, definimos la imagen de $A$ mediante $f$, notado por $f_*(A)$ o $f(A)$ por:
    \begin{equation*}
        f(A) = f_*(A) = \{f(x) \mid x \in A\} \subseteq Y
    \end{equation*}
\end{definicion}

\begin{definicion}[Conjunto imagen inversa de un conjunto]
    Dada una aplicación $f:X\rightarrow Y$ y un conjunto $B\subseteq Y$, definimos la imagen inversa de $B$ mediante $f$, notado por $f^*(B)$ o $f^{-1}(B)$ por:
    \begin{equation*}
        f^{-1}(B) = f^*(B) = \{x \in X \mid f(x) \in B\} \subseteq X
    \end{equation*}
\end{definicion}
No debemos confundir la notación $f^{-1}(B)$ con la aplicación inversa de $f$, pues no es necesario suponer nada sobre $f$ para hablar de la imagen inversa del conjunto $B$.

\begin{prop}
    La imagen inversa es compatible con todas las operaciones con conjuntos. Sea $f:X\rightarrow Y$ una aplicación y $A,B\subseteq Y$, se verifica:
    \begin{enumerate}
        \item $f^*(A \cup B) = f^*(A) \cup f^*(B)$
        \item $f^*(A\cap B) = f^*(A)\cap f^*(B)$
        \item $f^*(A - B) = f^*(A) - f^*(B)$
        \item $f^*(Y - A) = X - f^*(A)$
    \end{enumerate}
\end{prop}
\begin{proof}
    Demostramos cada una de las propiedades:
    \begin{align*}
        f^*(A \cup B) &= \{x \in X \mid f(x) \in A \cup B\} = \{x \in X \mid f(x) \in A \vee f(x) \in B\} =\\&= \{x \in X \mid f(x) \in A\} \cup \{x \in X \mid f(x) \in B\} = f^*(A) \cup f^*(B)\\
        f^*(A \cap B) &= \{x \in X \mid f(x) \in A \cap B\} = \{x \in X \mid f(x) \in A \y f(x) \in B\} =\\&= \{x \in X \mid f(x) \in A\} \cap \{x \in X \mid f(x) \in B\} = f^*(A) \cap f^*(B)\\
        f^*(A - B) &= \{x \in X \mid f(x) \in A - B\} = \{x \in X \mid f(x) \in A \y f(x) \notin B\} =\\&= \{x \in X \mid f(x) \in A\} - \{x \in X \mid f(x) \in B\} = f^*(A) - f^*(B)
    \end{align*}
    El último apartado se obtiene a partir del tercero, haciendo uso además de que $f^*(Y)=X$.
\end{proof}

\begin{definicion}[Aplicación característica de un conjunto]
    Sea $X$ un conjunto y $A\subseteq X$, podemos definir la \textbf{aplicación característica de $A$}, notada por $\chi_A$ como la aplicación $\chi_A:X\rightarrow \{0,1\}$ dada por:
    \begin{equation*}
        \chi_A(x) = \left\{\begin{array}{lcl}
                1 & \text{si} & x\in A \\
                0 & \text{si} & x\notin A
        \end{array}\right.
    \end{equation*}
\end{definicion}


\section{Relaciones de equivalencia}
\begin{definicion}[Relación binaria]
    Sea $X$ un conjunto no vacío, una \textbf{relación binaria en $X$} es un subconjunto $R \subseteq X \times X$. Dados $a, b \in X \mid (a, b) \in R$, diremos que $a$ está relacionado con $b$ por $R$, notado
    $aRb$.
\end{definicion}

Dado un conjunto $X$ y una relación binaria $R$ en $X$, algunas propiedades que puede cumplir $R$ son:
\begin{itemize}
    \item \textbf{Reflexividad:}  $\forall a \in X \Rightarrow aRa$
    \item \textbf{Simetría:} si: $\forall a,b \in X \mid aRb \Rightarrow bRa$
    \item \textbf{Transitividad:} $\forall a,b,c \in X \mid aRb \y bRc \Rightarrow aRc$
\end{itemize}
En el caso de que una relación $R$ cumpla las tres propiedades mencionadas, diremos que \textbf{$R$ es una relación binaria de equivalencia sobre el conjunto $X$}.

\begin{ejemplo} Algunos ejemplos de relaciones binarias son:
\begin{enumerate}
    \item Sea $X=\{a,b,c\}$. Son relaciones binarias:
    \begin{equation*}
        \begin{array}{l|ccc}
            & \text{Reflexividad} & \text{Simetría} & \text{Transitividad} \\ \hline
            R_1 = \{(a,a), (a,b), (b,c)\} & \text{No} & \text{No} & \text{No} \\
            R_2 = \{(a,a), (b,b), (c,c), (a,b), (b,c)\} & \text{Sí} & \text{No} & \text{No} \\
            R_3 = \{(a,b), (b,a)\} & \text{No} & \text{Sí} & \text{No} \\
            R_4 = \{(a,a), (b,b), (c,c), (a,b), (b,a)\} & \text{Sí} & \text{Sí} & \text{Sí} \\
        \end{array}
    \end{equation*}

    \item Sea $X=\bb{N}$, y consideramos la relación binaria:
    \begin{equation*}
        R=\{(a,b)\in \bb{N}\times \bb{N} \mid a+b \text{ es un número par}\}
    \end{equation*}

    Veamos que es una relación de equivalencia:
    \begin{itemize}
        \item \textbf{Reflexividad:}  Sea $a\in X$. Entonces, $aRa \Longleftrightarrow a+a=2a$ es par, lo cual es cierto.
        \item \textbf{Simetría:} Sean $a,b \in X \mid aRb\Longrightarrow a+b=2k \Longrightarrow b+a=2k \Longrightarrow bRa$, para cierto $k\in \bb{N}$.
        \item \textbf{Transitividad:} $\forall a,b,c \in X \mid aRb \y bRc$, se tiene que $\exists k,k'\in \mathbb{N}$:
        \begin{equation*}
            \left.\begin{array}{ccc}
                aRb & \Longrightarrow & a+b=2k \\
                \land&&\land\\
                bRc & \Longrightarrow & b+c=2k'
            \end{array} \right\}
            \Longrightarrow a+b+b+c = a+2b+c = 2k + 2k' = 2(k+k')
        \end{equation*}
        Por tanto, se tiene que $a+c=2(k+k'-b)$, para ciertos $k,k'\in \bb{N}$. Por tanto, tenemos que $aRc$.
    \end{itemize}

    \item Sea $X=\bb{R}^2$ y definimos $O=(0,0)$ como el origen del plano cartesiano. Entonces, consideramos la relación binaria:
    \begin{equation*}
        pRq \Longleftrightarrow d(O,p) = d(O,q)
    \end{equation*}

    Es directo comprobar (dejamos la demostración al lector) que esta relación es de equivalencia.
\end{enumerate}
\end{ejemplo}

\begin{definicion}[Clase de equivalencia]
    Sea $X$ un conjunto no vacío y $R$ una relación binaria de equivalencia. Para cada $a \in X$, definimos \textbf{la clase de equivalencia de $a$}, notada por $\overline{a}$ ó por $[a]$ como el conjunto:
    $$[a] = \{x \in X \mid xRa\} \subseteq X$$
\end{definicion}

Esto es, $[a]$ contiene aquellos elementos de $X$ que estén relacionados o que son equivalentes con $a$.
Por la propiedad reflexiva, tenemos que $aRa \Rightarrow a \in [a]$, por lo que $ [a] \neq \emptyset~\forall a \in X$.\\


A cada uno de los elementos de $X$ que pertenezcan a $[a]$ para algún $a \in X$ se les llama \textbf{representantes de la clase de $a$}.

\begin{ejemplo} Veamos algunos ejemplos de clase de equivalencia respecto de las relaciones binarias anteriores:
\begin{enumerate}
    \item[2.] Veamos la clase de equivalencia con representante de clase $0$ de la relación de equivalencia de los pares:
    \begin{equation*}
        [0] = \{x\in \bb{N}\mid xR0\} = \{x\in \bb{N}\mid x+0 \text{ es par}\} = \{x\in \bb{N}\mid x \text{ es par}\}
    \end{equation*}

    \item[3.] Veamos la clase de equivalencia con representante de clase el punto $(2,3)$ de la relación de equivalencia de la distancia:
    \begin{multline*}
        [(2,3)] = \{p\in \bb{R}^2\mid pR(2,3)\} = \{p\in \bb{R}^2\mid d(O,p)=d(O,(2,3))\} =\\= \left\{p\in \bb{R}^2\mid d(O,p)=\sqrt{13}\right\}
    \end{multline*}
\end{enumerate}
\end{ejemplo}

\begin{prop}
    Sea $X$ un conjunto no vacío y $R$ una relación de equivalencia en $X$. Sean $a,b \in X$. Son
    equivalentes:
    \begin{enumerate}[label=\roman*.]
        \item $aRb$
        \item $a \in [b]$
        \item $b \in [a]$
        \item $[a] \cap [b] \neq \emptyset$
        \item $[a] = [b]$
    \end{enumerate}
\end{prop}
\begin{proof} Demostramos por implicaciones sucesivas:
\begin{description}
    \item [I $\Longrightarrow$ II)] Por la definición de $[b]$, tenemos que si $aRb \Rightarrow a \in [b]$.

    \item [II $\Longrightarrow$ III)] Suponemos $a \in [b]$, es decir, $ aRb$. Por ser una relación de equivalencia, es simétrica, luego $bRa$, por lo que $b \in [a]$.
    
    \item [III $\Longrightarrow$ IV)] Hemos supuesto que $b\in [a]$. Además, se ha visto que $\forall b\in X$, se tiene que $b\in [b]$. Por tanto, $b\in [a]\cap [b]$, por lo que este último no es vacío.
    
    \item [IV $\Longrightarrow$ V)]
    Como $[a] \cap [b] \neq \emptyset \Rightarrow \exists c \in X \mid c \in [a] \cap [b] \Rightarrow cRa \y cRb$.
    \begin{gather*}
        \forall x \in [a] \Rightarrow xRa \mathop{\Rightarrow}^{cRa} aRc \Rightarrow xRc \mathop{\Rightarrow}^{cRb} xRb \Rightarrow x \in [b] \Rightarrow [a] \subseteq [b] \\
        \forall x \in [b] \Rightarrow xRb \mathop{\Rightarrow}^{cRb} bRc \Rightarrow xRc
        \mathop{\Rightarrow}^{cRa} xRa \Rightarrow x \in [a] \Rightarrow [b] \subseteq [a]
    \end{gather*}

    Tenemos que $[a] \subseteq [b] \y [b] \subseteq [a] \Rightarrow [a] = [b]$.
    
    \item [V $\Longrightarrow$ I)] Como $a \in [a] = [b] \Rightarrow aRb$. \qedhere
\end{description}
\end{proof}

\begin{definicion}[Conjunto cociente]
    Dado un conjunto $X$ no vacío y una relación de equivalencia $R$ sobre $X$, se define el
    \textbf{conjunto cociente de $X$ por la relación de equivalencia $R$}, notado $X/R$ como el conjunto:
    $$X/R = \{[a] \mid a \in X\}$$
\end{definicion}

\begin{ejemplo} Veamos algunos ejemplos de conjuntos cocientes por las relaciones binarias anteriores:
\begin{enumerate}
    \item[2.] Veamos las distintas clases de equivalencia que hay en la relación de equivalencia de los pares:
    \begin{equation*}
        \begin{split}
            [0] &= \{x\in \bb{N} \mid xR0\} = \{x\in \bb{N}\mid x+0 \text{ es par}\} = \{x\in \bb{N}\mid x \text{ es par}\} \\&= \{0,2,4,\dots\} \Longrightarrow [0]=[2]=[4]=\dots
        \end{split}
    \end{equation*}
    \begin{equation*}
        \begin{split}
            [1] &= \{x\in \bb{N} \mid xR1\} = \{x\in \bb{N}\mid x+1 \text{ es par}\} = \{x\in \bb{N}\mid x \text{ es impar}\} \\&= \{1,3,5,\dots\} \Longrightarrow [1]=[3]=[5]=\dots
        \end{split}
    \end{equation*}

    Por tanto, $\displaystyle \bb{N}/R = \{[0],[1]\}$.

    \item[3.] Veamos las distintas clases de equivalencia de la relación de equivalencia de la distancia:
    \begin{multline*}
        [p] = \{x\in \bb{R}^2\mid xRp\} = \{x\in \bb{R}^2\mid d(0,x)=d(0,p)\} = \{x\in \bb{R}^2\mid d(0,x)=r\} =\\= C_r \quad \text{(circunferencia de radio $r$ y centro $O$)}.
    \end{multline*}

    Por tanto, se tiene que $\displaystyle \bb{R}^2/R = \{C_r\mid r\in \bb{R}^+_0\}$.
\end{enumerate}
\end{ejemplo}

\begin{prop}
    Sea $f:X \rightarrow Y$ una aplicación y $R$ una relación de equivalencia en $X$. Supongamos que $f$ verifica la siguiente propiedad:
    \begin{center}
        Dados $a,b \in X \mid aRb \Rightarrow f(a) = f(b)$.
    \end{center}
    
    Entonces, podemos definir la siguiente aplicación:
    \begin{equation*}
        \begin{array}{rll}
            \overline{f}: X/R & \longrightarrow & Y\\
                \left[ a \right] & \longmapsto & \overline{f}\left([a]\right)=f(a)
        \end{array}
    \end{equation*}
    
    Se verifica que:
    \begin{enumerate}
        \item $Img\left(\overline{f}\right) = Img(f)$.
        \item $\overline{f}$ es sobreyectiva $\Longleftrightarrow f$ es es sobreyectiva.
        \item $\overline{f}$ es inyectiva $\Longleftrightarrow~\forall a,b \in X \mid f(a) = f(b) \Rightarrow aRb$.
    \end{enumerate}
\end{prop}
\begin{proof}
    Veamos en primer lugar que $\overline{f}$ está bien definida, es decir, que dos elementos iguales tienen la misma imagen. Nuestra definición de $\overline{f}$ depende del representante de la clase escogida, por lo que debemos comprobar que al cambiar el representante no cambia la imagen de $\overline{f}$:
    $$\forall a, b \in X \mid [a] = [b] \Rightarrow aRb \Rightarrow f(a) = f(b) \Rightarrow \overline{f}\left([a]\right) = \overline{f}\left([b]\right)$$

    Por tanto, tenemos que $\overline{f}$ es una aplicación. Comprobemos las tres propiedades que se enuncian:
    \begin{enumerate}
        \item Comprobemos que $Im\left(\overline{f}\right)=Im(f)$:
        $$Img\left(\overline{f}\right) = \left\{\overline{f}\left([a]\right) \mid [a] \in X/R\right\} = \{f(a) \mid [a] \in X/R\} = \{f(a) \mid a \in X\} = Img(f)$$

        \item $\overline{f}$ es sobreyectiva $\Longleftrightarrow Img\left(\overline{f}\right) = Y \Longleftrightarrow Img(f) = Y \Longleftrightarrow f$ es sobreyectiva.

        \item Comprobemos que $\overline{f}$ es inyectiva $\Longleftrightarrow~\forall a,b \in X \mid f(a) = f(b) \Rightarrow aRb$:
        \begin{description}
            \item[$\Longrightarrow)$] Sean $a,b \in X \mid f(a) = f(b) \Rightarrow \overline{f}\left([a]\right) = \overline{f}\left([b]\right) \Rightarrow [a] = [b] \Rightarrow aRb$

            \item[$\Longleftarrow)$] $\forall [a], [b] \in X/R \mid \overline{f}\left([a]\right) = \overline{f}\left([b]\right) \Rightarrow f(a) = f(b) \Rightarrow aRb \Rightarrow [a] = [b] \Rightarrow \overline{f}$ es inyectiva.
        \end{description}
    \end{enumerate}
\end{proof}

A la función $\overline{f}$ de la proposición anterior la llamaremos \textbf{aplicación inducida por $f$ en el conjunto cociente}.

    \chapter{Estadística descriptiva bidimensional}

\section{Distribución conjunta de dos caracteres estadísticos}

Sea una población formada por $n$ individuos en la que se desea estudiar simultáneamente dos caracteres, $X$ e $Y$.
Dichos caracteres podrán ser ambos cualitativos, uno cualitativo y otro cuantitativo o ambos cuantitativos (los dos
discretos, los dos continuos o uno discreto y otro continuo).\\


Si designamos por $x_1, x_2, \ldots, x_k$ las $k$ modalidades posibles del carácter $X$ y por $y_1, y_2, \ldots, y_p$
las $p$ modalidades posibles del carácter $Y$, las observaciones correspondientes a cada individuo serán de la forma
$(x_i, y_j)$, par ordenado que representa las modalidades tomadas por dicho individuo en los caracteres $X$ e $Y$.

\begin{itemize}
    \item $n_{ij}$: Número total de individuos en la población que presentan simultáneamente la modalidad $x_i$ del carácter $X$
          y la modalidad $y_j$ del carácter $Y$. Le llamamos frecuencia absoluta del par $(x_i, y_j)$.
    \item $f_{ij}$: Proporción de individuos en la población que presentan simultáneamente la modalidad $x_i$ del
          carácter $X$ y la modalidad $y_j$ del carácter $Y$. Le llamamos frecuencia relativa del par $(x_i, y_j)$.
          Por la definición de proporción sobre el total, tenemos que:
          $$f_{ij} = \dfrac{n_{ij}}{n} \qquad i \in \{1, 2, \ldots, k\} ~ j \in \{1, 2, \ldots, p\}$$
\end{itemize}

Gracias al principio de incompatibilidad y exhaustividad de las modalidades, tenemos que:
$$\sum_{i=1}^k\sum_{j=1}^p n_{ij} = n ~~ \sum_{i=1}^k\sum_{j=1}^p f_{ij}=1$$

La distribución $\left\{ (x_i,y_j), n_{ij}\right\}_{\substack{i=1,\dots,k\\j=1,\dots,p}}$ recibe el nombre de distribución conjunta de los caracteres $X$ e $Y$.

\begin{itemize}
    \item $n_{i.}$: Número total de individuos que presentan la modalidad $x_i$ del carácter $X$ sin tener en cuenta
          las modalidades que puedan tomar para el carácter $Y$:
          $$n_{i.} = \sum_{j=1}^p n_{ij} \qquad i\in \{1, \ldots, k\}$$
    \item $f_{i.}$: Proporción total de individuos que presentan la modalidad $x_i$ del carácter $X$ sin tener
          en cuenta el carácter $Y$:
          $$f_{i.} = \sum_{j=1}^p f_{ij} = \dfrac{n_{i.}}{n} \qquad i \in \{1, \ldots, k\}$$
    \item $n_{.j}$: Número total de individuos que presentan la modalidad $y_j$ del carácter $Y$ sin tener en cuenta
          las modalidades que puedan tomar para el carácter $X$:
          $$n_{.j} = \sum_{i=1}^k n_{ij} \qquad j\in \{1, \ldots, p\}$$
    \item $f_{.j}$: Proporción total de individuos que presentan la modalidad $y_j$ del carácter $Y$ sin tener
          en cuenta el carácter $X$:
          $$f_{.j} = \sum_{i=1}^k f_{ij} = \dfrac{n_{.j}}{n} \qquad j \in \{1, \ldots, p\}$$
\end{itemize}

Se tiene que:
$$\sum_{i=1}^k n_{i.} = \sum_{j=1}^p n_{.j} = n \hspace{2cm} \sum_{i=1}^k f_{i.} = \sum_{j=1}^p f_{.j} = 1$$

\section{Tablas estadísticas bidimensionales}

Para agrupar nuestros datos estadísticos, usaremos una tabla de doble entrada como la siguiente:

\begin{center}
    \begin{tabular}{c|c|c|c|c|c|c|c}
        $X \backslash Y$ & $y_1$    & $y_2$    & $\ldots$ & $y_j$    & $\ldots$ & $y_p$    & $n_{i.}$ \\
        \hline
        $x_1$            & $n_{11}$ & $n_{12}$ & $\ldots$ & $n_{1j}$ & $\ldots$ & $n_{1p}$ & $n_{1.}$ \\
        \hline
        $x_2$            & $n_{21}$ & $n_{22}$ & $\ldots$ & $n_{2j}$ & $\ldots$ & $n_{2p}$ & $n_{2.}$ \\
        \hline
        $\vdots$         & $\vdots$ & $\vdots$ & $\ldots$ & $\vdots$ & $\ldots$ & $\vdots$ & $\vdots$ \\
        \hline
        $x_i$            & $n_{i1}$ & $n_{i1}$ & $\ldots$ & $n_{ij}$ & $\ldots$ & $n_{ip}$ & $n_{i.}$ \\
        \hline
        $\vdots$         & $\vdots$ & $\vdots$ & $\ldots$ & $\vdots$ & $\ldots$ & $\vdots$ & $\vdots$ \\
        \hline
        $x_k$            & $n_{k1}$ & $n_{k2}$ & $\ldots$ & $n_{kj}$ & $\ldots$ & $n_{kp}$ & $n_{k.}$ \\
        \hline
        $n_{.j}$         & $n_{\text{.}1}$ & $n_{\text{.}2}$ & $\ldots$ & $n_{.j}$ & $\ldots$ & $n_{.p}$ & $n$
    \end{tabular}
\end{center}

En el caso de que uno de los carácteres sea cualitativo, esta tabla recibirá el nombre de tabla de contingencia.

\section{Representaciones gráficas}
\subsection{Diagrama de dispersión o nube de puntos}

Consiste en representar cada par de observaciones $(x_i, y_j)$ por un punto en un plano bidimensional.
Para representar la frecuecia absoluta de cada punto, se suele incluir un número pequeño al lado de cada punto.
En caso de representar variables cuantitativas continuas, usaremos las marcas de clase.

\subsection{Estereogramas}

Los estereogramas son gráficos tridimensionales formados por barras colocadas en cada punto $(x_i, y_j)$ con altura
$n_{ij}$.

En el caso de las variables cuantitativas continuas, las barras pasarán a ser prismas, donde la base del prisma
tiene dimensiones $e_i - e_{i-1} x e_j - e_{j-1}$. La altura de cada prisma vendrá dada por la fórmula:
$$h_{ij} = \dfrac{n_{ij}}{(L_i - L_{i-1})(L_j - L_{j-1})}$$
De tal forma que el volumen de cada prisma es igual a la frecuencia absoluta de cada pareja de intervalos de clase.

\section{Distribuciones marginales}

Las modalidades del carácter $X$ junto con las frecuencias $n_{i.}$ forman la distribución marginal del carácter $X$:
$$\{x_i, n_{i.}\}_{i=1,\ldots,k}$$

Mientras que las modalidades del carácter $Y$ junto con las frecuencias $n_{.j}$
forman la distribución marginal del carácter $Y$: $$\{y_j, n_{.j}\}_{j=1,\ldots,p}$$

Ambas son distribuciones unidimensionales, a las que es posible dar el tratamiento visto en el tema anterior.

Como ejemplo, la distribución marginal del carácter $X$ es la siguiente:
\begin{center}
    \begin{tabular}{c|c|c}
        $X$      & $n_{i.}$ & $f_{i.}$ \\
        \hline
        $x_1$    & $n_{1.}$ & $f_{1.}$ \\
        \hline
        $x_2$    & $n_{2.}$ & $f_{2.}$ \\
        \hline
        $\vdots$ & $\vdots$ & $\vdots$ \\
        \hline
        $x_i$    & $n_{i.}$ & $f_{i.}$ \\
        \hline
        $\vdots$ & $\vdots$ & $\vdots$ \\
        \hline
        $x_k$    & $n_{k.}$ & $f_{k.}$ \\
        \hline
                 & $n$        & $1$
    \end{tabular}
\end{center}

\section{Distribuciones condicionadas}

En ocasiones es interesante el estudio de un carácter sólo sobre los individuos que presentan una modalidad (o varias)
del otro carácter. por ejemplo, podría ser interesante el estudio del carácter $X$ en la subpoblación formada por
los individuos que presentan la modalidad $y_j$ del carácter $Y$, una subpoblación de $n_{.j}$ individuos.\\


Decimos que la frecuencia relativa de la modalidad $x_i$ del carácter $X$ en aquellos individuos que presentan la
modalidad $y_j$ del carácter $Y$ es:
$$f_{i/j} \equiv f_i^j = \dfrac{n_{ij}}{n_{.j}} \qquad i \in \{1, \ldots, k\}$$


Análogamente, podemos considerar la frecuencia relativa de la modalidad $y_j$ del carácter $Y$ en aquellos individuos
que presentan la modalidad $x_i$ del carácter $X$:
$$f_{j/i} \equiv f_j^i = \dfrac{n_{ij}}{n_{i.}} \qquad j \in \{1, \ldots, p\}$$

De esta forma, existen $p$ distribuciones condicionadas para el carácter $X$ según una única modalidad del carácter
$Y$ y $k$ distribuciones condicionadas para el carácter $Y$ según una única modalidad del carácter $X$.\\


Ejemplo de la distribución condicionada del carácter $X$ respecto a la modalidad $y_j$ del carácter $Y$, que
denotaremos por $X/Y=y_j$:

\begin{center}
    \begin{tabular}{c|c|c}
        $X$      & $n_{ij}$ & $f_i^j$  \\
        \hline
        $x_1$    & $n_{1j}$ & $f_1^j$  \\
        \hline
        $x_2$    & $n_{2j}$ & $f_2^j$  \\
        \hline
        $\vdots$ & $\vdots$ & $\vdots$ \\
        \hline
        $x_i$    & $n_{ij}$ & $f_i^j$  \\
        \hline
        $\vdots$ & $\vdots$ & $\vdots$ \\
        \hline
        $x_k$    & $n_{kj}$ & $f_k^j$  \\
        \hline
                 & $n_{.j}$ & 1
    \end{tabular}
\end{center}

De las definiciones anteriores, tenemos que:
$$f_{ij} = \dfrac{n_{ij}}{n} = \dfrac{n_{i.}}{n}\dfrac{n_{ij}}{n_{i.}} = \dfrac{n_{.j}}{n}\dfrac{n_{ij}}{n_{.j}} $$
$$ f_{ij} = f_{i.} f_j^i = f_{.j}f_i^j $$

\section{Dependencia e Independencia estadística}

Dos caracteres $X$ e $Y$ serán \underline{estadísticamente dependientes} cuando la variación en uno de ellos influya en la distribución del otro.\\


Por otra parte, se dice que el carácter $X$ es \underline{estadísticamente independiente} del carácter $Y$ si las distribuciones
de $X$ condicionadas a cada valor $y_j$ de $Y$ ($X/Y=y_j$) son idénticas para cualquier valor de $j$. En este caso,
cada distribución condicionada es idéntica a la distribución marginal de $X$:
$$\dfrac{n_{i1}}{n_{\text{.}1}} = \dfrac{n_{i2}}{n_{\text{.}2}} = \ldots = \dfrac{n_{ij}}{n_{.j}} = \ldots = \dfrac{n_{ip}}{n_{.p}} \qquad \forall i = 1, \ldots, k$$

De donde tenemos que:
$$f_{i/j} \equiv f_i^j = \dfrac{n_{ij}}{n_{.j}} = \dfrac{n_{i1} + n_{i2} + \ldots + n_{ip}}{n_{.1} + n_{.2} + \ldots + n_{.p}}
    = \dfrac{n_{i.}}{n} = f_{i.}$$

Por lo que si $f_i^j = f_{i.} \ \forall i \in \{1, \ldots, k\}$, entonces el carácter $X$ será independiente del
carácter $Y$. (Análogamente, se define la independencia del carácter $Y$ con el carácter $X$).

\begin{prop}
Si $X$ es una variable independiente de $Y$ $\Longrightarrow f_{i\cdot} = f_{i/j}$
\end{prop}
\begin{proof}
Suponemos $X$ e $Y$ variables independientes.
$$f_{i\cdot} = \sum_{j=1}^p f_{ij} = \sum_{j=1}^p f_{i/j}f_{\cdot j} \stackrel{(\ast)}{=} f_{i/j}\sum_{j=1}^pf_{\cdot j} = f_{i/j} f_{\cdot \cdot} = f_{i/j}$$

donde en $(\ast)$ he usado que las variables son independientes, por lo que la frecuencia condicionada a $Y$ no depende de $j$.
\end{proof}

\begin{teo}[Teorema de Caracterización de la Independencia]
Sean $X$ e $Y$ dos variables estadísticas.

\centering
$X$ e $Y$ son independientes $\Longleftrightarrow$ $f_{ij} = f_{i\cdot} f_{\cdot j} \Longleftrightarrow n_{ij} = \frac{n_{i\cdot} n_{\cdot j}}{n} \quad \forall i,j$
\end{teo}
\begin{proof} Demostramos mediante la doble implicación:
    \begin{description}
    \item [$\Longrightarrow$)] Suponemos $X$ e $Y$ independientes, es decir, $f_{i/j} = f_{i \cdot}$

    Por tanto,
    $$f_{ij} = f_{i/j}f_{\cdot j} = f_{i \cdot }f_{\cdot j}$$
    
    \item [$\Longleftarrow$)] Suponemos $f_{ij} = f_{i\cdot} f_{\cdot j}$.

    Probemos que $X$ e $Y$ son linealmente independientes.
    $$f_{\cdot j} = \frac{f_{ij}}{f_{i \cdot}} = f_{j/i}$$
    \end{description}
\end{proof}

\begin{prop}
    Si el carácter $X$ es independiente del carácter $Y$, entonces $Y$ es independiente de $X$ (la independencia es una propiedad recíproca).
\end{prop}
\begin{proof}
    Supuesto $X$ independiente de $Y$, tenemos $f_{i\cdot} = f_{i/j} \quad \forall j=1,\dots,p$
    \begin{equation*}
        f_{ij} = f_{i\cdot} f_{\cdot j} = f_{i\cdot}f_{j/i} \Longrightarrow f_{\cdot j} = f_{j/i}
    \end{equation*}
    demostrando así que $X$ es independiente de $Y$.
\end{proof}


Se dice que el carácter $X$ \underline{depende funcionalmente} del carácter $Y$ si a cada modalidad de $y_j$ de $Y$
le corresponde una única modalidad posible de $X$ con frecuencia no nula. Es decir:
$$\forall j \in \{1, \ldots, p\}, n_{ij} = 0 \text{ excepto para un valor } i = \varphi(j) \mid n_{ij} = n_{.j}$$

\begin{ejemplo} Consideramos la siguiente tabla estadística bidimensional:
    \begin{center}
    \begin{tabular}{c|c|c|c|c|c|c}
        $X \backslash Y$ & $y_1$ & $y_2$ & $y_3$ & $y_4$ & $y_5$ &    \\
        \hline
        $x_1$            & 3     & 0     & 6     & 0     & 0     & 9  \\
        \hline
        $x_2$            & 0     & 4     & 0     & 0     & 2     & 6  \\
        \hline
        $x_3$            & 0     & 0     & 0     & 5     & 0     & 5  \\
        \hline
                         & 3     & 4     & 6     & 5     & 2     & 20 \\
    \end{tabular}
    \end{center}
    
    
    En dicha distribución conjunta, $X$ depende funcionalmente del carácter $Y$. Sin embargo, $Y$ no depende
    funcionalmente del carácter $X$.
\end{ejemplo}

Si sucede que la dependencia funcional es bidireccional, hablaremos de una \underline{dependencia funcional recíproca}.
Notemos que esta es de poco interés estadístico.

\section{Momentos bidimensionales}

Dada una variable estadística bidimensional ($X,Y$) con una distribución conjunta
$\left\{ (x_i,y_j), n_{ij}\right\}_{\substack{i=1,\dots,k\\j=1,\dots,p}}$, se definen los momentos conjunto central y no central de órdenes $r$ y $s$
($r,s \in \N \cup \{0\}$) como:
$$\mu_{rs} = \sum_{i=1}^k \sum_{j=1}^p f_{ij} (x_i - \overline{x})^r (y_j - \overline{y})^s$$
$$m_{rs}=\sum_{i=1}^k \sum_{j=1}^p f_{ij} x_i^r y_j^s$$


Los momentos centrales más utilizados son las varianzas marginales, $\mu_{20} = \sigma_x^2$ y $\mu_{02}=\sigma_y^2$
y el momento $\mu_{11}$, cuya importancia se describe a continuación:

\subsection{Varianza}
\begin{definicion} Dadas dos variables estadísticas unidimensionales, $X$ y $Y$, se define la covarianza de las variables $X$ e $Y$ como:
\begin{equation*}
    \sigma_{xy} = Cov(X,Y) = \mu_{11}
\end{equation*}
\end{definicion}

\begin{prop} Dadas dos variables estadísticas unidimensionales, $X$ y $Y$, se tiene:
    $$\sigma_{xy} = \mu_{11} = m_{11} - m_{10}m_{01}$$
\end{prop}
\begin{proof}
    \begin{equation*}\begin{split}
    \sigma_{xy} &=  Cov(X,Y) = \mu_{11} = \sum_{i=1}^k \sum_{j=1}^p f_{ij}(x_i - \bar{x})(y_j - \bar{y}) = \sum_{i=1}^k \sum_{j=1}^p f_{ij}(x_iy_j - x_i\bar{y} - \bar{x}y_j + \bar{x}\bar{y})
    =\\&=
    \sum_{i=1}^k \sum_{j=1}^px_iy_jf_{ij} - \sum_{i=1}^k \sum_{j=1}^p x_i\bar{y}f_{ij} - \sum_{i=1}^k \sum_{j=1}^p\bar{x}y_jf_{ij} + \sum_{i=1}^k \sum_{j=1}^p \bar{x}\bar{y} f_{ij}
    =\\&=
    \sum_{i=1}^k \sum_{j=1}^px_iy_jf_{ij} - \bar{y}\sum_{i=1}^k x_if_{i\cdot} - \bar{x}\sum_{j=1}^p y_jf_{\cdot j} + \bar{x}\bar{y}\sum_{i=1}^k \sum_{j=1}^pf_{ij}
    =\\&=
    \sum_{i=1}^k \sum_{j=1}^p x_iy_jf_{ij} - \bar{y}\bar{x} - \bar{x}\bar{y} + \bar{x}\bar{y}
    =\\&=
    \sum_{i=1}^k \sum_{j=1}^px_iy_jf_{ij} - \bar{x}\bar{y} = m_{11} - m_{10}m_{01}
\end{split}\end{equation*}
\end{proof}


\begin{prop}
    Si $X$ e $Y$ son independientes $\Longrightarrow \left\{ \begin{array}{c}
        m_{rs}=m_{r0}m_{0s}  \\
        \mu_{rs}=\mu_{r0}\mu_{0s} 
    \end{array}\right.$
\end{prop}
\begin{proof} Supongo $X$ e $Y$ independientes, por lo que $f_{ij}=f_{i.}f_{.j}$. Entonces:
    \begin{equation*}
        m_{rs} = \sum_{i=1}^k \sum_{j=1}^p f_{ij}x_i^r y_j^s
        = \sum_{i=1}^k \sum_{j=1}^p f_{i.}f_{.j}x_i^r y_j^s
        = \sum_{i=1}^k f_{i.}x_i^r \sum_{j=1}^p f_{.j} y_j^s
        = m_{r0}m_{0s}
    \end{equation*}
    \begin{multline*}
        \mu_{rs} = \sum_{i=1}^k \sum_{j=1}^p f_{ij}(x_i-\bar{x})^r (y_j-\bar{y})^s
        = \sum_{i=1}^k \sum_{j=1}^p f_{i.}f_{.j}(x_i-\bar{x})^r (y_j-\bar{y})^s
        =\\=
        \sum_{i=1}^k f_{i.}(x_i-\bar{x})^r \sum_{j=1}^p f_{.j} (y_j-\bar{y})^s
        = \mu_{r0}\mu_{0s}
    \end{multline*}
\end{proof}

\begin{coro}
    Si $X$ e $Y$ son independientes $\Longrightarrow \sigma_{xy} = 0$
\end{coro}
\begin{proof} Supongo $X$ e $Y$ independientes, por lo que $m_{rs}=m_{r0}m_{0s}$. Entonces:
    \begin{equation*}
        \sigma_{xy} = m_{11} - m_{10}m_{01} = m_{10}m_{01} - m_{10}m_{01} = 0
    \end{equation*}
\end{proof}

\begin{prop}
    Si se transforman los valores de $x_i$ e $y_j$ mediante transformaciones lineales dadas por:
    \begin{equation*}
        \left\{\begin{array}{c}
            x_i' = ax_i + b  \\
            y_j' = cy_j + d 
        \end{array} \right.
    \end{equation*}
    La covarianza queda como:
    $$\sigma_{x'y'} = ac \sigma_{xy}$$
\end{prop}
\begin{proof}
    \begin{multline*}
        \sigma_{x'y'} = \sum_{i=1}^k \sum_{j=1}^p f_{ij} (x_i' - \bar{x}')(y_j'-\bar{y}')
        = \sum_{i=1}^k \sum_{j=1}^p f_{ij} [(ax_i+b) - (a\bar{x}+b)][(cy_j+d)-(c\bar{y}+d)]
        =\\=
        ac \sum_{i=1}^k \sum_{j=1}^p (x_i - \bar{x})(y_j- \bar{y})f_{ij} = ac \sigma_{xy}
    \end{multline*}
\end{proof}


Si expresamos nuevas variables a partir de otras, podemos calcular su covarianza a partir de la otra:
$$x_i' = ax_i+b~~y_j'=cy_j+d$$
$$\sigma_{X'Y'}=\sum_{i=1}^k \sum_{j=1}^p f_{ij} (ax_i+b-\overline{x'})(cy_j+d-\overline{y'}) = $$
$$=\sum_{i=1}^k \sum_{j=1}^p f_{ij}(ax_i+b-(a\overline{x}+b))(cy_j+d-(c\overline{y}+d))=$$
$$=\sum_{i=1}^k \sum_{j=1}^p f_{ij}(ax_i-a\overline{x})(cy_j-c\overline{y}) =
    ac \sum_{i=1}^k \sum_{j=1}^p f_{ij}(x_i-\overline{x})(y_j-\overline{y}) = ac\sigma_{xy}$$

\section{Regresión}

Pretendemos buscar que un número de magnitudes $X_1, \ldots, X_n$ se relacionen con una variable $Y$ mediante la expresión:
$$Y = f(X_1, \ldots, X_n)$$

Podemos abordar el problema desde dos enfoques:
\begin{itemize}
    \item Regresión: La determinación de la estructura de dependencia que mejor expresa la relación de la variable $Y$
          con las demás.
    \item Correlación: El estudio del grado de dependencia que existe entre las variables.
\end{itemize}


Si dos variables presentan una dependencia estadística (es decir, una dependencia no funcional), no es posible encontrar una ecuación tal que los valores que puedan presentar dichas variables la satisfagan. Es decir, no es posible encontrar una función que pase por todos los puntos del diagrama de dispersión que representa esa distribución conjunta. Por tato, tendremos que ajustar lo mejor posible una función a una serie de valores observados, encontrando una curva
que, aunque no pase por todos los puntos de la nube, más se aproxime a ellos. Dicho método recibe el nombre de \underline{ajuste por mínimos cuadrados}.

\subsection{Método de mínimos cuadrados}
Sea $f(x_i, a_0,\dots,a_n)$ la función que aproxima la variable $Y$ en función de los valores de $X$.
\begin{notacion}
    A los valores ajustados se les notará de la siguiente manera:
    \begin{equation*}
        \hat{y_j} = f(x_i;a_0,\dots,a_n)
    \end{equation*}
\end{notacion}

\begin{definicion}[Residuo]
Se define el residuo de la modalidad $y_j$ de la variable $Y$ como:
\begin{equation*}
    e_{ij} = y_j - \hat{y_j}
    = y_j - f(x_i, a_0, a_1, \ldots, a_n)
\end{equation*}
\end{definicion}



El método de mínimos cuadrados consiste en encontrar una función $f$ que minimice la media de los cuadrados de los
residuos:
\begin{equation*}
        ECM(f(x_i, a_0, a_1, \ldots, a_n)) = \psi(a_0, a_1, \ldots, a_n)
        = \sum_{i=1}^k \sum_{j=1}^p f_{ij} e_{ij}^2
\end{equation*}


La función $\psi$ se denomina el \underline{error cuadrático medio de la función} $f$, denotada $ECM(f(x_i, a_0, a_1, \ldots, a_n))$.
Como los parámetros $(x_i, a_0, a_1, \ldots, a_n)$ sólo están sometidos a sumas, productos y cuadrados dentro de $\psi$,
dicha función es derivable respecto a cada $a_i \ \forall i \in \{0, \ldots, n\}$. Además, se puede asegurar que
el punto $(\hat{a_0}, \hat{a_1}, \ldots, \hat{a_n})$ donde se anulan las derivadas parciales primeras respecto
de cada $a_i$ corresponde a un mínimo de la función $\psi$.


El cálculo de los parámetros de la función de ajuste óptima según el método de los mínimos cuadrados consiste en
resolver el siguiente sistema, llamado \underline{sistema de ecuaciones normales}:
$$\dfrac{\partial \psi}{\partial a_r}=0 \Rightarrow \sum_{i=1}^k \sum_{j=1}^p f_{ij} e_{ij}
    \dfrac{\partial f}{\partial a_r} = 0 \qquad \forall r \in \{0, \ldots n\}$$


Una de las funciones de regresión más utilizadas para expresar el comportamiento de una variable en función de la otra
es un polinomio de grado $n$ (comenzaremos con $n=1$).

\subsubsection{Ajuste lineal (recta de regresión)}\vspace{-0.5cm}
\begin{equation*}
    Y=f(X;a,b) = a + bX
\end{equation*}

Supongamos que queremos ajustar por el método de mínimos cuadrados una recta que exprese $Y$ en función de $X$.
La función sería $Y=f(X;a,b) = a + bX$, por lo que tendremos que calcular el mínimo en $a$ y $b$ de la función:
$$\psi(a,b) = ECM(a,b) = \sum_{i=1}^k \sum_{j=1}^p f_{ij} [y_j - (a + bx_i)]^2 $$

Obtenemos el sistema de ecuaciones normales:
\begin{equation*}
    \left\{
    \begin{array}{l}
        \dfrac{\partial \psi}{\partial a} = 0 \Rightarrow \displaystyle\sum_{i=1}^k \sum_{j=1}^p f_{ij} [y_j - (a+bx_i)]=0\\ \\
        \dfrac{\partial \psi}{\partial b} = 0 \Rightarrow \displaystyle\sum_{i=1}^k \sum_{j=1}^p f_{ij} [y_j - (a+bx_i)]x_i=0
    \end{array}
    \right\} \Longrightarrow
    \left\{
    \begin{array}{l}
        m_{01} = a + b m_{10}\\ \\
        m_{11} = a m_{10} + bm_{20}
    \end{array}
    \right.
\end{equation*}

La resolución de dicho sistema nos proporciona los coeficientes buscados:
$$\hat a = m_{01} - \dfrac{m_{11} - m_{10}m_{01}}{m_{02}-m_{01}^2}m_{10} = \overline{y} - \dfrac{\sigma_{xy}}{\sigma_x^2}\overline{x}$$
$$\hat b = \dfrac{m_{11} - m_{10}m_{01}}{m_{02}- m_{01}^2} = \dfrac{\sigma_{xy}}{\sigma_x^2}$$

Por tanto, \textbf{la recta de regresión de $Y$ sobre $X$} tiene por expresión:
\begin{equation*}
    Y = \dfrac{\sigma_{xy}}{\sigma_x^2}X + \overline{y} - \dfrac{\sigma_{xy}}{\sigma_x^2}\overline{x}
    \hspace{1cm}
    \left(\text{Equivalentemente}, Y - \overline{y} = \dfrac{\sigma_{xy}}{\sigma_x^2} (X - \overline{x})\right)
\end{equation*}


\begin{definicion}[Coeficiente de regresión lineal]
    Al coeficiente $\dfrac{\sigma_{xy}}{\sigma_x^2}$ se le denomina coeficiente de regresión lineal de $Y$ sobre $X$.

    Análogamente, se define el coeficiente de regresión lineal de $X$ sobre $Y$.
\end{definicion}


Análogamente, la recta de regresión mínimo cuadrática de $X$ sobre $Y$ es la recta $X = h(Y; c, d) = c+dY$ que minimiza
la función
$$\phi(c,d) = \sum_{i=1}^k \sum_{j=1}^p f_{ij} (x_i - \hat x_j )^2$$

donde $\hat x_j = c+dy_j$. Siguiente el procedimiento anterior, llegamos a que \textbf{la recta de regresión de $X$ sobre $Y$} es:
$$X - \overline{x} = \dfrac{\sigma_{xy}}{\sigma_y^2} (Y - \overline{y})$$


Los coeficientes de regresión son las pendientes de las rectas de regresión. Los signos de dichos coeficientes son
los mismos para ambas rectas e igual al signo de la covarianza. Cuando exista dependencia funcional lineal, las dos
rectas de regresión coincidirán con la recta de dependencia.\\

Algunas propiedades de la recta de regresión son:
\begin{lema}
    Las rectas de regresión pasan por el punto $(\bar{x}, \bar{y})$.
\end{lema}
\begin{proof}
    La recta de regresión de $Y$ sobre $X$ tiene la forma de $y-\bar{y} = K(x-\bar{x})$. Para $x=\bar{x}$, vemos que $y=\bar{y}$.

    Análogamente, la recta de regresión de $X$ sobre $Y$ tiene la forma de $x-\bar{x} = K(y-\bar{y})$. Para $y=\bar{y}$, vemos que $x=\bar{x}$.
\end{proof}

\begin{lema}\label{lema:2.8}
    La media de los valores ajustados coincide con la de los valores observados de la variable.
\end{lema}
\begin{proof}
    \begin{equation*}
        \overline{\hat{y}} = \sum_{i=1}^k \sum_{j=1}^p f_{ij}\hat{y_j}
        = \sum_{i=1}^k \sum_{j=1}^p f_{ij} (ax_i + b)
        = a\bar{x} + b = \bar{y}
    \end{equation*}
\end{proof}

\begin{coro}
    La media de los residuos vale 0.
\end{coro}
\begin{proof}
    \begin{equation*}
        \sum_{i=1}^k \sum_{j=1}^p f_{ij}e_{ij}
        = \sum_{i=1}^k \sum_{j=1}^p f_{ij} (y_j - \hat{y_j})
        =  \bar{y} - \overline{\hat{y}} = 0
    \end{equation*}
\end{proof}

\begin{coro}
    La media de los productos de los residuos por los valores de la variable explicativa vale cero.
\end{coro}
\begin{proof}
    \begin{equation*}
        \sum_{i=1}^k \sum_{j=1}^p f_{ij} e_{ij}x_i = \bar{x} \sum_{i=1}^k \sum_{j=1}^p f_{ij}e_{ij} = 0
    \end{equation*}
\end{proof}

\begin{coro}
    La media de los productos de los residuos por los valores ajustados vale cero.
\end{coro}
\begin{proof}
    \begin{equation*}
        \sum_{i=1}^k \sum_{j=1}^p f_{ij} e_{ij}\hat{y_j} = \bar{\hat{y}} \sum_{i=1}^k \sum_{j=1}^p f_{ij}e_{ij} = 0
    \end{equation*}
    donde se ha aplicado la Proposición \ref{prop:1.3}.
\end{proof}


\subsubsection{Ajuste polinómico}\vspace{-0.5cm}
\begin{equation*}
    Y=f(X;a_0,a_1,\dots,a_n) = a_0 + a_1X + \dots + a_nX^n
\end{equation*}

Si queremos aproximar mediante un polinomio de grado superior o igual a dos, el método
de mínimos cuadrados nos conducirá al sistema de ecuaciones:
\begin{equation*}
    \left\{
    \begin{array}{cl}
        m_{01} &= a_0 + a_1 m_{10} + \ldots + a_n m_{n0}\\
        m_{11} &= a_0m_{10} + a_1 m_{20} + \ldots + a_n m_{n+1,0}\\
        m_{21} &= a_0m_{20} + a_1 m_{30} + \ldots + a_n m_{n+2,0}\\
        &\vdots\\
        m_{n1} &= a_0 m_{n0} + a_1 m_{n+1,0} + \ldots + a_n m_{2n,0}
    \end{array}
    \right.
\end{equation*}



Para ajustar a la nube otro tipo de función, intentaremos pasar a un ajuste polinómico. Ejemplos de esto son los siguientes ajustes:

\subsubsection{Ajuste hiperbólico}\vspace{-0.5cm}
\begin{equation*}
    Y=f(X;a,b) = a+b\frac{1}{X}
\end{equation*}


Si queremos realizar un ajuste hiperbólico mediante una hipérbola equilátera realizamos la transformación $Z = \dfrac{1}{X}$, y realizamos el ajusta de mínimos cuadrados a la recta $Y = a + bZ$ sobre
las variables $(Z,Y)$.

\subsubsection{Ajuste potencial}\vspace{-0.5cm}
\begin{equation*}
    Y=f(X;a,b) = aX^b
\end{equation*}

De otra forma, si queremos aplicar el ajuste potencial hemos de aplicar el logaritmo y obtenemos la siguiente expresión:
$$\ln Y = \ln a + b \ln X$$

Llamando a las variables $V = \ln Y$, $U = \ln X$, $A = \ln a$, quedándonos la siguiente expresión a calcular el ajuste lineal: $$V= A + bU$$

\subsubsection{Ajuste exponencial}\vspace{-0.5cm}
\begin{equation*}
    Y=f(X;a,b) = ab^{x}
\end{equation*}

De otra forma, si queremos aplicar el ajuste exponencial hemos de aplicar el logaritmo y obtenemos la siguiente expresión:
$$\ln Y = \ln a + X \ln b$$

Llamando a las variables $V = \ln Y$, $A = \ln a$ y $B = \ln b$, quedándonos la siguiente expresión a calcular el ajuste lineal: $$V= A + BX$$

\subsection{Regresión de tipo I}

Podemos además realizar regresiones de una variable dependiente $Y$ dado el valor $x_i$ de una variable independiente
asociada $X$. Es decir, predecir el comportamiento de la variable condicionada $Y/X=x_i$.\\


Teniendo en cuenta la representatividad de le media en lo que al comportamiento de una variable se refiere, se define la curva de regresión de tipo I de $Y/X$ como la curva que pasa por los puntos $(x_i, \overline{y_i}) \
    \forall i \in \{1, \ldots, k\}$. Análogamente, se defien la curva de regresión de tipo I de $X/Y$ como la curva
que pasa por los puntos $(\overline{x_j}, y_j) \ \forall j \in \{1, \ldots, p\}$.

Estas curvas tienen la propiedad de ser entre todas las funciones las que mejor se ajustan a los datos observados
según el método de mínimos cuadrados. Estas curvas no son de gran utilidad práctica, pues el hecho de conocerla
solamente en puntos aislados hace que sea inútil para le predicción en los demás casos.

\section{Correlación}

El grado de asociación entre las variables nos indicará en qué medida la expresión encontrada mediante la regresión explica una variable en función de la otra. El estudio de la correlación también equivale al estudio de la bondad del ajuste de una curva a una nube de puntos.

Para ello, en primer lugar es importante diferenciar entre los ajustes lineales en los parámetros y los no lineales en los parámetros.

Los que sí son lineales en los parámetros son aquellos a los que no se les ha aplicado ninguna transformación a los parámetros. Ejemplo de estos son los ajustes mediante rectas, parábolas o hipérbolas equiláteras.

Los no lineales en los parámetros implican que a alguno de los parámetros se le ha aplicado alguna transformación. Ejemplos son el ajuste potencial o el exponencial.

\subsection{Varianza residual. Coeficiente de determinación}

El método de mínimos cuadrados toma como medida del error que se comente al ajustar una curva la siguiente medida:
\begin{definicion}[Varianza Residual]

    Se define la varianza residual del ajuste de $Y$ en función de $X$ como:
    $$\sigma_{ry}^2 = \sum_{i=1}^k \sum_{j=1}^p f_{ij} e_{ij}^2 = \sum_{i=1}^k \sum_{j=1}^p f_{ij}(y_j - \hat y_i)^2 = \sum_{i=1}^k \sum_{j=1}^p
    f_{ij} [y_j - f(x_i)]^2$$
\end{definicion}

Dicha cantidad se usa como medida de la bondad del ajuste. Por tanto, cuanto menor sea la varianza resiudal, mejor será el ajuste.
\begin{observacion}
    En funciones lineales de los parámetros la media de los residuos es cero (generalización del lema \ref{lema:2.8}), por lo que la expresión anterior es precisamente la varianza de los residuos, o \underline{varianza residual}.
    
    En funciones no lineales en los
    parámetros, la media de los residuos no es nula, aunque se sigue denominando varianza residual. Por ello, es importante no confundir la varianza residual con la varianza de los residuos en los ajustes no lineales en los parámetros.
\end{observacion}\bigskip


También se define la siguiente medida:
\begin{definicion}[Varianza Explicada]

    Se define la varianza residual de $Y$ como:
    \begin{equation*}
        \sigma_{ey}^2=\sum_{i=1}^k \sum_{j=1}^p f_{ij}
    (\hat y_j - \overline{y})^2
    \end{equation*}
\end{definicion}


Por norma general, se toma como medida del grado de ajuste el coeficiente de determinación.
\begin{definicion}[Coeficiente de determinación]
El coeficiente de determinación, o razón de correlación, es la proporción de la varianza total de la variable $Y$ explicada por la regresión. Esto es, el cociente o razón entre la varianza explicada y la total.
\begin{equation*}
    \eta_{Y/X}^2 = \dfrac{\sigma_{ey}^2}{\sigma_y^2}
\end{equation*}    
\end{definicion}


Por tanto, para comparar todo tipo de ajustes se puede emplear la \textbf{varianza residual} o el \textbf{coeficiente de determinación}, aunque se suele emplear la primera medida.

\subsubsection{Caso concreto de ajustes lineales en los parámetros}

En este caso, como la varianza residual coincide con la varianza de los residuos, se tiene que es posible descomponer la varianza en una suma de la varianza residual y la varianza explicada por la regresión:
$$\sigma_y^2 = \sigma_{ey}^2 + \sigma_{ry}^2$$

En este caso, se tiene que:
$$\eta_{Y/X}^2 := \dfrac{\sigma_{ey}^2}{\sigma_y^2} = 1 - \dfrac{\sigma_{ry}^2}{\sigma_y^2}$$


\subsubsection{Interpretación del coeficiente de correlación}

De la misma expresión se deduce que: $0 \leq \eta_{Y/X}^2 \leq 1$.

\begin{itemize}
    \item $\eta_{Y/X}^2 = 0 \Leftrightarrow \dfrac{\sigma_{ey}^2}{\sigma_y^2}=0 \Leftrightarrow \sigma_{ey}^2 = 0$. Es decir, el modelo no explica nada de $Y$ a partir de $X$. El ajuste es el peor posible que se puede hacer por mínimos cuadrados.
    \item $\eta_{Y/x}^2 = 1 \Leftrightarrow \dfrac{\sigma_{ey}^2}{\sigma_y^2}=1$. Es decir, todos los residuos son nulos y s explica la variable totalmente. El ajuste es perfecto.
    \item Para valores intermedios entre 0 y 1, según estén más próximos a un extremo o a otro nos indicarán un peor o mejor ajuste: Un ajute del 60\% explica que el 60\% de la variabilidad total de $Y$ la explica el modelo propuesto mediante la variable independiente.
\end{itemize}

\subsection{Correlación en el caso lineal}

En este caso, tenemos el siguiente resultado, muy útil para calcular la bondad de los ajustes lineales:
\begin{teo} En el caso de un ajuste lineal, el ajuste de determinación viene dado por:
    \begin{equation*}
        \eta_{Y/X}^2 = \eta_{X/Y}^2 = r^2 = \dfrac{\sigma_{xy}^2}{\sigma_x^2 \sigma_y^2}
    \end{equation*}
\end{teo}
\begin{proof}
    Demostramos para la recta de $Y$ sobre $X$, ya que en el otro caso sería análogo. La recta mencionada tiene por expresión:
    \begin{equation*}
        Y-\overline{y} = \dfrac{\sigma_{xy}}{\sigma_x^2}(X - \overline{x}) 
        \Longrightarrow
        Y = \overline{y} + \dfrac{\sigma_{xy}}{\sigma_x^2}(X - \overline{x})
    \end{equation*}

    Por tanto, la varianza residual en el caso de la recta de regresión es:
    \begin{equation*}\begin{split}
        \sigma_{ry}^2 &
        = \sum_{i=1}^k\sum_{j=1}^p f_{ij}[y_j - f(x_i)]^2
        = \sum_{i=1}^k\sum_{j=1}^p f_{ij}\left[y_j - \left(\overline{y} + \dfrac{\sigma_{xy}}{\sigma_x^2}(x_i - \overline{x})\right)\right]^2 = \\
        & = \sum_{i=1}^k\sum_{j=1}^p f_{ij}\left[(y_j - \overline{y}) - \left(\dfrac{\sigma_{xy}}{\sigma_x^2}(x_i - \overline{x})\right)\right]^2 = \\
        & = \sum_{i=1}^k\sum_{j=1}^p f_{ij}\left[(y_j - \overline{y})^2 + \dfrac{\sigma_{xy}^2}{(\sigma_x^2)^2}(x_i - \overline{x})^2 - 2\frac{\sigma_{xy}}{\sigma_x^2}(y_j-\overline{y})(x_i-\overline{x})\right] = \\
        &= \sigma_y^2 + \frac{\sigma_{xy}^2}{\sigma_x} - 2\frac{\sigma_{xy}^2}{\sigma_x^2}
        = \sigma_y^2 - \frac{\sigma_{xy}^2}{\sigma_x^2}
    \end{split}\end{equation*}

    Por tanto, como en este caso estamos ante un ajuste lineal en los parámetros, tenemos que:
    \begin{equation*}
        \eta_{Y/X}^2 = 1-\frac{\sigma_{ry}^2}{\sigma_y^2}
        = 1-\frac{\sigma_y^2 - \frac{\sigma_{xy}^2}{\sigma_x^2}}{\sigma_y^2}
        = 1-1+\frac{\sigma_{xy}^2}{\sigma_x^2\sigma_y^2} = \frac{\sigma_{xy}^2}{\sigma_x^2\sigma_y^2}
        \qedhere
    \end{equation*}
\end{proof}

Este resultado es de gran ayuda, ya que nos permite calcular $\eta_{Y/X}^2$ de una forma mucho más cómoda.

Es útil calcular la varianza residual en función de $r^2$, ya que el cálculo del segundo es mucho más sencillo. No obstante, es necesario a veces conocer la varianza residual para comparar con modelos no lineales en los parámetros. Por eso, se tiene que:
\begin{equation*}
    \eta_{Y/X}^2 = 1 - \dfrac{\sigma_{ry}^2}{\sigma_y^2}
    = r^2 \Longrightarrow \sigma_{ry}^2 = (1-r^2)\sigma_y^2
\end{equation*}


Por último, se introduce un nuevo coeficiente. La raíz cuadrada del coeficiente de determinación lineal anterior (con el signo de la covarianza) recibe el nombre
de \underline{coeficiente de correlación lineal}:
$$r = \pm \sqrt{r^2} = \dfrac{\sigma_{xy}}{\sigma_x \sigma_y}$$


Dicho coeficiente se usa para determinar el grado de dependencia lineal de la variable dependiente ante los valores
de la variable independiente. Esta dependencia puede ser directa (o positiva) o indirecta (o negativa), según
el signo de la covarianza. Adopta valroes entre $\pm1$ y 0.

\begin{itemize}
    \item \underline{Para la covarianza positiva}:

    Si $r=1$, existirá una dependencia lineal funcional, mientras que si $r=0$ no existirá ninguna dependencia o asociación entre las variables de tipo lineal, aunque sí puede haberla de otra naturaleza, convirtiéndose las rectas de regresión paralelas a los ejes de coordenadas.

    \item \underline{Para la covarianza negativa}:
    Si $r=-1$, existirá una correlación perfecta, con una dependencia funcional lineal, coincidiendo las dos rectas en una sola.
\end{itemize}

Para resumir, diremos que $-1 \leq r \leq 1$. Cuando varía de $-1$ a $0$ estamos en una correlación negativa y la dependencia
será mayor cuanto más se aproxime a $-1$ mientras que si varía de $0$ a $1$, la correlación es positiva y el grado de
dependencia será mayor cuanto más se aproxime a $1$.

\section{Predicciones}
Uno de los objetivos principales de la regresión y correlación es hacer predicciones de la variable dependiente en función de los valores que toma la variable independiente. Las predicciones se efectúan utilizando la función estimada por el método de mínimos cuadrados, $f$. Con la que obtenemos los valores teóricos que ajustan a los observados. La predicción será más fiable cuanto mayor sean los coeficientes de determinación correspondientes o razones de correlación, ya que menor será la varianza de los residuos, que nos indica la cuantía de la separación entre lo observado y estimado.

Hay que tener presente que la fiabilidad de las predicciones disminuye a medida que los valores de la variable independiente se aleja de su recorrido, pues puede que el modelo ajustado no sea válido para dicho valores en la medida dada por $\eta^2$
    \chapter{Compilación y Enlazado de Programas}

\begin{center}
    Cód. fuente (Leng. alto nivel) $\xrightarrow{(*)}$
    Cód. objeto (Cód. máquina o ensamblador)
\end{center}

\begin{definicion} [Compilación] Proceso mediante el cual se pasa de código fuente a código objeto $(*)$. Se emplea para:
    \begin{itemize}
        \item Comprobar que no hay errores en el código fuente.
        \item Generar ficheros objeto.
    \end{itemize}
\end{definicion}

\begin{definicion}[Enlazado]
    Proceso mediante el cual, a partir de los ficheros objeto, se obtienen los ficheros ejecutables.
\end{definicion}
\section{Gramática}
\begin{definicion} [Gramática]
    La gramática $G=\{V_N, V_T, P, S\}$ está formada por:
    \begin{enumerate}
        \item \underline{$V_N$ o símbolos no terminales}:\\
        Aquellos símbolos auxiliares que podemos usar para operar con la gramática.

        \item \underline{$V_T$ o símbolos terminales}:\\
        Aquellos símbolos que podemos usar al programar.

        \item \underline{$P$ o reglas de producción}:\\
        Combinaciones válidas de los símbolos.

        \item \underline{$S$ o axioma}:\\
        Uno de los símbolos no terminales que se usa como símbolo inicial.
    \end{enumerate}
\end{definicion}

\begin{ejemplo}
    Sea $G=\left(\{0,1\},\{S\}, S, P \right)$, con $P$:
    \begin{equation*}
        P=\{S::=\footnote{Aquí se ha empleado la notación de Backus.} \;0 | 1 | 0S1\}
    \end{equation*}
    Por tanto, tengo tres reglas de producción.

    \begin{itemize}
        \item $S\longrightarrow 0$: Sí es válido, usando la primera regla.
        \item $S\longrightarrow 1$: Sí es válido, usando la segunda regla.
        \item $S\longrightarrow 101$: No es válido, ya que no es posible que empiece con el 1.
    \end{itemize}
\end{ejemplo}

\begin{ejercicio}
La gramática definida por $G=(\{0,1,2,3,4,5,6,7,8,9\},\{N,C\}, N, P)$, con:
    \begin{equation*}
        P = \{N::=NC | C,
        \quad C::=[0-9]\}
    \end{equation*}

Esta gramática puede generar los naturales, pero también admite los 0 no significativos. Modificar para que no admita los 0 no significativos.\\

Las reglas de producción serían:
\begin{equation*}
        P = \{N::=D | ND | N0,
        \quad D::=[1-9]\}
    \end{equation*}
\end{ejercicio}

\begin{definicion}[Gramática Ambigua] Una gramática es ambigua cuando admite más de un árbol sintáctico para una misma secuencia de símbolos de entrada.

Ejemplo de esto es la precedencia de los operadores en un lenguaje de programación, ya que se usan los paréntesis para evitar la ambigüedad.
\end{definicion}

\begin{definicion}[Gramática libre de contexto]
    Se dice que una gramática es libre de contexto cuando en el lado izquierdo de cada regla de producción solo puede haber un símbolo no terminar. Formalmente, cada producción es de la forma:
    \begin{equation*}
        A\longrightarrow \alpha \qquad A\in V_N \qquad \alpha \in (V_N\cup V_T)^\ast
    \end{equation*}
    donde $(V_N\cup V_T)^\ast$ representa todas las combinaciones posibles de dichos conjuntos.

    Se dice que es libre de contexto porque $A$ se puede sustituir por $\alpha$ independientemente del contexto en el que aparezca.
\end{definicion}





\section{Traducción}
\begin{definicion}[Traductor]
    Un traductor es un programa que recibe como entrada un texto en un lenguaje de programación concreto y produce, como salida, un texto en lenguaje máquina equivalente.
\end{definicion}

Existen dos tipos de traductores, los compiladores y los intérpretes.
\subsection{Compilador}
\begin{definicion}[Compilador]
    Un compilador traduce la especificación de entrada (archivos fuente) a lenguaje máquina incompleto (archivos objeto) y con instrucciones máquina incompletas. Por tanto, se necesita un complemento llamado enlazador.
\end{definicion}
\begin{definicion}[Enlazador/Linker]
    El linker completa los programas ligando las instrucciones máquina necesarias y genera un programa ejecutable para la máquina real.
\end{definicion}

\subsection{Intérprete}
\begin{definicion}[Intérprete]
    Un intérprete hace que un programa fuente escrito en un lenguaje vaya, sentencia a sentencia, traduciéndose y ejecutándose directamente por el computador. Cabe destacar que no se genera ningún archivo objeto ni equivalente al descrito en el compilador.
\end{definicion}

Algunas ventajas del intérprete son:
\begin{itemize}
    \item Es más fácil detectar errores, ya que suele ser posible detenerlo para conocer los valores de las variables. Esto solo sería posible en otro caso con un \textit{debugger}.

    \item Es más pedagógico.
\end{itemize}

No obstante, los inconvenientes son:
\begin{itemize}
    \item Cada vez que se ejecute, se ha de interpretar de nuevo, ya que no se genera archivo objeto. Con un compilador, aunque la traducción sea más lenta, solo ha de realizarse una vez.
    \item Una instrucción que se encuentre en un bucle se ha de interpretar tantas veces como se ejecute el bucle.
    \item La optimización solo se puede realizar línea a línea, no se puede realizar a nivel del programa completo.
\end{itemize}


Un ejemplo de intérprete es Bash.


\subsection{Fases en el proceso de la traducción}
En primer lugar, ocurre la fase de análisis del código. Este se realiza en tres pasos: léxico, sintáctico y semántico. Posteriormente, en la fase de síntesis, se optimiza y se genera el código objeto.
\begin{enumerate}
    \item \textbf{Fase de Análisis.}
    \renewcommand{\theenumii}{\theenumi.\arabic{enumii}}
    \begin{enumerate}
        \item \underline{Análisis Léxico}

        Para entender esta etapa, son necesarios los siguientes conceptos:
        
        \begin{definicion}[Lexema/Palabra] Es un conjunto de caracteres del alfabeto que tienen significado propio. 
        \end{definicion}

        \begin{definicion}[Token] Es el concepto asociado a un conjunto de lexemas o palabras que, según la gramática del lenguaje fuente, tienen la misma misión sintáctica.

        En el caso del lenguaje español, un token podría ser los determinantes artículos, ya que tienen la misma función sintáctica independientemente de la oración.

        En el caso de un lenguaje de programación, un token podría ser ``identificador'' asociado a los nombres de las variables o funciones u ``operador aritmético'' asociado a estos operadores.

        Los tokens asociados a más de una palabra (la mayoría) deben ir acompañados del lexema reconocido anteriormente. Este es el denominado \underline{atributo}, y es necesario para las fases posteriores de la traducción. Por ejemplo, el token que recoja los operadores es necesario que tenga el atributo del operador en sí, ya que a la hora de la traducción será necesario saber si se trata de una multiplicación o una división, por ejemplo.
        \end{definicion}

        \begin{definicion}[Patrón]
            Es una descripción de la forma que pueden tomar los lexemas de un token. Se suelen emplear expresiones regulares.
            
            En el caso de una palabra clave como token, el patrón es sólo la secuencia de caracteres que forman dicha palabra clave. Para los identificadores y algunos otros tokens, el patrón es una estructura más compleja (normalmente dada con una expresión regular).
        \end{definicion}
        

        \vspace{1cm}
        Por tanto, la \textbf{función} del analizador léxico es leer el texto de origen, identificar lexemas, asociarles el token al que pertenecen pero, además, debe eliminar los comentarios y caracteres superfluos existentes en el texto de entrada (espacios en blanco, tabuladores y retornos de carro). La equivalencia entre cada lexema con su token se guardan en la \textbf{tabla de símbolos}, que es donde su guarda la información.

        \begin{ejemplo} Ejemplos de tokens son:
        \begin{itemize}
            \item \textbf{IF}: lexema asociado \textit{if}.
            \item \textbf{ELSE}: lexema asociado \textit{else}.
            \item \textbf{IDENT}: lexemas asociados \textit{pi}, \textit{dato3} o \textit{i}, por ejemplo.
        \end{itemize}

        Se producirá un \textbf{error léxico} cuando el carácter de la entrada no tenga asociado a ninguno de los patrones disponibles en nuestra lista de tokens. Por ejemplo, un carácter extraño en la formación de una palabra reservada, como \textit{wh\textbf{ñ}le}.
            
        \end{ejemplo}



        \item \underline{Análisis Sintáctico}

        Tiene como objetivo analizar las secuencias de tokens y comprobar que son correctas sintácticamente.
        
        A partir de una secuencia de tokens el analizador sintáctico nos devuelve el orden en el que hay que aplicar las producciones de la gramática para obtener la secuencia de entrada; es decir, el árbol sintáctico abstracto en el que aparece el token con el atributo. Este se guarda también en la tabla de símbolos.

        Se produce un \textbf{error sintáctico} cuando no se puede llegar desde el axioma hasta la palabra buscada; es decir, no se puede construir el árbol sintáctico. Ejemplo de esto podría ser, por ejemplo, paréntesis mal balanceados.




        \item \underline{Análisis Semántico}

        La semántica de un lenguaje de programación es el significado dado a las distintas construcciones sintácticas. En los lenguajes de programación, el significado está ligado a la estructura sintáctica de las sentencias.

        En el caso de que no se produzcan errores, actualiza en la tabla de símbolos el árbol semántico abstracto resuelto; es decir, el árbol semántico con los lexemas y su significado.

        Se producen \textbf{errores semánticos} cuando se detectan construcciones sin un significado correcto (p.e. variable no declarada, tipos incompatibles en una asignación, llamada a un procedimiento con número de argumentos incorrectos, \dots).
    \end{enumerate}
    \item \textbf{Fase de Síntesis}
    \begin{enumerate}
        \item \underline{Generación de código}
        En esta fase se genera un archivo con un código en lenguaje objeto (generalmente lenguaje máquina) con el mismo significado que el texto fuente.

        Es posible la generación de código intermedio para facilitar el proceso.

        \item \underline{Optimización de código}

        Esta fase existe para mejorar el código mediante comprobaciones locales a un grupo de instrucciones (bloque básico) o a nivel global. Se suele realizar en el código intermedio.

        Ejemplo de esto es una asignación del tipo $b=7.3$ dentro de un bucle $for$. En esta fase se sacará dicha instrucción del bucle.
    \end{enumerate}
\end{enumerate}


\section{Modelos de Memoria de un Proceso}
\subsection{Tipos de Datos (desde el punto de vista de su implementación en memoria)}

\begin{itemize}
    \item Datos estáticos - existen a lo largo de toda la vida del programa.
    \begin{itemize}
        \item \underline{Según el ámbito de visibilidad}.

        Las globales a todo el programa se encuentran fuera del main y no dependen de ningún archivo.
        
        Las de módulo se especifican con \verb|static| (del módulo en cuestión) o \verb|extern| de otro módulo, siendo un módulo cada uno de los archivos.
        
        Las de bloque pertenecen a una parte del programa delimitada por las llaves.

        Los datos estáticos de clase o función pertenecen a cada clase o función en concreto.

        \item \underline{Constantes o variables}.

        Según si se pueden modificar o no. Las constantes se almacenan en un espacio de la memoria de solo lectura, mientras que las variables han de poder modificarse también.

        \item \underline{Con o sin valor inicial}.
    \end{itemize}

    \item Datos dinámicos asociados a la ejecución de una función:

    Se almacenan en la pila \textit{(stack)} y se crean al ser llamada la función y se destruyen cuando esta termina.

    Corresponde a los datos locales y a los parámetros de una función.


    \item Datos dinámicos controlados por el programa.

    Se almacenan en el \textit{heap} (zona de memoria usada tiempo de ejecución para albergar los datos no conocidos en tiempo de compilación). En C++, se controlan mediante los operadores \verb|new| y \verb|delete|. Generalmente se gestionan mediante punteros.

    Su tiempo de vida no esta vinculado a la activación de una función sino bajo el control directo del programa que los crea cuando los necesita.
\end{itemize}

\begin{definicion}[Código Independiente de la Posición \textit{(PIC, Position Independent Code)}]

Un fragmento de código cumple esta propiedad si puede ejecutarse en cualquier parte de la memoria.

Es necesario que todas sus referencias a instrucciones o datos no sean absolutas sino relativas en función del valor de un registro.

Por ejemplo, una instrucción puede ser \verb|MOV R4, 32+PC|. Dependiendo del valor del \verb|PC|, se almacenará el valor de $R4$ en posiciones distintas.
\end{definicion}



\section{Ciclo de Vida de un Programa}

Una vez que el programador ha finalizado la escritura del programa, éste debe pasar por varias fases antes de que pueda ejecutarse. Estas son:
\begin{enumerate}
    \item Preprocesado (archivos \verb|.i| en C).

    Se procesan los includes, las directivas de preprocesador, etc.

    \item Compilación (archivos \verb|.s| en C).

    Se genera el código en lenguaje ensamblador.

    \item Ensamblado (archivos \verb|.o| en C).

    El código ensamblador se traduce por el \textit{assembler} a código máquina. Las referencias a símbolos que no están definidos en el módulo quedan pendientes de resolver.

    \item Enlazado (archivos .exe y a.out).

    El enlazador/\textit{linker} se encarga de, a partir de los archivos objeto y las bibliotecas, resolver las referencias pendientes y generar el archivo ejecutable.

    Como diferencia entre los archivos ejecutables y los archivos objeto, tenemos principalmente que el ejecutable contiene una cabecera en la que se indica el punto de inicio del mismo, es decir, la primera dirección que se cargará en el PC.

    \item Carga y ejecución.

    Por tanto el cargador/\textit{loader} es el que ayuda a la asignación y carga del programa como un proceso en MP en estado nuevo.
\end{enumerate}

\subsection{Compilación}
En esta fase, el compilador procesa cada uno de los archivos de código fuente para generar el correspondiente archivo objeto. Se realizan las siguientes acciones:
\begin{enumerate}
    \item Genera el código objeto y determina cuanto espacio ocupan los diferentes tipos de datos.

    \item Asigna direcciones a los \underline{símbolos estáticos definidos en el módulo}.
    
    Estas son consecutivas: en primer lugar van las constantes, luego las variables con valor inicial, y por último las que no tiene valor inicial.

    Como son estáticas y permanecen durante todo el programa, se puede asignar la dirección directamente.

    \item Resuelve las referencias a los \underline{símbolos estáticos externos definidos en el módulo}.

    Al ser estáticos, en el proceso de compilación ya tienen una dirección asignada. Al ser del mismo módulo, no se requiere aún del enlazador.

    Las referencias pueden resolverse mediante un direccionamiento absoluto (será necesario reubicación) o relativo al $PC$ (estaremos ante un PIC).

    \item Las referencias a los símbolos estáticos externos definidos fuera del módulo se resolverán en el enlazado.

    \item Resuelve las referencias a los \underline{símbolos dinámicos almacenados en la pila}.

    Se resuelven mediante direccionamiento relativo a pila. Al no aparecer en el fichero objeto (se generan en tiempo de ejecución), no requieren reubicación.

    \item Resuelve las referencias a los \underline{símbolos dinámicos almacenados en el \textit{heap}}.

    Se resuelven mediante direccionamiento indirecto mediante punteros. Al no aparecer en el fichero objeto (se generan en tiempo de ejecución), no requieren reubicación.

    \item Genera la Tabla de símbolos e información de depuración.
\end{enumerate}


\subsection{Enlazado}
El enlazador \textit{(linker)} debe agrupar los archivos objetos de la aplicación y las bibliotecas, y resolver las referencias entre ellos. Concretamente, se llevan a cabo las siguientes tareas:
\begin{enumerate}
    \item Se completa la etapa de resolución de símbolos.

    \item Se agrupan en regiones las zonas de las mismas características de los módulos .

    Es decir, todas la parte de código de todos los módulos se agrupa en una región; y lo mismo ocurre con datos inicializados y los no inicializados.

    Esto reduce el número de regiones que ha de gestionar el Sistema Operativo.

    \item Se realiza la reubicación de módulos formando regiones, ya que hay que transformar las referencias dentro de un módulo a referencias dentro de las regiones.
\end{enumerate}

Hay distintos tipos de enlazado, que determinan el ámbito de cada dato.
\begin{itemize}
    \item \underline{Enlazado externo.}
    
    Cada vez que aparece un identificador con enlazado externo representa el mismo objeto o función a través del total de ficheros y librerías que componen el programa. Por tanto, esto equivale a tener visibilidad global.

    \item \underline{Enlazado interno.}
    
    Cada vez que aparece un identificador con enlazado interno representa el mismo objeto o función solo dentro del mismo fichero. Los objetos con el mismo nombre en otros ficheros son objetos distintos. Por tanto, esto equivale a tener visibilidad de fichero.

    \item \underline{Sin enlazado.}
    
    Cada identificador sin enlazar representa unidades únicas. Los objetos con el mismo nombre en otros bloques son objetos distintos. Por tanto, esto equivale a tener visibilidad de bloque. Identificadores sin enlazado son:
    \begin{itemize}
        \item Cualquier identificador distinto a un objeto o función.
        \item Parámetros de funciones.
        \item Objetos de ámbito de bloque (entre llaves) sin el especificador \verb|extern|. 
    \end{itemize}
\end{itemize}


\section{Bibliotecas}
\begin{definicion}[Biblioteca] Colección de objetos, normalmente relacionados entre sí. Favorecen modularidad y reusabilidad de código.
\end{definicion}

Las bibliotecas se pueden clasificar según la forma en la que se enlazan:
\begin{enumerate}
    \item \underline{Bibliotecas Estáticas}

    Tienen extensión \verb|.a|, y que se ligan con el programa en el proceso de enlazado. El archivo ejecutable las contiene.

    Algunos inconvenientes que tiene son:
    \begin{itemize}
        \item El código de la biblioteca está en todos los ejecutables que la usan, lo que desperdicia disco y memoria principal.

        \item Si actualizamos una biblioteca estática, debemos recompilar los programas que la usan para que se puedan beneficiar de la nueva versión.

        \item Producen ejecutables grandes.
    \end{itemize}

    \item \underline{Bibliotecas Dinámicas}

    Por norma general, tienen extensión \verb|.so| y se integran con los procesos en tiempo de ejecución. En el proceso de montaje, se incluye un módulo de montaje dinámico (\textit{enlazador dinámico}) que se encarga de cargar y montar las bibliotecas dinámicas usadas por el programa durante su ejecución.

    El archivo correspondiente a una biblioteca dinámica se diferencia de un archivo ejecutable en los siguientes aspectos:
    \begin{enumerate}
        \item Contiene información de reubicación.
        \item Contiene una tabla de símbolos propios de la biblioteca.
        \item En la cabecera no se almacena información de punto de entrada.
    \end{enumerate}
\end{enumerate}
    \chapter{Aproximación}
\section{Motivación}
\noindent
El objetivo de esta sección es el de, dada una función $f(x)$, aproximarla mediante otra función de forma
que minimicemos el área que forman las gráficas de ambas funciones. Para evitar el realizar integrales con
valores absolutos, lo sustituiremos por elevar el área al cuadrado.\\

\noindent
\textbf{Problema de aproximación.} Dados $N$ puntos $\{(x_0, y_0), (x_1, y_1), \ldots, (n_N, y_N)\}$, el problema de aproximar
(ajustar) dichos puntos consiste en encontrar una función $g(x)$, con unas ciertas condiciones, que esté lo más cerca posible
de los puntos.\newline
Esto es, se busca una función $g(x)$ que minimice la distancia a los puntos $(x_i,y_i)~~i=0,1,\ldots, N$.\\

\noindent
Al igual que pasaba con la interpolación, buscamos funciones $g(x)$ que posean propiedades deseables:
\begin{itemize}
    \item Fáciles de implementar.
    \item Fáciles de evaluar.
    \item Simples de calcular.
    \item Suficientemente regulares.
    \item \ldots
\end{itemize}

\noindent
En los siguientes apartados formalizaremos los conceptos de distancia y aproximación.

\section{Producto escalar}
\begin{definicion}[Producto escalar]
    Sea $V$ un espacio vectorial, definimos un producto escalar (o forma bilineal simétrica definida positiva)
    como una aplicación:
    $$\begin{array}{ccccc}
            \langle\cdot,\cdot\rangle & : & V \times V & \longrightarrow & \R                  \\
                                      &   & (u,v)      & \longmapsto     & \langle u,v \rangle
        \end{array}$$

    \noindent
    Que cumple las siguientes propiedades:
    \begin{enumerate}
        \item $\langle v,v\rangle \geq 0~~~~\forall v \in V$
        \item $\langle v,w\rangle = \langle w,v\rangle~~~~\forall v,w \in V$
        \item $\langle \alpha v + \beta w, z\rangle = \alpha \langle v,z\rangle + \beta \langle w,z \rangle~~~~\forall \alpha, \beta \in \R, v,w,z \in V$
        \item $\langle v,v\rangle = 0 \Leftrightarrow v=0 \  \forall v \in V$
    \end{enumerate}

    \noindent
    A un espacio con un producto escalar, $(V, \langle\cdot,\cdot\rangle)$ se le denomina \textbf{espacio con producto escalar}.
\end{definicion}


\begin{ejemplo}Es fácil ver que los siguientes son productos escalares:
\begin{enumerate}
    \item Producto escalar en $\R^n$:

    Sean $u = (u_1, u_2, \ldots, u_n), v=(v_1, v_2, \ldots, v_n) \in V$

    Definimos $\langle u,v\rangle = u_1 v_1 + u_2 v_2 + \ldots + u_n v_n$

    \item Sea $[a,b] \subset \R$ con $a \neq b$, $V=\mathcal{C}([a,b])$ el espacio vectorial de las funciones continuas en $[a,b]$.
    Definimos $\forall f,g \in V$:
    $$\langle f,g\rangle = \int_a^b f(x)g(x)~dx$$

    \item Sea $V=\mathcal{C}([a,b])$, tratamos de definir un producto escalar de la siguiente forma:
    Sean $x_i \in [a,b]~~\forall i \in \{1, \ldots, N\}$. Definimos $\forall f,g \in V$:
    $$\langle f,g\rangle = \sum_{i=1}^N f(x_i)g(x_i)$$
    
    \noindent
    Nos centraremos en comprobar si este producto escalar es válido para un espacio de polinomios.\newline
    Notemos que $\langle f,f \rangle = 0 \Leftrightarrow \sum\limits_{i=1}^N f(x_i)f(x_i)=0 \Leftrightarrow x_i$ es raíz de $f$
    $\forall i \in \{1, \ldots, N\}$\\
    
    \noindent
    Por lo que en $\bb{P}_k$ con $k \geq N \Rightarrow$ podemos tener:
    $$f(x) = \prod_{i=1}^N(x-x_i)$$
    Con $f\neq0 \ \land \ \langle f,f \rangle = 0$, que no define un producto escalar.\\
    
    \noindent
    Sin embargo, en los espacios $\bb{P}_k$ con $k < N$, nuestra aplicación sí define un producto escalar, ya que:
    $$\langle f,f \rangle = 0 \Leftrightarrow \sum_{i=1}^N (f(x_i))^2 = 0 \Leftrightarrow f=0$$
    (Es rutinario comprobar que el producto escalar así definido verifica además las propiedades 1), 2) y 3), por lo que se deja
    al lector a modo de ejercicio su demostración).
\end{enumerate}
\end{ejemplo}

\bigskip
\section{Norma}
\begin{definicion}[Norma]
    Una norma es una aplicación:
    $$\begin{array}{ccccc}
            \|\cdot\| & : & V & \longrightarrow & \R    \\
                      &   & v & \longmapsto     & \|v\|
        \end{array}$$
    Que cumple las siguientes propiedades:
    \begin{enumerate}
        \item $\|v\| \geq 0~~~~\forall v \in V$
        \item $\|v+w\| \leq \|v\|+\|w\|~~~~\forall v,w \in V$
        \item $\|\lambda v\| = |\lambda| \|v\|~~~~\forall \lambda \in \R~~\forall v \in V$
        \item $\|v\|=0 \Leftrightarrow v=0$
    \end{enumerate}

    \noindent
    Un espacio vectorial en el que hay definida una norma se denomina \textbf{espacio vectorial normado}.
\end{definicion}

\begin{teo}[Desigualdad de Cauchy-Schwarz]
    Sea $(V, \langle \cdot , \cdot \rangle)$ un espacio con producto escalar, se verifica:
    $$|\langle u,v \rangle| \leq \|u\|\|v\|~~~~\forall u,v \in V$$
\end{teo}
\begin{proof}
    Sea $\lambda \in \R$, $\forall u,v \in V \Rightarrow \lambda u + v \in V$.
    $$0 \leq \|\lambda u + v\|^2 = \langle \lambda u + v, \lambda u + v \rangle = \lambda^2 \langle u,u \rangle + 2\lambda
        \langle u,v \rangle + \langle v,v \rangle=$$
    $$= \lambda^2 \|u\|^2 + 2\lambda \langle u,v \rangle + \|v\|^2 \geq 0 ~~\forall \lambda \in \R \Leftrightarrow \Delta \leq 0 $$
    $$\Delta = 4\langle u,v \rangle^2 - 4\|u\|^2 \|v\|^2 \leq 0 \Leftrightarrow \langle u,v \rangle^2 \leq \|u\|^2\|v\|^2 \Leftrightarrow$$
    $$\Leftrightarrow \sqrt{\langle u,v \rangle^2} \leq \sqrt{\|u\|^2\|v\|^2} \Leftrightarrow |\langle u,v \rangle| \leq \|u\|\|v\|$$
\end{proof}

\begin{ejemplo} La aplicación de dicha desigualdad con diferentes productos escalares es:
\begin{enumerate}
    \item Producto escalar en $\R^n$:
$$\forall u=(u_1, \ldots, u_n), v=(v_1, \ldots, v_n) \in V~~~~\langle u,v \rangle^2 \leq \|u\|^2\|v\|^2$$
$$(u_1v_1 + u_2v_2 + \ldots + u_nv_n)^2 \leq (u_1^2 + u_2^2 + \ldots + u_n^2)(v_1^2 + v_2^2 + \ldots + v_n^2)$$

    \item \item Sean $f,g\in \cc{C}([a,b])$. Se define $\langle f,g\rangle = \int_a^b f(x)g(x)dx$.

    La desigualdad se escribiría:
    \begin{equation*}
        \left[\int_a^b f(x)g(x)dx\right]^2 \leq \left[\int_a^b f^2(x)dx\right]\left[\int_a^b g^2(x)dx\right]
    \end{equation*}

Tomando $f=1$:
$$\left[ \int_a^b g(x)~dx \right]^2 \leq \int_a^b 1~dx \int_a^b (g(x))^2~dx = (b-a) \int_a^b (g(x))^2~dx$$
\end{enumerate}
    
\end{ejemplo}

\begin{ejercicio}
    Demostrar la desigualdad triangular desde la desigualdad de Cauchy-Schwarz.
    \begin{multline*}
        ||u+v||^2 = \langle u+v,u+v \rangle = ||u||^2 + ||v||^2 +2\langle u,v\rangle \leq ||u||^2 + ||v||^2 +2|\langle u,v\rangle| {\leq}\\ \stackrel{C-S}{\leq} ||u||^2 + ||v||^2 +2||u||||v|| = (||u|| + ||v||)^2
    \end{multline*}

    Tomando raíces cuadradas, tenemos que:
    \begin{equation*}
        ||u+v|| \leq ||u|| + ||v||
    \end{equation*}
\end{ejercicio}

\begin{teo}[Norma inducida]
    En todo espacio vectorial con producto escalar $(V,\langle \cdot,\cdot \rangle)$, podemos definir una norma como sigue:\newline
    Sea $v \in V$, definimos su norma como:
    $$\|v\| = \sqrt{\langle v, v\rangle} \in \R_0^{+}$$
\end{teo}
\begin{proof}
    Claramente la norma así definida es una aplicación $\|\cdot\|:V \rightarrow~\R$.Veamos que cumple las propiedades
    mencionadas en la definición:
    \begin{enumerate}
        \item $$\|v\| = \sqrt{\langle v,v \rangle} \geq 0~~~~\forall v \in V$$
        \item $$\forall u,v \in V: \|u+v\|^2 = \langle u+v, u+v \rangle = \|u\|^2 + 2\langle u,v \rangle + \|v\|^2 \leq$$

        Luego: $$\|u+v\| \leq \|u\|+\|v\|$$
        \item $\forall \lambda \in \R$,$\forall v \in V$:
    $$\|\lambda v\| = \sqrt{\langle \lambda v,\lambda v \rangle} = \sqrt{\lambda^2 \langle v,v \rangle} = |\lambda| \sqrt{\langle v,v \rangle}
        = |\lambda| \|v\|$$

        \item $$\|v\| = 0 \Leftrightarrow \sqrt{\langle v,v \rangle} = 0 \Leftrightarrow \langle v,v \rangle = 0 \Leftrightarrow v=0$$
    \end{enumerate}
\end{proof}

\section{Distancia}
\begin{definicion}[Distancia]
    Una distancia es una aplicación:
    $$\begin{array}{ccccc}
            d & : & V \times V & \longrightarrow & \R     \\
              &   & (u,v)      & \longmapsto     & d(u,v)
        \end{array}$$
    Que cumple las siguientes propiedades:
    \begin{enumerate}
        \item $d(u,v) \geq 0~~~~\forall u,v \in V$
        \item $d(u,v) = d(v,u)~~~~\forall u,v \in V$
        \item $d(u,v) \leq d(u,w) + d(w,v)~~~~\forall u,v,w \in V$
        \item $d(u,v)=0 \Leftrightarrow u=v$
    \end{enumerate}
    
    \noindent
    Un espacio vectorial en el que hay definida una distancia se denomina \textbf{espacio vectorial métrico}.
\end{definicion}

\begin{teo}[Distancia inducida]
    Sea $(V,\langle \cdot,\cdot \rangle)$ un espcio vectorial con producto escalar, podemos definir una distancia como sigue:
    $$d(u,v) = \|u-v\| = \sqrt{\langle u-v,u-v \rangle}~~~~\forall u,v \in V$$
    Aplicando este teorema y el anterior deducimos que todo espacio vectorial con producto escalar $(V,\langle \cdot, \cdot \rangle)$
    es normado y, por tanto, métrico.
\end{teo}
\begin{proof}
    Claramente la distancia así definida es una aplicación $d:V\times V \rightarrow~\R$. Veamos que cumple las propiedades mencionadas en la definición:
    \begin{enumerate}
        \item $$d(u,v) = \|u-v\| \geq 0~~~~\forall u,v \in V$$
        \item $$d(u,v) = \|u-v\| = |-1| \|u-v\| = \|v-u\| = d(v,u)~~~~\forall u,v \in V$$
        \item \begin{multline*}
            d(u,v) = \|u-v\| = \|u-w+w-v\| \leq \|u-w\| + \|w-v\| = d(u,w) + d(w,v)\\\forall u,v,w \in V
        \end{multline*}
    \item $$d(u,v)=0 \Leftrightarrow \|u-v\| = 0 \Leftrightarrow u-v=0 \Leftrightarrow u=v$$
    \end{enumerate}
\end{proof}

\section{Mejor aproximación}
\begin{definicion}[Mejor aproximación]
    Sea $(V, \langle \cdot, \cdot \rangle)$ y $U \subset V$ un subconjunto de $V$.
    Sea $f \in V$. Se dice que $u\in U$ es una mejor aproximación (m.a.) de $f$ en $U$ sii:
    $$d(f,u) =: d(f,U) = \|f-u\| = \inf\{d(f,v) \mid v \in U\}$$
\end{definicion}

\bigskip
\noindent
Nos planteamos a continuación las siguientes cuestiones:
\begin{itemize}
    \item ¿Existe siempre la mejor aproximación?\newline
          No, en el caso de un círculo sin la circunferencia, no existe la mejor aproximación a un punto exterior:
          $$U = \{(x_1,x_2)\in \R^2 \mid x_1^2 + x_2^2 < 1\}~~~~f \in V\setminus U$$
    \item ¿Es única la mejor aproximación?\newline
          No, por ejemplo, si consideramos uan circunferencia, la mejor aproximación a su centro es cada uno de los puntos que
          componen la circunferencia y, por tanto, hay infinitas mejores aproximaciones:
          $$U = \{(x_1, x_2)\in \R^2 \mid x_1^2 + x_2^2 = 1 \}~~~~f = (0,0)$$
\end{itemize}

\noindent
Notemos que minimizar el conjunto (Sea $U$ un subconjunto de $V$)
$$\{d(f,v) \mid v \in U\} = \{\sqrt{\langle f-v, f-v \rangle} \mid v \in U\}$$
es equivalente a minimizar el conjunto:
$$\{d(f,v)^2 \mid v \in U\} = \{\langle f-v, f-v \rangle \mid v \in U\}$$

\noindent
Que es más sencillo de tratar ante la ausencia de la raíz. A este problema se le llama \textbf{aproximación por mínimos cuadrados}.

\begin{definicion}[Ortogonalidad]
    Sean $u,v \in V$, se dice que son ortogonales si:
    $$\langle u,v \rangle = 0$$
    Notado: $u \perp v$.
\end{definicion}

\begin{prop}[Teorema de Pitágoras]
    Sean $u,v\in (V,\langle,\rangle)$.
    \begin{equation*}
        \langle u,v\rangle = 0 \Longrightarrow ||u+v||^2 = ||u||^2 + ||v||^2
    \end{equation*}
\end{prop}
\begin{proof}
    Tenemos que:
    \begin{equation*}
        ||u+v||^2 = \langle u+v,u+v \rangle = ||u||^2 + ||v||^2 +2\cancelto{0}{\langle u,v\rangle} = ||u||^2 + ||v||^2
    \end{equation*}
\end{proof}

\begin{teo}[Caracterización de la mejor aproximación]
    Sea $V$ un espacio con producto escalar, y $U$ un subespacio de $V$. Dada $f\in V$, un elemento $u\in U$ es mejor aproximación de $f$ en $U$ si y solo si:
\begin{equation*}
    \langle f-u,w\rangle = 0\qquad \forall w\in V
\end{equation*}
\end{teo}
\begin{proof}
    Procedemos mediante doble implicación:
    \begin{description}
        \item [$\Longleftarrow$)] Para todo $v\in U$, se cumple:
        \begin{multline*}
            ||f-v||^2 = ||(f-u)+(u-v)||^2 = ||f-u||^2 + ||u-v||^2 +2\cancelto{0}{\langle f-u,u-v\rangle} = \\=
            ||f-u||^2 +||u-v||^2 \leq 0 \Longrightarrow ||f-v||^2 \geq ||u-v||^2 \qquad \forall v\in V
        \end{multline*}

        donde he aplicado que $U$ es un sucespacio vectorial, por lo que $u-v\in U$, y por tanto $\langle f-u,u-v\rangle = 0$ por hipótesis.

        Por tanto, tenemos que $||u-v||\leq ||f-v|| \qquad \forall v\in V$, por lo que $u$ es la mejor aproximación en $U$ de $f$.

        \item [$\Longrightarrow$)] Por ser $u$ la mejor aproximación de $f$, tenemos que:
        \begin{equation*}
            ||f-u||\leq ||f-w|| \Longrightarrow ||f-u||^2\leq ||f-w||^2 \qquad \forall w\in U
        \end{equation*}

        Tomamos $v\in U$, y sea $w=u+\lambda v \in U \mid \lambda \in \bb{R}$. Por tanto, como $w\in U$, tenemos que:
        \begin{equation*}
            ||f-u||^2\leq ||f-u-\lambda v||^2 = \langle f-u-\lambda v, f-u-\lambda v\rangle = ||f-u||^2 -2\lambda \langle f-u,v\rangle +\lambda^2 ||v||^2
        \end{equation*}

        Por tanto,
        \begin{equation*}
            0 \leq -2\lambda \langle f-u,v\rangle +\lambda^2 ||v||^2 
            = \lambda (\lambda ||v||^2 -2 \langle f-u,v\rangle) 
            \qquad \forall \lambda\in \bb{R}, \forall v\in U.
        \end{equation*}

        Las raíces de dicha parábola en la incógnita $\lambda \in \bb{R}$ son:
        \begin{equation*}
            \lambda_1 = 0 \qquad \lambda_2 = \frac{2\langle f-u,v\rangle}{||v||^2}
        \end{equation*}

        Por tanto, como es siempre $\geq 0$, tenemos que las dos raíces son iguales. Por tanto, $\langle f-u,v\rangle = 0$.
    \end{description}
\end{proof}


\subsection{Método para el cálculo de la mejor aproximación}
\begin{teo}[existencia y unicidad de la mejor aproximación]
    Sea $(V, \langle \cdot, \cdot \rangle)$ y $U \subseteq V$ subespacio vectorial de $V$ de dimensión finita, entonces la mejor
    aproximación existe y es única.
\end{teo}
\begin{proof}
    Buscamos $u \in U = \cc{L}\{\varphi_0, \varphi_1, \ldots, \varphi_m\} \Rightarrow \exists a_0, a_1, \ldots, a_m \in \R$ tal que:
    $u = a_0 \varphi_0 + a_1 \varphi_1 + \ldots + a_m \varphi_m$. Buscamos calcular $a_i~~\forall i \in \{0, \ldots, m\}$:\\

    \noindent
    Se tiene que $u$ es la mejor aproximación de $f \in V$ en $U$:
    $$\Leftrightarrow \langle f-u, v\rangle = 0~~\forall v \in U \Leftrightarrow \langle f-u, \varphi_k\rangle = 0~~\forall k \in \{0, \ldots, m\} \Leftrightarrow$$
    $$\Leftrightarrow \langle f-u, \varphi_k\rangle = \langle f-(a_0\varphi_0 + a_1\varphi_1 + \ldots + a_m\varphi_m),\varphi_k\rangle = $$
    $$ =\langle f,\varphi_k \rangle -a_0 \langle \varphi_0,\varphi_k \rangle - a_1\langle \varphi_1,\varphi_k \rangle - \ldots - a_m \langle
        \varphi_m,\varphi_k \rangle = 0 \Leftrightarrow$$
    $$\Leftrightarrow a_0 \langle \varphi_0,\varphi_k \rangle + a_1\langle \varphi_1,\varphi_k \rangle + \ldots - a_m \langle
        \varphi_m,\varphi_k \rangle = \langle f,\varphi_k \rangle~~\forall k \in \{0, \ldots, m\}$$
    Que nos da el siguiente sistema de ecuaciones lineales de $m+1$ ecuaciones y $m+1$ incógnitas:
    $$\left( \begin{array}{cccc}
                \langle \varphi_0,\varphi_0 \rangle & \langle \varphi_1,\varphi_0 \rangle & \ldots & \langle \varphi_m,\varphi_0 \rangle \\
                \langle \varphi_0,\varphi_1 \rangle & \langle \varphi_1,\varphi_1 \rangle & \ldots & \langle \varphi_m,\varphi_1 \rangle \\
                \vdots                              & \vdots                              & \ddots & \vdots                              \\
                \langle \varphi_0,\varphi_m \rangle & \langle \varphi_1,\varphi_m \rangle & \ldots & \langle \varphi_m,\varphi_m \rangle
            \end{array} \right) \left( \begin{array}{c}
                a_0    \\
                a_1    \\
                \vdots \\
                a_m
            \end{array} \right) = \left( \begin{array}{c}
                \langle f,\varphi_0 \rangle \\
                \langle f,\varphi_1 \rangle \\
                \vdots                      \\
                \langle f,\varphi_m \rangle
            \end{array} \right)$$
    A la matriz de coeficientes anterior se le llama matriz de Gram y verifica que es simétrica y definida positiva, por lo que
    el sistema anterior es compatible determinado, luego sabemos que los coeficientes $a_i~~i\in\{0, \ldots, m\}$ existen y que
    son únicos.
\end{proof}


\begin{ejemplo}
    Calcular la mejor aproximación de la función $f(x)=x^3$ en $\bb{P}_1$, utilizando el producto escalar definido como:
$$\langle v,u \rangle = \int_{-1}^1v(x)u(x)~dx~~~~\forall u,v \in V$$


\noindent
$\bb{P}_1 = \cc{L}\{1,x\}$, la mejor aproximación de $f$ será $u(x) = a_0 \cdot 1 + a_1x \in \bb{P}_1~~~a_0,a_1 \in \R$
$$\left( \begin{array}{cc}
            \langle 1,1 \rangle & \langle x,1 \rangle \\
            \langle 1,x \rangle & \langle x,x \rangle
        \end{array} \right) \left( \begin{array}{c}
            a_0 \\
            a_1
        \end{array} \right) = \left( \begin{array}{c}
            \langle x^3,1 \rangle \\
            \langle x^3,x \rangle
        \end{array} \right)$$
$$\langle 1,1 \rangle = 2$$
$$\langle 1,x \rangle = \langle x,1 \rangle = 0$$
$$\langle x,x \rangle = \dfrac{2}{3} $$
$$\langle x^3,1 \rangle = 0$$
$$\langle x^3,x \rangle = \dfrac{2}{5}$$

$$\left. \begin{array}{cccc}
        2a_0 &                 & = & 0            \\
             & \dfrac{2}{3}a_1 & = & \dfrac{2}{5}
    \end{array} \right\} \Rightarrow \left\{ \begin{array}{c}
        a_0 = 0 \\
        a_1 = \dfrac{3}{5}
    \end{array} \right\} \Rightarrow u(x) = \dfrac{3}{5}x$$
\end{ejemplo}



\begin{ejemplo}
    Calcular la recta que mejor aproxima por mínimos cuadrados los datos $\{(1,0),(2,1),(3,2),(4,3),(5,4)\}$.
Siendo $U=\cc{L}\{1,x\}$ y el producto escalar en $U$ se define como:
$$\langle v,w \rangle = v(1)w(1) + v(2)w(2) + v(3)w(3) + v(4)w(4) + v(5)w(5)~~~~\forall v,w \in V$$
La mejor aproximación de los puntos será $u(x) = a_0 \cdot 1 + a_1x \in \bb{P}_1~~~a_0,a_1 \in \R$
$$\left( \begin{array}{cc}
            \langle 1,1 \rangle & \langle x,1 \rangle \\
            \langle 1,x \rangle & \langle x,x \rangle
        \end{array} \right) \left( \begin{array}{c}
            a_0 \\
            a_1
        \end{array} \right) = \left( \begin{array}{c}
            \langle f,1 \rangle \\
            \langle f,x \rangle
        \end{array} \right)$$
$$\langle 1,1 \rangle = 5$$
$$\langle 1,x \rangle = \langle x,1 \rangle = 15$$
$$\langle x,x \rangle = 55 $$
$$\langle f,1 \rangle = 10$$
$$\langle f,x \rangle = 40$$

$$\left. \begin{array}{cccc}
        5a_0  & 15a_1 & = & 10 \\
        15a_0 & 55a_1 & = & 40
    \end{array} \right\} \Rightarrow \left\{ \begin{array}{c}
        a_0 = -1 \\
        a_1 = 1
    \end{array} \right\} \Rightarrow u(x) = -1+x$$
\end{ejemplo}



\subsection{Tipos de aproximación por mínimos cuadrados}
\noindent
\subsubsection{Aproximación por mínimos cuadrados continua}

Para la aproximación por mínimos cuadrados continua en el intervalo $[a,b]\subset \bb{R}$ se emplea el producto escalar siguiente:
\begin{equation*}
    \langle f,g\rangle = \int_{a}^b \omega(x)f(x)g(x)dx
\end{equation*}

donde $\omega$ es denominada \emph{función peso} y ha de ser integrable y $\omega\geq 0 \quad \forall x\in [a,b]$. Si no se especifica lo contrario, $\omega(x)=1$.

\begin{ejemplo}
    Sea el producto escalar definido como
    \begin{equation*}
        \langle f,g\rangle = \int_{-1}^1 f(x)g(x)dx
    \end{equation*}

    Encontrar la mejor aproximación de $f(x)=x^3$ en $\bb{P}_1$.

    Sea $u\in \bb{P}_1$ la mejor aproximación de $f$. Tomamos como base $\bb{P}_1=\cc{L}\{1,x\}$, por lo que sea $u(x)=a_0\cdot 1 + a_1x \quad a_0,a_1\in \bb{R}$. El sistema a resolver, por tanto, es:
    \begin{equation*}
        \left\{\begin{array}{c}
            a_0\langle 1,1\rangle + a_1 \langle x,1\rangle = \langle x^3, 1 \rangle \\
            a_0\langle 1,x\rangle + a_1 \langle x,1\rangle = \langle x^3, x \rangle \\
        \end{array}\right.
    \end{equation*}

    Tras calcular cada integral definida, tenemos que:
    \begin{equation*}
        \left\{\begin{array}{c}
            2a_0 + 0\cdot a_1 = 0 \\
            0\cdot a_0 + \frac{2}{3} a_1 = \frac{2}{5}
        \end{array}\right.
    \end{equation*}

    Por tanto, $u(x)=\frac{3}{5}x$.
    \begin{figure}[H]
        \centering
        \begin{tikzpicture}
        \begin{axis}[
            xlabel=$x$,
            ylabel=$y$,
            xmin=-1.5,
            xmax=1.5,
            ymin=-1.5,
            ymax=1.5,
            axis lines=middle,
            width=5cm,
            height=5cm,
            samples=90 % número de muestras para la función
        ]
        
        \addplot[red ,domain=-1.5:1.5] {3/5*x};
        \addplot[blue ,domain=-1.5:1.5] {x^3};
            
        \end{axis}
        \end{tikzpicture}
    \end{figure}
\end{ejemplo}


\subsubsection{Aproximación por mínimos cuadrados discreta}
Para la aproximación por mínimos cuadrados discreta en los nodos $x_i\subset \bb{R}$, con $i=0,\dots, N$; se emplea el producto escalar siguiente:
\begin{equation*}
    \langle f,g\rangle = \sum_{i=0}^N \omega(x_i)f(x_i)g(x_i)dx
\end{equation*}

donde $\omega$ es denominada \emph{función peso} y ha de ser $\omega(x_i) > 0 \quad \forall i=0,\dots, N$. Si no se especifica lo contrario, $\omega(x)=1$.

El problema de la mejor aproximación por mínimos cuadrados discreta consiste en lo siguiente:

Sea $f:[a,b]\to \bb{R}$ para la que conocemos $(x_i, f(x_i))\quad i=0,\dots, N$. Encontrar $p\in \bb{P}_m$ donde $m<N$ tal que:
\begin{equation*}
    ||f-p||^2 = \min_{q\in \bb{P}_m} ||f(x)-q(x)||^2 = \lim_{q\in \bb{P}_m} \sum_{k=0}^N [f(x_k)-q(x_k)]^2
\end{equation*}

donde he considerado el producto escalar definido como:
\begin{equation*}
    \langle f,g \rangle = \sum_{k=0}^N f(x_k)g(x_k)
\end{equation*}

Alternativamente, trabajamos de la siguiente manera. Sabemos que $\dim \bb{P}_m = m+1$. Consideramos 
\begin{equation*}
    \bb{P}_m = \cc{L}\{\varphi_0,\dots,\varphi_N\}
\end{equation*}

y la aplicación
\begin{equation*}
    \varphi_k \longmapsto \Phi=\left(\begin{array}{c}
        \varphi_k (x_0) \\ \varphi_k(x_1) \\ \vdots \\ \varphi_k(x_N)
    \end{array}\right)
\end{equation*}


Por la misma aplicación, tenemos que
\begin{equation*}
    f\longmapsto F=\left(\begin{array}{c}
        f(x_0) \\ f(x_1) \\ \vdots \\ f(x_N)
    \end{array}\right)
\end{equation*}


Consideramos $\nu = \cc{L}\{\Phi_0, \Phi_1, \dots, \Phi_m\}$, y demostremos que forman base.
\begin{equation*}
    a_0 \Phi_0 + a_1\Phi_1 + \dots a_m\Phi_m = 0 \Longrightarrow
    a_0\varphi_0(x_k) + \dots + a_m\varphi_m(x_k)=0 \qquad \forall k=0,\dots,N
\end{equation*}

Por tanto, como se anulan para todo $k$, tenemos que:
\begin{equation*}
    a_0\varphi_0 + \dots + a_m\varphi_m = 0 \Longrightarrow a_0=\dots=a_m=0
\end{equation*}

Por tanto, el problema se reduce a encontrar $P$ mejor aproximación de $F$ en $\nu$ con el producto escalar euclídeo.

No obstante, tenemos que:
\begin{equation*}
    \langle \Phi_i, \Phi_j\rangle = \sum_{k=0}^N \varphi_i(x_k)\varphi_j(x_k)
\end{equation*}
por tanto, tenemos que hemos llegado al producto escalar discreto definido en el primer caso.


\begin{ejemplo}
    Calcular la recta que mejor aproxima por mínimos cuadrados los siguientes datos:
    \begin{equation*}
        (1,0) \quad (2,1) \quad (3,2) \quad (4,3) \quad (5,4)
    \end{equation*}

    Sea el producto escalar definidio como:
    \begin{equation*}
        \langle u,v\rangle = \sum_{i=0}^N u(x_i)v(x_i)
    \end{equation*}

    Buscamos aproximar en $U=\bb{P}_1=\cc{L}\{1,x\}$. $f(x)$ está definida por:
    \begin{equation*}
        \left(\begin{array}{c}
            f(1) \\ f(2) \\ f(3) \\ f(4) \\ f(5)
        \end{array}\right)
        = \left(\begin{array}{c}
            0 \\ 1 \\ 2 \\ 3 \\ 4
        \end{array}\right)
    \end{equation*}

    Sea $u \in \bb{P}_1$ la mejor aproximación de $f$. Sea $u=a_0 \cdot 1 + a_1\cdot x$. El sistema a resolver, por tanto, es:
    \begin{equation*}
        \left\{\begin{array}{c}
            a_0\langle 1,1\rangle + a_1 \langle x,1\rangle = \langle f, 1 \rangle \\
            a_0\langle 1,x\rangle + a_1 \langle x,x\rangle = \langle f, x \rangle \\
        \end{array}\right.
    \end{equation*}

    Tenemos que:
    \begin{equation*}
        \begin{array}{c|c|c|c|c|c|c}
            x_i & f_i & x_1^0 & x_i^1 & x_i^2 & x_i^0f_i & x_i^1f_i \\ \hline
            1 & 0 & 1 & 1 & 1 & 0 & 0 \\
            2 & 1 & 1 & 2 & 4 & 1 & 2 \\
            3 & 2 & 1 & 3 & 9 & 2 & 6 \\
            4 & 3 & 1 & 4 & 16 & 3 & 12 \\
            5 & 4 & 1 & 5 & 25 & 4 & 20 \\ \hline
            && 5 & 15 & 55 & 10 & 40 \\
            && (\langle 1,1 \rangle) & (\langle 1,x \rangle) &
            (\langle x,x \rangle) & 
            (\langle f,1 \rangle) &
            (\langle f,x \rangle)
            
        \end{array}
    \end{equation*}

    Calculando cada producto escalar, tenemos que el sistema a resolver es:
    \begin{equation*}
        \left\{\begin{array}{c}
            5a_0 + 15a_1 = 10 \\
            15a_0 + 55a_1 = 40
        \end{array}\right.
    \end{equation*}

    Resolviendo, tenemos que $a_0=-1,\;a_1=1$.

    Por tanto, $u(x)=-1+x$.
\end{ejemplo}

\subsection{Ejemplos de Chebyshev}
\begin{itemize}
    \item \textbf{Ejemplo de Chebyshev de primera especie.}
En este caso, se toma:
$$w(x) = \dfrac{1}{\sqrt{1-x^2}}~~\forall x \in ]-1,1[$$
Esta función da un gran peso a los puntos que se encuentran en los extremos y un peso menor a los puntos centrales.

        \begin{figure}[H]
            \centering
            \begin{tikzpicture}
            \begin{axis}[
                xlabel=$x$,
                ylabel=$y$,
                xmin=-1.5,
                xmax=1.5,
                ymin=-0.5,
                ymax=7,
                axis lines=middle,
                width=5cm,
                height=5cm,
                samples=90 % número de muestras para la función
            ]
            
            \addplot[blue, thick, domain=-1:1] {1/sqrt(1-x^2)};
            \end{axis}
            \end{tikzpicture}
        \end{figure}

    \item \textbf{Ejemplo de Chebyshev de segunda especie.}
$$w(x) = \sqrt{1-x^2}~~\forall x \in [-1,1]$$
Esta función da un gran peso a los puntos centrales y un menor peso (casi inapreciable) a los puntos en lo extremos.
        \begin{figure}[H]
            \centering
            \begin{tikzpicture}
            \begin{axis}[
                xlabel=$x$,
                ylabel=$y$,
                xmin=-1.5,
                xmax=1.5,
                ymin=-1.5,
                ymax=1.5,
                axis lines=middle,
                width=5cm,
                height=5cm,
                samples=90 % número de muestras para la función
            ]
            
            \addplot[blue, thick, domain=-1:1] {(1-x^2)};
            \end{axis}
            \end{tikzpicture}
        \end{figure}
\end{itemize}


\section{Bases ortogonales}
\noindent
Si la base que cogemos del subespacio $U$ es ortogonal, esto es que:
$$\langle \tilde{\varphi_i}, \tilde{\varphi_j} \rangle = 0 ~~\forall i\neq j$$
Entonces, el sistema de ecuaciones se convierte en un sistema diagonal:
$$\langle \tilde{\varphi_k}, \tilde{\varphi_k} \rangle a_k = \langle f, \tilde{\varphi_k} \rangle~~k \in \{0, \ldots, m\}$$
Por lo que:
$$a_k = \dfrac{\langle f, \tilde{\varphi_k} \rangle}{\langle \tilde{\varphi_k}, \tilde{\varphi_k} \rangle}~~k \in \{0, \ldots, m\}$$
Y la mejor aproximación se calcula de la forma:
$$u = \sum_{k=0}^m \dfrac{\langle f, \tilde{\varphi_k} \rangle}{\langle \tilde{\varphi_k}, \tilde{\varphi_k} \rangle}\tilde{\varphi_k}$$

\bigskip
\begin{definicion}[Suma de Fourier]
    A la expresión:
    $$u = \sum_{k=0}^m \dfrac{\langle f, \tilde{\varphi_k} \rangle}{\langle \tilde{\varphi_k}, \tilde{\varphi_k} \rangle}\tilde{\varphi_k}$$
    Se le llama \textbf{$m$-ésima suma de Fourier} de $f$ asociada a la base ortogonal $\{\tilde{\varphi_0}, \tilde{\varphi_1},
        \ldots, \tilde{\varphi_m}\}$.\\

    \noindent
    A la cantidad:
    $$a_k = \dfrac{\langle f, \tilde{\varphi_k} \rangle}{\langle \tilde{\varphi_k}, \tilde{\varphi_k} \rangle}~~~~k \in \{0, \ldots, m\}$$
    Se le llama \textbf{$k$-ésimo coeficiente de Fourier} de $f$ asociado a la base ortongonal $\{\tilde{\varphi_0}, \tilde{\varphi_1},
        \ldots, \tilde{\varphi_m}\}$.\\
\end{definicion}

\subsection{Algoritmo de Gram-Schmidt}
\begin{teo}[Algoritmo de Gram-Schmidt]
    Sea $\{\varphi_0, \varphi_1, \ldots, \varphi_m\}$ una base de $U$, es posible obtener una base ortogonal
    $\{\tilde{\varphi_0}, \tilde{\varphi_1}, \ldots, \tilde{\varphi_m}\}$ de la forma:
    $$\tilde{\varphi_0} = \varphi_0$$
    $$\tilde{\varphi_k} = \varphi_k - \sum_{j=0}^{k-1} \dfrac{\langle \varphi_k, \tilde{\varphi_j} \rangle}
        {\langle \tilde{\varphi_j}, \tilde{\varphi_j} \rangle} \tilde{\varphi_j}~~~~k \in \{1,2, \ldots, m\}$$
    Como consecuencia, sabemos de la existencia de las bases ortogonales.
\end{teo}
\begin{proof}
    Realizamos inducción sobre $k = \dim \cc{L} \{\varphi_0, \varphi_1, \ldots, \varphi_k\}$:
    \begin{itemize}
        \item \underline{Para $k=2$:}

        Sea $\{\varphi_0, \varphi_1\}$ una base de $\cc{L}\{\varphi_0, \varphi_1\}$ con $\dim
        \cc{L}\{\varphi_0, \varphi_1\} = 2$:\par
    Construimos la base:\newline
    $$\tilde{\varphi_0} = \varphi_0$$
    $$\tilde{\varphi_1} = \varphi_1 - \dfrac{\langle \varphi_1, \tilde{\varphi_0} \rangle}
        {\langle \tilde{\varphi_0}, \tilde{\varphi_0} \rangle} \tilde{\varphi_0}$$
    $$\tilde{\varphi_1} \in \cc{L}\{\tilde{\varphi_0}, \varphi_1\} = \cc{L}\{\varphi_0, \varphi_1\}$$

    \noindent
    Comprobemos que $\tilde{\varphi_1}$ es ortogonal a $\tilde{\varphi_0}$ y que son linealmente independientes
    (vamos a demostrar que esto segundo es consecuencia de lo primero):
    $$\langle \tilde{\varphi_1}, \tilde{\varphi_0} \rangle = \langle \varphi_1, \tilde{\varphi_0} \rangle -
        \dfrac{\langle \varphi_1, \tilde{\varphi_0} \rangle}{\langle \tilde{\varphi_0}, \tilde{\varphi_0} \rangle}
        \langle \tilde{\varphi_0}, \tilde{\varphi_0} \rangle = 0 \Rightarrow \tilde{\varphi_1} \perp \tilde{\varphi_0}$$

    Supongamos que son linealmente dependientes: $\exists a,b \in \R \mid a\tilde{\varphi_0} + b \tilde{\varphi_1} = 0$:
    $$0 = \langle a\tilde{\varphi_0} + b\tilde{\varphi_1}, \tilde{\varphi_0} \rangle = a\langle \tilde{\varphi_0},
        \tilde{\varphi_0} \rangle + b \langle \tilde{\varphi_0}, \tilde{\varphi_1} \rangle \mathop{=}^{\tilde{\varphi_0}
        \perp \tilde{\varphi_1}} a\langle \tilde{\varphi_0}, \tilde{\varphi_0} \rangle = 0 \Leftrightarrow a = 0$$
    $$0 = \langle a\tilde{\varphi_0} + b\tilde{\varphi_1}, \tilde{\varphi_1} \rangle = a \langle \tilde{\varphi_0},
        \tilde{\varphi_1} \rangle + b\langle \tilde{\varphi_1}, \tilde{\varphi_1} \rangle \mathop{=}^{\tilde{\varphi_0}
        \perp \tilde{\varphi_1}} b\langle \tilde{\varphi_1}, \tilde{\varphi_1} \rangle = 0 \Leftrightarrow b = 0$$
    Luego $a = b = 0 \Rightarrow$ son linealmente independientes (consecuencia de ser ortogonales).

        \item \underline{Sea cierto para $k-1$:}

        $$\{\tilde{\varphi_0}, \tilde{\varphi_1}, \ldots, \tilde{\varphi}_{k-1}\} \mbox{ es una base de }
        \cc{L}\{\varphi_0, \varphi_1, \ldots, \varphi_{k-1}\} \mbox{ en la que:}$$
    $$\tilde{\varphi_i} \perp \tilde{\varphi_j}~~~~ i \neq j~~ i,j \in \{0, 1, \ldots, k-1\}$$
    Construimos:
    $$\tilde{\varphi_k} = \varphi_k - \sum_{j=0}^{k-1} \dfrac{\langle \varphi_k, \tilde{\varphi_j} \rangle}
        {\langle \tilde{\varphi_j}, \tilde{\varphi_j} \rangle} \tilde{\varphi_j}$$
    Comprobemos que sea ortogonal al resto (y por tanto, linealmente independientes):
    $$\forall j \in \{0, \ldots, k-1\}: \langle \tilde{\varphi_k}, \tilde{\varphi_j} \rangle = \langle \varphi_k,
        \tilde{\varphi_j} \rangle - \sum_{i=0}^{k-1} \dfrac{\langle \tilde{\varphi_k}, \tilde{\varphi_i} \rangle}
        {\langle \tilde{\varphi_i}, \tilde{\varphi_i} \rangle} \langle \tilde{\varphi_i}, \tilde{\varphi_j} \rangle=$$
    $$= \langle \varphi_k, \tilde{\varphi_j} \rangle - \sum_{i=0}^{k-1} \dfrac{\langle \tilde{\varphi_k},
            \tilde{\varphi_i} \rangle} {\langle \tilde{\varphi_i}, \tilde{\varphi_i} \rangle} \delta_{ij} \langle \tilde{\varphi_j},
        \tilde{\varphi_j} \rangle = \langle \varphi_k,
        \tilde{\varphi_j} \rangle - \dfrac{\langle \tilde{\varphi_k}, \tilde{\varphi_j} \rangle}
        {\langle \tilde{\varphi_j}, \tilde{\varphi_j} \rangle} \langle \tilde{\varphi_j}, \tilde{\varphi_j} \rangle = 0$$
    Por lo que $\tilde{\varphi_k} \perp \tilde{\varphi_j}~~~~\forall j \in \{0, \ldots, k-1\} \Rightarrow$ es linealmente
    independiente con todos ellos $\Rightarrow \{\tilde{\varphi_0}, \tilde{\varphi_1}, \ldots, \tilde{\varphi}_{k-1},
        \tilde{\varphi_k}\}$ es una base de $\cc{L}\{\varphi_0, \varphi_1, \ldots, \varphi_{k-1}, \varphi_k\}$.
    \end{itemize}
    
\end{proof}


\begin{ejemplo}
 Dado el siguiente producto escalar dentro de $\bb{P}_2$:
$$\langle f,g \rangle = \int_0^1 f(x)g(x)~dx~~~~\forall f,g \in \bb{P}_2$$
Buscar una base ortogonal a partir de la base $\{1,x,x^2\}$\\

\noindent
Aplicamos el algoritmo de Gram-Schmidt:\par
Dada $\{\varphi_i\} \mid \varphi_i = x^i~~i \in \{0,1,2\}$, construimos $\{\tilde{\varphi_i}\}$ como sigue:
$$\tilde{\varphi_0} = \varphi_0$$
$$\tilde{\varphi_1} = \varphi_1 - \dfrac{\langle \varphi_1, \tilde{\varphi_0} \rangle}{\langle \tilde{\varphi_0},
        \tilde{\varphi_0} \rangle} \tilde{\varphi_0}$$
$$\tilde{\varphi_2} = \varphi_2 - \dfrac{\langle \varphi_2, \tilde{\varphi_0} \rangle}{\langle \tilde{\varphi_0},
        \tilde{\varphi_0} \rangle} \tilde{\varphi_0} - \dfrac{\langle \varphi_2, \tilde{\varphi_1} \rangle}{\langle \tilde{\varphi_1},
        \tilde{\varphi_1} \rangle} \tilde{\varphi_1}$$

Calculamos los respectivos productos escalares:
$$\langle \varphi_1, \tilde{\varphi_0} \rangle = \int_0^1 x~dx = \dfrac{1}{2}
\qquad 
\langle \tilde{\varphi_0}, \tilde{\varphi_0} \rangle = \int_0^1 dx = 1$$
$$\tilde{\varphi_1} = \varphi_1 - \dfrac{\langle \varphi_1, \tilde{\varphi_0} \rangle}{\langle \tilde{\varphi_0},
        \tilde{\varphi_0} \rangle} \tilde{\varphi_0} = x - \dfrac{1}{2} \cdot 1 = x - \dfrac{1}{2}$$
\begin{equation*}
    \langle \varphi_2, \tilde{\varphi_0} \rangle = \int_0^1 x^2~dx = \dfrac{1}{3}
\qquad
\langle \varphi_2, \tilde{\varphi_1} \rangle = \int_0^1 x^2(x - \dfrac{1}{2})~dx = \dfrac{1}{12}
\qquad
\langle \tilde{\varphi_1}, \tilde{\varphi_1} \rangle = \int_0^1 (x-\dfrac{1}{2})^2~dx = \dfrac{1}{12}
\end{equation*}
$$\tilde{\varphi_2} = \varphi_2 - \dfrac{\langle \varphi_2, \tilde{\varphi_0} \rangle}{\langle \tilde{\varphi_0},
        \tilde{\varphi_0} \rangle} \tilde{\varphi_0} - \dfrac{\langle \varphi_2, \tilde{\varphi_1} \rangle}{\langle \tilde{\varphi_1},
        \tilde{\varphi_1} \rangle} \tilde{\varphi_1} = x^2 - \dfrac{1}{3} \cdot 1 - \left(x- \dfrac{1}{2}\right) = x^2 - x + \dfrac{1}{6}$$

\noindent
Por tanto, la base buscada es:
$$\left\{1, x-\dfrac{1}{2}, x^2-x+\dfrac{1}{6}\right\}$$
\end{ejemplo}

\subsection{Bases ortonormales}
\noindent Las bases ortonormales son bases ortogonales en las que la norma de cada elemento de la base es igual a 1.\\

\noindent
A la hora de obtener bases ortonormales podemos hacer dos procedimientos a partir del anterior algoritmo de Gram-Schmidt:
\begin{itemize}
    \item Aplicar el algoritmo de Gram-Schmidt y normalizar la base: esto es, aplicar el algoritmo sobre uan base para
          obtener una base ortogonal y luego escalar los vectores de la base para que tengan norma 1.
    \item O aplicar el algoritmo de Gram-Schmidt modificado, que consiste en obtener un vector de la base de Gram-Schmidt,
          normalizarlo y volver a aplicar el algoritmo con este vector normalizado y repetir sucesivamente.
\end{itemize}

\noindent
En la práctica, suele ser más cómodo hacerlo de la primera forma aunque esto tiene un inconveniente, ya que este
método es inestable numéricamente mientras que el segundo es estable numéricamente. Por tanto, cuando queramos evitar
errores al conseguir una base ortonormal a partir de una base para un subespacio de una dimensión considerable,
es conveniente aplicar el segundo método.\\

\noindent
\textbf{Normalización de vectores.} Dado un vector $v \in V$ que queremos normalizar, esto es $\|v\|=1 \Leftrightarrow \|v\|^2 = 1$.
Escalaremos $v$ de la siguiente forma:
$$\overline{v} = \dfrac{1}{\langle v,v \rangle}v = \dfrac{1}{\sqrt{\|v\|^2}}v = \dfrac{1}{\|v\|}v$$

De esta forma:
$$\|\overline{v}\| = \sqrt{\|\overline{v}\|^2} = \sqrt{\langle \overline{v},\overline{v} \rangle} =
    \sqrt{\left\langle \dfrac{1}{\|v\|}v,\dfrac{1}{\|v\|}v \right\rangle} = \sqrt{\dfrac{\langle v,v \rangle}{\|v\|^2}} =
    \sqrt{\dfrac{\|v\|^2}{\|v\|^2}} = 1$$

\subsection{Utilidad de bases ortogonales}
\noindent
Dado un producto escalar de la forma:
$$\langle x^i, x^j \rangle = \int_0^1 x^{i+j}~dx = \dfrac{1}{i+j+1}$$
Tenemos que la matriz de coeficientes del sistema que nos permite calcular la mejor aproximación respecto de la base
del estilo $\{x^i\}~~i \in \{0, 1, \ldots, n\}$ del subespacio $U=\bb{P}_n$ del espacio vectorial de polinomios es:
$$G=\left(\begin{array}{ccccc}
            \langle 1,1 \rangle   & \langle x,1 \rangle   & \langle x^2,1 \rangle & \ldots & \langle x^n,1 \rangle   \\
            \langle 1,x \rangle   & \langle x,x \rangle   & \ldots                & \ldots & \langle x^n, x \rangle  \\
            \langle 1,x^2 \rangle & \vdots                & \ddots                & \ddots & \vdots                  \\
            \vdots                & \vdots                & \ddots                & \ddots & \vdots                  \\
            \langle 1,x^n \rangle & \langle x,x^n \rangle & \ldots                & \ldots & \langle x^n,x^n \rangle
        \end{array}\right) = \left( \begin{array}{ccccc}
            1            & \dfrac{1}{2}   & \dfrac{1}{3} & \ldots & \dfrac{1}{n}   \\
                         &                &              &                         \\
            \dfrac{1}{2} & \dfrac{1}{3}   & \ldots       & \ldots & \dfrac{1}{n+1} \\
            \dfrac{1}{3} & \vdots         & \ddots       & \ddots & \vdots         \\
            \vdots       & \vdots         & \ddots       & \ddots & \vdots         \\
            \dfrac{1}{n} & \dfrac{1}{n+1} & \ldots       & \ldots & \dfrac{1}{2n}
        \end{array} \right)$$
Llamada matriz de Hilbert, donde todos los elementos de cada antidiagonal son iguales entre sí (matriz de Hankel):
$$\langle x^2,1 \rangle = \langle x,x \rangle = \langle 1,x^2 \rangle = \int_0^1 x^2~dx = \dfrac{1}{3}$$

\noindent
Dicha matriz está muy mal condicionada y para solventar el problema numérico al que nos enfrentamos, se suele aplicar
el algoritmo de Gram-Schmidt sobre la base anterior para obtener una base ortogonal y construir nuestro sistema
con dicha base, formando un sistema diagonal cuya resolución no nos supone ningún problema.\\

\noindent
Este procedimiento puede llegar a ser hasta más eficiente que el trabajar directamente con la matriz de Hilbert para
dimensiones grandes.

\section{Polinomios ortogonales}
\begin{definicion}[Polinomios Ortogonales]
    Dado un producto escalar $\langle \cdot,\cdot \rangle$, una familia de polinomios $\{P_n(x)\}_{n\geq 0}$ se dice que es
    una sucesión de polinomios ortogonales (SPO) si verifica:
    \begin{enumerate}
        \item $grd(P_n) = n$
        \item $\langle P_n, P_m \rangle = 0~~~~\forall n \neq m$
    \end{enumerate}
    \noindent
    Por tanto, tenemos que $\{P_0, P_1, \ldots, P_m\}$ es una base ortogonal de $\bb{P}_m$
\end{definicion}

\noindent
Una sucesión de polinomios ortogonales mónica (SPOM) (es decir, de coeficiente líder 1) puede obtenerse aplicando el algoritmo de
Gram-Schmidt a la base $\{1, x, x^2, \ldots, x^m\}$ de $\bb{P}_m$.

\noindent
Además, las sucesiones de polinomios ortogonales son únicas salvo constante multiplicativa.

\begin{teo}
    Dado un producto escalar $\langle \cdot,\cdot \rangle$ y sea $\{P_n\}_{n \geq 0}$ la correspondiente SPOM, entonces existen
    $\{c_n\}_{n \geq 0}$ y $\{\lambda_n\}_{n\geq 1}$ con $\lambda_n>0$ $\forall n$, tales que:
    $$P_{n+1}(x) = (x-c_{n+1})P_n(x) - \lambda_{n+1}P_{n-1}(x)~~~~\forall n \geq 0$$
    Con $P_{-1}(x) = 0$ y $P_0(x) = 1$\\

    \noindent
    En particular:
    $$c_{n+1} = \dfrac{\langle xP_n(x), P_n(x) \rangle}{\langle P_n(x),P_n(x) \rangle}$$
    $$\lambda_{n+1} = \dfrac{\langle P_n(x), P_n(x) \rangle}{\langle P_{n-1}(x),P_{n-1}(x) \rangle}$$
\end{teo}
\begin{proof}
    Dados los $n+1$ primeros polinomios ortogonales mónicos de una sucesión, nos disponemos a calcular el siguiente:
    $$xP_n(x) \in \bb{P}_{n+1} \Rightarrow xP_n(x) = \sum_{j=1}^{n+1} d_jP_j(x)$$
    $$\langle xP_n, P_k \rangle = \sum_{j=0}^{n+1} d_j \langle P_j, P_k \rangle = \sum_{j=0}^{n+1} d_j \delta_{kj} \langle P_j, P_k \rangle
        = d_k \langle P_k, P_k \rangle$$
    Luego:
    $$d_k = \dfrac{\langle xP_n, P_k \rangle}{\langle P_k, P_k \rangle} = \dfrac{\langle P_n, xP_k \rangle}
        {\langle P_k, P_k \rangle} = 0 \Leftrightarrow k+1<n \Leftrightarrow k<n-1$$
    Por lo que:
    $$d_j = 0~~~~\forall j \in \{0, 1, \ldots, n-2\}$$
    De donde concluimos:
    $$xP_n(x) = \sum_{j=1}^{n+1} d_jP_j(x) = \sum_{j=n-1}^{n+1} d_jP_j(x) =$$
    $$=d_{n-1}P_{n-1}(x) + d_nP_n(x) + d_{n+1}P_{n+1}(x)$$

    \noindent
    $xP_n(x)$ es de grado $n+1$ y su coeficiente líder es 1 por ser mónico, en el término de la derecha, el único
    monomio de grado $n+1$ es el de $d_{n+1}P_{n+1}(x)$ que es $d_{n+1}$, por lo que concluimos que $d_{n+1}=1$
    $$xP_n(x)=d_{n-1}P_{n-1}(x) + d_nP_n(x) + P_{n+1}(x)$$
    $$P_{n+1}(x) = (x-d_n)P_n(x) - d_{n-1}P_{n-1}(x)$$
    Y tenemos que:
    $$c_{n+1}=d_n~~~~~~\lambda_{n+1}=d_{n-1}$$
\end{proof}

\bigskip
\noindent
A continuación, la siguiente parte de los apuntes no es relevante en lo que se refiere a la asignatura de
Métodos Numéricos I, el lector puede leerlo para su disfrute.

\subsection{Polinomios de Jacobi}
\noindent
Se obtienen como la familia de polinomios ortogonales asociados a la función de peso dada por:
$$w(x) = (1-x)^\alpha(1+x)^\beta~~~~x \in [-1,1]~~\alpha, \beta > -1$$
Denotamos por $P_n^{(\alpha, \beta)}(x)$ a los polinomios de Jacobi, que verifican la condición de normalización:
$$P_n^{(\alpha, \beta)}(1) = \binom{n+\alpha}{n}$$

\noindent
Estos presentan como casos particulaes a las familias de polinomios de:
\begin{itemize}
    \item Chebyshev de primera especie, con $\alpha = \beta = \dfrac{-1}{2}$.
    \item Chebyshev de segunda espacie, con $\alpha = \beta = \dfrac{1}{2}$.
    \item Legendre, con $\alpha = \beta = 0$.
    \item Gegenbauer o ultraesféricos, con $\alpha = \beta$.
\end{itemize}

\noindent
Todos ellos de coeficiente líder:
$$P_n^{(\alpha, \beta)}(x) = \dfrac{(n + \alpha + \beta + 1)}{2n!}x^n + \ldots$$

\noindent
\textbf{Fórmula de Rodrigues.}
$$P_n^{(\alpha, \beta)}(x) = \dfrac{(-1)^n}{2^nn!}(1-x)^{-\alpha}(1+x)^{-\beta}D^n((1-x)^{n+\alpha}(1+x)^{n+\beta})~~\alpha, \beta>$$

\noindent
\textbf{Ortogonalidad.}
$$\int_{-1}^1 P_n^{(\alpha,\beta)}(x) P_m^{(\alpha, \beta)}(x) (1-x)^\alpha(1+x)^\beta~dx = \left\{ \begin{array}{ll}
        \tau_n \delta_{mn} & n=m      \\
        0                  & n \neq m
    \end{array} \right.$$
$$\tau_n = \dfrac{2^{\alpha + \beta + 1}}{2n + \alpha + \beta + 1} \dfrac{\Gamma(n+ \alpha + 1) \Gamma(n+\beta + 1)}
    {\Gamma(n+\alpha + \beta + 1)n!}$$

\noindent
\textbf{Fórmula de recurrencia.}
$$xP_n^{(\alpha, \beta)}(x) = \dfrac{2(n+1)(n+\alpha + \beta + 1)}{(2n + \alpha + \beta +1)(2n + \alpha + \beta + 2)}P_{n+1}^{(\alpha,\beta)}(x) + $$
$$+ \dfrac{\beta^2-\alpha^2}{(2n+\alpha+\beta)(2n+\alpha+\beta+2)}P_n^{(\alpha + \beta)}(x) + $$
$$+ \dfrac{2(n+\alpha)(n+\beta)}{(2n+\alpha+\beta)(2n+\alpha+\beta+1)}P_{n-1}^{(\alpha,\beta)}(x)$$

\subsection{Polinomios de Lagrere}
\noindent
Se obtienen como la familia de polinomio ortogonales asociados a la función peso dada por:
$$w(x) = x^\alpha e^{-x}~~~~x \in [0,+\infty[~~\alpha >1$$
Los correspondientes momentos existen y verifican:
$$\int_0^{+\infty}x^nx^\alpha e^{-x}~dx = \int_0^{+\infty}x^{n+\alpha}e^{-x}~dx = \Gamma(n+\alpha)$$
Notemos por $\{L_n^{(\alpha)}(x)\}_{n\geq 0}$ a los polinomios de Laguerre con coeficiente líder $L_n^{(\alpha)} =
    \dfrac{(-1)^n}{n!}x^n + \ldots$

\noindent
\textbf{Fórmula de Rodrigues.}
$$L_n^{(\alpha)}(x) = \dfrac{1}{n!}x^{\alpha}e^{-x}D^n(x^{n+\alpha e^{-x}})~~~~\alpha >-1$$

\noindent
\textbf{Ortogonalidad.}
$$\int_0^{+\infty}L_m^{(\alpha)}(x) L_n^{(\alpha)}(x) x^\alpha e^{-x}~dx = \left\{ \begin{array}{ll}
        \dfrac{\Gamma(n + \alpha + 1)}{n!}\gamma_{mn} & n = m    \\
        0                                             & n \neq m
    \end{array} \right.$$

\noindent
\textbf{Relación de recurrencia.}
$$(n+1)L_{n+1}^{(\alpha)}(x) = (2n + \alpha + 1 - x)L_n^{(\alpha)}(x) - (n+\alpha)L_{n-1}^{(\alpha)}(x)$$

\subsection{Polinomios de Hermite}
\noindent
Se obtienen como la familia de polinomios ortogonales a la función peso dada por:
$$w(x) = e^{-x^2}~~~~\forall x \in \R$$
$\todon$, notaremos por $H_n(x)$ al polinomio ortogonal con respecto a $w(x)$ y cuyo coeficiente líder es $H_n(x) = 2^nx^n + \ldots$.\\

\noindent
\textbf{Fórmula de Rodrigues.}
$$H_n(x) = (-1)^n e^{x^2} D^n(e^{-x^2})$$

\noindent
\textbf{Ortogonalidad.}
$$\int_{-\infty}^{+\infty}H_m(x) H_n(x) e^{-x^2}~dx = \left\{ \begin{array}{ll}
        2^n n! \sqrt{\pi}\delta_{nm} & n = m    \\
        0                            & n \neq m
    \end{array} \right.$$

\noindent
\textbf{Relación de recurrencia.}
$$H_n(x) = 2xH_{n-1}(x) -2nH_{n-2}(x)$$


\section{Ejercicios}
Los ejercicios relativos a este tema están disponbles en la sección \ref{sec:Rel4}.

    \chapter{Relaciones de Problemas}
    \fancyhead[R]{\helv \nouppercase{\rightmark}}
    \section{Conexión por arcos}

\begin{ejercicio}
    Muestra que cualquier esfera de $\mathbb{R}^n$, $n\geq 2$ es arcoconexa con la topología usual.\\

    \noindent
    Es decir, queremos ver que $\bb{S}^n$ es arcoconexa para $n\geq 1$. \newline (notemos que $\bb{S}^0 = \{x\in \mathbb{R} : \|x\| = 1\} = \{-1,1\}$ no es un conjunto arcoconexo).\\

    \noindent
    Para ello, sea $n\geq 2$, sabemos que $\bb{S}^n\setminus\{p\}$ (con $p\in \bb{S}^n$) es homeomorfa a $\mathbb{R}^{n-1}$, que es un conjunto arcoconexo por ser convexo (es una espacio vectorial). Como la arcoconexión es una propiedad topológica, esta se conserva por homeomorfismo, luego $\bb{S}^n\setminus \{p\}$ es un conjunto arcoconexo, $\forall p\in \bb{S}^n$.

    Tomando $N = (0,\ldots,0,1), S =(0,\ldots,0,-1) \in \bb{S}^n$, podemos ver $\bb{S}^n$ como unión de dos conjuntos arcoconexos:
    \begin{equation*}
        \bb{S}^n = (\bb{S}^n\setminus\{N\}) \cup (\bb{S}^n\setminus\{S\})
    \end{equation*}

    no disjuntos:
    \begin{equation*}
        (\bb{S}^n\setminus\{N\}) \cap (\bb{S}^n\setminus\{S\}) = \bb{S}^n\setminus\{N,S\}
    \end{equation*}
    Por lo que $\bb{S}^n$ es un conjunto arcoconexo, $\forall n\geq 2$.
\end{ejercicio}

\begin{ejercicio}
    Demuestra que si $\{A_i\}_{i \in I}$ es una familia de arcoconexos de $X$ tales que todos intersecan a uno de ellos, es decir,
    \begin{equation*}
        A_i\cap A_{i_0} \neq \emptyset, \qquad \forall i \in I,
    \end{equation*}
    entonces $\bigcup\limits_{i \in I}A_i$ es arcoconexo.\\

    \noindent
    Sean $x,y\in \bigcup\limits_{i \in I}A_i$, entonces existen $i,j\in I$ de forma que $x\in A_i$ y $y\in A_j$. Como $A_i \cap A_{i_0}, A_j\cap A_{i_0}\neq \emptyset $, podemos tomar $a\in A_i\cap A_{i_0}$ y $b\in A_j\cap A_{i_0}$.
    \begin{itemize}
        \item $A_i$ es un conjunto arcoconexo con $x,a\in A_i$, por lo que existe un camino, $\alpha$, que une $x$ con $a$.
        \item $A_j$ también es un conjunto arcoconexo con $y,b\in A_j$, por lo que existe un camino, $\beta$, que une $y$ con $b$.
        \item Además, $A_{i_0}$ es un conjunto arcoconexo con $a,b\in A_{i_0}$, por lo que existe un tercer camino, $\gamma$, que une $a$ con $b$.
    \end{itemize}
    De esta forma, podemos tomar:
    \begin{equation*}
        \sigma = \alpha \ast \left(\gamma \ast \tilde{\beta}\right)
    \end{equation*}
    Que es un camino que une $x$ con $y$. Como $x$ e $y$ eran arbitrarios, podemos unir cualesquiera dos puntos de $\bigcup\limits_{i \in I}A_i$, por lo que dicho conjunto es arcoconexo.

    \begin{figure}[H]
        \centering
        \begin{tikzpicture}[scale=1.2]
        % Dibujar elipses (conjuntos)
        \draw[thick] (0,1.5) ellipse (2 and 0.8); % elipse superior horizontal
        \draw[thick] (0,-1.5) ellipse (2 and 0.8); % elipse inferior horizontal
        \draw[thick] (1.2,0) ellipse (0.8 and 2);  % elipse vertical

        % Puntos
        \node[circle,fill=black,inner sep=1pt,label=left:$x$] (x) at (-1,1.5) {};
        \node[circle,fill=black,inner sep=1pt,label=right:$a$] (a) at (1,1.5) {};
        \node[circle,fill=black,inner sep=1pt,label=right:$b$] (b) at (1,-1.5) {};
        \node[circle,fill=black,inner sep=1pt,label=left:$y$] (y) at (-1,-1.5) {};

        % Arcos dirigidos
        \draw[-stealth,thick,red,bend left=50] (x) to (a);
        \draw[-stealth,thick,red,bend left=10] (a) to (b);
        \draw[-stealth,thick,red,bend right=15] (y) to (b);
        \end{tikzpicture}
        \caption{Forma de unir dos puntos cualesquiera.}
    \end{figure}
\end{ejercicio}

\begin{ejercicio}
    Sea $X$ un conjunto, $x_0\in X$, y consideramos la topología (del punto incluido) dada por
    \begin{equation*}
        T = \{U\subset X : x_0 \in U\} \cup \{\emptyset \}
    \end{equation*}
    ¿Es $(X,T)$ arcoconexo?\\

    \noindent
    Sí: sea $x\in X$, veamos que la aplicación $\alpha:[0,1]\to X$ dada por
    \begin{equation*}
        \alpha(t) = \left\{\begin{array}{ll}
                x & \text{si } t\in [0,\nicefrac{1}{2}] \\
                x_0 & \text{si } t\in \left]\nicefrac{1}{2},1\right]
        \end{array}\right. \qquad \forall t\in [0,1]
    \end{equation*}
    es continua. Sea $U\in T$:
    \begin{itemize}
        \item Si $U = \emptyset $, entonces $\alpha^{-1}(U) = \emptyset \in \cc{T}_u\big|_{[0,1]}$.
        \item Si $x_0\in U$ y $x\notin U$, entonces $\alpha^{-1}(U) = \left]\nicefrac{1}{2},1\right]\in \cc{T}_u\big|_{[0,1]}$.
        \item Si $x_0,x\in U$, entonces $\alpha^{-1}(U) = [0,1] \in \cc{T}_u\big|_{[0,1]}$.
    \end{itemize}
    Como la preimagen de cualquier conjunto abierto es abierta, tenemos que $\alpha$ es continua, luego es un arco que une $x$ con $x_0$.\\

    \noindent
    Ahora, si $x,y\in X$, tenemos que existen $\alpha,\beta:[0,1]\to X$ de forma que $\alpha$ une $x$ con $x_0$ y $\beta$ une $y$ con $x_0$; por lo que $\alpha\ast\tilde{\beta}$ es un arco que une $x$ con $y$. Como $x$ e $y$ eran arbitrarios, concluimos que $X$ es arcoconexo.
\end{ejercicio}

\begin{ejercicio}
   Demustra que en $\mathbb{R}^n$  con la topología usual, todo abierto conexo es arcoconexo. ¿Es cierto que todo cerrado conexo de $\mathbb{R}^n$ es arcoconexo?\\

   \noindent
   En teoría vimos que:
   \begin{equation*}
       \text{Un conjunto es arcoconexo} \Longleftrightarrow \left\{\begin{array}{l}
           \text{Es conexo} \\
           \text{Todo punto admite un entorno arcoconexo}
       \end{array}\right.
   \end{equation*}
   Sea $U$ un abierto conexo de $(\mathbb{R}^n, \cc{T}_u)$, falta ver que todo punto suyo admite un entorno arcoconexo en la topología inducida en $U$ para ver que $U$ es arcoconexo. Para ello, sea $x\in U$, como $U$ es abierto existe $r\in \mathbb{R}^+$ de forma que $B(x,r)\subset U$. $B(x,r)$ es un conjunto arcoconexo por ser convexo, luego es un entorno arcoconexo de $x$ en $U$. Como $x$ era un punto arbitrario de $U$, todo punto suyo admite un entorno arcoconexo, y como $U$ era conexo, tenemos que $U$ es arcoconexo.\\

   \noindent
   Ahora, no es cierto que todo cerrado conexo de $\mathbb{R}^n$ es arcoconexo, ya que si consideramos $f:\mathbb{R}^+\to \mathbb{R}$ dada por:
   \begin{equation*}
       f(x) = \sen\left(\dfrac{1}{x}\right) \qquad \forall x\in \mathbb{R}^+
   \end{equation*}
   Tenemos que
   \begin{equation*}
       C = \overline{Gr(f)} = \overline{\{(x,f(x)) : x\in \mathbb{R}^+\}} = Gr(f) \cup (\{0\}\times [-1,1])
   \end{equation*}
   es un conjunto cerrado y conexo (se vio en Topología I) pero que no es arcoconexo, puede probarse por un razonamiento similar a un ejemplo visto en teoría.

   \begin{figure}[H]
       \centering
        \begin{tikzpicture}
          \begin{axis}[
            axis lines=middle,
            xlabel={$x$}, ylabel={$y$},
            xmin=0, xmax=1,
            ymin=-1.2, ymax=1.2,
            samples=200 % menos puntos = compila más rápido
          ]
            \addplot[blue, thick, domain=0.0005:1] {sin(deg(1/x))};
            \addplot[blue, ultra thick] coordinates {(0,-1) (0,1)};
          \end{axis}
        \end{tikzpicture}
        \caption{Dibujo de la adherencia de la gráfica de $f(x)$.}
   \end{figure}
\end{ejercicio}

\begin{ejercicio}
    Prueba que la componente arcoconexa de un punto $x_0$ está contenida en la componente conexa de $x_0$.\\

    \noindent
    Sea $(X,T)$ un espacio topológico, $x_0\in X$ y $C$ la componente arcoconexa de $x_0$ en $X$, en particular tenemos que $C$ es un conjunto arcoconexo, luego es conexo, por lo que está contenida en la componente conexa de $x$, al ser esta el mayor conjunto conexo que contiene a $x$.
\end{ejercicio}

\begin{ejercicio}
    En $\mathbb{R}$ con la topología de Sorgenfrey, esto es, la topología que tiene como base
    \begin{equation*}
        \cc{B}_S = \{[a,b) \subset \mathbb{R} : a<b\},
    \end{equation*}
    determina sus componentes arcoconexas.\\

    \noindent
    En Topología I vimos que las componentes conexas de la topología de Sorgenfrey eran los conjuntos de puntos unitarios $\{x\}$, ya que si tenemos un conjunto $A\subset \mathbb{R}$ con al menos dos puntos distintos $x$ e $y$ (suponemos $x<y$), entonces en la topología inducida en $A$ podemos considerar los abiertos:
    \begin{equation*}
        U = [-\infty,y)\cap A, \qquad V = [y,+\infty)\cap A
    \end{equation*}
    de forma que $U,V\neq \emptyset $, $U\cup V = A$ y $U\cap V = \emptyset $, por lo que $A$ (cualquier conjunto con al menos dos puntos distintos) es disconexo, luego las componentes conexas han de ser los conjuntos unitarios, ya que los conjuntos unitarios son conexos en cualquier topología.

    Como las componentes arcoconexas se encuentran contenidas en las componentes conexas, no queda más salida que las componentes arcoconexas de la topología de Sorgenfrey sean los conjuntos unitarios.
\end{ejercicio}

\begin{ejercicio}
    Sea $f:X\to Y$ un homeomorfismo entre espacios topológicos. Demuestra que $A\subset X$ es una componente arcoconexa de $X$ si y solo si $f(X)$ es una componente arcoconexa de $Y$. Deduce que el número de componentes arcoconexas es invariante por homeomorfismos.\\

    \noindent
    Sea $A\subset X$ una componente arcoconexa de $X$, veamos que $f(A)$ es una componente arcoconexa de $Y$. Para ello, por reducción al absurdo, si $f(A)$ no fuera una componente arcoconexa de $Y$ podría ser por dos razones:
    \begin{itemize}
        \item $f(A)$ no es un conjunto arcoconexo, algo que llevaría a una contradicción, ya que se vio que la imagen por una función continua de un conjunto arcoconexo era arcoconexa.
        \item Porque existe $B\subset Y$ un conjunto arcoconexo distinto de $f(A)$ de forma que $f(A)\subset B\subset Y$. En dicho caso, si aplicamos $f^{-1}$ en la anterior inclusión tenemos que:
            \begin{equation*}
                f^{-1}(f(A)) = A \subset f^{-1}(B) \subset X
            \end{equation*}
            Por lo que tenemos $f^{-1}(B)$, un conjunto arcoconexo\footnote{por ser imagen por una función continua de un conjunto arcoconexo.} distinto de $A$ que contiene a $A$, luego $A$ no era una componentes arcoconexa de $X$, contradicción.
    \end{itemize}
    En definitiva, si $A\subset X$ es una componente arcoconexa entonces $f(A)$ también lo es de $Y$. Ahora, si $f(A)$ es una componente arcoconexa de $Y$, basta aplicar que $f^{-1}$ también es un homeomorfismo para concluir que $f^{-1}(f(A)) = A$ es una componente arcoconexa de $X$.\\

    \noindent
    Sea $Z$ un espacio topológico, notaremos en este ejercicio:
    \begin{equation*}
        \Gamma_Z = \{U\subset Z : U \text{\ es una componente arcoconexa de\ } Z\}
    \end{equation*}
    Recuperando el homeomorfismo $f:X\to Y$, definimos
    \Func{\Phi}{\Gamma_X}{\Gamma_Y}{U}{f(U)}
    \begin{itemize}
        \item $\Phi$ está bien definida (es decir, $f(U)\in \Gamma_Y$ para $U\in \Gamma_X$), ya que hemos visto que la imagen de una componente arcoconexa de $X$ es una componente arcoconexa de $Y$.
        \item $\Phi$ es inyectiva, ya que si $U,V\in \Gamma_X$ con $f(U)=f(V)$, entonces por ser $f$ inyectiva tenemos que $U=V$.
        \item $\Phi$ es sobreyectiva, ya que si $W\in \Gamma_Y$, entonces $f^{-1}(W)\in \Gamma_X$, con:
            \begin{equation*}
                \Phi(f^{-1}(W)) = f(f^{-1}(W)) = W
            \end{equation*}
    \end{itemize}
    Por ser $\Phi$ biyectiva concluimos que $|\Gamma_X| = |\Gamma_Y|$; es decir, el número de componentes arcoconexas es invariante por homeomorfismos.
\end{ejercicio}

\begin{ejercicio}
    En $X=\mathbb{R}\times \{0,1\}$ se considera la topología que tiene por base
    \begin{equation*}
        \cc{B} = \{\left]a,b\right[\times \{0,1\} : a<b\}.
    \end{equation*}
    Demuestra que $X$ es arcoconexo. ¿Es $X$ homeomorfo a $\mathbb{R}$ con la topología usual?\\

    \noindent
    Sean $\alpha=(x,a),\beta=(y,b)\in X$, vamos a tratar de crear un arco que una $\alpha$ con $\beta$:
    \begin{itemize}
        \item Si $a=b$, entonces $\gamma:[0,1]\to X$ dada por:
            \begin{equation*}
                \gamma(t) = ((1-t)x + ty, a) \qquad \forall t\in [0,1]
            \end{equation*}
            Es una aplicación continua, ya que si tomamos $B = \left]a,b\right[\times \{0,1\}\in \cc{B}$, tenemos:
            \begin{equation*}
                \gamma^{-1}(B) = \gamma^{-1}(\left]a,b\right[\times \{0\}) \text{\ abierto de\ } [0,1]
            \end{equation*}
            Ya que el conjunto $\left]a,b\right[\times \{0\}$ es un abierto para la topología usual y $\alpha$ es una aplicación continua para la topología usual.
        \item Si $\alpha = (0,0)$ y $\beta = (0,1)$, entonces si tomamos $\gamma:[0,1]\to X$ dada por:
            \begin{equation*}
                \gamma(t) = \left\{\begin{array}{ll}
                        \alpha & \text{si\ } 0 \leq t \leq \nicefrac{1}{2}\\
                        \beta & \text{si\ } \nicefrac{1}{2}<t \leq 1
                \end{array}\right.
            \end{equation*}
            tenemos que $\gamma$ es continua, ya que si $B = \left]a,b\right[\times \{0,1\}\in \cc{B}$, tenemos que:
            \begin{equation*}
                \gamma^{-1}(B) = \left\{\begin{array}{cl}
                        \emptyset & \text{si\ } 0 \notin \left]a,b\right[ \\
                        \left[0,1\right] & \text{si\ } 0 \in \left]a,b\right[
                \end{array}\right.
            \end{equation*}
        \item Una vez discutidos dichos casos, suponemos ahora que $\alpha=(x,0)$ y $\beta = (y,1)$ (en caso contrario, sustituimos los papeles de $\alpha$ y $\beta$), en cuyo caso:
            \begin{itemize}
                \item Sabemos de la existencia de un arco $\gamma$ que une $\alpha$ con $(0,0)$.
                \item Sabemos de la existencia de un arco $\tau$ que une $(0,0)$ con $(0,1)$.
                \item Sabemos de la existencia de un arco $\pi$ que une $\beta$ con $(0,1)$.
            \end{itemize}
            Si consideramos el arco $\gamma \ast (\tau \ast \tilde{\pi})$ obtenemos un arco que une $\alpha$ con $\beta$.
    \end{itemize} 
    Por tanto, $X$ es arcoconexo, ya que somos capaces de unir cualesquiera dos puntos distintos de $X$ por un arco.\\

    \noindent
    Ahora, para responder a la pregunta de si $(\mathbb{R},\cc{T}_u)$ es homeomorfo a $X$, la respuesta es que no, y tenemos dos formas de justificar la respuesta:
    \begin{description}
        \item [Opción 1.] Sabemos que $(\mathbb{R},\cc{T}_u)$ es $T2$ por ser un espacio topológico metrizable, mientras que podemos probar que $X$ no es $T2$, ya que no existen ningún par de abiertos disjuntos uno conteniendo a $(0,0)$ y otro conteniendo a $(0,1)$, puesto que si $U$ es un abierto de $X$ que contiene a $(0,0)$, entonces como $\cc{B}$ es una base, existen $a,b\in \mathbb{R}$ de forma que:
            \begin{equation*}
                (0,0) \in \left]a,b\right[\times \{0,1\} \subset U
            \end{equation*}
            Sin embargo, tendríamos entonces que $(0,1)\in \left]a,b\right[\times \{0,1\}$, de donde $(0,1)\in U$, por lo que $X$ no es T2 y como ser T2 es una propiedad topológica, dichos espacios no pueden ser homeomorfos.
        \item [Opción 2.] Otra forma sería suponer que son homeomorfos, con lo que existe un homeomorfismo $f:\mathbb{R}\to X$. Sea $p\in \mathbb{R}$, resulta entonces que $\mathbb{R}\setminus \{p\}$ es homeomorfo a $X\setminus \{(p,0)\}$, pero:
            \begin{itemize}
                \item $\mathbb{R}\setminus\{p\}$ no es arcoconexo.
                \item $X\setminus \{(p,0)\}$ sí es arcoconexo, ya que podemos hacer que cualquier curva ``salte'' a $(p,1)$ sin perder su continuidad, con lo que podemos seguir conectando dos puntos cualesquiera.
            \end{itemize}
    \end{description}
\end{ejercicio}

\begin{ejercicio}
    En $\mathbb{R}^3$ con la topología usual, calcula las componentes arcoconexas de
    \begin{equation*}
        X = \{x,y,z) \in \mathbb{R}^3 : xyz = 1\}
    \end{equation*}

    \noindent
    Notemos que como $xyz = 1$, ninguno de ellos puede ser igual a 0, por lo que:
    \begin{equation*}
        X = \left\{(x,y,z)\in \mathbb{R}^3 : z=\dfrac{1}{xy} ,\quad  xy\neq 0\right\}
    \end{equation*}
    Si tomamos:
    \begin{equation*}
        \Gamma = \left\{(x,y) \in \mathbb{R}^2 : xy \neq 0\right\} = \mathbb{R}^\ast \times \mathbb{R}^\ast
    \end{equation*}
    y definimos $f:\Gamma\to \mathbb{R}$ dada por:
    \begin{equation*}
        f(x,y) = \dfrac{1}{xy} \qquad \forall (x,y)\in \Gamma
    \end{equation*}
    Tenemos que $X = Gr(f)$. Por tanto, definiendo $h:\Gamma\to X$ por:
    \begin{equation*}
        h(x,y) = (x,y,f(x,y)) \qquad \forall (x,y)\in \Gamma
    \end{equation*}
    Obtenemos (como vimos en Topología I) un homeomorfismo entre $\Gamma$ y $X$. Como $\Gamma$ tiene 4 componentes arcoconexas:
    \begin{equation*}
        \mathbb{R}^+\times \mathbb{R}^+, \qquad \mathbb{R}^+\times \mathbb{R}^-, \qquad \mathbb{R}^-\times\mathbb{R}^-, \qquad \mathbb{R}^-\times\mathbb{R}^-
    \end{equation*}
    y las componentes arcoconexas se convervan por homeomorfismos tal y como acabamos de ver en el ejercicio 7, tenemos que:
    \begin{equation*}
        h(\mathbb{R}^+\times \mathbb{R}^+), \qquad h(\mathbb{R}^+\times \mathbb{R}^-), \qquad h(\mathbb{R}^-\times\mathbb{R}^-), \qquad h(\mathbb{R}^-\times\mathbb{R}^-)
    \end{equation*}
    son las componentes arcoconexas de $X$.
\end{ejercicio}

\begin{ejercicio} % // TODO: HACER
    En $\mathbb{R}^2$ con la topología usual consideremos las rectas horizontales $A_n = \mathbb{R}\times \{\nicefrac{1}{n}\}$, $B_n = \mathbb{R}\times \{\nicefrac{-1}{n}\}$ y el eje de ordenadas menos el origen, esto es, $C=\{0\}\times (\mathbb{R}\setminus \{0\})$. Calcula las componentes conexas y arcoconexas de
    \begin{equation*}
        X = \left(\bigcup_{n\in \mathbb{N}}A_n\right) \cup \left(\bigcup_{n\in \mathbb{N}}B_n\right) \cup C \cup \{(1,0)\}.
    \end{equation*}
\end{ejercicio}


    \section{Topología del plano complejo}

\begin{ejercicio}
    Estudiar la continuidad de la función argumento principal; esta es, $\arg : \mathbb{C}^\ast \to \mathbb{R}$.\\

    Por el Ejercicio~\ref{ej:1.6}, sabemos que:
    \begin{equation*}
        \arg z = 2\arctan\left(\frac{\Im z}{\Re z + |z|}\right) \quad \forall z \in \mathbb{C}^\ast\setminus \bb{R}^-
    \end{equation*}

    Consideramos $\Omega = \mathbb{C}^\ast\setminus \bb{R}^-$. Como la función $Id$ es continua, tenemos que $\Re z,\Im z, |z|$ son continuas en $\cc{C}$. Además, como el denominador tan solo se anula en $\bb{R}^-_0$, el argumento de la arcotangente restringido a $\Omega$ es una función continua. Por ser la arcotangente continua en $\bb{R}$ y serlo el producto de funciones continuas, concluimos que $\arg_{\big| \Omega}$ es continua. Como $\Omega$ es abierto, por el carácter local de la continuidad, $\arg$ es continua en $\Omega=\mathbb{C}^\ast\setminus \bb{R}^-$.\\

    Tan falta por estudiar la continuidad en $\bb{R}^-$. Para ello, sea $z\in \bb{R}^-$, del que sabemos que $\arg z = \pi$. Sea la sucesión $\{\theta_n\}$ que recorre los ángulos desde $0$ en sentido horario hasta $-\pi$, límite de la sucesión:
    \begin{equation*}
        \{\theta_n\} = \left\{-\pi\left(1+\frac{1}{n}\right)\right\}\to -\pi
    \end{equation*}

    A partir de dicha sucesión, definimos $\{z_n\}$ como los números complejos de módulo $|z|$ y argumento $\theta_n$; que recorren los puntos de la circunferencia unitaria desde el eje positivo en sentido horario hasta el eje negativo.
    \begin{equation*}
        \{z_n\} = \left\{|z|\left(\cos\left(\theta_n\right)+i\sin\left(\theta_n\right)\right)\right\}\to |z|\left(\cos(-\pi)+i\sin(-\pi)\right) = -|z| = z
    \end{equation*}

    Por último, tenemos que:
    \begin{equation*}
        \{\arg z_n\} = \{\theta_n\} \to -\pi\neq \pi = \arg z
    \end{equation*}

    Por tanto, hemos encontrado una sucesión $\{z_n\}$ con $z_n\in \bb{C}^*~\forall n\in \mathbb{N}$, con $\{z_n\}\to z$ pero $\{\arg z_n\}\nrightarrow \arg z$. Por tanto, $\arg$ no es continua en $z$. Como $z$ era arbitrario, concluimos que $\arg$ no es continua en $\bb{R}^-$.\\

    Por tanto, concluimos que $\arg$ es continua en $\mathbb{C}^\ast\setminus \bb{R}^-$, pero no lo es en $\bb{R}^-$.
\end{ejercicio}

\begin{ejercicio}\label{ej:2.2}
    Dado $\theta \in \mathbb{R}$, se considera el conjunto $S_\theta = \{z \in \mathbb{C}^\ast \mid \theta \notin \Arg z\}$. Probar que existe una función $\varphi \in \cc{C}(S_\theta)$ que verifica $\varphi(z) \in \Arg (z)$ para todo $z \in S_\theta$.\\

    La elección del argumento principal de un número complejo realizada provoca que haya una discontinuidad en $\bb{R}^-=S_{\pi}$. Este ejercicio nos pide encontrar una función que, dado un argumento $\theta$, sea continua en $\bb{C}^*$ excepto en los puntos $z$ para los cuales $\theta\in \Arg z$.\\

    Dado $z\in S_{\theta}$, como $\arg$ es continua en $\mathbb{C}^\ast\setminus \bb{R}^-$, en primer lugar definiremos una función $g_{\theta}:S_{\theta}\to C^{\ast}\setminus \bb{R}^-$ que nos lleve $z$ a un punto $w\notin \bb{R}^-$ (esto lo haremos girando $z$ un ángulo de $\pi-\theta$); para poder aplicar luego $\arg$ y modificar el valor de forma que $\varphi(z)\in \Arg z$ (esto lo haremos restando $\pi-\theta$). Vamos a ello.\\

    Definimos en primer lugar $w_\theta=\cos(\pi-\theta) + i\sin(\pi-\theta)\in \mathbb{C}$, de forma que $|w_\theta|=1$ y $\pi-\theta\in \Arg w_\theta$. Definimos $g_{\theta}$ como:
    \Func{g_{\theta}}{S_{\theta}}{\mathbb{C}^{\ast}\setminus \bb{R}^-}{z}{zw_{\theta}}

    En primer lugar, como $g_{\theta}$ es polinómica, tenemos que $g_\theta\in \cc{C}(S_{\theta})$. Además, dado $z\in S_{\theta}$, tenemos que:
    \begin{align*}
        \Arg g_{\theta}(z) &= \Arg(zw_\theta) = \Arg z + \Arg w_{\theta} = (\arg z + \pi-\theta)+2\pi\bb{Z}
    \end{align*}

    Veamos que $g_{\theta}(z)\notin \bb{R}^-$. Supongamos que $g_{\theta}(z)\in \bb{R}^-$. Entonces, $\exists k\in \bb{Z}$ tal que $\arg z + \pi-\theta = 2k\pi$. Por tanto, $\arg z = 2k\pi - \pi + \theta = (2k-1)\pi + \theta$. Por tanto, $\theta\in \Arg z$, lo cual es una contradicción. Por tanto, $g_{\theta}(z)\notin \bb{R}^-$.\\

    A continuación, definimos $\varphi$ como sigue:
    \Func{\varphi}{S_{\theta}}{\bb{R}}{z}{\arg (g_{\theta}(z)) - (\pi-\theta)}

    De esta forma, tenemos que $\varphi$ es continua en $S_{\theta}$, puesto que $\arg$ es continua en $\mathbb{C}^{\ast}\setminus \bb{R}^-$ y $g_{\theta}$ es continua en $S_{\theta}$. Además, dado $z\in S_{\theta}$, tenemos que:
    \begin{equation*}
        \varphi(z) \in \Arg g_{\theta}(z) - \Arg w_{\theta} = \Arg g_{\theta}(z) + \Arg\frac{1}{w_{\theta}} = \Arg\left(\frac{g_{\theta}(z)}{w_{\theta}}\right) = \Arg\left(\frac{zw_{\theta}}{w_{\theta}}\right) = \Arg z
    \end{equation*}
\end{ejercicio}

\begin{ejercicio}
    Probar que no existe ninguna función $\varphi \in \cc{C}(\mathbb{C}^\ast)$ de forma que $\varphi(z) \in \Arg z$ para todo $z \in \mathbb{C}^\ast$, y que el mismo resultado es cierto, sustituyendo $\mathbb{C}^\ast$ por $\bb{T} = \{z \in \mathbb{C} \mid |z| = 1\}$.\\

    Por reducción al absurdo, supongamos que existe una función $\varphi\in \cc{C}(\mathbb{C}^{\ast})$ tal que $\varphi(z)\in \Arg z~\forall z\in \mathbb{C}^{\ast}$. Definimos la siguiente función auxiliar:
    \Func{f}{\mathbb{C}^{\ast}}{\bb{R}}{z}{\varphi(z)-\varphi(-z)}

    Por ser $\varphi$ continua, $f$ es continua. Además, dado $z\in \mathbb{C}^{\ast}$, tenemos que:
    \begin{align*}
        f(z) &= \varphi(z)-\varphi(-z)\\
        f(-z) &= \varphi(-z)-\varphi(z) = -(\varphi(z)-\varphi(-z)) = -f(z)
    \end{align*}

    Por tanto, fijado $w\in \mathbb{C}^{\ast}$, hay dos opciones:
    \begin{itemize}
        \item Si $f(w)=0$, entonces sea $z_0=w$, y se tiene que $f(z_0)=0$.
        \item Si $f(w)\neq 0$, entonces $f(w)f(-w)<0$. Como $\bb{C}^{\ast}$ es conexo, por el Teorema del Valor Intermedio $\exists z_0\in \mathbb{C}^{\ast}$ tal que $f(z_0)=0$.
    \end{itemize}
    En cualquier caso, $\exists z_0\in \mathbb{C}^{\ast}$ tal que $f(z_0)=0$. Por tanto, $\varphi(z_0)=\varphi(-z_0)$. Esto implica que $\Arg z_0 = \Arg (-z_0)$, lo cual es una contradicción ya que:
    \begin{align*}
        \Arg -z_0 &= (\arg z_0 + \pi) + 2\pi\bb{Z}
    \end{align*}

    Por tanto, no puede existir una función $\varphi\in \cc{C}(\mathbb{C}^{\ast})$ tal que $\varphi(z)\in \Arg z~\forall z\in \mathbb{C}^{\ast}$.\\

    Por otro lado, consideramos el caso para $\bb{T}$. Hay diversas formas de probarlo:
    \begin{itemize}
        \item De forma análoga, haciendo uso ahora de que $\bb{T}$ es conexo.
        \item Aplicando de forma directa el Teorema de Borsuk-Ulam a $\varphi$ (esto es lo que en realidad hacemos en la opción anterior).
        \item Haciendo uso de lo anteriormente demostrado.
    \end{itemize}

    Desarrollaremos la tercera opción, por ser aquella que difiere de lo anterior. De nuevo, supongamos por reducción al absurdo que existe una función $\varphi\in \cc{C}(\bb{T})$ tal que $\varphi(z)\in \Arg z~\forall z\in \bb{T}$. Definimos la siguiente función auxiliar:
    \Func{f}{\bb{C}^{\ast}}{\bb{R}}{z}{\varphi\left(\dfrac{z}{|z|}\right)}

    Tenemos que $f$ es continua, y verifica que:
    \begin{equation*}
        f(z) = \varphi\left(\frac{z}{|z|}\right) \in \Arg\left(\frac{z}{|z|}\right) = \Arg z - \Arg (|z|) = \Arg z - 2\pi\bb{Z} = \Arg z
    \end{equation*}
    No obstante, hemos demostrado que no puede existir una función $f\in \cc{C}(\mathbb{C}^{\ast})$ tal que $f(z)\in \Arg z~\forall z\in \mathbb{C}^{\ast}$. Por tanto, hemos llegado a una contradicción, y concluimos que no puede existir una función $\varphi\in \cc{C}(\bb{T})$ tal que $\varphi(z)\in \Arg z~\forall z\in \bb{T}$.
\end{ejercicio}

\begin{ejercicio}
    Probar que la función $\Arg : \mathbb{C}^\ast \to \mathbb{R}/2\pi\mathbb{Z}$ es continua, considerando en $\mathbb{R}/2\pi\mathbb{Z}$ la topología cociente. Más concretamente, se trata de probar que, si $\{z_n\}$ es una sucesión de números complejos no nulos, tal que $\{z_n\} \to z \in \mathbb{C}^\ast$ y $\theta \in \Arg z$, se puede elegir $\theta_n \in \Arg z_n$ para todo $n \in \mathbb{N}$, de forma que $\{\theta_n\} \to \theta$.

    \begin{description}
        \item[Usando sucesiones:] Usaremos la caracterización que en el mismo enunciado describen. Dada una sucesión $\{z_n\}$ de números complejos no nulos, tal que $\{z_n\}\to z\in \mathbb{C}^{\ast}$ y $\theta\in \Arg z$, definimos $\theta_n$ como sigue:
        \begin{itemize}
            \item \ul{Si $z\notin \bb{R}^-$}:
            
            Como $\arg z\in \Arg z$, tenemos que $\exists k\in \bb{Z}$ tal que $\theta=2k\pi+\arg z$. Por tanto, definimos $\theta_n$ como:
            \begin{equation*}
                \theta_n = \arg z_n + 2k\pi \in \Arg z_n\qquad \forall n\in \mathbb{N}
            \end{equation*}

            Además, tenemos que:
            \begin{equation*}
                \left\{\theta_n\right\} = \left\{\arg z_n + 2k\pi\right\}\to \arg z + 2k\pi = \theta
            \end{equation*}
            donde hemos usado que, al ser $\arg$ continua en $z\in \mathbb{C}^{\ast}\setminus \bb{R}^-$, como se tiene que $\{z_n\}\to z$, entonces $\{\arg z_n\}\to \arg z$.
            
            \item \ul{Si $z\in \bb{R}^-$}:
            
            Por el Ejercicio~\ref{ej:2.2}, $\exists \varphi\in \cc{C}(S_{0})$ tal que $\varphi(w)\in \Arg w~\forall w\in S_{0}$. En particular, $\varphi(z)\in \Arg z$, por lo que $\exists k\in \bb{Z}$ tal que $\varphi(z)=\theta+2k\pi$.\\

            Como $\{z_n\}\to z\in S_0=S_0^\circ$ abierto, $\exists N\in \mathbb{N}$ tal que $\forall n\geq N$ se tiene que $z_n\in S_{0}$. Por tanto, definimos $\theta_n$ como:
            \begin{equation*}
                \begin{cases}
                    \theta_n = \arg z_n & \text{si } n<N\\
                    \theta_n = \varphi(z_n)-2k\pi & \text{si } n\geq N
                \end{cases}
            \end{equation*}

            De esta forma, tenemos que $\theta_n\in \Arg z_n~\forall n\in \mathbb{N}$, y además:
            \begin{equation*}
                \left\{\theta_n\right\} \to \varphi(z)-2k\pi = \theta
            \end{equation*}
            donde hemos usado que, al ser $\varphi$ continua en $z\in S_0$, como $\{z_n\}\to z$, se tiene que $\{\varphi(z_n)\}\to \varphi(z)$.
        \end{itemize}
        \begin{observacion}
            Notemos que podríamos haber generalizado todo en el segundo caso, considerando $S_{\theta+\pi}$. No obstante, se ha optado por hacerlo de forma más explícita para facilitar la comprensión, ya que el primer caso seguramente sea más intuitivo.
        \end{observacion}
        
        \item[Usando el punto de vista topoógico:]~\\
        
        Definimos la función proyección:
        \Func{\pi}{\mathbb{R}}{\mathbb{R}/2\pi\mathbb{Z}}{x}{x+2\pi\mathbb{Z}}

        Tenemos la siguiente descomposición de $\bb{C}^*$:
        \begin{equation*}
            \bb{C}^* = \left(\bb{C}^*\setminus \bb{R}^-\right)\cup \left(\bb{C}^*\setminus \bb{R}^+\right)
        \end{equation*}
        Tenemos que:
        \begin{itemize}
            \item \ul{En $\bb{C}^*\setminus \bb{R}^-$}:
            \begin{equation*}
                \Arg\left(z\right)=\left(\pi\circ \arg\right)(z)\qquad \forall z\in \bb{C}^*\setminus \bb{R}^-
            \end{equation*}

            Por tanto, $\Arg$ es continua en $\bb{C}^*\setminus \bb{R}^-$.

            \item \ul{En $\bb{C}^*\setminus \bb{R}^+$}:
            
            Por el Ejercicio~\ref{ej:2.2}, sabemos que $\exists \varphi\in \cc{C}(S_{0})$ tal que:
            \begin{equation*}
                \Arg\left(z\right)=\left(\pi\circ \varphi\right)(z)\qquad \forall z\in S_0=\bb{C}^*\setminus \bb{R}^+
            \end{equation*}

            Por tanto, $\Arg$ es continua en $\bb{C}^*\setminus \bb{R}^+$.
        \end{itemize}

        Por el carácter local de la continuidad, $\Arg$ es continua en $\bb{C}^*$.
    \end{description}
\end{ejercicio}

\begin{ejercicio}
    Dado $z \in \mathbb{C}$, probar que la sucesión $\left\{\left(1 + \dfrac{z}{n}\right)^n\right\}$ es convergente y calcular su límite.\\

    Para facilitar la notación, sea:
    \begin{equation*}
        z_n = \left(1 + \dfrac{z}{n}\right)^n\qquad \forall n\in \mathbb{N}
    \end{equation*}

    En primer lugar, vamos a estudiar el límite de la sucesión $\{|z_n|\}$:
    \begin{align*}
        |z_n| &= \left|\left(1 + \dfrac{z}{n}\right)^n\right| = \left|1 + \dfrac{z}{n}\right|^n = \left(\sqrt{\left(1+\dfrac{\Re z}{n}\right)^2 + \left(\dfrac{\Im z}{n}\right)^2}\right)^n
        =\\&= \sqrt{\left(1 + \dfrac{\Re^2z}{n^2} + \dfrac{2\Re z}{n} + \dfrac{\Im^2 z}{n^2}\right)^{n}}
        = \sqrt{\left(1 + \dfrac{\dfrac{\Re^2z + \Im^2z}{n}+2\Re z}{n}\right)^{n}}
    \end{align*}

    Por tanto, tenemos que:
    \begin{align*}
        \lim_{n\to \infty}|z_n| &= \sqrt{\lim_{n\to \infty}\left(1 + \dfrac{\dfrac{\Re^2z + \Im^2z}{n}+2\Re z}{n}\right)^{n}}
        =\\&= \sqrt{\exp\left(\lim_{n\to \infty}\dfrac{\Re^2z + \Im^2z + 2n\Re z}{n}+2\Re z\right)}
        = \sqrt{\exp(2\Re z)} = e^{\Re z}
    \end{align*}
    donde en la primera igualdad hemos usado que la raíz es una función continua, y en la segunda igualdad hemos usado el Criterio de Euler. A continuación, estudiamos los argumentos de $z_n$. Para ello, definimos:
    \begin{equation*}
        w_n = 1 + \dfrac{z}{n}\qquad \forall n\in \mathbb{N}
    \end{equation*}

    Como $\{w_n\}\to 1$, $\exists N\in \mathbb{N}$ tal que $\forall n\geq N$ se tiene que $\Re w_n>0$. Por tanto, $\forall n\geq N$ se tiene que:
    \begin{equation*}
        \arg w_n = \arctan\left(\dfrac{\Im w_n}{\Re w_n}\right) = \arctan\left(\dfrac{\Im z}{n+\Re z}\right)
    \end{equation*}

    Como $\Arg(zw)=\Arg z + \Arg w$ para todo $z,w\in \mathbb{C}^{\ast}$, tenemos que:
    \begin{equation*}
        \Arg z_n = \Arg \left((w_n)^n\right) = n\Arg w_n \Longrightarrow
        n\arctan\left(\dfrac{\Im z}{n+\Re z}\right)\in \Arg z_n\qquad \forall n\geq N
    \end{equation*}

    Por tanto, definimos la sucesión $\{\theta_n\}$ como sigue:
    \begin{equation*}
        \theta_n = \begin{cases}
            \arg z_n & \text{si } n<N\\
            n\arctan\left(\dfrac{\Im z}{n+\Re z}\right) & \text{si } n\geq N
        \end{cases}
    \end{equation*}

    Por tanto, para todo $n\in \mathbb{N}$, tenemos que $\theta_n\in \Arg z_n$. 
    Calculemos el límite de la sucesión $\{\theta_n\}$:
    \begin{align*}
        \lim_{n\to \infty} \theta_n &= \lim_{n\to \infty}n\arctan\left(\dfrac{\Im z}{n+\Re z}\right) = \lim_{n\to \infty}\dfrac{\arctan\left(\dfrac{\Im z}{n+\Re z}\right)}{\dfrac{1}{n}}
        =\\&= \lim_{n\to \infty}\dfrac{-n^2}{1+\left(\dfrac{\Im z}{n+\Re z}\right)^2}\cdot \dfrac{-\Im z}{(n+\Re z)^2}
        = \lim_{n\to \infty}\dfrac{n^2\Im z}{(n+\Re z)^2+\Im^2 z} = \Im z
    \end{align*}
    
    
    Uniendo ambos resultados, tenemos que:
    \begin{align*}
        z_n = |z_n|\left(\cos(\theta_n) + i\sen(\theta_n)\right) \qquad \forall n\in \mathbb{N}
    \end{align*}

    Tomando límite, y como las funciones seno y coseno son continuas, tenemos que:
    \begin{equation*}
        \lim_{n\to \infty}z_n = \lim_{n\to \infty}|z_n|\left(\cos\left(\lim_{n\to \infty}\theta_n\right) + i\sen\left(\lim_{n\to \infty}\theta_n\right)\right) = e^{\Re z}\left(\cos(\Im z) + i\sen(\Im z)\right)
    \end{equation*}
\end{ejercicio}
    \section{Espacios recubridores}

\begin{ejercicio}
    Sea $R=\left]-1,1\right[\subset \mathbb{R}$. Demuestra que existe una aplicación recubridora $p:R\to \mathbb{S}^1$. ¿Se puede levantar al recubridor la aplicación $f:\mathbb{S}^1\to\mathbb{S}^1$ dada por $f(x,y) = (y,x)$?\\

    \noindent
    Sabemos que $R$ es homeomorfo a $\mathbb{R}$ y que la aplicación $p_0:\mathbb{R}\to\mathbb{S}^1$ dada por:
    \begin{equation*}
        p_0(x) = (\cos(2\pi x), \sen(2\pi x))
    \end{equation*}
    es una aplicación recubridora. Si tomamos $h:R\to \mathbb{R}$ cualquier homeomofismo tendremos entonces que $p=p_0\circ h:R\to\mathbb{S}^1$ es una aplicación recubridora (como vimos en el Tema 1). En vistas del segundo apartado daremos $h$ de forma explícita, como por ejemplo:
    \begin{equation*}
        h(t) = \tg\left(\frac{\pi}{2}t\right)
    \end{equation*}
    Si consideramos la aplicación $f$ enunciada, estamos en la siguiente situación:
    \begin{figure}[H]
        \centering
        \shorthandoff{""}
        \begin{tikzcd}
                                        & R \arrow[d, "p"] \\
            \mathbb{S}^1 \arrow[r, "f"] & \mathbb{S}^1    
        \end{tikzcd}
        \shorthandon{""}
    \end{figure}
    \noindent
    Fijado $x_0 = (0,1)\in \mathbb{S}^1$, tomamos $b_0 = f(x_0) = (1,0)$ y $r_0 = 0\in p^{-1}(b_0)$. Tenemos que existe un levantamiento $\hat{f}:\mathbb{S}^1\to R$ de $f$ con $\hat{f}(x_0) = r_0$ si y solo si:
    \begin{equation}\label{eq:ej1_rel3}
        f_\ast(\pi_1(\mathbb{S}^1,(0,1))) \subseteq p_\ast(\pi_1(R,0))
    \end{equation}
    Pero tenemos que:
    \begin{equation*}
        \pi_1(R,0) = \{[\varepsilon_0]\} \quad\Longrightarrow\quad p_\ast(\pi_1(R,0)) = \{[\varepsilon_{(1,0)}]\}
    \end{equation*}
    $f_\ast$ es un isomorfismo por ser $f$ un homeomorfismo, por lo que:
    \begin{equation*}
        \pi_1(\mathbb{S}^1,(0,1))\cong \mathbb{Z} \quad\Longrightarrow\quad f_\ast(\pi_1(\mathbb{S}^1,(0,1))) \cong \mathbb{Z}
    \end{equation*}
    como no se puede cumplir la ecuación~\eqref{eq:ej1_rel3} tenemos que no existe dicho levantamiento. El mismo razonamiento puede repetirse para todo punto $x_0\in \mathbb{S}^1$.
\end{ejercicio}

\begin{ejercicio} 
    Dada la aplicación recubridora estándar $p:\mathbb{R}\to\mathbb{S}^1$ definida por
    \begin{equation*}
        p(x) = (\cos(2\pi x), \sen(2\pi x)),
    \end{equation*}
    determina si la aplicación $f:\mathbb{S}^1\to\mathbb{S}^1$ dada por $f(x,y) = (x,|y|)$ puede ser levantada al recubridor. Y, en tal caso, calcula sus levantamientos.\\

    \noindent
    Fijado $x_0=(1,0)\in \mathbb{S}^1$, tomamos $b_0 = f(x_0) = (1,0)$ y consideramos el punto $r_0=0\in p^{-1}(b_0)$. Existirá un levantamiento $\hat{f}:\mathbb{S}^1\to \mathbb{R}$ de $f$ con $\hat{f}(x_0) = r_0$ si y solo si:
    \begin{equation*}
        f_\ast(\pi_1(\mathbb{S}^1,x_0)) \subseteq p_\ast(\pi_1(\mathbb{R},r_0))
    \end{equation*}
    Tenemos que:
    \begin{equation*}
        \pi_1(\mathbb{R},r_0) = \{[\varepsilon_{r_0}]\} \quad\Longrightarrow\quad p_\ast(\pi_1(\mathbb{R},r_0)) = \{[\varepsilon_{b_0}]\}
    \end{equation*}
    Por otra parte, $\pi_1(\mathbb{S}^1,x_0)$ está generado por $[\alpha]$, con $\alpha:[0,1]\to \mathbb{S}^1$ dado por:
    \begin{equation*}
        \alpha(t) = (\cos(2\pi t), \sen(2\pi t))
    \end{equation*}

    y tenemos que:
    \begin{equation*}
        f_\ast([\alpha]) = [f\circ \alpha] 
    \end{equation*}

    donde:
    \begin{equation*}
        (f\circ \alpha)(t) = (\cos(2\pi t), |\sen(2\pi t)|)
    \end{equation*}
    que claramente es un arco homotópico a $\varepsilon_{b_0}$, por lo que tenemos:
    \begin{equation*}
        f_\ast(\pi_1(\mathbb{S}^1,x_0)) = \{[\varepsilon_{b_0}]\}
    \end{equation*}
    De donde existe una única $\hat{f}:\mathbb{S}^1\to \mathbb{R}$ aplicación continua con $\hat{f}(x_0) = r_0$. % // TODO: Calcular levantamiento
\end{ejercicio}

\begin{ejercicio}
    ¿Existe una aplicación recubridora desde $\mathbb{S}^1\times \mathbb{S}^1$ en $\mathbb{S}^1$?\\

    \noindent
    No: por reducción al absurdo, si existiera una aplicación recubridora $p:\mathbb{S}^1\times \mathbb{S}^1\to \mathbb{S}^1$:
    \begin{description}
        \item [Opción 1.] Tendríamos entonces por el Teorema de Monodromía que la aplicación $p$ induce un homomorfismo inyectivo $p_\ast:\pi_1(\mathbb{S}^1\times \mathbb{S}^1,x_0)\to \pi_1(\mathbb{S}^1,p(x_0))$ con $\pi_1(\mathbb{S}^1\times \mathbb{S}^1,x_0)\cong \mathbb{Z}\times \mathbb{Z}$ y $\pi_1(\mathbb{S}^1,p(x_0))\cong \mathbb{Z}$. Sin embargo, ningún subgrupo de $\mathbb{Z}$ es isomorfo a $\mathbb{Z}\times \mathbb{Z}$, por lo que llegamos a una contradicción.
        \item [Opción 2.] $\mathbb{R}$ es el recubridor universal de $\mathbb{S}^1$, por lo que entonces este también recubre a $\mathbb{S}^1\times \mathbb{S}^1$, y $\mathbb{R}^2$ también es recubridor universal de $\mathbb{S}^1\times \mathbb{S}^1$, por lo que ha de existir un isomorfismo de recubridores (y por tanto, un homeomorfismo) entre $\mathbb{R}$ y $\mathbb{R}^2$, algo imposible, pues $\mathbb{R}$ menos un punto no es conexo y $\mathbb{R}^2$ menos un punto sí lo es.
    \end{description}
\end{ejercicio}

\begin{ejercicio}
    Determina, salvo isomorfismos, todos los recubridores del cilindro $\mathbb{S}^1\times \mathbb{R}$.\\

    \noindent
    Fijado $b_0 = (1,0,0)\in \mathbb{S}^1\times \mathbb{R}$, tenemos que $\pi_1(\mathbb{S}^1\times \mathbb{R},b_0)\cong \mathbb{Z}$, y los subgrupos de $\mathbb{Z}$ son:
    \begin{equation*}
        H_k = k\mathbb{Z} \qquad k \in \mathbb{N}\cup \{0\}
    \end{equation*}
    \begin{itemize}
        \item Asociado a $H_0 = \{0\}$ tenemos el recubridor $(\mathbb{R}^2,p)$, donde $p = p_0\times Id_\mathbb{R}$, con $p_0:\mathbb{R}\to \mathbb{S}^1$ la aplicación recubridora estándar.
        \item Asociado a $H_k$ con $k\geq 1$ tenemos el recubridor $(\mathbb{S}^1\times \mathbb{R}, p_k)$, con la aplicación recubridora $p_k:\mathbb{S}^1\times \mathbb{R}\to \mathbb{S}^1\times \mathbb{R}$ dada por:
            \begin{equation*}
                p_k(\cos\theta, \sen\theta, y) = (\cos(k\theta), \sen(k\theta), y)
            \end{equation*}
    \end{itemize}
\end{ejercicio}

\begin{ejercicio} % // TODO: HACER
    Demuestra que si $p:R\to B$ es un homeomorfismo local con $R$ compacto y $B$ Hausdorff (y conexo), entonces $p$ es una aplicación recubridora.\\

    \noindent
    Recordamos que si $p$ es un homeomorfismo local entonces es una aplicación continua de forma que para cada punto $x\in R$ existe un abierto $U_x$ entorno de $x$ de forma que $p(U_x)$ es abierto en $B$ y $p\big|_{U_x}:U_x\to p(U_x)$ es un homeomorfismo.

    \begin{itemize}
        \item Tenemos que $p$ es continua por ser un homeomorfismo local.
        \item Como $R$ es compacto y $B$ es Hausdorff tenemos que $p$ es cerrada.
        \item Sea $U\subseteq R$ un abierto, podemos escribir:
            \begin{equation*}
                U = \bigcup_{x\in U}U_x
            \end{equation*}
            de donde $U_x\cap U$ es un abierto para cada $x\in U$, por lo que
            \begin{equation*}
                U = \bigcup_{x\in U} U_x \cap U
            \end{equation*}
            Para cada $x\in X$ tenemos que $p\big|_{U_x}:U_x\to p(U_x)$ es un homeomorfismo, por lo que $p(U_x\cap U)$ es un abierto de $B$. Finalmente, vemos:
            \begin{equation*}
                p(U) = p\left(\bigcup_{x\in U} U_x\cap U\right) = \bigcup_{x\in U} p(U_x\cap U)
            \end{equation*}
            Por lo que $p(U)$ es abierto, de donde $p$ es una aplicación abierta.
    \end{itemize}
    Como $p$ es abierta y cerrada tenemos que $p(R)\subseteq B$ es abierto y cerrado, como $B$ es conexo ha de ser $p(R) = B$, por lo que $p$ es sobreyectiva. Falta probar que todo punto $b\in B$ tiene un entorno abierto regularmente recubierto. % // TODO:
\end{ejercicio}

\begin{ejercicio} % // TODO: HACER
    Sea $p:\mathbb{R}^n\to \mathbb{R}^n$ un homeomorfismo local tal que para todo $r>0$ se tiene que $p^{-1}(\overline{B}(0,r))$ es compacto. Demuestra que $p$ es un homeomorfismo.
\end{ejercicio}

\begin{ejercicio}
    Sea $X$ conexo y localmente arcoconexo con grupo fundamental finito. Si $f,g:X\to \mathbb{R}$ son aplicaciones continuas cumpliendo que $f(x)^2 + g(x)^2 = 1$ para todo $x\in X$. Prueba que existe $h:X\to\mathbb{R}$ continua tal que $\cos(h(x)) = f(x)$ y $\sen(h(x)) = g(x)$ para cada $x\in X$.\\

    \noindent
    Observando la condición que cumplen $f$ y $g$ así como de la presencia de senos y cosenos pensamos en que el problema está relacionado con una circunferencia. Definimos por tanto $F:X\to \mathbb{S}^1$ dada por:
    \begin{equation*}
        F(x) = (f(x),g(x))
    \end{equation*}
    que está bien definida, pues $f(x)^2+g(x)^2 = 1$, por lo que $F(x)\in \mathbb{S}^1$ para todo $x\in X$. Si consideramos la aplicación recubridora estándar $p_0:\mathbb{R}\to \mathbb{S}^1$, fijado $x_0\in X$, $b_0 = F(x_0)$ y tomando $r_0\in p^{-1}(b_0)$ tenemos que existe un levantamiento $\hat{F}:X\to \mathbb{R}$ de $F$ con $\hat{F}(x_0) = r_0$ si y solo si:
    \begin{equation*}
        F_\ast(\pi_1(X,x_0)) \subseteq p_\ast(\pi_1(\mathbb{R},r_0))
    \end{equation*}
    Por una parte:
    \begin{equation*}
        \pi_1(\mathbb{R},r_0) = \{[\varepsilon_{r_0}]\} \quad\Longrightarrow\quad p_\ast(\pi_1k(\mathbb{R},r_0)) = \{[\varepsilon_{b_0}]\}
    \end{equation*}
    Por otra tenemos que como $\pi_1(X,x_0)$ es un grupo finito ha de ser por tanto $F_\ast(\pi_1(X,x_0))$ un subgrupo finito de $\pi_1(\mathbb{S}^1,b_0)\cong \mathbb{Z}$, que solo tiene un subgrupo finito, $\{[\varepsilon_{b_0}]\}$, de donde deducimos que ha de ser $F_\ast(\pi_1(X,x_0)) = \{[\varepsilon_{b_0}]\}$. Sea por tanto $\hat{F}:X\to \mathbb{R}$ el único levantamiento de $F$ con $\hat{F}(x_0) = r_0$, tenemos que:
    \begin{equation*}
        (\cos(2\pi \hat{F}(x)), \sen(2\pi \hat{F}(x))) = p(\hat{F}(x)) = F(x) = (f(x),g(x)) \qquad \forall x\in X
    \end{equation*}
    Por lo que si definimos $h:X\to \mathbb{R}$ dada por $h(x) = 2\pi \hat{F}(x)$, tenemos que:
    \begin{equation*}
        (\cos(h(x)), \sen(h(x))) = F(x) = (f(x),g(x)) \qquad \forall x\in X
    \end{equation*}
\end{ejercicio}

\begin{ejercicio} 
    Sean $p:X\to Y$ y $f:Y\to Z$ dos aplicaciones continuas tales que $p$ y $f\circ p$ son aplicaciones recubridoras. Prueba que $f$ es también una aplicación recubridora.\\

    \noindent
    Como $f\circ p$ es una aplicación recubridora es en particular sobreyectiva, por lo que $f$ también es sobreyectiva. Basta ver que para todo $z\in Z$ existe un entorno abierto de $z$ regularmente recubierto. Fijado $z\in Z$, tomamos $O_z$ un entorno abierto y arcoconexo de $z$ regularmente recubierto por la aplicación recubridora $f\circ p$, por lo que:
    \begin{equation*}
        p^{-1}(f^{-1}(O_z)) = (f\circ p)^{-1}(O_z) = \biguplus_{i \in I}A_i
    \end{equation*}
    con $A_i\subseteq X$ abierto y $(f\circ p)\big|_{A_i}:A_i\to O_z$ homeomorfismo para cada $i \in I$. Si aplicamos $p$ a dicha igualdad, obtenemos que:
    \begin{equation*}
        \bigcup_{i \in I}p(A_i) = p\left(\bigcup_{i \in I}A_i\right) = p(p^{-1}(f^{-1}(O_z))) \AstIg f^{-1}(O_z)
    \end{equation*}
    donde en $(\ast)$ hemos usado que $p$ es sobreyectiva, y tenemos además que $p(A_i)$ es abierto para cada $i \in I$, ya que $p$ es una aplicación abierta por ser una aplicación recubridora.\\

    \noindent
    Veamos ahora que si $p(A_i)\cap p(A_j)\neq \emptyset  \Longrightarrow p(A_i) = p(A_j)$: sea $y \in p(A_i)\cap p(A_j)$ existen entonces $x_i \in A_i$, $x_j \in A_j$ de forma que $p(x_i) = y = p(x_j)$, basta demostrar que $p(A_i)\subseteq p(A_j)$ y la otra inclusión será análoga. Para ello, sea $y'\in p(A_i)$, tenemos que existe $x'\in A_i$ de forma que $p(x') = y'$. Tomamos ahora $w=f(y), w'=f(y')$ y tendremos que $w,w'\in O_z$. Como $O_z$ es arcoconexo, existirá $\gamma:[0,1]\to Z$ de forma que $\gamma(0) = w$, $\gamma(1) = w'$. Si tomamos como $\hat{\gamma}$ el único levantamiento de $\gamma$ por $f\circ p$ con $\hat{\gamma}(0) = x_i$, como $Im \hat{\gamma}$ es un conjunto conexo ha de ser $Im \hat{\gamma}\subseteq A_i$ y como $(f\circ p)\big|_{A_i}:A_i\to O_z$ es un homeomorfismo, ha de ser $\hat{\gamma}(1) = x'$, ya que $(f\circ p)^{-1}(\{w'\})\cap A_i = \{x'\}$.\\

    \noindent
    Si consideramos ahora $\delta = p\circ \hat{\gamma}$ tenemos que:
    \begin{equation*}
        \delta(0) = p(x_i) = y, \qquad \delta(1) = p(x') = y'
    \end{equation*}
    y consideramos como $\hat{\delta}$ el único levantamiento de $\delta$ por $p$ con $\hat{\delta}(0) = x_j$. Tendremos:
    \begin{equation*}
        (f\circ p)\circ \hat{\delta} = f\circ \delta = f\circ p \circ \hat{\gamma} = \gamma
    \end{equation*}
    Por lo que $\hat{\delta}$ es el único levantamiento de $\gamma$ por $f\circ p$ con $\hat{\delta}(0) = x_j \in A_j$. Tendrá que ser $Im \hat{\delta}\subseteq A_j$, luego $\hat{\delta}(1)\in A_j$ y tenemos que:
    \begin{equation*}
        p(\hat{\delta}(1)) = \delta(1) = y'
    \end{equation*}
    por lo que $y'\in p(A_j)$, como queríamos probar. De esta forma, si eliminamos de la unión:
    \begin{equation*}
        \bigcup_{i \in I}p(A_i)
    \end{equation*}
    los conjuntos repetidos, obtendremos una unión disjunta, luego será:
    \begin{equation*}
        f^{-1}(O_z) = \biguplus_{j \in J}p(A_j)
    \end{equation*}
    y como $(f\circ p)\big|_{A_j}:A_j\to O_z$ es un homeomorfismo para cada $j \in J$ tenemos que:
    \begin{equation*}
        (f\circ p)\big|_{A_j} = f\big|_{p(A_j)} \circ p\big|_{A_j}
    \end{equation*}
    Por lo que $f\big|_{p(A_j)}:A_j\to O_z$ es inyectiva y con todas las demás propiedades que hemos probado de $f$ deducimos que $f\big|_{p(A_j)}:p(A_j)\to O_z$ es un homeomorfismo para cada $j\in J$. En definitiva, hemos probado que la aplicación $f$ es una aplicación recubridora.
\end{ejercicio}

\begin{ejercicio}
    Sean $p_1:X\to Y$ y $p_2:Y\to Z$ dos aplicaciones recubridoras. Prueba que si $Z$ tiene recubridor universal, entonces $p_2\circ p_1:X\to Z$ es una aplicación recubridora.\\

    \noindent
    Si $Z$ tiene recubridor universal $(R,p)$ tenemos que como $(Y,p_2)$ también recubre a $Z$ existirá entonces un homomorfismo de recubridores $\phi_1$ de $(R,p)$ en $(Y,p_2)$, por lo que $R$ es recubridor universal de $Y$. Repetiendo el razonamiento con el recubridor $(X,p_1)$ de $Y$ tenemos que existe un homomorfismo de recubridores $\phi_2$ de $(R,p)$ en $(X,p_1)$, obteniendo que el siguiente diagrama es conmutativo:
    \begin{figure}[H]
        \centering
        \shorthandoff{""}
        \begin{tikzcd}
                               &                    & R \arrow[d, "p"] \arrow[ld, "\phi_1", bend right] \arrow[lld, "\phi_2", bend right] \\
            X \arrow[r, "p_1"] & Y \arrow[r, "p_2"] & Z                                                                                  
        \end{tikzcd}
        \shorthandon{""}
    \end{figure}
    \noindent
    De esta forma, tenemos que:
    \begin{equation*}
        (p_2\circ p_1)\circ \phi_2 = p
    \end{equation*}
    con $\phi_2$ y $p$ aplicaciones recubridoras, por lo que por el ejercicio anterior tenemos que $p_2\circ p_1$ es una aplicación recubridora.
\end{ejercicio}

\begin{ejercicio}
    Sean $p:R\to B$ una aplicación recubridora y $b_0\in B$. Definimos la aplicación (correspondencia del levantamiento generalizada)
    \Func{\phi}{p^{-1}(\{b_0\})\times \pi_1(B,b_0)}{p^{-1}(\{b_0\})}{(r,[\alpha])}{\hat{\alpha}_r(1)}
    donde $\hat{\alpha}_r(s)$ es el único levantamiento de $\alpha(s)$ con condición inicial $\hat{\alpha}_r(0)=r$. Demuestra que:
    \begin{enumerate}[label=\alph*)]
        \item $\phi$ está bien definida.

            Fijado $r\in p^{-1}(b_0)$ tenemos que la aplicación correspondencia del levantamiento 
            \Func{\phi_r}{\pi_1(B,b_0)}{p^{-1}(\{b_0\})}{[\alpha]}{\hat{\alpha}(1)}
            donde $\hat{\alpha}$ es el único levantamiento de $\alpha$ con condición inicial $\hat{\alpha}(0) = r$ está bien definida, por lo que $\phi$ estará bien definida.
        \item $\phi(r,[\varepsilon_{b_0}])=r$, para cualquier $r\in p^{-1}(\{b_0\})$.

            Sea $r\in p^{-1}(\{b_0\})$, si consideramos $\varepsilon_{r}$ tenemos que:
            \begin{equation*}
                p\circ \varepsilon_r = \varepsilon_{b_0}, \qquad \varepsilon_r(0) = r
            \end{equation*}
            Por lo que $\varepsilon_r$ es el único levantamiento de $\varepsilon_{b_0}$ con $\varepsilon_r(0) = r$, de donde ha de ser:
            \begin{equation*}
                \phi_r(r,[\varepsilon_{b_0}]) = \varepsilon_r(1) = r
            \end{equation*}
        \item $\phi(\phi(r,[\alpha]),[\beta]) = \phi(r,[\alpha]\ast[\beta])$, para cualesquiera $r\in p^{-1}(\{b_0\})$ y $[\alpha],[\beta]\in \pi_1(B,b_0)$.

            Sea $\hat{\alpha}_r$ el único levantamiento de $\alpha$ con $\hat{\alpha}_r(0) = r$ tenemos entonces que $\phi(r,[\alpha]) = \hat{\alpha}_r(1) = r_1\in p^{-1}(\{b_0\})$. Sea $\hat{\beta}_{r_1}$ el único levantamiento de $\beta$ con $\hat{\beta}_{r_1}(0) = r_1$, tenemos que $\phi(r_1,[\beta]) = \hat{\beta}_{r_1}(1)$.

            Veamos finalmente que $\hat{\alpha}_r\ast \hat{\beta}_{r_1}$ es un levantamiento de $\alpha\ast\beta$, que además cumple:
            \begin{equation*}
                (\hat{\alpha}_r\ast\hat{\beta}_{r_1})(0) = \hat{\alpha}_r(0) = r
            \end{equation*}
            Para ello, vemos que:
            \begin{align*}
                p((\hat{\alpha}_r\ast\hat{\beta}_{r_1})(t)) &= p\left(
                \left\{\begin{array}{ll}
                        \hat{\alpha}_r(2t) & \text{si\ } 0\leq t\leq \nicefrac{1}{2} \\ 
                        \hat{\beta}_{r_1}(2t-1) & \text{si\ } \nicefrac{1}{2}\leq t\leq 1
                \end{array}\right. \right) \\
                            &= \left\{\begin{array}{ll}
                                    p(\hat{\alpha}_r(2t)) & \text{si\ } 0\leq t\leq \nicefrac{1}{2} \\
                                    p(\hat{\beta}_{r_1}(2t-1)) & \text{si\ } \nicefrac{1}{2}\leq t\leq 1
                                    \end{array}\right.  \\ &= \left\{\begin{array}{ll}
                                \alpha(2t) & \text{si\ } 0\leq t\leq\nicefrac{1}{2} \\
                                \beta(2t-1) & \text{si\ } \nicefrac{1}{2}\leq t\leq 1
                            \end{array}\right.  = (\alpha\ast \beta)(t)
            \end{align*}
            Por lo que ha de ser:
            \begin{equation*}
                \phi(r,[\alpha]\ast [\beta]) = \phi(r,[\alpha\ast \beta]) = (\hat{\alpha}_r\ast\hat{\beta}_{r_1})(1) = \hat{\beta}_{r_1}(1) = \phi(\phi(r,[\alpha]),[\beta])
            \end{equation*}
        \item $\phi$ es sobreyectiva.

            En efecto, sea $r\in p^{-1}(b_0)$, tenemos que $\phi(r,[\varepsilon_{b_0}]) = r$.
        \item $\phi(r,[\alpha]) = r$ si y solo si $[\alpha] \in p_\ast(\pi_1(R,r))$.
            Por doble implicación:
            \begin{description}
                \item [$\Longleftarrow )$] Si $[\alpha]\in p_\ast(\pi_1(R,r))$ tenemos entonces que existe $[\beta]\in \pi_1(R,r)$ de forma que $[p\circ \beta] = p_\ast([\beta]) = [\alpha]$, por lo que $\beta$ es un levantamiento de $\alpha$, que además cumple $\beta(0) = r$, por lo que:
                    \begin{equation*}
                        \phi(r,[\alpha]) = \hat{\alpha}_r(0) = \beta(0) = r
                    \end{equation*}
                \item [$\Longrightarrow )$] Si $\phi(r,[\alpha]) = r$ tenemos entonces que el único levantamiento $\hat{\alpha}_r$ de $\alpha$ con $\hat{\alpha}_r(0) = r$ es un lazo en $R$, $[\hat{\alpha}_r]\in \pi_1(R,r)$, y tenemos que $p_\ast([\hat{\alpha}_r]) = [\alpha]$ por ser $\hat{\alpha}_r$ levantamiento de $\alpha$.
            \end{description}
        \item El cardinal de $p^{-1}(\{b_0\})$ coincide con el cardinal de $\pi_1(B,b_0)/p_\ast(\pi_1(R,r))$ (es decir, el índice de $p_\ast(\pi_1(R,r))$ como subgrupo de $\pi_1(B,b_0)$). % // TODO: TERMINAR
    \end{enumerate}
\end{ejercicio}

\begin{ejercicio}\label{ej:11_rel3}
    Sea $X$ un espacio topológico (conexo y localmente arcoconexo), $G$ un grupo de homeomorfismos de $X$ y $X/\cc{R}_G$ el espacio topológico cociente dado por la relación de equivalencia:
    \begin{equation*}
        x\cc{R}_G y \quad\Longleftrightarrow\quad\exists \varphi \in G : y = \varphi(x)
    \end{equation*}
    para cualesquiera $x,y\in X$.\newline
    Demuestra que la aplicación proyección $\pi:X\to X/\cc{R}_G$ es recubridora si y solo si para cada $x\in X$ existe un entorno suyo $U_x$ tal que $\varphi(U_x)\cap U_x=\emptyset $ para todo $\varphi\in G\setminus \{Id_X\}$.\newline
    Deduce que, además, $\varphi:(X,\pi)\to (X,\pi)$ es un isomorfismo de recubridores si y solo si $\varphi \in G$.
\end{ejercicio}

\begin{ejercicio}
    Para cada $n\in \mathbb{Z}$ se define $f_n:\mathbb{R}^2\to\mathbb{R}^2$ como $f_n(x,y) = (x+2n,{(-1)}^{n}y)$. Utiliza el ejercicio anterior para demostrar que:
    \begin{enumerate}[label=\alph*)]
        \item $G = \{f_n:n\in \mathbb{Z}\}$ es un grupo de homeomorfismos de $\mathbb{R}^2$ y para cada $x\in \mathbb{R}^2$ existe un entorno suyo $U_x$ tal que $f_n(U_x)\cap U_x = \emptyset $ para todo $n\in \mathbb{Z}\setminus \{0\}$.
        \item La proyección $p:\mathbb{R}^2\to\mathbb{R}^2/\cc{R}_G$ es una aplicación recubridora de $\mathbb{R}^2$ en la cinta de Moebius $\mathbb{R}^2/\cc{R}_G$.
    \end{enumerate}
\end{ejercicio}

\begin{ejercicio}
    Para cada $n,m\in \mathbb{Z}$ se define $f_{n,m}:\mathbb{R}^2\to\mathbb{R}^2$ como:
    \begin{equation*}
        f_{n,m}(x,y) = (x,{(-1)}^{n}y) + 2(n,{(-1)}^{n}m).
    \end{equation*}
    Utiliza el ejercicio~\ref{ej:11_rel3} para demostrar que:
    \begin{enumerate}[label=\alph*)]
        \item $G=\{f_{n,m}:n,m\in \mathbb{Z}\}$ es un grupo de homeomorfismos de $\mathbb{R}^2$ y para cada $x\in \mathbb{R}^2$ existe un entorno suyo $U_x$ tal que $f_{n,m}(U_x)\cap U_x=\emptyset $ para todo $n,m\in \mathbb{Z}$ con $(n,m)\neq (0,0)$.
        \item La proyección $p:\mathbb{R}^2\to\mathbb{R}^2/\cc{R}_G$ es una aplicación recubridora de $\mathbb{R}^2$ en la botella de Klein $\mathbb{R}^2/\cc{R}_G$.
    \end{enumerate}
\end{ejercicio}

\begin{ejercicio}
    Razona si son verdaderas o falsas las siguientes afirmaciones:
    \begin{enumerate}[label=\alph*)]
        \item Existe una aplicación recubridora $p:\mathbb{S}^1\to [0,1]$.

            Es falsa, ya que si existiera una aplicación recubridora $p:\mathbb{S}^1\to [0,1]$ el Teorema de Monodromía nos diría que el homomorfismo inducido por la aplicación $p$ $p_\ast:\pi_1(\mathbb{S}^1,x_0)\to \pi_1([0,1],p(x_0))$ es inyectivo, con $\pi_1(\mathbb{S}^1,x_0)\cong \mathbb{Z}$ y $\pi_1([0,1],p(x_0)) = \{[\varepsilon_{x_0}]\}$, lo que llevaría a una contradicción.
        \item Existe una aplicación recubridora $p:\mathbb{R}\bb{P}^2\to \mathbb{S}^1$.

            Es falsa, ya que si existiera una aplicación recubridora $p:\mathbb{R}\bb{P}^2\to \mathbb{S}^1$:
            \begin{description}
                \item [Opción 1.] el Teorema de Monodromía nos diría que el homomorfismo inducido $p_\ast:\pi_1(\mathbb{R}\bb{P}^2,x_0)\to \pi_1(\mathbb{S}^1,p(x_0))$ es inyectivo, con $\pi_1(\mathbb{R}\bb{P}^2,x_0)\cong \mathbb{Z}_2$, pero entonces tenemos que $p_\ast(\pi_1(\mathbb{R}\bb{P}^2,x_0)) \cong \mathbb{Z}_2$ y subgrupo de $\pi_1(\mathbb{S}^1,p(x_0))\cong \mathbb{Z}$, que es una contradicción.
                \item [Opción 2.] Como $\mathbb{R}$ es el recubridor universal de $\mathbb{S}^1$ tendríamos entonces que $\mathbb{R}$ recubriría a $\mathbb{R}\bb{P}^2$, y el recubridor universal de $\mathbb{R}\bb{P}^2$ es $\mathbb{S}^2$, por lo que tendríamos entonces un homeomorfismo entre $\mathbb{R}$ y $\mathbb{S}^2$, algo imposible, pues $\mathbb{S}^2$ es compacto y $\mathbb{R}$ no.
            \end{description}
        \item El semiplano $X=\{(x,y)\in \mathbb{R}^2 : y\geq 0\}$ es el recubridor universal de la bola cerrada punteada $Y = \{(x,y)\in \mathbb{R}^2 : 0<x^2+y^2 \leq 1\}$.

            Es verdadera, es claro que $X$ es simplemente conexo, por lo que basta ver que existe alguna aplicación recubridora $p:X\to Y$.
            \begin{figure}[H]
                \centering
                \begin{tikzpicture}
                    \filldraw[fill=blue!40, draw=blue, opacity=0.3] (0,0) rectangle (3,2);
                    \draw[thick] (0,0) -- (3,0);
                    \fill(1.5,1) node[] {$X$};
                    \draw[thick](6,1) circle(1cm);
                    \draw[fill = red!40, draw=red, opacity=0.3](6,1) circle(1cm);
                    \fill(5.5,1) node[] {$Y$};
                    \draw[fill=white] (6,1) circle (2pt);
                \end{tikzpicture}
            \end{figure}
            Sabemos ya de la existencia de una aplicación recubridora $p_0:\mathbb{R}\to \mathbb{S}^1$, por lo que sabemos llevarnos el ``borde'' de $X$ al ``borde'' de $Y$. Si repetimos el proceso subiendo la altura en $X$ y achicando el radio en $Y$ conseguiremos una aplicación recubridora $p:X\to Y$. La idea es buscar una aplicación que en $0$ valga $1$ y que su imagen tienda a $0$ en infinito. Tomamos por tanto $p:X\to Y$ dada por:
            \begin{equation*}
                p(x,y) = e^{-y}p_0(x)
            \end{equation*}
            \begin{figure}[H]
                \centering
                \begin{tikzpicture}
                    \filldraw[fill=blue!40, draw=blue, opacity=0.3] (0,0) rectangle (3,2);
                    \draw[thick] (0,0) -- (3,0);
                    \fill(1.5,1) node[] {$X$};
                    \draw[thick](6,1) circle(1cm);
                    \draw[fill = red!40, draw=red, opacity=0.3](6,1) circle(1cm);
                    \fill(5.5,1) node[] {$Y$};
                    \draw[fill=white] (6,1) circle (2pt);

                    \draw[thick, draw=green] (0,0.3) -- (3,0.3);
                    \draw[thick, draw=green] (6,1) circle(0.8cm);
                    \draw[thick, draw=purple] (0,1.7) -- (3,1.7);
                    \draw[thick, draw=purple] (6,1) circle(0.3cm);
                \end{tikzpicture}
            \end{figure}
            Es claro que $p$ es continua y sobreyectiva. Se comprueba además que $p$ es una aplicación recubridora.

        \item Si $X$ es un espacio topológico (conexo y localmente arcoconexo) con grupo fundamental finito y $p:X\to X$ es una aplicación recubridora, entonces $p$ es un homeomorfismo.

            Es verdadera. Para probar que $p$ es un homeomorfismo basta probar que $p$ es inyectiva, pues al ser una aplicación recubridora tenemos ya que es continua, sobreyectiva y abierta. Para ver que es inyectiva, tomamos dos puntos $x,y\in X$ con $p(x) = p(y)$, como $X$  es conexo y localmente arcoconexo vimos en el Tema 1 que entonces $X$ es arcoconexo, por lo que en particular ha de existir un arco $\alpha:[0,1]\to X$ de forma que $\alpha(0) = x$ y $\alpha(1) = y$. Si consideramos el arco $p\circ \alpha$ tenemos que:
            \begin{equation*}
                (p\circ \alpha)(0) = p(x) = p(y) = (p\circ \alpha)(1)
            \end{equation*}
            por lo que $p\circ \alpha$ es un lazo en $X$. Por otra parte, el Teorema de Monodromía nos dice que el homomorfismo $p_\ast:\pi_1(X,x) \to \pi_1(X,p(x))$ es inyectivo, por lo que será un isomorfismo de grupos, al ser $\pi_1(X,x)$ finito. De esta forma, como $[p\circ \alpha] \in \pi_1(X,p(x))$, ha de existir $\beta\in \Omega(X,x)$ de forma que $p_\ast([\beta]) = [p\circ \alpha]$. En este momento tenemos que tanto $\beta$ como $\alpha$ son dos levantamientos de $p\circ \alpha$ con:
            \begin{equation*}
                \beta(0) = x = \alpha(0)
            \end{equation*}
            por lo que han de ser iguales, de donde:
            \begin{equation*}
                y = \alpha(1) = \beta(1) = x
            \end{equation*}
    \end{enumerate}
\end{ejercicio}

    \newpage
\section{Estimación puntual. Insesgadez y mínima varianza}

\begin{ejercicio}
    Sea $(X_1, \ldots, X_n)$ una muestra de una variable $X\rightsquigarrow\cc{N}(\mu, \sigma^2)$ con $\mu\in \mathbb{R}$, $\sigma\in \mathbb{R}^+$. Probar que
    \begin{equation*}
        T(X_1, \ldots, X_n) = \left\{\begin{array}{ll}
            1 & \text{si\ } \overline{X}\leq 0 \\
            0 & \text{si\ } \overline{X} > 0
        \end{array}\right. 
    \end{equation*}
    es un estimador insesgado de la función paramétrica $\Phi\left(\frac{-\mu \sqrt{n}}{\sigma}\right)$, siendo $\Phi$ la función de distribución de la $\cc{N}(0,1)$.\\

    \noindent
    Tenemos $T(X_1, \ldots, X_n) = I_{\left]-\infty,0\right]}(\overline{X})$. Como $X\rightsquigarrow \cc{N}(\mu, \sigma^2)$, sabemos por lo visto en el Tema 1 que entonces:
    \begin{equation*}
        \overline{X} \rightsquigarrow \cc{N}\left(\mu, \frac{\sigma^2}{n}\right)
    \end{equation*}

    de donde (escribiendo $T=T(X_1,\ldots,X_n)$):
    \begin{equation*}
    T = I_{\left]-\infty,0\right]}(\overline{X}) \rightsquigarrow B(1,P[\overline{X}\leq 0])
    \end{equation*}

    estamos ya en condiciones de ver que $T$ es insesgado para dicha función:
    \begin{equation*}
        E[T] \AstIg P[\overline{X}\leq 0] \stackrel{\text{tipif.}}{=} P\left[Z \leq \dfrac{-\mu\sqrt{n}}{\sigma}\right] = \Phi\left(\dfrac{-\mu\sqrt{n}}{\sigma}\right) 
    \end{equation*}
    donde en $(\ast)$ usamos que conocemos bien la esperanza de una distribución Bernoulli.
\end{ejercicio}

\begin{ejercicio}
    Sea $(X_1, \ldots, X_n)$ una muestra aleatoria simple de $X\rightsquigarrow B(1,p)$ con $p\in \left]0,1\right[$ y sea $T=\sum\limits_{i=1}^{n}X_i$.
    \begin{enumerate}[label=\alph*)]
        \item Probar que si $k\in \mathbb{N}$ y $k\leq n$, el estadístico
            \begin{equation*}
                \dfrac{T(T-1)\cdot \ldots\cdot (T-k+1)}{n(n-1)\cdot \ldots\cdot (n-k+1)}
            \end{equation*}
            es un estimador insesgado de $p^k$. ¿Es este estimador el UMVUE?.
        \item Probar que si $k>n$, no existe ningún estimador insesgado para $p^k$.
        \item ¿Puede afirmarse que $\frac{T}{n}{\left(1-\frac{T}{n}\right)}^{2}$ es insesgado para $p{(1-p)}^{2}$? 
    \end{enumerate}
\end{ejercicio}

\begin{ejercicio}
    Sea $(X_1, \ldots, X_n)$ una muestra aleatoria simple de una variable $X\rightsquigarrow \cc{P}(\lm)$ con $\lm\in \mathbb{R}^+$. Encontrar, si existe, el UMVUE para $\lm^s$, siendo $s\in \mathbb{N}$ arbitrario.\\
    
    \noindent
    Veamos que $T(X_1, \ldots, X_n) = \sum\limits_{i=1}^{n}X_i$ es un estadístico suficiente y completo. Para ello, recordemos que $\{\cc{P}(\lm) : \lm>0\}$ es una familia exponencial:
    \begin{enumerate}
        \item El espacio paramétrico es $\mathbb{R}^+ \subseteq \mathbb{R}$.
        \item El espacio muestral es $\cc{X}=\mathbb{N}\cup \{0\}$, que no depende de $\lm$.
        \item Observamos que:
            \begin{align*}
                P_\lm[X=x] &= e^{-\lm}\dfrac{\lm^x}{x!} = exp\left[\ln\left(e^{-\lm}\dfrac{\lm^x}{x!}\right)\right] = exp\left(-\lm + x\ln\lm -\ln(x!)\right)
            \end{align*}
            por lo que basta tomar:
            \begin{equation*}
                Q(\lm) = \ln\lm, \qquad T(x) = x \qquad D(\lm) = -\lm, \qquad S(x) = -\ln(x!)
            \end{equation*}
    \end{enumerate}
    En consecuencia, por un Teorema visto en teoría, tenemos que el estadístico:
    \begin{equation*}
        T(X_1, \ldots, X_n) = \sum_{i=1}^{n}T(X_i) = \sum_{i=1}^{n}X_i
    \end{equation*}
    es suficiente y completo para $\lm$. Observemos que por la reproductividad de la Poisson tenemos que (notando $T=T(X_1, \ldots, X_n)$): 
    \begin{equation*}
        T\rightsquigarrow \cc{P}\left(\sum_{i=1}^{n}\lm\right) \equiv \cc{P}(n\lm)
    \end{equation*}
    Ahora, para buscar el UMVUE, buscamos una función $h$ medible de forma que:
    \begin{equation*}
        \lm^s = E[h(T)] = \sum_{t\in \mathbb{N}\cup \{0\}}h(t)P[T=t] = \sum_{t\in \mathbb{N}\cup \{0\}} h(t) e^{-n\lm} \dfrac{{(n\lm)}^{t}}{t!} 
    \end{equation*}

    por lo que:
    \begin{equation*}
        \lm^s e^{n\lm} = \sum_{t\in \mathbb{N}\cup \{0\}} h(t)\dfrac{{(n\lm)}^{t}}{t!}
    \end{equation*}

    y si aplicamos el desarrollo en serie de la exponencial, obtenemos:
    \begin{equation*}
        \lm^s \sum_{t\in \mathbb{N}\cup \{0\}} \dfrac{{(n\lm)}^{t}}{t!} = \lm^s e^{n\lm} = \sum_{t\in \mathbb{N}\cup\{0\}} h(t) \dfrac{{(n\lm)}^{t}}{t!} 
    \end{equation*}

    si desarrollamos cada uno de los términos:
    \begin{equation*}
        \lm^s + \lm^{s+1}n + \frac{\lm^{s+2}n^2}{2!} + \ldots = h(0) + h(1)(n\lm) + \ldots + h(s)\dfrac{{(n\lm)}^{s}}{s!} + \ldots
    \end{equation*}

    observamos que tomando:
    \begin{gather*}
        h(0) = \ldots = h(s-1) = 0,\quad  h(s) = \dfrac{s!}{n^s}\\h(s+1) = \dfrac{(s+1)!}{n^s}, \quad \ldots \quad  h(s+k) = \dfrac{(s+k)!}{n^s k!}
    \end{gather*}

    es decir:
    \begin{equation*}
        h(T) = \left\{\begin{array}{ll}
            0 & \text{si\ } T<s \\
            \dfrac{T!}{n^s(T-s)!}& \text{si\ } T\geq s
        \end{array}\right. 
    \end{equation*}
    tenemos que $h(T)$ es insesgado para $\lm^s$. Es claro además que $h(t)\in \mathbb{R}^+$ para cualquier valor de $t$, con lo que $h(T)$ es un estimador de $\lm^s$. Finalmente, observemos que:
    \begin{align*}
        E\left[{(h(T))}^{2}\right] &= \sum_{t\in \mathbb{N}\cup \{0\}} {(h(t))}^{2}P[T=t] = \sum_{t\geq s} {\left(\dfrac{t!}{n^s(t-s)!}\right)}^{2} e^{-n\lm} \dfrac{{(n\lm)}^{t}}{t!} \\ &= \dfrac{1}{n^s e^{n\lm}} \sum_{t\geq s} \dfrac{{(n\lm)}^{t}t!}{{((t-s)!)}^{2}}
    \end{align*}

    como:
    \begin{equation*}
        \dfrac{\dfrac{{(n\lm)}^{t+1}(t+1)!}{{((t+1-s)!)}^{2}}}{\dfrac{{(n\lm)}^{t}t!}{{((t-s)!)}^{2}}} = \dfrac{{(n\lm)}^{t+1}(t+1)!{((t-s)!)}^{2}}{{(n\lm)}^{t}t!{((t+1-s)!)}^{2}} = \dfrac{n\lm(t+1)}{{(t+1-s)}^{2}} \to 0 < 1
    \end{equation*}
    por el Criterio del cociente, tenemos que:
    \begin{equation*}
        \sum_{t\geq s} \dfrac{{(n\lm)}^{t}t!}{{((t-s)!)}^{2}} < \infty \Longrightarrow E\left[{(h(T))}^{2}\right]  =\dfrac{1}{n^s e^{n\lm}}\sum_{t\geq s} \dfrac{{(n\lm)}^{t}t!}{{((t-s)!)}^{2}}   < \infty
    \end{equation*}
    en consecuencia, tenemos que $h(T)$ es un estimador insesgado para $\lm^s$ y de momento de segundo orden finito y es función de un estadístico suficiente y completo, con lo que el Teorema de Lehmann-Scheffé nos dice que:
    \begin{equation*}
        E[h(T)/T] = h(T)
    \end{equation*}
    es un UMVUE para $\lm^s$.
\end{ejercicio}

\begin{ejercicio}
    Sea $(X_1, \ldots, X_n)$ una muestra aleatoria simple de una variable con distribución uniforme discreta en los puntos $\{1,\ldots,N\}$, siendo $N$ un número natural arbitrario. Encontrar el UMVUE para $N$.\\

    \noindent
    En el Ejercicio~\ref{ej:3.5} vimos que $T(X_1, \ldots, X_n) = X_{(n)}$ era un estadístico suficiente y completo. Si notamos $T = T(X_1,\ldots, X_n)$, tenemos que:
    \begin{equation*}
        F_T(t) = {(F_X(t))}^{n} \Longrightarrow P[T=t] = P[T\leq t] - P[T\leq t-1] = {(F_X(t))}^{n}-{(F_X(t-1))}^{n}
    \end{equation*}
    como $F_X(t) = \frac{t}{N}$, tenemos:
    \begin{equation*}
        P[T=t] = {(F_X(t))}^{n}-{(F_X(t-1))}^{n} = \dfrac{t^n - {(t-1)}^{n}}{N^n}
    \end{equation*} % // TODO: TERMINAR
\end{ejercicio}

\begin{ejercicio}
    Sea $(X_1, \ldots, X_n)$ una muestra aleatoria simple de una variable aleatoria $X$ cuya función de densidad es de la forma
    \begin{equation*}
        f_\theta(x) = \dfrac{1}{2\sqrt{x\theta}}, \quad 0<x<\theta
    \end{equation*}
    Calcular, si existe, el UMVUE para $\theta$.
\end{ejercicio}

\begin{ejercicio}
    Sea $(X_1, \ldots, X_n)$ una muestra aleatoria simple de una variable aleatoria $X$ con función de densidad
    \begin{equation*}
        f_\theta(x) = \dfrac{\theta}{x^2}, \quad x>\theta
    \end{equation*}
    Calcular, si existen, los UMVUE para $\theta$ y para $\nicefrac{1}{\theta}$.
\end{ejercicio}

\begin{ejercicio}
    Sea $X\rightsquigarrow P_\theta$ siendo $P_\theta$ una distribución con función de densidad
    \begin{equation*}
        f_\theta(x) = e^{\theta-x}, \quad x\geq \theta
    \end{equation*}
    Dada una muestra aleatoria simple de tamaño arbitrario, encontrar los UMVUE de $\theta$ y de $e^{\theta}$.
\end{ejercicio}

\begin{ejercicio}
    Sea $X$ la variable que describe el número de fracasos antes del primer éxito en una sucesión de pruebas de Bernoulli con probabilidad de éxito $\theta\in \left]0,1\right[$, y sea $(X_1, \ldots, X_n)$ una muestra aleatoria simple de $X$.
    \begin{enumerate}[label=\alph*)]
        \item Probar que la familia de distribuciones de $X$ es regular y calcular la función de infor- mación asociada a la muestra.
        \item Especificar la clase de funciones paramétricas que admiten estimadores eficientes y los correspondientes estimadores.
        \item Calcular la varianza de cada estimador eficiente y comprobar que coincide con las correspondiente cota de Fréchet-Cramér-Rao.
        \item Calcular, si existen, los UMVUE para $P_\theta[X=0]$ y para $E_\theta[X]$ y decir si son eficientes.
    \end{enumerate}
\end{ejercicio}

\begin{ejercicio}
    Sea $(X_1, \ldots, X_n)$ una muestra aleatoria simple de una variable aleatoria $X$ con distribución exponencial.
    \begin{enumerate}[label=\alph*)]
        \item Probar que la familia de distribuciones de $X$ es regular.
        \item Encontrar la clase de funciones paramétricas que admiten estimador eficiente y el estimador correspondiente. Calcular la varianza de estos estimadores.
        \item Basándose en el apartado anterior, encontrar el UMVUE para la media de $X$.
        \item Dar la cota de Fréchet-Cramér-Rao para la varianza de estimadores insesgados y regulares de $\lm^3$. ¿Es alcanzable dicha cota?
    \end{enumerate}
\end{ejercicio}

\begin{ejercicio}
    Sea $X$ una variable aleatoria con función de densidad de la forma
    \begin{equation*}
        f_\theta(x) = \theta x^{\theta-1}, \quad 0<x<1
    \end{equation*}
    \begin{enumerate}[label=\alph*)]
        \item Sabiendo que $E_\theta[\ln X] = -\frac{1}{\theta}$ y $Var_\theta[\ln X] = \frac{1}{\theta^2}$, comprobar que esta familia de distribuciones es regular.
        \item Basándose en una muestra aleatoria simple de $X$, dar la clase de funciones paramétricas con estimador eficiente, los estimadores y su varianza.
    \end{enumerate}
\end{ejercicio}

    \section{Funciones Elementales}

\begin{ejercicio}
    Sea $f : \mathbb{C} \to \mathbb{C}$ una función verificando que
    \[
        f(z + w) = f(z)f(w) \quad \forall z,w \in \mathbb{C}
    \]
    Probar que, si $f$ es derivable en algún punto del plano, entonces $f$ es entera. Encontrar todas las funciones enteras que verifiquen la condición anterior. Dar un ejemplo de una función que verifique dicha condición y no sea entera.
\end{ejercicio}

\begin{ejercicio}
    Calcular la imagen por la función exponencial de una banda horizontal o vertical y del dominio cuya frontera es un rectángulo de lados paralelos a los ejes.
\end{ejercicio}

\begin{ejercicio}
    Dado $\theta\in \left] -\pi, \pi \right]$, estudiar la existencia del límite en $+\infty$ de la función siguiente:
    \Func{\varphi}{\mathbb{R}^+}{\mathbb{C}}{r}{e^{re^{i\theta}}}
\end{ejercicio}

\begin{ejercicio}
    Probar que si $\{z_n\}$ y $\{w_n\}$ son sucesiones de números complejos, con $z_n \neq 0$ para todo $n \in \mathbb{N}$ y $\{z_n\} \to 1$, entonces
    \[
        \left\{w_n(z_n - 1)\right\} \to \lm \in \mathbb{C} \implies \left\{{z_n}^{w_n}\right\} \to e^{\lm}
    \]
\end{ejercicio}

\begin{ejercicio}
    Estudiar la convergencia puntual, absoluta y uniforme de la serie de funciones
    \[
        \sum_{n\geq 0} e^{-nz^2}
    \]
\end{ejercicio}

\begin{ejercicio}
    Probar que si $a,b,c \in \mathbb{T}$ son vértices de un triángulo equilátero si, y sólo si, $a+b+c = 0$.
\end{ejercicio}

\begin{ejercicio}
    Sea $\Omega$ un subconjunto abierto no vacío de $\mathbb{C}^*$ y $\varphi \in \mathcal{C}(\Omega)$ tal que $\varphi(z)^2 = z$ para todo $z \in \Omega$. Probar que $\varphi \in \mathcal{H}(\Omega)$ y calcular su derivada.
\end{ejercicio}

\begin{ejercicio}
    Probar que, para todo $z \in D(0,1)$ se tiene:
    \begin{enumerate}
        \item $\sum\limits_{n= 1}^\infty \dfrac{(-1)^{n+1}}{n}z^n = \log(1+z)$
        \item $\sum\limits_{n= 1}^\infty \dfrac{z^{2n+1}}{n(2n+1)} = 2z - (1+z)\log(1+z) + (1-z)\log(1-z)$
    \end{enumerate}
\end{ejercicio}

\begin{ejercicio}
    Sea la siguiente función:
    \Func{f}{\mathbb{C}\setminus\{1,-1\}}{\mathbb{C}}{z}{\log\left(\frac{1+z}{1-z}\right)}
    Probar que $f$ es holomorfa en el dominio $W = \mathbb{C} \setminus \{x \in \mathbb{R} : |x| \geq 1\}$ y calcular su derivada. Probar también que
    \[
        f(z) = 2\sum_{n=0}^\infty \frac{z^{2n+1}}{2n+1} \quad \forall z \in D(0,1)
    \]
\end{ejercicio}

\begin{ejercicio}
    Sean $\alpha,\beta \in \left[ -\pi, \pi \right]$ con $\alpha < \beta$, y $\rho \in \mathbb{R}^+$ tal que $\rho\alpha,\rho\beta \in \left[ -\pi, \pi \right]$. Consideramos los siguientes dominios:
    \begin{align*}
        \Omega &= \{z \in \mathbb{C}^* : \alpha < \arg z < \beta\} \\
        \Omega_\rho &= \{z \in \mathbb{C}^* : \rho\alpha < \arg z < \rho\beta\}
    \end{align*}
    Probar que la siguiente función define una biyección de $\Omega$ sobre el dominio $\Omega_\rho$:
    \Func{f}{\Omega}{\Omega_\rho}{z}{z^\rho}
\end{ejercicio}

\begin{ejercicio}
    Probar que el seno, el coseno y la tangente son funciones simplemente periódicas.
\end{ejercicio}

\begin{ejercicio}
    Estudiar la convergencia de la serie
    \[
        \sum_{n\geq 0} \frac{\sen(nz)}{2^n}
    \]
\end{ejercicio}

\begin{ejercicio}
    Sea $\Omega = \mathbb{C} \setminus \{x \in \mathbb{R} : |x| \geq 1\}$. Probar que existe $f \in \mathcal{H}(\Omega)$ tal que $\cos f(z) = z$ para todo $z \in \Omega$ y $f(x) = \arccos x$ para todo $x \in \left] -1, 1 \right[$. Calcular la derivada de $f$.
\end{ejercicio}

\begin{ejercicio}
    Para $z \in D(0,1)$ con $\Re z \neq 0$, probar que
    \[
        \arctan\left(\frac{1}{z}\right) + \sum_{n=0}^\infty \frac{(-1)^n}{2n+1}z^{2n+1} = \begin{cases}
            \nicefrac{\pi}{2} & \text{si } \Re z > 0 \\
            \nicefrac{-\pi}{2} & \text{si } \Re z < 0
        \end{cases}
    \]
\end{ejercicio}
    \section{$G-$conjuntos y $p$-grupos}

\begin{ejercicio}\label{ej:6.1}
    Si $X$ es un $G-$conjunto, demostrar que $x^g = \prescript{g^{-1}}{}{x},~ x \in X, g \in G$, define una acción por la derecha de $G$ sobre $X$.\\

    En primer lugar, vemos que se trata de una aplicación de $G \times X$ en $X$. Veamos ahora que cumple las condiciones necesarias para ser una acción por la derecha:
    \begin{itemize}
        \item $x^1 = x$ para todo $x \in X$.
        \begin{equation*}
            x^1 = \prescript{1^{-1}}{}{x} = \prescript{1}{}{x} = x
        \end{equation*}

        \item $(x^g)^h = x^{gh}$ para todo $x \in X$ y $g, h \in G$.
        \begin{equation*}
            (x^g)^h = \prescript{h^{-1}}{}{(x^g)} = \prescript{h^{-1}}{}{(\prescript{g^{-1}}{}{x})} = \prescript{h^{-1}g^{-1}}{}{x} =  \prescript{(gh)^{-1}}{}{x} = x^{gh}
        \end{equation*}
    \end{itemize}

    Por tanto, se trata de una acción por la derecha de $G$ sobre $X$.
\end{ejercicio}

\begin{ejercicio}\label{ej:6.2}
    Sea $G$ un grupo y $N$ un subgrupo normal abeliano de $G$. Demostrar que $G/N$ actúa sobre $N$ por conjugación y obtener entonces un homomorfismo $\varphi: G/N \to \Aut(N)$.\\

    Veamos en primer lugar que $G/N$ actúa sobre $N$ por conjugación. Es decir, que la siguiente aplicación es una acción de $G/N$ sobre $N$:
    \Func{ac}{G/N \times N}{N}{(gN, n)}{\prescript{gN}{}{n} = gng^{-1}}

    Veamos en primer lugar que está bien definida. Sean $g_1, g_2 \in G$ de forma que $g_1N = g_2N$. Entonces $\exists n'\in N$ tal que $g_1 = g_2n'$. Entonces:
    \begin{align*}
        \prescript{g_1N}{}{n} &= g_1ng_1^{-1} = g_2n' n (g_2n')^{-1} = g_2n' n (n')^{-1}g_2^{-1} \AstIg g_2 n'(n')^{-1} n g_2^{-1} = g_2 n g_2^{-1}
        = \prescript{g_2N}{}{n}
    \end{align*}
    donde en $(\ast)$ hemos usado que $N$ es abeliano. Por tanto, la acción está bien definida. Veamos ahora que se trata de una acción.
    \begin{itemize}
        \item $\prescript{1N}{}{n} = 1n1^{-1} = n$ para todo $n \in N$.
        \item Comprobemos la segunda propiedad:
        \begin{align*}
            \prescript{(g_1N)(g_2N)}{}{n} &= \prescript{g_1g_2N}{}{n} = g_1g_2ng_2^{-1}g_1^{-1} = g_1 \left(\prescript{g_2N}{}{n}\right) g_1^{-1}
            = \prescript{g_1N}{}{\left(\prescript{g_2N}{}{n}\right)}.
        \end{align*}
    \end{itemize}

    Buscamos ahora el homomorfismo $\varphi: G/N \to \Aut(N)$. En primer lugar, consideramos el siguiente homomorfismo:
    \Func{\Phi}{G/N}{\Perm(N)}{gN}{\prescript{gN}{}{(\cdot)}=ac(gN, \cdot)}


    Es necesario ver que, fijado $gN \in G/N$, la aplicación siguiente, además de pertenecer a $\Perm(N)$, pertenece a $\Aut(N)$:
    \Func{f}{N}{N}{n}{\prescript{gN}{}{n} = gng^{-1}}

    Sabemos que es biyectiva, por lo que tan solo nos queda probar que es un homomorfismo. Sean $n_1, n_2 \in N$:
    \begin{align*}
        f(n_1n_2) &= \prescript{gN}{}{(n_1n_2)} = g(n_1n_2)g^{-1} = g n_1 g^{-1} g n_2 g^{-1} = f(n_1)f(n_2).
    \end{align*}

    Por tanto, $f$ es un homomorfismo. La aplicación $\varphi$ pedida entonces es:
    \Func{\varphi}{G/N}{\Aut(N)}{gN}{f = \prescript{gN}{}{(\cdot)}}
\end{ejercicio}

\begin{ejercicio}\label{ej:6.3}
    Sean $S$ y $T$ dos $G-$conjuntos. Se define la \emph{acción diagonal} de $G$ sobre el producto cartesiano $S \times T$ mediante $\prescript{x}{}{(s,t)} = (\prescript{x}{}{s},\prescript{x}{}{t})$. Demostrar que, para la acción diagonal, el estabilizador de $(s, t)$ es la intersección de los estabilizadores de $s$ y $t$ en las acciones dadas.\\

    Fijados $s \in S$ y $t \in T$, el estabilizador de $(s,t)$ es:
    \begin{align*}
        \Stab_{G}(s,t) &= \{g \in G \mid \prescript{g}{}{(s,t)} = (s,t)\} = \{g \in G \mid (\prescript{g}{}{s},\prescript{g}{}{t}) = (s,t)\}\\
        &= \{g \in G \mid \prescript{g}{}{s} = s \land \prescript{g}{}{t} = t\} = \{g \in G \mid \prescript{g}{}{s} = s\} \cap \{g \in G \mid \prescript{g}{}{t} = t\}\\
        &= \Stab_{G}(s) \cap \Stab_{G}(t).
    \end{align*}
\end{ejercicio}

\begin{ejercicio}\label{ej:6.4}
    Demostrar que si $G$ contiene un elemento $x$ que tiene exactamente dos conjugados, entonces $G$ tiene un subgrupo normal propio.
    \begin{observacion}
        Considerar el centralizador de $x$.
    \end{observacion}

    Consideramos la acción por conjugación de $G$ sobre sí mismo:
    \Func{ac}{G \times G}{G}{(g,h)}{\prescript{g}{}{h} = ghg^{-1}}

    Calculamos el centralizador de $x$:
    \begin{align*}
        C_G(\{x\}) &= \{g \in G \mid gx = xg\} = \{g \in G \mid gxg^{-1} = x\} = \{g \in G \mid \prescript{g}{}{x} = x\} = \Stab_G(x)
    \end{align*}

    Por tanto, $C_G(\{x\}) = \Stab_G(x)<G$. Veamos ahora que es normal en $G$. Calculemos la órbita de $x$:
    \begin{align*}
        \Orb(x) &= \{y\in G \mid \exists g \in G \text{ tal que } y = \prescript{g}{}{x}\} = \{y\in G \mid \exists g \in G \text{ tal que } y = gxg^{-1}\}
        = \Cl_G(x)
    \end{align*}

    Como $x$ tiene exactamente dos conjugados (él mismo y otro elemento $y\in G$), tenemos que $|\Orb(x)| = 2$. Por tanto:
    \begin{align*}
        [G:C_G(\{x\})] &= |\Orb(x)| = 2 \implies C_G(\{x\})\lhd G
    \end{align*}
    \begin{observacion}
        Notemos que, aun sin saber si $G$ es finito, la igualdad anterior tiene perfecto sentido, puesto que $|\Orb(x)|=2$ y $[G:C_G(\{x\})]$ indica el número de clases en el conjunto cociente, que sabemos que es biyectivo con $\Orb(x)$, luego es $2$.
    \end{observacion}

    Por tanto, $C_G(\{x\})$ es un subgrupo normal de $G$. Tan solo falta por comprobar que es propio.
    \begin{itemize}
        \item Si $C_G(\{x\}) = G$, entonces:
        \begin{equation*}
            2 = |\Orb(x)| = [G:\Stab_G(x)] = [G:C_G(\{x\})] = 1 \implies \text{Contradicción.}
        \end{equation*}

        \item Si $C_G(\{x\}) = \{1\}$, entonces:
        \begin{equation*}
            2 = |\Orb(x)| = [G:\Stab_G(x)] = [G:C_G(\{x\})] = [G:\{1\}] = |G|
        \end{equation*}
        Por tanto, $G=\{1,x\}$. Calculemos el número de conjugados de $1$ y de $x$:
        \begin{align*}
            \Cl_G(1) &= \{g1g^{-1} \mid g \in G\} = \{1\} \\
            \Cl_G(x) &= \{gxg^{-1} \mid g \in G\} = \{1x1, xxx^{-1}\} = \{x\}
        \end{align*}
        Por tanto, ambos tienen un único conjugado. Por tanto, no se puede dar este caso.
    \end{itemize}
    Por tanto, $C_G(\{x\})$ es un subgrupo normal propio de $G$.
\end{ejercicio}

\begin{ejercicio}\label{ej:6.5}
    Encontrar todos los grupos finitos que tienen exactamente dos clases de conjugación.\\

    Sea $G$ un grupo finito con $|G| = n$ que tiene exactamente dos clases de conjugación; a saber, $\exists x_1, x_2 \in G$ tales que $\Cl_G(x_1) \neq \Cl_G(x_2)$. Considerando la acción de $G$ sobre sí mismo por conjugación, tenemos que:
    \begin{equation*}
        \Orb(x) = \Cl_G(x) \qquad \forall x \in G
    \end{equation*}

    Como las órbitas forman una partición de $G$, tenemos que:
    \begin{equation*}
        |G| = |\Orb(x_1)| + |\Orb(x_2)| = |\Cl_G(x_1)| + |\Cl_G(x_2)|
    \end{equation*}

    Calculamos no obstante la clase de conjugación del $1\in G$:
    \begin{align*}
        \Cl_G(1) &= \{g1g^{-1} \mid g \in G\} = \{g g^{-1} \mid g \in G\} = \{1\}
    \end{align*}
    Por tanto, $|\Cl_G(1)| = 1$. Supongamos sin pérdida de generalidad que $1\in \Cl_G(x_1)$. Entonces:
    \begin{equation*}
        n = |\Cl_G(x_1)| + |\Cl_G(x_2)| = 1 + |\Cl_G(x_2)|
        \Longrightarrow |\Cl_G(x_2)| = n - 1
    \end{equation*}

    Por otro lado, como $|\Cl_G(x_2)| = [G:\Stab_G(x_2)]$, tenemos que $|\Cl_G(x_2)|$ divide a $|G|$; es decir, $(n-1) \mid n$. Por tanto, $n=2$, y tenemos por tanto que:
    \begin{equation*}
        G\cong \bb{Z}_2
    \end{equation*}
\end{ejercicio}

\begin{ejercicio}\label{ej:6.6}
    Describir explícitamente las clases de conjugación del grupo $D_4$.\\

    Consideramos el grupo $D_4$:
    \begin{align*}
        D_4 &= \{1, r, r^2, r^3, s, sr, sr^2, sr^3\} \\
        &= \{s^ir^j \mid i = 0, 1, j = 0, 1, 2, 3\}
    \end{align*}

    Tenemos que:
    \begin{align*}
        \Cl_{D_4}(1) &= \{(s^i r^j)1(s^i r^j)^{-1} \mid i = 0, 1, j = 0, 1, 2, 3\} = \{1\} \\
        \Cl_{D_4}(r) &= \{(s^i r^j)r(s^i r^j)^{-1} \mid i = 0, 1, j = 0, 1, 2, 3\} = \{s^ir^j\ r\ r^{-j}s^{-i} \mid i = 0, 1, j = 0, 1, 2, 3\} \\
        &= \{s^i r s^i \mid i = 0, 1\} = \{r, r^3\} = \Cl_{D_4}(r^3) \\
        \Cl_{D_4}(r^2) &= \{(s^i r^j)r^2(s^i r^j)^{-1} \mid i = 0, 1, j = 0, 1, 2, 3\} = \{s^i r^j\ r^2\ r^{-j}s^{-i} \mid i = 0, 1, j = 0, 1, 2, 3\} \\
        &= \{s^i r^2 s^{-i} \mid i = 0, 1\} = \{r^2\}\\
        \Cl_{D_4}(s) &= \{(s^i r^j)s(s^i r^j)^{-1} \mid i = 0, 1, j = 0, 1, 2, 3\}
        = \{s^ir^j\ s\ r^{-j}s^{-i} \mid i = 0, 1, j = 0, 1, 2, 3\} 
    \end{align*}

    Este último no es tan sencillo, puesto que $r$ y $s$ no conmutan. Calculamos en primer lugar para $s=0$, sabiendo que las clases de conjugación son cerradas para inversos.
    \begin{align*}
        r\ s\ r^{-1} &= r\ s\ r^3 = sr^6 = sr^2 \in \Cl_{D_4}(s) \\
        r^2\ s\ r^{-2} &= r^2\ s\ r^2 = sr^6r^2=s\in \Cl_{D_4}(s) \\
        r^3\ s\ r^{-3} &= r^3\ s\ r = sr^9r = sr^2 \in \Cl_{D_4}(s)
    \end{align*}

    Por otro lado, para $s=1$, tenemos que:
    \begin{equation*}
        s\ s\ s = s\qquad \text{ y } s\ sr^2\ s = r^2s = sr^6 = sr^2
    \end{equation*}

    Por tanto, $\Cl_{D_4}(s) = \{s, sr^2\} = \Cl_{D_4}(sr^2)$. Tan solo queda por tanto calcular la clase de conjugación de $sr$ y de $sr^3$.
    \begin{equation*}
        r\ sr\ r^{-1} = r\ sr\ r^3 = rs = sr^3 \in \Cl_{D_4}(sr)
    \end{equation*}

    Por tanto, tenemos que $\Cl_{D_4}(sr) = \Cl_{D_4}(sr^3)$. Como las clases de conjugación forman una partición de $D_4$, tenemos que:
    \begin{align*}
        \Cl_{D_4}(1) &= \{1\} \\
        \Cl_{D_4}(r) &= \{r, r^3\} \\
        \Cl_{D_4}(r^2) &= \{r^2\} \\
        \Cl_{D_4}(s) &= \{s, sr^2\} \\
        \Cl_{D_4}(sr) &= \{sr, sr^3\}
    \end{align*}
\end{ejercicio}

\begin{ejercicio}\label{ej:6.7}
    Se dice que la acción de un grupo finito $G$ sobre un conjunto $X$ es \emph{transitiva} si hay una sola órbita para esta acción (es decir, si para cada $x, y \in X$ existe algún $g \in G$ tal que $\prescript{g}{}{x} = y$). Demostrar que si $G$ actúa transitivamente sobre un conjunto $X$ con $n$ elementos, entonces $|G|$ es un múltiplo de $n$.\\

    Sea $x\in X$. Como las órbitas forman una partición de $X$ y hay una única órbita, tenemos que:
    \begin{equation*}
        n = |X| = |\Orb(x)|
    \end{equation*}

    Como además tenemos que $|\Orb(x)|=[G:\Stab_G(x)]$, tenemos que:
    \begin{equation*}
        |G| = n\cdot |\Stab_G(x)|
    \end{equation*}
    Por tanto, $|G|$ es un múltiplo de $n$.
\end{ejercicio}

\begin{ejercicio}\label{ej:6.8}
    Un subgrupo $G \leq S_n$ se dice \emph{transitivo} si la acción de $G$ sobre $\{1, 2, \ldots, n\}$ es transitiva. Encontrar todos los subgrupos transitivos de $S_3$ y $S_4$.
    \begin{enumerate}
        \item $S_3$.
        
        Consideramos la acción natural de $S_3$ sobre $\{1, 2, 3\}$ dada por:
        \Func{ac}{S_3 \times \{1, 2, 3\}}{\{1, 2, 3\}}{(\sigma, i)}{\prescript{\sigma}{}{i} = \sigma(i)}

        Consideramos ahora la restricción de la acción a $G\leq S_3$, que sigue siendo una acción.
        Buscamos ahora los subgrupos transitivos $G\leq S_3$. En primer lugar, por el Ejercicio anterior, sabemos que $|G|$ es un múltiplo de $3$. Además, como $|S_3| = 6$, tenemos que $|G|$ divide a $6$. Por tanto, $|G|\in \{3,6\}$. Es decir, $G\in \{A_3, S_3\}$. Comprobemos si estos son transitivos.

        Dados $x,y\in \{1, 2, 3\}$ distintos, consideramos el tercer elemento $z\in \{1, 2, 3\}$. Sea ahora $\sigma=(x\ y\ z)\in S_3\cap A_3$. Entonces:
        \begin{align*}
            \prescript{\sigma}{}{x} &= y
        \end{align*}

        Entonces, $S_3$ y $A_3$ son transitivos. Por tanto, los únicos subgrupos transitivos de $S_3$ son $S_3$ y $A_3$.

        \item $S_4$.
        
        Consideramos la acción natural de $S_4$ sobre $\{1, 2, 3, 4\}$ dada por:
        \Func{ac}{S_4 \times \{1, 2, 3, 4\}}{\{1, 2, 3, 4\}}{(\sigma, i)}{\prescript{\sigma}{}{i} = \sigma(i)}

        Consideramos ahora la restricción de la acción a $G\leq S_4$, que sigue siendo una acción.
        Buscamos ahora los subgrupos transitivos $G\leq S_4$. En primer lugar, por el Ejercicio anterior, sabemos que $|G|$ es un múltiplo de $4$. Además, como $|S_4| = 24$, tenemos que $|G|$ divide a $24$. Por tanto, $|G|\in \{4,8,12,24\}$.
        \begin{itemize}
            \item Si $|G|=24$, entonces $G=S_4$. Dados por tanto $x,y\in \{1, 2, 3, 4\}$ distintos, consideramos un tercer elemento $z\in \{1, 2, 3, 4\}\setminus \{x,y\}$. Entonces, tomando $\sigma=(x\ y\ z)\in S_4$:
            \begin{align*}
                \prescript{\sigma}{}{x} &= y
            \end{align*}
            Entonces, $S_4$ es transitivo.

            \item Si $|G|=12$, entonces $G=A_4$. Empleando el mismo razonamiento que en el caso anterior, tenemos que $\sigma\in A_4$ y, por tanto, $\prescript{\sigma}{}{x} = y$. Entonces, $A_4$ es transitivo.
            
            \item Si $|G|=8$, entonces es un $2-$subgrupo de Sylow de $S_4$. Calculemos cuántos $2-$subgrupos de Sylow de $S_4$ hay. Como $|S_4|=24=2^3\cdot 3$, notando por $n_2$ al número de $2-$subgrupos de Sylow de $S_4$, por el Segundo Teorema de Sylow tenemos que:
            \begin{equation*}
                n_2 \equiv 1 \mod 2 \qquad\land \qquad n_2 \mid 3
            \end{equation*}

            Por tanto, puede ser $n_2=1$ o $n_2=3$. Puesto que $S_4$ no contiene subgrupos de orden $8$ normales, tenemos que $n_2=3$. Por tanto, hay tres subgrupos de orden $8$ en $S_4$. Probando, llegamos a que estos son:
            \begin{itemize}
                \item $\langle (1\ 2\ 3\ 4), (1\ 3)\rangle$.
                
                Sea $a=(1\ 2\ 3\ 4)$ y $b=(1\ 3)$. Entonces, tenemos que:
                \begin{align*}
                    ab &= (1\ 2\ 3\ 4)(1\ 3) = (1\ 4)(2\ 3)\\
                    ba^3 &= (1\ 3)(1\ 2\ 3\ 4)^3
                    = (1\ 3)(1\ 4\ 3\ 2) = (1\ 4)(2\ 3)
                \end{align*}
                Por tanto, $ab=ba^3$. Por el Teorema de Dyck, este grupo es isomorfo a $D_4$, luego es de orden $8$. Veamos si es transitivo. Para ello, vemos que:
                \begin{align*}
                    a^0(1) &= 1 \qquad
                    a^1(1) = 2 \qquad
                    a^2(1) = 3 \qquad
                    a^3(1) = 4
                \end{align*}

                Por tanto, $\Orb(1)=\{1, 2, 3, 4\}$. Por tanto, como $\Orb(x)$ es una partición de $\{1, 2, 3, 4\}$, tenemos que la única órbita es $\{1, 2, 3, 4\}$. Por tanto, es transitivo.

                \item $\langle (1\ 3\ 2\ 4), (1\ 2)\rangle$.
                
                Sea $a=(1\ 2\ 3\ 4)$ y $b=(1\ 2)$. Entonces, tenemos que:
                \begin{align*}
                    ab &= (1\ 3\ 2\ 4)(1\ 2) = (1\ 4)(2\ 3)\\
                    ba^3 &= (1\ 2)(1\ 3\ 2\ 4)^3
                    = (1\ 2)(1\ 4\ 2\ 3) = (1\ 4)(2\ 3)
                \end{align*}
                Por tanto, $ab=ba^3$. Por el Teorema de Dyck, este grupo es isomorfo a $D_4$, luego es de orden $8$. Veamos si es transitivo. Para ello, vemos que:
                \begin{align*}
                    a^0(1) &= 1 \qquad
                    a^1(1) = 3 \qquad
                    a^2(1) = 2 \qquad
                    a^3(1) = 4
                \end{align*}

                Por tanto, $\Orb(1)=\{1, 2, 3, 4\}$. Por tanto, como $\Orb(x)$ es una partición de $\{1, 2, 3, 4\}$, tenemos que la única órbita es $\{1, 2, 3, 4\}$. Por tanto, es transitivo.

                \item $\langle (1\ 2\ 3\ 4), (2\ 4)\rangle$.
                
                Sea $a=(1\ 2\ 3\ 4)$ y $b=(2\ 4)$. Entonces, tenemos que:
                \begin{align*}
                    ab &= (1\ 2\ 3\ 4)(2\ 4) = (1\ 2)(3\ 4)\\
                    ba^3 &= (2\ 4)(1\ 2\ 3\ 4)^3
                    = (2\ 4)(1\ 4\ 3\ 2) = (1\ 2)(3\ 4)
                \end{align*}

                Por tanto, $ab=ba^3$. Por el Teorema de Dyck, este grupo es isomorfo a $D_4$, luego es de orden $8$. Veamos si es transitivo. Para ello, vemos que:
                \begin{align*}
                    a^0(1) &= 1 \qquad
                    a^1(1) = 2 \qquad
                    a^2(1) = 3 \qquad
                    a^3(1) = 4
                \end{align*}

                Por tanto, $\Orb(1)=\{1, 2, 3, 4\}$. Por tanto, como $\Orb(x)$ es una partición de $\{1, 2, 3, 4\}$, tenemos que la única órbita es $\{1, 2, 3, 4\}$. Por tanto, es transitivo.
            \end{itemize}

            \item Si $|G|=4$, entonces es cíclico o isomorfo a $V$.
            \begin{itemize}
                \item Sea $G\cong \bb{Z}_4$, y sea $a\in G$ un generador. Entonces, por ser $a$ una biyección en $\{1, 2, 3, 4\}$, tenemos que:
                \begin{equation*}
                    \{a^0(1), a^1(1), a^2(1), a^3(1)\} = \{1,2,3,4\}
                \end{equation*}
                Por tanto, $\Orb(1)=\{1, 2, 3, 4\}$. Por tanto, como $\Orb(x)$ es una partición de $\{1, 2, 3, 4\}$, tenemos que la única órbita es $\{1, 2, 3, 4\}$. Por tanto, es transitivo.

                \item Sea $G\cong V$. Entonces, está formado por la identidad y $3$ elementos de orden $2$ de forma que el producto de dos de ellos es el tercero. Los elementos de orden $2$ son transposiciones o productos de transposiciones. Como es de orden $4$, ha de generarse con dos elementos.
                \begin{itemize}
                    \item Si $G$ está generado por dos productos de transposiciones disjuntas, entonces $G=V$. Veamos sí es transitivo. Sean $i,j\in \{1, 2, 3, 4\}$ distintos. Entonces, sean $k,l\in \{1, 2, 3, 4\}\setminus \{i,j\}$ distintos, de forma que $(i\ j)(k\ l)\in G$. Entonces:
                    \begin{align*}
                        \prescript{(i\ j)(k\ l)}{}{i} &= j
                    \end{align*}
                    Por tanto, como $j$ era arbitrario, tenemos que:
                    \begin{equation*}
                        \Orb(i) = \{1, 2, 3, 4\}
                    \end{equation*}
                    Por tanto, como $\Orb(x)$ es una partición de $\{1, 2, 3, 4\}$, tenemos que la única órbita es $\{1, 2, 3, 4\}$. Por tanto, es transitivo.

                    \item Si $G$ está generado por dos transposiciones no disjuntas, entonces $\exists i,j,k\in \{1, 2, 3, 4\}$ distintos tales que:
                    \begin{equation*}
                        G = \langle (i\ j), (i\ k) \rangle
                    \end{equation*}
                    Entonces:
                    \begin{equation*}
                        (i\ j)(i\ k) = (i\ k\ j)
                    \end{equation*}
                    Por tanto, contendría un elemento de orden $3$, que no puede ser, puesto que $G$ es de orden $4$. Por tanto, no se puede dar este caso.

                    \item Si $G$ está generado por dos transposiciones disjuntas, entonces $\exists i,j,k,l\in \{1, 2, 3, 4\}$ distintos tales que:
                    \begin{equation*}
                        G = \langle (i\ j), (k\ l) \rangle
                    \end{equation*}

                    Entonces, tenemos que:
                    \begin{equation*}
                        G = \langle (i\ j), (k\ l) \rangle = \{1, (i\ j), (k\ l), (i\ j)(k\ l)\}
                    \end{equation*}

                    En este caso, $\Orb(i) = \{1,i,j\}\neq \{1, 2, 3, 4\}$. Por tanto, no es transitivo.

                    \item Si $G$ está generado por una transposición y un producto de transposiciones disjuntas, caben dos casos:
                    \begin{itemize}
                        \item $G=\langle (i\ j), (i\ j)(k\ l) \rangle$, donde $i,j,k,l\in \{1, 2, 3, 4\}$ son distintos. Entonces:
                        \begin{equation*}
                            (i\ j)(i\ j)(k\ l) = (k\ l)
                        \end{equation*}
                        Por tanto, $G=\langle (i\ j), (k\ l) \rangle$, que es un caso ya visto.
                        \item $G=\langle (i\ j), (i\ k)(j\ l) \rangle$, donde $i,j,k,l\in \{1, 2, 3, 4\}$ son distintos. Entonces:
                        \begin{equation*}
                            (i\ j)(i\ k)(j\ l) = (i\ k\ j\ l)
                        \end{equation*}
                        No obstante, este elemento es de orden $4$, por lo que no puede ser, puesto que $G$ sería cíclico. Por tanto, no se puede dar este caso.
                    \end{itemize}
                \end{itemize}
            \end{itemize}
        \end{itemize}
        Como hemos visto, los únicos subgrupos transitivos de $S_4$ son:
        \begin{itemize}
            \item $S_4$.
            \item $A_4$.
            \item Los tres subgrupos de orden $8$ isomorfos a $D_4$:
            \begin{itemize}
                \item $\langle (1\ 2\ 3\ 4), (1\ 3)\rangle$.
                \item $\langle (1\ 3\ 2\ 4), (1\ 2)\rangle$.
                \item $\langle (1\ 2\ 3\ 4), (2\ 4)\rangle$.
            \end{itemize}
            \item Los grupos cíclicos de orden $4$ y $V$.
        \end{itemize}
    \end{enumerate}
\end{ejercicio}

\begin{ejercicio}\label{ej:6.9}
    Sea $n\in \bb{N}$. Una \emph{partición} de $n$ es una sucesión no decreciente de enteros positivos cuya suma es $n$. Dada una permutación $\sigma \in S_n$, la descomposición en ciclos disjuntos (incluyendo los ciclos de longitud 1) de $\sigma = \gamma_1 \gamma_2 \cdots \gamma_r$ determina una partición $n_1, n_2, \ldots, n_r$ de $n$ donde cada $n_i$ es la longitud del ciclo $\gamma_i$. Dos permutaciones en $S_n$ se dice que son del mismo tipo si determinan la misma partición de $n$. Demostrar:
    \begin{enumerate}
        \item Dos elementos de $S_n$ son conjugados si y solo si son del mismo tipo.
        \begin{description}
            \item[$\Longrightarrow$)] Sean $\sigma,\tau\in S_n$ dos elementos conjugados; es decir, $\exists \gamma\in S_n$ tal que $\gamma\sigma\gamma^{-1}=\tau$. Consideramos ahora la descomposición en ciclos disjuntos (incluyendo los de longitud $1$) de $\sigma$:
            \begin{equation*}
                \sigma = \sigma_1\cdots\sigma_r
            \end{equation*}
            Por tanto, tenemos que:
            \begin{equation*}
                \tau=\gamma\sigma\gamma^{-1} = (\gamma\sigma_1\gamma^{-1})\cdots(\gamma\sigma_r\gamma^{-1})
            \end{equation*}

            Además, sabemos que la longitud de $\sigma_i$ coincide con la de $\gamma\sigma_i\gamma^{-1}$ para todo $i\in \{1,\ldots,r\}$. Por tanto, $\tau$ y $\gamma$ determinan la misma partición y por tanto son del mismo tipo.

            \item[$\Longleftarrow$)] Sean $\sigma,\tau\in S_n$ dos elementos del mismo tipo; y sea $n_1,\dots,n_r$ la partición de $n$ que determinan. Consideramos por tanto ambas particiones en ciclos disjuntos (incluyendo los ciclos de longitud uno):
            \begin{align*}
                \sigma &= \sigma_1\cdots\sigma_r
                = (a_{11} \ a_{12} \cdots a_{1n_1})(a_{21} \ a_{22} \cdots a_{2n_2})\cdots(a_{r1} \ a_{r2} \cdots a_{rn_r})\\
                \tau &= \tau_1\cdots\tau_r
                = (b_{11} \ b_{12} \cdots b_{1n_1})(b_{21} \ b_{22} \cdots b_{2n_2})\cdots(b_{r1} \ b_{r2} \cdots b_{rn_r})
            \end{align*}

            Consideramos ahora $\gamma\in S_n$ cuya representación matricial es:
            \begin{equation*}
                \gamma = \begin{pmatrix}
                    a_{11} & a_{12} & \cdots & a_{1n_1} & \cdots & a_{r1} & a_{r2} & \cdots & a_{rn_r}\\
                    b_{11} & b_{12} & \cdots & b_{1n_1} & \cdots & b_{r1} & b_{r2} & \cdots & b_{rn_r}
                \end{pmatrix}
            \end{equation*}

            Entonces, tenemos que:
            \begin{align*}
                \gamma\sigma\gamma^{-1} &= \gamma(\sigma_1\cdots\sigma_r)\gamma^{-1} = \gamma\sigma_1\cdots\gamma\sigma_r\gamma^{-1}\\
                &= (\gamma\sigma_1\gamma^{-1})(\gamma\sigma_2\gamma^{-1})\cdots(\gamma\sigma_r\gamma^{-1}) = \tau_1\tau_2\cdots\tau_r = \tau
            \end{align*}
            Por tanto, $\sigma$ y $\tau$ son conjugados.
        \end{description}

        \item El número de clases de conjugación de $S_n$ es igual al número de particiones de $n$.
        
        \begin{description}
            \item[$\leq$)] Sean $\sigma,\tau\in S_n$ dos elementos tal que $\Cl_{S_n}(\sigma)\neq \Cl_{S_n}(\tau)$. Entonces, no son conjugados. Por tanto, no son del mismo tipo y determinan distintas particiones de $n$. Por tanto, el número de clases de conjugación de $S_n$ es menor o igual al número de particiones de $n$.
            
            \item[$\geq$)] Veamos en primer lugar que, dada una partición de $n$, existe al menos un elemento de $S_n$ que determina dicha partición. Sea $n_1, n_2, \ldots, n_r$ la partición de $n$. Consideramos el siguiente elemento de $S_n$:
            \begin{equation*}
                \sigma = (1\ 2\ \cdots\ n_1)(n_1+1\ n_1+2\ \cdots\ n_1+n_2)\cdots(n_1+\cdots+n_{r-1}+1\ n_1+\cdots+n_{r-1}+2\ \cdots\ n)
            \end{equation*}
            Entonces, $\sigma$ es un elemento de $S_n$ que determina la partición $n_1, n_2, \ldots, n_r$. Por tanto, existe al menos un elemento de $S_n$ que determina cada partición de $n$.

            Ahora, dados dos elementos $\sigma,\tau\in S_n$ que determinan particiones distintas de $n$, tenemos que $\sigma$ y $\tau$ no son del mismo tipo. Por tanto, no son conjugados. Por tanto, $\Cl_{S_n}(\sigma)\neq \Cl_{S_n}(\tau)$. Por tanto, el número de clases de conjugación de $S_n$ es mayor o igual al número de particiones de $n$.
        \end{description}

        Como conclusión, tenemos que el número de clases de conjugación de $S_n$ es igual al número de particiones de $n$.
    \end{enumerate}
\end{ejercicio}

\begin{ejercicio}\label{ej:6.10}
    Calcular el número de clases de conjugación de $S_5$. Dar un representante de cada una y encontrar el orden de cada clase. Calcular el estabilizador de $(1\ 2\ 3)$ bajo la acción de conjugación de $S_5$ sobre sí mismo.\\


    Por el ejercicio anterior, sabemos que hay tantas clases de conjugación como particiones de $n$:
    \begin{equation*}
        \begin{array}{l|c|c}
            \text{Partición} & \text{Representante} & \text{Orden}\\ \hline
            1\ 1\ 1\ 1\ 1 & id_{5} & 1\\
            1\ 1\ 1\ 2 & (1\ 2) & 10\\
            1\ 2\ 2 & (1\ 2)(3\ 4) & 15\\
            1\ 1\ 3 & (1\ 2\ 3) & 20\\
            2\ 3 & (1\ 2)(3\ 4\ 5) & 20\\
            1\ 4 & (1\ 2\ 3\ 4) & 30\\
            5 & (1\ 2\ 3\ 4\ 5) & 24
        \end{array}
    \end{equation*}
    
    Para calcular el orden de cada clase, usamos que:
    \begin{equation*}
        |\Cl_{S_5}(\sigma)| = \dfrac{5!}{\prod\limits_{i=1}^5 m_i! \cdot i^{m_i}}
    \end{equation*}
    donde $m_i$ es el número de ciclos de longitud $i$ en la descomposición en ciclos disjuntos de $\sigma$. Por tanto, tenemos que:
    \begin{align*}
        |\Cl_{S_5}(id_5)| &= \dfrac{5!}{5! \cdot 1^5} = 1\\
        |\Cl_{S_5}((1\ 2))| &= \dfrac{5!}{3!\cdot 1^3 \cdot 1!\cdot 2^1} = \dfrac{120}{6\cdot 2} = 10\\
        |\Cl_{S_5}((1\ 2)(3\ 4))| &= \dfrac{5!}{2!\cdot 2^2} = \dfrac{120}{2\cdot 4} = 15\\
        |\Cl_{S_5}((1\ 2\ 3))| &= \dfrac{5!}{2!\cdot 1^2 \cdot 3^1} = \dfrac{120}{2\cdot 3} = 20\\
        |\Cl_{S_5}((1\ 2)(3\ 4\ 5))| &= \dfrac{5!}{1!\cdot 1^1 \cdot 3^1 \cdot 2^1} = \dfrac{120}{1\cdot 3\cdot 2} = 20\\
        |\Cl_{S_5}((1\ 2\ 3\ 4))| &= \dfrac{5!}{1!\cdot 1^1 \cdot 4^1} = \dfrac{120}{1\cdot 4} = 30\\
        |\Cl_{S_5}((1\ 2\ 3\ 4\ 5))| &= \dfrac{5!}{1!\cdot 5^1} = \dfrac{120}{1\cdot 5} = 24
    \end{align*}


    Calculamos ahora el estabilizador de $(1\ 2\ 3)$:
    \begin{align*}
        \Stab_{S_5}((1\ 2\ 3)) &= \{\gamma\in S_5\mid \gamma(1\ 2\ 3)\gamma^{-1} = (1\ 2\ 3)\}
        =\\&= \{\gamma\in S_5\mid (\gamma(1)\ \gamma(2)\ \gamma(3)) = (1\ 2\ 3) = (2\ 3\ 1) = (3\ 1\ 2)\}
    \end{align*}

    A simple vista, vemos que:
    \begin{equation*}
        id_5,\ (4\ 5),\ (1\ 2\ 3),\ (1\ 3\ 2),\ (1\ 2\ 3)(4\ 5),\ (1\ 3\ 2)(4\ 5)\in \Stab_{S_5}((1\ 2\ 3))
    \end{equation*}

    No obstante, podría haber más. Comprobemos que no:
    \begin{equation*}
        |\Stab_{S_5}((1\ 2\ 3))| = \dfrac{|S_5|}{|\Orb((1\ 2\ 3))|}= \dfrac{|S_5|}{|\Cl_{S_5}((1\ 2\ 3))|}
        = \dfrac{120}{20} = 6
    \end{equation*}

    Por tanto:
    \begin{equation*}
        \Stab_{S_5}((1\ 2\ 3))=\{id_5,\ (4\ 5),\ (1\ 2\ 3),\ (1\ 3\ 2),\ (1\ 2\ 3)(4\ 5),\ (1\ 3\ 2)(4\ 5)\}
    \end{equation*}
\end{ejercicio}

\begin{ejercicio}
    Sea $G$ un grupo finito y $\Phi: G \to \Perm(G)$ la representación regular izquierda (que corresponde a la acción de $G$ sobre sí mismo por traslación por la izquierda).
    \begin{enumerate}
        \item Demostrar que si $x$ es un elemento de $G$ de orden $n$ y $|G| = nm$, entonces $\Phi(x)$ es un producto de $n-$ciclos. Deducir que $\Phi(x)$ es una permutación impar si y solo si el orden de $x$ es par y el cociente del orden de $G$ y el de $x$ es impar.
        
        Sea $x\in G$ con $\ord(x)=n$. Entonces, $\Phi(x)\in \Perm(G)$, y como $|G|=nm$, tenemos que $\Phi(x)\in S_{nm}$. Sea ahora $k\in G$. Con vistas de estudiar la descomposición de $\Phi(x)$ en ciclos disjuntos, veamos las imágenes sucesivas de $k$ bajo $\Phi(x)$:
        \begin{align*}
            k &\mapsto xk \mapsto x^2k \mapsto \cdots \mapsto x^{n-1}k \mapsto x^n k = k
        \end{align*}
        Por tanto, $k$ pertenece al ciclo $(k\ xk\ x^2k\ \cdots\ x^{n-1}k)$ de $\Phi(x)$. Como $k$ era arbitrario, hemos visto que $\Phi(x)$ es producto de $n-$ciclos. Además, todos estos son trivialmente disjuntos.
        \begin{comment}
        hemos visto que $\Phi(x)$ tiene un ciclo de longitud $n$ que contiene a $k$. Por tanto, para cada $k\in G$, $\Phi(x)$ tiene un ciclo de longitud $n$ que contiene a $k$.
        \begin{itemize}
            \item Supongamos que $\Phi(x)$ tiene un ciclo de longitud $j<n$. Entonces, fijado $g\in G$ perteneciente a dicho ciclo, como el orden de un ciclo es su longitud, tenemos que:
            \begin{equation*}
                g = \Phi^j(g) = x^j g\Longrightarrow x^j = 1
                \Longrightarrow n\mid j
                \Longrightarrow n\leq j
            \end{equation*}
            Por tanto, hemos llegado a una contradicción, luego $\Phi(x)$ no tiene ciclos de longitud $j<n$.
            \item Supongamos que $\Phi(x)$ tiene un ciclo de longitud $j>n$. Entonces:
            \begin{equation*}
                g \neq \Phi^n(g) = x^n g\Longrightarrow x^n \neq 1
            \end{equation*}
            Por tanto, hemos llegado a una contradicción, luego $\Phi(x)$ no tiene ciclos de longitud $j>n$.
        \end{itemize}

        Por tanto, la descomposición de $\Phi(x)$ en ciclos disjuntos está formada por $n-$ciclos. Veamos ahora por cuántos $n-$ciclos está formada. Veamos en primer lugar que dos $n-$ciclos distintos han de ser disjuntos. Supongamos que $\Phi(x)$ tiene dos $n-$ciclos $C_1$ y $C_2$ tales que $\exists k\in C_1\cap C_2$. Entonces, por cómo actúa $\Phi(x)$, tenemos que:
        \begin{align*}
            k &\mapsto xk \mapsto x^2k \mapsto \cdots \mapsto x^{n-1}k \mapsto x^n k = k
        \end{align*}
        Por tanto, el único ciclo que contiene a $k$ es $(k\ xk\ x^2k\ \cdots\ x^{n-1}k)$, luego $C_1=C_2$.\\

        Por tanto, $\Phi(x)$ está formada por ciclos de longitud $n$ disjuntos. 
        \end{comment}
        
        Como en la representación de $\Phi(x)$ deben aparecer los $nm$ elementos de $G$, tenemos que el número de ciclos de longitud $n$ en la descomposición de $\Phi(x)$ es:
        \begin{equation*}
            \dfrac{|G|}{n} = \dfrac{nm}{n} = m
        \end{equation*}

        Por tanto, $\Phi(x)$ está formada por $m$ ciclos de longitud $n$ disjuntos. Como una permutación es par si y solo si el número de ciclos de longitud par es par, tenemos que:
        \begin{equation*}
            \veps(\Phi(x)) = -1 \iff \text{el número de ciclos de longitud par es impar}
        \end{equation*}

        Como el $0$ es par, al menos uno de los ciclos de longitud $n$ ha de ser par, luego $n$ ha de ser par. Además, como el número de ciclos de longitud $n$ es $m$, tenemos que $m$ ha de ser impar. Por tanto:
        \begin{equation*}
            \veps(\Phi(x)) = -1 \iff n\text{ es par y }m\text{ es impar}
        \end{equation*}
        Como $m=\dfrac{|G|}{n}$, se tiene lo pedido.

        \item Demostrar que si $Im(\Phi)$ contiene una permutación impar entonces $G$ tiene un subgrupo de índice 2.
        
        Supongamos que $Im(\Phi)$ contiene una permutación impar. Entonces, existe $x\in G$ tal que $\veps(\Phi(x))=-1$. Como tanto $\Phi$ como $\veps$ son homomorfismos, consideramos el homomorfismo composición:
        \begin{equation*}
            \veps\circ\Phi: G \to \{-1, 1\}
        \end{equation*}

        Veamos si ese homomorfismo es sobreyectivo. Como $\veps(\Phi(x))=-1$, tenemos que:
        \begin{equation*}
            (\veps\circ\Phi)(x) = -1\Longrightarrow -1\in Im(\veps\circ\Phi)
        \end{equation*}

        Por otro lado, considerando $1\in G$, tenemos que:
        \begin{equation*}
            (\veps\circ\Phi)(1) = \veps(\Phi(1)) = \veps(id_{G}) = 1
        \end{equation*}

        Por tanto, $Im(\veps\circ\Phi)=\{-1, 1\}$. Por el Primero Teorema de Isomorfía, tenemos que:
        \begin{equation*}
            \dfrac{G}{\ker(\veps\circ\Phi)}\cong Im(\veps\circ\Phi) = \{-1, 1\}
        \end{equation*}

        Como $G$ es finito, tenemos que:
        \begin{equation*}
            [G:\ker(\veps\circ\Phi)] = |Im(\veps\circ\Phi)| = 2
        \end{equation*}

        Por tanto, $\ker(\veps\circ\Phi)$ es un subgrupo de $G$ de índice $2$.       

        \item Demostrar que si $|G| = 2k$ con $k$ impar, entonces $G$ tiene un subgrupo de índice 2.
        \begin{observacion}
            Usar el Teorema de Cauchy para obtener un elemento de orden 2 y entonces usar los dos apartados anteriores.
        \end{observacion}

        Por el Teorema de Cauchy, como $2\mid |G|$, existe $x\in G$ tal que $\ord(x)=2$. Además, como $|G|=2k$ con $k$ impar; y $\ord(x)=2$ par, por el primer apartado, tenemos que $\Phi(x)$ es una permutación impar. Por el segundo apartado, tenemos que $G$ tiene un subgrupo de índice $2$.
    \end{enumerate}
\end{ejercicio}

\begin{ejercicio}\label{ej:6.12}
    Sea $G$ un $p-$grupo actuando sobre un conjunto finito $X$. Demostrar que
    \[
        |X| \equiv |\Fix(X)| \mod p.
    \]

    Sea $G$ un $p-$grupo finito y sea $X$ un conjunto finito sobre el que actúa. Como $G$ es finito, $\exists n\in \bb{N}$ tal que $|G|=p^n$.
    En vistas de aplicar la fórmula de clases, definimos $\Gamma$ como un conjunto que tiene un representante de cada órbita no unitaria de la acción de $G$ sobre $X$. Entonces, tenemos que:
    \begin{equation*}
        |X| = |\Fix(X)| + \sum_{x\in \Gamma} |\Orb(x)|
    \end{equation*}

    Demostrar lo pedido equivale a demostrar que:
    \begin{equation*}
        X-|\Fix(X)| = \sum_{x\in \Gamma} |\Orb(x)| \equiv 0 \mod p
    \end{equation*}

    Fijado $x\in \Gamma$, tenemos que $|\Orb(x)|>1$. Asímismo, como $|\Orb(x)|=[G:\Stab_G(x)]$, tenemos que $|\Orb(x)|\mid |G|=p^n$. Uniendo ambas afirmaciones, tenemos que $|\Orb(x)|=p^{k_x}$ con $k_x\in \{1,\ldots,n\}$.\\

    Por tanto, $|\Orb(x)|\equiv 0 \mod p$ para todo $x\in \Gamma$. Por tanto, la suma de los términos de la suma es congruente a $0$ módulo $p$. Por tanto:
    \begin{equation*}
        |X| - |\Fix(X)| = \sum_{x\in \Gamma} |\Orb(x)| \equiv 0 \mod p
    \end{equation*}
    como queríamos demostrar.
\end{ejercicio}

\begin{ejercicio}
    Sea $G$ un $2-$grupo finito que actúa sobre un conjunto finito $X$ cuya cardinalidad es un número impar. ¿Podemos afirmar que existe al menos un punto de $X$ que queda fijo bajo la acción de $G$? ¿Podemos decir lo mismo si $|X|$ es par?

    Por la fórmula de clases, definiendo $\Gamma$ como un conjunto que tiene un representante de cada órbita no unitaria de la acción de $G$ sobre $X$, tenemos que:
    \begin{equation*}
        |X| = |\Fix(X)| + \sum_{x\in \Gamma} |\Orb(x)|
    \end{equation*}

    Dado $x\in \Gamma$, tenemos que $|\Orb(x)|=[G:\Stab_G(x)]$. Como $G$ es un $2-$grupo, tenemos que $|G|=2^k$ con $k\in \bb{N}$. Por tanto, $|\Orb(x)|$ es una potencia de $2$. Además, como $x\in \Gamma$, tenemos que $|\Orb(x)|>1$. Por tanto, $|\Orb(x)|$ es un número par para todo $x\in \Gamma$. Por tanto, la suma de los términos de la suma es par. Como $|X|$ es impar, tenemos que $|\Fix(X)|$ es impar. Por tanto, existe al menos un punto de $X$ que queda fijo bajo la acción de $G$.\\

    Si $|X|$ es par, entonces $|\Fix(X)|$ es par, pero podría ser $0$. Por tanto, no podemos afirmar que existe al menos un punto de $X$ que queda fijo bajo la acción de $G$.
\end{ejercicio}

\begin{ejercicio}\label{ej:6.14}
    Sea $C_n = \langle a \mid a^n = 1 \rangle$ un grupo cíclico de orden $n$. Describir sus subgrupos de Sylow.\\


    Sea $n=p_1^{k_1}p_2^{k_2}\cdots p_m^{k_m}$ la factorización de $n$ en primos. Entonces, para cada $i\in \{1,\ldots,m\}$, tenemos que el $p_i-$subgrupo de Sylow de $C_n$ es único, puesto que $C_n$ es cíclico luego abeliano, y por tanto sus subgrupos son normales. Por tanto, el $p_i-$subgrupo de Sylow de $C_n$ es el único subgrupo de orden $p_i^{k_i}$ de $C_n$.
    \begin{equation*}
        P_{p_i} = \left\langle a^{\left(\frac{n}{(p_i^{k_i})}\right)} \right\rangle
    \end{equation*}
\end{ejercicio}

\begin{ejercicio}\label{ej:6.15}
    Sea $G$ un grupo finito y $|G| = pn$ con $p$ primo y $p > n$. Demostrar que $G$ contiene un subgrupo normal de orden $p$ y que todo subgrupo de $G$ de orden $p$ es normal en $G$.\\

    Buscamos obtener $n_p$. Como $p>n$, sabemos que $\mcd(p,n)=1$. Por tanto, por el Teorema de Sylow, tenemos que:
    \begin{align*}
        n_p &\equiv 1 \mod p \\
        n_p &\mid n
    \end{align*}

    Por tanto, $n_p\leq n<p$, luego $n_p=1$. Por tanto, existe un único $p-$subgrupo de Sylow de $G$ (de orden $p$), que es normal. Llamémoslo $P_p$, y tendrá orden $|P_p|=p$.\\

    Sea ahora $H$ un subgrupo de $G$ de orden $p$. Entonces, este es un $p-$subgrupo de Sylow de $G$, luego $H=P_p$.
\end{ejercicio}

\begin{ejercicio}\label{ej:6.16}
    Sea $H$ un subgrupo de un grupo finito $G$ con $[G : H] = p$ primo y $p$ el menor primo que divide a $|G|$. Demostrar que entonces $H$ es normal en $G$.\\

    Como no sabemos si $H$ es normal en $G$, no podemos considerar el grupo cociente pero sí el conjunto de las clases laterales por la izquierda de $H$ en $G$, que denotamos por $G/\sim_H$. Consideramos la acción de $G$ sobre $G/\sim_H$ por traslación por la izquierda, y consideramos su representación por permutaciones
    \begin{equation*}
        \Phi: G \to \Perm(G/\sim_H)
    \end{equation*}

    Como $|G/\sim_H| = [G:H] = p$, tenemos que $\Phi$ es un homomorfismo de grupos entre $G$ y $S_p$:
    \begin{equation*}
        \Phi: G \to S_p
    \end{equation*}

    En vistas de aplicar el Primer Teorema de Isomorfía, calculamos el núcleo de $\Phi$:
    \begin{align*}
        \ker(\Phi) &= \{g\in G\mid \Phi(g) = id_{S_p}\}\\
        &= \{g\in G\mid g\cdot (aH) = (aH)\text{ para todo }a\in G\} \subset\\
        &\subset \{g\in G\mid gH = H\} = \{g\in G\mid g\in H\} = H
    \end{align*}

    Por el Primer Teorema de Isomorfía, tenemos que:
    \begin{equation*}
        G/\ker(\Phi) \cong Im(\Phi) \leq S_p
    \end{equation*}

    Por tanto, como $|S_p| = p!$ con $p$ primo, tenemos que $|G/\ker(\Phi)|$ divide a $p!$.
    Por otro lado, $|G/\ker(\Phi)|$ divide a $|G|$. Como $p$ es el menor primo que divide a $|G|$, tenemos que $|G/\ker(\Phi)|=p$.
    Por último, como $\ker(\Phi)\lhd G$ y $\ker(\Phi)\leq H$, tenemos que $\ker(\Phi)\lhd H$. Por tanto:
    \begin{equation*}
        p = [G:\ker(\Phi)] = [G:H]\cdot [H:\ker(\Phi)] = p\cdot [H:\ker(\Phi)]
        \Longrightarrow [H:\ker(\Phi)]=1
    \end{equation*}

    Por tanto, $|\ker(\Phi)|=|H|$. Como $\ker(\Phi)\leq H$, tenemos que $\ker(\Phi)=H$. Por tanto, $\ker(\Phi)=H$ es un subgrupo normal de $G$, concluyendo así la demostración.
\end{ejercicio}

\begin{ejercicio}\label{ej:6.17}
    Sea $p$ un número primo. Demostrar:
    \begin{enumerate}
        \item Todo grupo no abeliano de orden $p^3$ tiene un centro de orden $p$.
        
        Sea $G$ un grupo no abeliano de orden $p^3$. Entonces, $G$ es un $p-$grupo. Por ser $Z(G)<G$, tenemos que $|Z(G)|=p^k$ con $k\in \{0,1,2,3\}$.
        \begin{itemize}
            \item Por ser un $p-$grupo, $Z(G)$ es no trivial, luego $k>0$.
            \item Por no ser abeliano, $Z(G)\neq G$, luego $k<3$.
            \item Por ser un $p-$grupo, $|Z(G)|\neq p^{3-1}$, luego $k<2$.
        \end{itemize}
        Por tanto, $k=1$. Por tanto, $|Z(G)|=p$.
        \item Existen únicamente dos grupos no isomorfos de orden $p^2$.
        
        Todo grupo de orden $p^2$ es abeliano. Consideramos el grupo cíclico $C_{p^2}$ y el grupo directo $C_p\times C_p$. Estos son de orden $p^2$ y no son isomorfos entre sí, puesto que uno es cíclico y el otro no ($\mcd(p,p)=p\neq 1$).

        % // TODO: Hay más?
        \item Todo subgrupo normal de orden $p$ de un $p-$grupo finito está contenido en el centro.
        
        Sea $G$ un $p-$grupo finito y sea $H\lhd G$ un subgrupo normal de orden $p$. Como es de orden $p$, $\exists h\in H$ con $\ord(h)=p$, de forma que:
        \begin{equation*}
            H = \langle h \rangle
        \end{equation*}

        Por tanto, para ver que $H\leq Z(G)$, bastará con ver que $h\in Z(G)$. Fijado $g\in G$, buscamos ver que $gh=hg$. Como $H\lhd G$, tenemos que:
        \begin{equation*}
            ghg^{-1} \in H
            \Longrightarrow
            \exists k\in \{0,\ldots,p-1\}\text{ tal que }ghg^{-1} = h^k
        \end{equation*}

        Si demostramos que $k=1$, lo tendremos.
        \begin{itemize}
            \item Supongamos $k=0$. Entonces, $ghg^{-1}=1$, luego $gh=g$ y $h=1$, luego $\ord(h)=1\neq p$, lo cual es una contradicción.
            \item Supongamos $k>1$. Como $g\in G$ y $G$ es un $p-$grupo, $\exists m\in \bb{N}$ tal que $\ord(g)=p^m$. Entonces, tenemos que:
            % // TODO: Hacer
        \end{itemize}

    \end{enumerate}
\end{ejercicio}

\begin{ejercicio}\label{ej:6.18}
    Demostrar que si $N\lhd G$ y $N$ y $G/N$ son $p-$grupos entonces $G$ es un $p-$grupo.\\

    Para demostrar que $G$ es un $p-$grupo, puesto que $G$ no tiene por qué ser finito, hemos de comprobar que el orden de todo elemento de $G\setminus \{1\}$ es una potencia de $p$. Sea $g\in G\setminus \{1\}$ un elemento cualquiera. Distinguimos dos casos:
    \begin{itemize}
        \item Si $g\in N$, entonces como $N$ es un $p-$grupo y $g\neq 1$, tenemos que $\ord(g)$ es una potencia de $p$.
        \item Si $g\notin N$, entonces como $G/N$ es un $p-$grupo y $gN\neq N$, tenemos que $\ord(gN)$ es una potencia de $p$. Por tanto, existe $k\in \bb{N}$ tal que:
        \begin{equation*}
            (gN)^{p^k} = N
            \Longrightarrow
            g^{p^k}N = N
            \Longrightarrow
            g^{p^k} \in N
        \end{equation*}
        Si $g^{p^k}=1$, entonces $\ord(g)$ divide a $p^k$, y como $g\neq 1$, tenemos que $\ord(g)$ es una potencia de $p$. Si $g^{p^k}\neq 1$, entonces como $N$ es un $p-$grupo, tenemos que $\ord(g^{p^k})$ es una potencia de $p$. Por tanto, $\exists k'\in \bb{N}$ tal que:
        \begin{equation*}
            \left(g^{p^k}\right)^{p^{k'}} = 1
            \Longrightarrow
            g^{p^{k+k'}} = 1
        \end{equation*}
        Por tanto, $\ord(g)$ divide a $p^{k+k'}$, luego $\ord(g)$ es una potencia de $p$.
    \end{itemize}
\end{ejercicio}

\begin{ejercicio}\label{ej:6.19}
    Si $G$ es un grupo de orden $p^n$, $p$ primo, demostrar que para todo $k$, $0 \leq k \leq n$, existe un subgrupo normal de $G$ de orden $p^k$.\\

    Demostramos por inducción sobre $n$.
    \begin{itemize}
        \item Para $n=1$, tenemos que $|G|=p$, luego $G$ es cíclico, luego abeliano, luego todo subgrupo de $G$ es normal.
        \item Supongamos que para todo $p-$grupo de orden $p^m$, con $m<n$, se cumple que para todo $k$, $0\leq k\leq m$, existe un subgrupo normal de orden $p^k$.\\
        
        Sea $G$ un $p-$grupo de orden $p^n$, y consideramos su centro $Z(G)$. Como $Z(G)\neq \{1\}$ y $Z(G)<G$, tenemos que $|Z(G)|=p^k$ con $k\in \{1,\ldots,n\}$. Como $p\mid |Z(G)|$, por el Teorema de Cauchy, existe $N<Z(G)$ tal que $|N|=p$. Como $Z(G)\lhd G$, se tiene que $N\lhd G$, lo que nos permite considerar:
        \begin{equation*}
            |G/N| = \dfrac{|G|}{|N|} = \dfrac{p^n}{p} = p^{n-1}
        \end{equation*}
        Por tanto, $G/N$ es un $p-$grupo de orden $p^{n-1}$. Por la hipótesis de inducción, tenemos que para todo $k$, $0\leq k\leq n-1$, existe $L_k\lhd G/N$ tal que $|L_k|=p^k$. Por el Tercer Teorema de Isomorfía, tenemos que, para cada $k\in \{0,\ldots,n-1\}$, existe $H_k<G$, con $N\lhd H_k\lhd G$, tal que $L_k=H_k/N$. Por tanto, tenemos que:
        \begin{equation*}
            |H_k| = |H_k/N| \cdot |N| = |L_k| \cdot |N| = p^k\cdot p = p^{k+1}
        \end{equation*}

        Por tanto, para cada $k'\in \{1,\ldots,n\}$, existe $W_{k'}=H_{k'-1}$ tal que $|W_{k'}|=p^{k'}$ y $W_{k'}\lhd G$. Falta ver el resultado para $k=0$, pero esto es directo tomando $W_0 = \{1\}$, que es un subgrupo normal de $G$ de orden $1=p^0$.
        Por tanto, hemos visto que para todo $k\in \{0,\ldots,n\}$, existe un subgrupo normal de $G$ de orden $p^k$.
    \end{itemize}
    Por tanto, hemos demostrado que para todo $p-$grupo $G$ de orden $p^n$, con $p$ primo, y para todo $k\in \{0,\ldots,n\}$, existe un subgrupo normal de $G$ de orden $p^k$.
\end{ejercicio}

\begin{ejercicio}\label{ej:6.20}
    Hallar todos los subgrupos de Sylow de los grupos $S_3$ y $S_4$.
    \begin{observacion}
        Para los $2-$subgrupos de Sylow de $S_4$, observar primero que todos deben contener al subgrupo de Klein $V$, y, al menos, una trasposición $\tau$, y que como consecuencia se pueden obtener como producto de $V$ por el grupo cíclico generado por $\tau$.
    \end{observacion}
    \begin{enumerate}
        \item $S_3$.
        
        Sabemos que $|S_3|=6=2\cdot 3$. Calculamos los $p-$subgrupos de Sylow, con $p\in \{2,3\}$.
        \begin{itemize}
            \item $2-$subgrupo de Sylow.
            
            Por el Segundo Teorema de Sylow, tenemos que:
            \begin{equation*}
                n_2 \equiv 1 \mod 2 \qquad n_2 \mid 3
            \end{equation*}
            Por tanto, $n_2\in \{1,3\}$.
        \end{itemize}
        Como hay más de un grupo de orden $2$ en $S_3$, tenemos que $n_2=3$. Estos grupos son:
        \begin{align*}
            \langle (1\ 2) \rangle, \quad \langle (1\ 3) \rangle, \quad \langle (2\ 3) \rangle
        \end{align*}
        \begin{itemize}
            \item $3-$subgrupo de Sylow.
            
            Por el Segundo Teorema de Sylow, tenemos que:
            \begin{equation*}
                n_3 \equiv 1 \mod 3 \qquad n_3 \mid 2
            \end{equation*}
            Por tanto, $n_3=1$. El único $3-$subgrupo de Sylow de $S_3$ es:
            \begin{equation*}
                P_3 = \langle (1\ 2\ 3) \rangle
            \end{equation*}
        \end{itemize}

        \item $S_4$.
        
        Sabemos que $|S_4|=24=2^3\cdot 3$. Calculamos los $p-$subgrupos de Sylow, con $p\in \{2,3\}$.
        \begin{itemize}
            \item $3-$subgrupo de Sylow.
            
            Por el Segundo Teorema de Sylow, tenemos que:
            \begin{equation*}
                n_3 \equiv 1 \mod 3 \qquad n_3 \mid 8
            \end{equation*}
            Por tanto, $n_3\in \{1,4\}$. Como hay más de un grupo de orden $3$ en $S_4$, tenemos que $n_3=4$. Estos grupos son:
            \begin{align*}
                \langle (1\ 2\ 3) \rangle, \quad \langle (1\ 2\ 4) \rangle, \quad \langle (1\ 3\ 4) \rangle, \quad \langle (2\ 3\ 4) \rangle
            \end{align*}

            \item $2-$subgrupo de Sylow.
            
            Por el Segundo Teorema de Sylow, tenemos que:
            \begin{equation*}
                n_2 \equiv 1 \mod 2 \qquad n_2 \mid 3
            \end{equation*}
            Por tanto, $n_2\in \{1,3\}$. En el Ejercicio~\ref{ej:6.8}, vimos que estos grupos son:
            \begin{itemize}
                \item $\langle (1\ 2\ 3\ 4), (1\ 3)\rangle$.
                \item $\langle (1\ 3\ 2\ 4), (1\ 2)\rangle$.
                \item $\langle (1\ 2\ 3\ 4), (2\ 4)\rangle$.
            \end{itemize}
            
        \end{itemize}
    \end{enumerate}
\end{ejercicio}

\begin{ejercicio}\label{ej:6.21}
    Hallar todos los subgrupos de Sylow de los grupos $\bb{Z}_{600}$, $Q_2$, $D_5$, $D_6$, $A_4$, $A_5$, $S_5$.
    \begin{enumerate}
        \item $\bb{Z}_{600}$.
        
        Tenemos $600=2^3\cdot 3\cdot 5^2$. Calculamos los $p-$subgrupos de Sylow, para valores de $p\in \{2,3,5\}$. Como $\bb{Z}_{600}$ es cíclico, en particular es abeliano, y por tanto sus subgrupos son normales, luego son únicos.
        \begin{itemize}
            \item $2-$subgrupo de Sylow.
            
            Es un grupo cíclico de orden $8$, luego es isomorfo a $\bb{Z}_8$. De hecho:
            \begin{equation*}
                P_2 = \langle 3\cdot 5^2 \rangle = \langle 75 \rangle \cong \bb{Z}_8
            \end{equation*}

            \item $3-$subgrupo de Sylow.
            
            Es un grupo cíclico de orden $3$, luego es isomorfo a $\bb{Z}_3$. De hecho:
            \begin{equation*}
                P_3 = \langle 2^3\cdot 5^2 \rangle = \langle 200 \rangle \cong \bb{Z}_3
            \end{equation*}
            \item $5-$subgrupo de Sylow.
            
            Es un grupo cíclico de orden $25$, luego es isomorfo a $\bb{Z}_{25}$. De hecho:
            \begin{equation*}
                P_5 = \langle 2^3\cdot 3 \rangle = \langle 24 \rangle \cong \bb{Z}_{25}
            \end{equation*}
        \end{itemize}

        \item $Q_2$.
        
        Sabemos que $|Q_2|=8=2^3$. Además, como $Q_2\lhd Q_2$, el único $2-$subgrupo de Sylow de $Q_2$ es $Q_2$ mismo. Por tanto, el único subgrupo de Sylow de $Q_2$ es $Q_2$.

        \item $D_5$.
        
        Sabemos que $|D_5|=10=2\cdot 5$. Calculamos los $p-$subgrupos de Sylow, con $p\in \{2,5\}$.
        \begin{itemize}
            \item $2-$subgrupos de Sylow.
            
            Por el Segundo Teorema de Sylow, tenemos que:
            \begin{equation*}
                n_2 \equiv 1 \mod 2 \qquad n_2 \mid 5
            \end{equation*}
            Por tanto, $n_2\in \{1,5\}$. Se tiene que $n_2=5$, puesto que hay $5$ elementos de orden $2$ en $D_5$. Estos grupos son:
            \begin{align*}
                \langle sr^i \rangle \qquad \forall i\in \{0,1,2,3,4\}
            \end{align*}

            \item $5-$subgrupo de Sylow.
            
            Sea $H$ un $5-$subgrupo de Sylow de $D_5$. Como $|D_5|=10$ y $|H|=5$, tenemos que $[D_5:H]=2$, luego $H$ es normal en $D_5$, luego es el único $5-$subgrupo de Sylow de $D_5$. Como además $5$ es primo, $H$ es cíclico. Por tanto:
            \begin{equation*}
                H = \langle r \rangle
            \end{equation*}
        \end{itemize}

        \item $D_6$.
        
        Sabemos que $|D_6|=12=2^2\cdot 3$. Calculamos los $p-$subgrupos de Sylow, con $p\in \{2,3\}$.
        \begin{itemize}
            \item $2-$subgrupos de Sylow.
            
            Por el Segundo Teorema de Sylow, tenemos que:
            \begin{equation*}
                n_2 \equiv 1 \mod 2 \qquad n_2 \mid 3
            \end{equation*}
            Por tanto, $n_2\in \{1,3\}$. Como no hay elementos de orden $4$ en $D_6$, no puede ser cíclico. Por tanto, ha de estar generado por más de un elemento de orden 2:
            \begin{align*}
                \langle r^3,s\rangle &= \{1,r^3,s,sr^3\}\\
                \langle r^3,sr\rangle &= \{1,r^3,sr,sr^4\}\\
                \langle r^3,sr^2\rangle &= \{1,r^3,sr,sr^5\}
            \end{align*}

            Como estos son tres $2-$subgrupos de Sylow de $D_6$, estos son los únicos.


            \item $3-$subgrupos de Sylow.
            
            Por el Segundo Teorema de Sylow, tenemos que:
            \begin{equation*}
                n_3 \equiv 1 \mod 3 \qquad n_3 \mid 4
            \end{equation*}
            Por tanto, $n_3\in \{1,4\}$. Como además los subgrupos son de orden $3$, son cíclicos, luego buscamos elementos de orden $3$ en $D_6$. Todos los elementos de la forma $sr^i$ con $i\in \{0,\dots,5\}$ tienen orden 2. Por tanto, el único $3-$subgrupo es:
            \begin{equation*}
                \langle r^2\rangle
            \end{equation*}
        \end{itemize}

        \item $A_4$.
        
        Sabemos que $|A_4|=12=2^2\cdot 3$. Calculamos los $p-$subgrupos de Sylow, con $p\in \{2,3\}$.
        \begin{itemize}
            \item $2-$subgrupos de Sylow.
            
            Como $V$ es un $2-$subgrupo de Sylow de $A_4$ y $V\lhd A_4$, tenemos que $V$ es el único $2-$subgrupo de Sylow de $A_4$.
            \item $3-$subgrupos de Sylow.
            
            Por el Segundo Teorema de Sylow, tenemos que:
            \begin{equation*}
                n_3 \equiv 1 \mod 3 \qquad n_3 \mid 4
            \end{equation*}
            Por tanto, $n_3\in \{1,4\}$. Como $A_4$ tiene $8$ elementos de orden $3$, tenemos que $n_3=4$. Por tanto, los $3-$subgrupos de Sylow son:
            \begin{align*}
                \langle (1\ 2\ 3) \rangle &= \{1,(1\ 2\ 3),(1\ 3\ 2)\}\\
                \langle (1\ 2\ 4) \rangle &= \{1,(1\ 2\ 4),(1\ 4\ 2)\}\\
                \langle (1\ 3\ 4) \rangle &= \{1,(1\ 3\ 4),(1\ 4\ 3)\}\\
                \langle (2\ 3\ 4) \rangle &= \{1,(2\ 3\ 4),(2\ 4\ 3)\}
            \end{align*}
        \end{itemize}
        \item $A_5$.
        
        Sabemos que $|A_5|=60=2^2\cdot 3\cdot 5$. Calculamos los $p-$subgrupos de Sylow, con $p\in \{2,3,5\}$.
        \begin{itemize}
            \item $2-$subgrupos de Sylow.
            
            Por el Segundo Teorema de Sylow, tenemos que:
            \begin{equation*}
                n_2 \equiv 1 \mod 2 \qquad n_2 \mid 15
            \end{equation*}
            Por tanto, $n_2\in \{1,3,5,15\}$.
            En $A_5$ no hay elementos de orden $4$, y los únicos de orden $2$ son lo productos de transposiciones. Veamos cuántas hay:
            \begin{equation*}
                |\Cl_{A_5}((1\ 2)(3\ 4))| = \frac{5!}{2^2\cdot 2!} = \frac{120}{8} = 15
            \end{equation*}
            
            Por tanto, como mínimo habrá estos $5$ $2-$subgrupos de Sylow:
            \begin{align*}
                \langle (1\ 2)(3\ 4),\ (1\ 3)(2\ 4) \rangle
                &= \{1,(1\ 2)(3\ 4),(1\ 3)(2\ 4),(1\ 4)(2\ 3)\}\\
                \langle (1\ 2)(3\ 5),\ (1\ 3)(2\ 5) \rangle
                &= \{1,(1\ 2)(3\ 5),(1\ 3)(2\ 5),(1\ 5)(2\ 3)\}\\
                \langle (1\ 2)(4\ 5),\ (1\ 4)(2\ 5) \rangle
                &= \{1,(1\ 2)(4\ 5),(1\ 4)(2\ 5),(1\ 5)(2\ 4)\}\\
                \langle (1\ 3)(4\ 5),\ (1\ 4)(3\ 5) \rangle
                &= \{1,(1\ 3)(4\ 5),(1\ 4)(3\ 5),(1\ 5)(2\ 3)\}\\
                \langle (2\ 3)(4\ 5),\ (2\ 4)(3\ 5) \rangle
                &= \{1,(2\ 3)(4\ 5),(2\ 4)(3\ 5),(2\ 5)(3\ 4)\}
            \end{align*}

            Se comprueba que, efectivamente, estos son $5$ $2-$subgrupos de Sylow de $A_5$. Por tanto, $n_2=5$.

            \item $3-$subgrupos de Sylow.
            Por el Segundo Teorema de Sylow, tenemos que:
            \begin{equation*}
                n_3 \equiv 1 \mod 3 \qquad n_3 \mid 20
            \end{equation*}
            Por tanto, $n_3\in \{1,4,10\}$. Veamos cuántos elementos de orden $3$ hay en $A_5$:
            \begin{equation*}
                |\Cl_{A_5}((1\ 2\ 3))| = \frac{5!}{3!\cdot 2} = 20
            \end{equation*}

            Por tanto, hay $10$ $3-$subgrupos de Sylow, que son:
            \begin{align*}
                \langle (1\ 2\ 3) \rangle &= \{1,(1\ 2\ 3),(1\ 3\ 2)\}\\
                \langle (1\ 2\ 4) \rangle &= \{1,(1\ 2\ 4),(1\ 4\ 2)\}\\
                \langle (1\ 2\ 5) \rangle &= \{1,(1\ 2\ 5),(1\ 5\ 2)\}\\
                \langle (1\ 3\ 4) \rangle &= \{1,(1\ 3\ 4),(1\ 4\ 3)\}\\
                \langle (1\ 3\ 5) \rangle &= \{1,(1\ 3\ 5),(1\ 5\ 3)\}\\
                \langle (1\ 4\ 5) \rangle &= \{1,(1\ 4\ 5),(1\ 5\ 4)\}\\
                \langle (2\ 3\ 4) \rangle &= \{1,(2\ 3\ 4),(2\ 4\ 3)\}\\
                \langle (2\ 3\ 5) \rangle &= \{1,(2\ 3\ 5),(2\ 5\ 3)\}\\
                \langle (2\ 4\ 5) \rangle &= \{1,(2\ 4\ 5),(2\ 5\ 4)\}\\
                \langle (3\ 4\ 5) \rangle &= \{1,(3\ 4\ 5),(3\ 5\ 4)\}
            \end{align*}

            \item $5-$subgrupos de Sylow.
            
            Por el Segundo Teorema de Sylow, tenemos que:
            \begin{equation*}
                n_5 \equiv 1 \mod 5 \qquad n_5 \mid 12
            \end{equation*}
            Por tanto, $n_5\in \{1,6\}$. Veamos cuántos elementos de orden $5$ hay en $A_5$:
            \begin{equation*}
                |\Cl_{A_5}((1\ 2\ 3\ 4\ 5))| = \frac{5!}{5} = 24
            \end{equation*}
            Por tanto, hay $6$ $5-$subgrupos de Sylow, que son:
            \begin{align*}
                \langle (1\ 2\ 3\ 4\ 5)\rangle\\
                \langle (1\ 2\ 3\ 5\ 4)\rangle\\
                \langle (1\ 2\ 5\ 3\ 4)\rangle\\
                \langle (1\ 5\ 2\ 3\ 4)\rangle\\
                \langle (1\ 2\ 4\ 3\ 5)\rangle\\
                \langle (1\ 2\ 4\ 5\ 3)\rangle
            \end{align*}


        \end{itemize}


        \item $S_5$.
        
        % // TODO: Hacer
    \end{enumerate}
\end{ejercicio}

\begin{ejercicio}\label{ej:6.22}
    Demostrar que $D_4$ es isomorfo a los $2-$subgrupos de Sylow de $S_4$.
    \begin{observacion}
        Considerar la representación asociada a la acción de $D_4$ sobre los vértices del cuadrado.
    \end{observacion}
    
    
    Este ejercicio ya se resolvió en el Ejercicio~\ref{ej:6.8}, donde vimos que los $2-$subgrupos de Sylow de $S_4$ son isomorfos a $D_4$ aplicando el Teorema de Dyck.
\end{ejercicio}

\begin{ejercicio}\label{ej:6.23}
    Demostrar que todo grupo de orden $12$ con más de un $3-$subgrupo de Sylow es isomorfo al grupo alternado $A_4$.
    \begin{observacion}
        Considerar la acción por traslación de un tal grupo sobre el conjunto de clases módulo $P$, siendo $P$ un $3-$subgrupo de Sylow. Probar que dicha acción es fiel.
    \end{observacion}
    
    Sea $G$ un grupo de orden $12$ con más de un $3-$subgrupo de Sylow. Por el Segundo Teorema de Sylow, tenemos que:
    \begin{equation*}
        n_3 \equiv 1 \mod 3 \qquad n_3 \mid 4
    \end{equation*}
    Por tanto, $n_3\in \{1,4\}$. Como $G$ tiene más de un $3-$subgrupo de Sylow, tenemos que $n_3=4$. Sean por tanto:
    \begin{align*}
        \Syl_3 = \{P_1,P_2,P_3,P_4\}
    \end{align*}

    Como cada $P_i$ es un grupo de orden $3$, es cíclico. De esta forma, supongamos que $\exists x\neq 1$ tal que $x\in P_i\cap P_j$ con $i\neq j$. Entonces, tenemos que:
    \begin{equation*}
        P_i = \langle x \rangle = \{1,x,x^2\} = P_j
    \end{equation*}
    Por tanto, $P_i=P_j$, lo cual es una contradicción. Por tanto, tenemos que:
    \begin{equation*}
        P_1\cap P_2 = P_1\cap P_3 = P_1\cap P_4 = P_2\cap P_3 = P_2\cap P_4 = P_3\cap P_4 = \{1\}
    \end{equation*}
    Por tanto, los $3-$subgrupos de Sylow son disjuntos dos a dos.\\

    Consideramos la acción de $G$ sobre el conjunto de clases módulo $P_1$. Como $P_1$ no es normal en $G$, no podemos considerar el grupo cociente, pero consideramos el conjunto de las clases por la izquierda:
    \begin{equation*}
        G/{\sim}_{P_1} = \{gP_1\mid g\in G\}
    \end{equation*}
    
    Sea por tanto la siguiente acción:
    \Func{ac}{G\times G/{\sim}_{P_1}}{G/{\sim}_{P_1}}{(g,hP_1)}{(gh)P_1}

    Veamos que está bien definida. Sea $g\in G$ y $h_1P_1,h_2P_1\in G/{\sim}_{P_1}$ tales que $h_1P_1=h_2P_1$. Entonces:
    \begin{equation*}
        (gh_1)P_1 = (gh_2)P_1
        \iff (gh_1)^{-1}(gh_2) \in P_1
        \iff h_1^{-1}h_2 \in P_1
        \iff h_1P_1 = h_2P_1
    \end{equation*}

    Por tanto, la acción está bien definida. Veamos que efectivamente es una acción:
    \begin{align*}
        \prescript{1}{}{hP_1} &= (1h)P_1 = hP_1\qquad \forall hP_1\in G/{\sim}_{P_1}\\
        \prescript{g_1}{}{\left(\prescript{g_2}{}{hP_1}\right)} &= \prescript{g_1}{}{(g_2h)P_1} = (g_1g_2h)P_1 = \prescript{g_1g_2}{}{hP_1}\qquad \forall g_1,g_2\in G,\ hP_1\in G/{\sim}_{P_1}
    \end{align*}

    Por tanto, consideramos su representación por permutaciones asociada a la acción:
    \Func{\Phi}{G}{\Perm(G/{\sim}_{P_1})}{g}{\prescript{g}{}{\left(\cdot P_1\right)}}


    Veamos el cardinal del conjunto de clases:
    \begin{equation*}
        |G/{\sim}_{P_1}| = [G:P_1] = \frac{|G|}{|P_1|} = \frac{12}{3} = 4
    \end{equation*}

    Por tanto, tenemos que:
    \Func{\Phi}{G}{S_4}{g}{\left(g(\cdot)\right)P_1}

    Calculamos que se trata de una acción fiel:
    \begin{align*}
        \ker(\Phi) &= \{g\in G\mid \prescript{g}{}{\left(\cdot P_1\right)} = Id_{G/{\sim}_{P_1}}\}\\
        &= \{g\in G\mid \prescript{g}{}{\left(hP_1\right)} = hP_1\ \forall hP_1\in G/{\sim}_{P_1}\}\\
        &= \{g\in G\mid (gh)P_1 = hP_1\ \forall hP_1\in G/{\sim}_{P_1}\}\\
        &= \{g\in G\mid h^{-1}gh \in P_1\ \forall h\in G\}\\
        &= \{g\in G\mid g\in hP_1h^{-1}\ \forall h\in G\}\\
    \end{align*}

    Por el Segundo Teorema de Sylow, todos los $3-$subgrupos de Sylow son conjugados entre sí, luego $hP_1h^{-1}$ es un $3-$subgrupo de Sylow de $G$ para todo $h\in G$. Por tanto, tenemos que:
    \begin{align*}
        \ker(\Phi) &\subset \bigcap_{h\in G} hP_1h^{-1}
        \subset \bigcap_{i=1}^4 P_i = \{1\}
    \end{align*}
    Por tanto, $\ker(\Phi)=\{1\}$, luego la acción es fiel.\\

    Por el Primer Teorema de Isomorfía, tenemos que:
    \begin{equation*}
        G\cong G/\{1\} = G/{\ker(\Phi)} \cong \Im(\Phi) \subset S_4
    \end{equation*}

    Por tanto, hemos visto que $G$ es isomorfo a un subgrupo de $S_4$. Como $|G|=12$ y el único subgrupo de orden $12$ de $S_4$ es $A_4$, tenemos que:
    \begin{equation*}
        G\cong A_4
    \end{equation*}
    Por tanto, hemos demostrado que todo grupo de orden $12$ con más de un $3-$subgrupo de Sylow es isomorfo al grupo alternado $A_4$.
\end{ejercicio}

\begin{ejercicio}\label{ej:6.24}~
    \begin{enumerate}
        \item Demostrar que no existen grupos simples de orden $12$. Más concretamente, demostrar que todo grupo de orden $12$ admite un subgrupo normal de orden $3$ o de orden $4$.
        
        Sea $G$ un grupo de orden $12=3\cdot 2^2$. Por el Segundo Teorema de Sylow, tenemos que:
        \begin{equation*}
            n_3 \mid 4 \qquad n_3 \equiv 1 \mod 3
        \end{equation*}

        Por tanto, $n_3\in \{1,4\}$.
        \begin{itemize}
            \item Si $n_3=1$, entonces el $3-$subgrupo de Sylow es normal con cardinal $3$ (luego no es propio), por lo que $G$ no es simple.
            
            \item Si $n_3=4$, aplicamos de nuevo el Segundo Teorema de Sylow:
            \begin{equation*}
                n_2 \mid 3 \qquad n_2 \equiv 1 \mod 2
            \end{equation*}
            Por tanto, $n_2\in \{1,3\}$.
            \begin{itemize}
                \item Si $n_2=1$, entonces el $2-$subgrupo de Sylow es normal con cardinal $4$ (luego no es propio), por lo que $G$ no es simple.
                \item Si $n_2=3$, $n_3=4$.

                Estudiamos la situación.
                \begin{itemize}
                    \item Como $n_3=4$, tenemos $4$ $3-$subgrupos de Sylow (todos ellos disjuntos), por lo que tenemos $4\cdot 2=8$ elementos de orden $3$.
                    \item Como $n_2=3$, tenemos $3$ $2-$subgrupos de Sylow, pero no podemos garantizar que sean disjuntos. Fijado $P\in \Syl_2(G)$, este tendrá $3$ elementos de orden $2$ o $4$. Además, puesto que los $2-$subgrupos de Sylow son distintos, al menos habrá otro elemento de orden $2$ o $4$ distinto. Por tanto, tenemos al menos $4$ elementos de orden $2$ o $4$.
                \end{itemize}
                
                Por tanto, tenemos:
                \begin{itemize}
                    \item $1$ elemento de orden $1$.
                    \item $8$ elementos de orden $3$.
                    \item Al menos $4$ elementos de orden $2$ o $4$.
                \end{itemize}
                Esto implica que el grupo tiene al menos $13$ elementos, lo cual es una contradicción. Por tanto, este caso no puede darse.
            \end{itemize}
        \end{itemize}

        Por tanto, hemos visto que $n_3=1$ (en cuyo caso $G$ tiene un subgrupo normal de orden $3$) o $n_2=1$ (en cuyo caso $G$ tiene un subgrupo normal de orden $4$). Por tanto, todo grupo de orden $12$ admite un subgrupo normal de orden $3$ o de orden $4$.


        \item Demostrar que no existen grupos simples de orden $28$. Más concretamente, probar que todo grupo de orden $28$ contiene un subgrupo normal de orden $7$.
        
        Sea $G$ un grupo de orden $28=7\cdot 2^2$. Por el Segundo Teorema de Sylow, tenemos que:
        \begin{equation*}
            n_7 \mid 4 \qquad n_7 \equiv 1 \mod 7
        \end{equation*}
        Por tanto, $n_7=1$. Por tanto el $7-$subgrupo de Sylow es normal con cardinal $7$ (luego no es propio), por lo que $G$ no es simple.
        \item Demostrar que no existen grupos simples de orden $56$. Más concretamente, probar que todo grupo de orden $56$ contiene un subgrupo normal de orden $7$ o de orden $8$.
        
        Sea $G$ un grupo de orden $56=7\cdot 2^3$. Por el Segundo Teorema de Sylow, tenemos que:
        \begin{equation*}
            n_7 \mid 8 \qquad n_7 \equiv 1 \mod 7
        \end{equation*}
        Por tanto, $n_7\in \{1,8\}$.
        \begin{itemize}
            \item Si $n_7=1$, entonces el $7-$subgrupo de Sylow es normal con cardinal $7$ (luego no es propio), por lo que $G$ no es simple.
            \item Si $n_7=8$, aplicamos de nuevo el Segundo Teorema de Sylow:
            \begin{equation*}
                n_2 \mid 7 \qquad n_2 \equiv 1 \mod 2
            \end{equation*}
            Por tanto, $n_2\in \{1,7\}$.
            \begin{itemize}
                \item Si $n_2=1$, entonces el $2-$subgrupo de Sylow es normal con cardinal $8$ (luego no es propio), por lo que $G$ no es simple.
                \item Si $n_2=7$, $n_7=8$.
                \begin{itemize}
                    \item Como $n_7=8$, hay $8$ $7-$subgrupos de Sylow (todos ellos distintos), por lo que tenemos $8\cdot 6=48$ elementos de orden $7$. Como $n_2=7$, tenemos $7$ $2-$subgrupos de Sylow, pero no podemos garantizar que sean disjuntos. Fijado $P\in \Syl_2(G)$, este contendrá $7$ elementos de orden $2$, $4$ o $8$. Además, puesto que  los $2-$subgrupos de Sylow son distintos, al menos habrá otro elemento de orden $2$, $4$ o $8$ distinto. Por tanto, tenemos al menos $8$ elementos de orden $2$, $4$ o $8$. 
                \end{itemize}
                
                Por tanto, tenemos:
                \begin{itemize}
                    \item $1$ elemento de orden $1$.
                    \item $48$ elementos de orden $7$.
                    \item $8$ elementos de orden $2$, $4$ o $8$.
                \end{itemize}
                Esto implica que el grupo tiene al menos $57$ elementos, lo cual es una contradicción. Por tanto, este caso no puede darse.
            \end{itemize}
        \end{itemize}

        Por tanto, hemos visto que $n_7=1$ (en cuyo caso $G$ tiene un subgrupo normal de orden $7$) o $n_2=1$ (en cuyo caso $G$ tiene un subgrupo normal de orden $8$).
        \item Demostrar que no existen grupos simples de orden $148$.
        
        Sea $G$ un grupo de orden $148=37\cdot 2^2$. Por el Segundo Teorema de Sylow, tenemos que:
        \begin{equation*}
            n_{37} \mid 4 \qquad n_{37} \equiv 1 \mod 37
        \end{equation*}
        Por tanto, $n_{37}=1$. Por tanto el $37-$subgrupo de Sylow es normal con cardinal $37$ (luego no es propio), por lo que $G$ no es simple.
        \item Demostrar que no existen grupos simples de orden $200$.
        Sea $G$ un grupo de orden $200=5^2\cdot 2^3$. Por el Segundo Teorema de Sylow, tenemos que:
        \begin{equation*}
            n_5 \mid 8 \qquad n_5 \equiv 1 \mod 5
        \end{equation*}
        Por tanto, $n_5=1$. Por tanto el $5-$subgrupo de Sylow es normal con cardinal $5^2$ (luego no es propio), por lo que $G$ no es simple.
        \item Demostrar que no existen grupos simples de orden $351$.
        Sea $G$ un grupo de orden $351=3^3\cdot 13$. Por el Segundo Teorema de Sylow, tenemos que:
        \begin{equation*}
            n_{13} \mid 27 \qquad n_{13} \equiv 1 \mod 13
        \end{equation*}
        Por tanto, $n_{13}\in \{1,27\}$.
        \begin{itemize}
            \item Si $n_{13}=1$, entonces el $13-$subgrupo de Sylow es normal con cardinal $13$ (luego no es propio), por lo que $G$ no es simple.
            \item Si $n_{13}=27$, aplicamos de nuevo el Segundo Teorema de Sylow:
            \begin{equation*}
                n_3 \mid 13 \qquad n_3 \equiv 1 \mod 3
            \end{equation*}
            Por tanto, $n_3\in \{1,13\}$.
            \begin{itemize}
                \item Si $n_3=1$, entonces el $3-$subgrupo de Sylow es normal con cardinal $27$ (luego no es propio), por lo que $G$ no es simple.
                \item Si $n_3=13$, $n_{13}=27$.
                \begin{itemize}
                    \item Como $n_{13}=27$, tenemos $27$ $13-$subgrupos de Sylow (todos ellos distintos), por lo que tenemos $27\cdot 12=324$ elementos de orden $13$.
                    \item Como $n_3=13$, tenemos $13$ $3-$subgrupos de Sylow, pero no podemos garantizar que sean disjuntos. Fijado $P\in \Syl_3(G)$, este contendrá $26$ elementos de orden $3$, $9$ o $27$. Además, puesto que los $3-$subgrupos de Sylow son distintos, al menos habrá otro elemento de orden $3$, $9$ o $27$ distinto. Por tanto, tenemos al menos $27$ elementos de orden $3$, $9$ o $27$.
                \end{itemize}
                Por tanto, tenemos:
                \begin{itemize}
                    \item $1$ elemento de orden $1$.
                    \item $324$ elementos de orden $13$.
                    \item $27$ elementos de orden $3,9$ o $27$.
                \end{itemize}
                Esto implica que el grupo tiene más de $351$ elementos, lo cual es una contradicción. Por tanto, este caso no puede darse.
            \end{itemize}
        \end{itemize}

        Por tanto, hemos visto que $n_{13}=1$ (en cuyo caso $G$ tiene un subgrupo normal de orden $13$) o $n_3=1$ (en cuyo caso $G$ tiene un subgrupo normal de orden $27$).
    \end{enumerate}
\end{ejercicio}

\begin{ejercicio}\label{ej:6.25}
    Calcular el número de elementos de orden $7$ que tiene un grupo simple de orden $168$.\\


    Sabemos que $168=2^3\cdot 3\cdot 7$. Como cada elemento de orden $7$ va a generar un grupo cíclico de orden $7$, buscamos el número de subgrupos de Sylow de orden $7$. Por la descomposición de $168$, sabemos que dichos grupos serán $7-$subgrupos de Sylow. Por el Segundo Teorema de Sylow, tenemos que:
    \begin{equation*}
        n_7\equiv 1\mod 7\qquad n_7\mid 24
    \end{equation*}
    Por tanto, $n_7\in \{1, 8\}$.
    \begin{itemize}
        \item Si $n_7=1$, entonces el $7-$subgrupo de Sylow es normal con cardinal $7$ (luego no es propio), por lo que $G$ no es simple.
    \end{itemize}

    Por tanto, $n_7=8$. Por tanto, hay exactamente $8$ subgrupos de orden $7$. Cada uno de los elementos de orden $7$ del grupo pertenece a un único subgrupo de orden $7$, y será un generador de estos. Además, sabemos que el número de elementos de orden $7$ de un grupo cíclico de orden $7$ viene dado por la función $\varphi(7) = 6$. Por tanto, el número de elementos de orden $7$ de un grupo simple de orden $168$ es:
    \begin{equation*}
        8\cdot 6 = 48
    \end{equation*}
\end{ejercicio}

    \chapter{Cuestionarios} 
    \subsection{Cuestionario I}
\begin{ejercicio}
    Si $A$ es un conjunto finito arbitrario, la afirmación ``$|P(A)| > |A|$'' es:
    \begin{itemize}
        \item Siempre verdadera.
        \item Verdadera o falsa, depende de $A$.
        \item Siempre falsa.
    \end{itemize}
\end{ejercicio}

\begin{ejercicio}
    Si $A$, $B$, $C$ son conjuntos cualesquira con $B$ y $C$ disjuntos, selecciona la afirmación verdadera:
    \begin{itemize}
        \item $(A \cup B)\cap C = A$.
        \item $(A \cup B)\cap (A \cup C)=A$.
        \item $(A\cap B)\cup(A \cap C)=A$.
    \end{itemize}
\end{ejercicio}

\begin{ejercicio}
    Si $A$ y $B$ son subconjuntos de un conjunto, la afirmación \newline ``$c(A) \cap c(B) = c(A \cap B)$'' es:
    \begin{itemize}
        \item Siempre cierta.
        \item Siempre falsa.
        \item A veces verdadera y a veces falsa, depende de $A$ y $B$.
    \end{itemize}
\end{ejercicio}

\begin{ejercicio}
    Sean $P$ y $Q$ las propiedades referidas a los elementos de un conjunto. Las proposiciones $P \Rightarrow \neg Q$ y $Q \Rightarrow \neg P$ son:
    \begin{itemize}
        \item Siempre equivalentes.
        \item Nunca equivalentes.
        \item A veces equivalentes y a veces no, depende de $P$ y de $Q$.
    \end{itemize}
\end{ejercicio}

\begin{ejercicio}
    Sean $P$, $Q$ y $R$ propiedades referidas a los elementos de un conjunto tal que $P \Rightarrow Q \lor R$, entonces (seleccionar la afirmación correcta):
    \begin{itemize}
        \item $P \Rightarrow Q$ y $P \Rightarrow R$.
        \item $P \Rightarrow Q$ o $P \Rightarrow R$.
        \item $P \Rightarrow Q$ siempre que $R \Rightarrow Q$.
    \end{itemize}
\end{ejercicio}

\newpage
\ % --------------------------------------------------------------------------------
\resetearcontador

\begin{ejercicio}
    Si $A$ es un conjunto finito arbitrario, la afirmación ``$|P(A)| > |A|$'' es:
    \begin{itemize}
        \item \underline{Siempre verdadera.}
        \item Verdadera o falsa, depende de $A$.
        \item Siempre falsa.
    \end{itemize}

    \noindent
    \textbf{Justificación}:
    Si $A = \emptyset$, entonces $P(A) = \{\emptyset\}$ y $|P(A)|=1>0=|A|$.\newline
    Si $A \neq \emptyset$, entonces $P(A)$ contiene a todos los subconjuntos unitarios $\{a\}$, con $a \in A$ (luego, el cardinal de $P(A)$ es, como mínimo, igual al de $|A|$) y, además, contiene el subconjunto vacío, luego tiene al menos tantos elementos como $A$ más uno.\\

    \noindent
    Otra alternativa es usar la fórmula vista para el cardinal del conjunto potencia de un conjunto finito vista en teoría:\newline
    Sea $A$ un conjunto finito arbitrario con $|A| = n \in \bb{N}$, entonces $|\mathcal{P}(A)| = 2^n$.\newline
    Notemos que $2^n > n\quad\forall n \in \bb{N}$.
\end{ejercicio}

\begin{ejercicio}
    Si $A$, $B$, $C$ son conjuntos cualesquira con $B$ y $C$ disjuntos, selecciona la afirmación verdadera:
    \begin{itemize}
        \item $(A \cup B)\cap C = A$.
        \item \underline{$(A \cup B)\cap (A \cup C)=A$.}
        \item $(A\cap B)\cup(A \cap C)=A$.
    \end{itemize}

    \noindent
    \textbf{Justificación}:
    \begin{equation*}
        (A \cup B) \cap (A \cup C) = A \cup (B \cap C) = A \cup \emptyset = A    
    \end{equation*}
\end{ejercicio}

\begin{ejercicio}
    Si $A$ y $B$ son subconjuntos de un conjunto, la afirmación \newline ``$c(A) \cap c(B) = c(A \cap B)$'' es:
    \begin{itemize}
        \item Siempre cierta.
        \item Siempre falsa.
        \item \underline{A veces verdadera y a veces falsa, depende de $A$ y $B$.}
    \end{itemize}

    \noindent
    \textbf{Justificación}:
    Por las Leyes de Morgan: $c(A \cap B) = c(A) \cup c(B)$, por lo que podemos intuir que la afirmación no siempre es cierta. Podemos dar un contraejemplo para ilustrarlo:\newline
    Sea $X = \{1,2,3,4,5\}$, sean $A = \{1,2,3\}$, $B = \{4,5\} \subseteq X$:
    \begin{gather*}
        c(A) = B\qquad c(B) = A
        c(A \cap B) = c(\emptyset) = X \neq c(A) \cap c(B) = \emptyset
    \end{gather*}
    Además, como no impone nada sobre los conjuntos, podemos ver que si $A = B$, es cierta la afirmación. Supongamos que $A = B$:
    \begin{equation*}
        c(A \cap B) = c(A \cap A) = c(A) = c(A) \cup c(A) = c(A) \cup c(B)
    \end{equation*}
\end{ejercicio}

\newpage
\begin{ejercicio}
    Sean $P$ y $Q$ las propiedades referidas a los elementos de un conjunto. Las proposiciones $P \Rightarrow \neg Q$ y $Q \Rightarrow \neg P$ son:
    \begin{itemize}
        \item \underline{Siempre equivalentes.}
        \item Nunca equivalentes.
        \item A veces equivalentes y a veces no, depende de $P$ y de $Q$.
    \end{itemize}

    \noindent
    \textbf{Justificación}:
    $Q \Rightarrow \neg P$ es el contrarrecíproco de $P \Rightarrow \neg Q$.\newline
    Demostremos que $(Q \Rightarrow \neg P) \Leftrightarrow (P \Rightarrow \neg Q)$:\newline
    O, equivalentemente, que $X_Q \subseteq c(X_P) \Leftrightarrow X_P \subseteq c(X_Q)$.
    \begin{description}
        \item [$\Rightarrow)$]
            Sea $ x \in X_P \Rightarrow x \notin c(X_P) \Rightarrow x \notin X_Q \Rightarrow x \in c(X_Q)$\newline
            Para todo $x \in X_P$, luego $X_P \subseteq c(X_Q)$.
        \item [$\Leftarrow)$]
            Sea $ x \in X_Q \Rightarrow x \notin c(X_Q) \Rightarrow x \notin X_P \Rightarrow x \in c(X_P)$\newline
            Para todo $x \in X_Q$, luego $X_Q \subseteq c(X_P)$.
    \end{description}
\end{ejercicio}

\begin{ejercicio}
    Sean $P$, $Q$ y $R$ propiedades referidas a los elementos de un conjunto tal que $P \Rightarrow Q \lor R$, entonces (seleccionar la afirmación correcta):
    \begin{itemize}
        \item $P \Rightarrow Q$ y $P \Rightarrow R$.
        \item $P \Rightarrow Q$ o $P \Rightarrow R$.
        \item \underline{$P \Rightarrow Q$ siempre que $R \Rightarrow Q$.}
    \end{itemize}

    \noindent
    \textbf{Justificación}:
    Por hipótesis, $X_P \subseteq X_Q \cup X_R$.\newline
    Si $X_R \subseteq X_Q \Rightarrow X_P \subseteq X_Q = X_Q \cup X_R$.

\end{ejercicio}

\newpage
\resetearcontador

    \section{Cuestionario II}
\begin{ejercicio}
    Sean $X$ e $Y$ dos conjuntos finitos con $|X| = |Y|$ y $f:X \rightarrow Y$ una aplicación. La afirmación ``Si $f$ es inyectiva o sobreyectiva, entonces $f$ es biyectiva'' es:
    \begin{itemize}
        \item Verdadera o falsa, depende de $f$.
        \item Siempre verdadera.
        \item Siempre falsa.
    \end{itemize}
\end{ejercicio}

\begin{ejercicio}
    Sea $f:X \rightarrow Y$ una aplicación inyectiva y sean $A, B \subseteq X$. Selecciona la afirmación verdadera:
    \begin{itemize}
        \item $f_{*}(A) - f_{*}(B)$ es un subconjunto propio de $f_{*}(A-B)$.
        \item $f_{*}(A-B)$ es un subconjunto propio de $f_{*}(A) - f_{*}(B)$.
        \item $f_{*}(A-B) = f_{*}(A) - f_{*}(B)$.
    \end{itemize}
\end{ejercicio}

\begin{ejercicio}
    Sea $f:X \rightarrow X$ una aplicación tal que $f_{*}(c(A)) = c(f_{*}(A))$, para todo $A \in \mathcal{P}(X)$. Entonces:
    \begin{itemize}
        \item $f$ es inyectiva, pero no necesariamente sobreyectiva.
        \item $f$ es sobreyectiva, pero no necesariamente inyectiva.
        \item $f$ es biyectiva.
    \end{itemize}
\end{ejercicio}

\begin{ejercicio}
Sea $X$ un conjunto con $|X|\geq 2$. La afirmación ``Todo subconjunto de $X \times X$ es de la forma $A \times B$ para ciertos subconjuntos $A, B \subseteq X$'' es:
    \begin{itemize}
        \item Verdadera o falsa, depende de $X$.
        \item Siempre verdadera.
        \item Siempre falsa.
    \end{itemize}
\end{ejercicio}

\begin{ejercicio}
    Sea $R$ una relación simétrica y transitiva en un conjunto $X \neq \emptyset$ ¿Prueba el siguiente razonamiento que $R$ es reflexiva?:\newline
    ``Por simetría, $aRb$ implica $bRa$ y entonces, por transitividad, concluimos que $aRa$''.
    \begin{itemize}
        \item Sí.
        \item No.
    \end{itemize}
\end{ejercicio}

\newpage
\ % --------------------------------------------------------------------------------
\resetearcontador

\begin{ejercicio}
    Sean $X$ e $Y$ dos conjuntos finitos con $|X| = |Y|$ y $f:X \rightarrow Y$ una aplicación. La afirmación ``Si $f$ es inyectiva o sobreyectiva, entonces $f$ es biyectiva'' es:
    \begin{itemize}
        \item Verdadera o falsa, depende de $f$.
        \item \underline{Siempre verdadera.}
        \item Siempre falsa.
    \end{itemize}

    \noindent
    \textbf{Justificación}:
    Si $f$ es inyectiva, entonces $|X| = |Img(f)|$, luego $|Img(f)| = |Y|$ y por tanto, $Img(f) = Y$ y $f$ es sobreyectiva luego biyectiva.\newline
    Si $f$ es sobreyectiva, entonces $|Y|=|Img(f)|$, luego $|Img(f)| = |X|$ y por tanto, $f$ es necesariamente inyectiva luego biyectiva.
\end{ejercicio}

\begin{ejercicio}
    Sea $f:X \rightarrow Y$ una aplicación inyectiva y sean $A, B \subseteq X$. Selecciona la afirmación verdadera:
    \begin{itemize}
        \item $f_{*}(A) - f_{*}(B)$ es un subconjunto propio de $f_{*}(A-B)$.
        \item $f_{*}(A-B)$ es un subconjunto propio de $f_{*}(A) - f_{*}(B)$.
        \item \underline{$f_{*}(A-B) = f_{*}(A) - f_{*}(B)$.}
    \end{itemize}

    \noindent
    \textbf{Justificación}:
    Empezamos recordando la definición de $f_{*}(A)$ para $A \subseteq X$:
    \begin{equation*}
        f_{*}(A) = \{y \in X \mid \exists x \in X \mbox{\ con\ } f(x) = y \}
    \end{equation*}
    \begin{description}
        \item [$\subseteq)$]
            Sea $y \in f_{*}(A-B) \Rightarrow \exists x \in A -B \mid y = f(x)$.\newline
            Esto es, $\exists x \in A \land x \notin B \mid y = f(x)$.\newline
            Como $x \in A \Rightarrow y = f(x) \in f_{*}(A)$. Además, por ser $f$ inyectiva, se tiene que $y \notin f_{*}(B)$, ya que si suponemos que $y \in f_{*}(B)$:

            $y \in f_{*}(B) \Rightarrow \exists b \in B \mid y = f(b) \Rightarrow f(x) = f(b)$ con lo que $x = b \in B$, en contradicción con que $x \notin B$.

            \noindent
            Así, $y \in f_{*}(A) - f_{*}(B)$ para todo $y \in f_{*}(A-B)$. Luego:
            \begin{equation*}
                f_{*}(A-B) \subseteq f_{*}(A) - f_{*}(B)
            \end{equation*}
        \item [$\supseteq)$]
            Sea $y \in f_{*}(A) - f_{*}(B) \Rightarrow y \in f_{*}(A) \land y \notin f_{*}(B)$.\newline
            Como $y \in f_{*}(A) \Rightarrow \exists x \in A \mid y = f(x)$.\newline
            Como $y \notin f_{*}(B) \Rightarrow x \notin B$.\newline
            Luego $x \in A -B \Rightarrow y = f(x) \in f_{*}(A-B)$ para todo $y \in f_{*}(A) - f_{*}(B)$. Luego:
            \begin{equation*}
                f_{*}(A)-f_{*}(B) \subseteq f_{*}(A-B)
            \end{equation*}
    \end{description}
\end{ejercicio}

\begin{ejercicio}
    Sea $f:X \rightarrow X$ una aplicación tal que $f_{*}(c(A)) = c(f_{*}(A))$, para todo $A \in \mathcal{P}(X)$. Entonces:
    \begin{itemize}
        \item $f$ es inyectiva, pero no necesariamente sobreyectiva.
        \item $f$ es sobreyectiva, pero no necesariamente inyectiva.
        \item \underline{$f$ es biyectiva.}
    \end{itemize}

    \noindent
    \textbf{Justificación}:
    Procedemos a demostrar la inyectividad y sobreyectividad de la aplicación.\newline
    Para la sobreyectividad, consideramos $\emptyset \in \mathcal{P}(X)$:
    \begin{equation*}
        f_{*}(c(\emptyset)) = f_{*}(X) = Img(f) = c(f_{*}(\emptyset)) = c(\emptyset) = X
    \end{equation*}
    Para la inyectividad, podemos suponer sin perder generalidad que $|X| \geq 2$ (si no lo fuera, la aplicación sería automáticamente inyectiva).\newline
    Sean $x, x' \in X \mid x \neq x'$. Entonces, $x' \in c(\{x\})$ luego:
    \begin{equation*}
        f(x') \in f_{*}(c(\{x\})) = c(\{f(x)\})
    \end{equation*}
    Luego $f(x') \neq f(x)$.
\end{ejercicio}

\begin{ejercicio}
Sea $X$ un conjunto con $|X|\geq 2$. La afirmación ``Todo subconjunto de $X \times X$ es de la forma $A \times B$ para ciertos subconjuntos $A, B \subseteq X$'' es:
    \begin{itemize}
        \item Verdadera o falsa, depende de $X$.
        \item Siempre verdadera.
        \item \underline{Siempre falsa.}
    \end{itemize}

    \noindent
    \textbf{Justificación}: Supongamos que sí y consideremos el siguiente conjunto:\newline
    Sea $D = \{(x,x) \mid x \in X\} \subseteq X \times X$.\newline
    Si $D = A \times B$ para ciertos $A, B \subseteq X$, entonces para todo $x \in X$, $(x,x) \in A \times B$ y, por tanto, $x \in A$ y $x \in B$.\newline
    Así que $A = X = B$ y, necesariamente, $D = X \times X$. Pero $|X| \geq 2$, luego existen $a,b \in X$ con $a \neq b$, esto es, $(a,b) \notin D$ y $D \neq X \times X$.\newline
    Lo que nos lleva a contradicción.
\end{ejercicio}

\begin{ejercicio}
    Sea $R$ una relación simétrica y transitiva en un conjunto $X \neq \emptyset$ ¿Prueba el siguiente razonamiento que $R$ es reflexiva?:\newline
    ``Por simetría, $aRb$ implica $bRa$ y entonces, por transitividad, concluimos que $aRa$''.
    \begin{itemize}
        \item Sí.
        \item \underline{No.}
    \end{itemize}

    \noindent
    \textbf{Justificación}:
    Dado un $a \in X$, no tiene por qué existir a priori un elemento $b \in X$ tal que $aRb$. Por tanto, buscamos un contraejemplo para desmentir la afirmación:\\

    \noindent
    Dado $X = \{ a,b,c \} \neq \emptyset$ y la relación $R = \{ (a,b), (b,a), (b,b),(a,a) \} \subseteq X \times X$. \newline Observemos que $R$ es simétrica y transitiva pero no reflexiva:

    \noindent
    Es simétrica ya que para todos $\alpha, \beta \in X \mid \alpha R \beta \Rightarrow \beta R \alpha$:
    \begin{center}
        Ya que $aRb$, ¿se cumple que $bRa$?. Sí.\\
        Ya que $bRa$, ¿se cumple que $aRb$?. Sí.\\
        Ya que $bRb$, ¿se cumple que $bRb$?. Sí.\\
        Ya que $aRa$, ¿se cumple que $aRa$?. Sí.
    \end{center}
    Es transitiva ya que para todos $\alpha, \beta, \gamma \in X \mid \alpha R \beta \land \beta R \gamma \Rightarrow \alpha R \gamma$:
    \begin{center}
        Ya que $aRb$ y $bRa$, ¿se cumple que $aRa$?. Sí.\\
        Ya que $bRa$ y $aRb$, ¿se cumple que $bRb$?. Sí.\\
        Ya que $bRb$ y $bRb$, ¿se cumple que $bRb$?. Sí.\\
        Ya que $aRa$ y $aRa$, ¿se cumple que $aRa$?. Sí.
    \end{center}
    No es reflexiva, ya que $\exists c \in X \mid c\cancel{R}c$.
\end{ejercicio}

\newpage
\resetearcontador


    \subsection{Cuestionario III}
\begin{ejercicio}
    Sea $X$ un conjunto no vacío. Definimos en $\mathcal{P}(X)$ operaciones de suma y producto por $A+B = A \cup B$ y $A \cdot B = A \cap B$. Entonces (selecciona la respuesta correcta).
    \begin{itemize}
        \item $\mathcal{P}(X)$ es un anillo conmutativo.
        \item $\mathcal{P}(X)$ no es un anillo conmutativo, falla un axioma.
        \item $\mathcal{P}(X)$ no es un anillo conmutativo, fallan dos axiomas.
    \end{itemize}
\end{ejercicio}

\begin{ejercicio}
    Para enteros $m$ y $n$ tales que $2 \leq m < n$, la afirmación ``$\bb{Z}_m$ es un subanillo de $\bb{Z}_n$'' es:
    \begin{itemize}
        \item Verdadera o falsa, dependiendo de $m$ y de $n$.
        \item Siempre verdadera.
        \item Siempre falsa.
    \end{itemize}
\end{ejercicio}

\begin{ejercicio}
    En el anillo $\bb{Z}_8$ (seleccion la afirmación verdadera).
    \begin{itemize}
        \item $3$ es una unidad y $4 \cdot 3^{-1} = 4$.
        \item $3$ es una unidad, pero $4 \cdot 3^{-1} \neq 4$.
        \item $3$ no es una unidad.
    \end{itemize}
\end{ejercicio}

\begin{ejercicio}
    En el anillo $\bb{Z}[\sqrt{3}]$, la afirmación ``${(7+4\sqrt{3})}^n$ es una unidad para todo natural $n \geq 1$'' es:
    \begin{itemize}
        \item Verdadera o falsa, dependiendo de $n$.
        \item Siempre verdadera.
        \item Siempre falsa.
    \end{itemize}
\end{ejercicio}

\begin{ejercicio}
    Sea $A \subseteq \bb{R}$ un subanillo. La afirmación ``\bb{Z} es un subanillo de A'' es:
    \begin{itemize}
        \item Siempre verdadera.
        \item Siempre falsa.
        \item Verdadera o falsa, dependiendo de $A$.
    \end{itemize}
\end{ejercicio}

\newpage
\ % --------------------------------------------------------------------------------
\resetearcontador

\begin{ejercicio}
    Sea $X$ un conjunto no vacío. Definimos en $\mathcal{P}(X)$ operaciones de suma y producto por $A+B = A \cup B$ y $A \cdot B = A \cap B$. Entonces (selecciona la respuesta correcta).
    \begin{itemize}
        \item $\mathcal{P}(X)$ es un anillo conmutativo.
        \item \underline{$\mathcal{P}(X)$ no es un anillo conmutativo, falla un axioma.}
        \item $\mathcal{P}(X)$ no es un anillo conmutativo, fallan dos axiomas.
    \end{itemize}

    \noindent
    \textbf{Justificación}:
    En este caso, $0 = \emptyset$, ya que:
    \begin{equation*}
        \emptyset + A = \emptyset \cup A = A\quad\forall A \in \mathcal{P}(X)
    \end{equation*}
    Y no hay opuestos, sea $A\neq \emptyset \in \mathcal{P}(X)$:
    \begin{equation*}
        A + B = A \cup B \supseteq A \neq \emptyset\quad\forall B \in \mathcal{P}(X)
    \end{equation*}
    Podemos ver que el resto de axiomas se cumplen:
    
    \begin{itemize}
        \item Conmutativa de la suma:
        \begin{equation*}
            A + B = A \cup B = B \cup A = B + A\quad\forall A,B \in \mathcal{P}(X)
        \end{equation*}
        \item Asociativa de la suma:
        \begin{equation*}
            A + (B + C) = A \cup (B \cup C) = (A \cup B) \cup C = (A+B)+C\quad\forall A,B,C \in \mathcal{P}(X)
        \end{equation*}
        \item Elemento neutro de la suma (ya demostrado).
        \item Existencia de opuestos (ya se ha visto que no se cumple).
        \item Conmutativa del producto:
        \begin{equation*}
            A \cdot B = A \cap B = B \cap A = B \cdot A\quad\forall A,B \in \mathcal{P}(X)
        \end{equation*}
        \item Asociativa del producto:
        \begin{equation*}
            A \cdot (B \cdot C) = A \cap (B \cap C) = (A \cap B) \cap C = (A\cdot B)\cdot C\quad\forall A,B,C \in \mathcal{P}(X)
        \end{equation*}
        \item Elemento neutro del producto:
            \begin{equation*}
                A \cdot X = A\quad\forall A \in \mathcal{P}(X)
            \end{equation*}
        \item Distributiva del producto respecto de la suma:
            \begin{equation*}
                A \cdot (B + C) = A \cap (B \cup C) = (A \cap B) \cup (A \cap C) = (A \cdot B) +(A\cdot C)\quad\forall A,B,C \in \mathcal{P}(X)
            \end{equation*}
    \end{itemize}
\end{ejercicio}

\begin{ejercicio}
    Para enteros $m$ y $n$ tales que $2 \leq m < n$, la afirmación ``$\bb{Z}_m$ es un subanillo de $\bb{Z}_n$'' es:
    \begin{itemize}
        \item Verdadera o falsa, dependiendo de $m$ y de $n$.
        \item Siempre verdadera.
        \item \underline{Siempre falsa.}
    \end{itemize}

    \noindent
    \textbf{Justificación}:
    En $\bb{Z}_m$, se tiene que $m = 0$.\newline
    Sin embargo, por ser $2 \leq m < n$, tenemos que $m \neq 0$ en $\bb{Z}_n$.
\end{ejercicio}

\begin{ejercicio}
    En el anillo $\bb{Z}_8$ (seleccion la afirmación verdadera).
    \begin{itemize}
        \item \underline{$3$ es una unidad y $4 \cdot 3^{-1} = 4$.}
        \item $3$ es una unidad, pero $4 \cdot 3^{-1} \neq 4$.
        \item $3$ no es una unidad.
    \end{itemize}

    \noindent
    \textbf{Justificación}:
    $3$ es una unidad ya que $3 \cdot 3 = 9 = 1$, luego $3^{-1} = 3$.\newline
    Entonces, $4 \cdot 3^{-1} = 4 \cdot 3 = 12 = 4$.
\end{ejercicio}

\begin{ejercicio}
    En el anillo $\bb{Z}[\sqrt{3}]$, la afirmación ``${(7+4\sqrt{3})}^n$ es una unidad para todo natural $n \geq 1$'' es:
    \begin{itemize}
        \item Verdadera o falsa, dependiendo de $n$.
        \item Siempre falsa.
        \item \underline{Siempre verdadera.}
    \end{itemize}

    \noindent
    \textbf{Justificación}:
    Tenemos que $7 + 4\sqrt{3}$ es invertible, puesto que:
    \begin{equation*}
        N(7+4\sqrt{3}) = 7^2 - 3 \cdot 16 = 49 - 48 = 1
    \end{equation*}
    Como el producto de unidades es una unidad, cualquier potencia de una unidad también lo es.
\end{ejercicio}

\begin{ejercicio}
    Sea $A \subseteq \bb{R}$ un subanillo. La afirmación ``\bb{Z} es un subanillo de A'' es:
    \begin{itemize}
        \item \underline{Siempre verdadera.}
        \item Siempre falsa.
        \item Verdadera o falsa, dependiendo de $A$.
    \end{itemize}

    \noindent
    \textbf{Justificación}:
    Por inducción, veamos primero que $\bb{N} = \bb{Z}^{+} \subseteq A$.\newline
    Esto es, que $n \in A\quad\forall n \in \bb{N}$.
    \begin{enumerate}
        \item [$n=0$:]
            Por ser $A$ subanillo de $\bb{R}$, se tiene que $0 \in A$.
        \item [$n=1$:]
            Por ser $A$ subanillo de $\bb{R}$, se tiene que $1 \in A$.
        \item [$n>1$:]
            Como hipótesis de inducción, supongamos que $n \in A$ y veamos que $n+1 \in A$.\newline
            Por ser $A$ cerrado para la suma, tenemos que $1 \in A$ y que $n \in A$ por hipótesis de inducción, luego $n+1 \in A$.
    \end{enumerate}
    Por tanto, $\bb{N} = \bb{Z}^{+} \subseteq A$.\newline
    Ahora, para $n \in \bb{Z}$ con $n \geq 0$, $A$ es cerrado para opuestos, luego $-n \in A$.\newline
    Por tanto, $\bb{Z} \subseteq A$.\\

    \noindent
    Por ser $\bb{Z}$ cerrado para la suma, producto, opuestos y contiene al $0$ y al $1$, $\bb{Z}$ es subanillo de $A$. Por tanto, $\bb{Z}$ es el menor subanillo de $\bb{R}$.
\end{ejercicio}

\newpage
\resetearcontador


    \subsection{Cuestionario IV}
\begin{ejercicio}
    En el anillo $\bb{Z}_{10}$, la afirmación ``$3^{4k+3} = -3$, para cualquier $k \in \bb{Z}$'' es:
    \begin{itemize}
        \item Siempre falsa.
        \item Siempre cierta.
        \item A veces cierta y a veces falsa, depende de $k$.
    \end{itemize}
\end{ejercicio}

\begin{ejercicio}
    En el anillo $\bb{Z}_n[x]$, la afirmación ``la suma reiterada $n$ veces de cualquier polinomio es $0$'', es:
    \begin{itemize}
        \item Verdera o falsa, depende de $n$.
        \item Siempre falsa.
        \item Siempre verdadera.
    \end{itemize}
\end{ejercicio}

\begin{ejercicio}
    Un subanillo $A$ de un anillo $B$ se dice propio si $A \subsetneq B$. Seleccion el enunciado correcto:
    \begin{itemize}
        \item En anillo $\bb{Z}$ no tiene subanillos propios.
        \item El conjunto $A = \{ 5k \mid k \in \bb{Z} \}$ es un subanillo propio de $\bb{Z}$.
        \item El cuerpo $\bb{Q}$ no tiene subanillos propios.
    \end{itemize}
\end{ejercicio}

\begin{ejercicio}
    Homomorifismos $\phi : \bb{Z}_2 \rightarrow \bb{Z}$,
    \begin{itemize}
        \item Hay exactamente uno.
        \item Hay al menos dos.
        \item No hay ninguno.
    \end{itemize}
\end{ejercicio}

\begin{ejercicio}
    Sea $A$ un anillo comutativo, la afirmación ``Para cualesquiera indeterminadas $x$ e $y$, los anillos de polinomios $A[x]$ y $A[y]$ son isomorifos''. Es:
    \begin{itemize}
        \item Verdadera o falsa, depende de $A$.
        \item Siempre verdadera.
        \item Siempre falsa.
    \end{itemize}
\end{ejercicio}

\newpage
\ % --------------------------------------------------------------------------------
\resetearcontador

\begin{ejercicio}
    En el anillo $\bb{Z}_{10}$, la afirmación ``$3^{4k+3} = -3$, para cualquier $k \in \bb{Z}$'' es:
    \begin{itemize}
        \item Siempre falsa.
        \item \underline{Siempre cierta.}
        \item A veces cierta y a veces falsa, depende de $k$.
    \end{itemize}

    \noindent
    \textbf{Justificación}:
    \begin{equation*}
    3^{4k+3}={(3^4)}^k \cdot 3^3 = {(9 \cdot  9)}^k \cdot 9 \cdot 3 = 1^k \cdot 7 = 7\quad \forall k \in \mathbb{Z}
    \end{equation*}
\end{ejercicio}

\begin{ejercicio}
    En el anillo $\bb{Z}_n[x]$, la afirmación ``la suma reiterada $n$ veces de cualquier polinomio es $0$'', es:
    \begin{itemize}
        \item Verdera o falsa, depende de $n$.
        \item Siempre falsa.
        \item \underline{Siempre verdadera.}
    \end{itemize}

    \noindent
    \textbf{Justificación}:
    Sea $R_n:\mathbb{Z}[x]\to \mathbb{Z}_n[x]$ el homomorfismo de reducción módulo $n$. Para cualquier $f \in \mathbb{Z}_n[x]$:
    \begin{equation*}
        nf = nR_n(f) = R_n(nf) = R_n(n)R_n(f) = 0 \cdot f = 0
    \end{equation*}
\end{ejercicio}

\begin{ejercicio}
    Un subanillo $A$ de un anillo $B$ se dice propio si $A \subsetneq B$. Seleccion el enunciado correcto:
    \begin{itemize}
        \item \underline{En anillo $\bb{Z}$ no tiene subanillos propios.}
        \item El conjunto $A = \{ 5k \mid k \in \bb{Z} \}$ es un subanillo propio de $\bb{Z}$.
        \item El cuerpo $\bb{Q}$ no tiene subanillos propios.
    \end{itemize}

    \noindent
    \textbf{Justificación}:
    Si $A$ es un subanillo de $\mathbb{Z}$, entonces $1 \in A$ con lo que para todo $n \geq 0$, $\overbrace{1+\cdots+1}^{n \text{\ veces}}=n \in A$ y, como $A$ contiene a sus opuestos, entonces $\mathbb{Z}\subseteq A$. Por lo que $A = \mathbb{Z}$.
\end{ejercicio}

\begin{ejercicio}
    Homomorifismos $\phi : \bb{Z}_2 \rightarrow \bb{Z}$,
    \begin{itemize}
        \item Hay exactamente uno.
        \item Hay al menos dos.
        \item \underline{No hay ninguno.}
    \end{itemize}

    \noindent
    \textbf{Justificación}:
    Si $\phi:\mathbb{Z}_2\to \mathbb{Z}$ fuese un homomorfismo, tendríamos que:
    \begin{equation*}
        \phi(1+1) = \phi(1) + \phi(1) = 1+1 = 2
    \end{equation*}
    Pero en $\mathbb{Z}_2$, $1+1=0$ y por tanto, $\phi(1+1)=\phi(0)=0$, así que sería $0 = 2$ en $\mathbb{Z}$, lo que es una contradicción.
\end{ejercicio}

\begin{ejercicio}
    Sea $A$ un anillo comutativo, la afirmación ``Para cualesquiera indeterminadas $x$ e $y$, los anillos de polinomios $A[x]$ y $A[y]$ son isomorifos''. Es:
    \begin{itemize}
        \item Verdadera o falsa, depende de $A$.
        \item \underline{Siempre verdadera.}
        \item Siempre falsa.
    \end{itemize}

    \noindent
    \textbf{Justificación}:
    El automorfismo identidad $id_A:A \cong A$ extiende a un único homomorfismo $\phi:A[x]\to A[y]$ tal que $\phi(x)=y$. Explícitamente:
    \begin{equation*}
        \phi\left(\sum_{i=0}^{n} a_i x^i\right) = \sum_{i=0}^{n} a_i y^i
    \end{equation*}
    Claramente $\phi$ es biyectiva.
\end{ejercicio}

\newpage
\resetearcontador


    \subsection{Cuestionario V}

\begin{ejercicio}
    En relación con los anillos $\bb{Z}_6$ y $\bb{Z} \times \bb{Z}$, selecciona la afirmación correcta:
    \begin{itemize}
        \item Ambos son DI.
        \item Uno de ellos es DI, pero el otro no.
        \item Ninguno es DI.
    \end{itemize}
\end{ejercicio}

\begin{ejercicio}
    En relación a las siguientes proposiciones, referidas a los elementos de un Dominio de Integridad:
    \begin{enumerate}
        \item [(a)] $a\mid b \land a \nmid c \Rightarrow b \nmid b+c$.
        \item [(b)] $a\mid b \land a \nmid c \Rightarrow a \nmid b+c$.
    \end{enumerate}
    Selecciona la afirmación correcta:
    \begin{itemize}
        \item Ambas son verdad.
        \item Una es verdad y la otra es falsa.
        \item Ambas son falsas.
    \end{itemize}
\end{ejercicio}

\begin{ejercicio}
    Polinomios de grado uno que son unidades en el anillo de polinomios $\bb{Z}_4[x]$:
    \begin{itemize}
        \item No hay.
        \item Hay dos.
        \item Hay infinitos.
    \end{itemize}
\end{ejercicio}

\begin{ejercicio}
    En el anillo $\bb{Z}[i]$:
    \begin{itemize}
        \item $3$ es unidad.
        \item $3$ es irreducible.
        \item $3$ no es irreducible.
    \end{itemize}
\end{ejercicio}

\begin{ejercicio}
    En el anillo $\bb{Z}[i]$:
    \begin{itemize}
        \item $2$ es unidad.
        \item $2$ es irreducible.
        \item $2$ no es irreducible.
    \end{itemize}
\end{ejercicio}

\newpage
\ % --------------------------------------------------------------------------------
\resetearcontador

\begin{ejercicio}
    En relación con los anillos $\bb{Z}_6$ y $\bb{Z} \times \bb{Z}$, selecciona la afirmación correcta:
    \begin{itemize}
        \item Ambos son DI.
        \item Uno de ellos es DI, pero el otro no.
        \item \underline{Ninguno es DI.}
    \end{itemize}

    \noindent
    \textbf{Justificación}:
    \begin{itemize}
        \item En $\mathbb{Z}_6$, $2\cdot 3=0$.
        \item En $\mathbb{Z}\times \mathbb{Z}$, $(1,0)\cdot (0,1)=(0,0)$.
    \end{itemize}
\end{ejercicio}

\begin{ejercicio}
    En relación a las siguientes proposiciones, referidas a los elementos de un Dominio de Integridad:
    \begin{enumerate}
        \item [(a)] $a\mid b \land a \nmid c \Rightarrow b \nmid b+c$.
        \item [(b)] $a\mid b \land a \nmid c \Rightarrow a \nmid b+c$.
    \end{enumerate}
    Selecciona la afirmación correcta:
    \begin{itemize}
        \item Ambas son verdad.
        \item \underline{Una es verdad y la otra es falsa.}
        \item Ambas son falsas.
    \end{itemize}

    \noindent
    \textbf{Justificación}:
    \begin{itemize}
        \item La primera es cierta: si $b=ax$ y fuese $b+c=ay$, tendríamos que $c=ay-ax=a(x-y)$, así que $a\mid c$, lo que es contradictorio.
        \item La segunda es falsa: por ejemplo, en $\mathbb{Z}$, $2\nmid 1$ y $2\nmid 3$, pero $2\mid 1+3=4$.
    \end{itemize}
\end{ejercicio}

\begin{ejercicio}
    Polinomios de grado uno que son unidades en el anillo de polinomios $\bb{Z}_4[x]$:
    \begin{itemize}
        \item No hay.
        \item \underline{Hay dos.}
        \item Hay infinitos.
    \end{itemize}

    \noindent
    \textbf{Justificación}:
    La tabla de multiplicar en $\mathbb{Z}_4$ es:
    \begin{equation*}
       \begin{array}{c|cccc}
           (\mathbb{Z}_4, \cdot) & 0 & 1 & 2 & 3 \\
           \hline
           0 & 0 & 0 & 0 & 0 \\
           1 & 0 & 1 & 2 & 3 \\
           2 & 0 & 2 & 0 & 2 \\
           3 & 0 & 3 & 2 & 1
       \end{array} 
    \end{equation*}
    Buscamos estudiar el cardinal del conjunto:
    \begin{equation*}
        \left\{p \in U(\mathbb{Z}_4[x]) \mid \deg(p) =1\right\}
    \end{equation*}
    Sea $ax+b \in U(\mathbb{Z}_4[x])$ con $a\neq 0$:
    \begin{align*}
        (ax+b)(ax+b) &= 1 \Longrightarrow {(ax+b)}^{2}=1 \Longrightarrow a^2x + 2abx + b^2 = 1 \\
                     &\Longrightarrow a^2 = 0 \quad\land\quad 2ab = 0 \quad\land\quad b^2 = 1
    \end{align*}
    \begin{equation*}
        \left\{\begin{array}{lll}
                a^2 = 0 & \Longrightarrow & a = 2 \\
                2ab = 0 & \Longrightarrow & 4b = 0 \Longrightarrow 0b = 0 \Longrightarrow 0=0 \\
                b^2 = 1 & \Longrightarrow & b = 1 \quad\lor\quad b = 3
        \end{array}\right.
    \end{equation*}
    Luego:
    \begin{gather*}
        2x+1 \in U(\mathbb{Z}_4[x]) \\
        2x+3 \in U(\mathbb{Z}_4[x])
    \end{gather*}
    Tenemos dos polinomios que verifican la segunda opción. Además, la última no puede ser por ser $\mathbb{Z}_4[x]$ finito.
\end{ejercicio}

\begin{ejercicio}
    En el anillo $\bb{Z}[i]$:
    \begin{itemize}
        \item $3$ es unidad.
        \item \underline{$3$ es irreducible.}
        \item $3$ no es irreducible.
    \end{itemize}

    \noindent
    \textbf{Justificación}:
    \begin{equation*}
        N(3) = 9 \neq \pm 1 \Longrightarrow 3 \notin U(\mathbb{Z}[i])
    \end{equation*}
    Para probar que $3$ es irreducible, supongamos una factorización $3=\alpha \cdot \beta$ con $\alpha, \beta \in \mathbb{Z}[i]\setminus U(\mathbb{Z}[i])$. Entonces:
    \begin{equation*}
        N(3) = N(\alpha)N(\beta) \Longrightarrow 9 = N(\alpha)N(\beta) \quad N(\alpha), N(\beta) \in \mathbb{Z}
    \end{equation*}
    Como $\alpha, \beta \notin U(\mathbb{Z}[i]) \Longrightarrow N(\alpha), N(\beta)\neq \pm 1$
    Como $\alpha, \beta \in \mathbb{Z}[i]$, se tiene que:
    \begin{align*}
        N(\alpha) &= a^2 + b^2 \geq 1 \\
        N(\beta) &= {(a')}^{2} + {(b')}^{2} \geq 1
    \end{align*}
    Por tanto, $N(\alpha), N(\beta) \in \bb{N}$. Además, $9=N(\alpha)N(\beta)\Longrightarrow N(\alpha)=N(\beta)=3$.
    \begin{equation*}
        N(\alpha) = 3 \Longrightarrow a^2 + b^2 = 3
    \end{equation*}
    Pero $\nexists a,b \in \mathbb{Z} \mid a^2 + b^2 = 3$, por lo que 3 es irreducible.
\end{ejercicio}

\begin{ejercicio}
    En el anillo $\bb{Z}[i]$:
    \begin{itemize}
        \item $2$ es unidad.
        \item $2$ es irreducible.
        \item \underline{$2$ no es irreducible.}
    \end{itemize}

    \noindent
    \textbf{Justificación}:
    \begin{equation*}
        N(2) = 4 \neq 1 \Longrightarrow 2 \notin U(\mathbb{Z}[i])
    \end{equation*}
    Para ver que 2 no es irreducible, supongamos una factorización: $2=\alpha \cdot \beta \mid \alpha, \beta \in \mathbb{Z}[i]\setminus U(\mathbb{Z}[i])$.
    \begin{equation*}
        N(2) = N(\alpha \beta) \Longrightarrow 4=N(\alpha)N(\beta) \Longrightarrow N(\alpha) = N(\beta) = 2
    \end{equation*}
    Por ejemplo, $\alpha = \beta = 1+i$
    \begin{equation*}
        -i{(1+i)}^{2} = (1+i^2 + 2i)(-i) = (-i)(1-1+2i) = (-i)2i = -2i^2 = 2
    \end{equation*}
    Luego $2 = -i{(1+i)}^{2}$ es la factorización esencialmente única de 2 $\Longrightarrow$ es reducible.
\end{ejercicio}
\newpage
\resetearcontador

    \subsection{Cuestionario VI}

\begin{ejercicio}
    En relación a las siguientes proposiciones, referidas a elementos cualesquiera de un DI, selecciona las verdaderas:
    \begin{itemize}
        \item $c\mid ab \Longrightarrow c\mid a \lor c\mid b$.
        \item $a\mid c \land b\mid c \Longrightarrow ab\mid c$. 
        \item $c\mid a \lor c\mid b \Longrightarrow c\mid ab$.
    \end{itemize}
\end{ejercicio}

\begin{ejercicio}
    Entre los siguientes DE, selecciona aquellos en los que el máximo común divisor y el mínimo común múltiplo son únicos salvo signo:
    \begin{itemize}
        \item $\mathbb{Z}\left[\sqrt{-2}\right]$.
        \item $\mathbb{Z}\left[\sqrt{3}\right]$. 
        \item $\mathbb{Z}_3[x]$.
    \end{itemize}
\end{ejercicio}

\begin{ejercicio}
    En un DE, tenemos la ecuación diofántica $px+by=1$, donde $p$ es irreducible. Entre las siguientes afirmaciones, selecciona la que es verdad.
    \begin{itemize}
        \item Nunca tiene solución.
        \item Puede tener solución o no, depende de $b$. 
        \item Siempre tiene solución.
    \end{itemize}
\end{ejercicio}

\begin{ejercicio}
    En un DE, tenemos la ecuación diofántica $px+qy=c$, donde $p$ y $q$ son irreducibles no asociados entre sí. Entre las siguientes afirmaciones, selecciona la que es verdad.
    \begin{itemize}
        \item Nunca tiene solución.
        \item Puede tener solución o no, depende de $p$ y de $q$. 
        \item Siempre tiene solución.
    \end{itemize}
\end{ejercicio}

\begin{ejercicio}
    Entre las siguientes proposiciones, referidas a un DE, selecciona las verdaderas.
    \begin{itemize}
        \item Si la ecuación $ax+by=1$ tiene solución, entonces la ecuación $ax+by=c$ tiene solución para todo $c$.
        \item Si la ecuacióin $ax+bb'y=1$ tiene solución, entonces las ecuaciones $ax+by=1$ y $ax+b'y=1$ tienen solución. 
        \item Si las ecuaciones $ax+by=1$ y $ax+b'y=1$ tienen solución, entonces la ecuación $ax+bb'y=1$ tiene solución.
    \end{itemize}
\end{ejercicio}

\newpage
\ % --------------------------------------------------------------------------------
\resetearcontador

\begin{ejercicio}
    En relación a las siguientes proposiciones, referidas a elementos cualesquiera de un DI, selecciona las verdaderas:
    \begin{itemize}
        \item $c\mid ab \Longrightarrow c\mid a \lor c\mid b$.
        \item $a\mid c \land b\mid c \Longrightarrow ab\mid c$. 
        \item \underline{$c\mid a \lor c\mid b \Longrightarrow c\mid ab$.}
    \end{itemize}

    \noindent
    \textbf{Justificación}:
    \begin{itemize}
        \item La primera es falsa, en $\mathbb{Z}$, $6\mid 12 = 4 \cdot 3$ pero $6\nmid 4$.
        \item La segunda es falsa, en $\mathbb{Z}$, $2\mid 6$ pero $2 \cdot 2 \nmid 6$.
        \item La tercera es verdadera. De hecho, basta con que $c$ divida a uno de ellos para que divida al producto:
            \begin{equation*}
                a = ca' \Longrightarrow ab=c(a'b)
            \end{equation*}
    \end{itemize}
\end{ejercicio}

\begin{ejercicio}
    Entre los siguientes DE, selecciona aquellos en los que el máximo común divisor y el mínimo común múltiplo son únicos salvo signo:
    \begin{itemize}
        \item \underline{$\mathbb{Z}\left[\sqrt{-2}\right]$.}
        \item $\mathbb{Z}\left[\sqrt{3}\right]$. 
        \item \underline{$\mathbb{Z}_3[x]$.}
    \end{itemize}

    \noindent
    \textbf{Justificación}:
    Serán aquellos cuyas unidades sean $\pm 1$:
    \begin{itemize}
        \item En $\mathbb{Z}\left[\sqrt{-2}\right]$, $a+b\sqrt{-2}$ es unidad si y sólo si $a^2 + 2b^2 =1$, lo que sólo se verifica si $a=1$ y $b=0$.
        \item En $\mathbb{Z}\left[\sqrt{3}\right]$, $a+b\sqrt{3}$ es unidad si y sólo si $a^2 - 3b^2 =\pm 1$, lo que verifica por ejemplo $2+\sqrt{3}\neq \pm 1$, luego aquí el mcd y el mcm no son únicos salvo signo.
        \item En $\mathbb{Z}_3[x]$:
            \begin{equation*}
                U\left(\mathbb{Z}_3[x]\right) = U\left(\mathbb{Z}_3\right) = \{1,2\} = \{ 1, -1 \} = \{ \pm 1\}
            \end{equation*}
    \end{itemize}
\end{ejercicio}

\begin{ejercicio}
    En un DE, tenemos la ecuación diofántica $px+by=1$, donde $p$ es irreducible. Entre las siguientes afirmaciones, selecciona la que es verdad.
    \begin{itemize}
        \item Nunca tiene solución.
        \item \underline{Puede tener solución o no, depende de $b$.} 
        \item Siempre tiene solución.
    \end{itemize}

    \noindent
    \textbf{Justificación}:
    La ecuación tendrá solución $\Longleftrightarrow \text{mcd}(p,b)\mid 1 \Longleftrightarrow \text{mcd}(p,b)=1$. Como $p$ es irreducible, equivale a que $p \nmid b$, luego puede tener solución o no, dependiendo de $b$:
    \begin{itemize}
        \item Para $b=1$ sí tiene solución.
        \item Pero para $b=2p \Longrightarrow \text{mcd}(p,2p)=p\neq 1$ no tiene solución.
    \end{itemize}
\end{ejercicio}

\begin{ejercicio}
    En un DE, tenemos la ecuación diofántica $px+qy=c$, donde $p$ y $q$ son irreducibles no asociados entre sí. Entre las siguientes afirmaciones, selecciona la que es verdad.
    \begin{itemize}
        \item Nunca tiene solución.
        \item Puede tener solución o no, depende de $p$ y de $q$. 
        \item \underline{Siempre tiene solución.}
    \end{itemize}

    \noindent
    \textbf{Justificación}:
    La ecuación tendrá solución $\Longleftrightarrow \text{mcd}(p,q)\mid c$. Como $p$ y $q$ son irreducibles no asociados, tenemos que $\text{mcd}(p,q)=1$ y como $1\mid c$ $\forall c \in A$, la ecuación siempre tendrá solución.
\end{ejercicio}

\begin{ejercicio}
    Entre las siguientes proposiciones, referidas a un DE, selecciona las verdaderas.
    \begin{itemize}
    \item \underline{Si la ecuación $ax+by=1$ tiene solución, entonces la ecuación $ax+by=c$} \newline
        \underline{ tiene solución para todo $c$.}
\item \underline{Si la ecuacióin $ax+bb'y=1$ tiene solución, entonces las ecuaciones $ax+by=1$}
    \underline{ y $ax+b'y=1$ tienen solución.} 
\item \underline{Si las ecuaciones $ax+by=1$ y $ax+b'y=1$ tienen solución, entonces la }\newline
    \underline{ecuación $ax+bb'y=1$ tiene solución.}
    \end{itemize}

    \noindent
    \textbf{Justificación}:
    \begin{itemize}
        \item Sea $(x_0,y_0)$ solución de $ax+by=1 \Longrightarrow (cx_0, cy_0)$ es solución de $ax+by=c$.
        \item Sea $(x_0,y_0)$ solución de $ax+bb'y=1 \Longrightarrow (x_0, y_0b')$ es solución de $ax+by=1$ y $(x_0, y_0b)$ es solución de $ax+b'y=1$.
        \item 
            \begin{equation*}
                \left.
                    \begin{array}{lcr}
                        ax+by=1 \text{\ tiene\ solución} & \Longrightarrow & \text{mcd}(a,b)=1 \\
                        ax+b'y=1 \text{\ tiene\ solución} & \Longrightarrow & \text{mcd}(a,b')=1
                    \end{array}
                \right\} \Longrightarrow \text{mcd}(a,bb')=1
            \end{equation*}
            Luego $ax+bb'y=1$ tiene solución. 
    \end{itemize}
\end{ejercicio}

\newpage
\resetearcontador

    \section{Cuestionario VII}

\begin{ejercicio}
    En relación a las siguientes proposiciones sobre elementos de un DE, selecciona las verdaderas:
    \begin{itemize}
        \item Si $\text{mcd}(a,b)=1$, entonces $\text{mcd}(a,b^n)=1$ para todo $n \in \mathbb{N}$.
        \item Si $a \equiv a'\mod(b)$, entonces $\text{mcd}(a,b)=\text{mcd}(a',b)$.
        \item Si $a\equiv a'\mod(b)$, entonces $\text{mcm}(a,b)=\text{mcm}(a',b)$.
    \end{itemize}
\end{ejercicio}

\begin{ejercicio}
    Entre las siguientes ecuaciones en congruencias, selecciona las que tienen solución.
    \begin{itemize}
        \item En $\mathbb{Z}$, $6x\equiv 10 \mod (45)$.
        \item En $\mathbb{Z}$, $100x\equiv 20\mod (15)$.
        \item En $\mathbb{Z}[i]$, $(2+2i)x\equiv 5\mod(3-i)$.
    \end{itemize}
\end{ejercicio}

\begin{ejercicio}
    Entre las siguientes afirmaciones relativas a ecuaciones en el anillo $\mathbb{Z}_{64}$, selecciona las que son verdad.
    \begin{itemize}
        \item $12x=28$ tiene $4$ soluciones.
        \item $14x=28$ tiene $4$ soluciones.
        \item $12x=30$ tiene $4$ soluciones.
    \end{itemize}
\end{ejercicio}

\begin{ejercicio}
    Entre las siguientes proposiciones, selecciona las verdaderas.
    \begin{itemize}
        \item El anillo $\mathbb{Z}_{900}$ tiene 240 unidades.
        \item $14^{20}\equiv 1\mod (33)$.
        \item $3^{16}=3$ en $\mathbb{Z}_{16}$.
    \end{itemize}
\end{ejercicio}

\begin{ejercicio}
    Sea $p$ un número primo y considérese la congruencia $ax\equiv 1\mod (p^2)$. En relación a las siguientes proposiciones, selecciona las verdaderas:
    \begin{itemize}
        \item No tiene solución, pues $p^2$ no es primo.
        \item Tiene solución si y sólo si la congruencia $ax\equiv 1\mod (p)$ tiene solución.
        \item Tiene solución salvo que $a$ sea múltiplo de $p^2$.
    \end{itemize}
\end{ejercicio}

\newpage
\ % --------------------------------------------------------------------------------
\resetearcontador

\begin{ejercicio}
    En relación a las siguientes proposiciones sobre elementos de un DE, selecciona las verdaderas:
    \begin{itemize}
        \item \underline{Si $\text{mcd}(a,b)=1$, entonces $\text{mcd}(a,b^n)=1$ para todo $n \in \mathbb{N}$.}
        \item \underline{Si $a \equiv a'\mod(b)$, entonces $\text{mcd}(a,b)=\text{mcd}(a',b)$.}
        \item Si $a\equiv a'\mod(b)$, entonces $\text{mcm}(a,b)=\text{mcm}(a',b)$.
    \end{itemize}

    \noindent
    \textbf{Justificación}:
    \begin{itemize}
        \item Es cierto, lo probamos por inducción:
            \begin{description}
                \item [Para $n=0$:] 
                    $\text{mcd}(a,b^0) = \text{mcd}(a,1)=1$, cierto.
                \item [Para $n=1$:] 
                    $\text{mcd}(a,b)=1$, cierto.
                \item [Supuesto cierto para $n-1$, lo vemos para $n$:] 
                    \begin{equation*}
                        \left.\begin{array}{r}
                            \text{mcd}(a,b)=1 \\
                            \text{mcd}(a,b^{n-1}) = 1
                    \end{array}\right\} \text{mcd}(a,b^n) = \text{mcd}(a,b^{n-1}b) = 1
                    \end{equation*}
            \end{description}
        \item Es cierto, sea $A$ el DE:
            \begin{align*}
                a\equiv a'\mod(b) &\Longrightarrow \exists q\in A \mid a-a' = qb \\
                                  &\Longrightarrow  a'=a-qb
            \end{align*}
            \begin{equation*}
                \text{mcd}(a,b) = \text{mcd}(a-qb,b) = \text{mcd}(a',b)
            \end{equation*}
        \item Es falso, por ejemplo en $\mathbb{Z}$, sean $a=6$, $a' = 2$, $b = 4$
            \begin{gather*}
                6\equiv 2\mod (4) \\
                \text{mcm}(6,4) = 12 \neq 4 = \text{mcm}(2,4)
            \end{gather*}
    \end{itemize}
\end{ejercicio}

\begin{ejercicio}
    Entre las siguientes ecuaciones en congruencias, selecciona las que tienen solución.
    \begin{itemize}
        \item En $\mathbb{Z}$, $6x\equiv 10 \mod (45)$.
        \item \underline{En $\mathbb{Z}$, $100x\equiv 20\mod (15)$.}
        \item En $\mathbb{Z}[i]$, $(2+2i)x\equiv 5\mod(3-i)$.
    \end{itemize}

    \noindent
    \textbf{Justificación}:
    \begin{itemize}
        \item $\text{mcd}(6,45)=3$, como $3\nmid 10 \Longrightarrow$ no tiene solución.
        \item $\text{mcd}(100,15)=5$, como $5\mid 20 \Longrightarrow $ tiene solución:
            \begin{equation*}
                20x\equiv 4\mod (3) \qquad \text{mcd}(20,3)=1
            \end{equation*}
            \begin{align*}
                1 = 20(-1)+7\cdot 3 &\Longrightarrow 20\cdot 1=-1\mod (3) \\
                                    &\Longrightarrow 20(-4)\equiv 4\mod (3)
            \end{align*}
            \begin{gather*}
                x_0 = -4 \text{\ es\ solución\ particular} \\
                x_0 = 2 \text{\ es\ solución\ óptima} \\
                x_0 = 2+3k\quad k\in \mathbb{Z}
            \end{gather*}
        \item Calculamos $\text{mcd}(2+2i, 3-i)$ en $\mathbb{Q}[i]$:
            \begin{equation*}
                \dfrac{3-i}{2+2i} = \dfrac{(2-2i)(3-i)}{8} = \dfrac{6-2i-6i-2}{8}=\dfrac{4}{8}-\dfrac{8i}{8} = \dfrac{1}{2}-i
            \end{equation*}
            Tenemos $q=i$, $r = 1+i$
            \begin{equation*}
                \begin{array}{rcl}
                    r_i & u_i & v_i \\
                    3-i & 1 & 0 \\
                    2+2i & 0 & 1 \\
                    1+i & 1 & -i 
                \end{array}
            \end{equation*}
            Existe solución $\Longleftrightarrow 1+i\mid 5$, pero como $1+i\nmid 5$, no existe solución.
    \end{itemize}
\end{ejercicio}

\begin{ejercicio}
    Entre las siguientes afirmaciones relativas a ecuaciones en el anillo $\mathbb{Z}_{64}$, selecciona las que son verdad.
    \begin{itemize}
        \item \underline{$12x=28$ tiene $4$ soluciones.}
        \item $14x=28$ tiene $4$ soluciones.
        \item $12x=30$ tiene $4$ soluciones.
    \end{itemize}

    \noindent
    \textbf{Justificación}:
    \begin{itemize}
        \item 
            \begin{align*}
                12x &\equiv 28\mod(64)\\
                6x &\equiv 14\mod(32) \\
                3x &\equiv 7\mod(16)
            \end{align*}
            Como $\text{mcd}(16,3)=1$, tiene solución.
            \begin{align*}
                1 = 16\cdot 1 + 3(-5) &\Longrightarrow 3\cdot 5\equiv -1\mod(16) \\
                                      &\Longrightarrow 3\cdot 5(-7)\equiv 7\mod(16)
            \end{align*}
            \begin{gather*}
                5(-7) = -35 \text{\ es\ solución\ particular} \\
                x_0 = 13 \text{\ es\ solución\ óptima} \\
                x = 13 + 16k\quad k\in \mathbb{Z}
            \end{gather*}
            Por tanto:
            \begin{equation*}
                \begin{array}{ll}
                    x_1 = 13 & x_2 = 29 \\
                    x_3 = 45 & x_4 = 61
                \end{array}
            \end{equation*}
            Tiene 4 soluciones.
        \item 
            \begin{align*}
                14x &\equiv 28\mod(64) \\
                7x &\equiv 14\mod(32)
            \end{align*}
            $\text{mcd}(7,32)=1$, tiene solución.
            \begin{align*}
                1 = 32\cdot 2+7(-9) &\Longrightarrow 7\cdot 9\equiv -1\mod (32) \\
                                    &\Longrightarrow 7\cdot 9(-14)\equiv 14\mod(32)
            \end{align*}
            \begin{gather*}
                x_0 = 9(-14) = -126 \text{\ es\ solución\ particular} \\
                y_0 = 2 \text{\ es\ solución\ óptima} \\
                x = 2+23k\quad k\in \mathbb{Z}
            \end{gather*}
            Por tanto:
            \begin{gather*}
                x_1 = 2 \\
                x_2 = 34
            \end{gather*}
            No tiene 4 soluciones, es falso.
            
        \item 
            \begin{align*}
                12x &\equiv 30\mod(64) \\
                6x &\equiv 15\mod(32)
            \end{align*}
            \begin{equation*}
                \text{mcd}(6,32) = 2 \nmid 15 \Longrightarrow \text{\ no\ tiene\ solución}
            \end{equation*}
            Es falso.
    \end{itemize}
\end{ejercicio}

\begin{ejercicio}
    Entre las siguientes proposiciones, selecciona las verdaderas.
    \begin{itemize}
        \item \underline{El anillo $\mathbb{Z}_{900}$ tiene 240 unidades.}
        \item \underline{$14^{20}\equiv 1\mod (33)$.}
        \item $3^{16}=3$ en $\mathbb{Z}_{16}$.
    \end{itemize}

    \noindent
    \textbf{Justificación}:
    \begin{itemize}
        \item 
            \begin{equation*}
                |U(\mathbb{Z}_{900})| = \varphi(900) = \varphi(3^2 \cdot 2^2 \cdot 5^2) = 3\cdot 2\cdot 5\cdot 2\cdot 1\cdot 4 = 240
            \end{equation*}
        \item 
            \begin{equation*}
                \left.\begin{array}{r}
                    \varphi(33) = \varphi(3\cdot 11) = 2\cdot 10 = 20 \\
                    \text{mcd}(14,33) = 1
            \end{array}\right\} \mathop{\Longrightarrow}^{\text{Fermat}} 14^{20}\equiv 1\mod(33)
            \end{equation*}
        \item 
            \begin{align*}
                \left.\begin{array}{r}
                    \varphi(16) = \varphi(2^4) = 2^3 \cdot 1 =8 \\
                    \text{mcd}(3,16) = 1
            \end{array}\right\} &\Longrightarrow 3^8\equiv 1\mod (16) \Longrightarrow 3^{16}\equiv 1\mod(16) \\
            &\Longrightarrow 3^{16}\not\equiv 3\mod(16)
            \end{align*}
    \end{itemize}
\end{ejercicio}

\newpage
\begin{ejercicio}
    Sea $p$ un número primo y considérese la congruencia $ax\equiv 1\mod (p^2)$. En relación a las siguientes proposiciones, selecciona las verdaderas:
    \begin{itemize}
        \item No tiene solución, pues $p^2$ no es primo.
        \item \underline{Tiene solución si y sólo si la congruencia $ax\equiv 1\mod (p)$ tiene solución.}
        \item Tiene solución salvo que $a$ sea múltiplo de $p^2$.
    \end{itemize}

    \noindent
    \textbf{Justificación}:
    \begin{align*}
        \text{La\ equación\ tiene\ solución\ } &\Longleftrightarrow \text{mcd}(a,p^2) \mid 1 \Longleftrightarrow \text{mcd}(a,p^2) = 1 \\
                                              &\Longleftrightarrow \text{mcd}(a,p)= 1 \Longleftrightarrow ax\equiv 1\mod (p) \text{\ tiene\ solución}
    \end{align*}
    Luego la segunda opción es verdadera. Estudiamos ahora la tercera, si $a = kp^2$ con $k \in A \Longrightarrow \text{mcd}(a,p^2) = p^2$ por lo que es cierto que no tiene solución. Sin embargo, si $p^2$ es múltiplo de $a \Longrightarrow \text{mcd}(a,p^2) = a$, por lo que tampoco tiene solución.
    Luego la tercera es falsa, al existir más casos en los que no tiene solución.
\end{ejercicio}

\newpage
\resetearcontador

    \subsection{Cuestionario VIII}

\begin{ejercicio}
    En el anillo $\mathbb{Z}[i]$, selecciona las afirmaciones verdaderas:
    \begin{itemize}
        \item $2+ i$ y $2-i$ son unidades.
        \item $2+i$ y $2-i$ son asociados.
        \item $2+i$ y $2-i$ son irreducibles.
    \end{itemize}
\end{ejercicio}

\begin{ejercicio}
    Entre las siguientes afirmaciones, selecciona las afirmaciones verdaderas:
    \begin{itemize}
        \item En el anillo $\mathbb{Z}\left[\sqrt{2}\right]$, los número $2+\sqrt{2}$ y $2-\sqrt{2}$ son asociados.
        \item En el anillo $\mathbb{Z}\left[\sqrt{2}\right]$, los número $2+\sqrt{2}$ y $2-\sqrt{2}$ son primos.
        \item En el anillo $\mathbb{Z}\left[\sqrt{2}\right]$, el número 2 no es primo.
    \end{itemize}
\end{ejercicio}

\begin{ejercicio}
    Entre las siguientes afirmaciones, selecciona las correctas.
    \begin{itemize}
        \item En $\mathbb{Z}[x]$, todo polinomio de grado 1 es irreducible.
        \item En $\mathbb{Z}[x]$, todo polinomio mónico de grado menor o igual que 3 y sin raíces en $\mathbb{Z}$ es irreducible.
        \item Todo polinomio de grado mayor o igual que 1 en $\mathbb{Q}[x]$ es asociado a un primitivo de $\mathbb{Z}[x]$.
    \end{itemize}
\end{ejercicio}

\begin{ejercicio}
    Entre las siguientes afirmaciones relativas a un polinomio $f\in \mathbb{Z}[x]$, selecciona las que son verdad:
    \begin{itemize}
        \item Si el reducido $R_p(f)$ es irreducible en $\mathbb{Z}_p[x]$, entonces $f$ es irreducible.
        \item Si $f$ es mónico y el reducido $R_p(f)$ es irreducible en $\mathbb{Z}_p[x]$, entonces $f$ es irreducible.
        \item Si $f$ es primitivo y el reducido $R_p(f)$ es irreducible en $\mathbb{Z}_p[x]$, entonces $f$ es irreducible.
    \end{itemize}
\end{ejercicio}

\begin{ejercicio*}
    Entre las siguientes afirmaciones relativas a un polinomo mónimo $f\in \mathbb{Z}[x]$, selecciona las que son verdad:
    \begin{itemize}
        \item Si $f$ no tiene raíces en $\mathbb{Z}$ y para un primo entero $p\geq 2$, el reducido $R_p(f)$ factoriza en irreducibles $\mathbb{Z}_p[x]$ en la forma $R_p(f) = f_1 \cdot f_2$ con $\deg(f_1)=1$, entonces $f$ es irreducible en $\mathbb{Z}[x]$.
        \item Si para un entero primo $p\geq 2$, el reducido $R_p(f)$ factoriza en irreducibles $\mathbb{Z}_p[x]$ en la forma $R_p(f) = f_1^2$ con $\deg(f_1)=3$ y para un entero primo $q\geq 2$, el reducido $R_q(f)$ factoriza en irredcuibles $\mathbb{Z}_q[x]$ en la forma $R_q(f)=g_1g_2g_3$ con $\deg(g_1)=1=\deg(g_2)$ y $\deg(g_3)=4$, entonces $f$ es irreducible.
        \item Si para un entero primo $p\geq 2$, el reducido $R_p(f)$ factoriza en irreducibles $\mathbb{Z}_p[x]$ en la forma $R_p(f)=f_1^2$ con $\deg(f_1)=2$ y para un entero primo $q\geq 2$, el reducido $R_q(f)$ factoriza en irreducibles $\mathbb{Z}_q[x]$ en la forma $R_q(f)=g_1g_2g_3g_4$ con $\deg(g_1)=1$, entonces $f$ es irreducible.
    \end{itemize}
\end{ejercicio*}

\newpage
\ % --------------------------------------------------------------------------------
\resetearcontador


\newpage
\resetearcontador

\end{document}
