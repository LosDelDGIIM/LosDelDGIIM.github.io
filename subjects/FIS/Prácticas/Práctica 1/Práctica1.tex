\documentclass[12pt]{article}

% Idioma y codificación
\usepackage[spanish, es-tabla, es-notilde]{babel}       %es-tabla para que se titule "Tabla"
\usepackage[utf8]{inputenc}

% Márgenes
\usepackage[a4paper,top=3cm,bottom=2.5cm,left=3cm,right=3cm]{geometry}

% Comentarios de bloque
\usepackage{verbatim}

% Paquetes de links
\usepackage[hidelinks]{hyperref}    % Permite enlaces
\usepackage{url}                    % redirecciona a la web

% Más opciones para enumeraciones
\usepackage{enumitem}

% Personalizar la portada
\usepackage{titling}

% Paquetes de tablas
\usepackage{multirow}

% Para añadir el símbolo de euro
\usepackage{eurosym}


%------------------------------------------------------------------------

%Paquetes de figuras
\usepackage{caption}
\usepackage{subcaption} % Figuras al lado de otras
\usepackage{float}      % Poner figuras en el sitio indicado H.


% Paquetes de imágenes
\usepackage{graphicx}       % Paquete para añadir imágenes
\usepackage{transparent}    % Para manejar la opacidad de las figuras

% Paquete para usar colores
\usepackage[dvipsnames, table, xcdraw]{xcolor}
\usepackage{pagecolor}      % Para cambiar el color de la página

% Habilita tamaños de fuente mayores
\usepackage{fix-cm}

% Para los gráficos
\usepackage{tikz}
\usepackage{forest}

% Para poder situar los nodos en los grafos
\usetikzlibrary{positioning}


%------------------------------------------------------------------------

% Paquetes de matemáticas
\usepackage{mathtools, amsfonts, amssymb, mathrsfs}
\usepackage[makeroom]{cancel}     % Simplificar tachando
\usepackage{polynom}    % Divisiones y Ruffini
\usepackage{units} % Para poner fracciones diagonales con \nicefrac

\usepackage{pgfplots}   %Representar funciones
\pgfplotsset{compat=1.18}  % Versión 1.18

\usepackage{tikz-cd}    % Para usar diagramas de composiciones
\usetikzlibrary{calc}   % Para usar cálculo de coordenadas en tikz

%Definición de teoremas, etc.
\usepackage{amsthm}
%\swapnumbers   % Intercambia la posición del texto y de la numeración

\theoremstyle{plain}

\makeatletter
\@ifclassloaded{article}{
  \newtheorem{teo}{Teorema}[section]
}{
  \newtheorem{teo}{Teorema}[chapter]  % Se resetea en cada chapter
}
\makeatother

\newtheorem{coro}{Corolario}[teo]           % Se resetea en cada teorema
\newtheorem{prop}[teo]{Proposición}         % Usa el mismo contador que teorema
\newtheorem{lema}[teo]{Lema}                % Usa el mismo contador que teorema
\newtheorem*{lema*}{Lema}

\theoremstyle{remark}
\newtheorem*{observacion}{Observación}

\theoremstyle{definition}

\makeatletter
\@ifclassloaded{article}{
  \newtheorem{definicion}{Definición} [section]     % Se resetea en cada chapter
}{
  \newtheorem{definicion}{Definición} [chapter]     % Se resetea en cada chapter
}
\makeatother

\newtheorem*{notacion}{Notación}
\newtheorem*{ejemplo}{Ejemplo}
\newtheorem*{ejercicio*}{Ejercicio}             % No numerado
\newtheorem{ejercicio}{Ejercicio} [section]     % Se resetea en cada section


% Modificar el formato de la numeración del teorema "ejercicio"
\renewcommand{\theejercicio}{%
  \ifnum\value{section}=0 % Si no se ha iniciado ninguna sección
    \arabic{ejercicio}% Solo mostrar el número de ejercicio
  \else
    \thesection.\arabic{ejercicio}% Mostrar número de sección y número de ejercicio
  \fi
}


% \renewcommand\qedsymbol{$\blacksquare$}         % Cambiar símbolo QED
%------------------------------------------------------------------------

% Paquetes para encabezados
\usepackage{fancyhdr}
\pagestyle{fancy}
\fancyhf{}

\newcommand{\helv}{ % Modificación tamaño de letra
\fontfamily{}\fontsize{12}{12}\selectfont}
\setlength{\headheight}{15pt} % Amplía el tamaño del índice


%\usepackage{lastpage}   % Referenciar última pag   \pageref{LastPage}
%\fancyfoot[C]{%
%  \begin{minipage}{\textwidth}
%    \centering
%    ~\\
%    \thepage\\
%    \href{https://losdeldgiim.github.io/}{\texttt{\footnotesize losdeldgiim.github.io}}
%  \end{minipage}
%}
\fancyfoot[C]{\thepage}
\fancyfoot[R]{\href{https://losdeldgiim.github.io/}{\texttt{\footnotesize losdeldgiim.github.io}}}

%------------------------------------------------------------------------

% Conseguir que no ponga "Capítulo 1". Sino solo "1."
\makeatletter
\@ifclassloaded{book}{
  \renewcommand{\chaptermark}[1]{\markboth{\thechapter.\ #1}{}} % En el encabezado
    
  \renewcommand{\@makechapterhead}[1]{%
  \vspace*{50\p@}%
  {\parindent \z@ \raggedright \normalfont
    \ifnum \c@secnumdepth >\m@ne
      \huge\bfseries \thechapter.\hspace{1em}\ignorespaces
    \fi
    \interlinepenalty\@M
    \Huge \bfseries #1\par\nobreak
    \vskip 40\p@
  }}
}
\makeatother

%------------------------------------------------------------------------
% Paquetes de cógido
\usepackage{minted}
\renewcommand\listingscaption{Código fuente}

\usepackage{fancyvrb}
% Personaliza el tamaño de los números de línea
\renewcommand{\theFancyVerbLine}{\small\arabic{FancyVerbLine}}

% Estilo para C++
\newminted{cpp}{
    frame=lines,
    framesep=2mm,
    baselinestretch=1.2,
    linenos,
    escapeinside=||
}

% para minted
\definecolor{LightGray}{rgb}{0.95,0.95,0.92}
\setminted{
    linenos=true,
    stepnumber=5,
    numberfirstline=true,
    autogobble,
    breaklines=true,
    breakautoindent=true,
    breaksymbolleft=,
    breaksymbolright=,
    breaksymbolindentleft=0pt,
    breaksymbolindentright=0pt,
    breaksymbolsepleft=0pt,
    breaksymbolsepright=0pt,
    fontsize=\footnotesize,
    bgcolor=LightGray,
    numbersep=10pt
}


\usepackage{listings} % Para incluir código desde un archivo

\renewcommand\lstlistingname{Código Fuente}
\renewcommand\lstlistlistingname{Índice de Códigos Fuente}

% Definir colores
\definecolor{vscodepurple}{rgb}{0.5,0,0.5}
\definecolor{vscodeblue}{rgb}{0,0,0.8}
\definecolor{vscodegreen}{rgb}{0,0.5,0}
\definecolor{vscodegray}{rgb}{0.5,0.5,0.5}
\definecolor{vscodebackground}{rgb}{0.97,0.97,0.97}
\definecolor{vscodelightgray}{rgb}{0.9,0.9,0.9}

% Configuración para el estilo de C similar a VSCode
\lstdefinestyle{vscode_C}{
  backgroundcolor=\color{vscodebackground},
  commentstyle=\color{vscodegreen},
  keywordstyle=\color{vscodeblue},
  numberstyle=\tiny\color{vscodegray},
  stringstyle=\color{vscodepurple},
  basicstyle=\scriptsize\ttfamily,
  breakatwhitespace=false,
  breaklines=true,
  captionpos=b,
  keepspaces=true,
  numbers=left,
  numbersep=5pt,
  showspaces=false,
  showstringspaces=false,
  showtabs=false,
  tabsize=2,
  frame=tb,
  framerule=0pt,
  aboveskip=10pt,
  belowskip=10pt,
  xleftmargin=10pt,
  xrightmargin=10pt,
  framexleftmargin=10pt,
  framexrightmargin=10pt,
  framesep=0pt,
  rulecolor=\color{vscodelightgray},
  backgroundcolor=\color{vscodebackground},
}

%------------------------------------------------------------------------

% Comandos definidos
\newcommand{\bb}[1]{\mathbb{#1}}
\newcommand{\cc}[1]{\mathcal{#1}}

% I prefer the slanted \leq
\let\oldleq\leq % save them in case they're every wanted
\let\oldgeq\geq
\renewcommand{\leq}{\leqslant}
\renewcommand{\geq}{\geqslant}

% Si y solo si
\newcommand{\sii}{\iff}

% MCD y MCM
\DeclareMathOperator{\mcd}{mcd}
\DeclareMathOperator{\mcm}{mcm}

% Signo
\DeclareMathOperator{\sgn}{sgn}

% Letras griegas
\newcommand{\eps}{\epsilon}
\newcommand{\veps}{\varepsilon}
\newcommand{\lm}{\lambda}

\newcommand{\ol}{\overline}
\newcommand{\ul}{\underline}
\newcommand{\wt}{\widetilde}
\newcommand{\wh}{\widehat}

\let\oldvec\vec
\renewcommand{\vec}{\overrightarrow}

% Derivadas parciales
\newcommand{\del}[2]{\frac{\partial #1}{\partial #2}}
\newcommand{\Del}[3]{\frac{\partial^{#1} #2}{\partial #3^{#1}}}
\newcommand{\deld}[2]{\dfrac{\partial #1}{\partial #2}}
\newcommand{\Deld}[3]{\dfrac{\partial^{#1} #2}{\partial #3^{#1}}}


\newcommand{\AstIg}{\stackrel{(\ast)}{=}}
\newcommand{\Hop}{\stackrel{L'H\hat{o}pital}{=}}

\newcommand{\red}[1]{{\color{red}#1}} % Para integrales, destacar los cambios.

% Método de integración
\newcommand{\MetInt}[2]{
    \left[\begin{array}{c}
        #1 \\ #2
    \end{array}\right]
}

% Declarar aplicaciones
% 1. Nombre aplicación
% 2. Dominio
% 3. Codominio
% 4. Variable
% 5. Imagen de la variable
\newcommand{\Func}[5]{
    \begin{equation*}
        \begin{array}{rrll}
            \displaystyle #1:& \displaystyle  #2 & \longrightarrow & \displaystyle  #3\\
               & \displaystyle  #4 & \longmapsto & \displaystyle  #5
        \end{array}
    \end{equation*}
}

%------------------------------------------------------------------------


\title{Sistema de Gestión de Reservas de Actividades Turísticas\\\textbf{GERAT}}
\author{Pablo Linari Pérez\\Manuel Gómez Rubio\\Arturo Olivares Martos\\Joaquín Avilés de la Fuente}
\date{\today}

\usepackage[toc]{glossaries}
\makeglossaries

\newglossaryentry{ActividadTuristica}
{
    name=Actividad Turística,
    description={Conjunto de experiencias recreativas, culturales o de naturaleza realizadas tanto por visitantes como por residentes, incluyendo la exploración de monumentos, sitios históricos, paisajes naturales y actividades al aire libre}
}

\newglossaryentry{Reserva}
{
    name=Reserva,
    description={Acción realizada por el cliente para asegurar la disponibilidad de una actividad turística concreta}
}

\newglossaryentry{Actividad Ofertada}
{
    name=Actividad Ofertada,
    description={Actividad turística que se encuentra disponible para ser reservada por los clientes}
}

\newglossaryentry{Actividad Pasada}
{
    name=Actividad Pasada,
    description={Actividad turística que ya ha sido realizada}
}

\newglossaryentry{Guía Turístico}
{
    name=Guía Turístico,
    description={Persona encargada de acompañar y asistir a los visitantes en sus desplazamientos, proporcionando información y explicaciones sobre los lugares visitados}
}

\newglossaryentry{Reseña}
{
    name=Reseña,
    description={Opinión escrita por un cliente sobre una actividad turística que ha realizado}
}

\newglossaryentry{Catalogo Ofertado}
{
    name=Catálogo Ofertado,
    description={Listado de actividades turísticas disponibles para los clientes, incluyendo detalles sobre horarios, precios y requisitos}
}

\newglossaryentry{Cancelar Reserva}
{
    name=Cancelar Reserva,
    description={Acción de anular una reserva previamente realizada, impidiendo la participación en la actividad turística}
}

\newglossaryentry{Anular Actividad Turistica}
{
    name=Anular Actividad Turística,
    description={Suspensión de una actividad turística debido a condiciones climáticas adversas o la indisponibilidad de un guía turístico}
}

\newglossaryentry{Resguardo Reserva}
{
    name=Resguardo de Reserva,
    description={Documento o comprobante que certifica la reserva de una actividad turística}
}

\newglossaryentry{Parajes Naturales}
{
    name=Parajes Naturales,
    description={Espacios naturales con valor ecológico, paisajístico o cultural que pueden ser visitados con fines turísticos}
}

\newglossaryentry{Monumentos}
{
    name=Monumentos,
    description={Estructuras o sitios de valor histórico, cultural o arquitectónico que forman parte del patrimonio de un destino turístico}
}

\newglossaryentry{PeriodoCancelacion}
{
    name=Periodo de Cancelación,
    description={Plazo determinado en el cual se permite cancelar una reserva de actividad turística sin penalización o con condiciones específicas}
}

\newglossaryentry{Plataforma de pago segura}
{
    name=Plataforma de pago segura,
    description={Medio de pago online que nos permite pagar sin comprometer nuestros datos}
}



\newglossaryentry{Accesibilidad}
{
    name=Accesibilidad,
    description={Capacidad de la plataforma para ser utilizada por personas con diferentes necesidades, incluyendo compatibilidad con múltiples dispositivos y navegadores}
}

\begin{document}

    \begin{comment}
        A la hora de compilar para que se vea el glosario, es más complejo. Lo hace Arturo.
        El PDF no se subirá para que no haya conflictos. Arturo irá subiendo las versiones en _Visible
        Para trabajar, solo toco mi correspondiente archivo, y compilo Practica1.tex como si fuese normal. Y podré ver el PDF
    \end{comment}

    \fancyhead[L]{\helv \nouppercase{Fundamentos de Ingeniería del Software}}
    \fancyhead[R]{\helv \nouppercase{\leftmark}}

    \maketitle
    \begin{abstract}
        El presente proyecto consiste en el desarrollo de un sistema informático de gestión de reservas de actividades turísticas, denominado GERAT. Este documento consiste por tanto en la especificación de los requisitos del mencionado sistema, así como la descripción de los implicados en el mismo.
    \end{abstract}

    \tableofcontents

    \section{Objetivos}

Como se ha descrito en el breve resumen inicial, el presente documento muestra la especificación de requisistos de un sistema informático de gestión de reservas de actividades turísticas, denominado GERAT. Dicho sistema permite a las entidades organizadoras de las actividades publicar nuevas propuestas y gestionar las reservas de los clientes. A su vez, los clientes pueden visualizar las actividades ofertadas, reservar plazas y gestionar sus reservas.
        
GERAT se presenta como una herramienta que facilita la organización de actividades turísticas y la gestión de las reservas de los clientes. Esta plataforma permitirá así unificar la información de las actividades y las reservas que se dan en cierta localidad, provincia o país, facilitando así la gestión de las mismas y repercutiendo positivamente en la experiencia de los clientes y, por tanto, en la actividad turística y económica de la zona. Asimismo, puesto que el número de reservas aumentará, se espera que la empresas organizadoras de las actividades pueda aumentar su beneficio.

Una vez descrito el dominio del problema y la descripción general del sistema, detallamos a continuación los principales objetivos que se pretenden alcanzar con el desarrollo de GERAT.

\begin{enumerate}[label={\color{red}OBJ-\arabic{enumi}.}]
    \item El sistema deberá almacenar y gestionar la información relativa a las actividades ofertadas por las empresas organizadoras.
    \item El sistema deberá permitir a los clientes visualizar las actividades ofertadas y reservar (o cancelar) plazas en las mismas, tanto de forma individual como en grupo.
    \item El sistema automatizará los cobros de las reservas realizadas por los clientes, permitiendo el pago mediante tarjeta de crédito a través de una pasarela de pago segura.
\end{enumerate}


\begin{comment}
    El glosario, o a Glosario.tex (asegurandose de que nadie toca), o aquí en sucio y luego se pasará
\end{comment}
    \section{Descripción de los implicados}

%%%%%%%% JOAQUIN %%%%%%%%

En esta sección se hará una descripción de los implicados en el sistema de gestión de reservas turísticas GERAT, especificando sus roles y las responsabilidades que tienen en el mismo. Los usuarios directos del producto son los clientes y las empresas que proporcionan actividades turísticas, por los distintos parajes naturales, monumentos, ciudades, etc.

Veámos a continuación un resumen de los distintos implicados, indicando su descripción y sus roles de forma breve.

\begin{table}
    \makebox[\textwidth][c]{ % Esto centra la tabla en la página
    \begin{tabular}{|p{2.5cm}|p{5.5cm}|p{3.1cm}|p{5.5cm}|}
    \hline
    \multicolumn{1}{|c|}{\cellcolor[HTML]{FFCCC9}\textbf{Nombre}} & 
    \multicolumn{1}{c|}{\cellcolor[HTML]{FFCCC9}\textbf{Descripción}} & 
    \multicolumn{1}{c|}{\cellcolor[HTML]{FFCCC9}\textbf{Tipo}} & 
    \multicolumn{1}{c|}{\cellcolor[HTML]{FFCCC9}\textbf{Responsabilidad}} \\ \hline
    Cliente & Representa a posible persona que reserva una actividad & Usuario sistema y producto & Realizar reservas de actividades \\ \hline
    Grupo & Representa a un grupo de personas, que reservan una actividad mediante un único cliente & Usuario sistema y producto & Realizar reservas de actividades de forma colectiva \\ \hline
    Guía turístico & Representa a la persona que dirige a los clientes en las actividades turísticas. Empleado & Usuario sistema y producto & Acompañar, dirigir e informar al grupo en ciertas actividades turísticas \\ \hline 
    Encargado & Representa al encargado de la correcta gestión, planificación y reserva de actividades turísticas & Usuario producto & Planificar correctamente las distintas actividades, indicando las características de estas a los clientes en la plataforma\\ \hline
    Proveedor & Representa a la empresa que proporciona actividades turísticas & Usuario sistema y producto & Ofrecer actividades turísticas a través del sistema \\ \hline
    \end{tabular}
    }
    \caption{Resumen de los implicados en el sistema GERAT.}
    \label{tab:implicados_gerat}
\end{table}

\subsection{Perfil de los implicados}
A continuación desarrollaremos con más detalles las características de cada uno de los implicados, realizando, por tanto, un perfil de todos ellos, para poder tener un mayor conocimiento de su función y uso en el sistema.
\begin{description}
    \item[Cliente] Podemos ver el perfil del Cliente en Tabla~\ref{tab:per-cliente}.
    \item[Grupo] Podemos ver el perfil del Grupo en Tabla~\ref{tab:per-grupo}.
    \item[Guía turístico] Podemos ver el perfil del Guía turístico en Tabla~\ref{tab:per-turistica}.
    \item[Encargado] Podemos ver el perfil del Encargado en Tabla~\ref{tab:per-encargado}.
    \item[Proveedor] Podemos ver el perfil del Proveedor en Tabla~\ref{tab:per-proveedor}.
\end{description}



\begin{table}
    \centering
    \begin{tabular}{|p{5cm}|p{10cm}|}
        \hline
        \cellcolor[HTML]{FFCCC9} \textbf{Representante} & Raúl González \\ \hline
        \cellcolor[HTML]{FFCCC9}\textbf{Descripción} & Cliente \\ \hline
        \cellcolor[HTML]{FFCCC9}\textbf{Tipo} & Usuario del sistema. Usuario del producto, ya que es él que hace las reservas de forma autónoma \\ \hline
        \cellcolor[HTML]{FFCCC9}\textbf{Responsabilidades} & Crearse una cuenta personal. Hacer reservas y cancelaciones si es oportuno. Consulta del catálogo de actividades turísticas, así como de los horarios disponibles.\\ \hline
        \cellcolor[HTML]{FFCCC9}\textbf{Criterios de éxito} & Debe poder realizar consultas en el sistema sobre el catálogo ofertado, el horario y los precios fácilmente, permitiendo la reserva de actividades turísticas al instante. Todo esto lo hará desde su cuenta personal. \\ \hline
        \cellcolor[HTML]{FFCCC9}\textbf{Implicación} & Es el responsable de realizar las reservas de las actividades turísticas, haciendo uso del sistema de forma esporádica. \\ \hline
        \cellcolor[HTML]{FFCCC9}\textbf{Comentarios/Cuestiones} & No tiene porque tener conocimiento ni control de sistemas informáticos. \\ \hline
    \end{tabular}
    \caption{Perfil del Cliente.}
    \label{tab:per-cliente}
\end{table}
\begin{table}
    \centering
    \begin{tabular}{|p{5cm}|p{10cm}|}
        \hline
        \cellcolor[HTML]{FFCCC9}\textbf{Representante} & Luis Enrique \\ \hline
        \cellcolor[HTML]{FFCCC9}\textbf{Descripción} & Grupo \\ \hline
        \cellcolor[HTML]{FFCCC9}\textbf{Tipo} & Utiliza el sistema de forma directa, usuario producto, y además implica el uso de forma indirecta de otros clientes, dando lugar a un grupo de personas en una reserva. \\ \hline
        \cellcolor[HTML]{FFCCC9}\textbf{Responsabilidades} & Mismas responsabilidades que el cliente invidual. Incorporación de otros clientes a actividades, sin acceso de estos al sistema de forma directa. \\ \hline
        \cellcolor[HTML]{FFCCC9}\textbf{Criterios de éxito} & Debe poder realizar reservas de grupos de forma sencilla y rápida, así como la cancelación de estas de forma conjunta para todas las personas del grupo. \\ \cellcolor[HTML]{FFCCC9}& Consulta rápida y fácil del catálogo, precios y horarios. \\ \hline
        \cellcolor[HTML]{FFCCC9}\textbf{Implicación} & Es el responsable de realizar reservas de grupos para actividades turísticas, haciendo uso del sistema de forma esporádica. \\ \hline
        \cellcolor[HTML]{FFCCC9}\textbf{Comentarios/Cuestiones} & No tiene porque tener conocimiento ni control de sistemas informáticos. \\ \hline
    \end{tabular}
    \caption{Perfil del Grupo.}
    \label{tab:per-grupo}
\end{table}
\begin{table}
    \centering
    \begin{tabular}{|p{5cm}|p{10cm}|}
        \hline
        \cellcolor[HTML]{FFCCC9}\textbf{Representante} & Olga Carmona \\ \hline
        \cellcolor[HTML]{FFCCC9}\textbf{Descripción} & Guía turístico. Empleado \\ \hline
        \cellcolor[HTML]{FFCCC9}\textbf{Tipo} & Usuario del sistema, es incorporado a las actividades mediante la empresa, la cual tiene asignado para ello distintos empleados. \\ \hline
        \cellcolor[HTML]{FFCCC9}\textbf{Responsabilidades} & Acompañar, dirigir e informar al grupo en ciertas actividades turísticas. Controlar la asistencia de los clientes.\\ \hline
        \cellcolor[HTML]{FFCCC9}\textbf{Criterios de éxito} & Debe poder consultar las actividades que tiene asignadas, así como la información de los clientes que participan en ellas. \\ \hline
        \cellcolor[HTML]{FFCCC9}\textbf{Implicación} & Es el responsable de dirigir a los clientes en las actividades turísticas, haciendo uso del sistema de forma esporádica. \\ \hline
        \cellcolor[HTML]{FFCCC9}\textbf{Comentarios/Cuestiones} & Para anular una actividad por causa justificada debe tener que comunicarse con su empresa asociada, es decir, no puede anular las actividades por sí solo. \\ \hline
    \end{tabular}
    \caption{Perfil del Guía turístico.}
    \label{tab:per-turistica}
\end{table}
\begin{table}
    \centering
    \begin{tabular}{|p{5cm}|p{10cm}|}
        \hline
        \cellcolor[HTML]{FFCCC9}\textbf{Representante} & Sara Paralluelo \\ \hline
        \cellcolor[HTML]{FFCCC9}\textbf{Descripción} & Encargado \\ \hline
        \cellcolor[HTML]{FFCCC9}\textbf{Tipo} & Responsabalidad alta. \\ \hline
        \cellcolor[HTML]{FFCCC9}\textbf{Responsabilidades} & Realizar la correcta planificación y reserva de las actividades por parte de los clientes. \\ \cellcolor[HTML]{FFCCC9}& Añadir y cambiar datos sobre las actividades. \\ \cellcolor[HTML]{FFCCC9}& Posibilidad de anular actividades, indicando el motivo y la posible devolución de dinero a los clientes. \\ \hline
        \cellcolor[HTML]{FFCCC9}\textbf{Criterios de éxito} & Éxito si se puede obtener un claro esquema de las actividades diarias y de los horarios, para poder gestionar su planificación de forma correcta.\\ \cellcolor[HTML]{FFCCC9}& Modificación y gestión de las actividades. \\ \hline
        \cellcolor[HTML]{FFCCC9}\textbf{Implicación} & Responsable del correcto horario de actividades y del cambio de actividades a huecos libres. \\ \hline
        \cellcolor[HTML]{FFCCC9}\textbf{Comentarios/Cuestiones} & Tiene altos concomientos en sistemas informáticos. \\ \hline
    \end{tabular}
    \caption{Perfil del Encargado.}
    \label{tab:per-encargado}
\end{table}
\begin{table}
    \centering
    \begin{tabular}{|p{5cm}|p{10cm}|}
        \hline
        \cellcolor[HTML]{FFCCC9}\textbf{Representante} & Explora Aventura \\ \hline
        \cellcolor[HTML]{FFCCC9}\textbf{Descripción} & Proveedor \\ \hline
        \cellcolor[HTML]{FFCCC9}\textbf{Tipo} & Usuario del producto, ya que es él que proporciona actividades turísticas, con total detalle de estos. \\ \hline
        \cellcolor[HTML]{FFCCC9}\textbf{Responsabilidades} & Ofrecer actividades turísticas a través del sistema. \\ \hline
        \cellcolor[HTML]{FFCCC9}\textbf{Criterios de éxito} & Éxito si puede ofrecer un catálogo de actividades turísticas atractivas y variadas, con precios competitivos. \\ \hline
        \cellcolor[HTML]{FFCCC9}\textbf{Implicación} & Responsable de la oferta de actividades turísticas. \\ \hline
        \cellcolor[HTML]{FFCCC9}\textbf{Comentarios/Cuestiones} & Tiene altos concomientos en sistemas informáticos. \\ \hline
    \end{tabular}
    \caption{Perfil del Proveedor.}
    \label{tab:per-proveedor}
\end{table}

\subsection{Necesidades de los implicados}
En esta sección, y última en la que se comentará información de los implicados, desarrollaremos las distintas situaciones que pueden darse para los clientes y usuarios del sistema, y cómo se pueden resolver. Destacar tener en cuenta que 
la prioridad de algunas de ellas puede no ser esencial, pero es importante tener en cuentas todas ellas y hacer que se cumplan por igual, dando lugar a un sistema completo y funcional. Estas necesidades se ven reflejadas en la Tabla~\ref{tab:resumen_implicados_gerat}.

\begin{table}
    \makebox[\textwidth][c]{ % Esto centra la tabla en la página
    \begin{tabular}{|p{2.5cm}|p{0.8cm}|p{3.3cm}|p{3cm}|p{5.5cm}|}
    \hline
    \multicolumn{1}{|c|}{\cellcolor[HTML]{FFCCC9}\textbf{Necesidad}} & 
    \multicolumn{1}{c|}{\cellcolor[HTML]{FFCCC9}\textbf{Prioridad}} & 
    \multicolumn{1}{c|}{\cellcolor[HTML]{FFCCC9}\textbf{Problema}} & 
    \multicolumn{1}{c|}{\cellcolor[HTML]{FFCCC9}\textbf{Solución actual}} &
    \multicolumn{1}{c|}{\cellcolor[HTML]{FFCCC9}\textbf{Solución propuesta}} \\ \hline
    Cancelar reserva  (usuarios) & Alta & No se puede cancelar una reserva por parte de un cliente o grupo & No asistir a la actividad, perdiendo el dinero & Permitir la cancelación de actividades (en el correspondiente periodo de cancelación) por parte de los clientes y grupos a través del sistema, y posibilidad de ofrecer un cambio de horario o devolución del dinero. \\ \hline
    Anular actividad (encargado) & Alta & No se puede anular una actividad por parte de un encargado & Comunicación directa con la empresa & Permitir la cancelación de actividades por parte de los encargados a través del sistema. \\ \hline
    Avisar a usuarios de cancelación & Alta & No se avisa a los usuarios de la cancelación de una actividad & Comunicación directa con los usuarios uno a uno & Aviso a los usuarios de la cancelación de una actividad por parte de la empresa mediante un email común, indicando el motivo y la posible devolución del dinero. \\ \hline
    Histórico de reservas & Baja & No se puede ver el histórico de reservas & Mirar los resguardos de cada una de las actividades reservadas & Permitir a los usuarios ver el histórico de reservas realizadas, así como la posibilidad de repetir una reserva anterior. \\ \hline
    Consulta de actividades & Alta & No se puede ver las actividades disponibles, horarios y precios & Consulta directa con la empresa & Permitir a los usuarios ver las actividades disponibles, horarios y precios de forma rápida y sencilla, así como la posibilidad de reservarlas al instante. \\ \hline
    Consulta reservas & Media & No se puede ver las actividades reservadas & Mirar los resguardos de cada una de las actividades reservadas & Permitir a los usuarios ver las actividades reservadas, así como la posibilidad de cancelarlas o modificarlas. \\ \hline
    \end{tabular}
    }
    \caption{Resumen de los implicados en el sistema GERAT.}
    \label{tab:resumen_implicados_gerat}
\end{table}

\begin{comment}
    El glosario, o a Glosario.tex (asegurandose de que nadie toca), o aquí en sucio y luego se pasará
    Actividad turística, catálogo ofertado, guía turística, cancelar reserva, anular actividad, resguardo, parajes naturales, monumentos, periodo de cancelacion
\end{comment}

    \section{Requisitos Funcionales}

%%%%%%%% LINARI %%%%%%%%
En esta sección definiremos las características de alto nivel del sistema que son esenciales para 
cubrir las necesidades de los usuarios al hacer uso de la aplicación.

% Enumerate en el orden son RF-1, RF-2, etc.
\begin{enumerate}[label={\color{red}RF-\arabic{enumi}.}]
    \item \label{enum:RF-1}\textbf{Gestión de reservas.}
    
        El sistema debe realizar la gestión de las reservas de las actividades ofertadas que se puedan realizar 
        actualmente y de las actividades que estén planificadas para fechas posteriores.

    \begin{enumerate}[label={\color{red}\theenumi\arabic{enumii}.}]
        
        \item \label{enum:RF-1.1}Tener un control del estado de las reservas de todas las actividades disponibles en la agencia.
        \begin{enumerate}[label={\color{red}\theenumii\arabic{enumiii}.}]
            \item\label{enum:RF-1.1.1} Poder saber si una actividad está en curso, disponible, completada o anulada.
            \item\label{enum:RF-1.1.2} Mostrar plazas disponibles y plazas reservadas.
        \end{enumerate}
        \item\label{enum:RF-1.2} Gestionar los pagos de los clientes 
        \begin{enumerate}[label={\color{red}\theenumii\arabic{enumiii}.}]
            \item\label{enum:RF-1.2.1} Calcular descuentos y bonos (para estudiantes, minusválidos, personas mayores o jubilados, por ejemplo).
            \item\label{enum:RF-1.2.2} Tramitar devoluciones. 
            \item\label{enum:RF-1.2.3} Tramitar y confirmar pagos del cliente. 
        \end{enumerate}


        \item\label{enum:RF-1.3} Obtener información sobre quién va a realizar cada actividad. 
        \begin{enumerate}[label={\color{red}\theenumii\arabic{enumiii}.}]
            \item\label{enum:RF-1.3.1} Saber información necesaria sobre el cliente o grupo. 
            \item\label{enum:RF-1.3.2} Número de personas que van a realizar la actividad. 
            \item\label{enum:RF-1.3.3} Conocer si se trata de un grupo, familia, individual\ldots
            \item\label{enum:RF-1.3.4} Requerimientos especiales: discapacitados, menores de $x$ años, estudiantes\ldots
        \end{enumerate}

        \item\label{enum:RF-1.4} Poder anular una reserva de una actividad turística dentro del periodo de anulación.
        
        \item\label{enum:RF-1.5} Poder hacer reservas para un grupo de personas desde un único perfil y con un único pago, indicando el número de personas que van a realizar la actividad.
        
        \item\label{enum:RF-1.6} Valorar la actividad pasada realizada.
        \begin{enumerate}[label={\color{red}\theenumii\arabic{enumiii}.}]
            \item\label{enum:RF-1.6.1} Puntuar la actividad realizada.
            \item\label{enum:RF-1.6.2} Dejar un comentario sobre la actividad realizada.
            \item\label{enum:RF-1.6.3} Añadir archivos multimedia.
        \end{enumerate}
    \end{enumerate}
    \item \textbf{Gestión de actividades ofertadas.} \label{enum:RF-2}
    
        El sistema deberá gestionar las actividades que los proveedores decidan ofertar y debe sincronizar la información visible de dichas actividades con la disponible para los clientes.
    % Enumerate en el orden son RF-1.1, RF-1.2, etc.
    \begin{enumerate}[label={\color{red}\theenumi\arabic{enumii}.}]

        \item\label{enum:RF-2.1} Programar actividades. 
        \begin{enumerate}[label={\color{red}\theenumii\arabic{enumiii}.}]
            \item\label{enum:RF-2.1.1} Definir el período de disponibilidad de dicha actividad. 
            \item\label{enum:RF-2.1.2} Definir el período de fin de reserva y de anulación de reserva. 
        \end{enumerate}

        \item\label{enum:RF-2.2} Saber la disponibilidad y tipo de cada actividad.
        \begin{enumerate}[label={\color{red}\theenumii\arabic{enumiii}.}]
            \item\label{enum:RF-2.2.1} Ver las actividades que se pueden realizar actualmente.
            \item\label{enum:RF-2.2.2} Mostrar en qué consiste cada actividad a realizar.
            \item\label{enum:RF-2.2.3} Ver las especificaciones y requisitos de cada actividad.
        \end{enumerate}

        \item\label{enum:RF-2.3} Modificar reservas y actividades.
        \begin{enumerate}[label={\color{red}\theenumii\arabic{enumiii}.}]
            \item\label{enum:RF-2.3.1} Modificar fecha y hora de la actividad.
            \item\label{enum:RF-2.3.2} Cambiar el planteamiento de la actividad. 
            \item\label{enum:RF-2.3.3} Añadir o quitar plazas para la actividad. 
            \item\label{enum:RF-2.3.4} Mostrar posibles alternativas a la actividad en caso de que el cambio perjudique significativamente al cliente.
        \end{enumerate}
    \end{enumerate}

    \item \textbf{Gestión de Perfiles}\label{enum:RF-3}

    
    El sistema debe permitir crear y gestionar perfiles de usuario para todos los usuarios sistema (proveedores, clientes y guías turísticos).


    \begin{enumerate}[label={\color{red}\theenumi\arabic{enumii}.}]

        \item\label{enum:RF-3.1} Crear perfiles de usuario (indicando si es proveedor, cliente o guía turístico).
        
        \item\label{enum:RF-3.2} Obtener información sobre un perfil de usuario.
        \begin{enumerate}[label={\color{red}\theenumii\arabic{enumiii}.}]
            \item\label{enum:RF-3.2.1} Ver los datos personales de un perfil.
            \item\label{enum:RF-3.2.2} Consultar las actividades que ha realizado o contratado un cliente.
            \item\label{enum:RF-3.2.3} Ver las actividades ofertadas por un proveedor.
            \item\label{enum:RF-3.2.4} Consultar las actividades que ha dirigido un guía turístico.
        \end{enumerate}

        \item\label{enum:RF-3.3} Modificar los datos personales de un perfil de usuario.
        
        \item\label{enum:RF-3.4} Dar de baja a un perfil de usuario (siempre que este lo haya solicitado).
        

        \item\label{enum:RF-3.5} Hacer cliente a un consumidor de la actividad. 
        \begin{enumerate}[label={\color{red}\theenumii\arabic{enumiii}.}]
            \item\label{enum:RF-3.5.1} Poder crear perfiles de clientes para que se puedan reutilizar en próximas reservas. 
            \item\label{enum:RF-3.5.2} Poder crear perfiles de grupo para reutilizarlos en próximas reservas. 
        \end{enumerate}
    \end{enumerate}



\end{enumerate}

\begin{comment}
    Perfil, jubilados, archivos multimedia
\end{comment}



    \section{Requisitos No Funcionales}


%%%%%%%% MANU %%%%%%%%


\begin{description}
    \item[Usabilidad]~
    \begin{enumerate}[label={\color{red}RNF-\arabic{enumi}.}]
        \item La interfaz de usuario deberá ser \emph{intuitiva} y \emph{accesible} para clientes y guías turísticos, permitiendo la reserva y gestión de las actividades de forma rápida y cómoda.
        \item El sistema debe estar \emph{disponible en varios idiomas} (español, inglés, francés, alemán, y chino mandarín) para facilitar el proceso a personas del extranjero.
        \item Se ofrecerán instrucciones visuales y un apartado de preguntas frecuentes recogiendo las dudas más comunes para facilitar el uso de la plataforma, al igual que el método de pago y reserva.
    \end{enumerate}

    \item[Fiabilidad y Disponibilidad]~
    \begin{enumerate}[label={\color{red}RNF-\arabic{enumi}.}, resume]
        \item El sistema deberá estar \emph{disponible} en al menos un 95\% del tiempo del año, para no tener grandes pérdidas ecónmicas debido a problemas técnicos.
        \item Se deberán hacer \emph{copias de seguridad} con cierta frecuencia, cada 12h, para evitar pérdidas de los datos.
        \item En caso de caída, debe poderse volver a \emph{restaurar el sistema} en menos de 20 minutos.
        Además, durante dicho periodo de caída, se deberá mostrar un mensaje de error a los usuarios.
    \end{enumerate}

    \item[Rendimiento]~
    \begin{enumerate}[label={\color{red}RNF-\arabic{enumi}.}, resume]
        \item El \emph{tiempo de respuesta de carga} de la página no deben sobrepasar los 10 segundos bajo ninguna circunstancia. 
        \item Se debe poder trabajar con una \emph{cantidad usuarios} entorno a los 10000 usuarios simultáneos.
        \item La \emph{búsqueda de actividades túristicas} deberá de ofrecer la respuesta a los usarios en menos de 4 segundos, trabajando con unas 1000-1500 actividades por provincia.
    \end{enumerate}

    \item[Seguridad]~
    \begin{enumerate}[label={\color{red}RNF-\arabic{enumi}.}, resume]
        \item Todos los \emph{datos personales y de pago} deben ser almacenados cifrados siguiendo el estándar AES-256.
        \item Se debe establecer un \emph{sistema de roles y permisos} que permitan, únicamente al personal autorizado, modificar los precios y horarios de las actividades, así como añadir o eliminar actividades.
        \item Para poder acceder a la plataforma, se deberá \emph{verificar la identidad} de los usuarios mediante un correo electrónico y una contraseña.
        \item El pago deberá hacerse a través de una \emph{pasarela de pago segura} a través de interfaces de pago como Stripe, PayPal, ApplePay\dots
    \end{enumerate}~\\

    \item[Soporte y Restricciones Técnicas]~
    \begin{enumerate}[label={\color{red}RNF-\arabic{enumi}.}, resume]
        \item Los datos deben transformarse en un \emph{formato apto para usar aplicaciones de terceros}, como las de pago: PayPal, ApplePay\dots o las de reservas de actividades con terceros, como por ejemplo AlhambraGranada-Tickets, GetYourGuide\dots
        \item Debe contarse con un \emph{diseño accesible} por todo tipo de dispostivos (ordenadores, móviles\dots).
        \item La plataforma debe ser \emph{compatible} con los navegadores más usados(Google Chrome, Mozilla Firefox, Microsoft Edge, Safari\dots) en sus últimas versiones.
    \end{enumerate}

    \item[Restricciones de Implementación]~
    \begin{enumerate}[label={\color{red}RNF-\arabic{enumi}.}, resume]   
        \item El sistema debe desarrollarse en un \emph{entorno compatible con Java y bases de datos SQL}, permitiendo la escalabilidad.
        \item El código debe seguir una \emph{implementación modular} para hacer más fácil las futuras ampliaciones y mantenimientos.
    \end{enumerate}

    \item[Restricciones Físicas]~
    \begin{enumerate}[label={\color{red}RNF-\arabic{enumi}.}, resume]   
        \item Será necesario que los guías turisticos dispongan de dispositivos electrónicos de fácil uso para poder controlar la asistencia a las actividades.
    \end{enumerate}
\end{description}


\begin{comment}
    El glosario, o a Glosario.tex (asegurandose de que nadie toca), o aquí en sucio y luego se pasará

    Pasarela de pago
\end{comment}


    \section{Requisitos de Información}

%%%%%%%% ARTURO %%%%%%%%


A continuación, detallamos los datos que el sistema debe manejar para poder llevar a cabo las funcionalidades descritas en los apartados anteriores.
\begin{enumerate}[label={\color{red}RI-\arabic{enumi}.}]
    \item \textbf{Actividades Ofertadas}
    
    Información sobre las actividades que se ofertan en la plataforma.
    \begin{itemize}
        \item {\color{violet}Contenido:} Nombre de la actividad, fecha, hora, lugar, precio, plazas ofertadas, descripción.
        \item {\color{violet}Requisitos asociados:} \ref{enum:RF-1.1}, \ref{enum:RF-2.1}, \ref{enum:RF-2.2}, \ref{enum:RF-2.3}
    \end{itemize}
    
    \item \textbf{Actividades Pasadas}
    
    Información sobre las actividades que ya se han realizado, para guardar un cierto histórico.
    \begin{itemize}
        \item {\color{violet}Tiempo de Vida Máximo:} Se guardarán las actividades realizadas en los últimos 2 años.
        \item {\color{violet}Contenido:} Igual que en Actividades Ofertadas, pero incluimos el número de plazas ocupadas y los identificadores de las reservas realizadas.
        \item {\color{violet}Requisitos asociados:} \ref{enum:RF-1.6}
    \end{itemize}

    \item \textbf{Clientes}
    
    Información sobre los clientes que poseen una cuenta en la plataforma.
    \begin{itemize}
        \item {\color{violet}Contenido:} Nombre, apellidos, DNI, dirección, teléfono, correo electrónico, contraseña, tarjeta de crédito.
        \item {\color{violet}Requisitos asociados:} \ref{enum:RF-1.2.2}, \ref{enum:RF-1.2.3}, \ref{enum:RF-1.3}, \ref{enum:RF-3}
    \end{itemize}

    \item \textbf{Guías Turísticos}
    
    Información sobre los guías turísticos que realizan las actividades de la plataforma.
    \begin{itemize}
        \item {\color{violet}Contenido:} Nombre, apellidos, DNI, dirección, teléfono, correo electrónico, contraseña, número de identificación de guía, histórico de actividades asignadas.
        \item {\color{violet}Requisitos asociados:} \ref{enum:RF-3}
    \end{itemize}

    \item \textbf{Proveedores}
    
    Información sobre los proveedores de servicios turísticos.
    \begin{itemize}
        \item {\color{violet}Contenido:} Nombre, NIF, dirección, teléfono, correo electrónico, contraseña, IBAN.
        \item {\color{violet}Requisitos asociados:} \ref{enum:RF-3}
    \end{itemize}

    \item \textbf{Reservas de Actividades Ofertadas}
    
    Información sobre las reservas realizadas por los clientes.
    \begin{itemize}
        \item {\color{violet}Contenido:} Identificador de la reserva, identificador de la actividad, identificador del cliente, fecha y hora de la reserva, número de plazas reservadas, información adicional.
        \item {\color{violet}Requisitos asociados:} \ref{enum:RF-1}
    \end{itemize}

    \item \textbf{Reservas de Actividades Pasadas}
    
    Información sobre las reservas realizadas en actividades pasadas.
    \begin{itemize}
        \item {\color{violet}Contenido:} Igual que en Reservas de Actividades Ofertadas, pero incluimos la reseña correspondiente (valoración del cliente, fotos y comentarios).
        \item {\color{violet}Requisitos asociados:} \ref{enum:RF-1.6}
    \end{itemize}
\end{enumerate}
    

    \newpage
    \glsaddallunused
    \printglossary[numberedsection]
\end{document}