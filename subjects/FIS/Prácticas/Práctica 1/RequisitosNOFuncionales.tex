\section{Requisitos No Funcionales}


%%%%%%%% MANU %%%%%%%%


\begin{description}
    \item[Usabilidad]~
    \begin{enumerate}[label={\color{red}RNF-\arabic{enumi}.}]
        \item La interfaz de usuario deberá ser \emph{intuitiva} y \emph{accesible} para clientes y guías turísticos, permitiendo la reserva y gestión de las actividades de forma rápida y cómoda.
        \item El sistema debe estar \emph{disponible en varios idiomas} (español, inglés, francés, alemán, y chino mandarín) para facilitar el proceso a personas del extranjero.
        \item Se ofrecerán instrucciones visuales y un apartado de preguntas frecuentes recogiendo las dudas más comunes para facilitar el uso de la plataforma, al igual que el método de pago y reserva.
    \end{enumerate}

    \item[Fiabilidad y Disponibilidad]~
    \begin{enumerate}[label={\color{red}RNF-\arabic{enumi}.}, resume]
        \item El sistema deberá estar \emph{disponible} en al menos un 95\% del tiempo del año, para no tener grandes pérdidas ecónmicas debido a problemas técnicos.
        \item Se deberán hacer \emph{copias de seguridad} con cierta frecuencia, cada 12h, para evitar pérdidas de los datos.
        \item En caso de caída, debe poderse volver a \emph{restaurar el sistema} en menos de 20 minutos.
        Además, durante dicho periodo de caída, se deberá mostrar un mensaje de error a los usuarios.
    \end{enumerate}

    \item[Rendimiento]~
    \begin{enumerate}[label={\color{red}RNF-\arabic{enumi}.}, resume]
        \item El \emph{tiempo de respuesta de carga} de la página no deben sobrepasar los 10 segundos bajo ninguna circunstancia. 
        \item Se debe poder trabajar con una \emph{cantidad usuarios} entorno a los 10000 usuarios simultáneos.
        \item La \emph{búsqueda de actividades túristicas} deberá de ofrecer la respuesta a los usarios en menos de 4 segundos, trabajando con unas 1000-1500 actividades por provincia.
    \end{enumerate}

    \item[Seguridad]~
    \begin{enumerate}[label={\color{red}RNF-\arabic{enumi}.}, resume]
        \item Todos los \emph{datos personales y de pago} deben ser almacenados cifrados siguiendo el estándar AES-256.
        \item Se debe establecer un \emph{sistema de roles y permisos} que permitan, únicamente al personal autorizado, modificar los precios y horarios de las actividades, así como añadir o eliminar actividades.
        \item Para poder acceder a la plataforma, se deberá \emph{verificar la identidad} de los usuarios mediante un correo electrónico y una contraseña.
        \item El pago deberá hacerse a través de una \emph{pasarela de pago segura} a través de interfaces de pago como Stripe, PayPal, ApplePay\dots
    \end{enumerate}~\\

    \item[Soporte y Restricciones Técnicas]~
    \begin{enumerate}[label={\color{red}RNF-\arabic{enumi}.}, resume]
        \item Los datos deben transformarse en un \emph{formato apto para usar aplicaciones de terceros}, como las de pago: PayPal, ApplePay\dots o las de reservas de actividades con terceros, como por ejemplo AlhambraGranada-Tickets, GetYourGuide\dots
        \item Debe contarse con un \emph{diseño accesible} por todo tipo de dispostivos (ordenadores, móviles\dots).
        \item La plataforma debe ser \emph{compatible} con los navegadores más usados(Google Chrome, Mozilla Firefox, Microsoft Edge, Safari\dots) en sus últimas versiones.
    \end{enumerate}

    \item[Restricciones de Implementación]~
    \begin{enumerate}[label={\color{red}RNF-\arabic{enumi}.}, resume]   
        \item El sistema debe desarrollarse en un \emph{entorno compatible con Java y bases de datos SQL}, permitiendo la escalabilidad.
        \item El código debe seguir una \emph{implementación modular} para hacer más fácil las futuras ampliaciones y mantenimientos.
    \end{enumerate}

    \item[Restricciones Físicas]~
    \begin{enumerate}[label={\color{red}RNF-\arabic{enumi}.}, resume]   
        \item Será necesario que los guías turisticos dispongan de dispositivos electrónicos de fácil uso para poder controlar la asistencia a las actividades.
    \end{enumerate}
\end{description}


\begin{comment}
    El glosario, o a Glosario.tex (asegurandose de que nadie toca), o aquí en sucio y luego se pasará

    Pasarela de pago
\end{comment}

