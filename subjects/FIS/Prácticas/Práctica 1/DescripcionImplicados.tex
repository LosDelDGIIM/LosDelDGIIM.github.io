\section{Descripción de los implicados}

%%%%%%%% JOAQUIN %%%%%%%%

En esta sección se hará una descripción de los implicados en el sistema de gestión de reservas turísticas GERAT, especificando sus roles y las responsabilidades que tienen en el mismo. Los usuarios directos del producto son los clientes y las empresas que proporcionan actividades turísticas, por los distintos parajes naturales, monumentos, ciudades, etc.

Veámos a continuación un resumen de los distintos implicados, indicando su descripción y sus roles de forma breve.

\begin{table}
    \makebox[\textwidth][c]{ % Esto centra la tabla en la página
    \begin{tabular}{|p{2.5cm}|p{5.5cm}|p{3.1cm}|p{5.5cm}|}
    \hline
    \multicolumn{1}{|c|}{\cellcolor[HTML]{FFCCC9}\textbf{Nombre}} & 
    \multicolumn{1}{c|}{\cellcolor[HTML]{FFCCC9}\textbf{Descripción}} & 
    \multicolumn{1}{c|}{\cellcolor[HTML]{FFCCC9}\textbf{Tipo}} & 
    \multicolumn{1}{c|}{\cellcolor[HTML]{FFCCC9}\textbf{Responsabilidad}} \\ \hline
    Cliente & Representa a posible persona que reserva una actividad & Usuario sistema y producto & Realizar reservas de actividades \\ \hline
    Grupo & Representa a un grupo de personas, que reservan una actividad mediante un único cliente & Usuario sistema y producto & Realizar reservas de actividades de forma colectiva \\ \hline
    Guía turístico & Representa a la persona que dirige a los clientes en las actividades turísticas. Empleado & Usuario sistema y producto & Acompañar, dirigir e informar al grupo en ciertas actividades turísticas \\ \hline 
    Encargado & Representa al encargado de la correcta gestión, planificación y reserva de actividades turísticas & Usuario producto & Planificar correctamente las distintas actividades, indicando las características de estas a los clientes en la plataforma\\ \hline
    Proveedor & Representa a la empresa que proporciona actividades turísticas & Usuario sistema y producto & Ofrecer actividades turísticas a través del sistema \\ \hline
    \end{tabular}
    }
    \caption{Resumen de los implicados en el sistema GERAT.}
    \label{tab:implicados_gerat}
\end{table}

\subsection{Perfil de los implicados}
A continuación desarrollaremos con más detalles las características de cada uno de los implicados, realizando, por tanto, un perfil de todos ellos, para poder tener un mayor conocimiento de su función y uso en el sistema.
\begin{description}
    \item[Cliente] Podemos ver el perfil del Cliente en Tabla~\ref{tab:per-cliente}.
    \item[Grupo] Podemos ver el perfil del Grupo en Tabla~\ref{tab:per-grupo}.
    \item[Guía turístico] Podemos ver el perfil del Guía turístico en Tabla~\ref{tab:per-turistica}.
    \item[Encargado] Podemos ver el perfil del Encargado en Tabla~\ref{tab:per-encargado}.
    \item[Proveedor] Podemos ver el perfil del Proveedor en Tabla~\ref{tab:per-proveedor}.
\end{description}



\begin{table}
    \centering
    \begin{tabular}{|p{5cm}|p{10cm}|}
        \hline
        \cellcolor[HTML]{FFCCC9} \textbf{Representante} & Raúl González \\ \hline
        \cellcolor[HTML]{FFCCC9}\textbf{Descripción} & Cliente \\ \hline
        \cellcolor[HTML]{FFCCC9}\textbf{Tipo} & Usuario del sistema. Usuario del producto, ya que es él que hace las reservas de forma autónoma \\ \hline
        \cellcolor[HTML]{FFCCC9}\textbf{Responsabilidades} & Crearse una cuenta personal. Hacer reservas y cancelaciones si es oportuno. Consulta del catálogo de actividades turísticas, así como de los horarios disponibles.\\ \hline
        \cellcolor[HTML]{FFCCC9}\textbf{Criterios de éxito} & Debe poder realizar consultas en el sistema sobre el catálogo ofertado, el horario y los precios fácilmente, permitiendo la reserva de actividades turísticas al instante. Todo esto lo hará desde su cuenta personal. \\ \hline
        \cellcolor[HTML]{FFCCC9}\textbf{Implicación} & Es el responsable de realizar las reservas de las actividades turísticas, haciendo uso del sistema de forma esporádica. \\ \hline
        \cellcolor[HTML]{FFCCC9}\textbf{Comentarios/Cuestiones} & No tiene porque tener conocimiento ni control de sistemas informáticos. \\ \hline
    \end{tabular}
    \caption{Perfil del Cliente.}
    \label{tab:per-cliente}
\end{table}
\begin{table}
    \centering
    \begin{tabular}{|p{5cm}|p{10cm}|}
        \hline
        \cellcolor[HTML]{FFCCC9}\textbf{Representante} & Luis Enrique \\ \hline
        \cellcolor[HTML]{FFCCC9}\textbf{Descripción} & Grupo \\ \hline
        \cellcolor[HTML]{FFCCC9}\textbf{Tipo} & Utiliza el sistema de forma directa, usuario producto, y además implica el uso de forma indirecta de otros clientes, dando lugar a un grupo de personas en una reserva. \\ \hline
        \cellcolor[HTML]{FFCCC9}\textbf{Responsabilidades} & Mismas responsabilidades que el cliente invidual. Incorporación de otros clientes a actividades, sin acceso de estos al sistema de forma directa. \\ \hline
        \cellcolor[HTML]{FFCCC9}\textbf{Criterios de éxito} & Debe poder realizar reservas de grupos de forma sencilla y rápida, así como la cancelación de estas de forma conjunta para todas las personas del grupo. \\ \cellcolor[HTML]{FFCCC9}& Consulta rápida y fácil del catálogo, precios y horarios. \\ \hline
        \cellcolor[HTML]{FFCCC9}\textbf{Implicación} & Es el responsable de realizar reservas de grupos para actividades turísticas, haciendo uso del sistema de forma esporádica. \\ \hline
        \cellcolor[HTML]{FFCCC9}\textbf{Comentarios/Cuestiones} & No tiene porque tener conocimiento ni control de sistemas informáticos. \\ \hline
    \end{tabular}
    \caption{Perfil del Grupo.}
    \label{tab:per-grupo}
\end{table}
\begin{table}
    \centering
    \begin{tabular}{|p{5cm}|p{10cm}|}
        \hline
        \cellcolor[HTML]{FFCCC9}\textbf{Representante} & Olga Carmona \\ \hline
        \cellcolor[HTML]{FFCCC9}\textbf{Descripción} & Guía turístico. Empleado \\ \hline
        \cellcolor[HTML]{FFCCC9}\textbf{Tipo} & Usuario del sistema, es incorporado a las actividades mediante la empresa, la cual tiene asignado para ello distintos empleados. \\ \hline
        \cellcolor[HTML]{FFCCC9}\textbf{Responsabilidades} & Acompañar, dirigir e informar al grupo en ciertas actividades turísticas. Controlar la asistencia de los clientes.\\ \hline
        \cellcolor[HTML]{FFCCC9}\textbf{Criterios de éxito} & Debe poder consultar las actividades que tiene asignadas, así como la información de los clientes que participan en ellas. \\ \hline
        \cellcolor[HTML]{FFCCC9}\textbf{Implicación} & Es el responsable de dirigir a los clientes en las actividades turísticas, haciendo uso del sistema de forma esporádica. \\ \hline
        \cellcolor[HTML]{FFCCC9}\textbf{Comentarios/Cuestiones} & Para anular una actividad por causa justificada debe tener que comunicarse con su empresa asociada, es decir, no puede anular las actividades por sí solo. \\ \hline
    \end{tabular}
    \caption{Perfil del Guía turístico.}
    \label{tab:per-turistica}
\end{table}
\begin{table}
    \centering
    \begin{tabular}{|p{5cm}|p{10cm}|}
        \hline
        \cellcolor[HTML]{FFCCC9}\textbf{Representante} & Sara Paralluelo \\ \hline
        \cellcolor[HTML]{FFCCC9}\textbf{Descripción} & Encargado \\ \hline
        \cellcolor[HTML]{FFCCC9}\textbf{Tipo} & Responsabalidad alta. \\ \hline
        \cellcolor[HTML]{FFCCC9}\textbf{Responsabilidades} & Realizar la correcta planificación y reserva de las actividades por parte de los clientes. \\ \cellcolor[HTML]{FFCCC9}& Añadir y cambiar datos sobre las actividades. \\ \cellcolor[HTML]{FFCCC9}& Posibilidad de anular actividades, indicando el motivo y la posible devolución de dinero a los clientes. \\ \hline
        \cellcolor[HTML]{FFCCC9}\textbf{Criterios de éxito} & Éxito si se puede obtener un claro esquema de las actividades diarias y de los horarios, para poder gestionar su planificación de forma correcta.\\ \cellcolor[HTML]{FFCCC9}& Modificación y gestión de las actividades. \\ \hline
        \cellcolor[HTML]{FFCCC9}\textbf{Implicación} & Responsable del correcto horario de actividades y del cambio de actividades a huecos libres. \\ \hline
        \cellcolor[HTML]{FFCCC9}\textbf{Comentarios/Cuestiones} & Tiene altos concomientos en sistemas informáticos. \\ \hline
    \end{tabular}
    \caption{Perfil del Encargado.}
    \label{tab:per-encargado}
\end{table}
\begin{table}
    \centering
    \begin{tabular}{|p{5cm}|p{10cm}|}
        \hline
        \cellcolor[HTML]{FFCCC9}\textbf{Representante} & Explora Aventura \\ \hline
        \cellcolor[HTML]{FFCCC9}\textbf{Descripción} & Proveedor \\ \hline
        \cellcolor[HTML]{FFCCC9}\textbf{Tipo} & Usuario del producto, ya que es él que proporciona actividades turísticas, con total detalle de estos. \\ \hline
        \cellcolor[HTML]{FFCCC9}\textbf{Responsabilidades} & Ofrecer actividades turísticas a través del sistema. \\ \hline
        \cellcolor[HTML]{FFCCC9}\textbf{Criterios de éxito} & Éxito si puede ofrecer un catálogo de actividades turísticas atractivas y variadas, con precios competitivos. \\ \hline
        \cellcolor[HTML]{FFCCC9}\textbf{Implicación} & Responsable de la oferta de actividades turísticas. \\ \hline
        \cellcolor[HTML]{FFCCC9}\textbf{Comentarios/Cuestiones} & Tiene altos concomientos en sistemas informáticos. \\ \hline
    \end{tabular}
    \caption{Perfil del Proveedor.}
    \label{tab:per-proveedor}
\end{table}

\subsection{Necesidades de los implicados}
En esta sección, y última en la que se comentará información de los implicados, desarrollaremos las distintas situaciones que pueden darse para los clientes y usuarios del sistema, y cómo se pueden resolver. Destacar tener en cuenta que 
la prioridad de algunas de ellas puede no ser esencial, pero es importante tener en cuentas todas ellas y hacer que se cumplan por igual, dando lugar a un sistema completo y funcional. Estas necesidades se ven reflejadas en la Tabla~\ref{tab:resumen_implicados_gerat}.

\begin{table}
    \makebox[\textwidth][c]{ % Esto centra la tabla en la página
    \begin{tabular}{|p{2.5cm}|p{0.8cm}|p{3.3cm}|p{3cm}|p{5.5cm}|}
    \hline
    \multicolumn{1}{|c|}{\cellcolor[HTML]{FFCCC9}\textbf{Necesidad}} & 
    \multicolumn{1}{c|}{\cellcolor[HTML]{FFCCC9}\textbf{Prioridad}} & 
    \multicolumn{1}{c|}{\cellcolor[HTML]{FFCCC9}\textbf{Problema}} & 
    \multicolumn{1}{c|}{\cellcolor[HTML]{FFCCC9}\textbf{Solución actual}} &
    \multicolumn{1}{c|}{\cellcolor[HTML]{FFCCC9}\textbf{Solución propuesta}} \\ \hline
    Cancelar reserva  (usuarios) & Alta & No se puede cancelar una reserva por parte de un cliente o grupo & No asistir a la actividad, perdiendo el dinero & Permitir la cancelación de actividades (en el correspondiente periodo de cancelación) por parte de los clientes y grupos a través del sistema, y posibilidad de ofrecer un cambio de horario o devolución del dinero. \\ \hline
    Anular actividad (encargado) & Alta & No se puede anular una actividad por parte de un encargado & Comunicación directa con la empresa & Permitir la cancelación de actividades por parte de los encargados a través del sistema. \\ \hline
    Avisar a usuarios de cancelación & Alta & No se avisa a los usuarios de la cancelación de una actividad & Comunicación directa con los usuarios uno a uno & Aviso a los usuarios de la cancelación de una actividad por parte de la empresa mediante un email común, indicando el motivo y la posible devolución del dinero. \\ \hline
    Histórico de reservas & Baja & No se puede ver el histórico de reservas & Mirar los resguardos de cada una de las actividades reservadas & Permitir a los usuarios ver el histórico de reservas realizadas, así como la posibilidad de repetir una reserva anterior. \\ \hline
    Consulta de actividades & Alta & No se puede ver las actividades disponibles, horarios y precios & Consulta directa con la empresa & Permitir a los usuarios ver las actividades disponibles, horarios y precios de forma rápida y sencilla, así como la posibilidad de reservarlas al instante. \\ \hline
    Consulta reservas & Media & No se puede ver las actividades reservadas & Mirar los resguardos de cada una de las actividades reservadas & Permitir a los usuarios ver las actividades reservadas, así como la posibilidad de cancelarlas o modificarlas. \\ \hline
    \end{tabular}
    }
    \caption{Resumen de los implicados en el sistema GERAT.}
    \label{tab:resumen_implicados_gerat}
\end{table}

\begin{comment}
    El glosario, o a Glosario.tex (asegurandose de que nadie toca), o aquí en sucio y luego se pasará
    Actividad turística, catálogo ofertado, guía turística, cancelar reserva, anular actividad, resguardo, parajes naturales, monumentos, periodo de cancelacion
\end{comment}
