\section{Objetivos}

Como se ha descrito en el breve resumen inicial, el presente documento muestra la especificación de requisistos de un sistema informático de gestión de reservas de actividades turísticas, denominado GERAT. Dicho sistema permite a las entidades organizadoras de las actividades publicar nuevas propuestas y gestionar las reservas de los clientes. A su vez, los clientes pueden visualizar las actividades ofertadas, reservar plazas y gestionar sus reservas.
        
GERAT se presenta como una herramienta que facilita la organización de actividades turísticas y la gestión de las reservas de los clientes. Esta plataforma permitirá así unificar la información de las actividades y las reservas que se dan en cierta localidad, provincia o país, facilitando así la gestión de las mismas y repercutiendo positivamente en la experiencia de los clientes y, por tanto, en la actividad turística y económica de la zona. Asimismo, puesto que el número de reservas aumentará, se espera que la empresas organizadoras de las actividades pueda aumentar su beneficio.

Una vez descrito el dominio del problema y la descripción general del sistema, detallamos a continuación los principales objetivos que se pretenden alcanzar con el desarrollo de GERAT.

\begin{enumerate}[label={\color{red}OBJ-\arabic{enumi}.}]
    \item El sistema deberá almacenar y gestionar la información relativa a las actividades ofertadas por las empresas organizadoras.
    \item El sistema deberá permitir a los clientes visualizar las actividades ofertadas y reservar (o cancelar) plazas en las mismas, tanto de forma individual como en grupo.
    \item El sistema automatizará los cobros de las reservas realizadas por los clientes, permitiendo el pago mediante tarjeta de crédito a través de una pasarela de pago segura.
\end{enumerate}


\begin{comment}
    El glosario, o a Glosario.tex (asegurandose de que nadie toca), o aquí en sucio y luego se pasará
\end{comment}