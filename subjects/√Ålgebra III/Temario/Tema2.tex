\chapter{Extensiones de Galois}
\section{Extensiones de Galois}
\noindent
Del Capítulo anterior recordamos la Proposición~\ref{prop:comienzo_chap_2}, que nos servirá para comenzar este Capítulo:

\begin{center}
    Sea $F\leq K$ una extensión finita, entonces $|Aut_F(K)| \leq [K:F]$.
\end{center}

\noindent
Esto nos permite obtener grupos finitos de automorfismos a partir de extensiones finitas, y lo que haremos ahora será describir un procedimiento en sentido contrario.

\begin{ejemplo}
    Sea $Aut\left(\mathbb{Q}\left(\sqrt[3]{2}\right)\right)$, sabemos que:
    \begin{equation*}
        \left|Aut\left(\mathbb{Q}\left(\sqrt[3]{2}\right)\right)\right| \leq 3
    \end{equation*}
    Y afirmamos que es uno solo, ya que si observamos el diagrama:
    \begin{figure}[H]
        \centering
        \shorthandoff{""}
        \begin{tikzcd}
        \mathbb{Q} \arrow[r, "\iota"] \arrow[rd, "\iota"'] & {\mathbb{Q}\left(\sqrt[3]{2}\right)}                    \\
                                                           & {\mathbb{Q}\left(\sqrt[3]{2}\right)} \arrow[u, "\eta"']
        \end{tikzcd}
        \shorthandon{""}
    \end{figure}
    \noindent
    tenemos que raíces de $x^3-2$ en $\mathbb{Q}\left(\sqrt[3]{2}\right)$ solo hay 1. Sin embargo, anteriormente vimos que:
    \begin{equation*}
        \left|Aut\left(\mathbb{Q}\left(\sqrt[3]{2},w\right)\right)\right| = 6
    \end{equation*}
    Por lo que la idea intuitiva es que faltan raíces en el cuerpo.
\end{ejemplo}

\begin{definicion}
    Sea $K$ un cuerpo y $G\leq Aut(K)$ subgrupo, definimos el \underline{subcuerpo} \underline{fijo de $K$ bajo (la acción de) $G$} como el conjunto:
    \begin{equation*}
        K^G = \{a\in K : \sigma(a) = a \quad \forall \sigma\in G\}
    \end{equation*}
    Se verifica que $K^G$ es subcuerpo de $K$ (hágase), con lo que tenemos la extensión $K^G\leq K$.
\end{definicion}

\begin{prop}[Artin]
    Si $G$ es un subgrupo finito de $Aut(K)$, entonces.
    \begin{equation*}
        \left[K:K^G\right] \leq |G|
    \end{equation*}
    \begin{proof}
        Sea $n=|G|$, suponemos que $G = \{\sigma_1, \ldots, \sigma_n\}$ y tomamos $m$ (con $m>n$) elementos de $K$, $\alpha_1,\ldots,\alpha_m \in K$, basta probar que estos son $K^G-$linealmente dependientes. Para verlo, formamos la matriz:
        \begin{equation*}
            A = (\sigma_{j}(\alpha_i))_{i,j} \in  M_{m\times n}(K)
        \end{equation*}
        cuyo rango es menor o igual que $n$, luego menor o igual que $m$, es decir, existe un vector
        \begin{equation*}
            0\neq v = (v_1, \ldots, v_m) \in K^m
        \end{equation*}
        tal que $vA = 0$. Ahora, de entre todos los vectores que cumplen dichas condiciones, tomamos aquel con número de componentes no nulas mínimo y tal que alguna componente, digamos la $l-$ésima (con $1\leq l\leq m$), verifique que $v_l \in K^G$. Notemos que esto podemos conseguirlo siempre con $v_l = 1$, tras dividir todas las componentes del vector $v$ entre la $l-$ésima componente. Si escribimos la igualdad $vA = 0$:
        \begin{equation*}
            \sum_i v_i \sigma_j(\alpha_i) = 0 \qquad \forall j \in \{1,\ldots,n\}
        \end{equation*}
        Y observamos que para obtener la dependencia lineal de los $\alpha_i$ falta ver que realmente los coeficientes $v_i$ están en $K^G$ (por ahora solo sabemos que están en $K$). Para ello, supuesto que $v_{l'}\notin K^G$, tendremos que $v_{l'} \neq \sigma_k(v_{l'})$ para cierto índice $k$. Tomamos ahora cualquier $\sigma\in G$ y definimos:
        \begin{equation*}
            \sigma(v) = (\sigma(v_1,), \ldots, \sigma(v_n))
        \end{equation*}
        Si usamos esto para $\sigma_k$:
        \begin{equation*}
            \sigma_k(v) = (\sigma_k(v_1), \ldots, \sigma_k(v_m))
        \end{equation*}
        Aplicamos $\sigma_k$ a la igualdad anterior, con lo que:
        \begin{equation*}
            \sum_i \sigma_k(v_i) \sigma_k(\sigma_j(\alpha_i)) = 0 \qquad \forall j \in \{1,\ldots,n\}
        \end{equation*}
        Observemos que:
        \begin{equation*}
            G = \{\sigma_1, \ldots, \sigma_n\} = \{\sigma_k\sigma_1, \ldots, \sigma_k\sigma_n\}
        \end{equation*}
        y lo que hemos hecho ha sido permutar las ecuaciones, variando los coeficientes, con lo que:
        \begin{equation*}
            \sigma_k(v)A = 0
        \end{equation*}
        Como $vA = 0$ y $\sigma_k(v)A=0$, tenemos que:
        \begin{equation*}
            (v-\sigma_k(v))A = 0, \qquad v-\sigma_k(v) \neq 0
        \end{equation*}
        ya que si miramos las componentes $l'-$ésimas, estas son distintas. Sin embargo, las componentes $l-$ésimas son iguales, por lo. Y tenemos que $v-\sigma_k(v)$ tiene al menos una componente no nula menos que $v$, \underline{contradicción}, que viene de suponer que $v_{l'}\notin K^G$, lo que nos dice que los coeficientes $v_i$ estaban en $K^G$. Si en la igualdad: 
        \begin{equation*}
            \sum_i v_i \sigma_j(\alpha_i) = 0 \qquad \forall j \in \{1,\ldots,n\}
        \end{equation*}
        tomamos aquél índice $j$ que verifica que $\sigma_j = Id_K$, tendremos entonces que:
        \begin{equation*}
            \sum_{i}v_i \alpha_i = 0, \qquad v_i \in K^G
        \end{equation*}
        lo que implica que $\alpha_1, \ldots, \alpha_m$ eran $K^G-$linealmente dependientes, por lo que:
        \begin{equation*}
            [K:K^G] \leq n = |G|
        \end{equation*}
    \end{proof}
\end{prop}

\begin{lema}
    Para un cuerpo $K$, tenemos que:
    \begin{enumerate}
        \item Si $H\subseteq G$ son subgrupos de $Aut(K)$, entonces $K^H \geq K^G$. % Hay menos cosas que comprobar
        \item Si $F\leq E$ son subcuerpos de $K$, entonces $Aut_F(K)\supseteq Aut_E(K)$.
        \item Si $G$ es subgrupo de $Aut(K)$, entonces $G\subseteq Aut_{K^G}(K)$.
        \item Si $F\leq K$, entonces $F\leq F^{Aut_F(K)}$.
    \end{enumerate}
    \begin{proof}
        Demostramos cada uno de los apartados de forma muy sencilla:
        \begin{enumerate}
            \item Sea $a\in K^G$, si tomamos $\sigma\in H\subseteq G$, tendremos que $\sigma(a) = a$, con lo que $a\in K^H$.
            \item Sea $\sigma\in Aut_E(K)$, si tomamos $\lm \in F\leq E, x\in K$ observamos que:
                \begin{equation*}
                    \sigma(\lm \cdot x) = \lm\cdot \sigma(x)
                \end{equation*}
                Por lo que $\sigma\in Aut_F(K)$.
            \item Sea $\sigma\in G\subseteq Aut(K)$, si tomamos $x\in K$ y $y\in K^G$, observamos que:
                \begin{equation*}
                    \sigma(y\cdot x) = \sigma(y)\cdot \sigma(x) = y\cdot \sigma(x)
                \end{equation*}
                Por lo que $\sigma\in Aut_{K^G}(K)$.
            \item Sea $x\in F$ y $\sigma \in Aut_F(K)$, entonces:
                \begin{equation*}
                    \sigma(x) = \sigma(x\cdot 1) = x\cdot \sigma(1) = x
                \end{equation*}
                Por lo que $x\in F^{Aut_F(K)}$.
        \end{enumerate}
    \end{proof}
\end{lema}

\noindent
Veamos ahora dónde se da la igualdad en los apartados 2 y 3, que en general no se dan.

\begin{teo}\label{teo:antes_teo_angular}
    Sea $K$ un cuerpo, si $G$ es un subgrupo finito de $Aut(K)$, entonces:
    \begin{equation*}
        [K:K^G] = |G| \qquad \text{y}\qquad G = Aut_{K^G}(K)
    \end{equation*}
    \begin{proof}
        El Lema anterior nos dice que $G\leq Aut_{K^G}(K)$, y el Lema de Artin nos dice que $[K:K^G]\leq |G|$, con lo que en particular la extensión es finita, luego:
        \begin{equation*}
            |G| \leq |Aut_{K^G}(K)| \leq [K:K^G] \leq |G|
        \end{equation*}
        Por lo que $G = Aut_{K^G}(K)$.
    \end{proof}
\end{teo}

% // TODO: PASAR ESTO A LIMPIO

\begin{ejemplo}
    Sea $K = \mathbb{Q}(\sqrt[3]{2},w)$ con $w$ una raíz cúbica primitiva de la unidad, sabemos ya:
    \begin{equation*}
        Aut(K) = \{\eta_{j,k} : j\in \{0,1,2\}, k \in \{1,2\}\}
    \end{equation*}

    donde:
    \begin{equation*}
        \eta_{j,k}(\sqrt[3]{2}) = w^j \sqrt[3]{2} \qquad \eta_{j,k}(w) = w^k
    \end{equation*}
    Los subgrupos propios de $Aut(K)$ (por el Teorema de Lagrange) son de orden 2 o 3, todos ellos cíclicos, por lo que tenemos que buscar elementos de orden 2 y 3. Son:
    \begin{equation*}
        \langle \eta_{1,1} \rangle  \cong \langle \eta_{2,1} \rangle, \qquad \langle \eta_{0,2} \rangle  \cong \langle \eta_{1,2} \rangle  \cong \langle \eta_{2,2} \rangle 
    \end{equation*}
    Que hemos obtenido ya que por ejemplo:
    \begin{align*}
        \sqrt[3]{2} &\stackrel{\eta_{0,2}}{\longmapsto} \sqrt[3]{2} \\
        w &\longmapsto w^2 \longmapsto w^4 = w \\
        \\
        \sqrt[3]{2} &\stackrel{\eta_{1,2}}{\longmapsto} w\sqrt[3]{2} \stackrel{\eta_{1,2}}{\longmapsto} w^2 w\sqrt[3]{2} = \sqrt[3]{2} \\
        w &\longmapsto w^2 \longmapsto w 
    \end{align*}
    Si el grupo fuera cíclico, tendríamos un único subgrupo por cada divisor, pero como hemos encontrado dos elementos de orden 2 sabemos que no es cíclico.
    \begin{equation*}
        \sqrt[3]{2} \stackrel{\eta_{1,1}}{\longmapsto} w\sqrt[3]{2} \stackrel{\eta_{1,1}}{\longmapsto} ww\sqrt[3]{2} = w^2\sqrt[3]{2} \neq 1
    \end{equation*}
    hemos encontrado un elemento de orden que no es 2, por lo que ha de ser de orden 3 (puesto que no hay elementos de orden 6). Para calcular el segundo elemento de orden 3 calculamos el cuadrado a $\eta_{1,1}$, obteniendo el $\eta_{2,1}$. Finalmente, tenemos el elemento $\eta_{2,2}$, que automáticamente sabemos que es de orden 2, puesto que es el que queda.\\

    \noindent
    Buscamos ahora calcular $K^{\langle \eta_{1,1} \rangle }$, y sabemos que:
    \begin{equation*}
        \left[K:K^{\langle \eta_{1,1} \rangle }\right] = |\langle \eta_{1,1} \rangle | = 3
    \end{equation*}
    Por lo que aplicando el Lema de la torre:
    \begin{equation*}
        [K^{\eta_{1,1}}:\mathbb{Q}] = 2
    \end{equation*}
    buscamos una extensión de grado 2 de $\mathbb{Q}$ que esté dentro de $Aut(K)$. Heurísticamente, conocemos que $[\mathbb{Q}(w):\mathbb{Q}] = 2$, con lo que buscamos razonar que $K^{\langle \eta_{1,1} \rangle } = \mathbb{Q}(w)$, comprobémoslo: sabemos que $\eta_{1,1}(w) = w$, por lo que $w\in K^{\langle \eta_{1,1} \rangle }$, lo que implica que $\mathbb{Q}(w)\leq K^{\langle \eta_{1,1} \rangle }$. Además, como la extensión es 2 en ambos casos, ha de ser por tanto $\mathbb{Q}(w) = K^{\langle \eta_{1,1} \rangle }$.

    \noindent
    Se pide calcular $K^{\langle \eta_{2,2} \rangle }, K^{\langle \eta_{1,2} \rangle }, K^{\langle \eta_{2,2} \rangle }$ , para ello buscaremos extensiones de grado 3 en $\mathbb{Q}$, un elemento de grado 3 es $\sqrt[3]{2}$, otro será $w\sqrt[3]{2}$ y otro $w^2\sqrt[3]{2}$.
\end{ejemplo}

\begin{definicion}[Polinomio separable]
    Sea $f\in F[x]$ con $degf \geq 1$, se dice que $f$ es separable si todas sus raíces (en un cuerpo de descomposición de $f$) son simples.
\end{definicion}

\begin{observacion}
    Equivalentemente, un polinomio es separable si: 
    \begin{itemize}
        \item tiene $degf$ raíces distintas en su cuerpo de descomposición.
        \item $\mcd (f,f') = 1$.
    \end{itemize}
\end{observacion}

\begin{ejemplo}
    Para mostrar la abundancia de polinomios separables así como la existencia de polinomios no separables:
    \begin{itemize}
        \item Si $F$ es un cuerpo con $car(F)= 0$ y $f$ es irreducible, entonces $f$ es separable.

            Como $car(F)=0$ y $degf\geq 1$, tenemos al ser $f$ no constante que $f'\neq 0$, y como $f$ es irreducible tendremos que $\mcd(f,f')=1\neq 0$, de donde $f$ es separable.
        \item Sea $f=x^q-x\in \bb{F}_p[x]$, donde $q=p^n$, tenemos que $f$ es separable.

            Como $f'=qx^{q-1}-1 = -1\neq 0$, tenemos que $\mcd(f,f')=1$, por lo que $f$ es separable.
        \item Sea $\bb{F}_p(t)$ el cuerpo de fracciones del anillo de polinomios $\bb{F}_p[t]$, si consideramos el polinomio:
            \begin{equation*}
                f = x^p-t\in \bb{F}_p(t)[x]
            \end{equation*}
            tenemos que $f$ es irreducible (por Eisenstein para $t$) y que $f'=0$, con lo que $\mcd(f,f')=f\neq 1$, luego $f$ no es separable.
    \end{itemize}
\end{ejemplo}

\begin{definicion}[Extensión separable]
    Una extensión algebraica $F\leq K$ se dice separable si $Irr(\alpha,F)$ es separable, para todo $\alpha\in K$.
\end{definicion}

\begin{observacion}
    Toda extensión algebraica en característica $0$ es separable.
\end{observacion}

\begin{definicion}[Extensión normal]
    Una extensión algebraica $F\leq K$ se dice normal si $Irr(\alpha,F)$ se factoriza como producto de polinomios lineales en $K[x]$, para todo $\alpha\in K$.
\end{definicion}

\begin{ejemplo}
    Por ejemplo, la extensión $\mathbb{Q}\leq \mathbb{Q}(\sqrt[3]{2})$ no es normal pero sí es separable.
\end{ejemplo}

% // TODO: Comentar esto luego
% Lo único que sabemos por ahora es comprobar i.
% ii tiene que ver con la conexion de galois, ¿qué le hace falta a una extensión finita para estar en biyeccion con el grupo correspondiente?
% iii dice que si encontramos el grupo G entonces sabemos quién es F.
% una extensión de estas cumple que al comprobar i tienes iv

% // TODO: TEOREMA IMPORTANTE
\begin{teo}
    Sea $F\leq K$ una extensión de cuerpos. Son equivalentes: 
    \begin{enumerate}
        \item[$i)$] $K$ es cuerpo de descomposición de un $f\in F[x]$ separable.
        \item[$ii)$] $F\leq K$ es finita y $F = K^{Aut_F(K)}$.
        \item[$iii)$] $F = K^G$ para un subgrupo finito $G$ de $Aut(K)$.
        \item[$iv)$] $F\leq K$ es finita, normal y separable.
    \end{enumerate}
    \begin{proof}
        Veamos las equivalencias:
        \begin{description}
            \item [$i)\Longrightarrow  ii)$] Como $K$ es cuerpo de descomposición de cierto $f\in F[x]$, tenemos entonces que si $\alpha_1, \ldots, \alpha_s$ son las raíces de $f$ entonces:
                \begin{equation*}
                    K = F(\alpha_1, \ldots, \alpha_s)
                \end{equation*}
                Por lo que $F\leq K$ es finitamente generada. Si observamos ahora la demostración del Teorema~\ref{teo:finita_algebraica} observamos que solo usaba que los $\alpha_i$ eran algebraicos, por lo que podemos concluir que $F\leq K$ es finita.

                Sea $F' = K^{Aut_F(K)}$, es claro que $F\leq F'$. Además, como $F\leq K$ es finita, tendremos que $Aut_F(K)$ es finito. Por el Teorema~\ref{teo:antes_teo_angular} tenemos que tomando $G = Aut_F(K)$, se tiene que:
                \begin{equation*}
                    Aut_F(K) = Aut_{F'}(K)
                \end{equation*}
                $K$ es cuerpo de descomposición de $f\in F[x]$ y como $F\leq F'$, tenemos también que $K$ es cuerpo de descomposición de $f\in F'[x]$. Como $f$ es separable, tenemos que:
                \begin{equation*}
                    [K:F] = |Aut_F(K)| = |Aut_{F'}(K)| = [K:F']
                \end{equation*}
                Con $F\leq F'$, por lo que el Lema de la Torre nos dice que $F = F'$
            \item [$ii)\Longrightarrow iii)$] Si la extensión es finita, tenemos entonces que $Aut_F(K)$ es finita, con lo que tomando $G= Aut_F(G)$, tenemos que $F=K^G$.
            \item [$iii)\Longrightarrow iv)$] La Proposición de Artin nos dice que $K^G = F\leq K$ es finita. 

                Sean $\alpha\in K$ y $h = Irr(\alpha,F)\in F[x]$, como $G$ actúa sobre $K$, podemos considerar la órbita de $\alpha$ (considerando todos sus elementos distintos):
                \begin{equation*}
                    Orb(\alpha) = \{\alpha_1, \ldots, \alpha_t\} \subseteq K
                \end{equation*}
                y podemos considerar el polinomio:
                \begin{equation*}
                    g = \prod_{i=1}^{t}(x-\alpha_i) = \sum_{j=0}^{t} a_j x^j \in K[x]
                \end{equation*}
                veamos que $a_j \in F$ para todo $j\in \{1,\ldots,t\}$, usando que $F=K^G$. Dado $\sigma\in G$:
                \begin{equation*}
                    \prod_{i=1}^{t} (x-\sigma(\alpha_i)) = g^\sigma = \sum_{j=0}^{t} \sigma(a_j) x^j
                \end{equation*}
                y vemos que $g = \prod\limits_{i=1}^{t}(x-\sigma(\alpha_i))$, puesto que al aplicar $\sigma$ sobre los elementos de la órbita los permuta, con lo que de la igualdad de la derecha deducimos que $\sigma(a_j) = a_j$, para todo $j\in \{1,\ldots,t\}$, con lo que $a_j\in F[x]$ para todo $j \in \{1,\ldots,t\}$, luego $g\in F[x]$.

                Por una parte $g(\alpha) = 0$, puesto que $\alpha\in Orb(\alpha)$. Como $h=Irr(\alpha,F)$, tenemos que $h$ divide a $g$.

                Por otra parte, cada $\alpha_i$ es raíz de $h$, ya que $h(\alpha) = 0$ deducimos que si tomamos $\sigma\in G$, entonces:
                \begin{equation*}
                    0 = \sigma(0) = \sigma(h(\alpha)) = h(\alpha_i)
                \end{equation*}
                Como se cumple para todo $\sigma\in G$, tenemos pues que $h(\alpha_i) = 0$ para todo $i \in \{1,\ldots,t\}$. Como los elementos $\alpha_i$ son distintos, tenemos que $deg h \geq t$, pero como $g$ es un polinomio mónico de grado $t$ cuyas raíces son exactamente $Orb(\alpha)$, tenemos que $g=h$. Hemos probado que la extensión es normal y separable.
            \item [$iv)\Longrightarrow i)$] Como $F\leq K$ es finita, tenemos entonces que existen $\alpha_1, \ldots, \alpha_s\in K$ algebraicos de forma que $K = F(\alpha_1, \ldots, \alpha_s)$. Podemos por tanto considerar $f_i = Irr(\alpha_i, F)$, y tomamos como $f$ el producto de los $f_i$ eliminando repeticiones (es decir, multiplicamos todos los $f_i$ distintos). Como la extensión es normal y separable, cada uno de los $f_i$ se descompone como producto de polinomios lineales mónicos distintos. De donde $f$ es un polinomio separable\footnote{Notemos que para eso eliminamos antes las repeticiones.}, por lo que $K$ es un cuerpo de descomposición de $f$.
        \end{description}
    \end{proof}
\end{teo}

En definitiva, dado un cuerpo $K = F(\alpha_1, \ldots, \alpha_s)$ y queremos comprobar que es de Galois, lo que hacemos es repetir la demostración $iv)\Longrightarrow i)$.

\begin{definicion}
    La órbita de $\alpha$ bajo $G$ que ha aparecido en la demostración, $\{\alpha_1, \ldots, \alpha_s\}$ se llaman conjugados de $\alpha$ bajo $G$.
\end{definicion}

\noindent
Se trata de la generalización del concepto ``conjugado'' de un número complejo.

\begin{definicion}[Extensión de Galois]
    Una extensión $F\leq K$ se dice de Galois si es finita, normal y separable.
\end{definicion}

\begin{coro}
    En característica $0$, si $K$ es cuerpo de descomposición de $f\in F[x]$, entonces $F\leq K$ es de Galois.
    \begin{proof}
        Consideramos la descomposición de $f$ en irreducibles:
        \begin{equation*}
            f = p_1^{n_1}\cdot \ldots \cdot p_t^{n_t}
        \end{equation*}
        con $p_i$ distintos. Obsevemos que $K$ es cuerpo de descomposición de $p_1\cdot \ldots\cdot p_t$, que sí es separable, puesto que cada polinomio irreducible es separable. Por el Teorema anterior, la extensión es finita, normal y separable.
    \end{proof}
\end{coro}

\begin{coro}
    Si $F\leq K$ es de Galois y $F\leq E\leq K$ es una subextensión, entonces $E\leq K$ es de Galois.
    \begin{proof}
        Como $F\leq K$ es de Galois, entonces $K$ es cuerpo de descomposición de cierto $f\in F[x]$ separable, por lo que $K$ es cuerpo de descomposición de $f\in E[x]$, que sigue siendo separable.
    \end{proof}
\end{coro}

% // F <= E no tiene por qué ser de Galois

\begin{ejemplo}
    Si consideramos $\mathbb{Q}\leq \mathbb{Q}\left(\sqrt[3]{2}\right)$, tenemos una extensión finita, separable pero no es de Galois, porque no es normal. Sin embargo, $\mathbb{Q}\leq \mathbb{Q}\left(\sqrt[3]{2},w\right)$ sí que es de Galois. En consecuencia, $\mathbb{Q}\leq \mathbb{Q}\left(\sqrt[3]{2},w\right)$.\\

    \noindent
    Sabemos que $\mathbb{Q}\leq \mathbb{Q}\left(\sqrt[3]{2}\right)$ no es normal porque $Irr\left(\sqrt[3]{2},\mathbb{Q}\right) = x^3-2$ no se descompone como producto de polinomios lineales en $\mathbb{Q}\left(\sqrt[3]{2}\right)$, ya que $w\sqrt[3]{2}$ es una raíz del polinomio que no está en $\mathbb{Q}\left(\sqrt[3]{2}\right)$.
\end{ejemplo}

\begin{coro}
    Toda extensión de cuerpos finitos es de Galois.
    \begin{proof}
        Si tenemos una extensión $F\leq E$ de cuerpos finitos de característica $car(F)=p$, tenemos entonces que:
        \begin{equation*}
            \bb{F}_p\leq F \leq E
        \end{equation*}
        con $\bb{F}_p\leq E$ de Galois, puesto que $x^q-x \in \bb{F}_q[x]$ con $q=p^n = |E|$ es separable y $E$ es un cuerpo de descomposición suyo.
    \end{proof}
\end{coro}

\begin{ejemplo}
    Consideramos $\mathbb{Q}\leq E=\mathbb{Q}\left(\sqrt[3]{5},i\sqrt{5}\right)$, que es una extensión finita, con (por el Lema de la Torre) $[E:\mathbb{Q}] = 6$. Si esta extensión fuera de Galois, entonces la raíz $w\sqrt[3]{5}$ de $x^3-5=Irr(\sqrt[3]{5},\mathbb{Q})$ estaría en $E$, para $w = \frac{-1}{2} + i\frac{\sqrt{3}}{2}$.\\

    \noindent
    En dicho caso, $i\sqrt{3}\in E$, luego $\mathbb{Q}\left(i\sqrt{3},i\sqrt{5}\right)\leq E$. Buscamos calcular:
    \begin{equation*}
        \left[\mathbb{Q}\left(i\sqrt{3},i\sqrt{5}\right):\mathbb{Q}\right]
    \end{equation*}
    Sabemos que $\left[\mathbb{Q}\left(i\sqrt{3}\right):\mathbb{Q}\right] = 2$, así como que $\left[\mathbb{Q}\left(i\sqrt{5},i\sqrt{3}\right):\mathbb{Q}\left(i\sqrt{3}\right)\right]\leq 2$:
    \begin{itemize}
        \item Si $\left[\mathbb{Q}\left(i\sqrt{5},i\sqrt{3}\right)\right]=1$, esto es porque $i\sqrt{5}\in \mathbb{Q}\left(i\sqrt{3}\right)$. En dicho caso, tendríamos que:
            \begin{equation*}
                i\sqrt{5} = a+bi\sqrt{3} \qquad a,b\in \mathbb{Q}
            \end{equation*}
            de donde $a=0$, con lo que $i\sqrt{5}=bi\sqrt{3}$, y elevando al cuadrado tendríamos que:
            \begin{equation*}
                -5 = -3b^2
            \end{equation*}
            de donde $b\in \mathbb{Q}$ es raíz de $3x^2-5$, pero:
            \begin{description}
                \item [Opción 1.] $3x^2-5$ es irreducible por Eisenstein (notemos que es primitivo).
                \item [Opción 2.] Las posibles raíces racionales del polinomio enunciado son ciertas y ninguna es racional.
            \end{description}
        \item Tenemos por tanto que $\left[\mathbb{Q}\left(i\sqrt{5},i\sqrt{3}\right):\mathbb{Q}\left(i\sqrt{3}\right)\right] = 2$, y por el lema de la torre tenemos que $\left[\mathbb{Q}\left(i\sqrt{3},i\sqrt{5}\right):\mathbb{Q}\right] =4$, de donde $4$ divide a $6 = [E:\mathbb{Q}]$.
    \end{itemize}
\end{ejemplo}

\begin{notacion} % // TODO: POner esto en no cursiva
    Notaremos:
    \begin{itemize}
        \item Si $F\leq K$ es una extensión y $F\leq E \leq K$ se dice que $E$ es una \underline{subextensión} de $F\leq K$. Denotamos al conjunto de todas ellas por $Subex(F\leq K)$.
        \item Si $G$ es un grupo, llamamos $\Subgr(G)$ al conjunto de todos los subgrupos de $G$.
        \item Si $H\in \Subgr(G)$, denotamos por $(G:H)$ al índice de $H$ en $G$.
    \end{itemize}
\end{notacion}

% Un anti-isomorfismos es una biyeccion donde tomamos los órdenes de la inclusion
% anti significa que vuelve del revés el orden

\begin{teo} % // TODO: Poner que es el grupo de galois
    Sea $F\leq K$ una extensión de Galois con grupo de Galois $Aut_F(K) = G$. La aplicación 
    \Func{}{\Subgr(G)}{\Subex(F\leq K)}{H}{K^H}
    es un anti-isomorfismo de conjuntos ordenados cuya inversa es 
    \Func{}{\Subex(F\leq K)}{\Subgr(G)}{E}{Aut_E(K)}
    Si $H_1\subseteq H_2$ son subgrups de $G$ y $E_2\leq E_1$ son subextensiones de $F\leq K$ correspondientes según la anterior biyección, entonces:
    \begin{equation*}
        (H_2:H_1) = [E_1:E_2]
    \end{equation*}
\end{teo}



