\chapter{Extensiones de Galois}
\begin{prop}
    Sea $F\leq K$ una extensión finita, entonces $|Aut_F(K)| \leq [K:F]$.
    % // TODO: HACER
\end{prop}

\begin{ejemplo}
    Sea $Aut(\mathbb{Q}(\sqrt[3]{2}))$, sabemos que:
    \begin{equation*}
        |Aut(\mathbb{Q}(\sqrt[3]{2}))| \leq 3
    \end{equation*}
    Y afirmamos que es uno solo, ya que:
    \begin{figure}[H]
        \centering
        \shorthandoff{""}
        \begin{tikzcd}
        \mathbb{Q} \arrow[r, "\iota"] \arrow[rd, "\iota"'] & {\mathbb{Q}(\sqrt[3]{2})}                    \\
                                                           & {\mathbb{Q}(\sqrt[3]{2})} \arrow[u, "\eta"']
        \end{tikzcd}
        \shorthandon{""}
    \end{figure}
    \noindent
    y raíces de $x^3-2$ en dicho cuerpo solo hay 1. Sin embargo, anteriormente vimos que:
    \begin{equation*}
        |Aut(\mathbb{Q}(\sqrt[3]{2}))| = 6
    \end{equation*}
    Por lo que la idea intuitiva es que faltan raíces en el cuerpo.
\end{ejemplo}

\begin{definicion}
    Sea $K$ un cuerpo y $G\leq Aut(K)$ subgrupo, definimos el \underline{subcuerpo} \underline{fijo de $K$ bajo (la acción de) $G$}:
    \begin{equation*}
        K^G = \{a\in K : \sigma(a) = a \quad \forall \sigma\in G\}
    \end{equation*}
    Se verifica que $K^G$ es subcuerpo de $K$, con lo que tenemos la extensión $K^G\leq K$. % // TODO: HACER
\end{definicion}

\begin{prop}[Artin]
    Si $G$ es un subgrupo finito de $Aut(K)$, entonces.
    \begin{equation*}
        \left[K:K^G\right] \leq |G|
    \end{equation*}
    \begin{proof}
        Sea $n=|G|$, suponemos $G = \{\sigma_1, \ldots, \sigma_2\}$ y tomamos elementos $\alpha_1,\ldots,\alpha_m \in K$ con $m>n$, basta probar que los elementos $\alpha_i$ son linealmente dependientes. Formamos la matriz:
        \begin{equation*}
            A = (\sigma_{j}(\alpha_i))_{i,j} \in  M_{m\times n}(K)
        \end{equation*}
        cuyo rango es menor o igual que $n$, luego menor o igual que $m$, es decir, existe un vector
        \begin{equation*}
            0\neq v = (v_1, \ldots, v_m) \in K^m
        \end{equation*}
        tal que $vA = 0$. Ahora, de entre todos los vectores que cumplen dichas condiciones, tomamos aquel con número de componentes no nulas mínimo y tal que alguna componente, digamos $v_l \in K^G$. De hecho, podemos tomar $v_l = 1$. Si escribimos la igualdad $vA = 0$:
        \begin{equation*}
            \sum_i v_i \sigma_j(\alpha_i) = 0 \qquad \forall j \in \{1,\ldots,n\}
        \end{equation*}
        Para obtener la dependencia lineal falta ver que realmente los coeficientes $v_i$ están en $K^G$ (por ahora solo sabemos que están en $K$). Supuesto que algún coeficiente $v_{l'} \neq \sigma_k(v_{l'})$ para alguna pareja de índices $l,k$, tomamos cualquier $\sigma\in G$, con lo que:
        \begin{equation*}
            \sigma(v) = (\sigma(v_1,), \ldots, \sigma(v_n))
        \end{equation*}
        Tenemos que:
        \begin{equation*}
            \sigma_k(v) = (\sigma_k(v_1), \ldots, \sigma_k(v_m))
        \end{equation*}
        Aplicamos $\sigma_k$ a la igualdad anterior, con lo que:
        \begin{equation*}
            \sum_i \sigma_k(v_i) \sigma_k(\sigma_j(\alpha_i)) = 0 \qquad \forall j \in \{1,\ldots,n\}
        \end{equation*}
        Observemos que:
        \begin{equation*}
            G = \{\sigma_1, \ldots, \sigma_n\} = \{\sigma_k\sigma_1, \ldots, \sigma_k\sigma_n\}
        \end{equation*}
        y lo que hemos hecho ha sido permutar las ecuaciones, variando los coeficientes, con lo que:
        \begin{equation*}
            \sigma_k(v)A = 0
        \end{equation*}
        Tenemos pues que:
        \begin{equation*}
            (v-\sigma_k(v))A = 0
        \end{equation*}
        Además:
        \begin{equation*}
            v-\sigma_k(v) \neq 0
        \end{equation*}
        ya que si miramos la componentes $l'-$ésima, estas son distintas. Sin embargo, las componentes $l-$ésimas son iguales. Y tenemos que $v-\sigma_k(v)$ tiene al menos una componente no nula menos que $v$, \underline{contradicción}, que viene de suponer que $v_{l'}\neq \sigma_k(v_{l'})$, lo que nos dice que realmente los coeficientes $v_i$ estaban en $K^G$, tomamos $\sigma_j = id\in G$, con lo que:
        \begin{equation*}
            \sum_{i}v_i \alpha_i = 0
        \end{equation*}
        lo que implica que $\alpha_1, \ldots, \alpha_m$ eran linealmente dependientes, por lo que:
        \begin{equation*}
            [K:K^G] \leq n = |G|
        \end{equation*}
    \end{proof}
\end{prop}

\begin{lema} % // TODO: HACER DEMO
    Para un cuerpo $K$, tenemos que:
    \begin{enumerate}
        \item Si $H\subseteq G$ son subgrupos de $Aut(K)$, entonces $K^H \geq K^G$. % Hay menos cosas que comprobar
        \item Si $F\leq E$ son subcuerpos de $K$, entonces $Aut_F(K)\supseteq Aut_E(K)$.
        \item Si $G$ es subgrupo de $Aut(K)$, entonces $G\subseteq Aut_{K^G}(K)$.
        \item Si $F\leq K$, entonces $F\leq F^{Aut_F(K)}$.
    \end{enumerate}
\end{lema}
Veamos ahora dónde se da la igualdad en los apartados 2 y 3, que en general no se dan.

\begin{teo}
    Si $G$ es un subgrupo finito de $Aut(K)$ para un cuerpo $K$, entonces:
    \begin{equation*}
        [K:K^G] = |G| \qquad \text{y}\qquad G = Aut_{K^G}(K)
    \end{equation*}
    \begin{proof}
        El Lema anterior nos dice que $G\leq Aut_{K^G}(K)$, y el Lema de Artin nos dice que $[K:K^G]\leq |G|$, con lo que en particular la extensión es finita, luego:
        \begin{equation*}
            |G| \leq |Aut_{K^G}(K)| \leq [K:K^G] \leq |G|
        \end{equation*}
    \end{proof}
\end{teo}

\begin{ejemplo}
    Sea $K = \mathbb{Q}(\sqrt[3]{2},w)$ con $w$ una raíz cúbica primitiva de la unidad, sabemos ya:
    \begin{equation*}
        Aut(K) = \{\eta_{j,k} : j\in \{0,1,2\}, k \in \{1,2\}\}
    \end{equation*}

    donde:
    \begin{equation*}
        \eta_{j,k}(\sqrt[3]{2}) = w^j \sqrt[3]{2} \qquad \eta_{j,k}(w) = w^k
    \end{equation*}
    Los subgrupos propios de $Aut(K)$ (por el Teorema de Lagrange) son de orden 2 o 3, todos ellos cíclicos, por lo que tenemos que buscar elementos de orden 2 y 3. Son:
    \begin{equation*}
        \langle \eta_{1,1} \rangle  \cong \langle \eta_{2,1} \rangle, \qquad \langle \eta_{0,2} \rangle  \cong \langle \eta_{1,2} \rangle  \cong \langle \eta_{2,2} \rangle 
    \end{equation*}
    Que hemos obtenido ya que por ejemplo:
    \begin{align*}
        \sqrt[3]{2} &\stackrel{\eta_{0,2}}{\longmapsto} \sqrt[3]{2} \\
        w &\longmapsto w^2 \longmapsto w^4 = w \\
        \\
        \sqrt[3]{2} &\stackrel{\eta_{1,2}}{\longmapsto} w\sqrt[3]{2} \stackrel{\eta_{1,2}}{\longmapsto} w^2 w\sqrt[3]{2} = \sqrt[3]{2} \\
        w &\longmapsto w^2 \longmapsto w 
    \end{align*}
    Si el grupo fuera cíclico, tendríamos un único subgrupo por cada divisor, pero como hemos encontrado dos elementos de orden 2 sabemos que no es cíclico.
    \begin{equation*}
        \sqrt[3]{2} \stackrel{\eta_{1,1}}{\longmapsto} w\sqrt[3]{2} \stackrel{\eta_{1,1}}{\longmapsto} ww\sqrt[3]{2} = w^2\sqrt[3]{2} \neq 1
    \end{equation*}
    hemos encontrado un elemento de orden que no es 2, por lo que ha de ser de orden 3 (puesto que no hay elementos de orden 6). Para calcular el segundo elemento de orden 3 calculamos el cuadrado a $\eta_{1,1}$, obteniendo el $\eta_{2,1}$. Finalmente, tenemos el elemento $\eta_{2,2}$, que automáticamente sabemos que es de orden 2, puesto que es el que queda.\\

    \noindent
    Buscamos ahora calcular $K^{\langle \eta_{1,1} \rangle }$, y sabemos que:
    \begin{equation*}
        \left[K:K^{\langle \eta_{1,1} \rangle }\right] = |\langle \eta_{1,1} \rangle | = 3
    \end{equation*}
    Por lo que aplicando el Lema de la torre:
    \begin{equation*}
        [K^{\eta_{1,1}}:\mathbb{Q}] = 2
    \end{equation*}
    buscamos una extensión de grado 2 de $\mathbb{Q}$ que esté dentro de $Aut(K)$. Heurísticamente, conocemos que $[\mathbb{Q}(w):\mathbb{Q}] = 2$, con lo que buscamos razonar que $K^{\langle \eta_{1,1} \rangle } = \mathbb{Q}(w)$, comprobémoslo: sabemos que $\eta_{1,1}(w) = w$, por lo que $w\in K^{\langle \eta_{1,1} \rangle }$, lo que implica que $\mathbb{Q}(w)\leq K^{\langle \eta_{1,1} \rangle }$. Además, como la extensión es 2 en ambos casos, ha de ser por tanto $\mathbb{Q}(w) = K^{\langle \eta_{1,1} \rangle }$.

    \noindent
    Se pide calcular $K^{\langle \eta_{2,2} \rangle }, K^{\langle \eta_{1,2} \rangle }, K^{\langle \eta_{2,2} \rangle }$ , para ello buscaremos extensiones de grado 3 en $\mathbb{Q}$, un elemento de grado 3 es $\sqrt[3]{2}$, otro será $w\sqrt[3]{2}$ y otro $w^2\sqrt[3]{2}$.
\end{ejemplo}

