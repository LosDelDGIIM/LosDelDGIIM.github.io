\documentclass[12pt]{book}
% Idioma y codificación
\usepackage[spanish, es-tabla, es-notilde]{babel}       %es-tabla para que se titule "Tabla"
\usepackage[utf8]{inputenc}

% Márgenes
\usepackage[a4paper,top=3cm,bottom=2.5cm,left=3cm,right=3cm]{geometry}

% Comentarios de bloque
\usepackage{verbatim}

% Paquetes de links
\usepackage[hidelinks]{hyperref}    % Permite enlaces
\usepackage{url}                    % redirecciona a la web

% Más opciones para enumeraciones
\usepackage{enumitem}

% Personalizar la portada
\usepackage{titling}

% Paquetes de tablas
\usepackage{multirow}

% Para añadir el símbolo de euro
\usepackage{eurosym}


%------------------------------------------------------------------------

%Paquetes de figuras
\usepackage{caption}
\usepackage{subcaption} % Figuras al lado de otras
\usepackage{float}      % Poner figuras en el sitio indicado H.


% Paquetes de imágenes
\usepackage{graphicx}       % Paquete para añadir imágenes
\usepackage{transparent}    % Para manejar la opacidad de las figuras

% Paquete para usar colores
\usepackage[dvipsnames, table, xcdraw]{xcolor}
\usepackage{pagecolor}      % Para cambiar el color de la página

% Habilita tamaños de fuente mayores
\usepackage{fix-cm}

% Para los gráficos
\usepackage{tikz}
\usepackage{forest}

% Para poder situar los nodos en los grafos
\usetikzlibrary{positioning}


%------------------------------------------------------------------------

% Paquetes de matemáticas
\usepackage{mathtools, amsfonts, amssymb, mathrsfs}
\usepackage[makeroom]{cancel}     % Simplificar tachando
\usepackage{polynom}    % Divisiones y Ruffini
\usepackage{units} % Para poner fracciones diagonales con \nicefrac

\usepackage{pgfplots}   %Representar funciones
\pgfplotsset{compat=1.18}  % Versión 1.18

\usepackage{tikz-cd}    % Para usar diagramas de composiciones
\usetikzlibrary{calc}   % Para usar cálculo de coordenadas en tikz

%Definición de teoremas, etc.
\usepackage{amsthm}
%\swapnumbers   % Intercambia la posición del texto y de la numeración

\theoremstyle{plain}

\makeatletter
\@ifclassloaded{article}{
  \newtheorem{teo}{Teorema}[section]
}{
  \newtheorem{teo}{Teorema}[chapter]  % Se resetea en cada chapter
}
\makeatother

\newtheorem{coro}{Corolario}[teo]           % Se resetea en cada teorema
\newtheorem{prop}[teo]{Proposición}         % Usa el mismo contador que teorema
\newtheorem{lema}[teo]{Lema}                % Usa el mismo contador que teorema
\newtheorem*{lema*}{Lema}

\theoremstyle{remark}
\newtheorem*{observacion}{Observación}

\theoremstyle{definition}

\makeatletter
\@ifclassloaded{article}{
  \newtheorem{definicion}{Definición} [section]     % Se resetea en cada chapter
}{
  \newtheorem{definicion}{Definición} [chapter]     % Se resetea en cada chapter
}
\makeatother

\newtheorem*{notacion}{Notación}
\newtheorem*{ejemplo}{Ejemplo}
\newtheorem*{ejercicio*}{Ejercicio}             % No numerado
\newtheorem{ejercicio}{Ejercicio} [section]     % Se resetea en cada section


% Modificar el formato de la numeración del teorema "ejercicio"
\renewcommand{\theejercicio}{%
  \ifnum\value{section}=0 % Si no se ha iniciado ninguna sección
    \arabic{ejercicio}% Solo mostrar el número de ejercicio
  \else
    \thesection.\arabic{ejercicio}% Mostrar número de sección y número de ejercicio
  \fi
}


% \renewcommand\qedsymbol{$\blacksquare$}         % Cambiar símbolo QED
%------------------------------------------------------------------------

% Paquetes para encabezados
\usepackage{fancyhdr}
\pagestyle{fancy}
\fancyhf{}

\newcommand{\helv}{ % Modificación tamaño de letra
\fontfamily{}\fontsize{12}{12}\selectfont}
\setlength{\headheight}{15pt} % Amplía el tamaño del índice


%\usepackage{lastpage}   % Referenciar última pag   \pageref{LastPage}
%\fancyfoot[C]{%
%  \begin{minipage}{\textwidth}
%    \centering
%    ~\\
%    \thepage\\
%    \href{https://losdeldgiim.github.io/}{\texttt{\footnotesize losdeldgiim.github.io}}
%  \end{minipage}
%}
\fancyfoot[C]{\thepage}
\fancyfoot[R]{\href{https://losdeldgiim.github.io/}{\texttt{\footnotesize losdeldgiim.github.io}}}

%------------------------------------------------------------------------

% Conseguir que no ponga "Capítulo 1". Sino solo "1."
\makeatletter
\@ifclassloaded{book}{
  \renewcommand{\chaptermark}[1]{\markboth{\thechapter.\ #1}{}} % En el encabezado
    
  \renewcommand{\@makechapterhead}[1]{%
  \vspace*{50\p@}%
  {\parindent \z@ \raggedright \normalfont
    \ifnum \c@secnumdepth >\m@ne
      \huge\bfseries \thechapter.\hspace{1em}\ignorespaces
    \fi
    \interlinepenalty\@M
    \Huge \bfseries #1\par\nobreak
    \vskip 40\p@
  }}
}
\makeatother

%------------------------------------------------------------------------
% Paquetes de cógido
\usepackage{minted}
\renewcommand\listingscaption{Código fuente}

\usepackage{fancyvrb}
% Personaliza el tamaño de los números de línea
\renewcommand{\theFancyVerbLine}{\small\arabic{FancyVerbLine}}

% Estilo para C++
\newminted{cpp}{
    frame=lines,
    framesep=2mm,
    baselinestretch=1.2,
    linenos,
    escapeinside=||
}

% para minted
\definecolor{LightGray}{rgb}{0.95,0.95,0.92}
\setminted{
    linenos=true,
    stepnumber=5,
    numberfirstline=true,
    autogobble,
    breaklines=true,
    breakautoindent=true,
    breaksymbolleft=,
    breaksymbolright=,
    breaksymbolindentleft=0pt,
    breaksymbolindentright=0pt,
    breaksymbolsepleft=0pt,
    breaksymbolsepright=0pt,
    fontsize=\footnotesize,
    bgcolor=LightGray,
    numbersep=10pt
}


\usepackage{listings} % Para incluir código desde un archivo

\renewcommand\lstlistingname{Código Fuente}
\renewcommand\lstlistlistingname{Índice de Códigos Fuente}

% Definir colores
\definecolor{vscodepurple}{rgb}{0.5,0,0.5}
\definecolor{vscodeblue}{rgb}{0,0,0.8}
\definecolor{vscodegreen}{rgb}{0,0.5,0}
\definecolor{vscodegray}{rgb}{0.5,0.5,0.5}
\definecolor{vscodebackground}{rgb}{0.97,0.97,0.97}
\definecolor{vscodelightgray}{rgb}{0.9,0.9,0.9}

% Configuración para el estilo de C similar a VSCode
\lstdefinestyle{vscode_C}{
  backgroundcolor=\color{vscodebackground},
  commentstyle=\color{vscodegreen},
  keywordstyle=\color{vscodeblue},
  numberstyle=\tiny\color{vscodegray},
  stringstyle=\color{vscodepurple},
  basicstyle=\scriptsize\ttfamily,
  breakatwhitespace=false,
  breaklines=true,
  captionpos=b,
  keepspaces=true,
  numbers=left,
  numbersep=5pt,
  showspaces=false,
  showstringspaces=false,
  showtabs=false,
  tabsize=2,
  frame=tb,
  framerule=0pt,
  aboveskip=10pt,
  belowskip=10pt,
  xleftmargin=10pt,
  xrightmargin=10pt,
  framexleftmargin=10pt,
  framexrightmargin=10pt,
  framesep=0pt,
  rulecolor=\color{vscodelightgray},
  backgroundcolor=\color{vscodebackground},
}

%------------------------------------------------------------------------

% Comandos definidos
\newcommand{\bb}[1]{\mathbb{#1}}
\newcommand{\cc}[1]{\mathcal{#1}}

% I prefer the slanted \leq
\let\oldleq\leq % save them in case they're every wanted
\let\oldgeq\geq
\renewcommand{\leq}{\leqslant}
\renewcommand{\geq}{\geqslant}

% Si y solo si
\newcommand{\sii}{\iff}

% MCD y MCM
\DeclareMathOperator{\mcd}{mcd}
\DeclareMathOperator{\mcm}{mcm}

% Signo
\DeclareMathOperator{\sgn}{sgn}

% Letras griegas
\newcommand{\eps}{\epsilon}
\newcommand{\veps}{\varepsilon}
\newcommand{\lm}{\lambda}

\newcommand{\ol}{\overline}
\newcommand{\ul}{\underline}
\newcommand{\wt}{\widetilde}
\newcommand{\wh}{\widehat}

\let\oldvec\vec
\renewcommand{\vec}{\overrightarrow}

% Derivadas parciales
\newcommand{\del}[2]{\frac{\partial #1}{\partial #2}}
\newcommand{\Del}[3]{\frac{\partial^{#1} #2}{\partial #3^{#1}}}
\newcommand{\deld}[2]{\dfrac{\partial #1}{\partial #2}}
\newcommand{\Deld}[3]{\dfrac{\partial^{#1} #2}{\partial #3^{#1}}}


\newcommand{\AstIg}{\stackrel{(\ast)}{=}}
\newcommand{\Hop}{\stackrel{L'H\hat{o}pital}{=}}

\newcommand{\red}[1]{{\color{red}#1}} % Para integrales, destacar los cambios.

% Método de integración
\newcommand{\MetInt}[2]{
    \left[\begin{array}{c}
        #1 \\ #2
    \end{array}\right]
}

% Declarar aplicaciones
% 1. Nombre aplicación
% 2. Dominio
% 3. Codominio
% 4. Variable
% 5. Imagen de la variable
\newcommand{\Func}[5]{
    \begin{equation*}
        \begin{array}{rrll}
            \displaystyle #1:& \displaystyle  #2 & \longrightarrow & \displaystyle  #3\\
               & \displaystyle  #4 & \longmapsto & \displaystyle  #5
        \end{array}
    \end{equation*}
}

%------------------------------------------------------------------------


\begin{document}

    % 1. Foto de fondo
    % 2. Título
    % 3. Encabezado Izquierdo
    % 4. Color de fondo
    % 5. Coord x del titulo
    % 6. Coord y del titulo
    % 7. Fecha
    % 8. Autor

    % 1. Foto de fondo
% 2. Título
% 3. Encabezado Izquierdo
% 4. Color de fondo
% 5. Coord x del titulo
% 6. Coord y del titulo
% 7. Fecha
% 8. Autor

\newcommand{\portada}[8]{
    \portadaBase{#1}{#2}{#3}{#4}{#5}{#6}{#7}{#8}
    \portadaBook{#1}{#2}{#3}{#4}{#5}{#6}{#7}{#8}
}

\newcommand{\portadaFotoDif}[8]{
    \portadaBaseFotoDif{#1}{#2}{#3}{#4}{#5}{#6}{#7}{#8}
    \portadaBook{#1}{#2}{#3}{#4}{#5}{#6}{#7}{#8}
}

\newcommand{\portadaExamen}[8]{
    \portadaBase{#1}{#2}{#3}{#4}{#5}{#6}{#7}{#8}
    \portadaArticle{#1}{#2}{#3}{#4}{#5}{#6}{#7}{#8}
}

\newcommand{\portadaExamenFotoDif}[8]{
    \portadaBaseFotoDif{#1}{#2}{#3}{#4}{#5}{#6}{#7}{#8}
    \portadaArticle{#1}{#2}{#3}{#4}{#5}{#6}{#7}{#8}
}




\newcommand{\portadaBase}[8]{

    % Tiene la portada principal y la licencia Creative Commons
    
    % 1. Foto de fondo
    % 2. Título
    % 3. Encabezado Izquierdo
    % 4. Color de fondo
    % 5. Coord x del titulo
    % 6. Coord y del titulo
    % 7. Fecha
    % 8. Autor    
    
    \thispagestyle{empty}               % Sin encabezado ni pie de página
    \newgeometry{margin=0cm}        % Márgenes nulos para la primera página
    
    
    % Encabezado
    \fancyhead[L]{\helv #3}
    \fancyhead[R]{\helv \nouppercase{\leftmark}}
    
    
    \pagecolor{#4}        % Color de fondo para la portada
    
    \begin{figure}[p]
        \centering
        \transparent{0.3}           % Opacidad del 30% para la imagen
        
        \includegraphics[width=\paperwidth, keepaspectratio]{../../_assets/#1}
    
        \begin{tikzpicture}[remember picture, overlay]
            \node[anchor=north west, text=white, opacity=1, font=\fontsize{60}{90}\selectfont\bfseries\sffamily, align=left] at (#5, #6) {#2};
            
            \node[anchor=south east, text=white, opacity=1, font=\fontsize{12}{18}\selectfont\sffamily, align=right] at (9.7, 3) {\href{https://losdeldgiim.github.io/}{\textbf{Los Del DGIIM}, \texttt{\footnotesize losdeldgiim.github.io}}};
            
            \node[anchor=south east, text=white, opacity=1, font=\fontsize{12}{15}\selectfont\sffamily, align=right] at (9.7, 1.8) {Doble Grado en Ingeniería Informática y Matemáticas\\Universidad de Granada};
        \end{tikzpicture}
    \end{figure}
    
    
    \restoregeometry        % Restaurar márgenes normales para las páginas subsiguientes
    \nopagecolor      % Restaurar el color de página
    
    
    \newpage
    \thispagestyle{empty}               % Sin encabezado ni pie de página
    \begin{tikzpicture}[remember picture, overlay]
        \node[anchor=south west, inner sep=3cm] at (current page.south west) {
            \begin{minipage}{0.5\paperwidth}
                \href{https://creativecommons.org/licenses/by-nc-nd/4.0/}{
                    \includegraphics[height=2cm]{../../_assets/Licencia.png}
                }\vspace{1cm}\\
                Esta obra está bajo una
                \href{https://creativecommons.org/licenses/by-nc-nd/4.0/}{
                    Licencia Creative Commons Atribución-NoComercial-SinDerivadas 4.0 Internacional (CC BY-NC-ND 4.0).
                }\\
    
                Eres libre de compartir y redistribuir el contenido de esta obra en cualquier medio o formato, siempre y cuando des el crédito adecuado a los autores originales y no persigas fines comerciales. 
            \end{minipage}
        };
    \end{tikzpicture}
    
    
    
    % 1. Foto de fondo
    % 2. Título
    % 3. Encabezado Izquierdo
    % 4. Color de fondo
    % 5. Coord x del titulo
    % 6. Coord y del titulo
    % 7. Fecha
    % 8. Autor


}


\newcommand{\portadaBaseFotoDif}[8]{

    % Tiene la portada principal y la licencia Creative Commons
    
    % 1. Foto de fondo
    % 2. Título
    % 3. Encabezado Izquierdo
    % 4. Color de fondo
    % 5. Coord x del titulo
    % 6. Coord y del titulo
    % 7. Fecha
    % 8. Autor    
    
    \thispagestyle{empty}               % Sin encabezado ni pie de página
    \newgeometry{margin=0cm}        % Márgenes nulos para la primera página
    
    
    % Encabezado
    \fancyhead[L]{\helv #3}
    \fancyhead[R]{\helv \nouppercase{\leftmark}}
    
    
    \pagecolor{#4}        % Color de fondo para la portada
    
    \begin{figure}[p]
        \centering
        \transparent{0.3}           % Opacidad del 30% para la imagen
        
        \includegraphics[width=\paperwidth, keepaspectratio]{#1}
    
        \begin{tikzpicture}[remember picture, overlay]
            \node[anchor=north west, text=white, opacity=1, font=\fontsize{60}{90}\selectfont\bfseries\sffamily, align=left] at (#5, #6) {#2};
            
            \node[anchor=south east, text=white, opacity=1, font=\fontsize{12}{18}\selectfont\sffamily, align=right] at (9.7, 3) {\href{https://losdeldgiim.github.io/}{\textbf{Los Del DGIIM}, \texttt{\footnotesize losdeldgiim.github.io}}};
            
            \node[anchor=south east, text=white, opacity=1, font=\fontsize{12}{15}\selectfont\sffamily, align=right] at (9.7, 1.8) {Doble Grado en Ingeniería Informática y Matemáticas\\Universidad de Granada};
        \end{tikzpicture}
    \end{figure}
    
    
    \restoregeometry        % Restaurar márgenes normales para las páginas subsiguientes
    \nopagecolor      % Restaurar el color de página
    
    
    \newpage
    \thispagestyle{empty}               % Sin encabezado ni pie de página
    \begin{tikzpicture}[remember picture, overlay]
        \node[anchor=south west, inner sep=3cm] at (current page.south west) {
            \begin{minipage}{0.5\paperwidth}
                %\href{https://creativecommons.org/licenses/by-nc-nd/4.0/}{
                %    \includegraphics[height=2cm]{../../_assets/Licencia.png}
                %}\vspace{1cm}\\
                Esta obra está bajo una
                \href{https://creativecommons.org/licenses/by-nc-nd/4.0/}{
                    Licencia Creative Commons Atribución-NoComercial-SinDerivadas 4.0 Internacional (CC BY-NC-ND 4.0).
                }\\
    
                Eres libre de compartir y redistribuir el contenido de esta obra en cualquier medio o formato, siempre y cuando des el crédito adecuado a los autores originales y no persigas fines comerciales. 
            \end{minipage}
        };
    \end{tikzpicture}
    
    
    
    % 1. Foto de fondo
    % 2. Título
    % 3. Encabezado Izquierdo
    % 4. Color de fondo
    % 5. Coord x del titulo
    % 6. Coord y del titulo
    % 7. Fecha
    % 8. Autor


}


\newcommand{\portadaBook}[8]{

    % 1. Foto de fondo
    % 2. Título
    % 3. Encabezado Izquierdo
    % 4. Color de fondo
    % 5. Coord x del titulo
    % 6. Coord y del titulo
    % 7. Fecha
    % 8. Autor

    % Personaliza el formato del título
    \pretitle{\begin{center}\bfseries\fontsize{42}{56}\selectfont}
    \posttitle{\par\end{center}\vspace{2em}}
    
    % Personaliza el formato del autor
    \preauthor{\begin{center}\Large}
    \postauthor{\par\end{center}\vfill}
    
    % Personaliza el formato de la fecha
    \predate{\begin{center}\huge}
    \postdate{\par\end{center}\vspace{2em}}
    
    \title{#2}
    \author{\href{https://losdeldgiim.github.io/}{Los Del DGIIM, \texttt{\large losdeldgiim.github.io}}
    \\ \vspace{0.5cm}#8}
    \date{Granada, #7}
    \maketitle
    
    \tableofcontents
}




\newcommand{\portadaArticle}[8]{

    % 1. Foto de fondo
    % 2. Título
    % 3. Encabezado Izquierdo
    % 4. Color de fondo
    % 5. Coord x del titulo
    % 6. Coord y del titulo
    % 7. Fecha
    % 8. Autor

    % Personaliza el formato del título
    \pretitle{\begin{center}\bfseries\fontsize{42}{56}\selectfont}
    \posttitle{\par\end{center}\vspace{2em}}
    
    % Personaliza el formato del autor
    \preauthor{\begin{center}\Large}
    \postauthor{\par\end{center}\vspace{3em}}
    
    % Personaliza el formato de la fecha
    \predate{\begin{center}\huge}
    \postdate{\par\end{center}\vspace{5em}}
    
    \title{#2}
    \author{\href{https://losdeldgiim.github.io/}{Los Del DGIIM, \texttt{\large losdeldgiim.github.io}}
    \\ \vspace{0.5cm}#8}
    \date{Granada, #7}
    \thispagestyle{empty}               % Sin encabezado ni pie de página
    \maketitle
    \vfill
}
    \portada{ffccA4.jpg}{Lógica y Teoría\\Descriptiva\\de Conjuntos}{Lógica y TDC}{MidnightBlue}{-8}{28}{2025}{José Juan Urrutia Milán\\Arturo Olivares Martos}

    \newpage


    El presente documento es un resumen del microcredencial de ``Lógica y Teoría Descriptiva de Conjuntos'', que recoge los principales conceptos que se impartieron en el mismo. Si cursa el microcredencial se recomienda ver los recursos proporcionados por el profesorado. Si está cursando actualmente la asignatura de ``Lógica y Métodos Discretos'' del grado de Informática, los dos primeros capítulos pueden serle de gran ayuda.\\

    A lo largo del curso trabajaremos en $\mathbb{Z}_2$, por lo que se recomienda al lector repasar los apuntes de Álgebra I en caso de no estar familiarizado con dicho cuerpo.~\\

    \begin{observacion}
        Al ser adicional, es necesario aún completar correctamente el presente documento. Recomendamos por tanto a leer con detalle el contenido de este curso, prestando especial atención a pequeños errores que se hayan podido cometer, e instamos al lector a complementar el contenido del mismo.
    \end{observacion}



    \noindent
Se recomienda encarecidamente acompañar la asignatura de la lectura del libro ``Functional Analysis, Sobolev Spaces and Partial Diferential Equations'', de Haim Brezis, que puede encontrarse en la bibliografía de la asignatura.\newpage

% // TODO: Cambiar <> por ()

\chapter{El Espacio Dual}
El objetivo de este capítulo es definir el concepto de espacio dual de un espacio normado, así como sus principales propiedades, que nos dotan de muchos ejemplos de espacios de Banach. Para ello, será necesario primero repasar conceptos básicos vistos ya en asignaturas anteriores de Análisis Matemático.

\section{Repaso}
\begin{definicion}[Espacio métrico]
    Un espacio métrico es una tupla $(E, d)$ donde $E$ es un conjunto no vacío y $d:E\times E \to \mathbb{R}$ es una aplicación que verifica:
    \begin{itemize}
        \item \textbf{Desigualdad triangular.} $d(x,z) \leq d(x,y) + d(y,z) \qquad \forall x,y,z\in E$
        \item \textbf{Simetría.} $d(x,y) = d(y,x) \qquad \forall x,y\in E$
        \item \textbf{No degeneración.} $d(x,y) = 0 \Longleftrightarrow x = y$
    \end{itemize}
\end{definicion}

\begin{definicion}[Espacio normado]
    Un espacio normado es una tupla $(E, \|\cdot \|)$ donde $E$ es un espacio vectorial y $\|\cdot \|:E\to \mathbb{R}$ es una aplicación que verifica:
    \begin{itemize}
        \item \textbf{Desigualdad triangular.} $\|x + y\| \leq \|x\| + \|y\| \qquad \forall x,y\in E$
        \item \textbf{Homogeneidad por homotecia.} $\|\lm x\| = |\lm| \|x\| \qquad \forall \lm \in \mathbb{R}, \forall x\in E$
        \item \textbf{No degeneración.} $\|x\| = 0 \Longrightarrow x = 0$
    \end{itemize}
\end{definicion}

A partir de estas propiedades pueden deducirse muchas otras, entre las cuales destacamos:
\begin{prop}
    Si $(E, \|\cdot \|)$ es un espacio normado, entonces:
    \begin{itemize}
        \item $\|0\| = 0$.
        \item $\|x\| \geq 0 \qquad \forall x\in E$.
    \end{itemize}
    \begin{proof}
        Veamos cada propiedad:
        \begin{itemize}
            \item Para la primera: $\|0\| = \|0\cdot v\| = 0\|v\| = 0$.
            \item Para la segunda, basta observar que si $x\in E$, entonces:
                \begin{equation*}
                    0 = \|0\| = \|x + (-x)\| \leq 2\|x\| \Longrightarrow \|x\| \geq 0
                \end{equation*}
        \end{itemize}
    \end{proof}
\end{prop}

\begin{prop}
    Si $(E,\|\cdot \|)$ es un espacio normado y definimos la aplicación $d:E\times E\to \mathbb{R}$ dada por:
    \begin{equation*}
        d(x,y) = \|y-x\| \qquad \forall x,y\in E
    \end{equation*}
    Se verifica que $(E, d)$ es un espacio métrico.
\end{prop}

\begin{definicion}[Espacio métrico completo]
    Sea $(E,d)$ un espacio métrico, decimos que es completo (o que la distancia $d$ es completa) si toda sucesión de Cauchy para la distancia $d$ es también convergente a un elemento de $E$ para la distancia $d$.
\end{definicion}

Hemos visto ya que cualquier espacio normado puede dotarse de estructura de espacio métrico, así como la definición de espacio métrico completo, ambos conceptos tratados ya en asignaturas previas.

\begin{definicion}[Espacio de Banach]
    Sea $(E,\|\cdot \|)$ un espacio normado, decimos que es de Banach si el espacio métrico $(E,d)$ obtenido de la forma usual a partir de la norma $\|\cdot \|$ es un espacio métrico completo.
\end{definicion}

\begin{definicion}[Espacio prehilbertiano]
    Un espacio prehilbertiano es una tupla $(E,\langle \cdot,\cdot   \rangle )$ donde $H$ es un espacio vectorial y $\langle \cdot ,\cdot  \rangle:E\times E\to \mathbb{R} $ es una aplicación que verifica:
    \begin{itemize}
        \item \textbf{Bilinealidad.} La aplicación $\langle \cdot ,\cdot  \rangle $ es lineal en ambas variables.
        \item \textbf{Simetría.} $\langle x,y \rangle  = \langle y,x \rangle \qquad \forall x,y\in E$
        \item \textbf{Definida positiva.} $\langle x,x \rangle >0 \qquad \forall x\in H\setminus \{0\}$
    \end{itemize}
\end{definicion}

\begin{prop}
    Si $(E, \langle \cdot ,\cdot  \rangle )$ es un espacio prehilbertiano y definimos la aplicación $\|\cdot \|:E\to \mathbb{R}$ dada por:
    \begin{equation*}
        \|x\| = \sqrt{\langle x,x \rangle } \qquad \forall x\in E
    \end{equation*}
    Se verifica que $(E,\|\cdot \|)$ es un espacio normado.
\end{prop}

\begin{definicion}[Espacio de Hilbert]
    Sea $(E,\langle \cdot ,\cdot  \rangle )$ un espacio prehilbertiano, decimos que es de Hilbert si el espacio normado $(E,\|\cdot \|)$ obtenido de la forma usual a partir del producto escalar $\langle \cdot ,\cdot  \rangle $ es un espacio métrico de Banach.
\end{definicion}

\subsection{Ejemplos}
\begin{itemize}
    \item Sea $N\in \mathbb{N}$, en $\mathbb{R}^N$ podemos definir para cada $p\geq 1$ la aplicación\newline $\|\cdot \|_p:\mathbb{R}^N\to \mathbb{R}$ dada por:
        \begin{equation*}
            \|x\|_p = {\left(\sum_{i=1}^{N}|x_i|^p\right)}^{\dfrac{1}{p}} \qquad \forall x\in \mathbb{R}^N
        \end{equation*}
        Que hace que $(\mathbb{R}^N, \|\cdot \|_p)$ sea un espacio normado, que de hecho es de Banach, como se vió en Análisis Matemático II, puesto que todo espacio normado de dimensión finita es de Banach.
    \item En el caso anterior, si tomamos $p=2$ se verifica que además si definimos $\langle \cdot ,\cdot  \rangle :\mathbb{R}^N\times \mathbb{R}^N\to \mathbb{R}$ dada por:
        \begin{equation*}
            \langle x,y \rangle  = \sum_{i=1}^{N} x_iy_i \qquad \forall x,y\in \mathbb{R}^N
        \end{equation*}
        Obtenemos que $(\mathbb{R}^N,\langle \cdot ,\cdot  \rangle )$ es un espacio prehilbertiano (compruébese) cuyo espacio normado canónico coincide con $(\mathbb{R}^N,\|\cdot \|_2)$, por lo que es un espacio de Hilbert.
    \item Como otro ejemplo de espacio normado sobre $\mathbb{R}^N$, podemos definir $\|\cdot \|_\infty:\mathbb{R}^N\to \mathbb{R}$ dado por:
        \begin{equation*}
            \|x\|_\infty = \sup \{|x_i| : i \in \{0,\ldots, N\}\}
        \end{equation*}
        Se cumple igualmente que $(\mathbb{R}^N,\|\cdot \|_\infty)$ es un espacio normado que además es de Banach, por la misma razón que antes.
    \item Como primer ejemplo de espacio normado que no se construye sobre los vectores de un espacio de la forma $\mathbb{R}^N$, si tomamos un conjunto $A\subset \mathbb{R}^N$, y definimos\footnote{El subíndice ``b'' de $\cc{C}_b(A)$ viene de la palabra inglesa ``bounded''.}:
        \begin{equation*}
            \cc{C}_b(A) = \{f:A\to \mathbb{R} : f \text{\ es continua y } f \text{\ es acotada en\ } A\}
        \end{equation*}
        Junto con la aplicación $\|\cdot \|:\cc{C}_b(A)\to \mathbb{R}$ dada por:
        \begin{equation*}
            \|f\| = \sup \{\|f(x)\| : x \in A\}
        \end{equation*}
        Se verifica que $(\cc{C}_b(A), \|\cdot \|)$ es una espacio normado que de hecho es de Banach (compruébese). % // TODO: Hacer
    \item Sea ahora $K\subset \mathbb{R}^N$ un compacto, si definimos:
        \begin{equation*}
            \cc{C}(K) = \{f:K\to \mathbb{R}  : f \text{\ es continua}\}
        \end{equation*}
        resulta que podemos definir una aplicación $\langle \cdot ,\cdot  \rangle:\cc{C}(K)\times \cc{C}(K)\to \mathbb{R} $ dada por:
        \begin{equation*}
            \langle f,g \rangle  = \int_K f(x)g(x)~dx
        \end{equation*}
        que hace que $(\cc{C}(K), \langle \cdot ,\cdot  \rangle )$ sea un espacio prehilbertiano, que nos induce un espacio normado donde la norma es:
        \begin{equation*}
            \|f\|_2 = {\left(\int_{K}{f(x)}^{2}~dx \right)}^{\frac{1}{2}} \qquad \forall f\in \cc{C}(K)
        \end{equation*}
        Sin embargo, este espacio prehilbertiano \textbf{no es de Hilbert}:

        Por ejemplo, si tomamos $K = [0,2]\subset \mathbb{R}$, si tomamos $f_n:K\to \mathbb{R}$ de forma que la gráfica de $f_n$ sea algo parecido a la de la Figura~\ref{fig:no_hilbert}

        \begin{figure}[H]
            \centering
            \begin{tikzpicture}[scale=3]
              % parámetro n (cambiar aquí)
              \def\n{8}
              \pgfmathsetmacro{\a}{1-1/\n} % punto 1 - 1/n

              % ejes
              \draw[-stealth] (0,0) -- (2.2,0) node[right] {$x$};
              \draw[-stealth] (0,-0.15) -- (0,1.3) node[above] {$y$};

              % marcas en eje x
              \foreach \x in {0,1,\a,2}{
                \draw (\x,0.01) -- (\x,-0.01);
              }
              \node[below] at (1,0) {$$};
              \node[below] at (2,0) {$2$};
              \node[below] at (\a,0) {$1-\tfrac{1}{n}$};

              % marcas en eje y
              \draw (0,1) -- (-0.01,1);
              \node[left] at (0,1) {$1$};
              \node[left] at (0,0) {$0$};

              % tramos de la función
              % tramo constante 0 en [0, 1-1/n]
              \draw[line width=1pt] (0,0) -- ({\a},0);

              % tramo lineal en [1-1/n, 1]
              \draw[line width=1pt] ({\a},0) -- (1,1);

              % tramo constante 1 en [1,2]
              \draw[line width=1pt] (1,1) -- (2,1);

              % puntos (para destacar continuidad a la derecha/izquierda)
              \filldraw ({\a},0) circle (0.6pt);
              \filldraw (1,1) circle (0.6pt);

              % etiquetas de los tramos
              \node[below] at ({\a/2},0) {$0$};
              \node[above right] at ({(\a+1)/2},{(1/2)}) {$\displaystyle f_n(x)=n x-(n-1)$};
              \node[above] at (1.5,1) {$1$};

              % leyenda con el valor de n
              \node[anchor=north west] at (0.02,-0.12) { $\displaystyle n=\,$\n };
            \end{tikzpicture}
            \caption{Gráfica de la función $f_n$.}
            \label{fig:no_hilbert}
        \end{figure}
        Si definimos $f=  \chi_{[1,2]}$ la función característica del intervalo $[1,2]$ (que no pertence a $\cc{C}(K)$), tenemos que:
        \begin{equation*}
            \|f-f_n\|_2^2 = \int_{0}^{2} {(f(x)- f_n(x))}^{2}~dx  = \dfrac{1}{2n} \to 0
        \end{equation*}
        Por lo que $f_n$ es una sucesión de Cauchy pero cuyo límite no está en el espacio que consideramos, por lo que no es convergente, luego $\cc{C}(K)$ no es un espacio completo. % // TODO: Afinar más este razonamiento.
\end{itemize}

\section{Espacios de Lebesgue}
Un ejemplo interesante de espacios de Banach son los espacios de Lebesgue, que ya se trabajaron un poco en la asignatura de Análisis Matemático II. En este documento volveremos a definir dicho espacio, puesto que la construcción es importante tenerla clara. En un primer lugar, hemos de repasar ciertas desigualdades para poder construir la estructura de espacio normado.

\subsection{Desigualdades importantes}
Para la primera desigualdad, es conveniente la siguiente motivación, que nos dará una breve justificación del origen de la desigualdad: sean $a,b\in \mathbb{R}^+_0$ dos números reales no negativos, es bien conocido que:
\begin{equation*}
    0 \geq {(a-b)}^{2} = a^2 + b^2 - 2ab \Longrightarrow ab \leq \dfrac{a^2}{2} + \dfrac{b^2}{2}
\end{equation*}

\begin{definicion}
    Sea $p\geq 1$ un número real, definimos su ``exponente conjugado'' por:
    \begin{equation*}
        p' = \left\{\begin{array}{ll}
                \frac{p}{p-1} & \text{si\ } p\neq 1 \\
                \infty & \text{si\ } p = 1
        \end{array}\right.
    \end{equation*}
    De esta forma (admitiendo el convenio de que $0 = \nicefrac{1}{\infty}$ de la recta real extendida), tenemos que:
    \begin{equation*}
        \dfrac{1}{p} + \dfrac{1}{p'} = 1
    \end{equation*}
\end{definicion}
\noindent
Usaremos en esta sección la notación $p'$ para denotar al exponente conjugado de $p$.

\begin{prop}[Desigualdad de Young]
    Sean $a,b\in \mathbb{R}^+_0$ y $p\in \mathbb{R}$ con $p>1$, se verifica que:
    \begin{equation*}
        ab \leq \dfrac{a^p}{p} + \dfrac{b^{p'}}{p'}
    \end{equation*}
    \begin{proof}
        La concavidad\footnote{Recordamos que si $f$ era una función cóncava, entonces $f(tx+(1-t)y) \geq tf(x)+(1-t)f(y)$, para cualquier $t\in [0,1]$, $x$,$y$ en el dominio de definición de $f$.} del logaritmo nos dice:
        \begin{equation*}
            \log\left(\dfrac{a^p}{p} + \dfrac{b^{p'}}{p'}\right) \geq \dfrac{1}{p}\log(a^p) + \dfrac{1}{p'}\log\left(b^{p'}\right) = \log(a) + \log(b) = \log(ab)
        \end{equation*}
        Y si ahora aplicamos la función exponencial y usamos que es creciente obtenemos:
        \begin{equation*}
            ab \leq \dfrac{a^p}{p} + \dfrac{b^{p'}}{p'}
        \end{equation*}
    \end{proof}
\end{prop}

\noindent
Recordemos que en Análisis Matemático I definíamos para cualquier conjunto $\Omega\subset \mathbb{R}$ medible el conjunto de las funciones integrables sobre $\Omega$:
\begin{equation*}
    \cc{L}(\Omega) = \left\{f:\Omega\to \mathbb{R} : \int_\Omega f < \infty\right\}
\end{equation*}
Pues bien, dado $p\geq 1$, podemos definir ahora:
\begin{equation*}
    \cc{L}_p(\Omega) = \left\{f\in \cc{L}(\Omega) : \int_\Omega |f|^p < \infty\right\}
\end{equation*}

\begin{teo}[Desigualdad de Hölder]
    Sea $p>1$, si $f\in \cc{L}_p(\Omega)$ y $g\in \cc{L}_{p'}(\Omega)$, entonces $fg\in \cc{L}(\Omega)$ y además:
    \begin{equation*}
        \int_\Omega |fg| \leq {\left(\int_\Omega |f|^p\right)}^{\frac{1}{p}}{\left(\int_\Omega |g|^{p'}\right)}^{\frac{1}{p'}}
    \end{equation*}
    \begin{proof}
        Si notamos por comodidad:
        \begin{equation*}
            \alpha = {\left(\int_\Omega |f|^p\right)}^{\frac{1}{p}},  \qquad \beta ={\left(\int_\Omega |g|^{p'}\right)}^{\frac{1}{p'}}
        \end{equation*}
        Si $\alpha = 0$, entonces $f^p = 0$ casi por doquier, de donde $|fg| = 0$ casi por doquier, luego:
        \begin{equation*}
            \int_\Omega|fg| = 0
        \end{equation*}
        Si $\beta = 0$ la situación es simétrica. Suponiendo ahora que $\alpha,\beta\in \mathbb{R}^+$, la desigualdad de Young nos dice que:
        \begin{equation*}
            \dfrac{|f(x)|}{\alpha} \dfrac{|g(x)|}{\beta}\leq \dfrac{|f(x)|^p}{p\alpha^p} + \dfrac{|g(x)|^{p'}}{p' \beta^{p'}} \qquad \forall x\in \Omega
        \end{equation*}
        Si ahora aplicamos la integral de Lebesgue a ambos lados usando el crecimiento de dicho funcional, obtenemos que (usando la definición de $\alpha$ y $\beta$):
        \begin{equation*}
            \dfrac{1}{\alpha\beta} \int_\Omega |fg| \leq \dfrac{1}{p\alpha^p} \int_\Omega |f|^p + \dfrac{1}{p'\beta^{p'}} \int_\Omega |g|^{p'} = \dfrac{1}{p} + \dfrac{1}{p'} = 1
        \end{equation*}
        de donde $fg\in \cc{L}(\Omega)$ y despejando de la desigualdad:
        \begin{equation*}
            \dfrac{1}{\alpha\beta} \int_\Omega |fg| \leq 1
        \end{equation*}
        Obtenemos la desigualdad buscada.
    \end{proof}
\end{teo}

\noindent
La desigualdad de Hölder nos proporcionará la desigualdad de Cauchy-Schwartz de la norma del futuro espacio normado, y nos permitirá probar la desigualdad de Minkowski.

\begin{teo}[Desigualdad de Minkowski]
    Para $p\in \mathbb{R}$ con $p\geq 1$ y $f,g\in \cc{L}_p(\Omega)$, se cumple que:
    \begin{equation*}
        {\left(\int_\Omega |f+g|^p\right)}^{\frac{1}{p}} \leq {\left(\int_\Omega |f|^p\right)}^{\frac{1}{p}} + {\left(\int_\Omega |g|^{p'}\right)}^{\frac{1}{p'}}
    \end{equation*}
    \begin{proof}
        Si notamos por comodidad:
        \begin{equation*}
            \alpha = {\left(\int_\Omega |f|^p\right)}^{\frac{1}{p}},  \qquad \beta ={\left(\int_\Omega |g|^{p'}\right)}^{\frac{1}{p'}}, \qquad \gamma = {\left(\int_\Omega |f+g|^p\right)}^{\frac{1}{p}} 
        \end{equation*}
        Si $p=1$, entonces la desigualdad triangular nos dice que $|f+g| \leq |f| + |g|$, donde aplicamos el crecimiento de la integral y ya tenemos el Teorema demostrado. Sabemos por el resultado anterior que $\gamma<\infty$, puesto que $\cc{L}_p(\Omega)\subset \cc{L}(\Omega)$, y la desigualdad buscada es obvia si $\gamma = 0$.  Supuesto ahora que $p>1$ y $\gamma>0$, si tomamos:
        \begin{equation*}
            h = |f+g|^{p-1}
        \end{equation*}
        tenemos entonces que:
        \begin{equation*}
            h^{p'} = |f+g|^{(p-1)p'} = |f+g|^p
        \end{equation*}
        luego:
        \begin{equation*}
            \int_\Omega h^{p'} = \gamma^p < \infty
        \end{equation*}
        Por lo que $h\in \cc{L}_p(\Omega)$. Tenemos:
        \begin{equation*}
            |f+g|^p = |f+g|h \leq |f| h + |g| h
        \end{equation*}
        Y por la desigualdad de Hölder:
        \begin{equation*}
            \gamma^p \leq \int_\Omega |f| h + \int_\Omega |g|h \leq (\alpha+\beta){\left(\int_\Omega h^{p'}\right)}^{\frac{1}{p'}} = (\alpha + \beta)\gamma^{\frac{p}{p'}}
        \end{equation*}
        Y si dividimos por $\gamma^{\frac{p}{p'}}$ tenemos la desigualdad buscada.
    \end{proof}
\end{teo}

\subsection{Definición de los espacios de Lebesgue}
Fijado $p\geq 1$, podemos tratar de dotar a $\cc{L}_p(\Omega)$ de una norma. Pensamos en un principio en la aplicación $\varphi_p:\cc{L}_p(\Omega)\to \mathbb{R}$ dada por:
\begin{equation*}
    \varphi_p(f) = {\left(\int_\Omega |f|^p\right)}^{\frac{1}{p}} \qquad \forall f\in \cc{L}_p(\Omega)
\end{equation*}
Que:
\begin{itemize}
    \item Verifica la desigualdad triangular gracias a la desigualdad de Minkowski.
    \item Verifica la homegeneidad por homotecias, ya que:
        \begin{equation*}
            \varphi_p(\alpha f) = |\alpha| \varphi_p(f) \qquad \forall \alpha\in \mathbb{R}
        \end{equation*}
    \item $\varphi_p(f) = 0 \Longleftrightarrow f = 0$ casi por doquier.
\end{itemize}
Por lo que dicha función \textbf{no es una norma} en $\cc{L}_p(\Omega)$ al no verificar la no degeneración de la norma, puesto que la integral ``es ciega'' a la hora de diferenciar la función constantemente igual a 0 de otras funciones con integral cero.\\

\noindent
Para solucionar el problema con el que nos acabamos de topar (el problema de no poder definir una norma de dicha forma), podemos constuir una relación de equivalencia $\sim$ en $\cc{L}_p(\Omega)$ que identifique a las funciones que son iguales casi por doquier, pudiendo considerar el espacio cociente:
\begin{equation*}
    L_p(\Omega)= \dfrac{\cc{L}_p(\Omega)}{\sim}
\end{equation*}
Donde ya $(L_p(\Omega), \varphi_p)$ sí que es un espacio normado, donde denotaremos normalmente $\varphi_p = \|\cdot \|_p$.

\begin{teo}[Riesz-Fischer]
    Sea $\Omega\subset \mathbb{R}^N$ un conjunto medible y $p\geq 1$, se cumple que $(L_p(\Omega), \|\cdot \|_p)$ es un espacio de Banach.
\end{teo}

\subsection{Más ejemplos de espacios de Banach}
\begin{itemize}
    \item Sea $\Omega\subset \mathbb{R}$ un conjunto medible, si definimos:
        \begin{equation*}
            \sup_\Omega |f| = \inf\{M\geq 0 : |f(x)| \leq M \text{\ casi para todo\ } x\in \Omega\}
        \end{equation*}
        El conjunto:
        \begin{equation*}
            \cc{L}^\infty(\Omega) = \left\{f:\Omega\to \mathbb{R} : f \text{\ es medible y\ } \sup_\Omega |f| < \infty\right\}
        \end{equation*}
        junto con la norma:
        \begin{equation*}
            \|f\|_\infty = \sup_\Omega |f|
        \end{equation*}
        es un espacio de Banach, donde la desigualdad de Hölder se comple considerando que $p=\infty$ y $p'=1$:

        Si $f\in \cc{L}^\infty(\Omega)$ y $g\in \cc{L}(\Omega)$, entonces $fg\in \cc{L}(\Omega)$, con:
        \begin{equation*}
            \|fg\|_1 \leq \|f\|_\infty \|g\|_1
        \end{equation*}
    \item Para $1\leq p < \infty$ podemos considerar otro tipo de espacios:
        \begin{equation*}
            l^p = \left\{x:\mathbb{N}\to \mathbb{R} : \sum_{n=1}^{\infty}|x(n)|^p < \infty\right\}
        \end{equation*}
        que junto con la aplicación:
        \begin{equation*}
            \|x\|_p = {\left(\sum_{n=1}^{\infty}|x(n)|^p\right)}^{\frac{1}{p}} \qquad \forall x\in l^p
        \end{equation*}
        forman un espacio de Banach (compruébese).

        En dichos espacios, se tiene que si $x\in l^p$ y $y\in l^{p'}$, entonces $xy\in l$, con:
        \begin{equation*}
            \|xy\| \leq \|x\|_p \|y\|_{p'}
        \end{equation*}
    \item En el caso anterior, si $p=2$, podemos definir la aplicación:
        \begin{equation*}
            \langle x,y \rangle_2 = \sum_{n=1}^{\infty}x(n)y(n) \qquad \forall x,y\in l^2
        \end{equation*}
        Con lo que $(l^2, \langle \cdot ,\cdot  \rangle _2)$ es un espacio de Hilbert.
    \item Al igual que sucedía con las normas $p-$ésimas en $\mathbb{R}^N$, podemos considerar:
        \begin{equation*}
            l^\infty = \{x:\mathbb{N}\to \mathbb{R} : x \text{\ acotada}\}
        \end{equation*}
        junto con la aplicación $\|\cdot \|:l^{\infty}\to \mathbb{R}$ dada por:
        \begin{equation*}
            \|x\|_\infty = \sup\{|x(n)| : n\in \mathbb{N}\}
        \end{equation*}
        y obtenemos un espacio de Banach.
    \item $C = \{x:\mathbb{N}\to \mathbb{R} : x \text{\ es convergente}\}$ es un subespacio de $l^\infty$.
    \item $C_0 = \{x:\mathbb{N}\to \mathbb{R} : x \text{\ converge a\ }0\}$ es un subespacio de $C$.
\end{itemize}

\begin{prop}
    El espacio normado $l^p$ es de Banach para $1\leq p \leq \infty$.
    \begin{proof}
        \begin{description}
            \item [Para $p=\infty$.] Recordamos que trabajamos en el espacio:
                \begin{equation*}
                    l^\infty = \{x:\mathbb{N}\to\mathbb{R} : x \text{\ acotada}\}, \qquad \|x\|_\infty = \sup\{|x(n)|: n\in \mathbb{N}\}
                \end{equation*}
                Sea $\{x_m\}$ una sucesión de Cauchy de elementos de $l^\infty$, queremos probar que $\{x_m\}$ es convergente en $l^\infty$. Para ello, primero vemos que fijado $n\in \mathbb{N}$ tenemos que la sucesión de números reales $\{x_m(n)\}$ es de Cauchy, pues dado $\varepsilon>0$ podemos encontrar $m_0\in \mathbb{N}$ (gracias a que $\{x_m\}$ es de Cauchy) de foma que:
                \begin{equation*}
                    |x_p(n) - x_q(n)| \leq \sup_{k \in \mathbb{N}}|x_p(k)-x_q(k)| = \|x_p-x_q\|_\infty < \varepsilon \qquad \forall p,q\geq m_0
                \end{equation*}
                Como $\mathbb{R}$ es completo, tenemos que la sucesión $\{x_m(n)\}$ es convergente, para cada $n\in \mathbb{N}$, lo que nos permite definir
                \Func{x}{\bb{N}}{\bb{R}}{n}{\lim\{x_m(n)\}}
                Obteniendo que $x\in \mathbb{R}^\mathbb{N}$, y la demostración de este caso terminará probando que $x\in l^\infty$ y que $\{\|x_m - x\|\}\to 0$. Dado $\varepsilon>0$, existe $m_0\in \mathbb{N}$ de forma que de forma que:
                \begin{equation*}
                    \|x_p - x_q\|_\infty < \varepsilon\qquad \forall p,q\geq m_0
                \end{equation*}
                Fijado $m\geq m_0$, tenemos para todo $k\in \mathbb{N}$ que:
                \begin{equation*}
                    |x_m(k)- x(k)| = \lim_{n\to\infty}|x_m(k) - x_n(k)| \leq \varepsilon 
                \end{equation*}
                Por lo que $x_m-x\in l^\infty$, de donde:
                \begin{equation*}
                    x = x_m -(x_m - x) \in l^\infty
                \end{equation*}
                por ser $l^\infty$ un espacio vectorial. Más aún, hemos probado que:
                \begin{equation*}
                    \|x_m - x\|_\infty < \varepsilon\qquad \forall m\geq m_0
                \end{equation*}
                lo que nos dice que $\{\|x_m-x\|_\infty\}\to 0$.
            \item [Para $1\leq p < \infty$.] Trabajamos ahora en el espacio:
                \begin{equation*}
                    l^p = \left\{x:\mathbb{N}\to \mathbb{R} : \sum_{n=1}^{\infty}{|x(n)|}^{p}<\infty\right\}, \qquad \|x\|_p = {\left(\sum_{n=1}^{\infty}{|x(n)|}^{p}\right)}^{\frac{1}{p}}
                \end{equation*}
                Sea $\{x_m\}$ una sucesión de Cauchy de elementos de $l^p$, queremos probar que $\{x_m\}$ es convergente en $l^p$. Para ello, observemos primero que:
                \begin{equation*}
                    |x(k)| \leq \|x\|_p = {\left(\sum_{n=1}^{\infty}{|x(n)|}^{p}\right)}^{\frac{1}{p}} \qquad \forall x\in l^p, \quad \forall k\in \mathbb{N}
                \end{equation*}
                de donde deducimos que $\{x_m(k)\}$ es de Cauchy, para todo $k\in \mathbb{N}$. Por ser $\mathbb{R}$ completo tenemos que dicha sucesión es convergente, lo que nos permite definir
                \Func{x}{\bb{N}}{\bb{R}}{k}{\lim\{x_m(k)\}}
                Obteniendo que $x\in \mathbb{R}^\mathbb{N}$, y basta probar que $x\in l^p$ y que $\{\|x_m - x\|\}\to 0$. Fijado $\varepsilon>0$, existe $m_0\in \mathbb{N}$ de forma que:
                \begin{equation*}
                    \|x_t - x_s\|_p< \varepsilon\qquad \forall t,s\geq m_0
                \end{equation*}
                fijados $m\geq m_0$ y $N\in \mathbb{N}$, tenemos que:
                \begin{equation*}
                    \sum_{k=1}^{N}{|x_m(k) - x_n(k)|}^{p} \leq \sum_{k=1}^{\infty}{|x_m(k) - x_n(k)|}^{p} = {(\|x_m - x_n\|)}^{p} < \varepsilon^p \qquad \forall n\in \mathbb{N}
                \end{equation*}
                de donde:
                \begin{equation*}
                    \sum_{k=1}^{N}{|x_m(k) - x(k)|}^{p} = \lim_{n\to\infty}\sum_{n=1}^{\infty}{|x_m(k) -x_n(k)|}^{p} \leq \varepsilon^p  \qquad \forall N\in \mathbb{N}
                \end{equation*}
                por lo que la serie $\sum_{k\geq 1}{|x_m(k)-x(k)|}^{p}$ es convergente, de donde $x_m-x\in l^p$, por lo que:
                \begin{equation*}
                    x = x_m - (x_m - x) \in l^p
                \end{equation*}
                Más aún, la última desigualdad nos dice que:
                \begin{equation*}
                    \|x_m - x\|_p  = \lim_{N\to\infty}{\left(\sum_{n=1}^{\infty}{|x_m(k)-x(k)|}^{p}\right)}^{\frac{1}{p}} \leq \varepsilon \qquad \forall m\geq m_0
                \end{equation*}
                de donde deducimos que $\{\|x_m - x\|\}\to 0$.
        \end{description}
    \end{proof}
\end{prop}

\section{Espacio dual}
Para introducir la noción de espacio dual, nos será necesario primero destacar unos resultados:
\begin{prop}
    Si $H$ es un espacio prehilbertiano, entonces:
    \begin{enumerate}
        \item Se cumple la desigualdad de Cauchy-Schwartz:
            \begin{equation*}
                |\langle u,v \rangle | \leq \|u\|\|v\| \qquad \forall u,v\in H
            \end{equation*}
        \item Se cumple la identidad del paralelogramo:
            \begin{equation*}
                \left\|\dfrac{u+v}{2}\right\| + \left\| \dfrac{u-v}{2}\right\| = \dfrac{1}{2}(\|u\|^2 + \|v\|^2) \qquad \forall u,v\in H
            \end{equation*}
    \end{enumerate}
\end{prop}

\begin{teo}[de la Proyección]
    Sea $H$ un espacio de Hilbert, sea $\emptyset \neq K \subset H$ un conjunto convexo y cerrado, entonces $\forall f\in H~\exists_1 u\in K$ de forma que:
    \begin{equation*}
        \|f-u\| = d(f,K) = \inf \{d(f,v) : v\in K\}
    \end{equation*}
    Además, dicho elemento $u$ está caracterizado por:
    \begin{itemize}
        \item $u\in K$.
        \item $\langle f-u,v-u \rangle \leq 0 \qquad \forall v\in K$.
    \end{itemize}
    Por tanto, a dicho único elemento $u$ lo notaremos por $P_Kf$.
    \begin{proof}
        Como $0\leq d(f,v) \quad \forall v\in K$, tenemos entonces que dicho ínfimo existe. Tenemos por tanto que existe $\{v_n\}$ una sucesión de elementos de $K$ de forma que $\{d(f,v_n)\}\to d(f,K)$. Sean $n,m\in \mathbb{N}$ y usando la identidad del paralelogramo con $f-v_n$ y $f-v_m$, tenemos:
        \begin{gather*}
            \left\| \frac{f-v_n + f-v_m}{2} \right\|^2 + \left\| \frac{f-v_n-(f-v_m)}{2} \right\|^2 = \frac{1}{2}\left(\|f-v_n\|^2 + \|f-v_m\|^2\right)\\
            \left\| f - \frac{v_n+v_m}{2} \right\|^2 + \left\| \frac{v_m - v_n}{2} \right\|^2 = \frac{1}{2}\left(\|f-v_n\|^2 + \|f-v_m\|^2\right)\\
            \frac{\left\| v_m-v_n \right\|^2}{4} = \frac{1}{2}\left(\|f-v_n\|^2 + \|f-v_m\|^2\right) - \left\| f - \frac{v_n+v_m}{2} \right\|^2\\
            \left\| v_m-v_n \right\|^2 = 2\left(\|f-v_n\|^2 + \|f-v_m\|^2\right) - 4\left\| f - \frac{v_n+v_m}{2} \right\|^2
        \end{gather*}
        Como $K$ es convexo, tenemos que $\frac{v_n+v_m}{2}\in K$, por lo que:
        \begin{equation*}
            \left\| f-\dfrac{v_n+v_m}{2}\right\| \geq d(f,K)
        \end{equation*}
        Por lo que:
        \begin{equation*}
            0 \leq \|v_m - v_n\|^2 \leq 2(\|f-v_n\|^2 + \|f-v_m\|^2) - 4d(f,K)^2
        \end{equation*}
        Como $\{\|f-v_n\|^2\} \to d(f,K)^2$ y $\{\|f-v_m\|^2\}\to d(f,K)^2$, tenemos por el Lema del Sandwitch que $\{\|v_n - v_m\|^2\}\to 0$, por lo que $\{v_n\}$ es de Cauchy. Como $H$ es completo, existe $u\in H$ de forma que $\{v_n\}\to u$, pero por ser $K$ cerrado tendremos que $u\in K$.

        Como $\{v_n\}\to u$, tenemos entonces que $\{d(f,v_n)\}\to d(f,v)$, pero $\{d(f,v_n)\}$ convergía también a $d(f,K)$. No queda más salida que $d(f,v) = d(f,K)$.\\

        \noindent
        Una vez probada la existencia de $u$, veamos que:
        \begin{equation*}
            u\in K \text{\ con\ } \|f-u\| = d(f,K) \Longleftrightarrow u\in K \text{\ y\ } \langle f-u,v-u \rangle \leq 0 \quad \forall v\in K
        \end{equation*}
        \begin{description}
            \item [$\Longrightarrow)$] Supongamos que $u\in K$ y sabemos que $\|f-u\|\leq \|f-v\|$ para todo $v\in K$. Tomamos ahora $w\in K$ y consideramos el segmento que une $u$ con $w$. Entonces $\forall w\in K$ y $\forall t \in [0,1]$, al ser $K$ convexo tendremos que
            \begin{gather*}
                (1-t)u + tw \in K\ \  \text{ y }\ \ \|f-u\|^2 \leq \|f-(1-t)u-tw\|^2
            \end{gather*}
            Aplicando la bilinealidad podemos reescribir esta última expresión como 
            \begin{align*}
                \|f-(1-t)u-tw\|^2 &= \langle f-(1-t)u-tw,f-(1-t)u-tw \rangle  =\\
                &=\|f-u\|^2 + t^2\|w-u\|^2-2t(f-u,w-u)
            \end{align*}
            Sustituyendo en la expresión que teníamos anteriormente nos queda que:
            \begin{gather*}
                0\leq t^2\|w-u\|^2-2t\langle f-u,w-u \rangle  \ \ \ \forall t \in (0,1]
            \end{gather*}
            Al dividir entre $t$ nos queda
            \begin{gather*}
                0\leq t\|w-u\|^2-2\langle f-u,w-u \rangle  \ \ \ \forall t \in (0,1]
            \end{gather*}
            y tomando ahora el límite  cuando $t$ tiende a $0$ por la derecha queda que
            \begin{gather*}
                0\leq -2\langle f-u,w-u \rangle  \Rightarrow \langle f-u,w-u \rangle  \leq 0 \qquad \forall w\in K
            \end{gather*}
            \item [$\Longleftarrow)$] 
                \begin{equation*}
                    \|f-v\|^2 = \|f-u+u-v\|^2 = \|f-u\|^2 + 2\langle f-u,u-v \rangle  + \|u-v\|^2 \qquad \forall v\in K
                \end{equation*}
                De donde:
                \begin{equation*}
                    0\geq 2\langle f-u,v-u \rangle  - \|u-v\|^2 = \|f-u\|^2 - \|f-v\|^2
                \end{equation*}
                Luego:
                \begin{equation*}
                    \|f-u\|^2 \leq \|f-v\|^2 \qquad \forall v\in K
                \end{equation*}
        \end{description}
        Para probar finalmente la unicidad, supongamos que existen $u,w\in K$ de forma que:
        \begin{equation*}
            \langle f-u,v-u \rangle ,\langle f-w,v-w \rangle \leq 0 \qquad \forall v\in K
        \end{equation*}
        Entonces:
        \begin{equation*}
            \langle f-u,w-u \rangle , \langle f-w,u-w \rangle = \langle u-f,w-u \rangle  \leq 0
        \end{equation*}
        Por lo que:
        \begin{equation*}
            \langle f-u,w-u \rangle  + \langle w-f, w-u \rangle  = \langle w-u, w-u \rangle  \leq 0
        \end{equation*}
        de donde $\langle w-u,w-u \rangle = 0$, por lo que $\|w-u\|^2 = d(w,u)^2 = 0$, luego $w=u$.
    \end{proof}
\end{teo}

\begin{prop}
    Dado $\emptyset  \neq K\subset H$ un conjunto convexo y cerrado, tenemos que la aplicación
    \Func{P_K}{H}{H}{f}{P_Kf}
    es lipschitziana. De hecho:
    \begin{equation*}
        \|P_Kf_1 - P_Kf_2\| \leq \|f_1 - f_2\| \qquad \forall f_1,f_2\in H
    \end{equation*}
    \begin{proof}
        Sean $f_1, f_2\in H$, $u_1 = P_Kf_1$, $u_2=P_Kf_2$, estos verifican:
        \begin{equation*}
            \langle f_1-u_1, v-u_1 \rangle , \langle f_2-u_2,v-u_2 \rangle  \leq 0 \qquad \forall v\in K
        \end{equation*}
        Por lo que:
        \begin{gather*}
            \langle f_1 - u_1, u_2 - u_1 \rangle  \leq 0 \\
            \langle f_2 - u_2, u_1 - u_2 \rangle \leq 0 \Longrightarrow \langle f_2 - u_2, u_2 - u_1 \rangle \geq 0
        \end{gather*}
        De donde $\langle f_1 - u_2 - f_2 + u_2, u_2 - u_1 \rangle \leq 0$, por lo que:
        \begin{equation*}
            \langle f_1 - f_2 + (u_2 - u_1), (u_2 - u_1) \rangle  = \langle f_1 - f_2, u_2 - u_1 \rangle  + \langle u_2 - u_1, u_2 - u_1 \rangle 
        \end{equation*}
        Luego:
        \begin{equation*}
            \|u_2 - u_1\|^2 = \langle u_2 - u_1, u_2 - u_1 \rangle  \leq - \langle f_1 - f_2, u_2 - u_1 \rangle \stackrel{\text{Cauchy-Schwartz}}{\leq} \|f_1 - f_2\| \|u_2 - u_1\|
        \end{equation*}
        Por lo que:
        \begin{equation*}
            \|u_2 - u_1\| \leq \|f_1 - f_2\|
        \end{equation*}
        Si $\|u_2 - u_1\| \neq 0$, cierto también si $\|u_2 - u_1\| = 0$.
    \end{proof}
\end{prop}

\noindent
Pensemos ahora en un ejemplo de conjuntos convexos con propiedades interesantes, como lo son los espacios vectoriales:

\begin{coro}[Proyección Ortogonal]
    Sea $M\subset H$ un subespacio vectorial cerrado de $H$, un espacio de Hilbert, entonces:
    \begin{equation*}
        \forall f\in H~\exists _1 u \in M \text{\ tal que\ } \|f-u\| = d(f,M)
    \end{equation*}
    Además, la caracterización de $u$ puede mejorarse por:
    \begin{equation*}
        u\in M \qquad \text{y} \qquad \langle f-u,w \rangle  = 0 \quad \forall w\in M
    \end{equation*}
    \begin{proof}
        Bajo las hipótesis de que $M$ es un subespacio vectorial cerrado de un espacio de Hilbert $H$, basta probar:
        \begin{equation*}
            u\in M \land \langle f-u,v-u \rangle \leq 0 \quad \forall v\in M \Longleftrightarrow u\in M \land \langle f-u,w \rangle =0 \quad \forall w\in M
        \end{equation*}
        \begin{description}
            \item [$\Longleftarrow )$] Si $v\in M$, tenemos por ser $M$ un espacio vectorial que $v-u\in M$, de donde $\langle f-u,v-u \rangle =0$, por lo que en particular es menor o igual que 0.
            \item [$\Longrightarrow )$] Si tomamos $v\in M$ y $t\in \mathbb{R}^\ast$, como $M$ es un espacio vectorial tendremos que $\nicefrac{v}{t}\in M$, por lo que:
                \begin{equation*}
                    \left\langle f-u,\frac{v}{t}-u \right\rangle  \leq 0 \qquad \forall v\in M, \forall t\in \mathbb{R}^\ast
                \end{equation*}
                \begin{itemize}
                    \item Si $t>0$, entonces $\langle f-u,v-tu \rangle \leq 0$ $\forall v\in M$, de donde tomando límite cuando $t\to 0$, tenemos que $\langle f-u,v \rangle \leq 0$ $\forall v\in M$.
                    \item Si $t<0$, entonces $\langle f-u,v-tu \rangle \geq 0$ $\forall v\in M$, de donde tomando límite cuando $t\to 0$, tenemos que $\langle f-u,v \rangle \geq 0$ $\forall v\in M$.
                \end{itemize}
                En consecuencia, tenemos que $\langle f-u,v \rangle =0$ $\forall v\in M$.
        \end{description}
    \end{proof}
\end{coro}

\begin{prop}
    Sea $M\subset H$ un esubespacio vectorial cerrado de $H$, un espacio de Hilbert, la aplicación
    \Func{P_M}{H}{H}{f}{P_Mf}
    es lineal.
    \begin{proof}
        Sean $f_1,f_2\in H$, $u_1 = P_Mf_1$, $u_2 = P_Mf_2$, $\lm \in \mathbb{R}$, tenemos que:
        \begin{align*}
            \langle \lm f_1+f_2-(\lm u_1+u_2),w \rangle  &= \langle \lm f_1 - \lm u_1 +f_2 - u_2, w \rangle \\ &= \lm \langle f_1 -u_1,w \rangle + \langle f_2-u_2,w \rangle  = 0 \qquad \forall w\in M
        \end{align*}
        Por lo que por el Corolario anterior, tenemos que: 
        \begin{equation*}
            P_M(\lm f_1 + f_2) = \lm u_1 + u_2 = \lm P_M(f_1) + P_M(f_2)
        \end{equation*}
        de donde $P_M$ es lineal.
    \end{proof}
\end{prop}

\begin{definicion}
    Sea $(E,\|\cdot \|)$ un espacio normado, definimos el \underline{espacio dual topológico} de $E$ por:
    \begin{equation*}
        E^\ast = \{f:E\to \mathbb{R} : f \text{\ es lineal y continua}\}
    \end{equation*}
\end{definicion}

\noindent
Nos será necesaria la siguiente Proposición para comprender mejor las propiedades de las aplicaciones lineales. Más concretamente, la relación existente entre la acotación y la continuidad de una aplicación lineal.
\begin{prop}\label{prop:lineal_continuidad_acotacion}
    Sea $T:E\to F$ una aplicación lineal entre dos espacios normados $E$ y $F$, las siguientes afirmaciones son equivalentes:
    \begin{enumerate}[label=(\arabic*)]
        \item $\exists M\in \mathbb{R}^+$ de forma que $\|T(x)\| \leq M\|x\| \quad \forall x\in E$.
        \item $T$ es lipschitziana.
        \item $T$ es continua.
        \item $T$ es continua en $0$.
        \item $T$ es acotada (es decir, si $A\subset E$ es acotado, entonces $T(A)$ es acotado).
        \item $T(\overline{B}(0,1))$  es acotado.
        \item $T(B(0,1))$  es acotado.
    \end{enumerate}
    \begin{proof}
        Veamos la equivalencia entre todas ellas:
        \begin{description}
            \item [$(1)\Longleftrightarrow (2)$]  Por doble implicación:
                \begin{description}
                    \item [$\Longrightarrow )$] Sean $x,y\in E$, entonces $x-y\in E$, de donde:
                        \begin{equation*}
                            \|T(x) - T(y)\| = \|T(x-y)\| \leq M\|x-y\|
                        \end{equation*}
                        Por lo que $T$ es lipschitziana con constante de Lipschitz menor o igual que $M$.
                    \item [$\Longleftarrow )$] Sea $x\in E$, si $M$ es mayor o igual que la constante de Lipschitz de $T$, entonces:
                        \begin{equation*}
                            \|T(x)\| = \|T(2x-x)\| = \|T(2x) - T(x)\| \leq M\|2x-x\| = M\|x\|
                        \end{equation*}
                \end{description}
            \item [$(2)\Longrightarrow (3)$] Es conocida de Cálculo II.
            \item [$(3)\Longrightarrow (4)$] Si $T$ es continua, en particular lo es en $0$.
            \item [$(4)\Longrightarrow (1)$] Supuesto que $T$ es continua en $0$, es decir, que:
                \begin{equation*}
                    \forall \varepsilon>0~\exists \delta>0 : \|T(x)\| < \varepsilon~\forall x\in B(0,\delta)
                \end{equation*}
                Tomando $\varepsilon=1$, la continuidad nos da un $\delta$ cumpliendo la afirmación anterior. Sea $x\in E$ arbitrario, tenemos:
                \begin{equation*}
                    \|T(x)\| = \left\|T\left(\dfrac{x}{\|x\|}\dfrac{\delta}{2}\dfrac{2\|x\|}{\delta}\right)\right\| = \dfrac{2\|x\|}{\delta} \left\|T\left(\dfrac{x}{\|x\|}\dfrac{\delta}{2}\right)\right\| < \dfrac{2}{\delta} \|x\|
                \end{equation*}
                Ya que $\dfrac{x\delta}{2\|x\|}\in B(0,\delta)$, por lo que tomando $M=\frac{2}{\delta}$ tenemos la implicación.
            \item [$(5)\Longrightarrow (6)$] Como $\overline{B}(0,1)$ es acotado, $T(\overline{B}(0,1))$ será acotado por ser $T$ acotada.
            \item [$(6)\Longrightarrow (7)$] Como $B(0,1)\subset \overline{B}(0,1)$, entonces $T(B(0,1))\subset T(\overline{B}(0,1))$.
            \item [$(7)\Longrightarrow (4)$] Si $\exists R\in \mathbb{R}^+$ de forma que $\|T(x)\| \leq R$ para todo $x\in B(0,1)$, dado $\varepsilon>0$, si tomamos $\delta = \frac{\varepsilon}{2R}$, si $x\in B(0,\delta)$, entonces:
                \begin{equation*}
                    \|T(x)\| = \left\|T\left(\dfrac{x}{2\|x\|}2\|x\|\right)\right\| = 2\|x\|\left\|T\left(\dfrac{x}{2\|x\|}\right)\right\| \leq 2\|x\|R < 2\delta R = \varepsilon
                \end{equation*}
            \item [$(1)\Longrightarrow (5)$] Sea $A\subset E$ acotado, entonces $\exists r\in \mathbb{R}^+$ de forma que $A\subset B(0,r)$, por lo que:
                \begin{equation*}
                    \|T(x)\| \leq M\|x\| \leq Mr \qquad \forall x\in A
                \end{equation*}
                De donde $T(A)\subset B(0,Mr)$, por lo que es un conjunto acotado.
        \end{description}
    \end{proof}
\end{prop}

\begin{prop}
    Sea $E$ un espacio normado, observemos que $E^\ast$ es un espacio vectorial, sobre el que definimos la aplicación $\|\cdot \|:E^\ast \to \mathbb{R}$ dada por:
    \begin{equation*}
        \|f\| = \sup_{\|x\| \leq 1} |f(x)| \qquad \forall f\in E^\ast
    \end{equation*}
    Se verifica que:
    \begin{enumerate}
        \item $(E^\ast, \|\cdot \|)$ es un espacio normado.
        \item $(E^\ast, \|\cdot \|)$ es un espacio de Banach.
        \item Sea $f\in E^\ast$, entonces:
            \begin{equation*}
                \sup_{\|x\|\leq 1}|f(x)| = \|f\| = \sup_{\|x\|=1}|f(x)|
            \end{equation*}
        \item Sea $f\in E^\ast$, entonces:
            \begin{equation*}
                \sup_{\|x\| \leq 1} |f(x)| = \|f\| = \inf \{M\in \mathbb{R}^+_0 : |f(x)| \leq M\|x\| \quad \forall x\in E\}
            \end{equation*}
    \end{enumerate}
    \begin{proof}
        Veamos cada una de las propiedades:
        \begin{enumerate}
            \item Para la primera, hemos de probar:
                \begin{itemize}
                    \item \textbf{No degeneración.} Sea $f\in E^\ast$ de forma que $\sup\limits_{\|x\|\leq 1}|f(x)| = \|f\| = 0$, entonces:
                        \begin{equation*}
                            0\leq \|f(x)\|\leq 0 \quad \forall x\in \overline{B}(0,1) \Longrightarrow f(x) = 0 \quad \forall x\in \overline{B}(0,1)
                        \end{equation*}
                        de donde:
                        \begin{equation*}
                            f(x) = f\left(\dfrac{x}{\|x\|}\|x\|\right) = \|x\| f\left(\dfrac{x}{\|x\|}\right) = 0 \qquad \forall x\in E
                        \end{equation*}
                        Por lo que $f = 0$.
                    \item \textbf{Homogeneidad por homotecias.} Sea $f\in E^\ast$ y $\lm\in \mathbb{R}A:$
                        \begin{equation*}
                            \|\lm f\| = \sup_{\|x\|\leq 1} |\lm f(x)|  = \sup_{\|x\|\leq 1} |\lm||f(x)| = |\lm| \sup_{\|x\|\leq 1}|f(x)| = |\lm| \|f\|
                        \end{equation*}
                    \item \textbf{Desigualdad triangular.} Sean $f,g\in E^\ast$:
                        \begin{align*}
                            \|f+g\| &= \sup_{\|x\|\leq 1}|f(x) + g(x)| \leq \sup_{\|x\|\leq 1} (|f(x)| + |g(x)|) \\ &\leq \sup_{\|x\|\leqq 1}|f(x)| + \sup_{\|x\|\leq 1} |g(x)| = \|f\| + \|g\|
                        \end{align*}
                \end{itemize}
            \item Sea $\{f_n\}$ una sucesión de Cauchy de elementos de $E^\ast$, sean $\varepsilon,r>0$, la condición de Cauchy para $\nicefrac{\varepsilon}{r}$ nos da $m\in \mathbb{N}$ de forma que si $p,q\geq m$, entonces:
                \begin{equation*}
                    \sup_{\|x\|\leq 1} |f_p(x) - f_q(x)| = \|f_p - f_q\| < \dfrac{\varepsilon}{r}
                \end{equation*}
                de donde:
                \begin{equation*}
                    |f_p(x) - f_q(x)| < \dfrac{\varepsilon}{r}\qquad \forall x\in \overline{B}(0,1)
                \end{equation*}
                pero entonces:
                \begin{equation*}
                    |f_p(rx) - f_q(rx)| = r|f_p(x) - f_q(x)| < \varepsilon \qquad \forall x\in \overline{B}(0,1)
                \end{equation*}
                lo que equivale a que:
                \begin{equation*}
                    |f_p(x) - f_q(x)| < \varepsilon\qquad \forall x\in \overline{B}(0,r)
                \end{equation*}
                Por tanto, la sucesión $\{f_n(x)\}$ es de Cauchy para todo $x\in \overline{B}(0,r)$, pero como $r$ era arbitrario, dicha condición se cumple para todo $r\in \mathbb{R}^+$, tenemos que $\{f_n(x)\}$ es de Cauchy para todo $x\in E$. Como $\mathbb{R}$ es completo, la sucesión $\{f_n(x)\}$ es convergente para todo $x\in E$, lo que nos permite definir $f:E\to \mathbb{R}$ dada por:
                \begin{equation*}
                    f(x) = \lim \{f_n(x)\} \qquad \forall x\in E
                \end{equation*}
                Se verifica que $f$ es lineal, ya que:
                \begin{align*}
                    f(\lm x + y) &= \lim \{f_n(\lm x + y)\} = \lim \{\lm f_n(x)  + f_n(y)\} \\ &= \lm \lim\{f_n(x)\} +  \lim\{f_n(y)\} = \lm f(x) + f(y) \\
                                 & \forall \lm \in \mathbb{R}, \forall x,y\in E
                \end{align*}
                Ahora, como $\{f_n\}$ era de Cauchy, tenemos que fijado $r\in \mathbb{R}^+$ y dado $\varepsilon>0$, $\exists m\in \mathbb{N}$ de forma que para $p,q\geq m$ se tiene:
                \begin{equation*}
                    |f_p(x) - f_q(x)| < \dfrac{\varepsilon}{2} \qquad \forall x\in \overline{B}(0,r)
                \end{equation*}
                Fijado ahora dicho $p$, tenemos:
                \begin{equation*}
                    |f_p(x) - f(x)| = \lim_{q\to \infty} |f_p(x) - f_q(x)| \leq \dfrac{\varepsilon}{2} < \varepsilon
                \end{equation*}
                Por lo que $\{f_n\}$ converge uniformemente a $f$ en $B(0,r)$, para todo $r\in \mathbb{R}^+$. En particular, $\{f_n\}$ converge uniformemente a $f$ en cada conjunto acotado de $E$. Como $\{f_n\}$  es continua $\forall n\in \mathbb{N}$ y para cada $x\in E$ tenemos que $\{f_n\}$ converge uniformemente a $f$ en $B(x,1)$, entonces tenemos que $f $ es continua en $x$, de donde $f$ es continua en $E$. En consecuencia, $f\in E^\ast$.

                Por último, para ver que $\{f_n\}$ converge a $f$, dado $\varepsilon>0$, existe $m\in \mathbb{N}$ de forma que si $n\geq m$, entonces:
                \begin{equation*}
                    |f_n(x) - f(x)| < \dfrac{\varepsilon}{2} \qquad \forall x\in \overline{B}(0,1)
                \end{equation*}
                de donde:
                \begin{equation*}
                    \|f_n - f\| = \sup_{\|x\|\leq 1} |f_n(x) - f(x)| \leq \dfrac{\varepsilon}{2} < \varepsilon
                \end{equation*}
                Por lo que $\{f_n\}\to f$.
            \item La desigualdad $\geq$ es obvia. Para la otra, sea $x\in B(0,1)$:
                \begin{equation*}
                    |f(x)| = \left|f\left(\frac{x}{\|x\|}\right)\right|\|x\| \leq \sup_{\|x\|=1}|f(x)|\|x\| \leq \sup_{\|x\|=1}|f(x)|
                \end{equation*}
            \item Buscamos probar que:
                \begin{equation*}
                    \sup_{\|x\|\leq 1}|f(x)| = \inf\{M\in \mathbb{R}^+_0 : |f(x)| \leq M\|x\| \quad \forall x\in E\}
                \end{equation*}
                \begin{description}
                    \item [$\geq)$] Para ver que el supremo es mayor o igual que el ínfimo, veamos que el supremo pertence al conjunto de la derecha:
                        \begin{equation*}
                            |f(x)| = \|x\| \left|f\left(\dfrac{x}{\|x\|}\right)\right| \leq \|x\| \sup_{\|x\|\leq 1}|f(x)|
                        \end{equation*}
                        Por tanto, $\sup\limits_{\|x\|\leq1}|f(x)| \in \{M\in \mathbb{R}^+_0 : |f(x)|\leq M\|x\| \quad \forall x\in E\}$.
                    \item [$\leq)$] Para ver el el ínfimo es mayor o igual que el supremo, veamos que el ínfimo es un mayorante del conjunto de la izquierda, si tomamos:
                        \begin{equation*}
                            M_0 = \inf\{M\in \mathbb{R}^+_0 : |f(x)| \leq M\|x\| \quad \forall x\in E\}
                        \end{equation*}
                        entonces:
                        \begin{equation*}
                            |f(x)| \leq M\|x\| \leq M  \qquad \forall x\in \overline{B}(0,1)
                        \end{equation*}
                        Por lo que $M_0$ es un mayorante de $\{|f(x)| : \|x\| \leq 1\}$, por lo que es mayor o igual que su supremo.
                \end{description}
        \end{enumerate}
    \end{proof}
\end{prop}

\subsection{Espacio dual de un espacio de Hilbert}
\begin{prop}
    Se verifica que si $v\in H$, entonces la aplicación
    \Func{\varphi_v}{H}{\mathbb{R}}{u}{\langle u,v\rangle}
    verifica que $\varphi_v\in H^\ast$ y en cuyo caso, $\|\varphi_v\| = \|v\|$.\\

    \noindent
    Más aún, podemos definir 
    \Func{\Phi}{H}{H^\ast}{v}{\varphi_v}
    que es una aplicación lineal e inyectiva.
    \begin{proof}
        Como el producto escalar es bilineal es evidente que $\varphi_v$ es lineal. Vemos que:
        \begin{align*}
            |\varphi_v(u) - \varphi_v(w)| &= |\langle u,v \rangle -\langle w,v \rangle | = |\langle u-w,v \rangle | \leq \|u-w\|\|v\| \qquad \forall u,w\in E
        \end{align*}
        Por lo que $\varphi_v$ es lipschitziana, y por la última Proposición tenemos que $\|\varphi_v\| \leq \|v\|$. Si $v=0$ tenemos la igualdad de forma obvia y si $v\neq 0$, entonces:
        \begin{equation*}
            \|v\| = \dfrac{\|v\|}{\|v\|^2} = \dfrac{\langle v,v \rangle }{\|v\|} = \left\langle \dfrac{v}{\|v\|},v \right\rangle  = \varphi_v\left(\dfrac{v}{\|v\|}\right)
        \end{equation*}
        luego:
        \begin{equation*}
            \|v\| \leq \sup_{\|x\|\leq 1}|f(x)| = \|\varphi_v\|
        \end{equation*}
        Para ver que $\Phi$ es lineal, sean $\lm\in \mathbb{R}$ y $u,v\in H$:
        \begin{equation*}
            \Phi(\lm u+v) = \varphi_{\lm u+v} \stackrel{\text{?}}{=} \lm \varphi_u + \varphi_v = \lm \Phi(u) + \Phi(v)
        \end{equation*}
        donde la igualdad puede demostrarse por:
        \begin{equation*}
            \varphi_{\lm u + v}(w) = \langle w,\lm u + v \rangle  = \langle w,\lm u \rangle  + \langle w,v \rangle  = \lm \langle w,u \rangle  + \langle w,v \rangle  = \lm \varphi_u(w) + \varphi_v(w)
        \end{equation*}
        Como $\|\varphi_v\|  = \|v\|$, obtenemos de forma inmediata la continuidad de $\Phi$, por ser una isometría.\\

        \noindent
        Para ver que $\Phi$ es inyectiva, supongamos que $u,v\in H$ con $\Phi(u) = \Phi(v)$, de donde:
        \begin{equation*}
            \langle u,w \rangle  = \langle v,w \rangle  \qquad \forall w\in H
        \end{equation*}
        Luego:
        \begin{equation*}
            \langle u,w \rangle  - \langle v,w \rangle  = \langle u-v,w \rangle  = 0 \qquad \forall w\in H
        \end{equation*}
        En particular, tomando $w= u-v$, tenemos que:
        \begin{equation*}
            \|u-v\|^2 = \langle u-v,u-v \rangle  = 0
        \end{equation*}
        Por lo que $u=v$, de donde $\Phi$ es inyectiva.
    \end{proof}
\end{prop}

\begin{teo}[de Riesz-Fréchet, Representación del dual de un Hilbert]\ \\
    Sea $H$ un espacio de Hilbert, $\forall \varphi\in H^\ast$ $\exists _1 v\in H$ de forma que:
    \begin{equation*}
        \varphi(u) = \langle u,v \rangle  \qquad \forall u\in H
    \end{equation*}

    y además:
    \begin{equation*}
        \|\varphi\| = \|v\|
    \end{equation*}
    \begin{proof}
        Si conseguimos probar la primera parte del Teorema, la segunda la tendremos ya probada gracias a la Proposición anterior. Sea por tanto $f\in H^\ast$, si $f=0$ tomando $v=0$ se tiene la tesis. Suponemos por tanto que $f\neq 0$, por lo que $M = f^{-1}(\{0\})\subsetneq H$ es un espacio vectorial de $H$ distinto del trivial. Como $f$ es continua, tenemos además que $M$ es un conjunto cerrado.\\

        \noindent
        Como $M\subsetneq H$, podemos tomar $z_o\in H\setminus M$. Por el Teorema de la Proyección Ortogonal, tomamos $z_1 = P_Mz_0 \in M$, que verifica:
        \begin{equation*}
            \langle z_0 - z_1, v \rangle  = 0 \qquad \forall v\in M
        \end{equation*}
        Como $z_0\in H\setminus M$ y $z_1 \in M$, tenemos que $z_0 \neq z_1$, lo que nos permite definir:
        \begin{equation*}
            z = \dfrac{z_0 - z_1}{\|z_0 - z_1\|}
        \end{equation*}
        Con esta definición, es claro que $\|z\| = 1$, así como que:
        \begin{equation*}
            \langle z,v \rangle = \dfrac{1}{\|z_0 - z_1\|}\langle z_0 - z_1,v \rangle  = 0 \qquad \forall v\in M
        \end{equation*}
        Como $z_0 \notin M$ la situación $z\in M$ es imposible, por lo que $z\notin M$, luego $f(z) \neq 0$. Veamos ahora que:
        \begin{equation*}
            x-\dfrac{f(x)}{f(z)}z \in M \qquad \forall x\in H
        \end{equation*}
        ya que:
        \begin{equation*}
            f\left(x-\dfrac{f(x)}{f(z)}z\right) = f(x) - \dfrac{f(x)}{f(z)}f(z) = 0
        \end{equation*}
        Por lo que tenemos que:
        \begin{equation*}
            \left\langle z,x-\dfrac{f(x)}{f(z)}z \right\rangle  = 0
        \end{equation*}
        Pero tenemos:
        \begin{equation*}
            0 = \left\langle z,x-\dfrac{f(x)}{f(z)}z \right\rangle  = \langle z,x \rangle - \dfrac{f(x)}{f(z)} \langle z,z \rangle  = \langle z,x \rangle  - \dfrac{f(x)}{f(z)} \|z\|^2
        \end{equation*}
        Por lo que podemos despejar $f(x)$, obteniendo:
        \begin{equation*}
            f(x) = f(z)\langle z,x \rangle = \langle x,zf(z) \rangle   \qquad \forall x\in H
        \end{equation*}
        En consecuencia, tomando $v = zf(z)$ tenemos la existencia probada.\\

        \noindent
        Para la unicidad, supongamos que $\exists v,w\in H$ de forma que:
        \begin{equation*}
            \langle x,v \rangle  = f(x) = \langle x,w \rangle  \qquad \forall x\in H
        \end{equation*}
        En consecuencia:
        \begin{equation*}
            \langle x,v-w \rangle  = 0 \qquad \forall x\in H
        \end{equation*}
        Luego si tomamos $x=v-w$:
        \begin{equation*}
            \|v-w\|^2 = \langle v-w,v-w \rangle  = 0
        \end{equation*}
        Por lo que $v=w$.
    \end{proof}
\end{teo}

\noindent
A partir del Teorema anterior tenemos que $(\mathbb{R}^N, \|\cdot \|_2)$, $l^2$ y $L^2(\Omega)$ son todos isomorfos a sus duales.

\begin{ejercicio} % // TODO: HACER
    Calcular el dual de $l^p$, para $p>1$, $p\neq 2$.\\

    \noindent
    Puede encontrarse en la Sección~\ref{sec:duallp}.
\end{ejercicio}

\begin{notacion}
    Si $x\in E$ y $f\in E^\ast$, a menudo notaremos:
    \begin{equation*}
        \langle f,x \rangle = f(x)
    \end{equation*}
    Esta notación se debe a que la evaluación de una aplicación lineal y continua $f$ en un punto $x$ cumplen unas propiedades que nos recuerdan a la del producto escalar:
    \begin{enumerate}
        \item $\langle \lm f+ g,x \rangle = \lm \langle f,x  \rangle + \langle g,x \rangle   $.
        \item $\langle f,\lm x + y\rangle = \lm \langle f,x \rangle + \langle f,y \rangle   $.
        \item $\langle f,x \rangle \leq \|f\|\|x\|$.
    \end{enumerate}
\end{notacion}

\section{Teorema de Hahn-Banach}
\noindent
Si tenemos un espacio normado $E$ de dimensión finita, resulta fácil dar una aplicación lineal y continua $f:E\to \mathbb{R}$, pero en dimensión finita el problema se complica. A continuación veremos el Teorema de Hahn-Banach, que entre sus muchas utilidades una de ellas es probar que si $E$ es un espacio normado de dimensión infinita entonces $E^\ast\neq \{0\}$. Para resolver este problema, como somos capaces de calcular aplicciones lineales y continuas en dimensión finita y dentro de espacios de dimensión infinita somos capaces de encontrar espacios de dimensión finita, nos preguntamos:\\

\noindent
\textbf{Problema}\newline
Sea $E$ un espacio de Banach, $G\subset E$ un subespacio suyo y $g:G\to \mathbb{R}$ lineal y continua, ¿podemos garantizar entonces que existe $f:E\to \mathbb{R}$ lineal y continua tal que $f\big|_G = g$?\\

\noindent
Como ya vimos en la Proposición~\ref{prop:lineal_continuidad_acotacion}, que $g$ sea continua significa que $\exists k\in \mathbb{R}^+$ de forma que $|g(x)| \leq k\|x\|$ $\forall x\in G$. Para resolver el problema, necesitamos encontrar una aplicación $f:E\to \mathbb{R}$ lineal y $k'\in \mathbb{R}^+$ de forma que:
\begin{equation*}
    f\big|_G = g \qquad \text{y}\qquad |f(x)| \leq k'\|x\| \qquad \forall x\in E
\end{equation*}

\begin{ejercicio}\label{ej:aplicacion_p}
    Sea $p:(E,\|\cdot \|)\to \mathbb{R}$ dada por:
    \begin{equation*}
        p(x) = k\|x\| \qquad \forall x\in E
    \end{equation*}
    Demostrar que la función $p$ verifica:
    \begin{itemize}
        \item $p(x+y) \leq p(x) + p(y) \qquad \forall x,y\in E$.
        \item $p(\lm x) = \lm p(x) \qquad \forall \lm \in \mathbb{R}^+, \forall x\in E$.
    \end{itemize}
    \begin{proof}
        Sean $x,y\in  E$ y $\lm\in \mathbb{R}^+$:
        \begin{gather*}
            p(x+y)= k\|x+y\| \leq k(\|x\| + \|y\|) = k\|x\| + k\|y\| = p(x) + p(y) \\
            p(\lm x) = k\|\lm x\| = \lm k \|x\| = \lm p(x)
        \end{gather*}
    \end{proof}
\end{ejercicio}

\noindent
Aunque no lo demostraremos, el Teorema de Hahn-Banach resulta ser equivalente al axioma de elección. Para realizar la demostración del Teorema de Hahn-Banach es necesario usar el Lema de Zorn, por lo que conviene realizar un breve repaso del mismo. 

\subsubsection{Lema de Zorn}
\begin{definicion}
    Sea $\emptyset \neq P$ un conjunto con una relación $\leq$ de orden, es decir, una relación reflexiva, antisimétrica y transitiva, decimos que:
    \begin{itemize}
        \item Un subconjunto $Q\subset P$ es \underline{totalmente ordenado} si:
            \begin{equation*}
                \forall a,b\in Q \Longrightarrow a\leq b\ \lor\ b\leq a
            \end{equation*}
        \item Si $Q\subset P$ y $x\in P$, decimos que $x$ es una \underline{cota superior} de $Q$ si y solo si:
            \begin{equation*}
                a\leq x \qquad \forall a\in Q
            \end{equation*}
        \item Si $m\in P$, decimos que $m$ es un \underline{elemento maximal} de $P$ si y solo si:
            \begin{equation*}
                \{x\in P : m \leq x\} = \{m\}
            \end{equation*}
        \item Diremos que $P$ es \underline{inductivo} si todo subconjunto $Q\subset P$ que sea totalmente ordenado posee una cota superior.
    \end{itemize}
\end{definicion}

\begin{lema}[de Zorn]
    Si $P$ es un conjunto no vacío con una relación de orden $\leq$ y $P$  es inductivo, entonces $P$ tiene un elemento maximal.
\end{lema}

\begin{teo}[de Hahn-Banach, versión analítica]
    Sea $E$ un espacio vectorial, sea $p:E\to \mathbb{R}$ una aplicación verificando:
    \begin{itemize}
        \item $p(x+y) \leq p(x) + p(y) \qquad \forall x,y\in E$.
        \item $p(\lm x) = \lm p(x) \qquad \forall \lm \in \mathbb{R}^+ , \quad \forall x\in E$.
    \end{itemize}
    Sea $G\subset E$ un subespacio vectorial y $g:G\to \mathbb{R}$ una aplicación lineal tal que
    \begin{equation*}
        g(x) \leq p(x) \qquad \forall x\in G
    \end{equation*}
    Entonces, $\exists f:E\to \mathbb{R}$ lineal de forma que: 
    \begin{enumerate}
        \item $f(x) \leq p(x) \qquad \forall x\in E$.
        \item $f\big|_G = g$.
    \end{enumerate}
    \begin{proof}
        Definimos el conjunto $P$ de todas aquellas aplicaciones lineales $h$ que tienen por dominios subespacios vectoriales de $E$ que contienen a $G$ de forma que $h\big|_G = g$ y que cumplen la desigualdad $h(x) \leq p(x) \quad \forall x\in D(h)$ (donde $D(h)$ denota el dominio de $h$); es decir:
        \begin{equation*}
            P = \left\{h:D(h) \to \mathbb{R} : \left.\begin{array}{l}
                G\subset D(h) \text{\ subespacio vectorial de\ } E \\
                h(x) = g(x) \quad \forall x\in G\\
                h \text{\ lineal y\ } h(x) \leq p(x) \quad \forall x\in D(h) 
            \end{array}\right.\right\}
        \end{equation*}
        Tendremos entonces en $P$ todas aquellas alicaciones lineales definidas en espacios vectoriales que son extensiones de $g$ y que cumplen la condición de estar dominadas por $p$. Buscamos aplicar el Lema de Zorn sobre $P$, obteniendo un elemento maximal que luego probaremos que ha de tener como dominio $E$.\\

        \noindent
        Hemos pues de definir una relación de orden en $P$ que nos permita conseguir lo que queremos. Para ello, definiremos la relación $\leq$ de la siguiente forma:
        \begin{equation*}
            h_1 \leq h_2 \Longleftrightarrow \left\{\begin{array}{l}
                D(h_1) \subset D(h_2) \\
                h_2\big|_{D(h_1)} = h_1
            \end{array}\right. \qquad \forall h_1,h_2\in P
        \end{equation*}
        es decir, $h_1\leq h_2$ si $h_2$ es una extensión de $h_1$. Podemos comprobar que esta efectivamente es una relación de orden en $P$:
        \begin{itemize}
            \item \textbf{Reflexiva.} Si $h\in P$, trivialmente tenemos que $D(h)\subset D(h)$ y $h\big|_{D(h)} = h$, lo que nos dice que $h\leq h$.
            \item \textbf{Antisimétrica.} Sean $h_1,h_2\in P$ de forma que $h_1\leq h_2$ y $h_2\leq h_1$, entonces:
                \begin{equation*}
                    D(h_1)\subset D(h_2) \ \land\ D(h_2) \subset D(h_1) \Longrightarrow D(h_1) = D(h_2) 
                \end{equation*}
                Y de esta condición junto con $h_2 = h_2\big|_{D(h_2)} = h_2\big|_{D(h_1)} = h_1$ concluimos que $h_2 = h_1$.
            \item \textbf{Transitiva.} Si $h_1,h_2,h_3\in P$ con $h_1\leq h_2$ y $h_2\leq h_3$, tenemos entonces que $D(h_1)\subset D(h_2)$ y que $D(h_2)\subset D(h_3)$. La transitividad de $\subset$ nos dice que $D(h_1)\subset D(h_3)$. Ahora, si tenemos que $h_3\big|_{D(h_2)} = h_2$ y que $h_2\big|_{D(h_1)} = h_1$, obtenemos que:
                \begin{equation*}
                    h_3\big|_{D(h_1)} = h_2\big|_{D(h_1)} = h_1
                \end{equation*}
                De donde $h_1\leq h_3$.
        \end{itemize}

        \noindent
        Tratemos ahora de probar que $P$ es inductivo. Para ello, sea $Q\subset P$ un subconjunto totalmente ordenado, para buscar una cota superior de $Q$ consideraremos:
        \begin{equation*}
            V_0 = \bigcup_{h\in Q}D(h)
        \end{equation*}
        Vemos que $V_0$ es un subespacio vectorial de $E$, ya que si $\alpha\in \mathbb{R}$ y $u,v\in V_0$, tenemos entonces que $\exists h,h'\in Q$ de forma que $u\in D(h), v\in D(h')$. Como $Q$ es totalmente ordenado, tendremos entonces que $h\leq h'$ o que $h'\leq h$. Supondremos sin pérdida de generalidad que $h\leq h'$, lo que nos dice que $D(h)\subset D(h')$, por lo que $u\in D(h')$ y como $D(h')$ es un subespacio vectorial de $E$, tenemos entonces que:
        \begin{equation*}
            \alpha u + v \in D(h') \subset V_0
        \end{equation*}
        Una vez salvada esta cuestión, definimos $h_0:V_0\to \mathbb{R}$ por:
        \begin{equation*}
            h_0(x) = h(x) \qquad \text{si\ } x\in D(h)
        \end{equation*}
        que está bien definida, ya que si $x\in D(h_1)\cap D(h_2)$, sucederá bien $h_1 \leq h_2$ bien $h_2 \leq h_1$, luego suponiendo que $h_1\leq h_2$, tendremos que $h_2\big|_{D(h_1)} = h_1$, luego se cumplirá $h_1(x) = h_2(x)$. Además $h_0$ es lineal, ya que si $x,y\in V_0$, por ser $V_0$ espacio vectorial tendremos que $x+y\in V_0$, de donde $\exists h,h',h'' \in Q$ de forma que $x\in D(h)$, $y\in D(h')$, $x+y\in D(h'')$, con lo que:
        \begin{equation*}
            h_0(x+y) = h''(x+y) = h''(x) + h''(y) = h(x) + h'(y) = h_0(x) + h_0(y)
        \end{equation*}
        Y finalmente es claro que $h_0(x) \leq p(x)\quad \forall x\in V_0$, puesto que si $x\in V_0$, entonces $\exists h\in Q$ de forma que $x\in D(h)$, con lo que:
        \begin{equation*}
            h_0(x) = h(x) \leq p(x)
        \end{equation*}
        En definitiva, tenemos que $h_0$ es una aplicación lineal extensión de $g$ que cumple $h(x) \leq p(x)$ para todo $x\in V_0$ y con $V_0$ un subespacio vectorial de $E$ que claramente contiene a $G$, con lo que $h_0\in P$ y además tenemos que $h\leq h_0$ $\forall h\in Q$, por lo que $h_0$ es una cota superior de $Q$, de donde tenemos que $P$ es inductivo. Por el Lema de Zorn, existe $f\in P$ elemento maximal de $P$.\\

        \noindent
        Para concluir la demostración del Teorema, nos falta probar que si $f$ es un elemento maximal de $P$ entonces $D(f) = E$. Para ello, supongamos por reducción al absurdo que fuese $D(f)\subsetneq E$, luego existe $x_0\in E\setminus D(f)$. Si consideramos\footnote{Aquí hemos usado $\mathbb{R} x_0:= \{rx_0  : r\in \mathbb{R}\}$, que es un subespacio vectorial de $E$ de dimensión 1.}:
        \begin{equation*}
            D(f) \oplus \mathbb{R}x_0
        \end{equation*}
        Tenemos que si $v\in D(f)\oplus \mathbb{R}x_0$, entonces $v$ se escribe como $v = x+tx_0$, con $x\in D(f)$ y $t\in \mathbb{R}$, lo que nos permite definir $\hat{f}:D(f)\oplus \mathbb{R}x_0\to \mathbb{R}$ dada por:
        \begin{equation*}
            \hat{f}(x+tx_0) = f(x) + t\alpha
        \end{equation*}
        Siendo $\alpha$ un número real que por ahora no concretaremos (puesto que necesitamos buscar luego una condición sobre $\alpha$ para garantizar que $\hat{f}\in P$). Veamos que $\hat{f} \in P$:
        \begin{itemize}
            \item Es automático que $\hat{f}\big|_{D(f)} = f$, por lo que $\hat{f}(x) = g(x)\quad \forall x\in G$.
            \item $D(f)\oplus \mathbb{R}x_0$ es un subespacio vectorial de $E$ que contiene a $G$.
            \item Es fácil ver que $\hat{f}$ es lineal, ya que si $x,y\in D(f)$ y $t,t'\in \mathbb{R}$:
                \begin{align*}
                    \hat{f}(x+tx_0 + y+t'x_0) &= \hat{f}((x+y) + (t+t')x_0) = f(x+y) + (t+t')\alpha \\
                                              &= f(x) + f(y) + t\alpha + t'\alpha = \hat{f}(x+tx_0) + \hat{f}(y+t'x_0)
                \end{align*}
            \item Tenemos que ver finalmente que 
                \begin{equation}\label{eq:fgorromenorp}
                    \hat{f}(x+tx_0) \leq p(x+tx_0)\qquad \forall x\in D(f),\quad  \forall t\in \mathbb{R}
                \end{equation}
                que sucede si y solo si:
                \begin{equation*}
                    t\hat{f}(z+x_0) = \hat{f}(tz+tx_0) \leq p(tz+tx_0) = p(t(z+x_0)) \qquad \forall z\in D(f), \quad \forall t\in \mathbb{R}
                \end{equation*}
                \begin{itemize}
                    \item En el caso $t=0$ la desigualdad es obvia.
                    \item Si $t>0$, tenemos que:
                        \begin{equation*}
                            t(f(z) + \alpha) = t\hat{f}(z+x_0) \leq p(t(z+x_0)) = tp(z+x_0)  \qquad \forall z\in D(f)
                        \end{equation*}
                        que es equivalente a
                        \begin{equation*}
                            \alpha \leq -f(z) + p(z+z_0) \qquad \forall z \in D(f)
                        \end{equation*}
                    \item Si $t<0$, tenemos:
                        \begin{multline*}
                            -t(-f(z)-\alpha) = -t\hat{f}(-z-x_0) = t\hat{f}(z+x_0) \leq p(t(z+x_0)) = -tp(-z-x_0) \\ \forall z\in D(f)
                        \end{multline*}
                        que es equivalente a
                        \begin{equation*}
                            -f(z) - p(-z-x_0) \leq \alpha \qquad \forall z\in D(f)
                        \end{equation*}
                \end{itemize}
                En definitiva, ver~(\ref{eq:fgorromenorp}) es equivalente a ver que:
                \begin{equation*}
                    \sup\{f(-z)-p(-z-x_0) : z\in D(f)\} \leq \alpha \leq \inf\{-f(z)+p(z+x_0) : z\in D(f)\}
                \end{equation*}
                que a su vez equivale a:
                \begin{equation*}
                    \sup\{f(w)-p(w-x_0) : w\in D(f)\} \leq \alpha \leq \inf\{-f(z)+p(z+x_0) : z\in D(f)\}
                \end{equation*}
                Por tanto, si probamos que el supremo de la izquierda es menor o igual que el ínfimo de la derecha, elegiendo $\alpha$ cualquier valor real comprendido entre ambos (o incluso igual al supremo o al ínfimo) habremos construido una aplicación $\hat{f}$ que cumple con los tres puntos anteriores y con la condición~(\ref{eq:fgorromenorp}), que es la condición que veníamos buscando.

                Para demostrar la desigualdad entre supremo e ínfimo, basta observar que para $z,w\in D(f)$ se verifica:
                \begin{equation*}
                    f(z) + f(w) = f(z+w) \leq p(z+w) = p(z+x_0-x_0+w) \leq p(z+x_0) + p(w-x_0)
                \end{equation*}
                y despejando llegamos a que:
                \begin{equation*}
                    f(w) - p(w-x_0) \leq -f(z) + p(z+x_0) \qquad \forall z,w\in D(f)
                \end{equation*}
                Lo que demuestra que:
                \begin{equation*}
                    \sup\{f(-z)-p(-z-x_0) : z\in D(f)\} \leq \inf\{-f(z)+p(z+x_0) : z\in D(f)\}
                \end{equation*}
                Como hemos comentado anteriormente, tomando por ejemplo:
                \begin{equation*}
                    \alpha = \sup\{f(-z)-p(-z-x_0) : z\in D(f)\} \in \mathbb{R}
                \end{equation*}
                en la definición de $\hat{f}$ nos garantiza la condición~(\ref{eq:fgorromenorp}), que junto con las otras condiciones nos dice que $\hat{f}\in P$. Además, por la definición de $\hat{f}$ es claro que $f\leq \hat{f}$, donde $f$ era un elemento maximal de $P$. Hemos llegado a una \underline{contradicción}, que venía de suponer que $D(f)\subsetneq E$, por lo que $D(f)$ ha de ser igual a $E$, luego hemos encontrado la aplicación que el Teorema enunciaba, lo que concluye la demostración.
        \end{itemize}
    \end{proof}
\end{teo}

\noindent
Volviendo al caso que nos interesaba, tenemos ya respuesta al Teorema anteriormente planteado:

\begin{coro}\label{coro:hahn-banach}
    Sea $E$ un espacio vectorial, $G\subset E$ un subespacio vectorial suyo y $g:G\to \mathbb{R}$ lineal y continua, existe entonces $f:E\to \mathbb{R}$ lineal y continua de forma que $f\big|_G = g$. Además:
    \begin{equation*}
        \|f\| = \|g\|
    \end{equation*}
    \begin{proof}
        Como $g$ es una aplicación lineal y continua, si recordamos que:
        \begin{equation*}
            \|g\| = \inf \{M>0 : |g(x)| \leq M\|x\| \quad \forall x\in G\}
        \end{equation*}
        Si definimos $p:E\to \mathbb{R}$ dada por $p(x) = \|g\|\|x\|$ para $x\in E$, vimos en el Ejercicio~\ref{ej:aplicacion_p} que $p$ verificaba:
        \begin{itemize}
            \item $p(x+y)\leq p(x) + p(y) \qquad \forall x,y\in E$
            \item $p(\lm x)=\lm p(x) \qquad \forall \lm \in \mathbb{R}^+, \forall x\in E$
        \end{itemize}
        y la condición que hemos expresado arriba nos dice que $g(x) \leq p(x)$ para todo $x\in G$. Aplicando el Teorema de Hahn-Banach, tenemos que existe una aplicación $f:E\to \mathbb{R}$ lineal que verifica:
        \begin{itemize}
            \item $f\big|_G = g$
            \item $f(x) \leq p(x)\qquad \forall x\in E$
        \end{itemize}
        falta ver que $f$ es continua para acabar la demostración. Para ello, observemos que la condición $f(x) \leq p(x) \quad \forall x\in E$ implica:
        \begin{equation*}
            -f(x) = f(-x) \leq p(-x) = \|g\|\|- x\| = \|g\|\|x\| = p(x) \qquad \forall x\in E
        \end{equation*}
        Por lo que tenemos que $|f(x)| \leq \|g\|\|x\| \quad \forall x\in E$, y vimos en la Proposición~\ref{prop:lineal_continuidad_acotacion} que esta condición para una aplicación lineal era equivalente a que la aplicación sea continua. Además, esta desigualdad implica que $\|f\| \leq \|g\|$. Si notamos ahora que:
        \begin{equation*}
            |g(x)| = |f(x)| \leq \|f\| \|x\| \qquad \forall x\in G
        \end{equation*}
        Deducimos entonces que $\|g\| \leq \|f\|$, por lo que $\|f\| = \|g\|$.
    \end{proof}
\end{coro}

\begin{coro}\label{coro:existencia_f0}
    Sea $E$ un espacio vectorial, $\forall x_0\in E$ $\exists f_0\in E^\ast$ de forma que:
    \begin{equation*}
        \|f_0\| = \|x_0\| \quad \text{y} \quad  f_0(x_0)= \|x_0\|^2
    \end{equation*}
    \begin{proof}
        Si $x_0 = 0$, tomando $f_0 = 0$ se tiene. Suponemos por tanto que $x_0\neq 0$. Sea $G = \mathbb{R} x_0$, defino $g:G\to \mathbb{R}$ dada por:
            \begin{equation*}
                g(tx_0) = t\|x_0\|^2 \qquad \forall t\in \mathbb{R}
            \end{equation*}
            es fácil ver que $g$ es lineal. Además es continua, ya que:
            \begin{equation*}
                |g(tx_0)| = |t|\|x_0\|^2 = \|x_0\|\|tx_0\| \qquad \forall t\in \mathbb{R}
            \end{equation*}
            en particular, acabamos de ver que $\|g\| \leq \|x_0\|$, pero como:
            \begin{equation*}
                |g(x_0)| = \|x_0\|^2 = \|x_0\|\|x_0\|
            \end{equation*}
            deducimos que $\|g\| = \|x_0\|$. Aplicando el Corolario anterior, existe $f_0\in E^\ast$ lineal y continua de forma que:
            \begin{equation*}
                f_0\big|_G  = g \qquad \|f_0\| = \|g\| = \|x_0\|
            \end{equation*}

            de donde:
            \begin{equation*}
                f_0(x_0) = f_0\big|_G(x_0) = g(x_0) = \|x_0\|^2 
            \end{equation*}
    \end{proof}
\end{coro}

\begin{coro}\label{coro:calcular_norma_x}
    Para todo $x_0\in E$ se tiene que:
    \begin{equation*}
        \|x_0\| = \sup\{|f(x_0)| : f\in E^\ast, \|f\| \leq 1\} = \max\{|f(x_0)| : f\in E^\ast, \|f\| \leq 1\}
    \end{equation*}
    \begin{proof}
        Si $x_0=0$, cualquier aplicación lineal cumple $f(0)=0$, luego es obvio el resultado. Supuesto que $x_0\in E\setminus\{0\}$, dada $f\in E^\ast$ con $\|f\| \leq 1$, tenemos entonces que:
        \begin{equation*}
            |f(x_0)| \leq \|f\|\|x_0\| \leq \|x_0\| \Longrightarrow \|x_0\| \geq \sup\{|f(x_0)| : f\in E^\ast, \|f\| \leq 1\} 
        \end{equation*}
        Para la otra desigualdad, por el Corolario anterior para $x_0$ sabemos que $\exists f_0\in E^\ast$ de forma que:
        \begin{equation*}
            \|f_0\| = \|x_0\|, \qquad f_0(x_0) = \|x_0\|^2
        \end{equation*}
        Si tomamos $f = \nicefrac{f_0}{\|x_0\|}$, tenemos entonces que:
        \begin{equation*}
            \|f\| = \left\|\dfrac{f_0}{\|x_0\|}\right\| = \dfrac{\|f_0\|}{\|x_0\|} = 1
        \end{equation*}
        Y además:
        \begin{equation*}
            f(x_0) = \dfrac{f(x_0)}{\|x_0\|} = \dfrac{\|x_0\|^2}{\|x_0\|} = \|x_0\|
        \end{equation*}
        Por lo que el supremo se alcanza, luego es un máximo.
    \end{proof}
\end{coro}

\subsection{Versiones geométricas del Teorema}
\noindent
Aunque no lo demostraremos, las sucesivas versiones geométricas del teorema de Hahn-Banach son equivalentes a la ya vista. Para realizar la formulación del Teorema será necesario tener claros ciertos conceptos:

\begin{definicion}[Hiperplano afín]
    Sea $E$ un espacio vectorial, un hiperplano afín de $E$ es un subconjunto $H\subset E$ de la forma:
    \begin{equation*}
        H = \{x\in E : f(x) = \alpha\} = f^{-1}(\{\alpha\})
    \end{equation*}
    donde $f:E\to \mathbb{R}$ es una aplicación lineal no nula y $\alpha\in \mathbb{R}$. En dicho caso, escribiremos $H = [f=\alpha]$.
\end{definicion}

\begin{observacion}
    Cuando trabajábamos en asignaturas anteriores en espacios vectoriales de dimensión finita (digamos $n$), para nosotros un hiperplano era un subespacio vectorial de dimensión $n-1$. Ahora, si nos encontramos en un espacio vectorial $E$ genérico (no necesariamente de dimensión finita), el primer Teorema de Isomorfía de aplicaciones lineales aplicado a $f$ nos da el isomorfismo lineal
    \begin{equation*}
        E/\ker f \cong \text{Im}f
    \end{equation*}
    Como $f$ era lineal, ha de ser obligatoriamente $\text{dim}\ \text{Im}f = 1$. Observemos que en el caso $H = [f=0] = \ker f$, tenemos que $\text{dim}(E/H) = 1$, de donde si $E$ es de dimensión finita, tenemos $\text{dim}H = \text{dim}E - 1$. Si consideramos ahora $H=[f=\alpha]$ con $\alpha\neq 0$, tenemos que:
    \begin{equation*}
        H = \{x\in E : f(x) = \alpha\} = \{x+v : x\in E, v\in \ker f, f(x) = \alpha\}
    \end{equation*}
    Por lo que podemos ver $H$ como un trasladado de $\ker f$, como un hiperplano afín, con espacio de direcciones $\ker f$.
\end{observacion}

\begin{prop}
    El hiperplano $H = [f=\alpha]$ es cerrado si y solo si $f$ es continua.
    \begin{proof}
        Por doble implicación:
        \begin{description}
            \item [$\Longleftarrow )$] Si $f$ es continua, tenemos que $H = f^{-1}(\{\alpha\})$, por lo que $H$ será un conjunto cerrado, como imagen inversa de un conjunto cerrado por una aplicación continua.
            \item [$\Longrightarrow )$] Supuesto que $H$ es cerrado, tenemos entonces que $E\setminus H$ es abierto y no vacío (ya que $f$ no se anula totalmente). Sea $x_0\in E\setminus H$ de forma que $f(x_0)\neq \alpha$, podemos suponer sin pérdida de generalidad que $f(x_0)<\alpha$.

                Fijado $r>0$ de forma que $B(x_0,r)\subset E\setminus H$, se cumple que $f(x)<\alpha$\newline$\forall x\in B(x_0,r)$, ya que si $f(x_1)> \alpha$ para cierto $x_1\in B(x_0,r)$. El segmento:
                \begin{equation*}
                    \{x_t = (1-t)x_0 + tx_1 : t\in [0,1]\} 
                \end{equation*}
                está contenido en $B(x_0,r)\subset E\setminus H$, por lo que $f(x_t)\neq \alpha\quad \forall t\in [0,1]$, pero si tomamos:
                \begin{equation*}
                    t = \frac{f(x_1)-\alpha}{f(x_1)-f(x_0)}
                \end{equation*}
                tendremos que $f(x_t) = \alpha$, lo que lleva a una contradicción. En definitiva, tenemos que:
                \begin{equation*}
                    f(x_0+rz)<\alpha \qquad \forall z\in B(0,1)
                \end{equation*}
                de donde $f$ es continua.
        \end{description}
    \end{proof}
\end{prop}

\noindent
La condición que nos va a interesar es buscar bajo qué condiciones cuando nos dan dos subconjuntos de un espacio normado vamos a poder separarlos mediante un hiperplano afín. Para ello, es necesario formalizar la idea de ``separar dos subconjuntos de un espacio''.

\begin{definicion}
    Sea $E$ un espacio vectorial, $A,B\subset E$, diremos que el hiperplano $H=[f=\alpha]$ separa $A$ y $B$ si:
    \begin{equation*}
        f(x) \leq \alpha \leq f(y) \qquad \forall x\in A,\quad  \forall y\in B
    \end{equation*}
    Además, diremos que la separación es estricta (o que $H$ separa estrictamente $A$ y $B$) si $\exists \varepsilon>0$ de forma que:
    \begin{equation*}
        f(x) \leq \alpha - \varepsilon < \alpha+\varepsilon \leq f(y) \qquad \forall x\in A, \quad  \forall y\in B
    \end{equation*}
\end{definicion}

\begin{teo}[Hahn Banach, primera versión geométrica]
    Sea $E$ un espacio normado, $\emptyset \neq A,B\subset E$ con $A\cap B = \emptyset $, ambos convexos y $A$ abierto, entonces existe un hiperplano cerrado\footnote{Luego habrá una aplicación lineal y contina $f:E\to \mathbb{R}$, por lo que $E^\ast \neq \{0\}$.} $H = [f=\alpha]$ que separa $A$ y $B$.
    \begin{proof}
        El Teorema se demuestra en dos pasos:
        \begin{description}
            \item [Paso 1.] Supongamos en una versión más débil que $B$ se reduce a un punto, es decir, existe $x_0\in E$ de forma que $B = \{x_0\}$ y que $A\subset E$ es un conjunto abierto y convexo de forma que $x_0 \notin A$.

                Elegimos $z_0\in A$ y definimos $C = A-z_0$, que:
                \begin{itemize}
                    \item Contiene al $0$, ya que como $z_0 \in A$, entonces $0=z_0-z_0\in C$.
                    \item Es abierto, ya que si consideramos la traslación según el vector $z_0$:
                        \Func{t_{z_0}}{E}{E}{x}{x+z_0}
                        tenemos que $t_{z_0}$ es una aplicación continua, con inversa $t^{-1}_{z_0}=t_{-z_0}$. Como $C = t_{-z_0}(A) = t^{-1}_{z_0}(A)$ y tenemos que $A$ era abierto y $t_{z_0}$ una aplicación continua, concluimos que $C$ es abierto.
                    \item Es convexo, ya que si $x,y\in C=A-z_0$ tenemos entonces que existen $u,v\in A$ de forma que:
                        \begin{equation*}
                            x = u-z_0, \qquad y = v-z_0
                        \end{equation*}
                        Si tomamos $t\in [0,1]$, entonces:
                        \begin{align*}
                            tx + (1-t)y &= t(u-z_0) + (1-t)(v-z_0) = \underbrace{tu + (1-t)v}_{\in A} - z_0 \in C
                        \end{align*}
                \end{itemize}
                El punto $y_0 = x_ 0 - z_0\notin C$, de donde $y_0 \neq 0$. Por lo que $\mathbb{R}y_0$ es un subespacio vectorial de $E$ de dimensión 1. Definimos $G = \mathbb{R} y_0$ y tomamos 
                \Func{g}{G}{\bb{R}}{ty_0}{t}
            que es una aplicación lineal (compruébese) y verificando $g(y_0) = 1$. La función $g$ nos permitirá ``separar el corte de $C$ con $G$ y el punto $y_0$''. En este punto conviene estudiar el funcional de Minkowski del conjunto $C$, que se define en la Definición~\ref{def:f_minkowski} y cuyas propiedades se aclaran en la Proposición~\ref{prop:f_minkowski}. Sea $p$ el funcional de Minkowski de $C$, veamos que $p$ domina a $g$:
                \begin{itemize}
                    \item Si $t\geq 0$, como $y_0\notin C$ entonces\footnote{Usando la propiedad 3 del funcional.} $p(y_0) \geq 1$, de donde:
                        \begin{equation*}
                            g(ty_0) = t \leq tp(y_0) \stackrel{\text{(1)}}{=} p(ty_0)
                        \end{equation*}
                    \item Si $t<0$, tenemos que:
                        \begin{equation*}
                            g(ty_0) = t < 0 \leq p(ty_0)
                        \end{equation*}
                \end{itemize}
                En cualquier caso, $g(ty_0) \leq p(ty_0)$ $\forall t\in \mathbb{R}$. Nos encontramos en las hipótesis del Teorema de Hahn-Banach, por lo que podemos encontrar $f:E\to \mathbb{R}$ lineal de forma que:
                \begin{equation*}
                    f\big|_G = g \qquad \text{y}\qquad f(y)\leq p(y)\quad \forall y\in E
                \end{equation*}
                La propiedad 2 del funcional nos dice que $\exists M>0$ de forma que:
                \begin{equation*}
                    f(y) \leq p(y) \leq M\|y\| \qquad \forall y\in E
                \end{equation*}
                Si aplicamos esta propiedad para $-y$:
                \begin{equation*}
                    -f(y) = f(-y) \leq M\| -y\| = M\|y\|
                \end{equation*}
                De donde:
                \begin{equation*}
                    |f(y)| \leq M\|y\| \qquad \forall y\in E
                \end{equation*}
                Como $f$ es lineal, la Proposición~\ref{prop:lineal_continuidad_acotacion} nos dice que $f$ es continua.\\

                \noindent
                Si ahora usamos la propiedad 3 del funcional de Minkoski, observamos que:
                \begin{equation*}
                    f(x) \leq p(x) < 1 = f(y_0) \qquad \forall x\in C
                \end{equation*}
                por lo que el hiperplano cerrado $H'=[f=1]$ separa $C$ y $B'=\{y_0\}$.\\

                \noindent
                Si volvemos al problema de separar $A$ y $B=\{x_0\}$, observamos que:
                \begin{align*}
                    f(x) = f(y+z_0) = f(y) + f(z_0) \leq 1 + f(z_0) = f(y_0 + z_0) = f(x_0) \qquad \forall x\in A
                \end{align*}
                Por lo que el hiperplano cerrado $H=[f=f(x_0)]$ separa $A$ y $B=\{x_0\}$, como queríamos probar en este primer paso.
            \item [Paso 2.] Volviendo al caso que nos plantea el Teorema siendo $B$ un conjunto convexo y disjunto de $A$, tomamos:
                \begin{equation*}
                    A - B = \{a-b : a\in A, b\in B\}
                \end{equation*}
                Observemos que:
                \begin{itemize}
                    \item $0\notin A-B$, ya que $A\cap B = \emptyset $.
                    \item $A-B$ es abierto, ya que podemos escribir:
                        \begin{equation*}
                            A-B = \bigcup_{b\in B}(A-b)
                        \end{equation*}
                        y en la demostración del paso anterior ya probamos que la traslación de unconjunto abierto sigue siendo abierto.
                    \item $A-B$ es convexo, ya que si $\alpha,\beta\in A-B$, existen $a,a'\in A$, $b,b'\in B$ de forma que:
                        \begin{equation*}
                            \alpha = a-b, \qquad \beta = a'-b'
                        \end{equation*}
                        Por lo que:
                        \begin{equation*}
                            t\alpha + (1-t)\beta = t(a-b) + (1-t)(a'-b') = \underbrace{ta + (1-t)b}_{\in A} - [\underbrace{tb + (1-t)b'}_{\in B}] \in C
                        \end{equation*}
                        donde hemos usado que tanto $A$ como $B$ son convexos.
                \end{itemize}
                Estamos en las condiciones del paso anterior, por lo que existen $f:E\to \mathbb{R}$ lineal y continua y $\alpha\in \mathbb{R}$ de forma que el hiperplano cerrado $H=[f=\alpha]$ separa $A-B$ del conjunto $\{0\}$, es decir:
                \begin{equation*}
                    f(a) - f(b) = f(a-b) \leq \alpha \leq f(0) = 0 \qquad \forall a\in A, \quad \forall b\in B
                \end{equation*}
                de donde:
                \begin{equation*}
                    f(a) \leq \alpha - f(b) \leq f(b) \qquad \forall a\in A, \quad \forall b\in B
                \end{equation*}
                Por lo que el hiperplano cerrado $H' = [f=\alpha-f(b)]$ separa los conjuntos $A$ y $B$.
        \end{description}
    \end{proof}
\end{teo}

\subsubsection{Funcional de Minkowski de un conjunto}
\noindent
En este subapartado definiremos el funcional de Minkowski de un conjunto, una cierta aplicación con propiedades interesantes que nos permite realizar la demostración de la primera versión geométrica del Teorema de Hahn Banach y que además tiene cierto interés fuera de esta demostración, como luego se pondrá de manifiesto en los ejercicios a realizar.

\begin{definicion}[Funcional de Minkowski]\label{def:f_minkowski}
    Sea $E$ un espacio normado y $C\subset E$ un conjunto convexo, abierto y con $0\in C$, definimos el funcional de Minkowski de $C$ como la aplicación $p:E\to \mathbb{R}$ dada por:
    \begin{equation*}
        p(x) = \inf \left\{\alpha\in \mathbb{R}^+ : \dfrac{x}{\alpha}\in C\right\} \qquad \forall x\in E
    \end{equation*}
\end{definicion}

\begin{observacion}
    Bajo las hipótesis de la definición del funcional de Minkowski, observamos que lo que estamos haciendo es, fijado un punto $x\in E\setminus \{0\}$, tomar la recta de origen $0$ que pasa por $x$, y si multiplicamos $x$ por un escalar positivo, nos movemos por dicha recta. En particular, si multiplicamos $x$ por el inverso de un escalar positivo, si aumentamos dicho escalar, nos estaremos acercando a $0$, y si decrementamos dicho escalar, nos alejaremos de $0$. Notemos que lo que estamos haciendo por la definición del funcional de Minkowski es tomar aquel valor más ``pequeño'' para el cual si multiplicamos $x$ por el inverso de un escalar que se queda por encima suya no nos saldremos del conjunto $C$.
    \begin{figure}[H]
        \centering
        \begin{tikzpicture}[scale=1]
          % Ejes
          \draw[-stealth] (-2.5,0) -- (3.5,0) node[right] {$x$};
          \draw[-stealth] (0,-1.5) -- (0,2.5) node[above] {$y$};

          % Conjunto convexo C: elipse centrada en (0,0), borde discontinuo
          \draw[dashed, thick] (0,0) ellipse (1.8cm and 1.0cm);

          % Marcar el origen
          \fill (0,0) circle (1.5pt) node[below left] {$0$};

          % Punto x en el primer cuadrante (fuera de C)
          \coordinate (x) at (2.2,1.5);
          \fill (x) circle (1.5pt) node[below] {$x$};

          % Semirrecta (rayo) desde el origen que pasa por x, trazo discontinuo
          % se extiende más allá de x usando la notación de cálculo
          \draw[dashed, thick, -stealth] (0,0) -- ($(0,0)!1.5!(x)$);

          % Etiqueta para C
          \node[below right] at (0.4,-0.2) {$C$};
        \end{tikzpicture}
    \end{figure}
\end{observacion}

Observemos que $p(0) = 0$. Además, el funcional de Minkowski tiene ciertas propiedades resaltables.

\begin{prop}\label{prop:f_minkowski}
    Sea $E$ un espacio normado y $C\subset E$ un conjunto convexo, abierto y con $0\in C$, el funcional de Minkowski verifica:
    \begin{enumerate}
        \item $p(\lm x) = \lm p(x)\qquad \forall x\in E, \quad \forall \lm\in \mathbb{R}^+$
        \item $\exists M>0$ tal que $0\leq p(x) \leq M\|x\|\qquad \forall x\in E$
        \item $C = \{x\in E : p(x) < 1\}$
        \item $p(x+y) \leq p(x) + p(y)\qquad \forall x,y\in E$
    \end{enumerate}
    \begin{proof}
        Demostramos cada una de las propiedades:
        \begin{enumerate}
            \item Para la primera, basta usar que $\lm > 0$ y observar:
                \begin{align*}
                    p(\lm x) &= \inf \left\{\alpha\in \mathbb{R}^+ : \dfrac{x}{\nicefrac{\alpha}{\lm}} = \dfrac{\lm x}{\alpha} \in C\right\} = \inf\left\{\lm \alpha\in \mathbb{R}^+ : \dfrac{x}{\alpha}\in C\right\} \\
                             &= \lm\inf\left\{\alpha\in \mathbb{R}^+ : \dfrac{x}{\alpha}\in C\right\} = \lm p(x) \qquad \forall x\in E
                \end{align*}
            \item Dado $x\in E$, como $0\in C$ es abierto $\exists r>0$ de forma que $B(0,r)\subset C$. Si tomamos:
                \begin{equation*}
                    \alpha > \dfrac{\|x\|}{r} \Longrightarrow \left\|\dfrac{x}{\alpha}\right\| < r \Longrightarrow \dfrac{x}{\alpha} \in B(0,r)\subset C
                \end{equation*}
                Por tanto:
                \begin{equation*}
                    \left]\dfrac{\|x\|}{r},+\infty\right[ \subset \left\{\alpha\in \mathbb{R}^+:\frac{x}{\alpha}\in C \right\}
                \end{equation*}
                de donde el ínfimo de la derecha será menor o igual que el ínfimo de la izquierda:
                \begin{equation*}
                    p(x) \leq \dfrac{\|x\|}{r}
                \end{equation*}
                Tomamos $M = \frac{1}{r}$.
            \item Queremos ver que $C= \{x\in E : p(x) < 1\}$:
                \begin{description}
                    \item [$\supset )$] Sea $x\in E$ con $p(x)<1$, el ínfimo nos garantiza la existencia de $\alpha_0\in \mathbb{R}^+$ de forma que $\alpha_0 < 1$ y $\frac{x}{\alpha_0}\in C$. Como $C$ es convexo y $0\in C$, tenemos entonces que:
                        \begin{equation*}
                            x = \alpha_0\dfrac{x}{\alpha_0}+(1-\alpha_0)\cdot 0 \in C
                        \end{equation*}
                    \item [$\subset )$] Sea $x\in C$, por ser $C$ abierto $\exists r>0$ de forma que $B(x,r)\subset C$. Ahora, si tomamos $\varepsilon>0$ de forma que:
                        \begin{equation*}
                            \dfrac{\|x\|}{r}< \dfrac{1}{\varepsilon}
                        \end{equation*}
                        tendremos entonces que:
                        \begin{equation*}
                            \|(1+\varepsilon)x-x\| = \|\varepsilon x\| = \varepsilon \|x\| < r \Longrightarrow (1+\varepsilon)x\in B(x,r)
                        \end{equation*}
                        En dicho caso, tendremos que:
                        \begin{equation*}
                            p(x) \leq \dfrac{1}{1+\varepsilon}<1
                        \end{equation*}
                \end{description}
            \item Dados $x,y\in E$, sabemos que el conjunto:
                \begin{equation*}
                    \left\{\alpha> 0 : \dfrac{x}{\alpha}\in C\right\}
                \end{equation*}
                es un intervalo no acotado superiormente y acotado inferiormente por $p(x)$, pero no sabemos si el intervalo contiene a $p(x)$ (en cuyo caso se trataría de un mínimo) o si no. Sin embargo, lo que sí sabemos es que:
                \begin{equation*}
                    \dfrac{x}{p(x)+\varepsilon},\dfrac{y}{p(y)+\varepsilon} \in C \qquad \forall \varepsilon>0
                \end{equation*}
                Si usamos el apartado 3, tenemos que:
                \begin{equation*}
                    p\left(\dfrac{x}{p(x)+\varepsilon}\right) < 1
                \end{equation*}
                Como $C$ es convexo, si tomamos:
                \begin{equation*}
                    t = \dfrac{p(x)+\varepsilon}{p(x)+p(y)+2\varepsilon} \leq 1
                \end{equation*}
                tenemos entonces que:
                \begin{equation*}
                    \dfrac{x+y}{p(x) + p(y) + 2\varepsilon} = t\dfrac{x}{p(x)+\varepsilon} + (1-t)\dfrac{y}{p(y)+\varepsilon}\in C\qquad \forall x,y\in E
                \end{equation*}
                Usando de nuevo la propiedad 3:
                \begin{equation*}
                    p(x+y) \leq p(x) + p(y) + 2\varepsilon \qquad \forall x,y\in E, \quad \forall \varepsilon>0
                \end{equation*}
                De donde deducimos la propiedad buscada.\qedhere
        \end{enumerate}
    \end{proof}
\end{prop}

\begin{observacion}
    Notemos que si $C = B(0,1)$, tenemos entonces que:
    \begin{equation*}
        p(x) = \inf\left\{\alpha> 0 : \dfrac{x}{\alpha}\in C\right\} = \|x\| \qquad \forall x\in E
    \end{equation*}
\end{observacion}

\begin{ejercicio} % // TODO: HACER
    Sea $E$ un espacio vectorial y $C$ un conjunto abierto, convexo y que contiene al $0$, parece qer que $p$ tiene propiedades deseables para ser una norma en $E$ de forma que:
    \begin{equation*}
        B_p(0,1) = C
    \end{equation*}
    es decir, el funcional de Minkoski de alguna forma resuelve el problema de dado un conjunto que quiero que sea la bola unidad, ¿qué norma considero?.

    Se pide razonar las propiedades que ha de cumplir un conjunto $C\subset E$ abierto, convexo y que contiene al $0$ para garantizar que el funcional de Minkowski de $C$ sea una norma.
\end{ejercicio}

\begin{teo}[Hahn Banach, segunda versión geométrica]\label{teo:hahn-banach_2aversiongeometrica}
    Sea $E$ un espacio normado, $\emptyset \neq A, B\subset E$ con $A\cap B=\emptyset $ ambos convexos, $A$ cerrado y $B$ compacto, entonces existe un hiperplano que separa estrictamente $A$ y $B$. Es decir, existen $f:E\to \mathbb{R}$ lineal y continua, $\alpha\in \mathbb{R}$, $\varepsilon>0$ de forma que:
    \begin{equation*}
        f(a) \leq \alpha - \varepsilon < \alpha < \alpha + \varepsilon \leq f(b) \qquad \forall a\in A, \quad \forall b\in B
    \end{equation*}
    \begin{proof}
        Sea $C=A-B$, tenemos que:
        \begin{itemize}
            \item $0\notin C$, ya que $A\cap B = \emptyset $.
            \item $C$ es convexo, ya que $A$ y $B$ son convexos (se hizo en la prueba del Teorema anterior).
            \item $C$ es cerrado, ya que $A$ es cerrado y $B$ es compacto: sea $\{x_n\}\to x\in E$ con $x_n \in C \quad \forall n\in \mathbb{N}$, entonces existen $\{a_n\}$ sucesión de puntos de $A$ y $\{b_n\}$ sucesión de puntos de $B$ con:
                \begin{equation*}
                    x_n = a_n - b_n \qquad \forall n\in \mathbb{N}
                \end{equation*}
                Como $B$ es compacto, existe una parcial $\{b_{\sigma(n)}\}$ convergente a $b\in B$. Si vemos que:
                \begin{equation*}
                    x_{\sigma(n)} = a_{\sigma(n)} - b_{\sigma(n)} \Longrightarrow a_{\sigma(n)} = x_{\sigma(n)} + b_{\sigma(n)} \qquad \forall n\in \mathbb{N}
                \end{equation*}
                Tenemos entonces que $\{a_{\sigma(n)}\} \to x + b$ y como $A$ es cerrado, ha de ser $x+b\in A$. En definitiva:
                \begin{equation*}
                    x = x + b - b
                \end{equation*}
                Con $x+b\in A$ y $b\in B$, por lo que $x\in C$, lo que demuestra que $C$ es cerrado.
        \end{itemize}
        Como $C$ es cerrado y $0\notin C$, tenemos entonces que $E\setminus C$ es abierto y $0\in E\setminus C$, de donde $\exists r>0$ con que $B(0,r)\cap C = \emptyset $. Si usamos la primera versión geométrica del Teorema de Hahn-Banach para los conjuntos $B(0,r)$ y $C$, obtenemos un hiperplano cerrado $H=[f=\alpha]$ de forma que:
        \begin{equation*}
            f(x) \leq \alpha \leq f(y) \qquad \forall x\in C, \quad \forall y\in B(0,r)
        \end{equation*}
        es decir:
        \begin{equation*}
            f(a-b) \leq \alpha \leq f(-rz) = -rf(z) \leq -r\|f\| \qquad \forall a\in A, \quad \forall b\in B, \quad \forall z\in B(0,1)
        \end{equation*}
        Si tomamos $\varepsilon=\frac{1}{2}r\|f\|$, tenemos entonces que:
        \begin{equation*}
            f(a) + \varepsilon\leq f(b) - \varepsilon\qquad \forall a\in A, \quad \forall b\in B
        \end{equation*}
        Tomando $\beta = \min\{f(b) : b\in B\}+\varepsilon$, tenemos entonces que:
        \begin{equation*}
            f(a) \leq \beta - \varepsilon < \beta + \varepsilon \leq f(b) \qquad \forall a\in A, \quad \forall b\in B
        \end{equation*}
        Es decir, el hiperplano cerrado $H=[f=\beta]$ separa estrictamente $A$ y $B$.
    \end{proof}
\end{teo}

\noindent
Parece ser a priorio un Teorema más potente que la primera versión geométrica del Teorema de Hahn-Banach, pero en dimensión infinita apenas hay conjuntos compactos.

\begin{coro}
    Sea $E$ un espacio vectorial, $F\subset E$ un subespacio vectorial de forma que $\overline{F}\neq E$, entonces existe $f\in E^\ast$, $f\neq 0$ de forma que:
    \begin{equation*}
        f(x) = 0 \qquad \forall x\in F
    \end{equation*}
    \begin{proof}
        Como $\overline{F}\neq E$, tomamos $x_0\in E\setminus \overline{F}$ y tenemos que $\{x_0\}$ es compacto, así como que $\{x_0\}\cap \overline{F}=\emptyset $, con $\overline{F}$ cerrado. Además, como $F$ es un subespacio vectorial, tenemos que $\overline{F}$ es un subespacio vectorial, luego convexo. Si aplicamos la segunda versión geométrica del Teorema de Hahn-Banach, obtenemos $H=[f=\alpha]$ hiperplano cerrado que separa estrictamente $\overline{F}$ y $\{x_0\}$. Es decir, $\exists \varepsilon>0$ de forma que:
        \begin{equation*}
            f(x) \leq \alpha - \varepsilon < \alpha + \varepsilon \leq f(x_0) \qquad \forall x\in \overline{F}
        \end{equation*}
        Tenemos que $f\in E^\ast$ así como que $f\neq 0$ (ya que tenemos una separación estricta de $f(x_0)$). Finalmente:
        \begin{equation*}
            f(x) < \alpha < f(x_0) \qquad \forall x\in \overline{F}
        \end{equation*}
        Fijaremos $\lm\in \mathbb{R}^\ast$, tenemos que: 
        \begin{equation*}
            \lm f(x) = f(\lm x) < \alpha \qquad \forall x\in \overline{F}
        \end{equation*}
        \begin{itemize}
            \item Si $\lm > 0$, tenemos entonces que:
                \begin{equation*}
                    f(x) < \dfrac{\alpha}{\lm} \qquad \forall \lm > 0
                \end{equation*}
                luego $f(x) \leq 0\quad \forall x\in F$.
            \item Si $\lm < 0$, tenemos que:
                \begin{equation*}
                    f(x) > \dfrac{\alpha}{\lm} \qquad \forall \lm < 0
                \end{equation*}
                de la misma forma, $f(x) \geq 0\quad \forall x\in F$.
        \end{itemize}
    \end{proof}
\end{coro}

\begin{observacion}
    Notemos que el enunciado de este teorema es equivalente a:
    \begin{center}
        Sea $E$ un espacio vectorial, $G\subset E$ un subespacio vectorial cerrado de $E$ con $G\neq E$, entonces existe $f\in E^\ast$, $f\neq 0$ de forma que:
        \begin{equation*}
            f(x) = 0 \qquad \forall x\in F
        \end{equation*}
    \end{center}
    Aunque esta forma de enunciarlo parezca más sencilla, preferimos enunciarlo de la primera forma, ya que lo que nos va a interesar del enunciado es su contrarrecíproco:
    \begin{center}
        Sea $E$ un espacio vectorial y $F\subset E$ un subespacio vectorial, si $\forall f\in E^\ast$, $f\neq 0$, $\exists x\in F$ con $f(x)\neq 0$, entonces $F$ es denso en $E$.
    \end{center}
    Enunciado de otra forma más sencilla:
    \begin{center}
        Si $f\in E^\ast\setminus\{0\}$, si la condición $f\big|_F=0$ implica $f=0$, entonces $F$ es denso en $E$.
    \end{center}
    Acabamos de encontrar una condición suficiente que nos permite probar que ciertos subespacios vectoriales de un espacio vectorial son densos, mediante una idea muy ingeniosa.
\end{observacion}

\section{Espacio bidual} % // TODO: Arreglar esta sección
\noindent
Sea $E$ un espacio normado, habíamos ya definido el espacio $E^\ast$, que probamos que era también un espacio normado, con la norma:
\begin{equation*}
    \|f\| = \sup_{\|x\|\leq 1}|\langle f,x \rangle |
\end{equation*}
esto nos permite considerar ${(E^\ast)}^{\ast}$, al que llamaremos \underline{espacio bidual de $E$}, y notaremos por $E^{\ast\ast}$. Será costumbre denotar a sus elementos por letras griegas, y la norma en este espacio vendrá dada por:
\begin{equation*}
    \|\chi\| = \sup_{\|f\|\leq 1}|\langle \chi,f \rangle |
\end{equation*}

\begin{prop}
    Sea $E$ un espacio normado, $E^{\ast\ast}$ contiene una copia isométrica de $E$.
    \begin{proof}
        Es decir, queremos probar que existe un subconjunto de $E^{\ast\ast}$ que es isométrico con $E$, o en otras palabras, que existe una aplicación $J:E\to E^{\ast\ast}$ que sea lineal, inyectiva y que preserve la norma. Definimos la aplicación
        \Func{J}{E}{E^{\ast\ast}}{x}{\chi_x}

        donde $\chi_x$ es la aplicación dada por:
        \Func{\chi_x}{E^\ast}{\bb{R}}{f}{\langle f,x \rangle}
        Comprobemos primero que $J$ está bien definida, es decir, que $\chi_x\in E^{\ast\ast}$. Para ello:
        \begin{itemize}
            \item $\chi_x$ es lineal, puesto que:
                \begin{equation*}
                    \chi_x(\lm f + g) = \langle \lm f + g,x \rangle  = \lm\langle f,x \rangle  + \langle g,x \rangle  = \lm \chi_x(f) + \chi_x(g)\qquad \forall f,g\in E^\ast, \quad \forall \lm \in \mathbb{R}
                \end{equation*}
            \item $\chi_x$ es continua, ya que:
                \begin{equation*}
                    \|\chi_x\| = \sup_{\|f\|\leq 1}|\chi_x(f)| = \sup_{\|f\|\leq 1}|\langle f,x \rangle | \AstIg \|x\|
                \end{equation*}
                donde en $(\ast)$ hemos usado el Corolario~\ref{coro:calcular_norma_x} del Teorema de Hahn-Banach. Además, hemos probado que $\|\chi_x\| = \|x\|$, por lo que $J$ preserva la norma.
        \end{itemize}
        Nos falta comprobar que $J$ es lineal e inyectiva:
        \begin{itemize}
            \item Sean $\lm\in \mathbb{R}$, $x,y\in E$:
                \begin{align*}
                    J(\lm x + y) = \chi_{(\lm x + y)} \AstIg \lm \chi_x + \chi_y = \lm J(x) + J(y)
                \end{align*}
                donde en $(\ast)$ hemos usado que:
                \begin{equation*}
                    \chi_{(\lm x + y)}(f) = \langle f,\lm x + y \rangle  = \lm \langle f,x \rangle  + \langle f,y \rangle  = \lm \chi_x(f) + \chi_y(f) \qquad \forall f\in E^\ast
                \end{equation*}
            \item Sean $x,y\in E$ de forma que $J(x) = J(y)$, tenemos entonces que:
                \begin{description}
                    \item [Opción 1.] 
                        \begin{equation*}
                            J(x-y) = 0 \Longrightarrow 0 = \|J(x-y)\| \AstIg\|x-y\| \Longrightarrow x-y = 0
                        \end{equation*}
                        donde en $(\ast)$ usamos que $J$ conserva la norma.
                    \item [Opción 2.] Supuesto que $x\neq y$, si tomamos $v=x-y$ y consideramos $G = \mathbb{R} v$ podemos definir el funcional lineal y continuo:
                        \begin{equation*}
                            g(tv) = t\|v\|
                        \end{equation*}
                        El Corolario~\ref{coro:hahn-banach} nos da la existencia de $f\in E^\ast$ de forma que \newline$f(v) = \|v\|$, pero tenemos que:
                        \begin{equation*}
                            f(v) = f(x-y) = f(x) - f(y) = 0
                        \end{equation*}
                        por lo que $v=0$, lo que es una \underline{contradicción}.
                \end{description}
        \end{itemize}
        Hemos probado que $E$ es isométrico con $J(E)\subset E^{\ast\ast}$.
    \end{proof}
\end{prop}

\begin{definicion}
    Sea $E$ un espacio normado, decimos que es \underline{reflexivo} si la aplicación $J$ de la proposición anterior es sobreyectiva.
\end{definicion}

    \chapter{Principio de acotación uniforme y Tª de la gráfica cerrada}
\begin{definicion}
    Sean $E,F$ espacios normados, definimos:
    \begin{equation*}
        L(E,F) = \{f:E\to F : f \text{\ lineal y continua}\}
    \end{equation*}
    y notaremos normalmente $L(E) := L(E,E)$.
\end{definicion}

\begin{prop}
    Al igual que como sucedía con aplicaciones lineales y continuas de un espacio normado $E$ en $\mathbb{R}$, si $E,F$ son espacios normados y $T\in L(E,F)$ tenemos\footnote{Resultados análogos que se realizan con las mismas pruebas.}:
    \begin{enumerate}
        \item $T$ es continua $\Longleftrightarrow T$ es continua en $0 \Longleftrightarrow \sup_{\|x\|\leq 1}\|Tx\| < \infty$
        \item Si definimos:
            \begin{equation*}
                \|T\| := \sup_{\|x\|\leq 1}\|Tx\| \qquad \forall T\in L(E,F)
            \end{equation*}
            Tenemos que $\|\cdot \|$ es una norma en $L(E,F)$.
        \item Se verifica que:
            \begin{equation*}
                \|T\| = \inf\{M>0 : \|Tx\| \leq M\|x\|, \quad \forall x\in E\}
            \end{equation*}
    \end{enumerate}
\end{prop}

\section{Principio de acotación uniforme}
Con vistas a demostrar el Principio de acotación uniforme, demostramos el siguiente Lema:

\begin{lema}\label{lema:acotacion_uniforme}
    Sean $E,F$ espacios normados y $T\in L(E,F)$, entonces:
    \begin{equation*}
        \sup_{\|x-x_0\| \leq r} \|Tx\| \geq r\|T\|\qquad \forall x_0\in E, \quad \forall r>0
    \end{equation*}
    \begin{proof}
        Fijado $r\in \mathbb{R}^+$, sean $x_0,y\in E$ con $\|y\|\leq r$:
        \begin{align*}
            \|Ty\| &= \left\|T\left(\frac{1}{2}\left[x_0 + y - (x_0-y)\right]\right)\right\| = \dfrac{1}{2}\|T(x_0 + y - (x_0-y))\| \\ 
                   &\leq \dfrac{1}{2}\left(\|T(x_0+y)\| + \|T(x_0-y)\|\right) \leq \max\{\|T(x_0+y)\|,\|T(x_0-y\|)\} \\
                   &\leq \sup_{\|y\|\leq r} \max\{\|T(x_0+y)\|, \|T(x_0-y)\|\} \leq \sup_{\|z-x_0\| \leq r} \|Tz\|
        \end{align*}
        Si ahora observamos que:
        \begin{equation*}
            \sup_{\|y\|\leq r}\|Ty\| = r\sup_{\|z\|\leq 1}\|Tz\| = r\|T\|
        \end{equation*}
        Acabamos de probar que $\sup\limits_{\|x-x_0\|\leq r}\|Tx\| \geq r\|T\|$.
    \end{proof}
\end{lema}

\begin{teo}[Principio de acotación uniforme]\label{teo:principio_acotacion_uniforme}
Sea $E$ un espacio de Banach, $F$ un espacio normado y $\cc{F}$ un subconjunto de $L(E,F)$, entonces:
\begin{equation*}
    \sup_{T\in \cc{F}}\|Tx\| < +\infty \quad \forall x\in E \qquad \Longrightarrow \qquad  \sup_{T\in \cc{F}}\|T\| < +\infty
\end{equation*}
    \begin{proof}
        Demostraremos el contrarrecíproco:
        \begin{equation*}
            \sup_{T\in \cc{F}}\|T\| = \infty \qquad  \Longrightarrow \qquad  \sup_{T\in \cc{F}}\|Tx\| = \infty
        \end{equation*}
        Supongamos pues que $\sup\limits_{T\in \cc{F}}\|T\| = \infty$, por lo que existe una sucesión de elementos de $\cc{F}$, llamémosla $\{T_n\}$, de forma que:
        \begin{equation*}
            \|T_n\| \geq 4^n \qquad \forall n\in \mathbb{N}
        \end{equation*}
        Definimos por inducción una sucesión de puntos de $E$:
        \begin{itemize}
            \item $x_0 = 0\in E$.
            \item Tomando $r = \nicefrac{1}{3}$, el Lema~\ref{lema:acotacion_uniforme} nos dice:
                \begin{equation*}
                    \sup_{\|x-x_0\| < \frac{1}{3}}\|T_1x\| \geq \dfrac{1}{3}\|T_1\| > \dfrac{2}{3}\cdot \dfrac{1}{3}\|T_1\|
                \end{equation*}
                Como $\nicefrac{2}{3}\cdot \nicefrac{1}{3}\cdot \|T_1\|$ es estrictamente menor que el supremo de la izquierda, tenemos que no puede ser una cota superior de $\|T_1 x\|$ para $x\in B(x_0,\nicefrac{1}{3})$, con lo que $\exists x_1\in B(x_0,\nicefrac{1}{3})$ de forma que:
                \begin{equation*}
                    \|T_1x_1\| > \dfrac{2}{3}\cdot \dfrac{1}{3}\|T_1\|
                \end{equation*}
            \item Supuesto que hemos construido hasta $x_{n-1}$, veamos cómo construir $x_n$:

                Tomando $r = \nicefrac{1}{3^n}$, el Lema~\ref{lema:acotacion_uniforme} nos dice que:
                \begin{equation*}
                    \sup_{\|x-x_{n-1}\| < \frac{1}{3^n}}\|T_nx\| \geq \dfrac{1}{3^n}\|T_n\| > \dfrac{2}{3}\cdot \dfrac{1}{3^n} \|T_n\|
                \end{equation*}
                Y por el mismo razonamiento de antes podemos encontrar $x_n\in B(x_{n-1},\nicefrac{1}{3^n})$ de forma que:
                \begin{equation*}
                    \|T_nx_n\| > \dfrac{2}{3}\cdot \dfrac{1}{3^n}\|T_n\|
                \end{equation*}
        \end{itemize}
        Veamos ahora que $\{x_n\}$ es de Cauchy. Para ello, buscamos acotar $\|x_m - x_n\|$ para $n,m$ índices bastante avanzados. Supondremos sin pérdida de generalidad que $n,m\in \mathbb{N}$ con $m>n$, donde tendremos:
        \begin{align*}
            \|x_m-x_n\| &= \|x_m - x_{m-1} + x_{m-1} - x_{m-2} + \ldots + x_{n+1} - x_n\| \\
                        &\leq \|x_m - x_{m-1} \| + \| x_{m-1} - x_{m-2} \| + \ldots + \|x_{n+1} - x_n\| \\
                        &\leq \dfrac{1}{3^m} + \dfrac{1}{3^{m-1}} + \ldots + \dfrac{1}{3^{n+1}} = \dfrac{1}{3^n}\left[\dfrac{1}{3^{m-n}} + \ldots + \dfrac{1}{3}\right] \\
                        &\leq \dfrac{1}{3^n} \sum_{j=1}^{+\infty} \dfrac{1}{3^j} = \dfrac{1}{3^n} \dfrac{\frac{1}{3}}{1-\frac{1}{3}} = \dfrac{1}{3^n}\cdot \dfrac{1}{2}
        \end{align*}
        En definitiva, tenemos que:
        \begin{equation*}
            \|x_m-x_n\| = \|x_m - x_{m-1} + x_{m-1} - x_{m-2} + \ldots + x_{n+1} - x_n\| \leq \dfrac{1}{2}\cdot \dfrac{1}{3^n}
        \end{equation*}
        Por lo que $\{x_n\}$ es de Cauchy en $E$, que era de Banach, por lo que $\{x_n\}$ converge a cierto $x\in E$. Observemos que:
        \begin{equation*}
            \dfrac{1}{2}\cdot \dfrac{1}{3^n} \geq \lim_{m\to\infty}\|x_m-x_n\| \AstIg \left\|\lim_{m\to\infty}(x_m-x_n)\right\| = \|x-x_n\|
        \end{equation*}
        donde en $(\ast)$ hemos usado la continuidad de $\|\cdot \|$. Calculamos ahora:
        \begin{align*}
            \|T_nx\| &= \|T_n(x-x_n + x_n)\| \geq \|T_nx_n\| - \|T_n(x-x_n)\| \geq \dfrac{2}{3}\cdot \dfrac{1}{3^n}\|T_n\| - \|T_n\|\|x-x_n\| \\
                     &\geq \dfrac{2}{3}\cdot \dfrac{1}{3^n}\|T_n\| - \|T_n\|\dfrac{1}{2}\cdot \dfrac{1}{3^n} = \left(\dfrac{2}{3}-\dfrac{1}{2}\right) \dfrac{1}{3^n} \|T_n\| = \dfrac{1}{6}\cdot \dfrac{1}{3^n}\|T_n\| \geq \dfrac{1}{6}{\left(\dfrac{4}{3}\right)}^{n} \to \infty
        \end{align*}
        Por tanto:
        \begin{equation*}
            \sup_{T\in \cc{F}}\|Tx\| \geq \sup_{n\in \mathbb{N}}\|T_nx\| = \infty
        \end{equation*}
        Como queríamos probar.
    \end{proof}
\end{teo}

\noindent
Introducimos ahora una serie de Corolarios que nos da el Principio de acotación uniforme:

\begin{coro}
    Sean $E,F$ dos espacios de Banach, sea $\{T_n\}$ una sucesión de elementos de $L(E,F)$ de forma que $\{T_n(x)\} \to T(x)$ para todo $x\in E$. Entonces:
    \begin{enumerate}[label=(\alph*)]
        \item $\sup\limits_{n\in \mathbb{N}}\|T_n\| < \infty$.
        \item $T\in L(E,F)$.
        \item $\|T\| \leq \liminf\limits_{n\to\infty}\|T_n\|$.
    \end{enumerate}
    \begin{proof}
        Demostramos cada apartado:
        \begin{enumerate}[label=(\alph*)]
            \item Dado $x\in E$, como $\{T_n(x)\}\to T(x)$, tenemos que $\exists m\in \mathbb{N}$ de forma que:
                \begin{equation*}
                    T_n(x) \in B(T(x), 1) \qquad \forall n\geq m
                \end{equation*}
                Por lo que $\{T_n(x) : n\in \mathbb{N}\}$ es un conjunto acotado, puesto que podemos verlo como la unión de un conjunto acotado y uno finito:
                \begin{equation*}
                    \{T_n(x) : n\in \mathbb{N}\} = \{T_n(x) : n\geq m\} \cup \{T_n(x) : n < m\} 
                \end{equation*}
                En definitiva, tenemos que $\sup_{n\in \mathbb{N}}\|T_n(x)\| < \infty$ para todo $x\in E$, de donde aplicando el Principio de acotación uniforme tenemos que $\sup_{n\in \mathbb{N}}\|T_n\| < \infty$.
            \item Veamos que $T:E\to F$ es lineal y continua:
                \begin{itemize}
                    \item Es fácil ver que $T$ es lineal:
                        \begin{align*}
                            T(\lm x + y) &= \lim_{n\to\infty}T_n(\lm x + y) = \lim_{n\to\infty}(\lm T_n(x) + T_n(y)) \\ 
                                         &= \lm \lim_{n\to\infty}T_n(x) + \lim_{n\to\infty}T_n(y) = \lm T(x) + T(y) \qquad \forall x,y\in E
                        \end{align*}
                    \item Para ver que $T$ es continua podemos usar el apartado $(a)$:
                        \begin{equation*}
                            \|T_n(x)\| \leq \|T_n\|\|x\| \leq \sup_{n\in \mathbb{N}}\|T_n\|\|x\| \qquad \forall x\in E
                        \end{equation*}
                        Y como $\{\|T_n(x)\|\}\to \|T(x)\|$, tenemos que:
                        \begin{equation*}
                            \|T(x)\| \leq \sup_{n\in \mathbb{N}}\|T_n\|\|x\| \qquad \forall x\in E
                        \end{equation*}
                        lo que nos dice que $T$ es continua.
                \end{itemize}
            \item Para ver que $\|T\| \leq \liminf\limits_{n\to \infty} \|T_n\|$, notemos que:
                \begin{align*}
                    \|T(x)\| &= \left\|\lim_{n\to\infty}T_n(x)\right\| = \lim_{n\to\infty}\left\|T_n(x)\right\| = \liminf \|T_n(x)\| \leq \liminf(\|T_n\|\|x\|) \\
                             &= \sup_{k\in \mathbb{N}}\inf_{n\geq k}(\|T_n\|\|x\|) = \sup_{k\in \mathbb{N}}\inf_{n\geq k}\|T_n\| \cdot \|x\| = \liminf \|T_n\| \cdot \|x\| \qquad \forall x\in E
                \end{align*}
                De donde $\|T\|\leq \liminf \|T_n\|$.
        \end{enumerate}
    \end{proof}
\end{coro}

\begin{coro}
    Sea $G$ un espacio de Banach y $B\subset G$, si para toda $f\in G^\ast$ el conjunto $f(B)$ está acotado (en $\mathbb{R}$), entonces $B$ está acotado.
    \begin{proof}
        Comenzamos la demostración pensando a lo que queremos llegar, pues así nos será más fácil comenzar la demostración. Queremos probar que $B$ está acotado, es decir, que existe $M>0$ de forma que:
        \begin{equation*}
            \|b\| \leq M \qquad \forall b\in B
        \end{equation*}
        Si recordamos que el Corolario~\ref{coro:calcular_norma_x} nos dice que:
        \begin{equation*}
            \|b\| = \sup_{\|f\|\leq 1}|f(b)|
        \end{equation*}
        observemos que lo queremos es buscar una cota superior de $|f(b)|$, donde $b$ está fija y $f$ se mueve. Para ello, podemos pensar en definir ciertos funcionales $T_b(f)$ de forma que tras aplicar el Principio de Acotación Uniforme obtengamos para $\|f\|\leq 1$:
        \begin{equation*}
            |f(b)| = |T_b(f)| \leq \|T_b\|\|f\| \leq \|T_b\| \leq \sup_{b\in B}\|T_b\| < \infty
        \end{equation*}
        Con lo que nuestra constante $M$ la tomaremos como $\sup\limits_{b\in B}\|T_b\|$. Comenzando ahora con la demostración, fijado $b\in B$, definimos la aplicación $T_b:G^\ast\to \mathbb{R}$ dada por:
        \begin{equation*}
            T_b(f) = f(b) \qquad \forall f\in G^\ast
        \end{equation*}
        Con lo que $T_b\in L(G^\ast,\mathbb{R})$:
        \begin{itemize}
            \item Es claro que $T_b$ es lineal.
            \item $T_b$ es continua, ya que $|T_b(f)| = |f(b)| \leq \|b\|\|f\| \quad \forall f\in G^\ast$.
        \end{itemize}
        Como $f(B)$ es acotado para toda $f\in G^\ast$, tenemos entonces que:
        \begin{equation*}
            \sup_{b\in B}|T_b(f)| = \sup_{b\in B}|f(b)| < \infty \qquad \forall f\in G^\ast
        \end{equation*}
        Con lo que aplicando el Principio de acotación uniforme tenemos: 
        \begin{equation*}
            \sup\limits_{b\in B}\|T_b\| < \infty
        \end{equation*}
        Ahora, si tomamos $f\in G^\ast$ con $\|f\|\leq 1$, buscamos usar que $\|x\| = \sup\limits_{\|f\|\leq 1}|f(x)|$:
        \begin{equation*}
            |f(b)| = |T_b(f)| \leq \|T_b\|\|f\| \leq \sup_{b\in B}\|T_b\|\|f\| \leq \sup_{b\in B}\|T_b\|
        \end{equation*}
        Por lo que $\|b\| \leq \sup\limits_{b\in B}\|T_b\| < \infty \quad \forall b\in B$, lo que nos dice que $B$ está acotado.
    \end{proof}
\end{coro}

\noindent
Este último corolario nos dice que si $B\subset G^\ast$ es un conjunto cualquiera, una forma de estudiar si $B$ es un conjunto acotado, una posibilidad es tratar de calcular su imagen bajo cualquier función $f\in G^\ast$, que es un subconjunto de $\mathbb{R}$.

Recordemos que en $\mathbb{R}^n$ un conjunto era acotado si y solo si cada una de sus proyecciones es un conjunto acotado. Este Corolario hace ese papel en espacios de dimensión infinita, que junto con el siguiente son muy utilizados.
% // TODO: Estudiar la relacion con dicho resultado, cuando G sea un Banach de dimensión finita.

\begin{coro} % // TODO: HACER
    Sea $G$ un espacio de Banach y sea $B^\ast\subset G^\ast$, si el conjunto:
    \begin{equation*}
        B^\ast(x) = \{f(x) : f\in B^\ast\}
    \end{equation*}
    está acotado para todo $x\in G$, entonces $B^\ast$ está acotado.
    \begin{proof}
        Al igual que antes empezamos por el final, pues así nos será más fácil comenzar la demostración. Queremos concluir que $B^\ast$ está acotado, es decir, que:
        \begin{equation*}
            \|f\| \leq M\qquad \forall f\in B^\ast
        \end{equation*}
        para cierta constante $M>0$. Para ello, si recordamos la definición de $\|f\|$, vemos que:
        \begin{equation*}
            \|f\| = \sup_{\|x\|\leq 1}|f(x)|
        \end{equation*}
        donde $f$ está fija y movemos la $x$, con lo que trataremos de definir funcionales $T_f(x)$ de forma que para $\|x\|\leq 1$:
        \begin{equation*}
            |f(x)| = |T_f(x)| \leq \|T_f\|\|x\| \leq \|T_f\| \leq \sup_{f\in B^\ast}\|T_f\|
        \end{equation*}
        Comenzando ahora con la demostración, para cada $f\in B^\ast$ definimos la aplicación $T_f:G\to \mathbb{R}$ dada por:
        \begin{equation*}
            T_f(x) = f(x) \qquad \forall x\in G
        \end{equation*}
        con lo que $T_f\in G^\ast$ para todo $f\in B^\ast$:
        \begin{itemize}
            \item Es fácil ver que $T_f$ es lineal para cualquier $f\in B^\ast$.
            \item No es difícil ver que $T_f$ es continua para $f\in B^\ast$, ya que:
                \begin{equation*}
                    |T_f(x)| = |f(x)| \leq \|f\|\|x\| \qquad \forall x\in G
                \end{equation*}
        \end{itemize}
        Como el conjunto $B^\ast(x)$ está acotado para todo $x\in G$, tenemos que:
        \begin{equation*}
            \sup_{f\in B^\ast}\|T_f(x)\| = \sup_{f\in B^\ast}|f(x)| < \infty
        \end{equation*}
        nos encontramos en las hipótesis del Principio de acotación uniforme, que nos dice que entonces:
        \begin{equation*}
            \sup_{f\in B^\ast}\|T_f\| < \infty
        \end{equation*}

        en cuyo caso, si $\|x\|\leq 1$, entonces:
        \begin{equation*}
            |f(x)| = |T_f(x)| \leq \|T_f\|\|x\|\leq \|T_f\| \leq \sup_{f\in B^\ast}\|T_f\| \qquad \forall f\in B^\ast
        \end{equation*}

        con lo que:
        \begin{equation*}
            \|f\| = \sup_{\|x\|\leq 1}|f(x)| \leq \sup_{f\in B^\ast}\|T_f\| \qquad \forall f\in B^\ast
        \end{equation*}

        de donde deducimos que $B^\ast$ está acotado.
    \end{proof}
\end{coro}

\section{Otra demostración del Principio de acotación uniforme}
\noindent
Repetiremos ahora la demostración del Principio de acotación uniforme de otra forma, usando el Lema de Baire, un resultado clásico que nos da de forma no muy complicada la demostración del Principio.

\begin{lema}[de Baire]\label{lema:Baire}
    Sea $X$ un espacio métrico completo, sea $\{X_n\}$ una sucesión de subconjuntos de $X$ todos ellos cerrados y con interior vacío, entonces:
    \begin{equation*}
        \Int\left(\bigcup_{n\in \mathbb{N}} X_n\right) = \emptyset 
    \end{equation*}
    \begin{proof}
        Tomaremos $O_n = X\setminus X_n$ para cada $n\in \mathbb{N}$, con lo que $O_n$ es abierto y denso para cada $n\in \mathbb{N}$, ya que:
        \begin{equation*}
            \overline{O_n} = \overline{X\setminus X_n} = X\setminus \Int X_n = X\setminus \emptyset  = X \qquad \forall n\in \mathbb{N}
        \end{equation*}
        Y la prueba terminará probando que $G = \bigcap\limits_{n\in \mathbb{N}}O_n$ es denso, ya que en dicho caso tendremos:
        \begin{equation*}
            X = \overline{G} = \overline{\bigcap\limits_{n\in \mathbb{N}}O_n} = \overline{\bigcap_{n\in \mathbb{N}} X\setminus X_n} = \overline{X\setminus\bigcup_{n\in \mathbb{N}}X_n} = X\setminus \Int\left(\bigcup_{n\in \mathbb{N}}X_n\right)
        \end{equation*}
        de donde podremos deducir que $\Int\left(\bigcup\limits_{n\in \mathbb{N}}\right)X_n = \emptyset $. Para probar que $G$ es denso, sea $\omega$ un abierto no vacío de $X$, tenemos que probar que $\omega\cap G \neq \emptyset $. Como $\omega$ es abierto, dado $x_0\in \omega$ podemos encontrar $r_0>0$ de forma que:
        \begin{equation*}
            \overline{B(x_0,r_0)}\subset \omega
        \end{equation*}
        Tras esto, como $O_1$ es abierto y denso, podremos elegir $x_1 \in B(x_0,r_0)\cap O_1$ y $r_1>0$ de forma que:
        \begin{equation*}
            \overline{B(x_1,r_1)}\subset B(x_0,r_0) \cap O_1 \quad \text{y}\quad 0<r_1<\frac{r_0}{2}
        \end{equation*}
        De forma inductiva, como cada $O_n$ es abierto y denso, seremos capaces de encontrar dos sucesiones: $\{x_n\}$ de puntos de $X$ y $\{r_n\}$ de reales positivos de forma que se cumpla:
        \begin{equation*}
            \overline{B(x_{n+1},r_{n+1})}\subset B(x_n,r_n) \cap O_{n+1} \quad \text{y}\quad 0<r_{n+1}<\frac{r_n}{2} \qquad \forall n\in \mathbb{N}\cup \{0\}
        \end{equation*}
        Veamos que $\{x_n\}$ es de Cauchy. Para ello, sean $n,m\in \mathbb{N}$ con $m\geq n$, tendremos que:
        \begin{gather*}
            x_m \in B(x_{m-1},r_{m-1}) \subset \ldots \subset B(x_n,r_n) \\
            r_n < \dfrac{r_{n-1}}{2} < \dfrac{r_{n-2}}{2^2} < \ldots < \dfrac{r_0}{2^n}
        \end{gather*}
        Por lo que:
        \begin{equation*}
            \|x_m - x_n\| < r_n < \dfrac{r_0}{2^n}
        \end{equation*}
        de donde deducimos que $\{x_n\}$ es de Cauchy en $X$, y como $X$ era completo, existe $l\in X$ de forma que $\{x_n\}\to l$. Finalmente, como $x_{n+p}\in B(x_n,r_n)$ para $n,p\in \mathbb{N}\cup\{0\}$, tomando límite cuando $p\to \infty$ obtenemos que:
        \begin{equation*}
            l \in \overline{B(x_n,r_n)} \qquad \forall n\in \mathbb{N}\cup \{0\}
        \end{equation*}
        En particular, $l\in \omega \cap G$, por lo que $G$ es denso, lo que concluye la demostración.
    \end{proof}
\end{lema}

\noindent
Cabe destacar que una de las formas en las que más se utiliza el Lema de Baire es mediante su contrarrecíproco:
\begin{center}
    Sea $X$ un espacio métrico completo, sea $\{X_n\}$ una sucesión de subconjuntos de $X$ todos ellos cerrados, entonces:
    \begin{equation*}
        \Int\left(\bigcup_{n\in \mathbb{N}}X_n\right) \neq \emptyset \quad  \Longrightarrow \quad  \exists n_0\in \mathbb{N} : \Int(X_{n_0}) \neq \emptyset 
    \end{equation*}
\end{center}

\noindent
Ahora, volveremos a demostrar el Principio de acotación uniforme usando el Lema de Baire.

\begin{teo}[Principio de acotación uniforme]
    Sea $E$ un espacio de Banach, $F$ un espacio normado y $\cc{F}$ un subconjunto de $L(E,F)$, entonces:
    \begin{equation*}
        \sup_{T\in \cc{F}}\|Tx\| < +\infty \quad \forall x\in E \qquad \Longrightarrow \qquad  \sup_{T\in \cc{F}}\|T\| < +\infty
    \end{equation*}
    \begin{proof}
        Suponiendo que indexamos nuestra familia mediante un conjunto $I$: $\cc{F} = \{T_i\}_{i \in I}$, definimos para cada $n\in \mathbb{N}$:
        \begin{equation*}
            X_n = \{x\in E : \|T_i x\| \leq n, \quad \forall i \in I\}
        \end{equation*}
        que verifica:
        \begin{itemize}
            \item $X_n$ es cerrado para cada $n\in \mathbb{N}$, ya que si tomamos $\{x_m\}$ una sucesión de puntos de $X_n$ convergente a $x\in E$, tenemos entonces que $\|T_ix_m\| \leq n$ para cada $m\in \mathbb{N}$. Usando que $\|\cdot \|$ y que $T_i$ son las dos funciones continuas obtenemos que:
                \begin{equation*}
                    \|T_ix\| \leq n
                \end{equation*}
                con lo que $x\in X_n$.
            \item Usando que $\sup\limits_{T\in \cc{F}}\|Tx\| < \infty$, sabemos entonces que existe $M\in \mathbb{N}$ de forma que $\|Tx\| \leq M$ para todo $x\in E$ y $T\in \cc{F}$, con lo que $X_M = E$, luego:
                \begin{equation*}
                    \bigcup_{n\in \mathbb{N}}X_n = E
                \end{equation*}
        \end{itemize}
        Como $E$ es abierto y es un espacio vectorial, tenemos que $\Int E = E \neq \emptyset $. Por el Lema de Baire tenemos que existe $n_0\in \mathbb{N}$ de forma que $\Int(X_{n_0})\neq \emptyset $, lo que nos permite tomar $x_0\in E$ y $r>0$ de forma que $B(x_0,r)\subset X_{n_0}$, lo que nos dice por la definición de $X_{n_0}$ que:
        \begin{equation*}
            \|T_i(x_0 + rz)\| \leq n_0 \qquad \forall i \in I, \quad \forall z\in B(0,1)
        \end{equation*}

        como:
        \begin{multline*}
            n_0 \geq \|T_i(x_0 + rz)\| \geq \|T_i(rz)\| - \|T_i(x_0)\| \quad \Longrightarrow \quad  r\|T_i(z)\| \leq n_0 + \|T_i(x_0)\| \\
            \qquad \forall i \in I, \quad \forall z\in B(0,1)
        \end{multline*}
        
        tendremos:
        \begin{equation*}
            r\|T_i\| \leq n_0 + \|T_i(x_0)\| \leq n_0 + \sup_{T\in \cc{F}}\|T(x_0)\| < \infty \qquad \forall i \in I, \quad \forall z\in B(0,1)
        \end{equation*}
        
        de donde concluimos que $\sup\limits_{T\in \cc{F}}\|T\| < \infty$
    \end{proof}
\end{teo}

% // TODO: Ejercicio para mañana
% // TODO: Ver donde pongo esto
\begin{ejercicio}
    Sean $X,Y$ dos espacios de Banach, $T\in L(X,Y)$, definimos para cada $n\in \mathbb{N}$ y para cada $y \in Y$:
    \begin{equation*}
        \|y\|_n := \inf\{\|u\| + n\|v\| : u\in X, v\in Y \text{\ con\ } y = T(u)+v\}
    \end{equation*}
    Probar que $\|\cdot \|_n$ es una norma en $Y$ para todo $n\in \mathbb{N}$, que verifica:
    \begin{equation*}
        \|y\|_n \leq n\|y\| \qquad \forall y\in Y
    \end{equation*}
    Además, si $y=T(x)$ con $x\in X$, entonces:
    \begin{equation*}
        \|y\| \leq \|x\|
    \end{equation*}~\\

    \noindent
    Veamos en primer lugar que $\|\cdot \|_n$ es una norma en $Y$ para cada $n\in \mathbb{N}$. Para ello, fijaremos $n\in \mathbb{N}$ y veremos las propiedades de una norma:
    \begin{itemize}
        \item Para la no degeneración, supongamos que $y\in Y$ con $\|y\|_n = 0$. Por definición del ínfimo, existen sucesiones $\{u_m\}$ de puntos de $X$ y $\{v_m\}$ de puntos de $Y$ de forma que:
            \begin{equation*}
                \{\|u_m\| + n\|v_m\|\} \to 0 \qquad y = T(u_m)+ v_m \qquad \forall m\in \mathbb{N}
            \end{equation*}
            como $\|u_m\|, \|v_m\|\geq 0$ para todo $m\in \mathbb{N}$, tenemos entonces que:
            \begin{equation*}
                \{\|u_m\|\},\{\|v_m\|\}\to 0 \quad \Longrightarrow \quad  \{u_m\},\{v_m\} \to 0
            \end{equation*}
            usando ahora que $y = T(u_m)+v_m$ para todo $m\in \mathbb{N}$, observemos que:
            \begin{equation*}
                \{T(u_m)+v_m\}\to 0
            \end{equation*}
            donde hemos usado que $T$ y la suma son continuas, con lo que $y=0$.
        \item Para la homogeneidad por homotecias, sean $\lm\in \mathbb{R}$ y $y\in Y$:
            \begin{align*}
                |\lm|\|y\|_n &= \inf\{|\lm|(\|u\|+n\|v\|) : u\in X, v\in Y \text{\ con\ } y=T(u)+v\}  \\
                             &= \inf\{\|\lm u\| + n\|\lm v\| : u\in X, v\in Y \text{\ con\ } y=T(u)+v\}  \\
                             &= \inf\left\{\|u\| + n\|v\| : u\in X, v\in Y \text{\ con\ } y=T\left(\frac{u}{\lm}\right)+\frac{v}{\lm}\right\}   \\
                             &= \inf\{\|u\| + n\|v\| : u\in X, v\in Y \text{\ con\ } \lm y=T(u)+v\}  = \|\lm y\|_n
            \end{align*}
        \item Finalmente, para la desigualdad triangular, sean $y_1,y_2\in Y$, para todo $\varepsilon>0$ tenemos por la caracterización del ínfimo que existen $u_1,u_2\in X$, $v_1,v_2\in Y$ de forma que:
            \begin{equation*}
                y_i = T(u_i) + v_i \quad \text{y}\quad \|u_i\| + n\|v_i\| \leq \|y_i\|+\frac{\varepsilon}{2}\qquad \forall i \in \{1,2\}
            \end{equation*}
            de donde $y_1+y_2 = T(u_1)+v_1 + T(u_2)+v_2 = T(u_1+u_2)+v_1+v_2$, por lo que:
            \begin{align*}
                \|y_1 + y_2\| \leq \|u_1+u_2\| + n\|v_1+v_2\| &\leq \|u_1\|+n\|v_1\| + \|u_2\|+n\|v_2\| \\ &\leq \|y_1\|+\|y_2\| + \varepsilon \qquad \forall \varepsilon>0
            \end{align*}
            En definitiva, hemos probado que $\|y_1+y_2\| \leq \|y_1\| + \|y_2\|$.
    \end{itemize}
    % // TODO: Terminar
\end{ejercicio}

    \chapter{Topologías Débiles}
\section{Topologías iniciales}
\noindent
Trabajaremos sobre un conjunto $X$ y una familia de espacios topológicos $\{Y_i\}_{i \in I}$ junto con una familia de aplicaciones $\varphi_i:X\to Y_i, \quad \forall i \in I$.\\

\noindent
Observamos que si consideramos en $X$ la topología discreta:
\begin{equation*}
    \tau_d = \{A : A\subset X\}
\end{equation*}
tenemos que $\varphi_i$ es continua, para todo $i \in I$. Sin embargo, hemos sido ``muy brutos'' al considerar esta topología sobre $X$. Nos preguntamos por definir alguna topología en $X$ que haga que todas las funciones de la familia $\{\varphi_i\}_{i \in I}$ sean continuas con el menor número de abiertos.

\begin{observacion}
    Observemos que como pretendemos que las funciones $\varphi_i$ sean continuas, necesitaremos que esta topología $\tau$ buscada verifique que:
    \begin{equation*}
        \tau \supset \{\varphi_i^{-1}(\omega_i) : \omega_i \text{\ es un abierto de\ } Y_i, \quad \forall i \in I\}
    \end{equation*}
\end{observacion}
\noindent
De esta forma, el problema podemos reformularlo como:

\begin{center}
    Dado un conjunto $X$ y una familia $\cc{U} = \{U_\lm\subset X : \lm \in \Lambda\}$, buscar la topología $\tau$ con menor cantidad de abiertos de forma que $\cc{U}\subset \tau$.
\end{center}

\noindent
Para ello, si pretendemos que los $U_\lm$ estén en $\tau$, estos serán abiertos, luego toda intersección finita de ellos lo seguirá siendo, con lo que la intersección finita de los conjuntos $U_\lm$ también tiene que seguir estando en $\tau$. Es decir, si consideramos:
\begin{equation*}
    \cc{V} = \left\{V = \bigcap_{i=1}^n U_{\lm _i} : \lm_1, \ldots, \lm_n \in \Lambda, \quad n\in \mathbb{N}\right\}
\end{equation*}
Tenemos que $\cc{U}\subset \cc{V}$, y nos preguntamos si $\cc{V}$ es una topología. Observamos:
\begin{itemize}
    \item Primero, que $\cc{V}$ es estable por intersecciones finitas.
    \item Sin embargo, la familia no es cerrada por uniones arbitrarias de elementos del conjunto.
\end{itemize}
Para solucionar el segundo problema, consideramos:
\begin{equation*}
    \left\{\bigcup_{\eta \in \Lambda_0} V_\eta : V_\eta \in \cc{V}, \Lambda_0\subset \Lambda\right\}
\end{equation*}
Y tenemos que la topología más pequeña que buscamos debe contener este conjunto. Se demuestra que este conjunto es, de hecho, una topología.

\begin{observacion}
    Observemos que el vacío es resultado de una unión vacía.

    Sin embargo, faltaría unir el total.
\end{observacion}

\noindent
¿Cómo se forma una base de entornos de dicha topología en $X$?

Resulta que una base de entornos es:
\begin{equation*}
    \left\{\bigcap_{i \in J}\varphi_i^{-1}(V_i) : V_i \text{\ entorno de\ } \varphi_i(x)\in Y_i, \quad J\subset I \text{\ finito}\right\}
\end{equation*}

\noindent
Se verifica que el conjunto es una base de entornos.\\

\noindent
Aunque no conozcamos en profundidad la topología (puesto que no hemos dado de forma explícita quiénes son sus abiertos), es sencillo en ocasiones probar ciertas propiedades topológicas, usando para ellos las dos proposiciones siguientes, que nos permiten comprobar propiedades sobre la topología inicial sin tener que usarla, sino tratar de buscar problemas equivalentes realizando composiciones con las aplicaciones $\varphi_i$ de la familia que nos da la topología inicial.

\begin{prop}\label{prop:top_inicial_suc}
    Sea $(X,\tau)$ con $\tau$ la topología inicial asociada a una familia de aplicaciones $\{\varphi_i\}_{i \in I}$, sea $\{x_n\}$ una sucesión de puntos de $X$ y $x\in X$:
    \begin{equation*}
        \{x_n\}\to x \quad \Longleftrightarrow \quad  \{\varphi_i(x_n)\} \to \varphi_i(x) \quad \forall i \in I
    \end{equation*}
    \begin{proof}
        Por doble implicación:
        \begin{description}
            \item [$\Longrightarrow )$] Para cada $i \in I$, $\tau$ hace que $\varphi_i$ sea continua, por lo que:
                \begin{equation*}
                    \{x_n\} \to x \Longrightarrow \{\varphi_i(x_n)\} \to \varphi_i(x)
                \end{equation*}
            \item [$\Longleftarrow )$] Si consideramos un entorno $U$ de $x$, este ha de contener un entorno de la base de entornos, luego existe una familia finita $\{V_i\}_{i \in J}$ de entornos de $\varphi_i(x)$ en cada $Y_i$ con $i \in J$ de forma que:
                \begin{equation*}
                    W = \bigcap_{i \in J}\varphi_i^{-1}(V_i)
                \end{equation*}
                es un entorno básico contenido en $U$. Observemos que para cada $i \in J$ tenemos:
                \begin{equation*}
                    \left.\begin{array}{c}
                        \varphi_i(x) \in V_i \\
                        \{\varphi_i(x_n)\} \to \varphi_i(x)
                    \end{array}\right\} \Longrightarrow \exists N_i \in \mathbb{N} : \varphi_i(x_n)\in V_i \quad \forall n\geq N_i
                \end{equation*}
                Sin embargo, como $J$ es finito, podemos tomar $N = \max\limits_{i \in J}N_i$ y tendremos que:
                \begin{equation*}
                    n\geq N \Longrightarrow \varphi_i(x_n) \in V_i \qquad \forall i \in J
                \end{equation*}
                de donde:
                \begin{equation*}
                    x_n \in W = \bigcap_{i \in J}\varphi_i^{-1}(V_i) \subset U \quad \forall n\geq N
                \end{equation*}
                Por lo que $\{x_n\}\to x$.
        \end{description}
    \end{proof}
\end{prop}

\noindent
Por lo que conociendo la convergencia de las sucesiones en los espacios $Y_i$, estudiar la convergencia de $X$ con la topología inicial se reduce a estudiar las convergencias de sus imágenes por $\varphi_i$, esto hace fácil trabajar con sucesiones en la topología inicial. Sin embargo, no todos los conceptos topológicos se pueden caracterizar por sucesiones.

Otra propiedad útil de las topologías iniciales es la siguiente:

\begin{prop}
    Sea $(X,\tau)$ con $\tau$ la topología inicial asociada a una familia de aplicaciones $\{\varphi_i\}_{i \in I}$. Si $Z$ es un espacio topológico y tenemos una aplicación entre espacios topológicos $\psi:Z\to X$, entonces:
    \begin{equation*}
        \psi \text{\ es continua} \Longleftrightarrow \varphi_i\circ \psi:Z\to Y_i \text{\ es continua}\quad \forall i \in I
    \end{equation*}
    \begin{proof}
        Por doble implicación:
        \begin{description}
            \item [$\Longrightarrow )$] Si $\psi:Z\to X$ es continua, $\tau$ hace que cada $\varphi_i$ sea continua, por lo que $\varphi_i\circ\psi$ es continua.
            \item [$\Longleftarrow )$] Para esta, tenemos que si $U\in \tau$, entonces podemos escribir:
                \begin{equation*}
                    U = \bigcup \bigcap_{J} \varphi_i^{-1}(\omega_i)
                \end{equation*}
                para ciertos conjuntos abiertos $\omega_i$ de $Y_i$, de donde:
                \begin{equation*}
                    \psi^{-1}(U) = \bigcup\bigcap_{J}\psi^{-1}(\varphi^{-1}_i(\omega_i)) = \bigcup\bigcap_J {(\varphi_i \circ \psi_i)}^{-1}(\omega_i)
                \end{equation*}
                como la intersección finita de abiertos en $Z$ es un abierto de $Z$ y la unión arbitraria de abiertos de $Z$ también lo es, tenemos que $\psi^{-1}(U)$ es abierto en $Z$, para cada $U\in \tau$, por lo que $\psi$ es continua.
        \end{description}
    \end{proof}
\end{prop}

\noindent
Y la idea es la misma de la Proposición anterior: aunque no conozcamos con exactitud los abiertos de la topología inicial, estudiar las funciones continuas $Z\to X$ se reduce al problema de estudiar la continuidad de cada una de las funciones resultantes con componer con $\varphi_i$, obteniendo funciones $Z\to Y_i$. Este procedimiento hace que sea muy fácil comprobar qué aplicaciones $Z\to X$ son continuas.

\section{Topología débil}
\noindent
Sea $(E,\|\cdot \|_E)$ un espacio normado, tenemos ya sobre $E$ una topología, la asociada a la norma $\|\cdot \|_E$, que denotaremos a veces por $\tau_{\|\cdot \|_E}$. Definiremos sobre este espacio $E$ otra topología:

\begin{definicion}[Topología débil de un espacio normado]
    Sea $(E,\|\cdot \|_E)$ un espacio normado, definimos la topología débil en $E$ como la topología inicial en $E$ que hace que todas las aplicaciones de la familia $E^\ast$ sean continuas, y denotaremos a esta topología por $\sigma(E,E^\ast)$.
\end{definicion}

\begin{observacion}
    Observaciones que hay que tener en cuenta al trabajar con $\sigma(E,E^\ast)$:
    \begin{itemize}
        \item La notación $\sigma(E,E^\ast)$ hay que pensarla como la ``topología débil en $E$ es la topología inicial en $E$ que hace continuos todos aquellos elementos de $E^\ast$''.
        \item Tenemos $Y_f = \mathbb{R}$ para cada $f\in E^\ast$, donde tomamos como conjunto de índices $I = E^\ast$.
        \item Observemos que:
            \begin{equation*}
                \sigma(E,E^\ast) \subset \tau_{\|\cdot \|_E}
            \end{equation*}
            Ya que toda aplicación $f\in E^\ast$ es continua en $\tau_{\|\cdot \|_E}$ y $\sigma(E,E^\ast)$ es, por definición de topología inicial, la topología más pequeña que hace que las aplicaciones de $E^\ast$ sean continuas.
    \end{itemize}
\end{observacion}

\noindent
Destacaremos a continuación propiedades destacables de la topología débil de un espacio normado $E$, donde siempre que hagamos referencia $\sigma(E,E^\ast)$, estaremos trabajando sobre un espacio normado $E$ arbitrario.

\begin{prop}
    $\sigma(E,E^\ast)$ es Hausdorff.
    \begin{proof}
        Sean $x_1,x_2\in E$ distintos, tomamos:
        \begin{equation*}
            A = \{x_1\}, \qquad B = \{x_2\}
        \end{equation*}
        que son dos conjuntos convexos, disjuntos y cerrados para $\tau_{\|\cdot \|_E}$, lo que nos permite aplicar la segunda versión geométrica del Teorema de Hahn-Banach (Teorema~\ref{teo:hahn-banach_2aversiongeometrica}), obteniendo $f\in E^\ast\setminus\{0\}$ y $\alpha\in \mathbb{R}$ de forma que:
        \begin{equation*}
            \langle f,x_1 \rangle  < \alpha < \langle f,x_2 \rangle 
        \end{equation*}
        de donde tomando:
        \begin{align*}
            x_1 &\in \Theta_1 := \{x\in E : \langle f,x \rangle <\alpha\} = f^{-1}(\left]-\infty,\alpha\right[) \in \sigma(E,E^\ast) \\
            x_2 &\in \Theta_2 := \{x\in E : \langle f,x \rangle >\alpha\} = f^{-1}(\left]\alpha,+\infty\right[) \in \sigma(E,E^\ast) 
        \end{align*}
        Tenemos que $\Theta_1,\Theta_2$ son disjuntos entre sí, con lo que nos dan la condición de Hausdorff que buscábamos.
    \end{proof}
\end{prop}

\noindent
Veamos ahora una base de entornos en $\sigma(E,E^\ast)$, aplicando el procedimiento que hicimos anteriormente al construir la topología inicial.

\begin{prop}
    Dado $x_0\in E$ y $f_1, \ldots, f_k\in E^\ast$, tenemos que:
    \begin{enumerate}
        \item $V = V(f_1, \ldots, f_k;\varepsilon) = \{x\in E : |\langle f_i,x-x_0 \rangle | < \varepsilon, \quad i \in \{1,\ldots, k\}\}$ es un entorno de $x_0$ en $\sigma(E,E^\ast)$, para todo $\varepsilon>0$.
        \item Además, $\cc{V} = \{V(f_1, \ldots, f_k;\varepsilon) : \varepsilon>0, \quad f_1,\ldots,f_k\in E^\ast\}$ es base de entornos de $x_0$ en $\sigma(E,E^\ast)$.
    \end{enumerate}
    \begin{proof}
        Veamos cada apartado:
        \begin{enumerate}
            \item Si definimos:
                \begin{equation*}
                    a_i = \langle f_i ,x_0 \rangle  \qquad \forall i \in \{1,\ldots,k\}
                \end{equation*}
                dado $\varepsilon>0$, tenemos que:
                \begin{equation*}
                    |\langle f_i,x \rangle -\langle f_i,x_0 \rangle | < \varepsilon \Longleftrightarrow \langle f_i,x_0 \rangle -\varepsilon\leq \langle f_i,x \rangle \leq \langle f_i,x_0 \rangle  + \varepsilon
                \end{equation*}
                que a su vez equivale a:
                \begin{equation*}
                    x \in f_i^{-1}\left(\langle f_i,x_0 \rangle -\varepsilon,\langle f_i,x_0 \rangle +\varepsilon\right)
                \end{equation*}
                Por lo que podemos escribir:
                \begin{equation*}
                    V = \bigcap_{i = 1}^k f_i^{-1}(a_i - \varepsilon,a_i + \varepsilon)
                \end{equation*}
                como $f_i$ es continua para la topología débil, el conjunto $V$ ha de ser abierto para $\sigma(E,E^\ast)$, y $x_0\in V$, por lo que $V$ es un entorno abierto de $x_0$.
            \item Para ver que es base de entornos, si tomamos $U$ un entorno abierto de $x_0$ en $\sigma(E,E^\ast)$, tenemos entonces que existe un entorno de la base de entornos de $\sigma(E,E^\ast)$, luego existen $f_1, \ldots, f_k\in E^\ast$ de forma que:
                \begin{equation*}
                    \bigcap_{j=1}^k f_j^{-1}(V_j) \subset U
                \end{equation*}
                como $V_j$ es un entorno de $f_j(x_0)$ en la topología usual en $\mathbb{R}$, ha de existir $\varepsilon>0$ de forma que:
                \begin{equation*}
                    \left]f_j(x_0)-\varepsilon,f_j(x_0)+\varepsilon\right[\subset V_j \qquad \forall j \in \{1,\ldots,k\}
                \end{equation*}

                de donde:
                \begin{equation*}
                    V(f_1, \ldots, f_k;\varepsilon) \subset \bigcap_{j = 1}^k f_j^{-1}(V_j) \subset U
                \end{equation*}
        \end{enumerate}
    \end{proof}
\end{prop}

\noindent
Esta proposición nos permite, tomado $x\in E$ y $U$ un abierto de $x$, han de existir $f_1, \ldots, f_k\in E^\ast$ y $\varepsilon>0$ de forma que:
\begin{equation*}
    x\in V(f_1, \ldots, f_k;\varepsilon)\subset U
\end{equation*}

\begin{ejercicio} % // TODO: PONER BIEN
    Probar que $dim E < \infty \Longrightarrow \sigma(E,E^\ast) = \tau_{\|\cdot \|_E}$.\\

    \noindent
    Si $dim E < \infty$, $E$ ha de ser isométrico a $\mathbb{R}^N$, para $N = dim E$, y tenemos que:
    \begin{equation*}
        E^\ast = \{f:E\to \mathbb{R} \text{\ de forma que\ } f \text{\ es lineal}\}
    \end{equation*}
    Nos preguntamos ahora por la topología que hace que todas estas sean continuas. Si tomamos $\{x_n\}$ una sucesión de $\mathbb{R}^N$ Y $x\in \mathbb{R}^N$, tenemos que:
    \begin{equation*}
        \{x_n\} \stackrel{(\ast)}{\to} x \Longleftrightarrow \{f(x_n)\} \to f(x) \qquad \forall f\in E^\ast
    \end{equation*}
    Además, una base de $E^\ast$ es $\{\pi_1, \ldots, \pi_N\}$, donde $\pi_i$ es la proyección en $i-$ésima coordenada, de donde:
    \begin{equation*}
        \{f(x_n)\} \to f(x) \quad \forall f\in E^\ast \Longleftrightarrow \{x_n(i)\} =  \{\pi_i(x_n)\} \to \pi_i(x) = x(i) \quad \forall i \in \{1,\ldots, k\}
    \end{equation*}
    y esta última condición es equivalente a decir que $\{x_n\}$ converge a $x$ con la norma de $E$.
\end{ejercicio}

\noindent
Resumimos en la siguiente proposición cada una de las relaciones entre las convergencias de sucesiones en los distintos espacios topológicos que manejamos. Antes de ello, introducimos la siguiente notación:

\begin{notacion}
    Si $(E,\|\cdot \|_E)$ es un espacio normado, consideraremos sobre él habitualmente dos topologías posiblemente distintas (por lo que obtendremos distintas convergencias de sucesiones):
    \begin{equation*}
        \tau_{\|\cdot \|_E} \qquad \text{y}\qquad \sigma(E,E^\ast)
    \end{equation*}
    Si tenemos una sucesión $\{x_n\}$ de puntos de $E$ y un punto $x\in E$, será costumbre para nosotros:
    \begin{itemize}
        \item notar por ``$\{x_n\}\to x$'' si la sucesión $\{x_n\}$ es convergente en $\tau_{\|\cdot \|_E}$ al elemento $x$, diciendo en alguna ocasión que la sucesión $\{x_n\}$ ``converge'' o que ``converge fuertemente'' al elemento $x$.
        \item notar por ``$\{x_n\}\rightharpoonup x$'' si la sucesión $\{x_n\}$ es convergente en $\sigma(E,E^\ast)$ al elemento $x$, diciendo en alguna ocasión que la sucesión $\{x_n\}$ ``converge débilmente'' al elemento $x$.
    \end{itemize}
    Todavía no está del todo claro la relación entre estas dos convergencias distintas de sucesiones de puntos de $E$, que aclararemos en la siguiente Proposición, pero ya podremos hablar de convergencia de sucesiones de puntos de $E$ de forma cómoda, sin confundir en ningún momento la convergencia de $\sigma(E,E^\ast)$ con la de $\tau_{\|\cdot \|_E}$.
\end{notacion}

\begin{prop}
    Sea $E$ un espacio normado y $\{x_n\}$ una sucesión de puntos de $E$:
    \begin{enumerate}
        \item $\{x_n\} \rightharpoonup x \Longleftrightarrow \{\langle f,x_n \rangle \}\to \langle f,x \rangle \quad \forall f\in E^\ast$.
        \item $\{x_n\}\to x \Longrightarrow \{x_n\} \rightharpoonup x$.
        \item $\{x_n\}\rightharpoonup x \Longrightarrow \{\|x_n\|\}$ acotada y $\|x\|\leq \liminf \|x_n\|$.
        \item Tenemos:
            \begin{equation*}
                \left.\begin{array}{r}
                    \{x_n\}\rightharpoonup x \\
                    \{f_n\} \to f
                \end{array}\right\} \Longrightarrow \{\langle f_n,x_n \rangle \}\to \langle f,x \rangle 
            \end{equation*}
    \end{enumerate}
    \begin{proof}
        Demostramos cada una de las propiedades:
        \begin{enumerate}
            \item Es la Proposición~\ref{prop:top_inicial_suc} pero usando la notación para la topología débil de $E$.
            \item Si tenemos $\{x_n\} \to x$, entonces para $f\in E^\ast$:
                \begin{equation*}
                    |\langle f,x_n-x \rangle | \leq \|f\|\|x_n-x\|  \qquad \forall n\in \mathbb{N}
                \end{equation*}
                y como $\|x_n-x\| \to 0$, deducimos que $\langle f,x_n-x \rangle \to 0 $, luego tenemos que:
                \begin{equation*}
                    \{\langle f,x_n \rangle \}\to \langle f,x \rangle  \qquad \forall f\in E^\ast
                \end{equation*}
                y usando 1 tenemos que $\{x_n\}\rightharpoonup x$.
            \item Tomamos $B = \{x_n : n\in \mathbb{N}\}$, que verifica para $f\in E^\ast$:
                \begin{equation*}
                    f(B) = \{\langle f,x_n \rangle : n\in \mathbb{N}\}
                \end{equation*}
                Como $\{x_n\}\rightharpoonup x$, el apartado 1 nos dice que $f(B)$ es acotado, $\forall f\in E^\ast$. Por el Corolario~\ref{coro:entonces_B_acotado} deducimos que $B$ es acotado, es decir, que $\{\|x_n\|\}$ está acotada.

                Para la segunda parte, si tomamos $f\in E^\ast$, tenemos que:
                \begin{equation*}
                    |\langle f,x_n \rangle | \leq \|f\|\|x_n\| \qquad \forall n\in \mathbb{N}
                \end{equation*}
                si tomamos límite inferior:
                \begin{equation*}
                    \langle f,x \rangle = \lim_{n\to\infty}\langle f,x_n \rangle  = \liminf \langle f,x_n \rangle  \leq \|f\|\liminf \|x_n\| \qquad \forall f\in E^\ast
                \end{equation*}
                En particular, si tomamos $\|f\| = 1$, tenemos que:
                \begin{equation*}
                    \|x\| = \sup_{\|f\|\leq 1}\langle f,x \rangle  \leq \|f\| \liminf \|x_n\| = \liminf \|x_n\|
                \end{equation*}
            \item Estudiamos la diferencia:
                \begin{align*}
                    |\langle f_n,x_n \rangle  - \langle f,x \rangle | &\leq |\langle f_n-f,x_n \rangle | + |\langle f,x_n-x \rangle |  \\
                                                                      &\leq \|f_n-f\| \|x_n\| + |\langle f,x_n \rangle -\langle f,x \rangle | \qquad \forall n\in \mathbb{N}
                \end{align*}
                Y tenemos que $\|f_n-f\| \to 0$, que $\{\|x_n\|\}$ está acotada, y que $\langle f,x_n \rangle \to \langle f,x \rangle $, de donde deducimos que $|\langle f_n,x_n \rangle -\langle f,x \rangle |\to 0$.
        \end{enumerate}
    \end{proof}
\end{prop}

\noindent
Para entender mejor el punto 3 de esta Proposición, introducimos el siguiente concepto:
\begin{definicion}
    Sea $(E,\tau)$ un espcio topológico, sea $f:(E,\tau)\to \mathbb{R}$ una aplicción, decimos que la función $f$ es \underline{secuencialmente semicontinua inferiormente} si se cumple que:
    \begin{equation*}
        \{x_n\} \to x \Longrightarrow f(x) \leq \liminf f(x_n)
    \end{equation*}
\end{definicion}~\\

\noindent
Notemos que sabíamos que la aplicación 
\begin{equation*}
    \|\cdot \|:(E,\tau_{\|\cdot \|_E})\to \mathbb{R}
\end{equation*}
es continua. Sin embargo, en vista de la Proposición y la Definición anterior, sabemos que la aplicación 
\begin{equation*}
    \|\cdot \|:(E,\sigma(E,E^\ast))\to \mathbb{R}
\end{equation*}
es secuencialmente semicontinua inferiormente.\\

\noindent
Nos preguntamos ahora por el recícproco de la propiedad 3, si tenemos una sucesión $\{\|x_n\|\}$ acotada con $\{x_n\}$ una sucesión de puntos de $E$, ¿será cierto que $\{x_n\}\rightharpoonup x$? La respuesta a esta pregunta es rotundamente negativa, pues sabemos que en dimensión finita $\sigma(E,E^\ast) = \tau_{\|\cdot \|_E}$, y sabemos de la existencia de sucesiones acotadas que no son convergentes en cualquier espacio normado $N-$dimensional.

Sin embargo, si recordamos el Teorema de Bolzano-Weierstrass, en todo espacio normado $N-$dimensional siempre que teníamos una sucesión acotada podríamos extraer una parcial suya convergente. Veremos próximamente que una propiedad similar a esta se cumple en la topología débil de $E$, lo que nos permitirá llegar a un Teorema que relacione los conjuntos compactos de $\sigma(E,E^\ast)$ con los conjuntos cerrados y acotados, brindándonos un espacio topológico con una cantidad abundante de conjuntos compactos, cosa que no sucede en $\tau_{\|\cdot \|_E}$ cuando la dimensión del espacio $E$ no es finita.

Esta propiedad de $\sigma(E,E^\ast)$ es totalmente natural, pues al considerar como $\sigma(E,E^\ast)$ la menor topología sobre $E$ que hace que las aplicaciones de $E^\ast$ sean continuas lo que estamos haciendo es eliminar de $\tau_{\|\cdot \|_E}$ abiertos que no nos interesa considerar en ciertos momentos, haciendo más fácil que un conjunto sea compacto, pues cuantos menos abiertos contenga una topología más fácil será que un conjunto sea compacto, por la propia definición de conjunto compacto en un espacio topológico general.



    \chapter{Ejercicios}
    \section{Lógica Proposicional}

\begin{ejercicio}\label{ej:1.1}
    Demuestra que el conjunto de proposiciones es numerable.
\end{ejercicio}

\begin{ejercicio}\label{ej:1.2}
    Demuestra que las siguientes proposiciones son tautologías.
    \begin{enumerate}
        \item Ley de doble negación: $\neg\neg a \rightarrow a$.
        
        Por el Teorema de la Deducción, esto equivale a demostrar:
        \begin{equation*}
            \{\neg\neg a\} \models a.
        \end{equation*}

        Sea $I$ una interpretación tal que $I(\neg\neg a) = 1$. Entonces:
        \begin{align*}
            1=I(\neg\neg a) &= 1 + I(\neg a) = 1+1+I(a) = 2+I(a) = I(a)
        \end{align*}

        Por tanto, $I(a) = 1$, y por lo tanto, $\{\neg\neg a\} \models a$.
        \item Leyes de simplificación:
        \begin{enumerate}
            \item $(a\land b) \rightarrow a$.
            
            Por el Teorema de la Deducción, esto equivale a demostrar:
            \begin{equation*}
                \{a\land b\} \models a.
            \end{equation*}

            Sea $I$ una interpretación tal que $I(a\land b) = 1$. Entonces:
            \begin{align*}
                1=I(a\land b) &= I(a)I(b)
            \end{align*}
            
            Por ser $\bb{Z}_2$ un cuerpo, en particular es un DI. Si fuese $I(a) = 0$, entonces $I(a)I(b) = 0$, lo cual es una contradicción. Por tanto, $I(a) = 1$, y por lo tanto, $\{a\land b\} \models a$.


            \item $a \rightarrow (a\lor b)$.
            
            Por el Teorema de la Deducción, esto equivale a demostrar:
            \begin{equation*}
                \{a\} \models a\lor b.
            \end{equation*}

            Sea $I$ una interpretación tal que $I(a) = 1$. Entonces:
            \begin{align*}
                I(a\lor b) &= I(a) + I(b) + I(a)I(b) = 1 + I(b) + I(b) = 1 + 2I(b) = 1
            \end{align*}

            Por tanto, $\{a\} \models a\lor b$.
        \end{enumerate}
        
        \item Ley de conmutatividad de la conjunción: $(a\land b) \rightarrow (b\land a)$.
        
        Por el Teorema de la Deducción, esto equivale a demostrar:
        \begin{equation*}
            \{a\land b\} \models b\land a.
        \end{equation*}

        Sea $I$ una interpretación tal que $I(a\land b) = 1$. Entonces:
        \begin{align*}
            1=I(a\land b) &= I(a)I(b)\AstIg = I(b)I(a)=I(b\land a)
        \end{align*}
        donde en $(\ast)$ se usa la conmutatividad de la multiplicación en $\bb{Z}_2$. Por tanto, $\{a\land b\} \models b\land a$.
        \item Ley de conmutatividad de la disyunción: $(a\lor b) \rightarrow (b\lor a)$.
        
        Por el Teorema de la Deducción, esto equivale a demostrar:
        \begin{equation*}
            \{a\lor b\} \models b\lor a.
        \end{equation*}

        Sea $I$ una interpretación tal que $I(a\lor b) = 1$. Entonces:
        \begin{align*}
            I(a\lor b) &= I(a) + I(b) + I(a)I(b) \AstIg I(b) + I(a) + I(b)I(a) = I(b\lor a)
        \end{align*}
        donde en $(\ast)$ se usa la conmutatividad de la suma y la multiplicación en $\bb{Z}_2$. Por tanto, $\{a\lor b\} \models b\lor a$.
        \item Ley de Clavius: $(\neg a \rightarrow a) \rightarrow a$.
        
        Por el Teorema de la Deducción, esto equivale a demostrar:
        \begin{equation*}
            \{\neg a \rightarrow a\} \models a.
        \end{equation*}

        Sea $I$ una interpretación tal que $I(\neg a \rightarrow a) = 1$. Entonces:
        \begin{align*}
            1=I(\neg a \rightarrow a) &= 1+I(\neg a) + I(\neg a)I(a) = 1+1+I(a)+(1+I(a))I(a) =\\&= I(a)+I(a)+I(a)=I(a)
        \end{align*}

        Por tanto, $\{\neg a \rightarrow a\} \models a$.
        \item Ley de De Morgan: $\neg(a\land b) \rightarrow (\neg a \lor \neg b)$.
        
        Por el Teorema de la Deducción, esto equivale a demostrar:
        \begin{equation*}
            \{\neg(a\land b)\} \models \neg a \lor \neg b.
        \end{equation*}

        Sea $I$ una interpretación tal que $I(\neg(a\land b)) = 1$. Entonces:
        \begin{align*}
            1=I(\neg(a\land b)) &= 1+I(a\land b) = 1+I(a)I(b)
            \Longrightarrow
            0=I(a)I(b)
        \end{align*}

        Por tanto:
        \begin{align*}
            I(\neg a \lor \neg b) &= I(\neg a) + I(\neg b) + I(\neg a)I(\neg b) = 1+I(a) + 1+I(b) + (1+I(a))(1+I(b)) =\\&= 1+\cancel{I(a)} + 1+\bcancel{I(b)} + 1+\cancel{I(a)}+\bcancel{I(b)}+I(a)I(b) = 1+I(a)I(b)=1
        \end{align*}

        Por tanto, $\{\neg(a\land b)\} \models \neg a \lor \neg b$.
        \item Segunda ley de De Morgan: $\neg(a\lor b) \rightarrow (\neg a \land \neg b)$.
        
        Por el Teorema de la Deducción, esto equivale a demostrar:
        \begin{equation*}
            \{\neg(a\lor b)\} \models \neg a \land \neg b.
        \end{equation*}

        Sea $I$ una interpretación tal que $I(\neg(a\lor b)) = 1$. Entonces:
        \begin{align*}
            1=I(\neg(a\lor b)) &= 1+I(a\lor b) = 1+I(a) + I(b) + I(a)I(b)
        \end{align*}

        Por tanto:
        \begin{align*}
            I(\neg a \land \neg b) &= I(\neg a)I(\neg b) = (1+I(a))(1+I(b)) = 1+I(a)+I(b)+I(a)I(b) = 1
        \end{align*}

        Por tanto, $\{\neg(a\lor b)\} \models \neg a \land \neg b$.
        \item Ley de inferencia alternativa: $((a\lor b)\land \neg a) \rightarrow b$.
        
        Por el Teorema de la Deducción, esto equivale a demostrar:
        \begin{equation*}
            \{(a\lor b)\land \neg a\} \models b.
        \end{equation*}

        Sea $I$ una interpretación tal que $I((a\lor b)\land \neg a) = 1$. Entonces:
        \begin{align*}
            1=I((a\lor b)\land \neg a) &= I(a\lor b)I(\neg a) = \left(I(a)+I(b)+I(a)I(b)\right)\left(1+I(a)\right) =\\&= \bcancel{I(a)}+I(b)+I(a)I(b)\ +\ \bcancel{I(a)}+\cancel{I(a)I(b)} + \cancel{I(a)I(b)}
            =\\&= I(b)(1+I(a))
        \end{align*}

        Si fuese $I(b) = 0$, entonces $I(b)(1+I(a)) = 0$, lo cual es una contradicción. Por tanto, $I(b) = 1$, y por lo tanto, $\{(a\lor b)\land \neg a\} \models b$.
        \item Segunda ley de inferencia alternativa: $((a\lor b)\land \neg b) \rightarrow a$.
        
        Se tiene de forma directa por el apartado anterior (intercambiando los papeles de $a$ y $b$). La demostración es análoga empleando la conmutatividad de $\bb{Z}_2$.
        \item Modus ponendo ponens: $((a \rightarrow b)\land a) \rightarrow b$.
        
        Por el Teorema de la Deducción, esto equivale a demostrar:
        \begin{equation*}
            \{(a \rightarrow b)\land a\} \models b.
        \end{equation*}

        Sea $I$ una interpretación tal que $I((a \rightarrow b)\land a) = 1$. Entonces:
        \begin{align*}
            1=I((a \rightarrow b)\land a) &= I(a \rightarrow b)I(a) = \left(1+I(a)+I(a)I(b)\right)I(a) =\\&= I(a)+I(a)+I(a)I(b) = I(a)I(b)
        \end{align*}

        Si fuese $I(b) = 0$, entonces $I(a)I(b) = 0$, lo cual es una contradicción. Por tanto, $I(b) = 1$, y por lo tanto, $\{(a \rightarrow b)\land a\} \models b$.
        \item Modus tollendo tollens: $((a \rightarrow b)\land \neg b) \rightarrow \neg a$.
        
        Por el Teorema de la Deducción, esto equivale a demostrar:
        \begin{equation*}
            \{(a \rightarrow b)\land \neg b\} \models \neg a.
        \end{equation*}

        Sea $I$ una interpretación tal que $I((a \rightarrow b)\land \neg b) = 1$. Entonces:
        \begin{align*}
            1=I((a \rightarrow b)\land \neg b) &= I(a \rightarrow b)I(\neg b) = \left(1+I(a)+I(a)I(b)\right)(1+I(b)) =\\&= 1+I(a)+I(a)I(b)+I(b)+\cancel{I(a)I(b)}+\cancel{I(a)I(b)} =\\&= 1+I(a)+I(b)+I(a)I(b)
            =\\&= (1+I(b)) + I(a)(1+I(b))
            = (1+I(a))(1+I(b))
        \end{align*}
        Si fuese $1+I(a) = 0$, entonces $(1+I(a))(1+I(b)) = 0$, lo cual es una contradicción. Por tanto, $1+I(a) = 1 = I(\neg a)$. Por tanto, $\{(a \rightarrow b)\land \neg b\} \models \neg a$.
    \end{enumerate}
\end{ejercicio}

\begin{ejercicio}\label{ej:1.3}
    Dado un conjunto de proposiciones $\Gamma \cup \{\alpha, \beta\}$. Si $\Gamma \cup \{\alpha\} \models \beta$, entonces $\Gamma \models \alpha \rightarrow \beta$.\\

    Notemos que en este caso tan solo tenemos una de las implicaciones del Teorema de la Deducción. Sea $I$ una interpretación tal que $I(\gamma)=1$ para todo $\gamma\in\Gamma$.
    \begin{itemize}
        \item Si $I(\alpha)=1$, como $\Gamma \cup \{\alpha\} \models \beta$, entonces $I(\beta)=1$. Por tanto:
        \begin{align*}
            I(\alpha\rightarrow\beta) &= 1+I(\alpha)+I(\alpha)I(\beta) = 1+1+1 = 1
        \end{align*}

        \item Si $I(\alpha)=0$, entonces:
        \begin{align*}
            I(\alpha\rightarrow\beta) &= 1+I(\alpha)+I(\alpha)I(\beta) = 1+0+0 = 1
        \end{align*}
    \end{itemize}

    Por tanto, $\Gamma \models \alpha \rightarrow \beta$.
\end{ejercicio}

\begin{ejercicio}\label{ej:1.4}
    El señor Pérez, empadronador de la isla de Tururulandia, tiene como objetivo el censar la población de dicha isla. La tarea no es fácil debido al hecho de que la población se divide en dos grupos bien distinguidos: los honrados y los embusteros. Los honrados siempre dicen la verdad, mientras que un embustero solo es capaz de producir mentiras. El gobierno de la isla encarga como trabajo al señor Pérez la ardua tarea de contar los honrados y embusteros de la isla.
    He aquí cuatro de los muchos problemas con los que se encontró nuestro empadronador.
    \begin{enumerate}
        \item Llama a la puerta de una casa, en la que sabía a ciencia cierta que vivía un matrimonio, y el marido abre la puerta para ver quién es. El empadronador le dice: ``necesito información sobre usted y su esposa. ¿Cuál de ustedes, si alguno lo es, es honrado y cuál un embustero?,'' a lo que el hombre de la casa respondió ``ambos somos embusteros,'' cerrando la puerta de golpe. ¿Qué es el marido y qué es la mujer?
        
        Sean las siguientes proposiciones atómicas:
        \begin{itemize}
            \item $p\equiv$ ``El marido es honrado.''
            \item $q\equiv$ ``La mujer es honrada.''
        \end{itemize}

        Para empezar, sabemos que $p\longleftrightarrow \neg p\land \neg q$ es cierto. Por tanto, fijada una interpretación $I$ tal que $I(p\longleftrightarrow (\neg p\land \neg q))=1$, entonces:
        \begin{align*}
            1 &= I(p\longleftrightarrow (\neg p\land \neg q))
            = 1+I(p) + I(\neg p\land \neg q)
            = 1+I(p) + I(\neg p)I(\neg q)
            =\\&= 1+I(p) + (1+I(p))(1+I(q))
            = (1+I(p))(I(q))
        \end{align*}

        Por tanto:
        \begin{itemize}
            \item $I(p) = 0\Longrightarrow$ El marido es un embustero.
            \item $I(q) = 1\Longrightarrow$ La mujer es honrada.
        \end{itemize}
        \item La segunda casa que visita también está habitada por un matrimonio. Al llamar a la puerta y formular la misma pregunta que antes, el marido responde: ``Por lo menos uno de nosotros es un embustero,'' cerrando a continuación la puerta. ¿Qué es el marido y qué es la mujer?
        
        Sean las siguientes proposiciones atómicas:
        \begin{itemize}
            \item $p\equiv$ ``El marido es honrado.''
            \item $q\equiv$ ``La mujer es honrada.''
        \end{itemize}

        Para empezar, sabemos que $p\longleftrightarrow \neg p\lor \neg q$ es cierto. Por tanto, fijada una interpretación $I$ tal que $I(p\longrightarrow (\neg p\lor \neg q))=1$, entonces:
        \begin{align*}
            1 &= I(p\longleftrightarrow (\neg p\lor \neg q))
            = 1+I(p) + I(\neg p\lor \neg q)
            =\\&= 1+I(p) + I(\neg p)+I(\neg q) + I(\neg p)I(\neg q)
            =\\&= \cancel{1+I(p)} + \cancel{1+I(p)} + 1+I(q) + (1+I(p))(1+I(q))
            =\\&= (1+I(q))(1+1+I(p)) = I(p)(1+I(q))
        \end{align*}

        Por tanto:
        \begin{itemize}
            \item $I(p) = 1\Longrightarrow$ El marido es honrado.
            \item $I(q) = 0\Longrightarrow$ La mujer es una embustera.
        \end{itemize}
        \item Visita una tercera casa, y en las mismas condiciones de antes, recibe la respuesta: ``Si yo soy honrado, entonces mi mujer también lo es.'' ¿Qué es el marido y qué es la mujer?
        
        Sean las siguientes proposiciones atómicas:
        \begin{itemize}
            \item $p\equiv$ ``El marido es honrado.''
            \item $q\equiv$ ``La mujer es honrada.''
        \end{itemize}

        Para empezar, sabemos que $p\longleftrightarrow (p\longrightarrow q)$ es cierto. Por tanto, fijada una interpretación $I$ tal que $I(p\longleftrightarrow (p\longrightarrow q))=1$, entonces:
        \begin{align*}
            1 &= I(p\longleftrightarrow (p\longrightarrow q))
            = 1+I(p) + I(p\longrightarrow q)
            =\\&= 1+I(p) + 1+I(p)+I(p)I(q)
            = I(p)I(q)
        \end{align*}

        Por tanto:
        \begin{itemize}
            \item $I(p) = 1\Longrightarrow$ El marido es honrado.
            \item $I(q) = 1\Longrightarrow$ La mujer es honrada.
        \end{itemize}
        \item En la última casa que visita, pues ya estaba cansado de partirse el coco, la respuesta es ``Yo soy lo mismo que mi mujer.'' ¿Qué es el marido y qué es la mujer?
        Sean las siguientes proposiciones atómicas:
        \begin{itemize}
            \item $p\equiv$ ``El marido es honrado.''
            \item $q\equiv$ ``La mujer es honrada.''
        \end{itemize}

        Para empezar, sabemos que $p\longleftrightarrow (p\longleftrightarrow q)$ es cierto. Por tanto, fijada una interpretación $I$ tal que $I(p\longleftrightarrow (p\longleftrightarrow q))=1$, entonces:
        \begin{align*}
            1 &= I(p\longleftrightarrow (p\longleftrightarrow q))
            = 1+I(p) + I(p\longleftrightarrow q)
            =\\&= 1+I(p) + 1+I(p)+I(q)
            = I(q)
        \end{align*}

        Por tanto:
        \begin{itemize}
            \item $I(q) = 1\Longrightarrow$ La mujer es honrada.
            \item Respecto al marido, no podemos determinar si es honrado o no.
        \end{itemize}
        

        \item De vuelta a su casa se encuentra con tres individuos, A, B y C, en la calle, y pensando en que quizás podía tener más suerte con ellos decide preguntarles qué son cada uno de ellos. Le pregunta al primero, A, y no entiende la respuesta, ya que en ese momento pasa una de esas motos que hacen un ruido ensordecedor y no corren nada. El segundo, B, le aclara que lo que ha dicho el primero es que es un embustero, pero el tercero, C, le advierte que no haga caso del segundo, B, ya que es un embustero. ¿Puedes deducir algo de lo ocurrido?
        
        Sean las siguientes proposiciones atómicas:
        \begin{itemize}
            \item $a\equiv$ ``A es honrado.''
            \item $b\equiv$ ``B es honrado.''
            \item $c\equiv$ ``C es honrado.''
            \item $p\equiv$ ``A dice que es un embustero.''
        \end{itemize}

        Sabemos que:
        \begin{itemize}
            \item $a\longrightarrow \neg p$ es cierto.
            \item $\neg a\longrightarrow \neg p$ es cierto.
            \item $b\longleftrightarrow p$ es cierto.
            \item $c\longleftrightarrow \neg b$ es cierto.
        \end{itemize}

        Tenemos que:
        \begin{align*}
            1&= I(a\longrightarrow \neg p) = 1+I(a)+I(a)I(\neg p)\\
            1&= I(\neg a\longrightarrow \neg p) = 1+I(\neg a)+I(\neg a)I(\neg p)
            =\\&= 1+1+I(a) + (1+I(a))I(\neg p) =
            1+1+I(a)+I(\neg p)+I(a)I(\neg p)=\\&=1+I(\neg p)+1=I(\neg p)\Longrightarrow
            I(p)=0
        \end{align*}

        Además, sabemos que:
        \begin{align*}
            1&= I(b\longleftrightarrow p) = 1+I(b)+I(p)=1+I(b)\Longrightarrow
            I(b)=0\\
            1&= I(c\longleftrightarrow \neg b) = 1+I(c)+I(\neg b)=1+I(c)+1+I(b)=I(c)
        \end{align*}

        Por tanto:
        \begin{itemize}
            \item $I(b)=0\Longrightarrow$ B es un embustero.
            \item $I(c)=1\Longrightarrow$ C es honrado.
            \item Desconocemos el valor de $I(a)$, por lo que no podemos determinar si A es honrado o no.
        \end{itemize}
    \end{enumerate}
\end{ejercicio}


\begin{ejercicio}\label{ej:1.5}
    Probar que todo axioma del cálculo proposicional clásico es una tautología.\\

    Sean $\alpha,\beta,\gamma$ fórmulas proposicionales. Entonces, probaremos que cada uno de los axiomas del cálculo proposicional clásico es una tautología.
    \begin{enumerate}
        \item $\cc{A}_1 = \{\alpha\rightarrow(\beta\rightarrow\alpha)\}$.
        
        Aplicando dos veces el Teorema de la Deducción, hemos de probar que:
        \begin{equation*}
            \{\alpha,\beta\} \models \alpha
        \end{equation*}

        Sea $I$ una interpretación tal que $I(\alpha) = I(\beta) = 1$. Entonces, trivialmente $I(\alpha) = 1$. Por tanto, $\{\alpha,\beta\} \models \alpha$.
        \item $\cc{A}_2 = \{(\alpha\rightarrow(\beta\rightarrow\gamma))\rightarrow((\alpha\rightarrow\beta)\rightarrow(\alpha\rightarrow\gamma))\}$.
        
        Aplicando tres veces el Teorema de la Deducción, hemos de probar que:
        \begin{equation*}
            \{\alpha\rightarrow(\beta\rightarrow\gamma),\alpha\rightarrow\beta,\alpha\} \models \gamma
        \end{equation*}

        Sea $I$ una interpretación tal que $I\left(\alpha\rightarrow(\beta\rightarrow\gamma)\right) = I\left(\alpha\rightarrow\beta\right) = I(\alpha) = 1$. Entonces:
        \begin{align*}
            1&= I(\alpha\rightarrow\beta) = 1+I(\alpha)+I(\alpha)I(\beta)=1+1+I(\beta)\Longrightarrow I(\beta)=1\\
            1&=I\left(\alpha\rightarrow(\beta\rightarrow\gamma)\right) = 1+I(\alpha)+I(\alpha)I(\beta\rightarrow\gamma) = I(\beta\rightarrow\gamma)
            = 1+I(\beta)+I(\beta)I(\gamma) = I(\gamma)
        \end{align*}

        Por tanto, $\{\alpha\rightarrow(\beta\rightarrow\gamma),\alpha\rightarrow\beta,\alpha\} \models \gamma$.
        \item $\cc{A}_3 = \{(\neg\alpha\rightarrow\neg\beta)\rightarrow\left((\neg\alpha\rightarrow\beta)\rightarrow\alpha\right)\}$.
        
        Aplicando dos veces el Teorema de la Deducción, hemos de probar que:
        \begin{equation*}
            \{\neg\alpha\rightarrow\neg\beta,\neg\alpha\rightarrow\beta\} \models \alpha
        \end{equation*}

        Sea $I$ una interpretación tal que $I(\neg\alpha\rightarrow\neg\beta) = I(\neg\alpha\rightarrow\beta) = 1$. Entonces:
        \begin{align*}
            1&=I(\neg\alpha\rightarrow\beta) = 1+I(\neg\alpha)+I(\neg\alpha)I(\beta)\\
            1&=I(\neg\alpha\rightarrow\neg\beta) = 1+I(\neg\alpha)+I(\neg\alpha)I(\neg\beta) = 1+I(\neg\alpha)+I(\neg\alpha)\left(1+I(\beta)\right)
            =\\&= 1+I(\neg\alpha)+I(\neg\alpha)+I(\neg\alpha)I(\beta)=1+I(\neg\alpha)=1+1+I(\alpha)=I(\alpha)
        \end{align*}

        Por tanto, $\{\neg\alpha\rightarrow\neg\beta,\neg\alpha\rightarrow\beta\} \models \alpha$.
    \end{enumerate}
\end{ejercicio}

\begin{ejercicio}[Regla de ``reductio ad absurdum'' minimal o intuicionista]\label{ej:1.6}
    Si se tiene $\Gamma \cup \{\alpha\} \vdash \beta$ y $\Gamma \cup \{\alpha\} \vdash \neg\beta$, entonces $\Gamma \vdash \neg\alpha$.\\

    Por el Teorema de la Deducción, tenemos que:
    \begin{equation*}
        \Gamma\vdash \alpha\rightarrow\beta\qquad\qquad\Gamma\vdash \alpha\rightarrow\neg\beta
    \end{equation*}

    Por tanto:
    \begin{enumerate}
        \item[$1$.] \ldots\\\vdots
        \item[$p$.] $\alpha\rightarrow\beta$
        \item[$p+1$.] \ldots\\\vdots
        \item[$q$.] $\alpha\rightarrow\neg\beta$
        \item[$q+1$.] $\neg\neg\alpha\rightarrow \alpha$ por la Regla de la Doble Negación.
        \item[$q+2$.] $\neg\neg\alpha\rightarrow\beta$ por la Regla del Silogismo aplicada a $q+1$ y $p$.
        \item[$q+3$.] $\neg\neg\alpha\rightarrow\neg\beta$ por la Regla del Silogismo aplicada a $q+1$ y $q$.
    \end{enumerate}

    Como desde $1$ hasta $q$ tan solo se han empleado axiomas e hipótesis de $\Gamma$, entonces:
    \begin{equation*}
        \Gamma\vdash \neg\neg\alpha\rightarrow\beta\qquad\qquad\Gamma\vdash \neg\neg\alpha\rightarrow\neg\beta
    \end{equation*}

    Por tanto, aplicando el Teorema de la Deducción, tenemos que:
    \begin{equation*}
        \Gamma \cup \{\neg\neg\alpha\} \vdash \beta\qquad\qquad\Gamma \cup \{\neg\neg\alpha\} \vdash \neg\beta
    \end{equation*}

    Por la Regla de Reducción al Absurdo, tenemos que:
    \begin{equation*}
        \Gamma \vdash \neg\alpha
    \end{equation*}
\end{ejercicio}


\begin{ejercicio}[Leyes de Duns Scoto]\label{ej:1.7}~
    \begin{enumerate}
        \item $\vdash \neg\alpha \rightarrow (\alpha \rightarrow \beta)$.
        
        Por el Teorema de la Deducción, esto equivale a demostrar:
        \begin{equation*}
            \{\alpha,\neg\alpha\} \vdash \beta
        \end{equation*}

        Tenemos que:
        \begin{enumerate}[label=\arabic*.]
            \item $\alpha$ es una hipótesis.
            \item $\alpha\rightarrow(\neg\beta\rightarrow\alpha)\in \cc{A}_1$.
            \item $\neg\beta \rightarrow \alpha$ por Modus Ponens aplicado a $1$ y $2$.
            \item $\neg \alpha$ es una hipótesis.
            \item $\neg\alpha\rightarrow(\neg\beta\rightarrow\neg\alpha)\in \cc{A}_1$.
            \item $\neg\beta \rightarrow \neg\alpha$ por Modus Ponens aplicado a $4$ y $5$.
        \end{enumerate}

        Por tanto, y aplicando el Teorema de la Deducción, tenemos que:
        \begin{equation*}
            \{\alpha,\neg\alpha\} \cup \{\neg \beta\} \vdash \alpha\qquad\qquad\{\alpha,\neg\alpha\} \cup \{\neg \beta\} \vdash \neg\alpha
        \end{equation*}

        Por la Regla de Reducción al Absurdo, tenemos que:
        \begin{equation*}
            \{\alpha,\neg\alpha\} \vdash \beta
        \end{equation*}
        \item $\vdash \alpha \rightarrow (\neg\alpha \rightarrow \beta)$.
        
        Por el Teorema de la Deducción, esto equivale a demostrar:
        \begin{equation*}
            \{\alpha,\neg\alpha\} \vdash \beta
        \end{equation*}

        Esto ha sido demostrado en el apartado anterior.
    \end{enumerate}
\end{ejercicio}

\begin{ejercicio}[Principio de inconsistencia]\label{ej:1.8}
    Si $\Gamma \vdash \alpha$ y $\Gamma \vdash \neg\alpha$, entonces $\Gamma \vdash \beta$.
    \begin{enumerate}
        \item[$1$.] \ldots\\\vdots
        \item[$p$.] $\alpha$
        \item[$p+1$.] \ldots\\\vdots
        \item[$q$.] $\neg\alpha$
        \item[$q+1$.] $\neg\alpha\rightarrow(\alpha\rightarrow\beta)$ por la Ley de Duns Scoto (Ejercicio~\ref{ej:1.7}).
        \item[$q+2$.] $\alpha\rightarrow\beta$ por Modus Ponens aplicado a $q$ y $q+1$.
        \item[$q+3$.] $\beta$ por Modus Ponens aplicado a $p$ y $q+2$.
    \end{enumerate}

    Como desde $1$ hasta $q$ tan solo se han empleado axiomas e hipótesis de $\Gamma$:
    \begin{equation*}
        \Gamma \vdash \beta
    \end{equation*}
\end{ejercicio}

\begin{ejercicio}[Leyes débiles de Duns Scoto]\label{ej:1.9}~
    \begin{enumerate}
        \item $\vdash \neg\alpha \rightarrow (\alpha \rightarrow \neg\beta)$.
        \item $\vdash \alpha \rightarrow (\neg\alpha \rightarrow \neg\beta)$.
    \end{enumerate}
    
    
    Ambos casos se tienen de forma directa por el Ejercicio~\ref{ej:1.7}, puesto que lo tenemos demostrado para cualquier proposición $\beta$ (no es necesario que sea una proposición atómica), por lo que en particular se tiene para $\neg\beta$.
    \begin{observacion}
        Notemos que obtener las Leyes de Duns Scoto a partir de las Leyes débiles de Duns Scoto no sería directo y tendríamos que emplear la Regla de la Doble Negación.
    \end{observacion}        
\end{ejercicio}


\begin{ejercicio}[Principio de inconsistencia débil]\label{ej:1.10}
    Si $\Gamma \vdash \alpha$ y $\Gamma \vdash \neg\alpha$, entonces $\Gamma \vdash \neg\beta$.\\

    Al igual que ocurrió en el Ejercicio~\ref{ej:1.9}, esto se tiene de forma directa por el Ejercicio~\ref{ej:1.8}.
\end{ejercicio}

\begin{ejercicio}[Ley de contraposición fuerte o ``ponendo ponens'']\label{ej:1.11}
    $$\vdash (\neg\beta \rightarrow \neg\alpha) \rightarrow (\alpha \rightarrow \beta).$$
    Por el Teorema de la Deducción aplicado dos veces, esto equivale a demostrar:
    \begin{equation*}
        \{\neg\beta\rightarrow\neg\alpha,\alpha\} \vdash \beta
    \end{equation*}

    Supongamos además $\neg \beta$ como hipótesis (para poder aplicar reducción al absurso). Entonces:
    \begin{enumerate}
        \item $\neg\beta$ es una hipótesis.
        \item $\neg\beta\rightarrow \neg\alpha$ es una hipótesis.
        \item $\neg\alpha$ por Modus Ponens aplicado a $1$ y $2$.
        \item $\alpha$ es una hipótesis.
    \end{enumerate}

    Por tanto, tenemos que:
    \begin{equation*}
        \{\neg\beta\rightarrow\neg\alpha,\alpha\}\cup \{\neg\beta\} \vdash \alpha\qquad\qquad\{\neg\beta\rightarrow\neg\alpha,\alpha\}\cup \{\neg\beta\} \vdash \neg\alpha
    \end{equation*}

    Por la Regla de Reducción al Absurdo, tenemos que:
    \begin{equation*}
        \{\neg\beta\rightarrow\neg\alpha,\alpha\} \vdash \beta
    \end{equation*}
\end{ejercicio}

\begin{ejercicio}[Ley de contraposición ``ponendo tollens'']\label{ej:1.12}
    $$\vdash (\beta \rightarrow \neg\alpha) \rightarrow (\alpha \rightarrow \neg\beta).$$
    
    Por el Teorema de la Deducción aplicado dos veces, esto equivale a demostrar:
    \begin{equation*}
        \{\beta\rightarrow\neg\alpha,\alpha\} \vdash \neg\beta
    \end{equation*}
    
    Supongamos además $\beta$ como hipótesis (para poder aplicar reducción al absurso). Entonces:
    \begin{enumerate}
        \item $\beta$ es una hipótesis.
        \item $\beta\rightarrow \neg\alpha$ es una hipótesis.
        \item $\neg\alpha$ por Modus Ponens aplicado a $1$ y $2$.
        \item $\alpha$ es una hipótesis.
    \end{enumerate}

    Por tanto, tenemos que:
    \begin{equation*}
        \{\beta\rightarrow\neg\alpha,\alpha\}\cup \{\beta\} \vdash \alpha\qquad\qquad\{\beta\rightarrow\neg\alpha,\alpha\}\cup \{\beta\} \vdash \neg\alpha
    \end{equation*}

    Por la Regla de Reducción al Absurdo minimal (Ejercicio~\ref{ej:1.6}), tenemos que:
    \begin{equation*}
        \{\beta\rightarrow\neg\alpha,\alpha\} \vdash \neg\beta
    \end{equation*}
\end{ejercicio}

\begin{ejercicio}[Ley de contraposición ``tollendo ponens'']\label{ej:1.13}
    $$\vdash (\neg\alpha \rightarrow \beta) \rightarrow (\neg\beta \rightarrow \alpha).$$
    
    Por el Teorema de la Deducción aplicado dos veces, esto equivale a demostrar:
    \begin{equation*}
        \{\neg\alpha\rightarrow\beta,\neg\beta\} \vdash \alpha
    \end{equation*}

    Supongamos además $\neg \alpha$ como hipótesis (para poder aplicar reducción al absurso). Entonces:
    \begin{enumerate}
        \item $\neg\alpha$ es una hipótesis.
        \item $\neg\alpha\rightarrow \beta$ es una hipótesis.
        \item $\beta$ por Modus Ponens aplicado a $1$ y $2$.
        \item $\neg\beta$ es una hipótesis.
    \end{enumerate}

    Por tanto, tenemos que:
    \begin{equation*}
        \{\neg\alpha\rightarrow\beta,\neg\beta\}\cup \{\neg\alpha\} \vdash \beta\qquad\qquad\{\neg\alpha\rightarrow\beta,\neg\beta\}\cup \{\neg\alpha\} \vdash \neg\beta
    \end{equation*}

    Por la Regla de Reducción al Absurdo, tenemos que:
    \begin{equation*}
        \{\neg\alpha\rightarrow\beta,\neg\beta\} \vdash \alpha
    \end{equation*}
\end{ejercicio}

\begin{ejercicio}[Ley de contraposición débil o ``tollendo tollens'']\label{ej:1.14}
    $$\vdash (\alpha \rightarrow \beta) \rightarrow (\neg\beta \rightarrow \neg\alpha).$$
    
    Por el Teorema de la Deducción aplicado dos veces, esto equivale a demostrar:
    \begin{equation*}
        \{\alpha\rightarrow\beta,\neg\beta\} \vdash \neg\alpha
    \end{equation*}

    Supongamos además $\alpha$ como hipótesis (para poder aplicar reducción al absurso). Entonces:
    \begin{enumerate}
        \item $\alpha$ es una hipótesis.
        \item $\alpha\rightarrow \beta$ es una hipótesis.
        \item $\beta$ por Modus Ponens aplicado a $1$ y $2$.
        \item $\neg\beta$ es una hipótesis.
    \end{enumerate}

    Por tanto, tenemos que:
    \begin{equation*}
        \{\alpha\rightarrow\beta,\neg\beta\}\cup \{\alpha\} \vdash \beta\qquad\qquad\{\alpha\rightarrow\beta,\neg\beta\}\cup \{\alpha\} \vdash \neg\beta
    \end{equation*}

    Por la Regla de Reducción al Absurdo minimal (Ejercicio~\ref{ej:1.6}), tenemos que:
    \begin{equation*}
        \{\alpha\rightarrow\beta,\neg\beta\} \vdash \neg\alpha
    \end{equation*}
\end{ejercicio}

\begin{ejercicio}[Regla de prueba por casos]\label{ej:1.15}
    Si $\Gamma \cup \{\alpha\} \vdash \beta$ y $\Gamma \cup \{\neg\alpha\} \vdash \beta$, entonces $\Gamma \vdash \beta$.\\

    Supongamos (para poder aplicar reducción al absurso) $\neg\beta$ como hipótesis. Entonces, se tiene que:
    \begin{equation*}
        \Gamma \cup \{\alpha\}\cup \{\neg\beta\} \vdash \beta\qquad\qquad\Gamma \cup \{\alpha\}\cup \{\neg\beta\} \vdash \neg\beta
    \end{equation*}
    Por la Regla de Reducción al Absurdo minimal (Ejercicio~\ref{ej:1.6}), tenemos que:
    \begin{equation*}
        \Gamma \cup \{\neg\beta\} \vdash \neg\alpha
    \end{equation*}
    Por otro lado, tenemos que:
    \begin{equation*}
        \Gamma \cup \{\neg\alpha\}\cup \{\neg\beta\} \vdash \beta\qquad\qquad\Gamma \cup \{\neg\alpha\}\cup \{\neg\beta\} \vdash \neg\beta
    \end{equation*}
    Por la Regla de Reducción al Absurdo, tenemos que:
    \begin{equation*}
        \Gamma \cup \{\neg\beta\} \vdash \alpha
    \end{equation*}
    Por tanto, por la Regla de Reducción al Absurdo, tenemos que:
    \begin{equation*}
        \Gamma \vdash \beta
    \end{equation*}
\end{ejercicio}

\begin{ejercicio}[Ley débil de Clavius]\label{ej:1.16}
    $\vdash (\alpha \rightarrow \neg\alpha) \rightarrow \neg\alpha$.\\

    Por el Teorema de la Deducción, esto equivale a demostrar:
    \begin{equation*}
        \{\alpha\rightarrow\neg\alpha\} \vdash \neg\alpha
    \end{equation*}

    Supongamos además $\alpha$ como hipótesis (para poder aplicar reducción al absurso). Entonces:
    \begin{enumerate}
        \item $\alpha$ es una hipótesis.
        \item $\alpha\rightarrow\neg\alpha$ es una hipótesis.
        \item $\neg\alpha$ por Modus Ponens aplicado a $1$ y $2$.
    \end{enumerate}

    Por tanto, tenemos que:
    \begin{equation*}
        \{\alpha\rightarrow\neg\alpha\}\cup \{\alpha\} \vdash \alpha\qquad \qquad 
        \{\alpha\rightarrow\neg\alpha\}\cup \{\alpha\} \vdash \neg\alpha
    \end{equation*}

    Por la Regla de Reducción al Absurdo minimal (Ejercicio~\ref{ej:1.6}), tenemos que:
    \begin{equation*}
        \{\alpha\rightarrow\neg\alpha\} \vdash \neg\alpha
    \end{equation*}
\end{ejercicio}

\begin{ejercicio}[Ley de Clavius]\label{ej:1.17}
    $\vdash (\neg\alpha \rightarrow \alpha) \rightarrow \alpha$.\\

    Por el Teorema de la Deducción, esto equivale a demostrar:
    \begin{equation*}
        \{\neg\alpha\rightarrow\alpha\} \vdash \alpha
    \end{equation*}

    Supongamos además $\neg\alpha$ como hipótesis (para poder aplicar reducción al absurso). Entonces:
    \begin{enumerate}
        \item $\neg\alpha$ es una hipótesis.
        \item $\neg\alpha\rightarrow\alpha$ es una hipótesis.
        \item $\alpha$ por Modus Ponens aplicado a $1$ y $2$.
    \end{enumerate}

    Por tanto, tenemos que:
    \begin{equation*}
        \{\neg\alpha\rightarrow\alpha\}\cup \{\neg\alpha\} \vdash \alpha\qquad \qquad 
        \{\neg\alpha\rightarrow\alpha\}\cup \{\neg\alpha\} \vdash \neg\alpha
    \end{equation*}

    Por la Regla de Reducción al Absurdo, tenemos que:
    \begin{equation*}
        \{\neg\alpha\rightarrow\alpha\} \vdash \alpha
    \end{equation*}
\end{ejercicio}

\begin{ejercicio}[Regla de retorsión, regla de Clavius]\label{ej:1.18}
    Si $\Gamma \cup \{\neg\alpha\} \vdash \alpha$, entonces $\Gamma \vdash \alpha$.\\

    Tenemos que:
    \begin{equation*}
        \Gamma \cup \{\neg\alpha\} \vdash \alpha\qquad\qquad \Gamma \cup \{\neg\alpha\} \vdash \neg\alpha
    \end{equation*}

    Por la Regla de Reducción al Absurdo, tenemos que:
    \begin{equation*}
        \Gamma \vdash \alpha
    \end{equation*}
\end{ejercicio}

\begin{ejercicio}[Leyes de adjunción]\label{ej:1.19}~
    \begin{enumerate}
        \item $\vdash \alpha \rightarrow \alpha \lor \beta$.
        
        Semánticamente, $\alpha\lor\beta \equiv \neg\alpha\rightarrow\beta$. Por tanto, equivale a demostrar:
        \begin{equation*}
            \vdash \alpha\rightarrow(\neg\alpha\rightarrow\beta)
        \end{equation*}

        Esta es precisamente una Ley de Duns Scoto (Ejercicio~\ref{ej:1.7}), por lo que se tiene ya demostrada.
        \item $\vdash \beta \rightarrow \alpha \lor \beta$.
        
        Semánticamente, $\alpha\lor\beta \equiv \neg\alpha\rightarrow\beta$. Por tanto, equivale a demostrar:
        \begin{equation*}
            \vdash \beta\rightarrow(\neg\alpha\rightarrow\beta)
        \end{equation*}

        Esta es cierta por pertenecer a $\cc{A}_1$.
    \end{enumerate}
\end{ejercicio}

\begin{ejercicio}[Reglas de adjunción o de introducción de la disyunción]\label{ej:1.20}~
    \begin{enumerate}
        \item Si $\Gamma \vdash \alpha$, entonces $\Gamma \vdash \alpha \lor \beta$.
        \begin{enumerate}
            \item[$1$.] \ldots\\\vdots
            \item[$p$.] $\alpha$
            \item[$p+1$.] $\alpha\rightarrow \alpha\lor\beta$ por la Ley de Adjunción (Ejercicio~\ref{ej:1.19}).
            \item[$p+2$.] $\alpha\lor\beta$ por Modus Ponens aplicado a $p$ y $p+1$.
        \end{enumerate}

        Como desde $1$ hasta $p$ tan solo se han empleado axiomas e hipótesis de $\Gamma$:
        \begin{equation*}
            \Gamma \vdash \alpha \lor \beta
        \end{equation*}
        \item Si $\Gamma \vdash \beta$, entonces $\Gamma \vdash \alpha \lor \beta$.
        \begin{enumerate}
            \item[$1$.] \ldots\\\vdots
            \item[$p$.] $\beta$
            \item[$p+1$.] $\beta\rightarrow \alpha\lor\beta$ por la Ley de Adjunción (Ejercicio~\ref{ej:1.19}).
            \item[$p+2$.] $\alpha\lor\beta$ por Modus Ponens aplicado a $p$ y $p+1$.
        \end{enumerate}
        
        Como desde $1$ hasta $p$ tan solo se han empleado axiomas e hipótesis de $\Gamma$:
        \begin{equation*}
            \Gamma \vdash \alpha \lor \beta
        \end{equation*}
    \end{enumerate}
\end{ejercicio}

\begin{ejercicio}[Ley conmutativa de la disyunción]\label{ej:1.21}
    $\vdash \alpha \lor \beta \rightarrow \beta \lor \alpha$.\\

    Esto equivale a demostrar:
    \begin{equation*}
        \vdash (\neg\alpha\rightarrow\beta)\rightarrow(\neg\beta\rightarrow\alpha)
    \end{equation*}

    Esta es una de las leyes de contraposición, demostrada en el Ejericio~\ref{ej:1.13}.
\end{ejercicio}

\begin{ejercicio}\label{ej:1.22}~
    \begin{enumerate}
        \item $\vdash \alpha \land \beta \rightarrow \alpha$.
        
        Semánticamente, $\alpha\land\beta \equiv \neg(\alpha\rightarrow\neg\beta)$. Por tanto, equivale a demostrar:
        \begin{equation*}
            \vdash \neg(\alpha\rightarrow\neg\beta)\rightarrow\alpha
        \end{equation*}
        \begin{enumerate}[label=\arabic*.]
            \item $\neg\alpha\rightarrow (\alpha\rightarrow\neg\beta)$ por la Ley Débil de Duns Scoto.
            \item $\left(\neg\alpha\rightarrow (\alpha\rightarrow\neg\beta)\right)\rightarrow \left(\neg(\alpha\rightarrow\neg\beta)\rightarrow \alpha\right)$ por las Leyes de Contraposición.
            \item $\neg(\alpha\rightarrow\neg\beta)\rightarrow \alpha$ por Modus Ponens aplicado a $1$ y $2$.
        \end{enumerate}
        \item $\vdash \alpha \land \beta \rightarrow \beta$.
        
        Semánticamente, $\alpha\land\beta \equiv \neg(\alpha\rightarrow\neg\beta)$. Por tanto, equivale a demostrar:
        \begin{equation*}
            \vdash \neg(\alpha\rightarrow\neg\beta)\rightarrow\beta
        \end{equation*}
        \begin{enumerate}[label=\arabic*.]
            \item $\neg\beta\rightarrow (\alpha\rightarrow\neg\beta)\in \cc{A}_1$.
            \item $\left(\neg\beta\rightarrow (\alpha\rightarrow\neg\beta)\right)\rightarrow \left(\neg(\alpha\rightarrow\neg\beta)\rightarrow \beta\right)$ por las Leyes de Contraposición.
            \item $\neg(\alpha\rightarrow\neg\beta)\rightarrow \beta$ por Modus Ponens aplicado a $1$ y $2$.
        \end{enumerate}

    \end{enumerate}
\end{ejercicio}

\begin{ejercicio}[Reglas de simplificación o de eliminación de la conjunción]\label{ej:1.23}
    Si $\Gamma \vdash \alpha \land \beta$, entonces $\Gamma \vdash \alpha$ y $\Gamma \vdash \beta$.
    \begin{enumerate}
        \item[$1$.] \ldots\\\vdots
        \item[$p$.] $\alpha\land\beta\equiv \neg(\alpha\rightarrow\neg\beta)$.
        \item[$p+1$.] $\neg \alpha\rightarrow(\alpha\rightarrow\neg\beta)$ por la Ley Débil de Duns Scoto (Ejercicio~\ref{ej:1.9}).
        \item[$p+2$.] $[\neg \alpha\rightarrow(\alpha\rightarrow\neg\beta)]\rightarrow[\neg(\alpha\rightarrow\neg\beta)\rightarrow \alpha]$ por las Leyes de Contraposición.
        \item[$p+3$.] $\neg(\alpha\rightarrow\neg\beta)\rightarrow \alpha$ por Modus Ponens aplicado a $p+1$ y $p+2$.
        \item[$p+4$.] $\alpha$ por Modus Ponens aplicado a $p$ y $p+3$.
        \item[$p+5$.] $\neg \beta\rightarrow(\alpha\rightarrow\neg\beta)\in \cc{A}_1$.
        \item[$p+6$.] $[\neg \beta\rightarrow(\alpha\rightarrow\neg\beta)]\rightarrow[\neg(\alpha\rightarrow\neg\beta)\rightarrow \beta]$ por las Leyes de Contraposición.
        \item[$p+7$.] $\neg(\alpha\rightarrow\neg\beta)\rightarrow \beta$ por Modus Ponens aplicado a $p+5$ y $p+6$.
        \item[$p+8$.] $\beta$ por Modus Ponens aplicado a $p$ y $p+7$.
    \end{enumerate}

    Como desde $1$ hasta $p$ tan solo se han empleado axiomas e hipótesis de $\Gamma$:
    \begin{equation*}
        \Gamma \vdash \alpha\qquad\qquad \Gamma \vdash \beta
    \end{equation*}
\end{ejercicio}

\begin{ejercicio}\label{ej:1.24}
    $\vdash (\alpha \rightarrow \gamma) \rightarrow ((\beta \rightarrow \gamma) \rightarrow (\alpha \lor \beta \rightarrow \gamma))$.\\

    Por el Teorema de la Deducción, esto equivale a demostrar:
    \begin{equation*}
        \{\alpha\rightarrow\gamma,\beta\rightarrow\gamma,\alpha\lor\beta\} \vdash \gamma
    \end{equation*}

    Debido a la equivalencia semántica de la disyunción, definiendo el conjunto $\Gamma$ como $\Gamma=\{\alpha\rightarrow\gamma,\beta\rightarrow\gamma,\neg\alpha\rightarrow \beta\}$,    
    esto es equivalente a demostrar $\Gamma \vdash \gamma$.
    \begin{enumerate}[label=\arabic*.]
        \item $\neg\alpha\rightarrow \beta$ es una hipótesis.
        \item $\beta\rightarrow\gamma$ es una hipótesis.
        \item $\neg\alpha\rightarrow\gamma$ por la Regla del Silogismo aplicada a $1$ y $2$.
        \item $\alpha\rightarrow\gamma$ es una hipótesis.
   \end{enumerate}

    Por tanto, y aplicando el Teorema de la Deducción, tenemos que:
    \begin{equation*}
        \Gamma \cup \{\neg\alpha\} \vdash \gamma\qquad\qquad \Gamma \cup \{\alpha\} \vdash \gamma
    \end{equation*}

    Aplicando la Regla de la Prueba por Casos (Ejercicio~\ref{ej:1.15}), tenemos que:
    \begin{equation*}
        \Gamma \vdash \gamma
    \end{equation*}
\end{ejercicio}

\begin{ejercicio}[Otra regla de prueba por casos]\label{ej:1.25}
    Si $\Gamma \cup \{\alpha\} \vdash \gamma$ y $\Gamma \cup \{\beta\} \vdash \gamma$, entonces $\Gamma \cup \{\alpha \lor \beta\} \vdash \gamma$.\\

    Usando que $\Gamma\vdash \alpha\rightarrow\gamma$ y $\Gamma\vdash \beta\rightarrow\gamma$, tenemos que:
    \begin{enumerate}
        \item[$1$.] \ldots\\\vdots
        \item[$p$.] $\alpha\rightarrow\gamma$
        \item[$p+1$.] \ldots\\\vdots
        \item[$q$.] $\beta\rightarrow\gamma$
        \item[$q+1$.] $\neg\alpha\rightarrow\beta$ es una hipótesis.
        \item[$q+2$.] $\neg\alpha\rightarrow\gamma$ por la Regla del Silogismo aplicada a $q+1$ y $q$.
    \end{enumerate}

    Por tanto, y aplicando el Teorema de la Deducción, tenemos que:
    \begin{equation*}
        \Gamma \cup \{\alpha\lor \beta\} \cup \{\alpha\} \vdash \gamma\qquad\qquad \Gamma \cup \{\alpha\lor \beta\} \cup \{\neg\alpha\} \vdash \gamma
    \end{equation*}

    Por la Regla de Prueba por Casos (Ejercicio~\ref{ej:1.15}), tenemos que:
    \begin{equation*}
        \Gamma \cup \{\alpha\lor \beta\} \vdash \gamma
    \end{equation*}
\end{ejercicio}

\begin{ejercicio}[Ley de Peirce]\label{ej:1.26}
    $\vdash ((\alpha \rightarrow \beta) \rightarrow \alpha) \rightarrow \alpha$.\\

    Por el Teorema de la Deducción, esto equivale a demostrar:
    \begin{equation*}
        \{(\alpha\rightarrow\beta)\rightarrow\alpha\} \vdash \alpha
    \end{equation*}

    Supongamos además $\neg \alpha$ como hipótesis (para poder aplicar reducción al absurso). Entonces:
    \begin{enumerate}
        \item $\neg\alpha$ es una hipótesis.
        \item $\neg\alpha\rightarrow(\alpha\rightarrow\beta)$ por la Ley de Duns Scoto (Ejercicio~\ref{ej:1.7}).
        \item $\alpha\rightarrow\beta$ por Modus Ponens aplicado a $1$ y $2$.
        \item $(\alpha\rightarrow\beta)\rightarrow\alpha$ es una hipótesis.
        \item $\alpha$ por Modus Ponens aplicado a $3$ y $4$.
    \end{enumerate}

    Por tanto, tenemos que:
    \begin{equation*}
        \{(\alpha\rightarrow\beta)\rightarrow\alpha\} \cup \{\neg\alpha\} \vdash \alpha\qquad\qquad \{(\alpha\rightarrow\beta)\rightarrow\alpha\} \cup \{\neg\alpha\} \vdash \neg\alpha
    \end{equation*}

    Por la Regla de Reducción al Absurdo, tenemos que:
    \begin{equation*}
        \{(\alpha\rightarrow\beta)\rightarrow\alpha\} \vdash \alpha
    \end{equation*}
\end{ejercicio}

\begin{ejercicio}\label{ej:1.27}
    $\vdash \alpha \rightarrow (\beta \rightarrow \alpha \land \beta)$.\\

    Por la equivalencia semántica de la conjunción y el Teorema de la Deducción, esto equivale a demostrar:
    \begin{equation*}
        \{\alpha,\beta\} \vdash \neg(\alpha\rightarrow\neg\beta)
    \end{equation*}

    Supongamos además $\alpha\rightarrow\neg\beta$ como hipótesis (para poder aplicar reducción al absurso). Entonces:
    \begin{enumerate}
        \item $\alpha$ es una hipótesis.
        \item $\beta$ es una hipótesis.
        \item $\alpha\rightarrow\neg\beta$ es una hipótesis.
        \item $\neg\beta$ por Modus Ponens aplicado a $1$ y $3$.
    \end{enumerate}

    Por tanto, tenemos que:
    \begin{equation*}
        \{\alpha,\beta\}\cup \{\alpha\rightarrow\neg\beta\} \vdash \neg\beta\qquad\qquad \{\alpha,\beta\}\cup \{\alpha\rightarrow\neg\beta\} \vdash \beta
    \end{equation*}

    Por la Regla de Reducción al Absurdo minimal (Ejercicio~\ref{ej:1.6}), tenemos que:
    \begin{equation*}
        \{\alpha,\beta\} \vdash \neg(\alpha\rightarrow\neg\beta)
    \end{equation*}
\end{ejercicio}

\begin{ejercicio}[Regla del producto o de introducción de la conjunción]\label{ej:1.28}
    Si $\Gamma \vdash \alpha$ y $\Gamma \vdash \beta$, entonces $\Gamma \vdash \alpha \land \beta$.
    \begin{enumerate}
        \item[$1$.] \ldots\\\vdots
        \item[$p$.] $\alpha$
        \item[$p+1$.] \ldots\\\vdots
        \item[$q$.] $\beta$
        \item[$q+1$.] $\alpha\rightarrow (\beta\rightarrow \alpha \land \beta)$ por el Ejercicio~\ref{ej:1.27}.
        \item[$q+2$.] $\beta\rightarrow \alpha \land \beta$ por Modus Ponens aplicado a $p$ y $q+1$.
        \item[$q+3$.] $\alpha \land \beta$ por Modus Ponens aplicado a $q$ y $q+2$.
    \end{enumerate}

    Como desde $1$ hasta $q$ tan solo se han empleado axiomas e hipótesis de $\Gamma$:
    \begin{equation*}
        \Gamma \vdash \alpha\land\beta
    \end{equation*}
\end{ejercicio}

    \section{Lógica de Primer Orden}

\begin{ejercicio}\label{ej:2.1}
    Prueba que $\{\forall x(P(x) \rightarrow Q(x)), \neg Q(a)\} \models \neg P(a)$.\\

    Sea $(\veps,v)$ una $\cc{L}$-interpretación, verificando:
    \begin{align*}
        I^v(\forall x(P(x) \rightarrow Q(x))) &= 1,\\
        I^v(\neg Q(a)) &= 1.
    \end{align*}

    Por hipótesis, $I^v(\forall x(P(x) \rightarrow Q(x))) =~1$. En particular, para $a$ tenemos:
    \begin{multline*}
        1 = I^{v(x\mid a)}(P(x) \rightarrow Q(x))
        = 1+I^{v(x\mid a)}(P(x))+I^{v(x\mid a)}(Q(x))=1+P(a)+Q(a)
        \Longrightarrow \\ \Longrightarrow P(a) = Q(a).
    \end{multline*}
    
    
    Por otro lado, tenemos:
    \begin{align*}
        1 = I^v(\neg Q(a)) &= 1+I^v(Q(a))\Longrightarrow I^v(Q(a)) = 0
        = Q(v(a)) = Q(a)
    \end{align*}

    Por tanto:
    \begin{align*}
        I^v(\neg P(a)) &= 1+I^v(P(a)) = 1+P(v(a)) = 1+P(a) = 1+Q(a) = 1.
    \end{align*}

    Por tanto, $\{\forall x(P(x) \rightarrow Q(x)), \neg Q(a)\} \models \neg P(a)$.
\end{ejercicio}

\begin{ejercicio}\label{ej:2.2}
    Dada $\cc{U}$ una $\cc{L}$-estructura y $\varphi$ una sentencia, razona si equivalen el que $\varphi$ sea satisfacible y que sea válida en $\cc{U}$
    \begin{observacion}
        Usar el lema de coincidencia.
    \end{observacion}

    Demostraremos por doble implicación:
    \begin{itemize}
        \item[$\Longrightarrow)$] Supongamos que $\varphi$ es satisfacible en $\cc{U}$. Entonces existe una $\cc{L}$-interpretación $(\veps,v)$ tal que $I^v(\varphi) = 1$. No obstante, por tratarse de una sentencia, y usando el lema de coincidencia, tenemos que:
        \begin{equation*}
            1 = I^v(\varphi) = I^{w}(\varphi)
        \end{equation*}
        para cualquier otra $\cc{L}$-interpretación $(\veps,w)$. Por tanto, $\varphi$ es válida en $\cc{U}$.

        \item[$\Longleftarrow)$] Supongamos que $\varphi$ es válida en $\cc{U}$. Entonces para cualquier $\cc{L}$-interpretación $(\veps,w)$, tenemos que $I^v(\varphi) = 1$. En particular existe una $\cc{L}$-interpretación $(\veps,v)$ tal que $I^v(\varphi) = 1$. Por tanto, $\varphi$ es satisfacible en $\cc{U}$.
    \end{itemize}
\end{ejercicio}

\begin{ejercicio}\label{ej:2.3}
    Usando el Ejercicio~\ref{ej:2.2}, y bajo las mismas hipótesis, prueba que o bien $\varphi$ es válida o bien lo es $\neg \varphi$, pero no se pueden dar las dos posibilidades.\\

    Sea ahora una $\cc{L}-$interpretación $(\cc{U},v)$. Hay dos posibilidades:
    \begin{itemize}
        \item \ul{$I^v(\varphi) = 1$}. Entonces, $\varphi$ es satisfacible en $\cc{U}$, y por el Ejercicio~\ref{ej:2.2} $\varphi$ es válida en $\cc{U}$. Por tanto, $\neg \varphi$ no es válida en $\cc{U}$.
        
        \item \ul{$I^v(\varphi) = 0$}. Entonces, $\neg \varphi$ es satisfacible en $\cc{U}$, y por el Ejercicio~\ref{ej:2.2} $\neg \varphi$ es válida en $\cc{U}$. Por tanto, $\varphi$ no es válida en $\cc{U}$.
    \end{itemize}
\end{ejercicio}

\begin{ejercicio}\label{ej:2.4}
    Sea $\cc{U}$ una $\cc{L}$-estructura. Sea $\varphi \in Form(\cc{L})$. Si $x_1, \ldots, x_n$ son todas las variables de $\varphi$ con alguna ocurrencia libre, entonces equivalen:
    \begin{enumerate}
        \item $\varphi$ es válida en $\cc{U}$,
        \item $\forall x_1 \ldots \forall x_n \varphi$ es satisfacible en $\cc{U}$,
        \item $\forall x_1 \ldots \forall x_n \varphi$ es válida en $\cc{U}$.
    \end{enumerate}

    Demostraremos distintas implicaciones:
    \begin{description}
        \item[$(1)\Longrightarrow(2)$] Supongamos que $\varphi$ es válida en $\cc{U}$. Sea ahora una $\cc{L}$-interpretación $(\veps,v)$ fijada. Tenemos que:
        \begin{equation*}
            \forall x_1 \ldots \forall x_n I^v(\varphi) = 1\iff \forall a_1,\ldots,\forall a_n\in D,\qquad I^{v(x_1\mid a_1,\ldots,x_n\mid a_n)}(\varphi) = 1
        \end{equation*}
        No obstante, esto se tiene que es cierto, puesto que $\varphi$ es válida en $\cc{U}$ y $I^{v(x_1\mid a_1,\ldots,x_n\mid a_n)}$ es una $\cc{L}$-interpretación. Por tanto, $\forall x_1 \ldots \forall x_n \varphi$ es satisfacible en $\cc{U}$.

        \item[$(2)\Longrightarrow(3)$] $\forall x_1 \ldots \forall x_n \varphi$ es una sentencia, ya que estamos cuantificando todas las variables libres de $\varphi$. Por tanto, por el Ejercicio~\ref{ej:2.2}, se tiene la implicación.
        
        \item[$(3)\Longrightarrow(1)$] Supongamos que $\forall x_1 \ldots \forall x_n \varphi$ es válida en $\cc{U}$. Sea ahora una $\cc{L}$-interpretación $(\veps,v)$ fijada. Sea ahora $a_i=v(x_i)\in D$ para $i=1,\ldots,n$. Por ser $\forall x_1 \ldots \forall x_n \varphi$ válida en $\cc{U}$, tenemos que:
        \begin{equation*}
            I^{v(x_1\mid a_1,\ldots,x_n\mid a_n)}(\varphi) = 1.
        \end{equation*}

        Además, tenemos que:
        \begin{equation*}
            v(x_i) = v(x_1\mid a_1,\ldots,x_n\mid a_n)(x_i) = a_i\qquad i=1,\ldots,n.
        \end{equation*}
        Por tanto, ambas asignaciones coinciden en las variables libres de $\varphi$. Por el Lema de Coincidencia, tenemos que:
        \begin{equation*}
            1 = I^{v(x_1\mid a_1,\ldots,x_n\mid a_n)}(\varphi) = I^v(\varphi).
        \end{equation*}

        Por último, como esto era para cualquier $\cc{L}$-interpretación $(\veps,v)$, se tiene que $\varphi$ es válida en $\cc{U}$.
    \end{description}
\end{ejercicio}


\begin{ejercicio}\label{ej:2.5}
    Sea $\cc{U}$ una $\cc{L}$-estructura. Sea $\varphi \in Form(\cc{L})$. Si $x_1, \ldots, x_n$ son todas las variables de $\varphi$ con alguna ocurrencia libre, entonces equivalen:
    \begin{enumerate}
        \item $\varphi$ es satisfacible en $\cc{U}$,
        \item $\exists x_1 \ldots \exists x_n \varphi$ es satisfacible en $\cc{U}$.
    \end{enumerate}
    ¿Es también cierta la equivalencia cambiando en el segundo apartado satisfacible por válida?\\

    Demostramos las distintas implicaciones:
    \begin{description}
        \item[$(1)\Longrightarrow(2)$] Supongamos que $\varphi$ es satisfacible en $\cc{U}$. Entonces existe una $\cc{L}$-interpretación $(\veps,v)$ tal que $I^v(\varphi) = 1$. Consideramos $a_i=v(x_i)\in D$ para $i=1,\ldots,n$. Por tanto, por el Lema de Coincidencia, tenemos que:
        \begin{equation*}
            I^{v(x_1\mid a_1,\ldots,x_n\mid a_n)}(\varphi) = 1.
        \end{equation*}

        Por tanto, $\exists x_1 \ldots \exists x_n \varphi$ es satisfacible en $\cc{U}$.

        \item[$(2)\Longrightarrow(1)$] Supongamos que $\exists x_1 \ldots \exists x_n \varphi$ es satisfacible en $\cc{U}$. Entonces existen $a_i\in D$ para $i=1,\ldots,n$ y una $\cc{L}$-interpretación $(\veps,v)$ tal que:
        \begin{equation*}
            I^{v(x_1\mid a_1,\ldots,x_n\mid a_n)}(\varphi) = 1.
        \end{equation*}

        Por tanto, consideramos una $\cc{L}$-interpretación $(\veps,w)$ tal que $w(x_i) = a_i$ para $i=1,\ldots,n$. Por el Lema de Coincidencia, tenemos que:
        \begin{equation*}
            I^{v(x_1\mid a_1,\ldots,x_n\mid a_n)}(\varphi) = I^w(\varphi) = 1.
        \end{equation*}

        Por tanto, $\varphi$ es satisfacible en $\cc{U}$.
    \end{description}

    Por último, si cambiamos satisfacible por válida en el segundo apartado, la equivalencia sigue siendo cierta, puesto que:
    \begin{equation*}
        \exists x_1 \ldots \exists x_n \varphi\text{ es válida en }\cc{U}\Longrightarrow
        \exists x_1 \ldots \exists x_n \varphi\text{ es satisfacible en }\cc{U}.
    \end{equation*}
        
\end{ejercicio}

\begin{ejercicio}\label{ej:2.6}
    Demuestra que:
    \begin{enumerate}
        \item $\models \neg \forall x \psi \leftrightarrow \exists x \neg \psi$,
        % // TODO: Puedo negar así por la cara??
        
        Sea $(\veps,v)$ una $\cc{L}$-interpretación. Por definición:
        \begin{align*}
            1 = I^v(\exists x \neg \psi)
            &\iff \exists a\in D,\qquad I^{v(x\mid a)}(\neg \psi) = 1\\
            &\iff \exists a\in D,\qquad 1+I^{v(x\mid a)}(\psi) = 1\\
            &\iff \exists a\in D,\qquad I^{v(x\mid a)}(\psi) = 0\\\\
            1 = I^v(\neg \forall x \psi)
            &\iff 1+I^v(\forall x \psi) = 1\\
            &\iff I^{v}( \forall x \psi) = 0\\
            &\iff \exists a\in D,\qquad I^{v(x\mid a)}(\psi) = 0.
        \end{align*}

        Por tanto, y puesto que trabajamos en $\bb{Z}_2$, hemos probado que:
        \begin{equation*}
            I^v(\neg \forall x \psi) =I^v(\exists x \neg \psi).
        \end{equation*}

        Por tanto:
        \begin{align*}
            I( \neg \forall x \psi \leftrightarrow \exists x \neg \psi)
            &= 1+I(\neg \forall x \psi)+I(\exists x \neg \psi)=1
        \end{align*}
        \item $\models \neg \exists x \psi \leftrightarrow \forall x \neg \psi$,
        
        Sea $(\veps,v)$ una $\cc{L}$-interpretación. Por definición:
        \begin{align*}
            1 = I^v(\forall x \neg \psi)
            &\iff \forall a\in D,\qquad I^{v(x\mid a)}(\neg \psi) = 1\\
            &\iff \forall a\in D,\qquad 1+I^{v(x\mid a)}(\psi) = 1\\
            &\iff \forall a\in D,\qquad I^{v(x\mid a)}(\psi) = 0\\\\
            1 = I^v(\neg \exists x \psi)
            &\iff 1+I^v(\exists x \psi) = 1\\
            &\iff I^{v}( \exists x \psi) = 0\\
            &\iff \forall a\in D,\qquad I^{v(x\mid a)}(\psi) = 0.
        \end{align*}

        Por tanto, y puesto que trabajamos en $\bb{Z}_2$, hemos probado que:
        \begin{equation*}
            I^v(\neg \exists x \psi) =I^v(\forall x \neg \psi).
        \end{equation*}

        Por tanto:
        \begin{align*}
            I( \neg \exists x \psi \leftrightarrow \forall x \neg \psi)
            &= 1+I(\neg \exists x \psi)+I(\forall x \neg \psi)=1
        \end{align*}
        \item $\models \exists x \psi \leftrightarrow \neg \forall x \neg \psi$,
        
        Sea $(\veps,v)$ una $\cc{L}$-interpretación. Por definición:
        \begin{align*}
            1 = I^v(\neg \forall x \neg \psi)
            &\iff 1+I^v(\forall x \neg \psi) = 1\\
            &\iff I^{v}( \forall x \neg \psi) = 0\\
            &\iff \exists a\in D,\qquad I^{v(x\mid a)}(\neg \psi) = 0\\
            &\iff \exists a\in D,\qquad 1+I^{v(x\mid a)}(\psi) = 0\\
            &\iff \exists a\in D,\qquad I^{v(x\mid a)}(\psi) = 1\\\\
            1 = I^v(\exists x \psi)
            &\iff \exists a\in D,\qquad I^{v(x\mid a)}(\psi) = 1.
        \end{align*}

        Por tanto, y puesto que trabajamos en $\bb{Z}_2$, hemos probado que:
        \begin{equation*}
            I^v(\exists x \psi) =I^v(\neg \forall x \neg \psi).
        \end{equation*}

        Por tanto:
        \begin{align*}
            I( \exists x \psi \leftrightarrow \neg \forall x \neg \psi)
            &= 1+I(\exists x \psi)+I(\neg \forall x \neg \psi)=1
        \end{align*}
        \item $\models \forall x \psi \leftrightarrow \neg \exists x \neg \psi$,
        
        Sea $(\veps,v)$ una $\cc{L}$-interpretación. Por definición:
        \begin{align*}
            1 = I^v(\neg \exists x \neg \psi)
            &\iff 1+I^v(\exists x \neg \psi) = 1\\
            &\iff I^{v}( \exists x \neg \psi) = 0\\
            &\iff \forall a\in D,\qquad I^{v(x\mid a)}(\neg \psi) = 0\\
            &\iff \forall a\in D,\qquad 1+I^{v(x\mid a)}(\psi) = 0\\
            &\iff \forall a\in D,\qquad I^{v(x\mid a)}(\psi) = 1\\\\
            1 = I^v(\forall x \psi)
            &\iff \forall a\in D,\qquad I^{v(x\mid a)}(\psi) = 1.
        \end{align*}

        Por tanto, y puesto que trabajamos en $\bb{Z}_2$, hemos probado que:
        \begin{equation*}
            I^v(\forall x \psi) =I^v(\neg \exists x \neg \psi).
        \end{equation*}

        Por tanto:
        \begin{align*}
            I( \forall x \psi \leftrightarrow \neg \exists x \neg \psi)
            &= 1+I(\forall x \psi)+I(\neg \exists x \neg \psi)=1
        \end{align*}
        \item $\models \forall x \psi \wedge \varphi \leftrightarrow \forall x(\psi \wedge \varphi)$, si $x$ no aparece libre en $\varphi$,
        
        Sea $(\veps,v)$ una $\cc{L}$-interpretación. Por definición:
        \begin{align*}
            1 = I^v(\forall x(\psi \wedge \varphi))
            &\iff \forall a\in D,\qquad I^{v(x\mid a)}(\psi \wedge \varphi) = 1\\
            &\iff \forall a\in D,\qquad I^{v(x\mid a)}(\psi)I^{v(x\mid a)}(\varphi) = 1\\
            &\iff \forall a\in D,\qquad I^{v(x\mid a)}(\psi) = 1\text{ y }I^{v(x\mid a)}(\varphi) = 1\\
            &\stackrel{(\ast)}{\iff} \forall a\in D,\qquad I^{v(x\mid a)}(\psi) = 1\text{ y }I^{v}(\varphi) = 1\\
            &\iff I^v(\forall x\psi)=1\text{ y }I^v(\varphi)=1\\
            &\iff I^v(\forall x\psi\wedge\varphi)=1.
        \end{align*}
        donde $(\ast)$ se debe a que $x$ no aparece libre en $\varphi$ y $v$ y $v(x\mid a)$ tan solo difieren en el valor de $x$.
        Por el Lema de Coincidencia, $I^{v(x\mid a)}(\varphi) = I^v(\varphi)$ para cualquier $a\in D$.
        Por tanto:
        \begin{equation*}
            I^v(\forall x \psi \wedge \varphi) = I^v(\forall x(\psi \wedge \varphi)).
        \end{equation*}

        Por tanto:
        \begin{align*}
            I( \forall x \psi \wedge \varphi \leftrightarrow \forall x(\psi \wedge \varphi))
            &= 1+I(\forall x \psi \wedge \varphi)+I(\forall x(\psi \wedge \varphi))=1
        \end{align*}
        \item $\models \exists x \psi \wedge \varphi \leftrightarrow \exists x(\psi \wedge \varphi)$, si $x$ no aparece libre en $\varphi$,
        
        Sea $(\veps,v)$ una $\cc{L}$-interpretación. Por definición:
        \begin{align*}
            1 = I^v(\exists x(\psi \wedge \varphi))
            &\iff \exists a\in D,\qquad I^{v(x\mid a)}(\psi \wedge \varphi) = 1\\
            &\iff \exists a\in D,\qquad I^{v(x\mid a)}(\psi)I^{v(x\mid a)}(\varphi) = 1\\
            &\iff \exists a\in D,\qquad I^{v(x\mid a)}(\psi) = 1\text{ y }I^{v(x\mid a)}(\varphi) = 1\\
            &\stackrel{(\ast)}{\iff} \exists a\in D,\qquad I^{v(x\mid a)}(\psi) = 1\text{ y }I^{v}(\varphi) = 1\\
            &\iff I^v(\exists x\psi)=1\text{ y }I^v(\varphi)=1\\
            &\iff I^v(\exists x\psi\wedge\varphi)=1.
        \end{align*}
        donde $(\ast)$ se debe a que $x$ no aparece libre en $\varphi$ y $v$ y $v(x\mid a)$ tan solo difieren en el valor de $x$. Por el Lema de Coincidencia, $I^{v(x\mid a)}(\varphi) = I^v(\varphi)$ para cualquier $a\in D$.
        Por tanto:
        \begin{equation*}
            I^v(\exists x \psi \wedge \varphi) = I^v(\exists x(\psi \wedge \varphi)).
        \end{equation*}

        Por tanto:
        \begin{align*}
            I( \exists x \psi \wedge \varphi \leftrightarrow \exists x(\psi \wedge \varphi))
            &= 1+I(\exists x \psi \wedge \varphi)+I(\exists x(\psi \wedge \varphi))=1
        \end{align*}
        \item $\models \forall x \psi \vee \varphi \leftrightarrow \forall x(\psi \vee \varphi)$, si $x$ no aparece libre en $\varphi$,
        
        Sea $(\veps,v)$ una $\cc{L}$-interpretación. Por definición:
        \begin{align*}
            1 = I^v(\forall x(\psi \vee \varphi))
            &\iff \forall a\in D,\qquad I^{v(x\mid a)}(\psi \vee \varphi) = 1\\
            &\iff \forall a\in D,\qquad I^{v(x\mid a)}(\psi) + I^{v(x\mid a)}(\varphi) + I^{v(x\mid a)}(\psi)I^{v(x\mid a)}(\varphi) = 1\\
            &\stackrel{(\ast)}{\iff} \forall a\in D,\qquad I^{v(x\mid a)}(\psi) + I^{v}(\varphi) + I^{v(x\mid a)}(\psi)I^{v}(\varphi) = 1\\
            1 = I^v(\forall x \psi \vee \varphi)
            &\iff I^v(\forall x \psi \vee \varphi) = 1\\
            &\iff I^v(\forall x \psi) + I^v(\varphi) + I^v(\forall x \psi)I^v(\varphi) = 1
        \end{align*}
        donde en $(\ast)$ hemos usado que $x$ no aparece libre en $\varphi$ y $v$ y $v(x\mid a)$ tan solo difieren en el valor de $x$. Por el Lema de Coincidencia, $I^{v(x\mid a)}(\varphi) = I^v(\varphi)$ para cualquier $a\in D$.

        Veamos que se da la equivalencia:
        \begin{itemize}
            \item Si $I^v(\varphi)=1$, entonces:
            \begin{align*}
                1 = I^v(\forall x(\psi \vee \varphi)) &\iff \forall a\in D,\qquad I^{v(x\mid a)}(\psi) + 1 + I^{v(x\mid a)}(\psi) = 1\\
                &\iff \forall a\in D, \ 0=0\\
                1 = I^v(\forall x \psi \vee \varphi)
                &\iff I^v(\forall x \psi) + 1 + I^v(\forall x \psi) = 1 \iff 0=0
            \end{align*}
            Por tanto, en este caso se da $I^v(\forall x \psi \vee \varphi) = I^v(\forall x(\psi \vee \varphi))$.\\

            \item Si $I^v(\varphi)=0$, entonces:
            \begin{align*}
                1 = I^v(\forall x(\psi \vee \varphi)) &\iff \forall a\in D,\qquad I^{v(x\mid a)}(\psi) + 0 + 0= 1\\
                &\iff I^v(\forall x \psi) = 1\\
                &\iff I^v(\forall x \psi) + 0 + 0 = 1\\
                &\iff 1 = I^v(\forall x \psi \vee \varphi)
            \end{align*}
            En este caso, también se da $I^v(\forall x \psi \vee \varphi) = I^v(\forall x(\psi \vee \varphi))$.
        \end{itemize}

        Por tanto, hemos probado que:
        \begin{equation*}
            I^v(\forall x \psi \vee \varphi) = I^v(\forall x(\psi \vee \varphi)).
        \end{equation*}

        Por tanto:
        \begin{align*}
            I( \forall x \psi \vee \varphi \leftrightarrow \forall x(\psi \vee \varphi))
            &= 1+I(\forall x \psi \vee \varphi)+I(\forall x(\psi \vee \varphi))=1
        \end{align*}

        \item $\models \exists x \psi \vee \varphi \leftrightarrow \exists x(\psi \vee \varphi)$, si $x$ no aparece libre en $\varphi$,
        
        Sea $(\veps,v)$ una $\cc{L}$-interpretación. Por definición:
        \begin{align*}
            1 = I^v(\exists x(\psi \vee \varphi))
            &\iff \exists a\in D,\qquad I^{v(x\mid a)}(\psi \vee \varphi) = 1\\
            &\iff \exists a\in D,\qquad I^{v(x\mid a)}(\psi) + I^{v(x\mid a)}(\varphi) + I^{v(x\mid a)}(\psi)I^{v(x\mid a)}(\varphi) = 1\\
            &\stackrel{(\ast)}{\iff} \exists a\in D,\qquad I^{v(x\mid a)}(\psi) + I^{v}(\varphi) + I^{v(x\mid a)}(\psi)I^{v}(\varphi) = 1\\
            1 = I^v(\exists x \psi \vee \varphi)
            &\iff I^v(\exists x \psi \vee \varphi) = 1\\
            &\iff I^v(\exists x \psi) + I^v(\varphi) + I^v(\exists x \psi)I^v(\varphi) = 1
        \end{align*}
        donde en $(\ast)$ hemos usado que $x$ no aparece libre en $\varphi$ y $v$ y $v(x\mid a)$ tan solo difieren en el valor de $x$. Por el Lema de Coincidencia, $I^{v(x\mid a)}(\varphi) = I^v(\varphi)$ para cualquier $a\in D$.
        Veamos que se da la equivalencia:
        \begin{itemize}
            \item Si $I^v(\varphi)=1$, entonces:
            \begin{align*}
                1 = I^v(\exists x(\psi \vee \varphi)) &\iff \exists a\in D,\qquad I^{v(x\mid a)}(\psi) + 1 + I^{v(x\mid a)}(\psi) = 1\\
                &\iff \exists a\in D, \ 0=0\\
                1 = I^v(\exists x \psi \vee \varphi)
                &\iff I^v(\exists x \psi) + 1 + I^v(\exists x \psi) = 1 \iff 0=0
            \end{align*}
            Por tanto, en este caso se da $I^v(\exists x \psi \vee \varphi) = I^v(\exists x(\psi \vee \varphi))$.\\

            \item Si $I^v(\varphi)=0$, entonces:
            \begin{align*}
                1 = I^v(\exists x(\psi \vee \varphi)) &\iff \exists a\in D,\qquad I^{v(x\mid a)}(\psi) + 0 + 0= 1\\
                &\iff I^v(\exists x \psi) = 1\\
                &\iff I^v(\exists x \psi) + 0 + 0 = 1\\
                &\iff 1 = I^v(\exists x \psi \vee \varphi)
            \end{align*}
            En este caso, también se da $I^v(\exists x \psi \vee \varphi) = I^v(\exists x(\psi \vee \varphi))$.
        \end{itemize}
        
        Por tanto, hemos probado que:
        \begin{equation*}
            I^v(\exists x \psi \vee \varphi) = I^v(\exists x(\psi \vee \varphi)).
        \end{equation*}

        Por tanto:
        \begin{align*}
            I( \exists x \psi \vee \varphi \leftrightarrow \exists x(\psi \vee \varphi))
            &= 1+I(\exists x \psi \vee \varphi)+I(\exists x(\psi \vee \varphi))=1
        \end{align*}
        \item $\models \forall x \psi \wedge \forall x \varphi \leftrightarrow \forall x(\psi \wedge \varphi)$,
        
        Sea $(\veps,v)$ una $\cc{L}$-interpretación. Por definición:
        \begin{align*}
            1 = I^v(\forall x(\psi \wedge \varphi))
            &\iff \forall a\in D,\qquad I^{v(x\mid a)}(\psi \wedge \varphi) = 1\\
            &\iff \forall a\in D,\qquad I^{v(x\mid a)}(\psi)I^{v(x\mid a)}(\varphi) = 1\\
            &\iff \forall a\in D,\qquad I^{v(x\mid a)}(\psi) = 1\text{ y }I^{v(x\mid a)}(\varphi) = 1\\
            &\iff \forall a_1,a_2\in D,\qquad I^{v(x\mid a_1)}(\psi) = 1\text{ y }I^{v(x\mid a_2)}(\varphi) = 1\\
            &\iff I^v(\forall x\psi)=1\text{ y }I^v(\forall x\varphi)=1\\
            &\iff I^v(\forall x\psi)I^v(\forall x\varphi)=1\\
            &\iff I^v(\forall x\psi\wedge\forall x\varphi)=1.
        \end{align*}

        Por tanto, hemos probado que:
        \begin{equation*}
            I^v(\forall x \psi \wedge \forall x \varphi) = I^v(\forall x(\psi \wedge \varphi)).
        \end{equation*}

        Por tanto:
        \begin{align*}
            I( \forall x \psi \wedge \forall x \varphi \leftrightarrow \forall x(\psi \wedge \varphi))
            &= 1+I(\forall x \psi \wedge \forall x \varphi)+I(\forall x(\psi \wedge \varphi))=1
        \end{align*}
        \item $\models \exists x \psi \vee \exists x \varphi \leftrightarrow \exists x(\psi \vee \varphi)$,
        
        Sea $(\veps,v)$ una $\cc{L}$-interpretación. Por definición:
        \begin{align*}
            1 = I^v(\exists x(\psi \vee \varphi))
            &\iff \exists a\in D,\qquad I^{v(x\mid a)}(\psi \vee \varphi) = 1\\
            &\iff \exists a\in D,\qquad I^{v(x\mid a)}(\psi) + I^{v(x\mid a)}(\varphi) + I^{v(x\mid a)}(\psi)I^{v(x\mid a)}(\varphi) = 1\\
            1 = I^v(\exists x \psi \vee \exists x \varphi)
            &\iff I^v(\exists x \psi) + I^v(\exists x \varphi) + I^v(\exists x \psi)I^v(\exists x \varphi) = 1
        \end{align*}

        Veamos que se da la equivalencia:
        \begin{itemize}
            \item Si $I^v(\exists x \psi)=1$:
            
            En este caso, $\exists a\in D$ tal que $I^{v(x\mid a)}(\psi) = 1$. Por tanto, se tiene que:
            \begin{align*}
                \hspace{-2cm}&I^v(\exists x \psi \vee \exists x \varphi) = \cancelto{1}{I^v(\exists x \psi)} + I^v(\exists x \varphi) + \cancelto{1}{I^v(\exists x \psi)}I^v(\exists x \varphi) = 1\\
                \hspace{-2cm}\cancelto{1}{I^{v(x\mid a)}(\psi)} +& I^{v(x\mid a)}(\varphi) + \cancelto{1}{I^{v(x\mid a)}(\psi)}I^{v(x\mid a)}(\varphi)=1
                \Longrightarrow 1 = I^{v(x\mid a)}(\psi \vee \varphi)
                \Longrightarrow I^v(\exists x(\psi \vee \varphi)) = 1.
            \end{align*}

            Por tanto, en este caso se da $I^v(\exists x \psi \vee \exists x \varphi) = I^v(\exists x(\psi \vee \varphi))$.

            \item Si $I^v(\exists x \psi)=0$:
            
            En este caso, $\nexists a\in D$ tal que $I^{v(x\mid a)}(\psi) = 1$. Por tanto, se tiene que:
            \begin{align*}
                1 = I^v(\exists x(\psi \vee \varphi))
                &\iff \exists a\in D,\qquad 0+ I^{v(x\mid a)}(\varphi) + 0 = 1\\
                &\iff I^v(\exists x \varphi) = 1\\
                1 = I^v(\exists x \psi \vee \exists x \varphi)
                &\iff 0 + I^v(\exists x \varphi) + 0 = 1\iff I^v(\exists x \varphi) = 1
            \end{align*}

            De nuevo, se da $I^v(\exists x \psi \vee \exists x \varphi) = I^v(\exists x(\psi \vee \varphi))$.
        \end{itemize}

        En cualquier caso, hemos probado que:
        \begin{equation*}
            I^v(\exists x \psi \vee \exists x \varphi) = I^v(\exists x(\psi \vee \varphi)).
        \end{equation*}

        Por tanto:
        \begin{align*}
            I( \exists x \psi \vee \exists x \varphi \leftrightarrow \exists x(\psi \vee \varphi))
            &= 1+I(\exists x \psi \vee \exists x \varphi)+I(\exists x(\psi \vee \varphi))=1
        \end{align*}


        \item $\models \forall x \varphi(x) \leftrightarrow \forall y \varphi(y)$, $y$ variable que no aparece en $\forall x \varphi(x)$,
        
        Sea $(\veps,v)$ una $\cc{L}$-interpretación. Por definición:
        \begin{align*}
            1 = I^v(\forall x \varphi(x))
            &\iff \forall a\in D,\qquad I^{v(x\mid a)}(\varphi(x)) = 1\\
            &\stackrel{(\ast)}{\iff} \forall a\in D,\qquad I^{v(y\mid a)}(\varphi(y)) = 1\\
            &\iff I^v(\forall y \varphi(y)) = 1.
        \end{align*}
        donde vamos ahora a argumentar el paso dado en $(\ast)$. Los únicos valores en los que difieren $v(x\mid a)$ y $v(y\mid a)$ son en el valor de $x$ y $y$. Vamos a estudiar qué ocurre:
        \begin{itemize}
            \item Como $y$ no aparece en $\forall x \varphi(x)$, entonces $v(x\mid a)(x)$ no es relevante.
            \item Como, tras calcular $\varphi(y)$ se sustituyen todas las ocurrencias libres de $x$ por $y$, entonces $v(y\mid a)(x)$ tampoco es relevante (las ocurrencias ligadas, por el Lema de Coincidencia, no supondrán problema).
            \item En el caso de que haya más variables libres en $\varphi$, estas serán distintas a $y$ y $x$, por lo que, por el Lema de Coincidencia, no supondrán problema.
            \item Sabemos que $v(x\mid a)(x) = v(y\mid a)(y)=a$.
        \end{itemize}
        Por tanto, hemos probado que:
        \begin{equation*}
            I^v(\forall x \varphi(x)) = I^v(\forall y \varphi(y)).
        \end{equation*}

        Por tanto:
        \begin{align*}
            I( \forall x \varphi(x) \leftrightarrow \forall y \varphi(y))
            &= 1+I(\forall x \varphi(x))+I(\forall y \varphi(y))=1
        \end{align*}
        \item $\models \exists x \varphi(x) \leftrightarrow \exists y \varphi(y)$, $y$ variable que no aparece en $\forall x \varphi(x)$.
        
        
        Sea $(\veps,v)$ una $\cc{L}$-interpretación. Por definición:
        \begin{align*}
            1 = I^v(\exists x \varphi(x))
            &\iff \exists a\in D,\qquad I^{v(x\mid a)}(\varphi(x)) = 1\\
            &\stackrel{(\ast)}{\iff} \exists a\in D,\qquad I^{v(y\mid a)}(\varphi(y)) = 1\\
            &\iff I^v(\exists y \varphi(y)) = 1.
        \end{align*}
        donde, por el mismo argumento que en el caso anterior, hemos dado el paso $(\ast)$.
        Por tanto, hemos probado que:
        \begin{equation*}
            I^v(\exists x \varphi(x)) = I^v(\exists y \varphi(y)).
        \end{equation*}

        Por tanto:
        \begin{align*}
            I( \exists x \varphi(x) \leftrightarrow \exists y \varphi(y))
            &= 1+I(\exists x \varphi(x))+I(\exists y \varphi(y))=1
        \end{align*}
    \end{enumerate}
\end{ejercicio}

\begin{ejercicio}\label{ej:2.7}
    Demuestra que $\not\models \forall x(\psi \vee \varphi) \rightarrow (\forall x \psi \vee \forall x \varphi)$.

    Consideramos la siguiente estructura $\veps$:
    \begin{itemize}
        \item $D = \bb{N}$,
        \item $R,S:\bb{N}\rightarrow\bb{Z}_2$ tales que:
        \begin{align*}
            R(n) &= n\mod 2=\begin{cases}
                0 & \text{si }n\text{ es par},\\
                1 & \text{si }n\text{ es impar},
            \end{cases}\\
            S(n) &= n\mod 2+1=\begin{cases}
                1 & \text{si }n\text{ es par},\\
                0 & \text{si }n\text{ es impar},
            \end{cases}
        \end{align*}
    \end{itemize}

    Consideremos además que $\varphi=R(x)$ y $\psi=S(x)$. Como además no hay variables libres, no es relevante dar una asignación. Por tanto:
    \begin{align*}
        1 &= I^v(\forall x(\psi \vee \varphi))= I^v(\forall x(S(x) \vee R(x))) 
        \\&\iff \forall n\in\bb{N},\ I^{v(x\mid n)}(S(x) \vee R(x)) = 1\\
        &\iff \forall n\in\bb{N},\ I^{v(x\mid n)}(S(x)) + I^{v(x\mid n)}(R(x)) + I^{v(x\mid n)}(S(x))I^{v(x\mid n)}(R(x)) = 1\\
        &\iff \forall n\in\bb{N},\ S(n) + R(n) + S(n)R(n) = 1\\
    \end{align*}
    Dado $n\in \bb{N}$, si $n$ es par entonces $S(n)=1$ y $R(n)=0$, mientras que si $n$ es impar entonces $S(n)=0$ y $R(n)=1$. Por tanto, se tiene que:
    \begin{equation*}
        \forall n\in\bb{N},\ S(n) + R(n) + S(n)R(n) = 1 \Longrightarrow 1=I^v(\forall x(\psi \vee \varphi)).
    \end{equation*}

    Estudiemos ahora la otra fórmula:
    \begin{align*}
        1 = I^v(\forall x \psi)&=I^v(\forall x S(x)) \iff \forall n\in\bb{N},\ I^{v(x\mid n)}(S(x)) = 1\\
        &\iff \forall n\in\bb{N},\ S(n) = 1\iff \forall n\in\bb{N},\ n\mod 2 = 0.\\
        1 = I^v(\forall x \varphi)&=I^v(\forall x R(x)) \iff \forall n\in\bb{N},\ I^{v(x\mid n)}(R(x)) = 1\\
        &\iff \forall n\in\bb{N},\ R(n) = 1\iff \forall n\in\bb{N},\ n\mod 2 = 1.
    \end{align*}

    Por tanto, $I^v(\forall x \varphi)=I^v(\forall x \psi)=0$. Por tanto, con esta interpretación se tiene que:
    \begin{align*}
        I^v\left(\forall x(\psi \vee \varphi) \rightarrow (\forall x \psi \vee \forall x \varphi)\right)
        &= 1+I^v(\forall x(\psi \vee \varphi))+I^v(\forall x(\psi \vee \varphi))I^v(\forall x \psi \vee \forall x \varphi) =\\&= 1+1+1\cdot \left(I^v(\forall x \psi) + I^v(\forall x \varphi) + I^v(\forall x \psi)I^v(\forall x \varphi)\right) =\\&= 1+1+1\cdot(0+0+1\cdot 0) = 1+1+1\cdot 0 = 0
    \end{align*}

    Por tanto, $\not\models \forall x(\psi \vee \varphi) \rightarrow (\forall x \psi \vee \forall x \varphi)$.
\end{ejercicio}

\begin{ejercicio}\label{ej:2.8}
    $\vdash \forall x(\psi \rightarrow \varphi) \rightarrow (\exists x \psi \rightarrow \exists x \varphi)$.\\

    Por su valor semántico, sabemos que esto es equivalente a demostrar:
    \begin{equation*}
        \vdash \forall x(\psi \rightarrow \varphi) \rightarrow (\neg\forall x\neg \psi \rightarrow \neg\forall x\neg \varphi)
    \end{equation*}

    Con vistas a aplicar el Teorema de la deducción, intentaremos demostrar lo que sigue con cuidado:
    \begin{equation*}
        \{\forall x(\psi \rightarrow \varphi), \neg\forall x\neg \psi \}\vdash \neg\forall x\neg \varphi
    \end{equation*}

    Añadiremos como hipótesis $\forall x\neg \varphi$, con vistas a emplear el Teorema de Reducción al Absurdo débil. Por tanto:
    \begin{enumerate}
        \item $\forall x\neg \varphi$ es una hipótesis.
        \item $\forall x\neg\varphi \rightarrow \neg\varphi\in \cc{A}_4$.
        \item $\neg\varphi$ por Modus Ponens de 1 y 2.
        \item $\forall x(\psi \rightarrow \varphi)\rightarrow (\psi \rightarrow \varphi)\in \cc{A}_4$.
        \item $\forall x(\psi \rightarrow \varphi)\in \cc{A}_4$.
        \item $\psi \rightarrow \varphi$ por Modus Ponens de 4 y 5.
        \item $\neg \varphi \rightarrow \neg \psi$ por las leyes de Contraposición.
        \item $\neg \psi$ por Modus Ponens de 3 y 7.
        \item $\forall x\neg \psi$ por la Generalización de $8$ \qquad (Generalización sobre $x$).
        \item $\neg\forall x\neg \psi$ es una hipótesis.
    \end{enumerate}

    Por tanto, por el Teorema de Reducción al Absurdo débil, hemos probado que:
    \begin{equation*}
        \{\forall x(\psi \rightarrow \varphi), \neg\forall x\neg \psi \}\vdash \neg\forall x\neg \varphi
    \end{equation*}

    Como no hemos generalizado sobre variables libres (puesto que $x$ aparece cuantificada en ambas hipótesis), podemos aplicar el Teorema de la deducción y concluir que:
    \begin{equation*}
        \vdash \forall x(\psi \rightarrow \varphi) \rightarrow (\exists x \psi \rightarrow \exists x \varphi).
    \end{equation*}
    
\end{ejercicio}

\begin{ejercicio}\label{ej:2.9}
    $\vdash \exists x(\varphi \rightarrow \psi) \rightarrow (\forall x \varphi \rightarrow \psi)$, supuesto que $x$ no aparece libre en $\psi$
    \begin{observacion}
        Intenta probar $\{\neg \psi, \forall x \varphi\} \vdash \forall x \neg (\varphi \rightarrow \psi)$.
    \end{observacion}~

    Buscaremos en primer lugar demostrar $\{\neg \psi, \forall x \varphi\} \vdash \neg (\varphi \rightarrow \psi)$. Con vistas a aplicar el Teorema de reducción al Absurdo débil, añadimos como hipótesis $\varphi \rightarrow \psi$.
    \begin{enumerate}
        \item $\neg \psi$ es una hipótesis.
        \item $\forall x \varphi$ es una hipótesis.
        \item $\forall x \varphi \rightarrow \varphi\in \cc{A}_4$.
        \item $\varphi$ por Modus Ponens de 2 y 3.
        \item $\varphi \rightarrow \psi$ es una hipótesis.
        \item $\psi$ por Modus Ponens de 4 y 5.
    \end{enumerate}

    Por tanto, por el Teorema de Reducción al Absurdo débil, hemos probado que:
    \begin{equation*}
        \{\neg \psi, \forall x \varphi\} \vdash \neg (\varphi \rightarrow \psi).
    \end{equation*}

    Generalizando sobre $x$, se tiene que:
    \begin{equation*}
        \{\neg \psi, \forall x \varphi\} \vdash \forall x \neg (\varphi \rightarrow \psi).
    \end{equation*}

    Como $x$ no aparece libre en $\psi$, tampoco aparecerá libre en $\neg\psi$. Por tanto, por el Teorema de la deducción, podemos concluir que:
    \begin{equation*}
        \{\forall x\varphi\} \vdash \neg\psi\rightarrow \forall x \neg (\varphi \rightarrow \psi).
    \end{equation*}

    Por las Leyes de Contraposición, se tiene que:
    \begin{equation*}
        \{\forall x\varphi\} \vdash \neg\forall x \neg (\varphi \rightarrow \psi) \rightarrow \psi.
    \end{equation*}

    Por la implicación sencilla del Teorema de la deducción, se tiene que:
    \begin{equation*}
        \{\forall x\varphi, \neg\forall x \neg (\varphi \rightarrow \psi)\} \vdash \psi.
    \end{equation*}

    Como tan solo hemos generalizado sobre $x$ y esta aparece ligada en ambas hipótesis, podemos aplicar dos veces el Teorema de la deducción. En primer lugar:
    \begin{equation*}
        \{\neg\forall x \neg (\varphi \rightarrow \psi)\} \vdash \forall x\varphi \rightarrow \psi.
    \end{equation*}

    Y, en segundo lugar:
    \begin{equation*}
        \vdash \neg\forall x \neg (\varphi \rightarrow \psi) \rightarrow (\forall x\varphi \rightarrow \psi).
    \end{equation*}

    Por su equivalencia semántica, hemos probado que:
    \begin{equation*}
        \vdash \exists x(\varphi \rightarrow \psi) \rightarrow (\forall x \varphi \rightarrow \psi).
    \end{equation*}
\end{ejercicio}

\begin{ejercicio}\label{ej:2.10}
    $\vdash \neg \forall x \psi \rightarrow \exists x \neg \psi$.\\

    Por su equivalencia semántica, sabemos que esto es equivalente a demostrar:
    \begin{equation*}
        \vdash \neg \forall x \psi \rightarrow \neg\forall x(\neg\neg\psi)
    \end{equation*}

    Por las Leyes de Contraposición, sabemos que esto es equivalente a demostrar:
    \begin{equation*}
        \vdash \forall x(\neg\neg\psi) \rightarrow \forall x \psi
    \end{equation*}

    Con vistas a aplicar el Teorema de la deducción, intentaremos demostrar lo que sigue con cuidado:
    \begin{equation*}
        \{\forall x(\neg\neg\psi)\} \vdash \forall x \psi
    \end{equation*}
    \begin{enumerate}
        \item $\forall x(\neg\neg\psi)$ es una hipótesis.
        \item $\forall x(\neg\neg\psi) \rightarrow \neg\neg\psi\in \cc{A}_4$.
        \item $\neg\neg\psi$ por Modus Ponens de 1 y 2.
        \item $\psi$ por la Ley de Doble Negación.
        \item $\forall x\psi$ por la Generalización de 4 \qquad (Generalización sobre $x$).
    \end{enumerate}

    Como $x$ aparece ligada en la hipótesis, podemos aplicar el Teorema de la deducción y concluir que:
    \begin{equation*}
        \vdash \forall x(\neg\neg\psi) \rightarrow \forall x \psi.
    \end{equation*}

    Por las Leyes de Contraposición y su valor semántico, se tiene lo pedido.
\end{ejercicio}

\begin{ejercicio}\label{ej:2.11}
    Si $x$ no aparece libre en $\psi$, $\vdash (\forall x \varphi \rightarrow \psi) \rightarrow \exists x(\varphi \rightarrow \psi)$.\\

    Por su equivalencia semántica, sabemos que esto es equivalente a demostrar:
    \begin{equation*}
        \vdash (\forall x \varphi \rightarrow \psi) \rightarrow \neg\forall x \neg(\varphi \rightarrow \psi)
    \end{equation*}

    Con vistas a aplicar el Teorema de la deducción, intentaremos demostrar lo que sigue con cuidado:
    \begin{equation*}
        \{\forall x \varphi \rightarrow \psi\} \vdash \neg\forall x \neg(\varphi \rightarrow \psi)
    \end{equation*}

    Con vistas a aplicar el Teorema de reducción al Absurdo débil, añadimos como hipótesis $\forall x \neg(\varphi \rightarrow \psi)$.
    \begin{enumerate}
        \item $\forall x \neg(\varphi \rightarrow \psi)$ es una hipótesis.
        \item $\forall x \neg(\varphi \rightarrow \psi)\rightarrow \neg(\varphi \rightarrow \psi)\in \cc{A}_4$.
        \item $\neg(\varphi \rightarrow \psi)$ por Modus Ponens de 1 y 2.
        \item $\neg \varphi\rightarrow (\varphi \rightarrow \psi)$ por las Leyes de Duns Scoto.
        \item $\neg(\varphi \rightarrow \psi)\rightarrow \varphi$ por las Leyes de Contraposición aplicadas a $4$.
        \item $\varphi$ por Modus Ponens de 3 y 5.
        \item $\forall x \varphi$ por la Generalización de 6 \qquad (Generalización sobre $x$).
        \item $\forall x \varphi \rightarrow \psi$ es una hipótesis.
        \item $\psi$ por Modus Ponens de 7 y 8.
        \item $\psi\to (\varphi\to \psi)\in\cc{A}_1$.
        \item $\varphi\to \psi$ por Modus Ponens de 9 y 10. 
    \end{enumerate}

    Como $x$ no aparece libre en la hipótesis adicional que hemos añadido, entonces:
    \begin{equation*}
        \{\forall x \varphi \rightarrow \psi\} \vdash \neg\forall x \neg(\varphi \rightarrow \psi)
    \end{equation*}

    Como tan solo hemos generalizado sobre $x$ y esta no aparece libre en $\psi$, podemos aplicar el Teorema de la deducción y concluir que:
    \begin{equation*}
        \vdash (\forall x \varphi \rightarrow \psi) \rightarrow \neg\forall x \neg(\varphi \rightarrow \psi).
    \end{equation*}
\end{ejercicio}



\begin{ejercicio}
    En un sistema de primer orden con igualdad, demuestra que las siguientes fórmulas son teoremas:
    \begin{enumerate}
        \item $\forall x (x = x)$.
        
        Se tiene de forma directa por ser un axioma.
        \item $\forall x \forall y (x = y \rightarrow y = x)$.
        \begin{enumerate}[label=\arabic*.]
            \item $\forall x(x=x)\in \cc{A}_6$.
            \item $\forall x(x=x)\to (x=x)\in \cc{A}_4$.
            \item $x=x$ por modus ponens de 1 y 2.
            \item $(x=x)\to [(x=y)\to (x=x)]\in \cc{A}_1$.
            \item $(x=y)\to (x=x)$ por modus ponens de 3 y 4.
            \item $(x=y)\to [(x=x)\to (y=x)]\in \cc{A}_7$.
            \item $(x=x)\to [(x=y)\to (y=x)]$ por la Regla de Conmutación de las Premisas aplicada a 6.
            \item $(x=y)\to (y=x)$ por modus ponens de 3 y 7.
            \item $\forall x\forall y (x=y\to y=x)$ por generalización en $y,x$.
        \end{enumerate}
        
        \item $\forall x \forall y \forall z (x = y \rightarrow (y = z \rightarrow x = z))$.
        \begin{enumerate}[label=\arabic*.]
            \item $(y = x) \rightarrow (y = z \rightarrow x = z)\in \cc{A}_7$
            \item $\forall x\forall y (x=y\to y=x)$ por el apartado anterior.
            \item $(x=y\to y=x)$ tras aplicar dos veces el axioma $4$ y modus ponens.
            \item $(x = y) \rightarrow (y = z \rightarrow x = z)$ por la Regla del Silogismo aplicada a $3$ y $1$.
            \item $\forall x \forall y \forall z (x = y \rightarrow (y = z \rightarrow x = z))$ tras generalizar en $z,y,x$.
        \end{enumerate}
    \end{enumerate}
\end{ejercicio}

\begin{ejercicio}
    Sean $n, m \in \mathbb{N}$. En la aritmética de primer orden $\cc{N}$ prueba que:
    \begin{enumerate}
        \item si $n \neq m$, entonces $\vdash_{\cc{N}} \neg (s^n(0) = s^m(0))$,
        
        Con vistas a aplicar el Teorema de Reducción al Absurdo débil, suponemos como única hipótesis $s^n(0) = s^m(0)$. Además, podemos suponer $n>m$ sin pérdida de generalidad (en caso contrario, tendremos que aplicar la propiedad simétrica de la igualdad, demostrada en el ejercicio anterior). Entonces, tenemos que:
        \begin{enumerate}[label=\arabic*.]
            \item $s(s^{n-1}(0)) = s(s^{m-1}(0))$ es una hipótesis.
            \item $s(s^{n-1}(0)) = s(s^{m-1}(0))\to (s^{n-1}(0)=s^{m-1}(0))\in \cc{N}_2$.
            \item $s^{n-1}(0)=s^{m-1}(0)$ por modus ponens de 1 y 2.\\
            $\vdots$
            \item [$3(m-1).$] $s^{n-m+1}(0)=s(0)$ por modus ponens de $3(m-1)-1$ y $3(m-1)-2$.
            \item [$3(m-1)+1.$] $s\left(s^{n-m}(0)\right)=s(0)\to \left(s^{n-m}(0)=0\right)\in \cc{N}_2$.
            \item [$3(m-1)+2.$] $s^{n-m}(0)=0$ por modus ponens de $3(m-1)+1$ y $3(m-1)$.
            \item [$3m.$] $s(s^{n-m-1}(0))\neq 0\in \cc{N}_2$, puesto que $n-m-1\geq 0$.
        \end{enumerate}

        Por el Teorema de Reducción al Absurdo débil, se concluye que $s^n(0) \neq s^m(0)$.\\
        \item si $n = m$, entonces $\vdash_{\cc{N}} (s^n(0) = s^m(0))$,
        
        Como $n=m$, y $s$ es una aplicación, se tiene que $s^n(0)$ y $s^m(0)$ son el mismo elemento. Por lo tanto:
        \begin{enumerate}[label=\arabic*.]
            \item $s^n(0) = s^m(0)\in \cc{A}_6$.
        \end{enumerate}~
        \item $\vdash_{\cc{N}} (s^n(0) + s^m(0) = s^{n+m}(0))$.
        \begin{enumerate}[label=\arabic*.]
            \item $s^{n}(0)+ s(s^{m-1}(0)) = s(s^{n}(0)+s^{m-1}(0))\in \cc{N}_2$.\\
            \vdots
            \item[$m-1$.] $s^{m-2}(s^{n}(0)+s(s(0))) = s^{m-1}(s^{n}(0)+s(0))\in \cc{N}_2$.
            \item[$m$.] $s^{m-1}(s^{n}(0)+s(0))=s^{m}(s^{n}(0)+0)\in \cc{N}_2$.
            \item[$m+1$.] $s^{n}(0)+0=s^n(0)\in \cc{N}_3$.
            \item[$m+2$.] $s^{n}(0)+0=s^n(0) \to s^{m}(s^{n}(0)+0)=s^m(s^n(0))\in \cc{A}_7$.
            \item[$m+3$.] $s^{m-1}(s^{n}(0)+s(0))=s^{m}(s^{n}(0))$ por Modus ponens de $m$ y $m+2$ y transitividad.
            \item[$m+4$.] $s^n(0) + s^m(0) = s^{n+m}(0)$ tras aplicar la transitividad a $1,\dots,m-1$ y $m+3$.
        \end{enumerate}
    \end{enumerate}
\end{ejercicio}

\begin{ejercicio}
    Usando los ejercicios anteriores, prueba que para todo $n, m \in \mathbb{N}$, $n + m = m + n$ en $\cc{N}$, a saber,
    \[
        \vdash_{\cc{N}} s^n(0) + s^m(0) = s^m(0) + s^n(0).
    \]
    \begin{enumerate}
        \item $s^n(0) + s^m(0)=s^{n+m}(0)$ por el ejercicio anterior.
        \item $s^{n+m}(0)=s^{m+n}(0)$ por la conmutatividad en $\bb{N}$.
        \item $s^m(0) + s^n(0)=s^{m+n}(0)$ por el ejercicio anterior.
        \item $s^{m+n}(0)=s^m(0) + s^n(0)$ por la simetría de la igualdad.
        \item $s^n(0) + s^m(0)=s^m(0) + s^n(0)$ por la transitividad aplicada a $1,2,4$.
    \end{enumerate}
\end{ejercicio}
    \section{Teoría Descriptiva de Conjuntos}

    \begin{ejercicio}
    Demostrar que $(2^\mathbb{N}, d)$ es un espacio completo. \\

    Veamos en primer lugar que $d$ es una distancia.
    \begin{enumerate}
        \item No negatividad:
            \begin{equation*}
                d(x,y) = \frac{1}{2^{n+1}} \geq 0 \qquad \forall x,y\in 2^\mathbb{N}
            \end{equation*}
            Además, se tiene que $d(x,y) = 0$ si y solo si $x = y$.

        \item Simetría:
        
            Sea $n=\min\{k\in \mathbb{N} \mid x(k) \neq y(k)\}=\min\{k\in \mathbb{N} \mid y(k) \neq x(k)\}$, entonces:
            \begin{equation*}
                d(x,y) = \frac{1}{2^{n+1}} = d(y,x)
            \end{equation*}

        \item Desigualdad triangular:
        
            Sean los siguientes tres mínimos, que suponemos que existen (ya que si no existen, la desigualdad se verifica trivialmente):
            \begin{align*}
                n_1 &= \min\{k\in \mathbb{N} \mid x(k) \neq y(k)\} \\
                n_2 &= \min\{k\in \mathbb{N} \mid y(k) \neq z(k)\} \\
                n &= \min\{k\in \mathbb{N} \mid x(k) \neq z(k)\}
            \end{align*}

            Tenemos que $\min\{n_1,n_2\} \leq n$, puesto que para $k<\min\{n_1,n_2\}$, se verifica que $x(k) = y(k)$ y $y(k) = z(k)$, por lo que $x(k) = z(k)$. Por tanto, $n\geq \min\{n_1,n_2\}$. Por tanto:
            \begin{align*}
                d(x,z) &= \frac{1}{2^{n+1}} \leq \frac{1}{2^{\min\{n_1,n_2\}+1}} = \max\left\{\frac{1}{2^{n_1+1}},\frac{1}{2^{n_2+1}}\right\} =\\&= \max\{d(x,y),d(y,z)\}\leq d(x,y) + d(y,z)
            \end{align*}
    \end{enumerate}

    Por tanto, hemos demostrado que $d$ es una distancia, por lo que consideramos el espacio métrico $(2^\mathbb{N},d)$. Este será completo si toda sucesión de Cauchy es convergente a una sucesión de cantor, lo que veremos a continuación.\\

    Sea $\{x_n\}_{n\in \mathbb{N}}$ una sucesión de Cauchy, es decir:
    \begin{equation*}
        \forall \veps\in \mathbb{R}^+\ \exists N\in \mathbb{N}\ \forall m,n\geq N\ d(x_m,x_n) < \veps
    \end{equation*}

    Veamos ahora cómo demostrar que esta sucesión es convergente, para lo cual hemos de construir la sucesión $x$ que sea el límite de la sucesión de Cauchy. Para cada $j\in \bb{N}$, consideramos $\veps=\frac{1}{2^j+1}$, y por tanto, existe $N_j\in \mathbb{N}$ tal que:
    \begin{equation*}
        \forall m,n\geq N_j\qquad d(x_m,x_n) < \frac{1}{2^j+1}
    \end{equation*}

    Por tanto, para cada $m,n\geq N_j$, tenemos que $x_m(j)=x_n(j)$. Definimos por tanto:
    \begin{equation*}
        x(j) = x_{N_j}(j) = x_m(j) \qquad \forall m\geq N_j
    \end{equation*}

    Vemos que $x$ es una sucesión de Cantor, y ahora hemos de demostrar que es el límite de la sucesión de Cauchy. Fijado $\veps\in \mathbb{R}^+$, existe $k\in \mathbb{N}$ tal que $\frac{1}{2^k+1} < \veps$, y podemos considerar $N_k\in \mathbb{N}$ tal que:
    \begin{equation*}
        \forall m,n\geq N_k\ d(x_m,x_n) < \frac{1}{2^k+1}
    \end{equation*}

    Por tanto, para todo $m\geq N_k$, veamos que $x_m(j) = x(j)$ para todo $j\leq k$. Sea $j\leq k$, luego:
    \begin{equation*}
        \frac{1}{2^{k+1}}\leq \frac{1}{2^j+1}\Longrightarrow N_j\leq N_k
    \end{equation*}

    Por tanto, para todo $m\geq N_k\geq N_j$, se tiene que $x_m(j) = x_{N_j}(j) = x(j)$. Por tanto, para todo $m\geq N_k$, se verifica que:
    \begin{equation*}
        d(x_m,x) < \frac{1}{2^{k+1}} < \veps
    \end{equation*}

    Por tanto, hemos demostrado que la sucesión de Cauchy $\{x_n\}_{n\in \mathbb{N}}$ converge a $x\in 2^\mathbb{N}$, y por tanto, $(2^\mathbb{N},d)$ es completo.
\end{ejercicio}


\begin{ejercicio}
    Sea $(X,d)$ un espacio métrico. Dados $x\in X$ y $A\subseteq X$, definimos la distancia entre $a$ y $X$ como:
    \begin{equation*}
        d(x,A) = \inf\{d(x,a) \mid a\in A\}
    \end{equation*}
    Verificar que, dado $r>0$, el siguiente conjunto es un abierto:
    \begin{equation*}
        \{x\in X\mid d(x,A) < r\}
    \end{equation*}
\end{ejercicio}
\begin{proof}
    Dado $x\in X$ con $d(x,A) < r$, veamos que $\exists \veps\in \mathbb{R}^+$ de forma que $B(x,\veps) \subseteq \{x\in X\mid d(x,A) < r\}$.\\

    Sea $\veps=r-d(x,A)>0$, y sea $y\in B(x,\veps)$, es decir, $d(x,y) < \veps$. Veamos que $d(y,A) < r$.
    \begin{align*}
        d(y,a) \leq d(y,x) + d(x,a) \forall a\in A
    \end{align*}

    Por tanto:
    \begin{align*}
        d(y,A) &= \inf\{d(y,a) \mid a\in A\} \\
        &\leq \inf\{d(y,x) + d(x,a) \mid a\in A\}
        = d(y,x) + \inf\{d(x,a) \mid a\in A\} \\
        &= d(y,x) + d(x,A)< \veps + d(x,A) = r-d(x,A) + d(x,A) = r
    \end{align*}

    Por tanto, $y\in \{x\in X\mid d(x,A) < r\}$, y hemos demostrado que:
    \begin{equation*}
        B(x,\veps) \subseteq \{x\in X\mid d(x,A) < r\}
    \end{equation*}
    Por tanto, $\{x\in X\mid d(x,A) < r\}$ es un abierto.
\end{proof}


\begin{ejercicio}
    En el cubo de Hilbert ${[0,1]}^{\mathbb{N}}$, consideramos la métrica $d$ definida como:
    \begin{equation*}
        d(x,y) = \sum_{n=0}^{+\infty} \dfrac{d(x_n,y_n)}{2^n} \qquad \forall x,y\in {[0,1]}^{\mathbb{N}}
    \end{equation*}


    Demostrar que $d$ es una métrica en ${[0,1]}^{\mathbb{N}}$.
    \begin{proof}
        En primer lugar, hemos de ver que la distancia así definida está bien definida, es decir, que la suma converge. Aplicamos para ello el Criterio de Comparación:
        \begin{equation*}
            \sum_{n=0}^{+\infty} \dfrac{d(x_n,y_n)}{2^n} \leq \sum_{n=0}^{+\infty} \dfrac{1}{2^n} = \dfrac{1}{1-\nicefrac{1}{2}} = 2
        \end{equation*}

        Por tanto, $d$ está bien definida. Ahora, veamos que $d$ es una métrica:
        \begin{itemize}
            \item \underline{No-negatividad}: Por definición de $d$, tenemos que:
                \begin{equation*}
                    d(x,y) = \sum_{n=0}^{+\infty} \dfrac{d(x_n,y_n)}{2^n} \geq 0
                \end{equation*}

                Además, se tiene que $d(x,y) = 0$ si y solo si $x = y$.
            \item \underline{Simetría}: Por definición de $d$, tenemos que:
                \begin{equation*}
                    d(x,y) = \sum_{n=0}^{+\infty} \dfrac{d(x_n,y_n)}{2^n} = \sum_{n=0}^{+\infty} \dfrac{d(y_n,x_n)}{2^n} = d(y,x)
                \end{equation*}
            \item \underline{Desigualdad triangular}: Tenemos que:
                \begin{align*}
                    d(x,z) &= \sum_{n=0}^{+\infty} \dfrac{d(x_n,z_n)}{2^n}
                    \leq \sum_{n=0}^{+\infty} \dfrac{d(x_n,y_n) + d(y_n,z_n)}{2^n}
                    = \sum_{n=0}^{+\infty} \dfrac{d(x_n,y_n)}{2^n} + \sum_{n=0}^{+\infty} \dfrac{d(y_n,z_n)}{2^n} \\
                    &= d(x,y) + d(y,z)
                \end{align*}
            \end{itemize}
        Por tanto, hemos visto que $d$ es una métrica en ${[0,1]}^{\mathbb{N}}$.
    \end{proof}
\end{ejercicio}


\begin{ejercicio}
        Definimos los siguientes conjuntos:
        \begin{align*}
           Q_2 &= \left\{ \alpha \in \cc{C} \;\mid\; \exists A\subset \mathbb{N}\ \text{finito tal que}\ \alpha(n)=0\ \forall n\in \mathbb{N}\setminus\{A\}\right\} \\
           \ell^1 &= \left\{ x\in [0,1]^\mathbb{N}\ \middle|\ \sum_{n=1}^\infty x_n < \infty\right\}
        \end{align*}
        Demostrar que $\ell^1\in \Sigma_2^0$ y $Q_2 \leq_W \ell^1$.
    \end{ejercicio}

    \begin{ejercicio}
        Sea $\Gamma$ una clase de la Jerarquía Boreliana, y $X$ un conjunto. Si $A\subset X$ es $\Gamma-$completo, y $B\subset X$ es otro conjunto de la clase $\Gamma$ tal que $A\leq_W B$, entonces $B$ es $\Gamma-$completo.
        \begin{proof}
            Hemos de comprobar que:
        \begin{itemize}
            \item \ul{$B\in \Gamma$}: Se tiene por hipótesis.
            \item \ul{Para todo espacio polaco $X'$, si $C\in \Gamma(X')$ entonces $C\leq_W B$}:
            
            Sea $C\in \Gamma(X')$, y buscamos $f:X'\to X$ tal que $f$ es una función continua y $C=f^{-1}(B)$.
            
            Como $A$ es $\Gamma-$completo, existe $g:X'\to X$ tal que $g$ es continua y $C=g^{-1}(A)$. Por otro lado, como $A\leq_W B$, existe una función continua $h:X\to X$ tal que $A=h^{-1}(B)$. Entonces, la composición $f=h\circ g$ es continua y cumple que:
            \begin{equation*}
                f^{-1}(B) = g^{-1}(h^{-1}(B)) = g^{-1}(A) = C
            \end{equation*}

            Por tanto, $C\leq_W B$.
        \end{itemize}
        \end{proof}
    \end{ejercicio}


    \begin{ejercicio}
        Demostrar que $f:[0,1]\to \bb{R}$ es continuamente derivable si y solo si
        \begin{equation*}
            \forall\veps\in \bb{R}^+\ \exists\delta\in \bb{R}^+\ f\in A_{\veps,\delta}
        \end{equation*}
        donde:
        \begin{multline*}
            A_{\veps,\delta} = \left\{f\in C([0,1])\ \middle|\ \forall x,y,a,b\in [0,1]\ : a,b,x,y \text{ a distancia } \leq \delta\right. \Longrightarrow \\\Longrightarrow \left.\left|\frac{f(a)-f(b)}{a-b} - \frac{f(x)-f(y)}{x-y}\right| < \veps\right\}
        \end{multline*}
        \begin{proof}
            Sea $f:[0,1]\to \bb{R}$. Demostraremos por doble implicación.
            \begin{description}
                \item[$\Longrightarrow)$] Sea $f\in C^1([0,1])$, y sea $\veps\in \bb{R}^+$. Como $f$ es derivable en $[0,1]$, por el Teorema del Valor Medio existe $a'\in \left]a,b\right[, x'\in \left]x,y\right[$ tal que:
                \begin{equation*}
                    \frac{f(a)-f(b)}{a-b} = f'(a')\qquad \text{y}\qquad \frac{f(x)-f(y)}{x-y} = f'(x')
                \end{equation*}

                Por el Teorema de Heine, como $[0,1]$ es compacto y $f'$ es continua, existe $\delta'\in \bb{R}^+$ tal que:
                \begin{equation*}
                    |a'-x'| < \delta'
                    \Longrightarrow |f'(a') - f'(x')| < \veps
                \end{equation*}

                Sea ahora $\delta = \nicefrac{\delta'}{3}$. Usando que $a,b,x,y$ están a distancia $\leq \delta$, veamos que $|x'-a'| < \delta'$:
                \begin{align*}
                    |x'-a'| &\leq |x'-x| + |x-a| + |a-a'|< |x-y| + |x-a| + |a-b|\leq 3\delta = \delta'
                \end{align*}

                Por tanto, se verifica que:
                \begin{align*}
                    \left|\frac{f(a)-f(b)}{a-b} - \frac{f(x)-f(y)}{x-y}\right| &= |f'(a') - f'(x')| < \veps
                \end{align*}

                Por tanto, $f\in A_{\veps,\delta}$.


                \item[$\Longleftarrow)$] Hemos de demostrar que $f$ es continuamente derivable. Para ello, definimos el cociente incremental de $f$ en $t\in [0,1]$ como:
                \begin{equation*}
                    f_t(x) = \frac{f(x)-f(t)}{x-t} \qquad \forall x\in [0,1]\setminus\{t\}
                \end{equation*}

                En primer lugar, hemos de ver que $f$ es derivable, para lo cual hemos de comprobar que, para cada $t\in [0,1]$, el siguiente límite existe:
                \begin{equation*}
                    \lim_{x\to t} f_t(x)
                \end{equation*}

                Para comprobar que este límite existe, usaremos que $\bb{R}$ es completo, por lo que toda sucesión de Cauchy converge. Sea $t\in [0,1]$, y sea $\{x_n\}_{n\in \bb{N}}$ una sucesión de puntos de $[0,1]\setminus\{t\}$ tal que $\{x_n\}\to t$. Veamos que $\{f_t(x_n)\}_{n\in \bb{N}}$ es una sucesión de Cauchy. Para ello, fijamos $\veps\in \bb{R}^+$, por lo que $\exists \delta\in \bb{R}^+$ tal que $f\in A_{\veps,\delta}$. Por ser $\{x_n\}$ de Cauchy, existe $N\in \bb{N}$ tal que, para todo $m,n\geq N$, se verifica que:
                \begin{equation*}
                    |x_m-x_n| < \delta
                \end{equation*}

                Por tanto, $t,x_m,x_n$ están a distancia $\leq \delta$, y por tanto, como $f\in A_{\veps,\delta}$, se verifica que:
                \begin{align*}
                    \left|f_t(x_m) - f_t(x_n)\right| &= \left|\frac{f(x_m)-f(t)}{x_m-t} - \frac{f(x_n)-f(t)}{x_n-t}\right|< \veps
                \end{align*}

                Por tanto, $\{f_t(x_n)\}_{n\in \bb{N}}$ es una sucesión de Cauchy, y por tanto, converge a un límite $f'(t)\in \bb{R}$. Definimos por tanto:
                \begin{equation*}
                    f'(t) = \lim_{x\to t} f_t(x)
                \end{equation*}

                Ahora, queremos demostrar que \( f' \) es continua en todo \( [0,1] \). Para ello, fijamos un punto \( x \in [0,1] \), y tomamos una sucesión \( \{t_n\}_{n \in \mathbb{N}} \subset [0,1] \setminus \{x\} \) tal que \( \{t_n\} \to x \). Queremos ver que:
                \[
                \lim_{n \to \infty} f'(t_n) = f'(x)
                \]

                Recordemos que, por definición,
                \[
                f'(t_n) = \lim_{y \to t_n} \frac{f(y) - f(t_n)}{y - t_n}
                \quad \text{y} \quad
                f'(x) = \lim_{z \to x} \frac{f(z) - f(x)}{z - x}
                \]

                Fijado ahora $\veps\in \bb{R}^+$, consideramos $\delta \in \bb{R}^+$ tal que \( f \in A_{\varepsilon,\delta} \).  Como \( \{t_n\} \to x \), existe \( N \in \mathbb{N} \) tal que para todo \( n \geq N \), se tiene \( |t_n - x| < \nicefrac{\delta}{2} \). Fijamos tal \( n \geq N \), y tomamos \( y \in [0,1] \) con \( |y - t_n| < \nicefrac{\delta}{2} \). Entonces, por desigualdad triangular:
                \[
                |y - x| \leq |y - t_n| + |t_n - x| < \nicefrac{\delta}{2} + \nicefrac{\delta}{2} = \delta
                \]

                Por tanto, \( x, y, t_n \) están a distancia menor que \( \delta \), y podemos aplicar la hipótesis:
                \[
                \left| \frac{f(y) - f(t_n)}{y - t_n} - \frac{f(z) - f(x)}{z - x} \right| < \varepsilon
                \quad \forall z\in [0,1]\ \text{tal que}\ |x-z|<\delta
                \]

                Tomando el límite cuando \( y \to t_n \), se obtiene:
                \[
                \left| f'(t_n) - \frac{f(z) - f(x)}{z - x} \right| < \varepsilon
                \quad \forall z\in [0,1]\ \text{tal que}\ |x-z|<\delta
                \]

                Y tomando después el límite cuando \( z \to x \), se concluye que $|f'(t_n) - f'(x)| < \varepsilon$. Como $n\geq N$ era arbitrario, tenemos que:
                \[
                \lim_{n \to \infty} f'(t_n) = f'(x)
                \]

                Es decir, \( f' \) es continua en \( x \). Como \( x \in [0,1] \) era arbitrario, concluimos que \( f' \) es continua en todo el intervalo, y por tanto, \( f \in C^1([0,1]) \).
            \end{description}
        \end{proof}
    \end{ejercicio}


\end{document}
