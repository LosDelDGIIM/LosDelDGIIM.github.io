\chapter{Lógica Proposicional}
Consideraremos un conjunto finito de proposiciones atómicas, que serán para nosotros enunciados indivisibles. Nos interesará la veracidad o falsedad de cada una de estas proposiciones. Consideraremos sobre estas las conectivas $\lnot$, $\land$, $\lor$, $\to$ y $\leftrightarrow$. De esta forma, somos capaces de definir lo que es una \underline{proposición} en nuestro lenguaje.

\begin{definicion}[Proposición]
    Definimos las proposiciones de forma recursiva\footnote{Algo que será habitual en este curso.}:
    \begin{enumerate}
        \item Las proposiciones atómicas son proposiciones.
        \item Si $\alpha$ y $\beta$ son proposiciones, también lo son:
            \begin{equation*}
                \lnot\alpha,\ \alpha\land\beta,\ \alpha\lor\beta,\ \alpha\to\beta,\ \alpha\leftrightarrow\beta
            \end{equation*}
        \item No hay más proposiciones que las que se puedan obtener siguiendo una secuencia finita de pasos a partir de las enunciadas.
    \end{enumerate}
\end{definicion}

\section{Semántica}
Una vez definida lo que es una proposición, pasamos a lo que nos interesa, asignar un valor de verdad o de falsedad a cada una de las proposiciones que nos encontremos. Para ello, consideraremos una aplicación del conjunto de las proposiciones en $\mathbb{Z}_2$, e interpretaremos el valor de $0$ como falso y el valor de $1$ como verdad.

\begin{definicion}[Interpretación]
    Sea $\mathcal{P}$ el conjunto de todas las proposiciones de un lenguaje proposicional, una interpretación sobre el mismo es una aplicación\newline $I:\mathcal{P}\rightarrow\mathbb{Z}_2$ que verifica:
    \begin{enumerate}
        \item $I(\lnot a) = 1+ I(a)$.
        \item $I(a\land b) = I(a)I(b)$.
        \item $I(a\lor b) = I(a) + I(b) + I(a)I(b)$.
        \item $I(a\to b) = 1 + I(a) + I(a)I(b)$.
        \item $I(a\leftrightarrow b) = 1 + I(a) + I(b)$.
    \end{enumerate}
    Para cualesquiera proposiciones $a,b\in \mathcal{P}$.
\end{definicion}

\begin{observacion}
    Observemos que, gracias a la naturaleza recursiva de las interpretaciones, basta dar un valor de $\mathbb{Z}_2$ a cada proposición atómica para obtener una interpretación: conocidos los valores de las proposiciones atómicas conocemos el valor de cualquier proposición y viceversa.
\end{observacion}

\begin{definicion}
    Sea $\alpha$ y $\beta$ dos proposiciones de forma que $I(\alpha)=I(\beta)$ para cualquier interpretación $I$, entonces escribiremos que $\alpha\equiv\beta$ y podemos decir que $\alpha$ y $\beta$ son \underline{semánticamente equivalentes}.
\end{definicion}

\begin{definicion}
    Sea $\alpha$ una proposición: 
    \begin{itemize}
        \item Si existe una interpretación $I$ de forma que $I(\alpha)=1$, diremos que $p$ es \textbf{satisfacible}.
        \item Si existe una interpretación $I$ de forma que $I(\alpha)=0$, diremos que $p$ es \textbf{refutable}.
        \item Si $I(\alpha)=1$ para cualquier interpretación $I$, diremos que $p$ es una \textbf{tautología}.
        \item Si $I(\alpha)=0$ para cualquier interpretación $I$, diremos que $p$ es una \textbf{contradicción}.
    \end{itemize}
\end{definicion}

\begin{definicion}[Consecuencia lógica]
    Sea $\Gamma\cup\{p\}$ un conjunto de proposiciones, decimos que $p$ es consecuencia lógica de $\Gamma$ (notado por $\Gamma\vDash p$), si dada una interpretación $I$, siempre que se tenga que $I(\gamma) = 1$ para cualquier $\gamma\in \Gamma$, entonces se tiene que $I(p)= 1$.
\end{definicion}

\begin{notacion}
    Por comodidad, si $p$ es una proposición de forma que $\emptyset \vDash p$, entonces notaremos:
    \begin{equation*}
        \vDash p
    \end{equation*}
    Notemos que en este caso $p$ es una tautología, ya que estamos diciendo que $I(p)=1$ para cualquier\footnote{Cualquiera que haga ciertos todos los elementos del vacío.} interpretación $I$.
\end{notacion}

\begin{prop}
    Se verifica que $\Gamma\vDash p$ si y solo si $(1+I(p))\displaystyle\prod_{\gamma\in \Gamma}I(\gamma) = 0$.
    \begin{proof}
        Veamos las dos implicaciones:
    \begin{description}
        \item [$\Longrightarrow)$] 
            Sea $I$ una interpretación:
            \begin{itemize}
                \item Si existe un $\gamma\in \Gamma$ de forma que $I(\gamma)=0$, entonces tenemos el resultado.
                \item En caso contrario, tendremos que $I(\gamma)=1$ para cualquier $\gamma\in \Gamma$. En dicho caso, como $\Gamma\vDash p$, se tendrá que $I(p)=1$, por lo que:
                    \begin{equation*}
                        1 + I(p) = 0 \Longrightarrow (1+I(p))\displaystyle\prod_{\gamma\in \Gamma}I(\gamma) = 0
                    \end{equation*}
            \end{itemize}
        \item [$\Longleftarrow)$] 
            Sea $I$ una interpretación que verifica $I(\gamma)=1$ para cualquier $\gamma\in \Gamma$, como $\mathbb{Z}_2$ es un dominio de integridad, de $(1+I(p))\displaystyle\prod_{\gamma\in \Gamma}I(\gamma) = 0$ deducimos que $I(p) +1=0$, por lo que $I(p) = 1$ y entonces se tiene que $\Gamma\vDash p$.
    \end{description}
    \end{proof}
\end{prop}

\begin{teo}[de la deducción]\label{teo:deduccion}
    Sea $\Gamma\cup\{\alpha,\beta\}$ un conjunto de proposiciones, equivalen:
    \begin{enumerate}
        \item $\Gamma\vDash\alpha\to\beta$
        \item $\Gamma\cup\{\alpha\}\vDash\beta$
    \end{enumerate}
    \begin{proof}
        Demostramos las dos implicaciones:
        \begin{description}
            \item [$1)\Longrightarrow 2)$] 
                Sea $I$ una interpretación de forma que $I(\alpha)=1$ y que $I(\gamma)=1$ para todo $\gamma\in \Gamma$, entonces (por 1) deducimos que $I(\alpha\to\beta)=1$, luego:
                \begin{equation*}
                    1 = I(\alpha\to\beta) = 1 + \cancelto{1}{I(\alpha)} + \cancelto{1}{I(\alpha)}I(\beta) = 1 + 1 + I(\beta) = I(\beta)
                \end{equation*}
            \item [$2)\Longrightarrow 1)$] 
                Sea $I$ una interpretación de forma que $I(\gamma)=1$ para todo $\gamma\in \Gamma$: 
                \begin{itemize}
                    \item Si $I(\alpha)=0$, entonces:
                        \begin{equation*}
                            I(\alpha\to\beta) = 1 + I(\alpha) + I(\alpha)I(\beta) = 1
                        \end{equation*}
                        Por lo que se tiene 1.
                    \item Si $I(\alpha)=1$, como $\Gamma\cup\{\alpha\}\vDash\beta$, entonces $I(\beta)=1$, por lo que:
                        \begin{equation*}
                            I(\alpha\to\beta) = 1 + I(\alpha) + I(\alpha)I(\beta) = 1 + 1 + 1 = 1
                        \end{equation*}
                \end{itemize}
        \end{description}
    \end{proof}
\end{teo}

\begin{ejemplo}
    Demostraremos ahora que varias proposiciones son tautologías:
    \begin{description}
        \item [$\vDash\alpha\to\alpha$]~\\
            Por el Teorema de la deducción (\ref{teo:deduccion}), $\vDash\alpha\to\alpha$ es equivalente a ver que $\{\alpha\}\vDash\alpha$. En efecto, sea $I$ una interpretación de forma que $I(\alpha)=1$, tenemos que $I(\alpha)=1$.
        \item [$\vDash\alpha\to(\beta\to\alpha)$]~\\
            Por el Teorema de la deducción, es equivalente ver que $\{\alpha\}\vDash \beta\to\alpha$; que nuevamente por el Teorema de la deducción es equivalente ver que $\{\alpha,\beta\}\vDash \alpha$. En efecto, sea $I$ una interpretación de forma que $I(\alpha)=I(\beta)=1$, entonces $I(\alpha)=1$.
        \item [$\vDash(\alpha\to(\beta\to\gamma))\to((\alpha\to\beta)\to(\alpha\to\gamma))$]~\\
            Por el Teorema de la deducción aplicado 3 veces, es equivalente ver que:
            \begin{equation*}
                \{\alpha\to(\beta\to\gamma),\alpha\to\beta,\alpha\}\vDash \gamma
            \end{equation*}
            Sea $I$ una interpretación de forma que:
            \begin{equation*}
                1 = I(\alpha\to(\beta\to\gamma)) = I(\alpha\to\beta) = I(\alpha)
            \end{equation*}
            Entonces:
            \begin{align*}
                1 &= I(\alpha\to\beta) = 1 + I(\alpha) + I(\alpha)I(\beta) = 1 + 1 + I(\beta) = I(\beta) \Longrightarrow I(\beta) = 1 \\
                1 &= I(\alpha\to(\beta\to\gamma)) = 1 + I(\alpha) + I(\alpha)I(\beta\to\gamma) \\
                  &= 1 + I(\alpha) + I(\alpha)(1 + I(\beta) + I(\beta)I(\gamma)) = 1 + 1 + 1(1 + 1 + I(\gamma))  \\
                  &= I(\gamma) \Longrightarrow \underline{I(\gamma) = 1}
            \end{align*}
        \item [$\vDash(\lnot\alpha\to\lnot\beta) \to ((\lnot\alpha\to\beta)\to \alpha)$]~\\
            Por el Teorema de la deducción aplicado 2 veces, es equivalente ver que:
            \begin{equation*}
                \{\lnot\alpha\to\lnot\beta,\lnot\alpha\to\beta\} \vDash \alpha
            \end{equation*}
            Sea $I$ una interpretación de forma que:
            \begin{align*}
                1 = I(\lnot\alpha\to\lnot\beta) &= 1 + I(\lnot\alpha) + I(\lnot\alpha)I(\lnot\beta) \\
                1 = I(\lnot\alpha\to\beta) &= 1 + I(\lnot\alpha) + I(\lnot\alpha)I(\beta)
            \end{align*}
            Entonces (sumando):
            \begin{equation*}
                0 = I(\lnot\alpha\to\lnot\beta) + I(\lnot\alpha\to\beta) = I(\lnot\alpha)(I(\lnot\beta) + I(\beta)) \AstIg I(\lnot\alpha)
            \end{equation*}
            Donde en $(\ast)$ hemos usado que $I(\lnot\beta) = 1 + I(\beta)\Longrightarrow I(\lnot\beta)+I(\beta)=1$.

            Como $I(\lnot\alpha)=0$, se tiene que $I(\alpha)=1$, como queríamos demostrar.
    \end{description}
\end{ejemplo}

\begin{definicion}
    Sea $\Gamma$ un conjunto de proposiciones, decimos que $\Gamma$ es \textbf{inconsistente} si para toda interpretación $I$ existe $\gamma\in \Gamma$ de forma que $I(\gamma)=0$.
\end{definicion}

\begin{prop}
    Sea $\Gamma\cup\{\alpha\}$ un conjunto de proposiciones, equivalen:
    \begin{enumerate}
        \item $\Gamma\vDash \alpha$.
        \item $\Gamma\cup\{\lnot\alpha\}$ es inconsistente.
    \end{enumerate}
    \begin{proof}
        Demostramos las dos implicaciones:
        \begin{description}
            \item [$1) \Longrightarrow 2)$] Sea $I$ una interpretación:
                \begin{itemize}
                    \item Si existe un $\gamma\in \Gamma$ de forma que $I(\gamma)=0$, entonces $\Gamma$ es inconsistente, de donde $\Gamma\cup\{\lnot\alpha\}$ también lo es.
                    \item Si $I(\gamma)=1$ para cualquier $\gamma\in \Gamma$, aplicando que $\Gamma\vDash \alpha$ deducimos que\newline $I(\alpha)=1 \Longrightarrow I(\lnot\alpha) = 1 + I(\alpha) = 0$, por lo que $\Gamma\cup\{\lnot\alpha\}$ es inconsistente.
                \end{itemize}
            \item [$2) \Longrightarrow 1)$] Sea $I$ una interpretación de forma que $I(\gamma)=1$ para cualquier $\gamma\in \Gamma$, como $\Gamma\cup\{\lnot\alpha\}$ es inconsistente, deducimos que $I(\lnot\alpha) = 0$, luego $I(\alpha) = 1$.
        \qedhere
        \end{description}
    \end{proof}
\end{prop}

 \subsection{Algoritmo de Davis \& Putnam}
 \begin{definicion}
     Introducimos definiciones que nos serán útiles para llegar al algoritmo de Davis \& Putnam:
     \begin{itemize}
         \item Dada una proposición atómica $a$, entonces decimos que $a$ y $\lnot a$ son \underline{literales}.
         \item Sea $a$ una proposición atómica, denotamos $a^c = \lnot a$ y ${(\lnot a)}^{c} = a$. Para un literal $l$, decimos que $l^c$ es su \underline{complemento}.
         \item Sean $l_1,\ldots,l_n$ literales, entonces decimos que $l_1\lor \ldots \lor l_n$ es una \underline{cláusula}.
         \item Sea $\alpha$ una proposición, decimos que está en \underline{forma normal conjuntiva} (abreviado como fnc) si $\alpha$ es de la forma $c_1\land \ldots \land c_n$, con $c_1,\ldots,c_n$ cláusulas.
         \item A la cláusula sin literales (compuesta por la disyunción de 0 literales) la llamamos \underline{cláusula vacía}, y la denotamos por $\square$.
     \end{itemize}
 \end{definicion}

 \begin{prop}
     Sea $I$ una interpretación, entonces:
     \begin{equation*}
         I(\square) = 0
     \end{equation*}
     \begin{proof}
         Como $\square\lor a = a$ para cualquier proposición atómica $a$, entonces:
         \begin{equation*}
             I(\square\lor a) = I(\square) + I(a) + I(\square)I(a) = I(a)
         \end{equation*}
         De donde deducimos:
         \begin{equation*}
             I(\square) + I(\square)I(a) = I(\square)(1 + I(a)) = 0
         \end{equation*}
         Luego $I(\square)=0$ o $I(a)=1$, pero como la proposición atómica $a$ era arbitraria (y sabemos que hay proposiciones atómicas que no son tautologías), deducimos que ha de ser $I(\square)=0$.
     \end{proof}
 \end{prop}

 \begin{prop}
     Toda proposición puede expresarse en una proposición semánticamente equivalente que se encuentre en forma normal conjuntiva.
     \begin{proof}
         Aunque no lo demostraremos, el lector puede hacerse una idea de que el enunciado es cierto, con ayuda de las siguientes reglas:
         \begin{itemize}
             \item $\lnot\lnot a\equiv a$ (regla de la doble negación)
             \item $\lnot(a\lor b) \equiv \lnot a\land\lnot b$ y $\lnot(a\land b) \equiv \lnot a\lor \lnot b$ (reglas de De Morgan)
             \item $a\to b \equiv \lnot a\lor b$ y $a \leftrightarrow b \equiv (a\to b) \land (b\to a)$
             \item $a\lor (b\land c) \equiv (a\lor b) \land (a\lor c)$ (ley distributiva)
         \end{itemize}
     \end{proof}
 \end{prop}

 \begin{ejemplo}
     Sea $\alpha = (a\to b)\to a$, buscamos una proposición semánticamente equivalente en forma normal conjuntiva. Para ello, primero quitamos $\to$ de la proposición:
     \begin{equation*}
         (a\to b)\to a \equiv \lnot(\lnot a\lor b) \lor a
     \end{equation*}
     Posteriormente, aplicamos la regla de De Morgan, así como la de la doble negación:
     \begin{equation*}
         \lnot(\lnot a\lor b) \lor a \equiv (\lnot\lnot a\land \lnot b) \lor a \equiv (a\land \lnot b)\lor a
     \end{equation*}
     Aplicando la ley distributiva ya llegamos a una proposición semánticamente equivalente en forma normal conjuntiva:
     \begin{equation*}
         (a \lor a) \land (a \lor \lnot b)
     \end{equation*}
     Sin embargo, como $a \lor a \equiv a$, podemos seguir simplificando, obteniendo que:
     \begin{equation*}
         a \land (a \lor \lnot b)
     \end{equation*}
     Pero como:
     \begin{align*}
         I(\alpha \land (\alpha\lor \beta)) &= I(\alpha)(I(\alpha)+I(\beta)+I(\alpha)I(\beta)) \\
                                            &= I(\alpha)I(\alpha) + I(\alpha)I(\beta) + I(\alpha)I(\alpha)I(\beta) \\
                                            &= I(\alpha) + I(\alpha) I(\beta) + I(\alpha)I(\beta) = I(\alpha)
     \end{align*}
     Llegamos finalmente a que:
     \begin{equation*}
         (a\to b)\to a \equiv a
     \end{equation*}
 \end{ejemplo}

 \begin{prop}\label{prop:ej1}
        Dado el conjunto de proposiciones $\{\psi_1, \ldots, \psi_n\}$, son equivalentes:
        \begin{enumerate}
            \item $\{\psi_1, \ldots, \psi_n\}$ es inconsistente.
            \item $\{\psi_1 \land \ldots \land \psi_n\}$ es inconsistente.
        \end{enumerate}
        \begin{proof}
            Demostraremos el resultado mediante una doble implicación.
            \begin{description}
                \item[$\Longrightarrow)$]  Supongamos que $\{\psi_1, \ldots, \psi_n\}$ es inconsistente; y sea $I$ una interpretación arbitraria. Por ser dicho conjunto inconsistente, $\exists \psi_i \in \{\psi_1, \ldots, \psi_n\}$ tal que $I(\psi_i) = 0$. Por tanto:
                \begin{equation*}
                    I\left(\bigwedge_{i=1}^n \psi_i\right) = \prod_{k=1}^n I(\psi_k) = 0
                \end{equation*}
                puesto que uno de los factores ($I(\psi_i$)) es $0$. Por tanto, $\{\psi_1 \land \ldots \land \psi_n\}$ es inconsistente.

                \item[$\Longleftarrow)$] Supongamos que $\{\psi_1 \land \ldots \land \psi_n\}$ es inconsistente; y sea $I$ una interpretación arbitraria. Por ser dicho conjunto inconsistente, tenemos que:
                \begin{equation*}
                    I\left(\bigwedge_{i=1}^n \psi_i\right) = \prod_{k=1}^n I(\psi_k) = 0
                \end{equation*}
                Por tanto, por ser $\bb{Z}_2$ un cuerpo (y en particular un dominio de integridad), tenemos que $\exists \psi_i \in \{\psi_1, \ldots, \psi_n\}$ tal que $I(\psi_i) = 0$. Por tanto, $\{\psi_1, \ldots, \psi_n\}$ es inconsistente.
            \end{description}
        \end{proof}
    \end{prop}


\begin{comment}
 \begin{prop}\label{prop:ej1}
     Dado el conjunto de proposiciones $\{\psi_1,\ldots,\psi_n\}$, son equivalentes:
     \begin{itemize}
         \item $\{\psi_1,\ldots,\psi_n\}$ es inconsistente.
         \item $\{\psi_1\land\ldots\land\psi_n\}$ es inconsistente.
     \end{itemize}
     \begin{proof}
         Notemos que decir que $\{\psi_1\land \ldots \land \psi_n\}$ sea inconsistente significa que dada una interpretación $I$, entonces:
         \begin{equation*}
             I(\psi_1\land \ldots \land \psi_n) = \prod_{k=1}^{n}I(\psi_k) = 0
         \end{equation*}
         $\{\psi_1,\ldots,\psi_n\}$ será inconsistente si y solo si dada una interpretación $I$ hay alguna proposición $\psi_i$ de forma que $I(\psi_i)=0$, si y solo si $\prod\limits_{k=1}^{n}I(\psi_k)=0$, lo que equivale con que $\{\psi_1\land\ldots\land\psi_n\}$ sea inconsistente.
     \end{proof}
 \end{prop}
\end{comment}

 \begin{prop}
     Dado un conjunto de proposiciones $\Gamma$, si consideramos el conjunto $\Gamma'$ resultante de considerar para cada fórmula de $\Gamma$ su forma normal conjuntiva y luego tomar la unión de todas ellas, se tiene que $\Gamma$ es inconsistente si y solo si $\Gamma'$ es inconsistente.
     \begin{proof}
         Puede demostrarse fácilmente usando la Proposición~\ref{prop:ej1}.
     \end{proof}
 \end{prop}

 El último resultado es de gran importancia, ya que recordemos que, si $\Gamma\cup\{\varphi\}$ es un conjunto de proposiciones, que $\Gamma\vDash\varphi$ es equivalente a probar que $\Gamma\cup\{\lnot\varphi\}$ es inconsistente, conjunto que puede transformarse en un conjunto de cláusulas $\Delta$ que será inconsistente si y solo si lo era $\Gamma\cup\{\lnot\varphi\}$.

 De esta forma, notemos que el estudio de que una proposición sea consecuencia lógica de otra se reduce a estudiar si un conjunto de cláusulas es o no inconsistente.

 \subsubsection{El algoritmo}
 El algoritmo de Davis y Putnam consiste en aplicar a un conjunto de cláusulas las reglas que a continuación se exponen, intentando siempre aplicar la primera en el orden en que vienen dadas. Cada conjunto de cláusulas que obtengamos al aplicar las reglas de esta forma será inconsistente si y solo si lo era el original del que proviene (en el caso de la última regla, el conjunto de partida es inconsistente si y solo si lo son los dos conjuntos que se generan después de aplicar la regla):
 \begin{description}
     \item [Regla 1 (regla de las tautologías).] Quítense todas las fórmulas que sean tautologías, es decir, las que contengan un literal y su complementario.
     \item [Regla 2 (regla de los literales).] Si hay una cláusula que es un literal $L$ en $\Delta$, obténgase $\Delta'$ a partir de $\Delta$ eliminando todas las cláusulas de $\Delta$ que contengan a $L$.
         \begin{itemize}
             \item Si $\Delta'$ es el conjunto vacío, entonces $\Delta$ no será inconsistente, ya que el vacío no lo es.
             \item En otro caso, obténgase $\Delta''$ a partir de $\Delta'$ suprimiendo $L^c$ de $\Delta'$.

                 Notemos que si $L^c$ era una cláusula, entonces el resultado de suprimir $L^c$ es $\square$, la cláusula vacía, en cuyo caso, el conjunto $\{\square\}$ es inconsistente.
         \end{itemize}
     \item [Regla 3 (regla de los literales puros).] Si un literal $L$ aparece en algunas cláusulas y $L^c$ no aparece en ninguna, quítense todas las cláusulas conteniendo a $L$.
     \item [Regla 4 (regla de la generalización).] Si una cláusula $C$ tiene todos sus literales en otra $C'$ (es decir $C\subseteq C'$), quítese $C'$.
     \item [Regla 5 (regla de la subdivisión).] Si un literal $L$ y su complementario $L^c$ están presentes en el conjunto de cláusulas, construir dos nuevos conjuntos de cláusulas de la siguiente forma:
         \begin{itemize}
             \item El primero se obtiene quitando todas las cláusulas conteniendo a $L$ y borrando las ocurrencias de $L^c$.
             \item El segundo se obtiene quitando todas las cláusulas conteniendo a $L^c$ y borrando las ocurrencias de $L$.
         \end{itemize}
         En este caso, el conjunto original es inconsistente si y solo si lo son los dos conjuntos resultado de aplicar esta regla.
 \end{description}

 \begin{teo}[Funcionamiento de Davis \& Putnam]\label{teo:func_davis_putnam}
     Sea $\Delta$ un conjunto de cláusulas, $\Delta$ es inconsistente si y solo si lo son todos y cada uno de los conjuntos obtenidos tras aplicar las reglas del algoritmos de Davis y Putnam sobre $\Delta$.
     \begin{proof}
         Para ello, hemos de probar que el conjunto que obtenemos al aplicar cada regla sobre $\Delta$ es inconsistente si y solo si lo es $\Delta$:
         \begin{comment}
         \begin{itemize}
             \item \underline{Regla 1.} En este caso, basta ver que si $\Delta\cup\{\alpha\}$ es un conjunto de cláusulas de forma que $\alpha$ es una tautología, entonces $\Delta\cup\{\alpha\}$ es inconsistente si y solo si lo es $\Delta$. Para ello, supongamos que $\Delta = \{\varphi_1,\ldots,\varphi_n\}$:

                 $\Delta\cup\{\alpha\}$ es inconsistente si y solo si lo es $\{\varphi_1\land\ldots\land\varphi_n\land \alpha\}$, que es inconsistente si y solo si para cualquier interpretación $I$ se tiene que:
                 \begin{equation*}
                     \left(\prod_{k=1}^{n}I(\varphi_k)\right)I(\alpha) = 0
                 \end{equation*}
                 Que es equivalente a que $\prod\limits_{k=1}^{n}I(\varphi_k)=0$, ya que $\alpha$ es una tautología, por lo que $I(\alpha)=1$ para cualquier interpretación $I$. Esta última condición es equivalente con que $\{\varphi_1\land\ldots\land\varphi_k\}$ sea inconsistente, si y solo si lo es $\Delta$.

                 Si ahora el conjunto original tiene $n>1$ tautologías (el caso $n=0$ es trivial y el caso $n=1$ acaba de comprobarse), si quitamos una tautología cada vez, vamos obteniendo conjuntos que son inconsistentes si y solo si lo era el primero, hasta llegar a eliminar todas las tautologías del conjunto. Para este paso puede hacerse una demostración por inducción, aunque consideramos que no es necesario.
             \item \underline{Regla 2.} Sea $L$ un literal y $\Delta$ un conjunto de cláusulas de la forma: 
                 \begin{equation*}
                     \Delta = \{\varphi_1,\ldots,\varphi_m,\psi_1,\ldots,\psi_n,L\}
                 \end{equation*}
                 verificándo que:
                 \begin{itemize}
                     \item $\psi_i$ contiene a $L$ para cualquier $i \in \{1,\ldots,n\}$.
                     \item $\varphi_i$ no contiene a $L$, para todo $i \in \{1,\ldots,m\}$.
                 \end{itemize}
                 Por ser $L$ un literal, habrá alguna interpretación $I$ que lo haga cierto, $I(L)=~1$. En dicho caso, por ser $\psi_i$ una cláusula que contiene a $L$, tendremos $I(\psi_i)=1$ para todo $i \in \{1,\ldots,n\}$. Sea $\Delta' = \{\varphi_1,\ldots,\varphi_m\}$:

                 Si $\Delta'=\emptyset $ (es decir, que no había ningún $\varphi_i$ en $\Delta$), entonces $\Delta$ no era inconsistente, ya que una interpretación que hiciese cierta a $L$ hará cierta cualquier cláusula $\psi_i$, por contener a $L$.

                 En el caso $\Delta'\neq \emptyset $, tendremos $\Delta '' = \{\varphi_1',\ldots,\varphi_m'\}$ de forma que $\varphi_i'$ es $\varphi_i$ suprimiendo $L^c$, para todo $i \in \{1,\ldots,m\}$. Queremos ahora ver que $\Delta ''$ es inconsistente si y solo si lo es $\Delta$:
                 \begin{itemize}
                     \item Si $\Delta''$ es inconsistente, dada una interpretación $I$, se cumplirá $I(\varphi_i')=0$ para cierto $i \in \{1,\ldots,m\}$.
                         \begin{itemize}
                             \item Si $I(L)=1$, entonces $I(L^c)=0$, luego:
                                 \begin{equation*}
                                     I(\varphi_i) = I(\varphi_i'\lor L^c) = 0
                                 \end{equation*}
                             \item En el caso $I(L)=0$, tenemos $L\in \Delta$.
                         \end{itemize}
                         Concluimos que $\Delta$ es inconsistente.
                     \item Si $\Delta$ es inconsistente, supongamos que $\Delta ''$ no es inconsistente, por lo que existe una interpretación $I$ de forma que $I(\varphi_i')=1$ para todo \newline $i \in \{1,\ldots,m\}$. Como $\Delta ''$ no contiene en sus cláusulas ni a $L$ ni a $L^c$, podemos asignar a voluntad el valor de $I(L)$ (ya que su valor no cambia que $I(\varphi_i')=1$). De esta forma, si tomamos $I(L)=1$, entonces:
                         \begin{equation*}
                             I(\varphi_i) = I(\varphi_i'\lor L^c) = 1 \text{\ para todo\ } i \in \{1,\ldots,m\}
                         \end{equation*}
                         Como $\psi_i$ contiene a $L$, tenemos $I(\psi_i)=1$ para todo $i \in \{1,\ldots,n\}$. Hemos llegado a una contradicción, ya que $\Delta$ era inconsistente, luego $\Delta ''$ es inconsistente.
                 \end{itemize}
             \item \underline{Regla 3.} Sea $\Delta = \{\varphi_1,\ldots,\varphi_m,\psi_1,\ldots,\psi_n\}$ de forma que:
                 \begin{itemize}
                     \item $\psi_i$ contiene a $L$ y no a $L^c$, para todo $i \in \{1,\ldots,n\}$.
                     \item $\varphi_i$ no contiene ni a $L$ ni a $L^c$, para todo $i \in \{1,\ldots,m\}$.
                 \end{itemize}
                 Sea $\Delta' = \{\varphi_1,\ldots,\varphi_m\}$, queremos ver que $\Delta$ es inconsistente si y solo si lo es $\Delta'$:
                 \begin{itemize}
                     \item Si $\Delta'$ es inconsistente, por ser $\Delta'\subseteq \Delta$, tenemos que $\Delta$ también será inconsistente.
                     \item Si $\Delta$ es inconsistente, podemos repetir el razonamiento que hicimos en la regla 2: supongamos que $\Delta'$ no es inconsistente, con lo que existirá una interpretación $I$ de forma que $I(\varphi_i)=1$ para todo $i \in \{1,\ldots,m\}$. Como $\Delta'$ no contiene ni a $L$ ni a $L^c$ en ninguna de sus cláusulas, podemos asignar a voluntad el valor de $I(L)$. Si tomamos $I(L)=1$, entonces tendremos que $I(\psi_i)=1$ para todo $i \in \{1,\ldots,n\}$, por ser $\psi_i$ cláusulas que contienen a $L$. Hemos llegado a una contradicción, ya que $\Delta$ era inconsistente. Concluimos que $\Delta'$ ha de ser inconsistente.
                 \end{itemize}
             \item \underline{Regla 4.} Sea $\Delta\cup\{C,C'\}$ un conjunto de cláusulas de forma que $C$ tiene todos sus literales en $C'$, veamos que $\Delta\cup\{C,C'\}$ es inconsistente si y solo si lo es $\Delta\cup\{C\}$:
                 \begin{itemize}
                     \item Si $\Delta\cup\{C\}$ es inconsistente, por ser $\Delta\cup\{C\}\subseteq \Delta\cup\{C,C'\}$, entonces tenemos que este segundo también es inconsistente.
                     \item Si $\Delta\cup\{C,C'\}$ es inconsistente, dada una interpretación $I$, existirá una cláusula $\varphi \in \Delta\cup\{C,C'\}$ de modo que $I(\varphi)=0$:
                         \begin{itemize}
                             \item Si $\varphi\neq C'$, entonces $\varphi \in \Delta\cup\{C\}$ con $I(\varphi)=0$:
                             \item Si $\varphi=C'$, como $I(C')=0$ y $C'$ contiene más literales que $C$, ha de ser $I(C) =0$ (ya que si no tendríamos $I(C')=1$). Por tanto, tenemos $C \in \Delta\cup\{C\}$ con $I(C)=0$.
                         \end{itemize}
                         Concluimos que $\Delta\cup\{C\}$ es inconsistente.
                 \end{itemize}
             \item \underline{Regla 5.} Se deja como ejercicio.
         \end{itemize}
         \end{comment}
         \begin{description}
                \item[Regla 1.] Sea $\alpha$ una tautología y $\Delta$ un conjunto de cláusulas.
                Veamos que $\Delta\cup \{\alpha\}$ es inconsistente si y solo si $\Delta$ lo es.
                Por el Ejercicio~\ref{ej:1}, sabemos que:
                \begin{equation*}
                    \Delta\cup \{\alpha\}\ \text{es inconsistente} \iff \left(\bigvee_{\delta \in \Delta} \delta\right) \land \alpha\ \text{es inconsistente}
                \end{equation*}

                Sea ahora $I$ una interpretación arbitaria. Tenemos que $\Delta\cup \{\alpha\}$ es inconsistente si y solo si:
                \begin{equation*}
                    0 = I\left(\left(\bigvee_{\delta \in \Delta} \delta\right) \land \alpha\right) = I\left(\bigvee_{\delta \in \Delta} \delta\right)\cdot I(\alpha)
                \end{equation*}

                Por ser $\alpha$ una tautología, $I(\alpha) = 1$. Por tanto, por ser $\bb{Z}_2$ un DI, tenemos que:
                \begin{equation*}
                    0 = I\left(\bigvee_{\delta \in \Delta} \delta\right)
                \end{equation*}

                Por tanto, hemos probado que:
                \begin{equation*}
                    \left(\bigvee_{\delta \in \Delta} \delta\right) \land \alpha\ \text{es inconsistente}
                    \iff \left(\bigvee_{\delta \in \Delta} \delta\right)\ \text{es inconsistente}
                \end{equation*}

                Por último, usando de nuevo el Ejercicio~\ref{ej:1}, tenemos que:
                \begin{equation*}
                    \Delta\ \text{es inconsistente} \iff \left(\bigvee_{\delta \in \Delta} \delta\right)\ \text{es inconsistente}
                \end{equation*}

                Por tanto, hemos probado que $\Delta\cup \{\alpha\}$ es inconsistente si y solo si $\Delta$ lo es.\\

                Mediante una sencilla e inmediata inducción sobre el número de tautologías en $\Delta$, se prueba que un conjunto de cláusulas $\Delta$ es inconsistente si y solo si $\Delta'$ lo es, donde:
                \begin{equation*}
                    \Delta'=\Delta\setminus \left\{\delta\in \Delta\mid \delta\ \text{es una tautología}\right\}
                \end{equation*}

                \item[Regla 2.] Sea $L$ un literal, y sea $\Delta$ un conjunto de cláusulas tal que $L\in \Delta$.
                Definimos:
                \begin{align*}
                    \Delta'&= \Delta\setminus \{\alpha\in \Delta \mid \alpha\lor L=\alpha\}\\
                    &= \Delta\setminus \{\alpha\in \Delta \mid \alpha\ \text{contiene a}\ L\}
                \end{align*}

                Si $\Delta'=\emptyset$, consideramos una interpretación $I$ tal que $I(L)=1$.
                De esta forma, tenemos que para cada $\alpha\in \Delta$:
                \begin{equation*}
                    \alpha=\alpha\lor L
                    \Longrightarrow
                    I(\alpha)=I(\alpha\lor L)=I(\alpha)+I(L)+I(\alpha)\cdot I(L)=2I(\alpha)+1=1
                \end{equation*}
                Por tanto, hemos encontrado una interpretación $I$ tal que $I(\alpha)=1$ para todo $\alpha\in \Delta$. Por tanto, $\Delta$ no es inconsistente.\\

                Supongamos ahora que $\Delta'\neq \emptyset$. Consideramos el conjunto:
                \begin{align*}
                    \Delta''&= \{ \alpha\in \Delta' \mid \alpha\lor L^c\neq \alpha\} \cup \{ \alpha\mid (\alpha\lor L^c\neq \alpha) \ \land \ (\exists \alpha'\in \Delta'\ \text{con}\ \alpha'=\alpha\lor L^c)\}
                \end{align*}
                Notemos que este conjunto resulta de, partiendo de $\Delta'$, añadimos las cláusulas que no contienen a $L^c$ y, de las que lo tienen, eliminamos $L^c$.

                Demostramos que $\Delta$ es inconsistente si y solo si $\Delta''$ lo es.
                \begin{description}
                    \item[$\Longrightarrow)$] Demostraremos el recíproco. Supongamos que $\Delta''$ no es inconsistente, por lo que existe una interpretación $\wt{I}$ tal que $\wt{I}(\alpha)=1$ para todo $\alpha\in \Delta''$.

                    Recordemos que, para definir una interpretación, basta con asignarle imágenes a las proposiciones atómicas. Equivalentemente, podemos definir una interpretación asignando valores a los literales, aunque hemos de asegurarnos de que $I(\lm)=1+I(\lm^c)$ para todo literal $\lm$.
                    Consideramos por tanto la interpretación $I$ tal que:
                    \begin{align*}
                        I(L)&=1\\
                        I(\lm)&=\wt{I}(\lm)\ \text{para todo literal}\ \lm\ \text{tal que} \ \left(\bigvee_{\alpha\in \Delta''} \alpha\right)=\lm\lor \left(\bigvee_{\alpha\in \Delta''} \alpha\right)
                    \end{align*}
                    Notemos que simplemente hemos definido $I$ en $L$ y en los literales que aparecen en $\Delta''$. El valor que tomen el resto de literales no nos será de relevacia. Notemos además que está bien definida, puesto que $L$ no aparece en $\Delta'$ y por tanto tampoco en $\Delta''$.\\

                    Por tanto, para cada $\alpha\in \Delta$:
                    \begin{itemize}
                        \item Si $\alpha=\alpha\lor L$ ($L$ aparece en $\alpha$), entonces:
                        \begin{equation*}
                            I(\alpha)=I(\alpha\lor L)=I(\alpha)+I(L)+I(\alpha)\cdot I(L)=2I(\alpha)+1=1
                        \end{equation*}

                        \item Si $\alpha\neq \alpha\lor L$ ($L$ no aparece en $\alpha$), entonces $\alpha\in \Delta'$.
                        \begin{itemize}
                            \item Si $\alpha=\alpha\lor L^c$ ($L^c$ aparece en $\alpha$):
                            
                            Entonces consideramos $\alpha'\in \Delta''$ resultante de eliminar $L^c$ de $\alpha$; es decir, $\alpha=\alpha'\lor L^c$. Como $\alpha'\in \Delta''$, tenemos que $I(\alpha')=1$. Por tanto:
                            \begin{equation*}
                                I(\alpha)=I(\alpha'\lor L^c)=I(\alpha')+I(L^c)+I(\alpha')\cdot I(L^c)=I(\alpha')=1
                            \end{equation*}

                            \item Si $\alpha\neq \alpha\lor L^c$ ($L^c$ no aparece en $\alpha$):
                            
                            Entonces, $\alpha\in \Delta''$ y por tanto $I(\alpha)=1$.
                        \end{itemize}
                    \end{itemize}

                    Por tanto, hemos encontrado una interpretación $I$ tal que $I(\alpha)=1$ para todo $\alpha\in \Delta$. Por tanto, $\Delta$ no es inconsistente.\\

                    Por el recíproco, tenemos que si $\Delta$ es inconsistente, entonces $\Delta''$ también lo es.

                    \item[$\Longleftarrow)$] Supongamos ahora que $\Delta''$ es inconsistente, y consideramos una interpretación $I$ arbitraria. Veamos que existe $\gamma\in \Delta$ tal que $I(\gamma)=0$.
                    \begin{itemize}
                        \item Si $I(L)=0$, como $L\in \Delta$ basta con considerar $\gamma=L$.
                        \item Si $I(L)=1$, entonces $I(L^c)=0$.
                        
                        Por ser $\Delta''$ inconsistente, tenemos que $\exists \alpha\in \Delta''$ tal que $I(\alpha)=0$.
                        \begin{itemize}
                            \item Si $\alpha\in \Delta'\subset \Delta$, entonces basta con considerar $\gamma=\alpha$.
                            \item Si $\alpha\notin \Delta'$, entonces $\exists \alpha'\in \Delta'\subset \Delta$ tal que $\alpha'=\alpha\lor L^c$.
                            Entonces:
                            \begin{equation*}
                                I(\alpha')=I(\alpha\lor L^c)=I(\alpha)+I(L^c)+I(\alpha)\cdot I(L^c)=0
                            \end{equation*}
                            Por tanto, basta con considerar $\gamma=\alpha'$.
                        \end{itemize}
                    \end{itemize}

                    Por tanto, hemos probado que $\exists \gamma\in \Delta$ tal que $I(\gamma)=0$. Por tanto, $\Delta$ es inconsistente.
                \end{description}

                \item[Regla 3.] Sea $L$ un literal, y sea $\Delta$ un conjunto de cláusulas verificando:
                \begin{itemize}
                    \item $\exists \alpha\in \Delta$ tal que $\alpha=\alpha\lor L$.
                    \item $\nexists \alpha\in \Delta$ tal que $\alpha=\alpha\lor L^c$.
                \end{itemize}

                Consideramos el conjunto:
                \begin{equation*}
                    \Delta'=\{\alpha\in \Delta\mid \alpha\neq \alpha\lor L\}
                \end{equation*}

                Demostramos que $\Delta$ es inconsistente si y solo si $\Delta'$ lo es.
                \begin{description}
                    \item[$\Longrightarrow)$] Demostraremos el recíproco. Supongamos que $\Delta'$ no es inconsistente, por lo que existe una interpretación $\wt{I}$ tal que $\wt{I}(\alpha)=1$ para todo $\alpha\in \Delta'$.
                    
                    Al igual que hicimos en la Regla 2, consideramos la interpretación $I$ tal que:
                    \begin{align*}
                        I(L)&=1\\
                        I(\lm)&=\wt{I}(\lm)\ \text{para todo literal}\ \lm\ \text{tal que} \ \left(\bigvee_{\alpha\in \Delta'} \alpha\right)=\lm\lor \left(\bigvee_{\alpha\in \Delta'} \alpha\right)
                    \end{align*}

                    Por tanto, para cada $\alpha\in \Delta$:
                    \begin{itemize}
                        \item Si $\alpha=\alpha\lor L$ ($L$ aparece en $\alpha$), entonces:
                        \begin{equation*}
                            I(\alpha)=I(\alpha\lor L)=I(\alpha)+I(L)+I(\alpha)\cdot I(L)=2I(\alpha)+1=1
                        \end{equation*}

                        \item Si $\alpha\neq \alpha\lor L$ ($L$ no aparece en $\alpha$), entonces $\alpha\in \Delta'$. Por tanto, $I(\alpha)=1$.
                    \end{itemize}

                    Por tanto, hemos encontrado una interpretación $I$ tal que $I(\alpha)=1$ para todo $\alpha\in \Delta$. Por tanto, $\Delta$ no es inconsistente.\\

                    Por el recíproco, tenemos que si $\Delta$ es inconsistente, entonces $\Delta'$ también lo es.

                    \item[$\Longleftarrow)$] Supongamos ahora que $\Delta'$ es inconsistente. Como $\Delta'\subset \Delta$, tenemos que $\Delta$ también lo es.
                \end{description}

                \item[Regla 4.] Sea $\Delta$ un conjunto de cláusulas, y sean $C,C'\in \Delta$ dos cláusulas tales que $C'=C'\lor C$; es decir, todos los literales de $C$ están en $C'$.
                Consideramos el conjunto:
                \begin{equation*}
                    \Delta'=\Delta\setminus \{C'\}
                \end{equation*}
                
                Demostramos que $\Delta$ es inconsistente si y solo si $\Delta'$ lo es.
                \begin{description}
                    \item[$\Longrightarrow)$] Supongamos que $\Delta$ es inconsistente. Entonces, para cada interpretación $I$ existe $\alpha\in \Delta$ tal que $I(\alpha)=0$.
                    
                    Veamos que $\exists \gamma\in \Delta'$ tal que $I(\gamma)=0$.
                    \begin{itemize}
                        \item Si $\alpha\neq C'$, entonces $\alpha\in \Delta'$ y basta con considerar $\gamma=\alpha$.
                        
                        \item Si $\alpha=C'$, entonces:
                        \begin{equation*}
                            0=I(C')=I(C'\lor C)=I(C)+I(C')+I(C)\cdot I(C')=I(C)+0+0=I(C)
                        \end{equation*}

                        Por tanto, basta con considerar $\gamma=C$.
                    \end{itemize}

                    Por tanto, hemos probado que $\exists \gamma\in \Delta'$ tal que $I(\gamma)=0$. Por tanto, $\Delta'$ es inconsistente.

                    \item[$\Longleftarrow)$] Supongamos ahora que $\Delta'$ es inconsistente. Como $\Delta'\subset \Delta$, tenemos que $\Delta$ también lo es.
                \end{description}
            \end{description}
     \end{proof}
 \end{teo}

 \begin{ejemplo}
     Veamos ahora dos ejemplos de uso del algoritmo de Davis \& Putnam:
     \begin{enumerate}
         \item En este caso, vamos a demostrar que el conjunto:
             \begin{equation*}
                 \{P\lor Q, \lnot P\lor Q, P\lor \lnot Q, \lnot P\lor \lnot Q, P \lor \lnot P, P \lor Q \lor R, P \lor Q \lor \lnot R, \lnot P \lor S\}
             \end{equation*}
             es inconsistente. Como se trata de un conjunto de cláusulas, podemos aplicar directamente el algoritmo de Davis \& Putnam. En primer lugar, aplicamos la regla 1 (ya que $P\lor \lnot P$ es una tautología), obteniendo:
             \begin{equation*}
                 \{P\lor Q, \lnot P\lor Q, P\lor \lnot Q, \lnot P\lor \lnot Q, P \lor Q \lor R, P \lor Q \lor \lnot R, \lnot P \lor S\}
             \end{equation*}
             Ahora no podemos aplicar ni la regla 1 ni la 2, pero sí la 3 al literal $S$, obteniendo:
             \begin{equation*}
                 \{P\lor Q, \lnot P\lor Q, P\lor \lnot Q, \lnot P\lor \lnot Q, P \lor Q \lor R, P \lor Q \lor \lnot R\}
             \end{equation*}
             La primera regla que podemos aplicar ahora es la cuarta, ya que tenemos $P\lor Q$, $P\lor Q\lor R$ y $P\lor Q\lor \lnot R$; obteniendo:
             \begin{equation*}
                 \{P\lor Q, \lnot P\lor Q, P\lor \lnot Q, \lnot P \lor \lnot Q\}
             \end{equation*}
             Y ahora la única regla que podemos aplicar es la quinta, que si la aplicamos al literal $P$ obtenemos:
             \begin{equation*}
                 \Delta_1 = \{Q, \lnot Q\}, \qquad \Delta_2 = \{Q, \lnot Q\}
             \end{equation*}
             Finalmente, aplicando la segunda regla a cada uno de los dos conjuntos (notemos que son iguales), obtenemos:
             \begin{equation*}
                 \{\square\}, \qquad \{\square\}
             \end{equation*}
             De donde deducimos que el conjunto de partida era inconsistente si y solo si lo son $\{\square\}$ y $\{\square\}$, que efectivamente, son inconsistentes.
         \item \label{ej:DyP_3}
         Usando el algoritmo de Davis \& Putnam y el Teorema de la Deducción, buscamos demostrar que:
             \begin{equation*}
                 \vDash (\alpha\to\gamma)\to ((\beta\to\gamma)\to(\alpha\lor\beta\to\gamma))
             \end{equation*}
             
             \begin{comment}
             Para ello, si aplicamos 3 veces el Teorema de la Deducción (\ref{teo:deduccion}), llegamos a que demostrar lo de arriba es equivalente a ver que:
             \begin{equation*}
                 \{\alpha\to\gamma, \beta\to\gamma,\alpha\lor\beta\}\vDash \gamma
             \end{equation*}
             Que sabemos que a su vez es equivalente a ver que el conjunto:
             \begin{equation*}
                 \{\alpha\to\gamma, \beta\to\gamma,\alpha\lor\beta,\lnot\gamma\}
             \end{equation*}
             es inconsistente. Para ello, aplicaremos el algoritmo de Davis \& Putnam sobre este conjunto. En primer lugar, tenemos que convertir todas las proposiciones a cláusulas:
             \begin{equation*}
                 \alpha\to\gamma\equiv \lnot\alpha\lor \gamma \qquad \beta\to\gamma\equiv \lnot\beta\lor \gamma
             \end{equation*}
             Por lo que aplicaremos el algoritmo sobre el conjunto:
             \begin{equation*}
                 \Delta = \{\lnot\alpha\lor\gamma,\lnot\beta\lor\gamma,\alpha\lor\beta,\lnot\gamma\}
             \end{equation*}
             Aplicando ya el algoritmo, no podemos aplicar la regla 1, pero sí la 2 sobre $L=\lnot\gamma$, obteniendo:
             \begin{equation*}
                 \{\lnot\alpha\lor\gamma,\lnot\beta\lor\gamma,\alpha\lor\beta\} \neq \emptyset 
             \end{equation*}
             Si ahora eliminamos las ocurrencias de $L^c=\gamma$:
             \begin{equation*}
                 \{\lnot\alpha,\lnot\beta,\alpha\lor\beta\}
             \end{equation*}
             Llegamos a un conjunto en el que tampoco podemos aplicar la regla 1, pero sí podemos volver a aplicar la 2 sobre $L=\lnot\alpha$: $\{\lnot\beta,\alpha\lor\beta\}\neq \emptyset $, con lo que:
             \begin{equation*}
                 \{\lnot\beta,\beta\}
             \end{equation*}
             Sobre el que finalmente podemos volver a aplicar la regla 2: $\{\beta\}\neq \emptyset $, con lo que obtenemos finalmente $\{\square\}$.

             Por el Teorema~\ref{teo:func_davis_putnam}, sabemos que $\Delta$ es inconsistente si y solo si lo es $\{\square\}$, que sí es inconsistente.
             \end{comment}

             Aplicando tres veces el Teorema de la Deducción, eso equivale a demostrar que:
            \begin{equation*}
                \{\alpha \rightarrow \gamma, \beta \rightarrow \gamma, \alpha \vee \beta\} \vDash \gamma
            \end{equation*}

            Además, sabemos que demostrar esa consecuencia lógica equivale a probar que el siguiente conjunto es inconsistente:
            \begin{equation*}
                \{\alpha \rightarrow \gamma, \beta \rightarrow \gamma, \alpha \vee \beta, \neg \gamma\}
            \end{equation*}

            Para poder aplicar el Algoritmo de Davis-Putnam, necesitamos transformar las fórmulas en cláusulas. De esta forma:
            \begin{align*}
                \alpha \rightarrow \gamma &\equiv \neg \alpha \vee \gamma\\
                \beta \rightarrow \gamma &\equiv \neg \beta \vee \gamma
            \end{align*}

            Por tanto, el conjunto de cláusulas sobre el cual aplicaremos el Algoritmo de Davis-Putnam (y el cual será inconsistente si y solo si la consecuencia lógica de partida es cierta) es:
            \begin{equation*}
                \Delta = \{\neg \alpha \vee \gamma, \neg \beta \vee \gamma, \alpha \vee \beta, \neg \gamma\}
            \end{equation*}

            Por la Proposición~\ref{prop:ej1}, sabemos que $\Delta$ es inconsistente si y solo si lo es el conjunto obtenido tras aplicar el Algoritmo de Davis-Putnam.
            En la Figura~\ref{fig:DyP_3} se tiene que dicho conjunto es $\Delta_3=\{\square\}$. Por tanto:
            \begin{equation*}
                \Delta\ \text{es inconsistente} \iff \Delta_3=\{\square\}\ \text{es inconsistente}
            \end{equation*}

            Por tanto, como $\Delta_3$ es inconsistente, tenemos que $\Delta$ también lo es. Por tanto, tenemos probado que:
            \begin{equation*}
                \vDash \left(\alpha \rightarrow \gamma\right) \rightarrow \left(\beta \rightarrow \gamma\right) \rightarrow \left(\alpha \vee \beta \rightarrow \gamma\right)
            \end{equation*}

            \begin{figure}
                \centering
                \begin{forest}
                    for tree={draw=none, minimum size=2em, l=2cm, s sep=5mm, align=center, edge={-stealth}}
                    [
                        $\Delta {=} \{\neg \alpha \vee \gamma; \neg \beta \vee \gamma; \alpha \vee \beta; \neg \gamma\}$\\ \\
                        \fbox{R2. $\lm=\neg \gamma$. Eliminamos las cláusulas que contienen a $\lm$}
                        [
                            $\Delta' {=} \{\neg \alpha \vee \gamma; \neg \beta \vee \gamma; \alpha \vee \beta\}{\color{red}~\neq \emptyset}$\\ \\
                            \fbox{Cont R2. $\lm^c=\gamma$. Eliminamos las ocurrencias de $\lm^c$ (no las cláusulas)}
                            [
                                $\Delta_1 {=} \{\neg \alpha; \neg \beta; \alpha \vee \beta\}$\\ \\
                                \fbox{R2. $\lm=\neg \alpha$. Eliminamos las cláusulas que contienen a $\lm$}
                                [
                                    $\Delta_1' {=} \{\neg \beta; \alpha \vee \beta\}{\color{red}~\neq \emptyset}$\\ \\
                                    \fbox{Cont R2. $\lm^c=\alpha$. Eliminamos las ocurrencias de $\lm^c$ (no las cláusulas)}
                                    [
                                        $\Delta_2 {=} \{\neg \beta; \beta\}$\\ \\
                                        \fbox{R2. $\lm=\neg \beta$. Eliminamos las cláusulas que contienen a $\lm$}
                                        [
                                            $\Delta_2' {=} \{\beta\}{\color{red}~\neq \emptyset}$\\ \\
                                            \fbox{Cont R2. $\lm^c=\beta$. Eliminamos las ocurrencias de $\lm^c$ (no las cláusulas)}
                                            [
                                                $\Delta_3 {=} \{\square\}$\\ \\
                                            ]
                                        ]
                                    ]
                                ]
                            ]
                        ]
                    ]
                \end{forest}
                \caption{Algoritmo de Davis y Putnam del apartado~\ref{ej:DyP_3}.}
                \label{fig:DyP_3}
            \end{figure}
     \end{enumerate}
 \end{ejemplo}

\section{Demostraciones}
\begin{definicion}[Demostración]
    Sean $\mathcal{A}$ y $\Gamma\cup\{p\}$ dos conjuntos de proposiciones (nos referiremos al conjunto $\mathcal{A}$ como ``conjunto de axiomas'' y a $\Gamma$ como ``conjunto de hipótesis''), una demostración de $p$ a partir de $\Gamma$ (notado por $\Gamma\vdash p$) es una secuencia de proposiciones $\alpha_1,\alpha_2,\ldots,\alpha_n$ de forma que $\alpha_n=p$ y se verifica para todo $i$ menor o igual que $n$:
    \begin{itemize}
        \item bien $\alpha_i \in \mathcal{A}\cup\Gamma$.
        \item bien existen $j,k$ naturales con $j<k<i$ siendo $\alpha_k = \alpha_j\to \alpha_i$.

        En este caso, diremos que se tiene $\alpha_i$ por modus ponens de $j$ y $k$.
    \end{itemize}
\end{definicion}

\begin{notacion}
    Si $p$ es una proposición de forma que $\emptyset \vdash p$, podremos notar $\vdash p$ y diremos que $p$ es un teorema.
\end{notacion}

\begin{ejemplo}
    Como ejemplo de demostración, veamos que $\{\alpha,\alpha\to\beta\}\vdash \beta$ (regla conocida como ``Modus ponens''). Para ello, consideramos:
    \begin{align*}
        \alpha_1 &= \alpha \\
        \alpha_2 &= \alpha\to\beta \\
        \alpha_3 &= \beta
    \end{align*}
    Como vemos, es una demostración de $\beta$ a partir de $\{\alpha,\alpha\to\beta\}$ porque $\alpha_1,\alpha_2,\alpha_3$ son proposiciones, $\alpha_3=\beta$ y:
    \begin{itemize}
        \item $\alpha_1\in \Gamma$.
        \item $\alpha_2\in \Gamma$.
        \item $1,2<3$ y $\alpha_2 = \alpha_1\to \alpha_3$.
    \end{itemize}
\end{ejemplo}

\begin{notacion}
    Para abreviar las demostraciones, a partir de ahora no daremos una secuencia numerada de proposiciones $\alpha_1,\ldots,\alpha_n$, sino que numeraremos los pasos de la demostración y entenderemos que para formalizarla totalmente debemos coger como $\alpha_i$ el paso $i-$ésimo de la demostración.

    Más aún, para no pararnos a comprobar las condiciones abstractas que han de cumplir cada una de las propiedades de la demostrción, incluiremos junto a los pasos de la demostración un comentario sobre por qué dicho paso es válido.

    Con esta notación, la demostración de $\{\alpha,\alpha\to\beta\}\vdash \beta$ quedaría de la forma:
    \begin{enumerate}
        \item $\alpha$ es una hipótesis.
        \item $\alpha\to\beta$ es una hipótesis.
        \item $\beta$ por Modus Ponens de 1 y 2.
    \end{enumerate}
\end{notacion}~\\

\noindent
Finalmente, como conjunto $\mathcal{A}$ de axiomas, consideraremos:
\begin{equation*}
    \mathcal{A} = \mathcal{A}_1 \cup \mathcal{A}_2 \cup \mathcal{A}_3
\end{equation*}
Con:
\begin{align*}
    \mathcal{A}_1 &= \{\alpha\to(\beta\to\alpha) : \alpha,\beta \text{\ son proposiciones}\} \\
    \mathcal{A}_2 &= \{(\alpha\to(\beta\to\gamma))\to((\alpha\to\beta)\to(\alpha\to\gamma)) : \alpha,\beta,\gamma \text{\ son proposiciones}\} \\
    \mathcal{A}_3 &= \{(\lnot\alpha\to\lnot\beta)\to((\lnot\alpha\to\beta)\to\alpha) : \alpha,\beta \text{\ son proposiciones}\}
\end{align*}

\begin{ejemplo}
    Ejemplos de algunas demostraciones:
    \begin{itemize}
        \item $\{\alpha\}\vdash \beta\to\alpha$
            \begin{enumerate}
                \item $\alpha\to(\beta\to\alpha)\in \mathcal{A}_1$
                \item $\alpha$ es una hipótesis
                \item $\beta\to\alpha$ Modus ponens de 1 y 2.
            \end{enumerate}
        \item $\vdash \alpha\to\alpha$
            \begin{enumerate}
                \item $(\alpha\to((\alpha\to\alpha)\to\alpha))\to((\alpha\to(\alpha\to\alpha))\to(\alpha\to\alpha))\in \mathcal{A}_2$
                \item $\alpha\to((\alpha\to\alpha)\to\alpha)\in \mathcal{A}_1$
                \item $(\alpha\to(\alpha\to\alpha))\to(\alpha\to\alpha)$ Modus ponens de 1 y 2
                \item $\alpha\to(\alpha\to\alpha)\in \mathcal{A}_1$
                \item $\alpha\to\alpha$ Modus ponens de 3 y 4
            \end{enumerate}
    \end{itemize}
\end{ejemplo}

\begin{teo}[de Herbrand o de la deducción]\label{teo:herbrand}
    Sea $\Gamma\cup\{\alpha,\beta\}$ un conjunto de proposiciones, equivalen:
    \begin{enumerate}
        \item $\Gamma\vdash \alpha\to\beta$
        \item $\Gamma\cup\{\alpha\}\vdash \beta$
    \end{enumerate}
    \begin{proof}
        Demostramos las dos implicaciones:
        \begin{description}
            \item [$1) \Longrightarrow 2)$]
                Como $\Gamma\vdash \alpha\to\beta$, podemos construir una demostración de $n$ pasos de la proposición $\alpha\to\beta$ a partir de $\Gamma$. En cuyo caso, podemos añadir 2 pasos más a su demostración, de forma que:
                \begin{enumerate}
                    \item \ldots \\
                        \vdots 
                    \item[$n$.] $\alpha\to\beta$
                    \item[$n+1$.] $\alpha$ es hipótesis
                    \item[$n+2$.] $\beta$ por Modus ponens de $n$ y $n+1$
                \end{enumerate}
                Como en los $n$ primeros pasos solo hemos usado como hipótesis $\Gamma$, hemos conseguido demostrar en $n+2$ pasos que $\Gamma\cup\{\alpha\}\vdash \beta$.
            \item [$2) \Longrightarrow 1)$] 
                Como $\Gamma\cup\{\alpha\}\vdash \beta$, podemos obtener una demostración $\beta$ a partir de $\Gamma\cup\{\alpha\}$ de $n$ pasos: $\beta_1,\ldots,\beta_n$ (con $\beta_n=\beta$). Por inducción sobre $n$ (el número de pasos de la demostración):
                \begin{itemize}
                    \item \underline{Si $n=1$}: Como $\Gamma\cup\{\alpha\}\vdash \beta$ gracias a la demostración $\beta_1=\beta$, distinguimos casos:
                        \begin{enumerate}[label=(\alph*)]
                            \item $\beta_1\in \mathcal{A}$. En dicho caso, podemos considerar la demostración:
                                \begin{enumerate}[label=\arabic*.]
                                    \item $\beta_1\in \mathcal{A}$
                                    \item $\beta_1\to(\alpha\to\beta_1)\in \mathcal{A}_1$
                                    \item $\alpha\to\beta_1$ por Modus ponens de 1 y 2
                                \end{enumerate}
                                Y con esto tenemos que $\Gamma\vdash \alpha\to\beta$.
                            \item $\beta_1\in \Gamma$. En dicho caso, podemos considerar una demostración similar al caso anterior:
                                \begin{enumerate}[label=\arabic*.]
                                    \item $\beta_1\in \Gamma$
                                    \item $\beta_1\to(\alpha\to\beta_1)\in \mathcal{A}_1$
                                    \item $\alpha\to\beta_1$ por Modus ponens de 1 y 2
                                \end{enumerate}
                                Y con esto también tenemos que $\Gamma\vdash \alpha\to\beta$.
                            \item $\beta_1=\alpha$. En dicho caso, podemos copiar la demostración de $\vdash \beta\to\beta$ del ejemplo anterior, llegando a que $\Gamma\vdash \alpha\to\beta$.
                        \end{enumerate}
                    \item \underline{En el paso de inducción}, supuesto que de $\Gamma\cup\{\alpha\}\vdash \beta_m$ podemos deducir que $\Gamma\vdash \alpha\to\beta_m$ para todo $m\leq n$, suponemos ahora que $\Gamma\cup\{\alpha\}\vdash \beta_{n+1}$ y queremos ver que $\Gamma\vdash \alpha\to\beta_{n+1}$.

                        En dicho caso, supuesto que $\beta_{m+1}\notin \mathcal{A}\cup\Gamma\cup\{\alpha\}$ (ya que si no la demostración es análoga al caso $n=1$), la única posibilidad es que hayan de existir $i,j<n+1$ con $\beta_i=\gamma$ y $\beta_j=\gamma\to\beta_{m+1}$.

                        Si ahora consideramos los $i$ primeros pasos de la demostración, tenemos que $\Gamma\cup\{\alpha\}\vdash \gamma$ y si consideramos los $j$ primeros pasos, tenemos que $\Gamma\cup\{\alpha\}\vdash \gamma\to\beta_{n+1}$. Por hipótesis de inducción, como $i,j<n+1$, tenemos que $\Gamma\vdash \alpha\to\gamma$ y que $\Gamma\vdash \alpha\to(\gamma\to\beta_{n+1})$. En este momento, podemos realizar la demostración (con hipótesis $\Gamma$):
                        \begin{enumerate}
                            \item[$1$.] \ldots \\
                                \vdots
                            \item[$p$.] $\alpha\to\gamma$
                            \item[$p+1$.] \ldots \\
                                \vdots
                            \item[$q$.] $\alpha\to(\gamma\to\beta_{n+1})$
                            \item[$q+1$.] $(\alpha\to(\gamma\to\beta_{n+1}))\to((\alpha\to\gamma)\to(\alpha\to\beta_{n+1}))\in \mathcal{A}_2$
                            \item[$q+2$.] $(\alpha\to\gamma)\to(\alpha\to\beta_{n+1})$ por Modus ponens de $q$ y $q+1$.
                            \item[$q+3$.] $\alpha\to\beta_{n+1}$ por Modus ponens de $p$ y $q+2$.
                        \end{enumerate}
                \end{itemize}
        \end{description}
    \end{proof}
\end{teo}

\subsection{Resultados útiles a la hora de realizar demostraciones}
\begin{prop}[Regla de reducción al absurdo clásica]
    Sea $\Gamma\cup\{\alpha,\beta\}$ un conjunto de proposiciones: si $\Gamma\cup\{\lnot\alpha\}\vdash \beta$ y $\Gamma\cup\{\lnot\alpha\}\vdash \lnot\beta$, entonces $\Gamma\vdash \alpha$.
    \begin{proof}
        Supuesto que $\Gamma\cup\{\lnot\alpha\}\vdash \beta$ y que $\Gamma\cup\{\lnot\alpha\}\vdash \lnot\beta$, por el Teorema de Herbrand (\ref{teo:herbrand}), se tiene que $\Gamma\vdash \lnot\alpha\to\beta$ y que $\Gamma\vdash \lnot\alpha\to\lnot\beta$. En dicho caso:
        \begin{enumerate}
            \item[1.] \ldots \\
                \vdots
            \item[$p$.] $\lnot\alpha\to\lnot\beta$
            \item[$p+1$.] \ldots \\
                \vdots
            \item[$q$.] $\lnot\alpha\to\beta$
            \item[$q+1$.] $(\lnot\alpha\to\lnot\beta)\to((\lnot\alpha\to\beta)\to\alpha)\in \mathcal{A}_3$
            \item[$q+2$.] $((\lnot\alpha\to\beta)\to\alpha)$ por Modus ponens de $q+1$ y $p$.
            \item[$q+3$.] $\alpha$ por Modus ponens de $q+2$ y $q$.
        \end{enumerate}
        Como desde el paso 1 hasta el $q$ solo hemos usado como hipótesis $\Gamma$, deducimos que $\Gamma\vdash \alpha$.
    \end{proof}
\end{prop}

\begin{prop}[Leyes de silogismo o transitividad de la flecha]
    Sean $\alpha$, $\beta$ y $\gamma$ proposiciones, se verifican:
    \begin{enumerate}
        \item $\vdash (\alpha\to\beta)\to((\beta\to\gamma)\to(\alpha\to\gamma))$
        \item $\vdash (\beta\to\gamma)\to((\alpha\to\beta)\to(\alpha\to\gamma))$
    \end{enumerate}
    \begin{proof}
        Demostraremos la primera y dejamos la segunda como ejercicio. Para ello, aplicando el Teorema de Herbrand 3 veces, llegamos a que 1 es equivalente a ver que:
        \begin{equation*}
            \{\alpha\to\beta,\beta\to\gamma,\alpha\}\vdash \gamma
        \end{equation*}
        Para ello, nos sirve con la demostración:
        \begin{enumerate}
            \item $\alpha\to\beta$ es una hipótesis
            \item $\alpha$ es una hipótesis
            \item $\beta$ por Modus ponens de 1 y 2
            \item $\beta\to\gamma$ es una hipótesis
            \item $\gamma$ por Modus ponens de 3 y 4
        \end{enumerate}
    \end{proof}
\end{prop}

\begin{coro}[Regla del silogismo]
    Sea $\Gamma\cup\{\alpha,\beta\}$ un conjunto de proposiciones, si $\Gamma\vdash \alpha\to\beta$ y $\Gamma\vdash \beta\to\gamma$, entonces $\Gamma\vdash \alpha\to\gamma$.
    % \begin{proof} % // TODO: Hacer
    % \end{proof}
\end{coro}

\begin{prop}[Ley de conmutación de premisas]
    Sean $\alpha$, $\beta$ y $\gamma$ proposiciones:
    \begin{equation*}
        \vdash (\alpha\to(\beta\to\gamma))\to(\beta\to(\alpha\to\gamma))
    \end{equation*}
    \begin{proof}
        Aplicando el Teorema de Herbrand 3 veces, es equivalente a ver que:
        \begin{equation*}
            \{\alpha\to(\beta\to\gamma),\beta,\alpha\}\vdash \gamma
        \end{equation*}
        Para ello, nos sirve con:
        \begin{enumerate}
            \item $\alpha\to(\beta\to\gamma)$ es una hipótesis
            \item $\alpha$ es una hipótesis
            \item $\beta\to \gamma$ por Modus ponens de 1 y 2
            \item $\beta$ es una hipótesis
            \item $\gamma$ por Modus ponens de 3 y 4
        \end{enumerate}
    \end{proof}
\end{prop}

\begin{coro}[Regla de conmutación de premisas]
    Sea $\Gamma\cup\{\alpha,\beta,\gamma\}$ un conjunto de proposiciones, si $\Gamma\vdash \alpha\to(\beta\to\gamma)$, entonces $\Gamma\vdash \beta\to(\alpha\to\gamma)$.
\end{coro}

\begin{prop}[Ley de la doble negación]
    Sea $\alpha$ una proposición:
    \begin{equation*}
        \vdash \lnot\lnot\alpha\to \alpha
    \end{equation*}
    \begin{proof}
        Por el Teorema de Herbrand, es equivalente a ver que $\{\lnot\lnot\alpha\}\vdash \alpha$. Para ello, usamos la regla de la reducción al absurdo clásica, ya que:
        \begin{enumerate}
            \item $\{\lnot\lnot\alpha,\lnot\alpha\}\vdash \lnot\lnot\alpha$
            \item $\{\lnot\lnot\alpha,\lnot\alpha\}\vdash \lnot\alpha$
        \end{enumerate}
        Luego concluimos que $\{\lnot\lnot\alpha\}\vdash \alpha$.
    \end{proof}
\end{prop}

\begin{prop}[Ley débil de la doble negación]
    Sea $\alpha$ una proposición:
    \begin{equation*}
        \vdash \alpha\to\lnot\lnot\alpha
    \end{equation*}
    \begin{proof}
        Por el Teorema de Herbrand, es equivalente a ver que $\{\alpha\}\vdash \lnot\lnot\alpha$. Para ello, usamos la regla de la reducción al absurdo clásica, con lo que partimos que $\{\alpha,\lnot\lnot\lnot\alpha\}$ y tenemos que demostrar una proposición y su negación. Para ello:
        \begin{enumerate}
            \item $\lnot\lnot\lnot\alpha\to\lnot\alpha$ por la ley de la doble negación
            \item $\lnot\lnot\lnot\alpha$ es una hipótesis
            \item $\lnot\alpha$ por Modus ponens de 1 y 2
            \item $\alpha$ es una hipótesis
        \end{enumerate}
        Concluimos por la regla de la reducción al absurdo que $\{\alpha\}\vdash \lnot\lnot\alpha$.
    \end{proof}
\end{prop}

\section{Teoremas de coherencia y adecuación}
A lo largo de este capítulo hemos manejado en los lenguajes proposicionales dos conceptos fundamentales: las tautologías ($\vDash \alpha$), relacionadas con las interpretaciones; y los teorema ($\vdash \alpha$), relacionados con las demostraciones. Las primeras tienen un gran interés en informática y ciencias de la computación, gracias a las consecuencias semánticas que podemos realizar de forma autómatica, tal y como vimos con el Algoritmo de Davis \& Putnam. Por otra parte, las segundas tienen un gran interés matemático, por ser la principal herramienta que sustentan todo el conocimiento matemático. Veremos ahora dos teoremas que nos permiten relacionar las tautologías con los teoremas, de gran importancia en los lenguajes de primer orden.

\begin{teo}[de coherencia]
    Sea $\alpha$ una proposición, si $\vdash \alpha$, entonces $\vDash \alpha$. Es decir, todo teorema es una tautología.
    \begin{proof}
        Si $\alpha$ es un teorema, por definición este tendrá una demostración de $n$ pasos $\alpha_1,\ldots,\alpha_n$:
        \begin{itemize}
            \item $\alpha_1$ será un axioma y anteriormente probamos que todo axioma era una tautología.
            \item A $\alpha_2$ le ocurrirá lo mismo.
            \item $\alpha_i$ para $i\geq 3$ podrá ser un axioma, en cuyo caso ya sabemos cómo proceder o resultado de aplicar modus ponens sobre dos pasos anteriores. Sin embargo, anteriormente vimos que si $a$ y $a\to b$ eran tautologías, entonces $b$ era una tautología, por lo que $\alpha_i$ será una tautología.
        \end{itemize}
        Finalmente, llegaremos a $\alpha_n = \alpha$, con lo que podemos concluir que $\alpha$ es una tautología.
    \end{proof}
\end{teo}

Y además la otra implicación también es cierta, aunque su demostración excede los objetivos del curso.

\begin{teo}[de adecuación]
    Sea $\alpha$ una proposición, si $\vDash \alpha$, entonces $\vdash \alpha$. Es decir, toda tautología es un teorema.
\end{teo}
