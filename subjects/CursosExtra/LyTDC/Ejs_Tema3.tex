\section{Teoría Descriptiva de Conjuntos}

    \begin{ejercicio}
    Demostrar que $(2^\mathbb{N}, d)$ es un espacio completo. \\

    Veamos en primer lugar que $d$ es una distancia.
    \begin{enumerate}
        \item No negatividad:
            \begin{equation*}
                d(x,y) = \frac{1}{2^{n+1}} \geq 0 \qquad \forall x,y\in 2^\mathbb{N}
            \end{equation*}
            Además, se tiene que $d(x,y) = 0$ si y solo si $x = y$.

        \item Simetría:
        
            Sea $n=\min\{k\in \mathbb{N} \mid x(k) \neq y(k)\}=\min\{k\in \mathbb{N} \mid y(k) \neq x(k)\}$, entonces:
            \begin{equation*}
                d(x,y) = \frac{1}{2^{n+1}} = d(y,x)
            \end{equation*}

        \item Desigualdad triangular:
        
            Sean los siguientes tres mínimos, que suponemos que existen (ya que si no existen, la desigualdad se verifica trivialmente):
            \begin{align*}
                n_1 &= \min\{k\in \mathbb{N} \mid x(k) \neq y(k)\} \\
                n_2 &= \min\{k\in \mathbb{N} \mid y(k) \neq z(k)\} \\
                n &= \min\{k\in \mathbb{N} \mid x(k) \neq z(k)\}
            \end{align*}

            Tenemos que $\min\{n_1,n_2\} \leq n$, puesto que para $k<\min\{n_1,n_2\}$, se verifica que $x(k) = y(k)$ y $y(k) = z(k)$, por lo que $x(k) = z(k)$. Por tanto, $n\geq \min\{n_1,n_2\}$. Por tanto:
            \begin{align*}
                d(x,z) &= \frac{1}{2^{n+1}} \leq \frac{1}{2^{\min\{n_1,n_2\}+1}} = \max\left\{\frac{1}{2^{n_1+1}},\frac{1}{2^{n_2+1}}\right\} =\\&= \max\{d(x,y),d(y,z)\}\leq d(x,y) + d(y,z)
            \end{align*}
    \end{enumerate}

    Por tanto, hemos demostrado que $d$ es una distancia, por lo que consideramos el espacio métrico $(2^\mathbb{N},d)$. Este será completo si toda sucesión de Cauchy es convergente a una sucesión de cantor, lo que veremos a continuación.\\

    Sea $\{x_n\}_{n\in \mathbb{N}}$ una sucesión de Cauchy, es decir:
    \begin{equation*}
        \forall \veps\in \mathbb{R}^+\ \exists N\in \mathbb{N}\ \forall m,n\geq N\ d(x_m,x_n) < \veps
    \end{equation*}

    Veamos ahora cómo demostrar que esta sucesión es convergente, para lo cual hemos de construir la sucesión $x$ que sea el límite de la sucesión de Cauchy. Para cada $j\in \bb{N}$, consideramos $\veps=\frac{1}{2^j+1}$, y por tanto, existe $N_j\in \mathbb{N}$ tal que:
    \begin{equation*}
        \forall m,n\geq N_j\qquad d(x_m,x_n) < \frac{1}{2^j+1}
    \end{equation*}

    Por tanto, para cada $m,n\geq N_j$, tenemos que $x_m(j)=x_n(j)$. Definimos por tanto:
    \begin{equation*}
        x(j) = x_{N_j}(j) = x_m(j) \qquad \forall m\geq N_j
    \end{equation*}

    Vemos que $x$ es una sucesión de Cantor, y ahora hemos de demostrar que es el límite de la sucesión de Cauchy. Fijado $\veps\in \mathbb{R}^+$, existe $k\in \mathbb{N}$ tal que $\frac{1}{2^k+1} < \veps$, y podemos considerar $N_k\in \mathbb{N}$ tal que:
    \begin{equation*}
        \forall m,n\geq N_k\ d(x_m,x_n) < \frac{1}{2^k+1}
    \end{equation*}

    Por tanto, para todo $m\geq N_k$, veamos que $x_m(j) = x(j)$ para todo $j\leq k$. Sea $j\leq k$, luego:
    \begin{equation*}
        \frac{1}{2^{k+1}}\leq \frac{1}{2^j+1}\Longrightarrow N_j\leq N_k
    \end{equation*}

    Por tanto, para todo $m\geq N_k\geq N_j$, se tiene que $x_m(j) = x_{N_j}(j) = x(j)$. Por tanto, para todo $m\geq N_k$, se verifica que:
    \begin{equation*}
        d(x_m,x) < \frac{1}{2^{k+1}} < \veps
    \end{equation*}

    Por tanto, hemos demostrado que la sucesión de Cauchy $\{x_n\}_{n\in \mathbb{N}}$ converge a $x\in 2^\mathbb{N}$, y por tanto, $(2^\mathbb{N},d)$ es completo.
\end{ejercicio}


\begin{ejercicio}
    Sea $(X,d)$ un espacio métrico. Dados $x\in X$ y $A\subseteq X$, definimos la distancia entre $a$ y $X$ como:
    \begin{equation*}
        d(x,A) = \inf\{d(x,a) \mid a\in A\}
    \end{equation*}
    Verificar que, dado $r>0$, el siguiente conjunto es un abierto:
    \begin{equation*}
        \{x\in X\mid d(x,A) < r\}
    \end{equation*}
\end{ejercicio}
\begin{proof}
    Dado $x\in X$ con $d(x,A) < r$, veamos que $\exists \veps\in \mathbb{R}^+$ de forma que $B(x,\veps) \subseteq \{x\in X\mid d(x,A) < r\}$.\\

    Sea $\veps=r-d(x,A)>0$, y sea $y\in B(x,\veps)$, es decir, $d(x,y) < \veps$. Veamos que $d(y,A) < r$.
    \begin{align*}
        d(y,a) \leq d(y,x) + d(x,a) \forall a\in A
    \end{align*}

    Por tanto:
    \begin{align*}
        d(y,A) &= \inf\{d(y,a) \mid a\in A\} \\
        &\leq \inf\{d(y,x) + d(x,a) \mid a\in A\}
        = d(y,x) + \inf\{d(x,a) \mid a\in A\} \\
        &= d(y,x) + d(x,A)< \veps + d(x,A) = r-d(x,A) + d(x,A) = r
    \end{align*}

    Por tanto, $y\in \{x\in X\mid d(x,A) < r\}$, y hemos demostrado que:
    \begin{equation*}
        B(x,\veps) \subseteq \{x\in X\mid d(x,A) < r\}
    \end{equation*}
    Por tanto, $\{x\in X\mid d(x,A) < r\}$ es un abierto.
\end{proof}


\begin{ejercicio}
    En el cubo de Hilbert ${[0,1]}^{\mathbb{N}}$, consideramos la métrica $d$ definida como:
    \begin{equation*}
        d(x,y) = \sum_{n=0}^{+\infty} \dfrac{d(x_n,y_n)}{2^n} \qquad \forall x,y\in {[0,1]}^{\mathbb{N}}
    \end{equation*}


    Demostrar que $d$ es una métrica en ${[0,1]}^{\mathbb{N}}$.
    \begin{proof}
        En primer lugar, hemos de ver que la distancia así definida está bien definida, es decir, que la suma converge. Aplicamos para ello el Criterio de Comparación:
        \begin{equation*}
            \sum_{n=0}^{+\infty} \dfrac{d(x_n,y_n)}{2^n} \leq \sum_{n=0}^{+\infty} \dfrac{1}{2^n} = \dfrac{1}{1-\nicefrac{1}{2}} = 2
        \end{equation*}

        Por tanto, $d$ está bien definida. Ahora, veamos que $d$ es una métrica:
        \begin{itemize}
            \item \underline{No-negatividad}: Por definición de $d$, tenemos que:
                \begin{equation*}
                    d(x,y) = \sum_{n=0}^{+\infty} \dfrac{d(x_n,y_n)}{2^n} \geq 0
                \end{equation*}

                Además, se tiene que $d(x,y) = 0$ si y solo si $x = y$.
            \item \underline{Simetría}: Por definición de $d$, tenemos que:
                \begin{equation*}
                    d(x,y) = \sum_{n=0}^{+\infty} \dfrac{d(x_n,y_n)}{2^n} = \sum_{n=0}^{+\infty} \dfrac{d(y_n,x_n)}{2^n} = d(y,x)
                \end{equation*}
            \item \underline{Desigualdad triangular}: Tenemos que:
                \begin{align*}
                    d(x,z) &= \sum_{n=0}^{+\infty} \dfrac{d(x_n,z_n)}{2^n}
                    \leq \sum_{n=0}^{+\infty} \dfrac{d(x_n,y_n) + d(y_n,z_n)}{2^n}
                    = \sum_{n=0}^{+\infty} \dfrac{d(x_n,y_n)}{2^n} + \sum_{n=0}^{+\infty} \dfrac{d(y_n,z_n)}{2^n} \\
                    &= d(x,y) + d(y,z)
                \end{align*}
            \end{itemize}
        Por tanto, hemos visto que $d$ es una métrica en ${[0,1]}^{\mathbb{N}}$.
    \end{proof}
\end{ejercicio}


\begin{ejercicio}
        Definimos los siguientes conjuntos:
        \begin{align*}
           Q_2 &= \left\{ \alpha \in \cc{C} \;\mid\; \exists A\subset \mathbb{N}\ \text{finito tal que}\ \alpha(n)=0\ \forall n\in \mathbb{N}\setminus\{A\}\right\} \\
           \ell^1 &= \left\{ x\in [0,1]^\mathbb{N}\ \middle|\ \sum_{n=1}^\infty x_n < \infty\right\}
        \end{align*}
        Demostrar que $\ell^1\in \Sigma_2^0$ y $Q_2 \leq_W \ell^1$.
    \end{ejercicio}

    \begin{ejercicio}
        Sea $\Gamma$ una clase de la Jerarquía Boreliana, y $X$ un conjunto. Si $A\subset X$ es $\Gamma-$completo, y $B\subset X$ es otro conjunto de la clase $\Gamma$ tal que $A\leq_W B$, entonces $B$ es $\Gamma-$completo.
        \begin{proof}
            Hemos de comprobar que:
        \begin{itemize}
            \item \ul{$B\in \Gamma$}: Se tiene por hipótesis.
            \item \ul{Para todo espacio polaco $X'$, si $C\in \Gamma(X')$ entonces $C\leq_W B$}:
            
            Sea $C\in \Gamma(X')$, y buscamos $f:X'\to X$ tal que $f$ es una función continua y $C=f^{-1}(B)$.
            
            Como $A$ es $\Gamma-$completo, existe $g:X'\to X$ tal que $g$ es continua y $C=g^{-1}(A)$. Por otro lado, como $A\leq_W B$, existe una función continua $h:X\to X$ tal que $A=h^{-1}(B)$. Entonces, la composición $f=h\circ g$ es continua y cumple que:
            \begin{equation*}
                f^{-1}(B) = g^{-1}(h^{-1}(B)) = g^{-1}(A) = C
            \end{equation*}

            Por tanto, $C\leq_W B$.
        \end{itemize}
        \end{proof}
    \end{ejercicio}


    \begin{ejercicio}
        Demostrar que $f:[0,1]\to \bb{R}$ es continuamente derivable si y solo si
        \begin{equation*}
            \forall\veps\in \bb{R}^+\ \exists\delta\in \bb{R}^+\ f\in A_{\veps,\delta}
        \end{equation*}
        donde:
        \begin{multline*}
            A_{\veps,\delta} = \left\{f\in C([0,1])\ \middle|\ \forall x,y,a,b\in [0,1]\ : a,b,x,y \text{ a distancia } \leq \delta\right. \Longrightarrow \\\Longrightarrow \left.\left|\frac{f(a)-f(b)}{a-b} - \frac{f(x)-f(y)}{x-y}\right| < \veps\right\}
        \end{multline*}
        \begin{proof}
            Sea $f:[0,1]\to \bb{R}$. Demostraremos por doble implicación.
            \begin{description}
                \item[$\Longrightarrow)$] Sea $f\in C^1([0,1])$, y sea $\veps\in \bb{R}^+$. Como $f$ es derivable en $[0,1]$, por el Teorema del Valor Medio existe $a'\in \left]a,b\right[, x'\in \left]x,y\right[$ tal que:
                \begin{equation*}
                    \frac{f(a)-f(b)}{a-b} = f'(a')\qquad \text{y}\qquad \frac{f(x)-f(y)}{x-y} = f'(x')
                \end{equation*}

                Por el Teorema de Heine, como $[0,1]$ es compacto y $f'$ es continua, existe $\delta'\in \bb{R}^+$ tal que:
                \begin{equation*}
                    |a'-x'| < \delta'
                    \Longrightarrow |f'(a') - f'(x')| < \veps
                \end{equation*}

                Sea ahora $\delta = \nicefrac{\delta'}{3}$. Usando que $a,b,x,y$ están a distancia $\leq \delta$, veamos que $|x'-a'| < \delta'$:
                \begin{align*}
                    |x'-a'| &\leq |x'-x| + |x-a| + |a-a'|< |x-y| + |x-a| + |a-b|\leq 3\delta = \delta'
                \end{align*}

                Por tanto, se verifica que:
                \begin{align*}
                    \left|\frac{f(a)-f(b)}{a-b} - \frac{f(x)-f(y)}{x-y}\right| &= |f'(a') - f'(x')| < \veps
                \end{align*}

                Por tanto, $f\in A_{\veps,\delta}$.


                \item[$\Longleftarrow)$] Hemos de demostrar que $f$ es continuamente derivable. Para ello, definimos el cociente incremental de $f$ en $t\in [0,1]$ como:
                \begin{equation*}
                    f_t(x) = \frac{f(x)-f(t)}{x-t} \qquad \forall x\in [0,1]\setminus\{t\}
                \end{equation*}

                En primer lugar, hemos de ver que $f$ es derivable, para lo cual hemos de comprobar que, para cada $t\in [0,1]$, el siguiente límite existe:
                \begin{equation*}
                    \lim_{x\to t} f_t(x)
                \end{equation*}

                Para comprobar que este límite existe, usaremos que $\bb{R}$ es completo, por lo que toda sucesión de Cauchy converge. Sea $t\in [0,1]$, y sea $\{x_n\}_{n\in \bb{N}}$ una sucesión de puntos de $[0,1]\setminus\{t\}$ tal que $\{x_n\}\to t$. Veamos que $\{f_t(x_n)\}_{n\in \bb{N}}$ es una sucesión de Cauchy. Para ello, fijamos $\veps\in \bb{R}^+$, por lo que $\exists \delta\in \bb{R}^+$ tal que $f\in A_{\veps,\delta}$. Por ser $\{x_n\}$ de Cauchy, existe $N\in \bb{N}$ tal que, para todo $m,n\geq N$, se verifica que:
                \begin{equation*}
                    |x_m-x_n| < \delta
                \end{equation*}

                Por tanto, $t,x_m,x_n$ están a distancia $\leq \delta$, y por tanto, como $f\in A_{\veps,\delta}$, se verifica que:
                \begin{align*}
                    \left|f_t(x_m) - f_t(x_n)\right| &= \left|\frac{f(x_m)-f(t)}{x_m-t} - \frac{f(x_n)-f(t)}{x_n-t}\right|< \veps
                \end{align*}

                Por tanto, $\{f_t(x_n)\}_{n\in \bb{N}}$ es una sucesión de Cauchy, y por tanto, converge a un límite $f'(t)\in \bb{R}$. Definimos por tanto:
                \begin{equation*}
                    f'(t) = \lim_{x\to t} f_t(x)
                \end{equation*}

                Ahora, queremos demostrar que \( f' \) es continua en todo \( [0,1] \). Para ello, fijamos un punto \( x \in [0,1] \), y tomamos una sucesión \( \{t_n\}_{n \in \mathbb{N}} \subset [0,1] \setminus \{x\} \) tal que \( \{t_n\} \to x \). Queremos ver que:
                \[
                \lim_{n \to \infty} f'(t_n) = f'(x)
                \]

                Recordemos que, por definición,
                \[
                f'(t_n) = \lim_{y \to t_n} \frac{f(y) - f(t_n)}{y - t_n}
                \quad \text{y} \quad
                f'(x) = \lim_{z \to x} \frac{f(z) - f(x)}{z - x}
                \]

                Fijado ahora $\veps\in \bb{R}^+$, consideramos $\delta \in \bb{R}^+$ tal que \( f \in A_{\varepsilon,\delta} \).  Como \( \{t_n\} \to x \), existe \( N \in \mathbb{N} \) tal que para todo \( n \geq N \), se tiene \( |t_n - x| < \nicefrac{\delta}{2} \). Fijamos tal \( n \geq N \), y tomamos \( y \in [0,1] \) con \( |y - t_n| < \nicefrac{\delta}{2} \). Entonces, por desigualdad triangular:
                \[
                |y - x| \leq |y - t_n| + |t_n - x| < \nicefrac{\delta}{2} + \nicefrac{\delta}{2} = \delta
                \]

                Por tanto, \( x, y, t_n \) están a distancia menor que \( \delta \), y podemos aplicar la hipótesis:
                \[
                \left| \frac{f(y) - f(t_n)}{y - t_n} - \frac{f(z) - f(x)}{z - x} \right| < \varepsilon
                \quad \forall z\in [0,1]\ \text{tal que}\ |x-z|<\delta
                \]

                Tomando el límite cuando \( y \to t_n \), se obtiene:
                \[
                \left| f'(t_n) - \frac{f(z) - f(x)}{z - x} \right| < \varepsilon
                \quad \forall z\in [0,1]\ \text{tal que}\ |x-z|<\delta
                \]

                Y tomando después el límite cuando \( z \to x \), se concluye que $|f'(t_n) - f'(x)| < \varepsilon$. Como $n\geq N$ era arbitrario, tenemos que:
                \[
                \lim_{n \to \infty} f'(t_n) = f'(x)
                \]

                Es decir, \( f' \) es continua en \( x \). Como \( x \in [0,1] \) era arbitrario, concluimos que \( f' \) es continua en todo el intervalo, y por tanto, \( f \in C^1([0,1]) \).
            \end{description}
        \end{proof}
    \end{ejercicio}

