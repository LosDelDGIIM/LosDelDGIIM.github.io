\chapter{Lógica de Primer Orden}
Es necesario introducir ahora lenguajes en los que podamos cuantificar cosas. Como primer ejemplo, si sabemos que ``Todo hombre es mortal'' y que ``Sócrates es un hombre'', nos gustaría deducir que, entonces, ``Sócrates es mortal''. Sin embargo, para esto hemos de poder cuantificar, cosa que no es posible con los lenguajes proposicionales pero sí con los lenguajes de primer orden.\\

\noindent
Los lenguajes de primer orden estarán formados por:
\begin{itemize}
    \item Constantes: $c_1,c_2,\ldots, a,b,c,\ldots$
    \item Variables: $x_1,x_2,\ldots, x,y,z,\ldots$
    \item Símbolos de función: $f_1, f_2, \ldots, f,g,h, \ldots$
    \item Símbolos de relación: $R_1,R_2,\ldots,R,S,T,\ldots$
    \item Conectivas lógicas: $\lor$, $\land$, $\lnot$, $\to$, $\leftrightarrow $
    \item Cuantificadores: $\forall $, $\exists $
\end{itemize}

A los conjuntos de todas las constantes, de todas la variables y de todos los símbolos de función los notaremos por $Cons(\mathcal{L}), Var(\mathcal{L}), Fun(\mathcal{L})$, si $\mathcal{L}$ es nuestro lenguaje de primer orden.

\begin{notacion}
    En otros libros o contextos, en vez de denotar a los símbolos de función o variables con una letra que pueda llevar o no superíndice, estos las denotan con un superíndice:
    \begin{itemize}
        \item $f_1^{n_1},f_2^{n_2},\ldots$
        \item $R_1^{m_1},R_2^{m_2},\ldots$
    \end{itemize}
    En este caso, el superíndice indica la ariedad de la función o relación. Por ejemplo, si consideramos $f^3$, tenemos un símbolo de función que se aplica a 3 variables.
\end{notacion}

\begin{definicion}[Término]
    Un término es:
    \begin{enumerate}
        \item Cualquier constante.
        \item Cualquier variable.
        \item Si $t_1,t_2,\ldots,t_n$ son términos y $f$ es un símbolo de función $n-$ario, entonces $f(t_1,t_2,\ldots,t_n)$ es un término.
        \item No hay más términos que los que se puedan obtener siguiendo una secuencia finita de pasos a partir de las enunciadas.
    \end{enumerate}
    Al conjunto de todos los términos de nuestro lenguaje $\mathcal{L}$ lo denotamos por $Term(\mathcal{L})$.
\end{definicion}

\begin{ejemplo}\
    \begin{itemize}
        \item $f(x,f(x,y))$ es un término.
        \item $f(x,f(x))$ no es un término, ya que usamos un mismo símbolo de función, $f$, para denotar dos objetos: una función unaria y una función binaria.
    \end{itemize}
\end{ejemplo}

\begin{definicion}[Fórmulas atómicas]
    Si $t_1,\ldots,t_n$ son términos y $R$ es un símbolo de relación $n-$ario, entonces $R(t_1,\ldots,t_n)$ es una fórmula atómica (o simplemente, un átomo).
\end{definicion}

\begin{definicion}[Fórmulas]
    Son fórmulas:
    \begin{enumerate}
        \item Las fórmulas atómicas.
        \item Si $\varphi$ y $\psi$ son fórmulas, también lo son:
            \begin{equation*}
                \lnot\varphi,\ \varphi\land\psi,\ \varphi\lor\psi,\ \varphi\to\psi,\ \varphi\leftrightarrow\psi
            \end{equation*}
        \item Si $x$ es una variable y $\varphi$ es una fórmula, también lo son: $\forall x\varphi$, $\exists x\varphi$.
        \item No hay más fórmulas que las que se puedan obtener siguiendo una secuencia finita de pasos a partir de las enunciadas.
    \end{enumerate}
    Al conjunto de todas las fórmulas de nuestro lenguaje $\mathcal{L}$ lo denotamos por $Form(\mathcal{L})$.
\end{definicion}

\begin{definicion}
    Una \underline{ocurrencia} de una variable en una fórmula es una aparición de su escritura.
    \begin{itemize}
        \item En la fórmula $\forall x\varphi$, diremos que $\varphi$ es el \underline{radio de acción} de $\forall x$.
        \item En la fórmula $\exists x\varphi$, diremos que $\varphi$ es el \underline{radio de acción} de $\exists x$.
    \end{itemize}
    Diremos que $x$ se encuentra \underline{cuantificada} al ver $\forall x$ o $\exists x$.\newline
    Diremos que una ocurrencia de una variable $x$ es \underline{ligada} si aparece cuantificada o en el radio de acción de $\forall x$ o de $\exists x$.\newline
    Finalmente, diremos que una variable es \underline{libre} si no aparece ligada. Si $\varphi$ es una fórmula en la que las variables $x_1,\ldots,x_n$ aparecen libres, será usual denotar:
    \begin{equation*}
        \varphi(x_1,\ldots,x_n)
    \end{equation*}
    Que no debe confundirse con un término de una función o relación $n-$aria, ya que $\varphi$ no es ni un símbolo de función o relación, sino una fórmula.
\end{definicion}

\begin{ejemplo}
    En la siguiente fórmula:
    \begin{equation*}
        \forall x(\exists yR(x,y)\to Q(y))
    \end{equation*}
    \begin{itemize}
        \item $x$ aparece cuantificada en su primera ocurrencia.
        \item $y$ aparece cuantificada en su primera ocurrencia.
        \item $x$ aparece ligada en su segunda ocurrencia.
        \item $y$ aparece ligada en su segunda ocurrencia.
        \item $y$ aparece como variable libre en su tercera ocurrencia.
    \end{itemize}
\end{ejemplo}

\begin{definicion}[Sentencia]
    Una sentencia es una fórmula sin ocurrencias de variables libres.
\end{definicion}

\section{Semántica}
Trataremos de generalizar el concepto de ``interpretación'', ya visto para lenguajes proposicionales. Para ello, será necesario primero definir los conceptos de ``estructura'' y de ``asignación''.

\begin{definicion}[Estructura]
    Una estructura $\veps$ es una cuádrupla
    \begin{equation*}
        \veps = (D, \{c_i^\veps\}_{n\in \mathbb{N}}, \{f_i^\veps\}_{n\in \mathbb{N}}, \{R_i^\veps\}_{n\in \mathbb{N}})
    \end{equation*}
    de forma que:
    \begin{itemize}
        \item $D$ un conjunto no vacío al que llamamos \underline{universo} o \underline{dominio}.
        \item A cada constante $c_i$ de $\mathcal{L}$ le corresponde un elemento $c_i^{\veps}$ de $D$.
        \item A cada símbolo de función $f_i$ de $\mathcal{L}$ le corresponde una función $f_i^\veps:D^n\to D$.
        \item A cada símbolo de relación $R_i$ de $\mathcal{L}$ le corresponde una aplicación ${R_i^\veps:D^m\to\mathbb{Z}_2}$, de forma que $R_i^\veps(c_1^\veps,c_2^\veps)=1$ si $c_1^\veps$ y $c_2^\veps$ están relacionados y 0 en caso contrario.
    \end{itemize}
\end{definicion}

\begin{definicion}[Asignación]
    Una asignación $v$ en $\veps$ es una aplicación $v:Var(\mathcal{L})\to D$. Dada una asignación $v$, podremos extenderla a $v':Term(\mathcal{L})\to D$ de la forma:
    \begin{equation*}
        v'(t) = \left\{\begin{array}{ll}
                c^\veps & \text{si\ } t=c \text{\ una constante} \\
                v(x) & \text{si\ } t=x \text{\ una variable} \\
                f^\veps(v'(t_1),\ldots,v'(t_n)) &\text{si\ } t=f(t_1,\ldots,t_n)
        \end{array}\right.
    \end{equation*}
\end{definicion}

\begin{definicion}[Interpretación]
    Una interpretación es una tupla $(\veps,v)$ con $\veps$ una estructura y $v$ una asignación que tiene asociada una aplicación\footnote{A la que próximamente denotaremos simplemente como $I^v$, por simplicidad, entendiendo que la estructura $\veps$ viene dada por el contexto.} $I_\veps^v:Form(\mathcal{L})\to\mathbb{Z}_2$ que cumple para cualesquiera fórmulas $\varphi$ y $\psi$:
    \begin{enumerate}
        \item $I^v(\lnot \varphi) = 1+ I^v(\varphi)$.
        \item $I^v(\varphi\land \psi) = I^v(\varphi)I^v(\psi)$.
        \item $I^v(\varphi\lor \psi) = I^v(\varphi) + I^v(\psi) + I^v(\varphi)I^v(\psi)$.
        \item $I^v(\varphi\to \psi) = 1 + I^v(\varphi) + I^v(\varphi)I^v(\psi)$.
        \item $I^v(\varphi\leftrightarrow \psi) = 1 + I^v(\varphi) + I^v(\psi)$.
        \item $I^v(\forall x\varphi) = \left\{\begin{array}{ll}
                    1 & \text{si para todo\ } a\in D,\ I^{v(x\mid a)}(\varphi) = 1 \\
                    0 & \text{en caso contrario}
        \end{array}\right.$
        \item $I^v(\exists x\varphi) = \left\{\begin{array}{ll}
                    1 & \text{si existe\ } a\in D \text{\ con\ } I^{v(x\mid a)}(\varphi) = 1 \\
                    0 & \text{en caso contrario}
        \end{array}\right.$
    \end{enumerate}
    Siendo:
    \begin{equation*}
        v(x\mid a)(y) = \left\{\begin{array}{ll}
                v(y) & \text{si\ } y\neq x \\
                a & \text{si\ } y=x
        \end{array}\right.
    \end{equation*}
\end{definicion}
