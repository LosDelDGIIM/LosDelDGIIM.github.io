\chapter{Introducción a la Teoría Descriptiva de Conjuntos}

La Teoría Descriptiva de Conjuntos (TDC a partir de ahora) tiene como un objetivo clasificar enunciados o fórmulas según su complejidad, concepto que luego formalizaremos. Por ejemplo, si consideramos sobre las funciones del intervalo $[0,1]$ en $\mathbb{R}$ la propiedad de ``ser continua'':
\begin{equation*}
    \forall x\in \mathbb{R}\ \forall \veps > 0\ \exists \delta>0 : \forall x_0\ |x-x_0|<\delta \Longrightarrow |f(x)-f(x_0)|<\veps
\end{equation*}
como sabemos que toda función continua en $[0,1]$ es uniformemente continua y que toda función uniformemente continua es continua, podemos reescribir esta propiedad de una forma ``menos compleja'', eliminando en $\veps$ la dependencia de $x$
\begin{equation*}
    \forall \veps > 0\ \exists \delta>0 : \forall x_0\ |x-x_0|<\delta \Longrightarrow |f(x)-f(x_0)|<\veps
\end{equation*}
y obteniendo así una fórmula más corta, algo que nos interesará, puesto que intentaremos buscar las fórmulas más cortas (en función de las variables y cuantificadores que aparecen en ellas) que nos definan ciertas propiedades.

\section{Construcción de fórmulas}
Sin olvidar que la teoría sobre la que trabajamos (la de Zermelo-Fraenkel) es en particular un lenguaje de primer orden, recordamos que nuestras fórmulas van a estar formadas por, fijado un conjunto $X$ que contendrá los elementos a los que nos refiramos:
\begin{itemize}
    \item Términos, referidos a objetos.
    \item Conectores: $\lnot$, $\land$, $\lor$, $\rightarrow$, $\leftrightarrow $.
    \item Cuantificadores: $\forall $, $\exists $.
\end{itemize}
De esta forma, una fórmula para nosotros será una composición \underline{finita} de estos elementos.\\

Estaremos especialmente interesados en el hecho de que las fórmulas sean composiciones finitas de dichos elementos, así como en estudiar los cuantificadores que aparecen en las fórmulas.

\begin{ejemplo}
    Si consideramos $X = \mathbb{N}$ y en este contexto consideramos la aplicación ``sucesor'' $s:\mathbb{N}\to \mathbb{N}$, la fórmula:
    \begin{equation*}
        s\leq s(n)
    \end{equation*}
    Contiene a $n$ como variable libre, y (en general) podrá ser cierta o falsa en función de las sustituciones de $n$ realicemos (aunque en este caso hemos considerado una fórmula que ante cualquier sustitución siempre es cierta).
\end{ejemplo}

A partir de ahora, lo que nos interesará es dada una fórmula $P$ en la que hay una variable libre y trabajando sobre un conjunto $X$, podremos siempre considerar por el axioma de composición el conjunto formado por aquellos elementos de $X$ para los cuales la fórmula $P$ sea cierta:
\begin{equation*}
    X_P = \{x\in X \mid P(x) \}
\end{equation*}
Sin embargo, el recíproco de esta afirmación (que para cada cojunto siempre podemos encontrar una fórmula que cumplan exclusivamente los elementos del conjunto) no será generalmente cierta. Por ejemplo, si consideramos $X$ un conjunto finito, como $\cc{P}(X)$ es finito, siempre podremos hacerlo; pero en un caso general con $X$ cualquier conjunto (posiblemente no numerable), podemos pensar en que la cantidad de fórmulas que podemos construir es un conjunto numberable, por lo que no llegaremos a abarcar todas las posibilidades\footnote{Es mucho más complejo que esto, este argumento no es suficiente para demostrarlo.}.

\begin{ejemplo}
    Como primeros ejemplos de conjuntos a destacar, fijado un conjunto $X$, consideramos una cantidad numerable de subconjuntos de $X$: $\{X_n\}_{n\in \mathbb{N}}$ con $X_n\subseteq X$ para todo $n\in \mathbb{N}$.
    \begin{itemize}
        \item Si pensamos en la intersección de todos estos conjuntos:
            \begin{equation*}
                \bigcap_{n\in \mathbb{N}} X_n
            \end{equation*}
            Podemos tratar de buscar una fórmula que defina única y exclusivamente a todos los elementos de este conjunto, como por ejemplo:
            \begin{equation*}
                \forall n(n\in \mathbb{N} \Longrightarrow x\in X_n)
            \end{equation*}
        \item Si ahora consideramos la unión de todos ellos: 
            \begin{equation*}
                \bigcup_{n\in \mathbb{N}}X_n
            \end{equation*}
            Y tratamos de buscar una fórmula que lo defina, llegamos a:
            \begin{equation*}
                \exists n(n\in \mathbb{N} \land x\in X_n)
            \end{equation*}
    \end{itemize}
\end{ejemplo}

A partir de este primer ejemplo y viendo la relación existente entre los cuantificadores $\forall $ y $\exists $ con las operaciones $\cap$ y $\cup$, buscamos ahora cómo podemos expresar ciertas fórmulas sobre algún espacio $X$ previamente fijado en forma de conjuntos, proceso que ilustraremos con los siguientes ejemplos.

\begin{ejemplo}
    En cada caso, consideraremos un conjunto $X$ distinto.
    \begin{itemize}
        \item En el espacio de las suceiones de números reales: $X = \mathbb{R}^\mathbb{N} = \{x:\mathbb{N}\to \mathbb{R}\}$

            Pensamos en la propiedad de que una sucesión sea ``casi nula'', que intuitivamente podemos definir como que una sucesión tenga una cantidad infinita de términos nulos. De manera formal, podemos escribir que existe un término a partir del cual todos los términos de la sucesión son cero:
            \begin{equation*}
                \exists n\in \mathbb{N}\ \forall m\geq n\ x(m) = 0
            \end{equation*}
            Que podemos escribir de forma más rigurosa como (entendiendo que donde pone $m\geq n$ deberíamos escribir $m\geq n \land m\in \mathbb{N}$):
            \begin{equation*}
                \exists n(n\in \mathbb{N} \land \forall m(m\geq n \Longrightarrow x(m) = 0))
            \end{equation*}
            Donde observamos que en esta la variable $x$ aparece libre, por lo que podemos tratar de buscar un conjunto que contenga todos aquellos elementos que cumplan la fórmula para cierto $x$ y ninguno más, conjunto al que denotaremos por $C_{00}$.

            Para hayar este conjunto, lo que haremos será en primer lugar considerar los términos que aparecen en la fórmula y en segundo lugar, tratar de relacionarlos con los conectores y cuantificadores que aparecen. Para ello, los términos que aparecen en la fórmula son:
            \begin{equation*}
                n\in \mathbb{N} \qquad m\geq n \qquad x(m) = 0
            \end{equation*}
            Y para construir el conjunto que venga definido por la fórmula, lo que haremos será ir poco a poco de dentro hacia afuera, considerando primero el conjunto que cumpla:
            \begin{equation*}
                x(m) = 0
            \end{equation*}
            De esta forma, fijado $m$, definimos:
            \begin{equation*}
                X_m = \{x\in \mathbb{R}^\mathbb{N} : x(m) = 0\}
            \end{equation*}
            Ahora, buscamos el conjunto que venga definido por la fórmula:
            \begin{equation*}
                \forall m(m\geq n\Longrightarrow x(m) = 0)
            \end{equation*}
            Que podemos reescribir como:
            \begin{equation*}
                \forall m(m\geq n\Longrightarrow x\in X_m)
            \end{equation*}
            Observando el $\forall $, podemos pensar en reescribir este conjunto como en una intersección de conjuntos:
            \begin{equation*}
                \bigcap_{m\geq n}X_m
            \end{equation*}
            Finalmente, la fórmula entera:
            \begin{equation*}
                \exists n(n\in \mathbb{N} \land \forall m(m\geq n \Longrightarrow x(m) = 0))
            \end{equation*}
            La podemos reescribir como:
            \begin{equation*}
                \exists n\left(n\in \mathbb{N} \land x\in \bigcap_{m\geq n} X_m\right)
            \end{equation*}
            Que podemos expresar ahora como la unión de ciertos conjunto.
            \begin{equation*}
                \bigcup_{n\in \mathbb{N}}\bigcap_{m\geq n} X_m
            \end{equation*}
            Obteniendo así nuestro conjunto $C_{00}$:
            \begin{equation*}
                C_{00} = \bigcup_{n\in \mathbb{N}}\bigcap_{m\geq n} X_m
            \end{equation*}
        \item Sobre el mismo espacio $X= \mathbb{R}^\mathbb{N}$ podemos ahora considerar la propiedad de ``ser convergente a 0'', propiedad definida por:
            \begin{equation}\label{eq:conv_0}
                \forall \veps > 0\ \exists m\in \mathbb{N}\ \forall n\geq m \Longrightarrow |x(m)| < \veps
            \end{equation}
            Y que de forma rigurosa puede escribirse como (entendiendo que donde pone $n\geq m$ deberíamos escribir $n\geq m \land n\in \mathbb{N}$ y que donde pone $\veps > 0$ deberíamos poner $\veps \in \mathbb{R}^+$):
            \begin{equation*}
                \forall \veps(\veps > 0 \Longrightarrow \exists m(m\in \mathbb{N} \land \forall n(n\geq m \Longrightarrow |x(m)|<\veps)))
            \end{equation*}
            Fórmula a partir de la cual podemos extrar un conjunto de igual forma que hicimos anteriormente, identificando que los términos son:
            \begin{equation*}
                \veps > 0 \qquad m\in \mathbb{N} \qquad n\geq m \qquad |x(m)|<\veps
            \end{equation*}
            Y construyendo la fórmula de dentro hacia afuera. Para ello, en primer lugar, fijados $m$ y $\veps$ definimos el conjunto:
            \begin{equation*}
                X_{n,\veps} = \{x\in \mathbb{R}^\mathbb{N} : |x(n)|<\veps\}
            \end{equation*}
            Y posteriormente escribiendo las sucesivas fórmulas como uniones e intersecciones, llegando a:
            \begin{equation*}
                \bigcap_{\veps > 0} \bigcup_{m\in \mathbb{N}} \bigcap_{n\geq m} X_{n,\veps}
            \end{equation*}
            Sin embargo, hay una diferencia ahora entre la fórmula obtenida anteriormente y esta; resulta que en esta fórmula estamos considerando una intersección no numerable de elementos, al considerar la intersección de todos aquellos elementos de $\mathbb{R}^+$, un hecho que nos va a dificultar luego algo en lo que estamos interesados\footnote{Que es poder definir una sigma álgebra que contenga uniones e intersecciones numerables de ciertos conjuntos.}.

            A pesar de ello, la solución en este caso es bien sencilla. Resulta que la definición de convergencia a 0 cuya definición escribimos en~(\ref{eq:conv_0}) puede caracterizarse en función de una sucesión convergente a cero, pudiendo cambiar la fórmula que describe la propiedad de ``ser convergente a cero'' por la fórmula:
            \begin{equation*}
                \forall k\in \mathbb{N}\ \exists m\in \mathbb{N}\ \forall n\geq m \Longrightarrow |x(m)| < \dfrac{1}{k}
            \end{equation*}
            Si ahora realizamos nuevamente el proceso anterior de reescribir a qué conjunto llegamos, obtenemos ahora sí un conjunto dado por intersecciones y uniones numerables de ciertos conjuntos:
            \begin{equation*}
                \bigcap_{k\in \mathbb{N}} \bigcup_{m\in \mathbb{N}} \bigcap_{n\geq m} X_{n,\frac{1}{k}}
            \end{equation*}
        \item Si ahora consideramos $X = \{f:\mathbb{R}\to\mathbb{R} : f \text{\ continua}\}$ y fijado $L\in \mathbb{R}^+$, pensamos en la propiedad de $\lim\limits_{x\to\infty}f(x) = L$, definida por:
            \begin{equation*}
                \forall \veps > 0\ \exists M>0\ \forall x>M\ |f(x)-L|<\veps
            \end{equation*}
            Se nos plantea el problema anterior de que obtendríamos uniones o intersecciones no numerables de conjuntos. Sin embargo, vemos que es fácil reemplazar $\veps$ y $M$ para que esto no suceda, obteniendo una expresión equivalente:
            \begin{equation*}
                \forall k\in \mathbb{N}\ \exists M\in \mathbb{N}\ \forall x>M\ |f(x)-L|<\dfrac{1}{k}
            \end{equation*}
            Sin embargo, seguimos teniendo el problema de que $x\in \mathbb{R}$, que pensamos en cómo solucionar. Como estamos considerando funciones continuas de $\mathbb{R}$ en $\mathbb{R}$ y $\mathbb{Q}\subseteq \mathbb{R}$ es denso y numerable, podemos considerar $x\in \mathbb{Q}$ y reescribir la propiedad de forma equivalente como:
            \begin{equation*}
                \forall k\in \mathbb{N}\ \exists M\in \mathbb{N}\ \forall x\in \mathbb{Q}, x>M\ |f(x)-L|<\dfrac{1}{k}
            \end{equation*}
            Así, obtenemos una fórmula:
            \begin{equation*}
                \forall k\left(k\in \mathbb{N}\Longrightarrow \exists M\left(M\in \mathbb{N}\land \forall x\left(x\in Q\land x>M \Longrightarrow |f(x)-L|<\dfrac{1}{k}\right)\right)\right)
            \end{equation*}
            Fijado $x$ y $k$, definimos el conjunto:
            \begin{equation*}
                X_{x,k} = \left\{f\in C(\mathbb{R}) : |f(x)-L|<\dfrac{1}{k}\right\}
            \end{equation*}
            Y la fórmula nos da el conjunto:
            \begin{equation*}
                \bigcap_{k\in \mathbb{N}} \bigcup_{M\in \mathbb{N}} \bigcap_{\substack{x>M\\x\in \mathbb{Q}}} X_{x,k}
            \end{equation*}
        \item Finalmente, si ahora consideramos el mismo espacio $X$ y la pensamos en la propiedad de que una función tenga límite en infinito:
            \begin{equation*}
                \forall L\in \mathbb{R}\ \forall \veps > 0\ \exists M>0\ \forall x>M\ |f(x)-L|<\veps
            \end{equation*}
            Sabemos que podemos sustituir $\veps$, $M$ y $x$ para obtener intersecciones y uniones numerables, pero ¿cómo podemos ahora hacer esto con $L$? Pues bien, podemos usar un resultado bien conocido, y es que $\mathbb{R}$ es completo, por lo que cualquier sucesión de Cauchy es convergente y viceversa, con lo que podemos reescribir esta fórmula en una equivalente de la forma:
            \begin{equation*}
                \forall k\in \mathbb{N}\ \exists M\in \mathbb{N}\ \forall x,y\in \mathbb{Q}, x,y>M\ |f(x)-f(y)|<\veps
            \end{equation*}
    \end{itemize}
\end{ejemplo}~\\
\noindent
Con esta gran cantidad de ejemplos hemos visto cómo podemos obtener conjuntos a partir de fórmulas, así como tratar de buscar siempre una cantidad numerable de intersecciones y uniones, que generalmente obtendremos usando teoremas fundamentales del espacio en el que trabajemos.\\

En resumen, fijado un conjunto $X$, nos interesará dar propiedades mediante fórmulas, a partir de las cuales construir conjuntos mediante intersecciones y uniones (preferiblamente numerables) de conjuntos, las cuales vendrán dadas por los cuantificadores que hemos usado en la fórmula para describir una propiedad específica. De esta forma, dada una cierta propiedad y considerando una cierta topología $\cc{T}$ sobre el espacio $X$, los conjuntos que obtengamos podrán ser:
\begin{itemize}
    \item Abiertos.
    \item Cerrados.
    \item Intersecciones numerables de abiertos.
    \item Uniones arbitrarias de cerrados.
    \item Uniones numerables de intersecciones numerables de abiertos.
    \item Intersecciones numerables de uniones numerables de cerrados.
    \item \ldots
\end{itemize}
De esta forma, llegaremos luego a considerar una $\sigma-$álgebra de Borel, que será donde podamos trabajar.\\

Finalmente nos preguntamos si toda fórmula puede reducirse a unos cuantificadores numerables, pregunta cuya respuesta será que no\footnote{Se verá un leve razonamiento de por qué.}, y esta respuesta nos hará interesarnos por una noción más general de los cardinales de los conjuntos.
