\section{Sistemas Matemáticos}

\begin{ejercicio}
    En un sistema de primer orden con igualdad, demuestra que las siguientes fórmulas son teoremas:
    \begin{enumerate}
        \item $\forall x (x = x)$.
        
        Se tiene de forma directa por ser un axioma.
        \item $\forall x \forall y (x = y \rightarrow y = x)$.
        \begin{enumerate}[label=\arabic*.]
            \item $\forall x(x=x)\in \cc{A}_6$.
            \item $\forall x(x=x)\to (x=x)\in \cc{A}_4$.
            \item $x=x$ por modus ponens de 1 y 2.
            \item $(x=x)\to [(x=y)\to (x=x)]\in \cc{A}_1$.
            \item $(x=y)\to (x=x)$ por modus ponens de 3 y 4.
            \item $(x=y)\to [(x=x)\to (y=x)]\in \cc{A}_7$.
            \item $(x=x)\to [(x=y)\to (y=x)]$ por la Regla de Conmutación de las Premisas aplicada a 6.
            \item $(x=y)\to (y=x)$ por modus ponens de 3 y 7.
            \item $\forall x\forall y (x=y\to y=x)$ por generalización en $y,x$.
        \end{enumerate}
        
        \item $\forall x \forall y \forall z (x = y \rightarrow (y = z \rightarrow x = z))$.
        \begin{enumerate}[label=\arabic*.]
            \item $(y = x) \rightarrow (y = z \rightarrow x = z)\in \cc{A}_7$
            \item $\forall x\forall y (x=y\to y=x)$ por el apartado anterior.
            \item $(x=y\to y=x)$ tras aplicar dos veces el axioma $4$ y modus ponens.
            \item $(x = y) \rightarrow (y = z \rightarrow x = z)$ por la Regla del Silogismo aplicada a $3$ y $1$.
            \item $\forall x \forall y \forall z (x = y \rightarrow (y = z \rightarrow x = z))$ tras generalizar en $z,y,x$.
        \end{enumerate}
    \end{enumerate}
\end{ejercicio}

\begin{ejercicio}
    Sean $n, m \in \mathbb{N}$. En la aritmética de primer orden $\cc{N}$ prueba que:
    \begin{enumerate}
        \item si $n \neq m$, entonces $\vdash_{\cc{N}} \neg (s^n(0) = s^m(0))$,
        
        Con vistas a aplicar el Teorema de Reducción al Absurdo débil, suponemos como única hipótesis $s^n(0) = s^m(0)$. Además, podemos suponer $n>m$ sin pérdida de generalidad (en caso contrario, tendremos que aplicar la propiedad simétrica de la igualdad, demostrada en el ejercicio anterior). Entonces, tenemos que:
        \begin{enumerate}[label=\arabic*.]
            \item $s(s^{n-1}(0)) = s(s^{m-1}(0))$ es una hipótesis.
            \item $s(s^{n-1}(0)) = s(s^{m-1}(0))\to (s^{n-1}(0)=s^{m-1}(0))\in \cc{N}_2$.
            \item $s^{n-1}(0)=s^{m-1}(0)$ por modus ponens de 1 y 2.\\
            $\vdots$
            \item [$3(m-1).$] $s^{n-m+1}(0)=s(0)$ por modus ponens de $3(m-1)-1$ y $3(m-1)-2$.
            \item [$3(m-1)+1.$] $s\left(s^{n-m}(0)\right)=s(0)\to \left(s^{n-m}(0)=0\right)\in \cc{N}_2$.
            \item [$3(m-1)+2.$] $s^{n-m}(0)=0$ por modus ponens de $3(m-1)+1$ y $3(m-1)$.
            \item [$3m.$] $s(s^{n-m-1}(0))\neq 0\in \cc{N}_2$, puesto que $n-m-1\geq 0$.
        \end{enumerate}

        Por el Teorema de Reducción al Absurdo débil, se concluye que $s^n(0) \neq s^m(0)$.\\
        \item si $n = m$, entonces $\vdash_{\cc{N}} (s^n(0) = s^m(0))$,
        
        Como $n=m$, y $s$ es una aplicación, se tiene que $s^n(0)$ y $s^m(0)$ son el mismo elemento. Por lo tanto:
        \begin{enumerate}[label=\arabic*.]
            \item $s^n(0) = s^m(0)\in \cc{A}_6$.
        \end{enumerate}~
        \item $\vdash_{\cc{N}} (s^n(0) + s^m(0) = s^{n+m}(0))$.
        \begin{enumerate}[label=\arabic*.]
            \item $s^{n}(0)+ s(s^{m-1}(0)) = s(s^{n}(0)+s^{m-1}(0))\in \cc{N}_2$.\\
            \vdots
            \item[$m-1$.] $s^{m-2}(s^{n}(0)+s(s(0))) = s^{m-1}(s^{n}(0)+s(0))\in \cc{N}_2$.
            \item[$m$.] $s^{m-1}(s^{n}(0)+s(0))=s^{m}(s^{n}(0)+0)\in \cc{N}_2$.
            \item[$m+1$.] $s^{n}(0)+0=s^n(0)\in \cc{N}_3$.
            \item[$m+2$.] $s^{n}(0)+0=s^n(0) \to s^{m}(s^{n}(0)+0)=s^m(s^n(0))\in \cc{A}_7$.
            \item[$m+3$.] $s^{m-1}(s^{n}(0)+s(0))=s^{m}(s^{n}(0))$ por Modus ponens de $m$ y $m+2$ y transitividad.
            \item[$m+4$.] $s^n(0) + s^m(0) = s^{n+m}(0)$ tras aplicar la transitividad a $1,\dots,m-1$ y $m+3$.
        \end{enumerate}
    \end{enumerate}
\end{ejercicio}

\begin{ejercicio}
    Usando los ejercicios anteriores, prueba que para todo $n, m \in \mathbb{N}$, $n + m = m + n$ en $\cc{N}$, a saber,
    \[
        \vdash_{\cc{N}} s^n(0) + s^m(0) = s^m(0) + s^n(0).
    \]
    \begin{enumerate}
        \item $s^n(0) + s^m(0)=s^{n+m}(0)$ por el ejercicio anterior.
        \item $s^{n+m}(0)=s^{m+n}(0)$ por la conmutatividad en $\bb{N}$.
        \item $s^m(0) + s^n(0)=s^{m+n}(0)$ por el ejercicio anterior.
        \item $s^{m+n}(0)=s^m(0) + s^n(0)$ por la simetría de la igualdad.
        \item $s^n(0) + s^m(0)=s^m(0) + s^n(0)$ por la transitividad aplicada a $1,2,4$.
    \end{enumerate}
\end{ejercicio}
\begin{comment}




Ejercicio 1. En un sistema de primer orden con igualdad, demuestra que las siguientes formulas son teoremas. ´
1. ∀x(x = x).
2. ∀x∀y(x = y → y = x).
3. ∀x∀y∀z(x = y → (y = z → x = z)).
Ejercicio 2. Sean n, m ∈ N. En la aritmetica de primer orden ´ N prueba que
1. si n , m, entonces `N ¬(s
n
(0) = s
m(0)),
2. si n = m, entonces `N (s
n
(0) = s
m(0)),
3. `N (s
n
(0) + s
m(0) = s
n+m(0)).
Ejercicio 3. Usando los ejercicios anteriores, prueba que para todo n, m ∈ N, n+m =
m + n en N, a saber,
`N s
n
(0) + s
m
(0) = s
m
(0) + s
n
(0).
\end{comment}