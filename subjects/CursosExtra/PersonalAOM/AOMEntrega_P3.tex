\documentclass[12pt]{article}
% Idioma y codificación
\usepackage[spanish, es-tabla]{babel}       %es-tabla para que se titule "Tabla"
\usepackage[utf8]{inputenc}

% Márgenes
\usepackage[a4paper,top=3cm,bottom=2.5cm,left=3cm,right=3cm]{geometry}

% Comentarios de bloque
\usepackage{verbatim}

% Paquetes de links
\usepackage[hidelinks]{hyperref}    % Permite enlaces
\usepackage{url}                    % redirecciona a la web

% Más opciones para enumeraciones
\usepackage{enumitem}

% Personalizar la portada
\usepackage{titling}

% Paquetes de tablas
\usepackage{multirow}


%------------------------------------------------------------------------

%Paquetes de figuras
\usepackage{caption}
\usepackage{subcaption} % Figuras al lado de otras
\usepackage{float}      % Poner figuras en el sitio indicado H.


% Paquetes de imágenes
\usepackage{graphicx}       % Paquete para añadir imágenes
\usepackage{transparent}    % Para manejar la opacidad de las figuras

% Paquete para usar colores
\usepackage[dvipsnames]{xcolor}
\usepackage{pagecolor}      % Para cambiar el color de la página

% Habilita tamaños de fuente mayores
\usepackage{fix-cm}

% Para los gráficos
\usepackage{tikz}

% Para poder situar los nodos en los grafos
\usetikzlibrary{positioning}


%------------------------------------------------------------------------

% Paquetes de matemáticas
\usepackage{mathtools, amsfonts, amssymb, mathrsfs}
\usepackage[makeroom]{cancel}     % Simplificar tachando
\usepackage{polynom}    % Divisiones y Ruffini
\usepackage{units} % Para poner fracciones diagonales con \nicefrac

\usepackage{pgfplots}   %Representar funciones
\pgfplotsset{compat=1.18}  % Versión 1.18

\usepackage{tikz-cd}    % Para usar diagramas de composiciones
\usetikzlibrary{calc}   % Para usar cálculo de coordenadas en tikz

%Definición de teoremas, etc.
\usepackage{amsthm}
%\swapnumbers   % Intercambia la posición del texto y de la numeración

\theoremstyle{plain}

\makeatletter
\@ifclassloaded{article}{
  \newtheorem{teo}{Teorema}[section]
}{
  \newtheorem{teo}{Teorema}[chapter]  % Se resetea en cada chapter
}
\makeatother

\newtheorem{coro}{Corolario}[teo]           % Se resetea en cada teorema
\newtheorem{prop}[teo]{Proposición}         % Usa el mismo contador que teorema
\newtheorem{lema}[teo]{Lema}                % Usa el mismo contador que teorema

\theoremstyle{remark}
\newtheorem*{observacion}{Observación}

\theoremstyle{definition}

\makeatletter
\@ifclassloaded{article}{
  \newtheorem{definicion}{Definición} [section]     % Se resetea en cada chapter
}{
  \newtheorem{definicion}{Definición} [chapter]     % Se resetea en cada chapter
}
\makeatother

\newtheorem*{notacion}{Notación}
\newtheorem*{ejemplo}{Ejemplo}
\newtheorem*{ejercicio*}{Ejercicio}             % No numerado
\newtheorem{ejercicio}{Ejercicio} [section]     % Se resetea en cada section


% Modificar el formato de la numeración del teorema "ejercicio"
\renewcommand{\theejercicio}{%
  \ifnum\value{section}=0 % Si no se ha iniciado ninguna sección
    \arabic{ejercicio}% Solo mostrar el número de ejercicio
  \else
    \thesection.\arabic{ejercicio}% Mostrar número de sección y número de ejercicio
  \fi
}


% \renewcommand\qedsymbol{$\blacksquare$}         % Cambiar símbolo QED
%------------------------------------------------------------------------

% Paquetes para encabezados
\usepackage{fancyhdr}
\pagestyle{fancy}
\fancyhf{}

\newcommand{\helv}{ % Modificación tamaño de letra
\fontfamily{}\fontsize{12}{12}\selectfont}
\setlength{\headheight}{15pt} % Amplía el tamaño del índice


%\usepackage{lastpage}   % Referenciar última pag   \pageref{LastPage}
\fancyfoot[C]{\thepage}

%------------------------------------------------------------------------

% Conseguir que no ponga "Capítulo 1". Sino solo "1."
\makeatletter
\@ifclassloaded{book}{
  \renewcommand{\chaptermark}[1]{\markboth{\thechapter.\ #1}{}} % En el encabezado
    
  \renewcommand{\@makechapterhead}[1]{%
  \vspace*{50\p@}%
  {\parindent \z@ \raggedright \normalfont
    \ifnum \c@secnumdepth >\m@ne
      \huge\bfseries \thechapter.\hspace{1em}\ignorespaces
    \fi
    \interlinepenalty\@M
    \Huge \bfseries #1\par\nobreak
    \vskip 40\p@
  }}
}
\makeatother

%------------------------------------------------------------------------
% Paquetes de cógido
\usepackage{minted}
\renewcommand\listingscaption{Código fuente}

\usepackage{fancyvrb}
% Personaliza el tamaño de los números de línea
\renewcommand{\theFancyVerbLine}{\small\arabic{FancyVerbLine}}

% Estilo para C++
\newminted{cpp}{
    frame=lines,
    framesep=2mm,
    baselinestretch=1.2,
    linenos,
    escapeinside=||
}

% para minted
\definecolor{LightGray}{rgb}{0.95,0.95,0.92}
\setminted{
    linenos=true,
    stepnumber=5,
    numberfirstline=true,
    autogobble,
    breaklines=true,
    breakautoindent=true,
    breaksymbolleft=,
    breaksymbolright=,
    breaksymbolindentleft=0pt,
    breaksymbolindentright=0pt,
    breaksymbolsepleft=0pt,
    breaksymbolsepright=0pt,
    fontsize=\footnotesize,
    bgcolor=LightGray,
    numbersep=10pt
}


\usepackage{listings} % Para incluir código desde un archivo

\renewcommand\lstlistingname{Código Fuente}
\renewcommand\lstlistlistingname{Índice de Códigos Fuente}

% Definir colores
\definecolor{vscodepurple}{rgb}{0.5,0,0.5}
\definecolor{vscodeblue}{rgb}{0,0,0.8}
\definecolor{vscodegreen}{rgb}{0,0.5,0}
\definecolor{vscodegray}{rgb}{0.5,0.5,0.5}
\definecolor{vscodebackground}{rgb}{0.97,0.97,0.97}
\definecolor{vscodelightgray}{rgb}{0.9,0.9,0.9}

% Configuración para el estilo de C similar a VSCode
\lstdefinestyle{vscode_C}{
  backgroundcolor=\color{vscodebackground},
  commentstyle=\color{vscodegreen},
  keywordstyle=\color{vscodeblue},
  numberstyle=\tiny\color{vscodegray},
  stringstyle=\color{vscodepurple},
  basicstyle=\scriptsize\ttfamily,
  breakatwhitespace=false,
  breaklines=true,
  captionpos=b,
  keepspaces=true,
  numbers=left,
  numbersep=5pt,
  showspaces=false,
  showstringspaces=false,
  showtabs=false,
  tabsize=2,
  frame=tb,
  framerule=0pt,
  aboveskip=10pt,
  belowskip=10pt,
  xleftmargin=10pt,
  xrightmargin=10pt,
  framexleftmargin=10pt,
  framexrightmargin=10pt,
  framesep=0pt,
  rulecolor=\color{vscodelightgray},
  backgroundcolor=\color{vscodebackground},
}

%------------------------------------------------------------------------

% Comandos definidos
\newcommand{\bb}[1]{\mathbb{#1}}
\newcommand{\cc}[1]{\mathcal{#1}}

% I prefer the slanted \leq
\let\oldleq\leq % save them in case they're every wanted
\let\oldgeq\geq
\renewcommand{\leq}{\leqslant}
\renewcommand{\geq}{\geqslant}

% Si y solo si
\newcommand{\sii}{\iff}

% Letras griegas
\newcommand{\eps}{\epsilon}
\newcommand{\veps}{\varepsilon}
\newcommand{\lm}{\lambda}

\newcommand{\ol}{\overline}
\newcommand{\ul}{\underline}
\newcommand{\wt}{\widetilde}
\newcommand{\wh}{\widehat}

\let\oldvec\vec
\renewcommand{\vec}{\overrightarrow}

% Derivadas parciales
\newcommand{\del}[2]{\frac{\partial #1}{\partial #2}}
\newcommand{\Del}[3]{\frac{\partial^{#1} #2}{\partial #3^{#1}}}
\newcommand{\deld}[2]{\dfrac{\partial #1}{\partial #2}}
\newcommand{\Deld}[3]{\dfrac{\partial^{#1} #2}{\partial #3^{#1}}}


\newcommand{\AstIg}{\stackrel{(\ast)}{=}}
\newcommand{\Hop}{\stackrel{L'H\hat{o}pital}{=}}

\newcommand{\red}[1]{{\color{red}#1}} % Para integrales, destacar los cambios.

% Método de integración
\newcommand{\MetInt}[2]{
    \left[\begin{array}{c}
        #1 \\ #2
    \end{array}\right]
}

% Declarar aplicaciones
% 1. Nombre aplicación
% 2. Dominio
% 3. Codominio
% 4. Variable
% 5. Imagen de la variable
\newcommand{\Func}[5]{
    \begin{equation*}
        \begin{array}{rrll}
            #1:& #2 & \longrightarrow & #3\\
               & #4 & \longmapsto & #5
        \end{array}
    \end{equation*}
}

%------------------------------------------------------------------------


\author{Arturo Olivares Martos}
\date{\today}
\title{Entrega Ejercicios Microcredencial. Parte 3}

\begin{document}
    \maketitle
    \begin{abstract}
        En el presente documento, resolveremos ejercicios de la tercera parte de la Microcredencial de Lógica y Teoría Descriptiva de Conjuntos.
    \end{abstract}



    \begin{ejercicio}
        Definimos los siguientes conjuntos:
        \begin{align*}
           Q_2 &= \left\{ \alpha \in \cc{C} \;\mid\; \exists A\subset \mathbb{N}\ \text{finito tal que}\ \alpha(n)=0\ \forall n\in \mathbb{N}\setminus\{A\}\right\} \\
           \ell^1 &= \left\{ x\in [0,1]^\mathbb{N}\ \middle|\ \sum_{n=1}^\infty x_n < \infty\right\}
        \end{align*}
        Demostrar que $\ell^1\in \Sigma_2^0$ y $Q_2 \leq_W \ell^1$.
    \end{ejercicio}

    \begin{ejercicio}
        Sea $\Gamma$ una clase de la Jerarquía Boreliana, y $X$ un conjunto. Si $A\subset X$ es $\Gamma-$completo, y $B\subset X$ es otro conjunto de la clase $\Gamma$ tal que $A\leq_W B$, entonces $B$ es $\Gamma-$completo.
        \begin{proof}
            Hemos de comprobar que:
        \begin{itemize}
            \item \ul{$B\in \Gamma$}: Se tiene por hipótesis.
            \item \ul{Para todo espacio polaco $X'$, si $C\in \Gamma(X')$ entonces $C\leq_W B$}:
            
            Sea $C\in \Gamma(X')$, y buscamos $f:X'\to X$ tal que $f$ es una función continua y $C=f^{-1}(B)$.
            
            Como $A$ es $\Gamma-$completo, existe $g:X'\to X$ tal que $g$ es continua y $C=g^{-1}(A)$. Por otro lado, como $A\leq_W B$, existe una función continua $h:X\to X$ tal que $A=h^{-1}(B)$. Entonces, la composición $f=h\circ g$ es continua y cumple que:
            \begin{equation*}
                f^{-1}(B) = g^{-1}(h^{-1}(B)) = g^{-1}(A) = C
            \end{equation*}

            Por tanto, $C\leq_W B$.
        \end{itemize}
        \end{proof}
    \end{ejercicio}


    \begin{ejercicio}
        Demostrar que $f:[0,1]\to \bb{R}$ es continuamente derivable si y solo si
        \begin{equation*}
            \forall\veps\in \bb{R}^+\ \exists\delta\in \bb{R}^+\ f\in A_{\veps,\delta}
        \end{equation*}
        donde:
        \begin{multline*}
            A_{\veps,\delta} = \left\{f\in C([0,1])\ \middle|\ \forall x,y,a,b\in [0,1]\ : a,b,x,y \text{ a distancia } \leq \delta\right. \Longrightarrow \\\Longrightarrow \left.\left|\frac{f(a)-f(b)}{a-b} - \frac{f(x)-f(y)}{x-y}\right| < \veps\right\}
        \end{multline*}
        \begin{proof}
            Sea $f:[0,1]\to \bb{R}$. Demostraremos por doble implicación.
            \begin{description}
                \item[$\Longrightarrow)$] Sea $f\in C^1([0,1])$, y sea $\veps\in \bb{R}^+$. Como $f$ es derivable en $[0,1]$, por el Teorema del Valor Medio existe $a'\in \left]a,b\right[, x'\in \left]x,y\right[$ tal que:
                \begin{equation*}
                    \frac{f(a)-f(b)}{a-b} = f'(a')\qquad \text{y}\qquad \frac{f(x)-f(y)}{x-y} = f'(x')
                \end{equation*}

                Por el Teorema de Heine, como $[0,1]$ es compacto y $f'$ es continua, existe $\delta'\in \bb{R}^+$ tal que:
                \begin{equation*}
                    |a'-x'| < \delta'
                    \Longrightarrow |f'(a') - f'(x')| < \veps
                \end{equation*}

                Sea ahora $\delta = \nicefrac{\delta'}{3}$. Usando que $a,b,x,y$ están a distancia $\leq \delta$, veamos que $|x'-a'| < \delta'$:
                \begin{align*}
                    |x'-a'| &\leq |x'-x| + |x-a| + |a-a'|< |x-y| + |x-a| + |a-b|\leq 3\delta = \delta'
                \end{align*}

                Por tanto, se verifica que:
                \begin{align*}
                    \left|\frac{f(a)-f(b)}{a-b} - \frac{f(x)-f(y)}{x-y}\right| &= |f'(a') - f'(x')| < \veps
                \end{align*}

                Por tanto, $f\in A_{\veps,\delta}$.


                \item[$\Longleftarrow)$] Hemos de demostrar que $f$ es continuamente derivable. Para ello, definimos el cociente incremental de $f$ en $t\in [0,1]$ como:
                \begin{equation*}
                    f_t(x) = \frac{f(x)-f(t)}{x-t} \qquad \forall x\in [0,1]\setminus\{t\}
                \end{equation*}

                En primer lugar, hemos de ver que $f$ es derivable, para lo cual hemos de comprobar que, para cada $t\in [0,1]$, el siguiente límite existe:
                \begin{equation*}
                    \lim_{x\to t} f_t(x)
                \end{equation*}

                Para comprobar que este límite existe, usaremos que $\bb{R}$ es completo, por lo que toda sucesión de Cauchy converge. Sea $t\in [0,1]$, y sea $\{x_n\}_{n\in \bb{N}}$ una sucesión de puntos de $[0,1]\setminus\{t\}$ tal que $\{x_n\}\to t$. Veamos que $\{f_t(x_n)\}_{n\in \bb{N}}$ es una sucesión de Cauchy. Para ello, fijamos $\veps\in \bb{R}^+$, por lo que $\exists \delta\in \bb{R}^+$ tal que $f\in A_{\veps,\delta}$. Por ser $\{x_n\}$ de Cauchy, existe $N\in \bb{N}$ tal que, para todo $m,n\geq N$, se verifica que:
                \begin{equation*}
                    |x_m-x_n| < \delta
                \end{equation*}

                Por tanto, $t,x_m,x_n$ están a distancia $\leq \delta$, y por tanto, como $f\in A_{\veps,\delta}$, se verifica que:
                \begin{align*}
                    \left|f_t(x_m) - f_t(x_n)\right| &= \left|\frac{f(x_m)-f(t)}{x_m-t} - \frac{f(x_n)-f(t)}{x_n-t}\right|< \veps
                \end{align*}

                Por tanto, $\{f_t(x_n)\}_{n\in \bb{N}}$ es una sucesión de Cauchy, y por tanto, converge a un límite $f'(t)\in \bb{R}$. Definimos por tanto:
                \begin{equation*}
                    f'(t) = \lim_{x\to t} f_t(x)
                \end{equation*}

                Ahora, queremos demostrar que \( f' \) es continua en todo \( [0,1] \). Para ello, fijamos un punto \( x \in [0,1] \), y tomamos una sucesión \( \{t_n\}_{n \in \mathbb{N}} \subset [0,1] \setminus \{x\} \) tal que \( \{t_n\} \to x \). Queremos ver que:
                \[
                \lim_{n \to \infty} f'(t_n) = f'(x)
                \]

                Recordemos que, por definición,
                \[
                f'(t_n) = \lim_{y \to t_n} \frac{f(y) - f(t_n)}{y - t_n}
                \quad \text{y} \quad
                f'(x) = \lim_{z \to x} \frac{f(z) - f(x)}{z - x}
                \]

                Fijado ahora $\veps\in \bb{R}^+$, consideramos $\delta \in \bb{R}^+$ tal que \( f \in A_{\varepsilon,\delta} \).  Como \( \{t_n\} \to x \), existe \( N \in \mathbb{N} \) tal que para todo \( n \geq N \), se tiene \( |t_n - x| < \nicefrac{\delta}{2} \). Fijamos tal \( n \geq N \), y tomamos \( y \in [0,1] \) con \( |y - t_n| < \nicefrac{\delta}{2} \). Entonces, por desigualdad triangular:
                \[
                |y - x| \leq |y - t_n| + |t_n - x| < \nicefrac{\delta}{2} + \nicefrac{\delta}{2} = \delta
                \]

                Por tanto, \( x, y, t_n \) están a distancia menor que \( \delta \), y podemos aplicar la hipótesis:
                \[
                \left| \frac{f(y) - f(t_n)}{y - t_n} - \frac{f(z) - f(x)}{z - x} \right| < \varepsilon
                \quad \forall z\in [0,1]\ \text{tal que}\ |x-z|<\delta
                \]

                Tomando el límite cuando \( y \to t_n \), se obtiene:
                \[
                \left| f'(t_n) - \frac{f(z) - f(x)}{z - x} \right| < \varepsilon
                \quad \forall z\in [0,1]\ \text{tal que}\ |x-z|<\delta
                \]

                Y tomando después el límite cuando \( z \to x \), se concluye que $|f'(t_n) - f'(x)| < \varepsilon$. Como $n\geq N$ era arbitrario, tenemos que:
                \[
                \lim_{n \to \infty} f'(t_n) = f'(x)
                \]

                Es decir, \( f' \) es continua en \( x \). Como \( x \in [0,1] \) era arbitrario, concluimos que \( f' \) es continua en todo el intervalo, y por tanto, \( f \in C^1([0,1]) \).
            \end{description}
        \end{proof}
    \end{ejercicio}
\end{document}
