\section{Estructuras de SO's}

\begin{ejercicio}
    Cuestiones generales relacionadas con un SO:
    \begin{enumerate}
        \item ¿Qué es el núcleo (kernel) de un SO?

        El kernel de un SO es un programa que reside en RAM que contiene las llamadas al sistema (funciones relacionadas directamente con el hardware del ordenador).
        
        Como son funciones a tan bajo nivel, no se puede acceder a ellas en modo usuario. Cuando se ejecuta una llamada al sistema, se cambia el bit de modo de la PSW a 0, y se entra en modo kernel.
        
        \item ¿Qué es un modelo de memoria (interpretación del espacio de memoria) para un programa? Explique los diferentes modelos de memoria para la arquitectura IA-32.

        El modelo de memoria es la forma que tiene la CPU de interpretar los accesos a memoria. El espacio de direcciones son todas las direcciones disponibles en un ordenador, y va desde $0$ hasta $2^n-1$, siendo $n$ el número de bits del bus de direcciones. Los diferentes modelos de memoria del IA-32, cuyo espacio de direcciones es $\{0,\dots,2^{32}-1\}$, son:
        \begin{enumerate}
            \item \ul{Flat memory model}:

            Es un espacio lineal que direcciona todo el espacio de direcciones, desde $0$ hasta $2^{32}-1$. Puede direccionar cada byte, por lo que se dice que lo hace con granularidad de un byte.

            \item\ul{Segmented memory model}:

            Usa segmentación. Cada dirección lógica consta de un selector de segmento y de un desplazamiento dentro de dicho segmento.

            Consta también de una tabla de segmentos en la que, entre otros aspectos, se encuentra dónde comienza cada segmento.

            Se puede identificar una gran cantidad de segmentos, cada uno con un tamaño límite de $2^{32}$ bytes (tamaño de la memoria).

            \item \ul{Real-adress mode memory model}: 

            Muy similar al modelo de memoria segmentada, aunque este se mantiene por compatibilidad hacia atrás con otras arquitecturas. En este caso hay limitaciones de tamaño tanto para los segmentos como para la memoria total.
        \end{enumerate}

        \item ¿Cómo funciona el mecanismo de tratamiento de interrupciones mediante interrupciones vectorizadas? Explique que parte es realizada por el hardware y que parte por el software.
        
        \item Describa detalladamente los pasos que lleva a cabo el SO cuando un programa solicita una llamada al sistema.
    \end{enumerate}
\end{ejercicio}

\begin{ejercicio}
    Explique tres responsabilidades asignadas al gestor de memoria de un SO y tres asignadas al gestor de procesos.
\end{ejercicio}

\begin{ejercicio}
    ¿Cómo gestionaría el sistema operativo la posibilidad de anidamiento de interrupciones?
\end{ejercicio}

\begin{ejercicio}
    Contraste las ventajas e inconvenientes de una arquitectura de SO monolítica frente a una arquitectura microkernel.
\end{ejercicio}

\begin{ejercicio}
    Cuestiones relacionas con virtualización:
    \begin{enumerate}
        \item ¿Qué se entiende actualmente por virtualización mediante hipervisor?
        
        \item ¿Qué clases de hipervisores existen de manera general y que ventajas e inconvenientes plantea una clase con respecto a la otra?
    \end{enumerate}
\end{ejercicio}

\begin{ejercicio}
    Cuestiones relacionadas con RTOS:
    \begin{enumerate}
        \item ¿Qué característica distingue esencialmente a un proceso de tiempo real de otro que no lo es?

        \item ¿Cuáles son los factores determinantes del tiempo de respuesta en un RTOS, e.d. define determinismo y reactividad?
    \end{enumerate}
\end{ejercicio}