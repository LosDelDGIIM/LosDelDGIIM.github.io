\section{Vectores Aleatorios}

\begin{ejercicio}
    Asociadas al experimento aleatorio de lanzar un dado y una moneda no cargados, se define la variable $X$ como el valor del dado y la variable $Y$, que toma el valor 0 si sale cara en la moneda, y 1 si sale cruz. Calcular la función masa de probabilidad y la función de distribución del vector aleatorio $(X,Y)$.
\end{ejercicio}

\begin{ejercicio}
    El número de automóviles utilitarios, $X$, y el de automóviles de lujo, $Y$, que poseen las familias de una población se distribuye de acuerdo a las siguientes probabilidades:
    \begin{table}[H]
        \centering
        \begin{tabular}{c|ccc}
            $X\backslash Y$ & 0 & 1 & 2 \\
            \hline
            0 & \nicefrac{1}{3} & \nicefrac{1}{12} & \nicefrac{1}{24} \\
            1 & \nicefrac{1}{6} & \nicefrac{1}{24} & \nicefrac{1}{48} \\
            2 & \nicefrac{5}{22} & \nicefrac{5}{88} & \nicefrac{5}{176} \\
        \end{tabular}
    \end{table}
    Calcular la función de distribución del vector $(X,Y)$ en los puntos $(0,0)$; $(0,2)$; $(1,1)$ y $(2,1)$, y la probabilidad de que una familia tenga tres o más automóviles.
\end{ejercicio}

\begin{ejercicio}
    La función de densidad del vector aleatorio $(X,Y)$, donde $X$ denota los Kg. de naranjas, e $Y$ los Kg. de manzanas vendidos al día en una frutería está dada por
    \[
        f(x, y) = \frac{1}{400}; \quad 0 < x < 20; \quad 0 < y < 20.
    \]
    Determinar la función de distribución de $(X,Y)$ y la probabilidad de que en un día se vendan
    entre naranjas y manzanas, menos de 20 kilogramos.
\end{ejercicio}

\begin{ejercicio}
    La renta, $X$, y el consumo, $Y$, de los habitantes de una población, tienen por funciones de densidad
    \[
        f_X(x) = 2-2x; \quad 0 < x < 1; \quad f(y/x) = \frac{1}{x}; \quad 0 < y < x.
    \]
    Determinar la función de densidad conjunta del vector aleatorio $(X,Y)$ y la probabilidad de que el consumo sea inferior a la mitad de la renta.
\end{ejercicio}

\begin{ejercicio}
    Una gasolinera tiene $Y$ miles de litros en su depósito de gasóleo al comienzo de cada semana. A lo largo de una semana se venden $X$ miles de litros del citado combustible, siendo la función de densidad conjunta de $(X,Y)$ :
    \[
        f(x, y) = \frac{1}{8}; \quad 0 < x < y < 4.
    \]
    Se pide:
    \begin{enumerate}
        \item Probar que $f(x, y)$ es función de densidad y obtener la función de distribución.
        \item Probabilidad de que en una semana se venda más de la tercera parte de los litros de que se dispone al comienzo de la misma.
        \item Si en una semana se han vendido 3.000 litros de gasóleo, ¿cuál es la probabilidad de que al comienzo de la semana hubiese entre 3.500 y 3.750 litros de combustible?
    \end{enumerate}
\end{ejercicio}

\begin{ejercicio}
    Sea $(X,Y)$ un vector aleatorio continuo con función de densidad
    \[
        f(x, y) = k, \quad (x, y) \in R,
    \]
    siendo $R$ el rombo de vértices $(3,0)$; $(0,2)$; $(-3,0)$; $(0,-2)$. Calcular $k$ para que $f$ sea una función de densidad. Hallar las distribuciones marginales y condicionadas.
\end{ejercicio}

\begin{ejercicio}
    Sea $(X,Y)$ un vector aleatorio continuo con función de densidad
    \[
        f(x, y) = k, \quad x^2 \leq y \leq 1,
    \]
    anulándose fuera del recinto indicado. Hallar la constante $k$ para que $f$ sea una función de densidad de probabilidad y calcular la función de distribución de probabilidad. Calcular $P(X \geq Y)$. Calcular las distribuciones marginales y condicionadas.
\end{ejercicio}

\begin{ejercicio}
    Sea la función de densidad de probabilidad
    \[
        f(x, y) = \begin{cases}
            kxy^2 + 1, & 0 < x < 1, -1 < y < 1, \\
            0, & \text{en otro caso}.
        \end{cases}
    \]
    Calcular la función de distribución de probabilidad y las marginales.
\end{ejercicio}

\begin{ejercicio}
    Sea $(X,Y)$ un vector aleatorio bidimensional continuo, con función de densidad de probabilidad
    \[
        f(x, y) = \begin{cases}
            k, & 0 < x + y < 1, |y| < 1, 0 < x < 1, \\
            0, & \text{en otro caso}.
        \end{cases}
    \]
    Hallar la constante $k$ para que $f$ sea una función de densidad de probabilidad y calcular la función de distribución de probabilidad. Calcular las distribuciones marginales y condicionadas.
\end{ejercicio}

\begin{ejercicio}
    Sea $(X,Y)$ un vector aleatorio bidimensional continuo, con distribución de probabilidad uniforme sobre el triángulo de vértices $(0,0)$; $(0,1)$; $(1,1)$. Determinar la función de densidad de probabilidad, la función de distribución de probabilidad y las distribuciones marginales y condicionadas.
\end{ejercicio}

\begin{ejercicio}
    Sea $(X,Y)$ una variable aleatoria bidimensional con distribución uniforme en el recinto
    \[
        C = \{(x, y) \in \mathbb{R}^2; x^2 + y^2 < 1; x \geq 0, y \geq 0\}.
    \]
    Calcular:
    \begin{enumerate}
        \item La función de distribución conjunta.
        \item Las funciones de densidad marginales.
        \item Las funciones de densidad condicionadas.
    \end{enumerate}
\end{ejercicio}