\documentclass[12pt]{article}

% Idioma y codificación
\usepackage[spanish, es-tabla]{babel}       %es-tabla para que se titule "Tabla"
\usepackage[utf8]{inputenc}

% Márgenes
\usepackage[a4paper,top=3cm,bottom=2.5cm,left=3cm,right=3cm]{geometry}

% Comentarios de bloque
\usepackage{verbatim}

% Paquetes de links
\usepackage[hidelinks]{hyperref}    % Permite enlaces
\usepackage{url}                    % redirecciona a la web

% Más opciones para enumeraciones
\usepackage{enumitem}

% Personalizar la portada
\usepackage{titling}

% Paquetes de tablas
\usepackage{multirow}


%------------------------------------------------------------------------

%Paquetes de figuras
\usepackage{caption}
\usepackage{subcaption} % Figuras al lado de otras
\usepackage{float}      % Poner figuras en el sitio indicado H.


% Paquetes de imágenes
\usepackage{graphicx}       % Paquete para añadir imágenes
\usepackage{transparent}    % Para manejar la opacidad de las figuras

% Paquete para usar colores
\usepackage[dvipsnames]{xcolor}
\usepackage{pagecolor}      % Para cambiar el color de la página

% Habilita tamaños de fuente mayores
\usepackage{fix-cm}

% Para los gráficos
\usepackage{tikz}

% Para poder situar los nodos en los grafos
\usetikzlibrary{positioning}


%------------------------------------------------------------------------

% Paquetes de matemáticas
\usepackage{mathtools, amsfonts, amssymb, mathrsfs}
\usepackage[makeroom]{cancel}     % Simplificar tachando
\usepackage{polynom}    % Divisiones y Ruffini
\usepackage{units} % Para poner fracciones diagonales con \nicefrac

\usepackage{pgfplots}   %Representar funciones
\pgfplotsset{compat=1.18}  % Versión 1.18

\usepackage{tikz-cd}    % Para usar diagramas de composiciones
\usetikzlibrary{calc}   % Para usar cálculo de coordenadas en tikz

%Definición de teoremas, etc.
\usepackage{amsthm}
%\swapnumbers   % Intercambia la posición del texto y de la numeración

\theoremstyle{plain}

\makeatletter
\@ifclassloaded{article}{
  \newtheorem{teo}{Teorema}[section]
}{
  \newtheorem{teo}{Teorema}[chapter]  % Se resetea en cada chapter
}
\makeatother

\newtheorem{coro}{Corolario}[teo]           % Se resetea en cada teorema
\newtheorem{prop}[teo]{Proposición}         % Usa el mismo contador que teorema
\newtheorem{lema}[teo]{Lema}                % Usa el mismo contador que teorema

\theoremstyle{remark}
\newtheorem*{observacion}{Observación}

\theoremstyle{definition}

\makeatletter
\@ifclassloaded{article}{
  \newtheorem{definicion}{Definición} [section]     % Se resetea en cada chapter
}{
  \newtheorem{definicion}{Definición} [chapter]     % Se resetea en cada chapter
}
\makeatother

\newtheorem*{notacion}{Notación}
\newtheorem*{ejemplo}{Ejemplo}
\newtheorem*{ejercicio*}{Ejercicio}             % No numerado
\newtheorem{ejercicio}{Ejercicio} [section]     % Se resetea en cada section


% Modificar el formato de la numeración del teorema "ejercicio"
\renewcommand{\theejercicio}{%
  \ifnum\value{section}=0 % Si no se ha iniciado ninguna sección
    \arabic{ejercicio}% Solo mostrar el número de ejercicio
  \else
    \thesection.\arabic{ejercicio}% Mostrar número de sección y número de ejercicio
  \fi
}


% \renewcommand\qedsymbol{$\blacksquare$}         % Cambiar símbolo QED
%------------------------------------------------------------------------

% Paquetes para encabezados
\usepackage{fancyhdr}
\pagestyle{fancy}
\fancyhf{}

\newcommand{\helv}{ % Modificación tamaño de letra
\fontfamily{}\fontsize{12}{12}\selectfont}
\setlength{\headheight}{15pt} % Amplía el tamaño del índice


%\usepackage{lastpage}   % Referenciar última pag   \pageref{LastPage}
\fancyfoot[C]{\thepage}

%------------------------------------------------------------------------

% Conseguir que no ponga "Capítulo 1". Sino solo "1."
\makeatletter
\@ifclassloaded{book}{
  \renewcommand{\chaptermark}[1]{\markboth{\thechapter.\ #1}{}} % En el encabezado
    
  \renewcommand{\@makechapterhead}[1]{%
  \vspace*{50\p@}%
  {\parindent \z@ \raggedright \normalfont
    \ifnum \c@secnumdepth >\m@ne
      \huge\bfseries \thechapter.\hspace{1em}\ignorespaces
    \fi
    \interlinepenalty\@M
    \Huge \bfseries #1\par\nobreak
    \vskip 40\p@
  }}
}
\makeatother

%------------------------------------------------------------------------
% Paquetes de cógido
\usepackage{minted}
\renewcommand\listingscaption{Código fuente}

\usepackage{fancyvrb}
% Personaliza el tamaño de los números de línea
\renewcommand{\theFancyVerbLine}{\small\arabic{FancyVerbLine}}

% Estilo para C++
\newminted{cpp}{
    frame=lines,
    framesep=2mm,
    baselinestretch=1.2,
    linenos,
    escapeinside=||
}

% para minted
\definecolor{LightGray}{rgb}{0.95,0.95,0.92}
\setminted{
    linenos=true,
    stepnumber=5,
    numberfirstline=true,
    autogobble,
    breaklines=true,
    breakautoindent=true,
    breaksymbolleft=,
    breaksymbolright=,
    breaksymbolindentleft=0pt,
    breaksymbolindentright=0pt,
    breaksymbolsepleft=0pt,
    breaksymbolsepright=0pt,
    fontsize=\footnotesize,
    bgcolor=LightGray,
    numbersep=10pt
}


\usepackage{listings} % Para incluir código desde un archivo

\renewcommand\lstlistingname{Código Fuente}
\renewcommand\lstlistlistingname{Índice de Códigos Fuente}

% Definir colores
\definecolor{vscodepurple}{rgb}{0.5,0,0.5}
\definecolor{vscodeblue}{rgb}{0,0,0.8}
\definecolor{vscodegreen}{rgb}{0,0.5,0}
\definecolor{vscodegray}{rgb}{0.5,0.5,0.5}
\definecolor{vscodebackground}{rgb}{0.97,0.97,0.97}
\definecolor{vscodelightgray}{rgb}{0.9,0.9,0.9}

% Configuración para el estilo de C similar a VSCode
\lstdefinestyle{vscode_C}{
  backgroundcolor=\color{vscodebackground},
  commentstyle=\color{vscodegreen},
  keywordstyle=\color{vscodeblue},
  numberstyle=\tiny\color{vscodegray},
  stringstyle=\color{vscodepurple},
  basicstyle=\scriptsize\ttfamily,
  breakatwhitespace=false,
  breaklines=true,
  captionpos=b,
  keepspaces=true,
  numbers=left,
  numbersep=5pt,
  showspaces=false,
  showstringspaces=false,
  showtabs=false,
  tabsize=2,
  frame=tb,
  framerule=0pt,
  aboveskip=10pt,
  belowskip=10pt,
  xleftmargin=10pt,
  xrightmargin=10pt,
  framexleftmargin=10pt,
  framexrightmargin=10pt,
  framesep=0pt,
  rulecolor=\color{vscodelightgray},
  backgroundcolor=\color{vscodebackground},
}

%------------------------------------------------------------------------

% Comandos definidos
\newcommand{\bb}[1]{\mathbb{#1}}
\newcommand{\cc}[1]{\mathcal{#1}}

% I prefer the slanted \leq
\let\oldleq\leq % save them in case they're every wanted
\let\oldgeq\geq
\renewcommand{\leq}{\leqslant}
\renewcommand{\geq}{\geqslant}

% Si y solo si
\newcommand{\sii}{\iff}

% Letras griegas
\newcommand{\eps}{\epsilon}
\newcommand{\veps}{\varepsilon}
\newcommand{\lm}{\lambda}

\newcommand{\ol}{\overline}
\newcommand{\ul}{\underline}
\newcommand{\wt}{\widetilde}
\newcommand{\wh}{\widehat}

\let\oldvec\vec
\renewcommand{\vec}{\overrightarrow}

% Derivadas parciales
\newcommand{\del}[2]{\frac{\partial #1}{\partial #2}}
\newcommand{\Del}[3]{\frac{\partial^{#1} #2}{\partial #3^{#1}}}
\newcommand{\deld}[2]{\dfrac{\partial #1}{\partial #2}}
\newcommand{\Deld}[3]{\dfrac{\partial^{#1} #2}{\partial #3^{#1}}}


\newcommand{\AstIg}{\stackrel{(\ast)}{=}}
\newcommand{\Hop}{\stackrel{L'H\hat{o}pital}{=}}

\newcommand{\red}[1]{{\color{red}#1}} % Para integrales, destacar los cambios.

% Método de integración
\newcommand{\MetInt}[2]{
    \left[\begin{array}{c}
        #1 \\ #2
    \end{array}\right]
}

% Declarar aplicaciones
% 1. Nombre aplicación
% 2. Dominio
% 3. Codominio
% 4. Variable
% 5. Imagen de la variable
\newcommand{\Func}[5]{
    \begin{equation*}
        \begin{array}{rrll}
            #1:& #2 & \longrightarrow & #3\\
               & #4 & \longmapsto & #5
        \end{array}
    \end{equation*}
}

%------------------------------------------------------------------------

\DeclareMathOperator{\Var}{Var}
\DeclareMathOperator{\Cov}{Cov}
\usepgfplotslibrary{fillbetween}

\title{Ejercicios Probabilidad}
\date{\today}
\author{Arturo Olivares Martos}


\begin{document}
    \maketitle

    \begin{ejercicio}
        Sea $(X,Y)$ un vector aleatorio bidimensional con función de densidad de probabilidad
        \begin{equation*}
            f(x, y) = \begin{cases}
                k & 0<x<1,~0<y<1, \\
                0 & \text{en otro caso}.
            \end{cases}
        \end{equation*}
        \begin{enumerate}
            \item Calcular $k$ para que $f$ sea función de densidad de probabilidad de un vector aleatorio continuo $(X,Y)$.
            
            Para que $f$ sea función de densidad de probabilidad, necesitamos que:
            \begin{equation*}
                1=\int_{-\infty}^{+\infty} \int_{-\infty}^{+\infty} f(x, y) \, dx \, dy = k\int_{0}^{1} \int_{0}^{1} \, dy \, dx = k\int_{0}^{1} 1 \, dx = k\left[x\right]_0^1 = k.
            \end{equation*}
            \item Calcular la función de densidad de probabilidad conjunta del vector bidimensional $(Z,T)=(X+Y,X-Y)$.
            
            Definimos la transformación:
            \Func{g}{\bb{R}^2}{\bb{R}^2}{(X,Y)}{(Z,T)=(X+Y,X-Y)}
    
            Para obtener $g^{-1}$, buscamos obtener $X,Y$ en función de $Z,T$:
            \begin{equation*}
                \left\{\begin{aligned}
                    Z&=X+Y, \\
                    T&=X-Y.
                \end{aligned}\right\}\Longrightarrow
                \left\{\begin{aligned}
                    X&=\dfrac{Z+T}{2}, \\
                    Y&=\dfrac{Z-T}{2}.
                \end{aligned}\right.
            \end{equation*}
    
            Por tanto, tenemos que:
            \Func{g^{-1}}{\bb{R}^2}{\bb{R}^2}{(Z,T)}{(X,Y)=\left(\dfrac{Z+T}{2},\dfrac{Z-T}{2}\right)}
    
            Tenemos que todas las componentes de $g^{-1}$ son derivables:
            \begin{align*}
                \dfrac{\partial X}{\partial Z}(Z,T)&=\nicefrac{1}{2}, & \dfrac{\partial X}{\partial T}(Z,T)&=\nicefrac{1}{2},\\
                \dfrac{\partial Y}{\partial Z}(Z,T)&=\nicefrac{1}{2}, & \dfrac{\partial Y}{\partial T}(Z,T)&=\nicefrac{-1}{2}.
            \end{align*}
    
            Además, tenemos que:
            \begin{equation*}
                \det Jg^{-1}(z,t)=\begin{vmatrix}
                    \nicefrac{1}{2} & \nicefrac{1}{2} \\
                    \nicefrac{1}{2} & \nicefrac{-1}{2}
                \end{vmatrix}=\frac{1}{2^2}\begin{vmatrix}
                    1 & 1 \\
                    1 & -1
                \end{vmatrix}=\frac{1}{2^2}(-1-1)=-\frac{2}{2^2}=-\frac{1}{2}\neq 0 \qquad \forall (z,t)\in\bb{R}^2.
            \end{equation*}
    
            Por tanto, $(Z,T)=g(X,Y)$ es un vector aleatorio continuo. Veamos el valor de $g(X,Y)$ para $X,Y\in [0,1]$:
            \begin{align*}
                g(X,Y)&=\left\{(z,t)\in \bb{R}^2\mid 0<\dfrac{z+t}{2}<1,~0<\dfrac{z-t}{2}<1\right\}
                =\\&= \left\{(z,t)\in \bb{R}^2\mid 0<z+t<2,~0<z-t<2\right\}
            \end{align*}
            
            Veámoslo gráficamente:
            \begin{figure}[H]
                \centering
                \begin{tikzpicture}
                    \begin{axis}[
                        axis lines=middle,
                        xmin=-1.5, xmax=2.5,
                        ymin=-1.5, ymax=2.5,
                        xlabel={$Z$},
                        ylabel={$T$},
                        legend pos=outer north east, % Cambia la posición de la leyenda
                        axis equal,
                    ]
                        % Recta 0=z+t
                        \addplot[domain=-1:3, samples=2, color=red]{-x};
                        \addlegendentry{$0=z+t$}
                        % Recta 2=z+t
                        \addplot[domain=-1:3, samples=2, color=red, dashed]{2-x};
                        \addlegendentry{$2=z+t$}
                        % Recta 0=z-t
                        \addplot[domain=-1:3, samples=2, color=blue]{x};
                        \addlegendentry{$0=z-t$}
                        % Recta 2=z-t
                        \addplot[domain=-1:3, samples=2, color=blue, dashed]{x-2};
                        \addlegendentry{$2=z-t$}
    
                        % Relleno
                        \addplot[fill=gray, fill opacity=0.4] coordinates {(0,0) (1,1) (2,0) (1,-1) (0,0)};
                        \addlegendentry{$g(X,Y)$}
    
                    \end{axis}
                \end{tikzpicture}
            \end{figure}
            
            Buscamos ahora la función de densidad de probabilidad de $(Z,T)$:
            \begin{align*}
                f_{(Z,T)}(z, t)&=f_{(X,Y)}\left(\dfrac{z+t}{2},\dfrac{z-t}{2}\right)\cdot \left|\det Jg^{-1}(z,t)\right|
                =\\&= \begin{cases}
                    k\cdot \dfrac{1}{2} & (z,t)\in g(X,Y), \\
                    0 & \text{en otro caso}.
                \end{cases}
            \end{align*}
    
            Por tanto, la función de densidad de probabilidad de $(Z,T)$ es:
            \begin{equation*}
                f_{(Z,T)}(z, t) = \begin{cases}
                    \dfrac{1}{2} & 0<z+t<2,~0<z-t<2, \\
                    0 & \text{en otro caso}.
                \end{cases}
            \end{equation*}
    
            \item Determinar las funciones de densidad de probabilidad marginales del vector transformado $(Z,T)$.
            
            Para $z\in [0,2]$, tenemos que:
            \begin{align*}
                f_Z(z)&=\int_{-\infty}^{+\infty} f_{(Z,T)}(z, t) \, dt
            \end{align*}
    
            Los límites de integración los vemos claros en la gráfica anterior. Distinguimos en función del valor de $z$:
            \begin{itemize}
                \item Si $z\in [0,1]$, entonces:
                \begin{align*}
                    f_Z(z)&=\int_{-z}^{z} \dfrac{1}{2} \, dt = \dfrac{1}{2}\left[t\right]_{-z}^z = \dfrac{1}{2}(z-(-z)) = z.
                \end{align*}
    
                \item Si $z\in [1,2]$, entonces:
                \begin{align*}
                    f_Z(z)&=\int_{z-2}^{2-z} \dfrac{1}{2} \, dt = \dfrac{1}{2}\left[t\right]_{z-2}^{2-z} = \dfrac{1}{2}(2-z-(z-2)) = \dfrac{1}{2}(4-2z) = 2-z.
                \end{align*}
            \end{itemize}
    
            Por tanto, la función de densidad de probabilidad de $Z$ es:
            \begin{equation*}
                f_Z(z) = \begin{cases}
                    z & 0<z<1, \\
                    2-z & 1<z<2, \\
                    0 & \text{en otro caso}.
                \end{cases}
            \end{equation*}
    
            Para $t\in [-1,1]$, tenemos que:
            \begin{align*}
                f_T(t)&=\int_{-\infty}^{+\infty} f_{(Z,T)}(z, t) \, dz
            \end{align*}
    
            Los límites de integración los vemos claros en la gráfica anterior. Distinguimos en función del valor de $t$:
            \begin{itemize}
                \item Si $t\in [-1,0]$, entonces:
                \begin{align*}
                    f_T(t)&=\int_{-t}^{2+t} \dfrac{1}{2} \, dz = \dfrac{1}{2}\left[z\right]_{-t}^{2+t} = \dfrac{1}{2}(2+t-(-t)) = 1+t.
                \end{align*}
    
                \item Si $t\in [0,1]$, entonces:
                \begin{align*}
                    f_T(t)&=\int_{t}^{2-t} \dfrac{1}{2} \, dz = \dfrac{1}{2}\left[z\right]_{t}^{2-t} = \dfrac{1}{2}(2-t-t) = 1-t.
                \end{align*}
            \end{itemize}
    
            Por tanto, la función de densidad de probabilidad de $T$ es:
            \begin{equation*}
                f_T(t) = \begin{cases}
                    1+t & -1<t<0, \\
                    1-t & 0<t<1, \\
                    0 & \text{en otro caso}.
                \end{cases}
            \end{equation*}
            
    
            \item Determinar la función de distribución de probabilidad de $\nicefrac{X}{Y}$ y $XY$.
            % Hasta aquí hecho en clase. Preguntar a Joaquín
    
            Definimos la transformación:
            \Func{g}{\bb{R}^2}{\bb{R}^2}{(X,Y)}{(Z,T)=(\nicefrac{X}{Y},XY)}
    
            Para obtener $g^{-1}$, buscamos obtener $X,Y$ en función de $Z,T$:
            \begin{equation*}
                \left\{\begin{aligned}
                    Z&=\nicefrac{X}{Y}, \\
                    T&=XY.
                \end{aligned}\right\}\Longrightarrow
                \left\{\begin{aligned}
                    X&=ZY=\sqrt{ZT}, \\
                    Y&=\sqrt{\nicefrac{T}{Z}}.
                \end{aligned}\right.
            \end{equation*}
            Como $X,Y>0$, entonces $Z,T>0$, por lo que la inversa está bien definida.
            \Func{g^{-1}}{\bb{R}^+\times \bb{R}^+}{\bb{R}^2}{(Z,T)}{(X,Y)=(\sqrt{ZT},\sqrt{\nicefrac{T}{Z}})}
    
            Tenemos que todas las componentes de $g^{-1}$ son derivables:
            \begin{align*}
                \dfrac{\partial X}{\partial Z}(Z,T)&=\dfrac{T}{2\sqrt{ZT}}=\dfrac{\sqrt{T}}{2\sqrt{Z}}, & \dfrac{\partial X}{\partial T}(Z,T)&=\dfrac{\sqrt{Z}}{2\sqrt{T}},\\
                \dfrac{\partial Y}{\partial Z}(Z,T)&=\dfrac{-\frac{T}{Z^2}}{2\sqrt{\nicefrac{T}{Z}}}=-\dfrac{\sqrt{T}}{2Z\sqrt{Z}}, & \dfrac{\partial Y}{\partial T}(Z,T)&=\dfrac{1}{2Z\sqrt{\nicefrac{T}{Z}}}=\dfrac{1}{2\sqrt{TZ}}.
            \end{align*}
    
            Además, tenemos que:
            \begin{equation*}
                \det Jg^{-1}(z,t)=\begin{vmatrix}
                    \dfrac{\sqrt{T}}{2\sqrt{Z}} & \dfrac{\sqrt{Z}}{2\sqrt{T}} \\
                    -\dfrac{\sqrt{T}}{2Z\sqrt{Z}} & \dfrac{1}{2\sqrt{TZ}}
                \end{vmatrix}=\dfrac{1}{4Z}+\dfrac{1}{4Z}=\frac{1}{2Z}>0 \qquad \forall (z,t)\in\bb{R}^+\times \bb{R}^+.
            \end{equation*}
    
            Por tanto, $(Z,T)=g(X,Y)$ es un vector aleatorio continuo. Estudiamos ahora el conjunto $g(X,Y)$:
            \begin{align*}
                g(X,Y)&=\left\{(z,t)\in \bb{R}^+\times \bb{R}^+\mid 0<\sqrt{ZT}<1,~0<\sqrt{\nicefrac{T}{Z}}<1\right\}
                =\\&= \left\{(z,t)\in \bb{R}^+\times \bb{R}^+\mid 0<ZT<1,~0<T<Z\right\}
            \end{align*}
    
            Veamos el conjunto $g(X,Y)$ gráficamente:
            \begin{figure}[H]
                \centering
                \begin{tikzpicture}
                    \begin{axis}[
                        axis lines=middle,
                        xmin=-1.5, xmax=2.5,
                        ymin=-1.5, ymax=2.5,
                        xlabel={$Z$},
                        ylabel={$T$},
                        legend pos=outer north east, % Cambia la posición de la leyenda
                        axis equal,
                    ]
                        % Recta 0=z*t
                        \addplot[domain=-2:3, samples=2, color=red, dashed]{0};
                        \draw[red, dashed] (axis cs:0,-3) -- (axis cs:0,3);
                        \addlegendentry{$0=ZT$}
                        % Recta 1=z*t
                        \addplot[domain=-2:3, samples=100, color=red, name path=B]{1/x};
                        \addlegendentry{$1=ZT$}
                        % Recta 0=t
                        \addplot[name path=A, domain=-2:3, samples=2, color=blue, dashed]{0};
                        \addlegendentry{$0=T$}
                        % Recta z=t
                        \addplot[domain=-2:3, samples=2, color=blue, name path=C]{x};
                        \addlegendentry{$T=Z$}
    
                        % Marca del punto (1,1)
                        \addplot[color=black, mark=*, only marks, forget plot] coordinates {(1,1)};
                        \node[anchor=west] at (axis cs:1,1) {$(1,1)$};
    
                        % Relleno
                        \addplot [
                            thick,
                            color=orange,
                            fill=orange,
                            fill opacity=0.4
                        ]
                        fill between [
                            of=C and A,
                            soft clip={domain=0:1},
                        ];
                        \addplot [
                            thick,
                            color=orange,
                            fill=orange,
                            fill opacity=0.4
                        ]
                        fill between [
                            of=B and A,
                            soft clip={domain=1:4},
                        ];
                    \addlegendentry{$g(X,Y)$}
    
                    \end{axis}
                \end{tikzpicture}
            \end{figure}
    
            Buscamos ahora la función de densidad de probabilidad de $(Z,T)$:
            \begin{align*}
                f_{(Z,T)}(z, t)&=f_{(X,Y)}\left(\sqrt{ZT},\sqrt{\nicefrac{T}{Z}}\right)\cdot \left|\det Jg^{-1}(z,t)\right|
                =\\&= \begin{cases}
                    \dfrac{1}{2z} & (z,t)\in g(X,Y), \\
                    0 & \text{en otro caso}.
                \end{cases}
            \end{align*}
    
            Por tanto, la función de densidad de probabilidad de $(Z,T)$ es:
            \begin{equation*}
                f_{(Z,T)}(z, t) = \begin{cases}
                    \dfrac{1}{2z} & 0<ZT<1,~0<T<Z, \\
                    0 & \text{en otro caso}.
                \end{cases}
            \end{equation*}
    
            Para obtener la función de densidad de probabilidad de $Z=\nicefrac{X}{Y}$, tenemos que:
            \begin{align*}
                f_{Z}(z)&=\int_{-\infty}^{+\infty} f_{(Z,T)}(z, t) \, dt
            \end{align*}
    
            Los límites de integración los vemos claros en la gráfica anterior. Distinguimos en función del valor de $z$:
            \begin{itemize}
                \item Si $z\in [0,1]$, entonces:
                \begin{align*}
                    f_Z(z)&=\int_{0}^{z} \dfrac{1}{2z} \, dt = \dfrac{1}{2z}\left[t\right]_{0}^{z} = \dfrac{1}{2z}z = \dfrac{1}{2}.
                \end{align*}
    
                \item Si $z\in \left[1,+\infty\right[$, entonces:
                \begin{align*}
                    f_Z(z)&=\int_{0}^{\nicefrac{1}{z}} \dfrac{1}{2z} \, dt = \dfrac{1}{2z}\left[t\right]_{0}^{\nicefrac{1}{z}} = \dfrac{1}{2z^2}.
                \end{align*}
            \end{itemize}
    
            Por tanto, la función de densidad de probabilidad de $Z$ es:
            \begin{equation*}
                f_Z(z) = \begin{cases}
                    \dfrac{1}{2} & 0<z<1, \\ \\
                    \dfrac{1}{2z^2} & 1<z, \\
                    0 & \text{en otro caso}.
                \end{cases}
            \end{equation*}
    
            Para obtener la función de densidad de probabilidad de $T=XY$, tenemos que:
            \begin{align*}
                f_{T}(t)&=\int_{-\infty}^{+\infty} f_{(Z,T)}(z, w) \, dz
            \end{align*}
    
            Los límites de integración los vemos claros en la gráfica anterior. Para $t\in \left]0,1\right]$, tenemos que:
            \begin{align*}
                f_T(t)&=\int_{t}^{\nicefrac{1}{t}} \dfrac{1}{2z} \, dz = \dfrac{1}{2}\left[\ln(z)\right]_{t}^{\nicefrac{1}{t}} = \dfrac{1}{2}\left[\ln\left(\nicefrac{1}{t}\right)-\ln(t)\right]
                = \dfrac{1}{2}\left[\ln(1)-2\ln(t)\right] = -\ln(t)
            \end{align*}
    
            Por tanto, la función de densidad de probabilidad de $T$ es:
            \begin{equation*}
                f_T(t) = \begin{cases}
                    -\ln(t) & 0<t<1, \\
                    0 & \text{en otro caso}.
                \end{cases}
            \end{equation*}
    
            Una vez tenemos ambas marginales, es fácil obtener la función de distribución de cada una.
            Respecto de $Z$, distinguimos en función del valor de $z$:
            \begin{itemize}
                \item Si $z\in [0,1]$, entonces:
                \begin{align*}
                    F_Z(z)&=\int_{-\infty}^{z} f_Z(z) \, dz = \int_{0}^{z} \dfrac{1}{2} \, dz = \dfrac{1}{2}\left[z\right]_{0}^{z} = \dfrac{z}{2}
                \end{align*}
    
                \item Si $z\in \left[1,+\infty\right[$, entonces:
                \begin{align*}
                    F_Z(z)&=\int_{-\infty}^{z} f_Z(z) \, dz = \int_{0}^{1} \dfrac{1}{2} \, dz + \int_{1}^{z} \dfrac{1}{2z^2} \, dz
                    = \dfrac{1}{2} - \dfrac{1}{2}\left[\frac{1}{z}\right]_{1}^{z}
                    =\\&= \dfrac{1}{2} - \dfrac{1}{2}\left(\frac{1}{z}-1\right)
                    = 1-\dfrac{1}{2z}
                \end{align*}        
            \end{itemize}
    
            Por tanto, la función de distribución de probabilidad de $Z=\nicefrac{X}{Y}$ es:
            \begin{equation*}
                F_Z(z) = \begin{cases}
                    \dfrac{z}{2} & 0<z<1, \\ \\
                    1-\dfrac{1}{2z} & 1<z, \\
                    0 & \text{en otro caso}.
                \end{cases}
            \end{equation*}
    
            Respecto de $T$, para $t\in \left]0,1\right]$, tenemos que:
            \begin{align*}
                F_T(t)&=\int_{-\infty}^{t} f_T(t) \, dt = \int_{0}^{t} -\ln(t) \, dt = -\left[t\ln(t)-t\right]_{0}^{t} =\\&=
                -t\ln(t)+t +\lim_{t\to 0}t\ln(t) = -t\ln(t)+t
            \end{align*}
    
            Por tanto, la función de distribución de probabilidad de $T=XY$ es:
            \begin{equation*}
                F_T(t) = \begin{cases}
                    0 & t\leq 0, \\
                    -t\ln(t)+t & 0<t<1, \\
                    1 & 1\leq t, \\
                \end{cases}
            \end{equation*}
    
    
            \item Determinar la función de distribución de probabilidad de $\max(X,Y)$, y del $\min(X,Y)$.
            
            Dado $x\in \bb{R}$, tenemos que:
            \begin{equation*}
                P[\max(X,Y)\leq x]=P[X\leq x,Y\leq x] = P[(X,Y)\leq (x,x)]
            \end{equation*}
    
            Calculamos por tanto dicho valor sabiendo $f_{(X,Y)}$. Para $z\in [0,1]$, tenemos que:
            \begin{align*}
                P[(X,Y)\leq (z,z)]&=\int_{0}^{z}\int_{0}^{z} k \, dy \, dx = k\int_{0}^{z} z \, dx
                = kz\left[x\right]_{0}^{z} = kz^2 = z^2
            \end{align*}
    
            Por tanto, la función de distribución de probabilidad de $\max(X,Y)$ es:
            \begin{equation*}
                F_{\max(X,Y)}(z) = \begin{cases}
                    0 & z\leq 0, \\
                    z^2 & 0<z<1, \\
                    1 & 1\leq z.
                \end{cases}
            \end{equation*}
    
            Respecto del $\min(X,Y)$, dado $z\in \bb{R}$, tenemos que:
            \begin{equation*}
                P[\min(X,Y)\leq z]=1-P[\min(X,Y)>z]=1-P[(X,Y)>z]
            \end{equation*}
    
            Calculamos dicho valor sabiendo $f_{(X,Y)}$. Distinguimos en función de $z$:
            \begin{itemize}
                \item Si $z\leq 0$, entonces $P[(X,Y)>z]=1$.
    
                \item Si $0<z<1$, entonces:
                \begin{align*}
                    P[(X,Y)>z]&=\int_{z}^{1}\int_{z}^{1} k \, dy \, dz = k\int_{z}^{1} (1-z) \, dz
                    = k(1-z)\left[x\right]_{z}^{1} = (1-z)^2
                \end{align*}
    
                \item Si $z\geq 1$, entonces $P[(X,Y)>z]=0$.
            \end{itemize}
    
            Por tanto, la función de distribución de probabilidad de $\min(X,Y)$ es:
            \begin{equation*}
                F_{\min(X,Y)}(z) = 1-P[(X,Y)>z]=\begin{cases}
                    0 & z\leq 0, \\
                    1-(1-z)^2 & 0<z<1, \\
                    1 & 1\leq z.
                \end{cases}
            \end{equation*}
    
            \item Determinar la función de distribución de probabilidad conjunta del $\max(X,Y)$, y del $\min(X,Y)$.
            
            Dado $z,t\in \bb{R}$, distinguimos casos:
            \begin{itemize}
                \item Si $z\leq t$, como $\min (X,Y)\leq \max(X,Y)$, entonces:
                \begin{align*}
                    P[\max(X,Y)\leq z,\min(X,Y)\leq t]&=P[\max(X,Y)\leq z] = F_{\max(X,Y)}(z)
                \end{align*}
    
                Distinguimos en función de $z$:
                \begin{itemize}
                    \item Si $z\leq 0$, entonces $F_{\max(X,Y)}(z)=0$.
                    \item Si $0<z<1$, entonces $F_{\max(X,Y)}(z)=z^2$.
                    \item Si $z\geq 1$, entonces $F_{\max(X,Y)}(z)=1$.
                \end{itemize}
    
                \item Si $z>t$, entonces:
                \begin{align*}
                    P[\max(X,Y)\leq z,&\min(X,Y)\leq t]
                    \\&=P[\max(X,Y)\leq z]-P[\max(X,Y)\leq z,\min(X,Y)> t]
                    =\\&= P[\max(X,Y)\leq z]-P[t<X\leq z,t<Y\leq z]
                \end{align*}
    
                Distinguimos en función de $z$:
                \begin{itemize}
                    \item Si $z\leq 0$, entonces $P[\max(X,Y)\leq z]=0$. Además, $t<z\leq 0$. Por tanto:
                    \begin{equation*}
                        P[t<X\leq z,t<Y\leq z]=0
                    \end{equation*}
                    Por tanto, $P[\max(X,Y)\leq z,\min(X,Y)\leq t]=0$.
    
                    \item Si $0<z<1$, entonces $P[\max(X,Y)\leq z]=z^2$. Además, sabemos que $t<z<1$. Distinguimos en función de $t$:
                    \begin{itemize}
                        \item Si $t\leq 0$, entonces:
                        \begin{align*}
                            P[t<X\leq z,t<Y\leq z]&=\int_{0}^{z}\int_{0}^{z} k \, dy \, dx = kz^2 = z^2
                        \end{align*}
                        Por tanto, $P[\max(X,Y)\leq z,\min(X,Y)\leq t]=z^2-z^2=0$.
                        \item Si $0<t<z$, entonces:
                        \begin{align*}
                            P[t<X\leq z,t<Y\leq z]&=\int_{t}^{z}\int_{t}^{z} k \, dy \, dx = k(z-t)^2 = (z-t)^2
                        \end{align*}
                        Por tanto, tenemos que:
                        $$P[\max(X,Y)\leq z,\min(X,Y)\leq t]=z^2-(z-t)^2=z^2-z^2+2zt-t^2=2zt-t^2$$
                    \end{itemize}
                    \item Si $z\geq 1$, entonces $P[\max(X,Y)\leq z]=1$. Distinguimos en función de $t$:
                    \begin{itemize}
                        \item Si $t\leq 0$, entonces:
                        \begin{align*}
                            P[t<X\leq z,t<Y\leq z]&=\int_{0}^{1}\int_{0}^{1} k \, dy \, dx = k = 1
                        \end{align*}
                        Por tanto, $P[\max(X,Y)\leq z,\min(X,Y)\leq t]=1-1=0$.
                        \item Si $0<t<1$, entonces:
                        \begin{align*}
                            P[t<X\leq z,t<Y\leq z]&=\int_{t}^{1}\int_{t}^{1} k \, dy \, dx = k(1-t)^2 = (1-t)^2
                        \end{align*}
                        Por tanto, $$P[\max(X,Y)\leq z,\min(X,Y)\leq t]=1-(1-t)^2=1-1+2t-t^2=2t-t^2$$
                        \item Si $t\geq 1$, $t<z$, entonces:
                        \begin{align*}
                            P[t<X\leq z,t<Y\leq z]&=0
                        \end{align*}
                        Por tanto, $P[\max(X,Y)\leq z,\min(X,Y)\leq t]=1-0=1$.
                    \end{itemize}
                \end{itemize}
            \end{itemize}
            
                
            Por tanto, la función de distribución de probabilidad conjunta de $\max(X,Y)$ y $\min(X,Y)$ es:
            \begin{equation*}
                F_{\max(X,Y),\min(X,Y)}(z,t) = \begin{cases}
                    0 & z\leq 0 \lor t\leq 0\qquad (R_0), \\
                    z^2 & z\leq t \land 0<z<1 \qquad (R_1), \\
                    2zt-t^2 & 0<t<z<1 \qquad (R_2), \\
                    2t-t^2 & 0<t<1\leq z \qquad (R_3), \\
                    1 & 1\leq t\land 1\leq z \qquad (R_4).
                \end{cases}
            \end{equation*}
    
            Veamos gráficamente cada una de las regiones:
            \begin{figure}[H]
                \centering
                \begin{tikzpicture}
                    \begin{axis}[
                        axis lines=middle,
                        xmin=-0.5, xmax=2,
                        ymin=-0.5, ymax=2,
                        xlabel={$Z$},
                        ylabel={$T$},
                        legend pos=outer north east, % Cambia la posición de la leyenda
                        axis equal,
                    ]
                        % R0: (0,5) (-5,5) (-5,-5) (5,-5) (5,0) (0,0)
                        \addplot[fill=gray, fill opacity=0.3] coordinates {(0,5) (-5,5) (-5,-5) (5,-5) (5,0) (0,0)};
                        \node[anchor=south] at (-0.5,0.25) {$R_0$};
    
                        % R1: (0,0) (1,1) (1,5) (0,5)
                        \addplot[fill=green, fill opacity=0.3] coordinates {(0,0) (1,1) (1,5) (0,5)};
                        \node[anchor=south] at (0.5,1) {$R_1$};
    
                        % R2: (0,0) (1,0) (1,1)
                        \addplot[fill=blue, fill opacity=0.3] coordinates {(0,0) (1,0) (1,1)};
                        \node[anchor=west] at (axis cs:0.5,0.25) {$R_2$};
    
                        % R3: (1,0) (1,1) (5,1) (5,0)
                        \addplot[fill=red, fill opacity=0.3] coordinates {(1,0) (1,1) (5,1) (5,0)};
                        \node[anchor=west] at (axis cs:1.5,0.5) {$R_3$};
    
                        % R4: (1,1) (5,1) (5,5) (1,5)
                        \addplot[fill=orange, fill opacity=0.3] coordinates {(1,1) (5,1) (5,5) (1,5)};
                        \node[anchor=west] at (axis cs:1.5,1.5) {$R_4$};
                        
                    \end{axis}
                \end{tikzpicture}
            \end{figure}
    
    
    
            % // TODO: En clase no se ha hecho tanta distinción. Está bien?
        \end{enumerate}
    \end{ejercicio}



\end{document}