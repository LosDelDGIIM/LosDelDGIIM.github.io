\section{Teorema Central del Límite}

\begin{ejercicio}
    Se realizan 2500 lanzamientos independientes de una moneda correcta. Calcular la probabilidad aproximada de obtener $\nicefrac{1}{2}$ como frecuencia relativa de cara con un error máximo de $0.02$.
\end{ejercicio}

\begin{ejercicio}
    Calcular, aproximadamente, la probabilidad de que al lanzar 100 veces un dado, la media de los puntos obtenidos sea mayor que $3.7$.
\end{ejercicio}

\begin{ejercicio}
    El número de piezas correctas elaboradas en una fábrica cuadruplica el de piezas defectuosas, y la probabilidad de producir una pieza defectuosa se mantiene constante durante todo el proceso de fabricación. Se eligen al azar $200$ piezas, calcular aproximadamente la probabilidad de que el número de defectuosas oscile entre $40$ y $50$.
\end{ejercicio}

\begin{ejercicio}
    Se elige un punto aleatorio $(X_1, \ldots, X_{100})$ en el espacio $\mathbb{R}^{100}$. Suponiendo que las variables aleatorias son independientes e idénticamente distribuidas según una $\cc{U}([-1, 1])$, calcular, aproximadamente, la probabilidad de que el cuadrado de la distancia del punto al origen sea menor que $40$.
\end{ejercicio}

\begin{ejercicio}
    Cierta enfermedad afecta al $0.5\%$ de una población. Existe una prueba para la detección de la enfermedad, que da positiva en los individuos enfermos con probabilidad $0.99$ y da negativa en los individuos sanos con probabilidad $0.99$.
    \begin{enumerate}
        \item Calcular la probabilidad de que un individuo elegido al azar esté realmente enfermo si la prueba da resultado positivo.
        \item Calcular, aproximadamente, el número mínimo de personas con resultado positivo en la prueba que deben ser elegidas, de forma aleatoria e independiente, para asegurar una proporción de personas realmente enfermas en la muestra inferior a un $\nicefrac{1}{2}$, con probabilidad mayor o igual que $0.95$.
    \end{enumerate}
\end{ejercicio}

\begin{ejercicio}
    Una empresa necesita adquirir al menos $100$ vehículos. Para ello realiza una prueba a una población de coches compuesta de dos tipos distintos (un $40\%$ de tipo A y un $60\%$ de tipo B) y un coche es adquirido si supera la prueba. Un coche de tipo A supera la prueba con probabilidad $\nicefrac{1}{3}$, mientras que para uno de tipo B, dicha probabilidad es $\nicefrac{2}{3}$.
    \begin{enumerate}
        \item Calcular la probabilidad de que un coche elegido al azar supere la prueba.
        \item Calcula el número de coches que deben examinarse para cubrir las necesidades de la empresa con probabilidad mayor o igual que $0.901147$.
    \end{enumerate}
\end{ejercicio}

\begin{ejercicio}
    Para realizar una compra de un determinado material eléctrico del que se sabe que es defectuoso con probabilidad $0.25$, se le somete a una determinada prueba que da los resultados A, B y C, con probabilidades $0.8$, $0.15$ y $0.05$, si el material es válido, y probabilidades $0.2$, $0.3$ y $0.5$, si el material es defectuoso. Si cada lote se somete a seis pruebas de forma independiente, y se acepta para su compra si no aparece nunca el resultado C ni más de dos veces el resultado B, calcular la probabilidad de que un lote elegido al azar sea aceptado para su compra. Calcular el número de lotes que hay que probar para comprar al menos $20$ con probabilidad mayor o igual que $0.9452$.
\end{ejercicio}

\begin{ejercicio}
    La longitud de vida (en horas) de una determinada pieza de cierta máquina es una variable aleatoria que se distribuye de acuerdo a la función de densidad
    \[
        f (x) = \exp(1 - x), \quad x > 1.
    \]mathcalismas características, y se supone que las longitudes de vida de distintas piezas son independientes. Calcular aproximadamente el número de piezas de recambio necesario para asegurar el funcionamiento de la máquina al menos durante $1000$ horas, con probabilidad mayor que $0.95$.
\end{ejercicio}