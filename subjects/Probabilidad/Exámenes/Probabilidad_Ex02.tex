\documentclass[12pt]{article}

% Idioma y codificación
\usepackage[spanish, es-tabla]{babel}       %es-tabla para que se titule "Tabla"
\usepackage[utf8]{inputenc}

% Márgenes
\usepackage[a4paper,top=3cm,bottom=2.5cm,left=3cm,right=3cm]{geometry}

% Comentarios de bloque
\usepackage{verbatim}

% Paquetes de links
\usepackage[hidelinks]{hyperref}    % Permite enlaces
\usepackage{url}                    % redirecciona a la web

% Más opciones para enumeraciones
\usepackage{enumitem}

% Personalizar la portada
\usepackage{titling}

% Paquetes de tablas
\usepackage{multirow}


%------------------------------------------------------------------------

%Paquetes de figuras
\usepackage{caption}
\usepackage{subcaption} % Figuras al lado de otras
\usepackage{float}      % Poner figuras en el sitio indicado H.


% Paquetes de imágenes
\usepackage{graphicx}       % Paquete para añadir imágenes
\usepackage{transparent}    % Para manejar la opacidad de las figuras

% Paquete para usar colores
\usepackage[dvipsnames]{xcolor}
\usepackage{pagecolor}      % Para cambiar el color de la página

% Habilita tamaños de fuente mayores
\usepackage{fix-cm}

% Para los gráficos
\usepackage{tikz}

% Para poder situar los nodos en los grafos
\usetikzlibrary{positioning}


%------------------------------------------------------------------------

% Paquetes de matemáticas
\usepackage{mathtools, amsfonts, amssymb, mathrsfs}
\usepackage[makeroom]{cancel}     % Simplificar tachando
\usepackage{polynom}    % Divisiones y Ruffini
\usepackage{units} % Para poner fracciones diagonales con \nicefrac

\usepackage{pgfplots}   %Representar funciones
\pgfplotsset{compat=1.18}  % Versión 1.18

\usepackage{tikz-cd}    % Para usar diagramas de composiciones
\usetikzlibrary{calc}   % Para usar cálculo de coordenadas en tikz

%Definición de teoremas, etc.
\usepackage{amsthm}
%\swapnumbers   % Intercambia la posición del texto y de la numeración

\theoremstyle{plain}

\makeatletter
\@ifclassloaded{article}{
  \newtheorem{teo}{Teorema}[section]
}{
  \newtheorem{teo}{Teorema}[chapter]  % Se resetea en cada chapter
}
\makeatother

\newtheorem{coro}{Corolario}[teo]           % Se resetea en cada teorema
\newtheorem{prop}[teo]{Proposición}         % Usa el mismo contador que teorema
\newtheorem{lema}[teo]{Lema}                % Usa el mismo contador que teorema

\theoremstyle{remark}
\newtheorem*{observacion}{Observación}

\theoremstyle{definition}

\makeatletter
\@ifclassloaded{article}{
  \newtheorem{definicion}{Definición} [section]     % Se resetea en cada chapter
}{
  \newtheorem{definicion}{Definición} [chapter]     % Se resetea en cada chapter
}
\makeatother

\newtheorem*{notacion}{Notación}
\newtheorem*{ejemplo}{Ejemplo}
\newtheorem*{ejercicio*}{Ejercicio}             % No numerado
\newtheorem{ejercicio}{Ejercicio} [section]     % Se resetea en cada section


% Modificar el formato de la numeración del teorema "ejercicio"
\renewcommand{\theejercicio}{%
  \ifnum\value{section}=0 % Si no se ha iniciado ninguna sección
    \arabic{ejercicio}% Solo mostrar el número de ejercicio
  \else
    \thesection.\arabic{ejercicio}% Mostrar número de sección y número de ejercicio
  \fi
}


% \renewcommand\qedsymbol{$\blacksquare$}         % Cambiar símbolo QED
%------------------------------------------------------------------------

% Paquetes para encabezados
\usepackage{fancyhdr}
\pagestyle{fancy}
\fancyhf{}

\newcommand{\helv}{ % Modificación tamaño de letra
\fontfamily{}\fontsize{12}{12}\selectfont}
\setlength{\headheight}{15pt} % Amplía el tamaño del índice


%\usepackage{lastpage}   % Referenciar última pag   \pageref{LastPage}
\fancyfoot[C]{\thepage}

%------------------------------------------------------------------------

% Conseguir que no ponga "Capítulo 1". Sino solo "1."
\makeatletter
\@ifclassloaded{book}{
  \renewcommand{\chaptermark}[1]{\markboth{\thechapter.\ #1}{}} % En el encabezado
    
  \renewcommand{\@makechapterhead}[1]{%
  \vspace*{50\p@}%
  {\parindent \z@ \raggedright \normalfont
    \ifnum \c@secnumdepth >\m@ne
      \huge\bfseries \thechapter.\hspace{1em}\ignorespaces
    \fi
    \interlinepenalty\@M
    \Huge \bfseries #1\par\nobreak
    \vskip 40\p@
  }}
}
\makeatother

%------------------------------------------------------------------------
% Paquetes de cógido
\usepackage{minted}
\renewcommand\listingscaption{Código fuente}

\usepackage{fancyvrb}
% Personaliza el tamaño de los números de línea
\renewcommand{\theFancyVerbLine}{\small\arabic{FancyVerbLine}}

% Estilo para C++
\newminted{cpp}{
    frame=lines,
    framesep=2mm,
    baselinestretch=1.2,
    linenos,
    escapeinside=||
}

% para minted
\definecolor{LightGray}{rgb}{0.95,0.95,0.92}
\setminted{
    linenos=true,
    stepnumber=5,
    numberfirstline=true,
    autogobble,
    breaklines=true,
    breakautoindent=true,
    breaksymbolleft=,
    breaksymbolright=,
    breaksymbolindentleft=0pt,
    breaksymbolindentright=0pt,
    breaksymbolsepleft=0pt,
    breaksymbolsepright=0pt,
    fontsize=\footnotesize,
    bgcolor=LightGray,
    numbersep=10pt
}


\usepackage{listings} % Para incluir código desde un archivo

\renewcommand\lstlistingname{Código Fuente}
\renewcommand\lstlistlistingname{Índice de Códigos Fuente}

% Definir colores
\definecolor{vscodepurple}{rgb}{0.5,0,0.5}
\definecolor{vscodeblue}{rgb}{0,0,0.8}
\definecolor{vscodegreen}{rgb}{0,0.5,0}
\definecolor{vscodegray}{rgb}{0.5,0.5,0.5}
\definecolor{vscodebackground}{rgb}{0.97,0.97,0.97}
\definecolor{vscodelightgray}{rgb}{0.9,0.9,0.9}

% Configuración para el estilo de C similar a VSCode
\lstdefinestyle{vscode_C}{
  backgroundcolor=\color{vscodebackground},
  commentstyle=\color{vscodegreen},
  keywordstyle=\color{vscodeblue},
  numberstyle=\tiny\color{vscodegray},
  stringstyle=\color{vscodepurple},
  basicstyle=\scriptsize\ttfamily,
  breakatwhitespace=false,
  breaklines=true,
  captionpos=b,
  keepspaces=true,
  numbers=left,
  numbersep=5pt,
  showspaces=false,
  showstringspaces=false,
  showtabs=false,
  tabsize=2,
  frame=tb,
  framerule=0pt,
  aboveskip=10pt,
  belowskip=10pt,
  xleftmargin=10pt,
  xrightmargin=10pt,
  framexleftmargin=10pt,
  framexrightmargin=10pt,
  framesep=0pt,
  rulecolor=\color{vscodelightgray},
  backgroundcolor=\color{vscodebackground},
}

%------------------------------------------------------------------------

% Comandos definidos
\newcommand{\bb}[1]{\mathbb{#1}}
\newcommand{\cc}[1]{\mathcal{#1}}

% I prefer the slanted \leq
\let\oldleq\leq % save them in case they're every wanted
\let\oldgeq\geq
\renewcommand{\leq}{\leqslant}
\renewcommand{\geq}{\geqslant}

% Si y solo si
\newcommand{\sii}{\iff}

% Letras griegas
\newcommand{\eps}{\epsilon}
\newcommand{\veps}{\varepsilon}
\newcommand{\lm}{\lambda}

\newcommand{\ol}{\overline}
\newcommand{\ul}{\underline}
\newcommand{\wt}{\widetilde}
\newcommand{\wh}{\widehat}

\let\oldvec\vec
\renewcommand{\vec}{\overrightarrow}

% Derivadas parciales
\newcommand{\del}[2]{\frac{\partial #1}{\partial #2}}
\newcommand{\Del}[3]{\frac{\partial^{#1} #2}{\partial #3^{#1}}}
\newcommand{\deld}[2]{\dfrac{\partial #1}{\partial #2}}
\newcommand{\Deld}[3]{\dfrac{\partial^{#1} #2}{\partial #3^{#1}}}


\newcommand{\AstIg}{\stackrel{(\ast)}{=}}
\newcommand{\Hop}{\stackrel{L'H\hat{o}pital}{=}}

\newcommand{\red}[1]{{\color{red}#1}} % Para integrales, destacar los cambios.

% Método de integración
\newcommand{\MetInt}[2]{
    \left[\begin{array}{c}
        #1 \\ #2
    \end{array}\right]
}

% Declarar aplicaciones
% 1. Nombre aplicación
% 2. Dominio
% 3. Codominio
% 4. Variable
% 5. Imagen de la variable
\newcommand{\Func}[5]{
    \begin{equation*}
        \begin{array}{rrll}
            #1:& #2 & \longrightarrow & #3\\
               & #4 & \longmapsto & #5
        \end{array}
    \end{equation*}
}

%------------------------------------------------------------------------

\usepgfplotslibrary{fillbetween}

\begin{document}

    % 1. Foto de fondo
    % 2. Título
    % 3. Encabezado Izquierdo
    % 4. Color de fondo
    % 5. Coord x del titulo
    % 6. Coord y del titulo
    % 7. Fecha

    
    % 1. Foto de fondo
% 2. Título
% 3. Encabezado Izquierdo
% 4. Color de fondo
% 5. Coord x del titulo
% 6. Coord y del titulo
% 7. Fecha

\newcommand{\portada}[7]{

    \portadaBase{#1}{#2}{#3}{#4}{#5}{#6}{#7}
    \portadaBook{#1}{#2}{#3}{#4}{#5}{#6}{#7}
}

\newcommand{\portadaExamen}[7]{

    \portadaBase{#1}{#2}{#3}{#4}{#5}{#6}{#7}
    \portadaArticle{#1}{#2}{#3}{#4}{#5}{#6}{#7}
}




\newcommand{\portadaBase}[7]{

    % Tiene la portada principal y la licencia Creative Commons
    
    % 1. Foto de fondo
    % 2. Título
    % 3. Encabezado Izquierdo
    % 4. Color de fondo
    % 5. Coord x del titulo
    % 6. Coord y del titulo
    % 7. Fecha
    
    
    \thispagestyle{empty}               % Sin encabezado ni pie de página
    \newgeometry{margin=0cm}        % Márgenes nulos para la primera página
    
    
    % Encabezado
    \fancyhead[L]{\helv #3}
    \fancyhead[R]{\helv \nouppercase{\leftmark}}
    
    
    \pagecolor{#4}        % Color de fondo para la portada
    
    \begin{figure}[p]
        \centering
        \transparent{0.3}           % Opacidad del 30% para la imagen
        
        \includegraphics[width=\paperwidth, keepaspectratio]{assets/#1}
    
        \begin{tikzpicture}[remember picture, overlay]
            \node[anchor=north west, text=white, opacity=1, font=\fontsize{60}{90}\selectfont\bfseries\sffamily, align=left] at (#5, #6) {#2};
            
            \node[anchor=south east, text=white, opacity=1, font=\fontsize{12}{18}\selectfont\sffamily, align=right] at (9.7, 3) {\textbf{\href{https://losdeldgiim.github.io/}{Los Del DGIIM}}};
            
            \node[anchor=south east, text=white, opacity=1, font=\fontsize{12}{15}\selectfont\sffamily, align=right] at (9.7, 1.8) {Doble Grado en Ingeniería Informática y Matemáticas\\Universidad de Granada};
        \end{tikzpicture}
    \end{figure}
    
    
    \restoregeometry        % Restaurar márgenes normales para las páginas subsiguientes
    \pagecolor{white}       % Restaurar el color de página
    
    
    \newpage
    \thispagestyle{empty}               % Sin encabezado ni pie de página
    \begin{tikzpicture}[remember picture, overlay]
        \node[anchor=south west, inner sep=3cm] at (current page.south west) {
            \begin{minipage}{0.5\paperwidth}
                \href{https://creativecommons.org/licenses/by-nc-nd/4.0/}{
                    \includegraphics[height=2cm]{assets/Licencia.png}
                }\vspace{1cm}\\
                Esta obra está bajo una
                \href{https://creativecommons.org/licenses/by-nc-nd/4.0/}{
                    Licencia Creative Commons Atribución-NoComercial-SinDerivadas 4.0 Internacional (CC BY-NC-ND 4.0).
                }\\
    
                Eres libre de compartir y redistribuir el contenido de esta obra en cualquier medio o formato, siempre y cuando des el crédito adecuado a los autores originales y no persigas fines comerciales. 
            \end{minipage}
        };
    \end{tikzpicture}
    
    
    
    % 1. Foto de fondo
    % 2. Título
    % 3. Encabezado Izquierdo
    % 4. Color de fondo
    % 5. Coord x del titulo
    % 6. Coord y del titulo
    % 7. Fecha


}


\newcommand{\portadaBook}[7]{

    % 1. Foto de fondo
    % 2. Título
    % 3. Encabezado Izquierdo
    % 4. Color de fondo
    % 5. Coord x del titulo
    % 6. Coord y del titulo
    % 7. Fecha

    % Personaliza el formato del título
    \pretitle{\begin{center}\bfseries\fontsize{42}{56}\selectfont}
    \posttitle{\par\end{center}\vspace{2em}}
    
    % Personaliza el formato del autor
    \preauthor{\begin{center}\Large}
    \postauthor{\par\end{center}\vfill}
    
    % Personaliza el formato de la fecha
    \predate{\begin{center}\huge}
    \postdate{\par\end{center}\vspace{2em}}
    
    \title{#2}
    \author{\href{https://losdeldgiim.github.io/}{Los Del DGIIM}}
    \date{Granada, #7}
    \maketitle
    
    \tableofcontents
}




\newcommand{\portadaArticle}[7]{

    % 1. Foto de fondo
    % 2. Título
    % 3. Encabezado Izquierdo
    % 4. Color de fondo
    % 5. Coord x del titulo
    % 6. Coord y del titulo
    % 7. Fecha

    % Personaliza el formato del título
    \pretitle{\begin{center}\bfseries\fontsize{42}{56}\selectfont}
    \posttitle{\par\end{center}\vspace{2em}}
    
    % Personaliza el formato del autor
    \preauthor{\begin{center}\Large}
    \postauthor{\par\end{center}\vspace{3em}}
    
    % Personaliza el formato de la fecha
    \predate{\begin{center}\huge}
    \postdate{\par\end{center}\vspace{5em}}
    
    \title{#2}
    \author{\href{https://losdeldgiim.github.io/}{Los Del DGIIM}}
    \date{Granada, #7}
    \thispagestyle{empty}               % Sin encabezado ni pie de página
    \maketitle
    \vfill
}
    \portadaExamen{ffccA4.jpg}{Probabilidad\\Examen II}{Probabilidad. Examen II}{MidnightBlue}{-8}{28}{2024-2025}{Arturo Olivares Martos}

    \begin{description}
        \item[Asignatura] Probabilidad.
        \item[Curso Académico] 2024-25.
        \item[Grado] Doble Grado en Ingeniería Informática y Matemáticas.
        \item[Grupo] Único.
        \item[Profesor] Francisco Javier Esquivel Sánchez.
        \item[Descripción] Parcial de los Temas 1 y 2.1.
        \item[Fecha] 22 de octubre de 2024.
        \item[Duración] 50 minutos.
    \end{description}
    \newpage

    \begin{ejercicio}[$1$ punto]
        Calcular razonadamente la función generatriz de momentos de una variable aleatoria $\cc{N}(0, 1)$.
        Recordemos que la funsión de densidad de $\cc{N}(\mu, \sigma^2)$ es:
        \begin{equation*}
            f(x) = \dfrac{1}{\sqrt{2\pi}\sigma}e^{-\dfrac{(x-\mu)^2}{2\sigma^2}} \qquad x\in \bb{R}
        \end{equation*}

        Este ejercicio está demostrado en la Teoría. La función generatriz de momentos de una variable aleatoria $\cc{N}(0, 1)$ es:
        \begin{equation*}
            M_X(t) = e^{\nicefrac{t^2}{2}}
        \end{equation*}
    \end{ejercicio}

    \begin{ejercicio}
        Dado el vector bidimensional $(X,Y)$ distribuido uniformemente en el recinto limitado
        $$R = \{(x, y) \in \mathbb{R}^2 \mid -2 \leq x \leq y \leq -x\}$$

        \begin{enumerate}
            \item ($1$ punto) Obtener su función de densidad conjunta.
            
            Representamos en primer lugar dicho conjunto:
            \begin{figure}[H]
                \centering
                \begin{tikzpicture}
                    \begin{axis}
                        [
                            axis lines = center,
                            xlabel = $X$,
                            ylabel = $Y$,
                            xmin = -3, xmax = 3,
                            ymin = -3, ymax = 3,
                        ]
                        \addplot[fill=red, opacity=0.3] coordinates {(-2, -2) (0, 0) (-2, 2)} -- cycle;
                        \node at (-1.5, 0.5) {$R$};

                        % R0: (-2,3) (-2,-2) (3,-2) (3,-3) (-3,-3) (-3,3) (-2,3)
                        \addplot[fill=blue, opacity=0.2] coordinates {
                            (-2, 3) (-2, -2) (3, -2) (3, -3) (-3, -3) (-3, 3) (-2, 3)
                        } -- cycle;
                        \node at (-2.5, -2.5) {$R_0$};

                        % R1: (-2,-2) (0,0) (3,0) (3,-2)
                        \addplot[fill=green, opacity=0.2] coordinates {
                            (-2, -2) (0, 0) (3, 0) (3, -2)
                        } -- cycle;
                        \node at (1.5, -1) {$R_1$};

                        % R2: (-2,2) (0,0) (0,2)
                        \addplot[fill=yellow, opacity=0.2] coordinates {
                            (-2, 2) (0, 0) (0, 2)
                        } -- cycle;
                        \node at (-0.8, 1.3) {$R_2$};

                        % R3: (0,2) (0,0) (3,0) (3,2)
                        \addplot[fill=orange, opacity=0.2] coordinates {
                            (0, 2) (0, 0) (3, 0) (3, 2)
                        } -- cycle;
                        \node at (1.5, 1) {$R_3$};

                        % R4: (-2,2) (0,2) (0,3) (-2,3)
                        \addplot[fill=purple, opacity=0.2] coordinates {
                            (-2, 2) (0, 2) (0, 3) (-2, 3)
                        } -- cycle;
                        \node at (-1, 2.5) {$R_4$};

                        % R5: (0,2) (3,2) (3,3) (0,3)
                        \addplot[fill=gray, opacity=0.2] coordinates {
                            (0, 2) (3, 2) (3, 3) (0, 3)
                        } -- cycle;
                        \node at (1.5, 2.5) {$R_5$};
                    \end{axis}
                \end{tikzpicture}
            \end{figure}

            Tenemos que, para $x\in [-2, 0]$ y $y\in [x, -x]$, la función de densidad conjunta es:
            \begin{equation*}
                f_{(X,Y)}(x, y) = k,\qquad k\in \bb{R}^+
            \end{equation*}

            Para que sea una función de densidad, hemos de tener que:
            \begin{equation*}
                1 = \int_{\bb{R}^2} f = \int_R f = \int_R k
            \end{equation*}

            Tenemos dos opciones:
            \begin{description}
                \item[Integración Normal:]
                \begin{align*}
                    1 = \int_R k &= \int_{-2}^0 \int_{x}^{-x} k \, dy \, dx = \int_{-2}^0 k(-2x) \, dx =\\&= -2k\left[\dfrac{x^2}{2}\right]_{-2}^0 = -2k\left[0-2\right] = 4k \Longrightarrow k = \dfrac{1}{4}
                \end{align*}

                \item[Razonando según la Forma de $R$:]
                \begin{align*}
                    1 = \int_R k = k\lm(R) = k\cdot \dfrac{4\cdot 2}{2} = 4k \Longrightarrow k = \dfrac{1}{4}
                \end{align*}
            \end{description}
            En cualquier caso, para el valor de $k=\nicefrac{1}{4}$, la función de densidad conjunta integrable, no negativa e integra $1$.
            \item ($6.5$ puntos) Obtener su función de distribución conjunta.
            
            Distinguimos casos:
            \begin{itemize}
                \item \ul{Si $x < -2$ o $y < -2$} (Zona $R_0$):
                \begin{equation*}
                    F_{(X,Y)}(x, y) \int_{-\infty}^x \int_{-\infty}^y f_{(X,Y)}(u, v) \, dv \, du = 0
                \end{equation*}

                \item \ul{Si $x\in [-2, 0]$ y $y\in [x, -x]$} (Zona $R$):
                \begin{align*}
                    F_{(X,Y)}(x, y) &= \int_{-\infty}^x \int_{-\infty}^y f_{(X,Y)}(u, v) \, dv \, du = \int_{-2}^x \int_{u}^{y} \dfrac{1}{4} \, dv \, du
                    =\\&= \int_{-2}^x \dfrac{1}{4}(y-u) \, du = \dfrac{1}{4}\left[yu - \dfrac{u^2}{2}\right]_{-2}^x = \dfrac{1}{4}\left[yx - \dfrac{x^2}{2} + 2y+2\right]
                \end{align*}

                \item \ul{Si $y<0$ y $x > y$} (Zona $R_1$):
                \begin{align*}
                    F_{(X,Y)}(x, y) &= \int_{-\infty}^x \int_{-\infty}^y f_{(X,Y)}(u, v) \, dv \, du = \int_{-2}^y \int_{u}^{y} \dfrac{1}{4} \, dv \, du
                    =\\&= \int_{-2}^y \dfrac{1}{4}(y-u) \, du = \dfrac{1}{4}\left[yu - \dfrac{u^2}{2}\right]_{-2}^y = \dfrac{1}{4}\left[y^2 - \dfrac{y^2}{2} + 2y+2\right]
                    =\\&= \dfrac{1}{4}\left[\dfrac{y^2}{2} + 2y + 2\right]
                \end{align*}

                \item \ul{Si $x\in [-2, 0]$ y $y\in [-x, 2]$} (Zona $R_2$):
                \begin{align*}
                    F_{(X,Y)}(x, y) &= \int_{-\infty}^x \int_{-\infty}^y f_{(X,Y)}(u, v) \, dv \, du = \int_{-2}^{-y} \int_{u}^{y} \dfrac{1}{4} \, dv \, du+ \int_{-y}^{x} \int_{u}^{-u} \dfrac{1}{4} \, dv \, du
                    =\\&= \int_{-2}^{-y} \dfrac{1}{4}(y-u) \, du + \int_{-y}^{x} \dfrac{1}{4}(-2u) \, du = \dfrac{1}{4}\left[yu - \dfrac{u^2}{2}\right]_{-2}^{-y} + \dfrac{1}{4}\left[-u^2\right]_{-y}^{x}
                    =\\&= \dfrac{1}{4}\left[\cancel{y(-y)} - \dfrac{(-y)^2}{2} + 2y+2\right] + \dfrac{1}{4}\left[-x^2 + \cancel{y^2}\right]
                    =\\&= \dfrac{1}{4}\left[2y + 2 - \dfrac{y^2}{2} - x^2\right]
                \end{align*}

                \item \ul{Si $y\in [0, 2]$ y $x \geq 0$} (Zona $R_3$):
                \begin{align*}
                    F_{(X,Y)}(x, y) &= \int_{-\infty}^x \int_{-\infty}^y f_{(X,Y)}(u, v) \, dv \, du = \int_{-2}^{-y} \int_{u}^{y} \dfrac{1}{4} \, dv \, du+ \int_{-y}^{0} \int_{u}^{-u} \dfrac{1}{4} \, dv \, du
                    =\\&= \int_{-2}^{-y} \dfrac{1}{4}(y-u) \, du + \int_{-y}^{0} \dfrac{1}{4}(-2u) \, du = \dfrac{1}{4}\left[yu - \dfrac{u^2}{2}\right]_{-2}^{-y} + \dfrac{1}{4}\left[-u^2\right]_{-y}^{0}
                    =\\&= \dfrac{1}{4}\left[\cancel{y(-y)} - \dfrac{(-y)^2}{2} + 2y+2\right] + \dfrac{1}{4}\left[0 - \cancel{y^2}\right]
                    =\\&= \dfrac{1}{4}\left[2y + 2 - \dfrac{y^2}{2}\right]
                \end{align*}

                \item \ul{Si $x\in [-2, 0]$ y $y\geq 2$} (Zona $R_4$):
                \begin{align*}
                    F_{(X,Y)}(x, y) &= \int_{-\infty}^x \int_{-\infty}^y f_{(X,Y)}(u, v) \, dv \, du = \int_{-2}^{x} \int_{u}^{-u} \dfrac{1}{4} \, dv \, du
                    =\\&= \int_{-2}^{x} \dfrac{1}{4}(-2u) \, du = \dfrac{1}{4}\left[-u^2\right]_{-2}^{x} = \dfrac{1}{4}\left[-x^2 + 4\right]
                \end{align*}

                \item \ul{Si $x\geq 0, y\geq 2$} (Zona $R_5$):
                \begin{equation*}
                    F_{(X,Y)}(x, y) = 1
                \end{equation*}
            \end{itemize}

            Por tanto, la función de distribución conjunta es:
            \begin{equation*}
                F_{(X,Y)}(x, y) = \begin{cases}
                    0 & \text{si } x < -2 \text{ o } y < -2\\
                    \dfrac{1}{4}\left[yx - \dfrac{x^2}{2} + 2y+2\right] & \text{si } x\in [-2, 0] \text{ y } y\in [x, -x]\\
                    \dfrac{1}{4}\left[\dfrac{y^2}{2} + 2y + 2\right] & \text{si } y<0 \text{ y } x > y\\
                    \dfrac{1}{4}\left[2y + 2 - \dfrac{y^2}{2} - x^2\right] & \text{si } x\in [-2, 0] \text{ y } y\in [-x, 2]\\
                    \dfrac{1}{4}\left[2y + 2 - \dfrac{y^2}{2}\right] & \text{si } y\in [0, 2] \text{ y } x \geq 0\\
                    \dfrac{1}{4}\left[-x^2 + 4\right] & \text{si } x\in [-2, 0] \text{ y } y\geq 2\\
                    1 & \text{si } x\geq 0, y\geq 2
                \end{cases}
            \end{equation*}

            \item($1.5$ puntos) Obtener la probabilidad de que $X + Y + 1 \geq 0$.
            
            Veamos qué conjunto representa $X + Y + 1 \geq 0$:
            \begin{figure}[H]
                \centering
                \begin{tikzpicture}
                    \begin{axis}
                        [
                            axis lines = center,
                            xlabel = $X$,
                            ylabel = $Y$,
                            xmin = -3, xmax = 3,
                            ymin = -3, ymax = 3,
                        ]
                        \addplot[name path=A, fill=red, opacity=0.3] coordinates {(-2, -2) (0, 0) (-2, 2)} -- cycle;

                        % Recta y = -x-1
                        \addplot[name path=B, domain=-3:2.5, samples=2, dashed] {-x-1};

                        % Add a node to label the line
                        \node at (axis cs:1,-2) [anchor=west] {$y=-x-1$};

                        % R0: (-2,2) (0,0) (-0.5,-0.5) (-2, 1.5)
                        \addplot[fill=blue, opacity=0.2] coordinates {
                            (-2, 2) (0, 0) (-0.5, -0.5) (-2, 1)
                        } -- cycle;
                    \end{axis}
                \end{tikzpicture}
            \end{figure}

            Tenemos por tanto que:
            \begin{align*}
                P[X + Y + 1 \geq 0] &= \int_{-2}^{\nicefrac{-1}{2}} \int_{-x-1}^{-x} \dfrac{1}{4} \, dy \, dx + \int_{\nicefrac{-1}{2}}^{0} \int_{x}^{-x} \dfrac{1}{4} \, dy \, dx
                =\\&= \int_{-2}^{\nicefrac{-1}{2}} \dfrac{1}{4}(-x+1+x) \, dx + \int_{\nicefrac{-1}{2}}^{0} \dfrac{1}{4}(-2x) \, dx
                =\\&= \int_{-2}^{\nicefrac{-1}{2}} \dfrac{1}{4} \, dx + \int_{\nicefrac{-1}{2}}^{0} -\dfrac{1}{2}x \, dx
                =\\&= \dfrac{1}{4}\left[x\right]_{-2}^{\nicefrac{-1}{2}} - \dfrac{1}{2}\left[\dfrac{x^2}{2}\right]_{\nicefrac{-1}{2}}^{0}
                = \dfrac{1}{4}\left[\dfrac{-1}{2}+2\right] - \dfrac{1}{2}\left[0 - \dfrac{1}{8}\right] = \dfrac{7}{16}
            \end{align*}
        \end{enumerate}
    \end{ejercicio}
    \begin{observacion}
        Notamos que este ejercicio fue repetido de exámenes de otros años.
    \end{observacion}
\end{document}