\documentclass[12pt]{article}

% Idioma y codificación
\usepackage[spanish, es-tabla]{babel}       %es-tabla para que se titule "Tabla"
\usepackage[utf8]{inputenc}

% Márgenes
\usepackage[a4paper,top=3cm,bottom=2.5cm,left=3cm,right=3cm]{geometry}

% Comentarios de bloque
\usepackage{verbatim}

% Paquetes de links
\usepackage[hidelinks]{hyperref}    % Permite enlaces
\usepackage{url}                    % redirecciona a la web

% Más opciones para enumeraciones
\usepackage{enumitem}

% Personalizar la portada
\usepackage{titling}

% Paquetes de tablas
\usepackage{multirow}


%------------------------------------------------------------------------

%Paquetes de figuras
\usepackage{caption}
\usepackage{subcaption} % Figuras al lado de otras
\usepackage{float}      % Poner figuras en el sitio indicado H.


% Paquetes de imágenes
\usepackage{graphicx}       % Paquete para añadir imágenes
\usepackage{transparent}    % Para manejar la opacidad de las figuras

% Paquete para usar colores
\usepackage[dvipsnames]{xcolor}
\usepackage{pagecolor}      % Para cambiar el color de la página

% Habilita tamaños de fuente mayores
\usepackage{fix-cm}

% Para los gráficos
\usepackage{tikz}

% Para poder situar los nodos en los grafos
\usetikzlibrary{positioning}


%------------------------------------------------------------------------

% Paquetes de matemáticas
\usepackage{mathtools, amsfonts, amssymb, mathrsfs}
\usepackage[makeroom]{cancel}     % Simplificar tachando
\usepackage{polynom}    % Divisiones y Ruffini
\usepackage{units} % Para poner fracciones diagonales con \nicefrac

\usepackage{pgfplots}   %Representar funciones
\pgfplotsset{compat=1.18}  % Versión 1.18

\usepackage{tikz-cd}    % Para usar diagramas de composiciones
\usetikzlibrary{calc}   % Para usar cálculo de coordenadas en tikz

%Definición de teoremas, etc.
\usepackage{amsthm}
%\swapnumbers   % Intercambia la posición del texto y de la numeración

\theoremstyle{plain}

\makeatletter
\@ifclassloaded{article}{
  \newtheorem{teo}{Teorema}[section]
}{
  \newtheorem{teo}{Teorema}[chapter]  % Se resetea en cada chapter
}
\makeatother

\newtheorem{coro}{Corolario}[teo]           % Se resetea en cada teorema
\newtheorem{prop}[teo]{Proposición}         % Usa el mismo contador que teorema
\newtheorem{lema}[teo]{Lema}                % Usa el mismo contador que teorema

\theoremstyle{remark}
\newtheorem*{observacion}{Observación}

\theoremstyle{definition}

\makeatletter
\@ifclassloaded{article}{
  \newtheorem{definicion}{Definición} [section]     % Se resetea en cada chapter
}{
  \newtheorem{definicion}{Definición} [chapter]     % Se resetea en cada chapter
}
\makeatother

\newtheorem*{notacion}{Notación}
\newtheorem*{ejemplo}{Ejemplo}
\newtheorem*{ejercicio*}{Ejercicio}             % No numerado
\newtheorem{ejercicio}{Ejercicio} [section]     % Se resetea en cada section


% Modificar el formato de la numeración del teorema "ejercicio"
\renewcommand{\theejercicio}{%
  \ifnum\value{section}=0 % Si no se ha iniciado ninguna sección
    \arabic{ejercicio}% Solo mostrar el número de ejercicio
  \else
    \thesection.\arabic{ejercicio}% Mostrar número de sección y número de ejercicio
  \fi
}


% \renewcommand\qedsymbol{$\blacksquare$}         % Cambiar símbolo QED
%------------------------------------------------------------------------

% Paquetes para encabezados
\usepackage{fancyhdr}
\pagestyle{fancy}
\fancyhf{}

\newcommand{\helv}{ % Modificación tamaño de letra
\fontfamily{}\fontsize{12}{12}\selectfont}
\setlength{\headheight}{15pt} % Amplía el tamaño del índice


%\usepackage{lastpage}   % Referenciar última pag   \pageref{LastPage}
\fancyfoot[C]{\thepage}

%------------------------------------------------------------------------

% Conseguir que no ponga "Capítulo 1". Sino solo "1."
\makeatletter
\@ifclassloaded{book}{
  \renewcommand{\chaptermark}[1]{\markboth{\thechapter.\ #1}{}} % En el encabezado
    
  \renewcommand{\@makechapterhead}[1]{%
  \vspace*{50\p@}%
  {\parindent \z@ \raggedright \normalfont
    \ifnum \c@secnumdepth >\m@ne
      \huge\bfseries \thechapter.\hspace{1em}\ignorespaces
    \fi
    \interlinepenalty\@M
    \Huge \bfseries #1\par\nobreak
    \vskip 40\p@
  }}
}
\makeatother

%------------------------------------------------------------------------
% Paquetes de cógido
\usepackage{minted}
\renewcommand\listingscaption{Código fuente}

\usepackage{fancyvrb}
% Personaliza el tamaño de los números de línea
\renewcommand{\theFancyVerbLine}{\small\arabic{FancyVerbLine}}

% Estilo para C++
\newminted{cpp}{
    frame=lines,
    framesep=2mm,
    baselinestretch=1.2,
    linenos,
    escapeinside=||
}

% para minted
\definecolor{LightGray}{rgb}{0.95,0.95,0.92}
\setminted{
    linenos=true,
    stepnumber=5,
    numberfirstline=true,
    autogobble,
    breaklines=true,
    breakautoindent=true,
    breaksymbolleft=,
    breaksymbolright=,
    breaksymbolindentleft=0pt,
    breaksymbolindentright=0pt,
    breaksymbolsepleft=0pt,
    breaksymbolsepright=0pt,
    fontsize=\footnotesize,
    bgcolor=LightGray,
    numbersep=10pt
}


\usepackage{listings} % Para incluir código desde un archivo

\renewcommand\lstlistingname{Código Fuente}
\renewcommand\lstlistlistingname{Índice de Códigos Fuente}

% Definir colores
\definecolor{vscodepurple}{rgb}{0.5,0,0.5}
\definecolor{vscodeblue}{rgb}{0,0,0.8}
\definecolor{vscodegreen}{rgb}{0,0.5,0}
\definecolor{vscodegray}{rgb}{0.5,0.5,0.5}
\definecolor{vscodebackground}{rgb}{0.97,0.97,0.97}
\definecolor{vscodelightgray}{rgb}{0.9,0.9,0.9}

% Configuración para el estilo de C similar a VSCode
\lstdefinestyle{vscode_C}{
  backgroundcolor=\color{vscodebackground},
  commentstyle=\color{vscodegreen},
  keywordstyle=\color{vscodeblue},
  numberstyle=\tiny\color{vscodegray},
  stringstyle=\color{vscodepurple},
  basicstyle=\scriptsize\ttfamily,
  breakatwhitespace=false,
  breaklines=true,
  captionpos=b,
  keepspaces=true,
  numbers=left,
  numbersep=5pt,
  showspaces=false,
  showstringspaces=false,
  showtabs=false,
  tabsize=2,
  frame=tb,
  framerule=0pt,
  aboveskip=10pt,
  belowskip=10pt,
  xleftmargin=10pt,
  xrightmargin=10pt,
  framexleftmargin=10pt,
  framexrightmargin=10pt,
  framesep=0pt,
  rulecolor=\color{vscodelightgray},
  backgroundcolor=\color{vscodebackground},
}

%------------------------------------------------------------------------

% Comandos definidos
\newcommand{\bb}[1]{\mathbb{#1}}
\newcommand{\cc}[1]{\mathcal{#1}}

% I prefer the slanted \leq
\let\oldleq\leq % save them in case they're every wanted
\let\oldgeq\geq
\renewcommand{\leq}{\leqslant}
\renewcommand{\geq}{\geqslant}

% Si y solo si
\newcommand{\sii}{\iff}

% Letras griegas
\newcommand{\eps}{\epsilon}
\newcommand{\veps}{\varepsilon}
\newcommand{\lm}{\lambda}

\newcommand{\ol}{\overline}
\newcommand{\ul}{\underline}
\newcommand{\wt}{\widetilde}
\newcommand{\wh}{\widehat}

\let\oldvec\vec
\renewcommand{\vec}{\overrightarrow}

% Derivadas parciales
\newcommand{\del}[2]{\frac{\partial #1}{\partial #2}}
\newcommand{\Del}[3]{\frac{\partial^{#1} #2}{\partial #3^{#1}}}
\newcommand{\deld}[2]{\dfrac{\partial #1}{\partial #2}}
\newcommand{\Deld}[3]{\dfrac{\partial^{#1} #2}{\partial #3^{#1}}}


\newcommand{\AstIg}{\stackrel{(\ast)}{=}}
\newcommand{\Hop}{\stackrel{L'H\hat{o}pital}{=}}

\newcommand{\red}[1]{{\color{red}#1}} % Para integrales, destacar los cambios.

% Método de integración
\newcommand{\MetInt}[2]{
    \left[\begin{array}{c}
        #1 \\ #2
    \end{array}\right]
}

% Declarar aplicaciones
% 1. Nombre aplicación
% 2. Dominio
% 3. Codominio
% 4. Variable
% 5. Imagen de la variable
\newcommand{\Func}[5]{
    \begin{equation*}
        \begin{array}{rrll}
            #1:& #2 & \longrightarrow & #3\\
               & #4 & \longmapsto & #5
        \end{array}
    \end{equation*}
}

%------------------------------------------------------------------------

\usepgfplotslibrary{fillbetween}
\DeclareMathOperator{\Var}{Var}
\DeclareMathOperator{\Cov}{Cov}
\begin{document}

    % 1. Foto de fondo
    % 2. Título
    % 3. Encabezado Izquierdo
    % 4. Color de fondo
    % 5. Coord x del titulo
    % 6. Coord y del titulo
    % 7. Fecha

    
    % 1. Foto de fondo
% 2. Título
% 3. Encabezado Izquierdo
% 4. Color de fondo
% 5. Coord x del titulo
% 6. Coord y del titulo
% 7. Fecha

\newcommand{\portada}[7]{

    \portadaBase{#1}{#2}{#3}{#4}{#5}{#6}{#7}
    \portadaBook{#1}{#2}{#3}{#4}{#5}{#6}{#7}
}

\newcommand{\portadaExamen}[7]{

    \portadaBase{#1}{#2}{#3}{#4}{#5}{#6}{#7}
    \portadaArticle{#1}{#2}{#3}{#4}{#5}{#6}{#7}
}




\newcommand{\portadaBase}[7]{

    % Tiene la portada principal y la licencia Creative Commons
    
    % 1. Foto de fondo
    % 2. Título
    % 3. Encabezado Izquierdo
    % 4. Color de fondo
    % 5. Coord x del titulo
    % 6. Coord y del titulo
    % 7. Fecha
    
    
    \thispagestyle{empty}               % Sin encabezado ni pie de página
    \newgeometry{margin=0cm}        % Márgenes nulos para la primera página
    
    
    % Encabezado
    \fancyhead[L]{\helv #3}
    \fancyhead[R]{\helv \nouppercase{\leftmark}}
    
    
    \pagecolor{#4}        % Color de fondo para la portada
    
    \begin{figure}[p]
        \centering
        \transparent{0.3}           % Opacidad del 30% para la imagen
        
        \includegraphics[width=\paperwidth, keepaspectratio]{assets/#1}
    
        \begin{tikzpicture}[remember picture, overlay]
            \node[anchor=north west, text=white, opacity=1, font=\fontsize{60}{90}\selectfont\bfseries\sffamily, align=left] at (#5, #6) {#2};
            
            \node[anchor=south east, text=white, opacity=1, font=\fontsize{12}{18}\selectfont\sffamily, align=right] at (9.7, 3) {\textbf{\href{https://losdeldgiim.github.io/}{Los Del DGIIM}}};
            
            \node[anchor=south east, text=white, opacity=1, font=\fontsize{12}{15}\selectfont\sffamily, align=right] at (9.7, 1.8) {Doble Grado en Ingeniería Informática y Matemáticas\\Universidad de Granada};
        \end{tikzpicture}
    \end{figure}
    
    
    \restoregeometry        % Restaurar márgenes normales para las páginas subsiguientes
    \pagecolor{white}       % Restaurar el color de página
    
    
    \newpage
    \thispagestyle{empty}               % Sin encabezado ni pie de página
    \begin{tikzpicture}[remember picture, overlay]
        \node[anchor=south west, inner sep=3cm] at (current page.south west) {
            \begin{minipage}{0.5\paperwidth}
                \href{https://creativecommons.org/licenses/by-nc-nd/4.0/}{
                    \includegraphics[height=2cm]{assets/Licencia.png}
                }\vspace{1cm}\\
                Esta obra está bajo una
                \href{https://creativecommons.org/licenses/by-nc-nd/4.0/}{
                    Licencia Creative Commons Atribución-NoComercial-SinDerivadas 4.0 Internacional (CC BY-NC-ND 4.0).
                }\\
    
                Eres libre de compartir y redistribuir el contenido de esta obra en cualquier medio o formato, siempre y cuando des el crédito adecuado a los autores originales y no persigas fines comerciales. 
            \end{minipage}
        };
    \end{tikzpicture}
    
    
    
    % 1. Foto de fondo
    % 2. Título
    % 3. Encabezado Izquierdo
    % 4. Color de fondo
    % 5. Coord x del titulo
    % 6. Coord y del titulo
    % 7. Fecha


}


\newcommand{\portadaBook}[7]{

    % 1. Foto de fondo
    % 2. Título
    % 3. Encabezado Izquierdo
    % 4. Color de fondo
    % 5. Coord x del titulo
    % 6. Coord y del titulo
    % 7. Fecha

    % Personaliza el formato del título
    \pretitle{\begin{center}\bfseries\fontsize{42}{56}\selectfont}
    \posttitle{\par\end{center}\vspace{2em}}
    
    % Personaliza el formato del autor
    \preauthor{\begin{center}\Large}
    \postauthor{\par\end{center}\vfill}
    
    % Personaliza el formato de la fecha
    \predate{\begin{center}\huge}
    \postdate{\par\end{center}\vspace{2em}}
    
    \title{#2}
    \author{\href{https://losdeldgiim.github.io/}{Los Del DGIIM}}
    \date{Granada, #7}
    \maketitle
    
    \tableofcontents
}




\newcommand{\portadaArticle}[7]{

    % 1. Foto de fondo
    % 2. Título
    % 3. Encabezado Izquierdo
    % 4. Color de fondo
    % 5. Coord x del titulo
    % 6. Coord y del titulo
    % 7. Fecha

    % Personaliza el formato del título
    \pretitle{\begin{center}\bfseries\fontsize{42}{56}\selectfont}
    \posttitle{\par\end{center}\vspace{2em}}
    
    % Personaliza el formato del autor
    \preauthor{\begin{center}\Large}
    \postauthor{\par\end{center}\vspace{3em}}
    
    % Personaliza el formato de la fecha
    \predate{\begin{center}\huge}
    \postdate{\par\end{center}\vspace{5em}}
    
    \title{#2}
    \author{\href{https://losdeldgiim.github.io/}{Los Del DGIIM}}
    \date{Granada, #7}
    \thispagestyle{empty}               % Sin encabezado ni pie de página
    \maketitle
    \vfill
}
    \portadaExamen{ffccA4.jpg}{Probabilidad\\Examen V}{Probabilidad. Examen V}{MidnightBlue}{-8}{28}{2024-2025}{Arturo Olivares Martos \\ José Juan Urrutia Milán}

    \begin{description}
        \item[Asignatura] Probabilidad.
        \item[Curso Académico] 2021-22.
        \item[Grado] Doble Grado en Ingeniería Informática y Matemáticas.
        \item[Grupo] Único.
        %\item[Profesor] José María Espinar García.
        \item[Descripción] Examen Ordinario 
        \item[Fecha] 21 de enero de 2022.
        % \item[Duración] 60 minutos.
    
    \end{description}
    \newpage

    \subsection*{PARTE 1 (2.5 puntos)}
    \begin{ejercicio}[0.25 puntos]
        Sean $X_1$, $X_2$, $X_3$ variables aleatorias independientes e idénticamente distribuidas según una ley Binomial, $B(3,\nicefrac{1}{2})$. Justificar que: $$P[X_1 + X_2 + X_3 = 8] = \frac{9}{2^9} $$
            
        Tenemos que:
        \begin{equation*}
            X_1, X_2,X_3 \sim B(3, \nicefrac{1}{2}) 
        \end{equation*}

        Por la reproductividad de la distribución binomial, como son independientes, tenemos que:
        \begin{equation*}
            X_1 + X_2+X_3 \sim B(9, \nicefrac{1}{2})
        \end{equation*}

        Por tanto, usando la función masa de probabilidad de la distribución binomial, tenemos que:
        \begin{equation*}
            P[X_1 + X_2 + X_3 = 8] = \binom{9}{8} \left(\frac{1}{2}\right)^8 \left(\frac{1}{2}\right)^1 = \frac{9}{2^9}
        \end{equation*}
    \end{ejercicio}

    \begin{ejercicio}[0.25 puntos]
        Sean $X_1$ y $X_2$ variables aleatorias independientes e idénticamente distribuidas según una ley de Poisson, $\cc{P}(3)$. Justificar que:
        $$P[X_1 + X_2 > 0] = \frac{e^6 - 1}{e^6}$$

        Tenemos que:
            \begin{equation*}
                X_1, X_2 \sim \cc{P}(3)
            \end{equation*}

            Por la reproductividad de la distribución de Poisson, por ser independientes tenemos que:
            \begin{equation*}
                X_1 + X_2 \sim \cc{P}(6)
            \end{equation*}

            Por tanto, tenemos que:
            \begin{equation*}
                P[X_1 + X_2> 0] = 1 - P[X_1 + X_2= 0] = 1 - \frac{e^{-6}6^0}{0!} = 1 - e^{-6} = \frac{e^6 - 1}{e^6}
            \end{equation*}
    \end{ejercicio}

    \begin{ejercicio}
        Para predecir los valores de una variable aleatoria $X$ a partir de los de otra variable aleatoria $Y$ se considera un modelo lineal:
        \begin{enumerate}
            \item \textbf{(0.50 puntos)} Obtener de forma razonada los coeficientes del modelo lineal considerado.
            
            Se busca aproximar $X$ como $\wh{X}=aY+b$. Para ello, se minimiza el error cuadrático medio:
            \begin{align*}
                \text{E.C.M.}(X\mid Y) &= E[(X-\wh{X})^2]
                = E\left[(X-aY-b)^2\right]
                =\\&= E\left[X^2-2aXY-2bX+a^2Y^2+2abY+b^2\right]
                =\\&= E[X^2]-2aE[XY]-2bE[X]+a^2E[Y^2]+2abE[Y]+b^2
            \end{align*}

            Para ello, se busca minizar la siguiente función:
            \begin{equation*}
                L(a,b) = E.C.M.(X\mid Y) = E[X^2]-2aE[XY]-2bE[X]+a^2E[Y^2]+2abE[Y]+b^2
            \end{equation*}

            Se tiene demostrado en Teoría que llegamos a la siguiente expresión:
            \begin{equation*}
                \begin{cases}
                    a = \dfrac{\Cov[X,Y]}{\Var[Y]} \\
                    b = E[X]-aE[Y]
                \end{cases}
            \end{equation*}

            \item \textbf{(0.75 puntos)} Si $x-y=1$ y $2y-3x=-1$ son las dos rectas de regresión para el vector $(X,Y)$, se pide: identificar la recta de regresión del apartado anterior; obtener una medida de la bondad del ajuste y calcular la esperanza del vector $(X,Y)$.
            
            Suponemos que las rectas de regresión de $X$ sobre $Y$ y de $Y$ sobre $X$ son $x-y=1$ y $2y-3x=-1$, respectivamente. Por tanto, tenemos que:
            \begin{align*}
                x &= y+1 = \dfrac{\Cov[X,Y]}{\Var[Y]}\cdot y + E[X]-\dfrac{\Cov[X,Y]}{\Var[Y]}\cdot E[Y] \\
                y &= \dfrac{3}{2}x-\dfrac{1}{2} = \dfrac{\Cov[X,Y]}{\Var[X]}\cdot x + E[Y]-\dfrac{\Cov[X,Y]}{\Var[X]}\cdot E[X]
            \end{align*}

            Identificando términos, obtenemos que:
            \begin{equation*}
                \dfrac{\Cov[X,Y]}{\Var[X]}\cdot \dfrac{\Cov[X,Y]}{\Var[Y]} = \rho_{X,Y}^2 = 1\cdot \dfrac{3}{2} = \dfrac{3}{2}>1
            \end{equation*}

            Por tanto, llegamos a una contradicción, por lo que la suposición es incorrecta. La recta de regresión de $Y$ sobre $X$ es $y=x-1$ y la de $X$ sobre $Y$ es $x=\nicefrac{2}{3}y+\nicefrac{1}{3}$.

            La proporción de varianza de cada variable que queda explicada por el modelo de regresión lineal es el coeficiente de determinación, que en este caso es:
            \begin{equation*}
                \rho_{X,Y}^2 = \dfrac{2}{3}\cdot 1=\dfrac{2}{3}\approx 0.667\%
            \end{equation*}

            Por último, por identificación de términos, tenemos el siguiente sistema:
            \begin{equation*}
                \left\{\begin{array}{rcl}
                    E[X]-E[Y]&=&1 \\
                    E[Y]-\dfrac{3}{2}E[X]&=&-\dfrac{1}{2}
                \end{array}\right\}
                \Longrightarrow
                \begin{cases}
                    E[X]=-1 \\
                    E[Y]=-2
                \end{cases}
            \end{equation*}

            Por tanto, tenemos que:
            \begin{equation*}
                E[(X,Y)] = \begin{pmatrix}
                    -1 & -2
                \end{pmatrix}
            \end{equation*}
        \end{enumerate}
    \end{ejercicio}

    \begin{ejercicio}[0.75 puntos]
        Las componentes de un vector aleatorio continuo son variables aleatorias continuas. Sin embargo, en general, un conjunto de variables aleatorias continuas no da lugar a un vector aleatorio continuo. Justificar que este recíproco sí es cierto si consideramos un conjunto $X_1$, $X_2$, \ldots, $X_n$ de variables aleatorias continuas independientes.

        Definimos la siguiente función continua en $\bb{R}^n$:
        \Func{f_{(X_1,X_2,\ldots,X_n)}}{\bb{R}}{\bb{R}}{(x_1,x_2,\ldots,x_n)}{
            \prod\limits_{i=1}^n f_{X_i}(x_i)
        }
        y probaremos que es la función de densidad de $(X_1,X_2,\ldots,X_n)$. En primer lugar, está bien definida por estarlo cada una de las funciones de densidad de las variables aleatorias. Veamos que:
        \begin{align*}
            F_{(X_1,X_2,\ldots,X_n)}(x_1,x_2,\ldots,x_n) &= \int_{-\infty}^{x_1}\cdots\int_{-\infty}^{x_n} \prod\limits_{i=1}^n f_{X_i}(t_i) \,dt_1\cdots dt_n
        \end{align*}

        Para ello, como son independientes, tenemos que:
        \begin{equation*}
            P[X_1\leq x_1, X_2\leq x_2, \ldots, X_n\leq x_n] = \prod\limits_{i=1}^n P[X_i\leq x_i]
        \end{equation*}

        Por ser $f_{X_i}$ la función de densidad de $X_i$, tenemos que:
        \begin{equation*}
            F_{(X_1,X_2,\ldots,X_n)}(x_1,x_2,\ldots,x_n) = \prod\limits_{i=1}^n \int_{-\infty}^{x_i} f_{X_i}(t_i) \,dt_i
        \end{equation*}

        Por tanto, tenemos que la función que hemos definido efectivamente es la función de densidad de $(X_1,X_2,\ldots,X_n)$.\\

        Aunque no sería necesario, veamos que cumple las condiciones de toda función de densidad. En primer lugar, es no negativa, ya que cada término es mayor o igual que $0$. Veamos ahora que integra $1$:
        \begin{align*}
            \int_{\bb{R}^n} f_{(X_1,X_2,\ldots,X_n)}(x_1,x_2,\ldots,x_n) \,dx_1\cdots dx_n &= \int_{\bb{R}^n} \prod\limits_{i=1}^n f_{X_i}(x_i) \,dx_1\cdots dx_n
            =\\&= \prod\limits_{i=1}^n \int_{\bb{R}} f_{X_i}(x_i) \,dx_i
            = \prod\limits_{i=1}^n 1
            = 1
        \end{align*}
    \end{ejercicio}

    \subsection*{PARTE 2 (7.5 puntos)}
    \setcounter{ejercicio}{0}
    \begin{ejercicio}[5 puntos]
        Dado el vector aleatorio continuo $(X,Y)$ distribuido uniformemente en el recinto
        \begin{equation*}
            C = \{(x,y)\in \mathbb{R}^2 \mid x^2+y^2 < 1 \ \land\ x,y < 0\}
        \end{equation*}

    \begin{observacion}
        A tener en cuenta:
        \begin{itemize}
            \item En el \textbf{apartado 1.2} se obtiene \textbf{hasta 1 punto} si las integrales se dejan indicadas y \textbf{hasta 1.5 puntos} si se obtienen sus primitivas de forma explícita.
            \item Si necesitara obtener la primitiva de la función $f(x) = \sqrt{1-x^2}$, realizar el cambio de variable unidimensional $x=\sen(t)$.
            \item $\arcsen(0)=0$, $\arcsen(-1)=-\frac{\pi}{2}$, $\arcsen\left(-\frac{1}{\sqrt{2}}\right) = -\frac{\pi}{4}$.
            \item $\cos^2(x) = \frac{1+\cos(2x)}{2}$, $\sen^2(x) = \frac{1-\cos(2x)}{2}$.
        \end{itemize}
    \end{observacion}
        \begin{enumerate}
            \item \textbf{(0.25 puntos)} Obtener la función de densidad conjunta.
            \item \textbf{(1.50 puntos)} Obtener la función de distribuición de probabilidad conjunta.
            \item \textbf{(0.75 puntos)} Obtener las funciones de densidad condicionadas.
            \item \textbf{(0.50 puntos)} Obtener la probabilidad de que $X-Y>0$.
            \item \textbf{(1.50 puntos)} Obtener la mejor aproximaión mínimo cuadrática a la variable aleatoria $Y$ conocidos los valores de la variable $X$ y el error cuadrático medio de esta aproximación.
            \item \textbf{(0.50 puntos)} Obtener una media de la bondad del ajuste del apartado anterior.
        \end{enumerate}
    \end{ejercicio}

    \begin{ejercicio}[2.5 puntos]
        Dado el vector aleatorio $(X,Y)$ con función generatriz de momentos
        \begin{equation*}
            M_{(X,Y)}(t_1,t_2) = exp\left(\dfrac{2t_1 + 4t_1^2 + 9t_2^2 + 6t_1t_2}{2}\right)
        \end{equation*}
        \begin{enumerate}
            \item \textbf{(0.75 puntos)} Obtener la razón de correlación y el coeficiente de correlación lineal de las variables $(X,Y)$.
            \item \textbf{(0.75 puntos)} Indicar las distribuciones de las variables aleatorias $Y/X = 1$ y $X/Y=0$.
            \item \textbf{(1 punto)} Obtener la distribución de probabilidad del vector aleatorio $(2X, Y-X)$. Justificar que las variabes aleatorias $2X$ y $Y-X$ tienen cierta asociación lineal en sentido negativo.
        \end{enumerate}
    \end{ejercicio}


\end{document}
