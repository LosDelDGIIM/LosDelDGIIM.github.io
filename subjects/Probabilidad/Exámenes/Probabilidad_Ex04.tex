\documentclass[12pt]{article}

% Idioma y codificación
\usepackage[spanish, es-tabla]{babel}       %es-tabla para que se titule "Tabla"
\usepackage[utf8]{inputenc}

% Márgenes
\usepackage[a4paper,top=3cm,bottom=2.5cm,left=3cm,right=3cm]{geometry}

% Comentarios de bloque
\usepackage{verbatim}

% Paquetes de links
\usepackage[hidelinks]{hyperref}    % Permite enlaces
\usepackage{url}                    % redirecciona a la web

% Más opciones para enumeraciones
\usepackage{enumitem}

% Personalizar la portada
\usepackage{titling}

% Paquetes de tablas
\usepackage{multirow}


%------------------------------------------------------------------------

%Paquetes de figuras
\usepackage{caption}
\usepackage{subcaption} % Figuras al lado de otras
\usepackage{float}      % Poner figuras en el sitio indicado H.


% Paquetes de imágenes
\usepackage{graphicx}       % Paquete para añadir imágenes
\usepackage{transparent}    % Para manejar la opacidad de las figuras

% Paquete para usar colores
\usepackage[dvipsnames]{xcolor}
\usepackage{pagecolor}      % Para cambiar el color de la página

% Habilita tamaños de fuente mayores
\usepackage{fix-cm}

% Para los gráficos
\usepackage{tikz}

% Para poder situar los nodos en los grafos
\usetikzlibrary{positioning}


%------------------------------------------------------------------------

% Paquetes de matemáticas
\usepackage{mathtools, amsfonts, amssymb, mathrsfs}
\usepackage[makeroom]{cancel}     % Simplificar tachando
\usepackage{polynom}    % Divisiones y Ruffini
\usepackage{units} % Para poner fracciones diagonales con \nicefrac

\usepackage{pgfplots}   %Representar funciones
\pgfplotsset{compat=1.18}  % Versión 1.18

\usepackage{tikz-cd}    % Para usar diagramas de composiciones
\usetikzlibrary{calc}   % Para usar cálculo de coordenadas en tikz

%Definición de teoremas, etc.
\usepackage{amsthm}
%\swapnumbers   % Intercambia la posición del texto y de la numeración

\theoremstyle{plain}

\makeatletter
\@ifclassloaded{article}{
  \newtheorem{teo}{Teorema}[section]
}{
  \newtheorem{teo}{Teorema}[chapter]  % Se resetea en cada chapter
}
\makeatother

\newtheorem{coro}{Corolario}[teo]           % Se resetea en cada teorema
\newtheorem{prop}[teo]{Proposición}         % Usa el mismo contador que teorema
\newtheorem{lema}[teo]{Lema}                % Usa el mismo contador que teorema

\theoremstyle{remark}
\newtheorem*{observacion}{Observación}

\theoremstyle{definition}

\makeatletter
\@ifclassloaded{article}{
  \newtheorem{definicion}{Definición} [section]     % Se resetea en cada chapter
}{
  \newtheorem{definicion}{Definición} [chapter]     % Se resetea en cada chapter
}
\makeatother

\newtheorem*{notacion}{Notación}
\newtheorem*{ejemplo}{Ejemplo}
\newtheorem*{ejercicio*}{Ejercicio}             % No numerado
\newtheorem{ejercicio}{Ejercicio} [section]     % Se resetea en cada section


% Modificar el formato de la numeración del teorema "ejercicio"
\renewcommand{\theejercicio}{%
  \ifnum\value{section}=0 % Si no se ha iniciado ninguna sección
    \arabic{ejercicio}% Solo mostrar el número de ejercicio
  \else
    \thesection.\arabic{ejercicio}% Mostrar número de sección y número de ejercicio
  \fi
}


% \renewcommand\qedsymbol{$\blacksquare$}         % Cambiar símbolo QED
%------------------------------------------------------------------------

% Paquetes para encabezados
\usepackage{fancyhdr}
\pagestyle{fancy}
\fancyhf{}

\newcommand{\helv}{ % Modificación tamaño de letra
\fontfamily{}\fontsize{12}{12}\selectfont}
\setlength{\headheight}{15pt} % Amplía el tamaño del índice


%\usepackage{lastpage}   % Referenciar última pag   \pageref{LastPage}
\fancyfoot[C]{\thepage}

%------------------------------------------------------------------------

% Conseguir que no ponga "Capítulo 1". Sino solo "1."
\makeatletter
\@ifclassloaded{book}{
  \renewcommand{\chaptermark}[1]{\markboth{\thechapter.\ #1}{}} % En el encabezado
    
  \renewcommand{\@makechapterhead}[1]{%
  \vspace*{50\p@}%
  {\parindent \z@ \raggedright \normalfont
    \ifnum \c@secnumdepth >\m@ne
      \huge\bfseries \thechapter.\hspace{1em}\ignorespaces
    \fi
    \interlinepenalty\@M
    \Huge \bfseries #1\par\nobreak
    \vskip 40\p@
  }}
}
\makeatother

%------------------------------------------------------------------------
% Paquetes de cógido
\usepackage{minted}
\renewcommand\listingscaption{Código fuente}

\usepackage{fancyvrb}
% Personaliza el tamaño de los números de línea
\renewcommand{\theFancyVerbLine}{\small\arabic{FancyVerbLine}}

% Estilo para C++
\newminted{cpp}{
    frame=lines,
    framesep=2mm,
    baselinestretch=1.2,
    linenos,
    escapeinside=||
}

% para minted
\definecolor{LightGray}{rgb}{0.95,0.95,0.92}
\setminted{
    linenos=true,
    stepnumber=5,
    numberfirstline=true,
    autogobble,
    breaklines=true,
    breakautoindent=true,
    breaksymbolleft=,
    breaksymbolright=,
    breaksymbolindentleft=0pt,
    breaksymbolindentright=0pt,
    breaksymbolsepleft=0pt,
    breaksymbolsepright=0pt,
    fontsize=\footnotesize,
    bgcolor=LightGray,
    numbersep=10pt
}


\usepackage{listings} % Para incluir código desde un archivo

\renewcommand\lstlistingname{Código Fuente}
\renewcommand\lstlistlistingname{Índice de Códigos Fuente}

% Definir colores
\definecolor{vscodepurple}{rgb}{0.5,0,0.5}
\definecolor{vscodeblue}{rgb}{0,0,0.8}
\definecolor{vscodegreen}{rgb}{0,0.5,0}
\definecolor{vscodegray}{rgb}{0.5,0.5,0.5}
\definecolor{vscodebackground}{rgb}{0.97,0.97,0.97}
\definecolor{vscodelightgray}{rgb}{0.9,0.9,0.9}

% Configuración para el estilo de C similar a VSCode
\lstdefinestyle{vscode_C}{
  backgroundcolor=\color{vscodebackground},
  commentstyle=\color{vscodegreen},
  keywordstyle=\color{vscodeblue},
  numberstyle=\tiny\color{vscodegray},
  stringstyle=\color{vscodepurple},
  basicstyle=\scriptsize\ttfamily,
  breakatwhitespace=false,
  breaklines=true,
  captionpos=b,
  keepspaces=true,
  numbers=left,
  numbersep=5pt,
  showspaces=false,
  showstringspaces=false,
  showtabs=false,
  tabsize=2,
  frame=tb,
  framerule=0pt,
  aboveskip=10pt,
  belowskip=10pt,
  xleftmargin=10pt,
  xrightmargin=10pt,
  framexleftmargin=10pt,
  framexrightmargin=10pt,
  framesep=0pt,
  rulecolor=\color{vscodelightgray},
  backgroundcolor=\color{vscodebackground},
}

%------------------------------------------------------------------------

% Comandos definidos
\newcommand{\bb}[1]{\mathbb{#1}}
\newcommand{\cc}[1]{\mathcal{#1}}

% I prefer the slanted \leq
\let\oldleq\leq % save them in case they're every wanted
\let\oldgeq\geq
\renewcommand{\leq}{\leqslant}
\renewcommand{\geq}{\geqslant}

% Si y solo si
\newcommand{\sii}{\iff}

% Letras griegas
\newcommand{\eps}{\epsilon}
\newcommand{\veps}{\varepsilon}
\newcommand{\lm}{\lambda}

\newcommand{\ol}{\overline}
\newcommand{\ul}{\underline}
\newcommand{\wt}{\widetilde}
\newcommand{\wh}{\widehat}

\let\oldvec\vec
\renewcommand{\vec}{\overrightarrow}

% Derivadas parciales
\newcommand{\del}[2]{\frac{\partial #1}{\partial #2}}
\newcommand{\Del}[3]{\frac{\partial^{#1} #2}{\partial #3^{#1}}}
\newcommand{\deld}[2]{\dfrac{\partial #1}{\partial #2}}
\newcommand{\Deld}[3]{\dfrac{\partial^{#1} #2}{\partial #3^{#1}}}


\newcommand{\AstIg}{\stackrel{(\ast)}{=}}
\newcommand{\Hop}{\stackrel{L'H\hat{o}pital}{=}}

\newcommand{\red}[1]{{\color{red}#1}} % Para integrales, destacar los cambios.

% Método de integración
\newcommand{\MetInt}[2]{
    \left[\begin{array}{c}
        #1 \\ #2
    \end{array}\right]
}

% Declarar aplicaciones
% 1. Nombre aplicación
% 2. Dominio
% 3. Codominio
% 4. Variable
% 5. Imagen de la variable
\newcommand{\Func}[5]{
    \begin{equation*}
        \begin{array}{rrll}
            #1:& #2 & \longrightarrow & #3\\
               & #4 & \longmapsto & #5
        \end{array}
    \end{equation*}
}

%------------------------------------------------------------------------

\usepgfplotslibrary{fillbetween}

\DeclareMathOperator{\Var}{Var}
\DeclareMathOperator{\Cov}{Cov}
\begin{document}

    % 1. Foto de fondo
    % 2. Título
    % 3. Encabezado Izquierdo
    % 4. Color de fondo
    % 5. Coord x del titulo
    % 6. Coord y del titulo
    % 7. Fecha

    
    % 1. Foto de fondo
% 2. Título
% 3. Encabezado Izquierdo
% 4. Color de fondo
% 5. Coord x del titulo
% 6. Coord y del titulo
% 7. Fecha

\newcommand{\portada}[7]{

    \portadaBase{#1}{#2}{#3}{#4}{#5}{#6}{#7}
    \portadaBook{#1}{#2}{#3}{#4}{#5}{#6}{#7}
}

\newcommand{\portadaExamen}[7]{

    \portadaBase{#1}{#2}{#3}{#4}{#5}{#6}{#7}
    \portadaArticle{#1}{#2}{#3}{#4}{#5}{#6}{#7}
}




\newcommand{\portadaBase}[7]{

    % Tiene la portada principal y la licencia Creative Commons
    
    % 1. Foto de fondo
    % 2. Título
    % 3. Encabezado Izquierdo
    % 4. Color de fondo
    % 5. Coord x del titulo
    % 6. Coord y del titulo
    % 7. Fecha
    
    
    \thispagestyle{empty}               % Sin encabezado ni pie de página
    \newgeometry{margin=0cm}        % Márgenes nulos para la primera página
    
    
    % Encabezado
    \fancyhead[L]{\helv #3}
    \fancyhead[R]{\helv \nouppercase{\leftmark}}
    
    
    \pagecolor{#4}        % Color de fondo para la portada
    
    \begin{figure}[p]
        \centering
        \transparent{0.3}           % Opacidad del 30% para la imagen
        
        \includegraphics[width=\paperwidth, keepaspectratio]{assets/#1}
    
        \begin{tikzpicture}[remember picture, overlay]
            \node[anchor=north west, text=white, opacity=1, font=\fontsize{60}{90}\selectfont\bfseries\sffamily, align=left] at (#5, #6) {#2};
            
            \node[anchor=south east, text=white, opacity=1, font=\fontsize{12}{18}\selectfont\sffamily, align=right] at (9.7, 3) {\textbf{\href{https://losdeldgiim.github.io/}{Los Del DGIIM}}};
            
            \node[anchor=south east, text=white, opacity=1, font=\fontsize{12}{15}\selectfont\sffamily, align=right] at (9.7, 1.8) {Doble Grado en Ingeniería Informática y Matemáticas\\Universidad de Granada};
        \end{tikzpicture}
    \end{figure}
    
    
    \restoregeometry        % Restaurar márgenes normales para las páginas subsiguientes
    \pagecolor{white}       % Restaurar el color de página
    
    
    \newpage
    \thispagestyle{empty}               % Sin encabezado ni pie de página
    \begin{tikzpicture}[remember picture, overlay]
        \node[anchor=south west, inner sep=3cm] at (current page.south west) {
            \begin{minipage}{0.5\paperwidth}
                \href{https://creativecommons.org/licenses/by-nc-nd/4.0/}{
                    \includegraphics[height=2cm]{assets/Licencia.png}
                }\vspace{1cm}\\
                Esta obra está bajo una
                \href{https://creativecommons.org/licenses/by-nc-nd/4.0/}{
                    Licencia Creative Commons Atribución-NoComercial-SinDerivadas 4.0 Internacional (CC BY-NC-ND 4.0).
                }\\
    
                Eres libre de compartir y redistribuir el contenido de esta obra en cualquier medio o formato, siempre y cuando des el crédito adecuado a los autores originales y no persigas fines comerciales. 
            \end{minipage}
        };
    \end{tikzpicture}
    
    
    
    % 1. Foto de fondo
    % 2. Título
    % 3. Encabezado Izquierdo
    % 4. Color de fondo
    % 5. Coord x del titulo
    % 6. Coord y del titulo
    % 7. Fecha


}


\newcommand{\portadaBook}[7]{

    % 1. Foto de fondo
    % 2. Título
    % 3. Encabezado Izquierdo
    % 4. Color de fondo
    % 5. Coord x del titulo
    % 6. Coord y del titulo
    % 7. Fecha

    % Personaliza el formato del título
    \pretitle{\begin{center}\bfseries\fontsize{42}{56}\selectfont}
    \posttitle{\par\end{center}\vspace{2em}}
    
    % Personaliza el formato del autor
    \preauthor{\begin{center}\Large}
    \postauthor{\par\end{center}\vfill}
    
    % Personaliza el formato de la fecha
    \predate{\begin{center}\huge}
    \postdate{\par\end{center}\vspace{2em}}
    
    \title{#2}
    \author{\href{https://losdeldgiim.github.io/}{Los Del DGIIM}}
    \date{Granada, #7}
    \maketitle
    
    \tableofcontents
}




\newcommand{\portadaArticle}[7]{

    % 1. Foto de fondo
    % 2. Título
    % 3. Encabezado Izquierdo
    % 4. Color de fondo
    % 5. Coord x del titulo
    % 6. Coord y del titulo
    % 7. Fecha

    % Personaliza el formato del título
    \pretitle{\begin{center}\bfseries\fontsize{42}{56}\selectfont}
    \posttitle{\par\end{center}\vspace{2em}}
    
    % Personaliza el formato del autor
    \preauthor{\begin{center}\Large}
    \postauthor{\par\end{center}\vspace{3em}}
    
    % Personaliza el formato de la fecha
    \predate{\begin{center}\huge}
    \postdate{\par\end{center}\vspace{5em}}
    
    \title{#2}
    \author{\href{https://losdeldgiim.github.io/}{Los Del DGIIM}}
    \date{Granada, #7}
    \thispagestyle{empty}               % Sin encabezado ni pie de página
    \maketitle
    \vfill
}
    \portadaExamen{ffccA4.jpg}{Probabilidad\\Examen IV}{Probabilidad. Examen IV}{MidnightBlue}{-8}{28}{2024-2025}{Arturo Olivares Martos \\ José Juan Urrutia Milán}

    \begin{description}
        \item[Asignatura] Probabilidad.
        \item[Curso Académico] 2023-24.
        \item[Grado] Doble Grado en Ingeniería Informática y Matemáticas.
        \item[Grupo] Único.
        %\item[Profesor] José María Espinar García.
        \item[Descripción] Examen Ordinario 
        \item[Fecha] 17 de enero de 2024.
        \item[Duración] 3 horas.
    
    \end{description}
    \newpage

    \begin{ejercicio}[5 puntos]
        Dado el vector aleatorio continuo $(X,Y)$ distribuido uniformemente en el recinto
        \begin{equation*}
            C = \{(x,y)\in \mathbb{R}^2 \mid y-x<1 \ \land \ x<0,\ y>0\}
        \end{equation*}
        \begin{enumerate}
            \item \textbf{(0.25 puntos)} Obtener la función de densidad conjunta.
            
            Veamos el conjunto $C$ gráficamente en la Figura~\ref{fig:conjuntoC}.
            \begin{figure}[H]
                \centering
                \begin{tikzpicture}
                    \begin{axis}[
                        axis lines = center,
                        xlabel = $x$,
                        ylabel = $y$,
                        xmin = -2,
                        xmax = 1,
                        ymin = -1,
                        ymax = 2,
                        xtick = {-1},
                        ytick = {1},
                        xticklabels = {$-1$},
                        yticklabels = {$1$},
                        legend pos=north east,
                        axis equal,
                    ]
                    \addplot [
                        domain=-2:2,
                        samples=2,
                        color=red,
                        style=dashed,
                    ] {x+1};
                    \addlegendentry{$y-x=1$}


                    % Zona R0
                    % (-1,0) (-1,5) (-3,5) (-3,-5) (2,-5) (2,0) (-1,0)
                    \addplot [
                        fill=gray,
                        fill opacity=0.5,
                        draw=none,
                        forget plot,
                    ] coordinates {(-1,0) (-1,5) (-3,5) (-3,-5) (2,-5) (2,0) (-1,0)};
                    \node at (axis cs:-0.5,-0.5) {$R_0$};

                    % Zona R1 (C)
                    % (-1,0) (0,0) (0,1)
                    \addplot [
                        fill=orange,
                        fill opacity=0.5,
                        draw=none,
                    ] coordinates {(-1,0) (0,0) (0,1)};
                    \node at (axis cs:-0.3,0.3) {$R_1 (C)$};

                    % Zona R2
                    % (-1,0) (-1,4) (0,4) (0,0)
                    \addplot [
                        fill=blue,
                        fill opacity=0.5,
                        draw=none,
                        forget plot,
                    ] coordinates {(-1,0) (-1,4) (0,4) (0,1)};
                    \node at (axis cs:-0.65,1) {$R_2$};

                    % Zona R3
                    % (0,0) (0,1) (2,1) (2,0)
                    \addplot [
                        fill=green,
                        fill opacity=0.5,
                        draw=none,
                        forget plot,
                    ] coordinates {(0,0) (0,1) (2,1) (2,0)};
                    \node at (axis cs:0.7,0.5) {$R_3$};

                    % Zona R4
                    % (0,1) (0,4) (2,4) (2,1)
                    \addplot [
                        fill=yellow,
                        fill opacity=0.5,
                        draw=none,
                        forget plot,
                    ] coordinates {(0,1) (0,4) (2,4) (2,1)};
                    \node at (axis cs:0.7,1.3) {$R_4$};                    
                    \end{axis}
                \end{tikzpicture}
                \caption{Conjunto $C$.}
                \label{fig:conjuntoC}
            \end{figure}

            Como se distribuye uniformemente, la función de densidad conjunta es constante en la región $C$ y nula en el resto del plano. Por tanto, la función de densidad conjunta es:
            \begin{equation*}
                f_{X,Y}(x,y) = \begin{cases}
                    k & \text{si } (x,y) \in C \\
                    0 & \text{en otro caso}
                \end{cases}
            \end{equation*}

            Para obtener la constante $k$, calculamos la integral de la función de densidad conjunta en la región $C$:
            \begin{align*}
                1 &= \int_{\bb{R}^2} f_{X,Y} = \int_{C} k = k\lm(C) = k\cdot \frac{1}{2}\Longrightarrow
                k = 2
            \end{align*}
            \item \textbf{(1.50 puntos)} Obtener la función de distribución de probabilidad conjunta.
            \begin{observacion}
                Se obtiene \textbf{hasta 1 punto} si las integrales se dejan indicadas y \textbf{hasta 1.5 puntos} si se obtienen sus primitivas de forma explícita.
            \end{observacion}

            Estudiamos cada uno de los casos en función de los valores de $x$ e $y$:
            \begin{itemize}
                \item \ul{$Si (x,y)\in R_0$} ($x<-1$ o $y<0$):
                \begin{align*}
                    F_{X,Y}(x,y) &= \int_{-\infty}^{x}\int_{-\infty}^{y} f_{X,Y}(u,v) \ du \ dv
                    = \int_{-\infty}^{x}\int_{-\infty}^{y} 0 \ du \ dv = 0
                \end{align*}

                \item \ul{$Si (x,y)\in R_1$} ($-1\leq x<0$ y $0\leq y<1-x$):
                \begin{align*}
                    F_{X,Y}(x,y) &= \int_{-\infty}^{x}\int_{-\infty}^{y} f_{X,Y}(u,v) \ du \ dv
                    = \int_{-1}^{y-1}\int_{0}^{u+1} 2 \ du \ dv + \int_{y-1}^{x}\int_{0}^{y} 2 \ du \ dv
                    =\\&= 2\int_{-1}^{y-1} (u+1) \ du + 2\int_{y-1}^{x} y \ du
                    = 2\left[\frac{u^2}{2}+u\right]_{-1}^{y-1} + 2y\left[u\right]_{y-1}^{x}
                    =\\&= 2\left[\frac{(y-1)^2}{2} + y-1-\frac{1}{2}+1\right] + 2y(x-y+1)
                    =\\&= (y-1)^2 + 2(y-1) + 1 + 2y(x-y+1)
                    =\\&= (y-1+1)^2 + 2y(x-y+1) = y^2 + 2y(x-y+1)
                    =\\&= y^2 + 2xy - 2y^2 + 2y
                    = -y^2 + 2xy + 2y
                \end{align*}

                \item \ul{$Si (x,y)\in R_2$} ($-1\leq x<0$ y $y-x\geq 1$):
                \begin{align*}
                    F_{X,Y}(x,y) &= \int_{-\infty}^{x}\int_{-\infty}^{y} f_{X,Y}(u,v) \ du \ dv
                    = \int_{-1}^{x}\int_{0}^{u+1} 2 \ du \ dv
                    = 2\int_{-1}^{x} (u+1) \ du
                    =\\&= 2\left[\frac{u^2}{2}+u\right]_{-1}^{x}
                    = 2\left[\frac{x^2}{2}+x+\frac{1}{2}\right]
                    = x^2 + 2x + 1 = (x+1)^2
                \end{align*}

                \item \ul{$Si (x,y)\in R_3$} ($0\leq x$ y $0\leq y<1$):
                \begin{align*}
                    F_{X,Y}(x,y) &= \int_{-\infty}^{x}\int_{-\infty}^{y} f_{X,Y}(u,v) \ du \ dv
                    = \int_{-1}^{y-1}\int_{0}^{u+1} 2 \ du \ dv
                    + \int_{y-1}^{0}\int_{0}^{y} 2 \ du \ dv
                    =\\&= 2\int_{-1}^{y-1} (u+1) \ du + 2\int_{y-1}^{0} y \ du
                    = 2\left[\frac{u^2}{2}+u\right]_{-1}^{y-1} + 2y\left[u\right]_{y-1}^{0}
                    =\\&= 2\left[\frac{(y-1)^2}{2} + y-1+\frac{1}{2}\right] - 2y(y-1)
                    =\\&= (y-1)^2 + 2(y-1)+1 - 2y(y-1)
                    = (y-1+1)^2 - 2y(y-1)
                    =\\&= y^2 - 2y^2 + 2y = -y^2 + 2y = y(2-y)
                \end{align*}

                \item \ul{$Si (x,y)\in R_4$} ($0\leq x$ y $1\leq y$):
                \begin{align*}
                    F_{X,Y}(x,y) &= \int_{-\infty}^{x}\int_{-\infty}^{y} f_{X,Y}(u,v) \ du \ dv
                    = 1
                \end{align*}
            \end{itemize}

            Por tanto, la función de distribución de probabilidad conjunta es:
            \begin{equation*}
                F_{X,Y}(x,y) = \begin{cases}
                    0 & \text{si } x<-1 \ \lor \ y<0 \qquad (x,y)\in R_0 \\
                    -y^2 + 2xy + 2y & \text{si } -1\leq x<0 \ \land \ 0\leq y<1-x \qquad (x,y)\in R_1 \\
                    (x+1)^2 & \text{si } -1\leq x<0 \ \land \ y-x\geq 1 \qquad (x,y)\in R_2 \\
                    y(2-y) & \text{si } 0\leq x \ \land \ 0\leq y<1 \qquad (x,y)\in R_3 \\
                    1 & \text{si } 0\leq x \ \land \ 1\leq y \qquad (x,y)\in R_4
                \end{cases}
            \end{equation*}


            
            \item \textbf{(0.75 puntos)} Obtener las funciones de densidad condicionadas.
            
            Para obtener las funciones de densidad condicionadas, calculamos en primer lugar las funciones de densidad marginales. Tenemos que:
            \begin{align*}
                f_X(x) &= \int_{-\infty}^{\infty} f_{X,Y}(x,y) \ dy
                = \int_{0}^{1-x} 2 \ dy = 2\left[y\right]_{0}^{1-x} = 2(1-x) \qquad \forall x\in \left[-1,0\right] \\
                f_Y(y) &= \int_{-\infty}^{\infty} f_{X,Y}(x,y) \ dx
                = \int_{y-1}^{0} 2 \ dx = 2\left[x\right]_{y-1}^{0} = 2(1-y) \qquad \forall y\in \left[0,1\right]
            \end{align*}

            Por tanto, las funciones de densidad condicionadas son:
            \begin{align*}
                f_{X\mid Y=y}(x) &= \frac{f_{X,Y}(x,y)}{f_Y(y)} = \frac{2}{2(1-y)} = \frac{1}{1-y} \qquad \forall x\in \left[-1,y^*-1\right], \ y\in \left[0,1\right] \\
                f_{Y\mid X=x}(y) &= \frac{f_{X,Y}(x,y)}{f_X(x)} = \frac{2}{2(1-x)} = \frac{1}{1-x} \qquad \forall y\in \left[0,1+x\right], \ x\in \left[-1,0\right]
            \end{align*}
            \item \textbf{(0.25 puntos)} Obtener la probabilidad de que $X-Y > 0$.
            
            Tenemos que:
            \begin{equation*}
                \left\{(x,y)\in \bb{R}^2 \mid x-y>0\right\}\cap C=\emptyset
            \end{equation*}

            Por tanto, la probabilidad de que $X-Y>0$ es nula.
            \item \textbf{(0.25 puntos)} Obtener la probabilidad de que $X+Y < 0$.
            
            Veamos el conjunto $\left\{(x,y)\in \bb{R}^2 \mid x+y<0\right\}$ gráficamente en la Figura~\ref{fig:conjuntoXY}.
            \begin{figure}[H]
                \centering
                \begin{tikzpicture}
                    \begin{axis}[
                        axis lines = center,
                        xlabel = $x$,
                        ylabel = $y$,
                        xmin = -2,
                        xmax = 1,
                        ymin = -1,
                        ymax = 2,
                        xtick = {-1},
                        ytick = {1},
                        xticklabels = {$-1$},
                        yticklabels = {$1$},
                        legend pos=north east,
                        axis equal,
                    ]
                    \addplot [
                        domain=-2:2,
                        samples=2,
                        color=red,
                        style=dashed,
                    ] {x+1};
                    \addlegendentry{$y-x=1$}

                    \addplot [
                        domain=-2:2,
                        samples=2,
                        color=blue,
                        style=dashed,
                    ] {-x};
                    \addlegendentry{$x+y=0$}

                    \addplot [
                        fill=orange,
                        fill opacity=0.5,
                        draw=none,
                    ] coordinates {(-1,0) (0,0) (-0.5,0.5)};
                    
                    % Punto en (-0.5,0.5), con etiqueta
                    \addplot [mark=*] coordinates {(-0.5,0.5)};
                    \node[left] at (axis cs:-0.5,0.5) {$(\nicefrac{-1}{2},\nicefrac{1}{2})$};
                    \end{axis}
                \end{tikzpicture}
                \caption{Conjunto $\left\{(x,y)\in \bb{R}^2 \mid x+y<0\right\}$.}
                \label{fig:conjuntoXY}
            \end{figure}

            Calculamos la probabilidad de que $X+Y<0$:
            \begin{align*}
                P[X+Y<0] &= \int_{-1}^{\nicefrac{-1}{2}}\int_{0}^{1+x} 2 \ dy \ dx + \int_{\nicefrac{-1}{2}}^{0}\int_{0}^{-x} 2 \ dy \ dx
                =\\&= 2\int_{-1}^{\nicefrac{-1}{2}} (1+x) \ dx + 2\int_{\nicefrac{-1}{2}}^{0} -x \ dx
                = 2\left[\frac{x^2}{2}+x\right]_{-1}^{\nicefrac{-1}{2}} + 2\left[-\frac{x^2}{2}\right]_{\nicefrac{-1}{2}}^{0}
                =\\&= 2\left[\frac{1}{8}-\frac{1}{2}-\frac{1}{2}+1\right] + 2\left[0+\frac{1}{8}\right]
                = 2\cdot 2\cdot \frac{1}{8} = \frac{1}{2}
            \end{align*}
            \item \textbf{(1.50 puntos)} Obtener la mejor aproximación minimo cuadrática a la variable aleatoria $Y$ conocidos los valores de la variable $X$ y el error cuadrático medio de esta aproximación.
            
            Para obtener la mejor aproximación mínimos cuadrados de la variable aleatoria $Y$ conocidos los valores de la variable $X$, calculamos la curva de regresión de $Y$ sobre $X$. Para ello, tenemos que:
            \begin{align*}
                E[Y\mid X=x] &= \int_{-\infty}^{\infty} y\cdot f_{Y\mid X=x}(y) \ dy
                = \int_{0}^{1+x} y\cdot \frac{1}{1+x} \ dy
                = \frac{1}{1+x}\int_{0}^{1+x} y \ dy
                =\\&= \frac{1}{1+x}\left[\frac{y^2}{2}\right]_{0}^{1+x}
                = \frac{1}{1+x}\cdot \frac{(1+x)^2}{2}
                = \frac{1+x}{2}
            \end{align*}

            Por tanto, la mejor aproximación mínimos cuadrados de la variable aleatoria $Y$ conocidos los valores de la variable $X$ es:
            \begin{equation*}
                E[Y\mid X] = \frac{1+X}{2}
            \end{equation*}

            Para calcular el error cuadrático medio de esta aproximación, tenemos que:
            \begin{align*}
                \text{E.C.M.}(E[Y\mid X]) &= E[(Y-E[Y\mid X])^2]
                = E\left[\left(Y-\frac{1+X}{2}\right)^2\right]
                =\\&= E\left[Y^2 - Y(1+X) + \frac{(1+X)^2}{4}\right]
                =\\&= E[Y^2] - E[Y]-E[XY] + \frac{1}{4}E[X^2] + \frac{1}{2}E[X] + \frac{1}{4}
            \end{align*}

            De forma análoga, tenemos que:
            \begin{align*}
                \text{E.C.M.}(E[Y\mid X]) &= E[\Var[Y\mid X]]
                = E[E[Y^2\mid X] - E^2[Y\mid X]]
                =\\&= E[Y^2] - E\left[\left(E[Y\mid X]\right)^2\right]
                = E[Y^2] - E\left[\left(\frac{1+X}{2}\right)^2\right]
            \end{align*}

            Como podemos ver, en este caso se trata del cálculo de menos esperanzas, por lo que nos decantamos por esta opción.
            \begin{align*}
                E[Y^2] &= \int_{-\infty}^{\infty} y^2\cdot f_Y(y) \ dy
                = 2\int_{0}^{1} y^2(1-y) \ dy
                = 2\int_{0}^{1} y^2-y^3 \ dy
                =\\&= 2\left[\frac{y^3}{3}-\frac{y^4}{4}\right]_{0}^{1}
                = \frac{2}{3}-\frac{1}{2} = \frac{1}{6} \\
                E\left[\left(\frac{1+X}{2}\right)^2\right] &= \frac{1}{4}\int_{-1}^{0} \left(1+x\right)^2\cdot 2(1+x) \ dx
                = \frac{1}{2}\int_{-1}^{0} (1+x)^3 \ dx
                = \frac{1}{2}\left[\frac{(1+x)^4}{4}\right]_{-1}^{0}
                = \frac{1}{8}
            \end{align*}

            Por tanto, el error cuadrático medio de la mejor aproximación mínimos cuadrados de la variable aleatoria $Y$ conocidos los valores de la variable $X$ es:
            \begin{equation*}
                \text{E.C.M.}(E[Y\mid X]) = E[Y^2] - E\left[\left(\frac{1+X}{2}\right)^2\right] = \frac{1}{6} - \frac{1}{8} = \frac{1}{24}
            \end{equation*}
            \item \textbf{(0.50 puntos)} Obtener una medida de la bondad del ajuste del apartado anterior.\\
            
            Para obtener una medida de la bondad del ajuste del apartado anterior, calculamos el coeficiente de determinación. Para ello, tenemos que:
            \begin{align*}
                \eta^2_{Y / X} &= \frac{\Var[E[Y\mid X]]}{\Var[Y]}
                = 1-\frac{\text{E.C.M.}(E[Y\mid X])}{\Var[Y]}
            \end{align*}

            Calculamos la varianza de $Y$:
            \begin{align*}
                E[Y] &= \int_{-\infty}^{\infty} y\cdot f_Y(y) \ dy
                = 2\int_{0}^{1} y(1-y) \ dy
                = 2\int_{0}^{1} y-y^2 \ dy
                =\\&= 2\left[\frac{y^2}{2}-\frac{y^3}{3}\right]_{0}^{1}
                = 2\left[\frac{1}{2}-\frac{1}{3}\right]
                = \frac{1}{3} \\
                E[Y^2] &= \frac{1}{6} \\
                \Var[Y] &= E[Y^2] - E^2[Y] = \frac{1}{6} - \left(\frac{1}{3}\right)^2 = \frac{1}{18}
            \end{align*}

            Por tanto, el coeficiente de determinación es:
            \begin{align*}
                \eta^2_{Y / X} &= 1-\frac{\text{E.C.M.}(E[Y\mid X])}{\Var[Y]}
                = 1-\frac{\nicefrac{1}{24}}{\nicefrac{1}{18}}
                = 1-\frac{3}{4}
                = \frac{1}{4}
            \end{align*}

            Por tanto, la bondad del ajuste es del 25\%. Como vemos, no se trata de un ajuste de muy buena calidad.
        \end{enumerate}
    \end{ejercicio}

    \begin{ejercicio}[1 puntos]
        Dado un vector aleatorio $(X,Y)$ con función generatriz de momentos
        \begin{equation*}
            M_{(X,Y)}(t_1,t_2) = \exp\left(\dfrac{t_2+16t_1^2 + 4t_2^2 + 10t_1t_2}{2}\right)
        \end{equation*}
        \begin{enumerate}
            \item \textbf{(0.25 puntos)} Obtener la razón de correlación y el coeficiente de correlación lineal de las variables $(X,Y)$.
            
            La función generatriz de un vector aleatorio con distribución $\cc{N}_2(\mu,\Sigma)$ es:
            \begin{equation*}
                f(t_1,t_2) = \exp\left(\mu_1t_1+\mu_2t_2+\dfrac{\sigma_1^2t_1^2+\sigma_2^2t_2^2+2\rho\sigma_1\sigma_2t_1t_2}{2}\right)
            \end{equation*}

            Identificando términos, obtenemos que:
            \begin{equation*}
                \begin{cases}
                    \mu_1 = 0 \\
                    \mu_2 = \nicefrac{1}{2} \\
                    \sigma_1^2 = 16 \\
                    \sigma_2^2 = 4 \\
                    \rho\sigma_1\sigma_2 = 5
                \end{cases}
            \end{equation*}

            Por tanto, tenemos que $(X,Y)\sim \cc{N}_2\left(\mu,\Sigma\right)$ con:
            \begin{equation*}
                \mu = \begin{pmatrix}
                    0 \\ \nicefrac{1}{2}
                \end{pmatrix} \qquad \Sigma = \begin{pmatrix}
                    16 & 5 \\ 5 & 4
                \end{pmatrix}
            \end{equation*}

            Por tanto, tenemos que el coeficiente de correlación lineal es:
            \begin{equation*}
                \rho_{X,Y}=\rho=\dfrac{5}{\sqrt{16}\sqrt{4}} = \dfrac{5}{8}
            \end{equation*}

            Respecto a la razón de correlación, tenemos que:
            \begin{equation*}
                \eta_{X/Y}^2\AstIg \rho_{X,Y}^2 = \eta_{Y/X}^2=\left(\dfrac{5}{8}\right)^2 = \dfrac{25}{64}
            \end{equation*}
            donde en $(\ast)$ se ha utilizado que $(X,Y)$ sigue una distribución normal bivariante.

            \item \textbf{(0.25 puntos)} Indicar las distribuciones de las variables aleatorias $Y\mid X=0$ y $X\mid Y=2$.
            
            Tenemos que:
            \begin{align*}
                Y\mid X=x^* &\sim \cc{N}\left(\mu_2+\rho\cdot \dfrac{\sigma_2}{\sigma_1}(x^*-\mu_1),\sigma_2^2(1-\rho^2)\right) \\
                X\mid Y=y^* &\sim \cc{N}\left(\mu_1+\rho\cdot \dfrac{\sigma_1}{\sigma_2}(y^*-\mu_2),\sigma_1^2(1-\rho^2)\right)
            \end{align*}

            En concreto, para $x^*=0$ y $y^*=2$, tenemos que:
            \begin{align*}
                Y\mid X=0 &\sim \cc{N}\left(\nicefrac{1}{2},4\left(1-\left(\nicefrac{5}{8}\right)^2\right)\right) = \cc{N}\left(\nicefrac{1}{2},\nicefrac{39}{16}\right) \\
                X\mid Y=2 &\sim \cc{N}\left(0,16\left(1-\left(\nicefrac{5}{8}\right)^2\right)\right) = \cc{N}\left(0,\nicefrac{39}{4}\right)
            \end{align*}
            \item \textbf{(0.50 puntos)} Obtener la distribución de probabilidad del vector aleatorio $(2X, Y-X)$. Justificar que las variables aleatorias $2X$ y $Y-X$ tienen asociación lineal muy alta en sentido negativo.
            
            Tenemos que:
            \begin{equation*}
                (2X,Y-X) = \begin{pmatrix}
                    X & Y
                \end{pmatrix}A,\qquad A = 
                \begin{pmatrix}
                    2 & -1 \\ 0 & 1
                \end{pmatrix}
            \end{equation*}

            Por tanto, notando $X'=2X$ y $Y'=Y-X$, tenemos que $(X',Y')\sim \cc{N}(\mu A,A^t\Sigma A)$, donde:
            \begin{align*}
                \mu A &= \begin{pmatrix}
                    0 & \nicefrac{1}{2}
                \end{pmatrix}\begin{pmatrix}
                    2 & -1 \\ 0 & 1
                \end{pmatrix} = \begin{pmatrix}
                    0 & \nicefrac{1}{2}
                \end{pmatrix}
                \\
                A^t\Sigma A &= \begin{pmatrix}
                    2 & 0 \\ -1 & 1
                \end{pmatrix}\begin{pmatrix}
                    16 & 5 \\ 5 & 4
                \end{pmatrix}\begin{pmatrix}
                    2 & -1 \\ 0 & 1
                \end{pmatrix} = \begin{pmatrix}
                    32 & 10 \\ -11 & -1
                \end{pmatrix}
                \begin{pmatrix}
                    2 & -1 \\ 0 & 1
                \end{pmatrix}
                = \begin{pmatrix}
                    64 & -22 \\ -22 & 10
                \end{pmatrix}
            \end{align*}

            Por tanto, tenemos que:
            \begin{align*}
                \rho_{X',Y'} &= \dfrac{-22}{\sqrt{64}\sqrt{10}} \approx -0.87\\
                \rho_{X',Y'}^2 &= \dfrac{22^2}{64\cdot 10} = \dfrac{121}{160} = 0.75625
            \end{align*}

            Por tanto, como $\rho_{X',Y'}^2$ se aproxima a $1$, la asociación lineal entre $2X$ y $Y-X$ es muy alta, y al ser $\rho_{X',Y'}$ negativo, la correlación es negativa.
        \end{enumerate}
    \end{ejercicio}

    \begin{ejercicio}[3 puntos]
        Sea $(X,Y)$ un vector aleatorio. Se pretenden predecir, por mínimos cuadrados, los valores de la variable $Y$ a partir de una función lineal de la variable $X$, y viceversa.
        \begin{enumerate}
            \item \textbf{(2 puntos)} Obtener de forma razonada los coeficientes del modelo lineal de $X$ sobre $Y$.
            
            Se busca aproximar $X$ como $\wh{X}=aY+b$. Para ello, se minimiza el error cuadrático medio:
            \begin{align*}
                \text{E.C.M.}(X\mid Y) &= E[(X-\wh{X})^2]
                = E\left[(X-aY-b)^2\right]
                =\\&= E\left[X^2-2aXY-2bX+a^2Y^2+2abY+b^2\right]
                =\\&= E[X^2]-2aE[XY]-2bE[X]+a^2E[Y^2]+2abE[Y]+b^2
            \end{align*}

            Para ello, se busca minizar la siguiente función:
            \begin{equation*}
                L(a,b) = E.C.M.(X\mid Y) = E[X^2]-2aE[XY]-2bE[X]+a^2E[Y^2]+2abE[Y]+b^2
            \end{equation*}

            Se tiene demostrado en Teoría que llegamos a la siguiente expresión:
            \begin{equation*}
                \begin{cases}
                    a = \dfrac{\Cov[X,Y]}{\Var[Y]} \\
                    b = E[X]-aE[Y]
                \end{cases}
            \end{equation*}
            \item \textbf{(1 punto)} Si $3y-x+1=0$ y $x-2y-1=0$ son las rectas de regresión del vector $(X,Y)$: identificar la recta de regresión de Y sobre X; obtener una medida de la proporción de varianza de cada variable que queda explicada por el modelo de regresión lineal y calcular la esperanza del vector $(X,Y)$.\\
            
            Suponemos que las rectas de regresión de $X$ sobre $Y$ y de $Y$ sobre $X$ son $3y-x+1=0$ y $x-2y-1=0$, respectivamente. Por tanto, tenemos que:
            \begin{align*}
                x &= 3y+1 = \dfrac{\Cov[X,Y]}{\Var[Y]}\cdot y + E[X]-\dfrac{\Cov[X,Y]}{\Var[Y]}\cdot E[Y] \\
                y &= \dfrac{1}{2}x-\dfrac{1}{2} = \dfrac{\Cov[X,Y]}{\Var[X]}\cdot x + E[Y]-\dfrac{\Cov[X,Y]}{\Var[X]}\cdot E[X]
            \end{align*}

            Identificando términos, obtenemos que:
            \begin{equation*}
                \dfrac{\Cov[X,Y]}{\Var[X]}\cdot \dfrac{\Cov[X,Y]}{\Var[Y]} = \rho_{X,Y}^2 = 3\cdot \dfrac{1}{2} = \dfrac{3}{2}>1
            \end{equation*}

            Por tanto, llegamos a una contradicción, por lo que la suposición es incorrecta. La recta de regresión de $Y$ sobre $X$ es $y=\nicefrac{1}{3}x-\nicefrac{1}{3}$, y la de $X$ sobre $Y$ es $x=2y+1$.\\

            La proporción de varianza de cada variable que queda explicada por el modelo de regresión lineal es el coeficiente de determinación, que en este caso es:
            \begin{equation*}
                \rho_{X,Y}^2 = \dfrac{1}{3}\cdot 2=\dfrac{2}{3}\approx 0.667\%
            \end{equation*}

            Por último, por identificación de términos, tenemos el siguiente sistema:
            \begin{equation*}
                \left\{\begin{array}{rcl}
                    E[X]-2E[Y]&=&1 \\
                    E[Y]-\dfrac{1}{3}E[X]&=&-\dfrac{1}{3}
                \end{array}\right\}
                \Longrightarrow
                \begin{cases}
                    E[X]=1 \\
                    E[Y]=0
                \end{cases}
            \end{equation*}

            Por tanto, tenemos que:
            \begin{equation*}
                E[(X,Y)] = \begin{pmatrix}
                    1 & 0
                \end{pmatrix}
            \end{equation*}
        \end{enumerate}
    \end{ejercicio}

    \begin{ejercicio}[1 punto]
        Sean $X_1$, $X_2$, \ldots, $X_n$ $n$ variables aleatorias independientes e identicamente distribuidas según una ley uniforme en el intervalo $[0,\theta]$ con $\theta > 0$. Se considera la sucesión de variables aleatorias cuyo término general es de la forma $X_{(n)} = \max\{X_1, X_2, \ldots, X_n\}$. Probar que la sucesión anterior converge en ley a una variable aleatoria degenerada en $\theta$.\\

        Puesto que no son idénticamente distribuidas, no podemos aplicar el Teorema de Lévy, por lo que hemos de partir de la definición. Para ello, calculamos la función de distribución de $X_{(n)}$. Tenemos que:
        \begin{equation*}
            F_{X_{(n)}}(x) = P[X_{(n)}\leq x] = P[\max\{X_1,X_2,\ldots,X_n\}\leq x] = P[X_1\leq x, X_2\leq x,\ldots,X_n\leq x]
        \end{equation*}

        Puesto que las variables son independientes, tenemos que:
        \begin{equation*}
            F_{X_{(n)}}(x) = P[X_1\leq x]\cdot P[X_2\leq x]\cdot \ldots \cdot P[X_n\leq x]
        \end{equation*}

        Puesto que son idénticamente distribuidas, tenemos que:
        \begin{equation*}
            F_{X_{(n)}}(x) = \left(P[X_1\leq x]\right)^n
        \end{equation*}

        Calculamos por tanto la función de densidad de $X_1$. Como sigue una distribución uniforme en $[0,\theta]$, tenemos que:
        \begin{equation*}
            f_{X_1}(x) = \begin{cases}
                \dfrac{1}{\theta} & \text{si } x\in [0,\theta] \\
                0 & \text{en otro caso}
            \end{cases}
        \end{equation*}

        Por tanto, la función de distribución de $X_1$ es:
        \begin{equation*}
            F_{X_1}(x) = \begin{cases}
                0 & \text{si } x<0 \\
                \dfrac{x}{\theta} & \text{si } x\in [0,\theta] \\
                1 & \text{si } x>\theta
            \end{cases}
        \end{equation*}

        Por tanto, la función de distribución de $X_{(n)}$ es:
        \begin{equation*}
            F_{X_{(n)}}(x) = \begin{cases}
                0 & \text{si } x<0 \\
                \left(\dfrac{x}{\theta}\right)^n & \text{si } x\in [0,\theta[ \\
                1 & \text{si } x\geq\theta
            \end{cases}
        \end{equation*}

        Por tanto, tenemos que:
        \begin{equation*}
            \lim_{n\to\infty} F_{X_{(n)}}(x) = \begin{cases}
                0 & \text{si } x<\theta\\
                1 & \text{si } x\geq \theta
            \end{cases}
        \end{equation*}

        Como esta es la función de distribución de una variable aleatoria degenerada en $\theta$, tenemos que:
        \begin{equation*}
            X_{(n)}\overset{L}{\longrightarrow} \theta
        \end{equation*}
    \end{ejercicio}


\end{document}
