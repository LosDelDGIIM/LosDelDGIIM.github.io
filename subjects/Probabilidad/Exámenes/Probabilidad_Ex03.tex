\documentclass[12pt]{article}

% Idioma y codificación
\usepackage[spanish, es-tabla]{babel}       %es-tabla para que se titule "Tabla"
\usepackage[utf8]{inputenc}

% Márgenes
\usepackage[a4paper,top=3cm,bottom=2.5cm,left=3cm,right=3cm]{geometry}

% Comentarios de bloque
\usepackage{verbatim}

% Paquetes de links
\usepackage[hidelinks]{hyperref}    % Permite enlaces
\usepackage{url}                    % redirecciona a la web

% Más opciones para enumeraciones
\usepackage{enumitem}

% Personalizar la portada
\usepackage{titling}

% Paquetes de tablas
\usepackage{multirow}


%------------------------------------------------------------------------

%Paquetes de figuras
\usepackage{caption}
\usepackage{subcaption} % Figuras al lado de otras
\usepackage{float}      % Poner figuras en el sitio indicado H.


% Paquetes de imágenes
\usepackage{graphicx}       % Paquete para añadir imágenes
\usepackage{transparent}    % Para manejar la opacidad de las figuras

% Paquete para usar colores
\usepackage[dvipsnames]{xcolor}
\usepackage{pagecolor}      % Para cambiar el color de la página

% Habilita tamaños de fuente mayores
\usepackage{fix-cm}

% Para los gráficos
\usepackage{tikz}

% Para poder situar los nodos en los grafos
\usetikzlibrary{positioning}


%------------------------------------------------------------------------

% Paquetes de matemáticas
\usepackage{mathtools, amsfonts, amssymb, mathrsfs}
\usepackage[makeroom]{cancel}     % Simplificar tachando
\usepackage{polynom}    % Divisiones y Ruffini
\usepackage{units} % Para poner fracciones diagonales con \nicefrac

\usepackage{pgfplots}   %Representar funciones
\pgfplotsset{compat=1.18}  % Versión 1.18

\usepackage{tikz-cd}    % Para usar diagramas de composiciones
\usetikzlibrary{calc}   % Para usar cálculo de coordenadas en tikz

%Definición de teoremas, etc.
\usepackage{amsthm}
%\swapnumbers   % Intercambia la posición del texto y de la numeración

\theoremstyle{plain}

\makeatletter
\@ifclassloaded{article}{
  \newtheorem{teo}{Teorema}[section]
}{
  \newtheorem{teo}{Teorema}[chapter]  % Se resetea en cada chapter
}
\makeatother

\newtheorem{coro}{Corolario}[teo]           % Se resetea en cada teorema
\newtheorem{prop}[teo]{Proposición}         % Usa el mismo contador que teorema
\newtheorem{lema}[teo]{Lema}                % Usa el mismo contador que teorema

\theoremstyle{remark}
\newtheorem*{observacion}{Observación}

\theoremstyle{definition}

\makeatletter
\@ifclassloaded{article}{
  \newtheorem{definicion}{Definición} [section]     % Se resetea en cada chapter
}{
  \newtheorem{definicion}{Definición} [chapter]     % Se resetea en cada chapter
}
\makeatother

\newtheorem*{notacion}{Notación}
\newtheorem*{ejemplo}{Ejemplo}
\newtheorem*{ejercicio*}{Ejercicio}             % No numerado
\newtheorem{ejercicio}{Ejercicio} [section]     % Se resetea en cada section


% Modificar el formato de la numeración del teorema "ejercicio"
\renewcommand{\theejercicio}{%
  \ifnum\value{section}=0 % Si no se ha iniciado ninguna sección
    \arabic{ejercicio}% Solo mostrar el número de ejercicio
  \else
    \thesection.\arabic{ejercicio}% Mostrar número de sección y número de ejercicio
  \fi
}


% \renewcommand\qedsymbol{$\blacksquare$}         % Cambiar símbolo QED
%------------------------------------------------------------------------

% Paquetes para encabezados
\usepackage{fancyhdr}
\pagestyle{fancy}
\fancyhf{}

\newcommand{\helv}{ % Modificación tamaño de letra
\fontfamily{}\fontsize{12}{12}\selectfont}
\setlength{\headheight}{15pt} % Amplía el tamaño del índice


%\usepackage{lastpage}   % Referenciar última pag   \pageref{LastPage}
\fancyfoot[C]{\thepage}

%------------------------------------------------------------------------

% Conseguir que no ponga "Capítulo 1". Sino solo "1."
\makeatletter
\@ifclassloaded{book}{
  \renewcommand{\chaptermark}[1]{\markboth{\thechapter.\ #1}{}} % En el encabezado
    
  \renewcommand{\@makechapterhead}[1]{%
  \vspace*{50\p@}%
  {\parindent \z@ \raggedright \normalfont
    \ifnum \c@secnumdepth >\m@ne
      \huge\bfseries \thechapter.\hspace{1em}\ignorespaces
    \fi
    \interlinepenalty\@M
    \Huge \bfseries #1\par\nobreak
    \vskip 40\p@
  }}
}
\makeatother

%------------------------------------------------------------------------
% Paquetes de cógido
\usepackage{minted}
\renewcommand\listingscaption{Código fuente}

\usepackage{fancyvrb}
% Personaliza el tamaño de los números de línea
\renewcommand{\theFancyVerbLine}{\small\arabic{FancyVerbLine}}

% Estilo para C++
\newminted{cpp}{
    frame=lines,
    framesep=2mm,
    baselinestretch=1.2,
    linenos,
    escapeinside=||
}

% para minted
\definecolor{LightGray}{rgb}{0.95,0.95,0.92}
\setminted{
    linenos=true,
    stepnumber=5,
    numberfirstline=true,
    autogobble,
    breaklines=true,
    breakautoindent=true,
    breaksymbolleft=,
    breaksymbolright=,
    breaksymbolindentleft=0pt,
    breaksymbolindentright=0pt,
    breaksymbolsepleft=0pt,
    breaksymbolsepright=0pt,
    fontsize=\footnotesize,
    bgcolor=LightGray,
    numbersep=10pt
}


\usepackage{listings} % Para incluir código desde un archivo

\renewcommand\lstlistingname{Código Fuente}
\renewcommand\lstlistlistingname{Índice de Códigos Fuente}

% Definir colores
\definecolor{vscodepurple}{rgb}{0.5,0,0.5}
\definecolor{vscodeblue}{rgb}{0,0,0.8}
\definecolor{vscodegreen}{rgb}{0,0.5,0}
\definecolor{vscodegray}{rgb}{0.5,0.5,0.5}
\definecolor{vscodebackground}{rgb}{0.97,0.97,0.97}
\definecolor{vscodelightgray}{rgb}{0.9,0.9,0.9}

% Configuración para el estilo de C similar a VSCode
\lstdefinestyle{vscode_C}{
  backgroundcolor=\color{vscodebackground},
  commentstyle=\color{vscodegreen},
  keywordstyle=\color{vscodeblue},
  numberstyle=\tiny\color{vscodegray},
  stringstyle=\color{vscodepurple},
  basicstyle=\scriptsize\ttfamily,
  breakatwhitespace=false,
  breaklines=true,
  captionpos=b,
  keepspaces=true,
  numbers=left,
  numbersep=5pt,
  showspaces=false,
  showstringspaces=false,
  showtabs=false,
  tabsize=2,
  frame=tb,
  framerule=0pt,
  aboveskip=10pt,
  belowskip=10pt,
  xleftmargin=10pt,
  xrightmargin=10pt,
  framexleftmargin=10pt,
  framexrightmargin=10pt,
  framesep=0pt,
  rulecolor=\color{vscodelightgray},
  backgroundcolor=\color{vscodebackground},
}

%------------------------------------------------------------------------

% Comandos definidos
\newcommand{\bb}[1]{\mathbb{#1}}
\newcommand{\cc}[1]{\mathcal{#1}}

% I prefer the slanted \leq
\let\oldleq\leq % save them in case they're every wanted
\let\oldgeq\geq
\renewcommand{\leq}{\leqslant}
\renewcommand{\geq}{\geqslant}

% Si y solo si
\newcommand{\sii}{\iff}

% Letras griegas
\newcommand{\eps}{\epsilon}
\newcommand{\veps}{\varepsilon}
\newcommand{\lm}{\lambda}

\newcommand{\ol}{\overline}
\newcommand{\ul}{\underline}
\newcommand{\wt}{\widetilde}
\newcommand{\wh}{\widehat}

\let\oldvec\vec
\renewcommand{\vec}{\overrightarrow}

% Derivadas parciales
\newcommand{\del}[2]{\frac{\partial #1}{\partial #2}}
\newcommand{\Del}[3]{\frac{\partial^{#1} #2}{\partial #3^{#1}}}
\newcommand{\deld}[2]{\dfrac{\partial #1}{\partial #2}}
\newcommand{\Deld}[3]{\dfrac{\partial^{#1} #2}{\partial #3^{#1}}}


\newcommand{\AstIg}{\stackrel{(\ast)}{=}}
\newcommand{\Hop}{\stackrel{L'H\hat{o}pital}{=}}

\newcommand{\red}[1]{{\color{red}#1}} % Para integrales, destacar los cambios.

% Método de integración
\newcommand{\MetInt}[2]{
    \left[\begin{array}{c}
        #1 \\ #2
    \end{array}\right]
}

% Declarar aplicaciones
% 1. Nombre aplicación
% 2. Dominio
% 3. Codominio
% 4. Variable
% 5. Imagen de la variable
\newcommand{\Func}[5]{
    \begin{equation*}
        \begin{array}{rrll}
            #1:& #2 & \longrightarrow & #3\\
               & #4 & \longmapsto & #5
        \end{array}
    \end{equation*}
}

%------------------------------------------------------------------------

\usepgfplotslibrary{fillbetween}

\DeclareMathOperator{\Var}{Var}
\DeclareMathOperator{\Cov}{Cov}
\begin{document}

    % 1. Foto de fondo
    % 2. Título
    % 3. Encabezado Izquierdo
    % 4. Color de fondo
    % 5. Coord x del titulo
    % 6. Coord y del titulo
    % 7. Fecha

    
    % 1. Foto de fondo
% 2. Título
% 3. Encabezado Izquierdo
% 4. Color de fondo
% 5. Coord x del titulo
% 6. Coord y del titulo
% 7. Fecha

\newcommand{\portada}[7]{

    \portadaBase{#1}{#2}{#3}{#4}{#5}{#6}{#7}
    \portadaBook{#1}{#2}{#3}{#4}{#5}{#6}{#7}
}

\newcommand{\portadaExamen}[7]{

    \portadaBase{#1}{#2}{#3}{#4}{#5}{#6}{#7}
    \portadaArticle{#1}{#2}{#3}{#4}{#5}{#6}{#7}
}




\newcommand{\portadaBase}[7]{

    % Tiene la portada principal y la licencia Creative Commons
    
    % 1. Foto de fondo
    % 2. Título
    % 3. Encabezado Izquierdo
    % 4. Color de fondo
    % 5. Coord x del titulo
    % 6. Coord y del titulo
    % 7. Fecha
    
    
    \thispagestyle{empty}               % Sin encabezado ni pie de página
    \newgeometry{margin=0cm}        % Márgenes nulos para la primera página
    
    
    % Encabezado
    \fancyhead[L]{\helv #3}
    \fancyhead[R]{\helv \nouppercase{\leftmark}}
    
    
    \pagecolor{#4}        % Color de fondo para la portada
    
    \begin{figure}[p]
        \centering
        \transparent{0.3}           % Opacidad del 30% para la imagen
        
        \includegraphics[width=\paperwidth, keepaspectratio]{assets/#1}
    
        \begin{tikzpicture}[remember picture, overlay]
            \node[anchor=north west, text=white, opacity=1, font=\fontsize{60}{90}\selectfont\bfseries\sffamily, align=left] at (#5, #6) {#2};
            
            \node[anchor=south east, text=white, opacity=1, font=\fontsize{12}{18}\selectfont\sffamily, align=right] at (9.7, 3) {\textbf{\href{https://losdeldgiim.github.io/}{Los Del DGIIM}}};
            
            \node[anchor=south east, text=white, opacity=1, font=\fontsize{12}{15}\selectfont\sffamily, align=right] at (9.7, 1.8) {Doble Grado en Ingeniería Informática y Matemáticas\\Universidad de Granada};
        \end{tikzpicture}
    \end{figure}
    
    
    \restoregeometry        % Restaurar márgenes normales para las páginas subsiguientes
    \pagecolor{white}       % Restaurar el color de página
    
    
    \newpage
    \thispagestyle{empty}               % Sin encabezado ni pie de página
    \begin{tikzpicture}[remember picture, overlay]
        \node[anchor=south west, inner sep=3cm] at (current page.south west) {
            \begin{minipage}{0.5\paperwidth}
                \href{https://creativecommons.org/licenses/by-nc-nd/4.0/}{
                    \includegraphics[height=2cm]{assets/Licencia.png}
                }\vspace{1cm}\\
                Esta obra está bajo una
                \href{https://creativecommons.org/licenses/by-nc-nd/4.0/}{
                    Licencia Creative Commons Atribución-NoComercial-SinDerivadas 4.0 Internacional (CC BY-NC-ND 4.0).
                }\\
    
                Eres libre de compartir y redistribuir el contenido de esta obra en cualquier medio o formato, siempre y cuando des el crédito adecuado a los autores originales y no persigas fines comerciales. 
            \end{minipage}
        };
    \end{tikzpicture}
    
    
    
    % 1. Foto de fondo
    % 2. Título
    % 3. Encabezado Izquierdo
    % 4. Color de fondo
    % 5. Coord x del titulo
    % 6. Coord y del titulo
    % 7. Fecha


}


\newcommand{\portadaBook}[7]{

    % 1. Foto de fondo
    % 2. Título
    % 3. Encabezado Izquierdo
    % 4. Color de fondo
    % 5. Coord x del titulo
    % 6. Coord y del titulo
    % 7. Fecha

    % Personaliza el formato del título
    \pretitle{\begin{center}\bfseries\fontsize{42}{56}\selectfont}
    \posttitle{\par\end{center}\vspace{2em}}
    
    % Personaliza el formato del autor
    \preauthor{\begin{center}\Large}
    \postauthor{\par\end{center}\vfill}
    
    % Personaliza el formato de la fecha
    \predate{\begin{center}\huge}
    \postdate{\par\end{center}\vspace{2em}}
    
    \title{#2}
    \author{\href{https://losdeldgiim.github.io/}{Los Del DGIIM}}
    \date{Granada, #7}
    \maketitle
    
    \tableofcontents
}




\newcommand{\portadaArticle}[7]{

    % 1. Foto de fondo
    % 2. Título
    % 3. Encabezado Izquierdo
    % 4. Color de fondo
    % 5. Coord x del titulo
    % 6. Coord y del titulo
    % 7. Fecha

    % Personaliza el formato del título
    \pretitle{\begin{center}\bfseries\fontsize{42}{56}\selectfont}
    \posttitle{\par\end{center}\vspace{2em}}
    
    % Personaliza el formato del autor
    \preauthor{\begin{center}\Large}
    \postauthor{\par\end{center}\vspace{3em}}
    
    % Personaliza el formato de la fecha
    \predate{\begin{center}\huge}
    \postdate{\par\end{center}\vspace{5em}}
    
    \title{#2}
    \author{\href{https://losdeldgiim.github.io/}{Los Del DGIIM}}
    \date{Granada, #7}
    \thispagestyle{empty}               % Sin encabezado ni pie de página
    \maketitle
    \vfill
}
    \portadaExamen{ffccA4.jpg}{Probabilidad\\Examen III}{Probabilidad. Examen I}{MidnightBlue}{-8}{28}{2024-2025}{Arturo Olivares Martos \\ Miguel Ángel De la Vega Rodríguez}

    \begin{description}
        \item[Asignatura] Probabilidad.
        \item[Curso Académico] 2022-23.
        \item[Grado] Doble Grado en Ingeniería Informática y Matemáticas.
        \item[Grupo] Único.
        %\item[Profesor] José María Espinar García.
        \item[Descripción] Examen Ordinario 
        \item[Fecha] 11 de enero de 2023.
        % \item[Duración] 60 minutos.
    
    \end{description}
    \newpage

\begin{ejercicio}[1.5 puntos]
        Justificar las siguientes relaciones:
        \begin{enumerate}
            \item (0.25 puntos) Sean $X_1$ y $X_2$ variables aleatorias independientes e idénticamente distribuidas según una ley Binomial $B(3, \nicefrac{1}{2})$. Justificar que: $$P[X_1 + X_2 = 8] =~0$$
            
            Tenemos que:
            \begin{equation*}
                X_1, X_2 \sim B(3, \nicefrac{1}{2}) 
            \end{equation*}

            Por la reproductividad de la distribución binomial, tenemos que:
            \begin{equation*}
                X_1 + X_2 \sim B(6, \nicefrac{1}{2})
            \end{equation*}

            Por tanto, tenemos que:
            \begin{equation*}
                \sum_{k=0}^6 P[X_1 + X_2 = k] = 1
            \end{equation*}

            Por lo que:
            \begin{equation*}
                P[X_1 + X_2 = 8] = 0
            \end{equation*}
            \item (0.25 puntos) Sean $X_1$, $X_2$ y $X_3$ variables aleatorias independientes e idénticamente distribuidas según una ley de Poisson, $\cc{P}(3)$. Justificar que: $$P[X_1 + X_2 + X_3 > 0] = \frac{e^9 - 1}{e^9}$$
            
            Tenemos que:
            \begin{equation*}
                X_1, X_2, X_3 \sim \cc{P}(3)
            \end{equation*}

            Por la reproductividad de la distribución de Poisson, tenemos que:
            \begin{equation*}
                X_1 + X_2 + X_3 \sim \cc{P}(9)
            \end{equation*}

            Por tanto, tenemos que:
            \begin{equation*}
                P[X_1 + X_2 + X_3 > 0] = 1 - P[X_1 + X_2 + X_3 = 0] = 1 - \frac{e^{-9}9^0}{0!} = 1 - e^{-9} = \frac{e^9 - 1}{e^9}
            \end{equation*}

            \item (1 punto) Sea $\{X_n\}_{n \in \mathbb{N}}$ una sucesión de variables aleatorias independientes e idénticamente distribuidas según una ley de Bernoulli de parámetro $p$. Se considera $S_n = \sum\limits_{i=1}^n X_i$, la variable aleatoria que define las sumas parciales de la sucesión. Justificar que: $$\frac{S_n}{n} \xrightarrow{P} p$$
            
            Sabemos que:
            \begin{equation*}
                E[X_n] = p\qquad \forall n \in \mathbb{N}
            \end{equation*}

            Por tanto, tenemos que:
            \begin{equation*}
                E[S_n] = E\left[\sum_{i=1}^n X_i\right] = \sum_{i=1}^n E[X_i] = np\qquad \forall n \in \mathbb{N}
            \end{equation*}
            
            Por la Ley Débil de los Grandes Números, tenemos que:
            \begin{equation*}
                \frac{S_n}{n}-p = \frac{S_n-np}{n} \xrightarrow{P} 0
            \end{equation*}

            Por tanto, tenemos que:
            \begin{equation*}
                \frac{S_n}{n} \xrightarrow{P} p
            \end{equation*}
        \end{enumerate}
    \end{ejercicio}

    \begin{ejercicio}[1.5 puntos]
        Sean $X_1, \dots, X_n$ $n$ variables aleatorias continuas, independientes e idénticamente distribuidas con distribución uniforme en el intervalo $[0,1]$. Deducir la expresión analítica de la función de densidad de la variable aleatoria $Z = \max\{X_1, \dots, X_n\}$.\\

        Tenemos que:
        \begin{equation*}
            X_1, \dots, X_n \sim \cc{U}([0,1])
        \end{equation*}

        Por tanto, tenemos que:
        \begin{equation*}
            P[Z \leq z] = P[X_1 \leq z, \dots, X_n \leq z] \AstIg P[X_1 \leq z] \cdot \ldots \cdot P[X_n \leq z]
            \stackrel{(\ast\ast)}{=}
            \left( P[X_1 \leq z] \right)^n
        \end{equation*}
        donde en $(\ast)$ hemos usado la independencia de las variables aleatorias y en $(\ast\ast)$ hemos usado que son idénticamente distribuidas. Por tanto, tenemos que:
        \begin{align*}
            f_Z(z) &= \dfrac{d}{dz} P[Z \leq z] = \dfrac{d}{dz} \left( P[X_1 \leq z] \right)^n = n \left( P[X_1 \leq z] \right)^{n-1} \cdot \dfrac{d}{dz} P[X_1 \leq z] =\\&= n \left( P[X_1 \leq z] \right)^{n-1} \cdot f_{X_1}(z)\qquad \forall z \in \bb{R}
        \end{align*}

        Por ser $X_1$ uniformemente distibuida, tenemos que:
        \begin{equation*}
            f_{X_1}(z) = \begin{cases}
                1 & \text{si } z \in [0,1]\\
                0 & \text{en otro caso}
            \end{cases}
        \end{equation*}

        Además, tenemos que:
        \begin{equation*}
            P[X_1 \leq z] = \begin{cases}
                0 & \text{si } z < 0\\
                \int_0^z 1 \, dx = z & \text{si } z \in [0,1]\\
                1 & \text{si } z > 1
            \end{cases}
        \end{equation*}

        Por tanto, tenemos que:
        \begin{equation*}
            f_Z(z) = \begin{cases}
                n \cdot z^{n-1} & \text{si } z \in [0,1]\\
                0 & \text{en otro caso}
            \end{cases}
        \end{equation*}
    \end{ejercicio}

    \begin{ejercicio}[5 puntos]
        Dado el vector aleatorio $(X, Y)$ con distribución uniforme en el recinto acotado limitado por el exterior de la parábola $y = x^2$, la recta de ecuación $2y + x = 1$ y la recta de ecuación $y = 0$:
        \begin{enumerate}
            \item (0.25 puntos) Obtener la función de densidad conjunta.
            
            Buscamos en primer lugar entender el conjunto descrito, el cual se muestra en la Figura~\ref{fig:conjunto}.
            \begin{figure}
                \centering
                \begin{tikzpicture}
                    \begin{axis}[
                        axis lines = middle,
                        xlabel = {$x$},
                        ylabel = {$y$},
                        xmin = -0.5, xmax = 1.5,
                        ymin = -0.25, ymax = 0.5,
                        xtick = {-1, -0.5, 0, 0.5, 1},
                        ytick = {0, 0.25, 0.5, 1, 1.5},
                        legend pos=south east,
                        axis equal,
                    ]
                    \addplot[name path=A, domain=-1:1, samples=100, color=red]{x^2};
                    \addlegendentry{$y = x^2$}
                    \addplot[name path=B, domain=-1:3, samples=2, color=blue]{(1-x)/2};
                    \addlegendentry{$2y + x = 1$}
                    \addplot[name path=C, domain=-1:3, samples=2, draw=none, forget plot]{0};
                    \addplot[name path=D, domain=-1:3, samples=2, draw=none, forget plot]{1};
                    \addplot [
                        thick,
                        color=orange,
                        fill=orange,
                        fill opacity=0.4
                    ]
                    fill between [
                        of=A and C,
                        soft clip={domain=0:0.5},
                    ];
                    \addplot [
                        thick,
                        color=orange,
                        fill=orange,
                        fill opacity=0.4
                    ]
                    fill between [
                        of=B and C,
                        soft clip={domain=0.5:1},
                    ];


                    % Triángulo (0,0) (0,5) (-2,5) (-2,-2) (5,-2) (5,0)
                    \addplot[fill=blue, fill opacity=0.2, draw=none] coordinates {(0,0) (0,5) (-2,5) (-2,-2) (5,-2) (5,0)};
                    \node at (axis cs:-0.25,-0.5) {$R_0$};

                    \addplot [
                        thick,
                        color=teal,
                        fill=teal,
                        fill opacity=0.2
                    ]
                    fill between [
                        of=D and A,
                        soft clip={domain=0:0.5},
                    ];
                    \node at (axis cs:0.25,0.5) {$R_1$};

                    % R2: Triángulo (0.5,0.25) (1,0.25) (1,0)
                    \addplot[fill=green, fill opacity=0.4, draw=none] coordinates {(0.5,0.25) (1,0.25) (1,0)};
                    \node at (axis cs:0.81,0.17) {$R_2$};

                    % R3: Triángulo (1,0) (1,0.25) (3,0.25) (3,0)
                    \addplot[fill=yellow, fill opacity=0.4, draw=none] coordinates {(1,0) (1,0.25) (3,0.25) (3,0)};
                    \node at (axis cs:1.25,0.15) {$R_3$};

                    % R4: Triángulo (1,0.25) (1,2) (0.5,2)
                    \addplot[fill=purple, fill opacity=0.4, draw=none] coordinates {(1,0.25) (1,2) (0.5,2) (0.5,0.25)};
                    \node at (axis cs:0.75,0.5) {$R_4$};


                    % Triángulo (1,0.25) (1,5) (5,5) (5,0.25)
                    \addplot[fill=red, fill opacity=0.2, draw=none] coordinates {(1,0.25) (1,5) (5,5) (5,0.25)};
                    \node at (axis cs:1.25,0.5) {$R_5$};

                    \addlegendentry{$\Omega$}
                    \end{axis}
                \end{tikzpicture}
                \caption{Conjunto descrito por las ecuaciones dadas.}
                \label{fig:conjunto}
            \end{figure}

            Sea $\Omega$ el conjunto descrito por las ecuaciones dadas, donde el punto de corte dibujado es:
            \begin{equation*}
                x^2=\dfrac{1-x}{2}\Rightarrow 2x^2+x-1=0\Rightarrow x=\dfrac{-1\pm\sqrt{1+8}}{4}=\dfrac{-1\pm3}{4}\Longrightarrow
                \left\{\begin{array}{l}
                    \cancel{x_1=-1}\\
                    x_2=\dfrac{1}{2}
                \end{array}
                \right\}\Longrightarrow \left(\frac{1}{2},\frac{1}{4}\right)
            \end{equation*}
            
            
            Tenemos que:
            \begin{align*}
                1&=\int_{\bb{R}^2} f_{X,Y}
                = \int_{\Omega} k
                = k\int_{0}^{\nicefrac{1}{4}}\int_{\sqrt{y}}^{1-2y}dx\,dy
                = k\int_{0}^{\nicefrac{1}{4}}1-2y-\sqrt{y}dy
                = k\left[y-y^2-\dfrac{2}{3}y\sqrt{y}\right]_{0}^{\nicefrac{1}{4}}\\
                &= k\left(\dfrac{1}{4}-\dfrac{1}{16}-\dfrac{2}{3}\cdot \dfrac{1}{4}\cdot \dfrac{1}{2}\right)
                = k\cdot \dfrac{5}{48}
                \Longrightarrow
                k=\dfrac{48}{5}
            \end{align*}

            Por tanto, tenemos que:
            \begin{equation*}
                f_{X,Y}(x,y) = \begin{cases}
                    \dfrac{48}{5} & \text{si } (x,y) \in \Omega\\
                    0 & \text{en otro caso}
                \end{cases}
            \end{equation*}
            \item (1.50 puntos) Obtener la función de distribución de probabilidad conjunta.
            
            La división en conjuntos se encuentra en la Figura~\ref{fig:conjunto}. Calculamos el valor de la función de distribución en cada uno de los conjuntos:
            \begin{itemize}
                \item \ul{Si $(x,y) \in R_0$} ($x<0$ o $y<0$):
                \begin{align*}
                    F_{X,Y}(x,y) &= \int_{-\infty}^x\int_{-\infty}^y f_{X,Y}(u,v)du\,dv
                    = \int_{0}^x\int_{0}^y 0\,du\,dv = 0
                \end{align*}

                \item \ul{Si $(x,y) \in \Omega$} ($2y+x<1$ y $0\leq y\leq x^2$):
                \begin{align*}
                    F_{X,Y}(x,y) &= \int_{-\infty}^x\int_{-\infty}^y f_{X,Y}(u,v)du\,dv
                    = \int_{0}^y \int_{\sqrt{v}}^x k du\,dv
                    = k\int_{0}^y x-\sqrt{v}dv
                    =\\&= k\left[xv-\dfrac{2}{3}v\sqrt{v}\right]_{0}^y
                    = k\left[xy-\dfrac{2}{3}y\sqrt{y}\right]
                \end{align*}

                \item \ul{Si $(x,y) \in R_1$} ($0<x<\nicefrac{1}{2}$ y $y<x^2$):
                \begin{align*}
                    F_{X,Y}(x,y) &= \int_{-\infty}^x\int_{-\infty}^y f_{X,Y}(u,v)du\,dv
                    = \int_{0}^x\int_0^{u^2} k\,dv\,du
                    = k\int_{0}^x u^2\,du
                    = k\left[\dfrac{u^3}{3}\right]_{0}^x
                    =\\&= k\cdot \dfrac{x^3}{3}
                \end{align*}

                \item \ul{Si $(x,y) \in R_2$} ($0<y<\nicefrac{1}{4}$ y $2y+x>1$):
                \begin{align*}
                    F_{X,Y}(x,y) &= \int_{-\infty}^x\int_{-\infty}^y f_{X,Y}(u,v)du\,dv
                    =\\&= \int_0^{\sqrt{y}}\int_{0}^{u^2} k\,dv\,du
                    + \int_{\sqrt{y}}^{1-2y}\int_{0}^{y} k\,dv\,du
                    + \int_{1-2y}^x\int_{0}^{\frac{1-u}{2}} k\,dv\,du
                    =\\&= k\int_0^{\sqrt{y}}u^2\,du
                    + k\int_{\sqrt{y}}^{1-2y}y\,du
                    + k\int_{1-2y}^x \dfrac{1-u}{2}\,du
                    =\\&= k\left[\dfrac{u^3}{3}\right]_{0}^{\sqrt{y}}
                    + k\left[yu\right]_{\sqrt{y}}^{1-2y}
                    + k\left[\dfrac{u}{2}-\dfrac{u^2}{4}\right]_{1-2y}^x
                    =\\&= k\left[\dfrac{y\sqrt{y}}{3} + y(1-2y)-y\sqrt{y}+ \dfrac{x}{2}-\dfrac{x^2}{4}-\dfrac{1-2y}{2}+\dfrac{(1-2y)^2}{4}\right]
                \end{align*}

                \item \ul{Si $(x,y) \in R_3$} ($1\leq x$ y $0<y<\nicefrac{1}{4}$):
                \begin{align*}
                    F_{X,Y}(x,y) &= \int_{-\infty}^x\int_{-\infty}^y f_{X,Y}(u,v)du\,dv
                    =\\&= \int_0^{y}\int_{\sqrt{v}}^{1-2v} k\,du\,dv
                    = k\int_0^{y}1-2v-\sqrt{v}\,dv
                    = k\left[v-v^2-\dfrac{2}{3}v\sqrt{v}\right]_{0}^{y}
                    =\\&= k\left[y-y^2-\dfrac{2}{3}y\sqrt{y}\right]
                \end{align*}

                \item \ul{Si $(x,y) \in R_4$} ($\nicefrac{1}{2}<x<1$ y $\nicefrac{1}{4}<y$):
                \begin{align*}
                    F_{X,Y}(x,y) &= \int_{-\infty}^x\int_{-\infty}^y f_{X,Y}(u,v)du\,dv
                    =\\&= \int_0^{\nicefrac{1}{2}}\int_{0}^{u^2} k\,dv\,du
                    + \int_{\nicefrac{1}{2}}^{x}\int_{0}^{\frac{1-u}{2}} k\,dv\,du
                    = k\int_0^{\nicefrac{1}{2}}u^2\,du
                    + k\int_{\nicefrac{1}{2}}^{x} \dfrac{1-u}{2}\,du
                    =\\&= k\left[\dfrac{u^3}{3}\right]_{0}^{\nicefrac{1}{2}}
                    + k\left[\dfrac{u}{2}-\dfrac{u^2}{4}\right]_{\nicefrac{1}{2}}^x
                    = k\left[\dfrac{1}{24}+\dfrac{x}{2}-\dfrac{x^2}{4}-\dfrac{1}{4}+\dfrac{1}{16}\right]
                    =\\&= k\left[\dfrac{x}{2}-\dfrac{x^2}{4}-\dfrac{7}{48}\right]
                \end{align*}

                \item \ul{Si $(x,y) \in R_5$} ($1\leq x$ y $\nicefrac{1}{4}<y$):
                \begin{align*}
                    F_{X,Y}(x,y) &= \int_{-\infty}^x\int_{-\infty}^y f_{X,Y}(u,v)du\,dv
                    = 1
                \end{align*}
            \end{itemize}

            Por tanto, tenemos que:
            \begin{equation*}
                F_{X,Y}(x,y) = \begin{cases}
                    0 & \text{si } (x,y) \in R_0\\
                    k\cdot \dfrac{x^3}{3} & \text{si } (x,y) \in R_1\\
                    k\left[-\dfrac{2y\sqrt{y}}{3} + y(1-2y)- \dfrac{x}{2}-\dfrac{x^2}{4}-\dfrac{1-2y}{2}+\dfrac{(1-2y)^2}{4}\right] & \text{si } (x,y) \in R_2\\
                    k\left[y-y^2-\dfrac{2}{3}y\sqrt{y}\right] & \text{si } (x,y) \in R_3\\
                    k\left[\dfrac{x}{2}-\dfrac{x^2}{4}-\dfrac{7}{48}\right] & \text{si } (x,y) \in R_4\\
                    1 & \text{si } (x,y) \in R_5
                \end{cases}
            \end{equation*}
            \item (0.75 puntos) Obtener las funciones de densidad condicionadas.\\
            
            Calculamos en primer lugar las funciones de densidad marginales. Respecto a la marginal de $X$, tenemos que:
            \begin{itemize}
                \item \ul{Si $0<x<\nicefrac{1}{2}$}:
                \begin{align*}
                    f_X(x) &= \int_{-\infty}^{\infty} f_{X,Y}(x,y)dy
                    = \int_{0}^{x^2} k\,dy
                    = kx^2
                \end{align*}

                \item \ul{Si $\nicefrac{1}{2}<x<1$}:
                \begin{align*}
                    f_X(x) &= \int_{-\infty}^{\infty} f_{X,Y}(x,y)dy
                    = \int_{0}^{\frac{1-x}{2}} k\,dy
                    = k\left[\dfrac{1-x}{2}\right]
                \end{align*}
            \end{itemize}

            Por tanto, tenemos que:
            \begin{equation*}
                f_X(x) = \begin{cases}
                    kx^2 & \text{si } 0<x<\nicefrac{1}{2}\\
                    k\left[\dfrac{1-x}{2}\right] & \text{si } \nicefrac{1}{2}<x<1
                \end{cases}
            \end{equation*}

            Respecto a la marginal de $Y$, tenemos que, si $0<y<\nicefrac{1}{4}$:
            \begin{align*}
                f_Y(y) &= \int_{-\infty}^{\infty} f_{X,Y}(x,y)dx
                = \int_{\sqrt{y}}^{1-2y} k\,dx
                = k(1-2y-\sqrt{y})
            \end{align*}

            Por tanto, tenemos que:
            \begin{equation*}
                f_Y(y) = \begin{cases}
                    k(1-2y-\sqrt{y}) & \text{si } 0<y<\nicefrac{1}{4}\\
                    0 & \text{en otro caso}
                \end{cases}
            \end{equation*}

            Calculamos ahora las funciones de densidad condicionadas. Respecto a la condicionada de $X$ dado $y^*\in \left[0,\nicefrac{1}{4}\right]$, tenemos que:
            \begin{align*}
                f_{X\mid Y=y^*}(x) &= \dfrac{f_{X,Y}(x,y^*)}{f_Y(y^*)}= \dfrac{k}{k(1-2y^*-\sqrt{y^*})} = \dfrac{1}{1-2y^*-\sqrt{y^*}}
                \qquad \forall x\in \bb{R}\mid (x,y^*)\in \Omega
            \end{align*}

            Respecto a la condicionada de $Y$, distinguimos en función del valor de $x^*\in \left[0,1\right]$.
            \begin{itemize}
                \item \ul{Si $0<x^*<\nicefrac{1}{2}$}:
                \begin{align*}
                    f_{Y\mid X=x^*}(y) &= \dfrac{f_{X,Y}(x^*,y)}{f_X(x^*)}
                    = \dfrac{k}{kx^2} = \dfrac{1}{x^2}
                \end{align*}
                \item \ul{Si $\nicefrac{1}{2}<x^*<1$}:
                \begin{align*}
                    f_{Y\mid X=x^*}(y) &= \dfrac{f_{X,Y}(x^*,y)}{f_X(x^*)}
                    = \dfrac{k}{k\left[\dfrac{1-x^*}{2}\right]} = \dfrac{2}{1-x^*}
                \end{align*}
            \end{itemize}

            Por tanto, tenemos que:
            \begin{equation*}
                f_{Y\mid X=x^*}(y) = \begin{cases}
                    \dfrac{1}{x^2} & \text{si } 0<x^*<\nicefrac{1}{2}\\
                    \dfrac{2}{1-x^*} & \text{si } \nicefrac{1}{2}<x^*<1
                \end{cases}
                \qquad \forall y \in \bb{R}
            \end{equation*}

            \item (0.50 puntos) Obtener la probabilidad de que $X \geq \frac{1}{2}$.
            
            Del apartado anterior, tenemos la marginal de $X$. Por tanto:
            \begin{align*}
                P[X \geq \nicefrac{1}{2}] &= \int_{\nicefrac{1}{2}}^{1} f_X(x)dx
                = \int_{\nicefrac{1}{2}}^{1} k\left[\dfrac{1-x}{2}\right]dx
                = k\left[\dfrac{x}{2}-\dfrac{x^2}{4}\right]_{\nicefrac{1}{2}}^{1}
                =\\&= k\left[\dfrac{1}{2}-\dfrac{1}{4}-\dfrac{1}{4}+\dfrac{1}{16}\right]
                = \dfrac{k}{16}
            \end{align*}
            \item (1.25 puntos) Obtener la mejor aproximación mínimo cuadrática a la variable $X$ conocidos los valores de la variable $Y$.\\
            
            Del apartado anterior, tenemos que, para $y^*\in \left[0,\nicefrac{1}{4}\right]$:
            \begin{align*}
                E[X\mid Y=y^*] &= \int_{-\infty}^{\infty} x\cdot f_{X\mid Y=y^*}(x)dx
                = \int_{\sqrt{y^*}}^{1-2y^*} x\cdot \dfrac{1}{1-2y^*-\sqrt{y^*}}dx
                =\\&= \left[\dfrac{x^2}{2\left(1-2y^*-\sqrt{y^*}\right)}\right]_{\sqrt{y^*}}^{1-2y^*}
                = \dfrac{\left(1-2y^*\right)^2-\left(\sqrt{y^*}\right)^2}{2\left(1-2y^*-\sqrt{y^*}\right)}
                =\\&= \dfrac{(1-2y^*+\sqrt{y^*})\cancel{(1-2y^*-\sqrt{y^*})}}{2\cancel{(1-2y^*-\sqrt{y^*})}}
                = \dfrac{1-2y^*+\sqrt{y^*}}{2}
            \end{align*}

            Por tanto, la mejor aproximación mínimo cuadrática a la variable $X$ conocidos los valores de la variable $Y$ es:
            \begin{equation*}
                E[X\mid Y] = \dfrac{1-2Y+\sqrt{Y}}{2}
            \end{equation*}
            \item (0.50 puntos) Obtener la mejor aproximación de la variable aleatoria $Y$ sin observar la variable $X$ y dar una medida del error cuadrático medio cometido en esta aproximación.\\
            
            La mejor aproximación es su esperanza, y su error cuadrático medio es su varianza:
            \begin{align*}
                E[Y] &= \int_{-\infty}^{\infty} y\cdot f_Y(y)dy = \int_{0}^{\nicefrac{1}{4}} y\cdot k(1-2y-\sqrt{y})dy
                = k\left[\dfrac{y^2}{2}-\dfrac{2y^3}{3}-\dfrac{2y^2\sqrt{y}}{5}\right]_{0}^{\nicefrac{1}{4}} =\\&= \dfrac{k}{120} = \dfrac{2}{25} = 0.08\\
                E[Y^2] &= \int_{-\infty}^{\infty} y^2\cdot f_Y(y)dy = \int_{0}^{\nicefrac{1}{4}} y^2\cdot k(1-2y-\sqrt{y})dy
                = k\left[\dfrac{y^3}{3}-\dfrac{2y^4}{4}-\dfrac{2y^3\sqrt{y}}{7}\right]_{0}^{\nicefrac{1}{4}} =\\&= \dfrac{11k}{10752} = \dfrac{11}{1120}\\
                \Var[Y] &= E[Y^2] - E[Y]^2 = \dfrac{11}{1120} - \left(\dfrac{2}{25}\right)^2 = \dfrac{11}{1120} - \dfrac{4}{25^2} = \dfrac{479}{14\cdot 10^4} \approx 0.0034
            \end{align*}

            Por tanto, la mejor aproximación de la variable aleatoria $Y$ sin observar la variable $X$ es $0.08$ y el error cuadrático medio cometido en esta aproximación es de, aproximadamente, $0.0034$.
            \item (0.25 puntos) ¿Son $X$ e $Y$ variables aleatorias independientes? Justificar de forma muy breve la respuesta.
        \end{enumerate}
    \end{ejercicio}

    \begin{ejercicio}[2 puntos]
        Sea $(X, Y)$ un vector aleatorio con distribución normal bidimensional. La moda de $Y$ vale $4$ y $\mathrm{Var}[Y \mid X = x_0] = \frac{\mathrm{Var}(Y)}{2} \neq 0$. La curva de regresión de $Y/X$ es $y = x + 5$ y el error cuadrático medio asociado a esta aproximación es $3$.
        \begin{enumerate}
            \item (1.25 puntos) Determinar los parámetros de la distribución de $(X, Y)$.
            \item (0.75 puntos) Especificar la función generatriz de momentos de $(X, Y)$.
        \end{enumerate}
    \end{ejercicio}

    \begin{observacion}
        En el apartado $1$(c) hay que demostrar el/los resultados empleados para justificar la relación pedida, salvo la Desigualdad de Chebychev.
        En el apartado $3$(b) se obtienen hasta $1.25$ puntos si las integrales se dejan indicadas y hasta $1.50$ puntos si se obtienen sus valores de forma explícita.
    \end{observacion}

\end{document}
