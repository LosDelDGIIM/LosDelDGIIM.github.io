\documentclass[12pt]{article}

% Idioma y codificación
\usepackage[spanish, es-tabla]{babel}       %es-tabla para que se titule "Tabla"
\usepackage[utf8]{inputenc}

% Márgenes
\usepackage[a4paper,top=3cm,bottom=2.5cm,left=3cm,right=3cm]{geometry}

% Comentarios de bloque
\usepackage{verbatim}

% Paquetes de links
\usepackage[hidelinks]{hyperref}    % Permite enlaces
\usepackage{url}                    % redirecciona a la web

% Más opciones para enumeraciones
\usepackage{enumitem}

% Personalizar la portada
\usepackage{titling}

% Paquetes de tablas
\usepackage{multirow}


%------------------------------------------------------------------------

%Paquetes de figuras
\usepackage{caption}
\usepackage{subcaption} % Figuras al lado de otras
\usepackage{float}      % Poner figuras en el sitio indicado H.


% Paquetes de imágenes
\usepackage{graphicx}       % Paquete para añadir imágenes
\usepackage{transparent}    % Para manejar la opacidad de las figuras

% Paquete para usar colores
\usepackage[dvipsnames]{xcolor}
\usepackage{pagecolor}      % Para cambiar el color de la página

% Habilita tamaños de fuente mayores
\usepackage{fix-cm}

% Para los gráficos
\usepackage{tikz}

% Para poder situar los nodos en los grafos
\usetikzlibrary{positioning}


%------------------------------------------------------------------------

% Paquetes de matemáticas
\usepackage{mathtools, amsfonts, amssymb, mathrsfs}
\usepackage[makeroom]{cancel}     % Simplificar tachando
\usepackage{polynom}    % Divisiones y Ruffini
\usepackage{units} % Para poner fracciones diagonales con \nicefrac

\usepackage{pgfplots}   %Representar funciones
\pgfplotsset{compat=1.18}  % Versión 1.18

\usepackage{tikz-cd}    % Para usar diagramas de composiciones
\usetikzlibrary{calc}   % Para usar cálculo de coordenadas en tikz

%Definición de teoremas, etc.
\usepackage{amsthm}
%\swapnumbers   % Intercambia la posición del texto y de la numeración

\theoremstyle{plain}

\makeatletter
\@ifclassloaded{article}{
  \newtheorem{teo}{Teorema}[section]
}{
  \newtheorem{teo}{Teorema}[chapter]  % Se resetea en cada chapter
}
\makeatother

\newtheorem{coro}{Corolario}[teo]           % Se resetea en cada teorema
\newtheorem{prop}[teo]{Proposición}         % Usa el mismo contador que teorema
\newtheorem{lema}[teo]{Lema}                % Usa el mismo contador que teorema

\theoremstyle{remark}
\newtheorem*{observacion}{Observación}

\theoremstyle{definition}

\makeatletter
\@ifclassloaded{article}{
  \newtheorem{definicion}{Definición} [section]     % Se resetea en cada chapter
}{
  \newtheorem{definicion}{Definición} [chapter]     % Se resetea en cada chapter
}
\makeatother

\newtheorem*{notacion}{Notación}
\newtheorem*{ejemplo}{Ejemplo}
\newtheorem*{ejercicio*}{Ejercicio}             % No numerado
\newtheorem{ejercicio}{Ejercicio} [section]     % Se resetea en cada section


% Modificar el formato de la numeración del teorema "ejercicio"
\renewcommand{\theejercicio}{%
  \ifnum\value{section}=0 % Si no se ha iniciado ninguna sección
    \arabic{ejercicio}% Solo mostrar el número de ejercicio
  \else
    \thesection.\arabic{ejercicio}% Mostrar número de sección y número de ejercicio
  \fi
}


% \renewcommand\qedsymbol{$\blacksquare$}         % Cambiar símbolo QED
%------------------------------------------------------------------------

% Paquetes para encabezados
\usepackage{fancyhdr}
\pagestyle{fancy}
\fancyhf{}

\newcommand{\helv}{ % Modificación tamaño de letra
\fontfamily{}\fontsize{12}{12}\selectfont}
\setlength{\headheight}{15pt} % Amplía el tamaño del índice


%\usepackage{lastpage}   % Referenciar última pag   \pageref{LastPage}
\fancyfoot[C]{\thepage}

%------------------------------------------------------------------------

% Conseguir que no ponga "Capítulo 1". Sino solo "1."
\makeatletter
\@ifclassloaded{book}{
  \renewcommand{\chaptermark}[1]{\markboth{\thechapter.\ #1}{}} % En el encabezado
    
  \renewcommand{\@makechapterhead}[1]{%
  \vspace*{50\p@}%
  {\parindent \z@ \raggedright \normalfont
    \ifnum \c@secnumdepth >\m@ne
      \huge\bfseries \thechapter.\hspace{1em}\ignorespaces
    \fi
    \interlinepenalty\@M
    \Huge \bfseries #1\par\nobreak
    \vskip 40\p@
  }}
}
\makeatother

%------------------------------------------------------------------------
% Paquetes de cógido
\usepackage{minted}
\renewcommand\listingscaption{Código fuente}

\usepackage{fancyvrb}
% Personaliza el tamaño de los números de línea
\renewcommand{\theFancyVerbLine}{\small\arabic{FancyVerbLine}}

% Estilo para C++
\newminted{cpp}{
    frame=lines,
    framesep=2mm,
    baselinestretch=1.2,
    linenos,
    escapeinside=||
}

% para minted
\definecolor{LightGray}{rgb}{0.95,0.95,0.92}
\setminted{
    linenos=true,
    stepnumber=5,
    numberfirstline=true,
    autogobble,
    breaklines=true,
    breakautoindent=true,
    breaksymbolleft=,
    breaksymbolright=,
    breaksymbolindentleft=0pt,
    breaksymbolindentright=0pt,
    breaksymbolsepleft=0pt,
    breaksymbolsepright=0pt,
    fontsize=\footnotesize,
    bgcolor=LightGray,
    numbersep=10pt
}


\usepackage{listings} % Para incluir código desde un archivo

\renewcommand\lstlistingname{Código Fuente}
\renewcommand\lstlistlistingname{Índice de Códigos Fuente}

% Definir colores
\definecolor{vscodepurple}{rgb}{0.5,0,0.5}
\definecolor{vscodeblue}{rgb}{0,0,0.8}
\definecolor{vscodegreen}{rgb}{0,0.5,0}
\definecolor{vscodegray}{rgb}{0.5,0.5,0.5}
\definecolor{vscodebackground}{rgb}{0.97,0.97,0.97}
\definecolor{vscodelightgray}{rgb}{0.9,0.9,0.9}

% Configuración para el estilo de C similar a VSCode
\lstdefinestyle{vscode_C}{
  backgroundcolor=\color{vscodebackground},
  commentstyle=\color{vscodegreen},
  keywordstyle=\color{vscodeblue},
  numberstyle=\tiny\color{vscodegray},
  stringstyle=\color{vscodepurple},
  basicstyle=\scriptsize\ttfamily,
  breakatwhitespace=false,
  breaklines=true,
  captionpos=b,
  keepspaces=true,
  numbers=left,
  numbersep=5pt,
  showspaces=false,
  showstringspaces=false,
  showtabs=false,
  tabsize=2,
  frame=tb,
  framerule=0pt,
  aboveskip=10pt,
  belowskip=10pt,
  xleftmargin=10pt,
  xrightmargin=10pt,
  framexleftmargin=10pt,
  framexrightmargin=10pt,
  framesep=0pt,
  rulecolor=\color{vscodelightgray},
  backgroundcolor=\color{vscodebackground},
}

%------------------------------------------------------------------------

% Comandos definidos
\newcommand{\bb}[1]{\mathbb{#1}}
\newcommand{\cc}[1]{\mathcal{#1}}

% I prefer the slanted \leq
\let\oldleq\leq % save them in case they're every wanted
\let\oldgeq\geq
\renewcommand{\leq}{\leqslant}
\renewcommand{\geq}{\geqslant}

% Si y solo si
\newcommand{\sii}{\iff}

% Letras griegas
\newcommand{\eps}{\epsilon}
\newcommand{\veps}{\varepsilon}
\newcommand{\lm}{\lambda}

\newcommand{\ol}{\overline}
\newcommand{\ul}{\underline}
\newcommand{\wt}{\widetilde}
\newcommand{\wh}{\widehat}

\let\oldvec\vec
\renewcommand{\vec}{\overrightarrow}

% Derivadas parciales
\newcommand{\del}[2]{\frac{\partial #1}{\partial #2}}
\newcommand{\Del}[3]{\frac{\partial^{#1} #2}{\partial #3^{#1}}}
\newcommand{\deld}[2]{\dfrac{\partial #1}{\partial #2}}
\newcommand{\Deld}[3]{\dfrac{\partial^{#1} #2}{\partial #3^{#1}}}


\newcommand{\AstIg}{\stackrel{(\ast)}{=}}
\newcommand{\Hop}{\stackrel{L'H\hat{o}pital}{=}}

\newcommand{\red}[1]{{\color{red}#1}} % Para integrales, destacar los cambios.

% Método de integración
\newcommand{\MetInt}[2]{
    \left[\begin{array}{c}
        #1 \\ #2
    \end{array}\right]
}

% Declarar aplicaciones
% 1. Nombre aplicación
% 2. Dominio
% 3. Codominio
% 4. Variable
% 5. Imagen de la variable
\newcommand{\Func}[5]{
    \begin{equation*}
        \begin{array}{rrll}
            #1:& #2 & \longrightarrow & #3\\
               & #4 & \longmapsto & #5
        \end{array}
    \end{equation*}
}

%------------------------------------------------------------------------

\usepgfplotslibrary{fillbetween}
\DeclareMathOperator{\Var}{Var}
\DeclareMathOperator{\Cov}{Cov}

\begin{document}

    % 1. Foto de fondo
    % 2. Título
    % 3. Encabezado Izquierdo
    % 4. Color de fondo
    % 5. Coord x del titulo
    % 6. Coord y del titulo
    % 7. Fecha

    
    % 1. Foto de fondo
% 2. Título
% 3. Encabezado Izquierdo
% 4. Color de fondo
% 5. Coord x del titulo
% 6. Coord y del titulo
% 7. Fecha

\newcommand{\portada}[7]{

    \portadaBase{#1}{#2}{#3}{#4}{#5}{#6}{#7}
    \portadaBook{#1}{#2}{#3}{#4}{#5}{#6}{#7}
}

\newcommand{\portadaExamen}[7]{

    \portadaBase{#1}{#2}{#3}{#4}{#5}{#6}{#7}
    \portadaArticle{#1}{#2}{#3}{#4}{#5}{#6}{#7}
}




\newcommand{\portadaBase}[7]{

    % Tiene la portada principal y la licencia Creative Commons
    
    % 1. Foto de fondo
    % 2. Título
    % 3. Encabezado Izquierdo
    % 4. Color de fondo
    % 5. Coord x del titulo
    % 6. Coord y del titulo
    % 7. Fecha
    
    
    \thispagestyle{empty}               % Sin encabezado ni pie de página
    \newgeometry{margin=0cm}        % Márgenes nulos para la primera página
    
    
    % Encabezado
    \fancyhead[L]{\helv #3}
    \fancyhead[R]{\helv \nouppercase{\leftmark}}
    
    
    \pagecolor{#4}        % Color de fondo para la portada
    
    \begin{figure}[p]
        \centering
        \transparent{0.3}           % Opacidad del 30% para la imagen
        
        \includegraphics[width=\paperwidth, keepaspectratio]{assets/#1}
    
        \begin{tikzpicture}[remember picture, overlay]
            \node[anchor=north west, text=white, opacity=1, font=\fontsize{60}{90}\selectfont\bfseries\sffamily, align=left] at (#5, #6) {#2};
            
            \node[anchor=south east, text=white, opacity=1, font=\fontsize{12}{18}\selectfont\sffamily, align=right] at (9.7, 3) {\textbf{\href{https://losdeldgiim.github.io/}{Los Del DGIIM}}};
            
            \node[anchor=south east, text=white, opacity=1, font=\fontsize{12}{15}\selectfont\sffamily, align=right] at (9.7, 1.8) {Doble Grado en Ingeniería Informática y Matemáticas\\Universidad de Granada};
        \end{tikzpicture}
    \end{figure}
    
    
    \restoregeometry        % Restaurar márgenes normales para las páginas subsiguientes
    \pagecolor{white}       % Restaurar el color de página
    
    
    \newpage
    \thispagestyle{empty}               % Sin encabezado ni pie de página
    \begin{tikzpicture}[remember picture, overlay]
        \node[anchor=south west, inner sep=3cm] at (current page.south west) {
            \begin{minipage}{0.5\paperwidth}
                \href{https://creativecommons.org/licenses/by-nc-nd/4.0/}{
                    \includegraphics[height=2cm]{assets/Licencia.png}
                }\vspace{1cm}\\
                Esta obra está bajo una
                \href{https://creativecommons.org/licenses/by-nc-nd/4.0/}{
                    Licencia Creative Commons Atribución-NoComercial-SinDerivadas 4.0 Internacional (CC BY-NC-ND 4.0).
                }\\
    
                Eres libre de compartir y redistribuir el contenido de esta obra en cualquier medio o formato, siempre y cuando des el crédito adecuado a los autores originales y no persigas fines comerciales. 
            \end{minipage}
        };
    \end{tikzpicture}
    
    
    
    % 1. Foto de fondo
    % 2. Título
    % 3. Encabezado Izquierdo
    % 4. Color de fondo
    % 5. Coord x del titulo
    % 6. Coord y del titulo
    % 7. Fecha


}


\newcommand{\portadaBook}[7]{

    % 1. Foto de fondo
    % 2. Título
    % 3. Encabezado Izquierdo
    % 4. Color de fondo
    % 5. Coord x del titulo
    % 6. Coord y del titulo
    % 7. Fecha

    % Personaliza el formato del título
    \pretitle{\begin{center}\bfseries\fontsize{42}{56}\selectfont}
    \posttitle{\par\end{center}\vspace{2em}}
    
    % Personaliza el formato del autor
    \preauthor{\begin{center}\Large}
    \postauthor{\par\end{center}\vfill}
    
    % Personaliza el formato de la fecha
    \predate{\begin{center}\huge}
    \postdate{\par\end{center}\vspace{2em}}
    
    \title{#2}
    \author{\href{https://losdeldgiim.github.io/}{Los Del DGIIM}}
    \date{Granada, #7}
    \maketitle
    
    \tableofcontents
}




\newcommand{\portadaArticle}[7]{

    % 1. Foto de fondo
    % 2. Título
    % 3. Encabezado Izquierdo
    % 4. Color de fondo
    % 5. Coord x del titulo
    % 6. Coord y del titulo
    % 7. Fecha

    % Personaliza el formato del título
    \pretitle{\begin{center}\bfseries\fontsize{42}{56}\selectfont}
    \posttitle{\par\end{center}\vspace{2em}}
    
    % Personaliza el formato del autor
    \preauthor{\begin{center}\Large}
    \postauthor{\par\end{center}\vspace{3em}}
    
    % Personaliza el formato de la fecha
    \predate{\begin{center}\huge}
    \postdate{\par\end{center}\vspace{5em}}
    
    \title{#2}
    \author{\href{https://losdeldgiim.github.io/}{Los Del DGIIM}}
    \date{Granada, #7}
    \thispagestyle{empty}               % Sin encabezado ni pie de página
    \maketitle
    \vfill
}
    \portadaExamen{ffccA4.jpg}{Probabilidad\\Examen VII}{Probabilidad. Examen VII}{MidnightBlue}{-8}{28}{2024-2025}{Arturo Olivares Martos \\ José Juan Urrutia Milán}

    \begin{description}
        \item[Asignatura] Probabilidad.
        \item[Curso Académico] 2021-22.
        \item[Grado] Doble Grado en Ingeniería Informática y Matemáticas.
        \item[Grupo] Único.
        %\item[Profesor] José María Espinar García.
        \item[Descripción] Examen Extraordinario. 
        \item[Fecha] 15 de febrero de 2022.
        % \item[Duración] 60 minutos.
    
    \end{description}
    \newpage

    \begin{ejercicio}
        Sean $X_1$, $X_2$, \ldots, $X_n$ variables aleatorias continuas e independientes, tales que $\exists E[X_i]$ $\forall i=1,\ldots,n$, con momento no centrado de orden dos finito. Justificar que:
        \begin{enumerate}
            \item \textbf{(0.5 puntos)} $\exists \Var\left(\sum\limits_{i=1}^{n}a_iX_i\right) = \sum\limits_{i=1}^{n}a_i^2 \Var(X_i)$ $\forall a_1,\ldots,a_n\in \mathbb{R}$.
            \item \textbf{(0.5 puntos)} $(X_1, \ldots, X_n)$ es un vector aleatorio continuo.
        \end{enumerate}

        Ambos apartados se encuentran demostrados en el Tema correspondiente a independencia de variables aleatorias.
    \end{ejercicio}

    \begin{ejercicio}
        Sea $(X,Y)$ una variable aleatoria bidimensional con distribución uniforme en el recito
        \begin{equation*}
            C = \{(x,y)\in \mathbb{R}^2 \mid x^2+y^2 < 1 \quad x,y\geq 0\}
        \end{equation*}
        \begin{observacion}
            Si necesitara obtener la primitiva de la función $f(x) = \sqrt{1-x^2}$, realizar el cambio de variable unidimensional $x=\sen(t)$.
        \end{observacion}
        \begin{enumerate}
            \item \textbf{(1.25 puntos)} Calcular la función de distribución de probabilidad conjunta.
            \begin{observacion}
                Se obtiene \textbf{hasta 1 punto} si las integrales se dejan indicadas y \textbf{hasta 1.25 puntos} si se obtienen sus primitivas de forma explícita.
            \end{observacion}

            Calculamos en primer lugar la función de densidad conjunta.
            Esta es constante en $C$, por lo que:
            \begin{equation*}
                f(x, y) = \begin{cases}
                    k, & x^2+y^2<1, x,y\geq 0 \\
                    0, & \text{en otro caso}.
                \end{cases}
            \end{equation*}

            Para que $f$ sea una función de densidad, tenemos que:
            \begin{align*}
                1&=\int_{-\infty}^{+\infty} \int_{-\infty}^{+\infty} f(x, y) \, dx \, dy
            \end{align*}

            Hay dos opciones:
            \begin{description}
                \item[Integrando de la forma usual:]
                
                Es necesario que:
                \begin{equation*}
                    1=\int_{0}^{1} \int_{0}^{\sqrt{1-x^2}} k \, dy \, dx = k\int_{0}^{1} \sqrt{1-x^2} \, dx
                \end{equation*}

                Haciendo el cambio de variable $x=\sen(t)$, tenemos que:
                \begin{align*}
                    1&=k\int_{0}^{\frac{\pi}{2}} \cos(t)\cos(t) \, dt = k\int_{0}^{\frac{\pi}{2}} \cos^2(t) \, dt
                    = k\int_{0}^{\frac{\pi}{2}} \dfrac{1+\cos(2t)}{2} \, dt
                    =\\&= k\left[\dfrac{t}{2}+\dfrac{\sen(2t)}{4}\right]_0^{\frac{\pi}{2}}
                    = k\left[\dfrac{\pi}{4}\right] \Longrightarrow k=\dfrac{4}{\pi}.
                \end{align*}

                \item[Razonando la forma de $C$:]
                
                Sabemos que $C$ es un cuarto de círculo de radio 1, por lo que su área es $\nicefrac{\pi}{4}$. Por tanto, tenemos que:
                \begin{equation*}
                    1=\int_C f(x, y) = k\int_C 1 = k\cdot \lm(C) = k\cdot \dfrac{\pi}{4} \Longrightarrow k=\dfrac{4}{\pi}.
                \end{equation*}
            \end{description}


            Para calcular la función de distribución conjunta, dividimos el plano cartesiano en las distintas regiones:
            \begin{figure}[H]
                \centering
                \begin{tikzpicture}
                    \begin{axis}[
                        axis lines = center,
                        xlabel = $X$,
                        ylabel = $Y$,
                        xmin = -0.5, xmax = 1.5,
                        ymin = -0.5, ymax = 1.5,
                        xtick = {0,1},
                        ytick = {0,1},
                        yticklabel style = {yshift=-1.5ex},
                        xticklabel style = {xshift=-1.5ex},
                        axis equal,
                    ]
                        % R2: Cuarto de circunferencia
                        \addplot [name path=A, blue, thick, forget plot, samples=70, domain=0:1] {sqrt(1-x^2)};
                        
                        % Dibuja la línea horizontal en y=0
                        \addplot [name path=B, forget plot, draw=none] {0};

                        % Dibuja la línea horizontal en y=0
                        \addplot [name path=C, forget plot, draw=none] {1};
                    
                        % Rellena el área bajo la curva entre x=0 y x=1
                        \addplot [
                            thick,
                            color=orange,
                            fill=orange,
                            fill opacity=0.4
                        ]
                        fill between [
                            of=A and B,
                            soft clip={domain=0:1},
                        ];
                        
                        % Señalamos zona R2
                        \node at (0.5, 0.5) {$R_2$};

                        % R1: (0, 2), (-2, 2), (-2, -2), (2, -2), (2, 0), (0, 0)
                        \addplot[fill=red, opacity=0.2] coordinates {
                            (0, 2) (-2, 2) (-2, -2) (2, -2) (2, 0) (0, 0)
                        } --cycle;
                        \node at (-0.25, -0.25) {$R_1$};

                        \addplot [
                            thick,
                            color=teal,
                            fill=teal,
                            fill opacity=0.4
                        ]
                        fill between [
                            of=C and A,
                            soft clip={domain=0:1},
                        ];
                        \node at (0.8, 0.8) {$R_3$};

                        \addplot[fill=blue, opacity=0.2] coordinates {
                            (0, 1) (1,1) (1,2) (0,2)
                        } --cycle;
                        \node at (0.5, 1.25) {$R_4$};

                        \addplot[fill=green, opacity=0.2] coordinates {
                            (1, 1) (2,1) (2,0) (1,0)
                        } --cycle;
                        \node at (1.25, 0.5) {$R_5$};

                        \addplot[fill=olive, opacity=0.2] coordinates {
                            (1, 1) (2,1) (2,2) (1,2)
                        } --cycle;
                        \node at (1.25, 1.25) {$R_6$};
                    \end{axis}
                \end{tikzpicture}
            \end{figure}

            Distinguimos casos:
            \begin{itemize}
                \item \ul{Si $x\leq 0$ \quad o \quad $y\leq 0$} (zona $R_1$):
                \begin{equation*}
                    F_{(X,Y)}(x, y) = \int_{-\infty}^x \int_{-\infty}^y f(u, v) \, du \, dv = 0.
                \end{equation*}

                \item \ul{Si $x\in \left]0,1\right[$ \quad y \quad $y\in \left]0,\sqrt{1-x^2}\right[$} (zona $R_2$):
                \begin{align*}
                    F_{(X,Y)}(x, y) &= \int_{-\infty}^x \int_{-\infty}^y f(u, v) \, dv \, du = \int_{0}^x \int_{0}^y \dfrac{4}{\pi} \, dv \, du = \int_{0}^x \dfrac{4}{\pi}y \, du
                    =\\&= \dfrac{4}{\pi}\left[yu\right]_0^x = \dfrac{4}{\pi}\cdot xy
                \end{align*}

                \item \ul{Si $x\in \left]0,1\right[$ \quad y \quad $y\in \left]\sqrt{1-x^2},1\right[$} (zona $R_3$):
                \begin{align*}
                    F_{(X,Y)}(x, y) &= \int_{-\infty}^x \int_{-\infty}^y f(u, v) \, dv \, du =\\&= \int_{0}^{\sqrt{1-y^2}}\int_0^y \dfrac{4}{\pi} \, dv \, du + \int_{\sqrt{1-y^2}}^x \int_0^{\sqrt{1-u^2}} \dfrac{4}{\pi} \, dv \, du =\\
                    &= \int_{0}^{\sqrt{1-y^2}} \dfrac{4}{\pi}y \, du + \int_{\sqrt{1-y^2}}^x \dfrac{4}{\pi}\sqrt{1-u^2} \, du
                \end{align*}

                Para resolver la segunda integral, hacemos el cambio de variable dado por $u=\sen(t)$, $du=\cos(t)dt$:
                \begin{align*}
                    F_{(X,Y)}(x, y) &= \dfrac{4}{\pi} y\sqrt{1-y^2} + \dfrac{4}{\pi}\int_{\arcsen(\sqrt{1-y^2})}^{\arcsen(x)} \cos^2(t) \, dt
                    =\\&= \dfrac{4}{\pi} y\sqrt{1-y^2} + \dfrac{4}{\pi}\int_{\arcsen(\sqrt{1-y^2})}^{\arcsen(x)} \dfrac{1+\cos(2t)}{2} \, dt
                    =\\&= \dfrac{4}{\pi} y\sqrt{1-y^2} + \dfrac{4}{\pi}\left[\dfrac{t}{2}+\dfrac{\sen(2t)}{4}\right]_{\arcsen(\sqrt{1-y^2})}^{\arcsen(x)}
                    =\\&= \dfrac{4}{\pi} y\sqrt{1-y^2} + \dfrac{4}{\pi}\left[\dfrac{\arcsen(x)}{2}+\dfrac{\sen(2\arcsen(x))}{4}-\right.\\&\qquad -\left.\dfrac{\arcsen(\sqrt{1-y^2})}{2}-\dfrac{\sen(2\arcsen(\sqrt{1-y^2}))}{4}\right]
                \end{align*}

                Veamos cuánto vale anteriormente $\sen(2\arcsen(x))$ para cierto $x\in \bb{R}$:
                \begin{align*}
                    \sen(2\arcsen(x)) &= 2\sen(\arcsen(x))\cos(\arcsen(x)) = 2x\sqrt{1-x^2}.
                \end{align*}

                Por tanto, tenemos que:
                \begin{align*}
                    F_{(X,Y)}(x, y) &= \dfrac{4}{\pi} y\sqrt{1-y^2} + \dfrac{4}{\pi}\left[\dfrac{\arcsen(x)}{2}+\dfrac{2x\sqrt{1-x^2}}{4}-\right.\\&\qquad -\left.\dfrac{\arcsen(\sqrt{1-y^2})}{2}-\dfrac{2\sqrt{1-y^2}\sqrt{y^2}}{4}\right]
                    =\\&= \dfrac{4}{\pi} y\sqrt{1-y^2} + \dfrac{2}{\pi}\arcsen(x)+\dfrac{2}{\pi}x\sqrt{1-x^2}-\\&\qquad-\dfrac{2}{\pi}\arcsen(\sqrt{1-y^2})-\dfrac{2}{\pi}y\sqrt{1-y^2}.
                    =\\&= \dfrac{2}{\pi} y\sqrt{1-y^2} + \dfrac{2}{\pi}\arcsen(x)+\dfrac{2}{\pi}x\sqrt{1-x^2}-\dfrac{2}{\pi}\arcsen(\sqrt{1-y^2})
                \end{align*}

                \item \ul{Si $x\in \left]0,1\right[$ \quad y \quad $y\geq 1$} (zona $R_4$):
                \begin{align*}
                    F_{(X,Y)}(x, y) &= \int_{-\infty}^x \int_{-\infty}^y f(u, v) \, dv \, du = \int_{0}^x \int_{0}^{\sqrt{1-u^2}} \dfrac{4}{\pi} \, dv \, du
                    =\\&= \int_{0}^x \dfrac{4}{\pi}\sqrt{1-u^2} \, du
                \end{align*}

                Para resolver la integral, de nuevo hacemos el cambio de variable dado por $u=\sen(t)$, $du=\cos(t)dt$:
                \begin{align*}
                    F_{(X,Y)}(x, y) &= \dfrac{4}{\pi}\int_{0}^{\arcsen(x)} \cos^2(t) \, dt
                    = \dfrac{4}{\pi}\int_{0}^{\arcsen(x)} \dfrac{1+\cos(2t)}{2} \, dt
                    =\\&= \dfrac{4}{\pi}\left[\dfrac{t}{2}+\dfrac{\sen(2t)}{4}\right]_0^{\arcsen(x)}
                    = \dfrac{4}{\pi}\left[\dfrac{\arcsen(x)}{2}+\dfrac{\sen(2\arcsen(x))}{4}\right]
                    =\\&= \dfrac{4}{\pi}\left[\dfrac{\arcsen(x)}{2}+\dfrac{2x\sqrt{1-x^2}}{4}\right]
                    = \dfrac{2}{\pi}\arcsen(x)+\dfrac{2}{\pi}x\sqrt{1-x^2}.
                \end{align*}

                \item \ul{Si $y\in \left]0,1\right[$ \quad y \quad $x\geq 1$} (zona $R_5$):
                \begin{align*}
                    F_{(X,Y)}(x, y) &= \int_{-\infty}^x \int_{-\infty}^y f(u, v) \, dv \, du =\\
                    &= \int_{0}^{\sqrt{1-y^2}} \int_{0}^y \dfrac{4}{\pi} \, dv \, du + \int_{\sqrt{1-y^2}}^1 \int_{0}^{\sqrt{1-u^2}} \dfrac{4}{\pi} \, dv \, du
                    =\\&= \int_{0}^{\sqrt{1-y^2}} \dfrac{4}{\pi}y \, du + \int_{\sqrt{1-y^2}}^1 \dfrac{4}{\pi}\sqrt{1-u^2} \, du
                \end{align*}

                Para resolver la segunda integral, de nuevo hacemos el cambio de variable dado por $u=\sen(t)$, $du=\cos(t)dt$:
                \begin{align*}
                    F_{(X,Y)}(x, y) &= \dfrac{4}{\pi}y\left[u\right]_0^{\sqrt{1-y^2}} + \dfrac{4}{\pi}\int_{\arcsen(\sqrt{1-y^2})}^{\nicefrac{\pi}{2}} \cos^2(t) \, dt
                    =\\&= \dfrac{4}{\pi}y\sqrt{1-y^2} + \dfrac{4}{\pi}\int_{\arcsen(\sqrt{1-y^2})}^{\nicefrac{\pi}{2}} \dfrac{1+\cos(2t)}{2} \, dt
                    =\\&= \dfrac{4}{\pi}y\sqrt{1-y^2} + \dfrac{4}{\pi}\left[\dfrac{t}{2}+\dfrac{\sen(2t)}{4}\right]_{\arcsen(\sqrt{1-y^2})}^{\nicefrac{\pi}{2}}
                    =\\&= \dfrac{4}{\pi}y\sqrt{1-y^2} + \dfrac{4}{\pi}\left[\dfrac{\nicefrac{\pi}{2}}{2}+\dfrac{\sen(\pi)}{4}-\dfrac{\arcsen(\sqrt{1-y^2})}{2}-\right.\\&\qquad\left.-\dfrac{\sen(2\arcsen(\sqrt{1-y^2}))}{4}\right]
                    =\\&= \dfrac{4}{\pi}y\sqrt{1-y^2} + 1 - \dfrac{2}{\pi}\arcsen(\sqrt{1-y^2}) - \dfrac{2}{\pi}y\sqrt{1-y^2}
                    =\\&= \dfrac{2}{\pi}y\sqrt{1-y^2} + 1 - \dfrac{2}{\pi}\arcsen(\sqrt{1-y^2})
                \end{align*}

                \item \ul{Si $x,y\geq 1$} (zona $R_6$):
                \begin{equation*}
                    F_{(X,Y)}(x, y) = 1.
                \end{equation*}
            \end{itemize}

            Por tanto, tenemos que:
            \begin{equation*}
                F_{(X,Y)}(x, y) = \begin{cases}
                    0, & (x,y)\in R_1, \\
                    \nicefrac{4}{\pi}xy, & (x,y)\in R_2, \\
                    \nicefrac{2}{\pi}\left[y\sqrt{1-y^2} + x\sqrt{1-x^2}-\arcsen(\sqrt{1-y^2})\right], & (x,y)\in R_3 \\
                    \nicefrac{2}{\pi}\left[\arcsen(x)+x\sqrt{1-x^2}\right], & (x,y)\in R_4, \\
                    \nicefrac{2}{\pi}\left[y\sqrt{1-y^2} + \nicefrac{\pi}{2} - \arcsen(\sqrt{1-y^2})\right], & (x,y)\in R_5, \\
                    1, & (x,y)\in R_6.
                \end{cases}
            \end{equation*}
            \item \textbf{(1.25 puntos)} Calcular las funciones de densidad condicionadas.
            
            Para ello, calculamos en primer lugar las funciones de densidad marginales. Para $x\in [0,1]$, ya que la función de densidad es constante, tenemos que:
            \begin{align*}
                f_X(x) &= \int_{-\infty}^{+\infty} f(x, y) \, dy = \int_{0}^{\sqrt{1-x^2}} \dfrac{4}{\pi} \, dy = \dfrac{4}{\pi}\left[y\right]_{0}^{\sqrt{1-x^2}} = \dfrac{4}{\pi}\cdot \sqrt{1-x^2}.
            \end{align*}
    
            Para $y\in [0,1]$, ya que la función de densidad es constante, tenemos que:
            \begin{align*}
                f_Y(y) &= \int_{-\infty}^{+\infty} f(x, y) \, dx = \int_{0}^{\sqrt{1-y^2}} \dfrac{4}{\pi} \, dx = \dfrac{4}{\pi}\left[x\right]_{0}^{\sqrt{1-y^2}} = \dfrac{4}{\pi}\cdot \sqrt{1-y^2}.
            \end{align*}

            Una vez calculadas estas, calculamos las funciones de densidad condicionadas. Dado $x^\ast\in [0,1]$, tenemos para $y\in [0,\sqrt{1-(x^{\ast})^2}]$:
            \begin{equation*}
                f_{Y\mid X=x^\ast} (y) = \dfrac{f_{(X,Y)}(x^\ast, y)}{f_X(x^\ast)} = \dfrac{\nicefrac{4}{\pi}}{\dfrac{4}{\pi}\cdot \sqrt{1-(x^{\ast})^2}} = \dfrac{1}{\sqrt{1-(x^{\ast})^2}}.
            \end{equation*}
    
            Dado $y^\ast\in [0,1]$, tenemos para $x\in [0,\sqrt{1-(y^{\ast})^2}]$:
            \begin{equation*}
                f_{X\mid Y=y^\ast} (x) = \dfrac{f_{(X,Y)}(x, y^\ast)}{f_Y(y^\ast)} = \dfrac{\nicefrac{4}{\pi}}{\dfrac{4}{\pi}\cdot \sqrt{1-(y^{\ast})^2}} = \dfrac{1}{\sqrt{1-(y^{\ast})^2}}.
            \end{equation*}
        \end{enumerate}
    \end{ejercicio}

    \begin{ejercicio}
        Sea considera $(X,Y)$ la distribución uniforme en el cuadrado unidad.
        \begin{enumerate}
            \item \textbf{(1.25 puntos)} Calcular la función de densidad de probabilidad del vector aleatorio $Z = (X+Y,X-Y)$.
            
            Como cuadrado unidad, entendemos $C=[0,1]\times[0,1]$. La función de densidad conjunta es constante en $C$, por lo que:
            \begin{equation*}
                f_{(X,Y)}(x, y) = \begin{cases}
                    k, & (x,y)\in C, \\
                    0, & \text{en otro caso}.
                \end{cases}
            \end{equation*}

            Para que $f_{(X,Y)}$ sea una función de densidad, tenemos que:
            \begin{align*}
                1&=\int_{-\infty}^{+\infty} \int_{-\infty}^{+\infty} f_{(X,Y)}(x, y) \, dx \, dy
                =\\&= \int_{0}^{1} \int_{0}^{1} k \, dx \, dy
                = k\int_{0}^{1} \left[x\right]_{0}^{1} \, dy
                = k\int_{0}^{1} 1 \, dy
                = k\left[y\right]_{0}^{1}
                = k.
            \end{align*}
            \item \textbf{(1.25 puntos)} La función de densidad de proabbilidad conjunta del máximo y el mínimo.
        \end{enumerate}
    \end{ejercicio}

    \begin{ejercicio}
        Dado un vector aleatorio con función generatriz de momentos
        \begin{equation*}
            M_{X_1,X_2}(t_1,t_2) = {\left(\dfrac{e^{t_1}}{2}+\dfrac{e^{t_2}}{4}+\dfrac{1}{4}\right)}^{5} \qquad t_1,t_2 \in \mathbb{R}
        \end{equation*}
        Calcular la razón de correlación y el coeficiente de correlación lineal de las variables $X_1$ y $X_2$.
    \end{ejercicio}

    \begin{ejercicio}
        Dado el vector bidimensional $(X,Y)$ con la siguiente función masa de probabilidad conjunta:
        \begin{table}[H]
        \centering
        \begin{tabular}{|c|ccc|}
            \hline
            $X|Y$ & 0 & 1 & 2 \\
            \hline
            1 & $\nicefrac{1}{4}$ & 0 & 0 \\
            2 & 0 & $\nicefrac{1}{4}$ & 0 \\
            3 & $\nicefrac{1}{4}$ & 0 & $\nicefrac{1}{4}$ \\
            \hline
        \end{tabular}
        \end{table}
        \begin{enumerate}
            \item \textbf{(1.25 puntos)} Obtener la mejor aproximación minimo-cuadrática a la variable $Y$ conocidos valores de la variable $X$, así como calcular una medida de la bondad del ajuste.
            \item \textbf{(1.25 puntos)} Obtener las ecuaciones de la rectas de regresión de $Y|X$ y $X|Y$ y el error cuadrático medio.
        \end{enumerate}
    \end{ejercicio}

\end{document}
