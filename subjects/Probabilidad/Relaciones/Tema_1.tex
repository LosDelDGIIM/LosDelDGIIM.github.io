\newpage
\section{Distribuciones Continuas}

\begin{ejercicio}
    Sea $X$ una variable aleatoria continua que sigue una distribución uniforme en el intervalo $[a,b]$, es decir:
    \begin{equation*}
        X\sim \cc{U}[a,b]
    \end{equation*}

    \noindent
    Calcular su función generatriz de momentos.\\

    Distinguimos en función del valor de $t$:
    \begin{itemize}
        \item Para $t\neq 0$:

        \begin{align*}
            M_X(t) &= E\left(e^{tX}\right) = \int_{a}^{b} e^{tx} \frac{1}{b-a} \, dx =\\
            &= \dfrac{1}{b-a} \int_{a}^{b} e^{tx} \, dx = \dfrac{1}{b-a} \left[ \dfrac{e^{tx}}{t} \right]_{a}^{b} =
            \dfrac{e^{tb} - e^{ta}}{(b-a)t}
        \end{align*}

        \item Para $t=0$:
        
        \begin{equation*}
            M_X(0) = E\left(e^{0X}\right) = E(1) = 1
        \end{equation*}
    \end{itemize}
\end{ejercicio}


\begin{ejercicio}
    Comprueba que la función de densidad de la distribución normal es una función de densidad.\\
    
    Sea $X$ una variable aleatoria continua que sigue una distribución normal de media $\mu$ y varianza $\sigma^2$, es decir:
    \begin{equation*}
        X\sim \cc{N}(\mu,\sigma^2)
    \end{equation*}

    Para comprobar lo pedido, debemos comprobar que:
    \begin{itemize}
        \item $f(x)\geq 0$ para todo $x\in \mathbb{R}$.
        
        Tenemos que:
        \begin{equation*}
            f(x) = \dfrac{1}{\sqrt{2\pi}\sigma} e^{-\dfrac{(x-\mu)^2}{2\sigma^2}} \geq 0 \qquad \forall x\in \mathbb{R}
        \end{equation*}
        
        \item $\int_{-\infty}^{\infty} f(x) \, dx = 1$.
        
        \begin{align*}
            \int_{-\infty}^{\infty} f(x) \, dx &= \int_{-\infty}^{\infty} \dfrac{1}{\sqrt{2\pi}\sigma} e^{-\dfrac{(x-\mu)^2}{2\sigma^2}} \, dx = \dfrac{1}{\sqrt{2\pi}\sigma} \int_{-\infty}^{\infty} e^{-\dfrac{(x-\mu)^2}{2\sigma^2}} \, dx =\\
            &= \MetInt{t=\frac{x-\mu}{\sigma}}{dt = \frac{dx}{\sigma}} = \dfrac{1}{\sqrt{2\pi}\sigma} \int_{-\infty}^{\infty} e^{-\dfrac{t^2}{2}} \, dt \AstIg \dfrac{1}{\sqrt{2\pi}\sigma} \cdot \sigma \sqrt{2\pi} = 1
        \end{align*}
        donde en $(\ast)$ hemos aplicado la Integral de Gauss:
        \begin{equation*}
            \int_{-\infty}^{\infty} e^{-a(x+b)^2} \, dx = \sqrt{\dfrac{\pi}{a}} \qquad a,b\in \mathbb{R}^+
        \end{equation*}
    \end{itemize}
\end{ejercicio}


\begin{ejercicio}[Regla de la Probabilidad Normal]
    Sea $X$ una variable aleatoria continua que sigue una distribución normal de media $\mu$ y varianza $\sigma^2$, es decir:
    \begin{equation*}
        X\sim \cc{N}(\mu,\sigma^2)
    \end{equation*}

    Demostrar que:
    \begin{align*}
        P[\mu-\sigma\leq X\leq \mu+\sigma]
        &= P\left[\dfrac{\mu-\sigma-\mu}{\sigma} \leq Z\leq
        \dfrac{\mu+\sigma-\mu}{\sigma}\right]
        =\\&= P[-1\leq Z\leq 1]
        = P[Z\leq 1] -P[Z\leq -1]
        =\\&= P[Z\leq 1] -1+P[Z\leq 1]
        = 2P[Z\leq 1] -1 \approx\\&\approx 2\cdot 0.8413-1
        \approx 0.6826
    \end{align*}
    
\end{ejercicio}