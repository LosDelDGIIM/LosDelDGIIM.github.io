\section{Problemas de sesiones prácticas}

\begin{ejercicio}
    Demustre que para todo número natural $n$:
    $$\left(\sum_{k=0}^n k\right)^2 = \left(\sum_{k=0}^{n-1}\right)^2 + n^3$$
\end{ejercicio}
\begin{proof}
    La demostración es por casos:
    \begin{description}
        \item [$n = 0;$]
            $$\left(\sum_{n=0}^0 k\right)^2 = 0^2 = 0 = 0+0 = \left(\sum_{k=0}^{-1}k\right)^2 + 0^3$$
        \item [$n = 1;$]
            $$\left(\sum_{k=0}^1 k\right)^2 = 0+1 = \left(\sum_{k=0}^0 k\right)^2 + 1^3 = \left(\sum_{k=0}^{n-1} k\right)^2 + n^3$$
        \item [$n > 1;$]
            $$\left(\sum_{k=0}^n k\right)^2 = \left[\left(\sum_{k=0}^{n-1} k\right) + n\right]^2 = \left(\sum_{k=0}^{n-1}k\right)^2 + n^2 + 2\left(\sum_{k=0}^{n-1} k\right)n =$$
            $$= \left(\sum_{k=0}^{n-1}k\right)^2 + n^2 + 2 \dfrac{(n-1)n}{2} n = \left(\sum_{k=0}^{n-1}k\right)^2 + n^2 + (n-1)n^2 =$$ 
            $$= \left(\sum_{k=0}^{n-1}k\right)^2 + n^2 (1+n-1) = \left(\sum_{k=0}^{n-1}k\right)^2 + n^3$$
    \end{description}
    % // TODO: hacer por inducción
\end{proof}

\begin{ejercicio}[Teorema de Nicomachus]
    Demuestre que para todo número natural $n$ vale la siguiente igualdad:
    $$\sum_{k=0}^n k^3 = \left(\sum_{k=0}^n k\right)^2$$
\end{ejercicio}
\begin{proof}
    La demostración es por inducción según el principio de inducción matemática y predicado $P(n)$ del contenido literal (tenor):
    $$\mbox{''}\sum_{k=0}^n k^3 = \left(\sum_{k=0}^n k\right)^2\mbox{''}$$
    \begin{itemize}
        \item En el caso base $n = 0$;
        $$\sum_{k=0}^0 k^3 = 0^3 = 0 = 0^2 = \left(\sum_{k=0}^0 k\right)^2$$
        Y por tanto, $P(0)$ vale.\\
    \item Como hipótesis de inducción, supondremos que $n$ es un número natural y que $P(n)$ vale, es decir, 
        $$\sum_{k=0}^n k^3 = \left(\sum_{k=0}^n k\right)^2$$
        En el paso de inducción, demostraremos que $P(n+1)$ vale.
        $$\sum_{k=0}^n k^3 = \left(\sum_{k=0}^n k^3\right) + (n+1)^3 \AstIg \left(\sum_{k=0}^n k\right)^2 + (n+1)^3 \stackrel{(\ast\ast)}{=} \left(\sum_{k=0}^{n+1} k\right)^2  $$

        donde en $(\ast)$ he utilizado la hipótesis de inducción.\newline
        donde en $(\ast\ast)$ he utilizado el ejercicio 1.4.4.\newline
        Luego $P(n+1)$ vale.\newline
        Por el principio de inducción matemática para todo número natural $n$, $P(n)$ vale como se pedía.
    \end{itemize}
\end{proof}

\begin{observacion}
    El segundo principio de inducción matemática se utiliza cuando, en vez de usar como hipótesis una verdad sobre $n$, usar una verdad como $n-k$ con $k >1$, cuando estemos demostrado que el predicado vale para $n+1$.
\end{observacion}

