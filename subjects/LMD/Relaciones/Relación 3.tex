\section{Lógica Proposicional}

\begin{ejemplo}
    La fórmula proposicional $(a\to b)$ (siempre que $b\not\equiv a$) es refutable.\\

    Basta tomar una valoración $v$ tal que $v(a)=1$ y $v(b)=0$.

    De hecho, $a\to b$ es satisfacible, luego es una fórmula contingente.
\end{ejemplo}

\begin{ejemplo}
    Las siguientes fórmulas proposicionales son tautologías:
    \begin{enumerate}
        \item $(a\rightarrow a)$
        \begin{align*}
            v(a\rightarrow a) &= v(a)v(a)+v(a)+1 \\
                     &= v(a) + v(a) + 1 \\
                     &= 0 + 1 = 1
        \end{align*}
        \begin{observacion}
            Notemos que, dado $a\in \bb{Z}_2$, se tiene que $a^2=a$ y $2a=0$.
            Esto será usado constantemente, aunque en un principio el lector no
            esté acostumbrado a trabajar a trabajar en este anillo.
        \end{observacion}

        \item $(\alpha\lor \lnot \alpha)$. Esta tautología se conoce como el principio del tercio excluso.
        \begin{align*}
            v(\alpha\lor \lnot \alpha) &= v(\alpha)v(\lnot \alpha) + v(\alpha) + v(\lnot \alpha)\\
                                      &= v(\alpha)(v(\alpha)+1) + v(\alpha) + v(\alpha) + 1 \\
                                      &= v(\alpha)^2 + v(\alpha) + v(\alpha) + v(\alpha) + 1\\
                                      &= 0 + 0 + 1 = 1
        \end{align*}
    \end{enumerate}
\end{ejemplo}

\begin{ejemplo}
    Dada una valoración $v$ y dadas $\alpha,\beta$ fórmulas proposicionales, se tiene que:
    \begin{equation*}
        v\left((\alpha\rightarrow\beta) \land (\beta\rightarrow\alpha)\right) = 
        v(\alpha\leftrightarrow\beta)
    \end{equation*}

    En efecto, se tiene que:
    \begin{align*}
        v\left((\alpha\rightarrow\beta) \land (\beta\rightarrow\alpha)\right)
        &= v(\alpha\rightarrow\beta)v(\beta\rightarrow\alpha)\\
        &= \left(v(\alpha)v(\beta)+v(\alpha)+1\right)\left(v(\beta)v(\alpha)+v(\beta)+1\right)\\
        &= v(\alpha)v(\beta)v(\alpha)v(\beta) + v(\alpha)v(\beta)v(\beta) + v(\alpha)v(\beta) +\\
        &\qquad + v(\alpha)v(\beta)v(\alpha) + v(\alpha)v(\beta) + v(\alpha) +\\
        &\qquad + v(\alpha)v(\beta)+ v(\beta) + 1\\
        &= 6\cdot v(\alpha)v(\beta) + v(\alpha) + v(\beta) + 1\\
        &= v(\alpha) + v(\beta) + 1\\
        &= v(\alpha\leftrightarrow\beta)
    \end{align*}
\end{ejemplo}


\begin{ejercicio}
    Dadas $\alpha,~\beta$ fórmulas proposicionales,
    clasifique las siguientes fórmulas en función de si son tautologías, contradicciones o contingentes:
    \begin{enumerate}
        \item $(\alpha\rightarrow\beta)\rightarrow(\lnot\beta\rightarrow\lnot\alpha)$
        
        En lo que sigue, sea $v$ una valoración fija pero arbitraria.
        Para facilitar el desarrollo, notaremos que dada $\gamma$ fórmula proposicional, tenemos que:
        \begin{align*}
            v(\gamma)v(\lnot \gamma)
            &= v(\gamma)(v(\gamma)+1)\\
            &= v(\gamma)^2 + v(\gamma)\\
            &= v(\gamma) + v(\gamma) = 0
        \end{align*}
        
        Tenemos que:
        \begin{align*}
            v&((\alpha\rightarrow\beta)\rightarrow(\lnot\beta\rightarrow\lnot\alpha)) 
            = v(\alpha\rightarrow\beta)v(\lnot\beta\rightarrow\lnot\alpha)+ v(\alpha\rightarrow\beta) + 1 \\
            &= (v(\alpha)v(\beta)+v(\alpha) + 1) (v(\lnot\beta)v(\lnot\alpha)+v(\lnot\beta)+1)
            + v(\alpha)v(\beta)+v(\alpha)+\cancel{1}+\cancel{1} \\
            &= (v(\alpha)v(\beta)+v(\lnot \alpha)) (v(\lnot\beta)v(\lnot\alpha)+v(\beta))
            + v(\alpha)v(\beta)+v(\alpha) \\
            &= \cancelto{0}{v(\alpha)v(\beta)v(\lnot\beta)v(\lnot\alpha)} + \bcancel{v(\alpha)v(\beta)^2}
            + v(\lnot \alpha)^2 v(\lnot\beta) + v(\lnot\alpha)v(\beta) +\\
            &\quad + \bcancel{v(\alpha)v(\beta)}+v(\alpha)\\
            &= v(\lnot\alpha)v(\lnot\beta) + v(\lnot\alpha)v(\beta) + v(\alpha)\\
            &= v(\lnot\alpha)(v(\lnot\beta)+v(\beta))+v(\alpha)\\
            &= v(\lnot\alpha)(2v(\beta)+1)+v(\alpha)\\
            &= v(\lnot\alpha)+v(\alpha)\\
            &= 2v(\alpha) +1 = 1
        \end{align*}
        
        Por tanto, se trata de una tautología.
        
        \item $(\alpha\rightarrow\beta)\rightarrow((\alpha\rightarrow\beta)\rightarrow\lnot\alpha)$
        
        Tenemos que:
        \begin{align*}
            v&((\alpha\rightarrow\beta)\rightarrow((\alpha\rightarrow\beta)\rightarrow\lnot\alpha))\\
            &= v(\alpha\rightarrow\beta)v((\alpha\rightarrow\beta)\rightarrow\lnot\alpha)+v(\alpha\rightarrow\beta)+1\\
            &= v(\alpha\rightarrow\beta)(v(\alpha\rightarrow\beta)v(\lnot\alpha)+v(\alpha\rightarrow\beta)+1)+v(\alpha\rightarrow\beta)+1\\
            &= v(\alpha\rightarrow\beta)^2v(\lnot\alpha)+v(\alpha\rightarrow\beta)^2+\cancel{v(\alpha\rightarrow\beta)}+\cancel{v(\alpha\rightarrow\beta)}+1\\
            &= v(\alpha\rightarrow\beta)[v(\lnot \alpha)+1] +1\\
            &= v(\alpha\rightarrow\beta)v(\alpha)+1\\
            &= [v(\alpha)v(\beta) + v(\alpha) +1]v(\alpha) +1\\
            &= v(\alpha)^2v(\beta)+ \cancel{v(\alpha)^2} + \cancel{v(\alpha)} + 1\\
            &= v(\alpha)v(\beta) + 1\\
            &= v(\lnot(\alpha\land \beta))
        \end{align*}

        Por tanto, depende de la valoración.
        \begin{itemize}
            \item Si $v(\alpha)=1=v(\beta)$, entonces:
                \begin{equation*}
                    v((\alpha\rightarrow\beta)\rightarrow((\alpha\rightarrow\beta)\rightarrow\lnot\alpha))=0
                \end{equation*}
            \item Si $v(\alpha)=0$ ó $v(\beta)=0$, entonces:
                \begin{equation*}
                    v((\alpha\rightarrow\beta)\rightarrow((\alpha\rightarrow\beta)\rightarrow\lnot\alpha))=1
                \end{equation*}
        \end{itemize}
        En general, la fórmula ni será tautología ni contradicción. Es una una fórmula contingente.

        \item $(\alpha\rightarrow\lnot\beta)\rightarrow(\lnot\alpha\rightarrow\beta)$
        
        Tenemos que:
        \begin{align*}
            v&((\alpha\rightarrow\lnot\beta)\rightarrow(\lnot\alpha\rightarrow\beta))
            = v(\alpha\rightarrow\lnot\beta)v(\lnot\alpha\rightarrow\beta)+v(\alpha\rightarrow\lnot\beta) + 1 \\
            &= (v(\alpha)v(\lnot\beta)+v(\alpha)+1)(v(\lnot\alpha)v(\beta)+v(\lnot\alpha)+1) + v(\alpha)v(\lnot\beta)+v(\alpha)+\cancel{1}+\cancel{1} \\ 
            &= (v(\alpha)v(\lnot\beta)+v(\lnot \alpha))(v(\lnot\alpha)v(\beta)+v(\alpha)) + v(\alpha)v(\lnot\beta)+v(\alpha) \\ 
            &= \cancelto{0}{v(\alpha)v(\lnot\beta)v(\lnot\alpha)v(\beta)}+v(\lnot \alpha)^2v(\beta) +\cancel{v(\alpha)^2v(\lnot\beta)}+\cancelto{0}{v(\lnot \alpha)v(\alpha)} +\\&\qquad + \cancel{v(\alpha)v(\lnot\beta)}+v(\alpha) \\ 
            &= v(\alpha+1)v(\beta) + v(\alpha)\\
            &= v(\alpha)v(\beta) + v(\beta) + v(\alpha)\\
            &= v(\alpha\lor \beta)
        \end{align*}

        Por tanto, depende de la valoración.
        \begin{itemize}
            \item Si $v(\alpha)=0=v(\beta)$, entonces:
                \begin{equation*}
                    v((\alpha\rightarrow\lnot\beta)\rightarrow(\lnot\alpha\rightarrow\beta))=0
                \end{equation*}
            \item Si $v(\alpha)=1$ ó $v(\beta)=1$, entonces:
                \begin{equation*}
                    v((\alpha\rightarrow\lnot\beta)\rightarrow(\lnot\alpha\rightarrow\beta)) = 1
                \end{equation*}
        \end{itemize}
        En general, la fórmula ni será tautología ni contradicción. Es una una fórmula contingente.

    \end{enumerate}
\end{ejercicio}


\begin{ejercicio*}
    Para cualesquiera fórmulas proposicionales $\alpha,\beta,\gamma$, demuestre que las siguientes reglas son correctas:

    \begin{enumerate}
        \item $\alpha,\alpha\rightarrow\beta\vDash\beta$. Esta regla se conoce como \emph{modus ponens}.
        
        Tenemos que es equivalente a:
        \begin{equation*}
            \begin{array}{l}
                \alpha\\
                \alpha\rightarrow\beta\\ \hline
                \multicolumn{1}{c}{\beta}
            \end{array}
        \end{equation*}

        Sea $v$ una asignación fija pero arbitraria a condición de cumplir:
        \begin{equation*}
            v(\alpha)=1=v(\alpha\rightarrow\beta)
        \end{equation*}

        Entonces, tenemos que:
        \begin{align*}
            1 &= v(\alpha\rightarrow\beta)\\
            &= v(\alpha)v(\beta)+v(\alpha)+1\\
            &= 1\cdot v(\beta)+1+1\\
            &= v(\beta)
        \end{align*}

        Por tanto, como $v(\beta)=1$, la regla es correcta.

        \item $\alpha\rightarrow(\beta\rightarrow\gamma),~\alpha\land\beta \vDash \gamma$
        
        Tenemos que es quivalente a:
        \begin{equation*}
            \begin{array}{l}
                \alpha\rightarrow(\beta\rightarrow\gamma)\\
                \alpha\land\beta \\ \hline
                \multicolumn{1}{c}{\gamma}
            \end{array}
        \end{equation*}
        
        Sea $v$ una asignación fija pero arbitraria a condición de cumplir:
        \begin{equation*}
            v(\alpha\rightarrow(\beta\rightarrow\gamma)) = 1 = v(\alpha\land \beta)
        \end{equation*}
        Debemos demostrar que $v(\gamma)=1$.\\

        Como $v(\alpha\rightarrow(\beta\rightarrow\gamma))=1$, entonces:
        \begin{align*}
            1 &= v(\alpha)v(\beta\rightarrow\gamma)+v(\alpha)+1\\
            &= v(\alpha)(v(\beta)v(\gamma)+v(\beta)+1)+v(\alpha)+1\\
            &= v(\alpha)v(\beta)v(\gamma)+v(\alpha)v(\beta)+v(\alpha)+v(\alpha)+1\\
            &= v(\alpha)v(\beta)v(\gamma)+v(\alpha)v(\beta)+1
        \end{align*}

        Por tanto, tenemos que $v(\alpha)v(\beta)v(\gamma)=v(\alpha)v(\beta)$.
        Como por hipótesis también tenemos que $v(\alpha\land\beta)=1$, entonces:
        \begin{align*}
            1 &= v(\alpha \land \beta)\\
            &= v(\alpha)v(\beta)
        \end{align*}

        Uniendo ambos resultados, tenemos que:
        \begin{align*}
            v(\alpha)v(\beta)v(\gamma)&=v(\alpha)v(\beta)\\
            1\cdot v(\gamma)&=1\\
            v(\gamma)&=1
        \end{align*}
        
        Por tanto, $v(\gamma)=1$, y la regla es correcta.

        \item $\alpha\rightarrow \gamma, \beta\rightarrow \gamma \vDash \alpha\lor\beta$

        Tenemos que es equivalente a:
        \begin{equation*}
            \begin{array}{l}
                \alpha\rightarrow\gamma\\
                \beta\rightarrow\gamma\\ \hline
                \multicolumn{1}{c}{\alpha\lor\beta \rightarrow\gamma}        
            \end{array}
        \end{equation*}

        Sea $v$ una asignación fija pero arbitraria a condición de cumplir:
        \begin{equation*}
            v(\alpha\rightarrow\gamma)=1=v(\beta\rightarrow\gamma)
        \end{equation*}

        Entonces:
        \begin{align*}
            1 &= v(\alpha)v(\gamma)+v(\alpha)+1\\
            1 &= v(\beta)v(\gamma)+v(\beta)+1
        \end{align*}

        Por tanto:
        \begin{align*}
            v(\alpha)v(\gamma) &= v(\alpha) \\
            v(\beta)v(\gamma) &= v(\beta)
        \end{align*}

        Tenemos entonces que:
        \begin{align*}
            v &(\alpha\lor\beta\rightarrow\gamma)
            = v(\alpha\lor\beta)v(\gamma)+v(\alpha\lor\beta)+1\\
            &= (v(\alpha)v(\beta)+v(\alpha)+v(\beta))v(\gamma) + (v(\alpha)v(\beta)+v(\alpha)+v(\beta))+1\\
            &= v(\alpha)v(\beta)v(\gamma)+v(\alpha)v(\gamma)+v(\beta)v(\gamma) + v(\alpha)v(\beta)+v(\alpha)+v(\beta)+1
        \end{align*}

        Aplicando las igualdades anteriores, tenemos que:
        \begin{align*}
            v (\alpha\lor\beta\rightarrow\gamma)
            &= v(\alpha)v(\beta)+v(\alpha)+v(\beta) + v(\alpha)v(\beta)+v(\alpha)+v(\beta)+1 \\
            &= 2v(\alpha)v(\beta)+2v(\alpha)+2v(\beta)+1 \\
            &= 1
        \end{align*}

        Por tanto, la regla es correcta.

        \item $\gamma\rightarrow\alpha, \gamma\rightarrow\beta \vDash \gamma\rightarrow\alpha\land\beta$

        Tenemos que es equivalente a:
        \begin{equation*}
            \begin{array}{l}
                \gamma\rightarrow\alpha\\
                \gamma\rightarrow\beta\\ \hline
                \multicolumn{1}{c}{\gamma\rightarrow\alpha\land\beta}
            \end{array}
        \end{equation*}

        Sea $v$ una asignación fija pero arbitraria a condición de cumplir:
        \begin{equation*}
            v(\gamma\rightarrow\alpha)=1=v(\gamma\rightarrow\beta)
        \end{equation*}

        Entonces:
        \begin{align*}
            1 &= v(\gamma)v(\alpha)+v(\gamma)+1\\
            1 &= v(\gamma)v(\beta)+v(\gamma)+1
        \end{align*}

        Por tanto, tenemos que:
        \begin{equation*}
            v(\gamma)v(\alpha) = v(\gamma) = v(\gamma)v(\beta)
        \end{equation*}

        Comprobemos que la regla es cierta:
        \begin{align*}
            v(\gamma\rightarrow\alpha\land\beta) &= v(\gamma)v(\alpha\land\beta)+v(\gamma)+1\\
            &= v(\gamma)(v(\alpha)v(\beta))+v(\gamma)+1\\
            &= v(\gamma)v(\alpha)v(\beta)+v(\gamma)+1\\
            &\AstIg v(\gamma)v(\beta)+v(\gamma)+1\\
            &= v(\gamma)+v(\gamma)+1\\
            &= 1
        \end{align*}
        donde en $(\ast)$ hemos usado las hipótesis.
        Por tanto, la regla es correcta.

        \item $(\alpha\land\beta)\rightarrow\gamma,~\alpha,~\beta \vDash \gamma$

        Tenemos que es equivalente a:
        \begin{equation*}
            \begin{array}{l}
                (\alpha\land\beta)\rightarrow\gamma\\
                \alpha\\
                \beta\\ \hline
                \multicolumn{1}{c}{\gamma}
            \end{array}
        \end{equation*}

        Sea $v$ una asignación fija pero arbitraria a condición de cumplir:
        \begin{equation*}
            v((\alpha\land\beta)\rightarrow\gamma)=1=v(\alpha)=v(\beta)
        \end{equation*}

        Entonces:
        \begin{align*}
            1 &= v((\alpha\land\beta)\rightarrow\gamma) \\
            &= v(\alpha\land\beta)v(\gamma)+v(\alpha\land\beta)+1\\
            &= v(\alpha)v(\beta)v(\gamma)+v(\alpha)v(\beta)+1
        \end{align*}

        Usando que $v(\alpha)=v(\beta)=1$, tenemos que:
        \begin{equation*}
            1 = v(\gamma)+1+1 \Longrightarrow v(\gamma)=1
        \end{equation*}

        Por tanto, la regla es correcta.

        \item $\alpha\lor\beta,~\lnot\alpha\lor \gamma\vDash \beta\lor\gamma$. Esta regla se conoce como la regla de resolución.

        Tenemos que es equivalente a:
        \begin{equation*}
            \begin{array}{l}
                \alpha\lor\beta\\
                \lnot\alpha\lor\gamma\\ \hline
                \multicolumn{1}{c}{\beta\lor\gamma}
            \end{array}
        \end{equation*}

        Sea $v$ una asignación fija pero arbitraria a condición de cumplir:
        \begin{equation*}
            v(\alpha\lor\beta)=1=v(\lnot\alpha\lor\gamma)
        \end{equation*}

        Entonces:
        \begin{align*}
            1 &= v(\alpha\lor\beta) \\
            &= v(\alpha)v(\beta)+v(\alpha)+v(\beta)\\ \\
            1 &= v(\lnot\alpha\lor\gamma) \\
            &= v(\lnot \alpha)v(\gamma)+v(\lnot \alpha)+v(\gamma)\\
            &= (v(\alpha)+1)v(\gamma)+v(\alpha)+1+v(\gamma)\\
            &= v(\alpha)v(\gamma)+v(\gamma)+v(\alpha)+1+v(\gamma)\\
            &= v(\alpha)v(\gamma)+v(\alpha)+1
        \end{align*}

        Por tanto, tenemos que:
        \begin{align*}
            1 &= v(\alpha)v(\beta)+v(\alpha)+v(\beta)\\
            v(\alpha) &= v(\alpha)v(\gamma)
        \end{align*}

        Veamos ahora que la regla es correcta:
        \begin{align*}
            v(\beta\lor\gamma)
            &= v(\beta)v(\gamma)+v(\beta)+v(\gamma)\\
            &= (v(\alpha)v(\beta)+v(\alpha)+1)v(\gamma) + v(\beta) + v(\gamma)\\
            &= v(\alpha)v(\beta)v(\gamma)+v(\alpha)v(\gamma)+\cancel{v(\gamma)} + v(\beta) + \cancel{v(\gamma)}\\
            &= v(\alpha)v(\beta)+v(\alpha)+ v(\beta)\\
            &= v(\alpha\lor \beta)\\
            &= 1
        \end{align*}

        Por tanto, la regla es correcta. Hagamos ahora otro razonamiento distinto:
        \begin{itemize}
            \item Si $v(\alpha\lor\beta)=1$, entonces $v(\alpha)=1$ ó $v(\beta)=1$. 

            \item Si $v(\lnot\alpha\lor\gamma)=1$, entonces $v(\alpha)=0$ ó $v(\gamma)=1$.
        \end{itemize}
                
        Si $v(\beta)=1$, entonces $v(\beta\lor\gamma)=1$.
        Supongamos por tanto que $v(\beta)=0$, y demostremos que $v(\gamma)=1$.
        Como $v(\beta)=0$, entonces $v(\alpha)=1$, por lo que $v(\alpha)\neq 0$ y por tanto, $v(\gamma)=1$, de donde $v(\beta\lor\gamma)=1$.
        Queda demostrado por tanto que la regla es correcta.

        \item $\lnot\alpha\rightarrow\beta,~\lnot\alpha\rightarrow\lnot\beta\vDash\alpha$. Esta regla se conoce como la regla de reducción al absurdo clásica.
        
        Tenemos que es equivalente a:
        \begin{equation*}
            \begin{array}{l}
                \lnot\alpha\rightarrow\beta\\
                \lnot\alpha\rightarrow\lnot\beta\\ \hline
                \multicolumn{1}{c}{\alpha}
            \end{array}
        \end{equation*}

        Sea $v$ una asignación fija pero arbitraria a condición de cumplir:
        \begin{equation*}
            v(\lnot\alpha\rightarrow\beta)=1=v(\lnot\alpha\rightarrow\lnot\beta)
        \end{equation*}

        Entonces:
        \begin{align*}
            1 &= v(\lnot\alpha\rightarrow\beta)\\
            &= v(\lnot\alpha)v(\beta)+v(\lnot\alpha)+1\\
            &= (v(\alpha)+1)v(\beta)+v(\alpha)+1+1\\
            &= v(\alpha)v(\beta)+v(\beta)+v(\alpha)+1+1\\
            &= v(\alpha)v(\beta)+v(\alpha)+v(\beta)\\ \\
            1 &= v(\lnot\alpha\rightarrow\lnot\beta)\\
            &= v(\lnot\alpha)v(\lnot\beta)+v(\lnot\alpha)+1\\
            &= (v(\alpha)+1)(v(\beta)+1)+v(\alpha)+\cancel{1}+\cancel{1}\\
            &= v(\alpha)v(\beta)+\cancel{v(\alpha)}+v(\beta)+1+v\cancel{(\alpha)}\\
            &= v(\alpha)v(\beta)+v(\beta)+1
        \end{align*}

        Por tanto, tenemos que:
        \begin{align*}
            1 &= v(\alpha)v(\beta)+v(\alpha)+v(\beta)\\
            v(\alpha)v(\beta) = v(\beta)
        \end{align*}

        De forma directa, tenemos que $v(\alpha)=1$, y por tanto, la regla es correcta.

        \item $\alpha\rightarrow\beta,\alpha\rightarrow\lnot\beta\vDash\lnot\alpha$. Esta regla se conoce como la regla de reducción al absurdo intuicionista.
        
        Tenemos que es equivalente a:
        \begin{equation*}
            \begin{array}{l}
                \alpha\rightarrow\beta\\
                \alpha\rightarrow\lnot\beta\\ \hline
                \multicolumn{1}{c}{\lnot\alpha}
            \end{array}
        \end{equation*}

        Sea $v$ una asignación fija pero arbitraria a condición de cumplir:
        \begin{equation*}
            v(\alpha\rightarrow\beta)=1=v(\alpha\rightarrow\lnot\beta)
        \end{equation*}

        Entonces:
        \begin{align*}
            1 &= v(\alpha\rightarrow\beta)\\
            &= v(\alpha)v(\beta)+v(\alpha)+1\\ \\
            1 &= v(\alpha\rightarrow\lnot\beta)\\
            &= v(\alpha)v(\lnot\beta)+v(\alpha)+1\\
            &= v(\alpha)(v(\beta)+1)+v(\alpha)+1\\
            &= v(\alpha)v(\beta)+\cancel{v(\alpha)}+\cancel{v(\alpha)}+1\\
            &= v(\alpha)v(\beta)+1
        \end{align*}

        Por tanto, tenemos que:
        \begin{align*}
            v(\alpha)&= v(\alpha)v(\beta)\\
            v(\alpha)v(\beta)&= 0
        \end{align*}

        Por tanto, tenemos que $v(\alpha)=0$, por lo que
        $v(\lnot\alpha)=v(\alpha)+1=1$, y por tanto, la regla es correcta.

        \item $\alpha\rightarrow\beta,\lnot\alpha\rightarrow\beta\vDash\beta$. Esta regla se conoce como la regla de demostración por casos.
        
        Tenemos que es equivalente a:
        \begin{equation*}
            \begin{array}{l}
                \alpha\rightarrow\beta\\
                \lnot\alpha\rightarrow\beta\\ \hline
                \multicolumn{1}{c}{\beta}
            \end{array}
        \end{equation*}

        Sea $v$ una asignación fija pero arbitraria a condición de cumplir:
        \begin{equation*}
            v(\alpha\rightarrow\beta)=1=v(\lnot\alpha\rightarrow\beta)
        \end{equation*}

        Entonces:
        \begin{align*}
            1 &= v(\alpha\rightarrow\beta)\\
            &= v(\alpha)v(\beta)+v(\alpha)+1\\ \\
            1 &= v(\lnot\alpha\rightarrow\beta)\\
            &= v(\lnot\alpha)v(\beta)+v(\lnot\alpha)+1\\
            &= (v(\alpha)+1)v(\beta)+v(\alpha)+1+1\\
            &= v(\alpha)v(\beta)+v(\beta)+v(\alpha)
        \end{align*}

        De la primera hipótesis deducimos que $v(\alpha)v(\beta)=v(\alpha)$,
        y por tanto de la segunda hipótesis deducimos que $v(\beta)=1$, y por tanto, la regla es correcta.
    \end{enumerate}

\end{ejercicio*}

\begin{observacion}
    Sean $A,B$ dos conjuntos, y sea $f:A\to B$ una función. Entonces, definimos la aplicación $f_\ast$ por:
    \Func{f_\ast}{\cc{A}}{B}{C}{\{f(x)\mid x\in C\}}

    Notemos que, dado $\Gamma \cup \{\alpha\}$ un conjunto de fórmulas proposicionales,
    se tiene que $\Gamma \vDash \alpha$ si y sólo si, para toda valoración $v$, se tiene que $v(\alpha)=1$ siempre que $v_\ast(\Gamma)\subseteq \{1\}$.
\end{observacion}

\begin{prop}
    Dada una fórmula $\alpha$, demostrar que
    $\alpha$ es una tautología si y sólo si $\vDash\alpha$.
    \begin{proof}

        Tenemos que notar $\vDash\alpha$ es equivalente a notar $\emptyset\vDash\alpha$.
        Dada una valoración $v$, tenemos que $v_\ast(\emptyset )=\emptyset \subset \{1\}$.
        
        Por tanto, sabemos que que $\vDash\alpha$ si y sólo si
        para toda asignación se tiene que $v(\alpha)=1$ y esto se da si y sólo si $\alpha$ es una tautología.
    \end{proof}
    
\end{prop}

\begin{prop} Veamos algunos resultados sobre conjuntos satisfacibles.

    \begin{enumerate}
        \item El conjunto $\emptyset $ es satisfacible.
        \item Existen conjuntos insatisfacibles.
        \item Si $\Delta$ es insatisfacible y $\Delta \subseteq \Gamma$, entonces $\Gamma$ es insatisfacible.
        \item $\{\alpha,\alpha\rightarrow\lnot\alpha\}$ es insatisfacible.
    \end{enumerate}
    \begin{proof}
        \ 
        \begin{enumerate}
            \item Razonamos por vacuidad. Dado una fórmula $a_0$, sea $v=\chi_{\{a_0\}}$.

            Si $\emptyset $ fuese insatisfacible, existería $\varphi_v\in \emptyset $ tal que $v(\varphi_v)=0$, lo cual es absurdo.
            
            \item Sea el conjunto $\{a, \lnot a\}$. Sea $v$ tal que $v(a)=1$. Entonces $v(\lnot a) = v(a)+1 = 0$,
            por lo que dicho junto no es satisfacible.
            
            \item Como $\Delta$ es insatisfacible, entonces para toda valoraicón $v$ se tiene que existe $\varphi_v\in \Delta\subset \Gamma$ tal que $v(\varphi_v)=0$,
            por lo que $\Gamma$ es insatisfacible.

            \item Sea $v$ tal que $v(\alpha)=1$. Entonces:
                \begin{align*}
                    v(\alpha\rightarrow\lnot\alpha)&=v(\alpha)v(\lnot\alpha)+v(\alpha)+1\\
                    &= 1\cdot 0+1+1 \\
                    &= 0 + 0 = 0
                \end{align*}
                Si $\{\alpha,\alpha\rightarrow\lnot\alpha\}$ fuera satisfacible, debería ser $v(\alpha)=1$ y $v(\alpha\rightarrow\lnot\alpha)=1$, pero si $v(\alpha)=1$, entonces $v(\alpha\rightarrow\lnot\alpha)=0$.
        \end{enumerate}
    \end{proof}
\end{prop}

\begin{prop}
    Sea $\Gamma$ un conjunto de fórmulas proposicionales. Entonces,
    \begin{equation*}
        \Con(\Gamma)=\Con(\Con(\Gamma))
    \end{equation*}
    \begin{proof}
        Demostramos por doble inclusión:
        \begin{description}
            \item[$\subset$)] Como $\Gamma\subset \Con(\Gamma)$, entonces $\Con(\Gamma)\subset \Con(\Con(\Gamma))$.
            
            \item[$\supset$)] Dado $\alpha\in \Con(\Con(\Gamma))$, entonces por definición $\Con(\Gamma)\vDash\alpha$.
            
            Para ver que $\alpha\in \Con(\Gamma)$, basta ver que $\Gamma\vDash\alpha$. Sea $v$ una valoración con $v_\ast(\Gamma)\subseteq \{1\}$.
            Por tanto, $v_\ast(\Con(\Gamma))\subseteq \{1\}$, y por tanto $v(\alpha)=1$, por lo que $\Gamma\vDash\alpha$.
        \end{description}
    \end{proof}
\end{prop}


\begin{ejercicio}
    Sea $\Gamma \cup \{\alpha,\beta\}$ un conjunto de fórmulas proposicionales.
    Demostrar que si $\Gamma \vDash \alpha$ y $\Gamma \vDash \alpha\rightarrow \beta$, entonces $\Gamma \vDash \beta$.
    Es decir, si $\alpha,\alpha\to \beta \in \Con(\Gamma)$, entonces $\beta\in \Con(\Gamma)$.
    \begin{proof}
        Demostremos en primer lugar que $\Con(\Gamma)\vDash \beta$.
        Sea $v$ una valoración tal que $v_\ast(\Con(\Gamma))\subseteq \{1\}$. Entonces, tendremos que:
        \begin{equation*}
            v(\alpha)=1=v(\alpha\rightarrow\beta)
        \end{equation*}
        
        Por la regla de \emph{modus ponens}, como $\alpha,\alpha\rightarrow\beta\vDash \beta$, entonces $v(\beta)=1$.
        Por tanto, $\Con(\Gamma)\vDash \beta$.
        Deducimos entonces que $\beta \in \Con(\Con(\Gamma)) = \Con(\Gamma)$, por lo que $\Gamma \vDash \beta$.\\

        Este resultado se resume diciendo que \ul{$\Con(\Gamma)$ es cerrado por \emph{modus ponens}}.
    \end{proof}
\end{ejercicio}

\begin{prop}
    Para toda $\alpha,~\beta$ fórmulas proposicionales, $\beta\in \Con(\{\alpha\land\lnot\alpha\})$.
    \begin{proof}
        Sabemos que $(\alpha\land \lnot \alpha)\to \alpha,~(\alpha\land \lnot \alpha)\to \lnot \alpha$ son tautologías, por lo que:
        \begin{equation*}
            (\alpha\land \lnot \alpha)\to \alpha,~(\alpha\land \lnot \alpha)\to \lnot \alpha \in \Con(\emptyset)
        \end{equation*}

        Además, $\alpha\land \lnot \alpha~\in \Con(\alpha\land \lnot \alpha)$.
        Como $\emptyset\subset \{\alpha\land \lnot \alpha\}$, tenemos que:
        \begin{equation*}
            \alpha\land \lnot \alpha,~(\alpha\land \lnot \alpha)\to \alpha,~(\alpha\land \lnot \alpha)\to \lnot \alpha \in \Con(\alpha\land \lnot \alpha)
        \end{equation*}

        Por la Regla de \emph{modus ponens}, tenemos que $\alpha,~\lnot \alpha\in \Con(\alpha\land \lnot \alpha)$.
        Además, tenemos que otra tautología es:
        \begin{equation*}
            \alpha\rightarrow(\lnot\alpha\rightarrow\beta)\in \Con(\alpha\land \lnot \alpha)
        \end{equation*}

        Aplicando la Regla de \emph{modus ponens} dos veces de forma consecutiva, primero llegamos a que
        $\not\alpha\rightarrow\beta \in \Con(\alpha\land \lnot \alpha)$,
        y posteriormente llegamos a que se tiene que $\beta \in \Con(\alpha\land \lnot \alpha)$,
        que es lo que buscábamos demostrar.

    \end{proof}
\end{prop}



\begin{ejercicio}
    Demuestre que para todo conjunto de fórmulas $\Gamma\cup\{\alpha,\beta\}$, se cumple
    \begin{equation*}
        \Con(\Gamma,\alpha\lor\beta) = \Con(\Gamma,\alpha) \cap \Con(\Gamma,\beta)
    \end{equation*}

    \begin{proof}
        La demostración es por doble inclusión.
        \begin{description}
            \item [$\subseteq$)] Veamos en primer lugar que $\alpha, \alpha\rightarrow\alpha\lor\beta\in \Con(\Gamma,\alpha)$.
            
                Sabemos que $\alpha\rightarrow\alpha\lor\beta$ es una tautología, luego
                $\alpha\rightarrow\alpha\lor\beta\in \Con(\emptyset)$.
                Como $\emptyset\subseteq \Gamma\cup\{\alpha\}$, entonces $\Con(\emptyset)\subseteq \Con(\Gamma,\alpha)$, y por tanto $\alpha\rightarrow\alpha\lor\beta\in \Con(\Gamma,\alpha)$.

                Por otro lado, $\Gamma\cup \{\alpha\}\subseteq \Con(\Gamma,\alpha)$, y por tanto $\alpha\in \Con(\Gamma,\alpha)$.

                Por tanto, como $\Con(\Gamma,\alpha)$ es cerrado por \emph{modus ponens}, entonces tenemos que $\alpha\lor\beta\in \Con(\Gamma,\alpha)$.
                Como además $\Gamma\subset \Con(\Gamma,\alpha)$, entonces:
                \begin{equation*}
                    \Con(\Gamma,\alpha\lor\beta)\subseteq \Con(\Con(\Gamma,\alpha)) = \Con(\Gamma,\alpha)
                \end{equation*}

                Razonando de igual forma, tenemos que $\Con(\Gamma,\alpha\lor\beta)\subseteq \Con(\Gamma,\beta)$, por lo que:
                \begin{equation*}
                    \Con(\Gamma,\alpha\lor\beta)\subseteq \Con(\Gamma,\alpha)\cap \Con(\Gamma,\beta)
                \end{equation*}

            \item [$\supseteq$)] Sea $\gamma\in \Con(\Gamma,\alpha)\cap \Con(\Gamma,\beta)$, es decir:
                \begin{align*}
                    \gamma\in \Con(\Gamma,\alpha) &\Longrightarrow \Gamma \cup \{\alpha\} \vDash \gamma \\
                    \gamma\in \Con(\Gamma,\beta) &\Longrightarrow \Gamma \cup \{\beta\} \vDash \gamma
                \end{align*}
                
                Tenemos que demostrar que $\gamma\in \Con(\Gamma,\alpha\lor\beta)$, es decir, que $\Gamma\cup\{\alpha\lor\beta\}\vDash \gamma$. Demostremos antes el siguiente resultdado.\\

                Sea $v$ una asignación fija. Veamos que si $v(\alpha\lor \beta)=1$, entonces $v(\alpha)=1$ ó $v(\beta)=1$. Por reducción al absurdo, supongamos que no, es decir, que $v(\alpha)=0=v(\beta)$. Entonces:
                \begin{equation*}
                    1 = v(\alpha\lor \beta) = v(\alpha)v(\beta) + v(\alpha) + v(\beta) = 0
                \end{equation*}
                Por tanto, llegamos a una contradicción, por lo que $v(\alpha)=1$ o $v(\beta)=1$.\\


                Sabiendo esto, veamos si $\gamma\in \Con(\Gamma,\alpha\lor \beta)$. Sea $v$ una valoración tal que
                $v(\theta)=1$ para todo $\theta\in \Gamma$ y $v(\alpha\lor \beta)=1$. Por lo que acabamos de ver, $v(\alpha)=1$ ó $v(\beta)=1$, supongamos $v(\alpha)=1$ (el otro caso es idéntico).
                Entonces, como $\Gamma\cup \{\alpha\}\vDash \gamma$, entonces $v(\gamma)=1$, por lo que:
                $$\gamma\in \Con(\Gamma,\alpha\lor \beta)$$
                
                Por tanto, $\Con(\Gamma,\alpha)\cap \Con(\Gamma,\beta)\subseteq \Con(\Gamma,\alpha\lor\beta)$
        \end{description}
    \end{proof}
\end{ejercicio}

\begin{ejercicio*}
    Sea $\Gamma \cup \{\alpha,\beta\}$ un conjunto de fórmulas proposicionales.
    Entonces, si $\Gamma \vDash \lnot \alpha \rightarrow \beta$ y $\Gamma \vDash \alpha\rightarrow \psi$,
    entonces $\Gamma \vDash \lnot \beta \rightarrow \psi$ y $\Gamma \vDash \lnot \psi \rightarrow \beta$.
    \begin{proof}
        Sea $v$ una valoración tal que $v_\ast(\Gamma)\subseteq \{1\}$.
        Entonces, por hipótesis, tenemos que:
        \begin{align*}
            v(\lnot \alpha\rightarrow \beta)  = 1 = v(\alpha\rightarrow \psi)
        \end{align*}

        Entonces, tenemos:
        \begin{align*}
            1 &= v(\lnot \alpha \rightarrow \beta)\\
            &=v(\lnot \alpha)v(\beta)+v(\lnot \alpha)+1\\
            &= (v(\alpha)+1)v(\beta)+v(\alpha)+1+1\\
            &= v(\alpha)v(\beta)+v(\beta)+v(\alpha)\\
            &= v(\alpha\lor \beta)\\\\
            1 &= v(\alpha\rightarrow \psi)\\
            &= v(\alpha)v(\psi)+v(\alpha)+1
        \end{align*}

        Veamos que es cierto lo dicho. Tenemos que:
        \begin{align*}
            v(\lnot \beta\rightarrow \psi)
            &= v(\lnot \beta)v(\psi)+v(\lnot \beta)+1\\
            &= (v(\beta)+1)v(\psi)+v(\beta)+1+1\\
            &= v(\beta)v(\psi)+v(\psi)+v(\beta)\\
            &= v(\beta\lor \psi)\\ \\
            v(\lnot \psi\rightarrow \beta)
            &= v(\lnot \psi)v(\beta)+v(\lnot \psi)+1\\
            &= (v(\psi)+1)v(\beta)+v(\psi)+1+1\\
            &= v(\psi)v(\beta)+v(\beta)+v(\psi)\\
            &= v(\psi\lor \beta)
        \end{align*}

        Por tanto, nuestro problema se reduce a demostrar que $v(\alpha\lor \beta)=1$.
        \begin{itemize}
            \item Si $v(\beta)=1$, entonces $v(\alpha\lor \beta)=1$.
            \item Si $v(\beta)=0$, entonces por la primera hipótesis tenemos que $v(\alpha)=1$, y por la
            segunda hipótesis tenemos que $v(\psi)=1$, por lo que $v(\alpha\lor \beta)=1$. 
        \end{itemize}

        En cualquier caso, tenemos que $v(\alpha\lor \beta)=1$, por lo que $\Gamma \vDash \lnot \beta \rightarrow \psi$ y $\Gamma \vDash \lnot \psi \rightarrow \beta$.
    \end{proof}
\end{ejercicio*}


\begin{teo}[de la Reducción]
    Sea $\Gamma\cup \{\alpha,\beta\}$ un conjunto de fórmulas proposicionales. Equivalen:
    \begin{enumerate}
        \item $\Gamma,\alpha\vDash \beta$
        \item $\Gamma\vDash \alpha\rightarrow \beta$
    \end{enumerate}
    \begin{proof}
        La demostración es mediante doble implicación.
        \begin{description}
            \item[$1\Longrightarrow 2$)] Supongamos que $\Gamma,\alpha\vDash \beta$.
            Entonces, sea $v$ una valoración tal que $v_\ast(\Gamma)\subseteq \{1\}$.
            Caben dos posibilidades:
            \begin{itemize}
                \item $v(\alpha)=0$. Entonces, $v(\alpha\rightarrow \beta)=1$, por lo que $\Gamma\vDash \alpha\rightarrow \beta$.
                \item $v(\alpha)=1$. Entonces, por hipótesis tenemos que $v(\beta)=1$, por lo que $v(\alpha\rightarrow \beta)=1$, y por tanto $\Gamma\vDash \alpha\rightarrow \beta$.
            \end{itemize}

            \item[$2 \Longrightarrow 1$)] 
            Tenemis que $\alpha\to \beta\in \Con(\Gamma)\subset \Con(\Gamma,\alpha)$.
            Como $\alpha\in \Con(\Gamma,\alpha)$, entonces por la regla de \emph{modus ponens} tenemos que $\beta\in \Con(\Gamma,\alpha)$, es decir, $\Gamma,\alpha\vDash \beta$, que es lo que queríamos demostrar.
        \end{description}
    \end{proof}
\end{teo}

\begin{teo}
    Sea $\Gamma\cup \{\alpha\}$ un conjunto de fórmulas proposicionales. Entonces, equivalen:
    \begin{enumerate}
        \item $\Gamma\vDash \alpha$
        \item $\Gamma\cup \{\lnot \alpha\}$ es insatisfacible.
    \end{enumerate}

    \begin{proof}
        La demostración es mediante doble implicación.
        \begin{description}
            \item[$1\Longrightarrow 2$)] Supongamos que $\Gamma\vDash \alpha$, y sea $v$ una valoración fija pero arbitraria.
            Si $\exists \varphi\in \Gamma$ tal que $v(\varphi)=0$, entonces ya tenemos que $\Gamma \cup \{\lnot \alpha\}$ es insatisfacible.
            En caso contrario, tenemos que $v_\ast(\Gamma)\subseteq \{1\}$, por lo que $v(\alpha)=1$, y por tanto $v(\lnot \alpha)=0$, por lo que $\Gamma\cup \{\lnot \alpha\}$ es insatisfacible.

            \item[$2\Longrightarrow 1$)] Supongamos que $\Gamma\cup \{\lnot \alpha\}$ es insatisfacible,
            y sea $v$ una valoración tal que $v_\ast(\Gamma)\subseteq \{1\}$.
            Como $\Gamma\cup \{\lnot \alpha\}$ es insatisfacible, entonces $v(\lnot \alpha)=0$, por lo que $v(\alpha)=0$.
        \end{description}
    \end{proof}
\end{teo}
\begin{observacion}
    En numerosas ocasiones, estudiar la insatisfacibilidad de un conjunto de fórmulas es más sencillo que estudiar la validez de una fórmula
    (es decir, es más fácil estudiar el punto 2 que el punto 1 en el teorema anterior). Usaremos por tanto dicho resultado para demostrar la validez de una fórmula.
\end{observacion}


\begin{ejemplo}[Regla de la Cláusula Unit]
    Veamos algunos ejemplos de la Regla 2 del Algoritmo de Davis y Putnam.
    \begin{enumerate}
        \item Sea $\Sigma = \{a,a\lor b,\lnot a\lor b,\lnot a\lor b\lor c\}$ un conjunto de cláusulas.
        
        Tan solo cuenta con una cláusula unitaria, $\lm=a$. Eliminamos todas sus ampliaciones, y obtenemos:
        \begin{equation*}
            \Sigma' = \{\lnot a\lor b,\lnot a\lor b\lor c\}
        \end{equation*}

        Como $\Sigma'\neq \emptyset$, eliminamos ahora de cada cláusula de $\Sigma'$ todas las ocurrencias de la cláusula unitaria $\lm^c = \lnot a$:
        \begin{equation*}
            \Sigma'' = \{b,b\lor c\}
        \end{equation*}

        Por la regla 2, sabemos que $\Sigma''$ es insatisfacible si y solo si lo es $\Sigma$.

        \item Sea $\Sigma = \{a, a\lor b,a\lor \lnot c\}$ un conjunto de cláusulas.
        
        Tan solo cuenta con una cláusula unitaria, $\lm=a$. Eliminamos todas sus ampliaciones, y obtenemos:
        \begin{equation*}
            \Sigma' = \emptyset
        \end{equation*}

        Por tanto, por la regla 2, sabemos que $\Sigma$ es satisfacible.
    \end{enumerate}
\end{ejemplo}


\begin{ejemplo}[Regla del Literal Puro]
    Veamos algunos ejemplos de la Regla 3 del Algoritmo de Davis y Putnam.
    \begin{enumerate}
        \item Sea $\Sigma = \{\lnot a\lor b, \lnot a\lor c,\lnot b\lor c,b\lor \lnot d\}$ un conjunto de cláusulas.
        
        Tenemos que $\lm=\lnot a$ es un literal puro en $\Sigma$, puesto que $\lm^c=a$ no aparece en ninguna otra cláusula.
        Eliminamos todas las ampliaciones de $\lm$, y obtenemos:
        \begin{equation*}
            \Sigma' = \{\lnot b\lor c,b\lor \lnot d\}
        \end{equation*}

        Por la regla 3, sabemos que $\Sigma'$ es insatisfacible si y solo si lo es $\Sigma$.

        \item Sea $\Sigma = \{\lnot a\lor b, a\lor \lor b\}$ un conjunto de cláusulas.
        
        En este caso, no hay literales puros, por lo que no podemos aplicar la regla 3.
    \end{enumerate}
\end{ejemplo}


\begin{ejemplo}[Regla de Descomposición]
    Veamos un ejemplo de la Regla 4 del Algoritmo de Davis y Putnam.
    Sea el siguiente conjunto de cláusulas:
    \begin{equation*}
        \Sigma = \{a\lor \lnot b\lor \lnot c, b\lor d,\lnot a\lor b\lor c,b\lor \lnot d\}
    \end{equation*}

    Apliquemos las reglas vistas:
    \begin{enumerate}
        \item $\Sigma$ no tiene cláusulas tautológicas.
        \item $\Sigma$ no tiene cláusulas unitarias.
        \item $\Sigma$ no tiene literales puros.
        \item Considerando $\lm=a$, podemos dividir las cláusulas como sigue:
        \begin{align*}
            \Omega &= \{b\lor d,b\lor \lnot d\}\\
            \Sigma_1 &= \{\lnot b\lor \lnot c\} \cup \Omega = \{\lnot b\lor \lnot c,b\lor d,b\lor \lnot d\}\\
            \Sigma_2 &= \{b\lor c\} \cup \Omega = \{b\lor c,b\lor d,b\lor \lnot d\}
        \end{align*}

        Por la regla 4, sabemos que $\Sigma$ es insatisfacible si y solo si lo son $\Sigma_1$ y $\Sigma_2$.
    \end{enumerate}
\end{ejemplo}

\begin{ejercicio}\label{ej:1.3.4}
    Sean las siguientes fórmulas proposicionales:
    \begin{align*}
        \gamma_1 &= (a\lor b)\rightarrow (c\lor d)\\
        \gamma_2 &= (\lnot a\land \lnot d)\rightarrow (\lnot c\land (c\lor e))\\
        \gamma_3 &= a \rightarrow (\lnot c\land \lnot b \land (\lnot d\lor b))\\
        \varphi &= (d\rightarrow(b\lor a)) \rightarrow (d\land \lnot (a\lor \lnot b))        
    \end{align*}
    Estudie si $\gamma_1,\gamma_2,\gamma_3\vDash \varphi$; y en caso de no serlo dé una asignación que lo evidencie.

    Estudiaremos la satisfacibilidad del conjunto $\{\gamma_1,\gamma_2,\gamma_3,\lnot \varphi\}$
    mediante el Algoritmo de Davis y Putnam. Para hello, en primer
    lugar hemos de transformar las fórmulas a su forma normal conjuntiva.
    \begin{align*}
        \gamma_1 &= (a\lor b)\rightarrow (c\lor d)\\
        &= \lnot (a\lor b)\lor (c\lor d)\\
        &= (\lnot a\land \lnot b)\lor (c\lor d)\\
        &= (\lnot a\lor c\lor d)\land (\lnot b\lor c\lor d)\\ \\
        \gamma_2 &= (\lnot a\land \lnot d)\rightarrow (\lnot c\land (c\lor e))\\
        &= \lnot (\lnot a\land \lnot d)\lor (\lnot c\land (c\lor e))\\
        &= (\lnot \lnot a \lor \lnot \lnot d)\lor (\lnot c\land (c\lor e))\\
        &= (a\lor d)\lor (\lnot c\land (c\lor e))\\
        &= (a\lor d \lor \lnot c)\land (a\lor d\lor c\lor e)\\ \\
        \gamma_3 &= a \rightarrow (\lnot c\land \lnot b \land (\lnot d\lor b))\\
        &= \lnot a\lor (\lnot c\land \lnot b \land (\lnot d\lor b))\\
        &= (\lnot a\lor \lnot c) \land (\lnot a\lor \lnot b) \land (\lnot a\lor \lnot d\lor b)\\ \\
        \lnot \varphi &= \lnot \left[(d\rightarrow(b\lor a)) \rightarrow (d\land \lnot (a\lor \lnot b))\right]\\
        &= \lnot \left[\lnot (d\rightarrow(b\lor a))\lor (d\land \lnot (a\lor \lnot b))\right]\\
        &= \lnot \left[\lnot (\lnot d\lor b\lor a)\lor (d\land \lnot (a\lor \lnot b))\right]\\
        &= (a\lor b\lor \lnot d)\land \lnot(d\land \lnot a\land b)\\
        &= (a\lor b\lor \lnot d)\land (a \lor \lnot b\lor \lnot d)
    \end{align*}

    Queremos por tanto estudiar la satisfacibilidad del conjunto $\Sigma_0$:
    \begin{align*}
        \Sigma_0 &= \{\gamma_1,\gamma_2,\gamma_3,\lnot \varphi\}\\
        &= \{(\lnot a\lor c\lor d)\land (\lnot b\lor c\lor d),\\
        &\qquad (a\lor d \lor \lnot c)\land (a\lor d\lor c\lor e),\\
        &\qquad (\lnot a\lor \lnot c) \land (\lnot a\lor \lnot b) \land (\lnot a\lor \lnot d\lor b),\\
        &\qquad (a\lor b\lor \lnot d)\land (a \lor \lnot b\lor \lnot d)\}
    \end{align*}

    Estudiar la satisfacibilidad de $\Sigma_0$ equivale a estudiar la satisfacibilidad de $\Sigma$:
    \begin{align*}
        \Sigma &= \{(\lnot a\lor c\lor d),\quad (\lnot b\lor c\lor d),\\
        &\qquad (a\lor d \lor \lnot c),\quad (a\lor d\lor c\lor e),\\
        &\qquad (\lnot a\lor \lnot c),\quad (\lnot a\lor \lnot b),\quad (\lnot a\lor \lnot d\lor b),\\
        &\qquad (a\lor b\lor \lnot d),\quad (a \lor \lnot b\lor \lnot d)\}\\
        &= \{\lnot a\lor c\lor d,\\
        &\qquad \lnot b\lor c\lor d,\\
        &\qquad a\lor d \lor \lnot c,\\
        &\qquad a\lor d\lor c\lor e,\\
        &\qquad \lnot a\lor \lnot c,\\
        &\qquad \lnot a\lor \lnot b,\\
        &\qquad \lnot a\lor \lnot d\lor b,\\
        &\qquad a\lor b\lor \lnot d,\\
        &\qquad a \lor \lnot b\lor \lnot d\}
    \end{align*}

    Aplicamos ahora el Algoritmo de Davis y Putnam para estudiar la satisfacibilidad de $\Sigma$,
    representado en la Figura~\ref{fig:DyP_1.3.4}.
    % // TODO: Continuar
    \begin{figure}
        \centering
        \begin{forest}
            for tree={draw=none, minimum size=2em, l=2cm, s sep=5mm, align=center, edge={-stealth}}
            [
                $\Sigma$\\ \fbox{R3. $\lm=e$ es un literal puro. {\color{red}$v(e)=1$}}
                [
                    $\Sigma_{1} {=} \{\lnot a\lor c\lor d; \lnot b\lor c\lor d; a\lor \lnot c \lor d; \lnot a\lor \lnot c;$\\
                    \hspace{1.8cm}$\lnot a\lor \lnot b;\lnot a\lor b\lor \lnot d;a\lor b \lor \lnot d;a\lor \lnot b\lor \lnot d\}$\\\\
                    \fbox{R4. $\lm=\lnot a$. {\color{red}$v(\lnot a)=1$}}
                    [
                        \fbox{$\lm=\lnot a$}\\
                        $\Sigma_{11} {=}\{c\lor d; \lnot b\lor c\lor d; \lnot c; \lnot b;b\lor \lnot d\}$\\\\
                        \fbox{R2. $\lm=\lnot c$}\\
                        [
                            $\Sigma'_{11} {=}\{c\lor d; \lnot b\lor c\lor d; \lnot b;b\lor \lnot d\}{\color{red}~\neq \emptyset}$
                            [
                                $\Sigma_{111} {=}\{d; \lnot b\lor d; \lnot b;b\lor \lnot d\}$\\ \\
                                \fbox{R2. $\lm=d$}
                                [
                                    $\Sigma_{111}' {=}\{\lnot b;b\lor \lnot d\}{\color{red}~\neq \emptyset}$
                                    [
                                        $\Sigma_{1111} {=}\{\lnot b;b\}$\\ \\
                                        \fbox{R2. $\lm=\lnot b$}
                                        [
                                            $\Sigma_{1111}' {=}\{b\}{\color{red}=\{b\lor \square\}\neq \emptyset}$
                                            [
                                                $\Sigma_{11111} {=}\{\square\}$
                                            ]
                                        ]
                                    ]
                                ]
                            ]
                        ]
                    ]
                    [
                        \fbox{$\lm^c=a$}\\
                        $\Sigma_{12} {=}\{\lnot b\lor c\lor d; \lnot c \lor d; b \lor \lnot d;\lnot b\lor \lnot d\}$\\ \\
                        \fbox{R4. $\lm=\lnot b$. {\color{red}$v(\lnot b)=1$}}
                        [
                            \fbox{$\lm=\lnot b$}\\
                            $\Sigma_{121} {=}\{c\lor d; \lnot c \lor d; \lnot d\}$\\ \\
                            \fbox{R2. $\lm=\lnot d$}
                            [
                                $\Sigma'_{121} {=}\{c\lor d; \lnot c \lor d\}{\color{red}\neq \emptyset}$
                                [
                                    $\Sigma_{1211} {=}\{c; \lnot c\}$\\ \\
                                    \fbox{R2. $\lm=c$}
                                        [
                                            $\Sigma_{1211}' {=}\{\lnot c\}{\color{red}=\{\lnot c\lor \square\}\neq \emptyset}$
                                            [
                                                $\Sigma_{12111} {=}\{\square\}$
                                            ]
                                        ]
                                ]
                            ]
                        ]
                        [
                            \fbox{$\lm^c=b$}\\
                            $\Sigma_{122} {=}\{\lnot c \lor d; \lnot d\}$\\ \\
                            \fbox{R2. $\lm=\lnot d$. {\color{red}$v(\lnot d)=1$}}
                            [
                                $\Sigma_{122}' {=}\{\lnot c \lor d\}{\color{red}\neq \emptyset}$
                                [
                                    $\Sigma_{1222} {=}\{\lnot c\}$ \\ \\
                                    \fbox{R2. $\lm=\lnot c$. {\color{red}$v(\lnot c)=1$}}
                                    [
                                        $\Sigma_{1222}' {=}\emptyset$
                                    ]
                                ]
                            ]
                        ]
                    ]
                ]
            ]
        \end{forest}
        \caption{Algoritmo de Davis y Putman del Ejercicio~\ref{ej:1.3.4}.}
        \label{fig:DyP_1.3.4}
    \end{figure}
\end{ejercicio}