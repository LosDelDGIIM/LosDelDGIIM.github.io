\section{Recurrencia}

\begin{ejercicio}
    Resuelva la relación de recurrencia dada por $u_{n+2} = 4u_{n+1} - 4u_n$, para todo $n\geq 0$.
    Particularice el resultado suponiendo que $n\geq 0$, $u_0=1$, $u_1=3$.\\

    El orden de la recurrencia es $k=2$. La ecuación característica es:
    \begin{equation*}
        x^2-4x+4 = {(x-2)}^2=0
    \end{equation*}

    Por tanto, tan solo hay una raíz $r=2$ de multiplicidad $m=2$. La solución general de la recurrencia es:
    \begin{equation*}
        x_n = (c_1+c_2n)2^n
    \end{equation*}

    Y ahora buscar los valores de $c_1$ y $c_2$ usando las condiciones iniciales, para obtener la solución particular.
    Tenemos que $x_0=u_0=1$ y $x_1=u_1=3$, entonces:
    \begin{align*}
        1 &= (c_1+c_2\cdot 0) 2^0 = c_1\cdot 1 = c_1\\
        3 &= (c_1+c_2\cdot 1) 2^1 = (1+c_2) 2 = 2+2c_2 \Longrightarrow c_2 = \frac{3-2}{2} = \frac{1}{2}
    \end{align*}

    Por tanto, la solución particular es:
    \begin{equation*}
        x_n = \left(1+\frac{n}{2}\right)2^n = \left(\frac{2+n}{2}\right)2^n = (2+n)2^{n-1}
    \end{equation*}
\end{ejercicio}

\begin{ejercicio}\label{ej:recurrencia}
    Resuelva la recurrencia:
    \begin{equation*}
        u_{n+2} = u_{n+1} + u_n\qquad n\geq 0
    \end{equation*}

    El orden $k$ de la recurrencia es 2 ($k = 2$). La ecución característica:
    \begin{equation*}
        x^2 -x -1 = 0 
    \end{equation*}

    que tiene por soluciones
    \begin{equation*}
        x = \dfrac{1\pm \sqrt{1+4}}{2} 
    \end{equation*}

    En definitiva
    \begin{gather*}
        r_1 = \dfrac{1+\sqrt{5}}{2} \qquad m_1 = 1 \\
        r_2 = \dfrac{1+\sqrt{5}}{2} \qquad m_2 = 1
    \end{gather*}

    y se tiene: $m_1+m_2 = 1 +1 = 2 = k$

    Si $\{x_n\}$ es solución de la recurrencia, entonces sabemos que para todo $n \geq 0$:
    \begin{align*}
        x_n &= c_1 r_1^n + c_2 r_2^n\\
            &= c_1{\left(\dfrac{1+\sqrt{5}}{2}\right)}^n + c_2 {\left(\dfrac{1-\sqrt{5}}{2}\right)}^n
    \end{align*}

    para ciertos valores de $c_1$ y $c_2$.
\end{ejercicio}

\begin{ejercicio}
    Reuelva el problema de recurrencia 
    \begin{gather*}
        u_0 = 0\\    
        u_1 = 1 \\
        u_{n+2} = u_{n+1} + u_n \qquad n \geq 0
    \end{gather*}

    Por el ejercicio anterior, sabemos que la solución a la relación de recurrencia del problema es:
    \begin{equation*}
        x_n = c_1{\left(\dfrac{1+\sqrt{5}}{2}\right)}^n + c_2 {\left(\dfrac{1-\sqrt{5}}{2}\right)}^n
    \end{equation*}

    Si $\{x_n\}$ es solución del problema entonces $x_0 = u_0 = 0$ y $x_1 = u_1 = 1$
    \begin{align*}
        0 &= x_0 \\
          &= c_1{\left(\dfrac{1+\sqrt{5}}{2}\right)}^0 + c_2 {\left(\dfrac{1-\sqrt{5}}{2}\right)}^0\\
          &= c_1 + c_2 \Longrightarrow c_2 = -c_1
    \end{align*}
    \begin{align*}
        1 &= x_1 \\
          &= c_1 {\left(\dfrac{1+\sqrt{5}}{2}\right)}^1 + c_2 {\left(\dfrac{1-\sqrt{5}}{2}\right)}^1 \\
          &= c_1 \left(\dfrac{1+\sqrt{5}}{2}\right) - c_1 \left(\dfrac{1-\sqrt{5}}{2}\right) \\
          &= c_1 \left(\dfrac{1+\sqrt{5}}{2} - \dfrac{1-\sqrt{5}}{2}\right) \\
          &= c_1 \left(\dfrac{1+\sqrt{5}-1+\sqrt{5}}{2}\right) \\
          &= c_1 \dfrac{2\sqrt{5}}{2} \\
          &= c_1 \sqrt{5} \Longrightarrow c_1 = \dfrac{1}{\sqrt{5}} \Longrightarrow c_2 = -\dfrac{1}{\sqrt{5}}
    \end{align*}

    La solución del problema es:
    \begin{equation*}
    x_n = \dfrac{1}{\sqrt{5}}\left({\left(\dfrac{1+\sqrt{5}}{2}\right)}^n - {\left(\dfrac{1-\sqrt{5}}{2}\right)}^n\right)
    \end{equation*}
\end{ejercicio}

\begin{ejercicio}
    Calcular la solución del problema de recurrencia 
    \begin{gather*}
        u_0 = 2 \\
        u_1 = 1 \\
        u_{n+2} = u_{n+1} + u_n,\qquad n\geq 0
    \end{gather*}

    Gracias a la solución del ejercicio~\ref{ej:recurrencia}, sabemos que si:

    Si $\{x_n\}$ es solución del problema entonces $x_0 = u_0 = 2$ y $x_1 = u_1 = 1$
    \begin{align*}
        2 &= x_0 \\
          &= c_1{\left(\dfrac{1+\sqrt{5}}{2}\right)}^0 + c_2 {\left(\dfrac{1-\sqrt{5}}{2}\right)}^0\\
          &= c_1 + c_2 \Longrightarrow c_2 = 2-c_1
    \end{align*}

    \begin{align*}
        1 &= x_1 \\
          &= c_1 {\left(\dfrac{1+\sqrt{5}}{2}\right)}^1 + c_2 {\left(\dfrac{1-\sqrt{5}}{2}\right)}^1 \\
          &= c_1 \left(\dfrac{1+\sqrt{5}}{2}\right) + (2- c_1) \left(\dfrac{1-\sqrt{5}}{2}\right) \\
          &= c_1 \left(\dfrac{1+\sqrt{5}}{2}\right) + 2 \left(\dfrac{1-\sqrt{5}}{2}\right)-c_1\left(\dfrac{1-\sqrt{5}}{2}\right) \\
          &= c_1 \left(\dfrac{1+\sqrt{5}}{2}-\dfrac{1-\sqrt{5}}{2}\right) + \cancel{2}\left(\dfrac{1-\sqrt{5}}{\cancel{2}}\right) \\
          &= c_1 \sqrt{5} +1 - \sqrt{5} \\
          &\Longrightarrow 1 = c_1 \sqrt{5} + 1 -\sqrt{5}\\
          &\Longrightarrow 0 = c_1\sqrt{5}-\sqrt{5} \\
          &\Longrightarrow c_1\sqrt{5} = \sqrt{5} \Longrightarrow c_1 = 1 \Longrightarrow c_2 = 1
    \end{align*}

    La solución es:
    \begin{equation*}
        x_n = {\left(\dfrac{1+\sqrt{5}}{2}\right)}^n + {\left(\dfrac{1-\sqrt{5}}{2}\right)}^n
    \end{equation*}
\end{ejercicio}

% // TODO: hacer el ejercicio:
\begin{ejercicio}
    Resuelva el siguiente problema de recurrencia:
    \begin{gather*}
        u_0 = 1 \\
        u_1 = 3 \\
        u_{n+2} = 4u_{n+1} - 4u_n
    \end{gather*}

    % -------------------------------------------------
    % SOlución esquemática

    De orden: $k = 2$\\
    Ecuación característica: $x^2 -4x + 4 = 0$\\
    Soluciones: 
    \begin{equation*}
        x = \dfrac{1\pm \sqrt{16-16}}{2} = 2
    \end{equation*}
    \begin{equation*}
        r_1 = 2 \qquad m_1 = 2
    \end{equation*}

    Solución general:
    \begin{equation*}
        x_n = (c_1 + c_2) 2^n
    \end{equation*}

    % -------------------------------------------------

    Pasamos ahora a buscar los valores de $c_1$ y $c_2$:
    \begin{align*}
        1 &= u_0 \\
          &= x_0 \\
          &= (c_1+c_2 \cdot 0)2^0 \\
          &= c_1 \cdot 1  \\
          &= c_1 \Longrightarrow c_1 = 1\\
          & \\
        3 &= u_1 \\
          &= x_1 \\
          &= (1+c_2\cdot 1)2^1 \\
          &= (1+c_2)\cdot 2 \\
          &= 2 + 2 c_2 \Longrightarrow c_2 = \dfrac{3-2}{2} = \dfrac{1}{2}
    \end{align*}

    La solución
    \begin{equation*}
        x_n = \left(1 + \dfrac{1}{2}n\right) 2^n = \dfrac{2+n}{2} 2^n = (2+n)2^{n-1}
    \end{equation*}

\end{ejercicio}

\begin{ejercicio}
    De el número de pasos mínimo para completar una instancia del puzzle conocido como ``Torres de Hanoi'', en función del número de discos $n$ con los que cuente.\\

    Para $n \geq 0$ sea $u_n$ el número de pasos solicitado. Si el puzzle tuviese $n+1$ discos entonces (Intuimos que es mover $n-1$ piezas, mover la base y mover $n-1$ piezas):
    \begin{align*}
        u_{n+1} &= u_n + 1 + u_n \\
                &= 2u_n +1
    \end{align*}
    Así pues recurrentemente:
    \begin{equation*}
        u_{n+1} = 2u_n + 1
    \end{equation*}
    que resolveremos esquemáticamente:
    \begin{description}
        \item $k = 1$.
        \item Ecuación característica: $x-2$.
        \item Función de ajuste: $f(n) = 1^n = 1$.
        \item Multiplicidad de 1 como solución de la ecución característica: $m = 0$.
    \end{description}
    Śolución particular:
    \begin{equation*}
        x_n^{(p)} = n^m q(n)1^n \qquad \deg(g(n)) \leq \deg(p(n)) = 0
    \end{equation*}
    \begin{align*}
        x_n^{(p)} &= n^0 c_2 1^n = c_2 \\
        x_n &= x_n^{(h)} + x_n^{(p)} \\
            &= c_1 2^n + c_2 
    \end{align*}
    Calcularemos $c_2$ sin contar con valores de la sucesión. Sabemos que:
    \begin{align*}
        1 &= x_{n+1}^{(p)} - 2x_n^{(p)} \\
          &= c_2 - 2 c_2 \\
          &= c_2 (1-2) = -c_2 \Longrightarrow c_2 = -1
    \end{align*}
    Luego 
    \begin{equation*}
        x_n = c_1 2^n - 1
    \end{equation*}
    y como $x_0 = u_0 = 0$ se tiene que
    \begin{align*}
        0 &= c_1 2^0 - 1 \\
          &= c_1 - 1 \Longrightarrow c_1 = 1
    \end{align*}
    Respuesta:
    \begin{equation*}
        x_n = 2^n -1
    \end{equation*}
    \begin{align*}
        n = 0;& \quad x_n = 0 \\
        n = 1;& \quad x_1 = 2^1 - 1 = 1 \\
        n = 2;& \quad x_2 = 2^2 - 1 = 3 \\
        n = 3;& \quad x_3 = 2^3 - 1 = 7 \\
        n = 4;& \quad x_4 = 2^4 - 1 = 15
    \end{align*}
\end{ejercicio}

\begin{ejercicio}
    Sea 
    \begin{equation*}
        u_n = \sum_{k = 0}^n k 2^k
    \end{equation*}

    \begin{enumerate}
        \item Encuentre una expresión recurrente para $u_n$.
        \item Encuentre una fórmula explícita para calcular $u_n$.
    \end{enumerate}

    \begin{enumerate} 
        \item 
        \begin{align*}
            u_0 &= 0 \\
            u_n &= \sum_{k=0}^n k 2^k \quad n \geq 1\\
                &= \left(\sum_{k=0}^{n-1} k2^k\right) + n2^n \\
                &= u_{n-1} + n2^n
        \end{align*}
        O sea, $u_n = u_{n-1} + n 2^n$

        \item Resolvemos el problema
            \begin{align*}
                u_0 &= 0 \\
                u_n &= u_{n-1} + n2^n 
            \end{align*}
            \begin{description}
                \item Orden: $k=1$.
                \item Ecuación característica: $x-1$.
                \item Solución homogénea: $x_n^{(h)} = c_1$.
                \item Función de ajuste: $f(n) = p(n) s^n = n2^n$.
                \item $\Longrightarrow p(n) = n, \deg(p(n)) = 1$.
                    $S = 2$, que no es solución de la ecuación característica, luego $m = 0$.
            \end{description}
            Solución particular
            \begin{align*}
                x_n^{(p)} &= x^0 (c_2 + c_3 n) 2^n \\
                          &= (c_2 + c_3 n)2^n \\
            \end{align*}
            Calculemos $x_n^{(p)}$, o sea, los valores de $c_2$ y $c_2$.
            \begin{equation*}
                (n+1)2^{n+1} = x_{n+1}^{(p)} - x_{n}^{(p)}
            \end{equation*}
            \begin{align*}
                x_n^{(p)} &= (c_2+c_3n)2^n \\
                 x_{n+1}^{(p)} &= (c_2+c_3(n+1))2^{n+1} \\
                 (n+1)2^{n+1} &= (c_2 + c_3 (n+1))2^{n+1} - (c_2 + c_3n) 2^n \\
                 2(n+1)2^n &= 2(c_2+c_3(n+1))2^n - (c_2+c_3n)2^n \\
                           &= 2^n (2c_2+2c_3(n+1)-c_2-c_3n) \\
                           &= 2^n (2c_2 + 2c_3n + 2c_3 - c_2 -c_3 n) \\
                           &= 2^n (c_2 + (2c_3 - c_3)n + 2c_3) \\
                           &= 2^n (c_2 + 2c_3 + c_3n)
            \end{align*}
            luego
            \begin{equation*}
                2n+2 = 2(n+1) = c_3n + c_2 + 2c_3
            \end{equation*}
            basta con que 
            \begin{align*}
                2 &= c_3 \\
                2 &= c_2 + 2c_3 
            \end{align*}
            Por tanto $c_3 = 2$ y 
            \begin{align*}
                2 &= c_2 + 2 \cdot 2 \\
                  &= c_2 + 4 \Longrightarrow c_2 = -2
            \end{align*}
            luego 
            \begin{align*}
                x_n^{(p)} &= (-2+2n)2^n \\
                          &= (n-1)2^{n+1}
            \end{align*}
            En definitiva
            \begin{align*}
                x_n &= x_n^{(h)} + x_n^{(p)} \\
                    &= c_1 + (n-1) 2^{n+1}
            \end{align*}
            y como $x_0 = u_0 = 0$, entonces
            \begin{align*}
                0 &= c_1 + (0-1)2^{0+1} \\
                  &= c_2 - 2
            \end{align*}
            luego $c_1 = 2$ y por tanto: 
            \begin{equation*}
                x_n = 2 +(n-1)2^{n+1}
            \end{equation*}
            Respuesta:
            \begin{equation*}
                u_n = 2 + (n-1)2^{n+1}
            \end{equation*}
    \end{enumerate} 
\end{ejercicio}

\begin{ejercicio}
    Un ciudadano pide un préstamo por cantidad $S$ de dinero a pagar en $T$ plazos. Si $I$ es el interés del préstamos por plazo en tanto por uno, ¿qué pago constante $P$ debe hacer al final de cada plazo?\\

   $u_n$ es la cantidad de pŕestamo que todavía debe el ciudadano al final del $n-$ésimo plazo, es decir, a continuación del $n-$ésimo pago. 
   Entonces, para todo $0 \leq n \leq T-1$:
    \begin{align*}
        u_{n+1} &= u_n + I \cdot u_n - P \\
                &= (1+I) u_n - P
    \end{align*}
    y el problema de recurrencia a resolver es:
    \begin{align*}
        u_0 &= S \\
        u_T &= 0 \\
        u_{n+1} &= (1+I)u_n - P \qquad 0 \leq n \leq T - 1
    \end{align*}
    lo cual hacemos.
    \begin{description}
        \item $k = 1$.
        \item Ecuación característica: $x-(1+I) = 0$ = 0.
        \item Solución homogénea: $x_n^{(h)} = c{(1+I)}^n$.
        \item Función de ajuste: $f(n) = -P = -P\cdot 1^n$.
    \end{description}
    Los casos son los siguientes:
    \begin{itemize}
        \item $I = 0$;
        \begin{align*}
            x_n^{(h)} &= c \\
            x_n^{(p)} &= n^m \cdot A \cdot 1^n \\
                      &\mathop{=}^{m=1} nA
        \end{align*}
        \begin{align*}
            -P &= (n+1) -nA \\
               &= A(n+1-n) = A
        \end{align*}
        \begin{equation*}
            x_n = c - nP
        \end{equation*}
        como $x_0 = u_0 = S$ entonces:
        \begin{equation*}
            S = x_0 = C - 0\cdot P = C
        \end{equation*}
        luego
        \begin{equation*}
            x_n = S - nP
        \end{equation*}
        Como $x_T = 0$, entonces 
        \begin{equation*}
            0 = S - T\cdot P
        \end{equation*}
        luego
        \begin{equation*}
            P = \dfrac{S}{P}
        \end{equation*}

        \item $I \neq 0$;
        \begin{align*}
            x_n^{(p)} &= A \\
        \end{align*}
        \begin{align*}
            -P &= x_{n+1}^{(p)} - (1+I)x_n^{(p)} \\
                      &= A-(1+I)A \\
                      &= A (1-(1+I)) = -AI \Longrightarrow A = \dfrac{P}{I}
        \end{align*}
        \begin{align*}
            x_n &= x_n^{(h)} + x_n^{(p)} \\
                &= c + \dfrac{P}{I}
        \end{align*}
        Pero
        \begin{equation*}
            S = x_0 = c{(1+I)}^0 + \dfrac{P}{I} \Longrightarrow c = S - \dfrac{P}{I}
        \end{equation*}
        \begin{equation*}
            x_n = \left(S - \dfrac{P}{I}\right){(1+I)}^n + \dfrac{P}{I}
        \end{equation*}
        Sabemos que
        \begin{equation*}
            0 = x_T = \left(S-\dfrac{P}{I}\right){(1+I)}^T + \dfrac{P}{I}
        \end{equation*}
        y entonces
        \begin{equation*}
            \dfrac{P}{I} = \left(\dfrac{P}{I} - S\right){(1+I)}^T
        \end{equation*}
        \begin{align*}
            P &= (p-SI){(1+I)}^T \\
              &= P{{(1+I)}}^T - SI{{(1+I)}}^T
        \end{align*}
        \begin{equation*}
            P(1-{(1+I)}^T) = -SI{(1+I)}^T
        \end{equation*}
        \begin{align*}
            P &= \dfrac{SI{(1+I)}^T}{{(1+I)}^T-1} \\
              &= SI{(1-{(1+I)}^{-T})}^{-1}
        \end{align*}
        Respuesta
        \begin{equation*}
              P = SI{(1-{(1+I)}^{-T})}^{-1}
        \end{equation*}
    \end{itemize}

\end{ejercicio}
