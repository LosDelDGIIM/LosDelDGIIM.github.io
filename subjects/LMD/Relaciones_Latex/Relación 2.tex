\section{Recurrencia}

\begin{ejercicio}
    Resuelva la relación de recurrencia dada por $u_{n+2} = 4u_{n+1} - 4u_n$, para todo $n\geq 0$.
    Particularice el resultado suponiendo que $n\geq 0$, $u_0=1$, $u_1=3$.\\

    El orden de la recurrencia es $k=2$. La ecuación característica es:
    \begin{equation*}
        x^2-4x+4 = {(x-2)}^2=0
    \end{equation*}

    Por tanto, tan solo hay una raíz $r=2$ de multiplicidad $m=2$. La solución general de la recurrencia es:
    \begin{equation*}
        x_n = (c_1+c_2n)2^n
    \end{equation*}

    Y ahora buscar los valores de $c_1$ y $c_2$ usando las condiciones iniciales, para obtener la solución particular.
    Tenemos que $x_0=u_0=1$ y $x_1=u_1=3$, entonces:
    \begin{align*}
        1 &= (c_1+c_2\cdot 0) 2^0 = c_1\cdot 1 = c_1\\
        3 &= (c_1+c_2\cdot 1) 2^1 = (1+c_2) 2 = 2+2c_2 \Longrightarrow c_2 = \frac{3-2}{2} = \frac{1}{2}
    \end{align*}

    Por tanto, la solución particular es:
    \begin{equation*}
        x_n = \left(1+\frac{n}{2}\right)2^n = \left(\frac{2+n}{2}\right)2^n = (2+n)2^{n-1}
    \end{equation*}
\end{ejercicio}

\begin{ejercicio}\label{ej:recurrencia}
    Resuelva la recurrencia:
    \begin{equation*}
        u_{n+2} = u_{n+1} + u_n\qquad n\geq 0
    \end{equation*}

    El orden $k$ de la recurrencia es 2 ($k = 2$). La ecución característica:
    \begin{equation*}
        x^2 -x -1 = 0 
    \end{equation*}

    que tiene por soluciones
    \begin{equation*}
        x = \dfrac{1\pm \sqrt{1+4}}{2} 
    \end{equation*}

    En definitiva
    \begin{gather*}
        r_1 = \dfrac{1+\sqrt{5}}{2} \qquad m_1 = 1 \\
        r_2 = \dfrac{1+\sqrt{5}}{2} \qquad m_2 = 1
    \end{gather*}

    y se tiene: $m_1+m_2 = 1 +1 = 2 = k$

    Si $\{x_n\}$ es solución de la recurrencia, entonces sabemos que para todo $n \geq 0$:
    \begin{align*}
        x_n &= c_1 r_1^n + c_2 r_2^n\\
            &= c_1{\left(\dfrac{1+\sqrt{5}}{2}\right)}^n + c_2 {\left(\dfrac{1-\sqrt{5}}{2}\right)}^n
    \end{align*}

    para ciertos valores de $c_1$ y $c_2$.
\end{ejercicio}

\begin{ejercicio}
    Reuelva el problema de recurrencia 
    \begin{gather*}
        u_0 = 0\\    
        u_1 = 1 \\
        u_{n+2} = u_{n+1} + u_n \qquad n \geq 0
    \end{gather*}

    Por el ejercicio anterior, sabemos que la solución a la relación de recurrencia del problema es:
    \begin{equation*}
        x_n = c_1{\left(\dfrac{1+\sqrt{5}}{2}\right)}^n + c_2 {\left(\dfrac{1-\sqrt{5}}{2}\right)}^n
    \end{equation*}

    Si $\{x_n\}$ es solución del problema entonces $x_0 = u_0 = 0$ y $x_1 = u_1 = 1$
    \begin{align*}
        0 &= x_0 \\
          &= c_1{\left(\dfrac{1+\sqrt{5}}{2}\right)}^0 + c_2 {\left(\dfrac{1-\sqrt{5}}{2}\right)}^0\\
          &= c_1 + c_2 \Longrightarrow c_2 = -c_1
    \end{align*}
    \begin{align*}
        1 &= x_1 \\
          &= c_1 {\left(\dfrac{1+\sqrt{5}}{2}\right)}^1 + c_2 {\left(\dfrac{1-\sqrt{5}}{2}\right)}^1 \\
          &= c_1 \left(\dfrac{1+\sqrt{5}}{2}\right) - c_1 \left(\dfrac{1-\sqrt{5}}{2}\right) \\
          &= c_1 \left(\dfrac{1+\sqrt{5}}{2} - \dfrac{1-\sqrt{5}}{2}\right) \\
          &= c_1 \left(\dfrac{1+\sqrt{5}-1+\sqrt{5}}{2}\right) \\
          &= c_1 \dfrac{2\sqrt{5}}{2} \\
          &= c_1 \sqrt{5} \Longrightarrow c_1 = \dfrac{1}{\sqrt{5}} \Longrightarrow c_2 = -\dfrac{1}{\sqrt{5}}
    \end{align*}

    La solución del problema es:
    \begin{equation*}
    x_n = \dfrac{1}{\sqrt{5}}\left({\left(\dfrac{1+\sqrt{5}}{2}\right)}^n - {\left(\dfrac{1-\sqrt{5}}{2}\right)}^n\right)
    \end{equation*}
\end{ejercicio}

\begin{ejercicio}
    Calcular la solución del problema de recurrencia 
    \begin{gather*}
        u_0 = 2 \\
        u_1 = 1 \\
        u_{n+2} = u_{n+1} + u_n,\qquad n\geq 0
    \end{gather*}

    Gracias a la solución del ejercicio~\ref{ej:recurrencia}, sabemos que si:

    Si $\{x_n\}$ es solución del problema entonces $x_0 = u_0 = 2$ y $x_1 = u_1 = 1$
    \begin{align*}
        2 &= x_0 \\
          &= c_1{\left(\dfrac{1+\sqrt{5}}{2}\right)}^0 + c_2 {\left(\dfrac{1-\sqrt{5}}{2}\right)}^0\\
          &= c_1 + c_2 \Longrightarrow c_2 = 2-c_1
    \end{align*}

    \begin{align*}
        1 &= x_1 \\
          &= c_1 {\left(\dfrac{1+\sqrt{5}}{2}\right)}^1 + c_2 {\left(\dfrac{1-\sqrt{5}}{2}\right)}^1 \\
          &= c_1 \left(\dfrac{1+\sqrt{5}}{2}\right) + (2- c_1) \left(\dfrac{1-\sqrt{5}}{2}\right) \\
          &= c_1 \left(\dfrac{1+\sqrt{5}}{2}\right) + 2 \left(\dfrac{1-\sqrt{5}}{2}\right)-c_1\left(\dfrac{1-\sqrt{5}}{2}\right) \\
          &= c_1 \left(\dfrac{1+\sqrt{5}}{2}-\dfrac{1-\sqrt{5}}{2}\right) + \cancel{2}\left(\dfrac{1-\sqrt{5}}{\cancel{2}}\right) \\
          &= c_1 \sqrt{5} +1 - \sqrt{5} \\
          &\Longrightarrow 1 = c_1 \sqrt{5} + 1 -\sqrt{5}\\
          &\Longrightarrow 0 = c_1\sqrt{5}-\sqrt{5} \\
          &\Longrightarrow c_1\sqrt{5} = \sqrt{5} \Longrightarrow c_1 = 1 \Longrightarrow c_2 = 1
    \end{align*}

    La solución es:
    \begin{equation*}
        x_n = {\left(\dfrac{1+\sqrt{5}}{2}\right)}^n + {\left(\dfrac{1-\sqrt{5}}{2}\right)}^n
    \end{equation*}
\end{ejercicio}

% // TODO: hacer el ejercicio:
\begin{ejercicio}
    Resuelva el siguiente problema de recurrencia:
    \begin{gather*}
        u_0 = 1 \\
        u_1 = 3 \\
        u_{n+2} = 4u_{n+1} - 4u_n
    \end{gather*}

    % -------------------------------------------------
    % SOlución esquemática

    De orden: $k = 2$\\
    Ecuación característica: $x^2 -4x + 4 = 0$\\
    Soluciones: 
    \begin{equation*}
        x = \dfrac{1\pm \sqrt{16-16}}{2} = 2
    \end{equation*}
    \begin{equation*}
        r_1 = 2 \qquad m_1 = 2
    \end{equation*}

    Solución general:
    \begin{equation*}
        x_n = (c_1 + c_2) 2^n
    \end{equation*}

    % -------------------------------------------------

    Pasamos ahora a buscar los valores de $c_1$ y $c_2$:
    \begin{align*}
        1 &= u_0 \\
          &= x_0 \\
          &= (c_1+c_2 \cdot 0)2^0 \\
          &= c_1 \cdot 1  \\
          &= c_1 \Longrightarrow c_1 = 1\\
          & \\
        3 &= u_1 \\
          &= x_1 \\
          &= (1+c_2\cdot 1)2^1 \\
          &= (1+c_2)\cdot 2 \\
          &= 2 + 2 c_2 \Longrightarrow c_2 = \dfrac{3-2}{2} = \dfrac{1}{2}
    \end{align*}

    La solución
    \begin{equation*}
        x_n = \left(1 + \dfrac{1}{2}n\right) 2^n = \dfrac{2+n}{2} 2^n = (2+n)2^{n-1}
    \end{equation*}

\end{ejercicio}
