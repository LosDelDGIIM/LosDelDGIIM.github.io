\documentclass[12pt]{article}

% Idioma y codificación
\usepackage[spanish, es-tabla]{babel}       %es-tabla para que se titule "Tabla"
\usepackage[utf8]{inputenc}

% Márgenes
\usepackage[a4paper,top=3cm,bottom=2.5cm,left=3cm,right=3cm]{geometry}

% Comentarios de bloque
\usepackage{verbatim}

% Paquetes de links
\usepackage[hidelinks]{hyperref}    % Permite enlaces
\usepackage{url}                    % redirecciona a la web

% Más opciones para enumeraciones
\usepackage{enumitem}

% Personalizar la portada
\usepackage{titling}

% Paquetes de tablas
\usepackage{multirow}


%------------------------------------------------------------------------

%Paquetes de figuras
\usepackage{caption}
\usepackage{subcaption} % Figuras al lado de otras
\usepackage{float}      % Poner figuras en el sitio indicado H.


% Paquetes de imágenes
\usepackage{graphicx}       % Paquete para añadir imágenes
\usepackage{transparent}    % Para manejar la opacidad de las figuras

% Paquete para usar colores
\usepackage[dvipsnames]{xcolor}
\usepackage{pagecolor}      % Para cambiar el color de la página

% Habilita tamaños de fuente mayores
\usepackage{fix-cm}

% Para los gráficos
\usepackage{tikz}

% Para poder situar los nodos en los grafos
\usetikzlibrary{positioning}


%------------------------------------------------------------------------

% Paquetes de matemáticas
\usepackage{mathtools, amsfonts, amssymb, mathrsfs}
\usepackage[makeroom]{cancel}     % Simplificar tachando
\usepackage{polynom}    % Divisiones y Ruffini
\usepackage{units} % Para poner fracciones diagonales con \nicefrac

\usepackage{pgfplots}   %Representar funciones
\pgfplotsset{compat=1.18}  % Versión 1.18

\usepackage{tikz-cd}    % Para usar diagramas de composiciones
\usetikzlibrary{calc}   % Para usar cálculo de coordenadas en tikz

%Definición de teoremas, etc.
\usepackage{amsthm}
%\swapnumbers   % Intercambia la posición del texto y de la numeración

\theoremstyle{plain}

\makeatletter
\@ifclassloaded{article}{
  \newtheorem{teo}{Teorema}[section]
}{
  \newtheorem{teo}{Teorema}[chapter]  % Se resetea en cada chapter
}
\makeatother

\newtheorem{coro}{Corolario}[teo]           % Se resetea en cada teorema
\newtheorem{prop}[teo]{Proposición}         % Usa el mismo contador que teorema
\newtheorem{lema}[teo]{Lema}                % Usa el mismo contador que teorema

\theoremstyle{remark}
\newtheorem*{observacion}{Observación}

\theoremstyle{definition}

\makeatletter
\@ifclassloaded{article}{
  \newtheorem{definicion}{Definición} [section]     % Se resetea en cada chapter
}{
  \newtheorem{definicion}{Definición} [chapter]     % Se resetea en cada chapter
}
\makeatother

\newtheorem*{notacion}{Notación}
\newtheorem*{ejemplo}{Ejemplo}
\newtheorem*{ejercicio*}{Ejercicio}             % No numerado
\newtheorem{ejercicio}{Ejercicio} [section]     % Se resetea en cada section


% Modificar el formato de la numeración del teorema "ejercicio"
\renewcommand{\theejercicio}{%
  \ifnum\value{section}=0 % Si no se ha iniciado ninguna sección
    \arabic{ejercicio}% Solo mostrar el número de ejercicio
  \else
    \thesection.\arabic{ejercicio}% Mostrar número de sección y número de ejercicio
  \fi
}


% \renewcommand\qedsymbol{$\blacksquare$}         % Cambiar símbolo QED
%------------------------------------------------------------------------

% Paquetes para encabezados
\usepackage{fancyhdr}
\pagestyle{fancy}
\fancyhf{}

\newcommand{\helv}{ % Modificación tamaño de letra
\fontfamily{}\fontsize{12}{12}\selectfont}
\setlength{\headheight}{15pt} % Amplía el tamaño del índice


%\usepackage{lastpage}   % Referenciar última pag   \pageref{LastPage}
\fancyfoot[C]{\thepage}

%------------------------------------------------------------------------

% Conseguir que no ponga "Capítulo 1". Sino solo "1."
\makeatletter
\@ifclassloaded{book}{
  \renewcommand{\chaptermark}[1]{\markboth{\thechapter.\ #1}{}} % En el encabezado
    
  \renewcommand{\@makechapterhead}[1]{%
  \vspace*{50\p@}%
  {\parindent \z@ \raggedright \normalfont
    \ifnum \c@secnumdepth >\m@ne
      \huge\bfseries \thechapter.\hspace{1em}\ignorespaces
    \fi
    \interlinepenalty\@M
    \Huge \bfseries #1\par\nobreak
    \vskip 40\p@
  }}
}
\makeatother

%------------------------------------------------------------------------
% Paquetes de cógido
\usepackage{minted}
\renewcommand\listingscaption{Código fuente}

\usepackage{fancyvrb}
% Personaliza el tamaño de los números de línea
\renewcommand{\theFancyVerbLine}{\small\arabic{FancyVerbLine}}

% Estilo para C++
\newminted{cpp}{
    frame=lines,
    framesep=2mm,
    baselinestretch=1.2,
    linenos,
    escapeinside=||
}

% para minted
\definecolor{LightGray}{rgb}{0.95,0.95,0.92}
\setminted{
    linenos=true,
    stepnumber=5,
    numberfirstline=true,
    autogobble,
    breaklines=true,
    breakautoindent=true,
    breaksymbolleft=,
    breaksymbolright=,
    breaksymbolindentleft=0pt,
    breaksymbolindentright=0pt,
    breaksymbolsepleft=0pt,
    breaksymbolsepright=0pt,
    fontsize=\footnotesize,
    bgcolor=LightGray,
    numbersep=10pt
}


\usepackage{listings} % Para incluir código desde un archivo

\renewcommand\lstlistingname{Código Fuente}
\renewcommand\lstlistlistingname{Índice de Códigos Fuente}

% Definir colores
\definecolor{vscodepurple}{rgb}{0.5,0,0.5}
\definecolor{vscodeblue}{rgb}{0,0,0.8}
\definecolor{vscodegreen}{rgb}{0,0.5,0}
\definecolor{vscodegray}{rgb}{0.5,0.5,0.5}
\definecolor{vscodebackground}{rgb}{0.97,0.97,0.97}
\definecolor{vscodelightgray}{rgb}{0.9,0.9,0.9}

% Configuración para el estilo de C similar a VSCode
\lstdefinestyle{vscode_C}{
  backgroundcolor=\color{vscodebackground},
  commentstyle=\color{vscodegreen},
  keywordstyle=\color{vscodeblue},
  numberstyle=\tiny\color{vscodegray},
  stringstyle=\color{vscodepurple},
  basicstyle=\scriptsize\ttfamily,
  breakatwhitespace=false,
  breaklines=true,
  captionpos=b,
  keepspaces=true,
  numbers=left,
  numbersep=5pt,
  showspaces=false,
  showstringspaces=false,
  showtabs=false,
  tabsize=2,
  frame=tb,
  framerule=0pt,
  aboveskip=10pt,
  belowskip=10pt,
  xleftmargin=10pt,
  xrightmargin=10pt,
  framexleftmargin=10pt,
  framexrightmargin=10pt,
  framesep=0pt,
  rulecolor=\color{vscodelightgray},
  backgroundcolor=\color{vscodebackground},
}

%------------------------------------------------------------------------

% Comandos definidos
\newcommand{\bb}[1]{\mathbb{#1}}
\newcommand{\cc}[1]{\mathcal{#1}}

% I prefer the slanted \leq
\let\oldleq\leq % save them in case they're every wanted
\let\oldgeq\geq
\renewcommand{\leq}{\leqslant}
\renewcommand{\geq}{\geqslant}

% Si y solo si
\newcommand{\sii}{\iff}

% Letras griegas
\newcommand{\eps}{\epsilon}
\newcommand{\veps}{\varepsilon}
\newcommand{\lm}{\lambda}

\newcommand{\ol}{\overline}
\newcommand{\ul}{\underline}
\newcommand{\wt}{\widetilde}
\newcommand{\wh}{\widehat}

\let\oldvec\vec
\renewcommand{\vec}{\overrightarrow}

% Derivadas parciales
\newcommand{\del}[2]{\frac{\partial #1}{\partial #2}}
\newcommand{\Del}[3]{\frac{\partial^{#1} #2}{\partial #3^{#1}}}
\newcommand{\deld}[2]{\dfrac{\partial #1}{\partial #2}}
\newcommand{\Deld}[3]{\dfrac{\partial^{#1} #2}{\partial #3^{#1}}}


\newcommand{\AstIg}{\stackrel{(\ast)}{=}}
\newcommand{\Hop}{\stackrel{L'H\hat{o}pital}{=}}

\newcommand{\red}[1]{{\color{red}#1}} % Para integrales, destacar los cambios.

% Método de integración
\newcommand{\MetInt}[2]{
    \left[\begin{array}{c}
        #1 \\ #2
    \end{array}\right]
}

% Declarar aplicaciones
% 1. Nombre aplicación
% 2. Dominio
% 3. Codominio
% 4. Variable
% 5. Imagen de la variable
\newcommand{\Func}[5]{
    \begin{equation*}
        \begin{array}{rrll}
            #1:& #2 & \longrightarrow & #3\\
               & #4 & \longmapsto & #5
        \end{array}
    \end{equation*}
}

%------------------------------------------------------------------------

% Define a custom command for email addresses
\newcommand{\email}[1]{\href{mailto:#1}{{{\color{blue}#1}}}}


\begin{document}

     % 1. Foto de fondo
    % 2. Título
    % 3. Encabezado Izquierdo
    % 4. Color de fondo
    % 5. Coord x del titulo
    % 6. Coord y del titulo
    % 7. Fecha

    
    % 1. Foto de fondo
% 2. Título
% 3. Encabezado Izquierdo
% 4. Color de fondo
% 5. Coord x del titulo
% 6. Coord y del titulo
% 7. Fecha

\newcommand{\portada}[7]{

    \portadaBase{#1}{#2}{#3}{#4}{#5}{#6}{#7}
    \portadaBook{#1}{#2}{#3}{#4}{#5}{#6}{#7}
}

\newcommand{\portadaExamen}[7]{

    \portadaBase{#1}{#2}{#3}{#4}{#5}{#6}{#7}
    \portadaArticle{#1}{#2}{#3}{#4}{#5}{#6}{#7}
}




\newcommand{\portadaBase}[7]{

    % Tiene la portada principal y la licencia Creative Commons
    
    % 1. Foto de fondo
    % 2. Título
    % 3. Encabezado Izquierdo
    % 4. Color de fondo
    % 5. Coord x del titulo
    % 6. Coord y del titulo
    % 7. Fecha
    
    
    \thispagestyle{empty}               % Sin encabezado ni pie de página
    \newgeometry{margin=0cm}        % Márgenes nulos para la primera página
    
    
    % Encabezado
    \fancyhead[L]{\helv #3}
    \fancyhead[R]{\helv \nouppercase{\leftmark}}
    
    
    \pagecolor{#4}        % Color de fondo para la portada
    
    \begin{figure}[p]
        \centering
        \transparent{0.3}           % Opacidad del 30% para la imagen
        
        \includegraphics[width=\paperwidth, keepaspectratio]{assets/#1}
    
        \begin{tikzpicture}[remember picture, overlay]
            \node[anchor=north west, text=white, opacity=1, font=\fontsize{60}{90}\selectfont\bfseries\sffamily, align=left] at (#5, #6) {#2};
            
            \node[anchor=south east, text=white, opacity=1, font=\fontsize{12}{18}\selectfont\sffamily, align=right] at (9.7, 3) {\textbf{\href{https://losdeldgiim.github.io/}{Los Del DGIIM}}};
            
            \node[anchor=south east, text=white, opacity=1, font=\fontsize{12}{15}\selectfont\sffamily, align=right] at (9.7, 1.8) {Doble Grado en Ingeniería Informática y Matemáticas\\Universidad de Granada};
        \end{tikzpicture}
    \end{figure}
    
    
    \restoregeometry        % Restaurar márgenes normales para las páginas subsiguientes
    \pagecolor{white}       % Restaurar el color de página
    
    
    \newpage
    \thispagestyle{empty}               % Sin encabezado ni pie de página
    \begin{tikzpicture}[remember picture, overlay]
        \node[anchor=south west, inner sep=3cm] at (current page.south west) {
            \begin{minipage}{0.5\paperwidth}
                \href{https://creativecommons.org/licenses/by-nc-nd/4.0/}{
                    \includegraphics[height=2cm]{assets/Licencia.png}
                }\vspace{1cm}\\
                Esta obra está bajo una
                \href{https://creativecommons.org/licenses/by-nc-nd/4.0/}{
                    Licencia Creative Commons Atribución-NoComercial-SinDerivadas 4.0 Internacional (CC BY-NC-ND 4.0).
                }\\
    
                Eres libre de compartir y redistribuir el contenido de esta obra en cualquier medio o formato, siempre y cuando des el crédito adecuado a los autores originales y no persigas fines comerciales. 
            \end{minipage}
        };
    \end{tikzpicture}
    
    
    
    % 1. Foto de fondo
    % 2. Título
    % 3. Encabezado Izquierdo
    % 4. Color de fondo
    % 5. Coord x del titulo
    % 6. Coord y del titulo
    % 7. Fecha


}


\newcommand{\portadaBook}[7]{

    % 1. Foto de fondo
    % 2. Título
    % 3. Encabezado Izquierdo
    % 4. Color de fondo
    % 5. Coord x del titulo
    % 6. Coord y del titulo
    % 7. Fecha

    % Personaliza el formato del título
    \pretitle{\begin{center}\bfseries\fontsize{42}{56}\selectfont}
    \posttitle{\par\end{center}\vspace{2em}}
    
    % Personaliza el formato del autor
    \preauthor{\begin{center}\Large}
    \postauthor{\par\end{center}\vfill}
    
    % Personaliza el formato de la fecha
    \predate{\begin{center}\huge}
    \postdate{\par\end{center}\vspace{2em}}
    
    \title{#2}
    \author{\href{https://losdeldgiim.github.io/}{Los Del DGIIM}}
    \date{Granada, #7}
    \maketitle
    
    \tableofcontents
}




\newcommand{\portadaArticle}[7]{

    % 1. Foto de fondo
    % 2. Título
    % 3. Encabezado Izquierdo
    % 4. Color de fondo
    % 5. Coord x del titulo
    % 6. Coord y del titulo
    % 7. Fecha

    % Personaliza el formato del título
    \pretitle{\begin{center}\bfseries\fontsize{42}{56}\selectfont}
    \posttitle{\par\end{center}\vspace{2em}}
    
    % Personaliza el formato del autor
    \preauthor{\begin{center}\Large}
    \postauthor{\par\end{center}\vspace{3em}}
    
    % Personaliza el formato de la fecha
    \predate{\begin{center}\huge}
    \postdate{\par\end{center}\vspace{5em}}
    
    \title{#2}
    \author{\href{https://losdeldgiim.github.io/}{Los Del DGIIM}}
    \date{Granada, #7}
    \thispagestyle{empty}               % Sin encabezado ni pie de página
    \maketitle
    \vfill
}
    \portadaExamen{etsiitA4.jpg}{LMD\\Examen I}{Lógica y Métodos Discretos. Examen I}{MidnightBlue}{-8}{28}{2023-2024}{Antonio Romero Martín\\Carolina González Ríos\\Daniel Gómez García\\Arturo Olivares Martos}

    \begin{description}
        \item[Asignatura] Lógica y Métodos Discretos.
        \item[Curso Académico] 2023-24.
        \item[Grado] Doble Grado en Ingienería Informática y Matemáticas.
        \item[Grupo] Único.
        \item[Profesor] Francisco Miguel García Olmedo.
        \item[Descripción] Parcial de los Temas 1 y 2. Inducción y Recurrencia.
        \item[Fecha] 26 de abril de 2024.
        % \item[Duración] 60 minutos.
    
    \end{description}
    \newpage
    
    \begin{ejercicio}[Inducción]
        Demuestre por inducción que para todo número natural $n\in\omega$ existe un polinomio $f_n(x, y)$ cumpliendo:
        \begin{equation*}
            x^n - y^n = (x-y)\cdot f_n(x, y)
        \end{equation*}
        Tras perfeccionar la demostración, defina sin ambigüedad el factor $f_n(x, y)$ de $x^n - y^n$ cuya existencia ha concluido y calcule su valor para $n\in 4$.

        \begin{notacion}
            De aquí en adelante, para cualquier número natural $n\in\omega$, $f_n(x, y)$ denotará un polinomio.
        \end{notacion}
        \begin{proof}
            La demostración es por inducción según el segundo principio de inducción matemática y el predicado $P(n)$ del tenor:
            \begin{equation*}
                \text{`` }x^n - y^n = (x-y)\cdot \sum_{k=0}^{n-1}x^{n-1-k}y^k\text{ ''}
            \end{equation*}
            
            Distinguimos casos:
            \begin{itemize}
                \item Para $n=0$:
                    \begin{align*}
                        x^0 - y^0 = 1 - 1 = 0 = (x-y)\cdot 0
                    \end{align*}
                    Como $0 = 0\cdot (x-y)$, $P(0)$ es cierto.

                \item Para $n=1$:
                    \begin{align*}
                        x^1 - y^1 = x - y = (x-y)\cdot 1
                    \end{align*}
                    Por tanto, $P(1)$ es cierto.
    
                \item Para $n\in \omega$, $n\geq 2$:
                
                Como hipótesis de inducción, supondremos que $n$ es un número natural y que $P(k)$ es cierto para $k\in \omega,~0\leq k\leq n$, es decir, que:
                \begin{equation*}
                    x^k - y^k = (x-y)\cdot \sum_{j=0}^{k-1}x^{k-1-j}y^j
                \end{equation*}
                En el paso de inducción, demostraremos que $P(n+1)$ es cierto. Tenemos que:
                \begin{align*}
                    x^{n+1} - y^{n+1} &= x^{n+1} - y^{n+1} + x^ny -x^ny +xy^n -xy^n \\
                    &= (x+y)(x^n - y^n) -xy(x^{n-1} - y^{n-1}) \\
                    &\AstIg (x+y)(x-y)\sum_{k=0}^{n-1}x^{n-1-k}y^k -xy(x-y)\sum_{k=0}^{n-2}x^{n-2-k}y^k \\
                    &= (x-y)\left[(x+y)\sum_{k=0}^{n-1}x^{n-1-k}y^k - xy\sum_{k=0}^{n-2}x^{n-2-k}y^k\right] \\
                    &= (x-y)\left[\sum_{k=0}^{n-1}x^{n-k}y^k + \sum_{k=0}^{n-1}x^{n-1-k}y^{k+1} - \sum_{k=0}^{n-2}x^{n-1-k}y^{k+1}\right] \\
                    &= (x-y)\left[\sum_{k=0}^{n-1}x^{n-k}y^k + x^0y^n\right] \\
                    &= (x-y)\sum_{k=0}^{n}x^{n-k}y^k
                \end{align*}
                donde en $(\ast)$ hemos usado la hipótesis de inducción, puesto que $n,n-1\leq n$. Por tanto, $P(n+1)$ es cierto.
            \end{itemize}
    
            Así pues, por el segundo principio de inducción matemática, para todo número natural $n$, existe un polinomio $f_n(x, y)$ dado por:
            \begin{equation*}
                f_n(x, y) = \sum_{k=0}^{n-1}x^{n-1-k}y^k
            \end{equation*}
            tal que cumple:
            \begin{equation*}
                x^n - y^n = (x-y)\cdot f_n(x, y)
            \end{equation*}
        \end{proof}

        Veamos ahora para cada valor de $n\in 4= \{0, 1, 2, 3\}$ el valor de $f_n(x, y)$:
            \begin{itemize}
                \item Para $n=0$:
                    \begin{equation*}
                        f_0(x, y) = \sum_{k=0}^{-1}x^{-1-k}y^k = 0
                    \end{equation*}

                    Efectivamente, se tiene que:
                    \begin{equation*}
                        x^0 - y^0 = 1 - 1 = 0 = (x-y)\cdot 0
                    \end{equation*}
                \item Para $n=1$:
                    \begin{equation*}
                        f_1(x, y) = \sum_{k=0}^{0}x^{0-k}y^k = 1
                    \end{equation*}

                    Efectivamente, se tiene que:
                    \begin{equation*}
                        x^1 - y^1 = x - y = (x-y)\cdot 1
                    \end{equation*}
                \item Para $n=2$:
                    \begin{equation*}
                        f_2(x, y) = \sum_{k=0}^{1}x^{1-k}y^k = x + y
                    \end{equation*}

                    Efectivamente, se tiene que:
                    \begin{equation*}
                        x^2 - y^2 = (x-y)(x+y)
                    \end{equation*}
                \item Para $n=3$:
                    \begin{equation*}
                        f_3(x, y) = \sum_{k=0}^{2}x^{2-k}y^k = x^2 + xy + y^2
                    \end{equation*}

                    Efectivamente, se tiene que:
                    \begin{equation*}
                        x^3 - y^3 = (x-y)(x^2 + xy + y^2)
                    \end{equation*}
            \end{itemize}\newpage

            \begin{observacion}
                Proponemos ahora una demostración alternativa para la existencia de $f_n(x, y)$.
                \begin{proof}
                La demostración es mediante el principio de inducción matemática
                según el predicado $Q(n)$ del tenor:
                \begin{equation*}
                    \text{`` }x^n - y^n = (x-y)\cdot f_n(x, y)\text{ ''}
                \end{equation*}
                \begin{itemize}
                    \item En el caso base, $n=0$:
                        \begin{align*}
                            x^0 - y^0 &= 1 - 1 \\&= 0 \\&= (x-y)\cdot 0
                        \end{align*}
                        Como $0 = 0\cdot (x-y)$, $Q(0)$ es cierto.

                    \item Supongamos como hipótesis de inducción que $n$ es un número natural y que $Q(n)$ es cierto.
                    Es decir, que:
                        \begin{equation*}
                            x^n - y^n = (x-y)\cdot f_n(x, y)
                        \end{equation*}

                        En el paso de inducción, demostraremos que $Q(n+1)$ es cierto.
                        \begin{align*}
                            x^{n+1} - y^{n+1} &= x^{n+1} - y^{n+1} + xy^n - xy^n \\
                            &= x(x^n - y^n) +y^n(x-y) \\
                            &\AstIg x(x-y)f_n(x, y) + y^n(x-y) \\
                            &= (x-y)(xf_n(x, y) + y^n) \\
                            &= (x-y)f_{n+1}(x, y)
                        \end{align*}
                        donde en $(\ast)$ hemos usado la hipótesis de inducción. Por tanto, $Q(n+1)$ es cierto.
                \end{itemize}

                Así pues, por el principio de inducción matemática, para todo número natural $n$, existe un polinomio $f_n(x, y)$ tal que $x^n - y^n = (x-y)\cdot f_n(x, y)$.
                \end{proof}
            \end{observacion}
    \end{ejercicio}


    \begin{ejercicio}[Recurrencia]
        Encuentre la solución general de la recurrencia:
        \begin{equation*}
            u_{n+2} - 6u_{n+1} + 9u_n = 3\cdot 2^n + 7\cdot 3^n,\qquad n\geq 0
        \end{equation*}
        y posteriormente encuentre la solución particular que cumple con las condiciones iniciales $u_0 = 1$ y $u_1 = 4$.\\
    
        El orden de la recurrencia es $k=2$. La ecuación característica es:
        \begin{equation*}
            x^2 - 6x + 9 = 0 \Longrightarrow (x-3)^2 = 0
        \end{equation*}

        La única raíz de la ecuación característica es $x=3$ con multiplicidad doble. Por tanto, la solución de la parte homogénea de la recurrencia es:
        \begin{equation*}
            x_n^{(h)} = 3^n(c_1 + c_2n)
        \end{equation*}
    
        La función de ajuste es $f(n) = 3\cdot 2^n + 7\cdot 3^n$. Por lo visto en teoría, tenemos que
        una solución particular de la recurrencia es:
        \begin{equation*}
            x_n^{(p)} = c_3\cdot 2^n + c_4n^2\cdot 3^n
        \end{equation*}
    
        Para el cálculo de $c_3$ y $c_4$ no intervienen los valores iniciales.
        Calculamos primero $x_{n+2}^{(p)}$ y $x_{n+1}^{(p)}$:
        \begin{align*}
            x_{n+2}^{(p)} &= c_3\cdot 2^{n+2} + c_4(n+2)^2\cdot 3^{n+2} = 4c_3\cdot 2^n + 9c_4(n^2+4n+4)\cdot 3^{n} \\
            x_{n+1}^{(p)} &= c_3\cdot 2^{n+1} + c_4(n+1)^2\cdot 3^{n+1} = 2c_3\cdot 2^n + 3c_4(n^2+2n+1)\cdot 3^{n}
        \end{align*}
    
        Usando que $x_n^{(p)}$ es solución de la recurrencia, tenemos:
        \begin{align*}
            3\cdot 2^n + 7\cdot 3^n &= x_{n+2}^{(p)} - 6x_{n+1}^{(p)} + 9x_n^{(p)} \\
            &= 4c_3\cdot 2^n + 9c_4(n^2+4n+4)\cdot 3^{n} -\\&\qquad - 6(2c_3\cdot 2^n + 3c_4(n^2+2n+1)\cdot 3^{n}) +\\&\qquad + 9(c_3\cdot 2^n + c_4n^2\cdot 3^{n}) \\
            &= 4c_3\cdot 2^n + 9c_4(n^2+4n+4)\cdot 3^{n} -\\&\qquad - 12c_3\cdot 2^n - 18c_4(n^2+2n+1)\cdot 3^{n} +\\&\qquad + 9c_3\cdot 2^n + 9c_4n^2\cdot 3^{n} \\
            &= 2^n (4c_3 - 12c_3 + 9c_3) + 3^n(9c_4(n^2+4n+4) - 18c_4(n^2+2n+1) + 9c_4n^2) \\
            &= c_32^n + c_43^n(\cancel{9n^2} + \bcancel{36n} + 36 - \cancel{18n^2} - \bcancel{36n} - 18 + \cancel{9n^2}) \\
            &= c_32^n + 18\cdot c_43^n
        \end{align*}
    
        Igualando coeficientes, tenemos:
        \begin{equation*}
            \left\{\begin{array}{l}
                c_3 = 3 \\
                18c_4 = 7 \Longrightarrow c_4 = \nicefrac{7}{18}
            \end{array}\right.
        \end{equation*}
    
        Por tanto, una solución particular de la recurrencia es:
        \begin{equation*}
            x_n^{(p)} = 3\cdot 2^n + \frac{7}{18}n^2\cdot 3^n
        \end{equation*}
    
        Por tanto, la solución general de la recurrencia es:
        \begin{align*}
            x_n &= x_n^{(h)} + x_n^{(p)} \\&= c_1\cdot 3^n + c_2n\cdot 3^n + 3\cdot 2^n + \frac{7}{18}n^2\cdot 3^n=\\
            &= 3^n\left(c_1 + c_2n + \frac{7}{18}n^2\right) + 3\cdot 2^n
        \end{align*}
    
        Finalmente, imponemos las condiciones iniciales, sabiendo que $x_0=u_0=1$ y $x_1=u_1=4$:
        \begin{align*}
            x_0 = 1 &\Longrightarrow 1 = 3^0\left(c_1 + c_2\cdot 0 + \frac{7}{18}\cdot 0^2\right) + 3\cdot 2^0 = c_1 + 3 \Longrightarrow c_1 = -2\\
            x_1 = 4 &\Longrightarrow 4 = 3^1\left(-2 + c_2\cdot 1 + \frac{7}{18}\cdot 1^2\right) + 3\cdot 2^1 = \cancel{-6} + 3c_2 + \frac{7}{6} + \cancel{6} \Longrightarrow c_2 = \dfrac{4-\nicefrac{7}{6}}{3} = \dfrac{17}{18}
        \end{align*}
    
        Por tanto, para el caso particular dado tenemos que la solución de la recurrencia es:
        \begin{align*}
            x_n &= 3^n\left(-2 + \frac{17}{18}n + \frac{7}{18}n^2\right) + 3\cdot 2^n \\
            &= 3^n\left(-2 + \frac{n(17 + 7n)}{18}\right) + 3\cdot 2^n
        \end{align*}
        
    
    
    \end{ejercicio}


    
\end{document}