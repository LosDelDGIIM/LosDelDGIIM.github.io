\documentclass[12pt]{article}

% Idioma y codificación
\usepackage[spanish, es-tabla]{babel}       %es-tabla para que se titule "Tabla"
\usepackage[utf8]{inputenc}

% Márgenes
\usepackage[a4paper,top=3cm,bottom=2.5cm,left=3cm,right=3cm]{geometry}

% Comentarios de bloque
\usepackage{verbatim}

% Paquetes de links
\usepackage[hidelinks]{hyperref}    % Permite enlaces
\usepackage{url}                    % redirecciona a la web

% Más opciones para enumeraciones
\usepackage{enumitem}

% Personalizar la portada
\usepackage{titling}

% Paquetes de tablas
\usepackage{multirow}


%------------------------------------------------------------------------

%Paquetes de figuras
\usepackage{caption}
\usepackage{subcaption} % Figuras al lado de otras
\usepackage{float}      % Poner figuras en el sitio indicado H.


% Paquetes de imágenes
\usepackage{graphicx}       % Paquete para añadir imágenes
\usepackage{transparent}    % Para manejar la opacidad de las figuras

% Paquete para usar colores
\usepackage[dvipsnames]{xcolor}
\usepackage{pagecolor}      % Para cambiar el color de la página

% Habilita tamaños de fuente mayores
\usepackage{fix-cm}

% Para los gráficos
\usepackage{tikz}

% Para poder situar los nodos en los grafos
\usetikzlibrary{positioning}


%------------------------------------------------------------------------

% Paquetes de matemáticas
\usepackage{mathtools, amsfonts, amssymb, mathrsfs}
\usepackage[makeroom]{cancel}     % Simplificar tachando
\usepackage{polynom}    % Divisiones y Ruffini
\usepackage{units} % Para poner fracciones diagonales con \nicefrac

\usepackage{pgfplots}   %Representar funciones
\pgfplotsset{compat=1.18}  % Versión 1.18

\usepackage{tikz-cd}    % Para usar diagramas de composiciones
\usetikzlibrary{calc}   % Para usar cálculo de coordenadas en tikz

%Definición de teoremas, etc.
\usepackage{amsthm}
%\swapnumbers   % Intercambia la posición del texto y de la numeración

\theoremstyle{plain}

\makeatletter
\@ifclassloaded{article}{
  \newtheorem{teo}{Teorema}[section]
}{
  \newtheorem{teo}{Teorema}[chapter]  % Se resetea en cada chapter
}
\makeatother

\newtheorem{coro}{Corolario}[teo]           % Se resetea en cada teorema
\newtheorem{prop}[teo]{Proposición}         % Usa el mismo contador que teorema
\newtheorem{lema}[teo]{Lema}                % Usa el mismo contador que teorema

\theoremstyle{remark}
\newtheorem*{observacion}{Observación}

\theoremstyle{definition}

\makeatletter
\@ifclassloaded{article}{
  \newtheorem{definicion}{Definición} [section]     % Se resetea en cada chapter
}{
  \newtheorem{definicion}{Definición} [chapter]     % Se resetea en cada chapter
}
\makeatother

\newtheorem*{notacion}{Notación}
\newtheorem*{ejemplo}{Ejemplo}
\newtheorem*{ejercicio*}{Ejercicio}             % No numerado
\newtheorem{ejercicio}{Ejercicio} [section]     % Se resetea en cada section


% Modificar el formato de la numeración del teorema "ejercicio"
\renewcommand{\theejercicio}{%
  \ifnum\value{section}=0 % Si no se ha iniciado ninguna sección
    \arabic{ejercicio}% Solo mostrar el número de ejercicio
  \else
    \thesection.\arabic{ejercicio}% Mostrar número de sección y número de ejercicio
  \fi
}


% \renewcommand\qedsymbol{$\blacksquare$}         % Cambiar símbolo QED
%------------------------------------------------------------------------

% Paquetes para encabezados
\usepackage{fancyhdr}
\pagestyle{fancy}
\fancyhf{}

\newcommand{\helv}{ % Modificación tamaño de letra
\fontfamily{}\fontsize{12}{12}\selectfont}
\setlength{\headheight}{15pt} % Amplía el tamaño del índice


%\usepackage{lastpage}   % Referenciar última pag   \pageref{LastPage}
\fancyfoot[C]{\thepage}

%------------------------------------------------------------------------

% Conseguir que no ponga "Capítulo 1". Sino solo "1."
\makeatletter
\@ifclassloaded{book}{
  \renewcommand{\chaptermark}[1]{\markboth{\thechapter.\ #1}{}} % En el encabezado
    
  \renewcommand{\@makechapterhead}[1]{%
  \vspace*{50\p@}%
  {\parindent \z@ \raggedright \normalfont
    \ifnum \c@secnumdepth >\m@ne
      \huge\bfseries \thechapter.\hspace{1em}\ignorespaces
    \fi
    \interlinepenalty\@M
    \Huge \bfseries #1\par\nobreak
    \vskip 40\p@
  }}
}
\makeatother

%------------------------------------------------------------------------
% Paquetes de cógido
\usepackage{minted}
\renewcommand\listingscaption{Código fuente}

\usepackage{fancyvrb}
% Personaliza el tamaño de los números de línea
\renewcommand{\theFancyVerbLine}{\small\arabic{FancyVerbLine}}

% Estilo para C++
\newminted{cpp}{
    frame=lines,
    framesep=2mm,
    baselinestretch=1.2,
    linenos,
    escapeinside=||
}

% para minted
\definecolor{LightGray}{rgb}{0.95,0.95,0.92}
\setminted{
    linenos=true,
    stepnumber=5,
    numberfirstline=true,
    autogobble,
    breaklines=true,
    breakautoindent=true,
    breaksymbolleft=,
    breaksymbolright=,
    breaksymbolindentleft=0pt,
    breaksymbolindentright=0pt,
    breaksymbolsepleft=0pt,
    breaksymbolsepright=0pt,
    fontsize=\footnotesize,
    bgcolor=LightGray,
    numbersep=10pt
}


\usepackage{listings} % Para incluir código desde un archivo

\renewcommand\lstlistingname{Código Fuente}
\renewcommand\lstlistlistingname{Índice de Códigos Fuente}

% Definir colores
\definecolor{vscodepurple}{rgb}{0.5,0,0.5}
\definecolor{vscodeblue}{rgb}{0,0,0.8}
\definecolor{vscodegreen}{rgb}{0,0.5,0}
\definecolor{vscodegray}{rgb}{0.5,0.5,0.5}
\definecolor{vscodebackground}{rgb}{0.97,0.97,0.97}
\definecolor{vscodelightgray}{rgb}{0.9,0.9,0.9}

% Configuración para el estilo de C similar a VSCode
\lstdefinestyle{vscode_C}{
  backgroundcolor=\color{vscodebackground},
  commentstyle=\color{vscodegreen},
  keywordstyle=\color{vscodeblue},
  numberstyle=\tiny\color{vscodegray},
  stringstyle=\color{vscodepurple},
  basicstyle=\scriptsize\ttfamily,
  breakatwhitespace=false,
  breaklines=true,
  captionpos=b,
  keepspaces=true,
  numbers=left,
  numbersep=5pt,
  showspaces=false,
  showstringspaces=false,
  showtabs=false,
  tabsize=2,
  frame=tb,
  framerule=0pt,
  aboveskip=10pt,
  belowskip=10pt,
  xleftmargin=10pt,
  xrightmargin=10pt,
  framexleftmargin=10pt,
  framexrightmargin=10pt,
  framesep=0pt,
  rulecolor=\color{vscodelightgray},
  backgroundcolor=\color{vscodebackground},
}

%------------------------------------------------------------------------

% Comandos definidos
\newcommand{\bb}[1]{\mathbb{#1}}
\newcommand{\cc}[1]{\mathcal{#1}}

% I prefer the slanted \leq
\let\oldleq\leq % save them in case they're every wanted
\let\oldgeq\geq
\renewcommand{\leq}{\leqslant}
\renewcommand{\geq}{\geqslant}

% Si y solo si
\newcommand{\sii}{\iff}

% Letras griegas
\newcommand{\eps}{\epsilon}
\newcommand{\veps}{\varepsilon}
\newcommand{\lm}{\lambda}

\newcommand{\ol}{\overline}
\newcommand{\ul}{\underline}
\newcommand{\wt}{\widetilde}
\newcommand{\wh}{\widehat}

\let\oldvec\vec
\renewcommand{\vec}{\overrightarrow}

% Derivadas parciales
\newcommand{\del}[2]{\frac{\partial #1}{\partial #2}}
\newcommand{\Del}[3]{\frac{\partial^{#1} #2}{\partial #3^{#1}}}
\newcommand{\deld}[2]{\dfrac{\partial #1}{\partial #2}}
\newcommand{\Deld}[3]{\dfrac{\partial^{#1} #2}{\partial #3^{#1}}}


\newcommand{\AstIg}{\stackrel{(\ast)}{=}}
\newcommand{\Hop}{\stackrel{L'H\hat{o}pital}{=}}

\newcommand{\red}[1]{{\color{red}#1}} % Para integrales, destacar los cambios.

% Método de integración
\newcommand{\MetInt}[2]{
    \left[\begin{array}{c}
        #1 \\ #2
    \end{array}\right]
}

% Declarar aplicaciones
% 1. Nombre aplicación
% 2. Dominio
% 3. Codominio
% 4. Variable
% 5. Imagen de la variable
\newcommand{\Func}[5]{
    \begin{equation*}
        \begin{array}{rrll}
            #1:& #2 & \longrightarrow & #3\\
               & #4 & \longmapsto & #5
        \end{array}
    \end{equation*}
}

%------------------------------------------------------------------------

% Define a custom command for email addresses
\newcommand{\email}[1]{\href{mailto:#1}{{{\color{blue}#1}}}}

% Comando para \bf en vez de \mathbf
\renewcommand{\bf}[1]{\mathbf{#1}}

% Operador matemático para \Con
\DeclareMathOperator{\Con}{Con}

% Operador matemático para los átomos
\DeclareMathOperator{\Atm}{Atm}

% Operador matemático para el cardinal
\DeclareMathOperator{\Car}{Car}

% Comando para \mcd y \mcm
\DeclareMathOperator{\mcd}{mcd}
\DeclareMathOperator{\mcm}{mcm}


% Paquete para mapas de Karnaugh
\usepackage{karnaugh-map}

% Tachar filas
\usetikzlibrary{tikzmark}

\usepackage{forest}


\begin{document}

     % 1. Foto de fondo
    % 2. Título
    % 3. Encabezado Izquierdo
    % 4. Color de fondo
    % 5. Coord x del titulo
    % 6. Coord y del titulo
    % 7. Fecha

    
    % 1. Foto de fondo
% 2. Título
% 3. Encabezado Izquierdo
% 4. Color de fondo
% 5. Coord x del titulo
% 6. Coord y del titulo
% 7. Fecha

\newcommand{\portada}[7]{

    \portadaBase{#1}{#2}{#3}{#4}{#5}{#6}{#7}
    \portadaBook{#1}{#2}{#3}{#4}{#5}{#6}{#7}
}

\newcommand{\portadaExamen}[7]{

    \portadaBase{#1}{#2}{#3}{#4}{#5}{#6}{#7}
    \portadaArticle{#1}{#2}{#3}{#4}{#5}{#6}{#7}
}




\newcommand{\portadaBase}[7]{

    % Tiene la portada principal y la licencia Creative Commons
    
    % 1. Foto de fondo
    % 2. Título
    % 3. Encabezado Izquierdo
    % 4. Color de fondo
    % 5. Coord x del titulo
    % 6. Coord y del titulo
    % 7. Fecha
    
    
    \thispagestyle{empty}               % Sin encabezado ni pie de página
    \newgeometry{margin=0cm}        % Márgenes nulos para la primera página
    
    
    % Encabezado
    \fancyhead[L]{\helv #3}
    \fancyhead[R]{\helv \nouppercase{\leftmark}}
    
    
    \pagecolor{#4}        % Color de fondo para la portada
    
    \begin{figure}[p]
        \centering
        \transparent{0.3}           % Opacidad del 30% para la imagen
        
        \includegraphics[width=\paperwidth, keepaspectratio]{assets/#1}
    
        \begin{tikzpicture}[remember picture, overlay]
            \node[anchor=north west, text=white, opacity=1, font=\fontsize{60}{90}\selectfont\bfseries\sffamily, align=left] at (#5, #6) {#2};
            
            \node[anchor=south east, text=white, opacity=1, font=\fontsize{12}{18}\selectfont\sffamily, align=right] at (9.7, 3) {\textbf{\href{https://losdeldgiim.github.io/}{Los Del DGIIM}}};
            
            \node[anchor=south east, text=white, opacity=1, font=\fontsize{12}{15}\selectfont\sffamily, align=right] at (9.7, 1.8) {Doble Grado en Ingeniería Informática y Matemáticas\\Universidad de Granada};
        \end{tikzpicture}
    \end{figure}
    
    
    \restoregeometry        % Restaurar márgenes normales para las páginas subsiguientes
    \pagecolor{white}       % Restaurar el color de página
    
    
    \newpage
    \thispagestyle{empty}               % Sin encabezado ni pie de página
    \begin{tikzpicture}[remember picture, overlay]
        \node[anchor=south west, inner sep=3cm] at (current page.south west) {
            \begin{minipage}{0.5\paperwidth}
                \href{https://creativecommons.org/licenses/by-nc-nd/4.0/}{
                    \includegraphics[height=2cm]{assets/Licencia.png}
                }\vspace{1cm}\\
                Esta obra está bajo una
                \href{https://creativecommons.org/licenses/by-nc-nd/4.0/}{
                    Licencia Creative Commons Atribución-NoComercial-SinDerivadas 4.0 Internacional (CC BY-NC-ND 4.0).
                }\\
    
                Eres libre de compartir y redistribuir el contenido de esta obra en cualquier medio o formato, siempre y cuando des el crédito adecuado a los autores originales y no persigas fines comerciales. 
            \end{minipage}
        };
    \end{tikzpicture}
    
    
    
    % 1. Foto de fondo
    % 2. Título
    % 3. Encabezado Izquierdo
    % 4. Color de fondo
    % 5. Coord x del titulo
    % 6. Coord y del titulo
    % 7. Fecha


}


\newcommand{\portadaBook}[7]{

    % 1. Foto de fondo
    % 2. Título
    % 3. Encabezado Izquierdo
    % 4. Color de fondo
    % 5. Coord x del titulo
    % 6. Coord y del titulo
    % 7. Fecha

    % Personaliza el formato del título
    \pretitle{\begin{center}\bfseries\fontsize{42}{56}\selectfont}
    \posttitle{\par\end{center}\vspace{2em}}
    
    % Personaliza el formato del autor
    \preauthor{\begin{center}\Large}
    \postauthor{\par\end{center}\vfill}
    
    % Personaliza el formato de la fecha
    \predate{\begin{center}\huge}
    \postdate{\par\end{center}\vspace{2em}}
    
    \title{#2}
    \author{\href{https://losdeldgiim.github.io/}{Los Del DGIIM}}
    \date{Granada, #7}
    \maketitle
    
    \tableofcontents
}




\newcommand{\portadaArticle}[7]{

    % 1. Foto de fondo
    % 2. Título
    % 3. Encabezado Izquierdo
    % 4. Color de fondo
    % 5. Coord x del titulo
    % 6. Coord y del titulo
    % 7. Fecha

    % Personaliza el formato del título
    \pretitle{\begin{center}\bfseries\fontsize{42}{56}\selectfont}
    \posttitle{\par\end{center}\vspace{2em}}
    
    % Personaliza el formato del autor
    \preauthor{\begin{center}\Large}
    \postauthor{\par\end{center}\vspace{3em}}
    
    % Personaliza el formato de la fecha
    \predate{\begin{center}\huge}
    \postdate{\par\end{center}\vspace{5em}}
    
    \title{#2}
    \author{\href{https://losdeldgiim.github.io/}{Los Del DGIIM}}
    \date{Granada, #7}
    \thispagestyle{empty}               % Sin encabezado ni pie de página
    \maketitle
    \vfill
}
    \portadaExamen{etsiitA4.jpg}{LMD\\Examen II}{Lógica y Métodos Discretos. Examen I}{MidnightBlue}{-8}{28}{2023-2024}{Arturo Olivares Martos}

    \begin{description}
        \item[Asignatura] Lógica y Métodos Discretos.
        \item[Curso Académico] 2022-23.
        \item[Grado] Doble Grado en Ingienería Informática y Matemáticas.
        \item[Grupo] Único.
        \item[Profesor] Francisco Miguel García Olmedo.
        \item[Descripción] Convocatoria ordinaria.
        \item[Fecha] 19 de junio de 2023.
        % \item[Duración] 60 minutos.
    
    \end{description}
    \newpage

\begin{ejercicio} Demuestre que las siguientes recurrencias son dos definiciones
    equivalentes de una misma sucesión de números naturales, digamos $\{u_n\}_{n\geq 1}$.
    \begin{equation*}
        \begin{array}{ll}
            f_1 = 2 & g_1 = 2 \\
            f_2 = 4 & g_2 = 4 \\
            f_3 = 8 & g_3 = 8 \\
            & g_4 = 14 \\
            f_n = f_{n-1} + f_{n-2} + f_{n-3} \quad (n > 3) \qquad & g_n = 2g_{n-1} - g_{n-4} \quad (n > 4)
        \end{array}
    \end{equation*}
    ¿Puede decir, razonando la respuesta, cuánto valría el elemento $u_{49}$? ¿Puede imaginar un problema
    combinatorio de tamaño $n\geq 1$ que sea contado por cualquiera de las dos definiciones recurrentes?\\

    Fijado $n\in \bb{N}$, $n\geq 5$, tenemos que:
    \begin{align*}
        f_n &= f_{n-1} + f_{n-2} + f_{n-3} =\\
        &= 2f_{n-1} - f_{n-1} + f_{n-2} + f_{n-3} =\\
        &= 2f_{n-1} - \cancel{f_{n-2}} - \cancel{f_{n-3}} - f_{n-4} + \cancel{f_{n-2}} + \cancel{f_{n-3}} =\\
        &= 2f_{n-1} - f_{n-4}        
    \end{align*}

    Además, tenemos que $f_4=2+4+8=14=g_4$, por lo que las dos definiciones coinciden en los primeros cuatro términos.
    Por tanto, tenemos que:
    \begin{equation*}
        f_n = g_n \quad \forall n\in \bb{N}\setminus \{0\}
    \end{equation*}

    Esta recurrencia cuenta el número de formas de lanzar una moneda $n$ veces sin que aparezca
    una secuencia de $4$ caras seguidas, explicada en \href{https://oeis.org/A135491}{\color{blue}\underline{esta web}}.
    % // TODO: Calcular u_{49}
    % Problema combinatorio explicado
    % https://math.stackexchange.com/questions/3749177/fibonacci-and-tossing-coins
\end{ejercicio}


\begin{ejercicio}
    Establezca y seguidamente resuelva un problema de recurrencia que permita contar el número $a_n$
    $(n \geq 2)$ de cadenas de $n$ elementos del conjunto $\{0, 1, 2\}$ (i.e. elementos de $3^n$) cumpliendo cada
    una de ellas la condición de contener un único $0$ y un único $1$.\\

    Resolvamos en primer este problema mediante un enfoque combinatorio. Para ello, fijado $n\in \bb{N}$, $n\geq 2$, tenemos que:
    \begin{itemize}
        \item Hay $n$ formas de elegir la posición del $0$.
        \item Hay $n-1$ formas de elegir la posición del $1$.
        \item El resto de elementos, $n-2$, serán el número $2$, por lo que hay $1$ forma de elegirlos.
    \end{itemize}

    Por tanto, tenemos que:
    \begin{equation*}
        a_n = n(n-1) = n^2 - n \quad \forall n\in \bb{N}, n\geq 2
    \end{equation*}

    Ahora, establezcamos la recurrencia. Fijado $n\in \bb{N}$, $n\geq 2$, tenemos que:
    \begin{itemize}
        \item Si en la posición $n$ hay un $2$, entonces hay $a_{n-1}$ formas de completar la cadena.
        \item Si en la posición $n$ hay un $0$, entonces hay $n-1$ formas de elegir la posición del $1$ y $1$ forma de elegir el resto de elementos,
        por lo que hay $n-1$ formas de completar la cadena.
        \item Si en la posición $n$ hay un $1$, de igual forma hay $n-1$ formas de completar la cadena.
    \end{itemize}

    Por tanto, tenemos que la recurrencia queda establecida como:
    \begin{equation*}
        \left\{
            \begin{array}{ll}
                a_2 = 2 & \\
                a_n = a_{n-1} + 2(n-1) \quad \forall n\in \bb{N}, n\geq 3
            \end{array}
        \right.
    \end{equation*}

    Resolvamos ahora la recurrencia lineal no homogénea de primer orden. Tenemos que la ecuación
    característica es $x-1=0$, por lo que $\lm=1$ es la única raíz, con multiplicidad simple.
    Por tanto, la solución homogénea es:
    \begin{equation*}
        \{a_n^{(h)}\} = c_1\cdot 1^n = c_1 \quad \forall n\in \bb{N}\hspace{1cm} (c_1\in \bb{C})
    \end{equation*}
    
    
    Respecto a la parte no homogénea,
    tenemos que $f(n)=2(n-1)=1^n(2n-2)$, por lo que una solución particular es:
    \begin{equation*}
        \{a_n^{(p)}\} = n^1\cdot 1^n \cdot (c_2n + c_3) = n(c_2n + c_3) \quad \forall n\in \bb{N}\hspace{1cm} (c_2, c_3\in \bb{C})
    \end{equation*}

    Para determinar los valores de las constantes, como $\{a_n^{(p)}\}$ es una solución particular,
    para todo $n\in \bb{N}$, $n\geq 2$, tenemos que:
    \begin{equation*}
        a_n =a_{n-1} + 2(n-1) \Longrightarrow n(c_2n + c_3) = (n-1)(c_2(n-1) + c_3) + 2(n-1)
    \end{equation*}

    Operando, obtenemos para todo $n\in \bb{N}$, $n\geq 2$:
    \begin{align*}
        {c_2n^2}+{c_3n} &= c_2(n-1)^2 + (n-1)(2+c_3)\\
        \cancel{c_2n^2}+\bcancel{c_3n} &=  \cancel{c_2n^2} - 2c_2n + c_2 + 2n - 2 + \bcancel{nc_3} - c_3\\
        0 &= -2c_2n + c_2 + 2n - 2 - c_3\\
        0 &= 2n(1-c_2) + c_2 - 2 - c_3
    \end{align*}

    Igualando los coeficientes, obtenemos el sistema:
    \begin{equation*}
        \left\{
            \begin{array}{l}
                2(1-c_2) = 0  \Longrightarrow c_2 = 1\\
                c_2 - 2 - c_3 = 0  \Longrightarrow 1 - 2 - c_3 = 0 \Longrightarrow c_3 = -1
            \end{array}
        \right.
    \end{equation*}

    Por tanto, la solución particular es:
    \begin{equation*}
        \{a_n^{(p)}\} = n(1\cdot n - 1) = n(n-1) \quad \forall n\in \bb{N},~n\geq 2
    \end{equation*}

    Por tanto, la solución general es:
    \begin{equation*}
        a_n = \{a_n^{(h)}\} + \{a_n^{(p)}\} = c_1 + n(n-1) \quad \forall n\in \bb{N}\hspace{1cm} (c_1\in \bb{C})
    \end{equation*}

    Para determinar el valor de la constante, tenemos que $a_2=2$, por lo que:
    \begin{align*}
        a_2 = 2  = c_1 + 2(2-1) = c_1 + 2 \Longrightarrow c_1 = 0
    \end{align*}

    Por tanto, la solución a la recurrencia es:
    \begin{equation*}
        a_n = n(n-1) \quad \forall n\in \bb{N},~n\geq 2
    \end{equation*}

    Llegamos efectivamente al mismo resultado que en el enfoque combinatorio.
\end{ejercicio}


\begin{ejercicio} \label{ej:3}
    Considere el conjunto de fórmulas proposicionales $\Gamma$ siguiente:
    \begin{align*}
        \Gamma = \left\{ \right.&
            a\lor b, \\ &
            a\rightarrow (c\lor d), \\ &
            (a\land d) \rightarrow c, \\ &
            (a\land \lnot d) \rightarrow e, \\ &
            b\rightarrow (d\lor e), \\ &
            (c\lor \lnot d) \rightarrow e, \\ &
            e\rightarrow d \left.\right\}
    \end{align*}
    Considere también la fórmula $\varphi \equiv c\land d$. Haciendo uso del
    algoritmo de Davis\&Putman, decida si es cierta o no la afirmación $\Gamma \models \varphi$. En caso de
    no serlo, caracterice a las asignaciones que servirían para poner de manifiesto ese hecho.\\

    Para resolver este ejercicio, vamos a aplicar el algoritmo de Davis\&Putman. Para ello, primero
    vamos a transformar las fórmulas a su forma normal conjuntiva:
    \begin{align*}
        \varphi_1 :& = a\lor b\\ \\
        \varphi_2 :& = a\rightarrow (c\lor d)\\
        & = \lnot a \lor (c\lor d) = \lnot a \lor c \lor d\\ \\
        \varphi_3 :& = (a\land d) \rightarrow c\\
        & = \lnot (a\land d) \lor c = \lnot a \lor \lnot d \lor c\\ \\
        \varphi_4 :& = (a\land \lnot d) \rightarrow e\\
        & = \lnot (a\land \lnot d) \lor e = \lnot a \lor d \lor e\\ \\
        \varphi_5 :& = b\rightarrow (d\lor e)\\
        & = \lnot b \lor (d\lor e) = \lnot b \lor d \lor e\\ \\
        \varphi_6 :& = (c\lor \lnot d) \rightarrow e\\
        & = \lnot (c\lor \lnot d) \lor e = \lnot c \lor d \lor e\\ \\
        \varphi_7 :& = e\rightarrow d\\
        & = \lnot e \lor d \\ \\
        \lnot \varphi & = \lnot (c\land d) = \lnot c \lor \lnot d
    \end{align*}

    Queremos por tanto estudiar la satisfacibilidad del conjunto $\Sigma = \Gamma \cup \{\lnot \varphi\}$
    mediante el algoritmo de Davis\&Putman, cuyo árbol de decisión se muestra en la Figura~\ref{fig:DyP_3}.
    \begin{figure}
        \centering
        \begin{forest}
            for tree={draw=none, minimum size=2em, l=2cm, s sep=5mm, align=center, edge={-stealth}}
            [
                $\Sigma~{=}~\{a\lor b; \lnot a \lor c \lor d; \lnot a \lor \lnot d \lor c; \lnot a \lor d \lor e; \lnot b \lor d \lor e; \lnot c \lor d \lor e; \lnot e \lor d; \lnot c \lor \lnot d\}$\\~\\
                \fbox{R4. $\lm=a$. {\color{red} $v(\lnot a)=1$}}
                [
                    \fbox{$\lm=a$}\\
                    $\Sigma_1~{=}~\{b; \lnot b \lor d \lor e; \lnot c \lor d \lor e;\lnot e \lor d; \lnot c \lor \lnot d\}$\\~\\
                    \fbox{R2. $\lm=b$. {\color{red} $v(b)=1$}}
                    [
                        $\Sigma_1'~{=}~\{\lnot b \lor d \lor e; \lnot c \lor d \lor e;\lnot e \lor d; \lnot c \lor \lnot d\}{\color{red}~\neq \emptyset}$
                        [
                            $\Sigma_{11}~{=}~\{d \lor e; \lnot c \lor d \lor e;\lnot e \lor d; \lnot c \lor \lnot d\}$\\~\\
                            \fbox{R3. $\lm=\lnot c$ es un literal puro. {\color{red} $v(\lnot c)=1$}}                            
                            [
                                $\Sigma_{111}~{=}~\{d \lor e; \lnot e \lor d\}$\\~\\
                                \fbox{R3. $\lm=d$ es un literal puro. {\color{red} $v(d)=1$}}
                                [
                                    $\Sigma_{1111} ~{=}~\emptyset$
                                ]
                            ]
                        ]
                    ]
                ]
                [
                    \fbox{$\lm^c=\lnot a$}\\
                    $\Sigma_2~{=}~\{c \lor d; \lnot d \lor c; \lnot d \lor e; \lnot b \lor d \lor e;$\\\qquad  $ \lnot c \lor d \lor e;\lnot e \lor d; \lnot c \lor \lnot d\}$\\~\\
                ]
            ]
        \end{forest}
        \caption{Algoritmo de Davis y Putman del Ejercicio~\ref{ej:3}.}
        \label{fig:DyP_3}
    \end{figure}

    Como podemos ver, sin haber terminado el árbol de decisión, ya hemos visto que
    $\Sigma$ es satisfacible, con la asignación siguiente:
    \begin{align*}
        1 &= v(b) = v(d) \\
        0 &= v(a) = v(c)
    \end{align*}
    El valor de $v(e)$ es indiferente. Por tanto, como
    $\Sigma$ es satisfacible, tenemos que $\Gamma \not\models \varphi$.
\end{ejercicio}

\begin{ejercicio}
    Demuestre que, para todo conjunto de fórmulas proposicionales $\Gamma\cup \{\varphi\}$ son
    equivalentes las siguientes afirmaciones:
    \begin{enumerate}
        \item $\Gamma \models \varphi$.
        \item Existe un subconjunto finito de $\Gamma$, digamos $\Gamma_{f, \varphi}$, tal que $\Gamma_{f, \varphi} \models \varphi$.
    \end{enumerate}
    \begin{proof}
        Demostramos mediante una doble implicación.
        \begin{description}
            \item[$\Longrightarrow)$]
            \begin{comment}
            %Supongamos que $\Gamma \models \varphi$. Entonces, por definición, tenemos que para toda asignación $v$ tal que $v_{\ast}(\Gamma) = 1$, se tiene que $v(\varphi) = 1$.

            Demostraremos el recíproco. Supongamos que para todo subconjunto finito $\Gamma_{f, \varphi}\subset~\Gamma$ se tiene que $\Gamma_{f, \varphi} \not \models \varphi$,
            o equivalentemente, que $\Gamma_{f, \varphi} \cup \{\lnot \varphi\}$ es satisfacible.
            Es decir, supongamos que para todo subconjunto finito $\Gamma_{f, \varphi}\subset~\Gamma$, existe una asignación $v$ tal que $v_{\ast}(\Gamma_{f, \varphi} \cup \{\lnot \varphi\}) \subset \{1\}$.

            Buscamos ahora una asignación $v$ tal que $v_{\ast}(\Gamma \cup \{\lnot \varphi\}) \subset \{1\}$.
            Para cada $n\in \bb{N}$, $n\geq 1$, definimos $\Gamma_n = \{p_1, \ldots, p_n\}$, con $p_i\in \Gamma$ para todo $i\in \{1, \ldots, n\}$.
            Como $\Gamma_n \cup \{\lnot \varphi\}$ es finito, tenemos que existe una asignación $\wt{v}_n$ tal que se tiene $(\wt{v}_{n})_{\ast}(\Gamma_n \cup \{\lnot \varphi\}) \subset \{1\}$.

            Pues bien, queremos construir una secuencia de asignaciones $\{v_n\}$
            tal que $(v_n)_\ast(\Gamma_n\cup \{\lnot \varphi\})\subset \{1\}$ para todo $n\in \bb{N}$, $n\geq 1$
            y tal que $v_{n+1}$ extienda a $v_n$ para todo $n\in \bb{N}$, $n\geq 1$.
            \begin{itemize}
                \item Para $n=1$, definimos $v_1 = \wt{v}_1$.
                \item Supongamos que hemos definido $v_n$ para algún $n\in \bb{N}$, $n\geq 1$.
                Definimos $v_{n+1}$ como sigue:
                \begin{equation*}
                    v_{n+1}(p) = \left\{
                        \begin{array}{ll}
                            v_n(p) & \text{si } p\in \Gamma_n \cup \{\lnot \varphi\}\\
                            1 & \text{si } p = p_{n+1}\\
                        \end{array}
                    \right.
                \end{equation*}
            \end{itemize}

            Por tanto, tenemos que para todo $n\in \bb{N}$, $n\geq 1$, se tiene que $(v_n)_{\ast}(\Gamma_n \cup \{\lnot \varphi\}) \subset \{1\}$.
            Como $\Gamma$ es numerable, entonces $\Gamma = \bigcup_{n\geq 1} \Gamma_n$. Definimos entonces
            $v$ para cada $p_n\in \Gamma$ como sigue:
            \begin{equation*}
                v(p_n) = v_n(p_n) \quad \forall n\in \bb{N}, n\geq 1
            \end{equation*}

            Para $\varphi$, tenemos que:
            \begin{equation*}
                v(\lnot \varphi) = v_n(\lnot \varphi) = 1 \quad \forall n\in \bb{N}, n\geq 1
            \end{equation*}

            Por tanto, tenemos que $v_{\ast}(\Gamma \cup \{\lnot \varphi\}) \subset \{1\}$, por lo que $\Gamma \not \models \varphi$.
            Por haber demostrado en recíproco, tenemos que $\Gamma \models \varphi$.
            \end{comment}
            % // TODO: Subconjunto finito de $\Gamma$.


            \item[$\Longleftarrow)$]
            
            Supongamos que existe un subconjunto finito $\Gamma_{f, \varphi}\subset \Gamma$ tal que $\Gamma_{f, \varphi} \models \varphi$.
            Entonces, tenemos que para toda asignación $v$ tal que $v_{\ast}(\Gamma_{f, \varphi}) \subset \{1\}$, se tiene que $v(\varphi) = 1$.

            Sea entonces $v$ una asignación tal que $v_{\ast}(\Gamma) \subset \{1\}$. Como $\Gamma_{f, \varphi}\subset \Gamma$, entonces
            $v_{\ast}(\Gamma_{f, \varphi}) \subset \{1\}$, por lo que $v(\varphi) = 1$ y tenemos que $\Gamma \models \varphi$.
        \end{description}
    \end{proof}
\end{ejercicio}

\begin{ejercicio}
    Sea $\mathbf{B}$ un álgebra de Boole. Demuestre que, para todo $a, b, c \in B$, son
    equivalentes las siguientes afirmaciones:
    \begin{enumerate}
        \item $b=c$
        \item $a+b = a+c$ y $ab=ac$
    \end{enumerate}

    \begin{proof}
        Demostramos mediante una doble implicación.
        \begin{description}
            \item[$\Longleftarrow)$]
            
            Supongamos que $a+b=a+c$ y $ab=ac$. Entonces, tenemos que:
            \begin{figure}[H]
                \centering
                \begin{subfigure}[c]{0.4\linewidth}
                    \centering
                    \begin{align*}
                        b &= b + 0\\
                        &= b + (a\ol{a})\\
                        &= (b+a) \cdot (b+\ol{a})\\
                        &= (a+c) \cdot (\ol{a}+b)\\
                        &= (a+c)\ol{a} + (a+c)b\\
                        &= a\ol{a} + c\ol{a} + ab + cb\\
                        &= 0 + c\ol{a} + ac + cb\\
                        &= c\ol{a} + c(a+b)\\
                        &= c(\ol{a} + a + b)\\
                        &= c(1+b)\\
                        &= c\cdot 1\\
                        &= c.
                    \end{align*}
                    \caption{Opción 1.}
                \end{subfigure}\hfill
                \begin{subfigure}[c]{0.4\linewidth}
                    \centering
                    \begin{align*}
                        b &= b\cdot 1\\
                        &= b\cdot (1+a) \\
                        &= b+ab\\
                        &= b+ac\\
                        &= (b+a)(b+c)\\
                        &= (a+c)(b+c)\\
                        &= c+ab\\
                        &= c+ac\\
                        &= c(1+a)\\
                        &= c\cdot 1\\
                        &= c.
                    \end{align*}
                    \caption{Opción 2.}
                \end{subfigure}
            \end{figure}

            \item[$\Longrightarrow)$]
            
            Como $a=a$ y $b=c$, entonces trivialmente $a+b=a+c$ y $ab=ac$.
        \end{description}
    \end{proof}
\end{ejercicio}



\begin{ejercicio}
    Sea $f:B_2^5\rightarrow B_2$ la función:
    \begin{equation*}
        f(a,b,c,d,e) = \sum m(2,3,7,10,12,15,27) + \sum d(5,18,19,21,23)
    \end{equation*}

    Mediante el algoritmo de Quine-McCluskey, encuentre todas sus expresiones minimales a condición de ser suma de productos.\\

    Generamos los implicantes primos de la función $f$:
    \begin{table}[H]
        \centering
        \begin{tabular}{rcc|rcc|rcc}
            \multicolumn{3}{c}{Columna 1} & \multicolumn{3}{|c|}{Columna 2} & \multicolumn{3}{c}{Columna 3} \\ \hline
            2 & 00010 & \checkmark & \{2,3\} & 0001\_ & \checkmark & \{2,3,18,19\} & \_001\_ & $\ast$
            \\ \cline{1-3} \cline{7-9}
            3 & 00011 & \checkmark & \{2,10\} & 0\_010 & $\ast$ & \{3,7,19,23\} & \_0\_11 & $\ast$
            \\
            5 & 00101 & \checkmark & \{2,18\} & \_0010 & \checkmark & \{5,7,21,23\} & \_01\_1 & $\ast$
            \\ \cline{4-6} \cline{7-9}
            10 & 01010 & \checkmark & \{3,7\} & 00\_11 & \checkmark
            \\
            12 & 01100 & $\ast$ & \{3,19\} & \_0011 & \checkmark
            \\
            18 & 10010 & \checkmark & \{5,7\} & 001\_1 & \checkmark
            \\ \cline{1-3}
            7 & 00111 & \checkmark & \{5,21\} & \_0101 & \checkmark
            \\
            19 & 10011 & \checkmark & \{18,19\} & 1001\_ & \checkmark
            \\ \cline{4-6}
            21 & 10101 & \checkmark & \{7,15\} & 0\_111 & $\ast$
            \\ \cline{1-3}
            15 & 01111 & \checkmark & \{7,23\} & \_0111 & \checkmark
            \\
            23 & 10111 & \checkmark & \{19,23\} & 10\_11 & \checkmark
            \\
            27 & 11011 & \checkmark & \{19,27\} & 1\_011 & $\ast$
            \\ \cline{1-3}
            &&& \{21,23\} & 101\_1 & \checkmark
            \\ \cline{4-6}
        \end{tabular}
    \end{table}

    Los implicantes primos son, por tanto, los que se han marcado con $\ast$. Reducimos la tabla de implicantes primos:
    \begin{table}[H]
        \centering
        \begin{tabular}{c|ll|ccccccc}
            && & \tikzmark{col2Start}2 & 3 & \tikzmark{col7Start}7 & 10 & 12 & 15 & 27 \\ \hline
            {\color{red}$\ast$}& \{12\} & 01100 &\tikzmark{Ej5fil1Start} & & & & $\circ$ & &\tikzmark{Ej5fil1End}
            \\
            {\color{blue}$\ast$}& \{2,10\} & 0\_010 &\tikzmark{Ej5fil2Start}$\circ$ & & & $\circ$ & & &\tikzmark{Ej5fil2End}
            \\
            {\color{teal}$\ast$}& \{7,15\} & 0\_111 &\tikzmark{Ej5fil3Start} & & $\circ$ & & & $\circ$ &\tikzmark{Ej5fil3End}
            \\
            {\color{purple}$\ast$}& \{19,27\} & 1\_011 &\tikzmark{Ej5fil4Start} & & & & & & $\circ$ \tikzmark{Ej5fil4End}
            \\
            & \{2,3,18,19\} & \_001\_ & $\circ$ & $\circ$ & & & & &
            \\
            & \{3,7,19,23\} & \_0\_11 & & $\circ$ & $\circ$ & & & &
            \\
            & \{5,7,21,23\} & \_01\_1 &\tikzmark{col2End} & & \tikzmark{col7End}$\circ$ & & & &
            \\ \hline
        \end{tabular}
        \tikz[remember picture] \draw[overlay, red]  ([yshift=0.25em]pic cs:Ej5fil1Start) -- ([yshift=0.25em]pic cs:Ej5fil1End);
        \tikz[remember picture] \draw[overlay, blue] ([yshift=0.25em]pic cs:Ej5fil2Start) -- ([yshift=0.25em]pic cs:Ej5fil2End);
        \tikz[remember picture] \draw[overlay, teal] ([yshift=0.25em]pic cs:Ej5fil3Start) -- ([yshift=0.25em]pic cs:Ej5fil3End);
        \tikz[remember picture] \draw[overlay, purple] ([yshift=0.25em]pic cs:Ej5fil4Start) -- ([yshift=0.25em]pic cs:Ej5fil4End);

        \tikz[remember picture] \draw[overlay, blue] ([xshift=0.25em]pic cs:col2Start) -- (pic cs:col2End);
        \tikz[remember picture] \draw[overlay, teal] ([xshift=0.25em]pic cs:col7Start) -- ([xshift=0.25em]pic cs:col7End);
    \end{table}

    Tras haber llegado a este paso, hemos detectado ya cuatro implicantes primos esenciales. No obstante, para cubrir el minterm $3$ tenemos dos opciones,
    $\{2,3,18,19\}$ y $\{3,7,19,23\}$. Por tanto, las 2 expresiones minimales dadas en forma SOP son:
    \begin{align*}
        f(a,b,c,d,e) &= \ol{a}~b~c~\ol{d}~\ol{e} + \ol{a}~\ol{c}~d~\ol{e} + \ol{a}~c~d~e + a~\ol{c}~d~e + \ol{b}~\ol{c}~d\\
        f(a,b,c,d,e) &= \ol{a}~b~c~\ol{d}~\ol{e} + \ol{a}~\ol{c}~d~\ol{e} + \ol{a}~c~d~e + a~\ol{c}~d~e + \ol{b}~d~e
    \end{align*}
\end{ejercicio}


\begin{ejercicio}
    Considere las siguientes fórmulas en un cierto lenguaje de primer orden:
    \begin{align*}
        \varphi_1 & = q(x) \land \forall y\left( p(y) \rightarrow r(x,y) \right) \\
        \varphi_2 & = \forall x\left( q(x) \rightarrow \exists y\left( p(y) \land s(x,y) \right) \right) \\
        \varphi_3 & = \forall x\left( \exists y\left(s(x,y) \land r(x,y) \right) \rightarrow t(x) \right) \\
        \varphi_4 & = \exists x\left( t(x) \land q(x) \right)
    \end{align*}
    Diga justificadamente si es cierta o no la siguiente afirmación:
    \begin{equation*}
        \varphi_1, \varphi_2, \varphi_3 \models \varphi_4
    \end{equation*}

    Veamos que $\varphi_1, \varphi_2, \varphi_3 \not\models \varphi_4$.
    Sea $\bf{A}$ una estructura tal que:
    \begin{align*}
        A &= \{0,1\}\\
        (p)^{\bf{A}} &= \{0\}\\
        (q)^{\bf{A}} &= \{0\}\\
        (r)^{\bf{A}} &= \{\langle 0,0\rangle\}\\
        (s)^{\bf{A}} &= \{\langle 0,0\rangle\}\\
        (t)^{\bf{A}} &= \{1\}
    \end{align*}

    Sea una asignación $v$ tal que $v(x) = 0$, y consideramos la interpretación $\langle \bf{A}, v\rangle$.
    Veamos qué ocurre con cada una de las fórmulas:
    \begin{enumerate}
        \item \ul{Primera fórmula $\varphi_1$}:
        \begin{align*}
            I_{\bf{A}}^v(\varphi_1) = 1 &\Longleftrightarrow
            I_{\bf{A}}^v(q(x)) = 1 \text{ y } I_{\bf{A}}^v(\forall y\left( p(y) \rightarrow r(x,y) \right)) = 1\\
            & \Longleftrightarrow v(x)\in (q)^{\bf{A}} \text{ y } \forall a\in A, I_{\bf{A}}^{v(y|a)} (p(y) \rightarrow r(x,y)) = 1\\
            & \Longleftrightarrow v(x)\in (q)^{\bf{A}} \text{ y } \forall a\in A, I_{\bf{A}}^{v(y|a)}(p(y))I_{\bf{A}}^{v(y|a)}(r(x,y)) + I_{\bf{A}}^{v(y|a)}(p(y)) +1 = 1\\
            & \Longleftrightarrow v(x)\in (q)^{\bf{A}} \text{ y } \forall a\in A, I_{\bf{A}}^{v(y|a)}(p(y))I_{\bf{A}}^{v(y|a)}(r(x,y)) = I_{\bf{A}}^{v(y|a)}(p(y))
        \end{align*}
        Por tanto, en primer lugar, tenemos que $v(x)=0\in (q)^{\bf{A}}$. Veamos qué ocurre con cada $a\in A$:
        \begin{itemize}
            \item Si $a=0$, entonces:
            \begin{align*}
                I_{\bf{A}}^{v(y|0)}(p(y)) = 1 &\Longleftrightarrow v(y|0)(y) = 0 \in (p)^{\bf{A}} \qquad \checkmark \\
                I_{\bf{A}}^{v(y|0)}(r(x,y)) = 1 &\Longleftrightarrow \langle v(y|0)(x), v(y|0)(y)\rangle = \langle v(x), 0\rangle = \langle 0, 0\rangle \in (r)^{\bf{A}} \qquad \checkmark
            \end{align*}
            Por tanto, para $a=0$ tenemos $1\cdot 1 = 1$, lo cual es correcto.
    
            \item Si $a=1$, entonces:
            \begin{equation*}
                I_{\bf{A}}^{v(y|1)}(p(y)) = 1 \Longleftrightarrow v(y|1)(y) = 1 \in (p)^{\bf{A}} \qquad \times
            \end{equation*}
            Por tanto, para $a=1$ tenemos $0\cdot I_{\bf{A}}^{v(y|1)}(r(x,y)) = 0$, lo cual es correcto.
        \end{itemize}
        Por tanto, tenemos que $I_{\bf{A}}^v(\varphi_1) = 1$.
        \begin{comment}
            Además, para cada $a\in A$ tal que
            $I_{\bf{A}}^{v(y|a)}(p(y)) = 1$, tenemos que $I_{\bf{A}}^{v(y|a)}(r(x,y)) = 1$.
            Es decir, para cada $a\in A$ tal que $a\in (p)^{\bf{A}}$, tenemos que $\langle v(x), a\rangle\in (r)^{\bf{A}}$.
        \end{comment}

        \item \ul{Segunda fórmula $\varphi_2$}:
        \begin{align*}
            I_{\bf{A}}^v(\varphi_2) = 1 &\Longleftrightarrow
            I_{\bf{A}}^v(\forall x\left( q(x) \rightarrow \exists y\left( p(y) \land s(x,y) \right) \right)) = 1\\
            & \Longleftrightarrow \forall a\in A, I_{\bf{A}}^{v(x|a)}(q(x) \rightarrow \exists y\left( p(y) \land s(x,y) \right)) = 1\\
            & \Longleftrightarrow \forall a\in A, I_{\bf{A}}^{v(x|a)}(q(x))I_{\bf{A}}^{v(x|a)}(\exists y\left( p(y) \land s(x,y) \right))  + I_{\bf{A}}^{v(x|a)}(q(x)) + 1 = 1\\
            & \Longleftrightarrow \forall a\in A, I_{\bf{A}}^{v(x|a)}(q(x))I_{\bf{A}}^{v(x|a)}(\exists y\left( p(y) \land s(x,y) \right)) = I_{\bf{A}}^{v(x|a)}(q(x))
        \end{align*}
        Veamos qué ocurre con cada $a\in A$:
        \begin{itemize}
            \item Si $a=0$, entonces:
            \begin{align*}
                I_{\bf{A}}^{v(x|0)}(q(x)) = 1 &\Longleftrightarrow v(x|0)(x) = 0 \in (q)^{\bf{A}} \qquad \checkmark \\
                I_{\bf{A}}^{v(x|0)}(\exists y\left( p(y) \land s(x,y) \right)) = 1 &\Longleftrightarrow \exists b\in A, I_{\bf{A}}^{v(x|0,y|b)}(p(y) \land s(x,y)) = 1
            \end{align*}
            Para el segundo caso, para $b=0$ tenemos:
            \begin{align*}
                I_{\bf{A}}^{v(x|0,y|0)}(p(y) \land s(x,y)) = 1 &\Longleftrightarrow \left\{
                    \begin{array}{l}
                        v(x|0,y|0)(x) = 0 \in (p)^{\bf{A}}\\
                        \langle v(x|0,y|0)(x), v(x|0,y|0)(y)\rangle = \langle 0, 0\rangle \in (s)^{\bf{A}}
                    \end{array}
                \right. \checkmark
            \end{align*}
            Por tanto, para $a=0$ tenemos $1\cdot 1 = 1$, lo cual es correcto.
    
            \item Si $a=1$, entonces:
            \begin{equation*}
                I_{\bf{A}}^{v(x|1)}(q(x)) = 1 \Longleftrightarrow v(x|1)(x) = 1 \in (q)^{\bf{A}} \qquad \times
            \end{equation*}
            Por tanto, para $a=1$ tenemos $0\cdot I_{\bf{A}}^{v(x|1)}(\exists y\left( p(y) \land s(x,y) \right)) = 0$, lo cual es correcto.
        \end{itemize}
        Por tanto, tenemos que $I_{\bf{A}}^v(\varphi_2) = 1$.
        \begin{comment}
        Por tanto, para cada $a\in A$ tal que $I_{\bf{A}}^{v(x|a)}(q(x)) = 1$, por la segunda fórmula hemos deducido que $I_{\bf{A}}^{v(x|a)}(\exists y\left( p(y) \land s(x,y) \right)) = 1$.
        Es decir, para cada $a\in A$ tal que $a\in (q)^{\bf{A}}$, existe un $b\in A$ tal que $b\in (p)^{\bf{A}}$ y $\langle a, b\rangle\in (s)^{\bf{A}}$.
        \end{comment}


        \item \ul{Tercera fórmula $\varphi_3$}:
        \begin{align*}
            I_{\bf{A}}^v(\varphi_3) = 1 &\Longleftrightarrow
            I_{\bf{A}}^v(\forall x\left( \exists y\left(s(x,y) \land r(x,y) \right) \rightarrow t(x) \right)) = 1\\
            & \Longleftrightarrow \forall a\in A, I_{\bf{A}}^{v(x|a)}\left( \exists y\left(s(x,y) \land r(x,y) \right) \rightarrow t(x) \right) = 1\\
            & \Longleftrightarrow \forall a\in A, \exists b\in A, I_{\bf{A}}^{v(x|a,y|b)}\left((s(x,y) \land r(x,y)) \rightarrow t(x) \right) = 1\\
            & \Longleftrightarrow \forall a\in A, \exists b\in A,I_{\bf{A}}^{v(x|a,y|b)}\left(s(x,y)\right)I_{\bf{A}}^{v(x|a,y|b)}\left(r(x,y)\right)=\\&\hspace{3cm}=I_{\bf{A}}^{v(x|a,y|b)}\left(s(x,y)\right)I_{\bf{A}}^{v(x|a,y|b)}\left(r(x,y)\right)I_{\bf{A}}^{v(x|a,y|b)}\left(t(x)\right)
        \end{align*}
        Para cada $a\in A$, consideramos $b=1\in A$. Veamos qué ocurre:
        \begin{align*}
            I_{\bf{A}}^{v(x|a,y|1)}\left(s(x,y)\right) = 1 &\Longleftrightarrow \langle v(x|a,y|1)(x), v(x|a,y|1)(y)\rangle = \langle v(x|a)(x), 1\rangle = \langle a, 1\rangle \in (s)^{\bf{A}}
        \end{align*}
        Esto último no ocurreo, luego, independientemente de $a\in A$, tenemos:
        \begin{equation*}
            0 = 0 \cdot I_{\bf{A}}^{v(x|a,y|1)}\left(r(x,y)\right) = 0\cdot I_{\bf{A}}^{v(x|a,y|1)}\left(r(x,y)\right) \cdot I_{\bf{A}}^{v(x|a,y|1)}\left(t(x)\right) = 0
        \end{equation*}
        Esto es correcto, por lo que $I_{\bf{A}}^v(\varphi_3) = 1$.\\
        \begin{comment}
        Por tanto, para cada $a\in A$ existe un $b\in A$ tal que si $I_{\bf{A}}^{v(x|a,y|b)}\left(s(x,y)\right) = 1$ y $I_{\bf{A}}^{v(x|a,y|b)}\left(r(x,y)\right) = 1$, entonces $I_{\bf{A}}^{v(x|a,y|b)}\left(t(x)\right) = 1$.
        Es decir, para cada $a\in A$ existe un $b\in A$ tal que si $\langle a, b\rangle\in (s)^{\bf{A}}\cap (r)^{\bf{A}}$, entonces $a\in (t)^{\bf{A}}$.
        \end{comment}

        \item \ul{Cuarta fórmula $\varphi_4$}:
        \begin{align*}
            I_{\bf{A}}^v(\varphi_4) = 1 &\Longleftrightarrow
            I_{\bf{A}}^v(\exists x\left( t(x) \land q(x) \right)) = 1\\
            & \Longleftrightarrow \exists a\in A, I_{\bf{A}}^{v(x|a)}\left( t(x) \land q(x) \right) = 1\\
            & \Longleftrightarrow \exists a\in A, I_{\bf{A}}^{v(x|a)}\left( t(x) \right)I_{\bf{A}}^{v(x|a)}\left( q(x) \right) = 1\\
            & \Longleftrightarrow \exists a\in A, a\in (t)^{\bf{A}}\cap (q)^{\bf{A}}
        \end{align*}
    
        Tenemos que:
        \begin{align*}
            (t)^{\bf{A}}\cap (q)^{\bf{A}} &= \{1\}\cap \{0\} = \emptyset
            \Longrightarrow
            I_{\bf{A}}^v(\varphi_4) = 0
        \end{align*}
    \end{enumerate}
    
    En conclusión, tenemos que:   
    \begin{gather*}
        I_{\bf{A}}^v(\varphi_1) = I_{\bf{A}}^v(\varphi_2) = I_{\bf{A}}^v(\varphi_3) = 1\\
        I_{\bf{A}}^v(\varphi_4) = 0
    \end{gather*}

    Por tanto, concluimos que $\varphi_1, \varphi_2, \varphi_3 \not\models \varphi_4$.
\end{ejercicio}

\begin{ejercicio}
    ¿Es cierto que todo grafo (simple) con al menos dos nodos
    tiene al menos dos vértices con el mismo grado? Si es cierto,
    dé una demostración; en caso contrario, dé un contraejemplo.

    % // TODO: Dos nodos con el mismo grado.
    % https://www.quora.com/How-would-we-prove-that-every-graph-with-at-least-two-vertices-has-two-vertices-of-the-same-degree
\end{ejercicio}

    
\end{document}