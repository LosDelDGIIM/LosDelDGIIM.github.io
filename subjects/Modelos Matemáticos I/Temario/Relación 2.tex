\section{Ecuaciones en diferencias no lineales de orden 1}

\begin{ejercicio}
    Estudia el comportamiento local en torno a los puntos fijos de la ecuación en diferencias
    \begin{equation*}
        x_{n+1} = \frac{2x_n+4x_n^2-x_n^3}{5}
    \end{equation*}

    Su función asociada es:
    \Func{f}{\bb{R}}{\bb{R}}{x}{\dfrac{2x+4x^2-x^3}{5}}

    Calculamos los puntos fijos de dicha ecuación en diferencias:
    \begin{equation*}
        x=\dfrac{2x+4x^2-x^3}{5}
        \Longleftrightarrow
        5x = 2x+4x^2-x^3 
        \Longleftrightarrow
        x(x^2-4x+3)=0
        \Longleftrightarrow
        \left\{
        \begin{array}{l}
            x_1=0\\
            x_2=1\\
            x_3=3
        \end{array}
        \right.
    \end{equation*}

    Para aplicar el criterio de la 1ª derivada,
    calculamos dicha derivada:
    \begin{equation*}
        f'(x) = \dfrac{2+8x-3x^2}{5}
    \end{equation*}

    Tenemos entonces que:
    \begin{itemize}
        \item $|f'(0)|=\nicefrac{2}{5}<1$, por lo que $x_1=0$ es asintóticamente estable.
        \item $|f'(1)|=\nicefrac{7}{5}>1$, por lo que $x_2=1$ es inestable.
        \item $|f'(3)|=\nicefrac{1}{5}<1$, por lo que $x_3=3$ es asintóticamente estable.
    \end{itemize}
\end{ejercicio}

\begin{ejercicio}
    La \emph{ecuación logística de Pielou} es una ecuación en diferencias no lineal de la forma
    \begin{equation*}
        x_{n+1} = \frac{\alpha x_n}{1+\beta x_n},\qquad \alpha,\beta\in \bb{R}^+
    \end{equation*}
    Su uso es frecuente en dinámica de poblaciones. Dado que no tiene sentido hablar de poblaciones negativas, la
    ecuación se plantea en $[0, \infty[$.
    \begin{enumerate}
        \item Calcula los puntos de equilibrio de la ecuación y demuestra que, para las elecciones $\alpha = 2$ y $\beta = 1$, el punto de equilibrio positivo es asintóticamente estable.\\

        La función asociada es:
        \Func{f}{[0, \infty[}{[0, \infty[}{x}{\dfrac{\alpha x}{1+\beta x}}

        Los puntos fijos de dicha ecuación son:
        \begin{equation*}
            x = \dfrac{\alpha x}{1+\beta x}
            \Longleftrightarrow
            x + \beta x^2 = \alpha x
            \Longleftrightarrow
            x(1-\alpha+\beta x) = 0
            \Longleftrightarrow
            \left\{
            \begin{array}{l}
                x_1=0\\
                x_2=\dfrac{\alpha-1}{\beta}
            \end{array}
            \right.
        \end{equation*}

        Para estudiar la estabilidad usando el criterio de la primera derivada, derivamos:
        \begin{equation*}
            f'(x)=\frac{\alpha(1+\beta x)-\beta\cdot \alpha x}{(1+\beta x)^2}
            =\frac{\alpha}{(1+\beta x)^2}
        \end{equation*}

        Evaluando en $x_2$, que para los valores de $\alpha$ y $\beta$ dados es el punto de equilibrio positivo, se tiene:
        \begin{equation*}
            f'(x_2)
            = \frac{\alpha}{\alpha^2}
        \end{equation*}

        Para $\alpha=2$ tenemos que $f'(x_2)=\nicefrac{1}{2}<1$, por lo que $x_2$ es asintóticamente estable.
        
        \item Efectúa el cambio de variable $x_n=\nicefrac{1}{z_n}$ y calcula la expresión de todas las soluciones.\\

        Notemos que se supone que $x_0>0$, ya que si fuese igual a 0 tendríamos una solución constante y el ejercicio estaría resuelto. En caso contrario, el cambio de variable queda:
        \begin{equation*}
            \dfrac{1}{z_{n+1}} = \dfrac{\nicefrac{\alpha}{z_n}}{1+\nicefrac{\beta}{z_n}}
            = \dfrac{\alpha}{z_n+\beta}
            \Longrightarrow
            z_{n+1} = \dfrac{z_n+\beta}{\alpha}
        \end{equation*}

        Tenemos que se trata de una ecuación lineal de orden 1 no homogénea. Para resolverla, en primer lugar distinguimos en función del valor de $\alpha$:
        \begin{itemize}
            \item Si $\alpha\neq 1$:

            Hallamos en primer lugar su solución constante:
            \begin{equation*}
                z = \frac{z+\beta}{\alpha}
                \Longleftrightarrow
                \alpha z = z+\beta
                \Longleftrightarrow
                z = \frac{\beta}{\alpha-1}
            \end{equation*}
    
            Por tanto, aplicamos el cambio de variable $y_n = z_n - \dfrac{\beta}{\alpha-1}$, y tenemos que:
            \begin{align*}
                y_{n+1} &= z_{n+1} - \dfrac{\beta}{\alpha-1}
                = \dfrac{z_n+\beta}{\alpha} - \dfrac{\beta}{\alpha-1}
                = \dfrac{y_n + \dfrac{\beta}{\alpha-1}+\beta}{\alpha}
                =\\&= \dfrac{y_n + \dfrac{\beta}{\alpha-1}+\beta- \alpha\cdot \dfrac{\beta}{\alpha-1}}{\alpha}
                = \dfrac{y_n + (1-\alpha)\cdot \dfrac{\beta}{\alpha-1}+\beta}{\alpha}
                = \frac{y_n}{\alpha}
            \end{align*}
    
            Por tanto, tenemos que:
            \begin{equation*}
                z_{n} = y_n + \dfrac{\beta}{\alpha-1}
                = \frac{y_0}{\alpha^n}+ \dfrac{\beta}{\alpha-1}
                = \frac{z_0-\dfrac{\beta}{\alpha-1}}{\alpha^n}+ \dfrac{\beta}{\alpha-1}
                = \frac{(\alpha-1)z_0 + (\alpha^n-1)\beta}{\alpha^n(\alpha-1)}
            \end{equation*}
    
            Deshaciendo el cambio de variable, tenemos que:
            \begin{equation*}
                \frac{1}{x_n} = \frac{(\alpha-1)\cdot \frac{1}{x_0} + (\alpha^n-1)\beta}{\alpha^n(\alpha-1)}
                \Longrightarrow
                x_n = \frac{\alpha^n(\alpha-1)}{(\alpha-1)\cdot \frac{1}{x_0} + (\alpha^n-1)\beta}
            \end{equation*}

            \item Si $\alpha=1$:

            Tenemos que $z_{n+1}=z_n+\beta$, por lo que:
            \begin{equation*}
                z_n = z_0 + n\cdot \beta
            \end{equation*}

            Deshaciendo el cambio de variable, tenemos:
            \begin{equation*}
                \frac{1}{x_n} = \frac{1}{x_0}+n\cdot \beta =
                \frac{1+n\cdot x_0\cdot \beta}{x_0}
                \Longrightarrow
                x_n = \frac{x_0}{1+n\cdot x_0\cdot \beta}
            \end{equation*}
        \end{itemize}
        \item Determina el comportamiento asintótico de las soluciones de la ecuación.

        Distinguimos en función de los valores de $\alpha$:
        \begin{itemize}
            \item Si $\alpha=1$:
            \begin{equation*}
                \lim_{n\to \infty}x_n
                = \lim_{n\to \infty}\frac{x_0}{1+n\cdot x_0\cdot \beta}=0
            \end{equation*}

            \item Si $\alpha\in ~]0,1[$~:
            \begin{equation*}
                \lim_{n\to \infty}x_n
                = \lim_{n\to \infty}\frac{\alpha^n(\alpha-1)}{(\alpha-1)\cdot \frac{1}{x_0} + (\alpha^n-1)\beta}=0
            \end{equation*}
            \item Si $\alpha\in ~]1,+\infty[$~:
            \begin{equation*}
                \lim_{n\to \infty}x_n
                = \lim_{n\to \infty}\frac{\alpha^n(\alpha-1)}{(\alpha-1)\cdot \frac{1}{x_0} + (\alpha^n-1)\beta}=\frac{\alpha-1}{\beta}
            \end{equation*}
        \end{itemize}
    \end{enumerate}
\end{ejercicio}


\begin{ejercicio}\label{ej:2.3}
    Una población de ganado se rige por el modelo discreto dado por $p_{n+1}=10p_n\cdot e^{-p_n},~n\in \bb{N}$. Calcula sus puntos de equilibrio y comprueba que son todos inestables.\\

    Su función asociada es:
    \Func{f}{\bb{R}^+_0}{\bb{R}^+_0}{x}{10x\cdot e^{-x}}

    Sus puntos fijos son las soluciones de:
    \begin{equation*}
        x = 10xe^{-x}
    \end{equation*}

    Por tanto, una primera solución constante es $x_1=0$. Para la otra, resolvemos la siguiente ecuación:
    \begin{equation*}
        1 = 10e^{-x_2} \Longleftrightarrow
        0=\ln 1 = \ln(10e^{-x_2}) = \ln(10)-x_2 \Longleftrightarrow
        x_2 = \ln 10
    \end{equation*}

    Para estudiar la estabilidad, aplicamos el criterio de la primera derivada:
    \begin{equation*}
        f'(x)=10e^{-x}-10xe^{-x} = 10e^{-x}(1-x)
    \end{equation*}
    \begin{itemize}
        \item $f'(x_1)=10>1$, por lo que $x_1$ es inestable.
        \item $f'(x_2)=10\cdot \frac{1}{10}(1-\ln 10) = 1-\ln 10<-1 \Longleftrightarrow 2<\ln 10 \Longleftrightarrow e^2 < 10$. Por tato, como es cierto, tenemos que $x_2$ es inestable.
    \end{itemize}
\end{ejercicio}


\begin{ejercicio}
    En relación con el modelo del ejercicio anterior (Ejercicio \ref{ej:2.3}), se propone vender una fracción $\alpha~(0 < \alpha < 1)$ de la población en cada periodo de tiempo, lo que da lugar al siguiente otro modelo:
    \begin{equation*}
        p_{n+1} = 10(1-\alpha)p_n\cdot e^{-(1-\alpha)p_n}
    \end{equation*}
    \begin{enumerate}
        \item Encuentra el intervalo abierto (de amplitud máxima) al que debe pertenecer $\alpha$ para que esté asegurada la estabilidad asintótica del punto de equilibrio positivo.

        La función asociada es:
        \Func{f}{\bb{R}^+_0}{\bb{R}^+_0}{x}{10(1-\alpha)x\cdot e^{-(1-\alpha)x}}

        Calculamos en primer lugar los puntos fijos. Además de $x_1=0$, la otra solución constante es:
        \begin{equation*}
            1 = 10(1-\alpha)e^{-(1-\alpha)x}
            \Longrightarrow
            0 = \ln[10(1-\alpha)] -(1-\alpha)x
            \Longrightarrow
            x_2 = \dfrac{\ln[10(1-\alpha)]}{1-\alpha}
        \end{equation*}

        Para estudiar la estabilidad, aplicamos el criterio de la primera derivada:
        \begin{equation*}
            f'(x)=10(1-\alpha)e^{-(1-\alpha)x} -10(1-\alpha)^2xe^{-(1-\alpha)x}
            = 10(1-\alpha)e^{-(1-\alpha)x}[1-x(1-\alpha)]
        \end{equation*}

        Evaluando en $x_2$, tenemos:
        \begin{equation*}
            f'(x_2) = 10(1-\alpha)\cdot \frac{1}{10(1-\alpha)}[1-\ln[10(1-\alpha)]]
            = \ln \left(\frac{e}{10(1-\alpha)}\right)
        \end{equation*}

        Como necesitamos que $|f'(x_2)|<1$, tenemos que:
        \begin{align*}
            \ln \left(\frac{e}{10(1-\alpha)}\right)<1 &\Longleftrightarrow
            \frac{e}{10(1-\alpha)}<e 
            \Longleftrightarrow
            \frac{1}{10(1-\alpha)}<1
            \Longleftrightarrow \\&\hspace{2cm}\Longleftrightarrow
            \frac{1}{10}<1-\alpha
            \Longleftrightarrow
            \alpha < \frac{9}{10}\\
            \ln \left(\frac{e}{10(1-\alpha)}\right)>-1 &\Longleftrightarrow
            \frac{e}{10(1-\alpha)}>\frac{1}{e}
            \Longleftrightarrow
            \frac{e^2}{10(1-\alpha)}>1
            \Longleftrightarrow \\&\hspace{2cm}\Longleftrightarrow
            \frac{e^2}{10}>1-\alpha
            \Longleftrightarrow
            \frac{10-e^2}{10}<\alpha
        \end{align*}

        Por tanto, se asegura la estabilidad asintótica de $x_2$ si:
        \begin{equation*}
            \alpha\in \left]\frac{10-e^2}{10}, \frac{9}{10}\right[
        \end{equation*}
        
        \item Calcula el valor de $\alpha$ para el que la población (no nula) en equilibrio alcanza su valor máximo.\\

        La población no nula en equilibrio es $x_2 = \dfrac{\ln[10(1-\alpha)]}{1-\alpha}$. Para que alcance su valor máximo, es necesario que coincida con el máximo de $f$, calculémoslo:
        \begin{equation*}
            f'(x) = 0 \Longleftrightarrow 1-x(1-\alpha) = 0 \Longleftrightarrow x = \dfrac{1}{1-\alpha}
        \end{equation*}

        Comprobemos que efectivamente dicho valor es un máximo relativo:
        \begin{itemize}
            \item Si $x<\dfrac{1}{1-\alpha}$, entonces $f'(x)>0$, por lo que $f$ es creciente.
            \item Si $x>\dfrac{1}{1-\alpha}$, entonces $f'(x)<0$, por lo que $f$ es decreciente.
        \end{itemize}

        Por tanto, dicho valor efectivamente es un máximo relativo. Imponemos que dicho máximo sea igual al punto de equilibrio:
        \begin{equation*}
            \dfrac{\ln[10(1-\alpha)]}{1-\alpha} = \dfrac{1}{1-\alpha}
            \Longleftrightarrow
            \ln[10(1-\alpha)] = 1 \Longleftrightarrow
            1-\alpha = \frac{e}{10}
            \Longleftrightarrow
            \alpha=\frac{10-e}{10}
        \end{equation*}
        Por tanto, el valor de $\alpha$ buscado es $\alpha=\frac{10-e}{10}$.
    \end{enumerate}
\end{ejercicio}


\begin{ejercicio}
    Sea $g\in C^2(\bb{R})$ una función que satisface $g'(x)\neq 0$ para todo $x \in \bb{R}$. El método de Newton para resolver la ecuación $g(x) = 0$ se describe como
    \begin{equation*}
        x_{n+1} = x_n - \frac{g(x_n)}{g'(x_n)}.
    \end{equation*}
    Si $\alpha$ es una raíz de $g(x) = 0$, demuestra que el método es convergente para cualquier dato inicial suficientemente próximo a la raíz.\\

    La función asociada a dicha Ley de Recurrencia es:
    \Func{f}{\bb{R}}{\bb{R}}{x}{x-\dfrac{g(x)}{g'(x)}}

    En primer lugar, tenemos que:
    \begin{equation*}
        f(\alpha)=\alpha-0 = \alpha
    \end{equation*}

    Por tanto, $\alpha$ es un punto fijo de $f$. Como $g\in C^2(\bb{R})$, tenemos que $g'\in C^1(\bb{R})$, por lo que $f\in C^1(\bb{R})$. Tenemos que:
    \begin{equation*}
        f'(x) = 1 - \frac{(g'(x))^2-g(x)g''(x)}{(g'(x))^2}
        = 1 -1 + \frac{g(x)g''(x)}{(g'(x))^2}
        = \frac{g(x)g''(x)}{(g'(x))^2}
    \end{equation*}

    Por tanto, tenemos que $|f'(\alpha)|=0<1$, y por el Criterio de la Primera Derivada $\alpha$ es asintóticamente estable localmente para $f$. En concreto, es un atractor local, por lo que:
    \begin{equation*}
        \exists \delta\in \bb{R}^+\mid |x_0-\alpha|<\delta\Longrightarrow
        \lim_{n\to \infty} x_n = \alpha
    \end{equation*}
    Queda por tanto demostrado lo pedido.
\end{ejercicio}



\begin{ejercicio}
    Estudia la estabilidad de los puntos de equilibrio de la ecuación en diferencias $x_{n+1} = f(x_n)$, donde:
    \begin{enumerate}
        \item $f(x)=1-2x+3x^2-x^3$.

        Buscamos en primer lugar los puntos fijos de dicha función resolviendo la siguiente ecuación:
        \begin{equation*}
            1-2x+3x^2-x^3 = x \Longleftrightarrow 1-3x+3x^2-x^3 = 0
        \end{equation*}

        Aplicamos el Método de Ruffini:
        \begin{figure}[H]
            \centering
            \polyhornerscheme[x=1]{1-3x+3x^2-x^3}
            \caption{División mediante Ruffini donde se ve que $x=1$ es una solución.}
        \end{figure}

        De Ruffini, vemos que:
        \begin{equation*}
            f(x)=x \Longleftrightarrow 1-3x+3x^2-x^3=-(x-1)(x^2-2x+1) = (x-1)^3
        \end{equation*}

        Por tanto, tenemos que el único punto fijo es $x_c\equiv 1$. Para estudiar la estabilidad, como $f\in  C^\infty(\bb{R})$, tenemos que:
        \begin{equation*}
            f'(x)=-2+6x-3x^2 \qquad f'(1)=1
        \end{equation*}

        Como vemos que el Criterio de la Primera Derivada no aporta información, buscamos aplicar el Criterio de la Segunda Derivada:
        \begin{equation*}
            f''(x)=6-6x \qquad f''(1)=0
        \end{equation*}

        Este tampoco nos aporta información, por lo que recurrimos al Criterio de la Tercera Derivada:
        \begin{equation*}
            f'''(x)=-6<0
        \end{equation*}

        Por tanto, tenemos que $x=1$ es un punto de inflexión, en el que pasa de convexa a cóncava. Por el Criterio de la Tercera Derivada, tenemos que $x_c\equiv 1$ es asintóticamente estable localmente.
    
        \item $f(x)=x^2-x$.

        Buscamos en primer lugar los puntos fijos de dicha función resolviendo la siguiente ecuación:
        \begin{equation*}
            x^2-x = x \Longleftrightarrow x^2-2x = x(x-2) = 0 \Longrightarrow \left\{\begin{array}{c}
                x_{c_1} \equiv 0 \\
                \lor\\
                x_{c_2} \equiv 2 \\
            \end{array}\right.
        \end{equation*}
        
        Para estudiar la estabilidad, como $f\in  C^\infty(\bb{R})$, tenemos que:
        \begin{equation*}
            f'(x)=2x-1 \qquad f'(0)=-1,~f(2)=3
        \end{equation*}

        Por el Criterio de la Primera Derivada, deducimos que $x_{c_2}\equiv 2$ es inestable. Respecto al otro punto fijo, como es una parábola deducimos por la observación de la página \pageref{obs:polinomio2grado} tenemos que es asintóticamente estable localmente.
        
        \item $f(x)=\left\{\begin{array}{lll}
            \nicefrac{3}{2} - \nicefrac{x}{2} & \text{si} & x\in~]-\infty, 1[ \\
            x^2 & \text{si} & x\in [1,\infty[
        \end{array}\right.$

        Buscamos en primer lugar los puntos fijos de dicha función resolviendo las siguientes ecuaciones:
        \begin{align*}
            \frac{3}{2}-\frac{x}{2} &= x \Longleftrightarrow 3-x = 2x \Longleftrightarrow x=1\\
            x^2 &= x \Longleftrightarrow x=0,~1
        \end{align*}

        Considerando los intervalos de definición de cada parte de $f$, tenemos que tan solo hay un punto fijo, $x_c\equiv 1$. Estudiemos ahora su estabilidad. Tenemos que $f$ es derivable en $\bb{R}\setminus \{1\}$ con:
        \begin{equation*}
            f'(x)=\left\{\begin{array}{lll}
            - \nicefrac{1}{2} & \text{si} & x\in~]-\infty, 1[ \\
            2x & \text{si} & x\in ]1,\infty[
        \end{array}\right.
        \end{equation*}

        Veamos si es derivable en $x=1$. Tenemos:
        \begin{equation*}
            \lim_{x\to 1^-}f'(x) = \lim_{x\to 1^-}-\frac{1}{2} = -\frac{1}{2} \neq 2 = \lim_{x\to 1^+}2x = \lim_{x\to 1^+} f'(x)
        \end{equation*}

        Por tanto, tenemos que $f$ no es derivable en $x=1$, por lo que no podemos aplicar ninguno de los criterios ya conocidos. Veamos ahora qué ocurre en cada parte:
        \begin{itemize}
            \item \ul{Si $x_0>1$}:

            Demostramos por inducción que $x_n = x_0^{(2^n)}$ para todo $n\in \bb{N}$. Entonces, tenemos que:
            \begin{equation*}
                \lim_{n\to \infty} x_n = 
                \lim_{n\to \infty} x_0^{(2^n)} = \infty
            \end{equation*}

            Por tanto, tenemos que no es estable, ya que para valores cercanos a $1$, la solución diverge. Por tanto, es inestable por arriba.

            \item \ul{Si $x_0<1$}:

            Tenemos que:
            $$x_1=\frac{3}{2}-\frac{x_0}{2} = \frac{3-x_0}{2}>1 \Longleftrightarrow 3-x_0>2\Longleftrightarrow 1>x_0$$

            Por tanto, como $x_1>1$, aplicando lo anterior tenemos que:
            \begin{align*}
                x_n &= \left(\frac{3-x_0}{2}\right)^{(2^{n-1})}
                \qquad \forall n\in \bb{N}\setminus \{0\}
            \end{align*}

            Por tanto, con $n\to \infty$ tenemos que:
            \begin{equation*}
                \lim_{n\to \infty} x_n = 
                \lim_{n\to \infty} \left(\frac{3-x_0}{2}\right)^{(2^{n-1})} = \infty
            \end{equation*}
             Por tanto, es inestable por abajo.
        \end{itemize}

        Por tanto, deducimos que $x_c\equiv 1$ es inestable.

        
        \item $f(x)=\left\{\begin{array}{lll}
            0.25x+1.5 & \text{si} & x\leq 2 \\
            \sqrt{2x} & \text{si} & x>2
        \end{array}\right.$

        Buscamos en primer lugar los puntos fijos de dicha función resolviendo las siguientes ecuaciones:
        \begin{align*}
            0.25x+1.5 &= x \Longleftrightarrow 1.5=0.75x \Longleftrightarrow x=2\\
            \sqrt{2x} &= x \Longleftrightarrow 2x = x^2 \Longleftrightarrow x=0,2
        \end{align*}

        Considerando los intervalos de definición de cada parte de $f$, tenemos que tan solo hay un punto fijo, $x_c\equiv 2$. Estudiemos ahora su estabilidad. Tenemos que $f$ es derivable en $\bb{R}\setminus \{2\}$ con:
        \begin{equation*}
            f'(x)=\left\{\begin{array}{lll}
            0.25 & \text{si} & x<2 \\
            \frac{1}{2\sqrt{2x}}\cdot 2 = \frac{1}{\sqrt{2x}} & \text{si} & x>2
        \end{array}\right.
        \end{equation*}

        Veamos si es derivable en $x=2$. Tenemos:
        \begin{equation*}
            \lim_{x\to 2^-}f'(x) = \lim_{x\to 2^-}0.25 = 0.25 \neq \frac{1}{2} = \lim_{x\to 2^+}\frac{1}{\sqrt{2x}} = \lim_{x\to 2^+} f'(x)
        \end{equation*}

        Por tanto, tenemos que $f$ no es derivable en $x=2$, por lo que no podemos aplicar ninguno de los criterios ya conocidos.
       \begin{itemize}
           \item \ul{Si $x_0<2$}:
           
           Veamos en primer lugar que la sucesión solución está mayorada por 2:
           \begin{itemize}
               \item Para $n=0$, vemos que $x_0<2$.
               \item Supuesto cierto para $n$, demostramos para $n+1$:
               \begin{equation*}
                   x_n < 2 \Longrightarrow x_{n+1} = 0.25x_n +1.5 < 0.25\cdot 2 +1.5 = 2
               \end{equation*}
               Por tanto, para $n+1$ se tiene.
           \end{itemize}

           Veamos ahora que la sucesión solución es creciente:
           \begin{equation*}
                x_{n+1} = f(x_n) = 0.25\cdot x_n + 1.5 > x_n\Longleftrightarrow
                1.5 > 0.75x_n \Longleftrightarrow 
                2 > x_n
            \end{equation*}
            
           Por tanto, como $\{x_n\}$ es creciente y mayorada, tenemos que $\{x_n\}\to 2$. Por tanto, $x_c\equiv 2$ es un atractor local por debajo. Además, fijado $\varepsilon\in \bb{R}^+$, tomando $\delta=\veps$, tenemos que:
           \begin{equation*}
                | x_0 - 2 | = 2 - x_0 < \delta \Longrightarrow | x_n - 2 | = 2 - x_n < 2 - x_0 < \delta = \varepsilon
            \end{equation*}
            donde se ha empleado que $x_0<x_n$ para todo $n\in \bb{N}$. Por tanto, $x_c\equiv 2$ es estable por abajo.



            \item \ul{Si $x_0>2$}:
           
           Veamos en primer lugar que la sucesión solución está minorada por 2:
           \begin{itemize}
               \item Para $n=0$, vemos que $x_0>2$.
               \item Supuesto cierto para $n$, demostramos para $n+1$:
               \begin{equation*}
                   x_n > 2 \Longrightarrow x_{n+1} = \sqrt{2x_n} > \sqrt{2\cdot 2} = 2
               \end{equation*}
               Por tanto, para $n+1$ se tiene.
           \end{itemize}

           Veamos ahora que la sucesión solución es decreciente:
           \begin{equation*}
                x_{n+1} = f(x_n) = \sqrt{2\cdot x_n} < x_n\Longleftrightarrow
                2x_n<x_n^2 \Longleftrightarrow 
                2 < x_n
            \end{equation*}
            
           Por tanto, como $\{x_n\}$ es decreciente y minorada, tenemos que $\{x_n\}\to 2$. Por tanto, $x_c\equiv 2$ es un atractor local por encima. Además, fijado $\varepsilon\in \bb{R}^+$, tomando $\delta=\veps$, tenemos que:
           \begin{equation*}
                | x_0 - 2 | = x_0-2 < \delta \Longrightarrow | x_n - 2 | = x_n-2 < x_0-2 < \delta = \varepsilon
            \end{equation*}
            donde se ha empleado que $x_0>x_n$ para todo $n\in \bb{N}$. Por tanto, $x_c\equiv 2$ es estable por encima.
       \end{itemize} 
       Uniendo ambos resultados, tenemos que $x_c\equiv 2$ es estable y es un atractor local, por lo que es asintóticamente estable localmente.
        
        \item $f(x)=\left\{\begin{array}{lll}
            -1 & \text{si} & x<-1 \\
            x & \text{si} & -1\leq x\leq 1 \\
            1 & \text{si} & x>1 \\
        \end{array}\right.$

        En este caso, tenemos que $[-1,1]$ son todos ellos puntos fijos. No obstante, tenemos que $f$ no es derivable en $\pm 1$, por lo que no podemos aplicar los criterios ya conocidos.
        \begin{itemize}
            \item  Para cualquier $x^\ast \in ]-1,1[$, tenemos que $x$ no es un atractor local, ya que para $x_0\neq x^\ast$, tenemos que:
            \begin{equation*}
                \lim_{n\to \infty} x_n = x_0\neq x^\ast
            \end{equation*}
            No obstante, tomando $\delta=\veps$ obtenemos que sí es estable.
    
            \item Para $x^\ast=1$, tenemos que es estable en ambos lados y atractor local tan solo por arriba.

            \item Para $x^\ast=-1$, tenemos que es estable en ambos lados y atractor tan solo por debajo.
        \end{itemize}
    \end{enumerate}
\end{ejercicio}



\begin{ejercicio}
    Demuestra que $\left\{\nicefrac{2}{9}, \nicefrac{4}{9}, \nicefrac{8}{9}\right\}$ es un $3-$ciclo inestable para la función ``tienda''\emph{(tent map)} $T$ definida por
    \begin{equation*}
        T(x)=\left\{\begin{array}{lll}
            2x & \text{si} & 0\leq x\leq \nicefrac{1}{2} \\
            2(1-x) & \text{si} & \nicefrac{1}{2} \leq x \leq 1
        \end{array}\right.
    \end{equation*}
\end{ejercicio}

\begin{ejercicio} \label{ej:2.8}
    Sean $f: \bb{R} \to \bb{R}$ continua y $\{s_0, s_1\}$ un $2-$ciclo de $x_{n+1} = f(x_n)$. Demuestra que entre $s_0$ y $s_1$ hay un punto de equilibrio.\\

    Supongamos sin pérdida generalidad $s_0<s_1$. Buscamos demostrar que $\exists c\in~]s_0,s_1[$ tal que $f(c)=c$. Definiendo $g=f-Id$, tenemos que es continua por ser diferencia de continuas. Además, tenemos que:
    \begin{align*}
        g(s_0)&=f(s_0)-s_0 = s_1-s_0 \\
        g(s_1)&= f(s_1)-s_1 = s_0-s_1 = -(s_1-s_0)
    \end{align*}
    donde he hecho uso que, por ser un $2-$ciclo, tenemos que $f(s_0)=s_1$ y $f(s_1)=s_0$. Por tanto, tenemos que $g(s_0)g(s_1)=-(s_1-s_0)^2<0$, por lo que por el Teorema de los Ceros de Bolzano, $\exists c\in~]s_0,s_1[$ tal que $g(c)=0=f(c)-c$, por lo que $f(c)=c$.
\end{ejercicio}

\begin{ejercicio}
    Se considera la ecuación en diferencias $x_{n+1} = f(x_n)$ con función asociada $f(x) = \nicefrac{1}{2}\cdot x(1 - 3x^2)$. Estudia la estabilidad del ciclo $\{-1, 1\}$. Estudia también la estabilidad del punto de equilibrio deducido en el ejercicio anterior (Ejercicio \ref{ej:2.8}).\\

    Comprobemos en primer lugar que se trata de un $2-$ciclo:
    \begin{equation*}
        f(1)=\frac{1}{2}\cdot (1-3) = -1 \qquad f(-1)=-\frac{1}{2}(1-3) = 1
    \end{equation*}
    
    Por tanto, efectivamente se trata de un $2-$ciclo. Tenemos que $f\in C^\infty(\bb{R})$, por lo que podemos aplicar el Criterio de la Primera Derivada. Tenemos que:
    \begin{equation*}
        f'(x)=\frac{1}{2}-\frac{1}{2}\cdot 9x^2 = \frac{1-9x^2}{2}
    \end{equation*}

    Tenemos por tanto que $f'(1)=f(-1)=-\nicefrac{8}{2}=-4<-1$. Como $|f'(1)f'(-1)|=16>1$, tenemos que dicho ciclo es inestable. Además, en el ejercicio anterior hemos visto que tiene un punto fijo $c\in ~]-1,1[$. Calculémoslo explícitamente:
    \begin{equation*}
        f(x)=x \Longleftrightarrow \frac{1}{2}x(1-3x^2)=x \Longleftrightarrow
        \left\{\begin{array}{c}
            x = 0\\
            1-3x^2 = 2 \Longleftrightarrow 3x^2 = -1
        \end{array}\right.
    \end{equation*}
    Por tanto, tenemos que el punto fijo es $x_c\equiv 0$. Tenemos que $|f'(0)|=\frac{1}{2}<1$, y por el Criterio de la Primera Derivada tenemos que es asintóticamente estable localmente.

    \begin{comment}
        Tenemos que:
        \begin{align*}
            0\leq \frac{1-9x^2}{2}<1 &\Longleftrightarrow 0\leq 1-9x^2<2 \Longleftrightarrow -1\leq -9x^2<1
            \Longleftrightarrow \\&\hspace{1cm}\Longleftrightarrow \left\{\begin{array}{c}
                x^2 \leq \nicefrac{1}{9} \Longleftrightarrow |x|\leq \nicefrac{1}{3}\\
                \land \\
                -9x^2<1 \Longleftrightarrow x\in \bb{R}
            \end{array}\right.\\ 
            -1< \frac{1-9x^2}{2}\leq 0 &\Longleftrightarrow -2< 1-9x^2\leq 0 \Longleftrightarrow
            -3 < -9x^2\leq -1 \Longleftrightarrow 1\leq 9x^2 < 3
            \Longleftrightarrow \\&\hspace{1cm}\Longleftrightarrow
            \left\{\begin{array}{c}
                x^2 \geq \nicefrac{1}{9} \Longleftrightarrow |x|\geq \nicefrac{1}{3} \\
                \land \\
                x^2 < \frac{1}{3} \Longleftrightarrow |c|<\frac{\sqrt{3}}{3}
            \end{array}\right.
        \end{align*}
    
        Por tanto, tenemos que:
        \begin{align*}
            0\leq f'(x) <1 &\Longleftrightarrow x\in \left[\nicefrac{-1}{3}, \nicefrac{1}{3}\right] \\
            -1 < f'(x)\leq 0 &\Longleftrightarrow x \in \left]\nicefrac{-\sqrt{3}}{3}, \nicefrac{\sqrt{3}}{3}\right[ \setminus \left[\nicefrac{-1}{3}, \nicefrac{1}{3}\right]
        \end{align*}
    
        En cualquier caso, se tiene que $|f'(x)|<1$ si y solo si $x\in \left[\nicefrac{-1}{3}, \nicefrac{1}{3}\right]$. Veamos si el punto fijo $c$ pertenece a dicho intervalo:
        \begin{equation*}
            \left.\begin{array}{rl}
                f\left(\frac{1}{3}\right) &= \frac{1}{2}\cdot \frac{1}{3} \left(1-\frac{1}{3}\right)>0 \\
                f\left(-\frac{1}{3}\right) &= \frac{1}{2}\cdot \frac{-1}{3} \left(1-\frac{1}{3}\right)<0
            \end{array}\right\} \Longrightarrow \exists c\in \left]\nicefrac{-1}{3}, \nicefrac{1}{3}\right[
        \end{equation*}
    \end{comment}
\end{ejercicio}

\begin{ejercicio}
    Se considera la ecuación en diferencias
    \begin{equation*}
        x_{n+1} = x_ne^{r(1-x_n)},\qquad r\in \bb{R}
    \end{equation*}
    que describe la evolución de una población que se comporta como una exponencial cuando el tamaño de la población es bajo y tiene tendencia a disminuir cuando el tamaño es elevado. La cantidad
    \begin{equation*}
        \lm = e^{r(1-x_n)}
    \end{equation*}
    es la tasa de crecimiento de la población.
    \begin{enumerate}
        \item Calcula los puntos de equilibrio de la ecuación.

        Su función asociada es:
        \Func{f}{\bb{R}}{\bb{R}}{x}{xe^{r(1-x)}}

        Sus puntos fijos son, suponiendo $r\neq 0$, son:
        \begin{equation*}
            f(x)=x \Longleftrightarrow x=xe^{r(1-x)} \Longleftrightarrow
            \left\{\begin{array}{l}
                x= 0 \\
                \quad \lor\\
                1 = e^{r(1-x)} \Longleftrightarrow 0 = r(1-x) \Longleftrightarrow x=1
            \end{array}\right.
        \end{equation*}

        En el caso de $r=0$, se tiene que $f(x)=xe^0=x$ para todo $x\in \bb{R}$, por lo que todos los puntos son puntos fijos.
        
        \item Determina las condiciones bajo las que dichos puntos de equilibrio son asintóticamente estables para $r \neq 0, 2$.\\

        Tenemos que $f\in C^\infty(\bb{R})$, por lo que calculamos la primera derivada:
        \begin{equation*}
            f'(x)=e^{r(1-x)}(1-r\cdot x) =e^{r(1-x)}(1-rx)
        \end{equation*}

        Estudiamos ahora cada uno de los puntos:
        \begin{itemize}
            \item \ul{Para $x=0$}:

            Tenemos $f'(0) = e^r$. Tenemos que $e^x$ es una función estrictamente creciente y positiva. Por tanto, para $r>0$ $(f'(0)>1)$ se tiene que es inestable; mientras que para $r<0$ $(f'(0)<1)$ se tiene que es asintóticamente estable localmente.

            \item \ul{Para $x=1$}:

            Tenemos $f'(1) = e^0(1-r) = 1-r$. Tenemos que, para $r\in ]0,2[$ $(|f'(1)|<1)$ se tiene que es asintóticamente estable localmente; mientras que para $r>2$ y $r<0$ $(|f'(1)|>1)$ se tiene que es inestable.
        \end{itemize}

        Los resultados se resumen en la siguiente tabla, donde \emph{a.e.l.} representa \emph{asintóticamente estable localmente}.
        \begin{table}[H]
            \centering
            \begin{tabular}{l|l}
                $0< r$ & $0$ es a.e.l. y $1$ es inestable.\\ \hline
                $r=0$ & No lo sabemos. \\ \hline
                $0<r<2$ & $0$ es inestable y $1$ es a.e.l.\\ \hline
                $r=2$ & $0$ es inestable, y para el $1$ no lo sabemos. \\ \hline
                $2<r$ & $0$ y $1$ son inestables.\\
            \end{tabular}
        \end{table}
        
        \item Estudia el caso $r = 2$.\\

        Como hemos explicado antes, el $0$ es inestable. Veamos el caso de $x_c\equiv 1$. Tenemos que $f'(1)=1-r=-1$. Calculamos entonces la derivada segunda y tercera:
        \begin{align*}
            f''(x)&= -re^{r(1-x)}(1-rx +1) = -re^{r(1-x)}(2-rx)\\
            f'''(x)&= r^2e^{r(1-x)}(2-rx+1) = r^2e^{r(1-x)}(3-rx)
        \end{align*}

        Por tanto, tenemos que $f''(1)=-r(2-r) = 0$, $f'''(1)=r^2(3-r) = 4$. Usando el Teorema correspondiente, tenemos que:
        \begin{equation*}
            2f'''(1) + 3(f''(1))^2 = 2\cdot 4 > 0
        \end{equation*}
        Por tanto, tenemos que $x_c\equiv 1$ es asintóticamente estable localmente.
        
        \item Estudia el caso $r = 0$.

        En este caso, el modelo queda $x_{n+1}=x_n$, por lo que todos los valores iniciales son soluciones constantes. No es un atractor local, ya que $\lim\limits_{n\to \infty}x_n = x_0$ para cualquier valor de $x_0\in \bb{R}$. No obstante, tomando $\delta=\varepsilon$, tenemos que sí es estable.
    \end{enumerate}
\end{ejercicio}

\begin{ejercicio}
    Se considera la ecuación en diferencias $x_{n+1} = x_n + \alpha f(x_n)$, donde $f \in C^2(\bb{R})$ es una función que verifica las siguientes propiedades:
    \begin{itemize}
        \item $f(x)$ se anula solo en $x = -1$.
        \item $f'(x)$ es estrictamente decreciente con $f'(-1) = 0$.
    \end{itemize}
    Demuestra que $x^\ast = -1$ es un punto de equilibrio inestable siempre que $\alpha\neq 0$.\\

    Sea la función asociada  a dicha euación en diferencias la siguiente:
    \Func{g}{\bb{R}}{\bb{R}}{x}{x+\alpha f(x)}

    Veamos que $x^\ast=-1$ es un punto fijo:
    \begin{equation*}
        g(-1) = -1 + \alpha f(-1) = -1 + \alpha\cdot 0 = -1
    \end{equation*}

    Por tanto, $x^\ast$ es un punto fijo. Como $f\in C^2(\bb{R})$, tenemos que $g\in C^2(\bb{R})$. Calculemos $g'(x^\ast)$:
    \begin{equation*}
        g'(x) = 1+\alpha f'(x) \qquad g'(-1) = 1 + \alpha f'(-1) = 1
    \end{equation*}

    Por tanto, el Criterio de la Primera Derivada no aporta información. Veamos ahora qué ocurre en $x^\ast=-1$:
    \begin{itemize}
        \item Para $x<-1$, como $f'$ es estrictamente decreciente tenemos que $f'(x)>f'(-1)=0$. Por tanto, $f$ es estrictamente creciente.
        \item Para $x>-1$, como $f'$ es estrictamente decreciente tenemos que $f'(x)<f'(-1)=0$. Por tanto, $f$ es estrictamente decreciente.
    \end{itemize}

    Por tanto, podemos afirmar que $f$ tiene un máximo relativo en $x^\ast$ y que cambia de creciente a decreciente, por lo que es cóncava. \emph{Supongamos} que $f''(x)<0$ para todo $x\in \bb{R}$. Tenemos que por tanto que:
    \begin{equation*}
        g''(x)=\alpha f''(x)
    \end{equation*}

    Por tanto, como $\alpha,f''(-1)\neq 0$, tenemos que $g''(-1)\neq 0$. Por tanto, por el Criterio de la Segunda Derivada, tenemos que $x^\ast=-1$ será inestable bien por arriba o bien por abajo, pero en cualquier caso será inestable.

    \begin{observacion}
        Notemos que hemos supuesto $f''(x)<0$. Que una función sea cóncava no implica necesariamente que $f''(x)<0$ para todo $x\in \bb{R}$, y ejemplo de esto es $f''(x)=-\dfrac{(x-1)^4}{4}$.
    \end{observacion}
\end{ejercicio}

\begin{ejercicio}\label{ej:2.13}
    En cierto mercado las funciones de oferta y demanda vienen dadas por
    \begin{equation*}
        O(p)=1+p^2,\qquad D(p)=c-dp,\hspace{2cm}c\in~]1,+\infty[,~d\in \bb{R}^+
    \end{equation*}
    \begin{enumerate}
        \item Calcula el punto de equilibrio económicamente factible.
        \begin{equation*}
            O(p) = D(p) \Longleftrightarrow
            1+p^2 = c-dp \Longleftrightarrow
            p^2+dp +1-c = 0\Longleftrightarrow
            p = \frac{-d\pm \sqrt{d^2+4(c-1)}}{2}
        \end{equation*}

        En el caso de la solución que lleva un $-1$ como coeficiente de la raíz, podemos confirmar que el valor es negativo, por lo que el punto de equilibrio económicamente estable es:
        \begin{equation*}
            p^\ast = \frac{-d + \sqrt{d^2+4(c-1)}}{2}
        \end{equation*}
        
        \item Deduce las condiciones sobre $c$ y $d$ que aseguran la estabilidad asintótica de $p^\ast$. ¿Qué ocurre si $d = 2$ y $c = 4$?

        El modelo viene dado por la ecuación $O(p_{n-1})= D(p_n)$:
        \begin{equation*}
            1+p_{n-1}^2 = c-dp_n \Longrightarrow
            p_n = \frac{c-1-p_{n-1}^2}{d}
        \end{equation*}

        La función asociada es:
        \Func{f}{\bb{R}^+_0}{\bb{R}^+_0}{x}{\dfrac{c-1-x^2}{d}}

        Tenemos que $f\in C^\infty(\bb{R}^+_0)$, y su derivada es:
        \begin{align*}
            f'(x) &= -\frac{2}{d}\cdot x\\
            f'(p^\ast) &= -\frac{-d + \sqrt{d^2+4(c-1)}}{d} = \frac{d - \sqrt{d^2+4(c-1)}}{d}
            = 1-\frac{\sqrt{d^2+4(c-1)}}{d}
        \end{align*}

        Para poder asegurar la estabilidad asintótica de $p^\ast$, empleando el Criterio de la Primera Derivada necesitamos que $|f'(p^\ast)|<1$. Como sabemos que $\left(-d + \sqrt{d^2+4(c-1)}\right)>0$ por tener $p^\ast>0$ y, además, $d\in \bb{R}^+$, tenemos que $f'(p^\ast)\leq 0$. Por tanto, basta con imponer que $-1<f'(p^\ast)$:
        \begin{align*}
            -1 &< 1-\frac{\sqrt{d^2+4(c-1)}}{d}
            \Longleftrightarrow
            -2 < -\frac{\sqrt{d^2+4(c-1)}}{d}
            \Longleftrightarrow \\ &\Longleftrightarrow
            2 > \frac{\sqrt{d^2+4(c-1)}}{d} \Longleftrightarrow
            2d > \sqrt{d^2+4(c-1)}
            \Longleftrightarrow \\ &\Longleftrightarrow
            4d^2 > d^2+4(c-1) \Longleftrightarrow
            3d^2 > 4(c-1) \Longleftrightarrow
            3d^2-4c>-4
        \end{align*}

        Por tanto, hemos de imponer que $3d^2-4c>-4$. En el caso de $d=2$ y $c=4$, tenemos que:
        \begin{equation*}
            3\cdot 2^2 - 4\cdot 4 = 12-16 = -4
        \end{equation*}

        Por tanto, en este caso no basta con usar el Criterio de la Primera Derivada. Tenemos que:
        \begin{equation*}
            p^\ast = \frac{-2+ \sqrt{4+4\cdot 3}}{2} = \frac{-2+4}{2} = 1 \qquad f'(-1)=-1
        \end{equation*}

        En este caso, y debido a la observación de la página \pageref{obs:polinomio2grado}, tenemos que $p^\ast=1$ es asintóticamente localmente estable.
        
        \item Para $c = 3$ y $d = 2$, usa un diagrama de Cobweb para trazar los valores de $p_1$ y $p_2$ a partir de $p_0 = 1$. ¿Cómo se comportarán los precios a largo plazo en este caso?\\

        En este caso, tenemos:
        \begin{equation*}
            3\cdot 2^2 - 4\cdot 3 = 12-12 = 0 > -4
        \end{equation*}

        Por tanto, tenemos que el punto de equilibrio $p^\ast$ es asintóticamente estable localmente. En este caso, $p^\ast = -1+\sqrt{3}\approx 0.732$. Sabiendo la Ley de Recurrencia, tenemos que:
        \begin{equation*}
            p_0 = 1 \qquad p_1 = \nicefrac{1}{2} \qquad p_2 = \nicefrac{7}{8}
        \end{equation*}        
        
        \begin{figure}[H]
            \centering
            \begin{tikzpicture}
                \begin{axis}[
                    axis lines = center,
                    axis equal, % Asegura que los ejes tengan la misma escala
                    xlabel = $p$,
                    legend pos=outer north east
                ]
                
                % Curva de oferta
                \addplot[domain=-0.5:1.5, samples=100, color=red]{1+x^2}; 
                \addlegendentry{$O(p)$}
                
                % Curva de demanda
                \def\c{3}
                \def\d{2}
                \addplot[domain=-0.5:1.5, samples=2, color=blue]{\c-\d*x}; 
                \addlegendentry{$D(p)$}
                
                % Trayectoria de la telaraña
                \draw[dashed, -stealth] ( 1.00000, 0.00000) -- ( 1.00000, 2.00000);
                \draw[dashed, -stealth] ( 1.00000, 2.00000) -- ( 0.50000, 2.00000);
                \draw[dashed, -stealth] ( 0.50000, 2.00000) -- ( 0.50000, 1.25000);
                \draw[dashed, -stealth] ( 0.50000, 1.25000) -- ( 0.87500, 1.25000);
                \addlegendentry{Telaraña}
                \end{axis}
            \end{tikzpicture}
            \caption{Representación del modelo de la Telaraña del Ejercicio \ref{ej:2.13}.}
        \end{figure}
    \end{enumerate}
\end{ejercicio}

\begin{ejercicio}
    Determina los $2-$ciclos de los siguientes sistema dinámicos y estudia su estabilidad:
    \begin{enumerate}
        \item $x_{n+1}=1-x_n^2$.

        Su función asociada es:
        \Func{f}{\bb{R}}{\bb{R}}{x}{1-x^2}

        Para estudiar los $2-$ciclos, necesitamos obtener $f^2$. Tenemos que:
        \begin{equation*}
            f^2(x)=f(f(x)) = f(1-x^2) = 1-(1-x^2)^2 = 1-(1-2x^2+x^4)
            = 2x^2-x^4 = x^2(2-x^2)
        \end{equation*}

        Los elementos del $2-$ciclo serán los puntos fijos de $f^2$, calculémoslos:
        \begin{equation*}
            f^2(x)=x \Longleftrightarrow x^2(2-x^2) = x \Longleftrightarrow \left\{
            \begin{array}{l}
                x = 0\\
                \quad \lor \\
                x(2-x^2)=1 \Longleftrightarrow -x^3+2x-1
            \end{array}\right.
        \end{equation*}

        Aplicamos el Método de Ruffini para resolver dicha ecuación:
        \begin{figure}[H]
            \centering
            \polyhornerscheme[x=1]{2x-1-x^3}
            \caption{División mediante Ruffini donde se ve que $x=1$ es una solución.}
        \end{figure}

        Por tanto, tenemos que dos puntos fijos de $f^2$ son $0, 1$. Además, también tendrá como puntos fijos las soluciones de la ecuación $-x^2-x+1=0$, pero esas son las soluciones de $f(x)=x$; es decir, son soluciones constantes. Por tanto, el $2-$ciclo es:
        \begin{equation*}
            \{0, 1\}
        \end{equation*}
        
        \item $x_{n+1}=5-\frac{6}{x_n}$.

        Su función asociada es:
        \Func{f}{\bb{R}}{\bb{R}}{x}{5-\dfrac{6}{x}}

        Para estudiar los $2-$ciclos, necesitamos obtener $f^2$. Tenemos que:
        \begin{equation*}
            f^2(x)=f(f(x)) = f\left(5-\dfrac{6}{x}\right) = 5-\dfrac{6}{5-\dfrac{6}{x}}
            = 5-\dfrac{6x}{5x-6}
        \end{equation*}

        Los elementos del $2-$ciclo serán los puntos fijos de $f^2$, calculémoslos:
        \begin{align*}
            f^2(x)=x &\Longleftrightarrow 5-\dfrac{6x}{5x-6} = x \Longleftrightarrow
            (5-x)(5x-6) = 6x \Longleftrightarrow \\ &\Longleftrightarrow
            25x-30-5x^2+6x=6x \Longleftrightarrow
            x^2-5x+6 = 0 \Longleftrightarrow
            \left\{
            \begin{array}{c}
                x = 3\\
                \lor \\
                x=2
            \end{array}\right.
        \end{align*}

        Comprobemos ahora que no son soluciones constantes:
        \begin{equation*}
            f(x)=x \Longleftrightarrow
            5-\frac{6}{x} = x \Longleftrightarrow 5x-6=x^2
        \end{equation*}
        Como podemos ver, los dos valores anteriores cumplen la ecuación, por lo que son puntos fijos. Por tanto, no hay $2-$ciclos no triviales.
    \end{enumerate}
\end{ejercicio}


\begin{ejercicio}
    Para $a,b\in \bb{R}$ se considera la función:
    \begin{equation*}
        f(x)=\left\{\begin{array}{lll}
            ax & \text{si} & x<0\\
            bx & \text{si} & 0\leq x
        \end{array}\right.
    \end{equation*}
    demuestra las siguientes propiedades:
    \begin{enumerate}
        \item $\alpha = 0$ es un punto fijo de $f$ para cualesquiera $a$ y $b$.

        Para cualquier valores de $a,~b$, tenemos que $f(0)=b\cdot 0=0$, por lo que $\alpha=0$ es un punto fijo.
        
        \item Si $0 < a < 1$ y $0 < b < 1$ entonces $\alpha = 0$ es asintóticamente estable.\\

        Distinguimos en función del signo de $x_0$:
        \begin{itemize}
            \item \ul{Si $x_0<0$}:

            Demostremos en primer lugar que la solución está mayorada por $0$:
            \begin{itemize}
                \item Para $n=0$, tenemos $x_0<0$.
                \item Supuesto cierto para $n$, demostramos para $n+1$:
                \begin{equation*}
                    x_{n+1} = f(x_n) = a\cdot x_n < 0
                \end{equation*}
                donde he aplicado que $a\geq 0$ y $x_n<0$.
            \end{itemize}

            Veamos ahora que la solución es estrictamente creciente:
            \begin{equation*}
                x_{n+1} = f(x_n) = a\cdot x_n > x_n \Longleftrightarrow a < 1
            \end{equation*}
            donde, tras simplificar $x_n$ puesto que $x_n\neq 0$, he empleado que $x_n<0$, por lo que se invierte el sentido de la desigualdad. Por tanto, $\{x_n\}$ es creciente y mayorada, por lo que $\{x_n\}\to 0$. Para demostrar que es estable, tomando $\delta=\veps$, tenemos que:
            \begin{equation*}
                |x_0-0|=-x_0 <\delta \Longrightarrow |x_n-0| = -x_n < -x_0 <\veps
            \end{equation*}
            donde he empleado que $x_0<x_n$. Por tanto, tenemos que es asintóticamente estable localmente por abajo.

            \item \ul{Si $x_0>0$}:

            Demostremos en primer lugar que la solución está minorada por $0$:
            \begin{itemize}
                \item Para $n=0$, tenemos $x_0>0$.
                \item Supuesto cierto para $n$, demostramos para $n+1$:
                \begin{equation*}
                    x_{n+1} = f(x_n) = b\cdot x_n > 0
                \end{equation*}
                donde he aplicado que $b\geq 0$ y $x_n>0$.
            \end{itemize}

            Veamos ahora que la solución es estrictamente decreciente:
            \begin{equation*}
                x_{n+1} = f(x_n) = b\cdot x_n < x_n \Longleftrightarrow b < 1
            \end{equation*}
            donde, tras simplificar $x_n$ puesto que $x_n\neq 0$, he empleado que $x_n>0$, por lo que se mantiene el sentido de la desigualdad. Por tanto, $\{x_n\}$ es decreciente y minorada, por lo que $\{x_n\}\to 0$. Para demostrar que es estable, tomando $\delta=\veps$, tenemos que:
            \begin{equation*}
                |x_0-0|=x_0 <\delta \Longrightarrow |x_n-0| =x_n < x_0 <\veps
            \end{equation*}
            donde he empleado que la sucesión solución es creciente. Por tanto, tenemos que es asintóticamente estable localmente por encima.
        \end{itemize}
        Por tanto, deducimos que es asintóticamente estable localmente.
        
        \item Si $0 < a < 1$ y $b > 1$ entonces $\alpha = 0$ es inestable.

        Veamos que no es estable. Tomado $x_0>0$, veamos que la solución es creciente:
        \begin{equation*}
            x_{n+1} = f(x_n) = b\cdot x_n > x_n \Longleftrightarrow b > 1
        \end{equation*}
        
        Por tanto, la solución es estrictamente creciente. Veamos que no está acotada por reducción al absurdo. Supongamos que lo está, y tengamos $\{x_n\}\to L$. Tenemos que:
        \begin{equation*}
            L = \lim_{n\to \infty} x_n = \lim_{n\to \infty} x_{n+1} 
            = \lim_{n\to \infty} f(x_n) \AstIg f\left(\lim_{n\to \infty} x_n\right) = f(L)
        \end{equation*}
        donde en $(\ast)$ hemos empleado la continuidad de $f$. Por tanto, $f(L)=L$, por lo que $L$ es un punto fijo y ha de ser $\alpha=0$ por ser el único punto fijo. No obstante, la sucesión solución es estrictamente creciente con $x_0>0$, por lo que llegamos a una contradicción.
        Por tanto, $\{x_n\}$ no está acotada, y se tiene:
        \begin{equation*}
            \lim_{n\to \infty}x_n = \infty
        \end{equation*}

        En concreto, tenemos que es inestable por encima, y por tanto es inestable.
        
        \item Si $a < 0$ y $b < 0$ y $ab < 1$ entonces $\alpha = 0$ es asintóticamente estable.\\

        Veamos el valor de $f^2$. Tenemos que:
        \begin{itemize}
            \item Si $x<0$, tenemos que $f(x)=ax>0$, por lo que $f^2(x)=abx$.\\
            \item Si $x>0$, tenemos que $f(x)=bx<0$, por lo que $f^2(x)=abx$.\\
        \end{itemize}

        En cualquier caso, tenemos que $f^2(x)=abx$ para todo $x\in \bb{R}$. Tenemos que $f^2\in C^\infty(\bb{R})$, y su derivada es:
        \begin{equation*}
            (f^2)'(x) = ab \qquad \forall x\in \bb{R}
        \end{equation*}
        Como $|(f^2)'(\alpha)|=|ab|=ab<1$, tenemos que $\alpha$ es asintóticamente estable localmente para $f^2$. Debido al Lema \ref{lema:f2siif}, también lo es para $f$.
        
        \item ¿Qué puede decirse sobre la estabilidad de los puntos fijos cuando $b = 1$ y $a > 1$?

        En este caso, el conjunto de los puntos fijos es $\bb{R}^+_0$. Para los elementos de $\bb{R}^+$, tenemos que es estable tomando $\delta=\veps$. No obstante, no es asintóticamente estable, ya que $\lim\limits_{n\to \infty} x_n = x_0$ para cualquier valor de $x_0>0$, y por tanto no es un atractor local.

        En el caso de un punto $x_0<0$, tenemos que la solución es decreciente:
        \begin{equation*}
            x_{n+1} = f(x_n) = a\cdot x_n < x_n \Longleftrightarrow a> 1
        \end{equation*}
        
        Por tanto, la solución es estrictamente decreciente. Veamos que no está acotada por reducción al absurdo. Supongamos que lo está, y tengamos $\{x_n\}\to L$. Tenemos que:
        \begin{equation*}
            L = \lim_{n\to \infty} x_n = \lim_{n\to \infty} x_{n+1} 
            = \lim_{n\to \infty} f(x_n) \AstIg f\left(\lim_{n\to \infty} x_n\right) = f(L)
        \end{equation*}
        donde en $(\ast)$ hemos empleado la continuidad de $f$. Por tanto, $f(L)=L$, por lo que $L$ es un punto fijo y ha de ser $\alpha=0$ por ser el único punto fijo. No obstante, la sucesión solución es estrictamente decreciente con $x_0<0$, por lo que llegamos a una contradicción.
        Por tanto, $\{x_n\}$ no está acotada, y se tiene:
        \begin{equation*}
            \lim_{n\to \infty}x_n = -\infty
        \end{equation*}

        En concreto, tenemos que es inestable por debajo, y por tanto es inestable.
    \end{enumerate}
\end{ejercicio}

\begin{ejercicio}
    Sea $f:\bb{R}\to \bb{R}$ definida por
    \begin{equation*}
        f(x)=\left\{\begin{array}{lll}
            0 & \text{si} & x=0\\
            \frac{1}{2}x\sen\left(\frac{1}{x}\right) & \text{si} & x\neq 0
        \end{array}\right.
    \end{equation*}
    Estudia la estabilidad de los puntos fijos de $x_{n+1} = f(x_n)$. ¿Es $f$ contractiva?\\

    Tenemos que $f(0)=0$. Veamos si hay otro punto fijo:
    \begin{equation*}
        \frac{1}{2}x\sen\left(\frac{1}{x}\right) = x
        \Longleftrightarrow
        \sen\left(\frac{1}{x}\right) = 2
    \end{equation*}

    Por tanto, no hay más puntos fijos. Veamos en primer lugar que $|x_{n+1}| < |x_n|$ para todo $n\in \bb{N}$:
    \begin{equation*}
        |x_{n+1}| = \left|\frac{1}{2}x_n\sen\left(\nicefrac{1}{x_n}\right)\right| \leq \left|\frac{1}{2}x_n\right|<|x_n|
    \end{equation*}

    Veamos ahora que es estable. Fijado $\veps\in \bb{R}^+$, tomando $\delta=\veps$ tenemos que:
    \begin{equation*}
        |x_0-0| = |x_0|<\delta \Longrightarrow
        |x^n-0| = |x^n| < |x_0| < \veps
    \end{equation*}
    Por tanto, el $0$ es estable. Veamos ahora que es un atractor local. Como $|x_n|$ es estrictamente decreciente y minorada por $0$, tenemos que $\{|x_n|\}\to L$. Por tanto, por la unicidad del límite, como $x_{n+1}=f(x_n)$, tenemos que $L=f(L)$, por lo que $\{|x_n|\}\to 0$. Veamos que $\{x_n\}\to 0$. Por la convergencia del valor absoluto, tenemos que:
    \begin{equation*}
        \forall \veps\in \bb{R}^+,~\exists n\in \bb{N}\mid \text{Si } m\in \bb{N},~m\geq n\Longrightarrow
        ||x_m|-0| = |x_m| = |x_m-0|<\veps
    \end{equation*}
    Por tanto, tenemos que $\{x_n\}\to 0$, por lo que es un atractor local. Por tanto, $0$ es asintóticamente estable localmente.\\

    Por úĺtimo, veamos si es contractiva. Para ello, hemos de calcular la constante de Lipschitz acotando la derivada. Tenemos que:
    \begin{equation*}
        f'(x)=\frac{1}{2}\left(\sen\left(\nicefrac{1}{x}\right)-x\cos\left(\nicefrac{1}{x}\right)\cdot \frac{1}{x^2}\right)
        =\frac{1}{2}\left(\sen\left(\nicefrac{1}{x}\right)-\cos\left(\nicefrac{1}{x}\right)\cdot \frac{1}{x}\right)
    \end{equation*}

    Para que sea Lipschitziana es necesario que la derivada esté acotada. No obstante, cuando $x\to 0$ tenemos que $f'$ diverge (no está acotada), por lo que no es Lipschitziana y por tanto tampoco es contractiva.
\end{ejercicio}



\begin{ejercicio}
    La evolución de una determinada población viene descrita por la ecuación en diferencias
    \begin{equation*}
        x_{n+1} = x_ne^{a-x_n} \qquad a\in \bb{R}^+
    \end{equation*}
    \begin{enumerate}
        \item Determina los puntos de equilibrio y estudia su estabilidad en función del valor de $a$.

        Sea su función asociada:
        \Func{f}{\bb{R}}{\bb{R}}{x}{xe^{a-x}}

        Tenemos que los puntos fijos, además de $x=0$, son:
        \begin{equation*}
            1 = e^{a-x} \Longrightarrow 0 = a-x \Longrightarrow x=a
        \end{equation*}

        Además, sabemos que $f\in C^\infty(\bb{R})$. Tenemos que:
        \begin{equation*}
            f'(x) = e^{a-x}(1-x)
        \end{equation*}

        Estudiamos la estabilidad de cada uno de los puntos fijos empleando el Criterio de la Primera Derivada:
        \begin{itemize}
            \item \ul{$x=0$}:

            Tenemos que $f'(0)=e^a>0$. Para aplicar el criterio, imponemos que $f'(0)<1$:
            \begin{equation*}
                f'(0)<1 \Longleftrightarrow e^a<e^0 \Longleftrightarrow
                a<0
            \end{equation*}
            Por tanto, como $a\in \bb{R}^+$, tenemos que no se tiene, por lo que $x=0$ es un equilibrio inestable.

            \item \ul{$x=a$}:

            Tenemos que $f'(a)=e^0(1-a)=1-a$. Para aplicar el criterio, imponemos que $|f'(a)|<1$:
            \begin{align*}
                |f'(a)|<1 \Longleftrightarrow
                -1 < 1-a < 1 \Longleftrightarrow
                \left\{\begin{array}{c}
                    2 > a\\\land \\ a>0
                \end{array}\right\} \Longleftrightarrow
                a \in~]0,2[
            \end{align*}
            
            Para $a\in \bb{R}\setminus [0,2]$ sabemos que $|f'(a)|>1$, por lo que es inestable. Veamos qué ocurre para $a=0,2$.
            \begin{itemize}
                \item Para $a=0$, no se puede tener pues $a\in \bb{R}^+$.
                \item Para $a=2$, tenemos que $f'(2)=1-2=-1$. Calculamos $f''(2),~f'''(2)$:
                \begin{align*}
                    f''(x)&=e^{a-x}(-1(1-x) -1)=e^{a-x}(-2+x) \qquad f''(2)=0\\
                    f'''(x)&=e^{a-x}(-1(-2+x) +1)=e^{a-x}(3-x) \qquad f'''(2)=e^{a-2} = e^0 = 1
                \end{align*}

                Tenemos que:
                \begin{equation*}
                    2f'''(2) + 3(f''(2))^2 = 2 > 0
                \end{equation*}

                Por tanto, tenemos que $x=a=2$ es asintóticamente estable localmente. 
            \end{itemize}
        \end{itemize}
        
        \item Para el caso $a = 3$, comprueba que $\{0.424321, 5.57568\}$ es un $2-$ciclo (aproximado). ¿Es asintóticamente estable?\\

        Busquemos calcular $f^2$. Tenemos que:
        \begin{equation*}
            f^2(x) = f(f(x)) = f(xe^{a-x}) = xe^{a-x}\cdot e^{a-xe^{a-x}}
        \end{equation*}

        Los puntos fijos de $f^2$ son, bien el $0$, bien las soluciones de la siguiente ecuación:
        \begin{equation*}
            e^{a-x}\cdot e^{a-xe^{a-x}} = 1 \Longleftrightarrow
            e^{a-x + a-xe^{a-x}} = e^{2a-x-e^{a-x}} = e^0\Longleftrightarrow
            2a-x=e^{a-x}
        \end{equation*}

        Dicha ecuación es trascendente, por lo que no podemos obtener de forma explícita los puntos fijos, tan solo aproximarlos, como nos los dan en el enunciado. Sea el ciclo entonces $\{\ol{x_0}, \ol{x_1}\}$. Buscamos demostrar que $f(\ol{x_0})\approx \ol{x_1}$ y $f(\ol{x_1})\approx \ol{x_0}$:
        \begin{align*}
            f(\ol{x_0}) &= 5.5756782285075 \approx \ol{x_1}\\
            f(\ol{x_1}) &= 0.4243207104934 \approx \ol{x_0}\\
        \end{align*}

        Estudiemos ahora su estabilidad. Como $f\in C^\infty(\bb{R})$, buscamos ver el valor de la primera derivada evaluada en los elementos del ciclo.
        \begin{equation*}
            \left.
            \begin{array}{lr}
                f'(\ol{x_0}) =& 7.5645581220561\\
                f'(\ol{x_1}) =& -0.3482186546916\\
            \end{array}
            \right\} \Longrightarrow
            |f'(\ol{x_0})f'(\ol{x_1})| = |-2.6341202525984|>1
        \end{equation*}

        Por tanto, el $2-$ciclo visto es inestable.
    \end{enumerate}
\end{ejercicio}

\begin{ejercicio}
    En cierto mercado los precios de un determinado producto siguen una dinámica basada en los postulados del modelo de la telaraña, donde las funciones de oferta y demanda vienen dadas por:
    \begin{equation*}
        D(p)=5-p,\qquad O(p)=2+\frac{(p-2)^3}{3}.
    \end{equation*}
    Con el fin de estudiar la evolución de los precios se pide lo siguiente:
    \begin{enumerate}
        \item Prueba que si $f: \bb{R} \to \bb{R}$ es una función continua y decreciente, entonces $x_{n+1} = f(x_n)$ tiene un único punto de equilibrio.\\

        Está demostrado como parte de la demostración del ejemplo de la página \pageref{ej:decreciente_cont}.
        
        \item Construye una ecuación en diferencias para el precio y localiza un intervalo entre dos enteros consecutivos que contenga al precio de equilibrio.

        Para el modelo dinámico, tenemos que $O(p_{n-1})=D(p_n)$. Por tanto, tenemos que:
        \begin{equation*}
            2 + \dfrac{(p_{n-1}-2)^3}{3} = 5-p_n \Longrightarrow
            p_n = 3 - \dfrac{(p_{n-1}-2)^3}{3}
        \end{equation*}
        
        Podríamos intentar buscar explícitamente el punto de equilibrio $p^\ast$, pero llegaremos a un problema mayor:
        \begin{multline*}
            p^\ast = 3 - \dfrac{(p^\ast-2)^3}{3}
            \Longleftrightarrow
            3p^\ast = 9 -(p^\ast)^3 + 4p^\ast - 4
            \Longleftrightarrow
            (p^\ast)^3 - p^\ast - 5 = 0
        \end{multline*}

        Resolver dicha ecuación de grado $3$ explícitamente no es fácil, por lo que buscaremos aplicar el Teorema de Bolzano.

        Sea $f$ la función asociada a la ecuación en diferencias:
        \Func{f}{\bb{R}}{\bb{R}}{x}{3-\dfrac{(x-2)^3}{3}}

        Como buscamos una solución de $f(x)=x$, definimos $g=f-Id$. Tenemos que:
        \begin{equation*}
            g(2) = f(2)-2 = 3-2=1 \qquad g(3) = f(3)-3 = 3-\frac{1}{3} -3 = -\frac{1}{3}
        \end{equation*}
        Por tanto, por el Teorema de Bolzano, $\exists p^\ast\in~]2,3[$ tal que $g(p^\ast)=0$ o, equivalentemente, $f(p^\ast) = p^\ast$.
        
        \item Calcula la derivada de $f$ en el entorno anterior. Comprueba que el punto de equilibrio obtenido es asintóticamente estable.\\

        Tenemos que $f\in C^\infty(\bb{R})$, y su primera derivada es:
        \begin{equation*}
            f'(x) = -(x-2)^2 < 0
        \end{equation*}

        Tenemos que $f'$ es estrictamente decreciente, con $f'(2)=0$, $f'(3)=-1$. Por tanto:
        \begin{equation*}
            2 < p^\ast < 3 \Longrightarrow -1 = f'(3) < f'(p^\ast) < f'(2) = 0
        \end{equation*}
        En definitiva, tenemos que $|f'(p^\ast)|<1$, por lo que el punto de equilibrio obtenido es asintóticamente estable.
        
        \item Sea $g(x) = f(f(x))$, donde $f$ es la función del apartado anterior. Prueba que se verifica $g(3) < 3 < 4 < g(4)$, y deduce como consecuencia de ello que la ecuación en diferencias admite un $2-$ciclo.\\

        Tenemos que $g=f^2$. Tenemos que:
        \begin{align*}
            g(3) &= f(f(3)) = f\left(3-\frac{1}{3}\right)
            = 3 - \frac{\left(1-\frac{1}{3}\right)^3}{3}
            = 3 - \frac{\frac{2^3}{3^3}}{3}
            = 3 - \frac{2^3}{3^4} < 3\Longleftrightarrow
            \\ &\qquad \Longleftrightarrow \frac{2^3}{3^4}>0\\
            g(4) &= f(f(4)) = f\left(3-\frac{2^3}{3}\right)
            = 3 - \frac{\left(1-\frac{2^3}{3}\right)^3}{3}
            = 3 - \frac{-\frac{5^3}{3^3}}{3}
            = 3 + \frac{5^3}{3^4} > 4 \Longleftrightarrow
            \\ &\qquad \Longleftrightarrow \frac{5^3}{3^4} > 1 \Longleftrightarrow 5^3 = 125 > 81 = 3^4
        \end{align*}

        Por tanto, efectivamente tenemos que $g(3)<3<4<g(4)$. Definimos ahora $h=g-Id = f^2-Id$. Tenemos que:
        \begin{align*}
            h(3) &= g(3)-3 < 0\\
            h(4) &= g(4)-4 > 0
        \end{align*}
        Por tanto, por el Teorema de Bolzano, $\exists c\in ~]3,4[$ tal que $f^2(c)=c$. Además, como $c\neq p^\ast$ y en el primer apartado hemos visto que el punto fijo es único, tenemos que $f(c)\neq c$. Por tanto, tenemos que presenta un $2-$ciclo dado por:
        \begin{equation*}
            \{c, f(c)\}
        \end{equation*}
        
        \item ¿Es el precio de equilibrio un atractor global?\\

        Buscamos ver si, para cualquier valor de $x_0$, se tiene que:
        \begin{equation*}
            \lim_{n\to \infty} p_n = x^\ast
        \end{equation*}

        No es cierto, ya que si por ejemplo $p_0=c$ el primer elemento del $2-$ciclo descubierto en el apartado anterior, tenemos que $\{p_n\}$ no converge, ya que se mantiene en el $2-$ciclo.
        
        \item Estudia el comportamiento de los precios del modelo teniendo en cuenta la información obtenida en los apartados anteriores.\\

        Tenemos que si los precios comienzan cercanos al precio de equilibrio, convergerán a dicho precio. Además, presenta un $2-$ciclo.
    \end{enumerate}
\end{ejercicio}