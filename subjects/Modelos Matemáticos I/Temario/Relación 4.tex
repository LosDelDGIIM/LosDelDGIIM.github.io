\section{Sistemas de ecuaciones en diferencias lineales}

\begin{ejercicio}
Se consideran las siguientes matrices:

\begin{equation*}
    A_1 =
    \begin{pmatrix}
    6 & 2\\
    -3 & 2
    \end{pmatrix}
    , \hspace{1cm}
    A_2=
    \begin{pmatrix}
    -1 & 1 & 0\\
    1 & 1 & 2\\
    1 & -2 & -1
    \end{pmatrix}
    , \hspace{1cm}
    A_3=\begin{pmatrix}
    3 & 0 & 0 & 0\\
    2 & 7 & -2 & 2\\
    -8 & 1 & -2 & 1\\
    2 & 0 & 0 & 4
    \end{pmatrix}.
\end{equation*}

Determina en cada caso su radio espectral. Para cada matriz, ¿qué ecuación en diferencias de orden superior verifica cada componente del vector solución del sistema lineal homogéneo asociado?
\end{ejercicio}

\begin{ejercicio}
    Sea $A$ una matriz real tal que $|A| > 1$. Demuestra que el siguiente sistema no es convergente.
    \begin{equation*}
    X_{n+1} = AX_n
    \end{equation*}
    
    Proporciona un ejemplo de matriz $A$ tal que dicho sistema no sea convergente y $0 < |A| < 1$.
\end{ejercicio}

\begin{ejercicio}
Sea $A$ una matriz cuadrada con radio espectral menor que 1. Demuestra las siguientes propiedades:

\begin{enumerate}
    \item La matriz $I - A$ es invertible.
    \item El sistema $X = AX + B$ es compatible determinado para cualquier $B \in \mathbb{R}^k$.
    \item Dado $B \in \mathbb{R}^k$, la solución de $X_{n+1} = AX_n + B$ tiene límite para cualquier condición inicial.
\end{enumerate}
\end{ejercicio}

\begin{ejercicio}
Demuestra que las tres componentes de la solución del sistema:
\begin{equation*}
\left\{
    \begin{array}{rcrrrrr}
        a_{n+1} &=&  0.1a_n &+ 0.2b_n &+ 0.3c_n &+ 1\\
        b_{n+1} &=&  0.1a_n &+ 0.1b_n &- 0.2c_n &+ 1\\
        c_{n+1} &=& -0.2a_n &- 0.3b_n &+ 0.2c_n &+ 1
    \end{array}
\right.
\end{equation*}

con valor inicial $a_0 = 1.53$, $b_0 = 1.43$ y $c_0 = 2.2$, tienen límite y calcúlalo.
\end{ejercicio}


\begin{ejercicio}
Se supone que el precio del desayuno en los bares sigue el modelo de la oferta y la demanda en función del precio de la leche en el mercado. Concretamente, si denotamos $p^l$ el precio de la leche y $p^d$ el precio del desayuno, y consideramos:
\begin{itemize}
    \item La demanda de la leche $D_l(p^l) = 2 - 2p^l$,
    \item La oferta de la leche $O_l(p^l) = 1 + p^l$,
    \item La demanda del desayuno $D_d(p^d) = 1 - 3p^d$,
    \item La oferta del desayuno $O_d(p^d) = 1 + 2p^d - ep^l$, con $e > 0$,
\end{itemize}

Entonces, el modelo viene dado por:
\begin{align*}
D_l(p_n) &= O_l(p_{n-1}),\\
D_d(p_n) &= O_d(p_{n-1}),
\end{align*}
donde hemos denotado $p^d_n$ y $p^l_n$ el precio del desayuno y de la leche en el año $n$, respectivamente.
\begin{enumerate}
    \item Escribe el sistema de ecuaciones en diferencias del modelo.
    \item Resuelve el sistema en función del dato inicial y del parámetro $e$.
\end{enumerate}
\end{ejercicio}

\begin{ejercicio}
Los siguientes modelos representan una población compuesta por dos especies en competición. Estudia en cada caso la estabilidad de los puntos fijos.
\begin{enumerate}
    \item El modelo es:
    \begin{equation*}
        \begin{cases}
            x_{n+1} = x_n(1.7 - 0.02x_n - 0.08y_n)\\
            y_{n+1} = y_n(1.5 - 0.03x_n - 0.04y_n)
        \end{cases}
    \end{equation*}

    \item El modelo es:
    \begin{equation*}
        \begin{cases}
            x_{n+1} = x_n(1.8 - 0.06x_n - 0.03y_n)\\
            y_{n+1} = y_n(1.9 - 0.02x_n - 0.04y_n)
        \end{cases}
    \end{equation*} 
\end{enumerate}

\end{ejercicio}

\begin{ejercicio}
Si en el modelo de crecimiento de una población estructurada por sexos no se asume una distribución equitativa entre hembras y machos, el modelo resultante es

\begin{equation*}
\begin{cases}
x_{n+1} = x_n + \alpha_x x_n y_n - \mu_x x_n\\
y_{n+1} = y_n + \alpha_y x_n y_n - \mu_y y_n
\end{cases}
\end{equation*}
donde $x_n$ denota el número de hembras e $y_n$ el número de machos en el $n-$ésimo año. Los parámetros $\alpha_x$ y $\alpha_y$ representan la tasa de natalidad por pareja de hembras y machos, respectivamente, y $0 < \mu_x < 1$ y $0 < \mu_y < 1$ son las respectivas mortalidades. Dados $\alpha_x = 0.05$, $\alpha_y = 0.02$ y $\mu_x = \mu_y = 0.3$, estudia la estabilidad de los puntos de equilibrio.
\end{ejercicio}

\begin{ejercicio}
Estudia la estabilidad de los puntos de equilibrio del siguiente modelo de presa-depredador:
\begin{equation*}
\begin{cases}
x_{n+1} = 2x_n - x_n y_n\\
y_{n+1} = 1.5y_n - 2y_n^2 + x_n y_n
\end{cases}
\end{equation*}
\end{ejercicio}
