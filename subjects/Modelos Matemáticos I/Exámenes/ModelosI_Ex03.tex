\documentclass[12pt]{article}

% Idioma y codificación
\usepackage[spanish, es-tabla]{babel}       %es-tabla para que se titule "Tabla"
\usepackage[utf8]{inputenc}

% Márgenes
\usepackage[a4paper,top=3cm,bottom=2.5cm,left=3cm,right=3cm]{geometry}

% Comentarios de bloque
\usepackage{verbatim}

% Paquetes de links
\usepackage[hidelinks]{hyperref}    % Permite enlaces
\usepackage{url}                    % redirecciona a la web

% Más opciones para enumeraciones
\usepackage{enumitem}

% Personalizar la portada
\usepackage{titling}

% Paquetes de tablas
\usepackage{multirow}


%------------------------------------------------------------------------

%Paquetes de figuras
\usepackage{caption}
\usepackage{subcaption} % Figuras al lado de otras
\usepackage{float}      % Poner figuras en el sitio indicado H.


% Paquetes de imágenes
\usepackage{graphicx}       % Paquete para añadir imágenes
\usepackage{transparent}    % Para manejar la opacidad de las figuras

% Paquete para usar colores
\usepackage[dvipsnames]{xcolor}
\usepackage{pagecolor}      % Para cambiar el color de la página

% Habilita tamaños de fuente mayores
\usepackage{fix-cm}

% Para los gráficos
\usepackage{tikz}

% Para poder situar los nodos en los grafos
\usetikzlibrary{positioning}


%------------------------------------------------------------------------

% Paquetes de matemáticas
\usepackage{mathtools, amsfonts, amssymb, mathrsfs}
\usepackage[makeroom]{cancel}     % Simplificar tachando
\usepackage{polynom}    % Divisiones y Ruffini
\usepackage{units} % Para poner fracciones diagonales con \nicefrac

\usepackage{pgfplots}   %Representar funciones
\pgfplotsset{compat=1.18}  % Versión 1.18

\usepackage{tikz-cd}    % Para usar diagramas de composiciones
\usetikzlibrary{calc}   % Para usar cálculo de coordenadas en tikz

%Definición de teoremas, etc.
\usepackage{amsthm}
%\swapnumbers   % Intercambia la posición del texto y de la numeración

\theoremstyle{plain}

\makeatletter
\@ifclassloaded{article}{
  \newtheorem{teo}{Teorema}[section]
}{
  \newtheorem{teo}{Teorema}[chapter]  % Se resetea en cada chapter
}
\makeatother

\newtheorem{coro}{Corolario}[teo]           % Se resetea en cada teorema
\newtheorem{prop}[teo]{Proposición}         % Usa el mismo contador que teorema
\newtheorem{lema}[teo]{Lema}                % Usa el mismo contador que teorema

\theoremstyle{remark}
\newtheorem*{observacion}{Observación}

\theoremstyle{definition}

\makeatletter
\@ifclassloaded{article}{
  \newtheorem{definicion}{Definición} [section]     % Se resetea en cada chapter
}{
  \newtheorem{definicion}{Definición} [chapter]     % Se resetea en cada chapter
}
\makeatother

\newtheorem*{notacion}{Notación}
\newtheorem*{ejemplo}{Ejemplo}
\newtheorem*{ejercicio*}{Ejercicio}             % No numerado
\newtheorem{ejercicio}{Ejercicio} [section]     % Se resetea en cada section


% Modificar el formato de la numeración del teorema "ejercicio"
\renewcommand{\theejercicio}{%
  \ifnum\value{section}=0 % Si no se ha iniciado ninguna sección
    \arabic{ejercicio}% Solo mostrar el número de ejercicio
  \else
    \thesection.\arabic{ejercicio}% Mostrar número de sección y número de ejercicio
  \fi
}


% \renewcommand\qedsymbol{$\blacksquare$}         % Cambiar símbolo QED
%------------------------------------------------------------------------

% Paquetes para encabezados
\usepackage{fancyhdr}
\pagestyle{fancy}
\fancyhf{}

\newcommand{\helv}{ % Modificación tamaño de letra
\fontfamily{}\fontsize{12}{12}\selectfont}
\setlength{\headheight}{15pt} % Amplía el tamaño del índice


%\usepackage{lastpage}   % Referenciar última pag   \pageref{LastPage}
\fancyfoot[C]{\thepage}

%------------------------------------------------------------------------

% Conseguir que no ponga "Capítulo 1". Sino solo "1."
\makeatletter
\@ifclassloaded{book}{
  \renewcommand{\chaptermark}[1]{\markboth{\thechapter.\ #1}{}} % En el encabezado
    
  \renewcommand{\@makechapterhead}[1]{%
  \vspace*{50\p@}%
  {\parindent \z@ \raggedright \normalfont
    \ifnum \c@secnumdepth >\m@ne
      \huge\bfseries \thechapter.\hspace{1em}\ignorespaces
    \fi
    \interlinepenalty\@M
    \Huge \bfseries #1\par\nobreak
    \vskip 40\p@
  }}
}
\makeatother

%------------------------------------------------------------------------
% Paquetes de cógido
\usepackage{minted}
\renewcommand\listingscaption{Código fuente}

\usepackage{fancyvrb}
% Personaliza el tamaño de los números de línea
\renewcommand{\theFancyVerbLine}{\small\arabic{FancyVerbLine}}

% Estilo para C++
\newminted{cpp}{
    frame=lines,
    framesep=2mm,
    baselinestretch=1.2,
    linenos,
    escapeinside=||
}

% para minted
\definecolor{LightGray}{rgb}{0.95,0.95,0.92}
\setminted{
    linenos=true,
    stepnumber=5,
    numberfirstline=true,
    autogobble,
    breaklines=true,
    breakautoindent=true,
    breaksymbolleft=,
    breaksymbolright=,
    breaksymbolindentleft=0pt,
    breaksymbolindentright=0pt,
    breaksymbolsepleft=0pt,
    breaksymbolsepright=0pt,
    fontsize=\footnotesize,
    bgcolor=LightGray,
    numbersep=10pt
}


\usepackage{listings} % Para incluir código desde un archivo

\renewcommand\lstlistingname{Código Fuente}
\renewcommand\lstlistlistingname{Índice de Códigos Fuente}

% Definir colores
\definecolor{vscodepurple}{rgb}{0.5,0,0.5}
\definecolor{vscodeblue}{rgb}{0,0,0.8}
\definecolor{vscodegreen}{rgb}{0,0.5,0}
\definecolor{vscodegray}{rgb}{0.5,0.5,0.5}
\definecolor{vscodebackground}{rgb}{0.97,0.97,0.97}
\definecolor{vscodelightgray}{rgb}{0.9,0.9,0.9}

% Configuración para el estilo de C similar a VSCode
\lstdefinestyle{vscode_C}{
  backgroundcolor=\color{vscodebackground},
  commentstyle=\color{vscodegreen},
  keywordstyle=\color{vscodeblue},
  numberstyle=\tiny\color{vscodegray},
  stringstyle=\color{vscodepurple},
  basicstyle=\scriptsize\ttfamily,
  breakatwhitespace=false,
  breaklines=true,
  captionpos=b,
  keepspaces=true,
  numbers=left,
  numbersep=5pt,
  showspaces=false,
  showstringspaces=false,
  showtabs=false,
  tabsize=2,
  frame=tb,
  framerule=0pt,
  aboveskip=10pt,
  belowskip=10pt,
  xleftmargin=10pt,
  xrightmargin=10pt,
  framexleftmargin=10pt,
  framexrightmargin=10pt,
  framesep=0pt,
  rulecolor=\color{vscodelightgray},
  backgroundcolor=\color{vscodebackground},
}

%------------------------------------------------------------------------

% Comandos definidos
\newcommand{\bb}[1]{\mathbb{#1}}
\newcommand{\cc}[1]{\mathcal{#1}}

% I prefer the slanted \leq
\let\oldleq\leq % save them in case they're every wanted
\let\oldgeq\geq
\renewcommand{\leq}{\leqslant}
\renewcommand{\geq}{\geqslant}

% Si y solo si
\newcommand{\sii}{\iff}

% Letras griegas
\newcommand{\eps}{\epsilon}
\newcommand{\veps}{\varepsilon}
\newcommand{\lm}{\lambda}

\newcommand{\ol}{\overline}
\newcommand{\ul}{\underline}
\newcommand{\wt}{\widetilde}
\newcommand{\wh}{\widehat}

\let\oldvec\vec
\renewcommand{\vec}{\overrightarrow}

% Derivadas parciales
\newcommand{\del}[2]{\frac{\partial #1}{\partial #2}}
\newcommand{\Del}[3]{\frac{\partial^{#1} #2}{\partial #3^{#1}}}
\newcommand{\deld}[2]{\dfrac{\partial #1}{\partial #2}}
\newcommand{\Deld}[3]{\dfrac{\partial^{#1} #2}{\partial #3^{#1}}}


\newcommand{\AstIg}{\stackrel{(\ast)}{=}}
\newcommand{\Hop}{\stackrel{L'H\hat{o}pital}{=}}

\newcommand{\red}[1]{{\color{red}#1}} % Para integrales, destacar los cambios.

% Método de integración
\newcommand{\MetInt}[2]{
    \left[\begin{array}{c}
        #1 \\ #2
    \end{array}\right]
}

% Declarar aplicaciones
% 1. Nombre aplicación
% 2. Dominio
% 3. Codominio
% 4. Variable
% 5. Imagen de la variable
\newcommand{\Func}[5]{
    \begin{equation*}
        \begin{array}{rrll}
            #1:& #2 & \longrightarrow & #3\\
               & #4 & \longmapsto & #5
        \end{array}
    \end{equation*}
}

%------------------------------------------------------------------------



\begin{document}

    % 1. Foto de fondo
    % 2. Título
    % 3. Encabezado Izquierdo
    % 4. Color de fondo
    % 5. Coord x del titulo
    % 6. Coord y del titulo
    % 7. Fecha

    
    % 1. Foto de fondo
% 2. Título
% 3. Encabezado Izquierdo
% 4. Color de fondo
% 5. Coord x del titulo
% 6. Coord y del titulo
% 7. Fecha

\newcommand{\portada}[7]{

    \portadaBase{#1}{#2}{#3}{#4}{#5}{#6}{#7}
    \portadaBook{#1}{#2}{#3}{#4}{#5}{#6}{#7}
}

\newcommand{\portadaExamen}[7]{

    \portadaBase{#1}{#2}{#3}{#4}{#5}{#6}{#7}
    \portadaArticle{#1}{#2}{#3}{#4}{#5}{#6}{#7}
}




\newcommand{\portadaBase}[7]{

    % Tiene la portada principal y la licencia Creative Commons
    
    % 1. Foto de fondo
    % 2. Título
    % 3. Encabezado Izquierdo
    % 4. Color de fondo
    % 5. Coord x del titulo
    % 6. Coord y del titulo
    % 7. Fecha
    
    
    \thispagestyle{empty}               % Sin encabezado ni pie de página
    \newgeometry{margin=0cm}        % Márgenes nulos para la primera página
    
    
    % Encabezado
    \fancyhead[L]{\helv #3}
    \fancyhead[R]{\helv \nouppercase{\leftmark}}
    
    
    \pagecolor{#4}        % Color de fondo para la portada
    
    \begin{figure}[p]
        \centering
        \transparent{0.3}           % Opacidad del 30% para la imagen
        
        \includegraphics[width=\paperwidth, keepaspectratio]{assets/#1}
    
        \begin{tikzpicture}[remember picture, overlay]
            \node[anchor=north west, text=white, opacity=1, font=\fontsize{60}{90}\selectfont\bfseries\sffamily, align=left] at (#5, #6) {#2};
            
            \node[anchor=south east, text=white, opacity=1, font=\fontsize{12}{18}\selectfont\sffamily, align=right] at (9.7, 3) {\textbf{\href{https://losdeldgiim.github.io/}{Los Del DGIIM}}};
            
            \node[anchor=south east, text=white, opacity=1, font=\fontsize{12}{15}\selectfont\sffamily, align=right] at (9.7, 1.8) {Doble Grado en Ingeniería Informática y Matemáticas\\Universidad de Granada};
        \end{tikzpicture}
    \end{figure}
    
    
    \restoregeometry        % Restaurar márgenes normales para las páginas subsiguientes
    \pagecolor{white}       % Restaurar el color de página
    
    
    \newpage
    \thispagestyle{empty}               % Sin encabezado ni pie de página
    \begin{tikzpicture}[remember picture, overlay]
        \node[anchor=south west, inner sep=3cm] at (current page.south west) {
            \begin{minipage}{0.5\paperwidth}
                \href{https://creativecommons.org/licenses/by-nc-nd/4.0/}{
                    \includegraphics[height=2cm]{assets/Licencia.png}
                }\vspace{1cm}\\
                Esta obra está bajo una
                \href{https://creativecommons.org/licenses/by-nc-nd/4.0/}{
                    Licencia Creative Commons Atribución-NoComercial-SinDerivadas 4.0 Internacional (CC BY-NC-ND 4.0).
                }\\
    
                Eres libre de compartir y redistribuir el contenido de esta obra en cualquier medio o formato, siempre y cuando des el crédito adecuado a los autores originales y no persigas fines comerciales. 
            \end{minipage}
        };
    \end{tikzpicture}
    
    
    
    % 1. Foto de fondo
    % 2. Título
    % 3. Encabezado Izquierdo
    % 4. Color de fondo
    % 5. Coord x del titulo
    % 6. Coord y del titulo
    % 7. Fecha


}


\newcommand{\portadaBook}[7]{

    % 1. Foto de fondo
    % 2. Título
    % 3. Encabezado Izquierdo
    % 4. Color de fondo
    % 5. Coord x del titulo
    % 6. Coord y del titulo
    % 7. Fecha

    % Personaliza el formato del título
    \pretitle{\begin{center}\bfseries\fontsize{42}{56}\selectfont}
    \posttitle{\par\end{center}\vspace{2em}}
    
    % Personaliza el formato del autor
    \preauthor{\begin{center}\Large}
    \postauthor{\par\end{center}\vfill}
    
    % Personaliza el formato de la fecha
    \predate{\begin{center}\huge}
    \postdate{\par\end{center}\vspace{2em}}
    
    \title{#2}
    \author{\href{https://losdeldgiim.github.io/}{Los Del DGIIM}}
    \date{Granada, #7}
    \maketitle
    
    \tableofcontents
}




\newcommand{\portadaArticle}[7]{

    % 1. Foto de fondo
    % 2. Título
    % 3. Encabezado Izquierdo
    % 4. Color de fondo
    % 5. Coord x del titulo
    % 6. Coord y del titulo
    % 7. Fecha

    % Personaliza el formato del título
    \pretitle{\begin{center}\bfseries\fontsize{42}{56}\selectfont}
    \posttitle{\par\end{center}\vspace{2em}}
    
    % Personaliza el formato del autor
    \preauthor{\begin{center}\Large}
    \postauthor{\par\end{center}\vspace{3em}}
    
    % Personaliza el formato de la fecha
    \predate{\begin{center}\huge}
    \postdate{\par\end{center}\vspace{5em}}
    
    \title{#2}
    \author{\href{https://losdeldgiim.github.io/}{Los Del DGIIM}}
    \date{Granada, #7}
    \thispagestyle{empty}               % Sin encabezado ni pie de página
    \maketitle
    \vfill
}
    \portadaExamen{ffccA4.jpg}{Modelos \\Matemáticos I\\Examen III}{Modelos Matemáticos I. Examen III}{MidnightBlue}{-8}{28}{2025}{David Muñoz Gómez}

    \begin{description}
        \item[Asignatura] Modelos Matemáticos I.
        \item[Curso Académico] 2024-25.
        \item[Grado] Doble Grado en Ingeniería Informática y Matemáticas.
        \item[Grupo] Único.
        \item[Profesor] Maria José Cáceres Granados.
        \item[Descripción] Prueba 1. Temas 0,1 y parte del 2.
        \item[Fecha] 25 de abril de 2025.
        \item[Duración] 2 horas.
    
    \end{description}
    \newpage


    % ------------------------------------
    
    \begin{ejercicio}[3 puntos]
        Dada una sucesión $\{a_n\}_{n\geq 0}$ con $0 \neq a_n\in \mathbb{R}$ para todo $n \in \mathbb{N}_0$. Se considera la ecuación en diferencias:
        \begin{equation}\label{eq:1}
        	x_{n+1}=a_n x_n
        \end{equation}
        
		\begin{enumerate}
        \item Dado $x_0 \in \mathbb{R}$ determina $x_n$ en función de la sucesión $\{a_n\}_{n\geq 0}$ para que $\{x_n\}_{n\geq 0}$ sea solución de la ecuación.
        
        Dado $x_0\in\mathbb{R}$ si $\{x_n\}_{n\geq 0}$ es solución de la ecuación, entonces verifica:
        \begin{equation*}
            x_{n+1}=a_n x_n \;\; \forall n \in\mathbb{N}_0  
        \end{equation*}
        
        Por tanto, podemos intuir que:
        \begin{align*}
        x_1&=a_0 x_0\\
        x_2&=a_1 x_1 = a_1 a_0 x_0\\
        &\Downarrow\\
        x_n&= \prod_{k=0}^{n-1} a_k  x_0
        \end{align*}
        
        Lo demostramos por inducción:
        \begin{itemize}
        \item $n=1$\\
        \[x_1 =a_0 x_0 = \prod_{k=0}^{0} a_k  x_0\]
        \item Supuesto cierto para $n$, veamos para $n+1$
        \[x_{n+1} = a_n x_n = a_n \prod_{k=0}^{n-1} a_k x_0 = \prod_{k=0}^{n} a_k  x_0\hspace{3cm}\hfill\blacksquare\]
        \end{itemize}
        
        Por tanto la solución es:
        \[x_n=\prod_{k=0}^{n-1}a_k x_0\]
        
        \item ¿Qué debe verificar la sucesión $\{a_n\}_{n\geq 0}$ para que la Ecuación~\ref{eq:1}  admita soluciones constantes no triviales? ¿Cuáles son las soluciones constantes en este caso?\\\\
        Si buscamos las soluciones constantes de la ecuación Ecuación~\ref{eq:1} vemos que $c = a_n c \Rightarrow c = 0$ (Pero esta solución es trivial) o bien $a_n = 1 \;\forall n \in \mathbb{N}$. En este caso la ecuación en diferencias pasa a ser:
        \[x_{n+1}=x_n\]
        donde $\{x_n\} = c \;\;\forall n \in\mathbb{N}_0$ es solución constante $\forall c \in \mathbb{R}$.
        
        \item Encuentra el término general de $\{x_n\}_{n\geq 0}$ en el caso de que $a_n=\frac{1}{2^n}$\\
        \item 
        En este caso la ecuación en diferencias pasa a ser: $x_{n+1}=\frac{1}{2^n}x_n$
        
        Como en el apartado a hemos encontrado un término general en función de $\{a_n\}$ podemos aprovecharlo para este apartado, así sabemos que:
        
        \[x_n = \prod_{k=0}^{n-1} \frac{1}{2^k}x_0\]
        
        Es el término general de nuestra ecuación en diferencias, sin embargo para que el ejercicio esté enteramente bien debemos desarrollar la fórmula para encontrar una algo más simple.
        
        \begin{align*}
        x_n &= \prod_{k=0}^{n-1} \frac{1}{2^k}x_0\\
        &\Downarrow\\
        x_n &= \frac{1}{2^0}\times\frac{1}{2^1}\times\frac{1}{2^2}\dots\frac{1}{2^{n-1}} x_0\\
        x_n &= \frac{1}{2^{\sum_{k=0}^{n-1}k}}x_0\\
        &\Downarrow (*)\\
        x_n &= \frac{1}{2^{\frac{n(n-1)}{2}}}x_0 = \frac{1}{\sqrt{2^{n(n-1)}}}x_0
        \end{align*}
        
        Donde en $(*)$ se ha utilizado la fórmula de la suma de los n primeros naturales.
        \end{enumerate}
    \end{ejercicio}

    \begin{ejercicio}[3 puntos]
    Una empresa de pesticidas modela sus niveles de contaminación mediante la siguiente ecuación en diferencias:
    \begin{equation}\label{eq:2}
    x_{n+1}=\frac{x_n}{2} + b_n
    \end{equation}
    Donde $x_n$ representa el nivel de contaminación en el periodo $n$ y $\{b_n\}_{n\geq 0}$ es una sucesión de números reales.\\\\
    Se sabe que $\left\lbrace\left(1 + \frac{1}{n+1}\right)^{n+1}\right\rbrace_{n\geq 0}$ es una solución particular de la Ecuación~\ref{eq:2}.
    \begin{enumerate}
    \item Determina una solución de la Ecuación~\ref{eq:2} que cumple que $x_0=3$.\\
    
    En primer lugar, la solución qu enos otorgan cuando $n=0$ toma el valor 2, por tanto no nos sirve.
    
    Sabemos que todas las soluciones de la Ecuación~\ref{eq:2} se obtienen como suma de una solución de la parte homogénea y una solución particular.
    
    La parte homogénea tiene por solución:
    \begin{align*}
    h_{n+1}&=\frac{h_n}{2}\\
    &\Downarrow\\
    h_n &= \left(\frac{1}{2}\right)^n h_0
    \end{align*}
    
    Así podemos afirmar que toda solución de la Ecuación~\ref{eq:2} es de la forma:
    \[x_n = h_0\left(\frac{1}{2}\right)^n + \left(1 + \frac{1}{n+1}\right)^{n+1}\]
    
    Si imponemos que $x_0=3 \Rightarrow 3 = h_0 + 2 \Rightarrow h_0=1$\\
    Por tanto la solución que buscamos es:
    \[x_n = \left(\frac{1}{2}\right)^n + \left(1 + \frac{1}{n+1}\right)^{n+1}\]
    \item ¿Qué nivel de contaminación tendrá a largo plazo la empresa si en el periodo 0 el nivel de contaminación fue 3?
    
    Nos preguntan por el comportamiento asintótico de $x_n$ cuando $x_0=3$. Del ejercicio anterior sabemos que:
    \[x_n = \left(\frac{1}{2}\right)^n + \left(1 + \frac{1}{n+1}\right)^{n+1}\]
    
    Tomando límites:
    \[\lim_{n\to\infty}x_n=\lim_{n\to\infty}\left(\frac{1}{2}\right)^n + \left(1 + \frac{1}{n+1}\right)^{n+1} = e\]
    
    Por tanto se espera que a largo plazo el nivel de contaminación de la empresa sea $e$.
    
    \item Según este modelo. ¿Podría encontrarse un nivel de contaminación inicial para que a largo plazo el nivel de contaminación sea inferior a 1.8? Razona tu respuesta.
    
    Nos preguntan por un valor de $x_0$ para que la contaminación a largo plazo calculada anteriormente sea inferior a 1.8.
    
    Si cogemos la solución genérica de la ecuación:
    \[x_n = h_0\left(\frac{1}{2}\right)^n + \left(1 + \frac{1}{n+1}\right)^{n+1}\]
    
	Como $h_0$ queda determinado por $x_0$ tomamos límite aquí para ver si encontramos alguna condición. Sin embargo:
	\[\lim_{n\to\infty}\left(h_0\left(\frac{1}{2}\right)^n + \left(1 + \frac{1}{n+1}\right)^{n+1}\right) = e\]
	
	Como $h_0$ está multiplicado por $\frac{1}{2^n}\to 0$ no interviene en el valor a largo plazo. Por tanto da igual el valor de $x_0$ (que nos determina el valor de $h_0$) el valor de la contaminación a largo plazo siempre es $e$.
    \end{enumerate}
    \end{ejercicio}
    \begin{ejercicio}[4 puntos] Se considera la ecuación en diferencias real
    \begin{equation}\label{eq:3}
    x_{n+1}=\sqrt{2x_n^2-2}
    \end{equation}
    \begin{enumerate}
    \item Determina las soluciones constantes de la Ecuación~\ref{eq:3} y estudia su estabilidad.
    
    Notemos que la Ecuación~\ref{eq:3} viene dada por $x_{n+1}=f(x_n)$ donde:
    \Func{f}{\mathbb{R}\backslash\left]-1,1\right[}{\mathbb{R}^+_0}{x}{\sqrt{2x^2-2}}
    
    Buscamos un $c\in\mathbb{R}\backslash]-1,1[$ tal que:
    \begin{align*}
    c &= \sqrt{2c^2-2} \Rightarrow c^2 = 2c^2-2\\
    c &= \pm\sqrt{2}
    \end{align*}
    
    Sin embargo, como hemos visto, la imagen de $f$ es $\mathbb{R}^+_0$ y por tanto tomando $x_0=-\sqrt{2}$ tenemos que $x_1 = \sqrt{2}$ y por tanto no es solución constante.
    Por tanto nuestra única solución constante es $x_0=\sqrt{2}$
    
    Para estudiar su estabilidad podemos calcular $f'(x)$ para aplicar el criterio de la primera derivada:
	\[f'(x)=\frac{4x}{2\sqrt{2x^2-2}}\]
	Por lo que $f'(\sqrt{2})=2 > 1$ lo que nos indica que la solución constante $x_0=\sqrt{2}$ es inestable.
	\item Para la Ecuación~\ref{eq:3} ¿Existen ciclos no triviales? Razona tu respuesta.
	
	Hagamos unas observaciones sobre la función $f$.
	
	Para empezar, dado $x_0\leqslant-1 \Rightarrow x_1=f(x_0)>0$ Y como $Im(f)=\mathbb{R}^+_0$ Necesariamente $x_n\geq\;\;\forall n\in\mathbb{N}: n\geq 1$ 
	
	Por tanto los $x_0\leqslant-1$ no pueden formar n-ciclos de ningún tipo.
	
	Ahora bien, si $x_0 \geq 1$ el razonamiento es diferente. Ahora usaremos que $f$ es una función creciente ya que es composición de crecientes. Por tanto:
	\begin{itemize}
	\item Si $f(x_0) < x_0$
	Estaríamos diciendo que $x_1 < x_0$ y por tanto $f(x_1) < f(x_0) \Rightarrow x_2 < x_1$. Inductivamente la sucesión $x_n$ es decreciente y por tanto no admite ciclos de ningún tipo.
	\item Si $f(x_0) > x_0$
	Sucede lo mismo pero en este caso la sucesión $x_n$ es creciente y tampoco admite ciclos de ningún tipo.
	\end{itemize}
	
	Por tanto no existen ciclos distintos de los triviales para la Ecuación~\ref{eq:3}
	\item Dada una solución $\{x_n\}_{n\geq 0}$ de la Ecuación~\ref{eq:3} considera el cambio de variable $y_n := x_n^2$. Prueba que la sucesión $\{y_n\}_{n\geq 0}$ verifica una ecuación lineal de primer orden y resuélvela.
	
	Realizamos el cambio de variable:
	\begin{align*}
	y_{n+1}=x_{n+1}^2&=2x_n^2-2=2y_n-2\\
	y_{n+1} &= 2y_n-2
	\end{align*}
	
	Para resolverla buscamos una solución particular, en este caso constante nos sirve y la solución de su parte homogénea.
	\[c=2c-2 \iff c=2\]
	
	Por tanto la solución de $y_n$ es:
	\[y_n=2^nc_0 + 2\]
	
	Para dejarla en términos de $y_0$ calculamos $c_0$ en función suya:
	\[y_0=c_0+2\iff c_0=y_0-2\]
	
	Y tenemos que la solución es:
	\[y_n=2^n(y_0-2) + 2\]
	
	\item Utilizando el apartado anterior, demuestra que si $x_0\in (-\infty,-\sqrt{2})\cup(\sqrt{2},\infty)$. Entonces la solución de la Ecuación~\ref{eq:3} con ese dato inicial cumple que $x_n > \sqrt{2}$ para todo $n\geq 1$
	
	Dado $x_0\in (-\infty,-\sqrt{2})\cup(\sqrt{2},\infty) \Rightarrow x_0^2=y_0\in(2,\infty)$ Por tanto veamos que $y_n>2\;\; \forall n\geq 1$ Y por tanto $x_n>\sqrt{2}\;\;\forall n\geq 1$
	
	Como $y_0 > 2 \Rightarrow 2^n(y_0-2) > 0 \Rightarrow 2^n(y_0-2)+2 > 2$
	
	Si tenemos que $y_n > 2 \Rightarrow |x_n| > \sqrt{2}\;\; \forall n\geq 1$ Pero realmente este valor absoluto es innecesario pues con $n\geq 1$ necesariamente $x_n=f(x_{n-1}) > 0$ y por tanto se tiene lo que queríamos demostrar.\hfill $\blacksquare$
	\end{enumerate}
    \end{ejercicio}
\end{document}