\documentclass[12pt]{book}
% Idioma y codificación
\usepackage[spanish, es-tabla]{babel}       %es-tabla para que se titule "Tabla"
\usepackage[utf8]{inputenc}

% Márgenes
\usepackage[a4paper,top=3cm,bottom=2.5cm,left=3cm,right=3cm]{geometry}

% Comentarios de bloque
\usepackage{verbatim}

% Paquetes de links
\usepackage[hidelinks]{hyperref}    % Permite enlaces
\usepackage{url}                    % redirecciona a la web

% Más opciones para enumeraciones
\usepackage{enumitem}

% Personalizar la portada
\usepackage{titling}

% Paquetes de tablas
\usepackage{multirow}


%------------------------------------------------------------------------

%Paquetes de figuras
\usepackage{caption}
\usepackage{subcaption} % Figuras al lado de otras
\usepackage{float}      % Poner figuras en el sitio indicado H.


% Paquetes de imágenes
\usepackage{graphicx}       % Paquete para añadir imágenes
\usepackage{transparent}    % Para manejar la opacidad de las figuras

% Paquete para usar colores
\usepackage[dvipsnames]{xcolor}
\usepackage{pagecolor}      % Para cambiar el color de la página

% Habilita tamaños de fuente mayores
\usepackage{fix-cm}

% Para los gráficos
\usepackage{tikz}

% Para poder situar los nodos en los grafos
\usetikzlibrary{positioning}


%------------------------------------------------------------------------

% Paquetes de matemáticas
\usepackage{mathtools, amsfonts, amssymb, mathrsfs}
\usepackage[makeroom]{cancel}     % Simplificar tachando
\usepackage{polynom}    % Divisiones y Ruffini
\usepackage{units} % Para poner fracciones diagonales con \nicefrac

\usepackage{pgfplots}   %Representar funciones
\pgfplotsset{compat=1.18}  % Versión 1.18

\usepackage{tikz-cd}    % Para usar diagramas de composiciones
\usetikzlibrary{calc}   % Para usar cálculo de coordenadas en tikz

%Definición de teoremas, etc.
\usepackage{amsthm}
%\swapnumbers   % Intercambia la posición del texto y de la numeración

\theoremstyle{plain}

\makeatletter
\@ifclassloaded{article}{
  \newtheorem{teo}{Teorema}[section]
}{
  \newtheorem{teo}{Teorema}[chapter]  % Se resetea en cada chapter
}
\makeatother

\newtheorem{coro}{Corolario}[teo]           % Se resetea en cada teorema
\newtheorem{prop}[teo]{Proposición}         % Usa el mismo contador que teorema
\newtheorem{lema}[teo]{Lema}                % Usa el mismo contador que teorema

\theoremstyle{remark}
\newtheorem*{observacion}{Observación}

\theoremstyle{definition}

\makeatletter
\@ifclassloaded{article}{
  \newtheorem{definicion}{Definición} [section]     % Se resetea en cada chapter
}{
  \newtheorem{definicion}{Definición} [chapter]     % Se resetea en cada chapter
}
\makeatother

\newtheorem*{notacion}{Notación}
\newtheorem*{ejemplo}{Ejemplo}
\newtheorem*{ejercicio*}{Ejercicio}             % No numerado
\newtheorem{ejercicio}{Ejercicio} [section]     % Se resetea en cada section


% Modificar el formato de la numeración del teorema "ejercicio"
\renewcommand{\theejercicio}{%
  \ifnum\value{section}=0 % Si no se ha iniciado ninguna sección
    \arabic{ejercicio}% Solo mostrar el número de ejercicio
  \else
    \thesection.\arabic{ejercicio}% Mostrar número de sección y número de ejercicio
  \fi
}


% \renewcommand\qedsymbol{$\blacksquare$}         % Cambiar símbolo QED
%------------------------------------------------------------------------

% Paquetes para encabezados
\usepackage{fancyhdr}
\pagestyle{fancy}
\fancyhf{}

\newcommand{\helv}{ % Modificación tamaño de letra
\fontfamily{}\fontsize{12}{12}\selectfont}
\setlength{\headheight}{15pt} % Amplía el tamaño del índice


%\usepackage{lastpage}   % Referenciar última pag   \pageref{LastPage}
\fancyfoot[C]{\thepage}

%------------------------------------------------------------------------

% Conseguir que no ponga "Capítulo 1". Sino solo "1."
\makeatletter
\@ifclassloaded{book}{
  \renewcommand{\chaptermark}[1]{\markboth{\thechapter.\ #1}{}} % En el encabezado
    
  \renewcommand{\@makechapterhead}[1]{%
  \vspace*{50\p@}%
  {\parindent \z@ \raggedright \normalfont
    \ifnum \c@secnumdepth >\m@ne
      \huge\bfseries \thechapter.\hspace{1em}\ignorespaces
    \fi
    \interlinepenalty\@M
    \Huge \bfseries #1\par\nobreak
    \vskip 40\p@
  }}
}
\makeatother

%------------------------------------------------------------------------
% Paquetes de cógido
\usepackage{minted}
\renewcommand\listingscaption{Código fuente}

\usepackage{fancyvrb}
% Personaliza el tamaño de los números de línea
\renewcommand{\theFancyVerbLine}{\small\arabic{FancyVerbLine}}

% Estilo para C++
\newminted{cpp}{
    frame=lines,
    framesep=2mm,
    baselinestretch=1.2,
    linenos,
    escapeinside=||
}

% para minted
\definecolor{LightGray}{rgb}{0.95,0.95,0.92}
\setminted{
    linenos=true,
    stepnumber=5,
    numberfirstline=true,
    autogobble,
    breaklines=true,
    breakautoindent=true,
    breaksymbolleft=,
    breaksymbolright=,
    breaksymbolindentleft=0pt,
    breaksymbolindentright=0pt,
    breaksymbolsepleft=0pt,
    breaksymbolsepright=0pt,
    fontsize=\footnotesize,
    bgcolor=LightGray,
    numbersep=10pt
}


\usepackage{listings} % Para incluir código desde un archivo

\renewcommand\lstlistingname{Código Fuente}
\renewcommand\lstlistlistingname{Índice de Códigos Fuente}

% Definir colores
\definecolor{vscodepurple}{rgb}{0.5,0,0.5}
\definecolor{vscodeblue}{rgb}{0,0,0.8}
\definecolor{vscodegreen}{rgb}{0,0.5,0}
\definecolor{vscodegray}{rgb}{0.5,0.5,0.5}
\definecolor{vscodebackground}{rgb}{0.97,0.97,0.97}
\definecolor{vscodelightgray}{rgb}{0.9,0.9,0.9}

% Configuración para el estilo de C similar a VSCode
\lstdefinestyle{vscode_C}{
  backgroundcolor=\color{vscodebackground},
  commentstyle=\color{vscodegreen},
  keywordstyle=\color{vscodeblue},
  numberstyle=\tiny\color{vscodegray},
  stringstyle=\color{vscodepurple},
  basicstyle=\scriptsize\ttfamily,
  breakatwhitespace=false,
  breaklines=true,
  captionpos=b,
  keepspaces=true,
  numbers=left,
  numbersep=5pt,
  showspaces=false,
  showstringspaces=false,
  showtabs=false,
  tabsize=2,
  frame=tb,
  framerule=0pt,
  aboveskip=10pt,
  belowskip=10pt,
  xleftmargin=10pt,
  xrightmargin=10pt,
  framexleftmargin=10pt,
  framexrightmargin=10pt,
  framesep=0pt,
  rulecolor=\color{vscodelightgray},
  backgroundcolor=\color{vscodebackground},
}

%------------------------------------------------------------------------

% Comandos definidos
\newcommand{\bb}[1]{\mathbb{#1}}
\newcommand{\cc}[1]{\mathcal{#1}}

% I prefer the slanted \leq
\let\oldleq\leq % save them in case they're every wanted
\let\oldgeq\geq
\renewcommand{\leq}{\leqslant}
\renewcommand{\geq}{\geqslant}

% Si y solo si
\newcommand{\sii}{\iff}

% Letras griegas
\newcommand{\eps}{\epsilon}
\newcommand{\veps}{\varepsilon}
\newcommand{\lm}{\lambda}

\newcommand{\ol}{\overline}
\newcommand{\ul}{\underline}
\newcommand{\wt}{\widetilde}
\newcommand{\wh}{\widehat}

\let\oldvec\vec
\renewcommand{\vec}{\overrightarrow}

% Derivadas parciales
\newcommand{\del}[2]{\frac{\partial #1}{\partial #2}}
\newcommand{\Del}[3]{\frac{\partial^{#1} #2}{\partial #3^{#1}}}
\newcommand{\deld}[2]{\dfrac{\partial #1}{\partial #2}}
\newcommand{\Deld}[3]{\dfrac{\partial^{#1} #2}{\partial #3^{#1}}}


\newcommand{\AstIg}{\stackrel{(\ast)}{=}}
\newcommand{\Hop}{\stackrel{L'H\hat{o}pital}{=}}

\newcommand{\red}[1]{{\color{red}#1}} % Para integrales, destacar los cambios.

% Método de integración
\newcommand{\MetInt}[2]{
    \left[\begin{array}{c}
        #1 \\ #2
    \end{array}\right]
}

% Declarar aplicaciones
% 1. Nombre aplicación
% 2. Dominio
% 3. Codominio
% 4. Variable
% 5. Imagen de la variable
\newcommand{\Func}[5]{
    \begin{equation*}
        \begin{array}{rrll}
            #1:& #2 & \longrightarrow & #3\\
               & #4 & \longmapsto & #5
        \end{array}
    \end{equation*}
}

%------------------------------------------------------------------------


\usepackage{hhline}
\newcommand{\cell}[1]{\multicolumn{1}{|c|}{$#1$}}

% Para poder incluir árboles
\usepackage{forest}
\usepackage{booktabs}

% Para poder añadir autómatas
% https://www3.nd.edu/~kogge/courses/cse30151-fa17/Public/other/tikz_tutorial.pdf
\usetikzlibrary{automata} %, positioning, arrows}
\tikzset{
    -Stealth,
    node distance=3cm, % specifies the minimum distance between two nodes. Change if necessary.
    every state/.style={thick, fill=gray!10, shape=ellipse}, % sets the properties for each ’state’ node
    initial text=$ $, % sets the text that appears on the start arrow
    % Un tipo de nodo, que es error, que lo pone rojo
    error/.style={thick, fill=red!20},
}

\begin{document}

    % 1. Foto de fondo
    % 2. Título
    % 3. Encabezado Izquierdo
    % 4. Color de fondo
    % 5. Coord x del titulo
    % 6. Coord y del titulo
    % 7. Fecha
    % 8. Autor

    
    % 1. Foto de fondo
% 2. Título
% 3. Encabezado Izquierdo
% 4. Color de fondo
% 5. Coord x del titulo
% 6. Coord y del titulo
% 7. Fecha

\newcommand{\portada}[7]{

    \portadaBase{#1}{#2}{#3}{#4}{#5}{#6}{#7}
    \portadaBook{#1}{#2}{#3}{#4}{#5}{#6}{#7}
}

\newcommand{\portadaExamen}[7]{

    \portadaBase{#1}{#2}{#3}{#4}{#5}{#6}{#7}
    \portadaArticle{#1}{#2}{#3}{#4}{#5}{#6}{#7}
}




\newcommand{\portadaBase}[7]{

    % Tiene la portada principal y la licencia Creative Commons
    
    % 1. Foto de fondo
    % 2. Título
    % 3. Encabezado Izquierdo
    % 4. Color de fondo
    % 5. Coord x del titulo
    % 6. Coord y del titulo
    % 7. Fecha
    
    
    \thispagestyle{empty}               % Sin encabezado ni pie de página
    \newgeometry{margin=0cm}        % Márgenes nulos para la primera página
    
    
    % Encabezado
    \fancyhead[L]{\helv #3}
    \fancyhead[R]{\helv \nouppercase{\leftmark}}
    
    
    \pagecolor{#4}        % Color de fondo para la portada
    
    \begin{figure}[p]
        \centering
        \transparent{0.3}           % Opacidad del 30% para la imagen
        
        \includegraphics[width=\paperwidth, keepaspectratio]{assets/#1}
    
        \begin{tikzpicture}[remember picture, overlay]
            \node[anchor=north west, text=white, opacity=1, font=\fontsize{60}{90}\selectfont\bfseries\sffamily, align=left] at (#5, #6) {#2};
            
            \node[anchor=south east, text=white, opacity=1, font=\fontsize{12}{18}\selectfont\sffamily, align=right] at (9.7, 3) {\textbf{\href{https://losdeldgiim.github.io/}{Los Del DGIIM}}};
            
            \node[anchor=south east, text=white, opacity=1, font=\fontsize{12}{15}\selectfont\sffamily, align=right] at (9.7, 1.8) {Doble Grado en Ingeniería Informática y Matemáticas\\Universidad de Granada};
        \end{tikzpicture}
    \end{figure}
    
    
    \restoregeometry        % Restaurar márgenes normales para las páginas subsiguientes
    \pagecolor{white}       % Restaurar el color de página
    
    
    \newpage
    \thispagestyle{empty}               % Sin encabezado ni pie de página
    \begin{tikzpicture}[remember picture, overlay]
        \node[anchor=south west, inner sep=3cm] at (current page.south west) {
            \begin{minipage}{0.5\paperwidth}
                \href{https://creativecommons.org/licenses/by-nc-nd/4.0/}{
                    \includegraphics[height=2cm]{assets/Licencia.png}
                }\vspace{1cm}\\
                Esta obra está bajo una
                \href{https://creativecommons.org/licenses/by-nc-nd/4.0/}{
                    Licencia Creative Commons Atribución-NoComercial-SinDerivadas 4.0 Internacional (CC BY-NC-ND 4.0).
                }\\
    
                Eres libre de compartir y redistribuir el contenido de esta obra en cualquier medio o formato, siempre y cuando des el crédito adecuado a los autores originales y no persigas fines comerciales. 
            \end{minipage}
        };
    \end{tikzpicture}
    
    
    
    % 1. Foto de fondo
    % 2. Título
    % 3. Encabezado Izquierdo
    % 4. Color de fondo
    % 5. Coord x del titulo
    % 6. Coord y del titulo
    % 7. Fecha


}


\newcommand{\portadaBook}[7]{

    % 1. Foto de fondo
    % 2. Título
    % 3. Encabezado Izquierdo
    % 4. Color de fondo
    % 5. Coord x del titulo
    % 6. Coord y del titulo
    % 7. Fecha

    % Personaliza el formato del título
    \pretitle{\begin{center}\bfseries\fontsize{42}{56}\selectfont}
    \posttitle{\par\end{center}\vspace{2em}}
    
    % Personaliza el formato del autor
    \preauthor{\begin{center}\Large}
    \postauthor{\par\end{center}\vfill}
    
    % Personaliza el formato de la fecha
    \predate{\begin{center}\huge}
    \postdate{\par\end{center}\vspace{2em}}
    
    \title{#2}
    \author{\href{https://losdeldgiim.github.io/}{Los Del DGIIM}}
    \date{Granada, #7}
    \maketitle
    
    \tableofcontents
}




\newcommand{\portadaArticle}[7]{

    % 1. Foto de fondo
    % 2. Título
    % 3. Encabezado Izquierdo
    % 4. Color de fondo
    % 5. Coord x del titulo
    % 6. Coord y del titulo
    % 7. Fecha

    % Personaliza el formato del título
    \pretitle{\begin{center}\bfseries\fontsize{42}{56}\selectfont}
    \posttitle{\par\end{center}\vspace{2em}}
    
    % Personaliza el formato del autor
    \preauthor{\begin{center}\Large}
    \postauthor{\par\end{center}\vspace{3em}}
    
    % Personaliza el formato de la fecha
    \predate{\begin{center}\huge}
    \postdate{\par\end{center}\vspace{5em}}
    
    \title{#2}
    \author{\href{https://losdeldgiim.github.io/}{Los Del DGIIM}}
    \date{Granada, #7}
    \thispagestyle{empty}               % Sin encabezado ni pie de página
    \maketitle
    \vfill
}
    \portada{etsiitA4.jpg}{Modelos de\\Computación}{Modelos de Computación}{MidnightBlue}{-8}{27}{2024-2025}{Arturo Olivares Martos}

    %\chapter{Introducción a los fundamentos de redes}
\subsection{Objetivos}

\subsection{Historia}
Un servicio de banda ancha es un servicio de velocidad grande, que se inición a partir del 2000. Comenzaron por 2Mbps.
El ADSL en España comenzó transmitiendo 256kbps (el máximo teórico del ADSL son 20Mbps, aunque lo normal son 10 o 12).
Las redes de cable eran HFC (redes de cable y fibras híbridas), pero ahora tenemos FTTH (Fiber to the home), fibra directa a casa. La fibra es el mejor material para transmitir información del mundo, además de que no cuenta con interferencias, consiguiendo varios Gbps.

Estos servicios eran usados por:
\begin{itemize}
    \item Televisiones (contaban con una resolución de $640\times 240$, necesitando un servicio de 2Mbps sin comprimir).
\end{itemize}

Cerca del 70\% del tráfico de internet es debido al multimedia, a través de las CDNs, redes de transmisión de contenidos.
Por ejemplo, un vídeo de Youtube cuenta con varias copias del mismo alrededor del mundo.

\section{Sistemas de comunicación y redes}
El sistema de comunicación típico es:

En un sistema de comunicaciones contamos con una fuente y con un transmisor (ambos en el mismo equipo), de forma que la fuente genera datos.
Después del transmisor, contamos con un canal de comunicación, el cual proboca errores:
\begin{itemize}
    \item Ruidos.
    \item Interferencias.
    \item Diafonías: sucede mucho en ADSL, al tener muchos cables en paralelo juntos puede suceder que la información de un cable se meta en otro.
\end{itemize}

En el final del destino, conamos con un equipo que cuenta con un receptor y con el destino (que espera los datos a recibir).


Cuando hablamos de redes, tenemos que tener varios equipos interconectados, que funcionen de forma autónoma (sin interferencia de nadie) y que se realice de forma eficaz.

\subsection{Primera red de comunicaciones}

La primera red de comunicaciones era la red de telefonía móvil.

Contábamos con nuestra línea de teléfono, que conectaba con una central de conmutación local, luego regional y luego nacional, la cual debía conectar con la central local a la que queríamos llamar.
Se usaba la conmutación de circuitos: 
\begin{itemize}
    \item Inicialmente se creaba un camino físico juntando cables. A dicho camino se le llamaba circuito.

        Era ineficiente porque dicho cable cuenta con una eficiencia del 50\%, debido a que aproximadamente se habla la mitad del tiempo de la llamada.

        Era un problema de seguridad el mal funcionamiento de una central, ya que dejaba sin servicio a miles de teléfonos.
\end{itemize}

Si ahora cambiamos los teléfonos por ordenadores y las centrales de conmutación por routerse, contamos con muchísimos caminos para conectar dos ordenadores, haciendo mucho más segura la red (a expensas de la seguridad en la red).

Ahora ya no tenemos un circuito físico, sino que son los routers quienes deciden a dónde enviar los paquetes y en qué momento hacerlo. Con el inconveniente de generar retardo pero con la ventaja de usar mejor el canal (si hay silencios, puede usarlo otro).

El departamento de defensa americana y posteriormente la NSF crearon las primeras redes asemejables a internet.

De una red esperamos:
\begin{itemize}
    \item Autonomía.
    \item Interconexión.
    \item Eficiencia.
\end{itemize}

Una red clásica va a tener equipos terminales (hosts) y equipos de interconexión, que permiten conectar toda la red.

\subsection{Líneas de transmisión}
Podemos contar con enlaces inalámbricos y cableados.

Comenzó con los enlaces cableados con cables de pares (pensado para transmitir 4kHz, la media en la voz humana), luego con cables coaxiales y fibra óptica.
Este último es el mejor medio guiado existente.



    \fancyhead[R]{\helv \nouppercase{\rightmark}}
    \chapter{Relaciones de Problemas}
    %\section{Introducción a la Computación}


\begin{ejercicio}
    Sea la gramática $G=\left(V,T,P,S\right)$ dada por:
    \begin{align*}
        V &= \{S, X, Y\} \\
        T &= \{a,b\} \\
        S &= S
    \end{align*}
    \begin{enumerate}
        \item Describe el lenguaje generado por la gramática teniendo en cuenta que $P$ viene descrito por:
        \begin{align*}
            S &\rightarrow XYX \\
            X &\rightarrow aX \mid bX \mid \veps \\
            Y &\rightarrow bbb
        \end{align*}

        Sea $L=\{ubbbv\mid u,v\in\{a,b\}^\ast\}$. Demostraremos mediante doble inclusión que $L=\cc{L}(G)$.
        \begin{description}
            \item[$\subset)$] Sea $w\in L$. Entonces, $w=ubbbv$ con $u,v\in\{a,b\}^\ast$. Veamos que
            $S \stackrel{\ast}{\Longrightarrow} w$:
            \begin{equation*}
                S \Longrightarrow XYX \Longrightarrow XbbbX
            \end{equation*}

            Además, es fácil ver que la regla de producción $X\rightarrow aX \mid bX \mid \veps$ nos permite generar cualquier palabra $u\in\{a,b\}^\ast$. Por tanto, tenemos que $X \stackrel{\ast}{\Longrightarrow} u$ y $X \stackrel{\ast}{\Longrightarrow} v$; teniendo así que $S \stackrel{\ast}{\Longrightarrow} ubbbv$.

            \item[$\supset)$] Sea $w\in\cc{L}(G)$. Veamos la forma de $w$:
            \begin{equation*}
                S \Longrightarrow XYX \Longrightarrow XbbbX \Longrightarrow ubbbv \mid u,v\in\{a,b\}^\ast
            \end{equation*}
            donde en el último paso hemos empleado lo visto en el apartado anterior de la regla de producción $X\rightarrow aX \mid bX \mid \veps$. Por tanto, $w\in L$.
        \end{description}


        \item Describe el lenguaje generado por la gramática teniendo en cuenta que $P$ viene descrito por:
        \begin{align*}
            S &\rightarrow aX \\
            X &\rightarrow aX \mid bX \mid \veps
        \end{align*}

        Sea $L=\{au \mid u\in\{a,b\}^\ast\}$. Demostraremos mediante doble inclusión que $L=\cc{L}(G)$.
        \begin{description}
            \item[$\subset)$] Sea $w\in L$. Entonces, $w=au$ con $u\in\{a,b\}^\ast$. Veamos que
            $S \stackrel{\ast}{\Longrightarrow} w$:
            \begin{equation*}
                S \Longrightarrow aX \Longrightarrow au
            \end{equation*}
            donde en el último paso hemos empleado lo visto respecto a la regla de producción $X\rightarrow aX \mid bX \mid \veps$. Por tanto, $w\in \cc{L}(G)$.

            \item[$\supset)$] Sea $w\in\cc{L}(G)$. Veamos la forma de $w$:
            \begin{equation*}
                S \Longrightarrow aX \Longrightarrow au \mid u\in\{a,b\}^\ast
            \end{equation*}
            donde en el último paso hemos empleado lo visto respecto a la regla de producción $X\rightarrow aX \mid bX \mid \veps$. Por tanto, $w\in L$.
        \end{description}

        \item Describe el lenguaje generado por la gramática teniendo en cuenta que $P$ viene descrito por:
        \begin{align*}
            S &\rightarrow XaXaX \\
            X &\rightarrow aX \mid bX \mid \veps
        \end{align*}

        Sea $L=\{uavawa \mid u,v,w\in\{a,b\}^\ast\}$. Demostraremos mediante doble inclusión que $L=\cc{L}(G)$.
        \begin{description}
            \item[$\subset)$] Sea $z\in L$. Entonces, $z=uavawa$ con $u,v,w\in\{a,b\}^\ast$. Veamos que
            $S \stackrel{\ast}{\Longrightarrow} z$:
            \begin{equation*}
                S \Longrightarrow XaXaX \Longrightarrow uavawa
            \end{equation*}
            donde en el último paso hemos empleado lo visto respecto a la regla de producción $X\rightarrow aX \mid bX \mid \veps$. Por tanto, $z\in \cc{L}(G)$.

            \item[$\supset)$] Sea $z\in\cc{L}(G)$. Veamos la forma de $z$:
            \begin{equation*}
                S \Longrightarrow XaXaX \Longrightarrow uavawa \mid u,v,w\in\{a,b\}^\ast
            \end{equation*}
            donde en el último paso hemos empleado lo visto respecto a la regla de producción $X\rightarrow aX \mid bX \mid \veps$. Por tanto, $z\in L$.
        \end{description}

        \item Describe el lenguaje generado por la gramática teniendo en cuenta que $P$ viene descrito por:
        \begin{align*}
            S &\rightarrow SS \mid XaXaX \mid \veps \\
            X &\rightarrow bX \mid \veps
        \end{align*}

        Sea el lenguaje $L=\{b^i a b^j a b^k \mid i,j,k\in \bb{N}\cup \{0\}\}$. Demostraremos mediante doble inclusión que $L^\ast=\cc{L}(G)$.
        \begin{description}
            \item[$\subset)$] Sea $z\in L^\ast=\bigcup\limits_{i\in \bb{N}} L^i$.
            Sea $n$ el menor número natural tal que $z\in L^n$.
            Notando por $n_a(z)$ al número de $a$'s en $z$, tenemos que $n_a(z)=2n$.
            Entonces, $z\in L\cdot \ldots \cdot L$ ($n$ veces), por lo que existen
            $i_1,j_1,k_1,\ldots,i_n,j_n,k_n\in \bb{N}\cup \{0\}$ tales que $z=b^{i_1} a b^{j_1} a b^{k_1} \cdot \ldots \cdot b^{i_n} a b^{j_n} a b^{k_n}$. Veamos que
            $S \stackrel{\ast}{\Longrightarrow} z$:
            \begin{itemize}
                \item Para conseguir el número de $a$'s deseado, empleamos la regla de producción $S \rightarrow SS$ y reemplazamos una de las $S$ por $XaXaX$. Esto lo hacemos $n$ veces.
                \item Posteriormente, cada $X$ la sustituiremos tantas veces como sea necesario por $bX$ para conseguir el número de $b$'s deseado en cada posición, y finalizaremos con $X\rightarrow \veps$.
            \end{itemize}

            \item[$\supset)$] Sea $z\in\cc{L}(G)$, y sea $n_a(z)$ el número de $a$'s en $z$. Entonces, como el número de $a$ siempre aumenta de dos en dos, tenemos que $n_a(z)=2n$ para algún $n\in \bb{N}\cup \{0\}$.
            Veamos la forma de $z$:
            \begin{itemize}
                \item Para llegar a $z$, hemos tenido que emplear la regla de producción $S \rightarrow SS\rightarrow SXaXaX$ $n$ veces. Una vez llegados aquí, para eliminar la $S$ (ya que habremos llegado a $n_a(z)$ $a$'s), empleamos la regla de producción $S\rightarrow \veps$.
                \item Posteriormente, para cada $X$, tan solo podemos emplear la regla de producción $X\rightarrow bX \mid \veps$ para conseguir el número de $b$'s deseado en cada posición.
            \end{itemize}
            Por tanto, es directo ver que $z\in L^n\subseteq L^\ast$.
        \end{description}
    \end{enumerate}
\end{ejercicio}



\begin{ejercicio} \label{ej:1.2}
    Sea la gramática $G=\left(V,T,P,S\right)$. Determinar en cada caso el lenguaje generado por la gramática.
    \begin{enumerate}
        \item Tenga en cuenta que:
        \begin{align*}
            V &= \{S,A\}\\
            T &= \{a,b\}\\
            S &= S \\
            P &= \left\{
                \begin{array}{rcl}
                    S &\rightarrow & abAS \mid a \\
                    abA &\rightarrow & baab \\
                    A &\rightarrow & b
                \end{array}
            \right\}
        \end{align*}

        Sea $L=\{ua \mid u\in \{abb, baab\}^\ast\}$. Demostraremos mediante doble inclusión que $L=\cc{L}(G)$.
        \begin{description}
            \item[$\subset)$] Sea $w\in L$. Entonces, $w=ua$ con $u\in \{abb, baab\}^\ast$. Veamos que
            $S \stackrel{\ast}{\Longrightarrow} w$. Para ello, sabemos que $u\in \{abb, baab\}^\ast=\bigcup\limits_{i\in \bb{N}} \{abb, baab\}^i$.
            Sea $n$ el menor número natural tal que $u\in \{abb, baab\}^n$, es decir, es una concatenación de $n$ subcadenas, cada una de las cuales es o bien $abb$ o bien $baab$. Veamos que $S$ produce ambas subcadenas:
            \begin{itemize}
                \item Para producir $abb$, tenemos que $S\rightarrow abAS \rightarrow abbS$.
                \item Para producir $baab$, tenemos que $S\rightarrow abAS \rightarrow baabS$.
            \end{itemize}
            Como vemos, en cada caso podemos concatenar la subcadena necesaria, pero siempre nos quedará una $S$ al final. Usamos la regla de producción $S\rightarrow a$ para eliminarla, llegando así a $w$, por lo que $S \stackrel{\ast}{\Longrightarrow} w$ y $w\in \cc{L}(G)$.

            \item[$\supset)$] Sea $w\in\cc{L}(G)$. Veamos la forma de $w$, para lo cual hay dos opciones:
            \begin{itemize}
                \item $S\rightarrow a$: En este caso, habremos finalizado la palabra con $a$, por lo que habremos añadido la subcadena $a$ a la palabra al final.
                \item $S \rightarrow abAS$: En este caso, también hay dos opciones:
                \begin{itemize}
                    \item $S \rightarrow abAS \rightarrow baabS$: En este caso, habremos concatenado $baab$ con $S$, por lo que habremos añadido la subcadena $baab$ a la palabra.
                    \item $S \rightarrow abAS \rightarrow abbS$: En este caso, habremos concatenado $abb$ con $S$, por lo que habremos añadido la subcadena $abb$ a la palabra.
                \end{itemize}
            \end{itemize}
            Por tanto, $w$ es de la forma $ua$ con $u$ una concatenación de $abb$'s y $baab$'s, es decir, $u\in\{abb, baab\}^\ast$.
            Por tanto, $w\in L$.
        \end{description}

        \item \label{ej:1.2.b} Tenga en cuenta que:
        \begin{align*}
            V &= \{\langle \text{número} \rangle, \langle \text{dígito} \rangle\} \\
            T &= \{0,1,2,3,4,5,6,7,8,9\} \\
            S &= \langle \text{número} \rangle \\
            P &= \left\{
                \begin{array}{rcl}
                    \langle \text{número} \rangle &\rightarrow & \langle \text{número} \rangle \langle \text{dígito} \rangle \\
                    \langle \text{número} \rangle &\rightarrow & \langle \text{dígito} \rangle \\
                    \langle \text{dígito} \rangle &\rightarrow & 0 \mid 1 \mid 2 \mid 3 \mid 4 \mid 5 \mid 6 \mid 7 \mid 8 \mid 9
                \end{array}
            \right\}
        \end{align*}

        Tenemos que $\cc{L}(G)$ es el conjunto de los números naturales, permitiendo
        tantos ceros a la izquierda como se quiera. Es decir (usando la notación de potencia y concatenación vista para lenguajes):
        \begin{equation*}
            L = \{0^i n \mid i\in \bb{N}\cup\{0\},~n\in \bb{N}\cup \{0\}\}
        \end{equation*}
        Demostrémoslo mediante doble inclusión que $L=\cc{L}(G)$.
        \begin{description}
            \item[$\subset)$] Sea $w\in L$. Entonces, $w=0^i n$ con $i\in \bb{N}\cup\{0\}$ y $n\in \bb{N}\cup \{0\}$. Veamos que
            $\langle \text{número} \rangle \stackrel{\ast}{\Longrightarrow} w$:
            \begin{itemize}
                \item En primer lugar, aplicamos $|w|-1$ veces la regla de producción $\langle \text{número} \rangle \rightarrow \langle \text{número} \rangle \langle \text{dígito} \rangle$ y la regla
                que lleva de $\langle \text{dígito} \rangle$ a uno de los símbolos terminales, consiguiendo así en cada etapa reemplazar
                la última variable presente en la cadena por un dígito.
                \item Finalmente, aplicamos la regla de producción $\langle \text{número} \rangle \rightarrow \langle \text{dígito} \rangle$ para reemplazar la última variable por un dígito, que será el primero del número formado.
            \end{itemize}
            Por tanto, $\langle \text{número} \rangle \stackrel{\ast}{\Longrightarrow} w$, teniendo que $w\in \cc{L}(G)$.

            \item[$\supset)$] Sea $w\in\cc{L}(G)$. Como la única regla que
            aumenta la longitud es la regla de producción $\langle \text{número} \rangle \rightarrow \langle \text{número} \rangle \langle \text{dígito} \rangle$, tenemos que $w$ tiene la forma:
            \begin{align*}
                \langle \text{número} \rangle &\Longrightarrow \langle \text{número} \rangle \langle \text{dígito} \rangle \stackrel{|w|-1\text{ veces}}{\Longrightarrow} \\
                &\Longrightarrow
                \langle \text{número} \rangle \langle \text{dígito} \rangle \langle \text{dígito} \rangle \stackrel{|w|-1\text{ veces}}{\cdots} \langle \text{dígito} \rangle
                \Longrightarrow \\& \Longrightarrow
                \langle \text{dígito} \rangle \stackrel{|w|\text{ veces}}{\cdots} \langle \text{dígito} \rangle
            \end{align*}
            Por tanto, tenemos que se trata una sucesión de $|w|$ dígitos, lo que nos lleva a que $w\in L$.
        \end{description}

        \item Tenga en cuenta que:
        \begin{align*}
            V &= \{A,S\} \\
            T &= \{a,b\} \\
            S &= S \\
            P &= \left\{
                \begin{array}{rcl}
                    S &\rightarrow & aS \mid aA \\
                    A &\rightarrow & bA \mid b
                \end{array}
            \right\}
        \end{align*}

        Sea $L=\{a^nb^m \in \{a,b\}^\ast \mid n, m \in \bb{N}\}$. Demostraremos mediante doble inclusión que $L=\cc{L}(G)$.
        \begin{description}
            \item[$\subset)$] Sea $w\in L$. Entonces, $w=a^nb^m$ con $n,m\in \bb{N}$. Veamos que
            $S \stackrel{\ast}{\Longrightarrow} w$:
            \begin{itemize}
                \item En primer lugar, aplicamos $n-1$ veces la regla de producción $S \rightarrow aS$ para obtener $a^{n-1}S$,
                \begin{equation*}
                    S \stackrel{\ast}{\Longrightarrow} a^{n-1}S
                \end{equation*}

                \item Para cambiar a la etapa de añadir $b$'s, aplicamos la regla de producción $S \rightarrow aA$, obteniendo así $a^{n}A$,
                \item Después, aplicamos $m-1$ veces la regla de producción $A \rightarrow bA$ para obtener $a^nb^{m-1}A$.
                \item Para finalizar, aplicamos la regla de producción $A \rightarrow b$ para obtener $a^nb^m$.
            \end{itemize}
            Por tanto, $S \stackrel{\ast}{\Longrightarrow} w$, teniendo que $w\in \cc{L}(G)$.

            \item[$\supset)$] Sea $w\in\cc{L}(G)$. Vemos que en la palabra siempre
            va a haber tan solo una variable (ya sea $S$ o $A$). Se empezará con la $S$, y en cierto momento se cambiará a la $A$,
            sin poder entonces volver a la $S$.
            \begin{itemize}
                \item Cuando se está en la etapa en la que hay $S$, tan solo se pueden añadir $a$'s,
                o bien cambiar a la $A$.
                \item Cuando se está en la etapa en la que hay $A$, tan solo se pueden añadir $b$'s.
            \end{itemize}
            Por tanto, tenemos que $w$ estará formada por una sucesión de
            $a$'s seguida de una sucesión de $b$'s, lo que nos lleva a que $w\in L$.
        \end{description}
    \end{enumerate}
\end{ejercicio}

\begin{ejercicio}
    Encontrar gramáticas de tipo 2 para los siguientes lenguajes sobre el alfabeto $\{a, b\}$. En cada caso determinar si los lenguajes generados son de tipo 3, estudiando si existe una gramática de tipo 3 que los genera.
    \begin{enumerate}
        \item Palabras en las que el número de $b$ no es tres.
        
        Tenemos varias opciones:
        \begin{itemize}
            \item Que no tenga $b$'s.
            \item Que tenga una $b$.
            \item Que tenga dos $b$'s.
            \item Que tenga $4$ o más $b$'s.
        \end{itemize}

        Sea la gramática $G=\left(V,T,P,S\right)$ dada por:
        \begin{align*}
            V &= \{S, A, X\} \\
            T &= \{a,b\} \\
            S &= S \\
            P &= \left\{
                \begin{array}{rcl}
                    S &\rightarrow & A \mid AbA \mid AbAbA \mid XbXbXbXbX \\
                    A &\rightarrow & aA \mid \veps \\
                    X &\rightarrow & aX \mid bX \mid \veps
                \end{array}
            \right\}
        \end{align*}

        Esta gramática no obstante es de tipo $2$. Busquemos otra que sea de tipo 3.
        Sea la gramática $G'=\left(V',T',P',S'\right)$ dada por:
        \begin{align*}
            V' &= \{S, X,Y,Z, W\} \\
            T' &= \{a,b\} \\
            S' &= S \\
            P' &= \left\{
                \begin{array}{rcl}
                    S &\rightarrow & \veps \mid aS \mid bX \\
                    X &\rightarrow & \veps \mid aX \mid bY \\
                    Y &\rightarrow & \veps \mid aY \mid bZ \\
                    Z &\rightarrow & aZ \mid bW \\
                    W &\rightarrow & \veps \mid aW \mid bW
                \end{array}
            \right\}
        \end{align*}

        Esta sí es de tipo $3$, y genera el lenguaje deseado.



        \item Palabras que tienen 2 ó 3 $b$.
        
        Sea la gramática $G=\left(V,T,P,S\right)$ dada por:
        \begin{align*}
            V &= \{S, A, B\} \\
            T &= \{a,b\} \\
            S &= S \\
            P &= \left\{
                \begin{array}{rcl}
                    S &\rightarrow & AbAbABA \\
                    A &\rightarrow & aA \mid \veps \\
                    B &\rightarrow & b \mid \veps
                \end{array}
            \right\}
        \end{align*}

        Esta gramática no obstante es de tipo $2$. Busquemos otra que sea de tipo 3.
        Sea la gramática $G'=\left(V',T',P',S'\right)$ dada por:
        \begin{align*}
            V' &= \{S, X,Y,Z,W,V,T\} \\
            T' &= \{a,b\} \\
            S' &= S \\
            P' &= \left\{
                \begin{array}{rcl}
                    S &\rightarrow & aS \mid X \\
                    X &\rightarrow & bY \\
                    Y &\rightarrow & aY \mid Z \\
                    Z &\rightarrow & bW \\
                    W &\rightarrow & aW \mid \veps \mid V \\
                    V &\rightarrow & bT \\
                    T &\rightarrow & aT \mid \veps
                \end{array}
            \right\}
        \end{align*}

        Esta gramática ya es de tipo $3$, pero contiene un número elevado de variables. Veamos si podemos reducirlo:
        Sea la gramática $G''=\left(V'',T'',P'',S''\right)$ dada por:
        \begin{align*}
            V'' &= \{S, X,Y,Z\} \\
            T'' &= \{a,b\} \\
            S'' &= S \\
            P'' &= \left\{
                \begin{array}{rcl}
                    S &\rightarrow & aS \mid bX \\
                    X &\rightarrow & aX \mid bY \\
                    Y &\rightarrow & aY \mid \veps \mid bZ \\
                    Z &\rightarrow & aZ \mid \veps
                \end{array}
            \right\}
        \end{align*}

        Notemos que, en esta gramática de tipo $3$, ya hemos conseguido el menor número de variables posibles, que representan las $4$ etapas. Como la última es opcional, está la regla $Y\rightarrow \veps$, para así no agregar la tercera $b$.

    \end{enumerate}
\end{ejercicio}

\begin{ejercicio}
    Encontrar gramáticas de tipo 2 para los siguientes lenguajes sobre el alfabeto $\{a, b\}$. En cada caso determinar si los lenguajes generados son de tipo 3, estudiando si existe una gramática de tipo 3 que los genera.
    \begin{enumerate}
        \item Palabras que no contienen la subcadena $ab$.
        
        Sea la gramática $G=\left(V,T,P,S\right)$ dada por:
        \begin{align*}
            V &= \{S, A\} \\
            T &= \{a,b\} \\
            S &= S \\
            P &= \left\{
                \begin{array}{rcl}
                    S &\rightarrow & aA \mid bS \mid \veps \\
                    A &\rightarrow & aA \mid \veps \\
                \end{array}
            \right\}
        \end{align*}

        Notemos además que esta gramática es de tipo $3$, y se tiene que:
        \begin{equation*}
            \cc{L}(G) = \{b^i a^j \mid i,j\in \bb{N}\cup \{0\}\}
        \end{equation*}


        \item Palabras que no contienen la subcadena $baa$.
        
        Sea la gramática $G=\left(V,T,P,S\right)$ dada por:
        \begin{align*}
            V &= \{S, B\} \\
            T &= \{a,b\} \\
            S &= S \\
            P &= \left\{
                \begin{array}{rcl}
                    S &\rightarrow & aS \mid bB \mid \veps \\
                    B &\rightarrow & bB \mid abB \mid a \mid \veps
                \end{array}
            \right\}
        \end{align*}
        Notemos además que esta gramática es de tipo $3$.
    \end{enumerate}
\end{ejercicio}

\begin{ejercicio}
    Encontrar una gramática libre de contexto que genere el lenguaje sobre el alfabeto $\{a, b\}$ de las palabras que tienen más $a$ que $b$ (al menos una más).

    Sea la gramática $G=\left(V,T,P,S\right)$ dada por:
    \begin{align*}
        V &= \{S, S'\} \\
        T &= \{a,b\} \\
        S &= S \\
        P &= \left\{
            \begin{array}{rcl}
                S &\rightarrow & S'aS'\\
                S' &\rightarrow & S'aS' \mid aS'bS' \mid bS'aS' \mid \veps
            \end{array}
        \right\}
    \end{align*}
\end{ejercicio}

\begin{ejercicio}
    Encontrar, si es posible, una gramática regular (o, si no es posible, una gramática libre del contexto) que genere el lenguaje $L$ supuesto que $L \subset \{a, b\}^\ast$ y verifica:
    \begin{enumerate}
        \item $u \in L$ si, y solamente si, verifica que $u$ no contiene dos símbolos $b$ consecutivos.
        
        Sea la gramática $G=\left(V,T,P,S\right)$ dada por:
        \begin{align*}
            V &= \{S\} \\
            T &= \{a,b\} \\
            S &= S \\
            P &= \left\{
                \begin{array}{rcl}
                    S &\rightarrow & aS \mid baS \mid b\mid \veps
                \end{array}
            \right\}
        \end{align*}
        \item $u \in L$ si, y solamente si, verifica que $u$ contiene dos símbolos $b$ consecutivos.
        
        Sea la gramática $G=\left(V,T,P,S\right)$ dada por:
        \begin{align*}
            V &= \{S, B, F\} \\
            T &= \{a,b\} \\
            S &= S \\
            P &= \left\{
                \begin{array}{rcl}
                    S &\rightarrow & aS \mid bB \\
                    B &\rightarrow & bF \mid aS \\
                    F &\rightarrow & aF \mid bF \mid \veps
                \end{array}
            \right\}
        \end{align*}
        Notemos que, en este caso, tenemos tres estados:
        \begin{itemize}
            \item $S$: No hemos encontrado dos $b$'s consecutivas.
            \item $B$: Hemos encontrado una $b$, y puede ser que nos encontremos la segunda $b$.
            \item $F$: Hemos encontrado dos $b$'s consecutivas; ya hay libertad.
        \end{itemize}

        Sí es cierto que usamos tres variables. Para usar solo dos variables,
        podemos hacer lo siguiente.
        Sea la gramática $G'=\left(V',T',P',S'\right)$ dada por:
        \begin{align*}
            V' &= \{S, X\} \\
            T' &= \{a,b\} \\
            S' &= S \\
            P' &= \left\{
                \begin{array}{rcl}
                    S &\rightarrow & aS \mid bS \mid bbX \\
                    X &\rightarrow & aX \mid bX \mid \veps
                \end{array}
            \right\}
        \end{align*}
    \end{enumerate}
\end{ejercicio}

\begin{ejercicio}
    Encontrar, si es posible, una gramática regular (o, si no es posible, una gramática libre del contexto) que genere el lenguaje $L$ supuesto que $L \subset \{a, b\}^\ast$ y verifica:
    \begin{enumerate}
        \item $u \in L$ si, y solamente si, verifica que contiene un número impar de símbolos $a$.
        
        Sea la gramática $G=\left(V,T,P,S\right)$ dada por:
        \begin{align*}
            V &= \{S, X\} \\
            T &= \{a,b\} \\
            S &= S \\
            P &= \left\{
                \begin{array}{rcl}
                    S &\rightarrow & aX \mid bS\\
                    X &\rightarrow & aS \mid bX \mid \veps
                \end{array}
            \right\}
        \end{align*}
        \item $u \in L$ si, y solamente si, verifica que no contiene el mismo número de símbolos $a$ que de símbolos $b$.
        
        Sea la gramática $G=\left(V,T,P,S\right)$ dada por:
        \begin{align*}
            V &= \{S, A, B, X\} \\
            T &= \{a,b\} \\
            S &= S \\
            P &= \left\{
                \begin{array}{rcl}
                    S &\rightarrow & AaA \mid BbB \\
                    A &\rightarrow & AaA\mid X \\
                    B &\rightarrow & BbB \mid X \\
                    X &\rightarrow & aXbX \mid bXaX \mid \veps
                \end{array}
            \right\}
        \end{align*}
    \end{enumerate}
\end{ejercicio}


\begin{ejercicio}
    Dado el alfabeto $A = \{a, b\}$ determinar si es posible encontrar una gramática libre de contexto que:
    \begin{enumerate}
        \item Genere las palabras de longitud impar, y mayor o igual que 3, tales que la primera letra coincida con la letra central de la palabra.
        
        Sea la gramática $G=\left(V,T,P,S\right)$ dada por:
        \begin{align*}
            V &= \{S, X, A, B, C, D\} \\
            T &= \{a,b\} \\
            S &= S \\
            P &= \left\{
                \begin{array}{rcl}
                    S &A\mid B \\
                    A &\rightarrow & aCX \\
                    C & \rightarrow & a \mid XCX \\
                    B &\rightarrow & bDX \\
                    D & \rightarrow & b \mid XDX \\
                    X &\rightarrow & a\mid b
                \end{array}
            \right\}
        \end{align*}
        \item Genere las palabras de longitud par, y mayor o igual que 2, tales que las dos letras centrales coincidan.
        
        Sea la gramática $G=\left(V,T,P,S\right)$ dada por:
        \begin{align*}
            V &= \{S, X\} \\
            T &= \{a,b\} \\
            S &= S \\
            P &= \left\{
                \begin{array}{rcl}
                    S &\rightarrow & XSX\mid C \\
                    C &\rightarrow & aa \mid bb \\
                    X &\rightarrow & a \mid b
                \end{array}
            \right\}
        \end{align*}
    \end{enumerate}
\end{ejercicio}

\begin{ejercicio}
    Sea la gramática $G=\left(V,T,P,S\right)$ dada por:
    \begin{align*}
        V &= \{S, X\} \\
        T &= \{a,b\} \\
        S &= S\\
        P &= \left\{
            \begin{array}{rcl}
                S &\rightarrow & SS \\
                S &\rightarrow & XXX \\
                X &\rightarrow & aX \mid Xa \mid b
            \end{array}
        \right\}
    \end{align*}
    Determinar si el lenguaje generado por la gramática es regular. Justificar la respuesta.\\

    Sea la siguiente gramática regular $G'=\left(V',T',P',S'\right)$ dada por:
    \begin{align*}
        V' &= \{S, X\} \\
        T' &= \{a,b\} \\
        S' &= S \\
        P' &= \left\{
            \begin{array}{rcl}
                S &\rightarrow & aS \mid bX \\
                X &\rightarrow & aX \mid bY \\
                Y &\rightarrow & aY \mid bZ \\
                Z &\rightarrow & aZ \mid bW \mid \veps \\
                W &\rightarrow & aW \mid bU \\
                U &\rightarrow & aU \mid bV \\
                V &\rightarrow & aV \mid \veps
            \end{array}
        \right\}
    \end{align*}

    Tenemos que $\cc{L}(G) = \cc{L}(G')$, y como $G'$ es una gramática regular, tenemos que $\cc{L}(G)$ es regular.
    Sí es cierto que en el tema $2$ aprendemos otras maneras de demostrarlo más sencillas, como buscar un autómata finito que lo genere.
\end{ejercicio}

\begin{ejercicio}
    Dado un lenguaje $L$ sobre un alfabeto $A$, ¿es $L^{\ast}$ siempre numerable? ¿nunca lo es? ¿o puede serlo unas veces sí y otras, no? Pon ejemplos en este último caso.\\

    $L^{\ast}$ es siempre numerable, veámos por qué. $L^{\ast}$ es un lenguaje sobre el alfabeto $A$, por lo que $L^{\ast}\subseteq A^{\ast}$ y $A^{\ast}$ es numerable (visto en teoría), luego $L^{\ast}$ también lo es.
\end{ejercicio}

\begin{ejercicio}
    Dado un lenguaje $L$ sobre un alfabeto $A$, caracterizar cuando $L^{\ast} = L$. Esto es, dar un conjunto de propiedades sobre $L$ de manera que $L$ cumpla esas propiedades si y sólo si $L^{\ast} = L$.

    \begin{equation*}
        L = L^{\ast} \Longleftrightarrow \left\{
            \begin{array}{cl}
                \veps \in L \\ \land \\ u,v \in L & \Longrightarrow uv\in L
            \end{array}
        \right.
    \end{equation*}
    Es decir, $L=L^{\ast}$ si y solo si la cadena vacía está en $L$ y además es cerrado para concatenaciones.

    \begin{proof} Demostramos mediante doble implicación.
        \begin{description}
            \item [$\Longleftarrow)$] La inclusión $L\subseteq L^{\ast}$ es obvia, por lo que solo falta demostrar la otra inclusión.\\

                Sea $v\in L^{\ast}$:
                \begin{enumerate}
                    \item Si $v = \veps \Longrightarrow v\in L$ por hipótesis.
                    \item Si $v\neq \veps$, $\exists n\in \mathbb{N}$ tal que 
                        \begin{equation*}
                            v = a_1 a_2 \ldots a_n
                        \end{equation*}
                        con $a_i \in L$ $\forall i \in \{1, \ldots, n\}$, de donde tenemos que $v\in L$, por ser cerrado para concatenaciones. Luego $L^{\ast}\subseteq L$.
                \end{enumerate}
            \item [$\Longrightarrow)$] Hemos de probar dos cosas:
                \begin{enumerate}
                    \item $\veps \in L^{\ast}=L$.
                    \item Sean $u,v\in L=L^{\ast} \Longrightarrow uv\in L^{\ast}=L$.
                \end{enumerate}
        \end{description}
    \end{proof}
\end{ejercicio}

\begin{ejercicio}
    Dados dos homomorfismos $f : A^{\ast} \rightarrow B^{\ast}$, $g : A^{\ast} \rightarrow B^{\ast}$, se dice que son iguales si $f(x) = g(x)$, $\forall x \in A^{\ast}$. ¿Existe un procedimiento algorítmico para comprobar si dos homomorfismos son iguales?\\

    Sí, basta probar que su imagen coincide sobre un conjunto finito de elementos, los de $A$:
    \begin{equation*}
        f(x) = g(x) \quad \forall x\in A^{\ast} \Longleftrightarrow f(a)=g(a) \quad \forall a\in A
    \end{equation*}
    \begin{proof}\ 
        \begin{description}
            \item [$\Longleftarrow)$] Sea $v\in A^{\ast}$, $\exists n\in \mathbb{N}$ tal que $v=a_1a_2\ldots a_n$ con $a_i \in A$ $\forall i \in \{1,\ldots, n\}$
                \begin{equation*}
                    f(v) = f(a_1)f(a_2)\ldots f(a_n) = g(a_1)g(a_2)\ldots g(a_n) = g(v)
                \end{equation*}
            \item [$\Longrightarrow)$] Sea $a\in A \Longrightarrow a\in A^{\ast}\Longrightarrow f(a)=g(a)$.
        \end{description}
    \end{proof}
\end{ejercicio}

\begin{ejercicio}
    Sea $L \subseteq A^{\ast}$ un lenguaje arbitrario. Sea $C_0 = L$ y definamos los lenguajes $S_i$ y $C_i$, para todo $i \geq 1$, por $S_i = C_{i-1}^+$ y $C_i = \ol{S_i}$. 
    \begin{enumerate}
        \item ¿Es $S_1$ siempre, nunca o a veces igual a $C_2$? Justifica la respuesta.
        \item Demostrar que $S_2 = C_3$, cualquiera que sea $L$.
        \begin{observacion}
            Demuestra que $C_2$ es cerrado para la concatenación.
        \end{observacion}
    \end{enumerate}
    % // TODO: Hacer JJ
\end{ejercicio}

\begin{ejercicio}
    Demuestra que, para todo alfabeto $A$, el conjunto de los lenguajes finitos sobre dicho alfabeto es numerable.

    Sea $A=\{a_1, a_2, \ldots, a_n\}$, con $n\in \mathbb{N}$. Definimos el siguiente conjunto:
    \begin{equation*}
        \Gamma = \{L\subseteq A^{\ast} \mid L \text{ es finito}\}
    \end{equation*}

    Dado un símbolo $z\notin A$, definimos el conjunto $B=\{z\}\cup A$. Sea $B^{\ast}$ numerable, y buscamos una inyección de $\Gamma$ en $B^{\ast}$.
    Dado un lenguaje $L\in \Gamma$, sea $L=\{l_1, l_2, \ldots, l_m\}$, con $m\in \mathbb{N}$ y $l_i\in A^{\ast}$ $\forall i\in \{1, \ldots, m\}$. Definimos la siguiente función:
    \Func{f}{\Gamma}{B^\ast}{L}{zl_1zl_2\ldots zl_mz}

    Veamos que $f$ es inyectiva. Sean $L_1, L_2\in \Gamma$ tales que $f(L_1)=f(L_2)$. Entonces,
    \begin{equation*}
        zl_1zl_2\ldots zl_kz = zl'_1zl'_2\ldots zl'_{k'}z
    \end{equation*}
    Por ser ambas palabras iguales, tenemos que $k=k'$ y $l_i=l'_i$ $\forall i\in \{1, \ldots, k\}$, de donde $L_1=L_2$. Por tanto, $f$ es inyectiva, por lo que $\Gamma$ es inyectivo con un subconjunto de $B^{\ast}$, que es numerable. Por tanto, $\Gamma$ es numerable.
\end{ejercicio}




\subsection{Cálculo de gramáticas}

\begin{ejercicio}[Complejidad: Sencilla]
    Calcula, de forma razonada, gramáticas que generen cada uno de los siguientes lenguajes:
    \begin{enumerate}
        \item $\{ u\in \{0,1\}^\ast \mid |u|\leq 4 \}$
        
        Sea la gramática $G=\left(V,T,P,S\right)$ dada por:
        \begin{align*}
            V &= \{S, X\} \\
            T &= \{0,1\} \\
            S &= S \\
            P &= \left\{
                \begin{array}{rcl}
                    S &\rightarrow & XXXX \\
                    X &\rightarrow & 0 \mid 1 \mid \veps
                \end{array}
            \right\}
        \end{align*}

        No obstante, esta gramática es de tipo $2$. Busquemos una de tipo $3$.
        Sea la gramática $G'=\left(V',T',P',S'\right)$ dada por:
        \begin{align*}
            V' &= \{S, X, Y, Z\} \\
            T' &= \{0,1\} \\
            S' &= S \\
            P' &= \left\{
                \begin{array}{rcl}
                    S &\rightarrow & 0X \mid 1X \mid \veps \\
                    X &\rightarrow & 0Y \mid 1Y \mid \veps \\
                    Y &\rightarrow & 0Z \mid 1Z \mid \veps \\
                    Z &\rightarrow & 0 \mid 1
                \end{array}
            \right\}
        \end{align*}
        Tenemos que $\cc{L}(G) = \cc{L}(G')$, y es igual al lenguaje deseado. Tenemos por tanto que es un lenguaje regular.


        \item Palabras con 0's y 1's que no contengan dos 1's consecutivos y que empiecen por un 1 y que terminen por dos 0's.
        
        Sea la gramática $G=\left(V,T,P,S\right)$ dada por:
        \begin{align*}
            V &= \{S, X, Y\} \\
            T &= \{0,1\} \\
            S &= S \\
            P &= \left\{
                \begin{array}{rcl}
                    S &\rightarrow & 1X00 \\
                    X &\rightarrow & 0Y \mid \veps \\
                    Y &\rightarrow & 0Y \mid 1X \mid \veps \\
                \end{array}
            \right\}
        \end{align*}

        Notemos que esta gramática es de tipo 2 debido a la primera regla de producción. Busquemos una de tipo 3. 
        Sea la gramática $G'=\left(V',T',P',S'\right)$ dada por:
        \begin{align*}
            V' &= \{S, X, Y\} \\
            T' &= \{0,1\} \\
            S' &= S \\
            P' &= \left\{
                \begin{array}{rcl}
                    S &\rightarrow & 1X \\
                    X &\rightarrow & 0Y \mid F \\
                    Y &\rightarrow & 0Y \mid 1X \mid F \\
                    F &\rightarrow & 00
                \end{array}
            \right\}
        \end{align*}

        Tenemos que $\cc{L}(G) = \cc{L}(G')$, y es igual al lenguaje deseado. Tenemos por tanto que es un lenguaje regular. En esta última gramática, tenemos los siguientes estados:
        \begin{itemize}
            \item $S$: Es el estado inicial, empezamos con un $1$.
            \item $X$: Acabamos de escribir un $1$, por lo que ahora tan solo podemos escribir $0$'s.
            \item $Y$: Acabamos de escribir un $0$, por lo que ahora podemos escribir tanto $0$'s como $1$'s.
            \item $F$: Ya hemos terminado, y escribimos los dos $0$'s finales por la restricción impuesta.
        \end{itemize}
        

        \item El conjunto vacío.
        
        Sea la gramática $G=\left(V,T,P,S\right)$ dada por:
        \begin{align*}
            V &= \{S\} \\
            T &= \emptyset \\
            S &= S \\
            P &= \left\{
                \begin{array}{rcl}
                    S &\rightarrow & S
                \end{array}
            \right\}
        \end{align*}

        \item El lenguaje formado por los números naturales.
        
        Sea la gramática $G=\left(V,T,P,S\right)$ dada por:
        \begin{align*}
            V &= \{\langle \text{número no iniciado} \rangle, \langle \text{dígito no cero} \rangle, \langle \text{dígito} \rangle, \langle \text{número iniciado} \rangle\} \\
            T &= \{0,1,2,3,4,5,6,7,8,9\} \\
            S &= \langle \text{número no iniciado} \rangle \\
            P &= \left\{
                \begin{array}{rcl}
                    \langle \text{número no iniciado} \rangle &\rightarrow & \langle \text{dígito no cero} \rangle \mid \langle \text{dígito no cero} \rangle \langle \text{número iniciado} \rangle \\
                    \langle \text{número iniciado} \rangle &\rightarrow & \langle \text{dígito} \rangle \mid \langle \text{dígito} \rangle \langle \text{número iniciado} \rangle \\
                    \langle \text{dígito no cero} \rangle &\rightarrow & 1 \mid 2 \mid 3 \mid 4 \mid 5 \mid 6 \mid 7 \mid 8 \mid 9 \\
                    \langle \text{dígito} \rangle &\rightarrow & 0 \mid \langle \text{dígito no cero} \rangle
                \end{array}
            \right\}
        \end{align*}

        Notemos que esta gramática es similar a la descrita en el Ejercicio \ref{ej:1.2}.\ref{ej:1.2.b}, pero adaptada para que los números naturales no puedan empezar por $0$.
        No obstante, esta gramática es de tipo $2$. Busquemos una de tipo $3$.
        Sea la gramática $G'=\left(V',T',P',S'\right)$ dada por:
        \begin{align*}
            V' &= \{S, X, Y, Z\} \\
            T' &= \{0,1,2,3,4,5,6,7,8,9\} \\
            S' &= S \\
            P' &= \left\{
                \begin{array}{rcl}
                    S &\rightarrow & 0 \mid 1N \mid 2N \mid 3N \mid 4N \mid 5N \mid 6N \mid 7N \mid 8N \mid 9N\\
                    N &\rightarrow & 0N\mid 1N \mid 2N \mid 3N \mid 4N \mid 5N \mid 6N \mid 7N \mid 8N \mid 9N \mid \veps
                \end{array}
            \right\}
        \end{align*}
        \item $\{ a^n \in \{a,b\}^\ast \mid n\geq 0 \} \cup \{ a^nb^n \in \{a,b\}^\ast \mid n\geq 0 \}$
        
        Sea la gramática $G=\left(V,T,P,S\right)$ dada por:
        \begin{align*}
            V &= \{S, X, Y\} \\
            T &= \{a,b\} \\
            S &= S \\
            P &= \left\{
                \begin{array}{rcl}
                    S &\rightarrow & X \mid Y \mid \veps \\
                    X &\rightarrow & aX \mid \veps \\
                    Y &\rightarrow & aYb \mid \veps
                \end{array}
            \right\}
        \end{align*}
        \item $\{ a^nb^{2n}c^m \in \{a,b,c\}^\ast \mid n,m>0 \}$
        
        Sea la gramática $G=\left(V,T,P,S\right)$ dada por:
        \begin{align*}
            V &= \{S, X, Y, Z\} \\
            T &= \{a,b,c\} \\
            S &= S \\
            P &= \left\{
                \begin{array}{rcl}
                    S &\rightarrow & aXbbcY \\
                    X &\rightarrow & aXbb \mid \veps \\
                    Y &\rightarrow & cY \mid \veps
                \end{array}
            \right\}
        \end{align*}
        \item $\{ a^nb^ma^n \in \{a,b\}^\ast \mid m,n\geq 0 \}$
        
        Sea la gramática $G=\left(V,T,P,S\right)$ dada por:
        \begin{align*}
            V &= \{S, X\} \\
            T &= \{a,b\} \\
            S &= S \\
            P &= \left\{
                \begin{array}{rcl}
                    S &\rightarrow & aSa \mid bX \mid \veps \\
                    X &\rightarrow & bX \mid \veps \\
                \end{array}
            \right\}
        \end{align*}

        \item Palabras con 0's y 1's que contengan la subcadena 00 y 11.
        
        Sea la gramática $G=\left(V,T,P,S\right)$ dada por:
        \begin{align*}
            V &= \{S, X\} \\
            T &= \{0,1\} \\
            S &= S \\
            P &= \left\{
                \begin{array}{rcl}
                    S &\rightarrow & X00X11X \mid X11X00X \\
                    X &\rightarrow & 0X \mid 1X \mid \veps
                \end{array}
            \right\}
        \end{align*}

        Notemos que esta gramática es de tipo $2$. Busquemos una de tipo $3$.
        Sea la gramática $G'=\left(V',T',P',S'\right)$ dada por:
        \begin{align*}
            V' &= \{S, X, A, B, F\} \\
            T' &= \{0,1\} \\
            S' &= S \\
            P' &= \left\{
                \begin{array}{rcl}
                    S &\rightarrow & 0S \mid 1S \mid X\\
                    X &\rightarrow & 00A \mid 11B \\
                    A &\rightarrow & 0A \mid 1A \mid 11F \\
                    B &\rightarrow & 0B \mid 1B \mid 00F \\
                    F &\rightarrow & 0F \mid 1F \mid \veps
                \end{array}
            \right\}
        \end{align*}

        Notemos que:
        \begin{itemize}
            \item $S$: No hemos encontrado ninguna subcadena.
            \item $X$: Hemos encontrado una subcadena, y ahora buscamos la otra.
            \item $A$: Hemos encontrado la subcadena $00$, y ahora buscamos la subcadena $11$.
            \item $B$: Hemos encontrado la subcadena $11$, y ahora buscamos la subcadena $00$.
            \item $F$: Hemos encontrado ambas subcadenas.
        \end{itemize}
        
        \item Palíndromos formados con las letras $a$ y $b$.
        
        Sea la gramática $G=\left(V,T,P,S\right)$ dada por:
        \begin{align*}
            V &= \{S, X, Y\} \\
            T &= \{a,b\} \\
            S &= S \\
            P &= \left\{
                \begin{array}{rcl}
                    S &\rightarrow & aSa \mid bSb \mid \veps \mid a \mid b
                \end{array}
            \right\}
        \end{align*}
        Notemos que las reglas $S\rightarrow a\mid b$ se han añadido para añadir los palíndromos de longitud impar.
    \end{enumerate}
\end{ejercicio}

\begin{ejercicio}[Complejidad: Media]
    Calcula, de forma razonada, gramáticas que generen cada uno de los siguientes lenguajes:
    \begin{enumerate}
        \item $\{uv \in \{0,1\}^\ast \mid u^{-1} \text{ es un prefijo de } v\}$
        
        Sea la gramática $G=\left(V,T,P,S\right)$ dada por:
        \begin{align*}
            V &= \{S, X, Y\} \\
            T &= \{0,1\} \\
            S &= S \\
            P &= \left\{
                \begin{array}{rcl}
                    S &\rightarrow & XY \\
                    X &\rightarrow & 0X0 \mid 1X1 \mid \veps \\
                    Y &\rightarrow & 0Y \mid 1Y \mid \veps
                \end{array}
            \right\}
        \end{align*}
        Notemos que $X$ deriva en el palíndromo, $uu^{-1}$, y $Y$ en el resto de la palabra de $v$.
        \item $\{ucv \in \{a,b,c\}^\ast \mid |u| = |v|\}$
        
        Sea la gramática $G=\left(V,T,P,S\right)$ dada por:
        \begin{align*}
            V &= \{S, X\} \\
            T &= \{a,b,c\} \\
            S &= S \\
            P &= \left\{
                \begin{array}{rcl}
                    S &\rightarrow & XSX \mid c \\
                    X &\rightarrow & a \mid b \mid c
                \end{array}
            \right\}
        \end{align*}

        \item $\{u1^n \in \{0,1\}^\ast \mid |u| = n\}$
        
        Sea la gramática $G=\left(V,T,P,S\right)$ dada por:
        \begin{align*}
            V &= \{S, X\} \\
            T &= \{0,1\} \\
            S &= S \\
            P &= \left\{
                \begin{array}{rcl}
                    S &\rightarrow & XS1 \mid \veps \\
                    X &\rightarrow & 0 \mid 1
                \end{array}
            \right\}
        \end{align*}

        \item $\{a^nb^na^{n+1} \in \{a,b\}^\ast \mid n\geq 0\}$ (observar transparencias de teoría)
        
        Sea la gramática $G=\left(V,T,P,S\right)$ dada por:
        \begin{align*}
            V &= \{S, X, Y\} \\
            T &= \{a,b\} \\
            S &= S \\
            P &= \left\{
                \begin{array}{rcl}
                    S &\rightarrow & a\mid abaa\mid aXbaa\\
                    Xb & \rightarrow & bX\\
                    Xa & \rightarrow & Ybaa\\
                    bY & \rightarrow & Yb\\
                    aY & \rightarrow aa\mid aaX
                \end{array}
            \right\}
        \end{align*}
    \end{enumerate}
\end{ejercicio}


\begin{ejercicio}[Complejidad: Difícil]
    Calcula, de forma razonada, gramáticas que generen cada uno de los siguientes lenguajes:
    \begin{enumerate}
        \item $\{a^nb^mc^k \in \{a,b,c\}^\ast \mid k = m + n\}$
        
        Sea la gramática $G=\left(V,T,P,S\right)$ dada por:
        \begin{align*}
            V &= \{S, X\} \\
            T &= \{a,b,c\} \\
            S &= S \\
            P &= \left\{
                \begin{array}{rcl}
                    S &\rightarrow & aSc \mid X \\
                    X &\rightarrow & bXc \mid \veps
                \end{array}
            \right\}
        \end{align*}
        
        \item Palabras que son múltiplos de 7 en binario.
        
        % // TODO: Hacer JJ
    \end{enumerate}
\end{ejercicio}


\begin{ejercicio}[Complejidad: Extrema (no son libres de contexto)]
    Calcula, de forma razonada, gramáticas que generen cada uno de los siguientes lenguajes:
    \begin{enumerate}
        \item $\{ww \mid w \in \{0,1\}^\ast\}$
            % Para este lenguaje, hemos construido la gramática $G=(V,T,P,S)$ dada por:
            % \begin{align*}
            %     V &= \{S, \alpha, \beta, \gamma, X, E, E_1, E_0, E', B\} \\
            %     T &= \{0,1\} \\
            %     S &= S
            % \end{align*}
            % $P$ que contiene las siguientes reglas de producción, las cuales iremos explicando poco a poco para que pueda entenderse la gramática.

            % La idea principal es generar una palabra cualquiera del lenguaje ${\{0,1\}}^{\ast}$ entre las variables $\alpha$ y $\beta$. Posteriormente, iremos copiando dicha palabra a la derecha de $\beta$ usando para ello las variables $E$. Controlaremos con la variale $\gamma$ la parte de la palabra de la izquierda que ya hayamos copiado a la derecha de $\beta$.

            % Finalmente, utilizamos $B$ para eliminar las variables restantes una vez hecha la copia de la palabra de la izquierda en la parte derecha. 

            % Las reglas de producción de $P$ serán las siguientes:
            % \begin{itemize}
            %     \item Comenzamos construyendo el entorno en el que trabajaremos:
            %     \begin{equation*}
            %         S \rightarrow \alpha X \beta
            %     \end{equation*}
            %     Donde ya hemos comentado que entre $\alpha$ y $\beta$ generaremos una palabra cualquiera. Para ello, usaremos $X$:
            %     \begin{equation*}
            %         X \rightarrow 0X\ |\ 1X\ |\ E\gamma
            %     \end{equation*}

            %     \item Ahora, comenzará la copia de la palabra comprendida entre $\alpha$ Y $\beta$. La porción de palabra comprendida entre $\gamma$ y $\beta$ es la porción de palabra que tenemos ya copiada a la derecha de $\beta$. Para el 0:
            %         \begin{equation*}
            %             0E\gamma \rightarrow \gamma 0 E_0
            %         \end{equation*}
            %         donde avanzamos $\gamma$ un caracter y ahora nos movemos hacia la derecha hasta encontrar $\beta$:
            %         \begin{align*}
            %             E_0 0 &\rightarrow 0 E_0 \\
            %             E_0 1 &\rightarrow 1 E_0 \\
            %             E_0 \beta & \rightarrow E' \beta 0
            %         \end{align*}
            % \end{itemize}

        \item $\{a^{n^2} \in \{a\}^{\ast} \mid n\geq 0\}$
        \item $\{a^p \in \{a\}^{\ast} \mid p \text{ es primo}\}$
        \item $\{a^nb^m \in \{a,b\}^{\ast} \mid n\leq m^2\}$
    \end{enumerate}

\end{ejercicio}


    %\section{Vectores Aleatorios}

\begin{ejercicio}
    Asociadas al experimento aleatorio de lanzar un dado y una moneda no cargados, se define la variable $X$ como el valor del dado y la variable $Y$, que toma el valor 0 si sale cara en la moneda, y 1 si sale cruz. Calcular la función masa de probabilidad y la función de distribución del vector aleatorio $(X,Y)$.
\end{ejercicio}

\begin{ejercicio}
    El número de automóviles utilitarios, $X$, y el de automóviles de lujo, $Y$, que poseen las familias de una población se distribuye de acuerdo a las siguientes probabilidades:
    \begin{table}[H]
        \centering
        \begin{tabular}{c|ccc}
            $X\backslash Y$ & 0 & 1 & 2 \\
            \hline
            0 & \nicefrac{1}{3} & \nicefrac{1}{12} & \nicefrac{1}{24} \\
            1 & \nicefrac{1}{6} & \nicefrac{1}{24} & \nicefrac{1}{48} \\
            2 & \nicefrac{5}{22} & \nicefrac{5}{88} & \nicefrac{5}{176} \\
        \end{tabular}
    \end{table}
    Calcular la función de distribución del vector $(X,Y)$ en los puntos $(0,0)$; $(0,2)$; $(1,1)$ y $(2,1)$, y la probabilidad de que una familia tenga tres o más automóviles.
\end{ejercicio}

\begin{ejercicio}
    La función de densidad del vector aleatorio $(X,Y)$, donde $X$ denota los Kg. de naranjas, e $Y$ los Kg. de manzanas vendidos al día en una frutería está dada por
    \[
        f(x, y) = \frac{1}{400}; \quad 0 < x < 20; \quad 0 < y < 20.
    \]
    Determinar la función de distribución de $(X,Y)$ y la probabilidad de que en un día se vendan
    entre naranjas y manzanas, menos de 20 kilogramos.
\end{ejercicio}

\begin{ejercicio}
    La renta, $X$, y el consumo, $Y$, de los habitantes de una población, tienen por funciones de densidad
    \[
        f_X(x) = 2-2x; \quad 0 < x < 1; \quad f(y/x) = \frac{1}{x}; \quad 0 < y < x.
    \]
    Determinar la función de densidad conjunta del vector aleatorio $(X,Y)$ y la probabilidad de que el consumo sea inferior a la mitad de la renta.
\end{ejercicio}

\begin{ejercicio}
    Una gasolinera tiene $Y$ miles de litros en su depósito de gasóleo al comienzo de cada semana. A lo largo de una semana se venden $X$ miles de litros del citado combustible, siendo la función de densidad conjunta de $(X,Y)$ :
    \[
        f(x, y) = \frac{1}{8}; \quad 0 < x < y < 4.
    \]
    Se pide:
    \begin{enumerate}
        \item Probar que $f(x, y)$ es función de densidad y obtener la función de distribución.
        \item Probabilidad de que en una semana se venda más de la tercera parte de los litros de que se dispone al comienzo de la misma.
        \item Si en una semana se han vendido 3.000 litros de gasóleo, ¿cuál es la probabilidad de que al comienzo de la semana hubiese entre 3.500 y 3.750 litros de combustible?
    \end{enumerate}
\end{ejercicio}

\begin{ejercicio}
    Sea $(X,Y)$ un vector aleatorio continuo con función de densidad
    \[
        f(x, y) = k, \quad (x, y) \in R,
    \]
    siendo $R$ el rombo de vértices $(3,0)$; $(0,2)$; $(-3,0)$; $(0,-2)$. Calcular $k$ para que $f$ sea una función de densidad. Hallar las distribuciones marginales y condicionadas.
\end{ejercicio}

\begin{ejercicio}
    Sea $(X,Y)$ un vector aleatorio continuo con función de densidad
    \[
        f(x, y) = k, \quad x^2 \leq y \leq 1,
    \]
    anulándose fuera del recinto indicado. Hallar la constante $k$ para que $f$ sea una función de densidad de probabilidad y calcular la función de distribución de probabilidad. Calcular $P(X \geq Y)$. Calcular las distribuciones marginales y condicionadas.
\end{ejercicio}

\begin{ejercicio}
    Sea la función de densidad de probabilidad
    \[
        f(x, y) = \begin{cases}
            kxy^2 + 1, & 0 < x < 1, -1 < y < 1, \\
            0, & \text{en otro caso}.
        \end{cases}
    \]
    Calcular la función de distribución de probabilidad y las marginales.
\end{ejercicio}

\begin{ejercicio}
    Sea $(X,Y)$ un vector aleatorio bidimensional continuo, con función de densidad de probabilidad
    \[
        f(x, y) = \begin{cases}
            k, & 0 < x + y < 1, |y| < 1, 0 < x < 1, \\
            0, & \text{en otro caso}.
        \end{cases}
    \]
    Hallar la constante $k$ para que $f$ sea una función de densidad de probabilidad y calcular la función de distribución de probabilidad. Calcular las distribuciones marginales y condicionadas.
\end{ejercicio}

\begin{ejercicio}
    Sea $(X,Y)$ un vector aleatorio bidimensional continuo, con distribución de probabilidad uniforme sobre el triángulo de vértices $(0,0)$; $(0,1)$; $(1,1)$. Determinar la función de densidad de probabilidad, la función de distribución de probabilidad y las distribuciones marginales y condicionadas.
\end{ejercicio}

\begin{ejercicio}
    Sea $(X,Y)$ una variable aleatoria bidimensional con distribución uniforme en el recinto
    \[
        C = \{(x, y) \in \mathbb{R}^2; x^2 + y^2 < 1; x \geq 0, y \geq 0\}.
    \]
    Calcular:
    \begin{enumerate}
        \item La función de distribución conjunta.
        \item Las funciones de densidad marginales.
        \item Las funciones de densidad condicionadas.
    \end{enumerate}
\end{ejercicio}
    \section{Independencia de Variables Aleatorias}


\begin{ejercicio}
    Sea un experimento aleatorio que consiste en lanzar un tetraedro regular cuyas caras están numeradas del $1$ al $4$, y se definen los sucesos:
    \begin{align*}
        A &= \{1 \text{ ó } 2\} \\
        B &= \{2 \text{ ó } 3\} \\
        C &= \{2 \text{ ó } 4\}.
    \end{align*}

    Responder a los siguientes apartados:
    \begin{enumerate}
        \item Calcular $P(A)$, $P(B)$, $P(C)$.
        
        Usando la Ley de Laplace, tenemos que:
        \begin{align*}
            P(A) &= \frac{2}{4} = \frac{1}{2} \\
            P(B) &= \frac{2}{4} = \frac{1}{2} \\
            P(C) &= \frac{2}{4} = \frac{1}{2}.
        \end{align*}

        \item Calcular la probabilidad de obtener un dos.
        
        Usando de nuevo la Ley de Laplace, tenemos que:
        \[
            P(\{2\}) = \frac{1}{4}.
        \]
        \item Indicar si los sucesos $A$, $B$ y $C$ son independientes dos a dos.
        
        Para comprobar si dos sucesos son independientes, debemos comprobar si se cumplen las siguientes igualdades:
        \[
            P(A \cap B) = P(A)P(B), \quad P(A \cap C) = P(A)P(C), \quad P(B \cap C) = P(B)P(C).
        \]

        Tenemos que:
        \begin{align*}
            P(A \cap B) &= P(\{2\}) = \frac{1}{4} = P(A)P(B) = \frac{1}{4} \\
            P(A \cap C) &= P(\{2\}) = \frac{1}{4} = P(A)P(C) = \frac{1}{4} \\
            P(B \cap C) &= P(\{2\}) = \frac{1}{4} = P(B)P(C) = \frac{1}{4}.
        \end{align*}

        Por lo tanto, los sucesos $A$, $B$ y $C$ son independientes dos a dos.
        \item Indicar si los sucesos $A$, $B$ y $C$ son mutuamente independientes.
        
        Para comprobar si tres sucesos son mutuamente independientes, debemos comprobar si se cumple la siguiente igualdad:
        \[
            P(A \cap B \cap C) = P(A)P(B)P(C).
        \]

        Tenemos que:
        \[
            P(A \cap B \cap C) = P(\{2\}) = \frac{1}{4} \neq P(A)P(B)P(C) = \frac{1}{8}.
        \]
        Por lo tanto, los sucesos $A$, $B$ y $C$ no son mutuamente independientes.
    \end{enumerate}
\end{ejercicio}

\begin{ejercicio}
    Sea $(X_1, X_2, X_3)$ un vector aleatorio con función masa de probabilidad
    \[
        P[(X_1, X_2, X_3) = (x_1, x_2, x_3)] = \frac{1}{4},
    \]
    siendo $(x_1, x_2, x_3) = \{(1,0,0),(0,1,0),(0,0,1),(1,1,1)\}$.
    \begin{enumerate}
        \item Indicar si son $X_1$, $X_2$, $X_3$ independientes dos a dos.
        
        Calculamos en primer lugar las funciones masa de probabilidad marginales. Para el caso de $X_1$, tenemos que:
        \begin{align*}
            P[X_1=0]=P[(0,1,0)]+P[(0,0,1)] = \frac{1}{4} + \frac{1}{4} = \frac{1}{2} \\
            P[X_1=1]=P[(1,0,0)]+P[(1,1,1)] = \frac{1}{4} + \frac{1}{4} = \frac{1}{2}.
        \end{align*}

        Para $X_2$ y $X_3$ se obtienen los mismos resultados. Por lo tanto, las funciones masas de probabilidad marginales para $i\in \{1,2,3\}$ son:
        \begin{align*}
            P[X_i=0] &= \frac{1}{2},\\
            P[X_i=1] &= \frac{1}{2}.
        \end{align*}

        Calculemos ahora las funciones masa de probabilidad bidimensionales. Para el caso de $X_1$ y $X_2$, tenemos que:
        \begin{align*}
            P[(X_1, X_2) = (0,0)] &= P[(0,1,0)] = \frac{1}{4} \\
            P[(X_1, X_2) = (0,1)] &= P[(0,0,1)] = \frac{1}{4} \\
            P[(X_1, X_2) = (1,0)] &= P[(1,0,0)] = \frac{1}{4} \\
            P[(X_1, X_2) = (1,1)] &= P[(1,1,1)] = \frac{1}{4}.
        \end{align*}

        Para $(X_1, X_3)$ y $(X_2, X_3)$ se obtienen los mismos resultados. Por lo tanto, las funciones masa de probabilidad bidimensionales para $i,j \in \{1,2,3\}$, $i \neq j$, son:
        \begin{align*}
            P[(X_i, X_j) = (0,0)] &= \frac{1}{4}, \\
            P[(X_i, X_j) = (0,1)] &= \frac{1}{4}, \\
            P[(X_i, X_j) = (1,0)] &= \frac{1}{4}, \\
            P[(X_i, X_j) = (1,1)] &= \frac{1}{4}.
        \end{align*}

        Veamos ahora si son independientes dos a dos. Para $i,j \in \{1,2,3\}$, $i \neq j$ y $a,b\in \{0,1\}$, tenemos que:
        \begin{align*}
            \frac{1}{4} &= P[(X_i, X_j) = (a,b)] = P[X_i = a]P[X_j = b] = \frac{1}{2} \cdot \frac{1}{2} = \frac{1}{4}.
        \end{align*}

        Por lo tanto, las variables $X_1$, $X_2$ y $X_3$ son independientes dos a dos.

        \item Indicar si son $X_1$, $X_2$, $X_3$ mutuamente independientes.
        
        Tenemos que:
        \begin{equation*}
            0 = P[(X_1, X_2, X_3) = (0,0,0)] \neq P[X_1 = 0]P[X_2 = 0]P[X_3 = 0] = \frac{1}{8}.
        \end{equation*}
        Por lo tanto, las variables $X_1$, $X_2$ y $X_3$ no son mutuamente independientes.

        \item Indicar si $X_1 + X_2$ y $X_3$ son independientes.
        
        Representamos en la siguiente tabla los valores de la suma $Z:=X_1 + X_2$. Notemos que no es la función masa de probabilidad conjunta de $(X_1,X_2)$, sino la tabla de los valores de $Z$.
        \begin{equation*}
            \begin{array}{c|cc}
                X_1\backslash X_2 & 0 & 1 \\
                \hline
                0 & 0 & 1 \\
                1 & 1 & 2
            \end{array}
        \end{equation*}

        Notemos que $Z$ toma los valores $0$, $1$ y $2$, calculemos cada una de las probabilidades usando el Teorema de Cambio de Variable:
        \begin{align*}
            P[Z=0] &= P[(X_1, X_2) = (0,0)] = \nicefrac{1}{4} \\
            P[Z=1] &= P[(X_1, X_2) = (0,1)] + P[(X_1, X_2) = (1,0)] = \nicefrac{1}{4} + \nicefrac{1}{4} = \nicefrac{1}{2} \\
            P[Z=2] &= P[(X_1, X_2) = (1,1)] = \nicefrac{1}{4}.
        \end{align*}

        Tenemos que:
        \begin{equation*}
            0 = P[(Z, X_3) = (0,0)] = P[X_1=0, X_2=0, X_3=0] \neq P[Z=0]P[X_3=0] = \nicefrac{1}{4}\cdot \nicefrac{1}{2}.
        \end{equation*}

        Por lo tanto, las variables $X_1 + X_2$ y $X_3$ no son independientes.
    \end{enumerate}
\end{ejercicio}

\begin{ejercicio}
    Definimos sobre el experimento de lanzar diez veces una moneda las variables aleatorias $X$ como el número de lanzamientos hasta que aparece la primera cara (si no aparece cara $X = 0$), e $Y$ como el número de lanzamientos hasta que aparece la primera cruz (con $Y = 0$ si no aparece cruz). Indicar si $X$ e $Y$ son independientes.\\

    La probabilidad de que haga falta $1$ lanzamiento para obtener la primera cara, al igual que para obtener la primera cruz, usamos la Ley de Laplace:
    \begin{equation*}
        P[X=1] = P[Y=1] = \frac{1}{2}
    \end{equation*}

    No obstante, por no poder darse en un mismo lanzamiento cara y cruz a la vez, tenemos que:
    \begin{align*}
        P[X=1, Y=1] &= 0
    \end{align*}

    Por tanto, como $P[X=1]P[Y=1] \neq P[X=1, Y=1]$, las variables $X$ e $Y$ no son independientes.
\end{ejercicio}

\begin{ejercicio}
    El número de automóviles utilitarios, $X$, y el de automóviles de lujo, $Y$, que poseen las familias de una población se distribuye de acuerdo a las siguientes probabilidades:
    \begin{equation*}
        \begin{array}{c|ccc|c}
            X \backslash Y & 0 & 1 & 2 & \\
            \hline
            0 & \nicefrac{1}{3} & \nicefrac{1}{12} & \nicefrac{1}{24} & \nicefrac{11}{24}\\
            1 & \nicefrac{1}{6} & \nicefrac{1}{24} & \nicefrac{1}{48} & \nicefrac{11}{48} \\
            2 & \nicefrac{5}{22} & \nicefrac{5}{88} & \nicefrac{5}{176} & \nicefrac{5}{16} \\ \hline
            & \nicefrac{8}{11} & \nicefrac{2}{11} & \nicefrac{1}{11} &
        \end{array}
    \end{equation*}
    Comprobar que las variables $X$ e $Y$ son independientes.\\

    Notemos que hemos incluido las funciones masa de probabilidad marginales en la última fila y columna de la tabla. Podemos comprobar así fácilmente que las variables $X$ e $Y$ son independientes.
\end{ejercicio}

\begin{ejercicio}
    En los siguientes dos apartados, estudiar la independencia de las variables aleatorias $X$ e $Y$, cuando su densidad de probabilidad conjunta se define como sigue:
    \begin{enumerate}
        \item $f(x, y) = \nicefrac{1}{2}$, si $(x, y)$ pertenece al cuadrado de vértices $(1,0);(0,1);(-1,0);(0,-1)$.
        
        El recinto descrito es:
        \begin{figure}[H]
            \centering
            \begin{tikzpicture}
                \begin{axis}[
                    axis lines = center,
                    xlabel = $x$,
                    ylabel = $y$,
                    xmin = -1.5, xmax = 1.5,
                    ymin = -1.5, ymax = 1.5,
                    xtick = {-1,0,1},
                    ytick = {-1,0,1},
                    xticklabels = {$-1$,$0$,$1$},
                    yticklabels = {$-1$,$0$,$1$},
                    axis equal image
                ]
                    \addplot [
                        fill=gray,
                        fill opacity=0.5
                    ] coordinates {
                        (1,0) (0,1) (-1,0) (0,-1)
                    } -- cycle;
                \end{axis}
            \end{tikzpicture}
        \end{figure}

        Calculamos cada una de las funciones de densidad marginales:
        \begin{itemize}
            \item $f_X$ si $x \in [-1,0]$:
            \begin{equation*}
                f_X(x) = \int_{-x-1}^{x+1} \frac{1}{2} ~d{y} = \left[\frac{1}{2}y\right]_{-x-1}^{x+1} = \frac{1}{2}(x+1) - \frac{1}{2}(-x-1) = x+1.
            \end{equation*}

            \item $f_X$ si $x \in [0,1]$:
            \begin{equation*}
                f_X(x) = \int_{x-1}^{-x+1} \frac{1}{2} ~d{y} = \left[\frac{1}{2}y\right]_{x-1}^{-x+1} = \frac{1}{2}(-x+1) - \frac{1}{2}(x-1) = 1-x.
            \end{equation*}

            \item $f_Y$ si $y \in [-1,0]$:
            \begin{equation*}
                f_Y(y) = \int_{-y-1}^{y+1} \frac{1}{2} ~d{x} = \left[\frac{1}{2}x\right]_{-y-1}^{y+1} = \frac{1}{2}(y+1) - \frac{1}{2}(-y-1) = y+1.
            \end{equation*}

            \item $f_Y$ si $y \in [0,1]$:
            \begin{equation*}
                f_Y(y) = \int_{y-1}^{-y+1} \frac{1}{2} ~d{x} = \left[\frac{1}{2}x\right]_{y-1}^{-y+1} = \frac{1}{2}(-y+1) - \frac{1}{2}(y-1) = 1-y.
            \end{equation*}
        \end{itemize}

        Por tanto, tenemos que las funciones de densidad marginales son:
        \begin{equation*}
            f_X(x) = 1-|x|, \quad f_Y(y) = 1-|y|.
        \end{equation*}

        Por tanto, tenemos que:
        \begin{equation*}
            f_X(x) f_Y(y) = (1-|x|)(1-|y|) = 1-|x|-|y|+|x||y| \qquad \forall x\in [-1,1], y\in [-x,x].
        \end{equation*}

        Tomando como ejemplo el origen, tenemos que:
        \begin{equation*}
            \frac{1}{2} = f(0,0) \neq f_X(0)f_Y(0) = 1.
        \end{equation*}

        \item $f(x, y) = 1$, si $(x, y)$ pertenece al cuadrado de vértices $(0,0);(0,1);(1,0);(1,1)$.
        
        El recinto descrito es:
        \begin{figure}[H]
            \centering
            \begin{tikzpicture}
                \begin{axis}[
                    axis lines = center,
                    xlabel = $x$,
                    ylabel = $y$,
                    xmin = -0.5, xmax = 1.5,
                    ymin = -0.5, ymax = 1.5,
                    xtick = {0,1},
                    ytick = {0,1},
                    xticklabels = {$0$,$1$},
                    yticklabels = {$0$,$1$},
                    axis equal image
                ]
                    \addplot [
                        fill=gray,
                        fill opacity=0.5
                    ] coordinates {
                        (0,0) (0,1) (1,1) (1,0)
                    } -- cycle;
                \end{axis}
            \end{tikzpicture}
        \end{figure}

        Definimos funciones auxiliares:
        \Func{h_1,h_2}{\bb{R}}{\bb{R}}{t}{\begin{cases} 1 & t\in [0,1] \\ 0 & t\notin [0,1] \end{cases}}

        Tenemos que:
        \begin{equation*}
            1=f(x,y)=h_1(x)h_1(y) \qquad \forall x,y\in [0,1].
        \end{equation*}

        Por tanto, las variables $X$ e $Y$ son independientes.
    \end{enumerate}
\end{ejercicio}

\begin{ejercicio}
    Sean $X_1$ y $X_2$ variables aleatorias independientes con distribución Binomial con parámetros $n_i$, $i = 1,2$, y $p = \nicefrac{1}{2}$. Calcular la distribución de $X_1 - X_2 + n_2$.\\

    Sea $Z=X_1-X_2+n_2$. Calculamos generatriz de momentos de $Z$:
    \begin{equation*}
        M_Z(t) = E[e^{tZ}] = E[e^{t(X_1-X_2+n_2)}] = E[e^{tX_1}e^{-tX_2}e^{tn_2}] = e^{tn_2}E[e^{tX_1}e^{-tX_2}].
    \end{equation*}

    Fijado $t\in \bb{R}$, como la transformación $u\mapsto e^{tu}$ es medible, tenemos que $e^{tX_1}$ y $e^{-tX_2}$ son independientes. Por tanto, usando el Teorema de la Multiplicación de las Esperanzas, tenemos que:
    \begin{equation*}
        M_Z(t) = e^{tn_2}E[e^{tX_1}]E[e^{-tX_2}] = e^{tn_2}M_{X_1}(t)M_{X_2}(-t)
    \end{equation*}

    Usando la generatriz de momentos de la Binomial, tenemos que:
    \begin{align*}
        M_Z(t) &= e^{tn_2}\left(1+p(e^t-1)\right)^{n_1}\left(1+p(e^{-t}-1)\right)^{n_2} =\\
        &=(1+p(e^t-1))^{n_1}[e^t(1+p(e^{-t}-1))]^{n_2}=\\
        &=(1+p(e^t-1))^{n_1}(1+p(e^{t}-1))^{n_2}
        \AstIg\\&\AstIg
        (1+p(e^t-1))^{n_1+n_2}.
    \end{align*}
    donde en $(\ast)$ hemos usado que $p=\nicefrac{1}{2}$.

    Por tanto, la variable $Z$ sigue una distribución Binomial con parámetros $n_1+n_2$ y $p=\nicefrac{1}{2}$. Es decir,
    \begin{equation*}
        X_1-X_2+n_2\sim B(n_1+n_2,p).
    \end{equation*}
\end{ejercicio}

\begin{ejercicio}
    La demanda en miles de toneladas de un producto, $X$, y su precio por kilogramo en euros, $Y$, tienen por función de densidad conjunta
    \[
        f(x, y) = kx^2(1-x)^3y^3(1-y)^2, \quad x, y \in~]0,1[.
    \]
    Calcular la constante $k$ para que $f$ sea una función de densidad de probabilidad, y determinar si $X$ e $Y$ son independientes. Obtener la función de densidad de probabilidad del precio para una demanda fija.\\

    Para que $f$ sea una función de densidad de probabilidad, debe cumplir que:
    \begin{equation*}
        \int_{-\infty}^{\infty}\int_{-\infty}^{\infty} f(x,y) ~d{x}~d{y} = 1.
    \end{equation*}

    Por tanto, calculamos la constante $k$:
    \begin{align*}
        1 &= \int_{0}^{1}\int_{0}^{1} kx^2(1-x)^3y^3(1-y)^2 ~d{x}~d{y} = k\int_{0}^{1}x^2(1-x)^3 ~d{x}\int_{0}^{1}y^3(1-y)^2 ~d{y} =\\
        &= k\int_{0}^{1}x^2(1-3x+3x^2-x^3) ~d{x}\int_{0}^{1}y^3(1-2y+y^2) ~d{y} =\\
        &= k\int_{0}^{1}(x^2-3x^3+3x^4-x^5) ~d{x}\int_{0}^{1}(y^3-2y^4+y^5) ~d{y} =\\
        &= k\left[\frac{1}{3}x^3-\frac{3}{4}x^4+\frac{3}{5}x^5-\frac{1}{6}x^6\right]_{0}^{1}\left[\frac{1}{4}y^4-\frac{2}{5}y^5+\frac{1}{6}y^6\right]_{0}^{1} =\\
        &= k\left(\frac{1}{3}-\frac{3}{4}+\frac{3}{5}-\frac{1}{6}\right)\left(\frac{1}{4}-\frac{2}{5}+\frac{1}{6}\right) =\\
        &= k\cdot \frac{1}{60}\cdot \frac{1}{60} = \frac{k}{3600} \Longrightarrow k=3600.
    \end{align*}

    Comprobamos ahora si $X$ e $Y$ son independientes. Para ello, definimos funciones auxiliares:
    \Func{h_1}{\bb{R}}{\bb{R}}{x}{\begin{cases} x^2(1-x)^3 & x\in~]0,1[ \\ 0 & x\notin~]0,1[ \end{cases}}
    \Func{h_2}{\bb{R}}{\bb{R}}{y}{y^3(1-y)^2}

    Por tanto, tenemos que:
    \begin{equation*}
        f(x,y) = h_1(x)h_2(y) \qquad \forall x,y\in~\bb{R}
    \end{equation*}

    Por tanto, las variables $X$ e $Y$ son independientes. Buscamos ahora la función de densidad de probabilidad del precio para una demanda fija $D$. Para ello, como $X$ e $Y$ son independientes, tenemos que:
    \begin{align*}
        f_{Y\mid X=D}(y) &= f_Y(y) = \int_{-\infty}^{\infty} f(x,y) ~d{x} = ky^3(1-y)^2\int_{0}^{1}x^2(1-x)^3 ~d{x} =\\&= 3600\cdot y^3(1-y)^2 \cdot \dfrac{1}{60} = 60\cdot y^3(1-y)^2
    \end{align*}
\end{ejercicio}
    %\section{Esperanza Condicionada}

\begin{ejercicio}
    Sea $X$ una variable aleatoria que se distribuye uniformemente en el intervalo $\left]0,1\right[$. Comprobar si las variables aleatorias $X$ y $|\nicefrac{1}{2}-X|$ son incorreladas.
\end{ejercicio}

\begin{ejercicio}
    Calcular las curvas de regresión y las razones de correlación para las siguientes distribuciones, comentando los resultados.
    \begin{enumerate}
        \item Considerar las distribución conjunta $(X,Y)$ con función de masa de probabilidad dada por:
        \begin{equation*}
            \begin{array}{c|ccc||c}
                X\setminus Y & 10 & 15 & 20 & \\
                \hline
                1 & 0 & \nicefrac{2}{6} & 0 & \nicefrac{2}{6}\\
                2 & \nicefrac{1}{6} & 0 & 0 & \nicefrac{1}{6}\\
                3 & 0 & 0 & \nicefrac{3}{6} & \nicefrac{3}{6}\\
                \hline \hline
                & \nicefrac{1}{6} & \nicefrac{2}{6} & \nicefrac{3}{6} & 1
            \end{array}
        \end{equation*}

        Tras haber calculado las distribuciones marginales, calculamos ahora las distribuciones condicionadas.
        La siguiente tabla muestra la distribución condicionada de $Y$ dado $X$, $P[Y = y\mid X = x]$:
        \begin{equation*}
            \begin{array}{c|ccc}
                X\setminus Y & 10 & 15 & 20 \\
                \hline
                1 & 0 & 1 & 0\\
                2 & 1 & 0 & 0\\
                3 & 0 & 0 & 1
            \end{array}
        \end{equation*}

        La distribución condicionada de $X$ dado $Y$, $P[X = x\mid Y = y]$ viene dada por la misma tabla, ya que en este caso tenemos que:
        \begin{equation*}
            P[X = x\mid Y = y] = P[Y = y\mid X = x] \qquad \forall x, y
        \end{equation*}

        Calculemos ahora las curvas de regresión y las razones de correlación.
        \begin{itemize}
            \item Curva de regresión de $Y$ sobre $X$:
            \begin{equation*}
                \wh{Y}(x) = E[Y\mid X = x] = \sum_{y} y P[Y = y\mid X = x] \qquad \forall x\in E_x
            \end{equation*}
    
            Por tanto, la curva de regresión de $Y$ sobre $X$ es:
            \begin{align*}
                \wh{Y}(1) &= 15\\
                \wh{Y}(2) &= 10\\
                \wh{Y}(3) &= 20
            \end{align*}

            Para calcular la razón de correlación de $Y$ sobre $X$,
            hay dos opciones:
            \begin{description}
                \item[Opción 1.] Método rutinario.
                
                Usamos la fórmula:
                \begin{equation*}
                    \eta^2_{Y/X} = \dfrac{\Var(E[Y\mid X])}{\Var(Y)}
                \end{equation*}
                \begin{itemize}
                    \item Calculemos $E[Y]$:
                    \begin{align*}
                        E[Y] &= \sum_{y} y P[Y = y]\\
                        &= 10 \cdot \nicefrac{1}{6} + 15 \cdot \nicefrac{2}{6} + 20 \cdot \nicefrac{3}{6}\\
                        &= \nicefrac{10}{6} + 5 + 10\\
                        &= \frac{50}{3}
                    \end{align*}

                    \item Calculemos ahora $E[Y^2]$:
                    \begin{align*}
                        E[Y^2] &= \sum_{y} y^2 P[Y = y]\\
                        &= 10^2 \cdot \nicefrac{1}{6} + 15^2 \cdot \nicefrac{2}{6} + 20^2 \cdot \nicefrac{3}{6}\\
                        &= 100 \cdot \nicefrac{1}{6} + 225 \cdot \nicefrac{2}{6} + 400 \cdot \nicefrac{3}{6}\\
                        &= \nicefrac{100}{6} + 75 + 200\\
                        &= \frac{875}{3}
                    \end{align*}

                    \item Calculemos ahora $E[(E[Y\mid X])^2]$:
                    \begin{align*}
                        E[(E[Y\mid X])^2] &= \sum_{x} (E[Y\mid X = x])^2 P[X = x]\\
                        &= 10^2 \cdot \nicefrac{1}{6} + 15^2 \cdot \nicefrac{2}{6} + 20^2 \cdot \nicefrac{3}{6}\\
                        &= \frac{875}{3} = E[Y^2]
                    \end{align*}

                    \item Calculemos ahora $E[E[Y\mid X]]$.
                    \begin{equation*}
                        E[E[Y\mid X]] = E[Y]
                    \end{equation*}
                \end{itemize}

                Por tanto, usando lo anterior, tenemos:
                \begin{align*}
                    \eta^2_{Y/X} &= \dfrac{\Var(E[Y\mid X])}{\Var(Y)}
                    = \dfrac{E[(E[Y\mid X])^2] - E[E[Y\mid X]]^2}{E[Y^2] - E[Y]^2}\\
                    &= \dfrac{E[Y^2] - E[Y]^2}{E[Y^2] - E[Y]^2} = 1
                \end{align*}

                \item[Opción 2.] Razonando por dependencia funcional.
                
                En este caso, vemos que $Y$ es función de $X$. Por tanto:
                \begin{equation*}
                    \eta^2_{Y/X} = 1
                \end{equation*}
            \end{description}

            \item Curva de regresión de $X$ sobre $Y$:
            \begin{equation*}
                \wh{X}(y) = E[X\mid Y = y] = \sum_{x} x P[X = x\mid Y = y] \qquad \forall y\in E_y
            \end{equation*}

            Por tanto, la curva de regresión de $X$ sobre $Y$ es:
            \begin{align*}
                \wh{X}(10) &= 2\\
                \wh{X}(15) &= 1\\
                \wh{X}(20) &= 3
            \end{align*}

            De nuevo, razonando ahora por dependencia funcional, tenemos que:
            \begin{equation*}
                \eta^2_{X/Y} = 1
            \end{equation*}
        \end{itemize}

        Como vemos, en este caso, hay dependencia recíproca entre $X$ e $Y$. Por tanto,
        el ajuste es el idea, ya que $Y=f(X)$ y $X=g(Y)$.
        Cada una explica la totalidad de la variabilidad de la otra.


        \item Considerar las distribución conjunta $(X,Y)$ con función de masa de probabilidad dada por:
        \begin{equation*}
            \begin{array}{c|cccc||c}
                X\setminus Y & 10 & 15 & 20 & 25 &\\
                \hline
                1 & 0 & \nicefrac{3}{7} & 0 & \nicefrac{1}{7} & \nicefrac{4}{7}\\
                2 & 0 & 0 & \nicefrac{1}{7} & 0 & \nicefrac{1}{7}\\
                3 & \nicefrac{2}{7} & 0 & 0 & 0 & \nicefrac{2}{7}\\
                \hline \hline
                & \nicefrac{2}{7} & \nicefrac{3}{7} & \nicefrac{1}{7} & \nicefrac{1}{7} & 1
            \end{array}
        \end{equation*}

        Tras haber calculado las distribuciones marginales, calculamos ahora las distribuciones condicionadas.
        La siguiente tabla muestra la distribución condicionada de $Y$ dado $X$, $P[Y = y\mid X = x]$:
        \begin{equation*}
            \begin{array}{c|cccc}
                X\setminus Y & 10 & 15 & 20 & 25\\
                \hline
                1 & 0 & \nicefrac{3}{4} & 0 & \nicefrac{1}{4}\\
                2 & 0 & 0 & 1 & 0\\
                3 & 1 & 0 & 0 & 0
            \end{array}
        \end{equation*}

        La distribución condicionada de $X$ dado $Y$, $P[X = x\mid Y = y]$ viene dada por la siguiente tabla:
        \begin{equation*}
            \begin{array}{c|ccc}
                X\setminus Y & 10 & 15 & 20\\
                \hline
                1 & 0 & 1 & 0\\
                2 & 0 & 0 & 1\\
                3 & 1 & 0 & 0
            \end{array}
        \end{equation*}

        Calculemos ahora las curvas de regresión y las razones de correlación.
        \begin{itemize}
            \item Curva de regresión de $Y$ sobre $X$:
            \begin{equation*}
                \wh{Y}(x) = E[Y\mid X = x] = \sum_{y} y P[Y = y\mid X = x] \qquad \forall x\in E_x
            \end{equation*}

            Por tanto, la curva de regresión de $Y$ sobre $X$ es:
            \begin{align*}
                \wh{Y}(1) &= 15\cdot \nicefrac{3}{4} + 25\cdot \nicefrac{1}{4} = 17.5\\
                \wh{Y}(2) &= 20\\
                \wh{Y}(3) &= 10
            \end{align*}

            Para calcular la razón de correlación de $Y$ sobre $X$, tenemos que:
            \begin{equation*}
                \eta^2_{Y/X} = \dfrac{\Var(E[Y\mid X])}{\Var(Y)}
            \end{equation*}
            \begin{itemize}
                \item Calculemos $E[Y]$:
                \begin{align*}
                    E[Y] &= \sum_{y} y P[Y = y]\\
                    &= 10 \cdot \nicefrac{2}{7} + 15 \cdot \nicefrac{3}{7} + 20 \cdot \nicefrac{1}{7} + 25 \cdot \nicefrac{1}{7}\\
                    &= \frac{110}{7}
                \end{align*}

                \item Calculemos ahora $E[Y^2]$:
                \begin{align*}
                    E[Y^2] &= \sum_{y} y^2 P[Y = y]\\
                    &= 10^2 \cdot \nicefrac{2}{7} + 15^2 \cdot \nicefrac{3}{7} + 20^2 \cdot \nicefrac{1}{7} + 25^2 \cdot \nicefrac{1}{7}\\
                    &= \frac{1900}{7}
                \end{align*}

                \item Calculemos ahora $E[(E[Y\mid X])^2]$:
                \begin{align*}
                    E[(E[Y\mid X])^2] &= \sum_{x} (E[Y\mid X = x])^2 P[X = x]\\
                    &= 17.5^2 \cdot \nicefrac{4}{7} + 20^2 \cdot \nicefrac{1}{7} + 10^2 \cdot \nicefrac{2}{7}\\
                    &= \frac{1825}{7}
                \end{align*}

                \item Calculemos ahora $E[E[Y\mid X]]$.
                \begin{equation*}
                    E[E[Y\mid X]] = E[Y]
                \end{equation*}
            \end{itemize}

            Por tanto, usando lo anterior, tenemos:
            \begin{align*}
                \eta^2_{Y/X} &= \dfrac{\Var(E[Y\mid X])}{\Var(Y)}
                = \dfrac{E[(E[Y\mid X])^2] - E[E[Y\mid X]]^2}{E[Y^2] - E[Y]^2}\\
                &= \dfrac{E[(E[Y\mid X])^2] - E[Y]^2}{E[Y^2] - E[Y]^2}\\
                &= \dfrac{\dfrac{1825}{7} - \left(\dfrac{110}{7}\right)^2}{\dfrac{1900}{7} - \left(\dfrac{110}{7}\right)^2}\\
                &= \dfrac{9}{16} \approx 0.5625
            \end{align*}

            Por tanto, tenemos que $X$ explica el $56.25\%$ de la variabilidad de $Y$. Tenemos entonces que no es un ajuste ideal.

            \item Curva de regresión de $X$ sobre $Y$:
            
            En este caso, tenemos que $X=f(Y)$, por lo que:
            \begin{align*}
                \wh{X}(10) &= 3\\
                \wh{X}(15) &= 1\\
                \wh{X}(20) &= 2\\
                \wh{X}(25) &= 1
            \end{align*}

            Por tanto, como $X$ es función de $Y$, tenemos que:
            \begin{equation*}
                \eta^2_{X/Y} = 1
            \end{equation*}

            Tenemos que $Y$ explica la totalidad de la variabilidad de $X$, por lo que el ajuste es el ideal.
        \end{itemize}


    \end{enumerate}
\end{ejercicio}

\begin{ejercicio}
    % // TODO: Hacer
    Sea $X$ el número de balanzas e $Y$ el número de dependientes en los puntos de venta de un mercado. Determinar las rectas de regresión y el grado de ajuste a la distribución, si la función masa de probabilidad de $(X,Y)$ viene dada por:
    \begin{equation*}
        \begin{array}{c|cccc}
            X\setminus Y & 1 & 2 & 3 & 4\\
            \hline
            1 & \nicefrac{1}{24} & \nicefrac{2}{24} & 0 & 0\\
            2 & \nicefrac{1}{24} & \nicefrac{2}{24} & \nicefrac{3}{24} & \nicefrac{1}{24}\\
            3 & 0 & \nicefrac{1}{24} & \nicefrac{2}{24} & \nicefrac{6}{24}\\
            4 & 0 & 0 & \nicefrac{2}{24} & \nicefrac{3}{24}
        \end{array}
    \end{equation*}
\end{ejercicio}

\begin{ejercicio}
    Sea $(X,Y)$ un vector aleatorio con valores en $\{(x, y) \in \mathbb{R}^2/0 < x < y < 2\}$ y función de densidad constante. Calcular:
    \begin{enumerate}
        \item Curvas y rectas de regresión de $X$ sobre $Y$ y de $Y$ sobre $X$.
        \item Razones de correlación y coeficiente de correlación lineal.
        \item Error cuadrático medio asociado a cada una de las funciones de regresión.
    \end{enumerate}
\end{ejercicio}

\begin{ejercicio}
    Dada la función masa de probabilidad del vector aleatorio $(X,Y)$
    \begin{equation*}
        \begin{array}{c|cccc}
            X\setminus Y & 0 & 1 & 2 & 3\\
            \hline
            0 & 0.2 & 0.2 & 0.05 & 0\\
            1 & 0.1 & 0.1 & 0.1 & 0.05\\
            2 & 0 & 0.05 & 0.05 & 0.1
        \end{array}
    \end{equation*}
    % // TODO: Hacer
    \begin{enumerate}
        \item Determinar la aproximación lineal mínimo cuadrática de $Y$ para $X = 1$.
        \item Determinar la aproximación mínimo cuadrática de $Y$ para $X = 1$.
    \end{enumerate}
\end{ejercicio}

\begin{ejercicio}
    Dadas las siguientes distribuciones, determinar qué variable, $X$ ó $X'$, aproxima mejor a la variable $Y$:
    \begin{equation*}
        \begin{array}{c|ccc}
            X\setminus Y & 0 & 1 & 2\\
            \hline
            0 & \nicefrac{1}{5} & 0 & 0\\
            2 & 0 & \nicefrac{1}{5} & 0\\
            3 & \nicefrac{1}{5} & 0 & \nicefrac{1}{5}\\
            4 & 0 & 0 & \nicefrac{1}{5}
        \end{array}
        \qquad
        \begin{array}{c|ccc}
            X'\setminus Y & 0 & 1 & 2\\
            \hline
            0 & \nicefrac{1}{5} & 0 & \nicefrac{1}{5}\\
            2 & 0 & \nicefrac{1}{5} & 0\\
            3 & \nicefrac{1}{5} & 0 & 0\\
            4 & 0 & 0 & \nicefrac{1}{5}
        \end{array}
    \end{equation*}
\end{ejercicio}

\begin{ejercicio} \label{ej:4.7}
    Probar que las variables $X = U + V$ e $Y = U - V$ son incorreladas, pero no independientes, si $U$ y $V$ son variables aleatorias con función de densidad conjunta:
    \begin{equation*}
        f_{U,V}(u, v) = \exp(-u-v), \quad u, v > 0.
    \end{equation*}
\end{ejercicio}

\begin{ejercicio}
    % // TODO: Hacer
    Sea $X$ una variable aleatoria con distribución uniforme en el intervalo $[0,1]$, y sea $Y$ una variable aleatoria continua tal que
    \begin{equation*}
        f_{Y\mid X=x}(y) = \begin{cases}
            \nicefrac{1}{x^2} & y \in \left[0, x^2\right]\\
            0 & \text{en caso contrario}
        \end{cases}
    \end{equation*}
    \begin{enumerate}
        \item\label{ej:4.8.a} Calcular la función de densidad de probabilidad conjunta de $X$ e $Y$. Calcular la función de densidad de probabilidad marginal de $Y$.
        \item Calcular $E[X\mid Y = y]$ y $E[Y\mid X = x]$.
        \item Para la misma densidad de probabilidad condicionada del apartado \ref{ej:4.8.a}, considerando ahora que $X$ es una variable aleatoria continua con función de densidad de probabilidad:
        \begin{equation*}
            f_X(x) = \begin{cases}
                3x^2 & x \in \left[0,1\right]\\
                0 & \text{en caso contrario}
            \end{cases}
        \end{equation*}
        Calcular de nuevo la función de densidad de probabilidad conjunta de $X$ e $Y$, y la función de densidad de probabilidad marginal de $Y$, así como $E[X\mid Y = y]$ y $E[Y\mid X = x]$.
    \end{enumerate}
\end{ejercicio}

\begin{ejercicio}
    Sean $X$ e $Y$ variables aleatorias con función de densidad conjunta:
    \begin{equation*}
        f_{(X,Y)}(x, y) = \begin{cases}
            x+y & (x, y) \in [0,1] \times [0,1]\\
            0 & \text{en caso contrario}
        \end{cases}
    \end{equation*}
    \begin{enumerate}
        \item Calcular la predicción mínimo cuadrática de $Y$ a partir de $X$ y el error cuadrático medio asociado.
        \item Si se observa $X = \nicefrac{1}{2}$, qué predicción de $Y$ tiene menor error cuadrático medio? Calcular dicho error.
        \item Supóngase que una persona debe pagar una cantidad $C$ por la oportunidad de observar el valor de $X$ antes de predecir el valor de $Y$, o puede simplemente predecir el valor de $Y$ sin observar $X$. Si la persona considera que su pérdida total es la suma de $C$ y el error cuadrático medio de su predicción, qué valor máximo de $C$ estaría dispuesta a pagar?
    \end{enumerate}
\end{ejercicio}

\begin{ejercicio}
    Sea $(X,Y)$ un vector aleatorio con función de densidad:
    \begin{equation*}
        f_{(X,Y)}(x, y) = \exp(-y), \quad 0 < x < y
    \end{equation*}
    Obtener y representar las rectas y curvas de regresión. Calcular el coeficiente de correlación lineal, las razones de correlación y el error cuadrático medio cometido al predecir cada variable según cada una de las funciones de regresión. Interpretar los resultados.
\end{ejercicio}

\begin{ejercicio}
    Sea $(X,Y)$ un vector aleatorio con función de densidad uniforme sobre el cuadrado unidad. Obtener y representar las rectas y curvas de regresión. Calcular el coeficiente de correlación lineal, las razones de correlación y el error cuadrático medio cometido al predecir cada variable según cada una de las funciones de regresión. Interpretar los resultados.
\end{ejercicio}

\begin{ejercicio}
    Supongamos que $(X,Y)$ tiene función de densidad de probabilidad conjunta dada por:
    \begin{equation*}
        f_{(X,Y)}(x, y) = \begin{cases}
            1, & |y| < x, x \in \left]0,1\right[\\
            0, & \text{en otro caso}
        \end{cases}
    \end{equation*}
    Obtener y representar las rectas y curvas de regresión. Calcular el coeficiente de correlación lineal, las razones de correlación y el error cuadrático medio cometido al predecir cada variable según cada una de las funciones de regresión. Interpretar los resultados.
\end{ejercicio}

\begin{ejercicio}
    Sea $(X,Y)$ un vector aleatorio distribuido uniformemente en el paralelogramo de vértices $(0,0)$; $(2,0)$; $(3,1)$ y $(1,1)$. Calcular el error cuadrático medio asociado a la predicción de $X$ a partir de la variable $Y$ y a la predicción de $Y$ a partir de la variable aleatoria $X$. Determinar la predicción más fiable a la vista de los resultados obtenidos.
\end{ejercicio}

\begin{ejercicio}
    Sea $(X,Y)$ un vector aleatorio con rectas de regresión
    \begin{equation*}
        x+4y = 1 \qquad x+5y = 2
    \end{equation*}
    \begin{enumerate}
        \item Cuál es la recta de regresión de $Y$ sobre $X$?
        \item Calcular el coeficiente de correlación lineal y la proporción de varianza de cada variable que queda explicada por la regresión lineal.
        \item Calcular las medias de ambas variables
    \end{enumerate}
\end{ejercicio}
\end{document}
