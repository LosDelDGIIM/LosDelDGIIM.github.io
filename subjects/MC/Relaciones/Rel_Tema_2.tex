\newpage
\section{Autómatas Finitos}

\begin{comment}
\begin{tikzpicture}
    \node[state, initial] (q1) {$q_1$};
    \node[state, accepting, right of=q1] (q2) {$q_2$};
    \node[state, right of=q2] (q3) {$q_3$};
    \draw (q1) edge[loop above] node{0} (q1)
    (q1) edge[above] node{1} (q2)
    (q2) edge[loop above] node{1} (q2)
    (q2) edge[bend left, above] node{0} (q3)
    (q3) edge[bend left, below] node{0, 1} (q2);
\end{tikzpicture}
\end{comment}

\begin{ejercicio}
    Considera el siguiente Autómata Finito Determinista (AFD) $M = (Q, A, \delta, q_0, F)$, donde:
    \begin{itemize}
        \item $Q = \{q_0, q_1, q_2\}$
        \item $A = \{0, 1\}$
        \item La función de transición viene dada por:
        \begin{align*}
            \delta(q_0, 0) &= q_1, & \delta(q_0, 1) &= q_0 \\
            \delta(q_1, 0) &= q_2, & \delta(q_1, 1) &= q_0 \\
            \delta(q_2, 0) &= q_2, & \delta(q_2, 1) &= q_2
        \end{align*}
        \item $F = \{q_2\}$
    \end{itemize}
    Describe informalmente el lenguaje aceptado.
\end{ejercicio}

\begin{ejercicio} \label{ej:1.2.2}
    Dado el AFD de la Figura \ref{fig:ej:1.2.2}, describir el lenguaje aceptado por dicho autómata.
    \begin{figure}
        \centering
        \begin{tikzpicture}
            \node[state, initial] (q0) {$q_0$};
            \node[state, right of=q0] (q1) {$q_1$};
            \node[state, accepting, right of=q1] (q2) {$q_2$};

            \draw   (q0) edge[loop above] node{$b$} (q0)
                    (q0) edge[above] node{$a$} (q1)
                    (q1) edge[loop above] node{$b$} (q1)
                    (q1) edge[bend left, above] node{$a$} (q2)
                    (q2) edge[bend left, below] node{$a$} (q1)
                    (q2) edge[loop above] node{$b$} (q2);
        \end{tikzpicture}
        \caption{Autómata Finito Determinista del Ejercicio \ref{ej:1.2.2}}
        \label{fig:ej:1.2.2}
    \end{figure}
\end{ejercicio}