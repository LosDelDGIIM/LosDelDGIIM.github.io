\newpage
\section{Propiedades de los Lenguajes Regulares}

\begin{ejercicio}\label{ej:1.3.1}
    Determinar si los siguientes lenguajes son regulares o libres de contexto. Justificar las respuestas.
    \begin{enumerate}
        \item $L_1=\{0^i b^j \mid i = 2j \text{\ ó\ } 2i=j\}$
        
        Para todo $n\in \mathbb{N}$, consideramos la palabra $z=0^{2n}b^n\in L_1$ con $|z|=3n\geq n$. Toda descomposición $z=uvw$, con $u,v,w\in \{0,b\}^\ast$, $|uv|\leq n$ y $|v|\geq 1$ debe cumplir que:
        \begin{equation*}
            u=0^k \quad v=0^l \quad w=0^{2n-k-l}b^n \qquad \text{con } l,k\in \bb{N}\cup \{0\},~l\geq 1,~k+l\leq n
        \end{equation*}

        Para $i=2$, tenemos que $uv^iw=0^{k+2l+2n-k-l}b^n=0^{2n+l}b^n\notin L_1$, ya que, como $l\geq 1$:
        \begin{equation*}
            2n+l\neq 2n \quad \text{y} \quad 2(2n+l)\neq n
        \end{equation*}

        Por tanto, por el recíproco del Lema de Bombeo, no es regular. Veamos ahora que sí es libre de contexto. Consideramos la gramática $G=(\{S,S_1,S_2\},\{0,b\},P,S)$ con $P$ definido por:
        \begin{align*}
            S &\rightarrow S_1 \mid S_2 \\
            S_1 &\rightarrow 00S_1b \mid \varepsilon \\
            S_2 &\rightarrow 0S_2bb \mid \varepsilon
        \end{align*}

        Tenemos que $G$ es una gramática libre de contexto tal que $\cc{L}(G)=L_1$, por lo que $L_1$ es libre de contexto.

        \item $L_2=\{uu^{-1} \mid u \in {\{0,1\}}^\ast, |u|\leq 1000\}$
        
        Consideramos el lenguaje auxiliar:
        \begin{equation*}
            L_2' = \{u\in \{0,1\}^\ast \mid |u|\leq 2\cdot 1000\}
        \end{equation*}

        Veamos que $L_2'$ es finito. Como el número de combinaciones de $n$ elementos de $\{0,1\}$ es $2^n$, entonces el número de palabras de longitud menor o igual a $2\cdot 1000$ es:
        \begin{equation*}
            |L_2'| = \sum_{i=0}^{2\cdot 1000} 2^i <\infty
        \end{equation*}

        Por tanto, como $L_2\subset L_2'$ finito, tenemos que $L_2$ es finito y por tanto regular.

        \item $L_3=\{uu^{-1} \mid u \in {\{0,1\}}^\ast, |u|\geq 1000\}$
        
        Sabemos que el siguiente lenguaje es independiente del contexto:
        \begin{equation*}
            L_3' = \{uu^{-1} \mid u \in {\{0,1\}}^\ast\}
        \end{equation*}

        Además, tenemos que $L_3'=L_2\cup L_3$. Supongamos que $L_3$ es regular. Entonces, como $L_2$ es regular, tendríamos que $L_3'$ es regular, lo cual es una contradicción. Por tanto, $L_3$ no es regular.
        Para ver que es libre de contexto, consideramos la gramática $G=(V,\{0,1\},P,S)$ con:
        \begin{equation*}
            V=\{S\}\cup \{A_i\mid i\in \{1,\dots,1000\}\}
        \end{equation*}

        Tenemos que $P$ está definido por:
        \begin{align*}
            S &\rightarrow 0A_10 \mid 1A_11, \\
            A_i &\rightarrow 0A_{i+1}0 \mid 1A_{i+1},\qquad i\in \{1,\dots,999\},\\
            A_{1000} &\rightarrow 0A_{1000}0 \mid 1A_{1000}1 \mid \varepsilon
        \end{align*}

        Notemos que, en $A_i$, ya hemos leído $i$ caracteres de $u$ (la palabra que forma la mitad del palíndromo).
        Una vez hemos llegado a $A_{1000}$, hemos leído $1000$ caracteres de $u$. Por tanto, podemos añadir los que queramos sin restricción, y podemos también terminar.
        Como $L_3=\cc{L}(G)$, tenemos que $L_3$ es independiente del contexto.
        
        \item $L_4=\{0^i 1^j 2^k \mid i = j \text{\ ó\ } j=k\}$
        
        Para todo $n\in \mathbb{N}$, consideramos la palabra $z=0^n1^n2^{2n}\in L_4$ con $|z|=3n\geq n$. Toda descomposición $z=uvw$, con $u,v,w\in \{0,1,2\}^\ast$, $|uv|\leq n$ y $|v|\geq 1$ debe cumplir que:
        \begin{equation*}
            u=0^k \quad v=0^l \quad w=0^{n-k-l}1^n2^{2n} \qquad \text{con } l,k\in \bb{N}\cup \{0\},~l\geq 1,~k+l\leq n
        \end{equation*}

        Para $i=2$, tenemos que $uv^iw=0^{k+2l+n-k-l}1^n2^{2n}=0^{n+l}1^n2^{2n}\notin L_4$, ya que, como $l\geq 1$:
        \begin{equation*}
            n+l\neq n \quad \text{y} \quad n\neq 2n
        \end{equation*}

        Por tanto, por el recíproco del Lema de Bombeo, no es regular. Veamos ahora que sí es libre de contexto. Consideramos la gramática $G=(V,\{0,1,2\},P,S)$ donde $V=\{S,S_1,S_2,A_0,A_2\}$ y $P$ está definido por:
        \begin{align*}
            S &\rightarrow S_1 A_2 \mid A_0 S_2 \\
            S_1 &\rightarrow 0S_1 1 \mid \varepsilon \\
            A_2 &\rightarrow 2 A_2 \mid \varepsilon \\
            S_2 &\rightarrow 1S_2 2 \mid \varepsilon \\
            A_0 &\rightarrow 0 A_0 \mid \varepsilon
        \end{align*}

        Tenemos que $G$ es una gramática libre de contexto tal que $\cc{L}(G)=L_4$, por lo que $L_4$ es libre de contexto.
    \end{enumerate}
\end{ejercicio}

\begin{ejercicio}\label{ej:1.3.2}
    Determinar qué lenguajes son regulares o libres de contexto de los siguientes:
    \begin{enumerate}
        \item $\{u0u^{-1}\mid u \in {\{0,1\}}^\ast\}$
        
        Para todo $n\in \mathbb{N}$, consideramos la palabra $z=0^{n}10^n\in L$ con $|z|=2n+1\geq n$. Toda descomposición $z=uvw$, con $u,v,w\in \{0,1\}^\ast$, $|uv|\leq n$ y $|v|\geq 1$ debe cumplir que:
        \begin{equation*}
            u=0^k \quad v=0^l \quad w=0^{n-k-l}10^n \qquad \text{con } l,k\in \bb{N}\cup \{0\},~l\geq 1,~k+l\leq n
        \end{equation*}

        Para $i=2$, tenemos que $uv^iw=0^{k+2l+n-k-l}10^n=0^{n+l}10^n\notin L$, ya que, como $l\geq 1$:
        \begin{equation*}
            n+l\neq n
        \end{equation*}

        Por tanto, por el recíproco del Lema de Bombeo, no es regular. Veamos ahora que sí es libre de contexto. Consideramos la gramática $G=(\{S\},\{0,1\},P,S)$, con $P$ definido por:
        \begin{align*}
            S &\rightarrow 0S0 1S1 \mid 0 \mid 1
        \end{align*}

        Tenemos que $G$ es una gramática libre de contexto tal que $\cc{L}(G)=L$, por lo que $L$ es libre de contexto.

        \item Números en binario que sean múltiplos de 4
        
        Tenemos que todos los números en binario que son múltiplos de 4 terminan en $00$, por lo que vienen dados por la siguiente expresión regular:
        \begin{equation*}
            0^* + (1+0)^{\ast}00
        \end{equation*}

        Notemos que hemos incluido $0^*$, porque el $0$ también es múltiplo de $4$. Por tanto, es regular.

        \item Palabras de ${\{0,1\}}^\ast$ que no contienen la subcadena $0110$.
        
        Notemos que podríamos dar un autómata que reconociese ese lenguaje, pero no es la opción más sencilla.
        Veamos en primer lugar que el lenguaje formado por las palabras que sí contienen la subcadena $0110$ es regular dando una expresión regular asociada a él:
        \begin{equation*}
            (0+1)^*\red{0110}(0+1)^*
        \end{equation*}

        Como el lenguaje descrito es su complementario y el complementario de un regular es regular, tenemos que el lenguaje dado es regular.
    \end{enumerate}
\end{ejercicio}

\begin{ejercicio}\label{ej:1.3.3}
    Determinar qué lenguajes son regulares y qué lenguajes son libres de contexto entre los siguientes:
    \begin{enumerate}
        \item \label{ej:1.3.3.1}
        Conjunto de palabras sobre el alfabeto $\{0,1\}$ en las que cada 1 va precedido por un número par de ceros.
        
        Un reconocedor del lenguaje es el autómata de la Figura~\ref{fig:1.3.3-1},
        que tiene los siguientes estados:
        \begin{itemize}
            \item $q_0$: Llevo un número par de ceros consecutivos, puedo leer un $1$.
            \item $q_1$: Llevo un número impar de ceros consecutivos, no puedo leer un $1$.
            \item $q_2$: Acabo de leer un $1$.
        \end{itemize}
        \begin{figure}
            \centering
            \begin{tikzpicture}
                \node[state, accepting, initial] (q0) {$q_0$};
                \node[state, accepting, right of=q0] (q1) {$q_1$};
                \node[state, accepting, below of=q0] (q2) {$q_2$};
                \node[state, below of=q1, error] (E) {$E$};

                \draw   (q0) edge[above] node{0} (q1)
                        (q0) edge[left] node{1} (q2)
                        (q1) edge[above, bend right] node{0} (q0)
                        (q1) edge[right] node{1} (E)
                        (q2) edge[above] node{0} (q1)
                        (q2) edge[above] node{1} (E)
                        (E) edge[loop right] node{0,1} (E);
            \end{tikzpicture}
            \caption{Autómata que reconoce el lenguaje del ejercicio~\ref{ej:1.3.3}.\ref{ej:1.3.3.1}.}
            \label{fig:1.3.3-1}
        \end{figure}

        Por tanto, como hemos dado un autómata que reconoce el lenguaje, es regular.
        
        \item Conjunto $\{0^i 1^{2j}0^{i+j} \mid i,j\geq 0\}$
        
        Usaremos el Lema de Bombeo para demostrar que no es regular. Para todo $n\in \mathbb{N}$, consideramos la palabra $z=0^n1^{2n}0^{2n}$ con $|z|=5n\geq n$. Toda descomposición $z=uvw$, con $u,v,w\in \{0,1\}^\ast$, $|uv|\leq n$ y $|v|\geq 1$ debe cumplir que:
        \begin{equation*}
            u=0^k \quad v=0^l \quad w=0^{n-k-l}1^{2n}0^{2n} \qquad \text{con } l,k\in \bb{N}\cup \{0\},~l\geq 1,~k+l\leq n
        \end{equation*}

        Para $i=2$, tenemos que $uv^iw=0^{k+2l+n-k-l}1^{2n}0^{2n}=0^{n+l}1^{2n}0^{2n}\notin L$, ya que, como $l\geq 1$:
        \begin{equation*}
            2n\neq n+n+l
        \end{equation*}

        Por tanto, por el recíproco del Lema de Bombeo, no es regular.
        Veamos ahora que es libre de contexto. Consideramos la gramática $G=(\{S,X\},\{0,1\},P,S)$, con $P$ definido por:
        \begin{align*}
            S &\rightarrow 0S0 \mid X,\\
            X &\rightarrow 11X0 \mid \veps.
        \end{align*}

        Tenemos que $G$ es una gramática libre de contexto tal que $\cc{L}(G)=L$, por lo que $L$ es libre de contexto.

        \item Conjunto $\{0^i 1^{j} 0^{i\ast j}\mid i,j\geq 0\}$
        
        Usaremos el Lema de Bombeo para demostrar que no es regular. Para todo $n\in \mathbb{N}$, consideramos la palabra $z=0^n1^{n}0^{n^2}$ con $|z|=n+n+n^2\geq n$. Toda descomposición $z=uvw$, con $u,v,w\in \{0,1\}^\ast$, $|uv|\leq n$ y $|v|\geq 1$ debe cumplir que:
        \begin{equation*}
            u=0^k \quad v=0^l \quad w=0^{n-k-l}1^{n}0^{n^2} \qquad \text{con } l,k\in \bb{N}\cup \{0\},~l\geq 1,~k+l\leq n
        \end{equation*}

        Para $i=2$, tenemos que $uv^iw=0^{k+2l+n-k-l}1^{n}0^{n^2}=0^{n+l}1^{n}0^{n^2}\notin L$, ya que, como $l\geq 1$:
        \begin{equation*}
            (n+l)\cdot n = n^2+nl \neq n^2
        \end{equation*}

        Por tanto, por el recíproco del Lema de Bombeo, no es regular.
        Además, este lenguaje no es libre de contexto (algo que aún no podemos demostrar).
    \end{enumerate}
\end{ejercicio}

\begin{ejercicio}\label{ej:1.3.4}
    Determina si los siguientes lenguajes son regulares. Encuentra una gramática que los genere o un reconocedor que los acepte.
    \begin{enumerate}
        \item $L_1 = \{0^i 1^j \mid j < i\}$.
        
        Usaremos el Lema de Bombeo para demostrar que no es regular. Para todo $n\in \mathbb{N}$, consideramos la palabra $z=0^{n+1}1^{n}\in L_1$ con $|z|=2n+1\geq n$. Toda descomposición $z=uvw$, con $u,v,w\in \{0,1\}^\ast$, $|uv|\leq n$ y $|v|\geq 1$ debe cumplir que:
        \begin{equation*}
            u=0^k \quad v=0^l \quad w=0^{n+1-k-l}1^{n} \qquad \text{con } l,k\in \bb{N}\cup \{0\},~l\geq 1,~k+l\leq n
        \end{equation*}

        Para $i=0$, tenemos que $uv^iw=0^{k+n+1-k-l}1^{n}=0^{n+1-l}1^{n}\notin L_1$, ya que:
        \begin{equation*}
            n < n+1-l \Longleftarrow l<1
        \end{equation*}
        Pero esto es una contradicción, ya que $l\geq 1$. Por tanto, por el recíproco del Lema de Bombeo, no es regular.
        Veamos ahora que es libre de contexto. Consideramos la gramática $G=(\{S\},\{0,1\},P,S)$, con $P$ definido por:
        \begin{align*}
            S &\rightarrow 0S \mid 0S'\\
            S' &\rightarrow 0S'1 \mid \veps
        \end{align*}

        Tenemos que $G$ es una gramática libre de contexto tal que $\cc{L}(G)=L_1$, por lo que $L_1$ es libre de contexto. Notemos que
        la producción $S\rightarrow 0S'$ fuerza a que haya al menos un $0$ más que $1$, y la producción $S\rightarrow 0S$ permite que la diferencia no sea de una sola unidad, sino que pueda ser mayor.

        \item $L_2 = \{001^i 0^j \mid i,j \geq 1\}$.
        
        Tenemos que un reconocedor de $L_2$ es:
        \begin{equation*}
            001^+0^+
        \end{equation*}

        Por tanto, tenemos que $L_2$ es regular.
        \item $L_3 = \{010u \mid u \in {\{0,1\}}^{\ast}, u \text{ no contiene la subcadena } 010\}$.
        
        Sea $L'=\{u\in \{0,1\}^\ast \mid u \text{ contiene la subcadena } 010\}$. Entonces sabemos que $L'$ es regular con reconocedor:
        \begin{equation*}
            (0+1)^*\red{010}(0+1)^*
        \end{equation*}

        Por tanto, $\ol{L'}$ es regular, por lo que está asociado a una expresión regular, sea esta $\wt{r}$. Entonces, $L_3$ es regular y está asociado a la expresión regular:
        \begin{equation*}
            010\wt{r}
        \end{equation*}
    \end{enumerate}
\end{ejercicio}

\begin{ejercicio}\label{ej:1.3.5}
    Sea el alfabeto $A=\{0,1,+,=\}$, demostrar que el lenguaje
    \begin{equation*}
        \text{ADD} = \{x=y+z \mid x,y,z \text{ son números en binario, y } x \text{ es la suma de  } y \text{ y } z\}
    \end{equation*}
    no es regular.\\

    Usaremos el Lema de Bombeo para demostrar que no es regular. Para todo $n\in \mathbb{N}$, consideramos la palabra $w=[1^n = 0+1^n]$, donde hemos empleado los corchetes para facilitar la notación (ya que el $=$ lo estamos usando para igualdades entre cadenas y entre números en binario). Tenemos que $|w|=n+3+n\geq n$. Toda descomposición $w=uvw$, con $u,v,w\in \{0,1,+,=\}^\ast$, $|uv|\leq n$ y $|v|\geq 1$ debe cumplir que:
    \begin{equation*}
        u=1^k \quad v=1^l \quad w=[1^{n-k-l} = 0+1^n] \qquad \text{con } l,k\in \bb{N}\cup \{0\},~l\geq 1,~k+l\leq n
    \end{equation*}

    Para $i=2$, tenemos que $uv^iw=[1^{k+2l+n-k-l} = 0+1^n]=[1^{n+l} = 0+1^n]\notin \text{ADD}$, ya que, como $l\geq 1$, en números binarios:
    \begin{equation*}
        1^{n+l} \neq 0+1^n=1^n
    \end{equation*}

    Por tanto, por el recíproco del Lema de Bombeo, no es regular.
\end{ejercicio}

\begin{ejercicio}\label{ej:1.3.6}
    Determinar si los siguientes lenguajes son regulares o no:
    \begin{enumerate}
        \item $L=\{uvu^{-1} \mid u,v \in {\{0,1\}}^{\ast}\}$.\\
        
        Veamos en primer lugar que $L=\{0,1\}^*$ por doble inclusión:
        \begin{description}
            \item[$\subset$)] Se tiene trivialmente que $L\subset \{0,1\}^*$.
            \item[$\supset$)] Sea $w\in \{0,1\}^*$. Entonces, podemos tomar $u=\varepsilon$ y $v=w$ para obtener $w\in L$.
        \end{description}

        Por tanto, tenemos que $L$ es regular, con reconocedor:
        \begin{equation*}
            (0+1)^*
        \end{equation*}

        \item $L$ es el lenguaje sobre el alfabeto $\{0,1\}$ formado de las palabras de la forma $u0v$ donde $u^{-1}$ es un prefijo de $v$.
        
        Usaremos el Lema de Bombeo para demostrar que no es regular. Para todo $n\in \mathbb{N}$, consideramos la palabra $z=1^n01^n\in L$ con $|z|=2n+1\geq n$. Toda descomposición $z=uvw$, con $u,v,w\in \{0,1\}^\ast$, $|uv|\leq n$ y $|v|\geq 1$ debe cumplir que:
        \begin{equation*}
            u=1^k \quad v=1^l \quad w=1^{n-k-l}01^n \qquad \text{con } l,k\in \bb{N}\cup \{0\},~l\geq 1,~k+l\leq n
        \end{equation*}

        Para $i=2$, tenemos que $uv^iw=1^{k+2l+n-k-l}01^n=1^{n+l}01^n\notin L$, ya que, como $l\geq 1$:
        \begin{equation*}
            n+l\neq n\Longrightarrow (1^{n+l})^{-1}=1^{n+l}\neq 1^n
        \end{equation*}

        Por tanto, por el recíproco del Lema de Bombeo, no es regular.
        \item $L$ es el lenguaje sobre el alfebeto $\{0,1\}$ formado por las palabres en las que el tercer símbolo empezando por el final es un 1.
        
        Este lenguaje es regular, con reconocedor:
        \begin{equation*}
            (0+1)^*\red{1}(0+1)(0+1)
        \end{equation*}
    \end{enumerate}
\end{ejercicio}

\begin{ejercicio}\label{ej:1.3.7}
    Obtener autómatas finitos determinísticos para los siguientes lenguajes sobre el alfabeto $\{0,1\}$.
    \begin{enumerate}
        \item \label{ej:1.3.7.1}
        Palabras en las que el número de 1 es múltiplo de 3 y el número de 0 es par.
        
        El estado $q_{ij}$ para $i=0,1,2$ y $j=0,1$ indica que:
        \begin{itemize}
            \item $n_1(u) \text{mod} 3=i$,
            \item $n_0(u) \text{mod} 2=j$.
        \end{itemize}

        Entonces, el autómata es el de la Figura~\ref{fig:1.3.7.1}.
        \begin{figure}
            \centering
            \begin{tikzpicture}
                \node[state, initial, accepting] (q00) {$q_{00}$};
                \node[state, right of=q00] (q01) {$q_{01}$};
                \node[state, right of=q01] (q02) {$q_{02}$};
                \node[state, below of=q00] (q10) {$q_{10}$};
                \node[state, right of=q10] (q11) {$q_{11}$};
                \node[state, right of=q11] (q12) {$q_{12}$};
    
                \draw   (q00) edge[above] node{1} (q01)
                        (q01) edge[above] node{1} (q02)
                        (q02) edge[above, bend right] node{1} (q00)
                        (q10) edge[above] node{1} (q11)
                        (q11) edge[above] node{1} (q12)
                        (q12) edge[below, bend left] node{1} (q10)
                        (q00) edge[left] node{0} (q10)
                        (q10) edge[right, bend right] node{0} (q00)
                        (q01) edge[left] node{0} (q11)
                        (q11) edge[right, bend right] node{0} (q01)
                        (q02) edge[left] node{0} (q12)
                        (q12) edge[right, bend right] node{0} (q02);
            \end{tikzpicture}
            \caption{Autómata que reconoce el lenguaje del ejercicio~\ref{ej:1.3.7}.\ref{ej:1.3.7.1}.}
            \label{fig:1.3.7.1}
        \end{figure}


        \item \label{ej:1.3.7.2}
        $\{{(01)}^{2i} \mid i \geq 0\}$
        
        El autómata es el de la Figura~\ref{fig:1.3.7.2}.
        \begin{figure}
            \centering
            \begin{tikzpicture}
                \node[state, initial, accepting] (q0) {$q_0$};
                \node[state, right of=q0] (q1) {$q_1$};
                \node[state, right of=q1] (q2) {$q_2$};
                \node[state, right of=q2] (q3) {$q_3$};
                \node[state, below of=q1, xshift=4em, error] (E) {$E$};

                \draw   (q0) edge[above] node{0} (q1)
                        (q1) edge[above] node{1} (q2)
                        (q2) edge[above] node{0} (q3)
                        (q3) edge[above, bend right] node{1} (q0);

                \draw[opacity=0.6]
                        (q0) edge[right] node{1} (E)
                        (q1) edge[right] node{0} (E)
                        (q2) edge[right] node{1} (E)
                        (q3) edge[right] node{0} (E)
                        (E) edge[loop right] node{0,1} (E);


            \end{tikzpicture}
            \caption{Autómata que reconoce el lenguaje del ejercicio~\ref{ej:1.3.7}.\ref{ej:1.3.7.2}.}
            \label{fig:1.3.7.2}
        \end{figure}
        \item $L_3=\{(0^{2i}1^{2i}) \mid i \geq 0\}$
        
        Veamos que este lenguaje no es regular con el Lema de Bombeo. Para todo $n\in \mathbb{N}$, consideramos la palabra $z=0^{2n}1^{2n}\in L_3$ con $|z|=4n\geq n$. Toda descomposición $z=uvw$, con $u,v,w\in \{0,1\}^\ast$, $|uv|\leq n$ y $|v|\geq 1$ debe cumplir que:
        \begin{equation*}
            u=0^k \quad v=0^l \quad w=0^{2n-k-l}1^{2n} \qquad \text{con } l,k\in \bb{N}\cup \{0\},~l\geq 1,~k+l\leq n
        \end{equation*}

        Para $i=2$, tenemos que $uv^iw=0^{k+2l+2n-k-l}1^{2n}=0^{2n+l}1^{2n}\notin L_3$, ya que, como $l\geq 1$:
        \begin{equation*}
            2n+l\neq 2n
        \end{equation*}

        Por tanto, por el recíproco del Lema de Bombeo, no es regular. Por tanto, no es posible construir un autómata finito determinístico que reconozca $L_3$.
    \end{enumerate}
\end{ejercicio}

\begin{ejercicio}\label{ej:1.3.8}
    Dar una expresión regular para la intersección de los lenguajes asociados a las expresiones regulares ${(01+1)}^{\ast}0$ y ${(10+0)}^{\ast}$. Se valorará que se construya el autómata que acepta la intersección de estos lenguajes, se minimice y, a partir del resultado, se construya la expresión regular.\\

    Sea $r_1=(01+1)^{\ast}0$ y $r_2=(10+0)^{\ast}$. Construimos los AFND asociados a $r_1$ y $r_2$, mostrados en las figuras \ref{fig:1.3.8-AFND1} y \ref{fig:1.3.8-AFND2}, respectivamente.
    \begin{figure}
        \centering
        \begin{subfigure}[c]{0.45\textwidth}
            \centering
            \begin{tikzpicture}
                \node[state, initial] (q0) {$q_0$};
                \node[state, right of=q0] (q1) {$q_1$};
                \node[state, accepting, below of=q0] (q2) {$q_2$};

                \draw   (q0) edge[above] node{0} (q1)
                        (q0) edge[loop above] node{1} (q0)
                        (q1) edge[above, bend right] node{1} (q0)
                        (q0) edge[right] node{0} (q2);
            \end{tikzpicture}
            \caption{AFND asociado a $r_1$.}
            \label{fig:1.3.8-AFND1}
        \end{subfigure}
        \begin{subfigure}[c]{0.45\textwidth}
            \centering
            \begin{tikzpicture}
                \node[state, initial, accepting] (q0) {$q_0$};
                \node[state, right of=q0] (q1) {$q_1$};

                \draw   (q0) edge[above] node{1} (q1)
                        (q0) edge[loop above] node{0} (q0)
                        (q1) edge[above, bend right] node{0} (q0);
            \end{tikzpicture}
            \caption{AFND asociado a $r_2$.}
            \label{fig:1.3.8-AFND2}
        \end{subfigure}
        \caption{AFND asociados a las expresiones regulares $r_1$ y $r_2$.}
        \label{fig:1.3.8-AFND}
    \end{figure}

    Para poder aplicar el algoritmo de intersección de autómatas, antes hemos de convertir los autómatas en AFD. Los AFD asociados a $r_1$ y $r_2$ son los de las figuras \ref{fig:1.3.8-AFD1} y \ref{fig:1.3.8-AFD2}, respectivamente.
    \begin{figure}
        \centering
        \begin{subfigure}[c]{0.45\textwidth}
            \centering
            \begin{tikzpicture}
                \node[state, initial] (q10) {$q_{10}$};
                \node[state, right of=q0, accepting] (q11) {$q_{11}$};
                \node[state, error, below of=q11] (E1) {$E_1$};

                \draw   (q10) edge[above] node{0} (q11)
                        (q10) edge[loop above] node{1} (q10)
                        (q11) edge[above, bend right] node{1} (q10)
                        (q11) edge[right] node{0} (E1)
                        (E1) edge[loop left] node{0,1} (E1);
            \end{tikzpicture}
            \caption{AFD asociado a $r_1$.}
            \label{fig:1.3.8-AFD1}
        \end{subfigure}
        \begin{subfigure}[c]{0.45\textwidth}
            \centering
            \begin{tikzpicture}
                \node[state, initial, accepting] (q20) {$q_{20}$};
                \node[state, right of=q0] (q21) {$q_{21}$};
                \node[state, error, below of=q21] (E2) {$E_2$};

                \draw   (q0) edge[above] node{1} (q1)
                        (q0) edge[loop above] node{0} (q0)
                        (q1) edge[above, bend right] node{0} (q0)
                        (q1) edge[right] node{1} (E2)
                        (E2) edge[loop left] node{0,1} (E2);
            \end{tikzpicture}
            \caption{AFD asociado a $r_2$.}
            \label{fig:1.3.8-AFD2}
        \end{subfigure}
        \caption{AFD asociados a las expresiones regulares $r_1$ y $r_2$.}
        \label{fig:1.3.8-AFD}
    \end{figure}

    Por tanto, el AFD que acepta la intersección de los lenguajes asociados a $r_1$ y $r_2$ es el de la Figura~\ref{fig:1.3.8-AFD-Interseccion}.
    \begin{figure}
        \centering
        \begin{tikzpicture}
            \node[state, initial] (q10_20) {$q_{10}q_{20}$};
            \node[state, above right of=q10_20, accepting] (q11_20) {$q_{11}q_{20}$};
            \node[state, below right of=q10_20] (q10_21) {$q_{10}q_{21}$};
            \node[state, right of=q11_20] (qE1_20) {$E_1q_{20}$};
            \node[state, right of=qE1_20] (qE1_21) {$E_1q_{21}$};
            \node[state, below right of=qE1_21, error] (E1_E2) {$E_1E_2$};
            \node[state, right of=q10_21] (q10_E2) {$q_{10}E_2$};
            \node[state, right of=q10_E2] (q11_E2) {$q_{11}E_2$};

            \draw   (q10_20) edge[above] node{0} (q11_20)
                    (q10_20) edge[below] node{1} (q10_21)
                    (q11_20) edge[above] node{0} (qE1_20)
                    (q11_20) edge[right, bend right] node{1} (q10_21)
                    (q10_21) edge[above] node{1} (q10_E2)
                    (q10_21) edge[right, bend right] node{0} (q11_20)
                    (qE1_20) edge[above] node{1} (qE1_21)
                    (qE1_20) edge[loop below] node{0} (qE1_20)
                    (qE1_21) edge[above] node{1} (E1_E2)
                    (qE1_21) edge[bend left, below] node{0} (qE1_20)
                    (E1_E2) edge[loop right] node{0,1} (E1_E2)
                    (q10_E2) edge[above] node{0} (q11_E2)
                    (q10_E2) edge[loop above] node{1} (q10_E2)
                    (q11_E2) edge[above] node{0} (E1_E2)
                    (q11_E2) edge[bend right, above] node{1} (q10_E2);
        \end{tikzpicture}
        \caption{AFD que acepta la intersección de los lenguajes asociados a $r_1$ y $r_2$.}
        \label{fig:1.3.8-AFD-Interseccion}
    \end{figure}

    Para minimizarlo, consideramos en primer lugar que los siguientes estados son indistinguibles:
    \begin{equation*}
        q_E := \{E_{1}q_{20},E_{1}q_{21}, q_{10}E_{2},q_{11}E_{2}, E_{1}E_{2}\}
    \end{equation*}

    Estos son indistinguibles puesto que desde ninguno de ellos se puede llegar a un estado final. Por tanto, el AFD minimal es el de la Figura~\ref{fig:1.3.8-AFD-Interseccion-Minimal}.
    \begin{figure}
        \centering
        \begin{tikzpicture}
            \node[state, initial] (q10_20) {$q_{10}q_{20}$};
            \node[state, above right of=q10_20, accepting] (q11_20) {$q_{11}q_{20}$};
            \node[state, below right of=q10_20] (q10_21) {$q_{10}q_{21}$};
            \node[state, below right of=q11_20, error] (qE) {$q_{E}$};

            \draw   (q10_20) edge[above] node{0} (q11_20)
                    (q10_20) edge[below] node{1} (q10_21)
                    (q11_20) edge[bend right, right] node{1} (q10_21)
                    (q11_20) edge[above] node{0} (qE)
                    (q10_21) edge[below] node{1} (qE)
                    (q10_21) edge[bend right, left] node{0} (q11_20)
                    (qE) edge[loop right] node{0,1} (qE);
        \end{tikzpicture}
        \caption{AFD minimal que acepta la intersección de los lenguajes asociados a $r_1$ y $r_2$.}
        \label{fig:1.3.8-AFD-Interseccion-Minimal}
    \end{figure}

    Tenemos que es minimal, puesto que todos los estados son alcanzables y no hay estados distinguibles:
    \begin{itemize}
        \item $q_{11}q_{20}$ es distinguible del resto de estados por ser el único estado final.
        \item $q_{E}$ es distinguible de $q_{10}q_{20}$ y $q_{10}q_{21}$, ya que leyendo un $0$:
        \begin{equation*}
            \delta(q_{E},0)=q_{E}\notin F\qquad \delta(q_{10}q_{20},0)=\delta(q_{10}q_{21},0)=q_{11}q_{20}\in F
        \end{equation*}

        \item $q_{10}q_{20}$ y $q_{10}q_{21}$ son distinguibles, ya que leyendo $10$:
        \begin{equation*}
            \delta(q_{10}q_{20},1)=q_{11}q_{20}\in F\qquad \delta(q_{10}q_{21},1)=q_{E}\notin F
        \end{equation*}
    \end{itemize}

    Por tanto, el AFD minimal que acepta la intersección de los lenguajes asociados a $r_1$ y $r_2$ es el de la Figura~\ref{fig:1.3.8-AFD-Interseccion-Minimal}. Para obtener la expresión regular asociada, resolvemos el sistema de ecuaciones:
    \begin{align*}
        q_{10}q_{20} &= 0q_{11}q_{20}+1q_{10}q_{21}\\
        q_{11}q_{20} &= 0q_{E}+1q_{10}q_{21} + \veps\\
        q_{10}q_{21} &= 0q_{11}q_{20}+1q_{E}\\
        q_{E} &= 0q_{E}+1q_{E} = (0+1)q_{E}
    \end{align*}

    Por el Lema de Arden, como $q_E = (0+1)q_E + \emptyset$, tenemos que $q_E=(0+1)^*\emptyset = \emptyset$. Sustituyendo en las ecuaciones anteriores, obtenemos:
    \begin{align*}
        q_{10}q_{20} &= 0q_{11}q_{20}+1q_{10}q_{21}\\
        q_{11}q_{20} &= 1q_{10}q_{21} + \veps\\
        q_{10}q_{21} &= 0q_{11}q_{20}
    \end{align*}

    Sustituyendo $q_{10}q_{21}$ en la segunda ecuación, obtenemos:
    \begin{equation*}
        q_{11}q_{20} = 1q_{11}q_{20} + \veps \Longrightarrow q_{11}q_{20} = 10q_{11}q_{20}+ \veps = (10)^*\veps = (10)^*
    \end{equation*}

    Sustituyendo ambos en la primera ecuación, obtenemos:
    \begin{equation*}
        q_{10}q_{20} = 0(10)^* + 10q_{11}q_{20} = 0(10)^* + 10(10)^* = (0+10)(10)^*
    \end{equation*}

    Por tanto, la expresión regular asociada al AFD minimal de la Figura~\ref{fig:1.3.8-AFD-Interseccion-Minimal} es $$(0+10)(10)^*.$$    
\end{ejercicio}

\begin{ejercicio}\label{ej:1.3.9}
    Construir un Autómata Finito Determinista Minimal que acepte el lenguaje sobre el alfabeto $\{a,b,c\}$ de todas aquellas palabras que verifiquen simultáneamente las siguientes condiciones.
    \begin{enumerate}
        \item \label{ej:1.3.9.1}
        La palabra contiene un número par de a's.
        \item \label{ej:1.3.9.2}
        La longitud de la palabra es un múltiplo de 3.
        \item \label{ej:1.3.9.3}
        La palabra no contiene la subcadena $abc$.
    \end{enumerate}

    La condición \ref{ej:1.3.9.1} se puede cumplir con el autómata de la Figura~\ref{fig:1.3.9-AFD1}.
    \begin{figure}
        \centering
        \begin{tikzpicture}
            \node[state, accepting, initial] (a0) {$a_0$};
            \node[state, right of=a0] (a1) {$a_1$};

            \draw   (a0) edge[loop above] node{$b,c$} (a0)
                    (a0) edge[above, bend left] node{$a$} (a1)
                    (a1) edge[loop above] node{$b,c$} (a1)
                    (a1) edge[below, bend left] node{$a$} (a0);
        \end{tikzpicture}
        \caption{AFD que acepta palabras con número par de $a$'s.}
        \label{fig:1.3.9-AFD1}
    \end{figure}

    La condición \ref{ej:1.3.9.2} se puede cumplir con el autómata de la Figura~\ref{fig:1.3.9-AFD2}.
    \begin{figure}
        \centering
        \begin{tikzpicture}
            \node[state, initial, accepting] (b0) {$b_0$};
            \node[state, right of=b0] (b1) {$b_1$};
            \node[state, right of=b1] (b2) {$b_2$};

            \draw   (b0) edge[above] node{$a,b,c$} (b1)
                    (b1) edge[above] node{$a,b,c$} (b2)
                    (b2) edge[above, bend right] node{$a,b,c$} (b0);
        \end{tikzpicture}
        \caption{AFD que acepta palabras de longitud múltiplo de 3.}
        \label{fig:1.3.9-AFD2}
    \end{figure}

    La condición \ref{ej:1.3.9.3} se puede cumplir con el autómata de la Figura~\ref{fig:1.3.9-AFD3}.
    \begin{figure}
        \centering
        \begin{tikzpicture}
            \node[state, initial, accepting] (c0) {$c_0$};
            \node[state, right of=c0, accepting] (c1) {$c_1$};
            \node[state, right of=c1, accepting] (c2) {$c_2$};
            \node[state, right of=c2] (c3) {$c_3$};

            \draw   (c0) edge[above] node{$a$} (c1)
                    (c0) edge[loop above] node{$b,c$} (c0)
                    (c1) edge[above] node{$b$} (c2)
                    (c1) edge[loop below] node{$a$} (c1)
                    (c1) edge[below, bend left] node{$c$} (c0)
                    (c2) edge[above] node{$c$} (c3)
                    (c2) edge[below, bend left] node{$a$} (c1)
                    (c2) edge[above, bend right] node{$b$} (c0)
                    (c3) edge[loop above] node{$a,b,c$} (c3);
        \end{tikzpicture}
        \caption{AFD que acepta palabras que no contienen la subcadena $abc$.}
        \label{fig:1.3.9-AFD3}
    \end{figure}

    El autómata producto tiene los siguientes estados:
    \begin{align*}
        Q&=\{a_ib_jc_k\mid i\in\{0,1\},j\in\{0,1,2\},k\in\{0,1,2,3\}\}\\
        F&=\{a_0b_0c_0, a_0b_0c_1, a_0b_0c_2\}
    \end{align*}

    El autómata producto es $M=(Q,\{a,b,c\},\delta,a_0b_0c_0,F)$, donde $\delta$ viene dado por la Tabla~\ref{tab:1.3.9-AFD-Producto}.
    Además, todos sus estados son accesibles.
    \begin{table}
        \centering
        \resizebox{1.1\textwidth}{!}{
        \begin{tabular}{r|cccccccccccc}
            & $1$ & $2$ & $3$ & $4$ & $5$ & $6$ & $7$ & $8$ & $9$ & $10$ & $11$ & $12$ \\
            $\delta$ & $a_0b_0c_0$ & $a_0b_0c_1$ & $a_0b_0c_2$ & $a_0b_0c_3$ & $a_0b_1c_0$ & $a_0b_1c_1$ & $a_0b_1c_2$ & $a_0b_1c_3$ & $a_0b_2c_0$ & $a_0b_2c_1$ & $a_0b_2c_2$ & $a_0b_2c_3$ \\\hline
            $a$ & $a_1b_1c_1$ & $a_1b_1c_1$ & $a_1b_1c_1$ & $a_1b_1c_3$ & $a_1b_2c_1$ & $a_1b_2c_1$ & $a_1b_2c_1$ & $a_1b_2c_3$ & $a_1b_0c_1$ & $a_1b_0c_1$ & $a_1b_0c_1$ & $a_1b_0c_3$ \\
            $b$ & $a_0b_1c_0$ & $a_0b_1c_2$ & $a_0b_1c_0$ & $a_0b_1c_3$ & $a_0b_2c_0$ & $a_0b_2c_2$ & $a_0b_2c_0$ & $a_0b_2c_3$ & $a_0b_0c_0$ & $a_0b_0c_2$ & $a_0b_0c_0$ & $a_0b_0c_3$ \\
            $c$ & $a_0b_1c_0$ & $a_0b_1c_0$ & $a_0b_1c_3$ & $a_0b_1c_3$ & $a_0b_2c_0$ & $a_0b_2c_0$ & $a_0b_2c_3$ & $a_0b_2c_3$ & $a_0b_0c_0$ & $a_0b_0c_0$ & $a_0b_0c_3$ & $a_0b_0c_3$ \\
            \hline \hline \\ \hline \hline 
            & $13$ & $14$ & $15$ & $16$ & $17$ & $18$ & $19$ & $20$ & $21$ & $22$ & $23$ & $24$ \\
            $\delta$ & $a_1b_0c_0$ & $a_1b_0c_1$ & $a_1b_0c_2$ & $a_1b_0c_3$ & $a_1b_1c_0$ & $a_1b_1c_1$ & $a_1b_1c_2$ & $a_1b_1c_3$ & $a_1b_2c_0$ & $a_1b_2c_1$ & $a_1b_2c_2$ & $a_1b_2c_3$ \\\hline
            $a$ & $a_0b_1c_1$ & $a_0b_1c_1$ & $a_0b_1c_1$ & $a_0b_1c_3$ & $a_0b_2c_1$ & $a_0b_2c_1$ & $a_0b_2c_1$ & $a_0b_2c_3$ & $a_0b_0c_1$ & $a_0b_0c_1$ & $a_0b_0c_1$ & $a_0b_0c_3$ \\
            $b$ & $a_1b_1c_0$ & $a_1b_1c_2$ & $a_1b_1c_0$ & $a_1b_1c_3$ & $a_1b_2c_0$ & $a_1b_2c_2$ & $a_1b_2c_0$ & $a_1b_2c_3$ & $a_1b_0c_0$ & $a_1b_0c_2$ & $a_1b_0c_0$ & $a_1b_0c_3$ \\
            $c$ & $a_1b_1c_0$ & $a_1b_1c_0$ & $a_1b_1c_3$ & $a_1b_1c_3$ & $a_1b_2c_0$ & $a_1b_2c_0$ & $a_1b_2c_3$ & $a_1b_2c_3$ & $a_1b_0c_0$ & $a_1b_0c_0$ & $a_1b_0c_3$ & $a_1b_0c_3$
        \end{tabular}}
        \caption{Transiciones del autómata producto para el ejercicio~\ref{ej:1.3.9}.}
        \label{tab:1.3.9-AFD-Producto}
    \end{table}

    En primer lugar, vemos que los estados de la forma $a_ib_jc_3$ son indistinguibles, puesto que desde ellos no se puede llegar a un estado final. Por tanto, los agrupamos en un único estado $c_3$.

    La minimización del autómata producto se muestra en la Tabla~\ref{tab:1.3.9-AFD-Minimal}.
    \begin{table}
        \centering
        \resizebox{1.1\textwidth}{!}{
        \begin{tabular}{r c c c c c c c c c c c c c c c c c c c c c c c c c c c cc c c c c c c c c c c c c c c c c c c c c c c c c c c c}
            \hhline{~*{1}{-}}
            $2$ & \cell{\times} \\ \hhline{~*{2}{-}}
            $3$ & \cell{\xcancel{(9,10)}} & \cell{\times} \\ \hhline{~*{3}{-}}
            $5$ & \cell{\times} & \cell{\times} & \cell{\times}  \\ \hhline{~*{4}{-}}
            $6$ & \cell{\times} & \cell{\times} & \cell{\times} & \cell{\times} \\ \hhline{~*{5}{-}}
            $7$ & \cell{\times} & \cell{\times} & \cell{\times} & \cell{\times} & \cell{\times} \\ \hhline{~*{6}{-}}
            $9$ & \cell{\times} & \cell{\times} & \cell{\times} & \cell{\times} & \cell{\times} & \cell{\times} \\ \hhline{~*{7}{-}}
            $10$ & \cell{\times} & \cell{\times} & \cell{\times} & \cell{\times} & \cell{\times} & \cell{\times} & \cell{\times} & \\ \hhline{~*{8}{-}}
            $11$ & \cell{\times} & \cell{\times} & \cell{\times} & \cell{\times} & \cell{\times} & \cell{\times} & \cell{\times} & \cell{\times} \\ \hhline{~*{9}{-}}
            $13$ & \cell{\times} & \cell{\times} & \cell{\times} & \cell{\times} & \cell{\times} & \cell{\times} & \cell{\times} & \cell{\times} & \cell{\times}  \\ \hhline{~*{10}{-}}
            $14$ & \cell{\times} & \cell{\times} & \cell{\times} & \cell{\times} & \cell{\times} & \cell{\times} & \cell{\times} & \cell{\times} & \cell{\times} & \cell{\times} \\ \hhline{~*{11}{-}}
            $15$ & \cell{\times} & \cell{\times} & \cell{\times} & \cell{\times} & \cell{\times} & \cell{\times} & \cell{\times} & \cell{\times} & \cell{\times} & \cell{\times} & \cell{\times}  \\ \hhline{~*{12}{-}}
            $17$ & \cell{\times} & \cell{\times} & \cell{\times} & \cell{\times} & \cell{\times} & \cell{\times} & \cell{\times} & \cell{\times} & \cell{\times} & \cell{\times} & \cell{\times} & \cell{\times} \\ \hhline{~*{13}{-}}
            $18$ & \cell{\times} & \cell{\times} & \cell{\times} & \cell{\times} & \cell{\times} & \cell{\times} & \cell{\times} & \cell{\times} & \cell{\times} & \cell{\times} & \cell{\times} & \cell{\times} & \cell{\times} \\ \hhline{~*{14}{-}}
            $19$ & \cell{\times} & \cell{\times} & \cell{\times} & \cell{\times} & \cell{\times} & \cell{\times} & \cell{\times} & \cell{\times} & \cell{\xcancel{(21,22)}} & \cell{\times} & \cell{\times} & \cell{\times} & \cell{\times} & \cell{\times} \\ \hhline{~*{15}{-}}
            $21$ & \cell{\times} & \cell{\times} & \cell{\times} & \cell{\times} & \cell{\times} & \cell{\times} & \cell{\times} & \cell{\times} & \cell{\times} & \cell{\times} & \cell{\times} & \cell{\times} & \cell{\times} & \cell{\times} & \cell{\times} \\ \hhline{~*{16}{-}}
            $22$ & \cell{\times} & \cell{\times} & \cell{\times} & \cell{\times} & \cell{\times} & \cell{\times} & \cell{\times} & \cell{\times} & \cell{\times} & \cell{\times} & \cell{\times} & \cell{\times} & \cell{\times} & \cell{\times} & \cell{\times} & \cell{\times} \\ \hhline{~*{17}{-}}
            $23$ & \cell{\times} & \cell{\times} & \cell{\times} & \cell{\times} & \cell{\times} & \cell{\times} & \cell{\times} & \cell{\times} & \cell{\times} & \cell{\times} & \cell{\times} & \cell{\times} & \cell{\times} & \cell{\times} & \cell{\times} & \cell{\times} & \cell{\times} \\ \hhline{~*{18}{-}}
            $c_3$ & \cell{\times} & \cell{\times} & \cell{\times} & \cell{\times} & \cell{\times} & \cell{\times} & \cell{\times} & \cell{\times} & \cell{\times} & \cell{\times} & \cell{\times} & \cell{\times} & \cell{\times} & \cell{\times} & \cell{\times} & \cell{\times}& \cell{\times} & \cell{\times} \\ \hhline{~*{18}{-}}
            & $1$ & $2$ & $3$ & $5$ & $6$ & $7$ & $9$ & $10$ & $11$ &  $13$ & $14$ & $15$ & $17$ & $18$ & $19$& $21$ & $22$ & $23$
        \end{tabular}
        }
        \caption{Tabla de minimización del autómata producto para el ejercicio~\ref{ej:1.3.9}.}
        \label{tab:1.3.9-AFD-Minimal}
    \end{table}
    
    Por tanto, el autómata minimal que acepta el lenguaje del ejercicio~\ref{ej:1.3.9} es $M$, pero unificando los estados previamente descritos en $c_3$.
    Debido al gran número de estados (19), el grafo de dicho autómata no se muestra.
\end{ejercicio}

\begin{ejercicio}\label{ej:1.3.10}
    Encontrar un AFD minimal para el lenguaje
    \begin{equation*}
        {(a+b)}^{\ast}(aa+bb){(a+b)}^{\ast}
    \end{equation*}

    Para ello, primero construimos el AFND asociado a la expresión regular, mostrado en la Figura~\ref{fig:1.3.10-AFND}.
    \begin{figure}
        \centering
        \begin{tikzpicture}
            \node[state, initial] (q0) {$q_0$};
            \node[state, above right of=q0] (q1) {$q_1$};
            \node[state, below right of=q0] (q2) {$q_2$};
            \node[state, above right of=q2, accepting] (q3) {$q_3$};


            \draw   (q0) edge[loop above] node{$a,b$} (q0)
                    (q0) edge[above] node{$a$} (q1)
                    (q0) edge[below] node{$b$} (q2)
                    (q1) edge[above] node{$a$} (q3)
                    (q2) edge[below] node{$b$} (q3)
                    (q3) edge[loop above] node{$a,b$} (q3);
        \end{tikzpicture}
        \caption{AFND asociado a la expresión regular ${(a+b)}^{\ast}(aa+bb){(a+b)}^{\ast}$.}
        \label{fig:1.3.10-AFND}
    \end{figure}

    Convertimos el AFND en un AFD, mostrado en la Figura~\ref{fig:1.3.10-AFD}.
    \begin{figure}
        \centering
        \begin{tikzpicture}
            \node[state, initial] (q0) {$q_0$};
            \node[state, above right of=q0] (q0q1) {$q_0q_1$};
            \node[state, below right of=q0] (q0q2) {$q_0q_2$};
            \node[state, right of=q0q1, accepting] (q0q1q3) {$q_0q_1q_3$};
            \node[state, right of=q0q2, accepting] (q0q2q3) {$q_0q_2q_3$};

            \draw   (q0) edge[above] node{$a$} (q0q1)
                    (q0) edge[below] node{$b$} (q0q2)
                    (q0q1) edge[above] node{$a$} (q0q1q3)
                    (q0q1) edge[bend right, right] node{$b$} (q0q2)
                    (q0q2) edge[below] node{$b$} (q0q2q3)
                    (q0q2) edge[bend right, left] node{$a$} (q0q1)
                    (q0q1q3) edge[loop above] node{$a$} (q0q1q3)
                    (q0q1q3) edge[bend right, right] node{$b$} (q0q2q3)
                    (q0q2q3) edge[loop below] node{$b$} (q0q2q3)
                    (q0q2q3) edge[bend right, left] node{$a$} (q0q1q3);

        \end{tikzpicture}
        \caption{AFD asociado a la expresión regular ${(a+b)}^{\ast}(aa+bb){(a+b)}^{\ast}$.}
        \label{fig:1.3.10-AFD}
    \end{figure}

    No obstante, este no es minimal. En primer lugar, vemos que los estados $q_0q_1q_3$ y $q_0q_2q_3$ son indistinguibles, ya que para cualquier palabra $w\in \{a,b\}^*$:
    \begin{equation*}
        \delta(q_0q_1q_3,w)\in F\qquad \delta(q_0q_2q_3,w)\in F
    \end{equation*}

    Por tanto, notemos $q_F=\{q_0q_1q_3,q_0q_2q_3\}$. El AFD minimal es el de la Figura~\ref{fig:1.3.10-AFD-Minimal}.
    \begin{figure}
        \centering
        \begin{tikzpicture}
            \node[state, initial] (q0) {$q_0$};
            \node[state, above right of=q0] (q0q1) {$q_0q_1$};
            \node[state, below right of=q0] (q0q2) {$q_0q_2$};
            \node[state, above right of=q0q2, accepting] (qF) {$q_F$};

            \draw   (q0) edge[above] node{$a$} (q0q1)
                    (q0) edge[below] node{$b$} (q0q2)
                    (q0q1) edge[bend right, right] node{$b$} (q0q2)
                    (q0q2) edge[bend right, left] node{$a$} (q0q1)
                    (q0q1) edge[above] node{$a$} (qF)
                    (q0q2) edge[below] node{$b$} (qF)
                    (qF) edge[loop right] node{$a,b$} (qF);
        \end{tikzpicture}
        \caption{AFD minimal asociado a la expresión regular ${(a+b)}^{\ast}(aa+bb){(a+b)}^{\ast}$.}
        \label{fig:1.3.10-AFD-Minimal}
    \end{figure}
    \begin{itemize}
        \item $q_F$ es distinguible del resto de estados por ser el único estado final.
        \item $q_0q_1$ y $q_0q_2$ son distinguibles, ya que leyendo un $a$:
        \begin{equation*}
            \delta(q_0q_1,a)=q_F\in F\qquad \delta(q_0q_2,a)=q_0q_1\notin F
        \end{equation*}

        \item $q_0$ y $q_0q_1$ son distinguibles, ya que leyendo un $a$:
        \begin{equation*}
            \delta(q_0,a)=q_0q_1\notin F\qquad \delta(q_0q_1,a)=q_F\in F
        \end{equation*}

        \item $q_0$ y $q_0q_2$ son distinguibles, ya que leyendo un $b$:
        \begin{equation*}
            \delta(q_0,b)=q_0q_2\notin F\qquad \delta(q_0q_2,b)=q_F\in F
        \end{equation*}
    \end{itemize}

    Por tanto, el AFD minimal es el de la Figura~\ref{fig:1.3.10-AFD-Minimal}.
\end{ejercicio}

\begin{ejercicio}\label{ej:1.3.11}
    Para cada uno de los siguientes lenguajes regulares, encontrar el autómata minimal asociado, y a partir de dicho autómata minimal, determinar la gramática regular que genera el lenguaje:
    \begin{enumerate}
        \item \label{ej:1.3.11-1}
        $a^+ b^+$
        
        En primer lugar, construimos el AFD asociado al lenguaje, mostrado en la Figura~\ref{fig:1.3.11-1-AFD}.
        \begin{figure}
            \centering
            \begin{tikzpicture}
                \node[state, initial] (q0) {$q_0$};
                \node[state, right of=q0] (q1) {$q_1$};
                \node[state, accepting, right of=q1] (q2) {$q_2$};
                \node[state, error, below of=q1] (E) {$E$};

                \draw   (q0) edge[above] node{$a$} (q1)
                        (q0) edge[above] node{$b$} (E)
                        (q1) edge[above] node{$b$} (q2)
                        (q2) edge[loop above] node{$b$} (q2)
                        (q1) edge[loop above] node{$a$} (q1)
                        (q2) edge[above] node{$a$} (E)
                        (E) edge[loop right] node{$a,b$} (E);
            \end{tikzpicture}
            \caption{AFD asociado al lenguaje $a^+b^+$ del ejercicio \ref{ej:1.3.11}.\ref{ej:1.3.11-1}.}
            \label{fig:1.3.11-1-AFD}
        \end{figure}

        Veamos que este es minimal:
        \begin{itemize}
            \item $q_2$ es distinguible del resto de estados por ser el único estado final.
            \item $q_0$ y $q_1$ son distinguibles, ya que leyendo un $b$:
            \begin{equation*}
                \delta(q_0,b)=E\notin F\qquad \delta(q_1,b)=q_2\in F
            \end{equation*}

            \item $q_0$ y $E$ son distinguibles, ya que leyendo un $ab$:
            \begin{equation*}
                \delta^*(q_0,ab)=q_2\in F\qquad \delta^*(E,ab)=E\notin F
            \end{equation*}

            \item $q_1$ y $E$ son distinguibles, ya que leyendo un $b$:
            \begin{equation*}
                \delta(q_1,b)=q_2\in F\qquad \delta(E,b)=E\notin F
            \end{equation*}
        \end{itemize}

        Por tanto, el AFD minimal es el de la Figura~\ref{fig:1.3.11-1-AFD}. La gramática regular que genera el lenguaje es $G=(\{q_0,q_1,q_2\},\{a,b\},P,\{q_0\})$ con $P$:
        \begin{align*}
            q_0 &\longrightarrow aq_1\\
            q_1 &\longrightarrow aq_1\mid bq_2\\
            q_2 &\longrightarrow bq_2\mid \veps
        \end{align*}
        \item \label{ej:1.3.11-2}
        $a{(a+b)}^{\ast}b$
        
        En primer lugar, construimos el AFND asociado al lenguaje, mostrado en la Figura~\ref{fig:1.3.11-2-AFND}.
        \begin{figure}
            \centering
            \begin{tikzpicture}
                \node[state, initial] (q0) {$q_0$};
                \node[state, right of=q0] (q1) {$q_1$};
                \node[state, accepting, right of=q1] (q2) {$q_2$};

                \draw   (q0) edge[above] node{$a$} (q1)
                        (q1) edge[loop above] node{$a,b$} (q1)
                        (q1) edge[above] node{$b$} (q2);
            \end{tikzpicture}
            \caption{AFND asociado al lenguaje $a{(a+b)}^{\ast}b$ del ejercicio \ref{ej:1.3.11}.\ref{ej:1.3.11-2}.}
            \label{fig:1.3.11-2-AFND}
        \end{figure}

        Convertimos el AFND en un AFD, mostrado en la Figura~\ref{fig:1.3.11-2-AFD}.
        \begin{figure}
            \centering
            \begin{tikzpicture}
                \node[state, initial] (q0) {$q_0$};
                \node[state, right of=q0] (q1) {$q_1$};
                \node[state, accepting, right of=q1] (q2) {$q_1q_2$};
                \node[state, error, below of=q1] (E) {$E$};

                \draw   (q0) edge[above] node{$a$} (q1)
                        (q0) edge[above] node{$b$} (E)
                        (q1) edge[loop above] node{$a$} (q1)
                        (q1) edge[above] node{$b$} (q2)
                        (q2) edge[loop above] node{$b$} (q2)
                        (q2) edge[above, bend right] node{$a$} (q1)
                        (E) edge[loop right] node{$a,b$} (E);
            \end{tikzpicture}
            \caption{AFD asociado al lenguaje $a{(a+b)}^{\ast}b$ del ejercicio \ref{ej:1.3.11}.\ref{ej:1.3.11-2}.}
            \label{fig:1.3.11-2-AFD}
        \end{figure}

        Veamos que este es minimal:
        \begin{itemize}
            \item $q_1q_2$ es distinguible del resto de estados por ser el único estado final.
            \item $q_0$ y $q_1$ son distinguibles, ya que leyendo un $b$:
            \begin{equation*}
                \delta(q_0,b)=E\notin F\qquad \delta(q_1,b)=q_1q_2\in F
            \end{equation*}

            \item $q_0$ y $E$ son distinguibles, ya que leyendo un $ab$:
            \begin{equation*}
                \delta^*(q_0,ab)=q_1q_2\in F\qquad \delta^*(E,ab)=E\notin F
            \end{equation*}

            \item $q_1$ y $E$ son distinguibles, ya que leyendo un $b$:
            \begin{equation*}
                \delta(q_1,b)=q_1q_2\in F\qquad \delta(E,b)=E\notin F
            \end{equation*}
        \end{itemize}

        Por tanto, el AFD minimal es el de la Figura~\ref{fig:1.3.11-2-AFD}. La gramática regular que genera el lenguaje es $G=(\{q_0,q_1,q_2\},\{a,b\},P,\{q_0\})$ con $P$:
        \begin{align*}
            q_0 &\longrightarrow aq_1\\
            q_1 &\longrightarrow aq_1\mid bq_2\\
            q_2 &\longrightarrow bq_2\mid aq_1\mid \veps
        \end{align*}
    \end{enumerate}
\end{ejercicio}

\begin{ejercicio}\label{ej:1.3.12}
    Considera la gramática cuyas producciones se presentan a continuación y donde el símbolo inicial es $S$:
    \begin{align*}
        S &\rightarrow xN\mid  x \\
        N &\rightarrow yM\mid  y \\
        M &\rightarrow zN\mid  z
    \end{align*}
    \begin{enumerate}
        \item Escribe el diagrama de transiciones para el AFD que acepte el lenguaje $\cc{L}(G)$ generado por $G$.
        
        Las siguientes producciones generan el mismo lenguaje:
        \begin{align*}
            S &\rightarrow xN\\
            N &\rightarrow yM\mid \veps \\
            M &\rightarrow zN\mid \veps
        \end{align*}

        Por tanto, el AFD asociado al lenguaje $\cc{L}(G)$ es el de la Figura~\ref{fig:1.3.12-AFD}.
        \begin{figure}
            \centering
            \begin{tikzpicture}
                \node[state, initial] (S) {$S$};
                \node[state, right of=S, accepting] (N) {$N$};
                \node[state, right of=N, accepting] (M) {$M$};
                \node[state, error, right of=M] (E) {$E$};

                \draw   (S) edge[above] node{$x$} (N)
                        (N) edge[above] node{$y$} (M)
                        (M) edge[bend left, below] node{$z$} (N)
                        (S) edge[below, bend right] node{$y,z$} (E)
                        (N) edge[above, bend left] node{$x,z$} (E)
                        (M) edge[above] node{$x,y$} (E)
                        (E) edge[loop right] node{$x,y,z$} (E);
            \end{tikzpicture}
            \caption{AFD asociado al lenguaje $\cc{L}(G)$ del ejercicio \ref{ej:1.3.12}.}
            \label{fig:1.3.12-AFD}
        \end{figure}
        \item Encuentra una gramática regular por la izquierda que genere ese mismo lenguaje $\cc{L}(G)$.
        
        La expresión regular asociada al lenguaje $\cc{L}(G)$ es:
        \begin{equation*}
            x(yz)^{\ast}(y+\veps)
        \end{equation*}
        
        Sea $G'=(\{S,N,M, F\},\{x,y,z\},P,F)$ con $P$:
        \begin{align*}
            F & \rightarrow N\mid M\\
            N & \rightarrow Sx \mid Mz\\
            M & \rightarrow Ny\\
            S & \rightarrow \veps
        \end{align*}

        Tenemos que $G'$ es una gramática regular por la izquierda, y $\cc{L}(G')=\cc{L}(G)$.

        \item Encuentra el AFD que acepte el complementario del lenguaje $\cc{L}(G)$.
        
        Intercambiando los estados finales por no finales y viceversa en el AFD de la Figura~\ref{fig:1.3.12-AFD}, obtenemos el AFD asociado al complementario del lenguaje $\cc{L}(G)$, mostrado en la Figura~\ref{fig:1.3.12-AFD-Complementario}.
        \begin{figure}
            \centering
            \begin{tikzpicture}
                \node[state, initial, accepting] (S) {$S$};
                \node[state, right of=S] (N) {$N$};
                \node[state, right of=N] (M) {$M$};
                \node[state, error, right of=M, accepting] (E) {$E$};

                \draw   (S) edge[above] node{$x$} (N)
                        (N) edge[above] node{$y$} (M)
                        (M) edge[bend left, below] node{$z$} (N)
                        (S) edge[below, bend right] node{$y,z$} (E)
                        (N) edge[above, bend left] node{$x,z$} (E)
                        (M) edge[above] node{$x,y$} (E)
                        (E) edge[loop right] node{$x,y,z$} (E);
            \end{tikzpicture}
            \caption{AFD asociado al lenguaje $\ol{\cc{L}(G)}$ del ejercicio \ref{ej:1.3.12}.}
            \label{fig:1.3.12-AFD-Complementario}
        \end{figure}
    \end{enumerate}
\end{ejercicio}

\begin{ejercicio}\label{ej:1.3.13}
    Determinar autómatas minimales para los lenguajes $L(M_1) \cup L(M_2)$ y $L(M_1)\cap \overline{L(M_2)}$ donde,
    \begin{itemize}
        \item $M_1 = (\{q_0, q_1, q_2, q_3\}, \{a,b,c\},\delta_1,q_0, \{q_2\})$ donde
            \begin{equation*}
                \begin{array}{c|cccc}
                    \delta_1 & q_0 & q_1 & q_2 & q_3 \\ 
                    \hline
                    a & q_1 & q_1 & q_3 & q_3 \\ 
                    b & q_2 & q_1 & q_1 & q_3 \\ 
                    c & q_3 & q_3 & q_0 & q_3 
                \end{array}
            \end{equation*}
        \item $M_2 = (\{q_0, q_1, q_2, q_3\}, \{a,b,c\},\delta_2,q_0, \{q_2\})$ donde
            \begin{equation*}
                \begin{array}{c|cccc}
                    \delta_2 & q_0 & q_1 & q_2 & q_3 \\ 
                    \hline
                    a & q_1 & q_1 & q_3 & q_3 \\ 
                    b & q_1 & q_2 & q_2 & q_3 \\ 
                    c & q_3 & q_3 & q_0 & q_3 
                \end{array}
            \end{equation*}
    \end{itemize}

    En primer lugar, minimizamos $M_1$ y $M_2$. La miminización de $M_1$ se muestra en la Tabla \ref{tab:1.3.13-M1-Minimal}.
    \begin{table}
        \centering
        \begin{tabular}{r c c c}
            \hhline{~*{1}{-}}
            $q_1$ & \cell{\times} \\ \hhline{~*{2}{-}}
            $q_2$ & \cell{\times} & \cell{\times} \\ \hhline{~*{3}{-}}
            $q_3$ & \cell{\times} & \cell{(q_0,q_3)} & \cell{\times} \\ \hhline{~*{3}{-}}
            & $q_0$ & $q_1$ & $q_2$
        \end{tabular}
        \caption{Tabla de minimización de $M_1$.}
        \label{tab:1.3.13-M1-Minimal}
    \end{table}

    Por tanto, $\{q_1,q_3\}$ son indistinguibles, por lo que el AFD minimal asociado a $M_1$ es $M_1^m = (\{q_0,q_1,q_2\},\{a,b,c\},\delta_1^m,q_0,\{q_2\})$ donde:
    \begin{equation*}
        \begin{array}{c|ccc}
            \delta_1^m & q_0 & q_1 & q_2 \\ 
            \hline
            a & q_1 & q_1 & q_1 \\ 
            b & q_2 & q_1 & q_1 \\ 
            c & q_1 & q_1 & q_0 
        \end{array}
    \end{equation*}

    La miminización de $M_2$ se muestra en la Tabla \ref{tab:1.3.13-M2-Minimal}.
    \begin{table}
        \centering
        \begin{tabular}{r c c c}
            \hhline{~*{1}{-}}
            $q_1$ & \cell{\times} \\ \hhline{~*{2}{-}}
            $q_2$ & \cell{\times} & \cell{\times} \\ \hhline{~*{3}{-}}
            $q_3$ & \cell{\times} & \cell{\times} & \cell{\times} \\ \hhline{~*{3}{-}}
            & $q_0$ & $q_1$ & $q_2$
        \end{tabular}
        \caption{Tabla de minimización de $M_2$.}
        \label{tab:1.3.13-M2-Minimal}
    \end{table}

    Por tanto, el AFD minimal asociado a $M_2$ es $M_2^m=M_2$ minimal. A continuación, construimos los autómatas finitos deterministas asociados a $L(M_1) \cup L(M_2)$ y $L(M_1)\cap \overline{L(M_2)}$.
    \begin{enumerate}
        \item $L(M_1) \cup L(M_2)$
        
        En primer lugar, construimos el AFD asociado a $L(M_1) \cup L(M_2)$. Sea este $M=(Q,\{a,b,c\},\delta,(q_0,q_0'),F)$, donde:
        \begin{itemize}
            \item $Q=\{q_iq_j'\mid i=0,1,2\text{ y }j=0,1,2,3\}$.
            \item $F=\{(q_2,q_j')\mid j=0,1,2,3\} \cup \{(q_i,q_2')\mid i=0,1,2\}$.
            \item $\delta$ viene dada por la Tabla \ref{tab:transiciones}.            
            \begin{table}
                \centering
                \resizebox{1.1\textwidth}{!}{
                \begin{tabular}{r|cccccccccccc}
                    $\delta$ & $(q_0,q_0')$ & $(q_0,q_1')$ & $(q_0,q_2')$ & $(q_0,q_3')$ & $(q_1,q_0')$ & $(q_1,q_1')$ & $(q_1,q_2')$ & $(q_1,q_3')$ & $(q_2,q_0')$ & $(q_2,q_1')$ & $(q_2,q_2')$ & $(q_2,q_3')$ \\
                    \hline
                    $a$ & $q_1q_1'$ & $q_1q_1'$ & $q_1q_3'$ & $q_1q_3'$ & $q_1q_1'$ & $q_1q_1'$ & $q_1q_3'$ & $q_1q_3'$ & $q_1q_1'$ & $q_1q_1'$ & $q_1q_3'$ & $q_1q_3'$ \\
                    $b$ & $q_2q_1'$ & $q_2q_2'$ & $q_2q_2'$ & $q_2q_3'$ & $q_1q_1'$ & $q_1q_2'$ & $q_1q_2'$ & $q_1q_3'$ & $q_1q_1'$ & $q_1q_2'$ & $q_1q_2'$ & $q_1q_3'$ \\
                    $c$ & $q_1q_3'$ & $q_1q_3'$ & $q_1q_0'$ & $q_1q_3'$ & $q_1q_3'$ & $q_1q_3'$ & $q_1q_0'$ & $q_1q_3'$ & $q_0q_3'$ & $q_0q_3'$ & $q_0q_0'$ & $q_0q_3'$ \\
                \end{tabular}}
                \caption{Transiciones del autómata producto para $L(M_1),~L(M_2)$.}
                \label{tab:transiciones}
            \end{table}
        \end{itemize}

        Los estados accesibles son:
        \begin{equation*}
            \{q_0q_0',q_0q_3',q_1q_0',q_1q_1',q_1q_2',q_1q_3',q_2q_1',q_2q_3'\}
        \end{equation*}

        La minimización de $M$ se muestra en la Tabla \ref{tab:1.3.13-M-Union-Minimal}.
        \begin{table}
            \centering
            \begin{tabular}{r c c c c c c c}
                \hhline{~*{1}{-}}
                $q_0q_3'$ & \cell{\times} \\ \hhline{~*{2}{-}}
                $q_1q_0'$ & \cell{\times} & \cell{\times} \\ \hhline{~*{3}{-}}
                $q_1q_1'$ & \cell{\times} & \cell{\times} & \cell{\times} \\ \hhline{~*{4}{-}}
                $q_1q_2'$ & \cell{\times} & \cell{\times} & \cell{\times} & \cell{\times} \\ \hhline{~*{5}{-}}
                $q_1q_3'$ & \cell{\times} & \cell{\times} & \cell{\times} & \cell{\times} & \cell{\times} \\ \hhline{~*{6}{-}}
                $q_2q_1'$ & \cell{\times} & \cell{\times} & \cell{\times} & \cell{\times} & \cell{\times} & \cell{\times} \\ \hhline{~*{7}{-}}
                $q_2q_3'$ & \cell{\times} & \cell{\times}  & \cell{\times} & \cell{\times} & \cell{\times} & \cell{\times} & \cell{\times} \\ \hhline{~*{7}{-}}
                & $q_0q_0'$ & $q_0q_3'$ & $q_1q_0'$ & $q_1q_1'$ & $q_1q_2'$ & $q_1q_3'$ & $q_2q_1'$
            \end{tabular}
            \caption{Tabla de minimización de $M$ para $L(M_1) \cup L(M_2)$.}
            \label{tab:1.3.13-M-Union-Minimal}
        \end{table}

        Por tanto, el AFD minimal asociado a $L(M_1) \cup L(M_2)$ es $M^m=M$, el cual se puede ver en la Figura~\ref{fig:1.3.13-M-Union-Minimal}.
        \begin{figure}
            \centering
            \begin{tikzpicture}
                \node[state, initial] (q0q0) {$q_0q_0'$};
                \node[state, above right of=q0q0] (q1q1) {$q_1q_1'$};
                \node[state, below right of=q0q0, accepting] (q2q1) {$q_2q_1'$};
                \node[state, right of=q1q1, accepting] (q1q2) {$q_1q_2'$};
                \node[state, right of=q1q2] (q1q0) {$q_1q_0'$};
                \node[state, right of=q2q1] (q0q3) {$q_0q_3'$};
                \node[state, right of=q0q3, accepting] (q2q3) {$q_2q_3'$};
                \node[state, below right of=q1q0, error] (E) {$E$};

                \draw   (q0q0) edge[above] node{$a$} (q1q1)
                        (q0q0) edge[above] node{$b$} (q2q1)
                        (q1q1) edge[loop above] node{$a$} (q1q1)
                        (q1q1) edge[above] node{$b$} (q1q2)
                        (q1q2) edge[loop below] node{$b$} (q1q2)
                        (q1q2) edge[above] node{$c$} (q1q0)
                        (q1q0) edge[above, bend right] node{$a,b$} (q1q1)
                        (q2q1) edge[right] node{$a$} (q1q1)
                        (q2q1) edge[above] node{$b$} (q1q2)
                        (q2q1) edge[above] node{$c$} (q0q3)
                        (q0q3) edge[above] node{$b$} (q2q3)
                        (q2q3) edge[below, bend left] node{$c$} (q0q3);
                \draw[opacity=0.7]
                        (q0q0) edge[below, bend right=90, looseness=1.1] node{$c$} (E)
                        (q1q1) edge[bend left=90, above] node{$c$} (E)
                        (q1q2) edge[above] node{$a$} (E)
                        (q1q0) edge[above] node{$c$} (E)
                        (q0q3) edge[above] node[pos=0.1]{$a,c$} (E)
                        (q2q3) edge[above] node[pos=0.1]{$a,b$} (E)
                        (E) edge[loop right] node{$a,b,c$} (E);
            \end{tikzpicture}
            \caption{AFD minimal asociado a $L(M_1) \cup L(M_2)$.}
            \label{fig:1.3.13-M-Union-Minimal}
        \end{figure}

        \item $L(M_1)\cap \overline{L(M_2)}$
        
        Ya tenemos del apartado anterior el AFD minimal asociado a $L(M_1)$. La minimización de $\ol{L(M_2)}$ se muestra en la Tabla \ref{tab:1.3.13-M2-Complement-Minimal}.
        \begin{table}
            \centering
            \begin{tabular}{r c c c}
                \hhline{~*{1}{-}}
                $q_1$ & \cell{\times} \\ \hhline{~*{2}{-}}
                $q_2$ & \cell{\times} & \cell{\times} \\ \hhline{~*{3}{-}}
                $q_3$ & \cell{\times} & \cell{\times} & \cell{\times} \\ \hhline{~*{3}{-}}
                & $q_0$ & $q_1$ & $q_2$
            \end{tabular}
            \caption{Tabla de minimización de $\ol{M_2}$.}
            \label{tab:1.3.13-M2-Complement-Minimal}
        \end{table}

        Por tanto, el AFD minimal asociado a $\ol{L(M_2)}$ es ya minimal. A continuación, construimos el AFD asociado a $L(M_1)\cap \overline{L(M_2)}$. Sea este $M=(Q,\{a,b,c\},\delta,(q_0,q_0'),F)$, donde:
        \begin{itemize}
            \item $Q=\{q_iq_j'\mid i=0,1,2\text{ y }j=0,1,2,3\}$.
            \item $F=\{(q_2,q_j')\mid j=0,1,3\}$.
            \item $\delta$ viene dada por la Tabla \ref{tab:transiciones}.
        \end{itemize}

        Los estados accesibles son los mismos que antes:
        \begin{equation*}
            \{q_0q_0',q_0q_3',q_1q_0',q_1q_1',q_1q_2',q_1q_3',q_2q_1',q_2q_3'\}
        \end{equation*}

        La minimización de $M$ se muestra en la Tabla \ref{tab:1.3.13-M-Intersection-Minimal}.
        \begin{table}
            \centering
            \begin{tabular}{r c c c c c c c}
                \hhline{~*{1}{-}}
                $q_0q_3'$ & \cell{} \\ \hhline{~*{2}{-}}
                $q_1q_0'$ & \cell{\times} & \cell{\times} \\ \hhline{~*{3}{-}}
                $q_1q_1'$ & \cell{\times} & \cell{\times} & \cell{} \\ \hhline{~*{4}{-}}
                $q_1q_2'$ & \cell{\times} & \cell{\times} & \cell{} & \cell{{\begin{array}{c}(q_1q_2',q_1q_0')\\(q_qq_1',q_1q_0')\end{array}}} \\ \hhline{~*{5}{-}}
                $q_1q_3'$ & \cell{\times} & \cell{\times} & \cell{{\begin{array}{c}(q_1q_2',q_1q_3')\\(q_1q_2',q_1q_0')\\(q_1q_1',q_1q_2')\\(q0q_3',q_1q_0')\end{array}}} & \cell{{\begin{array}{c}(q_2q_3',q_2q_1')\\(q_1q_3',q_1q_0')\\(q_1q_1',q_1q_2')\end{array}}} & \cell{{\begin{array}{c}(q_2q_3',q_2q_1')\\(q_1q_1',q_1q_3')\end{array}}} \\ \hhline{~*{6}{-}}
                $q_2q_1'$ & \cell{\times} & \cell{\times} & \cell{\times} & \cell{\times} & \cell{\times} & \cell{\times} \\ \hhline{~*{7}{-}}
                $q_2q_3'$ & \cell{\times} & \cell{\times}  & \cell{\times} & \cell{\times} & \cell{\times} & \cell{\times} & \cell{(q_0q_3',q_0q_0')} \\ \hhline{~*{7}{-}}
                & $q_0q_0'$ & $q_0q_3'$ & $q_1q_0'$ & $q_1q_1'$ & $q_1q_2'$ & $q_1q_3'$ & $q_2q_1'$
            \end{tabular}
            \caption{Tabla de minimización de $M$ para $L(M_1) \cap \ol{L(M_2)}$.}
            \label{tab:1.3.13-M-Intersection-Minimal}
        \end{table}
        
        Por tanto, notando por $\equiv$ a la relación de indistinguibilidad, tenemos que:
        \begin{equation*}
            q_0q_3'\equiv q_0q_0':=q_0\qquad
            q_1q_1'\equiv q_1q_0'\equiv q_1q_2'\equiv q_1q_3':=q_1\qquad
            q_2q_1'\equiv q_2q_3' := q_2
        \end{equation*}

        Por tanto, el AFD minimal asociado a $L(M_1)\cap \overline{L(M_2)}$ es $M^m=(Q,\{a,b,c\},\delta^m,q_0,F)$ donde:
        \begin{itemize}
            \item $Q=\{q_0,q_0q_1',q_0q_2',q_1,q_2,q_2q_0',q_2q_2'\}$.
            \item $F=\{q_2, q_2q_0'\}$.
            \item Los estados accesibles son $\{q_0,q_1,q_2\}$.
            \item $\delta^m$ viene dada por la Tabla \ref{tab:transiciones-interseccion}, donde solo la definimos para los estados accesibles.
            \begin{table}
                \centering
                \begin{tabular}{r|cccccccccccc}
                    $\delta^m$ & $q_0$ & $q_1$ & $q_2$ \\
                    \hline
                    $a$ & $q_1$ & $q_1$ & $q_1$ \\
                    $b$ & $q_2$ & $q_1$ & $q_1$ \\
                    $c$ & $q_1$ & $q_1$ & $q_0$
                \end{tabular}
                \caption{Transiciones del autómata minimal para $L(M_1),~\ol{L(M_2)}$.}
                \label{tab:transiciones-interseccion}
            \end{table}
        \end{itemize}

        El AFD minimal asociado a $L(M_1)\cap \overline{L(M_2)}$ es el de la Figura~\ref{fig:1.3.13-M-Intersection-Minimal}.
        \begin{figure}
            \centering
            \begin{tikzpicture}
                \node[state, initial] (q0) {$q_0$};
                \node[state, below right of=q0] (q1) {$q_1$};
                \node[state, accepting, above right of=q1] (q2) {$q_2$};

                \draw   (q0) edge[below left] node{$a,c$} (q1)
                        (q0) edge[above] node{$b$} (q2)
                        (q1) edge[loop right] node{$a,b,c$} (q1)
                        (q2) edge[below right] node{$a,b$} (q1)
                        (q2) edge[above, bend right] node{$c$} (q0);
            \end{tikzpicture}
            \caption{AFD minimal asociado a $L(M_1)\cap \ol{L(M_2)}$.}
            \label{fig:1.3.13-M-Intersection-Minimal}
        \end{figure}
    \end{enumerate}
\end{ejercicio}

\begin{ejercicio}\label{ej:1.3.14}
    Dado el conjunto regular representado por la expresión regular $a^\ast b^\ast + b^\ast a^\ast$, construir un autómata finido determinístico minimal que lo acepte.\\

    El AFND con transiciones nulas asociado a la expresión regular dada es el de la Figura~\ref{fig:1.3.14-AFND}.
    \begin{figure}
        \centering
        \begin{tikzpicture}
            \node[state, initial] (q0) {$q_0$};
            \node[state, above right of=q0] (q1) {$q_1$};
            \node[state, below right of=q0] (q2) {$q_2$};
            \node[state, right of=q1, accepting] (q3) {$q_3$};
            \node[state, right of=q2, accepting] (q4) {$q_4$};

            \draw   (q0) edge[above] node{$\varepsilon$} (q1)
                    (q0) edge[below] node{$\varepsilon$} (q2)
                    (q1) edge[above] node{$\veps$} (q3)
                    (q1) edge[loop below] node{$a$} (q1)
                    (q3) edge[loop below] node{$b$} (q3)
                    (q2) edge[below] node{$\veps$} (q4)
                    (q2) edge[loop above] node{$b$} (q2)
                    (q4) edge[loop above] node{$a$} (q4);
        \end{tikzpicture}
        \caption{AFND con transiciones nulas asociado a la expresión regular $a^\ast b^\ast + b^\ast a^\ast$.}
        \label{fig:1.3.14-AFND}
    \end{figure}

    El AFD asociado a la expresión regular dada es el de la Figura~\ref{fig:1.3.14-AFD}.
    \begin{figure}
        \centering
        \begin{tikzpicture}
            \node[state, initial, accepting] (q0q1q2q3q4q5) {$q_0q_1q_2q_3q_4q_5$};
            \node[state, above right of=q0q1q2q3q4q5, accepting] (q1q4) {$q_1q_4$};
            \node[state, below right of=q0q1q2q3q4q5, accepting] (q2q3) {$q_2q_3$};
            \node[state, right of=q1q4, accepting] (q3) {$q_3$};
            \node[state, right of=q2q3, accepting] (q4) {$q_4$};
            \node[state, error, above right of=q4] (E) {$E$};

            \draw   (q0q1q2q3q4q5) edge[above] node{$a$} (q1q4)
                    (q0q1q2q3q4q5) edge[below] node{$b$} (q2q3)
                    (q1q4) edge[above] node{$b$} (q3)
                    (q1q4) edge[loop below] node{$a$} (q1q4)
                    (q2q3) edge[below] node{$a$} (q4)
                    (q2q3) edge[loop above] node{$b$} (q2q3)
                    (q3) edge[loop below] node{$b$} (q3)
                    (q4) edge[loop above] node{$a$} (q4)
                    (q3) edge[above] node{$a$} (E)
                    (q4) edge[below] node{$b$} (E)
                    (E) edge[loop right] node{$a,b$} (E);
        \end{tikzpicture}
        \caption{AFD asociado a la expresión regular $a^\ast b^\ast + b^\ast a^\ast$.}
        \label{fig:1.3.14-AFD}
    \end{figure}

    Veamos ahora que el AFD de la Figura~\ref{fig:1.3.14-AFD} es minimal.
    \begin{itemize}
        \item El estado $E$ es distinguible de cualquier otro estado pues ser el único estado que no es final.
        \item Veamos que $q_3$ es distinguible del resto de estados finales. Leyendo $a$, tenemos que:
        \begin{equation*}
            \delta(q_3,a)=E\notin F,\qquad \left\{\begin{array}{l}
                \delta(q_1q_4,a)=q_1q_4\in F,\\
                \delta(q_2q_3,a)=q_4\in F,\\
                \delta(q_4,a)=q_4\in F,\\
                \delta(q_0q_1q_2q_3q_4q_5,a)=q_1q_4\in F.
            \end{array}
            \right.
        \end{equation*}
        \item De forma análoga leyendo $b$, tenemos que $q_4$ es distinguible del resto de estados finales.
        \item El estado $q_1q_4$ es distinguible de $q_2q_3$ y $q_0q_1q_2q_3q_4q_5$ leyendo $ba$.
        \item Finalmente, $q_0q_1q_2q_3q_4q_5$ es distinguible de $q_2q_3$ leyendo $ab$.
    \end{itemize}
\end{ejercicio}

\begin{ejercicio}\label{ej:1.3.15}
    Sean los lenguajes:
    \begin{enumerate}
        \item $L_1={(01+1)}^{\ast}00$
        \item $L_2=01{(01+1)}^{\ast}$
    \end{enumerate}
    construir un autómata finito determinístico minimal que acepte el lenguaje $L_1 \setminus L_2$, a partir de autómatas que acepten $L_1$ y $L_2$.

    Veamos que $L_1\cap L_2=\emptyset$. Sea $z\in L_1$. Entonces, $z$ es admitida por la expresión regular $(01+1)^{\ast}00$, por lo que termina en $0$. Por tanto, $z\notin L_2$, ya que todas las palabras de $L_2$ terminan por $1$. Por tanto, $L_1\cap L_2=\emptyset$ y $L_1\setminus L_2=L_1$. Aun así, haremos el proceso algorítmico para obtener el AFD minimal asociado a $L_1\setminus L_2$.\\

    El AFD asociado a $L_1$ es el de la Figura~\ref{fig:1.3.15-L1}.
    \begin{figure}
        \centering
        \begin{tikzpicture}
            \node[state, initial] (q0) {$q_0$};
            \node[state, right of=q0] (q1) {$q_1$};
            \node[state, right of=q1, accepting] (q2) {$q_2$};
            \node[state, error, right of=q2] (E) {$q_E$};

            \draw   (q0) edge[above] node{$0$} (q1)
                    (q0) edge[loop above] node{$1$} (q0)
                    (q1) edge[above, bend right] node{$1$} (q0)
                    (q1) edge[above] node{$0$} (q2)
                    (q2) edge[above] node{$0,1$} (E)
                    (E) edge[loop above] node{$0,1$} (E);
        \end{tikzpicture}
        \caption{AFD minimal asociado a $L_1={(01+1)}^{\ast}00$.}
        \label{fig:1.3.15-L1}
    \end{figure}

    El AFD asociado a $L_2$ es el de la Figura~\ref{fig:1.3.15-L2}.
    \begin{figure}
        \centering
        \begin{tikzpicture}
            \node[state, initial] (q0) {$p_0$};
            \node[state, right of=q0] (q1) {$p_1$};
            \node[state, right of=q1, accepting] (q2) {$p_2$};
            \node[state, error, below of=q1] (E) {$p_E$};

            \draw   (q0) edge[above] node{$0$} (q1)
                    (q0) edge[below] node{$1$} (E)
                    (q1) edge[above] node{$1$} (q2)
                    (q1) edge[left] node{$0$} (E)
                    (q2) edge[loop above] node{$1$} (q2)
                    (q2) edge[above, bend right] node{$0$} (q1)
                    (E) edge[loop right] node{$0,1$} (E);
        \end{tikzpicture}
        \caption{AFD minimal asociado a $L_2=01{(01+1)}^{\ast}$.}
        \label{fig:1.3.15-L2}
    \end{figure}

    El AFD asociado a $L_1\setminus L_2$ es el de la Figura~\ref{fig:1.3.15-L1-L2}.
    \begin{figure}
        \centering
        \begin{tikzpicture}
            \node[state, initial] (q0q0) {$q_0p_0$};
            \node[state, above right of=q0q0] (q1p1) {$q_1p_1$};
            \node[state, below right of=q0q0] (q0pE) {$q_0p_E$};
            \node[state, above right of=q1q1, accepting, yshift=-2em] (q2pE) {$q_2p_E$};
            \node[state, below right of=q1q1, yshift=2em] (q0p2) {$q_0p_2$};
            \node[state, right of=q0pE] (q1pE) {$q_1p_E$};
            \node[state, right of=q2pE, error] (qEpE) {$q_Ep_E$};

            \draw   (q0q0) edge[above] node{$0$} (q1p1)
                    (q0q0) edge[below] node{$1$} (q0pE)
                    (q1p1) edge[above] node{$1$} (q0p2)
                    (q1p1) edge[above] node{$0$} (q2pE)
                    (q0pE) edge[above] node{$0$} (q1pE)
                    (q0pE) edge[loop above] node{$1$} (q0pE)
                    (q0p2) edge[below, bend left] node{$0$} (q1p1)
                    (q0p2) edge[loop above] node{$1$} (q0p2)
                    (q1pE) edge[below, bend left] node{$1$} (q0pE)
                    (q1pE) edge[right, bend right] node{$0$} (q2pE)
                    (q2pE) edge[above] node{$0,1$} (qEpE)
                    (qEpE) edge[loop below] node{$0,1$} (qEpE);
        \end{tikzpicture}
        \caption{AFD asociado a $L_1\setminus L_2$.}
        \label{fig:1.3.15-L1-L2}
    \end{figure}

    La minimización del AFD de la Figura~\ref{fig:1.3.15-L1-L2} se muestra en la Tabla~\ref{tab:1.3.15-L1-L2-Minimal}.
    \begin{table}
        \centering
        \begin{tabular}{r c c c c c c c}
            \hhline{~*{1}{-}}
            $q_1p_1$ & \cell{\times} \\ \hhline{~*{2}{-}}
            $q_2p_E$ & \cell{\times} & \cell{\times} \\ \hhline{~*{3}{-}}
            $q_0p_2$ & \cell{} & \cell{\times} & \cell{\times} \\ \hhline{~*{4}{-}}
            $q_0p_E$ & \cell{} & \cell{\times} & \cell{\times} & \cell{{\begin{array}{c}(q_1p_E,~q_1p_1)\\(q_0p_0,q_0p_2)\end{array}}} \\ \hhline{~*{5}{-}}
            $q_1p_E$ & \cell{\times} & \cell{{\begin{array}{c}(q_0p_E,q_0p_0)\\(q_0p_E,q_0p_2)\end{array}}} & \cell{\times} & \cell{\times} & \cell{\times} \\ \hhline{~*{6}{-}}
            $q_Ep_E$ & \cell{\times} & \cell{\times} & \cell{\times} & \cell{\times} & \cell{\times} & \cell{\times} \\ \hhline{~*{6}{-}}
            & $q_0p_0$ & $q_1p_1$ & $q_2p_E$ & $q_0p_2$ & $q_0p_E$ & $q_1p_E$
        \end{tabular}
        \caption{Tabla de minimización de $L_1\setminus L_2$.}
        \label{tab:1.3.15-L1-L2-Minimal}
    \end{table}

    Por tanto, notando por $\equiv$ a la relación de indistinguibilidad, tenemos que:
    \begin{equation*}
        q_0p_0\equiv q_0p_E\equiv q_0p_2
        \qquad q_1p_1\equiv q_1p_E
    \end{equation*}

    El AFD minimal asociado a $L_1\setminus L_2$ es el de la Figura~\ref{fig:1.3.15-L1-L2-Minimal}.
    Como vemos, este autómata es isomorfo al autómata de la Figura~\ref{fig:1.3.15-L1} que aceptaba $L_1$, como ya habíamos predicho.
    \begin{figure}
        \centering
        \begin{tikzpicture}
            \node[state, initial] (q0) {$q_0p_0,q_0p_E,q_0p_2$};
            \node[state, right of=q0, xshift=4em] (q1) {$q_1p_1,q_1p_E$};
            \node[state, right of=q1, accepting] (q2) {$q_2p_E$};
            \node[state, error, right of=q2] (E) {$q_Ep_E$};

            \draw   (q0) edge[above] node{$0$} (q1)
                    (q0) edge[loop above] node{$1$} (q0)
                    (q1) edge[above, bend right] node{$1$} (q0)
                    (q1) edge[above] node{$0$} (q2)
                    (q2) edge[above] node{$0,1$} (E)
                    (E) edge[loop above] node{$0,1$} (E);
        \end{tikzpicture}
        \caption{AFD minimal asociado a $L_1\setminus L_2$.}
        \label{fig:1.3.15-L1-L2-Minimal}
    \end{figure}
\end{ejercicio}

\begin{ejercicio}\label{ej:1.3.16}
    Dados los alfabetos $A=\{0,1,2,3\}$ y $B=\{0,1\}$ y el homomorfismo $f$ de $A^\ast$ en $B^\ast$ dado por:
    \begin{equation*}
        f(0)=00,\quad f(1)=01,\quad f(2)=10,\quad f(3)=11
    \end{equation*}
    Sea $L$ el conjunto de las palabras de $B^\ast$ en las que el número de símbolos 0 es par y el de símbolos 1 no es múltiplo de 3. Construir un autómata finito determinista que acepte el lenguaje $f^{-1}(L)$.\\

    El AFD que acepta a $L$ se desarrolló en el Ejercicio~\ref{ej:1.2.17}. Se muestra no obstante de nuevo en la Figura~\ref{fig:ej:1.3.16-L}.
    \begin{figure}
        \centering
        \begin{tikzpicture}
            \node[state, initial] (q00) {$q_{00}$};
            \node[state, right of=q00, accepting] (q01) {$q_{01}$};
            \node[state, right of=q01, accepting] (q02) {$q_{02}$};
            \node[state, below of=q00] (q10) {$q_{10}$};
            \node[state, right of=q10] (q11) {$q_{11}$};
            \node[state, right of=q11] (q12) {$q_{12}$};

            \draw   (q00) edge[above] node{1} (q01)
                    (q01) edge[above] node{1} (q02)
                    (q02) edge[above, bend right] node{1} (q00)
                    (q10) edge[above] node{1} (q11)
                    (q11) edge[above] node{1} (q12)
                    (q12) edge[below, bend left] node{1} (q10)
                    (q00) edge[left] node{0} (q10)
                    (q10) edge[right, bend right] node{0} (q00)
                    (q01) edge[left] node{0} (q11)
                    (q11) edge[right, bend right] node{0} (q01)
                    (q02) edge[left] node{0} (q12)
                    (q12) edge[right, bend right] node{0} (q02);
        \end{tikzpicture}
        \caption{AFD que acepta el lenguaje $L$ del Ejercicio~\ref{ej:1.3.16}.}
        \label{fig:ej:1.3.16-L}
    \end{figure}

    El AFD que acepta a $f^{-1}(L)$ es el de la Figura~\ref{fig:ej:1.3.16-f-1-L}.
    \begin{figure}
        \centering
        \begin{tikzpicture}
            \node[state, initial] (q00) {$q_{00}$};
            \node[state, right of=q00, accepting] (q01) {$q_{01}$};
            \node[state, right of=q01, accepting] (q02) {$q_{02}$};
            \node[state, below of=q02] (q10) {$q_{10}$};
            \node[state, below of=q00] (q11) {$q_{11}$};
            \node[state, below of=q01] (q12) {$q_{12}$};

            \draw   (q00) edge[above, loop above] node{0} (q00)
                    (q00) edge[left] node{1,2} (q11)
                    (q00) edge[above, bend left=60] node{3} (q02)
                    (q01) edge[above, loop above] node{0} (q01)
                    (q01) edge[left] node{1,2} (q12)
                    (q01) edge[above] node{3} (q00)
                    (q02) edge[above, loop above] node{0} (q02)
                    (q02) edge[left] node{1,2} (q10)
                    (q02) edge[above] node{3} (q01)
                    (q10) edge[below, loop below] node{0} (q10)
                    (q10) edge[above] node{1,2} (q01)
                    (q10) edge[above] node{3} (q12)
                    (q11) edge[below, loop below] node{0} (q11)
                    (q11) edge[below, bend right=90, looseness=2] node{1,2} (q02)
                    (q11) edge[below, bend right=60] node{3} (q10)
                    (q12) edge[below, loop below] node{0} (q12)
                    (q12) edge[above] node{1,2} (q00)
                    (q12) edge[above] node{3} (q11);
        \end{tikzpicture}
        \vspace{-2cm}
        \caption{AFD que acepta el lenguaje $L$ del Ejercicio~\ref{ej:1.3.16}.}
        \label{fig:ej:1.3.16-f-1-L}
    \end{figure}
\end{ejercicio}

\begin{ejercicio}\label{ej:1.3.17}
    Determinar un autómata finito determinístico minimal para el lenguaje sobre el alfabeto $A=\{a,b,c\}$ dado por la expresión regular $b{(a+b)}^{\ast}+cb^\ast$.\\

    El AFD asociado a la expresión regular dada es el de la Figura~\ref{fig:1.3.17-AFD}.
    \begin{figure}
        \centering
        \begin{tikzpicture}
            \node[state, initial] (q0) {$q_0$};
            \node[state, above right of=q0, accepting] (q1) {$q_1$};
            \node[state, below right of=q0, accepting] (q2) {$q_2$};
            \node[state, below right of=q1, error] (E) {$E$};

            \draw   (q0) edge[above] node{$b$} (q1)
                    (q0) edge[below] node{$c$} (q2)
                    (q0) edge[above] node{$a$} (E)
                    (q1) edge[loop above] node{$a,b$} (q1)
                    (q1) edge[above] node{$c$} (E)
                    (q2) edge[loop below] node{$b$} (q2)
                    (q2) edge[below] node{$a,b$} (E)
                    (E) edge[loop right] node{$a,b,c$} (E);
        \end{tikzpicture}
        \caption{AFD asociado a la expresión regular $b{(a+b)}^{\ast}+cb^\ast$.}
        \label{fig:1.3.17-AFD}
    \end{figure}

    Veamos que es minimal:
    \begin{itemize}
        \item $E$ es distinguible de cualquier otro estado puesto que desde él no se puede llegar a un estado final.
        \item $q_0$ es distinguible de $q_1$ y $q_2$ puesto que no es final.
        \item $q_1$ es distinguible de $q_2$ leyendo $a$.
    \end{itemize}
\end{ejercicio}

\begin{ejercicio}\label{ej:1.3.18}
    Determinar si las expresiones regulares siguientes representan el mismo lenguaje:
    \begin{enumerate}
        \item $L_1={(b+(c+a)a^\ast (b+c))}^{\ast} (c+a)a^\ast$
        \item $L_2=b^\ast (c+a) {((b+c)b^\ast (c+a))}^{\ast}a^\ast$
        \item $L_3=b^\ast (c+a){(a^\ast (b+c)b^\ast (c+a))}^{\ast}a^\ast$
    \end{enumerate}
    Justificar la respuesta.\\

    Veamos en primer lugar que $L_2\neq L_1,L_3$. Sea la palabra $u=caaccaaa$.
    Tenemos que $u\in L_1, L_3$ pero $u\notin L_2$. Por tanto, $L_2\neq L_1, L_3$.\\

    Veamos ahora que $L_1=L_3$ obteniendo el autómata finito minimal asociado a cada uno de ellos y viendo que es igual. Este se encuentra en la Figura~\ref{fig:1.3.18-L1-L3}.
    \begin{figure}
        \centering
        \begin{tikzpicture}
            \node[state, initial] (q0) {$q_0$};
            \node[state, right of=q0, accepting] (q1) {$q_1$};

            \draw   (q0) edge[loop above] node{$b$} (q0)
                    (q0) edge[above] node{$a,c$} (q1)
                    (q1) edge[loop above] node{$a$} (q1)
                    (q1) edge[below, bend left] node{$b,c$} (q0);
        \end{tikzpicture}
        \caption{AFD minimal asociado a $L_1=L_3$ del Ejercicio~\ref{ej:1.3.18}.}
        \label{fig:1.3.18-L1-L3}
    \end{figure}
\end{ejercicio}

\begin{ejercicio}\label{ej:1.3.19}
    Construir un autómata finito determinista minimal que acepte el conjunto de palabras sobre el alfabeto $A=\{0,1\}$ que representen números no divisibles por dos ni por tres (en binario).

    El AFD asociado a los múltiplos de $2$ se muestra en la Figura~\ref{fig:1.3.19-Multiplos-2}.
    \begin{figure}
        \centering
        \begin{tikzpicture}
            \node[state, accepting, initial] (q0) {$q_0$};
            \node[state, right of=q0] (q1) {$q_1$};

            \draw   (q0) edge[loop above] node{$0$} (q0)
                    (q0) edge[above] node{$1$} (q1)
                    (q1) edge[loop above] node{$1$} (q1)
                    (q1) edge[above, bend right] node{$0$} (q0);
        \end{tikzpicture}
        \caption{AFD minimal asociado a los múltiplos de $2$ del Ejercicio~\ref{ej:1.3.19}.}
        \label{fig:1.3.19-Multiplos-2}
    \end{figure}

    El AFD asociado a los múltiplos de $3$ se muestra en la Figura~\ref{fig:1.3.19-Multiplos-3}.
    \begin{figure}
        \centering
        \begin{tikzpicture}
            \node[state, accepting, initial] (p0) {$p_0$};
            \node[state, right of=p0] (p1) {$p_1$};
            \node[state, right of=p1] (p2) {$p_2$};

            \draw   (p0) edge[loop above] node{$0$} (p0)
                    (p0) edge[above] node{$1$} (p1)
                    (p1) edge[above] node{$0$} (p2)
                    (p1) edge[bend right, above] node{$1$} (p0)
                    (p2) edge[loop above] node{$1$} (p2)
                    (p2) edge[above, bend right] node{$0$} (p1);
        \end{tikzpicture}
        \caption{AFD minimal asociado a los múltiplos de $3$ del Ejercicio~\ref{ej:1.3.19}.}
        \label{fig:1.3.19-Multiplos-3}
    \end{figure}

    El autómata descrito en el enunciado es el autómata producto de los complementarios a los dos anteriores, mostrado en la Figura~\ref{fig:1.3.19-Producto}.
    \begin{figure}
        \centering
        \begin{tikzpicture}
            \node[state, initial] (q0p0) {$q_0p_0$};
            \node[state, right of=q0p0, accepting] (q1p1) {$q_1p_1$};
            \node[state, above right of=q1p1] (q1p0) {$q_1p_0$};
            \node[state, right of=q1p0] (q0p1) {$q_0p_1$};
            \node[state, below right of=q1p1] (q0p2) {$q_0p_2$};
            \node[state, right of=q0p2, accepting] (q1p2) {$q_1p_2$};

            \draw   (q0p0) edge[loop above] node{$0$} (q0p0)
                    (q0p0) edge[above] node{$1$} (q1p1)
                    (q0p1) edge[above] node{$0$} (q0p2)
                    (q0p1) edge[above] node{$1$} (q1p0)
                    (q0p2) edge[left, bend left] node{$0$} (q0p1)
                    (q0p2) edge[above] node{$1$} (q1p2)
                    (q1p0) edge[above] node{$0$} (q0p0)
                    (q1p0) edge[right, bend left] node{$1$} (q1p1)
                    (q1p1) edge[above] node{$0$} (q0p2)
                    (q1p1) edge[above] node{$1$} (q1p0)
                    (q1p2) edge[right] node{$0$} (q0p1)
                    (q1p2) edge[loop right] node{$1$} (q1p2);
        \end{tikzpicture}
        \caption{AFD asociado al lenguaje del Ejercicio~\ref{ej:1.3.19}.}
        \label{fig:1.3.19-Producto}
    \end{figure}

    La minimización del autómata de la Figura~\ref{fig:1.3.19-Producto} se muestra en la Tabla~\ref{tab:1.3.19-Producto-Minimal}.
    \begin{table}
        \centering
        \begin{tabular}{r c c c c c c c}
            \hhline{~*{1}{-}}
            $q_1p_1$ & \cell{\times} \\ \hhline{~*{2}{-}}
            $q_1p_0$ & \cell{} & \cell{\times} \\ \hhline{~*{3}{-}}
            $q_0p_2$ & \cell{\times} & \cell{\times} & \cell{\times} \\ \hhline{~*{4}{-}}
            $q_0p_1$ & \cell{\times} & \cell{\times} & \cell{\times} & \cell{\times} \\ \hhline{~*{5}{-}}
            $q_1p_2$ & \cell{\times} & \cell{\times} & \cell{\times} & \cell{\times} & \cell{\times} \\ \hhline{~*{6}{-}}
            & $q_0p_0$ & $q_1p_1$ & $q_1p_0$ & $q_0p_2$ & $q_1p_1$
        \end{tabular}
        \caption{Tabla de minimización del autómata de la Figura~\ref{fig:1.3.19-Producto}.}
        \label{tab:1.3.19-Producto-Minimal}
    \end{table}

    Por tanto, notando por $\equiv$ a la relación de indistinguibilidad, tenemos que:
    \begin{equation*}
        q_0p_0\equiv q_1p_0\equiv: p_0
    \end{equation*}

    Por tanto, el autómata minimal asociado al lenguaje del enunciado es el de la Figura~\ref{fig:1.3.19-Producto-Minimal}.
    \begin{figure}
        \centering
        \begin{tikzpicture}
            \node[state, initial] (p0) {$p_0$};
            \node[state, below of=p0, accepting] (q1p1) {$q_1p_1$};
            \node[state, right of=p0] (q0p1) {$q_0p_1$};
            \node[state, right of=q1p1] (q0p2) {$q_0p_2$};
            \node[state, right of=q0p2, accepting] (q1p2) {$q_1p_2$};

            \draw   (p0) edge[loop above] node{$0$} (p0)
                    (p0) edge[right] node{$1$} (q1p1)
                    (q0p1) edge[right] node{$0$} (q0p2)
                    (q0p1) edge[above] node{$1$} (p0)
                    (q0p2) edge[left, bend left] node{$0$} (q0p1)
                    (q0p2) edge[above] node{$1$} (q1p2)
                    (q1p1) edge[above] node{$0$} (q0p2)
                    (q1p1) edge[left, bend left] node{$1$} (p0)
                    (q1p2) edge[right] node{$0$} (q0p1)
                    (q1p2) edge[loop right] node{$1$} (q1p2);
        \end{tikzpicture}
        \caption{AFD Minimal asociado al lenguaje del Ejercicio~\ref{ej:1.3.19}.}
        \label{fig:1.3.19-Producto-Minimal}
    \end{figure}
\end{ejercicio}

\begin{ejercicio}\label{ej:1.3.20}
    Determinar una expresión regular para los siguientes lenguajes sobre el alfabeto $\{0,1\}$:
    \begin{itemize}
        \item Palabras en las que el tercer símbolo es un 0.
        \item Palabras en las que el antepenúltimo símbolo es un 1.
    \end{itemize}
    Construir un autómata finito minimal que acepte la intersección de ambos lenguajes.\\

    Respecto al leguaje de las palabras en las que el tercer símbolo es un $0$ (llamémoslo $L_1$), su expresión regular es:
    \begin{equation*}
        (0+1)(0+1) \ \red{0}\  {(0+1)}^{\ast}
    \end{equation*}
    

    Respecto al leguaje de las palabras en las que el antepenúltimo símbolo es un $1$ (llamémoslo $L_2$), su expresión regular es:
    \begin{equation*}
        (0+1)^*\ \red{1}\ (0+1)(0+1)
    \end{equation*}

    \begin{description}
        \item[Opción 1:]  Autómata Producto.
        
        Obtendremos el AFD de $L_1$ de forma directa, mostrado en la Figura~\ref{fig:ej:1.3.20-1:AFD}.
        \begin{figure}
            \centering
            \begin{tikzpicture}
                \node[state, initial] (p0) {$p_0$};
                \node[state, right of=p0] (p1) {$p_1$};
                \node[state, right of=p1] (p2) {$p_2$};
                \node[state, accepting, right of=p2] (p3) {$p_3$};
                \node[state, error, below of=p2] (E) {$E$};

                \draw   (p0) edge[above] node{$0,1$} (p1)
                        (p1) edge[above] node{$0,1$} (p2)
                        (p2) edge[above] node{$0$} (p3)
                        (p3) edge[loop above] node{$0,1$} (p3)
                        (p2) edge[right] node{$1$} (E)
                        (E) edge[loop below] node{$0,1$} (E);
            \end{tikzpicture}
            \caption{AFD minimal que acepta el lenguaje $L_1$ del ejercicio~\ref{ej:1.3.20}.}
            \label{fig:ej:1.3.20-1:AFD}
        \end{figure}

        Podríamos optar con construir el AFD de $L_2$ de forma directa, pero construiremos un AFND que acepte el lenguaje y luego lo convertiremos a AFD. El AFND se muestra en la Figura~\ref{fig:ej:1.3.20:AFND}, cuyos estados son:
        \begin{itemize}
            \item $q_0$: No estamos en la cadena final, por lo que podemos leer $0$'s y $1$'s.
            \item $q_1$: Acabo de empezar la cadena final. He leído un $1$.
            \item $q_2$: Estoy en la cadena final. El leído el $1$ y el segundo símbolo.
            \item $q_3$: Hemos terminado la cadena final.
        \end{itemize}
        \begin{figure}
            \centering
            \begin{tikzpicture}
                \node[state, initial] (q0) {$q_0$};
                \node[state, right of=q0] (q1) {$q_1$};
                \node[state, right of=q1] (q2) {$q_2$};
                \node[state, accepting, right of=q2] (q3) {$q_3$};

                \draw   (q0) edge[loop above] node{0,1} (q0)
                        (q0) edge[above] node{1} (q1)
                        (q1) edge[above] node{0,1} (q2)
                        (q2) edge[above] node{0,1} (q3);
            \end{tikzpicture}
            \caption{AFND que acepta el lenguaje $L_2$ del ejercicio~\ref{ej:1.3.20}.}
            \label{fig:ej:1.3.20:AFND}
        \end{figure}

        Convertimos ahora el AFND de la Figura \ref{fig:ej:1.3.20:AFND} en un AFD, representado en la Figura \ref{fig:ej:1.3.20:AFD}.
        \begin{figure}
            \centering
            \begin{tikzpicture}
                \node[state, initial] (q0) {$q_0$};
                \node[state, right of=q0] (q0q1) {$q_0q_1$};
                \node[state, above right of=q0q1, xshift=3em] (q0q1q2) {$q_0q_1q_2$};
                \node[state, below right of=q0q1, xshift=3em] (q0q2) {$q_0q_2$};
                \node[state, above right of=q0q1q2, accepting, xshift=3em] (q0q1q2q3) {$q_0q_1q_2q_3$};
                \node[state, below right of=q0q1q2, accepting, xshift=3em, yshift=3em] (q0q2q3) {$q_0q_2q_3$};
                \node[state, above right of=q0q2, accepting, xshift=3em, yshift=-3em] (q0q1q3) {$q_0q_1q_3$};
                \node[state, below right of=q0q2, accepting, xshift=3em] (q0q3) {$q_0q_3$};

                \draw   (q0) edge[loop above] node{0} (q0)
                        (q0) edge[above] node{1} (q0q1)
                        (q0q1) edge[above] node{0} (q0q2)
                        (q0q1) edge[above] node{1} (q0q1q2)
                        (q0q2) edge[above] node{0} (q0q3)
                        (q0q2) edge[above] node{1} (q0q1q3)
                        (q0q1q2) edge[above] node{0} (q0q2q3)
                        (q0q1q2) edge[above] node{1} (q0q1q2q3)
                        (q0q1q2q3) edge[right, bend left] node{0} (q0q2q3)
                        (q0q1q2q3) edge[loop above] node{1} (q0q1q2q3)
                        (q0q2q3) edge[right, bend left=45] node{0} (q0q3)
                        (q0q2q3) edge[right, bend left] node[pos=0.8]{1} (q0q1q3)
                        (q0q1q3) edge[below, bend left] node[pos=0.1]{0} (q0q2)
                        (q0q1q3) edge[right] node{1} (q0q1q2)
                        (q0q3) edge[below, bend left] node{1} (q0q1)
                        (q0q3) edge[below, bend left] node{0} (q0);
            \end{tikzpicture}
            \caption{AFD que acepta el lenguaje $L_2$ del ejercicio~\ref{ej:1.3.20}.}
            \label{fig:ej:1.3.20:AFD}
        \end{figure}

        El AFD de la Figura \ref{fig:ej:1.3.20:AFD} acepta el lenguaje $L_2$, y es idéntico al que podríamos haber razonado de forma directa. Veamos qué representa cada estado:
        \begin{itemize}
            \item $q_0$: No estamos en un candidado a ser cadena final. Si leemos un $1$, empezaremos la que puede ser la cadena final.
            \item $q_0q_1$: Hemos leído un $1$, por lo que hemos empezado la posible cadena final. Llevamos $1$.
            \item $q_0q_1q_2$: Hemos leído un $1$ y un $1$. Llevamos $11$ de cadena final.
            \item $q_0q_2$: Hemos leído un $1$ y un $0$. Llevamos $10$ de cadena final.
            \item $q_0q_1q_2q_3$: Hemos leído un $1$, un $1$ y un $1$. Llevamos $111$ de cadena final.
            \item $q_0q_2q_3$: Hemos leído un $1$, un $1$ y un $0$. Llevamos $110$ de cadena final.
            \item $q_0q_1q_3$: Hemos leído un $1$, un $0$ y un $1$. Llevamos $101$ de cadena final.
            \item $q_0q_3$: Hemos leído un $1$, un $0$ y un $0$. Llevamos $100$ de cadena final.
        \end{itemize}
        Lo complejo de hacerlo de forma directa sería ver las transiciones desde los estados finales. Razonando cuál es la cadena final leída, podríamos haberlo hecho de forma directa, pero el AFND nos ha ayudado a hacerlo de forma algorítmica.
        La minimización del autómata de la Figura~\ref{fig:ej:1.3.20:AFD} se muestra en la Tabla~\ref{tab:ej:1.3.20:AFD-Minimal}, donde vemos que este era minimal.
        \begin{table}
            \centering
            \begin{tabular}{r c c c c c c c c}
                \hhline{~*{1}{-}}
                $q_0q_1$ & \cell{\times} \\ \hhline{~*{2}{-}}
                $q_0q_1q_2$ & \cell{\times} & \cell{\xcancel{(q_0q_3,q_0q_1q_3)}} \\ \hhline{~*{3}{-}}
                $q_0q_2$ & \cell{\xcancel{(q_0q_3,q_0q_1q_3)}} & \cell{\times} & \cell{\times} \\ \hhline{~*{4}{-}}
                $q_0q_1q_2q_3$ & \cell{\times} & \cell{\times} & \cell{\times} & \cell{\times} \\ \hhline{~*{5}{-}}
                $q_0q_2q_3$ & \cell{\times} & \cell{\times} & \cell{\times} & \cell{\times} & \cell{\times} \\ \hhline{~*{6}{-}}
                $q_0q_1q_3$ & \cell{\times} & \cell{\times} & \cell{\times} & \cell{\times} & \cell{\times} & \cell{\times} \\ \hhline{~*{7}{-}}
                $q_0q_3$ & \cell{\times} & \cell{\times} & \cell{\times} & \cell{\times} & \cell{\times} & \cell{\times} & \cell{\times} \\ \hhline{~*{7}{-}}
                & $q_0$ & $q_0q_1$ & $q_0q_1q_2$ & $q_0q_2$ & $q_0q_1q_2q_3$ & $q_0q_2q_3$ & $q_0q_1q_3$
            \end{tabular}
            \caption{Tabla de minimización del autómata de la Figura~\ref{fig:ej:1.3.20:AFD}.}
            \label{tab:ej:1.3.20:AFD-Minimal}
        \end{table}

        Como vemos, obtener el autómata producto será complejo, puesto que tendrá $8\cdot 5=40$ estados. Por tanto, optamos por la segunda opción.

        \item[Opción 2:]  Expresión Regular.
        
        La expresión regular del lenguaje intersección ha de ser la siguiente:
        \begin{equation*}
            (0+1)(0+1) \ \red{0}\  {(0+1)}^{\ast} \ \red{1}\ (0+1)(0+1)
        \end{equation*}
        \begin{observacion}
            Notemos que esta expresión regular no contempla las palabras de longitud menor o igual a $5$. Estudiemos estas:
            \begin{itemize}
                \item Si $|z|=5$, es necesario que el tercer símbolo sea un $0$ y el antepenúltimo ($5-2=3$) un $1$. Por tanto, como el tercer símbolo no puede tomar ambos valores a la vez, no hay palabras de longitud $5$.
                \item Si $|z|=4$, es necesario que el tercer símbolo sea un $0$ y el antepenúltimo ($4-2=2$) un $1$. Por tanto, estas palabras son de la forma $(0+1) \ \red{10} \ (0+1)$.
                \item Si $|z|=3$, es necesario que el tercer símbolo sea un $0$ y el antepenúltimo ($3-2=1$) un $1$. Por tanto, estas palabras son de la forma $\red{1} \ (0+1) \ \red{0}$.
                \item Si $|z|<3$, no podemos considerar el tercer símbolo, por lo que no hay palabras de longitud menor que $3$.
            \end{itemize}

            Por tanto, la expresión regular correcta sería:
            \begin{equation*}
                (0+1)(0+1) \ \red{0}\  {(0+1)}^{\ast} \ \red{1}\ (0+1)(0+1)
                + (0+1) \ \red{10} \ (0+1)
                + \red{1} \ (0+1) \ \red{0}
            \end{equation*}
            No obstante, obtener el autómata finito minimal asociado a esta expresión regular sería muy complejo, por lo que no consideramos dichas palabras.
        \end{observacion}

        El AFND que acepta el lenguaje intersección se muestra en la Figura~\ref{fig:ej:1.3.20:AFND-Interseccion}.
        \begin{figure}
            \centering
            \begin{tikzpicture}[node distance=2.5cm]
                \node[state, initial] (q0) {$q_0$};
                \node[state, right of=q0] (q1) {$q_1$};
                \node[state, right of=q1] (q2) {$q_2$};
                \node[state, right of=q2] (q3) {$q_3$};
                \node[state, right of=q3] (q4) {$q_4$};
                \node[state, right of=q4] (q5) {$q_5$};
                \node[state, accepting, right of=q5] (q6) {$q_6$};

                \draw   (q0) edge[above] node{$0,1$} (q1)
                        (q1) edge[above] node{$0,1$} (q2)
                        (q2) edge[above] node{$0$} (q3)
                        (q3) edge[loop above] node{$0,1$} (q3)
                        (q3) edge[above] node{$1$} (q4)
                        (q4) edge[above] node{$0,1$} (q5)
                        (q5) edge[above] node{$0,1$} (q6);
            \end{tikzpicture}
            \caption{AFND que acepta la intersección de los lenguajes del ejercicio~\ref{ej:1.3.20}.}
            \label{fig:ej:1.3.20:AFND-Interseccion}
        \end{figure}

        Convertimos ahora el AFND de la Figura \ref{fig:ej:1.3.20:AFND-Interseccion} en un AFD, representado en la Figura \ref{fig:ej:1.3.20:AFD-Interseccion}.
        \begin{figure}
            \centering
            \resizebox{1.15\textwidth}{!}{
            \begin{tikzpicture}[node distance=2.7cm]
                \node[state, initial] (q0) {$q_0$};
                \node[state, right of=q0] (q1) {$q_1$};
                \node[state, right of=q1] (q2) {$q_2$};
                \node[state, right of=q2] (q3) {$q_3$};
                \node[state, right of=q3] (q3q4) {$q_3q_4$};
                \node[state, above right of=q3q4, xshift=3em] (q3q4q5) {$q_3q_4q_5$};
                \node[state, below right of=q3q4, xshift=3em] (q3q5) {$q_3q_5$};
                \node[state, above right of=q3q4q5, accepting, xshift=3em] (q3q4q5q6) {$q_3q_4q_5q_6$};
                \node[state, below right of=q3q4q5, accepting, xshift=3em, yshift=3em] (q3q5q6) {$q_3q_5q_6$};
                \node[state, above right of=q3q5, accepting, xshift=3em, yshift=-3em] (q3q4q6) {$q_3q_4q_6$};
                \node[state, below right of=q3q5, accepting, xshift=3em] (q3q6) {$q_3q_6$};
                \node[state, error, below of=q2] (E) {$E$};

                \draw   (q0) edge[above] node{$0,1$} (q1)
                        (q1) edge[above] node{$0,1$} (q2)
                        (q2) edge[above] node{$0$} (q3)
                        (q2) edge[right] node{$1$} (E)
                
                        (q3) edge[loop above] node{0} (q3)
                        (q3) edge[above] node{1} (q3q4)
                        (q3q4) edge[above] node{0} (q3q5)
                        (q3q4) edge[above] node{1} (q3q4q5)
                        (q3q5) edge[above] node{0} (q3q6)
                        (q3q5) edge[above] node{1} (q3q4q6)
                        (q3q4q5) edge[above] node{0} (q3q5q6)
                        (q3q4q5) edge[above] node{1} (q3q4q5q6)
                        (q3q4q5q6) edge[right, bend left] node{0} (q3q5q6)
                        (q3q4q5q6) edge[loop above] node{1} (q3q4q5q6)
                        (q3q5q6) edge[right, bend left=45] node{0} (q3q6)
                        (q3q5q6) edge[right, bend left] node[pos=0.8]{1} (q3q4q6)
                        (q3q4q6) edge[below, bend left] node[pos=0.1]{0} (q3q5)
                        (q3q4q6) edge[right] node{1} (q3q4q5)
                        (q3q6) edge[below, bend left] node{1} (q3q4)
                        (q3q6) edge[below, bend left] node{0} (q3);
            \end{tikzpicture}
            }
            \caption{AFD que acepta la intersección de los lenguajes del ejercicio~\ref{ej:1.3.20}.}
            \label{fig:ej:1.3.20:AFD-Interseccion}
        \end{figure}
    \end{description}
\end{ejercicio}

\begin{ejercicio}\label{ej:1.3.21}
    Construir un autómata finito minimal para los siguientes lenguajes sobre el alfabeto $\{0,1\}$:
    \begin{enumerate}
        \item \label{ej:1.3.21-1}
        Palabras que contienen como subcadena una palabra del conjunto ${\{00,11\}}^{2}$.
        
        Han de contener una subcadena del conjunto $\{0000,0011,1100,1111\}$. Para ello, construimos el autómata de la Figura~\ref{fig:ej:1.3.21-1:AFD}.
        \begin{figure}
            \centering
            \begin{tikzpicture}
                \node[state, initial] (q0) {$q_0$};
                \node[state, above right of=q0] (q1) {$q_{1}$};
                \node[state, below right of=q0] (q2) {$q_{2}$};
                \node[state, right of=q1] (q3) {$q_{3}$};
                \node[state, right of=q2] (q4) {$q_{4}$};
                \node[state, above right of=q3, yshift=-2em] (q5) {$q_{5}$};
                \node[state, below right of=q3, yshift=2em] (q6) {$q_{6}$};
                \node[state, above right of=q4, yshift=-2em] (q7) {$q_{7}$};
                \node[state, below right of=q4, yshift=2em] (q8) {$q_{8}$};
                \node[state, accepting, right of=q5] (q9) {$q_{9}$};
                \node[state, accepting, right of=q6] (q10) {$q_{10}$};
                \node[state, accepting, right of=q7] (q11) {$q_{11}$};
                \node[state, accepting, right of=q8] (q12) {$q_{12}$};

                \draw   (q0) edge[above] node{0} (q1)
                        (q1) edge[above] node{0} (q3)
                        (q3) edge[above] node{0} (q5)
                        (q5) edge[above] node{0} (q9)
                        (q3) edge[above] node{1} (q6)
                        (q6) edge[above] node{1} (q10)
                        (q0) edge[above] node{1} (q2)
                        (q2) edge[above] node{1} (q4)
                        (q4) edge[above] node{1} (q8)
                        (q8) edge[above] node{1} (q12)
                        (q4) edge[above] node{0} (q7)
                        (q7) edge[above] node{0} (q11);
                    
                \draw   (q1) edge[left, bend left] node{1} (q2)
                        (q2) edge[right, bend left] node{0} (q1)
                        (q5) edge[right] node{1} (q6)
                        (q6) edge[above, bend left] node{0} (q1)
                        (q8) edge[right] node{0} (q7)
                        (q7) edge[below, bend right] node{1} (q2);

                \draw   (q9) edge[loop right] node{0,1} (q9)
                        (q10) edge[loop right] node{0,1} (q10)
                        (q11) edge[loop right] node{0,1} (q11)
                        (q12) edge[loop right] node{0,1} (q12);
            \end{tikzpicture}
            \caption{AFD que acepta el lenguaje del ejercicio~\ref{ej:1.3.21}.\ref{ej:1.3.21-1}.}
            \label{fig:ej:1.3.21-1:AFD}
        \end{figure}

        La minimización del autómata de la Figura~\ref{fig:ej:1.3.21-1:AFD} se muestra en la Tabla~\ref{tab:ej:1.3.21-1:AFD-Minimal}.
        \begin{table}
            \centering
            \begin{tabular}{r cccccccccccccc}
                \hhline{~*{1}{-}}
                $q_1$ & \cell{\times} \\ \hhline{~*{2}{-}}
                $q_2$ & \cell{\times} & \cell{\times} \\ \hhline{~*{3}{-}}
                $q_3$ & \cell{\times} & \cell{\times} & \cell{\times} \\ \hhline{~*{4}{-}}
                $q_4$ & \cell{\times} & \cell{\times} & \cell{\times} & \cell{\times} \\ \hhline{~*{5}{-}}
                $q_5$ & \cell{\times} & \cell{\times} & \cell{\times} & \cell{\times} & \cell{\times} \\ \hhline{~*{6}{-}}
                $q_6$ & \cell{\times} & \cell{\times} & \cell{\times} & \cell{\times} & \cell{\times} & \cell{\times} \\ \hhline{~*{7}{-}}
                $q_7$ & \cell{\times} & \cell{\times} & \cell{\times} & \cell{\times} & \cell{\times} & \cell{\times} & \cell{\times} \\ \hhline{~*{8}{-}}
                $q_8$ & \cell{\times} & \cell{\times} & \cell{\times} & \cell{\times} & \cell{\times} & \cell{\times} & \cell{\times} & \cell{\times} \\ \hhline{~*{9}{-}}
                $q_9$ & \cell{\times} & \cell{\times} & \cell{\times} & \cell{\times} & \cell{\times} & \cell{\times} & \cell{\times} & \cell{\times}& \cell{\times} \\ \hhline{~*{10}{-}}
                $q_{10}$ & \cell{\times} & \cell{\times} & \cell{\times} & \cell{\times} & \cell{\times} & \cell{\times} & \cell{\times} & \cell{\times}& \cell{\times} & \cell{} \\ \hhline{~*{11}{-}}
                $q_{11}$ & \cell{\times} & \cell{\times} & \cell{\times} & \cell{\times} & \cell{\times} & \cell{\times} & \cell{\times} & \cell{\times}& \cell{\times} & \cell{} & \cell{} \\ \hhline{~*{12}{-}}
                $q_{12}$ & \cell{\times} & \cell{\times} & \cell{\times} & \cell{\times} & \cell{\times} & \cell{\times} & \cell{\times} & \cell{\times}& \cell{\times} & \cell{} & \cell{} & \cell{} \\ \hhline{~*{12}{-}}
                & $q_0$ & $q_1$ & $q_2$ & $q_3$ & $q_4$ & $q_5$ & $q_6$ & $q_7$ & $q_8$ & $q_9$ & $q_{10}$ & $q_{11}$
            \end{tabular}
            \caption{Tabla de minimización del autómata de la Figura~\ref{fig:ej:1.3.21-1:AFD}.}
            \label{tab:ej:1.3.21-1:AFD-Minimal}
        \end{table}

        Por tanto, tenemos que todos los estados finales eran indistinguibles. Notándolos por $q_F$, el autómata minimal es el de la Figura~\ref{fig:ej:1.3.21-1:AFD-Minimal}.
        \begin{figure}
            \centering
            \begin{tikzpicture}
                \node[state, initial] (q0) {$q_0$};
                \node[state, above right of=q0] (q1) {$q_{1}$};
                \node[state, below right of=q0] (q2) {$q_{2}$};
                \node[state, right of=q1] (q3) {$q_{3}$};
                \node[state, right of=q2] (q4) {$q_{4}$};
                \node[state, above right of=q3, yshift=-2em] (q5) {$q_{5}$};
                \node[state, below right of=q3, yshift=2em] (q6) {$q_{6}$};
                \node[state, above right of=q4, yshift=-2em] (q7) {$q_{7}$};
                \node[state, below right of=q4, yshift=2em] (q8) {$q_{8}$};
                \node[state, accepting, right of=q7, yshift=2em] (qF) {$q_{F}$};

                \draw   (q0) edge[above] node{0} (q1)
                        (q1) edge[above] node{0} (q3)
                        (q3) edge[above] node{0} (q5)
                        (q5) edge[above, bend left] node{0} (qF)
                        (q3) edge[above] node{1} (q6)
                        (q6) edge[above] node{1} (qF)
                        (q0) edge[above] node{1} (q2)
                        (q2) edge[above] node{1} (q4)
                        (q4) edge[above] node{1} (q8)
                        (q8) edge[above, bend right] node{1} (qF)
                        (q4) edge[above] node{0} (q7)
                        (q7) edge[above] node{0} (qF);
                    
                \draw   (q1) edge[left, bend left] node{1} (q2)
                        (q2) edge[right, bend left] node{0} (q1)
                        (q5) edge[right] node{1} (q6)
                        (q6) edge[above, bend left] node{0} (q1)
                        (q8) edge[right] node{0} (q7)
                        (q7) edge[below, bend right] node{1} (q2);

                \draw   (qF) edge[loop right] node{0,1} (qF);
            \end{tikzpicture}
            \caption{AFD minimal que acepta el lenguaje del ejercicio~\ref{ej:1.3.21}.\ref{ej:1.3.21-1}.}
            \label{fig:ej:1.3.21-1:AFD-Minimal}
        \end{figure}
        \item \label{ej:1.3.21-2}
        Palabras que contienen como subcadena una palabra del conjunto $\{0011,1100\}$.

        El AFD minimal que acepta este lenguaje es el de la Figura~\ref{fig:ej:1.3.21-2:AFD}.
        \begin{figure}
            \centering
            \begin{tikzpicture}
                \node[state, initial] (q0) {$q_0$};
                \node[state, above right of=q0] (q1) {$q_{1}$};
                \node[state, below right of=q0] (q2) {$q_{2}$};
                \node[state, right of=q1] (q3) {$q_{3}$};
                \node[state, right of=q2] (q4) {$q_{4}$};
                \node[state, right of=q3] (q6) {$q_{6}$};
                \node[state, right of=q4] (q7) {$q_{7}$};
                \node[state, accepting, above right of=q7] (qF) {$q_{F}$};

                \draw   (q0) edge[above] node{0} (q1)
                        (q1) edge[above] node{0} (q3)
                        (q3) edge[loop above] node{0} (q3)
                        (q3) edge[above] node{1} (q6)
                        (q6) edge[above] node{1} (qF)
                        (q0) edge[above] node{1} (q2)
                        (q2) edge[above] node{1} (q4)
                        (q4) edge[loop below] node{1} (q4)
                        (q4) edge[above] node{0} (q7)
                        (q7) edge[above] node{0} (qF);
                    
                \draw   (q1) edge[left, bend left] node{1} (q2)
                        (q2) edge[right, bend left] node{0} (q1)
                        (q6) edge[below, bend left] node{0} (q1)
                        (q7) edge[above, bend right] node{1} (q2);

                \draw   (qF) edge[loop right] node{0,1} (qF);
            \end{tikzpicture}
            \caption{AFD minimal que acepta el lenguaje del ejercicio~\ref{ej:1.3.21}.\ref{ej:1.3.21-2}.}
            \label{fig:ej:1.3.21-2:AFD}
        \end{figure}
    \end{enumerate}
\end{ejercicio}

\begin{ejercicio}\label{ej:1.3.22}
    Responda a los siguientes apartados:
    \begin{enumerate}
        \item Construye una gramática regular que genere el siguiente lenguaje:
            \begin{equation*}
                L_1 = \{u\in {\{0,1\}}^{\ast} \mid \text{ el número de 1's y el número de 0's en } u \text{ es par }\}
            \end{equation*}

            Este es muy similar al Ejercicio~\ref{ej:1.2.15}. Tenemos los siguientes estados:
            \begin{itemize}
                \item \ul{$S$}: La cadena leída es correcta, ya que el número de $0$'s y de $1$'s es par.
                \item \ul{$E_0$}: Tenemos un error en $0$, ya que el número de $0$'s es impar. El número de $1$'s es par.
                \item \ul{$E_1$}: Tenemos un error en $1$, ya que el número de $1$'s es impar. El número de $0$'s es par.
                \item \ul{$E_{01}$}: Tenemos un error en $0$ y en $1$, ya que el número de $0$'s y de $1$'s es impar.
            \end{itemize}
        
            La gramática obtenida es $G=(\{E_{01},E_0,E_1,S\}, \{0, 1\}, P, S)$, donde $P$ es:
            \begin{align*}
                S &\to 0E_0 \mid 1E_1 \mid \veps, \\
                E_0 &\to 0S \mid 1E_{01}, \\
                E_1 &\to 0E_{01} \mid 1S, \\
                E_{01} &\to 0E_1 \mid 1E_0
            \end{align*}

            El autómata finito determinista minimal asociado a la gramática obtenida es el de la Figura~\ref{fig:1.3.22-L1}.
            \begin{figure}
                \centering
                \begin{tikzpicture}
                    \node[state, initial, accepting] (S) {$S$};
                    \node[state, right of=S] (E0) {$E_0$};
                    \node[state, below of=S] (E1) {$E_1$};
                    \node[state, below of=E0] (E01) {$E_{01}$};

                    \draw   (S) edge[above] node{0} (E0)
                            (S) edge[left] node{1} (E1)
                            (E0) edge[above, bend right] node{0} (S)
                            (E0) edge[right] node{1} (E01)
                            (E1) edge[above] node{0} (E01)
                            (E1) edge[left, bend left] node{1} (S)
                            (E01) edge[below, bend left] node{0} (E1)
                            (E01) edge[right, bend right] node{1} (E0);
                \end{tikzpicture}
                \caption{AFD minimal asociado a $L_1$ del Ejercicio~\ref{ej:1.3.22}.}
                \label{fig:1.3.22-L1}
            \end{figure}
        \item Construye un autómata que reconozca el siguiente lenguaje:
            \begin{equation*}
                L_2 = \{0^n 1^m \mid n\geq 1, m\geq 0, n \text{ múltiplo de 3, } m \text{ par }\}
            \end{equation*}

            Sean los siguientes estados:
            \begin{itemize}
                \item \ul{$q_0$}: $n,m=0$.
                \item \ul{$q_1$}: $n \mod 3=1$, $m=0$.
                \item \ul{$q_2$}: $n \mod 3=2$, $m=0$.
                \item \ul{$q_3$}: $n \mod 3=0$, $n>1$, $m=0$.
                \item \ul{$q_4$}: $n \mod 3=0$, $m\mod 2=1$.
                \item \ul{$q_5$}: $n \mod 3=0$, $m\mod 2=0$.
            \end{itemize}

            El autómata finito determinista asociado a $L_2$ es el de la Figura~\ref{fig:1.3.22-L2}.
            \begin{figure}
                \centering
                \begin{tikzpicture}[node distance=2.5cm]
                    \node[state, initial] (q0) {$q_0$};
                    \node[state, right of=q0] (q1) {$q_1$};
                    \node[state, right of=q1] (q2) {$q_2$};
                    \node[state, error, below right of=q2, yshift=-2em] (E) {$E$};
                    \node[state, above right of=E, accepting, yshift=2em] (q3) {$q_3$};
                    \node[state, right of=q3] (q4) {$q_4$};
                    \node[state, right of=q4, accepting] (q5) {$q_5$};

                    \draw   (q0) edge[above] node{0} (q1)
                            (q1) edge[above] node{0} (q2)
                            (q2) edge[above] node{0} (q3)
                            (q3) edge[above, bend right] node{0} (q1)
                            (q3) edge[above] node{1} (q4)
                            (q4) edge[above] node{1} (q5)
                            (q5) edge[above, bend right] node{1} (q4);
                    \draw[opacity=0.6]
                            (q0) edge[above] node{1} (E)
                            (q1) edge[above] node{1} (E)
                            (q2) edge[above] node{1} (E)
                            (q4) edge[above] node{0} (E)
                            (q5) edge[above] node{0} (E)
                            (E) edge[loop right] node{0,1} (E);
                \end{tikzpicture}
                \caption{AFD minimal asociado a $L_2$ del Ejercicio~\ref{ej:1.3.22}.}
                \label{fig:1.3.22-L2}
            \end{figure}


        \item Diseña el AFD mínimo que reconoce el lenguaje $(L_1 \cup L_2)$.
        
        
        Resolvemos mediante autómata producto, tal y como se muestra en la Figura~\ref{fig:1.3.22-L1-L2}.
        \begin{figure}
            \centering
            \resizebox{1.1\textwidth}{!}{
            \begin{tikzpicture}[node distance=2.5cm]
                \node[state, initial, accepting] (Sq0) {$Sq_0$};
                \node[state, right of=Sq0] (E0q1) {$E_0q_1$};
                \node[state, right of=E0q1, accepting] (Sq2) {$Sq_2$};
                \node[state, right of=Sq2, accepting] (E0q3) {$E_0q_3$};
                \node[state, above right of=E0q3, accepting] (Sq1) {$Sq_1$};
                \node[state, below right of=E0q3] (E01q4) {$E_{01}q_4$};
                \node[state, right of=E01q4, accepting] (E0q5) {$E_0q_5$};
                \node[state, right of=Sq1] (E0q2) {$E_0q_2$};
                \node[state, right of=E0q2, accepting] (Sq3) {$Sq_3$};
                \node[state, right of=Sq3] (E1q4) {$E_1q_4$};
                \node[state, right of=E1q4, accepting] (Sq5) {$Sq_5$};

                \node[state, above of=Sq1] (E1E) {$E_1E$};
                \node[state, right of=E1E] (E01E) {$E_{01}E$};
                \node[state, above of=E01E] (E0E) {$E_0E$};
                \node[state, left of=E0E, accepting] (SE) {$SE$};

                \draw   (Sq0) edge[above] node{0} (E0q1)
                        (Sq0) edge[above] node{1} (E1E)
                        (E0q1) edge[above] node[pos=0.8]{0} (Sq2)
                        (Sq2) edge[above] node{0} (E0q3)
                        (E0q3) edge[above] node{0} (Sq1)
                        (E0q3) edge[left] node{1} (E01q4)
                        (Sq1) edge[above] node{0} (E0q2)
                        (Sq1) edge[left, bend right] node[pos=0.2]{1} (E1E)
                        (E01q4) edge[above] node{1} (E0q5)
                        (E0q5) edge[above, bend right] node{1} (E01q4)
                        (E0q2) edge[above] node{0} (Sq3)
                        (E0q2) edge[left] node{1} (E01E)
                        (Sq3) edge[above] node{1} (E1q4)
                        (Sq3) edge[above, bend left=70, looseness=1.4] node{0} (E0q1)
                        (E1q4) edge[above] node{1} (Sq5)
                        (E1q4) edge[above] node{0} (E01E)
                        (Sq5) edge[below, bend left] node{1} (E1q4)
                        (Sq5) edge[above] node{0} (E0E);

                \draw   (E1E) edge[above] node{0} (E01E)
                        (E1E) edge[left] node{1} (SE)
                        (E01E) edge[left] node{1} (E0E)
                        (E01E) edge[above, bend right] node{0} (E1E)
                        (E0E) edge[left, bend right] node{1} (E01E)
                        (E0E) edge[above] node{0} (SE)
                        (SE) edge[above, bend right] node{0} (E0E)
                        (SE) edge[left, bend right] node{1} (E1E);

                \draw[red]
                    (E0q1) edge[above] node[pos=0.2]{1} (E01E);
                
                \draw[blue]
                    (Sq2) edge[left] node[pos=0.2]{1} (E1E)
                    (E01q4) edge[left] node[pos=0.2]{0} (E1E)
                    (E0q5) edge[left] node[pos=0.2]{0} (SE);
            \end{tikzpicture}}
            \caption{AFD asociado a $(L_1 \cup L_2)$ del Ejercicio~\ref{ej:1.3.22}.}
            \label{fig:1.3.22-L1-L2}
        \end{figure}

        La minimización del autómata de la Figura~\ref{fig:1.3.22-L1-L2} se muestra en la Tabla~\ref{tab:1.3.22-L1-L2}, que informa de que el autómata de la Figura~\ref{fig:1.3.22-L1-L2} ya era minimal.
        \begin{table}
            \centering
            \resizebox{1.1\textwidth}{!}{
                \begin{tabular}{r ccccccccccccccccc}
                    \hhline{~*{1}{-}}
                    $E_0q_1$ & \cell{\times} \\ \hhline{~*{2}{-}}
                    $Sq_2$ & \cell{\times} & \cell{\times} \\ \hhline{~*{3}{-}}
                    $E_0q_3$ & \cell{\times} & \cell{\times} & \cell{\times} \\ \hhline{~*{4}{-}}
                    $Sq_1$ & \cell{\times} & \cell{\times} & \cell{\times} & \cell{\times} \\ \hhline{~*{5}{-}}
                    $E_{01}q_4$ & \cell{\times} & \cell{\times} & \cell{\times} & \cell{\times} & \cell{\times} \\ \hhline{~*{6}{-}}
                    $E_0q_5$ & \cell{\times} & \cell{\times} & \cell{\times} & \cell{\times} & \cell{\times} & \cell{\times} \\ \hhline{~*{7}{-}}
                    $E_0q_2$ & \cell{\times} & \cell{\times} & \cell{\times} & \cell{\times} & \cell{\times} & \cell{\times} & \cell{\times} \\ \hhline{~*{8}{-}}
                    $Sq_3$ & \cell{\times} & \cell{\times} & \cell{\times} & \cell{\times} & \cell{\times} & \cell{\times} & \cell{\times} & \cell{\times} \\ \hhline{~*{9}{-}}
                    $E_1q_4$ & \cell{\times} & \cell{\times} & \cell{\times} & \cell{\times} & \cell{\times} & \cell{\times} & \cell{\times} & \cell{\times} & \cell{\times} \\ \hhline{~*{10}{-}}
                    $Sq_5$ & \cell{\times} & \cell{\times} & \cell{\times} & \cell{\times} & \cell{\times} & \cell{\times} & \cell{\times} & \cell{\times} & \cell{\times} & \cell{\times} \\ \hhline{~*{11}{-}}
                    $E_1E$ & \cell{\times} & \cell{\times} & \cell{\times} & \cell{\times} & \cell{\times} & \cell{\times} & \cell{\times} & \cell{\times} & \cell{\times} & \cell{\xcancel{\begin{array}{c}(SE,Sq_3)\\(SE,Sq_5)\end{array}}} & \cell{\times} \\ \hhline{~*{12}{-}}
                    $E_{01}E$ & \cell{\times} & \cell{\times} & \cell{\times} & \cell{\times} & \cell{\times} & \cell{\times} & \cell{\times} & \cell{\times} & \cell{\times} & \cell{\times} & \cell{\times} & \cell{\times} \\ \hhline{~*{13}{-}}
                    $E_0E$ & \cell{\times} & \cell{\xcancel{\begin{array}{c}(SE,Sq_0)\\(SE,Sq_3)\end{array}}} & \cell{\times} & \cell{\times} & \cell{\times} & \cell{\times} & \cell{\times} & \cell{\xcancel{(SE,Sq_1)}} & \cell{\times} & \cell{\times} & \cell{\times} & \cell{\times} & \cell{\times} \\ \hhline{~*{14}{-}}
                    $SE$ & \cell{\times} & \cell{\times} & \cell{\times} & \cell{\times} & \cell{} & \cell{\times} & \cell{\times} & \cell{\times} & \cell{\times} & \cell{\times} & \cell{\times} & \cell{\times} & \cell{\times} & \cell{\times} \\ \hhline{~*{14}{-}}
                    & $Sq_0$ & $E_0q_1$ & $Sq_2$ & $E_0q_3$ & $Sq_1$ & $E_{01}q_4$ & $E_0q_5$ & $E_0q_2$ & $Sq_3$ & $E_1q_4$ & $Sq_5$ & $E_1E$ & $E_{01}E$ & $E_0E$
                \end{tabular}
            }
            \caption{Tabla de minimización del autómata de la Figura~\ref{fig:1.3.22-L1-L2}.}
            \label{tab:1.3.22-L1-L2}
        \end{table}
    \end{enumerate}
\end{ejercicio}

\begin{ejercicio}\label{ej:1.3.23}
    Sobre el alfabeto $\{0,1\}$:
    \begin{enumerate}
        \item Construye una gramática regular que genere el lenguaje $L_1$ de las palabras $u$ tales que:
            \begin{itemize}
                \item Si $|u|< 5$ entonces el número de 1's es impar.
                \item Si $|u|\geq 5$ entonces el número de 1's es par.
                \item $u$ tiene al menos un símbolo 1.
            \end{itemize}
            
            
            Sean los siguientes estados:
            \begin{itemize}
                \item $q_0$: Estado inicial.
                \item $q_{iNV}$ para $i\in \{1,\dots,4\}$: Si $u$ es la cadena leída, tenemos que $|u|=i$ pero $n_1(u)=0$, por lo que no es válido.
                \item $q_{jI}$ para $j\in \{1,\dots,4\}$: Si $u$ es la cadena leída, tenemos que $|u|=j$ y $n_1(u)$ es impar. Son estados finales.
                \item $q_I$: Si $u$ es la cadena leída, tenemos que $|u|\geq 5$ y $n_1(u)$ es impar.
                \item $q_{jP}$ para $j\in \{1,\dots,4\}$: Si $u$ es la cadena leída, tenemos que $|u|=j$ y $n_1(u)$ es par. Además, $n_1(u)>0$.
                \item $q_P$: Si $u$ es la cadena leída, tenemos que $|u|\geq 5$ y $n_1(u)$ es par. Además, $n_1(u)>0$. Es un estado final.
            \end{itemize}

            El AFD asociado a $L_1$ es el de la Figura~\ref{fig:1.3.23-L1}.
            \begin{figure}
                \centering
                \begin{tikzpicture}[node distance=2.5cm]
                    \node[state, initial] (q0) {$q_0$};
                    \node[state, above right of=q0] (q1NV) {$q_{1NV}$};
                    \node[state, below right of=q0, accepting] (q1I) {$q_{1I}$};
                    \node[state, right of=q1NV] (q2NV) {$q_{2NV}$};
                    \node[state, right of=q1I, accepting] (q2I) {$q_{2I}$};
                    \node[state, right of=q2NV] (q3NV) {$q_{3NV}$};
                    \node[state, right of=q2I, accepting] (q3I) {$q_{3I}$};
                    \node[state, right of=q3NV] (q4NV) {$q_{4NV}$};
                    \node[state, right of=q3I, accepting] (q4I) {$q_{4I}$};
                    \node[state, right of=q4I] (qI) {$q_{I}$};
                    \node[state, below of=q2I] (q2P) {$q_{2P}$};
                    \node[state, below of=q3I] (q3P) {$q_{3P}$};
                    \node[state, below of=q4I] (q4P) {$q_{4P}$};
                    \node[state, below of=qI, accepting] (qP) {$q_{P}$};

                    \draw   (q0) edge[above] node{0} (q1NV)
                            (q0) edge[below] node{1} (q1I)
                            (q1NV) edge[above] node{0} (q2NV)
                            (q1NV) edge[above right] node{1} (q2I)
                            (q2NV) edge[above] node{0} (q3NV)
                            (q2NV) edge[above right] node{1} (q3I)
                            (q3NV) edge[above] node{0} (q4NV)
                            (q3NV) edge[above right] node{1} (q4I)
                            (q4NV) edge[loop above] node{0} (q4NV)
                            (q4NV) edge[above right] node{1} (qI)
                            (qI) edge[loop above] node{0} (qI)
                            (qI) edge[left, bend right] node{1} (qP)
                            (qP) edge[loop below] node{0} (qP)
                            (qP) edge[right, bend right] node{1} (qI);

                    \draw[blue]
                            (q1I) edge[above] node{0} (q2I)
                            (q1I) edge[below] node[pos=0.2]{1} (q2P)
                            (q2I) edge[above] node{0} (q3I)
                            (q2I) edge[below] node[pos=0.2]{1} (q3P)
                            (q3I) edge[above] node{0} (q4I)
                            (q3I) edge[below] node[pos=0.2]{1} (q4P)
                            (q4I) edge[above] node{0} (qI)
                            (q4I) edge[below] node[pos=0.2]{1} (qP);
                    
                    \draw[red] 
                            (q2P) edge[above] node{0} (q3P)
                            (q2P) edge[above] node[pos=0.2]{1} (q3I)
                            (q3P) edge[above] node{0} (q4P)
                            (q3P) edge[above] node[pos=0.2]{1} (q4I)
                            (q4P) edge[above] node{1} (qI)
                            (q4P) edge[above] node[pos=0.2]{0} (qP);

                \end{tikzpicture}
                \caption{AFD asociado a $L_1$ del Ejercicio~\ref{ej:1.3.23}.}
                \label{fig:1.3.23-L1}
            \end{figure}

            Si el AFD de la Figura~\ref{fig:1.3.23-L1} es $M=(Q,\{0,1\},\delta,q_0,F)$, entonces la gramática pedida es $G=(Q,\{0,1\},P,q_0)$, donde $P$ es:
            \begin{align*}
                P=\left\{
                    \begin{aligned}
                        q_0 & \to 0q_{1NV} \mid 1q_{1I}, \\
                        q_{1NV} & \to 0q_{2NV} \mid 1q_{2I}, \\
                        q_{2NV} & \to 0q_{3NV} \mid 1q_{3I}, \\
                        q_{3NV} & \to 0q_{4NV} \mid 1q_{4I}, \\
                        q_{4NV} & \to 0q_{4NV} \mid 1q_{I}, \\
                        q_{I} & \to 0q_{I} \mid 1q_{P}, \\
                        q_{P} & \to 0q_{P} \mid 1q_{I}\mid \veps, \\
                        q_{1I} & \to 0q_{2I} \mid 1q_{2P}\mid \veps, \\
                        q_{2I} & \to 0q_{3I} \mid 1q_{3P}\mid \veps, \\
                        q_{3I} & \to 0q_{4I} \mid 1q_{4P}\mid \veps, \\
                        q_{4I} & \to 0q_{I} \mid 1q_{P}\mid \veps, \\
                        q_{2P} & \to 0q_{3P} \mid 1q_{3I}, \\
                        q_{3P} & \to 0q_{4P} \mid 1q_{4I}, \\
                        q_{4P} & \to 0q_{P} \mid 1q_{I}
                    \end{aligned}
                \right.
            \end{align*}

        \item \label{ej:1.3.23-2}
        Construye un autómata que reconozca el lenguaje $L_2$ dado por:
            \begin{equation*}
                L_2 = \{0^n 1^m \mid n\geq 0, m\geq 1, m \text{\ es múltiplo de\ } 6\}
            \end{equation*}

            El autómata finito determinista asociado a $L_2$ es el de la Figura~\ref{fig:1.3.23-L2}.
            \begin{figure}
                \centering
                \begin{tikzpicture}[node distance=2.5cm]
                    \node[state, initial] (q0) {$q_0$};
                    \node[state, right of=q0] (q1) {$q_1$};
                    \node[state, right of=q1] (q2) {$q_2$};
                    \node[state, right of=q2] (q3) {$q_3$};
                    \node[state, right of=q3] (q4) {$q_4$};
                    \node[state, right of=q4] (q5) {$q_5$};
                    \node[state, accepting, right of=q5] (q6) {$q_6$};
                    \node[state, below of=q3, error] (E) {$E$};

                    \draw   (q0) edge[above] node{1} (q1)
                            (q1) edge[above] node{1} (q2)
                            (q2) edge[above] node{1} (q3)
                            (q3) edge[above] node{1} (q4)
                            (q4) edge[above] node{1} (q5)
                            (q5) edge[above] node{1} (q6)
                            (q6) edge[above, bend right] node{1} (q1)
                            (q0) edge[loop above] node{0} (q0)
                            (q1) edge[above] node{0} (E)
                            (q2) edge[above] node{0} (E)
                            (q3) edge[above right] node{0} (E)
                            (q4) edge[above] node{0} (E)
                            (q5) edge[above] node{0} (E)
                            (q6) edge[above] node{0} (E)
                            (E) edge[loop below] node{0,1} (E);
                \end{tikzpicture}
                \caption{AFD minimal asociado a $L_2$ del Ejercicio~\ref{ej:1.3.23}.\ref{ej:1.3.23-2}.}
                \label{fig:1.3.23-L2}
            \end{figure}

        \item Diseña el AFD mínimo que reconozca el lenguaje $(L_1 \cup L_2)$.
        
        Razonando, llegamos a que $L_2\subset L_1$. Por tanto, $L_1\cup L_2=L_1$, por lo que el AFD minimal asociado a $L_1\cup L_2$ es el de la Figura~\ref{fig:1.3.23-L1}.
    \end{enumerate}
\end{ejercicio}

\begin{ejercicio}\label{ej:1.3.24}
    Encuentra para cada uno de los siguientes lenguajes una gramática de tipo 3 que lo genere o un autómata finito que lo reconozca:
    \begin{itemize}
        \item $L_1 = \{u\in {\{0,1\}}^{\ast} \mid u \text{\ no contiene la subcadena } \text{``}0101\text{''}\}$.
        \item $L_2 = \{0^i 1^j 0^k \mid i \geq 1, k\geq 0, i \text{\ impar\ }, k \text{\ múltiplo de\ } 3 \text{\ y\ } j\geq 2\}$ .
    \end{itemize}
    Diseña el AFD mínimo que reconoce el lenguaje $(L_1\cap L_2)$.\\

    El AFD minimal asociado a $L_1$ es el de la Figura~\ref{fig:1.3.24-L1}.
    \begin{figure}
        \centering
        \begin{tikzpicture}
            \node[state, initial, accepting] (q0) {$q_0$};
            \node[state, right of=q0, accepting] (q1) {$q_1$};
            \node[state, right of=q1, accepting] (q2) {$q_2$};
            \node[state, right of=q2, accepting] (q3) {$q_3$};
            \node[state, right of=q3] (q4) {$q_4$};

            \draw   (q0) edge[above] node{0} (q1)
                    (q0) edge[loop above] node{1} (q0)
                    (q1) edge[above] node{1} (q2)
                    (q1) edge[loop above] node{0} (q1)
                    (q2) edge[above] node{0} (q3)
                    (q2) edge[below, bend left] node{1} (q0)
                    (q3) edge[above] node{1} (q4)
                    (q3) edge[above, bend right] node{0} (q1)
                    (q4) edge[loop above] node{0,1} (q4);
        \end{tikzpicture}
        \caption{AFD minimal asociado a $L_1$ del Ejercicio~\ref{ej:1.3.24}.}
        \label{fig:1.3.24-L1}
    \end{figure}

    El AFD minimal asociado a $L_2$ es el de la Figura~\ref{fig:1.3.24-L2}.
    \begin{figure}
        \centering
        \begin{tikzpicture}[node distance=2.5cm]
            \node[state, initial] (p0) {$p_0$};
            \node[state, right of=p0] (p1) {$p_1$};
            \node[state, right of=p1] (p2) {$p_2$};
            \node[state, right of=p2, accepting] (p3) {$p_3$};
            \node[state, right of=p3] (p4) {$p_4$};
            \node[state, right of=p4] (p5) {$p_5$};
            \node[state, right of=p5, accepting] (p6) {$p_6$};
            \node[state, error, below of=p2] (E) {$E$};

            \draw   (p0) edge[above] node{0} (p1)
                    (p1) edge[above, bend right] node{0} (p0)
                    (p1) edge[above] node{1} (p2)
                    (p2) edge[above] node{1} (p3)
                    (p3) edge[loop above] node{1} (p3)
                    (p3) edge[above] node{0} (p4)
                    (p4) edge[above] node{0} (p5)
                    (p5) edge[above] node{0} (p6)
                    (p6) edge[above, bend right] node{0} (p4)
                    (p0) edge[above] node{1} (E)
                    (p2) edge[above right] node{0} (E)
                    (p4) edge[above] node{1} (E)
                    (p5) edge[above] node{1} (E)
                    (p6) edge[below] node{1} (E)
                    (E) edge[loop left] node{0,1} (E);
        \end{tikzpicture}
        \caption{AFD minimal asociado a $L_2$ del Ejercicio~\ref{ej:1.3.24}.}
        \label{fig:1.3.24-L2}
    \end{figure}

    El AFD asociado a $(L_1\cap L_2)$ es el de la Figura~\ref{fig:1.3.24-L1-L2}. Notemos que todos los estados de la forma $q_i,E$ se han agrupado en un único estado $E$, ya que todos ellos son indistingubles puesto que no se puede llegar desde ellos a ningún final.
    \begin{figure}
        \centering
        \begin{tikzpicture}[node distance=2.3cm]
            \node[state, initial] (q0p0) {$q_0p_0$};
            \node[state, right of=q0p0] (q1p1) {$q_1p_1$};
            \node[state, right of=q1p1] (q1p0) {$q_1p_0$};
            \node[state, below of=q1p1] (q2p2) {$q_2p_2$};
            \node[state, right of=q2p2, accepting] (q0p3) {$q_0p_3$};
            \node[state, right of=q0p3] (q1p4) {$q_1p_4$};
            \node[state, right of=q1p4] (q1p5) {$q_1p_5$};
            \node[state, right of=q1p5, accepting] (q1p6) {$q_1p_6$};
            \node[state, above of=q1p5, error] (E) {$E$};

            \draw   (q0p0) edge[above] node{0} (q1p1)
                    (q1p1) edge[below] node{0} (q1p0)
                    (q1p1) edge[left] node{1} (q2p2)
                    (q1p0) edge[above, bend right] node{0} (q1p1)
                    (q2p2) edge[below] node{1} (q0p3)
                    (q0p3) edge[above] node{0} (q1p4)
                    (q0p3) edge[loop below] node{1} (q0p3)
                    (q1p4) edge[above] node{0} (q1p5)
                    (q1p5) edge[above] node{0} (q1p6)
                    (q1p6) edge[below, bend left] node{0} (q1p4);

            \draw[opacity=0.6]
                    (q0p0) edge[above, bend left] node{1} (E)
                    (q1p0) edge[above] node{1} (E)
                    (q2p2) edge[above] node{0} (E)
                    (q1p4) edge[above] node{1} (E)
                    (q1p5) edge[right] node{1} (E)
                    (q1p6) edge[above] node{1} (E)
            (E) edge[loop above] node{0,1} (E);
        \end{tikzpicture}
        \caption{AFD asociado a $(L_1\cap L_2)$ del Ejercicio~\ref{ej:1.3.24}.}
        \label{fig:1.3.24-L1-L2}
    \end{figure}

    La minimización de este autómata se muestra en la Tabla~\ref{tab:1.3.24-L1-L2}.
    \begin{table}
        \centering
        \begin{tabular}{r cccccccccc}
            \hhline{~*{1}{-}} 
            $q_1p_1$ & \cell{\times} \\ \hhline{~*{2}{-}}
            $q_1p_0$ & \cell{} & \cell{\times} \\ \hhline{~*{3}{-}}
            $q_2p_2$ & \cell{\times} & \cell{\times} & \cell{\times} \\ \hhline{~*{4}{-}}
            $q_0p_3$ & \cell{\times} & \cell{\times} & \cell{\times} & \cell{\times} \\ \hhline{~*{5}{-}}
            $q_1p_4$ & \cell{\times} & \cell{\times} & \cell{\times} & \cell{\times} & \cell{\times} \\ \hhline{~*{6}{-}}
            $q_1p_5$ & \cell{\times} & \cell{\times} & \cell{\times} & \cell{\times} & \cell{\times} & \cell{\times} \\ \hhline{~*{7}{-}}
            $q_1p_6$ & \cell{\times} & \cell{\times} & \cell{\times} & \cell{\times} & \cell{} & \cell{\times} & \cell{\times} \\ \hhline{~*{8}{-}}
            $E$ & \cell{\times} & \cell{\times} & \cell{\times} & \cell{\times} & \cell{\times} & \cell{\times} & \cell{\times} & \cell{\times} \\ \hhline{~*{8}{-}}
            & $q_0p_0$ & $q_1p_1$ & $q_1p_0$ & $q_2p_2$ & $q_0p_3$ & $q_1p_4$ & $q_1p_5$ & $q_1p_6$
        \end{tabular}
        \caption{Tabla de minimización del autómata de la Figura~\ref{fig:1.3.24-L1-L2}.}
        \label{tab:1.3.24-L1-L2}
    \end{table}

    El AFD minimal asociado a $(L_1\cap L_2)$ es el de la Figura~\ref{fig:1.3.24-L1-L2-minimal}.
    \begin{figure}
        \centering
        \resizebox{1.1\textwidth}{!}{
            \begin{tikzpicture}[node distance=2.5cm]
                \node[state, initial] (p0) {$1_0p_0, q_1p_1$};
                \node[state, right of=p0] (p1) {$q_1p_1$};
                \node[state, right of=p1] (p2) {$q_2p_2$};
                \node[state, right of=p2, accepting] (p3) {$q_0p_3$};
                \node[state, right of=p3] (p4) {$q_1p_4$};
                \node[state, right of=p4] (p5) {$1_1p_5$};
                \node[state, right of=p5, accepting] (p6) {$q_1p_6$};
                \node[state, error, below of=p2] (E) {$E$};
    
                \draw   (p0) edge[above] node{0} (p1)
                        (p1) edge[above, bend right] node{0} (p0)
                        (p1) edge[above] node{1} (p2)
                        (p2) edge[above] node{1} (p3)
                        (p3) edge[loop above] node{1} (p3)
                        (p3) edge[above] node{0} (p4)
                        (p4) edge[above] node{0} (p5)
                        (p5) edge[above] node{0} (p6)
                        (p6) edge[above, bend right] node{0} (p4)
                        (p0) edge[above] node{1} (E)
                        (p2) edge[above right] node{0} (E)
                        (p4) edge[above] node{1} (E)
                        (p5) edge[above] node{1} (E)
                        (p6) edge[below] node{1} (E)
                        (E) edge[loop left] node{0,1} (E);
            \end{tikzpicture}}
        \caption{AFD minimal asociado a $(L_1\cap L_2)$ del Ejercicio~\ref{ej:1.3.24}.}
        \label{fig:1.3.24-L1-L2-minimal}
    \end{figure}
    Como podemos ver en el AFD minimal de la Figura~\ref{fig:1.3.24-L1-L2-minimal}, este es isomorfo al de la Figura~\ref{fig:1.3.24-L2}, por lo que $L_1\cap L_2 = L_2$. Esto es algo que podríamos haber deducido al principio, ya que $L_2\subset L_1$.
\end{ejercicio}

\begin{ejercicio}\label{ej:1.3.25}
    Dado el alfabeto $A=\{a,b,c\}$, encuentra:
    \begin{enumerate}
        \item \label{ej:1.3.25-1}
        Un AFD que reconozca las palabras en las que cada ``c'' va precedida de una ``a'' o una ``b''.
        \begin{figure}
            \centering
            \begin{tikzpicture}
                \node[state, initial, accepting] (q0) {$q_0$};
                \node[state, below right of=q0, error] (E) {$E$};
                \node[state, above right of=E, accepting] (q1) {$q_1$};

                \draw   (q0) edge[above] node{$a,b$} (q1)
                        (q0) edge[above] node{$c$} (E)
                        (q1) edge[loop right] node{$a,b$} (q1)
                        (q1) edge[above, bend right] node{$c$} (q0)
                        (E) edge[loop right] node{$a,b,c$} (E);
            \end{tikzpicture}
            \caption{AFD minimal asociado al lenguaje del Ejercicio~\ref{ej:1.3.25}.\ref{ej:1.3.25-1}.}
            \label{fig:1.3.25-1}
        \end{figure}

        \item Una expresión regular que represente el lenguaje compuesto por las palabras de longitud impar en las que el símbolo central es una ``c''.
        
        Veamos que este lenguaje no es regular usando el lema de bombeo. Para todo $n\in\bb{N}$, tomamos la palabra $z=a^nca^n\in L$, con $|z|=2n+1\geq n$. Toda descomposición de $z$ en $uvw$, con $u,v,w\in \{0,1\}^*$, $|uv|\leq n$ y $|v|\geq 1$ debe tener:
        \begin{equation*}
            u=a^k, \quad v=a^l, \quad w=a^{n-k-l}ca^n\qquad \text{con } l,k\in \bb{N}\cup \{0\}, l\geq 1, k+l\leq n
        \end{equation*}
        Para $i=2$, tenemos que $uv^2w=a^{k+2l+n-k-l}ca^n=a^{n+l}ca^n\notin L$ ya que, como $l\geq 1$, $n+l\neq n$, por lo que $c$ no es el símbolo central. Por lo tanto, por el contrarrecíproco del lema de bombeo, $L$ no es regular; y por tanto no podemos encontrar una expresión regular que lo represente.

        \item Una gramática regular que genere las palabras de longitud impar.
        
        Sea $G=(\{S,X\}, A, P, S)$, donde $P$ es:
        \begin{align*}
            S &\to aX \mid bX \mid cX, \\
            X &\to aS \mid bS \mid cS \mid \veps.
        \end{align*}

        Tenemos que $\cc{L}(G)=\{w\in A^* \mid |w|\text{ es impar}\}$.
    \end{enumerate}
\end{ejercicio}

\begin{ejercicio}\label{ej:1.3.26}
    Construir autómatas finitos para los siguientes lenguajes sobre el alfabeto $\{a,b,c\}$:
    \begin{enumerate}
        \item $L_1$: palabras del lenguaje ${(a+b)}^{\ast}{(b+c)}^{\ast}$.
        
        El AFND asociado a $L_1$ es el de la Figura~\ref{fig:1.3.26-L1-ND}.
        \begin{figure}
            \centering
            \begin{tikzpicture}
                \node[state, initial] (q0) {$q_0$};
                \node[state, right of=q0, accepting] (q1) {$q_1$};

                \draw   (q0) edge[loop above] node{$a,b$} (q0)
                        (q0) edge[above] node{$\veps$} (q1)
                        (q1) edge[loop above] node{$b,c$} (q1);
            \end{tikzpicture}
            \caption{AFND asociado a $L_1$ del Ejercicio~\ref{ej:1.3.26}.}
            \label{fig:1.3.26-L1-ND}
        \end{figure}

        El AFD minimal asociado a $L_1$ es el de la Figura~\ref{fig:1.3.26-L1}.
        \begin{figure}
            \centering
            \begin{tikzpicture}
                \node[state, initial, accepting] (q0q1) {$q_0q_1$};
                \node[state, right of=q0q1, accepting] (q1) {$q_1$};
                \node[state, right of=q1, error] (E) {$E$};

                \draw   (q0q1) edge[loop above] node{$a,b$} (q0q1)
                        (q0q1) edge[below] node{$c$} (q1)
                        (q1) edge[loop above] node{$b,c$} (q1)
                        (q1) edge[above] node{$a$} (E)
                        (E) edge[loop right] node{$a,b,c$} (E);
            \end{tikzpicture}
            \caption{AFD minimal asociado a $L_1$ del Ejercicio~\ref{ej:1.3.26}.}
            \label{fig:1.3.26-L1}
        \end{figure}
        \item $L_2$: palabras en las que nunca hay una ``a'' posterior a una ``c''.
        
        Vemos que $L_2$ tiene el mismo AFD que el de la Figura~\ref{fig:1.3.26-L1}, por lo que $L_1=L_2$.

        \item $(L_1 \setminus L_2)\cup (L_2 \setminus L_1)$
        
        Como $L_1=L_2$, tenemos que $(L_1 \setminus L_2)\cup (L_2 \setminus L_1)=\emptyset$.
    \end{enumerate}
\end{ejercicio}

\begin{ejercicio}\label{ej:1.3.27}
    Si $f:{\{0,1\}}^{\ast}\rightarrow{\{a,b,c\}}^{\ast}$ es un homomorfismo dado por
    \begin{equation*}
        f(0) = aab \qquad f(1) = bbc
    \end{equation*}
    dar autómatas finitos deterministas minimales para los lenguajes $L$ y $f^{-1}(L)$ donde $L\subseteq {\{a,b,c\}}^{\ast}$ es el lenguaje en el que el número de símbolos $a$ no es múltiplo de 4.\\

    El autómata finito minimal asociado a $L$ es el de la Figura~\ref{fig:1.3.27-L}, donde $q_i$ representa el estado en el que $n_a \mod 4 = i$.
    Empleando un distino número de cadenas de $a$'s, vemos que los estados son distinguibles, por lo que es minimal.
    \begin{figure}
        \centering
        \begin{tikzpicture}
            \node[state, initial] (q0) {$q_0$};
            \node[state, right of=q0, accepting] (q1) {$q_1$};
            \node[state, right of=q1, accepting] (q2) {$q_2$};
            \node[state, right of=q2, accepting] (q3) {$q_3$};

            \draw   (q0) edge[above] node{$a$} (q1)
                    (q0) edge[loop above] node{$b,c$} (q0)
                    (q1) edge[above] node{$a$} (q2)
                    (q1) edge[loop above] node{$b,c$} (q1)
                    (q2) edge[above] node{$a$} (q3)
                    (q2) edge[loop above] node{$b,c$} (q2)
                    (q3) edge[loop above] node{$b,c$} (q3)
                    (q3) edge[below, bend left] node{$a$} (q0);
        \end{tikzpicture}
        \caption{AFD minimal asociado a $L$ del Ejercicio~\ref{ej:1.3.27}.}
        \label{fig:1.3.27-L}
    \end{figure}

    Empleando el algoritmo, el AFD asociado a $f^{-1}(L)$ es el de la Figura~\ref{fig:1.3.27-f-1-L}.
    No obstante, y debido a que hay estados inaccesibles, este no es minimal. El AFD minimal asociado a $f^{-1}(L)$ es el de la Figura~\ref{fig:1.3.27-f-1-L-minimal}.
    \begin{figure}
        \centering
        \begin{tikzpicture}
            \node[state, initial] (q0) {$q_0$};
            \node[state, below of=q0, accepting] (q1) {$q_1$};
            \node[state, right of=q0, accepting] (q2) {$q_2$};
            \node[state, right of=q1, accepting] (q3) {$q_3$};

            \draw   (q0) edge[above] node{$0$} (q2)
                    (q0) edge[loop above] node{$1$} (q0)
                    (q1) edge[above] node{$0$} (q3)
                    (q1) edge[loop above] node{$1$} (q1)
                    (q2) edge[below, bend left] node{$0$} (q0)
                    (q2) edge[loop above] node{$1$} (q2)
                    (q3) edge[loop above] node{$1$} (q3)
                    (q3) edge[below, bend left] node{$0$} (q1);
        \end{tikzpicture}
        \caption{AFD asociado a $f^{-1}(L)$ del Ejercicio~\ref{ej:1.3.27}.}
        \label{fig:1.3.27-f-1-L}
    \end{figure}
    \begin{figure}
        \centering
        \begin{tikzpicture}
            \node[state, initial] (q0) {$q_0$};
            \node[state, right of=q0, accepting] (q2) {$q_2$};

            \draw   (q0) edge[above] node{$0$} (q2)
                    (q0) edge[loop above] node{$1$} (q0)
                    (q2) edge[below, bend left] node{$0$} (q0)
                    (q2) edge[loop above] node{$1$} (q2);
        \end{tikzpicture}
        \caption{AFD minimal asociado a $f^{-1}(L)$ del Ejercicio~\ref{ej:1.3.27}.}
        \label{fig:1.3.27-f-1-L-minimal}
    \end{figure}
\end{ejercicio}

\begin{ejercicio}\label{ej:1.3.28}
    Si $L_1$ es el lenguaje asociado a la expresión regular $01{(01+1)}^{\ast}$ y $L_2$ el lenguaje asociado a la expresión ${(1+10)}^{\ast}01$, encontrar un autómata minimal que acepte el lenguaje $L_1\setminus L_2$.\\

    El AFD minimal asociado a $L_1$ es el de la Figura~\ref{fig:1.3.28-L1}.
    \begin{figure}
        \centering
        \begin{tikzpicture}
            \node[state, initial] (q0) {$q_0$};
            \node[state, right of=q0] (q1) {$q_1$};
            \node[state, right of=q1, accepting] (q2) {$q_2$};
            \node[state, error, below of=q1] (E) {$E$};
            
            \draw   (q0) edge[above] node{0} (q1)
                    (q0) edge[above] node{1} (E)
                    (q1) edge[above] node{1} (q2)
                    (q1) edge[right] node{0} (E)
                    (q2) edge[loop above] node{1} (q2)
                    (q2) edge[above, bend right] node{0} (q1)
                    (E) edge[loop right] node{0,1} (E);
        \end{tikzpicture}
        \caption{AFD minimal asociado a $L_1$ del Ejercicio~\ref{ej:1.3.28}.}
        \label{fig:1.3.28-L1}
    \end{figure}

    El AFND asociado a $L_2$ es el de la Figura~\ref{fig:1.3.28-L2-ND}.
    \begin{figure}
        \centering
        \begin{tikzpicture}
            \node[state, initial] (q0) {$q_0$};
            \node[state, above right of=q0] (q1) {$q_1$};
            \node[state, right of=q0] (q2) {$q_2$};
            \node[state, right of=q2, accepting] (q3) {$q_3$};

            \draw   (q0) edge[above] node{1} (q1)
                    (q0) edge[loop above] node{1} (q0)
                    (q0) edge[below] node{0} (q2)
                    (q1) edge[below, bend left] node{0} (q0)
                    (q2) edge[above] node{1} (q3);
        \end{tikzpicture}
        \caption{AFND asociado a $L_2$ del Ejercicio~\ref{ej:1.3.28}.}
        \label{fig:1.3.28-L2-ND}
    \end{figure}

    El AFD asociado a $L_2$ es el de la Figura~\ref{fig:1.3.28-L2}.
    \begin{figure}
        \centering
        \begin{tikzpicture}
            \node[state, initial] (q0) {$q_0$};
            \node[state, above right of=q0] (q0q1) {$q_0q_1$};
            \node[state, right of=q0q1] (q0q2) {$q_0q_2$};
            \node[state, right of=q0q2, accepting] (q0q1q3) {$q_0q_1q_3$};
            \node[state, below right of=q0] (q2) {$q_2$};
            \node[state, right of=q2, accepting] (q3) {$q_3$};
            \node[state, error, above of=q3, yshift=-2em] (E) {$E$};

            \draw   (q0) edge[above] node{1} (q0q1)
                    (q0) edge[below] node{0} (q2)
                    (q0q1) edge[above] node{0} (q0q2)
                    (q0q1) edge[loop above] node{1} (q0q1)
                    (q0q2) edge[below] node{1} (q0q1q3)
                    (q0q2) edge[above, bend right] node{0} (q2)
                    (q2) edge[above] node{1} (q3)
                    (q2) edge[left] node{0} (E)
                    (q3) edge[right] node{0,1} (E)
                    (q0q1q3) edge[above, bend right] node[pos=0.9]{0} (q0q2)
                    (q0q1q3) edge[above, bend right] node{1} (q0q1)
                    (E) edge[loop right] node{0,1} (E);
        \end{tikzpicture}
        \caption{AFD minimal asociado a $L_2$ del Ejercicio~\ref{ej:1.3.28}.}
        \label{fig:1.3.28-L2}
    \end{figure}

    El autómata asociado a $L_1\setminus L_2$ es el de la Figura~\ref{fig:1.3.28-L1-L2}, donde hemos agrupado los estados de la forma $E,q_i$ en $E$ (ya que todos esos son indistinguibles puesto que no son finales y no se puede llegar a ellos desde un estado final).
    \begin{figure}
        \centering
        \begin{tikzpicture}
            \node[state, initial] (p0q0) {$p_0q_0$};
            \node[state, right of=p0q0] (p1q2) {$p_1q_2$};
            \node[state, right of=p1q2] (p2q3) {$p_2q_3$};
            \node[state, above right of=p2q3, accepting] (p2E) {$p_2E$};
            \node[state, below right of=p2q3] (p1E) {$p_1E$};
            \node[state, below of=p1q2, error, yshift=2.2em] (E) {$E$};

            \draw   (p0q0) edge[above] node{0} (p1q2)
                    (p0q0) edge[below] node{1} (E)
                    (p1q2) edge[above] node{1} (p2q3)
                    (p1q2) edge[right] node{0} (E)
                    (p2q3) edge[above] node{0} (p1E)
                    (p2q3) edge[above] node{1} (p2E)
                    (p1E) edge[above] node{0} (E)
                    (p1E) edge[right, bend left] node{1} (p2E)
                    (p2E) edge[left, bend left] node{0} (p1E)
                    (p2E) edge[loop above] node{1} (p2E)
                    (E) edge[loop below] node{0,1} (E);

        \end{tikzpicture}
        \caption{AFD minimal asociado a $L_1\setminus L_2$ del Ejercicio~\ref{ej:1.3.28}.}
        \label{fig:1.3.28-L1-L2}
    \end{figure}
\end{ejercicio}

\begin{ejercicio}\label{ej:1.3.29}
    Sean los alfabetos $A_1=\{a,b,c,d\}$ y $A_2=\{0,1\}$ y el lenguaje $L\subseteq A^\ast_2$ dado por la expresión regular ${(0+1)}^{\ast}0(0+1)$, calcular una expresión regular para el lenguaje $f^{-1}(L)$ donde $f$ es el homomorfismo entre $A^\ast_1$ y $A^\ast_2$ dado por
    \begin{equation*}
        f(a)=01 \qquad f(b) = 1 \qquad f(c)=0 \qquad f(d)=00
    \end{equation*}

    El AFND asociado a $L$ es el de la Figura~\ref{fig:1.3.29-L-ND}.
    \begin{figure}
        \centering
        \begin{tikzpicture}
            \node[state, initial] (q0) {$q_0$};
            \node[state, right of=q0] (q1) {$q_1$};
            \node[state, right of=q1, accepting] (q2) {$q_2$};

            \draw   (q0) edge[loop above] node{0,1} (q0)
                    (q0) edge[above] node{0} (q1)
                    (q1) edge[above] node{0,1} (q2);
        \end{tikzpicture}
        \caption{AFND asociado a $L$ del Ejercicio~\ref{ej:1.3.29}.}
        \label{fig:1.3.29-L-ND}
    \end{figure}

    El AFD minimal asociado a $L$ es el de la Figura~\ref{fig:1.3.29-L}.
    \begin{figure}
        \centering
        \begin{tikzpicture}
            \node[state, initial] (q0) {$q_0$};
            \node[state, right of=q0] (q0q1) {$q_0q_1$};
            \node[state, above right of=q0q1, accepting] (q0q1q2) {$q_0q_1q_2$};
            \node[state, below right of=q0q1, accepting] (q0q2) {$q_0q_2$};

            \draw   (q0) edge[loop above] node{1} (q0)
                    (q0) edge[above] node{0} (q0q1)
                    (q0q1) edge[above] node{0} (q0q1q2)
                    (q0q1) edge[above] node{1} (q0q2)
                    (q0q1q2) edge[loop above] node{0} (q0q1q2)
                    (q0q1q2) edge[right, bend left] node{1} (q0q2)
                    (q0q2) edge[left, bend left] node{0} (q0q1)
                    (q0q2) edge[below, bend left] node{1} (q0);
        \end{tikzpicture}
        \caption{AFD minimal asociado a $L$ del Ejercicio~\ref{ej:1.3.29}.}
        \label{fig:1.3.29-L}
    \end{figure}

    El AFD asociado a $f^{-1}(L)$ es el de la Figura~\ref{fig:1.3.29-f-1-L}.
    \begin{figure}
        \centering
        \begin{tikzpicture}
            \node[state, initial] (q0) {$q_0$};
            \node[state, right of=q0, xshift=2em] (q0q1) {$q_0q_1$};
            \node[state, above right of=q0q1, accepting, xshift=2em, yshift=2em] (q0q1q2) {$q_0q_1q_2$};
            \node[state, below right of=q0q1, accepting, xshift=2em, yshift=-2em] (q0q2) {$q_0q_2$};

            \draw   (q0) edge[loop above] node{$b$} (q0)
                    (q0) edge[above] node{$c$} (q0q1)
                    (q0) edge[above] node{$a$} (q0q2)
                    (q0) edge[above] node{$d$} (q0q1q2)
                    (q0q1) edge[above] node{$c,d$} (q0q1q2)
                    (q0q1) edge[above, bend left] node{$a,b$} (q0q2)
                    (q0q1q2) edge[loop above] node{$c,d$} (q0q1q2)
                    (q0q1q2) edge[right, bend left] node{$a,b$} (q0q2)
                    (q0q2) edge[below] node{$c$} (q0q1)
                    (q0q2) edge[below, bend left] node{$b$} (q0)
                    (q0q2) edge[loop below] node{$a$} (q0q2)
                    (q0q2) edge[left] node{$d$} (q0q1q2);
        \end{tikzpicture}
        \caption{AFD asociado a $f^{-1}(L)$ del Ejercicio~\ref{ej:1.3.29}.}
        \label{fig:1.3.29-f-1-L}
    \end{figure}

    Para obtener la expresión regular asociada a $f^{-1}(L)$, planteamos el sistema de ecuaciones siguiente:
    \begin{equation*}
        \begin{cases}
            q_0 = bq_0 + cq_0q_1 + aq_0q_2 + dq_0q_1q_2\\
            q_0q_1 = (c+d)q_0q_1q_2 + (a+b)q_0q_2\\
            q_0q_1q_2 = (c+d)q_0q_1q_2 + (a+b)q_0q_2 + \veps\\
            q_0q_2 = cq_0q_1 + bq_0 + aq_0q_2 + dq_0q_1q_2 + \veps
        \end{cases}
    \end{equation*}
    
    Sería necesario resolver el sistema para obtener la expresión regular.
\end{ejercicio}

\begin{ejercicio}\label{ej:1.3.30}
    Obtener un autómata finito determinista para el lenguaje asociado a la expresión regular: ${(01)}^{+}+{(010)}^{\ast}$. Minimizarlo.\\


    El AFND asociado a la expresión regular es el de la Figura~\ref{fig:1.3.30-ND}.
    \begin{figure}
        \centering
        \begin{tikzpicture}
            \node[state, initial, accepting] (q0) {$q_0$};
            \node[state, above right of=q0] (q1) {$q_1$};
            \node[state, right of=q1, accepting] (q2) {$q_2$};
            \node[state, right of=q2] (q3) {$q_3$};
            \node[state, below right of=q0] (q4) {$q_4$};
            \node[state, right of=q4] (q5) {$q_5$};
            \node[state, right of=q5, accepting] (q6) {$q_6$};

            \draw   (q0) edge[above] node{0} (q1)
                    (q1) edge[above] node{1} (q2)
                    (q2) edge[above] node{0} (q3)
                    (q3) edge[above, bend right] node{1} (q2)
                    (q0) edge[above] node{0} (q4)
                    (q4) edge[above] node{1} (q5)
                    (q5) edge[above] node{0} (q6)
                    (q6) edge[above, bend right] node{0} (q4);
        \end{tikzpicture}
        \caption{AFND asociado a la expresión regular del Ejercicio~\ref{ej:1.3.30}.}
        \label{fig:1.3.30-ND}
    \end{figure}

    El AFD asociado a la expresión regular es el de la Figura~\ref{fig:1.3.30}, donde las transiciones que faltan son nulas.
    \begin{figure}
        \centering
        \begin{tikzpicture}[node distance=2.5cm]
            \node[state, initial, accepting] (q0) {$q_0$};
            \node[state, right of=q0] (q1q4) {$q_1q_4$};
            \node[state, right of=q1q4, accepting] (q2q5) {$q_2q_5$};
            \node[state, right of=q2q5, accepting] (q3q6) {$q_3q_6$};
            \node[state, below right of=q3q6, accepting] (q2) {$q_2$};
            \node[state, right of=q2] (q3) {$q_3$};
            \node[state, above right of=q3q6] (q4) {$q_4$};
            \node[state, right of=q4] (q5) {$q_5$};
            \node[state, right of=q5, accepting] (q6) {$q_6$};
            \node[state, below of=q2q5, error] (E) {$E$};

            \draw   (q0) edge[above] node{0} (q1q4)
                    (q0) edge[below] node{1} (E)
                    (q1q4) edge[above] node{1} (q2q5)
                    (q1q4) edge[below] node{0} (E)
                    (q2q5) edge[above] node{0} (q3q6)
                    (q2q5) edge[right] node{1} (E)
                    (q3q6) edge[above] node{1} (q2)
                    (q3q6) edge[above] node{0} (q4)
                    (q2) edge[above] node{0} (q3)
                    (q2) edge[above] node{1} (E)
                    (q3) edge[below, bend left] node{0} (E)
                    (q3) edge[above, bend right] node{1} (q2)
                    (q4) edge[above] node{1} (q5)
                    (q5) edge[above] node{0} (q6)
                    (q6) edge[above, bend right] node{0} (q4)
                    (E) edge[loop below] node{0,1} (E);
        \end{tikzpicture}
        \caption{AFD asociado a la expresión regular del Ejercicio~\ref{ej:1.3.30}.}
        \label{fig:1.3.30}
    \end{figure}

    La minimización del autómata se encuentra en la Tabla~\ref{tab:1.3.30-min}.
    \begin{table}
        \centering
        \begin{tabular}{r ccccccccc}
            \hhline{~*{1}{-}}
            $q_1q_4$ & \cell{\times} \\ \hhline{~*{2}{-}}
            $q_2q_5$ & \cell{\times} & \cell{\times} \\ \hhline{~*{3}{-}}
            $q_3q_6$ & \cell{\times} & \cell{\times} & \cell{\times} \\ \hhline{~*{4}{-}}
            $q_2$ & \cell{\times} & \cell{\times} & \cell{\times} & \cell{\times} \\ \hhline{~*{5}{-}}
            $q_3$ & \cell{\times} & \cell{\times} & \cell{\times} & \cell{\times} & \cell{\times} \\ \hhline{~*{6}{-}}
            $q_4$ & \cell{\times} & \cell{\times} & \cell{\times} & \cell{\times} & \cell{\times} & \cell{\times} \\ \hhline{~*{7}{-}}
            $q_5$ & \cell{\times} & \cell{\times} & \cell{\times} & \cell{\times} & \cell{\times} & \cell{\times} & \cell{\times} \\ \hhline{~*{8}{-}}
            $q_6$ & \cell{\times} & \cell{\times} & \cell{\times} & \cell{\times} & \cell{\times} & \cell{\times} & \cell{\times} & \cell{\times} \\ \hhline{~*{9}{-}}
            $E$ & \cell{\times} & \cell{\times} & \cell{\times} & \cell{\times} & \cell{\times} & \cell{\times} & \cell{\times} & \cell{\times} & \cell{\times} \\ \hhline{~*{9}{-}}
            & $q_0$ & $q_1q_4$ & $q_2q_5$ & $q_3q_6$ & $q_2$ & $q_3$ & $q_4$ & $q_5$ & $q_6$
        \end{tabular}
        \caption{Minimización del autómata del Ejercicio~\ref{ej:1.3.30}.}
        \label{tab:1.3.30-min}
    \end{table}

    Por tanto, como todos los estados del AFD de la Figura~\ref{fig:1.3.30} son distinguibles, este es minimal.
\end{ejercicio}

\begin{ejercicio}\label{ej:1.3.31}
    Dado el lenguaje $L$ asociado a la expresión regular ${(01+011)}^{\ast}$ y el homomorfismo $f:{\{0,1\}}^{\ast}\rightarrow{\{0,1\}}^{\ast}$ dado por $f(0)=01$, $f(1)=1$, construir una expresión regular para el lenguaje $f^{-1}(L)$.

    El AFND asociado a la expresión regular es el de la Figura~\ref{fig:1.3.31-ND}.
    \begin{figure}
        \centering
        \begin{tikzpicture}
            \node[state, initial, accepting] (q0) {$q_0$};
            \node[state, above right of=q0] (q1) {$q_1$};
            \node[state, below right of=q0] (q2) {$q_2$};
            \node[state, right of=q2] (q3) {$q_3$};

            \draw   (q0) edge[above] node{0} (q1)
                    (q0) edge[below] node{0} (q2)
                    (q1) edge[above, bend right] node{1} (q0)
                    (q2) edge[above] node{1} (q3)
                    (q3) edge[above, bend right] node{1} (q0);
        \end{tikzpicture}
        \caption{AFND asociado al lenguaje $L$ del Ejercicio~\ref{ej:1.3.31}.}
        \label{fig:1.3.31-ND}
    \end{figure}

    El AFD asociado a la expresión regular es el de la Figura~\ref{fig:1.3.31}.
    \begin{figure}
        \centering
        \begin{tikzpicture}
            \node[state, initial, accepting] (q0) {$q_0$};
            \node[state, right of=q0] (q1q2) {$q_1q_2$};
            \node[state, right of=q1q2, accepting] (q0q3) {$q_0q_3$}
            node[state, below of=q1q2, error] (E) {$E$};

            \draw   (q0) edge[above] node{0} (q1q2)
                    (q0) edge[below] node{1} (E)
                    (q1q2) edge[above] node{1} (q0q3)
                    (q1q2) edge[left] node{0} (E)
                    (q0q3) edge[above, bend right] node{1} (q0)
                    (q0q3) edge[below, bend left] node{0} (q1q2)
                    (E) edge[loop right] node{0,1} (E);
        \end{tikzpicture}
        \caption{AFD asociado al lenguaje $L$ del Ejercicio~\ref{ej:1.3.31}.}
        \label{fig:1.3.31}
    \end{figure}

    El AFD asociado a $f^{-1}(L)$ es el de la Figura~\ref{fig:1.3.31-f-1-L}.
    \begin{figure}
        \centering
        \begin{tikzpicture}
            \node[state, initial, accepting] (q0) {$q_0$};
            \node[state, right of=q0, accepting] (q0q3) {$q_0q_3$};
            \node[state, below of=q0, error] (E) {$E$};
            \node[state, right of=E] (q1q2) {$q_1q_2$};

            \draw   (q0) edge[above] node{0} (q0q3)
                    (q0) edge[left] node{1} (E)
                    (q1q2) edge[right] node{1} (q0q3)
                    (q1q2) edge[above] node{0} (E)
                    (q0q3) edge[above, bend right] node{1} (q0)
                    (q0q3) edge[loop above] node{0} (q0q3)
                    (E) edge[loop left] node{0,1} (E);
        \end{tikzpicture}
        \caption{AFD asociado a $f^{-1}(L)$ del Ejercicio~\ref{ej:1.3.31}.}
        \label{fig:1.3.31-f-1-L}
    \end{figure}

    No obstante, este no es minimal, puesto que tiene estados inaccesibles. El AFD minimal asociado a $f^{-1}(L)$ es el de la Figura~\ref{fig:1.3.31-f-1-L-minimal}.
    \begin{figure}
        \centering
        \begin{tikzpicture}
            \node[state, initial, accepting] (q0) {$q_0$};
            \node[state, right of=q0, accepting] (q0q3) {$q_0q_3$};
            \node[state, below of=q0, error] (E) {$E$};

            \draw   (q0) edge[above] node{0} (q0q3)
                    (q0) edge[left] node{1} (E)
                    (q0q3) edge[above, bend right] node{1} (q0)
                    (q0q3) edge[loop above] node{0} (q0q3)
                    (E) edge[loop left] node{0,1} (E);
        \end{tikzpicture}
        \caption{AFD minimal asociado a $f^{-1}(L)$ del Ejercicio~\ref{ej:1.3.31}.}
        \label{fig:1.3.31-f-1-L-minimal}
    \end{figure}

    Para obtener la expresión regular asociada a $f^{-1}(L)$, planteamos el sistema de ecuaciones siguiente:
    \begin{equation*}
        \begin{cases}
            q_0 = 0q_0q_3 + 1E+\veps\\
            q_0q_3 = 0q_0q_3 + 1q_0+\veps\\
            E = (0+1)E
        \end{cases}
    \end{equation*}

    Por el Lema de Arden, tenemos que $E=(0+1)^{\ast}\emptyset=\emptyset$ y $q_0q_3=0^*(1q_0+\veps)$. Por tanto, tenemos que:
    \begin{align*}
        q_0 &= 00^*(1q_0+\veps) +\veps
        = 0^+(1q_0+\veps) + \veps
        \Longrightarrow \\
        \Longrightarrow 
        q_0 &= (0^{+}1)^*(0^+ + \veps)
    \end{align*}

    Por tanto, la expresión regular asociada a $f^{-1}(L)$ es $(0^{+}1)^*(0^+ + \veps)$.

\end{ejercicio}

\begin{ejercicio}\label{ej:1.3.32}
    Dar expresiones regulares para los siguientes lenguajes sobre el alfabeto $A_1=\{0,1,2\}$:
    \begin{enumerate}
        \item $L$ dado por el conjunto de palabras en las que cada 0 que no sea el último de la palabra va seguido por un 1 y cada 1 que no sea el último símbolo de la palabra va seguido por un 0.
        
        El AFD minimal asociado a $L$ es el de la Figura~\ref{fig:1.3.32-L}.
        \begin{figure}
            \centering
            \begin{tikzpicture}
                \node[state, initial, accepting] (q0) {$q_0$};
                \node[state, above right of=q0, accepting] (q1) {$q_1$};
                \node[state, below right of=q0, accepting] (q2) {$q_2$};
                \node[state, below right of=q1, error] (E) {$E$};

                \draw   (q0) edge[above] node{0} (q1)
                        (q0) edge[below] node{1} (q2)
                        (q0) edge[loop above] node{2} (q0)
                        (q1) edge[right, bend right] node{1} (q2)
                        (q1) edge[above right] node{0,2} (E)
                        (q2) edge[below right] node{1,2} (E)
                        (q2) edge[left, bend right] node{0} (q1)
                        (E) edge[loop right] node{0,1,2} (E);
            \end{tikzpicture}
            \caption{AFD minimal asociado a $L$ del Ejercicio~\ref{ej:1.3.32}.}
            \label{fig:1.3.32-L}
        \end{figure}

        Para obtener la expresión regular asociada a $L$, planteamos el sistema de ecuaciones siguiente:
        \begin{equation*}
            \begin{cases}
                q_0 = 0q_1 + 1q_2+2q_0+\veps\\
                q_1 = 1q_2 + (0+2)E+ \veps\\
                q_2 = (1+2)E + 0q_1+ \veps\\
                E = (0+1+2)E
            \end{cases}
        \end{equation*}

        Por el Lema de Arden, tenemos que $E=(0+1+2)^{\ast}\emptyset=\emptyset$, por lo que $q_2=0q_1+\veps$. Por tanto:
        \begin{align*}
            q_1 &= 1(0q_1+\veps) + 0\emptyset +\veps
            = 1(0q_1 +\veps) +\veps
            \Longrightarrow \\
            \Longrightarrow
            q_1 &= (10)^{\ast}(1+\veps)
        \end{align*}

        Por tanto, la expresión regular asociada a $L$ es:
        \begin{align*}
            q_0&= 0(10)^{\ast}(1+\veps)+1(0(10)^{\ast}(1+\veps)+\veps)+2q_0+\veps
            =\\&= 0(10)^{\ast}(1+\veps)+ 10(10)^{\ast}(1+\veps)+1+2q_0+\veps
            =\\&= [0(10)^{\ast}(1+\veps)+ 10(10)^{\ast}(1+\veps)+1+\veps] + 2q_0
            =\\&= [0(10)^{\ast}+ 10(10)^{\ast}+\veps](1+\veps) + 2q_0
            =\\&= [(1+\veps)0(10)^{\ast}+\veps](1+\veps) + 2q_0
            =\\&= 2^*[(1+\veps)0(10)^{\ast}+\veps](1+\veps)
        \end{align*}
        \item Considera el homomorfismo de $A_1$ en $A_2=\{0,1\}$  dado por $f(0)=001$, $f(1)=100$, $f(2)=0011$. Dar una expresión regular para $f(L)$.
        
        La expresión regular asociada a $f(L)$ es:
        \begin{align*}
            (001)^*[(100+\veps)001(100001)^{\ast}+\veps](100+\veps)
        \end{align*}
        \item Dar una expresión regular para $LL^{-1}$.
        
        Una expresión regular para $L^{-1}$ es:
        \begin{equation*}
            (1+\veps)[(01)^{\ast}0(1+\veps)+\veps]2^*
        \end{equation*}

        Por tanto, una expresión regular para $LL^{-1}$ es:
        \begin{equation*}
            2^*[(1+\veps)0(10)^{\ast}+\veps](1+\veps)
            (1+\veps)[(01)^{\ast}0(1+\veps)+\veps]2^*
        \end{equation*}
    \end{enumerate}
\end{ejercicio}

\begin{ejercicio}\label{ej:1.3.33}
    Dados los lenguajes
    \begin{align*}
        L_1 &= \{0^i 1^j \mid i\geq 1, j\text{ es par y } j\geq 2\}\\
        L_2 &= \{1^j 0^k \mid k\geq 1, j\text{ es impar y } j\geq 1\}
    \end{align*}
    Encuentre:
    \begin{enumerate}
        \item Una gramática regular que genere el lenguaje $L_1$.
        
        El AFD minimal asociado a $L_1$ es el de la Figura~\ref{fig:1.3.33-L1}.
        \begin{figure}
            \centering
            \begin{tikzpicture}
                \node[state, initial] (q0) {$q_0$};
                \node[state, right of=q0] (q1) {$q_1$};
                \node[state, right of=q1] (q2) {$q_2$};
                \node[state, right of=q2, accepting] (q3) {$q_3$};
                \node[state, below of=q2, error] (E) {$E$};

                \draw   (q0) edge[above] node{0} (q1)
                        (q0) edge[below] node{1} (E)
                        (q1) edge[loop above] node{0} (q1)
                        (q1) edge[above] node{1} (q2)
                        (q2) edge[above] node{1} (q3)
                        (q2) edge[left] node{0} (E)
                        (q3) edge[above, bend right] node{1} (q2)
                        (q3) edge[below] node{0} (E)
                        (E) edge[loop right] node{0,1} (E);
            \end{tikzpicture}
            \caption{AFD minimal asociado a $L_1$ del Ejercicio~\ref{ej:1.3.33}.}
            \label{fig:1.3.33-L1}
        \end{figure}

        La gramática regular asociada a $L_1$ por tanto que lo genera es $G=(V,\{0,1\},P,q_0)$ donde $P$ es:
        \begin{align*}
            V &= \{q_0,q_1,q_2,q_3\}\\
            P &= \begin{cases}
                q_0\rightarrow 0q_1\\
                q_1\rightarrow 0q_1\mid 1q_2\\
                q_2\rightarrow 1q_3 \\
                q_3\rightarrow 1q_2 \mid \veps
            \end{cases}
        \end{align*}
        

        \item Una expresión regular que represente al lenguaje $L_2$.
        
        El AFD minimal asociado a $L_2$ es el de la Figura~\ref{fig:1.3.33-L2}.
        \begin{figure}
            \centering
            \begin{tikzpicture}
                \node[state, initial] (q0) {$q_0$};
                \node[state, right of=q0] (q1) {$q_1$};
                \node[state, right of=q1, accepting] (q2) {$q_2$};
                \node[state, below of=q1, error] (E) {$E$};

                \draw   (q0) edge[above] node{1} (q1)
                        (q0) edge[below] node{0} (E)
                        (q1) edge[above, bend right] node{1} (q0)
                        (q1) edge[above] node{0} (q2)
                        (q2) edge[loop above] node{0} (q2)
                        (q2) edge[above] node{1} (E)
                        (E) edge[loop right] node{0,1} (E);
            \end{tikzpicture}
            \caption{AFD minimal asociado a $L_2$ del Ejercicio~\ref{ej:1.3.33}.}
            \label{fig:1.3.33-L2}
        \end{figure}

        La expresión regular asociada a $L_2$ se obtiene resolviendo el sistema de ecuaciones siguiente:
        \begin{equation*}
            \begin{cases}
                q_0 = 1q_1 + 0E\\
                q_1 = 1q_0 + 0q_2\\
                q_2 = 0q_2 + 1E+\veps\\
                E = (0+1)E
            \end{cases}
        \end{equation*}

        Por el Lema de Arden, tenemos que $E=(0+1)^{\ast}\emptyset=\emptyset$ y $q_2=0^*\veps=0^*$. Por tanto:
        \begin{equation*}
            q_1 = 1q_0 + 00^+ = 1q_0+0^+
        \end{equation*}

        Por tanto, la expresión regular asociada a $L_2$ es:
        \begin{equation*}
            q_0=1(1q_0+0^+) = (11)^*(10^+)
        \end{equation*}

        \item Un automata finito determinista que acepte las cadenas de la concatenación de los lenguajes $L_1$ y $L_2$. Aplica el algoritmo para minimizar este autómata.
        
        El AFD asociado a $L_1L_2$ es el de la Figura~\ref{fig:1.3.33-L1L2}. Este es:
        \begin{align*}
            L_1L_2&=\{0^i1^{j+j'}0^k \mid i\geq 1, j\text{ es par y } j\geq 2, j'\text{ es impar y } j'\geq 1, k\geq 1\}
            =\\&= \{0^i1^{j}0^k \mid i\geq 1, j\text{ es impar y } j\geq 3, k\geq 1\}
        \end{align*}
        \begin{figure}
            \centering
            \begin{tikzpicture}[node distance=2.6cm]
                \node[state, initial] (q0) {$q_0$};
                \node[state, right of=q0] (q1) {$q_1$};
                \node[state, right of=q1] (q2) {$q_2$};
                \node[state, right of=q2,] (q3) {$q_3=p_0$};
                \node[state, below of=q3, error] (E) {$E$};

                \draw   (q0) edge[above] node{0} (q1)
                        (q0) edge[below] node{1} (E)
                        (q1) edge[loop above] node{0} (q1)
                        (q1) edge[above] node{1} (q2)
                        (q2) edge[above] node{1} (q3)
                        (q2) edge[left] node{0} (E)
                        (q3) edge[right] node{0} (E)
                        (E) edge[loop right] node{0,1} (E);

                
                \node[state, right of=q3] (p1) {$p_1$};
                \node[state, right of=p1, accepting] (p2) {$p_2$};


                \draw   (q3) edge[above] node{1} (p1)
                        (p1) edge[above, bend right] node{1} (q3)
                        (p1) edge[above] node{0} (p2)
                        (p2) edge[loop above] node{0} (p2)
                        (p2) edge[above] node{1} (E);
            \end{tikzpicture}
            \caption{AFD asociado a $L_1L_2$ del Ejercicio~\ref{ej:1.3.33}.}
            \label{fig:1.3.33-L1L2}
        \end{figure}
        Este autómata vemos de forma directa que es minimal, puesto que desde cada estado para llegar al único estado final hemos de leer una cadena distinta.
    \end{enumerate}
\end{ejercicio}

\begin{ejercicio}\label{ej:1.3.34}
    Considerar los AFD $M_1=(\{A,B,C,D,E,F,G,H\}, \{0,1\}, \delta_1, A, \{C\})$ y $M_2=(\{A',B',C',D',G'\}, \{0,1\}, \delta_2, A', \{D'\})$ donde $\delta_1$ y $\delta_2$ están definidas por las Tablas~\ref{tab:1.3.34-M1} y~\ref{tab:1.3.34-M2} respectivamente.
    \begin{table}
        \centering
        \begin{tabular}{r|cccccccc}
            $\delta_1$ & $A$ & $B$ & $C$ & $D$ & $E$ & $F$ & $G$ & $H$ \\
            \hline
            $0$ & $B$ & $G$ & $A$ & $C$ & $H$ & $C$ & $G$ & $G$ \\
            $1$ & $F$ & $C$ & $C$ & $G$ & $F$ & $G$ & $E$ & $C$
        \end{tabular}
        \caption{Transiciones del autómata $M_1$ del Ejercicio~\ref{ej:1.3.34}.}
        \label{tab:1.3.34-M1}
    \end{table}
    \begin{table}
        \centering
        \begin{tabular}{r|ccccc}
            $\delta_2$ & $A'$ & $B'$ & $C'$ & $D'$ & $G'$ \\
            \hline
            $0$ & $G'$ & $B'$ & $D'$ & $A'$ & $B'$ \\
            $1$ & $C'$ & $A'$ & $B'$ & $D'$ & $D'$
        \end{tabular}
        \caption{Transiciones del autómata $M_2$ del Ejercicio~\ref{ej:1.3.34}.}
        \label{tab:1.3.34-M2}
    \end{table}
    Determinar si ambos autómatas finitos generan el mismo lenguaje.\\

    Como $M_1$ tiene más estados, comenzamos minimizando este autómata. Veamos cuáles son sus estados accesibles:
    \begin{equation*}
        \{A, B, F, G, C, E, H\} = Q\setminus \{D\}
    \end{equation*}
    Por tanto, tenemos que el único estado no accesible es $D$.
    La tabla de minimalización se encuentra en la Tabla~\ref{tab:1.3.34-M1-min}.
    \begin{table}
        \centering
        \begin{tabular}{r cccccc}
            \hhline{~*{1}{-}}
            $B$ & \cell{\times} \\ \hhline{~*{2}{-}}
            $C$ & \cell{\times} & \cell{\times} \\ \hhline{~*{3}{-}}
            $E$ & \cell{} & \cell{\times} & \cell{\times} \\ \hhline{~*{4}{-}}
            $F$ & \cell{\times} & \cell{\times} & \cell{\times} & \cell{\times} \\ \hhline{~*{5}{-}}
            $G$ & \cell{\times} & \cell{\times} & \cell{\times} & \cell{\times} & \cell{\times} \\ \hhline{~*{6}{-}}
            $H$ & \cell{\times} & \cell{(E,A)} & \cell{\times} & \cell{\times} & \cell{\times} & \cell{\times} \\ \hhline{~*{6}{-}}
            & $A$ & $B$ & $C$ & $E$ & $F$ & $G$ \\
        \end{tabular}
        \caption{Minimalización del autómata $M_1$ del Ejercicio~\ref{ej:1.3.34}.}
        \label{tab:1.3.34-M1-min}
    \end{table}

    Por tanto, notando por $\equiv$ a la relación de indistinguibilidad, tenemos que:
    \begin{equation*}
        H\equiv B\qquad A\equiv E
    \end{equation*}

    Por tanto, el autómata minimal de $M_1$ es:
    $$M_1^{\text{min}}=\{\{(AE),(BH),C,F,G\},\{0,1\},\delta_1^{\text{min}},(AE),\{C\}\}$$
    donde $\delta_1^{\text{min}}$ está definida por la Tabla~\ref{tab:1.3.34-M1-minimal}.
    \begin{table}
        \centering
        \begin{tabular}{r|ccccc}
            $\delta_1^{\text{min}}$ & $(AE)$ & $(BH)$ & $C$ & $F$ & $G$ \\
            \hline
            $0$ & $(BH)$ & $G$ & $(AE)$ & $C$ & $G$ \\
            $1$ & $F$ & $C$ & $C$ & $G$ & $(AE)$
        \end{tabular}
        \caption{Transiciones del autómata $M_1^{\text{min}}$ del Ejercicio~\ref{ej:1.3.34}.}
        \label{tab:1.3.34-M1-minimal}
    \end{table}

    Como podemos ver, $M_1^{\text{min}}$ y $M_2$ son isomorfos, con isomorfismo $f$ dado por:
    \begin{align*}
        f((AE)) &= A' \qquad (\text{inicial}) \\
        f(C) &= D' \qquad (\text{final})\\
        f((BH)) &= G'\\
        f(F) &= C'\\
        f(G) &= B'
    \end{align*}

    Por tanto, tenemos que $\cc{L}(M_1)=\cc{L}(M_2)$.
\end{ejercicio}

\begin{ejercicio}\label{ej:1.3.35}
    Comprobar si los autómatas de las Figuras~\ref{fig:1.3.35-M1} y~\ref{fig:1.3.35-M2} generan el mismo lenguaje.
    \begin{figure}
        \centering
        \begin{subfigure}[c]{0.45\textwidth}
            \hspace{-1cm}
            \begin{tikzpicture}[node distance=5.3em]
                \node[state, initial, accepting] (q0) {$q_0$};
                \node[state, above right of=q0, yshift=1em] (q1) {$q_1$};
                \node[state, below right of=q0, yshift=-1em] (q2) {$q_2$};
                \node[state, above right of=q2, yshift=1em] (q3) {$q_3$};
                \node[state, above right of=q3, yshift=1em] (q4) {$q_4$};
                \node[state, below right of=q3, yshift=-1em] (q5) {$q_5$};
                \node[state, above right of=q5, accepting, yshift=1em] (q6) {$q_6$};

                \draw   (q0) edge[above] node{$a$} (q1)
                        (q0) edge[above] node{$b$} (q3)
                        (q1) edge[above] node{$b$} (q4)
                        (q1) edge[above] node{$a$} (q3)
                        (q4) edge[above] node{$a$} (q6)
                        (q4) edge[above] node{$b$} (q3)
                        (q6) edge[left] node{$a$} (q5)
                        (q6) edge[above] node{$b$} (q3)
                        (q5) edge[above] node{$b$} (q2)
                        (q5) edge[above] node{$a$} (q3)
                        (q2) edge[above] node{$a$} (q0)
                        (q2) edge[above] node{$b$} (q3)
                        (q3) edge[loop above] node{$a,b$} (q3);
            \end{tikzpicture}
            \caption{Autómata $M_1$ del Ejercicio~\ref{ej:1.3.35}.}
            \label{fig:1.3.35-M1}
        \end{subfigure}
        \begin{subfigure}[c]{0.45\textwidth}
            \hspace{1cm}
            \begin{tikzpicture}[node distance=5.3em]
                \node[state, initial, accepting] (q0) {$p_0$};
                \node[state, right of=q0] (q1) {$p_1$};
                \node[state, right of=q1] (q2) {$p_2$};

                \draw   (q0) edge[above] node{$a$} (q1)
                        (q1) edge[above] node{$b$} (q2)
                        (q2) edge[below, bend left] node{$a$} (q0);
            \end{tikzpicture}
            \caption{Autómata $M_2$ del Ejercicio~\ref{ej:1.3.35}.}
            \label{fig:1.3.35-M2}
        \end{subfigure}
        \caption{Autómatas del Ejercicio~\ref{ej:1.3.35}.}
        \label{fig:1.3.35}
    \end{figure}
    
    En primer lugar, vemos que $q_3$ es distinguible del resto, pues que es el único estado desde el cual no se puede llegar a un estado final.
    La tabla de minimalización de $M_1$ se encuentra en la Tabla~\ref{tab:1.3.35-M1}.
    \begin{table}
        \centering
        \begin{tabular}{r cccccc}
            \hhline{~*{1}{-}}
            $q_1$ & \cell{\times} \\ \hhline{~*{2}{-}}
            $q_2$ & \cell{\times} & \cell{\times} \\ \hhline{~*{3}{-}}
            $q_3$ & \cell{\times} & \cell{\times} & \cell{\times} \\ \hhline{~*{4}{-}}
            $q_4$ & \cell{\times} & \cell{\times} & \cell{(q_1,q_5)} & \cell{\times} \\ \hhline{~*{5}{-}}
            $q_5$ & \cell{\times} & \cell{(q_0,q_6)} & \cell{\times} & \cell{\times} & \cell{\times} \\ \hhline{~*{6}{-}}
            $q_6$ & \cell{(q_2,q_4)} & \cell{\times} & \cell{\times} & \cell{\times} & \cell{\times} & \cell{\times} \\ \hhline{~*{6}{-}}
            & $q_0$ & $q_1$ & $q_2$ & $q_3$ & $q_4$ & $q_5$
        \end{tabular}
        \caption{Minimalización del autómata $M_1$ del Ejercicio~\ref{ej:1.3.35}.}
        \label{tab:1.3.35-M1}
    \end{table}

    Por tanto, el autómata minimal de $M_1$ es el de la Figura~\ref{fig:1.3.35-M1-min}.
    \begin{figure}
        \centering
        \begin{tikzpicture}
            \node[state, initial, accepting] (q0) {$q_0q_6$};
            \node[state, right of=q0] (q1) {$q_1q_5$};
            \node[state, right of=q1] (q2) {$q_2q_4$};
            \node[state, above of=q1, error] (q3) {$q_3$};

            \draw   (q0) edge[above] node{$a$} (q1)
                    (q0) edge[above] node{$b$} (q3)
                    (q1) edge[above] node{$b$} (q2)
                    (q1) edge[left] node{$a$} (q3)
                    (q2) edge[below, bend left] node{$a$} (q0)
                    (q2) edge[above] node{$b$} (q3)
                    (q3) edge[loop right] node{$a,b$} (q3);
        \end{tikzpicture}
        \caption{Autómata minimal asociado a $M_1$ del Ejercicio~\ref{ej:1.3.35}.}
        \label{fig:1.3.35-M1-min}
    \end{figure}

    Como vemos, el autómata minimal de $M_1$ es isomorfo al AFD asociado a $M_2$ (ya que cuenta con un estado de error con las transiciones restantes) con isomorfismo $f$ dado por:
    \begin{align*}
        f(q_0q_6) &= p_0 \qquad (\text{inicial})\\
        f(q_1q_5) &= p_1\\
        f(q_2q_4) &= p_2 \qquad (\text{final})\\
    \end{align*}

    Por tanto, tenemos que $\cc{L}(M_1)=\cc{L}(M_2)$.
\end{ejercicio}

\begin{ejercicio}\label{ej:1.3.36}
    Minimizar el autómata de la Figura~\ref{fig:1.3.36-M}.\\
    \begin{figure}
        \centering
        \begin{tikzpicture}
            \node[state, initial] (q0) {$q_0$};
            \node[state, right of=q0] (q1) {$q_1$};
            \node[state, right of=q1, accepting] (q2) {$q_2$};
            \node[state, below right of=q2] (q7) {$q_7$};
            \node[state, above right of=q7] (q3) {$q_3$};
            \node[state, below left of=q7, accepting] (q5) {$q_5$};
            \node[state, below right of=q7] (q6) {$q_6$};
            \node[state, left of=q5] (q4) {$q_4$};

            \draw   (q0) edge[above] node{$a$} (q1)
                    (q0) edge[above] node{$b$} (q4)
                    (q1) edge[above] node[pos=0.25]{$a$} (q5)
                    (q1) edge[above] node{$b$} (q2)
                    (q2) edge[above, bend left=20] node{$a$} (q3)
                    (q2) edge[above] node{$b$} (q7)
                    (q3) edge[below, bend left=20] node{$a$} (q2)
                    (q3) edge[above] node{$b$} (q7)
                    (q4) edge[above] node{$a$} (q5)
                    (q4) edge[above, bend right=20] node[pos=0.15]{$b$} (q2)
                    (q5) edge[above, bend left=20] node{$a$} (q6)
                    (q5) edge[above] node{$b$} (q7)
                    (q6) edge[below, bend left=20] node{$a$} (q5)
                    (q6) edge[above] node{$b$} (q7)
                    (q7) edge[loop right, right] node{$a,b$} (q7);
        \end{tikzpicture}
        \caption{Autómata a minimizar del Ejercicio~\ref{ej:1.3.36}.}
        \label{fig:1.3.36-M}
    \end{figure}

    En primer lugar, vemos que $q_7$ es distinguible del resto, pues que es el único estado desde el cual no se puede llegar a un estado final.
    La tabla de minimalización de $M$ se encuentra en la Tabla~\ref{tab:1.3.36-M}.
    \begin{table}
        \centering
        \begin{tabular}{r cccccccc}
            \hhline{~*{1}{-}}
            $q_1$ & \cell{\times} \\ \hhline{~*{2}{-}}
            $q_2$ & \cell{\times} & \cell{\times} \\ \hhline{~*{3}{-}}
            $q_3$ & \cell{\times} & \cell{\times} & \cell{\times} \\ \hhline{~*{4}{-}}
            $q_4$ & \cell{\times} & \cell{} & \cell{\times} & \cell{\times} \\ \hhline{~*{5}{-}}
            $q_5$ & \cell{\times} & \cell{\times} & \cell{(q_3, q_6)} & \cell{\times} & \cell{\times} \\ \hhline{~*{6}{-}}
            $q_6$ & \cell{\times} & \cell{\times} & \cell{\times} & \cell{(q_2, q_5)} & \cell{\times} & \cell{\times} \\ \hhline{~*{7}{-}}
            $q_7$ & \cell{\times} & \cell{\times} & \cell{\times} & \cell{\times} & \cell{\times} & \cell{\times} & \cell{\times} \\ \hhline{~*{7}{-}}
            & $q_0$ & $q_1$ & $q_2$ & $q_3$ & $q_4$ & $q_5$ & $q_6$
        \end{tabular}
        \caption{Minimalización del autómata $M$ del Ejercicio~\ref{ej:1.3.36}.}
        \label{tab:1.3.36-M}
    \end{table}

    Por tanto, si notamos por $\equiv$ a la relación de indistinguibilidad, tenemos que:
    \begin{equation*}
        q_4\equiv q_1 \qquad q_5\equiv q_2 \qquad q_6\equiv q_3
    \end{equation*}

    Por tanto, el autómata minimal de $M$ es el de la Figura~\ref{fig:1.3.36-M-min}.
    \begin{figure}
        \centering
        \begin{tikzpicture}
            \node[state, initial] (q0) {$q_0$};
            \node[state, right of=q0] (q1q4) {$q_1,q_4$};
            \node[state, right of=q1q4, accepting] (q2q5) {$q_2,q_5$};
            \node[state, below right of=q2q5, error] (E) {$E$};
            \node[state, above right of=E] (q3q6) {$q_3,q_6$};
            
            \draw   (q0) edge[above] node{$a,b$} (q1q4)
                    (q1q4) edge[above] node{$a,b$} (q2q5)
                    (q2q5) edge[above] node{$a$} (q3q6)
                    (q2q5) edge[above] node{$b$} (E)
                    (q3q6) edge[above, bend right] node{$a$} (q2q5)
                    (q3q6) edge[above] node{$b$} (E)
                    (E) edge[loop right] node{$a,b$} (E);
        \end{tikzpicture}
        \caption{Autómata minimal asociado a $M$ del Ejercicio~\ref{ej:1.3.36}.}
        \label{fig:1.3.36-M-min}
    \end{figure}
\end{ejercicio}

\begin{ejercicio} \label{ej:1.3.37}
    Si $L_1$ y $L_2$ son lenguajes sobre el alfabeto $A$, entonces \emph{la mezcla perfecta} de estos lenguajes se define como el lenguaje:
    \begin{equation*}
        \{w\mid w=a_1b_1\cdots a_kb_k \mid  a_1\ldots a_k\in L_1, b_1\ldots b_k\in L_2, a_i,b_i\in A\}
    \end{equation*}
    Demostrar que si $L_1$ y $L_2$ son regulares, entonces la mezcla perfecta de $L_1$ y $L_2$ es regular.\\

    Vamos a definir un autómata finito determinista que acepte la mezcla perfecta de $L_1$ y $L_2$.
    Sean los autómatas $M_1=(Q_1,A,\delta_1,q_0^1,F_1)$ el autómata asociado a $L_1$ y $M_2=(Q_2,A,\delta_2,q_0^2,F_2)$ el autómata asociado a $L_2$. Definimos el autómata $M=(Q,A,\delta,q_0,F)$ como sigue:
    \begin{align*}
        Q &= Q_1\times Q_2 \times \{\cc{L}_1,\cc{L}_2\}\\
        F &= F_1\times F_2 \times \{\cc{L}_1\}\\
        q_0 &= (q_0^1,q_0^2,\cc{L}_1)\\
        \delta((q_1,q_2,\cc{L}_1),a) &= (\delta_1(q_1,a),q_2,\cc{L}_2) \qquad \forall a \in A\\
        \delta((q_1,q_2,\cc{L}_2),a) &= (q_1,\delta_2(q_2,a),\cc{L}_1) \qquad \forall a \in A
    \end{align*}
    Notemos que el estado $(q_i, q_j, \cc{L}_k)$ indica que en el autómata $M_1$ se está en el estado $q_i$, en el autómata $M_2$ se está en el estado $q_j$ y, ahora mismo, debemos leer un símbolo de la de $L_k$.\\

    Veamos ahora que $\cc{L}(M)$ genera la mezcla perfecta de $L_1$ y $L_2$, para lo cual antes notamos que, para todo $u\in A^*$, con $|u|=2n$, $(n\in \bb{N})$, tenemos que:
    \begin{equation*}
        \delta^*((q_1,q_2,\cc{L}_1),u)=\left(\delta_1^*(q_1,u_1),\delta_2^*(q_2,u_2),\cc{L}_1\right)
    \end{equation*}
    donde:
    \begin{itemize}
        \item $u_1$ es la subcadena de $u$ formada por los símbolos en posiciones impares.
        \item $u_2$ es la subcadena de $u$ formada por los símbolos en posiciones pares.
    \end{itemize}

    Por tanto, demostremos el resultado por doble inclusión:
    \begin{description}
        \item[$\subseteq$)] Sea $u\in \cc{L}(M)$. Entonces, $\delta((q_0^1,q_0^2,\cc{L}_1),u)\in F$, de lo que deducimos que $|u=2n|$, con $n\in \bb{N}$. Por tanto, definiendo $u_1$ y $u_2$ como antes, tenemos que:
        \begin{align*}
            \delta^*((q_0^1,q_0^2,\cc{L}_1),u) &= \left(\delta_1^*(q_0^1,u_1),\delta_2^*(q_0^2,u_2),\cc{L}_1\right) \in F = F_1\times F_2 \times \{\cc{L}_1\}
        \end{align*}

        Por tanto, $u_1\in L_1$, $u_2\in L_2$ y, por tanto, $u$ pertenece a la mezcla perfecta de $L_1$ y $L_2$.

        \item[$\supseteq$)] Sea $u\in A^*$ tal que $u$ pertenece a la mezcla perfecta de $L_1$ y $L_2$. Entonces, $u=a_1b_1\cdots a_kb_k$, con $u_1=a_1\ldots a_k\in L_1$ y $u_2=b_1\ldots b_k\in L_2$. Por tanto, como $|u|=2k$, con $k\in \bb{N}$, tenemos que:
        \begin{equation*}
            \delta^*((q_0^1,q_0^2,\cc{L}_1),u) = \left(\delta_1^*(q_0^1,u_1),\delta_2^*(q_0^2,u_2),\cc{L}_1\right) \in F= F_1\times F_2 \times \{\cc{L}_1\}
        \end{equation*}
        Por tanto, $u\in \cc{L}(M)$.
    \end{description}   
    
    Por tanto, el autómata $M$ acepta la mezcla perfecta de $L_1$ y $L_2$, por lo que es regular.
\end{ejercicio}

\begin{ejercicio}\label{ej:1.3.38}
    Si $L$ es un lenguaje, sea $L_{\nicefrac{1}{2}}$ el conjunto de palabras que son las mitades de palabras de $L$ y $L_{\nicefrac{-1}{3}}$ el conjunto de palabras que son las dos terceras partes de palabras de $L$. Es decir:
    \begin{align*}
        L_{\nicefrac{1}{2}} &= \{x\mid \exists y\in A^\ast, |x|=|y|, xy\in L\} \\
        L_{\nicefrac{-1}{3}} &= \{xz\mid \exists y\in A^\ast, |x|=|y|=|z|, xyz\in L\}
    \end{align*}
    Demostrar que si $L$ es regular, entonces $L_{\nicefrac{1}{2}}$ también lo es, pero que $L_{\nicefrac{-1}{3}}$ no es necesariamente regular.\\

    Como $L$ es regular, sea $M=(Q,A,\delta,q_0,F)$ el autómata finito determinista asociado a $L$. Definimos el autómata $M'= (Q,A,\delta,q_0,F')$ donde:
    \begin{equation*}
        F' = \{q\in Q\mid \exists u,v\in A^*, |u|=|v|, \delta^*(q_0,u)=q, \delta^*(q,v)\in F\}
    \end{equation*}
    Es decir, $F'$ contiene los estados que se encuentran a mitad de camino entre el estado inicial y un estado final. Veamos que $L_{\nicefrac{1}{2}} = \cc{L}(M')$:
    \begin{description}
        \item[$\subseteq$)] Sea $u\in L_{\nicefrac{1}{2}}$. Entonces, existe $v\in A^*$ tal que $|u|=|v|$ y $uv\in L$. Entonces:
        \begin{equation*}
            \delta^*(q_0,uv) = \delta^*(\delta^*(q_0,u),v) \in F
            \Longrightarrow
            \delta^*(q_0,u)\in F'
        \end{equation*}
        Por tanto, $u\in \cc{L}(M')$.

        \item[$\supseteq$)] Sea $u\in \cc{L}(M')$. Entonces, $\delta^*(q_0,u)\in F'$. Por tanto, existe $v\in A^*$ tal que $|u|=|v|$ y:
        \begin{equation*}
            \delta^*(\delta^*(q_0,u),v) = \delta^*(q_0,uv) \in F
            \Longrightarrow uv\in \cc{L}(M)
            \Longrightarrow u\in L_{\nicefrac{1}{2}}
        \end{equation*}
    \end{description}

    Por tanto, $L_{\nicefrac{1}{2}}=\cc{L}(M')$, con lo que es regular.

    % // TODO: Demostrar que L_{-1/3} no es regular.
\end{ejercicio}

\subsection{Preguntas Tipo Test}
Se pide discutir la veracidad o falsedad de las siguientes afirmaciones:
\begin{enumerate}
    \item El lema de bombeo puede usarse para demostrar que un lenguaje determinado es regular.\\

        Falso, el lema de bombeo nos dice que si un lenguajes es regular, entonces este cumple una determinada propiedad. Podemos usar su contrarrecíproco para ver que si una palabra del lenguaje no cumple dicha propiedad entonces el lenguaje no es regular, pero no nos sirve para determinar si un lenguaje lo es o no.
    \item Todo lenguaje con un número finito de palabras es regular.\\

        Verdadero, ya que si tenemos $L=\{v_1,v_2,\ldots,v_n\}$ un lenguaje finito de $n\in \mathbb{N}$ palabras, entonces podemos construir la gramática $G=(\{S\}, A, P, S)$ con $A$ el alfabeto sobre el que está definido $L$ y $P$ el siguiente conjunto de producciones:
        \begin{equation*}
            P = \{S \rightarrow v_1\ |\ v_2\ |\ \ldots \ |\ v_n\}
        \end{equation*}
        Con lo que $G$ es una gramática regular de forma que $L = L(G)$, con lo que es regular.
    \item La intersección de lenguajes regulares es siempre regular.\\

        Verdadero. Como hemos visto en teoría, si tenemos dos lenguajes regulares, entonces podemos construir un autómata finito determinista para cada uno de ellos y construir el autómata producto, que genera la intersección de ambos lenguajes, con lo que el lenguaje como resultado de intersecar los dos lenguajes es regular.
    \item La demostración del lema de bombeo se basa en que si leemos una palabra de longitud mayor o igual al número de estados del autómata, entonces en el camino que se recorre en el diagrama de transición se produce un ciclo.\\

        Verdadero, si tenemos un autómata finito determinista de $n$ estados que reconoce un lenguaje regular, si leemos una palabra de longitud $m$ con $m\geq n$, entonces en el ``camino de lectura'' de la palabra, hemos de pasar por $m+1$ estados, con lo que pasaremos por al menos un estado dos veces o más, con lo que el autómata tendrá un ciclo.
    \item Es más fácil determinar si una palabra pertenece a un lenguaje regular cuando éste viene dado por una expresión regular que cuando viene dado por un autómata finito determinista.\\

        Falso, si tenemos un autómata finito determinista que acepta el lenguaje en cuestión, ver si la palabra está en el lenguaje será tan sencillo como ir realizando las transiciones entre estados en el autómata leyendo la palabra y comprobando si al final llegamos a un estado final o si no. Por otra parte, puede suceder que ver si una palabra está en un lenguaje o no a partir de la expresión regular puede que sea sencillo, pero si consideramos una expresión regular muy compleja, posiblemente no sea tan fácil determinar si la palabra está en el lenguaje o no. En cualquier caso, reconocer si una palabra está en el lenguaje por un autómata finito determinista es siempre el proceso más sencillo.
    \item En la demostración de que todo autómata finito tiene una expresión regular que representa el mismo lenguaje, el conjunto $R^k_{ij}$ se define como el lenguaje de todas las palabras que llevan al autómata del estado $q_i$ al estado $q_j$ pasando por el estado número $k$, $q_k$.\\

        Falso, (se trata de una pregunta del Tema 2) ya que $R^k_{ij}$ se define como el conjunto de todas las palabras que llevan al autómata del estado $q_i$ al estado $q_j$ pasando por estados de numeración menor o igual que $k$, pero no necesariamente por $q_k$.
    \item El conjunto de todas las expresiones regulares es un lenguaje regular.\\

        Falso, si consideramos que la buena parentización\footnote{El poner paréntesis.} forma parte de las expresiones regulares, podemos demostrar que el conjunto de todas las expresiones regulares no es un lenguaje regular usando el Lema de bombeo:

        Sea $n\in \mathbb{N}$ y $a\in A$, consideramos\footnote{Donde hemos usado corchetes para diferenciarlos de los paréntesis del lenguaje.} $z = (^n a [+ a)]^n$. Es decir:
        \begin{equation*}
            z = (((\ldots(((a+a)+a)+a)\ldots +a) +a) +a)
        \end{equation*}
        donde hay $n$ paréntesis de apertura y $n$ paréntesis de cierre y notemos que $|z|\geq n$. Supongamos que hay $u,v,w \in A^\ast$ tales que $z = uvw$ con $|v|\geq 1$ y $|uv|\leq n$. Entonces, $u = (^k$, $v = (^h$ con $h\geq 1$, $k+h\leq n$ y:
        \begin{equation*}
            w = (^{n-k-h}a+a)+a)+a)\ldots +a) +a) +a) = (^{n-k-h}a[+a)]^n
        \end{equation*}
        Sin embargo, $uv^0w = uw = (^{n-h}a[+a)]^n$, que no es una expresión regular correcta por no estar bien parentizada ($k\geq 1$), con lo que dicho lenguaje no es regular.

    \item A partir de la demostración de que si $R$ es regular y $L$ un lenguaje cualquiera, entonces $R/L$ es regular, se puede obtener un algoritmo para construir el autómata asociado a $R/L$.\\

        Falso en general: si el lenguaje $L$ es finito o regular, sí que nos dice cómo podemos construir el autómata asociado a $R/L$, pero si no puede suceder que no seamos capaces de hacerlo, por tener $L$ infinitas palabras y no poder considerar en el autómata todas ellas.
    \item En un autómata finito no-determinista, si intercambio entre sí los estados finales y no finales obtengo un autómata que acepta el lenguaje complementario.\\

        Falso, ya que un autómata finito no determinista puede que no tenga definidas alguna transición desde un estado leyendo algún caracter (lo que en el determinista asociado significaría ir a un estado de error), con lo que al realizar dicho cambio en el autómata no determinista, no consideramos dichas transiciones. Para ver esto más claro, consideramos el autómata finito no determinista de la Figura~\ref{fig:tipo_test9}, que acepta el lenguaje:
        \begin{equation*}
            L = \{01u \mid u \in {\{0,1\}}^{\ast}\}
        \end{equation*}

        \begin{figure}[H]
           \centering
           \begin{tikzpicture}
               \node[state, initial] (q0) {$q_0$};
               \node[state, right of=q0] (q1) {$q_1$};
               \node[state, accepting, right of=q1] (q2) {$q_2$};

               \draw   (q0) edge[above] node{0} (q1)
                       (q1) edge[above] node{1} (q2)
                       (q2) edge[loop right] node{0,1} (q2);
           \end{tikzpicture}
           \caption{Autómata finito no determinista para la pregunta 9.}
           \label{fig:tipo_test9}
       \end{figure}
       Si realizamos el cambio mencionado en la pregunta, entonces consideramos ahora como conjunto de estados finales $F' = \{q_0,q_1\}$, con lo que según esta pregunta, el nuevo autómata debería reconocer el lenguaje: 
       \begin{equation*}
           \overline{L}= \{w \mid w \neq 01u \quad \forall u\in {\{0,1\}}^{\ast}\}
       \end{equation*}
       Sin embargo, $110 \in \overline{L}$ y si intentamos leer esta palabra en el nuevo autómata finito no determinista, no la reconoce, como resultado de la falta de transiciones en el autómata de la Figura~\ref{fig:tipo_test9}, como habíamos enunciado anteriormente.

    \item Si en un autómata finito no hay estados distinguibles de nivel 2, ya no puede haber estados distinguibles de nivel 4.\\

        % // TODO: Hacer JJ (no lo sé)
    \item Todo lenguaje generado por una gramática lineal por la derecha es también generado por una gramática lineal por la izquierda.\\
        
        Verdadero, se vió en el Tema 2.
    \item Un autómata finito determinista sin estados inaccesibles ni indistinguibles es minimal.\\

        Verdadero, gracias a un resultado visto en teoría.
    \item Si $L$ es una lenguaje sobre el alfabeto $A$, entonces $\operatorname{CAB}(L)$ es siempre igual al cociente $L/A^\ast$.\\

        Verdadero, por definición de $\operatorname{CAB}(L)$ y de $L/A^\ast$:
        \begin{equation*}
            L/A^\ast = \{u\in A^\ast \mid \exists v\in A^\ast \text{\ verificando\ } uv \in L\} = \operatorname{CAB}(L)
        \end{equation*}
    \item El lenguaje de las palabras sobre $\{0,1\}$ en las que la diferencia entre el número de ceros y unos es impar es regular.\\

        Verdadero, ya que se puede reconocer por el autómata de la Figura~\ref{fig:tipo_test14}.
        \begin{figure}[H]
           \centering
           \begin{tikzpicture}
               \node[state, initial] (q0) {$q_0$};
               \node[state, accepting, right of=q0] (q1) {$q_1$};

               \draw   (q0) edge[above, bend left] node{0,1} (q1)
                       (q1) edge[below, bend left] node{0,1} (q0);
           \end{tikzpicture}
           \caption{Autómata finito determinista para la pregunta 14.}
           \label{fig:tipo_test14}
       \end{figure}
       Donde usamos el estado $q_0$ para representar que la diferencia entre el número de ceros y unos es par y $q_1$ para representar que la diferencia entre el número de ceros y unos es impar.
    \item En un autómata finito cualquiera, si las transiciones dan lugar a un ciclo, entonces el lenguaje aceptado es infinito.\\

        Falso, solo será cierto si tras salir de dicho ciclo se puede llegar a un estado final. Para ilustrar este caso, observamos el autómata finito determinista de la Figura~\ref{fig:tipo_test15}.
        \begin{figure}[H]
           \centering
           \begin{tikzpicture}
               \node[state, initial] (q0) {$q_0$};
               \node[state, accepting, right of=q0] (q1) {$q_1$};
               \node[state, below of=q0] (q4) {$q_4$};
               \node[state, right of=q4] (q5) {$q_5$};

               \draw   (q0) edge[above] node{0} (q1)
                       (q1) edge[left] node{0,1} (q5)
                       (q0) edge[left] node{1} (q4)
                       (q4) edge[above, bend left] node{0,1} (q5)
                       (q5) edge[below, bend left] node{0,1} (q4);
           \end{tikzpicture}
           \caption{Autómata finito determinista para la pregunta 15.}
           \label{fig:tipo_test15}
       \end{figure}
       En el autómata hay presente un ciclo y este reconoce el lenguaje $\{0\}$.
    \item La expresión recursiva que se emplea para obtener la expresión regular asociada a un autómata finito determinista es: $r^k_{ij} = r^{k-1}_{ij} + r^{k-1}_{i(k-1)}{(r^{k-1}_{(k-1)(k-1)})}^{\ast}r^{k-1}_{(k-1)j}$.\\

        Falso, aunque se trata de una pregunta del Tema 2, la expresión correcta es:
        \begin{equation*}
            r_{ij}^k = r_{ij}^{k-1} + r_{ik}^{k-1}{(r_{kk}^{k-1})}^{\ast}r_{kj}^{k-1}
        \end{equation*}
    \item Cuando se construye la expresión regular asociada a un autómata finito determinista, $r^0_{ii}$ no puede ser nunca vacío.\\

        Falso, sí puede serlo.
    \item El conjunto de las palabras $\{u0011v^{-1} \mid u,v\in {\{0,1\}}^{\ast}\}$ es regular.\\

        Verdadero, para verlo más claro (usando que puede $u\neq v$):
        \begin{equation*}
            \{u0011v^{-1}\mid u,v\in {\{0,1\}}^{\ast}\} = \{u0011w \mid u,w\in {\{0,1\}}^{\ast}\}
        \end{equation*}
        Y podemos reconocer este lenguaje mediante la gramática lineal por la derecha $G=(\{S,A\},\{0,1\},P,S)$ con $P$ el conjunto que contiene las siguientes producciones:
        \begin{align*}
            S &\rightarrow 0S\ |\ 1S\ |\ 0011A \\
            A &\rightarrow 0A\ |\ 1A\ |\ \veps
        \end{align*}
    \item Si $L$ es un lenguaje finito, entonces su complementario es siempre regular.\\

        Verdadero, ya que si $L$ es finito, entonces es regular, con lo que podemos construir un autómata finito determinista que reconozca dicho lenguaje. Una vez que obtengamos dicho autómata finito \textbf{determinista} $M=(Q,A,\delta,q_0,F)$, bastará considerar el autómata $M'=(Q,A,\delta,q_0,Q\setminus F)$, autómata finito determinista que aceptará el lenguaje $L(M') = \overline{L}$.
    \item En un autómata finito determinista la relación de indistinguibilidad es una relación de equivalencia.\\

        Verdadero, cumple las propiedades reflexiva, simétrica y transitiva, tal y como se ha visto en teoría.
    \item En un autómata finito determinista siempre debe de existir, al menos, un estado de error.\\

        Falso, el autómata que hicimos para la pregunta 14 que podemos ver en la Figura~\ref{fig:tipo_test14} era un autómata finito determinista totalmente válido y no tenía estados de error.

    \item El conjunto de los números en binario que son múltiplos de 7 es regular.\\

        Verdadero, como vimos en el Ejercicio~\ref{ej:1.1.17}.\ref{ej:1.1.17.b} y en el AFD de la Figura~\ref{fig:1.1.17.b}.
        
        \begin{comment}
        ya que podemos constuir un autómata finito determinista que reconozca dicho lenguaje, como vemos en la Figura~\ref{fig:tipo_test22}, donde hemos hecho uso de la tabla de estados~\ref{tab:estados}:
        \begin{table}[H]
        \centering
        \begin{tabular}{c|c|c|c|c|c|c}
            $7n$ & $7n+1$ & $7n+2$ & $7n+3$ & $7n+4$ & $7n+5$ & $7n+6$ \\
            \hline
            0 & 1 & 2 & 3 & 4 & 5 & 6  
        \end{tabular}
        \caption{Estados en relación a si el número es múltiplo de 7 o si no.}
        \label{tab:estados}
        \end{table}
        Y de las propiedades de los números binarios:
        \begin{itemize}
            \item Leer un 0 es equivalente a multiplicar el número por 2.
            \item Leer un 1 es equivalente a multiplicar el número por 2 y sumarle 1.
        \end{itemize}
        De esta forma, conseguimos el autómata finito determinista anteriormente mencionado, razonando los pasos entre estados.
        \begin{figure}[H]
           \centering
            \begin{tikzpicture}
                \node[state,initial,accepting] (q0) {$q_0$};
                \node[state] (q1) [right of=q0] {$q_1$};
                \node[state] (q2) [above right of=q1] {$q_2$};
                \node[state] (q3) [below right of=q1] {$q_3$};
                \node[state] (q4) [right of=q2] {$q_4$};
                \node[state] (q5) [right of=q3] {$q_5$};
                \node[state] (q6) [right of=q5] {$q_6$};

                \draw   (q0) edge[loop above] node {0} (q0)
                        (q0) edge[above] node {1} (q1)
                        (q1) edge[above] node {0} (q2)
                        (q1) edge[above] node {1} (q3)
                        (q2) edge[above] node {0} (q4)
                        (q2) edge[left, bend left] node[pos=0.8] {1} (q5)
                        (q3) edge[below, bend right] node {0} (q6)
                        (q3) edge[above] node {1} (q0)
                        (q4) edge[above, bend left] node {0} (q1)
                        (q4) edge[above, bend right] node {1} (q2)
                        (q5) edge[above] node {0} (q3)
                        (q5) edge[right] node {1} (q4)
                        (q6) edge[above] node {0} (q5)
                        (q6) edge[loop above] node {1} (q6);
            \end{tikzpicture}
           \caption{Autómata finito determinista para la pregunta 22.}
           \label{fig:tipo_test22}
       \end{figure}
       \end{comment}
    \item Hay situaciones en las que los estados inaccesibles de un AFD cumplen una función específica.\\

        Falso, no se puede llegar nunca a un estado inaccesible, por lo que son irrelevantes en el reconocimiento de una palabra.
    \item Si $R$ es un lenguaje regular y $L$ un lenguaje independiente del contexto, entonces $R/L$ es regular.\\

        Verdadero, ya que no hace falta exigir hipótesis sobre $L$, sea cual sea dicho lenguaje, mientras que $R$ sea regular, $R/L$ será regular.
    \item Si en un autómata dos estados son distinguibles de nivel $n$, entonces serán distinguibles de nivel $m$ para todo $m\geq n$.\\

        Verdadero, ya que dos estados son distinguibles de nivel $n$ si y solo si existe una palabra $u\in A^\ast$ de longitud menor o igual que $n$ tal que en el conjunto $\{\delta^\ast(p,u),\delta^\ast(q,u)\}$ hay un estado final y otro no final, por lo que si $p$ y $q$ son distinguibles de nivel $n$ por la dicha existencia de una palabra $u\in A^\ast$, entonces serán indistinguibles de nivel $m$ para $m\geq n$, ya que podemos considerar la misma palabra que considerábamos para el caso de $n$, por ser $|u|\leq n\leq m$.
    \item Si $h$ es un homomorfismo y $h(L)$ no es regular, podemos concluir que $L$ no es regular.\\

        Verdadero, es la implicación contrarrecíproca de ``si $h$ es un homomorfismo y $L$ es regular, entonces $h(L)$ es regular'', vista en teoría.
    \item El lenguaje de todas las palabras en las que los tres primeros símbolos son iguales a los tres últimos es regular.\\

        Verdadero, supongamos que trabajamos sobre el alfabeto $A = \{a_1,\ldots,a_n\}$. Sabemos que hay un número finito de combinaciones para coger tres símbolos de dicho lenguaje: $n^3$ combinaciones distintas. Podemos pues, construir una aplicación biyectiva $f:\{1,2,\ldots,n^3\}\rightarrow A\times A\times A$. De esta forma, podemos considerar la gramática:
        \begin{equation*}
            G = (\{S\}\cup \{A_i\}_{i \in \{1,2,\ldots,n^3\}}, A, P, S)
        \end{equation*}
        Siendo $P$ el conjunto de producciones que contienen las siguientes producciones que vamos a describir.

        $S$ va a poder generar cada una de las $n^3$ sucesiones de 3 símbolos sobre $A$, cada una seguida de una variable $A_i$ siendo $i$ el índice de dicha combinación. Posteriormente, todas las variables $A_i$ podrán generar un número indefinido de ceros y unos en cualquier orden, teniendo que terminar con la combinación de los 3 símbolos correspondiente al índice $i$:
        \begin{align*}
            S &\rightarrow f(1)A_1\ |\ f(2)A_2\ |\ \ldots\ |\ f(n^3)A_{n^3} \\
            A_1 &\rightarrow 0A_1\ |\ 1A_1\ |\ f(1) \\
            A_2 &\rightarrow 0A_2\ |\ 1A_2\ |\ f(2) \\
                &\vdots \\
            A_i &\rightarrow 0A_i\ |\ 1A_i\ |\ f(i) \\
                &\vdots \\
            A_{n^3} &\rightarrow 0A_{n^3}\ |\ 1A_{n^3}\ |\ f(n^3)
        \end{align*}
        Y tenemos que $G$ es una gramática regular por la derecha, por lo que el lenguaje que genera es regular.
    \item Si un lenguaje verifica la condición que aparece en el lema de bombeo para lenguajes regulares, ya no hay forma de demostrar que no es regular.\\

        Falso, el lenguaje puede verificar la condición del lema de bombeo y no ser regular. Sin embargo, puede ser posible demostrar que no es regular por otros métodos.
    \item Si $f$ es un homomorfismo entre alfabetos $f:A_1^\ast\rightarrow A_2^\ast$ y $L\subseteq A_1^\ast$ no es regular, podemos concluir que $f(L)$ tampoco es regular.\\

        % // TODO: this
    \item Todo lenguaje que cumple la condición del lema de bombeo para lenguajes regulares puede ser aceptado por un autómata finito no determinista.\\

        Falso, anteriormente comentamos que un lenguaje puede verificar la condición del lema de bombeo y no ser regular.
    \item No existe algoritmo para saber si el lenguaje generado por una gramática regular es finito.\\
    
        Falso, sí existe. En primer lugar, hemos de eliminar los estados inaccesibles y los estados desde los que no se puede llegar a un estado final (estados de error). Una vez hecho esto, si el autómata finito determinista resultante tiene ciclos, entonces el lenguaje es infinito, ya que podemos ir dando vueltas por el ciclo y generando palabras infinitas. Si no hay ciclos, entonces el lenguaje es finito.

        % // TODO: Hecha por Arturo, Revisar
    \item Dos autómatas finitos deterministas con diferente número de estados y que aceptan el lenguaje vacío tienen el mismo número de estados finales.\\

        Falso, solo sería cierto si todos los estados son alcanzables, ya que los autómatas de las Figuras~\ref{fig:tipo_test32_1} y~\ref{fig:tipo_test32_2} ambos aceptan el lenguaje vacío y el número de estados finales es distinto.
        \begin{figure}[H]
           \centering
            \begin{tikzpicture}
                \node[state,initial] (q0) {$q_0$};
                \draw   (q0) edge[loop right] node {0,1} (q0);
            \end{tikzpicture}
           \caption{Autómata finito determinista 1 para la pregunta 32.}
           \label{fig:tipo_test32_1}
       \end{figure}

       \begin{figure}[H]
           \centering
            \begin{tikzpicture}
                \node[state,initial] (q0) {$q_0$};
                \node[state,accepting] (q1) [right of=q0] {$q_1$};
                \draw   (q0) edge[loop right] node {0,1} (q0)
                        (q1) edge[loop right] node {0,1} (q1);
            \end{tikzpicture}
           \caption{Autómata finito determinista 2 para la pregunta 32.}
           \label{fig:tipo_test32_2}
       \end{figure}
    \item Si $A$ es un alfabeto y $L$ un lenguaje cualquiera distinto del vacío, entonces se verifica que $A^\ast / L = A^\ast$.\\

        Verdadero, si recordamos la definición de $A^\ast / L$:
        \begin{equation*}
            A^\ast / L = \{u\in A^\ast \mid \exists v\in L \text{\ verificando\ } uv\in A^\ast\}
        \end{equation*}
        \begin{description}
            \item [$\subseteq$)] Es trivial, por ser $A^\ast / L$ un lenguaje.
            \item [$\supseteq$)] Sea $u\in A^\ast$, como $L\neq \emptyset $, existirá $v\in L$, con lo que $uv\in A^\ast$ por ser el conjunto de todas las palabras, con lo que $u\in A^\ast/L$.
        \end{description}
    \item Si $R^k_{ij}$ son los lenguajes que se usan en la construcción de una expresión regular a partir de un autómata finito, siempre se verifica que $R^{i-1}_{ij}R^{j-1}_{jk}\subseteq R^{j}_{ik}$.\\

        % // TODO: Hacer
    \item El lema de bombeo es útil para demostrar que la intersección de dos lenguajes regulares no es regular.\\

        % // TODO: Hacer
    \item Existe un algoritmo para determinar si el lenguaje generado por una gramática regular es infinito.\\

        % // TODO: Hacer
    \item Existe un algoritmo para determinar si el lenguaje generado por una gramática regular es finito o infinito.\\

        % // TODO: Hacer
    \item La intersección de dos lenguajes regulares da lugar a un lenguaje independiente del contexto.\\

        Verdadero, ya que la intersección de dos lenguajes regulares da lugar a un lenguaje regular (tal y como veíamos en la pregunta 3), que a su vez es independiente del contexto.
    \item Si un lenguaje es infinito no se puede encontrar una expresión regular que lo represente.\\

        Falso, debido a la existencia de lenguajes regulares infinitos. Por ejemplo, podemos considerar $L = \{a^i \mid i \in \mathbb{N}\}$, un lenguaje infinito por ser biyectivo con $\mathbb{N}$:
        \begin{equation*}
            L = \{\veps, a, aa, aaa, aaaa, \ldots\}
        \end{equation*}
        Y este puede ser representado por la expresión regular ${(a)}^{\ast}$.
    \item En un autómata finito determinista sin estados inaccesibles la relación de indistiguibilidad entre los estados es una relación de equivalencia.\\

        Verdadero, tal y como vimos en teoría.
    \item En un autómata finito determinista, si no hay dos estados que sean indistinguibles entre sí, entonces el autómata es minimal.\\

        Falso, salvo si el autómata no tiene estados inalcanzables, en cuyo caso es verdadero.
    \item Dada una gramática lineal por la derecha, siempre existe otra gramática lineal por la izquierda que acepte el mismo lenguaje.\\

        Verdadero, tal y como vimos en la teoría del Tema 2.
    \item Si $R$ es un lenguaje regular y $L$ un lenguaje cualquiera, entonces $R/L$ es siempre un lenguaje regular.\\

        Verdadero, tal y como hemos visto en la teoría.
    \item Si un lenguaje cumple la condición del lema de bombeo para conjuntos regulares no nos asegura que sea un lenguaje regular.\\

        Verdadero, ya que la propiedad del lema de bombeo es una condición necesaria, no suficiente.
    \item Existe un algoritmo para determinar si los lenguajes generados por dos gramáticas regulares son iguales o no.\\

        Verdadero. En teoría hemos visto que si tenemos dos lenguajes, $L_1$ y $L_2$ generados por dos autómatas finitos de terministas, $M_1$ y $M_2$ (respectivamente), entonces podemos construir el autómata finito para el lenguaje diferencia simétrica $L_1 \ \Delta \ L_2=(L_1\setminus L_2)\cup (L_2\setminus L_1)$, y aplicarle el algoritmo de si el lenguaje que genera es vacío (en cuyo caso, $L_1=L_2$). Además, como tenemos un algoritmo para pasar de gramáticas regulares a autómatas, podemos mecanizar todo este proceso.
    \item El conjunto de cadenas aceptado por un autómata finito no determinista con transiciones nulas no puede ser generado por una gramática independiente del contexto.\\

        Falso, el conjunto de cadenas acpetadas por un autómata finito no determinista con transiciones nulas es un lenguaje regular, que puede ser generado por una gramática regular por la derecha, que a su vez es independiente del contexto.
    \item El lenguaje resultado de la unión de dos lenguajes regulares con un número infinito de palabras puede ser representado mediante una expresión regular.\\

        Verdadero, ya que la unión de dos lenguajes regulares es regular, con lo que puede ser representado mediante una expresión regular, independientemente de la cardinalidad de los lenguajes.
    \item Una expresión regular siempre representa a un lenguaje que puede ser generado por una gramática independiente del contexto.\\

        Verdadero, ya que una expresión regular siempre representa a un lenguaje que puede ser generado por una gramática regular por la derecha, que a su vez es independiente del contexto.
    \item Existe un algoritmo para comprobar si son iguales los lenguajes aceptados por dos autómatas finitos diferentes.\\

        Verdadero, en caso de ser autómatas finitos determinisitas, es el razonamiento que ya hicimos en la pregunta 45. En caso de ser autómatas finitos no deterministas, existe un algoritmo para pasarlos a deterministas y en cuyo caso, podemos aplicar el algoritmo ya mencionado.
    \item Si en un autómata finito no determinista intercambio entre sí los estados finales y no finales obtengo un autómata que acepta el lenguaje complementario del aceptado por el autómata original.\\

        Falso, esta pregunta ya fue respondida anteriormente, en la pregunta 9.
    \item Si $L$ es un lenguaje regular, entonces el lenguaje $LL^{-1}$ es también regular.\\

        Verdadero, ya que si $L$ es regular, entonces $L^{-1}$ es regular; y la concatenación de lenguajes regulares sigue siendo regular.
    \item El lema de bombeo para lenguajes regulares es útil para demostrar que un lenguaje determinado no es regular.\\

        Verdadero, puede ser útil para demostrar que un lenguaje no es regular, aunque puede haber casos en los que no nos dé información sobre si es regular o no, al tratarse de una condición necesaria para los lenguajes regulares.
    \item Si un lenguaje tiene un conjunto infinito de palabras sabemos que no es regular.\\

        Falso, el lenguaje $L=\{a^i \mid i \in \mathbb{N}\}$ tiene un conjunto infinito de palabras (es biyectivo con $\mathbb{N}$) y es regular, ya que puede representarse mediante la expresión regular ${(a)}^{\ast}$.
    \item Un autómata finito determinista sin estados inaccesibles ni indistinguibles es minimal.\\

        Verdadero, tal y como comentamos anteriormente en la pregunta 12.
    \item El conjunto de las palabras $\{u0011v^{-1}\mid u,v\in {\{0,1\}}^{\ast}\}$ es regular.\\

        Verdadero, tal y como comentamos anteriormente en la pregunta 18.
    \item Existe un algoritmo para determinar si el lenguaje generado por una gramática regular es infinito.\\

        Verdadero, como razonamos en la pregunta 31.
    \item Para cada autómata finito no determinista $M$ existe una gramática independiente de contexto $G$ tal que $\cc{L}(M) = \cc{L}(G)$.\\

        Verdadero, ya que sabemos que para cada autómata finito no determinista $M$ existe una gramática regular por la derecha $G$ tal que $\cc{L}(M) = \cc{L}(G)$ y sabemos que las gramáticas regulares por la derecha son a su vez independientes del contexto.
    \item El lenguaje formado por las cadenas sobre $\{0,1\}$ que tienen un número impar de 0 y un número par de 1 no es regular.\\

        Falso, ya que podemos construir un autómata finito determinista que acepte el lenguaje, tal y como vemos en la Figura~\ref{fig:tipo_test58}.
       \begin{figure}[H]
           \centering
            \begin{tikzpicture}
                \node[state, initial] (q0) {$q_0$};
                \node[state, accepting] (q1) [right of=q0] {$q_1$};
                \node[state] (q2) [below of=q0] {$q_2$};
                \node[state] (q3) [right of=q2] {$q_3$};
                \draw   (q0) edge[above, bend left] node {0} (q1)
                        (q1) edge[above, bend left] node {0} (q0)
                        (q0) edge[right, bend left] node {1} (q2)
                        (q2) edge[left, bend left] node {1} (q0)
                        (q1) edge[right, bend left] node {1} (q3)
                        (q3) edge[left, bend left] node {1} (q1)
                        (q2) edge[above, bend left] node {0} (q3)
                        (q3) edge[above, bend left] node {0} (q2);
            \end{tikzpicture}
           \caption{Autómata finito determinista para la pregunta 58.}
           \label{fig:tipo_test58}
       \end{figure}
       Donde pensamos en los estados como:
       \begin{itemize}
           \item $q_0$, la palabra tiene un número par de ceros y de unos.
           \item $q_1$, la palabra tiene un número impar de ceros y número par de unos.
           \item $q_2$, la palabra tiene un número par de ceros y número impar de unos.
           \item $q_3$, la palabra tiene un número impar de ceros y de unos.
       \end{itemize}
       Elegimos como estado inicial $q_0$ ya que la palabra $\veps$ tiene un número par de ceros y de unos.
    \item Si $L$ es un lenguaje regular, entonces la cabecera de $L$ ($\operatorname{CAB}(L)$) es siempre regular.\\

        Verdadero, si $L$ es un lenguaje regular, existirá un autómata finito determinista $M=(Q,A,\delta,q_0,F)$ que reconozca dicho lenguaje. Consideramos ahora el autómata $M'=(Q,A,\delta,q_0,F')$, un autómata igual que $M$ con la diferencia de que $q\in F'$ si y solo si existen dos palabras $u,v\in A^\ast$ tales que:
        \begin{equation*}
            \delta^\ast(q_0,u) = q \quad \land \quad \delta^\ast(q,v) \in F
        \end{equation*}
        Es decir, si $q$ es un estado alcanzable y se encuentra en el camino previo a un estado final, de forma que la palabra que se lee desde $q_0$ hasta $q$ será un prefijo de alguna palabra de $L$.
    \item En un autómata finito determinista, si no hay dos estados que sean indistinguibles entre si, entonces el autómata es minimal.\\

        Falso, es necesario exigir además que no haya estados inalcanzables.
    \item La intersección de dos lenguajes regulares da lugar a un lenguaje independiente del contexto.\\

        Verdadero, ya que la intersección de dos lenguajes regulares da lugar a un lenguaje regular, que a su vez es independiente del contexto.
    \item Si un lenguaje es infinito no se puede encontrar una expresión regular que lo represente.\\

        Falso, pregunta que ya fue contestada en la pregunta 39.
\end{enumerate}
