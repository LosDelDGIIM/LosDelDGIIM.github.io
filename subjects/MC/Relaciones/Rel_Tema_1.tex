\section{Introducción a la Computación}


\begin{ejercicio}
    Sea la gramática $G=\left(V,T,P,S\right)$ dada por:
    \begin{align*}
        V &= \{S, X, Y\} \\
        T &= \{a,b\} \\
        S &= S
    \end{align*}
    \begin{enumerate}
        \item Describe el lenguaje generado por la gramática teniendo en cuenta que $P$ viene descrito por:
        \begin{align*}
            S &\rightarrow XYX \\
            X &\rightarrow aX \mid bX \mid \veps \\
            Y &\rightarrow bbb
        \end{align*}

        Sea $L=\{ubbbv\mid u,v\in\{a,b\}^\ast\}$. Demostraremos mediante doble inclusión que $L=\cc{L}(G)$.
        \begin{description}
            \item[$\subset)$] Sea $w\in L$. Entonces, $w=ubbbv$ con $u,v\in\{a,b\}^\ast$. Veamos que
            $S \stackrel{\ast}{\Longrightarrow} w$:
            \begin{equation*}
                S \Longrightarrow XYX \Longrightarrow XbbbX
            \end{equation*}

            Además, es fácil ver que la regla de producción $X\rightarrow aX \mid bX \mid \veps$ nos permite generar cualquier palabra $u\in\{a,b\}^\ast$. Por tanto, tenemos que $X \stackrel{\ast}{\Longrightarrow} u$ y $X \stackrel{\ast}{\Longrightarrow} v$; teniendo así que $S \stackrel{\ast}{\Longrightarrow} ubbbv$.

            \item[$\supset)$] Sea $w\in\cc{L}(G)$. Veamos la forma de $w$:
            \begin{equation*}
                S \Longrightarrow XYX \Longrightarrow XbbbX \Longrightarrow ubbbv \mid u,v\in\{a,b\}^\ast
            \end{equation*}
            donde en el último paso hemos empleado lo visto en el apartado anterior de la regla de producción $X\rightarrow aX \mid bX \mid \veps$. Por tanto, $w\in L$.
        \end{description}


        \item Describe el lenguaje generado por la gramática teniendo en cuenta que $P$ viene descrito por:
        \begin{align*}
            S &\rightarrow aX \\
            X &\rightarrow aX \mid bX \mid \veps
        \end{align*}

        Sea $L=\{au \mid u\in\{a,b\}^\ast\}$. Demostraremos mediante doble inclusión que $L=\cc{L}(G)$.
        \begin{description}
            \item[$\subset)$] Sea $w\in L$. Entonces, $w=au$ con $u\in\{a,b\}^\ast$. Veamos que
            $S \stackrel{\ast}{\Longrightarrow} w$:
            \begin{equation*}
                S \Longrightarrow aX \Longrightarrow au
            \end{equation*}
            donde en el último paso hemos empleado lo visto respecto a la regla de producción $X\rightarrow aX \mid bX \mid \veps$. Por tanto, $w\in \cc{L}(G)$.

            \item[$\supset)$] Sea $w\in\cc{L}(G)$. Veamos la forma de $w$:
            \begin{equation*}
                S \Longrightarrow aX \Longrightarrow au \mid u\in\{a,b\}^\ast
            \end{equation*}
            donde en el último paso hemos empleado lo visto respecto a la regla de producción $X\rightarrow aX \mid bX \mid \veps$. Por tanto, $w\in L$.
        \end{description}

        \item Describe el lenguaje generado por la gramática teniendo en cuenta que $P$ viene descrito por:
        \begin{align*}
            S &\rightarrow XaXaX \\
            X &\rightarrow aX \mid bX \mid \veps
        \end{align*}

        Sea $L=\{uavawa \mid u,v,w\in\{a,b\}^\ast\}$. Demostraremos mediante doble inclusión que $L=\cc{L}(G)$.
        \begin{description}
            \item[$\subset)$] Sea $z\in L$. Entonces, $z=uavawa$ con $u,v,w\in\{a,b\}^\ast$. Veamos que
            $S \stackrel{\ast}{\Longrightarrow} z$:
            \begin{equation*}
                S \Longrightarrow XaXaX \Longrightarrow uavawa
            \end{equation*}
            donde en el último paso hemos empleado lo visto respecto a la regla de producción $X\rightarrow aX \mid bX \mid \veps$. Por tanto, $z\in \cc{L}(G)$.

            \item[$\supset)$] Sea $z\in\cc{L}(G)$. Veamos la forma de $z$:
            \begin{equation*}
                S \Longrightarrow XaXaX \Longrightarrow uavawa \mid u,v,w\in\{a,b\}^\ast
            \end{equation*}
            donde en el último paso hemos empleado lo visto respecto a la regla de producción $X\rightarrow aX \mid bX \mid \veps$. Por tanto, $z\in L$.
        \end{description}

        \item Describe el lenguaje generado por la gramática teniendo en cuenta que $P$ viene descrito por:
        \begin{align*}
            S &\rightarrow SS \mid XaXaX \mid \veps \\
            X &\rightarrow bX \mid \veps
        \end{align*}

        Sea el lenguaje $L=\{b^i a b^j a b^k \mid i,j,k\in \bb{N}\cup \{0\}\}$. Demostraremos mediante doble inclusión que $L^\ast=\cc{L}(G)$.
        \begin{description}
            \item[$\subset)$] Sea $z\in L^\ast=\bigcup\limits_{i\in \bb{N}} L^i$.
            Sea $n$ el menor número natural tal que $z\in L^n$.
            Notando por $n_a(z)$ al número de $a$'s en $z$, tenemos que $n_a(z)=2n$.
            Entonces, $z\in L\cdot \ldots \cdot L$ ($n$ veces), por lo que existen
            $i_1,j_1,k_1,\ldots,i_n,j_n,k_n\in \bb{N}\cup \{0\}$ tales que $z=b^{i_1} a b^{j_1} a b^{k_1} \cdot \ldots \cdot b^{i_n} a b^{j_n} a b^{k_n}$. Veamos que
            $S \stackrel{\ast}{\Longrightarrow} z$:
            \begin{itemize}
                \item Para conseguir el número de $a$'s deseado, empleamos la regla de producción $S \rightarrow SS$ y reemplazamos una de las $S$ por $XaXaX$. Esto lo hacemos $n$ veces.
                \item Posteriormente, cada $X$ la sustituiremos tantas veces como sea necesario por $bX$ para conseguir el número de $b$'s deseado en cada posición, y finalizaremos con $X\rightarrow \veps$.
            \end{itemize}

            \item[$\supset)$] Sea $z\in\cc{L}(G)$, y sea $n_a(z)$ el número de $a$'s en $z$. Entonces, como el número de $a$ siempre aumenta de dos en dos, tenemos que $n_a(z)=2n$ para algún $n\in \bb{N}\cup \{0\}$.
            Veamos la forma de $z$:
            \begin{itemize}
                \item Para llegar a $z$, hemos tenido que emplear la regla de producción $S \rightarrow SS\rightarrow SXaXaX$ $n$ veces. Una vez llegados aquí, para eliminar la $S$ (ya que habremos llegado a $n_a(z)$ $a$'s), empleamos la regla de producción $S\rightarrow \veps$.
                \item Posteriormente, para cada $X$, tan solo podemos emplear la regla de producción $X\rightarrow bX \mid \veps$ para conseguir el número de $b$'s deseado en cada posición.
            \end{itemize}
            Por tanto, es directo ver que $z\in L^n\subseteq L^\ast$.
        \end{description}
    \end{enumerate}
\end{ejercicio}



\begin{ejercicio} \label{ej:1.2}
    Sea la gramática $G=\left(V,T,P,S\right)$. Determinar en cada caso el lenguaje generado por la gramática.
    \begin{enumerate}
        \item Tenga en cuenta que:
        \begin{align*}
            V &= \{S,A\}\\
            T &= \{a,b\}\\
            S &= S \\
            P &= \left\{
                \begin{array}{rcl}
                    S &\rightarrow & abAS \mid a \\
                    abA &\rightarrow & baab \\
                    A &\rightarrow & b
                \end{array}
            \right\}
        \end{align*}

        Sea $L=\{ua \mid u\in \{abb, baab\}^\ast\}$. Demostraremos mediante doble inclusión que $L=\cc{L}(G)$.
        \begin{description}
            \item[$\subset)$] Sea $w\in L$. Entonces, $w=ua$ con $u\in \{abb, baab\}^\ast$. Veamos que
            $S \stackrel{\ast}{\Longrightarrow} w$. Para ello, sabemos que $u\in \{abb, baab\}^\ast=\bigcup\limits_{i\in \bb{N}} \{abb, baab\}^i$.
            Sea $n$ el menor número natural tal que $u\in \{abb, baab\}^n$, es decir, es una concatenación de $n$ subcadenas, cada una de las cuales es o bien $abb$ o bien $baab$. Veamos que $S$ produce ambas subcadenas:
            \begin{itemize}
                \item Para producir $abb$, tenemos que $S\rightarrow abAS \rightarrow abbS$.
                \item Para producir $baab$, tenemos que $S\rightarrow abAS \rightarrow baabS$.
            \end{itemize}
            Como vemos, en cada caso podemos concatenar la subcadena necesaria, pero siempre nos quedará una $S$ al final. Usamos la regla de producción $S\rightarrow a$ para eliminarla, llegando así a $w$, por lo que $S \stackrel{\ast}{\Longrightarrow} w$ y $w\in \cc{L}(G)$.

            \item[$\supset)$] Sea $w\in\cc{L}(G)$. Veamos la forma de $w$, para lo cual hay dos opciones:
            \begin{itemize}
                \item $S\rightarrow a$: En este caso, habremos finalizado la palabra con $a$, por lo que habremos añadido la subcadena $a$ a la palabra al final.
                \item $S \rightarrow abAS$: En este caso, también hay dos opciones:
                \begin{itemize}
                    \item $S \rightarrow abAS \rightarrow baabS$: En este caso, habremos concatenado $baab$ con $S$, por lo que habremos añadido la subcadena $baab$ a la palabra.
                    \item $S \rightarrow abAS \rightarrow abbS$: En este caso, habremos concatenado $abb$ con $S$, por lo que habremos añadido la subcadena $abb$ a la palabra.
                \end{itemize}
            \end{itemize}
            Por tanto, $w$ es de la forma $ua$ con $u$ una concatenación de $abb$'s y $baab$'s, es decir, $u\in\{abb, baab\}^\ast$.
            Por tanto, $w\in L$.
        \end{description}

        \item \label{ej:1.2.b} Tenga en cuenta que:
        \begin{align*}
            V &= \{\langle \text{número} \rangle, \langle \text{dígito} \rangle\} \\
            T &= \{0,1,2,3,4,5,6,7,8,9\} \\
            S &= \langle \text{número} \rangle \\
            P &= \left\{
                \begin{array}{rcl}
                    \langle \text{número} \rangle &\rightarrow & \langle \text{número} \rangle \langle \text{dígito} \rangle \\
                    \langle \text{número} \rangle &\rightarrow & \langle \text{dígito} \rangle \\
                    \langle \text{dígito} \rangle &\rightarrow & 0 \mid 1 \mid 2 \mid 3 \mid 4 \mid 5 \mid 6 \mid 7 \mid 8 \mid 9
                \end{array}
            \right\}
        \end{align*}

        Tenemos que $\cc{L}(G)$ es el conjunto de los números naturales, permitiendo
        tantos ceros a la izquierda como se quiera. Es decir (usando la notación de potencia y concatenación vista para lenguajes):
        \begin{equation*}
            L = \{0^i n \mid i\in \bb{N}\cup\{0\},~n\in \bb{N}\cup \{0\}\}
        \end{equation*}
        Demostrémoslo mediante doble inclusión que $L=\cc{L}(G)$.
        \begin{description}
            \item[$\subset)$] Sea $w\in L$. Entonces, $w=0^i n$ con $i\in \bb{N}\cup\{0\}$ y $n\in \bb{N}\cup \{0\}$. Veamos que
            $\langle \text{número} \rangle \stackrel{\ast}{\Longrightarrow} w$:
            \begin{itemize}
                \item En primer lugar, aplicamos $|w|-1$ veces la regla de producción $\langle \text{número} \rangle \rightarrow \langle \text{número} \rangle \langle \text{dígito} \rangle$ y la regla
                que lleva de $\langle \text{dígito} \rangle$ a uno de los símbolos terminales, consiguiendo así en cada etapa reemplazar
                la última variable presente en la cadena por un dígito.
                \item Finalmente, aplicamos la regla de producción $\langle \text{número} \rangle \rightarrow \langle \text{dígito} \rangle$ para reemplazar la última variable por un dígito, que será el primero del número formado.
            \end{itemize}
            Por tanto, $\langle \text{número} \rangle \stackrel{\ast}{\Longrightarrow} w$, teniendo que $w\in \cc{L}(G)$.

            \item[$\supset)$] Sea $w\in\cc{L}(G)$. Como la única regla que
            aumenta la longitud es la regla de producción $\langle \text{número} \rangle \rightarrow \langle \text{número} \rangle \langle \text{dígito} \rangle$, tenemos que $w$ tiene la forma:
            \begin{align*}
                \langle \text{número} \rangle &\Longrightarrow \langle \text{número} \rangle \langle \text{dígito} \rangle \stackrel{|w|-1\text{ veces}}{\Longrightarrow} \\
                &\Longrightarrow
                \langle \text{número} \rangle \langle \text{dígito} \rangle \langle \text{dígito} \rangle \stackrel{|w|-1\text{ veces}}{\cdots} \langle \text{dígito} \rangle
                \Longrightarrow \\& \Longrightarrow
                \langle \text{dígito} \rangle \stackrel{|w|\text{ veces}}{\cdots} \langle \text{dígito} \rangle
            \end{align*}
            Por tanto, tenemos que se trata una sucesión de $|w|$ dígitos, lo que nos lleva a que $w\in L$.
        \end{description}

        \item Tenga en cuenta que:
        \begin{align*}
            V &= \{A,S\} \\
            T &= \{a,b\} \\
            S &= S \\
            P &= \left\{
                \begin{array}{rcl}
                    S &\rightarrow & aS \mid aA \\
                    A &\rightarrow & bA \mid b
                \end{array}
            \right\}
        \end{align*}

        Sea $L=\{a^nb^m \in \{a,b\}^\ast \mid n, m \in \bb{N}\}$. Demostraremos mediante doble inclusión que $L=\cc{L}(G)$.
        \begin{description}
            \item[$\subset)$] Sea $w\in L$. Entonces, $w=a^nb^m$ con $n,m\in \bb{N}$. Veamos que
            $S \stackrel{\ast}{\Longrightarrow} w$:
            \begin{itemize}
                \item En primer lugar, aplicamos $n-1$ veces la regla de producción $S \rightarrow aS$ para obtener $a^{n-1}S$,
                \begin{equation*}
                    S \stackrel{\ast}{\Longrightarrow} a^{n-1}S
                \end{equation*}

                \item Para cambiar a la etapa de añadir $b$'s, aplicamos la regla de producción $S \rightarrow aA$, obteniendo así $a^{n}A$,
                \item Después, aplicamos $m-1$ veces la regla de producción $A \rightarrow bA$ para obtener $a^nb^{m-1}A$.
                \item Para finalizar, aplicamos la regla de producción $A \rightarrow b$ para obtener $a^nb^m$.
            \end{itemize}
            Por tanto, $S \stackrel{\ast}{\Longrightarrow} w$, teniendo que $w\in \cc{L}(G)$.

            \item[$\supset)$] Sea $w\in\cc{L}(G)$. Vemos que en la palabra siempre
            va a haber tan solo una variable (ya sea $S$ o $A$). Se empezará con la $S$, y en cierto momento se cambiará a la $A$,
            sin poder entonces volver a la $S$.
            \begin{itemize}
                \item Cuando se está en la etapa en la que hay $S$, tan solo se pueden añadir $a$'s,
                o bien cambiar a la $A$.
                \item Cuando se está en la etapa en la que hay $A$, tan solo se pueden añadir $b$'s.
            \end{itemize}
            Por tanto, tenemos que $w$ estará formada por una sucesión de
            $a$'s seguida de una sucesión de $b$'s, lo que nos lleva a que $w\in L$.
        \end{description}
    \end{enumerate}
\end{ejercicio}

\begin{ejercicio}
    Encontrar gramáticas de tipo 2 para los siguientes lenguajes sobre el alfabeto $\{a, b\}$. En cada caso determinar si los lenguajes generados son de tipo 3, estudiando si existe una gramática de tipo 3 que los genera.
    \begin{enumerate}
        \item Palabras en las que el número de $b$ no es tres.
        
        Tenemos varias opciones:
        \begin{itemize}
            \item Que no tenga $b$'s.
            \item Que tenga una $b$.
            \item Que tenga dos $b$'s.
            \item Que tenga $4$ o más $b$'s.
        \end{itemize}

        Sea la gramática $G=\left(V,T,P,S\right)$ dada por:
        \begin{align*}
            V &= \{S, A, X\} \\
            T &= \{a,b\} \\
            S &= S \\
            P &= \left\{
                \begin{array}{rcl}
                    S &\rightarrow & A \mid AbA \mid AbAbA \mid XbXbXbXbX \\
                    A &\rightarrow & aA \mid \veps \\
                    X &\rightarrow & aX \mid bX \mid \veps
                \end{array}
            \right\}
        \end{align*}

        Esta gramática no obstante es de tipo $2$. Busquemos otra que sea de tipo 3.
        Sea la gramática $G'=\left(V',T',P',S'\right)$ dada por:
        \begin{align*}
            V' &= \{S, X,Y,Z, W\} \\
            T' &= \{a,b\} \\
            S' &= S \\
            P' &= \left\{
                \begin{array}{rcl}
                    S &\rightarrow & \veps \mid aS \mid bX \\
                    X &\rightarrow & \veps \mid aX \mid bY \\
                    Y &\rightarrow & \veps \mid aY \mid bZ \\
                    Z &\rightarrow & aZ \mid bW \\
                    W &\rightarrow & \veps \mid aW \mid bW
                \end{array}
            \right\}
        \end{align*}

        Esta sí es de tipo $3$, y genera el lenguaje deseado.



        \item Palabras que tienen 2 ó 3 $b$.
        
        Sea la gramática $G=\left(V,T,P,S\right)$ dada por:
        \begin{align*}
            V &= \{S, A, B\} \\
            T &= \{a,b\} \\
            S &= S \\
            P &= \left\{
                \begin{array}{rcl}
                    S &\rightarrow & AbAbABA \\
                    A &\rightarrow & aA \mid \veps \\
                    B &\rightarrow & b \mid \veps
                \end{array}
            \right\}
        \end{align*}

        Esta gramática no obstante es de tipo $2$. Busquemos otra que sea de tipo 3.
        Sea la gramática $G'=\left(V',T',P',S'\right)$ dada por:
        \begin{align*}
            V' &= \{S, X,Y,Z,W,V,T\} \\
            T' &= \{a,b\} \\
            S' &= S \\
            P' &= \left\{
                \begin{array}{rcl}
                    S &\rightarrow & aS \mid X \\
                    X &\rightarrow & bY \\
                    Y &\rightarrow & aY \mid Z \\
                    Z &\rightarrow & bW \\
                    W &\rightarrow & aW \mid \veps \mid V \\
                    V &\rightarrow & bT \\
                    T &\rightarrow & aT \mid \veps
                \end{array}
            \right\}
        \end{align*}

        Esta gramática ya es de tipo $3$, pero contiene un número elevado de variables. Veamos si podemos reducirlo:
        Sea la gramática $G''=\left(V'',T'',P'',S''\right)$ dada por:
        \begin{align*}
            V'' &= \{S, X,Y,Z\} \\
            T'' &= \{a,b\} \\
            S'' &= S \\
            P'' &= \left\{
                \begin{array}{rcl}
                    S &\rightarrow & aS \mid bX \\
                    X &\rightarrow & aX \mid bY \\
                    Y &\rightarrow & aY \mid \veps \mid bZ \\
                    Z &\rightarrow & aZ \mid \veps
                \end{array}
            \right\}
        \end{align*}

        Notemos que, en esta gramática de tipo $3$, ya hemos conseguido el menor número de variables posibles, que representan las $4$ etapas. Como la última es opcional, está la regla $Y\rightarrow \veps$, para así no agregar la tercera $b$.

    \end{enumerate}
\end{ejercicio}

\begin{ejercicio}
    Encontrar gramáticas de tipo 2 para los siguientes lenguajes sobre el alfabeto $\{a, b\}$. En cada caso determinar si los lenguajes generados son de tipo 3, estudiando si existe una gramática de tipo 3 que los genera.
    \begin{enumerate}
        \item Palabras que no contienen la subcadena $ab$.
        
        Sea la gramática $G=\left(V,T,P,S\right)$ dada por:
        \begin{align*}
            V &= \{S, A\} \\
            T &= \{a,b\} \\
            S &= S \\
            P &= \left\{
                \begin{array}{rcl}
                    S &\rightarrow & aA \mid bS \mid \veps \\
                    A &\rightarrow & aA \mid \veps \\
                \end{array}
            \right\}
        \end{align*}

        Notemos además que esta gramática es de tipo $3$, y se tiene que:
        \begin{equation*}
            \cc{L}(G) = \{b^i a^j \mid i,j\in \bb{N}\cup \{0\}\}
        \end{equation*}


        \item Palabras que no contienen la subcadena $baa$.
        
        Sea la gramática $G=\left(V,T,P,S\right)$ dada por:
        \begin{align*}
            V &= \{S, B\} \\
            T &= \{a,b\} \\
            S &= S \\
            P &= \left\{
                \begin{array}{rcl}
                    S &\rightarrow & aS \mid bB \mid \veps \\
                    B &\rightarrow & bB \mid abB \mid a \mid \veps
                \end{array}
            \right\}
        \end{align*}
        Notemos además que esta gramática es de tipo $3$.
    \end{enumerate}
\end{ejercicio}

\begin{ejercicio}
    Encontrar una gramática libre de contexto que genere el lenguaje sobre el alfabeto $\{a, b\}$ de las palabras que tienen más $a$ que $b$ (al menos una más).

    Sea la gramática $G=\left(V,T,P,S\right)$ dada por:
    \begin{align*}
        V &= \{S, S'\} \\
        T &= \{a,b\} \\
        S &= S \\
        P &= \left\{
            \begin{array}{rcl}
                S &\rightarrow & S'aS'\\
                S' &\rightarrow & S'aS' \mid aS'bS' \mid bS'aS' \mid \veps
            \end{array}
        \right\}
    \end{align*}
\end{ejercicio}

\begin{ejercicio}
    Encontrar, si es posible, una gramática regular (o, si no es posible, una gramática libre del contexto) que genere el lenguaje $L$ supuesto que $L \subset \{a, b\}^\ast$ y verifica:
    \begin{enumerate}
        \item $u \in L$ si, y solamente si, verifica que $u$ no contiene dos símbolos $b$ consecutivos.
        
        Sea la gramática $G=\left(V,T,P,S\right)$ dada por:
        \begin{align*}
            V &= \{S\} \\
            T &= \{a,b\} \\
            S &= S \\
            P &= \left\{
                \begin{array}{rcl}
                    S &\rightarrow & aS \mid baS \mid b\mid \veps
                \end{array}
            \right\}
        \end{align*}
        \item $u \in L$ si, y solamente si, verifica que $u$ contiene dos símbolos $b$ consecutivos.
        
        Sea la gramática $G=\left(V,T,P,S\right)$ dada por:
        \begin{align*}
            V &= \{S, B, F\} \\
            T &= \{a,b\} \\
            S &= S \\
            P &= \left\{
                \begin{array}{rcl}
                    S &\rightarrow & aS \mid bB \\
                    B &\rightarrow & bF \mid aS \\
                    F &\rightarrow & aF \mid bF \mid \veps
                \end{array}
            \right\}
        \end{align*}
        Notemos que, en este caso, tenemos tres estados:
        \begin{itemize}
            \item $S$: No hemos encontrado dos $b$'s consecutivas.
            \item $B$: Hemos encontrado una $b$, y puede ser que nos encontremos la segunda $b$.
            \item $F$: Hemos encontrado dos $b$'s consecutivas; ya hay libertad.
        \end{itemize}

        Sí es cierto que usamos tres variables. Para usar solo dos variables,
        podemos hacer lo siguiente.
        Sea la gramática $G'=\left(V',T',P',S'\right)$ dada por:
        \begin{align*}
            V' &= \{S, X\} \\
            T' &= \{a,b\} \\
            S' &= S \\
            P' &= \left\{
                \begin{array}{rcl}
                    S &\rightarrow & aS \mid bS \mid bbX \\
                    X &\rightarrow & aX \mid bX \mid \veps
                \end{array}
            \right\}
        \end{align*}
    \end{enumerate}
\end{ejercicio}

\begin{ejercicio}
    Encontrar, si es posible, una gramática regular (o, si no es posible, una gramática libre del contexto) que genere el lenguaje $L$ supuesto que $L \subset \{a, b\}^\ast$ y verifica:
    \begin{enumerate}
        \item $u \in L$ si, y solamente si, verifica que contiene un número impar de símbolos $a$.
        
        Sea la gramática $G=\left(V,T,P,S\right)$ dada por:
        \begin{align*}
            V &= \{S, X\} \\
            T &= \{a,b\} \\
            S &= S \\
            P &= \left\{
                \begin{array}{rcl}
                    S &\rightarrow & aX \mid bS\\
                    X &\rightarrow & aS \mid bX \mid \veps
                \end{array}
            \right\}
        \end{align*}
        \item $u \in L$ si, y solamente si, verifica que no contiene el mismo número de símbolos $a$ que de símbolos $b$.
        
        Sea la gramática $G=\left(V,T,P,S\right)$ dada por:
        \begin{align*}
            V &= \{S, A, B, X\} \\
            T &= \{a,b\} \\
            S &= S \\
            P &= \left\{
                \begin{array}{rcl}
                    S &\rightarrow & AaA \mid BbB \\
                    A &\rightarrow & AaA\mid X \\
                    B &\rightarrow & BbB \mid X \\
                    X &\rightarrow & aXbX \mid bXaX \mid \veps
                \end{array}
            \right\}
        \end{align*}
    \end{enumerate}
\end{ejercicio}


\begin{ejercicio}
    Dado el alfabeto $A = \{a, b\}$ determinar si es posible encontrar una gramática libre de contexto que:
    \begin{enumerate}
        \item Genere las palabras de longitud impar, y mayor o igual que 3, tales que la primera letra coincida con la letra central de la palabra.
        
        Sea la gramática $G=\left(V,T,P,S\right)$ dada por:
        \begin{align*}
            V &= \{S, X, A, B, C, D\} \\
            T &= \{a,b\} \\
            S &= S \\
            P &= \left\{
                \begin{array}{rcl}
                    S &A\mid B \\
                    A &\rightarrow & aCX \\
                    C & \rightarrow & a \mid XCX \\
                    B &\rightarrow & bDX \\
                    D & \rightarrow & b \mid XDX \\
                    X &\rightarrow & a\mid b
                \end{array}
            \right\}
        \end{align*}
        \item Genere las palabras de longitud par, y mayor o igual que 2, tales que las dos letras centrales coincidan.
        
        Sea la gramática $G=\left(V,T,P,S\right)$ dada por:
        \begin{align*}
            V &= \{S, X\} \\
            T &= \{a,b\} \\
            S &= S \\
            P &= \left\{
                \begin{array}{rcl}
                    S &\rightarrow & XSX\mid C \\
                    C &\rightarrow & aa \mid bb \\
                    X &\rightarrow & a \mid b
                \end{array}
            \right\}
        \end{align*}
    \end{enumerate}
\end{ejercicio}

\begin{ejercicio}
    Sea la gramática $G=\left(V,T,P,S\right)$ dada por:
    \begin{align*}
        V &= \{S, X\} \\
        T &= \{a,b\} \\
        S &= S\\
        P &= \left\{
            \begin{array}{rcl}
                S &\rightarrow & SS \\
                S &\rightarrow & XXX \\
                X &\rightarrow & aX \mid Xa \mid b
            \end{array}
        \right\}
    \end{align*}
    Determinar si el lenguaje generado por la gramática es regular. Justificar la respuesta.\\

    Sea la siguiente gramática regular $G'=\left(V',T',P',S'\right)$ dada por:
    \begin{align*}
        V' &= \{S, X\} \\
        T' &= \{a,b\} \\
        S' &= S \\
        P' &= \left\{
            \begin{array}{rcl}
                S &\rightarrow & aS \mid bX \\
                X &\rightarrow & aX \mid bY \\
                Y &\rightarrow & aY \mid bZ \\
                Z &\rightarrow & aZ \mid bW \mid \veps \\
                W &\rightarrow & aW \mid bU \\
                U &\rightarrow & aU \mid bV \\
                V &\rightarrow & aV \mid \veps
            \end{array}
        \right\}
    \end{align*}

    Tenemos que $\cc{L}(G) = \cc{L}(G')$, y como $G'$ es una gramática regular, tenemos que $\cc{L}(G)$ es regular.
    Sí es cierto que en el tema $2$ aprendemos otras maneras de demostrarlo más sencillas, como buscar un autómata finito que lo genere.
\end{ejercicio}

\begin{ejercicio}
    Dado un lenguaje $L$ sobre un alfabeto $A$, ¿es $L^{\ast}$ siempre numerable? ¿nunca lo es? ¿o puede serlo unas veces sí y otras, no? Pon ejemplos en este último caso.\\

    $L^{\ast}$ es siempre numerable, veámos por qué. $L^{\ast}$ es un lenguaje sobre el alfabeto $A$, por lo que $L^{\ast}\subseteq A^{\ast}$ y $A^{\ast}$ es numerable (visto en teoría), luego $L^{\ast}$ también lo es.
\end{ejercicio}

\begin{ejercicio}
    Dado un lenguaje $L$ sobre un alfabeto $A$, caracterizar cuando $L^{\ast} = L$. Esto es, dar un conjunto de propiedades sobre $L$ de manera que $L$ cumpla esas propiedades si y sólo si $L^{\ast} = L$.

    \begin{equation*}
        L = L^{\ast} \Longleftrightarrow \left\{
            \begin{array}{cl}
                \veps \in L \\ \land \\ u,v \in L & \Longrightarrow uv\in L
            \end{array}
        \right.
    \end{equation*}
    Es decir, $L=L^{\ast}$ si y solo si la cadena vacía está en $L$ y además es cerrado para concatenaciones.

    \begin{proof} Demostramos mediante doble implicación.
        \begin{description}
            \item [$\Longleftarrow)$] La inclusión $L\subseteq L^{\ast}$ es obvia, por lo que solo falta demostrar la otra inclusión.\\

                Sea $v\in L^{\ast}$:
                \begin{enumerate}
                    \item Si $v = \veps \Longrightarrow v\in L$ por hipótesis.
                    \item Si $v\neq \veps$, $\exists n\in \mathbb{N}$ tal que 
                        \begin{equation*}
                            v = a_1 a_2 \ldots a_n
                        \end{equation*}
                        con $a_i \in L$ $\forall i \in \{1, \ldots, n\}$, de donde tenemos que $v\in L$, por ser cerrado para concatenaciones. Luego $L^{\ast}\subseteq L$.
                \end{enumerate}
            \item [$\Longrightarrow)$] Hemos de probar dos cosas:
                \begin{enumerate}
                    \item $\veps \in L^{\ast}=L$.
                    \item Sean $u,v\in L=L^{\ast} \Longrightarrow uv\in L^{\ast}=L$.
                \end{enumerate}
        \end{description}
    \end{proof}
\end{ejercicio}

\begin{ejercicio}
    Dados dos homomorfismos $f : A^{\ast} \rightarrow B^{\ast}$, $g : A^{\ast} \rightarrow B^{\ast}$, se dice que son iguales si $f(x) = g(x)$, $\forall x \in A^{\ast}$. ¿Existe un procedimiento algorítmico para comprobar si dos homomorfismos son iguales?\\

    Sí, basta probar que su imagen coincide sobre un conjunto finito de elementos, los de $A$:
    \begin{equation*}
        f(x) = g(x) \quad \forall x\in A^{\ast} \Longleftrightarrow f(a)=g(a) \quad \forall a\in A
    \end{equation*}
    \begin{proof}\ 
        \begin{description}
            \item [$\Longleftarrow)$] Sea $v\in A^{\ast}$, $\exists n\in \mathbb{N}$ tal que $v=a_1a_2\ldots a_n$ con $a_i \in A$ $\forall i \in \{1,\ldots, n\}$
                \begin{equation*}
                    f(v) = f(a_1)f(a_2)\ldots f(a_n) = g(a_1)g(a_2)\ldots g(a_n) = g(v)
                \end{equation*}
            \item [$\Longrightarrow)$] Sea $a\in A \Longrightarrow a\in A^{\ast}\Longrightarrow f(a)=g(a)$.
        \end{description}
    \end{proof}
\end{ejercicio}

\begin{ejercicio}
    Sea $L \subseteq A^{\ast}$ un lenguaje arbitrario. Sea $C_0 = L$ y definamos los lenguajes $S_i$ y $C_i$, para todo $i \geq 1$, por $S_i = C_{i-1}^+$ y $C_i = \ol{S_i}$. 
    \begin{enumerate}
        \item ¿Es $S_1$ siempre, nunca o a veces igual a $C_2$? Justifica la respuesta.
        \item Demostrar que $S_2 = C_3$, cualquiera que sea $L$.
        \begin{observacion}
            Demuestra que $C_2$ es cerrado para la concatenación.
        \end{observacion}
    \end{enumerate}
    % // TODO: Hacer JJ
\end{ejercicio}

\begin{ejercicio}
    Demuestra que, para todo alfabeto $A$, el conjunto de los lenguajes finitos sobre dicho alfabeto es numerable.

    Sea $A=\{a_1, a_2, \ldots, a_n\}$, con $n\in \mathbb{N}$. Definimos el siguiente conjunto:
    \begin{equation*}
        \Gamma = \{L\subseteq A^{\ast} \mid L \text{ es finito}\}
    \end{equation*}

    Dado un símbolo $z\notin A$, definimos el conjunto $B=\{z\}\cup A$. Sea $B^{\ast}$ numerable, y buscamos una inyección de $\Gamma$ en $B^{\ast}$.
    Dado un lenguaje $L\in \Gamma$, sea $L=\{l_1, l_2, \ldots, l_m\}$, con $m\in \mathbb{N}$ y $l_i\in A^{\ast}$ $\forall i\in \{1, \ldots, m\}$. Definimos la siguiente función:
    \Func{f}{\Gamma}{B^\ast}{L}{zl_1zl_2\ldots zl_mz}

    Veamos que $f$ es inyectiva. Sean $L_1, L_2\in \Gamma$ tales que $f(L_1)=f(L_2)$. Entonces,
    \begin{equation*}
        zl_1zl_2\ldots zl_kz = zl'_1zl'_2\ldots zl'_{k'}z
    \end{equation*}
    Por ser ambas palabras iguales, tenemos que $k=k'$ y $l_i=l'_i$ $\forall i\in \{1, \ldots, k\}$, de donde $L_1=L_2$. Por tanto, $f$ es inyectiva, por lo que $\Gamma$ es inyectivo con un subconjunto de $B^{\ast}$, que es numerable. Por tanto, $\Gamma$ es numerable.
\end{ejercicio}




\subsection{Cálculo de gramáticas}

\begin{ejercicio}[Complejidad: Sencilla]
    Calcula, de forma razonada, gramáticas que generen cada uno de los siguientes lenguajes:
    \begin{enumerate}
        \item $\{ u\in \{0,1\}^\ast \mid |u|\leq 4 \}$
        
        Sea la gramática $G=\left(V,T,P,S\right)$ dada por:
        \begin{align*}
            V &= \{S, X\} \\
            T &= \{0,1\} \\
            S &= S \\
            P &= \left\{
                \begin{array}{rcl}
                    S &\rightarrow & XXXX \\
                    X &\rightarrow & 0 \mid 1 \mid \veps
                \end{array}
            \right\}
        \end{align*}

        No obstante, esta gramática es de tipo $2$. Busquemos una de tipo $3$.
        Sea la gramática $G'=\left(V',T',P',S'\right)$ dada por:
        \begin{align*}
            V' &= \{S, X, Y, Z\} \\
            T' &= \{0,1\} \\
            S' &= S \\
            P' &= \left\{
                \begin{array}{rcl}
                    S &\rightarrow & 0X \mid 1X \mid \veps \\
                    X &\rightarrow & 0Y \mid 1Y \mid \veps \\
                    Y &\rightarrow & 0Z \mid 1Z \mid \veps \\
                    Z &\rightarrow & 0 \mid 1
                \end{array}
            \right\}
        \end{align*}
        Tenemos que $\cc{L}(G) = \cc{L}(G')$, y es igual al lenguaje deseado. Tenemos por tanto que es un lenguaje regular.


        \item Palabras con 0's y 1's que no contengan dos 1's consecutivos y que empiecen por un 1 y que terminen por dos 0's.
        
        Sea la gramática $G=\left(V,T,P,S\right)$ dada por:
        \begin{align*}
            V &= \{S, X, Y\} \\
            T &= \{0,1\} \\
            S &= S \\
            P &= \left\{
                \begin{array}{rcl}
                    S &\rightarrow & 1X00 \\
                    X &\rightarrow & 0Y \mid \veps \\
                    Y &\rightarrow & 0Y \mid 1X \mid \veps \\
                \end{array}
            \right\}
        \end{align*}

        Notemos que esta gramática es de tipo 2 debido a la primera regla de producción. Busquemos una de tipo 3. 
        Sea la gramática $G'=\left(V',T',P',S'\right)$ dada por:
        \begin{align*}
            V' &= \{S, X, Y\} \\
            T' &= \{0,1\} \\
            S' &= S \\
            P' &= \left\{
                \begin{array}{rcl}
                    S &\rightarrow & 1X \\
                    X &\rightarrow & 0Y \mid F \\
                    Y &\rightarrow & 0Y \mid 1X \mid F \\
                    F &\rightarrow & 00
                \end{array}
            \right\}
        \end{align*}

        Tenemos que $\cc{L}(G) = \cc{L}(G')$, y es igual al lenguaje deseado. Tenemos por tanto que es un lenguaje regular. En esta última gramática, tenemos los siguientes estados:
        \begin{itemize}
            \item $S$: Es el estado inicial, empezamos con un $1$.
            \item $X$: Acabamos de escribir un $1$, por lo que ahora tan solo podemos escribir $0$'s.
            \item $Y$: Acabamos de escribir un $0$, por lo que ahora podemos escribir tanto $0$'s como $1$'s.
            \item $F$: Ya hemos terminado, y escribimos los dos $0$'s finales por la restricción impuesta.
        \end{itemize}
        

        \item El conjunto vacío.
        
        Sea la gramática $G=\left(V,T,P,S\right)$ dada por:
        \begin{align*}
            V &= \{S\} \\
            T &= \emptyset \\
            S &= S \\
            P &= \left\{
                \begin{array}{rcl}
                    S &\rightarrow & S
                \end{array}
            \right\}
        \end{align*}

        \item El lenguaje formado por los números naturales.
        
        Sea la gramática $G=\left(V,T,P,S\right)$ dada por:
        \begin{align*}
            V &= \{\langle \text{número no iniciado} \rangle, \langle \text{dígito no cero} \rangle, \langle \text{dígito} \rangle, \langle \text{número iniciado} \rangle\} \\
            T &= \{0,1,2,3,4,5,6,7,8,9\} \\
            S &= \langle \text{número no iniciado} \rangle \\
            P &= \left\{
                \begin{array}{rcl}
                    \langle \text{número no iniciado} \rangle &\rightarrow & \langle \text{dígito no cero} \rangle \mid \langle \text{dígito no cero} \rangle \langle \text{número iniciado} \rangle \\
                    \langle \text{número iniciado} \rangle &\rightarrow & \langle \text{dígito} \rangle \mid \langle \text{dígito} \rangle \langle \text{número iniciado} \rangle \\
                    \langle \text{dígito no cero} \rangle &\rightarrow & 1 \mid 2 \mid 3 \mid 4 \mid 5 \mid 6 \mid 7 \mid 8 \mid 9 \\
                    \langle \text{dígito} \rangle &\rightarrow & 0 \mid \langle \text{dígito no cero} \rangle
                \end{array}
            \right\}
        \end{align*}

        Notemos que esta gramática es similar a la descrita en el Ejercicio \ref{ej:1.2}.\ref{ej:1.2.b}, pero adaptada para que los números naturales no puedan empezar por $0$.
        No obstante, esta gramática es de tipo $2$. Busquemos una de tipo $3$.
        Sea la gramática $G'=\left(V',T',P',S'\right)$ dada por:
        \begin{align*}
            V' &= \{S, X, Y, Z\} \\
            T' &= \{0,1,2,3,4,5,6,7,8,9\} \\
            S' &= S \\
            P' &= \left\{
                \begin{array}{rcl}
                    S &\rightarrow & 0 \mid 1N \mid 2N \mid 3N \mid 4N \mid 5N \mid 6N \mid 7N \mid 8N \mid 9N\\
                    N &\rightarrow & 0N\mid 1N \mid 2N \mid 3N \mid 4N \mid 5N \mid 6N \mid 7N \mid 8N \mid 9N \mid \veps
                \end{array}
            \right\}
        \end{align*}
        \item $\{ a^n \in \{a,b\}^\ast \mid n\geq 0 \} \cup \{ a^nb^n \in \{a,b\}^\ast \mid n\geq 0 \}$
        
        Sea la gramática $G=\left(V,T,P,S\right)$ dada por:
        \begin{align*}
            V &= \{S, X, Y\} \\
            T &= \{a,b\} \\
            S &= S \\
            P &= \left\{
                \begin{array}{rcl}
                    S &\rightarrow & X \mid Y \mid \veps \\
                    X &\rightarrow & aX \mid \veps \\
                    Y &\rightarrow & aYb \mid \veps
                \end{array}
            \right\}
        \end{align*}
        \item $\{ a^nb^{2n}c^m \in \{a,b,c\}^\ast \mid n,m>0 \}$
        
        Sea la gramática $G=\left(V,T,P,S\right)$ dada por:
        \begin{align*}
            V &= \{S, X, Y, Z\} \\
            T &= \{a,b,c\} \\
            S &= S \\
            P &= \left\{
                \begin{array}{rcl}
                    S &\rightarrow & aXbbcY \\
                    X &\rightarrow & aXbb \mid \veps \\
                    Y &\rightarrow & cY \mid \veps
                \end{array}
            \right\}
        \end{align*}
        \item $\{ a^nb^ma^n \in \{a,b\}^\ast \mid m,n\geq 0 \}$
        
        Sea la gramática $G=\left(V,T,P,S\right)$ dada por:
        \begin{align*}
            V &= \{S, X\} \\
            T &= \{a,b\} \\
            S &= S \\
            P &= \left\{
                \begin{array}{rcl}
                    S &\rightarrow & aSa \mid bX \mid \veps \\
                    X &\rightarrow & bX \mid \veps \\
                \end{array}
            \right\}
        \end{align*}

        \item Palabras con 0's y 1's que contengan la subcadena 00 y 11.
        
        Sea la gramática $G=\left(V,T,P,S\right)$ dada por:
        \begin{align*}
            V &= \{S, X\} \\
            T &= \{0,1\} \\
            S &= S \\
            P &= \left\{
                \begin{array}{rcl}
                    S &\rightarrow & X00X11X \mid X11X00X \\
                    X &\rightarrow & 0X \mid 1X \mid \veps
                \end{array}
            \right\}
        \end{align*}

        Notemos que esta gramática es de tipo $2$. Busquemos una de tipo $3$.
        Sea la gramática $G'=\left(V',T',P',S'\right)$ dada por:
        \begin{align*}
            V' &= \{S, X, A, B, F\} \\
            T' &= \{0,1\} \\
            S' &= S \\
            P' &= \left\{
                \begin{array}{rcl}
                    S &\rightarrow & 0S \mid 1S \mid X\\
                    X &\rightarrow & 00A \mid 11B \\
                    A &\rightarrow & 0A \mid 1A \mid 11F \\
                    B &\rightarrow & 0B \mid 1B \mid 00F \\
                    F &\rightarrow & 0F \mid 1F \mid \veps
                \end{array}
            \right\}
        \end{align*}

        Notemos que:
        \begin{itemize}
            \item $S$: No hemos encontrado ninguna subcadena.
            \item $X$: Hemos encontrado una subcadena, y ahora buscamos la otra.
            \item $A$: Hemos encontrado la subcadena $00$, y ahora buscamos la subcadena $11$.
            \item $B$: Hemos encontrado la subcadena $11$, y ahora buscamos la subcadena $00$.
            \item $F$: Hemos encontrado ambas subcadenas.
        \end{itemize}
        
        \item Palíndromos formados con las letras $a$ y $b$.
        
        Sea la gramática $G=\left(V,T,P,S\right)$ dada por:
        \begin{align*}
            V &= \{S, X, Y\} \\
            T &= \{a,b\} \\
            S &= S \\
            P &= \left\{
                \begin{array}{rcl}
                    S &\rightarrow & aSa \mid bSb \mid \veps \mid a \mid b
                \end{array}
            \right\}
        \end{align*}
        Notemos que las reglas $S\rightarrow a\mid b$ se han añadido para añadir los palíndromos de longitud impar.
    \end{enumerate}
\end{ejercicio}

\begin{ejercicio}[Complejidad: Media]
    Calcula, de forma razonada, gramáticas que generen cada uno de los siguientes lenguajes:
    \begin{enumerate}
        \item $\{uv \in \{0,1\}^\ast \mid u^{-1} \text{ es un prefijo de } v\}$
        
        Sea la gramática $G=\left(V,T,P,S\right)$ dada por:
        \begin{align*}
            V &= \{S, X, Y\} \\
            T &= \{0,1\} \\
            S &= S \\
            P &= \left\{
                \begin{array}{rcl}
                    S &\rightarrow & XY \\
                    X &\rightarrow & 0X0 \mid 1X1 \mid \veps \\
                    Y &\rightarrow & 0Y \mid 1Y \mid \veps
                \end{array}
            \right\}
        \end{align*}
        Notemos que $X$ deriva en el palíndromo, $uu^{-1}$, y $Y$ en el resto de la palabra de $v$.
        \item $\{ucv \in \{a,b,c\}^\ast \mid |u| = |v|\}$
        
        Sea la gramática $G=\left(V,T,P,S\right)$ dada por:
        \begin{align*}
            V &= \{S, X\} \\
            T &= \{a,b,c\} \\
            S &= S \\
            P &= \left\{
                \begin{array}{rcl}
                    S &\rightarrow & XSX \mid c \\
                    X &\rightarrow & a \mid b \mid c
                \end{array}
            \right\}
        \end{align*}

        \item $\{u1^n \in \{0,1\}^\ast \mid |u| = n\}$
        
        Sea la gramática $G=\left(V,T,P,S\right)$ dada por:
        \begin{align*}
            V &= \{S, X\} \\
            T &= \{0,1\} \\
            S &= S \\
            P &= \left\{
                \begin{array}{rcl}
                    S &\rightarrow & XS1 \mid \veps \\
                    X &\rightarrow & 0 \mid 1
                \end{array}
            \right\}
        \end{align*}

        \item $\{a^nb^na^{n+1} \in \{a,b\}^\ast \mid n\geq 0\}$ (observar transparencias de teoría)
        
        Sea la gramática $G=\left(V,T,P,S\right)$ dada por:
        \begin{align*}
            V &= \{S, X, Y\} \\
            T &= \{a,b\} \\
            S &= S \\
            P &= \left\{
                \begin{array}{rcl}
                    S &\rightarrow & a\mid abaa\mid aXbaa\\
                    Xb & \rightarrow & bX\\
                    Xa & \rightarrow & Ybaa\\
                    bY & \rightarrow & Yb\\
                    aY & \rightarrow aa\mid aaX
                \end{array}
            \right\}
        \end{align*}
    \end{enumerate}
\end{ejercicio}


\begin{ejercicio}[Complejidad: Difícil]
    Calcula, de forma razonada, gramáticas que generen cada uno de los siguientes lenguajes:
    \begin{enumerate}
        \item $\{a^nb^mc^k \in \{a,b,c\}^\ast \mid k = m + n\}$
        
        Sea la gramática $G=\left(V,T,P,S\right)$ dada por:
        \begin{align*}
            V &= \{S, X\} \\
            T &= \{a,b,c\} \\
            S &= S \\
            P &= \left\{
                \begin{array}{rcl}
                    S &\rightarrow & aSc \mid X \\
                    X &\rightarrow & bXc \mid \veps
                \end{array}
            \right\}
        \end{align*}
        
        \item Palabras que son múltiplos de 7 en binario.
        
        % // TODO: Hacer JJ
    \end{enumerate}
\end{ejercicio}


\begin{ejercicio}[Complejidad: Extrema (no son libres de contexto)]
    Calcula, de forma razonada, gramáticas que generen cada uno de los siguientes lenguajes:
    \begin{enumerate}
        \item $\{ww \mid w \in \{0,1\}^\ast\}$
            % Para este lenguaje, hemos construido la gramática $G=(V,T,P,S)$ dada por:
            % \begin{align*}
            %     V &= \{S, \alpha, \beta, \gamma, X, E, E_1, E_0, E', B\} \\
            %     T &= \{0,1\} \\
            %     S &= S
            % \end{align*}
            % $P$ que contiene las siguientes reglas de producción, las cuales iremos explicando poco a poco para que pueda entenderse la gramática.

            % La idea principal es generar una palabra cualquiera del lenguaje ${\{0,1\}}^{\ast}$ entre las variables $\alpha$ y $\beta$. Posteriormente, iremos copiando dicha palabra a la derecha de $\beta$ usando para ello las variables $E$. Controlaremos con la variale $\gamma$ la parte de la palabra de la izquierda que ya hayamos copiado a la derecha de $\beta$.

            % Finalmente, utilizamos $B$ para eliminar las variables restantes una vez hecha la copia de la palabra de la izquierda en la parte derecha. 

            % Las reglas de producción de $P$ serán las siguientes:
            % \begin{itemize}
            %     \item Comenzamos construyendo el entorno en el que trabajaremos:
            %     \begin{equation*}
            %         S \rightarrow \alpha X \beta
            %     \end{equation*}
            %     Donde ya hemos comentado que entre $\alpha$ y $\beta$ generaremos una palabra cualquiera. Para ello, usaremos $X$:
            %     \begin{equation*}
            %         X \rightarrow 0X\ |\ 1X\ |\ E\gamma
            %     \end{equation*}

            %     \item Ahora, comenzará la copia de la palabra comprendida entre $\alpha$ Y $\beta$. La porción de palabra comprendida entre $\gamma$ y $\beta$ es la porción de palabra que tenemos ya copiada a la derecha de $\beta$. Para el 0:
            %         \begin{equation*}
            %             0E\gamma \rightarrow \gamma 0 E_0
            %         \end{equation*}
            %         donde avanzamos $\gamma$ un caracter y ahora nos movemos hacia la derecha hasta encontrar $\beta$:
            %         \begin{align*}
            %             E_0 0 &\rightarrow 0 E_0 \\
            %             E_0 1 &\rightarrow 1 E_0 \\
            %             E_0 \beta & \rightarrow E' \beta 0
            %         \end{align*}
            % \end{itemize}

        \item $\{a^{n^2} \in \{a\}^{\ast} \mid n\geq 0\}$
        \item $\{a^p \in \{a\}^{\ast} \mid p \text{ es primo}\}$
        \item $\{a^nb^m \in \{a,b\}^{\ast} \mid n\leq m^2\}$
    \end{enumerate}

\end{ejercicio}

