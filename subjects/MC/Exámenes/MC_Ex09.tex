\documentclass[12pt]{article}

% Idioma y codificación
\usepackage[spanish, es-tabla]{babel}       %es-tabla para que se titule "Tabla"
\usepackage[utf8]{inputenc}

% Márgenes
\usepackage[a4paper,top=3cm,bottom=2.5cm,left=3cm,right=3cm]{geometry}

% Comentarios de bloque
\usepackage{verbatim}

% Paquetes de links
\usepackage[hidelinks]{hyperref}    % Permite enlaces
\usepackage{url}                    % redirecciona a la web

% Más opciones para enumeraciones
\usepackage{enumitem}

% Personalizar la portada
\usepackage{titling}

% Paquetes de tablas
\usepackage{multirow}


%------------------------------------------------------------------------

%Paquetes de figuras
\usepackage{caption}
\usepackage{subcaption} % Figuras al lado de otras
\usepackage{float}      % Poner figuras en el sitio indicado H.


% Paquetes de imágenes
\usepackage{graphicx}       % Paquete para añadir imágenes
\usepackage{transparent}    % Para manejar la opacidad de las figuras

% Paquete para usar colores
\usepackage[dvipsnames]{xcolor}
\usepackage{pagecolor}      % Para cambiar el color de la página

% Habilita tamaños de fuente mayores
\usepackage{fix-cm}

% Para los gráficos
\usepackage{tikz}

% Para poder situar los nodos en los grafos
\usetikzlibrary{positioning}


%------------------------------------------------------------------------

% Paquetes de matemáticas
\usepackage{mathtools, amsfonts, amssymb, mathrsfs}
\usepackage[makeroom]{cancel}     % Simplificar tachando
\usepackage{polynom}    % Divisiones y Ruffini
\usepackage{units} % Para poner fracciones diagonales con \nicefrac

\usepackage{pgfplots}   %Representar funciones
\pgfplotsset{compat=1.18}  % Versión 1.18

\usepackage{tikz-cd}    % Para usar diagramas de composiciones
\usetikzlibrary{calc}   % Para usar cálculo de coordenadas en tikz

%Definición de teoremas, etc.
\usepackage{amsthm}
%\swapnumbers   % Intercambia la posición del texto y de la numeración

\theoremstyle{plain}

\makeatletter
\@ifclassloaded{article}{
  \newtheorem{teo}{Teorema}[section]
}{
  \newtheorem{teo}{Teorema}[chapter]  % Se resetea en cada chapter
}
\makeatother

\newtheorem{coro}{Corolario}[teo]           % Se resetea en cada teorema
\newtheorem{prop}[teo]{Proposición}         % Usa el mismo contador que teorema
\newtheorem{lema}[teo]{Lema}                % Usa el mismo contador que teorema

\theoremstyle{remark}
\newtheorem*{observacion}{Observación}

\theoremstyle{definition}

\makeatletter
\@ifclassloaded{article}{
  \newtheorem{definicion}{Definición} [section]     % Se resetea en cada chapter
}{
  \newtheorem{definicion}{Definición} [chapter]     % Se resetea en cada chapter
}
\makeatother

\newtheorem*{notacion}{Notación}
\newtheorem*{ejemplo}{Ejemplo}
\newtheorem*{ejercicio*}{Ejercicio}             % No numerado
\newtheorem{ejercicio}{Ejercicio} [section]     % Se resetea en cada section


% Modificar el formato de la numeración del teorema "ejercicio"
\renewcommand{\theejercicio}{%
  \ifnum\value{section}=0 % Si no se ha iniciado ninguna sección
    \arabic{ejercicio}% Solo mostrar el número de ejercicio
  \else
    \thesection.\arabic{ejercicio}% Mostrar número de sección y número de ejercicio
  \fi
}


% \renewcommand\qedsymbol{$\blacksquare$}         % Cambiar símbolo QED
%------------------------------------------------------------------------

% Paquetes para encabezados
\usepackage{fancyhdr}
\pagestyle{fancy}
\fancyhf{}

\newcommand{\helv}{ % Modificación tamaño de letra
\fontfamily{}\fontsize{12}{12}\selectfont}
\setlength{\headheight}{15pt} % Amplía el tamaño del índice


%\usepackage{lastpage}   % Referenciar última pag   \pageref{LastPage}
\fancyfoot[C]{\thepage}

%------------------------------------------------------------------------

% Conseguir que no ponga "Capítulo 1". Sino solo "1."
\makeatletter
\@ifclassloaded{book}{
  \renewcommand{\chaptermark}[1]{\markboth{\thechapter.\ #1}{}} % En el encabezado
    
  \renewcommand{\@makechapterhead}[1]{%
  \vspace*{50\p@}%
  {\parindent \z@ \raggedright \normalfont
    \ifnum \c@secnumdepth >\m@ne
      \huge\bfseries \thechapter.\hspace{1em}\ignorespaces
    \fi
    \interlinepenalty\@M
    \Huge \bfseries #1\par\nobreak
    \vskip 40\p@
  }}
}
\makeatother

%------------------------------------------------------------------------
% Paquetes de cógido
\usepackage{minted}
\renewcommand\listingscaption{Código fuente}

\usepackage{fancyvrb}
% Personaliza el tamaño de los números de línea
\renewcommand{\theFancyVerbLine}{\small\arabic{FancyVerbLine}}

% Estilo para C++
\newminted{cpp}{
    frame=lines,
    framesep=2mm,
    baselinestretch=1.2,
    linenos,
    escapeinside=||
}

% para minted
\definecolor{LightGray}{rgb}{0.95,0.95,0.92}
\setminted{
    linenos=true,
    stepnumber=5,
    numberfirstline=true,
    autogobble,
    breaklines=true,
    breakautoindent=true,
    breaksymbolleft=,
    breaksymbolright=,
    breaksymbolindentleft=0pt,
    breaksymbolindentright=0pt,
    breaksymbolsepleft=0pt,
    breaksymbolsepright=0pt,
    fontsize=\footnotesize,
    bgcolor=LightGray,
    numbersep=10pt
}


\usepackage{listings} % Para incluir código desde un archivo

\renewcommand\lstlistingname{Código Fuente}
\renewcommand\lstlistlistingname{Índice de Códigos Fuente}

% Definir colores
\definecolor{vscodepurple}{rgb}{0.5,0,0.5}
\definecolor{vscodeblue}{rgb}{0,0,0.8}
\definecolor{vscodegreen}{rgb}{0,0.5,0}
\definecolor{vscodegray}{rgb}{0.5,0.5,0.5}
\definecolor{vscodebackground}{rgb}{0.97,0.97,0.97}
\definecolor{vscodelightgray}{rgb}{0.9,0.9,0.9}

% Configuración para el estilo de C similar a VSCode
\lstdefinestyle{vscode_C}{
  backgroundcolor=\color{vscodebackground},
  commentstyle=\color{vscodegreen},
  keywordstyle=\color{vscodeblue},
  numberstyle=\tiny\color{vscodegray},
  stringstyle=\color{vscodepurple},
  basicstyle=\scriptsize\ttfamily,
  breakatwhitespace=false,
  breaklines=true,
  captionpos=b,
  keepspaces=true,
  numbers=left,
  numbersep=5pt,
  showspaces=false,
  showstringspaces=false,
  showtabs=false,
  tabsize=2,
  frame=tb,
  framerule=0pt,
  aboveskip=10pt,
  belowskip=10pt,
  xleftmargin=10pt,
  xrightmargin=10pt,
  framexleftmargin=10pt,
  framexrightmargin=10pt,
  framesep=0pt,
  rulecolor=\color{vscodelightgray},
  backgroundcolor=\color{vscodebackground},
}

%------------------------------------------------------------------------

% Comandos definidos
\newcommand{\bb}[1]{\mathbb{#1}}
\newcommand{\cc}[1]{\mathcal{#1}}

% I prefer the slanted \leq
\let\oldleq\leq % save them in case they're every wanted
\let\oldgeq\geq
\renewcommand{\leq}{\leqslant}
\renewcommand{\geq}{\geqslant}

% Si y solo si
\newcommand{\sii}{\iff}

% Letras griegas
\newcommand{\eps}{\epsilon}
\newcommand{\veps}{\varepsilon}
\newcommand{\lm}{\lambda}

\newcommand{\ol}{\overline}
\newcommand{\ul}{\underline}
\newcommand{\wt}{\widetilde}
\newcommand{\wh}{\widehat}

\let\oldvec\vec
\renewcommand{\vec}{\overrightarrow}

% Derivadas parciales
\newcommand{\del}[2]{\frac{\partial #1}{\partial #2}}
\newcommand{\Del}[3]{\frac{\partial^{#1} #2}{\partial #3^{#1}}}
\newcommand{\deld}[2]{\dfrac{\partial #1}{\partial #2}}
\newcommand{\Deld}[3]{\dfrac{\partial^{#1} #2}{\partial #3^{#1}}}


\newcommand{\AstIg}{\stackrel{(\ast)}{=}}
\newcommand{\Hop}{\stackrel{L'H\hat{o}pital}{=}}

\newcommand{\red}[1]{{\color{red}#1}} % Para integrales, destacar los cambios.

% Método de integración
\newcommand{\MetInt}[2]{
    \left[\begin{array}{c}
        #1 \\ #2
    \end{array}\right]
}

% Declarar aplicaciones
% 1. Nombre aplicación
% 2. Dominio
% 3. Codominio
% 4. Variable
% 5. Imagen de la variable
\newcommand{\Func}[5]{
    \begin{equation*}
        \begin{array}{rrll}
            #1:& #2 & \longrightarrow & #3\\
               & #4 & \longmapsto & #5
        \end{array}
    \end{equation*}
}

%------------------------------------------------------------------------

% Para poder incluir árboles
\usepackage{forest}
\usepackage{booktabs}

\usepackage{hhline}
\newcommand{\cell}[1]{\multicolumn{1}{|c|}{$#1$}}

% Para poder añadir autómatas
% https://www3.nd.edu/~kogge/courses/cse30151-fa17/Public/other/tikz_tutorial.pdf
\usetikzlibrary{automata} %, positioning, arrows}
\tikzset{
    -Stealth,
    node distance=3cm, % specifies the minimum distance between two nodes. Change if necessary.
    every state/.style={thick, fill=gray!10, shape=ellipse}, % sets the properties for each ’state’ node
    initial text=$ $, % sets the text that appears on the start arrow
    % Un tipo de nodo, que es error, que lo pone rojo
    error/.style={thick, fill=red!20},
}


\begin{document}

    % 1. Foto de fondo
    % 2. Título
    % 3. Encabezado Izquierdo
    % 4. Color de fondo
    % 5. Coord x del titulo
    % 6. Coord y del titulo
    % 7. Fecha

    
    % 1. Foto de fondo
% 2. Título
% 3. Encabezado Izquierdo
% 4. Color de fondo
% 5. Coord x del titulo
% 6. Coord y del titulo
% 7. Fecha

\newcommand{\portada}[7]{

    \portadaBase{#1}{#2}{#3}{#4}{#5}{#6}{#7}
    \portadaBook{#1}{#2}{#3}{#4}{#5}{#6}{#7}
}

\newcommand{\portadaExamen}[7]{

    \portadaBase{#1}{#2}{#3}{#4}{#5}{#6}{#7}
    \portadaArticle{#1}{#2}{#3}{#4}{#5}{#6}{#7}
}




\newcommand{\portadaBase}[7]{

    % Tiene la portada principal y la licencia Creative Commons
    
    % 1. Foto de fondo
    % 2. Título
    % 3. Encabezado Izquierdo
    % 4. Color de fondo
    % 5. Coord x del titulo
    % 6. Coord y del titulo
    % 7. Fecha
    
    
    \thispagestyle{empty}               % Sin encabezado ni pie de página
    \newgeometry{margin=0cm}        % Márgenes nulos para la primera página
    
    
    % Encabezado
    \fancyhead[L]{\helv #3}
    \fancyhead[R]{\helv \nouppercase{\leftmark}}
    
    
    \pagecolor{#4}        % Color de fondo para la portada
    
    \begin{figure}[p]
        \centering
        \transparent{0.3}           % Opacidad del 30% para la imagen
        
        \includegraphics[width=\paperwidth, keepaspectratio]{assets/#1}
    
        \begin{tikzpicture}[remember picture, overlay]
            \node[anchor=north west, text=white, opacity=1, font=\fontsize{60}{90}\selectfont\bfseries\sffamily, align=left] at (#5, #6) {#2};
            
            \node[anchor=south east, text=white, opacity=1, font=\fontsize{12}{18}\selectfont\sffamily, align=right] at (9.7, 3) {\textbf{\href{https://losdeldgiim.github.io/}{Los Del DGIIM}}};
            
            \node[anchor=south east, text=white, opacity=1, font=\fontsize{12}{15}\selectfont\sffamily, align=right] at (9.7, 1.8) {Doble Grado en Ingeniería Informática y Matemáticas\\Universidad de Granada};
        \end{tikzpicture}
    \end{figure}
    
    
    \restoregeometry        % Restaurar márgenes normales para las páginas subsiguientes
    \pagecolor{white}       % Restaurar el color de página
    
    
    \newpage
    \thispagestyle{empty}               % Sin encabezado ni pie de página
    \begin{tikzpicture}[remember picture, overlay]
        \node[anchor=south west, inner sep=3cm] at (current page.south west) {
            \begin{minipage}{0.5\paperwidth}
                \href{https://creativecommons.org/licenses/by-nc-nd/4.0/}{
                    \includegraphics[height=2cm]{assets/Licencia.png}
                }\vspace{1cm}\\
                Esta obra está bajo una
                \href{https://creativecommons.org/licenses/by-nc-nd/4.0/}{
                    Licencia Creative Commons Atribución-NoComercial-SinDerivadas 4.0 Internacional (CC BY-NC-ND 4.0).
                }\\
    
                Eres libre de compartir y redistribuir el contenido de esta obra en cualquier medio o formato, siempre y cuando des el crédito adecuado a los autores originales y no persigas fines comerciales. 
            \end{minipage}
        };
    \end{tikzpicture}
    
    
    
    % 1. Foto de fondo
    % 2. Título
    % 3. Encabezado Izquierdo
    % 4. Color de fondo
    % 5. Coord x del titulo
    % 6. Coord y del titulo
    % 7. Fecha


}


\newcommand{\portadaBook}[7]{

    % 1. Foto de fondo
    % 2. Título
    % 3. Encabezado Izquierdo
    % 4. Color de fondo
    % 5. Coord x del titulo
    % 6. Coord y del titulo
    % 7. Fecha

    % Personaliza el formato del título
    \pretitle{\begin{center}\bfseries\fontsize{42}{56}\selectfont}
    \posttitle{\par\end{center}\vspace{2em}}
    
    % Personaliza el formato del autor
    \preauthor{\begin{center}\Large}
    \postauthor{\par\end{center}\vfill}
    
    % Personaliza el formato de la fecha
    \predate{\begin{center}\huge}
    \postdate{\par\end{center}\vspace{2em}}
    
    \title{#2}
    \author{\href{https://losdeldgiim.github.io/}{Los Del DGIIM}}
    \date{Granada, #7}
    \maketitle
    
    \tableofcontents
}




\newcommand{\portadaArticle}[7]{

    % 1. Foto de fondo
    % 2. Título
    % 3. Encabezado Izquierdo
    % 4. Color de fondo
    % 5. Coord x del titulo
    % 6. Coord y del titulo
    % 7. Fecha

    % Personaliza el formato del título
    \pretitle{\begin{center}\bfseries\fontsize{42}{56}\selectfont}
    \posttitle{\par\end{center}\vspace{2em}}
    
    % Personaliza el formato del autor
    \preauthor{\begin{center}\Large}
    \postauthor{\par\end{center}\vspace{3em}}
    
    % Personaliza el formato de la fecha
    \predate{\begin{center}\huge}
    \postdate{\par\end{center}\vspace{5em}}
    
    \title{#2}
    \author{\href{https://losdeldgiim.github.io/}{Los Del DGIIM}}
    \date{Granada, #7}
    \thispagestyle{empty}               % Sin encabezado ni pie de página
    \maketitle
    \vfill
}
    \portadaExamen{etsiitA4.jpg}{Modelos de\\Computación\\Examen IX}{MC. Examen IX}{MidnightBlue}{-8}{28}{2024-2025}{Arturo Olivares Martos}

    \begin{description}
        \item[Asignatura] Modelos de Computación
        \item[Curso Académico] 2024-25.
        \item[Grado] Doble Grado en Ingeniería Informática y Matemáticas.
        \item[Grupo] A2.
        \item[Profesor] Serafín Moral Callejón.
        \item[Descripción] Parcial Temas 3 y 4.
        \item[Fecha] 12 de diciembre de 2024.
        \item[Duración] 60 minutos.    
    \end{description}
    \newpage

\begin{ejercicio}
    Sobre el alfabeto $A=\{0,1\}$, se pide:
    \begin{enumerate}
        \item Construir un Autómata Finito Determinista para el lenguaje:
            \begin{equation*}
                L_1 = \{u\in {\{0,1\}}^{\ast} \mid u \text{\ no contiene la subcadena ``} 010\text{''}\}
            \end{equation*}

            El AFD se puede ver en la Figura~\ref{fig:afd1}.
            \begin{figure}
                \centering
                \begin{tikzpicture}
                    \node[state, initial, accepting] (q0) {$q_0$};
                    \node[state, right of=q0, accepting] (q1) {$q_1$};
                    \node[state, right of=q1, accepting] (q2) {$q_2$};
                    \node[state, right of=q2, error] (E1) {$E_1$};
                    \draw   (q0) edge[loop above] node{$1$} (q0)
                            (q0) edge[above] node{$0$} (q1)
                            (q1) edge[above] node{$1$} (q2)
                            (q1) edge[loop above] node{$0$} (q1)
                            (q2) edge[above] node{$0$} (E1)
                            (q2) edge[below, bend left] node{$1$} (q0)
                            (E1) edge[loop above] node{$0,1$} (E1);
                \end{tikzpicture}
                \caption{Autómata Finito Determinista para $L_1$.}
                \label{fig:afd1}
            \end{figure}
        \item Construir un Autómata Finito Determinista para el lenguaje $L_2$ dado por la expresión regular:
            \begin{equation*}
                01{(10+0)}^{\ast}
            \end{equation*}

            El AFD se puede ver en la Figura~\ref{fig:afd2}.
            \begin{figure}
                \centering
                \begin{tikzpicture}
                    \node[state, initial] (p0) {$p_0$};
                    \node[state, right of=p0] (p1) {$p_1$};
                    \node[state, right of=p1, accepting] (p2) {$p_2$};
                    \node[state, right of=p2] (p3) {$p_3$};
                    \node[state, below of=p2, error] (E2) {$E_2$};
                    \draw   (p0) edge[above] node{$0$} (p1)
                            (p0) edge[above] node{$1$} (E2)
                            (p1) edge[above] node{$1$} (p2)
                            (p1) edge[above] node{$0$} (E2)
                            (p2) edge[above] node{$1$} (p3)
                            (p2) edge[loop above] node{$0$} (p2)
                            (p3) edge[above, bend right] node{$0$} (p2)
                            (p3) edge[above] node{$1$} (E2)
                            (E2) edge[loop left] node{$0,1$} (E2);
                \end{tikzpicture}
                \caption{Autómata Finito Determinista para $L_2$.}
                \label{fig:afd2}
            \end{figure}
        \item Construir un Autómata Finito Determinista que acepte el lenguaje $L_1\cap L_2$ y minimizarlo.
        
        En primer lugar, todos los estados de error son indistinguibles, ya que desde ellos no se puede salir y no se puede alcanzar ninguno final. Notando por $E$ al estado de error, tenemos que:
        \begin{equation*}
            E = \left\{(q_iE_2,E_1p_j) \mid i\in \{0,1,2\}, j\in \{0,1,2,3\}\right\}
        \end{equation*}

        Por tanto, el AFD que acepta $L_1\cap L_2$ es el que se muestra en la Figura~\ref{fig:afd3}. No obstante, notando por $\equiv$ la relación de equivalencia de estados indistinguibles, vemos que:
        \begin{equation*}
            E\equiv q_2p_3
        \end{equation*}
        ya que, desde $q_2p_3$, vamos a $E$ con cualquier símbolo. Por tanto, el AFD minimizado es el que se muestra en la Figura~\ref{fig:afd4}.
        \begin{figure}
            \centering
            \begin{tikzpicture}
                \node[state, initial] (q0p0) {$q_0p_0$};
                \node[state, right of=q0p0] (q1p1) {$q_1p_1$};
                \node[state, right of=q1p1, accepting] (q2p2) {$q_2p_2$};
                \node[state, right of=q2p2] (q0p3) {$q_0p_3$};
                \node[state, right of=q0p3, accepting] (q1p2) {$q_1p_2$};
                \node[state, right of=q1p2] (q2p3) {$q_2p_3$};
                \node[state, below of=q2p2, error] (E) {$E$};
                \draw   (q0p0) edge[above] node{$0$} (q1p1)
                        (q0p0) edge[above] node{$1$} (E)
                        (q1p1) edge[above] node{$1$} (q2p2)
                        (q1p1) edge[above] node{$0$} (E)
                        (q2p2) edge[above] node{$1$} (q0p3)
                        (q2p2) edge[right] node{$0$} (E)
                        (q0p3) edge[above] node{$0$} (q1p2)
                        (q0p3) edge[above] node{$1$} (E)
                        (q1p2) edge[above] node{$1$} (q2p3)
                        (q1p2) edge[loop above] node{$0$} (q1p2)
                        (q2p3) edge[above] node{$0,1$} (E)
                        (E) edge[loop left] node{$0,1$} (E);
            \end{tikzpicture}
            \caption{Autómata Finito Determinista para $L_1\cap L_2$.}
            \label{fig:afd3}
        \end{figure}
        \begin{figure}
            \centering
            \begin{tikzpicture}
                \node[state, initial] (q0p0) {$q_0p_0$};
                \node[state, right of=q0p0] (q1p1) {$q_1p_1$};
                \node[state, right of=q1p1, accepting] (q2p2) {$q_2p_2$};
                \node[state, right of=q2p2] (q0p3) {$q_0p_3$};
                \node[state, right of=q0p3, accepting] (q1p2) {$q_1p_2$};
                \node[state, below of=q2p2, error] (E) {$E$};
                \draw   (q0p0) edge[above] node{$0$} (q1p1)
                        (q0p0) edge[above] node{$1$} (E)
                        (q1p1) edge[above] node{$1$} (q2p2)
                        (q1p1) edge[above] node{$0$} (E)
                        (q2p2) edge[above] node{$1$} (q0p3)
                        (q2p2) edge[right] node{$0$} (E)
                        (q0p3) edge[above] node{$0$} (q1p2)
                        (q0p3) edge[above] node{$1$} (E)
                        (q1p2) edge[above] node{$1$} (E)
                        (q1p2) edge[loop above] node{$0$} (q1p2)
                        (E) edge[loop left] node{$0,1$} (E);
            \end{tikzpicture}
            \caption{Autómata Finito Determinista minimal para $L_1\cap L_2$.}
            \label{fig:afd4}
        \end{figure}

        Además, como no hay ciclos, sería fácil comprobar que el autómata de la ~Figura~\ref{fig:afd4} es minimal.
    \end{enumerate}
\end{ejercicio}

\begin{ejercicio}
    Sea la gramática $G=(\{S\},\{a,b,c\},P,S)$ con $P$ el conjunto que contiene las producciones:
    \begin{equation*}
        S \rightarrow aaSbb\ |\ bbSaa\ |\ aaaSbbb\ |\ bbbAaaa\ |\ ccc
    \end{equation*}
    \begin{enumerate}
        \item Demuestra que $\cc{L}(G)$ no es regular.
        
        Lo haremos mediante el Lema de Bombeo. Para cada $n\in\mathbb{N}$, consideramos la palabra $z=a^{2n}c^3b^{2n}\in \cc{L}(G)$, con $|z|=4n+3\geq n$.
        Para cada descomposición $z=uvw$ con $|uv|\leq n$ y $|v|\geq 1$, tenemos que:
        \begin{equation*}
            u=a^k,\ v=a^l,\ w=a^{2n-k-l}c^3b^{2n}\qquad \text{con } 0\leq k+l\leq n,\ l\geq 1
        \end{equation*}

        Bombeando con $i=2$, obtenemos la palabra:
        \begin{equation*}
            uv^2w=a^{2n + l}c^3b^{2n}\notin \cc{L}(G)
        \end{equation*}
        la cual sabemos que no pertenece a $\cc{L}(G)$, ya que $2n+l\neq 2n$ por ser $l\geq 1$, y todas las palabras de $\cc{L}(G)$ tienen la misma longitud antes de $c^3$ y después.

        Por tanto, por el recíproco del Lema de Bombeo, $\cc{L}(G)$ no es regular.
        \item Demuestra que $G$ es una gramática ambigua.
        
        Sea la palabra $z=a^6c^3b^6\in \cc{L}(G)$. Podemos derivarla de dos formas distintas, como se muestra en la Figura~\ref{fig:arboles}. Por tanto, $G$ es ambigua.
        \begin{figure}
            \centering
                \begin{forest}for tree={
                    edge={-}, % Hace que las líneas no terminen en punta
                    fit=rectangle, % Ajusta el tamaño del rectángulo al texto
                }
                    [$S$
                        [$a$]
                        [$a$]
                        [$S$
                            [$a$]
                            [$a$]
                            [$S$
                                [$a$]
                                [$a$]
                                [$S$
                                    [$c$]
                                    [$c$]
                                    [$c$]
                                ]
                                [$b$]
                                [$b$]
                            ]
                            [$b$]
                            [$b$]
                        ]
                        [$b$]
                        [$b$]
                    ]
                \end{forest}
            \begin{forest}for tree={
                edge={-}, % Hace que las líneas no terminen en punta
                fit=rectangle, % Ajusta el tamaño del rectángulo al texto
            }
                [$S$
                    [$a$]
                    [$a$]
                    [$a$]
                    [$S$
                        [$a$]
                        [$a$]
                        [$a$]
                        [$S$
                            [$c$]
                            [$c$]
                            [$c$]
                        ]
                        [$b$]
                        [$b$]
                        [$b$]
                    ]
                    [$b$]
                    [$b$]
                    [$b$]
                ]
            \end{forest}
            \caption{Árboles de derivación para $z=a^6c^3b^6$.}
            \label{fig:arboles}
        \end{figure}
        \item Dar una gramática no ambigua que genere el lenguaje $\cc{L}(G)$.
        
        Consideramos la gramática $G'=(\{S', S_A, S_B, A, B\},\{a,b,c\},P',S')$ con $P'$ el conjunto que contiene las producciones:
        \begin{equation*}
            \begin{aligned}
                S' &\to S_A\mid S_B\mid c^3\\
                S_A &\to aaAbb\\
                S_B &\to bbBaa\\
                A &\to aAb \mid S_B\mid c^3\\
                B &\to bBa \mid S_A\mid c^3
            \end{aligned}
        \end{equation*}

        Es directo ver que $\cc{L}(G')=\cc{L}(G)$ y que $G'$ es una gramática no ambigua.
    \end{enumerate}
\end{ejercicio}

\end{document}
