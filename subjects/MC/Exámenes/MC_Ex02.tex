\documentclass[12pt]{article}

% Idioma y codificación
\usepackage[spanish, es-tabla]{babel}       %es-tabla para que se titule "Tabla"
\usepackage[utf8]{inputenc}

% Márgenes
\usepackage[a4paper,top=3cm,bottom=2.5cm,left=3cm,right=3cm]{geometry}

% Comentarios de bloque
\usepackage{verbatim}

% Paquetes de links
\usepackage[hidelinks]{hyperref}    % Permite enlaces
\usepackage{url}                    % redirecciona a la web

% Más opciones para enumeraciones
\usepackage{enumitem}

% Personalizar la portada
\usepackage{titling}

% Paquetes de tablas
\usepackage{multirow}


%------------------------------------------------------------------------

%Paquetes de figuras
\usepackage{caption}
\usepackage{subcaption} % Figuras al lado de otras
\usepackage{float}      % Poner figuras en el sitio indicado H.


% Paquetes de imágenes
\usepackage{graphicx}       % Paquete para añadir imágenes
\usepackage{transparent}    % Para manejar la opacidad de las figuras

% Paquete para usar colores
\usepackage[dvipsnames]{xcolor}
\usepackage{pagecolor}      % Para cambiar el color de la página

% Habilita tamaños de fuente mayores
\usepackage{fix-cm}

% Para los gráficos
\usepackage{tikz}

% Para poder situar los nodos en los grafos
\usetikzlibrary{positioning}


%------------------------------------------------------------------------

% Paquetes de matemáticas
\usepackage{mathtools, amsfonts, amssymb, mathrsfs}
\usepackage[makeroom]{cancel}     % Simplificar tachando
\usepackage{polynom}    % Divisiones y Ruffini
\usepackage{units} % Para poner fracciones diagonales con \nicefrac

\usepackage{pgfplots}   %Representar funciones
\pgfplotsset{compat=1.18}  % Versión 1.18

\usepackage{tikz-cd}    % Para usar diagramas de composiciones
\usetikzlibrary{calc}   % Para usar cálculo de coordenadas en tikz

%Definición de teoremas, etc.
\usepackage{amsthm}
%\swapnumbers   % Intercambia la posición del texto y de la numeración

\theoremstyle{plain}

\makeatletter
\@ifclassloaded{article}{
  \newtheorem{teo}{Teorema}[section]
}{
  \newtheorem{teo}{Teorema}[chapter]  % Se resetea en cada chapter
}
\makeatother

\newtheorem{coro}{Corolario}[teo]           % Se resetea en cada teorema
\newtheorem{prop}[teo]{Proposición}         % Usa el mismo contador que teorema
\newtheorem{lema}[teo]{Lema}                % Usa el mismo contador que teorema

\theoremstyle{remark}
\newtheorem*{observacion}{Observación}

\theoremstyle{definition}

\makeatletter
\@ifclassloaded{article}{
  \newtheorem{definicion}{Definición} [section]     % Se resetea en cada chapter
}{
  \newtheorem{definicion}{Definición} [chapter]     % Se resetea en cada chapter
}
\makeatother

\newtheorem*{notacion}{Notación}
\newtheorem*{ejemplo}{Ejemplo}
\newtheorem*{ejercicio*}{Ejercicio}             % No numerado
\newtheorem{ejercicio}{Ejercicio} [section]     % Se resetea en cada section


% Modificar el formato de la numeración del teorema "ejercicio"
\renewcommand{\theejercicio}{%
  \ifnum\value{section}=0 % Si no se ha iniciado ninguna sección
    \arabic{ejercicio}% Solo mostrar el número de ejercicio
  \else
    \thesection.\arabic{ejercicio}% Mostrar número de sección y número de ejercicio
  \fi
}


% \renewcommand\qedsymbol{$\blacksquare$}         % Cambiar símbolo QED
%------------------------------------------------------------------------

% Paquetes para encabezados
\usepackage{fancyhdr}
\pagestyle{fancy}
\fancyhf{}

\newcommand{\helv}{ % Modificación tamaño de letra
\fontfamily{}\fontsize{12}{12}\selectfont}
\setlength{\headheight}{15pt} % Amplía el tamaño del índice


%\usepackage{lastpage}   % Referenciar última pag   \pageref{LastPage}
\fancyfoot[C]{\thepage}

%------------------------------------------------------------------------

% Conseguir que no ponga "Capítulo 1". Sino solo "1."
\makeatletter
\@ifclassloaded{book}{
  \renewcommand{\chaptermark}[1]{\markboth{\thechapter.\ #1}{}} % En el encabezado
    
  \renewcommand{\@makechapterhead}[1]{%
  \vspace*{50\p@}%
  {\parindent \z@ \raggedright \normalfont
    \ifnum \c@secnumdepth >\m@ne
      \huge\bfseries \thechapter.\hspace{1em}\ignorespaces
    \fi
    \interlinepenalty\@M
    \Huge \bfseries #1\par\nobreak
    \vskip 40\p@
  }}
}
\makeatother

%------------------------------------------------------------------------
% Paquetes de cógido
\usepackage{minted}
\renewcommand\listingscaption{Código fuente}

\usepackage{fancyvrb}
% Personaliza el tamaño de los números de línea
\renewcommand{\theFancyVerbLine}{\small\arabic{FancyVerbLine}}

% Estilo para C++
\newminted{cpp}{
    frame=lines,
    framesep=2mm,
    baselinestretch=1.2,
    linenos,
    escapeinside=||
}

% para minted
\definecolor{LightGray}{rgb}{0.95,0.95,0.92}
\setminted{
    linenos=true,
    stepnumber=5,
    numberfirstline=true,
    autogobble,
    breaklines=true,
    breakautoindent=true,
    breaksymbolleft=,
    breaksymbolright=,
    breaksymbolindentleft=0pt,
    breaksymbolindentright=0pt,
    breaksymbolsepleft=0pt,
    breaksymbolsepright=0pt,
    fontsize=\footnotesize,
    bgcolor=LightGray,
    numbersep=10pt
}


\usepackage{listings} % Para incluir código desde un archivo

\renewcommand\lstlistingname{Código Fuente}
\renewcommand\lstlistlistingname{Índice de Códigos Fuente}

% Definir colores
\definecolor{vscodepurple}{rgb}{0.5,0,0.5}
\definecolor{vscodeblue}{rgb}{0,0,0.8}
\definecolor{vscodegreen}{rgb}{0,0.5,0}
\definecolor{vscodegray}{rgb}{0.5,0.5,0.5}
\definecolor{vscodebackground}{rgb}{0.97,0.97,0.97}
\definecolor{vscodelightgray}{rgb}{0.9,0.9,0.9}

% Configuración para el estilo de C similar a VSCode
\lstdefinestyle{vscode_C}{
  backgroundcolor=\color{vscodebackground},
  commentstyle=\color{vscodegreen},
  keywordstyle=\color{vscodeblue},
  numberstyle=\tiny\color{vscodegray},
  stringstyle=\color{vscodepurple},
  basicstyle=\scriptsize\ttfamily,
  breakatwhitespace=false,
  breaklines=true,
  captionpos=b,
  keepspaces=true,
  numbers=left,
  numbersep=5pt,
  showspaces=false,
  showstringspaces=false,
  showtabs=false,
  tabsize=2,
  frame=tb,
  framerule=0pt,
  aboveskip=10pt,
  belowskip=10pt,
  xleftmargin=10pt,
  xrightmargin=10pt,
  framexleftmargin=10pt,
  framexrightmargin=10pt,
  framesep=0pt,
  rulecolor=\color{vscodelightgray},
  backgroundcolor=\color{vscodebackground},
}

%------------------------------------------------------------------------

% Comandos definidos
\newcommand{\bb}[1]{\mathbb{#1}}
\newcommand{\cc}[1]{\mathcal{#1}}

% I prefer the slanted \leq
\let\oldleq\leq % save them in case they're every wanted
\let\oldgeq\geq
\renewcommand{\leq}{\leqslant}
\renewcommand{\geq}{\geqslant}

% Si y solo si
\newcommand{\sii}{\iff}

% Letras griegas
\newcommand{\eps}{\epsilon}
\newcommand{\veps}{\varepsilon}
\newcommand{\lm}{\lambda}

\newcommand{\ol}{\overline}
\newcommand{\ul}{\underline}
\newcommand{\wt}{\widetilde}
\newcommand{\wh}{\widehat}

\let\oldvec\vec
\renewcommand{\vec}{\overrightarrow}

% Derivadas parciales
\newcommand{\del}[2]{\frac{\partial #1}{\partial #2}}
\newcommand{\Del}[3]{\frac{\partial^{#1} #2}{\partial #3^{#1}}}
\newcommand{\deld}[2]{\dfrac{\partial #1}{\partial #2}}
\newcommand{\Deld}[3]{\dfrac{\partial^{#1} #2}{\partial #3^{#1}}}


\newcommand{\AstIg}{\stackrel{(\ast)}{=}}
\newcommand{\Hop}{\stackrel{L'H\hat{o}pital}{=}}

\newcommand{\red}[1]{{\color{red}#1}} % Para integrales, destacar los cambios.

% Método de integración
\newcommand{\MetInt}[2]{
    \left[\begin{array}{c}
        #1 \\ #2
    \end{array}\right]
}

% Declarar aplicaciones
% 1. Nombre aplicación
% 2. Dominio
% 3. Codominio
% 4. Variable
% 5. Imagen de la variable
\newcommand{\Func}[5]{
    \begin{equation*}
        \begin{array}{rrll}
            #1:& #2 & \longrightarrow & #3\\
               & #4 & \longmapsto & #5
        \end{array}
    \end{equation*}
}

%------------------------------------------------------------------------


% Para poder incluir árboles
\usepackage{forest}
\usepackage{booktabs}

% Para poder añadir autómatas
% https://www3.nd.edu/~kogge/courses/cse30151-fa17/Public/other/tikz_tutorial.pdf
\usetikzlibrary{automata} %, positioning, arrows}
\tikzset{
    -Stealth,
    node distance=3cm, % specifies the minimum distance between two nodes. Change if necessary.
    every state/.style={thick, fill=gray!10, shape=ellipse}, % sets the properties for each ’state’ node
    initial text=$ $, % sets the text that appears on the start arrow
    % Un tipo de nodo, que es error, que lo pone rojo
    error/.style={thick, fill=red!20},
}


\begin{document}

    % 1. Foto de fondo
    % 2. Título
    % 3. Encabezado Izquierdo
    % 4. Color de fondo
    % 5. Coord x del titulo
    % 6. Coord y del titulo
    % 7. Fecha

    
    % 1. Foto de fondo
% 2. Título
% 3. Encabezado Izquierdo
% 4. Color de fondo
% 5. Coord x del titulo
% 6. Coord y del titulo
% 7. Fecha

\newcommand{\portada}[7]{

    \portadaBase{#1}{#2}{#3}{#4}{#5}{#6}{#7}
    \portadaBook{#1}{#2}{#3}{#4}{#5}{#6}{#7}
}

\newcommand{\portadaExamen}[7]{

    \portadaBase{#1}{#2}{#3}{#4}{#5}{#6}{#7}
    \portadaArticle{#1}{#2}{#3}{#4}{#5}{#6}{#7}
}




\newcommand{\portadaBase}[7]{

    % Tiene la portada principal y la licencia Creative Commons
    
    % 1. Foto de fondo
    % 2. Título
    % 3. Encabezado Izquierdo
    % 4. Color de fondo
    % 5. Coord x del titulo
    % 6. Coord y del titulo
    % 7. Fecha
    
    
    \thispagestyle{empty}               % Sin encabezado ni pie de página
    \newgeometry{margin=0cm}        % Márgenes nulos para la primera página
    
    
    % Encabezado
    \fancyhead[L]{\helv #3}
    \fancyhead[R]{\helv \nouppercase{\leftmark}}
    
    
    \pagecolor{#4}        % Color de fondo para la portada
    
    \begin{figure}[p]
        \centering
        \transparent{0.3}           % Opacidad del 30% para la imagen
        
        \includegraphics[width=\paperwidth, keepaspectratio]{assets/#1}
    
        \begin{tikzpicture}[remember picture, overlay]
            \node[anchor=north west, text=white, opacity=1, font=\fontsize{60}{90}\selectfont\bfseries\sffamily, align=left] at (#5, #6) {#2};
            
            \node[anchor=south east, text=white, opacity=1, font=\fontsize{12}{18}\selectfont\sffamily, align=right] at (9.7, 3) {\textbf{\href{https://losdeldgiim.github.io/}{Los Del DGIIM}}};
            
            \node[anchor=south east, text=white, opacity=1, font=\fontsize{12}{15}\selectfont\sffamily, align=right] at (9.7, 1.8) {Doble Grado en Ingeniería Informática y Matemáticas\\Universidad de Granada};
        \end{tikzpicture}
    \end{figure}
    
    
    \restoregeometry        % Restaurar márgenes normales para las páginas subsiguientes
    \pagecolor{white}       % Restaurar el color de página
    
    
    \newpage
    \thispagestyle{empty}               % Sin encabezado ni pie de página
    \begin{tikzpicture}[remember picture, overlay]
        \node[anchor=south west, inner sep=3cm] at (current page.south west) {
            \begin{minipage}{0.5\paperwidth}
                \href{https://creativecommons.org/licenses/by-nc-nd/4.0/}{
                    \includegraphics[height=2cm]{assets/Licencia.png}
                }\vspace{1cm}\\
                Esta obra está bajo una
                \href{https://creativecommons.org/licenses/by-nc-nd/4.0/}{
                    Licencia Creative Commons Atribución-NoComercial-SinDerivadas 4.0 Internacional (CC BY-NC-ND 4.0).
                }\\
    
                Eres libre de compartir y redistribuir el contenido de esta obra en cualquier medio o formato, siempre y cuando des el crédito adecuado a los autores originales y no persigas fines comerciales. 
            \end{minipage}
        };
    \end{tikzpicture}
    
    
    
    % 1. Foto de fondo
    % 2. Título
    % 3. Encabezado Izquierdo
    % 4. Color de fondo
    % 5. Coord x del titulo
    % 6. Coord y del titulo
    % 7. Fecha


}


\newcommand{\portadaBook}[7]{

    % 1. Foto de fondo
    % 2. Título
    % 3. Encabezado Izquierdo
    % 4. Color de fondo
    % 5. Coord x del titulo
    % 6. Coord y del titulo
    % 7. Fecha

    % Personaliza el formato del título
    \pretitle{\begin{center}\bfseries\fontsize{42}{56}\selectfont}
    \posttitle{\par\end{center}\vspace{2em}}
    
    % Personaliza el formato del autor
    \preauthor{\begin{center}\Large}
    \postauthor{\par\end{center}\vfill}
    
    % Personaliza el formato de la fecha
    \predate{\begin{center}\huge}
    \postdate{\par\end{center}\vspace{2em}}
    
    \title{#2}
    \author{\href{https://losdeldgiim.github.io/}{Los Del DGIIM}}
    \date{Granada, #7}
    \maketitle
    
    \tableofcontents
}




\newcommand{\portadaArticle}[7]{

    % 1. Foto de fondo
    % 2. Título
    % 3. Encabezado Izquierdo
    % 4. Color de fondo
    % 5. Coord x del titulo
    % 6. Coord y del titulo
    % 7. Fecha

    % Personaliza el formato del título
    \pretitle{\begin{center}\bfseries\fontsize{42}{56}\selectfont}
    \posttitle{\par\end{center}\vspace{2em}}
    
    % Personaliza el formato del autor
    \preauthor{\begin{center}\Large}
    \postauthor{\par\end{center}\vspace{3em}}
    
    % Personaliza el formato de la fecha
    \predate{\begin{center}\huge}
    \postdate{\par\end{center}\vspace{5em}}
    
    \title{#2}
    \author{\href{https://losdeldgiim.github.io/}{Los Del DGIIM}}
    \date{Granada, #7}
    \thispagestyle{empty}               % Sin encabezado ni pie de página
    \maketitle
    \vfill
}
    \portadaExamen{etsiitA4.jpg}{Modelos de\\Computación\\Examen II}{MC. Examen II}{MidnightBlue}{-8}{28}{2024-2025}{Arturo Olivares Martos}

    \begin{description}
        \item[Asignatura] Modelos de Computación
        \item[Curso Académico] 2021-22.
        \item[Grado] Doble Grado en Ingeniería Informática y Matemáticas.
        %\item[Grupo] B.
        %\item[Profesor] Rafael Ortega Ríos.
        \item[Descripción] Parcial Tema 2.
        %\item[Fecha] 22 de marzo de 2018.
        %\item[Duración] 60 minutos.    
    \end{description}
    \newpage
    
    \begin{ejercicio} \label{ej:1}
        Dar un AFD que acepte el siguiente lenguaje:
        \begin{equation*}
            L=\{0^n1^m \mid n-m \text{ es múltiplo de } 3\}.
        \end{equation*}
        Usaremos la siguiente notación para los estados:
        \begin{itemize}
            \item $q_i$: $n \operatorname{mod} 3 = i$, $m=0$.
            \item $q_{ij}$: $n \operatorname{mod} 3 = i$, $m \operatorname{mod} 3 = j$.
        \end{itemize}

        Necesitamos que $n-m$ sea múltiplo de $3$. Para ello, neesitamos que el resto de dividir cada uno de ellos entre tres coincida, ya que se va a compensar haciendo que la resta sea múltiplo de tres. Tenemos por tanto el AFD de la Figura \ref{fig:ej:1}.

        Notemos que todas las conexiones de error, para evitar que una vez se han introducido $1$'s se pueda volver a introducir un $0$, se han hecho dibujado con menor opacidad para no saturar el diseño. Tal solo se ha establecido $\delta(q_{ij}, 0) = E$ para todo $i,j$.
        \begin{figure}[H]
            \centering
            \begin{tikzpicture}
                \node[state,initial, accepting] (q0) {$q_0$};
                \node[state,below of=q0] (q1) {$q_1$};
                \node[state,below of=q1] (q2) {$q_2$};

                \node[state,right of=q0] (q01) {$q_{01}$};
                \node[state,right of=q01] (q02) {$q_{02}$};
                \node[state,right of=q02, accepting] (q00) {$q_{00}$};

                \node[state,right of=q1, accepting] (q11) {$q_{11}$};
                \node[state,right of=q11] (q12) {$q_{12}$};
                \node[state,right of=q12] (q10) {$q_{10}$};

                \node[state,right of=q2] (q21) {$q_{21}$};
                \node[state,right of=q21, accepting] (q22) {$q_{22}$};
                \node[state,right of=q22] (q20) {$q_{20}$};

                \node[state, error, right of=q10] (qerror) {$E$};

                % Conexiones directas. Verticalmente 0's
                \draw   (q0) edge[left] node{0} (q1)
                        (q1) edge[left] node{0} (q2)
                        (q2) edge[left, bend left] node{0} (q0);

                % Conexiones directas. Horizontalmente 1's
                \draw   (q0) edge[above] node{1} (q01)
                        (q01) edge[above] node{1} (q02)
                        (q02) edge[above] node{1} (q00)
                        (q00) edge[above, bend right] node{1} (q01)
                        (q1) edge[above] node{1} (q11)
                        (q11) edge[above] node{1} (q12)
                        (q12) edge[above] node{1} (q10)
                        (q10) edge[above, bend right] node{1} (q11)
                        (q2) edge[below] node{1} (q21)
                        (q21) edge[below] node{1} (q22)
                        (q22) edge[below] node{1} (q20)
                        (q20) edge[below, bend left] node{1} (q21);

                % Conexiones al Error
                \draw[opacity=0.6]
                        (q01) edge[below] node[pos=0.07]{0} (qerror)
                        (q02) edge[below] node[pos=0.07]{0} (qerror)
                        (q00) edge[below] node[pos=0.07]{0} (qerror)
                        (q11) edge[below, bend right] node[pos=0.05]{0} (qerror)
                        (q12) edge[below, bend right] node[pos=0.07]{0} (qerror)
                        (q10) edge[above] node[pos=0.1]{0} (qerror)
                        (q21) edge[above] node[pos=0.07]{0} (qerror)
                        (q22) edge[above] node[pos=0.07]{0} (qerror)
                        (q20) edge[above] node[pos=0.07]{0} (qerror)
                        (qerror) edge[loop above] node{0,1} (qerror);
            \end{tikzpicture}
            \caption{AFD que acepta el lenguaje $L$ del ejercicio \ref{ej:1}.}
            \label{fig:ej:1}
        \end{figure}
    \end{ejercicio}

    \begin{ejercicio} \label{ej:2}
        Considerar el lenguaje siguiente:
        \begin{equation*}
            L=\{w \in \{0,1\}^* \mid \text{El tecer símbolo empezando por la derecha es } 1\}.
        \end{equation*}
        \begin{enumerate}
            \item Dar un AFD que acepte $L$.
            
            Podríamos optar con construir el AFD de forma directa, pero construiremos un AFND que acepte el lenguaje y luego lo convertiremos a AFD. El AFND se muestra en la Figura \ref{fig:ej:2:AFND}, cuyos estados son:
            \begin{itemize}
                \item $q_0$: No estamos en la cadena final, por lo que podemos leer $0$'s y $1$'s.
                \item $q_1$: Acabo de empezar la cadena final. He leído un $1$.
                \item $q_2$: Estoy en la cadena final. El leído el $1$ y el segundo símbolo.
                \item $q_3$: Hemos terminado la cadena final.
            \end{itemize}
            \begin{figure}[H]
                \centering
                \begin{tikzpicture}
                    \node[state, initial] (q0) {$q_0$};
                    \node[state, right of=q0] (q1) {$q_1$};
                    \node[state, right of=q1] (q2) {$q_2$};
                    \node[state, accepting, right of=q2] (q3) {$q_3$};

                    \draw   (q0) edge[loop above] node{0,1} (q0)
                            (q0) edge[above] node{1} (q1)
                            (q1) edge[above] node{0,1} (q2)
                            (q2) edge[above] node{0,1} (q3);
                \end{tikzpicture}
                \caption{AFND que acepta el lenguaje $L$ del ejercicio \ref{ej:2}.}
                \label{fig:ej:2:AFND}
            \end{figure}

            Convertimos ahora el AFND de la Figura \ref{fig:ej:2:AFND} en un AFD, representado en la Figura \ref{fig:ej:2:AFD}.
            \begin{figure}[H]
                \centering
                \begin{tikzpicture}
                    \node[state, initial] (q0) {$q_0$};
                    \node[state, right of=q0] (q0q1) {$q_0q_1$};
                    \node[state, above right of=q0q1, xshift=3em] (q0q1q2) {$q_0q_1q_2$};
                    \node[state, below right of=q0q1, xshift=3em] (q0q2) {$q_0q_2$};
                    \node[state, above right of=q0q1q2, accepting, xshift=3em] (q0q1q2q3) {$q_0q_1q_2q_3$};
                    \node[state, below right of=q0q1q2, accepting, xshift=3em, yshift=3em] (q0q2q3) {$q_0q_2q_3$};
                    \node[state, above right of=q0q2, accepting, xshift=3em, yshift=-3em] (q0q1q3) {$q_0q_1q_3$};
                    \node[state, below right of=q0q2, accepting, xshift=3em] (q0q3) {$q_0q_3$};

                    \draw   (q0) edge[loop above] node{0} (q0)
                            (q0) edge[above] node{1} (q0q1)
                            (q0q1) edge[above] node{0} (q0q2)
                            (q0q1) edge[above] node{1} (q0q1q2)
                            (q0q2) edge[above] node{0} (q0q3)
                            (q0q2) edge[above] node{1} (q0q1q3)
                            (q0q1q2) edge[above] node{0} (q0q2q3)
                            (q0q1q2) edge[above] node{1} (q0q1q2q3)
                            (q0q1q2q3) edge[right, bend left] node{0} (q0q2q3)
                            (q0q1q2q3) edge[loop above] node{1} (q0q1q2q3)
                            (q0q2q3) edge[right, bend left=45] node{0} (q0q3)
                            (q0q2q3) edge[right, bend left] node[pos=0.8]{1} (q0q1q3)
                            (q0q1q3) edge[below, bend left] node[pos=0.1]{0} (q0q2)
                            (q0q1q3) edge[right] node{1} (q0q1q2)
                            (q0q3) edge[below, bend left] node{1} (q0q1)
                            (q0q3) edge[below, bend left] node{0} (q0);
                \end{tikzpicture}
                \caption{AFD que acepta el lenguaje $L$ del ejercicio \ref{ej:2}.}
                \label{fig:ej:2:AFD}
            \end{figure}

            El AFD de la Figura \ref{fig:ej:2:AFD} acepta el lenguaje $L$, y es idéntico al que podríamos haber razonado de forma directa. Veamos qué representa cada estado:
            \begin{itemize}
                \item $q_0$: No estamos en un candidado a ser cadena final. Si leemos un $1$, empezaremos la que puede ser la cadena final.
                \item $q_0q_1$: Hemos leído un $1$, por lo que hemos empezado la posible cadena final. Llevamos $1$.
                \item $q_0q_1q_2$: Hemos leído un $1$ y un $1$. Llevamos $11$ de cadena final.
                \item $q_0q_2$: Hemos leído un $1$ y un $0$. Llevamos $10$ de cadena final.
                \item $q_0q_1q_2q_3$: Hemos leído un $1$, un $1$ y un $1$. Llevamos $111$ de cadena final.
                \item $q_0q_2q_3$: Hemos leído un $1$, un $1$ y un $0$. Llevamos $110$ de cadena final.
                \item $q_0q_1q_3$: Hemos leído un $1$, un $0$ y un $1$. Llevamos $101$ de cadena final.
                \item $q_0q_3$: Hemos leído un $1$, un $0$ y un $0$. Llevamos $100$ de cadena final.
            \end{itemize}
            Lo complejo de hacerlo de forma directa sería ver las transiciones desde los estados finales. Razonando cuál es la cadena final leída, odríamos haberlo hecho de forma directa, pero el AFND nos ha ayudado a hacerlo de forma algorítmica.

            \item Dar una expresión regular que genere $L$.
            
            Tenemos dos opciones:
            \begin{description}
                \item[De forma directa:] La expresión regular que genera $L$ es:
                \begin{equation*}
                    (0+1)^*\red{1}(0+1)(0+1)
                \end{equation*}

                \item[Pasando de Autómata a Expresión Regular:] Debido al gran número de estados del AFD, vamos a pasar el AFND sin transiciones nulas a una expresión regular. Para ello, establecemos una ecuación por cada estado y resolvemos el sistema:
                \begin{equation*}
                    \begin{cases}
                        q_0 =& 0q_0 + 1q_0 + 1q_1 = (0+1)q_0 + 1q_1 \\
                        q_1 =& 0q_2 + 1q_2 = (0+1)q_2 \\
                        q_2 =& 0q_3 + 1q_3 = (0+1)q_3 \\
                        q_3 =& \varepsilon
                    \end{cases}
                \end{equation*}

                Sustituyendo $q_3$ en la ecuación de $q_2$, llegamos a que:
                \begin{equation*}
                    q_2 = (0+1)\veps = (0+1)
                \end{equation*}

                Sustituyendo $q_2$ en la ecuación de $q_1$, llegamos a que:
                \begin{equation*}
                    q_1 = (0+1)(0+1)
                \end{equation*}

                Sustituyendo $q_1$ en la ecuación de $q_0$, llegamos a que:
                \begin{equation*}
                    q_0 = (0+1)q_0 + 1(0+1)(0+1) = (0+1)^*1(0+1)(0+1)
                \end{equation*}

                Por tanto, la expresión regular que genera $L$ es:
                \begin{equation*}
                    (0+1)^*\red{1}(0+1)(0+1)
                \end{equation*}

                Como podemos ver, hemos llegado a la misma expresión regular que de forma directa.
            \end{description}
            \item Dar una gramática regular por la izquierda que genere $L$.
            
            En primer lugar, invertimos el AFND de la Figura \ref{fig:ej:2:AFND} para obtener el AFND de la Figura \ref{fig:ej:2:AFND:inverso}.
            \begin{figure}[H]
                \centering
                \begin{tikzpicture}
                    \node[state, accepting] (q0) {$q_0$};
                    \node[state, right of=q0] (q1) {$q_1$};
                    \node[state, right of=q1] (q2) {$q_2$};
                    \node[state, initial right, right of=q2] (q3) {$q_3$};

                    \draw   (q0) edge[loop above] node{0,1} (q0)
                            (q1) edge[above] node{1} (q0)
                            (q2) edge[above] node{0,1} (q1)
                            (q3) edge[above] node{0,1} (q2);
                \end{tikzpicture}
                \caption{AFND que acepta el lenguaje $L^{-1}$, siendo $L$ el lenguaje del ejercicio \ref{ej:2}.}
                \label{fig:ej:2:AFND:inverso}
            \end{figure}

            Obtenemos ahora la gramática lineal por la derecha $G'=\{\{q_0,q_1,q_2,q_3\},\{0,1\},P',q_3\}$ que genera el lenguaje $L^{-1}$, siendo $P$ el conjunto de reglas:
            \begin{equation*}
                P'=\begin{cases}
                    q_3 \rightarrow & 0q_2 \mid 1q_2 \\
                    q_2 \rightarrow & 0q_1 \mid 1q_1 \\
                    q_1 \rightarrow & 1q_0 \\
                    q_0 \rightarrow & 0q_0 \mid 1q_0 \mid \varepsilon
                \end{cases}
            \end{equation*}

            Ahora, a partir de $G'$, obtenemos la gramática lineal por la izquierda, $G$, que genera el lenguaje $L$, donde $G=\{\{q_0,q_1,q_2,q_3\},\{0,1\},P,q_3\}$ siendo $P$ el conjunto de reglas:
            \begin{equation*}
                P=\begin{cases}
                    q_3 \rightarrow & q_2 0 \mid q_2 1 \\
                    q_2 \rightarrow & q_1 0 \mid q_1 1 \\
                    q_1 \rightarrow & q_0 1 \\
                    q_0 \rightarrow & q_0 0 \mid q_0 1 \mid \varepsilon
                \end{cases}
            \end{equation*}

        \end{enumerate}
    \end{ejercicio}
\end{document}
