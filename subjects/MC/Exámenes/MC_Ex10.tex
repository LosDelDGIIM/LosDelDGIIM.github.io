\documentclass[12pt]{article}

% Idioma y codificación
\usepackage[spanish, es-tabla]{babel}       %es-tabla para que se titule "Tabla"
\usepackage[utf8]{inputenc}

% Márgenes
\usepackage[a4paper,top=3cm,bottom=2.5cm,left=3cm,right=3cm]{geometry}

% Comentarios de bloque
\usepackage{verbatim}

% Paquetes de links
\usepackage[hidelinks]{hyperref}    % Permite enlaces
\usepackage{url}                    % redirecciona a la web

% Más opciones para enumeraciones
\usepackage{enumitem}

% Personalizar la portada
\usepackage{titling}

% Paquetes de tablas
\usepackage{multirow}


%------------------------------------------------------------------------

%Paquetes de figuras
\usepackage{caption}
\usepackage{subcaption} % Figuras al lado de otras
\usepackage{float}      % Poner figuras en el sitio indicado H.


% Paquetes de imágenes
\usepackage{graphicx}       % Paquete para añadir imágenes
\usepackage{transparent}    % Para manejar la opacidad de las figuras

% Paquete para usar colores
\usepackage[dvipsnames]{xcolor}
\usepackage{pagecolor}      % Para cambiar el color de la página

% Habilita tamaños de fuente mayores
\usepackage{fix-cm}

% Para los gráficos
\usepackage{tikz}

% Para poder situar los nodos en los grafos
\usetikzlibrary{positioning}


%------------------------------------------------------------------------

% Paquetes de matemáticas
\usepackage{mathtools, amsfonts, amssymb, mathrsfs}
\usepackage[makeroom]{cancel}     % Simplificar tachando
\usepackage{polynom}    % Divisiones y Ruffini
\usepackage{units} % Para poner fracciones diagonales con \nicefrac

\usepackage{pgfplots}   %Representar funciones
\pgfplotsset{compat=1.18}  % Versión 1.18

\usepackage{tikz-cd}    % Para usar diagramas de composiciones
\usetikzlibrary{calc}   % Para usar cálculo de coordenadas en tikz

%Definición de teoremas, etc.
\usepackage{amsthm}
%\swapnumbers   % Intercambia la posición del texto y de la numeración

\theoremstyle{plain}

\makeatletter
\@ifclassloaded{article}{
  \newtheorem{teo}{Teorema}[section]
}{
  \newtheorem{teo}{Teorema}[chapter]  % Se resetea en cada chapter
}
\makeatother

\newtheorem{coro}{Corolario}[teo]           % Se resetea en cada teorema
\newtheorem{prop}[teo]{Proposición}         % Usa el mismo contador que teorema
\newtheorem{lema}[teo]{Lema}                % Usa el mismo contador que teorema

\theoremstyle{remark}
\newtheorem*{observacion}{Observación}

\theoremstyle{definition}

\makeatletter
\@ifclassloaded{article}{
  \newtheorem{definicion}{Definición} [section]     % Se resetea en cada chapter
}{
  \newtheorem{definicion}{Definición} [chapter]     % Se resetea en cada chapter
}
\makeatother

\newtheorem*{notacion}{Notación}
\newtheorem*{ejemplo}{Ejemplo}
\newtheorem*{ejercicio*}{Ejercicio}             % No numerado
\newtheorem{ejercicio}{Ejercicio} [section]     % Se resetea en cada section


% Modificar el formato de la numeración del teorema "ejercicio"
\renewcommand{\theejercicio}{%
  \ifnum\value{section}=0 % Si no se ha iniciado ninguna sección
    \arabic{ejercicio}% Solo mostrar el número de ejercicio
  \else
    \thesection.\arabic{ejercicio}% Mostrar número de sección y número de ejercicio
  \fi
}


% \renewcommand\qedsymbol{$\blacksquare$}         % Cambiar símbolo QED
%------------------------------------------------------------------------

% Paquetes para encabezados
\usepackage{fancyhdr}
\pagestyle{fancy}
\fancyhf{}

\newcommand{\helv}{ % Modificación tamaño de letra
\fontfamily{}\fontsize{12}{12}\selectfont}
\setlength{\headheight}{15pt} % Amplía el tamaño del índice


%\usepackage{lastpage}   % Referenciar última pag   \pageref{LastPage}
\fancyfoot[C]{\thepage}

%------------------------------------------------------------------------

% Conseguir que no ponga "Capítulo 1". Sino solo "1."
\makeatletter
\@ifclassloaded{book}{
  \renewcommand{\chaptermark}[1]{\markboth{\thechapter.\ #1}{}} % En el encabezado
    
  \renewcommand{\@makechapterhead}[1]{%
  \vspace*{50\p@}%
  {\parindent \z@ \raggedright \normalfont
    \ifnum \c@secnumdepth >\m@ne
      \huge\bfseries \thechapter.\hspace{1em}\ignorespaces
    \fi
    \interlinepenalty\@M
    \Huge \bfseries #1\par\nobreak
    \vskip 40\p@
  }}
}
\makeatother

%------------------------------------------------------------------------
% Paquetes de cógido
\usepackage{minted}
\renewcommand\listingscaption{Código fuente}

\usepackage{fancyvrb}
% Personaliza el tamaño de los números de línea
\renewcommand{\theFancyVerbLine}{\small\arabic{FancyVerbLine}}

% Estilo para C++
\newminted{cpp}{
    frame=lines,
    framesep=2mm,
    baselinestretch=1.2,
    linenos,
    escapeinside=||
}

% para minted
\definecolor{LightGray}{rgb}{0.95,0.95,0.92}
\setminted{
    linenos=true,
    stepnumber=5,
    numberfirstline=true,
    autogobble,
    breaklines=true,
    breakautoindent=true,
    breaksymbolleft=,
    breaksymbolright=,
    breaksymbolindentleft=0pt,
    breaksymbolindentright=0pt,
    breaksymbolsepleft=0pt,
    breaksymbolsepright=0pt,
    fontsize=\footnotesize,
    bgcolor=LightGray,
    numbersep=10pt
}


\usepackage{listings} % Para incluir código desde un archivo

\renewcommand\lstlistingname{Código Fuente}
\renewcommand\lstlistlistingname{Índice de Códigos Fuente}

% Definir colores
\definecolor{vscodepurple}{rgb}{0.5,0,0.5}
\definecolor{vscodeblue}{rgb}{0,0,0.8}
\definecolor{vscodegreen}{rgb}{0,0.5,0}
\definecolor{vscodegray}{rgb}{0.5,0.5,0.5}
\definecolor{vscodebackground}{rgb}{0.97,0.97,0.97}
\definecolor{vscodelightgray}{rgb}{0.9,0.9,0.9}

% Configuración para el estilo de C similar a VSCode
\lstdefinestyle{vscode_C}{
  backgroundcolor=\color{vscodebackground},
  commentstyle=\color{vscodegreen},
  keywordstyle=\color{vscodeblue},
  numberstyle=\tiny\color{vscodegray},
  stringstyle=\color{vscodepurple},
  basicstyle=\scriptsize\ttfamily,
  breakatwhitespace=false,
  breaklines=true,
  captionpos=b,
  keepspaces=true,
  numbers=left,
  numbersep=5pt,
  showspaces=false,
  showstringspaces=false,
  showtabs=false,
  tabsize=2,
  frame=tb,
  framerule=0pt,
  aboveskip=10pt,
  belowskip=10pt,
  xleftmargin=10pt,
  xrightmargin=10pt,
  framexleftmargin=10pt,
  framexrightmargin=10pt,
  framesep=0pt,
  rulecolor=\color{vscodelightgray},
  backgroundcolor=\color{vscodebackground},
}

%------------------------------------------------------------------------

% Comandos definidos
\newcommand{\bb}[1]{\mathbb{#1}}
\newcommand{\cc}[1]{\mathcal{#1}}

% I prefer the slanted \leq
\let\oldleq\leq % save them in case they're every wanted
\let\oldgeq\geq
\renewcommand{\leq}{\leqslant}
\renewcommand{\geq}{\geqslant}

% Si y solo si
\newcommand{\sii}{\iff}

% Letras griegas
\newcommand{\eps}{\epsilon}
\newcommand{\veps}{\varepsilon}
\newcommand{\lm}{\lambda}

\newcommand{\ol}{\overline}
\newcommand{\ul}{\underline}
\newcommand{\wt}{\widetilde}
\newcommand{\wh}{\widehat}

\let\oldvec\vec
\renewcommand{\vec}{\overrightarrow}

% Derivadas parciales
\newcommand{\del}[2]{\frac{\partial #1}{\partial #2}}
\newcommand{\Del}[3]{\frac{\partial^{#1} #2}{\partial #3^{#1}}}
\newcommand{\deld}[2]{\dfrac{\partial #1}{\partial #2}}
\newcommand{\Deld}[3]{\dfrac{\partial^{#1} #2}{\partial #3^{#1}}}


\newcommand{\AstIg}{\stackrel{(\ast)}{=}}
\newcommand{\Hop}{\stackrel{L'H\hat{o}pital}{=}}

\newcommand{\red}[1]{{\color{red}#1}} % Para integrales, destacar los cambios.

% Método de integración
\newcommand{\MetInt}[2]{
    \left[\begin{array}{c}
        #1 \\ #2
    \end{array}\right]
}

% Declarar aplicaciones
% 1. Nombre aplicación
% 2. Dominio
% 3. Codominio
% 4. Variable
% 5. Imagen de la variable
\newcommand{\Func}[5]{
    \begin{equation*}
        \begin{array}{rrll}
            #1:& #2 & \longrightarrow & #3\\
               & #4 & \longmapsto & #5
        \end{array}
    \end{equation*}
}

%------------------------------------------------------------------------

% Para poder incluir árboles
\usepackage{forest}
\usepackage{booktabs}

\usepackage{hhline}
\newcommand{\cell}[1]{\multicolumn{1}{|c|}{$#1$}}

% Para poder añadir autómatas
% https://www3.nd.edu/~kogge/courses/cse30151-fa17/Public/other/tikz_tutorial.pdf
\usetikzlibrary{automata} %, positioning, arrows}
\tikzset{
    -Stealth,
    node distance=3cm, % specifies the minimum distance between two nodes. Change if necessary.
    every state/.style={thick, fill=gray!10, shape=ellipse}, % sets the properties for each ’state’ node
    initial text=$ $, % sets the text that appears on the start arrow
    % Un tipo de nodo, que es error, que lo pone rojo
    error/.style={thick, fill=red!20},
}


\begin{document}

    % 1. Foto de fondo
    % 2. Título
    % 3. Encabezado Izquierdo
    % 4. Color de fondo
    % 5. Coord x del titulo
    % 6. Coord y del titulo
    % 7. Fecha

    
    % 1. Foto de fondo
% 2. Título
% 3. Encabezado Izquierdo
% 4. Color de fondo
% 5. Coord x del titulo
% 6. Coord y del titulo
% 7. Fecha

\newcommand{\portada}[7]{

    \portadaBase{#1}{#2}{#3}{#4}{#5}{#6}{#7}
    \portadaBook{#1}{#2}{#3}{#4}{#5}{#6}{#7}
}

\newcommand{\portadaExamen}[7]{

    \portadaBase{#1}{#2}{#3}{#4}{#5}{#6}{#7}
    \portadaArticle{#1}{#2}{#3}{#4}{#5}{#6}{#7}
}




\newcommand{\portadaBase}[7]{

    % Tiene la portada principal y la licencia Creative Commons
    
    % 1. Foto de fondo
    % 2. Título
    % 3. Encabezado Izquierdo
    % 4. Color de fondo
    % 5. Coord x del titulo
    % 6. Coord y del titulo
    % 7. Fecha
    
    
    \thispagestyle{empty}               % Sin encabezado ni pie de página
    \newgeometry{margin=0cm}        % Márgenes nulos para la primera página
    
    
    % Encabezado
    \fancyhead[L]{\helv #3}
    \fancyhead[R]{\helv \nouppercase{\leftmark}}
    
    
    \pagecolor{#4}        % Color de fondo para la portada
    
    \begin{figure}[p]
        \centering
        \transparent{0.3}           % Opacidad del 30% para la imagen
        
        \includegraphics[width=\paperwidth, keepaspectratio]{assets/#1}
    
        \begin{tikzpicture}[remember picture, overlay]
            \node[anchor=north west, text=white, opacity=1, font=\fontsize{60}{90}\selectfont\bfseries\sffamily, align=left] at (#5, #6) {#2};
            
            \node[anchor=south east, text=white, opacity=1, font=\fontsize{12}{18}\selectfont\sffamily, align=right] at (9.7, 3) {\textbf{\href{https://losdeldgiim.github.io/}{Los Del DGIIM}}};
            
            \node[anchor=south east, text=white, opacity=1, font=\fontsize{12}{15}\selectfont\sffamily, align=right] at (9.7, 1.8) {Doble Grado en Ingeniería Informática y Matemáticas\\Universidad de Granada};
        \end{tikzpicture}
    \end{figure}
    
    
    \restoregeometry        % Restaurar márgenes normales para las páginas subsiguientes
    \pagecolor{white}       % Restaurar el color de página
    
    
    \newpage
    \thispagestyle{empty}               % Sin encabezado ni pie de página
    \begin{tikzpicture}[remember picture, overlay]
        \node[anchor=south west, inner sep=3cm] at (current page.south west) {
            \begin{minipage}{0.5\paperwidth}
                \href{https://creativecommons.org/licenses/by-nc-nd/4.0/}{
                    \includegraphics[height=2cm]{assets/Licencia.png}
                }\vspace{1cm}\\
                Esta obra está bajo una
                \href{https://creativecommons.org/licenses/by-nc-nd/4.0/}{
                    Licencia Creative Commons Atribución-NoComercial-SinDerivadas 4.0 Internacional (CC BY-NC-ND 4.0).
                }\\
    
                Eres libre de compartir y redistribuir el contenido de esta obra en cualquier medio o formato, siempre y cuando des el crédito adecuado a los autores originales y no persigas fines comerciales. 
            \end{minipage}
        };
    \end{tikzpicture}
    
    
    
    % 1. Foto de fondo
    % 2. Título
    % 3. Encabezado Izquierdo
    % 4. Color de fondo
    % 5. Coord x del titulo
    % 6. Coord y del titulo
    % 7. Fecha


}


\newcommand{\portadaBook}[7]{

    % 1. Foto de fondo
    % 2. Título
    % 3. Encabezado Izquierdo
    % 4. Color de fondo
    % 5. Coord x del titulo
    % 6. Coord y del titulo
    % 7. Fecha

    % Personaliza el formato del título
    \pretitle{\begin{center}\bfseries\fontsize{42}{56}\selectfont}
    \posttitle{\par\end{center}\vspace{2em}}
    
    % Personaliza el formato del autor
    \preauthor{\begin{center}\Large}
    \postauthor{\par\end{center}\vfill}
    
    % Personaliza el formato de la fecha
    \predate{\begin{center}\huge}
    \postdate{\par\end{center}\vspace{2em}}
    
    \title{#2}
    \author{\href{https://losdeldgiim.github.io/}{Los Del DGIIM}}
    \date{Granada, #7}
    \maketitle
    
    \tableofcontents
}




\newcommand{\portadaArticle}[7]{

    % 1. Foto de fondo
    % 2. Título
    % 3. Encabezado Izquierdo
    % 4. Color de fondo
    % 5. Coord x del titulo
    % 6. Coord y del titulo
    % 7. Fecha

    % Personaliza el formato del título
    \pretitle{\begin{center}\bfseries\fontsize{42}{56}\selectfont}
    \posttitle{\par\end{center}\vspace{2em}}
    
    % Personaliza el formato del autor
    \preauthor{\begin{center}\Large}
    \postauthor{\par\end{center}\vspace{3em}}
    
    % Personaliza el formato de la fecha
    \predate{\begin{center}\huge}
    \postdate{\par\end{center}\vspace{5em}}
    
    \title{#2}
    \author{\href{https://losdeldgiim.github.io/}{Los Del DGIIM}}
    \date{Granada, #7}
    \thispagestyle{empty}               % Sin encabezado ni pie de página
    \maketitle
    \vfill
}
    \portadaExamen{etsiitA4.jpg}{Modelos de\\Computación\\Examen X}{MC. Examen X}{MidnightBlue}{-8}{28}{2024-2025}{Arturo Olivares Martos}

    \begin{description}
        \item[Asignatura] Modelos de Computación
        \item[Curso Académico] 2022-23.
        \item[Grado] Doble Grado en Ingeniería Informática y Matemáticas o ADE.
        \item[Grupo] Único.
        %\item[Profesor] Serafín Moral García.
        \item[Descripción] Convocatoria Ordinaria.
        \item[Fecha] 16 de enero de 2023.
        \item[Duración] $2.5$ horas.    
    \end{description}
    \newpage

    \begin{ejercicio}[2.5 puntos]
        Decir cuales de los siguientes lenguajes sobre el alfabeto $\{a, b, c\}$ son regulares y/o independientes del contexto. Justificar las respuestas.
        \begin{enumerate}
            \item $L_1 = \{a^k b^m c^n : (k = n ~\lor~ m = n) ~\land~ k + m + n \geq 2\}$
            \item $L_2 = \{a^k b^m c^n : (k = n ~\lor~ m = n) ~\land~ k + m + n \leq 2\}$
            \item $L_3 = \{a^k b^m c^n : k + m + n \geq 2\}$
        \end{enumerate}
    \end{ejercicio}

    \begin{ejercicio}[2.5 puntos]
        Construir autómatas con pila que acepten los siguientes lenguajes sobre el alfabeto $\{a, b, c\}$, procurando que sean deterministas cuando sea posible.
        \begin{enumerate}
            \item $L_1 = \{a^i b^j c^k : i \neq j ~\lor~ j \neq k\}$
            \item $L_2$: conjunto de palabras $u$ tales que en todo prefijo de $u$ el número de $a$'s más el número de $b$'s es menor o igual al doble del número de $c$'s.
        \end{enumerate}
    \end{ejercicio}

    \begin{ejercicio}[1.25 puntos]
        Decir si las siguientes afirmaciones sobre expresiones regulares son verdaderas o falsas. Justificar las respuestas:
        \begin{enumerate}
            \item $(rr + \varepsilon)^* (r + \varepsilon) = r^*$
            \item $(r_1 r_1 + r_1 r_2 + r_2 r_1 + r_2 r_2)^* = (r_1 + r_2)^* (r_1 + r_2)$
        \end{enumerate}
    \end{ejercicio}

    \begin{ejercicio}[1.25 puntos]
        Pon ejemplos de lenguajes que cumplan las siguientes condiciones:
        \begin{enumerate}
            \item Un lenguaje independiente del contexto y no regular $L$ y un homomorfismo $f$, tal que $f(L)$ es regular.
            \item Un lenguaje independiente del contexto y no regular tal que su complementario es independiente del contexto.
            \item Un lenguaje independiente del contexto $L$ y otro regular $R$, tal que $R \cap L$ no es independiente del contexto.
        \end{enumerate}
    \end{ejercicio}

    \begin{ejercicio}[1.25 puntos]
        Describe la función $\text{ELIMINA}_2(A)$ en el algoritmo para pasar una gramática a forma normal de Greibach.
    \end{ejercicio}

    \begin{ejercicio}[1.25 puntos]
        Describe el paso de ``Terminación'' en el algoritmo de Early.
    \end{ejercicio}

    \begin{ejercicio}[1 punto]
        Dado un autómata finito determinista $M$ que acepta el lenguaje $L$, determinar cómo se construiría un autómata finito (puede ser no determinista) que acepta el lenguaje:
        \[
            \text{Ciclo}(L) = \{vu : uv \in L\}
        \]
        ¿Es cierto que si $\text{Ciclo}(L)$ es regular, entonces $L$ es siempre regular?
        \begin{observacion}
            Este ejercicio es opcional y sirve como nota complementaria (sólo suma). Tiene una dificultad mayor y no es recomendable
            hacerlo sin haber terminado antes las otras preguntas.
        \end{observacion}
    \end{ejercicio}




    \newpage
    \section*{Soluciones}
    \setcounter{ejercicio}{0}
    \begin{ejercicio}[2.5 puntos]
        Decir cuales de los siguientes lenguajes sobre el alfabeto $\{a, b, c\}$ son regulares y/o independientes del contexto. Justificar las respuestas.
        \begin{enumerate}
            \item $L_1 = \{a^k b^m c^n : (k = n ~\lor~ m = n) ~\land~ k + m + n \geq 2\}$
            
            Sea $L_1^1$ el siguiente lenguaje:
            \begin{equation*}
                L_1^1 = \{u\in \{a,b,c\}^* \mid |u| \geq 2\}
            \end{equation*}
            Este es un lenguaje regular por ser el complementario de uno regular (por ser finito). Definimos los siguientes lenguajes auxiliares:
            \begin{align*}
                L_1^2 &= \{a^k b^m c^n : k = n\} \\
                L_1^3 &= \{a^k b^m c^n : m = n\}
            \end{align*}
            
            Ambos lenguajes, $L_1^2$ y $L_1^3$, son independientes del contexto porque podemos construir sencillamente un APND que los acepta (por ejemplo, por el criterio de la pila vacía). Por tanto, como los lenguajes independientes del contexto son cerrados por unión, tenemos que $L_1^2 \cup L_1^3$ es independiente del contexto. Además, como la intersección de un lenguaje regular y uno independiente del contexto es independiente del contexto, tenemos que:
            \begin{equation*}
                L_1 = (L_1^2 \cup L_1^3) \cap L_1^1 \text{ es independiente del contexto}
            \end{equation*}

            Aunque no se pide (ya habríamos terminado), daremos la gramática que genera $L_1$. Tendremos $S\rightarrow A\mid B$, que representan:
            \begin{itemize}
                \item $A$: $k=n$ y $k+m+n\geq 2$. Sus producciones son:
                \begin{align*}
                    A &\rightarrow aAc \mid aA'c \mid bbA'\\
                    A' &\rightarrow bA' \mid \veps
                \end{align*}

                \item $B$: $m=n$ y $k+m+n\geq 2$. Sus producciones son:
                \begin{align*}
                    B &\rightarrow aaB''\mid B''B'\\
                    B' &\rightarrow bB'c \mid bc\\
                    B'' &\rightarrow aB'' \mid \veps
                \end{align*}
            \end{itemize}

            Por tanto, una gramática independiente del contexto que genera $L_1$ es:
            \begin{align*}
                S &\rightarrow A\mid B\\
                A &\rightarrow aAc \mid aA'c \mid bbA'\\
                A' &\rightarrow bA' \mid \veps\\
                B &\rightarrow aaB''\mid B''B'\\
                B' &\rightarrow bB'c \mid bc\\
                B'' &\rightarrow aB'' \mid \veps
            \end{align*}

            Veamos ahora que no es regular por el recíproco del Lema de Bombeo. Para cada $n \in \bb{N}$, tomamos la palabra $z = a^n b^{n+1}c^n\in L_1$ pues $N_a(z)=N_c(z)=n$, con $|z|=3n+1\geq n$. Toda descomposición de $z=uvw$ con $u,v,w\in \{a,b,c\}^*$, $|uv|\leq n$ y $|v|\geq 1$ debe cumplir que:
            \begin{equation*}
                u=a^k, \quad v=a^l, \quad w=a^{n-k-l}b^{n+1}c^n,\qquad 0\leq k+l\leq n, \quad l\geq 1
            \end{equation*}

            Bombeando con $i=2$, obtenemos la palabra:
            \begin{equation*}
                uv^2w=a^{k+l}a^l a^{n-k-l}b^{n+1}c^n=a^{n+l}b^{n+1}c^n\notin L_1
            \end{equation*}
            pues $n+l,n+1>n$. Por tanto, $L_1$ no es regular.

            \item $L_2 = \{a^k b^m c^n : (k = n ~\lor~ m = n) ~\land~ k + m + n \leq 2\}$
            
            Tenemos la siguiente inclusión entre lenguajes:
            \begin{equation*}
                L_2 \subset \{a^k b^m c^n : k + m + n \leq 2\}
                \subset
                \{u\in \{a,b,c\}^* \mid |u| \leq 2\}
            \end{equation*}

            El último lenguaje es finito, con un total de $3^2 = 9$ palabras. Por tanto, $L_2$ es finito y por tanto regular.
            \item $L_3 = \{a^k b^m c^n : k + m + n \geq 2\}$
            
            Sea $L_3^1$ el lenguaje regular con expresión regular $a^* b^* c^*$, y $L_3^2$ el siguiente lenguaje:
            \begin{equation*}
                L_3^2 = \{u \in \{a,b,c\}^* \mid |u| \geq 2\}
            \end{equation*}

            El lenguaje $L_3^2$ es regular por ser el complementario de uno regular (por ser finito). Por tanto, como los lenguajes independientes del contexto son cerrados por intersección, tenemos que:
            \begin{equation*}
                L_3 = L_3^1 \cap L_3^2 \text{ es regular}
            \end{equation*}
        \end{enumerate}
    \end{ejercicio}

    \begin{ejercicio}[2.5 puntos]
        Construir autómatas con pila que acepten los siguientes lenguajes sobre el alfabeto $\{a, b, c\}$, procurando que sean deterministas cuando sea posible.
        \begin{enumerate}
            \item $L_1 = \{a^i b^j c^k : i \neq j ~\lor~ j \neq k\}$
            
            Hay varias opciones, aunque como el lenguaje es inherentemente ambiguo no podremos conseguir un autómata determinista.
            \begin{description}
                \item[Opción 1] A partir de la gramática independiente del contexto que genera $L_1$.
                
                La gramática que lo genera es $G=(V,\{a,b,c\},P,S)$, con:
                \begin{equation*}
                    \begin{aligned}
                        V &= \{ S, X,X',Y,Y',A,B,C \} \\
                        P &= \left\{
                            \begin{aligned}
                                S &\rightarrow XC \mid AY \\
                                X &\rightarrow aX'b\\
                                X' &\rightarrow aX'b \mid A \mid B\\
                                Y &\rightarrow bY'c\\
                                Y' &\rightarrow bY'c \mid B \mid C\\
                                A &\rightarrow aA \mid a \\
                                B &\rightarrow bB \mid b \\
                                C &\rightarrow cC \mid c
                            \end{aligned}
                        \right.
                    \end{aligned}
                \end{equation*}

                Por tanto, el autómata es:
                \begin{equation*}
                    M=(\{q\},\{a,b,c\},\{a,b,c\}\cup V,\delta,q,S,\emptyset)
                \end{equation*}
                donde la función de transición $\delta$ es:
                \begin{align*}
                    \delta(q,\veps,D) &= \{(q,\alpha)\mid D\rightarrow \alpha\in P\}\qquad \text{para } D\in V\\
                    \delta(q,i,i) &= \{(q,i)\} \quad \text{para } i\in \{a,b,c\}
                \end{align*}

                \item [Opción 2] Autómata resultante de una intersección con un lenguaje regular.
                
                Definimos los siguientes lenguajes:
                \begin{align*}
                    L_1^1 &= \{u\in \{a,b,c\}^* \mid N_a(u)\neq N_b(u)~\lor~ N_b(u)\neq N_c(u)\}\\
                    L_1^2 & \text{ regular con expresión regular } a^+ b^+ c^+
                \end{align*}

                Tenemos que $L_1 = L_1^1 \cap L_1^2$. El AFD de $L_1^2$ se muestra en la Figura~\ref{fig:afd_L1_2}.
                \begin{figure}[h]
                    \centering
                    \begin{tikzpicture}[node distance=2cm]
                        \node[state,initial] (p_0) {$p_0$};
                        \node[state,right of=p_0] (p_1) {$p_1$};
                        \node[state,right of=p_1] (p_2) {$p_2$};
                        \node[state,right of=p_2, accepting] (p_3) {$p_3$};
                        \node[state, below of=p_1, error] (p_E) {$p_E$};

                        \draw   (p_0) edge node[below]{$b,c$} (p_E)
                                (p_0) edge node[above]{$a$} (p_1)
                                (p_1) edge node[above]{$b$} (p_2)
                                (p_1) edge[loop above] node{$a$} (p_1)
                                (p_1) edge node[left]{$c$} (p_E)
                                (p_2) edge node[above]{$c$} (p_3)
                                (p_2) edge[loop above] node{$b$} (p_2)
                                (p_2) edge node[above]{$a$} (p_E)
                                (p_3) edge[loop above] node{$c$} (p_3)
                                (p_3) edge node[below]{$a,b$} (p_E);
                    \end{tikzpicture}
                    \caption{AFD de $L_1^2$.}
                    \label{fig:afd_L1_2}
                \end{figure}

                Para el APND de $L_1^1$, tendremos varios estados. El estado inicial será $q_0$, al que iremos con una transición nula a $q_1$ y $q_2$. El estado $q_1$ representa las palabras con $N_a(u)\neq N_b(u)$, y el estado $q_2$ las palabras con $N_b(u)\neq N_c(u)$. Ambos estados, cuando efectivamente se cumpla dicha condición, irán a un estado de aceptación $q_f$ en el que no podremos añadir más caracteres. El autómata que lo acepta por el criterio de estados finales sería:
                \begin{equation*}
                    M=(\{q_0,q_1,q_2,q_f\},\{a,b,c\},\{Z_0,A,B,C\},\delta,q_0,Z_0,\{q_f\})
                \end{equation*}
                donde la función de transición $\delta$ la daremos por partes. Las transiciones nulas son:
                \begin{align*}
                    \delta(q_0,\varepsilon,Z_0) &= \{(q_1,Z_0),(q_2,Z_0)\}\\
                    \delta(q_i,\varepsilon,j) &= \{(q_f,j)\} \quad \text{para } i\in \{1,2\}, j\in \{A,B,C\}
                \end{align*}

                En $q_1$, las transiciones son:
                \begin{align*}
                    \delta(q_1,a,Z_0) &= \{(q_1,AZ_0)\}\\
                    \delta(q_1,a,A) &= \{(q_1,AA)\}\\
                    \delta(q_1,a,B) &= \{(q_1,\veps)\}\\
                    \delta(q_1,b,Z_0) &= \{(q_1,BZ_0)\}\\
                    \delta(q_1,b,A) &= \{(q_1,\veps)\}\\
                    \delta(q_1,b,B) &= \{(q_1,BB)\}\\
                    \delta(q_1,c,i) &= \{(q_1,i)\} \quad \text{para } i\in \{Z_0,A,B\}
                \end{align*}

                En $q_2$, las transiciones son:
                \begin{align*}
                    \delta(q_2,b,Z_0) &= \{(q_2,BZ_0)\}\\
                    \delta(q_2,b,B) &= \{(q_2,BB)\}\\
                    \delta(q_2,b,C) &= \{(q_2,\veps)\}\\
                    \delta(q_2,c,Z_0) &= \{(q_2,CZ_0)\}\\
                    \delta(q_2,c,B) &= \{(q_2,\veps)\}\\
                    \delta(q_2,c,C) &= \{(q_2,CC)\}\\
                    \delta(q_2,a,i) &= \{(q_2,i)\} \quad \text{para } i\in \{Z_0,B,C\}
                \end{align*}

                El estado de aceptación $q_f$, como hemos mencionado, no tiene transiciones. Por tanto, y nombrando $M'=(Q',\{a,b,c\},\delta',p_0,F')$ al AFD de la Figura~\ref{fig:afd_L1_2}, el autómata con pila que acepta $L_1=L_1^1\cap L_1^2$ por el criterio de estados finales es:
                \begin{equation*}
                    M''=(Q'\times Q,\{a,b,c\},\{Z_0,A,B,C\},\delta'',Z_0,F'\times\{q_f\})
                \end{equation*}
                donde la función de transición $\delta''$ viene definida, para cada estado $(p,q)\in Q'\times Q$ y cada símbolo del alfabeto de la pila $D\in \{Z_0,A,B,C\}$:
                \begin{align*}
                    \delta''((p,q),\varepsilon,D) &= \{((p,q'),\alpha)\mid (q',\alpha)\in \delta(q,\varepsilon,D)\} \\
                    \delta''((p,q),i,D) &= \{((p',q'),\alpha)\mid \delta'(p,i)=p'~\land~ (q',\alpha)\in \delta(q,i,D)\} \qquad i\in \{a,b,c\}
                \end{align*}
            \end{description}

            \item $L_2$: conjunto de palabras $u$ tales que en todo prefijo de $u$ el número de $a$'s más el número de $b$'s es menor o igual al doble del número de $c$'s.
            
            Tenemos que, para cada prefijo $v$ de $u$, se cumple que:
            \begin{equation*}
                N_a(v)+N_b(v)\leq 2N_c(v)
            \end{equation*}
            Podemos por tanto interpretarlo como que una $c$ permite que haya $2$ símbolos de $\{a,b\}$, y que un símbolo de $\{a,b\}$ no puede aparecer sin haber aparecido antes una $c$. Por tanto, el autómata con pila $M$ que acepta $L_2$ por el criterio de estados finales es:
            \begin{equation*}
                M=(\{q\},\{a,b,c\},\{Z_0,X\},\delta,q_0,Z_0,\{q\})
            \end{equation*}
            donde la función de transición $\delta$ es:
            \begin{align*}
                \delta(q,c,Z_0) &= \{(q,XXZ_0)\}\\
                \delta(q,c,X) &= \{(q,XXX)\}\\
                \delta(q,a,X) &= \{(q,\veps)\}\\
                \delta(q,b,X) &= \{(q,\veps)\}
            \end{align*}

            Además, notemos que es un autómata determinista.
        \end{enumerate}
    \end{ejercicio}

    \begin{ejercicio}[1.25 puntos]
        Decir si las siguientes afirmaciones sobre expresiones regulares son verdaderas o falsas. Justificar las respuestas:
        \begin{enumerate}
            \item $(rr + \varepsilon)^* (r + \varepsilon) = r^*$
            
            En primer lugar, tenemos que:
            \begin{equation*}
                (rr + \varepsilon)^* (r + \varepsilon)
                = (rr)^* (r + \varepsilon)
            \end{equation*}
            
            Sea $L$ el lenguaje asociado a $r$. Demostrémoslo mediante doble inclusión:
            \begin{description}
                \item[$\subseteq$)] Sea $u$ perteneciente al lenguaje asociado a $(rr)^* (r + \varepsilon)$. Entonces, $\exists n\in \bb{N}\cup \{0\}$ tal que:
                \begin{equation*}
                    u\in (LL)^n\left(L\cup \{\veps\}\right)
                \end{equation*}

                Por tanto, $u$ es de la forma $u=v_1v_2\cdots v_{2n}v_{2n+1}$ con $v_j\in L$ para todo $j=1,\ldots,2n$ y $v_{2n+1}\in L\cup \{\veps\}$; es decir, $v_{2n+1}\in L$ o $v_{2n+1}=\veps$. Si $v_{2n+1}=\veps$, entonces $u\in L^{2n}$, y si $v_{2n+1}\in L$, entonces $u\in L^{2n+1}$. En cualquier caso, $u\in L^*$, por lo que pertenece al lenguaje asociado a $r^*$.
                
                \item[$\supseteq$)] Sea $u\in L^*$. Entonces, $u$ es de la forma $u=v_1v_2\cdots v_n$ con $v_i\in L$ para todo $i=1,\ldots,n$. Si $n$ es par, entonces $u\in (LL)^{\nicefrac{n}{2}}=(LL)^{\nicefrac{n}{2}}\{\emptyset\}$, y si es impar entonces $u\in (LL)^{\nicefrac{n-1}{2}}L$. En cualquier caso:
                \begin{equation*}
                    u\in (LL)^*(L\cup \{\emptyset\})
                \end{equation*}
                que es el lenguaje asociado a $(rr)^* (r + \varepsilon)$.
            \end{description}

            Por tanto, es cierto.
            \item $(r_1 r_1 + r_1 r_2 + r_2 r_1 + r_2 r_2)^* = (r_1 + r_2)^* (r_1 + r_2)$
            
            En primer lugar, tenemos que:
            \begin{equation*}
                (r_1 r_1 + r_1 r_2 + r_2 r_1 + r_2 r_2)^*
                = (r_1 + r_2)^*
            \end{equation*}

            Por tanto, se reduce a demostrar:
            \begin{equation*}
                (r_1 + r_2)^* = (r_1 + r_2)^*(r_1 + r_2)
            \end{equation*}
            
            Sea $L_1$ el lenguaje asociado a $r_1$ y $L_2$ el asociado a $r_2$. Entonces, tan solo es cierto si $\veps\in (L_1\cup L_2)$.
            \begin{itemize}
                \item Si $\veps\notin (L_1\cup L_2)$, entonces $\veps$ pertenece al lenguaje asociado a la expresión regular de la izquierda, pero no pertenece al asociado por la expresión regular de la derecha.
                \item Si $\veps\in (L_1\cup L_2)$, sí es cierto.
            \end{itemize}
        \end{enumerate}
    \end{ejercicio}

    \begin{ejercicio}[1.25 puntos]
        Pon ejemplos de lenguajes que cumplan las siguientes condiciones:
        \begin{enumerate}
            \item Un lenguaje independiente del contexto y no regular $L$ y un homomorfismo $f$, tal que $f(L)$ es regular.\\
            
            Sea $L=\left\{a^ib^i\mid i\in \bb{N}\cup\{0\}\right\}$, el cual sabemos que es independiente del contexto y no regular. Sea $f:\{a,b\}^*\to \{0\}^*$ dado por:
            \begin{equation*}
                f(a)=0\qquad f(b)=0
            \end{equation*}

            Entonces, tenemos que:
            \begin{align*}
                f(L)&=\{f(u)\mid u\in L\}
                =\\&=\{f(u)\mid \exists i\in \bb{N}\cup \{0\}\mid u=a^ib^i\}
                =\\&=\{f(a^ib^i)\mid i\in \bb{N}\cup \{0\}\}
                =\\&=\{f(a)^if(b)^i\mid i\in \bb{N}\cup \{0\}\}
                =\\&=\{0^{2i}\mid i\in \bb{N}\cup \{0\}\}
            \end{align*}

            Este lenguaje sabemos que es regular, con expresión regular asociada $(00)^*$.
            \item Un lenguaje independiente del contexto y no regular tal que su complementario es independiente del contexto.
            
            El complementario de un lenguaje independiente del conexto, por norma general, no tiene por qué ser independiente del conexto. No obstante, los que son deterministas sí son cerrados por complementarios, por lo que buscaremos un lenguaje independiente del contexto determinista que no sea regular. Por ejemplo, sea el lenguaje:
            \begin{equation*}
                L=\{a^ib^i\mid i\in \bb{N}\cup \{0\}\}
            \end{equation*}

            Este lenguaje sabemos que es independiente del contexto determinista no regular. Además, por la teoría, sabemos que $\ol{L}$ es también independiente del contexto determinista.
            \item Un lenguaje independiente del contexto $L$ y otro regular $R$, tal que $R \cap L$ no es independiente del contexto.
            
            No es posible, ya que $R\cap L$ siempre será independiente del contexto.
        \end{enumerate}
    \end{ejercicio}

    \begin{ejercicio}[1.25 puntos]
        Describe la función $\text{ELIMINA}_2(A)$ en el algoritmo para pasar una gramática a forma normal de Greibach.\\

        Dada una variable $A$, esta función busca eliminar todas las producciones de la forma $A\to A\alpha$, con $\alpha\in (V\cup T)^*$, ya que esta no se encuentra en las condiciones de la forma normal de Greibach.\\

        Para esto, añadimos la variable $B_A$, y, para cada una de las producciones de la forma $A\to A\alpha$, con $\alpha\in (V\cup T)^*$; la eliminamos y añadimos la producción $B_A\rightarrow \alpha B_A\mid \alpha$.\\
        
        Por último, cuando ya no queden producciones de dicha forma, para cada producción de la forma $A\rightarrow \beta$ (donde $\beta$ ya no comenzará por $A$) añadimos la producción $A\rightarrow \beta B_A$.\\

        De esta forma, ya no tendremos producciones de la forma $A\to A\alpha$, con $\alpha\in (V\cup T)^*$; y no habremos cambiado el lenguaje generado.
    \end{ejercicio}

    \begin{ejercicio}[1.25 puntos]
        Describe el paso de ``Terminación'' en el algoritmo de Early.\\

        Este proceso se aplica sobre cada uno de los registros que se denominan completos; es decir, que son de la forma:
        \begin{equation*}
            (i,j,A,\alpha,\red{\veps})
        \end{equation*}
        donde la clave es que el objetivo es la cadena vacía. Por tanto, ese registro nos informa que $A\rightarrow \alpha$ es una producción, y $\alpha$ genera $u[i+1,\dots,j]$.

        Por tanto, buscamos en $\text{Registros}[i]$ registros de la forma $(h,i,B,\gamma,A\delta)$, que nos informan que $B\rightarrow \gamma A\delta$, donde $\gamma$ genera $u[h+1,\dots,i]$. Para cada uno de estos registros, tenemos que $\gamma A$ genera $u[h+1,j]$, luego introducimos en $\text{Registros}[j]$ el registro:
        \begin{equation*}
            (h,j,B,\gamma A,\delta)
        \end{equation*}
    \end{ejercicio}

    \begin{ejercicio}[1 punto]
        Dado un autómata finito determinista $M$ que acepta el lenguaje $L$, determinar cómo se construiría un autómata finito (puede ser no determinista) que acepta el lenguaje:
        \[
            \text{Ciclo}(L) = \{vu : uv \in L\}
        \]
        ¿Es cierto que si $\text{Ciclo}(L)$ es regular, entonces $L$ es siempre regular?
        \begin{observacion}
            Este ejercicio es opcional y sirve como nota complementaria (sólo suma). Tiene una dificultad mayor y no es recomendable
            hacerlo sin haber terminado antes las otras preguntas.
        \end{observacion}
        % // TODO: Hacer JJ
    \end{ejercicio}
\end{document}
