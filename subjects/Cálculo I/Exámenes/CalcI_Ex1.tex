\documentclass[12pt]{article}

% Idioma y codificación
\usepackage[spanish, es-tabla]{babel}       %es-tabla para que se titule "Tabla"
\usepackage[utf8]{inputenc}

% Márgenes
\usepackage[a4paper,top=3cm,bottom=2.5cm,left=3cm,right=3cm]{geometry}

% Comentarios de bloque
\usepackage{verbatim}

% Paquetes de links
\usepackage[hidelinks]{hyperref}    % Permite enlaces
\usepackage{url}                    % redirecciona a la web

% Más opciones para enumeraciones
\usepackage{enumitem}

% Personalizar la portada
\usepackage{titling}

% Paquetes de tablas
\usepackage{multirow}


%------------------------------------------------------------------------

%Paquetes de figuras
\usepackage{caption}
\usepackage{subcaption} % Figuras al lado de otras
\usepackage{float}      % Poner figuras en el sitio indicado H.


% Paquetes de imágenes
\usepackage{graphicx}       % Paquete para añadir imágenes
\usepackage{transparent}    % Para manejar la opacidad de las figuras

% Paquete para usar colores
\usepackage[dvipsnames]{xcolor}
\usepackage{pagecolor}      % Para cambiar el color de la página

% Habilita tamaños de fuente mayores
\usepackage{fix-cm}

% Para los gráficos
\usepackage{tikz}

% Para poder situar los nodos en los grafos
\usetikzlibrary{positioning}


%------------------------------------------------------------------------

% Paquetes de matemáticas
\usepackage{mathtools, amsfonts, amssymb, mathrsfs}
\usepackage[makeroom]{cancel}     % Simplificar tachando
\usepackage{polynom}    % Divisiones y Ruffini
\usepackage{units} % Para poner fracciones diagonales con \nicefrac

\usepackage{pgfplots}   %Representar funciones
\pgfplotsset{compat=1.18}  % Versión 1.18

\usepackage{tikz-cd}    % Para usar diagramas de composiciones
\usetikzlibrary{calc}   % Para usar cálculo de coordenadas en tikz

%Definición de teoremas, etc.
\usepackage{amsthm}
%\swapnumbers   % Intercambia la posición del texto y de la numeración

\theoremstyle{plain}

\makeatletter
\@ifclassloaded{article}{
  \newtheorem{teo}{Teorema}[section]
}{
  \newtheorem{teo}{Teorema}[chapter]  % Se resetea en cada chapter
}
\makeatother

\newtheorem{coro}{Corolario}[teo]           % Se resetea en cada teorema
\newtheorem{prop}[teo]{Proposición}         % Usa el mismo contador que teorema
\newtheorem{lema}[teo]{Lema}                % Usa el mismo contador que teorema

\theoremstyle{remark}
\newtheorem*{observacion}{Observación}

\theoremstyle{definition}

\makeatletter
\@ifclassloaded{article}{
  \newtheorem{definicion}{Definición} [section]     % Se resetea en cada chapter
}{
  \newtheorem{definicion}{Definición} [chapter]     % Se resetea en cada chapter
}
\makeatother

\newtheorem*{notacion}{Notación}
\newtheorem*{ejemplo}{Ejemplo}
\newtheorem*{ejercicio*}{Ejercicio}             % No numerado
\newtheorem{ejercicio}{Ejercicio} [section]     % Se resetea en cada section


% Modificar el formato de la numeración del teorema "ejercicio"
\renewcommand{\theejercicio}{%
  \ifnum\value{section}=0 % Si no se ha iniciado ninguna sección
    \arabic{ejercicio}% Solo mostrar el número de ejercicio
  \else
    \thesection.\arabic{ejercicio}% Mostrar número de sección y número de ejercicio
  \fi
}


% \renewcommand\qedsymbol{$\blacksquare$}         % Cambiar símbolo QED
%------------------------------------------------------------------------

% Paquetes para encabezados
\usepackage{fancyhdr}
\pagestyle{fancy}
\fancyhf{}

\newcommand{\helv}{ % Modificación tamaño de letra
\fontfamily{}\fontsize{12}{12}\selectfont}
\setlength{\headheight}{15pt} % Amplía el tamaño del índice


%\usepackage{lastpage}   % Referenciar última pag   \pageref{LastPage}
\fancyfoot[C]{\thepage}

%------------------------------------------------------------------------

% Conseguir que no ponga "Capítulo 1". Sino solo "1."
\makeatletter
\@ifclassloaded{book}{
  \renewcommand{\chaptermark}[1]{\markboth{\thechapter.\ #1}{}} % En el encabezado
    
  \renewcommand{\@makechapterhead}[1]{%
  \vspace*{50\p@}%
  {\parindent \z@ \raggedright \normalfont
    \ifnum \c@secnumdepth >\m@ne
      \huge\bfseries \thechapter.\hspace{1em}\ignorespaces
    \fi
    \interlinepenalty\@M
    \Huge \bfseries #1\par\nobreak
    \vskip 40\p@
  }}
}
\makeatother

%------------------------------------------------------------------------
% Paquetes de cógido
\usepackage{minted}
\renewcommand\listingscaption{Código fuente}

\usepackage{fancyvrb}
% Personaliza el tamaño de los números de línea
\renewcommand{\theFancyVerbLine}{\small\arabic{FancyVerbLine}}

% Estilo para C++
\newminted{cpp}{
    frame=lines,
    framesep=2mm,
    baselinestretch=1.2,
    linenos,
    escapeinside=||
}

% para minted
\definecolor{LightGray}{rgb}{0.95,0.95,0.92}
\setminted{
    linenos=true,
    stepnumber=5,
    numberfirstline=true,
    autogobble,
    breaklines=true,
    breakautoindent=true,
    breaksymbolleft=,
    breaksymbolright=,
    breaksymbolindentleft=0pt,
    breaksymbolindentright=0pt,
    breaksymbolsepleft=0pt,
    breaksymbolsepright=0pt,
    fontsize=\footnotesize,
    bgcolor=LightGray,
    numbersep=10pt
}


\usepackage{listings} % Para incluir código desde un archivo

\renewcommand\lstlistingname{Código Fuente}
\renewcommand\lstlistlistingname{Índice de Códigos Fuente}

% Definir colores
\definecolor{vscodepurple}{rgb}{0.5,0,0.5}
\definecolor{vscodeblue}{rgb}{0,0,0.8}
\definecolor{vscodegreen}{rgb}{0,0.5,0}
\definecolor{vscodegray}{rgb}{0.5,0.5,0.5}
\definecolor{vscodebackground}{rgb}{0.97,0.97,0.97}
\definecolor{vscodelightgray}{rgb}{0.9,0.9,0.9}

% Configuración para el estilo de C similar a VSCode
\lstdefinestyle{vscode_C}{
  backgroundcolor=\color{vscodebackground},
  commentstyle=\color{vscodegreen},
  keywordstyle=\color{vscodeblue},
  numberstyle=\tiny\color{vscodegray},
  stringstyle=\color{vscodepurple},
  basicstyle=\scriptsize\ttfamily,
  breakatwhitespace=false,
  breaklines=true,
  captionpos=b,
  keepspaces=true,
  numbers=left,
  numbersep=5pt,
  showspaces=false,
  showstringspaces=false,
  showtabs=false,
  tabsize=2,
  frame=tb,
  framerule=0pt,
  aboveskip=10pt,
  belowskip=10pt,
  xleftmargin=10pt,
  xrightmargin=10pt,
  framexleftmargin=10pt,
  framexrightmargin=10pt,
  framesep=0pt,
  rulecolor=\color{vscodelightgray},
  backgroundcolor=\color{vscodebackground},
}

%------------------------------------------------------------------------

% Comandos definidos
\newcommand{\bb}[1]{\mathbb{#1}}
\newcommand{\cc}[1]{\mathcal{#1}}

% I prefer the slanted \leq
\let\oldleq\leq % save them in case they're every wanted
\let\oldgeq\geq
\renewcommand{\leq}{\leqslant}
\renewcommand{\geq}{\geqslant}

% Si y solo si
\newcommand{\sii}{\iff}

% Letras griegas
\newcommand{\eps}{\epsilon}
\newcommand{\veps}{\varepsilon}
\newcommand{\lm}{\lambda}

\newcommand{\ol}{\overline}
\newcommand{\ul}{\underline}
\newcommand{\wt}{\widetilde}
\newcommand{\wh}{\widehat}

\let\oldvec\vec
\renewcommand{\vec}{\overrightarrow}

% Derivadas parciales
\newcommand{\del}[2]{\frac{\partial #1}{\partial #2}}
\newcommand{\Del}[3]{\frac{\partial^{#1} #2}{\partial #3^{#1}}}
\newcommand{\deld}[2]{\dfrac{\partial #1}{\partial #2}}
\newcommand{\Deld}[3]{\dfrac{\partial^{#1} #2}{\partial #3^{#1}}}


\newcommand{\AstIg}{\stackrel{(\ast)}{=}}
\newcommand{\Hop}{\stackrel{L'H\hat{o}pital}{=}}

\newcommand{\red}[1]{{\color{red}#1}} % Para integrales, destacar los cambios.

% Método de integración
\newcommand{\MetInt}[2]{
    \left[\begin{array}{c}
        #1 \\ #2
    \end{array}\right]
}

% Declarar aplicaciones
% 1. Nombre aplicación
% 2. Dominio
% 3. Codominio
% 4. Variable
% 5. Imagen de la variable
\newcommand{\Func}[5]{
    \begin{equation*}
        \begin{array}{rrll}
            #1:& #2 & \longrightarrow & #3\\
               & #4 & \longmapsto & #5
        \end{array}
    \end{equation*}
}

%------------------------------------------------------------------------

\usepackage{extarrows}
\usepackage{stackrel}

\begin{document}

    % 1. Foto de fondo
    % 2. Título
    % 3. Encabezado Izquierdo
    % 4. Color de fondo
    % 5. Coord x del titulo
    % 6. Coord y del titulo
    % 7. Fecha

    
    % 1. Foto de fondo
% 2. Título
% 3. Encabezado Izquierdo
% 4. Color de fondo
% 5. Coord x del titulo
% 6. Coord y del titulo
% 7. Fecha

\newcommand{\portada}[7]{

    \portadaBase{#1}{#2}{#3}{#4}{#5}{#6}{#7}
    \portadaBook{#1}{#2}{#3}{#4}{#5}{#6}{#7}
}

\newcommand{\portadaExamen}[7]{

    \portadaBase{#1}{#2}{#3}{#4}{#5}{#6}{#7}
    \portadaArticle{#1}{#2}{#3}{#4}{#5}{#6}{#7}
}




\newcommand{\portadaBase}[7]{

    % Tiene la portada principal y la licencia Creative Commons
    
    % 1. Foto de fondo
    % 2. Título
    % 3. Encabezado Izquierdo
    % 4. Color de fondo
    % 5. Coord x del titulo
    % 6. Coord y del titulo
    % 7. Fecha
    
    
    \thispagestyle{empty}               % Sin encabezado ni pie de página
    \newgeometry{margin=0cm}        % Márgenes nulos para la primera página
    
    
    % Encabezado
    \fancyhead[L]{\helv #3}
    \fancyhead[R]{\helv \nouppercase{\leftmark}}
    
    
    \pagecolor{#4}        % Color de fondo para la portada
    
    \begin{figure}[p]
        \centering
        \transparent{0.3}           % Opacidad del 30% para la imagen
        
        \includegraphics[width=\paperwidth, keepaspectratio]{assets/#1}
    
        \begin{tikzpicture}[remember picture, overlay]
            \node[anchor=north west, text=white, opacity=1, font=\fontsize{60}{90}\selectfont\bfseries\sffamily, align=left] at (#5, #6) {#2};
            
            \node[anchor=south east, text=white, opacity=1, font=\fontsize{12}{18}\selectfont\sffamily, align=right] at (9.7, 3) {\textbf{\href{https://losdeldgiim.github.io/}{Los Del DGIIM}}};
            
            \node[anchor=south east, text=white, opacity=1, font=\fontsize{12}{15}\selectfont\sffamily, align=right] at (9.7, 1.8) {Doble Grado en Ingeniería Informática y Matemáticas\\Universidad de Granada};
        \end{tikzpicture}
    \end{figure}
    
    
    \restoregeometry        % Restaurar márgenes normales para las páginas subsiguientes
    \pagecolor{white}       % Restaurar el color de página
    
    
    \newpage
    \thispagestyle{empty}               % Sin encabezado ni pie de página
    \begin{tikzpicture}[remember picture, overlay]
        \node[anchor=south west, inner sep=3cm] at (current page.south west) {
            \begin{minipage}{0.5\paperwidth}
                \href{https://creativecommons.org/licenses/by-nc-nd/4.0/}{
                    \includegraphics[height=2cm]{assets/Licencia.png}
                }\vspace{1cm}\\
                Esta obra está bajo una
                \href{https://creativecommons.org/licenses/by-nc-nd/4.0/}{
                    Licencia Creative Commons Atribución-NoComercial-SinDerivadas 4.0 Internacional (CC BY-NC-ND 4.0).
                }\\
    
                Eres libre de compartir y redistribuir el contenido de esta obra en cualquier medio o formato, siempre y cuando des el crédito adecuado a los autores originales y no persigas fines comerciales. 
            \end{minipage}
        };
    \end{tikzpicture}
    
    
    
    % 1. Foto de fondo
    % 2. Título
    % 3. Encabezado Izquierdo
    % 4. Color de fondo
    % 5. Coord x del titulo
    % 6. Coord y del titulo
    % 7. Fecha


}


\newcommand{\portadaBook}[7]{

    % 1. Foto de fondo
    % 2. Título
    % 3. Encabezado Izquierdo
    % 4. Color de fondo
    % 5. Coord x del titulo
    % 6. Coord y del titulo
    % 7. Fecha

    % Personaliza el formato del título
    \pretitle{\begin{center}\bfseries\fontsize{42}{56}\selectfont}
    \posttitle{\par\end{center}\vspace{2em}}
    
    % Personaliza el formato del autor
    \preauthor{\begin{center}\Large}
    \postauthor{\par\end{center}\vfill}
    
    % Personaliza el formato de la fecha
    \predate{\begin{center}\huge}
    \postdate{\par\end{center}\vspace{2em}}
    
    \title{#2}
    \author{\href{https://losdeldgiim.github.io/}{Los Del DGIIM}}
    \date{Granada, #7}
    \maketitle
    
    \tableofcontents
}




\newcommand{\portadaArticle}[7]{

    % 1. Foto de fondo
    % 2. Título
    % 3. Encabezado Izquierdo
    % 4. Color de fondo
    % 5. Coord x del titulo
    % 6. Coord y del titulo
    % 7. Fecha

    % Personaliza el formato del título
    \pretitle{\begin{center}\bfseries\fontsize{42}{56}\selectfont}
    \posttitle{\par\end{center}\vspace{2em}}
    
    % Personaliza el formato del autor
    \preauthor{\begin{center}\Large}
    \postauthor{\par\end{center}\vspace{3em}}
    
    % Personaliza el formato de la fecha
    \predate{\begin{center}\huge}
    \postdate{\par\end{center}\vspace{5em}}
    
    \title{#2}
    \author{\href{https://losdeldgiim.github.io/}{Los Del DGIIM}}
    \date{Granada, #7}
    \thispagestyle{empty}               % Sin encabezado ni pie de página
    \maketitle
    \vfill
}
    \portadaExamen{ffccA4.jpg}{Cálculo I\\Examen I}{Cálculo I. Examen I}{MidnightBlue}{-8}{28}{2023-2024}{Jesús Muñoz Velasco\\Arturo Olivares Martos}

    
    \begin{description}
        \item[Asignatura] Cálculo I.
        \item[Curso Académico] 2021-22.
        \item[Grado] Doble Grado en Ingeniería Informática y Matemáticas.
        \item[Grupo] Único.
        \item[Profesor] Jose Luis Gámez Ruiz.
        \item[Descripción] Convocatoria Ordinaria.
        \item[Fecha] 20 de enero de 2022.
        \item[Duración] 3 horas.
    
    \end{description}
    \newpage

    \begin{ejercicio}[3 puntos]
        Teorema (de los ceros) de Bolzano. Enunciado y demostración.

        Sean $a,b\in \bb{R}$ con $a<b$ y $f:[a,b] \longrightarrow \bb{R}$ continua verificando que
        \[
            f(a)f(b)<0 \,\, (f(a) \text{ y } f(b) \text{ tienen distinto signo})
        \]

        Entonces, $\exists c \in ]a,b[$ tal que $f(c)=0$

        \begin{proof}
            Supongamos que $f(a) < 0 < f(b)$. Definimos el conjunto $C$ como sigue:
            \begin{equation*}
                C= \{x \in [a,b] ~:~ f(x) < 0\}
            \end{equation*}
            Es fácil ver que $C$ es un conjunto de números reales no vacío y mayorado. Sea $c = \sup C$.
            Es claro que $c \in [a,b]$. Entonces, existe una sucesión $\{x_n\}$ de elementos de $C$ convergente
            a $c$ y por continuidad de $f$ en $c$ entonces $\{f(x_n)\} \longrightarrow f(c)$. Dado que
            \begin{equation*}
                f(x_n) < 0, ~ \forall n \in \mathbb{N}
            \end{equation*}
            entonces $f(c) \leq 0$. En particular, deducimos que $c \neq b$ y $c \leq b$, por lo que $c \in ~ ]a,b[$.\\
            
            Sea $\{z_n\} = \left\{c + \frac{b-c}{n}\right\}$. Es claro que
            \begin{equation*}
                z_n \in [a,b] ~ \text{y} ~ z_n \notin C, ~ \forall n \in \mathbb{N}
            \end{equation*}
            y por tanto ha de ser
            \begin{equation*}
                f(z_n) \geq 0, ~ \forall n \in \mathbb{N}
            \end{equation*}
            
            Evidentemente, $\{z_n\} \longrightarrow c$ y usando que $f$ es continua en $c$ y lo anterior deducimos que
            \begin{equation*}
                \{f(z_n)\} \longrightarrow f(c) \geq 0
            \end{equation*}
            y por tanto ha de ser $f(c) = 0$.\\
            
            Si fuera $f(b) < 0 < f(a)$, podemos razonar igual que antes o aplicar lo que acabamos de obtener a la función $-f$
            ($f$ es continua si y sólo si lo es $-f$).
    \end{proof}
    \end{ejercicio}
    
    \begin{ejercicio}[2 puntos]
        Un tren hace el recorrido Madrid-Zaragoza un día entre las 10 y las 12. Al día siguiente, dicho tren hace el mismo recorrido en dirección contraria y con el mismo horario.Prueba que existe una determinada hora del segundo día a la que el tren se encuentra exactamente a la misma distancia de Madrid que el primer día a esa misma hora.\\
    
        Sea $f:[10,12]\longrightarrow \bb{R}$ la función que el \underline{primer día} mide la ``distancia a Madrid'' en cada instante:
        \[
            f(x) = \text{``distancia a Madrid'' a la hora $x$ , } \forall x \in [10,12]
        \]
        \begin{itemize}
            \item $f$ continua en $[10,12]$
            \item $f(10)=0$
            \item $f(12) = D$ (distancia Madrid-Zaragoza, positiva)
        \end{itemize}

        Del mismo modo, $g:[10,12]\longrightarrow \bb{R}$ es la función ``distancia a Madrid'' en cada instante del \underline{segundo día}
        \begin{itemize}
            \item $g$ continua en $[10,12]$
            \item $g(10)=D>0$
            \item $g(12)=0$
        \end{itemize}

        Consideramos la función $h:[10,12]\longrightarrow \bb{R}$ dada por $h(x) = f(x)-g(x)$ $\forall x \in [10,12]$.
        Tenemos que $h$ continua en $[10,12]$. Además:
        \begin{itemize}
            \item $h(10)= f(10)-g(10) = -D$.
            \item $h(12)=f(12)-g(12) = D$.
        \end{itemize}

            Por tanto, $h(10) \cdot h(12) = -D^2 < 0$. Por el Teorema (de los ceros) de Bolzano:
            \[
                \exists x \in ]10,12[ : h(c) = 0 \text{ esto es:}
            \]
            \[
                0= h(c) = f(c) - g(c) \Longrightarrow f(c) = g(c)
            \]
            La hora buscada es $c \in]10,12[$.
        
    \end{ejercicio}

    \begin{ejercicio}[3 puntos]
        Estudia la convergencia de las siguientes sucesiones y calcula su límite (si existe):
        \begin{enumerate}
            \item $x_n = \dfrac{1}{\sqrt{n}}\left( \dfrac{1}{\sqrt{1}} +  \dfrac{1}{\sqrt{2}} + ... + \dfrac{1}{\sqrt{n}}\right)$ 

            Defino $x_n = \dfrac{a_n}{b_n}$ donde $a_n = \dfrac{1}{\sqrt{1}} +  \dfrac{1}{\sqrt{2}} + ... + \dfrac{1}{\sqrt{n}}$ y $b_n = \sqrt{n} \nearrow \nearrow +\infty$ (puedo aplicar Stolz)

            \[ \left(
                \begin{array}{c}
                    \text{Criterio de Stolz: }(b_n \nearrow \nearrow +\infty)\vspace{0.2cm}\\
                     \text{Si }\hspace{0.2cm} \dfrac{a_{n+1}-a_n}{b_{n+1}-b_n} \longrightarrow L \Longrightarrow \dfrac{a_n}{b_n} \longrightarrow L
                \end{array}
                \right)
            \]
            
            \[
                \dfrac{a_{n+1}-a_n}{b_{n+1}-b_n} = \dfrac{\frac{1}{\sqrt{n+1}}}{\sqrt{n+1} - \sqrt{n}}= \dfrac{\sqrt{n+1}+ \sqrt{n}}{\sqrt{n+1}} = 1 + \sqrt{\dfrac{n}{n+1}} \longrightarrow 2
            \]

            Luego $x_n = \dfrac{a_n}{b_n} \longrightarrow 2$

            
            \item $x_n = \dfrac{\sqrt[n]{n!}}{n}$
            \[
                \dfrac{\sqrt[n]{n!}}{n}= \sqrt[n]{\dfrac{n!}{n^n}} = \sqrt[n]{a_n} \text{ donde } a_n = \dfrac{n!}{n^n}
            \]

            \[ \left(
                \begin{array}{c}
                    \text{Criterio del cociente para sucesiones: }(a_n >0 \hspace{0.3cm} \forall n \in \bb{N})\vspace{0.2cm}\\
                     \text{Si }\hspace{0.2cm} \dfrac{a_{n+1}}{a_n} \longrightarrow L \Longrightarrow \sqrt[n]{a_n} \longrightarrow L
                \end{array}
                \right)
            \]

            \[
                \dfrac{a_{n+1}}{a_n} = \dfrac{(n+1)! \cdot n^n}{(n+1)^{n+1} \cdot n!} = \dfrac{\cancel{(n+1)}}{\cancel{(n+1)}} \left( \dfrac{n}{n+1} \right)^n = \left( \dfrac{n}{n+1} \right)^n = z_n^{y_n}
            \]
            
            \[
                \text{donde } y_n=n, \hspace{0.25cm} z_n = \dfrac{n}{n+1} \longrightarrow 1
            \]

            \[ \left(
                \begin{array}{c}
                    \text{Criterio ``exponencial'': }(z_n \longrightarrow 1)\vspace{0.2cm}\\
                     z_n^{y_n} \longrightarrow e^L \Longleftrightarrow y_n(z_n -1) \longrightarrow L
                \end{array}
                \right)
            \]

            \[
                y_n(z_n-1) = n \left( \dfrac{n}{n+1} -1 \right) = \dfrac{-n}{n+1} \longrightarrow -1 \Longrightarrow \left( \dfrac{n}{n+1} \right) ^n \longrightarrow e^{-1}
            \]

            Así, $\dfrac{a_{n+1}}{a_n} \longrightarrow e^{-1} \Longrightarrow x_n = \sqrt[n]{a_n} \longrightarrow e^{-1}$
            
            \item (Dada por recurrencia) $x_1 = 11$, $x_{n+1} = 2[\sqrt{5+x_n}-1]$, $\forall n \in \bb{N}$

            Probaremos que $x_n$ es decreciente y minorada por 4.\\
            (Por inducción, $4<x_{n+1}<x_n$ \, $\forall n \in \bb{N}$)

            \begin{itemize}
                \item \underline{n=1} \, ¿$4<x_2<x_1$? $\Longleftrightarrow 4 < 6 < 11$ \, Sí
                \item Supuesto que para un $n\in \bb{N}$, $4<x_{n+1}<x_n$ (hip. de ind.)\\
                ¿$\Rightarrow 4 \stackbin[(1)]{}{\,<\,} x_{n+2} \stackbin[(2)]{}{\,<\,} x_{n+1}$?\\

                \begin{itemize}
                    \item [(2)] $x_{n+2} = 2[\sqrt{5+x_{n+1}} -1 ] < 2[\sqrt{5+x_{n}} -1 ] = x_{n+1}$
                    \item [(1)] $x_{n+2} = 2[\sqrt{5+x_{n+1}} -1 ] > 2[\sqrt{5+4} -1 ] = 4$
                \end{itemize}
            \end{itemize}

            Luego $x_n$ es decreciente y minorada (por 4) $\Longrightarrow$ converge.\\
            Sea $L=\lim\{x_n\} \Longrightarrow \{x_{n+1}\} \longrightarrow L$ (parcial)
            \[
                \left .
                \begin{array}{r}
                     x_{n+1} \longrightarrow L \\\\
                     2[\sqrt{5+x_n}-1] \longrightarrow 2[\sqrt{5+L}-1]
                \end{array}
                \right\} \xLongrightarrow{\begin{array}{c}
                    \text{\scriptsize{(unicidad}} \\
                    \text{\scriptsize{del lim)}} 
                \end{array}} L=2[\sqrt{5+L}-1] 
            \]
            \[
                \Longleftrightarrow L^2 =16 \Longleftrightarrow 
                \left\{
                \begin{array}{l}
                     \, L=4\\
                     \xcancel{L=-4} \,\text{ (4 es minorante)}
                \end{array}
                \right.
            \]
        \end{enumerate}
    \end{ejercicio}

    \begin{ejercicio}[3 puntos]
        Estudia la convergencia de las series:
        \begin{enumerate}
            \item $\sum\limits_{n\geq 1} \left(\sqrt{n+1}- \sqrt{n}\right)^2$

            Defino $\sum\limits_{n\geq 1}a_n = \sum\limits_{n\geq 1} \left(\sqrt{n+1}- \sqrt{n}\right)^2$, donde $a_n = \left(\sqrt{n+1}- \sqrt{n}\right)^2 = \left(\dfrac{1}{\sqrt{n+1} + \sqrt{n}} \right)^2$\\
            
            Aplicaremos comparación (límite) con la serie $\sum\limits_{n \geq 1}\frac{1}{n} = \sum\limits_{n \geq 1}b_n$, con $b_n=\frac{1}{n}$

            \[
                \dfrac{a_n}{b_n} = \left( \dfrac{\sqrt{n}}{\sqrt{n+1} + \sqrt{n}}\right)^2 \longrightarrow \left( \dfrac{1}{2} \right)^2 = \dfrac{1}{4} \in \bb{R}^+
            \]

            Luego $\sum\limits_{n \geq 1} a_n$ converge $\Longleftrightarrow \sum\limits_{n \geq 1} b_n$\, , pero $\sum\limits_{n \geq 1} b_n$ no converge (Serie de Riemann, $\alpha = 1$).\\

            Por tanto $\sum\limits_{n \geq 1} a_n$ no converge.
            
            \item $\sum\limits_{n\geq 1} \dfrac{n!}{n^n}$

            Defino $\sum\limits_{n \geq 1} a_n = \sum\limits_{n \geq 1} \dfrac{n!}{n^n}$, con $a_n = \dfrac{n!}{n^n} \,\, \forall n \in \bb{N}$\\
            
            Aplicamos el criterio de la raíz y obtenemos, según se ha visto en el ejercicio anterior (Ejercicio 3.2), que $\sqrt[n]{a_n} \longrightarrow e^{-1}$.\\
            
            Por tanto, $\sqrt[n]{a_n} \longrightarrow e^{-1}= \frac{1}{e}<1 \Longrightarrow \sum\limits_{n \geq 1}a_n$ converge. 
            
            \item $ \sum\limits_{n\geq 1} \dfrac{\cos^3(n^2 + 7n - 10)}{n^2}$

            Defino $\sum\limits_{n \geq 1} a_n = \sum\limits_{n\geq 1} \dfrac{\cos^3(n^2 + 7n - 10)}{n^2}$, donde $a_n = \dfrac{\cos^3(n^2 + 7n - 10)}{n^2}$.\\
            (Términos sin restricción de signo)\\

            ¿Hay convergencia absoluta? $\left( \text{¿} \sum\limits_{n \geq 1}|a_n| \text{ converge?}\right)$\\

            Por el criterio de comparación:
            \[
                |a_n| = \left| \dfrac{\cos^3(n^2 + 7n - 10)}{n^2} \right| \leq \dfrac{1}{n^2}=b_n \,\, \forall n \in \bb{N}
            \]

            \[
                \sum\limits_{n\geq 1} b_n = \sum\limits_{n\geq 1}\dfrac{1}{n^2} \stackbin[\text{(Riemann, }\alpha=2)] {}{ \text{ converge}} \Longrightarrow \sum\limits_{n \geq 1} |a_n| \text{ converge}
            \]

            Así tenemos convergencia absoluta

            \[
                \sum\limits_{n \geq 1} |a_n| \text{ converge} \stackbin{\text{(crit. conv. abs.)}}{\Longrightarrow} \sum\limits_{n \geq 1} a_n \text{ converge}
            \]
            
        \end{enumerate}
    \end{ejercicio}



     
\end{document}
