\documentclass[12pt]{article}

% Idioma y codificación
\usepackage[spanish, es-tabla]{babel}       %es-tabla para que se titule "Tabla"
\usepackage[utf8]{inputenc}

% Márgenes
\usepackage[a4paper,top=3cm,bottom=2.5cm,left=3cm,right=3cm]{geometry}

% Comentarios de bloque
\usepackage{verbatim}

% Paquetes de links
\usepackage[hidelinks]{hyperref}    % Permite enlaces
\usepackage{url}                    % redirecciona a la web

% Más opciones para enumeraciones
\usepackage{enumitem}

% Personalizar la portada
\usepackage{titling}

% Paquetes de tablas
\usepackage{multirow}


%------------------------------------------------------------------------

%Paquetes de figuras
\usepackage{caption}
\usepackage{subcaption} % Figuras al lado de otras
\usepackage{float}      % Poner figuras en el sitio indicado H.


% Paquetes de imágenes
\usepackage{graphicx}       % Paquete para añadir imágenes
\usepackage{transparent}    % Para manejar la opacidad de las figuras

% Paquete para usar colores
\usepackage[dvipsnames]{xcolor}
\usepackage{pagecolor}      % Para cambiar el color de la página

% Habilita tamaños de fuente mayores
\usepackage{fix-cm}

% Para los gráficos
\usepackage{tikz}

% Para poder situar los nodos en los grafos
\usetikzlibrary{positioning}


%------------------------------------------------------------------------

% Paquetes de matemáticas
\usepackage{mathtools, amsfonts, amssymb, mathrsfs}
\usepackage[makeroom]{cancel}     % Simplificar tachando
\usepackage{polynom}    % Divisiones y Ruffini
\usepackage{units} % Para poner fracciones diagonales con \nicefrac

\usepackage{pgfplots}   %Representar funciones
\pgfplotsset{compat=1.18}  % Versión 1.18

\usepackage{tikz-cd}    % Para usar diagramas de composiciones
\usetikzlibrary{calc}   % Para usar cálculo de coordenadas en tikz

%Definición de teoremas, etc.
\usepackage{amsthm}
%\swapnumbers   % Intercambia la posición del texto y de la numeración

\theoremstyle{plain}

\makeatletter
\@ifclassloaded{article}{
  \newtheorem{teo}{Teorema}[section]
}{
  \newtheorem{teo}{Teorema}[chapter]  % Se resetea en cada chapter
}
\makeatother

\newtheorem{coro}{Corolario}[teo]           % Se resetea en cada teorema
\newtheorem{prop}[teo]{Proposición}         % Usa el mismo contador que teorema
\newtheorem{lema}[teo]{Lema}                % Usa el mismo contador que teorema

\theoremstyle{remark}
\newtheorem*{observacion}{Observación}

\theoremstyle{definition}

\makeatletter
\@ifclassloaded{article}{
  \newtheorem{definicion}{Definición} [section]     % Se resetea en cada chapter
}{
  \newtheorem{definicion}{Definición} [chapter]     % Se resetea en cada chapter
}
\makeatother

\newtheorem*{notacion}{Notación}
\newtheorem*{ejemplo}{Ejemplo}
\newtheorem*{ejercicio*}{Ejercicio}             % No numerado
\newtheorem{ejercicio}{Ejercicio} [section]     % Se resetea en cada section


% Modificar el formato de la numeración del teorema "ejercicio"
\renewcommand{\theejercicio}{%
  \ifnum\value{section}=0 % Si no se ha iniciado ninguna sección
    \arabic{ejercicio}% Solo mostrar el número de ejercicio
  \else
    \thesection.\arabic{ejercicio}% Mostrar número de sección y número de ejercicio
  \fi
}


% \renewcommand\qedsymbol{$\blacksquare$}         % Cambiar símbolo QED
%------------------------------------------------------------------------

% Paquetes para encabezados
\usepackage{fancyhdr}
\pagestyle{fancy}
\fancyhf{}

\newcommand{\helv}{ % Modificación tamaño de letra
\fontfamily{}\fontsize{12}{12}\selectfont}
\setlength{\headheight}{15pt} % Amplía el tamaño del índice


%\usepackage{lastpage}   % Referenciar última pag   \pageref{LastPage}
\fancyfoot[C]{\thepage}

%------------------------------------------------------------------------

% Conseguir que no ponga "Capítulo 1". Sino solo "1."
\makeatletter
\@ifclassloaded{book}{
  \renewcommand{\chaptermark}[1]{\markboth{\thechapter.\ #1}{}} % En el encabezado
    
  \renewcommand{\@makechapterhead}[1]{%
  \vspace*{50\p@}%
  {\parindent \z@ \raggedright \normalfont
    \ifnum \c@secnumdepth >\m@ne
      \huge\bfseries \thechapter.\hspace{1em}\ignorespaces
    \fi
    \interlinepenalty\@M
    \Huge \bfseries #1\par\nobreak
    \vskip 40\p@
  }}
}
\makeatother

%------------------------------------------------------------------------
% Paquetes de cógido
\usepackage{minted}
\renewcommand\listingscaption{Código fuente}

\usepackage{fancyvrb}
% Personaliza el tamaño de los números de línea
\renewcommand{\theFancyVerbLine}{\small\arabic{FancyVerbLine}}

% Estilo para C++
\newminted{cpp}{
    frame=lines,
    framesep=2mm,
    baselinestretch=1.2,
    linenos,
    escapeinside=||
}

% para minted
\definecolor{LightGray}{rgb}{0.95,0.95,0.92}
\setminted{
    linenos=true,
    stepnumber=5,
    numberfirstline=true,
    autogobble,
    breaklines=true,
    breakautoindent=true,
    breaksymbolleft=,
    breaksymbolright=,
    breaksymbolindentleft=0pt,
    breaksymbolindentright=0pt,
    breaksymbolsepleft=0pt,
    breaksymbolsepright=0pt,
    fontsize=\footnotesize,
    bgcolor=LightGray,
    numbersep=10pt
}


\usepackage{listings} % Para incluir código desde un archivo

\renewcommand\lstlistingname{Código Fuente}
\renewcommand\lstlistlistingname{Índice de Códigos Fuente}

% Definir colores
\definecolor{vscodepurple}{rgb}{0.5,0,0.5}
\definecolor{vscodeblue}{rgb}{0,0,0.8}
\definecolor{vscodegreen}{rgb}{0,0.5,0}
\definecolor{vscodegray}{rgb}{0.5,0.5,0.5}
\definecolor{vscodebackground}{rgb}{0.97,0.97,0.97}
\definecolor{vscodelightgray}{rgb}{0.9,0.9,0.9}

% Configuración para el estilo de C similar a VSCode
\lstdefinestyle{vscode_C}{
  backgroundcolor=\color{vscodebackground},
  commentstyle=\color{vscodegreen},
  keywordstyle=\color{vscodeblue},
  numberstyle=\tiny\color{vscodegray},
  stringstyle=\color{vscodepurple},
  basicstyle=\scriptsize\ttfamily,
  breakatwhitespace=false,
  breaklines=true,
  captionpos=b,
  keepspaces=true,
  numbers=left,
  numbersep=5pt,
  showspaces=false,
  showstringspaces=false,
  showtabs=false,
  tabsize=2,
  frame=tb,
  framerule=0pt,
  aboveskip=10pt,
  belowskip=10pt,
  xleftmargin=10pt,
  xrightmargin=10pt,
  framexleftmargin=10pt,
  framexrightmargin=10pt,
  framesep=0pt,
  rulecolor=\color{vscodelightgray},
  backgroundcolor=\color{vscodebackground},
}

%------------------------------------------------------------------------

% Comandos definidos
\newcommand{\bb}[1]{\mathbb{#1}}
\newcommand{\cc}[1]{\mathcal{#1}}

% I prefer the slanted \leq
\let\oldleq\leq % save them in case they're every wanted
\let\oldgeq\geq
\renewcommand{\leq}{\leqslant}
\renewcommand{\geq}{\geqslant}

% Si y solo si
\newcommand{\sii}{\iff}

% Letras griegas
\newcommand{\eps}{\epsilon}
\newcommand{\veps}{\varepsilon}
\newcommand{\lm}{\lambda}

\newcommand{\ol}{\overline}
\newcommand{\ul}{\underline}
\newcommand{\wt}{\widetilde}
\newcommand{\wh}{\widehat}

\let\oldvec\vec
\renewcommand{\vec}{\overrightarrow}

% Derivadas parciales
\newcommand{\del}[2]{\frac{\partial #1}{\partial #2}}
\newcommand{\Del}[3]{\frac{\partial^{#1} #2}{\partial #3^{#1}}}
\newcommand{\deld}[2]{\dfrac{\partial #1}{\partial #2}}
\newcommand{\Deld}[3]{\dfrac{\partial^{#1} #2}{\partial #3^{#1}}}


\newcommand{\AstIg}{\stackrel{(\ast)}{=}}
\newcommand{\Hop}{\stackrel{L'H\hat{o}pital}{=}}

\newcommand{\red}[1]{{\color{red}#1}} % Para integrales, destacar los cambios.

% Método de integración
\newcommand{\MetInt}[2]{
    \left[\begin{array}{c}
        #1 \\ #2
    \end{array}\right]
}

% Declarar aplicaciones
% 1. Nombre aplicación
% 2. Dominio
% 3. Codominio
% 4. Variable
% 5. Imagen de la variable
\newcommand{\Func}[5]{
    \begin{equation*}
        \begin{array}{rrll}
            #1:& #2 & \longrightarrow & #3\\
               & #4 & \longmapsto & #5
        \end{array}
    \end{equation*}
}

%------------------------------------------------------------------------


\begin{document}

    % 1. Foto de fondo
    % 2. Título
    % 3. Encabezado Izquierdo
    % 4. Color de fondo
    % 5. Coord x del titulo
    % 6. Coord y del titulo
    % 7. Fecha
    \newcommand{\N}{{\mathbb{N}}} % Enteros
    \newcommand{\Q}{{\mathbb{Q}}} % Racionales
    \newcommand{\R}{{\mathbb{R}}} % Reales
    
    % 1. Foto de fondo
% 2. Título
% 3. Encabezado Izquierdo
% 4. Color de fondo
% 5. Coord x del titulo
% 6. Coord y del titulo
% 7. Fecha

\newcommand{\portada}[7]{

    \portadaBase{#1}{#2}{#3}{#4}{#5}{#6}{#7}
    \portadaBook{#1}{#2}{#3}{#4}{#5}{#6}{#7}
}

\newcommand{\portadaExamen}[7]{

    \portadaBase{#1}{#2}{#3}{#4}{#5}{#6}{#7}
    \portadaArticle{#1}{#2}{#3}{#4}{#5}{#6}{#7}
}




\newcommand{\portadaBase}[7]{

    % Tiene la portada principal y la licencia Creative Commons
    
    % 1. Foto de fondo
    % 2. Título
    % 3. Encabezado Izquierdo
    % 4. Color de fondo
    % 5. Coord x del titulo
    % 6. Coord y del titulo
    % 7. Fecha
    
    
    \thispagestyle{empty}               % Sin encabezado ni pie de página
    \newgeometry{margin=0cm}        % Márgenes nulos para la primera página
    
    
    % Encabezado
    \fancyhead[L]{\helv #3}
    \fancyhead[R]{\helv \nouppercase{\leftmark}}
    
    
    \pagecolor{#4}        % Color de fondo para la portada
    
    \begin{figure}[p]
        \centering
        \transparent{0.3}           % Opacidad del 30% para la imagen
        
        \includegraphics[width=\paperwidth, keepaspectratio]{assets/#1}
    
        \begin{tikzpicture}[remember picture, overlay]
            \node[anchor=north west, text=white, opacity=1, font=\fontsize{60}{90}\selectfont\bfseries\sffamily, align=left] at (#5, #6) {#2};
            
            \node[anchor=south east, text=white, opacity=1, font=\fontsize{12}{18}\selectfont\sffamily, align=right] at (9.7, 3) {\textbf{\href{https://losdeldgiim.github.io/}{Los Del DGIIM}}};
            
            \node[anchor=south east, text=white, opacity=1, font=\fontsize{12}{15}\selectfont\sffamily, align=right] at (9.7, 1.8) {Doble Grado en Ingeniería Informática y Matemáticas\\Universidad de Granada};
        \end{tikzpicture}
    \end{figure}
    
    
    \restoregeometry        % Restaurar márgenes normales para las páginas subsiguientes
    \pagecolor{white}       % Restaurar el color de página
    
    
    \newpage
    \thispagestyle{empty}               % Sin encabezado ni pie de página
    \begin{tikzpicture}[remember picture, overlay]
        \node[anchor=south west, inner sep=3cm] at (current page.south west) {
            \begin{minipage}{0.5\paperwidth}
                \href{https://creativecommons.org/licenses/by-nc-nd/4.0/}{
                    \includegraphics[height=2cm]{assets/Licencia.png}
                }\vspace{1cm}\\
                Esta obra está bajo una
                \href{https://creativecommons.org/licenses/by-nc-nd/4.0/}{
                    Licencia Creative Commons Atribución-NoComercial-SinDerivadas 4.0 Internacional (CC BY-NC-ND 4.0).
                }\\
    
                Eres libre de compartir y redistribuir el contenido de esta obra en cualquier medio o formato, siempre y cuando des el crédito adecuado a los autores originales y no persigas fines comerciales. 
            \end{minipage}
        };
    \end{tikzpicture}
    
    
    
    % 1. Foto de fondo
    % 2. Título
    % 3. Encabezado Izquierdo
    % 4. Color de fondo
    % 5. Coord x del titulo
    % 6. Coord y del titulo
    % 7. Fecha


}


\newcommand{\portadaBook}[7]{

    % 1. Foto de fondo
    % 2. Título
    % 3. Encabezado Izquierdo
    % 4. Color de fondo
    % 5. Coord x del titulo
    % 6. Coord y del titulo
    % 7. Fecha

    % Personaliza el formato del título
    \pretitle{\begin{center}\bfseries\fontsize{42}{56}\selectfont}
    \posttitle{\par\end{center}\vspace{2em}}
    
    % Personaliza el formato del autor
    \preauthor{\begin{center}\Large}
    \postauthor{\par\end{center}\vfill}
    
    % Personaliza el formato de la fecha
    \predate{\begin{center}\huge}
    \postdate{\par\end{center}\vspace{2em}}
    
    \title{#2}
    \author{\href{https://losdeldgiim.github.io/}{Los Del DGIIM}}
    \date{Granada, #7}
    \maketitle
    
    \tableofcontents
}




\newcommand{\portadaArticle}[7]{

    % 1. Foto de fondo
    % 2. Título
    % 3. Encabezado Izquierdo
    % 4. Color de fondo
    % 5. Coord x del titulo
    % 6. Coord y del titulo
    % 7. Fecha

    % Personaliza el formato del título
    \pretitle{\begin{center}\bfseries\fontsize{42}{56}\selectfont}
    \posttitle{\par\end{center}\vspace{2em}}
    
    % Personaliza el formato del autor
    \preauthor{\begin{center}\Large}
    \postauthor{\par\end{center}\vspace{3em}}
    
    % Personaliza el formato de la fecha
    \predate{\begin{center}\huge}
    \postdate{\par\end{center}\vspace{5em}}
    
    \title{#2}
    \author{\href{https://losdeldgiim.github.io/}{Los Del DGIIM}}
    \date{Granada, #7}
    \thispagestyle{empty}               % Sin encabezado ni pie de página
    \maketitle
    \vfill
}
    \portadaExamen{ffccA4.jpg}{Cálculo I\\Examen X}{Cálculo I. Examen X}{MidnightBlue}{-8}{28}{2025}{Víctor Naranjo Cabrera}

    \begin{description}
        \item[Asignatura] Cálculo I.
        \item[Curso Académico] 2024-25.
        \item[Grado] Doble Grado en Ingeniería Informática y Matemáticas.
        \item[Grupo] Único.
        \item[Profesor] José Luis Gámez Ruiz.
        \item[Descripción] Convocatoria Ordinaria.
        \item[Fecha] 10 de enero de 2025.
        \item[Duración] 3 horas.
    
    \end{description}
    \newpage


    % ------------------------------------
    
    \begin{ejercicio}[]
        [\textbf{2 puntos}] Enuncia y demuestra el Teorema (de los ceros) de Bolzano. \\
    \end{ejercicio}
    
    \begin{ejercicio}[]
        [\textbf{1 punto}] Estudia la convergencia de la sucesión $\left\{\dfrac{n \log{n}}{\log(n!)}\right\}$. \\
    \end{ejercicio}

    \begin{ejercicio}
    Sea la sucesión recurrente $a_1 = 1, a_{n+1} = \sqrt{1 + a_1 + a_2 +  \dotsc + a_n}, {\forall n \in \N}$. Se pide:
        \begin{enumerate}[label=\alph*)]
            \item\ [\textbf{1 punto}] Prueba que $\{a_n\}$ es estrictamente creciente y diverge positivamente (observa que $a^2_{n+1} = a^2_n + a_n$, puedes usarlo).
            \item\ [\textbf{1 punto}] Prueba que $\left\{\dfrac{a^2_{n+1}}{a^2_n}\right\} \rightarrow 1$ (y, en particular, $\left\{\dfrac{a_{n+1}}{a_n}\right\} \rightarrow 1$).
            \item\ [\textbf{1 punto}] Prueba que $\{a_{n+1} - a_n\} \rightarrow \dfrac{1}{2}$.
            \item\ [\textbf{1 punto}] Prueba que $\left\{\dfrac{n}{a_n}\right\} \rightarrow 2$.
            \item\ [\textbf{1 punto}] Discute la convergencia de las series: \ $\displaystyle \sum\limits_{n\geq 1} \dfrac{1}{a_n}$ \ y \ $\displaystyle \sum\limits_{n\geq1} \dfrac{1}{a^2_n}$. \\
        \end{enumerate}
    \end{ejercicio}

    \begin{ejercicio}
        Sea $f: \R^+_0 \rightarrow \R$ continua, tal que $f(x) \geq x, \forall x \in \R^+_0.$ Se pide:
        \begin{enumerate}[label=\alph*)]
            \item\ [\textbf{1 punto}] Probar que $f$ alcanza un mínimo absoluto.
            \item\ [\textbf{1 punto}] Probar que $\exists c > 0 $ tal que $f(c) = \dfrac{1}{c}$. \\
        \end{enumerate}
    \end{ejercicio}

    \newpage
    \setcounter{ejercicio}{0} % Reseteo de contador para ejercicios resueltos

    \begin{ejercicio}[]
        [\textbf{2 puntos}] Enuncia y demuestra el Teorema (de los ceros) de Bolzano. \\

        \noindent
        Sean $a, b \in \R, a < b, f:[a, b] \rightarrow \R$ continua con $f(a) \cdot f(b) < 0$. Entonces:
        \begin{equation*}
            \exists c \in \left]a, b\right[ \colon f(c) = 0
        \end{equation*}
        \textit{Demostración}. Hay dos posibilidades, o bien $f(a) < 0 < f(b),$ o $f(a) > 0 > f(b)$
        \begin{description}
            \item[Caso f(a) $<$ 0 $<$ f(b).] Sea el conjunto $A = \{x \in [a,b]: f(x) < 0\}$ \\\\
            Claramente, el conjunto $A$ es no vacío ($a \in A$) y, por ser $A \subseteq [a, b] \Rightarrow A$ mayorado ($b \in M(A)$). Por tanto, $\exists c = sup(A),$ con $a \leq c \leq b$. \\ Usando la caracterización de supremo mediante sucesiones:
            \begin{align*}
                 \exists  \ & \{x_n\} \longrightarrow c. \text{ Por ser $f$ continua en c} \Rightarrow \{f(x_n)\} \rightarrow f(c) \\
                & ^{(x_n \in A \ \forall n \in \N)}
            \end{align*}
            y, como:
            \begin{equation}
                x_n \in A, f(x_n) < 0 \ \forall n \in \N \Rightarrow f(c) \leq 0
            \end{equation} \\
            En particular, $c \neq b$ (porque $f(b) > 0$) $\Rightarrow c < b$.
            Así, $\forall y \in \left]c, b\right] \Rightarrow f(y) \geq 0$. Podemos tomar:
            \begin{equation*}
                \{y_n\} = \left\{c + \frac{b-c}{n}\right\} \rightarrow c, \text{con } f(y_n) \geq 0 \ \forall n \in \N
            \end{equation*}
            De nuevo, por ser $f$ continua en c:
            \begin{align}
                &\{f(y_n)\} \rightarrow f(c) \Rightarrow f(c) \geq 0  \\
                & ^{(f(y_n) \geq 0)} \nonumber
            \end{align}
            Por (1) y (2) $\Rightarrow \boxed{f(c) = 0}$
        \item[Caso f(a) $>$ 0 $>$ f(b)] Podemos considerar $g:[a, b] \rightarrow \R, g = -f , \ g(x) = -f(x)$ ${\forall x \in [a, b]}$.
            La función $g$ verifica las hipótesis del caso anterior, luego
            \begin{equation*}
                \exists c \in \left]a, b\right[: g(c) = 0 \Rightarrow f(c) = -g(c) = 0.
            \end{equation*}
        \end{description}
    \end{ejercicio}
    
    \begin{ejercicio}[]
        [\textbf{1 punto}] Estudia la convergencia de la sucesión $\left\{\dfrac{n \log{n}}{\log(n!)}\right\}$. \\

        \noindent 
        Llamemos $\{a_n\} = \{n \log(n)\}, \{b_n\} = \{\log(n!)\}$. Por ser $\{b_n\} \nearrow\nearrow +\infty$ podemos aplicar el crit. de Stolz \\
        \begin{equation*}
            \left(Si \  \left\{\dfrac{a_{n+1} - a_n}{b_{n+1} - b_n}\right\} \longrightarrow L \Longrightarrow \left\{\dfrac{a_n}{b_n}\right\} \longrightarrow L \ (L \in \R \text{ ó } \pm \infty) \right)
        \end{equation*} \\
        Estudiemos, pues, la sucesión:
        \begin{gather*}
            \left\{\dfrac{a_{n+1} - a_n}{b_{n+1} - b_n}\right\} = \left\{\dfrac{(n+1)\log(n + 1) - n \log(n)}{log((n + 1)!) - log(n!)}\right\} = \left\{\dfrac{log(n+1) + n \log\left(\frac{n+1}{n}\right)}{log(n+1)}\right\} \\ =  \left\{1 + \dfrac{log((\frac{n+1}{n})^n)}{log(n+1)}\right\}
        \end{gather*}
        Llamemos $\{x_n\} = \left\{\left(\dfrac{n+ 1}{n}\right)\right\}$ y $\{y_n\} = \{n\}$
        Como $\{x_n\} \rightarrow 1$, podemos aplicar el crit. de Euler:
        \begin{equation*}
        (\{y_n(x_n-1)\} \rightarrow L \Longleftrightarrow \{x_n^{y_n} \} \rightarrow e^L, x_n > 0  \ \forall n \in \N)
        \end{equation*}
        Entonces:
        \begin{gather*}
            \{y_n(x_n - 1)\} = \left\{n\left(\dfrac{n + 1}{n} - 1\right) \right\} = \{1\}  \rightarrow 1 \Rightarrow \{x_n^{y_n}\} = \left\{\left(\dfrac{n+1}{n}\right)^n\right\} \rightarrow e \\
            \Rightarrow \left\{\log\left(\left(\dfrac{n+1}{n}\right)^n\right)\right\} \rightarrow log(e) = 1
        \end{gather*}
        Por lo tanto, $\left\{1 + \dfrac{log((\frac{n+1}{n})^n)}{log(n+1)}\right\} \rightarrow 1 + 0 = 1$. Y, por Stolz, la sucesión de partida $\boxed{\left\{\dfrac{a_n}{b_n}\right\} = \left\{\dfrac{n\log(n)}{log(n!)}\right\} \rightarrow 1}$
    \end{ejercicio}
    

    \begin{ejercicio}
        Sea la sucesión recurrente $a_1 = 1, a_{n+1} = \sqrt{1 + a_1 + a_2 +  ...  + a_n}, {\forall n \in \N}$. Se pide:
        \begin{enumerate}[label=\alph*)]
            \item\ [\textbf{1 punto}] Prueba que $\{a_n\}$ es estrictamente creciente y diverge positivamente (observa que $a^2_{n+1} = a^2_n + a_n$, puedes usarlo).

            Probemos que $a_n \geq 1 \ \forall n \in \N$ (en particular, $a_n > 0 \ \forall n \in \N$), por inducción. Sea el conjunto $A = \{n \in \N \mid a_n \geq 1\}$. Veamos si $A$ es inductivo:
            \begin{equation*}
                a_1 = 1 \geq 1 \Rightarrow 1 \in A
            \end{equation*}
            Sea $k \in A$, veamos si $(k + 1) \in A$:
            \begin{equation*}
                a_{k+1}^2 = a^2_k + a_k \Rightarrow a_{k+1} = \sqrt{a_k^2 + a_k} \geq 1 \Rightarrow (k + 1) \in A
            \end{equation*}
            Luego $a_n \geq 1 \ \forall n \in \N$. Así, $a_{n+1} > a_n \Leftrightarrow a_{n+1}^2 > a_n^2 \Leftrightarrow a_n^2 + a_n > a_n^2 \Leftrightarrow a_n > 0$, y $\{a_n\}$ estrictamente creciente. \\ Veamos si diverge por reducción al absurdo. Si $\{a_n\}$ no diverge, al ser estrictamente creciente, estará mayorada, luego $\{a_n\} \rightarrow L$.
            \begin{equation*}
             \left.\begin{array}{lll}
                  &  \{a_{n+1}^2\} \rightarrow L^2 \\
                  \\
                  &  \{a_n^2 + a_n\} \rightarrow L^2 + L
             \end{array} \right] \xRightarrow{\begin{array}{c}
                    \text{\scriptsize{(unicidad}} \\
                    \text{\scriptsize{del límite)}} 
                \end{array}} L^2 = L^2 + L \Rightarrow L = 0
            \end{equation*}
            \\
            Lo cual es una contradicción, ya que una sucesión creciente de positivos no puede converger a 0 y $\{a_n\} \rightarrow +\infty$
            
            \item\ [\textbf{1 punto}] Prueba que $\left\{\dfrac{a^2_{n+1}}{a^2_n}\right\} \xrightarrow{} 1$ y, en particular, $\left\{\dfrac{a_{n+1}}{a_n}\right\} \xrightarrow{} 1$.
            \begin{align*}
                 \left\{\dfrac{a_{n+1}^2}{a_n^2}\right\} = \left\{\dfrac{a_n^2 + a_n}{a_n^2}\right\}  = &\left\{ 1 + \dfrac{1}{a_n} \right\} 
                 \longrightarrow 1 + 0 = 1  \\
                & \left(\{a_n\} \rightarrow +\infty \Rightarrow \left\{\frac{1}{a_n}\right\} \rightarrow 0\right)\\
                 \text{En particular, } \left\{ \dfrac{a_{n+1}}{a_n}\right\}  = & \left\{\sqrt{ \dfrac{a_{n+1}^2}{a_n^2}}\right\} \rightarrow \sqrt{1} = 1
            \end{align*}
            \item\ [\textbf{1 punto}] Prueba que $\{a_{n+1} - a_n\} \rightarrow
            \dfrac{1}{2}$.
            \begin{gather*}
                \{a_{n+1} - a_n\} = \left\{\dfrac{(a_{n+1} - a_n)(a_{n+1} + a_n)}{a_{n+1} + a_n}\right\} = \left\{\dfrac{a_{n+1}^2 - a_n^2} {a_{n+1} + a_n}\right\} = \left\{\dfrac{\cancel{a_n^2} + a_n- \cancel{a_n^2}} {a_{n+1} + a_n}\right\} \\ = \left\{\frac{a_n}{a_{n+1} + a_n}\right\} = \left\{\frac{1}{\frac{a_{n+1}}{a_n} + 1}\right\}
            \end{gather*}
            Como  por el apartado anterior, $\left\{\dfrac{a_{n+1}}{a_n}\right\} \rightarrow 1 \Rightarrow\left\{\dfrac{1}{\frac{a_{n+1}}{a_n} + 1}\right\} \rightarrow \dfrac{1}{2}$
            \item\ [\textbf{1 punto}] Prueba que $\left\{\dfrac{n}{a_n}\right\} \xrightarrow{} 2$.
            Como el denominador es $\{a_n\} \nearrow \nearrow +\infty,$ podemos aplicar el criterio de Stolz (ya enunciado antes).
            Estudiamos, pues, esta otra sucesión:
            \begin{equation*}
                \left\{\dfrac{n + 1 - n}{a_{n+1} - a_n}\right\} = \left\{\dfrac{1}{a_{n+1} - a_n}\right\}
            \end{equation*}
            Como por el apartado anterior, $\{a_{n+1} - a_n\} \rightarrow \dfrac{1}{2} \Rightarrow \left\{\dfrac{1}{a_{n+1} - a_n}\right\} \rightarrow 2 \Rightarrow \left\{\dfrac{n}{a_n}\right\} \rightarrow 2.$
            \item\ [\textbf{1 punto}] Discute la convergencia de las series: \ $\displaystyle \sum\limits_{n\geq 1} \dfrac{1}{a_n}$ \ y \ $\displaystyle \sum\limits_{n\geq1} \dfrac{1}{a^2_n}$. \\
            Para estudiar la serie de térm. posit. $\displaystyle \sum\limits_{n \geq 1} \dfrac{1}{a_n},$ aplicaremos el criterio límite de comparación:
            \begin{equation*}
               \left(\begin{array}{l}
                   a_n \geq 0, b_n > 0 \ \forall n \in \N \land \left\{\dfrac{a_n}{b_n}\right\} \rightarrow L \ (L \in \R_0^+ \text{ ó } +\infty) \\
                  \text{Caso } L = 0: \text{Si} \displaystyle \sum\limits_{n \geq 1} b_n \text{ converge } \Rightarrow \displaystyle \sum\limits_{n \geq 1} a_n \text{ converge} \\
                 \text{Caso } L = +\infty: \text{Si} \displaystyle \sum\limits_{n \geq 1} a_n \text{ converge } \Rightarrow \displaystyle \sum\limits_{n \geq 1} b_n \text{ converge} \\
             \text{Caso } L = \R^+: \text{Si} \displaystyle \sum\limits_{n \geq 1} a_n \text{ converge } \Leftrightarrow \displaystyle \sum\limits_{n \geq 1} b_n \text{ converge}
                \end{array}\right)
            \end{equation*}
            Compararemos con la serie armónica $\displaystyle \sum\limits_{n \geq 1} \dfrac{1}{n}$:
            \begin{equation*}
                \left\{\dfrac{\frac{1}{a_n}}{\frac{1}{n}}\right\} = \left\{\frac{n}{a_n}\right\} \xrightarrow[\begin{array}{cc}
                     & \text{\scriptsize{(apdo.}} \\
                     & \text{\scriptsize{anterior)}}
                \end{array}]{} 2 \in \R^+
            \end{equation*}
                Así, $\displaystyle \sum\limits_{n \geq 1} \dfrac{1}{a_n}$ conv. $\Leftrightarrow \displaystyle \sum\limits_{n \geq 1} \dfrac{1}{n}$ conv. Como $\displaystyle \sum\limits_{n \geq 1} \dfrac{1}{n}$ no conv. $\Rightarrow \boxed{\displaystyle \sum\limits_{n \geq 1} \dfrac{1}{a_n}\ \text{no conv}.}$ \\
            Para estudiar la serie $\displaystyle \sum\limits_{n \geq 1} \frac{1}{a_n^2}$ aplicaremos de nuevo el criterio límite de comparación, comparado ahora con la serie $\displaystyle \sum\limits_{n \geq 1} \dfrac{1}{n^2}$
            \begin{equation*}
                \left\{\frac{\frac{1}{a_n^2}}{\frac{1}{ n^2}}\right\} = \left\{\frac{n^2}{a_n^2}\right\} = \left\{\left(\frac{n}{a_n}\right)^2\right\} \xrightarrow[\begin{array}{cc}
                     & \text{\scriptsize{(apdo.}} \\
                     & \text{\scriptsize{anterior)}}
                \end{array}]{} 2^2 = 4 \in \R^+
            \end{equation*}
            Así, $\displaystyle \sum\limits_{n \geq 1} \dfrac{1}{a_n^2}$ conv. $\Leftrightarrow \displaystyle \sum\limits_{n \geq 1} \dfrac{1}{n^2}$ conv. Como $\displaystyle \sum\limits_{n \geq 1} \dfrac{1}{n^2}$  conv. $\Rightarrow \boxed{\displaystyle \sum\limits_{n \geq 1} \dfrac{1}{a_n^2}\ \text{conv}.}$ \\
        \end{enumerate}
    \end{ejercicio}

    \begin{ejercicio}
        Sea $f: \R^+_0 \xrightarrow{} \R$ continua, tal que $f(x) \geq x, \forall x \in \R^+_0.$ Se pide:
        \begin{enumerate}[label=\alph*)]
            \item\ [\textbf{1 punto}] Probar que $f$ alcanza un mínimo absoluto. \\
            \  A partir de la hipótesis, probemos que Im(f) está minorada:
            \begin{equation*}
                \forall x \in \R_0^+, f(x) \geq x \geq 0 \Rightarrow 0 \text{ es minorante de Im(f)}. 
            \end{equation*}
            Como, además, $Im(f)$ es no vacío, $\exists \alpha = \inf(Im(f))$. Para probar si es el mínimo, bastará ver que $\alpha \in Im(f)$. Pör la caracterización de ínfimo mediante sucesiones:
            \begin{align*}
                \exists &\{ y_n\} \longrightarrow \alpha \\
                &^{(y_n \in Im(f) \forall n \in \N)}
            \end{align*}
            Observemos que:
            \begin{itemize}
                \item $\{y_n\}$ convergente $\Rightarrow \{y_n\}$ acotada, en particular. mayorada $\Rightarrow \exists M > 0: y_n \leq M, \forall n \in \N$
                \item $\forall n \in \N \  \exists x_n \in \R_0^+: y_n = f(x_n)$
            \end{itemize}
            Luego:
            \begin{align*}
                \forall n \in \N \ 0 \leq x_n \leq f(x_n) = y_n \leq M \Rightarrow \{x_n\} \text{ acotada}
            \end{align*}
            Por el Teorema de Bolzano-Weierstrass, por ser acotada, admitirá una parcial convergente, $\exists \{x_{\sigma(n)}\} \rightarrow x_0 \in \R^+_0$. \\ Luego
            \begin{equation*}
                \left.\begin{array}{ll}
                     &  \{f(x_{\sigma(n)})\} \xrightarrow{\text{\scriptsize{(f continua)}}} f(x_0) \\
                     & \{y_{\sigma(n)} \} \xrightarrow{\text{\scriptsize{(parcial)}}} \alpha
                \end{array}\right\} \xRightarrow{\begin{array}{cc}
                     & \text{\scriptsize{(unicidad}} \\
                     & \text{\scriptsize{del límite)}}
                \end{array}} \alpha = f(x_0)
            \end{equation*}
            Así, $\alpha \in Im(f)$ y es el valor mínimo absluto, que se alcanza en el punto $x_0$.
            \item\ [\textbf{1 punto}] Probar que $\exists c > 0 $ tal que $f(c) = \dfrac{1}{c}$. \\
            La proposición equivale a $c > 0: cf(c) - 1 = 0$. Consideremos entonces la función:
            $g:[0, 2] \rightarrow \R, g(x) = xf(x) - 1$. Vemos que es continua (pues es suma y producto de continuas). Además:
            \begin{gather*}
            \left.\begin{array}{ll}
                     & g(0) = 0 \cdot f(0) - 1 = -1 < 0 \\
                     & g(2) = 2\cdot f(2) - 1 \geq 2 \cdot 2 -1 = 3 > 0
            \end{array}\right\} \\\\ \text{Por el Tma. Bolzano (Ej 1) } \exists c \in ]0, 2[: g(c) = 0 \Rightarrow \exists c > 0, \text{ con } cf(c) - 1 = 0 \\ \Leftrightarrow f(c) = \nicefrac{1}{c}
            \end{gather*}
        \end{enumerate}
    \end{ejercicio}
    
\end{document}
