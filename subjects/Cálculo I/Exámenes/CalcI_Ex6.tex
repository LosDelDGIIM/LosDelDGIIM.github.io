\documentclass[12pt]{article}

% Idioma y codificación
\usepackage[spanish, es-tabla]{babel}       %es-tabla para que se titule "Tabla"
\usepackage[utf8]{inputenc}

% Márgenes
\usepackage[a4paper,top=3cm,bottom=2.5cm,left=3cm,right=3cm]{geometry}

% Comentarios de bloque
\usepackage{verbatim}

% Paquetes de links
\usepackage[hidelinks]{hyperref}    % Permite enlaces
\usepackage{url}                    % redirecciona a la web

% Más opciones para enumeraciones
\usepackage{enumitem}

% Personalizar la portada
\usepackage{titling}

% Paquetes de tablas
\usepackage{multirow}


%------------------------------------------------------------------------

%Paquetes de figuras
\usepackage{caption}
\usepackage{subcaption} % Figuras al lado de otras
\usepackage{float}      % Poner figuras en el sitio indicado H.


% Paquetes de imágenes
\usepackage{graphicx}       % Paquete para añadir imágenes
\usepackage{transparent}    % Para manejar la opacidad de las figuras

% Paquete para usar colores
\usepackage[dvipsnames]{xcolor}
\usepackage{pagecolor}      % Para cambiar el color de la página

% Habilita tamaños de fuente mayores
\usepackage{fix-cm}

% Para los gráficos
\usepackage{tikz}

% Para poder situar los nodos en los grafos
\usetikzlibrary{positioning}


%------------------------------------------------------------------------

% Paquetes de matemáticas
\usepackage{mathtools, amsfonts, amssymb, mathrsfs}
\usepackage[makeroom]{cancel}     % Simplificar tachando
\usepackage{polynom}    % Divisiones y Ruffini
\usepackage{units} % Para poner fracciones diagonales con \nicefrac

\usepackage{pgfplots}   %Representar funciones
\pgfplotsset{compat=1.18}  % Versión 1.18

\usepackage{tikz-cd}    % Para usar diagramas de composiciones
\usetikzlibrary{calc}   % Para usar cálculo de coordenadas en tikz

%Definición de teoremas, etc.
\usepackage{amsthm}
%\swapnumbers   % Intercambia la posición del texto y de la numeración

\theoremstyle{plain}

\makeatletter
\@ifclassloaded{article}{
  \newtheorem{teo}{Teorema}[section]
}{
  \newtheorem{teo}{Teorema}[chapter]  % Se resetea en cada chapter
}
\makeatother

\newtheorem{coro}{Corolario}[teo]           % Se resetea en cada teorema
\newtheorem{prop}[teo]{Proposición}         % Usa el mismo contador que teorema
\newtheorem{lema}[teo]{Lema}                % Usa el mismo contador que teorema

\theoremstyle{remark}
\newtheorem*{observacion}{Observación}

\theoremstyle{definition}

\makeatletter
\@ifclassloaded{article}{
  \newtheorem{definicion}{Definición} [section]     % Se resetea en cada chapter
}{
  \newtheorem{definicion}{Definición} [chapter]     % Se resetea en cada chapter
}
\makeatother

\newtheorem*{notacion}{Notación}
\newtheorem*{ejemplo}{Ejemplo}
\newtheorem*{ejercicio*}{Ejercicio}             % No numerado
\newtheorem{ejercicio}{Ejercicio} [section]     % Se resetea en cada section


% Modificar el formato de la numeración del teorema "ejercicio"
\renewcommand{\theejercicio}{%
  \ifnum\value{section}=0 % Si no se ha iniciado ninguna sección
    \arabic{ejercicio}% Solo mostrar el número de ejercicio
  \else
    \thesection.\arabic{ejercicio}% Mostrar número de sección y número de ejercicio
  \fi
}


% \renewcommand\qedsymbol{$\blacksquare$}         % Cambiar símbolo QED
%------------------------------------------------------------------------

% Paquetes para encabezados
\usepackage{fancyhdr}
\pagestyle{fancy}
\fancyhf{}

\newcommand{\helv}{ % Modificación tamaño de letra
\fontfamily{}\fontsize{12}{12}\selectfont}
\setlength{\headheight}{15pt} % Amplía el tamaño del índice


%\usepackage{lastpage}   % Referenciar última pag   \pageref{LastPage}
\fancyfoot[C]{\thepage}

%------------------------------------------------------------------------

% Conseguir que no ponga "Capítulo 1". Sino solo "1."
\makeatletter
\@ifclassloaded{book}{
  \renewcommand{\chaptermark}[1]{\markboth{\thechapter.\ #1}{}} % En el encabezado
    
  \renewcommand{\@makechapterhead}[1]{%
  \vspace*{50\p@}%
  {\parindent \z@ \raggedright \normalfont
    \ifnum \c@secnumdepth >\m@ne
      \huge\bfseries \thechapter.\hspace{1em}\ignorespaces
    \fi
    \interlinepenalty\@M
    \Huge \bfseries #1\par\nobreak
    \vskip 40\p@
  }}
}
\makeatother

%------------------------------------------------------------------------
% Paquetes de cógido
\usepackage{minted}
\renewcommand\listingscaption{Código fuente}

\usepackage{fancyvrb}
% Personaliza el tamaño de los números de línea
\renewcommand{\theFancyVerbLine}{\small\arabic{FancyVerbLine}}

% Estilo para C++
\newminted{cpp}{
    frame=lines,
    framesep=2mm,
    baselinestretch=1.2,
    linenos,
    escapeinside=||
}

% para minted
\definecolor{LightGray}{rgb}{0.95,0.95,0.92}
\setminted{
    linenos=true,
    stepnumber=5,
    numberfirstline=true,
    autogobble,
    breaklines=true,
    breakautoindent=true,
    breaksymbolleft=,
    breaksymbolright=,
    breaksymbolindentleft=0pt,
    breaksymbolindentright=0pt,
    breaksymbolsepleft=0pt,
    breaksymbolsepright=0pt,
    fontsize=\footnotesize,
    bgcolor=LightGray,
    numbersep=10pt
}


\usepackage{listings} % Para incluir código desde un archivo

\renewcommand\lstlistingname{Código Fuente}
\renewcommand\lstlistlistingname{Índice de Códigos Fuente}

% Definir colores
\definecolor{vscodepurple}{rgb}{0.5,0,0.5}
\definecolor{vscodeblue}{rgb}{0,0,0.8}
\definecolor{vscodegreen}{rgb}{0,0.5,0}
\definecolor{vscodegray}{rgb}{0.5,0.5,0.5}
\definecolor{vscodebackground}{rgb}{0.97,0.97,0.97}
\definecolor{vscodelightgray}{rgb}{0.9,0.9,0.9}

% Configuración para el estilo de C similar a VSCode
\lstdefinestyle{vscode_C}{
  backgroundcolor=\color{vscodebackground},
  commentstyle=\color{vscodegreen},
  keywordstyle=\color{vscodeblue},
  numberstyle=\tiny\color{vscodegray},
  stringstyle=\color{vscodepurple},
  basicstyle=\scriptsize\ttfamily,
  breakatwhitespace=false,
  breaklines=true,
  captionpos=b,
  keepspaces=true,
  numbers=left,
  numbersep=5pt,
  showspaces=false,
  showstringspaces=false,
  showtabs=false,
  tabsize=2,
  frame=tb,
  framerule=0pt,
  aboveskip=10pt,
  belowskip=10pt,
  xleftmargin=10pt,
  xrightmargin=10pt,
  framexleftmargin=10pt,
  framexrightmargin=10pt,
  framesep=0pt,
  rulecolor=\color{vscodelightgray},
  backgroundcolor=\color{vscodebackground},
}

%------------------------------------------------------------------------

% Comandos definidos
\newcommand{\bb}[1]{\mathbb{#1}}
\newcommand{\cc}[1]{\mathcal{#1}}

% I prefer the slanted \leq
\let\oldleq\leq % save them in case they're every wanted
\let\oldgeq\geq
\renewcommand{\leq}{\leqslant}
\renewcommand{\geq}{\geqslant}

% Si y solo si
\newcommand{\sii}{\iff}

% Letras griegas
\newcommand{\eps}{\epsilon}
\newcommand{\veps}{\varepsilon}
\newcommand{\lm}{\lambda}

\newcommand{\ol}{\overline}
\newcommand{\ul}{\underline}
\newcommand{\wt}{\widetilde}
\newcommand{\wh}{\widehat}

\let\oldvec\vec
\renewcommand{\vec}{\overrightarrow}

% Derivadas parciales
\newcommand{\del}[2]{\frac{\partial #1}{\partial #2}}
\newcommand{\Del}[3]{\frac{\partial^{#1} #2}{\partial #3^{#1}}}
\newcommand{\deld}[2]{\dfrac{\partial #1}{\partial #2}}
\newcommand{\Deld}[3]{\dfrac{\partial^{#1} #2}{\partial #3^{#1}}}


\newcommand{\AstIg}{\stackrel{(\ast)}{=}}
\newcommand{\Hop}{\stackrel{L'H\hat{o}pital}{=}}

\newcommand{\red}[1]{{\color{red}#1}} % Para integrales, destacar los cambios.

% Método de integración
\newcommand{\MetInt}[2]{
    \left[\begin{array}{c}
        #1 \\ #2
    \end{array}\right]
}

% Declarar aplicaciones
% 1. Nombre aplicación
% 2. Dominio
% 3. Codominio
% 4. Variable
% 5. Imagen de la variable
\newcommand{\Func}[5]{
    \begin{equation*}
        \begin{array}{rrll}
            #1:& #2 & \longrightarrow & #3\\
               & #4 & \longmapsto & #5
        \end{array}
    \end{equation*}
}

%------------------------------------------------------------------------

\usepackage{extarrows} 
\usepackage{stackrel}

\begin{document}
    % 1. Foto de fondo
    % 2. Título
    % 3. Encabezado Izquierdo
    % 4. Color de fondo
    % 5. Coord x del titulo
    % 6. Coord y del titulo
    % 7. Fecha

    
    % 1. Foto de fondo
% 2. Título
% 3. Encabezado Izquierdo
% 4. Color de fondo
% 5. Coord x del titulo
% 6. Coord y del titulo
% 7. Fecha

\newcommand{\portada}[7]{

    \portadaBase{#1}{#2}{#3}{#4}{#5}{#6}{#7}
    \portadaBook{#1}{#2}{#3}{#4}{#5}{#6}{#7}
}

\newcommand{\portadaExamen}[7]{

    \portadaBase{#1}{#2}{#3}{#4}{#5}{#6}{#7}
    \portadaArticle{#1}{#2}{#3}{#4}{#5}{#6}{#7}
}




\newcommand{\portadaBase}[7]{

    % Tiene la portada principal y la licencia Creative Commons
    
    % 1. Foto de fondo
    % 2. Título
    % 3. Encabezado Izquierdo
    % 4. Color de fondo
    % 5. Coord x del titulo
    % 6. Coord y del titulo
    % 7. Fecha
    
    
    \thispagestyle{empty}               % Sin encabezado ni pie de página
    \newgeometry{margin=0cm}        % Márgenes nulos para la primera página
    
    
    % Encabezado
    \fancyhead[L]{\helv #3}
    \fancyhead[R]{\helv \nouppercase{\leftmark}}
    
    
    \pagecolor{#4}        % Color de fondo para la portada
    
    \begin{figure}[p]
        \centering
        \transparent{0.3}           % Opacidad del 30% para la imagen
        
        \includegraphics[width=\paperwidth, keepaspectratio]{assets/#1}
    
        \begin{tikzpicture}[remember picture, overlay]
            \node[anchor=north west, text=white, opacity=1, font=\fontsize{60}{90}\selectfont\bfseries\sffamily, align=left] at (#5, #6) {#2};
            
            \node[anchor=south east, text=white, opacity=1, font=\fontsize{12}{18}\selectfont\sffamily, align=right] at (9.7, 3) {\textbf{\href{https://losdeldgiim.github.io/}{Los Del DGIIM}}};
            
            \node[anchor=south east, text=white, opacity=1, font=\fontsize{12}{15}\selectfont\sffamily, align=right] at (9.7, 1.8) {Doble Grado en Ingeniería Informática y Matemáticas\\Universidad de Granada};
        \end{tikzpicture}
    \end{figure}
    
    
    \restoregeometry        % Restaurar márgenes normales para las páginas subsiguientes
    \pagecolor{white}       % Restaurar el color de página
    
    
    \newpage
    \thispagestyle{empty}               % Sin encabezado ni pie de página
    \begin{tikzpicture}[remember picture, overlay]
        \node[anchor=south west, inner sep=3cm] at (current page.south west) {
            \begin{minipage}{0.5\paperwidth}
                \href{https://creativecommons.org/licenses/by-nc-nd/4.0/}{
                    \includegraphics[height=2cm]{assets/Licencia.png}
                }\vspace{1cm}\\
                Esta obra está bajo una
                \href{https://creativecommons.org/licenses/by-nc-nd/4.0/}{
                    Licencia Creative Commons Atribución-NoComercial-SinDerivadas 4.0 Internacional (CC BY-NC-ND 4.0).
                }\\
    
                Eres libre de compartir y redistribuir el contenido de esta obra en cualquier medio o formato, siempre y cuando des el crédito adecuado a los autores originales y no persigas fines comerciales. 
            \end{minipage}
        };
    \end{tikzpicture}
    
    
    
    % 1. Foto de fondo
    % 2. Título
    % 3. Encabezado Izquierdo
    % 4. Color de fondo
    % 5. Coord x del titulo
    % 6. Coord y del titulo
    % 7. Fecha


}


\newcommand{\portadaBook}[7]{

    % 1. Foto de fondo
    % 2. Título
    % 3. Encabezado Izquierdo
    % 4. Color de fondo
    % 5. Coord x del titulo
    % 6. Coord y del titulo
    % 7. Fecha

    % Personaliza el formato del título
    \pretitle{\begin{center}\bfseries\fontsize{42}{56}\selectfont}
    \posttitle{\par\end{center}\vspace{2em}}
    
    % Personaliza el formato del autor
    \preauthor{\begin{center}\Large}
    \postauthor{\par\end{center}\vfill}
    
    % Personaliza el formato de la fecha
    \predate{\begin{center}\huge}
    \postdate{\par\end{center}\vspace{2em}}
    
    \title{#2}
    \author{\href{https://losdeldgiim.github.io/}{Los Del DGIIM}}
    \date{Granada, #7}
    \maketitle
    
    \tableofcontents
}




\newcommand{\portadaArticle}[7]{

    % 1. Foto de fondo
    % 2. Título
    % 3. Encabezado Izquierdo
    % 4. Color de fondo
    % 5. Coord x del titulo
    % 6. Coord y del titulo
    % 7. Fecha

    % Personaliza el formato del título
    \pretitle{\begin{center}\bfseries\fontsize{42}{56}\selectfont}
    \posttitle{\par\end{center}\vspace{2em}}
    
    % Personaliza el formato del autor
    \preauthor{\begin{center}\Large}
    \postauthor{\par\end{center}\vspace{3em}}
    
    % Personaliza el formato de la fecha
    \predate{\begin{center}\huge}
    \postdate{\par\end{center}\vspace{5em}}
    
    \title{#2}
    \author{\href{https://losdeldgiim.github.io/}{Los Del DGIIM}}
    \date{Granada, #7}
    \thispagestyle{empty}               % Sin encabezado ni pie de página
    \maketitle
    \vfill
}
    \portadaExamen{ffccA4.jpg}{Cálculo I\\Examen VI}{Cálculo I. Examen VI}{MidnightBlue}{-8}{28}{2023-2024}{Jesús Muñoz Velasco}

    \begin{description}
        \item[Asignatura] Cálculo I.
        \item[Curso Académico] 2022-23.
        \item[Grado] Doble Grado en Ingeniería Informática y Matemáticas.
        \item[Grupo] Único.
        \item[Profesor] Jose Luis Gámez Ruiz.
        \item[Descripción] Examen de evaluación continua.
        \item[Fecha] 16 de noviembre de 2022
        %\item[Duración] 3 horas.
    
    \end{description}
    \newpage

    \begin{ejercicio}[2 puntos]
        Escribe los siguientes enunciados:
        \begin{enumerate}
            \item Axioma del supremo.\\
            ``Todo conjunto de números reales no vacío y mayorado tiene supremo''
            \item Teorema de Bolzano-Weierstrass.\\
            ``Toda sucesión acotada de números reales admite una parcial convergente''
            \item Definición de sucesión de Cauchy.\\
            Sea $\{x_n\}$ una sucesión de números reales. Se dice que $\{x_n\}$ es ``de Cauchy'' si:
            \[
                \forall \varepsilon>0 \ \ \exists m \in \bb{N} : 
                \begin{array}{c}
                    \text{si }p,q \in \bb{N}\\
                    p,q \geq m
                \end{array}
                \text{, se tiene } |x_p - x_q|<\varepsilon
            \]
            \item Criterio de Stolz. $\left(\text{para sucesiones del tipo} \left\{ \frac{a_n}{b_n}\right\}\right)$\\
            Sean $\{a_n\},\ \{b_n\}$ sucesiones de números reales tales que $\{b_n\}\nearrow \nearrow + \infty $ (estrictamente creciente y divergente). Entonces,\\
            \[
                \text{Si } \left\{ \dfrac{a_{n+1}-a_n}{b_{n+1}-b_n}\right\}\longrightarrow L \Longrightarrow \left\{ \dfrac{a_n}{b_n}\right\}\longrightarrow L \ \ \ (L \in \bb{R}\  \text{ ó } \pm \infty)
            \]
 
            
            
        \end{enumerate}
    \end{ejercicio}

    \begin{ejercicio}[2 puntos]
        Prueba que la suma de los cubos de tres naturales consecutivos es siempre múltiplo de 9.\\

        Esto es equivalente a probar que 
        \[
            \forall n \in \bb{N} \ \ \exists k_n \in \bb{N} : n^3 + (n+1)^3 + (n+2)^3 = 9k_n
        \]

        Lo haremos por inducción:
        \begin{itemize}[label=$\ast$]
            \item Caso $n=1$\\
            \[
                n^3 + (n+1)^3 + (n+2)^3 = 1 + 8 + 27=36 = 9 \cdot 4 \ \ \ \text{Sí} \ \ (k_1=4)
            \]

            \item Supuesto cierto para $n$ (hipótesis de inducción) $n^3 + (n+1)^3 + (n+2)^3 = 9k_n\ \ (k_n \in \bb{N})$\\
            ¿Será ciero para $n+1$? Veamos...\\

            \begin{gather*}
                (n+1)^3+(n+2)^3+(n+3)^3 = (n+1)^3+(n+2)^3+n^3 + 9 n^2 + 27n + 27 =\\
                = n^3 + (n+1) ^3 + (n+3)^3 + 9(n^2 + 3n + 3) \stackbin[(1)]{}{=} 9 k_n + 9(n^2 + 3n + 3) =\\
                = 9\underbrace{(k_n + n^2 + 3n + 3)}_{k_{n+1}} \Longrightarrow \text{ Sí}
            \end{gather*}
        \end{itemize}

        Luego queda demostrado por inducción.
        
    \end{ejercicio}
    
    \begin{ejercicio}[2 puntos]
        Sea el número real $a\leq 1$ (fijo). Estudia la convergencia de la sucesión definida por recurrencia como: $x_1 = a, \, x_{n+1} = 1- \sqrt{1-x_n}\,\,\, \forall n \in \bb{N}$.\\

        \textbf{Indicación:} distingue casos, según sea $a=1, \, 0<a<1,\, a=0, \, a<0$.\\

        \begin{enumerate}[label=Caso\ \arabic*)]
            \item $a=1$ $\Longrightarrow x_n=1\ \ \forall n \in \bb{N}$ $\ \ \{x_n\}=\{1\}\longrightarrow 1$\\

            \item $0<a<1$. Probemos que $\{x_n\}$ $\underbrace{\text{decreciente}}_{(2)}$ y $\underbrace{\text{minorada}}_{(1)}$ (por 0).
            \begin{enumerate}[label=(\arabic*)]
                \item Por inducción.
                \begin{itemize}[label=$\ast$]
                    \item $n=1\ \ x_1=a>0$
                    \item Suponiendo $x_n >0$ (hipótesis de inducción) ¿$\Rightarrow x_{n+1}>0$?
                    \begin{gather*}
                        x_{n+1}>0 \Longleftrightarrow 1- \sqrt{1-x_n}>0 \Longleftrightarrow \sqrt{1-x_n}<1 \Longleftrightarrow \\
                        \Longleftrightarrow 1-x_n<1 \Longleftrightarrow x_n > 0 \ \ \text{Sí}
                    \end{gather*}
                \end{itemize}
                \item $x_{n+1}<x_n \Longleftrightarrow 1- \sqrt{1-x_n}<x_n \Longleftrightarrow 1-x_n < \sqrt{1-x_n} \Longleftrightarrow\\
                    \Longleftrightarrow (1-x_n)^2<1-x_n \Longleftrightarrow 1-x_n<1 \Longleftrightarrow x_n > 0 \ \ \text{Sí}$
            \end{enumerate} 
            Por tanto, $\{x_n\}$ converge (a un límite ``L''). Tomando límites en la fórmula de recurrencia,
                    \begin{gather*}
                    \left.
                        \begin{array}{c l}
                           x_{n+1} & \longrightarrow L\\
                           \ \ \shortparallel &\\
                           1- \sqrt{1-x_n} &\longrightarrow 1- \sqrt{1-L}
                        \end{array} \right\} \xRightarrow{
                        \begin{array}{c}
                            \scriptsize\text{(unicidad  } \\
                            \scriptsize \text{del\ límite)} 
                        \end{array}
                        } L = 1- \sqrt{1-L} \Rightarrow ...\left\{
                        \begin{array}{c}
                            L=1\\
                            \vee\\
                            L=0
                        \end{array}
                        \right.
                    \end{gather*}
                    (El caso $L=1$ no puede darse porque $\{x_n\} \searrow \searrow\ ,\ <1$)\\
                    Luego en el caso $0<a<1$ obtenemos $\{x_n\} \searrow \searrow 0$.
                    
            \item $a=0\ \Longrightarrow x_n=0\ \ \forall n \in \bb{N}\Longrightarrow\{x_n\}=\{0\} \longrightarrow 0$.
            \item $a<0$. Probemos que $\{x_n\}$ es $\underbrace{\text{creciente}}_{(2)}$ y $\underbrace{\text{mayorada}}_{(1)}$ (por 0).
            \begin{enumerate}[label=(\arabic*)]
                \item Por inducción
                \begin{itemize}[label=$\ast$]
                    \item $n=1\ \ x_1=a<0$
                    \item Suponiendo $x_n<0$ (hipótesis de inducción) ¿$\Rightarrow x_{n+1}<0$?
                    \begin{gather*}
                        x_{n+1}<0 \Longleftrightarrow 1- \sqrt{1-x_n}<0 \Longleftrightarrow \sqrt{1-x_n}<1 \Longleftrightarrow \\
                        \Longleftrightarrow 1-x_n<1 \Longleftrightarrow x_n < 0 \ \ \text{Sí}
                    \end{gather*}
                \end{itemize}
                \item $x_{n+1}>x_n \Longleftrightarrow 1- \sqrt{1-x_n}>x_n \Longleftrightarrow 1-x_n >\sqrt{1-x_n} \Longleftrightarrow\\
                    \Longleftrightarrow (1-x_n)^2>1-x_n \Longleftrightarrow 1-x_n>1 \Longleftrightarrow x_n < 0 \ \ \text{Sí}$
            \end{enumerate}
            
            Por tanto, $\{x_n\}$ converge (a un límite ``L''). Tomando límites en la fórmula de recurrencia,
                    \begin{gather*}
                    \left.
                        \begin{array}{c l}
                           x_{n+1} & \longrightarrow L\\
                           \ \ \shortparallel &\\
                           1- \sqrt{1-x_n} &\longrightarrow 1- \sqrt{1-L}
                        \end{array} \right\} \xRightarrow{
                        \begin{array}{c}
                            \scriptsize\text{(unicidad  } \\
                            \scriptsize \text{del\ límite)} 
                        \end{array}
                        } L = 1- \sqrt{1-L} \Rightarrow ...\left\{
                        \begin{array}{c}
                            L=1\\
                            \vee\\
                            L=0
                        \end{array}
                        \right.
                    \end{gather*}
                    (El caso $L=1$ no puede darse porque $x_n < 0 \ \ \forall n \in \bb{N})$\\
                    Luego en el caso $a<0$ obtenemos $\{x_n\} \nearrow \nearrow 0$.
        \end{enumerate}
    \end{ejercicio}

    \begin{ejercicio}[2 puntos]
        Sean las sucesiones de números reales positivos $\{a_n\}$ y $\{b_n\}$, que verifican que $\{a_n\}\longrightarrow +\infty$ y que $\left\{ \frac{a_n}{b_n}\right\}\longrightarrow 2$.\\

        Estudiar la \textbf{convergencia o divergencia} de $\{b_n\}$ y de $\left\{ \dfrac{\ln(a_n)}{\ln(b_n)}\right\}$
        
        \begin{enumerate}
            \item Convergencia de $\{b_n\}$.\\
            \begin{gather*}
                \{b_n\} = \left\{ \dfrac{b_n}{a_n} \cdot a_n\right\}\stackbin{(\ast)}{\longrightarrow} +\infty
            \end{gather*}
            Donde en $(\ast)$ he aplicado que $\left\{ \dfrac{b_n}{a_n}\right\}\longrightarrow \dfrac{1}{2}>0$ y que $\{a_n\}\longrightarrow +\infty$.\\
            Por tanto, la sucesión $\{b_n\}$ diverge positivamente $(\{b_n\} \nearrow \nearrow +\infty)$

            \item Convergencia de $\left\{ \dfrac{\ln(a_n)}{\ln(b_n)}\right\}$.\\
            \begin{gather*}
                \dfrac{\ln(a_n)}{\ln(b_n)} = \dfrac{ \ln\left( b_n \cdot \frac{a_n}{b_n}\right)}{\ln(b_n)} = \dfrac{ \ln(b_n) + \ln \left(\frac{a_n}{b_n}\right)}{\ln(b_n)} = 1+ \dfrac{\ln \left(\frac{a_n}{b_n}\right)}{\ln(b_n)}\stackbin{(\ast)}{\longrightarrow} 1
            \end{gather*}

            Donde en $(\ast)$ he aplicado que $\left\{ \dfrac{a_n}{b_n}\right\}\longrightarrow \ln(2)$ y que $\{b_n\}\longrightarrow +\infty$
        \end{enumerate}

        
    \end{ejercicio}

    \begin{ejercicio}[2 puntos]
        Estudia la convergencia de la sucesión:
        \begin{gather*}
            \left\{ \left[ 1+\ln\left( \dfrac{3n^2 + 2n + 1 }{3n^2 + 5n} \right)\right]^{4n+1}\right\}
        \end{gather*}
        
        $\left.\begin{array}{l}
            \text{Defino } \{x_n\}= \left\{1+\ln \underbrace{\left( \dfrac{3n^2 + 2n + 1 }{3n^2 + 5n} \right)}_{\rightarrow 1} \right\}\longrightarrow 1 \\\\\\
            \{y_n\}= \{4n+1\}
        \end{array}\hspace{2cm}\right\}$  $\left( \begin{array}{c}
            \text{puedo aplicar el} \\
            \text{criterio de Euler}
        \end{array}\right)$\\

        \[
        x_n ^{y_n} \longrightarrow e^L \Longleftrightarrow y_n(x_n-1) \longrightarrow L
        \]
        \begin{gather*}
            y_n(x_n-1) = (4n+1) \ln\left( \dfrac{3n^2 + 2n + 1}{3n^2 + 5n}\right) = \ln\left( \dfrac{3n^2 + 2n + 1}{3n^2 + 5n}\right)^{(4n+1)}=\\
            =\ln\underbrace{(x_n)^{y_n}}_{(\ast) \rightarrow e^{-4}} \longrightarrow \ln(e^{-4}) = -4
        \end{gather*}

        Donde en $(\ast)$ he aplicado de nuevo el criterio de Euler:\\
    
        $\left.\begin{array}{l}
            \{x_n\} =\left\{ \dfrac{3n^2 + 2n + 1}{3n^2 + 5n}\right\} \longrightarrow 1 \\\\\\
            \{y_n\}= \{4n+1\}
        \end{array}\hspace{2cm}\right\}$  $\left( \begin{array}{c}
            \text{puedo aplicar el} \\
            \text{criterio de Euler}
        \end{array}\right)$\\

        \[
        x_n ^{y_n} \longrightarrow e^H \Longleftrightarrow y_n(x_n-1) \longrightarrow H
        \]

        \begin{gather*}
            y_n(x_n-1)= (4n+1) \left( \dfrac{3n^2 + 2n +1}{3n^2 + 5n} -1 \right) =\\
            = (4n+1) \left( \dfrac{\cancel{3n^2} + 2n +1 -\cancel{3n^2} - 5n}{3n^2 + 5n}\right) \longrightarrow -4
        \end{gather*}

        Luego $x_n ^{y_n} \longrightarrow e^{-4}$.\\
        \begin{flushright}
            $\square$
        \end{flushright} 

        Así, $L=-4$ y por el criterio de Euler, $x_n ^{y_n} \longrightarrow e^{-4}$.
    \end{ejercicio}


     
\end{document}
