\documentclass[12pt]{article}

% Idioma y codificación
\usepackage[spanish, es-tabla]{babel}       %es-tabla para que se titule "Tabla"
\usepackage[utf8]{inputenc}

% Márgenes
\usepackage[a4paper,top=3cm,bottom=2.5cm,left=3cm,right=3cm]{geometry}

% Comentarios de bloque
\usepackage{verbatim}

% Paquetes de links
\usepackage[hidelinks]{hyperref}    % Permite enlaces
\usepackage{url}                    % redirecciona a la web

% Más opciones para enumeraciones
\usepackage{enumitem}

% Personalizar la portada
\usepackage{titling}

% Paquetes de tablas
\usepackage{multirow}


%------------------------------------------------------------------------

%Paquetes de figuras
\usepackage{caption}
\usepackage{subcaption} % Figuras al lado de otras
\usepackage{float}      % Poner figuras en el sitio indicado H.


% Paquetes de imágenes
\usepackage{graphicx}       % Paquete para añadir imágenes
\usepackage{transparent}    % Para manejar la opacidad de las figuras

% Paquete para usar colores
\usepackage[dvipsnames]{xcolor}
\usepackage{pagecolor}      % Para cambiar el color de la página

% Habilita tamaños de fuente mayores
\usepackage{fix-cm}

% Para los gráficos
\usepackage{tikz}

% Para poder situar los nodos en los grafos
\usetikzlibrary{positioning}


%------------------------------------------------------------------------

% Paquetes de matemáticas
\usepackage{mathtools, amsfonts, amssymb, mathrsfs}
\usepackage[makeroom]{cancel}     % Simplificar tachando
\usepackage{polynom}    % Divisiones y Ruffini
\usepackage{units} % Para poner fracciones diagonales con \nicefrac

\usepackage{pgfplots}   %Representar funciones
\pgfplotsset{compat=1.18}  % Versión 1.18

\usepackage{tikz-cd}    % Para usar diagramas de composiciones
\usetikzlibrary{calc}   % Para usar cálculo de coordenadas en tikz

%Definición de teoremas, etc.
\usepackage{amsthm}
%\swapnumbers   % Intercambia la posición del texto y de la numeración

\theoremstyle{plain}

\makeatletter
\@ifclassloaded{article}{
  \newtheorem{teo}{Teorema}[section]
}{
  \newtheorem{teo}{Teorema}[chapter]  % Se resetea en cada chapter
}
\makeatother

\newtheorem{coro}{Corolario}[teo]           % Se resetea en cada teorema
\newtheorem{prop}[teo]{Proposición}         % Usa el mismo contador que teorema
\newtheorem{lema}[teo]{Lema}                % Usa el mismo contador que teorema

\theoremstyle{remark}
\newtheorem*{observacion}{Observación}

\theoremstyle{definition}

\makeatletter
\@ifclassloaded{article}{
  \newtheorem{definicion}{Definición} [section]     % Se resetea en cada chapter
}{
  \newtheorem{definicion}{Definición} [chapter]     % Se resetea en cada chapter
}
\makeatother

\newtheorem*{notacion}{Notación}
\newtheorem*{ejemplo}{Ejemplo}
\newtheorem*{ejercicio*}{Ejercicio}             % No numerado
\newtheorem{ejercicio}{Ejercicio} [section]     % Se resetea en cada section


% Modificar el formato de la numeración del teorema "ejercicio"
\renewcommand{\theejercicio}{%
  \ifnum\value{section}=0 % Si no se ha iniciado ninguna sección
    \arabic{ejercicio}% Solo mostrar el número de ejercicio
  \else
    \thesection.\arabic{ejercicio}% Mostrar número de sección y número de ejercicio
  \fi
}


% \renewcommand\qedsymbol{$\blacksquare$}         % Cambiar símbolo QED
%------------------------------------------------------------------------

% Paquetes para encabezados
\usepackage{fancyhdr}
\pagestyle{fancy}
\fancyhf{}

\newcommand{\helv}{ % Modificación tamaño de letra
\fontfamily{}\fontsize{12}{12}\selectfont}
\setlength{\headheight}{15pt} % Amplía el tamaño del índice


%\usepackage{lastpage}   % Referenciar última pag   \pageref{LastPage}
\fancyfoot[C]{\thepage}

%------------------------------------------------------------------------

% Conseguir que no ponga "Capítulo 1". Sino solo "1."
\makeatletter
\@ifclassloaded{book}{
  \renewcommand{\chaptermark}[1]{\markboth{\thechapter.\ #1}{}} % En el encabezado
    
  \renewcommand{\@makechapterhead}[1]{%
  \vspace*{50\p@}%
  {\parindent \z@ \raggedright \normalfont
    \ifnum \c@secnumdepth >\m@ne
      \huge\bfseries \thechapter.\hspace{1em}\ignorespaces
    \fi
    \interlinepenalty\@M
    \Huge \bfseries #1\par\nobreak
    \vskip 40\p@
  }}
}
\makeatother

%------------------------------------------------------------------------
% Paquetes de cógido
\usepackage{minted}
\renewcommand\listingscaption{Código fuente}

\usepackage{fancyvrb}
% Personaliza el tamaño de los números de línea
\renewcommand{\theFancyVerbLine}{\small\arabic{FancyVerbLine}}

% Estilo para C++
\newminted{cpp}{
    frame=lines,
    framesep=2mm,
    baselinestretch=1.2,
    linenos,
    escapeinside=||
}

% para minted
\definecolor{LightGray}{rgb}{0.95,0.95,0.92}
\setminted{
    linenos=true,
    stepnumber=5,
    numberfirstline=true,
    autogobble,
    breaklines=true,
    breakautoindent=true,
    breaksymbolleft=,
    breaksymbolright=,
    breaksymbolindentleft=0pt,
    breaksymbolindentright=0pt,
    breaksymbolsepleft=0pt,
    breaksymbolsepright=0pt,
    fontsize=\footnotesize,
    bgcolor=LightGray,
    numbersep=10pt
}


\usepackage{listings} % Para incluir código desde un archivo

\renewcommand\lstlistingname{Código Fuente}
\renewcommand\lstlistlistingname{Índice de Códigos Fuente}

% Definir colores
\definecolor{vscodepurple}{rgb}{0.5,0,0.5}
\definecolor{vscodeblue}{rgb}{0,0,0.8}
\definecolor{vscodegreen}{rgb}{0,0.5,0}
\definecolor{vscodegray}{rgb}{0.5,0.5,0.5}
\definecolor{vscodebackground}{rgb}{0.97,0.97,0.97}
\definecolor{vscodelightgray}{rgb}{0.9,0.9,0.9}

% Configuración para el estilo de C similar a VSCode
\lstdefinestyle{vscode_C}{
  backgroundcolor=\color{vscodebackground},
  commentstyle=\color{vscodegreen},
  keywordstyle=\color{vscodeblue},
  numberstyle=\tiny\color{vscodegray},
  stringstyle=\color{vscodepurple},
  basicstyle=\scriptsize\ttfamily,
  breakatwhitespace=false,
  breaklines=true,
  captionpos=b,
  keepspaces=true,
  numbers=left,
  numbersep=5pt,
  showspaces=false,
  showstringspaces=false,
  showtabs=false,
  tabsize=2,
  frame=tb,
  framerule=0pt,
  aboveskip=10pt,
  belowskip=10pt,
  xleftmargin=10pt,
  xrightmargin=10pt,
  framexleftmargin=10pt,
  framexrightmargin=10pt,
  framesep=0pt,
  rulecolor=\color{vscodelightgray},
  backgroundcolor=\color{vscodebackground},
}

%------------------------------------------------------------------------

% Comandos definidos
\newcommand{\bb}[1]{\mathbb{#1}}
\newcommand{\cc}[1]{\mathcal{#1}}

% I prefer the slanted \leq
\let\oldleq\leq % save them in case they're every wanted
\let\oldgeq\geq
\renewcommand{\leq}{\leqslant}
\renewcommand{\geq}{\geqslant}

% Si y solo si
\newcommand{\sii}{\iff}

% Letras griegas
\newcommand{\eps}{\epsilon}
\newcommand{\veps}{\varepsilon}
\newcommand{\lm}{\lambda}

\newcommand{\ol}{\overline}
\newcommand{\ul}{\underline}
\newcommand{\wt}{\widetilde}
\newcommand{\wh}{\widehat}

\let\oldvec\vec
\renewcommand{\vec}{\overrightarrow}

% Derivadas parciales
\newcommand{\del}[2]{\frac{\partial #1}{\partial #2}}
\newcommand{\Del}[3]{\frac{\partial^{#1} #2}{\partial #3^{#1}}}
\newcommand{\deld}[2]{\dfrac{\partial #1}{\partial #2}}
\newcommand{\Deld}[3]{\dfrac{\partial^{#1} #2}{\partial #3^{#1}}}


\newcommand{\AstIg}{\stackrel{(\ast)}{=}}
\newcommand{\Hop}{\stackrel{L'H\hat{o}pital}{=}}

\newcommand{\red}[1]{{\color{red}#1}} % Para integrales, destacar los cambios.

% Método de integración
\newcommand{\MetInt}[2]{
    \left[\begin{array}{c}
        #1 \\ #2
    \end{array}\right]
}

% Declarar aplicaciones
% 1. Nombre aplicación
% 2. Dominio
% 3. Codominio
% 4. Variable
% 5. Imagen de la variable
\newcommand{\Func}[5]{
    \begin{equation*}
        \begin{array}{rrll}
            #1:& #2 & \longrightarrow & #3\\
               & #4 & \longmapsto & #5
        \end{array}
    \end{equation*}
}

%------------------------------------------------------------------------

\usepackage{extarrows} 
\usepackage{stackrel}

\begin{document}
    % 1. Foto de fondo
    % 2. Título
    % 3. Encabezado Izquierdo
    % 4. Color de fondo
    % 5. Coord x del titulo
    % 6. Coord y del titulo
    % 7. Fecha

    
    % 1. Foto de fondo
% 2. Título
% 3. Encabezado Izquierdo
% 4. Color de fondo
% 5. Coord x del titulo
% 6. Coord y del titulo
% 7. Fecha

\newcommand{\portada}[7]{

    \portadaBase{#1}{#2}{#3}{#4}{#5}{#6}{#7}
    \portadaBook{#1}{#2}{#3}{#4}{#5}{#6}{#7}
}

\newcommand{\portadaExamen}[7]{

    \portadaBase{#1}{#2}{#3}{#4}{#5}{#6}{#7}
    \portadaArticle{#1}{#2}{#3}{#4}{#5}{#6}{#7}
}




\newcommand{\portadaBase}[7]{

    % Tiene la portada principal y la licencia Creative Commons
    
    % 1. Foto de fondo
    % 2. Título
    % 3. Encabezado Izquierdo
    % 4. Color de fondo
    % 5. Coord x del titulo
    % 6. Coord y del titulo
    % 7. Fecha
    
    
    \thispagestyle{empty}               % Sin encabezado ni pie de página
    \newgeometry{margin=0cm}        % Márgenes nulos para la primera página
    
    
    % Encabezado
    \fancyhead[L]{\helv #3}
    \fancyhead[R]{\helv \nouppercase{\leftmark}}
    
    
    \pagecolor{#4}        % Color de fondo para la portada
    
    \begin{figure}[p]
        \centering
        \transparent{0.3}           % Opacidad del 30% para la imagen
        
        \includegraphics[width=\paperwidth, keepaspectratio]{assets/#1}
    
        \begin{tikzpicture}[remember picture, overlay]
            \node[anchor=north west, text=white, opacity=1, font=\fontsize{60}{90}\selectfont\bfseries\sffamily, align=left] at (#5, #6) {#2};
            
            \node[anchor=south east, text=white, opacity=1, font=\fontsize{12}{18}\selectfont\sffamily, align=right] at (9.7, 3) {\textbf{\href{https://losdeldgiim.github.io/}{Los Del DGIIM}}};
            
            \node[anchor=south east, text=white, opacity=1, font=\fontsize{12}{15}\selectfont\sffamily, align=right] at (9.7, 1.8) {Doble Grado en Ingeniería Informática y Matemáticas\\Universidad de Granada};
        \end{tikzpicture}
    \end{figure}
    
    
    \restoregeometry        % Restaurar márgenes normales para las páginas subsiguientes
    \pagecolor{white}       % Restaurar el color de página
    
    
    \newpage
    \thispagestyle{empty}               % Sin encabezado ni pie de página
    \begin{tikzpicture}[remember picture, overlay]
        \node[anchor=south west, inner sep=3cm] at (current page.south west) {
            \begin{minipage}{0.5\paperwidth}
                \href{https://creativecommons.org/licenses/by-nc-nd/4.0/}{
                    \includegraphics[height=2cm]{assets/Licencia.png}
                }\vspace{1cm}\\
                Esta obra está bajo una
                \href{https://creativecommons.org/licenses/by-nc-nd/4.0/}{
                    Licencia Creative Commons Atribución-NoComercial-SinDerivadas 4.0 Internacional (CC BY-NC-ND 4.0).
                }\\
    
                Eres libre de compartir y redistribuir el contenido de esta obra en cualquier medio o formato, siempre y cuando des el crédito adecuado a los autores originales y no persigas fines comerciales. 
            \end{minipage}
        };
    \end{tikzpicture}
    
    
    
    % 1. Foto de fondo
    % 2. Título
    % 3. Encabezado Izquierdo
    % 4. Color de fondo
    % 5. Coord x del titulo
    % 6. Coord y del titulo
    % 7. Fecha


}


\newcommand{\portadaBook}[7]{

    % 1. Foto de fondo
    % 2. Título
    % 3. Encabezado Izquierdo
    % 4. Color de fondo
    % 5. Coord x del titulo
    % 6. Coord y del titulo
    % 7. Fecha

    % Personaliza el formato del título
    \pretitle{\begin{center}\bfseries\fontsize{42}{56}\selectfont}
    \posttitle{\par\end{center}\vspace{2em}}
    
    % Personaliza el formato del autor
    \preauthor{\begin{center}\Large}
    \postauthor{\par\end{center}\vfill}
    
    % Personaliza el formato de la fecha
    \predate{\begin{center}\huge}
    \postdate{\par\end{center}\vspace{2em}}
    
    \title{#2}
    \author{\href{https://losdeldgiim.github.io/}{Los Del DGIIM}}
    \date{Granada, #7}
    \maketitle
    
    \tableofcontents
}




\newcommand{\portadaArticle}[7]{

    % 1. Foto de fondo
    % 2. Título
    % 3. Encabezado Izquierdo
    % 4. Color de fondo
    % 5. Coord x del titulo
    % 6. Coord y del titulo
    % 7. Fecha

    % Personaliza el formato del título
    \pretitle{\begin{center}\bfseries\fontsize{42}{56}\selectfont}
    \posttitle{\par\end{center}\vspace{2em}}
    
    % Personaliza el formato del autor
    \preauthor{\begin{center}\Large}
    \postauthor{\par\end{center}\vspace{3em}}
    
    % Personaliza el formato de la fecha
    \predate{\begin{center}\huge}
    \postdate{\par\end{center}\vspace{5em}}
    
    \title{#2}
    \author{\href{https://losdeldgiim.github.io/}{Los Del DGIIM}}
    \date{Granada, #7}
    \thispagestyle{empty}               % Sin encabezado ni pie de página
    \maketitle
    \vfill
}
    \portadaExamen{ffccA4.jpg}{Cálculo I\\Examen VII}{Cálculo I. Examen VII}{MidnightBlue}{-8}{28}{2021-2022}{Jesús Muñoz Velasco\\Arturo Olivares Martos}

    \begin{description}
        \item[Asignatura] Cálculo I.
        \item[Curso Académico] 2021-22.
        \item[Grado] Doble Grado en Ingeniería Informática y Matemáticas.
        \item[Grupo] Único.
        \item[Profesor] Jose Luis Gámez Ruiz.
        \item[Descripción] Examen de evaluación continua.
        \item[Fecha] 23 de noviembre de 2021
        %\item[Duración] 3 horas.
    
    \end{description}
    \newpage

    \begin{ejercicio}[2 puntos]
        \textbf{Enuncia} el Teorema de Bolzano-Weierstrass y el Teorema de Complitud de $\bb{R}$.\\

        \begin{itemize}
            \item \underline{Teorema de Bolzano-Weierstrass:}\\
            
            Toda sucesión (de números reales) \underline{acotada} admite una \underline{parcial convergente}
            \begin{gather*}
            \left(
                \begin{array}{l}
                    \{x_n\} \text{ acotada} \Longrightarrow
                \end{array}
                \begin{array}{c}
                    \exists \ \sigma: \bb{N} \longrightarrow \bb{N} \text{ estrictamente creciente}\\
                    \text{ tal que } \{x_{\sigma(n)} \} \text{ converge}
                \end{array}
            \right)
            \end{gather*}

            \item \underline{Teorema de Complitud de $\bb{R}$:}\\
            
            Sea $\{x_n\}$ una sucesión de números reales, entonces: 
            \[
                \{x_n\} \text{ convergente} \Longleftrightarrow \{x_n\} \text{ de Cauchy}
            \]
        \end{itemize}
    \end{ejercicio}

    \begin{ejercicio}[2 puntos]
        \textbf{Justifica} si las siguientes afirmaciones son verdaderas o falsas:
        \begin{enumerate}
            \item Toda sucesión monótona y mayorada es convergente.\\
            \textbf{Falso}. Contraejemplo:\\
            La sucesión $\{-n\}$ es monótona (decreciente), mayorada (por $0$) y \underline{no} converge.
            \item Si un $A \subseteq \bb{R}$ es no vacío y mayorado, existe al menos un mayorante positivo.
            Sea $M(A)$ el conjunto de los mayorantes de $A$ y sea $k\in M(A)$. Entonces
            \begin{gather*}
                \left.
                \begin{array}{l}
                     k'=\max\{k,5\}\in M(A)\\\\
                     k'\geq 5 \Rightarrow k' \text{ positivo}
                \end{array}
                \right\} \Longrightarrow \textbf{Verdadero}
            \end{gather*}
            \item Si un $A \subseteq \bb{R}$ es no vacío y minorado, existe al menos un minorante positivo.
            \textbf{Falso}. Contraejemplo:
            \begin{gather*}
                A=\{-2,8\},\ \ \ m(A)=]-\infty , -2] \Longrightarrow m(A) \cap \bb{R}^+ = \emptyset\\
                \text{(ningún minorante es positivo)}
            \end{gather*}
            \item Toda sucesión que admita una parcial de Cauchy, es acotada.\\
            \textbf{Falso.} Contraejemplo:
            \begin{gather*}
                \begin{array}{ccc}
                    \{x_n\} \ \ \text{ tal que }&
                    \begin{array}{c}
                         x_{2n}=0\\
                         x_{2n-1}=n
                    \end{array}& \forall n \in \bb{N}\\
                \end{array}
            \end{gather*}
            Admite una parcial de Cauchy, $\{x_{2n}\}$. Sin embargo, es \underline{no acotada}.
            \item Si $\{x_n\}$ es una sucesión no acotada, admite una parcial divergente.\\
            \textbf{Verdadero.} Demostración:\\
            (Dado $k \in \bb{R}^+,\ \{p \in \bb{N} : |x_p|>k\}$ es infinito)\\

            Sea $\sigma(1) = \min\{p\in \bb{N}: |x_p|>1\}$. Supuesto conocido $\sigma(n)$, ¿$\sigma(n+1)$?\\
            \begin{gather*}
                \sigma(n+1) = \min\left\{ p \in \bb{N} : \begin{array}{c}
                     p>\sigma(n)\\
                     |x_p|>n+1
                \end{array}\right\}
            \end{gather*}
            Se tiene:
            \begin{gather*}
                \sigma(n+1) > \sigma (n)\ \  \text{ ($\sigma$ es estrictamente creciente)}\\
                |x_{\sigma(n)}| > n \Longrightarrow \{|x_{\sigma(n)}|\}\longrightarrow +\infty
            \end{gather*}
        \end{enumerate}
    \end{ejercicio}

    \begin{ejercicio}[2 puntos]
        Sean $a$ y $b$ dos números reales distintos. Demuestra que:
        \begin{gather*}
            \sum_{k=0}^n a^{n-k}b^k = \dfrac{a^{n+1}-b^{n+1}}{a-b}, \ \forall n \in \bb{N}.
        \end{gather*}

        Por inducción:\\
        
        Sea $A=\left\{n \in \bb{N} : \sum\limits_{k=0}^n a^{n-k}b^k = \dfrac{a^{n+1}-b^{n+1}}{a-b}\right\}\subseteq \bb{N}$\\
        
        ¿Es $A$ inductivo? ($1 \in A$ y Si $n\in A \Longrightarrow n+1 \in A)$

        \begin{itemize}[label=$\ast$]
            \item ¿$1\in A$?\ $1 \in \bb{N},\ \sum\limits_{k=0}^1 a^{1-k}b^k = a + b =\dfrac{a^2 - b^2}{a-b}$\ \ Sí.
            \item Si $n\in A$, \ \ $\sum\limits_{k=0}^n a^{n-k}b^k=\dfrac{a^{n+1}-b^{n+1}}{a-b}$\ \ (hipótesis de inducción)
            \[
                \text{¿}n+1\in A\text{?} \Longleftrightarrow \text{¿}\sum\limits_{k=0}^{n+1} a^{n+1-k}b^k = \dfrac{a^{n+2}-b^{n+2}}{a-b}\text{?}
            \]
            Veamos:
            \begin{gather*}
                \sum\limits_{k=0}^{n+1} a^{n+1-k}b^k = \sum\limits_{k=0}^{n} a^{n+1-k}b^k + b^{n+1} =
                 a\sum\limits_{k=0}^{n} a^{n-k}b^k + b^{n+1} \stackbin{(\ast)}{=} a \dfrac{a^{n+1}-b^{n+1}}{a-b} + b^{n+1} =\\\\
                 =\dfrac{a^{n+2}-\cancel{ab^{n+1}}+\cancel{ab^{n+1}}-b^{n+2}}{a-b} = \dfrac{a^{n+2}-b^{n+2}}{a-b}\ \ \ \text{ Sí}
            \end{gather*}

            Donde en $(\ast)$ he aplicado la hipótesis de inducción.
        \end{itemize}

        Luego $A$ es inductivo (por el principio de inducción). Por tanto $A=\bb{N}$, es decir, 
        \[
            \sum\limits_{k=0}^n a^{n-k}b^k = \dfrac{a^{n+1}-b^{n+1}}{a-b}\ \forall n \in \bb{N}.
        \]
        
    \end{ejercicio}

    \begin{ejercicio}[4 puntos]
        Estudia la convergencia de las sucesiones:
        \begin{enumerate}
            \item $\left\{ \dfrac{n 3^n (\sqrt{n+1}-\sqrt{n})}{3^n \sqrt{n+1}+2^n} \right\}$
            
            \begin{gather*}
                \left\{ \dfrac{n 3^n (\sqrt{n+1}-\sqrt{n})}{3^n \sqrt{n+1}+2^n} \right\} = 
                \left\{ \dfrac{n \cancel{3^n} (\sqrt{n+1}-\sqrt{n})}{\cancel{3^n} \sqrt{n+1}+\cancel{3^n}\left(\frac{2}{3}\right)^n} \right\} = 
                 \left\{ \dfrac{n(\sqrt{n+1}-\sqrt{n})}{\sqrt{n+1}+\left(\frac{2}{3}\right)^n} \right\} =\\
                \left\{ \dfrac{\cancel{\sqrt{n+1}}\left(\frac{n}{\sqrt{n+1}}\right) (\sqrt{n+1}-\sqrt{n})}{\cancel{\sqrt{n+1}}+\cancel{\sqrt{n+1}}\left(\frac{\left(\frac{2}{3}\right)^n}{\sqrt{n+1}}\right)} \right\}=
                \left\{ \dfrac{\left(\frac{n}{\sqrt{n+1}}\right) (\sqrt{n+1}-\sqrt{n})}{1+\left(\frac{\left(\frac{2}{3}\right)^n}{\sqrt{n+1}}\right)} \right\}
            \end{gather*}

            Estudiemos ahora por comodidad numerador y denominador por separado:
            \begin{itemize}[label=$\ast$]
                \item \underline{Numerador:}
                \begin{gather*}
                    \dfrac{n (\sqrt{n+1}-\sqrt{n})}{\sqrt{n+1}} = \dfrac{n (\sqrt{n+1}-\sqrt{n})(\sqrt{n+1}+\sqrt{n}) }{\sqrt{n+1}(\sqrt{n+1}+\sqrt{n})} = \dfrac{n(\cancel{n}+1-\cancel{n})}{n+1+\sqrt{n(n+1)}}=\\
                    =\dfrac{\cancel{n}}{\cancel{n}+\cancel{n}\left( \frac{1}{n}\right)+\cancel{n} \sqrt{\frac{\cancel{n}(n+1)}{n^{\cancel{2}}}}} = \dfrac{1}{1+\left( \frac{1}{n}\right) + \sqrt{\frac{n+1}{n}}} \longrightarrow \dfrac{1}{1+0+\sqrt{1}}=\dfrac{1}{2}
                \end{gather*}

                \item \underline{Denominador:}
                \begin{gather*}
                    1+\left(\frac{\left(\frac{2}{3}\right)^n}{\sqrt{n+1}} \right) \longrightarrow 1 + 0= 1
                \end{gather*}
            \end{itemize}

            Por tanto,
            \begin{gather*}
                \left\{ \dfrac{n 3^n (\sqrt{n+1}-\sqrt{n})}{3^n \sqrt{n+1}+2^n} \right\} \longrightarrow \dfrac{\left(\frac{1}{2}\right)}{1} = \dfrac{1}{2}
            \end{gather*}
            
            \item $\{x_n\}$ Definida por recurrencia: $x_1=5,\ \ x_{n+1}=\dfrac{x_n^2 + 9}{2x_n}\ \ \forall n \in \bb{N}$.\\

            Voy a demostrar que $\{x_n\}$ es decreciente y minorada por 3.\\
            
            ¿$3<x_{n+1}<x_n \ \ \forall n \in \bb{N}$?\ \ (por inducción)
            \begin{itemize}[label=$\ast$]
                \item \underline{n=1}\ \ $x_2=\frac{34}{10}=3,4\Longrightarrow \ 3<3,4<5\ \ $Sí.
                \item Supuesto $3<x_{n+1}<x_n$ (hipótesis de inducción), ¿$\Rightarrow 3 \stackbin[(1)]{}{<\,} x_{n+2} \stackbin[(2)]{}{<\,}x_{n+1}$?

                \begin{enumerate}[label=(\arabic*)]
                    \item \begin{gather*}
                        x_{n+2}=\dfrac{x_{n+1}^2 + 9}{2x_{n+1}}> 3 \Longleftrightarrow x_{n+1}^2 + 9 > 6x_{n+1} \Longleftrightarrow \\
                        \Longleftrightarrow x_{n+1}^2 -6x_{n+1} + 9 > 0 \Longleftrightarrow (x_{n+1}-3)^2 > 0\ \ \text{ Sí}
                    \end{gather*}

                    \item \begin{gather*}
                        x_{n+2}=\dfrac{x_{n+1}^2 + 9}{2x_{n+1}}< x_{n+1} \Longleftrightarrow x_{n+1}^2 + 9 < 2x_{n+1}^2 \Longleftrightarrow\\
                        \Longleftrightarrow 9 < x_{n+1}^2 \Longleftarrow 3 < x_{n+1}\ \ \text{ Sí (por hipótesis de inducción)}
                    \end{gather*}
                \end{enumerate}
            \end{itemize}

            Luego $\{x_n\}$ decreciente y minorada (por 3), luego es convergente. Por la unicidad del límite y sabiendo que $x_{n+1}=\frac{x_n^2 + 9}{2x_n}$,
            \begin{gather*}
                L=\dfrac{L^2 + 9}{2L} \Longrightarrow L^2 = 9 \Longrightarrow \begin{array}{c}
                     L=3 \\
                     \cancel{L=-3}
                \end{array}
            \end{gather*}

            Descartamos el $-3$ ya que $x_n>3\ \ \forall n \in \bb{N}$.\\
            Finalmente tenemos que la sucesión dada converge a 3.
        \end{enumerate}
        
    \end{ejercicio}


     
\end{document}
