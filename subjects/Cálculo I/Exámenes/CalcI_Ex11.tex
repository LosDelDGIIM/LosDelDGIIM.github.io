\documentclass[12pt]{article}

% Idioma y codificación
\usepackage[spanish, es-tabla]{babel}       %es-tabla para que se titule "Tabla"
\usepackage[utf8]{inputenc}

% Márgenes
\usepackage[a4paper,top=3cm,bottom=2.5cm,left=3cm,right=3cm]{geometry}

% Comentarios de bloque
\usepackage{verbatim}

% Paquetes de links
\usepackage[hidelinks]{hyperref}    % Permite enlaces
\usepackage{url}                    % redirecciona a la web

% Más opciones para enumeraciones
\usepackage{enumitem}

% Personalizar la portada
\usepackage{titling}

% Paquetes de tablas
\usepackage{multirow}


%------------------------------------------------------------------------

%Paquetes de figuras
\usepackage{caption}
\usepackage{subcaption} % Figuras al lado de otras
\usepackage{float}      % Poner figuras en el sitio indicado H.


% Paquetes de imágenes
\usepackage{graphicx}       % Paquete para añadir imágenes
\usepackage{transparent}    % Para manejar la opacidad de las figuras

% Paquete para usar colores
\usepackage[dvipsnames]{xcolor}
\usepackage{pagecolor}      % Para cambiar el color de la página

% Habilita tamaños de fuente mayores
\usepackage{fix-cm}

% Para los gráficos
\usepackage{tikz}

% Para poder situar los nodos en los grafos
\usetikzlibrary{positioning}


%------------------------------------------------------------------------

% Paquetes de matemáticas
\usepackage{mathtools, amsfonts, amssymb, mathrsfs}
\usepackage[makeroom]{cancel}     % Simplificar tachando
\usepackage{polynom}    % Divisiones y Ruffini
\usepackage{units} % Para poner fracciones diagonales con \nicefrac

\usepackage{pgfplots}   %Representar funciones
\pgfplotsset{compat=1.18}  % Versión 1.18

\usepackage{tikz-cd}    % Para usar diagramas de composiciones
\usetikzlibrary{calc}   % Para usar cálculo de coordenadas en tikz

%Definición de teoremas, etc.
\usepackage{amsthm}
%\swapnumbers   % Intercambia la posición del texto y de la numeración

\theoremstyle{plain}

\makeatletter
\@ifclassloaded{article}{
  \newtheorem{teo}{Teorema}[section]
}{
  \newtheorem{teo}{Teorema}[chapter]  % Se resetea en cada chapter
}
\makeatother

\newtheorem{coro}{Corolario}[teo]           % Se resetea en cada teorema
\newtheorem{prop}[teo]{Proposición}         % Usa el mismo contador que teorema
\newtheorem{lema}[teo]{Lema}                % Usa el mismo contador que teorema

\theoremstyle{remark}
\newtheorem*{observacion}{Observación}

\theoremstyle{definition}

\makeatletter
\@ifclassloaded{article}{
  \newtheorem{definicion}{Definición} [section]     % Se resetea en cada chapter
}{
  \newtheorem{definicion}{Definición} [chapter]     % Se resetea en cada chapter
}
\makeatother

\newtheorem*{notacion}{Notación}
\newtheorem*{ejemplo}{Ejemplo}
\newtheorem*{ejercicio*}{Ejercicio}             % No numerado
\newtheorem{ejercicio}{Ejercicio} [section]     % Se resetea en cada section


% Modificar el formato de la numeración del teorema "ejercicio"
\renewcommand{\theejercicio}{%
  \ifnum\value{section}=0 % Si no se ha iniciado ninguna sección
    \arabic{ejercicio}% Solo mostrar el número de ejercicio
  \else
    \thesection.\arabic{ejercicio}% Mostrar número de sección y número de ejercicio
  \fi
}


% \renewcommand\qedsymbol{$\blacksquare$}         % Cambiar símbolo QED
%------------------------------------------------------------------------

% Paquetes para encabezados
\usepackage{fancyhdr}
\pagestyle{fancy}
\fancyhf{}

\newcommand{\helv}{ % Modificación tamaño de letra
\fontfamily{}\fontsize{12}{12}\selectfont}
\setlength{\headheight}{15pt} % Amplía el tamaño del índice


%\usepackage{lastpage}   % Referenciar última pag   \pageref{LastPage}
\fancyfoot[C]{\thepage}

%------------------------------------------------------------------------

% Conseguir que no ponga "Capítulo 1". Sino solo "1."
\makeatletter
\@ifclassloaded{book}{
  \renewcommand{\chaptermark}[1]{\markboth{\thechapter.\ #1}{}} % En el encabezado
    
  \renewcommand{\@makechapterhead}[1]{%
  \vspace*{50\p@}%
  {\parindent \z@ \raggedright \normalfont
    \ifnum \c@secnumdepth >\m@ne
      \huge\bfseries \thechapter.\hspace{1em}\ignorespaces
    \fi
    \interlinepenalty\@M
    \Huge \bfseries #1\par\nobreak
    \vskip 40\p@
  }}
}
\makeatother

%------------------------------------------------------------------------
% Paquetes de cógido
\usepackage{minted}
\renewcommand\listingscaption{Código fuente}

\usepackage{fancyvrb}
% Personaliza el tamaño de los números de línea
\renewcommand{\theFancyVerbLine}{\small\arabic{FancyVerbLine}}

% Estilo para C++
\newminted{cpp}{
    frame=lines,
    framesep=2mm,
    baselinestretch=1.2,
    linenos,
    escapeinside=||
}

% para minted
\definecolor{LightGray}{rgb}{0.95,0.95,0.92}
\setminted{
    linenos=true,
    stepnumber=5,
    numberfirstline=true,
    autogobble,
    breaklines=true,
    breakautoindent=true,
    breaksymbolleft=,
    breaksymbolright=,
    breaksymbolindentleft=0pt,
    breaksymbolindentright=0pt,
    breaksymbolsepleft=0pt,
    breaksymbolsepright=0pt,
    fontsize=\footnotesize,
    bgcolor=LightGray,
    numbersep=10pt
}


\usepackage{listings} % Para incluir código desde un archivo

\renewcommand\lstlistingname{Código Fuente}
\renewcommand\lstlistlistingname{Índice de Códigos Fuente}

% Definir colores
\definecolor{vscodepurple}{rgb}{0.5,0,0.5}
\definecolor{vscodeblue}{rgb}{0,0,0.8}
\definecolor{vscodegreen}{rgb}{0,0.5,0}
\definecolor{vscodegray}{rgb}{0.5,0.5,0.5}
\definecolor{vscodebackground}{rgb}{0.97,0.97,0.97}
\definecolor{vscodelightgray}{rgb}{0.9,0.9,0.9}

% Configuración para el estilo de C similar a VSCode
\lstdefinestyle{vscode_C}{
  backgroundcolor=\color{vscodebackground},
  commentstyle=\color{vscodegreen},
  keywordstyle=\color{vscodeblue},
  numberstyle=\tiny\color{vscodegray},
  stringstyle=\color{vscodepurple},
  basicstyle=\scriptsize\ttfamily,
  breakatwhitespace=false,
  breaklines=true,
  captionpos=b,
  keepspaces=true,
  numbers=left,
  numbersep=5pt,
  showspaces=false,
  showstringspaces=false,
  showtabs=false,
  tabsize=2,
  frame=tb,
  framerule=0pt,
  aboveskip=10pt,
  belowskip=10pt,
  xleftmargin=10pt,
  xrightmargin=10pt,
  framexleftmargin=10pt,
  framexrightmargin=10pt,
  framesep=0pt,
  rulecolor=\color{vscodelightgray},
  backgroundcolor=\color{vscodebackground},
}

%------------------------------------------------------------------------

% Comandos definidos
\newcommand{\bb}[1]{\mathbb{#1}}
\newcommand{\cc}[1]{\mathcal{#1}}

% I prefer the slanted \leq
\let\oldleq\leq % save them in case they're every wanted
\let\oldgeq\geq
\renewcommand{\leq}{\leqslant}
\renewcommand{\geq}{\geqslant}

% Si y solo si
\newcommand{\sii}{\iff}

% Letras griegas
\newcommand{\eps}{\epsilon}
\newcommand{\veps}{\varepsilon}
\newcommand{\lm}{\lambda}

\newcommand{\ol}{\overline}
\newcommand{\ul}{\underline}
\newcommand{\wt}{\widetilde}
\newcommand{\wh}{\widehat}

\let\oldvec\vec
\renewcommand{\vec}{\overrightarrow}

% Derivadas parciales
\newcommand{\del}[2]{\frac{\partial #1}{\partial #2}}
\newcommand{\Del}[3]{\frac{\partial^{#1} #2}{\partial #3^{#1}}}
\newcommand{\deld}[2]{\dfrac{\partial #1}{\partial #2}}
\newcommand{\Deld}[3]{\dfrac{\partial^{#1} #2}{\partial #3^{#1}}}


\newcommand{\AstIg}{\stackrel{(\ast)}{=}}
\newcommand{\Hop}{\stackrel{L'H\hat{o}pital}{=}}

\newcommand{\red}[1]{{\color{red}#1}} % Para integrales, destacar los cambios.

% Método de integración
\newcommand{\MetInt}[2]{
    \left[\begin{array}{c}
        #1 \\ #2
    \end{array}\right]
}

% Declarar aplicaciones
% 1. Nombre aplicación
% 2. Dominio
% 3. Codominio
% 4. Variable
% 5. Imagen de la variable
\newcommand{\Func}[5]{
    \begin{equation*}
        \begin{array}{rrll}
            #1:& #2 & \longrightarrow & #3\\
               & #4 & \longmapsto & #5
        \end{array}
    \end{equation*}
}

%------------------------------------------------------------------------



\begin{document}
	
	% 1. Foto de fondo
	% 2. Título
	% 3. Encabezado Izquierdo
	% 4. Color de fondo
	% 5. Coord x del titulo
	% 6. Coord y del titulo
	% 7. Fecha
	
	\newcommand{\N}{{\mathbb{N}}} % Enteros
	\newcommand{\Q}{{\mathbb{Q}}} % Racionales
	\newcommand{\R}{{\mathbb{R}}} % Reales
	
	% 1. Foto de fondo
% 2. Título
% 3. Encabezado Izquierdo
% 4. Color de fondo
% 5. Coord x del titulo
% 6. Coord y del titulo
% 7. Fecha

\newcommand{\portada}[7]{

    \portadaBase{#1}{#2}{#3}{#4}{#5}{#6}{#7}
    \portadaBook{#1}{#2}{#3}{#4}{#5}{#6}{#7}
}

\newcommand{\portadaExamen}[7]{

    \portadaBase{#1}{#2}{#3}{#4}{#5}{#6}{#7}
    \portadaArticle{#1}{#2}{#3}{#4}{#5}{#6}{#7}
}




\newcommand{\portadaBase}[7]{

    % Tiene la portada principal y la licencia Creative Commons
    
    % 1. Foto de fondo
    % 2. Título
    % 3. Encabezado Izquierdo
    % 4. Color de fondo
    % 5. Coord x del titulo
    % 6. Coord y del titulo
    % 7. Fecha
    
    
    \thispagestyle{empty}               % Sin encabezado ni pie de página
    \newgeometry{margin=0cm}        % Márgenes nulos para la primera página
    
    
    % Encabezado
    \fancyhead[L]{\helv #3}
    \fancyhead[R]{\helv \nouppercase{\leftmark}}
    
    
    \pagecolor{#4}        % Color de fondo para la portada
    
    \begin{figure}[p]
        \centering
        \transparent{0.3}           % Opacidad del 30% para la imagen
        
        \includegraphics[width=\paperwidth, keepaspectratio]{assets/#1}
    
        \begin{tikzpicture}[remember picture, overlay]
            \node[anchor=north west, text=white, opacity=1, font=\fontsize{60}{90}\selectfont\bfseries\sffamily, align=left] at (#5, #6) {#2};
            
            \node[anchor=south east, text=white, opacity=1, font=\fontsize{12}{18}\selectfont\sffamily, align=right] at (9.7, 3) {\textbf{\href{https://losdeldgiim.github.io/}{Los Del DGIIM}}};
            
            \node[anchor=south east, text=white, opacity=1, font=\fontsize{12}{15}\selectfont\sffamily, align=right] at (9.7, 1.8) {Doble Grado en Ingeniería Informática y Matemáticas\\Universidad de Granada};
        \end{tikzpicture}
    \end{figure}
    
    
    \restoregeometry        % Restaurar márgenes normales para las páginas subsiguientes
    \pagecolor{white}       % Restaurar el color de página
    
    
    \newpage
    \thispagestyle{empty}               % Sin encabezado ni pie de página
    \begin{tikzpicture}[remember picture, overlay]
        \node[anchor=south west, inner sep=3cm] at (current page.south west) {
            \begin{minipage}{0.5\paperwidth}
                \href{https://creativecommons.org/licenses/by-nc-nd/4.0/}{
                    \includegraphics[height=2cm]{assets/Licencia.png}
                }\vspace{1cm}\\
                Esta obra está bajo una
                \href{https://creativecommons.org/licenses/by-nc-nd/4.0/}{
                    Licencia Creative Commons Atribución-NoComercial-SinDerivadas 4.0 Internacional (CC BY-NC-ND 4.0).
                }\\
    
                Eres libre de compartir y redistribuir el contenido de esta obra en cualquier medio o formato, siempre y cuando des el crédito adecuado a los autores originales y no persigas fines comerciales. 
            \end{minipage}
        };
    \end{tikzpicture}
    
    
    
    % 1. Foto de fondo
    % 2. Título
    % 3. Encabezado Izquierdo
    % 4. Color de fondo
    % 5. Coord x del titulo
    % 6. Coord y del titulo
    % 7. Fecha


}


\newcommand{\portadaBook}[7]{

    % 1. Foto de fondo
    % 2. Título
    % 3. Encabezado Izquierdo
    % 4. Color de fondo
    % 5. Coord x del titulo
    % 6. Coord y del titulo
    % 7. Fecha

    % Personaliza el formato del título
    \pretitle{\begin{center}\bfseries\fontsize{42}{56}\selectfont}
    \posttitle{\par\end{center}\vspace{2em}}
    
    % Personaliza el formato del autor
    \preauthor{\begin{center}\Large}
    \postauthor{\par\end{center}\vfill}
    
    % Personaliza el formato de la fecha
    \predate{\begin{center}\huge}
    \postdate{\par\end{center}\vspace{2em}}
    
    \title{#2}
    \author{\href{https://losdeldgiim.github.io/}{Los Del DGIIM}}
    \date{Granada, #7}
    \maketitle
    
    \tableofcontents
}




\newcommand{\portadaArticle}[7]{

    % 1. Foto de fondo
    % 2. Título
    % 3. Encabezado Izquierdo
    % 4. Color de fondo
    % 5. Coord x del titulo
    % 6. Coord y del titulo
    % 7. Fecha

    % Personaliza el formato del título
    \pretitle{\begin{center}\bfseries\fontsize{42}{56}\selectfont}
    \posttitle{\par\end{center}\vspace{2em}}
    
    % Personaliza el formato del autor
    \preauthor{\begin{center}\Large}
    \postauthor{\par\end{center}\vspace{3em}}
    
    % Personaliza el formato de la fecha
    \predate{\begin{center}\huge}
    \postdate{\par\end{center}\vspace{5em}}
    
    \title{#2}
    \author{\href{https://losdeldgiim.github.io/}{Los Del DGIIM}}
    \date{Granada, #7}
    \thispagestyle{empty}               % Sin encabezado ni pie de página
    \maketitle
    \vfill
}
	\portadaExamen{ffccA4.jpg}{Cálculo I\\Examen XI}{Cálculo I. Examen XI}{MidnightBlue}{-8}{28}{2025}{Roxana Acedo Parra}
	
	\begin{description}
        \item[Asignatura] Cálculo I.
		\item[Curso Académico] 2024-25.
		\item[Grado] Doble Grado en Ingeniería Informática y Matemáticas.
		\item[Grupo] Único.
		\item[Profesor] José Luis Gámez Ruiz.
		\item[Descripción] Convocatoria Extraordinaria.
		\item[Fecha] 6 de febrero de 2025.
		\item[Duración] 3 horas.
		
	\end{description}
	\newpage
	
	
	\begin{ejercicio}[1 punto] Enuncia:
		\begin{enumerate}
			\item El criterio de condensación para series de términos positivos.
			\item El criterio de convergencia absoluta para series de números reales.
		\end{enumerate}
	\end{ejercicio}
	
	\begin{ejercicio}[2 puntos] Decir si las siguientes afirmaciones son verdaderas o falsas, justificando las respuestas:
		\begin{enumerate}
			\item Si $n \in \N$ es primo, $\sqrt{n}$ es irracional.
			\item Todo conjunto de números naturales, no vacío y mayorado, es finito.
			\item Toda sucesión monótona de números reales, que admita una parcial divergente, es divergente.
			\item Toda sucesión de números positivos, convergente a cero, es decreciente.
		\end{enumerate}
	\end{ejercicio}
	
	\begin{ejercicio}[2 puntos] Estudiar la convergencia de:
		\begin{enumerate}
			\item La sucesión $\left\{\dfrac{n^2 \sqrt{n}}{1+2\sqrt{2}+ \dots + n\sqrt{n}}\right\}$.
			\item La serie \ $\displaystyle \sum\limits_{n\geq 1} \dfrac{2^na^n}{1+a^n}$ según el valor de $a \in \R^+$.
		\end{enumerate}
	\end{ejercicio}

	\begin{ejercicio}[2 puntos] Se considera la sucesión $\left\{x_n\right\}$ definida de forma recurrente por $x_1=1$ y $x_{n+1} = \sqrt{1+2x_n}~-1$. Estudiar:
		\begin{enumerate}
			\item La convergencia de la sucesión $\left\{x_n\right\}$.
			\item El carácter de la serie \ $\sum\limits_{n\geq 1} \left(-1\right)^n \frac{(n+1)x_n}{n^2}$ \ .
		\end{enumerate}
	\end{ejercicio}

	\begin{ejercicio}[3 puntos] Sea $f: \R \rightarrow \R$ la función definida por:
		\[
		f(x)=
		\begin{cases}
			\sqrt[3]{x^2+8} & \text{si $x \in \bb{R}^-$,} \\ \\
			e^{-x}\dfrac{1}{1+x} & \text{si $x \in \bb{R}_0^+$.}
		\end{cases}
		\]
		\begin{enumerate}
			\item Estudiar la continuidad de $f$.
			\item Calcular $f(\R)$ y $f([-1,1])$.
			\item ¿Tiene inversa la función $f$?¿Y la función $f_{\big| [-1,1]}$? En caso afirmativo, discutir el dominio y la continuidad de dicha inversa.
		\end{enumerate}		
	\end{ejercicio}
	
	\newpage
	\setcounter{ejercicio}{0} % Reseteo de contador para ejercicios resueltos
	
	\begin{ejercicio}[1 punto] Enuncia:
		\begin{enumerate}
			\item El criterio de condensación para series de términos positivos. \\
			\textbf{Criterio de condensación para series de términos positivos:} \\
			Si $\left\{a_n\right\}$ \underline{decreciente}, con $a_n \geq 0 \quad \forall n \in \N$, entonces:
			$$ \sum\limits_{n\geq 1}a_n \text{ converge }\iff \sum\limits_{n\geq 1}2^na_{2n} \text{ converge }$$
			
			\item El criterio de convergencia absoluta para series de números reales.
			\textbf{Criterio de convergencia absoluta para series:} \\
			Si la serie $\sum\limits_{n\geq 1}a_n$ converge absolutamente (esto es, si $\sum\limits_{n\geq 1}\left|a_n\right|$ converge) $\Rightarrow$ $\sum\limits_{n\geq 1}a_n$ converge.
		\end{enumerate}
	\end{ejercicio}
	
	\begin{ejercicio}[2 puntos] Decir si las siguientes afirmaciones son verdaderas o falsas, justificando las respuestas:
		\begin{enumerate}
			\item Si $n \in \N$ es primo, $\sqrt{n}$ es irracional. \boxed{Verdadera} \\
			\textbf{Demostración }(por reducción al absurdo): \\
			Si fuese $\sqrt{n} \in \bb{Q} \ \Rightarrow  \ \sqrt{n}= \frac{p}{q}$ con $p,q \in \N$, fracción irreducible. \\
			$ \sqrt{n}= \frac{p}{q} \ \Rightarrow \ n= \frac{p^2}{q^2} \ \Rightarrow \ p^2=nq^2 \ \Rightarrow p$ \underline{múltiplo}  de $n \ \Rightarrow \ \exists r \in \N : p=nr \ \Rightarrow $
			(sustituimos $p$ en la expresión $p^2=nq^2$) $ n^2r^2=nq^2 \ \Rightarrow \ q^2=nr^2 \ \Rightarrow$ (igual) $q$ \underline{múltiplo} de $n$ \\
			Luego $p$ y $q$ son, ambos, múltiplos de n. \textbf{¡Contradicción!}
				
			\item Todo conjunto de números naturales, no vacío y mayorado, es finito. \boxed{Verdadera} \\
			\textbf{Demostración:}\\
			Si $A \subseteq \N$, no vacío y mayorado $\Rightarrow$ tiene supremo $\alpha= sup(A)$. Por la propiedad arquimediana, $\exists m \in \N : \alpha < m$. Así $A \subseteq \left\{k \in \N : k \leq m\right\}=S(m)$ finito $\Rightarrow$ $A$ finito.
			
			\item Toda sucesión monótona de números reales, que admita una parcial divergente, es divergente. \boxed{Verdadera} \\
			\textbf{Demostración }(hagamos el caso creciente): \\
			$\left\{x_n\right\}$ monótona creciente y $\exists$ parcial $\left\{x_{\sigma_{(n)}}\right\} \ \to +\infty$ \\
			$\forall k>0 \ \exists m \in \N : $ si $ p \in \N \land p \geq m$, se tiene que  $x_{\sigma_{(p)}} > k$ (en particular $x_{\sigma_{(m)}} > k$)\\
			Si $n \in \N \land n \geq \sigma_{(p)}$ se tiene que $x_n \geq x_{\sigma_{(m)}} > k$. Luego  $\left\{x_n\right\} \ \to +\infty$
			\item Toda sucesión de números positivos, convergente a cero, es decreciente. \boxed{Falsa} \\
			\textbf{Contraejemplos:}\\
			$$\left\{x_n\right\} = \left\{\frac{1}{n + (-1)^{n+1}}\right\}$$
			\[
			\left\{y_n\right\}=
			\left \{
			\begin{array}{c l}
				\frac{1}{n} & \text{si $n$ par} \\ \\
				\frac{1}{n^2} & \text{si $n$ impar} \\ \\
			\end{array}
			\right .
			\]
		\end{enumerate}
	\end{ejercicio}
	
	\newpage
	
	\begin{ejercicio}[2 puntos] Estudiar la convergencia de:
		\begin{enumerate}
			\item la sucesión $\left\{\dfrac{n^2 \sqrt{n}}{1+2\sqrt{2}+ \dots + n\sqrt{n}}\right\}$. \\
			\\
			Llamamos $a_n = n^2\sqrt{n}$,  $b_n = 1+2\sqrt{2}+ \dots + n\sqrt{n}$. \\
			La sucesión es $\left\{\frac{a_n}{b_n}\right\}$, con $\{b_n\} \nearrow\nearrow +\infty$, por lo tanto podemos aplicar el criterio de Stolz.
			\begin{equation*}
				\left(Si \  \left\{\dfrac{a_{n+1} - a_n}{b_{n+1} - b_n}\right\} \longrightarrow L \Longrightarrow \left\{\dfrac{a_n}{b_n}\right\} \longrightarrow L \ (L \in \R \text{ ó } \pm \infty) \right)
			\end{equation*} \\
			Estudiemos la sucesión:
			\begin{gather*}
				\left\{\dfrac{a_{n+1} - a_n}{b_{n+1} - b_n}\right\} = \left\{\dfrac{(n+1)^{\frac{5}{2}} - n^{\frac{5}{2}}}{(n+1)^{\frac{3}{2}}}\right\} = \left\{\dfrac{(n+1)^5- n^5}{(n+1)^{\frac{3}{2}} \left[(n+1)^{\frac{5}{2}} + n^{\frac{5}{2}}\right]}\right\} \\ \left\{\dfrac{\cancel{n^5} +5n^4+10n^3+10n^2+5n+1 - \cancel{n^5}}{(n+1)^4 + (n+1)^\frac{3}{2}n^\frac{5}{2}}\right\}= \left\{\dfrac{\cancel{n^4}\left(5+\dfrac{10}{n}+\dfrac{10}{n^2} + \dfrac{5}{n^3} + \dfrac{1}{n^4}\right)}{\cancel{n^4} \left[\left(\dfrac{n+1}{n}\right)^4 + \left(\dfrac{n+1}{n}\right)^\frac{3}{2}\right]}\right\} \longrightarrow \dfrac{5}{2}
			\end{gather*}
			
			Luego aplicando Stolz, $\boxed{\boxed{\left\{\dfrac{a_n}{b_n}\right\} \rightarrow \frac{5}{2}}}$
			
			 \hrulefill 
			
			También podría haberse calculado así: 
			\begin{gather*}
				\left\{\dfrac{a_{n+1} - a_n}{b_{n+1} - b_n}\right\} = \left\{\dfrac{(n+1)^{\frac{5}{2}} - n^{\frac{5}{2}}}{(n+1)^{\frac{3}{2}}}\right\} = \left\{\frac{n^{\frac{5}{2}}\left[\left(\frac{n+1}{n}\right)^{\frac{5}{2}} - 1\right]}{n^{\frac{3}{2}}\left(\frac{n+1}{1}\right)^{\frac{3}{2}} }\right\} = \\
				\left\{\frac{n\left[\left(\frac{n+1}{n}\right)^{\frac{5}{2}} - 1\right]}{\left(\frac{n+1}{1}\right)^{\frac{3}{2}} }\right\} 
				\xrightarrow[ \left(\ast\right) \, \text{\scriptsize debajo }]{} \frac{5}{2} \xRightarrow[\left(\text{\scriptsize Stolz}\right)]{} \boxed{\boxed{\left\{\dfrac{a_n}{b_n}\right\} \rightarrow \frac{5}{2}}}
			\end{gather*}
      	  	
      	  	(*) El denominador tiende a 1, estudiemos por separado el numerador:
      	  	
      	  	$\left.\begin{array}{l}
			\{x_n\} = \left(\dfrac{n+1}{1}\right)^\frac{5}{2} \longrightarrow 1
			 \\\\\\
			\{y_n\}= \{n\}
			\end{array}\hspace{2cm}\right\}$  $\left( \begin{array}{c}
				\text{puedo aplicar el} \\
				\text{criterio de Euler $(\ast)$}
			\end{array}\right)$\\
	
			\[
			(\ast) \ x_n ^{y_n} \longrightarrow e^H \Longleftrightarrow y_n(x_n-1) \longrightarrow H
			\]			
		
			$\left\{x_n^{y_n}\right\} = \left\{\left(\dfrac{n+1}{1}\right)^{\frac{5}{2}n}\right\} = \left\{\left[\left(\dfrac{n+1}{1}\right)^{n}\right]^\frac{5}{2}\right\} \longrightarrow \, e^\frac{5}{2}$
			
			Luego $\left\{n\left[\left(\dfrac{n+1}{1}\right)^{\frac{5}{2}}-1\right]\right\} = \left\{y_n(x_n-1)\right\} \, \longrightarrow \frac{5}{2}$
			
			\item la serie \ $\displaystyle \sum\limits_{n\geq 1} \dfrac{2^na^n}{1+a^n}$ según el valor de $a \in \R^+$. \\
			Convergencia de la serie \ $\displaystyle \sum\limits_{n\geq 1} \dfrac{2^na^n}{1+a^n}$, según el valor de $a \in \R^+$.
			
			Consideremos los casos: $\underline{0 < a < \frac{1}{2}, \, a=\frac{1}{2}, \, \frac{1}{2} < a < 1, \, a=1, \, a>1.}$
			\begin{itemize}[]
				\item[\textasteriskcentered] Caso $0 < a < \frac{1}{2}$. Criterio de la raíz $\left[ \text{Si } \sqrt[n]{a_n} \rightarrow L \geq 0, 
				\begin{cases}
					\text{Si } L > 1 \Rightarrow \sum a_n \text{ converge} \\
					\text{Si } L < 1 \Rightarrow \sum a_n \text{ no converge}
				\end{cases}
				\right]$
				
				\begin{gather*}
					\left\{ \sqrt[n]{\frac{2^na^n}{1+a^n}} \right\} = \left\{\frac{2a}{\sqrt[n]{1+a^n}}\right\} \rightarrow 2a<1 \, \Rightarrow 
				\end{gather*}
				\underline{\underline{Para $0 < a < \frac{1}{2}$ la serie \textbf{converge}.}}
				
				\item[\textasteriskcentered] Caso $a=\frac{1}{2}$. Los sumandos $\left\{\frac{2^na^n}{1+a^n} \right\} = \left\{ \frac{1}{1 + \left(\frac{1}{2}\right)^n}\right\} \rightarrow 1 \quad \text{¡¡No tienden a 0 !!}$ \\
				Luego, \underline{\underline{para $a=\frac{1}{2}$, la serie \textbf{no converge}.}}
				
				\item[\textasteriskcentered] Caso $\frac{1}{2} < a < 1$. Los sumandos $\left\{\frac{2^na^n}{1+a^n} \right\} = \left\{ \frac{(2a)^n}{1+a^n}\right\} \rightarrow +\infty \quad \text{¡¡No tienden a 0 !!}$ \\
				Luego, \underline{\underline{para $\frac{1}{2} < a < 1$, la serie \textbf{no converge}.}}
				
				\item[\textasteriskcentered] Caso $a = 1$. Los sumandos $\left\{\frac{2^na^n}{1+a^n} \right\} = \left\{2^{n-1}\right\} \rightarrow +\infty \quad \text{¡¡No tienden a 0 !!}$ \\
				Luego, \underline{\underline{para $a=1$, la serie \textbf{no converge}.}}
				
				\item[\textasteriskcentered] Caso $a > 1$. Los sumandos $\left\{\frac{2^na^n}{1+a^n} \right\} = \left\{ \frac{2^n\cancel{a^n}}{\cancel{a^n}\left(\left(\frac{1}{a}\right)^n + 1\right)}\right\} \rightarrow +\infty \quad \text{¡¡No tienden a 0 !!}$ \\
				Luego, \underline{\underline{para $a > 1$, la serie \textbf{no converge}.}}
			\end{itemize}
		Alternativamente, todos los casos con $\boxed{a > \frac{1}{2}}$ también se resuelven comparando la serie a estudiar con la del caso $a=\frac{1}{2}$. Bastará con observar que, si $x>y>0 \Rightarrow \frac{x}{1+x} = \frac{1+x-1}{1+x} = 1- \frac{1}{1+x} > \frac{1}{1+y} = \frac{y}{1+y}$, con lo que $x=a^n \quad y=\left(\frac{1}{2}\right)^n$ obtenemos la comparación:
		$$
		2^n\frac{a^n}{1+a^n} > 2^n\frac{\left(\frac{1}{2}\right)^n}{1+\left(\frac{1}{2}\right)^n} \quad (\text{sumandos del caso } a=1)
		$$
		Por el criterio de comparación, para $a>\frac{1}{2}$,\\
		$$
		\text{ya que } \sum\limits_{n \geq 1} 2^n\frac{\left(\frac{1}{2}\right)^n}{1+\left(\frac{1}{2}\right)^n}  \text{ no converge } \Rightarrow \, \sum\limits_{n \geq 1}2^n\frac{a^n}{1+a^n}\text{ no converge.} 
		$$			
		\end{enumerate}
	\end{ejercicio}
	
	\newpage
	
		\begin{ejercicio}[2 puntos] Se considera la sucesión $\left\{x_n\right\}$ definida por $x_1 = \sqrt{1+2x_n} -1$. Estudiar:
		\begin{enumerate}
			\item La convergencia de la sucesión $\left\{x_n\right\}$. \\
			Probaremos que $\left\{x_n\right\}$ es estrictamente decreciente y minorada por 0. Más concretamente, demostraremos que $\boxed{0<x_{n+1}<x_n \forall n \in \N}$ por inducción: \\
			\\
			Sea $A= \left\{n \in \N : 0<x_{n+1}<x_n \right\} \subseteq \N$ \\
			¿Es A inductivo? $\Leftrightarrow 1) \land 2)$
			\begin{enumerate}[label=\arabic*)]
				\item $\text{¿} 1\in A? \Leftrightarrow \text{¿}0<x_2<x_1? \Leftrightarrow \text{¿}0< \sqrt{3} -1 < 1?$ \fbox{Sí}
				\item Si $k \in A$ {\tiny (hipótesis de inducción)} ¿$\Rightarrow k+1 \in A$? \\
				La hipótesis de inducción dice: $0<x_{n+1}<x_n$, y queremos probar ¿$0<x_{n+2}<x_{n+1}$?
				\[
				\left.
				\begin{array}{l}
				\text{Claramente } x_{k+2}= \sqrt{1+2x_{k+1}}-1 > \text{{\tiny (ya que $x_{k+1}>0$)}}\sqrt{1} - 1 = 0 \\
				\text{Además, } x_{k+2} = \sqrt{1+2x_{k+1}} -1 < \sqrt{1+2x_k} -1 = x_{k+1}				
				\end{array}
				\right]
				\Rightarrow \text{\fbox{Sí}}
				\]
			\end{enumerate}
			
			Luego $\left\{x_n\right\}$ estrictamente decreciente y minorada. En particular será convergente $\left\{x_n\right\} \searrow L \geq 0$ 
			\[
			\left.
			\begin{array}{l}
			\left\{x_{n+1}\right\} \longrightarrow L \text{ (parcial)} \\
			\left\{ \sqrt{1+2x_n} -1 \right\} \longrightarrow \sqrt{1+2L} -1
			\end{array}
			\right]
			\xRightarrow[]{\text{{\tiny (uni. del lím.)}}} L = \sqrt{1+2L} -1 \Rightarrow \boxed{L=0}
			\]
				
			Luego $\left\{x_n\right\} \searrow 0$.
		
			
			\item El carácter de la serie \ $\sum\limits_{n\geq 1} \left(-1\right)^n \frac{(n+1)x_n}{n^2}$ \ .	\\
			Es una serie alternada, podemos aplicar el criterio de Leibniz, nos preguntamos si ¿$ \left\{\frac{(n+1)}{n^2}x_n\right\} \searrow 0$? (ya sabemos que $\left\{x_n\right\} \searrow 0$) \\
			Veamos que $\left\{\frac{(n+1)}{n^2}\right\}$ es decreciente:
			\begin{gather*}
				\text{¿} \frac{n+2}{(n+1)^2} < \frac{n+1}{n^2} \text{?} \Leftrightarrow \text{¿} n^2(n+2) < (n+1)^3 \text{?} \Leftrightarrow \text{¿} n^3 + 2n^2 < n^3 +3n^2+3n+1 \text{? \fbox{Sí}}  
			\end{gather*}
			Luego $\left\{\frac{(n+1)}{n^2}x_n\right\}$ es producto de dos sucesiones decrecientes y positivas y, por lo tanto, es decreciente y su límite es (producto de límites) 0. Así, por el criterio de Leibniz, \underline{\underline{ $\sum\limits_{n\geq 1} \left(-1\right)^n \frac{(n+1)x_n}{n^2}$ converge.}}
		\end{enumerate}
	\end{ejercicio}
	
	\begin{ejercicio}[3 puntos] Sea $f: \R \rightarrow \R$ la función definida por:
		\[
		f(x)=
		\left \{
		\begin{array}{c l}
			\sqrt[3]{x^2+8} & \text{si $x \in \bb{R}^-$,} \\ \\
			e^{-x}\dfrac{1}{1+x} & \text{si $x \in \bb{R}_0^+$.} \\ \\
		\end{array}
		\right .
		\]
		\begin{enumerate}
			\item Estudiar la continuidad de $f$. \\
			Por el carácter local de la continuidad, $f$ será continua en todo punto de $\R^-$ y también en todo punto de $\R^+$. Falta saber si es continua en 0. Por un resultado visto en clase, consecuencia de la caracterización de la continuidad por sucesiones monótonas:
			\begin{gather*}
				f \text{ continua en } 0 \Leftrightarrow \sqrt[3]{0^2+8} = e^{-0} + \frac{1}{1+0} \Leftrightarrow \sqrt[3]{8} = 1+1 \qquad \fbox{Sí}  
			\end{gather*}
			Luego \underline{\underline{$f$ es continua en $\R$}}.
			\item Calcular $f(\R)$ y $f([-1,1])$.\\
			¿$f(\R)$? $f(\R)= f(\R^- \cup \R^+_0) = f(\R^-) \cup f(\R^+_0)$ \\
			\begin{itemize}
				\item[\textasteriskcentered] Comencemos por $f(\R^-)$: por el T.V.I. será un intervalo. Además $\forall x,y \in \R^- x < y \Rightarrow x^2 > y^2 \Rightarrow f(x) > f(y)$ \\
				Luego \underline{$f$ es estrictamente decreciente en $\R^-$} (en particular \underline{inyectiva} en $\R^-$).
				\[
				\left.
				\begin{array}{l}
					\text{Si} \left\{x_n\right\} (x_n \in \R^- \forall n \in \N) \rightarrow 0 \Rightarrow \left\{f(x_n)\right\} \rightarrow 2 \\
					\text{Si} \left\{x_n\right\} \rightarrow -\infty \Rightarrow \left\{f(x_n)\right\}
					\rightarrow +\infty
				\end{array}
				\right]
				\Rightarrow \boxed{f(\R^-) = \left(2, +\infty\right)}
				\]
				\item[\textasteriskcentered] Hagamos ahora $f(\R^+)$: de nuevo, por el T.V.I. será un intervalo. Además, en $\R_0^+ \, f$ es suma de funciones estrictamente decrecientes. \\
				Luego \underline{es estrictamente decreciente en $\R_0^+$} (en particular inyectiva en $\R_0^+$). 
				\[
				\left.
				\begin{array}{l}
					f(0)=2 \\
					\text{Si} \left\{y_n\right\} \rightarrow +\infty \Rightarrow \left\{f(y_n)\right\} \rightarrow 0
				\end{array}
				\right]
				\Rightarrow \boxed{f(\R_0^+) = \left(0, 2 \right]}
				\]
				\item[\textasteriskcentered] En consecuencia, $f(\R)= f(\R^-) \cup f(\R^+_0) = \left(2, +\infty\right) \cup  \left(0, 2 \right] = \left(0, +\infty \right)$
			\end{itemize}
			
			Para calcular $f(\left[-1,1\right])$, ya que $f$ es estrictamente decreciente en todo su dominio y continua, tendremos: $$f(\left[-1,1\right]) = \left[f(1), f(-1)\right]= \left[\frac{1}{e} + \frac{1}{2}, \sqrt[3]{9}\right]$$
			
			\item ¿Tiene inversa la función $f$?¿Y la función $f\restriction_{[-1,1]}$? En caso afirmativo, discutir el dominio y la continuidad de dicha inversa. \\
			$f$ es estrictamente decreciente en su dominio ($\Rightarrow$ inyectiva) y por tanto  tiene inversa, cuyo dominio será la imagen de $f$. Además, dado que el dominio de $f$ es un intervalo ($\R$), la inversa será continua (*). \\
			$$ f^{-1}: (0, +\infty) \longrightarrow \R \text{ continua.}$$
			Por el mismo motivo, al ser $g= f\restriction_{[-1,1]}$ estrictamente monótona tendrá inversa cuyo dominio será la imagen de $g$, es decir, $f([-1,1])$. De nuevo, por ser el dominio de $g$ un intervalo, $g^{-1}$ será continua (*). \\
			$$ g^{-1}=(f\restriction_{[-1,1]})^{-1}: \left[\frac{1}{e} + \frac{1}{2}, \sqrt[3]{9}\right] \longrightarrow \left[-1,1\right] \quad \text{continua.}$$ \\
			
			(*) Hemos usado un resultado de teoría sobre monotonía y continuidad:
			\[
			\left.
			\begin{array}{l}
				\text{Si } I \subseteq \R \text{ intervalo} \\
				\text{y } h:I \longrightarrow \R \text{ estrictamente monótona}
			\end{array}
			\right]
			\Rightarrow \exists \text{ inversa } h^{-1} \text{ y es continua.}
			\]
			

		\end{enumerate}		
	\end{ejercicio}

\end{document}	