\documentclass[12pt]{article}

% Idioma y codificación
\usepackage[spanish, es-tabla]{babel}       %es-tabla para que se titule "Tabla"
\usepackage[utf8]{inputenc}

% Márgenes
\usepackage[a4paper,top=3cm,bottom=2.5cm,left=3cm,right=3cm]{geometry}

% Comentarios de bloque
\usepackage{verbatim}

% Paquetes de links
\usepackage[hidelinks]{hyperref}    % Permite enlaces
\usepackage{url}                    % redirecciona a la web

% Más opciones para enumeraciones
\usepackage{enumitem}

% Personalizar la portada
\usepackage{titling}

% Paquetes de tablas
\usepackage{multirow}


%------------------------------------------------------------------------

%Paquetes de figuras
\usepackage{caption}
\usepackage{subcaption} % Figuras al lado de otras
\usepackage{float}      % Poner figuras en el sitio indicado H.


% Paquetes de imágenes
\usepackage{graphicx}       % Paquete para añadir imágenes
\usepackage{transparent}    % Para manejar la opacidad de las figuras

% Paquete para usar colores
\usepackage[dvipsnames]{xcolor}
\usepackage{pagecolor}      % Para cambiar el color de la página

% Habilita tamaños de fuente mayores
\usepackage{fix-cm}

% Para los gráficos
\usepackage{tikz}

% Para poder situar los nodos en los grafos
\usetikzlibrary{positioning}


%------------------------------------------------------------------------

% Paquetes de matemáticas
\usepackage{mathtools, amsfonts, amssymb, mathrsfs}
\usepackage[makeroom]{cancel}     % Simplificar tachando
\usepackage{polynom}    % Divisiones y Ruffini
\usepackage{units} % Para poner fracciones diagonales con \nicefrac

\usepackage{pgfplots}   %Representar funciones
\pgfplotsset{compat=1.18}  % Versión 1.18

\usepackage{tikz-cd}    % Para usar diagramas de composiciones
\usetikzlibrary{calc}   % Para usar cálculo de coordenadas en tikz

%Definición de teoremas, etc.
\usepackage{amsthm}
%\swapnumbers   % Intercambia la posición del texto y de la numeración

\theoremstyle{plain}

\makeatletter
\@ifclassloaded{article}{
  \newtheorem{teo}{Teorema}[section]
}{
  \newtheorem{teo}{Teorema}[chapter]  % Se resetea en cada chapter
}
\makeatother

\newtheorem{coro}{Corolario}[teo]           % Se resetea en cada teorema
\newtheorem{prop}[teo]{Proposición}         % Usa el mismo contador que teorema
\newtheorem{lema}[teo]{Lema}                % Usa el mismo contador que teorema

\theoremstyle{remark}
\newtheorem*{observacion}{Observación}

\theoremstyle{definition}

\makeatletter
\@ifclassloaded{article}{
  \newtheorem{definicion}{Definición} [section]     % Se resetea en cada chapter
}{
  \newtheorem{definicion}{Definición} [chapter]     % Se resetea en cada chapter
}
\makeatother

\newtheorem*{notacion}{Notación}
\newtheorem*{ejemplo}{Ejemplo}
\newtheorem*{ejercicio*}{Ejercicio}             % No numerado
\newtheorem{ejercicio}{Ejercicio} [section]     % Se resetea en cada section


% Modificar el formato de la numeración del teorema "ejercicio"
\renewcommand{\theejercicio}{%
  \ifnum\value{section}=0 % Si no se ha iniciado ninguna sección
    \arabic{ejercicio}% Solo mostrar el número de ejercicio
  \else
    \thesection.\arabic{ejercicio}% Mostrar número de sección y número de ejercicio
  \fi
}


% \renewcommand\qedsymbol{$\blacksquare$}         % Cambiar símbolo QED
%------------------------------------------------------------------------

% Paquetes para encabezados
\usepackage{fancyhdr}
\pagestyle{fancy}
\fancyhf{}

\newcommand{\helv}{ % Modificación tamaño de letra
\fontfamily{}\fontsize{12}{12}\selectfont}
\setlength{\headheight}{15pt} % Amplía el tamaño del índice


%\usepackage{lastpage}   % Referenciar última pag   \pageref{LastPage}
\fancyfoot[C]{\thepage}

%------------------------------------------------------------------------

% Conseguir que no ponga "Capítulo 1". Sino solo "1."
\makeatletter
\@ifclassloaded{book}{
  \renewcommand{\chaptermark}[1]{\markboth{\thechapter.\ #1}{}} % En el encabezado
    
  \renewcommand{\@makechapterhead}[1]{%
  \vspace*{50\p@}%
  {\parindent \z@ \raggedright \normalfont
    \ifnum \c@secnumdepth >\m@ne
      \huge\bfseries \thechapter.\hspace{1em}\ignorespaces
    \fi
    \interlinepenalty\@M
    \Huge \bfseries #1\par\nobreak
    \vskip 40\p@
  }}
}
\makeatother

%------------------------------------------------------------------------
% Paquetes de cógido
\usepackage{minted}
\renewcommand\listingscaption{Código fuente}

\usepackage{fancyvrb}
% Personaliza el tamaño de los números de línea
\renewcommand{\theFancyVerbLine}{\small\arabic{FancyVerbLine}}

% Estilo para C++
\newminted{cpp}{
    frame=lines,
    framesep=2mm,
    baselinestretch=1.2,
    linenos,
    escapeinside=||
}

% para minted
\definecolor{LightGray}{rgb}{0.95,0.95,0.92}
\setminted{
    linenos=true,
    stepnumber=5,
    numberfirstline=true,
    autogobble,
    breaklines=true,
    breakautoindent=true,
    breaksymbolleft=,
    breaksymbolright=,
    breaksymbolindentleft=0pt,
    breaksymbolindentright=0pt,
    breaksymbolsepleft=0pt,
    breaksymbolsepright=0pt,
    fontsize=\footnotesize,
    bgcolor=LightGray,
    numbersep=10pt
}


\usepackage{listings} % Para incluir código desde un archivo

\renewcommand\lstlistingname{Código Fuente}
\renewcommand\lstlistlistingname{Índice de Códigos Fuente}

% Definir colores
\definecolor{vscodepurple}{rgb}{0.5,0,0.5}
\definecolor{vscodeblue}{rgb}{0,0,0.8}
\definecolor{vscodegreen}{rgb}{0,0.5,0}
\definecolor{vscodegray}{rgb}{0.5,0.5,0.5}
\definecolor{vscodebackground}{rgb}{0.97,0.97,0.97}
\definecolor{vscodelightgray}{rgb}{0.9,0.9,0.9}

% Configuración para el estilo de C similar a VSCode
\lstdefinestyle{vscode_C}{
  backgroundcolor=\color{vscodebackground},
  commentstyle=\color{vscodegreen},
  keywordstyle=\color{vscodeblue},
  numberstyle=\tiny\color{vscodegray},
  stringstyle=\color{vscodepurple},
  basicstyle=\scriptsize\ttfamily,
  breakatwhitespace=false,
  breaklines=true,
  captionpos=b,
  keepspaces=true,
  numbers=left,
  numbersep=5pt,
  showspaces=false,
  showstringspaces=false,
  showtabs=false,
  tabsize=2,
  frame=tb,
  framerule=0pt,
  aboveskip=10pt,
  belowskip=10pt,
  xleftmargin=10pt,
  xrightmargin=10pt,
  framexleftmargin=10pt,
  framexrightmargin=10pt,
  framesep=0pt,
  rulecolor=\color{vscodelightgray},
  backgroundcolor=\color{vscodebackground},
}

%------------------------------------------------------------------------

% Comandos definidos
\newcommand{\bb}[1]{\mathbb{#1}}
\newcommand{\cc}[1]{\mathcal{#1}}

% I prefer the slanted \leq
\let\oldleq\leq % save them in case they're every wanted
\let\oldgeq\geq
\renewcommand{\leq}{\leqslant}
\renewcommand{\geq}{\geqslant}

% Si y solo si
\newcommand{\sii}{\iff}

% Letras griegas
\newcommand{\eps}{\epsilon}
\newcommand{\veps}{\varepsilon}
\newcommand{\lm}{\lambda}

\newcommand{\ol}{\overline}
\newcommand{\ul}{\underline}
\newcommand{\wt}{\widetilde}
\newcommand{\wh}{\widehat}

\let\oldvec\vec
\renewcommand{\vec}{\overrightarrow}

% Derivadas parciales
\newcommand{\del}[2]{\frac{\partial #1}{\partial #2}}
\newcommand{\Del}[3]{\frac{\partial^{#1} #2}{\partial #3^{#1}}}
\newcommand{\deld}[2]{\dfrac{\partial #1}{\partial #2}}
\newcommand{\Deld}[3]{\dfrac{\partial^{#1} #2}{\partial #3^{#1}}}


\newcommand{\AstIg}{\stackrel{(\ast)}{=}}
\newcommand{\Hop}{\stackrel{L'H\hat{o}pital}{=}}

\newcommand{\red}[1]{{\color{red}#1}} % Para integrales, destacar los cambios.

% Método de integración
\newcommand{\MetInt}[2]{
    \left[\begin{array}{c}
        #1 \\ #2
    \end{array}\right]
}

% Declarar aplicaciones
% 1. Nombre aplicación
% 2. Dominio
% 3. Codominio
% 4. Variable
% 5. Imagen de la variable
\newcommand{\Func}[5]{
    \begin{equation*}
        \begin{array}{rrll}
            #1:& #2 & \longrightarrow & #3\\
               & #4 & \longmapsto & #5
        \end{array}
    \end{equation*}
}

%------------------------------------------------------------------------

\usepackage{extarrows} 

\begin{document}

    % 1. Foto de fondo
    % 2. Título
    % 3. Encabezado Izquierdo
    % 4. Color de fondo
    % 5. Coord x del titulo
    % 6. Coord y del titulo
    % 7. Fecha

    
    % 1. Foto de fondo
% 2. Título
% 3. Encabezado Izquierdo
% 4. Color de fondo
% 5. Coord x del titulo
% 6. Coord y del titulo
% 7. Fecha

\newcommand{\portada}[7]{

    \portadaBase{#1}{#2}{#3}{#4}{#5}{#6}{#7}
    \portadaBook{#1}{#2}{#3}{#4}{#5}{#6}{#7}
}

\newcommand{\portadaExamen}[7]{

    \portadaBase{#1}{#2}{#3}{#4}{#5}{#6}{#7}
    \portadaArticle{#1}{#2}{#3}{#4}{#5}{#6}{#7}
}




\newcommand{\portadaBase}[7]{

    % Tiene la portada principal y la licencia Creative Commons
    
    % 1. Foto de fondo
    % 2. Título
    % 3. Encabezado Izquierdo
    % 4. Color de fondo
    % 5. Coord x del titulo
    % 6. Coord y del titulo
    % 7. Fecha
    
    
    \thispagestyle{empty}               % Sin encabezado ni pie de página
    \newgeometry{margin=0cm}        % Márgenes nulos para la primera página
    
    
    % Encabezado
    \fancyhead[L]{\helv #3}
    \fancyhead[R]{\helv \nouppercase{\leftmark}}
    
    
    \pagecolor{#4}        % Color de fondo para la portada
    
    \begin{figure}[p]
        \centering
        \transparent{0.3}           % Opacidad del 30% para la imagen
        
        \includegraphics[width=\paperwidth, keepaspectratio]{assets/#1}
    
        \begin{tikzpicture}[remember picture, overlay]
            \node[anchor=north west, text=white, opacity=1, font=\fontsize{60}{90}\selectfont\bfseries\sffamily, align=left] at (#5, #6) {#2};
            
            \node[anchor=south east, text=white, opacity=1, font=\fontsize{12}{18}\selectfont\sffamily, align=right] at (9.7, 3) {\textbf{\href{https://losdeldgiim.github.io/}{Los Del DGIIM}}};
            
            \node[anchor=south east, text=white, opacity=1, font=\fontsize{12}{15}\selectfont\sffamily, align=right] at (9.7, 1.8) {Doble Grado en Ingeniería Informática y Matemáticas\\Universidad de Granada};
        \end{tikzpicture}
    \end{figure}
    
    
    \restoregeometry        % Restaurar márgenes normales para las páginas subsiguientes
    \pagecolor{white}       % Restaurar el color de página
    
    
    \newpage
    \thispagestyle{empty}               % Sin encabezado ni pie de página
    \begin{tikzpicture}[remember picture, overlay]
        \node[anchor=south west, inner sep=3cm] at (current page.south west) {
            \begin{minipage}{0.5\paperwidth}
                \href{https://creativecommons.org/licenses/by-nc-nd/4.0/}{
                    \includegraphics[height=2cm]{assets/Licencia.png}
                }\vspace{1cm}\\
                Esta obra está bajo una
                \href{https://creativecommons.org/licenses/by-nc-nd/4.0/}{
                    Licencia Creative Commons Atribución-NoComercial-SinDerivadas 4.0 Internacional (CC BY-NC-ND 4.0).
                }\\
    
                Eres libre de compartir y redistribuir el contenido de esta obra en cualquier medio o formato, siempre y cuando des el crédito adecuado a los autores originales y no persigas fines comerciales. 
            \end{minipage}
        };
    \end{tikzpicture}
    
    
    
    % 1. Foto de fondo
    % 2. Título
    % 3. Encabezado Izquierdo
    % 4. Color de fondo
    % 5. Coord x del titulo
    % 6. Coord y del titulo
    % 7. Fecha


}


\newcommand{\portadaBook}[7]{

    % 1. Foto de fondo
    % 2. Título
    % 3. Encabezado Izquierdo
    % 4. Color de fondo
    % 5. Coord x del titulo
    % 6. Coord y del titulo
    % 7. Fecha

    % Personaliza el formato del título
    \pretitle{\begin{center}\bfseries\fontsize{42}{56}\selectfont}
    \posttitle{\par\end{center}\vspace{2em}}
    
    % Personaliza el formato del autor
    \preauthor{\begin{center}\Large}
    \postauthor{\par\end{center}\vfill}
    
    % Personaliza el formato de la fecha
    \predate{\begin{center}\huge}
    \postdate{\par\end{center}\vspace{2em}}
    
    \title{#2}
    \author{\href{https://losdeldgiim.github.io/}{Los Del DGIIM}}
    \date{Granada, #7}
    \maketitle
    
    \tableofcontents
}




\newcommand{\portadaArticle}[7]{

    % 1. Foto de fondo
    % 2. Título
    % 3. Encabezado Izquierdo
    % 4. Color de fondo
    % 5. Coord x del titulo
    % 6. Coord y del titulo
    % 7. Fecha

    % Personaliza el formato del título
    \pretitle{\begin{center}\bfseries\fontsize{42}{56}\selectfont}
    \posttitle{\par\end{center}\vspace{2em}}
    
    % Personaliza el formato del autor
    \preauthor{\begin{center}\Large}
    \postauthor{\par\end{center}\vspace{3em}}
    
    % Personaliza el formato de la fecha
    \predate{\begin{center}\huge}
    \postdate{\par\end{center}\vspace{5em}}
    
    \title{#2}
    \author{\href{https://losdeldgiim.github.io/}{Los Del DGIIM}}
    \date{Granada, #7}
    \thispagestyle{empty}               % Sin encabezado ni pie de página
    \maketitle
    \vfill
}
    \portadaExamen{ffccA4.jpg}{Cálculo I\\Examen V}{Cálculo I. Examen V}{MidnightBlue}{-8}{28}{2023-2024}{Jesús Muñoz Velasco}

    \begin{description}
        \item[Asignatura] Cálculo I.
        \item[Curso Académico] 2022-23.
        \item[Grado] Doble Grado en Ingeniería Informática y Matemáticas.
        \item[Grupo] Único.
        \item[Profesor] Jose Luis Gámez Ruiz.
        \item[Descripción] Convocatoria Ordinaria.
        \item[Fecha] 19 de enero de 2023.
        %\item[Duración] 3 horas.
    
    \end{description}
    \newpage

    \begin{ejercicio}[2.5 puntos]
        Escribe los siguientes enunciados:
        \begin{enumerate}
            \item Definición de sucesión Convergente:

            Una sucesión de números reales $\{x_n\}$ se dice ``convergente" si $\exists L \in \bb{R}$ tal que $\forall \varepsilon > 0 $ $\exists m \in \bb{N}: $ si $n \in \bb{N}, n\geq m$, se tiene $|x_n-L|<\varepsilon$.
            
            \item Criterio del cociente para sucesiones:

            Sea $a_n > 0$  $\forall n \in \bb{N}$,
            \[
            \text{Si} \left\{ \dfrac{a_{n+1}}{a_n}\right\} \longrightarrow L \Longrightarrow \{ \sqrt[n]{a_n}\} \longrightarrow L \hspace{1cm} (L \in \bb{R}_0^+ \text{ o } ``+\infty")
            \]
            
            \item Criterio de condensación para series:

            Sean $a_n \geq 0$ $\forall n \in \bb{N}$, con \underline{$a_n$ decreciente}. Entonces:
            \[
            \sum_{n\geq 1}a_n \text{ convergente} \Longleftrightarrow \sum_{n\geq1}2^n a_{2^n} \text{ convergente}
            \]
            \item Definición de función continua en un punto \textit{a} de su dominio:

            Sea $A \subseteq \bb{R}$, $f:A \rightarrow\bb{R}$, $a \in A$. Se dice ``f continua en el punto \textit{a}" si
            \[
            \forall\{a_n\} \longrightarrow a \Longrightarrow \{f(a_n)\} \longrightarrow f(a) \hspace{1cm} (a_n \in A \text{ } \forall n \in \bb{N})
            \]
            \item Teorema de Bolzano-Weierstrass:
            
            ``La imagen, por una función continua, de un intervalo cerrado y acotado, es un intervalo cerrado y acotado" \\
            En particular, si $f:[a,b] \rightarrow \bb{R}$ es continua, entonces f \underline{alcanza} un máximo absoluto y un mínimo absoluto.

            
        \end{enumerate}
    \end{ejercicio}

    \begin{ejercicio}[2 puntos]
        Estudia la convergencia de las siguientes sucesiones:
        \begin{enumerate}
            \item $x_1 = 2,$ $x_{n+1} = \dfrac{x_n+\frac{2}{x_n}}{2}$ (1 punto)

            Probaré que $\{x_n\}$ $\underbrace{\text{decreciente}}_{(2)}$ y $\underbrace{\text{minorada por } \sqrt{2}}_{(1)}$
            \begin{itemize}
                \item [(1)] ¿ $x_n \geq \sqrt{2}$ $\forall n \in \bb{N}$ ? (por inducción)
                \begin{itemize}
                    \item $\underline{n=1}$ $x_1=2 \geq \sqrt{2}$. Sí
                    \item $\underbrace{x_n \geq \sqrt{2}}_{hip.\ de\ ind}$ ¿$\Rightarrow x_{n+1} \geq \sqrt{2}$?
                    \[
                        x_{n+1} \geq \sqrt{2} \Longleftrightarrow \dfrac{x_n+\frac{2}{x_n}}{2} \geq \sqrt{2} \Longleftrightarrow x_n+\dfrac{2}{x_n} \geq 2\sqrt{2} \Longleftrightarrow x_n^2 +2 \geq 2 \sqrt{2}x_n 
                    \]
                    \[
                        \Longleftrightarrow x_n^2 +2 -2\sqrt{2} \geq 0 \Longleftrightarrow (x_n - \sqrt{2})^2 \geq 0 \hspace{1cm} \text{Sí}
                    \]
                    \\
                    Luego $x_{n+1}\geq \sqrt{2}$ y concluimos (1).

                    \item [(2)] ¿$\{x_n\}$ decreciente ? $\Longleftrightarrow$ ¿$x_{n+1} \leq x_n$ $\forall n \in \bb{N}$?
                    \[
                        x_{n+1} \leq x_n \Longleftrightarrow \dfrac{x_n+\frac{2}{x_n}}{2} \leq x_n \Longleftrightarrow x_n + \dfrac{2}{x_n} \leq 2 x_n \Longleftrightarrow \dfrac{2}{x_n} \leq x_n 
                    \]
                    \[
                        \Longleftrightarrow 2 \leq x_n^2 \hspace{1cm} \text{Sí, por (1)}
                    \]
                    \\
                    Así concluimos (2).
                \end{itemize}
            \end{itemize}

            Por ser $\{x_n\}$ decreciente y minorada (por $\sqrt{2}$) $\Longrightarrow \{x_n\}$ converge.\\
            Sea $L=\lim\{x_n\} \Longrightarrow \{x_{n+1}\} \longrightarrow L$ (parcial)
            \[
                \left .
                \begin{array}{r}
                     x_{n+1} \longrightarrow L \\\\
                     \dfrac{x_n+\frac{2}{x_n}}{2} \longrightarrow \dfrac{L+\frac{2}{L}}{2}
                \end{array}
                \right\} \xLongrightarrow{\begin{array}{c}
                    \text{\scriptsize{(unicidad}} \\
                    \text{\scriptsize{del lim)}} 
                \end{array}} L=\dfrac{L+\frac{2}{L}}{2} \Longrightarrow L^2 =2 \Longrightarrow 
                \left\{
                \begin{array}{c}
                     L=\sqrt{2}\\
                     \xcancel{L=-\sqrt{2}}
                \end{array}
                \right.
            \]

            (el ``candidato" $-\sqrt{2}$ se descarta, pues $x_n \geq \sqrt{2}$ $\forall n \in \bb{N}$)\\
            En conclusión, $\{x_n\} \searrow \sqrt{2}$
            
            \item $\left\{ \dfrac{n \log n}{\log(n!)} \right\}$ (1 punto)\\
            
            Llamamos $a_n=n \log(n)$, $b_n=\log(n!) \nearrow \nearrow +\infty$ (puedo aplicar Stolz)
            \[ 
                \dfrac{a_{n+1} - a_n}{b_{n+1}-b_n} = \dfrac{(n+1) \log(n+1) - n \log(n)}{\log(n+1)! - \log(n)} = \dfrac{\log(n+1) + \log\left( \frac{n+1}{n}\right)^n}{\log(n+1)} =
            \]
            \[
            = 1 + \dfrac{\log\left( \frac{n+1}{n}\right)^n}{\log(n+1)} \xlongrightarrow{(\ast)} 1+0 = 1
            \]

            Donde en $(\ast)$ he aplicado que $\log\left( \frac{n+1}{n}\right)^n \rightarrow \log(e)=1$ y que $\log(n+1) \rightarrow +\infty$
            
        \end{enumerate}
    \end{ejercicio}

    \begin{ejercicio}[3 puntos]
        Sea la sucesión $\{a_n\}$ verificando $|a_n-1| \leq \dfrac{1}{n}, \forall n \in \bb{N}$. Se pide:
        \begin{enumerate}
            \item Probar que la serie $\sum\limits_{n\geq 1} (-1)^{n} \frac{(a_n-1)}{n}$ converge absolutamente.

            La pregunta se reduce a probar si $\sum\limits_{n\geq 1}\frac{|a_n-1|}{n}$ converge. Por ``comparación", usando la hipótesis:
            \[
            \dfrac{|a_n-1|}{n} \leq \dfrac{\frac{1}{n}}{n} = \dfrac{1}{n^2}
            \]

            Así, por ser $\sum\limits_{n\geq 1} \frac{1}{n^2}$ convergente $\xRightarrow{(crit. comp.)}$ $\sum\limits_{n\geq 1} (-1)^{n} \frac{(a_n-1)}{n}$ convergente.
            
            \item Estudiar la convergencia absoluta y la convergencia de la serie $\sum\limits_{n\geq 1} (-1)^{n} \dfrac{a_n}{n}$

            \begin{itemize}
                \item La convergencia absoluta es ver si $\sum\limits_{n\geq 1} \frac{a_n}{n}$ converge.\\
                Comparación (criterio límite) con la serie $\sum\limits_{n\geq 1} \frac{1}{n}$
                \[
                    \dfrac{\frac{a_n}{n}}{\frac{1}{n}} = a_n \longrightarrow 1\in \bb{R}^+
                \]
                Así, por el criterio límite de comparación,
                \[
                    \sum\limits_{n\geq 1} \frac{1}{n} \text{ no converge } \Longrightarrow \sum\limits_{n\geq 1} \frac{a_n}{n} \text{ no converge}
                \]
                Luego $\sum\limits_{n\geq 1} (-1)^{n} \dfrac{a_n}{n}$ no converge absolutamente

                \item ¿converge?
                \[
                    \sum\limits_{n\geq 1} (-1)^{n} \dfrac{a_n}{n} = \sum\limits_{n\geq 1} (-1)^{n} \dfrac{(a_n-1)}{n} + \sum\limits_{n\geq 1} (-1)^{n} \dfrac{1}{n}
                \]

                Por el apartado anterior tenemos que $\sum\limits_{n\geq 1} (-1)^{n} \dfrac{(a_n -1)}{n}$ converge absolutamente, luego converge.\\
                Por el criterio de Leibnitz tenemos que $\sum\limits_{n\geq 1} (-1)^{n} \dfrac{1}{n}$ converge.\\
                Al ser $\sum\limits_{n\geq 1} (-1)^{n} \dfrac{a_n}{n}$ suma de series convergentes tenemos que converge.
                
            \end{itemize}
        \end{enumerate}
    \end{ejercicio}

    \begin{ejercicio}[2.5 puntos]
        Sea $f:[0,1]\cup \{2\} \rightarrow \bb{R}$ la función definida por:
        \[
            f(x)=
            \left \{
            \begin{array}{c l}
                 a & \text{(si $x=0$)} \\ \\
                 \dfrac{x^2}{1+x^2} & \text{(si $0 < x \leq 1$)} \\ \\
                 b & \text{(si $x=2$)} \\
            \end{array}
            \right .
        \]

        \begin{enumerate}
            \item (1 punto) ¿Para qué valores de \textit{a} y \textit{b} es $f$ continua? 

            \begin{itemize}
                \item Continuidad en el punto $0\in A$
                \[
                \left.
                \begin{array}{l}
                    f(0)=a \\
                    \lim\limits_{x\to 0}f(x) = \lim\limits_{x\to 0^+}f(x) = 0
                \end{array}
                \right\} f \text{ continua en 0 } \Longleftrightarrow a=0
                \]

                \item  $f$ continua en cualquier punto de $]0,1]$ (por el carácter local de la continuidad)
                \item Continuidad en el punto $2\in A$ \\
                2 es un \underline{punto aislado} del dominio y, por tanto, $f$ es continua en 2 (independientemente del valor de \textit{b})
            \end{itemize}

            Conclusión:
            \[
                f \text{ continua }\Longleftrightarrow a=0
            \]
            
            \item (1 punto) ¿Para qué valores de \textit{a} y \textit{b} es $f$ monótona? 

            Observemos que $\frac{x^2}{1+x^2} = \frac{1+ x^2}{1+x^2} - \frac{1}{1+x^2} = 1 - \frac{1}{1+x^2}$ es creciente en $]0,1]$.\\
            
            Así, $f_{|_{]0,1]}}$ es creciente. Para que \underline{f creciente} en $A$ debe ocurrir que:
            \[
            \left.
            \begin{array}{l c l c c c}
                a=f(0) \leq f(x) \text{ } \forall x \in ]0,1] & \Longleftrightarrow & a \leq 0 &&& \text{(En ningún caso podrá} \\
                b=f(2) \geq f(1)=\frac{1}{2} &  \Longleftrightarrow & b \geq \frac{1}{2} &&& \text{ser $f$ decreciente)}
            \end{array}
            \right.
            \]
            \item (0.5 puntos) Calcular la imagen de $f$. 
            \[
            Im(f) = \{a,b\} \cup f(]0,1])
            \]
            Por el T.V.I. y la monotonía de $f_{|_{]0,1]}}$ sabemos que $f(]0,1])=]0,\frac{1}{2}]$, luego
            \[
                Im(f) = \{a,b\} \cup \left]0,\frac{1}{2}\right]
            \]
        \end{enumerate}
    \end{ejercicio}



     
\end{document}
