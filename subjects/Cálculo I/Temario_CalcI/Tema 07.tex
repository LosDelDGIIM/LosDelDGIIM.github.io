\chapter{Sucesiones divergentes}\label{chp:Tema7}

%########################################################################################################
% Definiciones básicas.
%########################################################################################################

\section{Definiciones básicas}
\begin{definicion}
    Se dice que una sucesión de números reales $\{x_n\}$ \textbf{diverge positivamente} si
    \begin{equation*}
        \forall K \in \mathbb{R} ~ \exists m \in \mathbb{N} ~:~ n \in \mathbb{N}, ~ n \geq m \implies x_n > K.
    \end{equation*}
    
    Escribiremos $\{x_n\} \longrightarrow + \infty$.
\end{definicion}

En la definición se exige que todos los números reales cumplan una determinada condición. Es claro que si la condición anterior se cumple para un cierto $K \in \mathbb{R}$, también se cumple para cualquier número real menor que él. Por tanto, para que una sucesión de números reales diverja positivamente, basta con que dicha condición sea cierta para un conjunto de números reales no mayorado, como $\mathbb{R}^+$. Por ello, es usual poner en la definición ``$\forall K \in \mathbb{R}^+$''.

\begin{definicion}
    Se dice que una sucesión de números reales $\{x_n\}$ \textbf{diverge negativamente} si
    \begin{equation*}
        \forall K \in \mathbb{R} ~ \exists m \in \mathbb{N} ~:~ n \in \mathbb{N}, ~ n \geq m \implies x_n < K.
    \end{equation*}
    Escribiremos $\{x_n\} \longrightarrow - \infty$.
\end{definicion}

Por un razonamiento similar al de la aclaración anterior, basta con exigir que la condición sea cierta para un conjunto de números reales no minorado, como como $\mathbb{R}^-$. Por ello, es usual poner en la definición ``$\forall K \in \mathbb{R}^-$''.

\begin{definicion}
    Diremos que una sucesión de números reales $\{x_n\}$ \textbf{diverge} cuando diverja positivamente o negativamente, escribiremos $\{|x_n|\} \longrightarrow + \infty$.
\end{definicion}

Es obvio que toda sucesión de números reales divergente es no acotada, pero merece la pena destacar que existen sucesiones de números reales que, \textbf{a pesar de no estar acotadas, no son divergentes}.\\

Aunque la gran mayoría de resultados de este tema se proponen como ejercicio, vamos a ver un resultado
que nos será de utilidad más adelante.
\begin{teo}\label{teo:7.1.4}
    Sean $\{x_n\}$ e $\{y_n\}$ dos sucesiones de números reales. Supongamos que $\{x_n\}$ diverge positivamente y que $\{y_n\}$ verifica la siguiente condición:
    \begin{equation*}
        \exists \varepsilon > 0, ~ \exists p \in \mathbb{N} ~:~ n \geq p \Longrightarrow y_n > \varepsilon
    \end{equation*}
    Entonces $\{x_n y_n\} \longrightarrow + \infty$.
\end{teo}
\begin{proof}
    Dado $K > 0$, por ser $\{x_n\} \longrightarrow + \infty$, podemos encontrar un natural $q$ tal que para $n \geq q$ se tenga $x_n > \frac{K}{\varepsilon}$. Tomando $m = \max \{p,q\}$ tenemos
    \begin{equation*}
        n \geq m \Longrightarrow x_n y_n > \frac{K}{\varepsilon} \varepsilon = K
    \end{equation*}
    lo que prueba que $\{x_n y_n\} \longrightarrow + \infty$, como queríamos.
\end{proof}


%########################################################################################################
% Ejercicios de Sucesiones divergentes.
%########################################################################################################

\section{Ejercicios}

\textbf{Probar los siguientes resultados:}
\begin{ejercicio}
    Si $\{x_n\} \longrightarrow +\infty$, entonces $\{x_n\}$ está minorada.
\end{ejercicio}

\begin{ejercicio}
    Si $\{x_n\} \longrightarrow -\infty$, entonces $\{x_n\}$ está mayorada.
\end{ejercicio}

\begin{ejercicio}
    Probar que
    \begin{equation*}
        \{x_n\} \longrightarrow + \infty \Longleftrightarrow \{-x_n\} \longrightarrow -\infty.
    \end{equation*}
\end{ejercicio}

\begin{ejercicio}\label{ej:7.2.4}
    Si $\{x_n\} \longrightarrow + \infty$ e $\{y_n\}$ está minorada, $\{x_n+y_n\} \longrightarrow + \infty$.
\end{ejercicio}

\begin{ejercicio}
    Si $\{x_n\} \longrightarrow - \infty$ e $\{y_n\}$ está mayorada, $\{x_n+y_n\} \longrightarrow - \infty$.
\end{ejercicio}

\begin{ejercicio}
    Si $\{x_n\}$ es divergente y $\{y_n\}$ está acotada, entonces $\{x_n + y_n\}$ diverge.
\end{ejercicio}

\begin{ejercicio}\label{ej:7.2.7}
    Si $\{x_n\}$ e $\{y_n\}$ divergen ambas positivamente o ambas negativamente, entonces $\{x_n y_n\} \longrightarrow +\infty$.
\end{ejercicio}

\begin{ejercicio}\label{ej:7.2.8}
    Si $\{x_n\} \longrightarrow +\infty$ e $\{y_n\} \longrightarrow -\infty$, entonces $\{x_n y_n\} \longrightarrow -\infty$.
\end{ejercicio}

\begin{ejercicio}\label{ej:7.2.9}
    Si $\{x_n\} \longrightarrow +\infty$ e $\{y_n\} \longrightarrow y > 0$, entonces $\{x_n y_n\} \longrightarrow +\infty$.
\end{ejercicio}

\begin{ejercicio}
    Si $x_n \neq 0$, $\forall n \in \mathbb{N}$, entonces
    \begin{equation*}
        \{x_n\} ~ \text{diverge} \Longleftrightarrow \left\{\frac{1}{x_n}\right\} \longrightarrow 0.
    \end{equation*}
\end{ejercicio}

\begin{ejercicio}\label{ej:7.2.11}
    Sea $\{x_n\}$ una sucesión de números reales no nulos. Entonces, $\{x_n\}$ converge a cero si, y sólo si, la sucesión $\left\{ \frac{1}{|x_n|} \right\}$ diverge positivamente.
\end{ejercicio}