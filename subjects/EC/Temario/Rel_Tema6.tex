\setcounter{section}{0}
\begin{ejercicio}
    Suponga una jerarquía de memoria de dos niveles, \( M_1 \) y \( M_2 \), con capacidades \( s_1 \) y \( s_2 \) bits, tiempo de acceso a los circuitos de cada nivel \( t_1 \) y \( t_2 \), y coste por bit \( c_1 \) y \( c_2 \), respectivamente. Obtenga:
    \begin{itemize}
        \item[a)] El costo por bit de toda la memoria.\\ \\
            Viene dado por:
            \begin{equation*}
                c(n) = \frac{\sum_{i=1}^{n} c_i \cdot s_i}{\sum_{i=1}^{n} s_i} + c_0
            \end{equation*}
            En este caso, y suponiendo que el coste de interconexión es despreciable, se tiene que:
            \begin{equation*}
                c(2) = \frac{c_1 \cdot s_1 + c_2 \cdot s_2}{s_1 + s_2}
            \end{equation*}
        \item[b)] El tiempo medio de acceso.
            \\ \\
            Dado que no se proporciona la frecuencia de acceso con exito a cada nivel, calculamos el tiempo medio de acceso como:
            \begin{equation*}
                \overline{T} = t_1 \cdot A_1 + (t_1 + t_2) \cdot (1 - A_1)
            \end{equation*}

        \item[c)] La eficacia de la jerarquía de memoria en términos de la razón de velocidad del nivel 2 respecto al 1. \\ \\
            En este caso:
            \begin{equation*}
                E = \frac{t_1}{\overline{T}}
            \end{equation*}
    \end{itemize}
\end{ejercicio}

\begin{ejercicio}
    Considere una jerarquía de memoria de dos niveles, \( M_1 \) (más cercano al procesador) y \( M_2 \), con tiempos de acceso \( t_1 \) y \( t_2 \), costes por byte \( c_1 \) y \( c_2 \), y tamaños \( s_1 \) y \( s_2 \), respectivamente. La razón de aciertos de \( M_1 \) es \( A_1 = 0.9 \).
    \begin{itemize}
        \item[a)] Escriba una fórmula mostrando el coste total \( C_{tot} \) del sistema de memoria, sin tener en cuenta el coste de conexión de los dos niveles de la jerarquía.
            \\ \\
            Se puede calcular como:
            \begin{equation*}
                C_{tot} = \frac{c_1 \cdot s_1 + c_2 \cdot s_2}{s_1 + s_2}
            \end{equation*}
        \item[b)] Escriba una fórmula que modele el tiempo de acceso efectivo \( T_{ef} \) del sistema de memoria, teniendo en cuenta que el acceso a \( M_1 \) y \( M_2 \) no se puede llevar a cabo simultáneamente.
            \\ \\
            Viene modelado por la fórmula:
            \begin{equation*}
                T_{ef} = 0.9 \cdot t_1 + 0.1 \cdot (t_1 + t_2)
            \end{equation*}
        \item[c)] Suponga que \( t_1 = 20 \, \text{ns} \), \( t_2 \) es desconocido, \( s_1 = 512 \, \text{KB} \), \( s_2 \) es desconocido, \( c_1 = 0.00015 \, \text{€ / byte} \), \( c_2 = 0.0000006 \, \text{€ / byte} \).
        \begin{itemize}
            \item[i)] ¿Cuántos GB de \( M_2 \) (\( s_2 = ? \)) se pueden adquirir sin exceder un presupuesto total aproximado de unos 400 €, si se desprecia el coste de la conexión de los dos niveles de la jerarquía?
                \\ \\
                Con un presupuesto de 400 €, se tiene que:
                \begin{equation*}
                    C_{tot} = c_1 \cdot s_1 + c_2 \cdot s_2 \Rightarrow 400 = 0.00015 \cdot 512 \cdot 10^3 B+ 6 \cdot 10^{-7}\cdot s_2 \cdot 10^9 \Rightarrow s_2 = 0.539
                \end{equation*}
            \item[ii)] ¿De qué velocidad hay que comprar la memoria \( M_2 \) (\( t_2 = ? \)) para conseguir un tiempo de acceso medio \( T_{ef} = 30 \, \text{ns} \) en el sistema de memoria completo bajo la suposición de tasa de aciertos anterior?
                \\ \\
                Usando la fórmula de \( T_{ef} \), se tiene que:
                \begin{equation*}
                    30ns = 0.9 \cdot 20ns + 0.1 \cdot (20ns + t_2) \Rightarrow t_2 = 100ns
                \end{equation*}
        \end{itemize}
    \end{itemize}
\end{ejercicio}

\begin{ejercicio}
    Teniendo en cuenta que la gráfica de la Figura 1 pretende estar relacionada con el pseudocódigo siguiente, etiquete los ejes de coordenadas, 
    rodee con un círculo los puntos que le parezcan relacionados, etiquételos según la nomenclatura del pseudocódigo e indicando qué tipo de localidad se para cada caso, y deduzca el 
    valor de \( N \).
    \begin{itemize}
        \item Para \( i = 1 \dots N \), \( a[i] = i \times i \)
        \item Para \( j = 1 \dots N \), \( b[j] = j \times j \times j \)
        \item Para \( k = 1 \dots N \), \( c[k] = a[k] + b[k] \)
    \end{itemize}
    Veamos directamente la solución, (el problema original viene únicamente con los puntos)
    \begin{center}
        \includesvg[width=0.8\textwidth]{Imagenes/Relacion/ejer3}
    \end{center}
    En este caso, podemos ver que el eje de ordenadas corresponde con las direcciones de memoria de cada elemento,
    en el eje de abscisas se representa el tiempo. La parte señalada como A
    corresponde con lso accesos a los datos, para saberlo, notemos que los accesos a lo largo del tiempo a los iteradores
    tienen que estar en la misma posición, de ahí se deducen i,j,k. Para distinguir A de B además notamos que a la derecha
    se producen accesos a posiciones anteriores muy seguidos, lo que se realiza en el pseudocódigo para c, a partir de esto,
    tanto a como b son fáciles de ver ya que en memoria están muy juntos. Viendo el número de iteraciones para cada uno de ellos
    se deduce que $N = 3$. En este caso se aprovecha la localidad temporal siempre y la espacial solo para a y b, que se acceden secuencialmente.
    \\ \\
    Respecto a la zona etiquetada como B, se puede deducir que se trata de las instrucciones, además podemos decir que tiene localidad espacial y
    temporal en la mayoría de su duración.
\end{ejercicio}

\begin{ejercicio}
    Un computador con un bus de datos de 32 bits usa circuitos de memoria RAM dinámica de \( 1M \times 4 \).
    \begin{itemize}
        \item[a)] ¿Cuál es la memoria más pequeña (en bytes) que puede tener ese computador?
            \\ \\
            Como se tienen 4 bytes por dirección y como mínimo hay 32 direcciones ya que se usan 32 bits para el bus de datos, 
            la más pequeña sería de 128B.
        \item[b)] Dibuje un esquema de la misma, detallando las líneas de direcciones y de datos.
    \end{itemize}
\end{ejercicio}

\begin{ejercicio}
    Una placa madre de un IBM PC basada en el microprocesador Intel 8088 dispone de 256 KB de DRAM con 1 bit de paridad, constituida por circuitos integrados 4164 (Figura 2). Las señales que conectan esta memoria con los buses y otros circuitos (circuitería de refresco de memoria y de generación/comprobación de paridad) se muestran en la Figura 3 (PAR\_IN y PAR\_OUT son las señales de paridad).
    \begin{itemize}
        \item[a)] Dibuje el sistema de memoria y su conexionado con las señales de la Figura 3. No incluya ninguna circuitería de refresco.
        \item[b)] Especifique en el dibujo del apartado (a) o en una tabla las direcciones de memoria asociadas a cada chip o grupo de chips. Indique asimismo en qué chips y en qué posición dentro de esos chips se encuentra almacenado el byte de memoria cuya dirección es 20001h.
    \end{itemize}
\end{ejercicio}

\begin{ejercicio}
    Una placa madre de un PC basada en el microprocesador Intel 8088 disponía de 512 KB de DRAM más bit de paridad, constituida por circuitos integrados 41256 (Figura 4). Las señales que conectaban esta memoria con los buses y otros circuitos (circuitería de refresco de memoria y de generación/comprobación de paridad) se muestran en la Figura 5 (PAR\_IN y PAR\_OUT son las señales de paridad).
    \begin{itemize}
        \item[a)] Dibuje el sistema de memoria completo (todos los chips) y su conexionado con las señales de la Figura 5. No hay que incluir la circuitería de refresco.
        \item[b)] Especifique en el dibujo el rango de direcciones de memoria asociado a cada chip.
    \end{itemize}
\end{ejercicio}

\begin{ejercicio}
    Suponga que dispone de circuitos integrados de memoria SRAM de \( 32K \times 8 \) (Figura 6) y desea construir con ellos una memoria de 256 K palabras de 32 bits, organizada con entrelazado de orden inferior con bancos de 64 K palabras.
    \begin{itemize}
        \item[a)] Dibuje el esquema del sistema de memoria, detallando las líneas de dirección.
        \item[b)] Indique en el dibujo en qué chips y en qué posición dentro de esos chips se encuentra almacenada la palabra cuya dirección es 20001h (el direccionamiento de memoria se realiza a nivel de palabras de 32 bits).
    \end{itemize}
\end{ejercicio}

\begin{ejercicio}
    La placa madre de un procesador de 32 bits (486) dispone de 4 MB de memoria DRAM distribuida en módulos SIMM de 30 contactos sin paridad de \( 1M \times 8 \). Cada SIMM contiene 8 chips de DRAM. Aparte de 9 contactos para alimentación, tierra y no conectados, cada SIMM tiene los siguientes contactos:
    \begin{itemize}
        \item A0-A9: Entradas de dirección.
        \item RAS: Row Address Strobe.
        \item CAS: Column Address Strobe.
        \item WE: Write Enable.
        \item DQ1-DQ8: Entradas/salidas de datos.
    \end{itemize}
    \begin{itemize}
        \item[a)] ¿Qué tamaño tiene cada chip de DRAM?
        \item[b)] Dibuje un esquema de uno de los SIMM con las conexiones entre sus contactos y las patillas de los chips de DRAM.
        \item[c)] Dibuje un esquema con todos los SIMM (sin dibujar el interior de cada uno) y su conexión con los buses de direcciones y datos del sistema.
    \end{itemize}
\end{ejercicio}

\begin{ejercicio}
    Un computador dispone de 16 MB de memoria principal entrelazada de orden inferior y acceso simultáneo (Tipo S), constituida por módulos de 1 M Byte. Suponga que el procesador desea acceder a los bytes cuyas direcciones son:
    \begin{itemize}
        \item AB0010h, AB0021h, 0010AAh, 227AAAh, 0101AAh, 01016Ah, 010163h
    \end{itemize}
    \begin{itemize}
        \item[a)] ¿En qué módulo se encuentra cada uno de esos bytes?
            \\ \\ En primer lugar, comenzamos notando que al tratarse de entrelazado inferior, los bits inferiores seleccionan el módulo de
            memoria, además el acceso simultáneo indica que se hace \textit{latch} a la salida de datos, es decir, se muestra la misma
            dirección a todos los módulos en cada acceso. Dicho esto, cada módulo es de $1MB = 2^{20}B$ y hay 16 módulos, cada dirección tiene
            24 bits, por lo que, restándo los 4 bits para el módulo tenemos 20 bits para la dirección y 4 para el módulo:
            \begin{itemize}
                \item AB0010h: Módulo 0
                \item AB0021h: Módulo 1
                \item 0010AAh: Módulo 10
                \item 227AAAh: Módulo 10
                \item 0101AAh: Módulo 10
                \item 01016Ah: Módulo 10
                \item 010163h: Módulo 3
            \end{itemize}
        \item[b)] ¿Cuántos accesos simultáneos a memoria se necesitan como mínimo?
            \\ \\
            6 accesos simultáneos ya que únicamente los dos últimos accesos coinciden en la dirección.
    \end{itemize}
\end{ejercicio}

\begin{ejercicio}
    Se tiene un computador con una memoria principal entrelazada con esquema de entrelazado de orden inferior con latch en las entradas (acceso C) de 256 K palabras dividida en 4 módulos.
    \begin{itemize}
        \item[a)] ¿Cuántos bytes hay en cada módulo?
        \item[b)] ¿Cuántos bits necesita cada módulo para acceder a \( 128 \) palabras?
    \end{itemize}
\end{ejercicio}

\begin{ejercicio}
    En un computador que dispone de caché, las instrucciones necesitan 8.5 ciclos de reloj para ejecutarse en caso de que no haya falta. Si la tasa de faltas de la caché es del 11\%, cada falta da lugar a un incremento de tiempo de 6 ciclos de reloj, y por término medio cada instrucción necesita 3 accesos a memoria. ¿Cuál debe ser el número medio de ciclos de una instrucción (\(CPI\)) en el computador sin caché para que sea beneficioso utilizarla?
\end{ejercicio}

\begin{ejercicio}
    Suponga que el tiempo de ejecución de un programa es directamente proporcional al tiempo de acceso a instrucciones, y que el acceso a una instrucción en la caché es ocho veces más rápido que el acceso a una instrucción en la memoria principal. Suponga que una instrucción pedida por el procesador se encuentra en la caché con una probabilidad de \(0.9\), y suponga también que si una instrucción no se encuentra en la caché primero debe ser captada desde la memoria principal a la caché y, a continuación, captada de la caché para ser ejecutada.
    \begin{itemize}
        \item[a)] Calcule la eficiencia como la relación entre el tiempo de ejecución de un programa sin la caché y el tiempo de ejecución con la caché.
        \item[b)] Siempre que se dobla el tamaño de la caché, la probabilidad de no encontrar en ella una instrucción buscada se reduce a la mitad. Dibuje una gráfica que represente la eficiencia, tal como se define en el apartado a), en función del tamaño de la caché, indicando cuál es la eficiencia mínima (tamaño \(0\)) y máxima (tamaño \(\to \infty\)).
    \end{itemize}
\end{ejercicio}

\begin{ejercicio}
    El tiempo de acceso a la memoria caché de un sistema es de \(50 \, \text{ns}\), y el tiempo de acceso a la memoria principal es de \(500 \, \text{ns}\). Se estima que el \(80\%\) de las peticiones de acceso a la memoria son para lectura y el resto para escritura. La razón de aciertos es de \(0.9\).
    \begin{itemize}
        \item[a)] Determinar el tiempo de acceso promedio considerando sólo los ciclos de lectura.
        \item[b)] Determinar el tiempo de acceso promedio considerando también los ciclos de escritura.
    \end{itemize}
\end{ejercicio}

\begin{ejercicio}
    Un computador con una memoria principal de \(1 \, \text{M}\) palabra dispone de una memoria caché de correspondencia directa de \(4 \, \text{K}\) palabras con marcos de bloque de \(16 \, \text{palabras}\).
    \begin{itemize}
        \item[a)] Indique el número de faltas de caché que se producen si el procesador genera la secuencia de accesos a memoria:
            \[ \texttt{ABC13h, CDC14h, ABC1Fh, AB305h, CAC13h, CDC1Ah, CA00Fh, ABC10h} \]
        \item[b)] Para esa misma secuencia, indique los marcos de bloque de caché en los que se carga cada posición de memoria principal a la que se pretende acceder.
    \end{itemize}
\end{ejercicio}

\begin{ejercicio}
    Un computador direccionable por bytes tiene una pequeña caché de datos capaz de almacenar \(8\) palabras de \(32 \, \text{bits}\). Cada bloque de caché consiste en una palabra de \(32 \, \text{bits}\). Al ejecutar un determinado programa, el procesador lee datos de la siguiente secuencia de direcciones (en hexadecimal):
    \[ \texttt{200, 204, 208, 20C, 2F4, 2F0, 200, 204, 218, 21C, 24C, 2F4} \]
    Este patrón se repite cuatro veces en total.
    \begin{itemize}
        \item[a)] Muestre los contenidos de la caché al final de cada pasada a ese bucle si se usa correspondencia directa. Calcule la tasa de aciertos para este ejemplo. Asuma que la caché está inicialmente vacía.
        \item[b)] Repita la parte a) para una correspondencia completamente asociativa que usa el algoritmo de reemplazo LRU.
        \item[c)] Repita la parte a) para una caché asociativa por conjuntos de cuatro vías.
    \end{itemize}
\end{ejercicio}

\begin{ejercicio}
    Un computador tiene una caché de \(128 \, \text{bytes}\). Usa correspondencia asociativa por conjuntos, de cuatro vías, con \(8 \, \text{bytes}\) en cada bloque. El tamaño de la dirección física es \(32 \, \text{bits}\), y la unidad mínima direccionable es \(1 \, \text{byte}\).
    \begin{itemize}
        \item[a)] Dibuje un diagrama que muestre la organización de la caché e indique cómo se relaciona una dirección física con la caché (campos).
        \item[b)] ¿En qué marcos de bloque de la caché puede residir el byte de memoria principal cuya dirección es \texttt{000010AFh}?
        \item[c)] Si las direcciones \texttt{000010AFh} y \texttt{FFFF7AXYh} pueden asignarse simultáneamente al mismo conjunto de caché, ¿cuántos posibles valores distintos puede tener \texttt{XY} en la segunda dirección?
    \end{itemize}
\end{ejercicio}

\begin{ejercicio}
    Suponga una memoria caché de \(4 \, \text{K}\) palabras asociativa por conjuntos, con marcos de bloque de \(64 \, \text{palabras}\) y \(8\) conjuntos, y una memoria principal de \(8 \, \text{M}\) palabras.
    \begin{itemize}
        \item[a)] ¿En qué posición de caché se situaría la palabra de memoria principal cuya dirección es \texttt{3A0A39h}?
        \item[b)] ¿Qué direcciones de memoria principal no pueden encontrarse en caché al mismo tiempo que esa dirección?
    \end{itemize}
\end{ejercicio}

\begin{ejercicio}
    Un sistema de cómputo tiene una memoria principal de \(32 \, \text{K}\) palabras de \(16 \, \text{bits}\). También tiene una memoria caché de \(4 \, \text{K}\) palabras organizadas de forma asociativa por conjuntos con \(4\) bloques por conjunto y \(64 \, \text{palabras}\) por bloque.
    \begin{itemize}
        \item[a)] Calcule el número de bits de cada uno de los campos: Etiqueta, Conjunto y Palabra.
        \item[b)] Si el tiempo de acceso a la memoria principal es \(10\) veces mayor que el de la memoria caché, y la razón de aciertos de la caché es de \(0.9\), determinar la eficiencia de acceso (razón entre el tiempo de acceso a la memoria caché y el tiempo medio de acceso).
    \end{itemize}
\end{ejercicio}

\begin{ejercicio}
    Un ordenador dispone de \(32 \, \text{KB}\) de memoria principal direccionable por bytes y una memoria caché completamente asociativa de \(4 \, \text{KB}\). El tamaño del bloque de la memoria caché es de \(8 \, \text{palabras}\) de \(32 \, \text{bits}\). El tiempo de acceso a la memoria principal es \(10\) veces mayor que el de la memoria caché.
    \begin{itemize}
        \item[a)] ¿Cuántos comparadores hardware se necesitan?
        \item[b)] ¿Cuál es el tamaño del campo identificador?
        \item[c)] Si se utiliza el esquema de sustitución directa, ¿cuál sería el tamaño del campo identificador?
        \item[d)] Suponer que la eficiencia de acceso se define como la razón entre el tiempo de acceso con memoria caché y el tiempo de acceso sin memoria caché. Determinar la eficiencia de acceso suponiendo que la razón de aciertos de la memoria caché es \(0.9\).
        \item[e)] Si el tiempo de acceso a la memoria caché es de \(200 \, \text{ns}\), ¿cuál será la razón de aciertos necesaria para lograr un tiempo de acceso de \(500 \, \text{ns}\)?
    \end{itemize}
\end{ejercicio}

% Ejercicio 20
\begin{ejercicio}
    Los parámetros que definen la memoria de un computador son los siguientes:
    \begin{itemize}
        \item Tamaño de la memoria principal: \(32 \, \text{K}\) palabras.
        \item Tamaño de la memoria caché: \(4 \, \text{K}\) palabras.
        \item Tamaño de bloque: \(64 \, \text{palabras}\).
    \end{itemize}
    Determine el tamaño de cada campo de una dirección de memoria y explique brevemente cómo se obtiene, para las siguientes políticas de colocación:
    \begin{itemize}
        \item[a)] Totalmente asociativa.
            \\ \\ En este caso se tienen $m$ bits para la dirección de la memoria principal y el resto para los bytes, como en este caso hay $2^6$ palabras se 
            necesitan 6 bits para el campo $b$ y como para la memoria principal se necesitan $2^{15}$, es decir 15 bits, el campo de etiqueta tiene 9 bits.
        \item[b)] Por correspondencia directa.
            \\ \\ Como antes, tenemos 6 bits para el campo b y 15 bits en total, la caché ocupa 12 bits, por lo que tenemos $2^6$ lineas que direccionaremos con 6 bits
            luego el campo de etiqueta ocupará 3 bits.
        \item[c)] Asociativa por conjuntos con \(16\) bloques por conjunto.
            \\ \\
            16 Bloques necesitan 4 bits, es decir, mantenemos 6 bits para el campo $b$, por tanto tenemos 15 - 6 - 4 bits para el campo de etiqueta
    \end{itemize}
\end{ejercicio}

\begin{ejercicio}
    Los parámetros que definen la memoria de un computador son los siguientes:
    \begin{itemize}
        \item Tamaño de la memoria principal: \(4 \, \text{G}\) bytes.
        \item Tamaño de la memoria caché: \(1 \, \text{MB}\).
        \item Tamaño de bloque: \(256 \, \text{bytes}\).
    \end{itemize}
    Determine el tamaño de cada campo de una dirección de memoria desde el punto de vista de la caché, para las siguientes políticas de colocación:
    \begin{itemize}
        \item[a)] Totalmente asociativa.
            \\ \\
            Como tenemos 32 bits para la dirección principal y cada bloque son 8 bits, tenemos 32 - 8 = 24 bits para la etiqueta.
        \item[b)] Por correspondencia directa.
            \\ \\
            Sabemos que para el offset tenemos 8 bits, la caché ocupa $2^{20}$ bytes, luego hay $2^{12}$ lineas de caché y se necesitan 12 bits para direccionarlas,
            como resultado nos quedan 30 - 8 - 12 = 10 bits para la etiqueta

        \item[c)] Asociativa por conjuntos con \(4\) bloques por conjunto.
            \\ \\
            Se utilizan 2 bits para el conjunto, además 8 bits para el offset, de donde se tienen 20 - 8 - 2 = 10 bits para la etiqueta.
    \end{itemize}
\end{ejercicio}

\begin{ejercicio}
    Los parámetros que definen la memoria de un computador son los siguientes:
    \begin{itemize}
        \item Tamaño de la memoria principal: \(4 \, \text{GB}\).
        \item Tamaño de la memoria caché: \(16 \, \text{MB}\).
        \item Tamaño de bloque: \(256 \, \text{bytes}\).
    \end{itemize}
    Determine el tamaño de cada campo de una dirección de memoria desde el punto de vista de la caché, para las siguientes políticas de colocación:
    \begin{itemize}
        \item[a)] Totalmente asociativa.
            \\ \\
            Como tenemos 32 bits para la dirección principal y cada bloque son 8 bits, tenemos 32 - 8 = 24 bits para la etiqueta.
        \item[b)] Directa.
            \\ \\
            Sabemos que para el offset tenemos 8 bits, como la caché ocupa $2^{24}B$ tenemos 16 bits para las lineas, como la memoria principal tiene direcciones de 32 bits,
            se tienen 32 - 8 - 16 = 8 bits para el campo de etiqueta.
        \item[c)] Asociativa por conjuntos, de \(8\) vías.
            \\ \\
            Necesitamos 3 bits para las vias, además se tienen 8 bits para el bloque de datos, es decir, 13 bits para conjunto, si tomamaos ahora los 32 bits de la memoria principal y le restamos el de conjunto
            y el de los bloques de datos, nos quedan 11 bits para la etiqueta.
        \item[d)] Por sectores con \(16\) bloques por sector.
    \end{itemize}
\end{ejercicio}

\begin{ejercicio}
    Una memoria caché asociativa por conjuntos consiste en un total de \(64\) bloques divididos en conjuntos de \(4\) bloques. La memoria principal contiene \(4096\) bloques. Cada bloque contiene \(128\) palabras.
    \begin{itemize}
        \item[a)] Indicar el número de bits que hay en una dirección de memoria principal.
        \item[b)] Indicar el número de bits que hay en cada uno de los campos Marca, Conjunto y Palabra.
    \end{itemize}
\end{ejercicio}

\begin{ejercicio}
    Un ordenador dispone de \(32 \, \text{K}\) palabras de memoria principal y una memoria caché con colocación asociativa por conjuntos. El tamaño de un bloque es de \(16 \, \text{palabras}\) y el campo identificador de \(5\) bits. Si la misma memoria caché se sustituye directamente, el campo identificador tendría una longitud de \(3\) bits.
    \begin{itemize}
        \item[a)] Determinar el número de palabras que alberga la memoria caché.
        \item[b)] Determinar el número de bloques que alberga un conjunto de la memoria caché.
    \end{itemize}
\end{ejercicio}

\begin{ejercicio}
    Los parámetros que definen la memoria de un computador son los siguientes:
    \begin{itemize}
        \item Tamaño de la memoria principal: \(8 \, \text{K}\) líneas.
        \item Tamaño de la memoria caché: \(512\) líneas.
        \item Tamaño de la línea: \(8 \, \text{palabras}\).
    \end{itemize}
    Determinar el tamaño de los distintos campos de una dirección en las siguientes condiciones:
    \begin{itemize}
        \item[a)] Colocación completamente asociativa.
        \item[b)] Colocación directa.
        \item[c)] Colocación asociativa por conjuntos con \(16\) líneas por conjunto.
        \item[d)] Colocación por sectores con \(16\) líneas por sector.
    \end{itemize}
\end{ejercicio}

\begin{ejercicio}
    Sea un sistema de memoria con caché, con las siguientes características:
    \begin{itemize}
        \item Tamaño de la memoria principal: \(4 \, \text{GB}\).
        \item Tamaño de la memoria caché: \(256 \, \text{KB}\).
        \item Tamaño de palabra (anchura bus de datos): \(32 \, \text{bits}\).
        \item Correspondencia: Asociativa por conjuntos.
        \item Número de bloques / líneas por conjunto (número de vías): \(4\).
        \item Número de palabras por bloque / línea: \(16\).
    \end{itemize}
    Dibuje un esquema detallado de la memoria caché y su conexión con la CPU para una de las dos alternativas:
    \begin{itemize}
        \item[a)] Utilizar \(1024\) memorias asociativas de \(4 \times 16\), un decodificador de \(10\) a \(1024\), \(1024\) codificadores de \(4\) a \(2\), \(1024\) puertas OR de \(4\) entradas y \(4096\) circuitos SRAM de \(64 \times 8\).
        \item[b)] Utilizar \(8\) circuitos SRAM de \(1 \, \text{K} \times 8\), \(4\) comparadores de dos entradas de \(16 \, \text{bits}\) y \(16\) circuitos SRAM de \(16 \, \text{K} \times 8\).
    \end{itemize}
\end{ejercicio}

