\documentclass[12pt]{article}

% Idioma y codificación
\usepackage[spanish, es-tabla, es-notilde]{babel}       %es-tabla para que se titule "Tabla"
\usepackage[utf8]{inputenc}

% Márgenes
\usepackage[a4paper,top=3cm,bottom=2.5cm,left=3cm,right=3cm]{geometry}

% Comentarios de bloque
\usepackage{verbatim}

% Paquetes de links
\usepackage[hidelinks]{hyperref}    % Permite enlaces
\usepackage{url}                    % redirecciona a la web

% Más opciones para enumeraciones
\usepackage{enumitem}

% Personalizar la portada
\usepackage{titling}

% Paquetes de tablas
\usepackage{multirow}

% Para añadir el símbolo de euro
\usepackage{eurosym}


%------------------------------------------------------------------------

%Paquetes de figuras
\usepackage{caption}
\usepackage{subcaption} % Figuras al lado de otras
\usepackage{float}      % Poner figuras en el sitio indicado H.


% Paquetes de imágenes
\usepackage{graphicx}       % Paquete para añadir imágenes
\usepackage{transparent}    % Para manejar la opacidad de las figuras

% Paquete para usar colores
\usepackage[dvipsnames, table, xcdraw]{xcolor}
\usepackage{pagecolor}      % Para cambiar el color de la página

% Habilita tamaños de fuente mayores
\usepackage{fix-cm}

% Para los gráficos
\usepackage{tikz}
\usepackage{forest}

% Para poder situar los nodos en los grafos
\usetikzlibrary{positioning}


%------------------------------------------------------------------------

% Paquetes de matemáticas
\usepackage{mathtools, amsfonts, amssymb, mathrsfs}
\usepackage[makeroom]{cancel}     % Simplificar tachando
\usepackage{polynom}    % Divisiones y Ruffini
\usepackage{units} % Para poner fracciones diagonales con \nicefrac

\usepackage{pgfplots}   %Representar funciones
\pgfplotsset{compat=1.18}  % Versión 1.18

\usepackage{tikz-cd}    % Para usar diagramas de composiciones
\usetikzlibrary{calc}   % Para usar cálculo de coordenadas en tikz

%Definición de teoremas, etc.
\usepackage{amsthm}
%\swapnumbers   % Intercambia la posición del texto y de la numeración

\theoremstyle{plain}

\makeatletter
\@ifclassloaded{article}{
  \newtheorem{teo}{Teorema}[section]
}{
  \newtheorem{teo}{Teorema}[chapter]  % Se resetea en cada chapter
}
\makeatother

\newtheorem{coro}{Corolario}[teo]           % Se resetea en cada teorema
\newtheorem{prop}[teo]{Proposición}         % Usa el mismo contador que teorema
\newtheorem{lema}[teo]{Lema}                % Usa el mismo contador que teorema
\newtheorem*{lema*}{Lema}

\theoremstyle{remark}
\newtheorem*{observacion}{Observación}

\theoremstyle{definition}

\makeatletter
\@ifclassloaded{article}{
  \newtheorem{definicion}{Definición} [section]     % Se resetea en cada chapter
}{
  \newtheorem{definicion}{Definición} [chapter]     % Se resetea en cada chapter
}
\makeatother

\newtheorem*{notacion}{Notación}
\newtheorem*{ejemplo}{Ejemplo}
\newtheorem*{ejercicio*}{Ejercicio}             % No numerado
\newtheorem{ejercicio}{Ejercicio} [section]     % Se resetea en cada section


% Modificar el formato de la numeración del teorema "ejercicio"
\renewcommand{\theejercicio}{%
  \ifnum\value{section}=0 % Si no se ha iniciado ninguna sección
    \arabic{ejercicio}% Solo mostrar el número de ejercicio
  \else
    \thesection.\arabic{ejercicio}% Mostrar número de sección y número de ejercicio
  \fi
}


% \renewcommand\qedsymbol{$\blacksquare$}         % Cambiar símbolo QED
%------------------------------------------------------------------------

% Paquetes para encabezados
\usepackage{fancyhdr}
\pagestyle{fancy}
\fancyhf{}

\newcommand{\helv}{ % Modificación tamaño de letra
\fontfamily{}\fontsize{12}{12}\selectfont}
\setlength{\headheight}{15pt} % Amplía el tamaño del índice


%\usepackage{lastpage}   % Referenciar última pag   \pageref{LastPage}
%\fancyfoot[C]{%
%  \begin{minipage}{\textwidth}
%    \centering
%    ~\\
%    \thepage\\
%    \href{https://losdeldgiim.github.io/}{\texttt{\footnotesize losdeldgiim.github.io}}
%  \end{minipage}
%}
\fancyfoot[C]{\thepage}
\fancyfoot[R]{\href{https://losdeldgiim.github.io/}{\texttt{\footnotesize losdeldgiim.github.io}}}

%------------------------------------------------------------------------

% Conseguir que no ponga "Capítulo 1". Sino solo "1."
\makeatletter
\@ifclassloaded{book}{
  \renewcommand{\chaptermark}[1]{\markboth{\thechapter.\ #1}{}} % En el encabezado
    
  \renewcommand{\@makechapterhead}[1]{%
  \vspace*{50\p@}%
  {\parindent \z@ \raggedright \normalfont
    \ifnum \c@secnumdepth >\m@ne
      \huge\bfseries \thechapter.\hspace{1em}\ignorespaces
    \fi
    \interlinepenalty\@M
    \Huge \bfseries #1\par\nobreak
    \vskip 40\p@
  }}
}
\makeatother

%------------------------------------------------------------------------
% Paquetes de cógido
\usepackage{minted}
\renewcommand\listingscaption{Código fuente}

\usepackage{fancyvrb}
% Personaliza el tamaño de los números de línea
\renewcommand{\theFancyVerbLine}{\small\arabic{FancyVerbLine}}

% Estilo para C++
\newminted{cpp}{
    frame=lines,
    framesep=2mm,
    baselinestretch=1.2,
    linenos,
    escapeinside=||
}

% para minted
\definecolor{LightGray}{rgb}{0.95,0.95,0.92}
\setminted{
    linenos=true,
    stepnumber=5,
    numberfirstline=true,
    autogobble,
    breaklines=true,
    breakautoindent=true,
    breaksymbolleft=,
    breaksymbolright=,
    breaksymbolindentleft=0pt,
    breaksymbolindentright=0pt,
    breaksymbolsepleft=0pt,
    breaksymbolsepright=0pt,
    fontsize=\footnotesize,
    bgcolor=LightGray,
    numbersep=10pt
}


\usepackage{listings} % Para incluir código desde un archivo

\renewcommand\lstlistingname{Código Fuente}
\renewcommand\lstlistlistingname{Índice de Códigos Fuente}

% Definir colores
\definecolor{vscodepurple}{rgb}{0.5,0,0.5}
\definecolor{vscodeblue}{rgb}{0,0,0.8}
\definecolor{vscodegreen}{rgb}{0,0.5,0}
\definecolor{vscodegray}{rgb}{0.5,0.5,0.5}
\definecolor{vscodebackground}{rgb}{0.97,0.97,0.97}
\definecolor{vscodelightgray}{rgb}{0.9,0.9,0.9}

% Configuración para el estilo de C similar a VSCode
\lstdefinestyle{vscode_C}{
  backgroundcolor=\color{vscodebackground},
  commentstyle=\color{vscodegreen},
  keywordstyle=\color{vscodeblue},
  numberstyle=\tiny\color{vscodegray},
  stringstyle=\color{vscodepurple},
  basicstyle=\scriptsize\ttfamily,
  breakatwhitespace=false,
  breaklines=true,
  captionpos=b,
  keepspaces=true,
  numbers=left,
  numbersep=5pt,
  showspaces=false,
  showstringspaces=false,
  showtabs=false,
  tabsize=2,
  frame=tb,
  framerule=0pt,
  aboveskip=10pt,
  belowskip=10pt,
  xleftmargin=10pt,
  xrightmargin=10pt,
  framexleftmargin=10pt,
  framexrightmargin=10pt,
  framesep=0pt,
  rulecolor=\color{vscodelightgray},
  backgroundcolor=\color{vscodebackground},
}

%------------------------------------------------------------------------

% Comandos definidos
\newcommand{\bb}[1]{\mathbb{#1}}
\newcommand{\cc}[1]{\mathcal{#1}}

% I prefer the slanted \leq
\let\oldleq\leq % save them in case they're every wanted
\let\oldgeq\geq
\renewcommand{\leq}{\leqslant}
\renewcommand{\geq}{\geqslant}

% Si y solo si
\newcommand{\sii}{\iff}

% MCD y MCM
\DeclareMathOperator{\mcd}{mcd}
\DeclareMathOperator{\mcm}{mcm}

% Signo
\DeclareMathOperator{\sgn}{sgn}

% Letras griegas
\newcommand{\eps}{\epsilon}
\newcommand{\veps}{\varepsilon}
\newcommand{\lm}{\lambda}

\newcommand{\ol}{\overline}
\newcommand{\ul}{\underline}
\newcommand{\wt}{\widetilde}
\newcommand{\wh}{\widehat}

\let\oldvec\vec
\renewcommand{\vec}{\overrightarrow}

% Derivadas parciales
\newcommand{\del}[2]{\frac{\partial #1}{\partial #2}}
\newcommand{\Del}[3]{\frac{\partial^{#1} #2}{\partial #3^{#1}}}
\newcommand{\deld}[2]{\dfrac{\partial #1}{\partial #2}}
\newcommand{\Deld}[3]{\dfrac{\partial^{#1} #2}{\partial #3^{#1}}}


\newcommand{\AstIg}{\stackrel{(\ast)}{=}}
\newcommand{\Hop}{\stackrel{L'H\hat{o}pital}{=}}

\newcommand{\red}[1]{{\color{red}#1}} % Para integrales, destacar los cambios.

% Método de integración
\newcommand{\MetInt}[2]{
    \left[\begin{array}{c}
        #1 \\ #2
    \end{array}\right]
}

% Declarar aplicaciones
% 1. Nombre aplicación
% 2. Dominio
% 3. Codominio
% 4. Variable
% 5. Imagen de la variable
\newcommand{\Func}[5]{
    \begin{equation*}
        \begin{array}{rrll}
            \displaystyle #1:& \displaystyle  #2 & \longrightarrow & \displaystyle  #3\\
               & \displaystyle  #4 & \longmapsto & \displaystyle  #5
        \end{array}
    \end{equation*}
}

%------------------------------------------------------------------------


\newcommand\digitstyle{\color{blue}}
\makeatletter
\newcommand{\ProcessDigit}[1]
{%
  \ifnum\lst@mode=\lst@Pmode\relax%
   {\digitstyle #1}%
  \else
    #1%
  \fi
}
\makeatother
\lstset{literate=
    {0}{{{\ProcessDigit{0}}}}1
    {1}{{{\ProcessDigit{1}}}}1
    {2}{{{\ProcessDigit{2}}}}1
    {3}{{{\ProcessDigit{3}}}}1
    {4}{{{\ProcessDigit{4}}}}1
    {5}{{{\ProcessDigit{5}}}}1
    {6}{{{\ProcessDigit{6}}}}1
    {7}{{{\ProcessDigit{7}}}}1
    {8}{{{\ProcessDigit{8}}}}1
    {9}{{{\ProcessDigit{9}}}}1
    {<=}{{\(\leq\)}}1,
    morestring=[b]",
    morestring=[b]',
    morecomment=[l]//,
}

\definecolor{codegreen}{rgb}{0,0.6,0}
\definecolor{codegray}{rgb}{0.5,0.5,0.5}
\definecolor{codepurple}{rgb}{0.58,0,0.82}
\definecolor{backcolour}{rgb}{0.95,0.95,0.92}

\definecolor{backcolor}{RGB}{30,30,30}
\definecolor{codeblue}{RGB}{153,255,255}
\definecolor{codepink}{RGB}{200,80,180}
\definecolor{normalblue}{RGB}{12,144,244}

\lstset{
    language=C++,
    morekeywords={int,char,ld,ll,double}
}

\lstdefinestyle{mystyle}{
    backgroundcolor=\color{backcolour},   
    commentstyle=\color{codegreen},
    keywordstyle=\color{magenta},
    numberstyle=\tiny\color{codegray},
    stringstyle=\color{codepurple},
    %identifierstyle=\color{blue},
    basicstyle=\ttfamily\footnotesize,
    breakatwhitespace=false,         
    breaklines=true,                 
    captionpos=b,                    
    keepspaces=true,                 
    numbers=left,                    
    numbersep=5pt,                  
    showspaces=false,                
    showstringspaces=false,
    showtabs=false,                  
    tabsize=2
}
%\usemintedstyle[C++]{default}
\lstset{style=mystyle}


\begin{document}

     % 1. Foto de fondo
    % 2. Título
    % 3. Encabezado Izquierdo
    % 4. Color de fondo
    % 5. Coord x del titulo
    % 6. Coord y del titulo
    % 7. Fecha

    
    % 1. Foto de fondo
% 2. Título
% 3. Encabezado Izquierdo
% 4. Color de fondo
% 5. Coord x del titulo
% 6. Coord y del titulo
% 7. Fecha
% 8. Autor

\newcommand{\portada}[8]{
    \portadaBase{#1}{#2}{#3}{#4}{#5}{#6}{#7}{#8}
    \portadaBook{#1}{#2}{#3}{#4}{#5}{#6}{#7}{#8}
}

\newcommand{\portadaFotoDif}[8]{
    \portadaBaseFotoDif{#1}{#2}{#3}{#4}{#5}{#6}{#7}{#8}
    \portadaBook{#1}{#2}{#3}{#4}{#5}{#6}{#7}{#8}
}

\newcommand{\portadaExamen}[8]{
    \portadaBase{#1}{#2}{#3}{#4}{#5}{#6}{#7}{#8}
    \portadaArticle{#1}{#2}{#3}{#4}{#5}{#6}{#7}{#8}
}

\newcommand{\portadaExamenFotoDif}[8]{
    \portadaBaseFotoDif{#1}{#2}{#3}{#4}{#5}{#6}{#7}{#8}
    \portadaArticle{#1}{#2}{#3}{#4}{#5}{#6}{#7}{#8}
}




\newcommand{\portadaBase}[8]{

    % Tiene la portada principal y la licencia Creative Commons
    
    % 1. Foto de fondo
    % 2. Título
    % 3. Encabezado Izquierdo
    % 4. Color de fondo
    % 5. Coord x del titulo
    % 6. Coord y del titulo
    % 7. Fecha
    % 8. Autor    
    
    \thispagestyle{empty}               % Sin encabezado ni pie de página
    \newgeometry{margin=0cm}        % Márgenes nulos para la primera página
    
    
    % Encabezado
    \fancyhead[L]{\helv #3}
    \fancyhead[R]{\helv \nouppercase{\leftmark}}
    
    
    \pagecolor{#4}        % Color de fondo para la portada
    
    \begin{figure}[p]
        \centering
        \transparent{0.3}           % Opacidad del 30% para la imagen
        
        \includegraphics[width=\paperwidth, keepaspectratio]{../../_assets/#1}
    
        \begin{tikzpicture}[remember picture, overlay]
            \node[anchor=north west, text=white, opacity=1, font=\fontsize{60}{90}\selectfont\bfseries\sffamily, align=left] at (#5, #6) {#2};
            
            \node[anchor=south east, text=white, opacity=1, font=\fontsize{12}{18}\selectfont\sffamily, align=right] at (9.7, 3) {\href{https://losdeldgiim.github.io/}{\textbf{Los Del DGIIM}, \texttt{\footnotesize losdeldgiim.github.io}}};
            
            \node[anchor=south east, text=white, opacity=1, font=\fontsize{12}{15}\selectfont\sffamily, align=right] at (9.7, 1.8) {Doble Grado en Ingeniería Informática y Matemáticas\\Universidad de Granada};
        \end{tikzpicture}
    \end{figure}
    
    
    \restoregeometry        % Restaurar márgenes normales para las páginas subsiguientes
    \nopagecolor      % Restaurar el color de página
    
    
    \newpage
    \thispagestyle{empty}               % Sin encabezado ni pie de página
    \begin{tikzpicture}[remember picture, overlay]
        \node[anchor=south west, inner sep=3cm] at (current page.south west) {
            \begin{minipage}{0.5\paperwidth}
                \href{https://creativecommons.org/licenses/by-nc-nd/4.0/}{
                    \includegraphics[height=2cm]{../../_assets/Licencia.png}
                }\vspace{1cm}\\
                Esta obra está bajo una
                \href{https://creativecommons.org/licenses/by-nc-nd/4.0/}{
                    Licencia Creative Commons Atribución-NoComercial-SinDerivadas 4.0 Internacional (CC BY-NC-ND 4.0).
                }\\
    
                Eres libre de compartir y redistribuir el contenido de esta obra en cualquier medio o formato, siempre y cuando des el crédito adecuado a los autores originales y no persigas fines comerciales. 
            \end{minipage}
        };
    \end{tikzpicture}
    
    
    
    % 1. Foto de fondo
    % 2. Título
    % 3. Encabezado Izquierdo
    % 4. Color de fondo
    % 5. Coord x del titulo
    % 6. Coord y del titulo
    % 7. Fecha
    % 8. Autor


}


\newcommand{\portadaBaseFotoDif}[8]{

    % Tiene la portada principal y la licencia Creative Commons
    
    % 1. Foto de fondo
    % 2. Título
    % 3. Encabezado Izquierdo
    % 4. Color de fondo
    % 5. Coord x del titulo
    % 6. Coord y del titulo
    % 7. Fecha
    % 8. Autor    
    
    \thispagestyle{empty}               % Sin encabezado ni pie de página
    \newgeometry{margin=0cm}        % Márgenes nulos para la primera página
    
    
    % Encabezado
    \fancyhead[L]{\helv #3}
    \fancyhead[R]{\helv \nouppercase{\leftmark}}
    
    
    \pagecolor{#4}        % Color de fondo para la portada
    
    \begin{figure}[p]
        \centering
        \transparent{0.3}           % Opacidad del 30% para la imagen
        
        \includegraphics[width=\paperwidth, keepaspectratio]{#1}
    
        \begin{tikzpicture}[remember picture, overlay]
            \node[anchor=north west, text=white, opacity=1, font=\fontsize{60}{90}\selectfont\bfseries\sffamily, align=left] at (#5, #6) {#2};
            
            \node[anchor=south east, text=white, opacity=1, font=\fontsize{12}{18}\selectfont\sffamily, align=right] at (9.7, 3) {\href{https://losdeldgiim.github.io/}{\textbf{Los Del DGIIM}, \texttt{\footnotesize losdeldgiim.github.io}}};
            
            \node[anchor=south east, text=white, opacity=1, font=\fontsize{12}{15}\selectfont\sffamily, align=right] at (9.7, 1.8) {Doble Grado en Ingeniería Informática y Matemáticas\\Universidad de Granada};
        \end{tikzpicture}
    \end{figure}
    
    
    \restoregeometry        % Restaurar márgenes normales para las páginas subsiguientes
    \nopagecolor      % Restaurar el color de página
    
    
    \newpage
    \thispagestyle{empty}               % Sin encabezado ni pie de página
    \begin{tikzpicture}[remember picture, overlay]
        \node[anchor=south west, inner sep=3cm] at (current page.south west) {
            \begin{minipage}{0.5\paperwidth}
                %\href{https://creativecommons.org/licenses/by-nc-nd/4.0/}{
                %    \includegraphics[height=2cm]{../../_assets/Licencia.png}
                %}\vspace{1cm}\\
                Esta obra está bajo una
                \href{https://creativecommons.org/licenses/by-nc-nd/4.0/}{
                    Licencia Creative Commons Atribución-NoComercial-SinDerivadas 4.0 Internacional (CC BY-NC-ND 4.0).
                }\\
    
                Eres libre de compartir y redistribuir el contenido de esta obra en cualquier medio o formato, siempre y cuando des el crédito adecuado a los autores originales y no persigas fines comerciales. 
            \end{minipage}
        };
    \end{tikzpicture}
    
    
    
    % 1. Foto de fondo
    % 2. Título
    % 3. Encabezado Izquierdo
    % 4. Color de fondo
    % 5. Coord x del titulo
    % 6. Coord y del titulo
    % 7. Fecha
    % 8. Autor


}


\newcommand{\portadaBook}[8]{

    % 1. Foto de fondo
    % 2. Título
    % 3. Encabezado Izquierdo
    % 4. Color de fondo
    % 5. Coord x del titulo
    % 6. Coord y del titulo
    % 7. Fecha
    % 8. Autor

    % Personaliza el formato del título
    \pretitle{\begin{center}\bfseries\fontsize{42}{56}\selectfont}
    \posttitle{\par\end{center}\vspace{2em}}
    
    % Personaliza el formato del autor
    \preauthor{\begin{center}\Large}
    \postauthor{\par\end{center}\vfill}
    
    % Personaliza el formato de la fecha
    \predate{\begin{center}\huge}
    \postdate{\par\end{center}\vspace{2em}}
    
    \title{#2}
    \author{\href{https://losdeldgiim.github.io/}{Los Del DGIIM, \texttt{\large losdeldgiim.github.io}}
    \\ \vspace{0.5cm}#8}
    \date{Granada, #7}
    \maketitle
    
    \tableofcontents
}




\newcommand{\portadaArticle}[8]{

    % 1. Foto de fondo
    % 2. Título
    % 3. Encabezado Izquierdo
    % 4. Color de fondo
    % 5. Coord x del titulo
    % 6. Coord y del titulo
    % 7. Fecha
    % 8. Autor

    % Personaliza el formato del título
    \pretitle{\begin{center}\bfseries\fontsize{42}{56}\selectfont}
    \posttitle{\par\end{center}\vspace{2em}}
    
    % Personaliza el formato del autor
    \preauthor{\begin{center}\Large}
    \postauthor{\par\end{center}\vspace{3em}}
    
    % Personaliza el formato de la fecha
    \predate{\begin{center}\huge}
    \postdate{\par\end{center}\vspace{5em}}
    
    \title{#2}
    \author{\href{https://losdeldgiim.github.io/}{Los Del DGIIM, \texttt{\large losdeldgiim.github.io}}
    \\ \vspace{0.5cm}#8}
    \date{Granada, #7}
    \thispagestyle{empty}               % Sin encabezado ni pie de página
    \maketitle
    \vfill
}
    \portadaExamen{etsiitA4.jpg}{Algorítmica\\Examen III}
    {Algorítmica. Examen III}{MidnightBlue}{-8}{28}{2023-2024}
    {Laura Mandow Fuentes}

    \begin{description}
        \item[Asignatura] Algorítmica.
        \item[Curso Académico] 2018/2019.
        \item[Grado] Doble Grado en Ingienería Informática y Matemáticas.
        \item[Grupo] Único.
        \item[Profesor] Juan Francisco Huete Guadix\footnote{El examen lo pone el departamento.}.
        \item[Descripción] Convocatoria Extraordinaria. 
        \item[Fecha] 28 de junio de 2019.
        % \item[Duración] 60 minutos.
    
    \end{description}
    \newpage
    
    \begin{ejercicio}
        (2 puntos) Calcula el orden de eficiencia de la siguiente función recursiva:

        \begin{minted}{C++}
            int calcula(int ini, int fin) {
                if (ini < fin)
                    return calcula(ini,fin-1) + 
                            calcula(ini+1,fin)+
                            calcula(ini+1,fin-1);
                else return ini-fin;
            }
        \end{minted}

        Notemos por $n$ a \verb|fin - ini|. Tenemos que la recursión 
        nos queda:
        \begin{equation*}
            T(n) = \left \{ \begin{array}{lcc} 
                1 & n = 0  \\ 
                2 \cdot T(n-1) + T(n-2) + 1 & n > 0
            \end{array} \right .
        \end{equation*}

        Aplicando el cambio de variable $T(n) = x^{n}$, tenemos 
        que la parte homogénea nos queda:
        \begin{equation*}
            x^{n} - 2x^{n-1} - x^{n-2} = 0 
            \implies x^{n-2}(x^{2} - 2x - 1) = 0
        \end{equation*}

        Como $x^{n-1} > 0$ para todo $n \in \mathbb{N}$, 
        se tiene que las únicas
         soluciones del
        polinomio característico de la parte homogénea son las de la ecuación de 
        segundo grado:
        \begin{equation*}
            x^{2} - 2x - 1 = 0 \implies x = 1 \pm \sqrt{2}
        \end{equation*}
        
        Añadiendo
        la parte no homogénea, se tiene que $1 = 1^{n}n^{0}$, 
        por lo que el polinomio
        caractertístico de la ecuación en recurrente es:
        \begin{equation*}
            p(x) = (x - 1)(x - 1 - \sqrt{2})(x - 1 + \sqrt{2})
        \end{equation*}

        Por tanto, la solución general de la ecuación 
        recurrente es:
        \begin{equation*}
            T(x) = x^{n} = c_{1}1^{n} + 
                            c_{2}(1 + \sqrt{2})^{n} + 
                            c_{3}(1 - \sqrt{2})^{n} 
                            \in O((1 + \sqrt{2})^{n})
        \end{equation*}

        Por tanto, tenemos que el código dado es de orden 
        \textbf{exponencial}.
    \end{ejercicio}

    \begin{ejercicio}
        (2 puntos) Dado el grafo de la figura, donde aparecen las distancias entre los vértices adyacentes, se
        desea calcular cuáles son los caminos mínimos entre el vértice $S$ 
        y cada uno de los demás vértices.
        Describid detalladamente un algoritmo para resolver ese problema en general y aplicadlo paso a
        paso para ese grafo en concreto.

        \begin{center}
            \begin{tikzpicture}
            %Nodes
            \node[draw, circle] (0) at (0,0) {S};
            \node[draw, circle] (1) at (2,1.5) {A};
            \node[draw, circle] (2) at (2, -1.5) {B};
            \node[draw, circle] (3) at (4, 0) {C};
            \node[draw, circle] (4) at (6, 1.5) {D};
            \node[draw, circle] (5) at (6, -1.5) {E};

            %Edges
            \draw (0) -- node[above] {6} (1) ;
            \draw (1) -- node[left] {1} (2);
            \draw (2) -- node[below] {3} (0);
            \draw (1) -- node[below] {7} (3);
            \draw (2) -- node[above] {6} (3);
            \draw (1) -- node[above] {6} (4);
            \draw (3) -- node[above] {2} (4);
            \draw (2) -- node[above] {12} (5);
            \draw (3) -- node[above] {4} (5);
            \end{tikzpicture}
        \end{center}

        El ejercicio nos pide un algoritmo para calcular los caminos 
        mínimos de \textbf{UN nodo a TODOS} para un grafo cuyos pesos 
        son todos 
        \textbf{positivos}. Por tanto, la solución al problema se trata 
        de aplicar algoritmo de \textit{Dijkstra}.

        El algoritmo de Dijkstra calcula la distancia mínima desde 
        un nodo ($S$ en nuestro caso) a todos los demás. El algoritmo
        funciona de la siguiente forma:
        \begin{enumerate}
            \item De todos nodos a los que todavía no les hemos calulado su 
                distancia mínima a $S$ (no han sido 
                seleccionados), seleccionamos aquel cuya distancia a $S$ 
                es mínima.
            \item Actualizamos todos sus vecinos, de forma que si 
                se puede llegar a uno de sus vecinos pasando por el nodo 
                seleccionado por un camino más corto que el que ya tiene 
                actualizamos su distancia y almacenamos que el nodo 
                que va antes que el vecino es el nodo seleccionado.
            \item Repetir hasta que todos los nodos hayan sido seleccionados.
        \end{enumerate}

        Dado que el grafo de ejemplo tiene muchos arcos respecto al 
        número de nodos, optaremos por la implementación para grafos 
        densos.
        \lstinputlisting[language=C++]
        {./Programas_Laura/dijkstra_dense.cpp}

        Apliquemos ahora el algoritmo al grafo de ejemplo.
        \begin{enumerate}
            \item 
            Inicialmente el vector de distancias comienza así:
            \begin{center}
                \begin{tabular}{|c|c|c|c|c|c|}
                    \hline
                    \textcolor{blue}{S}&A&B&C&D&E \\
                    \hline
                    \textcolor{blue}{0}&$\infty$&$\infty$&$\infty$&$\infty$&$\infty$ \\
                    \hline
                \end{tabular}
            \end{center}
            Ya que lo único que sabemos es que la distancia de $S$ a 
            si mismo es 0. Procedemos entonces a seleccionar $S$ 
            por ser el nodo no seleccionado cuya a distancia a $S$ 
            es menor. Noteremos el nuevo nodo seleccionado en 
            \textcolor{blue}{azul} y los ya seleccionados en 
            \textbf{negrita}.

            \item 
            Actualizamos los vecinos de $S$ ($A$ y $B$) y seleccionamos el nodo 
            cuya distancia a $S$ es menor de los no seleccionados.
            \begin{center}
                \begin{tabular}{|c|c|c|c|c|c|}
                    \hline
                    \textbf{S}&A&\textcolor{blue}{B}&C&D&E \\
                    \hline
                    \textbf{0}&{6}&\textcolor{blue}{3}&
                    $\infty$&$\infty$&$\infty$ \\
                    \hline
                \end{tabular}
            \end{center}
            En este caso es $B$. Ambos vecinos han sido actualizados 
            ya que previamente su distancia a $S$ era infinito.

            \item 
            Actualizamos los vecinos de $B$ ($A$,$C$, $E$ ($S$ ya fue 
            seleccionado por tanto ya se calculo su distancia mínima 
            y por tanto esta no será actualizada nunca más)) 
            y seleccionamos el nodo 
            cuya distancia a $S$ es menor de los no seleccionados.
            \begin{center}
                \begin{tabular}{|c|c|c|c|c|c|}
                    \hline
                    \textbf{S}&\textcolor{blue}{A}&\textbf{B}&C&D&E \\
                    \hline
                    \textbf{0}&\textcolor{blue}{4}&\textbf{3}&
                    9&$\infty$&15 \\
                    \hline
                \end{tabular}
            \end{center}
            En este caso es $A$. Todos los vecinos han sido 
            actualizados (incluyendo $A$) salvo los ya seleccionados.

            \item 
            Actualizamos los vecinos (no seleccionados) de $A$ ($C$ y $D$) 
            y seleccionamos el nodo 
            cuya distancia a $S$ es menor de los no seleccionados.
            \begin{center}
                \begin{tabular}{|c|c|c|c|c|c|}
                    \hline
                    \textbf{S}&\textbf{A}&\textbf{B}&\textcolor{blue}{C}&D&E \\
                    \hline
                    \textbf{0}&\textbf{4}&\textbf{3}&
                    \textcolor{blue}{9}&10&15 \\
                    \hline
                \end{tabular}
            \end{center}
            En este caso es $C$. En este caso $D$ fue actualizado al 
            ser infinito pero $C$ no por ser mejor el camino ya 
            calculado.

            \item 
            Actualizamos los vecinos (no seleccionados) de $C$ ($E$ y $D$) 
            y seleccionamos el nodo 
            cuya distancia a $S$ es menor de los no seleccionados.
            \begin{center}
                \begin{tabular}{|c|c|c|c|c|c|}
                    \hline
                    \textbf{S}&\textbf{A}&\textbf{B}&
                    \textbf{C}&\textcolor{blue}{D}&E \\
                    \hline
                    \textbf{0}&\textbf{4}&\textbf{3}&
                    \textbf{9}&\textcolor{blue}{10}&13 \\
                    \hline
                \end{tabular}
            \end{center}
            En este caso es $D$. En este caso $E$ fue actualizado al 
            ser mejor el camino que pasa por $C$ pero $E$ 
            no por ser mejor el camino ya 
            calculado.
            \item Finalmente tenemos que $D$ no tiene vecinos no seleccionados,
            y ya no quedan más nodos aparte de $E$ por actualizar, por tanto hemos acabado
            y tenenos que el vector de distancias a $S$ que nos queda es:
            \begin{center}
                \begin{tabular}{|c|c|c|c|c|c|}
                    \hline
                    {S}&A&B&C&D&E \\
                    \hline
                    0&4&3&
                    9&10&13 \\
                    \hline
                \end{tabular}
            \end{center}
        \end{enumerate}
    \end{ejercicio}

    \begin{ejercicio}
        (2 puntos) Diseña 
        un algoritmo para calcular 
        $x^{n},   n \in \mathbb{N}$, 
        con un coste $O(\log{n})$ en términos del
        número de multiplicaciones.\\

        Claramente nos están pidiento que implementemos la función 
        \verb|potencia| para exponentes \textbf{naturales} 
        mediante el método de exponenciación rápida. Este método 
        consiste en abusar de la propiedad matématica:
        \begin{equation*}
            x^{2n} = (x^{n})^{2} = x^{n} \cdot x^{n}
        \end{equation*}

        Y también en aplicar 
        \begin{equation*}
            x^{n+1} = x^{n} \cdot x
        \end{equation*}

        De esta forma, podemos definir la función potencia para 
        exponentes naturales recursivamente:
        \begin{equation*}
            x^{n} = \left \{ \begin{array}{lcc} 
                1 & n = 0  \\ 
                x^{\nicefrac{n}{2}} \cdot x^{\nicefrac{n}{2}}  & n > 0 \wedge n\: par \\
                x^{\nicefrac{n}{2}} \cdot x^{\nicefrac{n}{2}} \cdot x  & n > 0 \wedge n\: impar
            \end{array} \right .
        \end{equation*}

        Donde notamos por $\nicefrac{n}{2}$ al máximo número entero $k$
        tal que $2k \leq n$ (dicho de otro modo, entedemos la división 
        de enteros igual que c++). Por ejemplo $\frac{1}{2} = 0$
        y $\frac{3}{2} = 1$. Equivalentemente, si $n$ es par 
        la división es la que conocemos de toda la vida, y si 
        $n$ es impar se corresponde con $\frac{n-1}{2}$, ya que sabemos
        que $n-1$ es par.\\
        Definimos la siguiente función recursiva:
        \begin{equation*}
            potencia(x,n) = \left \{ \begin{array}{ll} 
                1 & n = 0  \\ 
                potencia(x,\nicefrac{n}{2}) \cdot potencia(x,\nicefrac{n}{2})  & n > 0 \wedge n\: par \\
                potencia(x,\nicefrac{n}{2}) \cdot potencia(x,\nicefrac{n}{2}) \cdot x  & n > 0 \wedge n\: impar
            \end{array} \right .
        \end{equation*}

        Nos fijamos en que $potencia(x,\nicefrac{n}{2})$ hace falta dos veces 
        en cada paso, motivo por el cual hará falta almacenar este resultado 
        en alguna variable para evitar tener que recalcularlo.\\
        La implementación en c++ sería:

        \begin{minipage}{0.9\linewidth} %avoid page breaks
            \lstinputlisting[language=C++]
            {./Programas_Laura/potencia.cpp}
        \end{minipage}

        La función de eficiencia sería:
        \begin{equation*}
            T(n) = T(\nicefrac{n}{2}) + 1
        \end{equation*}
        Por tanto es claramente de orden \textbf{logarítmico}.
    \end{ejercicio}

    \begin{ejercicio}
        (2 puntos) Consideremos un mapa formado por $N$ países. 
        Queremos viajar entre países. Cada vez
        que atravesemos una frontera tenemos que pagar una tasa. 
        Todas las tasas son conocidas. Construid
        un algoritmo de programación dinámica que determine el 
        coste del viaje más barato entre dos países
        dados.

        Tenemos que se trata de un problema de caminos mínimos.
        En este caso el coste de ir de un nodo (país) a otro 
        es la tasa que nos imponen en la frontera.
        Entendemos 
        que todas las tasas son positivas (no tendría mucho 
        sentido que nos dieran dinero por viajar a otro país) 
        y que el enunciado desea saber el viaje más barato entre 
        dos países \textbf{cualesquiera}. Por tanto, se trata 
        de calcular el camino mínimo de \textbf{TODOS a TODOS} los nodos 
        y que los pesos de los arcos son \textbf{positivos}.
        Claramente debemos usar el algoritmo de \textit{Floyd-Warshall}.
        Este algoritmo emplea la técnica de programación dinámica,
        tal y como se pide en el enunciado.

        La implementación de este algoritmo en C++ es la siguiente:

        \begin{minipage}{0.9\linewidth} %avoid page breaks
            \lstinputlisting[language=C++]
            {./Programas_Laura/floyd-warshall.cpp}
        \end{minipage}
    \end{ejercicio}

    \begin{ejercicio}
        (2 puntos) Diseñad un algoritmo de exploración de grafos que, 
        dado un número natural $n$, devuelva
        todas las formas posibles en que un conjunto ascendente de 
        números positivos sume exactamente $n$.
        Por ejemplo, si $n = 10$, la salida debería ser:
        \begin{verbatim}
1+2+3+4
1+2+7
1+3+6
1+4+5
1+9
2+3+5
2+8
3+7
4+6
10
        \end{verbatim}
        Especificad claramente, además del algoritmo, 
        la representación de la solución, las restricciones
        explícitas e implícitas, así como las posibles cotas a utilizar.

        Tenemos pues que se trata de un problema de exploración de 
        grafos, por tanto debemos aplicar o \textit{backtracking}
        o \textit{branch and bound}.

        Dado que el problema nos pide que imprimamos \textbf{todas}
        las soluciones, vamos a optar por aplicar \textit{backtracking}
        al ser más sencillo, ya que no nos podemos beneficiar 
        de ``descartar'' soluciones ``peores'' mediante la cola de 
        prioridad de \textit{branch and bound}, y por tanto nos supondría
        una complejidad añadida.

        La solucion consistirá en un conjunto de conjuntos $s$, 
        donde cada conjunto $s$ es una $k$-tupla de numeros naturales 
        (el cero no es natural en este caso) ordenada ascendentemente 
        sin numeros repetidos cuya suma en $n$. Las 
        escribiremos como $(x_1,x_2,...,x_k)$. Notamos distintas 
        tuplas pueden tener tamaños distintos.

        Representaremos los tuplas de números con \verb|set|
        ya que no podemos repetir números y vamos a realizar muchas 
        inserciones y borrados por lo que nos conviene que sean 
        operaciones eficientes.\\
        Por comodidad emplearemos la siguiente estructura:
        \begin{lstlisting}
            struct tupla {
                set<int> tupla;
                int suma;

                tupla(){
                    tupla = {};
                    suma = 0;
                }

                void add(int x){
                    suma += x;
                    tupla.insert(x);
                }

                void erase(int x){
                    suma -= x;
                    tupla.erase(x);
                }
            };
        \end{lstlisting}
        Notamos que las estructuras incluyen el conjunto en cuestión 
        y un campo donde iremos almacenando la suma del conjunto 
        para poder acceder a ella en tiempo \textbf{constante}.
        Además se incluyen funciones para añadir y eliminar 
        numeros del conjunto.

        Las \textbf{restricciones explícitas} son que para toda 
        tupla $(x_1,x_2,...,x_k)$ se tiene que $x_i \in \mathbb{N}$
        para todo $i$ tal que $0 < i \leq k$.

        Las \textbf{restricciones implícitas} son que para toda 
        tupla $(x_1,x_2,...,x_k)$ se tiene que $x_i < x_{i+1}$
        para todo $i$ tal que $0 < i \leq k$ (debe estar ordenada)
        y además $\sum\limits_{i=1}^{k}x_i = n$.

        Procedamos ahora a implementar el algoritmo mediante un TDA (clase) 
        \verb|Solucion|. 
        Tendremos las funciones \verb|esSolucion| para ver si ya hemos 
        alcanzado una solucion, \verb|procesaSolucion| para cuando 
        alcancemos una imprimirla y guardarla, \verb|factible| para ver 
        si un conjunto puede llegar a ser solucion o todos sus caminos 
        acabaran en fracaso. También tenemos \verb|f_cota| que 
        representa la función de cota usada para por \verb|factible|.

        En este caso al tener que sacar todas las solucione posibles 
        y deber alcanzar una suma, tenemos cota inferior 
        (para comprobar si nos vamos a pasar de la suma deseada) 
        como cota local (no tiene sentido una cota superior ya que 
        siempre podemos seguir sumando). La cota global o superior 
        en este caso sería el valor a alcanzar sumando.
        \begin{lstlisting}
            class Solucion{
                // Almacenamos las soluciones aqui
                set<tupla> soluciones; 
                // Numero cuyas sumas debemos alcanzar
                int n;

                void backtracking(int k,tupla & s){
                    if(esSolucion(s))
                        procesaSolucion(s);

                    // Seleccionamos el siguiente numero para el conjunto
                    // Como los conjuntos deben estar ordenados,
                    // el siguiente numero debe ser mayor que el anterior
                    // Ademas no puede ser mayor que el n-s.suma
                    // ya que entonces al agregarlo al conjunto 
                    // la suma seria mayor que n
                    int tope = n - s.suma;
                    for(int i=k+1; i<=tope; ++i){
                        // Si la solucion (junto con el nuevo numero) es factible
                        // proseguimos este camino
                        if(factible(i,s)){
                            s.insert(i);
                            backtracking(i,s);
                            s.erase(i);
                        }
                    }
                }

                bool esSolucion(tupla & s){
                    // Si la suma de todos los numeros del conjunto 
                    // es igual n tenemos una solucion valida
                    return s.suma == n;
                }

                bool factible(int k,tupla & s){
                    // Sacamos la cota inferior de cual puede ser la suma 
                    // y comprobamos que es menor igual que la suma 
                    // que queremos alcanzar
                    int cota_inf = f_cota(k,s);
                    return cota_inf <= n;
                }

                int f_cota(int k, tupla & s){
                    // Tenemos que para que una solucion sea factible
                    // al agregarle el valor k la suma del conjunto 
                    // debe ser menor o igual que n
                    return k + s.suma;
                }

                void procesaSolucion(tupla & s){
                    // Imprimos la solucion y la guardamos
                    for(auto it = s.begin(); it != s.end(); ++it)
                        cout << *it << "+"
                    cout << endl;
                    soluciones.insert(s);
                }
            };
        \end{lstlisting} 

    \end{ejercicio}
    
\end{document}