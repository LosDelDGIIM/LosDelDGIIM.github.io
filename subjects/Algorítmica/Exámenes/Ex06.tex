\documentclass[12pt]{article}

% Idioma y codificación
\usepackage[spanish, es-tabla, es-notilde]{babel}       %es-tabla para que se titule "Tabla"
\usepackage[utf8]{inputenc}

% Márgenes
\usepackage[a4paper,top=3cm,bottom=2.5cm,left=3cm,right=3cm]{geometry}

% Comentarios de bloque
\usepackage{verbatim}

% Paquetes de links
\usepackage[hidelinks]{hyperref}    % Permite enlaces
\usepackage{url}                    % redirecciona a la web

% Más opciones para enumeraciones
\usepackage{enumitem}

% Personalizar la portada
\usepackage{titling}

% Paquetes de tablas
\usepackage{multirow}

% Para añadir el símbolo de euro
\usepackage{eurosym}


%------------------------------------------------------------------------

%Paquetes de figuras
\usepackage{caption}
\usepackage{subcaption} % Figuras al lado de otras
\usepackage{float}      % Poner figuras en el sitio indicado H.


% Paquetes de imágenes
\usepackage{graphicx}       % Paquete para añadir imágenes
\usepackage{transparent}    % Para manejar la opacidad de las figuras

% Paquete para usar colores
\usepackage[dvipsnames, table, xcdraw]{xcolor}
\usepackage{pagecolor}      % Para cambiar el color de la página

% Habilita tamaños de fuente mayores
\usepackage{fix-cm}

% Para los gráficos
\usepackage{tikz}
\usepackage{forest}

% Para poder situar los nodos en los grafos
\usetikzlibrary{positioning}


%------------------------------------------------------------------------

% Paquetes de matemáticas
\usepackage{mathtools, amsfonts, amssymb, mathrsfs}
\usepackage[makeroom]{cancel}     % Simplificar tachando
\usepackage{polynom}    % Divisiones y Ruffini
\usepackage{units} % Para poner fracciones diagonales con \nicefrac

\usepackage{pgfplots}   %Representar funciones
\pgfplotsset{compat=1.18}  % Versión 1.18

\usepackage{tikz-cd}    % Para usar diagramas de composiciones
\usetikzlibrary{calc}   % Para usar cálculo de coordenadas en tikz

%Definición de teoremas, etc.
\usepackage{amsthm}
%\swapnumbers   % Intercambia la posición del texto y de la numeración

\theoremstyle{plain}

\makeatletter
\@ifclassloaded{article}{
  \newtheorem{teo}{Teorema}[section]
}{
  \newtheorem{teo}{Teorema}[chapter]  % Se resetea en cada chapter
}
\makeatother

\newtheorem{coro}{Corolario}[teo]           % Se resetea en cada teorema
\newtheorem{prop}[teo]{Proposición}         % Usa el mismo contador que teorema
\newtheorem{lema}[teo]{Lema}                % Usa el mismo contador que teorema
\newtheorem*{lema*}{Lema}

\theoremstyle{remark}
\newtheorem*{observacion}{Observación}

\theoremstyle{definition}

\makeatletter
\@ifclassloaded{article}{
  \newtheorem{definicion}{Definición} [section]     % Se resetea en cada chapter
}{
  \newtheorem{definicion}{Definición} [chapter]     % Se resetea en cada chapter
}
\makeatother

\newtheorem*{notacion}{Notación}
\newtheorem*{ejemplo}{Ejemplo}
\newtheorem*{ejercicio*}{Ejercicio}             % No numerado
\newtheorem{ejercicio}{Ejercicio} [section]     % Se resetea en cada section


% Modificar el formato de la numeración del teorema "ejercicio"
\renewcommand{\theejercicio}{%
  \ifnum\value{section}=0 % Si no se ha iniciado ninguna sección
    \arabic{ejercicio}% Solo mostrar el número de ejercicio
  \else
    \thesection.\arabic{ejercicio}% Mostrar número de sección y número de ejercicio
  \fi
}


% \renewcommand\qedsymbol{$\blacksquare$}         % Cambiar símbolo QED
%------------------------------------------------------------------------

% Paquetes para encabezados
\usepackage{fancyhdr}
\pagestyle{fancy}
\fancyhf{}

\newcommand{\helv}{ % Modificación tamaño de letra
\fontfamily{}\fontsize{12}{12}\selectfont}
\setlength{\headheight}{15pt} % Amplía el tamaño del índice


%\usepackage{lastpage}   % Referenciar última pag   \pageref{LastPage}
%\fancyfoot[C]{%
%  \begin{minipage}{\textwidth}
%    \centering
%    ~\\
%    \thepage\\
%    \href{https://losdeldgiim.github.io/}{\texttt{\footnotesize losdeldgiim.github.io}}
%  \end{minipage}
%}
\fancyfoot[C]{\thepage}
\fancyfoot[R]{\href{https://losdeldgiim.github.io/}{\texttt{\footnotesize losdeldgiim.github.io}}}

%------------------------------------------------------------------------

% Conseguir que no ponga "Capítulo 1". Sino solo "1."
\makeatletter
\@ifclassloaded{book}{
  \renewcommand{\chaptermark}[1]{\markboth{\thechapter.\ #1}{}} % En el encabezado
    
  \renewcommand{\@makechapterhead}[1]{%
  \vspace*{50\p@}%
  {\parindent \z@ \raggedright \normalfont
    \ifnum \c@secnumdepth >\m@ne
      \huge\bfseries \thechapter.\hspace{1em}\ignorespaces
    \fi
    \interlinepenalty\@M
    \Huge \bfseries #1\par\nobreak
    \vskip 40\p@
  }}
}
\makeatother

%------------------------------------------------------------------------
% Paquetes de cógido
\usepackage{minted}
\renewcommand\listingscaption{Código fuente}

\usepackage{fancyvrb}
% Personaliza el tamaño de los números de línea
\renewcommand{\theFancyVerbLine}{\small\arabic{FancyVerbLine}}

% Estilo para C++
\newminted{cpp}{
    frame=lines,
    framesep=2mm,
    baselinestretch=1.2,
    linenos,
    escapeinside=||
}

% para minted
\definecolor{LightGray}{rgb}{0.95,0.95,0.92}
\setminted{
    linenos=true,
    stepnumber=5,
    numberfirstline=true,
    autogobble,
    breaklines=true,
    breakautoindent=true,
    breaksymbolleft=,
    breaksymbolright=,
    breaksymbolindentleft=0pt,
    breaksymbolindentright=0pt,
    breaksymbolsepleft=0pt,
    breaksymbolsepright=0pt,
    fontsize=\footnotesize,
    bgcolor=LightGray,
    numbersep=10pt
}


\usepackage{listings} % Para incluir código desde un archivo

\renewcommand\lstlistingname{Código Fuente}
\renewcommand\lstlistlistingname{Índice de Códigos Fuente}

% Definir colores
\definecolor{vscodepurple}{rgb}{0.5,0,0.5}
\definecolor{vscodeblue}{rgb}{0,0,0.8}
\definecolor{vscodegreen}{rgb}{0,0.5,0}
\definecolor{vscodegray}{rgb}{0.5,0.5,0.5}
\definecolor{vscodebackground}{rgb}{0.97,0.97,0.97}
\definecolor{vscodelightgray}{rgb}{0.9,0.9,0.9}

% Configuración para el estilo de C similar a VSCode
\lstdefinestyle{vscode_C}{
  backgroundcolor=\color{vscodebackground},
  commentstyle=\color{vscodegreen},
  keywordstyle=\color{vscodeblue},
  numberstyle=\tiny\color{vscodegray},
  stringstyle=\color{vscodepurple},
  basicstyle=\scriptsize\ttfamily,
  breakatwhitespace=false,
  breaklines=true,
  captionpos=b,
  keepspaces=true,
  numbers=left,
  numbersep=5pt,
  showspaces=false,
  showstringspaces=false,
  showtabs=false,
  tabsize=2,
  frame=tb,
  framerule=0pt,
  aboveskip=10pt,
  belowskip=10pt,
  xleftmargin=10pt,
  xrightmargin=10pt,
  framexleftmargin=10pt,
  framexrightmargin=10pt,
  framesep=0pt,
  rulecolor=\color{vscodelightgray},
  backgroundcolor=\color{vscodebackground},
}

%------------------------------------------------------------------------

% Comandos definidos
\newcommand{\bb}[1]{\mathbb{#1}}
\newcommand{\cc}[1]{\mathcal{#1}}

% I prefer the slanted \leq
\let\oldleq\leq % save them in case they're every wanted
\let\oldgeq\geq
\renewcommand{\leq}{\leqslant}
\renewcommand{\geq}{\geqslant}

% Si y solo si
\newcommand{\sii}{\iff}

% MCD y MCM
\DeclareMathOperator{\mcd}{mcd}
\DeclareMathOperator{\mcm}{mcm}

% Signo
\DeclareMathOperator{\sgn}{sgn}

% Letras griegas
\newcommand{\eps}{\epsilon}
\newcommand{\veps}{\varepsilon}
\newcommand{\lm}{\lambda}

\newcommand{\ol}{\overline}
\newcommand{\ul}{\underline}
\newcommand{\wt}{\widetilde}
\newcommand{\wh}{\widehat}

\let\oldvec\vec
\renewcommand{\vec}{\overrightarrow}

% Derivadas parciales
\newcommand{\del}[2]{\frac{\partial #1}{\partial #2}}
\newcommand{\Del}[3]{\frac{\partial^{#1} #2}{\partial #3^{#1}}}
\newcommand{\deld}[2]{\dfrac{\partial #1}{\partial #2}}
\newcommand{\Deld}[3]{\dfrac{\partial^{#1} #2}{\partial #3^{#1}}}


\newcommand{\AstIg}{\stackrel{(\ast)}{=}}
\newcommand{\Hop}{\stackrel{L'H\hat{o}pital}{=}}

\newcommand{\red}[1]{{\color{red}#1}} % Para integrales, destacar los cambios.

% Método de integración
\newcommand{\MetInt}[2]{
    \left[\begin{array}{c}
        #1 \\ #2
    \end{array}\right]
}

% Declarar aplicaciones
% 1. Nombre aplicación
% 2. Dominio
% 3. Codominio
% 4. Variable
% 5. Imagen de la variable
\newcommand{\Func}[5]{
    \begin{equation*}
        \begin{array}{rrll}
            \displaystyle #1:& \displaystyle  #2 & \longrightarrow & \displaystyle  #3\\
               & \displaystyle  #4 & \longmapsto & \displaystyle  #5
        \end{array}
    \end{equation*}
}

%------------------------------------------------------------------------



\begin{document}

    % 1. Foto de fondo
    % 2. Título
    % 3. Encabezado Izquierdo
    % 4. Color de fondo
    % 5. Coord x del titulo
    % 6. Coord y del titulo
    % 7. Fecha

    
    % 1. Foto de fondo
% 2. Título
% 3. Encabezado Izquierdo
% 4. Color de fondo
% 5. Coord x del titulo
% 6. Coord y del titulo
% 7. Fecha
% 8. Autor

\newcommand{\portada}[8]{
    \portadaBase{#1}{#2}{#3}{#4}{#5}{#6}{#7}{#8}
    \portadaBook{#1}{#2}{#3}{#4}{#5}{#6}{#7}{#8}
}

\newcommand{\portadaFotoDif}[8]{
    \portadaBaseFotoDif{#1}{#2}{#3}{#4}{#5}{#6}{#7}{#8}
    \portadaBook{#1}{#2}{#3}{#4}{#5}{#6}{#7}{#8}
}

\newcommand{\portadaExamen}[8]{
    \portadaBase{#1}{#2}{#3}{#4}{#5}{#6}{#7}{#8}
    \portadaArticle{#1}{#2}{#3}{#4}{#5}{#6}{#7}{#8}
}

\newcommand{\portadaExamenFotoDif}[8]{
    \portadaBaseFotoDif{#1}{#2}{#3}{#4}{#5}{#6}{#7}{#8}
    \portadaArticle{#1}{#2}{#3}{#4}{#5}{#6}{#7}{#8}
}




\newcommand{\portadaBase}[8]{

    % Tiene la portada principal y la licencia Creative Commons
    
    % 1. Foto de fondo
    % 2. Título
    % 3. Encabezado Izquierdo
    % 4. Color de fondo
    % 5. Coord x del titulo
    % 6. Coord y del titulo
    % 7. Fecha
    % 8. Autor    
    
    \thispagestyle{empty}               % Sin encabezado ni pie de página
    \newgeometry{margin=0cm}        % Márgenes nulos para la primera página
    
    
    % Encabezado
    \fancyhead[L]{\helv #3}
    \fancyhead[R]{\helv \nouppercase{\leftmark}}
    
    
    \pagecolor{#4}        % Color de fondo para la portada
    
    \begin{figure}[p]
        \centering
        \transparent{0.3}           % Opacidad del 30% para la imagen
        
        \includegraphics[width=\paperwidth, keepaspectratio]{../../_assets/#1}
    
        \begin{tikzpicture}[remember picture, overlay]
            \node[anchor=north west, text=white, opacity=1, font=\fontsize{60}{90}\selectfont\bfseries\sffamily, align=left] at (#5, #6) {#2};
            
            \node[anchor=south east, text=white, opacity=1, font=\fontsize{12}{18}\selectfont\sffamily, align=right] at (9.7, 3) {\href{https://losdeldgiim.github.io/}{\textbf{Los Del DGIIM}, \texttt{\footnotesize losdeldgiim.github.io}}};
            
            \node[anchor=south east, text=white, opacity=1, font=\fontsize{12}{15}\selectfont\sffamily, align=right] at (9.7, 1.8) {Doble Grado en Ingeniería Informática y Matemáticas\\Universidad de Granada};
        \end{tikzpicture}
    \end{figure}
    
    
    \restoregeometry        % Restaurar márgenes normales para las páginas subsiguientes
    \nopagecolor      % Restaurar el color de página
    
    
    \newpage
    \thispagestyle{empty}               % Sin encabezado ni pie de página
    \begin{tikzpicture}[remember picture, overlay]
        \node[anchor=south west, inner sep=3cm] at (current page.south west) {
            \begin{minipage}{0.5\paperwidth}
                \href{https://creativecommons.org/licenses/by-nc-nd/4.0/}{
                    \includegraphics[height=2cm]{../../_assets/Licencia.png}
                }\vspace{1cm}\\
                Esta obra está bajo una
                \href{https://creativecommons.org/licenses/by-nc-nd/4.0/}{
                    Licencia Creative Commons Atribución-NoComercial-SinDerivadas 4.0 Internacional (CC BY-NC-ND 4.0).
                }\\
    
                Eres libre de compartir y redistribuir el contenido de esta obra en cualquier medio o formato, siempre y cuando des el crédito adecuado a los autores originales y no persigas fines comerciales. 
            \end{minipage}
        };
    \end{tikzpicture}
    
    
    
    % 1. Foto de fondo
    % 2. Título
    % 3. Encabezado Izquierdo
    % 4. Color de fondo
    % 5. Coord x del titulo
    % 6. Coord y del titulo
    % 7. Fecha
    % 8. Autor


}


\newcommand{\portadaBaseFotoDif}[8]{

    % Tiene la portada principal y la licencia Creative Commons
    
    % 1. Foto de fondo
    % 2. Título
    % 3. Encabezado Izquierdo
    % 4. Color de fondo
    % 5. Coord x del titulo
    % 6. Coord y del titulo
    % 7. Fecha
    % 8. Autor    
    
    \thispagestyle{empty}               % Sin encabezado ni pie de página
    \newgeometry{margin=0cm}        % Márgenes nulos para la primera página
    
    
    % Encabezado
    \fancyhead[L]{\helv #3}
    \fancyhead[R]{\helv \nouppercase{\leftmark}}
    
    
    \pagecolor{#4}        % Color de fondo para la portada
    
    \begin{figure}[p]
        \centering
        \transparent{0.3}           % Opacidad del 30% para la imagen
        
        \includegraphics[width=\paperwidth, keepaspectratio]{#1}
    
        \begin{tikzpicture}[remember picture, overlay]
            \node[anchor=north west, text=white, opacity=1, font=\fontsize{60}{90}\selectfont\bfseries\sffamily, align=left] at (#5, #6) {#2};
            
            \node[anchor=south east, text=white, opacity=1, font=\fontsize{12}{18}\selectfont\sffamily, align=right] at (9.7, 3) {\href{https://losdeldgiim.github.io/}{\textbf{Los Del DGIIM}, \texttt{\footnotesize losdeldgiim.github.io}}};
            
            \node[anchor=south east, text=white, opacity=1, font=\fontsize{12}{15}\selectfont\sffamily, align=right] at (9.7, 1.8) {Doble Grado en Ingeniería Informática y Matemáticas\\Universidad de Granada};
        \end{tikzpicture}
    \end{figure}
    
    
    \restoregeometry        % Restaurar márgenes normales para las páginas subsiguientes
    \nopagecolor      % Restaurar el color de página
    
    
    \newpage
    \thispagestyle{empty}               % Sin encabezado ni pie de página
    \begin{tikzpicture}[remember picture, overlay]
        \node[anchor=south west, inner sep=3cm] at (current page.south west) {
            \begin{minipage}{0.5\paperwidth}
                %\href{https://creativecommons.org/licenses/by-nc-nd/4.0/}{
                %    \includegraphics[height=2cm]{../../_assets/Licencia.png}
                %}\vspace{1cm}\\
                Esta obra está bajo una
                \href{https://creativecommons.org/licenses/by-nc-nd/4.0/}{
                    Licencia Creative Commons Atribución-NoComercial-SinDerivadas 4.0 Internacional (CC BY-NC-ND 4.0).
                }\\
    
                Eres libre de compartir y redistribuir el contenido de esta obra en cualquier medio o formato, siempre y cuando des el crédito adecuado a los autores originales y no persigas fines comerciales. 
            \end{minipage}
        };
    \end{tikzpicture}
    
    
    
    % 1. Foto de fondo
    % 2. Título
    % 3. Encabezado Izquierdo
    % 4. Color de fondo
    % 5. Coord x del titulo
    % 6. Coord y del titulo
    % 7. Fecha
    % 8. Autor


}


\newcommand{\portadaBook}[8]{

    % 1. Foto de fondo
    % 2. Título
    % 3. Encabezado Izquierdo
    % 4. Color de fondo
    % 5. Coord x del titulo
    % 6. Coord y del titulo
    % 7. Fecha
    % 8. Autor

    % Personaliza el formato del título
    \pretitle{\begin{center}\bfseries\fontsize{42}{56}\selectfont}
    \posttitle{\par\end{center}\vspace{2em}}
    
    % Personaliza el formato del autor
    \preauthor{\begin{center}\Large}
    \postauthor{\par\end{center}\vfill}
    
    % Personaliza el formato de la fecha
    \predate{\begin{center}\huge}
    \postdate{\par\end{center}\vspace{2em}}
    
    \title{#2}
    \author{\href{https://losdeldgiim.github.io/}{Los Del DGIIM, \texttt{\large losdeldgiim.github.io}}
    \\ \vspace{0.5cm}#8}
    \date{Granada, #7}
    \maketitle
    
    \tableofcontents
}




\newcommand{\portadaArticle}[8]{

    % 1. Foto de fondo
    % 2. Título
    % 3. Encabezado Izquierdo
    % 4. Color de fondo
    % 5. Coord x del titulo
    % 6. Coord y del titulo
    % 7. Fecha
    % 8. Autor

    % Personaliza el formato del título
    \pretitle{\begin{center}\bfseries\fontsize{42}{56}\selectfont}
    \posttitle{\par\end{center}\vspace{2em}}
    
    % Personaliza el formato del autor
    \preauthor{\begin{center}\Large}
    \postauthor{\par\end{center}\vspace{3em}}
    
    % Personaliza el formato de la fecha
    \predate{\begin{center}\huge}
    \postdate{\par\end{center}\vspace{5em}}
    
    \title{#2}
    \author{\href{https://losdeldgiim.github.io/}{Los Del DGIIM, \texttt{\large losdeldgiim.github.io}}
    \\ \vspace{0.5cm}#8}
    \date{Granada, #7}
    \thispagestyle{empty}               % Sin encabezado ni pie de página
    \maketitle
    \vfill
}
    \portadaExamen{etsiitA4.jpg}{Algorítmica\\Examen VI}{Algorítmica. Examen VI}{MidnightBlue}{-8}{28}{2025}{José Juan Urrutia Milán}

    \begin{description}
        \item[Asignatura] Algorítmica.
        \item[Curso Académico] 2024/25.
        \item[Grado] Doble grado en Ingeniería Informática y Matemáticas.
        \item[Grupo] Único.
        \item[Profesor] Francisco Javier Cabrerizo Lorite.
        \item[Descripción] Examen de la Convocatoria Ordinaria.
        \item[Fecha] 16 de junio de 2025.
        \item[Duración] 2 horas y media.
    
    \end{description}
    \newpage


    % ------------------------------------
    
    \begin{ejercicio}[2 puntos]
        La ecuación en recurrencias resultante del análisis de eficiencia de un algoritmo depende de una condición dada $C$, de modo que tenemos $T(n) = 2T(\nicefrac{n}{4}) + n^2$ si $C$ es cierta, y $T(n) = 4T(\nicefrac{n}{2}) + \log_2(n)$ si $C$ es falsa. Determine los órdenes de complejidad $O$ y $\Omega$ del algoritmo y justifique si este tiene, o no, orden de complejidad $\Theta$.
    \end{ejercicio}
    
    \begin{ejercicio}[2 puntos]
        Dado un vector de números enteros de tamaño $n$, ordenado de menor a mayor, y donde puede haber números duplicados, se quiere contar cuántas veces aparece un determinado número $x$ o indicar que dicho número no aparece ninguna vez. Diseñe un algoritmo basado en la técnica ``divide y vencerás'' que resuelva el problema de la forma más eficiente posible. ¿Cuál es el orden de complejidad $O$ del algoritmo diseñado?
    \end{ejercicio}

    \begin{ejercicio}[2 puntos]
        Una empresa de transportes hace de forma regular una ruta con su flota de vehículos. A lo largo de esta ruta, se ha identificado un conjunto de $n$ puntos kilométricos específicos $(k_1, k_2, \ldots, k_n)$ en los que sus vehículos deben detenerse. Con el fin de facilitar las tareas de mantenimiento y control, se desea instalar un conjunto de estaciones a lo largo de la ruta. Cada estación puede cubrir los vehículos que se encuentran en un rango de $L$ kilómetros. Se desea identificar dónde ubicar el \textit{menor} conjunto de estaciones, garantizando que todos los puntos de parada prefijados queden cubiertos por al menos una estación. Diseñe un algoritmo voraz que resuelva este problema de forma óptima.
    \end{ejercicio}

    \begin{ejercicio}[2 puntos]
        Una empresa de mantenimiento eléctrico debe inspeccionar una red de $n$ transformadores interconectados.  Cada transformador está identificado por un número del 1 al $n$, y se conoce el conjunto de cables subterráneos que conectan pares de transformadores, representado mediante pares $(i,j)$. Por razones de seguridad, no es posible inspeccionar directamente los cables. En cambio, se debe enviar personal técnico a ciertos transformadores para monitorear los cables conectados a ellos desde ahí. Cada equipo técnico ubicado en un transformador puede inspeccionar todos los cables que están conectados directamente a ese transformador. Por ejemplo, si hay 7 transformadores y 7 cables conectados como sigue: $(1, 2), (1, 3), (3, 4), (3, 5), (4, 6), (5, 6), (6, 7)$, entonces ubicar personal técnico en los transformadores $\{1,3,6\}$ permite cubrir todos los cables de la red. Diseñe un algoritmo basado en la técnica de exploración en grafos que permita conocer el número \textit{mínimo} de transformadores donde se debe ubicar personal técnico para que cada cable esté supervisado al menos desde uno de sus extremos.
    \end{ejercicio}

    \begin{ejercicio}[2 puntos]
        Se dispone de $K$ euros para hacer la compra y tenemos una lista de $n$ posibles productos que podemos comprar. Cada producto $i$ tiene un precio, $p(i)$ (que será siempre un número entero), y una utilidad, $u(i)$. De cada producto podemos comprar como máximo 2 unidades. Además, tenemos una oferta según la cual la segunda unidad nos cuesta 1 euro menos. Queremos elegir los productos a comprar, y cuántos de cada tipo, \textit{maximizando} la utilidad de los productos comprados. Diseñe un algoritmo basado en la técnica de programación dinámica para resolver el problema. Aplíquelo, construyendo la tabla correspondiente, para el caso en que hay $n= 3$ productos, con precios $p = (3,4,6)$, utilidades $u=(7,8,11)$ y el presupuesto es $K=10$.
    \end{ejercicio}

    \newpage
    \setcounter{ejercicio}{0}

    \begin{ejercicio}[2 puntos]
        La ecuación en recurrencias resultante del análisis de eficiencia de un algoritmo depende de una condición dada $C$, de modo que tenemos $T(n) = 2T(\nicefrac{n}{4}) + n^2$ si $C$ es cierta, y $T(n) = 4T(\nicefrac{n}{2}) + \log_2(n)$ si $C$ es falsa. Determine los órdenes de complejidad $O$ y $\Omega$ del algoritmo y justifique si este tiene, o no, orden de complejidad $\Theta$.\\

        \noindent
        Primero resolvemos las dos recurrencias por separado. Para la primera:
        \begin{equation*}
            T(n) = 2T(\nicefrac{n}{4}) + n^2
        \end{equation*}
        Hacemos el cambio de variable $n = 2^h$:
        \begin{equation*}
            T\left(2^h\right) = 2T\left(2^{h-2}\right) + {\left(2^h\right)}^{2}
        \end{equation*}
        Escribiendo $t_h = T(2^h)$:
        \begin{equation*}
            t_h = 2t_{h-2} + 4^h \Longrightarrow t_h - 2t_{h-2} = 4^h
        \end{equation*}
        Obtenemos:
        \begin{itemize}
            \item $b = 4$, $q(h) = 1$, de grado 0; lo que nos da el término $(x-4)$.
            \item $p(x) = x^2-2 \Longrightarrow x = \pm \sqrt{2}$, lo que nos da los términos $\left(x-\sqrt{2}\right)\left(x+\sqrt{2}\right)$.
        \end{itemize}
        En definitiva, el polinomio característico de la recurrencia es:
        \begin{equation*}
            p(x) = \left(x-\sqrt{2}\right)\left(x+\sqrt{2}\right)(x-4)
        \end{equation*}
        Por lo que:
        \begin{equation*}
            t_h = C_1{\left(\sqrt{2}\right)}^{h} + c_2 {\left(-\sqrt{2}\right)}^{h} + c_34^h
        \end{equation*}
        Al deshacer el cambio de variable, tenemos $h = \log_2n$, por lo que:
        \begin{align*}
            T(n) &= c_1 {\left(\sqrt{2}\right)}^{\log_2n} + c_2 {\left(-\sqrt{2}\right)}^{\log_2n} + c_3 4^{\log_2n} \\
                 &= C-1 \sqrt{n} + c_2 {\left(-\sqrt{2}\right)}^{\log_2n} + c_3n^2
        \end{align*}
        De donde $T(n) \in O(n^2)$ si $C$ es cierta.\\

        \noindent
        Para la segunda recurrencia, procedemos de forma análoga:
        \begin{equation*}
            T(n) = 4T(\nicefrac{n}{2}) + \log_2n
        \end{equation*}
        Hacemos el cambio de variable $n = 2^h$:
        \begin{equation*}
            T\left(2^h\right) = 4T\left(2^{h-1}\right) + \log_2 2^n
        \end{equation*}
        Escribiendo $t_h = T\left(2^h\right)$:
        \begin{equation*}
            t_h = 4t_{h-1}+h \Longrightarrow t_h - 4t_{h-1} = h
        \end{equation*}
        Obtenemos:
        \begin{itemize}
            \item $b=1$, $q(h) = h$, de grado 1; lo que nos da el término ${(x-1)}^{2}$.
            \item $p(x) = (x-4)$
        \end{itemize}
        En definitiva, el polinomio característico de la recurrencia es:
        \begin{equation*}
            p(x) = (x-4){(x-1)}^{2}
        \end{equation*}
        Por lo que:
        \begin{equation*}
            t_h = c_1 4^h + c_2 1^h + c_3 h1^h = c_1 4^h + c_2 + c_3h
        \end{equation*}
        Si deshacemos el cambio, tenemos $h = \log_2n$, con lo que:
        \begin{align*}
            T(n) &= C_1 4^{\log_2n} + c_2 + c_3\log_2n \\
                 &= c_1 n^2 + c_2 + c_3 \log_2n
        \end{align*}
        De donde tenemos que $T(n)\in O(n^2)$ si $C$ es falsa.\\

        \noindent
        En ambos casos tenemos que $T(n) \in O(n^2)$, por lo que tanto el mejor como el peor caso coinciden, es decir, $T(n) \in O(n^2)$ y $T(n) \in \Omega(n^2)$. Como coinciden, tendremos también que $T(n) \in \Theta(n^2)$.
    \end{ejercicio}
    
    \begin{ejercicio}[2 puntos]
        Dado un vector de números enteros de tamaño $n$, ordenado de menor a mayor, y donde puede haber números duplicados, se quiere contar cuántas veces aparece un determinado número $x$ o indicar que dicho número no aparece ninguna vez. Diseñe un algoritmo basado en la técnica ``divide y vencerás'' que resuelva el problema de la forma más eficiente posible. ¿Cuál es el orden de complejidad $O$ del algoritmo diseñado?\\

        \noindent
        La idea consiste en modificar la búsqueda binaria para que localice la primera ocurrencia de $x$ y la última. Así, si la primera ocurrencia de $x$ está en la posición $p$ y la última está en la posición $u$, el número de ocurrencias de $x$ es $u-p+1$. Como se hacen dos búsquedas binarias, que tienen una eficiencia $O(\log_2n)$, el algoritmo resultante es $O(\log_2n)$. 

        Mostramos a continuación el pseudocódigo de lo que podría ser una implementación de esta idea:
        \begin{minted}{c++}
            cuantos(v[], inicio, fin, buscado){
               p = posicion_izquierda(v, inicio, fin, buscado);
               if(v[p] == buscado){
                  u = posicion_derecha(v, inicio, fin, buscado);
                  return u - p + 1;
               }else
                  return 0;
            }

            posicion_izquierda(v[], inicio, fin, buscado){
               resultado = -1;
               while(inicio <= fin){
                  mitad = (inicio + fin)/2;
                  if(v[mitad] == buscado){
                     resultado = mitad;
                     fin = mitad - 1;
                  }else if(buscado < v[mitad])
                     fin = mitad - 1;
                  else
                     inicio = mitad + 1;
               }

               return resultado;
            }

            posicion_derecha(v[], inicio, fin, buscado){
               resultado = -1;
               while(inicio <= fin){
                  mitad = (inicio + fin)/2;
                  if(v[mitad] == buscado){
                     resultado = mitad;
                     inicio = mitad + 1;
                  }else if(buscado < v[mitad])
                     fin = mitad - 1;
                  else
                     inicio = mitad + 1;
               }

               return resultado;
            }
        \end{minted}
    \end{ejercicio}

    \begin{ejercicio}[2 puntos]
        Una empresa de transportes hace de forma regular una ruta con su flota de vehículos. A lo largo de esta ruta, se ha identificado un conjunto de $n$ puntos kilométricos específicos $(k_1, k_2, \ldots, k_n)$ en los que sus vehículos deben detenerse. Con el fin de facilitar las tareas de mantenimiento y control, se desea instalar un conjunto de estaciones a lo largo de la ruta. Cada estación puede cubrir los vehículos que se encuentran en un rango de $L$ kilómetros. Se desea identificar dónde ubicar el \textit{menor} conjunto de estaciones, garantizando que todos los puntos de parada prefijados queden cubiertos por al menos una estación. Diseñe un algoritmo voraz que resuelva este problema de forma óptima.\\

        \noindent
        \begin{description}
            \item [Conjunto de candidatos.] Los $n$ puntos kilométricos específicos, $k_1, k_2, \ldots, k_n$ en los que los vehículos deben detenerse.
            \item [Conjunto de candidatos elegidos.]  La solución es una tupla $(x_1, x_2, \ldots, x_s)$ son $s \leq n$ que en cada $x_i$ (con $i \in \{1,\ldots,s\}$) contiene un punto kilométrico específico en el que se ubicuará una estación.
            \item [Función solución.]  Todos los puntos kilométricos específicos $k_1,k_2, \ldots, k_n$ están cubiertos por al menos una estación de las ubicadas en $x_1, x_2, \ldots, x_s$:
                \begin{equation*}
                    \forall k_i (i \in \{1,\ldots,n\})~\exists j \in \{1,\ldots,s\} \text{\ tal que\ } |k_i - x_j| \leq \nicefrac{L}{2}
                \end{equation*}
            \item [Función de factibilidad.] La selección de cualquier punto kilométrico específico para ubicar en él una estación siempre da lugar a una solución factible.
            \item [Función objetivo.] Número de puntos kilométricos específicos elegidos para ubicar en ellos una estación.
            \item [Función selección.] Considerando el punto kilométrico específico que cubrirá más a la izquierda de los restantes, se selecciona el punto kilométrico específico más a la derecha de este dentro del rango de cobertura permitida ($\nicefrac{L}{2}$), y se ubica ahí una estación.
        \end{description}
        La idea consiste en que siempre que se encuentre un punto kilométrico específico sin cubrir, se coloca una estación en el punto kilométrico específico más a la derecha dentro del rango de cobertura permitida ($\nicefrac{L}{2}$).
        \begin{minted}{c++}
            puntos[1..n];   // ordenados de menor a mayor
            x[];
            i = 1;
            num_estaciones = 0;
            while(i <= n){
               inicio = puntos[i];
               j = i;
               while(j <= n && puntos[j] <= inicio + L/2)
                  j++;
               estacion = puntos[j-1];
               num_estaciones++;
               x[num_estaciones] = estacion;
               while(i <= n && puntos[i] <= estacion + L/2)
                  i++;
            }
        \end{minted}

        \noindent
        \textbf{Demostración de la optimalidad.} Sea $X = \{x_1,\ldots,x_s\}$ la solución dada por el algoritmo voraz. Supongamos que existe otra solución $Y=\{y_1,\ldots,y_r\}$ tal que $r<s$. Sea $x_1$ la primera estación en la solución voraz. Por definición, cubre un conjunto de puntos kilométricos que vienen desde $k_1$ hasta $k_j \leq x_1 + \nicefrac{L}{2}$. La estación $x_1$ fue seleccionada porque es el punto kilométrico más a la derecha que cubre el primer punto kilométrico $k_1$, es decir, $x_1 \leq k_1 + \nicefrac{L}{2}$. Cualquier estación colocada más a la izquierda que $x_1$ cubre menos puntos, ya que el intervalo se desplaza a la izquierda. Por tanto, ninguna estación más a la izquierda que $x_1$ puede cubrir más puntos kilométricos que la que elige el algoritmo voraz. Ahora, supongamos que la solución óptima $Y$, usa una estación $y_1$ en lugar de $x_1$ para cubrir esos puntos. Como el algoritmo voraz elige la mejor (más a la derecha) estación que cubre desde $k_1$, $y_1 \leq x_1$. Por tanto, no cubre más puntos que $x_1$. Si repetimos el proceso, por cada atracción $y_i$ de la solución óptima que cubre un bloque que también cubre $x_i$, podemos reemplazar $y_i$ por $x_i$ sin perder puntos kilométricos cubiertos. Haciendo esta transformación convertimos la solución $Y$ en $X$, pero habíamos supuesto que $|Y| < |X|$, contradicción.
    \end{ejercicio}

    \begin{ejercicio}[2 puntos]
        Una empresa de mantenimiento eléctrico debe inspeccionar una red de $n$ transformadores interconectados.  Cada transformador está identificado por un número del 1 al $n$, y se conoce el conjunto de cables subterráneos que conectan pares de transformadores, representado mediante pares $(i,j)$. Por razones de seguridad, no es posible inspeccionar directamente los cables. En cambio, se debe enviar personal técnico a ciertos transformadores para monitorear los cables conectados a ellos desde ahí. Cada equipo técnico ubicado en un transformador puede inspeccionar todos los cables que están conectados directamente a ese transformador. Por ejemplo, si hay 7 transformadores y 7 cables conectados como sigue: $(1, 2), (1, 3), (3, 4), (3, 5), (4, 6), (5, 6), (6, 7)$, entonces ubicar personal técnico en los transformadores $\{1,3,6\}$ permite cubrir todos los cables de la red. Diseñe un algoritmo basado en la técnica de exploración en grafos que permita conocer el número \textit{mínimo} de transformadores donde se debe ubicar personal técnico para que cada cable esté supervisado al menos desde uno de sus extremos.\\

        \noindent
        Vamos a diseñar un algoritmo de vuelta atrás (backtracking). La solución se va a representar como una tupla $X = (x_1,x_2,\ldots,x_n)$, donde cada $x_i$ tendrá un valor de 0 para indicar que no se enviará un técnico al transformador $i-$ésimo, o de 1 para indicar que sí se enviará un técnico al transformador $i-$ésimo.
        \begin{description}
            \item [Restricciones explícitas.] $x_i \in \{0,1\}$, $\forall i \in \{1,\ldots,n\}$.
            \item [Restricciones implícitas.] Sea $M[i][j]$ la matriz de adyacencia con valores 1 y 0 para indicar que existe un cable (1) o no (0) entre los transformadores $i$ y $j$, entonces:
                \begin{equation*}
                    \forall i,j \text{\ tales que\ } M[i][j] = 1 \Longrightarrow x_i = 1 \lor x_j = 1
                \end{equation*}
            \item [Árbol de estados.] Árbol binario de altura $n+1$. En cada nivel $i$, se decide si se manda (1) o no (0) un técnico al transformador $i-$ésimo.

\begin{tikzpicture}[
  level distance=2cm,
  sibling distance=1.5cm,
  every node/.style={circle, draw},
  edge from parent/.style={draw, -latex},
  level 1/.style={sibling distance=5cm},
  level 2/.style={sibling distance=2cm}
]

\node {}
  child {node {$x_1=1$}
    child {node {\shortstack{$x_1 = 1$\\$x_2=1$}}}
    child {node {\shortstack{$x_1 = 1$\\$x_2=0$}}}
  }
  child {node {$x_1=0$}
    child {node {\shortstack{$x_1 = 0$\\$x_2=1$}}}
    child {node {\shortstack{$x_1 = 0$\\$x_2=0$}}}
  };

\end{tikzpicture}

            \item[Función objetivo.] Número de técnicos enviados a los transformadores, $\sum_{i=1}^n x_i$.
            \item [Función de acotación.] Función que poda el árbol si al añadir a la solución parcial $x_1,\ldots,x_i$ el valor correspondiente de $x_{i+1}$ no se satisfacen las restricciones implícitas. También se poda si el numero de técnicos asignados a los transformadores de la solución parcial es igual o mayor que el de la mejor solución encontrada hasta el momento.
        \end{description}
        Una implementación en pseudocódigo de esta solución es la siguiente:
        \begin{minted}{c++}
            transformadores(x[1..n], M[1..n][1..n], mejor_solucion, k){
               if(k == n+1){   // nodo hoja (solución completa)
                  if(num_tecnicos(x,n) < num_tecnicos(mejor_solucion, n))
                     mejor_solucion = x;
               }else{
                  x[k] = 1; // asignar trabajador al transformador k-ésimo
                  if(num_tecnicos(x,k) < num_tecnicos(mejor_solucion, k))
                     transformadores(x, M, mejor_solucion, k+1);

                  x[k] = 0; // no se asigna un trabajador al k-ésimo transformador
                  if(factible(x,M,k)) // si se cumplen las restricciones implícitas
                     transformadores(x, M, mejor_esolucion, k+1);
               }
            }

            // devuelve si se satisfacen o no las restricciones implícitas
            factible(x[1..n], M[1..n][1..n], k) {
               for(i = 1; i <= k; i++){
                  if(M[k][i] == 1)
                     if(x[i] == 0)
                        return false;
               }
               return true;
            }

            // devuelve el nº de técnicos asignados a transformadores
            num_tecnicos(x[1..n], k){
               tecnicos = 0;
               for(i = 1; i <= k; i++)
                  tecnicos = tecnicos + x[i];
               return tecnicos;
            }
        \end{minted}
    \end{ejercicio}

    \begin{ejercicio}[2 puntos]
        Se dispone de $K$ euros para hacer la compra y tenemos una lista de $n$ posibles productos que podemos comprar. Cada producto $i$ tiene un precio, $p(i)$ (que será siempre un número entero), y una utilidad, $u(i)$. De cada producto podemos comprar como máximo 2 unidades. Además, tenemos una oferta según la cual la segunda unidad nos cuesta 1 euro menos. Queremos elegir los productos a comprar, y cuántos de cada tipo, \textit{maximizando} la utilidad de los productos comprados. Diseñe un algoritmo basado en la técnica de programación dinámica para resolver el problema. Aplíquelo, construyendo la tabla correspondiente, para el caso en que hay $n= 3$ productos, con precios $p = (3,4,6)$, utilidades $u=(7,8,11)$ y el presupuesto es $K=10$.\\

        \noindent
        \begin{description}
            \item [Definición de la solución.] Se plantea la solución como una secuencia de decisiones $x_1,x_2,\ldots,x_n$, donde cada $x_i$ indica cuántas unidades se compran del producto $i-$ésimo.

                Sea $U(i,j)$ La máxima utilidad que se puede conseguir con $j$ euros cuando se consideran los productos del 1 al $i$, la solución viene dada por $U(n,K)$.
            \item [Verificación del principio de optimalidad.] Si $x_1,x_2,\ldots,x_n$ es óptima para $U(n,K)$, entonces hay que demostrar que $x_1,x_2,\ldots,x_{n-1}$ es óptima para:
                \begin{itemize}
                    \item $U(n-1,K)$ si $x_n = 0$.
                    \item $U(n-1, K-p(n))$ si $x_n = 1$.
                    \item $U(n-1,K-2p(n)+1)$ si $x_n = 2$.
                \end{itemize}
                Se demuestra por reducción al absurdo. Para el caso $x_n = 0$, si no fuese así, entonces existiría una solución $y_1,\ldots,y_{n-1}$ de forma que:
                \begin{equation*}
                    \sum_{i=1}^{n-1} y_i u(i) > \sum_{i=1}^{n-1}x_i u(i), \quad \text{con} \quad \sum_{i=1}^{n-1}y_ip(i) \leq K
                \end{equation*}
                Pero entonces, haciendo $y_n = x_n = 0$, tenemos que:
                \begin{equation*}
                    \sum_{i=1}^{n} y_ip(i) \leq K
                \end{equation*}
                Luego $y_1,\ldots,y_n$ es una solución para $U(n,K)$ y además:
                \begin{equation*}
                    \sum_{i=1}^{n}y_i u(i) > \sum_{i=1}^{n}x_i u(i)
                \end{equation*}
                Por lo que $x_1,\ldots,x_n$ no sería óptima, contradicción. Los casos $x_n \in \{1,2\}$ se demustran de forma análoga.
            \item[Definición recursiva de la solución óptima.] Definimos:
                \begin{equation*}
                    {\scriptsize U(i,j) = \left\{\begin{array}{ll}
                            0 & \text{si\ } i = 0 \lor j = 0 \\
                            U(i-1,j) & \text{si\ } p(i) > j \\
                            \max\{U(i-1,j),U(i-1,j-p(i)) + u(i)\} & \text{si\ } 2p(i)-1 > j \\
                            \max\{U(i-1,j),U(i-1,j-p(i))+u(i), U(i-1,j-2p(i)+1)+2u(i)\} & \text{en otro caso}
                    \end{array}\right.}
                \end{equation*}
        \end{description}
        Por lo que una implementación en pseudocódigo del algoritmo sería:
        \begin{minted}{c++}
            for i = 0 to n:
               U(i,0) = 0;
            for j = 1 to n:
               U(0,j) = 0;
            for i = 1 to n:
               for j = 1 to K:
                  U(i,j) = U(i-1,j);
                  V(i,j) = 0;    // 0 unidades
                  if(p(i) <= j)
                     if(U(i-1,j-p(i)) + u(i) > U(i,j))
                        U(i,j) = U(i-1,j-p(i)) + u(i);
                        V(i,j) = 1;    // 1 unidad
                  if(2p(i) + 1 <= j)
                     if(U(i-1,j-2p(i)+1) + 2*u(i) > U(i,j))
                        U(i,j) = U(i-1,j-2*p(i)+1) + 2*u(i);
                        V(i,j) = 2;    // 2 unidades

            // La tabla V se utiliza para recuperar la solución:
            i = n
            j = K
            while(i >= 1)
               solucion(i) = V(i,j);
               if(V(i,j) == 2)
                  j = j - 2*p(i) + 1;
               else
                  j = j - V(i,j)*p(i);
               i--;
        \end{minted}

        Construimos las tablas $U$ y $V$ para el caso $n = 3$, $K= 10$, $p=(3,4,6)$ y $u = (7,8,11)$:
        \begin{table}[H]
        \centering
        \begin{tabular}{c|ccccccccccc}
            $U$ & 0 & 1 &2 &3 &4 &5 &6 &7 &8 &9 &10 \\
            \hline
            0 & 0 &0 &0 &0 &0 &0 &0 &0 &0 &0 &0 \\
            1 & 0 &0 &0 &7 &7 &14 &14 &14 &14 &14 &14 \\
            2 & 0 &0 &0 &7 &8 &14 &14 &16 &16 &22 &23 \\
            3 & 0 &0 &0 &7 &8 &14 &14 &16 &16 &22 &23 
        \end{tabular}
        \end{table}

        \begin{table}[H]
        \centering
        \begin{tabular}{c|cccccccccc}
            $V$ & 1 &2 &3 &4 &5 &6 &7 &8 &9 &10 \\
            \hline
            1 & 0 &0 &1 &1 &2 &2 &2 &2 &2 &2 \\
            2 & 0 &0 &0 &1 &0 &0 &2 &2 &1 &2 \\
            3 & 0 &0 &0 &0 &0 &0 &0 &0 &0 &0 
        \end{tabular}
        \end{table}
    \end{ejercicio}

\end{document}
