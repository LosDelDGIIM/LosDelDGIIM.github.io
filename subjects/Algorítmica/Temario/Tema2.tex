\chapter{Algoritmos Divide y Vencerás}
Dado un problema $P$, lo dividimos en subproblemas que han de ser del mismo tipo: $P = \{P_i\}_{i\in I}$. Encontramos las soluciones de todos estos problemas: $S_i \mid i \in I$ y con estas, las juntamos de forma adecuada para obtener la solución al problema $P$: $S = \{S_i\}_{i \in I}$. Notemos que podemos hacerlo de forma recursiva, podemos aplicar la técnica de divide y vencerás a los subproblemas. Debemos fijar un caso base, donde el problema sea suficientemente pequeño como para resolverlo.

Ejemplos de algoritmos que implementan la búsqueda binaria son Quicksort, Mergesort, Búsqueda Binaria, $\ldots$

Problemas como el de las Torres de Hanoi pueden resolverse mediante esta técnica de resolución de problemas.

Tratamos cuestiones importantes:
\begin{itemize}
    \item Cómo descomponer un problema en subproblemas.
    \item Cómo resolver subproblemas.
    \item Cómo combinar las soluciones.
    \item ¿Mereció la pena aplicar esta técnica?
\end{itemize}

Para responder al último punto observemos el siguiente ejemplo, que trata un problema de forma genérica.
\begin{ejemplo}
    Supongamos un problema $P$ de tamaño $n$ que sabemos que puede resolverse con un algoritmo (básico) $A$, con:
    \begin{equation*}
        t_A(n) \leq cn^2 
    \end{equation*}

    Dividimos $P$ en 3 subproblemas de tamaño $\frac{n}{2}$, isneod cada uno de ellos del mismo tipo que $A$ y consumiendo un tiempo lineal la combinación de sus soluciones: $t(n) \leq dn$.
    Tenemos, por tanto, un nuevo algoritmo $B$, Divide y Vencerás, que consumirá un tiempo de:
    \begin{equation*}
        t_B(n) = 3t_A\left(\dfrac{n}{2}\right)+t(n) \leq 3t_A\left(\dfrac{n}{2}\right) + dn \leq \left(\dfrac{3c}{4}\right)n^2 + dn
    \end{equation*}
    $B$ tiene un tiempo de ejecución mejor que el algoritmo $A$, ya que reduce la constante oculta.

    Si volvemos a reducir el tamaño del subproblema, obtenemos un nuevo algoritmo $C$, que tendría un tiempo:
    \begin{equation*}
        t_c(n) = \left\{ \begin{array}{ll}
            t_A(n) & \text{si\ } n \leq n_0 \\
            3t_C\left(\dfrac{n}{2}\right) + t(n) & \text{si\ } n > n_0
        \end{array}\right.
    \end{equation*}
    $C$ es mejor en eficiencia que los algoritmos $A$ y $B$ ($t_c(n) \leq bn^{1.58}$). 
\end{ejemplo}
Al valor $n_0$ se le denomina umbral y al algoritmo que se ejecuta debajo de este umbral, \verb|ad hoc| son fundamentales para que la técnica funcione bien.

Requisitos lógicos para usar Divide y Vencerás:
\begin{itemize}
    \item El problema se tiene que poder reducir a subproblemas del mismo tipo.
    \item Los subproblemas han de poder resolverse de manera independiente.
    \item Las soluciones de los subproblemas no deben afectar entre sí, no debe existir solapamiento entre subproblemas.
    \item Ha de poderse juntar las soluciones de los subproblemas para resolver el problema.
\end{itemize}
Una forma matemática de entender el punto 3 es decir que los subproblemas deben formar una \ul{partición} del problema.

Para que la técnica sea eficiente:
\begin{itemize}
    \item Selección cuidadosa de cuando usar el algoritmo \verb|ad hoc|.
    \item El número de subproblemas debe ser razonablemente pequeño (para no reducir mucho la eficiencia).
    \item Los subproblemas deben tener el menor tamaño posible.
\end{itemize}

Normalmente, al aplicar Divide y Vencerás nos encontraremos con eficiencias del estilo:
\begin{equation*}
    T(n) = \left\{ \begin{array}{ll}
            t(n) & \text{si\ } n \leq n_0\\
            IT\left(\dfrac{n}{b}\right) + G(n) & \text{si\ }n > n_0
    \end{array}\right.
\end{equation*}
Donde $I$ es el número de subproblemas, $\dfrac{n}{b}$ el tamaño de estos y:
\begin{equation*}
    G(n) = D(n) + C(n)
\end{equation*}
Con $D(n)$ el tiempo de dividir el problema en subproblemas y $C(n)$ el tiempo de combinación.
$t(n)$ era el tiempo del algoritmo \verb|ad hoc|.


\begin{ejemplo}
    Dado un vector de elementos, determinar la posición que ocupa el máximo del mismo.

    Proponemos como solución el siguiente código:

\begin{listing}[H]
    \begin{minted}[xleftmargin=2cm]{c++}
        int Maximo(int* v, int inf, int sup){
            if(sup == inf) return v[inf];
            else{
                int med = (inf+sup)/2;
                int izda = Maximo(v, inf, med);
                int dcha = Maximo(v, med+1, sup);

                return max(izda, dcha);
            }
        }
    \end{minted}
\end{listing}

Con un tiempo de ejecución de:
\begin{equation*}
    T(n) = 2T\left(\dfrac{n}{2}\right) +1 \in O(n)
\end{equation*}

Otra solución sería:
\begin{listing}[H]
    \begin{minted}[xleftmargin=2cm]{c++}
        int Maximo2(int* v, int inf, int sup){
            if(sup == inf) return v[inf];
            else{
                int izda = Maximo2(v, inf, sup);

                return max(izda, sup);
            }
        }
    \end{minted}
\end{listing}

En este caso:
\begin{equation*}
    T(n) = T(n-1)+1 \in O(n)
\end{equation*}

Los dos del mismo orden, pero encontramos un problema entre uno y otro: el número de llamadas recursivas:
\begin{itemize}
    \item \verb|Maximo2| Realiza $n$ llamadas recursivas, pero el tamaño de la pila ocupado es de $n$.
    \item \verb|Maximo| Realiza $n$ llamadas recursivas, pero el tamaño de la pila es $\log n$.
\end{itemize}
El número de los elementos de la pila de \verb|Maximo| puede razonarse dibujándolo con un árbol.

Por tanto, el primer algoritmo que planteamos era mejor ya que usa menos pila (evitando su desbordamiento).

\end{ejemplo}
\begin{observacion}
    Notemos que este ejemplo ha sido un caso instructivo, es más eficiente resolverlo de la forma canónica que aplicando Divide y Vencerás.
\end{observacion}

\begin{ejercicio*}
    Dada una secuencia de números, definimos \ul{pico} al valor $i$ tal que $x[i-1] < x[i] > x[i+1]$ y \ul{valle} al valor $j$ tal que $x[j-1]>x[j]<x[j+1]$. Un pico $i$ es consecutivo a un vallo $j$ si ningún valor entre $i$ y $j$ es pico o valle.

    Diseñar un algoritmo que permita encontrar la mayor diferencia entre un pico y un valle.
\end{ejercicio*}

\begin{ejercicio*}
    Un camión va desde Granada a Moscú siguiendo una ruta determinada. La capacidad del tanque de combustible es $C$ y conocemos el consumo por kilómetros del camión. Conocemos las gasolineras que se encuentran en la ruta y la distantica entre ellas. Se pide minimizar el número de paradas que hace el conductor.


\end{ejercicio*}
