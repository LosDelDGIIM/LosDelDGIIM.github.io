\chapter{Introducción}
test2

% // TODO: Pasar esto a limpio
% Una geometría es el estudio cuantitativo de las formas y su evolución para ciertas figuras de un espacio
% 
%         Espacio ambiente                     | Figuras                                          | Soporte Matemático                                                                        |
%         -------------------------------------|--------------------------------------------------|-------------------------------------------------------------------------------------------|
% Gª I:   Espacio vectorial                    | Subespacios vectoriales: rectas, planos, ...     | Espacio vectorial, intersecciones y sumas                                                 |
% Gª II:  Espacio vect euclideo                | Subespacios vectoriales: rectas, planos, ...     | Espacio vectorial con una métrica, intersecciones, sumas, perpendicularidad, medir ángulos|
% Gª III: Espacio afin euclideo                | Subespacios afines: puntos, rectas, planos, ...  | Espacio afin con una métrica en el esp. vect. asociado                                    |
%                                              | cónicas y cuádricas                              | intersecciones, sumas, perpendicularidad, medir ángulos, medir distancias                 |
% Tª I:   Ninguno o un e.t.                    | Subespacios topológicos                          | Topología en un conjunto, determinar homeomorfismos                                       |
% Tª II:  Ninguno o un e.t.                    | Subespacios topológicos                          | Topología en un conjunto, determinar homeomorfismos, teoría de grupos                     |
% CyS:    Curvas en R² o R³, Superficies en R³ | Curvas en R² o R³, Superficies en R³             | Todas las anteriores + análisis                                                           |
% 
% En Gª I y II los soportes matemáticos tienen que ver con polinomios de primer grado homogéneos
% En Gª III son polinomios de primer y segundo grado, se va a la geometría algebraica
% 
% CyS / Introducción a la Geometría Diferencial (Análisis Geométrico)
% 
%
% Cronológicamente:
% GII y GIII, puntos, rectas, planos, ...
% CyS, curvas y superficies, sigo XVII
% Topologías, siglo XX
% GI, espacio vectorial
%
%
% CyS:
% epicicloide: curva que describe un punto de una rueda al girar
% 
% Protagonistas:
% siglo 17: Leibniz, Newton
% siglo 18: Euler, Monge
% siglo 19: Frenet, Senet, Escuela Francesa, Cauchy
% siglo 20: Darboux, Cartan

% // TODO: Empieza aquí
\chapter{Curvas en el plano y en el espacio}
\section{Definición. Curva regular. Longitud de arco}
\begin{definicion}
    Una \textbf{curva} en el espacio es una aplicación diferenciable $\alpha:I\to \mathbb{R}^3$ donde $I\subseteq \mathbb{R}$ es un intervalo abierto.\\

    \noindent
    Normalmente, si queremos señalar explícitamente las componentes de $\alpha$, escribiremos:
    \begin{equation*}
        \alpha(t) = (x(t), y(t), z(t))
    \end{equation*}
    En realidad, a un geómetra solo le interesa la llamada ``traza de la curva'':
    \begin{equation*}
        \tr \alpha = \im \alpha = \alpha(I)
    \end{equation*}
\end{definicion}

\begin{ejemplo}
    Sean $a\in \mathbb{R}^3$, $v\in \mathbb{R}^3$, definimos $\alpha:\mathbb{R}\to \mathbb{R}^3$ dada por:
    \begin{equation*}
        \alpha(t) = a + tv 
    \end{equation*}
    En dicho caso:
    \begin{equation*}
        \tr \alpha = \left\{\begin{array}{ll}
            \{a\} & \text{si\ } v=0 \\
             R_{a,v} & \text{si\ } v\neq 0
        \end{array}\right. 
    \end{equation*}
    donde denotamos por $R_{a,v}$ a la única recta que pasar por $a$ y con dirección $v$:
    \begin{equation*}
        R_{a,v} = a + \langle v \rangle 
    \end{equation*}
\end{ejemplo}

\begin{definicion}
    Una curva $\alpha:I\to \mathbb{R}^3$ se llama ``\textbf{plana}'' si existe un plano $P\subset \mathbb{R}^3$ tal que $\tr \alpha \subset P$.\\

    \noindent
    Como $P$ y $P(z=0)$ son equivalentes salvo un movimiento rígido\footnote{Es decir, una aplicación afín que conserve longitudes y ángulos.} de $\mathbb{R}^3$, podemos considerar que la curva $\alpha$ está definida como $\alpha:I\to P(z=0)\subset \mathbb{R}^3$, cuyas componentes son:
    \begin{equation*}
        \alpha(t) = (x(t), y(t), 0) \equiv (x(t), y(t))
    \end{equation*}
    De esta forma, podemos abstraernos y pensar que una curva plana es una aplicación diferenciable $\alpha:I\to \mathbb{R}^2$.
\end{definicion}~\\

\noindent
En los libros es común llamar a estar curvas ``parametrizadas'' y ``diferenciables''. En nuestra definición imponemos que una curva ha de ser diferenciable, y el adjetivo parametrizada se debe a la dependencia de la variable independiente.

% // TODO: Buscar software para dibujar curvas en R³.

\begin{ejemplo}
    Varios ejemplos de curvas:
    \begin{enumerate}
        \item Extrapolando el ejemplo anterior:
            \begin{equation*}
                \alpha(t) = a + tv
            \end{equation*}
            para $a\in \mathbb{R}^2, v\in \mathbb{R}^2$ y $\alpha'(t) = v$. Se trata de un Movimiento Rectilíneo Uniforme. Ahora, vemos que $\alpha''(t) = 0$, no hay aceleración ninguna.
        \item Tomando $a\in \mathbb{R}^2$ y $v\in \mathbb{R}^2\setminus \{0\}$, consideramos $\beta:\mathbb{R}\to \mathbb{R}^2$ dada por:
            \begin{equation*}
                \beta(t) = a+t^2v
            \end{equation*}
            Observamos ahora que tenemos $\tr \beta\subset \tr \alpha$, y la traza de esta curva es la semirrecta de extremo $a$ y dirección $v$:
            \begin{equation*}
                \tr \beta = R_0^+(a,v) 
            \end{equation*}

            Tenemos $\beta'(t) = 2tv$, la velocidad depende de $t$, cuanto más nos acercamos a $a$ de forma negativa vamos frenando, en $a$ estamos a velocidad $0$ y luego la velocidad aumenta.

            $\beta''(t) = 2v$, la aceleración es constante, se trata de un Movimiento Rectilíneo Uniformemente Acelerado.
    \end{enumerate}
\end{ejemplo}

\begin{ejercicio} % // TODO: HACER
    Estudiar $\gamma(t) = a+t^3 v$, con $a\in \mathbb{R}^2$ y $v\in \mathbb{R}^2\setminus \{0\}$.
\end{ejercicio}

\begin{ejemplo}
    \begin{enumerate}
        \item Consideramos ahora $\delta:\mathbb{R}\to \mathbb{R}^2$ dada por:
            \begin{equation*}
                \delta(t) = (t^3-4t, t^2-4)
            \end{equation*}
            Se trata de una curva con autointersecciones. Folio 1. 

            Su velocidad es $\delta'(t) = (3t^2-4,2t)$, que no se anula en ningún punto.
        \item Si consideramos $\varepsilon:\mathbb{R}\to \mathbb{R}^2$ dada por $\varepsilon(t) = (t^3,t^2)$.

            Su velocidad es $\varepsilon'(t) = (3t^2, 2t)$, que se anula en el origen. Observamos que en el dibujo de la traza vemos un pico.

            Aunque $\varepsilon$ sea diferenciable, $\tr \varepsilon$ tiene ``picos''. Observamos además que $\im \varepsilon$ es la gráfica de la aplicación\footnote{Lo hemos obtenido igualando $x=t^2$, $y=t^3$, despejando $t$ de la segunda e igualando en la primera.} $y=x^{\nicefrac{2}{3}}$. Esta función no es derivable en el origen.
    \end{enumerate}
\end{ejemplo}
