\subsection{Ejercicios adicionales}
\noindent
Los siguientes ejercicios provienen en gran parte de los apuntes de Javier Pérez, cuyos apuntes pueden encontrarse en la bibliografía de la asignatura. Contamos además con ejercicios pendientes del Capítulo 1 que son interesantes de hacer o de conocer.

\begin{ejercicio}
    Todo espacio de Hilbert es estrictamente convexo.\\

    \noindent
    Recordamos que un espacio normado (o un conjunto convexo dentro de un espacio normado) es estricatamente convexo si siempre que tengamos dos elementos $u$ y $v$ distintos con $\|u\| = 1 = \|v\|$, entonces tendremos que:
    \begin{equation*}
        \|tu + (1-t)v\| < 1 \qquad \forall t\in \left]0,1\right[
    \end{equation*}~\\

    \noindent
    Sea $H$ un espacio de Hilbert y sean $u,v\in H$ con $u\neq v$ y $\|u\| = 1 = \|v\|$, dado $t\in [0,1]$ calculamos:
    \begin{align*}
        \|tu + (1-t)v\| &= \langle tu+(1-t)v, tu+(1-t)v \rangle  = t^2\langle u,u \rangle  + {(1-t)}^{2}\langle v,v \rangle  + 2\langle tu,(1-t)v \rangle  \\
        &= t^2\|u\|^2 + {(1-t)}^{2}\|v\|^2 + 2t(1-t)\langle u,v \rangle \stackrel{(\ast)}{\leq} t^2 + {(1-t)}^{2}+ 2t(1-t)\|u\|\|v\| \\
        &= t^2+ {(1-t)}^{2} + 2t(1-t) = t^2 + t^2 + 1 - 2t + 2t - 2t^2 =  1
    \end{align*}
    Donde en $(\ast)$ hemos usado la desigualdad de Cauchy-Schwartz. Sabemos que en dicha desigualdad se da la igualdad si y solo si $u$ y $v$ son linealmente dependientes de forma positiva. De esta forma, tenemos que:
    \begin{equation*}
        \|tu + (1-t)v\| \geq 1 \Longleftrightarrow \exists \lm\in \mathbb{R}^+ : u = \lm v
    \end{equation*}
    Suponiendo que estamos en este caso, existe $\lm\in \mathbb{R}$ de forma que $u = \lm v$, luego tenemos que:
    \begin{equation*}
        1 = \|u\| = \|\lm v\| = \lm \|v\| = \lm \Longrightarrow \lm = 1
    \end{equation*}
    Pero teníamos que $u\neq v$, por lo que este caso es imposible, luego tenemos que:
    \begin{equation*}
        \|tu + (1-t)v\| < 1 \qquad \forall t\in \left]0,1\right[
    \end{equation*}
    Por lo que $H$ es estrictamente convexo.
\end{ejercicio}

\begin{ejercicio}
    Sean $X\neq \{0\}$ e $Y$ espacios normados. Si el espacio $L(X,Y)$ es completo, entonces $Y$ es completo.\newline
    (\textbf{Pista:} Considerar $T_n(x) = f(x)y_n$.)\\

    \noindent
    Sea $\{y_n\}$ una sucesión de Cauchy de puntos de $Y$, como $X\neq \{0\}$ podemos encontrar $0\neq x_0\in X$. Si consideramos $g:\mathbb{R}x_0\to \mathbb{R}$ dada por:
    \begin{equation*}
        g(tx_0) = t\|x_0\| \qquad \forall t\in \mathbb{R}
    \end{equation*}
    Tenemos que $g$ es lineal y continua:
    \begin{gather*}
        g(\lm tx_0 + \mu x_0) = g((\lm t + \mu)x_0) = (\lm t + \mu)\|x_0\| = \lm g(tx_0) + g(\mu x_0) \qquad \forall \lm,t,\mu\in \mathbb{R} \\
        |g(tx_0)| = |t\|x_0\|| = |t|\|x_0\| = \|tx_0\|
    \end{gather*}
    Si consideramos $p:E\to \mathbb{R}$ dada por:
    \begin{equation*}
        p(x) = \|g\|\|x\| \qquad \forall x\in E
    \end{equation*}
    Tenemos entones que:
    \begin{gather*}
        p(x+y) = \|g\|\|x + y\| \leq \|g\| (\|x\| + \|y\|) = p(x) + p(y) \qquad \forall x,y\in X \\
        g(tx_0) \leq |g(tx_0)| \leq \|g\|\|tx_0\| = p(tx_0) \qquad \forall t\in \mathbb{R}
    \end{gather*}
    Estamos en las condiciones del Teorema de Hahn-Banach, por lo que existe $f\in E^\ast$ de forma que $f\big|_{\mathbb{R}x_0} = g$. En particular, tenemos que:
    \begin{equation*}
        f(x_0) = g(x_0) = 1 \neq 0
    \end{equation*}
    Si consideramos ahora la sucesión de aplicaciones $\{T_n\}$ donde:
    \begin{equation*}
        T_n(x) = f(x)y_n \qquad \forall x\in E, \quad \forall n\in \mathbb{N}
    \end{equation*}
    Tenemos que $T_n\in L(X,Y)$, ya que:
    \begin{itemize}
        \item Para cada $n\in \mathbb{N}$ $T_n$ es lineal:
            \begin{multline*}
                T_n(\lm x + y) = f(\lm x + y)y_n = (\lm f(x) + f(y))y_n = \lm f(x)y_n + f(y)y_n = \lm T_n(x) + T_n(y) \\ \forall x,y\in X, \quad \forall \lm\in \mathbb{R}
            \end{multline*}
        \item Para cada $n\in \mathbb{N}$ tenemos que $T_n$ es continua:
            \begin{equation*}
                \|T_n(x)\| = \|f(x)y_n\| = \|y_n\||f(x)| \leq \|y_n\|\|f\|\|x\| \qquad \forall x\in X
            \end{equation*}
    \end{itemize}
    Además, $\{T_n\}$ es de Cauchy, pues si $n,m\in \mathbb{N}$ entonces:
    \begin{align*}
        \|T_n - T_m\| &= \sup_{\|x\|\leq 1}\|T_n(x) - T_m(x)\| = \sup_{\|x\|\leq 1}\|f(x) (y_n - y_m)\| \\ &= \|y_n - y_m\| \sup_{\|x\|\leq 1}|f(x)| = \|y_n - y_m\|\|f\| 
    \end{align*}
    Y teníamos que la sucesión $\{y_n\}$ era de Cauchy. Como $L(X,Y)$ es completo, ha de existir $T\in L(X,Y)$ de forma que $\{T_n\}\to T$. Si consideramos ahora (recordemos que $f(x_0)\neq 0$):
    \begin{equation*}
        y = \frac{T(x_0)}{f(x_0)} \in Y
    \end{equation*}
    Vemos finalmente que:
    \begin{align*}
        \|y_n - y\| &= \frac{1}{|f(x_0)|}\|f(x_0)y_n - f(x_0)y\| = \frac{1}{|f(x_0)|}\|T_n(x_0) - T(x_0)\| \\
                    &= \frac{1}{|f(x_0)|} \|(T_n-T)(x_0)\| \leq \frac{\|x_0\|}{|f(x_0)|} \|T_n-T\|
    \end{align*}
    Por lo que $\{y_n\}\to y$, de donde $Y$ es completo.
\end{ejercicio}

\begin{ejercicio}
    Sean $X$ e $Y$ espacios normados, dados $a\in X$ con $a\neq 0$ y $b\in Y$, prueba que existe un operador $T\in L(X,Y)$ tal que $T(a) = b$ y $\|T\|\|a\| = \|b\|$.\\

    \noindent
    Podemos suponer sin pérdida de generalidad que $b\neq 0$, pues si $b=0$ el operador $T$ constantemente igual a $0$ verifica todas las propiedades del enunciado.\\

    \noindent
    Como $a\neq 0$, tenemos que $\mathbb{R} a$ es un subespacio vectorial de $X$ de dimensión $1$. Definimos $g:\mathbb{R}a\to \mathbb{R}$ dada por:
    \begin{equation*}
        g(ta) = t\|b\| \qquad \forall t\in \mathbb{R}
    \end{equation*}
    Que es lineal y continua:
    \begin{itemize}
        \item $g(\lm ta + \mu a) = g((\lm t + \mu)a) = (\lm t + \mu)\|b\| = \lm g(ta) + g(\mu a) \quad \forall \lm,t,\mu\in \mathbb{R}$.
        \item Si $t\in \mathbb{R}$:
            \begin{equation*}
                |g(ta)| = |t|\|b\| = |t| \frac{\|b\|}{\|a\|} \|a\| = \frac{\|b\|}{\|a\|}\|ta\|
            \end{equation*}
    \end{itemize}
    En particular, en el segundo punto hemos probado que $\|g\| = \nicefrac{\|b\|}{\|a\|}$. Por el Corolario~\ref{coro:hahn-banach} de Hahn-Banach, existe $f\in X^\ast$ con $\|f\| = \|g\|$ de forma que $f\big|_{\mathbb{R}a} = g$. Si consideramos ahora la aplicación $h:\mathbb{R}\to Y$  dada por:
    \begin{equation*}
        h(t) = t\frac{b}{\|b\|} \qquad \forall t\in \mathbb{R}
    \end{equation*}
    Tenemos que $h$ es claramente lineal y continua. Consideramos ahora la aplicación $T = h\circ f:X\to Y$, que es una aplicación lineal y continua como composición de dos aplicaciones lineales y continuas. Veamos que $T$ verifica las dos condiciones pedidas:
    \begin{equation*}
        T(a) = (h\circ f)(a) = h(f(a)) = h(g(a)) = h(\|b\|) = \|b\|\frac{b}{\|b\|} = b
    \end{equation*}
    Y además:
    \begin{align*}
        \|T\| &= \sup_{\|x\|\leq 1}\|T(x)\| = \sup_{\|x\|\leq 1} \|h(f(x))\| = \sup_{\|x\|\leq 1}\left\|f(x)\frac{b}{\|b\|}\right\| = \sup_{\|x\|\leq 1}\|f(x)\| = \|f\|  = \frac{\|b\|}{\|a\|}
    \end{align*}
\end{ejercicio}

\begin{ejercicio}
    Sea $X$ un espacio normado separable. Prueba que existe un conjunto $\{x_n^\ast : n\in \mathbb{N}\}\subset X^\ast$ tal que para todo $x\in X$ se verifica que $\|x\| = \sup\limits_{n\in \mathbb{N}}|x_n^\ast(x)|$
\end{ejercicio}
