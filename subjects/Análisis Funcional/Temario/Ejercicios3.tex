\section{Topologías Débiles}

\begin{ejercicio}
    Sea $E$ un espacio de Banach y sea $A\subset E$ un subconjunto que es compacto en la topología débil $\sigma(E,E^\ast)$. Prueba que $A$ es acotado.\\

    \noindent
    Usaremos el Corolario~\ref{coro:entonces_B_acotado}:
    \begin{center}
        Si $f(A)$ está acotado $\forall f\in E^\ast \Longrightarrow A$ está acotado.
    \end{center}
    Si $f\in E^\ast$, tenemos que $A$ es compacto y $f$ continua, por lo que $f(A)\subset \mathbb{R}$ es compacto, luego acotado, de donde deducimos que $A$ es acotado.
\end{ejercicio}

\begin{ejercicio}
    Sea $E$ un espacio de Banach y sea $\{x_n\}$ una sucesión de forma que $\{x_n\}\rightharpoonup x$ en la topología débil $\sigma(E,E^\ast)$. Consideramos:
    \begin{equation*}
        \sigma_n = \frac{1}{n}(x_1 + x_2 + \ldots + x_n) \qquad \forall n \in \mathbb{N}
    \end{equation*}
    Prueba que $\{\sigma_n\}\rightharpoonup x$ en la topología débil $\sigma(E,E^\ast)$.\\

    \noindent
    Tenemos que:
    \begin{equation*}
        \{\sigma_n\}\rightharpoonup x \Longleftrightarrow \{\langle f,\sigma_n \rangle \}\to \langle f,x \rangle  \quad \forall f\in E^\ast
    \end{equation*}
    Fijada $f\in E^\ast$ y $\varepsilon>0$, sabemos que $\{\langle f,x_n \rangle \}\to \langle f,x \rangle $, por lo que existe $n_0\in \mathbb{N}$ de forma que:
    \begin{equation*}
        |\langle f,x_n \rangle -\langle f,x \rangle | < \frac{\varepsilon}{2} \qquad \forall n>n_0
    \end{equation*}
    Si consideramos ahora:
    \begin{equation*}
        n_1 = \max \left\{n_0, \left\lceil\sum_{i=1}^{n_0}|\langle f,x_i-x \rangle |\right\rceil\frac{2}{\varepsilon}\right\}
    \end{equation*}
    y tomamos $n>n_1$ tenemos entonces que:
    \begin{align*}
        |\langle f,\sigma_n \rangle -\langle f,x \rangle | &= \left|\frac{1}{n}\sum_{i=1}^{n} \langle f,x_i-x \rangle \right| \leq \frac{1}{n} \sum_{i=1}^{n} |\langle f,x_i-x \rangle | \\
                                                           &\leq \frac{1}{n} \sum_{i=1}^{n_0} |\langle f,x_i-x \rangle | + \frac{1}{n}\sum_{i=n_0+1}^{n}|\langle f,x_i-x \rangle |\leq \frac{\varepsilon}{2} + \frac{1}{n}\cdot n\frac{\varepsilon}{2} = \varepsilon
    \end{align*} % // TODO: TMB sale por Stolz
\end{ejercicio}

\begin{ejercicio}
    Sea $E$ un espacio de Banach. Sea $A\subset E$ un conjunto convexo. Prueba que el cierre de $A$ coincide con su cierre en la topología $\sigma(E,E^\ast)$.
\end{ejercicio}

\begin{ejercicio}
    
\end{ejercicio}

\begin{ejercicio}
    Sea $E$ un espacio de Banach y sea $K\subset E$ un subconjunto de $E$ compacto. Sea $\{x_n\}$ una sucesión en $K$ de forma que $\{x_n\}\rightharpoonup x$ en $\sigma(E,E^\ast)$. Prueba que $\{x_n\}\to x$.\newline
    (\textbf{Pista:} Argumentar por reducción al absurdo)\\

    \noindent
    \begin{description}
        \item [Opción 1.] Por reducción al absurdo, supuesto que $\{x_n\}\not\to x$, tenemos entonces que existe una sucesión parcial $\{x_{\sigma(n)}\}$ y un $\varepsilon>0$ de forma que:
            \begin{equation*}
                \|x_{\sigma(n)} - x\|>\varepsilon
            \end{equation*}
            Como $K$ es compacto, $\{x_{\sigma(n)}\}$ admite una sucesión parcial $\{x_{(\tau\circ \sigma)(n)}\}$ convergente a cierto $y\in K$. Por la unicidad del límite, tenemos que $y=x$, por lo que:
            \begin{equation*}
                \|x_{(\tau\circ \sigma)(n)} -x\| \to 0
            \end{equation*}
            lo que es una contradicción.
        \item [Opción 2.]  % // TODO: ESto es un argumento del profe

            Queremos probar $\{x_n\}\to x$ con $\{x_n\}$, $x$ dados. Esto equivale a decir que para toda parcial $\{y_{\sigma(n)}\}$ se tiene que existe $\{x_{\tau(\sigma(n))}\}$ de $\{x_{\sigma(n)}\}$ tal que $x_{\tau(\sigma(n))}$ es convergente a $x$. % // TODO: Probar el resultado

            \noindent
            Tenemos que $\{x_n\}\rightharpoonup x$, por lo que para toda $\sigma:\mathbb{N}\to\mathbb{N}$ estrictamente creciente se tien que $\{x_{\sigma(n)}\}\subset K$ es compacto, por lo que existe una sucesión parcial $\{x_{\tau(\sigma(n))}\}$ convergente hacia (como $\{x_n\}\rightharpoonup x$, tendremos que $\{x_{\tau(\sigma(n))}\}\rightharpoonup x$), luego ha de ser $\{x_{\tau(\sigma(n))}\}\to x$, por lo que por el resultado anterior tenemos que $\{x_n\}\to x$.
    \end{description}
\end{ejercicio}

\begin{ejercicio}
    Sea $X$ un espacio topológico y sea $E$ un espacio de Banach. Sean $u,v:X\to E$ dos aplicaciones continuas de $X$ en $(E,\sigma(E,E^\ast))$.
    \begin{enumerate}
        \item Prueba que la aplicación $x\mapsto u(x)+v(x)$ es continua de $X$ en $(E,\sigma(E,E^\ast))$.
        \item Sea $a:X\to \mathbb{R}$ una función continua. Prueba que la aplicación $x\mapsto a(x)u(x)$ es continua de $X$ en $(E,\sigma(E,E^\ast))$.
    \end{enumerate}~\\

    \noindent
    Usaremos la Proposición 2:
    \begin{center}
        $\psi:X\to (E,\sigma(E,E^\ast))$  es continua $\Longleftrightarrow f\circ \psi:X\to \mathbb{R}$ es continua $\quad \forall f\in E^\ast$.
    \end{center}
    \begin{enumerate}
        \item Para el primero, si $\psi(x) = u(x) + v(x)$, tenemos que para toda $f\in E^\ast$:
            \begin{equation*}
                f(\psi(x)) = f(u(x) + v(x)) = f(u(x)) + f(v(x))
            \end{equation*}
            que es continua, puesto que $f,u,v$ son continuas, de donde $\psi$ es continua.
        \item Consideramos ahora $\psi(x) = a(x)u(x)$, y si $f\in E^\ast$ tenemos:
            \begin{equation*}
                f(\psi(x)) = f(a(x)u(x)) = a(x)f(u(x))
            \end{equation*}
            con la última función continua, por lo que $\psi$ es continua.
    \end{enumerate} % // TODO: Ver que XxR --> E tmb es continuo
\end{ejercicio}

% // TODO: Ver donde meter esto
\setcounter{ejercicio}{-1}
\begin{ejercicio}
    Sea $E$ un espacio normado y $M\subset E$ espacio vectorial, se pide probar que:
    \begin{equation*}
        \sigma(M,M^\ast) = \sigma(E,E^\ast)\big|_{M} = \{U\cap M : U\in \sigma(E,E^\ast)\}
    \end{equation*}
    \begin{proof}
        Siempre que tengamos $f\in M^\ast$, denotaremos por $\hat{f}$ a una extensión continua de $f$ dada por Hahn-Banach. Fijamos ahora $x_0\in M$ y consideramos:
        \begin{align*}
            \cc{V}_1 &= \{V_M(f_1,\ldots,f_k,\varepsilon): \varepsilon>0, f_1,\ldots, f_k\in M^\ast \} \\
            \cc{V}_2 &= \{V_E(g_1,\ldots,g_k,\varepsilon)\cap M:\varepsilon>0, g_1,\ldots,g_k\in E^\ast \}
        \end{align*}
        donde recordamos que:
        \begin{equation*}
            V_M(f_1,\ldots,f_k,\varepsilon) = \{x\in M : |\langle f_i,x-x_0 \rangle |<\varepsilon,\quad \forall i \in \{1,\ldots,k\}\}
        \end{equation*}
        Vemos que $\cc{V}_1$ y $\cc{V}_2$ son bases de entornos de las dos correspondientes topologías. Como $M$ es espacio vectorial, $x-x_0\in M$ para todo $x\in M$, por lo que:
        \begin{equation*}
            f_i(x-x_0) = \hat{f}_1(x-x_0) \qquad \forall x\in M, \quad \forall i \in \{1,\ldots,k\}
        \end{equation*}
        Bajo esta premisa, se prueba que $\cc{V}_1 = \cc{V}_2$  por doble inclusión, ya que cada $g\in E^\ast$ da una $g\big|_{M^\ast} \in M^\ast$ y para cada $f\in M^\ast$ obtenemos una $\hat{f}\in E^\ast$.
        Por lo que las dos topologías han de ser iguales, pues tienen uan misma base de entornos y las bases de entornos determinan de forma única las topologías.
    \end{proof}
\end{ejercicio}
