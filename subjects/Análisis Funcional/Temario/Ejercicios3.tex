\section{Topologías Débiles}
% // TODO: PUedo salir al 3, 4, 

\begin{ejercicio}
    Sea $E$ un espacio de Banach y sea $A\subset E$ un subconjunto que es compacto en la topología débil $\sigma(E,E^\ast)$. Prueba que $A$ es acotado.\\

    \noindent
    Con vistas a usar el Corolario~\ref{coro:entonces_B_acotado}:
    \begin{center}
        Si $f(A)$ está acotado $\forall f\in E^\ast \Longrightarrow A$ está acotado.
    \end{center}
    Sea $f\in E^\ast$, tenemos entonces que $f:(E,\sigma(E,E^\ast))\to (\mathbb{R},\tau_u)$ es continua, por lo que $f(A)\subset \mathbb{R}$ será compacto (como imagen de un compacto por una función continua), luego será $f(A)$ acotado. La arbitrariedad de $f\in E^\ast$ nos permite aplicar el Corolario enunciado, obteniendo entones que $A$ es acotado.
\end{ejercicio}

\begin{ejercicio}
    Sea $E$ un espacio de Banach y sea $\{x_n\}$ una sucesión de forma que $\{x_n\}\rightharpoonup x$ en la topología débil $\sigma(E,E^\ast)$. Consideramos:
    \begin{equation*}
        \sigma_n = \frac{1}{n}(x_1 + x_2 + \ldots + x_n) \qquad \forall n \in \mathbb{N}
    \end{equation*}
    Prueba que $\{\sigma_n\}\rightharpoonup x$ en la topología débil $\sigma(E,E^\ast)$.\\

    \noindent
    Tenemos que:
    \begin{equation*}
        \{\sigma_n\}\rightharpoonup x \Longleftrightarrow \{\langle f,\sigma_n \rangle \}\to \langle f,x \rangle  \quad \forall f\in E^\ast
    \end{equation*}
    \begin{description}
        \item [Opción 1.] Fijada $f\in E^\ast$ y $\varepsilon>0$, sabemos que $\{\langle f,x_n \rangle \}\to \langle f,x \rangle $, por lo que existe $n_0\in \mathbb{N}$ de forma que:
            \begin{equation*}
                |\langle f,x_n \rangle -\langle f,x \rangle | < \frac{\varepsilon}{2} \qquad \forall n>n_0
            \end{equation*}
            Si consideramos ahora:
            \begin{equation*}
                \mathbb{N}\ni n_1 \geq  \max \left\{n_0, \left(\sum_{i=1}^{n_0}|\langle f,x_i-x \rangle |\right)\frac{2}{\varepsilon}\right\}
            \end{equation*}
            y tomamos $n>n_1$ tenemos entonces que:
            \begin{align*}
                |\langle f,\sigma_n \rangle -\langle f,x \rangle | &= \left|\frac{1}{n}\sum_{i=1}^{n} \langle f,x_i-x \rangle \right| \leq \frac{1}{n} \sum_{i=1}^{n} |\langle f,x_i-x \rangle | \\
                                                                   &\leq \frac{1}{n} \sum_{i=1}^{n_0} |\langle f,x_i-x \rangle | + \frac{1}{n}\sum_{i=n_0+1}^{n}|\langle f,x_i-x \rangle |\\
                                                                   &\leq \frac{n_1}{n}\cdot  \frac{\varepsilon}{2} + \frac{1}{n}\cdot (n-n_0-1)\frac{\varepsilon}{2} \leq \frac{\varepsilon}{2} + \frac{\varepsilon}{2} = \varepsilon
            \end{align*} 
        \item [Opción 2.] Fijada $f\in E^\ast$, queremos acabar probando que $\{f(\sigma_n)\}\to f(x)$. Observemos que para cada $n\in \mathbb{N}$ tenemos que:
            \begin{equation*}
                f(\sigma_n) = f\left(\frac{1}{n}\sum_{k=1}^{n}x_k\right) = \frac{1}{n}\sum_{k=1}^{n}f(x_k)
            \end{equation*}
            Si notamos:
            \begin{equation*}
                a_n = \sum_{k=1}^{n}f(x_k) \qquad b_n = n \qquad \forall n\in \mathbb{N}
            \end{equation*}
            Tenemos que $b_n$ es estrictamente creciente y divergente, así como que:
            \begin{equation*}
                \frac{a_{n+1}-a_n}{b_{n+1}-b_n} = \frac{\sum\limits_{k=1}^{n+1}f(x_k) - \sum\limits_{k=1}^{n}f(x_k)}{(n+1)-n} = \frac{f(x_{n+1})}{1} = f(x_{n+1}) \to f(x)
            \end{equation*}
            Aplicando el Criterio de Stolz obtenemos que $\{\frac{a_n}{b_n}\}=\{f(\sigma_n)\}\to f(x)$.

            Observemos la similaritud entre esta prueba y la que se hizo en Cálculo I para probar que la sucesión de medias aritméticas de una sucesión convergente también tiende a su límite.
    \end{description}
\end{ejercicio}

\begin{ejercicio}
    Sea $E$ un espacio de Banach. Sea $A\subset E$ un conjunto convexo. Prueba que el cierre de $A$ coincide con su cierre en la topología $\sigma(E,E^\ast)$.\\

    \noindent
    Si denotamos por $\tau_E$ a la topología de $E$ dada por su norma y por $\overline{A}^\sigma$ al cierre de $A$ en la topología $\sigma(E,E^\ast)$, queremos ver que $\overline{A} = \overline{A}^\sigma$:
    \begin{description}
        \item[$\subseteq )$] Como $\sigma(E,E^\ast)\subset \tau_E$ y $\overline{A}^\sigma$ es $\sigma(E,E^\ast)-$cerrado tenemos entonces que $\overline{A}^\sigma$ es cerrado en $\tau_E$ y es claro que $A\subset \overline{A}^\sigma$, por lo que $\overline{A}\subset \overline{A}^\sigma$.
        \item[$\supseteq )$] Como $A$ es convexo tenemos (visto por ejemplo en el Ejercicio~\ref{ej:7_rel1}) que $\overline{A}$ es convexo y además es cerrado, por lo que $\overline{A}$ es $\sigma(E,E^\ast)-$cerrado, y es claro que $A\subset \overline{A}$, con lo que $\overline{A}^\sigma\subset \overline{A}$.
    \end{description}
\end{ejercicio}

\begin{ejercicio}
    Sea $E$ un espacio de Banach y sea $\{x_n\}$ una sucesión de puntos de $E$ de forma que $\{x_n\}\rightharpoonup x$ en la topología débil $\sigma(E,E^\ast)$.
    \begin{enumerate}
        \item Prueba que existe una sucesión $\{y_n\}$ de puntos de $E$ de forma que
            \begin{enumerate}[label=\alph*)]
                \item $y_n \in \conv\left(\bigcup\limits_{i =n}^\infty\{x_i\}\right)\quad \forall n \in \mathbb{N}$
                \item $\{y_n\}\to x$.
            \end{enumerate}
        \item Prueba que existe una sucesión $\{z_n\}$ de puntos de $E$ de forma que
            \begin{enumerate}[label=\alph*)]
                \item $z_n \in \conv\left(\bigcup\limits_{i=1}^n\{x_i\}\right)\quad \forall n \in \mathbb{N}$
                \item $\{z_n\}\to x$.
            \end{enumerate}
    \end{enumerate}
    \noindent
    \textbf{Solución.}\newline
    A lo largo del ejercicio denotaremos por $\overline{A}^\sigma$ al cierre del conjunto $A$ en la toplogía $\sigma(E,E^\ast)$.
    \begin{enumerate}
        \item Para todo $k\in \mathbb{N}$ definimos:
            \begin{equation*}
                C_k = \conv\left(\bigcup_{i=k}^\infty\{x_i\}\right)
            \end{equation*}
            Se verifica que:
            \begin{equation*}
                \{x_n\}\rightharpoonup x \Longleftrightarrow \{x_{n+k}\} \rightharpoonup x \quad \forall k \in \mathbb{N}
            \end{equation*}
            de la segunda afirmación deducimos que:
            \begin{equation*}
                x\in \overline{\{x_{n+k} : k\in \mathbb{N}\}}^\sigma \subset \overline{C_k}^\sigma \AstIg \overline{C_k} \qquad \forall k \in \mathbb{N}
            \end{equation*}
            donde en $(\ast)$ hemos usado el ejercicio anterior. Así, para cada $k \in \mathbb{N}$ tenemos que $x\in \overline{C_k}$, por lo que podemos tomar $y_k \in C_k$ de forma que $\|x-y_k\|<\frac{1}{k}$, obteniendo la sucesión $\{y_k\}$ que buscábamos.
        \item Para todo $k\in \mathbb{N}$ definimos ahora:
            \begin{equation*}
                P_k = \conv\left(\bigcup_{i=1}^n \{x_i\}\right) , \qquad C = \conv\left(\bigcup_{i \in \mathbb{N}} \{x_i\}\right)
            \end{equation*}
            Por el Teorema de Mazur tenemos que existe una sucesión $\{y_n\}$ de puntos de $C$ con $\{y_n\}\to x$. Para cada $k\in \mathbb{N}$ tenemos que $y_k\in C$, es decir, $y_k$ es combinación convexa de una cantidad finita de puntos de $\{x_n : n\in \mathbb{N}\}$, por lo que existe $m\in \mathbb{N}$ de forma que $y_n \in P_m$. Podemos definir por tanto:
            \begin{equation*}
                m(k) = \min\{n\in \mathbb{N} : y_k \in P_n\} \qquad \forall k\in \mathbb{N}
            \end{equation*}
            Observemos además que si $z\in P_k$ para cierto $k\in \mathbb{N}$ tenemos entonces que $z\in P_n$ para todo $n\geq k$, por lo que si consideramos la aplicación $\sigma:\mathbb{N}\to \mathbb{N}$ dada por:
            \begin{equation*}
                \sigma(k) = \max\{k, m(k)\}
            \end{equation*}
            Observamos que es claramente estrictamente creciente, así como que:
            \begin{equation*}
                y_{\sigma(k)} \in P_k \qquad \forall k\in \mathbb{N}
            \end{equation*}
            ya que para cada $k\in \mathbb{N}$ tenemos que $y_{m(k)} \in P_k$ y $\sigma(k)\geq m(k)$. Si tomamos $z_n = y_{\sigma(n)}$ para cada $n\in \mathbb{N}$ tenemos que $\{z_n\} \to x$ por ser una sucesión parcial de $\{y_n\}$ y que $z_n = y_{\sigma(k)} \in P_k \quad \forall k\in \mathbb{N}$.
    \end{enumerate}
\end{ejercicio}

\begin{ejercicio} % // TODO: A partir de aqui
    Sea $E$ un espacio de Banach y sea $K\subset E$ un subconjunto de $E$ compacto. Sea $\{x_n\}$ una sucesión en $K$ de forma que $\{x_n\}\rightharpoonup x$ en $\sigma(E,E^\ast)$. Prueba que $\{x_n\}\to x$.\newline
    (\textbf{Pista:} Argumentar por reducción al absurdo)\\

    \noindent
    \begin{description}
        \item [Opción 1.] Por reducción al absurdo, supuesto que $\{x_n\}\not\to x$, tenemos entonces que existe una sucesión parcial $\{x_{\sigma(n)}\}$ y cierto $\varepsilon_0>0$ de forma que:
            \begin{equation}\label{eq:ejercicio_5_rel3}
                \|x_{\sigma(n)} - x\|\geq \varepsilon_0 \qquad \forall n \in \mathbb{N}
            \end{equation}
            Como $K$ es compacto, $\{x_{\sigma(n)}\}$ admite una sucesión parcial $\{x_{\tau(\sigma(n))}\}$ convergente a cierto $z\in K$, luego tenemos que $\{x_{\tau(\sigma(n))}\}\rightharpoonup z$ por esta última convergencia y que $\{x_{\tau(\sigma(n))}\}\rightharpoonup x$ por ser $\{x_n\}\rightharpoonup x$. Como $\sigma(E,E^\ast)$ es Hausdorff tiene que ser $x = z$, por lo que:
            \begin{equation*}
                \|x_{\tau(\sigma(n))} -x\| \to 0
            \end{equation*}
            lo que contradice la afirmación~\eqref{eq:ejercicio_5_rel3}.
        \item [Opción 2.]  Hay una Proposición muy útil a la hora de probar la convergencia de una sucesión a un punto:

            \begin{prop}
                Sea $X$ un espacio topológico, $\{x_n\}$ una sucesión de puntos de $X$ y $x\in X$:
                \begin{equation*}
                    \{x_n\} \to x \Longleftrightarrow \begin{array}{l}
                        \text{Para toda parcial\ } \{x_{\sigma(n)}\} \text{\ existe} \\
                        \text{\ una parcial\ } \{x_{\tau(\sigma(n))}\} \to x
                    \end{array}
                \end{equation*}
                \begin{proof}
                    La implicación de izquierda a derecha es trivial, por lo que basta probar:
                    \begin{description} 
                        \item [$\Longleftarrow )$] Por reducción al absurdo, supongamos que $\{x_n\}\not\to x$, por lo que podemos encontrar un entorno $U_0$ de $x$ y una parcial $\{x_{\sigma(n)}\}$ de forma que:
                            \begin{equation}\label{eq:ejercicio_5_rel3_bis}
                                x_{\sigma(n)} \notin U_0 \qquad \forall n \in \mathbb{N}
                            \end{equation}
                            Sin embargo, tenemos que existe una parcial $\{x_{\tau(\sigma(n))}\}$ convergente a $x$, lo que contradice la afirmación~\eqref{eq:ejercicio_5_rel3_bis}.\qedhere
                    \end{description}
                \end{proof}
            \end{prop}

            \noindent
            Sea $\sigma:\mathbb{N}\to \mathbb{N}$ estrictamente creciente, consideramos $\{x_{\sigma(n)}\} \subset K$. Como $K$ es compacto, existe una parcial $\{x_{\tau(\sigma(n))}\}$  convergente a cierto punto $z\in K$. Razonando igual que en la opción 1 vemos que $z = x$ y aplicando la Proposición superior tenemos el resultado probado.
    \end{description}
\end{ejercicio}

\begin{ejercicio}
    Sea $X$ un espacio topológico y sea $E$ un espacio de Banach. Sean $u,v:X\to E$ dos aplicaciones continuas de $X$ en $(E,\sigma(E,E^\ast))$.
    \begin{enumerate}
        \item Prueba que la aplicación $x\mapsto u(x)+v(x)$ es continua de $X$ en $(E,\sigma(E,E^\ast))$.
        \item Sea $a:X\to \mathbb{R}$ una función continua. Prueba que la aplicación $x\mapsto a(x)u(x)$ es continua de $X$ en $(E,\sigma(E,E^\ast))$.
    \end{enumerate}

    \noindent
    Usaremos la Proposición~\ref{prop:sii_debil}:
    \begin{center}
        $\psi:X\to (E,\sigma(E,E^\ast))$  es continua $\Longleftrightarrow f\circ \psi:X\to \mathbb{R}$ es continua $\quad \forall f\in E^\ast$.
    \end{center}
    \begin{enumerate}
        \item Para el primero, si definimos $\varphi:X\to (E,\sigma(E,E^\ast))$ dada por:
            \begin{equation*}
                \varphi(x) = u(x) + v(x)
            \end{equation*}
            Sea $f\in E^\ast$ y $\{x_n\}\to x$ tenemos que:
            \begin{equation*}
                (f\circ \varphi)(x_n) = f(u(x_n) + v(x_n)) = f(u(x_n)) + f(v(x_n)) \qquad \forall n\in \mathbb{N}
            \end{equation*}
            Y como $(f\circ u), (f\circ v)$ son continuas por ser $u$ y $v$ continuas de $X$ en $(E,\sigma(E,E^\ast))$ tenemos pues que:
            \begin{equation*}
                \{(f\circ \varphi)(x_n)\} = \{f(u(x_n)) + f(v(x_n))\} \to f(u(x)) + f(v(x)) = (f\circ \varphi)(x)
            \end{equation*}
            De donde $\{\varphi(x_n)\}\rightharpoonup \varphi(x)$, lo que demostra que $\varphi$ es continua.
        \item Consideramos ahora $\varphi:X\to (E,\sigma(E,E^\ast))$ dada por:
            \begin{equation*}
                \varphi(x) = a(x)u(x)
            \end{equation*}
            Sea $f\in E^\ast$ y $\{x_n\}\to x$ tenemos que:
            \begin{equation*}
                (f\circ \varphi)(x_n) = f(a(x_n)u(x_n)) = a(x_n)f(u(x_n)) \qquad \forall n \in \mathbb{N}
            \end{equation*}
            Y como $f\circ u$ es continua por ser $u$ continua tenemos que:
            \begin{equation*}
                \{(f\circ \varphi)(x_n)\} = \{a(x_n)f(u(x_n))\} \to a(x)f(u(x)) = f(\varphi(x))
            \end{equation*}
            De donde $\{\varphi(x_n)\}\rightharpoonup \varphi(x)$, lo que demuestra que $\varphi$ es continua.
    \end{enumerate} % // TODO: Ver que XxR --> E tmb es continuo
\end{ejercicio}

\begin{ejercicio}
    Sea $E$ un espacio de Banach y sea $A\subset E$ un conjunto cerrado en la topología débil $\sigma(E,E^\ast)$. Sea $B\subset E$ un subconjunto compacto en la topología débil $\sigma(E,E^\ast)$:
    \begin{enumerate}[label=\alph*)]
        \item Prueba que $A+B$ es cerrado en $\sigma(E,E^\ast)$.
        \item Si además $A$ y $B$ son convexos, no vacíos y disjuntos; prueba que existe un hiperplano cerrado que separa estrictamente $A$ y $B$.
    \end{enumerate}
\end{ejercicio}

\begin{ejercicio}
    Sea $E$ un espacio de Banach de dimensión infinita. Nuestro propósito es demostrar que $(E,\sigma(E,E^\ast))$ no es metrizable. Supongamos por reducción al absurdo que existe una distancia $d$ en $E$ que induce la topología $\sigma(E,E^\ast)$.
    \begin{enumerate}
        \item Para cada $k\in \mathbb{N}$ denotamos por $V_k$ a un entorno de $0$ en la topología $\sigma(E,E^\ast)$ de forma que:
            \begin{equation*}
                V_k\subset \left\{x\in E : d(x,0) < \frac{1}{k}\right\}
            \end{equation*}
            Prueba que existe una sucesión $\{f_n\}$ de puntos de $E^\ast$ de forma que cada $g\in E^\ast$ es una combinación lineal finita de puntos de $\{f_n : n\in \mathbb{N}\}$.\newline
            (\textbf{Pista:} Usa el Ejercicio~\ref{ej:lineal_debil*})
        \item Dedue que $E^\ast$ tiene dimensión finita.\newline
            (\textbf{Pista:} Usa el Lema de la categoría de Baire como en el Ejercicio~\ref{ej:5_rel1})
        \item Finaliza la prueba.
        \item Prueba por un método similar que $(E^\ast,\sigma(E^\ast,E))$ no es metrizable.
    \end{enumerate}
\end{ejercicio}

\begin{ejercicio}
    Sea $E$ Un espacio de Banach, sea $M\subset E$ un subespacio vectorial y sea $f_0\in E^\ast$. Prueba que existe $g_0\in M^\perp$ de forma que
    \begin{equation*}
        \inf_{g\in M^\perp}\|f_0-g\| = \|f_0-g_0\|
    \end{equation*}
    Puede hacerse por dos métodos:
    \begin{enumerate}
        \item Usar el Teorema 1.12 del Brezis.
        \item Usar la topología débil$-\ast$ $\sigma(E^\ast,E)$.
    \end{enumerate}
\end{ejercicio}

\begin{ejercicio}
    Sean $E$ y $F$ espacios de Banach. Sea $T\in L(E,F)$ de forma que $T^\ast \in L(F^\ast, E^\ast)$. Prueba que $T^\ast:(F^\ast,\sigma(F^\ast,F))\to (E^\ast,\sigma(E^\ast,E))$.\\

    \noindent
    Tenemos que $T^\ast:F^\ast\to E^\ast$ es la dada por:
    \begin{equation*}
        T^\ast(g) = g\circ T \qquad \forall g\in F^\ast
    \end{equation*}
    Notaremos:
    \begin{equation*}
        T^\ast_{\sigma} :(F^\ast,\sigma(F^\ast,F))\to (E^\ast,\sigma(E^\ast,E))
    \end{equation*}
    Para demostrar el ejercicio, dado $x\in E$ tenemos que $T^\ast_\sigma $ es continua si y solo si $J(x) \circ T^\ast:(F^\ast,\sigma(F^\ast,F))\to (\mathbb{R},\tau_u)$  es continua. Para ello, tenemos:
    \begin{equation*}
        (J(x)\circ T^\ast)(g) = J(x)(g\circ T) = (g\circ T)(x) = g(T(x)) = J_F(T(x))(g) \qquad \forall g\in F^\ast
    \end{equation*}
    Como $J_F(T(x)):(F^\ast,\sigma(F^\ast,F))\to (\mathbb{R},\tau_u)$ es continua y $J(x)\circ T^\ast = J_F(T(x))$ tenemos entonces que $J(x)\circ T^\ast$ es continua, luego $T^\ast_\sigma$ es continua.

\end{ejercicio}

\begin{ejercicio}
    Sea $E$ un espacio de Banach y sea $A:E\to E^\ast$ una aplicación monótona definida en $D(A) = E$, vea el Ejercicio~\ref{ej:6_rel2}. Supongamos que para cada $x,y\in E$ la aplicación $\mathbb{R}\to \mathbb{R}$ dada por:
    \begin{equation*}
        t\longmapsto \langle A(x+ty),y \rangle 
    \end{equation*}
    es continua en $t=0$. Prueba que $A:(E,\tau_E)\to (E,\sigma(E^\ast,E))$ es continua.
\end{ejercicio}

\setcounter{ejercicio}{15}
\begin{ejercicio}
    Sea $E$ un espacio de Banach:
    \begin{enumerate}[label=\alph*)]
        \item Sea $\{f_n\}$ una sucesión de puntos de $E^\ast$ de forma que para cada $x\in E$ $\{\langle f_n,x \rangle \}$ es convergete a un límite. Prueba que existe $f\in E^\ast$ de forma que $\{f_n\}\stackrel{\ast}{\rightharpoonup} f$ en $\sigma(E^\ast,E)$.

            Fijado $x\in E$ tenemos que existe un único $y_x\in \mathbb{R}$ de forma que:
            \begin{equation*}
                \{\langle f_n,x \rangle \}\to y_x
            \end{equation*}
            Esto nos permite definir la aplicación $f:E\to \mathbb{R}$ que a cada $x$ le hace corresponder $y_x$. En vista de la definición de $f$ es sencillo comprobar que $f$ es lineal. Si observamos ahora que para todo $x\in E$ tenemos que:
            \begin{equation*}
                |\langle f_n,x \rangle | \leq \|f_n\|\|x\| \qquad \forall n\in \mathbb{N}
            \end{equation*}
            Podemos tomar límite inferior a ambos lados de la igualdad, obteniendo que:
            \begin{equation*}
                |\langle f,x \rangle | \leq \|x\|\liminf \|f_n\|
            \end{equation*}
            Por lo que $f\in E^\ast$. Para comprobar que $\{f_n\}\stackrel{\ast}{\rightharpoonup} f$ solo hemos de observar que para cada $x\in E$ tenemos que:
            \begin{equation*}
                \langle Jx,f_n \rangle  = \langle f_n,x \rangle  \to \langle f,x \rangle  = \langle Jx,f \rangle  \qquad \forall n\in \mathbb{N}
            \end{equation*}
            por lo que $\{f_n\}\stackrel{\ast}{\rightharpoonup} f$.
        \item Supongamos aquí que además $E$ es reflexivo. Sea $\{x_n\}$ una sucesión de puntos de $E$ de forma que para cada $f\in E^\ast$ se tiene que $\{\langle f,x_n \rangle \}$ es convergente a un límite. Prueba que existe $x\in E$ de forma que $\{x_n\}\rightharpoonup x$ en $\sigma(E,E^\ast)$.

            Para cada $f\in E^\ast$ tenemos que la sucesión:
            \begin{equation*}
                \{\langle f,x_n \rangle \} = \{\langle Jx_n,f \rangle \}
            \end{equation*}
            es convergente. Si consideramos la sucesión $\{Jx_n\}$ de puntos de $E^{\ast\ast}$, observamos que $\{\langle Jx_n,f \rangle \}$ es convergente para cada $f\in E^\ast$, por lo que aplicando el apartado que acabamos de ver tenemos que existe $\varphi \in E^{\ast\ast}$ de forma que:
            \begin{equation*}
                \{Jx_n\}\stackrel{\ast}{\rightharpoonup} \varphi
            \end{equation*}
            Como $E$ es reflexivo tenemos que existe $x\in E$ de forma que $\varphi = Jx$. Bajo esta hipótesis vemos que:
            \begin{equation*}
                \{Jx_n\} \stackrel{\ast}{\rightharpoonup} \varphi \quad\Longleftrightarrow\quad \{Jx_n\}\stackrel{\ast}{\rightharpoonup} Jx
            \end{equation*}
            Basta observar que para $f\in E^\ast$ se tiene que:
            \begin{equation*}
                \{\langle f,x_n \rangle \} = \{\langle Jx_n,f \rangle \} \to \langle Jx,f \rangle   = \langle f,x \rangle 
            \end{equation*}
            Por lo que $\{x_n\}\rightharpoonup x$.
        \item Construye un ejemplo de un espacio no reflexivo $E$ donde la afirmación anterior falle. 
            (\textbf{Pista:} Tomar $E=c_0$ y $x_n = (1, \ldots, 1, 0, \ldots)$).
    \end{enumerate}
\end{ejercicio}

\setcounter{ejercicio}{19}
\begin{ejercicio}
    Sea $E$ un espacio de Banach.
    \begin{enumerate}[label=\alph*)]
        \item Existe un espacio topológico $K$ y una isometría $\psi:E\to C(K)$.\newline
            (\textbf{Pista:} Tomar $K = (\overline{B}_{E^\ast}, \sigma(E^\ast,E)\big|_{\overline{B}_{E^\ast}})$)\\
            
            Consideraremos:
            \begin{equation*}
                \|f\|_\infty = \max_{x\in K}|f(x)| \qquad \forall f\in C(K)
            \end{equation*}
            Tenemos que:
            \begin{equation*}
                \|x\| = \max_{f\in \overline{B}_{E^\ast}} |\langle f,x \rangle | = \max_{f\in \overline{B}_{E^\ast}} |\langle Jx,f \rangle | = \left\|(Jx)\big|_{\overline{B}_{E^\ast}}\right\|_\infty
            \end{equation*}
            Definimos $\psi:E\to C(K)$ dada por:
            \begin{equation*}
                \psi(x) = (Jx)\big|_{\overline{B}_{E^\ast}}
            \end{equation*}
        \item Supuesto que $E$ es separable, prueba que exite una isometría $\psi:E\to l^\infty$.

            Veamos qué pinta ha de tener dicha supuesta isometría: para $x\in E$ su imagen debe tener la siguiente forma:
            \begin{equation*}
                \psi(x) = \{\psi_n(x)\} \qquad \psi_n \in E^\ast
            \end{equation*}
            y neesitamos que $\{\psi_n\}$ esté acotada, por lo que lo ideal es tomar $\psi_n \in \overline{B}_{E^\ast}$. De esta forma:
            \begin{equation*}
                |\psi_n(x)| \leq \|\psi_n\|\|x\| \leq \|x\|\qquad \forall n \in \mathbb{N}
            \end{equation*}
            y la aplicación estará bien definidad. Necesitamos además que:
            \begin{equation*}
                \|\psi_n(x)\|_\infty = \sup_{n\in \mathbb{N}}|\psi_n(x)| = \max_{f\in \overline{B}_{E^\ast}}|\langle f,x \rangle | = \|x\| = \langle f_x,x \rangle 
            \end{equation*}
            para cierto $f_x\in \overline{B}_{E^\ast}$. La desigualdad $(\leq)$ ya es clara para nuestra sucesión. Para la otra, buscamos una sucesión de la forma $\{\psi_{n_m}(x)\}\to \langle f_x,x \rangle $. Esto incita a pensar que $\{\psi_n : n\in \mathbb{N}\}$ tiene que ser denso en $\overline{B}_{E^\ast}$, ya que:
            \begin{equation*}
                \{f_n\}\stackrel{\ast}{\rightharpoonup} f \Longleftrightarrow  \{f_n(x)\} \to f
            \end{equation*}
            El problema se reduce a probar que $K$ es separable. Como $E$ es separable tenemos que $K$ es metrizable. Además, todo espacio métrico compacto es separable.

            \begin{center}
                Todo espacio métrico $K$ compacto es separable.
            \end{center}
            \begin{proof}
                Fijado $n\in \mathbb{N}$, tenemos que:
                \begin{equation*}
                    K = \bigcup_{x\in K} B\left(x,\frac{1}{n}\right)
                \end{equation*}
                Y como $K$ es compacto, existe $F_n\subset K$ finito de forma que:
                \begin{equation*}
                    K = \bigcup_{x\in F_n} B\left(x,\frac{1}{n}\right)
                \end{equation*}
                Tomamos ahora $D = \bigcup\limits_{n\in \mathbb{N}}F_n$ que es un conjunto numerable. Para ver que es denso, basta ver que para todo $y\in K$ existe una sucesión de elementos de $D$ convergente a él. Para cada $n\in \mathbb{N}$ tenemos que:
                \begin{equation*}
                    y \in K = \bigcup_{x\in F_n}B\left(x,\frac{1}{n}\right)
                \end{equation*}
                Por lo que existe $x_n\in F_n$ tal que $y\in B\left(x_n,\frac{1}{n}\right)$, de donde:
                \begin{equation*}
                    \{x_n\} \to y
                \end{equation*}
                Porque $d(x_n,y)<\frac{1}{n}$ para cada $n\in \mathbb{N}$. Por lo que $D$ es denso y numerable, luego $K$ es separable.
            \end{proof}
            Por tanto, $K$ es separable, luego existe $\{\psi_n\}\subset K$ densa en $K$. Definimos ahora $\psi_n:E\to l^\infty$ dada por:
            \begin{equation*}
                \psi(x) = \{\psi_n(x)\} \qquad \forall x\in E
            \end{equation*}
            Que está bien definida, pues:
            \begin{equation*}
                |\psi_n(x)| \leq \|\psi_n\|\|x\| \leq \|x\| \qquad \forall n \in \mathbb{N}
            \end{equation*}
            además, hemos visto que $\|\psi_n(x)\|_\infty\leq \|x\|\infty\quad \forall x\in E$. Para ver que se da la igualdad, fijado $x\in E$ tenemos que existe $f_x\in \overline{B}_{E^\ast}$ de forma que:
            \begin{equation*}
                \langle f_x,x \rangle  = \|x\| = \max_{f\in \overline{B}_{E^\ast}}|\langle f,x \rangle |
            \end{equation*}
            de aquí, tenemos que $\psi_n\in \overline{B}_{E^\ast}$ para cada $n\in \mathbb{N}$, de donde:
            \begin{equation*}
                \|\psi_n(x)\|_\infty \leq \|x\|
            \end{equation*}
            Como el conjunto es denso, existe una sucesión $\{\psi_{n_m}\}_{m\in \mathbb{N}}\subset K$ de forma que:
            \begin{equation*}
                \{\psi_{n_m}\}\stackrel{\ast}{\rightharpoonup} f_x
            \end{equation*}
            ahora, tenemos que $\{\psi_{n_m}(x)\}\to\langle f_x,x \rangle = \|x\|$. Lo que nos da la igualdad, de donde tenemos que es una isometría.
    \end{enumerate}
\end{ejercicio}


\setcounter{ejercicio}{21}
\begin{ejercicio}
    Sea $E$ un espacio de Banach de dimensión infinita verificando alguna de las siguientes condiciones:
    \begin{enumerate}[label=\alph*)]
        \item $E^\ast$ es separable.
        \item $E$ es reflexivo.
    \end{enumerate}
    Prueba que entonces existe una sucesión $\{x_n\}$ de $E$ de forma que:
    \begin{equation*}
        \|x_n\| = 1 \quad \forall n\in \mathbb{N}\quad \text{y}\quad \{x_n\}\rightharpoonup 0 
    \end{equation*}
    \begin{enumerate}[label=\alph*)]
        \item Supuesto que $E^\ast$ es separable, tenemos entonces por uno de los últimos Teoremas del capítulo 3 que $\overline{B}_E$ es metrizable. Tenemos además que:
            \begin{equation*}
                0 \in \overline{B}_E = \overline{S}_E^\sigma
            \end{equation*}
            Como $\overline{B}_E$ es metrizable, tenemos entonces que existe una sucesión $\{x_n\}\subset S_E$  con $\{x_n\}\rightharpoonup 0$.
        \item Supuesto que $E$ es reflexivo. Por ser $E$ de dimensión infinita tenemos que existe una sucesión $\{x_n\}$ de vectores de $E$ de forma que el conjunto $\{x_n : n\in \mathbb{N}\}$ es linealmente independiente. En dicho caso, podemos tomar:
            \begin{equation*}
                E_0 = \overline{\cc{L}(\{x_n : n\in \mathbb{N}\})}
            \end{equation*}
            Y tenemos que $E_0$ es un subespacio vectorial de $E$, con:
            \begin{itemize}
                \item dimensión infinita, pues $\{x_n : n\in \mathbb{N}\}\subset E_0$.
                \item es cerrado.
                \item el conjunto:
                    \begin{equation*}
                        D = \left\{\sum_{i=1}^{n} r_i x_i : r_i \in \mathbb{Q} \quad \forall i \in \{1,\ldots,n\}, n\in \mathbb{N}\right\}
                    \end{equation*}
                    es denso y numerable, pues:
                    \begin{equation*}
                        D = \bigcup_{n\in \mathbb{N}}D_n, \qquad D_n = \left\{\sum_{i=1}^{n} r_ix_i : r_i \in \mathbb{Q} \quad \forall  i \in \{1,\ldots,n\}\right\}
                    \end{equation*}
                    Y podemos tomar $f:\mathbb{Q}^n\to D_n$ inyectiva dada por:
                    \begin{equation*}
                        f(r_1,\ldots, r_n) = \sum_{k=1}^{n}r_ix_i
                    \end{equation*}
                    Por lo que $E_0$ es separable.
            \end{itemize}
            Por ser $E_0$ cerrado y $E_0\subset E$ tenemos que $E_0$ es también reflexivo, y como era también separable tendremos entonces que $E_0^\ast$ es reflexivo y separable. Por el apartado $a)$ tenemos entonces que existe una sucesión $\{x_n\}$ de puntos de $E_0$ de forma que:
            \begin{equation*}
                \|x_n\| = 1 \qquad \forall n \in \mathbb{N}
            \end{equation*}
            con $\{x_n\}\rightharpoonup 0$ en $\sigma(E_0,E_0^\ast) = \sigma(E,E^\ast)\big|_{E_0}$, lo que implica que $\{x_n\}\rightharpoonup 0$ en $\sigma(E,E^\ast)$ pues si $U$ es un entorno de $0\in E_0$ en $\sigma(E,E^\ast)\big|_{E_0}$ tenemos entonces que:
            \begin{equation*}
                \exists m\in \mathbb{N} : n\geq m \Longrightarrow x_n \in U\cap E_0 \subset U
            \end{equation*}
    \end{enumerate}
\end{ejercicio}
