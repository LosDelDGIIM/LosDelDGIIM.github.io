\section{Topologías Débiles}
% // TODO: PUedo salir al 3, 4, 

\begin{ejercicio}
    Sea $E$ un espacio de Banach y sea $A\subset E$ un subconjunto que es compacto en la topología débil $\sigma(E,E^\ast)$. Prueba que $A$ es acotado.\\

    \noindent
    Con vistas a usar el Corolario~\ref{coro:entonces_B_acotado}:
    \begin{center}
        Si $f(A)$ está acotado $\forall f\in E^\ast \Longrightarrow A$ está acotado.
    \end{center}
    Sea $f\in E^\ast$, tenemos entonces que $f:(E,\sigma(E,E^\ast))\to (\mathbb{R},\tau_u)$ es continua, por lo que $f(A)\subset \mathbb{R}$ será compacto (como imagen de un compacto por una función continua), luego será $f(A)$ acotado. La arbitrariedad de $f\in E^\ast$ nos permite aplicar el Corolario enunciado, obteniendo entones que $A$ es acotado.
\end{ejercicio}

\begin{ejercicio}
    Sea $E$ un espacio de Banach y sea $\{x_n\}$ una sucesión de forma que $\{x_n\}\rightharpoonup x$ en la topología débil $\sigma(E,E^\ast)$. Consideramos:
    \begin{equation*}
        \sigma_n = \frac{1}{n}(x_1 + x_2 + \ldots + x_n) \qquad \forall n \in \mathbb{N}
    \end{equation*}
    Prueba que $\{\sigma_n\}\rightharpoonup x$ en la topología débil $\sigma(E,E^\ast)$.\\

    \noindent
    Tenemos que:
    \begin{equation*}
        \{\sigma_n\}\rightharpoonup x \Longleftrightarrow \{\langle f,\sigma_n \rangle \}\to \langle f,x \rangle  \quad \forall f\in E^\ast
    \end{equation*}
    \begin{description}
        \item [Opción 1.] Fijada $f\in E^\ast$ y $\varepsilon>0$, sabemos que $\{\langle f,x_n \rangle \}\to \langle f,x \rangle $, por lo que existe $n_0\in \mathbb{N}$ de forma que:
            \begin{equation*}
                |\langle f,x_n \rangle -\langle f,x \rangle | < \frac{\varepsilon}{2} \qquad \forall n>n_0
            \end{equation*}
            Si consideramos ahora:
            \begin{equation*}
                \mathbb{N}\ni n_1 \geq  \max \left\{n_0, \left(\sum_{i=1}^{n_0}|\langle f,x_i-x \rangle |\right)\frac{2}{\varepsilon}\right\}
            \end{equation*}
            y tomamos $n>n_1$ tenemos entonces que:
            \begin{align*}
                |\langle f,\sigma_n \rangle -\langle f,x \rangle | &= \left|\frac{1}{n}\sum_{i=1}^{n} \langle f,x_i-x \rangle \right| \leq \frac{1}{n} \sum_{i=1}^{n} |\langle f,x_i-x \rangle | \\
                                                                   &\leq \frac{1}{n} \sum_{i=1}^{n_0} |\langle f,x_i-x \rangle | + \frac{1}{n}\sum_{i=n_0+1}^{n}|\langle f,x_i-x \rangle |\\
                                                                   &\leq \frac{n_1}{n}\cdot  \frac{\varepsilon}{2} + \frac{1}{n}\cdot (n-n_0-1)\frac{\varepsilon}{2} \leq \frac{\varepsilon}{2} + \frac{\varepsilon}{2} = \varepsilon
            \end{align*} 
        \item [Opción 2.] Fijada $f\in E^\ast$, queremos acabar probando que $\{f(\sigma_n)\}\to f(x)$. Observemos que para cada $n\in \mathbb{N}$ tenemos que:
            \begin{equation*}
                f(\sigma_n) = f\left(\frac{1}{n}\sum_{k=1}^{n}x_k\right) = \frac{1}{n}\sum_{k=1}^{n}f(x_k)
            \end{equation*}
            Si notamos:
            \begin{equation*}
                a_n = \sum_{k=1}^{n}f(x_k) \qquad b_n = n \qquad \forall n\in \mathbb{N}
            \end{equation*}
            Tenemos que $b_n$ es estrictamente creciente y divergente, así como que:
            \begin{equation*}
                \frac{a_{n+1}-a_n}{b_{n+1}-b_n} = \frac{\sum\limits_{k=1}^{n+1}f(x_k) - \sum\limits_{k=1}^{n}f(x_k)}{(n+1)-n} = \frac{f(x_{n+1})}{1} = f(x_{n+1}) \to f(x)
            \end{equation*}
            Aplicando el Criterio de Stolz obtenemos que $\{\frac{a_n}{b_n}\}=\{f(\sigma_n)\}\to f(x)$.

            Observemos la similaritud entre esta prueba y la que se hizo en Cálculo I para probar que la sucesión de medias aritméticas de una sucesión convergente también tiende a su límite.
    \end{description}
\end{ejercicio}

\begin{ejercicio}
    Sea $E$ un espacio de Banach. Sea $A\subset E$ un conjunto convexo. Prueba que el cierre de $A$ coincide con su cierre en la topología $\sigma(E,E^\ast)$.\\

    \noindent
    Si denotamos por $\tau_E$ a la topología de $E$ dada por su norma y por $\overline{A}^\sigma$ al cierre de $A$ en la topología $\sigma(E,E^\ast)$, queremos ver que $\overline{A} = \overline{A}^\sigma$:
    \begin{description}
        \item[$\subseteq )$] Como $\sigma(E,E^\ast)\subset \tau_E$ y $\overline{A}^\sigma$ es $\sigma(E,E^\ast)-$cerrado tenemos entonces que $\overline{A}^\sigma$ es cerrado en $\tau_E$ y es claro que $A\subset \overline{A}^\sigma$, por lo que $\overline{A}\subset \overline{A}^\sigma$.
        \item[$\supseteq )$] Como $A$ es convexo tenemos (visto por ejemplo en el Ejercicio~\ref{ej:7_rel1}) que $\overline{A}$ es convexo y además es cerrado, por lo que $\overline{A}$ es $\sigma(E,E^\ast)-$cerrado, y es claro que $A\subset \overline{A}$, con lo que $\overline{A}^\sigma\subset \overline{A}$.
    \end{description}
\end{ejercicio}

\begin{ejercicio}
    Sea $E$ un espacio de Banach y sea $\{x_n\}$ una sucesión de puntos de $E$ de forma que $\{x_n\}\rightharpoonup x$ en la topología débil $\sigma(E,E^\ast)$.
    \begin{enumerate}
        \item Prueba que existe una sucesión $\{y_n\}$ de puntos de $E$ de forma que
            \begin{enumerate}[label=\alph*)]
                \item $y_n \in \conv\left(\bigcup\limits_{i =n}^\infty\{x_i\}\right)\quad \forall n \in \mathbb{N}$
                \item $\{y_n\}\to x$.
            \end{enumerate}
        \item Prueba que existe una sucesión $\{z_n\}$ de puntos de $E$ de forma que
            \begin{enumerate}[label=\alph*)]
                \item $z_n \in \conv\left(\bigcup\limits_{i=1}^n\{x_i\}\right)\quad \forall n \in \mathbb{N}$
                \item $\{z_n\}\to x$.
            \end{enumerate}
    \end{enumerate}
    \noindent
    \textbf{Solución.}\newline
    A lo largo del ejercicio denotaremos por $\overline{A}^\sigma$ al cierre del conjunto $A$ en la toplogía $\sigma(E,E^\ast)$.
    \begin{enumerate}
        \item Para todo $k\in \mathbb{N}$ definimos:
            \begin{equation*}
                C_k = \conv\left(\bigcup_{i=k}^\infty\{x_i\}\right)
            \end{equation*}
            Se verifica que:
            \begin{equation*}
                \{x_n\}\rightharpoonup x \Longleftrightarrow \{x_{n+k}\} \rightharpoonup x \quad \forall k \in \mathbb{N}
            \end{equation*}
            de la segunda afirmación deducimos que:
            \begin{equation*}
                x\in \overline{\{x_{n+k} : k\in \mathbb{N}\}}^\sigma \subset \overline{C_k}^\sigma \AstIg \overline{C_k} \qquad \forall k \in \mathbb{N}
            \end{equation*}
            donde en $(\ast)$ hemos usado el ejercicio anterior. Así, para cada $k \in \mathbb{N}$ tenemos que $x\in \overline{C_k}$, por lo que podemos tomar $y_k \in C_k$ de forma que $\|x-y_k\|<\frac{1}{k}$, obteniendo la sucesión $\{y_k\}$ que buscábamos.
        \item Para todo $k\in \mathbb{N}$ definimos ahora:
            \begin{equation*}
                P_k = \conv\left(\bigcup_{i=1}^n \{x_i\}\right) , \qquad C = \conv\left(\bigcup_{i \in \mathbb{N}} \{x_i\}\right)
            \end{equation*}
            Por el Teorema de Mazur tenemos que existe una sucesión $\{y_n\}$ de puntos de $C$ con $\{y_n\}\to x$. Para cada $k\in \mathbb{N}$ tenemos que $y_k\in C$, es decir, $y_k$ es combinación convexa de una cantidad finita de puntos de $\{x_n : n\in \mathbb{N}\}$, por lo que existe $m\in \mathbb{N}$ de forma que $y_n \in P_m$. Podemos definir por tanto:
            \begin{equation*}
                m(k) = \min\{n\in \mathbb{N} : y_k \in P_n\} \qquad \forall k\in \mathbb{N}
            \end{equation*}
            Observemos además que si $z\in P_k$ para cierto $k\in \mathbb{N}$ tenemos entonces que $z\in P_n$ para todo $n\geq k$, por lo que si consideramos la aplicación $\sigma:\mathbb{N}\to \mathbb{N}$ dada por:
            \begin{equation*}
                \sigma(k) = \max\{k, m(k)\}
            \end{equation*}
            Observamos que es claramente estrictamente creciente, así como que:
            \begin{equation*}
                y_{\sigma(k)} \in P_k \qquad \forall k\in \mathbb{N}
            \end{equation*}
            ya que para cada $k\in \mathbb{N}$ tenemos que $y_{m(k)} \in P_k$ y $\sigma(k)\geq m(k)$. Si tomamos $z_n = y_{\sigma(n)}$ para cada $n\in \mathbb{N}$ tenemos que $\{z_n\} \to x$ por ser una sucesión parcial de $\{y_n\}$ y que $z_n = y_{\sigma(k)} \in P_k \quad \forall k\in \mathbb{N}$.
    \end{enumerate}
\end{ejercicio}

\begin{ejercicio} % // TODO: A partir de aqui
    Sea $E$ un espacio de Banach y sea $K\subset E$ un subconjunto de $E$ compacto. Sea $\{x_n\}$ una sucesión en $K$ de forma que $\{x_n\}\rightharpoonup x$ en $\sigma(E,E^\ast)$. Prueba que $\{x_n\}\to x$.\newline
    (\textbf{Pista:} Argumentar por reducción al absurdo)\\

    \noindent
    \begin{description}
        \item [Opción 1.] Por reducción al absurdo, supuesto que $\{x_n\}\not\to x$, tenemos entonces que existe una sucesión parcial $\{x_{\sigma(n)}\}$ y cierto $\varepsilon_0>0$ de forma que:
            \begin{equation}\label{eq:ejercicio_5_rel3}
                \|x_{\sigma(n)} - x\|\geq \varepsilon_0 \qquad \forall n \in \mathbb{N}
            \end{equation}
            Como $K$ es compacto, $\{x_{\sigma(n)}\}$ admite una sucesión parcial $\{x_{\tau(\sigma(n))}\}$ convergente a cierto $z\in K$, luego tenemos que $\{x_{\tau(\sigma(n))}\}\rightharpoonup z$ por esta última convergencia y que $\{x_{\tau(\sigma(n))}\}\rightharpoonup x$ por ser $\{x_n\}\rightharpoonup x$. Como $\sigma(E,E^\ast)$ es Hausdorff tiene que ser $x = z$, por lo que:
            \begin{equation*}
                \|x_{\tau(\sigma(n))} -x\| \to 0
            \end{equation*}
            lo que contradice la afirmación~\eqref{eq:ejercicio_5_rel3}.
        \item [Opción 2.]  Hay una Proposición muy útil a la hora de probar la convergencia de una sucesión a un punto:

            \begin{prop}
                Sea $X$ un espacio topológico, $\{x_n\}$ una sucesión de puntos de $X$ y $x\in X$:
                \begin{equation*}
                    \{x_n\} \to x \Longleftrightarrow \begin{array}{l}
                        \text{Para toda parcial\ } \{x_{\sigma(n)}\} \text{\ existe} \\
                        \text{\ una parcial\ } \{x_{\tau(\sigma(n))}\} \to x
                    \end{array}
                \end{equation*}
                \begin{proof}
                    La implicación de izquierda a derecha es trivial, por lo que basta probar:
                    \begin{description} 
                        \item [$\Longleftarrow )$] Por reducción al absurdo, supongamos que $\{x_n\}\not\to x$, por lo que podemos encontrar un entorno $U_0$ de $x$ y una parcial $\{x_{\sigma(n)}\}$ de forma que:
                            \begin{equation}\label{eq:ejercicio_5_rel3_bis}
                                x_{\sigma(n)} \notin U_0 \qquad \forall n \in \mathbb{N}
                            \end{equation}
                            Sin embargo, tenemos que existe una parcial $\{x_{\tau(\sigma(n))}\}$ convergente a $x$, lo que contradice la afirmación~\eqref{eq:ejercicio_5_rel3_bis}.\qedhere
                    \end{description}
                \end{proof}
            \end{prop}

            \noindent
            Sea $\sigma:\mathbb{N}\to \mathbb{N}$ estrictamente creciente, consideramos $\{x_{\sigma(n)}\} \subset K$. Como $K$ es compacto, existe una parcial $\{x_{\tau(\sigma(n))}\}$  convergente a cierto punto $z\in K$. Razonando igual que en la opción 1 vemos que $z = x$ y aplicando la Proposición superior tenemos el resultado probado.
    \end{description}
\end{ejercicio}

\begin{ejercicio}
    Sea $X$ un espacio topológico y sea $E$ un espacio de Banach. Sean $u,v:X\to E$ dos aplicaciones continuas de $X$ en $(E,\sigma(E,E^\ast))$.
    \begin{enumerate}
        \item Prueba que la aplicación $x\mapsto u(x)+v(x)$ es continua de $X$ en $(E,\sigma(E,E^\ast))$.
        \item Sea $a:X\to \mathbb{R}$ una función continua. Prueba que la aplicación $x\mapsto a(x)u(x)$ es continua de $X$ en $(E,\sigma(E,E^\ast))$.
    \end{enumerate}

    \noindent
    Usaremos la Proposición~\ref{prop:sii_debil}:
    \begin{center}
        $\psi:X\to (E,\sigma(E,E^\ast))$  es continua $\Longleftrightarrow f\circ \psi:X\to \mathbb{R}$ es continua $\quad \forall f\in E^\ast$.
    \end{center}
    \begin{enumerate}
        \item Para el primero, si definimos $\varphi:X\to (E,\sigma(E,E^\ast))$ dada por:
            \begin{equation*}
                \varphi(x) = u(x) + v(x)
            \end{equation*}
            Sea $f\in E^\ast$ y $\{x_n\}\to x$ tenemos que:
            \begin{equation*}
                (f\circ \varphi)(x_n) = f(u(x_n) + v(x_n)) = f(u(x_n)) + f(v(x_n)) \qquad \forall n\in \mathbb{N}
            \end{equation*}
            Y como $(f\circ u), (f\circ v)$ son continuas por ser $u$ y $v$ continuas de $X$ en $(E,\sigma(E,E^\ast))$ tenemos pues que:
            \begin{equation*}
                \{(f\circ \varphi)(x_n)\} = \{f(u(x_n)) + f(v(x_n))\} \to f(u(x)) + f(v(x)) = (f\circ \varphi)(x)
            \end{equation*}
            De donde $\{\varphi(x_n)\}\rightharpoonup \varphi(x)$, lo que demostra que $\varphi$ es continua.
        \item Consideramos ahora $\varphi:X\to (E,\sigma(E,E^\ast))$ dada por:
            \begin{equation*}
                \varphi(x) = a(x)u(x)
            \end{equation*}
            Sea $f\in E^\ast$ y $\{x_n\}\to x$ tenemos que:
            \begin{equation*}
                (f\circ \varphi)(x_n) = f(a(x_n)u(x_n)) = a(x_n)f(u(x_n)) \qquad \forall n \in \mathbb{N}
            \end{equation*}
            Y como $f\circ u$ es continua por ser $u$ continua tenemos que:
            \begin{equation*}
                \{(f\circ \varphi)(x_n)\} = \{a(x_n)f(u(x_n))\} \to a(x)f(u(x)) = f(\varphi(x))
            \end{equation*}
            De donde $\{\varphi(x_n)\}\rightharpoonup \varphi(x)$, lo que demuestra que $\varphi$ es continua.
    \end{enumerate} % // TODO: Ver que XxR --> E tmb es continuo
\end{ejercicio}

\begin{ejercicio}
    Sea $E$ un espacio de Banach y sea $A\subset E$ un conjunto cerrado en la topología débil $\sigma(E,E^\ast)$. Sea $B\subset E$ un subconjunto compacto en la topología débil $\sigma(E,E^\ast)$:
    \begin{enumerate}[label=\alph*)]
        \item Prueba que $A+B$ es cerrado en $\sigma(E,E^\ast)$.
        \item Si además $A$ y $B$ son convexos, no vacíos y disjuntos; prueba que existe un hiperplano cerrado que separa estrictamente $A$ y $B$.
    \end{enumerate}
\end{ejercicio}

\begin{ejercicio}
    Sea $E$ un espacio de Banach de dimensión infinita. Nuestro propósito es demostrar que $(E,\sigma(E,E^\ast))$ no es metrizable. Supongamos por reducción al absurdo que existe una distancia $d$ en $E$ que induce la topología $\sigma(E,E^\ast)$.
    \begin{enumerate}
        \item Para cada $k\in \mathbb{N}$ denotamos por $V_k$ a un entorno de $0$ en la topología $\sigma(E,E^\ast)$ de forma que:
            \begin{equation*}
                V_k\subset \left\{x\in E : d(x,0) < \frac{1}{k}\right\}
            \end{equation*}
            Prueba que existe una sucesión $\{f_n\}$ de puntos de $E^\ast$ de forma que cada $g\in E^\ast$ es una combinación lineal finita de puntos de $\{f_n : n\in \mathbb{N}\}$.\newline
            (\textbf{Pista:} Usa el Ejercicio~\ref{ej:lineal_debil*})
        \item Dedue que $E^\ast$ tiene dimensión finita.\newline
            (\textbf{Pista:} Usa el Lema de la categoría de Baire como en el Ejercicio~\ref{ej:5_rel1})
        \item Finaliza la prueba.
        \item Prueba por un método similar que $(E^\ast,\sigma(E^\ast,E))$ no es metrizable.
    \end{enumerate}
\end{ejercicio}

\begin{ejercicio}
    Sea $E$ Un espacio de Banach, sea $M\subset E$ un subespacio vectorial y sea $f_0\in E^\ast$. Prueba que existe $g_0\in M^\perp$ de forma que
    \begin{equation*}
        \inf_{g\in M^\perp}\|f_0-g\| = \|f_0-g_0\|
    \end{equation*}
    Puede hacerse por dos métodos:
    \begin{enumerate}
        \item Usar el Teorema 1.12 del Brezis.
        \item Usar la topología débil$-\ast$ $\sigma(E^\ast,E)$.
    \end{enumerate}
\end{ejercicio}

% // TODO: Falta el ejercicio 10, creo k no se puede hacer porque no hemos visto T*
\setcounter{ejercicio}{10}
\begin{ejercicio}
    Sea $E$ un espacio de Banach y sea $A:E\to E^\ast$ una aplicación monótona definida en $D(A) = E$, vea el Ejercicio~\ref{ej:6_rel2}. Supongamos que para cada $x,y\in E$ la aplicación $\mathbb{R}\to \mathbb{R}$ dada por:
    \begin{equation*}
        t\longmapsto \langle A(x+ty),y \rangle 
    \end{equation*}
    es continua en $t=0$. Prueba que $A:(E,\tau_E)\to (E,\sigma(E^\ast,E))$ es continua.
\end{ejercicio}
