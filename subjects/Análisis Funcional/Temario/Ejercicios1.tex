\chapter{Relaciones de Ejercicios}
\noindent
Las siguientes relaciones de ejercicios corresponden a los ejercicios que uno puede encontrar en el libro ``Functional Analysis'', de Haim Brezis. Concretamente, se encuentran los primeros ejercicios de los Capítulos 1 y 2.
\section{El Espacio Dual}

\begin{ejercicio}
    Sea $E$ un espacio normado, definimos $\forall x\in E$:
    \begin{equation*}
        F(x) = \left\{f\in E^\ast : \|f\| = \|x\| \quad \text{y}\quad f(x)=\|x\|^2\right\}
    \end{equation*}
    Se pide:
    \begin{enumerate}[label=\alph*)]
        \item Probar que
            \begin{equation*}
                F(x) = \{f\in E^\ast : \|f\|\leq \|x\| \quad \text{y}\quad f(x)=\|x\|^2\}
            \end{equation*}
            y deducir que $F(x)$ es no vacío, cerrado y convexo.
            \begin{description}
                \item [$\subset)$] Es evidente.
                \item [$\supset)$] Si $x=0$ la igualdad es evidente. Supuesto que $x\neq 0$, denotamos por $\tilde{F}(x)$ al conjunto de la derecha y tenemos que si $f\in \tilde{F}(x)$, entonces:
                    \begin{equation*}
                        \|x\|^2 = |f(x)| = f(x) \leq \|f\|\|x\| \Longrightarrow \|f\|\geq \|x\|
                    \end{equation*}
                    Por lo que $\|f\| = \|x\|$, de donde $\tilde{F}(x) = F(x)$.
            \end{description}
            \begin{itemize}
                \item Por el Corolario~\ref{coro:existencia_f0} sabemos que $F(x)\neq \emptyset $.
                \item Sea $\{f_n\}\to f\in E^\ast$ con $f_n\in F(x) \quad \forall n\in \mathbb{N}$, entonces:
                    \begin{align*}
                        f(x) &= \lim_{n\to\infty}f_n(x) = \lim_{n\to\infty}\|x\|^2 = \|x\|^2 \\
                        \|f\| &= \left\|\lim_{n\to\infty}f_n\right\| = \lim_{n\to\infty}\|f_n\| = \lim_{n\to\infty}\|x\| = \|x\|
                    \end{align*}
                    Por lo que $f\in F(x)$, de donde $F(x)$ es cerrado.
                \item Sean $f,g\in F(x)$, si tomamos $t\in [0,1]$, definimos:
                    \begin{equation*}
                        h = tf + (1-t)g
                    \end{equation*}
                    $h$ es lineal y continua, y además:
                    \begin{align*}
                        h(x) &= tf(x) + (1-t)g(x) = t\|x\|^2 + (1-t)\|x\|^2 = \|x\|^2 \\
                        \|h\| &= \|tf + (1-t)g\| \leq t\|f\| + (1-t)\|g\| = t\|x\| + (1-t)\|x\| = \|x\|
                    \end{align*}
                    Por lo que $h\in \{f\in E^\ast : \|f\| \leq \|x\| \quad \text{y}\quad f(x) = \|x\|^2\} = F(x)$, lo que demuestra que $F(x)$ es convexo.
            \end{itemize}
        \item Probar que si $E^\ast$ es estrictamente convexo, entonces $F(x)$ se reduce a un punto.

            Que $E^\ast$ sea estrictamente convexo significa que si tomamos $f,g\in E^\ast$ con $f\neq g$ y $\|f\| = 1 = \|g\|$, entonces:
            \begin{equation*}
                \|tf + (1-t)g\| < 1 \qquad \forall t\in \left]0,1\right[
            \end{equation*}
            Si $x=0$ entonces $F(x)$ es unitario. Si $x\neq 0$, supongamos que existen dos funciones $g_1,g_2\in F(x)$ con $g_1\neq g_2$. En cuyo caso, podemos tomar:
            \begin{equation*}
                f_1 = \dfrac{g_1}{\|x\|}, \qquad f_2 = \dfrac{g_2}{\|x\|}
            \end{equation*}
            que verifican $f_1\neq f_2$ y $\|f_1\| = 1 = \|f_2\|$. Por la convexidad estricta de $E^\ast$ tenemos que:
            \begin{equation*}
                \|tf_1 + (1-t)f_2\| < 1 \qquad \forall t\in \left]0,1\right[
            \end{equation*}
            Sin embargo, fijado $t\in \left]0,1\right[$, vemos que:
            \begin{equation*}
                \|x\| = t\|x\| + (1-t)\|x\| = (tf_1+(1-t)f_2)(x) \leq \|tf_1+(1-t)f_2\|\|x\|
            \end{equation*}
            de donde deducimos que $\|tf_1 + (1-t)f_2\|\geq 1$, \underline{contradicción}, que viene de suponer que $F(x)$ contiene dos elementos distintos.
        \item Probar que:
            \begin{equation*}
                F(x) = \left\{f\in E^\ast : \dfrac{1}{2}\|y\|^2 - \dfrac{1}{2}\|x\|^2 \geq f(y-x)\quad \forall y\in E\right\}
            \end{equation*}
            \begin{description}
                \item [$\subset )$] Si $f\in F(x)$, entonces:
                    \begin{equation*}
                        f(y) \leq \|f\|\|y\| \leq \dfrac{1}{2}\|f\|^2 + \dfrac{1}{2}\|y\|^2 = \dfrac{1}{2}\|x\|^2 + \dfrac{1}{2}\|y\|^2 \qquad \forall y\in E
                    \end{equation*}
                    De donde si restamos $f(x) = \|x\|^2$ a ambos lados:
                    \begin{equation*}
                        f(y-x) = f(y) - f(x)\leq \dfrac{1}{2} \|x\|^2 + \dfrac{1}{2}\|y\|^2 - \|x\|^2 = \dfrac{1}{2}\|y\|^2 - \dfrac{1}{2}\|x\|^2 \quad \forall y\in E
                    \end{equation*}
                \item [$\supset )$] Supongamos que tenemos $f\in E^\ast$, $x\in E$ fijo de forma que se cumple:
                    \begin{equation*}
                        f(y-x) \leq \dfrac{1}{2}\|y\|^2 - \dfrac{1}{2}\|x\|^2 \qquad \forall y\in E
                    \end{equation*}
                    Para probar primero que $f(x) = \|x\|^2$, tomaremos $y=\lm x$, con $\lm\in \mathbb{R}^+$:
                    \begin{equation*}
                        (\lm - 1)f(x) = f(\lm x- x) \leq \dfrac{1}{2}\|\lm x\|^2 - \dfrac{1}{2}\|x\|^2 = \dfrac{1}{2}\|x\|^2 \left(\lm^2 - 1\right) \quad \forall \lm\in \mathbb{R}^+
                    \end{equation*}
                    Distinguimos casos (notemos que si $\lm=1$ la desigualdad sigue siendo cierta):
                    \begin{itemize}
                        \item Si $\lm > 1$, entonces:
                            \begin{equation*}
                                f(x) \leq \dfrac{1}{2}\|x\|^2 \left(\dfrac{\lm^2-1}{\lm - 1}\right) \qquad \forall \lm >1
                            \end{equation*}
                        \item Si $\lm < 1$, entonces:
                            \begin{equation*}
                                f(x) \geq \dfrac{1}{2}\|x\|^2 \left(\dfrac{\lm^2-1}{\lm - 1}\right) \qquad \forall \lm \in \left]0,1\right[
                            \end{equation*}
                    \end{itemize}
                    Como tenemos que:
                    \begin{equation*}
                        \lim_{\lm\to1}\dfrac{\lm^2 - 1}{\lm-1} = \lim_{\lm\to1} \dfrac{(\lm+1)(\lm-1)}{\lm-1} = \lim_{\lm\to1} (\lm + 1)= 2
                    \end{equation*}
                    Del primer punto deducimos que $f(x) \leq \|x\|^2$, y del segundo punto que $f(x) \geq \|x\|^2$. Por tanto, tenemos que $f(x) = \|x\|^2$.

                    Para ver que $\|f\|\leq \|x\|$, tomamos $y\in E$ con $\|y\| = \delta>0$, con lo que:
                    \begin{equation*}
                        f(y)-f(x) = f(y-x) \leq \dfrac{1}{2}\delta^2 - \dfrac{1}{2}\|x\|^2
                    \end{equation*}
                    de donde:
                    \begin{equation*}
                        f(y) \leq \dfrac{1}{2}\delta^2 +\dfrac{1}{2}\|x\|^2
                    \end{equation*}
                    Si ahora observamos que:
                    \begin{equation*}
                        \delta\|f\| = \delta\sup_{\|x\|=1}|f(x)| = \sup_{\|x\| =1}|f(\delta x)| = \sup_{\|x\| = \delta}|f(x)| \leq \dfrac{1}{2}\delta^2 + \dfrac{1}{2}\|x\|^2
                    \end{equation*}
                    Si tomamos $\delta=\|x\|$, tenemos que:
                    \begin{equation*}
                        \|x\|\|f\| \leq \|x\|^2 \Longrightarrow \|f\|\leq \|x\|
                    \end{equation*}
            \end{description}
        \item Deducir que:
            \begin{equation*}
                (f-g)(x-y) \geq 0 \qquad \forall x,y\in E, \quad \forall f\in F(x), \quad \forall g\in F(y)
            \end{equation*}
            de hecho:
            \begin{equation*}
                (f-g)(x-y) \geq {(\|x\|- \|y\|)}^{2} \qquad \forall x,y\in E, \quad \forall f\in F(x), \quad \forall g\in F(y)
            \end{equation*}

            Para probar la primera desigualdad, sean $x,y\in E$, si tomamos $f\in F(x)$, $g\in F(y)$, entonces por el apartado anterior tenemos que:
            \begin{gather*}
                f(z-x) \leq \dfrac{1}{2}\|z\|^2 - \dfrac{1}{2}\|x\|^2 \qquad \forall z\in E \\
                g(z-y) \leq \dfrac{1}{2} \|z\|^2 - \dfrac{1}{2}\|y\|^2 \qquad \forall z\in E
            \end{gather*}
            Tomando en la primera desigualdad $z=y$, $z=x$ en la segunda y sumando ambas obtenemos:
            \begin{equation*}
                f(y-x) + g(x-y) \leq 0
            \end{equation*}
            De donde:
            \begin{equation*}
                (f-g)(x-y) = f(x-y) - g(x-y) = -(f(y-x) +g(x-y)) \geq 0
            \end{equation*}
            Para probar que bajo las mismas hipótesis tenemos $(f-g)(x-y)\geq {(\|x\|-\|y\|)}^{2}$:
            \begin{equation*}
                (f-g)(x-y) = f(x) - f(y) -g(x) + g(y) = \|x\|^2 -f(y)-g(x) + \|y\|^2
            \end{equation*}
            Ahora, observamos que:
            \begin{equation*}
                f(y)+g(x) \leq \|f\|\|y\| + \|g\|\|x\| = 2\|x\|\|y\| \Longrightarrow -2\|x\|\|y\|\leq -f(y)-g(x)
            \end{equation*}
            de donde:
            \begin{equation*}
                (f-g)(x-y) = \|x\|^2 - f(y) - g(x) + \|y\|^2 \geq \|x\|^2 -2\|x\|\|y\| + \|y\|^2 = {(\|x\|+\|y\|)}^{2}
            \end{equation*}
        \item Sea $E^\ast$ un espacio estrictamente convexo con $x,y\in E$ de forma que:
            \begin{equation*}
                (f-g)(x-y) = 0 \qquad \forall x,y\in E, \quad \forall f\in F(x), \quad \forall g\in F(y)
            \end{equation*}
            Probar que $F(x) = F(y)$.

            \noindent
            Sean $x,y\in E$, $f\in F(x)$, $g\in F(y)$, si aplicamos la desigualdad del apartado anterior junto con la propiedad que nos dan ahora:
            \begin{equation*}
                0 = (f-g)(x-y) \geq {(\|x\|-\|y\|)}^{2}\geq 0 \Longrightarrow \|x\| = \|y\|
            \end{equation*}
            Del apartado $c)$ obtenemos que (usando además que $\|x\|=\|y\|$):
            \begin{gather*}
                f(y) - \|x\|^2 = f(y-x) \leq \dfrac{1}{2}\|y\|^2 - \dfrac{1}{2}\|x\|^2 = 0 \Longrightarrow f(y)\leq \|x\|^2 \\
                g(x) - \|y\|^2 = g(x-y) \leq \dfrac{1}{2}\|x\|^2 - \dfrac{1}{2}\|y\|^2 = 0 \Longrightarrow g(x) \leq \|y\|^2 = \|x\|^2
            \end{gather*}
            Además, tenemos que:
            \begin{equation*}
                0 = (f-g)(x-y) = f(x)-f(y)-g(x)+g(y) = \|x\|^2 -f(y)-g(x)+\|y\|^2 
            \end{equation*}
            luego:
            \begin{equation*}
                f(y)+g(x) = \|x\|^2 + \|y\|^2 = 2\|x\|^2
            \end{equation*}
            Sin embargo, como $f(y),g(x)\leq \|x\|^2$, concluimos que ha de ser:
            \begin{equation*}
                g(x) = \|x\|^2 = \|y\|^2 = f(y)
            \end{equation*}
            Finalmente, como $E^\ast$ es un espacio estrictamente convexo, tenemos por el apartado $b)$ que tanto $F(x)$ como $F(y)$ se reducen a un punto:
            \begin{equation*}
                \{f\} = F(x) = \{f\in E^\ast : \|f\| = \|x\| \quad \text{y}\quad f(x)=\|x\|^2\}
            \end{equation*}
            Sin embargo, tenemos que $\|g\| = \|x\|$ y que $g(x) = \|x\|^2$, lo que nos dice que $g\in F(x) = \{f\}$, por lo que $f=g$ y tenemos $F(x) = F(y)$.
    \end{enumerate}
\end{ejercicio}

\begin{ejercicio}\label{ej:2_rel1}
    Sea $E$ un espacio vectorial de dimensión $n$ y sea $\{e_i\}_{1\leq i \leq n}$ una base de $E$, dado $x\in E$ escribimos $x =\sum_{i=1}^{n}x_ie_i$ con $x_i\in \mathbb{R}$. Dado $f\in E^\ast$, definimos $f_i = f(e_i)$.
    \begin{enumerate}[label=\alph*)]
        \item Considerar en $E$ la norma:
            \begin{equation*}
                \|x\|_1 = \sum_{i=1}^{n}|x_i|
            \end{equation*}
            \begin{enumerate}
                \item Calcular explícitamente, en términos de $f_i$, la norma $\|f\|$ para $f\in E^\ast$.\\

                    Hemos visto que $\|f\| = \sup_{\|x\|_1=1}|f(x)|$, por lo que si tomamos $x\in E$ con $\|x\|_1=1$, tenemos entonces que $\sum_{i=1}^{n}|x_i| = 1$, de donde:
                    \begin{align*}
                        |f(x)| &= \left|f\left(\sum_{i=1}^{n}x_ie_i\right)\right| = \left|\sum_{i=1}^{n}x_if_i\right| \leq \sum_{i=1}^{n}|x_i||f_i|  \leq \sum_{i=1}^{n}|x_i|\max_{1\leq i \leq n}|f_i| \\
                               &= \max_{1\leq i\leq n}|f_i| \sum_{i=1}^{n}|x_i| = \max_{1\leq i\leq n}|f_i|
                    \end{align*}
                    Luego $\|f\|\leq \max_{1\leq i\leq n}|f_i|$. Sin embargo, si tenemos que $p\in \{1,\ldots,n\}$ es el índice en el cual se maximiza $|f_i|$, es decir, $|f_p| = \max_{1\leq i\leq n}|f_i|$, si tomamos:
                    \begin{equation*}
                        x = e_j
                    \end{equation*}
                    Tenemos que $\|x\|_1=1$, así como que:
                    \begin{equation*}
                        |f(x)| = |f(e_j)| = |f_j| = \max_{1\leq i \leq n}|f_i|
                    \end{equation*}
                    Por lo que el supremo se alcanza, luego:
                    \begin{equation*}
                        \|f\| = \max_{1\leq i \leq n}|f_i|
                    \end{equation*}
                \item Determinar explícitamente el conjunto $F(x)$, para todo $x\in E$.

                    Veamos que:
                    \begin{gather*}
                        f\in F(x) = \{f\in E^\ast : \|f\| = \|x\|_1 \quad \text{y}\quad f(x) = \|x\|_1^2\} \\
                        \text{si y solo si} \\
                        f_i = \left\{\begin{array}{cl}
                            sgn(x_i)\|x\|_1 & \text{si\ } x_i\neq 0  \\
                            \text{cualquier valor en } [-\|x\|_1, \|x\|_1]& \text{si\ } x_i=0 
                        \end{array}\right. \qquad \forall i \in \{1,\ldots,n\}
                    \end{gather*}
                    \begin{description}
                        \item [$\Longleftarrow )$] Notemos que:
                            \begin{equation*}
                                f(x) = \sum_{i=1}^{n}x_if_i = \sum_{i=1}^{n} x_isgn(x_i)\|x\|_1 = \|x\|_1 \sum_{i=1}^{n} |x_i| = \|x\|_1 \|x\|_1 = \|x\|_1^2
                            \end{equation*}
                            y que:
                            \begin{equation*}
                                \|f\| = \max_{1\leq i\leq n}|f_i| = \max\left\{\max_{x_i\neq 0} |sgn(x_i)\|x\|_1|,~\max_{x_i = 0}\{f_i\} \right\} \AstIg \max_{x_i\neq 0} |\|x\|_1| = \|x\|_1
                            \end{equation*}
                            donde en $(\ast)$ hemos usado que si $x_i=0$, entonces el valor de $f_i$ está en el intervalo $[-\|x\|_1, \|x\|_1]$.
                        \item [$\Longrightarrow )$] Sea $f\in E^\ast$ con $\|f\| = \|x\|_1$ y $f(x) = \|x\|_1^2$, entonces:
                            \begin{equation*}
                                \sum_{i=1}^{n}x_if_i = f(x) = \|x\|_1^2 = \sum_{i=1}^{n} |x_i| \|x\|_1
                            \end{equation*}
                            Ahora, como $\max_{1\leq i\leq n}|f_i| = \|f\| = \|x\|_1$, tenemos que:
                            \begin{equation*}
                                |f_i| \leq \|x\|_1 \qquad \forall i \in \{1,\ldots,n\}
                            \end{equation*}
                            Por lo que tenemos:
                            \begin{gather*}
                                x_if_i \leq |x_i||f_i| \leq |x_i|\|x\|_1 \qquad \text{y} \qquad 
                                \sum_{i=1}^{n}x_if_i = \sum_{i=1}^{n}|x_i|\|x\|_1
                            \end{gather*}
                            Luego ha de ser:
                            \begin{equation*}
                                x_if_i = |x_i|\|x\|_1 \qquad \forall i \in \{1,\ldots,n\}
                            \end{equation*}
                            De donde (si $x_i \neq 0$):
                            \begin{equation*}
                                f_i = \dfrac{|x_i|\|x\|_1}{x_i} = sgn(x_i)\|x\|_1
                            \end{equation*}
                            y para el resto de valores podemos tomar cualquier valor que no se salga del intervalo $[-\|x\|_1,\|x\|_1]$, para no alterar el valor de $\|f\|$.
                    \end{description}
            \end{enumerate}
        \item Las mismas preguntas pero para la norma:
            \begin{equation*}
                \|x\|_\infty = \max_{1\leq i \leq n}|x_i|
            \end{equation*}

            Sea $x\in E$ con $\|x\|_\infty=1$, tenemos entonces que $\max_{1\leq i\leq n}|x_i| =1$, de donde:
            \begin{align*}
                |f(x)| = \left|f\left(\sum_{i=1}^{n}x_ie_i\right)\right| = \left|\sum_{i=1}^{n}x_if_i\right| \leq \sum_{i=1}^{n}|x_i||f_i| \leq \sum_{i=1}^{n}|f_i|
            \end{align*}
            por lo que $\|f\| \leq \sum_{i=1}^{n}|f_i|$, pero si tomamos:
            \begin{equation*}
                x = (sgn(f_1),sgn(f_2),\ldots,sgn(f_n))
            \end{equation*}
            tenemos entonces que $\|x\|_\infty=1$, con:
            \begin{equation*}
                f(x) = \sum_{i=1}^{n}x_if_i = \sum_{i=1}^{n}sgn(f_i)f_i = \sum_{i=1}^{n}|f_i|
            \end{equation*}
            Tenemos que el supremo se alcanza, por lo que:
            \begin{equation*}
                \|f\| = \sum_{i=1}^{n}|f_i|
            \end{equation*}

            Si pensamos ahora en el conjunto $F(x)$, si definimos:
            \begin{equation*}
                I = \{i \in \{1,\ldots,n\} : |x_i| = \|x\|_\infty\}
            \end{equation*}
            veamos que:
            \begin{gather*}
                f\in F(x) = \{f\in E^\ast : \|f\| = \|x\|_\infty \quad \text{y}\quad f(x) = \|x\|_\infty^2\} \\
                \text{si y solo si} \\
                \left\{\begin{array}{l}
                    f_i = 0 \qquad \forall i \notin I \\
                    x_if_i \geq 0 \quad \forall i \in I \quad \text{y}\quad \sum_{i \in I}|f_i| = \|x\|_\infty
                \end{array}\right.
            \end{gather*}
            \begin{description}
                \item [$\Longleftarrow )$] Si las $f_i$ cumplen lo enunciado, entonces:
                    \begin{equation*}
                        f(x) = \sum_{i=1}^{n} x_if_i = \sum_{i \in I} x_if_i = \sum_{i \in I}|x_i||f_i| = \|x\|_\infty \sum_{i \in I}|f_i| = \|x\|_\infty \|x\|_\infty = \|x\|_\infty^2
                    \end{equation*}
                    y también:
                    \begin{equation*}
                        \|f\| = \sum_{i=1}^{n} |f_i| = \sum_{i \in I} |f_i| = \|x\|_\infty
                    \end{equation*}
                \item [$\Longrightarrow )$] Sea $f\in E^\ast$ con $\|f\| = \|x\|_\infty$ y $f(x) = \|x\|_\infty^2$, entonces:
                    \begin{itemize}
                        \item Si $f_i = 0\quad \forall i\notin I$, entonces:
                            \begin{equation*}
                                \|x\|_\infty = \|f\| = \sum_{i=1}^{n}|f_i| = \sum_{i \in I}|f_i|
                            \end{equation*}
                            Además:
                            \begin{equation*}
                                \sum_{i \in I} x_if_i = \sum_{i=1}^{n}x_if_i = f(x) =  \|x\|_\infty^2 = \sum_{i=1}^{n}|f_i|\|x\|_\infty = \sum_{i \in I}|f_i|\|x\|_\infty
                            \end{equation*}
                            y tenemos las desigualdades:
                            \begin{equation*}
                                x_i \leq |x_i| \leq \|x\|, \qquad f_i \leq |f_i| \qquad \forall i \in \{1,\ldots,n\}
                            \end{equation*}
                            lo que nos permite igualar término a término en la suma anterior, obteniendo:
                            \begin{equation*}
                                x_if_i = |f_i|\|x\|_\infty = |f_i| \max_{1\leq i\leq n}\{|x_i|\} \Longrightarrow x_if_i \geq 0 \quad \forall i \in I
                            \end{equation*}
                            y se tienen las dos condiciones buscadas.
                        \item Si suponemos ahora que existe $f_j\neq 0$ para $j\notin I$, tendremos entonces que $|x_j|<\|x\|$. Si observamos que:
                            \begin{equation*}
                                \sum_{i=1}^{n}x_if_i = f(x) = \|x\|_\infty^2 = \sum_{i=1}^{n}|f_i|\|x\|
                            \end{equation*}
                            y las desigualdades:
                            \begin{equation*}
                                x_i\leq |x_i|\leq \|x\|, \qquad f_i\leq |f_i| \qquad \forall i \in \{1,\ldots,n\}
                            \end{equation*}
                            deducimos entonces que:
                            \begin{equation*}
                                x_if_i = |f_i|\|x\| \qquad \forall i \in \{1,\ldots,n\}
                            \end{equation*}
                            Sin embargo, tendríamos entonces que (para $i=j$ tenemos $f_j\neq 0$):
                            \begin{equation*}
                                x_j = \dfrac{|f_j|\|x\|}{f_j}= sgn(f_j)\|x\| 
                            \end{equation*}
                            de donde deducimos que $|x_j| = \|x\|$, \underline{contradicción}, por lo que este caso es imposible.
                    \end{itemize}
            \end{description}
        \item Las mismas preguntas pero para la norma:
            \begin{equation*}
                \|x\|_2 = {\left(\sum_{i=1}^{n}|x_i|^2\right)}^{\frac{1}{2}}
            \end{equation*}
            y más generalmente para la norma:
            \begin{equation*}
                \|x\|_p = {\left(\sum_{i=1}^{n}|x_i|^p\right)}^{\frac{1}{p}}, \qquad p\in \left]1,+\infty\right[
            \end{equation*}

            \noindent
            Sea $x\in E$ con $\|x\|_p = 1$, tenemos entonces que ${\left(\sum_{i=1}^{n}|x_i|^p\right)}^{\frac{1}{p}}=1$, de donde:
            \begin{align*}
                |f(x)| = \left|\sum_{i=1}^{n}x_if_i\right| \leq \sum_{i=1}^{n}|x_i||f_i| \stackrel{(\ast)}{\leq} {\left(\sum_{i=1}^{n}|x_i|^p\right)}^{\frac{1}{p}}{\left(\sum_{i=1}^{n}|f_i|^{p'}\right)}^{\frac{1}{p'}} = {\left(\sum_{i=1}^{n}|f_i|^{p'}\right)}^{\frac{1}{p'}}
            \end{align*}
            donde en $(\ast)$ he usado la desigualdad de Hölder. Deducimos por tanto que:
            \begin{equation*}
                \|f\| \leq {\left(\sum_{i=1}^{n}|f_i|^{p'}\right)}^{\frac{1}{p'}}
            \end{equation*}
    \end{enumerate}
\end{ejercicio}

\begin{ejercicio}
    Sea $E = \{u\in C([0,1],\mathbb{R}) : u(0) = 0\}$ con la norma:
    \begin{equation*}
        \|u\| = \max_{t\in [0,1]}|u(t)|
    \end{equation*}
    considera el funcional lineal $f:E\to \mathbb{R}$ dado por:
    \begin{equation*}
        f(u) = \int_{0}^{1} u(t)~dt 
    \end{equation*}
    \begin{enumerate}[label=\alph*)]
        \item Demuestra que $f\in E^\ast$ y calcula $\|f\|$.

            Hemos de probar que $f$ es lineal y continua:
            \begin{itemize}
                \item Por la forma de definir $f$ es claro que es lineal:
                    \begin{multline*}
                        f(\lm u + v) = \int_{0}^{1} \lm u(t) + v(t)~dt  = \lm \int_{0}^{1} u(t)~dt  + \int_{0}^{1} v(t)~dt  = \lm f(u) + f(v) \\ \forall u,v\in E, \quad  \forall \lm \in \mathbb{R}
                    \end{multline*}
                \item $f$ es continua, ya que:
                    \begin{align*}
                        |f(u)| &= \left|\int_{0}^{1} u(t)~dt \right| \leq \left|\int_{0}^{1} \|u\|~dt \right| = \|u\| \qquad \forall u\in E
                    \end{align*}
            \end{itemize}
            En consecuencia, $f\in E^\ast$. En este último punto hemos probado además que $\|f\|\leq 1$. Para probar la otra desigualdad: 
            \begin{description}
                \item [Opción 1.] si para cada $\alpha>0$ definimos $u_\alpha \in E$ dada por:
                    \begin{equation*}
                        u_\alpha(t) = t^{\alpha} \qquad \forall t\in [0,1]
                    \end{equation*}
                    tenemos entonces que:
                    \begin{equation*}
                        f(u_\alpha) = \int_{0}^{1} t^{\alpha}~dt  = \left[\dfrac{t^{\alpha+1}}{\alpha+1}\right]^1_0 = \dfrac{1}{1+\alpha} \qquad \forall \alpha>0
                    \end{equation*}
                    de donde:
                    \begin{equation*}
                        |f(u_\alpha)|  \leq \|f\| \leq 1
                    \end{equation*}
                    con $\lim\limits_{\alpha\to0} |f(u_\alpha)| = 1$, por lo que $\|f\| = 1$.
                \item [Opción 2.] Definimos para cada $n\in \mathbb{N}$ $u_n\in E$  dada por:
                    \begin{equation*}
                        u_n(t) = \left\{\begin{array}{ll}
                            nt & \text{si\ } 0\leq t \leq n \\
                            1 & \text{si\ } \frac{1}{n}\leq t \leq 1
                        \end{array}\right. 
                    \end{equation*}
                    Tenemos que:
                    \begin{equation*}
                        f(u_n) = \int_{0}^{\frac{1}{n}} nt~dt  + \int_{\frac{1}{n}}^{1} 1~dt  = \frac{nt}{2}\int_{0}^{\frac{1}{n}} 1+t~dt  = 1 -\frac{1}{2n}
                    \end{equation*}
                    por lo que:
                    \begin{equation*}
                        \lim_{n\to\infty}f(u_n) = 1
                    \end{equation*}
                    luego $\|f\| = 1$.
            \end{description}
        \item ¿Puede encontrarse $u\in E$ con $\|u\| = 1$ y $f(u) = \|f\|$?
            \begin{description}
                \item [Opción 1.] No, ya que la única función $f:[0,1]\to \mathbb{R}$ continua con integral 1 en $[0,1]$ y con máximo 1 es la constantemente igual a 1, que no pertence a $E$ (por no valer 0 en 0): 

                    Supongamos que tenemos una función $f:[0,1]\to \mathbb{R}$ continua con integral $1$ en $[0,1]$ y con máximo 1 distinta de la constantemente igual a 1. En dicho caso, ha de existir $x_0\in [0,1]$ de forma que $f(x_0) < 1$. Por continuidad de $f$ podemos encontrar $\varepsilon,\delta>0$ de forma que:
                    \begin{equation*}
                        f(x) \leq 1 -\varepsilon \qquad \forall x\in \left]x_0-\delta,x_0+\delta\right[\cap [0,1]
                    \end{equation*}
                    En cuyo caso, llamando $I = \left]x_0-\delta,x_0+\delta\right[\cap [0,1]$:
                    \begin{align*}
                        \int_{0}^{1} f(x)~dx &= \int_{[0,1]\setminus I} f(x)~dx + \int_I f(x)~dx \leq \int_{[0,1]\setminus I} 1~dx + \int_I (1-\varepsilon)~dx \\ &= (1-l(I)) + l(I)(1-\varepsilon) = 1-\varepsilon < 1
                    \end{align*}
                    Por lo que $f$ no tiene integral 1 en $[0,1]$, \underline{contradicción}, que viene de suponer que $f$ no es constantemente igual a 1. 

                    Tras este resultado, como la pertenencia al conjunto $E$ obliga a que la función $u$ no sea constantemente igual a $1$, tenemos pues que no puede existir una tal función.
                \item [Opción 2.] Supongamos que existe $u\in E$ de forma que $\|u\| = 1$ con $f(u) = 1$, esto es equivalente a que:
                    \begin{equation*}
                        f(u) = \int_{0}^{1} u(t)~dt  = 1  \Longleftrightarrow \int_{0}^{1} u(t)-1~dt  = 0
                    \end{equation*}
                    Tomando $g(t) = u(t)-1$, como $\|u\| = 1$ tenemos entonces que $u(t)\leq 1$ para todo $t\in [0,1]$, con lo que $g(t)\leq 0$ para $t\leq [0,1]$, y como su integral en $[0,1]$ vale 0 concluimos que $g(t) = 0$ para todo $t\in [0,1]$, con lo que:
                    \begin{equation*}
                        u(t) = 1 \qquad \forall t\in [0,1]
                    \end{equation*}
                    Pero teníamos que $u\in E$, luego $0 = u(0) = 1$, \underline{contradicción}.
            \end{description}
    \end{enumerate}
\end{ejercicio}

\begin{ejercicio}
    Considera el espacio $E=C_0$ de sucesiones de números reales que convergen a cero con la norma 
    \begin{equation*}
        \|x\| = \sup\{|x(n)| : n\in \mathbb{N}\} \qquad \forall x\in E
    \end{equation*}
    Para cada elemento $u\in E$ definimos:
    \begin{equation*}
        f(u) = \sum_{n=1}^{+\infty}\dfrac{1}{2^n}u(n)
    \end{equation*}
    \begin{enumerate}[label=\alph*)]
        \item Comprueba que $f\in E^\ast$ y calcula $\|f\|$.

            \begin{itemize}
                \item Para ver que $f$ es lineal:
                    \begin{multline*}
                        f(\lm u + v) = \sum_{n=1}^{\infty} \dfrac{\lm u(n)+v(n)}{2^n} = \lm \sum_{n=1}^{\infty}u(n) + \sum_{n=1}^{\infty}v(n) = \lm f(u)+f(v) \\ \forall u,v\in E, \quad \forall \lm\in \mathbb{R}
                    \end{multline*}
                \item Para ver que $f$ es continuo:
                    \begin{align*}
                        |f(u)-f(v)| &= \left|\sum_{n=1}^{\infty}\dfrac{u(n)}{2^n} - \sum_{n=1}^{\infty}\dfrac{v(n)}{2^n}\right| = \left|\sum_{n=1}^{\infty}\dfrac{u(n)-v(n)}{2^n}\right| = \left|\sum_{n=1}^{\infty}\dfrac{\|u-v\|}{2^n}\right| \\ &= \|u-v\| \left|\sum_{n=1}^{\infty}\dfrac{1}{2^n}\right| = \|u-v\|
                    \end{align*}
            \end{itemize}
            Por lo que $f\in E^\ast$. En este último punto hemos visto también que $\|f\|\leq 1$.

            % // TODO: HACER
            % Si ahora tomamos $u(n) = 1$ para todo $n\in \mathbb{N}$, tenemos que $\|u\|=1$, con:
            % \begin{equation*}
            %     f(u) = \sum_{n=1}^{\infty} \dfrac{1}{2^n} = 1
            % \end{equation*}
            % Por lo que $\|f\|=\sup\limits_{\|x\|\leq 1}|f(x)| \geq 1$, de donde tenemos que $\|f\| = 1$.
        \item ¿Puede encontrarse $u\in E$ con $\|u\| = 1$ y $f(u) = \|f\|$?

            % // No, no se puede
            % Sí, tomando $u(n) = 1$ para todo $n\in \mathbb{N}$ obtenemos $\|u\| =1$ y $f(u) = 1 = \|f\|$.
    \end{enumerate}
\end{ejercicio}

\begin{ejercicio}
    Sea $E$ un espacio normado de dimensión infinita:
    \begin{enumerate}[label=\alph*)]
        \item Demuestra (usando el Lema de Zorn) que existe una base algebraica $\{e_i\}_{i \in I}$ en $E$ de forma que $\|e_i\| = 1\quad \forall i \in I$.

            Recordamos que una base algebraica (o de Hamel) es un subconjunto $\{e_i\}_{i \in I}$ de $E$ de forma que todo $x\in E$ puede ser escrito de forma única como:
            \begin{equation*}
                x = \sum_{i \in J} x_i e_i, \quad \text{con\ } \quad J\subset I,\quad  J \text{\ finito}
            \end{equation*}

            Consideramos (que es no vacío puesto que $\{x\}\in P$ para todo $x\in E$):
            \begin{equation*}
                P = \{C\subset E : \text{los elementos de\ } C \text{\ son linealmente independientes}\}
            \end{equation*}
            Y buscamos aplicar el Lema de Zorn a $P$. Para ello, definimos:
            \begin{equation*}
                C \leq D \Longleftrightarrow C \subset D \qquad \forall C,D\in P
            \end{equation*}
            Ahora, hemos de probar primero que $P$ es inductivo. Para ello, sea $Q\subset P$ un subconjunto totalmente ordenado, tratemos de probar que $\cup Q$ es una cota superior de $Q$. Es claro que $C\subset \cup Q$ para todo $C\in Q$, por lo que basta probar que $\cup Q \in P$. Si tomamos $x,y\in \cup Q$, tendremos entonces que existen $C,D\in Q$ de forma que $x\in C$ y $y\in D$. Como $Q$ es totalmente ordenado, tendremos bien $C\subset D$ bien $D\subset C$. Aprovechando la simetría de la situación, supondremos que $C\subset D$, con lo que también tenemos $x\in D\in Q\subset P$, de donde $x$ e $y$ son linealmente independientes, como queríamos probar, lo que nos dice que $\cup Q \in P$.

            Aplicando el Lema de Zorn, tenemos que $P$ tiene un elemento maximal, es decir, existe $\cc{B}\in P$ de forma que si $C\subset E$ es un conjunto tal que todos sus elementos son linealmente independientes entonces $C\subset \cc{B}$. Probaremos ahora que $\cc{B}$ es una base de $E$. Para ello, sea $x\in E$, si $x$ es linealmente independiente de los elementos de $\cc{B}$, entonces consideramos $B = \cc{B}\cup \{x\}$, que es un elemento de $P$ mayor que el elemento maximal de $P$, \underline{contradicción}, por lo que $x$ ha de ser linealmente dependiente de los elementos de $\cc{B}$, es decir, existe una cantidad finita de ellos determinada por $J\subset I$ finito y unos escalares $a_i$ de forma que:
            \begin{equation*}
                x + \sum_{j\in J}a_j x_j = 0, \qquad x_j \in \cc{B}, \quad a_j\in \mathbb{R}, \quad \forall j\in J
            \end{equation*}
            Dicho de otra forma, si tomamos $x_j = -a_j \quad \forall j\in J$:
            \begin{equation*}
                x = \sum_{j\in J} a_jx_j
            \end{equation*}
            Como la normalización de los vectores no modifica su independiencia lineal, podemos normalizar todos los elementos del conjunto $\cc{B}$ y este seguirá cumpliendo lo enunciado.

        \item Construye un funcional lineal $f:E\to \mathbb{R}$ que no sea continuo. 
            \begin{description}
                \item [Ejemplo particular.] Si consideramos $E = \mathbb{R}^{(\mathbb{N})}$ el espacio de sucesiones casi nulas, una base del mismo es:
                    \begin{equation*}
                        \cc{B} = \{e_n \in \mathbb{R}^{(\mathbb{N})} : e_n(n) = 1, e_n(m) = 0, m\neq n\}
                    \end{equation*}
                    Si consideramos la norma:
                    \begin{equation*}
                        \|x\|_\infty = \max_{n\in \mathbb{N}}|x(n)| \qquad \forall x\in E
                    \end{equation*}
                    Veamos que la aplicación lineal dada por la base:
                    \Func{f}{E}{\bb{R}^{(\bb{N})}}{e_n}{2^n}
                    Veamos que el funcional no es continuo en 0, pues para $\varepsilon=1$, para todo $\delta>0$ existe $n\in \mathbb{N}$ con $\frac{1}{2^n}<\delta$ de forma que $\|\delta e_n\| \leq \delta$:
                    \begin{equation*}
                        |f(\delta e_n)| = \delta|f(e^{n})| > \frac{1}{2^n}2^n = 1
                    \end{equation*}
                \item [En general.] Sea $E$ un espacio normado de dimensión infinita, sabemos que existe una base unitaria $\cc{B}$ de $E$. Consideramos $\overline{\cc{B}}\subset \cc{B}$ un subconjunto infinito numerable, por lo que podemos enumerar sus elementos, definimos la aplicación $f:E\to \mathbb{R}$ dada por:
                    \begin{align*}
                        f(e_n) &= 2^n \qquad \forall e_n \in \overline{\cc{B}} \\
                        f(e) &= 0 \qquad \forall e \in \cc{B}\setminus\overline{\cc{B}}
                    \end{align*}
                    Al igual que antes, $f$ no es continua en $0$, pues para $\varepsilon=1$, $\forall \delta>0$ existe $n\in \mathbb{N}$ de forma que $\frac{1}{2^n}<\delta$ con $\|\delta e_n\| = \delta$ y:
                    \begin{equation*}
                        |f(\delta e_n)| = 1
                    \end{equation*}
            \end{description}
        \item Suponiendo que además $E$ es un espacio de Banach, prueba que $I$ no es numerable (\textbf{Pista:} usar ``el Teorema de categoría de Baire'').

            Sea $E$ un espacio de Banach de dimensión infinita, suponemos que existe $\cc{B}$ una base numerable del mismo. Definimos para cada $n\in \mathbb{N}$:
            \begin{equation*}
                A_n = \left\{x\in E : x = \sum_{i=1}^{n}\lm_i e_{i}\right\}
            \end{equation*}
            Veamos que $A_n$ es cerrado y que $\Int(A_n)=\emptyset $, $\forall n\in \mathbb{N}$:
            \begin{itemize}
                \item Sea $\{x_m\}\to x\in E$ con $x_m\in A_n$ $\forall m\in \mathbb{N}$, como $A_n$ es un subespacio vectorial de $E$ de dimensión finita, entonces $A_n\cong \mathbb{R}^n$, luego $A_n$ es completo y como $\{x_m\}$ es de Cauchy, ha de ser convergente en $A_n$, de donde $x\in A_n$.
                \item Sea $x\in A_n$, consideramos para cierto $\varepsilon>0$ la bola $B(x,\varepsilon)$, y vemos que:
                    \begin{gather*}
                        x+\frac{\varepsilon}{2} e_{n+1}\in B(x,\varepsilon) \\
                        x+\frac{\varepsilon}{2}\notin A_n
                    \end{gather*}
                    De donde $B(x,\varepsilon)\not\subset A_n$
            \end{itemize}
            Si consideramos $\bigcup_{n\in \mathbb{N}}A_n = E$, el Lema de Baire nos dice que $\Int E = \emptyset $, \underline{contradicción}, puesto que $\Int E = E$, que viene de suponer que la base $\cc{B}$ es numerable.
    \end{enumerate}
\end{ejercicio}

\begin{ejercicio} % // TODO: HACER
    Sea $E$ un espacio normado y $H\subset E$ un hiperplano. Sea $V\subset E$ un subespacio afín que contiene a $H$.
    \begin{enumerate}[label=\alph*)]
        \item Probar que $V=H$ o $V=E$.

            Recordamos que un hiperplano es, dada $0\neq f:E\to \mathbb{R}$ lineal y $\alpha\in \mathbb{R}$:
            \begin{equation*}
                H = \{x\in E : f(x)=\alpha\} = f^{-1}(\{\alpha\})
            \end{equation*}

            % Sea $V$ un espacio afín que contiene a un hiperplano $H =[f=\alpha]$, como $H$ es un espacio afín podría ser $V=H$. Suponemos que $V\neq H$, con lo que ha de existir $v\in E\setminus H$ de forma que $v\in V$, 

            % Suponemos $V\neq H$, y tomamos $0\neq x_0 \in E\setminus H$, con lo que $0\neq x \in V$, de donde ha de existir $v\in E\setminus H^{\rightarrow}$, con $v\in V^{\rightarrow}$. De donde:
            % \begin{equation*}
            %     E = H^{\rightarrow} \oplus \mathbb{R}v \subseteq V^{\rightarrow}
            % \end{equation*}
            % Por lo que $V^{\rightarrow}=E$, de donde $V = E$.

            % // TODO: Lo dice el profe
            % Si H es el hiperplano, quién es el subespacio vectorial asociado a dicho espacio afin
            % x0, x en H <=> x0 - x in ker f, con lo que H = x0 + kerf
        \item Deducir que $H$ es cerrado o denso en $E$.

            Supuesto que $H$ no es cerrado, tenemos que $\overline{H}\neq H$, y además $\overline{H}$ es un espacio afín conteniendo $H$, por lo que por el apartado anterior ha de ser $\overline{H}=E$, es decir, $H$ es denso en $E$.
    \end{enumerate}
\end{ejercicio}


\begin{ejercicio}
    Sea $E$ un espacio normado y $C\subset E$ un subconjunto convexo.
    \begin{enumerate}[label=\alph*)]
        \item Prueba que $\overline{C}$ y $\Int C$ son convexos.

            \begin{itemize}
                \item Para $\overline{C}$, sean $x,y\in \overline{C}$, entonces existen sucesiones de puntos de $C$ $\{x_n\},\{y_n\}$ con $\{x_n\}\to x$ y $\{y_n\}\to y$, de donde si tomamos $t\in [0,1]$, tenemos que:
                    \begin{equation*}
                        \{(1-t)x_n + ty_n\} \to (1-t)x+ty
                    \end{equation*}
                    Por lo que $(1-t)x+ty\in \overline{C}$ para todo $t\in [0,1]$.
                \item Para $\Int C$, sean $x,y\in \Int C$, entonces existe $r>0$ de forma que $B(x,r),B(y,r)\subset C$. En dicho caso, como $C$ es convexo, tenemos que:
                    \begin{equation*}
                        tB(x,r) + (1-t)B(y,r) \subset C \qquad \forall t\in [0,1]
                    \end{equation*}
                    Pero como:
                    \begin{equation*}
                        tB(x,r) + (1-t)B(y,r) = B(tx+(1-t)y,r)
                    \end{equation*}
                    tenemos entonces que $tx + (1-t)y \in B(tx+(1-t)y,r)\subset C\quad \forall t\in [0,1]$, con lo que $\Int C$ es convexo. La igualdad entre conjuntos que hemos usado se debe principalmente a:
                    \begin{equation*}
                        B(x,r) = x+rB(0,1) \qquad \forall x\in E, \quad \forall r\in \mathbb{R}^+
                    \end{equation*}
                    \begin{description}
                        \item [$\supseteq )$] Si $z\in B(0,1)$ entonces $\|z\|<1$, de donde $\|rz\| < r$, por lo que $\|x-x+rz\| < r$, luego $x+rz\in B(x,r)$.
                        \item [$\subseteq )$] Si $z\in B(x,r)$, entonces $\|x-z\| < r$, por lo que:
                            \begin{equation*}
                                z = x + r\left(\frac{z-x}{r}\right) \quad \text{con} \quad \left\|\frac{z-x}{r}\right\| <1
                            \end{equation*}
                    \end{description}
                    Para ver:
                    \begin{equation*}
                        tB(x,r) + (1-t)B(y,r) = B(tx+(1-t)y,r)
                    \end{equation*}
                    \begin{description}
                        \item [$\subseteq )$] Si tomamos $z\in tB(x,r)+(1-t)B(y,r)$, tenemos entonces que existen $\alpha\in B(x,r)$ y $\beta\in B(y,r)$ de forma que:
                            \begin{equation*}
                                z = t(x+r\alpha) + (1-t)(y+r\beta) = tx+ (1-t)y +r(t\alpha+(1-t)\beta)
                            \end{equation*}
                            y como $t\alpha+(1-t)\beta\in B(0,1)$ por ser convexa, hemos probado que $z\in B(tx+(1-t)y,r)$.
                        \item [$\supseteq )$] Si tomamos ahora $z\in B(tx+(1-t)y,r)$, tenemos que existe un elemento $\alpha\in B(0,1)$ de forma que:
                            \begin{align*}
                                z &= tx +(1-t)y+r\alpha = tx + (1-t)y + tr\alpha + (1-t)r\alpha \\ &= t(x+r\alpha) + (1-t)(y+r\alpha) \in tB(x,r)+ (1-t)B(y,r)
                            \end{align*}
                    \end{description}
            \end{itemize}
        \item Dado $x\in C$ y $y\in \Int C$, prueba que $tx+(1-t)y\in \Int C\quad \forall t\in \left]0,1\right[$ .

            Sea $r>0$ de forma que $B(y,r)\subset C$, como $C$ es convexo tenemos que:
            \begin{equation*}
                tx+(1-t)B(y,r)\subset C \qquad \forall t\in [0,1]
            \end{equation*}
            lo que nos dice que:
            \begin{equation*}
                C\ni tx + (1-t)(y+r\alpha) = tx + (1-t)y + (1-t)r\alpha \qquad \forall \alpha\in B(0,1), \quad \forall t\in [0,1]
            \end{equation*}
            por tanto, tendremos que $B(tx+(1-t)y,(1-t)r)\subset C$ para todo $t\in [0,1]$, de donde $tx+(1-t)y\in \Int C$, para todo $t\in [0,1]$.
        \item Deduce que $\overline{C} = \overline{\Int C}$ siempre que $\Int C\neq \emptyset $.

            Como $\Int C\subset C$, tenemos que $\overline{\Int C}\subset \overline{C}$. Sea ahora $x\in \overline{C}$, tenemos que existe una sucesión $\{x_n\}$ de puntos de $C$ con $\{x_n\}\to x$. Como $\Int C \neq \emptyset $, podemos tomar $y\in \Int C$. Sea $\{t_n\}$ una sucesión de puntos de $[0,1]$ convergente a $1$ (por ejemplo, $t_n = 1-\nicefrac{1}{n}$), tenemos entonces que:
            \begin{equation*}
                \{t_n x_n + (1-t_n)y\} \to x
            \end{equation*}
            con $t_nx_n + (1-t_n)y\in \Int C$ para todo $n\in \mathbb{N}$ (por el apartado $b)$), de donde concluimos que $x\in \overline{\Int C}$.
    \end{enumerate}
\end{ejercicio}

\begin{ejercicio}
    Sea $E$ un espacio normado con norma $\|\cdot \|$. Sea $C\subset E$ un abierto convexo de forma que $0\in C$. Si $p$ denota el funcional de Minkowski de $C$:
    \begin{enumerate}[label=\alph*)]
        \item Suponiendo que $C$ es simétrico (es decir, que $-C=C$) y que es acotado, prueba que $p$ es una norma equivalente a $\|\cdot \|$.

            Veamos en primer lugar que $p$ es una norma en $E$:
            \begin{itemize}
                \item $p$ verifica la desigualdad triangular, como vimos en la Proposición~\ref{prop:f_minkowski}.
                \item Sea $x\in E$ de forma que $p(x) = 0$, entonces $\lm x\in C \quad \forall \lm \in \mathbb{R}^+$. Si $x\neq 0$ podemos tomar la sucesión:
                    \begin{equation*}
                        \left\{n \frac{x}{\|x\|}\right\}
                    \end{equation*}
                    que verifica:
                    \begin{equation*}
                        \left\|n\frac{x}{\|x\|}\right\| = \|n\| = n 
                    \end{equation*}
                    por lo que $\{\|n\cdot \nicefrac{x}{\|x\|}\|\}\to \infty$, lo que contradice que $C$ esté acotado, contradicción que viene de suponer que $x\neq 0$, luego ha de ser $x = 0$.
                \item Sean $\lm \in \mathbb{R}$, $x\in E$, distinguimos casos:
                    \begin{itemize}
                        \item Si $\lm \in \mathbb{R}^+$, la Proposición~\ref{prop:f_minkowski} nos dice que $p(\lm x) = \lm p(x)$.
                        \item Si $\lm  = 0$, tenemos que $\lm p(x) = 0 = p(0) = p(\lm x)$.
                        \item Si $\lm\in \mathbb{R}^-$, tenemos que:
                            \begin{equation*}
                                p(\lm x) = p((-\lm)(-x)) \AstIg -\lm p(-x) = |\lm| p(-x)
                            \end{equation*}
                            donde en $(\ast)$ usamos que $-\lm\in \mathbb{R}^+$. De la simetría de $C$ concluimos que $x\in C\Longleftrightarrow -x\in C$. Supuesto que $p(x) \neq p(-x)$, podemos suponer sin pérdida de generalidad que $p(x)<p(-x)$:
                            \begin{equation*}
                                \inf\left\{\alpha\in \mathbb{R}^+ : \frac{x}{\alpha}\in C\right\} = p(x) < p(-x) = \inf\left\{\alpha\in \mathbb{R}^+ : \frac{-x}{\alpha}\in C\right\}
                            \end{equation*}
                            por definición de ínfimo, ha de existir $\alpha\in \mathbb{R}^+$ de forma que:
                            \begin{equation*}
                                \frac{x}{\alpha} \in C \quad \text{y}\quad \frac{-x}{\alpha}\notin C
                            \end{equation*}
                            lo que contradice que $\nicefrac{x}{\alpha}\in C\Longleftrightarrow -\nicefrac{x}{\alpha}\in C$, que viene de suponer que $p(x) \neq p(-x)$.
                    \end{itemize}
                    En conclusión, tenemos que $p$ es una norma en $E$. Para ver que $p$ es equivalente a $\|\cdot \|$:
                    \begin{itemize}
                        \item La Proposición~\ref{prop:f_minkowski} nos dice que $\exists M>0$ de forma que:
                            \begin{equation*}
                                p(x) \leq M\|x\| \qquad \forall x\in E
                            \end{equation*}
                        \item Como $C$ está acotado, ha de existir $L\geq 0$ de forma que:
                            \begin{equation*}
                                \|x\| \leq L \qquad \forall x\in C
                            \end{equation*}
                            Fijado $x\in E$, sea $\alpha \in \left\{\alpha\in \mathbb{R}^+ : \frac{x}{\alpha}\in C\right\}$, tenemos que:
                            \begin{equation*}
                                \frac{1}{\alpha}\|x\| = \left\|\frac{x}{\alpha}\right\| \leq L \Longrightarrow \frac{1}{L}\|x\| \leq \alpha
                            \end{equation*}
                            de donde deducimos que $\frac{1}{L}\|x\| \leq p(x) \quad \forall x\in E$, lo que nos da la otra desigualdad.
                    \end{itemize}
                    Por lo que $p$ y $\|\cdot \|$ son normas equivalentes.
            \end{itemize}
        \item Sea $E = C([0,1], \mathbb{R})$ con la norma:
            \begin{equation*}
                \|u\| = \max_{t\in [0,1]}|u(t)|
            \end{equation*}
            sea:
            \begin{equation*}
                C = \left\{u\in E: \int_{0}^{1} |u(t)|^2~dt < 1 \right\}
            \end{equation*}
            Comprueba que $C$ es convexo, simétrico y que $0\in C$. ¿Está $C$ acotado en $E$? Calcula el funcional de Minkowski $p$ de $C$ y prueba que $p$ es una norma en $E$. ¿Es $p$ equivalente a $\|\cdot \|$?

            Veamos que $C$ es convexo, simétrico y que $0\in C$:
            \begin{itemize}
                \item Si consideramos la función $c_0:[0,1]\to \mathbb{R}$ tal que $c_0(x) = 0\quad \forall x\in [0,1]$, tenemos que:
                    \begin{equation*}
                        \int_{0}^{1} {|c_0(t)|}^{2}~dt  = \int_{0}^{1} 0~dt  = 0 < 1
                    \end{equation*}
                    por lo que $c_0\in C$.
                \item Sea $u\in E$, tenemos que:
                    \begin{equation*}
                        \int_{0}^{1} {|u(t)|}^{2}~dt  = \int_{0}^{1} {| - u(t)|}^{2}~dt 
                    \end{equation*}
                    de donde deducimos que $u\in C \Longleftrightarrow -u\in C$, lo que nos dice que $C = -C$.
                \item Si consideramos:
                    \begin{equation*}
                        \|u\|_2 = {\left(\int_{0}^{1} {|u(t)|}^{2}~dt \right)}^{\frac{1}{2}}
                    \end{equation*}
                    tenemos que % // TODO: TERMINAR
            \end{itemize}
    \end{enumerate}
\end{ejercicio}

\begin{ejercicio}% // TODO: HACER
    Sea $E$ un espacio normado de dimensión finita, sea $C\subset E$ un conjunto no vacío convexo con $0\notin C$. Siempre hay un hiperplano que separa $C$ y $\{0\}$ (Notemos que todo hiperplano es cerrado (¿por qué?). El mayor punto de este ejercicio es que no hace falta exigir nada más sobre $C$).\\

    \noindent
    Como un hiperplano de $E$ es un subespcio vectorial de $E$ de dimensión finita, este será cerrado.

    \begin{enumerate}[label=\alph*)]
        \item Sea $\{x_n\}_{n\in \mathbb{N}}$ subconjunto numerable de $C$ que es denso en $C$ (¿por qué existe?). Para cada $n$ definimos:
            \begin{equation*}
                C_n = conv\{x_1, \ldots, x_n\} = \left\{x=\sum_{i=1}^{n}t_ix_i : t_i\geq 0 \quad \forall i \quad \text{y}\quad \sum_{i=1}^{n}t_i = 1 \right\}
            \end{equation*}
            Comprueba que $C_n$ es compacto y que $\bigcup\limits_{n=1}^\infty C_n$ es denso en $C$.
        \item Prueba que existe un $f_n \in E^\ast$ de forma que:
            \begin{equation*}
                \|f_n\| = 1 \quad \text{y}\quad f_n(x) \geq 0 \quad \forall x\in C_n
            \end{equation*}
        \item Deduce que existe $f\in E^\ast$ de forma que:
            \begin{equation*}
                \|f\| = 1 \quad \text{y}\quad f(x)\geq 0 \quad \forall x\in C
            \end{equation*}
        \item Sean $A,B\subset E$ conjuntos no vacíos disjuntos y convexos. Prueba que existe algún hiperplano $H$ que separa $A$ y $B$.
    \end{enumerate}
\end{ejercicio}

\begin{ejercicio}% // TODO: HACER
    Sea $E$ un espacio normado y sea $I$ cualquier conjunto de índices, fijado un subconjunto $\{x_i\}_{i \in I}$ en $E$ y otro $\{\alpha_i\}_{i \in I}$ en $\mathbb{R}$. Demuestra que las siguientes propiedades son equivalentes:
    \begin{enumerate}
        \item Existe $f\in E^\ast$ de forma que $f(x_i) = \alpha_i\quad \forall i \in I$.
        \item Existe una constante $M\geq 0$ de forma que para cada conjunto finito $J\subset I$ y para cada elección de números reales $\{\beta_i\}_{i \in J}$ tenemos:
            \begin{equation*}
                \left|\sum_{i \in J}\beta_i \alpha_i\right| \leq M \left\|\sum_{i \in J} \beta_i x_i\right\|
            \end{equation*}
    \end{enumerate}
    Notemos que en la prueba $2\Longrightarrow 1$ uno puede encontrar alguna $f\in E^\ast$ con $\|f\| \leq M$. (\textbf{Pista:} intenta primero definir $f$ en el espacio lineal generado por $\{x_i\}_{i \in I}$).\\

    \noindent
    \begin{description}
        \item [$1 \Longrightarrow 2)$] Sea $J\subset I$ finito y $\{\beta_i\}_{i \in J}\subset \mathbb{R}$, tenemos:
            \begin{equation*}
                \left|\sum_{i \in J}\beta_i \alpha_i\right| = \left|\sum_{i \in J} \beta_i f(x_i)\right| = \left|f\left(\sum_{i \in J} \beta_i x_i\right)\right| \leq M \left\|\sum_{i \in J} \beta_i x_i\right\|
            \end{equation*}
        \item [$2 \Longrightarrow 1)$] Sea:
            \begin{equation*}
                G = \left\{\sum_{i \in J} \beta_i x_i : J\subset I \text{\ finito,\ } \beta_i \in \mathbb{R}\right\}
            \end{equation*}
            Definimos:
            \begin{equation*}
                g(x) = g\left(\sum_{i \in J}\beta_i x_i\right) = \sum_{i \in J} \beta_i \alpha_i
            \end{equation*}
            Que está bien definida, puesto que si $0 = x = \sum_{i \in J}\beta_i x_i = 0$, con lo que:
            \begin{equation*}
                \left|\sum_{i \in J}\beta_i x_i\right| \leq M \left\|\sum_{i \in J} \beta_i x_i \right\| = 0 \Longrightarrow g(0) = 0
            \end{equation*} 
            De donde si tenemos:
            \begin{equation*}
                \sum_{i \in J_1}\gamma_i x_i = x = \sum_{i \in J_2}\beta_i x_i
            \end{equation*}
            Entones:
            \begin{equation*}
                \sum_{i \in J_1\cup J_2}(\beta_i - \gamma_i)x_i = 0
            \end{equation*}
            De donde deducimos que $g$ está bien definida. Veamos que $g$ es lineal y continua:
            \begin{itemize}
                \item Si tomamos
                    \begin{equation*}
                        x = \sum_{i \in J_1} \beta_i x_i, \qquad y = \sum_{i \in J_2}\gamma_i x_i
                    \end{equation*}
                    definiendo:
                    \begin{equation*}
                        \beta_i = \left\{\begin{array}{ll}
                            \beta_i & \text{si\ } j\in J_1  \\
                             0& \text{si\ } j \in J_2\setminus J_1
                        \end{array}\right.  \qquad 
                        \gamma_i = \left\{\begin{array}{ll}
                            \gamma_i & \text{si\ } j\in J_2  \\
                             0& \text{si\ } j \in J_1\setminus J_2
                        \end{array}\right.  
                    \end{equation*}
                    tenemos:
                    \begin{equation*}
                        x = \sum_{i \in J_1\cup J_2} \beta_i x_i, \quad y = \sum_{i \in J_1\cup J_2}\gamma_i x_i \Longrightarrow x+y = \sum_{i \in J_1\cup J_2}(\beta_i + \gamma_i)x_i
                    \end{equation*}
                    Luego:
                    \begin{equation*}
                        g(x+y) = \sum_{i \in J_1\cup J_2}(\beta_i + \gamma_i)\alpha_i = \sum_{i \in J_1} \beta_i \alpha_i + \sum_{i \in J_2}\gamma_i \alpha_i = g(x) + g(y)
                    \end{equation*}
                    Si ahora $\lm \in \mathbb{R}$:
                    \begin{equation*}
                        g(\lm x) = g\left(\sum_{i \in J}\lm_i \beta_i x_i\right) = \sum_{i \in J} \lm \beta_i \alpha_i = \lm \sum_{i \in J} \beta_i \alpha_i = \lm g(x)
                    \end{equation*}
                \item Por 2 sabemos que existe $M\geq 0$ de forma que:
                    \begin{equation*}
                        |g(x)| = \left|\sum_{i \in J}\beta_i \alpha_i\right| \leq M\left\|\sum_{i \in J}\beta_i x_i\right\| = M\|x\|
                    \end{equation*}
            \end{itemize}
            Tenemos por tanto que $g\in G^\ast$ (en particular, al probar que $g$ es continua hemos probado que $\|g\| \leq M$), y por un Corolario del Teorema de Hahn-Banach, existe $f\in E^\ast$ de forma que $f\big|_G = g$ y $\|f\| = \|g\|$. Tenemos pues que:
            \begin{equation*}
                f(x_i) = g(x_i) = \alpha_i \qquad \forall  i \in I
            \end{equation*}
    \end{description}
\end{ejercicio}

\begin{ejercicio}
\end{ejercicio}

\begin{ejercicio}
\end{ejercicio}

\begin{ejercicio}
\end{ejercicio}

\begin{ejercicio}
\end{ejercicio}

% // TODO: EJERCICIO 15, no lo he hecho yo

\begin{ejercicio}
    Sea $E$ un espacio normado y $C\subset E$ convexo con $0\in C$. Consideramos:
    \begin{align*}
        C^\ast &= \{f\in E^\ast : \langle f,x \rangle \leq 1 \quad \forall x\in C \} \\
        C^{\ast\ast} &= \{x\in E : \langle f,x \rangle \leq 1  \quad \forall f\in C^\ast \}
    \end{align*}
    Se pide:
    \begin{enumerate}
        \item Probar que $C^{\ast\ast} = \overline{C}$.
            
            \begin{description}
                \item [$\subseteq )$] Para esta inclusión:
                    \begin{description}
                        \item [Opcion 1.] Podemos probarlo directamente:

                            Si $x\in \overline{C}$, existe entonces una sucesión $\{x_n\}$ de puntos de $C$ con $\{x_n\}\to x$. Sea $f\in C^\ast$, tenemos entonces que:
                            \begin{equation*}
                                \langle f,x_n \rangle \leq 1 \qquad \forall n\in \mathbb{N}
                            \end{equation*}
                            Y como $f$ es continua, tenemos que $\{\langle f,x_n \rangle \}\to \langle f,x \rangle $, por lo que ha de ser $\langle f,x \rangle\leq 1 $, de donde $x\in C^{\ast\ast}$.
                        \item [Opcion 2.] Veamos que $C\subset C^{\ast\ast}$ y que $C^{\ast\ast}$ es cerrado:
                            \begin{itemize}
                                \item Si $x\in C$ y tomamos $f\in C^\ast$ tenemos entonces que $\langle f,x \rangle \leq 1$, por lo que $x\in C^{\ast\ast}$.
                                \item Podemos ver:
                                    \begin{equation*}
                                        C^{\ast\ast} = \bigcap_{f\in C^\ast} \{x\in E : \langle f,x \rangle \leq 1\}
                                    \end{equation*}
                                    donde cada uno de dichos conjuntos es la preimagen por una función continua de un conjunto cerrado, luego $C^{\ast\ast}$ es cerrado, como intersección de conjuntos cerrados.
                            \end{itemize}
                            Por lo que $\overline{C}\subset C^{\ast\ast}$.
                    \end{description}
                \item [$\supseteq)$] Supongamos que existe $x_0\in C^{\ast\ast}\setminus \overline{C}$ , tenemos que $\{x_0\}$ es compacto y $\overline{C}$ cerrado, por lo que por la segunda versión geométrica del Teorema de Hahn-Banach, existen $f_0\in E^\ast$, $\alpha_0\in \mathbb{R}$ de manera que:
                    \begin{equation*}
                        \langle f_0,x \rangle  < \alpha_0 < \langle f_0,x_0 \rangle  \qquad \forall x\in \overline{C}
                    \end{equation*}
                    Sabemos por hipótesis que $0\in C\subset \overline{C}$, por lo que:
                    \begin{equation*}
                        0 = \langle f_0,0 \rangle  < \alpha_0
                    \end{equation*}
                    Podemos tomar $f=\frac{1}{\alpha_0}f_0$, con lo que:
                    \begin{equation*}
                        \frac{1}{\alpha_0} \langle f_0,x \rangle  < 1 < \frac{1}{\alpha_0}\langle f_0,x_0 \rangle  \quad \Longrightarrow \quad \langle f,x \rangle  < 1 < \langle f,x_0 \rangle  \qquad \forall x\in C
                    \end{equation*}
                    por lo que $f\in C^{\ast}$, pero esto es una contradicción, pues $x_0\in C^{\ast\ast}$, que venía de suponer que existe $x_0\in C^{\ast\ast}\setminus \overline{C}$.
            \end{description}
        \item ¿Qué le sucede a $C^\ast$ si $C$ es un subespacio vectorial de $E$?.

            Bajo estas condiciones, si $x\in C$, entonces $\lm x\in C$ para todo $\lm \in \mathbb{R}$, por lo que:
            \begin{equation*}
                C^\ast = \{f\in E^\ast : \langle f,\lm x \rangle \leq 1 \quad \forall x\in C \}
            \end{equation*}
            Por lo que:
            \begin{equation*}
                \langle f,\lm x \rangle  \leq 1 \stackrel{f\in E^\ast}{\Longrightarrow } \lm \langle f,x \rangle  \leq 1 \qquad \forall \lm \in \mathbb{R}
            \end{equation*}
            Si fuese $\langle f,x \rangle \neq 0 $, tenemos por tanto que:
            \begin{itemize}
                \item $\lm \leq \frac{1}{\langle f,x \rangle }$
                \item $\lm \geq \frac{1}{\langle f,x \rangle }$
            \end{itemize}
            en ambos casos se llega a contradicción, por lo que ha de ser $\langle f,x \rangle = 0 $. Para todo $x\in C$, y por definición de $C^\ast$:
            \begin{equation*}
                C^\ast = \{f\in E^\ast : \langle f,x \rangle = 0 \quad \forall x\in C \} = an(C)
            \end{equation*}
            A veces se llama a $an(C)$ por $C^\perp$, ya que en espacios de Hilbert realmente los elementos del dual son correspondientes por cada vector.
    \end{enumerate}
\end{ejercicio}


% // TODO: Ejercicios

% // TODO: Todo espacio de hilbert es estrictamente convexo
