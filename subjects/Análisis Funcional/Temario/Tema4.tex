\chapter{Espacios $L^p(\Omega)$} 
% // TODO: Para esta sección, reformularlo todo hasta el TODO TODO:
% // TODO: Leer del Brezis la parte introductoria

\noindent
Recordemos\footnote{Se vio ya en Análisis Matemático II y se recordó en la Sección~\ref{sec:Lp}.} que para $1\leq p < \infty$ definíamos para un conjunto medible $\Omega\subset \mathbb{R}^N$:
\begin{equation*}
    L^p(\Omega) = \left\{[f] \text{\ con\ } f:\Omega\to \mathbb{R} \text{\ medible tal que\ } \int_\Omega |f|^p~dx < \infty\right\}
\end{equation*}

donde las clases de equivalencia vienen dadas por:
\begin{equation*}
    f\sim g \quad \Longleftrightarrow \quad f = g \text{\ casi por doquier en\ } \Omega
\end{equation*}
De esta forma, al considerar:
\begin{equation*}
    \cc{L}^p(\Omega) = \left\{f:\Omega\to \mathbb{R} \text{\ con\ } f \text{\ medible tal que\ } \int_\Omega |f|^p~dx\right\}
\end{equation*}

tenemos que:
\begin{equation*}
    L^p(\Omega) = \frac{\cc{L}^p(\Omega)}{\sim}
\end{equation*}

\begin{notacion}
    Normalmente realizaremos un abuso de notación, denotando simplemente por $f$ a $[f]$.
\end{notacion}

\noindent
Y teníamos que $L^p(\Omega)$ era un espacio normado, con la norma:
\begin{equation*}
    \|[f]\|_p = {\left(\int_\Omega|f|^p~dx\right)}^{\nicefrac{1}{p}}
\end{equation*}
Y el Teorema de Riesz-Fischer nos dice que es de Banach, y tenía como Corolario:

\begin{coro}\label{coro:corolp}
    Si $\{f_n\}$ es convergente en $L^p(\Omega)$ hacia $f\in L^p(\Omega)$, entonces existe una parcial de forma que $\{f_{\sigma(n)}\}\to f$ casi por doquier en $\Omega$. Además, existe $h\in L^p(\Omega)$ tal que:
    \begin{equation*}
        |f_{\sigma(n)}(x)| \leq h(x) \qquad \text{casi para todo\ } x\in \Omega
    \end{equation*}
\end{coro}

Y el Teorema de Riesz-Fischer usaba en gran medida:
\begin{teo}[convergencia dominada] % // TODO: Escribirlo bien a partir de Paya
    Si $\{f_n\}$ es una sucesión de funciones medibles con $\{f_n(x)\}\to f(x)$ para casi todo $x\in \Omega$ y existe $h\in L^1(\Omega)$ de forma que:
    \begin{equation*}
        |f_n(x)| \leq h(x) \qquad \text{para casi todo\ } x\in \Omega
    \end{equation*}
    Entonces:
    \begin{equation*}
        \left\{\int_\Omega f_n~dx\right\} \to \int_\Omega f~dx
    \end{equation*}
    Si supiésemos además que $f\in L^1(\Omega)$ entonces obtenemos que:
    \begin{equation*}
        \left\{\int_\Omega |f_n-f|~dx\right\} \to 0
    \end{equation*}
\end{teo}

\begin{ejercicio}
    Si tenemos $\{f_n\}$ una sucesión de funciones medibles con $\{f_n(x)\}\to f(x)$ para casi todo $x\in \Omega$ con $f\in L^p(\Omega)$ y existe $h\in L^p(\Omega)$ tal que:
    \begin{equation*}
        |f_n(x)| \leq h(x) \qquad \text{para casi todo\ } x\in \Omega
    \end{equation*}
    Entonces:
    \begin{equation*}
        \left\{\int_\Omega |f_n-f|^p~dx\right\} \to 0
    \end{equation*}
\end{ejercicio}

\noindent
Y observamos que el Corolario~\ref{coro:corolp} es casi el recíproco de este último ejercicio.

\begin{notacion}
    Usaremos ``a.e.'' (\textit{almost everywhere}) para escribir ``c.p.d'' (casi por doquier).
\end{notacion}

\subsubsection{Si $p=\infty$}
Definimos entonces:
\begin{equation*}
    L^\infty(\Omega) = \{[f] \text{\ con\ } f:\Omega\to \mathbb{R} \text{\ medible y acotada\ }\}
\end{equation*}
que es un espacio normado, con la norma:
\begin{equation*}
    \|[f]\|_\infty = \inf \left\{k\geq 0 : |f(x)|\leq k \quad \text{para casi todo\ } x\in \Omega\right\}
\end{equation*}

\begin{ejemplo}
    Si tomamos $\Omega = \left]0,1\right[$ y tomamos $f:\Omega\to \mathbb{R}$ la función $f(x) = 1$ $\forall x\in \Omega$, tenemos que:
    \begin{align*}
        &\inf\{k\geq 0 : |f(x)|\leq k \quad \text{para casi todo\ } x\in \Omega\} \AstIg \\
        &\inf\{k\geq 0 : |f(x)| \leq k \quad \forall x\in \left]0,1\right[\} = \\
        &\sup_{x\in \left]0,1\right[}|f(x)| = \|f\|_\infty
    \end{align*}
    donde en $(\ast)$ usamos que $f$ es continua. Tenemos entonces que $\|f\|_\infty = 1$. Si consideramos ahora:
    \begin{equation*}
        f(x) = \left\{\begin{array}{ll}
            1 & \text{si\ } x\in \Omega\setminus \{\nicefrac{1}{2}\} \\
             2 & \text{si\ } x = \nicefrac{1}{2}
        \end{array}\right. 
    \end{equation*}
    Tenemos entonces que:
    \begin{gather*}
        \inf \{k\geq 0 : |f(x)|=1 \leq k \quad \text{casi para todo\ } x\in \Omega\} = 1 \\
        \inf \{k\geq 0 : |f(x)|\leq k \quad \forall x\in \left]0,1\right[\} = 2
    \end{gather*}
    Al primero se le suele denotar normalmente por:
    \begin{equation*}
        \text{esssup}_{x\in \left]0,1\right[}|f(x)|
    \end{equation*}
\end{ejemplo}

\noindent
Se verifica que $(L^\infty(\Omega),\|\cdot \|_\infty)$ es un espacio de Banach.\\

\noindent
Todos estos espacios verifican: % // TODO: Tener en cuenta que no es discreto
\begin{equation*}
    L^1(\Omega) \supset \ldots \supset L^p(\Omega) \supset \ldots \supset L^\infty(\Omega)
\end{equation*}
La desigualdad de Hölder nos da para $p_2<p_1$:
\begin{equation*}
    \|f\|_{p_2}^{p_2} = \int_\Omega |f|^{p_2}~dx \leq {\left(\int_\Omega {(|f|^{p_2})}^{\nicefrac{p_1}{p_2}}~dx\right)}^{\nicefrac{p_2}{p_1}} {(|\Omega|)}^{1-\nicefrac{p_2}{p_1}} = \|f\|_{p_1}^{p_2}|\Omega|^{1-\nicefrac{p_2}{p_1}}
\end{equation*}
de donde:
\begin{equation*}
    \|f\|_{p_2} \leq \|f\|_{p_1} |\Omega|^{\nicefrac{1}{p_2}-\nicefrac{1}{p_1}}
\end{equation*}
En particular tenemos la inclusión de forma continua, por esta última igualdad. Finalmente, es trivial que si $f\in L^\infty(\Omega)$ entonces $|f|^p\in L^\infty(\Omega)$, por lo que esta última función será integrable, de donde $f\in L^p(\Omega)$. De esta forma:
\begin{equation*}
    \|f\|_p \leq \|f\|_\infty \cdot |\Omega|^{\nicefrac{1}{p}} \qquad \forall f\in L^\infty(\Omega)
\end{equation*}
Por lo que la aplicación
\Func{}{L^\infty(\Omega)}{L^p(\Omega)}{f}{f}
es continua.\\

\noindent
A partir de estas inclusiones:
\begin{equation*}
    L^\infty(\Omega) \subset \bigcap_{1\leq p < \infty}L^p(\Omega)
\end{equation*}
\begin{ejercicio}
    Pensar si se da la igualdad. En el caso que se dé, consideramos $\|\cdot \|_p$ con $p\to \infty$ y nos preguntamos si es igual a $\|\cdot \|_\infty$. De la desigualdad:
    \begin{equation*}
        \|f\|_p \leq \|f\|_\infty \cdot {|\Omega|}^{\nicefrac{1}{p}}
    \end{equation*}
    se deduce que $\|f\|$ limite $\leq \|f\|$
\end{ejercicio}


% // TODO: No se que es esto, derivada debil
\begin{ejercicio}
    Si tenemos $u\in C^1(a,b)$ y $v\in C^0(a,b)$ de forma que:
    \begin{equation*}
        \int_{a}^{b} u\varphi '~dx  = -\int_{a}^{b} v\varphi~dx  \qquad \forall \varphi \in C^1_c(a,b)
    \end{equation*}
    (donde $C_c^1$ son las de clase 1 y soporte compacto). Entonces:
    \begin{equation*}
        \int_{a}^{b} [u'(x) -v(x)]\varphi(x)~dx  = 0 \qquad \forall \varphi \in C^1_c(a,b)
    \end{equation*}
    Si en cierto punto $x_0$ tenemos $u'(x_0)-v(x_0)\neq 0$ entonces tenemos un intervalo donde es distinto de cero, y podemos encontrar $\varphi$ conveniente para que la integral no salga cero, llegando a contradicción. De aquí, $u' = v$.

    \noindent
    Definiremos la derivada débil de $u$ (integrable en el soporte de $\varphi$) como la $v$ que cumple:
    \begin{equation*}
        \int_{a}^{b} u\varphi '~dx  = -\int_{a}^{b} v\varphi~dx 
    \end{equation*}
    Por tanto, solo hace falta que $u$ sea integrable en compactos contenidos en $a,b$.
\end{ejercicio}

\begin{definicion}
    \begin{equation*}
        L^1_{\text{loc}}(\Omega) = \{u:\Omega\to \mathbb{R} \text{\ medibles con\ } u\big|_K \text{\ integrables\ } \forall K\subset \Omega \text{\ compacto}\}
    \end{equation*}
\end{definicion}

\noindent
Tenemos entonces que $u$ es derivable si y solo si existe $v\in L^1_{\text{loc}}(\Omega)$ tal que:
\begin{equation*}
    \int_\Omega u\varphi ' ~dx = -\int_\Omega v\varphi~dx \qquad \forall \varphi \in C^1_c(\Omega)
\end{equation*}
\begin{ejercicio}
    Con esta definición tenemos que $|\cdot |:\left]-1,1\right[\to \left]-1,1\right[$ es derivable.\\

    \noindent
    Sean $f\in L^1_{\text{loc}}(a,b)$ y $g\in L^1_{\text{loc}}(a,b)$ $g$ es derivada de $f$ si:
    \begin{equation*}
        \int_{a}^{b} f(x)\varphi'(x)~dx  = -\int_{a}^{b} g(x)\varphi(x)~dx  \qquad \forall \varphi \in C^1_c(a,b)
    \end{equation*}
    Si tomamos $f(x) = |x|$ para $x\in \left]-1,1\right[$, y $g(x) = sgn(x)$ para $x\in \left]-1,1\right[$ tenemos:
    \begin{align*}
        \int_{-1}^{1} |x|\varphi'(x)~dx  &= \int_{-1}^{0} -x\varphi'(x)~dx  + \int_{0}^{1} x\varphi'(x)~dx  \\
                                         &\AstIg \int_{-1}^{0} \varphi(x)~dx  - \int_{0}^{1} \varphi(x)~dx  = \int_{-1}^{1} sgn(x)\varphi(x)~dx 
    \end{align*}
    donde en $(\ast)$ hemos hecho integración por partes. 
\end{ejercicio}
Sustituimos de esta forma $C^1(\Omega)$ por $L^1_{\text{loc}}(\Omega)$, pero no por $L^1(\Omega)$.

\noindent
Para ver que una función está en $L^1_{\text{loc}}$ basta ver que tiene integral finita en un compacto pequeño en cada punto, algo que no es verdad en $L^1(\Omega)$.

% // TODO TODO:
\section{Dual de $L^p(\Omega)$ para $1<p<\infty$}
\noindent
Sea $1<p<\infty$, consideramos la aplicación
\Func{T}{L^{p'}(\Omega)}{L^p(\Omega)^\ast}{v}{Tv}
donde $Tv:L^p(\Omega)\to \mathbb{R}$ viene dada por:
\begin{equation*}
    \langle Tv,w \rangle  := \int_\Omega vw
\end{equation*}
Fijado $v\in L^{p'}(\Omega)$ tenemos que $Tv$ está bien definida, ya que para cada $w\in L^p(\Omega)$ la desigualdad de Hölder nos dice que:
\begin{equation*}
    \int_{\Omega}|vw| \leq {\left(\int_\Omega |v|^{p'}\right)}^{\nicefrac{1}{p'}} {\left(\int_\Omega |w|^p\right)}^{\nicefrac{1}{p}}
\end{equation*}
de donde la función $|vw|$ medible y positiva tiene integral finita, luego es integrable, lo que equivale a que $vw$ sea una función integrable. De aquí tenemos que $Tv$ está bien definida. De la desigualdad deducimos también que:
\begin{equation*}
    |\langle Tv,w \rangle |\leq \|v\|_{p'} \|w\|_p \qquad \forall w\in L^p(\Omega), \quad \forall v\in L^{p'}(\Omega)
\end{equation*}
Fijado $v\in L^{p'}(\Omega)$, como $Tv$ es lineal esta última desigualdad nos dice ahora que $Tv\in L^p(\Omega)^\ast$. Esto último nos dice que $T$ está bien definida. Es claro también que $T$ es lineal. Si para cada $v\in L^{p'}(\Omega)$ calculamos $\|Tv\|$, tenemos por un lado que $\|Tv\| \leq \|v\|_{p'}$ y por otro que si tomamos\footnote{Para el caso $2<p<\infty$ tenemos $1<p'<2$, por lo que si $v(x) = 0$ tendríamos una división por cero, que se evita con una función a trozos.}:
\begin{equation*}
    w(x) = \left\{\begin{array}{ll}
        |v(x)|^{p'-2}\cdot v(x) & \text{si\ } v(x)\neq 0 \\
         0& \text{si\ } v(x) = 0
    \end{array}\right. 
\end{equation*}
Tenemos que $w\in L^{p}(\Omega)$, ya que (tomando $\Omega' = \{x\in \Omega : v(x)\neq 0\}$):
\begin{equation*}
    \|w\|_p = {\left[\int_{\Omega'} {\left(|v(x)|^{p'-1}\right)}^{p}\right]}^{\nicefrac{1}{p}} = {\left(\int_{\Omega'}|v|^{p'}\right)}^{\nicefrac{1}{p}} = {\left[\int_{\Omega} {\left(|v|^{p'}\right)}^{\nicefrac{1}{p'}}\right]}^{\nicefrac{p'}{p}} = \|v\|_{p'}^{p'-1}
\end{equation*}
Y por otra parte si aplicamos $Tv$ a $w$ obtenemos:
\begin{equation*}
    \langle Tv,w \rangle  = \int_\Omega vw  = \int_{\Omega'} |v|^{p'} = \int_{\Omega}|v|^{p'} = \|v\|_{p'}^{p'} = \|v\|_{p'}^{p'-1}\|v\|_{p'} = \|w\|_{p}\|v\|_{p'}
\end{equation*}
de donde ha de ser $\|Tv\| = \|v\|_{p'}$, para cada $v\in L^{p'}(\Omega)$, luego $T$ es una isometría, y en particular es continua e inyectiva. Falta por tanto probar que $T$ es sobreyectiva para ver que $L^p(\Omega)$ y $L^{p'}(\Omega)$ son isométricos.\\

\noindent
Antes de ello podemos decir ya varias cosas, como por ejemplo: 
\begin{itemize}
    \item Como $L^{p'}(\Omega)$ es completo y $T$ es una isometría tenemos por tanto que $T(L^{p'}(\Omega))$ es un subespacio completo de $L^p(\Omega)^\ast$. 
    \item Además, como en un espacio commpleto tenemos que un subespacio es completo si y solo si es cerrado, deducimos que $T(L^{p'}(\Omega))$ es un subespacio cerrado de $L^p(\Omega)^\ast$.
    \item Observamos también que como $(p')'=p$, tenemos por el mismo argumento que hemos visto que hay una isometría
        \begin{equation*}
            L^p(\Omega) \to L^{p'}(\Omega)^\ast
        \end{equation*}
\end{itemize}
Para ver la sobreyectividad de $T$ necesitaremos ver previamente ciertos resultados, a partir de las hipótesis en las que nos encontramos.

\begin{teo}
    $L^p(\Omega)$ es reflexivo para $2\leq p < \infty$.
    \begin{proof}
        Fijado $p\in \left[2,+\infty\right[$, se verifica que:
        \begin{equation*}
            \left|\frac{a+b}{2}\right|^p + \left|\frac{a-b}{2}\right|^p \leq \frac{1}{2}(|a|^p+|b|^p) \qquad \forall a,b\in \mathbb{R}
        \end{equation*}
            Sin lugar a dudas, resolver tal trivialidad sería un insulto a la autonomía e inteligencia de nuestro estimado lector, por lo que se estima conveniente que dicho individuo sea quien lo realice por sí mismo. Tenemos por tanto que:
            \begin{equation*}
                \int_{\Omega}\left|\frac{w(x)+\overline{w}(x)}{2}\right|^p \int_{\Omega} \left|\frac{w(x)-\overline{w}(x)}{2}\right|^p \leq \int_{\Omega}\frac{1}{2}(|w(x)|^p + |\overline{w}(x)|^p)
            \end{equation*}

            o equivalentemente:
            \begin{equation*}
                \left\|\frac{w+\overline{w}}{2}\right\|_p^p + \left\|\frac{w-\overline{w}}{2}\right\|_p^p  \leq \frac{1}{2} (\|w\|_p^p + \|\overline{w}\|_p^p) \qquad \forall w,\overline{w}\in L^p(\Omega)
            \end{equation*}
            desigualdad que recibe el nombre de ``Primera\footnote{La segunda identidad es para el caso $p<2$, que es ligeramente más complicada.} Identidad de Clarkson''. Como consecuencia, tenemos que para $\varepsilon \in \left]0,1\right[$ dado y $w,\overline{w}\in L^p(\Omega)$ con $\|w\|_p=1=\|\overline{w}\|_p$ y $\|w-\overline{w}\|_p>\varepsilon$ se tiene que:
            \begin{equation*}
                \left\|\frac{w+\overline{w}}{2}\right\|_p^p \leq \frac{1}{2}(1+1) - \left\|\frac{w-\overline{w}}{2}\right\|_p^p = 1 - \frac{\|w-\overline{w}\|_p^p}{2^p} < 1 - \frac{\varepsilon^p}{2^p}
            \end{equation*}

            de donde:
            \begin{equation*}
                \left\|\frac{w+\overline{w}}{2}\right\|_p \leq {\left(1-\frac{\varepsilon^p}{2^p}\right)}^{\nicefrac{1}{p}} < 1
            \end{equation*}
            Por lo que si tomamos:
            \begin{equation*}
                \delta = 1-{\left(1-\frac{\varepsilon^p}{2^p}\right)}^{\nicefrac{1}{p}}
            \end{equation*}

            tenemos que $0<\delta<1$, de donde:
            \begin{equation*}
                \left\|\frac{w+\overline{w}}{2}\right\|_p < 1-\delta
            \end{equation*}
            Por tanto, tenemos que $L^p(\Omega)$ es estrictamente convexo, luego es reflexivo por el Teorema de Milman-Pettis.
    \end{proof}
\end{teo}

\begin{coro}
    $L^p(\Omega)$ también es reflexivo para $1<p<2$.
    \begin{proof}
        Fijado $p\in \left]1,2\right[$, sabemos ya que existe una isometría 
        \begin{equation*}
            T:L^p(\Omega) \to L^{p'}(\Omega)^\ast
        \end{equation*}
        Así como que $T(L^p(\Omega))$ es un subespacio cerrado de $L^{p'}(\Omega)^\ast$. 

        \noindent
        Tendremos que $p'\in \left]2,+\infty\right[$, por lo que por el Teorema anterior ha de ser $L^{p'}(\Omega)$ reflexivo, y como el dual de un espacio reflexivo es reflexivo, tenemos que $L^{p'}(\Omega)^{\ast}$ es también reflexivo. Como $T(L^p(\Omega))$ es un subesapcio cerrado de un espacio reflexivo ha de ser también reflexivo. Como $T$ es una isometría obtenemos que $L^p(\Omega)$ es también reflexivo.
    \end{proof}
\end{coro}

\noindent
Usando la Segunda Identidad de Clarkson (es ligeramente más enrevesada) se puede probar que $L^p(\Omega)$ es uniformemente convexo para $1<p<2$, de donde también son reflexivos:
\begin{equation*}
    \left\|\frac{w+\overline{w}}{2}\right\|_p^{p'} + \left\|\frac{w-\overline{w}}{2}\right\|^{p'}_p \leq {\left[\frac{1}{2}(\|w\|_p^p + \|\overline{w}\|_p^p)\right]}^{\nicefrac{1}{p-1}}
\end{equation*}

\noindent
Tratamos de probar ahora que $V:=T(L^{p'}(\Omega))$ es denso en $L^p(\Omega)^\ast$, y como es cerrado tendremos por tanto que $T(L^{p'}(\Omega)) = L^p(\Omega)^\ast$, lo que nos dará la sobreyectividad de $T$.

\begin{prop}
    $V:=T(L^{p'}(\Omega))$ es denso en $L^p(\Omega)^\ast$ para $1<p<\infty$.
    \begin{proof}
        Supongamos que $\varphi \in L^p(\Omega)^{\ast\ast}$ con:
        \begin{equation*}
            \langle \varphi,Tv \rangle  = 0 \qquad \forall v\in L^{p'}(\Omega)
        \end{equation*}
        Si conseguimos probar que entonces $\varphi = 0$ tendremos por un Corolario del Teorema de Hahn-Banach que $V$ es denso en $L^p(\Omega)^\ast$.\\

        \noindent
        Hemos visto ya que $L^p(\Omega)$ es reflexivo, por lo que $J(L^p(\Omega)) = L^p(\Omega)^{\ast\ast}$. De aquí deducimos que existe $w\in L^p(\Omega)$ de forma que $\varphi = J(w)$, por lo que la hipótesis se traduce en:
        \begin{equation*}
            0 = \langle J(w),Tv \rangle  = \langle Tv,w \rangle  = \int_{\Omega}vw \qquad \forall v\in L^{p'}(\Omega)
        \end{equation*}
        Tomando ahora:
        \begin{equation*}
            v(x) = \left\{\begin{array}{ll}
                |w(x)|^{p-2}w(x) & \text{si\ } w(x)\neq 0 \\
                 0& \text{si\ } w(x) = 0
            \end{array}\right. 
        \end{equation*}
        Se deduce que $w=0$, por lo que ha de ser $\varphi = J(0) = 0$, y podemos aplicar el Corolario~\ref{coro:comprobar_denso}.
    \end{proof}
\end{prop}

\noindent
En definitiva, tenemos que $V=T(L^{p'}(\Omega))\subset L^p(\Omega)^\ast$ es cerrado y denso, por lo que ha de ser $T(L^{p'}(\Omega)) = L^p(\Omega)^\ast$, luego la aplicación $T$ que definíamos al principio de esta sección es una isometría sobreyectiva, por lo que $L^{p'}(\Omega)$ es isométrico a $L^p(\Omega)^\ast$.\\

\noindent
Esta prueba recibe el nombre \textbf{Teorema de Representación de Riesz} para $L^p(\Omega)$.

\section{Dual de $L^1(\Omega)$}

\begin{observacion}
    Observemos que si $u\in L^\infty(\Omega)$ entonces la aplicación
    \Func{}{L^1(\Omega)}{\bb{R}}{f}{\int_\Omega fu}

    es lineal y continua:
    \begin{equation*}
        \left|\int_\Omega fu\right| \leq \left(\int_\Omega |f|\right) \|u\|_\infty = \|f\|_1 \|u\|_\infty
    \end{equation*}
    Por lo que la aplicación definida está en $L^1(\Omega)^\ast$
\end{observacion}

\begin{teo}[Representación de Riesz para $L^1(\Omega)$]
    \begin{equation*}
        \forall \phi \in L^1(\Omega)^\ast ~\exists_1 u\in L^\infty(\Omega) \quad \text{con}\quad  \langle \phi,f \rangle  = \int_{\Omega} fu \qquad \forall f\in L^1(\Omega)
    \end{equation*}
    Además veremos que $\|\phi\| = \|u\|_\infty$
    \begin{proof}
        Para ello, consideramos los conjuntos $\{\Omega_n\}$ de forma que:
        \begin{equation*}
            \Omega = \bigcup_{n\in \mathbb{N}} \Omega_n
        \end{equation*}
        con $\Omega_n \subset \Omega_{n+1}$ y $0<|\Omega_n|<\infty$ para cada $n\in \mathbb{N}$. Exigiremos también que $0<|\Omega_n\setminus \Omega_{n-1}|$\\

        \begin{description}
            \item [Unicidad.] Para ver la unicidad de $u$, si tenemos $u_1,u_2$ cumpliendo la condición y tomamos $v = u_1-u_2$ tenemos entonces que:
                \begin{equation*}
                    \int_\Omega fv = 0 \qquad \forall f\in L^1(\Omega)
                \end{equation*}
                Si tomamos $f = \chi_{\Omega_n}sgn v$ donde:
                \begin{equation*}
                    sgn v(x) = \left\{\begin{array}{ll}
                        \frac{v(x)}{|v(x)|} & \text{si\ } v(x)\neq 0 \\
                        0 & \text{si\ } v(x) = 0
                    \end{array}\right. 
                \end{equation*}
                Es claro que $f \in L^1(\Omega)$, de donde tendremos que:
                \begin{equation*}
                    0 = \int_\Omega \chi_{\Omega_n}(sgn v) v = \int_{\Omega_n} (sgn v) v = \int_{\Omega_n} |v|
                \end{equation*}
                de donde $v=0$ casi por doquier en $\Omega_n$ para cada $n\in \mathbb{N}$; por lo que $v=0$  casi por doquier en $\Omega$, de donde $v=0$ en $L^1(\Omega)$.\\
            \item [Existencia.] Tomamos $\{\alpha_n\}$ una sucesión de números reales positivos y definimos la función $\theta:\Omega\to \mathbb{R}$ dada por:
                \begin{equation*}
                    \theta(x) = \left\{\begin{array}{ll}
                        \alpha_1 & \text{si\ } x\in \Omega_1 \\
                        \alpha_2 & \text{si\ } x\in \Omega_2\setminus \Omega_1 \\
                                 &\vdots \\
                        \alpha_n & \text{si\ } x\in \Omega_n\setminus \Omega_{n-1} \\
                                 &\vdots
                    \end{array}\right. 
                \end{equation*}
                Y $\theta$ es el límite de una sucesión de funciones simples positivas, por lo que $\theta$ es medible. Calculamos ahora (tomando $\Omega_0 = \emptyset $):
                \begin{equation*}
                    \int_\Omega \theta(x)^2~dx = \sum_{n=1}^{\infty} \alpha_n^2 |\Omega_n\setminus\Omega_{n-1}|
                \end{equation*}
                Que puede ser o no convergente, en función de la sucesión $\{\alpha_n\}$. Tomaremos pues la sucesión $\{\alpha_n\}$ que verifique:
                \begin{equation*}
                    \alpha_n^2 |\Omega_n\setminus \Omega_{n-1}| = \frac{1}{n^2} \quad\Longleftrightarrow\quad \alpha_n = \frac{1}{n\sqrt{|\Omega_n\setminus \Omega_{n-1}|}} \qquad \forall n \in \mathbb{N}
                \end{equation*}
                De esta forma tenemos entonces que $\int_\Omega\theta(x)^2~dx < \infty$, por lo que $\theta \in L^2(\Omega)$. Consideramos ahora la aplicación
                \Func{\phi_\theta}{L^2(\Omega)}{\bb{R}}{f}{\langle\phi, \theta f\rangle}
                Que está bien definida ($\theta f\in L^1(\Omega)$ para toda $f\in L^2(\Omega)$) ya que si tomamos $f\in L^2(\Omega)$ tenemos por la desigualdad de Cauchy-Schwartz (o la de Hölder para $p=2$) que $\theta f\in L^1(\Omega)$. Es claro además que la aplicación es lineal, así como que:
                \begin{equation*}
                    |\langle \phi,\theta f \rangle | \leq \|\phi\| \|\theta f\|_1 \leq \|\phi\|\|\theta\|_2\|f\|_2
                \end{equation*}
                Por lo que la aplicación es también continua, con lo que es un elemento de $L^2(\Omega)^\ast$. El Teorema de Representación de Riesz para $L^2(\Omega)$ nos dice que existe un único elemento $v\in L^2(\Omega)$ tal que $\langle \phi_\theta,f \rangle = \langle \phi,\theta f \rangle  = \int_\Omega fv$ para toda $f\in L^2(\Omega)$.

                \noindent
                Consideramos ahora $u = \frac{v}{\theta}$, que es una función medible como cociente de dos funciones medibles. Vemos ahora que para cada $n\in \mathbb{N}$ tenemos:
                \begin{equation*}
                    u\chi_{\Omega_n} \in L^2(\Omega)
                \end{equation*}
                ya que:
                \begin{equation*}
                    \int_\Omega{(u\chi_{\Omega_n})}^{2} = \int_{\Omega_n} u^2 = \int_{\Omega_n} \frac{v^2}{\theta^2} \leq \int_{\Omega} \frac{v^2}{\min \{\alpha^2_1,\alpha^2_2, \ldots, \alpha^2_n\}} = \frac{\int_\Omega v^2 }{\min\{\alpha^2_1,\alpha^2_2,\ldots,\alpha^2_n\}}  < \infty
                \end{equation*}
                Sea $g\in L^\infty(\Omega)$, consideramos $f = \chi_{\Omega_n}\cdot \frac{g}{\theta}$. Tenemos que $g$ está acotada y que $\nicefrac{1}{\theta}$ está acotado por $\nicefrac{1}{\min\{\alpha_1,\ldots,\alpha_n\}}$, de donde $f\in L^\infty(\Omega)$, por estar acotada. Observamos además que $f\big|_{\Omega\setminus \Omega_n} = 0$, por lo que:
                \begin{equation*}
                    \int_\Omega f^2 \leq \int_{\Omega_n} f^2 \leq \|f\|_\infty \cdot |\Omega_n| < \infty
                \end{equation*}
                de donde $f\in L^2(\Omega)$. Además, tenemos que: 
                \begin{equation}\label{eq:representacion_riesz_1}
                    \langle \phi,\chi_{\Omega_n} g\rangle = \langle \phi, \theta f \rangle  = \int_\Omega fv = \int_\Omega \chi_{\Omega_n} \frac{g}{\theta} v = \int_\Omega \chi_{\Omega_n} gu   \qquad \forall g\in L^\infty(\Omega)
                \end{equation}
                Veamos finalmente que $\|u\|_\infty \leq \|\phi\|$ para concluir que $u\in L^\infty(\Omega)$. Tomamos $C\in \mathbb{R}^+_0$ de forma que $C\geq \|\phi\|$. Consideramos:
                \begin{equation*}
                    A = \{x\in \Omega:  |u(x)| > C\}
                \end{equation*}
                Veamos que $|A| = 0$, con lo que tendremos que $|u(x)|\leq C$ casi por doquier en $\Omega$, por lo que el supremo esencial de $u$, o $\|u\|_\infty$ será menor o igual que $C$ para todo $C>\|\phi\|$, de donde tendiendo $C\to \|\phi\|$ queda que $\|u\|_\infty \leq \|\phi\|$.\\

                \noindent
                Para ver que $|A| = 0$, veamos que $|A\cap \Omega_n| = 0$ para cada $n\in \mathbb{N}$, ya que:
                \begin{equation*}
                    \bigcup_{n\in \mathbb{N}} A\cap \Omega_n = A\cap \bigcup_{n\in \mathbb{N}}\Omega_n  = A
                \end{equation*}
                En vista de~\eqref{eq:representacion_riesz_1}, tomamos $g=\chi_A sgn u$, que es medible y está en (es acotada) $L^1(\Omega)\subset L^\infty(\Omega)$, obteniendo que:
                \begin{equation*}
                    \int_{\Omega_n\cap A}|u| = \int_{\Omega_n} \chi_A|u| = \langle \phi,\chi_A\chi_{\Omega_n}sgn u \rangle  = \langle \phi,\chi_{A\cap \Omega_n} u \rangle 
                \end{equation*}
                Si vemos ahora que:
                \begin{equation*}
                    C\cdot |\Omega_n\cap A| \leq \int_{\Omega_n\cap A} |u|
                \end{equation*}
        \end{description}
    \end{proof}
\end{teo}

\section{Teorema espectral}
\subsection{Operadores compactos}
% // TODO: Repasar compacidad en cosas distintas a R^n
\noindent
Sean $E$ y $F$ espacios de Banach y $T:E\to F$, decimos que $T$ es compacto si $T(A)$ es relativamente compacto (es decir, que $\overline{T(A)}$ sea compacto) para todo conjunto $A\subset E$ acotado.
% // TODO: POner esto en definicion

\begin{ejercicio} % // TODO: HACER
    Algunas propiedades de los operadores compactos:
    \begin{enumerate}
        \item $T$ es compacto si y solo si para toda sucesión de puntos de $E$ existe una parcial suya $\{x_{\sigma(n)}\}$ tal que $\{T(x_{\sigma(n)})\}$ es convergente en $F$.
        \item Si $E$ es reflexivo, tenemos que $T$ es compacto si y solo sí para toda sucesión de puntos de $E$ con $\{x_n\}\rightharpoonup x\in E$ es posible encontrar una parcial $\{x_{\sigma(n)}\}$ de forma que $\{T(x_{\sigma(n)})\}\to Tx$.
    \end{enumerate}
\end{ejercicio}

\begin{prop}
    Sean $E$, $F$ espacios de Banach y $T:E\to F$ una aplicación lineal:
    \begin{center}
        $T$ es compacto $\Longleftrightarrow \overline{T\left(\overline{B}_E\right)}$ es compacto en $F\Longrightarrow \{x_n\}\rightharpoonup x \in E \Longrightarrow \{T(x_n)\}\to Tx$ .
    \end{center}
\end{prop}

\begin{observacion}
    Vemos por la propiedad 3 de la Prosposición anterior que si $T$ es lineal y compacta tenemos que $T$ es continua.
\end{observacion}

\begin{ejercicio} % // TODO: HACER
    Más ejercicios:
    \begin{enumerate}
        \item  $Id:E\to E$ es compacta si y solo si $dim E < \infty$.
        \item Si $T:E\to F$ es lineal, acotada y de rango finito (que $dim T(E) < \infty$) entonces $T$ es compacto.
    \end{enumerate}
\end{ejercicio}

\begin{definicion}[Aplicación simétrica]
    Sea $H$ un espacio de Hilbert y $T:H\to H$ una aplicación lineal y continua, decimos que es simétrica si:
    \begin{equation*}
        \langle Tx,y \rangle  = \langle x,Ty \rangle 
    \end{equation*}
    A veces se dice que $T$ es un \underline{operador autoadjunto}.
\end{definicion}~\\

\subsection{Teorema espectral}
\noindent
El objetivo del Teorema siguiente es calcular todos aquellos valores $\lm \in \mathbb{R}$ para los que $\exists u\in H\setminus\{0\}$ con $Tu = \lm u$.

\noindent
Esto ya lo vimos en Geometría II para el caso de $\mathbb{R}^n$, pues si consideramos $T(x) = Ax$ para $A$ una matriz, buscamos valores propios y para estos teníamos que tener que $A$ era una matriz simétrica, que $T$ es lineal y que $T$ es compacta, por ser continua.\\

\noindent
\begin{observacion}
    Observemos que:
    \begin{center}
        $0$ es valor propio de $T\Longleftrightarrow \ker T \neq \{0\}$.
    \end{center}
    Por lo que no nos preocupa este valor.
\end{observacion}

\begin{prop}
    Si $H$ es un espacio de Hilbert y $T:H\to H$ es una aplicación lineal:
    \begin{enumerate}
        \item Si $T$ es además continua y $\lm$ es un valor propio de $T$ entonces:
            \begin{equation*}
                -\|T\| \leq \lm \leq \|T\|
            \end{equation*}
        \item Si $T$ es además simétrico, funciones\footnote{Vectores en un espacio de funciones.} propias asociadas a valores propios distintos son ortogonales.
        \item Si $T$ es además compacto, $dim \ker(T-\lm I) < \infty$ para todo $\lm\in \mathbb{R}\setminus\{0\}$ valor propio de $T$. 

            El valor de esta dimensión recibirá el nombre ``multiplicidad algebraica de $\lm$''.
        \item Si $T$ es compacto y simétrico, entonces $0$ es el único posible\footnote{No necesariamente lo es.} punto de acumulación de valores propios no nulos.
    \end{enumerate}
    \begin{proof}
        Demostramos cada apartado:
        \begin{enumerate}
            \item Si $\lm$ es un valor propio de $T$, existe $u\in H\setminus\{0\}$ de forma que $T(u) = \lm u$, y podemos suponer sin pérdida de generalidad que $\|u\| = 1$, por lo que:
                \begin{equation*}
                    |\lm| \|u\|^2 = |\langle \lm u,u \rangle | = |\langle Tu, u \rangle | \leq \|Tu\|\|u\|
                \end{equation*}
                de donde:
                \begin{equation*}
                    |\lm| \leq \|Tu\| \leq \|T\|\|u\| = \| T\|
                \end{equation*}
            \item Si tenemos que $Tu = \lm u$ y que $Tv = \beta u$ con $\lm\neq \beta$ y $u,v\neq 0$. Tenemos que:
                \begin{equation*}
                    \langle u,\beta v \rangle = \langle u,Tv \rangle =  \langle Tu, v \rangle  = \langle \lm u,v \rangle 
                \end{equation*}
                de donde $\lm \langle u,v \rangle = \beta\langle u,v \rangle  $ con $\lm\neq \beta$, por lo que ha de ser $\langle u,v \rangle  = 0$.
            \item Por reducción al absurdo, si tomamos $V=\ker (T-\lm I)$ con $\lm\neq 0$, supondremos que $\dim V = \infty$. Es claro que $V$ es un subespacio de $H$ que además es cerrado ($V = {(T-\lm I)}^{-1}(\{0\})$), por lo que $V$ también es un espacio de Hilbert, y todos ellos son reflexivos. 

                Somos capaces de encontrar (pensar en la bola unidad, que no es compacta\footnote{O algo de eso dijo.}) $\{u_k\}$ en $V$ con $\|u_k\| = 1$, $\{u_k\}\rightharpoonup u$ y $\{u_k\}\to u$. Como $T$ es compacta, tenemos por la Proposición anterior que $\{T(u_k)\}\to Tu$, pero por otra parte teneos que $T(u_k) = \lm u_k$, por ser $\{u_k\}\subset V$. En dicho caso:
                \begin{equation*}
                    \{\lm u_k\} = \{T(u_k)\} \to Tu
                \end{equation*}
                Como $\lm\neq 0$, tenemos entonces que $\{u_k\}\to \nicefrac{Tu}{\lm}$, por lo que también tenemos $\{u_k\}\rightharpoonup \nicefrac{Tu}{\lm}$, de donde $u = \nicefrac{Tu}{\lm}$, luego tendríamos:
                \begin{equation*}
                    \{ u_k\} \to u
                \end{equation*}
                y esto es una contradicción.
            \item % // TODO: HACER, suponer que hay un valor propio no nulo punto de acumulacion de valores propios no nulos. Para cada valor propio coger una funcion propia (hay infinitos, puesto que tenemos infinitos valores). Las funciones son ortogonales entre si => el espacio que generan tiene dimension infinita => podemos sacar una sucesión mala como la de la prueba anterior, contradiccion.
        \end{enumerate}
    \end{proof}
\end{prop}

\begin{ejercicio} % // TODO: HACER
    Sea $H$ un espacio de Hilbert, $T:H\to H$ lineal y simétrica. Sea $V\subset H$ un subespacio cerrado de $H$ con $T(V)\subset V$, tenemos entonces que $T:V^\perp\to V^\perp$ es lineal y simétrica (como restricción de una aplicación lineal y simétrica).\newline
    Si además $T$ fuera compacto, entonces $T:V^\perp\to V^\perp$ también es compacto.
\end{ejercicio}

\begin{teo}[espectral]
    Sea $H$ un espacio de Hilbert y $T:H\to H$ una aplicación lineal, compacta y simétrica
\end{teo}

\begin{lema}
    Sea $H$ un espacio de Hilbert, $T:H\to H$ una aplicación lineal, compacta y simétrica. Sea:
    \begin{equation*}
        \lm_1 = \sup\{\langle Tu,u \rangle : \|u\| = 1\}
    \end{equation*}
    Observemos que $\lm_1$ es valor propio si existe $u_1$ de módulo 1 con $T(u_1) = \lm_1u_1$.
    \begin{enumerate}
        \item Si $dim H = \infty\Longrightarrow \lm_1\geq 0$.
        \item Si $\lm$ es un valor propio de $T$ entonces $\lm\leq \lm_1$.
        \item Si $\lm_1>0$ entonces $\lm_1$ es un valor propio (el más grande) de $T$.
        \item $T$ tiene un valor propio positivo $\Longleftrightarrow \lm_1>0$.
    \end{enumerate}
    \begin{proof}
        Veamos cada apartado:
        \begin{enumerate}
            \item Si $dim H = \infty$, podemos tomar una sucesión $\{u_k\}$ de elementos de $H$ de forma que $\|u_k\| = 1$ y que $\{u_k\}\rightharpoonup 0$ (observemos que no puede ser $\{u_k\}\to 0$). Como $T$ es lineal y compacto tenemos que $\{T(u_k)\}\to T(0) = 0$, por lo que:
                \begin{equation*}
                    \{\langle T(u_k),u_k \rangle \} \to \langle 0,0 \rangle  = 0
                \end{equation*}
                Como el conjunto $\{\langle Tu,u \rangle : \|u\| = 1 \}$ contiene una sucesión de elementos convergente a $0$, tiene que ser $\lm_1\geq 0$.
            \item Si $\lm$ es un valor propio de $T$ ha de existir $u\in H$ con $\|u\| = 1$ y $T(u) = \lm u$, por lo que:
                \begin{equation*}
                    \lm_1 \geq \langle Tu,u \rangle = \lm \langle u,u \rangle  = \lm 
                \end{equation*}
            \item Lo demostraremos a continuación.
            \item Por doble implicación:
                \begin{description}
                    \item [$\Longrightarrow )$] Si tiene un valor propio positivo, tenemos que $0<\lm \leq \lm_1$.
                    \item [$\Longleftarrow )$] Supongamos que $\lm_1>0$, 3 nos dice que entonces $\lm_1$ es un valor propio, por lo que hay un valor propio positivo.
                \end{description}
        \end{enumerate}

        \noindent
        Para el tercer apartado:
        \begin{center}
            Si $\lm_1>0\quad \Longrightarrow\quad  \lm_1$ es un valor propio.
        \end{center}
        Lo que haremos será buscar $u_1\in H\setminus\{0\}$ con $\|u_1\| = 1$ de forma que $T(u_1) = \lm u_1$, por lo que $\lm_1 = \langle T(u_1),u_1 \rangle $. Es decir, deberemos probar además que $\lm_1$ en realidad es un máximo.\\

        \noindent
        Como $\lm_1$ es un supremo, podemos encontrar $\{u_n\}$ en $H$ con $\|u_n\| = 1\quad \forall n \in \mathbb{N}$ y $\{\langle T(u_n),u_n \rangle \}\to \lm_1$. Como la sucesión está acotada, tenemos que existe una parcial suya $\{u_{\sigma(n)}\}$ con $\{u_{\sigma(n)}\}\rightharpoonup u_0$. Supondremos sin pérdida de generalidad que la sucesión $\{u_n\}$ que habíamos tomado al inicio verifica que $\{u_n\}\rightharpoonup u_0$. Como $T$ es lineal y compacta, tenemos que $\{T(u_n)\}\to T(u_0)$, por lo que entonces: 
        \begin{equation*}
            \lm_1 \leftarrow \{\langle T(u_n),u_n \rangle \} \to \langle T(u_0),u_0 \rangle 
        \end{equation*}
        de donde $\lm_1 = \langle T(u_0),u_0 \rangle $, y necesitamos ver que $\|u_0\| = 1$ para concluir que $\lm_1$ en realidad es un máximo. Observamos que solo sabemos que: 
        \begin{equation*}
            \|u_0\| \leq \lim_{n\to\infty}\|u_n\| = 1
        \end{equation*}
        Como $\lm_1>0$ no puede ser $u_0=0$. Afirmamos que:
        \begin{center}
            Si $\langle Tu,u \rangle = \lm_1$ con $0<\|u\|\leq 1 \quad\Longrightarrow\quad T(u) = \lm_1 u$
            \begin{proof}
                Tomamos $v = \nicefrac{u}{\|u\|}$ con lo que $\|v\| = 1$ y:
                \begin{equation*}
                    \langle Tv,v \rangle  = \frac{\langle Tu,u \rangle }{\|u\|^2} = \frac{\lm_1}{\|u\|^2} \geq \lm_1
                \end{equation*}
                Pero por definición de $\lm_1$ no puede ser $\langle Tv,v \rangle > \lm_1 $, por lo que en realidad:
                \begin{equation*}
                    \langle Tv,v \rangle  = \frac{\langle Tu,u \rangle }{\|u\|^2} = \frac{\lm_1}{\|u\|^2} = \lm_1
                \end{equation*}
                de donde $\|u\| = 1$, por lo que $\lm_1 = \langle Tu,u \rangle $, de donde:
                \begin{equation*}
                    \lm_1 = \max\{\langle Tu,u \rangle :\|u\| = 1\}
                \end{equation*}
                Si tomamos ahora $A = T-\lm_1 I$, veamos ahora que:
                \begin{equation*}
                    \langle Av,v \rangle  \leq 0 \qquad \forall v\in H
                \end{equation*}
                Y si consideramos $v = u+tv$ obtenemos:
                \begin{equation*}
                    f(t) = \langle A(u+tv),u+tv \rangle 
                \end{equation*}
                con:
                \begin{equation*}
                    f(0) = \langle Au,u \rangle  = \langle Tu,u \rangle  - \lm_1\langle u,u \rangle  = \lm_1 - \lm_1 = 0
                \end{equation*}
                Así, $0$ es el máximo de la función $f$. Tenemos que demostrar que a partir de esta función real (es polinomio de grado 2) que tiene un máximo en $0$, calculando derivada e igualando a $0$ llegamos a que $Tu = \lm_1 u$.
            \end{proof}
        \end{center}
        Y si aplicamos esta afirmación a $u=u_0$ obtenemos lo que buscábamos.


        % // TODO: NO sé donde va esto
        \begin{equation*}
            \langle Ax,x \rangle  = \langle Tx,x \rangle  - \lm_1 \langle x,x \rangle  \leq 0
        \end{equation*}
        trivial si $x= 0$ y si $x\neq0$ tenemos que:
        \begin{equation*}
            \left\langle T\left(\frac{x}{\|x\|}\right),\frac{x}{\|x\|} \right\rangle  \leq \lm_1 \left\langle \frac{x}{\|x\|},\frac{x}{\|x\|} \right\rangle  = \lm_1
        \end{equation*}
        Veamos ahora que la función
        \Func{}{H}{\bb{R}}{x}{\langle Ax,x\rangle}
        tiene máximo en 0. Para ello, definimos $f:\mathbb{R}\to \mathbb{R}$ dada por:
        \begin{equation*}
            f(t) = \langle A(u_0 + tv),u_0+tv \rangle 
        \end{equation*}
        tenemos que:
        \begin{equation*}
            f(t) \leq f(0) = \langle Au_0,u_0 \rangle  = 0
        \end{equation*}
        Como $T$ y la identidad son lineales tenemos que $A$ es lineal, por lo que:
        \begin{align*}
            f(t) &= \langle A(u_0+tv),u_0+tv \rangle  = \langle Au_0 + tAv,u_0+tv \rangle  \\
                 &= \langle Au_0,u_0 \rangle  + t\left[\langle Au_0,v \rangle  + \langle Av,u_0 \rangle \right] + t^2 \langle Av,v \rangle 
        \end{align*}
        Si tiene un máximo en $0$ podemos decir que $f'(0) = 0$. % idk why
        Y la derivada en $0$ nos queda:
        \begin{equation*}
            0 = f'(0) = \langle Au_0,v \rangle  + \langle Av,u_0 \rangle  = \langle Au_0,v \rangle  + \langle v,Au_0 \rangle  = 2\langle Au_0,v \rangle 
        \end{equation*}
        de donde:
        \begin{equation*}
            0 = \langle Au_0,v \rangle  \qquad \forall v\in H
        \end{equation*}
        o equivalentemente, que:
        \begin{equation*}
            \langle Tu_0,v \rangle  = \lm_1\langle u_0,v \rangle \quad \forall v\in H \quad  \Longleftrightarrow\quad  Tu_0 = \lm_1 u_0
        \end{equation*}
    \end{proof}
\end{lema}


\noindent
Si el supremo nos sale 0 no podremos tener ningún valor propio positivo. Si nos sale distinto de $0$, probaremos que $\lm_1$ es un valor propio positivo. 

\noindent
Qué pasa si no hay ningún valor propio positivo: podemos trabajar con los negativos y es análogo:
\begin{equation*}
    \lm_{-1} = \inf\{\langle Tu,u \rangle : \|u\| = 1\}
\end{equation*}
El Teorema anterior sigue siendo válido, cambiando grande por pequeño y $<0$ por $>0$:
\begin{lema}
    Sea $H$ un espacio de Hilbert, $T:H\to H$ una aplicación lineal, compacta y simétrica. Sea:
    \begin{equation*}
        \lm_{-1} = \inf\{\langle Tu,u \rangle : \|u\| = 1\}
    \end{equation*}
    Observemos que $\lm_1$ es valor propio si existe $u_1$ de módulo 1 con $T(u_1) = \lm_1u_1$.
    \begin{enumerate}
        \item Si $dim H = \infty\Longrightarrow \lm_1\leq 0$.
        \item Si $\lm$ es un valor propio de $T$ entonces $\lm_{-1}\leq \lm$.
        \item Si $\lm_{-1}<0$ entonces $\lm_{-1}$ es un valor propio (el más pequeño) de $T$.
        \item $T$ tiene un valor propio negativo $\Longleftrightarrow \lm_{-1}<0$.
    \end{enumerate}
\end{lema}

\noindent
Una vez tenemos un valor propio, aplicamos el Ejercicio anterior y buscamos en $T:V^\perp\to V^\perp$; buscando el valor propio más grande. Tendremos una sucesión positiva decreciente que solo puede converger a 0, así como una sucesión negativa creciente.\\

\noindent
Nos preguntamos finalmente si están todos los valores en la sucesión o si falta alguno.\\

\noindent
La idea para obtener los valores propios es la siguiente: dado un operador $T:H\to H$ bajo las condiciones enunciadas, tomamos $\lm_1$ el visto en el Lema anterior, y tomamos $u_1$ una función propia de $T$ con $\|u_1\| = 1$.

\noindent
Tomamos ahora $V = \langle u_1 \rangle $, con lo que $T(V)\subseteq V$. Podemos ahora aplicar el Ejercicio anterior, obteniendo que $T:V^\perp\to V^\perp$ para:
\begin{equation*}
    V^\perp = \{v\in H : \langle v,u_1 \rangle =0\}
\end{equation*}
que claramente es un subespacio cerrado, luego también Hilbert, y va a seguir siendo compacta y simétrica. Podemos volver a aplicar el Lema a esta aplicación, obteniendo:
\begin{equation*}
    \lm_2 = \sup\{\langle Tx,x \rangle :x\in V^\perp, \|x\| = 1\}
\end{equation*}
\begin{enumerate}
    \item Como teníamos que $dim H = \infty \Longrightarrow dim V^\perp = \infty$
    \item Todo valor propio que no sea $\lm_1$ tendrá una función propia, que será ortogonal a $u_1$, por lo que estará en $V^\perp$. No puede haber ningún valor propio entre $\lm_1$ y $\lm_2$. 

        Puede que nos salga $\lm_2 = \lm_1$.
\end{enumerate}
Podemos repetir el razonamiento, que se acaba si $dim H < \infty$ y que no se acaba si $dim H = +\infty$, bien por arriba o por abajo. Supondremos ahora que $dim H = \infty$.

% // TODO: Meter definición de ortogonal en algún punto

\begin{lema}
    Supongamos que tenemos construida una sucesión de valores propios:
    \begin{equation*}
        0 < \lm_{n-1} \leq \lm_{n-2} \leq \ldots \leq \lm_1, \qquad n\geq 2
    \end{equation*}
    con $u_{n-1},u_{n-2}, \ldots, u_1$ funciones propias asociadas respectivamente a $\lm_{n-1},\lm_{n-2},\ldots,\lm_1$, de módulo 1 y ortogonales dos a dos. Si tomamos:
    \begin{equation*}
        \lm_n = \sup\{\langle Tx,x \rangle : \|x\| = 1, x\perp u_1,u_2,\ldots,u_{n-1}\}
    \end{equation*}
    Entonces se tiene que:
    \begin{enumerate}
        \item Si $dim H = \infty \quad \Longrightarrow\quad  \lm_n\geq 0$.
        \item Si $\lm$ es valor propio de $T$ con $\lm<\lm_{n-1}\quad\Longrightarrow\quad \lm\leq \lm_n$.
        \item Si $\lm_n>0 \quad\Longrightarrow\quad \lm_n$ es un valor propio de $T$.
    \end{enumerate}
\end{lema}

\noindent
Y existe un Lema análogo para valores propios negativos.

\noindent
No tenemos dos sucesiones una de positivos y otra negativa, ya que el paso recursivo depende de la tercera condición: ``Si $\lm_n>0$ entonces $\lm_n$ es un valor propio de $T$''. Podríamos tener un operador con un número finito de valores propios positivos. Sin embargo, en dicho caso tendríamos una cantidad infinita de números negativos, ya que estamos suponiendo que la dimensión de $H$ es infinita, por lo que hay al menos una sucesión de valores propios.\\

\noindent
Además, como $0$ era el único posible punto de acumulación:
\begin{itemize}
    \item Si hay infinitos valores propios positivos, estos convergen a $0$.
    \item Si hay infinitos valores propios negativos, estos convergen a $0$.
\end{itemize}
Sabemos que entre dos valores propios no hay ninguno más, pero queremos ver si la sucesión recoge todos los valores, que es el Teorema espectral.\\

\begin{ejercicio} % // TODO: HACER
    Sea $H$ un espacio de Hilbert, si $T:H\to H$ es lineal, acotada y simétrica, entonces:
    \begin{equation*}
        \|T\| = \sup\{\|Tx\| : \|x\| = 1\} = \ßup_{|\langle Tu,u \rangle : \|u\| = 1|}
    \end{equation*}
\end{ejercicio}

\noindent
Consideramos:
\begin{equation*}
    A = \left\{n\in \mathbb{Z}\setminus\{0\} : 
        \left\{\begin{array}{ll}
            \lm_n>0 & \text{si\ } n>0 \\
            \lm_n<0 & \text{si\ } n<0
        \end{array}\right.\right\}
\end{equation*}
Tenemos que como $dim H = \infty \quad\Longrightarrow\quad A$ tiene cardinal infinito. Tenemos que:
\begin{equation*}
    Tx = \sum_{n\in A} \lm_n \langle x,u_n \rangle u_n \qquad \forall x\in H
\end{equation*}
\begin{proof}
    Sea $N\in \mathbb{N}$, tomamos:
    \begin{equation*}
        x_N = x- \sum_{\substack{n=-N\\n\in A}}^N \langle x,u_n \rangle u_n
    \end{equation*}
    Como $T$ es lineal tenemos que:
    \begin{equation*}
        Tx_N = Tx - \sum_{\substack{n=-N\\n\in A}}^n \langle x,u_n \rangle \lm_n u_n
    \end{equation*}
    Para ver que $\{Tx_N\}\to 0$, tenemos que:
    \begin{equation*}
        \langle x_n,u_k \rangle  = \langle x,u_k \rangle  - \sum_{\substack{n=-N\\n\in A}}^N \langle x,u_n \rangle \langle u_n,u_k \rangle  = \langle x,u_k \rangle  - \langle x,u_k \rangle\cdot  1 = 0
    \end{equation*}
    De aquí tenemos que:
    \begin{equation*}
        x_n \in \langle u_1,u_2,\ldots,u_N,u_{-1},u_{-2},\ldots,u_{-N} \rangle^\perp = V^\perp
    \end{equation*}
    luego:
    \begin{equation*}
        \|Tx_N\| \leq \|T\big|_{V^\perp}\| \|x_k\| = \sup\{|\langle Tu,u \rangle |: \|u\| = 1 , u\in V^\perp\} \cdot \|x_N\|
    \end{equation*}
    El supremo será igual si observamos solo $u_1,u_2,\ldots,u_N$ o si observamos solo $u_{-1},u_{-2},\ldots,u_{-N}$, por lo que:
    \begin{equation*}
        \|Tx_N\| \leq \min\{|\lm_N|, |\lm_{-N}|\} \|x_N\|
    \end{equation*}
    De aquí:
    \begin{equation*}
        \langle x_N,x_N \rangle  = \langle x,x_N \rangle  - \sum_{\substack{n=-N\\n\in A}}^N \langle x,u_n \rangle  \langle u_n,x \rangle  = \langle x,x \rangle \ldots \leq \|x\|^2
    \end{equation*}
    Por lo que:
    \begin{equation*}
        \|Tx_N\| \leq \min\{\lm_N, -\lm_{-N}\}\|x\|
    \end{equation*}
    Y como $\{\min\{\lm_N, -\lm_{-N}\}\}\to 0$ tenemos entonces que $\{\|Tx_n\|\}\to 0$.
\end{proof}
\noindent
Así, la fórmula de diagonalización es:
\begin{equation*}
    \boxed{Tx = \sum_{n\in A} \lm_n \langle x,u_n \rangle u_n\qquad \forall x\in H}
\end{equation*}

\begin{teo}
    Sea $H$ un espacio de Hilbert con $dim H = \infty$ y $T:H\to H$ lineal, simétrica y compacta, entonces los valores propios no nulos de $T$ están caracterizados por ``nuestra'' construcción inductiva $\{\lm_n : n\in A\}$.
    \begin{proof}
        Sea $\lm\neq 0$ valor propio de $T$, entonces existe $u\in H\setminus\{0\}$ con $Tu = \lm u$, y podemos suponer sin pérdida de generalidad que $\|u\| = 1$. Supongamos por reducción al absurdo que $\lm\neq \lm_n$ para todo $n\in A$, por lo que $u\perp u_n$. Tenemos entonces que:
        \begin{equation*}
            \lm u = Tu = \sum_{n\in A} \lm_n \langle u,u_n \rangle u_n = 0
        \end{equation*}
        de donde $\lm u = 0$ con $\lm\neq 0$ y $u\neq 0$, contradicción.
    \end{proof}
\end{teo}

% // TODO: HACER los siguientes ejercicios
\begin{ejercicio}
    Sea $H$ un espacio de Hilbert y $T:H\to H$ lineal, simétrico y compacto. Probar:
    \begin{equation*}
        T = 0 \quad \Longleftrightarrow\quad  A = \emptyset 
    \end{equation*}
\end{ejercicio}

\begin{ejercicio}
    Sea $H$ un espacio de Hilbert y $T:H\to H$ lineal, simétrico y compacto, si observamos:
    \begin{equation*}
        H = \ker T \oplus {(\ker T)}^{\perp}
    \end{equation*}
    Tenemos que $\{u_n: n\in A\}$ es una base ortonormal de ${(\ker T)}^{\perp}$. Es decir:
    \begin{equation*}
        \forall x\in {(\ker T)}^{\perp} \qquad x = \sum_{n\in A}\langle x,u_n \rangle u_n
    \end{equation*}
\end{ejercicio}
