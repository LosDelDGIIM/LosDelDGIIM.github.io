\chapter{Espacios $L^p(\Omega)$} 
% // TODO: Para esta sección, reformularlo todo hasta el TODO TODO:
% // TODO: Leer del Brezis la parte introductoria

\noindent
Recordemos\footnote{Se vio ya en Análisis Matemático II y se recordó en la Sección~\ref{sec:Lp}.} que para $1\leq p < \infty$ definíamos para un conjunto medible $\Omega\subset \mathbb{R}^N$:
\begin{equation*}
    L^p(\Omega) = \left\{[f] \text{\ con\ } f:\Omega\to \mathbb{R} \text{\ medible tal que\ } \int_\Omega |f|^p~dx < \infty\right\}
\end{equation*}

donde las clases de equivalencia vienen dadas por:
\begin{equation*}
    f\sim g \quad \Longleftrightarrow \quad f = g \text{\ casi por doquier en\ } \Omega
\end{equation*}
De esta forma, al considerar:
\begin{equation*}
    \cc{L}^p(\Omega) = \left\{f:\Omega\to \mathbb{R} \text{\ con\ } f \text{\ medible tal que\ } \int_\Omega |f|^p~dx\right\}
\end{equation*}

tenemos que:
\begin{equation*}
    L^p(\Omega) = \frac{\cc{L}^p(\Omega)}{\sim}
\end{equation*}

\begin{notacion}
    Normalmente realizaremos un abuso de notación, denotando simplemente por $f$ a $[f]$.
\end{notacion}

\noindent
Y teníamos que $L^p(\Omega)$ era un espacio normado, con la norma:
\begin{equation*}
    \|[f]\|_p = {\left(\int_\Omega|f|^p~dx\right)}^{\nicefrac{1}{p}}
\end{equation*}
Y el Teorema de Riesz-Fischer nos dice que es de Banach, y tenía como Corolario:

\begin{coro}\label{coro:corolp}
    Si $\{f_n\}$ es convergente en $L^p(\Omega)$ hacia $f\in L^p(\Omega)$, entonces existe una parcial de forma que $\{f_{\sigma(n)}\}\to f$ casi por doquier en $\Omega$. Además, existe $h\in L^p(\Omega)$ tal que:
    \begin{equation*}
        |f_{\sigma(n)}(x)| \leq h(x) \qquad \text{casi para todo\ } x\in \Omega
    \end{equation*}
\end{coro}

Y el Teorema de Riesz-Fischer usaba en gran medida:
\begin{teo}[convergencia dominada] % // TODO: Escribirlo bien a partir de Paya
    Si $\{f_n\}$ es una sucesión de funciones medibles con $\{f_n(x)\}\to f(x)$ para casi todo $x\in \Omega$ y existe $h\in L^1(\Omega)$ de forma que:
    \begin{equation*}
        |f_n(x)| \leq h(x) \qquad \text{para casi todo\ } x\in \Omega
    \end{equation*}
    Entonces:
    \begin{equation*}
        \left\{\int_\Omega f_n~dx\right\} \to \int_\Omega f~dx
    \end{equation*}
    Si supiésemos además que $f\in L^1(\Omega)$ entonces obtenemos que:
    \begin{equation*}
        \left\{\int_\Omega |f_n-f|~dx\right\} \to 0
    \end{equation*}
\end{teo}

\begin{ejercicio}
    Si tenemos $\{f_n\}$ una sucesión de funciones medibles con $\{f_n(x)\}\to f(x)$ para casi todo $x\in \Omega$ con $f\in L^p(\Omega)$ y existe $h\in L^p(\Omega)$ tal que:
    \begin{equation*}
        |f_n(x)| \leq h(x) \qquad \text{para casi todo\ } x\in \Omega
    \end{equation*}
    Entonces:
    \begin{equation*}
        \left\{\int_\Omega |f_n-f|^p~dx\right\} \to 0
    \end{equation*}
\end{ejercicio}

\noindent
Y observamos que el Corolario~\ref{coro:corolp} es casi el recíproco de este último ejercicio.

\begin{notacion}
    Usaremos ``a.e.'' (\textit{almost everywhere}) para escribir ``c.p.d'' (casi por doquier).
\end{notacion}

\subsubsection{Si $p=\infty$}
Definimos entonces:
\begin{equation*}
    L^\infty(\Omega) = \{[f] \text{\ con\ } f:\Omega\to \mathbb{R} \text{\ medible y acotada\ }\}
\end{equation*}
que es un espacio normado, con la norma:
\begin{equation*}
    \|[f]\|_\infty = \inf \left\{k\geq 0 : |f(x)|\leq k \quad \text{para casi todo\ } x\in \Omega\right\}
\end{equation*}

\begin{ejemplo}
    Si tomamos $\Omega = \left]0,1\right[$ y tomamos $f:\Omega\to \mathbb{R}$ la función $f(x) = 1$ $\forall x\in \Omega$, tenemos que:
    \begin{align*}
        &\inf\{k\geq 0 : |f(x)|\leq k \quad \text{para casi todo\ } x\in \Omega\} \AstIg \\
        &\inf\{k\geq 0 : |f(x)| \leq k \quad \forall x\in \left]0,1\right[\} = \\
        &\sup_{x\in \left]0,1\right[}|f(x)| = \|f\|_\infty
    \end{align*}
    donde en $(\ast)$ usamos que $f$ es continua. Tenemos entonces que $\|f\|_\infty = 1$. Si consideramos ahora:
    \begin{equation*}
        f(x) = \left\{\begin{array}{ll}
            1 & \text{si\ } x\in \Omega\setminus \{\nicefrac{1}{2}\} \\
             2 & \text{si\ } x = \nicefrac{1}{2}
        \end{array}\right. 
    \end{equation*}
    Tenemos entonces que:
    \begin{gather*}
        \inf \{k\geq 0 : |f(x)|=1 \leq k \quad \text{casi para todo\ } x\in \Omega\} = 1 \\
        \inf \{k\geq 0 : |f(x)|\leq k \quad \forall x\in \left]0,1\right[\} = 2
    \end{gather*}
    Al primero se le suele denotar normalmente por:
    \begin{equation*}
        \text{esssup}_{x\in \left]0,1\right[}|f(x)|
    \end{equation*}
\end{ejemplo}

\noindent
Se verifica que $(L^\infty(\Omega),\|\cdot \|_\infty)$ es un espacio de Banach.\\

\noindent
Todos estos espacios verifican: % // TODO: Tener en cuenta que no es discreto
\begin{equation*}
    L^1(\Omega) \supset \ldots \supset L^p(\Omega) \supset \ldots \supset L^\infty(\Omega)
\end{equation*}
La desigualdad de Hölder nos da para $p_2<p_1$:
\begin{equation*}
    \|f\|_{p_2}^{p_2} = \int_\Omega |f|^{p_2}~dx \leq {\left(\int_\Omega {(|f|^{p_2})}^{\nicefrac{p_1}{p_2}}~dx\right)}^{\nicefrac{p_2}{p_1}} {(|\Omega|)}^{1-\nicefrac{p_2}{p_1}} = \|f\|_{p_1}^{p_2}|\Omega|^{1-\nicefrac{p_2}{p_1}}
\end{equation*}
de donde:
\begin{equation*}
    \|f\|_{p_2} \leq \|f\|_{p_1} |\Omega|^{\nicefrac{1}{p_2}-\nicefrac{1}{p_1}}
\end{equation*}
En particular tenemos la inclusión de forma continua, por esta última igualdad. Finalmente, es trivial que si $f\in L^\infty(\Omega)$ entonces $|f|^p\in L^\infty(\Omega)$, por lo que esta última función será integrable, de donde $f\in L^p(\Omega)$. De esta forma:
\begin{equation*}
    \|f\|_p \leq \|f\|_\infty \cdot |\Omega|^{\nicefrac{1}{p}} \qquad \forall f\in L^\infty(\Omega)
\end{equation*}
Por lo que la aplicación
\Func{}{L^\infty(\Omega)}{L^p(\Omega)}{f}{f}
es continua.\\

\noindent
A partir de estas inclusiones:
\begin{equation*}
    L^\infty(\Omega) \subset \bigcap_{1\leq p < \infty}L^p(\Omega)
\end{equation*}
\begin{ejercicio}
    Pensar si se da la igualdad. En el caso que se dé, consideramos $\|\cdot \|_p$ con $p\to \infty$ y nos preguntamos si es igual a $\|\cdot \|_\infty$. De la desigualdad:
    \begin{equation*}
        \|f\|_p \leq \|f\|_\infty \cdot {|\Omega|}^{\nicefrac{1}{p}}
    \end{equation*}
    se deduce que $\|f\|$ limite $\leq \|f\|$
\end{ejercicio}


% // TODO: No se que es esto, derivada debil
\begin{ejercicio}
    Si tenemos $u\in C^1(a,b)$ y $v\in C^0(a,b)$ de forma que:
    \begin{equation*}
        \int_{a}^{b} u\varphi '~dx  = -\int_{a}^{b} v\varphi~dx  \qquad \forall \varphi \in C^1_c(a,b)
    \end{equation*}
    (donde $C_c^1$ son las de clase 1 y soporte compacto). Entonces:
    \begin{equation*}
        \int_{a}^{b} [u'(x) -v(x)]\varphi(x)~dx  = 0 \qquad \forall \varphi \in C^1_c(a,b)
    \end{equation*}
    Si en cierto punto $x_0$ tenemos $u'(x_0)-v(x_0)\neq 0$ entonces tenemos un intervalo donde es distinto de cero, y podemos encontrar $\varphi$ conveniente para que la integral no salga cero, llegando a contradicción. De aquí, $u' = v$.

    \noindent
    Definiremos la derivada débil de $u$ (integrable en el soporte de $\varphi$) como la $v$ que cumple:
    \begin{equation*}
        \int_{a}^{b} u\varphi '~dx  = -\int_{a}^{b} v\varphi~dx 
    \end{equation*}
    Por tanto, solo hace falta que $u$ sea integrable en compactos contenidos en $a,b$.
\end{ejercicio}

\begin{definicion}
    \begin{equation*}
        L^1_{\text{loc}}(\Omega) = \{u:\Omega\to \mathbb{R} \text{\ medibles con\ } u\big|_K \text{\ integrables\ } \forall K\subset \Omega \text{\ compacto}\}
    \end{equation*}
\end{definicion}

\noindent
Tenemos entonces que $u$ es derivable si y solo si existe $v\in L^1_{\text{loc}}(\Omega)$ tal que:
\begin{equation*}
    \int_\Omega u\varphi ' ~dx = -\int_\Omega v\varphi~dx \qquad \forall \varphi \in C^1_c(\Omega)
\end{equation*}
\begin{ejercicio}
    Con esta definición tenemos que $|\cdot |:\left]-1,1\right[\to \left]-1,1\right[$ es derivable.\\

    \noindent
    Sean $f\in L^1_{\text{loc}}(a,b)$ y $g\in L^1_{\text{loc}}(a,b)$ $g$ es derivada de $f$ si:
    \begin{equation*}
        \int_{a}^{b} f(x)\varphi'(x)~dx  = -\int_{a}^{b} g(x)\varphi(x)~dx  \qquad \forall \varphi \in C^1_c(a,b)
    \end{equation*}
    Si tomamos $f(x) = |x|$ para $x\in \left]-1,1\right[$, y $g(x) = sgn(x)$ para $x\in \left]-1,1\right[$ tenemos:
    \begin{align*}
        \int_{-1}^{1} |x|\varphi'(x)~dx  &= \int_{-1}^{0} -x\varphi'(x)~dx  + \int_{0}^{1} x\varphi'(x)~dx  \\
                                         &\AstIg \int_{-1}^{0} \varphi(x)~dx  - \int_{0}^{1} \varphi(x)~dx  = \int_{-1}^{1} sgn(x)\varphi(x)~dx 
    \end{align*}
    donde en $(\ast)$ hemos hecho integración por partes. 
\end{ejercicio}
Sustituimos de esta forma $C^1(\Omega)$ por $L^1_{\text{loc}}(\Omega)$, pero no por $L^1(\Omega)$.

\noindent
Para ver que una función está en $L^1_{\text{loc}}$ basta ver que tiene integral finita en un compacto pequeño en cada punto, algo que no es verdad en $L^1(\Omega)$.

% // TODO TODO:
\section{Dual de $L^p(\Omega)$ para $1<p<\infty$}
\noindent
Sea $1<p<\infty$, consideramos la aplicación
\Func{T}{L^{p'}(\Omega)}{L^p(\Omega)^\ast}{v}{Tv}
donde $Tv:L^p(\Omega)\to \mathbb{R}$ viene dada por:
\begin{equation*}
    \langle Tv,w \rangle  := \int_\Omega vw
\end{equation*}
Fijado $v\in L^{p'}(\Omega)$ tenemos que $Tv$ está bien definida, ya que para cada $w\in L^p(\Omega)$ la desigualdad de Hölder nos dice que:
\begin{equation*}
    \int_{\Omega}|vw| \leq {\left(\int_\Omega |v|^{p'}\right)}^{\nicefrac{1}{p'}} {\left(\int_\Omega |w|^p\right)}^{\nicefrac{1}{p}}
\end{equation*}
de donde la función $|vw|$ medible y positiva tiene integral finita, luego es integrable, lo que equivale a que $vw$ sea una función integrable. De aquí tenemos que $Tv$ está bien definida. De la desigualdad deducimos también que:
\begin{equation*}
    |\langle Tv,w \rangle |\leq \|v\|_{p'} \|w\|_p \qquad \forall w\in L^p(\Omega), \quad \forall v\in L^{p'}(\Omega)
\end{equation*}
Fijado $v\in L^{p'}(\Omega)$, como $Tv$ es lineal esta última desigualdad nos dice ahora que $Tv\in L^p(\Omega)^\ast$. Esto último nos dice que $T$ está bien definida. Es claro también que $T$ es lineal. Si para cada $v\in L^{p'}(\Omega)$ calculamos $\|Tv\|$, tenemos por un lado que $\|Tv\| \leq \|v\|_{p'}$ y por otro que si tomamos\footnote{Para el caso $2<p<\infty$ tenemos $1<p'<2$, por lo que si $v(x) = 0$ tendríamos una división por cero, que se evita con una función a trozos.}:
\begin{equation*}
    w(x) = \left\{\begin{array}{ll}
        |v(x)|^{p'-2}\cdot v(x) & \text{si\ } v(x)\neq 0 \\
         0& \text{si\ } v(x) = 0
    \end{array}\right. 
\end{equation*}
Tenemos que $w\in L^{p}(\Omega)$, ya que (tomando $\Omega' = \{x\in \Omega : v(x)\neq 0\}$):
\begin{equation*}
    \|w\|_p = {\left[\int_{\Omega'} {\left(|v(x)|^{p'-1}\right)}^{p}\right]}^{\nicefrac{1}{p}} = {\left(\int_{\Omega'}|v|^{p'}\right)}^{\nicefrac{1}{p}} = {\left[\int_{\Omega} {\left(|v|^{p'}\right)}^{\nicefrac{1}{p'}}\right]}^{\nicefrac{p'}{p}} = \|v\|_{p'}^{p'-1}
\end{equation*}
Y por otra parte si aplicamos $Tv$ a $w$ obtenemos:
\begin{equation*}
    \langle Tv,w \rangle  = \int_\Omega vw  = \int_{\Omega'} |v|^{p'} = \int_{\Omega}|v|^{p'} = \|v\|_{p'}^{p'} = \|v\|_{p'}^{p'-1}\|v\|_{p'} = \|w\|_{p}\|v\|_{p'}
\end{equation*}
de donde ha de ser $\|Tv\| = \|v\|_{p'}$, para cada $v\in L^{p'}(\Omega)$, luego $T$ es una isometría, y en particular es continua e inyectiva. Falta por tanto probar que $T$ es sobreyectiva para ver que $L^p(\Omega)$ y $L^{p'}(\Omega)$ son isométricos.\\

\noindent
Antes de ello podemos decir ya varias cosas, como por ejemplo: 
\begin{itemize}
    \item Como $L^{p'}(\Omega)$ es completo y $T$ es una isometría tenemos por tanto que $T(L^{p'}(\Omega))$ es un subespacio completo de $L^p(\Omega)^\ast$. 
    \item Además, como en un espacio commpleto tenemos que un subespacio es completo si y solo si es cerrado, deducimos que $T(L^{p'}(\Omega))$ es un subespacio cerrado de $L^p(\Omega)^\ast$.
    \item Observamos también que como $(p')'=p$, tenemos por el mismo argumento que hemos visto que hay una isometría
        \begin{equation*}
            L^p(\Omega) \to L^{p'}(\Omega)^\ast
        \end{equation*}
\end{itemize}
Para ver la sobreyectividad de $T$ necesitaremos ver previamente ciertos resultados, a partir de las hipótesis en las que nos encontramos.

\begin{teo}
    $L^p(\Omega)$ es reflexivo para $2\leq p < \infty$.
    \begin{proof}
        Fijado $p\in \left[2,+\infty\right[$, se verifica que:
        \begin{equation*}
            \left|\frac{a+b}{2}\right|^p + \left|\frac{a-b}{2}\right|^p \leq \frac{1}{2}(|a|^p+|b|^p) \qquad \forall a,b\in \mathbb{R}
        \end{equation*}
            Sin lugar a dudas, resolver tal trivialidad sería un insulto a la autonomía e inteligencia de nuestro estimado lector, por lo que se estima conveniente que dicho individuo sea quien lo realice por sí mismo. Tenemos por tanto que:
            \begin{equation*}
                \int_{\Omega}\left|\frac{w(x)+\overline{w}(x)}{2}\right|^p \int_{\Omega} \left|\frac{w(x)-\overline{w}(x)}{2}\right|^p \leq \int_{\Omega}\frac{1}{2}(|w(x)|^p + |\overline{w}(x)|^p)
            \end{equation*}

            o equivalentemente:
            \begin{equation*}
                \left\|\frac{w+\overline{w}}{2}\right\|_p^p + \left\|\frac{w-\overline{w}}{2}\right\|_p^p  \leq \frac{1}{2} (\|w\|_p^p + \|\overline{w}\|_p^p) \qquad \forall w,\overline{w}\in L^p(\Omega)
            \end{equation*}
            desigualdad que recibe el nombre de ``Primera\footnote{La segunda identidad es para el caso $p<2$, que es ligeramente más complicada.} Identidad de Clarkson''. Como consecuencia, tenemos que para $\varepsilon \in \left]0,1\right[$ dado y $w,\overline{w}\in L^p(\Omega)$ con $\|w\|_p=1=\|\overline{w}\|_p$ y $\|w-\overline{w}\|_p>\varepsilon$ se tiene que:
            \begin{equation*}
                \left\|\frac{w+\overline{w}}{2}\right\|_p^p \leq \frac{1}{2}(1+1) - \left\|\frac{w-\overline{w}}{2}\right\|_p^p = 1 - \frac{\|w-\overline{w}\|_p^p}{2^p} < 1 - \frac{\varepsilon^p}{2^p}
            \end{equation*}

            de donde:
            \begin{equation*}
                \left\|\frac{w+\overline{w}}{2}\right\|_p \leq {\left(1-\frac{\varepsilon^p}{2^p}\right)}^{\nicefrac{1}{p}} < 1
            \end{equation*}
            Por lo que si tomamos:
            \begin{equation*}
                \delta = 1-{\left(1-\frac{\varepsilon^p}{2^p}\right)}^{\nicefrac{1}{p}}
            \end{equation*}

            tenemos que $0<\delta<1$, de donde:
            \begin{equation*}
                \left\|\frac{w+\overline{w}}{2}\right\|_p < 1-\delta
            \end{equation*}
            Por tanto, tenemos que $L^p(\Omega)$ es estrictamente convexo, luego es reflexivo por el Teorema de Milman-Pettis.
    \end{proof}
\end{teo}

\begin{coro}
    $L^p(\Omega)$ también es reflexivo para $1<p<2$.
    \begin{proof}
        Fijado $p\in \left]1,2\right[$, sabemos ya que existe una isometría 
        \begin{equation*}
            T:L^p(\Omega) \to L^{p'}(\Omega)^\ast
        \end{equation*}
        Así como que $T(L^p(\Omega))$ es un subespacio cerrado de $L^{p'}(\Omega)^\ast$. 

        \noindent
        Tendremos que $p'\in \left]2,+\infty\right[$, por lo que por el Teorema anterior ha de ser $L^{p'}(\Omega)$ reflexivo, y como el dual de un espacio reflexivo es reflexivo, tenemos que $L^{p'}(\Omega)^{\ast}$ es también reflexivo. Como $T(L^p(\Omega))$ es un subesapcio cerrado de un espacio reflexivo ha de ser también reflexivo. Como $T$ es una isometría obtenemos que $L^p(\Omega)$ es también reflexivo.
    \end{proof}
\end{coro}

\noindent
Usando la Segunda Identidad de Clarkson (es ligeramente más enrevesada) se puede probar que $L^p(\Omega)$ es uniformemente convexo para $1<p<2$, de donde también son reflexivos:
\begin{equation*}
    \left\|\frac{w+\overline{w}}{2}\right\|_p^{p'} + \left\|\frac{w-\overline{w}}{2}\right\|^{p'}_p \leq {\left[\frac{1}{2}(\|w\|_p^p + \|\overline{w}\|_p^p)\right]}^{\nicefrac{1}{p-1}}
\end{equation*}

\noindent
Tratamos de probar ahora que $V:=T(L^{p'}(\Omega))$ es denso en $L^p(\Omega)^\ast$, y como es cerrado tendremos por tanto que $T(L^{p'}(\Omega)) = L^p(\Omega)^\ast$, lo que nos dará la sobreyectividad de $T$.

\begin{prop}
    $V:=T(L^{p'}(\Omega))$ es denso en $L^p(\Omega)^\ast$ para $1<p<\infty$.
    \begin{proof}
        Supongamos que $\varphi \in L^p(\Omega)^{\ast\ast}$ con:
        \begin{equation*}
            \langle \varphi,Tv \rangle  = 0 \qquad \forall v\in L^{p'}(\Omega)
        \end{equation*}
        Si conseguimos probar que entonces $\varphi = 0$ tendremos por un Corolario del Teorema de Hahn-Banach que $V$ es denso en $L^p(\Omega)^\ast$.\\

        \noindent
        Hemos visto ya que $L^p(\Omega)$ es reflexivo, por lo que $J(L^p(\Omega)) = L^p(\Omega)^{\ast\ast}$. De aquí deducimos que existe $w\in L^p(\Omega)$ de forma que $\varphi = J(w)$, por lo que la hipótesis se traduce en:
        \begin{equation*}
            0 = \langle J(w),Tv \rangle  = \langle Tv,w \rangle  = \int_{\Omega}vw \qquad \forall v\in L^{p'}(\Omega)
        \end{equation*}
        Tomando ahora:
        \begin{equation*}
            v(x) = \left\{\begin{array}{ll}
                |w(x)|^{p-2}w(x) & \text{si\ } w(x)\neq 0 \\
                 0& \text{si\ } w(x) = 0
            \end{array}\right. 
        \end{equation*}
        Se deduce que $w=0$, por lo que ha de ser $\varphi = J(0) = 0$, y podemos aplicar el Corolario~\ref{coro:comprobar_denso}.
    \end{proof}
\end{prop}

\noindent
En definitiva, tenemos que $V=T(L^{p'}(\Omega))\subset L^p(\Omega)^\ast$ es cerrado y denso, por lo que ha de ser $T(L^{p'}(\Omega)) = L^p(\Omega)^\ast$, luego la aplicación $T$ que definíamos al principio de esta sección es una isometría sobreyectiva, por lo que $L^{p'}(\Omega)$ es isométrico a $L^p(\Omega)^\ast$.\\

\noindent
Esta prueba recibe el nombre \textbf{Teorema de Representación de Riesz} para $L^p(\Omega)$.
