\chapter{Topologías Débiles}
\section{Topologías iniciales}
\noindent
Trabajaremos sobre un conjunto $X$ y una familia de espacios topológicos $\{Y_i\}_{i \in I}$ junto con una familia de aplicaciones $\varphi_i:X\to Y_i, \quad \forall i \in I$.\\

\noindent
Observamos que si consideramos en $X$ la topología discreta:
\begin{equation*}
    \tau_d = \{A : A\subset X\}
\end{equation*}
tenemos que $\varphi_i$ es continua, para todo $i \in I$. Sin embargo, hemos sido ``muy brutos'' al considerar esta topología sobre $X$. Nos preguntamos por definir alguna topología en $X$ que haga que todas las funciones de la familia $\{\varphi_i\}_{i \in I}$ sean continuas con el menor número de abiertos.

\begin{observacion}
    Observemos que como pretendemos que las funciones $\varphi_i$ sean continuas, necesitaremos que esta topología $\tau$ buscada verifique que:
    \begin{equation*}
        \tau \supset \{\varphi_i^{-1}(\omega_i) : \omega_i \text{\ es un abierto de\ } Y_i, \quad \forall i \in I\}
    \end{equation*}
\end{observacion}
\noindent
De esta forma, el problema podemos reformularlo como:

\begin{center}
    Dado un conjunto $X$ y una familia $\cc{U} = \{U_\lm\subset X : \lm \in \Lambda\}$, buscar la topología $\tau$ con menor cantidad de abiertos de forma que $\cc{U}\subset \tau$.
\end{center}

\noindent
Para ello, si pretendemos que los $U_\lm$ estén en $\tau$, estos serán abiertos, luego toda intersección finita de ellos lo seguirá siendo, con lo que la intersección finita de los conjuntos $U_\lm$ también tiene que seguir estando en $\tau$. Es decir, si consideramos:
\begin{equation*}
    \cc{V} = \left\{V = \bigcap_{i=1}^n U_{\lm _i} : \lm_1, \ldots, \lm_n \in \Lambda, \quad n\in \mathbb{N}\right\}
\end{equation*}
Tenemos que $\cc{U}\subset \cc{V}$, y nos preguntamos si $\cc{V}$ es una topología. Observamos:
\begin{itemize}
    \item Primero, que $\cc{V}$ es estable por intersecciones finitas.
    \item Sin embargo, la familia no es cerrada por uniones arbitrarias de elementos del conjunto.
\end{itemize}
Para solucionar el segundo problema, consideramos:
\begin{equation*}
    \left\{\bigcup_{\eta \in \Lambda_0} V_\eta : V_\eta \in \cc{V}, \Lambda_0\subset \Lambda\right\}
\end{equation*}
Y tenemos que la topología más pequeña que buscamos debe contener este conjunto. Se demuestra que este conjunto es, de hecho, una topología.

\begin{observacion}
    Observemos que el vacío es resultado de una unión vacía.

    Sin embargo, faltaría unir el total.
\end{observacion}

\noindent
¿Cómo se forma una base de entornos de dicha topología en $X$?

\begin{definicion}
    Sea $X$ un conjunto, $\{Y_i\}_{i \in I}$ una familia de espacios topológicos y $\{\varphi_i\}_{i \in I}$ una familia de aplicaciones $\varphi_i:X\to Y_i$, definimos la topología inicial para la familia $\{\varphi_i\}_{i \in I}$ sobre $X$ a la topología dada por la base de entornos:
    \begin{equation*}
        \left\{\bigcap_{i \in J}\varphi_i^{-1}(V_i) : V_i \text{\ entorno de\ }\varphi_i(x)\in Y_i, \quad J\subset I \text{\ finito} \right\}
    \end{equation*}
\end{definicion}

\noindent
Aunque no conozcamos en profundidad la topología (puesto que no hemos dado de forma explícita quiénes son sus abiertos), es sencillo en ocasiones probar ciertas propiedades topológicas, usando para ellos las dos proposiciones siguientes, que nos permiten comprobar propiedades sobre la topología inicial sin tener que usarla, sino tratar de buscar problemas equivalentes realizando composiciones con las aplicaciones $\varphi_i$ de la familia que nos da la topología inicial.

\begin{prop}\label{prop:top_inicial_suc}
    Sea $(X,\tau)$ con $\tau$ la topología inicial asociada a una familia de aplicaciones $\{\varphi_i\}_{i \in I}$, sea $\{x_n\}$ una sucesión de puntos de $X$ y $x\in X$:
    \begin{equation*}
        \{x_n\}\to x \quad \Longleftrightarrow \quad  \{\varphi_i(x_n)\} \to \varphi_i(x) \quad \forall i \in I
    \end{equation*}
    \begin{proof}
        Por doble implicación:
        \begin{description}
            \item [$\Longrightarrow )$] Para cada $i \in I$, $\tau$ hace que $\varphi_i$ sea continua, por lo que:
                \begin{equation*}
                    \{x_n\} \to x \Longrightarrow \{\varphi_i(x_n)\} \to \varphi_i(x)
                \end{equation*}
            \item [$\Longleftarrow )$] Si consideramos un entorno $U$ de $x$, este ha de contener un entorno de la base de entornos, luego existe una familia finita $\{V_i\}_{i \in J}$ de entornos de $\varphi_i(x)$ en cada $Y_i$ con $i \in J$ de forma que:
                \begin{equation*}
                    W = \bigcap_{i \in J}\varphi_i^{-1}(V_i)
                \end{equation*}
                es un entorno básico contenido en $U$. Observemos que para cada $i \in J$ tenemos:
                \begin{equation*}
                    \left.\begin{array}{c}
                        \varphi_i(x) \in V_i \\
                        \{\varphi_i(x_n)\} \to \varphi_i(x)
                    \end{array}\right\} \Longrightarrow \exists N_i \in \mathbb{N} : \varphi_i(x_n)\in V_i \quad \forall n\geq N_i
                \end{equation*}
                Sin embargo, como $J$ es finito, podemos tomar $N = \max\limits_{i \in J}N_i$ y tendremos que:
                \begin{equation*}
                    n\geq N \Longrightarrow \varphi_i(x_n) \in V_i \qquad \forall i \in J
                \end{equation*}
                de donde:
                \begin{equation*}
                    x_n \in W = \bigcap_{i \in J}\varphi_i^{-1}(V_i) \subset U \quad \forall n\geq N
                \end{equation*}
                Por lo que $\{x_n\}\to x$.\qedhere
        \end{description}
    \end{proof}
\end{prop}

\noindent
Por lo que conociendo la convergencia de las sucesiones en los espacios $Y_i$, estudiar la convergencia de $X$ con la topología inicial se reduce a estudiar las convergencias de sus imágenes por $\varphi_i$, esto hace fácil trabajar con sucesiones en la topología inicial. Sin embargo, no todos los conceptos topológicos se pueden caracterizar por sucesiones.

Otra propiedad útil de las topologías iniciales es la siguiente:

\begin{prop}\label{prop:sii_debil}
    Sea $(X,\tau)$ con $\tau$ la topología inicial asociada a una familia de aplicaciones $\{\varphi_i\}_{i \in I}$. Si $Z$ es un espacio topológico y tenemos una aplicación entre espacios topológicos $\psi:Z\to X$, entonces:
    \begin{equation*}
        \psi \text{\ es continua} \Longleftrightarrow \varphi_i\circ \psi:Z\to Y_i \text{\ es continua}\quad \forall i \in I
    \end{equation*}
    \begin{proof}
        Por doble implicación:
        \begin{description}
            \item [$\Longrightarrow )$] Si $\psi:Z\to X$ es continua, $\tau$ hace que cada $\varphi_i$ sea continua, por lo que cada aplicación $\varphi_i\circ\psi$ es continua.
            \item [$\Longleftarrow )$] Para esta, tenemos que si $U\in \tau$, entonces podemos escribir:
                \begin{equation*}
                    U = \bigcup \bigcap_{J} \varphi_i^{-1}(\omega_i)
                \end{equation*}
                para ciertos conjuntos abiertos $\omega_i$ de $Y_i$, de donde:
                \begin{equation*}
                    \psi^{-1}(U) = \bigcup\bigcap_{J}\psi^{-1}(\varphi^{-1}_i(\omega_i)) = \bigcup\bigcap_J {(\varphi_i \circ \psi_i)}^{-1}(\omega_i)
                \end{equation*}
                como la intersección finita de abiertos en $Z$ es un abierto de $Z$ y la unión arbitraria de abiertos de $Z$ también lo es, tenemos que $\psi^{-1}(U)$ es abierto en $Z$, para cada $U\in \tau$, por lo que $\psi$ es continua.
        \end{description}
    \end{proof}
\end{prop}

\noindent
Y la idea es la misma de la Proposición anterior: aunque no conozcamos con exactitud los abiertos de la topología inicial, estudiar las funciones continuas $Z\to X$ se reduce al problema de estudiar la continuidad de cada una de las funciones resultantes tras componer con $\varphi_i$, obteniendo funciones $Z\to Y_i$. Este procedimiento hace que sea muy fácil comprobar qué aplicaciones $Z\to X$ son continuas.

\section{Topología débil}
\noindent
Sea $(E,\|\cdot \|_E)$ un espacio de Banach, tenemos ya sobre $E$ una topología, la asociada a la norma $\|\cdot \|_E$, que denotaremos a veces por $\tau_{\|\cdot \|_E}$. Definiremos sobre este espacio $E$ otra topología:

\begin{definicion}[Topología débil de un espacio normado]
    Sea $(E,\|\cdot \|_E)$ un espacio normado, definimos la topología débil en $E$ como la topología inicial en $E$ que hace que todas las aplicaciones de la familia $E^\ast$ sean continuas, y denotaremos a esta topología por $\sigma(E,E^\ast)$.
\end{definicion}

\begin{observacion}
    Observaciones que hay que tener en cuenta al trabajar con $\sigma(E,E^\ast)$:
    \begin{itemize}
        \item La notación $\sigma(E,E^\ast)$ hay que pensarla como la ``topología débil en $E$ es la topología inicial en $E$ que hace continuos todos aquellos elementos de $E^\ast$''.
        \item Tenemos $Y_f = \mathbb{R}$ para cada $f\in E^\ast$, donde tomamos como conjunto de índices $I = E^\ast$.
        \item Observemos que:
            \begin{equation*}
                \sigma(E,E^\ast) \subset \tau_{\|\cdot \|_E}
            \end{equation*}
            Ya que toda aplicación $f\in E^\ast$ es continua en $\tau_{\|\cdot \|_E}$ y $\sigma(E,E^\ast)$ es, por definición de topología inicial, la topología más pequeña que hace que las aplicaciones de $E^\ast$ sean continuas.
    \end{itemize}
\end{observacion}

\noindent
Destacaremos a continuación propiedades destacables de la topología débil de un espacio normado $E$, donde siempre que hagamos referencia a $\sigma(E,E^\ast)$, estaremos trabajando sobre un espacio normado $E$ arbitrario.

\begin{prop}
    $\sigma(E,E^\ast)$ es Hausdorff.
    \begin{proof}
        Sean $x_1,x_2\in E$ distintos, tomamos:
        \begin{equation*}
            A = \{x_1\}, \qquad B = \{x_2\}
        \end{equation*}
        que son dos conjuntos convexos, disjuntos y cerrados para $\tau_{\|\cdot \|_E}$, lo que nos permite aplicar la segunda versión geométrica del Teorema de Hahn-Banach (Teorema~\ref{teo:hahn-banach_2aversiongeometrica}), obteniendo $f\in E^\ast\setminus\{0\}$ y $\alpha\in \mathbb{R}$ de forma que:
        \begin{equation*}
            \langle f,x_1 \rangle  < \alpha < \langle f,x_2 \rangle 
        \end{equation*}
        de donde tomando:
        \begin{align*}
            x_1 &\in \Theta_1 := \{x\in E : \langle f,x \rangle <\alpha\} = f^{-1}(\left]-\infty,\alpha\right[) \in \sigma(E,E^\ast) \\
            x_2 &\in \Theta_2 := \{x\in E : \langle f,x \rangle >\alpha\} = f^{-1}(\left]\alpha,+\infty\right[) \in \sigma(E,E^\ast) 
        \end{align*}
        Tenemos que $\Theta_1,\Theta_2$ son disjuntos entre sí, con lo que nos dan la condición de Hausdorff que buscábamos.
    \end{proof}
\end{prop}

\noindent
Veamos ahora una base de entornos en $\sigma(E,E^\ast)$, aplicando el procedimiento que hicimos anteriormente al construir la topología inicial.

\begin{prop}\label{prop:bde_debil}
    Dado $x_0\in E$ y $f_1, \ldots, f_k\in E^\ast$, tenemos que:
    \begin{enumerate}
        \item $V = V(f_1, \ldots, f_k;\varepsilon) = \{x\in E : |\langle f_i,x-x_0 \rangle | < \varepsilon, \quad i \in \{1,\ldots, k\}\}$ es un entorno de $x_0$ en $\sigma(E,E^\ast)$, para todo $\varepsilon>0$.
        \item Además, $\cc{V} = \{V(f_1, \ldots, f_k;\varepsilon) : \varepsilon>0, \quad f_1,\ldots,f_k\in E^\ast\}$ es base de entornos de $x_0$ en $\sigma(E,E^\ast)$.
    \end{enumerate}
    \begin{proof}
        Veamos cada apartado:
        \begin{enumerate}
            \item Dado $\varepsilon>0$ y $x\in V(f_1,\ldots,f_k,\varepsilon)$, tenemos que:
                \begin{equation*}
                    |\langle f_i,x-x_0 \rangle | = |\langle f_i,x \rangle -\langle f_i,x_0 \rangle | < \varepsilon \Longleftrightarrow \langle f_i,x_0 \rangle -\varepsilon\leq \langle f_i,x \rangle \leq \langle f_i,x_0 \rangle  + \varepsilon
                \end{equation*}
                que a su vez equivale a:
                \begin{equation*}
                    x \in f_i^{-1}\left(\left]\langle f_i,x_0 \rangle -\varepsilon,\langle f_i,x_0 \rangle +\varepsilon\right[\right)
                \end{equation*}
                Si definimos:
                \begin{equation*}
                    a_i = \langle f_i ,x_0 \rangle  \qquad \forall i \in \{1,\ldots,k\}
                \end{equation*}
                podemos escribir:
                \begin{equation*}
                    V = \bigcap_{i = 1}^k f_i^{-1}(\left]a_i - \varepsilon,a_i + \varepsilon\right[)
                \end{equation*}
                como cada $f_i$ es continua para la topología débil, el conjunto $V$ ha de ser abierto para $\sigma(E,E^\ast)$, como intersección finita de conjuntos abiertos; y es claro que $x_0\in V$, por lo que $V$ es un entorno abierto de $x_0$.
            \item Para ver que es base de entornos, si tomamos $U$ un entorno abierto de $x_0$ en $\sigma(E,E^\ast)$, tenemos entonces que existe un entorno de la base de entornos de $\sigma(E,E^\ast)$, luego existen $f_1, \ldots, f_k\in E^\ast$ y $V_1,\ldots,V_k\subset \mathbb{R}$ entornos de su correspondiente punto $f_1(x_0),\ldots,f_k(x_0)$ de forma que:
                \begin{equation*}
                    \bigcap_{j=1}^k f_j^{-1}(V_j) \subset U
                \end{equation*}
                como cada $V_j$ es un entorno de $f_j(x_0)$ en la topología usual en $\mathbb{R}$ y tenemos una cantidad finita de ellos, ha de existir $\varepsilon>0$ de forma que:
                \begin{equation*}
                    \left]f_j(x_0)-\varepsilon,f_j(x_0)+\varepsilon\right[\subset V_j \qquad \forall j \in \{1,\ldots,k\}
                \end{equation*}

                de donde:
                \begin{equation*}
                    V(f_1, \ldots, f_k;\varepsilon) \subset \bigcap_{j = 1}^k f_j^{-1}(V_j) \subset U
                \end{equation*}
        \end{enumerate}
    \end{proof}
\end{prop}

\noindent
Esta proposición nos permite, tomado $x\in E$ y $U$ un entorno de $x$, han de existir $f_1, \ldots, f_k\in E^\ast$ y $\varepsilon>0$ de forma que:
\begin{equation*}
    x\in V(f_1, \ldots, f_k;\varepsilon)\subset U
\end{equation*}

\begin{ejercicio} % // TODO: PONER BIEN
    Probar que $dim E < \infty \Longrightarrow \sigma(E,E^\ast) = \tau_{\|\cdot \|_E}$.\\

    \begin{description}
        \item [$\subseteq )$] Tenemos que $\sigma(E,E^\ast)$ es la topología más pequeña que hace continuos todos aquellos elementos de $E^\ast$, y tenemos que $\tau_{\|\cdot \|_E}$ hace continuos todos aquellos elementos de $E^\ast$, por lo que se tiene esta inclusión.
        \item [$\supseteq )$] Podemos hacerla de dos formas distintas:
            \begin{description}
                \item [Opción 1.] En el caso de que $dim E$ es finita, si fijamos una base $\{x_1, \ldots, x_N\}$ tenemos que $\tau_{\|\cdot \|_E}$ es la topología inicial que hace continuas las aplicaciones $\{\pi_1, \ldots, \pi_N\}$, donde $\pi_i:E\to \mathbb{R}$ viene dada por:
                    \begin{equation*}
                        \pi_i\left(\sum_{k=1}^{N} \lm_k x_k\right) = \lm_i \qquad i \in \{1,\ldots,N\}
                    \end{equation*}
                    Como $\{\pi_1, \ldots, \pi_N\}\subset E^\ast$, tendremos que $\tau_{\|\cdot \|_E}\subset \sigma(E,E^\ast)$.
                \item [Opción 2.] Fijado $x_0\in E$, si $U$ es un entorno suyo en $\tau_{\|\cdot \|_E}$ tenemos entonces que existe $r>0$ de forma que $B(x_0,r)\subset U$. Fijada una base $\{e_1,\ldots,e_N\}$ de $E$ de vectores de norma 1, podemos considerar las aplicaciones $\pi_1,\ldots,\pi_N$, donde $\pi_i:E\to \mathbb{R}$ viene dada por:
                    \begin{equation*}
                        \pi_i\left(\sum_{k=1}^{N} \lm_k x_k\right) = \lm_i \qquad i \in \{1,\ldots,N\}
                    \end{equation*}
                    y observamos que:
                    \begin{equation*}
                        \|x-x_0\| = \left\|\sum_{k=1}^{N}\pi_i(x-x_0)\|e_k\|\right\| \leq \sum_{k=1}^{N} |\pi_i(x-x_0)| \qquad \forall x\in E
                    \end{equation*}
                    Por lo que si consideramos $x\in V(\pi_1,\ldots,\pi_N,\frac{r}{N})$ tendremos que:
                    \begin{equation*}
                        \|x-x_0\| \leq \sum_{k=1}^{N}|\pi_i(x-x_0)| < \sum_{k=1}^{N} \frac{r}{N} = r
                    \end{equation*}
                    de donde $x\in B(x_0,r)$, es decir:
                    \begin{equation*}
                        V\left(\pi_1,\ldots,\pi_N,\frac{r}{N}\right) \subset B(x_0,r) \subset U
                    \end{equation*}
                    Por lo que $U$ también es un entorno de $\sigma(E,E^\ast)$, lo que prueba que $\tau_{\|\cdot \|_E}\subset \sigma(E,E^\ast)$.
            \end{description}
    \end{description}
    Observemos que como $\sigma(E,E^\ast) = \tau_{\|\cdot \|_E}$ tendremos en particular que:
    \begin{equation*}
        \{x_n\} \stackrel{\sigma(E,E^\ast)}{\longrightarrow} x \Longleftrightarrow \{x_n\} \stackrel{\|\cdot \|_E}{\longrightarrow} x
    \end{equation*}
\end{ejercicio}

\noindent
Resumimos en la siguiente proposición cada una de las relaciones entre las convergencias de sucesiones en los distintos espacios topológicos que manejamos. Antes de ello, introducimos la siguiente notación:

\begin{notacion}
    Si $(E,\|\cdot \|_E)$ es un espacio normado, consideraremos sobre él habitualmente dos topologías posiblemente distintas (por lo que obtendremos distintas convergencias de sucesiones):
    \begin{equation*}
        \tau_{\|\cdot \|_E} \qquad \text{y}\qquad \sigma(E,E^\ast)
    \end{equation*}
    Si tenemos una sucesión $\{x_n\}$ de puntos de $E$ y un punto $x\in E$, será costumbre para nosotros:
    \begin{itemize}
        \item notar por ``$\{x_n\}\to x$'' si la sucesión $\{x_n\}$ es convergente en $\tau_{\|\cdot \|_E}$ al elemento $x$, diciendo en alguna ocasión que la sucesión $\{x_n\}$ ``converge'' o que ``converge fuertemente'' al elemento $x$.
        \item notar por ``$\{x_n\}\rightharpoonup x$'' si la sucesión $\{x_n\}$ es convergente en $\sigma(E,E^\ast)$ al elemento $x$, diciendo en alguna ocasión que la sucesión $\{x_n\}$ ``converge débilmente'' al elemento $x$.
    \end{itemize}
    Todavía no está del todo claro la relación entre estas dos convergencias distintas de sucesiones de puntos de $E$, que aclararemos en la siguiente Proposición, pero ya podremos hablar de convergencia de sucesiones de puntos de $E$ de forma cómoda, sin confundir en ningún momento la convergencia de $\sigma(E,E^\ast)$ con la de $\tau_{\|\cdot \|_E}$.
\end{notacion}

\begin{prop}\label{prop:convergencia_debil} % // TODO: MIrar si es de banach o normado
    Sea $E$ un espacio de Banach y $\{x_n\}$ una sucesión de puntos de $E$:
    \begin{enumerate}
        \item $\{x_n\} \rightharpoonup x \Longleftrightarrow \{\langle f,x_n \rangle \}\to \langle f,x \rangle \quad \forall f\in E^\ast$.
        \item $\{x_n\}\to x \Longrightarrow \{x_n\} \rightharpoonup x$.
        \item $\{x_n\}\rightharpoonup x \Longrightarrow \{\|x_n\|\}$ acotada y $\|x\|\leq \liminf \|x_n\|$.
        \item Tenemos:
            \begin{equation*}
                \left.\begin{array}{r}
                    \{x_n\}\rightharpoonup x \\
                    \{f_n\} \to f
                \end{array}\right\} \Longrightarrow \{\langle f_n,x_n \rangle \}\to \langle f,x \rangle 
            \end{equation*}
    \end{enumerate}
    \begin{proof}
        Demostramos cada una de las propiedades:
        \begin{enumerate}
            \item Es la Proposición~\ref{prop:top_inicial_suc} pero usando la notación para la topología débil de $E$.
            \item Si tenemos $\{x_n\} \to x$, entonces para $f\in E^\ast$:
                \begin{equation*}
                    |\langle f,x_n-x \rangle | \leq \|f\|\|x_n-x\|  \qquad \forall n\in \mathbb{N}
                \end{equation*}
                y como $\|x_n-x\| \to 0$, deducimos que $\langle f,x_n-x \rangle \to 0 $, luego tenemos que:
                \begin{equation*}
                    \{\langle f,x_n \rangle \}\to \langle f,x \rangle  \qquad \forall f\in E^\ast
                \end{equation*}
                y usando 1 tenemos que $\{x_n\}\rightharpoonup x$.
            \item Tomamos $B = \{x_n : n\in \mathbb{N}\}$, que verifica para $f\in E^\ast$:
                \begin{equation*}
                    f(B) = \{\langle f,x_n \rangle : n\in \mathbb{N}\}
                \end{equation*}
                Como $\{x_n\}\rightharpoonup x$, el apartado 1 nos dice que $f(B)$ es acotado, $\forall f\in E^\ast$. Por el Corolario~\ref{coro:entonces_B_acotado} deducimos que $B$ es acotado, es decir, que $\{\|x_n\|\}$ está acotada.

                Para la segunda parte, si tomamos $f\in E^\ast$, tenemos que:
                \begin{equation*}
                    |\langle f,x_n \rangle | \leq \|f\|\|x_n\| \qquad \forall n\in \mathbb{N}
                \end{equation*}
                si tomamos límite inferior:
                \begin{equation*}
                    \langle f,x \rangle = \lim_{n\to\infty}\langle f,x_n \rangle  = \liminf \langle f,x_n \rangle  \leq \|f\|\liminf \|x_n\| \qquad \forall f\in E^\ast
                \end{equation*}
                En particular, si tomamos $\|f\| = 1$, tenemos que:
                \begin{equation*}
                    \|x\| = \sup_{\|f\|\leq 1}\langle f,x \rangle  \leq \|f\| \liminf \|x_n\| = \liminf \|x_n\|
                \end{equation*}
            \item Estudiamos la diferencia:
                \begin{align*}
                    |\langle f_n,x_n \rangle  - \langle f,x \rangle | &\leq |\langle f_n-f,x_n \rangle | + |\langle f,x_n-x \rangle |  \\
                                                                      &\leq \|f_n-f\| \|x_n\| + |\langle f,x_n \rangle -\langle f,x \rangle | \qquad \forall n\in \mathbb{N}
                \end{align*}
                Y tenemos que $\|f_n-f\| \to 0$, que $\{\|x_n\|\}$ está acotada, y que $\langle f,x_n \rangle \to \langle f,x \rangle $, de donde deducimos que $|\langle f_n,x_n \rangle -\langle f,x \rangle |\to 0$.
        \end{enumerate}
    \end{proof}
\end{prop}

\noindent
Para entender mejor el punto 3 de esta Proposición, introducimos el siguiente concepto:
\begin{definicion}
    Sea $(E,\tau)$ un espcio topológico, sea $f:(E,\tau)\to \mathbb{R}$ una aplicación, decimos que la función $f$ es \underline{secuencialmente semicontinua inferiormente} si se cumple que:
    \begin{equation*}
        \{x_n\} \to x \Longrightarrow f(x) \leq \liminf f(x_n)
    \end{equation*}
\end{definicion}~\\

\noindent
Notemos que sabíamos que la aplicación 
\begin{equation*}
    \|\cdot \|:(E,\tau_{\|\cdot \|_E})\to \mathbb{R}
\end{equation*}
es continua. Sin embargo, en vista de la Proposición y la Definición anterior, sabemos que la aplicación 
\begin{equation*}
    \|\cdot \|:(E,\sigma(E,E^\ast))\to \mathbb{R}
\end{equation*}
es secuencialmente semicontinua inferiormente.\\

\noindent
Nos preguntamos ahora por el recícproco de la propiedad 3, si tenemos una sucesión $\{\|x_n\|\}$ acotada con $\{x_n\}$ una sucesión de puntos de $E$, ¿será cierto que $\{x_n\}\rightharpoonup x$? La respuesta a esta pregunta es rotundamente negativa, pues sabemos que en dimensión finita $\sigma(E,E^\ast) = \tau_{\|\cdot \|_E}$, y sabemos de la existencia de sucesiones acotadas que no son convergentes en cualquier espacio normado $N-$dimensional.

Sin embargo, si recordamos el Teorema de Bolzano-Weierstrass, en todo espacio normado $N-$dimensional siempre que teníamos una sucesión acotada podríamos extraer una parcial suya convergente. Veremos próximamente que una propiedad similar a esta se cumple en la topología débil de $E$, lo que nos permitirá llegar a un Teorema que relacione los conjuntos compactos de $\sigma(E,E^\ast)$ con los conjuntos cerrados y acotados, brindándonos un espacio topológico con una cantidad abundante de conjuntos compactos, cosa que no sucede en $\tau_{\|\cdot \|_E}$ cuando la dimensión del espacio $E$ no es finita.

Esta propiedad de $\sigma(E,E^\ast)$ es totalmente natural, pues al considerar como $\sigma(E,E^\ast)$ la menor topología sobre $E$ que hace que las aplicaciones de $E^\ast$ sean continuas lo que estamos haciendo es eliminar de $\tau_{\|\cdot \|_E}$ abiertos que no nos interesa considerar en ciertos momentos, haciendo más fácil que un conjunto sea compacto, pues cuantos menos abiertos contenga una topología más fácil será que un conjunto sea compacto, por la propia definición de conjunto compacto en un espacio topológico general.

\subsection{Cierre de la esfera}
\begin{notacion}
    Sea $E$ un espacio normado, denotaremos:
    \begin{equation*}
        B = B(0,1), \qquad \overline{B} = \overline{B}(0,1), \qquad S = S(0,1)
    \end{equation*}
    Además, llamaremos a los conjuntos abiertos de $\sigma(E,E^\ast)$ \underline{débilmente abiertos}, y análogamente débilmente cerrados a los conjuntos cerrados de $\sigma(E,E^\ast)$.
\end{notacion}

\begin{observacion}
    Si $E = \mathbb{R}^N$, tenemos que $\tau_{\|\cdot \|}$ y $\sigma(E,E^\ast)$ son iguales (con $\|\cdot \|$ cualquier norma en $\mathbb{R}^N$, puesto que todas son equivalentes). De esta forma, vemos que:
    \begin{equation*}
        \overline{S}^{\sigma(E,E^\ast)} = \overline{S} = S
    \end{equation*}
    Si ahora tenemos que $dim E = \infty$, nos va a interesar calcular también $\overline{S}^{\sigma(E,E^\ast)}$, y resulta que:
    \begin{equation*}
        \overline{S}^{\sigma(E,E^\ast)} = \overline{B}
    \end{equation*}
    Nos da un resultado muy sorprendente, pues no es nada intuitivo.
\end{observacion}

\begin{prop}\label{prop:cierre_esfera}
    Sea $E$ un espacio de Banach de dimensión infinita, se tiene que:
    \begin{equation*}
        \overline{S}^{\sigma(E,E^\ast)} = \overline{B}
    \end{equation*}
    \begin{proof}
        Por doble inclusión:
        \begin{description}
            \item [$\subseteq )$] Para esta inclusión, como tenemos que $S\subset \overline{B}$, basta probar que $\overline{B}$ es débilmente cerrado, con lo que:
                \begin{equation*}
                    \overline{S}^{\sigma(E,E^\ast)} \subset \overline{\left(\overline{B}\right)}^{\sigma(E,E^\ast)} = \overline{B}
                \end{equation*}
                Para ello, recordamos que:
                \begin{equation*}
                    \overline{B} = \{x\in E : \|x\|\leq 1\}
                \end{equation*}
                Y el Corolario~\ref{coro:calcular_norma_x} del Teorema de Hahn-Banach nos dice que:
                \begin{equation*}
                    \|x\| = \sup_{\|f\|\leq 1}|f(x)| \qquad \forall x\in E
                \end{equation*}
                Por lo que podemos escribir:
                \begin{align*} 
                    \overline{B} = \{x\in E : |f(x)| \leq 1 \quad \forall f\in E^\ast \text{\ con\ } \|f\|\leq 1\} &= \bigcap_{\substack{f\in E^\ast \\ \|f\|\leq 1}} \{x\in E : |f(x)|\leq 1\} \\
                    &= \bigcap_{\substack{f\in E^\ast \\ \|f\|\leq 1}} f^{-1}([-1,1])
                \end{align*}
                De donde deducimos que $\overline{B}$ es $\sigma(E,E^\ast)-$cerrado.
                
            \item [$\supseteq )$] Probemos primero que:
                \begin{equation*}
                    B\subset \overline{S}^{\sigma(E,E^\ast)}
                \end{equation*}
                Para ello, sea $x_0\in B$, si consideramos $V$ un entorno de $x_0$ en la topología débil, tenemos que existen $f_1,\ldots,f_k\in E^\ast$ y $\varepsilon>0$ de forma que:
                \begin{equation*}
                    V \supseteq  V(f_1, \ldots, f_k;\varepsilon) = \{x\in E : |\langle f_i,x-x_0 \rangle |<\varepsilon\quad i \in \{1,\ldots,k\}\}
                \end{equation*}
                La demostración termina probando que $V\cap S\neq \emptyset $. Para ello, definimos la función $\psi:E\to \mathbb{R}^k$ dada por:
                \begin{equation*}
                    \psi(x) = \langle \psi,x \rangle =   (\langle f_1,x \rangle , \langle f_2,x \rangle , \ldots, \langle f_k,x \rangle )
                \end{equation*}
                Tenemos que $\psi$ es lineal y que no puede ser inyectiva, ya que $dimE = \infty$ y tenemos que $dim \mathbb{R}^k = k$. Como $\psi$ no es inyectiva, tenemos que existe $y_0\in E\setminus \{0\}$ con $\langle \psi,y_0 \rangle = 0$.\\

                \noindent
                Consideramos ahora $g:\mathbb{R}\to \mathbb{R}$ dada por:
                \begin{equation*}
                    g(t) = \|x_0 + ty_0\| \qquad \forall t\in \mathbb{R}
                \end{equation*}
                que es una función continua, con $g(0) = \|x_0\| < 1$, así como que $\lim\limits_{t\to\infty}g(t) = \infty$, ya que:
                \begin{equation*}
                    \|x_0+ty_0\| \geq |\|x_0\| + |t|\|y_0\||
                \end{equation*}
                Por el Teorema de Bolzano, tenemos que existe $t_0\in \mathbb{R}^+$ de forma que $g(t_0) = 1$, es decir, $x_0+t_0y_0\in S$. Además, tenemos que $x_0+t_0y_0\in V(f_1, \ldots, f_k;\varepsilon)$, ya que:
                \begin{equation*}
                    |\langle f_i,x_0+t_0y_0-x_0 \rangle | = |\langle f_i,t_0y_0 \rangle | = |t_0\langle f_i,y_0 \rangle | \AstIg 0 < \varepsilon \qquad \forall i \in \{1,\ldots,k\}
                \end{equation*}
                donde en $(\ast)$ usamos que $y_0\in \ker\psi$. En definitiva, tenemos que $x_0+t_0y_0 \in S\cap V$, donde $V$ era un entorno arbitrario de $x_0$ para la topología débil. Hemos probado que $x_0\in \overline{S}^{\sigma(E,E^\ast)}$ para cada $x_0\in B$, por lo que tenemos la inclusión que queríamos:
                \begin{equation*}
                    B\subset \overline{S}^{\sigma(E,E^\ast)}
                \end{equation*}
                Por otra parte, es obvio que $S\subset \overline{S}^{\sigma(E,E^\ast)}$, por lo que:
                \begin{equation*}
                    \overline{B} = B\cup S \subset \overline{S}^{\sigma(E,E^\ast)}
                \end{equation*}
                \qedhere
        \end{description}
    \end{proof}
\end{prop}

\begin{observacion}
    Observemos que según la prueba anterior, la elección de $t_0$ no nos da la condición de $x_0+t_0y_0\in V(f_1,\ldots,f_k;\varepsilon)$, sino que la existencia de $y_0\in \ker\psi$ nos dice que:
    \begin{equation*}
        x_0 + ty_0 \in V(f_1,\ldots,f_k;\varepsilon) \qquad \forall t\in \mathbb{R}
    \end{equation*}
    Es decir, dicho entorno básico contiene a toda una recta afín de $E$, algo que no sigue la idea intuitiva de entorno básico.
\end{observacion}

\begin{coro}
    Como consecuencias a destacar, si $E$ es un espacio de Banach de dimensión infinita:
    \begin{itemize}
        \item $S$ no es débilmente cerrado.
        \item $B$ no es débilmente abierta.
    \end{itemize}
    \begin{proof}
        Para la segunda, supongamos por reducción al absurdo que $B$ fuera $\sigma(E,E^\ast)-$abierta, con lo que $E\setminus B$ es $\sigma(E,E^\ast)-$cerrado, con lo que el conjunto
        \begin{equation*}
            S = (E\setminus B) \cap \overline{B} 
        \end{equation*}
        es $\sigma(E,E^\ast)-$cerrado, lo que contradice el primer punto, que viene de la Proposición anterior.
    \end{proof}
\end{coro}

\subsection{Relación entre débilmente cerrados y cerrados}
\noindent
Sea $E$ un espacio de Banach: 
\begin{itemize}
    \item la Proposición~\ref{prop:cierre_esfera} nos dice que los cerrados de $E$ no son necesariamente débilmente cerrados.
    \item como $\sigma(E,E^\ast)\subset \tau_{\|\cdot \|}$, tenemos que todo conjunto débilmente cerrado tambien será cerrado.
\end{itemize}

\noindent
Buscamos ahora una condición sencilla que podemos añadir a los conjuntos cerrados para que siempre sean también débilmente cerrados. Para ello:

\begin{teo}
    Sea $E$ un espacio de Banach y $A\subset E$ un conjunto convexo, entonces:
    \begin{equation*}
        A \text{\ es\ } \sigma(E,E^\ast)-\text{cerrado} \Longleftrightarrow A\text{\ es cerrado}
    \end{equation*}
    \begin{proof}
        Por doble implicación:
        \begin{description}
            \item [$\Longrightarrow )$] La hemos discutido anteriormente, pues se tiene que:
                \begin{equation*}
                    \sigma(E,E^\ast)\subset \tau_{\|\cdot \|}
                \end{equation*}
            \item [$\Longleftarrow )$]  Si $A$ es un subconjunto de $E$ que es cerrado y convexo, queremos ver que es $\sigma(E,E^\ast)-$cerrado. Para ello, veamos que $E\setminus A$ es débilmente abierto. Para esto último, si tomamos $x_0\in E\setminus A$ tenemos por la segunda versión geométrica del Teorema de Hahn-Banach ($\{x_0\}$ es compacto y $A$ cerrado con $x_0\notin A$) que existen $f\in E^\ast$  y $\alpha\in \mathbb{R}$ tales que:
                \begin{equation*}
                    \langle f,x_0 \rangle < \alpha < \langle f,x \rangle  \qquad \forall x\in A
                \end{equation*}
                Tenemos por tanto que:
                \begin{equation*}
                    x_0 \in \{y\in E : \langle f,y \rangle <\alpha\} = f^{-1}(\left]-\infty,\alpha\right[)
                \end{equation*}
                con $f^{-1}(\left]-\infty,\alpha\right[)$ un conjunto $\sigma(E,E^\ast)-$abierto\footnote{También podríamos haber dicho que $f^{-1}(\left]-\infty,\alpha\right[) = V(f;\alpha)$.}, por la definición de $\sigma(E,E^\ast)$. Como $f^{-1}(\left]-\infty,\alpha\right[)\cap A = \emptyset $, tenemos que:
                \begin{equation*}
                    x_0 \in f^{-1}(\left]-\infty,\alpha\right[) \subset E\setminus A
                \end{equation*}
                Como $x_0$ era un punto de $E\setminus A$ arbitrario, tenemos que $E\setminus A$ es $\sigma(E,E^\ast)-$abierto, lo que concluye la demostración.\qedhere
        \end{description}
    \end{proof}
\end{teo}

\begin{observacion}
    Por tanto, para decir que un conjunto abierto es abierto en la topología débil basta ver que su complementario es convexo.
\end{observacion}

\begin{coro}[Teorema de Mazur]
    Supongamos que $\{x_n\}$ es una sucesión de puntos de $E$ débilmente convergente a $x\in E$, $\{x_n\}\rightharpoonup x$. Entonces existe una sucesión $\{y_n\}$ de puntos de $E$ tal que:
    \begin{enumerate}
        \item $\forall n\in \mathbb{N}$, $y_n$ es una combinación convexa finita de $\{x_k:k\in \mathbb{N}\}$.
        \item $\{y_n\}\to x$.
    \end{enumerate}
\end{coro}

\subsubsection{Envolvente convexa de un conjunto}
\noindent
Para realizar su demostración, conviene tener claros ciertos conceptos:
\begin{definicion}
    Sea $E$ un espacio vectorial, si $\emptyset \neq X\subset E$ definimos la envolvente convexa de $X$ como:
    \begin{equation*}
        \conv(X) = \bigcap \{C\subset E : C\text{\ es convexo y\ }X\subset C\}
    \end{equation*}
    Este subconjunto de $E$ verifica ser el menor conjunto convexo que contiene a $X$.
\end{definicion}

\begin{prop}
    Sea $E$ un espacio vectorial y $\emptyset \neq X\subset E$, tenemos que:
    \begin{equation*}
        \conv(X) = \left\{\sum_{i \in I}t_ix_i : 
            \begin{array}{l}
                I \text{\ es finito} \\
                t_i \in \mathbb{R}^+_0, x_i \in X \quad \forall i \in I \\
                \sum\limits_{i \in I}t_i = 1
            \end{array}\right\}
    \end{equation*}
    \begin{proof}
        Por doble inclusión y llamando $Y$ al conjunto de la derecha:
        \begin{description} % // TODO: Se puede facilitar la prueba suponiendo que los dos elementos son combinación convexa de los mismos vectores xi, tomando los alphai y betai = 0 cuando haga falta
            \item [$\subseteq )$] Para esta inclusión, veamos que $Y$ es convexo. Para ello, si $\sum\limits_{i \in I}t_ix_i, \sum\limits_{j \in J}\lm_jx_j \in Y$ y $t\in [0,1]$ observamos que:
                \begin{align*}
                    t\sum_{i \in I}t_ix_i + (1-t)\sum_{j\in J}\lm_jx_j &= \sum_{i \in I}tt_ix_i + \sum_{j \in J}(1-t)\lm_jx_j \\
                                                                       &\AstIg \sum_{i \in I\setminus J} tt_ix_i + \sum_{j \in J\setminus I}(1-t)\lm_jx_j + \sum_{k \in I\cap J}(tt_k + (1-t)\lm_k)x_k
                \end{align*}
                donde en $(\ast)$ hemos agrupado por cada $x_k$, si definimos:
                \begin{equation*}
                    \alpha_i = \left\{\begin{array}{ll}
                        tt_i + (1-t)\lm_i & \text{si\ } i \in I\cap J  \\
                         tt_i & \text{si\ } i \in I\setminus J \\
                         (1-t)\lm_i & \text{si\ } i \in J\setminus I
                    \end{array}\right. 
                \end{equation*}
                tenemos que:
                \begin{equation*}
                    \sum_{i \in I\setminus J} tt_ix_i + \sum_{j \in J\setminus I}(1-t)\lm_jx_j + \sum_{k \in I\cap J}(tt_k + (1-t)\lm_k)x_k = \sum_{i \in I\cup J}\alpha_i x_i
                \end{equation*}
                con $I\cup J$ finito y además:
                \begin{align*}
                    \sum_{i \in I\cup J}\alpha_i &= \sum_{i \in I\setminus J}tt_i + \sum_{j \in J\setminus I}(1-t)\lm_j + \sum_{k \in I\cap J}(tt_k + (1-t)\lm_k) \\
                                                 &= \sum_{i \in I} tt_i + \sum_{j \in J}(1-t)\lm_j = t\sum_{i \in I} t_i + (1-t)\sum_{j \in J}\lm_j \\ 
                                                 &\AstIg t + (1-t)  = 1
                \end{align*}
                donde en $(\ast)$ hemos usado que $\sum\limits_{i \in I}t_i = 1 = \sum\limits_{j\in J}\lm_j$.
            \item [$\supseteq )$] Sea $\sum\limits_{i \in I}t_ix_i \in Y$, veamos que $\sum\limits_{i \in I}t_ix_i \in \conv(X)$, por inducción sobre $|I| = n$:
                \begin{itemize}
                    \item Para $n=1$ tenemos $tx$ con $t=1$, $x\in X$, luego $x = tx \in X\subset \conv(X)$.
                    \item Para $n=2$ tenemos que $I = \{p,q\}$:
                        \begin{equation*}
                            \sum_{i \in I}t_ix_i = t_px_p + t_qx_q \qquad \text{con}\qquad t_p+t_q = 1 \Longrightarrow t_q=1-t_p
                        \end{equation*}
                        de donde $\sum\limits_{i \in I}t_ix_i$ es combinación convexa de dos elementos de $X$, por lo que ha de ser $\sum\limits_{i \in I}t_ix_i \in \conv(X)$.
                    \item Supuesto para $n=m-1$, veámoslo si $|I| = m$. Tomando $J = I\setminus \{k\}$ con $k\in I$, vemos que:
                        \begin{equation*}
                            \sum_{i \in I}t_ix_i = t_kx_k + \sum_{j \in J}t_jx_j = t_kx_k + (1-t_k)\sum_{j \in J}\frac{t_j}{(1-t_k)}x_j
                        \end{equation*}
                        Observemos que: 
                        \begin{equation*}
                            t_k + \sum_{j \in J}t_j = \sum_{i \in I}x_i = 1 \Longrightarrow \sum_{j \in J}t_j = 1-t_k
                        \end{equation*}
                        Por lo que:
                        \begin{equation*}
                            \sum_{j \in J}\frac{t_j}{(1-t_k)} = \frac{1}{1-t_k}\sum_{j \in J} = 1
                        \end{equation*}
                        Por hipótesis de inducción (recordemos que $|J| = m-1$) tenemos que $\sum\limits_{j \in J}\frac{t_j}{1-t_k}x_j\in \conv(X)$, por lo que hemos probado que $\sum\limits_{i \in I}t_ix_i$ es conbinación convexa de dos elementos de $\conv(X)$, por lo que tiene que estar en $\conv(X)$.\qedhere
                \end{itemize}
        \end{description}
    \end{proof}
\end{prop}

Estamos ya en condiciones de probar el Corolario anterior:
\begin{coro}[Teorema de Mazur]
    Supongamos que $\{x_n\}$ es una sucesión de puntos de $E$ débilmente convergente a $x\in E$, $\{x_n\}\rightharpoonup x$. Entonces existe una sucesión $\{y_n\}$ de puntos de $E$ tal que:
    \begin{enumerate}
        \item $\forall n\in \mathbb{N}$, $y_n$ es una combinación convexa finita de $\{x_k:k\in \mathbb{N}\}$.
        \item $\{y_n\}\to x$.
    \end{enumerate}
    \begin{proof}
        Llamando $C = \conv(\{x_k : k\in \mathbb{N}\})$, si $\{x_n\}\rightharpoonup x$ tenemos entonces que $x\in \overline{\{x_k : k\in \mathbb{N}\}}^{\sigma(E,E^\ast)}$, luego:
        \begin{equation*}
            x\in \overline{\{x_k : k\in \mathbb{N}\}}^{\sigma(E,E^\ast)}\subset \overline{C}^{\sigma(E,E^\ast)}
        \end{equation*}
        tenemos que $\overline{C}^{\sigma(E,E^\ast)}$ es un conjunto débilmente cerrado y convexo (ya que la clausura de un conjunto convexo sigue siendo un conjunto convexo), por lo que por el Teorema anterior ha de ser $\overline{C}^{\sigma(E,E^\ast)}$ un conjunto cerrado para la norma de $E$. En estas condiciones:
        \begin{itemize}
            \item Como $\overline{C}^{\sigma(E,E^\ast)}$ es un cerrado que contiene a $C$, ha de ser $\overline{C}\subset \overline{C}^{\sigma(E,E^\ast)}$.
            \item Como tenemos también que $C\subset \overline{C}$, tomando cierre débil a ambos lados obtenemos:
                \begin{equation*}
                    \overline{C}^{\sigma(E,E^\ast)} \subset \overline{(\overline{C})}^{\sigma(E,E^\ast)}
                \end{equation*}
                Pero como $\overline{C}$ es un conjunto cerrado y convexo, es débilmente cerrado, por lo que:
                \begin{equation*}
                    \overline{C}^{\sigma(E,E^\ast)} \subset \overline{(\overline{C})}^{\sigma(E,E^\ast)} = \overline{C}
                \end{equation*}
        \end{itemize}
        En definitiva, $\overline{C} = \overline{C}^{\sigma(E,E^\ast)}$ y tenemos que $x\in \overline{C}$, por lo que existe una sucesión $\{y_n\}$ de puntos de $C$ de forma que $\{y_n\}\to x$.
    \end{proof}
\end{coro}

\begin{teo}
    Sean $E$, $F$ dos espacios de Banach y $T:E\to F$ lineal, entonces equivalen: 
    \begin{enumerate}
        \item $T:(E,\tau_{\|\cdot \|_E})\to (F,\tau_{\|\cdot \|_F})$ es continua.
        \item $T:(E,\tau_{\|\cdot \|_E})\to (F,\sigma(F,F^\ast))$ es continua.
        \item $T:(E,\sigma(E,E^\ast)) \to (F,\sigma(F,F^\ast))$ es continua.
    \end{enumerate}
    \begin{proof}
        Demostremos todas las implicaciones: 
        \begin{description}
            \item [$1\Longrightarrow 2)$] Para probar que $T:(E,\tau_{\|\cdot \|_E})\to (F,\sigma(F,F^\ast))$ es continua:
                \begin{description}
                    \item [Opción 1.] Si tomamos un abierto $U\in \sigma(F,F^\ast)\subset \tau_{\|\cdot \|_F}$ tendremos entonces que $T^{-1}(U)\in \tau_{\|\cdot \|_E}$.
                    \item [Opción 2.] Si $\{x_n\}\to x$, la primera condición nos dice que $\{f(x_n)\}\to f(x)$, de donde $\{f(x_n)\}\rightharpoonup f(x)$.
                    \item [Opción 3.] Si tomamos cualquier $f\in F^\ast$, tenemos que $f:(F,\tau_{\|\cdot \|_F})\to \mathbb{R}$ es continua, por lo que $f\circ T:(E,\tau_{\|\cdot \|_E})\to \mathbb{R}$ será también continua, y podemos aplicar la Proposición~\ref{prop:sii_debil} 
                \end{description}
            \item [$2\Longrightarrow 3)$] Para esta implicación:
                \begin{description}
                    \item [Opción 1.] Sea $f\in F^\ast$, tenemos que:
                        \begin{equation*}
                            f\circ T:(E,\tau_{\|\cdot \|_E})\to (\mathbb{R},\tau_u) \in E^\ast
                        \end{equation*}
                        Por lo que $f\circ T:(E,\sigma(E,E^\ast))\to (\mathbb{R},\tau_u)$ ha de ser continua, para toda $f\in F^\ast$, de donde $T:(E,\sigma(E,E^\ast))\to (F,\sigma(F,F^\ast))$ es continua.
                    \item [Opción 2.] Si $U\in \sigma(F,F^\ast)$, tenemos entonces que:
                        \begin{equation*}
                            U = \bigcup_{arb}\bigcap_{fin}f^{-1}(\omega)
                        \end{equation*}
                        con $f\in F^\ast$, $\omega$ abierto de $\mathbb{R}$, de donde:
                        \begin{equation*}
                            T^{-1}(U) = \bigcup_{arb}\bigcap_{fin}T^{-1}(f^{-1}(\omega)) = \bigcup_{arb}\bigcap_{fin}{(f\circ T)}^{-1}(\omega)
                        \end{equation*}
                        Como $f\circ T\in E^\ast$, tenemos por definición de topología débil que:
                        \begin{equation*}
                            T^{-1}(U) = \bigcup_{arb}\bigcap_{fin}{(f\circ T)}^{-1}(\omega) \in \sigma(E,E^\ast)
                        \end{equation*}
                    \item [Opción 3.] $T:(E,\tau_{\|\cdot \|_E})\to (F,\sigma(F,F^\ast))$ es continua si y solo si la aplicación $f\circ T:(E,\tau_{\|\cdot \|_E})\to \mathbb{R}$ es contiua para toda $f\in F^\ast$, si y solo si $f\circ T\in E^\ast$ para toda $f\in F^\ast$, si y solo si (por la definición de topología débil) $f\circ T:(E,\sigma(E,E^\ast))\to\mathbb{R}$ es continua para toda $f\in F^\ast$, que ocurre si y solo si $T:(E,\sigma(E,E^\ast))\to(F,\sigma(F,F^\ast))$ es continua.
                \end{description}
            \item [$3\Longrightarrow 1)$] Para esta última implicación:
                \begin{description}
                    \item [Opción 1.] Si tomamos una sucesión de puntos de la gráfica convergente, $\{(x_n,T(x_n))\}\to (x,y)$, tenemos:
                        \begin{equation*}
                            \{x_n\} \rightharpoonup x \qquad \{T(x_n)\}\rightharpoonup y
                        \end{equation*}
                        la primera implica $\{T(x_n)\}\rightharpoonup T(x)$, de donde $T(x) = y$ (ya que $\sigma(F,F^\ast)$ es Hausdorff), por lo que $(x,y)\in Gr(T)$, de donde la gráfica es cerrada en $(E\times F,\tau_E \times \tau_F)$. Como $E$ y $F$ son Banach, tenemos por el Teorema de la aplicación cerrada que $T$ es continua.
                    \item [Opción 2.] Si consideramos:
                        \begin{equation*}
                            Gr(T) = \{(x,Tx) : x\in E\}\subset E\times F
                        \end{equation*}
                        Si $T$ es continua, entonces $Gr(T)$ es cerrado en la topología producto $(E\times F, \sigma(E,E^\ast)\times \sigma(F,F^\ast))$. Puede probarse que: % // TODO: EJERCICIO
                        \begin{equation*}
                            \sigma(E,E^\ast)\times \sigma(F,F^\ast) = \sigma(E\times F, {(E\times F)}^{\ast})
                        \end{equation*}
                        Finalmente hay que probar que $Gr T$ es cerrado en $(E\times F, \tau_{\|\cdot \|_{E\times F}})$, donde consideramos por ejemplo la norma:
                        \begin{equation*}
                            \|(x,y)\|_{E\times F} = \|x\|_E + \|y\|_F
                        \end{equation*}
                        De donde $Gr(T)$ es cerrado en $\tau_{\|\cdot \|_{E\times F}}$, y por el Teorema de la Gráfica cerrada se llega a que $T$ es continua.
                \end{description}
        \end{description}
    \end{proof}
\end{teo}

\section{Topología débil$-\ast$}
\noindent
Sea $E$ un espacio normado, tenemos que $E^\ast$ es de Banach, con lo que podemos considerar su topología débil $\sigma(E^\ast,E^{\ast\ast})$. Recordemos que siempre tenemos una inyección canónica
\Func{J}{E}{E^{\ast\ast}}{x}{J(x)}

donde teníamos:
\Func{J(x)}{E^\ast}{\bb{R}}{f}{\langle f,x\rangle}

\begin{definicion}
    Sea $E$ un espacio normado, si consideramos como $J$ la inyección canónica en su bidual, la topología débil$-\ast$ de $E$ será $\sigma(E^\ast,J(E))$, es decir, la topología inicial sobre $E^\ast$ que hace continuos todos aquellos elementos de $J(E)\subset E^{\ast\ast}$.
\end{definicion}

\begin{observacion}
    Sea $E$ un espacio normado, si consideramos la topología débil de su dual, $\sigma(E^\ast, E^{\ast\ast})$ tenemos que esta es la topología inicial sobre $E^\ast$ que hace continuos todos aquellos elementos de $E^{\ast\ast}$. Como $J(E)\subset E^{\ast\ast}$, tendremos pues que:
    \begin{equation*}
        \sigma(E^\ast,J(E)) \subset \sigma(E^\ast,E^{\ast\ast})
    \end{equation*}
    Por lo que la toplogía débil$-\ast$ de $E$ está contenida en la topología débil de $E^\ast$.
\end{observacion}

\begin{notacion}
    A veces notaremos:
    \begin{equation*}
        \sigma(E^\ast,E) = \sigma(E^\ast,J(E))
    \end{equation*}
    Pensando en identificar $J(E)$ con $E$ dentro de $E^{\ast\ast}$, puesto que $J$ es inyectiva y preserva la norma.
\end{notacion}

\begin{prop}
    $\sigma(E^\ast,J(E))$ es Hausdorff.
    \begin{proof}
        Sean $f_1,f_2\in E^\ast$ distintos, entonces existe $x\in E$ de forma que $\langle f_1,x \rangle \neq \langle f_2,x \rangle $. Podemos suponer sin pérdida de generalidad que:
        \begin{equation*}
            \langle f_1,x \rangle < \langle f_2,x \rangle 
        \end{equation*}
        Con lo que existe $\alpha\in \mathbb{R}$ de forma que:
        \begin{equation*}
            \langle f_1,x \rangle < \alpha < \langle f_2,x \rangle 
        \end{equation*}
        Tomando ahora:
        \begin{align*}
            O_1 &= \{f\in E^\ast : \langle f,x \rangle < \alpha\} = \{f\in E^\ast : J(x)(f)<\alpha\} = {J(x)}^{-1}(\left]-\infty,\alpha\right[) \in \sigma(E^\ast,J(E)) \\
            O_2 &= \{f\in E^\ast : \langle f,x \rangle > \alpha\} = \{f\in E^\ast : J(x)(f)>\alpha\} = {J(x)}^{-1}(\left]\alpha,+\infty\right[) \in \sigma(E^\ast,J(E)) 
        \end{align*}
        Tenemos que $f_1\in O_1$, $f_2\in O_2$ con $O_1\cap O_2 = \emptyset $, por lo que $\sigma(E^\ast,J(E))$ es Hausdorff.
    \end{proof}
\end{prop}

\begin{prop}
    Dada $f_0\in E^\ast$ y $x_1,\ldots,x_k\in E$, tenemos que:
    \begin{enumerate}
        \item $V=V(x_1,\ldots,x_k;\varepsilon) = \{f\in E^\ast : |\langle f-f_0,x_i \rangle |<\varepsilon,\quad i \in \{1,\ldots,k\}\}$ es un entorno de $f_0$ en $\sigma(E^\ast,J(E))$, para todo $\varepsilon>0$.
        \item Además $\cc{V} = \{V(x_1,\ldots,x_k;\varepsilon) : \varepsilon>0, x_1, \ldots, x_k\in E\}$ es una base de entornos de $f_0$ en $\sigma(E^\ast,J(E))$.
    \end{enumerate}
    \begin{proof}
        Es análoga a la de la Proposición~\ref{prop:bde_debil}:
        \begin{enumerate}
            \item Para cada $i \in \{1,\ldots,k\}$, defino $a_i = \langle f_0,x_i \rangle $. Vemos que:
                \begin{equation*}
                    |\langle f,x_i \rangle - a_i | = |\langle f,x_i \rangle - \langle f_0,x_i \rangle  | = |\langle f-f_0,x_i \rangle | < \varepsilon
                \end{equation*}
                Por lo que:
                \begin{equation*}
                    V= \bigcap_{i=1}^k {J(x_i)}^{-1}\left(\left]a_i-\varepsilon,a_i+\varepsilon\right[\right)
                \end{equation*}
                Con lo que $V$ es abierto en $\sigma(E^\ast,J(E))$, por intersección finita de abiertos.
            \item Sea $U$ un entorno de $f_0$ en $\sigma(E^\ast,J(E))$, tenemos que existe $k\in \mathbb{N}$ de forma que $f_0\subset W \subset U$, donde:
                \begin{equation*}
                    W = \bigcap_{i=1}^k {J(x_i)}^{-1}(\omega_i)
                \end{equation*}
                donde cada $\omega_i$ es un abierto de $\mathbb{R}$ que contiene a $a_i$. Podemos tomar $\varepsilon>0$ de forma que $\left]a_i-\varepsilon,a_i+\varepsilon\right[\subset \omega_i\quad \forall \varepsilon\in \{1,\ldots,k\}$. Si tomamos ahora:
                \begin{equation*}
                    V = V(x_1,\ldots,x_k;\varepsilon)
                \end{equation*}
                Tenemos entonces que $f_0\in V\subset W\subset U$.
        \end{enumerate}
    \end{proof}
\end{prop}

\begin{notacion}
    Si tenemos una sucesión de elementos de $E^\ast$ que converge a $f$ en la topología $\sigma(E^\ast,J(E))$, escribiremos:
    \begin{equation*}
        \{f_n\}\stackrel{\ast}{\rightharpoonup} f
    \end{equation*}
\end{notacion}

\begin{prop}
    Sea $E$ un espacio de Banach y $\{f_n\}$ una sucesión de elementos de $E^\ast$:
    \begin{enumerate}
        \item $\{f_n\}\stackrel{\ast}{\rightharpoonup} f\Longleftrightarrow \{f_n(x)\}\to f(x)\quad \forall x\in E$.
        \item $\{f_n\}\to f \Longrightarrow \{f_n\}\rightharpoonup f \Longrightarrow \{f_n\}\stackrel{\ast}{\rightharpoonup}f $.
        \item $\{f_n\}\stackrel{\ast}{\rightharpoonup} f\Longrightarrow \{\|f_n\|\}$ acotada y $\|f\|\leq \liminf\|f_n\|$.
        \item Tenemos:
            \begin{equation*}
                \left.\begin{array}{l}
                    \{f_n\}\stackrel{\ast}{\rightharpoonup} f \\
                    \{x_n\}\rightarrow x
                \end{array}\right\} \Longrightarrow \{\langle f_n,x_n \rangle \}\to \langle f,x \rangle 
            \end{equation*}
    \end{enumerate}
    \begin{proof}
        Es análoga a la de la Proposición~\ref{prop:convergencia_debil}:
        \begin{enumerate}
            \item Es caso particular de la Proposición~\ref{prop:sii_debil}.
            \item Sabemos otra vez por la Proposición~\ref{prop:sii_debil} que:
                \begin{equation*}
                    \{f_n\}\rightharpoonup f \Longleftrightarrow \{\varphi(f_n)\}\to \varphi(f) \quad \forall \varphi \in E^{\ast\ast}
                \end{equation*}
                En particular, tendremos que $\{Jx(f)\}\to Jx(f)\quad \forall x\in E$ que equivale, nuevamente por la Proposición anterior, a que $\{f_n\}\stackrel{\ast}{\rightharpoonup} f$.
            \item Sabemos por 1 que $\{f_n(x)\}\to f(x)$ para todo $x\in E$, por lo que en particular la sucesión $\{f_n(x)\}$ está acotada, es decir, el conjunto:
                \begin{equation*}
                    \{f_n(x) : n\in \mathbb{N}\}
                \end{equation*}
                está acotado $\forall n\in \mathbb{N}$. Usando el Corolario~\ref{coro:entonces_Bast_acotado} tenemos entonces que $\{f_n\}$ está acotada, es decir, que $\{\|f_n\|\}$ está acotada. Por otra parte:
                \begin{equation*}
                    |f_n(x)| \leq \|f_n\|\|x\| \qquad \forall x\in E, \quad \forall n\in \mathbb{N}
                \end{equation*}
                Como $\{f_n\}$ está acotada, podemos tomar límite inferior a ambos lados, obteniendo que:
                \begin{equation*}
                    |f(x)| \leq \liminf\|f_n\| \|x\| \qquad \forall x\in E
                \end{equation*}
                Esto prueba que $f$ es continua y que $\|f\|\leq \liminf\|f_n\|$.
            \item Tenemos para todo $n\in \mathbb{N}$ que:
                \begin{align*}
                    |f_n(x_n) - f(x)| &= |f_n(x_n) - f_n(x) + f_n(x) - f(x)| \leq |f_n(x-x_n)| + |(f_n-f)(x)| \\ &\leq \|f_n\|\|x-x_n\| + |(f_n-f)(x)|
                \end{align*}
                Por el apartado 1 tenemos que $(f_n-f)(x)$ tiende a 0, y por el apartado 3 tenemos que $\|f_n\|$ está acotada. Finalmente, como $\{x_n\}\to x$, hemos probado que $\{f_n(x_n)\}\to f(x)$. \qedhere
        \end{enumerate}
    \end{proof}
\end{prop}

\noindent
Nos preguntamos ahora por las aplicaciones $\varphi:E^\ast\to \mathbb{R}$ lineales que son $\sigma(E^\ast,E)-$continuos.

\begin{ejercicio}\label{ej:lineal_debil*} % // TODO: HACER
    Sea $X$ un espacio vectorial y $\varphi_1, \ldots, \varphi_n,\varphi:X\to \mathbb{R}$ lineales. Si se cumple que: 
    \begin{equation*}
        \varphi_i(v) = 0 \quad \forall  i \in \{1,\ldots, n\} \Longrightarrow \varphi(v) = 0
    \end{equation*}
    Entonces, existen $\lm_1, \ldots, \lm_n\in \mathbb{R}$ tales que:
    \begin{equation*}
        \varphi = \sum_{i=1}^{n} \lm_i\varphi_i
    \end{equation*}
    \begin{proof}
        Obsevemos que
        \begin{equation*}
            \varphi_i(v) = 0 \quad \forall  i \in \{1,\ldots, n\} \Longleftrightarrow v\in \bigcap_{i=1}^n \ker\varphi_i
        \end{equation*}
        y que:
        \begin{equation*}
            \varphi(v) = 0 \Longleftrightarrow v\in \ker\varphi
        \end{equation*}
        Por lo que la conclusión es equivalente a que $\ker\varphi\subset \bigcap\limits_{i=1}^n \ker\varphi_i$.
    \end{proof}
\end{ejercicio}

\begin{prop}
    Sea $E$ de Banach, si consideramos $\varphi:E^\ast\to \mathbb{R}$ lineal:
    \begin{equation*}
        \text{Si\ } \varphi \text{\ es\ } \sigma(E^\ast,E)-\text{continua} \Longrightarrow \exists x\in E : J(x) = \varphi
    \end{equation*}
    la condición $J(x) = \varphi$ es equivalente a que:
    \begin{equation*}
        \langle \varphi,f \rangle  = \langle f,x \rangle  \qquad \forall f\in E^\ast
    \end{equation*}
    \begin{proof}
        Que $\varphi$ sea continua significa que existe $V$ un entorno de $0$ en $\sigma(E^\ast,E)$ de forma que:
        \begin{equation*}
            |\varphi(f)| \leq M \qquad \forall f\in V
        \end{equation*}
        Tenemos que entonces:
        \begin{equation*}
            \exists V(0;x_1, \ldots, x_n;\varepsilon) = \{f\in E^\ast : |\langle f,x_i \rangle |<\varepsilon\quad \forall i \in \{1,\ldots,n\}\} \subset V
        \end{equation*}
        Observemos que si $f\in E^\ast$ verifica:
        \begin{equation*}
            \langle f,x_i \rangle  = 0 \qquad \forall i \in \{1,\ldots,n\}
        \end{equation*}
        tenemos entonces que $\lm f \in V(0;x_1, \ldots, x_n;\varepsilon)\subset V$, por lo que:
        \begin{equation*}
            |\lm| |\varphi(f)| = |\varphi(\lm f)| \leq M \qquad \forall \lm \in \mathbb{R}
        \end{equation*}
        de donde $\varphi(f) = 0$.\\

        \noindent
        Si definimos $\varphi_i:E^\ast \to \mathbb{R}$ dadas por:
        \begin{equation*}
            \varphi_i(f) = \langle f,x_i \rangle \qquad \forall i \in \{1,\ldots,n\}
        \end{equation*}
        tenemos por el Ejercicio~\ref{ej:lineal_debil*} que existen $\lm_1, \ldots, \lm_n\in \mathbb{R}$ tales que:
        \begin{equation*}
            \varphi = \sum_{i=1}^{n}\lm_i \varphi_i
        \end{equation*}
        Y observamos que:
        \begin{equation*}
            \varphi(f) = \sum_{k=1}^{n}\lm_i \varphi_i(f) = \sum_{i=1}^{n}\lm_i \left\langle f,x_i \right\rangle  = \left\langle f,\sum_{i=1}^{n}\lm_i x_i \right\rangle  = \langle f,x \rangle  
        \end{equation*}
    \end{proof}
\end{prop}

\noindent
Por lo que las únicas aplicaciones lineales que son $\sigma(E^\ast,E)-$continuas son las que ya sabíamos.\\

\noindent
Si nos preguntamos ahora por los hiperplanos que son cerrados para $\sigma(E^\ast,E)$:

\begin{coro}
    Si $H$ es un hiperplano en $E^\ast$ que es $\sigma(E^\ast,E)-$cerrado, entonces existe $x\in E$ de forma que:
    \begin{equation*}
        H = \{f\in E^\ast : f(x) = \alpha\}
    \end{equation*}
    \begin{proof}
        Si $H$ es un hiperplano en $E^\ast$, existen entonces $0\neq \varphi:E^\ast\to \mathbb{R}$ lineal y $\alpha\in \mathbb{R}$ de forma que:
        \begin{equation*}
            H = \{f\in E^\ast : \varphi(f) = \alpha \}
        \end{equation*}
        Como $H$ es débil$-\ast$ cerrado, veamos a continuación que $\varphi$ ha de ser $\sigma(E^\ast,E)-$continua. En dicho caso, tendremos por la Proposición anterior que existirá $x\in E$ de forma que:
        \begin{equation*}
            H = \{f\in E^\ast : f(x) = \alpha\}
        \end{equation*}
        Veamos entonces que $\varphi$ es $\sigma(E^\ast,E)-$continua, sabiendo que $H$ es débil$-\ast$ cerrado. Por el contrarrecíproco, supongamos que $\varphi$ no es $\sigma(E^\ast,E)-$continua. Como $\varphi$ es lineal, no será continua en $0$, por lo que podemos suponer que existe una parcial de cierta sucesión $\{f_n\}$ de puntos de $E^\ast$ de forma que para cierto $\varepsilon_0>0$ se tiene que $\{f_n\}\stackrel{\ast}{\rightharpoonup} 0$ y $|\varphi(f_n)| \geq \varepsilon_0$.\\

        \noindent
        Por tanto tenemos que $\varphi(f_0)\neq 0$. Podemos tomar por tanto $\lm$ para que se cumpla:
        \begin{equation*}
            \frac{f_n}{\varphi(f_n)} - \lm \frac{f_0}{\varphi(f_0)} \in H \qquad \forall n\in \mathbb{N}
        \end{equation*}

        Ya que:
        \begin{equation*}
            \varphi\left(\frac{f_n}{\varphi(f_n)} - \lm \frac{f_0}{\varphi(f_0)}\right) = \frac{1}{\varphi(f_n)}\varphi(f_n) - \frac{\lm}{\varphi(f_0)}\varphi(f_0) = 1 - \lm 
        \end{equation*}
        Por lo que tomando $\lm = 1-\alpha$ se tiene que dicho elemento está en $H$. Nos preguntamos ahora por la convergencia de dicha cantidad:
        \begin{itemize}
            \item Tenemos que $\frac{f_n}{\varphi(f_n)}\to 0$.
            \item La segunda cantidad es constante
        \end{itemize}
        Por lo que:
        \begin{equation*}
            \left\{\frac{f_n}{\varphi(f_n)} - \lm \frac{f_0}{\varphi(f_0)}\right\}  \to -(1-\alpha)\frac{f_0}{\varphi(f_0)}
        \end{equation*}
        Y tenemos que:
        \begin{equation*}
            \varphi\left(-(1-\alpha)\frac{f_0}{\varphi(f_0)}\right) = \frac{-(1-\alpha)}{\varphi(f_0)} \varphi(f_0) = \alpha - 1 \neq \alpha
        \end{equation*}
        Por lo que tenemos que:
        \begin{equation*}
            0 - \frac{(1-\alpha)}{\varphi(f_0)} f_0 \notin H
        \end{equation*}
        Por lo que $H$ no es débil$-\ast$ cerrado, lo que prueba el contrarrecíproco de la afirmación que queríamos probar.
    \end{proof}
\end{coro}

\noindent
Si buscamos ahora la relación entre $\sigma(E^\ast,E)$ y $\sigma(E^\ast,E^{\ast\ast})$. Tenemos que:
\begin{equation*}
    \sigma(E^\ast,E) \subset \sigma(E^\ast,E^{\ast\ast})
\end{equation*}

\begin{ejercicio} % // TODO: HACER
    Si $E$ no es reflexivo, entones:
    \begin{equation*}
        \sigma(E^\ast,E) \subsetneq \sigma(E^\ast,E^{\ast\ast})
    \end{equation*}~\\

    \noindent % // TODO: Lo hizo Jorge
    Si $E$ no es reflexivo tenemos entonces que $J(E)\subsetneq E^{\ast\ast}$, por lo que existe $\xi \in E^{\ast\ast}\setminus J(E)$, y tomamos:
    \begin{equation*}
        H = \{f\in E^\ast : \xi(f) = 0\} = \ker \xi
    \end{equation*}
    de donde $H = \xi^{-1}(\{0\})$ que es $\sigma(E^\ast,E^{\ast\ast})-$cerrado, puesto que $\xi \in E^{\ast\ast}$. Además, no existe $x\in E$ tal que $f(x) = 0$, porque $\xi \notin J(E)$.
\end{ejercicio} % Buscar un hiperplano cerrado en la derecha que no sea cerrado en la izquierda, o buscar uan aplicación E*--> R que sea continua en la derecha y que no lo sea en la izquierda.

\section{Teorema de Banach-Alaoglu-Bourbaki}
\noindent
Consideraremos para nuestro espacio $E$:
\begin{equation*}
    \mathbb{R}^E = \{w:E\to \mathbb{R}\} = \left\{w=\{w_x\}_{x\in E}\right\}
\end{equation*}
Que es un espacio vectorial.\\ % // TODO: HACER

\noindent
Para definir una topología sobre $\mathbb{R}^E$, como queremos que la convergencia en la topología de $\mathbb{R}^E$ equivalga a la convergencia en cada una de las proyecciones, tomamos lo topología inicial de la familia de proyecciones:
\Func{\pi_x}{\bb{R}^E}{\bb{R}}{w}{w_x}
para cada $x\in E$.\\

\noindent
Usaremos el Teorema de Tijonov\footnote{Que también es equivalente al axioma de elección.}:
\begin{teo}[de Tijonov]
    El producto arbitrario de compactos es compacto.
\end{teo}

\begin{teo}[de Banach-Alaoglu-Bourbaki]
    \begin{equation*}
        \overline{B}_{E^\ast} = \{f\in E^\ast : \|f\|\leq 1 \} \text{\ es débil$-\ast$ compacta}
    \end{equation*}
    \begin{proof} % // TODO: REVISAR ESTA DEMO
        Es trivial que $E^\ast\subset \mathbb{R}^E$. Consideramos en $\mathbb{R}^E$ la topología $\tau$, la topología inicial que hace continuas todas las proyecciones. De esta forma, cada vez que tengamos una aplicación $\varphi$ que llegue a $(\mathbb{R}^E,\tau)$, tenemos que comprobar que la composición $\pi_x\circ \varphi$ (que llegan a $\mathbb{R}$) sea continua. Usaremos que:
        \begin{equation*}
            (E^\ast,\sigma(E^\ast,E)) \stackrel{\Phi}{\to} (\mathbb{R}^E, \tau) \text{\ es continua} \Longleftrightarrow (E^\ast,\sigma(E^\ast,E)) \stackrel{\pi_x\circ\Phi}{\longrightarrow}\mathbb{R} \text{\ es continua} \quad \forall x\in E
        \end{equation*}
        Si consideramos
        \Func{\Phi}{E^\ast}{\bb{R}^E}{f}{f}
        Por definición de $\sigma(E^\ast,E)$ tenemos que $\phi_x\circ \Phi = J(x)$ es continua, por lo que $\Phi$ es continua.\\ 

        \noindent
        Tenemos que $\Phi$ es claramente inyectiva y es sobreyectiva, puesto que $\Phi(E^\ast) = E^\ast$.
        \begin{equation*}
            (E^\ast,\tau\big|_{E^\ast}) \stackrel{\Phi^{-1}}{\longrightarrow}(E^\ast,\sigma(E^\ast,E))
        \end{equation*}
        Para ver que $\Phi^{-1}$ sea continua, es equivalente a ver que:
        \begin{equation*}
            (E^\ast,\tau\big|_{E^\ast})\stackrel{J(x)\circ \Phi^{-1}}{\longrightarrow}\mathbb{R}
        \end{equation*}
        es continua, para cada $x\in E$. Sin embargo, como cada una de estas es continua por hipótesis, tenemos que $\Phi^{-1}$ es continua. En definitiva:
        \begin{equation*}
            (E^\ast,\sigma(E^\ast,E)) \stackrel{\Phi}{\to}(E^\ast,\tau\big|_{E^\ast})
        \end{equation*}
        es un homeomorfismo.\\

        \noindent
        Bajo estas condiciones (tenemos además que $\|f\| = 1$), veamos quién es $\Phi(\overline{B}_{E^\ast})$:
        \begin{equation*}
            K = \Phi(\overline{B}_{E^\ast}) = \left\{w\in \mathbb{R}^E : 
                \begin{array}{l}
                    |w_x|\leq \|x\| \\
                    w_{x+y} = w_x + w_y \\
                    w_{\lm x} = \lm w_x
                \end{array}
            \qquad \forall x,y\in E, \quad \forall \lm\in \mathbb{R} \right\}
        \end{equation*}
        Veamos que $K$ es compacto, pues si consideramos:
        \begin{align*}
            K_1 &= \{w\in \mathbb{R}^E : |w_x|\leq \|x\| \qquad \forall x\in E\} \\
            K_2 &= \{w\in \mathbb{R}^E : w_{x+y} = w_x + w_y \qquad \forall x,y\in E\} \\
            K_3 &= \{w\in \mathbb{R}^E : w_{\lm x} = \lm w_x \forall x,y\in E\} \\
        \end{align*}
        Tenemos que $K = K_1 \cap K_2 \cap K_3$.
        \begin{itemize}
            \item Vemos fácilmente que:
                \begin{equation*}
                    K_2 = \{w\in \mathbb{R}^E : w_{x+y} - w_x - w_y= 0 \qquad \forall x,y\in E\} 
                \end{equation*}
                Que claramente es un conjunto cerrado.
            \item Con la misma idea:
                \begin{equation*}
                    K_3 = \{w\in \mathbb{R}^E : w_{\lm x} - \lm w_x= 0 \qquad \forall x,y\in E\} 
                \end{equation*}
        \end{itemize}
        Podemos reescribir $K_1$ como:
        \begin{align*}
            K_1 = \{w\in \mathbb{R}^E : -\|x\| \leq w_x \leq \|x\| \qquad \forall x\in E\}= {\left[-\|x\|, \|x\|\right]}^{E}
        \end{align*}
        Como $[-\|x\|, \|x\|]$ es compacto para todo $x\in E$, por el Teorema de Tijonov tenemos que $K_1$ es compacto, como producto de compactos. Finalmente, tenemos que $K$ es un subconjunto cerrado de $K_1$, que es compacto, por lo que $K$ es compacto.
    \end{proof}
\end{teo}

\begin{teo}
    \begin{equation*}
        \text{Si\ } E \text{\ es un espacio de Banach reflexivo, entonces\ } \overline{B}_{E^\ast} \text{\ es débil-compacto}
    \end{equation*}
    \begin{proof}
        Sabemos que $J:E\to E^{\ast\ast}$ es lineal, inyectiva y continua, por lo que por el Teorema de la aplicación abierta restringiendo $J$ a su imagen, tenemos que $J$ es un embebimiento. Si $E$ es reflexivo, tenemos de hecho que $J$ es un homemomorfismo, pues $J(E) = E^{\ast\ast}$. Como $J$ es una isometría, es claro que:
        \begin{equation*}
            J(\overline{B}_{E}) = \overline{B}_{E^{\ast\ast}} = \overline{B}_{{(E^\ast)}^{\ast}}
        \end{equation*}
        Hemos visto en el teorema anterior que $\overline{B}_{{(E^\ast)}^{\ast}}$ es $\sigma(E^{\ast\ast},E^\ast)-$compacta. Dado cualquier $f\in E^\ast$, si consideramos:
        \Func{f\circ J^{-1}}{(E^{\ast\ast}, \sigma(E^{\ast\ast}, E^\ast))}{\bb{R}}{\xi}{\xi(f)}
        Tenemos que es continua, para cada $f\in E^\ast$, por lo que la aplicación:
        \begin{equation*}
            (E^{\ast\ast},\sigma(E^{\ast\ast},E^\ast)) \stackrel{J^{-1}}{\longrightarrow} (E,\sigma(E,E^\ast))
        \end{equation*}
        es continua. Tenemos que $\overline{B}_E$ es compacto.
    \end{proof}
\end{teo}

\noindent
La afirmación recíproca es también cierta\footnote{Su demostracion excede los conocimientos del curso, aunque el resultado se usará.}, y se conoce como Teorema de Kakutani.

\begin{prop}
    Sea $E$ un espacio de Banach reflexivo, si $M$ es un subespacio vectorial cerrado de $E$ entonces $M$ es reflexivo.
    \begin{proof}
        Tenemos que $M$ es un espacio de Banach, por ser cerrado. Para ver que $M$ es reflexivo, veamos que:
        \begin{equation*}
            \overline{B}_M = \{x\in M : \|x\|\leq 1\}
        \end{equation*}
        es $\sigma(M,M^\ast)-$compacto, o equivalentemente, que es $\sigma(E,E^\ast)-$compacto.

        Es sencillo probar que el Teorema de Hahn-Banach nos da la igualdad: % // TODO: HACER
        \begin{equation*}
            \sigma(E,E^\ast)\big|_{M} = \sigma(M,M^\ast)
        \end{equation*}
        Como $M$ es cerrado en $E$, entonces el conjunto $\overline{B}_M$ es también cerrado en $E$, así como que está dentro de un conjunto convexo, por lo que $\overline{B}_M$ es $\sigma(E,E^\ast)-$cerrado, y tenemos que $\overline{B}_M\subset \overline{B}_E$, y esta última es $\sigma(E,E^\ast)-$compacto, porque $E$ es reflexivo. Como $\overline{B}_M$ es $\sigma(E,E^\ast)-$cerrado dentro de un $\sigma(E,E^\ast)-$compacto, tenemos que es $\sigma(E,E^\ast)-$compacto.
    \end{proof}
\end{prop}

\begin{coro}\label{coro:reflexivo_sii}
    Sea $E$ un espacio de Banach:
    \begin{center}
        $E$ es reflexivo $\Longleftrightarrow E^\ast$ es reflexivo
    \end{center}
    \begin{proof}
        Por doble implicación:
        \begin{description}
            \item [$\Longrightarrow) $] Tenemos que la aplicación
                \Func{J}{E}{E^{\ast\ast}}{x}{J(x)}
                donde $J(x)(f) = f(x)$ para cada $f\in E^\ast$ es sobreyectiva.

                \noindent
                Queremos llegar a ver que $E^\ast$ es reflexivo, es decir, que su aplicación
                \Func{J}{E^\ast}{(E^\ast)^{\ast\ast}}{f}{J(f)}
                donde $J(f)(\xi) = \xi(f)$ para cada $\xi \in E^{\ast\ast}$, es sobreyectiva. Para ello, sea $\varphi \in {(E^\ast)}^{\ast\ast}$, definimos $f:E\to \mathbb{R}$ por:
                \begin{equation*}
                    f(x) = \varphi(J(x)) \qquad \forall x\in E
                \end{equation*}
                Y la demostración terminará viendo que $J(f) = \varphi$. % // TODO: HACER
            \item [$\Longleftarrow) $] A partir de la otra implicación, tenemos que $E^{\ast\ast} = {(E^\ast)}^{\ast}$ es reflexivo, y teníamos que la aplicación $J:E\to E^{\ast\ast}$ era una inyección, por lo que $J(E)$ es un subespacio vectorial de $E^{\ast\ast}$. Además, $J(E)$ es cerrado por ser de Banach, puesto que si $\{J(x_n)\}\to y$ con $J(x_n)\in J(E)\quad \forall n\in \mathbb{N}$ tendremos que (como $J$ es una isometría) $\{x_n\}$ es de Cauchy en $E$. Como $E$ es completo, existe $x\in E$ con $\{x_n\}\to x$, de donde ha de ser $y=J(x)$.

                \noindent
                En definitiva, la Proposición anterior nos dice que $J(E)$ es reflexivo, y como $J$ es una isometría, tendremos que $E$ es reflexivo. % // TODO: HACER
        \end{description}
    \end{proof}
\end{coro}

\noindent
El siguiente resultado es muy útil a la hora de trabajar con análisis funcional aplicado:

\begin{coro}
    Sea $E$ un espacio de Banach reflexivo y $K\subset E$ un conjunto acotado, convexo y cerrado, entonces $K$ es $\sigma(E,E^\ast)-$compacto.
    \begin{proof}
        Si $K$ es convexo y cerrado entonces $K$ es $\sigma(E,E^\ast)-$cerrado. Como $K$ es acotado, podemos encontrar cierto radio $R\in \mathbb{R}^+$ de forma que $K\subset R\overline{B}_E$. Como $E$ es reflexivo, tenemos que $\overline{B}_E$ es $\sigma(E,E^\ast)-$compacto, luego $R\overline{B}_E$ seguirá siendo $\sigma(E,E^\ast)-$compacto. Como $K$ es $\sigma(E,E^\ast)-cerrado$ tenemos automáticamente que $K$ es $\sigma(E,E^\ast)-$compacto.
    \end{proof}
\end{coro}

\noindent
Observemos que podemos sustituir las hipótesis de ``convexo y cerrado'' por que $K$ sea $\sigma(E,E^\ast)-$cerrado, pues en realidad en la demostración del Corolario solo usamos esta propiedad.

\begin{coro}
    Toda sucesión acotada de puntos de $E$ con $E$ un espacio de Banach reflexivo admite una parcial débilmente convergente. % // TODO: HACER: se podrá hacer al terminar espacios reflexivos, mover de sitio
\end{coro}

% // TODO: familiarizar espacios reflexivos
\begin{ejercicio}
    Los espacios:
    \begin{equation*}
        \mathbb{R}^N,\quad  H \text{\ Hilbert},\quad  l^p~(1<p<\infty), \quad L^p~(1<p<\infty)
    \end{equation*}
    son reflexivos.
    \begin{itemize}
        \item $C(K)$ no es reflexivo, con $K$ compacto.
        \item ¿$l^1$ o $l^\infty$ son reflexivos?
        \item ¿$L^1$ o $L^\infty$ son reflexivos?
    \end{itemize}
\end{ejercicio}

% // TODO: Pensar justificacion R^N
% // TODO: Escribir H reflexivo, es fácil
% // TODO: HACER l^p como ejercicio, salen en el cap 11
% Para toda f del dual de l^p existe una unica x en l^q tal que f(x) = suma x_k y_k para todo x en l^p, Teorema de Riesz-Fréchet para l^p
% // TODO: Pensar que C(K) no es reflexivo

\section{Espacios separables}
\begin{definicion}[Espacio separable]
    Sea $E$ un espacio métrico, se dice que es separable si y solo si existe $D\subset E$ denso y numerable.
\end{definicion}

\begin{ejercicio} % // TODO: HACER?
    Estudiar la separabilidad de los siguientes conjuntos:
    \begin{itemize}
        \item $l^p$ es separable para $1\leq p < \infty$.
        \item $L^p$ es separable para $1\leq p < \infty$. % // TODO: Se hará
        \item $l^\infty$ no es separable.
        \item $L^\infty$ no es separable.
    \end{itemize}
\end{ejercicio}

\begin{prop} % // TODO: Arrelgar ejercicio de cap 1 que me decia tomar {xn}
    Sea $E$ es un espacio métrico separable y $F\subset E$, entonces $F$ es separable.
    \begin{proof}
        Si $E$ es separable ha de existir $D\subset E$ denso y numerable, por lo que podemos escribirlo como:
        \begin{equation*}
            D = \{x_n : n\in \mathbb{N}\}
        \end{equation*}
        Como $D$ es denso, para todo $x\in F$, y para todo $m\in \mathbb{N}$ existe $n_0\in \mathbb{N}$ de forma que:
        \begin{equation*}
            d(x_{n_0} , x) < \frac{1}{m} 
        \end{equation*}
        O equivalentemente, que $x\in B\left(x_{n_0},\frac{1}{m}\right)\cap F$, por lo que dicha intersección es no vacía. Para cada $m,n\in \mathbb{N}$ consideramos por tanto: % // TODO: quizas definir antes de tomar x en F
        \begin{equation*}
            B\left(x_n,\frac{1}{m}\right)\cap F
        \end{equation*}
        Si la intersección no es vacía, existe $a_{m,n}\in B\left(x_n,\frac{1}{m}\right)\cap F$. Tomamos:
        \begin{equation*}
            C = \left\{a_{m,n} : B\left(x_n,\frac{1}{m}\right)\cap F \neq \emptyset \right\} \subset F
        \end{equation*}
        Y $C$ es claramente numerable. Calculamos ahora:
        \begin{equation*}
            d(x, a_{m,n_{0}}) \leq d(x,x_0) + d(x_{n_0}, a_{m,n_0}) \leq \frac{1}{m}+ \frac{1}{m} = \frac{2}{m}
        \end{equation*}
        De aquí deducimos que $C$ es denso en $F$.
    \end{proof}
\end{prop}

\begin{teo}
    Sea $E$ un espacio de Banach con $E^\ast$ separable, entonces $E$ es separable.
    \begin{proof}
        Como $E^\ast$ es separable, entonces existe $\{f_n:n\in \mathbb{N}\}\subset E^\ast$ denso. Recordemos que para cada $n\in \mathbb{N}$ tenemos:
        \begin{equation*}
            \|f_n\| = \sup_{\|x\|=1} f_n(x) \geq \frac{1}{2}\sup_{\|x\| = 1}f_n(x)
        \end{equation*}
        Existe por tanto $x_n\in E$ con $\|x_n\| = 1$ de forma que:
        \begin{equation*}
            \frac{1}{2}\|f_n\| \leq f_n(x_n) \leq \|f_n\|
        \end{equation*}
        Hay que tener en cuenta que si $\|f_n\| = 0$ es trivial y si no entonces la desigualdad anterior es estricta, lo que nos permite tomar $x_n$. Consideramos ahora el subespacio vectorial que genera $\{x_n\}_{n\in \mathbb{N}}$:
        \begin{equation*}
            L = \left\{\sum_{k=1}^{n}\alpha_k x_k : \alpha_1, \ldots, \alpha_n \in \mathbb{R}, n\in \mathbb{N}\right\}
        \end{equation*}
        Y también:
        \begin{equation*}
            L_0 = \left\{\sum_{k=1}^{n}r_k x_k : r_1, \ldots, r_n \in \mathbb{Q}, n\in \mathbb{N}\right\}
        \end{equation*}
        Que es numerable, como unión numerable de conjuntos numerables:
        \begin{equation*}
            L_0 = \bigcup_{n\in \mathbb{N}} \left\{\sum_{k=1}^{n}r_k x_k : r_1,\ldots, r_n\in \mathbb{Q}\right\}
        \end{equation*}
        Es cierto además que $L_0$ es denso en $L$. Si probamos que $L$ es denso en $E$ tendremos entonces que $L_0$ es denso en $E$. Con vistas a usar el Corolario~\ref{coro:comprobar_denso}, tomamos $f\in E^\ast$ con $f\big|_L = 0$. Dado $\varepsilon>0$, existe $n\in \mathbb{N}$ de forma que:
        \begin{equation*}
            \|f_n-f\| < \varepsilon
        \end{equation*}
        Y tenemos entonces que:
        \begin{equation*}
            \frac{1}{2}\|f_n\| \leq \langle f_n,x_n \rangle  \AstIg \langle f_n - f,x_n \rangle \leq \|f_n-f\| \cancelto{1}{\|x_n\|} <\varepsilon
        \end{equation*}
        donde en $(\ast)$ usamos que $f(x_n) = 0$. Si observamos ahora que:
        \begin{equation*}
            \|f\| \leq \|f-f_n\| + \|f_n\| < \varepsilon + 2\varepsilon = 3\varepsilon \qquad \forall \varepsilon>0
        \end{equation*}
        Por lo que $\|f\| = 0$, de donde $f=0$, por lo que aplicando el Corolario~\ref{coro:comprobar_denso} tenemos que $L$ es denso en $E$.
    \end{proof}
\end{teo}

\noindent
El recíproco es falso: $l^\infty$ no es separable y $l^1$ sí, siendo ${(l^1)}^{\ast} = l^\infty$.

\begin{coro}
    Si $E$ es de Banach:
    \begin{center}
        $E$ es reflexivo y separable $\Longleftrightarrow E^\ast$ es reflexivo y separable
    \end{center}
    \begin{description}
        \item [$\Longleftarrow )$] Si $E^\ast$ es separable, entonces $E$ es separable por el últimio Teorema, y si $E^\ast$ es reflexivo, entonces $E$ será también reflexivo, por el Corolario~\ref{coro:reflexivo_sii}.
        \item [$\Longrightarrow )$] Si $E$ es reflexivo y separable, tenemos entonces que $J(E) = E^{\ast\ast}$ con $J$ una isometría, por lo que $E^{\ast\ast}$ es reflexivo y separable. Como ${(E^\ast)}^{\ast} = E^{\ast\ast}$, la implicación anterior nos dice que $E^\ast$ es reflexivo y separable.
    \end{description}
\end{coro}

\begin{teo}
    Sea $E$ un espacio de Banach:
    \begin{center}
        $E$ es separable $ \Longleftrightarrow \left(\overline{B}_{E^\ast}, \sigma(E^\ast,E)\big|_{\overline{B}_{E^\ast}}\right)$ es metrizable.
    \end{center}
    \begin{proof}
        Demostramos la doble implicación:
        \begin{enumerate}
            \item[$\Longrightarrow )$] Como $E$ es separable, será también $\overline{B}_{E}$ un conjunto separable, luego existe $\{x_n:n\in \mathbb{N}\}\subset \overline{B}_E$ denso. Definimos ahora para cada $f\in E^\ast$:
                \begin{equation*}
                    [f] = \sum_{n=1}^{\infty} \frac{1}{2^n} |\langle f,x_n \rangle |
                \end{equation*}
                La aplicación está bien definida, puesto que:
                \begin{equation*}
                    \frac{1}{2^n}|\langle f,x_n \rangle| \leq \frac{1}{2^n}\|f\| \|x_n\| \leq \frac{1}{2^n}\|f\| \qquad \forall n\in \mathbb{N}
                \end{equation*} 
                de donde:
                \begin{equation*}
                    [f] = \sum_{n=1}^{\infty}\frac{1}{2^n}|\langle f,x_n \rangle | \leq \sum_{n=1}^{\infty}\frac{1}{2^n}\|f\| = \|f\| \sum_{n=1}^{\infty}\frac{1}{2^n} = \|f\|
                \end{equation*}
                Por lo que el límite que consideramos a la hora de definir $[f]$ tiene sentido. En particular, tenemos que $[f]\leq \|f\|$. Vemos ahora que:
                \begin{itemize}
                    \item $[f]$ es una norma:
                        \begin{itemize}
                            \item Si $f,g\in E^\ast$ tenemos entonces que:
                                \begin{align*}
                                    [f+g] &= \sum_{n=1}^{\infty}\frac{1}{2^n}|\langle f+g,x_n \rangle|  \leq \sum_{n=1}^{\infty}\frac{1}{2^n}\left(|\langle f,x_n \rangle| +|\langle g,x_n \rangle |\right) \\
                                          &= \sum_{n=1}^{\infty}\frac{1}{2^n}|\langle f,x_n \rangle | + \sum_{n=1}^{\infty}\frac{1}{2^n}|\langle g,x_n \rangle | = [f] + [g]
                                \end{align*}
                            \item Si $f\in E^\ast$ y considero $\lm\in \mathbb{R}$, tenemos que:
                                \begin{equation*}
                                    [\lm f] = \sum_{n=1}^{\infty}\frac{1}{2^n}|\langle \lm f,x_n \rangle | = \sum_{n=1}^{\infty}\frac{1}{2^n}|\lm| |\langle f,x_n \rangle | = |\lm|\sum_{n=1}^{\infty}\frac{1}{2^n}|\langle f,x_n \rangle | =  |\lm|[f]
                                \end{equation*}
                            \item Si tenemos $f\in E^\ast$ tal que:
                                \begin{equation*}
                                    0 = [f] = \sum_{n=1}^{\infty}\frac{1}{2^n}|\langle f,x_n \rangle |
                                \end{equation*}
                                Tendremos entones que $\langle f,x_n \rangle =0\quad \forall n\in \mathbb{N}$. Si tomamos ahora $x\in \overline{B}_E$, como $\{x_n:n\in \mathbb{N}\}$ es un conjunto denso, podemos tomar una sucesión de dichos puntos $\{y_n\}$ convergente a $x$. De esta forma, como $f$ es continua:
                                \begin{equation*}
                                    \{0\} = \{f(y_n)\} \to f(x)
                                \end{equation*}
                                Por lo que ha de ser $f(x) = 0$ para todo punto $x\in \overline{B}_E$.
                        \end{itemize}
                \end{itemize}
                Consideraremos el espacio métrico inducido por el espacio normado:
                \begin{equation*}
                    d(f,g) = [f-g]
                \end{equation*}
                Observemos que $[f] \leq \|f\|$. Probaremos ahora que
                \begin{equation*}
                    (\overline{B}_{E^\ast}, \tau_{d}) = \left(\overline{B}_{E^\ast}, \sigma(E^\ast,E)\big|_{\overline{B}_{E^\ast}}\right)
                \end{equation*}
                donde $\tau_d$ es la topología asociada a $d$ inducida en $\overline{B}_{E^\ast}$:
                    \begin{description}
                        \item [$\supseteq )$] Sea $f_0\in \overline{B}_{E^\ast}$ y $U$ un entorno de $f_0$ en $\sigma(E^\ast,E)$, existe $V$ un entorno básico (de la base concreta) de $f_0$ que está contenido en $U$, por lo que existirán $y_1,\ldots,y_k\in E$ y $\varepsilon>0$ de manera que:
                            \begin{equation*}
                                V = \{f\in E^\ast : |\langle f-f_0,y_i \rangle |<\varepsilon\quad \forall i \in \{1,...,k\}\} \subset U
                            \end{equation*}
                            Además, podemos multiplicar cada $y_i$ por una constante (ajustando luego $\varepsilon$), por lo que podemos suponer sin pérdida de generalidad que $\|y_i\|\leq 1 \quad \forall i \in \{1,\ldots,k\}$.\\

                            \noindent
                            Queremos probar que $U\cap \overline{B}_{E^\ast}$ contiene un abierto de $\tau_d$ que contenga a $f_0$ o equivalentemente, que existe $r\in \mathbb{R}^+$ de forma que:
                            \begin{equation*}
                                \{f\in E^\ast : d(f,f_0)<r\} \cap \overline{B}_{E^\ast} \subset V\cap \overline{B}_{E^\ast} \subset U\cap \overline{B}_{E^\ast}
                            \end{equation*}
                            Para ello, como $y_i \in \overline{B}_E$ para cada $i \in \{1,\ldots,k\}$, podemos tomar para cada $i \in \{1,\ldots,k\}$ cierto elemento $x_{n_i}$ de forma que:
                            \begin{equation*}
                                \|y_i - x_{n_i}\| < \frac{\varepsilon}{4}
                            \end{equation*}
                            y tomamos $0<r<\min\limits_{i \in \{1,\ldots,k\}} \left\{\dfrac{\varepsilon}{2^{1+n_{i}}} \right\}$. Veamos ahora que este $r$ nos sirve para la condición que buscábamos: sea $f\in \overline{B}_{E^\ast}$ de forma que $d(f,f_0)<r$, tenemos entonces que:
                            \begin{equation*}
                                \frac{1}{2^{n_i}}|\langle f-f_0,x_{n_i} \rangle | \leq \sum_{n=1}^{\infty}\frac{1}{2^n}|\langle f-f_0,x_n \rangle | < r \qquad \forall i \in \{1,\ldots,k\}
                            \end{equation*}
                            de donde para cada $i \in \{1,\ldots,k\}$:
                            \begin{align*}
                                |\langle f-f_0,y_i \rangle | &\leq |\langle f-f_0,y_i-x_{n_i} \rangle | + |\langle f-f_0,x_{n_i} \rangle |\\ 
                                                             &\leq \|f-f_0\|\|y_i - x_{n_i}\| + 2^{n_i}r \\
                                                             &\leq (\|f\| + \|f_0\|)\|y_i - x_{n_i}\| + 2^{n_i}r  \\
                                                             &\leq 2\cdot \frac{\varepsilon}{4} + 2^{n_i}r = \frac{\varepsilon}{2} + 2^{n_i}r < \frac{\varepsilon}{2} + \frac{\varepsilon}{2} = \varepsilon
                            \end{align*}
                            De aquí tenemos que $f\in V\cap \overline{B}_{E^\ast}$, por lo que hemos demostrado esta inclusión.
                        \item [$\subseteq )$] Si tenemos ahora que $f_0\in \overline{B}_{E^\ast}$, tomamos un entorno suyo en $\tau_d$, que contendrá una bola de cierto radio $r$. Queremos probar que:
                            \begin{equation*}
                                \{f\in E^\ast : d(f,f_0)<r\} \cap \overline{B}_{E^\ast} \supseteq V\cap \overline{B}_{E^\ast}
                            \end{equation*}
                            para un cierto $V$ entorno básico de $f_0$ en $\sigma(E^\ast,E)\big|_{\overline{B}_{E^\ast}}$. Escogemos $\varepsilon>0$ y $m\in \mathbb{N}$ de forma que se tenga $\varepsilon + \frac{1}{2^{m-1}}<r$, y consideramos:
                            \begin{equation*}
                                V = \{f\in E^\ast : |\langle f-f_0,x_i \rangle |<\varepsilon\quad \forall i \in \{1,\ldots,m\}\}
                            \end{equation*}
                            Observamos que $V$ es un entorno básico de $f_0$ en $\sigma(E^\ast,E)\big|_{\overline{B}_{E^\ast}}$. Veamos que tenemos la inclusión enunciada anteriormente, pues si $f\in V\cap \overline{B}_{E^\ast}$, tendremos entonces que:
                            \begin{align*}
                                d(f,f_0) &= [f-f_0] = \sum_{n=1}^{\infty}\frac{1}{2^n}|\langle f-f_0,x_n \rangle | \\
                                         &= \sum_{n=1}^{m}\frac{1}{2^n}|\langle f-f_0,x_n \rangle | + \sum_{n=m+1}^{\infty}\frac{1}{2^n}|\langle f-f_0,x_n \rangle | \\
                                         &\leq \varepsilon \sum_{n=1}^{m} \frac{1}{2^n} + \sum_{n=m+1}^{\infty}\frac{1}{2^n}|\langle f-f_0,x_n \rangle | \leq \varepsilon \sum_{n=1}^{\infty}\frac{1}{2^n} + \sum_{n=m+1}^{\infty}\frac{1}{2^n}|\langle f-f_0,x_n \rangle | \\
                                         &\leq \varepsilon + \sum_{n=m+1}^{\infty}  \frac{1}{2^n} \|f-f_0\|\|x_n\| \leq \varepsilon + \left(\|f\| + \|f_0\|\right) \sum_{n=m+1}^{\infty} \frac{1}{2^n} \\
                                         &\leq \varepsilon + 2\cdot \frac{1}{2^{m}} = \varepsilon + \frac{1}{2^{m-1}} < r
                            \end{align*}
                    \end{description}
            \item[$\Longleftarrow )$] Tenemos que existe una distancia $d$ de forma que:
                \begin{equation*}
                    \sigma(E^\ast,E)\big|_{\overline{B}_{E^\ast}} = \tau_d
                \end{equation*}
                Para cada $n\in \mathbb{N}$, consideramos:
                \begin{equation*}
                    U_n = \left\{f\in E^\ast : d(f,0)<\frac{1}{n}\right\}\cap \overline{B}_{E^\ast} = \left\{f\in \overline{B}_{E^\ast} : d(f,0)<\frac{1}{n}\right\}
                \end{equation*}
                Por lo que $U_n$ es un entorno en la topología débil$-\ast$ inducida a $\overline{B}_{E^\ast}$ para cada $n\in \mathbb{N}$, luego ha de contener a un entorno básico $V_n$, es decir, existen $\varepsilon_n>0$ y $\Phi_n$ un subconjunto finito de $E$ de forma que:
                \begin{equation*}
                    V_n = \{f\in \overline{B}_{E^\ast} : |\langle f,x \rangle |<\varepsilon_n \quad \forall x\in  \Phi_n\}
                \end{equation*}
                Si tomamos:
                \begin{equation*}
                    D = \bigcup_{n\in \mathbb{N}}\Phi_n
                \end{equation*}
                tenemos que $D$ es numerable, como unión numerable de conjuntos  numerables (en realidad son finitos). Veamos ahora que $D$ es denso en $E$, con vistas de aplicar el Corolario~\ref{coro:comprobar_denso}, supuesto que tenemos $f\in \overline{B}_{E^\ast}$ con $\langle f,x \rangle =0$ para cada $x\in D$, tendremos entonces que $f\in V_n$ para cada $n\in \mathbb{N}$, luego $f\in U_n$ para cada $n\in \mathbb{N}$, es decir:
                \begin{equation*}
                    d(f,0) < \frac{1}{n} \qquad \forall n\in \mathbb{N} 
                \end{equation*}
                Por lo que $d(f,0) = 0$, luego $f=0$. Aplicando el Corolario~\ref{coro:comprobar_denso} tenemos que $D$ es denso, por lo que $E$ es separable.
        \end{enumerate}
    \end{proof}
\end{teo}

\begin{observacion}
    Notemos que cualesquiera dos bolas cerradas de un mismo espacio métrico son homeomorfas, por lo que podemos sustituir la bola del enunciado del Teorema por cualquier otra bola cerrada del mismo espacio.
\end{observacion}

\begin{coro}
    Si $E$ es un espacio de Banach separable, entonces toda sucesión acotada en $E^\ast$ admite una sucesión parcial convergente en $\sigma(E^\ast,E)$.
    \begin{proof}
        Sea $\{f_n\}$ una sucesión de puntos de $E^\ast$ de forma que:
        \begin{equation*}
            \|f_n\| \leq R \qquad \forall n \in \mathbb{N}
        \end{equation*}
        para cierto $R\in \mathbb{R}^+_0$, tenemos entonces que $f_n\in \overline{B}_{E^\ast}(0,R)$. El Teorema anterior nos dice que esta bola es metrizable, y el Teorema de Banach-Alaouglú-Bourbaki nos dice que $\overline{B}_{E^\ast}(0,R)$ es $\sigma(E^\ast,E)-$compacta. Como dicha bola es metrizable, tenemos entonces que la bola es secuencialmente compacta, por lo que la sucesión $\{f_n\}$ admite una parcial convergente, al estar contenida totalmente en $\overline{B}_{E^\ast}(0,R)$.
    \end{proof}
\end{coro}

\begin{teo}
    Sea $E$ un espacio de Banach:
    \begin{center}
        $E^\ast$ es separable $\Longleftrightarrow \left(\overline{B}_E, \sigma(E,E^\ast)\big|_{\overline{B}_E}\right)$ es metrizable.
    \end{center}
    \begin{proof}
        La demostración es análoga a la anterior, cambiando $E$ por $E^\ast$ y $E^\ast$ por $E$:
        \begin{enumerate}
            \item[$\Longrightarrow )$] Como $E^\ast$ es separable, será también $\overline{B}_{E^\ast}$ un conjunto separable, luego existe $\{f_n:n\in \mathbb{N}\}\subset \overline{B}_{E^\ast}$ denso. Definimos ahora para cada $x\in E$:
                \begin{equation*}
                    [x] = \sum_{n=1}^{\infty} \frac{1}{2^n} |\langle f_n,x \rangle |
                \end{equation*}
                La aplicación está bien definida, puesto que:
                \begin{equation*}
                    \frac{1}{2^n}|\langle f_n,x \rangle| \leq \frac{1}{2^n}\|f_n\| \|x\| \leq \frac{1}{2^n}\|x\| \qquad \forall n\in \mathbb{N}
                \end{equation*} 
                de donde:
                \begin{equation*}
                    [x] = \sum_{n=1}^{\infty}\frac{1}{2^n}|\langle f_n,x \rangle | \leq \sum_{n=1}^{\infty}\frac{1}{2^n}\|x\| = \|x\| \sum_{n=1}^{\infty}\frac{1}{2^n} = \|x\|
                \end{equation*}
                Por lo que el límite que consideramos a la hora de definir $[x]$ tiene sentido. En particular, tenemos que $[x]\leq \|x\|$. Vemos ahora que:
                \begin{itemize}
                    \item $[x]$ es una norma:
                        \begin{itemize}
                            \item Si $x,y\in E$ tenemos entonces que:% // TODO: CAMBIAR
                                \begin{align*}
                                    [f+g] &= \sum_{n=1}^{\infty}\frac{1}{2^n}|\langle f+g,x_n \rangle|  \leq \sum_{n=1}^{\infty}\frac{1}{2^n}\left(|\langle f,x_n \rangle| +|\langle g,x_n \rangle |\right) \\
                                          &= \sum_{n=1}^{\infty}\frac{1}{2^n}|\langle f,x_n \rangle | + \sum_{n=1}^{\infty}\frac{1}{2^n}|\langle g,x_n \rangle | = [f] + [g]
                                \end{align*}
                            \item Si $f\in E^\ast$ y considero $\lm\in \mathbb{R}$, tenemos que:% // TODO: CAMBIAR
                                \begin{equation*}
                                    [\lm f] = \sum_{n=1}^{\infty}\frac{1}{2^n}|\langle \lm f,x_n \rangle | = \sum_{n=1}^{\infty}\frac{1}{2^n}|\lm| |\langle f,x_n \rangle | = |\lm|\sum_{n=1}^{\infty}\frac{1}{2^n}|\langle f,x_n \rangle | =  |\lm|[f]
                                \end{equation*}
                            \item Si tenemos $x\in E$ tal que:
                                \begin{equation*}
                                    0 = [x] = \sum_{n=1}^{\infty}\frac{1}{2^n}|\langle f_n,x \rangle |
                                \end{equation*}
                                Tendremos entones que $\langle f_n,x \rangle =0\quad \forall n\in \mathbb{N}$. Si tomamos ahora $f\in \overline{B}_{E^\ast}$, como $\{f_n:n\in \mathbb{N}\}$ es un conjunto denso, podemos tomar una sucesión de dichas aplicaciones $\{g_n\}$ convergente a $f$. De esta forma:
                                \begin{equation*}
                                    \{0\} = \{g_n(x)\} \to f(x)
                                \end{equation*}
                                Por lo que ha de ser $f(x) = 0$ para toda $f\in \overline{B}_{\overline{B}_{E^\ast}}$.
                        \end{itemize}
                \end{itemize}
                Consideraremos el espacio métrico inducido por el espacio normado:
                \begin{equation*}
                    d(x,y) = [x-y]
                \end{equation*}
                Observemos que $[x] \leq \|x\|$. Probaremos ahora que
                \begin{equation*}
                    (\overline{B}_{E}, \tau_{d}) = \left(\overline{B}_{E}, \sigma(E,E^\ast)\big|_{\overline{B}_{E}}\right)
                \end{equation*}
                donde $\tau_d$ es la topología asociada a $d$ inducida en $\overline{B}_{E}$:
                    \begin{description}
                        \item [$\supseteq )$] Sea $x_0\in \overline{B}_{E}$ y $U$ un entorno de $x_0$ en $\sigma(E,E^\ast)$, existe $V$ un entorno básico (de la base concreta) de $x_0$ que está contenido en $U$, por lo que existirán $g_1,...,g_k\in E^\ast$ y $\varepsilon>0$ de manera que:
                            \begin{equation*}
                                V = \{x\in E: |\langle g_i,x-x-0 \rangle |<\varepsilon\quad \forall i \in \{1,...,k\}\} \subset U
                            \end{equation*}
                            Además, podemos multiplicar cada $g_i$ por una constante (ajustando luego $\varepsilon$), por lo que podemos suponer sin pérdida de generalidad que $\|g_i\|\leq 1 \quad \forall i \in \{1,\ldots,k\}$.\\

                            \noindent
                            Queremos probar que $U\cap \overline{B}_{E}$ contiene un abierto de $\tau_d$ que contenga a $x_0$ o equivalentemente, que existe $r\in \mathbb{R}^+$ de forma que:
                            \begin{equation*}
                                \{x\in E: d(x,x_0)<r\} \cap \overline{B}_{E} \subset V\cap \overline{B}_{E} \subset U\cap \overline{B}_{E}
                            \end{equation*}
                            Para ello, como $g_i \in \overline{B}_{E^\ast}$ para cada $i \in \{1,\ldots,k\}$, podemos tomar para cada $i \in \{1,\ldots,k\}$ cierto elemento $f_{n_i}$ de forma que:
                            \begin{equation*}
                                \|g_i - f_{n_i}\| < \frac{\varepsilon}{4}
                            \end{equation*}
                            y tomamos $0<r<\min\limits_{i \in \{1,\ldots,k\}} \left\{\dfrac{\varepsilon}{2^{1+n_{i}}} \right\}$. Veamos ahora que este $r$ nos sirve para la condición que buscábamos: sea $x\in \overline{B}_{E}$ de forma que $d(x,x_0)<r$, tenemos entonces que:
                            \begin{equation*}
                                \frac{1}{2^{n_i}}|\langle f_{n_i},x-x_0\rangle | \leq \sum_{n=1}^{\infty}\frac{1}{2^n}|\langle f_n, x-x_0\rangle | < r \qquad \forall i \in \{1,\ldots,k\}
                            \end{equation*}
                            de donde para cada $i \in \{1,\ldots,k\}$:
                            \begin{align*}
                                |\langle g_i,x-x_0\rangle | &\leq |\langle g_i-f_{n_i},x-x_0\rangle | + |\langle f_{n_i},x-x_0\rangle |\\ 
                                                             &\leq \|x-x_0\|\|g_i - f_{n_i}\| + 2^{n_i}r \\
                                                             &\leq (\|x\| + \|x_0\|)\|g_i - f_{n_i}\| + 2^{n_i}r  \\
                                                             &\leq 2\cdot \frac{\varepsilon}{4} + 2^{n_i}r = \frac{\varepsilon}{2} + 2^{n_i}r < \frac{\varepsilon}{2} + \frac{\varepsilon}{2} = \varepsilon
                            \end{align*}
                            De aquí tenemos que $x\in V\cap \overline{B}_{E}$, por lo que hemos demostrado esta inclusión.
                        \item [$\subseteq )$] Si tenemos ahora que $x_0\in \overline{B}_{E}$, tomamos un entorno suyo en $\tau_d$, que contendrá una bola de cierto radio $r$. Queremos probar que:
                            \begin{equation*}
                                \{x\in E^\ast : d(x,x_0)<r\} \cap \overline{B}_{E} \supseteq V\cap \overline{B}_{E}
                            \end{equation*}
                            para un cierto $V$ entorno básico de $x_0$ en $\sigma(E,E^\ast)\big|_{\overline{B}_{E}}$. Escogemos $\varepsilon>0$ y $m\in \mathbb{N}$ de forma que se tenga $\varepsilon + \frac{1}{2^{m-1}}<r$, y consideramos:
                            \begin{equation*}
                                V = \{x\in E^\ast : |\langle f_i,x-x_0 \rangle |<\varepsilon\quad \forall i \in \{1,\ldots,m\}\}
                            \end{equation*}
                            Observamos que $V$ es un entorno básico de $x_0$ en $\sigma(E,E^\ast)\big|_{\overline{B}_{E}}$. Veamos que tenemos la inclusión enunciada anteriormente, pues si $x\in V\cap \overline{B}_{E}$, tendremos entonces que:
                            \begin{align*}
                                d(x,x_0) &= [x-x_0] = \sum_{n=1}^{\infty}\frac{1}{2^n}|\langle f_n,x-x_0\rangle | \\
                                         &= \sum_{n=1}^{m}\frac{1}{2^n}|\langle f_n,x-x_0\rangle | + \sum_{n=m+1}^{\infty}\frac{1}{2^n}|\langle f_n,x-x_0\rangle | \\
                                         &\leq \varepsilon \sum_{n=1}^{m} \frac{1}{2^n} + \sum_{n=m+1}^{\infty}\frac{1}{2^n}|\langle f_n,x-x_0\rangle | \leq \varepsilon \sum_{n=1}^{\infty}\frac{1}{2^n} + \sum_{n=m+1}^{\infty}\frac{1}{2^n}|\langle f_n,x-x_0\rangle | \\
                                         &\leq \varepsilon + \sum_{n=m+1}^{\infty}  \frac{1}{2^n} \|f_n\|\|x-x_0\| \leq \varepsilon + \left(\|x\| + \|x_0\|\right) \sum_{n=m+1}^{\infty} \frac{1}{2^n} \\
                                         &\leq \varepsilon + 2\cdot \frac{1}{2^{m}} = \varepsilon + \frac{1}{2^{m-1}} < r
                            \end{align*}
                    \end{description}
            \item[$\Longleftarrow )$] Es el Ejercicio 3.24. % // TODO: Es el ejercicio 3.24
        \end{enumerate}
    \end{proof}
\end{teo}

\begin{coro}
    Si $E$ es un espacio de Banach reflexivo, entonces toda sucesión acotada en $E$ admite una sucesión parcial convergente en $\sigma(E,E^\ast)$.
    \begin{proof}
        Sea $\{x_n\}$ una sucesión de puntos de $E$ de forma que:
        \begin{equation*}
            \|x_n\| \leq R \qquad \forall n \in \mathbb{N}
        \end{equation*}
        para cierto $R\in \mathbb{R}^+_0$. Consideramos ahora $M=\overline{\cc{L}\{x_n : n\in \mathbb{N}\}}$, que es un subespacio vectorial cerrado de $E$, por lo que $M$ será reflexivo. Además, tenemos que $M$ es separable, puesto que $\cc{L}\{x_n:n\in \mathbb{N}\}$ es separable (basta considerar combinaciones lineales con escalares racionales). Como $M$ es reflexivo y separable, tendremos entonces que $M^\ast$ es reflexivo y separable, de donde $\left(\overline{B}_{M}(0,R), \sigma(M,M^\ast)\big|_{\overline{B}_M(0,R)}\right)$ es metrizable por el Teorema anterior.\\

        \noindent
        Anteriormente probamos que $\left(\overline{B}_M(0,R), \sigma(M,M^\ast)\big|_{\overline{B}_M(0,R)}\right)$ es compacto, por lo que por ser también metrizable, será secuencialmente compacto, de donde $\{x_n\}$ admite una sucesión parcial convergente.
    \end{proof}
\end{coro}

% // TODO: Conjuntos uniformemente convexos se verá luego

\section{Ejemplos de espacios reflexivos}
\subsection{Espacios $l^p$}
\begin{prop}
    $l^p$ es reflexivo para $1<p<\infty$
    \begin{proof}
        Tenemos que ver que la aplicación
        \Func{J}{l^p}{(l^p)^{\ast\ast}}{x}{Jx}
        Dada $\xi \in {(l^p)}^{\ast\ast}$, como tenemos que ${(l^p)}^{\ast}\cong l^q$ para:
        \begin{equation*}
            \frac{1}{p} + \frac{1}{q} = 1
        \end{equation*}
        Tenemos que existe una isometría lineal:
        \Func{\varphi}{l^q}{(l^p)^\ast}{y}{f}
        de forma que:
        \begin{equation*}
            \langle f,x \rangle  = \sum_{n=1}^{\infty}x_ny_n \qquad \forall x\in l^p
        \end{equation*}
        Y tenemos que:
        \begin{equation*}
            \langle \xi,f \rangle  = \langle \xi, \varphi(y) \rangle  = \langle \xi\circ \varphi, y \rangle 
        \end{equation*}
        Tenemos por tanto la aplicación $\xi\circ \varphi:l^q\to \mathbb{R}$ continua y lineal, con lo que $\xi \circ \varphi \in {(l^q)}^{\ast}$. Por tanto, $\exists ! x\in l^p$ de forma que:
        \begin{equation*}
            (\xi \circ \varphi)(y) = \sum_{n=1}^{\infty}y_nx_n \qquad \forall y\in l^q
        \end{equation*}
        Por tanto:
        \begin{equation*}
            \langle \xi\circ\varphi,y \rangle  = \sum_{n=1}^{\infty}y_nx_n = \langle f,x \rangle = \langle Jx,f \rangle  \qquad \forall y\in l^q
        \end{equation*}
        Por tanto, tenemos que $\xi = Jx$.
    \end{proof}
\end{prop}

\noindent
Necesitaremos la siguiente proposición:
\begin{prop}
    Sea $E$ un espacio normado:
    \begin{center}
        $E$ es separable $\Longleftrightarrow $ existe $Y\subset E$ subespacio denso y de dimensión numerable.
    \end{center}
    \begin{proof}
        Por doble implicación:
        \begin{description}
            \item [$\Longrightarrow )$] Tomamos $D\subset E$ denso y numerable, basta considerar $Y = \cc{L}(D)$.
            \item [$\Longleftarrow )$] Suponemos que $Y = \cc{L}(U)$, con $U$ un sistema de generadores de cardinal numerable. Fijado $n\in \mathbb{N}$, definimos:
                \begin{equation*}
                    E_n = \left\{\sum_{k=1}^{n}\lm_k u_k : \lm_1,\ldots,\lm_n \in \mathbb{Q},\quad  u_1,\ldots,u_n\in U\right\}
                \end{equation*}
                Que es un conjunto numerable, pues la aplicación $f:\mathbb{Q}^n\times U^n\to E_n$ de forma que:
                \begin{equation*}
                    f((\lm_1,\ldots,\lm_n),(u_1,\ldots,u_n)) = \sum_{k=1}^{n}\lm_k u_k
                \end{equation*}
                Y vemos que $f$ es sobreyectiva con $\mathbb{Q}^n\times U^n$ numerable, con lo que $E_n$ es numerable para cada $n\in \mathbb{N}$. Si tomamos ahora:
                \begin{equation*}
                    D = \bigcup_{n\in \mathbb{N}}E_n
                \end{equation*}
                Tenemos que $D$ es numerable, así como que $\overline{D}$ contiene a $Y$, por lo que:
                \begin{equation*}
                    E = \overline{Y} \subset \overline{D} \subset E
                \end{equation*}
                Por lo que $E$ es separable.
        \end{description}
    \end{proof}
\end{prop}

\begin{coro}
    $C_0$ y $l_p$ con $1\leq p<\infty$ es separable.
    \begin{proof}
        Tenemos que:
        \begin{equation*}
            \mathbb{R}^{(\mathbb{N})} = \{x\in \mathbb{N}\to \mathbb{R} : \exists n_0\in \mathbb{N} \text{\ tal que\ } x(n) = 0 \quad n\geq n_0\} = \cc{L}(\{e_n : n\in \mathbb{N}\})
        \end{equation*}
        Este espacio está dentro de $c_0$ y $l^p$ con $1\leq p<\infty$ es separable.
    \end{proof}
\end{coro}

\begin{coro} % // TODO: Parece ser necesario usar el axioma de eleccion para probar esto
    $l^\infty$ no es separable.
    \begin{proof}
        Tomamos $J\in \cc{P}(\mathbb{N})$ y consideramos $\chi_J:\mathbb{N}\to \mathbb{R}$ dada por:
        \begin{equation*}
            \chi_J(n) = \left\{\begin{array}{ll}
                0 & \text{si\ } n\notin J \\
                1 & \text{si\ } n\in J
            \end{array}\right. 
        \end{equation*}
        Vemos que $\chi_J \in l^\infty$ para todo $J\in \cc{P}(\mathbb{N})$. Tomamos ahora:
        \begin{equation*}
            B_J = B\left(\chi_J, \frac{1}{2}\right)
        \end{equation*}
        Vemos que si tomamos $J,P\in \cc{P}(\mathbb{N})$ tal que $J\neq P$ vemos que existe $n\in \mathbb{N}$ de forma que:
        \begin{equation*}
            |\chi_J(n) - \chi_P(n)|  = 1 \geq \frac{1}{2} \quad\Longrightarrow\quad \|\chi_J-\chi_P\|_\infty \geq 1
        \end{equation*}
        Lo que nos dice que $B_J$ y $B_P$ son disjuntas, ya que si existiera $x\in B_J\cap B_P$ tendríamos entonces que:
        \begin{equation*}
            1 = \|\chi_J - \chi_P\|_\infty = \|(\chi_J-x) + (x-\chi_P)\|_\infty \leq \|(\chi_J-x)\|_\infty + \|(x-\chi_P)\|_\infty < 1
        \end{equation*}
        Por tanto, tenemos una familia de abiertos de $l^\infty$ dos a dos disjuntos.\\

        \noindent
        Sea ahora $D\subset l^\infty$ denso, tenemos por tanto que $D\cap B_J\neq \emptyset  \quad \forall J\in \cc{P}(\mathbb{N})$. Usando el axioma de elección, existe $f:\cc{P}(\mathbb{N})\to D$ de forma que
        \begin{equation*}
            f(J)\in D\cap B_J\quad \forall J\in \cc{P}(\mathbb{N})
        \end{equation*}
        Además tenemos que $f$ es inyectiva, puesto que los conjuntos $B_J$ eran dos a dos disjuntos. Como $\cc{P}(\mathbb{N})$ no es numerable, tendremos por tanto que $D$ tiene un conjunto no numerable, por lo que $D$ no puede ser numerable, de donde $l^\infty$ no es separable.
    \end{proof}
\end{coro}

\begin{prop}
    $l^1$ y $l^\infty$ no son reflexivos.
    \begin{proof}
        Vimos ya que:
        \begin{equation*}
            {(l^1)}^{\ast}\cong l^\infty \qquad {(c_0)}^{\ast} \cong l^1
        \end{equation*}
        Y tenemos:
        \begin{gather*}
            J:c_0\to c_0^{\ast\ast} \cong {(l^1)}^{\ast} \cong l^\infty
        \end{gather*}
        Por lo que $c_0\cong l^\infty$.
    \end{proof}
\end{prop}
