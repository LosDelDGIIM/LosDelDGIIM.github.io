\subsection{Ejercicios adicionales}
\noindent
La siguiente relación de ejercicios proviene de los apuntes de Javier Pérez, cuyos apuntes pueden encontrarse en la bibliografía de la asignatura

\begin{ejercicio}
    Sea $E$ un espacio métrico completo y supongamos que $E = \bigcup\limits_{n\in \mathbb{N}}F_n$ donde los $F_n$ son conjuntos cerrados. Prueba que $\bigcup\limits_{n\in \mathbb{N}}\Int F_n$ es un abierto denso en $E$.
\end{ejercicio}

\begin{ejercicio}
    Sean $X$ un espacio de Banach, $Y$ un espacio normado y $T:X\to Y$ una aplicación lineal. Se define una nueva norma en $X$ mediante la expresión $\||x|\| = \|x\| + \|T(x)\|$ para todo $x\in X$. Prueba que las siguientes afirmaciones son equivalentes:
    \begin{enumerate}[label=\alph*)]
        \item $T$ es continua.
        \item $\|\cdot \|$ y $\||\cdot |\|$ son equivalentes.
        \item $\||\cdot |\|$ es una norma completa en $X$.
    \end{enumerate}

    \noindent
    Probamos las implicaciones:
    \begin{description}
        \item [$c)\Longrightarrow b)$] Como tenemos trivialmente que:
            \begin{equation*}
                \|x\| \leq \|x\| + \|T(x)\| = \||x|\| \qquad \forall x\in X
            \end{equation*}
            y $\||\cdot |\|$ es también una norma completa, el Corolario~\ref{coro:equivalencia_normas} nos dice que las dos normas son equivalentes.
        \item [$b)\Longrightarrow a)$] Como las dos normas son equivalentes, ha de existir $C\in \mathbb{R}^+$ de forma que:
            \begin{equation*}
                \||x|\| \leq C\|x\| \qquad \forall x\in X
            \end{equation*}
            Por lo que:
            \begin{equation*}
                \|x\| + \|T(x)\| = \||x|\| \leq C \|x\| \quad \Longrightarrow \quad \|T(x)\| \leq (C-1)\|x\| \qquad \forall x\in X
            \end{equation*}
        \item [$a)\Longrightarrow c)$] Si tomamos una sucesión de puntos de $X$ que sea de Cauchy para $\||\cdot |\|$: $\{x_n\}$, tenemos entonces que:
            \begin{equation*}
                \forall \varepsilon>0~\exists m\in \mathbb{N} : p,q\geq m \Longrightarrow \||x_p-x_q|\|<\varepsilon
            \end{equation*}
            Pero tenemos:
            \begin{equation*}
                \||x_p - x_q|\| = \|x_p - x_q\| + \|T(x_p) - T(x_q)\| < \varepsilon \Longrightarrow \left\{\begin{array}{l}
                    \|x_p - x_q\| < \varepsilon \\
                    \|T(x_p) - T(x_q)\| < \varepsilon
                \end{array}\right.
            \end{equation*}
            Por lo que $\{x_n\}$ es de Cauchy para $\|\cdot \|$ y como esta norma es completa tenemos que existe $x\in X$ de forma que $\{x_n\}\stackrel{\|\cdot \|}{\longrightarrow}x$. Como $T$ es continua, tendremos entones que $\{T(x_n)\}\to T(x)$. Finalmente, de la definición de $\||\cdot |\|$ vemos que $\{x_n\}\stackrel{\||\cdot |\|}{\longrightarrow} x$, pues:
            \begin{equation*}
                \||x_n - x|\| = \|x_n - x\| + \|T(x_n) - T(x)\|
            \end{equation*}
    \end{description}
\end{ejercicio}

\begin{ejercicio}
    Sean $X$ e $Y$ espacios de Banach y $T:X\to Y$ una aplicación lineal y continua. Prueba que $T$ es inyectiva y $T(X)$ es cerrado en $Y$ si, y sólo si, existe $M>0$ tal que $\|x\|\leq M\|T(x)\|$ para todo $x\in X$.
\end{ejercicio}

\begin{ejercicio}
    Sean $X$, $Y$ espacios de Banach y $T\in L(X,Y)$. Prueba que las siguentes afirmaciones son equivalentes:
    \begin{enumerate}[label=\alph*)]
        \item $T(X)$ es cerrado en $Y$.
        \item $T$ es una aplicación abierta de $X$ sobre $T(X)$.
        \item Existe $K>0$ tal que $\|x+\ker T\| \leq K \|Tx\|$ para todo $x\in X$.
        \item Existe $M>0$ tal que para todo $y\in T(X)$ existe $x\in T^{-1}(y)$ verificando $\|x\|\leq M\|y\|$.
    \end{enumerate}

    \noindent
    Probamos las implicaciones:
    \begin{description}
        \item [$a)\Longrightarrow b)$] Si $T(X)\subset Y$ es cerrado, como $Y$ es de Banach tendremos que $T(X)$ será también de Banach, y como $T$ es sobreyectiva sobre su imagen, por el Teorema de la aplicación abierta tenemos que $T$ es una aplicación abierta de $X$ sobre $T(X)$.
        \item [$b)\Longrightarrow c)$] 
    \end{description}
\end{ejercicio}

\begin{ejercicio}
    Prueba que si $X$ e $Y$ son espacios normados, $T:X\to Y$ es un operador lineal con gráfica cerrada y $T(X)$ tiene dimensión finita, entonces $T$ es continuo.
\end{ejercicio}

\begin{ejercicio}
    Sean $X$ un espacio de Banach, $Y$ un espacio normado y $\cc{F}$ un subconjunto de $L(X,Y)$. Las siguientes afirmaciones son equivalentes:
    \begin{enumerate}[label=\alph*)]
        \item $\cc{F}$ está acotado.
        \item Para cada $x\in X$ y cada $g\in Y^\ast$ el conjunto $\{g(T(x)) : T\in \cc{F}\}$ está acotado.
    \end{enumerate}
\end{ejercicio}

\begin{ejercicio}
    Sea $E$ un espacio métrico y $X$ un espacio normado. Prueba que una aplicación $T:E\to X$ es lipschitziana si, y sólo si, lo es $x^\ast\circ T$, para todo $x^\ast \in X^\ast$.
\end{ejercicio}
