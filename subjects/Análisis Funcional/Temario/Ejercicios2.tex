\section{Principio de acotación uniforme y Tª de la gráfica cerrada}
\begin{ejercicio}
    Continuidad de las funciones convexas.\newline
    Sea $E$ un espacio de Banach y sea $\varphi:E\to \left]-\infty,+\infty\right]$ una función convexa secuencialmente semicontinua inferiormente. Supón que $x_0\in \Int D(\varphi)$.\\

    \noindent
    Antes de comenzar con el ejercicio:
    \begin{itemize}
        \item $D(\varphi) = \{x\in E : \varphi(x) < +\infty\}$.
        \item Que $\varphi$ sea convexa significa que:
            \begin{equation*}
                \varphi(tx + (1-t)y) \leq t\varphi(x) + (1-t)\varphi(y)\qquad \forall x,y\in E, \quad \forall t\in \mathbb{R}
            \end{equation*}
        \item Que $\varphi$ sea secuencialmente semicontinua inferiormente significa que:
            \begin{equation*}
                \{x_n\}\to x \Longrightarrow \varphi(x) \leq \liminf \varphi(x_n)
            \end{equation*}
    \end{itemize}
    \begin{enumerate}
        \item Prueba que existen dos constantes $R>0$ y $M$ de forma que:
            \begin{equation*}
                \varphi(x)\leq M \qquad \forall x\in E \quad \text{con}\quad  \|x-x_0\| \leq R
            \end{equation*}
            (\textbf{Pista:} Dado un $\rho>0$ apropiado, considera los conjuntos
            \begin{equation*}
                F_n = \{x\in E : \|x-x_0\| \leq \rho \quad \text{y} \quad \varphi(x)\leq n\})
            \end{equation*}

            \textbf{Solución.} Como $x_0\in \Int D(\varphi)$, tenemos que existe $\rho>0$ de forma que $\overline{B}(x_0,\rho)\subset D(\varphi)$. Consideramos:
            \begin{equation*}
                F_n = \left\{x\in \overline{B}(x_0,\rho) : \varphi(x) \leq n\right\} \qquad \forall n\in \mathbb{N}
            \end{equation*}
            Tenemos que:
            \begin{itemize}
                \item Para cada $n\in \mathbb{N}$ veamos que $F_n$ es cerrado, ya que si $\{x_m\}\to x$ con $x_m \in F_n\quad \forall m\in \mathbb{N}$ y $x\in E$ tendremos entonces que $x\in \overline{B}(x_0,\rho)$ por ser este conjunto cerrado y:
                    \begin{equation*}
                        \varphi(x) \leq \liminf\varphi(x_n) \leq n
                    \end{equation*}
                    Por lo que $x\in F_n$.
                \item Como $\overline{B}(x_0,\rho)\subset D(\varphi)$, es claro que:
                    \begin{equation*}
                        \bigcup_{n \in \mathbb{N}}F_n = \overline{B}(x_0,\rho)
                    \end{equation*}
            \end{itemize}
            Por el Lema de Baire, tenemos que existe $m\in \mathbb{N}$ de forma que $\Int F_m\neq \emptyset $, por lo que existen $x_1\in \Int F_m$ y $r>0$ de forma que $\overline{B}(x_1,r)\subset F_m$. Es decir:
            \begin{equation*}
                \varphi(x) \leq m \qquad \forall x\in \overline{B}(x_1,r)
            \end{equation*}
            Sea ahora $x\in \overline{B}(x_0, \nicefrac{r}{2})$ tenemos que existe $z\in \overline{B}(0,1)$ de forma que:
            \begin{equation*}
                x = x_0 + \frac{r}{2}z = x_0 + \frac{r}{2}z + \frac{x_1}{2} - \frac{x_1}{2} = \frac{1}{2}(x_1 + rz) + \frac{1}{2}(2x_0-x_1)
            \end{equation*}
            Si usamos la convexidad de $\varphi$:
            \begin{equation*}
                \varphi(x) = \varphi\left(\frac{1}{2}(x_1 + rz) + \frac{1}{2}(2x_0-x_1)\right) \leq \frac{1}{2}\varphi(x_1 + rz) + \frac{1}{2}\varphi(2x_0-x_1)
            \end{equation*}
            Como tenemos que $x_1+rz\in \overline{B}(x_1,r)\subset F_m$, tenemos que:
            \begin{equation*}
                \varphi(x) \leq  \frac{1}{2}\varphi(x_1 + rz) + \frac{1}{2}\varphi(2x_0-x_1) \leq \frac{m}{2} + \frac{1}{2}\varphi(2x_0 - x_1)
            \end{equation*}
            Por lo que tomando:
            \begin{equation*}
                R = \frac{r}{2}, \qquad M = \frac{m}{2}+\frac{1}{2}\varphi(2x_0 - x_1)
            \end{equation*}
            Tenemos que:
            \begin{equation*}
                \varphi(x)\leq M \qquad \forall x\in \overline{B}(x_0,R)
            \end{equation*}
        \item Prueba que $\forall r<R$, $\exists L\geq 0$ de forma que
            \begin{equation*}
                |\varphi(x_1)-\varphi(x_2)| \leq L\|x_1-x_2\| \qquad \forall x_1,x_2\in E \quad \text{con}\quad \|x_i-x_0\| \leq r, \quad i \in \{1,2\}
            \end{equation*}
            Más precisamente, uno puede tomar $L = \frac{2(M-\varphi(x_0))}{R-r}$.

            \textbf{Solución.} Fijado $r<R$, si tomamos $x_1,x_2\in \overline{B}(x_0,r)$, tenemos que existen $u\in S(x_0,r)$ y $t\in [0,1]$ de forma que:
            \begin{equation*}
                x_2 = tx_1 + (1-t)u
            \end{equation*}
            Por lo que usando la convexidad de $\varphi$:
            \begin{equation*}
                \varphi(x_2) = \varphi(tx_1 + (1-t)u) \leq t\varphi(x_1) + (1-t)\varphi(u)
            \end{equation*}
            de donde:
            \begin{equation*}
                \varphi(x_2)-\varphi(x_1) \leq (1-t)(\varphi(u) - \varphi(x_1)) \leq (1-t)(M-\varphi(x_1))
            \end{equation*}
            Si observamos ahora que:
            \begin{equation*}
                x_1 - x_2 = (1-t)(u-x_1)
            \end{equation*}
            tomando normas:
            \begin{align*}
                \|x_1 - x_2\| &= (1-t)\|u-x_1\| = (1-t)\|u-x_0+x_0-x_1\| \\ &\geq (1-t)\left|\|u-x_0\| - \|x_0-x_1\|\right| = (1-t)(R-r)
            \end{align*}
            por lo que:
            \begin{equation*}
                (1-t)\leq \frac{\|x_1-x_2\|}{R-r}
            \end{equation*}
            de donde:
            \begin{equation*}
                \varphi(x_2) - \varphi(x_1) \leq (1-t)(M-\varphi(x_1)) \leq \frac{M-\varphi(x_1)}{R-r}\|x_1-x_2\|
            \end{equation*}
    \end{enumerate}
\end{ejercicio}

\begin{ejercicio}
    Sea $E$ un espacio vectorial y sea $p:E\to \mathbb{R}$ una función que cumple las siguientes propiedades:
    \begin{enumerate}
        \item $p(x+y)\leq p(x)+p(y)\quad \forall x,y\in E$.
        \item Para cada $x\in E$ fijo la función $\lm\longmapsto p(\lm x)$ es continua.
        \item Siempre que una sucesión de puntos de $E$ $\{y_n\}$ verifique que $\{p(y_n)\}\to 0$, entonces $\{p(\lm y_n)\}\to 0$ para cada $\lm \in \mathbb{R}$.
    \end{enumerate}
    Supongamos que $\{x_n\}$ es una sucesión de puntos de $E$ de forma que $\{p(x_n)\}\to 0$ y $\{\alpha_n\}$ es una sucesión de números reales acotada. Probar que $p(0)=0$ y que $\{p(\alpha_n x_n)\}\to 0$.\newline
    (\textbf{Pista:} Dado $\varepsilon>0$, considera los conjuntos:
    \begin{equation*}
        F_n = \{\lm \in \mathbb{R} : |p(\lm x_k)| \leq \varepsilon\quad \forall k\geq n\}
    \end{equation*}
    Deduce que si $\{x_n\}$ es una sucesión de $E$ de forma que $\{p(x_n-x)\}\to 0$ para algún $x\in E$ y $\{\alpha_n\}$ es una sucesión de forma que $\{\alpha_n\}\to \alpha$, entonces $\{p(\alpha_n x_n)\}\to p(\alpha x)$.)\\

    \noindent
    Por un lado tenemos que:
    \begin{equation*}
        p(0) \leq p(0) + p(0) = 2 p(0) \Longrightarrow p(0) \geq 0 
    \end{equation*}
    Por otro:
    \begin{equation*}
        p(0) \leq p(x_n) + p(-x_n) \qquad \forall n\in \mathbb{N}
    \end{equation*}
    con $\{p(x_n) + p(-x_n)\}\to 0$, por lo que $p(0)\leq 0$, de donde tenemos que $p(0) = 0$.

    \noindent
    Siguiendo la pista, dado $\varepsilon>0$ definimos:
    \begin{equation*}
        F_n = \{\lm \in \mathbb{R} : |p(\lm x_k)| \leq \varepsilon\quad \forall k\geq n\}
    \end{equation*}
    Por reducción al absurdo, supongamos que $\{p(\alpha_n x_n)\}\not\to 0$, de donde existe una parcial con $|p(\alpha_{\sigma(n)}x_{\sigma(n)})|\geq \varepsilon$, para todo $n\in \mathbb{N}$. Como $\{\alpha_n\}$ está acotada, el Teorema de Weierstrass nos permite encontrar una parcial convergente. Supongmos que la parcial $\sigma$ verifica esto, con lo que $\{\alpha_{\sigma(n)}\}\to \alpha$. Observemos que:
    \begin{itemize}
        \item $F_n$ es cerrado para cada $n\in \mathbb{N}$, ya que:
            \begin{equation*}
                F_n = \bigcap_{k\geq n}\{\lm \in \mathbb{R} : |p(\lm x_k)|\leq \varepsilon\}
            \end{equation*}
            Como $(2)$ nos dice que $\lm \longmapsto p(\lm x_k)$ es continua, tenemos que cada uno de dichos conjuntos son cerrados, como preimagen de un conjunto cerrado por una función continua.
        \item Veamos que $\bigcup_{n\geq 1}F_n = \mathbb{R}$. Para ello, tomamos $\lm \in \mathbb{R}$ y como por hipótesis $\{p(\lm x_n)\}\to 0$, con lo que existe $n_0\in \mathbb{N}$ de forma que:
            \begin{equation*}
                |p(\lm x_n)| < \varepsilon \qquad \forall n\geq n_0
            \end{equation*}
            Luego $\lm\in  F_{n_0}$
    \end{itemize}
    Por el contrarrecíproco del Lema de Baire, existe $F_{\overline{n}}$ con $F_{\overline{n}}^\circ \neq \emptyset $, por lo que existen $\lm_0 \in \mathbb{R}$ y $\delta>0$ de forma que
    \begin{equation*}
        B(\lm_0, \delta)\subset F_{\overline{n}}^\circ
    \end{equation*}
    En otras palabras:
    \begin{equation*}
        (|p((\lm_0+t)x_{\sigma(k)})| \leq \varepsilon\quad \forall k\geq n_0 ) \quad \forall t\in \left]-\delta,\delta\right[
    \end{equation*}
    Ahora:
    \begin{equation*}
        p(\alpha_{\sigma(k)}x_{\sigma(k)})\leq p((\lm_0+\alpha_{\sigma(k)}-\alpha)x_{\sigma(k)}) + p((\alpha-\lm_0)x_{\sigma(k)})
    \end{equation*}
    con $\{ p((\alpha-\lm_0)x_{\sigma(k)})\}\to 0$ y podemos acotar el primer sumando en valor absoluto:
    \begin{equation*}
        |p((\lm_0+\alpha_{\sigma(k)}-\alpha)x_{\sigma(k)}) |\leq \varepsilon
    \end{equation*}
    Luego:
    \begin{equation*}
        p((\lm_0 + \alpha_{\sigma(k)} - \alpha)x_{\sigma(k)}) \leq p((\lm_0-\alpha)x_{\sigma(k)}) + p(\alpha_{\sigma(k)}x_{\sigma(k)})
    \end{equation*}
    de donde:
    \begin{equation*}
        p(x_{\sigma(k)}x_{\sigma(k)}) \geq p((\lm_0 + \alpha_{\sigma(k)}-\alpha)x_{\sigma(k)}) - p((\lm_0-\alpha)x_{\sigma(k)})
    \end{equation*}
    de forma que el segundo sumando tiende a 0 y el primero está acotado en valor absoluto por $\varepsilon$, de donde deducimos:
    \begin{equation*}
        p(\alpha_{\sigma(k)}x_{\sigma(k)}) \leq 2\varepsilon
    \end{equation*}
    Seguimos:
    \begin{equation*}
        p(\alpha_n x_n) - p(\alpha x) \leq p(\alpha_n (x_n - x)) + \underbrace{p(\alpha_n x) - p(\alpha  x)}_{\text{tiende a\ } 0}
    \end{equation*}
    Además, como $\{\alpha_n\}$ está acotada y $\{x_n - x\}\to 0$, nos queda simplemente acotar por debajo para aplicar el Lema del Sandwich:
    \begin{equation*}
        p(\alpha_n x) \leq p(\alpha_n (x-x_n)) + p(\alpha_n x_n)
    \end{equation*}

    de donde:
    \begin{equation*}
        p(\alpha_n x) - p(\alpha_n(x_n - x)) - p(\alpha x) \leq p(\alpha_n x_n) - p(\alpha x)
    \end{equation*}
    de donde $\{p(\alpha_n x_n)\}\to p(\alpha x)$.
\end{ejercicio}

\begin{ejercicio}
    Sean $E$ y $F$ dos espacios de Banach y $\{T_n\}$ una sucesión en $L(E,F)$. Supongamos que para todo $x\in E$ se tiene que $\{T_nx\}$ converge a un cierto límite $Tx$. Probar que si $\{x_n\}\to x$ en $E$, entonces $\{T_n(x_n)\}\to Tx$ en $F$.\\

    \noindent
    Dado $x\in E$, como $\{T_nx\}$ es convergente a $Tx$ tenemos entonces que $\{T_nx\}$ es acotada, con lo que:
    \begin{equation*}
        \sup_{n\in \mathbb{N}}\|T_nx\| < \infty
    \end{equation*}
    Por el Principio de acotación uniforme, tenemos que $C=\sup\limits_{n\in \mathbb{N}}\|Tn\| < \infty$, de donde:
    \begin{align*}
        \|T_nx_n - Tx\| &= \|T_nx_n - T_nx + T_nx - Tx\| \leq \|T_nx_n - T_nx\| + \|T_nx - Tx\| \\
                        &= \|T_n(x_n - x)\| + \|T_nx-Tx\| \leq C\|x_n-x\| + \|T_nx-Tx\|
    \end{align*}
    Como $\|x_n-x\|, \|T_nx-Tx\|\to 0$, tenemos pues que $\|T_nx_n - Tx\| \to 0$, de donde $\{T_n(x_n)\}\to Tx$.
\end{ejercicio}

\begin{ejercicio}\label{ej:4_rel2} % // TODO: Ejercicio de examen
    Sean $E,F$ dos espacios de Banach y sea $a:E\times F\to \mathbb{R}$ una aplicación bilineal que verifica:
    \begin{enumerate}
        \item Para cada $x\in E$, la aplicación $y\longmapsto a(x,y)$ es continua.
        \item Para cada $y\in F$, la aplicación $x\longmapsto a(x,y)$ es continua.
    \end{enumerate}
    Probar que existe una constante $C\geq 0$ de forma que:
    \begin{equation*}
        |a(x,y)| \leq C\|x\|\|y\| \qquad \forall x\in E, \quad \forall y\in F
    \end{equation*}
    (\textbf{Pista:} Introduce un operador lineal $T:E\to F^\ast$ y prueba que $T$ está acotada con ayuda del Corolario~\ref{coro:entonces_Bast_acotado}).\\

    \begin{description}
        \item [Opción 1.] Fijado $x\in E$ definimos: % // TODO: ENTENDER ESTA DEMO
            \Func{f_x}{F}{\bb{R}}{y}{a(x,y)}
            y la linealidad de $a$ en segunda variable así como la continuidad de la función $y\longmapsto a(x,y)$ nos dice que $f_x\in F^\ast\quad \forall x\in E$, con lo que podemos definir la función
            \Func{T}{E}{F^\ast}{x}{f_x}
            Queremos ver que si $B\subset E$ es un conjunto acotado entonces $T(B)$ es acotado. Para ello consideramos:
            \begin{equation*}
                \langle T(B),y \rangle  = \{\langle f_x,y \rangle :f_x\in T(B)\} = \{a(x,y):x\in B\} = \{f_y(x) : x\in B\}
            \end{equation*}

            de donde:
            \begin{equation*}
                |f_y(x)| \leq M \|x\| \qquad \forall x\in E
            \end{equation*}
            Luego por el Corolario~\ref{coro:entonces_Bast_acotado} tenemos que $T(B)$ está acotado, así como que $T$ es continua (acabamos de probar que es una aplicación acotada), por lo que existe $C\geq 0$ tal que $\|T(x)\|\leq C\|x\|\quad \forall x\in E$. Luego:
            \begin{equation*}
                \|f_x\| = \sup_{\|y\|\leq 1}|f_x(y)| \leq C\|x\|
            \end{equation*}

            de donde:
            \begin{equation*}
                \left|a\left(x,\frac{y}{\|y\|}\right)\right| \leq C\|x\| \Longrightarrow a(x,y) \leq C\|x\|\|y\|
            \end{equation*}
        \item [Opción 2.] Definimos:
            \Func{T}{E}{F^\ast}{x}{Tx}
            Donde el operador $Tx$ viene dado para cada $x\in E$ por:
            \Func{Tx}{F}{\bb{R}}{y}{a(x,y)}
            Y vemos que:
            \begin{itemize}
                \item Como fijado $x\in E$ la aplicación $y\longmapsto a(x,y)$ es continua, $Tx$ es continua para cada $x\in E$.
                \item Como $a$ es lineal en segunda variable, $Tx$ es lineal para cada $x\in E$, por lo que la aplicación $T$ está bien definida (ya que $Tx\in F^\ast$ para cada $x\in E$).
                \item Como $a$ es lineal en primera variable, $T$ es lineal, puesto que si $x,z\in E$ y $\lm\in \mathbb{R}$, tendremos que $T(\lm x + z) = \lm Tx + Tz$, ya que:
                    \begin{multline*}
                        T(\lm x + z )(y) = a(\lm x + z,y) = \lm a(x,y) + a(z,y) = \lm Tx(y) + Tz(y) \\ \forall y\in F
                    \end{multline*}
                \item Veamos que $T$ es continua usando para ello el Teorema de la Gráfica cerrada. Sea $\{(x,Tx)\}$ una sucesión de puntos de $GrT$ convergente a cierto punto $(x,L)\in E\times F^\ast$, tenemos entonces que $\|x_n-x\|, \|Tx_n-L\| \to 0$. De la segunda desigualdad deducimos que para todo $y\in F$ se tiene:
                    \begin{equation*}
                        \|a(x_n,y) - L(y)\| = \|Tx_n(y) - L(y)\| = \|(Tx_n-L)(y)\| \to 0
                    \end{equation*}
                    Por lo que $\{a(x_n,y)\}\to L(y)$, pero usando que fijado $y\in F$ la aplicación $x\longmapsto a(x,y)$ es continua, teníamos que $\{a(x_n,y)\}\to a(x,y)$, por lo que ha de ser:
                    \begin{equation*}
                        L(y) = a(x,y) = Tx(y) \qquad \forall y\in F \quad \Longrightarrow \quad L = Tx
                    \end{equation*}
                    de donde $GrT$ es cerrada, luego $T$ es continua.
            \end{itemize}
            Podemos ya terminar la demostración, observando que:
            \begin{equation*}
                |a(x,y)| = |Tx(y)| \leq \|Tx\|\|y\| \leq \|T\|\|x\|\|y\| \qquad \forall x\in E, \quad \forall y\in F
            \end{equation*}
    \end{description}
\end{ejercicio}

\begin{ejercicio}\label{ej:5_rel2}
    Sea $E$ un espacio de Banach y sea $\{\varepsilon_n\}$ una sucesión de reales positivos de forma que $\{\varepsilon_n\}\to 0$. Además, sea $\{f_n\}$ una sucesión de elementos de $E^\ast$ que cumple la propiedad:
    \begin{center}
        $\exists r>0~\forall x\in E$ con $\|x\|<r$, $\exists C(x)\in \mathbb{R}$ de forma que $\langle f_n,x \rangle \leq \varepsilon_n\|f_n\| + C(x)\qquad \forall n\in \mathbb{N}$
    \end{center}
    Prueba que la sucesión $\{f_n\}$ está acotada.\newline
    (\textbf{Pista:} Introduce $g_n = \frac{f_n}{(1+\varepsilon_n\|f_n\|)}$).\\

    \noindent
    Consideramos:
    \begin{equation*}
        g_n = \frac{f_n}{1+\varepsilon_n\|f_n\|} \qquad \forall n\in \mathbb{N}
    \end{equation*}
    Como $f_n\in E^\ast$ tendremos también que $g_n\in E^\ast$ para cada $n\in \mathbb{N}$. Sabemos que $\exists r>0$ tal que $\forall x\in B(0,r)$ existe $C(x)\in \mathbb{R}$ de forma que:
    \begin{equation*}
        f_n(x) \leq \varepsilon_n\|f_n\| + C(x) \qquad \forall n\in \mathbb{N}
    \end{equation*}
    De esta forma:
    \begin{itemize}
        \item Si $x\in B(0,r)$, tenemos para cada $n\in \mathbb{N}$ que:
            \begin{align*}
                g_n(x) &= \dfrac{f_n(x)}{1+\varepsilon_n\|f_n\|} \leq \frac{\varepsilon_n\|f_n\|+C(x)}{1+\varepsilon_n\|f_n\|} = \frac{1+\varepsilon_n\|f_n\|+C(x)-1}{1+\varepsilon_n\|f_n\|} = 1 + \frac{C(x)-1}{1+\varepsilon_n\|f_n\|} \\
                       &\stackrel{(\ast)}{\leq} 1+C(x)-1 = C(x)
            \end{align*}
            donde en $(\ast)$ hemos usado que $1+\varepsilon_n\|f_n\|\geq 1$, al ser $\varepsilon_n\in \mathbb{R}^+$.
        \item Si $x\in E$ con $\|x\|\geq r$, si consideramos $y=\frac{r}{2}\frac{x}{\|x\|}$ tenemos que $y\in B(0,r)$, por lo que para cada $n\in \mathbb{N}$ podemos aplicar la cota anteriormente conseguida a $y$, y usando además que $g_n$ es lineal obtenemos:
            \begin{equation*}
                C(y) \geq g_n(y) = \frac{r}{2\|x\|}g_n(x) \quad \Longrightarrow \quad g_n(x) \leq \frac{2\|x\|}{r}C(y)
            \end{equation*}
    \end{itemize}
    En definitiva, hemos probado que:
    \begin{equation*}
        \sup_{n\in \mathbb{N}}|g_n(x)| < \infty \qquad \forall x\in E
    \end{equation*}
    Por lo que aplicando el Teorema de Banach-Steinhaus obtenemos que \newline $C=\sup\limits_{n\in \mathbb{N}}\|g_n\| < \infty$, de donde:
    \begin{equation*}
        \|g_n\| = \frac{\|f_n\|}{1+\varepsilon_n\|f_n\|} \quad \Longrightarrow \quad \|g_n\|(1+\varepsilon_n\|f_n\|) = \|f_n\|
    \end{equation*}
    Por lo que:
    \begin{equation*}
        \|g_n\| = \|f_n\|(1-\varepsilon_n\|g_n\|) 
    \end{equation*}
    Si observamos ahora que $\|g_n\|\leq C$ y que $\{\varepsilon_n\}\to 0$, tenemos que: % // TODO: TERMINAR


    % // TODO: VERSION DE JORGE, tengo que meter esto en lo mio
    Para $x\in B(0,r)$ tenemos que:
    \begin{align*}
        f_n(x) &\leq \varepsilon_n\|f_n\| + C(x) \\
        -f_n(x) = f(-x) &\leq \varepsilon_n\|f_n\| + C(-x)
    \end{align*}

    de donde:
    \begin{equation*}
        |f_n(x)|  \leq \varepsilon_n\|f_n\| + \tilde{C}(x)
    \end{equation*}

    donde:
    \begin{equation*}
        \tilde{C}(x) = \max\{C(x),C(-x)\}
    \end{equation*}
    \begin{itemize}
        \item Si $x\in B(0,r)$, entonces:
            \begin{equation*}
                |g_n(x)| = \frac{|f_n(x)|}{1+\varepsilon_n\|f_n\|} \leq \frac{\varepsilon_n\|f_n\| + \tilde{C}(x)}{1+\varepsilon_n\|f_n\|}
            \end{equation*}
        \item Si tomamos ahora $x\in E$ con $\|x\|\geq r$, tomamos $\lm\in \mathbb{R}^+$ con $\lm < \frac{r}{\|x\|}$, de donde $\|\lm x \| < r$. Tenemos que:
            \begin{equation*}
                |g_n(x)| = \frac{1}{\lm}|g_n(\lm x)| \leq \frac{1}{\lm} \frac{\varepsilon_n\|f_n\| + \tilde{C}(\lm x)}{1+\varepsilon_n\|f_n\|}
            \end{equation*}
    \end{itemize}
    Para ver que esta última cantidad está acotada, tomando:
    \begin{equation*}
        t_n = \varepsilon_n\|f_n\| \geq 0 \qquad \forall n\in \mathbb{N}
    \end{equation*}
    tenemos que:
    \begin{equation*}
        \left|\frac{t_n + \tilde{C}(x)}{t_n+1}\right| = \left|1+\frac{\tilde{C}(x)-1}{t_n+1}\right| \leq 1 + \left|\frac{\tilde{C}(x)-1}{t_n+1}\right| \leq 1 + |\tilde{C}(x)-1|
    \end{equation*}
    Tenemos pues que:
    \begin{equation*}
        \sup_{n\in \mathbb{N}}|g_n(x)| < \infty \qquad \forall x\in E
    \end{equation*}
    Por el Principio de acotación uniforme tenemos que $M = \sup\limits_{n\in \mathbb{N}}\|g_n\|<\infty$. Tenemos ahora:
    \begin{align*}
        M > \|g_n\| = \frac{\|f_n\|}{1+\varepsilon_n\|f_n\|} \Longleftrightarrow \|f_n\| < M(1+\varepsilon_n\|f_n\|) &\Longleftrightarrow \|f_n\| < M + M\varepsilon_n\|f_n\| \\
                                                             &\Longleftrightarrow \|f_n\|(1-M\varepsilon_n) < M
    \end{align*}
    Como $\{\varepsilon_n\}\to 0$, existe $m\in \mathbb{N}$ de forma que si $n\geq m$ tenemos $M\varepsilon_n<\frac{1}{2}$ , por lo que $\|f_n\| < 2M$ para $n\geq m$.
    Como $\{f_n : n<m\}$ es un conjunto finito, podemos acotar su norma por $\max\limits_{n<m}\|f_n\|$. En conclusión:
    \begin{equation*}
        \|f_n\| \leq \max\{2M, \max\limits_{n<m}\|f_n\|\} \qquad \forall n\in \mathbb{N}
    \end{equation*}
\end{ejercicio}

\begin{ejercicio}\label{ej:6_rel2}
    (Operadores no lineales monótonos y localmente acotados)\newline
    Sea $E$ un espacio de Banach y sea $D(A)$ cualquier subconjunto de $E$. Una aplicación (no lineal) $A:D(A)\subseteq E\to E^\ast$ se dice ``monótona'' si verifica
    \begin{equation*}
        \langle Ax - Ay,x-y \rangle \geq 0 \qquad \forall x,y\in D(A)
    \end{equation*}
    \begin{enumerate}
        \item Sea $x_0\in \Int D(A)$. Prueba que existen dos constantes $R>0$ y $C$ de forma que
            \begin{equation*}
                \|Ax\|\leq C \qquad \forall x\in D(A) \quad \text{con}\quad \|x-x_0\|<R
            \end{equation*}
            (\textbf{Pista:} Razona por reducción al absurdo y construye una sucesión $\{x_n\}$ de puntos de $D(A)$ de forma que $\{x_n\}\to x_0$ y $\{\|Ax_n\|\}\to \infty$. Elije $r>0$ de forma que $B(x_0,r)\subset D(A)$. Usa la monotonía de $A$ en $x_n$ y en $(x_0+r)$ con $\|x\|<r$. Aplica el Ejercicio~\ref{ej:5_rel2}).
        \item Prueba la misma conclusión para un punto $x_0\in \Int(\conv D(A))$.
        \item Extiende la conclusión de la pregunta 1 al caso de que $A$ sea multivaluada, es decir, para cada $x\in D(A)$, $Ax$ es un conjunto no vacío de $E^\ast$, en este caso la monotonía se define como sigue:
            \begin{equation*}
                \langle f-g,x-y \rangle \geq 0 \qquad \forall x,y\in D(A), \quad \forall f\in Ax, \quad \forall g\in Ay
            \end{equation*}
    \end{enumerate}

    \noindent
    \textbf{Solución.}
    \begin{enumerate}
        \item Sea $x_0\in \Int D(A)$. Prueba que existen dos constantes $R>0$ y $C$ de forma que
            \begin{equation*}
                \|Ax\| \leq C \qquad \forall x\in D(A) \quad \text{con}\quad \|x-x_0\|<R
            \end{equation*}

            Por reducción al absurdo, supongamos que:
            \begin{equation*}
                \forall R,C\geq 0\quad \exists x\in D(A) \quad \text{con}\quad \|x-x_0\|<R \quad \text{y}\quad  \|Ax\|>C
            \end{equation*}
            Por tanto, para cada $n\in \mathbb{N}$
            \begin{equation*}
                \exists x_n\in D(A) \quad \text{con}\quad \|x_n-x_0\|<\frac{1}{n} \quad \text{y}\quad  \|Ax_n\|>C
            \end{equation*}
            Por lo que $\{x_n\}\to x_0$ y $\{\|Ax_n\|\}\to \infty$. Como $x_0\in \Int D(A)$, sea $r>0$ de forma que $B(x_0,r)\subset D(A)$, si tomamos $x\in E$ con $\|x\|<r$ tendremos entonces que:
            \begin{equation*}
                \|x_0 + x - x_0\| = \|x\| < r \Longrightarrow x_0+x\in B(x_0,r)\subset D(A)
            \end{equation*}
            Como $A$ es monótona, tenemos que:
            \begin{align*}
                0&\leq \langle Ax_n-A(x_0+x),x_n-x_0-x \rangle  \\
                 &= \langle Ax_n, x \rangle  + \langle Ax_n, x_n-x_0 \rangle  + \langle A(x_0+x),x \rangle  + \langle A(x_0+x),x_0-x_n \rangle 
            \end{align*}

            de donde:
            \begin{align*}
                \langle Ax_n,x \rangle  &\leq \|Ax_n\| \|x_n-x_0\| + \|A(x_0+x)\|\|x\| + \|A(x_0+x)\|\|x_0-x_n\| \\
                                        &= \|Ax_n\| \|x_n - x_0\| + \|A(x_0+x)\|(\|x\| + \|x_0-x_n\|)\\
                                        &\stackrel{(\ast)}{\leq} \|Ax_n\| \underbrace{\|x_n - x_0\|}_{\varepsilon_n} + \underbrace{\|A(x_0+x)\|(\|x\| + 1)}_{C(x)}
            \end{align*}
            Donde en $(\ast)$ hemos usado que $\|x_n-x_0\|<1 \quad \forall n\in \mathbb{N}$. Aplicando el Ejercicio~\ref{ej:5_rel2}, tenemos que $\{Ax_n\}$ está acotada, contradicción con que $\{\|Ax_n\|\}\to \infty$, por lo que tenemos el primer apartado.
        \item Prueba la misma conclusión para un punto $x_0\in \Int(\conv D(A))$.

            Tenemos que $\conv D(A)$ es el cierre convexo de $D(A)$:
            \begin{equation*}
                \conv D(A) = \left\{\sum_{k=1}^{n}\lm_k v_k : v_k \in D(A), \text{\ con\ } \sum_{k=1}^{n}\lm_k = 1\right\}
            \end{equation*}
            Repetiremos la misma prueba que en el apartado 1, cambiando un poco el final. Por Reducción al absurdo, de la misma forma podemos encontrar $\{x_n\}\to x_0$ con $\|x_n-x_0\| \leq 1$ para todo $n\in \mathbb{N}$ y $\{\|Ax_n\|\}\to \infty$. Como $x_0\in \Int(\conv D(A))$, sea $r>0$ de forma que $B(x_0,r)\subset \conv D(A)$, si tomamos $x\in E$ con $\|x\|<r$ tendremos que $x_0+x\in \conv D(A)$, por lo que existen $\lm_1,\ldots,\lm_n\in \mathbb{R}$, $v_1,\ldots, v_n\in D(A)$ con:
            \begin{equation*}
                x_0 + x = \sum_{k=1}^{n} \lm_k v_k \qquad \sum_{k=1}^{n}\lm_k = 1
            \end{equation*}
            Fijado $k\in \{1,\ldots,n\}$, podemos usar la monotonía de $A$:
            \begin{equation*}
                0\leq \langle Ax_n - Av_k, x_n - v_k \rangle  \Longrightarrow \langle Ax_n,x_n-v_k \rangle  \geq \langle Av_k, x_n-v_k \rangle 
            \end{equation*}
            Por lo que:
            \begin{equation*}
                \lm_k \langle Ax_n,x_n-v_k \rangle  \geq \lm_k\langle Av_k, x_n-v_k \rangle \qquad \forall k \in \{1,\ldots,n\}
            \end{equation*}
            de donde:
            \begin{equation*}
                \sum_{k=1}^{n}\lm_k \langle Ax_n,x_n-v_k \rangle  \geq\sum_{k=1}^{n} \lm_k\langle Av_k, x_n-v_k \rangle 
            \end{equation*}
            Vemos que:
            \begin{align*}
                \sum_{k=1}^{n}\lm_k \langle Ax_n,x_n-v_k \rangle   &= \left\langle Ax_n, \sum_{k=1}^{n}(\lm_kx_n-\lm_kv_k) \right\rangle  = \left\langle Ax_n, x_n - \sum_{k=1}^{n}\lm_kv_k \right\rangle  \\ &= \langle Ax_n, x_n - x_0-x\rangle  = \langle Ax_n, x_n-x_0 \rangle  - \langle Ax_n, x \rangle 
            \end{align*}
            Por lo que:
            \begin{align*}
                \langle Ax_n,x \rangle &\leq \langle Ax_n, x_n-x_0 \rangle + \sum_{k=1}^{n}\lm_k\langle Av_k, v_k - x_n \rangle  \\
                                       &\leq \|Ax_n\|\|x_n-x_0\| + \sum_{k=1}^{n}\lm_k \|Av_k\|\|v_k-x_n\|
            \end{align*}
            Pero:
            \begin{equation*}
                \|v_k - x_n\| \leq \|v_k - x_0\| + \|x_0 - x_n\| \leq \|v_k - x_0\| + 1
            \end{equation*}
            Por lo que:
            \begin{align*}
                \langle Ax_n,x \rangle &\leq \langle Ax_n, x_n-x_0 \rangle + \sum_{k=1}^{n}\lm_k\langle Av_k, v_k - x_n \rangle  \\
                                       &\leq \|Ax_n\|\|x_n-x_0\| + \sum_{k=1}^{n}\lm_k \|Av_k\|\|v_k-x_n\| \\
                                       &\leq \|Ax_n\|\underbrace{\|x_n-x_0\|}_{\varepsilon_n} + \underbrace{\sum_{k=1}^{n}\lm_k \|Av_k\|(\|v_k-x_0\|+1)}_{C(x)}
            \end{align*}
            Lo que nuevamente lleva a contradicción.
        \item Extiende la conclusión de la pregunta 1 al caso de que $A$ sea multivaluada.

            En este caso, queremos probar que existen $R>0,C\geq 0$ de forma que:
            \begin{equation*}
                \text{Si}\quad  x\in D(A) \quad \text{con}\quad \|x-x_0\|<R \Longrightarrow \|f\| \leq C \qquad \forall f\in Ax
            \end{equation*}
            Por reducción al absurdo, suponemos que
            \begin{equation*}
                \forall R,C\geq 0 \quad \exists  x\in D(A) \quad \text{con}\quad \|x-x_0\|<R \quad \text{y}\quad f\in Ax \quad \text{con}\quad  \|f\| > C 
            \end{equation*}
            Por lo que para cada $n\in \mathbb{N}$ podemos tomar un elemento $x_n\in D(A)$ con $\|x_n-x_0\| < \frac{1}{n}$, y $f_n\in Ax_n$ con $\|f_n\|>C$. En definitiva, tenemos $\{x_n\}\to x_0$ y $\{\|f_n\|\}\to \infty$. Sea ahora $r>0$ de forma que $B(x_0,r)\subset D(A)$, entonces si $x\in E$ con $\|x\|<r$ tendremos que $x_0+x\in D(A)$. Si tomamos $g\in A(x_0+x)$ y usamos la monotonía de $A$:
            \begin{equation*}
                \langle f_n - g,x_n-x_0-x \rangle \geq 0
            \end{equation*}
            luego:
            \begin{align*}
                \langle f_n,x \rangle &\leq \langle f_n,x_n-x_0 \rangle  + \langle g,x+x_0-x_n \rangle  \leq \|f_n\|\|x_n-x_0\| + \|g\|\|x + x_0 - x_n\| \\
                                      &\leq \|f_n\|\|x_n-x_0\| + \|g\|(\|x\| + \|x_0-x_n\|) \leq \|f_n\|\underbrace{\|x_n-x_0\|}_{\varepsilon_n} + \underbrace{\|g\|(\|x\|+1)}_{C(x)}
            \end{align*}
            Lo que nos lleva a contradicción.
    \end{enumerate}
\end{ejercicio}

\begin{ejercicio}
    Sea $\alpha = \{\alpha_n\}$ una sucesión de números reales y sea $1\leq p \leq \infty$. Supongamos que $\sum_{n=1}^{\infty} |\alpha_n||x_n| < \infty$ para cada elemento $x=\{x_n\}$ de $l_p$. Prueba que $\alpha\in l_{p'}$.\\

    \noindent
    Recordamos que para $1\leq p < \infty$ teníamos:
    \begin{equation*}
        l_p = \left\{x:\mathbb{N}\to\mathbb{R} \text{\ tales que\ } \sum_{n=1}^{\infty}{|x_n|}^{p}<\infty\right\}
    \end{equation*}
    que era un espacio normado, con la norma:
    \begin{equation*}
        \|x\|_p = {\left(\sum_{n=1}^{\infty}{|x_n|}^{p}\right)}^{\frac{1}{p}} \qquad \forall x\in l_p
    \end{equation*}
    Y que:
    \begin{equation*}
        l_\infty = \left\{x:\mathbb{N}\to\mathbb{R} \text{\ tales que\ } \sup_{n\in \mathbb{N}}|x_n|<\infty\right\}, \qquad \|x\|_\infty = \sup_{n\in \mathbb{N}}|x_n| , \quad x\in l_\infty
    \end{equation*} % // TODO: HACER
\end{ejercicio}

\begin{ejercicio}
    Sea $E$ un espacio de Banach y sea $T:E\to E^\ast$ un operador lineal verificando que
    \begin{equation*}
        \langle Tx, x \rangle \geq 0 \qquad \forall x\in E
    \end{equation*}
    Prueba que $T$ es acotado.\newline
    (Se pueden aplicar dos métodos, bien usar el Ejercicio~\ref{ej:6_rel2}, bien aplicar el Teorema de la Gráfica Cerrada.)

    \begin{description}
        \item [Usando el Ejercicio~\ref{ej:6_rel2}.] Es fácil ver que $T$ es monótona, ya que:
            \begin{equation*}
                \langle Tx-Ty,x-y \rangle  = \langle T(x-y),x-y \rangle \geq 0 \qquad \forall x,y\in E
            \end{equation*}
            En dicho caso, por el Ejercicio~\ref{ej:6_rel2} tenemos que para todo $x\in E$ existen $R>0$, $C\geq 0$ de forma que:
            \begin{equation*}
                y\in E \quad \text{con}\quad \|y-x\|<R \Longrightarrow \|Ty\| \leq C
            \end{equation*}
            Es decir, si $y\in B(x,R)$ tenemos entonces que $\|Ty\| \leq C$. Sea ahora $z\in B(0,1)$, tenemos que:
            \begin{equation*}
                \|Tz\| = \dfrac{\|T(Rz)\|}{R} \leq \dfrac{C}{R}
            \end{equation*}
            Por lo que $T(B(0,1))$ es un conjunto acotado, y por una proposición vista en teoría concluimos que $T$ es acotada.
        \item [Usando el Teorema de la Gráfica Cerrada.] Sea $\{(x_n,Tx_n)\}$ una sucesión de $GrT$ convergente a $(x,L)\in E\times E^\ast$, queremos ver que $L=Tx$ para concluir que $GrT$ es cerrada. Para ello, observemos que para todo $y\in E$ tenemos:
            \begin{equation*}
                0\leq \langle T(x_n-y),x_n-y \rangle  = \langle Tx_n-Ty,x_n-y \rangle \rightarrow \langle L-T_y,x-y \rangle 
            \end{equation*}
            En particular, tomando $y = x+tz$ para $t\in \mathbb{R}$, tenemos:
            \begin{equation*}
                0\leq \langle L-T(x+tz),-tz \rangle = \langle L-Tx-T(tz),-tz \rangle  = -t\langle L-Tx,z  \rangle + t^2\langle Tz,z \rangle 
            \end{equation*}
            de donde:
            \begin{equation*}
                t^2 \langle Tz,z \rangle \geq t\langle L-Tx,z \rangle  \qquad \forall z\in E, \quad \forall t\in \mathbb{R}
            \end{equation*}
            \begin{itemize}
                \item Si $t>0$, tenemos que:
                    \begin{equation*}
                        t\langle Tz,z \rangle \geq \langle L-Tx,z \rangle \qquad \forall z\in E
                    \end{equation*}
                    Por lo que tendiendo $t\rightarrow 0^+$, tenemos:
                    \begin{equation*}
                        0\geq \langle L-Tx,z \rangle  \qquad \forall z\in E
                    \end{equation*}
                \item Si $t<0$, tenemos que:
                    \begin{equation*}
                        t\langle Tz,z \rangle \leq \langle L-Tx,z \rangle \qquad \forall z\in E
                    \end{equation*}
                    Por lo que tendiendo $t\rightarrow 0^-$, tenemos:
                    \begin{equation*}
                        0\leq \langle L-Tx,z \rangle  \qquad \forall z\in E
                    \end{equation*}
            \end{itemize}
            En definitiva, tenemos que $\langle L-Tx,z \rangle =0$ para todo $z\in E$, por lo que $L = Tx$, luego $GrT$ es un conjunto cerrado. Por el Teorema de la gráfica cerrada, concluimos que $T$ es continua, es decir, acotada.
    \end{description}
\end{ejercicio}

\begin{ejercicio} % // TODO: SE puede hacer sin usar el ejercicio 4, pero se repite dentro la demostración
    Sea $E$ un espacio de Banach y sea $T:E\to E^\ast$ un operador lineal verificando 
    \begin{equation*}
        \langle Tx,y \rangle  = \langle Ty,x \rangle  \qquad \forall x,y\in E
    \end{equation*}
    Prueba que $T$ es acotado.\\

    \begin{description}
        \item [Usando el Ejercicio~\ref{ej:4_rel2}.] Definimos $a:E\times E \to \mathbb{R}$ dada por:
            \begin{equation*}
                a(x,y) = \langle Tx,y \rangle  \qquad \forall x,y\in E
            \end{equation*}
            Tenemos que:
            \begin{itemize}
                \item $a$ es lineal en primera variable por ser $T$ una aplicación lineal.
                \item $a$ es lineal en segunda variable por ser $Tx$ lineal, para cada $x\in E$.
                \item Fijado $x\in E$ tenemos que $Tx$ es continua, es decir, la aplicación: 
                    \begin{equation*}
                        y\longmapsto a(x,y) = \langle Tx,y \rangle 
                    \end{equation*}
                    es continua.
                \item Fijado $y\in E$, tenemos que la aplicación $Ty$ es continua, por lo que la aplicación:
                    \begin{equation*}
                        x\longmapsto a(x,y) = \langle Tx,y \rangle = \langle Ty,x \rangle 
                    \end{equation*}
                    también será continua.
            \end{itemize}
            Aplicando el Ejercicio~\ref{ej:4_rel2}, existe $C\geq 0$ de forma que:
            \begin{equation*}
                |a(x,y)| \leq C\|x\|\|y\|\qquad \forall x,y\in E
            \end{equation*}
            De esta forma, tenemos que:
            \begin{equation*}
                \|Tx\| = \sup_{\|y\|\leq 1}|\langle Tx,y \rangle | = \sup_{\|y\|\leq 1}|a(x,y)| \leq \sup_{\|y\|\leq 1}C\|x\|\|y\| = C\|x\|\qquad \forall x\in E
            \end{equation*}
            Por lo que $T$ es acotado.
        \item [Usando el Teorema de la gráfica cerrada.] Sea $\{(x_n,Tx_n)\}$ una sucesión de puntos de $GrT$ convergente a $(x,L)\in E\times E^\ast$, buscamos probar que $Tx_n = L$ para probar que $GrT$ es cerrado. Para ello, observamos que:
            \begin{equation*}
                \langle L,z \rangle  = \lim_{n\to\infty}\langle Tx_n,z \rangle  = \lim_{n\to\infty}\langle Tz,x_n \rangle  = \langle Tz,x \rangle = \langle Tx,z \rangle  \qquad \forall z\in E
            \end{equation*}
            de donde deducimos que $L = Tx$, por lo que $GrT$ es cerrada. Aplicando el Teorema de la gráfica cerrada concluimos que $T$ es continua, luego acotada.
    \end{description}
\end{ejercicio}

\begin{ejercicio}
    Sean $E$ y $F$ dos espacios de Banach y sea $T\in L(E,F)$ una aplicación sobreyectiva.
    \begin{enumerate}
        \item Sea $M\subset E$. Prueba que $T(M)$ es cerrado en $F$ si y solo si $M+N(T)$ es cerrado en $E$.

            Donde $N(T)$ es el núcleo de $T$, por doble implicación tenemos que:
            \begin{description}
                \item [$\Longleftarrow )$] Si $M+N(T)$ es cerrado, entonces $E\setminus(M+N(T))$ es abierto, y como:
                    \begin{equation*}
                        T(E\setminus(M+N(T))) = F\setminus T(M)
                    \end{equation*}
                    ya que:
                    \begin{description}
                        \item [$\supseteq )$] Si $y\in F\setminus T(M)$, por ser $T$ sobreyectiva existe $x\in E$ de forma que $T(x) = y$. Afirmo que $x\notin M + N(T)$, ya que si $x\in M+N(T)$ entonces existen $m\in M$, $n\in N(T)$ de forma que $x = m+ n$, por lo que:
                            \begin{equation*}
                                y = T(x) = T(m+n) = T(m) + T(n) = T(m) \in T(M)
                            \end{equation*}
                            \underline{contradicción}, luego $x\in E\setminus(M+N(T))$, de donde tenemos que $y \in T(E\setminus (M+N(T)))$.
                        \item [$\subseteq )$] Si $y\in T(E\setminus(M+N(T)))$ entonces existe $x\in E\setminus(M+N(T))$ de forma que $T(x) = y$. Si existiera $z\in M$ de forma que $T(z) = y$, tendríamos entonces que:
                            \begin{equation*}
                                0 = y-y = T(x)-T(z) = T(x-z) 
                            \end{equation*}
                            Por lo que existe $v\in N(T)$ de forma que $v = x-z$, de donde:
                            \begin{equation*}
                                x = z+v \in M+N(T)
                            \end{equation*}
                            \underline{contradicción}, luego $y\in F\setminus T(M)$.
                    \end{description}
                    si aplicamos ahora el Teorema de la Aplicación Abierta, obtenemos que $F\setminus T(M)$ es abierto, por lo que $T(M)$ es cerrado.
                \item [$\Longrightarrow )$] Si $T(M)$ es cerrado, por ser $T$ continua tenemos que $T^{-1}(T(M))$ es cerrado, y:
                    \begin{equation*}
                        T^{-1}(T(M)) = M+N(T)
                    \end{equation*}
                    \begin{description}
                        \item [$\supseteq )$] Si $x+n\in M+N(T)$, entonces:
                            \begin{equation*}
                                T(x+n) = T(x)+T(n) = T(x) \in T(M)
                            \end{equation*}
                            Por lo que $x+n\in T^{-1}(T(x))$.
                        \item [$\subseteq )$] Si $x\in T^{-1}(T(M))$, entonces:
                            \begin{equation*}
                                T(x) \in T(M) \Longrightarrow \exists m\in M \text{\ con\ } T(x) = T(m)
                            \end{equation*}
                            de donde $T(x-m) = T(x)-T(m) = 0$. En conclusión, tenemos que:
                            \begin{equation*}
                                x = m + x - m
                            \end{equation*}
                            con $m\in M$, $x-m\in N(T)$.
                    \end{description}
            \end{description}
        \item Deduce que si $M$ es un subespacio vectorial cerrado de $E$ y si $dim N(T)<\infty$, entonces $T(M)$ es cerrado.

            Si $dim N(T)<\infty$ tenemos entonces que $N(T)$ es un espacio vectorial de dimensión finita, luego es un conjunto cerrado. Como Además, $M$ es un espacio vectorial cerrado, tendremos que $M+N(T)$ es cerrado, luego $T(M)$ será cerrado por el primer apartado.
    \end{enumerate}

    En este último apartado hemos usado que:
    \begin{center}
        Si $E$ es un espacio normado y $M,N\subset E$ son subespacios vectoriales con $M$ cerrado y $N$ de dimensión finita, entonces $M+N$ es cerrado.
    \end{center}
    \begin{proof}
        Distingamos ciertos casos triviales que nos facilitan la prueba:
        \begin{itemize}
            \item Si $N\subset M$, entonces $M+ N = M$, por lo que la prueba es trivial. Suponemos pues que $M\cap N \neq N$.
            \item Si $M\cap N \neq \{0\}$, entonces como $N$ es de dimensión finita, podemos tomar una base suya $B$, de forma que $N = \cc{L}(B)$, y como $M\cap N$ es un subespacio de $N$ ha de existir $B'\subsetneq B$ de forma que $M\cap N = \cc{L}(B')$, por lo que si consideramos $N' = \cc{L}(B\setminus B')$ tenemos que $M\cap N' = \{0\}$, con:
                \begin{equation*}
                    M + N = M + N'
                \end{equation*}
                Podemos por tanto suponer que $M\cap N = \{0\}$.
        \end{itemize}
        Si tenemos una sucesión $\{x_n\}$ de puntos de $M+N$ convergente a $x\in E$, queremos ver que $x\in M+N$. Para ello, para cada $n\in \mathbb{N}$ existen $u_n\in M$ y $v_n\in N$ de forma que:
        \begin{equation*}
            x_n = u_n + v_n
        \end{equation*}
        Y la demostración se completa en dos pasos:
        \begin{description}
            \item [Ver que $\{v_n\}$ está acotada.]  Por reducción al absurdo, supongamos que $\{v_n\}$ no está acotada, con lo que podemos encontrar una parcial divergente \newline $\{\|v_{\sigma(n)}\|\}\to \infty$, lo que nos dirá que (usando que $\{x_n\}$ está acotada por ser convergente):
                \begin{equation*}
                    \left\{\frac{u_{\sigma(n)} + v_{\sigma(n)}}{\|v_{\sigma(n)}\|}\right\}= \left\{\frac{x_{\sigma(n)}}{\|v_{\sigma(n)}\|}\right\} \to 0
                \end{equation*}
                Si tomamos:
                \begin{equation*}
                    w_n = \frac{v_n}{\|v_n\|} \qquad \forall n\in \mathbb{N}
                \end{equation*}
                tenemos que $\|w_n\| = 1 \quad \forall n\in \mathbb{N}$ y que $w_n\in N\quad \forall n\in \mathbb{N}$. Es decir, una sucesión acotada en un espacio de dimensión finita, propiedad que también cumple $\{w_{\sigma(n)}\}$, por lo que ha de existir una parcial de esta última, $\{w_{\gamma(n)}\}$ que sea convergente a cierto $w\in E$ (por el Teorema de Bolzano-Weierstrass). Notemos que por ser $N$ cerrado ha de ser $w\in N$, así como $\|w\| = 1$ por ser $\|\cdot \|$ una aplicación continua. Tenemos entonces que:
                \begin{equation*}
                    \left\{\frac{u_{\gamma(n)}}{\|v_{\gamma(n)}\|} + w_{\gamma(n)}\right\} = \left\{\frac{x_{\gamma(n)}}{\|v_{\gamma(n)}\|}\right\} \to 0
                \end{equation*}
                Fijado $\varepsilon>0$, esta última convergencia nos da $n_1\in \mathbb{N}$ de forma que:
                \begin{equation*}
                    \left\|\frac{u_{\gamma(n)}}{\|v_{\gamma(n)}\|} + w_{\gamma(n)}\right\| < \frac{\varepsilon}{2} \qquad \forall n\geq n_1
                \end{equation*}
                y la convergencia $\{w_{\gamma(n)}\}\to w$ nos da $n_2\in \mathbb{N}$ de forma que
                \begin{equation*}
                    \|w_{\gamma(n)} - w\| < \frac{\varepsilon}{2}\qquad \forall n\geq n_2
                \end{equation*}
                Por lo que tomando $n_0=\max\{n_1,n_2\}$ obtenemos que:
                \begin{equation*}
                    \left\|\frac{u_{\gamma(n)}}{\|v_{\gamma(n)}\|} + w\right\| \leq \left\|\frac{u_{\gamma(n)}}{\|v_{\gamma(n)}\|} + w_{\gamma(n)}\right\| + \|w_{\gamma(n)}  - w\| < \varepsilon \qquad \forall n\geq n_0
                \end{equation*}
                Es decir:
                \begin{equation*}
                    \left\{\frac{u_{\gamma(n)}}{\|v_{\gamma(n)}\|} + w\right\} \to 0 \quad \Longrightarrow \quad \left\{\frac{u_{\gamma(n)}}{\|v_{\gamma(n)}\|}\right\} \to -w
                \end{equation*}
                Y como $M$ es cerrado y la sucesión que tenemos es de puntos de $M$ deducimos que $-w\in M$, por lo que $w\in M$ por ser $M$ un espacio vectorial. En definitiva, hemos probado que si la sucesión $\{v_n\}$ no está acotada, entonces podemos encontrar (recordamos que $\|w\|=1$) $0\neq w \in M\cap N$, \underline{contradicción} con que $M\cap N = \{0\}$.
            \item [Ver que $x\in M+N$.] Una vez sabemos que $\{v_n\}$ está acotada, como $dim N<\infty$, por Bolzano-Weierstrass ha de existir una parcial $\{v_{\alpha(n)}\}$ convergente a cierto $v\in E$, y por ser $N$ de dimensión finita tenemos que es cerrado, con lo que $v\in N$. Fijado $\varepsilon>0$, la convergencia $\{u_{\alpha(n)} + v_{\alpha(n)}\}\to x$ nos da $n_1\in \mathbb{N}$ de forma que:
                \begin{equation*}
                    \|u_{\alpha(n)} + v_{\alpha(n)} - x \| < \frac{\varepsilon}{2} \qquad \forall n\geq n_1
                \end{equation*}
                y la convergencia $\{v_{\alpha(n)}\}\to v$ nos da $n_2\in \mathbb{N}$ de forma que:
                \begin{equation*}
                    \|v_{\alpha(n)} - v\| < \frac{\varepsilon}{2}\qquad \forall n\geq n_2
                \end{equation*}
                por lo que tomando $n_0 = \max\{n_1,n_2\}$ tenemos que:
                \begin{align*}
                    \|u_{\alpha(n)} - x + v\| &= \|u_{\alpha(n)} + v_{\alpha(n)} - x - (v_{\alpha(n)} - v) \|\\ & \leq \|u_{\alpha(n)} + v_{\alpha(n)} - x\| + \|v_{\alpha(n)} - v\| < \varepsilon\qquad  \forall n\geq n_0
                \end{align*}
                Por lo que $\{u_{\alpha(n)}\}\to x-v$ y como $M$ es cerrado, $x-v\in M$. En definitiva, tenemos que:
                \begin{equation*}
                    x = x - v + v
                \end{equation*}
                con $x-v\in M$ y $v\in N$, por lo que $x\in M+N$.
        \end{description}
    \end{proof}
\end{ejercicio}

\begin{ejercicio}
    Sea $E$ un espacio de Banach y $F = l^1$, sea $T\in L(E,F)$ una aplicación sobreyectiva. Prueba que existe $S\in L(F,E)$ de forma que $T\circ S = I_F$, es decir, $S$ es la inversa por la derecha de $T$.\newline
    (\textbf{Pista:} Intenta definir $S$ de forma explícita usando la base canónica de $l^1$.)\\

    \noindent
    Recordamos que:
    \begin{equation*}
        l^1 = \left\{x:\mathbb{N}\to \mathbb{R}\text{\ tales que\ } \sum_{n=1}^{\infty}|x_n|<\infty\right\}, \qquad \|x\|_1 = \sum_{n=1}^{\infty}|x_n|, \quad x\in l^1
    \end{equation*}
    Consideramos:
    \begin{equation*}
        \cc{B} = \{e_k : k\in \mathbb{N}\} \subset l^1
    \end{equation*}
    donde para cada $k\in \mathbb{N}$ tenemos que $e_k(i) = \delta_{k,i} \quad \forall i \in \mathbb{N}$. De esta forma, es evidente que $\|e_k\| = 1 \quad \forall k\in \mathbb{N}$. Como $T$ es sobreyectiva y $l^1$ es de Banach, el Teorema de la aplicación abierta nos dice que $T$ es abierta. Como $0\in T(B(0,1))$, existe $R\in \mathbb{R}^+$ de forma que $\overline{B}(0,R)\subset T(B(0,1))$. Para cada $k\in \mathbb{N}$, tenemos que $Re_k\in \overline{B}(0,R)$, por lo que existe $v_k\in B(0,1)$ de forma que $T(v_k) = Re_k$, de donde tomando:
    \begin{equation*}
        u_k = \frac{v_k}{R}
    \end{equation*}
    tenemos que $T(u_k) = e_k\quad \forall k\in \mathbb{N}$. Definimos $S:l^1\to E$ por:
    \begin{equation*}
        S(x) = \sum_{n=1}^{\infty}x_nu_n \qquad \forall x\in l^1
    \end{equation*}
    Que está bien definida (la suma es convergente), porque:
    \begin{equation*}
        \|x_nu_n\| = |x_n|\|u_n\| = \frac{1}{R}|x_n|\|v_n\| \leq \frac{|x_n|}{R} \qquad \forall n\in \mathbb{N}
    \end{equation*}
    De donde:
    \begin{equation*}
        \sum_{n=1}^{\infty}\|x_nu_n\| \leq \frac{1}{R}\sum_{n=1}^{\infty}|x_n| < \infty
    \end{equation*}
    Por lo que la suma de la definición de $S$ es absolutamente convergente, luego es convergente por ser $E$ de Banach. Además:
    \begin{itemize}
        \item $S$ es lineal, ya que si $x,y\in l^1$ y $\lm \in \mathbb{R}$, entonces:
            \begin{align*}
                S(\lm x+y) &= \sum_{n=1}^{\infty}(\lm x_n + y_n)u_n = \sum_{n=1}^{\infty}\lm x_nu_n + y_nu_n = \lm \sum_{n=1}^{\infty}x_nu_n + \sum_{n=1}^{\infty}y_nu_n \\ &= \lm S(x) + S(y)
            \end{align*}
        \item $S$ es continua, ya que:
            \begin{equation*}
                \|S(x)\| = \left\|\sum_{n=1}^{\infty}x_nu_n\right\| \leq \sum_{n=1}^{\infty} \|x_nu_n\| \leq \frac{1}{R}\sum_{n=1}^{\infty}|x_n| = \frac{1}{R}\|x\| \qquad \forall x\in l^1
            \end{equation*}
    \end{itemize}
    Finalmente, vemos que:
    \begin{align*}
        T(S(x)) &= T\left(\sum_{n=1}^{\infty} x_nu_n\right) = T\left(\lim_{n\to\infty}\sum_{k=1}^{n}x_ku_k \right) \stackrel{T\text{\ cont.}}{=} \lim_{n\to\infty}T\left(\sum_{k=1}^{n}x_ku_k\right) \\ 
                &= \lim_{n\to\infty}\sum_{k=1}^{n}x_kT(u_k) = \sum_{n=1}^{\infty}x_nT(u_n) = \sum_{n=1}^{\infty}x_n e_n = x \qquad \forall x\in l^1
    \end{align*}
\end{ejercicio}

\begin{ejercicio}
    Sean $E,F$ dos espacios de Banach a cuyas normas denotamos por $\|\cdot \|$. Sea $T\in L(E,F)$ de forma que $Im T$ es cerrado y $dim\ker T<\infty$. Sea $|\cdot |$ otra norma definida sobre $E$ que cumple:
    \begin{equation*}
        |x| \leq M \|x\|\qquad \forall x\in E
    \end{equation*}
    Prueba que existe una constante $C$ de forma que:
    \begin{equation*}
        \|x\| \leq C(\|Tx\| + |x|) \qquad \forall x\in E
    \end{equation*}
    (\textbf{Pista:} Razonar por reducción al absurdo)\\

    \noindent
    Por reducción al absurdo, supongamos que:
    \begin{equation*}
        \forall C\in \mathbb{R}^+~\exists x\in E \qquad \text{tal que}\qquad \|x\|>C(\|Tx\|+|x|)
    \end{equation*}
    Por tanto, para todo $n\in \mathbb{N}$ existe $u_n\in E$ de forma que:
    \begin{equation*}
        \|u_n\| > n(\|Tu_n\| + |u_n|) \qquad \forall n\in \mathbb{N}
    \end{equation*}
    Si tomamos:
    \begin{equation*}
        x_n = \frac{u_n}{\|u_n\|} \qquad \forall n\in \mathbb{N}
    \end{equation*}
    Tenemos entonces que:
    \begin{equation*}
        \|x_n\| = \frac{\|u_n\|}{\|u_n\|} > \frac{n}{\|u_n\|}(\|Tu_n\| + |u_n|) = n\left(\left\|T\left(\frac{u_n}{\|u_n\|}\right)\right\| + \left|\frac{u_n}{\|u_n\|}\right|\right) = n(\|Tx_n\| + |x_n|)
    \end{equation*}
    Con $\|x_n\| = 1\quad \forall n\in \mathbb{N}$. Podemos suponer sin pérdida de generalidad que $T$ es sobreyectiva, ya que como $Im T$ es un subespacio vectorial cerrado de $F$ tendremos que $Im T$ también es de Banach. Bajo esta suposición tenemos por el Teorema de la aplicación abierta que $T$ es abierta, con lo que existe $r\in \mathbb{R}^+$ de forma que $B(0,r)\subset T(B(0,1))$. De la desigualdad superior vemos que:
    \begin{equation*}
        \frac{1}{n} > \|Tx_n\| + |x_n| \Longrightarrow \|Tx_n\|,|x_n| \leq \frac{1}{n} \qquad \forall n\in \mathbb{N}
    \end{equation*}
    De esta forma, vemos que:
    \begin{equation*}
        Tx_n \in B\left(0,\frac{1}{n}\right) = \frac{1}{n}B(0,1) = \frac{1}{nr}B(0,r) \subset \frac{1}{nr}T(B(0,1)) = T\left(B\left(0,\frac{1}{nr}\right)\right)
    \end{equation*}
    Por lo que para cada $n\in \mathbb{N}$, existe $y_n\in B\left(0,\frac{1}{nr}\right)$ tal que:
    \begin{equation*}
        Ty_n = Tx_n
    \end{equation*}
    Tenemos entonces que $\{\|y_n\|\}\to 0$. Si definimos:
    \begin{equation*}
        z_n = x_n - y_n \in \ker T \qquad \forall n\in \mathbb{N}
    \end{equation*}
    Tenemos que esta sucesión verifica:
    \begin{itemize}
        \item $\|z_n\| = \|x_n - y_n\| \leq \|x_n\| + \|y_n\| < 1+ \frac{1}{nr}$ por lo que $\{\|z_n\|\}\to 1$.
        \item $|z_n| = |x_n - y_n| \leq |x_n| + |y_n| < \frac{1}{n} + M\|y_n\|$, por lo que $\{|z_n|\}\to 0$.
    \end{itemize}
    Como $dim \ker T < \infty$ tenemos que las normas $\|\cdot \|$ y $|\cdot |$ son equivalentes, pero sin embargo su convergencia para la sucesión $\{z_n\}$ no es la misma, hemos llegado a una \underline{contradicción}.
\end{ejercicio}

\begin{ejercicio}
    Sean $E$ y $F$ dos espacios de Banach, probar que el conjunto
    \begin{equation*}
        \Omega = \{T\in L(E,F) : T \text{\ admite una inversa por la izquierda}\}
    \end{equation*}
    es abierto en $L(E,F)$. \newline(\textbf{Pista:} probar primero que el conjunto
    \begin{equation*}
        \cc{O} = \{T\in L(E,F): T \text{\ es biyectiva}\}
    \end{equation*}
    es abierto en $L(E,F)$.) 
\end{ejercicio}

\begin{ejercicio}
\end{ejercicio}

\begin{ejercicio}
    Sean $E_1, E_2$ y $F$ tres espacios de Banach, consideramos $T_1\in L(E_1, F)$ y $T_2\in L(E_2,F)$ de forma que:
    \begin{equation*}
        ImT_1 \cap ImT_2 = \{0\} \qquad \text{y}\qquad ImT_1 + ImT_2 = F
    \end{equation*}
    Prueba que $ImT_1$ y $ImT_2$ son cerrados.
\end{ejercicio}

\begin{ejercicio}
    Sea $E$ un espacio de Banach y sean $G$ y L dos subespacios cerrados de $E$. Si existe una constante $C$ de forma que:
    \begin{equation*}
        dist(x,G\cap L) \leq C dist(x,L) \qquad \forall x\in G
    \end{equation*}
    Prueba que $G+L$ es cerrado.
\end{ejercicio}
