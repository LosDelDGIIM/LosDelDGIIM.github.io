\section{Principio de acotación uniforme y Tª de la gráfica cerrada}
\begin{ejercicio}
    % (Continuidad de funciones convexas)\newline
    % Sea $E$ un espacio de Banach y sea $\varphi:E\to \left]-\infty,+\infty\right]$ 
\end{ejercicio}

\begin{ejercicio}
    Sea $E$ un espacio vectorial y sea $p:E\to \mathbb{R}$ una función que cumple las siguientes propiedades:
    \begin{enumerate}
        \item $p(x+y)\leq p(x)+p(y)\quad \forall x,y\in E$.
        \item Para cada $x\in E$ fijo la función $\lm\longmapsto p(\lm x)$ es continua.
        \item Siempre que una sucesión de puntos de $E$ $\{y_n\}$ verifique que $\{p(y_n)\}\to 0$, entonces $\{p(\lm y_n)\}\to 0$ para cada $\lm \in \mathbb{R}$.
    \end{enumerate}
    Supongamo que $\{x_n\}$ es una sucesión de puntos de $E$ de forma que $\{p(x_n)\}\to 0$ y $\{\alpha_n\}$ es una sucesión de números reales acotada. Probar que $p(0)=0$ y que $\{p(\alpha_n x_n)\}\to 0$.\newline
    (\textbf{Pista:} Dado $\varepsilon>0$, considera los conjuntos:
    \begin{equation*}
        F_n = \{\lm \in \mathbb{R} : |p(\lm x_k)| \leq \varepsilon\quad \forall k\geq n\}
    \end{equation*}
    Deduce que si $\{x_n\}$ es una sucesión de $E$ de forma que $\{p(x_n-x)\}\to 0$ para algún $x\in E$ y $\{\alpha_n\}$ es una sucesión de forma que $\{\alpha_n\}\to \alpha$, entonces $\{p(\alpha_n x_n)\}\to p(\alpha x)$.)\\

    \noindent
    Usaando que $\{x_n\}$ y 3 con $\lm = 0$ obtenemos que: % // TODO: suponer en 2 que \lm \neq 0
    \begin{equation*}
        \{p(0\cdot x_n)\} = \{p(0)\} \to 0
    \end{equation*}
    de donde $p(0) = 0$. Siguiendo la pista, dado $\varepsilon>0$ definimos:
    \begin{equation*}
        F_n = \{\lm \in \mathbb{R} : |p(\lm x_k)| \leq \varepsilon\quad \forall k\geq n\}
    \end{equation*}
    Por reducción al absurdo, supongamos que $\{p(\alpha_n x_n)\}\not\to 0$, de donde existe una parcial con $|p(\alpha_{\sigma(n)}x_{\sigma(n)})|\geq \varepsilon$, para todo $n\in \mathbb{N}$. Como $\{\alpha_n\}$ está acotada, el Teorema de Weierstrass nos permite encontrar una parcial convergente. Supongmos que la parcial $\sigma$ verifica esto, con lo que $\{\alpha_{\sigma(n)}\}\to \alpha$. Observemos que:
    \begin{itemize}
        \item $F_n$ es cerrado para cada $n\in \mathbb{N}$, ya que:
            \begin{equation*}
                F_n = \bigcap_{k\geq n}\{\lm \in \mathbb{R} : |p(\lm x_k)|\leq \varepsilon\}
            \end{equation*}
            Como $(2)$ nos dice que $\lm \longmapsto p(\lm x_k)$ es continua, tenemos que cada uno de dichos conjuntos son cerrados, como preimagen de un conjunto cerrado por una función continua.
        \item Veamos que $\bigcup_{n\geq 1}F_n = \mathbb{R}$. Para ello, tomamos $\lm \in \mathbb{R}$ y como por hipótesis $\{p(\lm x_n)\}\to 0$, con lo que existe $n_0\in \mathbb{N}$ de forma que:
            \begin{equation*}
                |p(\lm x_n)| < \varepsilon \qquad \forall n\geq n_0
            \end{equation*}
            Luego $\lm\in  F_{n_0}$
    \end{itemize}
    Por el contrarrecíproco del Lema de Baire, existe $F_{\overline{n}}$ con $F_{\overline{n}}^\circ \neq \emptyset $, por lo que existen $\lm_0 \in \mathbb{R}$ y $\delta>0$ de forma que
    \begin{equation*}
        B(\lm_0, \delta)\subset F_{\overline{n}}^\circ
    \end{equation*}
    En otras palabras:
    \begin{equation*}
        (|p((\lm_0+t)x_{\sigma(k)})| \leq \varepsilon\quad \forall k\geq n_0 ) \quad \forall t\in \left]-\delta,\delta\right[
    \end{equation*}
    Ahora:
    \begin{equation*}
        p(\alpha_{\sigma(k)}x_{\sigma(k)})\leq p((\lm_0+\alpha_{\sigma(k)}-\alpha)x_{\sigma(k)}) + p((\alpha-\lm_0)x_{\sigma(k)})
    \end{equation*}
    con $\{ p((\alpha-\lm_0)x_{\sigma(k)})\}\to 0$ y podemos acotar el primer sumando en valor absoluto:
    \begin{equation*}
        |p((\lm_0+\alpha_{\sigma(k)}-\alpha)x_{\sigma(k)}) |\leq \varepsilon
    \end{equation*}
    Luego:
    \begin{equation*}
        p((\lm_0 + \alpha_{\sigma(k)} - \alpha)x_{\sigma(k)}) \leq p((\lm_0-\alpha)x_{\sigma(k)}) + p(\alpha_{\sigma(k)}x_{\sigma(k)})
    \end{equation*}
    de donde:
    \begin{equation*}
        p(x_{\sigma(k)}x_{\sigma(k)}) \geq p((\lm_0 + \alpha_{\sigma(k)}-\alpha)x_{\sigma(k)}) - p((\lm_0-\alpha)x_{\sigma(k)})
    \end{equation*}
    de forma que el segundo sumando tiende a 0 y el primero está acotado en valor absoluto por $\varepsilon$, de donde deducimos:
    \begin{equation*}
        p(\alpha_{\sigma(k)}x_{\sigma(k)}) \leq 2\varepsilon
    \end{equation*}
\end{ejercicio}

\begin{ejercicio}
    Sean $E$ y $F$ dos espacios de Banach y $\{T_n\}$ una sucesión en $L(E,F)$. Supongamos que para todo $x\in E$ se tiene que $\{T_nx\}$ converge a un cierto límite $Tx$. Probar que si $\{x_n\}\to x$ en $E$, entonces $\{T_n(x_n)\}\to Tx$ en $F$.\\

    \noindent
    Dado $x\in E$, como $\{T_nx\}$ es convergente a $Tx$, entonces $\{\|T_nx\|\}\to \|Tx\|$, luego:
    \begin{equation*}
        \sup_{n\in \mathbb{N}}\|T_nx\| < \infty
    \end{equation*}
    Por el Principio de acotación uniforme, tenemos que $C=\sup\limits_{n\in \mathbb{N}}\|Tn\| < \infty$, de donde:
    \begin{align*}
        \|T_nx_n - Tx\| &= \|T_nx_n - T_nx + T_nx - Tx\| \leq \|T_nx_n - T_nx\| + \|T_nx - Tx\| \\
                        &= \|T_n(x_n - x)\| + \|T_nx-Tx\| \leq C\|x_n-x\| + \|T_nx-Tx\|
    \end{align*}
    Como $\|x_n-x\|, \|T_nx-Tx\|\to 0$, tenemos pues que $\|T_nx_n - Tx\| \to 0$, de donde $\{T_n(x_n)\}\to Tx$.
\end{ejercicio}

\begin{ejercicio} % // TODO: Ejercicio de examen
    Sean $E,F$ dos espacios de Banach y sea $a:E\times F\to \mathbb{R}$ una forma bilineal que verifica:
    \begin{enumerate}
        \item Para cada $x\in E$, la aplicación $fx:y\longmapsto a(x,y)$ es continua.
        \item Para cada $y\in F$, la aplicación $f_y:x\longmapsto a(x,y)$ es continua.
    \end{enumerate}
    Probar que existe una constante $C\geq 0$ de forma que:
    \begin{equation*}
        |a(x,y)| \leq C\|x\|\|y\| \qquad \forall x\in E, \quad \forall y\in F
    \end{equation*}
    (\textbf{Pista:} Introduce un operador lineal $T:E\to F^\ast$ y prueba que $T$ está acotada con ayuda del Corolario~\ref{coro:entonces_Bast_acotado}).\\

    \begin{description}
        \item [Opción 1.] Es claro que $f_x$ es lineal, con lo que $f_x\in F^\ast$.
            \Func{T}{E}{F^\ast}{x}{f_x}
            Queremos ver que si $B\subset E$ es acotada entonces $T(B)$ es acotada. Para ello:
            \begin{equation*}
                \langle T(B),y \rangle  = \{\langle f_x,y \rangle :f_x\in T(B)\} = \{a(x,y):x\in B\} = \{f_y(x) : x\in B\}
            \end{equation*}

            de donde:
            \begin{equation*}
                |f_y(x)| \leq M \|x\| \qquad \forall x\in E
            \end{equation*}
            Luego por el Corolario~\ref{coro:entonces_Bast_acotado} tenemos que $T(B)$ está acotado, así como $T$ es continua, por lo que existe $C\geq 0$ tal que $\|T(x)\|\leq C\|x\|\quad \forall x\in E$. Luego:
            \begin{equation*}
                \|f_x\| = \sup_{\|y\|\leq 1}|f_x(y)| \leq C\|x\|
            \end{equation*}

            de donde:
            \begin{equation*}
                \left|a\left(x,\frac{y}{\|y\|}\right)\right| \leq C\|x\| \Longrightarrow a(x,y) \leq C\|x\|\|y\|
            \end{equation*}
        \item [Opción 2.] Definimos:
            \Func{T}{E}{F^\ast}{x}{f_x}
            Es claro que $T$ es lineal por ser $a$ bilineal. Lo que haremos será probar que $T$ es continua tratando de usar el Teorema de la Gráfica cerrada. Recordamos que:
            \begin{equation*}
                Gr(T) = \{(x,Tx) : x\in E\}
            \end{equation*}
            Para ver que $Gr(T)$ es cerrado, tomamos una sucesión de puntos de $Gr(T)$: $\{(x_n,Tx_n)\}$ que suponemso convergente a un punto $(x,w)\in Gr(T)$. Tenemos entonces que:
            \begin{equation*}
                \|x_n-x\|, \|Tx_n - w\| \to 0
            \end{equation*}
            De la segunda podemos deducir que $\langle Tx_n - w,y \rangle\to 0 \quad \forall y\in F$ :
            \begin{equation*}
                \langle Tx_n - w,y \rangle  = \langle Tx_n,y \rangle - \langle w-y \rangle  = a(x_n,y) - \langle w,y \rangle 
            \end{equation*}
            Por lo que $\{a(x_n,y)\}\to \langle w,y \rangle $, pero por la propiedad 2 se tiene que $\{a(x_n,y)\}\to a(x,y)$, de donde deducimos que $\langle T_x,y \rangle= a(x,y) = \langle w,y \rangle $ $\forall y\in F$, de donde deducimos que $w=T_x$, con lo que $(x,w)\in Gr(T)$, luego $Gr(T)$ es cerrada y por el Teorema de la gráfica cerrada concluimos que $T$ es continua.
    \end{description}
\end{ejercicio}

\begin{ejercicio}\label{ej:5_rel2}
    Sea $E$ un espacio de Banach y sea $\{\varepsilon_n\}$ una sucesión de reales positivos de forma que $\{\varepsilon_n\}\to 0$. Además, sea $\{f_n\}$ una sucesión de elementos de $E^\ast$ que cumple la propiedad:
    \begin{center}
        $\exists r>0~\forall x\in E$ con $\|x\|<r$, $\exists C(x)\in \mathbb{R}$ de forma que $\langle f_n,x \rangle \leq \varepsilon_n\|f_n\| + C(x)\qquad \forall n\in \mathbb{N}$
    \end{center}
    Prueba que la sucesión $\{f_n\}$ está acotada.\newline
    (\textbf{Pista:} Introduce $g_n = \frac{f_n}{(1+\varepsilon_n\|f_n\|)}$).
\end{ejercicio}

\begin{ejercicio}\label{ej:6_rel2}
    (Operadores no lineales monótonos y localmente acotados)\newline
    Sea $E$ un espacio de Banach y sea $D(A)$ cualquier subconjunto de $E$. Una aplicación (no lineal) $A:D(A)\subseteq E\to E^\ast$ se dice ``monótona'' si verifica
    \begin{equation*}
        \langle Ax - Ay,x-y \rangle \geq 0 \qquad \forall x,y\in D(A)
    \end{equation*}
    \begin{enumerate}
        \item Sea $x_0\in \Int D(A)$. Prueba que existen dos constantes $R>0$ y $C$ de forma que
            \begin{equation*}
                \|Ax\|\leq C \qquad \forall x\in D(A) \quad \text{con}\quad \|x-x_0\|<R
            \end{equation*}
            (\textbf{Pista:} Razona por reducción al absurdo y construye una sucesión $\{x_n\}$ de puntos de $D(A)$ de forma que $\{x_n\}\to x_0$ y $\{\|Ax_n\|\}\to \infty$. Elije $r>0$ de forma que $B(x_0,r)\subset D(A)$. Usa la monotonía de $A$ en $x_n$ y en $(x_0+r)$ con $\|x\|<r$. Aplica el Ejercicio~\ref{ej:5_rel2}).
        \item Prueba la misma conclusión para un punto $x_0\in \Int(\conv D(A))$.
        \item Extiende la conclusión de la pregunta 1 al caso de que $A$ sea multivaluada, es decir, para cada $x\in D(A)$, $Ax$ es un conjunto no vacío de $E^\ast$, en este caso la monotonía se define como sigue:
            \begin{equation*}
                \langle f-g,x-y \rangle \geq 0 \qquad \forall x,y\in D(A), \quad \forall f\in Ax, \quad \forall g\in Ay
            \end{equation*}
    \end{enumerate}

    \noindent
    \textbf{Solución.}
    \begin{enumerate}
        \item Sea $x_0\in \Int D(A)$. Prueba que existen dos constantes $R>0$ y $C$ de forma que
            \begin{equation*}
                \|Ax\| \leq C \qquad \forall x\in D(A) \quad \text{con}\quad \|x-x_0\|<R
            \end{equation*}

            Por reducción al absurdo, supongamos que:
            \begin{equation*}
                \forall R,C\geq 0\quad \exists x\in D(A) \quad \text{con}\quad \|x-x_0\|<R \quad \text{y}\quad  \|Ax\|>C
            \end{equation*}
            Por tanto, para cada $n\in \mathbb{N}$
            \begin{equation*}
                \exists x_n\in D(A) \quad \text{con}\quad \|x_n-x_0\|<\frac{1}{n} \quad \text{y}\quad  \|Ax_n\|>C
            \end{equation*}
            Por lo que $\{x_n\}\to x_0$ y $\{\|Ax_n\|\}\to \infty$. Como $x_0\in \Int D(A)$, sea $r>0$ de forma que $B(x_0,r)\subset D(A)$, si tomamos $x\in E$ con $\|x\|<r$ tendremos entonces que:
            \begin{equation*}
                \|x_0 + x - x_0\| = \|x\| < r \Longrightarrow x_0+x\in B(x_0,r)\subset D(A)
            \end{equation*}
            Como $A$ es monótona, tenemos que:
            \begin{align*}
                0&\leq \langle Ax_n-A(x_0+x),x_n-x_0-x \rangle  \\
                 &= \langle Ax_n, x \rangle  + \langle Ax_n, x_n-x_0 \rangle  + \langle A(x_0+x),x \rangle  + \langle A(x_0+x),x_0-x_n \rangle 
            \end{align*}

            de donde:
            \begin{align*}
                \langle Ax_n,x \rangle  &\leq \|Ax_n\| \|x_n-x_0\| + \|A(x_0+x)\|\|x\| + \|A(x_0+x)\|\|x_0-x_n\| \\
                                        &= \|Ax_n\| \|x_n - x_0\| + \|A(x_0+x)\|(\|x\| + \|x_0-x_n\|)\\
                                        &\stackrel{(\ast)}{\leq} \|Ax_n\| \underbrace{\|x_n - x_0\|}_{\varepsilon_n} + \underbrace{\|A(x_0+x)\|(\|x\| + 1)}_{C(x)}
            \end{align*}
            Donde en $(\ast)$ hemos usado que $\|x_n-x_0\|<1 \quad \forall n\in \mathbb{N}$. Aplicando el Ejercicio~\ref{ej:5_rel2}, tenemos que $\{Ax_n\}$ está acotada, contradicción con que $\{\|Ax_n\|\}\to \infty$, por lo que tenemos el primer apartado.
        \item Prueba la misma conclusión para un punto $x_0\in \Int(\conv D(A))$.

            Tenemos que $\conv D(A)$ es el cierre convexo de $D(A)$:
            \begin{equation*}
                \conv D(A) = \left\{\sum_{k=1}^{n}\lm_k v_k : v_k \in D(A), \text{\ con\ } \sum_{k=1}^{n}\lm_k = 1\right\}
            \end{equation*}
            Repetiremos la misma prueba que en el apartado 1, cambiando un poco el final. Por Reducción al absurdo, de la misma forma podemos encontrar $\{x_n\}\to x_0$ con $\|x_n-x_0\| \leq 1$ para todo $n\in \mathbb{N}$ y $\{\|Ax_n\|\}\to \infty$. Como $x_0\in \Int(\conv D(A))$, sea $r>0$ de forma que $B(x_0,r)\subset \conv D(A)$, si tomamos $x\in E$ con $\|x\|<r$ tendremos que $x_0+x\in \conv D(A)$, por lo que existen $\lm_1,\ldots,\lm_n\in \mathbb{R}$, $v_1,\ldots, v_n\in D(A)$ con:
            \begin{equation*}
                x_0 + x = \sum_{k=1}^{n} \lm_k v_k \qquad \sum_{k=1}^{n}\lm_k = 1
            \end{equation*}
            Fijado $k\in \{1,\ldots,n\}$, podemos usar la monotonía de $A$:
            \begin{equation*}
                0\leq \langle Ax_n - Av_k, x_n - v_k \rangle  \Longrightarrow \langle Ax_n,x_n-v_k \rangle  \geq \langle Av_k, x_n-v_k \rangle 
            \end{equation*}
            Por lo que:
            \begin{equation*}
                \lm_k \langle Ax_n,x_n-v_k \rangle  \geq \lm_k\langle Av_k, x_n-v_k \rangle \qquad \forall k \in \{1,\ldots,n\}
            \end{equation*}
            de donde:
            \begin{equation*}
                \sum_{k=1}^{n}\lm_k \langle Ax_n,x_n-v_k \rangle  \geq\sum_{k=1}^{n} \lm_k\langle Av_k, x_n-v_k \rangle 
            \end{equation*}
            Vemos que:
            \begin{align*}
                \sum_{k=1}^{n}\lm_k \langle Ax_n,x_n-v_k \rangle   &= \left\langle Ax_n, \sum_{k=1}^{n}(\lm_kx_n-\lm_kv_k) \right\rangle  = \left\langle Ax_n, x_n - \sum_{k=1}^{n}\lm_kv_k \right\rangle  \\ &= \langle Ax_n, x_n - x_0-x\rangle  = \langle Ax_n, x_n-x_0 \rangle  - \langle Ax_n, x \rangle 
            \end{align*}
            Por lo que:
            \begin{align*}
                \langle Ax_n,x \rangle &\leq \langle Ax_n, x_n-x_0 \rangle + \sum_{k=1}^{n}\lm_k\langle Av_k, v_k - x_n \rangle  \\
                                       &\leq \|Ax_n\|\|x_n-x_0\| + \sum_{k=1}^{n}\lm_k \|Av_k\|\|v_k-x_n\|
            \end{align*}
            Pero:
            \begin{equation*}
                \|v_k - x_n\| \leq \|v_k - x_0\| + \|x_0 - x_n\| \leq \|v_k - x_0\| + 1
            \end{equation*}
            Por lo que:
            \begin{align*}
                \langle Ax_n,x \rangle &\leq \langle Ax_n, x_n-x_0 \rangle + \sum_{k=1}^{n}\lm_k\langle Av_k, v_k - x_n \rangle  \\
                                       &\leq \|Ax_n\|\|x_n-x_0\| + \sum_{k=1}^{n}\lm_k \|Av_k\|\|v_k-x_n\| \\
                                       &\leq \|Ax_n\|\underbrace{\|x_n-x_0\|}_{\varepsilon_n} + \underbrace{\sum_{k=1}^{n}\lm_k \|Av_k\|(\|v_k-x_0\|+1)}_{C(x)}
            \end{align*}
            Lo que nuevamente lleva a contradicción.
        \item Extiende la conclusión de la pregunta 1 al caso de que $A$ sea multivaluada.

            En este caso, queremos probar que existen $R>0,C\geq 0$ de forma que:
            \begin{equation*}
                \text{Si}\quad  x\in D(A) \quad \text{con}\quad \|x-x_0\|<R \Longrightarrow \|f\| \leq C \qquad \forall f\in Ax
            \end{equation*}
            Por reducción al absurdo, suponemos que
            \begin{equation*}
                \forall R,C\geq 0 \quad \exists  x\in D(A) \quad \text{con}\quad \|x-x_0\|<R \quad \text{y}\quad f\in Ax \quad \text{con}\quad  \|f\| > C 
            \end{equation*}
            Por lo que para cada $n\in \mathbb{N}$ podemos tomar un elemento $x_n\in D(A)$ con $\|x_n-x_0\| < \frac{1}{n}$, y $f_n\in Ax_n$ con $\|f_n\|>C$. En definitiva, tenemos $\{x_n\}\to x_0$ y $\{\|f_n\|\}\to \infty$. Sea ahora $r>0$ de forma que $B(x_0,r)\subset D(A)$, entonces si $x\in E$ con $\|x\|<r$ tendremos que $x_0+x\in D(A)$. Si tomamos $g\in A(x_0+x)$ y usamos la monotonía de $A$:
            \begin{equation*}
                \langle f_n - g,x_n-x_0-x \rangle \geq 0
            \end{equation*}
            luego:
            \begin{align*}
                \langle f_n,x \rangle &\leq \langle f_n,x_n-x_0 \rangle  + \langle g,x+x_0-x_n \rangle  \leq \|f_n\|\|x_n-x_0\| + \|g\|\|x + x_0 - x_n\| \\
                                      &\leq \|f_n\|\|x_n-x_0\| + \|g\|(\|x\| + \|x_0-x_n\|) \leq \|f_n\|\underbrace{\|x_n-x_0\|}_{\varepsilon_n} + \underbrace{\|g\|(\|x\|+1)}_{C(x)}
            \end{align*}
            Lo que nos lleva a contradicción.
    \end{enumerate}
\end{ejercicio}

\begin{ejercicio}
    Sea $\alpha = \{\alpha_n\}$ una sucesión de números reales y sea $1\leq p \leq \infty$. Supongamos que $\sum_{n=1}^{\infty} |\alpha_n||x_n| < \infty$ para cada elemento $x=\{x_n\}$ de $l_p$. Prueba que $\alpha\in l_{p'}$.
\end{ejercicio}

\begin{ejercicio}
    Sea $E$ un espacio de Banach y sea $T:E\to E^\ast$ un operador lineal verificando que
    \begin{equation*}
        \langle Tx, x \rangle \geq 0 \qquad \forall x\in E
    \end{equation*}
    Prueba que $T$ es acotado.\newline
    (Se pueden aplicar dos métodos, bien usar el Ejercicio~\ref{ej:6_rel2}, bien aplicar el Teorema de la Gráfica Cerrada.)

    \begin{description}
        \item [Usando el Ejercicio~\ref{ej:6_rel2}.] Es fácil ver que $T$ es monótona, ya que:
            \begin{equation*}
                \langle Tx-Ty,x-y \rangle  = \langle T(x-y),x-y \rangle \geq 0 \qquad \forall x,y\in E
            \end{equation*}
            En dicho caso, por el Ejercicio~\ref{ej:6_rel2} tenemos que para todo $x\in E$ existen $R>0$, $C\geq 0$ de forma que:
            \begin{equation*}
                y\in E \quad \text{con}\quad \|y-x\|<R \Longrightarrow \|Ty\| \leq C
            \end{equation*}
            Es decir, si $y\in B(x,R)$ tenemos entonces que $\|Ty\| \leq C$. Sea ahora $z\in B(0,1)$, tenemos que:
            \begin{equation*}
                \|Tz\| = \dfrac{\|T(Rz)\|}{R} \leq \dfrac{C}{R}
            \end{equation*}
            Por lo que $T(B(0,1))$ es un conjunto acotado, y por una proposición vista en teoría concluimos que $T$ es acotada.
        \item [Usando el Teorema de la Gráfica Cerrada.]  % // TODO: 
    \end{description}
\end{ejercicio}

\begin{ejercicio}
    Sea $E$ un espacio de Banach y sea $T:E\to E^\ast$ un operador lineal verificando 
    \begin{equation*}
        \langle Tx,y \rangle  = \langle Ty,x \rangle  \qquad \forall x,y\in E
    \end{equation*}
    Prueba que $T$ es acotado.
\end{ejercicio}

\begin{ejercicio}
    Sean $E$ y $F$ dos espacios de Banach y sea $T\in L(E,F)$ una aplicación sobreyectiva.
    \begin{enumerate}
        \item Sea $M\subset E$. Prueba que $T(M)$ es cerrado en $F$ si y solo si $M+N(T)$ es cerrado en $E$.

            Donde $N(T)$ es el núcleo de $T$, por doble inclusión tenemos que:
            \begin{description}
                \item [$\Longleftarrow )$] Si $M+N(T)$ es cerrado, entonces $E\setminus(M+N(T))$ es abierto, y como:
                    \begin{equation*}
                        T(E\setminus(M+N(T))) = T(E)\setminus T(M+N(T)) = F\setminus T(M)
                    \end{equation*}
                    donde hemos usado que $T$ es sobreyectiva, usamos además el Teorema de la Aplicación Abierta, obteniendo que $F\setminus T(M)$ es abierto, por lo que $T(M)$ es cerrado.
                \item [$\Longrightarrow )$] Si $T(M)$ es cerrado, por ser $T$ continua tenemos que $T^{-1}(T(M))$ es cerrado, y:
                    \begin{equation*}
                        T^{-1}(T(M)) = M+N(T)
                    \end{equation*}
                    \begin{description}
                        \item [$\subseteq )$] Si $x+n\in M+N(T)$, entonces:
                            \begin{equation*}
                                T(x+n) = T(x)+T(n) = T(x) \in T(M)
                            \end{equation*}
                            Por lo que $x+n\in T^{-1}(T(x))$.
                        \item [$\supseteq )$] Si $x\in T^{-1}(T(M))$, entonces:
                            \begin{equation*}
                                T(x) \in T(M) \Longrightarrow \exists m\in M \text{\ con\ } T(x) = T(m)
                            \end{equation*}
                            de donde $T(x-m) = T(x)-T(m) = 0$. En conclusión, tenemos que:
                            \begin{equation*}
                                x = m + x - m
                            \end{equation*}
                            con $m\in M$, $x-m\in N(T)$.
                    \end{description}
            \end{description}
        \item Deduce que si $M$ es un espacio vectorial cerrado de $E$ y si $dim N(T)<\infty$, entonces $T(M)$ es cerrado.

            Si $dim N(T)<\infty$ tenemos entonces que $N(T)$ es un espacio vectorial de dimensión finita, luego es un conjunto cerrado. Como Además, $M$ es un espacio vectorial cerrado, tendremos que $M+N(T)$ es cerrado, luego $T(M)$ será cerrado por el primer apartado.
    \end{enumerate}
\end{ejercicio}
