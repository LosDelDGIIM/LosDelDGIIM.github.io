\documentclass[12pt]{article}

% Idioma y codificación
\usepackage[spanish, es-tabla]{babel}       %es-tabla para que se titule "Tabla"
\usepackage[utf8]{inputenc}

% Márgenes
\usepackage[a4paper,top=3cm,bottom=2.5cm,left=3cm,right=3cm]{geometry}

% Comentarios de bloque
\usepackage{verbatim}

% Paquetes de links
\usepackage[hidelinks]{hyperref}    % Permite enlaces
\usepackage{url}                    % redirecciona a la web

% Más opciones para enumeraciones
\usepackage{enumitem}

% Personalizar la portada
\usepackage{titling}

% Paquetes de tablas
\usepackage{multirow}


%------------------------------------------------------------------------

%Paquetes de figuras
\usepackage{caption}
\usepackage{subcaption} % Figuras al lado de otras
\usepackage{float}      % Poner figuras en el sitio indicado H.


% Paquetes de imágenes
\usepackage{graphicx}       % Paquete para añadir imágenes
\usepackage{transparent}    % Para manejar la opacidad de las figuras

% Paquete para usar colores
\usepackage[dvipsnames]{xcolor}
\usepackage{pagecolor}      % Para cambiar el color de la página

% Habilita tamaños de fuente mayores
\usepackage{fix-cm}

% Para los gráficos
\usepackage{tikz}

% Para poder situar los nodos en los grafos
\usetikzlibrary{positioning}


%------------------------------------------------------------------------

% Paquetes de matemáticas
\usepackage{mathtools, amsfonts, amssymb, mathrsfs}
\usepackage[makeroom]{cancel}     % Simplificar tachando
\usepackage{polynom}    % Divisiones y Ruffini
\usepackage{units} % Para poner fracciones diagonales con \nicefrac

\usepackage{pgfplots}   %Representar funciones
\pgfplotsset{compat=1.18}  % Versión 1.18

\usepackage{tikz-cd}    % Para usar diagramas de composiciones
\usetikzlibrary{calc}   % Para usar cálculo de coordenadas en tikz

%Definición de teoremas, etc.
\usepackage{amsthm}
%\swapnumbers   % Intercambia la posición del texto y de la numeración

\theoremstyle{plain}

\makeatletter
\@ifclassloaded{article}{
  \newtheorem{teo}{Teorema}[section]
}{
  \newtheorem{teo}{Teorema}[chapter]  % Se resetea en cada chapter
}
\makeatother

\newtheorem{coro}{Corolario}[teo]           % Se resetea en cada teorema
\newtheorem{prop}[teo]{Proposición}         % Usa el mismo contador que teorema
\newtheorem{lema}[teo]{Lema}                % Usa el mismo contador que teorema

\theoremstyle{remark}
\newtheorem*{observacion}{Observación}

\theoremstyle{definition}

\makeatletter
\@ifclassloaded{article}{
  \newtheorem{definicion}{Definición} [section]     % Se resetea en cada chapter
}{
  \newtheorem{definicion}{Definición} [chapter]     % Se resetea en cada chapter
}
\makeatother

\newtheorem*{notacion}{Notación}
\newtheorem*{ejemplo}{Ejemplo}
\newtheorem*{ejercicio*}{Ejercicio}             % No numerado
\newtheorem{ejercicio}{Ejercicio} [section]     % Se resetea en cada section


% Modificar el formato de la numeración del teorema "ejercicio"
\renewcommand{\theejercicio}{%
  \ifnum\value{section}=0 % Si no se ha iniciado ninguna sección
    \arabic{ejercicio}% Solo mostrar el número de ejercicio
  \else
    \thesection.\arabic{ejercicio}% Mostrar número de sección y número de ejercicio
  \fi
}


% \renewcommand\qedsymbol{$\blacksquare$}         % Cambiar símbolo QED
%------------------------------------------------------------------------

% Paquetes para encabezados
\usepackage{fancyhdr}
\pagestyle{fancy}
\fancyhf{}

\newcommand{\helv}{ % Modificación tamaño de letra
\fontfamily{}\fontsize{12}{12}\selectfont}
\setlength{\headheight}{15pt} % Amplía el tamaño del índice


%\usepackage{lastpage}   % Referenciar última pag   \pageref{LastPage}
\fancyfoot[C]{\thepage}

%------------------------------------------------------------------------

% Conseguir que no ponga "Capítulo 1". Sino solo "1."
\makeatletter
\@ifclassloaded{book}{
  \renewcommand{\chaptermark}[1]{\markboth{\thechapter.\ #1}{}} % En el encabezado
    
  \renewcommand{\@makechapterhead}[1]{%
  \vspace*{50\p@}%
  {\parindent \z@ \raggedright \normalfont
    \ifnum \c@secnumdepth >\m@ne
      \huge\bfseries \thechapter.\hspace{1em}\ignorespaces
    \fi
    \interlinepenalty\@M
    \Huge \bfseries #1\par\nobreak
    \vskip 40\p@
  }}
}
\makeatother

%------------------------------------------------------------------------
% Paquetes de cógido
\usepackage{minted}
\renewcommand\listingscaption{Código fuente}

\usepackage{fancyvrb}
% Personaliza el tamaño de los números de línea
\renewcommand{\theFancyVerbLine}{\small\arabic{FancyVerbLine}}

% Estilo para C++
\newminted{cpp}{
    frame=lines,
    framesep=2mm,
    baselinestretch=1.2,
    linenos,
    escapeinside=||
}

% para minted
\definecolor{LightGray}{rgb}{0.95,0.95,0.92}
\setminted{
    linenos=true,
    stepnumber=5,
    numberfirstline=true,
    autogobble,
    breaklines=true,
    breakautoindent=true,
    breaksymbolleft=,
    breaksymbolright=,
    breaksymbolindentleft=0pt,
    breaksymbolindentright=0pt,
    breaksymbolsepleft=0pt,
    breaksymbolsepright=0pt,
    fontsize=\footnotesize,
    bgcolor=LightGray,
    numbersep=10pt
}


\usepackage{listings} % Para incluir código desde un archivo

\renewcommand\lstlistingname{Código Fuente}
\renewcommand\lstlistlistingname{Índice de Códigos Fuente}

% Definir colores
\definecolor{vscodepurple}{rgb}{0.5,0,0.5}
\definecolor{vscodeblue}{rgb}{0,0,0.8}
\definecolor{vscodegreen}{rgb}{0,0.5,0}
\definecolor{vscodegray}{rgb}{0.5,0.5,0.5}
\definecolor{vscodebackground}{rgb}{0.97,0.97,0.97}
\definecolor{vscodelightgray}{rgb}{0.9,0.9,0.9}

% Configuración para el estilo de C similar a VSCode
\lstdefinestyle{vscode_C}{
  backgroundcolor=\color{vscodebackground},
  commentstyle=\color{vscodegreen},
  keywordstyle=\color{vscodeblue},
  numberstyle=\tiny\color{vscodegray},
  stringstyle=\color{vscodepurple},
  basicstyle=\scriptsize\ttfamily,
  breakatwhitespace=false,
  breaklines=true,
  captionpos=b,
  keepspaces=true,
  numbers=left,
  numbersep=5pt,
  showspaces=false,
  showstringspaces=false,
  showtabs=false,
  tabsize=2,
  frame=tb,
  framerule=0pt,
  aboveskip=10pt,
  belowskip=10pt,
  xleftmargin=10pt,
  xrightmargin=10pt,
  framexleftmargin=10pt,
  framexrightmargin=10pt,
  framesep=0pt,
  rulecolor=\color{vscodelightgray},
  backgroundcolor=\color{vscodebackground},
}

%------------------------------------------------------------------------

% Comandos definidos
\newcommand{\bb}[1]{\mathbb{#1}}
\newcommand{\cc}[1]{\mathcal{#1}}

% I prefer the slanted \leq
\let\oldleq\leq % save them in case they're every wanted
\let\oldgeq\geq
\renewcommand{\leq}{\leqslant}
\renewcommand{\geq}{\geqslant}

% Si y solo si
\newcommand{\sii}{\iff}

% Letras griegas
\newcommand{\eps}{\epsilon}
\newcommand{\veps}{\varepsilon}
\newcommand{\lm}{\lambda}

\newcommand{\ol}{\overline}
\newcommand{\ul}{\underline}
\newcommand{\wt}{\widetilde}
\newcommand{\wh}{\widehat}

\let\oldvec\vec
\renewcommand{\vec}{\overrightarrow}

% Derivadas parciales
\newcommand{\del}[2]{\frac{\partial #1}{\partial #2}}
\newcommand{\Del}[3]{\frac{\partial^{#1} #2}{\partial #3^{#1}}}
\newcommand{\deld}[2]{\dfrac{\partial #1}{\partial #2}}
\newcommand{\Deld}[3]{\dfrac{\partial^{#1} #2}{\partial #3^{#1}}}


\newcommand{\AstIg}{\stackrel{(\ast)}{=}}
\newcommand{\Hop}{\stackrel{L'H\hat{o}pital}{=}}

\newcommand{\red}[1]{{\color{red}#1}} % Para integrales, destacar los cambios.

% Método de integración
\newcommand{\MetInt}[2]{
    \left[\begin{array}{c}
        #1 \\ #2
    \end{array}\right]
}

% Declarar aplicaciones
% 1. Nombre aplicación
% 2. Dominio
% 3. Codominio
% 4. Variable
% 5. Imagen de la variable
\newcommand{\Func}[5]{
    \begin{equation*}
        \begin{array}{rrll}
            #1:& #2 & \longrightarrow & #3\\
               & #4 & \longmapsto & #5
        \end{array}
    \end{equation*}
}

%------------------------------------------------------------------------


\newcommand{\norma}[1]{\lVert #1 \rVert}
\newcommand{\normagenerica}{\lVert \cdot \rVert}
\newcommand{\dual}[1]{#1^{*}}
\newcommand{\prodescalar}[2]{\langle #1, #2 \rangle}

\begin{document}

    % 1. Foto de fondo
    % 2. Título
    % 3. Encabezado Izquierdo
    % 4. Color de fondo
    % 5. Coord x del titulo
    % 6. Coord y del titulo
    % 7. Fecha

    
    % 1. Foto de fondo
% 2. Título
% 3. Encabezado Izquierdo
% 4. Color de fondo
% 5. Coord x del titulo
% 6. Coord y del titulo
% 7. Fecha

\newcommand{\portada}[7]{

    \portadaBase{#1}{#2}{#3}{#4}{#5}{#6}{#7}
    \portadaBook{#1}{#2}{#3}{#4}{#5}{#6}{#7}
}

\newcommand{\portadaExamen}[7]{

    \portadaBase{#1}{#2}{#3}{#4}{#5}{#6}{#7}
    \portadaArticle{#1}{#2}{#3}{#4}{#5}{#6}{#7}
}




\newcommand{\portadaBase}[7]{

    % Tiene la portada principal y la licencia Creative Commons
    
    % 1. Foto de fondo
    % 2. Título
    % 3. Encabezado Izquierdo
    % 4. Color de fondo
    % 5. Coord x del titulo
    % 6. Coord y del titulo
    % 7. Fecha
    
    
    \thispagestyle{empty}               % Sin encabezado ni pie de página
    \newgeometry{margin=0cm}        % Márgenes nulos para la primera página
    
    
    % Encabezado
    \fancyhead[L]{\helv #3}
    \fancyhead[R]{\helv \nouppercase{\leftmark}}
    
    
    \pagecolor{#4}        % Color de fondo para la portada
    
    \begin{figure}[p]
        \centering
        \transparent{0.3}           % Opacidad del 30% para la imagen
        
        \includegraphics[width=\paperwidth, keepaspectratio]{assets/#1}
    
        \begin{tikzpicture}[remember picture, overlay]
            \node[anchor=north west, text=white, opacity=1, font=\fontsize{60}{90}\selectfont\bfseries\sffamily, align=left] at (#5, #6) {#2};
            
            \node[anchor=south east, text=white, opacity=1, font=\fontsize{12}{18}\selectfont\sffamily, align=right] at (9.7, 3) {\textbf{\href{https://losdeldgiim.github.io/}{Los Del DGIIM}}};
            
            \node[anchor=south east, text=white, opacity=1, font=\fontsize{12}{15}\selectfont\sffamily, align=right] at (9.7, 1.8) {Doble Grado en Ingeniería Informática y Matemáticas\\Universidad de Granada};
        \end{tikzpicture}
    \end{figure}
    
    
    \restoregeometry        % Restaurar márgenes normales para las páginas subsiguientes
    \pagecolor{white}       % Restaurar el color de página
    
    
    \newpage
    \thispagestyle{empty}               % Sin encabezado ni pie de página
    \begin{tikzpicture}[remember picture, overlay]
        \node[anchor=south west, inner sep=3cm] at (current page.south west) {
            \begin{minipage}{0.5\paperwidth}
                \href{https://creativecommons.org/licenses/by-nc-nd/4.0/}{
                    \includegraphics[height=2cm]{assets/Licencia.png}
                }\vspace{1cm}\\
                Esta obra está bajo una
                \href{https://creativecommons.org/licenses/by-nc-nd/4.0/}{
                    Licencia Creative Commons Atribución-NoComercial-SinDerivadas 4.0 Internacional (CC BY-NC-ND 4.0).
                }\\
    
                Eres libre de compartir y redistribuir el contenido de esta obra en cualquier medio o formato, siempre y cuando des el crédito adecuado a los autores originales y no persigas fines comerciales. 
            \end{minipage}
        };
    \end{tikzpicture}
    
    
    
    % 1. Foto de fondo
    % 2. Título
    % 3. Encabezado Izquierdo
    % 4. Color de fondo
    % 5. Coord x del titulo
    % 6. Coord y del titulo
    % 7. Fecha


}


\newcommand{\portadaBook}[7]{

    % 1. Foto de fondo
    % 2. Título
    % 3. Encabezado Izquierdo
    % 4. Color de fondo
    % 5. Coord x del titulo
    % 6. Coord y del titulo
    % 7. Fecha

    % Personaliza el formato del título
    \pretitle{\begin{center}\bfseries\fontsize{42}{56}\selectfont}
    \posttitle{\par\end{center}\vspace{2em}}
    
    % Personaliza el formato del autor
    \preauthor{\begin{center}\Large}
    \postauthor{\par\end{center}\vfill}
    
    % Personaliza el formato de la fecha
    \predate{\begin{center}\huge}
    \postdate{\par\end{center}\vspace{2em}}
    
    \title{#2}
    \author{\href{https://losdeldgiim.github.io/}{Los Del DGIIM}}
    \date{Granada, #7}
    \maketitle
    
    \tableofcontents
}




\newcommand{\portadaArticle}[7]{

    % 1. Foto de fondo
    % 2. Título
    % 3. Encabezado Izquierdo
    % 4. Color de fondo
    % 5. Coord x del titulo
    % 6. Coord y del titulo
    % 7. Fecha

    % Personaliza el formato del título
    \pretitle{\begin{center}\bfseries\fontsize{42}{56}\selectfont}
    \posttitle{\par\end{center}\vspace{2em}}
    
    % Personaliza el formato del autor
    \preauthor{\begin{center}\Large}
    \postauthor{\par\end{center}\vspace{3em}}
    
    % Personaliza el formato de la fecha
    \predate{\begin{center}\huge}
    \postdate{\par\end{center}\vspace{5em}}
    
    \title{#2}
    \author{\href{https://losdeldgiim.github.io/}{Los Del DGIIM}}
    \date{Granada, #7}
    \thispagestyle{empty}               % Sin encabezado ni pie de página
    \maketitle
    \vfill
}
    \portadaExamen{ffccA4.jpg}{Análisis Funcional\\Examen XII}{Análisis Funcional. Examen XII}{MidnightBlue}{-8}{28}{2025}{}

    \begin{description}
        \item[Asignatura] Análisis Funcional.
        \item[Curso Académico] 2025-26.
        \item[Grado] Doble Grado en Ingeniería Informática y Matemáticas.
        \item[Grupo] Único.
        \item[Profesor] David Arcoya Álvarez.
        \item[Descripción] Primer Parcial.
        \item[Fecha] 19 de Noviembre de 2025.
        \item[Duración] 2 horas.
    
    \end{description}
    \newpage


    % ------------------------------------
    
    \begin{ejercicio}[3.5 Puntos]
        Sea $E$ un espacio normado, $M \subset E$ un subespacio vectorial y $x_0 \in E$ verificando
        $$d:= \inf_{y \in M} \norma{x_0 - y} > 0.$$
        Prueba que existe un funcional $f \in \dual{E}$ tal que 
        $$\prodescalar{f}{x_0} = d, \quad \norma{f} = 1 \quad \text{ y } \quad \prodescalar{f}{y} = 0 \quad \forall y \in M$$
    \end{ejercicio}

    \begin{ejercicio}[3.5 Puntos]
        Sean $(E, \normagenerica_E)$ y $(F, \normagenerica_F)$ dos espacios de Banach y $T : E \to F$ una aplicación lineal. Prueba que si
        $$G(T) = \{(x,Tx) : x \in E\} \quad \text{ y } \quad \norma{x}_1 = \norma{x}_E + \norma{Tx}_F,\footnote{En el examen se
        aclaró que si se necesita usar que $\normagenerica_1$ es una norma, hay que demostrarlo.}$$
        entonces
        $$G(T) \text{ es cerrado} \iff (E, \normagenerica_1) \text{ es un espacio de Banach.}$$
    \end{ejercicio}

    \begin{ejercicio}[3 Puntos]
        Enunciar y demostrar el Teorema de Riesz-Fréchet.
    \end{ejercicio}

    \newpage
    \setcounter{ejercicio}{0}
    \noindent
    \textbf{Solución.}

    \begin{ejercicio}[3.5 Puntos]
        Sea $E$ un espacio normado, $M \subset E$ un subespacio vectorial y $x_0 \in E$ verificando
        $$d:= \inf_{y \in M} \norma{x_0 - y} > 0.$$
        Prueba que existe un funcional $f \in \dual{E}$ tal que 
        $$\prodescalar{f}{x_0} = d, \quad \norma{f} = 1 \quad \text{ y } \quad \prodescalar{f}{y} = 0 \quad \forall y \in M$$

        \noindent
        \begin{description}
            \item [Opción 1.] Observemos que como $d>0$ tenemos que $x_0\notin M$. Más aún, veamos que $B(x_0,d)\cap M = \emptyset $, pues si existiera $y_0\in B(x_0,d)\cap M$ tendríamos entonces que $y_0\in M$ con $\|x_0-y_0\| < d$, lo que contradice la definición de $d$. Como $B(x_0,d)$ y $M$ son conjuntos convexos y $B(x_0,d)$ es abierto, podemos aplicar la primera versión geométrica del Teorema de Hahn-Banach, obteniendo $\alpha\in \mathbb{R}$ y una aplicación $g\in E^\ast$ no constantemente igual a $0$ de forma que:
                \begin{equation*}
                    g(y) \leq \alpha \leq g(x_0 + dz) \qquad \forall y\in M,\quad \forall z\in B(0,1)
                \end{equation*}
                De la primera desigualdad deducimos\footnote{Ya se ha usado varias veces este argumento.} que $g(y) = 0 \quad \forall y\in M$, puesto que $M$ es un espacio vectorial y $g\big|_M:M\to \mathbb{R}$ es una aplicación lineal cuya imagen está acotada por $\alpha$. Tenemos por tanto que:
                \begin{equation*}
                    0 \leq g(x_0 + dz) = g(x_0) + dg(z) \qquad \forall z\in B(0,1)
                \end{equation*}
                de donde tomando $-z$ en lugar de $z$ vemos que:
                \begin{equation*}
                    0 \leq g(x_0) - dg(z) \qquad \forall z\in B(0,1)
                \end{equation*}
                Deducimos por tanto que:
                \begin{equation*}
                    d\|g\| \leq g(x_0)
                \end{equation*}
                Para la otra desigualdad, por la definición de $d$ podemos encontrar $\{y_n\}$ sucesión de puntos de $M$ de forma que $\{\|x_0-y_n\|\}\to d$, por lo que para cada $n\in \mathbb{N}$ tenemos:
                \begin{equation*}
                    g(x_0) = g(x_0 - y_n) \leq \|g\|\|x_0 - y_n\|
                \end{equation*}
                y como $\{\|g\|\|x_0-y_n\|\}\to \|g\|d$ concluimos que $g(x_0)\leq \|g\|d$, por lo que ha de ser:
                \begin{equation*}
                    g(x_0) = d\|g\|
                \end{equation*}
                Como $g\neq 0$ tenemos que $\|g\|\neq 0$, por lo que si tomamos ahora $f = \frac{g}{\|g\|}\in E^\ast$, tenemos que $\|f\| = 1$, así como que:
                \begin{equation*}
                    f(y) = \frac{g(y)}{\|g\|} = 0 \qquad \forall y\in M, \qquad 
                    f(x_0) = \frac{g(x_0)}{\|g\|} = \frac{d\|g\|}{\|g\|} = d
                \end{equation*}
            \item [Opción 2.] Como $d>0$ tenemos entonces que $x_0\notin M$ (ya que si $x_0\in M$ tendríamos entonces que $\inf\limits_{y\in M}\|x_0-y\| = \|x_0-x_0\| = 0$), por lo que si tomamos $E_0=\cc{L}\{x_0\}$ tendremos entonces que $E_0\cap M = \{0\}$. Si tomamos $E_1 = E_0\oplus M$ podemos definir la aplicación lineal $g:E_1\to \mathbb{R}$ dada por:
                \begin{equation*}
                    g(tx_0 + y) = td \qquad \forall t\in \mathbb{R}, \quad \forall y\in M
                \end{equation*}
                Observemos que si $y\in M$ y $t\in \mathbb{R}^\ast$ tenemos entonces que $-\frac{y}{t}\in M$, así como que:
                \begin{equation*}
                    d \leq \left\|x_0 - \left(-\frac{y}{t}\right)\right\| = \frac{1}{|t|}\|tx_0 + y\| \qquad \forall y\in M
                 \end{equation*}
                 de donde deducimos que para todo $x=tx_0 + y\in E_1$ con $t\in \mathbb{R}$ y $y\in M$ se tiene:
                 \begin{equation*}
                     |g(x)| = |t|d \leq \|tx_0 + y\| = \|x\|
                 \end{equation*}
                 por lo que $g\in E_1^\ast$, con $\|g\|\leq 1$. Para ver que $\|g\|=1$, fijado $\varepsilon>0$ elegimos $y\in M$ de forma que $\|x_0-y\|<d+\varepsilon$ y tomando
                 \begin{equation*}
                     z = \frac{x_0-y}{\|x_0-y\|}
                 \end{equation*}
                 se tiene que:
                 \begin{equation*}
                     |g(z)| = \frac{|g(x_0-y)|}{\|x_0-y\|} = \frac{d}{\|x_0-y\|} > \frac{d}{d+\varepsilon} = \frac{d}{d+\varepsilon}\|z\|, \qquad \|g\|\geq \frac{d}{d+\varepsilon}
                 \end{equation*}
                 Como $\varepsilon>0$ era arbitrario, obtenemos que $\|g\|\geq 1$, de donde $\|g\| = 1$. Si aplicamos el Teorema de Hahn-Banach, obtenemos que existe $f\in E^\ast$ con $\|f\| = \|g\| = 1$ y $f\big|_{E_1} = g$, en particular:
                 \begin{equation*}
                     f(y) = g(y) = 0 \quad \forall y\in M, \qquad f(x_0) = g(x_0) = d
                 \end{equation*}
        \end{description}
    \end{ejercicio}

    \begin{ejercicio}[3.5 Puntos]
        Sean $(E, \normagenerica_E)$ y $(F, \normagenerica_F)$ dos espacios de Banach y $T : E \to F$ una aplicación lineal. Prueba que si
        $$G(T) = \{(x,Tx) : x \in E\} \quad \text{ y } \quad \norma{x}_1 = \norma{x}_E + \norma{Tx}_F,$$ entonces
        $$G(T) \text{ es cerrado} \iff (E, \normagenerica_1) \text{ es un espacio de Banach.}$$

        \noindent
        Veamos primero que $\|\cdot \|_1$ es una norma en $E$:
        \begin{itemize}
            \item Si $\|x\|_1 = 0$ entonces $\|x\|_E+\|Tx\|_F= 0$, por lo que como $\|x\|_E,\|Tx\|_F\geq 0$ tenemos entonces que $\|x\|_E = 0$, de donde $x=0$.
            \item Si $x,y\in E$ entonces:
                \begin{align*}
                    \|x+y\|_1 &= \|x+y\|_E + \|T(x+y)\|_F = \|x+y\|_E + \|Tx+Ty\|_F \\
                              &\leq \|x\|_E + \|Tx\|_F + \|y\|_E + \|Ty\|_F = \|x\|_1 + \|y\|_1
                \end{align*}
            \item Si $x\in E$ y $\lm\in \mathbb{R}$ entonces:
                \begin{align*}
                    \|\lm x\|_1 &= \|\lm x\|_E + \|T(\lm x)\|_F = \|\lm x\|_E + \|\lm Tx\|_F \\
                                &= |\lm| \|x\|_E + |\lm| \|Tx\|_F = |\lm| (\|x\|_E + \|Tx\|_F) = |\lm| \|x\|_1
                \end{align*}
        \end{itemize}
        \begin{description}
            \item [$\Longleftarrow )$] Para esta implicación:
                \begin{description}
                    \item [Opción 1.] Si $(E,\|\cdot \|_1)$ es de Banach, como $\|Tx\|_F\geq 0 \quad \forall x\in E$ tenemos entonces que:
                        \begin{equation*}
                            \|x\|_E \leq \|x\|_E + \|Tx\|_F = \|x\|_1 \qquad \forall x\in E
                        \end{equation*}
                        Como tanto $\|\cdot \|_E$ como $\|\cdot \|_1$ son completas en $E$ tenemos por un Corolario del Teorema de la aplicación abierta que existe $k\in \mathbb{R}^+_0$ de forma que:
                        \begin{equation*}
                            \|x\|_E + \|Tx\|_F = \|x\|_1 \leq k\|x\|_E \qquad \forall x\in E
                        \end{equation*}
                        Por lo que:
                        \begin{equation*}
                            \|Tx\|_F \leq (k-1)\|x\|_E \qquad \forall x\in E
                        \end{equation*}
                        Luego $T$ es continua, de donde $G(T)$ es cerrado.
                    \item [Opción 2.] Sea $\{(x_n,T(x_n))\}$ una sucesión en $G(T)$ que converge a $(x,y)$ en $E\times F$, si consideramos la norma:
                        \begin{equation*}
                            \|(x,y)\|_{E\times F} = \|x\|_E + \|y\|_F \qquad \forall (x,y)\in E\times F
                        \end{equation*}
                        tenemos entonces que la sucesión $\{(x_n,T(x_n))\}$ es de Cauchy:
                        \begin{equation*}
                            \forall \varepsilon>0~\exists n_0\in \mathbb{N} : m>n\geq n_0 \Longrightarrow \|x_n-x_m\|_E + \|Tx_n - Tx_m\|_F = \|x_n-x_m\|_1 < \varepsilon
                        \end{equation*}
                        Por lo que $\{x_n\}$ es de Cauchy en el espacio $(E,\|\cdot \|_1)$, y como este es de Banach, ha de ser la sucesión convergente hacia cierto punto $z\in E$. Si observamos que:
                        \begin{align*}
                            \|(z,Tz) - (x,y)\|_{E\times F} &\leq \|(z,Tz) - (x_n,Tx_n)\|_{E\times F} + \|(x,y) - (x_n,Tx_n)\|_{E\times F} \\
                                                           &= \|z-x_n\|_1 + \|(x,y)-(x_n,Tx_n)\|_{E\times F}
                        \end{align*}
                        las convergencias de $\{x_n\}$ en $(E,\|\cdot \|_1)$ y de $\{(x_n,Tx_n)\}$ en $E\times F$ nos dicen que:
                        \begin{equation*}
                            z=x, \qquad y=Tz = Tx
                        \end{equation*}
                        Por lo que $(x,y) = (x,Tx)\in G(T)$ y $G(T)$ es cerrado.
                \end{description}

            \item [$\Longrightarrow )$] Para esta otra implicación:
                \begin{description}
                    \item [Opción 1.] Para ver que $(E,\|\cdot \|_1)$ es de Banach, sea $\{x_n\}$ una sucesión de puntos de $E$ de Cauchy para $\|\cdot \|_1$ queremos ver que $\{x_n\}$ es convergente para $\|\cdot \|_1$. Para ello, como $\{x_n\}$ es de Cauchy para $\|\cdot \|_1$ tenemos que:
                        \begin{equation*}
                            \forall \varepsilon>0~\exists m\in \mathbb{N}:p,q\geq m \Longrightarrow \|x_p-x_q\|_1 < \varepsilon
                        \end{equation*}
                        Fijado $\varepsilon>0$ y considerando $m\in \mathbb{N}$ que nos da la condición de Cauchy, tenemos entonces que para $p,q\geq m$:
                        \begin{equation*}
                            \|x_p-x_q\|_E + \|Tx_p - Tx_q\|_F = \|x_p-x_q\|_E + \|T(x_p-x_q)\|_F = \|x_p - x_q\|_1 < \varepsilon
                        \end{equation*}
                        de donde deducimos que $\|x_p-x_q\|_E, \|Tx_p - Tx_q\|_F< \varepsilon$, por lo que $\{x_n\}$ es de Cauchy para $\|\cdot \|_E$ y $\{T(x_n)\}$ es de Cauchy en $F$. Como $(E,\|\cdot \|_E)$ y $(F,\|\cdot \|_F)$ son de Banach, existen $x\in E$ y $y\in F$ de forma que $\{x_n\}\to x$ para $\|\cdot \|_E$ y $\{T(x_n)\}\to y$ en $F$. Observemos que $\{(x_n,T(x_n))\}$ es una sucesión de puntos de $G(T)$ convergente a $(x,y)$, siendo $G(T)$ un conjunto cerrado, por lo que $(x,y)\in G(T)$, de donde $y = T(x)$. Si tomamos ahora $n_0\in \mathbb{N}$ de forma que para $n\geq n_0$ tengamos:
                        \begin{equation*}
                            \|x_n - x\|_E < \frac{\varepsilon}{2},\qquad \|T(x_n) - T(x)\|_F < \frac{\varepsilon}{2}
                        \end{equation*}
                        Tenemos entonces que para $n\neq n_0$:
                        \begin{equation*}
                            \|x_n - x\|_1 = \|x_n - x\|_E + \|T(x_n) - T(x)\|_F < \frac{\varepsilon}{2} + \frac{\varepsilon}{2} = \varepsilon
                        \end{equation*}
                        Por lo que $\{x_n\}$ es convergente a $x$ para $\|\cdot \|_1$, de donde deducimos que $(E,\|\cdot \|_1)$ es de Banach.
                    \item [Opción 2.] Como $G(T)$ es cerrado, el teorema de la gráfica cerrada implica que $T$ es acotado. Así:
                        \begin{equation*}
                            \|x\|_E \leq \|x\|_1 = \|x\|_E + \|Tx\|_F \leq \|x\|_E + \|T\|\|x\|_E = (1+\|T\|)\|x\|_E \quad \forall x\in E
                        \end{equation*}
                        por lo que $\|\cdot \|_1$ y $\|\cdot \|_E$ son equivalentes, y como $\|\cdot \|_E$ es una norma completa concluimos que $\|\cdot \|_1$ también ha de serlo.
                \end{description}
        \end{description}
    \end{ejercicio}
\end{document}
