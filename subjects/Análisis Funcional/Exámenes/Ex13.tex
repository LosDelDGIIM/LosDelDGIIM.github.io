\documentclass[12pt]{article}

% Idioma y codificación
\usepackage[spanish, es-tabla]{babel}       %es-tabla para que se titule "Tabla"
\usepackage[utf8]{inputenc}

% Márgenes
\usepackage[a4paper,top=3cm,bottom=2.5cm,left=3cm,right=3cm]{geometry}

% Comentarios de bloque
\usepackage{verbatim}

% Paquetes de links
\usepackage[hidelinks]{hyperref}    % Permite enlaces
\usepackage{url}                    % redirecciona a la web

% Más opciones para enumeraciones
\usepackage{enumitem}

% Personalizar la portada
\usepackage{titling}

% Paquetes de tablas
\usepackage{multirow}


%------------------------------------------------------------------------

%Paquetes de figuras
\usepackage{caption}
\usepackage{subcaption} % Figuras al lado de otras
\usepackage{float}      % Poner figuras en el sitio indicado H.


% Paquetes de imágenes
\usepackage{graphicx}       % Paquete para añadir imágenes
\usepackage{transparent}    % Para manejar la opacidad de las figuras

% Paquete para usar colores
\usepackage[dvipsnames]{xcolor}
\usepackage{pagecolor}      % Para cambiar el color de la página

% Habilita tamaños de fuente mayores
\usepackage{fix-cm}

% Para los gráficos
\usepackage{tikz}

% Para poder situar los nodos en los grafos
\usetikzlibrary{positioning}


%------------------------------------------------------------------------

% Paquetes de matemáticas
\usepackage{mathtools, amsfonts, amssymb, mathrsfs}
\usepackage[makeroom]{cancel}     % Simplificar tachando
\usepackage{polynom}    % Divisiones y Ruffini
\usepackage{units} % Para poner fracciones diagonales con \nicefrac

\usepackage{pgfplots}   %Representar funciones
\pgfplotsset{compat=1.18}  % Versión 1.18

\usepackage{tikz-cd}    % Para usar diagramas de composiciones
\usetikzlibrary{calc}   % Para usar cálculo de coordenadas en tikz

%Definición de teoremas, etc.
\usepackage{amsthm}
%\swapnumbers   % Intercambia la posición del texto y de la numeración

\theoremstyle{plain}

\makeatletter
\@ifclassloaded{article}{
  \newtheorem{teo}{Teorema}[section]
}{
  \newtheorem{teo}{Teorema}[chapter]  % Se resetea en cada chapter
}
\makeatother

\newtheorem{coro}{Corolario}[teo]           % Se resetea en cada teorema
\newtheorem{prop}[teo]{Proposición}         % Usa el mismo contador que teorema
\newtheorem{lema}[teo]{Lema}                % Usa el mismo contador que teorema

\theoremstyle{remark}
\newtheorem*{observacion}{Observación}

\theoremstyle{definition}

\makeatletter
\@ifclassloaded{article}{
  \newtheorem{definicion}{Definición} [section]     % Se resetea en cada chapter
}{
  \newtheorem{definicion}{Definición} [chapter]     % Se resetea en cada chapter
}
\makeatother

\newtheorem*{notacion}{Notación}
\newtheorem*{ejemplo}{Ejemplo}
\newtheorem*{ejercicio*}{Ejercicio}             % No numerado
\newtheorem{ejercicio}{Ejercicio} [section]     % Se resetea en cada section


% Modificar el formato de la numeración del teorema "ejercicio"
\renewcommand{\theejercicio}{%
  \ifnum\value{section}=0 % Si no se ha iniciado ninguna sección
    \arabic{ejercicio}% Solo mostrar el número de ejercicio
  \else
    \thesection.\arabic{ejercicio}% Mostrar número de sección y número de ejercicio
  \fi
}


% \renewcommand\qedsymbol{$\blacksquare$}         % Cambiar símbolo QED
%------------------------------------------------------------------------

% Paquetes para encabezados
\usepackage{fancyhdr}
\pagestyle{fancy}
\fancyhf{}

\newcommand{\helv}{ % Modificación tamaño de letra
\fontfamily{}\fontsize{12}{12}\selectfont}
\setlength{\headheight}{15pt} % Amplía el tamaño del índice


%\usepackage{lastpage}   % Referenciar última pag   \pageref{LastPage}
\fancyfoot[C]{\thepage}

%------------------------------------------------------------------------

% Conseguir que no ponga "Capítulo 1". Sino solo "1."
\makeatletter
\@ifclassloaded{book}{
  \renewcommand{\chaptermark}[1]{\markboth{\thechapter.\ #1}{}} % En el encabezado
    
  \renewcommand{\@makechapterhead}[1]{%
  \vspace*{50\p@}%
  {\parindent \z@ \raggedright \normalfont
    \ifnum \c@secnumdepth >\m@ne
      \huge\bfseries \thechapter.\hspace{1em}\ignorespaces
    \fi
    \interlinepenalty\@M
    \Huge \bfseries #1\par\nobreak
    \vskip 40\p@
  }}
}
\makeatother

%------------------------------------------------------------------------
% Paquetes de cógido
\usepackage{minted}
\renewcommand\listingscaption{Código fuente}

\usepackage{fancyvrb}
% Personaliza el tamaño de los números de línea
\renewcommand{\theFancyVerbLine}{\small\arabic{FancyVerbLine}}

% Estilo para C++
\newminted{cpp}{
    frame=lines,
    framesep=2mm,
    baselinestretch=1.2,
    linenos,
    escapeinside=||
}

% para minted
\definecolor{LightGray}{rgb}{0.95,0.95,0.92}
\setminted{
    linenos=true,
    stepnumber=5,
    numberfirstline=true,
    autogobble,
    breaklines=true,
    breakautoindent=true,
    breaksymbolleft=,
    breaksymbolright=,
    breaksymbolindentleft=0pt,
    breaksymbolindentright=0pt,
    breaksymbolsepleft=0pt,
    breaksymbolsepright=0pt,
    fontsize=\footnotesize,
    bgcolor=LightGray,
    numbersep=10pt
}


\usepackage{listings} % Para incluir código desde un archivo

\renewcommand\lstlistingname{Código Fuente}
\renewcommand\lstlistlistingname{Índice de Códigos Fuente}

% Definir colores
\definecolor{vscodepurple}{rgb}{0.5,0,0.5}
\definecolor{vscodeblue}{rgb}{0,0,0.8}
\definecolor{vscodegreen}{rgb}{0,0.5,0}
\definecolor{vscodegray}{rgb}{0.5,0.5,0.5}
\definecolor{vscodebackground}{rgb}{0.97,0.97,0.97}
\definecolor{vscodelightgray}{rgb}{0.9,0.9,0.9}

% Configuración para el estilo de C similar a VSCode
\lstdefinestyle{vscode_C}{
  backgroundcolor=\color{vscodebackground},
  commentstyle=\color{vscodegreen},
  keywordstyle=\color{vscodeblue},
  numberstyle=\tiny\color{vscodegray},
  stringstyle=\color{vscodepurple},
  basicstyle=\scriptsize\ttfamily,
  breakatwhitespace=false,
  breaklines=true,
  captionpos=b,
  keepspaces=true,
  numbers=left,
  numbersep=5pt,
  showspaces=false,
  showstringspaces=false,
  showtabs=false,
  tabsize=2,
  frame=tb,
  framerule=0pt,
  aboveskip=10pt,
  belowskip=10pt,
  xleftmargin=10pt,
  xrightmargin=10pt,
  framexleftmargin=10pt,
  framexrightmargin=10pt,
  framesep=0pt,
  rulecolor=\color{vscodelightgray},
  backgroundcolor=\color{vscodebackground},
}

%------------------------------------------------------------------------

% Comandos definidos
\newcommand{\bb}[1]{\mathbb{#1}}
\newcommand{\cc}[1]{\mathcal{#1}}

% I prefer the slanted \leq
\let\oldleq\leq % save them in case they're every wanted
\let\oldgeq\geq
\renewcommand{\leq}{\leqslant}
\renewcommand{\geq}{\geqslant}

% Si y solo si
\newcommand{\sii}{\iff}

% Letras griegas
\newcommand{\eps}{\epsilon}
\newcommand{\veps}{\varepsilon}
\newcommand{\lm}{\lambda}

\newcommand{\ol}{\overline}
\newcommand{\ul}{\underline}
\newcommand{\wt}{\widetilde}
\newcommand{\wh}{\widehat}

\let\oldvec\vec
\renewcommand{\vec}{\overrightarrow}

% Derivadas parciales
\newcommand{\del}[2]{\frac{\partial #1}{\partial #2}}
\newcommand{\Del}[3]{\frac{\partial^{#1} #2}{\partial #3^{#1}}}
\newcommand{\deld}[2]{\dfrac{\partial #1}{\partial #2}}
\newcommand{\Deld}[3]{\dfrac{\partial^{#1} #2}{\partial #3^{#1}}}


\newcommand{\AstIg}{\stackrel{(\ast)}{=}}
\newcommand{\Hop}{\stackrel{L'H\hat{o}pital}{=}}

\newcommand{\red}[1]{{\color{red}#1}} % Para integrales, destacar los cambios.

% Método de integración
\newcommand{\MetInt}[2]{
    \left[\begin{array}{c}
        #1 \\ #2
    \end{array}\right]
}

% Declarar aplicaciones
% 1. Nombre aplicación
% 2. Dominio
% 3. Codominio
% 4. Variable
% 5. Imagen de la variable
\newcommand{\Func}[5]{
    \begin{equation*}
        \begin{array}{rrll}
            #1:& #2 & \longrightarrow & #3\\
               & #4 & \longmapsto & #5
        \end{array}
    \end{equation*}
}

%------------------------------------------------------------------------


\newcommand{\norma}[1]{\lVert #1 \rVert}
\newcommand{\normagenerica}{\lVert \cdot \rVert}
\newcommand{\dual}[1]{#1^{*}}
\newcommand{\prodescalar}[2]{\langle #1, #2 \rangle}

\begin{document}

    % 1. Foto de fondo
    % 2. Título
    % 3. Encabezado Izquierdo
    % 4. Color de fondo
    % 5. Coord x del titulo
    % 6. Coord y del titulo
    % 7. Fecha

    
    % 1. Foto de fondo
% 2. Título
% 3. Encabezado Izquierdo
% 4. Color de fondo
% 5. Coord x del titulo
% 6. Coord y del titulo
% 7. Fecha

\newcommand{\portada}[7]{

    \portadaBase{#1}{#2}{#3}{#4}{#5}{#6}{#7}
    \portadaBook{#1}{#2}{#3}{#4}{#5}{#6}{#7}
}

\newcommand{\portadaExamen}[7]{

    \portadaBase{#1}{#2}{#3}{#4}{#5}{#6}{#7}
    \portadaArticle{#1}{#2}{#3}{#4}{#5}{#6}{#7}
}




\newcommand{\portadaBase}[7]{

    % Tiene la portada principal y la licencia Creative Commons
    
    % 1. Foto de fondo
    % 2. Título
    % 3. Encabezado Izquierdo
    % 4. Color de fondo
    % 5. Coord x del titulo
    % 6. Coord y del titulo
    % 7. Fecha
    
    
    \thispagestyle{empty}               % Sin encabezado ni pie de página
    \newgeometry{margin=0cm}        % Márgenes nulos para la primera página
    
    
    % Encabezado
    \fancyhead[L]{\helv #3}
    \fancyhead[R]{\helv \nouppercase{\leftmark}}
    
    
    \pagecolor{#4}        % Color de fondo para la portada
    
    \begin{figure}[p]
        \centering
        \transparent{0.3}           % Opacidad del 30% para la imagen
        
        \includegraphics[width=\paperwidth, keepaspectratio]{assets/#1}
    
        \begin{tikzpicture}[remember picture, overlay]
            \node[anchor=north west, text=white, opacity=1, font=\fontsize{60}{90}\selectfont\bfseries\sffamily, align=left] at (#5, #6) {#2};
            
            \node[anchor=south east, text=white, opacity=1, font=\fontsize{12}{18}\selectfont\sffamily, align=right] at (9.7, 3) {\textbf{\href{https://losdeldgiim.github.io/}{Los Del DGIIM}}};
            
            \node[anchor=south east, text=white, opacity=1, font=\fontsize{12}{15}\selectfont\sffamily, align=right] at (9.7, 1.8) {Doble Grado en Ingeniería Informática y Matemáticas\\Universidad de Granada};
        \end{tikzpicture}
    \end{figure}
    
    
    \restoregeometry        % Restaurar márgenes normales para las páginas subsiguientes
    \pagecolor{white}       % Restaurar el color de página
    
    
    \newpage
    \thispagestyle{empty}               % Sin encabezado ni pie de página
    \begin{tikzpicture}[remember picture, overlay]
        \node[anchor=south west, inner sep=3cm] at (current page.south west) {
            \begin{minipage}{0.5\paperwidth}
                \href{https://creativecommons.org/licenses/by-nc-nd/4.0/}{
                    \includegraphics[height=2cm]{assets/Licencia.png}
                }\vspace{1cm}\\
                Esta obra está bajo una
                \href{https://creativecommons.org/licenses/by-nc-nd/4.0/}{
                    Licencia Creative Commons Atribución-NoComercial-SinDerivadas 4.0 Internacional (CC BY-NC-ND 4.0).
                }\\
    
                Eres libre de compartir y redistribuir el contenido de esta obra en cualquier medio o formato, siempre y cuando des el crédito adecuado a los autores originales y no persigas fines comerciales. 
            \end{minipage}
        };
    \end{tikzpicture}
    
    
    
    % 1. Foto de fondo
    % 2. Título
    % 3. Encabezado Izquierdo
    % 4. Color de fondo
    % 5. Coord x del titulo
    % 6. Coord y del titulo
    % 7. Fecha


}


\newcommand{\portadaBook}[7]{

    % 1. Foto de fondo
    % 2. Título
    % 3. Encabezado Izquierdo
    % 4. Color de fondo
    % 5. Coord x del titulo
    % 6. Coord y del titulo
    % 7. Fecha

    % Personaliza el formato del título
    \pretitle{\begin{center}\bfseries\fontsize{42}{56}\selectfont}
    \posttitle{\par\end{center}\vspace{2em}}
    
    % Personaliza el formato del autor
    \preauthor{\begin{center}\Large}
    \postauthor{\par\end{center}\vfill}
    
    % Personaliza el formato de la fecha
    \predate{\begin{center}\huge}
    \postdate{\par\end{center}\vspace{2em}}
    
    \title{#2}
    \author{\href{https://losdeldgiim.github.io/}{Los Del DGIIM}}
    \date{Granada, #7}
    \maketitle
    
    \tableofcontents
}




\newcommand{\portadaArticle}[7]{

    % 1. Foto de fondo
    % 2. Título
    % 3. Encabezado Izquierdo
    % 4. Color de fondo
    % 5. Coord x del titulo
    % 6. Coord y del titulo
    % 7. Fecha

    % Personaliza el formato del título
    \pretitle{\begin{center}\bfseries\fontsize{42}{56}\selectfont}
    \posttitle{\par\end{center}\vspace{2em}}
    
    % Personaliza el formato del autor
    \preauthor{\begin{center}\Large}
    \postauthor{\par\end{center}\vspace{3em}}
    
    % Personaliza el formato de la fecha
    \predate{\begin{center}\huge}
    \postdate{\par\end{center}\vspace{5em}}
    
    \title{#2}
    \author{\href{https://losdeldgiim.github.io/}{Los Del DGIIM}}
    \date{Granada, #7}
    \thispagestyle{empty}               % Sin encabezado ni pie de página
    \maketitle
    \vfill
}
    \portadaExamen{ffccA4.jpg}{Análisis Funcional\\Examen XIII}{Análisis Funcional. Examen XIII}{MidnightBlue}{-8}{28}{2026}{Daniel Morán Sánchez}

    \begin{description}
        \item[Asignatura] Análisis Funcional.
        \item[Curso Académico] 2024-25.
        \item[Grado] Doble Grado en Ingeniería Informática y Matemáticas.
        \item[Grupo] Único.
        \item[Profesor] David Arcoya Álvarez.
        \item[Descripción] Examen Ordinario de Incidencias.
        % \item[Fecha] 19 de Noviembre de 2025.
        % \item[Duración] 2 horas.
    
    \end{description}
    \newpage


    % ------------------------------------

    \begin{ejercicio}[3 puntos]
        Sean:
        \begin{gather*}
            F=\{f\in C([-1,1],\mathbb{R}) : f(1) = 0\}, \qquad G = \{f\in C^1([-1,1],\mathbb{R}) : f(1) = 0\}, \\
            \|f\|_\infty := \sup\{|f(t)|:t\in [0,1]\}, \qquad \forall f\in X
        \end{gather*}
        \begin{enumerate}[label=\alph*)]
            \item ¿Es $(F,\|\cdot \|_\infty)$ un espacio de Banach? Justifica tu respuesta.
            \item ¿Es $(G,\|\cdot \|_\infty)$ un espacio de Banach? Justifica tu respuesta.
        \end{enumerate}
    \end{ejercicio}

    \begin{ejercicio}[3 puntos]
        Supongamos que $\|\cdot \|_1$ y $\|\cdot \|_2$ son dos normas completas en el espacio vectorial $E$. Prueba que si se verifica la propiedad
        \begin{center}
            ``toda sucesión $\{x_n\}\subset E$ verificando $
                \left\{\begin{array}{l}
                    \{x_n\} \stackrel{(E,\|\cdot \|_1)}{\longrightarrow} x \\
                    \{x_n\} \stackrel{(E,\|\cdot \|_2)}{\longrightarrow} y
                \end{array}\right.$ cumple $x=y$''
        \end{center}
        entonces $\|\cdot \|_1$ y $\|\cdot \|_2$ son equivalentes.
    \end{ejercicio}

    \begin{ejercicio}
        Sea $E$ un espacio de Banach.
        \begin{enumerate}[label=\alph*)]
            \item $\textbf{[1.5 puntos]}$ Si $C$ es un subconjunto convexo de $E$, prueba que
                \begin{center}
                    $C$ es cerrado en la topología de la norma $\Longleftrightarrow C$ es cerrado en la topología débil $\sigma(E,E^\ast)$.
                \end{center}
            \item $\textbf{[0.75 puntos]}$ Sea una sucesión $\{f_n\}\subset E^\ast$ verificando
                \begin{center}
                    $\{\langle f_n,x \rangle \}$ es convergente para todo $x\in E$.
                \end{center}
                Prueba que existe $f\in E^\ast$ tal que
                \begin{equation*}
                    \{f_n\}\stackrel{\ast}{\rightharpoonup} f \text{\ en\ } \sigma(E^\ast,E)
                \end{equation*}
            \item $\textbf{[0.75 puntos]}$ Supongamos que $E$ es reflexivo y sea $\{x_n\}$ una sucesión en $E$ tal que
                \begin{center}
                    $\{\langle f,x_n \rangle \}$ es convergente para cada $f\in E^\ast$.
                \end{center}
                Prueba que existe $x\in E$ tal que
                \begin{equation*}
                    \{x_n\}\rightharpoonup x \text{\ en\ } \sigma(E,E^\ast)
                \end{equation*}
            \item $\textbf{[1 punto]}$ Da un ejemplo de una sucesión $\{x_n\}\subset E = c_0$ tal que $\{\langle f,x_n \rangle \}$ es convergente para cada $f\in E^\ast$, pero no sea convergente débilmente en $\sigma(E,E^\ast)$.
        \end{enumerate}
    \end{ejercicio}
    \newpage
    \setcounter{ejercicio}{0} % Reiniciar contador de ejercicios

    \begin{ejercicio}[3 puntos]
        Sean:
        \begin{gather*}
            F=\{f\in C([-1,1],\mathbb{R}) : f(1) = 0\}, \qquad G = \{f\in C^1([-1,1],\mathbb{R}) : f(1) = 0\}, \\
            \|f\|_\infty := \sup\{|f(t)|:t\in [0,1]\}, \qquad \forall f\in X
        \end{gather*}
        Este es el enunciado oficial pero supongamos que $X=C([-1,1])$ y que la norma está mal escrita y en realidad se refiere a $t\in [-1,1]$
        \begin{enumerate}[label=\alph*)]
            \item ¿Es $(F,\|\cdot \|_\infty)$ un espacio de Banach? Justifica tu respuesta.\\
            Sabemos que $(C([-1,1],\mathbb{R}),\|\cdot\|_\infty)$ es un espacio de Banach.
            Por tanto, para probar que $(F,\|\cdot\|_\infty)$ es de Banach basta ver que
            $F$ es un subespacio cerrado de $C([-1,1],\mathbb{R})$.
            
            \medskip
            
            \textbf{Paso 1:} $F$ es un subespacio vectorial.
            
            Sea $f,g\in F$. Entonces $f,g\in C([-1,1],\mathbb{R})$ y $f(1)=g(1)=0$.
            Se tiene:
            \[
            (f+g)(1)=f(1)+g(1)=0,
            \]
            luego $f+g\in F$.
            
            Además, si $\alpha\in\mathbb{R}$ y $f\in F$, entonces
            \[
            (\alpha f)(1)=\alpha f(1)=\alpha\cdot 0=0,
            \]
            por lo que $\alpha f\in F$.
            Así, $F$ es un subespacio vectorial de $C([-1,1],\mathbb{R})$.
            
            \medskip
            
            \textbf{Paso 2:} $F$ es cerrado en $C([-1,1],\mathbb{R})$.
            
            Sea $\{f_n\}\subset F$ una sucesión tal que
            \[
            \|f_n-f\|_\infty \longrightarrow 0
            \quad \text{con } f\in C([-1,1],\mathbb{R}).
            \]
            La convergencia uniforme implica convergencia puntual, en particular,
            \[
            f_n(1)\longrightarrow f(1).
            \]
            Pero como $f_n\in F$, se cumple $f_n(1)=0$ para todo $n$, luego
            \[
            f(1)=0,
            \]
            y por tanto $f\in F$.
            Esto prueba que $F$ es cerrado.
            
            \medskip
            
            Como $F$ es un subespacio cerrado de un espacio de Banach,
            concluimos que $(F,\|\cdot\|_\infty)$ es un espacio de Banach.
            
            \item ¿Es $(G,\|\cdot \|_\infty)$ un espacio de Banach? Justifica tu respuesta.

            \noindent
            No. Aunque $G\subset C^1([-1,1],\mathbb{R})\subset C([-1,1],\mathbb{R})$, con la norma
            $\|\cdot\|_\infty$ el espacio $G$ no es completo porque no es cerrado en
            $\big(C([-1,1],\mathbb{R}),\|\cdot\|_\infty\big)$.
            
            \medskip
            
            \noindent
            Para verlo, consideramos la sucesión
            \begin{description}
            \item[Opción 1.] 
            \[
            f_n(t):=|t|^{\frac{n+1}{n}}-1,\qquad t\in[-1,1].
            \]
            Entonces $f_n\in G$ para todo $n$ porque $|t|^{\alpha}$ es derivable en $[-1,1]$ cuando
            $\alpha>1$, y además $f_n(1)=1-1=0$.
            
            \medskip
            
            \noindent
            Veamos que $f_n\to f$ uniformemente, donde
            \[
            f(t):=|t|-1.
            \]
            En efecto,
            \[
            \|f_n-f\|_\infty
            =\sup_{t\in[-1,1]}\bigl||t|^{\frac{n+1}{n}}-|t|\bigr|
            \overset{|\cdot| \text{ es par}}{=}\max_{0\le t\le 1}\left|t^{\frac{n+1}{n}}-t\right|,
            \]
            pues la expresión depende sólo de $|t|$.
            
            Definimos
            \[
            \varphi(t):=t^{\frac{n+1}{n}}-t,\qquad t\in[0,1].
            \]
            Se tiene $\varphi(0)=\varphi(1)=0$ y
            \[
            \varphi'(t)=\frac{n+1}{n}\,t^{\frac1n}-1.
            \]
            Luego $\varphi'(t)=0$ si y sólo si
            \[
            t^{\frac1n}=\frac{n}{n+1}
            \quad\Longleftrightarrow\quad
            t_0=\left(\frac{n}{n+1}\right)^n.
            \]
            Por tanto, el máximo de $|\varphi|$ en $[0,1]$ se alcanza en $t_0$, y como $\varphi(t)\le 0$
            en $[0,1]$, se obtiene
            \[
            \|f_n-f\|_\infty=\max_{0\le t\le 1}|\varphi(t)|=|\varphi(t_0)|
            = t_0-\;t_0^{\frac{n+1}{n}}.
            \]
            Pero
            \[
            t_0^{\frac{n+1}{n}}=\left(\left(\frac{n}{n+1}\right)^n\right)^{\frac{n+1}{n}}
            =\left(\frac{n}{n+1}\right)^{n+1},
            \]
            y así
            \begin{multline*}
            |\varphi(t_0)|
            =\left(\frac{n}{n+1}\right)^n-\left(\frac{n}{n+1}\right)^{n+1}
            =\left(\frac{n}{n+1}\right)^n\left(1-\frac{n}{n+1}\right)\\
            =\left(\frac{n}{n+1}\right)^n\frac{1}{n+1}\xrightarrow[n\to\infty]{}0.
            \end{multline*}
            Concluimos que $\|f_n-f\|_\infty\to 0$, es decir, $f_n\to f$ uniformemente.
            
            \medskip
            
            \noindent
            Sin embargo, $f(t)=|t|-1$ no pertenece a $G$ porque no es derivable en $t=0$.
            Luego $G$ no es cerrado en $\big(C([-1,1]),\|\cdot\|_\infty\big)$ y, por tanto,
            $(G,\|\cdot\|_\infty)$ no es completo (no es de Banach).

            \item[Opción 2.] Usaremos la aproximación usual del valor absoluto $\sqrt{t^2+\frac1n}$ y le restamos su valor en 1 $\sqrt{1+\frac1n}$ para que pertenezca a $G$
            \[
            f_n(t):=\sqrt{t^2+\frac1n}-\sqrt{1+\frac1n}.
            \]
            Entonces $f_n\in C^1([-1,1])$ y además
            \[
            f_n(1)=\sqrt{1+\frac1n}-\sqrt{1+\frac1n}=0,
            \]
            luego $f_n\in G$ para todo $n$.
            
            Sea ahora
            \[
            f(t):=|t|-1,\qquad t\in[-1,1].
            \]
            Probamos que $f_n\to f$ uniformemente. En efecto, para todo $t\in[-1,1]$,
            \[
            \Bigl|\sqrt{t^2+\frac1n}-|t|\Bigr|
            =
            \frac{\left(\sqrt{t^2+\frac1n}\right)^2+|t|^2}{\sqrt{t^2+\frac1n}+|t|}
            =
            \frac{\frac1n}{\sqrt{t^2+\frac1n}+|t|}
            \le
            \frac{\frac1n}{\sqrt{\frac1n}}
            =
            \frac1{\sqrt{n}},
            \]
            y análogamente
            \[
            \Bigl|\sqrt{1+\frac1n}-1\Bigr|
            =
            \frac{\frac1n}{\sqrt{1+\frac1n}+1}
            \le
            \frac1{\sqrt{n}}.
            \]
            Por tanto,
            \[
            \|f_n-f\|_\infty
            =
            \sup_{t\in[-1,1]}\Bigl| \bigl(\sqrt{t^2+\tfrac1n}-|t|\bigr)
            -\bigl(\sqrt{1+\tfrac1n}-1\bigr)\Bigr|
            \le
            \frac1{\sqrt{n}}+\frac1{\sqrt{n}}
            =
            \frac{2}{\sqrt{n}}
            \xrightarrow[n\to\infty]{}0.
            \]
            Luego $\{f_n\}$ converge uniformemente a $f$, en particular es de Cauchy en $\|\cdot\|_\infty$.
            
            Sin embargo, $f(t)=|t|-1$ no pertenece a $G$ porque no es derivable en $t=0$ (luego no es $C^1$).
            Concluimos que $(G,\|\cdot\|_\infty)$ no es completo y, por tanto, no es un espacio de Banach.
            
            \end{description}
            
        \end{enumerate}
    \end{ejercicio}
    \newpage

    \begin{ejercicio}[3 puntos]
        Supongamos que $\|\cdot \|_1$ y $\|\cdot \|_2$ son dos normas completas en el espacio vectorial $E$. Prueba que si se verifica la propiedad
        \begin{center}
            ``toda sucesión $\{x_n\}\subset E$ verificando $
                \left\{\begin{array}{l}
                    \{x_n\} \stackrel{(E,\|\cdot \|_1)}{\longrightarrow} x \\
                    \{x_n\} \stackrel{(E,\|\cdot \|_2)}{\longrightarrow} y
                \end{array}\right.$ cumple $x=y$''
        \end{center}
        entonces $\|\cdot \|_1$ y $\|\cdot \|_2$ son equivalentes.\\


        Consideremos la aplicación identidad
        \[
        T:(E,\|\cdot\|_1)\longrightarrow (E,\|\cdot\|_2),
        \qquad T(x)=x.
        \]
        Es claro que $T$ es lineal.
        
        Como $(E,\|\cdot\|_1)$ y $(E,\|\cdot\|_2)$ son espacios de Banach, para probar que $T$ es continua basta, por el teorema de la gráfica cerrada, ver que su gráfica es cerrada.
        
        Sea $\{x_n\}\subset E$ tal que
        \[
        x_n \xrightarrow[(E,\|\cdot\|_1)]{} x
        \qquad\text{y}\qquad
        Tx_n = x_n \xrightarrow[(E,\|\cdot\|_2)]{} y.
        \]
        Por la hipótesis del enunciado, se deduce necesariamente que $x=y$.
        Por tanto,
        \[
        (x_n,Tx_n)\longrightarrow (x,x)\in \operatorname{Graf}(T),
        \]
        lo que prueba que la gráfica de $T$ es cerrada.
        
        Aplicando el teorema de la gráfica cerrada, concluimos que $T$ es continua, es decir, existe una constante $C>0$ tal que
        \[
        \|x\|_2 \le C\,\|x\|_1
        \qquad \forall x\in E.
        \]
        Para ver la misma desigualdad en el otro sentido:
        \begin{quote}
        \begin{description}
        \item[Opción 1.] 
        Intercambiando el papel de las normas y considerando ahora:
        \[
        S:(E,\|\cdot\|_2)\longrightarrow (E,\|\cdot\|_1),
        \qquad S(x)=x,
        \]
        razonando de forma análoga .
        \item[Opción 2.]Por el 2do corolario del teorema de la aplicación abierta.
        \end{description}
        \end{quote}
        
        \noindent Se obtiene la existencia de $C'>0$ tal que
        \[
        \|x\|_1 \le C'\,\|x\|_2
        \qquad \forall x\in E.
        \]
        Por tanto, existen constantes $C,C'>0$ tales que
        \[
        C'^{-1}\|x\|_1 \le \|x\|_2 \le C\,\|x\|_1
        \qquad \forall x\in E,
        \]
        y se concluye que $\|\cdot\|_1$ y $\|\cdot\|_2$ son normas equivalentes.
    \end{ejercicio}
    \newpage

    \begin{ejercicio}
        Sea $E$ un espacio de Banach.
        \begin{enumerate}[label=\alph*)]
            \item $\textbf{[1.5 puntos]}$ Si $C$ es un subconjunto convexo de $E$, prueba que
                \begin{center}
                    $C$ es cerrado en la topología de la norma $\Longleftrightarrow C$ es cerrado en la topología débil $\sigma(E,E^\ast)$.
                \end{center}
                Es el Teorema 3.7. del temario, en la parte de "Relación entre débilmente cerrados y cerrados"
            \begin{teo}
                Sea $E$ un espacio de Banach y $A\subset E$ un conjunto convexo, entonces:
                \begin{equation*}
                    A \text{\ es\ } \sigma(E,E^\ast)-\text{cerrado} \Longleftrightarrow A\text{\ es cerrado}
                \end{equation*}
                \begin{proof}
                    Por doble implicación:
                    \begin{description}
                        \item [$\Longrightarrow )$] La hemos discutido anteriormente, pues se tiene que:
                            \begin{equation*}
                                \sigma(E,E^\ast)\subset \tau_{\|\cdot \|}
                            \end{equation*}
                        \item [$\Longleftarrow )$]  Si $A$ es un subconjunto de $E$ que es cerrado y convexo, queremos ver que es $\sigma(E,E^\ast)-$cerrado. Para ello, veamos que $E\setminus A$ es débilmente abierto. Para esto último, si tomamos $x_0\in E\setminus A$ tenemos por la segunda versión geométrica del Teorema de Hahn-Banach ($\{x_0\}$ es compacto y $A$ cerrado con $x_0\notin A$) que existen $f\in E^\ast$  y $\alpha\in \mathbb{R}$ tales que:
                            \begin{equation*}
                                \langle f,x_0 \rangle < \alpha < \langle f,x \rangle  \qquad \forall x\in A
                            \end{equation*}
                            Tenemos por tanto que:
                            \begin{equation*}
                                x_0 \in \{y\in E : \langle f,y \rangle <\alpha\} = f^{-1}(\left]-\infty,\alpha\right[)
                            \end{equation*}
                            con $f^{-1}(\left]-\infty,\alpha\right[)$ un conjunto $\sigma(E,E^\ast)-$abierto\footnote{También podríamos haber dicho que $f^{-1}(\left]-\infty,\alpha\right[) = V(f;\alpha)$.}, por la definición de $\sigma(E,E^\ast)$. Como $f^{-1}(\left]-\infty,\alpha\right[)\cap A = \emptyset $, tenemos que:
                            \begin{equation*}
                                x_0 \in f^{-1}(\left]-\infty,\alpha\right[) \subset E\setminus A
                            \end{equation*}
                            Como $x_0$ era un punto de $E\setminus A$ arbitrario, tenemos que $E\setminus A$ es $\sigma(E,E^\ast)-$abierto, lo que concluye la demostración.\qedhere
                    \end{description}
                \end{proof}
            \end{teo}
            \newpage
            
            \item $\textbf{[0.75 puntos]}$ Sea una sucesión $\{f_n\}\subset E^\ast$ verificando
                \begin{center}
                    $\{\langle f_n,x \rangle \}$ es convergente para todo $x\in E$.
                \end{center}
                Prueba que existe $f\in E^\ast$ tal que
                \begin{equation*}
                    \{f_n\}\stackrel{\ast}{\rightharpoonup} f \text{\ en\ } \sigma(E^\ast,E)
                \end{equation*}
            
            Definimos la aplicación
            \[
            f:E\longrightarrow \mathbb{R}, 
            \qquad 
            \langle f,x\rangle := \lim_{n\to\infty}\langle f_n,x\rangle .
            \]
            
            \begin{enumerate}
            \item \textit{$f$ es lineal.}
            
            Como cada $f_n$ es lineal, el paso al límite preserva la linealidad, luego $f$ es lineal.
            
            \item \textit{$f$ es continuo.}
            
            Por $(\ast)$, para todo $x\in E$ la sucesión $\{\langle f_n,x\rangle\}_{n\in\mathbb{N}}$ es convergente y, en particular, acotada:
            \[
            \sup_{n\in\mathbb{N}} |\langle f_n,x\rangle| < \infty
            \qquad \forall x\in E.
            \]
            Esto significa que la familia $\{f_n\}\subset E^\ast$ es puntualmente acotada.
            Por el teorema de Banach--Steinhaus(Acot. Unif.), existe $M>0$ tal que
            \[
            \sup_{n\in\mathbb{N}}\|f_n\|_{E^\ast} \le M.
            \]
            
            Entonces, para todo $x\in E$,
            \[
            |\langle f_n,x\rangle|
            \le \|f_n\|_{E^\ast}\,\|x\|
            \le M\|x\|,
            \]
            considerando el límite cuando $n\to\infty$ obtenemos
            \[
            |\langle f,x\rangle| \le M\|x\| \qquad \forall x\in E.
            \]
            Por tanto, $f$ es un funcional lineal continuo, es decir, $f\in E^\ast$.
            \end{enumerate}
            
            Además, por definición de $f$,
            \[
            \langle f_n,x\rangle \longrightarrow \langle f,x\rangle
            \qquad \forall x\in E,
            \]
            lo que, por la caracterización de la convergencia débil-$\ast$, implica que
            \[
            \{f_n\}\stackrel{\ast}{\rightharpoonup} f \text{\ en\ } \sigma(E^\ast,E)
            \]
            \newpage
                
            \item $\textbf{[0.75 puntos]}$ Supongamos que $E$ es reflexivo y sea $\{x_n\}$ una sucesión en $E$ tal que
                \begin{center}
                    $\{\langle f,x_n \rangle \}$ es convergente para cada $f\in E^\ast$.
                \end{center}
                Prueba que existe $x\in E$ tal que
                \begin{equation*}
                    \{x_n\}\rightharpoonup x \text{\ en\ } \sigma(E,E^\ast)
                \end{equation*}
            Definimos la aplicación
            \[
            \chi:E^\ast\longrightarrow \mathbb{R}, 
            \qquad 
            \langle \chi,f\rangle := \lim_{n\to\infty}\langle f,x_n\rangle .
            \]
            
            \begin{enumerate}
            \item \textit{$\chi$ es lineal.}
            
            Como para cada $n$ la aplicación $f\mapsto \langle f,x_n\rangle$ es lineal en $E^\ast$,
            el paso al límite preserva la linealidad, luego $\chi$ es lineal.
            
            \item \textit{$\chi$ es continua.}
            
            Para cada $f\in E^\ast$, la sucesión $\{\langle f,x_n\rangle\}_{n\in\mathbb{N}}$ es convergente y, en particular, acotada:
            \[
            \sup_{n\in\mathbb{N}} |\langle f,x_n\rangle| < \infty
            \qquad \forall f\in E^\ast.
            \]
            Esto significa que la familia de funcionales
            \[
            \chi_n:E^\ast\to \mathbb{R},\qquad \chi_n(f):=\langle f,x_n\rangle
            \]
            es puntualmente acotada en $(E^\ast)^\ast=E^{\ast\ast}$.
            
            Por el teorema de Banach--Steinhaus, existe $M>0$ tal que
            \[
            \sup_{n\in\mathbb{N}}\|\chi_n\|_{E^{\ast\ast}}\le M.
            \]
            Además $\langle \chi_n, f\rangle = \langle f,x_n\rangle$, por lo que
            \[
            \|\chi_n\|_{E^{\ast\ast}}
            =
            \sup_{\|f\|_{E^\ast}\le 1} |\langle \chi_n, f\rangle|
            =
            \sup_{\|f\|_{E^\ast}\le 1} |\langle f,x_n\rangle|
            =
            \|x_n\|,
            \]
            luego $\sup_{n\in\mathbb{N}} \|x_n\|<\infty$.
            
            Entonces, para todo $f\in E^\ast$ y todo $n$,
            \[
            |\langle f,x_n\rangle|\le \|f\|_{E^\ast}\,\|x_n\|\le M\|f\|_{E^\ast},
            \]
            y pasando a límite en $n\to\infty$ obtenemos
            \[
            |\langle \chi,f\rangle| \le M\|f\|_{E^\ast}\qquad \forall f\in E^\ast.
            \]
            Por tanto $\chi$ es un funcional lineal continuo, es decir, $\chi\in E^{\ast\ast}$.
            \end{enumerate}
            
            Como $E$ es reflexivo, $J:E\to E^{\ast\ast}$ es sobreyectiva, luego existe $x\in E$ tal que
            \[
            \chi = J(x),
            \qquad \text{es decir,}\qquad
            \langle \chi,f\rangle = \langle J(x),f\rangle = \langle f,x\rangle
            \quad \forall f\in E^\ast.
            \]
            Pero por definición de $\chi$,
            \[
            \langle \chi,f\rangle = \lim_{n\to\infty}\langle f,x_n\rangle.
            \]
            Concluimos que
            \[
            \langle f,x_n\rangle \longrightarrow \langle f,x\rangle
            \qquad \forall f\in E^\ast,
            \]
            lo cual equivale a
            \[
            x_n \rightharpoonup x \quad \text{en } \sigma(E,E^\ast).
            \]







                
            \item $\textbf{[1 punto]}$ Da un ejemplo de una sucesión $\{x_n\}\subset E = c_0$ tal que $\{\langle f,x_n \rangle \}$ es convergente para cada $f\in E^\ast$, pero no sea convergente débilmente en $\sigma(E,E^\ast)$.

            Consideremos $E=c_0$, el espacio de las sucesiones reales que convergen a $0$ con la norma $\|\cdot\|_\infty$.
            Definimos la sucesión $\{x_n\}\subset c_0$ dada por
            \[
            x_n := (\underbrace{1,1,\dots,1}_{n\ \text{veces}},0,0,\dots) \in c_0.
            \]
            
            Recordemos que $(c_0)^\ast \simeq \ell^1$. Más precisamente, a cada $y=(y_k)\in \ell^1$ le corresponde el funcional
            \[
            \varphi_y \in (c_0)^\ast, 
            \qquad 
            \varphi_y(x) := \sum_{k=1}^\infty y_k x_k,
            \quad \forall x=(x_k)\in c_0,
            \]
            y toda forma lineal continua sobre $c_0$ es de este tipo.
            
            Para $y\in \ell^1$ arbitrario, se tiene
            \[
            \varphi_y(x_n)
            =
            \sum_{k=1}^n y_k
            \xrightarrow[n\to\infty]{}
            \sum_{k=1}^\infty y_k
            \]
            ya que las sumas parciales de una serie absolutamente convergente convergen.
            Por tanto,
            \[
            \{\langle f,x_n\rangle\} \text{ es convergente para todo } f\in (c_0)^\ast.
            \]
            
            Supongamos ahora, por contradicción, que existe $x\in c_0$ tal que
            \[
            x_n \rightharpoonup x \quad \text{en } \sigma(E,E^\ast)=\sigma(c_0,(c_0)^\ast).
            \]
            Entonces
            \[
            \langle f,x_n\rangle \longrightarrow \langle f,x\rangle
            \qquad \forall f\in (c_0)^\ast.
            \]
            En particular, para todo $y\in \ell^1$,
            \[
            \sum_{k=1}^n y_k
            \xrightarrow[n\to\infty]{}
            \sum_{k=1}^\infty y_k x_k.
            \]
            
            Eligiendo $y=e_i=(0,\dots,0,\overset{i)}{1},0,\dots)\in \ell^1$, obtenemos
            \[
            1 = \lim_{n\to\infty} \sum_{k=1}^n e_i(k) = x_i,
            \qquad \forall i\in \mathbb{N}.
            \]
            Por tanto,
            \[
            x=(1,1,1,\dots),
            \]
            lo cual contradice que $x\in c_0$, ya que esta sucesión no converge a $0$.
            
            Concluimos que no existe $x\in c_0$ tal que $x_n \rightharpoonup x$. Es decir,
            \[
            \{\langle f,x_n\rangle\} \text{ es convergente para todo } f\in (c_0)^\ast \text{ pero } \{x_n\} \text{ no converge débilmente en } c_0.
            \]
            
        \end{enumerate}
    \end{ejercicio}
    
\end{document}
