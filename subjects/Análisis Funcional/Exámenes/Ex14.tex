\documentclass[12pt]{article}

% Idioma y codificación
\usepackage[spanish, es-tabla]{babel}       %es-tabla para que se titule "Tabla"
\usepackage[utf8]{inputenc}

% Márgenes
\usepackage[a4paper,top=3cm,bottom=2.5cm,left=3cm,right=3cm]{geometry}

% Comentarios de bloque
\usepackage{verbatim}

% Paquetes de links
\usepackage[hidelinks]{hyperref}    % Permite enlaces
\usepackage{url}                    % redirecciona a la web

% Más opciones para enumeraciones
\usepackage{enumitem}

% Personalizar la portada
\usepackage{titling}

% Paquetes de tablas
\usepackage{multirow}


%------------------------------------------------------------------------

%Paquetes de figuras
\usepackage{caption}
\usepackage{subcaption} % Figuras al lado de otras
\usepackage{float}      % Poner figuras en el sitio indicado H.


% Paquetes de imágenes
\usepackage{graphicx}       % Paquete para añadir imágenes
\usepackage{transparent}    % Para manejar la opacidad de las figuras

% Paquete para usar colores
\usepackage[dvipsnames]{xcolor}
\usepackage{pagecolor}      % Para cambiar el color de la página

% Habilita tamaños de fuente mayores
\usepackage{fix-cm}

% Para los gráficos
\usepackage{tikz}

% Para poder situar los nodos en los grafos
\usetikzlibrary{positioning}


%------------------------------------------------------------------------

% Paquetes de matemáticas
\usepackage{mathtools, amsfonts, amssymb, mathrsfs}
\usepackage[makeroom]{cancel}     % Simplificar tachando
\usepackage{polynom}    % Divisiones y Ruffini
\usepackage{units} % Para poner fracciones diagonales con \nicefrac

\usepackage{pgfplots}   %Representar funciones
\pgfplotsset{compat=1.18}  % Versión 1.18

\usepackage{tikz-cd}    % Para usar diagramas de composiciones
\usetikzlibrary{calc}   % Para usar cálculo de coordenadas en tikz

%Definición de teoremas, etc.
\usepackage{amsthm}
%\swapnumbers   % Intercambia la posición del texto y de la numeración

\theoremstyle{plain}

\makeatletter
\@ifclassloaded{article}{
  \newtheorem{teo}{Teorema}[section]
}{
  \newtheorem{teo}{Teorema}[chapter]  % Se resetea en cada chapter
}
\makeatother

\newtheorem{coro}{Corolario}[teo]           % Se resetea en cada teorema
\newtheorem{prop}[teo]{Proposición}         % Usa el mismo contador que teorema
\newtheorem{lema}[teo]{Lema}                % Usa el mismo contador que teorema

\theoremstyle{remark}
\newtheorem*{observacion}{Observación}

\theoremstyle{definition}

\makeatletter
\@ifclassloaded{article}{
  \newtheorem{definicion}{Definición} [section]     % Se resetea en cada chapter
}{
  \newtheorem{definicion}{Definición} [chapter]     % Se resetea en cada chapter
}
\makeatother

\newtheorem*{notacion}{Notación}
\newtheorem*{ejemplo}{Ejemplo}
\newtheorem*{ejercicio*}{Ejercicio}             % No numerado
\newtheorem{ejercicio}{Ejercicio} [section]     % Se resetea en cada section


% Modificar el formato de la numeración del teorema "ejercicio"
\renewcommand{\theejercicio}{%
  \ifnum\value{section}=0 % Si no se ha iniciado ninguna sección
    \arabic{ejercicio}% Solo mostrar el número de ejercicio
  \else
    \thesection.\arabic{ejercicio}% Mostrar número de sección y número de ejercicio
  \fi
}


% \renewcommand\qedsymbol{$\blacksquare$}         % Cambiar símbolo QED
%------------------------------------------------------------------------

% Paquetes para encabezados
\usepackage{fancyhdr}
\pagestyle{fancy}
\fancyhf{}

\newcommand{\helv}{ % Modificación tamaño de letra
\fontfamily{}\fontsize{12}{12}\selectfont}
\setlength{\headheight}{15pt} % Amplía el tamaño del índice


%\usepackage{lastpage}   % Referenciar última pag   \pageref{LastPage}
\fancyfoot[C]{\thepage}

%------------------------------------------------------------------------

% Conseguir que no ponga "Capítulo 1". Sino solo "1."
\makeatletter
\@ifclassloaded{book}{
  \renewcommand{\chaptermark}[1]{\markboth{\thechapter.\ #1}{}} % En el encabezado
    
  \renewcommand{\@makechapterhead}[1]{%
  \vspace*{50\p@}%
  {\parindent \z@ \raggedright \normalfont
    \ifnum \c@secnumdepth >\m@ne
      \huge\bfseries \thechapter.\hspace{1em}\ignorespaces
    \fi
    \interlinepenalty\@M
    \Huge \bfseries #1\par\nobreak
    \vskip 40\p@
  }}
}
\makeatother

%------------------------------------------------------------------------
% Paquetes de cógido
\usepackage{minted}
\renewcommand\listingscaption{Código fuente}

\usepackage{fancyvrb}
% Personaliza el tamaño de los números de línea
\renewcommand{\theFancyVerbLine}{\small\arabic{FancyVerbLine}}

% Estilo para C++
\newminted{cpp}{
    frame=lines,
    framesep=2mm,
    baselinestretch=1.2,
    linenos,
    escapeinside=||
}

% para minted
\definecolor{LightGray}{rgb}{0.95,0.95,0.92}
\setminted{
    linenos=true,
    stepnumber=5,
    numberfirstline=true,
    autogobble,
    breaklines=true,
    breakautoindent=true,
    breaksymbolleft=,
    breaksymbolright=,
    breaksymbolindentleft=0pt,
    breaksymbolindentright=0pt,
    breaksymbolsepleft=0pt,
    breaksymbolsepright=0pt,
    fontsize=\footnotesize,
    bgcolor=LightGray,
    numbersep=10pt
}


\usepackage{listings} % Para incluir código desde un archivo

\renewcommand\lstlistingname{Código Fuente}
\renewcommand\lstlistlistingname{Índice de Códigos Fuente}

% Definir colores
\definecolor{vscodepurple}{rgb}{0.5,0,0.5}
\definecolor{vscodeblue}{rgb}{0,0,0.8}
\definecolor{vscodegreen}{rgb}{0,0.5,0}
\definecolor{vscodegray}{rgb}{0.5,0.5,0.5}
\definecolor{vscodebackground}{rgb}{0.97,0.97,0.97}
\definecolor{vscodelightgray}{rgb}{0.9,0.9,0.9}

% Configuración para el estilo de C similar a VSCode
\lstdefinestyle{vscode_C}{
  backgroundcolor=\color{vscodebackground},
  commentstyle=\color{vscodegreen},
  keywordstyle=\color{vscodeblue},
  numberstyle=\tiny\color{vscodegray},
  stringstyle=\color{vscodepurple},
  basicstyle=\scriptsize\ttfamily,
  breakatwhitespace=false,
  breaklines=true,
  captionpos=b,
  keepspaces=true,
  numbers=left,
  numbersep=5pt,
  showspaces=false,
  showstringspaces=false,
  showtabs=false,
  tabsize=2,
  frame=tb,
  framerule=0pt,
  aboveskip=10pt,
  belowskip=10pt,
  xleftmargin=10pt,
  xrightmargin=10pt,
  framexleftmargin=10pt,
  framexrightmargin=10pt,
  framesep=0pt,
  rulecolor=\color{vscodelightgray},
  backgroundcolor=\color{vscodebackground},
}

%------------------------------------------------------------------------

% Comandos definidos
\newcommand{\bb}[1]{\mathbb{#1}}
\newcommand{\cc}[1]{\mathcal{#1}}

% I prefer the slanted \leq
\let\oldleq\leq % save them in case they're every wanted
\let\oldgeq\geq
\renewcommand{\leq}{\leqslant}
\renewcommand{\geq}{\geqslant}

% Si y solo si
\newcommand{\sii}{\iff}

% Letras griegas
\newcommand{\eps}{\epsilon}
\newcommand{\veps}{\varepsilon}
\newcommand{\lm}{\lambda}

\newcommand{\ol}{\overline}
\newcommand{\ul}{\underline}
\newcommand{\wt}{\widetilde}
\newcommand{\wh}{\widehat}

\let\oldvec\vec
\renewcommand{\vec}{\overrightarrow}

% Derivadas parciales
\newcommand{\del}[2]{\frac{\partial #1}{\partial #2}}
\newcommand{\Del}[3]{\frac{\partial^{#1} #2}{\partial #3^{#1}}}
\newcommand{\deld}[2]{\dfrac{\partial #1}{\partial #2}}
\newcommand{\Deld}[3]{\dfrac{\partial^{#1} #2}{\partial #3^{#1}}}


\newcommand{\AstIg}{\stackrel{(\ast)}{=}}
\newcommand{\Hop}{\stackrel{L'H\hat{o}pital}{=}}

\newcommand{\red}[1]{{\color{red}#1}} % Para integrales, destacar los cambios.

% Método de integración
\newcommand{\MetInt}[2]{
    \left[\begin{array}{c}
        #1 \\ #2
    \end{array}\right]
}

% Declarar aplicaciones
% 1. Nombre aplicación
% 2. Dominio
% 3. Codominio
% 4. Variable
% 5. Imagen de la variable
\newcommand{\Func}[5]{
    \begin{equation*}
        \begin{array}{rrll}
            #1:& #2 & \longrightarrow & #3\\
               & #4 & \longmapsto & #5
        \end{array}
    \end{equation*}
}

%------------------------------------------------------------------------


\newcommand{\norma}[1]{\lVert #1 \rVert}
\newcommand{\normagenerica}{\lVert \cdot \rVert}
\newcommand{\dual}[1]{#1^{*}}
\newcommand{\prodescalar}[2]{\langle #1, #2 \rangle}

\begin{document}

    % 1. Foto de fondo
    % 2. Título
    % 3. Encabezado Izquierdo
    % 4. Color de fondo
    % 5. Coord x del titulo
    % 6. Coord y del titulo
    % 7. Fecha

    
    % 1. Foto de fondo
% 2. Título
% 3. Encabezado Izquierdo
% 4. Color de fondo
% 5. Coord x del titulo
% 6. Coord y del titulo
% 7. Fecha

\newcommand{\portada}[7]{

    \portadaBase{#1}{#2}{#3}{#4}{#5}{#6}{#7}
    \portadaBook{#1}{#2}{#3}{#4}{#5}{#6}{#7}
}

\newcommand{\portadaExamen}[7]{

    \portadaBase{#1}{#2}{#3}{#4}{#5}{#6}{#7}
    \portadaArticle{#1}{#2}{#3}{#4}{#5}{#6}{#7}
}




\newcommand{\portadaBase}[7]{

    % Tiene la portada principal y la licencia Creative Commons
    
    % 1. Foto de fondo
    % 2. Título
    % 3. Encabezado Izquierdo
    % 4. Color de fondo
    % 5. Coord x del titulo
    % 6. Coord y del titulo
    % 7. Fecha
    
    
    \thispagestyle{empty}               % Sin encabezado ni pie de página
    \newgeometry{margin=0cm}        % Márgenes nulos para la primera página
    
    
    % Encabezado
    \fancyhead[L]{\helv #3}
    \fancyhead[R]{\helv \nouppercase{\leftmark}}
    
    
    \pagecolor{#4}        % Color de fondo para la portada
    
    \begin{figure}[p]
        \centering
        \transparent{0.3}           % Opacidad del 30% para la imagen
        
        \includegraphics[width=\paperwidth, keepaspectratio]{assets/#1}
    
        \begin{tikzpicture}[remember picture, overlay]
            \node[anchor=north west, text=white, opacity=1, font=\fontsize{60}{90}\selectfont\bfseries\sffamily, align=left] at (#5, #6) {#2};
            
            \node[anchor=south east, text=white, opacity=1, font=\fontsize{12}{18}\selectfont\sffamily, align=right] at (9.7, 3) {\textbf{\href{https://losdeldgiim.github.io/}{Los Del DGIIM}}};
            
            \node[anchor=south east, text=white, opacity=1, font=\fontsize{12}{15}\selectfont\sffamily, align=right] at (9.7, 1.8) {Doble Grado en Ingeniería Informática y Matemáticas\\Universidad de Granada};
        \end{tikzpicture}
    \end{figure}
    
    
    \restoregeometry        % Restaurar márgenes normales para las páginas subsiguientes
    \pagecolor{white}       % Restaurar el color de página
    
    
    \newpage
    \thispagestyle{empty}               % Sin encabezado ni pie de página
    \begin{tikzpicture}[remember picture, overlay]
        \node[anchor=south west, inner sep=3cm] at (current page.south west) {
            \begin{minipage}{0.5\paperwidth}
                \href{https://creativecommons.org/licenses/by-nc-nd/4.0/}{
                    \includegraphics[height=2cm]{assets/Licencia.png}
                }\vspace{1cm}\\
                Esta obra está bajo una
                \href{https://creativecommons.org/licenses/by-nc-nd/4.0/}{
                    Licencia Creative Commons Atribución-NoComercial-SinDerivadas 4.0 Internacional (CC BY-NC-ND 4.0).
                }\\
    
                Eres libre de compartir y redistribuir el contenido de esta obra en cualquier medio o formato, siempre y cuando des el crédito adecuado a los autores originales y no persigas fines comerciales. 
            \end{minipage}
        };
    \end{tikzpicture}
    
    
    
    % 1. Foto de fondo
    % 2. Título
    % 3. Encabezado Izquierdo
    % 4. Color de fondo
    % 5. Coord x del titulo
    % 6. Coord y del titulo
    % 7. Fecha


}


\newcommand{\portadaBook}[7]{

    % 1. Foto de fondo
    % 2. Título
    % 3. Encabezado Izquierdo
    % 4. Color de fondo
    % 5. Coord x del titulo
    % 6. Coord y del titulo
    % 7. Fecha

    % Personaliza el formato del título
    \pretitle{\begin{center}\bfseries\fontsize{42}{56}\selectfont}
    \posttitle{\par\end{center}\vspace{2em}}
    
    % Personaliza el formato del autor
    \preauthor{\begin{center}\Large}
    \postauthor{\par\end{center}\vfill}
    
    % Personaliza el formato de la fecha
    \predate{\begin{center}\huge}
    \postdate{\par\end{center}\vspace{2em}}
    
    \title{#2}
    \author{\href{https://losdeldgiim.github.io/}{Los Del DGIIM}}
    \date{Granada, #7}
    \maketitle
    
    \tableofcontents
}




\newcommand{\portadaArticle}[7]{

    % 1. Foto de fondo
    % 2. Título
    % 3. Encabezado Izquierdo
    % 4. Color de fondo
    % 5. Coord x del titulo
    % 6. Coord y del titulo
    % 7. Fecha

    % Personaliza el formato del título
    \pretitle{\begin{center}\bfseries\fontsize{42}{56}\selectfont}
    \posttitle{\par\end{center}\vspace{2em}}
    
    % Personaliza el formato del autor
    \preauthor{\begin{center}\Large}
    \postauthor{\par\end{center}\vspace{3em}}
    
    % Personaliza el formato de la fecha
    \predate{\begin{center}\huge}
    \postdate{\par\end{center}\vspace{5em}}
    
    \title{#2}
    \author{\href{https://losdeldgiim.github.io/}{Los Del DGIIM}}
    \date{Granada, #7}
    \thispagestyle{empty}               % Sin encabezado ni pie de página
    \maketitle
    \vfill
}
    \portadaExamen{ffccA4.jpg}{Análisis Funcional\\Examen XIV}{Análisis Funcional. Examen XIV}{MidnightBlue}{-8}{28}{2026}{José Juan Urrutia Milán}

    \begin{description}
        \item[Asignatura] Análisis Funcional.
        \item[Curso Académico] 2025-26.
        \item[Grado] Doble Grado en Ingeniería Informática y Matemáticas.
        \item[Grupo] Único.
        \item[Profesor] David Arcoya Álvarez.
        \item[Descripción] Examen Ordinario.
        \item[Fecha] 22 de Enero de 2026.
        \item[Duración] 3 horas.
    
    \end{description}
    \newpage


    % ------------------------------------

    \begin{ejercicio}
        Para $\alpha\in \left]0,1\right]$, sea
        \begin{equation*}
            X = \left\{f:[0,1]\to \mathbb{R} : f\text{\ es continua y\ } \sup_{x\neq y \in [0,1]}\frac{|f(x)-f(y)|}{|x-y|^{\alpha}}<\infty\right\}
        \end{equation*}
        con
        \begin{equation*}
        \|f\| := \|f\|_\infty + \sup_{x\neq y \in [0,1]}\frac{|f(x)-f(y)|}{|x-y|^{\alpha}}, \qquad \forall f\in X
        \end{equation*}
        Prueba que $(X,\|\cdot \|)$ es un espacio de Banach.
    \end{ejercicio}

    \begin{ejercicio}
        Sean $(X,\|\cdot \|_X)$ y $(Y,\|\cdot \|_Y)$ dos espacios de Banach y $T:X\to Y$ lineal y acotada para la que existe $c>0$ tal que
        \begin{equation*}
            \|Tx\|_Y \geq c\|x\|_X,\qquad \forall x\in X
        \end{equation*}
        Prueba que $T$ es compacto si y solo si $\dim X < \infty$.
    \end{ejercicio}

    \begin{ejercicio}
        Sea $X$ un espacio de Banach reflexivo y $f:[0,1]\to X$ una función continua. Prueba que existe $x\in X$ tal que
        \begin{equation*}
            \int_{0}^{1} \langle \varphi,f(s) \rangle ~ds  = \langle \varphi,x \rangle , \qquad \forall \varphi \in X^\ast
        \end{equation*}
    \end{ejercicio}

    \begin{ejercicio}
        (\textbf{Teoría}) Sea $H$ un espacio de Hilbert con producto escalar $(\cdot ,\cdot )$ y norma asociada $\|\cdot \|$. Prueba que si $T:H\to H$ es lineal, compacto y simétrico tal que
        \begin{equation*}
            \lm_1 = \sup\{(Tx,x):\|x\| = 1\} > 0,
        \end{equation*}
        entonces $\lm_1$ es un valor propio de $T$ y cualquier otro valor propio de $T$ es menor que $\lm_1$.
    \end{ejercicio}

    \newpage
    \setcounter{ejercicio}{0}
    \noindent
    \textbf{Solución.}
    
    \begin{ejercicio}
        Para $\alpha\in \left]0,1\right]$, sea
        \begin{equation*}
            X = \left\{f:[0,1]\to \mathbb{R} : f\text{\ es continua y\ } \sup_{x\neq y \in [0,1]}\frac{|f(x)-f(y)|}{|x-y|^{\alpha}}<\infty\right\}
        \end{equation*}
        con
        \begin{equation*}
        \|f\| := \|f\|_\infty + \sup_{x\neq y \in [0,1]}\frac{|f(x)-f(y)|}{|x-y|^{\alpha}}, \qquad \forall f\in X
        \end{equation*}
        Prueba que $(X,\|\cdot \|)$ es un espacio de Banach.\\

        \noindent
        Comenzaremos primero probado que $(X,\|\cdot \|)$ es un espacio normado, es decir, probando que $\|\cdot \|$ es una norma en $X$:
        \begin{itemize}
            \item Sean $\lm\in \mathbb{R}$ y $f\in X$ vemos que:
                \begin{align*}
                    \|\lm f\| &= \|\lm f\|_\infty + \sup_{x\neq y \in [0,1]}\frac{|\lm f(x)-\lm f(y)|}{|x-y|^{\alpha}} = |\lm|\|f\|_\infty + \sup_{x\neq y \in [0,1]}|\lm|\frac{| f(x)- f(y)|}{|x-y|^{\alpha}}  \\
                              &= |\lm|\left[\|f\|_\infty + \sup_{x\neq y \in [0,1]}\frac{|\lm f(x)-\lm f(y)|}{|x-y|^{\alpha}} \right] = |\lm|\|f\|
                \end{align*}
            \item Sean $f,g\in X$ tenemos que:
                \begin{align*}
                    \|f+g\| &= \|f+g\|_\infty + \sup_{x\neq y \in [0,1]}\frac{|(f+g)(x)-(f+g)(y)|}{|x-y|^{\alpha}}  \\ &= \|f+g\|_\infty + \sup_{x\neq y \in [0,1]}\frac{|f(x)-f(y)+g(x)-g(y)|}{|x-y|^{\alpha}}  \\ 
                            &\leq \|f\|_\infty + \|g\|_\infty + \sup_{x\neq y \in [0,1]}\frac{|f(x)-f(y)|+|g(x)-g(y)|}{|x-y|^{\alpha}}   \\
                            &\leq \|f\|_\infty + \|g\|_\infty + \sup_{x\neq y \in [0,1]}\frac{| f(x)- f(y)|}{|x-y|^{\alpha}}  + \sup_{x\neq y \in [0,1]}\frac{| g(x)- g(y)|}{|x-y|^{\alpha}} = \|f\| + \|g\|
                \end{align*}
            \item Finalmente, si tenemos que:
                \begin{equation*}
                    0 = \|f\| = \|f\|_\infty + \sup_{x\neq y \in [0,1]}\frac{| f(x)- f(y)|}{|x-y|^{\alpha}}
                \end{equation*}
                como $\|f\|_\infty, \sup\limits_{x\neq y \in [0,1]}\frac{| f(x)- f(y)|}{|x-y|^{\alpha}}\geq 0$ para toda $f\in X$ deducimos entonces que ambos sumandos son iguales a cero, por lo que en particular $\|f\|_\infty = 0$, de donde $f=0$.
        \end{itemize}
    \end{ejercicio}

    \begin{ejercicio}
        Sean $(X,\|\cdot \|_X)$ y $(Y,\|\cdot \|_Y)$ dos espacios de Banach y $T:X\to Y$ lineal y acotada para la que existe $c>0$ tal que
        \begin{equation*}
            \|Tx\|_Y \geq c\|x\|_X,\qquad \forall x\in X
        \end{equation*}
        Prueba que $T$ es compacto si y solo si $\dim X < \infty$.\\

        \noindent
        Por doble implicación (notaremos a ambas normas por $\|\cdot \|$):
        \begin{description}
            \item [$\Longleftarrow )$] Si $\dim X = N\in \mathbb{N}$ es conocido entonces que existe una ismetría lineal sobreyectiva $\Phi:X\to \mathbb{R}^N$. Sea $A\subset X$ un conjunto acotado, hemos de probar que $\overline{T(A)}$ es un conjunto compacto para probar que $T$ es un operador compacto. Como $T$ es acotada vemos que $T(A)$ es un conjunto acotado, es decir, existe $M\geq 0$ tal que $\|x\|\leq M \quad \forall x\in A$. Además, si $x\in \overline{T(A)}$ tenemos entonces existe una sucesión $\{x_n\}$ de puntos de $A$ convergente a $x$, tendremos que:
                \begin{equation*}
                    \|x_n\| \leq M \qquad \forall n\in \mathbb{N}
                \end{equation*}
                Por lo que también $\|x\|\leq M$, de donde $\overline{T(A)}$ también es un conjunto acotado. Observamos además que $\overline{T(A)}$ es un conjunto cerrado. De esta forma, $\Phi(\overline{T(A)})\subset \mathbb{R}^N$ será también un conjunto cerrado (puesto que $\Phi$ es un homeomorfismo) y acotado (puesto que $\Phi$ es isometría), por lo que $\Phi(\overline{T(A)})$ es un conjunto compacto de $\mathbb{R}^N$, de donde:
                \begin{equation*}
                    \overline{T(A)} = \Phi^{-1}\left(\Phi(\overline{T(A)})\right)
                \end{equation*}
                es un conjunto compacto, como imagen de un conjunto compacto por una aplicación continua.
            \item [$\Longrightarrow )$] Sea $\{x_n\}$ una sucesión de puntos de $\overline{B}(0,1)\subset X$, tenemos entonces que existe una parcial suya $\{x_{\sigma(n)}\}$ de forma que $\{T(x_{\sigma(n)})\}$ es una sucesión convergente, por lo que en particular esta será de Cauchy. Fijado $\varepsilon>0$ existe $m\in \mathbb{N}$ tal que si $p,q\in \mathbb{N}$ con $p,q\geq m$ se tiene que:
                \begin{equation*}
                    \|T(x_{\sigma(p)}) - T(x_{\sigma(q)}) \| < \varepsilon
                \end{equation*}
                Si observamos que:
                \begin{multline*}
                    \|T(x_{\sigma(p)})-T(x_{\sigma(q)})\| = \|T(x_{\sigma(p)}-x_{\sigma(q)})\| \geq c\|x_{\sigma(p)} - x_{\sigma(q)}\| = \|cx_{\sigma(p)}-cx_{\sigma(q)}\| \\ \forall p,q\in \mathbb{N}
                \end{multline*}
                Observamos que obtenemos que la sucesión $\{cx_{\sigma(n)}\}$ es de Cauchy, y por ser $X$ completo existe $x\in X$ con $\{cx_{\sigma(n)}\}\to x$. Si usamos ahora que $c>0$ vemos que tenemos que $\{x_{\sigma(n)}\}\to \frac{x}{c}$, y que $\frac{x}{c}\in \overline{B}(0,1)$ por ser este un conjunto cerrado.

                Hemos probado que toda sucesión de $\overline{B}(0,1)$ admite una parcial convergente a un punto del mismo conjunto, es decir, hemos probado que $\overline{B}(0,1)$ es secuencialmente compacto, condición equivalente a compacto en un espacio métrico. Finalmente, como:
                \begin{center}
                    $\dim X < \infty \quad\Longleftrightarrow\quad \overline{B}(0,1)$ es compacto
                \end{center}
                concluimos que $\dim X < \infty$.
        \end{description}
    \end{ejercicio}

    \begin{ejercicio}
        Sea $X$ un espacio de Banach reflexivo y $f:[0,1]\to X$ una función continua. Prueba que existe $x\in X$ tal que
        \begin{equation*}
            \int_{0}^{1} \langle \varphi,f(s) \rangle ~ds  = \langle \varphi,x \rangle , \qquad \forall \varphi \in X^\ast
        \end{equation*}~\\

        \noindent
        Definimos $\chi:X^\ast\to \mathbb{R}$ dada por:
        \begin{equation*}
            \langle \chi,\varphi \rangle  = \int_{0}^{1} \langle \varphi,f(s) \rangle ~ds  \qquad \forall \varphi \in X^\ast
        \end{equation*}
        Observamos que es una aplicación lineal, ya que si $a\in \mathbb{R}$ y $\varphi,\psi\in X^\ast$ tenemos entonces que:
        \begin{align*}
            \langle \chi, a\varphi + \psi \rangle  &= \int_{0}^{1} \langle a\varphi + \psi,f(s) \rangle ~ds  = \int_{0}^{1}\left( a\langle \varphi,f(s) \rangle +\langle \psi,f(s) \rangle \right)~ds  \\
                                                   &= a\int_{0}^{1} \langle \varphi,f(s) \rangle ~ds  + \int_{0}^{1} \langle \psi,f(s) \rangle ~ds  = a\langle \chi,\varphi \rangle  + \langle \chi,\psi \rangle 
        \end{align*}
        Notando por $\|\cdot \|$ a la norma de $X$, para ver que $\chi$ es continua usaremos que $\|\cdot \|\circ f:[0,1]\to \mathbb{R}$ es una aplicación continua de un conjunto compacto en $\mathbb{R}$, por lo que existe:
        \begin{equation*}
            |f| := \max\{\|f(s)\|:s\in [0,1]\}
        \end{equation*}
        así vemos para cada $\varphi \in X^\ast$ que:
        \begin{align*}
            |\langle \chi,\varphi \rangle | &= \left|\int_{0}^{1} \langle \varphi,f(s) \rangle ~ds \right| \leq \int_{0}^{1} |\langle \varphi,f(s) \rangle |~ds  \leq \int_{0}^{1} \|\varphi\|\|f(s)\|~ds \\ &= \|\varphi\|\int_{0}^{1} \|f(s)\|~ds \leq \|\varphi\|\int_{0}^{1} |f|~ds  = \|\varphi\||f|\int_{0}^{1} 1~ds = \|\varphi\||f|
        \end{align*}
        por lo que deducimos que $\chi$ es continua, de donde $\chi \in X^{\ast\ast}$. Como $X$ es reflexivo tenemos que la aplicación
        \Func{J}{X}{X^{\ast\ast}}{x}{Jx}
        es sobreyectiva, por lo que para $\chi \in X^{\ast\ast}$ existe $x\in X$ de forma que:
        \begin{equation*}
            \chi = Jx
        \end{equation*}
        Por lo que en particular tenemos para cada $\varphi \in X^\ast$ que:
        \begin{equation*}
            \langle \varphi,x \rangle  = \langle Jx,\varphi \rangle  = \langle \chi,\varphi \rangle  = \int_{0}^{1} \langle \varphi,f(s) \rangle ~ds 
        \end{equation*}
    \end{ejercicio}

    \begin{ejercicio}
        (\textbf{Teoría}) Sea $H$ un espacio de Hilbert con producto escalar $(\cdot ,\cdot )$ y norma asociada $\|\cdot \|$. Prueba que si $T:H\to H$ es lineal, compacto y simétrico tal que
        \begin{equation*}
            \lm_1 = \sup\{(Tx,x):\|x\| = 1\} > 0,
        \end{equation*}
        entonces $\lm_1$ es un valor propio de $T$ y cualquier otro valor propio de $T$ es menor que $\lm_1$.\\

        \noindent
        Consulte los apuntes de teoría.
    \end{ejercicio}
\end{document}
