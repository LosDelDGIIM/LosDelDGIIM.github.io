\documentclass[12pt]{article}

% Idioma y codificación
\usepackage[spanish, es-tabla]{babel}       %es-tabla para que se titule "Tabla"
\usepackage[utf8]{inputenc}

% Márgenes
\usepackage[a4paper,top=3cm,bottom=2.5cm,left=3cm,right=3cm]{geometry}

% Comentarios de bloque
\usepackage{verbatim}

% Paquetes de links
\usepackage[hidelinks]{hyperref}    % Permite enlaces
\usepackage{url}                    % redirecciona a la web

% Más opciones para enumeraciones
\usepackage{enumitem}

% Personalizar la portada
\usepackage{titling}

% Paquetes de tablas
\usepackage{multirow}


%------------------------------------------------------------------------

%Paquetes de figuras
\usepackage{caption}
\usepackage{subcaption} % Figuras al lado de otras
\usepackage{float}      % Poner figuras en el sitio indicado H.


% Paquetes de imágenes
\usepackage{graphicx}       % Paquete para añadir imágenes
\usepackage{transparent}    % Para manejar la opacidad de las figuras

% Paquete para usar colores
\usepackage[dvipsnames]{xcolor}
\usepackage{pagecolor}      % Para cambiar el color de la página

% Habilita tamaños de fuente mayores
\usepackage{fix-cm}

% Para los gráficos
\usepackage{tikz}

% Para poder situar los nodos en los grafos
\usetikzlibrary{positioning}


%------------------------------------------------------------------------

% Paquetes de matemáticas
\usepackage{mathtools, amsfonts, amssymb, mathrsfs}
\usepackage[makeroom]{cancel}     % Simplificar tachando
\usepackage{polynom}    % Divisiones y Ruffini
\usepackage{units} % Para poner fracciones diagonales con \nicefrac

\usepackage{pgfplots}   %Representar funciones
\pgfplotsset{compat=1.18}  % Versión 1.18

\usepackage{tikz-cd}    % Para usar diagramas de composiciones
\usetikzlibrary{calc}   % Para usar cálculo de coordenadas en tikz

%Definición de teoremas, etc.
\usepackage{amsthm}
%\swapnumbers   % Intercambia la posición del texto y de la numeración

\theoremstyle{plain}

\makeatletter
\@ifclassloaded{article}{
  \newtheorem{teo}{Teorema}[section]
}{
  \newtheorem{teo}{Teorema}[chapter]  % Se resetea en cada chapter
}
\makeatother

\newtheorem{coro}{Corolario}[teo]           % Se resetea en cada teorema
\newtheorem{prop}[teo]{Proposición}         % Usa el mismo contador que teorema
\newtheorem{lema}[teo]{Lema}                % Usa el mismo contador que teorema

\theoremstyle{remark}
\newtheorem*{observacion}{Observación}

\theoremstyle{definition}

\makeatletter
\@ifclassloaded{article}{
  \newtheorem{definicion}{Definición} [section]     % Se resetea en cada chapter
}{
  \newtheorem{definicion}{Definición} [chapter]     % Se resetea en cada chapter
}
\makeatother

\newtheorem*{notacion}{Notación}
\newtheorem*{ejemplo}{Ejemplo}
\newtheorem*{ejercicio*}{Ejercicio}             % No numerado
\newtheorem{ejercicio}{Ejercicio} [section]     % Se resetea en cada section


% Modificar el formato de la numeración del teorema "ejercicio"
\renewcommand{\theejercicio}{%
  \ifnum\value{section}=0 % Si no se ha iniciado ninguna sección
    \arabic{ejercicio}% Solo mostrar el número de ejercicio
  \else
    \thesection.\arabic{ejercicio}% Mostrar número de sección y número de ejercicio
  \fi
}


% \renewcommand\qedsymbol{$\blacksquare$}         % Cambiar símbolo QED
%------------------------------------------------------------------------

% Paquetes para encabezados
\usepackage{fancyhdr}
\pagestyle{fancy}
\fancyhf{}

\newcommand{\helv}{ % Modificación tamaño de letra
\fontfamily{}\fontsize{12}{12}\selectfont}
\setlength{\headheight}{15pt} % Amplía el tamaño del índice


%\usepackage{lastpage}   % Referenciar última pag   \pageref{LastPage}
\fancyfoot[C]{\thepage}

%------------------------------------------------------------------------

% Conseguir que no ponga "Capítulo 1". Sino solo "1."
\makeatletter
\@ifclassloaded{book}{
  \renewcommand{\chaptermark}[1]{\markboth{\thechapter.\ #1}{}} % En el encabezado
    
  \renewcommand{\@makechapterhead}[1]{%
  \vspace*{50\p@}%
  {\parindent \z@ \raggedright \normalfont
    \ifnum \c@secnumdepth >\m@ne
      \huge\bfseries \thechapter.\hspace{1em}\ignorespaces
    \fi
    \interlinepenalty\@M
    \Huge \bfseries #1\par\nobreak
    \vskip 40\p@
  }}
}
\makeatother

%------------------------------------------------------------------------
% Paquetes de cógido
\usepackage{minted}
\renewcommand\listingscaption{Código fuente}

\usepackage{fancyvrb}
% Personaliza el tamaño de los números de línea
\renewcommand{\theFancyVerbLine}{\small\arabic{FancyVerbLine}}

% Estilo para C++
\newminted{cpp}{
    frame=lines,
    framesep=2mm,
    baselinestretch=1.2,
    linenos,
    escapeinside=||
}

% para minted
\definecolor{LightGray}{rgb}{0.95,0.95,0.92}
\setminted{
    linenos=true,
    stepnumber=5,
    numberfirstline=true,
    autogobble,
    breaklines=true,
    breakautoindent=true,
    breaksymbolleft=,
    breaksymbolright=,
    breaksymbolindentleft=0pt,
    breaksymbolindentright=0pt,
    breaksymbolsepleft=0pt,
    breaksymbolsepright=0pt,
    fontsize=\footnotesize,
    bgcolor=LightGray,
    numbersep=10pt
}


\usepackage{listings} % Para incluir código desde un archivo

\renewcommand\lstlistingname{Código Fuente}
\renewcommand\lstlistlistingname{Índice de Códigos Fuente}

% Definir colores
\definecolor{vscodepurple}{rgb}{0.5,0,0.5}
\definecolor{vscodeblue}{rgb}{0,0,0.8}
\definecolor{vscodegreen}{rgb}{0,0.5,0}
\definecolor{vscodegray}{rgb}{0.5,0.5,0.5}
\definecolor{vscodebackground}{rgb}{0.97,0.97,0.97}
\definecolor{vscodelightgray}{rgb}{0.9,0.9,0.9}

% Configuración para el estilo de C similar a VSCode
\lstdefinestyle{vscode_C}{
  backgroundcolor=\color{vscodebackground},
  commentstyle=\color{vscodegreen},
  keywordstyle=\color{vscodeblue},
  numberstyle=\tiny\color{vscodegray},
  stringstyle=\color{vscodepurple},
  basicstyle=\scriptsize\ttfamily,
  breakatwhitespace=false,
  breaklines=true,
  captionpos=b,
  keepspaces=true,
  numbers=left,
  numbersep=5pt,
  showspaces=false,
  showstringspaces=false,
  showtabs=false,
  tabsize=2,
  frame=tb,
  framerule=0pt,
  aboveskip=10pt,
  belowskip=10pt,
  xleftmargin=10pt,
  xrightmargin=10pt,
  framexleftmargin=10pt,
  framexrightmargin=10pt,
  framesep=0pt,
  rulecolor=\color{vscodelightgray},
  backgroundcolor=\color{vscodebackground},
}

%------------------------------------------------------------------------

% Comandos definidos
\newcommand{\bb}[1]{\mathbb{#1}}
\newcommand{\cc}[1]{\mathcal{#1}}

% I prefer the slanted \leq
\let\oldleq\leq % save them in case they're every wanted
\let\oldgeq\geq
\renewcommand{\leq}{\leqslant}
\renewcommand{\geq}{\geqslant}

% Si y solo si
\newcommand{\sii}{\iff}

% Letras griegas
\newcommand{\eps}{\epsilon}
\newcommand{\veps}{\varepsilon}
\newcommand{\lm}{\lambda}

\newcommand{\ol}{\overline}
\newcommand{\ul}{\underline}
\newcommand{\wt}{\widetilde}
\newcommand{\wh}{\widehat}

\let\oldvec\vec
\renewcommand{\vec}{\overrightarrow}

% Derivadas parciales
\newcommand{\del}[2]{\frac{\partial #1}{\partial #2}}
\newcommand{\Del}[3]{\frac{\partial^{#1} #2}{\partial #3^{#1}}}
\newcommand{\deld}[2]{\dfrac{\partial #1}{\partial #2}}
\newcommand{\Deld}[3]{\dfrac{\partial^{#1} #2}{\partial #3^{#1}}}


\newcommand{\AstIg}{\stackrel{(\ast)}{=}}
\newcommand{\Hop}{\stackrel{L'H\hat{o}pital}{=}}

\newcommand{\red}[1]{{\color{red}#1}} % Para integrales, destacar los cambios.

% Método de integración
\newcommand{\MetInt}[2]{
    \left[\begin{array}{c}
        #1 \\ #2
    \end{array}\right]
}

% Declarar aplicaciones
% 1. Nombre aplicación
% 2. Dominio
% 3. Codominio
% 4. Variable
% 5. Imagen de la variable
\newcommand{\Func}[5]{
    \begin{equation*}
        \begin{array}{rrll}
            #1:& #2 & \longrightarrow & #3\\
               & #4 & \longmapsto & #5
        \end{array}
    \end{equation*}
}

%------------------------------------------------------------------------



\begin{document}

    % 1. Foto de fondo
    % 2. Título
    % 3. Encabezado Izquierdo
    % 4. Color de fondo
    % 5. Coord x del titulo
    % 6. Coord y del titulo
    % 7. Fecha

    
    % 1. Foto de fondo
% 2. Título
% 3. Encabezado Izquierdo
% 4. Color de fondo
% 5. Coord x del titulo
% 6. Coord y del titulo
% 7. Fecha

\newcommand{\portada}[7]{

    \portadaBase{#1}{#2}{#3}{#4}{#5}{#6}{#7}
    \portadaBook{#1}{#2}{#3}{#4}{#5}{#6}{#7}
}

\newcommand{\portadaExamen}[7]{

    \portadaBase{#1}{#2}{#3}{#4}{#5}{#6}{#7}
    \portadaArticle{#1}{#2}{#3}{#4}{#5}{#6}{#7}
}




\newcommand{\portadaBase}[7]{

    % Tiene la portada principal y la licencia Creative Commons
    
    % 1. Foto de fondo
    % 2. Título
    % 3. Encabezado Izquierdo
    % 4. Color de fondo
    % 5. Coord x del titulo
    % 6. Coord y del titulo
    % 7. Fecha
    
    
    \thispagestyle{empty}               % Sin encabezado ni pie de página
    \newgeometry{margin=0cm}        % Márgenes nulos para la primera página
    
    
    % Encabezado
    \fancyhead[L]{\helv #3}
    \fancyhead[R]{\helv \nouppercase{\leftmark}}
    
    
    \pagecolor{#4}        % Color de fondo para la portada
    
    \begin{figure}[p]
        \centering
        \transparent{0.3}           % Opacidad del 30% para la imagen
        
        \includegraphics[width=\paperwidth, keepaspectratio]{assets/#1}
    
        \begin{tikzpicture}[remember picture, overlay]
            \node[anchor=north west, text=white, opacity=1, font=\fontsize{60}{90}\selectfont\bfseries\sffamily, align=left] at (#5, #6) {#2};
            
            \node[anchor=south east, text=white, opacity=1, font=\fontsize{12}{18}\selectfont\sffamily, align=right] at (9.7, 3) {\textbf{\href{https://losdeldgiim.github.io/}{Los Del DGIIM}}};
            
            \node[anchor=south east, text=white, opacity=1, font=\fontsize{12}{15}\selectfont\sffamily, align=right] at (9.7, 1.8) {Doble Grado en Ingeniería Informática y Matemáticas\\Universidad de Granada};
        \end{tikzpicture}
    \end{figure}
    
    
    \restoregeometry        % Restaurar márgenes normales para las páginas subsiguientes
    \pagecolor{white}       % Restaurar el color de página
    
    
    \newpage
    \thispagestyle{empty}               % Sin encabezado ni pie de página
    \begin{tikzpicture}[remember picture, overlay]
        \node[anchor=south west, inner sep=3cm] at (current page.south west) {
            \begin{minipage}{0.5\paperwidth}
                \href{https://creativecommons.org/licenses/by-nc-nd/4.0/}{
                    \includegraphics[height=2cm]{assets/Licencia.png}
                }\vspace{1cm}\\
                Esta obra está bajo una
                \href{https://creativecommons.org/licenses/by-nc-nd/4.0/}{
                    Licencia Creative Commons Atribución-NoComercial-SinDerivadas 4.0 Internacional (CC BY-NC-ND 4.0).
                }\\
    
                Eres libre de compartir y redistribuir el contenido de esta obra en cualquier medio o formato, siempre y cuando des el crédito adecuado a los autores originales y no persigas fines comerciales. 
            \end{minipage}
        };
    \end{tikzpicture}
    
    
    
    % 1. Foto de fondo
    % 2. Título
    % 3. Encabezado Izquierdo
    % 4. Color de fondo
    % 5. Coord x del titulo
    % 6. Coord y del titulo
    % 7. Fecha


}


\newcommand{\portadaBook}[7]{

    % 1. Foto de fondo
    % 2. Título
    % 3. Encabezado Izquierdo
    % 4. Color de fondo
    % 5. Coord x del titulo
    % 6. Coord y del titulo
    % 7. Fecha

    % Personaliza el formato del título
    \pretitle{\begin{center}\bfseries\fontsize{42}{56}\selectfont}
    \posttitle{\par\end{center}\vspace{2em}}
    
    % Personaliza el formato del autor
    \preauthor{\begin{center}\Large}
    \postauthor{\par\end{center}\vfill}
    
    % Personaliza el formato de la fecha
    \predate{\begin{center}\huge}
    \postdate{\par\end{center}\vspace{2em}}
    
    \title{#2}
    \author{\href{https://losdeldgiim.github.io/}{Los Del DGIIM}}
    \date{Granada, #7}
    \maketitle
    
    \tableofcontents
}




\newcommand{\portadaArticle}[7]{

    % 1. Foto de fondo
    % 2. Título
    % 3. Encabezado Izquierdo
    % 4. Color de fondo
    % 5. Coord x del titulo
    % 6. Coord y del titulo
    % 7. Fecha

    % Personaliza el formato del título
    \pretitle{\begin{center}\bfseries\fontsize{42}{56}\selectfont}
    \posttitle{\par\end{center}\vspace{2em}}
    
    % Personaliza el formato del autor
    \preauthor{\begin{center}\Large}
    \postauthor{\par\end{center}\vspace{3em}}
    
    % Personaliza el formato de la fecha
    \predate{\begin{center}\huge}
    \postdate{\par\end{center}\vspace{5em}}
    
    \title{#2}
    \author{\href{https://losdeldgiim.github.io/}{Los Del DGIIM}}
    \date{Granada, #7}
    \thispagestyle{empty}               % Sin encabezado ni pie de página
    \maketitle
    \vfill
}
    \portadaExamen{ffccA4.jpg}{Análisis Funcional\\Examen II}{Análisis Funcional. Examen II}{MidnightBlue}{-8}{28}{2025}{Daniel Morán Sánchez}

    \begin{description}
        \item[Asignatura] Análisis Funcional.
        \item[Curso Académico] 2023/24.
        \item[Grado] Doble Grado en Ingeniería Informática y Matemáticas.
        % \item[Grupo] ---.
        % \item[Profesor] ---.
        \item[Descripción] Examen Ordinario.
        % \item[Fecha] 20 de diciembre de 2024.
        % \item[Duración] ---.
    
    \end{description}
    \newpage


    % ------------------------------------
    
    \begin{ejercicio}[3 puntos]
        Teorema de representación de Riesz-Fréchet en espacios de Hilbert.
    \end{ejercicio}

    \begin{ejercicio}[2.5 puntos]
        Sea $E = \{u\in C([0,1]) : u(0) = 0\}$ con la norma usual
        \begin{equation*}
            \|u\| = \max_{t\in [0,1]}|u(t)|
        \end{equation*}
        Consideremos el funcional lineal $f:E\to \mathbb{R}$ dado por
        \begin{equation*}
            f(u) = \int_{0}^{1} u(t)~dt \qquad \forall u\in E
        \end{equation*}
        \begin{enumerate}[label=\alph*)]
            \item Probad que $f\in E^\ast$ y calculad $\|f\|_{E^\ast}$.
            \item ¿Existe $u\in E$ tal que $\|u\| = 1$ y $f(u) = \|f\|_{E^\ast}$?
        \end{enumerate}
    \end{ejercicio}

    \begin{ejercicio}[2.5 puntos]
        Sea $E$ un espacio de Banach. Probad que si un subconjunto $A\subset E$ es compacto en la topología débil $\sigma(E,E^\ast)$, entonces $A$ es acotado.
    \end{ejercicio}

    \begin{ejercicio}[2 puntos]
        Sea $E$ un espacio de Banach reflexivo y $K\subset E$ un subconjunto convexo, cerrado y acotado. Dotamos a $K$ con la topología débil $\sigma(E,E^\ast)$ (que hace compacto a $K$). Sea $F=C(K)$ con la norma usual. Si $\mu\in F^\ast$ con $\|\mu\| = 1$ y
        \begin{equation*}
            \langle \mu,u \rangle \geq 0 \qquad \forall u\in C(K) \quad \text{tal que}\quad u\geq 0 \text{\ en\ } K
        \end{equation*}
        probad que existe un único elemento $x_0\in K$ tal que
        \begin{equation*}
            \langle \mu, f\big|_{K} \rangle  = \langle f,x_0 \rangle  \qquad \forall f\in E^\ast
        \end{equation*}
        (\textbf{Pista:} Encontrad primero $x_0\in E$ verificando $\langle \mu,f\big|_K \rangle = \langle f,x_0 \rangle\quad  \forall f\in E^\ast  $ y a continuación probad que $x_0\in K$ usando el teorema de Hahn-Banach).
    \end{ejercicio}

    % // TODO: La resolucion esta en un pdf en esta misma carpeta

    \newpage
    \setcounter{ejercicio}{0} % Reiniciar contador de ejercicios

    \begin{ejercicio}[3 puntos]
        Teorema de representación de Riesz-Fréchet en espacios de Hilbert.
    \end{ejercicio}
    Consultar los apuntes de teoría.
    \newpage

    
    \begin{ejercicio}[2.5 puntos]
        Sea $E = \{u\in C([0,1]) : u(0) = 0\}$ con la norma usual
        \begin{equation*}
            \|u\| = \max_{t\in [0,1]}|u(t)|
        \end{equation*}
        Consideremos el funcional lineal $f:E\to \mathbb{R}$ dado por
        \begin{equation*}
            f(u) = \int_{0}^{1} u(t)~dt \qquad \forall u\in E
        \end{equation*}
        \begin{enumerate}[label=\alph*)]
            \item Probad que $f\in E^\ast$ y calculad $\|f\|_{E^\ast}$.\\
                
                El funcional $f$ es claramente lineal(pues la integral es lineal).\\ Además para todo $u\in E$ se tiene
                \begin{equation*}
                    |f(u)|
                    =\left|\int_0^1 u(t)\,dt\right|
                    \le \int_0^1 |u(t)|\,dt
                    \le \|u\|\int_0^1 1\,dt
                    =\|u\|.
                \end{equation*}
                Luego $f$ es continuo y
                \begin{equation*}
                \|f\|_{E^\ast}\le 1.
                \end{equation*}
                
                Consideremos la sucesión $(u_n)\subset E$ dada por
                \begin{equation*}
                    u_n(t)=
                    \begin{cases}
                        nt, & 0\le t\le \frac{1}{n},\\
                        1, & \frac{1}{n}\le t\le 1.
                    \end{cases}
                \end{equation*}
                Se verifica que $\|u_n\|=1$ y
                \begin{equation*}
                    f(u_n)
                    =\int_0^{1/n} nt\,dt+\int_{1/n}^1 1\,dt
                    =1-\frac{1}{2n}.
                \end{equation*}
                Por tanto,
                \begin{equation*}
                    \|f\|_{E^\ast}=1.
                \end{equation*}
                
            \item ¿Existe $u\in E$ tal que $\|u\| = 1$ y $f(u) = \|f\|_{E^\ast}$?\\
                Supongamos que existe $u\in E$ tal que $\|u\|=1$ y $f(u)=1$. Entonces
                \begin{equation*}
                \int_0^1 u(t)\,dt=1.
                \end{equation*}
                Como $|u(t)|\le 1$ para todo $t\in[0,1]$ y $u$ es continua, se deduce que $u(t)=1$ para todo $t\in[0,1]$,
                lo cual contradice la condición $u(0)=0$.
                
                Por tanto, no existe ningún $u\in E$ que alcance la norma de $f$.
        \end{enumerate}    
    \end{ejercicio}
    \newpage

    
    \begin{ejercicio}[2.5 puntos]
        Sea $E$ un espacio de Banach. Probad que si un subconjunto $A\subset E$ es compacto en la topología débil $\sigma(E,E^\ast)$, entonces $A$ es acotado.\\

    Sea $A\subset E$ un subconjunto compacto para la topología débil $\sigma(E,E^\ast)$.\\
    Comenzamos probando que, para todo funcional lineal $f\in E^\ast$, el conjunto $f(A)\subset\mathbb{R}$ es compacto.
    
    En efecto, sea $\{f(x_n)\}$ una sucesión cualquiera en $f(A)$, con $\{x_n\}\subset A$.
    Como $A$ es $\sigma(E,E^\ast)$-compacto, existe una subsucesión $\{x_{n_k}\}$ que converge débilmente hacia algún $x\in A$.
    
    Dado que la topología débil $\sigma(E,E^\ast)$ es la topología inicial inducida por los funcionales de $E^\ast$, todo funcional $f\in E^\ast$ es continuo para dicha topología(es decir $f$ lleva sucesiones $\sigma(E,E^\ast)$-convergentes en sucesiones convergentes).
    Por tanto,
    \begin{equation*}
        \lim_{k\to\infty} f(x_{n_k}) = f(x).
    \end{equation*}
    Observando que $f(x)\in f(A)$, concluimos que toda sucesión $\{f(x_n)\}$ en $f(A)$ admite una subsucesión convergente $\{f(x_{n_k})\}$ con límite $f(x)$ en $f(A)$, y por tanto $f(A)$ es compacto en $\mathbb{R}$.
    
    En particular, $f(A)$ es un conjunto acotado de $\mathbb{R}$ para todo $f\in E^\ast$.
    Por uno de los corolarios del teorema de Banach--Steinhaus, se deduce que $A$ es un subconjunto acotado de $E$.

    \end{ejercicio}
    \newpage

    \begin{ejercicio}[2 puntos]
        Sea $E$ un espacio de Banach reflexivo y $K\subset E$ un subconjunto convexo, cerrado y acotado. Dotamos a $K$ con la topología débil $\sigma(E,E^\ast)$ (que hace compacto a $K$). Sea $F=C(K)$ con la norma usual. Si $\mu\in F^\ast$ con $\|\mu\| = 1$ y
        \begin{equation*}
            \langle \mu,u \rangle \geq 0 \qquad \forall u\in C(K) \quad \text{tal que}\quad u\geq 0 \text{\ en\ } K
        \end{equation*}
        probad que existe un único elemento $x_0\in K$ tal que
        \begin{equation*}
            \langle \mu, f\big|_{K} \rangle  = \langle f,x_0 \rangle  \qquad \forall f\in E^\ast
        \end{equation*}
        (\textbf{Pista:} Encontrad primero $x_0\in E$ verificando $\langle \mu,f\big|_K \rangle = \langle f,x_0 \rangle\quad  \forall f\in E^\ast  $ y a continuación probad que $x_0\in K$ usando el teorema de Hahn-Banach).\\

    \textbf{Unicidad.}
        Supongamos que existen $x_0,y_0\in K$ tales que
        \begin{equation*}
        \langle \mu,f|_K\rangle = \langle f,x_0\rangle = \langle f,y_0\rangle
        \qquad \forall f\in E^\ast.
        \end{equation*}
        Entonces
        \begin{equation*}
        \langle f,x_0-y_0\rangle = 0 \qquad \forall f\in E^\ast.
        \end{equation*}
        Por el teorema de Hahn--Banach se deduce que $x_0-y_0=0$, y por tanto $x_0=y_0$.\\
        
    \textbf{Existencia.}
        Procedemos en dos pasos, primero veremos que existe dicho $x_0$ y después que pertenece a $K$.\\
                
        \textbf{Paso 1.}
        Definimos $\varphi:E^\ast\to\mathbb{R}$ mediante
        \begin{equation*}
        \langle \varphi,f\rangle := \langle \mu,f|_K\rangle,
        \qquad \forall f\in E^\ast.
        \end{equation*}
        Es claro que $\varphi$ es lineal. Además, para todo $f\in E^\ast$ se tiene
        \begin{equation*}
            |\langle \varphi,f\rangle|
            = |\langle \mu,f|_K\rangle|
            \overset{(*)}{\le} \|\mu\|\,\|f|_K\|_\infty
            = \max_{x\in K} |\langle f,x\rangle|
            \overset{(**)}{\le} \max_{x\in K} \|f\|_{E^\ast}\|x\|_E.
        \end{equation*}
        Donde en $(*)$ y $(**)$ hemos usado la continuidad de $\mu \in F^\ast$ y de $f \in E^\ast$, respectivamente\\
        
        Como $K$ es acotado, existe $C>0$ tal que $K\subset B(0,C)$, y por tanto
        \begin{equation*}
        |\langle \varphi,f\rangle|
        \le C\,\|f\|_{E^\ast}
        \qquad \forall f\in E^\ast.
        \end{equation*}
        Luego $\varphi$ es un funcional lineal continuo sobre $E^\ast$, es decir,
        $\varphi\in E^{\ast\ast}$.
        
        Como $E$ es reflexivo, la aplicación canónica $J:E\to E^{\ast\ast}$ es sobreyectiva.
        Por tanto, existe $x_0\in E$ tal que $\varphi=J(x_0)$, es decir,
        \begin{equation*}
        \langle \mu,f|_K\rangle = \langle f,x_0\rangle
        \qquad \forall f\in E^\ast.
        \end{equation*}
        
        \medskip
        
        \textbf{Paso 2.}
        Veamos que necesariamente $x_0\in K$.
        Supongamos por contradicción que $x_0\notin K$. Como $K$ es convexo y cerrado y $\{x_0\}$ es compacto, por la 2da forma geométrica del teorema de Hahn--Banach existen $f_0\in E^\ast\setminus\{0\}$, $\alpha\in\mathbb{R}$ y $\varepsilon>0$ tales que
        \begin{equation}
        \langle f_0,x\rangle \le \alpha-\varepsilon
        < \alpha+\varepsilon \le \langle f_0,x_0\rangle,
        \qquad \forall x\in K.
        \tag{1}
        \end{equation}
        
        Como $K$ es $\sigma(E,E^\ast)$-compacto y $f_0$ es $\sigma(E,E^\ast)$-continuo, en particular alcanza su máximo, existe $y\in K$ tal que
        \begin{equation*}
        \langle f_0,x\rangle \le \langle f_0,y\rangle
        \qquad \forall x\in K \qquad \text{ y definimos } M:=\langle f_0,y\rangle
        \end{equation*}
        
        Usando la positividad de $\mu$ y notando $M, 1$ como funciones constantes
        \begin{equation*}
        \langle f_0,x_0\rangle
        = \langle \mu,f_0|_K\rangle
        = \langle \mu,f_0|_K-M\rangle + \langle \mu,M\rangle
        \overset{(*)}{\le} M\langle \mu,1\rangle.
        \end{equation*}
        Donde en $(\ast)$ hemos usado que $f_0|_K - M \le 0$ en $K$ (por definición de $M$) y, como $\mu\ge 0$, se tiene $\langle \mu, f_0|_K - M\rangle \le 0$.

        Como $\|\mu\|=1$ y $\|1\|_\infty=1$, se tiene $\langle \mu,1\rangle\le \|\mu\|\|1\|_\infty \le 1$, y por tanto
        \begin{equation*}
        \langle f_0,x_0\rangle \le M = \langle f_0,y\rangle,
        \end{equation*}
        lo cual contradice $(1)$. Por tanto, $x_0\in K$.
        
    \end{ejercicio}

\end{document}
