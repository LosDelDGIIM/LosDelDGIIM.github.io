\documentclass[12pt]{article}

% Idioma y codificación
\usepackage[spanish, es-tabla]{babel}       %es-tabla para que se titule "Tabla"
\usepackage[utf8]{inputenc}

% Márgenes
\usepackage[a4paper,top=3cm,bottom=2.5cm,left=3cm,right=3cm]{geometry}

% Comentarios de bloque
\usepackage{verbatim}

% Paquetes de links
\usepackage[hidelinks]{hyperref}    % Permite enlaces
\usepackage{url}                    % redirecciona a la web

% Más opciones para enumeraciones
\usepackage{enumitem}

% Personalizar la portada
\usepackage{titling}

% Paquetes de tablas
\usepackage{multirow}


%------------------------------------------------------------------------

%Paquetes de figuras
\usepackage{caption}
\usepackage{subcaption} % Figuras al lado de otras
\usepackage{float}      % Poner figuras en el sitio indicado H.


% Paquetes de imágenes
\usepackage{graphicx}       % Paquete para añadir imágenes
\usepackage{transparent}    % Para manejar la opacidad de las figuras

% Paquete para usar colores
\usepackage[dvipsnames]{xcolor}
\usepackage{pagecolor}      % Para cambiar el color de la página

% Habilita tamaños de fuente mayores
\usepackage{fix-cm}

% Para los gráficos
\usepackage{tikz}

% Para poder situar los nodos en los grafos
\usetikzlibrary{positioning}


%------------------------------------------------------------------------

% Paquetes de matemáticas
\usepackage{mathtools, amsfonts, amssymb, mathrsfs}
\usepackage[makeroom]{cancel}     % Simplificar tachando
\usepackage{polynom}    % Divisiones y Ruffini
\usepackage{units} % Para poner fracciones diagonales con \nicefrac

\usepackage{pgfplots}   %Representar funciones
\pgfplotsset{compat=1.18}  % Versión 1.18

\usepackage{tikz-cd}    % Para usar diagramas de composiciones
\usetikzlibrary{calc}   % Para usar cálculo de coordenadas en tikz

%Definición de teoremas, etc.
\usepackage{amsthm}
%\swapnumbers   % Intercambia la posición del texto y de la numeración

\theoremstyle{plain}

\makeatletter
\@ifclassloaded{article}{
  \newtheorem{teo}{Teorema}[section]
}{
  \newtheorem{teo}{Teorema}[chapter]  % Se resetea en cada chapter
}
\makeatother

\newtheorem{coro}{Corolario}[teo]           % Se resetea en cada teorema
\newtheorem{prop}[teo]{Proposición}         % Usa el mismo contador que teorema
\newtheorem{lema}[teo]{Lema}                % Usa el mismo contador que teorema

\theoremstyle{remark}
\newtheorem*{observacion}{Observación}

\theoremstyle{definition}

\makeatletter
\@ifclassloaded{article}{
  \newtheorem{definicion}{Definición} [section]     % Se resetea en cada chapter
}{
  \newtheorem{definicion}{Definición} [chapter]     % Se resetea en cada chapter
}
\makeatother

\newtheorem*{notacion}{Notación}
\newtheorem*{ejemplo}{Ejemplo}
\newtheorem*{ejercicio*}{Ejercicio}             % No numerado
\newtheorem{ejercicio}{Ejercicio} [section]     % Se resetea en cada section


% Modificar el formato de la numeración del teorema "ejercicio"
\renewcommand{\theejercicio}{%
  \ifnum\value{section}=0 % Si no se ha iniciado ninguna sección
    \arabic{ejercicio}% Solo mostrar el número de ejercicio
  \else
    \thesection.\arabic{ejercicio}% Mostrar número de sección y número de ejercicio
  \fi
}


% \renewcommand\qedsymbol{$\blacksquare$}         % Cambiar símbolo QED
%------------------------------------------------------------------------

% Paquetes para encabezados
\usepackage{fancyhdr}
\pagestyle{fancy}
\fancyhf{}

\newcommand{\helv}{ % Modificación tamaño de letra
\fontfamily{}\fontsize{12}{12}\selectfont}
\setlength{\headheight}{15pt} % Amplía el tamaño del índice


%\usepackage{lastpage}   % Referenciar última pag   \pageref{LastPage}
\fancyfoot[C]{\thepage}

%------------------------------------------------------------------------

% Conseguir que no ponga "Capítulo 1". Sino solo "1."
\makeatletter
\@ifclassloaded{book}{
  \renewcommand{\chaptermark}[1]{\markboth{\thechapter.\ #1}{}} % En el encabezado
    
  \renewcommand{\@makechapterhead}[1]{%
  \vspace*{50\p@}%
  {\parindent \z@ \raggedright \normalfont
    \ifnum \c@secnumdepth >\m@ne
      \huge\bfseries \thechapter.\hspace{1em}\ignorespaces
    \fi
    \interlinepenalty\@M
    \Huge \bfseries #1\par\nobreak
    \vskip 40\p@
  }}
}
\makeatother

%------------------------------------------------------------------------
% Paquetes de cógido
\usepackage{minted}
\renewcommand\listingscaption{Código fuente}

\usepackage{fancyvrb}
% Personaliza el tamaño de los números de línea
\renewcommand{\theFancyVerbLine}{\small\arabic{FancyVerbLine}}

% Estilo para C++
\newminted{cpp}{
    frame=lines,
    framesep=2mm,
    baselinestretch=1.2,
    linenos,
    escapeinside=||
}

% para minted
\definecolor{LightGray}{rgb}{0.95,0.95,0.92}
\setminted{
    linenos=true,
    stepnumber=5,
    numberfirstline=true,
    autogobble,
    breaklines=true,
    breakautoindent=true,
    breaksymbolleft=,
    breaksymbolright=,
    breaksymbolindentleft=0pt,
    breaksymbolindentright=0pt,
    breaksymbolsepleft=0pt,
    breaksymbolsepright=0pt,
    fontsize=\footnotesize,
    bgcolor=LightGray,
    numbersep=10pt
}


\usepackage{listings} % Para incluir código desde un archivo

\renewcommand\lstlistingname{Código Fuente}
\renewcommand\lstlistlistingname{Índice de Códigos Fuente}

% Definir colores
\definecolor{vscodepurple}{rgb}{0.5,0,0.5}
\definecolor{vscodeblue}{rgb}{0,0,0.8}
\definecolor{vscodegreen}{rgb}{0,0.5,0}
\definecolor{vscodegray}{rgb}{0.5,0.5,0.5}
\definecolor{vscodebackground}{rgb}{0.97,0.97,0.97}
\definecolor{vscodelightgray}{rgb}{0.9,0.9,0.9}

% Configuración para el estilo de C similar a VSCode
\lstdefinestyle{vscode_C}{
  backgroundcolor=\color{vscodebackground},
  commentstyle=\color{vscodegreen},
  keywordstyle=\color{vscodeblue},
  numberstyle=\tiny\color{vscodegray},
  stringstyle=\color{vscodepurple},
  basicstyle=\scriptsize\ttfamily,
  breakatwhitespace=false,
  breaklines=true,
  captionpos=b,
  keepspaces=true,
  numbers=left,
  numbersep=5pt,
  showspaces=false,
  showstringspaces=false,
  showtabs=false,
  tabsize=2,
  frame=tb,
  framerule=0pt,
  aboveskip=10pt,
  belowskip=10pt,
  xleftmargin=10pt,
  xrightmargin=10pt,
  framexleftmargin=10pt,
  framexrightmargin=10pt,
  framesep=0pt,
  rulecolor=\color{vscodelightgray},
  backgroundcolor=\color{vscodebackground},
}

%------------------------------------------------------------------------

% Comandos definidos
\newcommand{\bb}[1]{\mathbb{#1}}
\newcommand{\cc}[1]{\mathcal{#1}}

% I prefer the slanted \leq
\let\oldleq\leq % save them in case they're every wanted
\let\oldgeq\geq
\renewcommand{\leq}{\leqslant}
\renewcommand{\geq}{\geqslant}

% Si y solo si
\newcommand{\sii}{\iff}

% Letras griegas
\newcommand{\eps}{\epsilon}
\newcommand{\veps}{\varepsilon}
\newcommand{\lm}{\lambda}

\newcommand{\ol}{\overline}
\newcommand{\ul}{\underline}
\newcommand{\wt}{\widetilde}
\newcommand{\wh}{\widehat}

\let\oldvec\vec
\renewcommand{\vec}{\overrightarrow}

% Derivadas parciales
\newcommand{\del}[2]{\frac{\partial #1}{\partial #2}}
\newcommand{\Del}[3]{\frac{\partial^{#1} #2}{\partial #3^{#1}}}
\newcommand{\deld}[2]{\dfrac{\partial #1}{\partial #2}}
\newcommand{\Deld}[3]{\dfrac{\partial^{#1} #2}{\partial #3^{#1}}}


\newcommand{\AstIg}{\stackrel{(\ast)}{=}}
\newcommand{\Hop}{\stackrel{L'H\hat{o}pital}{=}}

\newcommand{\red}[1]{{\color{red}#1}} % Para integrales, destacar los cambios.

% Método de integración
\newcommand{\MetInt}[2]{
    \left[\begin{array}{c}
        #1 \\ #2
    \end{array}\right]
}

% Declarar aplicaciones
% 1. Nombre aplicación
% 2. Dominio
% 3. Codominio
% 4. Variable
% 5. Imagen de la variable
\newcommand{\Func}[5]{
    \begin{equation*}
        \begin{array}{rrll}
            #1:& #2 & \longrightarrow & #3\\
               & #4 & \longmapsto & #5
        \end{array}
    \end{equation*}
}

%------------------------------------------------------------------------



\begin{document}

    % 1. Foto de fondo
    % 2. Título
    % 3. Encabezado Izquierdo
    % 4. Color de fondo
    % 5. Coord x del titulo
    % 6. Coord y del titulo
    % 7. Fecha

    
    % 1. Foto de fondo
% 2. Título
% 3. Encabezado Izquierdo
% 4. Color de fondo
% 5. Coord x del titulo
% 6. Coord y del titulo
% 7. Fecha

\newcommand{\portada}[7]{

    \portadaBase{#1}{#2}{#3}{#4}{#5}{#6}{#7}
    \portadaBook{#1}{#2}{#3}{#4}{#5}{#6}{#7}
}

\newcommand{\portadaExamen}[7]{

    \portadaBase{#1}{#2}{#3}{#4}{#5}{#6}{#7}
    \portadaArticle{#1}{#2}{#3}{#4}{#5}{#6}{#7}
}




\newcommand{\portadaBase}[7]{

    % Tiene la portada principal y la licencia Creative Commons
    
    % 1. Foto de fondo
    % 2. Título
    % 3. Encabezado Izquierdo
    % 4. Color de fondo
    % 5. Coord x del titulo
    % 6. Coord y del titulo
    % 7. Fecha
    
    
    \thispagestyle{empty}               % Sin encabezado ni pie de página
    \newgeometry{margin=0cm}        % Márgenes nulos para la primera página
    
    
    % Encabezado
    \fancyhead[L]{\helv #3}
    \fancyhead[R]{\helv \nouppercase{\leftmark}}
    
    
    \pagecolor{#4}        % Color de fondo para la portada
    
    \begin{figure}[p]
        \centering
        \transparent{0.3}           % Opacidad del 30% para la imagen
        
        \includegraphics[width=\paperwidth, keepaspectratio]{assets/#1}
    
        \begin{tikzpicture}[remember picture, overlay]
            \node[anchor=north west, text=white, opacity=1, font=\fontsize{60}{90}\selectfont\bfseries\sffamily, align=left] at (#5, #6) {#2};
            
            \node[anchor=south east, text=white, opacity=1, font=\fontsize{12}{18}\selectfont\sffamily, align=right] at (9.7, 3) {\textbf{\href{https://losdeldgiim.github.io/}{Los Del DGIIM}}};
            
            \node[anchor=south east, text=white, opacity=1, font=\fontsize{12}{15}\selectfont\sffamily, align=right] at (9.7, 1.8) {Doble Grado en Ingeniería Informática y Matemáticas\\Universidad de Granada};
        \end{tikzpicture}
    \end{figure}
    
    
    \restoregeometry        % Restaurar márgenes normales para las páginas subsiguientes
    \pagecolor{white}       % Restaurar el color de página
    
    
    \newpage
    \thispagestyle{empty}               % Sin encabezado ni pie de página
    \begin{tikzpicture}[remember picture, overlay]
        \node[anchor=south west, inner sep=3cm] at (current page.south west) {
            \begin{minipage}{0.5\paperwidth}
                \href{https://creativecommons.org/licenses/by-nc-nd/4.0/}{
                    \includegraphics[height=2cm]{assets/Licencia.png}
                }\vspace{1cm}\\
                Esta obra está bajo una
                \href{https://creativecommons.org/licenses/by-nc-nd/4.0/}{
                    Licencia Creative Commons Atribución-NoComercial-SinDerivadas 4.0 Internacional (CC BY-NC-ND 4.0).
                }\\
    
                Eres libre de compartir y redistribuir el contenido de esta obra en cualquier medio o formato, siempre y cuando des el crédito adecuado a los autores originales y no persigas fines comerciales. 
            \end{minipage}
        };
    \end{tikzpicture}
    
    
    
    % 1. Foto de fondo
    % 2. Título
    % 3. Encabezado Izquierdo
    % 4. Color de fondo
    % 5. Coord x del titulo
    % 6. Coord y del titulo
    % 7. Fecha


}


\newcommand{\portadaBook}[7]{

    % 1. Foto de fondo
    % 2. Título
    % 3. Encabezado Izquierdo
    % 4. Color de fondo
    % 5. Coord x del titulo
    % 6. Coord y del titulo
    % 7. Fecha

    % Personaliza el formato del título
    \pretitle{\begin{center}\bfseries\fontsize{42}{56}\selectfont}
    \posttitle{\par\end{center}\vspace{2em}}
    
    % Personaliza el formato del autor
    \preauthor{\begin{center}\Large}
    \postauthor{\par\end{center}\vfill}
    
    % Personaliza el formato de la fecha
    \predate{\begin{center}\huge}
    \postdate{\par\end{center}\vspace{2em}}
    
    \title{#2}
    \author{\href{https://losdeldgiim.github.io/}{Los Del DGIIM}}
    \date{Granada, #7}
    \maketitle
    
    \tableofcontents
}




\newcommand{\portadaArticle}[7]{

    % 1. Foto de fondo
    % 2. Título
    % 3. Encabezado Izquierdo
    % 4. Color de fondo
    % 5. Coord x del titulo
    % 6. Coord y del titulo
    % 7. Fecha

    % Personaliza el formato del título
    \pretitle{\begin{center}\bfseries\fontsize{42}{56}\selectfont}
    \posttitle{\par\end{center}\vspace{2em}}
    
    % Personaliza el formato del autor
    \preauthor{\begin{center}\Large}
    \postauthor{\par\end{center}\vspace{3em}}
    
    % Personaliza el formato de la fecha
    \predate{\begin{center}\huge}
    \postdate{\par\end{center}\vspace{5em}}
    
    \title{#2}
    \author{\href{https://losdeldgiim.github.io/}{Los Del DGIIM}}
    \date{Granada, #7}
    \thispagestyle{empty}               % Sin encabezado ni pie de página
    \maketitle
    \vfill
}
    \portadaExamen{ffccA4.jpg}{Análisis Funcional\\Examen III}{Análisis Funcional. Examen III}{MidnightBlue}{-8}{28}{2025}{}

    \begin{description}
        \item[Asignatura] Análisis Funcional.
        \item[Curso Académico] 2024/25.
        \item[Grado] Grado en Matemáticas.
        % \item[Grupo] ---.
        % \item[Profesor] ---.
        \item[Descripción] Examen Ordinario.
        \item[Fecha] 15 de enero de 2025.
        % \item[Duración] ---.
    
    \end{description}
    \newpage


    % ------------------------------------
    
    \begin{ejercicio}[2 puntos]
        Sea $X = C([0,1],\mathbb{R})$ el espacio vectorial de las funciones continuas de $[0,1]$ a $\mathbb{R}$. Se define
        \begin{equation*}
            \|f\|_1 := \int_{0}^{1} |f(t)|~dt , \qquad \|f\|_\infty := \sup\{|f(t)| : t\in [0,1]\}\qquad \forall f\in X
        \end{equation*}
        y se escribe $X_1 = (X,\|\cdot \|_1)$, $X_{\infty} = (X,\|\cdot \|_\infty)$. Definimos el funcional lineal $\delta:X\to \mathbb{R}$ por
        \begin{equation*}
            \delta(f) = f(0) \qquad \forall f\in X
        \end{equation*}
        \begin{enumerate}[label=\alph*)]
            \item Demuestra que $\delta\in X_\infty^\ast$ y calcula su norma.
            \item ¿Es $\delta$ continuo respecto a $\|\cdot \|_1$? Justifica tu respuesta.
            \item ¿Son equivalente $\|\cdot \|_1$ y $\|\cdot \|_\infty$ en $X$? Justifica tu respuesta.
            \item ¿Es $X_1$ completo? Justifica tu respuesta.
        \end{enumerate}
    \end{ejercicio}

    \begin{ejercicio}[2 puntos]
        Sea $X$ un espacio reflexivo e $Y$ un espacio de Banach. Pruébese que si existe $T\in L(X,Y)$ sobreyectiva, entonces $Y$ es reflexivo.
    \end{ejercicio}

    \begin{ejercicio}[2 puntos]
        Sean $S:c_0\to c_0$ y $T:l_1\to l_1$ operadores lineales y supongamos que
        \begin{equation*}
            \sum_{k=1}^{\infty}[Sx](k)y(k) = \sum_{k=1}^{\infty}x(k)[Ty](k) \qquad \forall x\in c_0, \quad \forall y\in l_1
        \end{equation*}
        Demuestra que $S$ y $T$ son continuos.
    \end{ejercicio}

    \begin{ejercicio}[4 puntos]
        Desearrolla el siguiente tema: ``\textit{Mejor aproximación en espacios de Hilbert; teorema de la proyección ortogonal; teorema de Riesz-Fréchet}''.
    \end{ejercicio}

    \begin{ejercicio}
        (Ejercicio extra)\newline
        Sean $X,Y$ espacios normados y $T:X\to Y$ un operador verificando que $\overline{T(B_X)}$ es un subconjunto compacto de $Y$. Demuestra que $T^\ast$ alcanza su norma.
    \end{ejercicio}

    % // TODO: La solucion está en un pdf de esta carpeta

    \newpage
    \setcounter{ejercicio}{0} % Reiniciar contador de ejercicios

    \begin{ejercicio}[2 puntos]
        Sea $X = C([0,1],\mathbb{R})$ el espacio vectorial de las funciones continuas de $[0,1]$ a $\mathbb{R}$. Se define
        \begin{equation*}
            \|f\|_1 := \int_{0}^{1} |f(t)|~dt , \qquad \|f\|_\infty := \sup\{|f(t)| : t\in [0,1]\}\qquad \forall f\in X
        \end{equation*}
        y se escribe $X_1 = (X,\|\cdot \|_1)$, $X_{\infty} = (X,\|\cdot \|_\infty)$. Definimos el funcional lineal $\delta:X\to \mathbb{R}$ por
        \begin{equation*}
            \delta(f) = f(0) \qquad \forall f\in X
        \end{equation*}
        \begin{enumerate}[label=\alph*)]
            \item Demuestra que $\delta\in X_\infty^\ast$ y calcula su norma.

                Veamos que $\delta\in X_\infty^\ast$:
                \begin{itemize}
                    \item $\delta$ es lineal, pues si $f,g\in X$ y $\lm\in \mathbb{R}$ tenemos que:
                        \begin{equation*}
                            \delta(\lm f + g) = (\lm f + g)(0) = \lm f(0) + g(0) = \lm\delta(f) + \delta(g)
                        \end{equation*}
                    \item $\delta$ es continuo, pues si $f\in X$ tenemos:
                        \begin{equation*}
                            |\delta(f)| = |f(0)| \leq \sup_{t\in [0,1]}|f(t)| = \|f\|_\infty
                        \end{equation*}
                \end{itemize}
                Por lo que $\delta\in X_\infty^\ast$. De hecho, en el último apartado hemos probado además que $\|\delta\|\leq 1$. Veamos que $\|\delta\| = 1$, puesto que si consideramos cualquier función $f\in X$ de forma que $\max\limits_{t\in [0,1]}|f(t)| = |f(0)|$ tenemos entonces que:
                \begin{equation*}
                    |\delta(f)| = |f(0)| = \max_{t\in [0,1]}|f(t)| = \|f\|_\infty
                \end{equation*}
            \item ¿Es $\delta$ continuo respecto a $\|\cdot \|_1$? Justifica tu respuesta.

                Si $\delta$ fuera continuo respecto a $\|\cdot \|_1$, en particular será continua en la función constantemente igual a $0$, por lo que:
                \begin{equation*}
                    \forall \varepsilon>0~\exists \eta>0 : \|f\|_1 < \eta \Longrightarrow |\delta(f)|<\varepsilon
                \end{equation*}
                Es decir, si $\delta$ fuera continua podríamos acotar el valor de cualquier función $f\in X$ en $0$ sabiendo acotar el valor de su integral. No parece que esto sea posible, por lo que tratamos de probar que $\delta$ no es continua. Para ello, buscamos probar que:
                \begin{equation*}
                    \exists \varepsilon>0 : \forall \eta>0~\exists f\in X \quad \text{con}\quad \|f\|_1<\eta \quad \text{y}\quad |\delta(f)|>\varepsilon
                \end{equation*}
                Si consideramos $\varepsilon=\nicefrac{1}{2}$ y nos dan $\eta>0$, si consideramos la función cuya gráfica es:
                \begin{figure}[H]
                    \centering
                \begin{tikzpicture}
                \begin{axis}[
                    axis lines=middle,
                    xmin=0, xmax=1.1,
                    ymin=0, ymax=3.2,
                    xtick={0,1/3+0.11,1},
                    ytick={3},
                    xticklabels={$0$,$\eta$,$1$},
                    yticklabels={$1$},
                    width=10cm,
                    height=6cm,
                    samples=100,
                    domain=0:1,
                    thick
                ]

                % Parte lineal [0,1/3]
                \addplot[blue,domain=0:1/3+0.11]{3 - (3 - 3/4)/(1/3)*x};

                % Parte constante [1/3,1]
                \addplot[blue,domain=1/3+0.11:1]{0};
                \end{axis}
                \end{tikzpicture}
                \end{figure}
                \noindent
                Es decir, que en el intervalo $[0,\eta]$ es la recta que une el punto $(0,1)$ con el $(\eta,0)$ y en el intervalo $[\eta,1]$ vale $0$, tenemos que $f\in X$, así como que:
                \begin{equation*}
                    \|f\|_1 = \int_{0}^{1} |f(t)|~dt  = \int_{0}^{1} f(t)~dt  = \int_{0}^{\eta}f(t) ~dt  + \int_{\eta}^{1} f(t)~dt  = \frac{\eta}{2} < \eta
                \end{equation*}
                (ya que el área del triángulo es base por altura entre 2) y tenemos que:
                \begin{equation*}
                    |\delta(f)| = |f(0)| = 1 > \frac{1}{2}
                \end{equation*}
                Acabamos de probar que $\delta$ no es continua para $\|\cdot \|_1$.

            \item ¿Son equivalentes $\|\cdot \|_1$ y $\|\cdot \|_\infty$ en $X$? Justifica tu respuesta.

                No pueden ser equivalentes: si fueran equivalentes tendríamos que las topologías que da cada norma serían iguales, pero sin embargo tenemos que $\delta$ es continua para $\|\cdot \|_\infty$ y no es continua para $\|\cdot \|_1$, por lo que sus topologías no pueden contener los mismos abiertos (recordamos que $\delta$ es continua si y solo si la preimagen de todo abierto de $\mathbb{R}$ es un abierto en la topología que consideramos en $X$), por lo que las dos normas no pueden ser equivalentes.
            \item ¿Es $X_1$ completo? Justifica tu respuesta.

                No es completo, si consideramos la sucesión $\{f_n\}$ de funciones de $X$ donde cada $f_n$ es la función que en el intervalo $[0,\nicefrac{1}{n}]$ une el punto $(0,1)$ con el $(\nicefrac{1}{n},0)$ y en el intervalo $[\nicefrac{1}{n},1]$ es constantemente igual a 0, tendremos que la gráfica de cada función es:
                \begin{figure}[H]
                    \centering
                \begin{tikzpicture}
                \begin{axis}[
                    axis lines=middle,
                    xmin=0, xmax=1.1,
                    ymin=0, ymax=3.2,
                    xtick={0,1/3+0.11,1},
                    ytick={3},
                    xticklabels={$0$,$\tfrac{1}{n}$,$1$},
                    yticklabels={$1$},
                    width=10cm,
                    height=6cm,
                    samples=100,
                    domain=0:1,
                    thick
                ]

                % Parte lineal [0,1/3]
                \addplot[blue,domain=0:1/3+0.11]{3 - (3 - 3/4)/(1/3)*x};

                % Parte constante [1/3,1]
                \addplot[blue,domain=1/3+0.11:1]{0};
                \end{axis}
                \end{tikzpicture}
                \end{figure}
        \end{enumerate}
        Tenemos que esta sucesión es de Cauchy para $X_1$, pues si $n,m\in \mathbb{N}$ con $n<m$:
        \begin{equation*}
            \|f_n - f_m\|_1 = \int_{0}^{1} |f_n(t) - f_m(t)|~dt  = \int_{0}^{1} f_n(t)~dt -\int_{0}^{1} f_m(t)~dt  = \frac{1}{2}\left(\frac{1}{n}-\frac{1}{m}\right) < \frac{1}{2n}
        \end{equation*}
        Y sin embargo dicha sucesión de funciones no es convergente, pues en $L^1([0,1])$ convergen (con la misma norma) a la función:
        \begin{equation*}
            f(x) = \left\{\begin{array}{ll}
                1 & \text{si\ } x=0 \\
                0 & \text{si\ } x\neq 0
            \end{array}\right. 
        \end{equation*}
        Y tenemos que $f\notin X_1$, por lo que $\{f_n\}$ no es convergente en $X_1$ pero sí es de Cauchy, con lo que $X_1$ no puede ser completo.
    \end{ejercicio}

    % \begin{ejercicio}[2 puntos]
    %     Sea $X$ un espacio reflexivo e $Y$ un espacio de Banach. Pruébese que si existe $T\in L(X,Y)$ sobreyectiva, entonces $Y$ es reflexivo.
    % \end{ejercicio}

    % \begin{ejercicio}[2 puntos]
    %     Sean $S:c_0\toc_0$ y $T:l_1\to l_1$ operadores lineales y supongamos que
    %     \begin{equation*}
    %         \sum_{k=1}^{\infty}[Sx](k)y(k) = \sum_{k=1}^{\infty}x(k)[Ty](k) \qquad \forall x\in c_0, \quad \forall y\in l_1
    %     \end{equation*}
    %     Demuestra que $S$ y $T$ son continuos.
    % \end{ejercicio}

    % \begin{ejercicio}[4 puntos]
    %     Desearrolla el siguiente tema: ``\textit{Mejor aproximación en espacios de Hilbert; teorema de la proyección ortogonal; teorema de Riesz-Fréchet}''.
    % \end{ejercicio}

    % \begin{ejercicio}
    %     (Ejercicio extra)\newline
    %     Sean $X,Y$ espacios normados y $T:X\to Y$ un operador verificando que $\overline{T(B_X)}$ es un subconjunto compacto de $Y$. Demuestra que $T^\ast$ alcanza su norma.
    % \end{ejercicio}

\end{document}
