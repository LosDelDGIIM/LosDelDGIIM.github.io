\documentclass[12pt]{article}

% Idioma y codificación
\usepackage[spanish, es-tabla]{babel}       %es-tabla para que se titule "Tabla"
\usepackage[utf8]{inputenc}

% Márgenes
\usepackage[a4paper,top=3cm,bottom=2.5cm,left=3cm,right=3cm]{geometry}

% Comentarios de bloque
\usepackage{verbatim}

% Paquetes de links
\usepackage[hidelinks]{hyperref}    % Permite enlaces
\usepackage{url}                    % redirecciona a la web

% Más opciones para enumeraciones
\usepackage{enumitem}

% Personalizar la portada
\usepackage{titling}

% Paquetes de tablas
\usepackage{multirow}


%------------------------------------------------------------------------

%Paquetes de figuras
\usepackage{caption}
\usepackage{subcaption} % Figuras al lado de otras
\usepackage{float}      % Poner figuras en el sitio indicado H.


% Paquetes de imágenes
\usepackage{graphicx}       % Paquete para añadir imágenes
\usepackage{transparent}    % Para manejar la opacidad de las figuras

% Paquete para usar colores
\usepackage[dvipsnames]{xcolor}
\usepackage{pagecolor}      % Para cambiar el color de la página

% Habilita tamaños de fuente mayores
\usepackage{fix-cm}

% Para los gráficos
\usepackage{tikz}

% Para poder situar los nodos en los grafos
\usetikzlibrary{positioning}


%------------------------------------------------------------------------

% Paquetes de matemáticas
\usepackage{mathtools, amsfonts, amssymb, mathrsfs}
\usepackage[makeroom]{cancel}     % Simplificar tachando
\usepackage{polynom}    % Divisiones y Ruffini
\usepackage{units} % Para poner fracciones diagonales con \nicefrac

\usepackage{pgfplots}   %Representar funciones
\pgfplotsset{compat=1.18}  % Versión 1.18

\usepackage{tikz-cd}    % Para usar diagramas de composiciones
\usetikzlibrary{calc}   % Para usar cálculo de coordenadas en tikz

%Definición de teoremas, etc.
\usepackage{amsthm}
%\swapnumbers   % Intercambia la posición del texto y de la numeración

\theoremstyle{plain}

\makeatletter
\@ifclassloaded{article}{
  \newtheorem{teo}{Teorema}[section]
}{
  \newtheorem{teo}{Teorema}[chapter]  % Se resetea en cada chapter
}
\makeatother

\newtheorem{coro}{Corolario}[teo]           % Se resetea en cada teorema
\newtheorem{prop}[teo]{Proposición}         % Usa el mismo contador que teorema
\newtheorem{lema}[teo]{Lema}                % Usa el mismo contador que teorema

\theoremstyle{remark}
\newtheorem*{observacion}{Observación}

\theoremstyle{definition}

\makeatletter
\@ifclassloaded{article}{
  \newtheorem{definicion}{Definición} [section]     % Se resetea en cada chapter
}{
  \newtheorem{definicion}{Definición} [chapter]     % Se resetea en cada chapter
}
\makeatother

\newtheorem*{notacion}{Notación}
\newtheorem*{ejemplo}{Ejemplo}
\newtheorem*{ejercicio*}{Ejercicio}             % No numerado
\newtheorem{ejercicio}{Ejercicio} [section]     % Se resetea en cada section


% Modificar el formato de la numeración del teorema "ejercicio"
\renewcommand{\theejercicio}{%
  \ifnum\value{section}=0 % Si no se ha iniciado ninguna sección
    \arabic{ejercicio}% Solo mostrar el número de ejercicio
  \else
    \thesection.\arabic{ejercicio}% Mostrar número de sección y número de ejercicio
  \fi
}


% \renewcommand\qedsymbol{$\blacksquare$}         % Cambiar símbolo QED
%------------------------------------------------------------------------

% Paquetes para encabezados
\usepackage{fancyhdr}
\pagestyle{fancy}
\fancyhf{}

\newcommand{\helv}{ % Modificación tamaño de letra
\fontfamily{}\fontsize{12}{12}\selectfont}
\setlength{\headheight}{15pt} % Amplía el tamaño del índice


%\usepackage{lastpage}   % Referenciar última pag   \pageref{LastPage}
\fancyfoot[C]{\thepage}

%------------------------------------------------------------------------

% Conseguir que no ponga "Capítulo 1". Sino solo "1."
\makeatletter
\@ifclassloaded{book}{
  \renewcommand{\chaptermark}[1]{\markboth{\thechapter.\ #1}{}} % En el encabezado
    
  \renewcommand{\@makechapterhead}[1]{%
  \vspace*{50\p@}%
  {\parindent \z@ \raggedright \normalfont
    \ifnum \c@secnumdepth >\m@ne
      \huge\bfseries \thechapter.\hspace{1em}\ignorespaces
    \fi
    \interlinepenalty\@M
    \Huge \bfseries #1\par\nobreak
    \vskip 40\p@
  }}
}
\makeatother

%------------------------------------------------------------------------
% Paquetes de cógido
\usepackage{minted}
\renewcommand\listingscaption{Código fuente}

\usepackage{fancyvrb}
% Personaliza el tamaño de los números de línea
\renewcommand{\theFancyVerbLine}{\small\arabic{FancyVerbLine}}

% Estilo para C++
\newminted{cpp}{
    frame=lines,
    framesep=2mm,
    baselinestretch=1.2,
    linenos,
    escapeinside=||
}

% para minted
\definecolor{LightGray}{rgb}{0.95,0.95,0.92}
\setminted{
    linenos=true,
    stepnumber=5,
    numberfirstline=true,
    autogobble,
    breaklines=true,
    breakautoindent=true,
    breaksymbolleft=,
    breaksymbolright=,
    breaksymbolindentleft=0pt,
    breaksymbolindentright=0pt,
    breaksymbolsepleft=0pt,
    breaksymbolsepright=0pt,
    fontsize=\footnotesize,
    bgcolor=LightGray,
    numbersep=10pt
}


\usepackage{listings} % Para incluir código desde un archivo

\renewcommand\lstlistingname{Código Fuente}
\renewcommand\lstlistlistingname{Índice de Códigos Fuente}

% Definir colores
\definecolor{vscodepurple}{rgb}{0.5,0,0.5}
\definecolor{vscodeblue}{rgb}{0,0,0.8}
\definecolor{vscodegreen}{rgb}{0,0.5,0}
\definecolor{vscodegray}{rgb}{0.5,0.5,0.5}
\definecolor{vscodebackground}{rgb}{0.97,0.97,0.97}
\definecolor{vscodelightgray}{rgb}{0.9,0.9,0.9}

% Configuración para el estilo de C similar a VSCode
\lstdefinestyle{vscode_C}{
  backgroundcolor=\color{vscodebackground},
  commentstyle=\color{vscodegreen},
  keywordstyle=\color{vscodeblue},
  numberstyle=\tiny\color{vscodegray},
  stringstyle=\color{vscodepurple},
  basicstyle=\scriptsize\ttfamily,
  breakatwhitespace=false,
  breaklines=true,
  captionpos=b,
  keepspaces=true,
  numbers=left,
  numbersep=5pt,
  showspaces=false,
  showstringspaces=false,
  showtabs=false,
  tabsize=2,
  frame=tb,
  framerule=0pt,
  aboveskip=10pt,
  belowskip=10pt,
  xleftmargin=10pt,
  xrightmargin=10pt,
  framexleftmargin=10pt,
  framexrightmargin=10pt,
  framesep=0pt,
  rulecolor=\color{vscodelightgray},
  backgroundcolor=\color{vscodebackground},
}

%------------------------------------------------------------------------

% Comandos definidos
\newcommand{\bb}[1]{\mathbb{#1}}
\newcommand{\cc}[1]{\mathcal{#1}}

% I prefer the slanted \leq
\let\oldleq\leq % save them in case they're every wanted
\let\oldgeq\geq
\renewcommand{\leq}{\leqslant}
\renewcommand{\geq}{\geqslant}

% Si y solo si
\newcommand{\sii}{\iff}

% Letras griegas
\newcommand{\eps}{\epsilon}
\newcommand{\veps}{\varepsilon}
\newcommand{\lm}{\lambda}

\newcommand{\ol}{\overline}
\newcommand{\ul}{\underline}
\newcommand{\wt}{\widetilde}
\newcommand{\wh}{\widehat}

\let\oldvec\vec
\renewcommand{\vec}{\overrightarrow}

% Derivadas parciales
\newcommand{\del}[2]{\frac{\partial #1}{\partial #2}}
\newcommand{\Del}[3]{\frac{\partial^{#1} #2}{\partial #3^{#1}}}
\newcommand{\deld}[2]{\dfrac{\partial #1}{\partial #2}}
\newcommand{\Deld}[3]{\dfrac{\partial^{#1} #2}{\partial #3^{#1}}}


\newcommand{\AstIg}{\stackrel{(\ast)}{=}}
\newcommand{\Hop}{\stackrel{L'H\hat{o}pital}{=}}

\newcommand{\red}[1]{{\color{red}#1}} % Para integrales, destacar los cambios.

% Método de integración
\newcommand{\MetInt}[2]{
    \left[\begin{array}{c}
        #1 \\ #2
    \end{array}\right]
}

% Declarar aplicaciones
% 1. Nombre aplicación
% 2. Dominio
% 3. Codominio
% 4. Variable
% 5. Imagen de la variable
\newcommand{\Func}[5]{
    \begin{equation*}
        \begin{array}{rrll}
            #1:& #2 & \longrightarrow & #3\\
               & #4 & \longmapsto & #5
        \end{array}
    \end{equation*}
}

%------------------------------------------------------------------------



\begin{document}

    % 1. Foto de fondo
    % 2. Título
    % 3. Encabezado Izquierdo
    % 4. Color de fondo
    % 5. Coord x del titulo
    % 6. Coord y del titulo
    % 7. Fecha

    
    % 1. Foto de fondo
% 2. Título
% 3. Encabezado Izquierdo
% 4. Color de fondo
% 5. Coord x del titulo
% 6. Coord y del titulo
% 7. Fecha

\newcommand{\portada}[7]{

    \portadaBase{#1}{#2}{#3}{#4}{#5}{#6}{#7}
    \portadaBook{#1}{#2}{#3}{#4}{#5}{#6}{#7}
}

\newcommand{\portadaExamen}[7]{

    \portadaBase{#1}{#2}{#3}{#4}{#5}{#6}{#7}
    \portadaArticle{#1}{#2}{#3}{#4}{#5}{#6}{#7}
}




\newcommand{\portadaBase}[7]{

    % Tiene la portada principal y la licencia Creative Commons
    
    % 1. Foto de fondo
    % 2. Título
    % 3. Encabezado Izquierdo
    % 4. Color de fondo
    % 5. Coord x del titulo
    % 6. Coord y del titulo
    % 7. Fecha
    
    
    \thispagestyle{empty}               % Sin encabezado ni pie de página
    \newgeometry{margin=0cm}        % Márgenes nulos para la primera página
    
    
    % Encabezado
    \fancyhead[L]{\helv #3}
    \fancyhead[R]{\helv \nouppercase{\leftmark}}
    
    
    \pagecolor{#4}        % Color de fondo para la portada
    
    \begin{figure}[p]
        \centering
        \transparent{0.3}           % Opacidad del 30% para la imagen
        
        \includegraphics[width=\paperwidth, keepaspectratio]{assets/#1}
    
        \begin{tikzpicture}[remember picture, overlay]
            \node[anchor=north west, text=white, opacity=1, font=\fontsize{60}{90}\selectfont\bfseries\sffamily, align=left] at (#5, #6) {#2};
            
            \node[anchor=south east, text=white, opacity=1, font=\fontsize{12}{18}\selectfont\sffamily, align=right] at (9.7, 3) {\textbf{\href{https://losdeldgiim.github.io/}{Los Del DGIIM}}};
            
            \node[anchor=south east, text=white, opacity=1, font=\fontsize{12}{15}\selectfont\sffamily, align=right] at (9.7, 1.8) {Doble Grado en Ingeniería Informática y Matemáticas\\Universidad de Granada};
        \end{tikzpicture}
    \end{figure}
    
    
    \restoregeometry        % Restaurar márgenes normales para las páginas subsiguientes
    \pagecolor{white}       % Restaurar el color de página
    
    
    \newpage
    \thispagestyle{empty}               % Sin encabezado ni pie de página
    \begin{tikzpicture}[remember picture, overlay]
        \node[anchor=south west, inner sep=3cm] at (current page.south west) {
            \begin{minipage}{0.5\paperwidth}
                \href{https://creativecommons.org/licenses/by-nc-nd/4.0/}{
                    \includegraphics[height=2cm]{assets/Licencia.png}
                }\vspace{1cm}\\
                Esta obra está bajo una
                \href{https://creativecommons.org/licenses/by-nc-nd/4.0/}{
                    Licencia Creative Commons Atribución-NoComercial-SinDerivadas 4.0 Internacional (CC BY-NC-ND 4.0).
                }\\
    
                Eres libre de compartir y redistribuir el contenido de esta obra en cualquier medio o formato, siempre y cuando des el crédito adecuado a los autores originales y no persigas fines comerciales. 
            \end{minipage}
        };
    \end{tikzpicture}
    
    
    
    % 1. Foto de fondo
    % 2. Título
    % 3. Encabezado Izquierdo
    % 4. Color de fondo
    % 5. Coord x del titulo
    % 6. Coord y del titulo
    % 7. Fecha


}


\newcommand{\portadaBook}[7]{

    % 1. Foto de fondo
    % 2. Título
    % 3. Encabezado Izquierdo
    % 4. Color de fondo
    % 5. Coord x del titulo
    % 6. Coord y del titulo
    % 7. Fecha

    % Personaliza el formato del título
    \pretitle{\begin{center}\bfseries\fontsize{42}{56}\selectfont}
    \posttitle{\par\end{center}\vspace{2em}}
    
    % Personaliza el formato del autor
    \preauthor{\begin{center}\Large}
    \postauthor{\par\end{center}\vfill}
    
    % Personaliza el formato de la fecha
    \predate{\begin{center}\huge}
    \postdate{\par\end{center}\vspace{2em}}
    
    \title{#2}
    \author{\href{https://losdeldgiim.github.io/}{Los Del DGIIM}}
    \date{Granada, #7}
    \maketitle
    
    \tableofcontents
}




\newcommand{\portadaArticle}[7]{

    % 1. Foto de fondo
    % 2. Título
    % 3. Encabezado Izquierdo
    % 4. Color de fondo
    % 5. Coord x del titulo
    % 6. Coord y del titulo
    % 7. Fecha

    % Personaliza el formato del título
    \pretitle{\begin{center}\bfseries\fontsize{42}{56}\selectfont}
    \posttitle{\par\end{center}\vspace{2em}}
    
    % Personaliza el formato del autor
    \preauthor{\begin{center}\Large}
    \postauthor{\par\end{center}\vspace{3em}}
    
    % Personaliza el formato de la fecha
    \predate{\begin{center}\huge}
    \postdate{\par\end{center}\vspace{5em}}
    
    \title{#2}
    \author{\href{https://losdeldgiim.github.io/}{Los Del DGIIM}}
    \date{Granada, #7}
    \thispagestyle{empty}               % Sin encabezado ni pie de página
    \maketitle
    \vfill
}
    \portadaExamen{ffccA4.jpg}{Análisis Funcional\\Examen III}{Análisis Funcional. Examen III}{MidnightBlue}{-8}{28}{2025}{Daniel Morán Sánchez}

    \begin{description}
        \item[Asignatura] Análisis Funcional.
        \item[Curso Académico] 2024/25.
        \item[Grado] Grado en Matemáticas.
        % \item[Grupo] ---.
        % \item[Profesor] ---.
        \item[Descripción] Examen Ordinario.
        \item[Fecha] 15 de enero de 2025.
        % \item[Duración] ---.
    
    \end{description}
    \newpage


    % ------------------------------------
    
    \begin{ejercicio}[2 puntos]
        Sea $X = C([0,1],\mathbb{R})$ el espacio vectorial de las funciones continuas de $[0,1]$ a $\mathbb{R}$. Se define
        \begin{equation*}
            \|f\|_1 := \int_{0}^{1} |f(t)|~dt , \qquad \|f\|_\infty := \sup\{|f(t)| : t\in [0,1]\}\qquad \forall f\in X
        \end{equation*}
        y se escribe $X_1 = (X,\|\cdot \|_1)$, $X_{\infty} = (X,\|\cdot \|_\infty)$. Definimos el funcional lineal $\delta:X\to \mathbb{R}$ por
        \begin{equation*}
            \delta(f) = f(0) \qquad \forall f\in X
        \end{equation*}
        \begin{enumerate}[label=\alph*)]
            \item Demuestra que $\delta\in X_\infty^\ast$ y calcula su norma.
            \item ¿Es $\delta$ continuo respecto a $\|\cdot \|_1$? Justifica tu respuesta.
            \item ¿Son equivalente $\|\cdot \|_1$ y $\|\cdot \|_\infty$ en $X$? Justifica tu respuesta.
            \item ¿Es $X_1$ completo? Justifica tu respuesta.
        \end{enumerate}
    \end{ejercicio}

    \begin{ejercicio}[2 puntos]
        Sea $X$ un espacio reflexivo e $Y$ un espacio de Banach. Pruébese que si existe $T\in L(X,Y)$ sobreyectiva, entonces $Y$ es reflexivo.
    \end{ejercicio}

    \begin{ejercicio}[2 puntos]
        Sean $S:c_0\to c_0$ y $T:\ell^1\to \ell^1$ operadores lineales y supongamos que
        \begin{equation*}
            \sum_{k=1}^{\infty}[Sx](k)y(k) = \sum_{k=1}^{\infty}x(k)[Ty](k) \qquad \forall x\in c_0, \quad \forall y\in \ell^1
        \end{equation*}
        Demuestra que $S$ y $T$ son continuos.
    \end{ejercicio}

    \begin{ejercicio}[4 puntos]
        Desearrolla el siguiente tema: ``\textit{Mejor aproximación en espacios de Hilbert; teorema de la proyección ortogonal; teorema de Riesz-Fréchet}''.
    \end{ejercicio}

    \begin{ejercicio}
        (Ejercicio extra)\newline
        Sean $X,Y$ espacios normados y $T:X\to Y$ un operador verificando que $\overline{T(B_X)}$ es un subconjunto compacto de $Y$. Demuestra que $T^\ast$ alcanza su norma.
    \end{ejercicio}

    % // TODO: La solucion está en un pdf de esta carpeta

    \newpage
    \setcounter{ejercicio}{0} % Reiniciar contador de ejercicios

    \begin{ejercicio}[2 puntos]
        Sea $X = C([0,1],\mathbb{R})$ el espacio vectorial de las funciones continuas de $[0,1]$ a $\mathbb{R}$. Se define
        \begin{equation*}
            \|f\|_1 := \int_{0}^{1} |f(t)|~dt , \qquad \|f\|_\infty := \sup\{|f(t)| : t\in [0,1]\}\qquad \forall f\in X
        \end{equation*}
        y se escribe $X_1 = (X,\|\cdot \|_1)$, $X_{\infty} = (X,\|\cdot \|_\infty)$. Definimos el funcional lineal $\delta:X\to \mathbb{R}$ por
        \begin{equation*}
            \delta(f) = f(0) \qquad \forall f\in X
        \end{equation*}
        \begin{enumerate}[label=\alph*)]
            \item Demuestra que $\delta\in X_\infty^\ast$ y calcula su norma.

                Veamos que $\delta\in X_\infty^\ast$:
                \begin{itemize}
                    \item $\delta$ es lineal, pues si $f,g\in X$ y $\lm\in \mathbb{R}$ tenemos que:
                        \begin{equation*}
                            \delta(\lm f + g) = (\lm f + g)(0) = \lm f(0) + g(0) = \lm\delta(f) + \delta(g)
                        \end{equation*}
                    \item $\delta$ es continuo, pues si $f\in X$ tenemos:
                        \begin{equation*}
                            |\delta(f)| = |f(0)| \leq \sup_{t\in [0,1]}|f(t)| = \|f\|_\infty
                        \end{equation*}
                \end{itemize}
                Por lo que $\delta\in X_\infty^\ast$. De hecho, en el último apartado hemos probado además que $\|\delta\|\leq 1$. Veamos que $\|\delta\| = 1$, puesto que si consideramos cualquier función $f\in X$ de forma que $\max\limits_{t\in [0,1]}|f(t)| = |f(0)|$ tenemos entonces que:
                \begin{equation*}
                    |\delta(f)| = |f(0)| = \max_{t\in [0,1]}|f(t)| = \|f\|_\infty
                \end{equation*}
            \item ¿Es $\delta$ continuo respecto a $\|\cdot \|_1$? Justifica tu respuesta.

                Si $\delta$ fuera continuo respecto a $\|\cdot \|_1$, en particular será continua en la función constantemente igual a $0$, por lo que:
                \begin{equation*}
                    \forall \varepsilon>0~\exists \eta>0 : \|f\|_1 < \eta \Longrightarrow |\delta(f)|<\varepsilon
                \end{equation*}
                Es decir, si $\delta$ fuera continua podríamos acotar el valor de cualquier función $f\in X$ en $0$ sabiendo acotar el valor de su integral. No parece que esto sea posible, por lo que tratamos de probar que $\delta$ no es continua. Para ello, buscamos probar que:
                \begin{equation*}
                    \exists \varepsilon>0 : \forall \eta>0~\exists f\in X \quad \text{con}\quad \|f\|_1<\eta \quad \text{y}\quad |\delta(f)|>\varepsilon
                \end{equation*}
                Si consideramos $\varepsilon=\nicefrac{1}{2}$ y nos dan $\eta>0$, si consideramos la función cuya gráfica es:
                \begin{figure}[H]
                    \centering
                \begin{tikzpicture}
                \begin{axis}[
                    axis lines=middle,
                    xmin=0, xmax=1.1,
                    ymin=0, ymax=3.2,
                    xtick={0,1/3+0.11,1},
                    ytick={3},
                    xticklabels={$0$,$\eta$,$1$},
                    yticklabels={$1$},
                    width=10cm,
                    height=6cm,
                    samples=100,
                    domain=0:1,
                    thick
                ]

                % Parte lineal [0,1/3]
                \addplot[blue,domain=0:1/3+0.11]{3 - (3 - 3/4)/(1/3)*x};

                % Parte constante [1/3,1]
                \addplot[blue,domain=1/3+0.11:1]{0};
                \end{axis}
                \end{tikzpicture}
                \end{figure}
                \noindent
                Es decir, que en el intervalo $[0,\eta]$ es la recta que une el punto $(0,1)$ con el $(\eta,0)$ y en el intervalo $[\eta,1]$ vale $0$, tenemos que $f\in X$, así como que:
                \begin{equation*}
                    \|f\|_1 = \int_{0}^{1} |f(t)|~dt  = \int_{0}^{1} f(t)~dt  = \int_{0}^{\eta}f(t) ~dt  + \int_{\eta}^{1} f(t)~dt  = \frac{\eta}{2} < \eta
                \end{equation*}
                (ya que el área del triángulo es base por altura entre 2) y tenemos que:
                \begin{equation*}
                    |\delta(f)| = |f(0)| = 1 > \frac{1}{2}
                \end{equation*}
                Acabamos de probar que $\delta$ no es continua para $\|\cdot \|_1$.

            \item ¿Son equivalentes $\|\cdot \|_1$ y $\|\cdot \|_\infty$ en $X$? Justifica tu respuesta.

                No pueden ser equivalentes: si fueran equivalentes tendríamos que las topologías que da cada norma serían iguales, pero sin embargo tenemos que $\delta$ es continua para $\|\cdot \|_\infty$ y no es continua para $\|\cdot \|_1$, por lo que sus topologías no pueden contener los mismos abiertos (recordamos que $\delta$ es continua si y solo si la preimagen de todo abierto de $\mathbb{R}$ es un abierto en la topología que consideramos en $X$), por lo que las dos normas no pueden ser equivalentes.
            \item ¿Es $X_1$ completo? Justifica tu respuesta.

                No es completo, si consideramos la sucesión $\{f_n\}$ de funciones de $X$ donde cada $f_n$ es la función que en el intervalo $[0,\nicefrac{1}{n}]$ une el punto $(0,1)$ con el $(\nicefrac{1}{n},0)$ y en el intervalo $[\nicefrac{1}{n},1]$ es constantemente igual a 0, tendremos que la gráfica de cada función es:
                \begin{figure}[H]
                    \centering
                \begin{tikzpicture}
                \begin{axis}[
                    axis lines=middle,
                    xmin=0, xmax=1.1,
                    ymin=0, ymax=3.2,
                    xtick={0,1/3+0.11,1},
                    ytick={3},
                    xticklabels={$0$,$\tfrac{1}{n}$,$1$},
                    yticklabels={$1$},
                    width=10cm,
                    height=6cm,
                    samples=100,
                    domain=0:1,
                    thick
                ]

                % Parte lineal [0,1/3]
                \addplot[blue,domain=0:1/3+0.11]{3 - (3 - 3/4)/(1/3)*x};

                % Parte constante [1/3,1]
                \addplot[blue,domain=1/3+0.11:1]{0};
                \end{axis}
                \end{tikzpicture}
                \end{figure}
        \end{enumerate}
        Tenemos que esta sucesión es de Cauchy para $X_1$, pues si $n,m\in \mathbb{N}$ con $n<m$:
        \begin{equation*}
            \|f_n - f_m\|_1 = \int_{0}^{1} |f_n(t) - f_m(t)|~dt  = \int_{0}^{1} f_n(t)~dt -\int_{0}^{1} f_m(t)~dt  = \frac{1}{2}\left(\frac{1}{n}-\frac{1}{m}\right) < \frac{1}{2n}
        \end{equation*}
        Y sin embargo dicha sucesión de funciones no es convergente, pues en $L^1([0,1])$ convergen (con la misma norma) a la función:
        \begin{equation*}
            f(x) = \left\{\begin{array}{ll}
                1 & \text{si\ } x=0 \\
                0 & \text{si\ } x\neq 0
            \end{array}\right. 
        \end{equation*}
        Y tenemos que $f\notin X_1$, por lo que $\{f_n\}$ no es convergente en $X_1$ pero sí es de Cauchy, con lo que $X_1$ no puede ser completo.
    \end{ejercicio}
    \newpage
    
    \begin{ejercicio}[2 puntos]
         Sea $X$ un espacio reflexivo e $Y$ un espacio de Banach. Pruébese que si existe $T\in L(X,Y)$ sobreyectiva, entonces $Y$ es reflexivo.\\

         Sea $N:=\ker T$, que es un subespacio cerrado de $X$ (por continuidad de $T$).
        Consideremos el cociente $X/N$ con la norma cociente y la aplicación inducida, con la idea de obtener una aplicación biyectiva y aplicar el 1er corolario del teorema de la aplicación abierta
        \begin{equation*}
        \widetilde{T}:X/N \longrightarrow Y,
        \qquad \widetilde{T}([x]) := T(x),
        \end{equation*}
        donde $[x]=x+N$.

        \begin{itemize}
            \item $\widetilde{T}$ está bien definida y es lineal. Si $[x]=[x']$, entonces $x-x'\in N=\ker T$, luego $T(x)=T(x')$, y por tanto $\widetilde{T}$ está bien definida. La linealidad es inmediata.

            \item $\widetilde{T}$ es biyectiva.
                \begin{itemize}
                    \item Es sobreyectiva porque $T$ lo es: dado $y\in Y$ existe $x\in X$ con $T(x)=y$,
                    y entonces $\widetilde{T}([x])=y$.
                    \item Es inyectiva porque si $\widetilde{T}([x])=0$, entonces $T(x)=0$,
                    luego $x\in \ker T=N$, y por tanto $[x]=0$ en $X/N$.
                \end{itemize}

                \item $\widetilde{T}$ es un isomorfismo de Banach. Como $X$ es Banach y $N$ es cerrado, $X/N$ es Banach. Además, $\widetilde{T}$ es continua y biyectiva entre espacios de Banach, así que por el 1er corolario del teorema de la aplicación abierta su inversa $\widetilde{T}^{-1}$ es continua. Por tanto, $\widetilde{T}$ es un isomorfismo topológico.

                \item Como $X$ es reflexivo y $N$ es cerrado, el cociente $X/N$ es reflexivo. Finalmente, al ser $Y$ isomorfo a $X/N$, se deduce que $Y$ es reflexivo.
            
        \end{itemize}
    \end{ejercicio}
    \newpage

    \begin{ejercicio}[2 puntos]\ \\
        \noindent
        \textbf{Opción 1:} Directamente:\\
        Sean $S:c_0\to c_0$ y $T:\ell^1\to \ell^1$ operadores lineales y supongamos que
        \begin{equation*}
            \sum_{k=1}^{\infty}[Sx](k)y(k) = \sum_{k=1}^{\infty}x(k)[Ty](k) \qquad \forall x\in c_0, \quad \forall y\in \ell^1
        \end{equation*}
        Demuestra que $S$ y $T$ son continuos.\\
        Recordemos que las normas de dichos espacios son $\|\cdot\|_\infty$ para $c_0$ y $\|\cdot\|_1$ para $\ell^1$

        Recordemos el emparejamiento $c_0$--$\ell^1$:
        para $x\in c_0$ e $y\in \ell^1$ definimos
        \begin{equation*}
        \langle x,y\rangle:=\sum_{k=1}^\infty x(k)y(k),
        \end{equation*}
        que está bien definido (convergencia absoluta) y satisface
        \begin{equation*}
        |\langle x,y\rangle|\le \|x\|_\infty\|y\|_1.
        \end{equation*}
        La hipótesis se reescribe como
        \begin{equation*}
        \langle Sx,y\rangle=\langle x,Ty\rangle
        \qquad \forall x\in c_0,\ \forall y\in \ell^1.
        \tag{$\ast$}
        \end{equation*}
        
        \textbf{1) $S$ es continuo.}
        Sea $n\in\mathbb N$ y consideremos $e_n\in\ell^1$, definido por
        \begin{equation*}
        e_n(k)=
        \begin{cases}
        1, & k=n,\\
        0, & k\neq n.
        \end{cases}
        \end{equation*}
        Aplicando la identidad $(\ast)$ con $y=e_n$, se obtiene para todo $x\in c_0$
        \begin{equation*}
        (Sx)(n)=\sum_{k=1}^\infty x(k)\,(Te_n)(k).
        \end{equation*}
        Por tanto,
        \begin{equation*}
        |(Sx)(n)|
        \le
        \sum_{k=1}^\infty |x(k)|\,|(Te_n)(k)|
        \le
        \|x\|_\infty \sum_{k=1}^\infty |(Te_n)(k)|
        =
        \|x\|_\infty\,\|Te_n\|_1.
        \end{equation*}
        Tomando supremo en $n\in\mathbb N$ se deduce
        \begin{equation*}
        \|Sx\|_\infty
        =
        \sup_{n\in\mathbb N}|(Sx)(n)|
        \le
        \Big(\sup_{n\in\mathbb N}\|Te_n\|_1\Big)\,\|x\|_\infty,
        \qquad \forall x\in c_0.
        \tag{1}
        \end{equation*}
        Por tanto, si $\sup_n\|Te_n\|_1<\infty$, el operador $S$ es acotado.\\

        
    Definimos, para cada $n\in\mathbb N$, el funcional lineal
        \begin{equation*}
        \varphi_n:c_0\to\mathbb R,\qquad \varphi_n(x):=(Sx)(n).
        \end{equation*}
        Por $(\ast)$ con $y=e_n$ se tiene, para todo $x\in c_0$,
        \begin{equation*}
        \varphi_n(x)=(Sx)(n)=\langle Sx,e_n\rangle=\langle x,Te_n\rangle.
        \end{equation*}
        En particular
        \begin{equation*}
        |\varphi_n(x)|=|\langle x,Te_n\rangle|\le \|x\|_\infty\,\|Te_n\|_1,
        \end{equation*}
        luego $\varphi_n\in (c_0)^\ast$ y además $\|\varphi_n\|\le \|Te_n\|_1$.
        
        Por otro lado, como $Sx\in c_0\subset \ell^\infty$, para cada $x\in c_0$ se cumple
        \begin{equation*}
        \sup_{n\in\mathbb N}|\varphi_n(x)|
        =\sup_{n\in\mathbb N}|(Sx)(n)|
        =\|Sx\|_\infty
        <\infty.
        \end{equation*}
        Por tanto, la familia $\{\varphi_n\}_{n\in\mathbb N}\subset (c_0)^\ast$ es puntualmente acotada.
        Aplicando el teorema de Banach--Steinhaus, concluimos que
        \begin{equation*}
        \sup_{n\in\mathbb N}\|\varphi_n\|<\infty.
        \tag{2}
        \end{equation*}
        Finalmente, para cada $n$,
        \begin{equation*}
        \|Te_n\|_1
        = \sup_{\|x\|_\infty\le 1} |\langle x,Te_n\rangle|
        = \sup_{\|x\|_\infty\le 1} |\varphi_n(x)|
        =\|\varphi_n\|.
        \end{equation*}
        Juntando esto con (2) obtenemos
        \begin{equation*}
        \sup_{n\in\mathbb N}\|Te_n\|_1<\infty.
        \end{equation*}
        Sustituyendo en (1) concluimos que $S$ es acotado (luego continuo).
                
        \textbf{2) $T$ es continuo.}
        Sea $y\in \ell^1$ arbitrario. Para $N\in\mathbb{N}$ definimos $x^{(N)}\in c_0$ por
        \begin{equation*}
        x^{(N)}(k)=
        \begin{cases}
        \operatorname{sgn}([Ty](k)), & 1\le k\le N,\\
        0, & k>N.
        \end{cases}
        \end{equation*}
        Entonces $\|x^{(N)}\|_\infty\le 1$ y, usando $(\ast)$,
        \begin{equation*}
        \sum_{k=1}^{N} |[Ty](k)|
        = \sum_{k=1}^{N} x^{(N)}(k)[Ty](k)
        = \sum_{k=1}^{\infty} x^{(N)}(k)[Ty](k)
        = \langle x^{(N)},Ty\rangle
        = \langle Sx^{(N)},y\rangle.
        \end{equation*}
        Por la "desigualdad de Hölder" para el emparejamiento antes definido $\langle \cdot , \cdot \rangle$ para $c_0$--$\ell^1$,
        \begin{equation*}
        |\langle Sx^{(N)},y\rangle|
        \le \|Sx^{(N)}\|_\infty\,\|y\|_1
        \le \|S\|\,\|x^{(N)}\|_\infty\,\|y\|_1
        \le \|S\|\,\|y\|_1.
        \end{equation*}
        Luego, para todo $N$,
        \begin{equation*}
        \sum_{k=1}^{N} |[Ty](k)| \le \|S\|\,\|y\|_1.
        \end{equation*}
        Haciendo $N\to\infty$ obtenemos
        \begin{equation*}
        \|Ty\|_1=\sum_{k=1}^{\infty}|[Ty](k)| \le \|S\|\,\|y\|_1,
        \end{equation*}
        lo cual prueba que $T$ es acotado (y por tanto continuo).
        
    \end{ejercicio}
    \newpage

    \setcounter{ejercicio}{2} % Reiniciar contador de ejercicios
    \begin{ejercicio}[2 puntos]\ \\
        \noindent
        \textbf{Opción 2:} Usando el teorema de la gráfica cerrada.\\
        Sean $S:c_0\to c_0$ y $T:\ell^1\to \ell^1$ operadores lineales y supongamos que
        \begin{equation*}
        \sum_{k=1}^{\infty}(Sx)(k)\,y(k)=\sum_{k=1}^{\infty}x(k)\,(Ty)(k)
        \qquad \forall x\in c_0,\ \forall y\in \ell^1.
        \end{equation*}
        Demuestra que $S$ y $T$ son continuos.\\
        Recordemos el emparejamiento $c_0$--$\ell^1$:
        para $x\in c_0$ e $y\in \ell^1$ definimos
        \begin{equation*}
        \langle x,y\rangle:=\sum_{k=1}^\infty x(k)y(k),
        \end{equation*}
        que está bien definido (convergencia absoluta) y satisface
        \begin{equation*}
        |\langle x,y\rangle|\le \|x\|_\infty\|y\|_1.
        \end{equation*}
        La hipótesis se reescribe como
        \begin{equation*}
        \langle Sx,y\rangle=\langle x,Ty\rangle
        \qquad \forall x\in c_0,\ \forall y\in \ell^1.
        \tag{$\ast$}
        \end{equation*}
        
        
        \textbf{1) $S$ es continuo.}
        Como $c_0$(dominio e imagen) es espacio de Banach, por el teorema de la gráfica cerrada
        basta probar que la gráfica de $S$ es cerrada.
        
        Sea $\{x_n\}\subset c_0$ tal que $x_n\to x$ en $c_0$ y $Sx_n\to z$ en $c_0$.
        Queremos probar que $z=Sx$.
        
        Fijado $y\in \ell^1$, por la continuidad del emparejamiento en cada variable,
        \begin{equation*}
        \langle z,y\rangle=\lim_{n\to\infty}\langle Sx_n,y\rangle.
        \end{equation*}
        Usando $(\ast)$,
        \begin{equation*}
        \langle Sx_n,y\rangle=\langle x_n,Ty\rangle.
        \end{equation*}
        Además, como $x_n\to x$ en $c_0$ y $Ty\in \ell^1$, de nuevo por continuidad del
        emparejamiento,
        \begin{equation*}
        \lim_{n\to\infty}\langle x_n,Ty\rangle=\langle x,Ty\rangle.
        \end{equation*}
        
        En resumen:
        \begin{equation*}
        \langle z,y\rangle
        \xleftarrow[n\to\infty]{}
        \langle Sx_n,y\rangle=
        \langle x_n,Ty\rangle
        \xrightarrow[n\to\infty]{}
        \langle x,Ty\rangle,
        \end{equation*}
        y como $\langle Sx_n,y\rangle=\langle x_n,Ty\rangle$ para todo $n$, se deduce que
        \begin{equation*}
        \langle z,y\rangle=\langle x,Ty\rangle.
        \end{equation*}
        
        Por tanto,
        \begin{equation*}
        \langle z,y\rangle=\langle x,Ty\rangle.
        \end{equation*}
        Aplicando otra vez $(\ast)$ (pero con $x$ fijo),
        \begin{equation*}
        \langle x,Ty\rangle=\langle Sx,y\rangle,
        \end{equation*}
        y obtenemos
        \begin{equation*}
        \langle z,y\rangle=\langle Sx,y\rangle \qquad \forall y\in \ell^1.
        \end{equation*}
        Luego
        \begin{equation*}
        \langle z-Sx,y\rangle=0 \qquad \forall y\in \ell^1.
        \end{equation*}
        Como $\ell^1=(c_0)^\ast$ y los funcionales separan puntos\footnote{es decir, si $\forall y \in \ell^1$ visto como funcional aplicado a $x \in c_0$ es igual a $0$ entonces $x=0$}, se deduce $z-Sx=0$, es decir,
        $z=Sx$. La gráfica de $S$ es cerrada y, por el teorema de la gráfica cerrada, $S$ es continuo.
        
        \newpage
        
        \textbf{2) $T$ es continuo.}

        \begin{itemize}
            \item \textbf{Opción 1}
        
            Análogamente, como $\ell^1$ y $\ell^1$ son Banach, basta probar que la gráfica de $T$ es cerrada.
            
            Sea $(y_n)\subset \ell^1$ tal que $y_n\to y$ en $\ell^1$ y $Ty_n\to w$ en $\ell^1$.
            Queremos ver que $w=Ty$.
            
            Fijado $x\in c_0$, por continuidad del emparejamiento,
            \begin{equation*}
            \langle x,w\rangle=\lim_{n\to\infty}\langle x,Ty_n\rangle.
            \end{equation*}
            Usando $(\ast)$,
            \begin{equation*}
            \langle x,Ty_n\rangle=\langle Sx,y_n\rangle.
            \end{equation*}
            Como $Sx\in c_0$ y $y_n\to y$ en $\ell^1$, otra vez por continuidad,
            \begin{equation*}
            \lim_{n\to\infty}\langle Sx,y_n\rangle=\langle Sx,y\rangle.
            \end{equation*}

            En resumen:
            \begin{equation*}
            \langle x,w\rangle
            \xleftarrow[n\to\infty]{}
            \langle x,Ty_n\rangle=
            \langle Sx,y_n\rangle
            \xrightarrow[n\to\infty]{}
            \langle Sx,y\rangle,
            \end{equation*}
            
            Por tanto,
            \begin{equation*}
            \langle x,w\rangle=\langle Sx,y\rangle.
            \end{equation*}
            Y por $(\ast)$ aplicado a $(x,y)$,
            \begin{equation*}
            \langle Sx,y\rangle=\langle x,Ty\rangle,
            \end{equation*}
            con lo que
            \begin{equation*}
            \langle x,w\rangle=\langle x,Ty\rangle \qquad \forall x\in c_0.
            \end{equation*}
            Luego
            \begin{equation*}
            \langle x,w-Ty\rangle=0 \qquad \forall x\in c_0.
            \end{equation*}
            Como $c_0=(\ell^1)^\ast$ separa puntos de $\ell^1$, se concluye $w-Ty=0$, esto es,
            $w=Ty$. La gráfica de $T$ es cerrada y, por el teorema de la gráfica cerrada, $T$ es continuo.
            
            \item \textbf{Opción 2}

            Una vez probado que $S$ es continuo, deducimos la continuidad de $T$ identificándolo con el adjunto de $S$.

            Recordemos que $c_0^\ast\simeq \ell^1$ mediante el emparejamiento
            \begin{equation*}
            \langle x,y\rangle=\sum_{k=1}^\infty x(k)y(k),
            \qquad x\in c_0,\ y\in \ell^1.
            \end{equation*}
            Sea $y\in \ell^1$. Definimos $S^\ast y\in \ell^1$ como el funcional de $c_0^\ast$ dado por
            \begin{equation*}
            \langle x,S^\ast y\rangle:=\langle Sx,y\rangle
            \qquad \forall x\in c_0.
            \end{equation*}
            Entonces, usando la hipótesis $(\ast)$, para todo $x\in c_0$ se tiene
            \begin{equation*}
            \langle x,S^\ast y\rangle
            =\langle Sx,y\rangle
            =\langle x,Ty\rangle.
            \end{equation*}
            Es decir,
            \begin{equation*}
            \langle x,S^\ast y-Ty\rangle=0
            \qquad \forall x\in c_0.
            \end{equation*}
            Como los elementos de $c_0$ separan puntos de $\ell^1$ (por ejemplo, tomando $x=e_n\in c_0$
            se obtiene la $n$-ésima coordenada), concluimos que $S^\ast y=Ty$ para todo $y\in \ell^1$.
            Por tanto,
            \begin{equation*}
            T=S^\ast.
            \end{equation*}
            
            Finalmente, como $S$ es continuo (acotado), su adjunto $S^\ast:\ell^1\to \ell^1$ también es continuo y satisface
            \begin{equation*}
            \|T\|=\|S^\ast\|=\|S\|.
            \end{equation*}
            En particular, $T$ es continuo.
            
        \end{itemize}
    \end{ejercicio}
    \newpage

    \begin{ejercicio}[4 puntos]
        Desearrolla el siguiente tema: ``\textit{Mejor aproximación en espacios de Hilbert; teorema de la proyección ortogonal; teorema de Riesz-Fréchet}''.
        \\
        \\
        Consultar los apuntes de teoría
    \end{ejercicio}
    \newpage

    \begin{ejercicio}
        (Ejercicio extra)\newline
        Sean $X,Y$ espacios normados y $T:X\to Y$ un operador verificando que $\overline{T(B_X)}$ es un subconjunto compacto de $Y$. Demuestra que $T^\ast$ alcanza su norma.\\

        \begin{enumerate}
            \item Sea $K:=\overline{T(B_X)}\subset Y$. Por hipótesis, $K$ es compacto; en particular, es acotado, luego $T(B_X)$ es acotado y por tanto $T$ es acotado (continuo).
        
            \item Si $T=0$, entonces $T^\ast=0$ y alcanza su norma trivialmente. Supongamos $T\neq 0$.
            Por definición de norma de un funcional,
            \begin{equation*}
                \|T\|=\sup_{\|x\|\le 1}\|Tx\|=\sup_{y\in T(B_X)}\|y\|=\sup_{y\in K}\|y\|.
            \end{equation*}
            Como $K$ es compacto y la función $y\mapsto \|y\|$ es continua, existe $y_0\in K$ tal que
            \begin{equation*}
                \|y_0\|=\max_{y\in K}\|y\|=\|T\|.
            \end{equation*}
            Además, como $y_0\in K=\overline{T(B_X)}$, existe una sucesión $\{x_n\}\subset B_X$ tal que
            \begin{equation*}
                Tx_n \to y_0 \quad \text{en } Y.
            \end{equation*}
            Aplicando continuidad de la norma,
            \begin{equation*}
                \|Tx_n\|\to \|y_0\|=\|T\|.
            \end{equation*}
            
            \item Por el teorema de Hahn--Banach (forma de soporte), existe $y_0^\ast\in Y^\ast$ con
            \begin{equation*}
                \|y_0^\ast\|=1
                \qquad\text{y}\qquad
                y_0^\ast(y_0)=\|y_0\|=\|T\|.
            \end{equation*}
            Entonces, para cada $n$,
            \begin{equation*}
                (T^\ast y_0^\ast)(x_n)=y_0^\ast(Tx_n).
            \end{equation*}
            Tomando valores absolutos y pasando al límite usando que $Tx_n\to y_0$ y la continuidad de $y_0^\ast$,
            \begin{equation*}
                \lim_{n\to\infty} |(T^\ast y_0^\ast)(x_n)|
                =
                \lim_{n\to\infty} |y_0^\ast(Tx_n)|
                =
                |y_0^\ast(y_0)|
                =
                \|T\|.
            \end{equation*}
            
            \item Pero como $\|x_n\|\le 1$ para todo $n$,
            \begin{equation*}
                |(T^\ast y_0^\ast)(x_n)|\le \|T^\ast y_0^\ast\| \qquad \forall n \in \mathbb{N},
            \end{equation*}
            y por tanto
            \begin{equation*}
                \|T^\ast y_0^\ast\|\ge \|T\|.
            \end{equation*}
            Por otro lado, siempre se cumple $\|T^\ast\|=\|T\|$ y, en particular,
            \begin{equation*}
                \|T^\ast y_0^\ast\|\le \|T^\ast\|\,\|y_0^\ast\|=\|T\|.
            \end{equation*}
            Concluimos que
            \begin{equation*}
                \|T^\ast y_0^\ast\|=\|T^\ast\|.
            \end{equation*}
            Es decir, $T^\ast$ alcanza su norma en el funcional $y_0^\ast\in Y^\ast$ con $\|y_0^\ast\|=1$.
            
    \end{enumerate}
    \end{ejercicio}

\end{document}
