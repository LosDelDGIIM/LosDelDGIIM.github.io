\documentclass[12pt]{article}

% Idioma y codificación
\usepackage[spanish, es-tabla]{babel}       %es-tabla para que se titule "Tabla"
\usepackage[utf8]{inputenc}

% Márgenes
\usepackage[a4paper,top=3cm,bottom=2.5cm,left=3cm,right=3cm]{geometry}

% Comentarios de bloque
\usepackage{verbatim}

% Paquetes de links
\usepackage[hidelinks]{hyperref}    % Permite enlaces
\usepackage{url}                    % redirecciona a la web

% Más opciones para enumeraciones
\usepackage{enumitem}

% Personalizar la portada
\usepackage{titling}

% Paquetes de tablas
\usepackage{multirow}


%------------------------------------------------------------------------

%Paquetes de figuras
\usepackage{caption}
\usepackage{subcaption} % Figuras al lado de otras
\usepackage{float}      % Poner figuras en el sitio indicado H.


% Paquetes de imágenes
\usepackage{graphicx}       % Paquete para añadir imágenes
\usepackage{transparent}    % Para manejar la opacidad de las figuras

% Paquete para usar colores
\usepackage[dvipsnames]{xcolor}
\usepackage{pagecolor}      % Para cambiar el color de la página

% Habilita tamaños de fuente mayores
\usepackage{fix-cm}

% Para los gráficos
\usepackage{tikz}

% Para poder situar los nodos en los grafos
\usetikzlibrary{positioning}


%------------------------------------------------------------------------

% Paquetes de matemáticas
\usepackage{mathtools, amsfonts, amssymb, mathrsfs}
\usepackage[makeroom]{cancel}     % Simplificar tachando
\usepackage{polynom}    % Divisiones y Ruffini
\usepackage{units} % Para poner fracciones diagonales con \nicefrac

\usepackage{pgfplots}   %Representar funciones
\pgfplotsset{compat=1.18}  % Versión 1.18

\usepackage{tikz-cd}    % Para usar diagramas de composiciones
\usetikzlibrary{calc}   % Para usar cálculo de coordenadas en tikz

%Definición de teoremas, etc.
\usepackage{amsthm}
%\swapnumbers   % Intercambia la posición del texto y de la numeración

\theoremstyle{plain}

\makeatletter
\@ifclassloaded{article}{
  \newtheorem{teo}{Teorema}[section]
}{
  \newtheorem{teo}{Teorema}[chapter]  % Se resetea en cada chapter
}
\makeatother

\newtheorem{coro}{Corolario}[teo]           % Se resetea en cada teorema
\newtheorem{prop}[teo]{Proposición}         % Usa el mismo contador que teorema
\newtheorem{lema}[teo]{Lema}                % Usa el mismo contador que teorema

\theoremstyle{remark}
\newtheorem*{observacion}{Observación}

\theoremstyle{definition}

\makeatletter
\@ifclassloaded{article}{
  \newtheorem{definicion}{Definición} [section]     % Se resetea en cada chapter
}{
  \newtheorem{definicion}{Definición} [chapter]     % Se resetea en cada chapter
}
\makeatother

\newtheorem*{notacion}{Notación}
\newtheorem*{ejemplo}{Ejemplo}
\newtheorem*{ejercicio*}{Ejercicio}             % No numerado
\newtheorem{ejercicio}{Ejercicio} [section]     % Se resetea en cada section


% Modificar el formato de la numeración del teorema "ejercicio"
\renewcommand{\theejercicio}{%
  \ifnum\value{section}=0 % Si no se ha iniciado ninguna sección
    \arabic{ejercicio}% Solo mostrar el número de ejercicio
  \else
    \thesection.\arabic{ejercicio}% Mostrar número de sección y número de ejercicio
  \fi
}


% \renewcommand\qedsymbol{$\blacksquare$}         % Cambiar símbolo QED
%------------------------------------------------------------------------

% Paquetes para encabezados
\usepackage{fancyhdr}
\pagestyle{fancy}
\fancyhf{}

\newcommand{\helv}{ % Modificación tamaño de letra
\fontfamily{}\fontsize{12}{12}\selectfont}
\setlength{\headheight}{15pt} % Amplía el tamaño del índice


%\usepackage{lastpage}   % Referenciar última pag   \pageref{LastPage}
\fancyfoot[C]{\thepage}

%------------------------------------------------------------------------

% Conseguir que no ponga "Capítulo 1". Sino solo "1."
\makeatletter
\@ifclassloaded{book}{
  \renewcommand{\chaptermark}[1]{\markboth{\thechapter.\ #1}{}} % En el encabezado
    
  \renewcommand{\@makechapterhead}[1]{%
  \vspace*{50\p@}%
  {\parindent \z@ \raggedright \normalfont
    \ifnum \c@secnumdepth >\m@ne
      \huge\bfseries \thechapter.\hspace{1em}\ignorespaces
    \fi
    \interlinepenalty\@M
    \Huge \bfseries #1\par\nobreak
    \vskip 40\p@
  }}
}
\makeatother

%------------------------------------------------------------------------
% Paquetes de cógido
\usepackage{minted}
\renewcommand\listingscaption{Código fuente}

\usepackage{fancyvrb}
% Personaliza el tamaño de los números de línea
\renewcommand{\theFancyVerbLine}{\small\arabic{FancyVerbLine}}

% Estilo para C++
\newminted{cpp}{
    frame=lines,
    framesep=2mm,
    baselinestretch=1.2,
    linenos,
    escapeinside=||
}

% para minted
\definecolor{LightGray}{rgb}{0.95,0.95,0.92}
\setminted{
    linenos=true,
    stepnumber=5,
    numberfirstline=true,
    autogobble,
    breaklines=true,
    breakautoindent=true,
    breaksymbolleft=,
    breaksymbolright=,
    breaksymbolindentleft=0pt,
    breaksymbolindentright=0pt,
    breaksymbolsepleft=0pt,
    breaksymbolsepright=0pt,
    fontsize=\footnotesize,
    bgcolor=LightGray,
    numbersep=10pt
}


\usepackage{listings} % Para incluir código desde un archivo

\renewcommand\lstlistingname{Código Fuente}
\renewcommand\lstlistlistingname{Índice de Códigos Fuente}

% Definir colores
\definecolor{vscodepurple}{rgb}{0.5,0,0.5}
\definecolor{vscodeblue}{rgb}{0,0,0.8}
\definecolor{vscodegreen}{rgb}{0,0.5,0}
\definecolor{vscodegray}{rgb}{0.5,0.5,0.5}
\definecolor{vscodebackground}{rgb}{0.97,0.97,0.97}
\definecolor{vscodelightgray}{rgb}{0.9,0.9,0.9}

% Configuración para el estilo de C similar a VSCode
\lstdefinestyle{vscode_C}{
  backgroundcolor=\color{vscodebackground},
  commentstyle=\color{vscodegreen},
  keywordstyle=\color{vscodeblue},
  numberstyle=\tiny\color{vscodegray},
  stringstyle=\color{vscodepurple},
  basicstyle=\scriptsize\ttfamily,
  breakatwhitespace=false,
  breaklines=true,
  captionpos=b,
  keepspaces=true,
  numbers=left,
  numbersep=5pt,
  showspaces=false,
  showstringspaces=false,
  showtabs=false,
  tabsize=2,
  frame=tb,
  framerule=0pt,
  aboveskip=10pt,
  belowskip=10pt,
  xleftmargin=10pt,
  xrightmargin=10pt,
  framexleftmargin=10pt,
  framexrightmargin=10pt,
  framesep=0pt,
  rulecolor=\color{vscodelightgray},
  backgroundcolor=\color{vscodebackground},
}

%------------------------------------------------------------------------

% Comandos definidos
\newcommand{\bb}[1]{\mathbb{#1}}
\newcommand{\cc}[1]{\mathcal{#1}}

% I prefer the slanted \leq
\let\oldleq\leq % save them in case they're every wanted
\let\oldgeq\geq
\renewcommand{\leq}{\leqslant}
\renewcommand{\geq}{\geqslant}

% Si y solo si
\newcommand{\sii}{\iff}

% Letras griegas
\newcommand{\eps}{\epsilon}
\newcommand{\veps}{\varepsilon}
\newcommand{\lm}{\lambda}

\newcommand{\ol}{\overline}
\newcommand{\ul}{\underline}
\newcommand{\wt}{\widetilde}
\newcommand{\wh}{\widehat}

\let\oldvec\vec
\renewcommand{\vec}{\overrightarrow}

% Derivadas parciales
\newcommand{\del}[2]{\frac{\partial #1}{\partial #2}}
\newcommand{\Del}[3]{\frac{\partial^{#1} #2}{\partial #3^{#1}}}
\newcommand{\deld}[2]{\dfrac{\partial #1}{\partial #2}}
\newcommand{\Deld}[3]{\dfrac{\partial^{#1} #2}{\partial #3^{#1}}}


\newcommand{\AstIg}{\stackrel{(\ast)}{=}}
\newcommand{\Hop}{\stackrel{L'H\hat{o}pital}{=}}

\newcommand{\red}[1]{{\color{red}#1}} % Para integrales, destacar los cambios.

% Método de integración
\newcommand{\MetInt}[2]{
    \left[\begin{array}{c}
        #1 \\ #2
    \end{array}\right]
}

% Declarar aplicaciones
% 1. Nombre aplicación
% 2. Dominio
% 3. Codominio
% 4. Variable
% 5. Imagen de la variable
\newcommand{\Func}[5]{
    \begin{equation*}
        \begin{array}{rrll}
            #1:& #2 & \longrightarrow & #3\\
               & #4 & \longmapsto & #5
        \end{array}
    \end{equation*}
}

%------------------------------------------------------------------------


\begin{document}

    % 1. Foto de fondo
    % 2. Título
    % 3. Encabezado Izquierdo
    % 4. Color de fondo
    % 5. Coord x del titulo
    % 6. Coord y del titulo
    % 7. Fecha

    
    % 1. Foto de fondo
% 2. Título
% 3. Encabezado Izquierdo
% 4. Color de fondo
% 5. Coord x del titulo
% 6. Coord y del titulo
% 7. Fecha

\newcommand{\portada}[7]{

    \portadaBase{#1}{#2}{#3}{#4}{#5}{#6}{#7}
    \portadaBook{#1}{#2}{#3}{#4}{#5}{#6}{#7}
}

\newcommand{\portadaExamen}[7]{

    \portadaBase{#1}{#2}{#3}{#4}{#5}{#6}{#7}
    \portadaArticle{#1}{#2}{#3}{#4}{#5}{#6}{#7}
}




\newcommand{\portadaBase}[7]{

    % Tiene la portada principal y la licencia Creative Commons
    
    % 1. Foto de fondo
    % 2. Título
    % 3. Encabezado Izquierdo
    % 4. Color de fondo
    % 5. Coord x del titulo
    % 6. Coord y del titulo
    % 7. Fecha
    
    
    \thispagestyle{empty}               % Sin encabezado ni pie de página
    \newgeometry{margin=0cm}        % Márgenes nulos para la primera página
    
    
    % Encabezado
    \fancyhead[L]{\helv #3}
    \fancyhead[R]{\helv \nouppercase{\leftmark}}
    
    
    \pagecolor{#4}        % Color de fondo para la portada
    
    \begin{figure}[p]
        \centering
        \transparent{0.3}           % Opacidad del 30% para la imagen
        
        \includegraphics[width=\paperwidth, keepaspectratio]{assets/#1}
    
        \begin{tikzpicture}[remember picture, overlay]
            \node[anchor=north west, text=white, opacity=1, font=\fontsize{60}{90}\selectfont\bfseries\sffamily, align=left] at (#5, #6) {#2};
            
            \node[anchor=south east, text=white, opacity=1, font=\fontsize{12}{18}\selectfont\sffamily, align=right] at (9.7, 3) {\textbf{\href{https://losdeldgiim.github.io/}{Los Del DGIIM}}};
            
            \node[anchor=south east, text=white, opacity=1, font=\fontsize{12}{15}\selectfont\sffamily, align=right] at (9.7, 1.8) {Doble Grado en Ingeniería Informática y Matemáticas\\Universidad de Granada};
        \end{tikzpicture}
    \end{figure}
    
    
    \restoregeometry        % Restaurar márgenes normales para las páginas subsiguientes
    \pagecolor{white}       % Restaurar el color de página
    
    
    \newpage
    \thispagestyle{empty}               % Sin encabezado ni pie de página
    \begin{tikzpicture}[remember picture, overlay]
        \node[anchor=south west, inner sep=3cm] at (current page.south west) {
            \begin{minipage}{0.5\paperwidth}
                \href{https://creativecommons.org/licenses/by-nc-nd/4.0/}{
                    \includegraphics[height=2cm]{assets/Licencia.png}
                }\vspace{1cm}\\
                Esta obra está bajo una
                \href{https://creativecommons.org/licenses/by-nc-nd/4.0/}{
                    Licencia Creative Commons Atribución-NoComercial-SinDerivadas 4.0 Internacional (CC BY-NC-ND 4.0).
                }\\
    
                Eres libre de compartir y redistribuir el contenido de esta obra en cualquier medio o formato, siempre y cuando des el crédito adecuado a los autores originales y no persigas fines comerciales. 
            \end{minipage}
        };
    \end{tikzpicture}
    
    
    
    % 1. Foto de fondo
    % 2. Título
    % 3. Encabezado Izquierdo
    % 4. Color de fondo
    % 5. Coord x del titulo
    % 6. Coord y del titulo
    % 7. Fecha


}


\newcommand{\portadaBook}[7]{

    % 1. Foto de fondo
    % 2. Título
    % 3. Encabezado Izquierdo
    % 4. Color de fondo
    % 5. Coord x del titulo
    % 6. Coord y del titulo
    % 7. Fecha

    % Personaliza el formato del título
    \pretitle{\begin{center}\bfseries\fontsize{42}{56}\selectfont}
    \posttitle{\par\end{center}\vspace{2em}}
    
    % Personaliza el formato del autor
    \preauthor{\begin{center}\Large}
    \postauthor{\par\end{center}\vfill}
    
    % Personaliza el formato de la fecha
    \predate{\begin{center}\huge}
    \postdate{\par\end{center}\vspace{2em}}
    
    \title{#2}
    \author{\href{https://losdeldgiim.github.io/}{Los Del DGIIM}}
    \date{Granada, #7}
    \maketitle
    
    \tableofcontents
}




\newcommand{\portadaArticle}[7]{

    % 1. Foto de fondo
    % 2. Título
    % 3. Encabezado Izquierdo
    % 4. Color de fondo
    % 5. Coord x del titulo
    % 6. Coord y del titulo
    % 7. Fecha

    % Personaliza el formato del título
    \pretitle{\begin{center}\bfseries\fontsize{42}{56}\selectfont}
    \posttitle{\par\end{center}\vspace{2em}}
    
    % Personaliza el formato del autor
    \preauthor{\begin{center}\Large}
    \postauthor{\par\end{center}\vspace{3em}}
    
    % Personaliza el formato de la fecha
    \predate{\begin{center}\huge}
    \postdate{\par\end{center}\vspace{5em}}
    
    \title{#2}
    \author{\href{https://losdeldgiim.github.io/}{Los Del DGIIM}}
    \date{Granada, #7}
    \thispagestyle{empty}               % Sin encabezado ni pie de página
    \maketitle
    \vfill
}
    \portadaExamen{ffccA4.jpg}{Geometría I\\Examen XI}{Geometría I. Examen XI}{MidnightBlue}{-8}{28}{2023-2024}{Jesús Muñoz Velasco}

    \begin{description}
        \item[Asignatura] Geometría I.
        \item[Curso Académico] 2023-24.
        \item[Grado] Doble Grado en Ingeniería Informática y Matemáticas.
        \item[Grupo] Único.
        \item[Profesor] Antonio Ros Mulero.
        \item[Descripción] Convocatoria Extraordinaria\footnote{El examen lo pone el departamento.}.
        \item[Fecha] 8 de febrero de 2024.
        % \item[Duración] 3 horas.
    \end{description}
    \newpage

    \begin{ejercicio}[2.5 puntos]
        Sea $V(\bb{K})$ un espacio vectorial sobre $\bb{K} = \bb{R}$ ó $\bb{C}$ con dimensión $n\geq 2$, y sean $U,W$ dos subespacios vectoriales no triviales de $V$ tales que $V = U \oplus W$.

        \begin{enumerate}
            \item \textbf{1.25 puntos} Construir razonadamente una base de $V/U$ (es decir, probando que cumple las condiciones para ser base).
            
            Demostrado en clase.

            \item \textbf{1.25 puntos} Encontrar explícitamente un isomorfismo de $V/U$ en $W$.
            
            Como $\dim(V/U) = \dim(V) - \dim(U) = \dim(W)$, los dos espacios vectoriales son isomorfos. Para construir el isomorfismo vamos a asignar imágenes a una vase de $V/U$ de forma que estas imágenes constituyan una base de $W$, ya que en este caso la aplicación lineal estará totalmente determinada y como la aplicación lineal lleva una base de $V/U$ en una base de $W$ entonces será un isomorfismo de espacios vectoriales.

            Como se ha visto en el apartado anterior, si $\{w_1, \dots, w_k\}$ es una base de $W$, entonces $\{w_1 + U, \dots, w_k + U\}$ es una vase de $V/U$. Definimos $f$ como la única aplicación lineal $f : V/U \rightarrow W$ tal que $f(w_i+U) = w_i$ para todo $i = 1, \dots, k$. Por lo comentado anteriormente, $f$ es un isomorfismo.
        \end{enumerate}
    \end{ejercicio}

    \begin{ejercicio}[2.5 puntos]
        Sea $V(\bb{R})$ un espacio vectorial real con dimensión finita y $f$ un endomorfismo de $V$ que cumple $f \circ f = 4f$.

        \begin{enumerate}
            \item \textbf{1.25 puntos} Probar que $V = \ker(f) \oplus Im(f)$.
            
            Veamos que $\ker(f) \cap Im(f) = \{0\}$: Sea $x \in Im(f)$. Como $x\in Im(f)$, existe $z \in V$ tal que $f(z)=x$. Como $x\in \ker(f)$, tenemos $0 = f(x) = f(f(z)) = (f \circ f)(z) = 4f(z) = 4x$, de donde $x=0$.

            Como $\ker(f) \cap Im(f) = \{0\}$ y $\dim(\ker(f)) + \dim(Im(f)) = \dim(V)$, por la fórmula de las dimensiones concluimos que
            \begin{gather*}
                \dim(\ker(f)+Im(f)) = \dim(\ker(f)) + \dim(Im(f)) - \dim(\ker(f)\cap Im(f)) = \dim(V)
            \end{gather*}
            y $V = \ker(f) \oplus Im(f).$

            \item \textbf{1.25 puntos} Demostrar que existe una base $\cc{B}$ de $V$ tal que 
            \begin{gather*}
                M(f, \cc{B}) = 
                \left(
                \begin{array}{c|c}
                    4 I_r &  0\\
                    \hline
                    0 & 0 \\
                \end{array}
                \right)
            \end{gather*}
            para algún $r \in \{0,1,\dots,\dim(V)\}$.

            Sea $r$ el rango de $f$. Como $V = \ker(f) + Im(f)$, por el apartado (a), concluimos de la fórmula de la nulidad y el rango que la nulidad de $f$ es $n -r$, siendo $n = \dim_{\bb{K}}(V)$.

            Tenemos ahora bases $\{x_1, \dots, x_r\}$ de $Im(f)$ y $\{x_{r+1}, \dots, x_n\}$ de $\ker(f)$. De nuevo, por ser $V = \ker(f) \oplus Im(f)$, concluimos que $\{x_1, \dots, x_r, x_{r+1}, \dots, x_n\}$ es base de $V$. Ordenamos esa base tal y como la hemos escrito, llamamos $\cc{B}$ a la base ordenada y calculamos $;(f,\cc{B})$:

            Para cada $i = 1, \dots, r$ existe $z_i \in V$ tal que $f(z_i = x_i)$ (porque $x_i \in Im(f)$), luego $f(x_i) = f(f(z_i)) = (f \circ f)(z_i) = 4f(z_i) = 4z_i$, que en coordenadas respecto de $\cc{B}$ es $(0,\dots,4,\dots,0)$ donde $4$ está en la posición $i$. Para cada $i = r+1,\dots,n$ tenemos $x_i \in \ker(f)$, luego $f(x_i) = 0$, que en coordenadas respecto de $\cc{B}$ es $(0,\dots, 0)$. Juntando por columnas todo esto, deducimos que
            \begin{gather*}
                M(f, \cc{B}) = 
                \left(
                \begin{array}{c|c}
                    4 I_r &  0\\
                    \hline
                    0 & 0 \\
                \end{array}
                \right)
            \end{gather*} 
        \end{enumerate}
    \end{ejercicio}
    
    \begin{ejercicio}[2.5 puntos]
        En el espacio vectorial $\bb{R}_2[x]$ de los polinomios con coeficientes reales y grado $\leq 2$, se considera la base ordenada usual $\cc{B}=(1,x,x^2)$ y el endomorfismo $f_k$ de $\bb{R}_2[x]$ cuya matriz respecto de $\cc{B}$ es 
        \begin{gather*}
            M(f_k, \cc{B}) = 
            \begin{pmatrix}
                1 & 1 & -2 \\
                0 & k-1 & 1 \\
                1 & k^2 & 0 \\
            \end{pmatrix}
        \end{gather*}
        siendo $k\in \bb{R}$ un parámetro.
        \begin{enumerate}
            \item \textbf{(1.25 puntos)} Hallar $\ker(f_k)$ e $Im(f_k)$ explícitamente en función de $k$. ¿Para qué valores de $k$ es $f_k$ inyectiva? ¿Y sobreyectiva?
            
            Los valores de $k\in \bb{R}$ para los que $k$ es inyectiva son aquellos para los que la matriz $M(f_k, \cc{B})$ es regular, es decir, su determinante es distinto de cero. Dicho determinante es $-(k-1)^2$, luego $f_k$ es inyectiva si y sólo si $k\neq 1$. Por ser $f_k$ un endomorfismo, es inyectiva si y sólo si es automorfismo, con lo que los mismos valores $k \neq 1$ son aquellos para los que $f_k$ es sobreyectiva.

            Ahora determinamos el núcleo e imagen de $f_k$. Si $k\neq 1$, sabemos que $f_k$ es un automorfismo, luego $\ker(f_k) = \{0\}$ e $Im(f_k) = \bb{R}_2[x]$. Queda hallar $\ker(f_1)$ e $Im(f_1)$:

            Trabajaremos en coordenadas respecto de $\cc{B}$: un polinomio $p(x) \in \bb{R}_2[x]$ con coordenadas $(a_1,a_2,a_3)$ respecto de $\cc{B}$ está en el núcleo de $f_1$ si y sólo si,
            \begin{gather*}
                \begin{pmatrix}
                    1 & 1 & -2 \\
                    0 & 0 & 1 \\
                    1 & 1 & 0 \\
                \end{pmatrix}
                \begin{pmatrix}
                    a_1 \\
                    a_2 \\
                    a_3 \\
                \end{pmatrix}
                =
                \begin{pmatrix}
                    0 \\
                    0 \\
                    0 \\
                \end{pmatrix}
            \end{gather*}
            es decir, las ecuaciones cartesianas de $ker(f_1)$ respecto a $\cc{B}$ son
            \begin{gather*}
                \left\{
                    \begin{array}{ccccc}
                        a_1 & +a_2 & -2a_3 & = & 0 \\
                         &  & a_3 & = & 0 \\
                        a_1 & +a_2 &  & = & 0 \\
                    \end{array}
                \right.
            \end{gather*}
            o equivalentemente,
            \begin{gather} \label{eq_1}
                \left\{
                    \begin{array}{ccccc}
                        a_1 & +a_2 &  & = & 0 \\
                         &  & a_3 & = & 0 \\
                    \end{array}
                \right.
            \end{gather}
            Esto determina $\ker(f_1)$ como el espacio de polinomios dentro de $\bb{R}_2[x]$ cuyas coordenadas respecto de $\cc{B}$ son las soluciones del sistema homogéneo (\ref{eq_1}), que tiene dimención 1 y está generado por el polinomio de coordenadas $(1,-1,0)$ respecto de $\cc{B}$, es decir, por $q(x)=1-x$.

            Por otro lado, $Im(f_1) = L(\{f_1(1), f_1(x), f_1(x^2)\})$, es decir, $Im(f_1)$ está generado por los polinomios de $\bb{R}_2[x]$ cuyas coordenadas respecto a $\cc{B}$ son las columnas de $M(f_1, \cc{B})$:
            \begin{gather} \label{eq_2}
                f_1(1)_{\cc{B}} = (1,0,1) = f_1(x)_{\cc{B}}, \ \ \ \  f_1(x^2)_{\cc{B}} = (-2,1,0)
            \end{gather}
            Pasando de nuevo las coordenadas a polinomio en $\bb{R}_2[x]$, tenemos
            \begin{gather} \label{eq_3}
                f_1(1) = 1+x^2 = f_1(x), \ \ \ \  f_1(x^2) = -2+x
            \end{gather} 
            con lo que $Im(f_1) = L(\{1+x^2, -2+x\})$ y el rango de $f_1$ es $2$ (en particular, $\{1+x^2, -2+x\}$ es base de $Im(f_1)$).

            \item \textbf{(1.25 puntos)} Para los valores de $k \in \bb{R}$ que sea posible, dar bases ordenadas $\cc{B}_k$, $\cc{B}_k'$ de $\bb{R}_2[x]$ tal que la matriz de $f_k$ respecto de dicho par de bases sea
            \begin{gather*}
                M(f_k, \cc{B}_k' \leftarrow \cc{B}_k ) = 
                \begin{pmatrix}
                    1 & 0 & 0 \\
                    0 & 1 & 0 \\
                    0 & 0 & 0 \\
                \end{pmatrix}
            \end{gather*}
            
            La matriz anterior no es regular (su rango es 2), así que $f_k$ no puede ser sobreyectiva. Por lo obtenido en el apartado (a), deducimos que $k=1$.
            También del apartado (a) tenemos que una base del núcleo de $f_1$ es $q(x) = 1-x$. Ampliamos esta base de $\ker(f_1)$ a una base de $\bb{R}_2[x]$, tomando $p_1(x) = 1$, $p_2(x)=x^2$ (notemos que $\{p_1(x), p_2(x), q(x)\}$ son linealmente independientes porque son polinomios de grados distintos, y por tanto son base de $\bb{R}_2[x]$) . Ordenamos dicha base definiendo $\cc{B}_1 := (p1_(x), p_2(x), q(x))$. De la expresión de la matriz que nos dan en este apartado deducimos que los dos primeros vectores de la segunda base $\cc{B}_1$ deben tomarse como 
            \begin{gather*}
                f_1(p_1) = f_1(1) \overset{(\ref{eq_3})}{=} 1 +x^2, \ \ \ \ f_1(p_2) = f_1(x^2) \overset{(\ref{eq_3})}{=} -2 + x,
            \end{gather*}
            que forman base de $Im(f_1)$ como se vio en el apartado (a). Ampliamos esta base a una de $\bb{R}_2[x]$: Usando (\ref{eq_2}) y que el determinante
            \begin{gather*}
                \begin{vmatrix}
                    1 & 0 & 1 \\
                    -2 & 1 & 0 \\
                    0 & 0 & 1 \\
                \end{vmatrix}
                = 1 \neq 0
            \end{gather*}
            concluimos que los polinomios $1+x^2$, $-2+x$, $x^2$ son base de $\bb{R}_2[x]$. Llamamos $\cc{B}_1'=(1+x^2, -2+x, x^2)$. Entonces, las coordenadas de $f_1(p_1)$ en $\cc{B}_1'$ son mat(1,0,0), las de $f_1(p_2)$ en $\cc{B}_1'$ son $(0,1,0)$, y las de $f_1(q)$ en $\cc{B}_1'$ son $(0,0,0)$. Juntando todo esto por columnas, deducimos que
            \begin{gather*}
                M(f_k, \cc{B}) = 
                \begin{pmatrix}
                    1 & 1 & -2 \\
                    0 & k-1 & 1 \\
                    1 & k^2 & 0 \\
                \end{pmatrix}
            \end{gather*}
        \end{enumerate}
    \end{ejercicio}

    \newpage

    \begin{ejercicio}[2.5 puntos] \ 
        \begin{enumerate}
            \item \textbf{(1.25 puntos)} Determinar un endomorfismo $f$ de $\bb{R}^4$ que cumpla las condiciones
            \begin{gather*}
                Im(f^t) = an(L(\{(1,0,-1,0),(0,2,1,1)\})), \ \ \ \ \ker(f) \oplus Im(f) = \bb{R}^4
            \end{gather*}

            Como $Im(f^t)=an(\ker(f))$, la primera condición anterior equivale a imponer $an(\ker(f)) = an(L(\{(1,0,-1,0),(0,2,1,1)\}))$, o lo que es lo mismo,
            \begin{gather*}
                L(\{(1,0,-1,0),(0,2,1,1)\}) = \ker(f)
            \end{gather*}
            Sean $x_1 = (1,0,-1,0)$, $x_2 = (0,2,1,1) \in \bb{R}^4$. Es claro que $x_2, x_2$ son linealmente independientes (por ejemplo, porque el menor de orden $2$ formado por las dos primeras coordenadas de ambos vectores es $2 \neq 0$). Ampliamos $\{x_1, x_2\}$ a una base de $\bb{R}^4$ con los vectores $x_3 = (0,0,1,0)$, $x_4 = (0,0,0,1)$ ($\{x_2, x_2, x_3, x_4\}$ forman base de $\bb{R}^4$ porque el determinante de la matriz que forman es $2 \neq 0$). Planteamos ahora el cuadro
            \begin{gather*}
                \begin{array}{ccc}
                    \bb{R}^4 & \rightarrow & \bb{R}^4\\
                    x_1 & \mapsto & 0 \\
                    x_2 & \mapsto & 0 \\
                    x_3 & \mapsto & x_3 \\
                    x_4 & \mapsto & x_4 \\
                \end{array}
            \end{gather*}
            Por el teorema fundamental de las aplicaciones lineales, exite un único endomorfismo $f$ de $\bb{R}^4$ que cumple el cuadro anterior. Así, la imagen de $f$ está generada por $x_3, x_4$ (luego $rango(f) = 2$, $\{x_3,x-4\}$ es base de $Im(f)$ y $nulidad(f) = 4 -2 = 2$), y el núcleo de $f$ tiene por base a $\{x_1, x_2\}$. Además,
            \begin{gather*}
                \bb{R}^4 = L(\{x_1, x_2, x_3, x_4\}) = L(\{x_1, x_2\}) \oplus L(\{x_3, x_4\}) = \ker(f) \oplus Im(f)
            \end{gather*}
            luego $f$ cumple las condiciones pedidas (no es única con estas condiciones).

            \item \textbf{(1.25 puntos)} Calcular la matriz de $f^t$ en la base dual de la base canónica de $\bb{R}^4$.
            
            Si $\cc{B}_u$ es la base ordenada usual de $\bb{R}^4$, la matriz de $f$ respecto de las bases ordenadas $\cc{B} = (x_1, x_2, x_3, x_4)$ y $\cc{B}_u$ es 
            \begin{gather} \label{eq_4}
                M(f, \cc{B}_u \leftarrow \cc{B}) = 
                \begin{pmatrix}
                    0 & 0 & 0 & 0 \\
                    0 & 0 & 0 & 0 \\
                    0 & 0 & 1 & 0 \\
                    0 & 0 & 0 & 1 \\
                \end{pmatrix}
            \end{gather}

            Por tanto,
            \begin{align*}
                M(f, \cc{B}_u) &= M(f, \cc{B}_u \leftarrow \cc{B}) \cdot M(1_{\bb{R}^4}, \cc{B} \leftarrow \cc{B}_u) \\
                &= M(f, \cc{B}_u \leftarrow \cc{B}) \cdot M(1_{\bb{R}^4}, \cc{B}_u \leftarrow \cc{B})^{-1} \\
                &= 
                \begin{pmatrix}
                    0 & 0 & 0 & 0 \\
                    0 & 0 & 0 & 0 \\
                    0 & 0 & 1 & 0 \\
                    0 & 0 & 0 & 1 \\
                \end{pmatrix}
                \cdot
                \begin{pmatrix}
                    1 & 0 & 0 & 0 \\
                    0 & 2 & 0 & 0 \\
                    -1 & 1 & 1 & 0 \\
                    0 & 1 & 0 & 1 \\
                \end{pmatrix}^{-1}\\
                &= 
                \begin{pmatrix}
                    0 & 0 & 0 & 0 \\
                    0 & 0 & 0 & 0 \\
                    0 & 0 & 1 & 0 \\
                    0 & 0 & 0 & 1 \\
                \end{pmatrix}
                \cdot
                \begin{pmatrix}
                    1 & 0 & 0 & 0 \\
                    0 & 1/2 & 0 & 0 \\
                    1 & -1/2 & 1 & 0 \\
                    0 & -1/2 & 0 & 1 \\
                \end{pmatrix}\\
                &= 
                \begin{pmatrix} \tag{5} \label{eq_5}
                    0 & 0 & 0 & 0 \\
                    0 & 0 & 0 & 0 \\
                    1 & -1/2 & 1 & 0 \\
                    0 & -1/2 & 0 & 1 \\
                \end{pmatrix}
            \end{align*}
            Finalmente, la matriz que nos piden es
            \begin{gather*}
                M(f^t, \cc{B}_u^*) = M(f, \cc{B}_u)^t =
                \begin{pmatrix}
                    0 & 0 & 0 & 0 \\
                    0 & 0 & 0 & 0 \\
                    1 & -1/2 & 1 & 0 \\
                    0 & -1/2 & 0 & 1 \\
                \end{pmatrix}^t
                =
                \begin{pmatrix} 
                    0 & 0 & 1 & 0 \\
                    0 & 0 & -1/2 & -1/2 \\
                    0 & 0 & 1 & 0 \\
                    0 & 0 & 0 & 1 \\
                \end{pmatrix}
            \end{gather*}
            También se puede obtener $M(f, \cc{B}_u)$ calculando directamente lo que valen las imágenes de los vectores de la base usual de $\bb{R}^4$. Ya sabemos que $f(0,0,1,0) = (0,0,1,0)$ y que $f(0,0,0,1) = (0,0,0,1)$. Por otro lado,
            \begin{align*}
                f(1,0,0,0) &= f(1,0,-1,0) + f(0,0,1,0) = (0,0,1,0)\\
                f(0,1,0,0) &= \frac{1}{2} f(0,2,1,1) - \frac{1}{2}(0,0,1,0) - \frac{1}{2}f(0,0,0,1) = (0,0,\nicefrac{-1}{2}, \nicefrac{-1}{2})
            \end{align*} 
            Poniendo estos valores por columnas en el orden adecuado obtenemos la matriz (\ref{eq_5}).
        \end{enumerate}
    \end{ejercicio}


\end{document}