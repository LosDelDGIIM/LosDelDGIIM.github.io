\documentclass[12pt]{article}

% Idioma y codificación
\usepackage[spanish, es-tabla]{babel}       %es-tabla para que se titule "Tabla"
\usepackage[utf8]{inputenc}

% Márgenes
\usepackage[a4paper,top=3cm,bottom=2.5cm,left=3cm,right=3cm]{geometry}

% Comentarios de bloque
\usepackage{verbatim}

% Paquetes de links
\usepackage[hidelinks]{hyperref}    % Permite enlaces
\usepackage{url}                    % redirecciona a la web

% Más opciones para enumeraciones
\usepackage{enumitem}

% Personalizar la portada
\usepackage{titling}

% Paquetes de tablas
\usepackage{multirow}


%------------------------------------------------------------------------

%Paquetes de figuras
\usepackage{caption}
\usepackage{subcaption} % Figuras al lado de otras
\usepackage{float}      % Poner figuras en el sitio indicado H.


% Paquetes de imágenes
\usepackage{graphicx}       % Paquete para añadir imágenes
\usepackage{transparent}    % Para manejar la opacidad de las figuras

% Paquete para usar colores
\usepackage[dvipsnames]{xcolor}
\usepackage{pagecolor}      % Para cambiar el color de la página

% Habilita tamaños de fuente mayores
\usepackage{fix-cm}

% Para los gráficos
\usepackage{tikz}

% Para poder situar los nodos en los grafos
\usetikzlibrary{positioning}


%------------------------------------------------------------------------

% Paquetes de matemáticas
\usepackage{mathtools, amsfonts, amssymb, mathrsfs}
\usepackage[makeroom]{cancel}     % Simplificar tachando
\usepackage{polynom}    % Divisiones y Ruffini
\usepackage{units} % Para poner fracciones diagonales con \nicefrac

\usepackage{pgfplots}   %Representar funciones
\pgfplotsset{compat=1.18}  % Versión 1.18

\usepackage{tikz-cd}    % Para usar diagramas de composiciones
\usetikzlibrary{calc}   % Para usar cálculo de coordenadas en tikz

%Definición de teoremas, etc.
\usepackage{amsthm}
%\swapnumbers   % Intercambia la posición del texto y de la numeración

\theoremstyle{plain}

\makeatletter
\@ifclassloaded{article}{
  \newtheorem{teo}{Teorema}[section]
}{
  \newtheorem{teo}{Teorema}[chapter]  % Se resetea en cada chapter
}
\makeatother

\newtheorem{coro}{Corolario}[teo]           % Se resetea en cada teorema
\newtheorem{prop}[teo]{Proposición}         % Usa el mismo contador que teorema
\newtheorem{lema}[teo]{Lema}                % Usa el mismo contador que teorema

\theoremstyle{remark}
\newtheorem*{observacion}{Observación}

\theoremstyle{definition}

\makeatletter
\@ifclassloaded{article}{
  \newtheorem{definicion}{Definición} [section]     % Se resetea en cada chapter
}{
  \newtheorem{definicion}{Definición} [chapter]     % Se resetea en cada chapter
}
\makeatother

\newtheorem*{notacion}{Notación}
\newtheorem*{ejemplo}{Ejemplo}
\newtheorem*{ejercicio*}{Ejercicio}             % No numerado
\newtheorem{ejercicio}{Ejercicio} [section]     % Se resetea en cada section


% Modificar el formato de la numeración del teorema "ejercicio"
\renewcommand{\theejercicio}{%
  \ifnum\value{section}=0 % Si no se ha iniciado ninguna sección
    \arabic{ejercicio}% Solo mostrar el número de ejercicio
  \else
    \thesection.\arabic{ejercicio}% Mostrar número de sección y número de ejercicio
  \fi
}


% \renewcommand\qedsymbol{$\blacksquare$}         % Cambiar símbolo QED
%------------------------------------------------------------------------

% Paquetes para encabezados
\usepackage{fancyhdr}
\pagestyle{fancy}
\fancyhf{}

\newcommand{\helv}{ % Modificación tamaño de letra
\fontfamily{}\fontsize{12}{12}\selectfont}
\setlength{\headheight}{15pt} % Amplía el tamaño del índice


%\usepackage{lastpage}   % Referenciar última pag   \pageref{LastPage}
\fancyfoot[C]{\thepage}

%------------------------------------------------------------------------

% Conseguir que no ponga "Capítulo 1". Sino solo "1."
\makeatletter
\@ifclassloaded{book}{
  \renewcommand{\chaptermark}[1]{\markboth{\thechapter.\ #1}{}} % En el encabezado
    
  \renewcommand{\@makechapterhead}[1]{%
  \vspace*{50\p@}%
  {\parindent \z@ \raggedright \normalfont
    \ifnum \c@secnumdepth >\m@ne
      \huge\bfseries \thechapter.\hspace{1em}\ignorespaces
    \fi
    \interlinepenalty\@M
    \Huge \bfseries #1\par\nobreak
    \vskip 40\p@
  }}
}
\makeatother

%------------------------------------------------------------------------
% Paquetes de cógido
\usepackage{minted}
\renewcommand\listingscaption{Código fuente}

\usepackage{fancyvrb}
% Personaliza el tamaño de los números de línea
\renewcommand{\theFancyVerbLine}{\small\arabic{FancyVerbLine}}

% Estilo para C++
\newminted{cpp}{
    frame=lines,
    framesep=2mm,
    baselinestretch=1.2,
    linenos,
    escapeinside=||
}

% para minted
\definecolor{LightGray}{rgb}{0.95,0.95,0.92}
\setminted{
    linenos=true,
    stepnumber=5,
    numberfirstline=true,
    autogobble,
    breaklines=true,
    breakautoindent=true,
    breaksymbolleft=,
    breaksymbolright=,
    breaksymbolindentleft=0pt,
    breaksymbolindentright=0pt,
    breaksymbolsepleft=0pt,
    breaksymbolsepright=0pt,
    fontsize=\footnotesize,
    bgcolor=LightGray,
    numbersep=10pt
}


\usepackage{listings} % Para incluir código desde un archivo

\renewcommand\lstlistingname{Código Fuente}
\renewcommand\lstlistlistingname{Índice de Códigos Fuente}

% Definir colores
\definecolor{vscodepurple}{rgb}{0.5,0,0.5}
\definecolor{vscodeblue}{rgb}{0,0,0.8}
\definecolor{vscodegreen}{rgb}{0,0.5,0}
\definecolor{vscodegray}{rgb}{0.5,0.5,0.5}
\definecolor{vscodebackground}{rgb}{0.97,0.97,0.97}
\definecolor{vscodelightgray}{rgb}{0.9,0.9,0.9}

% Configuración para el estilo de C similar a VSCode
\lstdefinestyle{vscode_C}{
  backgroundcolor=\color{vscodebackground},
  commentstyle=\color{vscodegreen},
  keywordstyle=\color{vscodeblue},
  numberstyle=\tiny\color{vscodegray},
  stringstyle=\color{vscodepurple},
  basicstyle=\scriptsize\ttfamily,
  breakatwhitespace=false,
  breaklines=true,
  captionpos=b,
  keepspaces=true,
  numbers=left,
  numbersep=5pt,
  showspaces=false,
  showstringspaces=false,
  showtabs=false,
  tabsize=2,
  frame=tb,
  framerule=0pt,
  aboveskip=10pt,
  belowskip=10pt,
  xleftmargin=10pt,
  xrightmargin=10pt,
  framexleftmargin=10pt,
  framexrightmargin=10pt,
  framesep=0pt,
  rulecolor=\color{vscodelightgray},
  backgroundcolor=\color{vscodebackground},
}

%------------------------------------------------------------------------

% Comandos definidos
\newcommand{\bb}[1]{\mathbb{#1}}
\newcommand{\cc}[1]{\mathcal{#1}}

% I prefer the slanted \leq
\let\oldleq\leq % save them in case they're every wanted
\let\oldgeq\geq
\renewcommand{\leq}{\leqslant}
\renewcommand{\geq}{\geqslant}

% Si y solo si
\newcommand{\sii}{\iff}

% Letras griegas
\newcommand{\eps}{\epsilon}
\newcommand{\veps}{\varepsilon}
\newcommand{\lm}{\lambda}

\newcommand{\ol}{\overline}
\newcommand{\ul}{\underline}
\newcommand{\wt}{\widetilde}
\newcommand{\wh}{\widehat}

\let\oldvec\vec
\renewcommand{\vec}{\overrightarrow}

% Derivadas parciales
\newcommand{\del}[2]{\frac{\partial #1}{\partial #2}}
\newcommand{\Del}[3]{\frac{\partial^{#1} #2}{\partial #3^{#1}}}
\newcommand{\deld}[2]{\dfrac{\partial #1}{\partial #2}}
\newcommand{\Deld}[3]{\dfrac{\partial^{#1} #2}{\partial #3^{#1}}}


\newcommand{\AstIg}{\stackrel{(\ast)}{=}}
\newcommand{\Hop}{\stackrel{L'H\hat{o}pital}{=}}

\newcommand{\red}[1]{{\color{red}#1}} % Para integrales, destacar los cambios.

% Método de integración
\newcommand{\MetInt}[2]{
    \left[\begin{array}{c}
        #1 \\ #2
    \end{array}\right]
}

% Declarar aplicaciones
% 1. Nombre aplicación
% 2. Dominio
% 3. Codominio
% 4. Variable
% 5. Imagen de la variable
\newcommand{\Func}[5]{
    \begin{equation*}
        \begin{array}{rrll}
            #1:& #2 & \longrightarrow & #3\\
               & #4 & \longmapsto & #5
        \end{array}
    \end{equation*}
}

%------------------------------------------------------------------------


\begin{document}

    % 1. Foto de fondo
    % 2. Título
    % 3. Encabezado Izquierdo
    % 4. Color de fondo
    % 5. Coord x del titulo
    % 6. Coord y del titulo
    % 7. Fecha

    
    % 1. Foto de fondo
% 2. Título
% 3. Encabezado Izquierdo
% 4. Color de fondo
% 5. Coord x del titulo
% 6. Coord y del titulo
% 7. Fecha

\newcommand{\portada}[7]{

    \portadaBase{#1}{#2}{#3}{#4}{#5}{#6}{#7}
    \portadaBook{#1}{#2}{#3}{#4}{#5}{#6}{#7}
}

\newcommand{\portadaExamen}[7]{

    \portadaBase{#1}{#2}{#3}{#4}{#5}{#6}{#7}
    \portadaArticle{#1}{#2}{#3}{#4}{#5}{#6}{#7}
}




\newcommand{\portadaBase}[7]{

    % Tiene la portada principal y la licencia Creative Commons
    
    % 1. Foto de fondo
    % 2. Título
    % 3. Encabezado Izquierdo
    % 4. Color de fondo
    % 5. Coord x del titulo
    % 6. Coord y del titulo
    % 7. Fecha
    
    
    \thispagestyle{empty}               % Sin encabezado ni pie de página
    \newgeometry{margin=0cm}        % Márgenes nulos para la primera página
    
    
    % Encabezado
    \fancyhead[L]{\helv #3}
    \fancyhead[R]{\helv \nouppercase{\leftmark}}
    
    
    \pagecolor{#4}        % Color de fondo para la portada
    
    \begin{figure}[p]
        \centering
        \transparent{0.3}           % Opacidad del 30% para la imagen
        
        \includegraphics[width=\paperwidth, keepaspectratio]{assets/#1}
    
        \begin{tikzpicture}[remember picture, overlay]
            \node[anchor=north west, text=white, opacity=1, font=\fontsize{60}{90}\selectfont\bfseries\sffamily, align=left] at (#5, #6) {#2};
            
            \node[anchor=south east, text=white, opacity=1, font=\fontsize{12}{18}\selectfont\sffamily, align=right] at (9.7, 3) {\textbf{\href{https://losdeldgiim.github.io/}{Los Del DGIIM}}};
            
            \node[anchor=south east, text=white, opacity=1, font=\fontsize{12}{15}\selectfont\sffamily, align=right] at (9.7, 1.8) {Doble Grado en Ingeniería Informática y Matemáticas\\Universidad de Granada};
        \end{tikzpicture}
    \end{figure}
    
    
    \restoregeometry        % Restaurar márgenes normales para las páginas subsiguientes
    \pagecolor{white}       % Restaurar el color de página
    
    
    \newpage
    \thispagestyle{empty}               % Sin encabezado ni pie de página
    \begin{tikzpicture}[remember picture, overlay]
        \node[anchor=south west, inner sep=3cm] at (current page.south west) {
            \begin{minipage}{0.5\paperwidth}
                \href{https://creativecommons.org/licenses/by-nc-nd/4.0/}{
                    \includegraphics[height=2cm]{assets/Licencia.png}
                }\vspace{1cm}\\
                Esta obra está bajo una
                \href{https://creativecommons.org/licenses/by-nc-nd/4.0/}{
                    Licencia Creative Commons Atribución-NoComercial-SinDerivadas 4.0 Internacional (CC BY-NC-ND 4.0).
                }\\
    
                Eres libre de compartir y redistribuir el contenido de esta obra en cualquier medio o formato, siempre y cuando des el crédito adecuado a los autores originales y no persigas fines comerciales. 
            \end{minipage}
        };
    \end{tikzpicture}
    
    
    
    % 1. Foto de fondo
    % 2. Título
    % 3. Encabezado Izquierdo
    % 4. Color de fondo
    % 5. Coord x del titulo
    % 6. Coord y del titulo
    % 7. Fecha


}


\newcommand{\portadaBook}[7]{

    % 1. Foto de fondo
    % 2. Título
    % 3. Encabezado Izquierdo
    % 4. Color de fondo
    % 5. Coord x del titulo
    % 6. Coord y del titulo
    % 7. Fecha

    % Personaliza el formato del título
    \pretitle{\begin{center}\bfseries\fontsize{42}{56}\selectfont}
    \posttitle{\par\end{center}\vspace{2em}}
    
    % Personaliza el formato del autor
    \preauthor{\begin{center}\Large}
    \postauthor{\par\end{center}\vfill}
    
    % Personaliza el formato de la fecha
    \predate{\begin{center}\huge}
    \postdate{\par\end{center}\vspace{2em}}
    
    \title{#2}
    \author{\href{https://losdeldgiim.github.io/}{Los Del DGIIM}}
    \date{Granada, #7}
    \maketitle
    
    \tableofcontents
}




\newcommand{\portadaArticle}[7]{

    % 1. Foto de fondo
    % 2. Título
    % 3. Encabezado Izquierdo
    % 4. Color de fondo
    % 5. Coord x del titulo
    % 6. Coord y del titulo
    % 7. Fecha

    % Personaliza el formato del título
    \pretitle{\begin{center}\bfseries\fontsize{42}{56}\selectfont}
    \posttitle{\par\end{center}\vspace{2em}}
    
    % Personaliza el formato del autor
    \preauthor{\begin{center}\Large}
    \postauthor{\par\end{center}\vspace{3em}}
    
    % Personaliza el formato de la fecha
    \predate{\begin{center}\huge}
    \postdate{\par\end{center}\vspace{5em}}
    
    \title{#2}
    \author{\href{https://losdeldgiim.github.io/}{Los Del DGIIM}}
    \date{Granada, #7}
    \thispagestyle{empty}               % Sin encabezado ni pie de página
    \maketitle
    \vfill
}
    \portadaExamen{ffccA4.jpg}{Geometría I\\Examen VI}{Geometría I. Examen VI}{MidnightBlue}{-8}{28}{2023}{Arturo Olivares Martos}

    \begin{description}
        \item[Asignatura] Geometría I.
        \item[Curso Académico] 2021-22.
        \item[Grado] Doble Grado en Ingeniería Informática y Matemáticas.
        \item[Grupo] Único.
        \item[Profesor] Juan de Dios Pérez Jiménez\footnote{El examen lo pone el departamento.}.
        \item[Descripción] Convocatoria Ordinaria.
        \item[Fecha] 20 de enero de 2022.
        %\item[Duración] 3 horas.
    
    \end{description}
    \newpage

\begin{ejercicio}\textbf{[2 puntos]} Sea $U_1,\dots,U_n$ una familia finita de subespacios vectoriales de un espacio vectorial $V$. Demostrar que:
\begin{equation*}
    \sum_{i=1}^n U_i = \{u_1 + \dots + u_n \mid u_i\in U_i,\;\forall\ i=1,\dots,n\}
\end{equation*}
\begin{observacion}
    Por definición: $\sum_{i=1}^n U_i = \cc{L}\left(\cup_{i=1}^n U_i\right)$.
\end{observacion}
\end{ejercicio}


\begin{ejercicio}\textbf{[2 puntos]}
    Sea $V$ y $V'$ espacios vectoriales de dimensión finita y $\Phi:V\to V^{\ast\ast},\; \Phi':V'\to (V')^{\ast\ast}$ los correspondientes isomorfismos del Teorema de reflexividad. Demostrar que si $f:V\to V'$ es una aplicación lineal, entonces $\Phi'\circ f = (f^t)^t \circ \Phi$.

    \begin{proof}
        Describimos en primer lugar las aplicaciones mencionadas:
        \begin{equation*}\begin{array}{llll}
            f:V\to V' &&&\\
            \Phi:V\to V^{\ast\ast} & \Phi(v)=\Phi_v & \forall \phi\in V^\ast,\quad &\Phi_v(\phi)=\phi(v) \\
            \Phi':V'\to (V')^{\ast\ast} & \Phi'(v')=\Phi'_{v'} & \forall \phi'\in (V')^\ast,\quad &\Phi'_{v'}(\phi')=\phi'(v') \\
            f^t:(V')^\ast \to V^\ast & f^t(\phi')=\phi'\circ f &&\\
            (f^t)^t:V^{\ast\ast} \to (V')^{\ast\ast} & (f^t)^t(\Phi_v)=\Phi_v\circ f^t &&\\
        \end{array}\end{equation*}
    
        Veamos ahora la igualdad pedida:
        \begin{gather*}
            V\xrightarrow{\quad f \quad} V' \xrightarrow{\quad \Phi' \quad} (V')^{\ast\ast}\\
            V\xrightarrow{\quad \Phi \quad} V^{\ast\ast} \xrightarrow{\quad (f^t)^t \quad} (V')^{\ast\ast}
        \end{gather*}
    
        Por tanto, vemos que ambas composiciones tienen el mismo dominio y codominio. Veamos ahora si, $\forall v\in V$, se cumple que $(\Phi'\circ f)(v) = ((f^t)^t \circ \Phi)(v)$.
        \begin{equation*}
            (\Phi'\circ f)(v) = \Phi'[f(v)] = \Phi'_{f(v)} \quad  \in (V')^{\ast\ast}
        \end{equation*}
        \begin{equation*}
            ((f^t)^t \circ \Phi)(v) = (f^t)^t(\Phi(v)) = (f^t)^t(\Phi_v) = \Phi_v\circ f^t \quad \in (V')^{\ast\ast}
        \end{equation*}
    
        Como ambos pertenecen a $(V')^{\ast\ast}$, les aplicamos $\phi'\in (V')^\ast$.
        \begin{equation*}
            \Phi'_{f(v)}(\phi') = \phi'(f(v)) = (\phi' \circ f)(v) \quad \in \bb{K}
        \end{equation*}
        \begin{equation*}
            (\Phi_v\circ f^t)(\phi') = \Phi_v(f^t(\phi')) = \Phi_v(\phi'\circ f) = (\phi'\circ f)(v) \quad \in \bb{K}
        \end{equation*}
    
        Por tanto, tenemos que se verifica lo pedido.
        \end{proof}
\end{ejercicio}

\begin{ejercicio}\textbf{[3 puntos]}
    En el espacio vectorial de las matrices reales antisimétricas de orden 3, $\cc{A}_3(\bb{R})$, se considera $U=\{A\in \cc{A}_3(\bb{R})\mid tr(AM)=0\}$, siendo
    \begin{equation*}
        M=\left(\begin{array}{ccc}
            1 & -1 & 1 \\
            1 & -1 & 1 \\
            1 & -1 & 1
        \end{array}\right)
    \end{equation*}
    \begin{enumerate}
        \item Calcular un complementario $W$ de $U$ en $\cc{A}_3(\bb{R})$.
        
        Calculo en primer lugar una base $\cc{B}$ de $\cc{A}_3(\bb{R})$.
        \begin{equation*}\begin{split}
            \forall A\in \cc{A}_3(\bb{R}),\quad A&=-A^t \Longrightarrow
            \left(\begin{array}{ccc}
                a_{11} & a_{12} & a_{13} \\
                a_{21} & a_{22} & a_{23} \\
                a_{31} & a_{32} & a_{33} \\
            \end{array}\right)
            = \left(\begin{array}{ccc}
                -a_{11} & -a_{21} & -a_{31} \\
                -a_{12} & -a_{22} & -a_{32} \\
                -a_{13} & -a_{23} & -a_{33} \\
            \end{array}\right)
            \Longrightarrow \\ \Longrightarrow
            A &= \left(\begin{array}{ccc}
                0 & a_{12} & a_{13} \\
                -a_{12} & 0 & a_{23} \\
                -a_{13} & -a_{23} & 0 \\
            \end{array}\right)
            =\\&= a_{12} \left(\begin{array}{ccc}
                0 & 1 & 0 \\
                -1 & 0 & 0 \\
                0 & 0 & 0 \\
            \end{array}\right)
            + a_{13}\left(\begin{array}{ccc}
                0 & 0 & 1 \\
                0 & 0 & 0 \\
                -1 & 0 & 0 \\
            \end{array}\right)
            + a_{23}\left(\begin{array}{ccc}
                0 & 0 & 0 \\
                0 & 0 & 1 \\
                0 & -1 & 0 \\
            \end{array}\right)
        \end{split}\end{equation*}

        Por tanto, tenemos que una base de $\cc{A}_3(\bb{R})$ es:
        \begin{equation*}
            \cc{B}=\left\{\left(\begin{array}{ccc}
                0 & 1 & 0 \\
                -1 & 0 & 0 \\
                0 & 0 & 0 \\
            \end{array}\right),
            \left(\begin{array}{ccc}
                0 & 0 & 1 \\
                0 & 0 & 0 \\
                -1 & 0 & 0 \\
            \end{array}\right),
            \left(\begin{array}{ccc}
                0 & 0 & 0 \\
                0 & 0 & 1 \\
                0 & -1 & 0 \\
            \end{array}\right)\right\}
        \end{equation*}


        Calculo ahora un base de $U$.
        \begin{equation}\label{ec:ParamU}\begin{split}
            U&= \left\{A\in \cc{A}_3(\bb{R}) \left|\;tr\left(AM\right)=0\right.\right\} =\\
            &= \left\{A\in \cc{A}_3(\bb{R}) \left|\;tr\left[
                \left(\begin{array}{ccc}
                    0 & a_{12} & a_{13} \\
                    -a_{12} & 0 & a_{23} \\
                    -a_{13} & -a_{23} & 0 \\
                \end{array}\right)
                \left(\begin{array}{ccc}
                    1 & -1 & 1 \\
                    1 & -1 & 1 \\
                    1 & -1 & 1
                \end{array}\right)
            \right]=0\right.\right\} =\\
            &= \left\{A\in \cc{A}_3(\bb{R}) \left|\;
            a_{12} + \cancel{a_{13}} + a_{12} - a_{23} - \cancel{a_{13}} - a_{23} = 0
            \right.\right\} =\\
            &= \left\{A\in \cc{A}_3(\bb{R}) \left|\;
            2a_{12} - 2a_{23} = 0
            \right.\right\} = \left\{A\in \cc{A}_3(\bb{R}) \left|\;
            a_{12} = a_{23}
            \right.\right\}
        \end{split}\end{equation}

        Por tanto, tenemos que la base $\cc{B}_U$ de $U$ es:
        \begin{equation*}
            \cc{B}_U=\left\{
            \left(\begin{array}{ccc}
                0 & 0 & 1 \\
                0 & 0 & 0 \\
                -1 & 0 & 0 \\
            \end{array}\right),
            \left(\begin{array}{ccc}
                0 & 1 & 0 \\
                -1 & 0 & 1 \\
                0 & -1 & 0 \\
            \end{array}\right)\right\}
        \end{equation*}

        Por tanto, para obtener un complementario de $U$ es necesario ampliar la base $\cc{B}_U$ a una base de $\cc{A}_3(\bb{R})$. Sea dicha base ampliada $\bar{\cc{B}}$.
        \begin{equation*}
            \bar{\cc{B}}=\left\{
            \left(\begin{array}{ccc}
                0 & 0 & 1 \\
                0 & 0 & 0 \\
                -1 & 0 & 0 \\
            \end{array}\right),
            \left(\begin{array}{ccc}
                0 & 1 & 0 \\
                -1 & 0 & 1 \\
                0 & -1 & 0 \\
            \end{array}\right),
            \left(\begin{array}{ccc}
                0 & 1 & 0 \\
                -1 & 0 & 0 \\
                0 & 0 & 0 \\
            \end{array}\right)\right\}
        \end{equation*}

        Por tanto, tenemos que el complementario de $U$, $W$, es:
        \begin{equation*}
          W=\cc{L}\left(\left\{\left(\begin{array}{ccc}
                0 & 1 & 0 \\
                -1 & 0 & 0 \\
                0 & 0 & 0 \\
            \end{array}\right)\right\}\right)  
        \end{equation*}

        Es fácil ver que $\cc{A}_3(\bb{R}) = U\oplus W$.
        

        \item Calcular un complementario de
        \begin{equation*}
            T=\cc{L}\left(\left\{
                \left(\begin{array}{ccc}
                    0 & 0 & 0 \\
                    0 & 0 & 0 \\
                    0 & -1 & 0 \\
                \end{array}\right) + U
            \right\}\right)
        \end{equation*}
        en $\cc{A}_3(\bb{R})/U$.

        Calculo en primer lugar la dimensión del espacio cociente:
        \begin{equation*}
            \dim_{\bb{R}}\left(\cc{A}_3(\bb{R})/U\right) = 
            \dim_{\bb{R}}\left(\cc{A}_3(\bb{R})\right) - \dim_{\bb{R}}\left(U\right) = 3-2 = 1
        \end{equation*}

        Como la suma de las dimensiones de un subespacio y las de su complementario dan la dimensión del espacio vectorial, tenemos que la dimensión del subespacio complementario es nula, ya que:
        \begin{equation*}
            1 = \dim_{\bb{R}}\left(\cc{A}_3(\bb{R})/U\right) = dim_{\bb{R}}\left(T\right) + 0 = 1+0=1
        \end{equation*}

        Por tanto, tenemos que es complementario de $T$ es $\{0\}$.

        \item Sea $f:\cc{A}_3(\bb{R})\to \bb{R}$ dada por $f(A)=tr(AM)$. Comprobar que $f\in \left(\cc{A}_3(\bb{R})\right)^\ast$ y calcular una base de $\cc{A}_3(\bb{R})^\ast$ que contenga a $f$.

        Supuesto que $f$ es lineal, al ser una forma lineal tendríamos directamente el resultado buscado. Por tanto, comprobamos que $f$ es lineal.

        Sean $A,B\in \cc{A}_3(\bb{R})$,\qquad $a,b\in \bb{R}$.
        \begin{equation*}\begin{split}
            f(aA+bB) &= tr[(aA+bB)M] = tr(aA+bB)\cdot tr(M) = [atr(A) + btr(B)]tr(M) =\\
            &= atr(AM) + btr(BM) = af(A) + bf(B)
        \end{split}\end{equation*}

        Por tanto, tenemos que $f$ es lineal y, por tanto, $f\in (\cc{A}_3(\bb{R}))^\ast$. Calculamos ahora una base $\cc{B}^\ast$ de $(\cc{A}_3(\bb{R}))^\ast$ tal que $f\in \cc{B}^\ast$.

        Sea $\cc{B}_0$ la base usual de $\cc{A}_3(\bb{R})$, es decir.
        \begin{equation*}
            \cc{B}_0=\left\{\left(\begin{array}{ccc}
                0 & 1 & 0 \\
                -1 & 0 & 0 \\
                0 & 0 & 0 \\
            \end{array}\right),
            \left(\begin{array}{ccc}
                0 & 0 & 1 \\
                0 & 0 & 0 \\
                -1 & 0 & 0 \\
            \end{array}\right),
            \left(\begin{array}{ccc}
                0 & 0 & 0 \\
                0 & 0 & 1 \\
                0 & -1 & 0 \\
            \end{array}\right)\right\}
        \end{equation*}

        Sea $\cc{B}_0^\ast=\{\varphi_1, \varphi_2, \varphi_3\}$ su base dual tal que:
        \begin{equation*}
            \varphi_i \left(\begin{array}{ccc}
                0 & a_1 & a_2 \\
                -a_1 & 0 & a_3 \\
                -a_2 & -a_3 & 0 \\
            \end{array}\right) = a_i \hspace{2cm} \forall i=1,2,3
        \end{equation*}

        Como $f\in (\cc{A}_3(\bb{R}))^\ast$, calculamos sus coordenadas en la base especificada.
        \begin{equation*}
            f\left(\begin{array}{ccc}
                0 & a_1 & a_2 \\
                -a_1 & 0 & a_3 \\
                -a_2 & -a_3 & 0 \\
            \end{array}\right)
            =tr\left[
                \left(\begin{array}{ccc}
                    0 & a_1 & a_2 \\
                    -a_1 & 0 & a_3 \\
                    -a_2 & -a_3 & 0 \\
                \end{array}\right)
                \left(\begin{array}{ccc}
                    1 & -1 & 1 \\
                    1 & -1 & 1 \\
                    1 & -1 & 1
                \end{array}\right)
            \right] \stackrel{Ec.\;\ref{ec:ParamU}}{=} 2a_1 - 2a_3
        \end{equation*}
        Por tanto, tenemos que $f=2\varphi_2 -2\varphi_3 \Longrightarrow f=(2,-2,0)_{\cc{B}_0^\ast}$.

        Veamos si $\varphi_2, \varphi_3$ son linealmente independientes a $f$ y, por tanto, forman base:
        \begin{equation*}
            \left|\begin{array}{ccc}
                2 & 0 & 0\\
                -2 & 1 & 0 \\
                0 & 0 & 1
            \end{array}\right| = 2 \neq 0 \Longrightarrow \cc{B}^\ast = \{f,\varphi_2, \varphi_3\} \text{ base de } \left(\cc{A}_3(\bb{R})\right)^\ast
        \end{equation*}
        
    \end{enumerate}
\end{ejercicio}



\begin{ejercicio}\textbf{[3 puntos]}
    Se considera la aplicación lineal $f:\bb{R}^3 \to \bb{R}^3$ cuya matriz en la base usual verifica:
    \begin{equation*}
        M(f,\cc{B}_u)=\left(\begin{array}{ccc}
            3 & 2 & 0 \\
            -3 & -2 & 0 \\
            1 & 1 & 1 \\
        \end{array}\right)
    \end{equation*}

    \begin{enumerate}
        \item Calcular, en caso de que existan, bases $\cc{B}$ y $\cc{B}'$ de $\bb{R}^3$ tales que la matriz de $f$ en esas bases sea:
        \begin{equation*}
            M(f, \cc{B}\to \cc{B}')=
                \left(\begin{array}{ccc}
                1 & 0 & 0 \\
                0 & 1 & 0 \\
                0 & 0 & 0 \\
            \end{array}\right)
        \end{equation*}

        Calculo en primer lugar $Ker(f)$:
        \begin{equation*}\begin{split}
            Ker(f)&= \left\{(x,y,z)\in \bb{R}^3 \left|
            \left(\begin{array}{ccc}
                3 & 2 & 0 \\
                -3 & -2 & 0 \\
                1 & 1 & 1 \\
            \end{array}\right)
            \left(\begin{array}{c}
                x \\ y \\ z
            \end{array}\right) = 0
            \right.\right\} =\\
            &= \left\{(x,y,z)\in \bb{R}^3 \left|
            \begin{array}{c}
                3x+2y=0\\
                x+y+z=0
            \end{array}
            \right.\right\}
            = \cc{L}\left(\left\{(-2,3,-1)\right\}\right)
        \end{split}\end{equation*}

        Sean ahora las bases buscadas $\cc{B}=\{v_1,v_2,v_3\}, \cc{B}'=\{v_1',v_2',v_3'\}$. Por la matriz dada, es necesario que:
        \begin{equation*}
            \left\{\begin{array}{c}
                f(v_1)=v_1' \\
                f(v_2)=v_2' \\
                f(v_3)=0 \Longrightarrow v_3\in Ker(f)
            \end{array}\right.
        \end{equation*}

        Una posible solución es:
        \begin{equation*}\begin{split}
            \cc{B}&=\{(1,0,0), (0,1,0), (-2,3,-1)\} \\
            \cc{B}'&=\{(3, -3, 1), (2, -2, 1), (1,0,0)\}
        \end{split}\end{equation*}

        donde los siguientes determinantes prueban que forman base:
        \begin{equation*}
            \left|\begin{array}{ccc}
                1 & 0 & -2 \\
                0 & 1 & 3 \\
                0 & 0 & -1
            \end{array}\right| = -1\neq 0 \hspace{2cm}
            \left|\begin{array}{ccc}
                3 & 2 & 1 \\
                -3 & -2 & 0 \\
                1 & 1 & 0
            \end{array}\right| \neq 0
        \end{equation*}

        \item Determinar si es posible resolver el apartado anterior con una única base (esto es, con $\cc{B}=\cc{B}'$) y, en caso de ser posible, calcular esa base.

        Sea ahora la bases buscada $\cc{B}=\{v_1,v_2,v_3\}$. Por la matriz dada, es necesario que:
        \begin{equation*}
            \left\{\begin{array}{c}
                f(v_1)=v_1 \\
                f(v_2)=v_2 \\
                f(v_3)=0 \Longrightarrow v_3\in Ker(f)
            \end{array}\right.
        \end{equation*}

        Notamos la base usual de $\bb{R}^3$ como $\cc{B}_u=\{e_1,e_2,e_3\}$. Una posible solución consiste en fijar $v_3=(-2,3,-1)$ y $v_2=e_3=(0,0,1)$. Notamos $v_1 = (x,y,z) = xe_1 + ye_2 + ze_3$.
        Para calcular $v_1$, imponemos la siguiente condición:
        \begin{equation*}\begin{split}
            v_1 = f(v_1) &\Longrightarrow xe_1 + ye_2 + ze_3 = f(xe_1 + ye_2 + ze_3)= xf(e_1) + yf(e_2) + zf(e_3)
            \Longrightarrow \\
            &\quad \Longrightarrow
            xe_1 + ye_2 + ze_3 = x[3e_1 -3e_2 +e_3] + y[2e_1-2e_2+e_3] + ze_3 \Longrightarrow \\
            &\quad \Longrightarrow
            x[2e_1 -3e_2 +e_3] + y[2e_1-3e_2+e_3] = 0 \Longrightarrow \\
            &\quad \Longrightarrow [2e_1 -3e_2 +e_3](x+y)=0
        \end{split}\end{equation*}

        Por tanto, tenemos que la condición que ha de cumplir $v_1$ es $x+y=0$, de lo que deducimos que una posible solución es:
        \begin{equation*}
            \cc{B}=\{(1,-1, 0), (0,0,1), (-2, 3, -1)\}
        \end{equation*}
    \end{enumerate}
\end{ejercicio}



\end{document}