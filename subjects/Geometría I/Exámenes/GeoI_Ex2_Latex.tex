\documentclass[12pt]{article}

% Idioma y codificación
\usepackage[spanish, es-tabla]{babel}       %es-tabla para que se titule "Tabla"
\usepackage[utf8]{inputenc}

% Márgenes
\usepackage[a4paper,top=3cm,bottom=2.5cm,left=3cm,right=3cm]{geometry}

% Comentarios de bloque
\usepackage{verbatim}

% Paquetes de links
\usepackage[hidelinks]{hyperref}    % Permite enlaces
\usepackage{url}                    % redirecciona a la web

% Más opciones para enumeraciones
\usepackage{enumitem}

% Personalizar la portada
\usepackage{titling}

% Paquetes de tablas
\usepackage{multirow}


%------------------------------------------------------------------------

%Paquetes de figuras
\usepackage{caption}
\usepackage{subcaption} % Figuras al lado de otras
\usepackage{float}      % Poner figuras en el sitio indicado H.


% Paquetes de imágenes
\usepackage{graphicx}       % Paquete para añadir imágenes
\usepackage{transparent}    % Para manejar la opacidad de las figuras

% Paquete para usar colores
\usepackage[dvipsnames]{xcolor}
\usepackage{pagecolor}      % Para cambiar el color de la página

% Habilita tamaños de fuente mayores
\usepackage{fix-cm}

% Para los gráficos
\usepackage{tikz}

% Para poder situar los nodos en los grafos
\usetikzlibrary{positioning}


%------------------------------------------------------------------------

% Paquetes de matemáticas
\usepackage{mathtools, amsfonts, amssymb, mathrsfs}
\usepackage[makeroom]{cancel}     % Simplificar tachando
\usepackage{polynom}    % Divisiones y Ruffini
\usepackage{units} % Para poner fracciones diagonales con \nicefrac

\usepackage{pgfplots}   %Representar funciones
\pgfplotsset{compat=1.18}  % Versión 1.18

\usepackage{tikz-cd}    % Para usar diagramas de composiciones
\usetikzlibrary{calc}   % Para usar cálculo de coordenadas en tikz

%Definición de teoremas, etc.
\usepackage{amsthm}
%\swapnumbers   % Intercambia la posición del texto y de la numeración

\theoremstyle{plain}

\makeatletter
\@ifclassloaded{article}{
  \newtheorem{teo}{Teorema}[section]
}{
  \newtheorem{teo}{Teorema}[chapter]  % Se resetea en cada chapter
}
\makeatother

\newtheorem{coro}{Corolario}[teo]           % Se resetea en cada teorema
\newtheorem{prop}[teo]{Proposición}         % Usa el mismo contador que teorema
\newtheorem{lema}[teo]{Lema}                % Usa el mismo contador que teorema

\theoremstyle{remark}
\newtheorem*{observacion}{Observación}

\theoremstyle{definition}

\makeatletter
\@ifclassloaded{article}{
  \newtheorem{definicion}{Definición} [section]     % Se resetea en cada chapter
}{
  \newtheorem{definicion}{Definición} [chapter]     % Se resetea en cada chapter
}
\makeatother

\newtheorem*{notacion}{Notación}
\newtheorem*{ejemplo}{Ejemplo}
\newtheorem*{ejercicio*}{Ejercicio}             % No numerado
\newtheorem{ejercicio}{Ejercicio} [section]     % Se resetea en cada section


% Modificar el formato de la numeración del teorema "ejercicio"
\renewcommand{\theejercicio}{%
  \ifnum\value{section}=0 % Si no se ha iniciado ninguna sección
    \arabic{ejercicio}% Solo mostrar el número de ejercicio
  \else
    \thesection.\arabic{ejercicio}% Mostrar número de sección y número de ejercicio
  \fi
}


% \renewcommand\qedsymbol{$\blacksquare$}         % Cambiar símbolo QED
%------------------------------------------------------------------------

% Paquetes para encabezados
\usepackage{fancyhdr}
\pagestyle{fancy}
\fancyhf{}

\newcommand{\helv}{ % Modificación tamaño de letra
\fontfamily{}\fontsize{12}{12}\selectfont}
\setlength{\headheight}{15pt} % Amplía el tamaño del índice


%\usepackage{lastpage}   % Referenciar última pag   \pageref{LastPage}
\fancyfoot[C]{\thepage}

%------------------------------------------------------------------------

% Conseguir que no ponga "Capítulo 1". Sino solo "1."
\makeatletter
\@ifclassloaded{book}{
  \renewcommand{\chaptermark}[1]{\markboth{\thechapter.\ #1}{}} % En el encabezado
    
  \renewcommand{\@makechapterhead}[1]{%
  \vspace*{50\p@}%
  {\parindent \z@ \raggedright \normalfont
    \ifnum \c@secnumdepth >\m@ne
      \huge\bfseries \thechapter.\hspace{1em}\ignorespaces
    \fi
    \interlinepenalty\@M
    \Huge \bfseries #1\par\nobreak
    \vskip 40\p@
  }}
}
\makeatother

%------------------------------------------------------------------------
% Paquetes de cógido
\usepackage{minted}
\renewcommand\listingscaption{Código fuente}

\usepackage{fancyvrb}
% Personaliza el tamaño de los números de línea
\renewcommand{\theFancyVerbLine}{\small\arabic{FancyVerbLine}}

% Estilo para C++
\newminted{cpp}{
    frame=lines,
    framesep=2mm,
    baselinestretch=1.2,
    linenos,
    escapeinside=||
}

% para minted
\definecolor{LightGray}{rgb}{0.95,0.95,0.92}
\setminted{
    linenos=true,
    stepnumber=5,
    numberfirstline=true,
    autogobble,
    breaklines=true,
    breakautoindent=true,
    breaksymbolleft=,
    breaksymbolright=,
    breaksymbolindentleft=0pt,
    breaksymbolindentright=0pt,
    breaksymbolsepleft=0pt,
    breaksymbolsepright=0pt,
    fontsize=\footnotesize,
    bgcolor=LightGray,
    numbersep=10pt
}


\usepackage{listings} % Para incluir código desde un archivo

\renewcommand\lstlistingname{Código Fuente}
\renewcommand\lstlistlistingname{Índice de Códigos Fuente}

% Definir colores
\definecolor{vscodepurple}{rgb}{0.5,0,0.5}
\definecolor{vscodeblue}{rgb}{0,0,0.8}
\definecolor{vscodegreen}{rgb}{0,0.5,0}
\definecolor{vscodegray}{rgb}{0.5,0.5,0.5}
\definecolor{vscodebackground}{rgb}{0.97,0.97,0.97}
\definecolor{vscodelightgray}{rgb}{0.9,0.9,0.9}

% Configuración para el estilo de C similar a VSCode
\lstdefinestyle{vscode_C}{
  backgroundcolor=\color{vscodebackground},
  commentstyle=\color{vscodegreen},
  keywordstyle=\color{vscodeblue},
  numberstyle=\tiny\color{vscodegray},
  stringstyle=\color{vscodepurple},
  basicstyle=\scriptsize\ttfamily,
  breakatwhitespace=false,
  breaklines=true,
  captionpos=b,
  keepspaces=true,
  numbers=left,
  numbersep=5pt,
  showspaces=false,
  showstringspaces=false,
  showtabs=false,
  tabsize=2,
  frame=tb,
  framerule=0pt,
  aboveskip=10pt,
  belowskip=10pt,
  xleftmargin=10pt,
  xrightmargin=10pt,
  framexleftmargin=10pt,
  framexrightmargin=10pt,
  framesep=0pt,
  rulecolor=\color{vscodelightgray},
  backgroundcolor=\color{vscodebackground},
}

%------------------------------------------------------------------------

% Comandos definidos
\newcommand{\bb}[1]{\mathbb{#1}}
\newcommand{\cc}[1]{\mathcal{#1}}

% I prefer the slanted \leq
\let\oldleq\leq % save them in case they're every wanted
\let\oldgeq\geq
\renewcommand{\leq}{\leqslant}
\renewcommand{\geq}{\geqslant}

% Si y solo si
\newcommand{\sii}{\iff}

% Letras griegas
\newcommand{\eps}{\epsilon}
\newcommand{\veps}{\varepsilon}
\newcommand{\lm}{\lambda}

\newcommand{\ol}{\overline}
\newcommand{\ul}{\underline}
\newcommand{\wt}{\widetilde}
\newcommand{\wh}{\widehat}

\let\oldvec\vec
\renewcommand{\vec}{\overrightarrow}

% Derivadas parciales
\newcommand{\del}[2]{\frac{\partial #1}{\partial #2}}
\newcommand{\Del}[3]{\frac{\partial^{#1} #2}{\partial #3^{#1}}}
\newcommand{\deld}[2]{\dfrac{\partial #1}{\partial #2}}
\newcommand{\Deld}[3]{\dfrac{\partial^{#1} #2}{\partial #3^{#1}}}


\newcommand{\AstIg}{\stackrel{(\ast)}{=}}
\newcommand{\Hop}{\stackrel{L'H\hat{o}pital}{=}}

\newcommand{\red}[1]{{\color{red}#1}} % Para integrales, destacar los cambios.

% Método de integración
\newcommand{\MetInt}[2]{
    \left[\begin{array}{c}
        #1 \\ #2
    \end{array}\right]
}

% Declarar aplicaciones
% 1. Nombre aplicación
% 2. Dominio
% 3. Codominio
% 4. Variable
% 5. Imagen de la variable
\newcommand{\Func}[5]{
    \begin{equation*}
        \begin{array}{rrll}
            #1:& #2 & \longrightarrow & #3\\
               & #4 & \longmapsto & #5
        \end{array}
    \end{equation*}
}

%------------------------------------------------------------------------


\begin{document}

    % 1. Foto de fondo
    % 2. Título
    % 3. Encabezado Izquierdo
    % 4. Color de fondo
    % 5. Coord x del titulo
    % 6. Coord y del titulo
    % 7. Fecha

    
    % 1. Foto de fondo
% 2. Título
% 3. Encabezado Izquierdo
% 4. Color de fondo
% 5. Coord x del titulo
% 6. Coord y del titulo
% 7. Fecha

\newcommand{\portada}[7]{

    \portadaBase{#1}{#2}{#3}{#4}{#5}{#6}{#7}
    \portadaBook{#1}{#2}{#3}{#4}{#5}{#6}{#7}
}

\newcommand{\portadaExamen}[7]{

    \portadaBase{#1}{#2}{#3}{#4}{#5}{#6}{#7}
    \portadaArticle{#1}{#2}{#3}{#4}{#5}{#6}{#7}
}




\newcommand{\portadaBase}[7]{

    % Tiene la portada principal y la licencia Creative Commons
    
    % 1. Foto de fondo
    % 2. Título
    % 3. Encabezado Izquierdo
    % 4. Color de fondo
    % 5. Coord x del titulo
    % 6. Coord y del titulo
    % 7. Fecha
    
    
    \thispagestyle{empty}               % Sin encabezado ni pie de página
    \newgeometry{margin=0cm}        % Márgenes nulos para la primera página
    
    
    % Encabezado
    \fancyhead[L]{\helv #3}
    \fancyhead[R]{\helv \nouppercase{\leftmark}}
    
    
    \pagecolor{#4}        % Color de fondo para la portada
    
    \begin{figure}[p]
        \centering
        \transparent{0.3}           % Opacidad del 30% para la imagen
        
        \includegraphics[width=\paperwidth, keepaspectratio]{assets/#1}
    
        \begin{tikzpicture}[remember picture, overlay]
            \node[anchor=north west, text=white, opacity=1, font=\fontsize{60}{90}\selectfont\bfseries\sffamily, align=left] at (#5, #6) {#2};
            
            \node[anchor=south east, text=white, opacity=1, font=\fontsize{12}{18}\selectfont\sffamily, align=right] at (9.7, 3) {\textbf{\href{https://losdeldgiim.github.io/}{Los Del DGIIM}}};
            
            \node[anchor=south east, text=white, opacity=1, font=\fontsize{12}{15}\selectfont\sffamily, align=right] at (9.7, 1.8) {Doble Grado en Ingeniería Informática y Matemáticas\\Universidad de Granada};
        \end{tikzpicture}
    \end{figure}
    
    
    \restoregeometry        % Restaurar márgenes normales para las páginas subsiguientes
    \pagecolor{white}       % Restaurar el color de página
    
    
    \newpage
    \thispagestyle{empty}               % Sin encabezado ni pie de página
    \begin{tikzpicture}[remember picture, overlay]
        \node[anchor=south west, inner sep=3cm] at (current page.south west) {
            \begin{minipage}{0.5\paperwidth}
                \href{https://creativecommons.org/licenses/by-nc-nd/4.0/}{
                    \includegraphics[height=2cm]{assets/Licencia.png}
                }\vspace{1cm}\\
                Esta obra está bajo una
                \href{https://creativecommons.org/licenses/by-nc-nd/4.0/}{
                    Licencia Creative Commons Atribución-NoComercial-SinDerivadas 4.0 Internacional (CC BY-NC-ND 4.0).
                }\\
    
                Eres libre de compartir y redistribuir el contenido de esta obra en cualquier medio o formato, siempre y cuando des el crédito adecuado a los autores originales y no persigas fines comerciales. 
            \end{minipage}
        };
    \end{tikzpicture}
    
    
    
    % 1. Foto de fondo
    % 2. Título
    % 3. Encabezado Izquierdo
    % 4. Color de fondo
    % 5. Coord x del titulo
    % 6. Coord y del titulo
    % 7. Fecha


}


\newcommand{\portadaBook}[7]{

    % 1. Foto de fondo
    % 2. Título
    % 3. Encabezado Izquierdo
    % 4. Color de fondo
    % 5. Coord x del titulo
    % 6. Coord y del titulo
    % 7. Fecha

    % Personaliza el formato del título
    \pretitle{\begin{center}\bfseries\fontsize{42}{56}\selectfont}
    \posttitle{\par\end{center}\vspace{2em}}
    
    % Personaliza el formato del autor
    \preauthor{\begin{center}\Large}
    \postauthor{\par\end{center}\vfill}
    
    % Personaliza el formato de la fecha
    \predate{\begin{center}\huge}
    \postdate{\par\end{center}\vspace{2em}}
    
    \title{#2}
    \author{\href{https://losdeldgiim.github.io/}{Los Del DGIIM}}
    \date{Granada, #7}
    \maketitle
    
    \tableofcontents
}




\newcommand{\portadaArticle}[7]{

    % 1. Foto de fondo
    % 2. Título
    % 3. Encabezado Izquierdo
    % 4. Color de fondo
    % 5. Coord x del titulo
    % 6. Coord y del titulo
    % 7. Fecha

    % Personaliza el formato del título
    \pretitle{\begin{center}\bfseries\fontsize{42}{56}\selectfont}
    \posttitle{\par\end{center}\vspace{2em}}
    
    % Personaliza el formato del autor
    \preauthor{\begin{center}\Large}
    \postauthor{\par\end{center}\vspace{3em}}
    
    % Personaliza el formato de la fecha
    \predate{\begin{center}\huge}
    \postdate{\par\end{center}\vspace{5em}}
    
    \title{#2}
    \author{\href{https://losdeldgiim.github.io/}{Los Del DGIIM}}
    \date{Granada, #7}
    \thispagestyle{empty}               % Sin encabezado ni pie de página
    \maketitle
    \vfill
}
    \portadaExamen{ffccA4.jpg}{Geometría I\\Examen II}{Geometría I. Examen II}{MidnightBlue}{-8}{28}{2023}

    \begin{description}
        \item[Asignatura] Geometría I.
        \item[Curso Académico] 2021-22.
        \item[Grado] Doble Grado en Ingeniería Informática y Matemáticas.
        \item[Grupo] Único.
        \item[Profesor] Juan de Dios Pérez Jiménez.
        \item[Descripción] 1ª Prueba. Temas 1 y 2.
        \item[Fecha] 3 de diciembre de 2021.
        \item[Duración] 90 minutos.
    
    \end{description}
    \newpage
    

    \begin{ejercicio}
    En el espacio vectorial $\mathbb{R}^4$, se consideran los subconjuntos
    $$ U = \left\{ \left( x_1, x_2, x_3, x_4\right) \in \mathbb{R}^4 :\; x_1 - 2x_3 + 2x_4 = 0 \right\}, $$
    $$ W = \mathcal{L}\left(\{(1,0,1,1),(0,1,2,3)\}\right). $$

    \begin{enumerate}
        \item \textbf{[1 punto]} Demuestra que $U$ es un subespacio vectorial de $\mathbb{R}^4$.

        Obtengo unas ecuaciones paramétricas de $U$:
        \begin{equation*}\begin{split}
            U&=\left\{(x_1, x_2, x_3, x_4)\in \bb{R}^4\left|\begin{array}{c}
                x_1=2x_3-2x_4 \\
                x_i=x_i \quad i=2,3,4
            \end{array}\right.\right\} \\
            &= \left\{(2x_3-2x_4,x_2,x_3, x_4)\in \bb{R}^4\right\}
        \end{split}\end{equation*}

        Sea $u,u'\in U$, y sean $a,b\in \bb{R}$. Para ver que es un subespacio vectorial, comprobemos que $au+au'\in U$.
        \begin{equation*}\begin{split}
            au+bu' &= a(2(x_3-x_4),x_2,x_3, x_4) + b(2(x_3'-x_4'),x_2',x_3', x_4')
            =\\&= \left(2(ax_3+bx_3') -2(ax_4+bx_4'), ax_2+bx_2', ax_3+bx_3', ax_4+bx_4'\right) \quad \in U
        \end{split}\end{equation*}
        Como $au+au'\in U$, tenemos que $U$ es cerrado para sumas y producto por escalares, por lo que es un subespacio vectorial.
        
        \item \textbf{[1 punto]} Calcula una base $\mathcal{B}_U$ de $U$.

        Sea $(0,0,1,1),(4,3,1,-1), (2,0,2,1)\in U$.
        \begin{equation*}
            A = \left(\begin{array}{ccc}
                0 & 4 & 2 \\
                0 & 3 & 0 \\
                1 & 1 & 2 \\
                1 & -1 & 1
            \end{array}\right) \qquad \left|\begin{array}{ccc}
                0 & 3 & 0 \\
                1 & 1 & 2 \\
                1 & -1 & 1
            \end{array}\right| = -3 \left|\begin{array}{cc}
                1 & 2 \\
                1 & 1
            \end{array}\right| \neq 0 \Longrightarrow rg(A)=3
        \end{equation*}

        Por tanto, esos tres vectores son linealmente independientes en $\bb{R}^4$. Además, como $\dim_{\bb{R}}(\bb{R}^4)-\dim_{\bb{R}}(U)=$nº de ec. implícitas, tenemos que:
        \begin{equation*}
            \dim_{\bb{R}}(U) = \dim_{\bb{R}}(\bb{R}^4) - \text{nº de ec. implícitas} = 4-1=3
        \end{equation*}

        Como $\dim_{\bb{R}}(U)=3$ y esos tres vectores son linealmente independientes, tenemos que forman una base.
        \begin{equation*}
            \cc{B}_U = \left\{(0,0,1,1),(4,3,1,-1), (2,0,2,1)\right\}
        \end{equation*}
        
        \item \textbf{[1 punto]} Amplía la base $\mathcal{B}$ a una base $\mathcal{B}$ de $\mathbb{R}^4$.

        Como $\dim_{\bb{R}}(\bb{R}^4)=4$, tenemos que la base $\cc{B}$ ha de tener 4 vectores:
        \begin{equation*}
            \left|\begin{array}{cccc}
                0 & 4 & 2 & 0 \\
                0 & 3 & 0 & 0\\
                1 & 1 & 2 & 0\\
                1 & -1 & 1 & 1
            \end{array}\right| = \left|\begin{array}{ccc}
                0 & 4 & 2 \\
                0 & 3 & 0\\
                1 & 1 & 2\\
            \end{array}\right| = 
            \left|\begin{array}{ccc}
                4 & 2 \\
                3 & 0\\
            \end{array}\right| = -6 \neq 0
        \end{equation*}
        Por tanto, los 4 vectores son linealmente independientes. Como la dimensión es 4, tenemos que forman base.
        \begin{equation*}
            \cc{B} = \left\{(0,0,1,1),(4,3,1,-1), (2,0,2,1), (0,0,0,1)\right\}
        \end{equation*}
        
        \item \textbf{[1 punto]} Calcula las coordenadas del vector $w = (3, -1, 1, 1)$ respecto de $\mathcal{B}$.

        Tomando la base usual,
        \begin{equation*}
            \cc{B}_u=\left\{(1,0,0,0),(0,1,0,0),(0,0,1,0),(0,0,0,1)\right\}    
        \end{equation*}

        Tenemos que:
        \begin{equation*}\begin{split}
            w &= (3,-1,1,1)_{\cc{B}_u} \\
            &= 3(1,0,0,0) -1(0,1,0,0) +1(0,0,1,0) +1(0,0,0,1) \\
            &= a(0,0,1,1) +b(4,3,1,-1) +c(2,0,2,1) +d(0,0,0,1) \\
            &= (4b+2c, 3b, a+b+2c, a-b+c+d)_{\cc{B}_u} \\
            &= (a,b,c,d)_{\cc{B}}
        \end{split}\end{equation*}

        Por la unicidad de expresión respecto a una base:
        \begin{equation*}
            \left\{\begin{array}{l}
                3 = 4b+2c \longrightarrow c=\frac{3-4b}{2} = \frac{3+\frac{4}{3}}{2} = \frac{\frac{13}{3}}{2} = \frac{13}{6} \\
                -1 = 3b \longrightarrow b=-\frac{1}{3} \\
                1=a+b+2c \longrightarrow a=1-b-2c = 1+\frac{1}{3} -\frac{13}{3} = -3 \\
                1 = a-b+c+d \longrightarrow d=1-a+b-c = 1+3-\frac{1}{3} - \frac{13}{6} = \frac{3}{2}
            \end{array}\right.
        \end{equation*}

        Por tanto,
        \begin{equation*}
            w = (3,-1,1,1)_{\cc{B}_u} = \left(-3,-\frac{1}{3},\frac{13}{6},\frac{13}{2}\right)_{\cc{B}}
        \end{equation*}
        
        \item \textbf{[1 punto]} Calcula la dimensión de $U+W$. Comprueba si dicha suma es directa.

        $\cc{B}_U = \left\{(0,0,1,1),(4,3,1,-1), (2,0,2,1)\right\}$ base de $U$. Calculamos ahora $\cc{B}_W$ una base de $W$.
        \begin{equation*}
            rg\left(\begin{array}{cc}
                1 & 0 \\
                0 & 1 \\
                1 & 2 \\
                1 & 2
            \end{array}\right) = 2, \qquad \text{ya que }\left|\begin{array}{cc}
                1 & 0 \\
                0 & 1
            \end{array}\right| = 1\neq 0.
        \end{equation*}
        Por tanto, ambos vectores son linealmente independientes. Como además son sistema generador, forman base.
        \begin{equation*}
            \cc{B}_W=\{(1,0,1,1),(0,1,2,3)\}
        \end{equation*}

        Por tanto, tenemos que:
        \begin{equation*}
            U+W = \mathcal{L}\left(\{(0,0,1,1),(4,3,1,-1), (2,0,2,1),(1,0,1,1),(0,1,2,3)\}\right)
        \end{equation*}

        Veamos cuáles de ellos son linealmente independientes:
        \begin{equation*}
            \left|\begin{array}{cccc}
                0 & 4 & 2 & 1 \\
                0 & 3 & 0 & 0 \\
                1 & 1 & 2 & 1 \\
                1 & -1 & 1 & 1 \\
            \end{array}\right|
            = 3\left|\begin{array}{ccc}
                0 & 2 & 1 \\
                1 & 2 & 1 \\
                1 & 1 & 1 \\
            \end{array}\right|
            = 3\left|\begin{array}{ccc}
                0 & 2 & 1 \\
                0 & 1 & 0 \\
                1 & 1 & 1 \\
            \end{array}\right|
            = 3\left|\begin{array}{cc}
                2 & 1 \\
                1 & 0 \\
            \end{array}\right|\neq 0
        \end{equation*}

        Por tanto, tenemos que los primeros 4 vectores son linealmente dependientes. Como pertenecen a $\bb{R}^4$, no puede haber 5 vectores linealmente independientes, por lo que el 5º vector será linealmente dependiente.
        \begin{equation*}
            U+W = \mathcal{L}\left(\{(0,0,1,1),(4,3,1,-1), (2,0,2,1),(1,0,1,1)\}\right)
        \end{equation*}

        Como son linealmente independientes y forman sistema generador, forman base. Por tanto,
        \begin{equation*}
            \dim_{\bb{R}}(U+W)=4\Longrightarrow U+W=\bb{R}^4
        \end{equation*}

        Para ver si es directa o no, necesito calcular $U\cap W$.
        \begin{equation*}
            \dim_{\bb{R}}(U\cap W)=\dim_{\bb{R}}(U) + \dim_{\bb{R}}(W) - \dim_{\bb{R}}(U+W) = 3+2-4=1
        \end{equation*}
        Como $\dim_{\bb{R}}(U\cap W)=1\neq 0 \Longrightarrow U\cap W\neq \{0\}$. Por tanto, la suma no es directa.
        
        \item \textbf{[1 punto]} Calcula unas ecuaciones cartesianas de $U+W$ y de $U\cup W$.

        Como $U+W=\bb{R}^4$, tenemos que no tiene ecuaciones cartesianas. Para calcular las ecuaciones cartesianas de $U\cap W$, calculo primero las de $W$.

        Sea $(x,y,z,t)\in W$. Este debe ser linealmente dependiente de los elementos de su base, luego:
        \begin{equation*}
            rg\left(\begin{array}{ccc}
                1 & 0 & x\\
                0 & 1 & y\\
                1 & 2 & z\\
                1 & 2 & t
            \end{array}\right) = 2
        \end{equation*}

        Por tanto,
        \begin{equation*}
            \left|\begin{array}{ccc}
                1 & 0 & x\\
                0 & 1 & y\\
                1 & 2 & z\\
            \end{array}\right| = 0 = \left|\begin{array}{cc}
                1 & y\\
                2 & z\\
            \end{array}\right| + x\left|\begin{array}{cc}
                0 & 1\\
                1 & 2\\
            \end{array}\right| = z-2y-x=0
        \end{equation*}
        \begin{equation*}
            \left|\begin{array}{ccc}
                1 & 0 & x\\
                0 & 1 & y\\
                1 & 2 & t\\
            \end{array}\right| = t-2y-x=0
        \end{equation*}

        Por tanto,
        \begin{equation*}\begin{split}
            W&= \left\{(x,y,z,t)\in \bb{R}^4\left|\begin{array}{c}
                x+2y-z=0 \\
                x+2y-t=0 \\
            \end{array}\right.\right\} \\
            &= \left\{(x,y,z,t)\in \bb{R}^4\left|\begin{array}{c}
                x+2y=z \\
                z=t \\
            \end{array}\right.\right\} \\
        \end{split}\end{equation*}

        Por tanto, tenemos que las ecuaciones cartesianas de la intersección son:
        \begin{equation*}\begin{split}
            U\cap W&= \left\{(x,y,z,t)\in \bb{R}^4\left|\begin{array}{c}
                x-2z+2t=0 \\
                x+2y=z \\
                z=t \\
            \end{array}\right.\right\} \\
            &= \left\{(x,y,z,t)\in \bb{R}^4\left|\begin{array}{c}
                x=0 \\
                x+2y=z \\
                z=t \\
            \end{array}\right.\right\}
        \end{split}\end{equation*}
    \end{enumerate}
\end{ejercicio}

\begin{ejercicio}\textbf{[4 puntos]}
    Discute y resuelve, cuando sea posible, en función del parámetro $a\in \mathbb{R}$ el siguiente sistema:

    $$
        \left\{
        \begin{array}{rcl}
            ax + y + z + t + u &=& 1  \\
            \\
            x + y + az + t + u &=& -1  \\
            \\
            x + y + z + t + au &=& 1  \\
        \end{array}
        \right.
    $$

    Sea $A$ la matriz de coeficientes del sistema, y $A|C$ la matriz ampliada:
    \begin{equation*}
        A=\left(\begin{array}{ccccc}
            a & 1 & 1 & 1 & 1 \\
            1 & 1 & a & 1 & 1 \\
            1 & 1 & 1 & 1 & a \\
        \end{array}\right)
        \qquad \qquad
        A|C=\left(\begin{array}{cccccc}
            a & 1 & 1 & 1 & 1 & 1\\
            1 & 1 & a & 1 & 1 & -1\\
            1 & 1 & 1 & 1 & a & 1\\
        \end{array}\right)
    \end{equation*}
    \begin{itemize}
        \item \underline{Si $a=1$}:
        \begin{equation*}
            rg(A)=1 \qquad rg(A|C)=2
        \end{equation*}
        Como $rg(A)\neq rg(A|C)$, por el Teorema de Rouché-Frobenious, tenemos que se trata de un \textbf{Sistema Incompatible (SI)}.

        \item \underline{Si $a\neq 1$}:
        \begin{equation*}
            rg(A)=3 \qquad rg(A|C)=3
        \end{equation*}
        Como $rg(A)=rg(A|C)<$nº de incógnitas, por el Teorema de Rouché-Frobenious, tenemos que se trata de un \textbf{Sistema Compatible Indeterminado (SCI)}.
        \begin{equation*}
            \text{nº parámetros} = \text{nº incógnitas} - rg(A)=5-3=2
        \end{equation*}
        Resuelvo por tanto para $a\neq 1$ con el método de Cramer. Sea $t,u\in \bb{R}$ los dos parámetros libres. El sistema de Cramer queda:
        $$
            \left\{
            \begin{array}{rcl}
                ax + y + z &=& 1 -t-u \\
                \\
                x + y + az &=& -1-t-u  \\
                \\
                x + y + z &=& 1 -t-au  \\
            \end{array}
            \right.
        $$
        El determinante de la matriz de coeficientes es:
        \begin{equation*}
            \left|\begin{array}{ccc}
                a & 1 & 1 \\
                1 & 1 & a \\
                1 & 1 & 1
            \end{array}\right| = a + a +1-1-a^2-1 = -a^2+2a-1 = -(a-1)^2
        \end{equation*}

        Por tanto, por la regla de Cramer:
        \begin{equation*}
            x = \frac{\left|\begin{array}{ccc}
                1-t-u & 1 & 1 \\
                -1-t-u & 1 & a \\
                1-t-au & 1 & 1
            \end{array}\right|}{-(a-1)^2}
            = \frac{\left|\begin{array}{ccc}
                u(a-1) & 0 & 0 \\
                -1-t-u & 1 & a \\
                1-t-au & 1 & 1
            \end{array}\right|}{-(a-1)^2}
            = \frac{u(a-1)(1-a)}{-(a-1)^2} = \frac{u(a-1)^2}{(a-1)^2} = u
        \end{equation*}
        \begin{equation*}
            y = \frac{\left|\begin{array}{ccc}
                a & 1-t-u & 1 \\
                1 & -1-t-u & a \\
                1 & 1-t-au & 1
            \end{array}\right|}{-(a-1)^2} = \dots = \frac{-a^2u-at-au+a+t+2u+1}{a-1}
        \end{equation*}
        \begin{multline*}
            z = \frac{\left|\begin{array}{ccc}
                a & 1 & 1-t-u \\
                1 & 1 & -1-t-u \\
                1 & 1 & 1-t-au
            \end{array}\right|}{-(a-1)^2}
            = \frac{\left|\begin{array}{ccc}
                a & 1 & 1-t-u \\
                1 & 1 & -1-t-u \\
                0 & 0 & 2+u(1-a)
            \end{array}\right|}{-(a-1)^2}
            = -\frac{(a-1)(2+u(1-a))}{(a-1)^2}
            =\\= -\frac{2+u(1-a)}{a-1}
            = \frac{2+u(1-a)}{1-a} = \frac{1}{1-a} +u
        \end{multline*}
    \end{itemize}
\end{ejercicio}
\end{document}