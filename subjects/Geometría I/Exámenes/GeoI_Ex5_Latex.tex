\documentclass[12pt]{article}

% Idioma y codificación
\usepackage[spanish, es-tabla]{babel}       %es-tabla para que se titule "Tabla"
\usepackage[utf8]{inputenc}

% Márgenes
\usepackage[a4paper,top=3cm,bottom=2.5cm,left=3cm,right=3cm]{geometry}

% Comentarios de bloque
\usepackage{verbatim}

% Paquetes de links
\usepackage[hidelinks]{hyperref}    % Permite enlaces
\usepackage{url}                    % redirecciona a la web

% Más opciones para enumeraciones
\usepackage{enumitem}

% Personalizar la portada
\usepackage{titling}

% Paquetes de tablas
\usepackage{multirow}


%------------------------------------------------------------------------

%Paquetes de figuras
\usepackage{caption}
\usepackage{subcaption} % Figuras al lado de otras
\usepackage{float}      % Poner figuras en el sitio indicado H.


% Paquetes de imágenes
\usepackage{graphicx}       % Paquete para añadir imágenes
\usepackage{transparent}    % Para manejar la opacidad de las figuras

% Paquete para usar colores
\usepackage[dvipsnames]{xcolor}
\usepackage{pagecolor}      % Para cambiar el color de la página

% Habilita tamaños de fuente mayores
\usepackage{fix-cm}

% Para los gráficos
\usepackage{tikz}

% Para poder situar los nodos en los grafos
\usetikzlibrary{positioning}


%------------------------------------------------------------------------

% Paquetes de matemáticas
\usepackage{mathtools, amsfonts, amssymb, mathrsfs}
\usepackage[makeroom]{cancel}     % Simplificar tachando
\usepackage{polynom}    % Divisiones y Ruffini
\usepackage{units} % Para poner fracciones diagonales con \nicefrac

\usepackage{pgfplots}   %Representar funciones
\pgfplotsset{compat=1.18}  % Versión 1.18

\usepackage{tikz-cd}    % Para usar diagramas de composiciones
\usetikzlibrary{calc}   % Para usar cálculo de coordenadas en tikz

%Definición de teoremas, etc.
\usepackage{amsthm}
%\swapnumbers   % Intercambia la posición del texto y de la numeración

\theoremstyle{plain}

\makeatletter
\@ifclassloaded{article}{
  \newtheorem{teo}{Teorema}[section]
}{
  \newtheorem{teo}{Teorema}[chapter]  % Se resetea en cada chapter
}
\makeatother

\newtheorem{coro}{Corolario}[teo]           % Se resetea en cada teorema
\newtheorem{prop}[teo]{Proposición}         % Usa el mismo contador que teorema
\newtheorem{lema}[teo]{Lema}                % Usa el mismo contador que teorema

\theoremstyle{remark}
\newtheorem*{observacion}{Observación}

\theoremstyle{definition}

\makeatletter
\@ifclassloaded{article}{
  \newtheorem{definicion}{Definición} [section]     % Se resetea en cada chapter
}{
  \newtheorem{definicion}{Definición} [chapter]     % Se resetea en cada chapter
}
\makeatother

\newtheorem*{notacion}{Notación}
\newtheorem*{ejemplo}{Ejemplo}
\newtheorem*{ejercicio*}{Ejercicio}             % No numerado
\newtheorem{ejercicio}{Ejercicio} [section]     % Se resetea en cada section


% Modificar el formato de la numeración del teorema "ejercicio"
\renewcommand{\theejercicio}{%
  \ifnum\value{section}=0 % Si no se ha iniciado ninguna sección
    \arabic{ejercicio}% Solo mostrar el número de ejercicio
  \else
    \thesection.\arabic{ejercicio}% Mostrar número de sección y número de ejercicio
  \fi
}


% \renewcommand\qedsymbol{$\blacksquare$}         % Cambiar símbolo QED
%------------------------------------------------------------------------

% Paquetes para encabezados
\usepackage{fancyhdr}
\pagestyle{fancy}
\fancyhf{}

\newcommand{\helv}{ % Modificación tamaño de letra
\fontfamily{}\fontsize{12}{12}\selectfont}
\setlength{\headheight}{15pt} % Amplía el tamaño del índice


%\usepackage{lastpage}   % Referenciar última pag   \pageref{LastPage}
\fancyfoot[C]{\thepage}

%------------------------------------------------------------------------

% Conseguir que no ponga "Capítulo 1". Sino solo "1."
\makeatletter
\@ifclassloaded{book}{
  \renewcommand{\chaptermark}[1]{\markboth{\thechapter.\ #1}{}} % En el encabezado
    
  \renewcommand{\@makechapterhead}[1]{%
  \vspace*{50\p@}%
  {\parindent \z@ \raggedright \normalfont
    \ifnum \c@secnumdepth >\m@ne
      \huge\bfseries \thechapter.\hspace{1em}\ignorespaces
    \fi
    \interlinepenalty\@M
    \Huge \bfseries #1\par\nobreak
    \vskip 40\p@
  }}
}
\makeatother

%------------------------------------------------------------------------
% Paquetes de cógido
\usepackage{minted}
\renewcommand\listingscaption{Código fuente}

\usepackage{fancyvrb}
% Personaliza el tamaño de los números de línea
\renewcommand{\theFancyVerbLine}{\small\arabic{FancyVerbLine}}

% Estilo para C++
\newminted{cpp}{
    frame=lines,
    framesep=2mm,
    baselinestretch=1.2,
    linenos,
    escapeinside=||
}

% para minted
\definecolor{LightGray}{rgb}{0.95,0.95,0.92}
\setminted{
    linenos=true,
    stepnumber=5,
    numberfirstline=true,
    autogobble,
    breaklines=true,
    breakautoindent=true,
    breaksymbolleft=,
    breaksymbolright=,
    breaksymbolindentleft=0pt,
    breaksymbolindentright=0pt,
    breaksymbolsepleft=0pt,
    breaksymbolsepright=0pt,
    fontsize=\footnotesize,
    bgcolor=LightGray,
    numbersep=10pt
}


\usepackage{listings} % Para incluir código desde un archivo

\renewcommand\lstlistingname{Código Fuente}
\renewcommand\lstlistlistingname{Índice de Códigos Fuente}

% Definir colores
\definecolor{vscodepurple}{rgb}{0.5,0,0.5}
\definecolor{vscodeblue}{rgb}{0,0,0.8}
\definecolor{vscodegreen}{rgb}{0,0.5,0}
\definecolor{vscodegray}{rgb}{0.5,0.5,0.5}
\definecolor{vscodebackground}{rgb}{0.97,0.97,0.97}
\definecolor{vscodelightgray}{rgb}{0.9,0.9,0.9}

% Configuración para el estilo de C similar a VSCode
\lstdefinestyle{vscode_C}{
  backgroundcolor=\color{vscodebackground},
  commentstyle=\color{vscodegreen},
  keywordstyle=\color{vscodeblue},
  numberstyle=\tiny\color{vscodegray},
  stringstyle=\color{vscodepurple},
  basicstyle=\scriptsize\ttfamily,
  breakatwhitespace=false,
  breaklines=true,
  captionpos=b,
  keepspaces=true,
  numbers=left,
  numbersep=5pt,
  showspaces=false,
  showstringspaces=false,
  showtabs=false,
  tabsize=2,
  frame=tb,
  framerule=0pt,
  aboveskip=10pt,
  belowskip=10pt,
  xleftmargin=10pt,
  xrightmargin=10pt,
  framexleftmargin=10pt,
  framexrightmargin=10pt,
  framesep=0pt,
  rulecolor=\color{vscodelightgray},
  backgroundcolor=\color{vscodebackground},
}

%------------------------------------------------------------------------

% Comandos definidos
\newcommand{\bb}[1]{\mathbb{#1}}
\newcommand{\cc}[1]{\mathcal{#1}}

% I prefer the slanted \leq
\let\oldleq\leq % save them in case they're every wanted
\let\oldgeq\geq
\renewcommand{\leq}{\leqslant}
\renewcommand{\geq}{\geqslant}

% Si y solo si
\newcommand{\sii}{\iff}

% Letras griegas
\newcommand{\eps}{\epsilon}
\newcommand{\veps}{\varepsilon}
\newcommand{\lm}{\lambda}

\newcommand{\ol}{\overline}
\newcommand{\ul}{\underline}
\newcommand{\wt}{\widetilde}
\newcommand{\wh}{\widehat}

\let\oldvec\vec
\renewcommand{\vec}{\overrightarrow}

% Derivadas parciales
\newcommand{\del}[2]{\frac{\partial #1}{\partial #2}}
\newcommand{\Del}[3]{\frac{\partial^{#1} #2}{\partial #3^{#1}}}
\newcommand{\deld}[2]{\dfrac{\partial #1}{\partial #2}}
\newcommand{\Deld}[3]{\dfrac{\partial^{#1} #2}{\partial #3^{#1}}}


\newcommand{\AstIg}{\stackrel{(\ast)}{=}}
\newcommand{\Hop}{\stackrel{L'H\hat{o}pital}{=}}

\newcommand{\red}[1]{{\color{red}#1}} % Para integrales, destacar los cambios.

% Método de integración
\newcommand{\MetInt}[2]{
    \left[\begin{array}{c}
        #1 \\ #2
    \end{array}\right]
}

% Declarar aplicaciones
% 1. Nombre aplicación
% 2. Dominio
% 3. Codominio
% 4. Variable
% 5. Imagen de la variable
\newcommand{\Func}[5]{
    \begin{equation*}
        \begin{array}{rrll}
            #1:& #2 & \longrightarrow & #3\\
               & #4 & \longmapsto & #5
        \end{array}
    \end{equation*}
}

%------------------------------------------------------------------------


\begin{document}

    % 1. Foto de fondo
    % 2. Título
    % 3. Encabezado Izquierdo
    % 4. Color de fondo
    % 5. Coord x del titulo
    % 6. Coord y del titulo
    % 7. Fecha

    
    % 1. Foto de fondo
% 2. Título
% 3. Encabezado Izquierdo
% 4. Color de fondo
% 5. Coord x del titulo
% 6. Coord y del titulo
% 7. Fecha

\newcommand{\portada}[7]{

    \portadaBase{#1}{#2}{#3}{#4}{#5}{#6}{#7}
    \portadaBook{#1}{#2}{#3}{#4}{#5}{#6}{#7}
}

\newcommand{\portadaExamen}[7]{

    \portadaBase{#1}{#2}{#3}{#4}{#5}{#6}{#7}
    \portadaArticle{#1}{#2}{#3}{#4}{#5}{#6}{#7}
}




\newcommand{\portadaBase}[7]{

    % Tiene la portada principal y la licencia Creative Commons
    
    % 1. Foto de fondo
    % 2. Título
    % 3. Encabezado Izquierdo
    % 4. Color de fondo
    % 5. Coord x del titulo
    % 6. Coord y del titulo
    % 7. Fecha
    
    
    \thispagestyle{empty}               % Sin encabezado ni pie de página
    \newgeometry{margin=0cm}        % Márgenes nulos para la primera página
    
    
    % Encabezado
    \fancyhead[L]{\helv #3}
    \fancyhead[R]{\helv \nouppercase{\leftmark}}
    
    
    \pagecolor{#4}        % Color de fondo para la portada
    
    \begin{figure}[p]
        \centering
        \transparent{0.3}           % Opacidad del 30% para la imagen
        
        \includegraphics[width=\paperwidth, keepaspectratio]{assets/#1}
    
        \begin{tikzpicture}[remember picture, overlay]
            \node[anchor=north west, text=white, opacity=1, font=\fontsize{60}{90}\selectfont\bfseries\sffamily, align=left] at (#5, #6) {#2};
            
            \node[anchor=south east, text=white, opacity=1, font=\fontsize{12}{18}\selectfont\sffamily, align=right] at (9.7, 3) {\textbf{\href{https://losdeldgiim.github.io/}{Los Del DGIIM}}};
            
            \node[anchor=south east, text=white, opacity=1, font=\fontsize{12}{15}\selectfont\sffamily, align=right] at (9.7, 1.8) {Doble Grado en Ingeniería Informática y Matemáticas\\Universidad de Granada};
        \end{tikzpicture}
    \end{figure}
    
    
    \restoregeometry        % Restaurar márgenes normales para las páginas subsiguientes
    \pagecolor{white}       % Restaurar el color de página
    
    
    \newpage
    \thispagestyle{empty}               % Sin encabezado ni pie de página
    \begin{tikzpicture}[remember picture, overlay]
        \node[anchor=south west, inner sep=3cm] at (current page.south west) {
            \begin{minipage}{0.5\paperwidth}
                \href{https://creativecommons.org/licenses/by-nc-nd/4.0/}{
                    \includegraphics[height=2cm]{assets/Licencia.png}
                }\vspace{1cm}\\
                Esta obra está bajo una
                \href{https://creativecommons.org/licenses/by-nc-nd/4.0/}{
                    Licencia Creative Commons Atribución-NoComercial-SinDerivadas 4.0 Internacional (CC BY-NC-ND 4.0).
                }\\
    
                Eres libre de compartir y redistribuir el contenido de esta obra en cualquier medio o formato, siempre y cuando des el crédito adecuado a los autores originales y no persigas fines comerciales. 
            \end{minipage}
        };
    \end{tikzpicture}
    
    
    
    % 1. Foto de fondo
    % 2. Título
    % 3. Encabezado Izquierdo
    % 4. Color de fondo
    % 5. Coord x del titulo
    % 6. Coord y del titulo
    % 7. Fecha


}


\newcommand{\portadaBook}[7]{

    % 1. Foto de fondo
    % 2. Título
    % 3. Encabezado Izquierdo
    % 4. Color de fondo
    % 5. Coord x del titulo
    % 6. Coord y del titulo
    % 7. Fecha

    % Personaliza el formato del título
    \pretitle{\begin{center}\bfseries\fontsize{42}{56}\selectfont}
    \posttitle{\par\end{center}\vspace{2em}}
    
    % Personaliza el formato del autor
    \preauthor{\begin{center}\Large}
    \postauthor{\par\end{center}\vfill}
    
    % Personaliza el formato de la fecha
    \predate{\begin{center}\huge}
    \postdate{\par\end{center}\vspace{2em}}
    
    \title{#2}
    \author{\href{https://losdeldgiim.github.io/}{Los Del DGIIM}}
    \date{Granada, #7}
    \maketitle
    
    \tableofcontents
}




\newcommand{\portadaArticle}[7]{

    % 1. Foto de fondo
    % 2. Título
    % 3. Encabezado Izquierdo
    % 4. Color de fondo
    % 5. Coord x del titulo
    % 6. Coord y del titulo
    % 7. Fecha

    % Personaliza el formato del título
    \pretitle{\begin{center}\bfseries\fontsize{42}{56}\selectfont}
    \posttitle{\par\end{center}\vspace{2em}}
    
    % Personaliza el formato del autor
    \preauthor{\begin{center}\Large}
    \postauthor{\par\end{center}\vspace{3em}}
    
    % Personaliza el formato de la fecha
    \predate{\begin{center}\huge}
    \postdate{\par\end{center}\vspace{5em}}
    
    \title{#2}
    \author{\href{https://losdeldgiim.github.io/}{Los Del DGIIM}}
    \date{Granada, #7}
    \thispagestyle{empty}               % Sin encabezado ni pie de página
    \maketitle
    \vfill
}
    \portadaExamen{ffccA4.jpg}{Geometría I\\Examen V}{Geometría I. Examen V}{MidnightBlue}{-8}{28}{2023}{Arturo Olivares Martos}

    \begin{description}
        \item[Asignatura] Geometría I.
        \item[Curso Académico] 2021-22.
        \item[Grado] Doble Grado en Ingeniería Informática y Matemáticas.
        \item[Grupo] Único.
        \item[Profesor] Juan de Dios Pérez Jiménez\footnote{El examen lo pone el departamento.}.
        \item[Descripción] Convocatoria Extraordinaria.
        \item[Fecha] 15 de febrero de 2022.
        %\item[Duración] 3 horas.
    
    \end{description}
    \newpage


\begin{ejercicio}
    \textbf{[4 puntos]} Sean $V$ y $V'$ espacios vectoriales sobre un mismo cuerpo $K$. Razonar si son verdaderas o falsas las siguientes afirmaciones:
    \begin{enumerate}
        \item Si $f:V\to V'$ es una aplicación lineal y $\{w_1,\dots,w_k\}\subset V$ es un conjunto linealmente independiente, entonces las siguientes afirmaciones son equivalentes:
        \begin{enumerate}
            \item $\{f(w_1),\dots,f(w_k)\} \subset V'$ es linealmente independiente.

            \item $\cc{L}(\{w_1,\dots,w_k\})\cap Ker(f)=\{0\}$
        \end{enumerate}
        \begin{proof} Procedemos mediante doble implicación:
        \begin{description}
            \item [$\mathbf{a)\Longrightarrow b)}$]
            Partimos de que $\{w_1,\dots,w_k\}\subset V$ es linealmente independiente. Como además $\{f(w_1),\dots,f(w_k)\} \subset V'$ es linealmente independiente, tenemos que un conjunto linealmente independiente se aplica en otro linealmente independiente, de lo que deducimos que $f$ es un monomorfismo. Por tanto:
            \begin{equation*}
                f \text{ monomorfismo} \Longrightarrow Ker(f)=\{0\} \Longrightarrow \cc{L}(\{w_1,\dots,w_k\})\cap Ker(f)=\{0\}
            \end{equation*}
            quedando demostrada la primera implicación.

            \item [$\mathbf{b)\Longrightarrow a)}$] Partimos de que $\cc{L}(\{w_1,\dots,w_k\})\cap Ker(f)=\{0\}$ y buscamos demostrar que $\{f(w_1),\dots,f(w_k)\} \subset V'$ es linealmente independiente. Equivalentemente, demostraremos que $f$ es un monomorfismo.

            Sea $v\in \cc{L}(\{w_1,\dots,w_k\})\subseteq V, \quad v\neq 0$. Si fuese $f(v)=0$, llegamos a la siguiente contradicción:
            \begin{equation*}
                f(v)=0\Longrightarrow v\in Ker(f)\Longrightarrow v\in \{0\}\Longrightarrow v=0
            \end{equation*}
            pero es una contradicción, ya que $v\neq 0$. Por tanto, tenemos que $f(v)\neq 0$.

            Como $v\in \cc{L}(\{w_1,\dots,w_k\})$, es combinación lineal de los vectores del sistema generador. Por tanto,
            \begin{equation*}
                v=a_1w_1 + \dots + a_kw_k \hspace{2cm} a_i\in \bb{K}
            \end{equation*}
            Aplicando $f$, sabiendo que esta es una aplicación lineal, y que $f(v)\neq 0$; tenemos que:
            \begin{equation*}
                f(v)=f(a_1w_1 + \dots + a_kw_k) = a_1f(w_1) + \dots + a_kf(w_k) \neq 0
            \end{equation*}

            Por tanto, $f(v)=0\Longleftrightarrow v=0$. Por tanto, $Ker(f)=0$ y por tanto $f$ es un monomorfismo. Por tanto, queda demostrado que $\{f(w_1),\dots,f(w_k)\} \subset V'$ es linealmente independiente.
        \end{description}
        \end{proof}

        \item Si $V$ y $V'$ son finitamente generados y $f:V\to V'$ es una aplicación lineal, entonces las siguientes afirmaciones son equivalentes:
        \begin{enumerate}
            \item $f$ y $f^t$ son ambas inyectivas.
            \item $f$ es biyectiva.
        \end{enumerate}
        \begin{proof}
        Tenemos que:
        \begin{equation*}
            f^t:(V')^\ast \longrightarrow V^\ast
        \end{equation*}

        Además, adoptamos las siguientes notaciones:
        \begin{equation*}
            \dim_{\bb{K}}(V)=n \hspace{2cm} \dim_{\bb{K}}(V')=n'
        \end{equation*}

        Sean también $\cc{B}, \cc{B}'$ bases de $V,V'$ respectivamente y $\cc{B}^\ast, (\cc{B}')^\ast$ sus respectivas bases duales.
        
        Procedemos mediante doble implicación:
        \begin{description}
            \item [$\mathbf{a)\Longrightarrow b)}$]
            Partimos de que $f,f^t$ son inyectivas.
            \begin{equation*}\begin{split}
                f\text{ inyectiva} &\Longrightarrow rg(M(f,\cc{B}'\leftarrow \cc{B})) = n\\
                f^t\text{ inyectiva} &\Longrightarrow rg(M(f^t,\cc{B}^\ast\leftarrow (\cc{B}')^\ast)) = \dim_{\bb{K}}(V')^\ast = n'
            \end{split}\end{equation*}

            Por tanto, como $(M(f,\cc{B}'\leftarrow \cc{B}))^t = M(f^t,\cc{B}^\ast\leftarrow (\cc{B}')^\ast) \Longrightarrow n=n'$. Por tanto,
            \begin{equation*}
                rg(M(f,\cc{B}'\leftarrow \cc{B})) = \dim_{\bb{K}}(Im(f)) = n = n' = \dim_{\bb{K}}(V')
            \end{equation*}
            Por tanto, tenemos que $Im(f)=V'$ y, por tanto, $f$ es sobreyectiva. Como hemos supuesto que es inyectiva, tenemos que es biyectiva.

            \item [$\mathbf{b)\Longrightarrow a)}$] Partimos de que $f$ es biyectiva, por lo que tenemos de forma directa que $f$ es inyectiva.

            Como $f$ es sobreyectiva, tenemos que:
            \begin{equation*}
                \dim_{\bb{K}}(V') = \dim_{\bb{K}}(Im(f)) = \dim_{\bb{K}}(Im(f^t))
            \end{equation*}

            Aplicando las propiedades del espacio dual, tenemos que:
            \begin{equation*}
                \dim_{\bb{K}}(V')^\ast = \dim_{\bb{K}}(Im(f^t)) \Longrightarrow \dim_{\bb{K}}(Ker(f)) = 0 \Longrightarrow Ker(f^t)=\{0\}
            \end{equation*}

            por tanto, tenemos que $f^t$ es inyectiva.
        \end{description}
        \end{proof}

        \item Si $U \subset V$ es un subespacio vectorial y $\{w_1+U,\dots,w_k+U\}$ es un conjunto linealmente independiente en $V/U$, entonces $\{w_1,\dots,w_k\}$ es linealmente independiente en $V$.
        \begin{proof}
            Sea $0=a_1w_1 + \dots + a_kw_k\in V \qquad a_i\in \bb{K}$. 
            
            Veamos que $a_1=\dots = a_k=0$.
            \begin{multline*}
                0=a_1w_1 + \dots + a_kw_k \Longrightarrow\\
                \Longrightarrow
                0+U=(a_1w_1 + \dots + a_kw_k)+U = a_1(w_1+U)+\dots + a_k(w_k+U)
            \end{multline*}

            Como tenemos que $\{w_1+U,\dots,w_k+U\}$ es un conjunto linealmente independiente en $V/U$, tenemos que $a_1=\dots=a_k=0$. Por tanto, queda demostrado que $\{w_1,\dots,w_k\}$ es linealmente independiente en $V$.
        \end{proof}
    \end{enumerate}
\end{ejercicio}


\begin{ejercicio}
    \textbf{[2 puntos]} Sean $V$ y $V'$ espacios vectoriales finitamente generados sobre un mismo cuerpo~$K$. Demostrar que la aplicación transposición
    \begin{equation*}
    \begin{array}{rcl}
        ^t:Hom_{\bb{K}}(V,V') & \longrightarrow & Hom_{\bb{K}}((V')^\ast,V^\ast)\\
        f & \longmapsto & f^t
    \end{array}
    \end{equation*}
    es un isomorfismo de espacios vectoriales.

    \begin{proof}
        Suponemos que es una forma lineal (habría que demostrarlo).
        
        Veamos en primer lugar que es un monomorfismo.
        \begin{equation*}
            Ker(^t)=\{f\in Hom_{\bb{K}}(V,V')\mid f^t=c_0\}
        \end{equation*}
        para todo $f\in Ker(^t)$, se tiene que $f^t=0$. Por tanto,
        \begin{equation*}
            \forall \varphi'\in (V')^\ast, f^t(\varphi')=\varphi'\circ f=c_0 \Longrightarrow f=c_0 \Longrightarrow Ker(^t)=0 \Longrightarrow ^t \text{ inyectiva}
        \end{equation*}

        Veamos ahora que es un epiformismo, es decir, $Im(^t)=Hom_{\bb{K}}((V')^\ast,V^\ast)$.
        \begin{equation*}
            \dim_{\bb{K}}(Hom_{\bb{K}}(V,V')) = \cancelto{0}{\dim_{\bb{K}}(Ker(^t))} + \dim_{\bb{K}}(Im(^t)) = \dim_{\bb{K}}(Im(^t))
        \end{equation*}
        \begin{equation*}
            \dim_{\bb{K}}(Hom_{\bb{K}}(V,V')) = \dim_{\bb{K}}(V)\cdot \dim_{\bb{K}}(V') = \dim_{\bb{K}}(V^\ast)\cdot \dim_{\bb{K}}((V')^\ast) = \dim_{\bb{K}}(Hom_{\bb{K}}((V')^\ast,V^\ast))
        \end{equation*}

        Por tanto, tenemos que:
        \begin{equation*}
            \dim_{\bb{K}}(Hom_{\bb{K}}(V,V'))=\dim_{\bb{K}}(Hom_{\bb{K}}((V')^\ast,V^\ast)) = \dim_{\bb{K}}(Im(^t))
        \end{equation*}

        Como además tenemos que $Im(^t)\subseteq Hom_{\bb{K}}((V')^\ast,V^\ast)$, tenemos que $Im(^t) = Hom_{\bb{K}}((V')^\ast,V^\ast)$. Por tanto, $^t$ es un epiformismo.

        Por tanto, $^t$ es un isomorfismo.
    \end{proof}
\end{ejercicio}


\begin{ejercicio}
    \textbf{[4 puntos]} Dado $k\in \mathbb{R}$, se consideran los subespacios vectoriales de $\bb{R}^4$ siguientes:
    \begin{equation*}
    \begin{array}{rcl}
        U_k & = & \mathcal{L}\left( \left\{ (1,2,k,1),(k+1,4,2,2),(2,2,2-k,1) \right\}\right), \\
        V & = & \left\{(x,y,z,t)\in \mathbb{R}^4 \left|\begin{array}{c}
            y-x=0 \\
            t-x=0 
        \end{array}\right.\right\}.
    \end{array}
    \end{equation*}

    \begin{enumerate}
        \item Obtener una base y la dimensión de $U_k$ para todo $k\in \bb{R}$.

        Para hallar el número de vectores linealmente independientes del sistema generador y, por tanto la dimensión y la base del subespacio vectorial, calculamos el rango de la siguiente matriz:
        \begin{equation*}
            rg(A) = rg\left(\begin{array}{ccc}
                1 & k+1 & 2 \\
                2 & 4 & 2 \\
                k & 2 & 2-k \\
                1 & 2 & 1
            \end{array}\right)
        \end{equation*}

        Para facilitar los cálculos, realizamos la transformación elemental $F'_4 = 2F_4 - F_2$, que no cambia el rango.
        \begin{equation*}
            rg(A) = rg\left(\begin{array}{ccc}
                1 & k+1 & 2 \\
                2 & 4 & 2 \\
                k & 2 & 2-k \\
                0 & 0 & 0
            \end{array}\right)
        \end{equation*}

        Para estudiar el rango, vemos el valor del determinante:
        \begin{equation*}\begin{split}
            \left|\begin{array}{ccc}
                1 & k+1 & 2 \\
                2 & 4 & 2 \\
                k & 2 & 2-k
            \end{array}\right|
            &= 2\left|\begin{array}{ccc}
                1 & k+1 & 2 \\
                1 & 2 & 1 \\
                k & 2 & 2-k
            \end{array}\right| =\\
            &= 2[2(2-k) +k(k+1) +4 - 4k -2 -(k+1)(2-k)] = \\
            &= 2[4-2k +k^2 +k+4-4k-2-2k+k^2 -2 +k] =\\
            &= 2[2k^2 -6k +4] = 4[k^2 -3k +2] = 0 \Longleftrightarrow
            \left\{\begin{array}{c}
                k=2 \\ \lor \\ k=1 
            \end{array}\right.
        \end{split}\end{equation*}

        Realizamos por tanto la siguiente distinción de casos:
        \begin{itemize}
            \item \underline{Para $k=1,2$}:

            Tenemos que $rg(A)=2$, por lo que hay dos vectores linealmente independientes. Veamos cuáles son:
            \begin{equation*}
                \left|\begin{array}{cc}
                k+1 & 2 \\
                4 & 2 \\
            \end{array}\right|
            = 2(k+1) -8 = 2[k+1-4] = 2[k-3] = 0 \Longleftrightarrow k=3
            \end{equation*}

            Por tanto, tenemos que:
            \begin{gather*}
                U_k = \mathcal{L}\left( \left\{ (k+1,4,2,2),(2,2,2-k,1) \right\}\right) \\
                \dim_{\bb{R}}(U_k)=2
            \end{gather*}

            \item \underline{Para $k\neq 1,2$}:

            Tenemos que $rg(A)=3$, por lo que los tres vectores son linealmente independientes.
            \begin{gather*}
                U_k = \mathcal{L}\left( \left\{ (1,2,k,1),(k+1,4,2,2),(2,2,2-k,1) \right\}\right) \\
                \dim_{\bb{R}}(U_k)=3
            \end{gather*}
        \end{itemize}

        \item Calcular una base de $U_k + V$ y de $U_k \cap V$. ¿Existe $k\in \mathbb{R}$ tal que $\mathbb{R}^4= U_k \oplus V$?
        \begin{equation*}
            V = \left\{(x,y,z,t)\in \mathbb{R}^4 \left|\begin{array}{c}
                y-x=0 \\
                t-x=0 
            \end{array}\right.\right\}
            = \cc{L}\left(\{(0,0,1,0), (1,1,1,1)\}\right)
        \end{equation*}

        Tenemos que:
        \begin{equation*}
            U_k+V = \mathcal{L}\left( \left\{ (0,0,1,0),(1,1,1,1),(1,2,k,1),(k+1,4,2,2),(2,2,2-k,1) \right\}\right)
        \end{equation*}

        Veamos cuántos vectores son linealmente independientes:
        \begin{equation*}
            \left|\begin{array}{cccc}
                0 & 1 & 1 & 2 \\
                0 & 1 & 2 & 2 \\
                1 & 1 & k & 2-k \\
                0 & 1 & 1 & 1
            \end{array}\right|
            = \left|\begin{array}{ccc}
                1 & 1 & 2 \\
                1 & 2 & 2 \\
                1 & 1 & 1
            \end{array}\right| = 2+2+2-4-2-1=-1\neq 0
        \end{equation*}

        Por tanto, tenemos que esos 4 vectores son linealmente independientes.
        \begin{equation*}
            U_k+V = \mathcal{L}\left( \left\{ (0,0,1,0),(1,1,1,1),(1,2,k,1),(2,2,2-k,1) \right\}\right) = \bb{R}^4 \qquad \quad \forall k\in \bb{R}
        \end{equation*}


        Ahora procedemos a calcular $U_k\cap V$:
        \begin{multline*}
            \dim_{\bb{R}}(U_k\cap V) = \dim_{\bb{R}}(U_k) + \dim_{\bb{R}}(V) - \dim_{\bb{R}}(U_k+V) = \dim_{\bb{R}}(U_k)+2-4 =\\= \dim_{\bb{R}}(U_k)-4
        \end{multline*}

        Realizamos por tanto la siguiente distinción de casos:
        \begin{itemize}
            \item \underline{Para $k=1,2$}:

            \begin{equation*}
                \dim_{\bb{R}}(U_k)=2 \Longrightarrow
                \dim_{\bb{R}}(U_k\cap V)=0 \Longrightarrow U_k\cap V =\{0\}
            \end{equation*}

            \item \underline{Para $k\neq 1,2$}:

            \begin{equation*}
                \dim_{\bb{R}}(U_k)=3 \Longrightarrow
                \dim_{\bb{R}}(U_k\cap V)=1
            \end{equation*}

            Calculo unas ecuaciones implícitas de $U_k$:
            \begin{multline*}
                \left|\begin{array}{cccc}
                    1 & k+1 & 2 & x \\
                    2 & 4 & 2 & y \\
                    k & 2 & 2-k & z \\
                    1 & 2 & 1 & t
                \end{array}\right| = 0 =\\
                =-x\left|\begin{array}{ccc}
                    2 & 4 & 2 \\
                    k & 2 & 2-k \\
                    1 & 2 & 1
                \end{array}\right|
                +y\left|\begin{array}{ccc}
                    1 & k+1 & 2 \\
                    k & 2 & 2-k \\
                    1 & 2 & 1
                \end{array}\right|
                -z\left|\begin{array}{ccc}
                    1 & k+1 & 2 \\
                    2 & 4 & 2 \\
                    1 & 2 & 1
                \end{array}\right|
                +t\left|\begin{array}{ccc}
                    1 & k+1 & 2 \\
                    2 & 4 & 2 \\
                    k & 2 & 2-k
                \end{array}\right|
            \end{multline*}

            Calculo en primer lugar los determinantes:
            \begin{equation*}
                \left|\begin{array}{ccc}
                    2 & 4 & 2 \\
                    k & 2 & 2-k \\
                    1 & 2 & 1
                \end{array}\right| = 0 \qquad \qquad (F_1=2F_3)
            \end{equation*}
            \begin{equation*}
                \left|\begin{array}{ccc}
                    1 & k+1 & 2 \\
                    k & 2 & 2-k \\
                    1 & 2 & 1
                \end{array}\right|
                =-\left|\begin{array}{ccc}
                    1 & k+1 & 2 \\
                    1 & 2 & 1 \\
                    k & 2 & 2-k
                \end{array}\right| = -2[k^2 -3k +2] \qquad \text{(Ya calculado)}
            \end{equation*}
            \begin{equation*}
                \left|\begin{array}{ccc}
                    1 & k+1 & 2 \\
                    2 & 4 & 2 \\
                    1 & 2 & 1
                \end{array}\right| = 0 \qquad \qquad (F_2=2F_3)
            \end{equation*}
            \begin{equation*}
                \left|\begin{array}{ccc}
                    1 & k+1 & 2 \\
                    2 & 4 & 2 \\
                    k & 2 & 2-k
                \end{array}\right|
                = -2\left|\begin{array}{ccc}
                    1 & k+1 & 2 \\
                    k & 2 & 2-k \\
                    1 & 2 & 1
                \end{array}\right|
                = 4[k^2 -3k +2]
            \end{equation*}

            Por tanto, tenemos que la ecuación implícita es:
            \begin{equation*}
                0 = -0x -2y[k^2 -3k +2] -0z +4t[k^2 -3k +2] = 2(k^2 -3k +2)(-y+2t)
            \end{equation*}

            Como $k\neq 1,2\Longrightarrow k^2 -3k +2\neq 0$. Por tanto, las ecuaciones implícitas son:
            \begin{equation*}
                U_k = \left\{(x,y,z,t)\in \mathbb{R}^4 \left|\begin{array}{c}
                    -y+2t=0
                \end{array}\right.\right\}
            \end{equation*}

            Por tanto, el subespacio intersección es:
            \begin{equation*}
                U_k\cap V = \left\{(x,y,z,t)\in \mathbb{R}^4 \left|\begin{array}{c}
                    x=y\\ x=t \\ y=2t
                \end{array}\right.\right\}
                = \cc{L}(\{(0,0,1,0)\})
            \end{equation*}
        \end{itemize}
        

        Para ver si existe $k\in \mathbb{R}$ tal que $\mathbb{R}^4= U_k \oplus V$, necesitamos que su intersección sea nula. Por lo visto anteriormente, tenemos que:
        \begin{equation*}
            U_k\oplus V = \bb{R}^4 \Longleftrightarrow k=1,2
        \end{equation*}

        \item Para $k=1$, encontrar una aplicación lineal $f:\bb{R}^4\to \bb{R}^4$ tal que $Ker(f) = U_1,\;Im(f)=~V$ y $f \circ f = f$. Calcular $M(f, B_u)$, donde $B_u$ representa la base usual de $\mathbb{R}^4$.

        Tenemos que:
        \begin{equation*}
            V=Im(f)=\cc{L}(\{(0,0,1,0), (1,1,1,1)\})
        \end{equation*}
        \begin{equation*}
            U_1=Ker(f)=\cc{L}(\{(2,4,2,2), (2,2,1,1)\}) = \cc{L}(\{(1,2,1,1), (2,2,1,1)\})
        \end{equation*}

        Notamos la base usual $\cc{B}_u$ como $\cc{B}_u=\{e_1,e_2,e_3,e_4\}$.
        \begin{equation*}\begin{split}
            (1,2,1,1)\in Ker(f) & \Longrightarrow f(1,2,1,1)=f(e_1) + 2f(e_2) + f(e_3) + f(e_4)=0\\
            (2,2,1,1)\in Ker(f) & \Longrightarrow f(2,2,1,1)=2f(e_1) + 2f(e_2) + f(e_3) + f(e_4)=0\\
            (0,0,1,0)\in Im(f) & \Longrightarrow f(e_3)=e_3\\
            (1,1,1,1)\in Im(f) & \Longrightarrow f(e_1)+f(e_2)+f(e_3)+f(e_4)=(1,1,1,1)\\
        \end{split}\end{equation*}

        Por tanto, las ecuaciones son:
        \begin{equation*}
            \left\{\begin{array}{l}
                f(e_1) + 2f(e_2) + f(e_3) + f(e_4)=0\\
                2f(e_1) + 2f(e_2) + f(e_3) + f(e_4)=0 \\
                f(e_3)=e_3 \\
                f(e_1)+f(e_2)+f(e_3)+f(e_4)=(1,1,1,1)
            \end{array}\right\}\Longrightarrow
            \left\{\begin{array}{l}
                f(e_1)=0\\
                f(e_2)=(-1,-1,-1,-1) \\
                f(e_3)=e_3 \\
                f(e_4)=(2,2,1,2)
            \end{array}\right\}
        \end{equation*}

        Por tanto, tenemos que:
        \begin{equation*}
            A=M(f,\;\cc{B}_u)=\left(\begin{array}{cccc}
                0 & -1 & 0 & 2 \\
                0 & -1 & 0 & 2 \\
                0 & -1 & 1 & 1 \\
                0 & -1 & 0 & 2 \\
            \end{array}\right)
        \end{equation*}

        Como tenemos que $A^2=A$, efectivamente se cumple que $f\circ f=f$.

        \item Para la aplicación $f$ calculada en el apartado anterior, hallar $Im(f^t)$ y $ker(f^t)$.

        Sea la aplicación $f^t:(\bb{R}^4)^\ast\to (\bb{R}^4)^\ast$, y definimos la base dual de la usual como:
        \begin{equation*}
            (\cc{B}_u)^\ast = \{\varphi_1,\varphi_2,\varphi_3,\varphi_4\} \hspace{2cm} \varphi_i(a_1,a_2,a_3,a_4)=a_i\qquad \forall i=1,\dots,4
        \end{equation*}

        Por la propiedades de la aplicación lineal traspuesta, tenemos que:
        \begin{equation*}
            A^t=M(f^t,\;(\cc{B}_u)^\ast)=\left(\begin{array}{cccc}
                0 & 0 & 0 & 0 \\
                -1 & -1 & -1 & -1 \\
                0 & 0 & 1 & 0 \\
                2 & 2 & 1 & 2 \\
            \end{array}\right)
        \end{equation*}

        Por tanto, tenemos que:
        \begin{equation*}\begin{split}
            Im(f^t)&=\cc{L}\left(\{(0,-1,0,2)_{(\cc{B}_u)^\ast}, (0,-1,1,1)_{(\cc{B}_u)^\ast}\}\right)
        \end{split}\end{equation*}
        \begin{equation*}\begin{split}
            Ker(f^t)&=\left\{(a_1,a_2,a_3,a_4)_{(\cc{B}_u)^\ast}\in (\bb{R}^4)^\ast \left|\left(\begin{array}{cccc}
                0 & 0 & 0 & 0 \\
                -1 & -1 & -1 & -1 \\
                0 & 0 & 1 & 0 \\
                2 & 2 & 1 & 2 \\
            \end{array}\right)\left(\begin{array}{c}
                a_1\\a_2\\a_3\\a_4
            \end{array}\right)=0\right.\right\}=\\
            &=\left\{(a_1,a_2,a_3,a_4)_{(\cc{B}_u)^\ast}\in (\bb{R}^4)^\ast \left|\begin{array}{c}
                a_1+a_2+a_3+a_4=0
                a_3=0
            \end{array}\right.\right\} =\\
            &=\cc{L}\left(\{(1,-1,0,0)_{(\cc{B}_u)^\ast}, (1,0,0,-1)_{(\cc{B}_u)^\ast}\}\right)
        \end{split}\end{equation*}
        
    \end{enumerate}
\end{ejercicio}



\end{document}