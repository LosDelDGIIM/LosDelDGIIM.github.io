\documentclass[12pt]{article}

% Idioma y codificación
\usepackage[spanish, es-tabla]{babel}       %es-tabla para que se titule "Tabla"
\usepackage[utf8]{inputenc}

% Márgenes
\usepackage[a4paper,top=3cm,bottom=2.5cm,left=3cm,right=3cm]{geometry}

% Comentarios de bloque
\usepackage{verbatim}

% Paquetes de links
\usepackage[hidelinks]{hyperref}    % Permite enlaces
\usepackage{url}                    % redirecciona a la web

% Más opciones para enumeraciones
\usepackage{enumitem}

% Personalizar la portada
\usepackage{titling}

% Paquetes de tablas
\usepackage{multirow}


%------------------------------------------------------------------------

%Paquetes de figuras
\usepackage{caption}
\usepackage{subcaption} % Figuras al lado de otras
\usepackage{float}      % Poner figuras en el sitio indicado H.


% Paquetes de imágenes
\usepackage{graphicx}       % Paquete para añadir imágenes
\usepackage{transparent}    % Para manejar la opacidad de las figuras

% Paquete para usar colores
\usepackage[dvipsnames]{xcolor}
\usepackage{pagecolor}      % Para cambiar el color de la página

% Habilita tamaños de fuente mayores
\usepackage{fix-cm}

% Para los gráficos
\usepackage{tikz}

% Para poder situar los nodos en los grafos
\usetikzlibrary{positioning}


%------------------------------------------------------------------------

% Paquetes de matemáticas
\usepackage{mathtools, amsfonts, amssymb, mathrsfs}
\usepackage[makeroom]{cancel}     % Simplificar tachando
\usepackage{polynom}    % Divisiones y Ruffini
\usepackage{units} % Para poner fracciones diagonales con \nicefrac

\usepackage{pgfplots}   %Representar funciones
\pgfplotsset{compat=1.18}  % Versión 1.18

\usepackage{tikz-cd}    % Para usar diagramas de composiciones
\usetikzlibrary{calc}   % Para usar cálculo de coordenadas en tikz

%Definición de teoremas, etc.
\usepackage{amsthm}
%\swapnumbers   % Intercambia la posición del texto y de la numeración

\theoremstyle{plain}

\makeatletter
\@ifclassloaded{article}{
  \newtheorem{teo}{Teorema}[section]
}{
  \newtheorem{teo}{Teorema}[chapter]  % Se resetea en cada chapter
}
\makeatother

\newtheorem{coro}{Corolario}[teo]           % Se resetea en cada teorema
\newtheorem{prop}[teo]{Proposición}         % Usa el mismo contador que teorema
\newtheorem{lema}[teo]{Lema}                % Usa el mismo contador que teorema

\theoremstyle{remark}
\newtheorem*{observacion}{Observación}

\theoremstyle{definition}

\makeatletter
\@ifclassloaded{article}{
  \newtheorem{definicion}{Definición} [section]     % Se resetea en cada chapter
}{
  \newtheorem{definicion}{Definición} [chapter]     % Se resetea en cada chapter
}
\makeatother

\newtheorem*{notacion}{Notación}
\newtheorem*{ejemplo}{Ejemplo}
\newtheorem*{ejercicio*}{Ejercicio}             % No numerado
\newtheorem{ejercicio}{Ejercicio} [section]     % Se resetea en cada section


% Modificar el formato de la numeración del teorema "ejercicio"
\renewcommand{\theejercicio}{%
  \ifnum\value{section}=0 % Si no se ha iniciado ninguna sección
    \arabic{ejercicio}% Solo mostrar el número de ejercicio
  \else
    \thesection.\arabic{ejercicio}% Mostrar número de sección y número de ejercicio
  \fi
}


% \renewcommand\qedsymbol{$\blacksquare$}         % Cambiar símbolo QED
%------------------------------------------------------------------------

% Paquetes para encabezados
\usepackage{fancyhdr}
\pagestyle{fancy}
\fancyhf{}

\newcommand{\helv}{ % Modificación tamaño de letra
\fontfamily{}\fontsize{12}{12}\selectfont}
\setlength{\headheight}{15pt} % Amplía el tamaño del índice


%\usepackage{lastpage}   % Referenciar última pag   \pageref{LastPage}
\fancyfoot[C]{\thepage}

%------------------------------------------------------------------------

% Conseguir que no ponga "Capítulo 1". Sino solo "1."
\makeatletter
\@ifclassloaded{book}{
  \renewcommand{\chaptermark}[1]{\markboth{\thechapter.\ #1}{}} % En el encabezado
    
  \renewcommand{\@makechapterhead}[1]{%
  \vspace*{50\p@}%
  {\parindent \z@ \raggedright \normalfont
    \ifnum \c@secnumdepth >\m@ne
      \huge\bfseries \thechapter.\hspace{1em}\ignorespaces
    \fi
    \interlinepenalty\@M
    \Huge \bfseries #1\par\nobreak
    \vskip 40\p@
  }}
}
\makeatother

%------------------------------------------------------------------------
% Paquetes de cógido
\usepackage{minted}
\renewcommand\listingscaption{Código fuente}

\usepackage{fancyvrb}
% Personaliza el tamaño de los números de línea
\renewcommand{\theFancyVerbLine}{\small\arabic{FancyVerbLine}}

% Estilo para C++
\newminted{cpp}{
    frame=lines,
    framesep=2mm,
    baselinestretch=1.2,
    linenos,
    escapeinside=||
}

% para minted
\definecolor{LightGray}{rgb}{0.95,0.95,0.92}
\setminted{
    linenos=true,
    stepnumber=5,
    numberfirstline=true,
    autogobble,
    breaklines=true,
    breakautoindent=true,
    breaksymbolleft=,
    breaksymbolright=,
    breaksymbolindentleft=0pt,
    breaksymbolindentright=0pt,
    breaksymbolsepleft=0pt,
    breaksymbolsepright=0pt,
    fontsize=\footnotesize,
    bgcolor=LightGray,
    numbersep=10pt
}


\usepackage{listings} % Para incluir código desde un archivo

\renewcommand\lstlistingname{Código Fuente}
\renewcommand\lstlistlistingname{Índice de Códigos Fuente}

% Definir colores
\definecolor{vscodepurple}{rgb}{0.5,0,0.5}
\definecolor{vscodeblue}{rgb}{0,0,0.8}
\definecolor{vscodegreen}{rgb}{0,0.5,0}
\definecolor{vscodegray}{rgb}{0.5,0.5,0.5}
\definecolor{vscodebackground}{rgb}{0.97,0.97,0.97}
\definecolor{vscodelightgray}{rgb}{0.9,0.9,0.9}

% Configuración para el estilo de C similar a VSCode
\lstdefinestyle{vscode_C}{
  backgroundcolor=\color{vscodebackground},
  commentstyle=\color{vscodegreen},
  keywordstyle=\color{vscodeblue},
  numberstyle=\tiny\color{vscodegray},
  stringstyle=\color{vscodepurple},
  basicstyle=\scriptsize\ttfamily,
  breakatwhitespace=false,
  breaklines=true,
  captionpos=b,
  keepspaces=true,
  numbers=left,
  numbersep=5pt,
  showspaces=false,
  showstringspaces=false,
  showtabs=false,
  tabsize=2,
  frame=tb,
  framerule=0pt,
  aboveskip=10pt,
  belowskip=10pt,
  xleftmargin=10pt,
  xrightmargin=10pt,
  framexleftmargin=10pt,
  framexrightmargin=10pt,
  framesep=0pt,
  rulecolor=\color{vscodelightgray},
  backgroundcolor=\color{vscodebackground},
}

%------------------------------------------------------------------------

% Comandos definidos
\newcommand{\bb}[1]{\mathbb{#1}}
\newcommand{\cc}[1]{\mathcal{#1}}

% I prefer the slanted \leq
\let\oldleq\leq % save them in case they're every wanted
\let\oldgeq\geq
\renewcommand{\leq}{\leqslant}
\renewcommand{\geq}{\geqslant}

% Si y solo si
\newcommand{\sii}{\iff}

% Letras griegas
\newcommand{\eps}{\epsilon}
\newcommand{\veps}{\varepsilon}
\newcommand{\lm}{\lambda}

\newcommand{\ol}{\overline}
\newcommand{\ul}{\underline}
\newcommand{\wt}{\widetilde}
\newcommand{\wh}{\widehat}

\let\oldvec\vec
\renewcommand{\vec}{\overrightarrow}

% Derivadas parciales
\newcommand{\del}[2]{\frac{\partial #1}{\partial #2}}
\newcommand{\Del}[3]{\frac{\partial^{#1} #2}{\partial #3^{#1}}}
\newcommand{\deld}[2]{\dfrac{\partial #1}{\partial #2}}
\newcommand{\Deld}[3]{\dfrac{\partial^{#1} #2}{\partial #3^{#1}}}


\newcommand{\AstIg}{\stackrel{(\ast)}{=}}
\newcommand{\Hop}{\stackrel{L'H\hat{o}pital}{=}}

\newcommand{\red}[1]{{\color{red}#1}} % Para integrales, destacar los cambios.

% Método de integración
\newcommand{\MetInt}[2]{
    \left[\begin{array}{c}
        #1 \\ #2
    \end{array}\right]
}

% Declarar aplicaciones
% 1. Nombre aplicación
% 2. Dominio
% 3. Codominio
% 4. Variable
% 5. Imagen de la variable
\newcommand{\Func}[5]{
    \begin{equation*}
        \begin{array}{rrll}
            #1:& #2 & \longrightarrow & #3\\
               & #4 & \longmapsto & #5
        \end{array}
    \end{equation*}
}

%------------------------------------------------------------------------


\begin{document}

% 1. Foto de fondo
% 2. Título
% 3. Encabezado Izquierdo
% 4. Color de fondo
% 5. Coord x del titulo
% 6. Coord y del titulo
% 7. Fecha
\newcommand{\F}{{\mathcal{F}}} % Funcion
\newcommand{\R}{{\mathbb{R}}}  % Reales
\newcommand{\Q}{{\mathbb{Q}}}  % Racionales
\newcommand{\Z}{{\mathbb{Z}}}  % Enteros
\newcommand{\N}{{\mathbb{N}}}  % Naturales
\newcommand{\C}{{\mathbb{C}}}  % Complejos
\newcommand{\U}{{\mathcal{U}}} % Unidades
\hbadness=10000

% 1. Foto de fondo
% 2. Título
% 3. Encabezado Izquierdo
% 4. Color de fondo
% 5. Coord x del titulo
% 6. Coord y del titulo
% 7. Fecha

\newcommand{\portada}[7]{

    \portadaBase{#1}{#2}{#3}{#4}{#5}{#6}{#7}
    \portadaBook{#1}{#2}{#3}{#4}{#5}{#6}{#7}
}

\newcommand{\portadaExamen}[7]{

    \portadaBase{#1}{#2}{#3}{#4}{#5}{#6}{#7}
    \portadaArticle{#1}{#2}{#3}{#4}{#5}{#6}{#7}
}




\newcommand{\portadaBase}[7]{

    % Tiene la portada principal y la licencia Creative Commons
    
    % 1. Foto de fondo
    % 2. Título
    % 3. Encabezado Izquierdo
    % 4. Color de fondo
    % 5. Coord x del titulo
    % 6. Coord y del titulo
    % 7. Fecha
    
    
    \thispagestyle{empty}               % Sin encabezado ni pie de página
    \newgeometry{margin=0cm}        % Márgenes nulos para la primera página
    
    
    % Encabezado
    \fancyhead[L]{\helv #3}
    \fancyhead[R]{\helv \nouppercase{\leftmark}}
    
    
    \pagecolor{#4}        % Color de fondo para la portada
    
    \begin{figure}[p]
        \centering
        \transparent{0.3}           % Opacidad del 30% para la imagen
        
        \includegraphics[width=\paperwidth, keepaspectratio]{assets/#1}
    
        \begin{tikzpicture}[remember picture, overlay]
            \node[anchor=north west, text=white, opacity=1, font=\fontsize{60}{90}\selectfont\bfseries\sffamily, align=left] at (#5, #6) {#2};
            
            \node[anchor=south east, text=white, opacity=1, font=\fontsize{12}{18}\selectfont\sffamily, align=right] at (9.7, 3) {\textbf{\href{https://losdeldgiim.github.io/}{Los Del DGIIM}}};
            
            \node[anchor=south east, text=white, opacity=1, font=\fontsize{12}{15}\selectfont\sffamily, align=right] at (9.7, 1.8) {Doble Grado en Ingeniería Informática y Matemáticas\\Universidad de Granada};
        \end{tikzpicture}
    \end{figure}
    
    
    \restoregeometry        % Restaurar márgenes normales para las páginas subsiguientes
    \pagecolor{white}       % Restaurar el color de página
    
    
    \newpage
    \thispagestyle{empty}               % Sin encabezado ni pie de página
    \begin{tikzpicture}[remember picture, overlay]
        \node[anchor=south west, inner sep=3cm] at (current page.south west) {
            \begin{minipage}{0.5\paperwidth}
                \href{https://creativecommons.org/licenses/by-nc-nd/4.0/}{
                    \includegraphics[height=2cm]{assets/Licencia.png}
                }\vspace{1cm}\\
                Esta obra está bajo una
                \href{https://creativecommons.org/licenses/by-nc-nd/4.0/}{
                    Licencia Creative Commons Atribución-NoComercial-SinDerivadas 4.0 Internacional (CC BY-NC-ND 4.0).
                }\\
    
                Eres libre de compartir y redistribuir el contenido de esta obra en cualquier medio o formato, siempre y cuando des el crédito adecuado a los autores originales y no persigas fines comerciales. 
            \end{minipage}
        };
    \end{tikzpicture}
    
    
    
    % 1. Foto de fondo
    % 2. Título
    % 3. Encabezado Izquierdo
    % 4. Color de fondo
    % 5. Coord x del titulo
    % 6. Coord y del titulo
    % 7. Fecha


}


\newcommand{\portadaBook}[7]{

    % 1. Foto de fondo
    % 2. Título
    % 3. Encabezado Izquierdo
    % 4. Color de fondo
    % 5. Coord x del titulo
    % 6. Coord y del titulo
    % 7. Fecha

    % Personaliza el formato del título
    \pretitle{\begin{center}\bfseries\fontsize{42}{56}\selectfont}
    \posttitle{\par\end{center}\vspace{2em}}
    
    % Personaliza el formato del autor
    \preauthor{\begin{center}\Large}
    \postauthor{\par\end{center}\vfill}
    
    % Personaliza el formato de la fecha
    \predate{\begin{center}\huge}
    \postdate{\par\end{center}\vspace{2em}}
    
    \title{#2}
    \author{\href{https://losdeldgiim.github.io/}{Los Del DGIIM}}
    \date{Granada, #7}
    \maketitle
    
    \tableofcontents
}




\newcommand{\portadaArticle}[7]{

    % 1. Foto de fondo
    % 2. Título
    % 3. Encabezado Izquierdo
    % 4. Color de fondo
    % 5. Coord x del titulo
    % 6. Coord y del titulo
    % 7. Fecha

    % Personaliza el formato del título
    \pretitle{\begin{center}\bfseries\fontsize{42}{56}\selectfont}
    \posttitle{\par\end{center}\vspace{2em}}
    
    % Personaliza el formato del autor
    \preauthor{\begin{center}\Large}
    \postauthor{\par\end{center}\vspace{3em}}
    
    % Personaliza el formato de la fecha
    \predate{\begin{center}\huge}
    \postdate{\par\end{center}\vspace{5em}}
    
    \title{#2}
    \author{\href{https://losdeldgiim.github.io/}{Los Del DGIIM}}
    \date{Granada, #7}
    \thispagestyle{empty}               % Sin encabezado ni pie de página
    \maketitle
    \vfill
}
\portadaExamen{ffccA4.jpg}{Geometría I\\Examen X}{Geometría I. Examen X}{MidnightBlue}{-8}{28}{2023-2024}{Miguel Ángel De la Vega Rodríguez\\Jesús Muñoz Velasco}

\begin{description}
	\item[Asignatura] Geometría I.
	\item[Curso Académico] 2022-23.
	\item[Grado] Doble Grado en Ingeniería Informática y Matemáticas.
	\item[Grupo] Único.
	\item[Profesor] Juan de Dios Pérez Jiménez.
	\item[Descripción] Convocatoria Extraordinaria\footnote{El examen lo pone el departamento.}.
	\item[Fecha] 17 de febrero de 2023.
	\item[Duración] 3 horas.

\end{description}
\newpage

\begin{ejercicio}[2,5 puntos]
	Enuncia y demuestra el Teorema de Reflexividad.
\end{ejercicio}
\begin{proof}
	La aplicación $\Phi$ está definida consistentemente. Para demostrar que es lineal, debemos comprobar que $\Phi_{av+bw} = \Phi(a v + b w)$ es igual a $a\Phi_v + b\Phi_w$ para todo $v, w \in V$, $a, b \in K$. Para ello, aplicamos ambos a una forma lineal genérica $\varphi \in V^*$:
	\begin{align*}
		\Phi_{av+bw}(\varphi)          & = \varphi(a v + b w) = a \varphi(v) + b \varphi(w)                \\
		(a\Phi_v + b \Phi_w )(\varphi) & = a\Phi_v(\varphi) + b\Phi_w(\varphi) = a\varphi(v) + b\varphi(w)
	\end{align*}
	comprobándose que son iguales. Para demostrar la biyectividad de $\Phi$, basta con comprobar su inyectividad (al tener $V$ y $V^{**}$ la misma dimensión finita),
	esto es, Nuc$(\Phi) = \{0\}$. Sea $v \in V $ tal que $\Phi_v$ es la forma nula del bidual. Esto quiere decir $\Phi_v(\varphi) = 0$
	, esto es, $\varphi(v) = 0$ para todo $\varphi \in V^*$. lo que implica que $v = 0$.
\end{proof}

\begin{ejercicio}[2 puntos]
	Sea $V(\bb{R})$ un espacio vectorial de dimensión $n\in \bb{N}$ y $f$ un endomorfismo suyo que verifica $f\circ f=-I_V$ ($I_V$ es la aplicación identidad en $V$).
	Demuestra que $f$ es un automorfismo y que $n$ no puede ser impar.
\end{ejercicio}
\begin{itemize}
	\item Biyectividad de \( f \): Para ello, vamos a demostrar que \( \text{Ker}(f) = \{ 0 \} \). \\ \\
	      Sea \( v \in \text{Ker}(f) \), entonces \( f(v) = 0 \). Por ser $f$ lineal $(f \circ f)(v) = 0$, Pero sabemos que
	      $(f \circ f)(v) = -v$, luego $-v = 0 \Rightarrow v=0$ . Por tanto, $f$ es inyectiva.
	\item \( n \) no puede ser impar: \\ \\
	      Para ello, observamos lo siguiente:
	      \begin{equation*}
		      \det(f \circ f) = \det(-I_V) \Rightarrow \det(f) \cdot \det(f) = (-1)^n \cdot 1 \leftrightarrow (\det(f))^2 = (-1)^n
	      \end{equation*}
	      Que para $n$ impar es imposible, por tanto, $n$ no puede ser impar.
\end{itemize}

\begin{ejercicio}[5,5 puntos]
	Se consideran los espacios vectoriales $S_2(\bb{R})$ y $\bb{R}_2[x]$.
	\begin{enumerate}
		\item \textbf{[3 Puntos]} Construye una aplicación lineal \( f : S_2(\mathbb{R}) \rightarrow \mathbb{R}_2[x] \) que verifique:
		      \begin{equation*}
			      \begin{aligned}
				      \text{ker}(f)             & = \{ A \in S_2(\mathbb{R}) : \text{traza}(A) = 0 \} \text{ y ker}(f^t) = L\{ \phi, \psi \} \text{ donde} \\
				      \phi(a_0 + a_1x + a_2x^2) & = a_0 - a_2, \quad \psi(a_0 + a_1x + a_2x^2) = a_1 - a_2, \quad \forall a_0, a_1, a_2 \in \mathbb{R}.
			      \end{aligned}
		      \end{equation*}
		      Determina explícitamente \( f \begin{pmatrix}
			      a & b \\
			      b & c \\
		      \end{pmatrix} \), \(\forall a, b, c \in \mathbb{R} \). \\ \\
		      En primer lugar, observamos que:
		      \begin{equation*}
			      f\left(\begin{pmatrix}
				      a & b  \\
				      b & -a \\
			      \end{pmatrix}\right) = 0
		      \end{equation*}
		      También sabemos que ker$(f^t) = L\{ \phi, \psi \}$ = an(Im$(f))$, es decir,
		      \begin{equation*}
			      \text{Im}(f) = \left\{ a_0 + a_1x + a_2x^2 : \begin{array}{l}
				      a_0 - a_2 = 0 \\
				      a_1 - a_2 = 0
			      \end{array} \right\} = \mathcal{L}\left\{ (1,1,1)\right\}
		      \end{equation*}
		      De donde deducimos que $\dim\text{Ker}(f) = 2$, vamos a construir una base de Ker$(f)$:
		      \begin{equation*}
			      \text{Ker}(f) = \mathcal{L}\left\{\begin{pmatrix}
				      1 & 0  \\
				      0 & -1 \\
			      \end{pmatrix},\begin{pmatrix}
				      0 & 1 \\
				      1 & 0 \\
			      \end{pmatrix}\right\}
		      \end{equation*}
		      Ahora, para construir la aplicación lineal $f$, basta con calcular la imagen de cada uno de los vectores de la base de $S_2(\R)$:
		      \begin{align*}
			      f\left(\begin{pmatrix}
				             1 & 0  \\
				             0 & -1 \\
			             \end{pmatrix}\right) & = 0           \\
			      f\left(\begin{pmatrix}
				             0 & 1 \\
				             1 & 0 \\
			             \end{pmatrix}\right) & = 0           \\
			      f\left(\begin{pmatrix}
				             1 & 0 \\
				             0 & 0 \\
			             \end{pmatrix}\right) & = 1 + x + x^2
		      \end{align*}
		      De donde se obtine:
		      \begin{align*}
			      f(E_{11}) & = 1 + x + x^2                                                             \\
			      f(E_{12}) & = 0                                                                       \\
			      f(E_{21}) & = 0                                                                       \\
			      f(E_{22}) & = f\left(\begin{pmatrix}
					                           1 & 0 \\
					                           0 & 0 \\
				                           \end{pmatrix}\right) - f\left(\begin{pmatrix}
					                                                         1 & 0  \\
					                                                         0 & -1 \\
				                                                         \end{pmatrix}\right) = 1 + x + x^2
		      \end{align*}
		      Por tanto, la aplicación lineal $f$ es:
		      \begin{equation*}
			      f\left(\begin{pmatrix}
				      a & b \\
				      b & c \\
			      \end{pmatrix}\right) = a(1+x+x^2) + c(1+x+x^2) = (a+c)(1 + x + x^2)
		      \end{equation*}
		\item \textbf{[1 Punto]} \textit{Construye, si es posible, un endomorfismo \( h \) de \( S_2(\mathbb{R}) \) distinto del endomorfismo nulo tal que \( f \circ h \) sea la aplicación lineal nula.}
		      \\ \\
		      Usando el apartado anterior, sabemos que $$f\left(\begin{pmatrix}
					      a & b \\
					      b & c \\
				      \end{pmatrix}\right) = (a+c)(1 + x + x^2), \text{ por tanto, tomando } h\left(\begin{pmatrix}
					      a & b \\
					      b & c \\
				      \end{pmatrix}\right) = \begin{pmatrix}
				      a & b  \\
				      b & -a \\
			      \end{pmatrix}
		      $$
		      un endomorfismo no nulo, con $a,b,c \in \R$. Tendríamos que $f\circ h = 0$.
		      Veamos que $h$ es un endomorfismo, para ello, comprobaremos que es lineal:
		      \begin{itemize}
			      \item Aditividad:
			            \begin{align*}
				            h\left(\begin{pmatrix}
					                   a & b \\
					                   b & c \\
				                   \end{pmatrix} + \begin{pmatrix}
					                                   a' & b' \\
					                                   b' & c' \\
				                                   \end{pmatrix}\right) & = h\left(\begin{pmatrix}
					                                                                   a+a' & b+b' \\
					                                                                   b+b' & c+c' \\
				                                                                   \end{pmatrix}\right) = \begin{pmatrix}
					                                                                                          a+a' & b+b'    \\
					                                                                                          b+b' & -(a+a') \\
				                                                                                          \end{pmatrix}                                  \\
				                                            & = \begin{pmatrix}
					                                                a & b  \\
					                                                b & -a \\
				                                                \end{pmatrix} + \begin{pmatrix}
					                                                                a' & b'  \\
					                                                                b' & -a' \\
				                                                                \end{pmatrix} = h\left(\begin{pmatrix}
					                                                                                       a & b \\
					                                                                                       b & c \\
				                                                                                       \end{pmatrix}\right) + h\left(\begin{pmatrix}
					                                                                                                                     a' & b' \\
					                                                                                                                     b' & c' \\
				                                                                                                                     \end{pmatrix}\right)
			            \end{align*}
			      \item Homogeneidad:
			            \begin{align*}
				            h\left(\lambda \begin{pmatrix}
					                           a & b \\
					                           b & c \\
				                           \end{pmatrix}\right) & = h\left(\begin{pmatrix}
					                                                           \lambda a & \lambda b \\
					                                                           \lambda b & \lambda c \\
				                                                           \end{pmatrix}\right) = \begin{pmatrix}
					                                                                                  \lambda a & \lambda b  \\
					                                                                                  \lambda b & -\lambda a \\
				                                                                                  \end{pmatrix}
				            = \lambda \begin{pmatrix}
					                      a & b  \\
					                      b & -a \\
				                      \end{pmatrix} = \lambda h\left(\begin{pmatrix}
					                                                     a & b \\
					                                                     b & c \\
				                                                     \end{pmatrix}\right)
			            \end{align*}
		      \end{itemize}
		      Así, dada una matriz $A \in S_2(\R)$, $f \circ h(A) = f(h(A)) = f\left(\begin{pmatrix}
					      a & b  \\
					      b & -a \\
				      \end{pmatrix}\right) = (a-a)(1+x+x^2) = 0$.
		\item \textbf{[1,5 Puntos]} \textit{Calcula una base del espacio cociente \( \mathbb{R}_2[x] / \text{Im}(f) \) y determina si este espacio es isomorfo a \( S_2(\mathbb{R})/\text{ker}(f) \).}
		      \\ \\
		      En primer lugar, comenzamos observando que $\dim{\R_2[x] / \text{Im}(f)} = \dim{\R_2[x]} - \dim{\text{Im}(f)} = 3 - 1 = 2$.
		      y por los apartados anteriores, $B_{\text{Im}(f)} = \{1+x+x^2\}$. Ahora, ampliamos esta base a una base $B'$ de $\R_2[x]$:
		      \begin{equation*}
			      B' = \{1+x+x^2,x,x^2\}
		      \end{equation*}
		      Donde se tiene que $W = \mathcal{L}\{x,x^2\}$ es suma directa con $\text{Im}(f)$, por tanto, $\R_2[x] = \text{Im}(f) \oplus W$.
		      Y por esto, podemos tomar $B = \{x + \text{Im}(f), x^2 +\text{Im}(f)\}$ como base de $\R_2[x] / \text{Im}(f)$.
		      Por otro lado, $\dim{S_2(\R) / \text{ker}(f)} = \dim{S_2(\R)} - \dim{\text{ker}(f)} = 3 - 2 = 1$, por tanto, no pueden ser isomorfos.
	\end{enumerate}
\end{ejercicio}



\end{document}
