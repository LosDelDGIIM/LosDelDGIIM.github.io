\documentclass[12pt]{article}

% Idioma y codificación
\usepackage[spanish, es-tabla]{babel}       %es-tabla para que se titule "Tabla"
\usepackage[utf8]{inputenc}

% Márgenes
\usepackage[a4paper,top=3cm,bottom=2.5cm,left=3cm,right=3cm]{geometry}

% Comentarios de bloque
\usepackage{verbatim}

% Paquetes de links
\usepackage[hidelinks]{hyperref}    % Permite enlaces
\usepackage{url}                    % redirecciona a la web

% Más opciones para enumeraciones
\usepackage{enumitem}

% Personalizar la portada
\usepackage{titling}

% Paquetes de tablas
\usepackage{multirow}


%------------------------------------------------------------------------

%Paquetes de figuras
\usepackage{caption}
\usepackage{subcaption} % Figuras al lado de otras
\usepackage{float}      % Poner figuras en el sitio indicado H.


% Paquetes de imágenes
\usepackage{graphicx}       % Paquete para añadir imágenes
\usepackage{transparent}    % Para manejar la opacidad de las figuras

% Paquete para usar colores
\usepackage[dvipsnames]{xcolor}
\usepackage{pagecolor}      % Para cambiar el color de la página

% Habilita tamaños de fuente mayores
\usepackage{fix-cm}

% Para los gráficos
\usepackage{tikz}

% Para poder situar los nodos en los grafos
\usetikzlibrary{positioning}


%------------------------------------------------------------------------

% Paquetes de matemáticas
\usepackage{mathtools, amsfonts, amssymb, mathrsfs}
\usepackage[makeroom]{cancel}     % Simplificar tachando
\usepackage{polynom}    % Divisiones y Ruffini
\usepackage{units} % Para poner fracciones diagonales con \nicefrac

\usepackage{pgfplots}   %Representar funciones
\pgfplotsset{compat=1.18}  % Versión 1.18

\usepackage{tikz-cd}    % Para usar diagramas de composiciones
\usetikzlibrary{calc}   % Para usar cálculo de coordenadas en tikz

%Definición de teoremas, etc.
\usepackage{amsthm}
%\swapnumbers   % Intercambia la posición del texto y de la numeración

\theoremstyle{plain}

\makeatletter
\@ifclassloaded{article}{
  \newtheorem{teo}{Teorema}[section]
}{
  \newtheorem{teo}{Teorema}[chapter]  % Se resetea en cada chapter
}
\makeatother

\newtheorem{coro}{Corolario}[teo]           % Se resetea en cada teorema
\newtheorem{prop}[teo]{Proposición}         % Usa el mismo contador que teorema
\newtheorem{lema}[teo]{Lema}                % Usa el mismo contador que teorema

\theoremstyle{remark}
\newtheorem*{observacion}{Observación}

\theoremstyle{definition}

\makeatletter
\@ifclassloaded{article}{
  \newtheorem{definicion}{Definición} [section]     % Se resetea en cada chapter
}{
  \newtheorem{definicion}{Definición} [chapter]     % Se resetea en cada chapter
}
\makeatother

\newtheorem*{notacion}{Notación}
\newtheorem*{ejemplo}{Ejemplo}
\newtheorem*{ejercicio*}{Ejercicio}             % No numerado
\newtheorem{ejercicio}{Ejercicio} [section]     % Se resetea en cada section


% Modificar el formato de la numeración del teorema "ejercicio"
\renewcommand{\theejercicio}{%
  \ifnum\value{section}=0 % Si no se ha iniciado ninguna sección
    \arabic{ejercicio}% Solo mostrar el número de ejercicio
  \else
    \thesection.\arabic{ejercicio}% Mostrar número de sección y número de ejercicio
  \fi
}


% \renewcommand\qedsymbol{$\blacksquare$}         % Cambiar símbolo QED
%------------------------------------------------------------------------

% Paquetes para encabezados
\usepackage{fancyhdr}
\pagestyle{fancy}
\fancyhf{}

\newcommand{\helv}{ % Modificación tamaño de letra
\fontfamily{}\fontsize{12}{12}\selectfont}
\setlength{\headheight}{15pt} % Amplía el tamaño del índice


%\usepackage{lastpage}   % Referenciar última pag   \pageref{LastPage}
\fancyfoot[C]{\thepage}

%------------------------------------------------------------------------

% Conseguir que no ponga "Capítulo 1". Sino solo "1."
\makeatletter
\@ifclassloaded{book}{
  \renewcommand{\chaptermark}[1]{\markboth{\thechapter.\ #1}{}} % En el encabezado
    
  \renewcommand{\@makechapterhead}[1]{%
  \vspace*{50\p@}%
  {\parindent \z@ \raggedright \normalfont
    \ifnum \c@secnumdepth >\m@ne
      \huge\bfseries \thechapter.\hspace{1em}\ignorespaces
    \fi
    \interlinepenalty\@M
    \Huge \bfseries #1\par\nobreak
    \vskip 40\p@
  }}
}
\makeatother

%------------------------------------------------------------------------
% Paquetes de cógido
\usepackage{minted}
\renewcommand\listingscaption{Código fuente}

\usepackage{fancyvrb}
% Personaliza el tamaño de los números de línea
\renewcommand{\theFancyVerbLine}{\small\arabic{FancyVerbLine}}

% Estilo para C++
\newminted{cpp}{
    frame=lines,
    framesep=2mm,
    baselinestretch=1.2,
    linenos,
    escapeinside=||
}

% para minted
\definecolor{LightGray}{rgb}{0.95,0.95,0.92}
\setminted{
    linenos=true,
    stepnumber=5,
    numberfirstline=true,
    autogobble,
    breaklines=true,
    breakautoindent=true,
    breaksymbolleft=,
    breaksymbolright=,
    breaksymbolindentleft=0pt,
    breaksymbolindentright=0pt,
    breaksymbolsepleft=0pt,
    breaksymbolsepright=0pt,
    fontsize=\footnotesize,
    bgcolor=LightGray,
    numbersep=10pt
}


\usepackage{listings} % Para incluir código desde un archivo

\renewcommand\lstlistingname{Código Fuente}
\renewcommand\lstlistlistingname{Índice de Códigos Fuente}

% Definir colores
\definecolor{vscodepurple}{rgb}{0.5,0,0.5}
\definecolor{vscodeblue}{rgb}{0,0,0.8}
\definecolor{vscodegreen}{rgb}{0,0.5,0}
\definecolor{vscodegray}{rgb}{0.5,0.5,0.5}
\definecolor{vscodebackground}{rgb}{0.97,0.97,0.97}
\definecolor{vscodelightgray}{rgb}{0.9,0.9,0.9}

% Configuración para el estilo de C similar a VSCode
\lstdefinestyle{vscode_C}{
  backgroundcolor=\color{vscodebackground},
  commentstyle=\color{vscodegreen},
  keywordstyle=\color{vscodeblue},
  numberstyle=\tiny\color{vscodegray},
  stringstyle=\color{vscodepurple},
  basicstyle=\scriptsize\ttfamily,
  breakatwhitespace=false,
  breaklines=true,
  captionpos=b,
  keepspaces=true,
  numbers=left,
  numbersep=5pt,
  showspaces=false,
  showstringspaces=false,
  showtabs=false,
  tabsize=2,
  frame=tb,
  framerule=0pt,
  aboveskip=10pt,
  belowskip=10pt,
  xleftmargin=10pt,
  xrightmargin=10pt,
  framexleftmargin=10pt,
  framexrightmargin=10pt,
  framesep=0pt,
  rulecolor=\color{vscodelightgray},
  backgroundcolor=\color{vscodebackground},
}

%------------------------------------------------------------------------

% Comandos definidos
\newcommand{\bb}[1]{\mathbb{#1}}
\newcommand{\cc}[1]{\mathcal{#1}}

% I prefer the slanted \leq
\let\oldleq\leq % save them in case they're every wanted
\let\oldgeq\geq
\renewcommand{\leq}{\leqslant}
\renewcommand{\geq}{\geqslant}

% Si y solo si
\newcommand{\sii}{\iff}

% Letras griegas
\newcommand{\eps}{\epsilon}
\newcommand{\veps}{\varepsilon}
\newcommand{\lm}{\lambda}

\newcommand{\ol}{\overline}
\newcommand{\ul}{\underline}
\newcommand{\wt}{\widetilde}
\newcommand{\wh}{\widehat}

\let\oldvec\vec
\renewcommand{\vec}{\overrightarrow}

% Derivadas parciales
\newcommand{\del}[2]{\frac{\partial #1}{\partial #2}}
\newcommand{\Del}[3]{\frac{\partial^{#1} #2}{\partial #3^{#1}}}
\newcommand{\deld}[2]{\dfrac{\partial #1}{\partial #2}}
\newcommand{\Deld}[3]{\dfrac{\partial^{#1} #2}{\partial #3^{#1}}}


\newcommand{\AstIg}{\stackrel{(\ast)}{=}}
\newcommand{\Hop}{\stackrel{L'H\hat{o}pital}{=}}

\newcommand{\red}[1]{{\color{red}#1}} % Para integrales, destacar los cambios.

% Método de integración
\newcommand{\MetInt}[2]{
    \left[\begin{array}{c}
        #1 \\ #2
    \end{array}\right]
}

% Declarar aplicaciones
% 1. Nombre aplicación
% 2. Dominio
% 3. Codominio
% 4. Variable
% 5. Imagen de la variable
\newcommand{\Func}[5]{
    \begin{equation*}
        \begin{array}{rrll}
            #1:& #2 & \longrightarrow & #3\\
               & #4 & \longmapsto & #5
        \end{array}
    \end{equation*}
}

%------------------------------------------------------------------------


\begin{document}

    % 1. Foto de fondo
    % 2. Título
    % 3. Encabezado Izquierdo
    % 4. Color de fondo
    % 5. Coord x del titulo
    % 6. Coord y del titulo
    % 7. Fecha
    \newcommand{\N}{{\mathbb{N}}} % Enteros
    \newcommand{\Q}{{\mathbb{Q}}} % Racionales
    \newcommand{\R}{{\mathbb{R}}} % Reales
    \newcommand{\K}{{\mathbb{K}}} % Cuerpo arbitrario
    
    % 1. Foto de fondo
% 2. Título
% 3. Encabezado Izquierdo
% 4. Color de fondo
% 5. Coord x del titulo
% 6. Coord y del titulo
% 7. Fecha

\newcommand{\portada}[7]{

    \portadaBase{#1}{#2}{#3}{#4}{#5}{#6}{#7}
    \portadaBook{#1}{#2}{#3}{#4}{#5}{#6}{#7}
}

\newcommand{\portadaExamen}[7]{

    \portadaBase{#1}{#2}{#3}{#4}{#5}{#6}{#7}
    \portadaArticle{#1}{#2}{#3}{#4}{#5}{#6}{#7}
}




\newcommand{\portadaBase}[7]{

    % Tiene la portada principal y la licencia Creative Commons
    
    % 1. Foto de fondo
    % 2. Título
    % 3. Encabezado Izquierdo
    % 4. Color de fondo
    % 5. Coord x del titulo
    % 6. Coord y del titulo
    % 7. Fecha
    
    
    \thispagestyle{empty}               % Sin encabezado ni pie de página
    \newgeometry{margin=0cm}        % Márgenes nulos para la primera página
    
    
    % Encabezado
    \fancyhead[L]{\helv #3}
    \fancyhead[R]{\helv \nouppercase{\leftmark}}
    
    
    \pagecolor{#4}        % Color de fondo para la portada
    
    \begin{figure}[p]
        \centering
        \transparent{0.3}           % Opacidad del 30% para la imagen
        
        \includegraphics[width=\paperwidth, keepaspectratio]{assets/#1}
    
        \begin{tikzpicture}[remember picture, overlay]
            \node[anchor=north west, text=white, opacity=1, font=\fontsize{60}{90}\selectfont\bfseries\sffamily, align=left] at (#5, #6) {#2};
            
            \node[anchor=south east, text=white, opacity=1, font=\fontsize{12}{18}\selectfont\sffamily, align=right] at (9.7, 3) {\textbf{\href{https://losdeldgiim.github.io/}{Los Del DGIIM}}};
            
            \node[anchor=south east, text=white, opacity=1, font=\fontsize{12}{15}\selectfont\sffamily, align=right] at (9.7, 1.8) {Doble Grado en Ingeniería Informática y Matemáticas\\Universidad de Granada};
        \end{tikzpicture}
    \end{figure}
    
    
    \restoregeometry        % Restaurar márgenes normales para las páginas subsiguientes
    \pagecolor{white}       % Restaurar el color de página
    
    
    \newpage
    \thispagestyle{empty}               % Sin encabezado ni pie de página
    \begin{tikzpicture}[remember picture, overlay]
        \node[anchor=south west, inner sep=3cm] at (current page.south west) {
            \begin{minipage}{0.5\paperwidth}
                \href{https://creativecommons.org/licenses/by-nc-nd/4.0/}{
                    \includegraphics[height=2cm]{assets/Licencia.png}
                }\vspace{1cm}\\
                Esta obra está bajo una
                \href{https://creativecommons.org/licenses/by-nc-nd/4.0/}{
                    Licencia Creative Commons Atribución-NoComercial-SinDerivadas 4.0 Internacional (CC BY-NC-ND 4.0).
                }\\
    
                Eres libre de compartir y redistribuir el contenido de esta obra en cualquier medio o formato, siempre y cuando des el crédito adecuado a los autores originales y no persigas fines comerciales. 
            \end{minipage}
        };
    \end{tikzpicture}
    
    
    
    % 1. Foto de fondo
    % 2. Título
    % 3. Encabezado Izquierdo
    % 4. Color de fondo
    % 5. Coord x del titulo
    % 6. Coord y del titulo
    % 7. Fecha


}


\newcommand{\portadaBook}[7]{

    % 1. Foto de fondo
    % 2. Título
    % 3. Encabezado Izquierdo
    % 4. Color de fondo
    % 5. Coord x del titulo
    % 6. Coord y del titulo
    % 7. Fecha

    % Personaliza el formato del título
    \pretitle{\begin{center}\bfseries\fontsize{42}{56}\selectfont}
    \posttitle{\par\end{center}\vspace{2em}}
    
    % Personaliza el formato del autor
    \preauthor{\begin{center}\Large}
    \postauthor{\par\end{center}\vfill}
    
    % Personaliza el formato de la fecha
    \predate{\begin{center}\huge}
    \postdate{\par\end{center}\vspace{2em}}
    
    \title{#2}
    \author{\href{https://losdeldgiim.github.io/}{Los Del DGIIM}}
    \date{Granada, #7}
    \maketitle
    
    \tableofcontents
}




\newcommand{\portadaArticle}[7]{

    % 1. Foto de fondo
    % 2. Título
    % 3. Encabezado Izquierdo
    % 4. Color de fondo
    % 5. Coord x del titulo
    % 6. Coord y del titulo
    % 7. Fecha

    % Personaliza el formato del título
    \pretitle{\begin{center}\bfseries\fontsize{42}{56}\selectfont}
    \posttitle{\par\end{center}\vspace{2em}}
    
    % Personaliza el formato del autor
    \preauthor{\begin{center}\Large}
    \postauthor{\par\end{center}\vspace{3em}}
    
    % Personaliza el formato de la fecha
    \predate{\begin{center}\huge}
    \postdate{\par\end{center}\vspace{5em}}
    
    \title{#2}
    \author{\href{https://losdeldgiim.github.io/}{Los Del DGIIM}}
    \date{Granada, #7}
    \thispagestyle{empty}               % Sin encabezado ni pie de página
    \maketitle
    \vfill
}
    \portadaExamen{ffccA4.jpg}{Geometría I\\Examen XIII}{Geometría I. Examen XIII}{MidnightBlue}{-8}{28}{2021}{Víctor Naranjo Cabrera}

    \begin{description}
        \item[Asignatura] Geometría I.
        \item[Curso Académico] 2021-22.
        \item[Grado] Grado en Matemáticas.
        \item[Grupo] B
        \item[Profesor] Antonio Alarcón López.
        \item[Fecha] 21 de diciembre de 2021.
    
    \end{description}
    \newpage


    % ------------------------------------
    
    \begin{ejercicio}[3 puntos] Sea $f: V \rightarrow V'$ una aplicación lineal entre espacios vectoriales de dimensión finita. Demostrar que las siguientes afirmaciones son equivalentes:
    \begin{enumerate}[label=\alph*.]
        \item $f$ es un monomorfismo.
        \item Para todo conjunto linealmente independiente $S \subset V$, se tiene que $f(S) \subset V'$ es linealmente independiente
        \item Existe una base $\mathcal{B}$ de $V$ tal que $f(\mathcal{B})$ es un conjunto linealmente independiente en $V'$
    \end{enumerate}
    \end{ejercicio}
    
    \begin{ejercicio}[3 puntos] Sea $V$ un espacio vectorial de dimensión finita. Demostrar que toda base de $V$ tiene una única base dual en $V^*$
    \end{ejercicio}

    \begin{ejercicio}[4 puntos]
        Sea $f: \R^3 \rightarrow \R^3$ un endomorfismo de $\R^3$ del que se sabe que:
        \begin{equation*}
            f(1, 0 , 2) = (-2, 1, 1), f(0, 1, 0) = (-1, 1, 0), \text{ y } f(0, 0, 1) = (-1, 0, 1)
        \end{equation*}
        Se pide:
        \begin{enumerate}[label=\alph*.]
            \item Obtener bases del núcleo y la imagen de $f$. Calcular la matriz asociada a $f$ respecto de la base usual de $\R^3, A=\mathcal{M}(f, \mathcal{B}_u)$.
        \item Obtener, si es posible, bases $\mathcal{B}$ y $\mathcal{B}'$ de $R^3$ tales que
            \begin{equation*}
                \mathcal{M}(f, \mathcal{B'} \rightarrow \mathcal{B}) = C = \begin{pmatrix}
                    1 & 0 & 0 \\
                    0 & 1 & 0 \\
                    0 & 0 & 0
                \end{pmatrix}
            \end{equation*}
            Encontrar las matrices regulares, $P, Q \in Gl(3, \R)$ tales que $C = Q^{-1} \cdot A \cdot P$.
        \item Calcular la base dual $\mathcal{B}^*$ de la base $\mathcal{B}$ de $\R^3$ obtenida en el apartado anterior.
        \end{enumerate}
    \end{ejercicio}

    \newpage
    \setcounter{ejercicio}{0} % Reseteo de contador para ejercicios resueltos

\begin{ejercicio}[3 puntos] Sea $f: V \rightarrow V'$ una aplicación lineal entre espacios vectoriales de dimensión finita. Demostrar que las siguientes afirmaciones son equivalentes:
    \begin{enumerate}[label=\alph*.]
    \item $f$ es un monomorfismo.
    \item Para todo conjunto linealmente independiente $S \subset V$, se tiene que $f(S) \subset V'$ es linealmente independiente
    \item Existe una base $\mathcal{B}$ de $V$ tal que $f(\mathcal{B})$ es un conjunto linealmente independiente en $V'$ 
    \end{enumerate}
        
        \noindent
        Sean $V$ y $V'$ espacios vectoriales sobre el mismo cuerpo $\K$:
        \begin{description}
            \item[$\underline{a) \Rightarrow b)}$] \ \\\\
            Sea $f$ un monomorfismo. Sea $S = \{v_1,\dotsc, v_n\}$ l.i $\subseteq V\Rightarrow a_1v_1 + \dotsc + a_nv_n = 0 \Rightarrow a_1 = \dotsc = a_n. = 0$, Sea ahora  $f(S) = \{f(v_1), \dotsc, f(v_n)\}$ y $b_1,\dotsc,b_n, \text{ con } b_i \in \K \ \forall i = 1,\dotsc,n$ tal que
            \begin{equation*}
                b_1f(v_1) + \dotsc + b_nf(v_n) = 0 \Rightarrow f(b_1v_1 + \dotsc + b_nv_n) = 0
            \end{equation*}
            Por ser inyectiva, $b_1v_1+ \dotsc + b_nv_n = 0$. Como $\{v_1,\dotsc,v_n\}$ l.i $\Rightarrow b_1 = \dotsc = b_n = 0 \Rightarrow f(S)$ l.i.
            \item[$\underline{b) \Rightarrow c)}$] \ \\\\
            Como la dimensión de $V$ es finita, $\exists \mathcal{B}$ base de $V$. Por definición, $\mathcal{B}$ es l.i $\Rightarrow f(\mathcal{B})$ es l.i.

            \item[$\underline{c) \Rightarrow a)}$] \ \\\\
            Sea $\mathcal{B} = \{v_1,\dotsc,v_n\}$ y $x \in Ker(f) \mid x = a_1v_1+\dotsc+a_nv_n, \text{ con } a_i \in \K$ $\forall i = 0,\dotsc,n$. Entonces:
            \begin{equation*}
                0 = f(x) = a_1f(v_1)+\dotsc+a_nf(v_n) = 0 \xRightarrow{f(\mathcal{B}) \ l.i} a_1 = a_2 = \dotsc = a_n= 0 \Rightarrow x = 0
            \end{equation*}
            Por tanto, $Ker(f) = \{0\} \Rightarrow f \text{ monomorfismo }$
        \end{description}
    \end{ejercicio}
    
    \begin{ejercicio}[3 puntos] Sea $V$ un espacio vectorial de dimensión finita. Demostrar que toda base de $V$ tiene una única base dual en $V^*$ \\

    \noindent
    Sea $V$ esp. vect. con  $dim_\K(V) = n \in \N$. Queremos probar que $\forall \mathcal{B} = \{v_1, \dotsc, v_n\}$ base de $V, \exists! \ \mathcal{B^*}$ base de $V^*$. \\
    Como $\forall \varphi \in V^*, \varphi$ viene determinada por $\mathcal{M}(\varphi, \{1\} \leftarrow \mathcal{B})$ y esta por las imágenes de cada vector de $\mathcal{B}$, entonces cada $\varphi \in V^*$ viene determinada por $\varphi(v_i) \ \forall i = 1,\dotsc, n$. Como $\varphi_i(v_j) = \delta_{ij}, \forall i, j = 1,\dotsc, n,$ se tienen determinadas $\varphi_1,\dotsc,\varphi_n$  de forma única $\Rightarrow \exists! \ \mathcal{B^*} \subseteq V^*$. \\\\ Veamos que efectivamente $\mathcal{B^*}$ forma base, demostrando para ello que $\mathcal{B^*}$ es l.i. Sean $a_1\varphi_1 + \dotsc+ a_n\varphi_n = \varphi_0 = 0 \ \forall i = 1,\dotsc,n$
    \begin{equation*}
        \varphi_0(v_i) = 0 = a_i\varphi_i(v_i) = a_i = 0 \Rightarrow \mathcal{B^*} \text{ l.i } \Rightarrow \mathcal{B^*} \text{ es base } 
    \end{equation*}
    \end{ejercicio}

    \begin{ejercicio}[4 puntos]
        Sea $f: \R^3 \rightarrow \R^3$ un endomorfismo de $\R^3$ del que se sabe que:
        \begin{equation*}
            f(1, 0 , 2) = (-2, 1, 1), f(0, 1, 0) = (-1, 1, 0), \text{ y } f(0, 0, 1) = (-1, 0, 1)
        \end{equation*}
        Se pide:
        \begin{enumerate}[label=\alph*.]
            \item Obtener bases del núcleo y la imagen de $f$. Calcular la matriz asociada a $f$ respecto de la base usual de $\R^3, A=\mathcal{M}(f, \mathcal{B}_u)$. \\\\
            \noindent
            Primero calcular la matriz A. Sea $\mathcal{B}_u = \{e_1, e_2, e_3\}$:
            \begin{gather*}
                \left\{\begin{array}{l}
                    f(0, 1, 0) = f(e_2) = (-1, 1, 0) \\
                    f(0, 0, 1) = f(e_3) = (-1, 0, 1) \\
                     f(1, 0, 2) = f(e_1) + 2f(e_3) \Rightarrow f(e_1) = (-2, 1, 1) - 2(-1, 0, 1) = (0, 1, -1) \\
                \end{array}\right.
            \end{gather*}
            Por tanto:
            \begin{equation*}
                \mathcal{M}(f, \mathcal{B}_u) = \begin{pmatrix}
                    0 & -1 & -1 \\
                    1 & 1 & 0 \\
                    -1 & 0 & 1 \\
                \end{pmatrix}
            \end{equation*}
            Calcular base de $Ker(f)$. 
            \begin{gather*}
                Ker(f) = \left\{(x, y, z) \in \R^3 \left| \begin{pmatrix}
                0 & -1 & -1 \\
                1 & 1 & 0 \\
                -1 & 0 & 1 \\
            \end{pmatrix} 
            \begin{pmatrix}
                x \\
                y \\
                z \\
            \end{pmatrix} =\begin{pmatrix}
                0 \\
                0 \\
                0 \\
            \end{pmatrix} \right.\right\} \\\\ = \left\{(x, y, z) \in \R^3 \left|\begin{array}{ll}
                 & -y -z = 0 \\
                 & x + y = 0 \\
                 & -x + z = 0 \\
            \end{array}
            \right.\right\} 
             = \left\{(x, y, z) \in \R^3 \left|\begin{array}{ll}
                 & y = -z \\
                 & x = -y \\
                 & x = z \\
            \end{array}
            \right.\right\} \\\\ = \left\{(x, y, z) \in \R^3 \left|\begin{array}{ll}
                 & y = -z \\
                 & x = z \\
            \end{array}
            \right.\right\}  = \mathcal{L}(\{(1, -1, 1\}) \\ \Rightarrow \mathcal{B} = \{(1, -1, 1)\} \text{ base de } Ker(f) 
            \end{gather*}
            Obtener base de $Im(f)$.
            \begin{equation*}
                \begin{vmatrix}
                    0 & -1 & 1 \\
                    1 & 1 & 0 \\
                    -1 & 0 & 1 \\
                \end{vmatrix} = -1+1 = 0\Rightarrow rg(A) \neq 3, 
                \begin{vmatrix}
                    0 & -1 \\
                    1 & 1 \\
                \end{vmatrix} \neq 0 \Rightarrow rg(A) = 2
            \end{equation*}
             Por lo que $\mathcal{B} = \{(0, -1, 1), (-1, 1, 0)\}$ l.i  y base de $ Im(f)$ \\
        \item Obtener, si es posible, bases $\mathcal{B}$ y $\mathcal{B}'$ de $R^3$ tales que
            \begin{equation*}
                \mathcal{M}(f, \mathcal{B'} \rightarrow \mathcal{B}) = C = \begin{pmatrix}
                    1 & 0 & 0 \\
                    0 & 1 & 0 \\
                    0 & 0 & 0
                \end{pmatrix}
            \end{equation*}
            Encontrar las matrices regulares, $P, Q \in Gl(3, \R)$ tales que $C = Q^{-1} \cdot A \cdot P$.
        \noindent
        Sean $\mathcal{B} = \{(0, 0, 1), (0, 1, 0), (1, -1, 1)\}$ y $\mathcal{B'} = \{(-1, 0, 1), (-1, 1, 0), (0, 0, 1)\} $:
        \begin{gather*}
                \left\{\begin{array}{ll}
                    f(0, 0, 1) = (-1, 0, 1) \\
                    f(0, 0, 1) = (-1, 1, 0) \\
                    f(1, -1, 1) = (0, 0, 0) 
                \end{array}\right. \Rightarrow M(f, \mathcal{B'} \leftarrow \mathcal{B}) = \begin{pmatrix}
                    1 & 0 & 0 \\
                    0 & 1 & 0 \\
                    0 & 0 & 0 \\
                \end{pmatrix} = C
        \end{gather*}
        Encontrar $P, Q \in Gl(3, \R) \mid C = Q^{-1} \cdot A \cdot P$ \\
        Sabemos que $\mathcal{M} (f, \mathcal{B}' \leftarrow \mathcal{B}) = \mathcal{M}(I_{\R^3}, \mathcal{B'} \leftarrow \mathcal{B}_u) \cdot \mathcal{M}(f, \mathcal{B}_u) \cdot {\mathcal{M}(I_{\R^3}, \mathcal{B}_u \leftarrow \mathcal{B})} \Rightarrow C = \mathcal{M}(I_{\R^3}, \mathcal{B}_u \leftarrow \mathcal{B'}) \cdot A \cdot \mathcal{M}(I_{\R^3}, \mathcal{B}_u \leftarrow \mathcal{B})$. Por lo que:
        \begin{equation*}
            Q = \mathcal{M}(I_{\R^3}, \mathcal{B}_u \leftarrow \mathcal{B´}) = \begin{pmatrix}
                -1 & 1 & 0 \\
                0 & 1 & 0 \\
                1 & 0 & 1 \\
            \end{pmatrix},
             P = \mathcal{M}(I_{\R^3}, \mathcal{B}_u \leftarrow \mathcal{B}) = \begin{pmatrix}
                0 & 0 & 1 \\
                0 & 1 & -1 \\
                1 & 0 & 1 \\
            \end{pmatrix}
        \end{equation*}
        \item Calcular la base dual $\mathcal{B}^*$ de la base $\mathcal{B}$ de $\R^3$ obtenida en el apartado anterior.
        \noindent
        Sea $\mathcal{B}_u^* = \{\varphi_1, \varphi_2, \varphi_3\}, \varphi_i(a_1, a_2, a_3) = a_i, \forall i = 1, 2, 3$  y $\mathcal{B'} = \{\psi_1, \psi_2, \psi_3\}, \psi_i\equiv(a_i, b_i, c_i)_{\mathcal{B}_u^*}$:
        \begin{gather*}
            \begin{pmatrix}
                a_1 & b_1 & c_1 \\
                a_2 & b_2 & c_2 \\
                a_3 & b_3 & c_3 \\
            \end{pmatrix} 
            \begin{pmatrix}
                0 & 0 & 1 \\
                0 & 1 & -1 \\
                1 & 0 & 1 \\
            \end{pmatrix} = I_3 \Rightarrow 
             \begin{pmatrix}
                a_1 & b_1 & c_1 \\
                a_2 & b_2 & c_2 \\
                a_3 & b_3 & c_3 \\
            \end{pmatrix} =
            \begin{pmatrix}
                0 & 0 & 1 \\
                0 & 1 & -1 \\
                1 & 0 & 1 \\
            \end{pmatrix}^{-1} = 
            \begin{pmatrix}
                -1 & 0 & 1 \\
                1 & 1 & 0 \\
                1 & 0 & 0 \\
            \end{pmatrix} \\\\
            \Rightarrow \psi_1 = (-1, 0, 1)_{\mathcal{B}_u^*}, \psi_2 = (1, 1, 0)_{\mathcal{B}_u^*} \psi_3 = (1, 0, 0)_{\mathcal{B}_u^*} \Rightarrow \mathcal{B^*} = \{-\varphi_1 + \varphi_3, \varphi_1 + \varphi_2, \varphi_1\}
        \end{gather*}
        \end{enumerate}
    \end{ejercicio}

    
\end{document}
