\documentclass[12pt]{article}

% Idioma y codificación
\usepackage[spanish, es-tabla]{babel}       %es-tabla para que se titule "Tabla"
\usepackage[utf8]{inputenc}

% Márgenes
\usepackage[a4paper,top=3cm,bottom=2.5cm,left=3cm,right=3cm]{geometry}

% Comentarios de bloque
\usepackage{verbatim}

% Paquetes de links
\usepackage[hidelinks]{hyperref}    % Permite enlaces
\usepackage{url}                    % redirecciona a la web

% Más opciones para enumeraciones
\usepackage{enumitem}

% Personalizar la portada
\usepackage{titling}

% Paquetes de tablas
\usepackage{multirow}


%------------------------------------------------------------------------

%Paquetes de figuras
\usepackage{caption}
\usepackage{subcaption} % Figuras al lado de otras
\usepackage{float}      % Poner figuras en el sitio indicado H.


% Paquetes de imágenes
\usepackage{graphicx}       % Paquete para añadir imágenes
\usepackage{transparent}    % Para manejar la opacidad de las figuras

% Paquete para usar colores
\usepackage[dvipsnames]{xcolor}
\usepackage{pagecolor}      % Para cambiar el color de la página

% Habilita tamaños de fuente mayores
\usepackage{fix-cm}

% Para los gráficos
\usepackage{tikz}

% Para poder situar los nodos en los grafos
\usetikzlibrary{positioning}


%------------------------------------------------------------------------

% Paquetes de matemáticas
\usepackage{mathtools, amsfonts, amssymb, mathrsfs}
\usepackage[makeroom]{cancel}     % Simplificar tachando
\usepackage{polynom}    % Divisiones y Ruffini
\usepackage{units} % Para poner fracciones diagonales con \nicefrac

\usepackage{pgfplots}   %Representar funciones
\pgfplotsset{compat=1.18}  % Versión 1.18

\usepackage{tikz-cd}    % Para usar diagramas de composiciones
\usetikzlibrary{calc}   % Para usar cálculo de coordenadas en tikz

%Definición de teoremas, etc.
\usepackage{amsthm}
%\swapnumbers   % Intercambia la posición del texto y de la numeración

\theoremstyle{plain}

\makeatletter
\@ifclassloaded{article}{
  \newtheorem{teo}{Teorema}[section]
}{
  \newtheorem{teo}{Teorema}[chapter]  % Se resetea en cada chapter
}
\makeatother

\newtheorem{coro}{Corolario}[teo]           % Se resetea en cada teorema
\newtheorem{prop}[teo]{Proposición}         % Usa el mismo contador que teorema
\newtheorem{lema}[teo]{Lema}                % Usa el mismo contador que teorema

\theoremstyle{remark}
\newtheorem*{observacion}{Observación}

\theoremstyle{definition}

\makeatletter
\@ifclassloaded{article}{
  \newtheorem{definicion}{Definición} [section]     % Se resetea en cada chapter
}{
  \newtheorem{definicion}{Definición} [chapter]     % Se resetea en cada chapter
}
\makeatother

\newtheorem*{notacion}{Notación}
\newtheorem*{ejemplo}{Ejemplo}
\newtheorem*{ejercicio*}{Ejercicio}             % No numerado
\newtheorem{ejercicio}{Ejercicio} [section]     % Se resetea en cada section


% Modificar el formato de la numeración del teorema "ejercicio"
\renewcommand{\theejercicio}{%
  \ifnum\value{section}=0 % Si no se ha iniciado ninguna sección
    \arabic{ejercicio}% Solo mostrar el número de ejercicio
  \else
    \thesection.\arabic{ejercicio}% Mostrar número de sección y número de ejercicio
  \fi
}


% \renewcommand\qedsymbol{$\blacksquare$}         % Cambiar símbolo QED
%------------------------------------------------------------------------

% Paquetes para encabezados
\usepackage{fancyhdr}
\pagestyle{fancy}
\fancyhf{}

\newcommand{\helv}{ % Modificación tamaño de letra
\fontfamily{}\fontsize{12}{12}\selectfont}
\setlength{\headheight}{15pt} % Amplía el tamaño del índice


%\usepackage{lastpage}   % Referenciar última pag   \pageref{LastPage}
\fancyfoot[C]{\thepage}

%------------------------------------------------------------------------

% Conseguir que no ponga "Capítulo 1". Sino solo "1."
\makeatletter
\@ifclassloaded{book}{
  \renewcommand{\chaptermark}[1]{\markboth{\thechapter.\ #1}{}} % En el encabezado
    
  \renewcommand{\@makechapterhead}[1]{%
  \vspace*{50\p@}%
  {\parindent \z@ \raggedright \normalfont
    \ifnum \c@secnumdepth >\m@ne
      \huge\bfseries \thechapter.\hspace{1em}\ignorespaces
    \fi
    \interlinepenalty\@M
    \Huge \bfseries #1\par\nobreak
    \vskip 40\p@
  }}
}
\makeatother

%------------------------------------------------------------------------
% Paquetes de cógido
\usepackage{minted}
\renewcommand\listingscaption{Código fuente}

\usepackage{fancyvrb}
% Personaliza el tamaño de los números de línea
\renewcommand{\theFancyVerbLine}{\small\arabic{FancyVerbLine}}

% Estilo para C++
\newminted{cpp}{
    frame=lines,
    framesep=2mm,
    baselinestretch=1.2,
    linenos,
    escapeinside=||
}

% para minted
\definecolor{LightGray}{rgb}{0.95,0.95,0.92}
\setminted{
    linenos=true,
    stepnumber=5,
    numberfirstline=true,
    autogobble,
    breaklines=true,
    breakautoindent=true,
    breaksymbolleft=,
    breaksymbolright=,
    breaksymbolindentleft=0pt,
    breaksymbolindentright=0pt,
    breaksymbolsepleft=0pt,
    breaksymbolsepright=0pt,
    fontsize=\footnotesize,
    bgcolor=LightGray,
    numbersep=10pt
}


\usepackage{listings} % Para incluir código desde un archivo

\renewcommand\lstlistingname{Código Fuente}
\renewcommand\lstlistlistingname{Índice de Códigos Fuente}

% Definir colores
\definecolor{vscodepurple}{rgb}{0.5,0,0.5}
\definecolor{vscodeblue}{rgb}{0,0,0.8}
\definecolor{vscodegreen}{rgb}{0,0.5,0}
\definecolor{vscodegray}{rgb}{0.5,0.5,0.5}
\definecolor{vscodebackground}{rgb}{0.97,0.97,0.97}
\definecolor{vscodelightgray}{rgb}{0.9,0.9,0.9}

% Configuración para el estilo de C similar a VSCode
\lstdefinestyle{vscode_C}{
  backgroundcolor=\color{vscodebackground},
  commentstyle=\color{vscodegreen},
  keywordstyle=\color{vscodeblue},
  numberstyle=\tiny\color{vscodegray},
  stringstyle=\color{vscodepurple},
  basicstyle=\scriptsize\ttfamily,
  breakatwhitespace=false,
  breaklines=true,
  captionpos=b,
  keepspaces=true,
  numbers=left,
  numbersep=5pt,
  showspaces=false,
  showstringspaces=false,
  showtabs=false,
  tabsize=2,
  frame=tb,
  framerule=0pt,
  aboveskip=10pt,
  belowskip=10pt,
  xleftmargin=10pt,
  xrightmargin=10pt,
  framexleftmargin=10pt,
  framexrightmargin=10pt,
  framesep=0pt,
  rulecolor=\color{vscodelightgray},
  backgroundcolor=\color{vscodebackground},
}

%------------------------------------------------------------------------

% Comandos definidos
\newcommand{\bb}[1]{\mathbb{#1}}
\newcommand{\cc}[1]{\mathcal{#1}}

% I prefer the slanted \leq
\let\oldleq\leq % save them in case they're every wanted
\let\oldgeq\geq
\renewcommand{\leq}{\leqslant}
\renewcommand{\geq}{\geqslant}

% Si y solo si
\newcommand{\sii}{\iff}

% Letras griegas
\newcommand{\eps}{\epsilon}
\newcommand{\veps}{\varepsilon}
\newcommand{\lm}{\lambda}

\newcommand{\ol}{\overline}
\newcommand{\ul}{\underline}
\newcommand{\wt}{\widetilde}
\newcommand{\wh}{\widehat}

\let\oldvec\vec
\renewcommand{\vec}{\overrightarrow}

% Derivadas parciales
\newcommand{\del}[2]{\frac{\partial #1}{\partial #2}}
\newcommand{\Del}[3]{\frac{\partial^{#1} #2}{\partial #3^{#1}}}
\newcommand{\deld}[2]{\dfrac{\partial #1}{\partial #2}}
\newcommand{\Deld}[3]{\dfrac{\partial^{#1} #2}{\partial #3^{#1}}}


\newcommand{\AstIg}{\stackrel{(\ast)}{=}}
\newcommand{\Hop}{\stackrel{L'H\hat{o}pital}{=}}

\newcommand{\red}[1]{{\color{red}#1}} % Para integrales, destacar los cambios.

% Método de integración
\newcommand{\MetInt}[2]{
    \left[\begin{array}{c}
        #1 \\ #2
    \end{array}\right]
}

% Declarar aplicaciones
% 1. Nombre aplicación
% 2. Dominio
% 3. Codominio
% 4. Variable
% 5. Imagen de la variable
\newcommand{\Func}[5]{
    \begin{equation*}
        \begin{array}{rrll}
            #1:& #2 & \longrightarrow & #3\\
               & #4 & \longmapsto & #5
        \end{array}
    \end{equation*}
}

%------------------------------------------------------------------------


\begin{document}

% 1. Foto de fondo
% 2. Título
% 3. Encabezado Izquierdo
% 4. Color de fondo
% 5. Coord x del titulo
% 6. Coord y del titulo
% 7. Fecha
\newcommand{\F}{{\mathcal{F}}} % Funcion
\newcommand{\R}{{\mathbb{R}}}  % Reales
\newcommand{\Q}{{\mathbb{Q}}}  % Racionales
\newcommand{\Z}{{\mathbb{Z}}}  % Enteros
\newcommand{\N}{{\mathbb{N}}}  % Naturales
\newcommand{\C}{{\mathbb{C}}}  % Complejos
\newcommand{\U}{{\mathcal{U}}} % Unidades
\hbadness=10000

% 1. Foto de fondo
% 2. Título
% 3. Encabezado Izquierdo
% 4. Color de fondo
% 5. Coord x del titulo
% 6. Coord y del titulo
% 7. Fecha

\newcommand{\portada}[7]{

    \portadaBase{#1}{#2}{#3}{#4}{#5}{#6}{#7}
    \portadaBook{#1}{#2}{#3}{#4}{#5}{#6}{#7}
}

\newcommand{\portadaExamen}[7]{

    \portadaBase{#1}{#2}{#3}{#4}{#5}{#6}{#7}
    \portadaArticle{#1}{#2}{#3}{#4}{#5}{#6}{#7}
}




\newcommand{\portadaBase}[7]{

    % Tiene la portada principal y la licencia Creative Commons
    
    % 1. Foto de fondo
    % 2. Título
    % 3. Encabezado Izquierdo
    % 4. Color de fondo
    % 5. Coord x del titulo
    % 6. Coord y del titulo
    % 7. Fecha
    
    
    \thispagestyle{empty}               % Sin encabezado ni pie de página
    \newgeometry{margin=0cm}        % Márgenes nulos para la primera página
    
    
    % Encabezado
    \fancyhead[L]{\helv #3}
    \fancyhead[R]{\helv \nouppercase{\leftmark}}
    
    
    \pagecolor{#4}        % Color de fondo para la portada
    
    \begin{figure}[p]
        \centering
        \transparent{0.3}           % Opacidad del 30% para la imagen
        
        \includegraphics[width=\paperwidth, keepaspectratio]{assets/#1}
    
        \begin{tikzpicture}[remember picture, overlay]
            \node[anchor=north west, text=white, opacity=1, font=\fontsize{60}{90}\selectfont\bfseries\sffamily, align=left] at (#5, #6) {#2};
            
            \node[anchor=south east, text=white, opacity=1, font=\fontsize{12}{18}\selectfont\sffamily, align=right] at (9.7, 3) {\textbf{\href{https://losdeldgiim.github.io/}{Los Del DGIIM}}};
            
            \node[anchor=south east, text=white, opacity=1, font=\fontsize{12}{15}\selectfont\sffamily, align=right] at (9.7, 1.8) {Doble Grado en Ingeniería Informática y Matemáticas\\Universidad de Granada};
        \end{tikzpicture}
    \end{figure}
    
    
    \restoregeometry        % Restaurar márgenes normales para las páginas subsiguientes
    \pagecolor{white}       % Restaurar el color de página
    
    
    \newpage
    \thispagestyle{empty}               % Sin encabezado ni pie de página
    \begin{tikzpicture}[remember picture, overlay]
        \node[anchor=south west, inner sep=3cm] at (current page.south west) {
            \begin{minipage}{0.5\paperwidth}
                \href{https://creativecommons.org/licenses/by-nc-nd/4.0/}{
                    \includegraphics[height=2cm]{assets/Licencia.png}
                }\vspace{1cm}\\
                Esta obra está bajo una
                \href{https://creativecommons.org/licenses/by-nc-nd/4.0/}{
                    Licencia Creative Commons Atribución-NoComercial-SinDerivadas 4.0 Internacional (CC BY-NC-ND 4.0).
                }\\
    
                Eres libre de compartir y redistribuir el contenido de esta obra en cualquier medio o formato, siempre y cuando des el crédito adecuado a los autores originales y no persigas fines comerciales. 
            \end{minipage}
        };
    \end{tikzpicture}
    
    
    
    % 1. Foto de fondo
    % 2. Título
    % 3. Encabezado Izquierdo
    % 4. Color de fondo
    % 5. Coord x del titulo
    % 6. Coord y del titulo
    % 7. Fecha


}


\newcommand{\portadaBook}[7]{

    % 1. Foto de fondo
    % 2. Título
    % 3. Encabezado Izquierdo
    % 4. Color de fondo
    % 5. Coord x del titulo
    % 6. Coord y del titulo
    % 7. Fecha

    % Personaliza el formato del título
    \pretitle{\begin{center}\bfseries\fontsize{42}{56}\selectfont}
    \posttitle{\par\end{center}\vspace{2em}}
    
    % Personaliza el formato del autor
    \preauthor{\begin{center}\Large}
    \postauthor{\par\end{center}\vfill}
    
    % Personaliza el formato de la fecha
    \predate{\begin{center}\huge}
    \postdate{\par\end{center}\vspace{2em}}
    
    \title{#2}
    \author{\href{https://losdeldgiim.github.io/}{Los Del DGIIM}}
    \date{Granada, #7}
    \maketitle
    
    \tableofcontents
}




\newcommand{\portadaArticle}[7]{

    % 1. Foto de fondo
    % 2. Título
    % 3. Encabezado Izquierdo
    % 4. Color de fondo
    % 5. Coord x del titulo
    % 6. Coord y del titulo
    % 7. Fecha

    % Personaliza el formato del título
    \pretitle{\begin{center}\bfseries\fontsize{42}{56}\selectfont}
    \posttitle{\par\end{center}\vspace{2em}}
    
    % Personaliza el formato del autor
    \preauthor{\begin{center}\Large}
    \postauthor{\par\end{center}\vspace{3em}}
    
    % Personaliza el formato de la fecha
    \predate{\begin{center}\huge}
    \postdate{\par\end{center}\vspace{5em}}
    
    \title{#2}
    \author{\href{https://losdeldgiim.github.io/}{Los Del DGIIM}}
    \date{Granada, #7}
    \thispagestyle{empty}               % Sin encabezado ni pie de página
    \maketitle
    \vfill
}
\portadaExamen{ffccA4.jpg}{Geometría I\\Examen IV}{Geometría I. Examen IV}{MidnightBlue}{-8}{28}{2023}{ Miguel Ángel De la Vega Rodríguez\\Arturo Olivares Martos}


\begin{description}
	\item[Asignatura] Geometría I.
	\item[Curso Académico] 2022-23.
	\item[Grado] Doble Grado en Ingeniería Informática y Matemáticas.
	\item[Grupo] Único.
	\item[Profesor] Juan de Dios Pérez Jiménez\footnote{El examen lo pone el departamento}.
	\item[Descripción] Convocatoria Ordinaria.
	\item[Fecha] 23 de enero de 2023.
	\item[Duración] 3 horas.

\end{description}
\newpage


\begin{enumerate}
	\item \textbf{[2 puntos]} Enuncia y demuestra el Teorema del Rango.
	      \begin{proof}
		      Tomemos una base $B_{\text{Ker}f}=\{ v_1,...,v_{n(f)}\}$ y ampliémosla a una base de $V(K)$. $B=\{v_1,...,v_{n(f)},
			      v_{n(f)+1}, ..., v(n)\}$. Sabemos que $f(B)$ es un sistema de generadores de Im$(f)$ y, puesto que
		      los vectores $v_i$ con $i \leq n(f)$ se aplican en 0, el conjunto $\{f(v_{n(f)+1}),...,f(v_n)$ es también un sistema
		      de generadores de Im$(f)$. Como este conjunto tiene $n - n(f)$ vectores, bastará ver que los vectores del conjunto son
		      linealmente independientes. Para ello tomamos una combinación lineal suya igualada a 0:
		      \begin{equation*}
			      0 = a_{n(f)+1} f(v_{n(f)+1})+...+a_n f(v_n) = f(a_{n(f)+1}v_{n(f)+1}+...+a_n v_n)
		      \end{equation*}
		      La expresión anterior claramente implica que $a_{n(f)+1}v_{n(f)+1}+...+a_n v_n \in$ Ker$(f)$. Pero entonces ese vector
		      se puede expresa como combinación lineal de elementos de $B_{\text{Ker}f}$. Esto es, para ciertos escalares
		      $a_1,...,a_{n(f)}$ se tiene:
		      \begin{equation*}
			      a_{n(f)+1}v_{n(f)+1}+...+a_n v_n = a_1 v_1+...+a_{n(f)}v_{n(f)}
		      \end{equation*}
		      Pasando los términos del segundo miembro al primero, se tiene una combinación lineal de la base $B$ igualada al vector
		      0. Por tanto, todso los coeficientes de la combinación deben ser nulos, y en particular, $a_n(f)+1 = ... = a_v = 0$
		      como se queria demostrar.
	      \end{proof}

	\item Sea $\displaystyle U = \left\{ M \in \mathcal{M}_2(\mathbb{R}) : MA = AM \right\}$, donde $\displaystyle A = \left( \begin{array}{cc}
				      1 & 1 \\
				      0 & 0
			      \end{array}\right)$.

	      \begin{enumerate}
		      \item \textbf{[2 Puntos]} Demostrar que $U$ es un subespacio vectorial de $M_2 (\R)$ y calcular un complementario.
		            \\ \\
		            Para ello, en primer lugar veamos que:
		            \begin{align*}
			            U & = \left\{ \begin{pmatrix}
				                          a & b \\
				                          c & d
			                          \end{pmatrix} \in M_2 (\R) : \begin{pmatrix}
				                                                       a & b \\
				                                                       c & d
			                                                       \end{pmatrix} \cdot \begin{pmatrix}
				                                                                           1 & 1 \\
				                                                                           0 & 0
			                                                                           \end{pmatrix} = \begin{pmatrix}
				                                                                                           1 & 1 \\
				                                                                                           0 & 0
			                                                                                           \end{pmatrix}\cdot\begin{pmatrix}
				                                                                                                             a & b \\
				                                                                                                             c & d
			                                                                                                             \end{pmatrix} \right\} \\ &=\left\{ \begin{array}{l}
				            a = a+c \\
				            a = b+d \\
				            c= 0
			            \end{array}\right\}
		            \end{align*}
		            que es el conjunto de soluciones de un SEL homogéneo por tanto un subespacio vectorial,
		            Y como $\dim{U} = \dim{M_2 (\R)} - 2 = 2$ y resolviendo el sistema obtenemos la siguiente base de $U$:
		            \begin{equation*}
			            B_U = \left\{ \begin{pmatrix}
				            1 & 1 \\
				            0 & 0
			            \end{pmatrix}, \begin{pmatrix}
				            1 & 0 \\
				            0 & 1
			            \end{pmatrix} \right\}
		            \end{equation*}
		            Y como el complementario de $U$, $W$ está generado por dos vectores más, tomamos dos vectores linealmente independientes:
		            \begin{equation*}
			            B_{W} = \left\{ \begin{pmatrix}
				            0 & 0 \\
				            1 & 0
			            \end{pmatrix},\begin{pmatrix}
				            1 & 0 \\
				            0 & 0
			            \end{pmatrix}
			            \right\}
		            \end{equation*}
		      \item \textbf{[1 Punto]} Hallar una base de $M_2 (\R) / U$ y las coordenadas en esa base de $\begin{pmatrix}
				            1  & 1 \\
				            -1 & 0
			            \end{pmatrix} + U$ \\ \\
		            Como $W$ es un subespacio complementario de $U$, y $B_{W} = \left\{ w_1,...,w_k\right\}$ es una base de $W$ entonces
		            \begin{equation*}
			            \pi (B_W) = \{ w_1 + U,..., w_k + U \}
		            \end{equation*}
		            es una base de $V /U$. Por tanto, $\dim_K(V/ U) = \dim_{\mathbb{K}} V - \dim_{\mathbb{K}} U = \dim{(M_2(\R) / U)} = \dim{(M_2(\R)} - \dim{U} = 4-2 =2$.
		            y tomando
		            \begin{equation*}
			            B_{V/U} = \left\{ \begin{pmatrix}
				            0 & 0 \\
				            1 & 0
			            \end{pmatrix}+ U,\begin{pmatrix}
				            1 & 0 \\
				            0 & 0
			            \end{pmatrix}+ U
			            \right\}
		            \end{equation*}
		            Vamos a calcular ahora las coordenadas de $\begin{pmatrix}
				            1  & 1 \\
				            -1 & 0
			            \end{pmatrix} + U$
		            Para ello, calculamos sus coordenadas en la siguiente base de $V$:
		            \begin{equation*}
			            B = \left\{ \begin{pmatrix}
				            1 & 1 \\
				            0 & 0
			            \end{pmatrix}, \begin{pmatrix}
				            1 & 0 \\
				            0 & 1
			            \end{pmatrix}, \begin{pmatrix}
				            0 & 0 \\
				            1 & 0
			            \end{pmatrix},\begin{pmatrix}
				            1 & 0 \\
				            0 & 0
			            \end{pmatrix} \right\}
		            \end{equation*}
		            Para ello, sean $a,b,c,d \in \R$, entonces:
		            \begin{align*}
			             & a\begin{pmatrix}
				                1 & 1 \\
				                0 & 0
			                \end{pmatrix}+ b\begin{pmatrix}
				                                1 & 0 \\
				                                0 & 1
			                                \end{pmatrix}+ c\begin{pmatrix}
				                                                0 & 0 \\
				                                                1 & 0
			                                                \end{pmatrix}+d\begin{pmatrix}
				                                                               1 & 0 \\
				                                                               0 & 0
			                                                               \end{pmatrix} = \\&= \begin{pmatrix}
				            1  & 1 \\
				            -1 & 0
			            \end{pmatrix} \Rightarrow a=1,b=0,c=-1, d = 0
		            \end{align*}
		            Por tanto obtenemos que
		            \begin{align*}
			            \begin{pmatrix}
				            1  & 1 \\
				            -1 & 0
			            \end{pmatrix} + U & = \begin{pmatrix}
				                                  1 & 1 \\
				                                  0 & 0
			                                  \end{pmatrix} + U - \begin{pmatrix}
				                                                      0 & 0 \\
				                                                      1 & 0
			                                                      \end{pmatrix} + U = 0 + U - \begin{pmatrix}
				                                                                                  0 & 0 \\
				                                                                                  1 & 0
			                                                                                  \end{pmatrix} + U \\&\Rightarrow \begin{pmatrix}
				            1  & 1 \\
				            -1 & 0
			            \end{pmatrix} + U = - \begin{pmatrix}
				            0 & 0 \\
				            1 & 0
			            \end{pmatrix} + U
		            \end{align*}
		      \item \textbf{[2 Puntos]} Construir una aplicación lineal $f : M_2(\R) \mapsto \R_3[x] $ cuyo núcleo sea $U$ y cuya
		            imagen tenga por sistema de generadores $\{1+x,1-x\}$\\ \\
		            Para que la aplicación lineal tenga Ker$(f)=U$ e Im$(f)=\mathcal{L}\left\{(1+x),(1-x)\right\}$ podemos hacer uso de lo
		            ya obtenido y tomar los vectores de $U$ y $W$:
		            \begin{align*}
			            f\left(\begin{pmatrix}
				                   1 & 1 \\
				                   0 & 0
			                   \end{pmatrix}\right) & = 0    \\
			            f\left(\begin{pmatrix}
				                   1 & 0 \\
				                   0 & 1
			                   \end{pmatrix}\right) & = 0    \\
			            f\left(\begin{pmatrix}
				                   0 & 0 \\
				                   1 & 0
			                   \end{pmatrix} \right) & = 1+x \\
			            f\left(\begin{pmatrix}
				                   1 & 0 \\
				                   0 & 0
			                   \end{pmatrix} \right) & = 1-x
		            \end{align*}
		            De donde obtenemos que
		            \begin{align*}
			            f(e_1)       & = 1-x                                                        \\
			            f(e_1 + e_2) & = 0 \Rightarrow f(e_1) = -f(e_2) \Rightarrow f(e_2) = -1 + x \\
			            f(e_3)       & = 1+x                                                        \\
			            f(e_1 + e_4) & = 0 \Rightarrow f(e_1) = -f(e_4) \Rightarrow f(e_4) = -1 + x \\
		            \end{align*}
		            Por tanto, la aplicación lineal $f$ es:
		            \begin{align*}
			            f\left(\begin{pmatrix}
				                   a & b \\
				                   c & d
			                   \end{pmatrix}\right) & = a(1-x) + b(-1+x) + c(1+x) + d(-1+x) \\ &= a-b+c-d + x(-a+b+c+d)
		            \end{align*}
		      \item \textbf{[1 Punto]} Calcular una matriz de $f$ respecto a las bases usuales $B_U = \{E_{ij} : 1 \leq i, j \leq 2\}$ (
		            con la ordenación que se escoja) de $M_2(\R)$ y $B_U^{'} = \{1,x,x^2,x^3\}$ de $\R_3[x]$. \\ \\
		            Para ello, calcularemos la imágen de cada uno de los vectores de la base de $M_2(\R)$, y lo expresaremos en la base
		            $B_U^{'}$:
		            \begin{align*}
			            f(E_{11}) & = 1-x = 1\cdot 1 + -1\cdot x + 0\cdot x^2 + 0\cdot x^3 \Rightarrow  (1,-1,0,0) \\
			            f(E_{12}) & = -1+x = -1\cdot 1 + 1\cdot x + 0\cdot x^2 + 0\cdot x^3 \Rightarrow (-1,1,0,0) \\
			            f(E_{21}) & = 1+x = 1\cdot 1 + 1\cdot x + 0\cdot x^2 + 0\cdot x^3 \Rightarrow (1,1,0,0)    \\
			            f(E_{22}) & = -1+x = -1\cdot 1 + 1\cdot x + 0\cdot x^2 + 0\cdot x^3 \Rightarrow (-1,1,0,0)
		            \end{align*}
		            De donde obtenemos que la matriz de $f$ es:
		            \begin{equation*}
			            M(f:B_U^{'} \leftarrow B_U) = \begin{pmatrix}
				            1  & -1 & 1 & -1 \\
				            -1 & 1  & 1 & 1  \\
				            0  & 0  & 0 & 0  \\
				            0  & 0  & 0 & 0
			            \end{pmatrix}
		            \end{equation*}
		      \item \textbf{[1 Punto]} Encontrar, si es posible, bases $B$ de $M_2(\R)$ y $B^{'}$ de $\R_3[x]$ tales que:
		            \begin{equation*}
			            M(f : B^{'} \leftarrow B) = \begin{pmatrix}
				            1 & 0 & 0 & 0 \\
				            0 & 1 & 0 & 0 \\
				            0 & 0 & 0 & 0 \\
				            0 & 0 & 0 & 0
			            \end{pmatrix}
		            \end{equation*}
		            Como $B_U=\{E_{11},E_{12},E_{21},E_{22}\}$ y $B_U^{'} = \{1,x,x^2,x^3\}$ son bases de $M_2(\R)$ y $\R_3[x]$ respectivamente,
		            \begin{align*}
			            f(E_{11}) & = (1,0,0,0) = e_1 \\
			            f(E_{12}) & = (0,1,0,0) = e_2 \\
			            f(E_{21}) & = (0,0,0,0) = 0   \\
			            f(E_{22}) & = (0,0,0,0) = 0
		            \end{align*}
		            Por tanto, si $B'=\{(1,1,0,0),(1,-1,0,0),(0,0,1,0),(0,0,0,1)\}$, entonces:
		            \begin{align*}
			            v_1 & = \begin{pmatrix}
				                    1 & 0 \\
				                    0 & 0
			                    \end{pmatrix} = E_{11}                                                  \\
			            v_2 & = \begin{pmatrix}
				                    0 & 1 \\
				                    0 & 0
			                    \end{pmatrix} = E_{12}                                                  \\
			            v_3 & \in \text{Nuc}(f) \Rightarrow f(v_3) = 0 \Rightarrow v_3 = \begin{pmatrix}
				                                                                             1 & 1 \\
				                                                                             0 & 0
			                                                                             \end{pmatrix} \\
			            v_4 & \in \text{Nuc}(f) \Rightarrow f(v_4) = 0 \Rightarrow v_4 = \begin{pmatrix}
				                                                                             1 & 0 \\
				                                                                             0 & 1
			                                                                             \end{pmatrix}
		            \end{align*}
		            $B=\left\{\begin{pmatrix}
				            1 & 0 \\
				            0 & 0
			            \end{pmatrix},\begin{pmatrix}
				            0 & 1 \\
				            0 & 0
			            \end{pmatrix},\begin{pmatrix}
				            1 & 1 \\
				            0 & 0
			            \end{pmatrix},\begin{pmatrix}
				            1 & 0 \\
				            0 & 1
			            \end{pmatrix}\right\}$ y  \\$B'=\{(1,1,0,0),(1,-1,0,0),(0,0,1,0),(0,0,0,1)\}$.
		      \item \textbf{[1 Punto]} Hallar bases de an$(U)$ y ker$(f^t)$
		            \begin{itemize}
			            \item an$(U)$ \\ \\
			                  Para calcular una base de an$(U)$ construimos una base de $U$ (ya la tenemos) y la ampliamos a una base de $M_2(\R)$:
			                  \begin{equation*}
				                  B = \left\{ \begin{pmatrix}
					                  1 & 1 \\
					                  0 & 0
				                  \end{pmatrix}, \begin{pmatrix}
					                  1 & 0 \\
					                  0 & 1
				                  \end{pmatrix}, \begin{pmatrix}
					                  0 & 0 \\
					                  1 & 0
				                  \end{pmatrix},\begin{pmatrix}
					                  1 & 0 \\
					                  0 & 0
				                  \end{pmatrix} \right\}
			                  \end{equation*}
			                  Sea ahora $B^* = \{ \varphi^1,\varphi^2,\varphi^3,\varphi^4\}$ la base dual de $B$, entonces, deducimos que
			                  $\{\varphi^3,\varphi^4\}$ es una base de an($U$), por tanto:
			                  \begin{equation*}
				                  \begin{pmatrix}
					                  a_1 & a_2 & a_3 & a_4 \\
					                  b_1 & b_2 & b_3 & b_4 \\
					                  c_1 & c_2 & c_3 & c_4 \\
					                  d_1 & d_2 & d_3 & d_4
				                  \end{pmatrix}
				                  \cdot
				                  \begin{pmatrix}
					                  1 & 0 & 0 & 0 \\
					                  1 & 0 & 0 & 0 \\
					                  0 & 1 & 1 & 0 \\
					                  0 & 0 & 0 & 1
				                  \end{pmatrix} =
				                  \begin{pmatrix}
					                  1 & 0 & 0 & 0 \\
					                  0 & 1 & 0 & 0 \\
					                  0 & 0 & 1 & 0 \\
					                  0 & 0 & 0 & 1
				                  \end{pmatrix}
			                  \end{equation*}
			                  De donde se obtiene que:
			                  \begin{equation*}
				                  \varphi^3\left(\begin{pmatrix}
					                  a & b \\
					                  c & d
				                  \end{pmatrix}\right) = c \quad \varphi^4\left(\begin{pmatrix}
					                  a & b \\
					                  c & d
				                  \end{pmatrix}\right) = a-b -d
			                  \end{equation*}
			            \item ker$(f^t)$ \\ \\
			                  Para ello hacemos uso de la propiedad an(Im($f$)) = ker($f^t$), en primer lugar, sabemos que
			                  \begin{equation*}
				                  \text{Im}(f) = \mathcal{L}\left\{(1+x),(1-x)\right\} = \mathcal{L}(1,x)
			                  \end{equation*}
			                  Que en coordenadas cartesianas es
			                  \begin{equation*}
				                  \text{Im}(f) = \left\{ a_0 + a_1 x + a_2 x^2 + a_3 x^3 : a_2 = a_3 = 0\right\}
			                  \end{equation*}
			                  Por tanto obtenemos que una base de an(Im($f$)) = ker($f^t$) es $\{ \varphi^1,\varphi^2\}$, donde:
			                  \begin{equation*}
				                  \varphi^1(a_0 + a_1 x + a_2 x^2 + a_3 x^3) = a_2 \quad \varphi^2(a_0 + a_1 x + a_2 x^2 + a_3 x^3) = a_3
			                  \end{equation*}
		            \end{itemize}	      \end{enumerate}
\end{enumerate}
\end{document}
