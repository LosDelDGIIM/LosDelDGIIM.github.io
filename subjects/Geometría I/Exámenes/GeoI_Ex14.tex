\documentclass[12pt]{article}

% Idioma y codificación
\usepackage[spanish, es-tabla]{babel}       %es-tabla para que se titule "Tabla"
\usepackage[utf8]{inputenc}

% Márgenes
\usepackage[a4paper,top=3cm,bottom=2.5cm,left=3cm,right=3cm]{geometry}

% Comentarios de bloque
\usepackage{verbatim}

% Paquetes de links
\usepackage[hidelinks]{hyperref}    % Permite enlaces
\usepackage{url}                    % redirecciona a la web

% Más opciones para enumeraciones
\usepackage{enumitem}

% Personalizar la portada
\usepackage{titling}

% Paquetes de tablas
\usepackage{multirow}


%------------------------------------------------------------------------

%Paquetes de figuras
\usepackage{caption}
\usepackage{subcaption} % Figuras al lado de otras
\usepackage{float}      % Poner figuras en el sitio indicado H.


% Paquetes de imágenes
\usepackage{graphicx}       % Paquete para añadir imágenes
\usepackage{transparent}    % Para manejar la opacidad de las figuras

% Paquete para usar colores
\usepackage[dvipsnames]{xcolor}
\usepackage{pagecolor}      % Para cambiar el color de la página

% Habilita tamaños de fuente mayores
\usepackage{fix-cm}

% Para los gráficos
\usepackage{tikz}

% Para poder situar los nodos en los grafos
\usetikzlibrary{positioning}


%------------------------------------------------------------------------

% Paquetes de matemáticas
\usepackage{mathtools, amsfonts, amssymb, mathrsfs}
\usepackage[makeroom]{cancel}     % Simplificar tachando
\usepackage{polynom}    % Divisiones y Ruffini
\usepackage{units} % Para poner fracciones diagonales con \nicefrac

\usepackage{pgfplots}   %Representar funciones
\pgfplotsset{compat=1.18}  % Versión 1.18

\usepackage{tikz-cd}    % Para usar diagramas de composiciones
\usetikzlibrary{calc}   % Para usar cálculo de coordenadas en tikz

%Definición de teoremas, etc.
\usepackage{amsthm}
%\swapnumbers   % Intercambia la posición del texto y de la numeración

\theoremstyle{plain}

\makeatletter
\@ifclassloaded{article}{
  \newtheorem{teo}{Teorema}[section]
}{
  \newtheorem{teo}{Teorema}[chapter]  % Se resetea en cada chapter
}
\makeatother

\newtheorem{coro}{Corolario}[teo]           % Se resetea en cada teorema
\newtheorem{prop}[teo]{Proposición}         % Usa el mismo contador que teorema
\newtheorem{lema}[teo]{Lema}                % Usa el mismo contador que teorema

\theoremstyle{remark}
\newtheorem*{observacion}{Observación}

\theoremstyle{definition}

\makeatletter
\@ifclassloaded{article}{
  \newtheorem{definicion}{Definición} [section]     % Se resetea en cada chapter
}{
  \newtheorem{definicion}{Definición} [chapter]     % Se resetea en cada chapter
}
\makeatother

\newtheorem*{notacion}{Notación}
\newtheorem*{ejemplo}{Ejemplo}
\newtheorem*{ejercicio*}{Ejercicio}             % No numerado
\newtheorem{ejercicio}{Ejercicio} [section]     % Se resetea en cada section


% Modificar el formato de la numeración del teorema "ejercicio"
\renewcommand{\theejercicio}{%
  \ifnum\value{section}=0 % Si no se ha iniciado ninguna sección
    \arabic{ejercicio}% Solo mostrar el número de ejercicio
  \else
    \thesection.\arabic{ejercicio}% Mostrar número de sección y número de ejercicio
  \fi
}


% \renewcommand\qedsymbol{$\blacksquare$}         % Cambiar símbolo QED
%------------------------------------------------------------------------

% Paquetes para encabezados
\usepackage{fancyhdr}
\pagestyle{fancy}
\fancyhf{}

\newcommand{\helv}{ % Modificación tamaño de letra
\fontfamily{}\fontsize{12}{12}\selectfont}
\setlength{\headheight}{15pt} % Amplía el tamaño del índice


%\usepackage{lastpage}   % Referenciar última pag   \pageref{LastPage}
\fancyfoot[C]{\thepage}

%------------------------------------------------------------------------

% Conseguir que no ponga "Capítulo 1". Sino solo "1."
\makeatletter
\@ifclassloaded{book}{
  \renewcommand{\chaptermark}[1]{\markboth{\thechapter.\ #1}{}} % En el encabezado
    
  \renewcommand{\@makechapterhead}[1]{%
  \vspace*{50\p@}%
  {\parindent \z@ \raggedright \normalfont
    \ifnum \c@secnumdepth >\m@ne
      \huge\bfseries \thechapter.\hspace{1em}\ignorespaces
    \fi
    \interlinepenalty\@M
    \Huge \bfseries #1\par\nobreak
    \vskip 40\p@
  }}
}
\makeatother

%------------------------------------------------------------------------
% Paquetes de cógido
\usepackage{minted}
\renewcommand\listingscaption{Código fuente}

\usepackage{fancyvrb}
% Personaliza el tamaño de los números de línea
\renewcommand{\theFancyVerbLine}{\small\arabic{FancyVerbLine}}

% Estilo para C++
\newminted{cpp}{
    frame=lines,
    framesep=2mm,
    baselinestretch=1.2,
    linenos,
    escapeinside=||
}

% para minted
\definecolor{LightGray}{rgb}{0.95,0.95,0.92}
\setminted{
    linenos=true,
    stepnumber=5,
    numberfirstline=true,
    autogobble,
    breaklines=true,
    breakautoindent=true,
    breaksymbolleft=,
    breaksymbolright=,
    breaksymbolindentleft=0pt,
    breaksymbolindentright=0pt,
    breaksymbolsepleft=0pt,
    breaksymbolsepright=0pt,
    fontsize=\footnotesize,
    bgcolor=LightGray,
    numbersep=10pt
}


\usepackage{listings} % Para incluir código desde un archivo

\renewcommand\lstlistingname{Código Fuente}
\renewcommand\lstlistlistingname{Índice de Códigos Fuente}

% Definir colores
\definecolor{vscodepurple}{rgb}{0.5,0,0.5}
\definecolor{vscodeblue}{rgb}{0,0,0.8}
\definecolor{vscodegreen}{rgb}{0,0.5,0}
\definecolor{vscodegray}{rgb}{0.5,0.5,0.5}
\definecolor{vscodebackground}{rgb}{0.97,0.97,0.97}
\definecolor{vscodelightgray}{rgb}{0.9,0.9,0.9}

% Configuración para el estilo de C similar a VSCode
\lstdefinestyle{vscode_C}{
  backgroundcolor=\color{vscodebackground},
  commentstyle=\color{vscodegreen},
  keywordstyle=\color{vscodeblue},
  numberstyle=\tiny\color{vscodegray},
  stringstyle=\color{vscodepurple},
  basicstyle=\scriptsize\ttfamily,
  breakatwhitespace=false,
  breaklines=true,
  captionpos=b,
  keepspaces=true,
  numbers=left,
  numbersep=5pt,
  showspaces=false,
  showstringspaces=false,
  showtabs=false,
  tabsize=2,
  frame=tb,
  framerule=0pt,
  aboveskip=10pt,
  belowskip=10pt,
  xleftmargin=10pt,
  xrightmargin=10pt,
  framexleftmargin=10pt,
  framexrightmargin=10pt,
  framesep=0pt,
  rulecolor=\color{vscodelightgray},
  backgroundcolor=\color{vscodebackground},
}

%------------------------------------------------------------------------

% Comandos definidos
\newcommand{\bb}[1]{\mathbb{#1}}
\newcommand{\cc}[1]{\mathcal{#1}}

% I prefer the slanted \leq
\let\oldleq\leq % save them in case they're every wanted
\let\oldgeq\geq
\renewcommand{\leq}{\leqslant}
\renewcommand{\geq}{\geqslant}

% Si y solo si
\newcommand{\sii}{\iff}

% Letras griegas
\newcommand{\eps}{\epsilon}
\newcommand{\veps}{\varepsilon}
\newcommand{\lm}{\lambda}

\newcommand{\ol}{\overline}
\newcommand{\ul}{\underline}
\newcommand{\wt}{\widetilde}
\newcommand{\wh}{\widehat}

\let\oldvec\vec
\renewcommand{\vec}{\overrightarrow}

% Derivadas parciales
\newcommand{\del}[2]{\frac{\partial #1}{\partial #2}}
\newcommand{\Del}[3]{\frac{\partial^{#1} #2}{\partial #3^{#1}}}
\newcommand{\deld}[2]{\dfrac{\partial #1}{\partial #2}}
\newcommand{\Deld}[3]{\dfrac{\partial^{#1} #2}{\partial #3^{#1}}}


\newcommand{\AstIg}{\stackrel{(\ast)}{=}}
\newcommand{\Hop}{\stackrel{L'H\hat{o}pital}{=}}

\newcommand{\red}[1]{{\color{red}#1}} % Para integrales, destacar los cambios.

% Método de integración
\newcommand{\MetInt}[2]{
    \left[\begin{array}{c}
        #1 \\ #2
    \end{array}\right]
}

% Declarar aplicaciones
% 1. Nombre aplicación
% 2. Dominio
% 3. Codominio
% 4. Variable
% 5. Imagen de la variable
\newcommand{\Func}[5]{
    \begin{equation*}
        \begin{array}{rrll}
            #1:& #2 & \longrightarrow & #3\\
               & #4 & \longmapsto & #5
        \end{array}
    \end{equation*}
}

%------------------------------------------------------------------------


\begin{document}

    % 1. Foto de fondo
    % 2. Título
    % 3. Encabezado Izquierdo
    % 4. Color de fondo
    % 5. Coord x del titulo
    % 6. Coord y del titulo
    % 7. Fecha
    \newcommand{\R}{{\mathbb{R}}} % Reales
    \newcommand{\K}{{\mathbb{K}}} % Cuerpo arbitrario
    
    % 1. Foto de fondo
% 2. Título
% 3. Encabezado Izquierdo
% 4. Color de fondo
% 5. Coord x del titulo
% 6. Coord y del titulo
% 7. Fecha

\newcommand{\portada}[7]{

    \portadaBase{#1}{#2}{#3}{#4}{#5}{#6}{#7}
    \portadaBook{#1}{#2}{#3}{#4}{#5}{#6}{#7}
}

\newcommand{\portadaExamen}[7]{

    \portadaBase{#1}{#2}{#3}{#4}{#5}{#6}{#7}
    \portadaArticle{#1}{#2}{#3}{#4}{#5}{#6}{#7}
}




\newcommand{\portadaBase}[7]{

    % Tiene la portada principal y la licencia Creative Commons
    
    % 1. Foto de fondo
    % 2. Título
    % 3. Encabezado Izquierdo
    % 4. Color de fondo
    % 5. Coord x del titulo
    % 6. Coord y del titulo
    % 7. Fecha
    
    
    \thispagestyle{empty}               % Sin encabezado ni pie de página
    \newgeometry{margin=0cm}        % Márgenes nulos para la primera página
    
    
    % Encabezado
    \fancyhead[L]{\helv #3}
    \fancyhead[R]{\helv \nouppercase{\leftmark}}
    
    
    \pagecolor{#4}        % Color de fondo para la portada
    
    \begin{figure}[p]
        \centering
        \transparent{0.3}           % Opacidad del 30% para la imagen
        
        \includegraphics[width=\paperwidth, keepaspectratio]{assets/#1}
    
        \begin{tikzpicture}[remember picture, overlay]
            \node[anchor=north west, text=white, opacity=1, font=\fontsize{60}{90}\selectfont\bfseries\sffamily, align=left] at (#5, #6) {#2};
            
            \node[anchor=south east, text=white, opacity=1, font=\fontsize{12}{18}\selectfont\sffamily, align=right] at (9.7, 3) {\textbf{\href{https://losdeldgiim.github.io/}{Los Del DGIIM}}};
            
            \node[anchor=south east, text=white, opacity=1, font=\fontsize{12}{15}\selectfont\sffamily, align=right] at (9.7, 1.8) {Doble Grado en Ingeniería Informática y Matemáticas\\Universidad de Granada};
        \end{tikzpicture}
    \end{figure}
    
    
    \restoregeometry        % Restaurar márgenes normales para las páginas subsiguientes
    \pagecolor{white}       % Restaurar el color de página
    
    
    \newpage
    \thispagestyle{empty}               % Sin encabezado ni pie de página
    \begin{tikzpicture}[remember picture, overlay]
        \node[anchor=south west, inner sep=3cm] at (current page.south west) {
            \begin{minipage}{0.5\paperwidth}
                \href{https://creativecommons.org/licenses/by-nc-nd/4.0/}{
                    \includegraphics[height=2cm]{assets/Licencia.png}
                }\vspace{1cm}\\
                Esta obra está bajo una
                \href{https://creativecommons.org/licenses/by-nc-nd/4.0/}{
                    Licencia Creative Commons Atribución-NoComercial-SinDerivadas 4.0 Internacional (CC BY-NC-ND 4.0).
                }\\
    
                Eres libre de compartir y redistribuir el contenido de esta obra en cualquier medio o formato, siempre y cuando des el crédito adecuado a los autores originales y no persigas fines comerciales. 
            \end{minipage}
        };
    \end{tikzpicture}
    
    
    
    % 1. Foto de fondo
    % 2. Título
    % 3. Encabezado Izquierdo
    % 4. Color de fondo
    % 5. Coord x del titulo
    % 6. Coord y del titulo
    % 7. Fecha


}


\newcommand{\portadaBook}[7]{

    % 1. Foto de fondo
    % 2. Título
    % 3. Encabezado Izquierdo
    % 4. Color de fondo
    % 5. Coord x del titulo
    % 6. Coord y del titulo
    % 7. Fecha

    % Personaliza el formato del título
    \pretitle{\begin{center}\bfseries\fontsize{42}{56}\selectfont}
    \posttitle{\par\end{center}\vspace{2em}}
    
    % Personaliza el formato del autor
    \preauthor{\begin{center}\Large}
    \postauthor{\par\end{center}\vfill}
    
    % Personaliza el formato de la fecha
    \predate{\begin{center}\huge}
    \postdate{\par\end{center}\vspace{2em}}
    
    \title{#2}
    \author{\href{https://losdeldgiim.github.io/}{Los Del DGIIM}}
    \date{Granada, #7}
    \maketitle
    
    \tableofcontents
}




\newcommand{\portadaArticle}[7]{

    % 1. Foto de fondo
    % 2. Título
    % 3. Encabezado Izquierdo
    % 4. Color de fondo
    % 5. Coord x del titulo
    % 6. Coord y del titulo
    % 7. Fecha

    % Personaliza el formato del título
    \pretitle{\begin{center}\bfseries\fontsize{42}{56}\selectfont}
    \posttitle{\par\end{center}\vspace{2em}}
    
    % Personaliza el formato del autor
    \preauthor{\begin{center}\Large}
    \postauthor{\par\end{center}\vspace{3em}}
    
    % Personaliza el formato de la fecha
    \predate{\begin{center}\huge}
    \postdate{\par\end{center}\vspace{5em}}
    
    \title{#2}
    \author{\href{https://losdeldgiim.github.io/}{Los Del DGIIM}}
    \date{Granada, #7}
    \thispagestyle{empty}               % Sin encabezado ni pie de página
    \maketitle
    \vfill
}
    \portadaExamen{ffccA4.jpg}{Geometría I\\Examen XIV}{Geometría I. Examen XIV}{MidnightBlue}{-8}{28}{2025}{Víctor Naranjo Cabrera}

    \begin{description}
        \item[Asignatura] Geometría I.
        \item[Curso Académico] 2024-25.
        \item[Grado] Doble Grado de Ingeniería Informática y Matemáticas.
        \item[Grupo] Único.
        \item[Profesor] Ana María Hurtado Cortegana y Antonio Ros Mulero.
        \item[Descripción] Convocatoria ordinaria.
        \item[Fecha] 17 de enero de 2025.    
    \end{description}
    \newpage


    % ------------------------------------
    
    \begin{ejercicio}[2.5 puntos] En el espacio $\R^4$, con vectores $(x, y, z, t)$, y para todo $a$ real consideramos los subespacios vectoriales definidos, uno por sistema de generadores y otro por ecuaciones implícitas,
    \begin{equation*}
        U = \mathcal{L}(\{(0, 1, 1, 1), (1, 0, -a, -1)\}), W=\left\{(x, y, z, t) \in \R^4 \left|\begin{array}{rr}
             & ax-y + t = 0 \\
             & x + z = 0 \\
        \end{array}  \right. \text{ } \right\}
    \end{equation*}
    \begin{enumerate}[label=(\alph*)]        \item Calcula, para todo $a$, unas ecuaciones implícitas de $U$ y una base de $W$.
    \item Estudia si para $a = 1$ se verifica o no la descomposición en suma directa $\R^4 = U \oplus W.$ 
    
    \end{enumerate}
    \end{ejercicio}
    
    \begin{ejercicio}[2.5 puntos] Sea $f : \mathcal{M}_2(\R) \rightarrow \R^3$ la aplicación lineal dada por
    \begin{equation*}
        f\begin{pmatrix}
            \ a & b \ \\
            \ c & d \ \\
        \end{pmatrix} = (b-c , a+d, a-d)
    \end{equation*}
    \begin{enumerate}[label=(\alph*)]
        \item Calcula el núcleo y la imagen de $f$.
        \item Halla las coordenadas respecto de la base dual de la base canónica de $\mathcal{M}_2(\R)$ de $f^t(\varphi),$ donde $\varphi$ es la forma lineal de $(\R^3)^*$ dada por $\varphi(x, y, z) = x + y + z$.
    \end{enumerate}
    \end{ejercicio}

    \begin{ejercicio}[2.5 puntos]
        Sea $V$ un espacio vectorial sobre un cuerpo $\K$ de dimensión finita.
        \begin{enumerate}[label=(\alph*)]
        \item Prueba que si $f \in \text{End}(V)$ tal que $f \circ f = f \Rightarrow V = \ker(f) \oplus \text{Im}(f).$
        \item Dado $v \in V \setminus \{0\}$ prueba que existe $f \in \text{End}(V)$ tal que $f \circ f = f$ y $\text{ker}(f^t) = \text{an}(\{v\})$.
        \end{enumerate}
    \end{ejercicio}

    \begin{ejercicio}[2.5 puntos]
        Decide de forma razonada si las siguientes afirmaciones son verdaderas o falsas:
        \begin{enumerate}[label=(\alph*)]
            \item Para toda matriz antisimétrica de orden impar con coeficientes en $\R$ existe una fila que es combinación lineal del resto.
            \item Sean $V$ un espacio vectorial sobre un cuerpo $\K$ de dimensión finita mayor o igual que 2 y $v \in V \setminus \{\vec{0}\}.$ Entonces existen dos bases distintas $\mathcal{B}$ y $\mathcal{B'}$ tales que $v_{\mathcal{B}} = v_{\mathcal{B'}}$. ¿Es cierta la afirmación para cualesquiera par de bases de $V$?
            \item Sean $V$ y $V'$ espacios vectoriales sobre un cuerpo $\K$ de dimensión finita y $f: V \rightarrow V'$ una aplicación lineal. Entonces, $f$ es inyectiva $\Leftrightarrow f^t$ es sobreyectiva. 
        \end{enumerate}
    \end{ejercicio}

    \newpage
    \setcounter{ejercicio}{0} % Reseteo de contador para ejercicios resueltos

\begin{ejercicio}[2.5 puntos] En el espacio $\R^4$, con vectores $(x, y, z, t)$, y para todo $a$ real consideramos los subespacios vectoriales definidos, uno por sistema de generadores y otro por ecuaciones implícitas,
    \begin{equation*}
        U = \mathcal{L}(\{(0, 1, 1, 1), (1, 0, -a, -1)\}), W=\left\{(x, y, z, t) \in \R^4 \left|\begin{array}{rr}
             & ax-y + t = 0 \\
             & x + z = 0 \\
        \end{array}  \right. \text{ } \right\}
    \end{equation*}
    \begin{enumerate}[label=(\alph*)]        \item Calcula, para todo $a$, unas ecuaciones implícitas de $U$ y una base de $W$. \\\\
    Un vector $(x, y, z, t) \in \R^4$ está en $U$ si y sólo si es combinación lineal de $v_1 := (0, 1, 1, 1), v_2 := (1, 0, -a, -1),$ es decir, si y sólo si la matriz
    \begin{equation*}
        \begin{pmatrix}
            0 & 1 & 1 & 1 \\
            1 & 0 & -a & -1 \\
            x & y & z & t \\
        \end{pmatrix}
    \end{equation*}
    tiene rango 2. Vemos que $\begin{vmatrix}
        0 & 1 \\
        1 & 0 \\
    \end{vmatrix} \neq 0$, así que vamos orlando el menor anterior con la tercera fila y las columnas tercera y cuarta, obteniendo las dos ecuaciones implícitas de $U$ respecto de la base usual de $\R^4$:
    \begin{equation*}
        0  = \begin{vmatrix}
            0 & 1 & 1 \\
            1  & 0 & -a \\
            x & y & z \\
        \end{vmatrix} = -ax  + y -z, 
        \ 0 = \begin{vmatrix}
            0 & 1 & 1 \\
            1 & 0 & -1 \\
            x & y & t \\
        \end{vmatrix} = -x + y - t
    \end{equation*}
    Para calcular una base de $W$, solucionamos el sistema homogéneo dado por sus ecuaciones implícitas. La matriz de coeficientes de este sistema homogéneo es
    \begin{equation*}
        \begin{pmatrix}
            a & -1 & 0 & 1 \\
            1 & 0 & 1 & 0 \\
         \end{pmatrix}.
    \end{equation*}
    Como el menor $\begin{vmatrix}
        -1 & 0 \\
        0 & 1 \\
    \end{vmatrix} \neq 0$, tomamos $y, z$ como incógnitas principales y $x, t$ como incógnitas secundarias (parámetros), es decir, escribimos el sistema como
    \begin{equation*}
        \left.\begin{array}{lll}
            y &  & = ax+t \\
             & z & = -x 
        \end{array}\right\}
    \end{equation*}
    Dando valores a los parámetros obtenemos: si $(x, t) = (1, 0)$, entonces $(y, z) = (a, -1);$ y si $(x, t) = (0, 1),$ entonces $(y, z) = (1, 0).$ Por tanto, los vectores
    \begin{equation*}
        v_3 := (1, a, -1, 0), \ v_4:= (0, 1, 0, 1)
    \end{equation*}
    forman base de $W$.
    \item Estudia si para $a = 1$ se verifica o no la descomposición en suma directa $\R^4 = U \oplus W.$ \\\\
    Para $a = 1,$ los vectores que hemos calculado en el apartado anterior son
    \begin{equation*}
        v_1 := (0, 1, 1, 1), \ v_2 := (1, 0, -1, -1), \ v_3 := (1, 1, -1, 0), \ 
        v_4 := (0, 1, 0, 1).
    \end{equation*}
    Como
    \begin{equation*}
        \begin{vmatrix}
            0 & 1 & 1 & 1 \\
            1 & 0 & -1 & -1 \\
            1 & 1 & -1 & 0 \\
            0 & 1 & 0 & 1 \\
        \end{vmatrix} = 0
    \end{equation*}
    concluimos que $v_1, v_2, v_3, v_4$ son linealmente dependientes. Así que no pueden formar base, lo que impide que $\R^4$ sea suma directa de $U$ y $W$. Como $U$ y $W$ tienen dimensión $2$, tenemos $U \cap W \neq \{0\}$ y $U + W \neq \R^4.$
    \end{enumerate}
    \end{ejercicio}
    
    \begin{ejercicio}[2.5 puntos] Sea $f : \mathcal{M}_2(\R) \rightarrow \R^3$ la aplicación lineal dada por
    \begin{equation*}
        f\begin{pmatrix}
            \ a & b \ \\
            \ c & d \ \\
        \end{pmatrix} = (b-c, a+d, a-d)
    \end{equation*}
    \begin{enumerate}[label=(\alph*)]
        \item Calcula el núcleo y la imagen de $f$. \\\\
        Calculamos primero $ker(f)$:
        \begin{gather*}
            ker(f) = \left\{
            \begin{pmatrix}
                a & b \\
                c & d \\
            \end{pmatrix} \in \mathcal{M}_2(\R) : (b-c, a+d, a-d) = (0, 0, 0)\right\} \\
            = \left\{
            \begin{pmatrix}
                a & b \\
                c & d \\
            \end{pmatrix} \in \mathcal{M}_2(\R) : b = c, a = d = 0\right\} = \mathcal{L}\left(\left\{\begin{pmatrix}
                0 & 1 \\
                1 & 0 \\
            \end{pmatrix} \right\}\right).
        \end{gather*}
        Podemos calcular $\text{Im}(f)$ de dos maneras:
        \begin{description}
            \item[Fórmula de la nulidad y el rango.] Como $\dim_\K\ker(f) \ + \ \dim \text{Im}(f) = \dim_\K V$,  $\dim_\K\ker(f) = 1 \Rightarrow 4 = 1 + \dim \text{Im}(f)$, luego $\dim \text{Im}(f) = 3$ y por tanto, $f$ es sobreyectiva. Así que la imagen de $f$ es $\R^3$.
            \item[Formando una base de Im$(f)$] Llamamos $E_{ij}$ a las matrices de la base usual de $\mathcal{M}_2(\R).$ Así, $f(E_{11}) = (0, 1, 1), f(E_{12}) = (1, 0, 0), f(E_{21}) = (-1, 0, 0),$ $f(E_{22}) = (0, 1, -1)$, luego
            \begin{gather*}
                \text{Im}(f) = \mathcal{L}(\{f(E_{11}), f(E_{12}), f(E_{21}), f(E_{22})\}) \\ 
                = \mathcal{L}(\{(0, 1, 1), (1, 0, 0), (-1, 0, 0), (0, 1, -1)\})  \\
                = \mathcal{L}(\{(0, 1, 1), (1, 0, 0), (-1, 0, 0), (0, 1, -1)\})   = \R^3,
            \end{gather*}
            donde en la última igualdad hemos usado que los vectores $(0, 1, 1), (1, 0, 0),$ $(0, 1, -1)$ son linealmente independientes, porque
            \begin{equation*}
                \begin{vmatrix}
                    0 & 1 & 1 \\
                    1 & 0 & 0 \\
                    0 & 1 & -1 \\
                \end{vmatrix} = 2 \neq 0.
            \end{equation*}
        \end{description}
        \item Halla las coordenadas respecto de la base dual de la base canónica de $\mathcal{M}_2(\R)$ de $f^t(\varphi),$ donde $\varphi$ es la forma lineal de $(\R^3)^*$ dada por $\varphi(x, y, z) = x + y + z$. \\\\
        Llamemos $\mathcal{B} = (E_{11}, E_{12}, E_{21}, E_{22})$ a la base ordenada usual de $\mathcal{M}_2(\R)$ y $\mathcal{B^*} = (\varphi_{11}, \varphi_{12}, \varphi_{21}, \varphi_{22})$ a su base ordenada dual. Entonces, las coordenadas que nos piden son
        \begin{equation*}
            \begin{array}{lll}
            (f^t(\varphi))_{\mathcal{B^*}} & = ([f^t(\varphi)](E_{11}),[f^t(\varphi)](E_{12}),
            [f^t(\varphi)](E_{21}),
            [f^t(\varphi)](E_{22})) \\
             & = ((\varphi \circ f)(E_{11}),(\varphi \circ f)(E_{12}),
            (\varphi \circ f)(E_{21}),
            (\varphi \circ f)(E_{22})) \\
             & =  (\varphi(0, 1, 1),  \varphi(1, 0, 0), \varphi(-1, 0, 0), \varphi(0, 1, -1)) = (2, 1, -1, 0). \\
            \end{array}
        \end{equation*}
    \end{enumerate}
    \end{ejercicio}

    \begin{ejercicio}[2.5 puntos]
        Sea $V$ un espacio vectorial sobre un cuerpo $\K$ de dimensión finita.
        \begin{enumerate}[label=(\alph*)]
        \item Prueba que si $f \in \text{End}(V)$ tal que $f \circ f = f \Rightarrow V = \ker(f) \oplus \text{Im}(f).$ \\\\
        Veamos que $ \ker(f) \cap \text{Im}(f) = \{0\}:$ sea $x \in \ker(f) \cap \text{Im}(f).$ Como $x \in \text{Im}(f)$ existe $z \in V$ tal que $x = f(z)$. Como $x \in \ker(f),$ tenemos
        \begin{equation*}
            0 = f(x) = f(f(z)) = (f \circ f)(z) = f(z) = x.
        \end{equation*}
        Por tanto, $x = 0$, y así, $\ker(f) \cap \text{Im}(f) = \{0\}.$
        Por la fórmula de la nulidad y el rango, concluimos que $V = \ker(f) \oplus \text{Im}(f)$.
        \item Dado $v \in V \setminus \{0\}$ prueba que existe $f \in \text{End}(V)$ tal que $f \circ f = f$ y $\ker(f^t) = \text{an}(\{v\})$. \\\\
        Sabemos que 
        \begin{equation*}
            \ker(f^t) = \text{an}(\{v\}) = \text{an(Im}(f)) \Rightarrow \text{Im}(f) = \text{an(an}(\{v\})) = \mathcal{L}(\{v\}).
        \end{equation*}
        Entonces, sea $n = \dim_\K V$, tomemos $x_1,\dotsc,x_{n-1}$ de un subespacio suplementario de $\mathcal{L}(\{v\})$ en $V$, tal que $\mathcal{B} = (x_1,\dotsc, x_{n-1}, v)$ es base ordenada de $V$. Sea $f$ el único endomorfismo dado por el teorema fundamental de las aplicaciones lineales a partir de
        \begin{equation*}
            f(x_i) = 0, \forall i = 1,\dotsc, n-1, f(v) = v.
        \end{equation*}
        Entonces
        \begin{equation*}
            \text{Im}(f) = \mathcal{L}(\{f(x_1),\dotsc, f(x_{n-1}), f(v)\}) = \mathcal{L}(\{0, \dotsc, 0, v\}) = \mathcal{L}(\{v\})
        \end{equation*}
        y por tanto $\ker(f^t) = \text{an(Im}(f)) = \text{an}(\mathcal{L}(\{v\})) = \text{an}(\{v\}),$ mientras que $f \circ f = f$ ya que $f \circ f$ y $f$ coinciden sobre los vectores de la base $\mathcal{B}$ 
        \end{enumerate}
    \end{ejercicio}

    \begin{ejercicio}
        Decide de forma razonada si las siguientes afirmaciones son verdaderas o falsas:
        \begin{enumerate}[label=(\alph*)]
            \item Para toda matriz antisimétrica de orden impar con coeficientes en $\R$ existe una fila que es combinación lineal del resto. \\\\
            Verdadero: Sea $A \in \mathcal{M}_n(\R)$ antisimétrica, es decir $A^t = -A.$ Tomando determinantes
            \begin{equation*}
                det(A) = det(A^t) = det(-A) = (-1)^ndet(A).
            \end{equation*}
             Como estamos suponiendo $n$ impar, tenemos $det(A) = -det(A),$ luego $det(A) = 0.$ Esto equivale a que el rango de $A$ por filas es menor que $n$, es decir, al menos una fila de $A$ es combinación lineal del resto.
            \item Sean $V$ un espacio vectorial sobre un cuerpo $K$ de dimensión finita mayor o igual que 2 y $v \in V \setminus \{\vec{0}\}.$ Entonces existen dos bases distintas $\mathcal{B}$ y $\mathcal{B'}$ tales que $v_{\mathcal{B}} = v_{\mathcal{B'}}$. ¿Es cierta la afirmación para cualesquiera par de bases de $V$? \\\\
            
            La primera afirmación es verdadera: Como $v \neq 0,$ el conjunto $\{v\}$ es linealmente independiente. Por el Teorema de Ampliación de la Base, existe una base ordenada de $V$ del tipo $\mathcal{B} = (v, v_2, \dotsc, v_n)$ para vectores $v_2,\dotsc,v_n \in V$. Tomemos $\mathcal{B'} = (v, v_2, ..., v_{n-1}, 2v_n).$, también base ordenada de V. Vemos entonces que $v_{\mathcal{B}} = (1, 0, ..., 0) = v_{\mathcal{B'}}$.
            La segunda afirmación es falsa: 
            Consideremos las bases ordenadas $\mathcal{B} = (v, v_2,..., v_n), \mathcal{B}'' = (v_2, v, v_3,..., v_n)$ Entonces
            \begin{equation*}
                v_\mathcal{B} = (1, 0, \dotsc, 0) \neq (0, 1, 0, \dotsc, 0) = v_{\mathcal{B}''}.
            \end{equation*}
            Y no es cierta para cualesquiera par de bases de $V$.
            \item Sean $V$ y $V'$ espacios vectoriales sobre un cuerpo $\K$ de dimensión finita y $f: V \rightarrow V'$ una aplicación lineal. Entonces, $f$ es inyectiva $\Leftrightarrow f^t$ es sobreyectiva. \\\\
            
            La afirmación es verdadera:
            \begin{description}
                \item[$\underline{\Rightarrow)}$] 
                $f$ es inyectiva si y sólo si $n(f) = 0.$ Por la fórmula de la nulidad y el rango aplicada a $f$, esto equivale a que $\dim_\K V = r(f)$. Por el Teorema del Rango, $r(f) = r(f^t)$, luego $dim_\K V = r(f^t)$ Como $\dim_\K V = \dim_\K V^*$, lo anterior equivale a que $\dim_\K V^* = r(f^t)$, luego $f^t$ es sobreyectiva.

                \item[$\underline{\Leftarrow)}$]
                $f^t: (V')^* \rightarrow V^*$ es sobreyectiva si y solo si $\text{Im}(f^t) = V^* = \text{an}(\ker(f))$ Tomando anuladores, esto equivale a que $\ker(f) = \text{an}(V^*) = \{0\}$, es decir, a que $f$ es inyectiva.
            \end{description}
        \end{enumerate}
    \end{ejercicio}

    
\end{document}