\documentclass[12pt]{article}

% Idioma y codificación
\usepackage[spanish, es-tabla]{babel}       %es-tabla para que se titule "Tabla"
\usepackage[utf8]{inputenc}

% Márgenes
\usepackage[a4paper,top=3cm,bottom=2.5cm,left=3cm,right=3cm]{geometry}

% Comentarios de bloque
\usepackage{verbatim}

% Paquetes de links
\usepackage[hidelinks]{hyperref}    % Permite enlaces
\usepackage{url}                    % redirecciona a la web

% Más opciones para enumeraciones
\usepackage{enumitem}

% Personalizar la portada
\usepackage{titling}

% Paquetes de tablas
\usepackage{multirow}


%------------------------------------------------------------------------

%Paquetes de figuras
\usepackage{caption}
\usepackage{subcaption} % Figuras al lado de otras
\usepackage{float}      % Poner figuras en el sitio indicado H.


% Paquetes de imágenes
\usepackage{graphicx}       % Paquete para añadir imágenes
\usepackage{transparent}    % Para manejar la opacidad de las figuras

% Paquete para usar colores
\usepackage[dvipsnames]{xcolor}
\usepackage{pagecolor}      % Para cambiar el color de la página

% Habilita tamaños de fuente mayores
\usepackage{fix-cm}

% Para los gráficos
\usepackage{tikz}

% Para poder situar los nodos en los grafos
\usetikzlibrary{positioning}


%------------------------------------------------------------------------

% Paquetes de matemáticas
\usepackage{mathtools, amsfonts, amssymb, mathrsfs}
\usepackage[makeroom]{cancel}     % Simplificar tachando
\usepackage{polynom}    % Divisiones y Ruffini
\usepackage{units} % Para poner fracciones diagonales con \nicefrac

\usepackage{pgfplots}   %Representar funciones
\pgfplotsset{compat=1.18}  % Versión 1.18

\usepackage{tikz-cd}    % Para usar diagramas de composiciones
\usetikzlibrary{calc}   % Para usar cálculo de coordenadas en tikz

%Definición de teoremas, etc.
\usepackage{amsthm}
%\swapnumbers   % Intercambia la posición del texto y de la numeración

\theoremstyle{plain}

\makeatletter
\@ifclassloaded{article}{
  \newtheorem{teo}{Teorema}[section]
}{
  \newtheorem{teo}{Teorema}[chapter]  % Se resetea en cada chapter
}
\makeatother

\newtheorem{coro}{Corolario}[teo]           % Se resetea en cada teorema
\newtheorem{prop}[teo]{Proposición}         % Usa el mismo contador que teorema
\newtheorem{lema}[teo]{Lema}                % Usa el mismo contador que teorema

\theoremstyle{remark}
\newtheorem*{observacion}{Observación}

\theoremstyle{definition}

\makeatletter
\@ifclassloaded{article}{
  \newtheorem{definicion}{Definición} [section]     % Se resetea en cada chapter
}{
  \newtheorem{definicion}{Definición} [chapter]     % Se resetea en cada chapter
}
\makeatother

\newtheorem*{notacion}{Notación}
\newtheorem*{ejemplo}{Ejemplo}
\newtheorem*{ejercicio*}{Ejercicio}             % No numerado
\newtheorem{ejercicio}{Ejercicio} [section]     % Se resetea en cada section


% Modificar el formato de la numeración del teorema "ejercicio"
\renewcommand{\theejercicio}{%
  \ifnum\value{section}=0 % Si no se ha iniciado ninguna sección
    \arabic{ejercicio}% Solo mostrar el número de ejercicio
  \else
    \thesection.\arabic{ejercicio}% Mostrar número de sección y número de ejercicio
  \fi
}


% \renewcommand\qedsymbol{$\blacksquare$}         % Cambiar símbolo QED
%------------------------------------------------------------------------

% Paquetes para encabezados
\usepackage{fancyhdr}
\pagestyle{fancy}
\fancyhf{}

\newcommand{\helv}{ % Modificación tamaño de letra
\fontfamily{}\fontsize{12}{12}\selectfont}
\setlength{\headheight}{15pt} % Amplía el tamaño del índice


%\usepackage{lastpage}   % Referenciar última pag   \pageref{LastPage}
\fancyfoot[C]{\thepage}

%------------------------------------------------------------------------

% Conseguir que no ponga "Capítulo 1". Sino solo "1."
\makeatletter
\@ifclassloaded{book}{
  \renewcommand{\chaptermark}[1]{\markboth{\thechapter.\ #1}{}} % En el encabezado
    
  \renewcommand{\@makechapterhead}[1]{%
  \vspace*{50\p@}%
  {\parindent \z@ \raggedright \normalfont
    \ifnum \c@secnumdepth >\m@ne
      \huge\bfseries \thechapter.\hspace{1em}\ignorespaces
    \fi
    \interlinepenalty\@M
    \Huge \bfseries #1\par\nobreak
    \vskip 40\p@
  }}
}
\makeatother

%------------------------------------------------------------------------
% Paquetes de cógido
\usepackage{minted}
\renewcommand\listingscaption{Código fuente}

\usepackage{fancyvrb}
% Personaliza el tamaño de los números de línea
\renewcommand{\theFancyVerbLine}{\small\arabic{FancyVerbLine}}

% Estilo para C++
\newminted{cpp}{
    frame=lines,
    framesep=2mm,
    baselinestretch=1.2,
    linenos,
    escapeinside=||
}

% para minted
\definecolor{LightGray}{rgb}{0.95,0.95,0.92}
\setminted{
    linenos=true,
    stepnumber=5,
    numberfirstline=true,
    autogobble,
    breaklines=true,
    breakautoindent=true,
    breaksymbolleft=,
    breaksymbolright=,
    breaksymbolindentleft=0pt,
    breaksymbolindentright=0pt,
    breaksymbolsepleft=0pt,
    breaksymbolsepright=0pt,
    fontsize=\footnotesize,
    bgcolor=LightGray,
    numbersep=10pt
}


\usepackage{listings} % Para incluir código desde un archivo

\renewcommand\lstlistingname{Código Fuente}
\renewcommand\lstlistlistingname{Índice de Códigos Fuente}

% Definir colores
\definecolor{vscodepurple}{rgb}{0.5,0,0.5}
\definecolor{vscodeblue}{rgb}{0,0,0.8}
\definecolor{vscodegreen}{rgb}{0,0.5,0}
\definecolor{vscodegray}{rgb}{0.5,0.5,0.5}
\definecolor{vscodebackground}{rgb}{0.97,0.97,0.97}
\definecolor{vscodelightgray}{rgb}{0.9,0.9,0.9}

% Configuración para el estilo de C similar a VSCode
\lstdefinestyle{vscode_C}{
  backgroundcolor=\color{vscodebackground},
  commentstyle=\color{vscodegreen},
  keywordstyle=\color{vscodeblue},
  numberstyle=\tiny\color{vscodegray},
  stringstyle=\color{vscodepurple},
  basicstyle=\scriptsize\ttfamily,
  breakatwhitespace=false,
  breaklines=true,
  captionpos=b,
  keepspaces=true,
  numbers=left,
  numbersep=5pt,
  showspaces=false,
  showstringspaces=false,
  showtabs=false,
  tabsize=2,
  frame=tb,
  framerule=0pt,
  aboveskip=10pt,
  belowskip=10pt,
  xleftmargin=10pt,
  xrightmargin=10pt,
  framexleftmargin=10pt,
  framexrightmargin=10pt,
  framesep=0pt,
  rulecolor=\color{vscodelightgray},
  backgroundcolor=\color{vscodebackground},
}

%------------------------------------------------------------------------

% Comandos definidos
\newcommand{\bb}[1]{\mathbb{#1}}
\newcommand{\cc}[1]{\mathcal{#1}}

% I prefer the slanted \leq
\let\oldleq\leq % save them in case they're every wanted
\let\oldgeq\geq
\renewcommand{\leq}{\leqslant}
\renewcommand{\geq}{\geqslant}

% Si y solo si
\newcommand{\sii}{\iff}

% Letras griegas
\newcommand{\eps}{\epsilon}
\newcommand{\veps}{\varepsilon}
\newcommand{\lm}{\lambda}

\newcommand{\ol}{\overline}
\newcommand{\ul}{\underline}
\newcommand{\wt}{\widetilde}
\newcommand{\wh}{\widehat}

\let\oldvec\vec
\renewcommand{\vec}{\overrightarrow}

% Derivadas parciales
\newcommand{\del}[2]{\frac{\partial #1}{\partial #2}}
\newcommand{\Del}[3]{\frac{\partial^{#1} #2}{\partial #3^{#1}}}
\newcommand{\deld}[2]{\dfrac{\partial #1}{\partial #2}}
\newcommand{\Deld}[3]{\dfrac{\partial^{#1} #2}{\partial #3^{#1}}}


\newcommand{\AstIg}{\stackrel{(\ast)}{=}}
\newcommand{\Hop}{\stackrel{L'H\hat{o}pital}{=}}

\newcommand{\red}[1]{{\color{red}#1}} % Para integrales, destacar los cambios.

% Método de integración
\newcommand{\MetInt}[2]{
    \left[\begin{array}{c}
        #1 \\ #2
    \end{array}\right]
}

% Declarar aplicaciones
% 1. Nombre aplicación
% 2. Dominio
% 3. Codominio
% 4. Variable
% 5. Imagen de la variable
\newcommand{\Func}[5]{
    \begin{equation*}
        \begin{array}{rrll}
            #1:& #2 & \longrightarrow & #3\\
               & #4 & \longmapsto & #5
        \end{array}
    \end{equation*}
}

%------------------------------------------------------------------------


\begin{document}
	
	% 1. Foto de fondo
	% 2. Título
	% 3. Encabezado Izquierdo
	% 4. Color de fondo
	% 5. Coord x del titulo
	% 6. Coord y del titulo
	% 7. Fecha
	\newcommand{\R}{{\mathbb{R}}} % Reales
	\newcommand{\K}{{\mathbb{K}}} % Cuerpo arbitrario
	
	% 1. Foto de fondo
% 2. Título
% 3. Encabezado Izquierdo
% 4. Color de fondo
% 5. Coord x del titulo
% 6. Coord y del titulo
% 7. Fecha

\newcommand{\portada}[7]{

    \portadaBase{#1}{#2}{#3}{#4}{#5}{#6}{#7}
    \portadaBook{#1}{#2}{#3}{#4}{#5}{#6}{#7}
}

\newcommand{\portadaExamen}[7]{

    \portadaBase{#1}{#2}{#3}{#4}{#5}{#6}{#7}
    \portadaArticle{#1}{#2}{#3}{#4}{#5}{#6}{#7}
}




\newcommand{\portadaBase}[7]{

    % Tiene la portada principal y la licencia Creative Commons
    
    % 1. Foto de fondo
    % 2. Título
    % 3. Encabezado Izquierdo
    % 4. Color de fondo
    % 5. Coord x del titulo
    % 6. Coord y del titulo
    % 7. Fecha
    
    
    \thispagestyle{empty}               % Sin encabezado ni pie de página
    \newgeometry{margin=0cm}        % Márgenes nulos para la primera página
    
    
    % Encabezado
    \fancyhead[L]{\helv #3}
    \fancyhead[R]{\helv \nouppercase{\leftmark}}
    
    
    \pagecolor{#4}        % Color de fondo para la portada
    
    \begin{figure}[p]
        \centering
        \transparent{0.3}           % Opacidad del 30% para la imagen
        
        \includegraphics[width=\paperwidth, keepaspectratio]{assets/#1}
    
        \begin{tikzpicture}[remember picture, overlay]
            \node[anchor=north west, text=white, opacity=1, font=\fontsize{60}{90}\selectfont\bfseries\sffamily, align=left] at (#5, #6) {#2};
            
            \node[anchor=south east, text=white, opacity=1, font=\fontsize{12}{18}\selectfont\sffamily, align=right] at (9.7, 3) {\textbf{\href{https://losdeldgiim.github.io/}{Los Del DGIIM}}};
            
            \node[anchor=south east, text=white, opacity=1, font=\fontsize{12}{15}\selectfont\sffamily, align=right] at (9.7, 1.8) {Doble Grado en Ingeniería Informática y Matemáticas\\Universidad de Granada};
        \end{tikzpicture}
    \end{figure}
    
    
    \restoregeometry        % Restaurar márgenes normales para las páginas subsiguientes
    \pagecolor{white}       % Restaurar el color de página
    
    
    \newpage
    \thispagestyle{empty}               % Sin encabezado ni pie de página
    \begin{tikzpicture}[remember picture, overlay]
        \node[anchor=south west, inner sep=3cm] at (current page.south west) {
            \begin{minipage}{0.5\paperwidth}
                \href{https://creativecommons.org/licenses/by-nc-nd/4.0/}{
                    \includegraphics[height=2cm]{assets/Licencia.png}
                }\vspace{1cm}\\
                Esta obra está bajo una
                \href{https://creativecommons.org/licenses/by-nc-nd/4.0/}{
                    Licencia Creative Commons Atribución-NoComercial-SinDerivadas 4.0 Internacional (CC BY-NC-ND 4.0).
                }\\
    
                Eres libre de compartir y redistribuir el contenido de esta obra en cualquier medio o formato, siempre y cuando des el crédito adecuado a los autores originales y no persigas fines comerciales. 
            \end{minipage}
        };
    \end{tikzpicture}
    
    
    
    % 1. Foto de fondo
    % 2. Título
    % 3. Encabezado Izquierdo
    % 4. Color de fondo
    % 5. Coord x del titulo
    % 6. Coord y del titulo
    % 7. Fecha


}


\newcommand{\portadaBook}[7]{

    % 1. Foto de fondo
    % 2. Título
    % 3. Encabezado Izquierdo
    % 4. Color de fondo
    % 5. Coord x del titulo
    % 6. Coord y del titulo
    % 7. Fecha

    % Personaliza el formato del título
    \pretitle{\begin{center}\bfseries\fontsize{42}{56}\selectfont}
    \posttitle{\par\end{center}\vspace{2em}}
    
    % Personaliza el formato del autor
    \preauthor{\begin{center}\Large}
    \postauthor{\par\end{center}\vfill}
    
    % Personaliza el formato de la fecha
    \predate{\begin{center}\huge}
    \postdate{\par\end{center}\vspace{2em}}
    
    \title{#2}
    \author{\href{https://losdeldgiim.github.io/}{Los Del DGIIM}}
    \date{Granada, #7}
    \maketitle
    
    \tableofcontents
}




\newcommand{\portadaArticle}[7]{

    % 1. Foto de fondo
    % 2. Título
    % 3. Encabezado Izquierdo
    % 4. Color de fondo
    % 5. Coord x del titulo
    % 6. Coord y del titulo
    % 7. Fecha

    % Personaliza el formato del título
    \pretitle{\begin{center}\bfseries\fontsize{42}{56}\selectfont}
    \posttitle{\par\end{center}\vspace{2em}}
    
    % Personaliza el formato del autor
    \preauthor{\begin{center}\Large}
    \postauthor{\par\end{center}\vspace{3em}}
    
    % Personaliza el formato de la fecha
    \predate{\begin{center}\huge}
    \postdate{\par\end{center}\vspace{5em}}
    
    \title{#2}
    \author{\href{https://losdeldgiim.github.io/}{Los Del DGIIM}}
    \date{Granada, #7}
    \thispagestyle{empty}               % Sin encabezado ni pie de página
    \maketitle
    \vfill
}
	\portadaExamen{ffccA4.jpg}{Geometría I\\Examen XV}{Geometría I. Examen XV}{MidnightBlue}{-8}{28}{2025}{Roxana Acedo Parra}
	
	\begin{description}
		\item[Asignatura] Geometría I.
		\item[Curso Académico] 2024-25.
		\item[Grado] Doble Grado de Ingeniería Informática y Matemáticas.
		\item[Grupo] Único.
		\item[Profesor] Ana María Hurtado Cortegana y Antonio Ros Mulero.
		\item[Descripción] Convocatoria extraordinaria.
		\item[Fecha] 10 de febrero de 2025.    
	\end{description}
	\newpage
	
	\begin{ejercicio}[2.5 puntos] Dado $a \in \R$, consideremos el subespacio de $\R^4$.
		\begin{equation*}
			U_a := \cc{L}(\{(0, -1, a, 3), (2-a, 1, 2, -3), (0, a, -2-a, 3)\})
		\end{equation*}
		
		\begin{enumerate} 
			\item Calcula la dimensión de $U_a$ en función de los valores de $a$. Determina una base y ecuaciones
			implícitas de $U_a$ para cada $a \in \R$.
			\item Para $a$ satisfaciendo $\dim_{\R} U_a = 2$, encuentra un subespacio $W$ de $\R^4$ tal que $\R^4 = U_a\oplus W$. Determina las ecuaciones implícitas de $W$.
		\end{enumerate}
	\end{ejercicio}
		
	\begin{ejercicio}[2.5 puntos] Sea $f : \mathcal{M}_2(\R) \rightarrow \R^3$ la aplicación lineal dada por:
		\begin{equation*}
			f{\tiny \begin{pmatrix}
				\ 0 & 1 \ \\
				\ 1 & 0 \ \\
			\end{pmatrix}} = 1+x^2, \quad
			f{\tiny \begin{pmatrix}
				\ 1 & 1 \ \\
				\ 0 & 0 \ \\
			\end{pmatrix}} = 2-x+x^2, \quad	
			f{\tiny \begin{pmatrix}
				\ 0 & 0 \ \\
				\ 0 & 1 \ \\
			\end{pmatrix}} = 1+x^2, \quad
			f{\tiny \begin{pmatrix}
				\ 1 & -1 \ \\
				\ 0 & 0 \ \\
			\end{pmatrix}} =2-3x-x^2										
		\end{equation*}
		
		\begin{enumerate} 
			\item Calcula la matriz asociada a $f$ respecto de las bases usuales de $\cc{M}_2(\R)$ y de $\R_2\left[x\right]$.
			\item Encontrar bases ordenadas $B$ y $B'$ de $\cc{M}_2(\R)$ y $\R_2\left[x\right]$ respectivamente, para las cuales
			\[
			M(f, B, B') = 
			\left(
				\begin{array}{c|c}
					I_r & 0 \\
					\hline
					0 & 0
				\end{array}
			\right)
			,
			\]
			donde $I_r$ es la matriz identidad de orden $r$ para un cierto $r$.
			\item Calcula la base dual de la base $B'$ obtenida en el apartado anterior.
		\end{enumerate}
	\end{ejercicio}	
	
	\begin{ejercicio}[2.5 puntos] Sea $V$ un espacio vectorial de dimensión finita sobre $\K$ y $f,g \in \text{End}_{\K}V$ tales que $f \circ g = g \circ f$. Prueba que:
		
		\begin{enumerate} 
			\item $f( \ker(g)) \subseteq  \ker(g)$ y $\text{nulidad}(g) = \dim_{\K}( \ker(f) \cap  \ker(g)) + \dim_{\K}(f( \ker(g)))$.
			\item $f(\text{Im}(g)) \subseteq \text{Im}(g)$ y rango$(g) = \dim_{\K}( \ker(f) \cap \text{Im}(g)) +$ rango$(f \circ g)$.
		\end{enumerate}
	\end{ejercicio}
	
	\begin{ejercicio}[2.5 puntos] Decide de forma razonada si las siguientes afirmaciones son verdaderas o falsas:
		
		\begin{enumerate} 
			\item Sea $V$ un espacio vectorial de dimensión finita, $U,W \subseteq V$ dos subespacios vectoriales no
			triviales y $B_U$, $B_W$ una base de $U$ y otra de $W$.
			\begin{enumerate}
				\item  Si $B_U \cup B_W$ es una base de $V$ entonces $U + W = V$.
				\item  Si $B_U \cap B_W = \emptyset$ entonces $U \cap W = \left\{\vec{0}\right\}$.
			\end{enumerate}
			
			\item Si $\cc{M}_n(\R)$ es el espacio de las matrices cuadradas reales de orden $n \geq 2$ entonces
			\begin{enumerate}
				\item  Existe una base de $\cc{M}_n(\R)$ formada por matrices de traza igual a 0.
				\item  Existe una base de $\cc{M}_n(\R)$ con una matriz de traza 1 y todas las demás con traza igual
				a 0.
			\end{enumerate}
		\end{enumerate}
	\end{ejercicio}
	
	\newpage
	\setcounter{ejercicio}{0} % Reseteo de contador para ejercicios resueltos
	
		%1
	\begin{ejercicio}[2.5 puntos] Dado $a \in \R$, consideremos el subespacio de $\R^4$.
		\begin{equation*}
			U_a := \cc{L}(\{(0, -1, a, 3), (2-a, 1, 2, -3), (0, a, -2-a, 3)\})
		\end{equation*}
		
		\begin{enumerate} 
			\item Calcula la dimensión de $U_a$ en función de los valores de $a$. Determina una base y ecuaciones
			implícitas de $U_a$ para cada $a \in \R$.
			
			Ponemos los vectores que generan $U_a$ por filas para hacer una matriz.
			$$
			A(a):=\begin{pmatrix}
				0& -1& a& 3 \\
				2-a& 1& 2& -3 \\
				0& a& -2-a& 3 
			\end{pmatrix}
			$$
			
			El determinante del menor de $A(a)$ formado por sus columnas primera, segunda y tercera es.
			
			$$
			\begin{vmatrix}
				0& -1& 3 \\
				2-a& 1& -3 \\
				0& a& 3 
			\end{vmatrix} = 3(2-a)(1+a)
			$$
			
			Por tanto:
			\begin{enumerate}[label=\roman*.]
				\item Si $a\ne -1,2$, el rango de $A(a)$ es 3. Esto quiere decir que los tres vectores que forman el sistema de generadores de $U_a$ del enunciado son linealmente independientes, y por tanto, $\dim_{\R}U_a=3$. Una base de $U_a$ es
				$$ B_a := \{(0, -1, a, 3), (2-a, 1, 2, -3), (0, a, -2-a, 3)\}$$
				Para calcular las ecuaciones implícitas de $U_a$, obligamos a que la matriz.
				
				$$
				\begin{pmatrix}
					0 & -1 & a & 3 \\
					2 - a & 1 & 2 & -3 \\
					0 & a & -2 - a & 3 \\
					x & y & z & t
				\end{pmatrix}
				$$
				
				tenga rango 3, es decir, que su determinante sea cero. Este determinante vale
				
				$$ -(a+1) \left[ -3(a+2)x + 6(2 - a)y + 3(2 - a)z + (a^2 - 4a + 4)t \right],$$
				
				así que la (única) ecuación implícita de $U_a$ es
				\begin{equation*}
					-3(a+2)x + 6(2 - a)y + 3(2 - a)z + (a^2 - 4a + 4)t = 0.
				\end{equation*}
				
				\item Si $a = -1$, nos queda $ \displaystyle A(-1) = 
				\begin{pmatrix}
					0 & -1 & -1 & 3 \\
					3 & 1 & 2 & -3 \\
					0 & -1 & -1 & 3
				\end{pmatrix} $, que tiene iguales sus filas primera y tercera, luego el rango de $A(-1)$ es como mucho 2. Como el menor $ \displaystyle M := \begin{pmatrix}
					0 & -1 \\
					3 & 1 
				\end{pmatrix}$ 
				
				de $A(-1)$ tiene determinante $3 \ne 0$, el rango de $A(-1)$ es 2, y por tanto $\dim_{\R} U_{-1} = 2$. Una base de $U_{-1}$ es
				
				\begin{equation*}
					B_{-1} = \{(0, -1, -1, 3),\ (3, 1, 2, -3)\}.
				\end{equation*}
				
				Para calcular las ecuaciones implícitas de $U_{-1}$, obligamos a que la matriz
				
				$$
				\begin{pmatrix}
					0 & -1 & -1 & 3 \\
					3 & 1 & 2 & -3 \\
					x & y & z & t
				\end{pmatrix}
				$$
				
				tenga rango 2, es decir, a que los siguientes dos determinantes se anulen (se obtienen orlando el menor $M$):
				
				$$ \begin{vmatrix}
					0&-1&-1 \\
					3&1&2 \\
					x&y&z \\
				\end{vmatrix} = -x -3y +3z, \quad
				\begin{vmatrix}
					0&-1&3 \\
					3&1&-3 \\
					x&y&t
				\end{vmatrix} = 3t+9y $$
				
				luego las ecuaciones implícitas de $U_{-1}$ son
				$$ -x -3y +3z = 0, \quad 3t+9y = 0 $$
				
				\item Si $a = 2$, nos queda $ \displaystyle A(2) = 
				\begin{pmatrix}
					0 & -1 & 2 & 3 \\
					0 & 1 & 2 & -3 \\
					0 & 2 & -4 & 3
				\end{pmatrix} $. El determinante del menor formado por las columnas segunda, tercera y cuarta de $A(2)$ es $-36 \ne 0$, luego como pasaba en el apartado i), los tres vectores que forman el sistema de generadores de $U_2$ del enunciado son linealmente independientes, y por tanto, $\dim_{\R} U_2 = 3$. Una base de $U_a$ es $B_2$ (en el apartado $B_a$ teníamos definida $B_a$ para $a \in \R \setminus \{-1, 2\}$, ahora extendemos esa misma definición al caso $a = 2$):
				
				$$B_{2} = \{(0, -1, 2, 3),\ (0, 1, 2, -3),\ (0, 2, -4, 3)\}.$$
				
				Para calcular las ecuaciones implícitas de $U_2$ seguimos el mismo procedimiento del apartado 1 con la base $B_2$, es decir, imponemos que la matriz
				
				$$\begin{pmatrix}
					0 & -1 & 2 & 3 \\
					0 & 1 & 2 & -3 \\
					0 & 2 & -4 & 3 \\
					x & y & z & t
				\end{pmatrix}$$
				
				tenga rango 3, es decir, que su determinante sea cero. Desarrollando por la primera columna, este determinante vale $36x$, luego la ecuación implícita de $U_2$ es
				
				$$x=0.$$
			\end{enumerate}
			
			\item Para $a$ satisfaciendo $\dim_{\R} U_a = 2$, encuentra un subespacio $W$ de $\R^4$ tal que $\R^4 = U_a\oplus W$. Determina las ecuaciones implícitas de $W$.
			
			Por el apartado (a), el único caso en el que $\dim_{\R} U_a = 2$ es para $a = -1$, y en este caso $B_{-1}$ dada por (2). Ampliamos $B_{-1}$ una base de $\R^4$, por ejemplo con los vectores $(0,0,1,0)$, $(0,0,0,1)$. La ampliación
			
			$$B' := \{(0, -1, -1, 3),\ (3, 1, 2, -3),\ (0, 0, 1, 0),\ (0, 0, 0, 1)\}$$
			
			es una base de $\R^4$ porque los cuatro vectores son linealmente independientes, ya que
			
			$$ 
			\begin{vmatrix}
				0 & -1 & -1 & 3 \\
				3 & 1 & 2 & -3 \\
				0 & 0 & 1 & 0 \\
				0 & 0 & 0 & 1
			\end{vmatrix} = 3 \ne 0$$
			
			Así que $W := \cc{L}(\{(0, 0, 1, 0),\ (0, 0, 0, 1)\})$ cumple 
			$\R^4 = U_{-1}\oplus W$. Finalmente, las ecuaciones implícitas de $W$ son
			
			$$ x=0, \quad y=0.$$
		\end{enumerate}
	\end{ejercicio}
		
		%2
	\begin{ejercicio}[2.5 puntos] Sea $f : \mathcal{M}_2(\R) \rightarrow \R^3$ la aplicación lineal dada por:
		\begin{equation*}
			f{\tiny \begin{pmatrix}
					\ 0 & 1 \ \\
					\ 1 & 0 \ \\
			\end{pmatrix}} = 1+x^2, \quad
			f{\tiny \begin{pmatrix}
					\ 1 & 1 \ \\
					\ 0 & 0 \ \\
			\end{pmatrix}} = 2-x+x^2, \quad	
			f{\tiny \begin{pmatrix}
					\ 0 & 0 \ \\
					\ 0 & 1 \ \\
			\end{pmatrix}} = 1+x^2, \quad
			f{\tiny \begin{pmatrix}
					\ 1 & -1 \ \\
					\ 0 & 0 \ \\
			\end{pmatrix}} =2-3x-x^2										
		\end{equation*}
		
		\begin{enumerate} 
			\item Calcula la matriz asociada a $f$ respecto de las bases usuales de $\cc{M}_2(\R)$ y de $\R_2\left[x\right]$.
			Llamamos 
			$$
			M_1 :={\tiny \begin{pmatrix}
					\ 0 & 1 \ \\
					\ 1 & 0 \ \\
			\end{pmatrix}},\quad
			M_2 :={\tiny \begin{pmatrix}
					\ 1 & 1 \ \\
					\ 0 & 0 \ \\
			\end{pmatrix}}, \quad	
			M_3 :={\tiny \begin{pmatrix}
					\ 0 & 0 \ \\
					\ 0 & 1 \ \\
			\end{pmatrix}}, \quad
			M_4 :={\tiny \begin{pmatrix}
					\ 1 & -1 \ \\
					\ 0 & 0 \ \\
			\end{pmatrix}}.
			$$
			Las coordenadas de $M_1, M_2, M_3, M_4$ en la base ordenada usual de $\cc{M_2}{\R})$ son respectivamente:
			
			$$(M_1)_{B_u} = (0,1,1,0), \quad (M_2)_{B_u} = (1,1,0,0), $$ 
			$$ (M_3)_{B_u} = (0,0,0,1), \quad (M_4)_{B_u} = (1,-1,0,0).$$
			
			Como el determinante 
			
			$$ \begin{vmatrix}
				0 & 1 & 1 & 0 \\
				1 & 1 & 0 & 0 \\
				0 & 0 & 0 & 1 \\
				1 & -1 & 0 & 0 
			\end{vmatrix} = 2 \ne 0 $$
			
			Deducimos que $B' := (M_1, M_2, M_3, M_4)$ es una base ordenada de $M_2(\R)$. Esto hace que tenga sentido la definición de $f$ del enunciado, vía el teorema fundamental de las aplicaciones lineales. Además, los datos que nos dan implican que
			
			$$ M(f, B', B'_u)= \begin{pmatrix}
				1 & 2 & 1 & 2 \\
				0 & -1 & 0 & -3 \\
				1 & 1 & 1 & -1
			\end{pmatrix}. $$
			
			Sea $B'_u = (1, x, x^2)$ la base ordenada usual de $\R_2[x]$. La matriz que nos piden es:
			\begin{align*}
				M(f, B_u, B'_u) &= M(f \circ 1_{\cc{M}_2(\R)}, B_u, B'_u) = M(f, B', B'_u) \cdot M(1_{\cc{M}_2(\R)}, B_u, B')
				=\\&= M(f, B', B'_u) \cdot M(1_{\cc{M}_2(\R)}, B', B_u)^{-1}
				=\\&= \begin{pmatrix}
					1 & 2 & 1 & 2 \\
					0 & -1 & 0 & -3 \\
					1 & 1 & 1 & -1
				\end{pmatrix} \cdot 
				\begin{pmatrix}
					0 & 1 & 0 & 1 \\
					1 & 1 & 0 & -1 \\
					1 & 0 & 0 & 0 \\
					0 & 0 & 1 & 0 
				\end{pmatrix}^{-1} 
				=\\&= \begin{pmatrix}
					1 & 2 & 1 & 2 \\
					0 & -1 & 0 & -3 \\
					1 & 1 & 1 & -1
				\end{pmatrix} \cdot 
				\begin{pmatrix}
					0 & 0 & 1 & 0 \\
					\dfrac{1}{2} & \dfrac{1}{2} & -\dfrac{1}{2} & 0 \\
					0 & 0 & 0 & 1 \\
					\dfrac{1}{2} & -\dfrac{1}{2} & \dfrac{1}{2} & 0 
				\end{pmatrix}
				=\\&=\begin{pmatrix}
						2 & 0 & 1 & 1 \\
						-2 & 1 & -1 & 0 \\
						0 & 1 & 0 & 1
					\end{pmatrix}.
			\end{align*}


			\begin{align*}
				f(x) &= 7x+3\\
					&= 7x+3
			\end{align*}
		
			
			
			\item Encontrar bases ordenadas $B$ y $B'$ de $\cc{M}_2(\R)$ y $\R_2\left[x\right]$ respectivamente, para las cuales
			\[
			M(f, B, B') = 
			\left(
			\begin{array}{c|c}
				I_r & 0 \\
				\hline
				0 & 0
			\end{array}
			\right)
			,
			\]
			donde $I_r$ es la matriz identidad de orden $r$ para un cierto $r$.
			
			Como el rango de $f$ es el rango de cualquiera de sus matrices (en cualquier par de bases de su dominio y codominio), tenemos $r = \text{rango}(f)$. Para calcular $r$, nos podemos basar, por ejemplo, en la matriz $M(f, B_u, B'_u)$ que habíamos calculado en (3): la tercera fila de esta matriz es suma de sus dos primeras filas, luego $M(f, B_u, B'_u)$ tiene como mucho rango 2. Su rango es exactamente 2, ya que el menor $\displaystyle \begin{pmatrix}
				2 & 0 \\ 
				-2 & 1
			\end{pmatrix}$ de $M(f, B_u, B'_u)$ tiene determinante $2 \ne 0$. Así que $r=2$. Por la fórmula de la nulidad y el rango, la nulidad de $f$ es $4-2=2$.\\
			La base ordenada $B$ Será de la forma $B = (N_1, N_2, N_3, N_4)$ siendo $\{N_3, N_4\}$ una base del núcleo de $f$. Usando (3), tenemos
			
			\begin{align*}
			 \ker(f) &= 
			\left\{
			\begin{pmatrix}
				 a & b \\ 
				 c & d
			\end{pmatrix} \in \cc{M}_2(\R) : 
			\begin{pmatrix}
				2 & 0 & 1 & 1 \\
				-2 & 1 & -1 & 0 \\
				0 & 1 & 0 & 1
			\end{pmatrix} \cdot
			\begin{pmatrix}
				a \\
				b \\
				c \\
				d
			\end{pmatrix} = 
			\begin{pmatrix}
				0 \\
				0 \\
				0 \\
				0
			\end{pmatrix}
			\right\}=\\&=
			\left\{
			\begin{pmatrix}
				a & b \\
				c & d
			\end{pmatrix} \in \cc{M}_2(\R) : 
			\begin{array}{c}
				2a +c +d =0 \\
				-2a +b -c =0
			\end{array}
			\right\}
			=\\&=
			\left\{
			\begin{pmatrix}
				a & b \\
				c & d
			\end{pmatrix} \in \cc{M}_2(\R) : 
			\begin{array}{c}
				c +d = -2a \\
				c = -2a +b
			\end{array}
			\right\} = \cc{L}(\{N_3, N_4\}),
			\end{align*}
			donde:
			$$ N_3 := \begin{pmatrix}
				1 & 0 \\
				-2 & 0
			\end{pmatrix}, \quad
			N_4 := \begin{pmatrix}
				0 & 1 \\
				1 & -1
			\end{pmatrix}. $$
			
			Tomamos $N_1, N_2$ como una ampliación donde $\{N_3, N_4\}$ es una base de $M_2(\R)$; por ejemplo, con
			
			$$ N_1 := \begin{pmatrix}
				1 & 0 \\
				0 & 0
			\end{pmatrix}, \quad
			N_2 := \begin{pmatrix}
				0 & 1 \\
				0 & 0
			\end{pmatrix}. $$
			
			Que $B := (N_1, N_2, N_3, N_4)$ es base de $\cc{M}_2(\R)$ es consecuencia de que el determinante de la matriz de las coordenadas de $N_1, N_2, N_3, N_4$ en la base usual es no nulo; escribiendo estas coordenadas por filas, este determinante es
			$$\begin{vmatrix}
				1 & 0 & 0 & 0 \\
				0 & 1 & 0 & 0 \\
				1 & 0 & -2 & 0 \\
				0 & 1 & 1 & -1
			\end{vmatrix} = 2 \ne 0 $$
			y ya tenemos la base $B$. Para calcular la base $B'$, notemos que 
			$$\text{Im}(f) = L(\{f(N_1), f(N_2), f(N_3), f(N_4)\}) = L(\{f(N_1), f(N_2)\}) = L(\{2 - 2x, x + x^2\}),$$ 
			donde la última igualdad se deduce de las dos primeras columnas de la matriz $M(f, B_u, B'_u)$ que calculamos en el apartado (a). Ahora llamamos $$B' := (2 - 2x, x + x^2, 1),$$ 
			que es base (ordenada) de $\R_2[x]$ porque está formada por tres polinomios de grados distintos 1, 2, 0. Por construcción, la matriz de $f$ en $B, B'$ es
			$$
			M(f, B, B') = \begin{pmatrix}
				1 & 0 & 0 & 0 \\
				0 & 1 & 0 & 0 \\
				0 & 0 & 0 & 0
			\end{pmatrix}. $$
			
			
			\item Calcula la base dual de la base $B'$ obtenida en el apartado anterior.
			
			Llamamos $B'^\ast = (\varphi_1, \varphi_2, \varphi_3)$ a la base dual de $B'$. Entonces, 
			$$
			\varphi_1(2 - 2x) = 1, \quad \varphi_1(x + x^2) = 0, \quad \varphi_1(1) = 0.
			$$
			De la primera y tercera ecuación tenemos
			$ 1 = 2[\varphi_1(1) - \varphi_1(x)] = 2[0 - \varphi_1(x)] = -2\varphi_1(x),$ luego $ \displaystyle \varphi_1(x) = -\frac{1}{2}$. De esto y la segunda ecuación deducimos que $\displaystyle \varphi_1(x^2) = -\varphi_1(x) = \frac{1}{2}.$
			Por tanto,
			$$
			\varphi_1(a_0 + a_1 x + a_2 x^2) = -\frac{a_1}{2} + \frac{a_2}{2}, \quad \forall a_0 + a_1 x + a_2 x^2 \in \R_2[x].
			$$
			$\varphi_2$ y $\varphi_3$ se calculan análogamente:
			$$\varphi_2(2 - 2x) = 0, \quad \varphi_2(x + x^2) = 1, \quad \varphi_2(1) = 0,$$
			luego $ 0 = 2[\varphi_2(1) - \varphi_2(x)] = 2[0 - \varphi_2(x)] = -2\varphi_2(x)$, de donde $\varphi_2(x) = 0$. $\varphi_2(x^2) = 1 - \varphi_2(x) = 1.$ Por tanto,
		
			\begin{gather*}
				\varphi_2(a_0 + a_1 x + a_2 x^2) = a_2, \quad \forall a_0 + a_1 x + a_2 x^2 \in \R_2[x].\\
				\varphi_3(2 - 2x) = 0, \quad \varphi_3(x + x^2) = 0, \quad \varphi_3(1) = 1,
			\end{gather*}
			luego $ 0 = 2[\varphi_3(1) - \varphi_3(x)] = 2[1 - \varphi_3(x)]$, de donde $\varphi_3(x) = 1$. $\varphi_3(x^2) = -\varphi_3(x) = -1.$ Por tanto,
			$$
			\varphi_3(a_0 + a_1 x + a_2 x^2) = a_0 + a_1 - a_2, \quad \forall a_0 + a_1 x + a_2 x^2 \in \R_2[x].
			$$
		\end{enumerate}
	\end{ejercicio}	
	
		%3
	\begin{ejercicio}[2.5 puntos] Sea $V$ un espacio vectorial de dimensión finita sobre $\K$ y $f,g \in \text{End}_{\K}V$ tales que $f \circ g = g \circ f$. Prueba que:
			
			\begin{enumerate} 
				\item $f( \ker(g)) \subseteq  \ker(g)$ y $\text{nulidad}(g) = \dim_{\K}( \ker(f) \cap  \ker(g)) + \dim_{\K}(f( \ker(g)))$.
				
				Sea $x \in \ker(g)$. Veamos que $f(x) \in \ker(g)$ y tendremos la inclusión que nos piden. 
				
				$g(f(x)) = (g \circ f)(x) = (f \circ g)(x) = f(g(x)) = f(0) = 0$. Por tanto, $f(\ker(g)) \subseteq \ker(g)$. 
				
				Como $f(\ker(g)) \subseteq \ker(g)$, podemos restringir $f$ a $\ker(g)$ obteniendo un endomorfismo 
				$$f_{\big| \ker(g)} : \ker(g) \to \ker(g).$$ 
				
				Aplicando la fórmula de la nulidad y el rango a $f_{\big| \ker(g)}$ (notemos que el espacio total es ahora $\ker(g)$, que es de dimensión finita por serlo $V$), tenemos
				$$ \text{nulidad}(g) = \dim_{\K} \ker(g) = \dim_{\K} \ker\left(f_{\big| \ker(g)}\right) + \dim_{\K} \text{Im}\left(f_{\big| \ker(g)}\right), $$ 
				luego para terminar basta demostrar que :
				$$ \ker\left(f_{\big| \ker(g)}\right) = \ker(f) \cap \ker(g), \quad \text{Im}\left(f_{\big| \ker(g)}\right) = f(\ker(g)). $$

				Tenemos que:
				\begin{align*}
					\ker\left(f_{\big| \ker(g)}\right) &= \{ x \in \ker(g) \mid f(x) = 0 \} = \{ x \in \ker(g) \mid x \in \ker(f) \} = \ker(f) \cap \ker(g),\\
					\text{Im}\left(f_{\big| \ker(g)}\right) &= \{ f(x) \mid x \in \ker(g) \} = f(\ker(g)).
				\end{align*}


				\item $f(\text{Im}(g)) \subseteq \text{Im}(g)$ y rango$(g) = \dim_{\K}( \ker(f) \cap \text{Im}(g)) +$ rango$(f \circ g)$.
				
				Este apartado es muy parecido al anterior. Sea $x \in \text{Im}(g)$; y veamos que se tiene $f(x) \in \text{Im}(g)$. 
				
				Como $x \in \text{Im}(g)$, existe $y \in V$ tal que $g(y) = x$. Así, $f(x) = f(g(y)) = (f \circ g)(y) = (g \circ f)(y) = g(f(y))$, luego $f(x)$ es la imagen por $g$ de un elemento de $V$. Por tanto, 
				$f(\text{Im}(g)) \subseteq \text{Im}(g)$. 
				
				Como $f(\text{Im}(g)) \subseteq \text{Im}(g)$, podemos restringir $f$ a $\text{Im}(g)$ obteniendo un endomorfismo
				$$f_{\big| \text{Im}(g)} : \text{Im}(g) \to \text{Im}(g).$$
				Aplicando la fórmula de la nulidad y el rango a $f_{\big| \text{Im}(g)}$ (ahora el espacio total es $\text{Im}(g)$, que de nuevo tiene dimensión finita), tenemos
				$$ \text{rango}(g) = \dim_{\K} \text{Im}(g) = \dim_{\K} \ker\left(f_{\big| \text{Im}(g)}\right) + \dim_{\K} \text{Im}\left(f_{\big| \text{Im}(g)}\right), $$ 
				luego para terminar basta demostrar que:
				$$ \ker\left(f_{\big| \text{Im}(g)}\right) = \ker(f) \cap \text{Im}(g), \quad \text{Im}\left(f_{\big| \text{Im}(g)}\right) = \text{Im}(f \circ g). $$

				Tenemos que:
				\begin{align*}
					\ker\left(f_{\big| \text{Im}(g)}\right) &= \{ x \in \text{Im}(g) \mid f(x) = 0 \} = \{ x \in \text{Im}(g) \mid x \in \ker(f) \} = \ker(f) \cap \text{Im}(g),
					\\\text{Im}\left(f_{\big| \text{Im}(g)}\right) &= \{ f(x) \mid x \in \text{Im}(g) \} = \{ f(g(y)) \mid y \in V \} = \text{Im}(f \circ g).
				\end{align*}
				
			\end{enumerate}
		\end{ejercicio}
		
		%4
	\begin{ejercicio}[2.5 puntos] Decide de forma razonada si las siguientes afirmaciones son verdaderas o falsas:
			
			\begin{enumerate} 
				\item Sea $V$ un espacio vectorial de dimensión finita, $U,W \subseteq V$ dos subespacios vectoriales no
				triviales y $B_U$, $B_W$ una base de $U$ y otra de $W$.
				\begin{enumerate}
					\item  Si $B_U \cup B_W$ es una base de $V$ entonces $U + W = V$. \\
					\textbf{Verdadero:} como $B_U \cup B_W$ es base de $V$, en particular es sistema de generadores de $V$. Por tanto, $V = \cc{L}(B_U \cup B_W) = \cc{L}(B_U) + \cc{L}(B_W) = U + W$, donde en la última igualdad hemos usado que $B_U$ es sistema de generadores de $U$ y $B_W$ lo es de $W$.\\
					
					\item  Si $B_U \cap B_W = \emptyset$ entonces $U \cap W = \left\{\vec{0}\right\}$. \\
					\textbf{Falso:} Tomemos $V(\K) = \R^3(\R)$, $U = \{(a,b,c) \in \R^3 \mid a = 0\}$, $W = \{(a,b,c) \in \R^3 \mid c = 0\}$ y $B_U = \{(0,1,0),(0,0,1)\}$, $B_W = \{(1,0,0),(1,1,0)\}$. Entonces, $B_U$ es base de $U$ y $B_W$ lo es de $W$, $B_U \cap B_W = \emptyset$ y $U \cap W = \cc{L}(\{(0,1,0)\}) \ne \left\{\vec{0}\right\}$.\\
					
				\end{enumerate}
				
				\item Si $\cc{M}_n(\R)$ es el espacio de las matrices cuadradas reales de orden $n \geq 2$ entonces
				\begin{enumerate}
					\item  Existe una base de $\cc{M}_n(\R)$ formada por matrices de traza igual a 0. \\
					\textbf{Falso:} Llamemos $U = \{A \in \cc{M}_n(\R) \mid \text{traza}(A) = 0\} = \ker(\text{traza})$. Como la traza es una forma lineal no nula (porque la traza de $I_n$ es $n \neq 0$), deducimos que $U$ tiene dimensión $n^2 - 1$. \\
					Si existiera una base $B$ de $\cc{M}_n(\R)$ formada por matrices de traza igual a 0, entonces $B$ estaría contenida en $U$, luego $\cc{M}_n(\R) = \cc{L}(B) \subseteq U$, lo que contradice que $\dim_{\K}(U) = n^2 - 1 < n^2 = \dim_{\K}(\cc{M}_n(\R))$. \\
					
					\item  Existe una base de $\cc{M}_n(\R)$ con una matriz de traza 1 y todas las demás con traza igual
					a 0. \\
					\textbf{Verdadero:} Tomemos una base $B_U$ de $U$ (el mismo subespacio definido en (a)), que tendrá dimensión $\dim_{\K}(U) = n^2 - 1$ matrices, todas con traza cero. Como $I_n$ no tiene traza cero, $I_n \notin U$. Por tanto, $B := B_U \cup \{I_n\}$ es linealmente independiente. Como $B$ tiene $n^2$ matrices y $\dim_{\K} M_n(\R) = n^2$, concluimos que $B$ es base de $M_n(\R)$. Claramente, $B$ cumple las condiciones del enunciado.\\
					
				\end{enumerate}
			\end{enumerate}
		\end{ejercicio}
	
\end{document}