\section{Sistemas de ecuaciones Lineales}
\begin{ejercicio}Decidir cuáles de los siguientes sistemas de ecuaciones son lineales. Para los que lo sean, escribir la matriz de coeficientes y la matriz ampliada del sistema:

	\begin{align*}
		\begin{cases}
			x + y + \sqrt{z} = 0 \\
			y - z = 3x           \\
			x + y + z = 0
		\end{cases} & \begin{cases}
			              x + 2z = -3y       \\
			              \sin(2)z = 35 - 2x \\
			              y + x = \sqrt{3}z
		              \end{cases}              \\
		\begin{cases}
			x + y + 2z = 0 \\
			y - z = 35     \\
			x + 2y + 3z = 2007
		\end{cases}      & \begin{cases}
			                   x + y + z = 28 \\
			                   z^2 = 35       \\
			                   \sin(x) + \cos(y) = \tan(z)
		                   \end{cases}
	\end{align*}


	Los sistemas segundo y tercero son de ecuaciones lineales. Las matrices de coeficientes y ampliada del segundo sistema son:
	\[
		A = \begin{pmatrix}
			1 & 1 & 2  \\
			0 & 1 & -1 \\
			1 & 2 & 3
		\end{pmatrix}, \quad (A|b) = \left(\begin{array}{ccc|c}
				1 & 1 & 2  & 0    \\
				0 & 1 & -1 & 35   \\
				1 & 2 & 3  & 2007
			\end{array}\right)
	\]
	Las matrices de coeficientes y ampliadas del tercer sistema son:
	\[
		A = \begin{pmatrix}
			1 & 3 & 2         \\
			2 & 0 & \sin(2)   \\
			1 & 1 & -\sqrt{3}
		\end{pmatrix}, \quad (A|b) = \left(\begin{array}{ccc|c}
				1 & 3 & 2         & 0  \\
				2 & 0 & \sin(2)   & 35 \\
				1 & 1 & -\sqrt{3} & 0
			\end{array}\right)
	\]
\end{ejercicio}

\begin{ejercicio} Resolver los siguientes sistemas de ecuaciones lineales escalonados:
	\begin{align*}
		\begin{cases}
			x + y + z + t = 1 \\
			y + z + t = 2     \\
			z + t = 3
		\end{cases}
		\begin{cases}
			x + y = -z - t \\
			y + z + t = 3
		\end{cases}
		\begin{cases}
			x - y + 10z = 8
		\end{cases}
	\end{align*}


	Solución: Todos son compatibles e indeterminados. Las es del primero son:
	\[
		x = -1, \quad y = -1, \quad z = 3 - \lambda, \quad t = \lambda, \quad \lambda \in \mathbb{R}
	\]

	Las es del segundo son:
	\[
		x = -3, \quad y = 3 - \lambda - \mu, \quad z = \lambda, \quad t = \mu, \quad \lambda, \mu \in \mathbb{R}
	\]

	Las es del tercero son:
	\[
		x = 8 + \lambda - 10\mu, \quad y = \lambda, \quad z = \mu, \quad \lambda, \mu \in \mathbb{R}
	\]
\end{ejercicio}
\begin{ejercicio}Discutir y resolver los siguientes sistemas de ecuaciones lineales:

	\begin{align*}
		\begin{cases}
			2x + 2y + 10z = 18 \\
			2x + 3y + 12z = 23 \\
			2y + 5z = 11
		\end{cases} & \quad
		\begin{cases}
			x + 2y - 3z = -1 \\
			3x - y + 2z = 1  \\
			5x + 3y - 4z = 2
		\end{cases}       \\
		\begin{cases}
			x + 2y - 3z = 6 \\
			2x - y + 4z = 2 \\
			4x + 3y - 2z = 14
		\end{cases}    & \quad
		\begin{cases}
			x - 3y + 4z - 2t = 5 \\
			2y + 5z + t = 2      \\
			y - 3z = 4
		\end{cases}
	\end{align*}

	Solución: El primer sistema es compatible determinado con solución:
	\[
		x = 1, \quad y = 3, \quad z = 1
	\]
	El segundo sistema es incompatible. El tercer sistema es compatible indeterminado con es:
	\[
		x = 2 - \lambda, \quad y = 2 + 2\lambda, \quad z = \lambda, \quad \lambda \in \mathbb{R}
	\]
	El ultimo sistema es compatible indeterminado con es:
	\begin{equation*}
		x = \frac{157 + 15\lambda}{11}, \quad y = \frac{26-3\lambda}{11}, \quad z = \frac{-6-\lambda}{11},
		\quad t = \lambda, \quad \lambda \in \mathbb{R}
	\end{equation*}
\end{ejercicio}
\begin{ejercicio}Discutir y resolver los siguientes sistemas de ecuaciones lineales:

	\begin{align}
		\begin{cases}
			-4y - z    & = -7 \\
			x + y + z  & = 2  \\
			x - 2y + z & = -2 \\
			-x + 2y    & = 3
		\end{cases} \qquad
		\begin{cases}
			y + z - 2s + t      & = 2 \\
			x + 2y + s          & = 7 \\
			2x - y + z + 4s + t & = 0
		\end{cases}
	\end{align}

	Solución: El primer sistema es incompatible. El segundo es compatible indeterminado con es:
	\begin{align}
		x & = \frac{5 - 7\lambda}{3}, \quad y = \frac{8 + 2\lambda}{3}, \quad z = \frac{-2 + 4\lambda - 3\mu}{3}, \quad s = \lambda, \quad t = \mu, \quad \lambda \in \mathbb{R}
	\end{align}

\end{ejercicio}
\begin{ejercicio}Discutir y resolver, cuando sea posible, los sistemas de ecuaciones lineales siguientes en función de
	los parámetros $a$ y $b$:

	\begin{align*}
		\left\{ \begin{array}{rcl}
			        x + 2y + z & = 1  \\
			        -x - y + z & = 2  \\
			        x + az     & = -1
		        \end{array} \right. & \qquad
		\left\{ \begin{array}{rcl}
			        x + y + z    & = 1 \\
			        3x + ay + az & = 5 \\
			        4x + ay      & = 5
		        \end{array} \right.          \\
		\left\{ \begin{array}{rcl}
			        x + t           & = a \\
			        x - 2y + z      & = 1 \\
			        -x + y + az - t & = 0
		        \end{array} \right. & \qquad
		\left\{ \begin{array}{rcl}
			        az         & = b \\
			        y + z      & = 0 \\
			        x + ay + z & = 0
		        \end{array} \right.          \\
		\left\{ \begin{array}{rcl}
			        ax + y + z & = 1   \\
			        x + ay + z & = a   \\
			        x + y + az & = a^2
		        \end{array} \right. & \qquad
		\left\{ \begin{array}{rcl}
			        x - 3y + z & = 1 \\
			        2x - 3z    & = a \\
			        x + y + 2z & = 0 \\
			        2x + y - z & = 1
		        \end{array} \right.
	\end{align*}
	\begin{align*}
		\left\{ \begin{array}{rcl}
			        ax + y + z  & = 1 \\
			        x + y + z   & = b \\
			        ax + by + z & = 1
		        \end{array} \right. & \qquad
		\left\{ \begin{array}{rcl}
			        ax + y + z & = 1 \\
			        x + y + z  & = 2
		        \end{array} \right.
	\end{align*}

	Discutimos cada uno de los sistemas:

	\begin{itemize}
		\item \textbf{Primer sistema.} Si $a = -3$, es incompatible. Si $a \neq -3$, es compatible determinado con solución:
		      \[
			      x = \frac{3a - 3}{a + 3}, \quad y = \frac{5 - a}{a + 3}, \quad z = \frac{-4}{a + 3}.
		      \]

		\item \textbf{Segundo sistema.} Si $a = 0$ o $a = 3$, es incompatible. Si $a \neq 0, 3$, es compatible determinado con solución:
		      \[
			      x = \frac{a - 5}{a - 3}, \quad y = \frac{a + 5}{a(a - 3)}, \quad z = \frac{a - 5}{a(a - 3)}.
		      \]

		\item \textbf{Tercer sistema.} Siempre es compatible indeterminado. Si $a = -\frac{1}{2}$, las es son:
		      \[
			      x = 0, \quad y = \frac{\lambda - 1}{2}, \quad z = \lambda, \quad t = -\frac{1}{2}, \quad \lambda \in \mathbb{R}.
		      \]
		      Si $a \neq -\frac{1}{2}$, entonces las es son:
		      \[
			      x = a - \lambda, \quad y = \frac{a(a - \lambda)}{2a + 1}, \quad z = \frac{a + \lambda + 1}{2a + 1}, \quad t = \lambda, \quad \lambda \in \mathbb{R}.
		      \]

		\item \textbf{Cuarto sistema.} Si $a = 0$ y $b \neq 0$, es incompatible. Si $a = 0$ y $b = 0$, es compatible indeterminado con es:
		      \[
			      x = -\lambda, \quad y = -\lambda, \quad z = \lambda, \quad \lambda \in \mathbb{R}.
		      \]
		      Si $a \neq 0$, es compatible determinado con solución:
		      \[
			      x = \frac{b(a - 1)}{a}, \quad y = \frac{-b}{a}, \quad z = \frac{b}{a}.
		      \]

		\item \textbf{Quinto sistema.} Si $a = -2$, es incompatible. Si $a = 1$, es compatible indeterminado con es:
		      \[
			      x = 1 - \lambda - \mu, \quad y = \lambda, \quad z = \mu, \quad \lambda, \mu \in \mathbb{R}.
		      \]
		      Si $a \neq -2, 1$, es compatible determinado con solución (tras varias simplificaciones):
		      \[
			      x = \frac{-a + 1}{a + 2}, \quad y = \frac{1}{a + 2}, \quad z = \frac{(a + 1)^2}{a + 2}.
		      \]

		\item \textbf{Sexto sistema.} Si $a \neq \frac{29}{19}$, es incompatible. Si $a = \frac{29}{19}$, es compatible determinado con solución:
		      \[
			      x = \frac{10}{19}, \quad y = \frac{-4}{19}, \quad z = \frac{-3}{19}.
		      \]

		\item \textbf{Séptimo sistema.} Si $a = 1$ y $b \neq 1$, es incompatible. Si $a = 1$ y $b = 1$, es compatible indeterminado con es:
		      \[
			      x = 1 - \lambda - \mu, \quad y = \lambda, \quad z = \mu, \quad \lambda, \mu \in \mathbb{R}.
		      \]
		      Si $a \neq 1$ y $b \neq 1$, es compatible determinado con solución:
		      \[
			      x = \frac{b - 1}{1 - a}, \quad y = 0, \quad z = \frac{1 - ab}{1 - a}.
		      \]
		      Si $a \neq 1$ y $b = 1$, es compatible indeterminado con es:
		      \[
			      x = 0, \quad y = 1 - \lambda, \quad z = \lambda, \quad \lambda \in \mathbb{R}.
		      \]

		\item \textbf{Octavo sistema.} Si $a = 1$, es incompatible. Si $a \neq 1$, es compatible indeterminado con es:
		      \[
			      x = \frac{1}{1 - a}, \quad y = \frac{1 - 2a}{1 - a} - \lambda, \quad z = \lambda, \quad \lambda \in \mathbb{R}.
		      \]
	\end{itemize}


\end{ejercicio}
\begin{ejercicio}
	Las tres cifras de un número suman 21. Si a ese número se le resta el que resulta de invertir el orden de sus cifras se obtiene 198. Se sabe también que la cifra de las decenas coincide con la media aritmética entre las otras dos. Calcular dicho número.
	Se esta planteando el siguiente sistema, sean $x, y, z$ las cifras del número, entonces:
	\begin{equation*}
		\left.
		\begin{array}{rcl}
			x + y + z                         & = & 21              \\
			100x + 10y + z - (100z + 10y + x) & = & 198             \\
			y                                 & = & \frac{x + z}{2}
		\end{array} \right\} \Rightarrow \left. \begin{array}{rcl}
			x + y + z & = & 21              \\
			x - z     & = & 2               \\
			y         & = & \frac{x + z}{2}
		\end{array} \right\}
	\end{equation*}

	{El número buscado es el 876.}

\end{ejercicio}
\begin{ejercicio}Dados tres puntos planos $(x_1, y_1)$, $(x_2, y_2)$, $(x_3, y_3)$ de forma que sus primeras coordenadas son dos a dos distintas, probar que existe una única parábola $y = ax^2 + bx + c$ (incluyendo el caso límite de rectas, esto es, $a = 0$) cuya gráfica contiene a dichos puntos. ¿Qué parábola se obtiene para los puntos $(2, 0)$, $(3, 0)$ y $(-1, 12)$?


	\textbf{Solución:} Queremos probar que hay una única parábola de tipo \( y = ax^2 + bx + c \) que pasa por \( (x_1, y_1), (x_2, y_2) \) y \( (x_3, y_3) \). Esto se consigue resolviendo la ecuación de la parábola. La ecuación es un SEL con matriz ampliada:

	\[
		A|b =
		\left( \begin{array}{ccc}
				1 & x_1 & x_{1}^{2} \\
				1 & x_2 & x_{2}^{2} \\
				1 & x_3 & x_{3}^{2}
			\end{array} \right|
		\left. \begin{array}{c}
				y_{1} \\
				y_{2} \\
				y_{3}
			\end{array} \right)
	\]

	Realizamos transformaciones elementales por filas hasta obtener un SEL escalonado. Co-mo mi \( X_T A = X_T Y\) es parte de este caso particular de determinado, lo resuelvo el SEL por el método del sustitución hacia atrás, en sus componentes de la primera posición que se llega a la parábola \( y = x^2 - 5x + 6 \). Es inmediato comprobar que, efectivamente, esta parábola pasa por \( (2.0), (3.0) \) y \( (-1,12) \).


\end{ejercicio}
\begin{ejercicio}Para la construcción de un almacén se necesita una unidad de hierro y ninguna de madera. Para la construcción de un piso se necesita una unidad de cada material y para la construcción de una torre se necesitan cuatro unidades de hierro y una de madera. Si poseemos en reserva 14 unidades de hierro y cuatro de madera, decidir cuántos almacenes, pisos y torres se pueden construir de manera que se utilicen todas las reservas.

		{Hay 4 posibilidades, a saber:

			\begin{itemize}
				\item 10 almacenes, 4 pisos y ninguna torre,
				\item 7 almacenes, 3 pisos y una torre,
				\item 4 almacenes, 2 pisos y 2 torres,
				\item 1 almacén, un piso y 3 torres.
			\end{itemize}}

\end{ejercicio}
\begin{ejercicio}En un examen tipo test de 50 preguntas se dan 2 puntos por cada acierto y se quita medio punto por cada fallo. Para aprobar hay que obtener al menos 40 puntos y es obligatorio contestar a todas las preguntas. Si se quiere aprobar, ¿cuántas preguntas hay que contestar correctamente y cuántas se pueden fallar?

	\begin{equation*}
		2(50-x) - 0.5x = 40 \Rightarrow 100 - 2x - 0.5x = 40 \Rightarrow 100 - 2.5x = 40 \Rightarrow 2.5x = 60 \Rightarrow x = 24
	\end{equation*}
	Por lo que hay que acertar 26 preguntas y se pueden fallar 24.
\end{ejercicio}

\begin{ejercicio}En una ciudad los taxis cobran 1 euro por la bajada de bandera y 10 céntimos por cada 200 metros recorridos. En otra ciudad, la bajada de bandera es de 90 céntimos y por cada 200 metros que se recorran se cobran 12 céntimos. ¿Existe alguna distancia para la que coincidan los precios de las carreras en ambas ciudades?

	Para que coincidan los precios de las carreras en ambas ciudades, se debe cumplir que el precio de la bajada de bandera y el de cada 200 metros recorridos sea el mismo. Por tanto, se debe cumplir que:
	\begin{equation*}
		\left.
		\begin{array}{l}
			1 + 0.1x = y \\
			0.9 + 0.12x = y
		\end{array} \right\}
	\end{equation*}
	Se usa la incógnita $y$ para que se entienda mejor, notese que realmente
	estamos ante una simple ecuación con una incógnita $x$ representando $200$m, como para $x=5$ tenemos una solución, para 1km, el precio sería 1.5 euros en ambos
	taxis.

\end{ejercicio}
\begin{ejercicio}Existe un SEL con 2 ecuaciones y 3 incógnitas que sea compatible determinado? ¿Y si el SEL tiene 3 ecuaciones y 2 incógnitas?

	\begin{itemize}
		\item No, no existe un SEL con 2 ecuaciones y 3 incógnitas que sea compatible determinado ya que al tener más incógnitas que ecuaciones, el sistema es incompatible o compatible indeterminado.
		\item Si, existe un SEL con 3 ecuaciones y 2 incógnitas que sea compatible determinado ya que al tener más ecuaciones que incógnitas, se puede añadir una ecuación redundante y el sistema será compatible determinado.
	\end{itemize}

\end{ejercicio}


