\documentclass[12pt]{article}

% Idioma y codificación
\usepackage[spanish, es-tabla]{babel}       %es-tabla para que se titule "Tabla"
\usepackage[utf8]{inputenc}

% Márgenes
\usepackage[a4paper,top=3cm,bottom=2.5cm,left=3cm,right=3cm]{geometry}

% Comentarios de bloque
\usepackage{verbatim}

% Paquetes de links
\usepackage[hidelinks]{hyperref}    % Permite enlaces
\usepackage{url}                    % redirecciona a la web

% Más opciones para enumeraciones
\usepackage{enumitem}

% Personalizar la portada
\usepackage{titling}

% Paquetes de tablas
\usepackage{multirow}


%------------------------------------------------------------------------

%Paquetes de figuras
\usepackage{caption}
\usepackage{subcaption} % Figuras al lado de otras
\usepackage{float}      % Poner figuras en el sitio indicado H.


% Paquetes de imágenes
\usepackage{graphicx}       % Paquete para añadir imágenes
\usepackage{transparent}    % Para manejar la opacidad de las figuras

% Paquete para usar colores
\usepackage[dvipsnames]{xcolor}
\usepackage{pagecolor}      % Para cambiar el color de la página

% Habilita tamaños de fuente mayores
\usepackage{fix-cm}

% Para los gráficos
\usepackage{tikz}

% Para poder situar los nodos en los grafos
\usetikzlibrary{positioning}


%------------------------------------------------------------------------

% Paquetes de matemáticas
\usepackage{mathtools, amsfonts, amssymb, mathrsfs}
\usepackage[makeroom]{cancel}     % Simplificar tachando
\usepackage{polynom}    % Divisiones y Ruffini
\usepackage{units} % Para poner fracciones diagonales con \nicefrac

\usepackage{pgfplots}   %Representar funciones
\pgfplotsset{compat=1.18}  % Versión 1.18

\usepackage{tikz-cd}    % Para usar diagramas de composiciones
\usetikzlibrary{calc}   % Para usar cálculo de coordenadas en tikz

%Definición de teoremas, etc.
\usepackage{amsthm}
%\swapnumbers   % Intercambia la posición del texto y de la numeración

\theoremstyle{plain}

\makeatletter
\@ifclassloaded{article}{
  \newtheorem{teo}{Teorema}[section]
}{
  \newtheorem{teo}{Teorema}[chapter]  % Se resetea en cada chapter
}
\makeatother

\newtheorem{coro}{Corolario}[teo]           % Se resetea en cada teorema
\newtheorem{prop}[teo]{Proposición}         % Usa el mismo contador que teorema
\newtheorem{lema}[teo]{Lema}                % Usa el mismo contador que teorema

\theoremstyle{remark}
\newtheorem*{observacion}{Observación}

\theoremstyle{definition}

\makeatletter
\@ifclassloaded{article}{
  \newtheorem{definicion}{Definición} [section]     % Se resetea en cada chapter
}{
  \newtheorem{definicion}{Definición} [chapter]     % Se resetea en cada chapter
}
\makeatother

\newtheorem*{notacion}{Notación}
\newtheorem*{ejemplo}{Ejemplo}
\newtheorem*{ejercicio*}{Ejercicio}             % No numerado
\newtheorem{ejercicio}{Ejercicio} [section]     % Se resetea en cada section


% Modificar el formato de la numeración del teorema "ejercicio"
\renewcommand{\theejercicio}{%
  \ifnum\value{section}=0 % Si no se ha iniciado ninguna sección
    \arabic{ejercicio}% Solo mostrar el número de ejercicio
  \else
    \thesection.\arabic{ejercicio}% Mostrar número de sección y número de ejercicio
  \fi
}


% \renewcommand\qedsymbol{$\blacksquare$}         % Cambiar símbolo QED
%------------------------------------------------------------------------

% Paquetes para encabezados
\usepackage{fancyhdr}
\pagestyle{fancy}
\fancyhf{}

\newcommand{\helv}{ % Modificación tamaño de letra
\fontfamily{}\fontsize{12}{12}\selectfont}
\setlength{\headheight}{15pt} % Amplía el tamaño del índice


%\usepackage{lastpage}   % Referenciar última pag   \pageref{LastPage}
\fancyfoot[C]{\thepage}

%------------------------------------------------------------------------

% Conseguir que no ponga "Capítulo 1". Sino solo "1."
\makeatletter
\@ifclassloaded{book}{
  \renewcommand{\chaptermark}[1]{\markboth{\thechapter.\ #1}{}} % En el encabezado
    
  \renewcommand{\@makechapterhead}[1]{%
  \vspace*{50\p@}%
  {\parindent \z@ \raggedright \normalfont
    \ifnum \c@secnumdepth >\m@ne
      \huge\bfseries \thechapter.\hspace{1em}\ignorespaces
    \fi
    \interlinepenalty\@M
    \Huge \bfseries #1\par\nobreak
    \vskip 40\p@
  }}
}
\makeatother

%------------------------------------------------------------------------
% Paquetes de cógido
\usepackage{minted}
\renewcommand\listingscaption{Código fuente}

\usepackage{fancyvrb}
% Personaliza el tamaño de los números de línea
\renewcommand{\theFancyVerbLine}{\small\arabic{FancyVerbLine}}

% Estilo para C++
\newminted{cpp}{
    frame=lines,
    framesep=2mm,
    baselinestretch=1.2,
    linenos,
    escapeinside=||
}

% para minted
\definecolor{LightGray}{rgb}{0.95,0.95,0.92}
\setminted{
    linenos=true,
    stepnumber=5,
    numberfirstline=true,
    autogobble,
    breaklines=true,
    breakautoindent=true,
    breaksymbolleft=,
    breaksymbolright=,
    breaksymbolindentleft=0pt,
    breaksymbolindentright=0pt,
    breaksymbolsepleft=0pt,
    breaksymbolsepright=0pt,
    fontsize=\footnotesize,
    bgcolor=LightGray,
    numbersep=10pt
}


\usepackage{listings} % Para incluir código desde un archivo

\renewcommand\lstlistingname{Código Fuente}
\renewcommand\lstlistlistingname{Índice de Códigos Fuente}

% Definir colores
\definecolor{vscodepurple}{rgb}{0.5,0,0.5}
\definecolor{vscodeblue}{rgb}{0,0,0.8}
\definecolor{vscodegreen}{rgb}{0,0.5,0}
\definecolor{vscodegray}{rgb}{0.5,0.5,0.5}
\definecolor{vscodebackground}{rgb}{0.97,0.97,0.97}
\definecolor{vscodelightgray}{rgb}{0.9,0.9,0.9}

% Configuración para el estilo de C similar a VSCode
\lstdefinestyle{vscode_C}{
  backgroundcolor=\color{vscodebackground},
  commentstyle=\color{vscodegreen},
  keywordstyle=\color{vscodeblue},
  numberstyle=\tiny\color{vscodegray},
  stringstyle=\color{vscodepurple},
  basicstyle=\scriptsize\ttfamily,
  breakatwhitespace=false,
  breaklines=true,
  captionpos=b,
  keepspaces=true,
  numbers=left,
  numbersep=5pt,
  showspaces=false,
  showstringspaces=false,
  showtabs=false,
  tabsize=2,
  frame=tb,
  framerule=0pt,
  aboveskip=10pt,
  belowskip=10pt,
  xleftmargin=10pt,
  xrightmargin=10pt,
  framexleftmargin=10pt,
  framexrightmargin=10pt,
  framesep=0pt,
  rulecolor=\color{vscodelightgray},
  backgroundcolor=\color{vscodebackground},
}

%------------------------------------------------------------------------

% Comandos definidos
\newcommand{\bb}[1]{\mathbb{#1}}
\newcommand{\cc}[1]{\mathcal{#1}}

% I prefer the slanted \leq
\let\oldleq\leq % save them in case they're every wanted
\let\oldgeq\geq
\renewcommand{\leq}{\leqslant}
\renewcommand{\geq}{\geqslant}

% Si y solo si
\newcommand{\sii}{\iff}

% Letras griegas
\newcommand{\eps}{\epsilon}
\newcommand{\veps}{\varepsilon}
\newcommand{\lm}{\lambda}

\newcommand{\ol}{\overline}
\newcommand{\ul}{\underline}
\newcommand{\wt}{\widetilde}
\newcommand{\wh}{\widehat}

\let\oldvec\vec
\renewcommand{\vec}{\overrightarrow}

% Derivadas parciales
\newcommand{\del}[2]{\frac{\partial #1}{\partial #2}}
\newcommand{\Del}[3]{\frac{\partial^{#1} #2}{\partial #3^{#1}}}
\newcommand{\deld}[2]{\dfrac{\partial #1}{\partial #2}}
\newcommand{\Deld}[3]{\dfrac{\partial^{#1} #2}{\partial #3^{#1}}}


\newcommand{\AstIg}{\stackrel{(\ast)}{=}}
\newcommand{\Hop}{\stackrel{L'H\hat{o}pital}{=}}

\newcommand{\red}[1]{{\color{red}#1}} % Para integrales, destacar los cambios.

% Método de integración
\newcommand{\MetInt}[2]{
    \left[\begin{array}{c}
        #1 \\ #2
    \end{array}\right]
}

% Declarar aplicaciones
% 1. Nombre aplicación
% 2. Dominio
% 3. Codominio
% 4. Variable
% 5. Imagen de la variable
\newcommand{\Func}[5]{
    \begin{equation*}
        \begin{array}{rrll}
            #1:& #2 & \longrightarrow & #3\\
               & #4 & \longmapsto & #5
        \end{array}
    \end{equation*}
}

%------------------------------------------------------------------------


\begin{document}
	
	% 1. Foto de fondo
	% 2. Título
	% 3. Encabezado Izquierdo
	% 4. Color de fondo
	% 5. Coord x del titulo
	% 6. Coord y del titulo
	% 7. Fecha
	
	
	% 1. Foto de fondo
% 2. Título
% 3. Encabezado Izquierdo
% 4. Color de fondo
% 5. Coord x del titulo
% 6. Coord y del titulo
% 7. Fecha

\newcommand{\portada}[7]{

    \portadaBase{#1}{#2}{#3}{#4}{#5}{#6}{#7}
    \portadaBook{#1}{#2}{#3}{#4}{#5}{#6}{#7}
}

\newcommand{\portadaExamen}[7]{

    \portadaBase{#1}{#2}{#3}{#4}{#5}{#6}{#7}
    \portadaArticle{#1}{#2}{#3}{#4}{#5}{#6}{#7}
}




\newcommand{\portadaBase}[7]{

    % Tiene la portada principal y la licencia Creative Commons
    
    % 1. Foto de fondo
    % 2. Título
    % 3. Encabezado Izquierdo
    % 4. Color de fondo
    % 5. Coord x del titulo
    % 6. Coord y del titulo
    % 7. Fecha
    
    
    \thispagestyle{empty}               % Sin encabezado ni pie de página
    \newgeometry{margin=0cm}        % Márgenes nulos para la primera página
    
    
    % Encabezado
    \fancyhead[L]{\helv #3}
    \fancyhead[R]{\helv \nouppercase{\leftmark}}
    
    
    \pagecolor{#4}        % Color de fondo para la portada
    
    \begin{figure}[p]
        \centering
        \transparent{0.3}           % Opacidad del 30% para la imagen
        
        \includegraphics[width=\paperwidth, keepaspectratio]{assets/#1}
    
        \begin{tikzpicture}[remember picture, overlay]
            \node[anchor=north west, text=white, opacity=1, font=\fontsize{60}{90}\selectfont\bfseries\sffamily, align=left] at (#5, #6) {#2};
            
            \node[anchor=south east, text=white, opacity=1, font=\fontsize{12}{18}\selectfont\sffamily, align=right] at (9.7, 3) {\textbf{\href{https://losdeldgiim.github.io/}{Los Del DGIIM}}};
            
            \node[anchor=south east, text=white, opacity=1, font=\fontsize{12}{15}\selectfont\sffamily, align=right] at (9.7, 1.8) {Doble Grado en Ingeniería Informática y Matemáticas\\Universidad de Granada};
        \end{tikzpicture}
    \end{figure}
    
    
    \restoregeometry        % Restaurar márgenes normales para las páginas subsiguientes
    \pagecolor{white}       % Restaurar el color de página
    
    
    \newpage
    \thispagestyle{empty}               % Sin encabezado ni pie de página
    \begin{tikzpicture}[remember picture, overlay]
        \node[anchor=south west, inner sep=3cm] at (current page.south west) {
            \begin{minipage}{0.5\paperwidth}
                \href{https://creativecommons.org/licenses/by-nc-nd/4.0/}{
                    \includegraphics[height=2cm]{assets/Licencia.png}
                }\vspace{1cm}\\
                Esta obra está bajo una
                \href{https://creativecommons.org/licenses/by-nc-nd/4.0/}{
                    Licencia Creative Commons Atribución-NoComercial-SinDerivadas 4.0 Internacional (CC BY-NC-ND 4.0).
                }\\
    
                Eres libre de compartir y redistribuir el contenido de esta obra en cualquier medio o formato, siempre y cuando des el crédito adecuado a los autores originales y no persigas fines comerciales. 
            \end{minipage}
        };
    \end{tikzpicture}
    
    
    
    % 1. Foto de fondo
    % 2. Título
    % 3. Encabezado Izquierdo
    % 4. Color de fondo
    % 5. Coord x del titulo
    % 6. Coord y del titulo
    % 7. Fecha


}


\newcommand{\portadaBook}[7]{

    % 1. Foto de fondo
    % 2. Título
    % 3. Encabezado Izquierdo
    % 4. Color de fondo
    % 5. Coord x del titulo
    % 6. Coord y del titulo
    % 7. Fecha

    % Personaliza el formato del título
    \pretitle{\begin{center}\bfseries\fontsize{42}{56}\selectfont}
    \posttitle{\par\end{center}\vspace{2em}}
    
    % Personaliza el formato del autor
    \preauthor{\begin{center}\Large}
    \postauthor{\par\end{center}\vfill}
    
    % Personaliza el formato de la fecha
    \predate{\begin{center}\huge}
    \postdate{\par\end{center}\vspace{2em}}
    
    \title{#2}
    \author{\href{https://losdeldgiim.github.io/}{Los Del DGIIM}}
    \date{Granada, #7}
    \maketitle
    
    \tableofcontents
}




\newcommand{\portadaArticle}[7]{

    % 1. Foto de fondo
    % 2. Título
    % 3. Encabezado Izquierdo
    % 4. Color de fondo
    % 5. Coord x del titulo
    % 6. Coord y del titulo
    % 7. Fecha

    % Personaliza el formato del título
    \pretitle{\begin{center}\bfseries\fontsize{42}{56}\selectfont}
    \posttitle{\par\end{center}\vspace{2em}}
    
    % Personaliza el formato del autor
    \preauthor{\begin{center}\Large}
    \postauthor{\par\end{center}\vspace{3em}}
    
    % Personaliza el formato de la fecha
    \predate{\begin{center}\huge}
    \postdate{\par\end{center}\vspace{5em}}
    
    \title{#2}
    \author{\href{https://losdeldgiim.github.io/}{Los Del DGIIM}}
    \date{Granada, #7}
    \thispagestyle{empty}               % Sin encabezado ni pie de página
    \maketitle
    \vfill
}
	\portadaExamen{ffccA4.jpg}{MN I\\Examen VI}{Métodos Numéricos I. Examen VI}{MidnightBlue}{-8}{28}{2024}{Roxana Acedo Parra}
	
	\begin{description}
		\item[Asignatura] Métodos Numéricos I.
		\item[Curso Académico] 2024-25.
		\item[Grado] Doble Grado en Ingeniería Informática y Matemáticas. También Grado en Matemáticas y Doble Grado en Matemáticas y Física.
		\item[Grupo] Único.
		\item[Profesor] Juan José Nieto Muñoz.
		\item[Fecha] 9 de abril de 2025.
		\item[Duración] 2 horas.
		\item[Descripción] Primer Parcial.
		\item[Observaciones] 
		Contiene respuestas a varias versiones, no a una única prueba. Este profesor, al darnos la corrección de este examen no diferenció entre las preguntas de nuestro examen y las del examen de Matemáticas y Matemáticas y Física. Los del DGIIM tuvimos que hacer el 1. a), c), d) y e), el 3) y 4).
	\end{description}
	\newpage
	
	\begin{ejercicio}[4 puntos]
		Justifica razonadamente si las siguientes afirmaciones son verdaderas o falsas.
		\begin{enumerate}[label=\alph*)]
			\item Si $f \in C^1(\mathbb{R})$ y $x_0 \in \mathbb{R}$, entonces se verifica que
			$$ f(y) = f(x_0) + f'(x_0)(y - x_0) + f[y, x_0, x_0](y - x_0)^2, \quad \forall y \in \mathbb{R}.$$
			
			\item Si $f \in C^1(\mathbb{R})$ y $x_0 \in \mathbb{R}$, y definimos $g(x) := f[x, x_0]$, entonces se verifica que
			$$ g'(x) = \frac{f'(x) - g(x)}{x - x_0}, \quad \forall x \in \mathbb{R}, \ x \ne x_0.$$
			
			\item Dados dos números reales: $y_0$ y  $y_1$ cumpliendo $y_0 < y_1$, y otros 2 $d_0$ y $d_1$ estrictamente positivos, entonces el polinomio de grado menor o igual a 3 que resuelve el siguiente problema de interpolación de tipo Hermite, es estrictamente creciente.
			
			$$ \begin{array}{c|c|c}
				x_i & 0 & 1 \\
				\hline
				p(x_i) & y_0 & y_1 \\
				\hline
				p'(x_i) & d_0 & d_1 \\
			\end{array} $$
			
			\item Dados 3 números reales: $y_0$, $y_1$, $y_2$, cumpliendo $y_0 < y_1 < y_2$ y otro $d_1$ estrictamente positivo, entonces el spline cuadrático $s \in S_2(0,1,2)$ que resuelve el siguiente problema, es estrictamente creciente;
			$$ s(0) = y_0, \quad s(1) = y_1, \quad s(2) = y_2, \quad s'(1) = d_1.$$
			
			\item Al interpolar una función $f(x)$ cualquiera en 5 puntos diferentes, sabemos que puede haber varios polinomios de grado menor o igual a 5 que la interpolen; pero si la función de partida es un polinomio de grado 3, entonces hay un único polinomio de grado menor o igual a 5 que la interpola: el propio polinomio.
			
			\item Al aproximar una función continua $f(x)$ cualquiera mediante mínimos cuadrados discretos usando su valor en 5 nodos diferentes: $\{x_1 < x_2 < \dots < x_5\}$, hay un único polinomio $ p(x)$ de grado menor o igual a 4 que la ajusta y que, además, minimiza el siguiente error:
			$$\sum_{i=1}^{5} |p(x_i) - f(x_i)|.\qquad \qquad \text{ (es sin elevar al cuadrado, no una errata) }$$
			
		\end{enumerate}
	\end{ejercicio}
	
	\begin{ejercicio}[2 puntos]
		Considera la función $f(x) = e^{-(\nicefrac{x^2}{2})}$. 
		\begin{enumerate}[label=\alph*)]
			\item Calcula, si es posible, un polinomio $p(x) \in \bb{P}_2[x]$ que interpole a $f$ en los nodos $1$ y $-1$ y a su derivada en $0$. ¿Cuántos polinomios hay cumpliendo estas condiciones?
			
			\item Si añadimos una cuarta condición: $p(0) = f(0) = 1$, pero seguimos buscando en $\bb{P}_2[x]$, que tiene dimensión 3, ¿habrá solución? En caso afirmativo, encuéntrala.
			
			\item Sabiendo que $|f^{(3)}(x)| \leq 2$ en $[-1, 1]$, ¿podrías dar una estimación del mayor error cometido al aproximar $f(x)$ en $[-1, 1]$ por las evaluaciones del polinomio del apartado anterior?
			
		\end{enumerate}
	\end{ejercicio}
	
	\begin{ejercicio}[2 puntos]
		Dada la nube de puntos: 
		$\begin{array}{c|c|c|c|c|c|c}
			x_i & 0 & 1 & 2 & 2 & 1 & 0 \\
			\hline
			y_i & 2 & 1 & 3 & 1 & 2 & 0 \\
		\end{array}$
		
		\begin{enumerate}[label=\alph*)]
			\item Calcula el polinomio de grado menor o igual a 1 que mejor aproxima esta nube de puntos.
			
			\item Si añadimos el punto $(4, 3)$ y rehacemos el ajuste, ¿qué polinomio obtendríamos y por qué?
			
			\item Si cambiamos los datos $y_i$ por $\displaystyle \frac{y_i}{2}$ y rehacemos el ajuste, ¿qué polinomio obtendríamos y por qué?
			
		\end{enumerate}
	\end{ejercicio}
	
	\begin{ejercicio}[4 puntos]
		Considera la función $f(x) = |x|$, el espacio $V = C\left([-\pi, \pi]\right)$ con el producto escalar continuo:
		
		$$\langle f, g \rangle = \frac{1}{2\pi} \int_{-\pi}^{\pi} f(x)g(x) \, dx,$$
		
		y el subespacio $H$ generado por las funciones: $\varphi_1(x) = \cos(x)$ y $\varphi_2(x) = \cos(2x).$
		
		\begin{enumerate}[label=\alph*)]
			\item ¿Es posible encontrar alguna mejor aproximación por mínimos cuadrados continuos de $f$ en $H$? En caso afirmativo, calcúlalas todas.
			
			\item Calcula la distancia entre $f$ y el subespacio $H$ y determina si se alcanza en algún $u \in H$.
			
			\item Determina de manera justificada, una base ortogonal de $H$.
		\end{enumerate}
	\end{ejercicio}
	
	\newpage
	\setcounter{ejercicio}{0}
	% (1)
	\begin{ejercicio}[4 puntos]
		Justifica razonadamente si las siguientes afirmaciones son verdaderas o falsas.
		\begin{enumerate}[label=\alph*)]
			\item 	Si $f \in C^1(\mathbb{R})$ y $x_0 \in \mathbb{R}$, entonces se verifica que
			$$ f(y) = f(x_0) + f'(x_0)(y - x_0) + f[y, x_0, x_0](y - x_0)^2, \quad \forall y \in \mathbb{R}.$$
			
				\textbf{Solución:} \fbox{Cierto.} Simplemente calculamos $f[y, x_0, x_0]$ a partir de la correspondiente tabla de diferencias divididas con el nodo $x_0$ repetido e $y$ como tercer nodo:
					$$
				\begin{array}{|c|c|}
					\hline
					x_i & f(x_i) \\
					\hline
					x_0 & f(x_0) \\
					\hline
					x_0 & f(x_0) \\
					\hline
					y & f(y) \\
					\hline
				\end{array}
				\begin{array}{|c|}
					\\
					\\
					\hline
					f'(x_0) \\
					\hline
					\displaystyle \frac{f(y)-f(x_0)}{y-x_0} \\
					\hline
				\end{array}
				\begin{array}{c|} 
					\\
					\\
					\hline
					\displaystyle \frac{\frac{f(y)-f(x_0)}{y-x_0} - f'(x_0)}{y-x_0} := f[x_0,x_0,y]\\
					\hline
				\end{array}
				$$
				
				Y basta despejar $f(y)$ de la última expresión.
			
			\item Si $f \in C^1(\mathbb{R})$ y $x_0 \in \mathbb{R}$, y definimos $g(x) := f[x, x_0]$, entonces se verifica que
			$$ g'(x) = \frac{f'(x) - g(x)}{x - x_0}, \quad \forall x \in \mathbb{R}, \ x \ne x_0.$$
			
				\textbf{Solución:} \fbox{Cierto.} Reescribimos $g(x)$ usando las propiedades de las diferencias divididas (se puede usar la tabla de diferencias divididas con 2 nodos: $x_0$ y $x$):
				
				$$g(x) := f[x,x_0] = \frac{f[x] - f[x_0]}{x - x_0} = \frac{f(x) - f(x_0)}{x - x_0}$$
				
				y, derivando, obtenemos el resultado
				$$g'(x) = \frac{f'(x)(x - x_0) - (f(x) - f(x_0))}{(x - x_0)^2} = \frac{f'(x) - g(x)}{x - x_0}.$$
				
			\item Dados dos números reales: $y_0$ y  $y_1$ cumpliendo $y_0 < y_1$, y otros 2 $d_0$ y $d_1$ estrictamente positivos, entonces el polinomio de grado menor o igual a 3 que resuelve el siguiente problema de interpolación de tipo Hermite, es estrictamente creciente.
			$$ \begin{array}{c|c|c}
				x_i & 0 & 1 \\
				\hline
				p(x_i) & y_0 & y_1 \\
				\hline
				p'(x_i) & d_0 & d_1 \\
			\end{array} $$
			
				\textbf{Solución:} \fbox{Falso.} Basta en pensar en polinomios de grado 3 (que son los que interpolan 4 datos que suban y bajen entre 0 y 1. Por concretar un contraejemplo: 
				$$ p(x) = (x - a)(x - b)(x - c), \quad \text{con } 0 < a < b < c < 1. $$
				
				\begin{figure}[H]
					\centering 
					\begin{tikzpicture}[scale=1]
						\begin{axis}[xmin=-1, xmax=1.5, ymin=-0.14, ymax= 0.08, axis x line=center, axis y line=center, xlabel=$x$, ylabel=$y$, xtick={0,1}, ytick={0}, thick]
							\addplot[blue!80!white, ultra thick, samples=200, domain= -1:2]{(x-0.2)*(x-0.6)*(x-0.9)};
							\filldraw[black] (0.2,0) circle (2pt) node[anchor=south]{a};
							\filldraw[black] (0.6,0) circle (2pt) node[anchor=south]{b};
							\filldraw[black] (0.9,0) circle (2pt) node[anchor=south]{c};
							\draw node[anchor=west, blue!80!white] at (1.1, 0.07) {$p(x)$};
							
							\filldraw[black] (1,0.032) circle (3pt);
							\draw[dashed, thick] (1,0)--(1, 0.032)--(-0.4, 0.032) node[anchor=east, black] at (-0.4, 0.032) {$y_1$};
							
							\filldraw[black] (0,-0.11) circle (3pt);
							\draw[dashed, thick] (0,-0.11)--(-0.4,-0.11) node[anchor=east, black] at (-0.4, -0.11) {$y_0$};
						\end{axis}
					\end{tikzpicture}
					\caption{Ejemplo: $p(x)=(x-0.2)(x-0.6)(x-0.9)$}
				\end{figure}
				
			\item Dados 3 números reales: $y_0$, $y_1$, $y_2$, cumpliendo $y_0 < y_1 < y_2$ y otro $d_1$ estrictamente positivo, entonces el spline cuadrático $s \in S_2(0,1,2)$ que resuelve el siguiente problema, es estrictamente creciente;
			$$ s(0) = y_0, \quad s(1) = y_1, \quad s(2) = y_2, \quad s'(1) = d_1.$$
			
				\textbf{Solución:} \fbox{Falso.} Basta pensar en dos polinomios de grado dos, uno convexo y otro cóncavo, que peguen bien en el nodo $x=1$. \\
				Por concretar un contraejemplo (pensamos en $y_1=0$ de modo que $x=1$ sea un cero y nodo en común):
				$$
				s(x) =
				\begin{cases}
					s(x)_1 =(x + a)(x - 1), & x \leq 1, \\
					s(x)_2 =(2 + a - x)(x - 1), & x \geq 1,
				\end{cases}
				$$
				con $a>0$ (el poner los ceros simétricos es para que salga derivable en 0 sin añadir nada).
				
				\begin{figure}[H]
					\centering 
					\begin{tikzpicture}[scale=1]
						\begin{axis}[xmin=-1.2, xmax=3.2, ymin=-0.6, ymax= 0.6, axis x line=center, axis y line=center, xlabel=$x$, ylabel=$y$, xtick={0,1,2}, ytick={0}, thick]
							\addplot[red!80!white, ultra thick, samples=200, domain= -1:1]{(x+0.3)*(x-1)};
							\addplot[blue!80!white, ultra thick, samples=200, domain= 1:3]{(2.3-x)*(x-1)};
							
							\filldraw[black] (-0.3,0) circle (2pt);
							\filldraw[black] (2.3,0) circle (2pt);
							\filldraw[black] (1,0) circle (3pt);
							
							\draw node[anchor=north, red!80!white] at (1, -0.3) {$s_1(x)$};
							\draw node[anchor=west, blue!80!white] at (2.5, -0.3) {$s_2(x)$};
							
							\filldraw[black] (2, 0.3) circle (3pt);
							\draw[dashed, thick] (2,0)--(2, 0.3)--(-0.7, 0.3) node[anchor=east, black] at (-0.7, 0.3) {$y_2$};
							
							\filldraw[black] (0, -0.3) circle (3pt);
							\draw[dashed, thick] (-0.7, -0.3)--(0, -0.3) node[anchor=east, black] at (-0.7, -0.3) {$y_0$};
							
							\draw node[anchor=north, black] at (-0.7, 0) {$y_1=0$};
							
							\draw node[anchor=east, black] at (-0.35, 0.08) {$(-a)$};
							\draw node[anchor=west, black] at (2.3, -0.08) {$(2+a)$};
						\end{axis}
					\end{tikzpicture}
					\caption{Ejemplo graficado con $a=0.3$}
				\end{figure}
				
			\item Al interpolar una función $f(x)$ cualquiera en 5 puntos diferentes, sabemos que puede haber varios polinomios de grado menor o igual a 5 que la interpolen; pero si la función de partida es un polinomio de grado 3, entonces hay un único polinomio de grado menor o igual a 5 que la interpola: el propio polinomio.
			
				\textbf{Solución.} \fbox{Falso.} La primera parte es, como sabemos, cierta: al interpolar en 5 puntos, hay unicidad en $\cc{P}_4$, pero nunca en $\cc{P}_5$; sea quien sea la $f$ que origina dichos puntos. Por tanto, que $f$ sea un polinomio de grado 3 no cambia nada.
			
			\item Al aproximar una función continua $f(x)$ cualquiera mediante mínimos cuadrados discretos usando su valor en 5 nodos diferentes: $\{x_1 < x_2 < \dots < x_5\}$, hay un único polinomio $ p(x)$ de grado menor o igual a 4 que la ajusta y que, además, minimiza el siguiente error:
			
			$$\sum_{i=1}^{5} |p(x_i) - f(x_i)|. \text{ (es sin elevar al cuadrado, no una errata) }$$
			
				\textbf{Solución:} \fbox{Cierto.} Al aproximar en 5 puntos con polinomios de $ \cc{P}_4[x] $ (que tiene dimensión 5), el polinomio resultante es en realidad el que la interpola, ya que este minimiza el error (cuadrático), pues justo da un error igual a 0 (el mínimo posible). Por tanto $|p(x_i) - f(x_i)| = 0$, con o sin cuadrado, también es el mínimo posible: cero.
			
		\end{enumerate} 
	\end{ejercicio}
	% (2)
	\begin{ejercicio}[2 puntos]
		Considera la función $f(x) = e^{-(\nicefrac{x^2}{2})}$. 
		\begin{enumerate}[label=\alph*)]
			\item Calcula, si es posible, un polinomio $p(x) \in \bb{P}_2[x]$ que interpole a $f$ en los nodos $1$ y $-1$ y a su derivada en $0$. ¿Cuántos polinomios hay cumpliendo estas condiciones? \\
			
			Al interpolar en 2 nodos de tipo Lagrange, con un dato tipo Hermite en otro nodo (dejando un hueco), a priori no sabemos qué puede pasar. Elegimos en este caso coeficientes indeterminados, 
			$\boxed{p(x) = A + Bx + Cx^2}$, e imponemos las condiciones a ver qué pasa:
			
			$$
			\begin{cases}
				p(-1) = f(-1) = \frac{1}{\sqrt{e}} \\
				p(1) = f(1) = \frac{1}{\sqrt{e}} \\
				p'(0) = f'(0) = 0.+
			\end{cases} \Rightarrow
			\begin{cases}
				A-B+C = \frac{1}{\sqrt{e}} \\
				A+B+C= \frac{1}{\sqrt{e}} \\
				B= 0
			\end{cases} \Rightarrow
			\begin{cases}
				A = \frac{1}{\sqrt{e}} -C \\
				B= 0
			\end{cases}
			$$
			
			$$ p_C(x)=\frac{1}{\sqrt{e}}+C(x^2-1), \quad C \in \bb{R}, $$
			
			lo que responde a la pregunta, hay infinitas posibilidades.
			
			\item Si añadimos una cuarta condición: $p(0) = f(0) = 1$, pero seguimos buscando en $\bb{P}_2[x]$, que tiene dimensión 3, ¿habrá solución? En caso afirmativo, encuéntrala. \\
			
			Al añadir el dato $p(0) = f(0)$, hemos rellenado el hueco y ya así tenemos datos tipo Hermite que dan unisolvencia en $\bb{P}_3[x]$. Pero como nos piden en $\bb{P}_2[x]$ de nuevo, podría no haber solución. 
			
			Aunque, visto lo ocurrido en el apartado anterior, parece que este dato simplemente va a fijar el valor de $C$: basta imponer ese dato más al que ya tenemos $p_C(x) = 1 + C(x^2 - 1)$ a ver qué pasa:
			
			$$ p_C(0)= f(0) = 1 \Rightarrow \frac{1}{\sqrt{e}}-C=1 \Rightarrow C=\frac{1}{\sqrt{e}}-1=\frac{1-\sqrt{e}}{\sqrt{e}}$$
			
			$$ \Rightarrow p(x)= 1+\frac{1-\sqrt{e}}{\sqrt{e}}x^2$$
			
			\item Sabiendo que $|f^{(3)}(x)| \leq 2$ en $[-1, 1]$, ¿podrías dar una estimación del mayor error cometido al aproximar $f(x)$ en $[-1, 1]$ por las evaluaciones del polinomio del apartado anterior? \\
			
			La clave es que el polinomio obtenido: $p(x) = 1 + \frac{\rho + 1}{\rho} x^2$ interpola a $f(x)$ en 3 nodos: $x_0 = -1$, $x_1 = 1$ y $x_2 = 0$ (además de a su derivada, pero eso lo obviamos), por lo que podemos usar la fórmula para el error de interpolación con $n = 2$:
	
			$$E(x) = f(x) - p(x) = \frac{f^{(3)}(\xi)}{3!} (x + 1)x(x - 1), \, x \in [-1,1]$$
			
			A partir de aquí, podemos hacer varias estimaciones, no se pide la óptima, por ejemplo:
			
			$$ |f(x)-p(x)| \leq \frac{2}{6}\cdot 2 \cdot 1 \cdot 2 = \frac{4}{3}, \,  x \in [-1,1] $$
			
		\end{enumerate}
	\end{ejercicio}
	% (3)
	\begin{ejercicio}[2 puntos]
		Dada la nube de puntos: 
		$\begin{array}{c|c|c|c|c|c|c}
			x_i & 0 & 1 & 2 & 2 & 1 & 0 \\
			\hline
			y_i & 2 & 1 & 3 & 1 & 2 & 0 \\
		\end{array}$
		
		\begin{enumerate}[label=\alph*)]
			\item Calcula el polinomio de grado menor o igual a 1 que mejor aproxima esta nube de puntos. \\
			
			Para buscar $p(x) = c_0 + c_1 x$, aplicamos los resultados de clase:
			$$
			A = 
			\begin{pmatrix}
				1 & 0 \\
				1 & 1 \\
				1 & 2 \\
				1 & 2 \\
				1 & 1 \\
				1 & 0
			\end{pmatrix}, \quad
			y =
			\begin{pmatrix}
				2 \\
				1 \\
				3 \\
				1 \\
				2 \\
				0 
			\end{pmatrix} \Rightarrow G = A^T A =
			\begin{pmatrix}
				6 & 6 \\
				6 & 10
			\end{pmatrix}, \,
			b = A^T y =
			\begin{pmatrix}
				9 \\
				11
			\end{pmatrix}
			$$
			Resolvemos
			
			$$\begin{pmatrix}
				6 & 6 \\
				6 & 10
			\end{pmatrix}
			\begin{pmatrix}
				c_0 \\
				c_1
			\end{pmatrix}=
			\begin{pmatrix}
				9 \\
				11
			\end{pmatrix} \Rightarrow
			\begin{pmatrix}
				c_0 \\
				c_1
			\end{pmatrix}=
			\begin{pmatrix}
				1 \\
				1/2
			\end{pmatrix} \Rightarrow
			p(x)=1+\frac{x}{2}. $$
			
			\item Si añadimos el punto $(4, 3)$ y rehacemos el ajuste, ¿qué polinomio obtendríamos y por qué? \\
			
			Puesto que $p(4) = 1 + \frac{4}{2} = 3$, el punto adicional $(4, 3)$ resulta que es un punto de la gráfica de $p(x)$ y, aunque se pueden repetir los cálculos con un punto más, sabemos que saldrá el mismo polinomio, pues cualquier otro aumentaría el error.
			
			
			\item Si cambiamos los datos $y_i$ por $\displaystyle \frac{y_i}{2}$ y rehacemos el ajuste, ¿qué polinomio obtendríamos y por qué? \\
			
			Aunque se puede repetir la cuenta, en general, al multiplicar los datos por un escalar no nulo y rehacer el ajuste, se obtiene el mismo polinomio de antes multiplicado por el mismo escalar; veámoslo usando el sistema abstracto. Llamamos $c_{\lambda}$ a los nuevos coeficientes del nuevo polinomio.
			
			$$(A^T A)c_{\lambda} = A^T(\lambda y) = \lambda A^T y \, \Rightarrow \, c_{\lambda} = (A^T A)^{-1} (\lambda A^T y) = \lambda (A^TA)^{-1}A^Ty=\lambda c.$$
			
			Por tanto, en nuestro caso, se obtendría $p(x) = \displaystyle \frac{1}{2} + \frac{x}{4}.$
			
			
		\end{enumerate}
	\end{ejercicio}
	% (4)
	\begin{ejercicio}[4 puntos]
		Considera la función $f(x) = |x|$, el espacio $V = C\left([-\pi, \pi]\right)$ con el producto escalar continuo:
		
		$$\langle f, g \rangle = \frac{1}{2\pi} \int_{-\pi}^{\pi} f(x)g(x) \, dx,$$
		
		y el subespacio $H$ generado por las funciones: $\varphi_1(x) = \cos(x)$ y $\varphi_2(x) = \cos(2x).$
		
		\begin{enumerate}[label=\alph*)]
			\item ¿Es posible encontrar alguna mejor aproximación por mínimos cuadrados continuos de $f$ en $H$? En caso afirmativo, calcúlalas todas.\\
			
				Este ejercicio es aplicación directa del teorema de mínimos cuadrados, puesto que $(V, \langle \cdot, \cdot \rangle)$ es prehilbertiano y $H \subseteq V$ es de dimensión finita. Por lo tanto, sí hay una única mejor aproximación de $f$ en $H$ y sabemos calcularla, pues incluso nos han dado una base de $H$: $\mathcal{B} = \{\cos(x), \cos(2x)\}$. Concretamente $\boxed{u_f = c_0 \cos(x) + c_1 \cos(2x)}$ y $c = (c_0, c_1)$ es la única solución del sistema de Gramm.
				$$Gc = b \text{ con } G = 
				\begin{pmatrix}
					\langle \cos(x), \cos(x) \rangle & \langle \cos(x), \cos(2x) \rangle \\
					\langle \cos(2x), \cos(x) \rangle & \langle \cos(2x), \cos(2x) \rangle
				\end{pmatrix}, \, b = 
				\begin{pmatrix}
					\langle f(x), \cos(x) \rangle \\
					\langle f(x), \cos(2x) \rangle
				\end{pmatrix}$$
				
				Haciendo las (tediosas pero triviales) cuentas, el sistema obtenido es
				$$\begin{pmatrix}
					\nicefrac{1}{2} & 0 \\ 
					0 & \nicefrac{1}{2}
				\end{pmatrix}
				\begin{pmatrix}
					c_0 \\
					c_1
				\end{pmatrix}
				=
				\begin{pmatrix}
					\nicefrac{-2}{\pi}\\
					0
				\end{pmatrix} \Rightarrow 
				\begin{pmatrix}
					c_0 \\
					c_1
				\end{pmatrix}
				=
				\begin{pmatrix}
					\nicefrac{-4}{\pi} \\
					0
				\end{pmatrix} \Rightarrow p(x) = \frac{-4}{\pi} \cos(x) $$
				
				
				
			\item Calcula la distancia entre $f$ y el subespacio $H$ y determina si se alcanza en algún $u \in H$. \\
			
			De nuevo usando el teorema de la mejor aproximación, sabemos que sí se alcanza, en $u_f$:
			$$\|f - u_f\| = \min_{u \in H} \|f - u\| = \operatorname{dist}(f, H).$$
			
			Por tanto solo hay que calcular
			$$ \|f - u_f\| = \sqrt{\langle f - u_f, f - u_f \rangle} = \\
			\left( \frac{1}{2\pi} \int_{-\pi}^{\pi} \left( |x| + \frac{4}{\pi} \cos(x) \right)^2 dx \right)^{1/2} = \\
			\sqrt{\frac{\pi^4 -24}{ 3\pi^2}}. $$
			
			\item Determina de manera justificada, una base ortogonal de $H$. \\
			
			Al calcular la matriz de Gramm ya hemos visto que $\langle \cos(x), \cos(2x) \rangle = 0$, por lo que la base que nos han dado ya es ortogonal, y no hay que hacer nada, salvo indicarlo. Y si uno no se da cuenta y aplica Gramm-Schmidt, todas las operaciones también están incluidas en el cálculo de $G$:
			$$\varphi_0 = \text{cos}(x), \quad \bar{\varphi_1}= \varphi_1 - \frac{\langle \varphi_1, \varphi_0 \rangle }{\langle \varphi_1, \varphi_0 \rangle}\varphi_0 = \text{cos}(2x) - \frac{\langle \text{cos}(2x), \text{cos}(x) \rangle}{\langle \text{cos}(x), \text{cos}(x) \rangle}\text{cos}(x) = \text{cos}(2x)$$
			
			
		\end{enumerate}
	\end{ejercicio}

	
\end{document}
