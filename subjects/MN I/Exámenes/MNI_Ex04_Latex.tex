\documentclass[12pt]{article}

% Idioma y codificación
\usepackage[spanish, es-tabla]{babel}       %es-tabla para que se titule "Tabla"
\usepackage[utf8]{inputenc}

% Márgenes
\usepackage[a4paper,top=3cm,bottom=2.5cm,left=3cm,right=3cm]{geometry}

% Comentarios de bloque
\usepackage{verbatim}

% Paquetes de links
\usepackage[hidelinks]{hyperref}    % Permite enlaces
\usepackage{url}                    % redirecciona a la web

% Más opciones para enumeraciones
\usepackage{enumitem}

% Personalizar la portada
\usepackage{titling}

% Paquetes de tablas
\usepackage{multirow}


%------------------------------------------------------------------------

%Paquetes de figuras
\usepackage{caption}
\usepackage{subcaption} % Figuras al lado de otras
\usepackage{float}      % Poner figuras en el sitio indicado H.


% Paquetes de imágenes
\usepackage{graphicx}       % Paquete para añadir imágenes
\usepackage{transparent}    % Para manejar la opacidad de las figuras

% Paquete para usar colores
\usepackage[dvipsnames]{xcolor}
\usepackage{pagecolor}      % Para cambiar el color de la página

% Habilita tamaños de fuente mayores
\usepackage{fix-cm}

% Para los gráficos
\usepackage{tikz}

% Para poder situar los nodos en los grafos
\usetikzlibrary{positioning}


%------------------------------------------------------------------------

% Paquetes de matemáticas
\usepackage{mathtools, amsfonts, amssymb, mathrsfs}
\usepackage[makeroom]{cancel}     % Simplificar tachando
\usepackage{polynom}    % Divisiones y Ruffini
\usepackage{units} % Para poner fracciones diagonales con \nicefrac

\usepackage{pgfplots}   %Representar funciones
\pgfplotsset{compat=1.18}  % Versión 1.18

\usepackage{tikz-cd}    % Para usar diagramas de composiciones
\usetikzlibrary{calc}   % Para usar cálculo de coordenadas en tikz

%Definición de teoremas, etc.
\usepackage{amsthm}
%\swapnumbers   % Intercambia la posición del texto y de la numeración

\theoremstyle{plain}

\makeatletter
\@ifclassloaded{article}{
  \newtheorem{teo}{Teorema}[section]
}{
  \newtheorem{teo}{Teorema}[chapter]  % Se resetea en cada chapter
}
\makeatother

\newtheorem{coro}{Corolario}[teo]           % Se resetea en cada teorema
\newtheorem{prop}[teo]{Proposición}         % Usa el mismo contador que teorema
\newtheorem{lema}[teo]{Lema}                % Usa el mismo contador que teorema

\theoremstyle{remark}
\newtheorem*{observacion}{Observación}

\theoremstyle{definition}

\makeatletter
\@ifclassloaded{article}{
  \newtheorem{definicion}{Definición} [section]     % Se resetea en cada chapter
}{
  \newtheorem{definicion}{Definición} [chapter]     % Se resetea en cada chapter
}
\makeatother

\newtheorem*{notacion}{Notación}
\newtheorem*{ejemplo}{Ejemplo}
\newtheorem*{ejercicio*}{Ejercicio}             % No numerado
\newtheorem{ejercicio}{Ejercicio} [section]     % Se resetea en cada section


% Modificar el formato de la numeración del teorema "ejercicio"
\renewcommand{\theejercicio}{%
  \ifnum\value{section}=0 % Si no se ha iniciado ninguna sección
    \arabic{ejercicio}% Solo mostrar el número de ejercicio
  \else
    \thesection.\arabic{ejercicio}% Mostrar número de sección y número de ejercicio
  \fi
}


% \renewcommand\qedsymbol{$\blacksquare$}         % Cambiar símbolo QED
%------------------------------------------------------------------------

% Paquetes para encabezados
\usepackage{fancyhdr}
\pagestyle{fancy}
\fancyhf{}

\newcommand{\helv}{ % Modificación tamaño de letra
\fontfamily{}\fontsize{12}{12}\selectfont}
\setlength{\headheight}{15pt} % Amplía el tamaño del índice


%\usepackage{lastpage}   % Referenciar última pag   \pageref{LastPage}
\fancyfoot[C]{\thepage}

%------------------------------------------------------------------------

% Conseguir que no ponga "Capítulo 1". Sino solo "1."
\makeatletter
\@ifclassloaded{book}{
  \renewcommand{\chaptermark}[1]{\markboth{\thechapter.\ #1}{}} % En el encabezado
    
  \renewcommand{\@makechapterhead}[1]{%
  \vspace*{50\p@}%
  {\parindent \z@ \raggedright \normalfont
    \ifnum \c@secnumdepth >\m@ne
      \huge\bfseries \thechapter.\hspace{1em}\ignorespaces
    \fi
    \interlinepenalty\@M
    \Huge \bfseries #1\par\nobreak
    \vskip 40\p@
  }}
}
\makeatother

%------------------------------------------------------------------------
% Paquetes de cógido
\usepackage{minted}
\renewcommand\listingscaption{Código fuente}

\usepackage{fancyvrb}
% Personaliza el tamaño de los números de línea
\renewcommand{\theFancyVerbLine}{\small\arabic{FancyVerbLine}}

% Estilo para C++
\newminted{cpp}{
    frame=lines,
    framesep=2mm,
    baselinestretch=1.2,
    linenos,
    escapeinside=||
}

% para minted
\definecolor{LightGray}{rgb}{0.95,0.95,0.92}
\setminted{
    linenos=true,
    stepnumber=5,
    numberfirstline=true,
    autogobble,
    breaklines=true,
    breakautoindent=true,
    breaksymbolleft=,
    breaksymbolright=,
    breaksymbolindentleft=0pt,
    breaksymbolindentright=0pt,
    breaksymbolsepleft=0pt,
    breaksymbolsepright=0pt,
    fontsize=\footnotesize,
    bgcolor=LightGray,
    numbersep=10pt
}


\usepackage{listings} % Para incluir código desde un archivo

\renewcommand\lstlistingname{Código Fuente}
\renewcommand\lstlistlistingname{Índice de Códigos Fuente}

% Definir colores
\definecolor{vscodepurple}{rgb}{0.5,0,0.5}
\definecolor{vscodeblue}{rgb}{0,0,0.8}
\definecolor{vscodegreen}{rgb}{0,0.5,0}
\definecolor{vscodegray}{rgb}{0.5,0.5,0.5}
\definecolor{vscodebackground}{rgb}{0.97,0.97,0.97}
\definecolor{vscodelightgray}{rgb}{0.9,0.9,0.9}

% Configuración para el estilo de C similar a VSCode
\lstdefinestyle{vscode_C}{
  backgroundcolor=\color{vscodebackground},
  commentstyle=\color{vscodegreen},
  keywordstyle=\color{vscodeblue},
  numberstyle=\tiny\color{vscodegray},
  stringstyle=\color{vscodepurple},
  basicstyle=\scriptsize\ttfamily,
  breakatwhitespace=false,
  breaklines=true,
  captionpos=b,
  keepspaces=true,
  numbers=left,
  numbersep=5pt,
  showspaces=false,
  showstringspaces=false,
  showtabs=false,
  tabsize=2,
  frame=tb,
  framerule=0pt,
  aboveskip=10pt,
  belowskip=10pt,
  xleftmargin=10pt,
  xrightmargin=10pt,
  framexleftmargin=10pt,
  framexrightmargin=10pt,
  framesep=0pt,
  rulecolor=\color{vscodelightgray},
  backgroundcolor=\color{vscodebackground},
}

%------------------------------------------------------------------------

% Comandos definidos
\newcommand{\bb}[1]{\mathbb{#1}}
\newcommand{\cc}[1]{\mathcal{#1}}

% I prefer the slanted \leq
\let\oldleq\leq % save them in case they're every wanted
\let\oldgeq\geq
\renewcommand{\leq}{\leqslant}
\renewcommand{\geq}{\geqslant}

% Si y solo si
\newcommand{\sii}{\iff}

% Letras griegas
\newcommand{\eps}{\epsilon}
\newcommand{\veps}{\varepsilon}
\newcommand{\lm}{\lambda}

\newcommand{\ol}{\overline}
\newcommand{\ul}{\underline}
\newcommand{\wt}{\widetilde}
\newcommand{\wh}{\widehat}

\let\oldvec\vec
\renewcommand{\vec}{\overrightarrow}

% Derivadas parciales
\newcommand{\del}[2]{\frac{\partial #1}{\partial #2}}
\newcommand{\Del}[3]{\frac{\partial^{#1} #2}{\partial #3^{#1}}}
\newcommand{\deld}[2]{\dfrac{\partial #1}{\partial #2}}
\newcommand{\Deld}[3]{\dfrac{\partial^{#1} #2}{\partial #3^{#1}}}


\newcommand{\AstIg}{\stackrel{(\ast)}{=}}
\newcommand{\Hop}{\stackrel{L'H\hat{o}pital}{=}}

\newcommand{\red}[1]{{\color{red}#1}} % Para integrales, destacar los cambios.

% Método de integración
\newcommand{\MetInt}[2]{
    \left[\begin{array}{c}
        #1 \\ #2
    \end{array}\right]
}

% Declarar aplicaciones
% 1. Nombre aplicación
% 2. Dominio
% 3. Codominio
% 4. Variable
% 5. Imagen de la variable
\newcommand{\Func}[5]{
    \begin{equation*}
        \begin{array}{rrll}
            #1:& #2 & \longrightarrow & #3\\
               & #4 & \longmapsto & #5
        \end{array}
    \end{equation*}
}

%------------------------------------------------------------------------



\begin{document}

    % 1. Foto de fondo
    % 2. Título
    % 3. Encabezado Izquierdo
    % 4. Color de fondo
    % 5. Coord x del titulo
    % 6. Coord y del titulo
    % 7. Fecha

    
    % 1. Foto de fondo
% 2. Título
% 3. Encabezado Izquierdo
% 4. Color de fondo
% 5. Coord x del titulo
% 6. Coord y del titulo
% 7. Fecha

\newcommand{\portada}[7]{

    \portadaBase{#1}{#2}{#3}{#4}{#5}{#6}{#7}
    \portadaBook{#1}{#2}{#3}{#4}{#5}{#6}{#7}
}

\newcommand{\portadaExamen}[7]{

    \portadaBase{#1}{#2}{#3}{#4}{#5}{#6}{#7}
    \portadaArticle{#1}{#2}{#3}{#4}{#5}{#6}{#7}
}




\newcommand{\portadaBase}[7]{

    % Tiene la portada principal y la licencia Creative Commons
    
    % 1. Foto de fondo
    % 2. Título
    % 3. Encabezado Izquierdo
    % 4. Color de fondo
    % 5. Coord x del titulo
    % 6. Coord y del titulo
    % 7. Fecha
    
    
    \thispagestyle{empty}               % Sin encabezado ni pie de página
    \newgeometry{margin=0cm}        % Márgenes nulos para la primera página
    
    
    % Encabezado
    \fancyhead[L]{\helv #3}
    \fancyhead[R]{\helv \nouppercase{\leftmark}}
    
    
    \pagecolor{#4}        % Color de fondo para la portada
    
    \begin{figure}[p]
        \centering
        \transparent{0.3}           % Opacidad del 30% para la imagen
        
        \includegraphics[width=\paperwidth, keepaspectratio]{assets/#1}
    
        \begin{tikzpicture}[remember picture, overlay]
            \node[anchor=north west, text=white, opacity=1, font=\fontsize{60}{90}\selectfont\bfseries\sffamily, align=left] at (#5, #6) {#2};
            
            \node[anchor=south east, text=white, opacity=1, font=\fontsize{12}{18}\selectfont\sffamily, align=right] at (9.7, 3) {\textbf{\href{https://losdeldgiim.github.io/}{Los Del DGIIM}}};
            
            \node[anchor=south east, text=white, opacity=1, font=\fontsize{12}{15}\selectfont\sffamily, align=right] at (9.7, 1.8) {Doble Grado en Ingeniería Informática y Matemáticas\\Universidad de Granada};
        \end{tikzpicture}
    \end{figure}
    
    
    \restoregeometry        % Restaurar márgenes normales para las páginas subsiguientes
    \pagecolor{white}       % Restaurar el color de página
    
    
    \newpage
    \thispagestyle{empty}               % Sin encabezado ni pie de página
    \begin{tikzpicture}[remember picture, overlay]
        \node[anchor=south west, inner sep=3cm] at (current page.south west) {
            \begin{minipage}{0.5\paperwidth}
                \href{https://creativecommons.org/licenses/by-nc-nd/4.0/}{
                    \includegraphics[height=2cm]{assets/Licencia.png}
                }\vspace{1cm}\\
                Esta obra está bajo una
                \href{https://creativecommons.org/licenses/by-nc-nd/4.0/}{
                    Licencia Creative Commons Atribución-NoComercial-SinDerivadas 4.0 Internacional (CC BY-NC-ND 4.0).
                }\\
    
                Eres libre de compartir y redistribuir el contenido de esta obra en cualquier medio o formato, siempre y cuando des el crédito adecuado a los autores originales y no persigas fines comerciales. 
            \end{minipage}
        };
    \end{tikzpicture}
    
    
    
    % 1. Foto de fondo
    % 2. Título
    % 3. Encabezado Izquierdo
    % 4. Color de fondo
    % 5. Coord x del titulo
    % 6. Coord y del titulo
    % 7. Fecha


}


\newcommand{\portadaBook}[7]{

    % 1. Foto de fondo
    % 2. Título
    % 3. Encabezado Izquierdo
    % 4. Color de fondo
    % 5. Coord x del titulo
    % 6. Coord y del titulo
    % 7. Fecha

    % Personaliza el formato del título
    \pretitle{\begin{center}\bfseries\fontsize{42}{56}\selectfont}
    \posttitle{\par\end{center}\vspace{2em}}
    
    % Personaliza el formato del autor
    \preauthor{\begin{center}\Large}
    \postauthor{\par\end{center}\vfill}
    
    % Personaliza el formato de la fecha
    \predate{\begin{center}\huge}
    \postdate{\par\end{center}\vspace{2em}}
    
    \title{#2}
    \author{\href{https://losdeldgiim.github.io/}{Los Del DGIIM}}
    \date{Granada, #7}
    \maketitle
    
    \tableofcontents
}




\newcommand{\portadaArticle}[7]{

    % 1. Foto de fondo
    % 2. Título
    % 3. Encabezado Izquierdo
    % 4. Color de fondo
    % 5. Coord x del titulo
    % 6. Coord y del titulo
    % 7. Fecha

    % Personaliza el formato del título
    \pretitle{\begin{center}\bfseries\fontsize{42}{56}\selectfont}
    \posttitle{\par\end{center}\vspace{2em}}
    
    % Personaliza el formato del autor
    \preauthor{\begin{center}\Large}
    \postauthor{\par\end{center}\vspace{3em}}
    
    % Personaliza el formato de la fecha
    \predate{\begin{center}\huge}
    \postdate{\par\end{center}\vspace{5em}}
    
    \title{#2}
    \author{\href{https://losdeldgiim.github.io/}{Los Del DGIIM}}
    \date{Granada, #7}
    \thispagestyle{empty}               % Sin encabezado ni pie de página
    \maketitle
    \vfill
}
    \portadaExamen{ffccA4.jpg}{MN I\\Examen IV}{MN I. Examen IV}{MidnightBlue}{-8}{28}{2023}

    \begin{description}
        \item[Asignatura] Métodos Numéricos I.
        \item[Curso Académico] 2021-22.
        \item[Grado] Matemáticas.
        \item[Grupo] B.
        \item[Profesor] Teresa Encarnación Pérez Fernández.
        \item[Descripción] Convocatoria Ordinaria.
        \item[Fecha] 13 de junio de 2022.
        %\item[Duración] 60 minutos.
    
    \end{description}
    \newpage
    
    \textbf{Primera Parte} [4 puntos]
\begin{ejercicio}
    Se pretende resolver el sistema $Ax = b$ donde $A$ es la matriz
    \begin{equation*}
        A=\left( \begin{array}{ccc}
            -3 & a & -a \\
            -4 & 3 & -4 \\
            2 & 7 & -4
        \end{array} \right)
    \end{equation*}
    y $b = (a, 3, 1)^T $, donde $a\in \bb{R}$ es un parámetro.

    \begin{enumerate}
        \item Determine para que valores del parámetro $a$:
        \begin{enumerate}
            \item Se puede resolver utilizando el método de Gauss sin intercambio de filas.\\

            Para ello, es necesario que $a_{kk}^{(k)}\neq 0 \;\forall k=1,\dots, n$.

            \begin{multline*}
                \left( \begin{array}{ccc}
                    -3 & a & -a \\
                    -4 & 3 & -4 \\
                    2 & 7 & -4
                \end{array} \right)
                \xrightarrow[F'_3=F_3 +\frac{2}{3}F_1]{F'_2=F_2 -\frac{4}{3}F_1}
                \left( \begin{array}{ccc}
                    -3 & a & -a \\
                    0 & 3-\frac{4a}{3} & -4+\frac{4a}{3} \\
                    0 & 7+\frac{2a}{3} & -4-\frac{2a}{3}
                \end{array} \right)
                \longrightarrow \\
                \xrightarrow[m_{3,2}=-\frac{7+\frac{2a}{3}}{3-\frac{4a}{3}} = \frac{21+2a}{4a-9}]{F'_3=F_3 +m_{3,2}F_2}
                \left( \begin{array}{ccc}
                    -3 & a & -a \\
                    0 & 3-\frac{4a}{3} & -4+\frac{4a}{3} \\
                    0 & 0 & \frac{10a-48}{4a-9}
                \end{array} \right)
            \end{multline*}

            $a_{11}^{(1)}=-3\neq 0$\\
            $a_{22}^{(2)}=3-\frac{4a}{3} = 0 \Longleftrightarrow 9=4a \Longleftrightarrow a=\frac{9}{4}$\\
            $a_{33}^{(3)}=\frac{10a-48}{4a-9} = 0 \Longleftrightarrow 10a=48 \Longleftrightarrow a=\frac{48}{10}$

            Por tanto, se puede resolver siempre que $a\neq \{\frac{9}{4}, \frac{48}{10}\}$.
            
            \item Se puede resolver usando una descomposición LU de $A$.\\

            En este caso, es necesario que todos sus menores principales sean no nulos.
            \begin{equation*}
                |-3|=-3 \qquad 
                \left| \begin{array}{cc}
                    -3 & a \\
                    -4 & 3 \\
                \end{array} \right| = -9+4a = 0 \Longleftrightarrow a=\frac{9}{4}
            \end{equation*}
            \begin{equation*}
                |A|=36 -8a +28a +6a -84-16a = 10a -48 = 0 \Longleftrightarrow a=\frac{48}{10}
            \end{equation*}
            Por tanto, se puede resolver siempre que $a\neq \{\frac{9}{4}, \frac{48}{10}\}$. Como vemos, es análogo al método de Gauss sin intercambio de filas.

            \item El método iterativo de Gauss-Seidel es convergente.\\

            En el caso de Gauss-Seidel, la matriz de descomposición es $Q=D+L$. Por tanto,
            \begin{multline*}
                Ax=b \Longrightarrow (Q-(Q-A))x=b \Longrightarrow Qx = (Q-A)x + b \Longrightarrow \\ \Longrightarrow x=Q^{-1}(Q-A)x + Q^{-1}b = (I-Q^{-1}A)x + Q^{-1}b
            \end{multline*}
            Por tanto, $B_{G-S} = I-Q^{-1}A = I-(D+L)^{-1}A$

            \begin{equation*}
                (D+L)^{-1} = \left( \begin{array}{ccc}
                    -3 & 0 & 0 \\
                    -4 & 3 & 0 \\
                    2 & 7 & -4
                \end{array} \right)^{-1} = \left( \begin{array}{ccc}
                      -\frac{1}{3} &  0	&   0\\
                      -\frac{4}{9}	& \frac{1}{3}	&   0\\
                    -\frac{17}{18}	& \frac{7}{12}	& -\frac{1}{4}
                \end{array} \right)
            \end{equation*}

            Por tanto,
            \begin{equation*}
                B_{G-S} = I-Q^{-1}A = I-(D+L)^{-1}A = \left( \begin{array}{ccc}
                    0 & \frac{a}{3} & -\frac{a}{3} \\
                    0 & \frac{4a}{9} & \frac{-4a+12}{9} \\
                    0 & \frac{17a}{18} & \frac{-17a+42}{18}
                \end{array} \right)
            \end{equation*}

            Para ver si es convergente, calculamos sus valores propios.
            \begin{multline*}
                P_{B_{G-S}}(\lambda) = \left| \begin{array}{ccc}
                    -\lambda & \frac{a}{3} & -\frac{a}{3} \\
                    0 & \frac{4a}{9}-\lambda & \frac{-4a+12}{9} \\
                    0 & \frac{17a}{18} & \frac{-17a+42}{18}-\lambda
                \end{array} \right| = -\lambda\cdot \frac{1}{9}\cdot \frac{1}{18}\left| \begin{array}{cc}
                    4a-9\lambda & -4a+12 \\
                    17a & -17a+42-18\lambda
                \end{array} \right| \\
                = -\frac{\lambda}{9\cdot 18} [(4a-9\lambda)(-17a+42-18\lambda) -17a(-4a+12)]
                =\\=
                -\frac{\lambda}{9\cdot 18} (-68a^2 +168a - 72a\lambda+153a\lambda-378\lambda+162\lambda^2 +68a^2-204a)
                =\\=
                -\frac{\lambda}{9\cdot 18}[162\lambda^2 +(81a-378)\lambda-36a]
                =
                -\frac{\lambda}{18}[18\lambda^2 +(9a-42)\lambda-4a]
            \end{multline*}
            \begin{multline*}
                \lambda_2
                = \frac{-9a+42 \pm \sqrt{(9a-42)^2+16\cdot 18a}}{2\cdot 18} 
                =\\=
                \frac{-9a+42 \pm \sqrt{81a^2+1764-756a+16\cdot 18a}}{2\cdot 18} 
                =\\=
                \frac{-9a+42 \pm \sqrt{81a^2-468+1764}}{2\cdot 18}
                =
                \frac{-3a+14 \pm \sqrt{9a^2-52a+196}}{2\cdot 6} 
            \end{multline*}
        \end{enumerate}

        \item Para $a = 1$ escriba las ecuaciones explícitas del método de Jacobi y Gauss-Seidel y realice tres iteraciones de ambos métodos.\\
        \begin{equation*}
            \left. \begin{array}{rrrrrrrr}
                -&3x_1&+&x_2&-&x_3 &=& 1 \\
                -&4x_1&+&3x_2&-&4x_3&=&3 \\
                &2x_1&+&7x_2&-&4x_3 &=& 1
            \end{array}\right\} \Longrightarrow
            \left. \begin{array}{rcl}
                x_1 &=& -\frac{1}{3}(1-x_2+x_3) \\
                x_2 &=& \frac{1}{3}(3+4x_1+4x_3) \\
                x_3 &=& -\frac{1}{4}(1-2x_1-7x_2)
            \end{array}\right\}
        \end{equation*}

        Ecuaciones de Jacobi:
        \begin{equation*}
            \left. \begin{array}{rcl}
                x_1^{(k+1)} &=& -\frac{1}{3}(1-x_2^{(k)}+x_3^{(k)}) \\
                x_2^{(k+1)} &=& \frac{1}{3}(3+4x_1^{(k)}+4x_3^{(k)}) \\
                x_3^{(k+1)} &=& -\frac{1}{4}(1-2x_1^{(k)}-7x_2^{(k)})
            \end{array}\right\}
        \end{equation*}
        Tres iteraciones del método de Jacobi:
        \begin{equation*}
            \begin{array}{c|ccc}
                k & x_1 & x_2 & x_3 \\ \hline
                0 & 0 & 0 & 0 \\
                1 & -\frac{1}{3} & 1 & -\frac{1}{4} \\
                2 & \frac{1}{12} & \frac{2}{9} & \frac{4}{3} \\
                3 & -\frac{19}{27} &+\frac{26}{9}& +\frac{13}{72}
            \end{array}
        \end{equation*}

        Ecuaciones de Gauss-Seidel:
        \begin{equation*}
            \left. \begin{array}{rcl}
                x_1^{(k+1)} &=& -\frac{1}{3}(1-x_2^{(k)}+x_3^{(k)}) \\
                x_2^{(k+1)} &=& \frac{1}{3}(3+4x_1^{(k+1)}+4x_3^{(k)}) \\
                x_3^{(k+1)} &=& -\frac{1}{4}(1-2x_1^{(k+1)}-7x_2^{(k+1)})
            \end{array}\right\}
        \end{equation*}
        Tres iteraciones del método de Gauss-Seidel:
        \begin{equation*}
            \begin{array}{c|ccc}
                k & x_1 & x_2 & x_3 \\ \hline
                0 & 0 & 0 & 0 \\
                1 & -\frac{1}{3} & \frac{5}{9} & \frac{5}{9} \\
                2 & -\frac{1}{3} & +\frac{35}{27} & +\frac{50}{27} \\
                3 & -\frac{14}{27}  & \frac{25}{9} & \frac{235}{54} \\
            \end{array}
        \end{equation*}

        \item Para $a = 0$ resuelva el sistema utilizando el método de Gauss con pivote total.
        \begin{multline*}
            \left( \begin{array}{ccc|c}
                -3 & 0 & 0 & 0 \\
                -4 & 3 & -4 & 3\\
                2 & 7 & -4 & 1
            \end{array} \right)
            \xrightarrow[F'_3=F_3 +\frac{2}{3}F_1]{F'_2=F_2 -\frac{4}{3}F_1}
            \left( \begin{array}{ccc|c}
                -3 & 0 & 0 & 0\\
                0 & 3 & -4 & 3\\
                0 & 7 & -4 & 1
            \end{array} \right)
            \longrightarrow \\
            \xrightarrow[m_{3,2}=-\frac{7}{3}]{F'_3=F_3 +m_{3,2}F_2}
            \left( \begin{array}{ccc|c}
                -3 & 0 & 0 & 0\\
                0 & 3 & -4 & 3\\
                0 & 0 & \frac{16}{3} & -6
            \end{array} \right)
        \end{multline*}
        Por tanto,
        \begin{equation*}
            x_3 = -\frac{18}{16} = -\frac{9}{8}
            \qquad \qquad
            x_2 = \frac{3+4x_3}{3} = \frac{5}{2}
            \qquad \qquad
            x_1 = 0
        \end{equation*}
    \end{enumerate}
\end{ejercicio}

\begin{ejercicio}
    Dada la matriz $A = (a_{ij})^n_{i,j=1}$, para $n \geq 1$, se considera la expresión
    \begin{equation} \label{Norma}
        ||A||_S = \sum_{i=1}^n \sum_{j=1}^n |a_{ij}|
    \end{equation}
    \begin{enumerate}
        \item Demuestre que (\ref{Norma}) define una norma matricial.
        \begin{itemize}
            \item $||A||_S \geq 0$, ya que cada sumando es $\geq 0$ debido al valor absoluto. Además,
            \begin{equation*}
                ||A||_S = \sum_{i=1}^n \sum_{j=1}^n |a_{ij}| = 0 \Longleftrightarrow |a_{ij}| = 0 \;\forall i,j \Longleftrightarrow A=0
            \end{equation*}

            \item $\displaystyle ||cA||_S = \sum_{i=1}^n \sum_{j=1}^n |ca_{ij}| = \sum_{i=1}^n \sum_{j=1}^n |c||a_{ij}| = |c|\sum_{i=1}^n \sum_{j=1}^n |a_{ij}| = |c| \cdot ||A||_S$

            \item $\displaystyle ||A+B||_S = \sum_{i=1}^n \sum_{j=1}^n |a_{ij}+b_{ij}| \leq \sum_{i=1}^n \sum_{j=1}^n |a_{ij}| + |b_{ij}| = \sum_{i=1}^n \sum_{j=1}^n |a_{ij}| + \sum_{i=1}^n \sum_{j=1}^n |b_{ij}| = ||A||_S + ||B||_S$

            \item Veamos que $||AB||_S \leq ||A||_S ||B||_S$
            \begin{multline*}
                 ||AB||_S = \sum_{i,j=1}^n |(ab)_{ij}|
                = \sum_{i,j=1}^n \left|\sum_{k=1}^n a_{ik}b_{kj}\right|
                \leq \sum_{i,j,k=1}^n |a_{ik}b_{kj}|
                = \sum_{i,j,k=1}^n |a_{ik}|\cdot~|b_{kj}|
                \leq \\ \leq 
                \sum_{i,j,k,s=1}^n |a_{ik}|\cdot~|b_{sj}|
                = \sum_{i,k=1}^n |a_{ik}| \sum_{s,j=1}^n|b_{sj}|
                \stackrel{Asociativa}{=}
                \left(\sum_{i,k=1}^n |a_{ik}|\right)\left( \sum_{s,j=1}^n|b_{sj}|\right)
                 =\\= ||A||_S ||B||_S
            \end{multline*}
        \end{itemize}
        
        \item Calcule $||I_n||_S$ donde $I_n$ es la matriz identidad de orden $n$.
        \begin{equation*}
            ||I_n||_S = \sum_{i=1}^n \sum_{j=1}^n |(I_n)_{ij}| =\sum_{i=1}^n 1 =  n
        \end{equation*}
        
        \item ¿Es una $||\cdot||_S$ una norma inducida? Justique la respuesta.\\
        
        Supongamos que lo fuese. Por tanto, $\exists ||\cdot||$ norma vectorial tal que:
        \begin{equation*}
            ||A||_S = \max_{x\neq 0} \frac{||Ax||}{||x||}
        \end{equation*}

        Por tanto, suponiendo que $||\cdot||_S$ es una norma inducida:
        \begin{equation*}
            ||I_n||_S = \max_{x\neq 0} \frac{||I_nx||}{||x||} = \max_{x\neq 0} \frac{||x||}{||x||} = \max_{x\neq 0} 1 = 1
        \end{equation*}

        Por tanto, como hemos llegado a un absurdo, ya que $n\neq 1$, entonces la suposición es falsa. No es una norma inducida.

        \begin{observacion}
        Se ha demostrado que, para toda norma inducida $||\cdot||_M$, la norma de la identidad es $||I_n||_M=1$.
        \end{observacion}
        
        \item Estime el número de condición asociado a la norma (\ref{Norma}) de la matriz
        \begin{equation*}
            B=\left(\begin{array}{cc}
                5 & 2 \\
                2 & 1
            \end{array} \right)
        \end{equation*}

        \begin{equation*}
            \kappa(B) = ||B||_S ||B^{-1}||_S
        \end{equation*}

        En primer, lugar, calculo $B^{-1}$:
        \begin{equation*}
            B^{-1}=\left(\begin{array}{cc}
                1 & -2 \\
                -2 & 5
            \end{array} \right)
        \end{equation*}

        Calculo cada norma:
        \begin{equation*}
            ||B||_S = 5+2+2+1 = 10 \qquad ||B^{-1}||_S = 1+2+2+5 = 10
        \end{equation*}

        Por tanto,
        \begin{equation*}
            \kappa(B) = ||B||_S ||B^{-1}||_S = 10^2
        \end{equation*}

        Por tanto, podemos esperar la pérdida de 2 dígitos significativos en el cálculo de la solución de un sistema con $B$ como matriz de coeficientes.
    \end{enumerate}
\end{ejercicio}

\newpage
\textbf{Segunda Parte} [4 puntos]
\begin{ejercicio}
    Se consideran los datos $f(-1)=f(1)=0.345,\quad f(0)=f(2)=1.67.$
    \begin{enumerate}
        \item Estime el valor de $f(0,5)$ utilizando el algoritmo de Newton-Horner en aritmética finita de tres dígitos por redondeo.

        Calculo en primer lugar la tabla de diferencias divididas:
            \begin{equation*}
                \begin{array}{c|cccc}
                    x_i & f[x_i] \\
                    \\
                    -1 & \mathbf{0.345} \\
                    && \mathbf{1.33}\\
                    0 & 1.67 & & \mathbf{-1.33}\\
                    && -1.33&&\mathbf{0.887}\\
                    1 & 0.345 && 1.33\\
                    & & 1.33\\
                    2 &1.67
                \end{array}
            \end{equation*}
            Por tanto, el polinomio de interpolación es
            \begin{equation*}
                \begin{split}
                    p_3(x) &= 0.345 +1.33(x+1) -1.33x(x+1) +0.887x(x+1)(x-1) \\
                     &= 0.345 +(x+1)[1.33 +x[-1.33 +0.887(x-1)]]
                \end{split}
            \end{equation*}

            Para reducir el error por redondeo, evaluamos usando el algoritmo de Newton-Horner:
            \begin{equation*}\begin{split}
                 p_3(x) & = 0.345 +1.5[1.33+0.5[-1.33-0.5\cdot 0.887]] \\
                & = 0.345 +1.5[1.33+0.5[-1.33-0.444]]\\
                & = 0.345 +1.5[1.33-0.885] =  0.345 +1.5[0.445]\\& = 0.345 + 0.668 = 1.01
            \end{split} \end{equation*}

        \item Estime el error cometido en el apartado anterior, sabiendo que $|f^{(k)}|<0.3$, para todo $x$, y para cualquier orden de derivación $k$.

        El error cometido viene dado por:
        \begin{equation*}
            |e(x)| = \frac{|f^{4)}(\xi)|}{(4!)}|x(x+1)(x-1)(x-2)|
        \end{equation*}

        Por tanto, evaluando en $x=0.5$, tenemos:
        \begin{equation*}
            |e(x)| = \frac{|f^{4)}(\xi)|}{24}\cdot \frac{9}{16}
        \end{equation*}

        Como sabemos que $|f^{(k)}|<0.3$, para todo $x$, y para cualquier orden de derivación $k$, tenemos, por tanto, que:
        \begin{equation*}
            |e(x)| < \frac{0.3\cdot 9}{24\cdot 16} = \frac{9}{1280} = 7.031\cdot 10^{-3}
        \end{equation*}
        
    \end{enumerate}
\end{ejercicio}

\begin{ejercicio}
    Se considera la tabla de datos
    \begin{equation*}
        \begin{array}{c|c|c|c|c|c}
            x_i & -2 & -1 & 0 & 1 & 2 \\ \hline
            f(x_i) & -6.5 & -1.5 & -0.5 & 3.5 & 9.5
        \end{array}
    \end{equation*}
    \begin{enumerate}
        \item Calcule la aproximación por mínimos cuadrados de la función $f(x)$ en el espacio vectorial $\mathcal{U}=\mathcal{L}\{x,x^2\}$

        Sea la mejor aproximación $u(x)=ax + bx^2\in \mathcal{U}$, y consideramos el producto escalar discreto siguiente:
        \begin{equation*}
            \langle f,g\rangle = \sum_{i=1}^5 f(x_i)g(x_i)
        \end{equation*}

        Calculamos los siguientes productos escalares:
        \begin{equation*}
            \langle x,x\rangle = 10 \qquad \langle x^2, x^2\rangle = 34 \qquad \langle x,x^2\rangle = 0
        \end{equation*}
        \begin{equation*}
            \langle f,x\rangle = 37 \hspace{2cm} \langle f, x^2\rangle = 14
        \end{equation*}

        Por trabajar con una base ortogonal, tenemos que:
        \begin{equation*}
            a_i = \frac{\langle e_i, f\rangle}{\langle e_i, e_i \rangle}
        \end{equation*}

        Por tanto,
        \begin{equation*}
            a_1 = \frac{37}{10} \hspace{2cm} a_2 = \frac{14}{34} = \frac{7}{17}
        \end{equation*}

        Por tanto, la mejor aproximación de $f$ en $\mathcal{U}$ es:
        \begin{equation*}
            u(x) = \frac{37}{10}x + \frac{7}{17}x^2
        \end{equation*}
        

        \item Calcule las diferencias divididas de orden 1 (con dos argumentos) para los datos de la tabla y llámelas $P_1$, $P_2$, $P_3$ y $P_4$.
        \begin{gather*}
            P_1 = f[-2, -1] = \frac{-1.5+6.5}{-1+2} = 5
            \qquad
            P_2 = f[-1, 0] = 1
            \\
            P_3 = f[0,1] = 4
            \qquad
            P_4 = f[1,2] = 6
        \end{gather*}

        \item Calcule el spline cúbico de clase 1 en los nodos $-2$, $0$ y $2$, $s(x)\in S^1_3 (-2, 0, 2)$, tomando como derivadas en los nodos:
        \begin{equation*}
            d_0 = P_1,\qquad d_1=\frac{P_2+P_3}{2}, \qquad d_2=P_4.
        \end{equation*}

        Interpolamos mediante Hermite en cada intervalo:
        \begin{equation*}
            \begin{array}{c|cccc}
                &&\mathbf{[-2, 0]} \\ \\
                x_i & f(x_i) \\ \\
                -2 & \mathbf{-6.5} \\
                && \mathbf{d_0=5}\\
                -2 & -6.5 && \mathbf{-1}\\
                && 3 && \mathbf{\frac{3}{8}}\\ 
                0 & -0.5 && -\frac{1}{4}\\
                && d_1=\frac{5}{2}\\
                0 & -0.5
            \end{array}
            \quad \left\|\quad
            \begin{array}{c|cccc}
                &&\mathbf{[0,2]} \\ \\
                x_i & f(x_i) \\ \\
                0 & \mathbf{-0.5} \\
                && \mathbf{d_1=\frac{5}{2}}\\
                0 & -0.5 && \mathbf{\frac{5}{4}}\\
                && 5 && \mathbf{-\frac{3}{8}}\\ 
                2 & 9.5 && \frac{1}{2}\\
                && d_2=6\\
                2 & 9.5
            \end{array}\right.
        \end{equation*}
    
        Por tanto, el spline queda:
        \begin{equation*}
            s(x)=\left\{\begin{array}{lll}
                -6.5 +5(x+2) -(x+2)^2 +\frac{3}{8}x(x+2)^2 & \text{si} & x\in [-2, 0] \\
                -0.5+\frac{5}{2}x + \frac{5}{4}x^2 -\frac{3}{8}x^2(x-2) & \text{si} & x\in [0,2] \\
            \end{array} \right.
        \end{equation*}

        \item Compare los valores que proporcionan el spline y la aproximación por mínimos cuadrados en los nodos $-1$ y $1$. ¿Qué modelo elegiría?

        En $x=-1$, tenemos:
        \begin{equation*}
            s(-1)= -2.875 \hspace{2cm} u(-1)\approx -3.288
        \end{equation*}

        En $x=1$, tenemos:
        \begin{equation*}
            s(1)=3.625 \hspace{2cm} u(1)\approx 4.112
        \end{equation*}

        Por tanto, en ambos casos tenemos que el spline se aproxima más a los valores correctos de $f$.
    \end{enumerate}
\end{ejercicio}


    
\end{document}