\documentclass[12pt]{article}

% Idioma y codificación
\usepackage[spanish, es-tabla]{babel}       %es-tabla para que se titule "Tabla"
\usepackage[utf8]{inputenc}

% Márgenes
\usepackage[a4paper,top=3cm,bottom=2.5cm,left=3cm,right=3cm]{geometry}

% Comentarios de bloque
\usepackage{verbatim}

% Paquetes de links
\usepackage[hidelinks]{hyperref}    % Permite enlaces
\usepackage{url}                    % redirecciona a la web

% Más opciones para enumeraciones
\usepackage{enumitem}

% Personalizar la portada
\usepackage{titling}

% Paquetes de tablas
\usepackage{multirow}


%------------------------------------------------------------------------

%Paquetes de figuras
\usepackage{caption}
\usepackage{subcaption} % Figuras al lado de otras
\usepackage{float}      % Poner figuras en el sitio indicado H.


% Paquetes de imágenes
\usepackage{graphicx}       % Paquete para añadir imágenes
\usepackage{transparent}    % Para manejar la opacidad de las figuras

% Paquete para usar colores
\usepackage[dvipsnames]{xcolor}
\usepackage{pagecolor}      % Para cambiar el color de la página

% Habilita tamaños de fuente mayores
\usepackage{fix-cm}

% Para los gráficos
\usepackage{tikz}

% Para poder situar los nodos en los grafos
\usetikzlibrary{positioning}


%------------------------------------------------------------------------

% Paquetes de matemáticas
\usepackage{mathtools, amsfonts, amssymb, mathrsfs}
\usepackage[makeroom]{cancel}     % Simplificar tachando
\usepackage{polynom}    % Divisiones y Ruffini
\usepackage{units} % Para poner fracciones diagonales con \nicefrac

\usepackage{pgfplots}   %Representar funciones
\pgfplotsset{compat=1.18}  % Versión 1.18

\usepackage{tikz-cd}    % Para usar diagramas de composiciones
\usetikzlibrary{calc}   % Para usar cálculo de coordenadas en tikz

%Definición de teoremas, etc.
\usepackage{amsthm}
%\swapnumbers   % Intercambia la posición del texto y de la numeración

\theoremstyle{plain}

\makeatletter
\@ifclassloaded{article}{
  \newtheorem{teo}{Teorema}[section]
}{
  \newtheorem{teo}{Teorema}[chapter]  % Se resetea en cada chapter
}
\makeatother

\newtheorem{coro}{Corolario}[teo]           % Se resetea en cada teorema
\newtheorem{prop}[teo]{Proposición}         % Usa el mismo contador que teorema
\newtheorem{lema}[teo]{Lema}                % Usa el mismo contador que teorema

\theoremstyle{remark}
\newtheorem*{observacion}{Observación}

\theoremstyle{definition}

\makeatletter
\@ifclassloaded{article}{
  \newtheorem{definicion}{Definición} [section]     % Se resetea en cada chapter
}{
  \newtheorem{definicion}{Definición} [chapter]     % Se resetea en cada chapter
}
\makeatother

\newtheorem*{notacion}{Notación}
\newtheorem*{ejemplo}{Ejemplo}
\newtheorem*{ejercicio*}{Ejercicio}             % No numerado
\newtheorem{ejercicio}{Ejercicio} [section]     % Se resetea en cada section


% Modificar el formato de la numeración del teorema "ejercicio"
\renewcommand{\theejercicio}{%
  \ifnum\value{section}=0 % Si no se ha iniciado ninguna sección
    \arabic{ejercicio}% Solo mostrar el número de ejercicio
  \else
    \thesection.\arabic{ejercicio}% Mostrar número de sección y número de ejercicio
  \fi
}


% \renewcommand\qedsymbol{$\blacksquare$}         % Cambiar símbolo QED
%------------------------------------------------------------------------

% Paquetes para encabezados
\usepackage{fancyhdr}
\pagestyle{fancy}
\fancyhf{}

\newcommand{\helv}{ % Modificación tamaño de letra
\fontfamily{}\fontsize{12}{12}\selectfont}
\setlength{\headheight}{15pt} % Amplía el tamaño del índice


%\usepackage{lastpage}   % Referenciar última pag   \pageref{LastPage}
\fancyfoot[C]{\thepage}

%------------------------------------------------------------------------

% Conseguir que no ponga "Capítulo 1". Sino solo "1."
\makeatletter
\@ifclassloaded{book}{
  \renewcommand{\chaptermark}[1]{\markboth{\thechapter.\ #1}{}} % En el encabezado
    
  \renewcommand{\@makechapterhead}[1]{%
  \vspace*{50\p@}%
  {\parindent \z@ \raggedright \normalfont
    \ifnum \c@secnumdepth >\m@ne
      \huge\bfseries \thechapter.\hspace{1em}\ignorespaces
    \fi
    \interlinepenalty\@M
    \Huge \bfseries #1\par\nobreak
    \vskip 40\p@
  }}
}
\makeatother

%------------------------------------------------------------------------
% Paquetes de cógido
\usepackage{minted}
\renewcommand\listingscaption{Código fuente}

\usepackage{fancyvrb}
% Personaliza el tamaño de los números de línea
\renewcommand{\theFancyVerbLine}{\small\arabic{FancyVerbLine}}

% Estilo para C++
\newminted{cpp}{
    frame=lines,
    framesep=2mm,
    baselinestretch=1.2,
    linenos,
    escapeinside=||
}

% para minted
\definecolor{LightGray}{rgb}{0.95,0.95,0.92}
\setminted{
    linenos=true,
    stepnumber=5,
    numberfirstline=true,
    autogobble,
    breaklines=true,
    breakautoindent=true,
    breaksymbolleft=,
    breaksymbolright=,
    breaksymbolindentleft=0pt,
    breaksymbolindentright=0pt,
    breaksymbolsepleft=0pt,
    breaksymbolsepright=0pt,
    fontsize=\footnotesize,
    bgcolor=LightGray,
    numbersep=10pt
}


\usepackage{listings} % Para incluir código desde un archivo

\renewcommand\lstlistingname{Código Fuente}
\renewcommand\lstlistlistingname{Índice de Códigos Fuente}

% Definir colores
\definecolor{vscodepurple}{rgb}{0.5,0,0.5}
\definecolor{vscodeblue}{rgb}{0,0,0.8}
\definecolor{vscodegreen}{rgb}{0,0.5,0}
\definecolor{vscodegray}{rgb}{0.5,0.5,0.5}
\definecolor{vscodebackground}{rgb}{0.97,0.97,0.97}
\definecolor{vscodelightgray}{rgb}{0.9,0.9,0.9}

% Configuración para el estilo de C similar a VSCode
\lstdefinestyle{vscode_C}{
  backgroundcolor=\color{vscodebackground},
  commentstyle=\color{vscodegreen},
  keywordstyle=\color{vscodeblue},
  numberstyle=\tiny\color{vscodegray},
  stringstyle=\color{vscodepurple},
  basicstyle=\scriptsize\ttfamily,
  breakatwhitespace=false,
  breaklines=true,
  captionpos=b,
  keepspaces=true,
  numbers=left,
  numbersep=5pt,
  showspaces=false,
  showstringspaces=false,
  showtabs=false,
  tabsize=2,
  frame=tb,
  framerule=0pt,
  aboveskip=10pt,
  belowskip=10pt,
  xleftmargin=10pt,
  xrightmargin=10pt,
  framexleftmargin=10pt,
  framexrightmargin=10pt,
  framesep=0pt,
  rulecolor=\color{vscodelightgray},
  backgroundcolor=\color{vscodebackground},
}

%------------------------------------------------------------------------

% Comandos definidos
\newcommand{\bb}[1]{\mathbb{#1}}
\newcommand{\cc}[1]{\mathcal{#1}}

% I prefer the slanted \leq
\let\oldleq\leq % save them in case they're every wanted
\let\oldgeq\geq
\renewcommand{\leq}{\leqslant}
\renewcommand{\geq}{\geqslant}

% Si y solo si
\newcommand{\sii}{\iff}

% Letras griegas
\newcommand{\eps}{\epsilon}
\newcommand{\veps}{\varepsilon}
\newcommand{\lm}{\lambda}

\newcommand{\ol}{\overline}
\newcommand{\ul}{\underline}
\newcommand{\wt}{\widetilde}
\newcommand{\wh}{\widehat}

\let\oldvec\vec
\renewcommand{\vec}{\overrightarrow}

% Derivadas parciales
\newcommand{\del}[2]{\frac{\partial #1}{\partial #2}}
\newcommand{\Del}[3]{\frac{\partial^{#1} #2}{\partial #3^{#1}}}
\newcommand{\deld}[2]{\dfrac{\partial #1}{\partial #2}}
\newcommand{\Deld}[3]{\dfrac{\partial^{#1} #2}{\partial #3^{#1}}}


\newcommand{\AstIg}{\stackrel{(\ast)}{=}}
\newcommand{\Hop}{\stackrel{L'H\hat{o}pital}{=}}

\newcommand{\red}[1]{{\color{red}#1}} % Para integrales, destacar los cambios.

% Método de integración
\newcommand{\MetInt}[2]{
    \left[\begin{array}{c}
        #1 \\ #2
    \end{array}\right]
}

% Declarar aplicaciones
% 1. Nombre aplicación
% 2. Dominio
% 3. Codominio
% 4. Variable
% 5. Imagen de la variable
\newcommand{\Func}[5]{
    \begin{equation*}
        \begin{array}{rrll}
            #1:& #2 & \longrightarrow & #3\\
               & #4 & \longmapsto & #5
        \end{array}
    \end{equation*}
}

%------------------------------------------------------------------------



\begin{document}

    % 1. Foto de fondo
    % 2. Título
    % 3. Encabezado Izquierdo
    % 4. Color de fondo
    % 5. Coord x del titulo
    % 6. Coord y del titulo
    % 7. Fecha

    
    % 1. Foto de fondo
% 2. Título
% 3. Encabezado Izquierdo
% 4. Color de fondo
% 5. Coord x del titulo
% 6. Coord y del titulo
% 7. Fecha

\newcommand{\portada}[7]{

    \portadaBase{#1}{#2}{#3}{#4}{#5}{#6}{#7}
    \portadaBook{#1}{#2}{#3}{#4}{#5}{#6}{#7}
}

\newcommand{\portadaExamen}[7]{

    \portadaBase{#1}{#2}{#3}{#4}{#5}{#6}{#7}
    \portadaArticle{#1}{#2}{#3}{#4}{#5}{#6}{#7}
}




\newcommand{\portadaBase}[7]{

    % Tiene la portada principal y la licencia Creative Commons
    
    % 1. Foto de fondo
    % 2. Título
    % 3. Encabezado Izquierdo
    % 4. Color de fondo
    % 5. Coord x del titulo
    % 6. Coord y del titulo
    % 7. Fecha
    
    
    \thispagestyle{empty}               % Sin encabezado ni pie de página
    \newgeometry{margin=0cm}        % Márgenes nulos para la primera página
    
    
    % Encabezado
    \fancyhead[L]{\helv #3}
    \fancyhead[R]{\helv \nouppercase{\leftmark}}
    
    
    \pagecolor{#4}        % Color de fondo para la portada
    
    \begin{figure}[p]
        \centering
        \transparent{0.3}           % Opacidad del 30% para la imagen
        
        \includegraphics[width=\paperwidth, keepaspectratio]{assets/#1}
    
        \begin{tikzpicture}[remember picture, overlay]
            \node[anchor=north west, text=white, opacity=1, font=\fontsize{60}{90}\selectfont\bfseries\sffamily, align=left] at (#5, #6) {#2};
            
            \node[anchor=south east, text=white, opacity=1, font=\fontsize{12}{18}\selectfont\sffamily, align=right] at (9.7, 3) {\textbf{\href{https://losdeldgiim.github.io/}{Los Del DGIIM}}};
            
            \node[anchor=south east, text=white, opacity=1, font=\fontsize{12}{15}\selectfont\sffamily, align=right] at (9.7, 1.8) {Doble Grado en Ingeniería Informática y Matemáticas\\Universidad de Granada};
        \end{tikzpicture}
    \end{figure}
    
    
    \restoregeometry        % Restaurar márgenes normales para las páginas subsiguientes
    \pagecolor{white}       % Restaurar el color de página
    
    
    \newpage
    \thispagestyle{empty}               % Sin encabezado ni pie de página
    \begin{tikzpicture}[remember picture, overlay]
        \node[anchor=south west, inner sep=3cm] at (current page.south west) {
            \begin{minipage}{0.5\paperwidth}
                \href{https://creativecommons.org/licenses/by-nc-nd/4.0/}{
                    \includegraphics[height=2cm]{assets/Licencia.png}
                }\vspace{1cm}\\
                Esta obra está bajo una
                \href{https://creativecommons.org/licenses/by-nc-nd/4.0/}{
                    Licencia Creative Commons Atribución-NoComercial-SinDerivadas 4.0 Internacional (CC BY-NC-ND 4.0).
                }\\
    
                Eres libre de compartir y redistribuir el contenido de esta obra en cualquier medio o formato, siempre y cuando des el crédito adecuado a los autores originales y no persigas fines comerciales. 
            \end{minipage}
        };
    \end{tikzpicture}
    
    
    
    % 1. Foto de fondo
    % 2. Título
    % 3. Encabezado Izquierdo
    % 4. Color de fondo
    % 5. Coord x del titulo
    % 6. Coord y del titulo
    % 7. Fecha


}


\newcommand{\portadaBook}[7]{

    % 1. Foto de fondo
    % 2. Título
    % 3. Encabezado Izquierdo
    % 4. Color de fondo
    % 5. Coord x del titulo
    % 6. Coord y del titulo
    % 7. Fecha

    % Personaliza el formato del título
    \pretitle{\begin{center}\bfseries\fontsize{42}{56}\selectfont}
    \posttitle{\par\end{center}\vspace{2em}}
    
    % Personaliza el formato del autor
    \preauthor{\begin{center}\Large}
    \postauthor{\par\end{center}\vfill}
    
    % Personaliza el formato de la fecha
    \predate{\begin{center}\huge}
    \postdate{\par\end{center}\vspace{2em}}
    
    \title{#2}
    \author{\href{https://losdeldgiim.github.io/}{Los Del DGIIM}}
    \date{Granada, #7}
    \maketitle
    
    \tableofcontents
}




\newcommand{\portadaArticle}[7]{

    % 1. Foto de fondo
    % 2. Título
    % 3. Encabezado Izquierdo
    % 4. Color de fondo
    % 5. Coord x del titulo
    % 6. Coord y del titulo
    % 7. Fecha

    % Personaliza el formato del título
    \pretitle{\begin{center}\bfseries\fontsize{42}{56}\selectfont}
    \posttitle{\par\end{center}\vspace{2em}}
    
    % Personaliza el formato del autor
    \preauthor{\begin{center}\Large}
    \postauthor{\par\end{center}\vspace{3em}}
    
    % Personaliza el formato de la fecha
    \predate{\begin{center}\huge}
    \postdate{\par\end{center}\vspace{5em}}
    
    \title{#2}
    \author{\href{https://losdeldgiim.github.io/}{Los Del DGIIM}}
    \date{Granada, #7}
    \thispagestyle{empty}               % Sin encabezado ni pie de página
    \maketitle
    \vfill
}
    \portadaExamen{ffccA4.jpg}{MN I\\Examen III}{MN I. Examen III}{MidnightBlue}{-8}{28}{2023}

    \begin{description}
        \item[Asignatura] Métodos Numéricos I.
        \item[Curso Académico] 2021-22.
        \item[Grado] Matemáticas.
        \item[Grupo] B.
        \item[Profesor] Teresa Encarnación Pérez Fernández.
        \item[Descripción] Prueba 2. Temas 3 y 4.
        \item[Fecha] 7 de junio de 2022.
        %\item[Duración] 60 minutos.
    
    \end{description}
    \newpage
    
    \begin{ejercicio} [\textbf{1.5 puntos}]
    Determine $s'(0)$ y $s'(2)$ para que la siguiente expresión defina un spline cúbico de extremo sujeto:
    \begin{equation*}
        s(x)=\left\{\begin{array}{cc}
             2x^3 + 2x^2 +\alpha x +1, & 0\leq x \leq 1,  \\
             x^3+\beta x^2 + 2x + \gamma, & 1\leq x \leq 2,
        \end{array}\right.
    \end{equation*}

    En primer lugar, calculamos la derivada. Por el carácter local de la derivabilidad:
    \begin{equation*}
        s'(x)=\left\{\begin{array}{cc}
             6x^2 + 4x + \alpha , & 0\leq x \leq 1,  \\
             3x^2+2\beta x + 2, & 1\leq x \leq 2,
        \end{array}\right.
    \end{equation*}
    \begin{equation*}
        s''(x)=\left\{\begin{array}{cc}
             12x +4, & 0\leq x \leq 1,  \\
             6x+2\beta, & 1\leq x \leq 2,
        \end{array}\right.
    \end{equation*}

    Para que sea un spline cúbico, hemos de tener en primer lugar que sea continua. Por tanto,
    \begin{equation*}
        5+\alpha = \lim_{x\to 1^-}s(x) = \lim_{x\to 1^+}s(x) = \beta + 3 + \gamma \Longrightarrow -\alpha + \beta + \gamma = 2
    \end{equation*}

    Además, como es de clase 1, ha de ser derivable con derivada continua. Por tanto,
    \begin{equation*}
        10+\alpha = \lim_{x\to 1^-}s'(x) = \lim_{x\to 1^+}s'(x) = 5 + 2\beta
        \Longrightarrow -\alpha + 2\beta = 5
    \end{equation*}

    Además, como es de clase 2, su segunda derivada ha de ser continua. Por tanto,
    \begin{equation*}
        16 = \lim_{x\to 1^-}s''(x) = \lim_{x\to 1^+}s''(x) = 6 + 2\beta
        \Longrightarrow \beta = 5
    \end{equation*}

    Por tanto, para que sea un spline ha de ser:
    \begin{equation*}
        \alpha = \beta = 5 \qquad \qquad \gamma = 2
    \end{equation*}

    Por tanto, para que sea un spline cúbico de extremo sujeto ha de ser:
    \begin{equation*}
        s'(0)=5  \hspace{2cm} s'(2) = 34
    \end{equation*}
\end{ejercicio}


\begin{ejercicio} [\textbf{1.5 puntos}]
    Usando aritmética de tres cifras por redondeo, calcule el polinomio de interpolación para los siguientes datos: $f(0.8) = 0.224, f'(0.8)~=~2.17,$\\$f(1.0)=0.658,$ $f'(1.0) = 2.04$.

    \begin{equation*}
        \begin{array}{c|cc}
            x_i & 0.8 & 1.0 \\ \hline
            f_i & 0.224 & 0.658 \\ \hline
            f'_i & 2.17 & 2.04
        \end{array}
    \end{equation*}
    Calculo la tabla de diferencias divididas:
    \begin{equation*}
        \begin{array}{c|cccccc}
            x_i & f[x_i] \\
            \\
            0.8 & \textbf{0.224} \\
            && \textbf{2.17}\\
            0.8 & 0.224 && \textbf{0}\\
            && 2.17&&\mathbf{-3.25}\\
            1 & 0.658 && -0.65 &&\\
            && 2.04\\
            1 & 0.658
        \end{array}
    \end{equation*}

    Por tanto, tengo que el polinomio de interpolación es:
    \begin{equation*}
        \begin{split}
            p_3(x)&= 0.224 + 2.17(x-0.8) -3.25(x-1)(x-0.8)^2
        \end{split}
    \end{equation*}
\end{ejercicio}


\begin{ejercicio}[\textbf{1.5 puntos}]
    Dada la función $f(x) = \frac{x^4-x}{4}$ definida en el intervalo $[0,a],\;a>0$, determine el número mínimo de nodos equiespaciados $n$ (abcisas) que se deben tomar en el intervalo $[0,a]$ para que el error de interpolación entre cada dos nodos consecutivos sea menor que un $\varepsilon>0$ dado.\\

    Tenemos que $f\in \mathbb{P}_4$, por lo que queda definido por $5$ puntos. Es decir, tomando $5$ puntos, el polinomio de interpolación pertenecerá a $\mathbb{P}_4$ y, de hecho, será $f$. Por tanto, el error de interpolación será nulo, ya que se habría interpolado el mismo polinomio.

    De hecho, al tomar $5$ puntos tenemos que el error de interpolación es:
    \begin{equation*}
        e(x) = \frac{f^{(5)}(\xi)}{5!}\prod_{i=0}^4 (x-x_i) \qquad \xi \in [a,b]
    \end{equation*}

    No obstante, tenemos que $f^{(5)}(x)=0 \;\forall x$, por lo que como podemos ver, el error es nulo.
\end{ejercicio}

\begin{ejercicio} [\textbf{1.5 puntos}]
    Sea la función
    \begin{equation*}
        f(x)=\left\{\begin{array}{cc}
            0 & x\leq c \\
            2 & x>c
        \end{array}\right.
    \end{equation*}

    Sabemos que la recta que mejor aproxima a $f(x)$ por mínimos cuadrados continuos en el intervalo $[0, 3]$ es $\displaystyle p(x) = \frac{8}{9}x$. Calcule $c$.\\

    Al ser la aproximación por mínimos cuadrados continua en el intervalo $[0,3]$, tomamos el producto escalar:
    \begin{equation*}
        \langle f,g\rangle = \int_0^3 f(x)g(x)\;dx
    \end{equation*}

    Buscamos la mejor aproximación en $\mathbb{P}_1=\mathcal{L}\{1,x\}$. Calculamos los productos escalares:
    \begin{gather*}
        \langle 1,1\rangle = 3
        \qquad
        \langle 1,x\rangle = \frac{9}{2}
        \qquad
        \langle x,x\rangle = 9
        \\
        \langle f,1\rangle = \int_c^3 2\;dx = 2(3-c)
        \qquad
        \langle f,x\rangle = \int_c^3 2x\;dx = 9-c^2 
    \end{gather*}

    Por tanto, la mejor aproximación es $p(x)=a_0+a_1x \in \bb{P}_1$ tal que:
    \begin{equation*}
        \left(\begin{array}{cc}
            3 & 9/2 \\
            9/2 & 9
        \end{array}\right)
        \left(\begin{array}{c}
            a_0 \\ a_1    
        \end{array}\right)
        = 
        \left(\begin{array}{c}
            2(3-c) \\ 9-c^3
        \end{array}\right)
    \end{equation*}

    Como tenemos que la mejor aproximación es $p(x)=\frac{8}{9}x$, tenemos que $a_0=0$, $a_1=\frac{8}{9}$. Por tanto:
    \begin{equation*}
        \left\{\begin{array}{c}
            \displaystyle \frac{9}{2}\cdot \frac{8}{9} = 6-2c \Longrightarrow 4 = 6-2c  \\
            \\
            \displaystyle 9\cdot \frac{8}{9} = 9-c^3 \Longrightarrow 8=9-c^3
        \end{array}\right.
    \end{equation*}

    Por tanto, de ambas ecuaciones deducimos que $c=1$.
\end{ejercicio}

\begin{ejercicio} [\textbf{4 puntos}]
    Se considera el producto escalar
    \begin{equation*}
        \langle f,g \rangle = \int_{-2}^{-1} f(x) g(x)\;dx + f(0) g(0) +\int_1^2f(x)g(x)\;dx.
    \end{equation*}

    \begin{enumerate}
        \item (\textbf{1.25 puntos}) Calcule los tres primeros polinomios ortogonales.

        Consideramos el espacio vectorial $\mathbb{P}_n=\mathcal{L}\{1,x,x^2,\dots, x^n\}$ y buscamos una base ortogonal $\mathcal{B}_o = \{e_1, e_2, e_3, \dots, e_n\}$. Partimos de $e_1=1$, y usando el algoritmo de Gram-Schmidt, tenemos que:
        \begin{equation*}
            e_2 = x-\frac{\langle x, e_1\rangle}{\langle e_1, e_1\rangle} \cdot e_1
        \end{equation*}
        \begin{equation*}
            \langle x,1\rangle = \int_{-2}^{-1} x\;dx + 0 +\int_1^2 x\;dx = 0
        \end{equation*}

        Por tanto, definimos $e_2=x$. Calculamos ahora $e_3$
        \begin{equation*}
            e_3 = x^2-\frac{\langle x^2, e_1\rangle}{\langle e_1, e_1\rangle} \cdot e_1 -\frac{\langle x^2, e_2\rangle}{\langle e_2, e_2\rangle} \cdot e_2
        \end{equation*}
        \begin{equation*}
            \langle x^2,1\rangle = \int_{-2}^{-1} x^2\;dx + 0 +\int_1^2 x^2\;dx = \frac{14}{3}
            \qquad
            \langle x^2,x\rangle = \int_{-2}^{-1} x^3\;dx + 0 +\int_1^2 x^3\;dx = 0
        \end{equation*}
        \begin{equation*}
            \langle 1,1\rangle = \int_{-2}^{-1} 1\;dx + 1 +\int_1^2 1\;dx = 3
        \end{equation*}

        Por tanto, $e_3=x^2-\frac{14}{9}$. Es decir, los tres primeros polinomios ortogonales son:
        \begin{equation*}
            \left\{1,x,x^2-\frac{14}{9}\right\}
        \end{equation*}

        
        \item (\textbf{1.25 punto}) Utilizando el apartado anterior, proporcione la parábola $u(x)$ mejor aproximación por mínimos cuadrados de la función $f(x) = x^3$.

        Sea $\mathbb{P}_2=\mathcal{L}\left\{1,x,x^2-\frac{14}{9}\right\}$, y consideramos $u(x)=a_1 + a_2x + a_3\left(x^2-\frac{14}{9}\right)$.

        Calculamos productos escalares necesarios, sabiendo que se trata de una base ortogonal:
        \begin{equation*}
            \langle 1,1\rangle = 3 \qquad
            \langle x,x\rangle = \frac{14}{3}
        \end{equation*}
        \begin{equation*}
            \langle f,1\rangle = 0 \qquad
            \langle f,x\rangle = \frac{62}{5} \qquad
            \left\langle f, x^2-\frac{14}{9}\right\rangle = 0
        \end{equation*}

        Por trabajar con una base ortogonal, tenemos que:
        \begin{equation*}
            a_i = \frac{\langle e_i, f\rangle}{\langle e_i, e_i \rangle}
        \end{equation*}

        Por tanto, $a_1=a_3 = 0$. Además,
        \begin{equation*}
            a_2 = \frac{\langle x, f\rangle}{\langle x,x \rangle} = \frac{\frac{62}{5}}{\frac{14}{3}} = \frac{93}{35}
        \end{equation*}
        
        Es decir, la mejor aproximación en $\bb{P}_2$ es $$u(x)=\frac{93}{35}x$$
        
        \item (\textbf{1 punto})  Compruebe que $f(x) - u(x)$ es ortogonal a todos los polinomios de grado menor o igual a dos.

        Por ser $u$ la mejor aproximación de $f$, tenemos que:
        \begin{equation*}
            ||f-u||\leq ||f-p_2|| \Longrightarrow ||f-u||^2\leq ||f-p_2||^2 \qquad \forall p_2\in \bb{P}_2
        \end{equation*}

        Tomamos $g\in \bb{P}_2$, y sea $p_2=u+\lambda g \in \bb{P}_2 \mid \lambda \in \bb{R}$. Por tanto, como $p_2 \in \bb{P}_2$, tenemos que:
        \begin{equation*}
            ||f-u||^2\leq ||f-u-\lambda g||^2 = \langle f-u-\lambda g, f-u-\lambda g\rangle = ||f-u||^2 -2\lambda \langle f-u,g\rangle +\lambda^2 ||g||^2
        \end{equation*}

        Por tanto,
        \begin{equation*}
            0 \leq -2\lambda \langle f-u,g\rangle +\lambda^2 ||g||^2
            \qquad \forall \lambda\in \bb{R}, \forall g\in \bb{P}_2.
        \end{equation*}

        Considerando la expresión anterior como una parábola en la incógnita $\lambda$, tenemos:
        \begin{equation*}
            \Delta = 4(\langle f-u,g\rangle)^2 \geq 0
        \end{equation*}

        Por tanto, para que la parábola siempre sea positiva, no puede tener dos raíces. Por tanto, $\Delta=0$, lo que implica que:
        \begin{equation*}
            \langle f-u,g\rangle = 0 \qquad \forall g\in \bb{P}_2
        \end{equation*}

        Por tanto, hemos demostrado que $f(x) - u(x)$ es ortogonal a todos los polinomios de grado menor o igual a dos.
        
        \item (\textbf{0.5 puntos}) Interprete geométricamente la distancia que induce este producto escalar.
        
        Tenemos que:
        \begin{equation*}
            d(u,v) := ||u-v||
            = \sqrt{\int_{-2}^{-1}(u-v)^2(x)\;dx + (u-v)^2(0) + \int_{1}^{2}(u-v)^2(x)\;dx}
        \end{equation*}

        Por tanto, la distancia es la raíz del área encerrada por ambas funciones entre $[-2, -1]$ y $[1,2]$ y el cuadrado de las imágenes de las dos funciones en $x=0$.

        Por tanto, al minimizar mediante mínimos cuadrados, lo que buscamos es minimizar el área entre dichas funciones en los intervalos mencionados y minimizar la distancia en el punto $x=0$.
        

        
    \end{enumerate}
\end{ejercicio}


    
\end{document}