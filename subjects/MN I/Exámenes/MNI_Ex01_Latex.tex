\documentclass[12pt]{article}

% Idioma y codificación
\usepackage[spanish, es-tabla]{babel}       %es-tabla para que se titule "Tabla"
\usepackage[utf8]{inputenc}

% Márgenes
\usepackage[a4paper,top=3cm,bottom=2.5cm,left=3cm,right=3cm]{geometry}

% Comentarios de bloque
\usepackage{verbatim}

% Paquetes de links
\usepackage[hidelinks]{hyperref}    % Permite enlaces
\usepackage{url}                    % redirecciona a la web

% Más opciones para enumeraciones
\usepackage{enumitem}

% Personalizar la portada
\usepackage{titling}

% Paquetes de tablas
\usepackage{multirow}


%------------------------------------------------------------------------

%Paquetes de figuras
\usepackage{caption}
\usepackage{subcaption} % Figuras al lado de otras
\usepackage{float}      % Poner figuras en el sitio indicado H.


% Paquetes de imágenes
\usepackage{graphicx}       % Paquete para añadir imágenes
\usepackage{transparent}    % Para manejar la opacidad de las figuras

% Paquete para usar colores
\usepackage[dvipsnames]{xcolor}
\usepackage{pagecolor}      % Para cambiar el color de la página

% Habilita tamaños de fuente mayores
\usepackage{fix-cm}

% Para los gráficos
\usepackage{tikz}

% Para poder situar los nodos en los grafos
\usetikzlibrary{positioning}


%------------------------------------------------------------------------

% Paquetes de matemáticas
\usepackage{mathtools, amsfonts, amssymb, mathrsfs}
\usepackage[makeroom]{cancel}     % Simplificar tachando
\usepackage{polynom}    % Divisiones y Ruffini
\usepackage{units} % Para poner fracciones diagonales con \nicefrac

\usepackage{pgfplots}   %Representar funciones
\pgfplotsset{compat=1.18}  % Versión 1.18

\usepackage{tikz-cd}    % Para usar diagramas de composiciones
\usetikzlibrary{calc}   % Para usar cálculo de coordenadas en tikz

%Definición de teoremas, etc.
\usepackage{amsthm}
%\swapnumbers   % Intercambia la posición del texto y de la numeración

\theoremstyle{plain}

\makeatletter
\@ifclassloaded{article}{
  \newtheorem{teo}{Teorema}[section]
}{
  \newtheorem{teo}{Teorema}[chapter]  % Se resetea en cada chapter
}
\makeatother

\newtheorem{coro}{Corolario}[teo]           % Se resetea en cada teorema
\newtheorem{prop}[teo]{Proposición}         % Usa el mismo contador que teorema
\newtheorem{lema}[teo]{Lema}                % Usa el mismo contador que teorema

\theoremstyle{remark}
\newtheorem*{observacion}{Observación}

\theoremstyle{definition}

\makeatletter
\@ifclassloaded{article}{
  \newtheorem{definicion}{Definición} [section]     % Se resetea en cada chapter
}{
  \newtheorem{definicion}{Definición} [chapter]     % Se resetea en cada chapter
}
\makeatother

\newtheorem*{notacion}{Notación}
\newtheorem*{ejemplo}{Ejemplo}
\newtheorem*{ejercicio*}{Ejercicio}             % No numerado
\newtheorem{ejercicio}{Ejercicio} [section]     % Se resetea en cada section


% Modificar el formato de la numeración del teorema "ejercicio"
\renewcommand{\theejercicio}{%
  \ifnum\value{section}=0 % Si no se ha iniciado ninguna sección
    \arabic{ejercicio}% Solo mostrar el número de ejercicio
  \else
    \thesection.\arabic{ejercicio}% Mostrar número de sección y número de ejercicio
  \fi
}


% \renewcommand\qedsymbol{$\blacksquare$}         % Cambiar símbolo QED
%------------------------------------------------------------------------

% Paquetes para encabezados
\usepackage{fancyhdr}
\pagestyle{fancy}
\fancyhf{}

\newcommand{\helv}{ % Modificación tamaño de letra
\fontfamily{}\fontsize{12}{12}\selectfont}
\setlength{\headheight}{15pt} % Amplía el tamaño del índice


%\usepackage{lastpage}   % Referenciar última pag   \pageref{LastPage}
\fancyfoot[C]{\thepage}

%------------------------------------------------------------------------

% Conseguir que no ponga "Capítulo 1". Sino solo "1."
\makeatletter
\@ifclassloaded{book}{
  \renewcommand{\chaptermark}[1]{\markboth{\thechapter.\ #1}{}} % En el encabezado
    
  \renewcommand{\@makechapterhead}[1]{%
  \vspace*{50\p@}%
  {\parindent \z@ \raggedright \normalfont
    \ifnum \c@secnumdepth >\m@ne
      \huge\bfseries \thechapter.\hspace{1em}\ignorespaces
    \fi
    \interlinepenalty\@M
    \Huge \bfseries #1\par\nobreak
    \vskip 40\p@
  }}
}
\makeatother

%------------------------------------------------------------------------
% Paquetes de cógido
\usepackage{minted}
\renewcommand\listingscaption{Código fuente}

\usepackage{fancyvrb}
% Personaliza el tamaño de los números de línea
\renewcommand{\theFancyVerbLine}{\small\arabic{FancyVerbLine}}

% Estilo para C++
\newminted{cpp}{
    frame=lines,
    framesep=2mm,
    baselinestretch=1.2,
    linenos,
    escapeinside=||
}

% para minted
\definecolor{LightGray}{rgb}{0.95,0.95,0.92}
\setminted{
    linenos=true,
    stepnumber=5,
    numberfirstline=true,
    autogobble,
    breaklines=true,
    breakautoindent=true,
    breaksymbolleft=,
    breaksymbolright=,
    breaksymbolindentleft=0pt,
    breaksymbolindentright=0pt,
    breaksymbolsepleft=0pt,
    breaksymbolsepright=0pt,
    fontsize=\footnotesize,
    bgcolor=LightGray,
    numbersep=10pt
}


\usepackage{listings} % Para incluir código desde un archivo

\renewcommand\lstlistingname{Código Fuente}
\renewcommand\lstlistlistingname{Índice de Códigos Fuente}

% Definir colores
\definecolor{vscodepurple}{rgb}{0.5,0,0.5}
\definecolor{vscodeblue}{rgb}{0,0,0.8}
\definecolor{vscodegreen}{rgb}{0,0.5,0}
\definecolor{vscodegray}{rgb}{0.5,0.5,0.5}
\definecolor{vscodebackground}{rgb}{0.97,0.97,0.97}
\definecolor{vscodelightgray}{rgb}{0.9,0.9,0.9}

% Configuración para el estilo de C similar a VSCode
\lstdefinestyle{vscode_C}{
  backgroundcolor=\color{vscodebackground},
  commentstyle=\color{vscodegreen},
  keywordstyle=\color{vscodeblue},
  numberstyle=\tiny\color{vscodegray},
  stringstyle=\color{vscodepurple},
  basicstyle=\scriptsize\ttfamily,
  breakatwhitespace=false,
  breaklines=true,
  captionpos=b,
  keepspaces=true,
  numbers=left,
  numbersep=5pt,
  showspaces=false,
  showstringspaces=false,
  showtabs=false,
  tabsize=2,
  frame=tb,
  framerule=0pt,
  aboveskip=10pt,
  belowskip=10pt,
  xleftmargin=10pt,
  xrightmargin=10pt,
  framexleftmargin=10pt,
  framexrightmargin=10pt,
  framesep=0pt,
  rulecolor=\color{vscodelightgray},
  backgroundcolor=\color{vscodebackground},
}

%------------------------------------------------------------------------

% Comandos definidos
\newcommand{\bb}[1]{\mathbb{#1}}
\newcommand{\cc}[1]{\mathcal{#1}}

% I prefer the slanted \leq
\let\oldleq\leq % save them in case they're every wanted
\let\oldgeq\geq
\renewcommand{\leq}{\leqslant}
\renewcommand{\geq}{\geqslant}

% Si y solo si
\newcommand{\sii}{\iff}

% Letras griegas
\newcommand{\eps}{\epsilon}
\newcommand{\veps}{\varepsilon}
\newcommand{\lm}{\lambda}

\newcommand{\ol}{\overline}
\newcommand{\ul}{\underline}
\newcommand{\wt}{\widetilde}
\newcommand{\wh}{\widehat}

\let\oldvec\vec
\renewcommand{\vec}{\overrightarrow}

% Derivadas parciales
\newcommand{\del}[2]{\frac{\partial #1}{\partial #2}}
\newcommand{\Del}[3]{\frac{\partial^{#1} #2}{\partial #3^{#1}}}
\newcommand{\deld}[2]{\dfrac{\partial #1}{\partial #2}}
\newcommand{\Deld}[3]{\dfrac{\partial^{#1} #2}{\partial #3^{#1}}}


\newcommand{\AstIg}{\stackrel{(\ast)}{=}}
\newcommand{\Hop}{\stackrel{L'H\hat{o}pital}{=}}

\newcommand{\red}[1]{{\color{red}#1}} % Para integrales, destacar los cambios.

% Método de integración
\newcommand{\MetInt}[2]{
    \left[\begin{array}{c}
        #1 \\ #2
    \end{array}\right]
}

% Declarar aplicaciones
% 1. Nombre aplicación
% 2. Dominio
% 3. Codominio
% 4. Variable
% 5. Imagen de la variable
\newcommand{\Func}[5]{
    \begin{equation*}
        \begin{array}{rrll}
            #1:& #2 & \longrightarrow & #3\\
               & #4 & \longmapsto & #5
        \end{array}
    \end{equation*}
}

%------------------------------------------------------------------------



\begin{document}

    % 1. Foto de fondo
    % 2. Título
    % 3. Encabezado Izquierdo
    % 4. Color de fondo
    % 5. Coord x del titulo
    % 6. Coord y del titulo
    % 7. Fecha

    
    % 1. Foto de fondo
% 2. Título
% 3. Encabezado Izquierdo
% 4. Color de fondo
% 5. Coord x del titulo
% 6. Coord y del titulo
% 7. Fecha

\newcommand{\portada}[7]{

    \portadaBase{#1}{#2}{#3}{#4}{#5}{#6}{#7}
    \portadaBook{#1}{#2}{#3}{#4}{#5}{#6}{#7}
}

\newcommand{\portadaExamen}[7]{

    \portadaBase{#1}{#2}{#3}{#4}{#5}{#6}{#7}
    \portadaArticle{#1}{#2}{#3}{#4}{#5}{#6}{#7}
}




\newcommand{\portadaBase}[7]{

    % Tiene la portada principal y la licencia Creative Commons
    
    % 1. Foto de fondo
    % 2. Título
    % 3. Encabezado Izquierdo
    % 4. Color de fondo
    % 5. Coord x del titulo
    % 6. Coord y del titulo
    % 7. Fecha
    
    
    \thispagestyle{empty}               % Sin encabezado ni pie de página
    \newgeometry{margin=0cm}        % Márgenes nulos para la primera página
    
    
    % Encabezado
    \fancyhead[L]{\helv #3}
    \fancyhead[R]{\helv \nouppercase{\leftmark}}
    
    
    \pagecolor{#4}        % Color de fondo para la portada
    
    \begin{figure}[p]
        \centering
        \transparent{0.3}           % Opacidad del 30% para la imagen
        
        \includegraphics[width=\paperwidth, keepaspectratio]{assets/#1}
    
        \begin{tikzpicture}[remember picture, overlay]
            \node[anchor=north west, text=white, opacity=1, font=\fontsize{60}{90}\selectfont\bfseries\sffamily, align=left] at (#5, #6) {#2};
            
            \node[anchor=south east, text=white, opacity=1, font=\fontsize{12}{18}\selectfont\sffamily, align=right] at (9.7, 3) {\textbf{\href{https://losdeldgiim.github.io/}{Los Del DGIIM}}};
            
            \node[anchor=south east, text=white, opacity=1, font=\fontsize{12}{15}\selectfont\sffamily, align=right] at (9.7, 1.8) {Doble Grado en Ingeniería Informática y Matemáticas\\Universidad de Granada};
        \end{tikzpicture}
    \end{figure}
    
    
    \restoregeometry        % Restaurar márgenes normales para las páginas subsiguientes
    \pagecolor{white}       % Restaurar el color de página
    
    
    \newpage
    \thispagestyle{empty}               % Sin encabezado ni pie de página
    \begin{tikzpicture}[remember picture, overlay]
        \node[anchor=south west, inner sep=3cm] at (current page.south west) {
            \begin{minipage}{0.5\paperwidth}
                \href{https://creativecommons.org/licenses/by-nc-nd/4.0/}{
                    \includegraphics[height=2cm]{assets/Licencia.png}
                }\vspace{1cm}\\
                Esta obra está bajo una
                \href{https://creativecommons.org/licenses/by-nc-nd/4.0/}{
                    Licencia Creative Commons Atribución-NoComercial-SinDerivadas 4.0 Internacional (CC BY-NC-ND 4.0).
                }\\
    
                Eres libre de compartir y redistribuir el contenido de esta obra en cualquier medio o formato, siempre y cuando des el crédito adecuado a los autores originales y no persigas fines comerciales. 
            \end{minipage}
        };
    \end{tikzpicture}
    
    
    
    % 1. Foto de fondo
    % 2. Título
    % 3. Encabezado Izquierdo
    % 4. Color de fondo
    % 5. Coord x del titulo
    % 6. Coord y del titulo
    % 7. Fecha


}


\newcommand{\portadaBook}[7]{

    % 1. Foto de fondo
    % 2. Título
    % 3. Encabezado Izquierdo
    % 4. Color de fondo
    % 5. Coord x del titulo
    % 6. Coord y del titulo
    % 7. Fecha

    % Personaliza el formato del título
    \pretitle{\begin{center}\bfseries\fontsize{42}{56}\selectfont}
    \posttitle{\par\end{center}\vspace{2em}}
    
    % Personaliza el formato del autor
    \preauthor{\begin{center}\Large}
    \postauthor{\par\end{center}\vfill}
    
    % Personaliza el formato de la fecha
    \predate{\begin{center}\huge}
    \postdate{\par\end{center}\vspace{2em}}
    
    \title{#2}
    \author{\href{https://losdeldgiim.github.io/}{Los Del DGIIM}}
    \date{Granada, #7}
    \maketitle
    
    \tableofcontents
}




\newcommand{\portadaArticle}[7]{

    % 1. Foto de fondo
    % 2. Título
    % 3. Encabezado Izquierdo
    % 4. Color de fondo
    % 5. Coord x del titulo
    % 6. Coord y del titulo
    % 7. Fecha

    % Personaliza el formato del título
    \pretitle{\begin{center}\bfseries\fontsize{42}{56}\selectfont}
    \posttitle{\par\end{center}\vspace{2em}}
    
    % Personaliza el formato del autor
    \preauthor{\begin{center}\Large}
    \postauthor{\par\end{center}\vspace{3em}}
    
    % Personaliza el formato de la fecha
    \predate{\begin{center}\huge}
    \postdate{\par\end{center}\vspace{5em}}
    
    \title{#2}
    \author{\href{https://losdeldgiim.github.io/}{Los Del DGIIM}}
    \date{Granada, #7}
    \thispagestyle{empty}               % Sin encabezado ni pie de página
    \maketitle
    \vfill
}
    \portadaExamen{ffccA4.jpg}{MN I\\Examen I}{MN I. Examen I}{MidnightBlue}{-8}{28}{2023}{Arturo Olivares Martos}

    \begin{description}
        \item[Asignatura] Métodos Numéricos I.
        \item[Curso Académico] 2021-22.
        \item[Grado] Doble Grado en Ingeniería Informática y Matemáticas.
        \item[Grupo] Único.
        \item[Profesor] Lidia Fernández Rodríguez.
        \item[Descripción] Prueba 1. Temas 1 y 2.
        %\item[Fecha] 10 de noviembre de 2023.
        %\item[Duración] 60 minutos.
    
    \end{description}
    \newpage
    
    \begin{ejercicio} [\textbf{1.5 puntos}]
    Dado el sistema de ecuaciones lineales
    \begin{equation*}
        \left\{ \begin{array}{rrrrrrr}
            2x_1 & + & x_2 & + & 3x_3 &=& 1  \\
            6x_1 & + & \alpha x_2 & + & 10x_3 &=&  5 \\
            4x_1 & + & 6x_2 & + & 8x_3 &=&  5
        \end{array}\right.
    \end{equation*}
    ¿Qué debe cumplir el parámetro $\alpha$ para que se pueda resolver el sistema usando el método de Gauss sin intercambio de filas?\\

    Como este método es equivalente a la descomposición LU, basta con que todos sus menores principales sean no nulos.
    \begin{equation*}
        |2|=2\neq 0 \qquad \left|\begin{array}{cc}
            2 & 1 \\
            6 & \alpha
        \end{array} \right| = 2\alpha-6 \neq 0 \Longrightarrow \alpha \neq 3
    \end{equation*}
    \begin{equation*}
        \left|\begin{array}{ccc}
            2 & 1 & 3\\
            6 & \alpha & 10 \\
            4 & 6 & 8    
        \end{array} \right| = 16\alpha +40 +108-12\alpha-120-48 = 4\alpha -20 \neq 0 \Longrightarrow \alpha\neq 5
    \end{equation*}

    Por tanto, se puede resolver siempre que $\alpha \neq \{3,5\}$. 
    
    Alternativamente, se puede siempre que $a_{kk}^{(k)} \neq 0$. Veamos:
    \begin{multline*}
        \left(\begin{array}{ccc|c}
            2 & 1 & 3 & 1\\
            6 & \alpha & 10 & 5 \\
            4 & 6 & 8 & 5   
        \end{array} \right)
        \xrightarrow[F'_3=F_3-2F1]{F'_2=F_2-3F_1}
        \left(\begin{array}{ccc|c}
            2 & 1 & 3 & 1\\
            0 & \alpha-3 & 1 & 2 \\
            0 & 4 & 2 & 3   
        \end{array} \right)
        \longrightarrow \\
        \xrightarrow[F'_3=F_3+m_{3,2}F_2]{m_{3,2}=\frac{-4}{\alpha-3}}
        \left(\begin{array}{ccc|c}
            2 & 1 & 3 & 1\\
            0 & \alpha-3 & 1 & 2 \\
            0 & 0 & 2-\frac{4}{\alpha-3} & 3-\frac{8}{\alpha-3}
        \end{array} \right)
    \end{multline*}

    $a_{11}^{(1)}=2\neq 0$
    
    $a_{22}^{(2)}=\alpha- 3\neq 0 \Longleftrightarrow \alpha \neq 3$
    
    $a_{33}^{(3)}=2-\frac{4}{\alpha-3}\neq 0 \Longleftrightarrow 2\alpha-6\neq 4 \Longleftrightarrow \alpha\neq 5$\\
    
    Podemos ver que, efectivamente, se cumple que se puede resolver si $\alpha\neq \{3,5\}$
\end{ejercicio}


\begin{ejercicio} [\textbf{2 puntos}]
    Resuelve el sistema
    \begin{equation*}
        \left\{ \begin{array}{rrrrr}
            0.0300x_1 & + & 58.9x_2 &=& 59.2  \\
            5.31x_1 & - & 6.10 x_2 &=&  47.0 \\
        \end{array}\right.
    \end{equation*}
    utilizando el método de Gauss con pivote parcial y aritmética de tres dígitos. Realiza los cálculos uno a uno indicando claramente el redondeo a tres dígitos en cada paso.
    \begin{multline*}
        \left( \begin{array}{cc|c}
            0.03 & 58.9 & 59.2 \\
            5.31 & -6.10 & 47.0
        \end{array}\right)
        \xrightarrow{F_1 \Longleftrightarrow F_2}
        \left( \begin{array}{cc|c}
            5.31 & -6.10 & 47.0\\
            0.03 & 58.9 & 59.2 \\
        \end{array}\right)
        \longrightarrow \\
        \xrightarrow[m_{2,1} = -\frac{0.03}{5.31}\approx -0.00565]{F'_2 = F_2 + m_{2,1}F_1}
        \left( \begin{array}{cc|c}
            5.31 & -6.10 & 47.0\\
            0 & 58.9 & 58.9
        \end{array}\right)
    \end{multline*}
    \begin{equation*}
        x_2 \approx \frac{58.9}{58.9} =1 \qquad \qquad x_1 \approx \frac{47+6.1x_2}{5.31} \approx \frac{47+6.1}{5.31} \approx \frac{53.1}{5.31} =10
    \end{equation*}
\end{ejercicio}

\begin{ejercicio} [\textbf{1.5 puntos}]
     Pon un ejemplo de matriz simétrica que admita factorización de Cholesky y otra que no la admita. Justifica tu respuesta.
     \begin{equation*}
         A = \left( \begin{array}{ccc}
             2 & 1 & 1 \\
             1 & 2 & 1 \\
             1 & 1 & 2
         \end{array}\right)
         \qquad \qquad
         B = \left( \begin{array}{ccc}
             -2 & 1 & 1 \\
             1 & 2 & 1 \\
             1 & 1 & 2
         \end{array}\right)
     \end{equation*}

     Una condición necesaria y suficiente para que una matriz admita factorización de Cholesky es que sea simétrica y definida positiva. Como $A$ y $B$ son simétricas, una de ellas debe ser definida positiva y la otra no.

     Es fácil ver que $A$ es definida positiva, ya que:
     \begin{equation*}
         |2|=2 \qquad \left| \begin{array}{cc}
             2 & 1 \\
             1 & 2 \\
         \end{array}\right| = 3 \qquad |A|=10-6=4
     \end{equation*}

     Por tanto, $A$ sí admite factorización de Cholesky. Sin embargo, $B$ no es definida positiva ya que su menor principal de orden 1 es negativo, por lo que $B$ no admite factorización de Cholesky.
\end{ejercicio}

\begin{ejercicio} [\textbf{1.5 puntos}]
      Se considera una norma vectorial $||\cdot||$ en $\bb{R}^n$ y la correspondiente norma matricial inducida $||\cdot||$. Dada una matriz cuadrada regular $S$ de orden $n$, se define la norma vectorial $||\cdot||_S$ por:
     $$||x||_S = ||Sx||$$

     \begin{enumerate}
         \item Prueba que así definida es una norma en $\bb{R}^n$
         \begin{itemize}
             \item $||x||_S = ||Sx|| \geq 0$ por ser $||\cdot||$ una norma vectorial. Además, se comprueba que $||x||_S = 0 \Longleftrightarrow x=0$
             $$||x||_S = 0 \Longleftrightarrow ||Sx|| = 0 \Longleftrightarrow Sx = 0 \Longleftrightarrow x = S^{-1}\cdot 0 = 0$$

             \item $||cx||_S = ||S(cx)|| = ||c(Sx)|| = |c|\cdot ||Sx|| = |c| \cdot ||x||_S$

             \item $||x+y||_S = ||S(x+y)|| = ||Sx + Sy|| \leq ||Sx|| + ||Sy|| = ||x||_S + ||y||_S$
         \end{itemize}

         \item Prueba que la norma matricial inducida es
         $$||A||_S = ||SAS^{-1}||$$
         \begin{proof}
             \begin{multline*}
                 ||A||_S = \max_{x\neq 0} \frac{||Ax||_S}{||x||_S}
                 = \max_{x\neq 0} \frac{||SAx||}{||Sx||}
                 = \max_{x\neq 0} \frac{||SAS^{-1}Sx||}{||Sx||}
                 =\\ \stackrel{(\ast)}{=}
                 \max_{Sx\neq 0} \frac{||SAS^{-1}Sx||}{||Sx||} = ||SAS^{-1}||
             \end{multline*}
             Donde en $(\ast)$ he usado que, por ser $S$ regular,
             $$\{x\in \bb{R}^n \mid x\neq 0\} = \{x\in \bb{R}^n \mid Sx\neq 0\}$$
             Como ambos conjuntos son los mismos, el máximo se alcanzará en el mismo valor.
        \end{proof}

         \item Si denotamos por $\kappa(A)$ y $\kappa_S(A)$ el número de condición de la matriz $A$ respecto de las normas $||\cdot||$ y $||\cdot||_S$ respectivamente, prueba que:
         $$\kappa_S (A) \leq \kappa(S)^2\kappa(A)$$
         \begin{proof}
             \begin{multline*}
                 \kappa_S(A) = ||A||_S ||A^{-1}||_S = ||SAS^{-1}||\cdot ||SA^{-1}S^{-1}||
                 \leq \\ \leq
                 ||S||^2 ||S^{-1}||^2 ||A||||A^{-1}|| = \kappa(S)^2 \kappa(A)
             \end{multline*}
         \end{proof}
     \end{enumerate}
\end{ejercicio}

\begin{ejercicio}\textbf{[3 puntos]}
    Dadas las matrices
    $$A=\left( \begin{array}{cc}
        -2 & 1/2 \\
        -1/2 & -2
    \end{array}\right)
    \qquad \qquad
    b=\left( \begin{array}{c}
        8\\32
    \end{array}\right)$$
    se pretende resolver el sistema $Ax=b$.

    \begin{enumerate}
        \item ¿Se puede garantizar la convergencia de los métodos de Jacobi y Gauss-Seidel? Justifica la respuesta.\\
        
        Sí, ya que la matriz $A$ es E.D.D., ya que $2>1/2$ y $2>1/2$.\\

        Alternativamente, y solo para el caso del método de Jacobi, se demuestra de manera general.
        
        La matriz de descomposición del método de Jacobi es:
        $$Q=D=\left( \begin{array}{cc}
            -2 & 0 \\
            0 & -2
        \end{array}\right) = -2I$$
        Por tanto, el sistema de punto fijo de Jacobi $x=B_J x + c$  tiene como $B_J$ a la matriz:
        $$B_J = I-Q^{-1}A = I+\frac{1}{2}A=I + \left( \begin{array}{cc}
            -1 & 1/4 \\
            -1/4 & -1
        \end{array}\right)
        = \left( \begin{array}{cc}
            0 & 1/4 \\
            -1/4 & 0
        \end{array}\right)$$

        Por tanto, como $||B_J||_1 < 1 \Longrightarrow $ este método iterativo converge.

        \item Escribe las ecuaciones de los métodos y realiza dos iteraciones del método de Gauss-Seidel partiendo de $x^{(0)}=(0,0)$.\\

        Las ecuaciones del método de Jacobi son:
        \begin{equation*}
            \left\{\begin{array}{l}
                x_1^{(k+1)} = -\frac{1}{2}\left( -\frac{1}{2}x_2^{(k)} + 8 \right) = \frac{1}{4}x_2^{(k)} - 4 \\
                x_2^{(k+1)} = -\frac{1}{2}\left( \frac{1}{2}x_1^{(k)} + 32 \right) =  -\frac{1}{4}x_1^{(k)} - 16\\
            \end{array} \right.
        \end{equation*}
        
        Las ecuaciones del método de Gauss-Seidel son:
        \begin{equation*}
            \left\{\begin{array}{l}
                x_1^{(k+1)} = \frac{1}{4}x_2^{(k)} - 4 \\
                x_2^{(k+1)} = -\frac{1}{4}x_1^{(k+1)} - 16\\
            \end{array} \right.
        \end{equation*}

        Realizamos ahora dos iteraciones del método de Gauss-Seidel:
        \begin{equation*}
            \begin{array}{c|c|c}
                k & x_1^{(k)} & x_2^{(k)} \\ \hline
                0 & 0 & 0 \\
                1 & -4 & -15 \\
                2 & -\frac{31}{4} & -\frac{225}{16} \\
            \end{array}
        \end{equation*}


        \item Se propone el método iterativo
        $$x^{k+1} = (I-\omega A)x^{(k)} + \omega b$$
        Prueba que para $\omega=-\frac{1}{2}$ el método converge a la solución del sistema para cualquier valor inicial $x^{(0)}$. ¿Que debe cumplir $\omega$ para que el método sea convergente? Indica algún otro valor para el que así sea.

        \begin{equation*}
            I-\omega A = \left( \begin{array}{cc}
                1+2\omega & -\frac{\omega}{2} \\
                \frac{\omega}{2} & 1+2\omega
            \end{array}\right)
        \end{equation*}

        \begin{equation*}
            P_{I-\omega A}(\lambda) = \lambda^2 - (2+4\omega)\lambda + (1+2\omega)^2 + \frac{\omega^2}{4}
        \end{equation*}

        Los valores propios de dicha matriz son: $$\lambda\in \bb{R} \mid \lambda^2 - (2+4\omega)\lambda + (1+2\omega)^2 + \frac{\omega^2}{4} =0$$
        \begin{equation*}\begin{split}
            \lambda &= \frac{2+4\omega \pm \sqrt{(2+4\omega)^2 -4(1+2\omega)^2 - \omega^2}}{2}\\
            &= \frac{2+4\omega \pm \sqrt{\cancel{2^2(1+2\omega)^2} \cancel{-4(1+2\omega)^2} - \omega^2}}{2} \\
            &= 1+2\omega \pm \frac{|\omega|}{2}i = 1 + 2\omega \pm \frac{\omega}{2}i
        \end{split}\end{equation*}

        Por tanto, los valores propios son: $\left\{1 + 2\omega \pm \frac{\omega}{2}i\right\}$. Para que el método iterativo converga, necesitamos que
        $$\max \left\{\left|1 + 2\omega + \frac{\omega}{2}i\right| ,\left| 1 + 2\omega - \frac{\omega}{2}i\right| \right\} < 1$$

        Como $\left| 1 + 2\omega - \frac{\omega}{2}i\right| = \left| 1 + 2\omega + \frac{\omega}{2}i\right|$, la inecuación a resolver es:
        \begin{multline*}
            \left| 1 + 2\omega - \frac{\omega}{2}i\right| < 1 \Longleftrightarrow \sqrt{(1+2\omega)^2 + \frac{\omega^2}{4}} < 1 \Longleftrightarrow 1 + 4\omega + 4\omega^2 + \frac{\omega^2}{4} < 1 
            \Longleftrightarrow \\ \Longleftrightarrow
            \omega\left(4 + 4\omega + \frac{\omega}{4}\right) < 0
        \end{multline*}

        Esta última desigualdad se cumple solo si uno de los dos términos es negativos.
        \begin{equation*}
            \omega < 0 \hspace{3cm} 4 + 4\omega + \frac{\omega}{4} = 4 + \frac{17}{4}\omega < 0 \Longleftrightarrow \omega < \frac{-16}{17}
        \end{equation*}

        Por tanto, el método iterativo converge si: $$\omega \in \left]-\frac{16}{17},0\right[$$
    \end{enumerate}
\end{ejercicio}
\end{document}