\documentclass[12pt]{article}

% Idioma y codificación
\usepackage[spanish, es-tabla]{babel}       %es-tabla para que se titule "Tabla"
\usepackage[utf8]{inputenc}

% Márgenes
\usepackage[a4paper,top=3cm,bottom=2.5cm,left=3cm,right=3cm]{geometry}

% Comentarios de bloque
\usepackage{verbatim}

% Paquetes de links
\usepackage[hidelinks]{hyperref}    % Permite enlaces
\usepackage{url}                    % redirecciona a la web

% Más opciones para enumeraciones
\usepackage{enumitem}

% Personalizar la portada
\usepackage{titling}

% Paquetes de tablas
\usepackage{multirow}


%------------------------------------------------------------------------

%Paquetes de figuras
\usepackage{caption}
\usepackage{subcaption} % Figuras al lado de otras
\usepackage{float}      % Poner figuras en el sitio indicado H.


% Paquetes de imágenes
\usepackage{graphicx}       % Paquete para añadir imágenes
\usepackage{transparent}    % Para manejar la opacidad de las figuras

% Paquete para usar colores
\usepackage[dvipsnames]{xcolor}
\usepackage{pagecolor}      % Para cambiar el color de la página

% Habilita tamaños de fuente mayores
\usepackage{fix-cm}

% Para los gráficos
\usepackage{tikz}

% Para poder situar los nodos en los grafos
\usetikzlibrary{positioning}


%------------------------------------------------------------------------

% Paquetes de matemáticas
\usepackage{mathtools, amsfonts, amssymb, mathrsfs}
\usepackage[makeroom]{cancel}     % Simplificar tachando
\usepackage{polynom}    % Divisiones y Ruffini
\usepackage{units} % Para poner fracciones diagonales con \nicefrac

\usepackage{pgfplots}   %Representar funciones
\pgfplotsset{compat=1.18}  % Versión 1.18

\usepackage{tikz-cd}    % Para usar diagramas de composiciones
\usetikzlibrary{calc}   % Para usar cálculo de coordenadas en tikz

%Definición de teoremas, etc.
\usepackage{amsthm}
%\swapnumbers   % Intercambia la posición del texto y de la numeración

\theoremstyle{plain}

\makeatletter
\@ifclassloaded{article}{
  \newtheorem{teo}{Teorema}[section]
}{
  \newtheorem{teo}{Teorema}[chapter]  % Se resetea en cada chapter
}
\makeatother

\newtheorem{coro}{Corolario}[teo]           % Se resetea en cada teorema
\newtheorem{prop}[teo]{Proposición}         % Usa el mismo contador que teorema
\newtheorem{lema}[teo]{Lema}                % Usa el mismo contador que teorema

\theoremstyle{remark}
\newtheorem*{observacion}{Observación}

\theoremstyle{definition}

\makeatletter
\@ifclassloaded{article}{
  \newtheorem{definicion}{Definición} [section]     % Se resetea en cada chapter
}{
  \newtheorem{definicion}{Definición} [chapter]     % Se resetea en cada chapter
}
\makeatother

\newtheorem*{notacion}{Notación}
\newtheorem*{ejemplo}{Ejemplo}
\newtheorem*{ejercicio*}{Ejercicio}             % No numerado
\newtheorem{ejercicio}{Ejercicio} [section]     % Se resetea en cada section


% Modificar el formato de la numeración del teorema "ejercicio"
\renewcommand{\theejercicio}{%
  \ifnum\value{section}=0 % Si no se ha iniciado ninguna sección
    \arabic{ejercicio}% Solo mostrar el número de ejercicio
  \else
    \thesection.\arabic{ejercicio}% Mostrar número de sección y número de ejercicio
  \fi
}


% \renewcommand\qedsymbol{$\blacksquare$}         % Cambiar símbolo QED
%------------------------------------------------------------------------

% Paquetes para encabezados
\usepackage{fancyhdr}
\pagestyle{fancy}
\fancyhf{}

\newcommand{\helv}{ % Modificación tamaño de letra
\fontfamily{}\fontsize{12}{12}\selectfont}
\setlength{\headheight}{15pt} % Amplía el tamaño del índice


%\usepackage{lastpage}   % Referenciar última pag   \pageref{LastPage}
\fancyfoot[C]{\thepage}

%------------------------------------------------------------------------

% Conseguir que no ponga "Capítulo 1". Sino solo "1."
\makeatletter
\@ifclassloaded{book}{
  \renewcommand{\chaptermark}[1]{\markboth{\thechapter.\ #1}{}} % En el encabezado
    
  \renewcommand{\@makechapterhead}[1]{%
  \vspace*{50\p@}%
  {\parindent \z@ \raggedright \normalfont
    \ifnum \c@secnumdepth >\m@ne
      \huge\bfseries \thechapter.\hspace{1em}\ignorespaces
    \fi
    \interlinepenalty\@M
    \Huge \bfseries #1\par\nobreak
    \vskip 40\p@
  }}
}
\makeatother

%------------------------------------------------------------------------
% Paquetes de cógido
\usepackage{minted}
\renewcommand\listingscaption{Código fuente}

\usepackage{fancyvrb}
% Personaliza el tamaño de los números de línea
\renewcommand{\theFancyVerbLine}{\small\arabic{FancyVerbLine}}

% Estilo para C++
\newminted{cpp}{
    frame=lines,
    framesep=2mm,
    baselinestretch=1.2,
    linenos,
    escapeinside=||
}

% para minted
\definecolor{LightGray}{rgb}{0.95,0.95,0.92}
\setminted{
    linenos=true,
    stepnumber=5,
    numberfirstline=true,
    autogobble,
    breaklines=true,
    breakautoindent=true,
    breaksymbolleft=,
    breaksymbolright=,
    breaksymbolindentleft=0pt,
    breaksymbolindentright=0pt,
    breaksymbolsepleft=0pt,
    breaksymbolsepright=0pt,
    fontsize=\footnotesize,
    bgcolor=LightGray,
    numbersep=10pt
}


\usepackage{listings} % Para incluir código desde un archivo

\renewcommand\lstlistingname{Código Fuente}
\renewcommand\lstlistlistingname{Índice de Códigos Fuente}

% Definir colores
\definecolor{vscodepurple}{rgb}{0.5,0,0.5}
\definecolor{vscodeblue}{rgb}{0,0,0.8}
\definecolor{vscodegreen}{rgb}{0,0.5,0}
\definecolor{vscodegray}{rgb}{0.5,0.5,0.5}
\definecolor{vscodebackground}{rgb}{0.97,0.97,0.97}
\definecolor{vscodelightgray}{rgb}{0.9,0.9,0.9}

% Configuración para el estilo de C similar a VSCode
\lstdefinestyle{vscode_C}{
  backgroundcolor=\color{vscodebackground},
  commentstyle=\color{vscodegreen},
  keywordstyle=\color{vscodeblue},
  numberstyle=\tiny\color{vscodegray},
  stringstyle=\color{vscodepurple},
  basicstyle=\scriptsize\ttfamily,
  breakatwhitespace=false,
  breaklines=true,
  captionpos=b,
  keepspaces=true,
  numbers=left,
  numbersep=5pt,
  showspaces=false,
  showstringspaces=false,
  showtabs=false,
  tabsize=2,
  frame=tb,
  framerule=0pt,
  aboveskip=10pt,
  belowskip=10pt,
  xleftmargin=10pt,
  xrightmargin=10pt,
  framexleftmargin=10pt,
  framexrightmargin=10pt,
  framesep=0pt,
  rulecolor=\color{vscodelightgray},
  backgroundcolor=\color{vscodebackground},
}

%------------------------------------------------------------------------

% Comandos definidos
\newcommand{\bb}[1]{\mathbb{#1}}
\newcommand{\cc}[1]{\mathcal{#1}}

% I prefer the slanted \leq
\let\oldleq\leq % save them in case they're every wanted
\let\oldgeq\geq
\renewcommand{\leq}{\leqslant}
\renewcommand{\geq}{\geqslant}

% Si y solo si
\newcommand{\sii}{\iff}

% Letras griegas
\newcommand{\eps}{\epsilon}
\newcommand{\veps}{\varepsilon}
\newcommand{\lm}{\lambda}

\newcommand{\ol}{\overline}
\newcommand{\ul}{\underline}
\newcommand{\wt}{\widetilde}
\newcommand{\wh}{\widehat}

\let\oldvec\vec
\renewcommand{\vec}{\overrightarrow}

% Derivadas parciales
\newcommand{\del}[2]{\frac{\partial #1}{\partial #2}}
\newcommand{\Del}[3]{\frac{\partial^{#1} #2}{\partial #3^{#1}}}
\newcommand{\deld}[2]{\dfrac{\partial #1}{\partial #2}}
\newcommand{\Deld}[3]{\dfrac{\partial^{#1} #2}{\partial #3^{#1}}}


\newcommand{\AstIg}{\stackrel{(\ast)}{=}}
\newcommand{\Hop}{\stackrel{L'H\hat{o}pital}{=}}

\newcommand{\red}[1]{{\color{red}#1}} % Para integrales, destacar los cambios.

% Método de integración
\newcommand{\MetInt}[2]{
    \left[\begin{array}{c}
        #1 \\ #2
    \end{array}\right]
}

% Declarar aplicaciones
% 1. Nombre aplicación
% 2. Dominio
% 3. Codominio
% 4. Variable
% 5. Imagen de la variable
\newcommand{\Func}[5]{
    \begin{equation*}
        \begin{array}{rrll}
            #1:& #2 & \longrightarrow & #3\\
               & #4 & \longmapsto & #5
        \end{array}
    \end{equation*}
}

%------------------------------------------------------------------------



\begin{document}

    % 1. Foto de fondo
    % 2. Título
    % 3. Encabezado Izquierdo
    % 4. Color de fondo
    % 5. Coord x del titulo
    % 6. Coord y del titulo
    % 7. Fecha

    
    % 1. Foto de fondo
% 2. Título
% 3. Encabezado Izquierdo
% 4. Color de fondo
% 5. Coord x del titulo
% 6. Coord y del titulo
% 7. Fecha

\newcommand{\portada}[7]{

    \portadaBase{#1}{#2}{#3}{#4}{#5}{#6}{#7}
    \portadaBook{#1}{#2}{#3}{#4}{#5}{#6}{#7}
}

\newcommand{\portadaExamen}[7]{

    \portadaBase{#1}{#2}{#3}{#4}{#5}{#6}{#7}
    \portadaArticle{#1}{#2}{#3}{#4}{#5}{#6}{#7}
}




\newcommand{\portadaBase}[7]{

    % Tiene la portada principal y la licencia Creative Commons
    
    % 1. Foto de fondo
    % 2. Título
    % 3. Encabezado Izquierdo
    % 4. Color de fondo
    % 5. Coord x del titulo
    % 6. Coord y del titulo
    % 7. Fecha
    
    
    \thispagestyle{empty}               % Sin encabezado ni pie de página
    \newgeometry{margin=0cm}        % Márgenes nulos para la primera página
    
    
    % Encabezado
    \fancyhead[L]{\helv #3}
    \fancyhead[R]{\helv \nouppercase{\leftmark}}
    
    
    \pagecolor{#4}        % Color de fondo para la portada
    
    \begin{figure}[p]
        \centering
        \transparent{0.3}           % Opacidad del 30% para la imagen
        
        \includegraphics[width=\paperwidth, keepaspectratio]{assets/#1}
    
        \begin{tikzpicture}[remember picture, overlay]
            \node[anchor=north west, text=white, opacity=1, font=\fontsize{60}{90}\selectfont\bfseries\sffamily, align=left] at (#5, #6) {#2};
            
            \node[anchor=south east, text=white, opacity=1, font=\fontsize{12}{18}\selectfont\sffamily, align=right] at (9.7, 3) {\textbf{\href{https://losdeldgiim.github.io/}{Los Del DGIIM}}};
            
            \node[anchor=south east, text=white, opacity=1, font=\fontsize{12}{15}\selectfont\sffamily, align=right] at (9.7, 1.8) {Doble Grado en Ingeniería Informática y Matemáticas\\Universidad de Granada};
        \end{tikzpicture}
    \end{figure}
    
    
    \restoregeometry        % Restaurar márgenes normales para las páginas subsiguientes
    \pagecolor{white}       % Restaurar el color de página
    
    
    \newpage
    \thispagestyle{empty}               % Sin encabezado ni pie de página
    \begin{tikzpicture}[remember picture, overlay]
        \node[anchor=south west, inner sep=3cm] at (current page.south west) {
            \begin{minipage}{0.5\paperwidth}
                \href{https://creativecommons.org/licenses/by-nc-nd/4.0/}{
                    \includegraphics[height=2cm]{assets/Licencia.png}
                }\vspace{1cm}\\
                Esta obra está bajo una
                \href{https://creativecommons.org/licenses/by-nc-nd/4.0/}{
                    Licencia Creative Commons Atribución-NoComercial-SinDerivadas 4.0 Internacional (CC BY-NC-ND 4.0).
                }\\
    
                Eres libre de compartir y redistribuir el contenido de esta obra en cualquier medio o formato, siempre y cuando des el crédito adecuado a los autores originales y no persigas fines comerciales. 
            \end{minipage}
        };
    \end{tikzpicture}
    
    
    
    % 1. Foto de fondo
    % 2. Título
    % 3. Encabezado Izquierdo
    % 4. Color de fondo
    % 5. Coord x del titulo
    % 6. Coord y del titulo
    % 7. Fecha


}


\newcommand{\portadaBook}[7]{

    % 1. Foto de fondo
    % 2. Título
    % 3. Encabezado Izquierdo
    % 4. Color de fondo
    % 5. Coord x del titulo
    % 6. Coord y del titulo
    % 7. Fecha

    % Personaliza el formato del título
    \pretitle{\begin{center}\bfseries\fontsize{42}{56}\selectfont}
    \posttitle{\par\end{center}\vspace{2em}}
    
    % Personaliza el formato del autor
    \preauthor{\begin{center}\Large}
    \postauthor{\par\end{center}\vfill}
    
    % Personaliza el formato de la fecha
    \predate{\begin{center}\huge}
    \postdate{\par\end{center}\vspace{2em}}
    
    \title{#2}
    \author{\href{https://losdeldgiim.github.io/}{Los Del DGIIM}}
    \date{Granada, #7}
    \maketitle
    
    \tableofcontents
}




\newcommand{\portadaArticle}[7]{

    % 1. Foto de fondo
    % 2. Título
    % 3. Encabezado Izquierdo
    % 4. Color de fondo
    % 5. Coord x del titulo
    % 6. Coord y del titulo
    % 7. Fecha

    % Personaliza el formato del título
    \pretitle{\begin{center}\bfseries\fontsize{42}{56}\selectfont}
    \posttitle{\par\end{center}\vspace{2em}}
    
    % Personaliza el formato del autor
    \preauthor{\begin{center}\Large}
    \postauthor{\par\end{center}\vspace{3em}}
    
    % Personaliza el formato de la fecha
    \predate{\begin{center}\huge}
    \postdate{\par\end{center}\vspace{5em}}
    
    \title{#2}
    \author{\href{https://losdeldgiim.github.io/}{Los Del DGIIM}}
    \date{Granada, #7}
    \thispagestyle{empty}               % Sin encabezado ni pie de página
    \maketitle
    \vfill
}
    \portadaExamen{ffccA4.jpg}{MN I\\Examen II}{MN I. Examen II}{MidnightBlue}{-8}{28}{2023}{Arturo Olivares Martos}

    \begin{description}
        \item[Asignatura] Métodos Numéricos I.
        \item[Curso Académico] 2021-22.
        \item[Grado] Matemáticas.
        \item[Grupo] B.
        \item[Profesor] Teresa Encarnación Pérez Fernández.
        \item[Descripción] Prueba 1. Temas 1 y 2.
        \item[Fecha] 20 de abril de 2022.
        %\item[Duración] 60 minutos.
    
    \end{description}
    \newpage
    
    \begin{ejercicio} [\textbf{1.5 puntos}]
    Calcule $p(1.1)$ para el polinomio $p(x) = x^3 -3 x^2 + 3 x$ utilizando la expresión explícita y el algoritmo de Horner en aritmética de redondeo a dos dígitos. ¿Cambian los resultados? ¿Qué valor es más exacto? ¿Porqué?\\

    En el caso de el algoritmo de Horner, el valor es $p(1.1)=0.99$.
    \begin{equation*}
        \begin{array}{c|cccc}
             & 1 & -3 & 3 & 0 \\
            1.1  & & 1.1 & -2.1 & 0.99\\ \hline
            & 1 & -1.9 & 0.9 & 0.99
        \end{array}
    \end{equation*}

    Evaluando en la expresión analítica:
    \begin{equation*}
        p(1.1)=(1.1)^3 -3 (1.1)^2 + 3 (1.1) = 1.3 -3(1.2) + 3(1.1) = 1.3 -3.6 + 3.3 = 1
    \end{equation*}

    El valor exacto es $p(1.1)=1.001$. Aunque por norma general es más exacto el algoritmo de Horner, en este caso es más exacto evaular en la expresión analítica. Esto se debe a que al evaluar al cubo se produce un error por defecto, mientras que al evaluar en el monomio de grado dos se produce un error por exceso. Por tanto, los errores, aunque son mayores, se compensan.
\end{ejercicio}


\begin{ejercicio} [\textbf{2 puntos}]
    Dada la matriz
    \begin{equation*}
        A= \left( \begin{array}{ccc}
            3 & 6 & 7 \\
            6 & 12 & 5 \\
            3 & 5 & 7 \\
        \end{array}\right)
    \end{equation*}

    \begin{enumerate}
        \item Razone que $A$ no admite factorización LU sin realizar cálculos.
        \begin{equation*}
            \left| \begin{array}{cc}
                3 & 6 \\
                6 & 12
            \end{array}\right| = 36-36 = 0
        \end{equation*}
        Como el determinante principal de orden $2$ de $A$ es nulo, entonces no admite factorización LU sin intercambio de filas.

        \item Razone que existe una matriz obtenida mediante permutación de las filas de la matriz $A$ que sí admite factorización LU, y encuéntrela.
        \begin{equation*}
            |A|=\cancel{7\cdot 3 \cdot 15} + 6\cdot 5\cdot 3 + 6\cdot 5 \cdot 7 - \cancel{7\cdot3\cdot15} - 6^2\cdot 7 - 5^2\cdot 3 = -27
        \end{equation*}
        Como $|A|\neq 0 \Longrightarrow A$ es regular, y como $A$ es regular, con una permutación de filas podremos obtener una matriz que admita factorización LU. En este caso, esa permutación es $F_2\Longleftrightarrow F_3$.
        \begin{equation*}
            A'= \left( \begin{array}{ccc}
                3 & 6 & 7 \\
                3 & 5 & 7 \\
                6 & 12 & 5 \\
            \end{array}\right)
        \end{equation*}
        \begin{equation*}
            |3|=3 \qquad \left| \begin{array}{cc}
                3 & 6 \\
                3 & 5
            \end{array}\right| = 15 - 18 = -3 \qquad |A|=27
        \end{equation*}

        Como los menores principales de $A'$ son no nulos, entonces admite factorización $LU$.
    \end{enumerate}
\end{ejercicio}

\begin{ejercicio} [\textbf{2.5 puntos}]
     Se considera el sistema de ecuaciones $Ax = b$, donde
    \begin{equation*}
        A= \left( \begin{array}{cccc}
            1/a & a & \dots & a \\
            a & 1/a & \dots & a \\
            \vdots & \vdots & \ddots & \vdots \\
            a & a & \dots & 1/a\\
        \end{array}\right), \quad a\in \bb{R}
    \end{equation*}

    \begin{enumerate}
        \item Estudie para qué valores del parámetro $a$ el método de Jacobi es convergente. Escriba las ecuaciones del método, indicando explícitamente cuál es la matriz $B_J$.

        \begin{multline*}
            \left\{ \begin{array}{l}
                 \frac{1}{a}x_1 + ax_2 + \dots + ax_n = b_1 \\
                 ax_1 + \frac{1}{a}x_2 + \dots + ax_n = b_2 \\
                 \vdots \\
                 ax_1 + ax_2 + \dots + \frac{1}{a}x_n = b_n
            \end{array} \right.
            \Longrightarrow
            \left\{ \begin{array}{l}
                 x_1^{(k+1)} = ab_1 - a^2x_2^{(k)} - \dots - a^2x_n^{(k)} \\
                 x_2^{(k+1)} = ab_2 - a^2x_1^{(k)} -a^2x_3^{(k)} - \dots - a^2x_n^{(k)} \\
                 \vdots \\
                 x_n^{(k+1)} = ab_n - a^2x_1^{(k)} - \dots - a^2x_{n-1}^{(k)} \\
            \end{array} \right.
            \\
            \Longrightarrow x_i^{(k+1)} = a\left( b_i - \sum_{j=0, j\neq i}^n ax_j^{(k)}\right) = ab_i - \sum_{j=0, j\neq i}^n a^2x_j^{(k)}
        \end{multline*}

        Por tanto,
        \begin{equation*}
            B_J = \left( \begin{array}{cccc}
                0 & -a^2 & \dots & -a^2 \\
                -a^2 & 0 & \dots & -a^2 \\
                \vdots & \vdots & \ddots & \vdots \\
                -a^2 & -a^2 & \dots & 0\\
            \end{array}\right)
        \end{equation*}

        Calculemos su polinomio característico:
        \begin{equation*}\begin{split}
            P_{B_J}(\lambda)=|B_J-\lambda I|&=
            \left| \begin{array}{cccc}
                -\lambda & -a^2 & \dots & -a^2 \\
                -a^2 & -\lambda & \dots & -a^2 \\
                \vdots & \vdots & \ddots & \vdots \\
                -a^2 & -a^2 & \dots & -\lambda\\
            \end{array}\right|
            =\\ & \stackrel{(1)}{=}
            \left| \begin{array}{cccc}
                -\lambda-(n-1)a^2 & -\lambda-(n-1)a^2 & \dots & -\lambda-(n-1)a^2 \\
                -a^2 & -\lambda & \dots & -a^2 \\
                \vdots & \vdots & \ddots & \vdots \\
                -a^2 & -a^2 & \dots & -\lambda\\
            \end{array}\right|
            =\\ &=(-\lambda-(n-1)a^2)
            \left| \begin{array}{cccc}
                1 & 1 & \dots & 1 \\
                -a^2 & -\lambda & \dots & -a^2 \\
                \vdots & \vdots & \ddots & \vdots \\
                -a^2 & -a^2 & \dots & -\lambda\\
            \end{array}\right|
            = \\ &\stackrel{(2)}{=}
            (-\lambda-(n-1)a^2)
            \left| \begin{array}{cccc}
                1 & 1 & \dots & 1 \\
                0 & -\lambda+a^2 & \dots & 0 \\
                \vdots & \vdots & \ddots & \vdots \\
                0 &0  & \dots & -\lambda+a^2\\
            \end{array}\right|
            = \\ &= (-\lambda-(n-1)a^2)(-\lambda+a^2)^{n-1}
        \end{split}\end{equation*}
        donde en $(1)$ he sumado fila por fila a la primera fila. Es decir,
        $$(1)\Longrightarrow F'_1 = F_1 + \sum_{k=2}^nF_k$$
        y en $(2)$ he sumado a cada fila $a^2F_1$. Es decir,
        $$(2)\Longrightarrow F'_k = F_k + a^2F_1 \qquad \forall k=2,\dots,n$$

        Por tanto, sus valores propios son: $\sigma(B_J)=\{a^2, -(n-1)a^2\}$.

        Para que el sistema sea convergente para cualquier valor de $x^{(0)}$ inicial, es necesario que $\rho(B_J)<1$. Por tanto,
        \begin{equation*}
            |a^2|=a^2<1 \Longrightarrow |a|<1
        \end{equation*}
        \begin{equation*}
            |-(n-1)a^2|=(n-1)a^2<1 \Longrightarrow a^2 < \frac{1}{n-1} \Longrightarrow |a| < \sqrt{\frac{1}{n-1}}
        \end{equation*}

        Sin embargo, podemos ver que
        \begin{equation*}
            \sqrt{\frac{1}{n-1}} \leq 1 \Longleftrightarrow \frac{1}{n-1} \leq 1 \Longleftrightarrow 1 \leq n-1 \Longleftrightarrow 2 \leq n\text{, trivialmente cierto}
        \end{equation*}

        Por tanto, para que sea convergente es necesario que:
        \begin{equation*}
            |a| < \sqrt{\frac{1}{n-1}}
        \end{equation*}
        
        \item Calcule la matriz $B_J$ y el vector $c$ del método para $n = 3, a = 0,\frac{1}{2},1$, y $b = (1,-1, 2)^T$ . ¿Es convergente el método de Jacobi en estos casos?\\

        Para $n=3$, es convergente para cualquier valor inicial siy solo si $$|a|<\sqrt{\frac{1}{2}} = \frac{\sqrt{2}}{2}\approx 0.7$$
        \begin{equation*}
            B_J^{a=0} = 0.\text{ No tiene sentido, ya que } a_{kk}=\frac{1}{a}=\frac{1}{0}
        \end{equation*}
        \begin{equation*}
            B_J^{a=\frac{1}{2}} = \left( \begin{array}{ccc}
                0 & -\frac{1}{4} & -\frac{1}{4} \\
                -\frac{1}{4} & 0 & -\frac{1}{4} \\
                -\frac{1}{4} & -\frac{1}{4} & 0\\
            \end{array}\right)\qquad \qquad
            B_J^{a=1} = \left( \begin{array}{ccc}
                0 & -1 & -1 \\
                -1 & 0 & -1 \\
                -1 & -1 & 0\\
            \end{array}\right)
        \end{equation*}
    \end{enumerate}

    Para $a=\frac{1}{2}$, podemos ver que $||B_J^{a=\frac{1}{2}}||_1 = ||B_J^{a=\frac{1}{2}}||_\infty = \frac{1}{2} < 1$, por lo que el sistema es convergente. Además, podemos ver que $|a|<\frac{\sqrt{2}}{2}$.

    Para $a=1$, podemos ver que $||B_J^{a=1}||_1 = ||B_J^{a=1}||_\infty = 2 \nless 1$, por lo que no se garantiza que sea convergente. Como $|a|=1\nless \frac{\sqrt{2}}{2} \Longrightarrow$ No convergerá para cualquier valor inicial.
\end{ejercicio}

\begin{ejercicio} [\textbf{4 puntos}]
     El tema de resolución de sistemas de ecuaciones tiene muchos nombres y métodos, y es fácil confundirse, sobre todo cuando te has estudiado los métodos la noche anterior. El método de Gauss hace sustitución hacia atrás (de abajo arriba), y hay un método iterativo de Gauss-no-se-qué (o qué-se-yo-Gauss) que actualiza los valores... ¡el método de Seidel-Gauss es iterativo con actualización de abajo arriba!
     
     Eso es. Dado un sistema de ecuaciones cuadrado $Ax = b$ de dimensión $n \geq 1$ y un vector inicial $x^{(0)}$, el método de Seidel-Gauss es iterativo como todos, en la forma $x^{(k+1)} = B x^{(k)} + c;\quad k \geq 0$, es decir,

     \begin{equation*}
         \left( \begin{array}{c}
            x_1^{(k+1)} \\
            \vdots \\
            x_n^{(k+1)}
         \end{array} \right) =
         \left( \begin{array}{ccc}
            b_{1,1} & \dots & b_{1,n} \\
            \vdots & \ddots & \vdots \\
            b_{n,1} & \dots & b_{n,n} \\
         \end{array} \right)  \left( \begin{array}{c}
            x_1^{(k)} \\
            \vdots \\
            x_{n}^{(k)}
         \end{array} \right)
         + \left( \begin{array}{c}
            c_1 \\
            \vdots \\
            c_n
         \end{array} \right)
     \end{equation*}

     de forma que vamos actualizando las variables de abajo hacia arriba. Para conseguir la siguiente iteración, primero calculamos $x_n^{(k+1)}$ de la última ecuación, después sustituimos esta variable actualizada en la penúltima ecuación junto con los valores anteriores necesarios del paso $k$, obteniendo $x^{(k+1)}_{n-1}$ , y así sucesivamente hasta obtener $x_1^{(k+1)}.$

     \begin{enumerate}
         \item Describa matricialmente el método, esto es, deduzca los valores de la matriz $B$ y el vector $c$ en términos de la matriz de coeficientes $A$.

         \item Escriba las ecuaciones del método de Seidel-Gauss para el sistema de ecuaciones:
         \begin{equation*}\begin{array}{rrrrrrr}
            3x_1 & + & x_2 & + &x_3 & = & 5 \\
            x_1 & + & 3x_2 & - &x_3 & = & 3 \\
            3x_1 & + & x_2 & -& 5x_3 & = & -1 \\
         \end{array}
         \end{equation*}

         y calcule tres iteraciones partiendo del vector inicial $x^{(0)} = (0, 0, 0)^T $.

        \begin{comment}
         \begin{equation*}
             \left.
             \begin{array}{l}
                x_1^{(k+1)} = \frac{1}{3}\left( 5-x_2^{(k+1)}-x_3^{(k+1)}\right)\\
                x_2^{(k+1)} = \frac{1}{3}\left( 3+x_3^{(k+1)}-x_1^{(k)}\right)\\
                x_3^{(k+1)} = -\frac{1}{5}\left( -1-x_2^{(k)}-3x_1^{(k)}\right)\\
            \end{array} \right\}
         \end{equation*}

         Las tres iteraciones son:
         \begin{equation*}
            \renewcommand{\arraystretch}{2}
             \begin{array}{c|ccc}
                 k & x_1 & x_2 & x_3 \\ \hline
                 0 & 0 & 0 & 0 \\
                 \displaystyle 1 & \displaystyle \frac{56}{45} & \displaystyle \frac{16}{15} & \displaystyle \frac{1}{5} \\
                 \displaystyle 2 & \displaystyle \frac{1936}{2025} & \displaystyle \frac{656}{675} & \displaystyle \frac{29}{25} \\
                 \displaystyle 3 & \displaystyle \frac{91976}{91125} & \displaystyle \frac{30496}{30375} & \displaystyle \frac{121}{125} \\
             \end{array}
         \end{equation*}
         \end{comment}

         \item ¿Qué puede decirse acerca de la convergencia?
     \end{enumerate}
\end{ejercicio}


    
\end{document}