\documentclass[12pt]{article}

% Idioma y codificación
\usepackage[spanish, es-tabla, es-notilde]{babel}       %es-tabla para que se titule "Tabla"
\usepackage[utf8]{inputenc}

% Márgenes
\usepackage[a4paper,top=3cm,bottom=2.5cm,left=3cm,right=3cm]{geometry}

% Comentarios de bloque
\usepackage{verbatim}

% Paquetes de links
\usepackage[hidelinks]{hyperref}    % Permite enlaces
\usepackage{url}                    % redirecciona a la web

% Más opciones para enumeraciones
\usepackage{enumitem}

% Personalizar la portada
\usepackage{titling}

% Paquetes de tablas
\usepackage{multirow}

% Para añadir el símbolo de euro
\usepackage{eurosym}


%------------------------------------------------------------------------

%Paquetes de figuras
\usepackage{caption}
\usepackage{subcaption} % Figuras al lado de otras
\usepackage{float}      % Poner figuras en el sitio indicado H.


% Paquetes de imágenes
\usepackage{graphicx}       % Paquete para añadir imágenes
\usepackage{transparent}    % Para manejar la opacidad de las figuras

% Paquete para usar colores
\usepackage[dvipsnames, table, xcdraw]{xcolor}
\usepackage{pagecolor}      % Para cambiar el color de la página

% Habilita tamaños de fuente mayores
\usepackage{fix-cm}

% Para los gráficos
\usepackage{tikz}
\usepackage{forest}

% Para poder situar los nodos en los grafos
\usetikzlibrary{positioning}


%------------------------------------------------------------------------

% Paquetes de matemáticas
\usepackage{mathtools, amsfonts, amssymb, mathrsfs}
\usepackage[makeroom]{cancel}     % Simplificar tachando
\usepackage{polynom}    % Divisiones y Ruffini
\usepackage{units} % Para poner fracciones diagonales con \nicefrac

\usepackage{pgfplots}   %Representar funciones
\pgfplotsset{compat=1.18}  % Versión 1.18

\usepackage{tikz-cd}    % Para usar diagramas de composiciones
\usetikzlibrary{calc}   % Para usar cálculo de coordenadas en tikz

%Definición de teoremas, etc.
\usepackage{amsthm}
%\swapnumbers   % Intercambia la posición del texto y de la numeración

\theoremstyle{plain}

\makeatletter
\@ifclassloaded{article}{
  \newtheorem{teo}{Teorema}[section]
}{
  \newtheorem{teo}{Teorema}[chapter]  % Se resetea en cada chapter
}
\makeatother

\newtheorem{coro}{Corolario}[teo]           % Se resetea en cada teorema
\newtheorem{prop}[teo]{Proposición}         % Usa el mismo contador que teorema
\newtheorem{lema}[teo]{Lema}                % Usa el mismo contador que teorema
\newtheorem*{lema*}{Lema}

\theoremstyle{remark}
\newtheorem*{observacion}{Observación}

\theoremstyle{definition}

\makeatletter
\@ifclassloaded{article}{
  \newtheorem{definicion}{Definición} [section]     % Se resetea en cada chapter
}{
  \newtheorem{definicion}{Definición} [chapter]     % Se resetea en cada chapter
}
\makeatother

\newtheorem*{notacion}{Notación}
\newtheorem*{ejemplo}{Ejemplo}
\newtheorem*{ejercicio*}{Ejercicio}             % No numerado
\newtheorem{ejercicio}{Ejercicio} [section]     % Se resetea en cada section


% Modificar el formato de la numeración del teorema "ejercicio"
\renewcommand{\theejercicio}{%
  \ifnum\value{section}=0 % Si no se ha iniciado ninguna sección
    \arabic{ejercicio}% Solo mostrar el número de ejercicio
  \else
    \thesection.\arabic{ejercicio}% Mostrar número de sección y número de ejercicio
  \fi
}


% \renewcommand\qedsymbol{$\blacksquare$}         % Cambiar símbolo QED
%------------------------------------------------------------------------

% Paquetes para encabezados
\usepackage{fancyhdr}
\pagestyle{fancy}
\fancyhf{}

\newcommand{\helv}{ % Modificación tamaño de letra
\fontfamily{}\fontsize{12}{12}\selectfont}
\setlength{\headheight}{15pt} % Amplía el tamaño del índice


%\usepackage{lastpage}   % Referenciar última pag   \pageref{LastPage}
%\fancyfoot[C]{%
%  \begin{minipage}{\textwidth}
%    \centering
%    ~\\
%    \thepage\\
%    \href{https://losdeldgiim.github.io/}{\texttt{\footnotesize losdeldgiim.github.io}}
%  \end{minipage}
%}
\fancyfoot[C]{\thepage}
\fancyfoot[R]{\href{https://losdeldgiim.github.io/}{\texttt{\footnotesize losdeldgiim.github.io}}}

%------------------------------------------------------------------------

% Conseguir que no ponga "Capítulo 1". Sino solo "1."
\makeatletter
\@ifclassloaded{book}{
  \renewcommand{\chaptermark}[1]{\markboth{\thechapter.\ #1}{}} % En el encabezado
    
  \renewcommand{\@makechapterhead}[1]{%
  \vspace*{50\p@}%
  {\parindent \z@ \raggedright \normalfont
    \ifnum \c@secnumdepth >\m@ne
      \huge\bfseries \thechapter.\hspace{1em}\ignorespaces
    \fi
    \interlinepenalty\@M
    \Huge \bfseries #1\par\nobreak
    \vskip 40\p@
  }}
}
\makeatother

%------------------------------------------------------------------------
% Paquetes de cógido
\usepackage{minted}
\renewcommand\listingscaption{Código fuente}

\usepackage{fancyvrb}
% Personaliza el tamaño de los números de línea
\renewcommand{\theFancyVerbLine}{\small\arabic{FancyVerbLine}}

% Estilo para C++
\newminted{cpp}{
    frame=lines,
    framesep=2mm,
    baselinestretch=1.2,
    linenos,
    escapeinside=||
}

% para minted
\definecolor{LightGray}{rgb}{0.95,0.95,0.92}
\setminted{
    linenos=true,
    stepnumber=5,
    numberfirstline=true,
    autogobble,
    breaklines=true,
    breakautoindent=true,
    breaksymbolleft=,
    breaksymbolright=,
    breaksymbolindentleft=0pt,
    breaksymbolindentright=0pt,
    breaksymbolsepleft=0pt,
    breaksymbolsepright=0pt,
    fontsize=\footnotesize,
    bgcolor=LightGray,
    numbersep=10pt
}


\usepackage{listings} % Para incluir código desde un archivo

\renewcommand\lstlistingname{Código Fuente}
\renewcommand\lstlistlistingname{Índice de Códigos Fuente}

% Definir colores
\definecolor{vscodepurple}{rgb}{0.5,0,0.5}
\definecolor{vscodeblue}{rgb}{0,0,0.8}
\definecolor{vscodegreen}{rgb}{0,0.5,0}
\definecolor{vscodegray}{rgb}{0.5,0.5,0.5}
\definecolor{vscodebackground}{rgb}{0.97,0.97,0.97}
\definecolor{vscodelightgray}{rgb}{0.9,0.9,0.9}

% Configuración para el estilo de C similar a VSCode
\lstdefinestyle{vscode_C}{
  backgroundcolor=\color{vscodebackground},
  commentstyle=\color{vscodegreen},
  keywordstyle=\color{vscodeblue},
  numberstyle=\tiny\color{vscodegray},
  stringstyle=\color{vscodepurple},
  basicstyle=\scriptsize\ttfamily,
  breakatwhitespace=false,
  breaklines=true,
  captionpos=b,
  keepspaces=true,
  numbers=left,
  numbersep=5pt,
  showspaces=false,
  showstringspaces=false,
  showtabs=false,
  tabsize=2,
  frame=tb,
  framerule=0pt,
  aboveskip=10pt,
  belowskip=10pt,
  xleftmargin=10pt,
  xrightmargin=10pt,
  framexleftmargin=10pt,
  framexrightmargin=10pt,
  framesep=0pt,
  rulecolor=\color{vscodelightgray},
  backgroundcolor=\color{vscodebackground},
}

%------------------------------------------------------------------------

% Comandos definidos
\newcommand{\bb}[1]{\mathbb{#1}}
\newcommand{\cc}[1]{\mathcal{#1}}

% I prefer the slanted \leq
\let\oldleq\leq % save them in case they're every wanted
\let\oldgeq\geq
\renewcommand{\leq}{\leqslant}
\renewcommand{\geq}{\geqslant}

% Si y solo si
\newcommand{\sii}{\iff}

% MCD y MCM
\DeclareMathOperator{\mcd}{mcd}
\DeclareMathOperator{\mcm}{mcm}

% Signo
\DeclareMathOperator{\sgn}{sgn}

% Letras griegas
\newcommand{\eps}{\epsilon}
\newcommand{\veps}{\varepsilon}
\newcommand{\lm}{\lambda}

\newcommand{\ol}{\overline}
\newcommand{\ul}{\underline}
\newcommand{\wt}{\widetilde}
\newcommand{\wh}{\widehat}

\let\oldvec\vec
\renewcommand{\vec}{\overrightarrow}

% Derivadas parciales
\newcommand{\del}[2]{\frac{\partial #1}{\partial #2}}
\newcommand{\Del}[3]{\frac{\partial^{#1} #2}{\partial #3^{#1}}}
\newcommand{\deld}[2]{\dfrac{\partial #1}{\partial #2}}
\newcommand{\Deld}[3]{\dfrac{\partial^{#1} #2}{\partial #3^{#1}}}


\newcommand{\AstIg}{\stackrel{(\ast)}{=}}
\newcommand{\Hop}{\stackrel{L'H\hat{o}pital}{=}}

\newcommand{\red}[1]{{\color{red}#1}} % Para integrales, destacar los cambios.

% Método de integración
\newcommand{\MetInt}[2]{
    \left[\begin{array}{c}
        #1 \\ #2
    \end{array}\right]
}

% Declarar aplicaciones
% 1. Nombre aplicación
% 2. Dominio
% 3. Codominio
% 4. Variable
% 5. Imagen de la variable
\newcommand{\Func}[5]{
    \begin{equation*}
        \begin{array}{rrll}
            \displaystyle #1:& \displaystyle  #2 & \longrightarrow & \displaystyle  #3\\
               & \displaystyle  #4 & \longmapsto & \displaystyle  #5
        \end{array}
    \end{equation*}
}

%------------------------------------------------------------------------

\begin{document}
	
	% 1. Foto de fondo
	% 2. Título
	% 3. Encabezado Izquierdo
	% 4. Color de fondo
	% 5. Coord x del titulo
	% 6. Coord y del titulo
	% 7. Fecha
	
	
	% 1. Foto de fondo
% 2. Título
% 3. Encabezado Izquierdo
% 4. Color de fondo
% 5. Coord x del titulo
% 6. Coord y del titulo
% 7. Fecha
% 8. Autor

\newcommand{\portada}[8]{
    \portadaBase{#1}{#2}{#3}{#4}{#5}{#6}{#7}{#8}
    \portadaBook{#1}{#2}{#3}{#4}{#5}{#6}{#7}{#8}
}

\newcommand{\portadaFotoDif}[8]{
    \portadaBaseFotoDif{#1}{#2}{#3}{#4}{#5}{#6}{#7}{#8}
    \portadaBook{#1}{#2}{#3}{#4}{#5}{#6}{#7}{#8}
}

\newcommand{\portadaExamen}[8]{
    \portadaBase{#1}{#2}{#3}{#4}{#5}{#6}{#7}{#8}
    \portadaArticle{#1}{#2}{#3}{#4}{#5}{#6}{#7}{#8}
}

\newcommand{\portadaExamenFotoDif}[8]{
    \portadaBaseFotoDif{#1}{#2}{#3}{#4}{#5}{#6}{#7}{#8}
    \portadaArticle{#1}{#2}{#3}{#4}{#5}{#6}{#7}{#8}
}




\newcommand{\portadaBase}[8]{

    % Tiene la portada principal y la licencia Creative Commons
    
    % 1. Foto de fondo
    % 2. Título
    % 3. Encabezado Izquierdo
    % 4. Color de fondo
    % 5. Coord x del titulo
    % 6. Coord y del titulo
    % 7. Fecha
    % 8. Autor    
    
    \thispagestyle{empty}               % Sin encabezado ni pie de página
    \newgeometry{margin=0cm}        % Márgenes nulos para la primera página
    
    
    % Encabezado
    \fancyhead[L]{\helv #3}
    \fancyhead[R]{\helv \nouppercase{\leftmark}}
    
    
    \pagecolor{#4}        % Color de fondo para la portada
    
    \begin{figure}[p]
        \centering
        \transparent{0.3}           % Opacidad del 30% para la imagen
        
        \includegraphics[width=\paperwidth, keepaspectratio]{../../_assets/#1}
    
        \begin{tikzpicture}[remember picture, overlay]
            \node[anchor=north west, text=white, opacity=1, font=\fontsize{60}{90}\selectfont\bfseries\sffamily, align=left] at (#5, #6) {#2};
            
            \node[anchor=south east, text=white, opacity=1, font=\fontsize{12}{18}\selectfont\sffamily, align=right] at (9.7, 3) {\href{https://losdeldgiim.github.io/}{\textbf{Los Del DGIIM}, \texttt{\footnotesize losdeldgiim.github.io}}};
            
            \node[anchor=south east, text=white, opacity=1, font=\fontsize{12}{15}\selectfont\sffamily, align=right] at (9.7, 1.8) {Doble Grado en Ingeniería Informática y Matemáticas\\Universidad de Granada};
        \end{tikzpicture}
    \end{figure}
    
    
    \restoregeometry        % Restaurar márgenes normales para las páginas subsiguientes
    \nopagecolor      % Restaurar el color de página
    
    
    \newpage
    \thispagestyle{empty}               % Sin encabezado ni pie de página
    \begin{tikzpicture}[remember picture, overlay]
        \node[anchor=south west, inner sep=3cm] at (current page.south west) {
            \begin{minipage}{0.5\paperwidth}
                \href{https://creativecommons.org/licenses/by-nc-nd/4.0/}{
                    \includegraphics[height=2cm]{../../_assets/Licencia.png}
                }\vspace{1cm}\\
                Esta obra está bajo una
                \href{https://creativecommons.org/licenses/by-nc-nd/4.0/}{
                    Licencia Creative Commons Atribución-NoComercial-SinDerivadas 4.0 Internacional (CC BY-NC-ND 4.0).
                }\\
    
                Eres libre de compartir y redistribuir el contenido de esta obra en cualquier medio o formato, siempre y cuando des el crédito adecuado a los autores originales y no persigas fines comerciales. 
            \end{minipage}
        };
    \end{tikzpicture}
    
    
    
    % 1. Foto de fondo
    % 2. Título
    % 3. Encabezado Izquierdo
    % 4. Color de fondo
    % 5. Coord x del titulo
    % 6. Coord y del titulo
    % 7. Fecha
    % 8. Autor


}


\newcommand{\portadaBaseFotoDif}[8]{

    % Tiene la portada principal y la licencia Creative Commons
    
    % 1. Foto de fondo
    % 2. Título
    % 3. Encabezado Izquierdo
    % 4. Color de fondo
    % 5. Coord x del titulo
    % 6. Coord y del titulo
    % 7. Fecha
    % 8. Autor    
    
    \thispagestyle{empty}               % Sin encabezado ni pie de página
    \newgeometry{margin=0cm}        % Márgenes nulos para la primera página
    
    
    % Encabezado
    \fancyhead[L]{\helv #3}
    \fancyhead[R]{\helv \nouppercase{\leftmark}}
    
    
    \pagecolor{#4}        % Color de fondo para la portada
    
    \begin{figure}[p]
        \centering
        \transparent{0.3}           % Opacidad del 30% para la imagen
        
        \includegraphics[width=\paperwidth, keepaspectratio]{#1}
    
        \begin{tikzpicture}[remember picture, overlay]
            \node[anchor=north west, text=white, opacity=1, font=\fontsize{60}{90}\selectfont\bfseries\sffamily, align=left] at (#5, #6) {#2};
            
            \node[anchor=south east, text=white, opacity=1, font=\fontsize{12}{18}\selectfont\sffamily, align=right] at (9.7, 3) {\href{https://losdeldgiim.github.io/}{\textbf{Los Del DGIIM}, \texttt{\footnotesize losdeldgiim.github.io}}};
            
            \node[anchor=south east, text=white, opacity=1, font=\fontsize{12}{15}\selectfont\sffamily, align=right] at (9.7, 1.8) {Doble Grado en Ingeniería Informática y Matemáticas\\Universidad de Granada};
        \end{tikzpicture}
    \end{figure}
    
    
    \restoregeometry        % Restaurar márgenes normales para las páginas subsiguientes
    \nopagecolor      % Restaurar el color de página
    
    
    \newpage
    \thispagestyle{empty}               % Sin encabezado ni pie de página
    \begin{tikzpicture}[remember picture, overlay]
        \node[anchor=south west, inner sep=3cm] at (current page.south west) {
            \begin{minipage}{0.5\paperwidth}
                %\href{https://creativecommons.org/licenses/by-nc-nd/4.0/}{
                %    \includegraphics[height=2cm]{../../_assets/Licencia.png}
                %}\vspace{1cm}\\
                Esta obra está bajo una
                \href{https://creativecommons.org/licenses/by-nc-nd/4.0/}{
                    Licencia Creative Commons Atribución-NoComercial-SinDerivadas 4.0 Internacional (CC BY-NC-ND 4.0).
                }\\
    
                Eres libre de compartir y redistribuir el contenido de esta obra en cualquier medio o formato, siempre y cuando des el crédito adecuado a los autores originales y no persigas fines comerciales. 
            \end{minipage}
        };
    \end{tikzpicture}
    
    
    
    % 1. Foto de fondo
    % 2. Título
    % 3. Encabezado Izquierdo
    % 4. Color de fondo
    % 5. Coord x del titulo
    % 6. Coord y del titulo
    % 7. Fecha
    % 8. Autor


}


\newcommand{\portadaBook}[8]{

    % 1. Foto de fondo
    % 2. Título
    % 3. Encabezado Izquierdo
    % 4. Color de fondo
    % 5. Coord x del titulo
    % 6. Coord y del titulo
    % 7. Fecha
    % 8. Autor

    % Personaliza el formato del título
    \pretitle{\begin{center}\bfseries\fontsize{42}{56}\selectfont}
    \posttitle{\par\end{center}\vspace{2em}}
    
    % Personaliza el formato del autor
    \preauthor{\begin{center}\Large}
    \postauthor{\par\end{center}\vfill}
    
    % Personaliza el formato de la fecha
    \predate{\begin{center}\huge}
    \postdate{\par\end{center}\vspace{2em}}
    
    \title{#2}
    \author{\href{https://losdeldgiim.github.io/}{Los Del DGIIM, \texttt{\large losdeldgiim.github.io}}
    \\ \vspace{0.5cm}#8}
    \date{Granada, #7}
    \maketitle
    
    \tableofcontents
}




\newcommand{\portadaArticle}[8]{

    % 1. Foto de fondo
    % 2. Título
    % 3. Encabezado Izquierdo
    % 4. Color de fondo
    % 5. Coord x del titulo
    % 6. Coord y del titulo
    % 7. Fecha
    % 8. Autor

    % Personaliza el formato del título
    \pretitle{\begin{center}\bfseries\fontsize{42}{56}\selectfont}
    \posttitle{\par\end{center}\vspace{2em}}
    
    % Personaliza el formato del autor
    \preauthor{\begin{center}\Large}
    \postauthor{\par\end{center}\vspace{3em}}
    
    % Personaliza el formato de la fecha
    \predate{\begin{center}\huge}
    \postdate{\par\end{center}\vspace{5em}}
    
    \title{#2}
    \author{\href{https://losdeldgiim.github.io/}{Los Del DGIIM, \texttt{\large losdeldgiim.github.io}}
    \\ \vspace{0.5cm}#8}
    \date{Granada, #7}
    \thispagestyle{empty}               % Sin encabezado ni pie de página
    \maketitle
    \vfill
}
	\portadaExamen{ffccA4.jpg}{MN I\\Examen VIII}{Métodos Numéricos I. Examen VIII}{MidnightBlue}{-8}{28}{2025}{Roxana Acedo Parra}
	
	\begin{description}
		\item[Asignatura] Métodos Numéricos I.
		\item[Curso Académico] 2024-25.
		\item[Grado] Doble Grado en Ingeniería Informática y Matemáticas.
		\item[Grupo] Único.
		\item[Profesor] Juan José Nieto Muñoz.
		\item[Fecha] 10 de junio de 2025.
		\item[Duración] 3 horas.
		\item[Descripción] Convocatoria Ordinaria.\\
		Los que se examinaban de la parte uno, debían hacer los ejercicios 1 (sobre 2), 2 (sobre 4), 3 (sobre 4) y 4 (sobre 4). Mientras que los que se examinaban de la segunda, los ejercicios 5 (sobre 2), 6 (sobre 4), 7 (sobre 4) y 8 (sobre 4). Por otro lado, los que se examinaban de ambas hicieron el  1 (sobre 2), 2 (sobre 4), 3 (sobre 4), 5 (sobre 2), 6 (sobre 4) y 7 (sobre 4).
	\end{description}
	\newpage
	
	\begin{ejercicio}
		Determina razonadamente la veracidad o falsedad de las siguientes afirmaciones:
		\begin{enumerate}[label=\alph*)]
			\item Si dos polinomios de grado exactamente 5 interpolan los mismos 5 datos, entonces esos polinomios han de ser iguales.
			
			\item Si en un espacio vectorial de funciones prehilbertiano $(V, \langle \cdot, \cdot \rangle)$ tenemos una función $f$ y su mejor aproximación $u_f$ en un cierto subespacio $H$ de dimensión finita, entonces se verifica:
			$$ d(f, H)^2 = \text{(distancia de } f \text{ a } H)^2 = \langle f - u_f, f - u_f \rangle^2 = \|f\|^2 - \|u_f\|^2.$$
		\end{enumerate}
	\end{ejercicio}
	
	\begin{ejercicio}
		Considera el espacio vectorial $V = C([0,1])$ con el producto escalar con peso siguiente:
		$$ \langle f, g \rangle = \int_0^1 f(x) g(x) x \, dx,$$
		y el subespacio $ H = \{ u(x) \in V : u(x) = a + b x^3 + c(x^3 - x) \mid a, b, c \in \mathbb{R} \}.$
		\begin{enumerate}[label=\alph*)]
			\item Calcula una base del subespacio $H$ e indica si es ortogonal.
			
			\item Dada la función $f(x) = x$, calcula su mejor aproximación $u_f$ en $H$.
		\end{enumerate}
	\end{ejercicio}
	
	\begin{ejercicio}
		Considera la función $f(x) = e^{x^2}$ y los nodos $X = \{-1, 0, 1\}$.
		
		\begin{enumerate}[label=\alph*)]
			\item Calcula un polinomio $p(x) \in \bb{P}_2[x]$ que interpole a $f$ en los nodos dados sin usar el método de los coeficientes indeterminados.
			
			\item Sabiendo que $|f'''(x)| < 4$ en el intervalo $[-1, 1]$, da una estimación lo más ajustada posible del error cometido en el intervalo.
			
			\item Usa $p(x)$ para aproximar la integral $\displaystyle \int_{-1}^{1} f(x) \, dx.$
			
			\item Estima el error cometido al calcular la integral anterior.
		\end{enumerate}
	\end{ejercicio}
	
	\begin{ejercicio}
		Dado el siguiente spline:
		$$ s(x)=
		\begin{cases}
			3+x-9x^2 & \text{si } 0 \leq x \leq 1, \\
			a + b(x - 1) + c(x - 1)^2 + x^3 & \text{si } 1 < x \leq 2.
		\end{cases}$$
		
		\begin{enumerate}[label=\alph*)]
			\item Determina los valores de $a$, $b$ y $c$ para que $s(x)$ sea un spline cúbico de clase 2.
			
			\item Determina qué problema de interpolación de tipo Lagrange en los nodos $X = \{0, 1, 2\}$ resuelve $s(x)$.
			
			\item Justifica si $s(x)$ es el único spline cúbico de clase $2$ que resuelve dicho problema.
		\end{enumerate}
	\end{ejercicio}
	
	\begin{ejercicio}
		Determina razonadamente la veracidad o falsedad de las siguientes afirmaciones:
		\begin{enumerate}[label=\alph*)]
			\item Toda matriz $A$ para la cual sea posible encontrar una matriz $L$ triangular inferior tal que se verifique $A = LL^T$ ha de ser simétrica y regular.
			
			\item Toda matriz estrictamente diagonal dominante (por filas) posee un valor propio que es simple y 
			mayor en módulo que todos los demás valores propios.
		\end{enumerate}
	\end{ejercicio}
	
	\begin{ejercicio}
		Al aplicar el método de Jacobi a un SEL $2 \times 2$ se han obtenido las siguientes iteraciones:
		$$ \begin{array}{c|c|c|c|c|c|c}
			k & 0 & 1 & 2 & 3 & 4 & \ldots \\
			\hline
			x_k & 0 & \nicefrac{3}{2} & \nicefrac{7}{5} & \nicefrac{37}{20} & \nicefrac{91}{50} & \ldots \\
			\hline
			y_k & 0 & \nicefrac{-1}{5} & \nicefrac{7}{10} & \nicefrac{16}{25} & \nicefrac{91}{100} & \ldots 
		\end{array}$$
		
		\begin{enumerate}[label=\alph*)]
			\item Determina un SEL del cual provengan estas iteraciones.
			
			\item Decide si el SEL encontrado es el único que corresponde a tales iteraciones.
			
			\item Justifica si el método convergerá o no.
		\end{enumerate}
	\end{ejercicio}
	
	\begin{ejercicio}
		Considera el sistema de ecuaciones:
		$$ \begin{pmatrix}
			1 & -1 & 3 \\
			-1 & 2 & -3 \\
			-2 & 4 & -3 \\
		\end{pmatrix} 
		\begin{pmatrix}
			x \\
			y \\
			z
		\end{pmatrix} = 
		\begin{pmatrix}
			3 \\
			-2 \\
			-1
		\end{pmatrix}$$
		
		\begin{enumerate}[label=\alph*)] 
			\item Demuestra que la matriz de coeficientes del sistema admite una descomposición de tipo $LU$.
			
			\item Halla la descomposición $LU$ que verifica $u_{i,i} = i$, para $i = 1, 2, 3$.
			
			\item Utiliza dicha factorización para hallar la solución del SEL.
			
			\item Aparte de la factorización anterior, ¿es posible encontrar (otra) matriz $L$ tal que $A = L L^T$?
		\end{enumerate}
	\end{ejercicio}
	
	\begin{ejercicio}
		Una señora dejó junto a la ventana los resultados de su análisis de sangre y con el sol se borraron los correspondientes al Hematocrito (medido en \%), la Hemoglobina (medida en gramos/decilitro) y los Hematíes (medidos en millones/mililitro). Recordaba sin embargo que el Hematocrito triplicaba la Hemoglobina y sextuplicaba los Hematíes, y curiosamente sabía que la cantidad de Hematocrito menos 33 era igual a la mitad de los Hematíes.
		
		\begin{enumerate}[label=\alph*)] 
			\item Determina el SEL que han de resolver estas tres cantidades.
			
			\item Ordénalo para que no se pueda aplicar el método de Jacobi.
			
			\item Ordénalo de nuevo para que se pueda aplicar el método de Gauss–Seidel y sea convergente.
			
			\item Teniendo en cuenta que los Hematíes han de estar entre 3 y 5 millones por mililitro, determina si la
			señora los tiene bajos, altos o dentro de lo normal.
		\end{enumerate}
	\end{ejercicio}
	
	\newpage
	\setcounter{ejercicio}{0}
	% 1
	\begin{ejercicio}
		Determina razonadamente la veracidad o falsedad de las siguientes afirmaciones:
		\begin{enumerate}[label=\alph*)]
			\item Si dos polinomios de grado exactamente 5 interpolan los mismos 5 datos, entonces esos polinomios han de ser iguales. \\
			
				\textbf{Solución. }\fbox{Falso.} El espacio $\bb{P}_5[x]$ tiene dimensión 6; por tanto, al interpolar solo 5 datos sabemos que hay existencia pero no unicidad. Por tanto, damos un contraejemplo sencillo (interpolando ceros).\\
				$$p_1(x) = x(x - 1)(x - 2)(x - 3)(x - 4), \quad \text{ y} \quad p_2(x) = 43 \cdot p_1(x), $$
				$$\quad \textit{que interpolan los puntos } \quad
				\begin{array}{c|ccccc}
					x_i & 0 & 1 & 2 & 3 & 4 \\
					\hline
					y_i & 0 & 0 & 0 & 0 & 0 \\
				\end{array}$$
				
			\item Si en un espacio vectorial de funciones prehilbertiano $(V, \langle \cdot, \cdot \rangle)$ tenemos una función $f$ y su mejor aproximación $u_f$ en un cierto subespacio $H$ de dimensión finita, entonces se verifica:
			$$ d(f, H)^2 = \text{(distancia de } f \text{ a } H)^2 = \langle f - u_f, f - u_f \rangle^2 = \|f\|^2 - \|u_f\|^2.$$
			
				\textbf{Solución. }\fbox{Falso.} Sobra un cuadrado:
				$$ d(f, H)^2 = \langle f - u_f, f - u_f \rangle = \|f\|^2 - \|u_f\|^2.$$
		\end{enumerate}
	\end{ejercicio}
	
	% 2
	\begin{ejercicio}
		Considera el espacio vectorial $V = C([0,1])$ con el producto escalar con peso siguiente:
		$$ \langle f, g \rangle = \int_0^1 f(x) g(x) x \, dx,$$
		y el subespacio $ H = \{ u(x) \in V : u(x) = a + b x^3 + c(x^3 - x) \mid a, b, c \in \mathbb{R} \}.$
		\begin{enumerate}[label=\alph*)]
			\item Calcula una base del subespacio $H$ e indica si es ortogonal. \\
			
				Por la propia definición de $H$, es claro que $u_1(x) = 1, \quad u_2(x) = x^3, \quad u_3(x) = x^3 - x,$ 
				constituyen una base (generan $H$ y son polinomios linealmente independientes). Veamos que no son ortogonales haciendo solo un producto escalar, aunque en realidad ninguno de los $\langle u_i, u_j \rangle$ sale $0$.
				
				$$ \langle u_1, u_2 \rangle = \int_0^1 1 \cdot x^3 \cdot x \, dx 
				= \int_0^1 x^4 \, dx 
				= \left[ \frac{x^5}{5} \right]_{x=0}^{x=1}
				= \frac{1}{5}.$$
				
			\item Dada la función $f(x) = x$, calcula su mejor aproximación $u_f$ en $H$. \\
			
				Podríamos aplicar el teorema principal de aproximación visto en clase, que dice que la mejor aproximación es la única función $u_f(x) = a u_1(x) + b u_2(x) + c u_3(x)$ cuyos coeficientes resuelven el sistema de Gramm:
				
				$$\begin{pmatrix}
					\langle u_1, u_1 \rangle & \langle u_1, u_2 \rangle & \langle u_1, u_3 \rangle \\
					\langle u_2, u_1 \rangle & \langle u_2, u_2 \rangle & \langle u_2, u_3 \rangle \\
					\langle u_3, u_1 \rangle & \langle u_3, u_2 \rangle & \langle u_3, u_3 \rangle
				\end{pmatrix}
				\begin{pmatrix}
					a \\ 
					b \\ 
					c
				\end{pmatrix}
				=
				\begin{pmatrix}
					\langle f, u_1 \rangle \\
					\langle f, u_2 \rangle \\
					\langle f, u_3 \rangle
				\end{pmatrix}$$
				
				Y todos estos productos son integrales de polinomios fáciles de calcular, aunque llevan un ratito... Pero, en este caso, simplemente observamos que $f(x) = x = x - x^3 + x^3 = u_3(x) + u_2(x),$ y, por lo tanto, $f$ está en $H$ y ¡ella misma es su mejor aproximación!
		\end{enumerate}
	\end{ejercicio}
	
	% 3
	\begin{ejercicio}
		Considera la función $f(x) = e^{x^2}$ y los nodos $X = \{-1, 0, 1\}$.
		
		\begin{enumerate}[label=\alph*)]
			\item Calcula un polinomio $p(x) \in \bb{P}_2[x]$ que interpole a $f$ en los nodos dados sin usar el método de los coeficientes indeterminados. \\
			
				Los datos a interpolar, a partir de $f$, son:
				
				$$
				\begin{array}{c|c|c|c}
					x_i & -1 & 0 & 1 \\
					\hline
					f(x_i) & \nicefrac{1}{e} & 1 & \nicefrac{1}{e}
				\end{array}
				$$
				
				Por lo que el polinomio interpolador es (usamos, por ejemplo, la fórmula de Lagrange):
				\begin{align*}
					p(x) &= \frac{1}{e} \cdot \frac{(x)(x - 1)}{(-1)(-1 - 1)} + 1 \cdot \frac{(x+1)(x - 1)}{(0+1)(0 - 1)} + \frac{1}{e} \cdot \frac{(x+1)(x)}{(1+1)(1)} \\
					&= \frac{1}{2e}(x^2 - x) - (x^2 - 1) + \frac{1}{2e}(x^2 + x) \\
					&= \left( \frac{1}{2e} + \frac{1}{2e} \right) x^2 + \left( -\frac{1}{2e} + \frac{1}{2e} \right) x - x^2 + 1\\
					&=\left( \frac{1}{e} - 1 \right) x^2 + 1\\
					&\approx 1 -0.6321x^2 
				\end{align*}
			
			\item Sabiendo que $|f'''(x)| < 4$ en el intervalo $[-1, 1]$, da una estimación lo más ajustada posible del error cometido en el intervalo.
			
				Aquí usamos el teorema que caracteriza el error de interpolación mediante una derivada de $f$ junto con la cota de las derivadas que nos proporciona el enunciado, obtenemos:
				
				$$ f(x) - p(x) := E(x) = \frac{f'''(\xi_x)}{3!} \cdot \Pi(x)$$
				$$ \Rightarrow |f(x) - p(x)| \leq \frac{4}{6} \cdot |x(x^2 - 1)| \leq \frac{4}{6} \cdot \frac{2}{3\sqrt{3}} \approx 0.2566$$
				
				donde hemos calculado el siguiente máximo valor absoluto:
				
				$$\max_{x \in [-1,1]} |x(x^2 - 1)| = \frac{2}{3\sqrt{3}}$$
			
			\item Usa $p(x)$ para aproximar la integral $\displaystyle \int_{-1}^{1} f(x) \, dx.$
				
					$$ \int_{-1}^{1} f(x) \, dx \approx \int_{-1}^{1} (\nicefrac{1}{e}) -1)x^2 +1\, dx = \left[\frac{1-e}{3e}x^3+x\right]_{-1}^{1}=2\frac{1-e}{3e}+2\approx 1.5132.$$
					
			\item Estima el error cometido al calcular la integral anterior.
			
				$$ \left| \int_{-1}^{1} f(x) \, dx -1.579 \right| = \left| \int_{-1}^{1} \left(f(x)-p(x)\right) \, dx \right|
				\leq \int_{-1}^{1} |E(x)| \, dx$$
				$$ \leq \int_{-1}^{1} \frac{4}{9\sqrt{3}} \, dx = \frac{8}{9\sqrt{3}} \approx 0.5132 $$
				
		\end{enumerate}
	\end{ejercicio}
	
	% 4
	\begin{ejercicio}
		Dado el siguiente spline:
		$$ s(x)=
		\begin{cases}
			3+x-9x^2 & \text{si } 0 \leq x \leq 1, \\
			a + b(x - 1) + c(x - 1)^2 + x^3 & \text{si } 1 < x \leq 2.
		\end{cases}$$
		
		\begin{enumerate}[label=\alph*)]
			\item Determina los valores de $a$, $b$ y $c$ para que $s(x)$ sea un spline cúbico de clase 2. \\
			
				Sólo hemos de recordar que un spline cúbico de clase 2 es una función a trozos, que en cada trozo es un
				polinomio de grado 3, como mucho, y que es 2 veces derivable (lo que se chequea sólo en las “juntas” porque
				en el resto de puntos, por ser polinómica, ya es derivable todas las veces que se quiera). En este caso, sólo
				hemos de imponer que los dos trozos a la izquierda y derecha de $x = 1$ (la única junta) se “peguen bien”
				tanto las funciones, como sus derivadas primera y segunda.
				
				Para simplificar ponemos nombres a las expresiones en cada trozo:
				$$ s_1(x):=3+x-9x^2 \quad \text{ y } \quad s_2(x):= a + b(x - 1) + c(x - 1)^2 + x^3 $$
				
				Comenzamos imponiendo continuidad (que se peguen bien las funciones en cada trozo).
				$$ s_1(1) = s_2(1), \quad -5 = a + 0 + 0 + 1\iff \boxed{a = -6}.$$
				
				Continuamos imponiendo derivabilidad (que se peguen bien las derivadas de las funciones en cada trozo)
				$$s_1'(1) = s_2'(1) \iff  (1 - 18x) \big|_{x=1} = \left( b + 2c(x - 1) + 3x^2 \right) \big|_{x=1}$$
				$$ \iff -17 = b + 0 + 3 \iff \boxed{b = -20}.$$
				
				Y concluimos imponiendo clase 2 (que se peguen bien las segundas derivadas de las funciones en cada trozo)
				$$s_1''(1) = s_2''(1) \iff (-18) \big|_{x=1} = \left( 2c + 6x \right) \big|_{x=1} \iff -18 = 2c + 6\iff\boxed{ c = -12}.$$
				
				 Por lo tanto, basta que $a = -6$, $b = -20$ y $c = -12$ para que s(x) sea un spline cúbico de clase 2.
			
			\item Determina qué problema de interpolación de tipo Lagrange en los nodos $X = \{0, 1, 2\}$ resuelve $s(x)$. \\
			
				Obviamente los datos tipo Lagrange que interpola son $(0, s(0))$, $(1, s(1))$ y $(2, s(2))$; en este caso:  
				$$s(0) = s_1(0) = 3, \quad s(1) = s_1(1) = 3 + 1 - 9 = -5,$$
				$$ s(2) = s_2(2) = -6 - 20(2 - 1) - 12(2 - 1)^2 + 2^3 = -30.$$
				
				Si no se especificase “de tipo Lagrange”, se pueden proponer montones de problemas de interpolación de  
				los cuales es solución nuestro spline; pero entonces, habría que ser consecuentes en el siguiente apartado...
			
			\item Justifica si $s(x)$ es el único spline cúbico de clase $2$ que resuelve dicho problema.
			
				No, no es el único. Sabemos que el espacio de splines cúbicos de clase dos, en tres nodos, tiene
				dimensión 5 y por tanto, hay infinitos splines de este espacio resolviendo las tres condiciones pedidas. Como
				ya sabemos, hacen falta dos condiciones adicionales para que quede unívocamente determinado.
		\end{enumerate}
	\end{ejercicio}
	
	% 5
	\begin{ejercicio}
		Determina razonadamente la veracidad o falsedad de las siguientes afirmaciones:
		\begin{enumerate}[label=\alph*)]
			\item Toda matriz $A$ para la cual sea posible encontrar una matriz $L$ triangular inferior tal que se verifique $A = LL^T$ ha de ser simétrica y regular.\\
			
				\textbf{Solución. }\fbox{Falso.} La regularidad no es necesaria; contraejemplo: $A=0$ y $L=0$.
			
			\item Toda matriz estrictamente diagonal dominante (por filas) posee un valor propio que es simple y 
			mayor en módulo que todos los demás valores propios. \\
			
				\textbf{Solución. }\fbox{Falso.} Ser EDD y tener valor propio dominante no tienen nada que ver. Contraejemplo: $A=I$.
		\end{enumerate}
	\end{ejercicio}
	
	% 6
	\begin{ejercicio}
		Al aplicar el método de Jacobi a un SEL $2 \times 2$ se han obtenido las siguientes iteraciones:
		$$ \begin{array}{c|c|c|c|c|c|c}
			k & 0 & 1 & 2 & 3 & 4 & \ldots \\
			\hline
			x_k & 0 & \nicefrac{3}{2} & \nicefrac{7}{5} & \nicefrac{37}{20} & \nicefrac{91}{50} & \ldots \\
			\hline
			y_k & 0 & \nicefrac{-1}{5} & \nicefrac{7}{10} & \nicefrac{16}{25} & \nicefrac{91}{100} & \ldots 
		\end{array}$$
		
		\begin{enumerate}[label=\alph*)]
			\item Determina un SEL del cual provengan estas iteraciones. \\
			
				Como sabemos, al aplicar el método de Jacobi a un sistema $2 \times 2$ nos queda un método de la forma
				
				$$\begin{pmatrix}
					x \\
					y
				\end{pmatrix}_{k+1}
				=
				\begin{pmatrix}
					0 & a \\
					c & 0
				\end{pmatrix}
				\begin{pmatrix}
					x \\
					y
				\end{pmatrix}_k
				+
				\begin{pmatrix}
					b \\
					d
				\end{pmatrix}, \quad \text{ó} \quad 
				\begin{cases}
					x_{k+1} = a \cdot y_k + b, \\
					y_{k+1} = c \cdot x_k + d,
				\end{cases}$$
				
				por lo que calcularemos $a$, $b$, $c$ y $d$. Usando el paso de $k=0$ a $k=1$ obtenemos
				
				$$\begin{cases}
					\nicefrac{3}{2} = 0 \cdot a + b, \\
					\nicefrac{-1}{5} = 0 \cdot c + d,
				\end{cases} \Rightarrow \quad \boxed{b = \nicefrac{3}{2}}\quad \text{ y } \quad \boxed{d = \nicefrac{-1}{5}}.$$
				
				Usando ahora el paso de $k=1$ a $k=2$, junto con los valores calculados, obtenemos
				
				$$\begin{cases}
					\nicefrac{7}{5} = \nicefrac{-1}{5} \cdot a + \nicefrac{3}{2}, \\
					\nicefrac{7}{10} = \nicefrac{3}{2} \cdot c - \nicefrac{1}{5},
				\end{cases} \Rightarrow \quad \boxed{a = \nicefrac{1}{2}} \quad \text{ y } \quad \boxed{c = \nicefrac{3}{5}}.$$
				
				El resto de iteraciones no son necesarias, pero podemos usarlas para verificar que no nos hemos equivocado. Recordando ahora que las ecuaciones de Jacobi se calculan despejando cada incógnita de su correspondiente ecuación, deducimos un SEL:
				
				$$\begin{cases}
					x_{k+1} = \nicefrac{1}{2} \cdot y^k + \nicefrac{3}{2}, \\
					y_{k+1} = \nicefrac{3}{5} \cdot x^k - \nicefrac{1}{5},
				\end{cases} \Rightarrow 
				\begin{cases}
					2x - y = 3, \\
					-3x + 5y = -1.
				\end{cases} \quad \text{ó} \quad
				\begin{pmatrix}
					2 & -1 \\
					-3 & 5
				\end{pmatrix}
				\begin{pmatrix}
					x \\
					y
				\end{pmatrix}
				=
				\begin{pmatrix}
					3 \\
					-1
				\end{pmatrix}$$
				
			
			\item Decide si el SEL encontrado es el único que corresponde a tales iteraciones. \\
			
				El sistema, obviamente, no es único (hablamos del sistema, no de la solución), ya que si multiplicamos
				cualquier ecuación del sistema por una constante no nula (cosa que, de hecho, ya hemos hecho para simplificar
				denominadores), obtendríamos el mismo método de Jacobi.
			
			\item Justifica si el método convergerá o no. \\
			
				Como tenemos 
				$$B_J = 
				\begin{pmatrix}
					0 & 0.5 \\
					0.6 & 0
				\end{pmatrix},$$
				se puede calcular que $\|B_J\|_1 = \|B_J\|_\infty = 0.6 < 1$, condición suficiente
				para la convergencia. También, en este caso, la matriz del sistema obtenido es claramente EDD, y esto es
				otra condición suficiente.
				
		\end{enumerate}
	\end{ejercicio}
	
	% 7
	\begin{ejercicio}
		Considera el sistema de ecuaciones:
		$$ \begin{pmatrix}
			1 & -1 & 3 \\
			-1 & 2 & -3 \\
			-2 & 4 & -3 \\
		\end{pmatrix} 
		\begin{pmatrix}
			x \\
			y \\
			z
		\end{pmatrix} = 
		\begin{pmatrix}
			3 \\
			-2 \\
			-1
		\end{pmatrix}$$
		
		\begin{enumerate}[label=\alph*)] 
			\item Demuestra que la matriz de coeficientes del sistema admite una descomposición de tipo $LU$. \\
			
				En rigor, cuando resolvamos 7.b) habremos respondido al 7.a)... No obstante, se puede verificar a priori
				que los determinantes principales de $A$ son no nulos, lo que garantiza la descomposición $LU$.
			
			\item Halla la descomposición $LU$ que verifica $u_{i,i} = i$, para $i = 1, 2, 3$. \\
			
				Basta seguir los pasos de clase ordenadamente, colocando en primer lugar la diagonal de $U$ que nos
				han prescrito: $u_{1,1} = \boxed{1}$, $u_{2,2} = \boxed{2}$ y $u_{3,3} = \boxed{3}$, obteniendo:
				
				$$A =
				\begin{pmatrix}
					1 & -1 & 3 \\
					-1 & 2 & -3 \\
					-2 & 4 & -3
				\end{pmatrix}
				=
				\begin{pmatrix}
					1 & 0 & 0 \\
					-1 & 0.5 & 0 \\
					-2 & 1 & 1
				\end{pmatrix}
				\begin{pmatrix}
					\boxed{1} & -1 & 3 \\
					0 & \boxed{2} & 0 \\
					0 & 0 & \boxed{3}
				\end{pmatrix}
				= LU.$$
				
			
			\item Utiliza dicha factorización para hallar la solución del SEL. \\
			
				Basta recordar el orden de resolución: primero $L y = b$ y luego $U x = y$. Lo hacemos:
				$$\left[ \begin{pmatrix}
					1 & 0 & 0 \\
					-1 & 0.5 & 0 \\
					-2 & 1 & 1
				\end{pmatrix}
				y =
				\begin{pmatrix}
					3 \\
					-2 \\
					-1
				\end{pmatrix}
				\Rightarrow
				y =
				\begin{pmatrix}
					3 \\
					2 \\
					3
				\end{pmatrix} \right] \Rightarrow \left[
				\begin{pmatrix}
					1 & -1 & 3 \\
					0 & 2 & 0 \\
					0 & 0 & 3
				\end{pmatrix}
				x =
				\begin{pmatrix}
					3 \\
					2 \\
					3
				\end{pmatrix}
				\Rightarrow
				x =
				\begin{pmatrix}
					1 \\
					1 \\
					1
				\end{pmatrix} \right].$$
				
			
			\item Aparte de la factorización anterior, ¿es posible encontrar (otra) matriz $L$ tal que $A = L L^T$? \\
			
				No, basta observar que, cuando esto ocurre, $A = L L^T$ resulta ser simétrica, pero nuestra matriz $A$ no
				lo es, puesto que $a_{1,3} = 3$ y $a_{3,1} = 2$ no son iguales.
		\end{enumerate}
	\end{ejercicio}
	
	% 8
	\begin{ejercicio}
		Una señora dejó junto a la ventana los resultados de su análisis de sangre y con el sol se borraron los correspondientes al Hematocrito (medido en \%), la Hemoglobina (medida en gramos/decilitro) y los Hematíes (medidos en millones/mililitro). Recordaba sin embargo que el Hematocrito triplicaba la Hemoglobina y sextuplicaba los Hematíes, y curiosamente sabía que la cantidad de Hematocrito menos 33 era igual a la mitad de los Hematíes.
		
		\begin{enumerate}[label=\alph*)] 
			\item Determina el SEL que han de resolver estas tres cantidades. \\
			
				Para comenzar, ponemos nombres a las variables del problema: \\
				$x = \text{Hematocrito (medido en \%)}$; \\
				$y = \text{Hemoglobina (medida en gramos/decilitro)}$; \\
				$z = \text{Hematíes (medidos en millones/mililitro)}$; \\
				
				Y, trasladando los datos a fórmulas, obtenemos nuestro SEL:  
				– el Hematocrito triplicaba la Hemoglobina: $x = 3y$; \\
				– el Hematocrito sextuplicaba los Hematíes: $x = 6z$; \\
				– la cantidad de Hematocrito menos 33 era igual a la mitad de los Hematíes: $x - 33 = \dfrac{z}{2}$ \\
					
			\item Ordénalo para que no se pueda aplicar el método de Jacobi. \\
			
				Pasando todas las incógnitas a la izquierda, nos queda:
				$$ \begin{cases}
					x - 3y = 0 \\
					x - 6z = 0 \\
					x - \nicefrac{1}{2} z = 33
				\end{cases} \Rightarrow
				\begin{pmatrix}
					1 & -3 & 0 \\
					1 & 0 & -6 \\
					1 & 0 & -0.5
				\end{pmatrix} 
				\begin{pmatrix}
					x \\
					y \\
					z
				\end{pmatrix}
				\begin{pmatrix}
					0 \\
					0 \\
					33
				\end{pmatrix}$$
				
				Y, tal y como está, ya no se puede aplicar Jacobi, pues $a_{2,2} = 0$ no puede pasar dividiendo.
				
			\item Ordénalo de nuevo para que se pueda aplicar el método de Gauss–Seidel y sea convergente. \\
			
				Basta reordenar las ecuaciones (las filas de la matriz ampliada $(A|b)$) para que, primero, los elementos
				de la diagonal sean no nulos y se pueda aplicar y, segundo, buscando la convergencia del método; en este
				caso valen varias opciones, pero lo hacemos para que quede EDD, lo que produce convergencia:
				
					$$\begin{pmatrix}
						1 & -3 & 0 \\
						1 & 0 & -6 \\
						1 & 0 & -0.5
					\end{pmatrix} 
					\begin{pmatrix}
						x \\
						y \\
						z
					\end{pmatrix}
					\begin{pmatrix}
						0 \\
						0 \\
						33
					\end{pmatrix} \iff
					\begin{pmatrix}
					1 & 0 & -0.5 \\
					1 & -3 & 0 \\
					1 & 0 & -6 \\
					\end{pmatrix} 
					\begin{pmatrix}
					x \\
					y \\
					z
					\end{pmatrix}
					\begin{pmatrix}
					33 \\
					0 \\
					0
					\end{pmatrix}$$
			
			\item Teniendo en cuenta que los Hematíes han de estar entre 3 y 5 millones por mililitro, determina si la
			señora los tiene bajos, altos o dentro de lo normal. \\
			
				Resolviendo el sistema (en realidad, solo calculando $z$) obtenemos $z = 6$ y como han de estar entre $z = 3$ y $z = 5$, concluimos que están altos.
				
		\end{enumerate}
	\end{ejercicio}

\end{document}
