\section{Interpolación polinómica}\label{sec:Rel3.1}

\begin{ejercicio}
    Utilice el método más adecuado para calcular el polinomio $p(x)$ de grado mínimo que interpola los datos de la tabla
    \begin{equation*}
        \begin{array}{c|ccccc}
            x_i & -1 & 0 & 1 & 2 \\ \hline
            f_i & 2 & 1 & 2 & -7 \\
        \end{array}
    \end{equation*}

    \begin{enumerate}
        \item Utilice el algoritmo de Newton–Horner para calcular $p(3)$.

        Calculo en primer lugar el polinomio de interpolación mediante el método de Newton. La tabla de diferencias dividas es:
        \begin{equation*}
            \begin{array}{c|cccc}
                x_i & f[x_i] \\
                \\
                -1 & \textbf{2} \\
                && \textbf{-1}\\
                0 & 1 & & \textbf{1}\\
                && 1 &&\textbf{-2}\\
                1 & 2 && -5\\
                & & -9\\
                2 &-7
            \end{array}
        \end{equation*}
        Por tanto, el polinomio de interpolación es:
        \begin{equation*}
            p_3(x) = 2-(x+1) +(x+1)x -2(x+1)x(x-1)
        \end{equation*}

        Para evaluar, usamos el método de Newton-Horner:
        \begin{equation*}
            \begin{split}
                p_3(x) &= 2-(x+1) +(x+1)x -2(x+1)x(x-1) \\
                &= 2+(x+1)[-1+x[1-2(x-1)]]
            \end{split}
        \end{equation*}

        Por tanto, evaluando, $p_3(3) = -38$.
        

        \item ¿Qué término habrá que añadir al polinomio $p(x)$ para que el nuevo polinomio interpole también el dato $(3, 10)$?

        Ampliamos la tabla de diferencias divididas con una diagonal más:
        \begin{equation*}
            \begin{array}{c|ccccc}
                x_i & f[x_i] \\
                \\
                -1 & \textbf{2} \\
                && \textbf{-1}\\
                0 & 1 & & \textbf{1}\\
                && 1 &&\textbf{-2}\\
                1 & 2 && -5 && \textbf{2}\\
                & & -9 && 6\\
                2 &-7 && 13 \\
                && 17\\
                3 & 10
            \end{array}
        \end{equation*}
        
        Por tanto, el polinomio de interpolación es:
        \begin{equation*}
            p_4(x) = 2-(x+1) +(x+1)x -2(x+1)x(x-1) +2(x+1)x(x-1)(x-2)
        \end{equation*}

        Podemos ver que el último término será:
        \begin{equation*}
            f[x_0,\dots, x_n]\prod_{i=0}^{n-1}(x-x_i) = 2(x+1)x(x-1)(x-2)
        \end{equation*}
    \end{enumerate}
\end{ejercicio}

\begin{ejercicio}
    Dados los puntos:
    \begin{equation*}
        \begin{array}{c|ccccc}
            x_i & -1 & 0 & 4 & -2 \\ \hline
            f_i & 0 & 1 & 305 & -31 \\
        \end{array}
    \end{equation*}

    \begin{enumerate}
        \item Construya, usando el método de los coeficientes indeterminados, la fórmula de Lagrange y la fórmula de Newton, el polinomio que interpola a dichos puntos.\\

        Como hay cuatro puntos, el grado del polinomio de interpolación es 3.
        \begin{itemize}
            \item \underline{Método de Coeficientes Indeterminados}

            El polinomio quedará como $p_{3}(x) = a_0+a_1x + a_2x^2 + a_3x^3$.

            Las condiciones de interpolación son:
            \begin{equation*}
                \left. \begin{array}{l}
                    p_3(-1)=0 \\
                    p_3(0)=1 \\
                    p_3(4)=305 \\
                    p_3(-2)=-31 \\
                \end{array} \right\} \Longrightarrow
                \left. \begin{array}{rrrrrr}
                    a_0 & -a_1 & +a_2 & -a_3&=&0 \\
                    a_0 &&&&=&1 \\
                    a_0 & + 4a_1 &+16a_2 & +64a_3&=& 305 \\
                    a_0 &  -2a_1 &+4a_2 & -8a_3&=& -31 \\
                \end{array} \right\} \Longrightarrow
                \left. \begin{array}{rcr}
                    a_0&=&1 \\
                    a_1&=&-4 \\
                    a_2&=&0 \\
                    a_3&=&5 \\
                \end{array} \right.
            \end{equation*}
            Por tanto, el polinomio queda: $p_{3}(x) = 1-4x + 5x^3$

            \item \underline{Método de Lagrange}
            
            Calculo los polinomios básicos de Lagrange $\ell_i$
            \begin{equation*}
                \ell_i(x) = \prod_{\substack{k=0\\k\neq i}}^n \frac{x-x_k}{x_i-x_k}
            \end{equation*}

            Por tanto,
            \begin{equation*}
                \ell_0(x) = \prod_{\substack{k=0\\k\neq i}}^n \frac{(x-0)(x-4)(x-(-2))}{(-1-0)(-1-4)(-1-(-2))} = \frac{x(x-4)(x+2)}{5}
            \end{equation*}
            \begin{equation*}
                \ell_1(x) = \frac{(x+1)(x-4)(x+2)}{-8}
                \quad
                \ell_2(x) = \frac{(x+1)x(x+2)}{120}
                \quad
                \ell_3(x) = \frac{(x+1)x(x-4)}{-12}
            \end{equation*}

            Por tanto, el polinomio de interpolación queda:
            \begin{equation*}
                \begin{split}
                    p_3(x) &= f_0 \ell_0(x) + f_1 \ell_1(x) + f_2 \ell_2(x) + f_3 \ell_3(x) \\
                    &= 0\cdot \frac{x(x-4)(x+2)}{5} +1\cdot \frac{(x+1)(x-4)(x+2)}{-8}
                    +\\&
                    \qquad +305\cdot \frac{(x+1)x(x+2)}{120} -31\cdot \frac{(x+1)x(x-4)}{-12}
                \end{split}
            \end{equation*}

            \item \underline{Método de Newton}

            Calculo en primer lugar la tabla de diferencias divididas:
            \begin{equation*}
                \begin{array}{c|cccc}
                    x_i & f[x_i] \\
                    \\
                    -1 & \textbf{0} \\
                    && \textbf{1}\\
                    0 & 1 & & \textbf{15}\\
                    && 76&&\textbf{5}\\
                    4 & 305 && 10\\
                    & & 56\\
                    -2 &-31
                \end{array}
            \end{equation*}
            Por tanto, el polinomio de interpolación es
            \begin{equation*}
                \begin{split}
                    p_3(x) &= 0+1(x-(-1))+15(x-(-1))(x-0) +5(x-(-1))(x-0)(x-4) \\
                    &= 0+(x+1)+15(x+1)x + 5(x+1)x(x-4) \\
                    &= 0+(x+1)[1+x[15+5(x-4)]]
                \end{split}
            \end{equation*}
        \end{itemize}

        \item ¿Sigue siendo válido el mismo polinomio si agregamos el punto $(1, 2)$? ¿Y si fuera el punto $(3, 0)$?

        Evaluando en cualquiera de los polinomios del apartado anterior,
        \begin{equation*}
            p_3(1)=2
        \end{equation*}
        Por tanto, sí es válido el mismo polinomio para el punto $(1,2)$, ya que lo interpola.

        Sin enbargo,
        \begin{equation*}
            p_3(3)=124
        \end{equation*}
        Por tanto, no es válido el mismo polinomio para el punto $(3,0)$, ya que no lo interpola.
    \end{enumerate}
\end{ejercicio}


\begin{ejercicio}
    Sean $\ell_i(t),i=1,\dots,n$ los polinomios básicos de Lagrange. Demuestre que:
    \begin{enumerate}
        \item $\{\ell_0(t),\ell_1(t),\dots, \ell_n(t)\}$ constituyen una base de $\bb{P}_n$,

        Sabemos que $\ell_k(x_j) = \delta_{k,j}$.
        
        Supongamos que son linealmente dependientes, es decir, que $\exists a_0,\dots, a_{n}\in~\bb{R}$ no todos nulos tal que:
        \begin{equation*}
            0 = a_0 \ell_0(t) + \dots + a_j\ell_j(t) + \dots + a_n\ell_n(t)
        \end{equation*}
        Evaluando en cada $x_j$, obtenemos que $a_j=0 \quad \forall j$. Por tanto, son nulos, por lo que llegamos a una contradicción.

        Como tenemos $n+1$ vectores linealmente independientes en un espacio vectorial de dimensión $n+1$, tenemos que forman base.

        \item $\sum_{i=0}^n \ell_i(t)=1$.

        Sean los puntos $(x_i, f_i)$, con $f_i=1\quad \forall i$. Es decir, un conjunto de puntos alineados sobre la recta $y=1$. Por tanto, sabemos que el polinomio de interpolación es $p_n(x)=1$.

        El polinomio de interpolación de Lagrange es:
        \begin{equation*}
            p_n(x) = \sum_{i=0}^ny_i\ell_i(x) = \sum_{i=0}^n \ell_i(x)
        \end{equation*}

        Por la unicidad del polinomio de interpolación:
        \begin{equation*}
            p_n(x) = \sum_{i=0}^n \ell_i(x) = 1
        \end{equation*}
    \end{enumerate}
\end{ejercicio}

\begin{ejercicio}    
    Estudie para que valores de $a\in \bb{R}$ es unisolvente el siguiente problema de interpolación:\\
    Encontrar $p\in \bb{P}_2$ tal que:
    \begin{equation*}
        \left.\begin{array}{ccc}
            p(-1) &=& \omega_0  \\
            p'(a) &=& \omega_1  \\
            p(1) &=& \omega_2  \\
        \end{array}\right\}
    \end{equation*}

    Procedemos al cálculo del polinomio mediante el método de coeficientes indeterminados. Sea $p_2(x)=b_0 + b_1x + b_2x^2$.
    \begin{equation*}
        \left.\begin{array}{ccc}
            p(-1) &=& \omega_0  \\
            p'(a) &=& \omega_1  \\
            p(1) &=& \omega_2  \\
        \end{array}\right\}
        \Longrightarrow
        \left.\begin{array}{rrrrrcc}
            b_0&-&b_1& +&b_2 &=& \omega_0  \\
            &&b_1&+&2b_2a &=& \omega_1  \\
            b_0&+&b_1&+&b_2 &=& \omega_2  \\
        \end{array}\right\}
    \end{equation*}

    Por el teorema de Rouché-Frobenius, tenemos que la solución será única si el determinante de la matriz de coeficientes no es nulo. Por tanto,
    \begin{equation*}
        \left|\begin{array}{ccc}
            1 & -1 & 1 \\
            0 & 1 & 2a \\
            1 & 1 & 1
        \end{array}\right| \neq 0
        \Longleftrightarrow 1-2a-1-2a = -4a \neq 0\Longleftrightarrow a\neq 0
    \end{equation*}

    Por tanto, el problema tendrá solución única si $a\neq 0$. Para $a=0$, tendrá infinitas soluciones si $\omega_1=0$; mientras que no existirá solución en caso contrario.
\end{ejercicio}

\begin{ejercicio}
    Demuestre que el determinante de Vandermonde
    \begin{equation*}
        V(x_0, \dots, x_n):= \left|
        \begin{array}{cccc}
            1 & x_0 & \dots & x_0^n \\
            1 & x_1 & \dots & x_1^n \\
            \vdots & \vdots &  & \vdots \\
            1 & x_n & \dots & x_n^n \\
        \end{array}\right|
    \end{equation*}
    verifica
    \begin{equation*}
        V(x_0, \dots, x_n) = \prod_{n\geq i>j\geq 0}(x_i-x_j)
    \end{equation*}
    y que por tanto $V(x_0, \dots, x_n) \neq 0 \Longleftrightarrow x_i\neq x_j$ para $i\neq j$.\\

    Demostramos por inducción sobre $n$.
    \begin{itemize}
        \item \underline{Para $n=1$}:
        \begin{equation*}
            V(x_0, x_1):= \left|
            \begin{array}{cc}
                1 & x_0 \\
                1 & x_1\\
            \end{array}\right| = x_1 - x_0 = \prod_{1\geq i>j\geq 0}(x_i-x_j)
        \end{equation*}

        \item \underline{Supuesto cierto para $n-1$, demostramos para $n$}:
        \begin{equation*}\begin{split}
            V(x_0, \dots, x_n):&= \left|
            \begin{array}{cccc}
                1 & x_0 & \dots & x_0^n \\
                1 & x_1 & \dots & x_1^n \\
                \vdots & \vdots &  & \vdots \\
                1 & x_n & \dots & x_n^n \\
            \end{array}\right|
            \stackrel{C'_j = C_j -x_0C_{j-1}}{=}
             \left| \begin{array}{cccc}
                1 & 0 & \dots & 0 \\
                1 & x_1-x_0 & \dots & x_1^n-x_1^{n-1}x_0 \\
                \vdots & \vdots &  & \vdots \\
                1 & x_n-x_0 & \dots & x_n^n-x_n^{n-1}x_0 \\
            \end{array}\right|
            =\\&=
            \left| \begin{array}{ccc}
                x_1-x_0 & \dots & x_1^n-x_1^{n-1}x_0 \\
                \vdots &  & \vdots \\
                x_n-x_0 & \dots & x_n^n-x_n^{n-1}x_0 \\
            \end{array}\right|
            =
            \left| \begin{array}{ccc}
                x_1-x_0 & \dots & x_1^{n-1}(x_1-x_0) \\
                \vdots &  & \vdots \\
                x_n-x_0 & \dots & x_n^{n-1}(x_n-x_0) \\
            \end{array}\right|
            =\\&=
            (x_1-x_0)\dots(x_n-x_0)
            \left|\begin{array}{cccc}
                1 & x_1 & \dots & x_1^{n-1} \\
                1 & x_2 & \dots & x_2^{n-1} \\
                \vdots & \vdots &  & \vdots \\
                1 & x_n & \dots & x_n^{n-1} \\
            \end{array}\right|
            =\\&=
            \prod_{i=1}^n (x_i-x_0) \prod_{n\geq i>j\geq 1}(x_i - x_j)
            = \prod_{n\geq i>j\geq 0}(x_i - x_j)
        \end{split}\end{equation*}

        Demostrándolo así para $n$. Por tanto,
        $$V(x_0, \dots, x_n) \neq 0 \Longleftrightarrow \prod_{n\geq i>j\geq 0}(x_i-x_j) \neq 0 \Longleftrightarrow x_i\neq x_j \qquad \forall i,j$$
    \end{itemize}
\end{ejercicio}

\begin{ejercicio}
    Al medir $f$ en una serie de puntos $x_i$, se han obtenido los siguientes valores:
    \begin{equation*}
        \begin{array}{c|cccccccc}
            x_i & 0 & 1 & 2 & 3 & 4 & 5 & 6 \\ \hline
            f_i & 0 & 1 & 8 & 26 & 64 & 125 & 216 \\
        \end{array}
    \end{equation*}

    \begin{enumerate}
        \item Calcule la tabla de diferencias divididas.
        \begin{equation*}
            \begin{array}{c|ccccccc}
                x_i & f[x_i] \\
                \\
                0 & \textbf{0} \\
                && \textbf{1}\\
                1 & 1 && \textbf{3}\\
                && 7&&\mathbf{\frac{5}{6}}\\
                2 & 8 && \frac{11}{2} && \mathbf{\frac{1}{6}}\\
                && 18 && \frac{3}{2} && \mathbf{-\frac{1}{12}}\\
                3 & 26 && 10 && -\frac{1}{4} && \mathbf{\frac{1}{36}}\\
                && 38 && \frac{1}{2} && \frac{1}{12}\\
                4 & 64 && \frac{23}{2} && \frac{1}{6}\\
                && 61  && \frac{7}{6}\\
                5 & 125 && 15\\
                && 91\\
                6 & 216
            \end{array}
        \end{equation*}
        
        \item Al medir $f$ en el punto $x = 3$ se cometió un error, ya que el valor exacto era $f(3) = 27$ y se obtuvo $26$. Estudie la propagación de dicho error en la tabla de diferencias divididas.
        \begin{equation*}
            \begin{array}{c|ccccccc}
                x_i & f[x_i] \\
                \\
                0 & 0 \\
                && 1\\
                1 & 1 && 3\\
                && 7&&\textbf{1}\\
                2 & 8 && \textbf{6} && \textbf{0}\\
                && \textbf{19} && \textbf{1} && \textbf{0}\\
                3 & \textbf{27} && \textbf{9} && \textbf{0} && \textbf{0}\\
                && \textbf{37} && \textbf{1} && \textbf{0}\\
                4 & 64 && \textbf{12} && \textbf{0}\\
                && 61  && \textbf{1}\\
                5 & 125 && 15\\
                && 91\\
                6 & 216
            \end{array}
        \end{equation*}

        Como podemos ver, el error se propaga a lo largo de la tabla de diferencias, afectando a gran cantidad de diferencias divididas. Además, los errores son significativos, ya que se obtendría un polinomio de grado 3, mientras que el anterior sería de grado 6.

        \item Supongamos que los valores $f_i$ no son todos exactos sino que unos son más fiables que otros. Si se desea que los menos fiables intervengan en la obtención del menor número posible de coeficientes en la fórmula de Newton, ¿cómo hay que ordenar los cálculos?

        Hay que situar los datos $(x_i,f_i)$ más fiables en la zona intermedia de la tabla de diferencias divididas, mientras que los menos fiables se debe situar en la parte superior o en la parte inferior, para afectar así al menor número posible de diferencias divididas.
        
    \end{enumerate}
\end{ejercicio}

\begin{ejercicio}
    Utilice las propiedades de las diferencias divididas para determinar de qué grado es el polinomio $p$ del que se conocen los siguientes valores:
    \begin{equation*}
        \begin{array}{c|ccccccc}
            x_i & -2 & -1 & 0 & 1 & 2 & 3 \\ \hline
            f_i & -5 & 1 & 1 & 1 & 7 & 25 \\
        \end{array}
    \end{equation*}
    Calculo la tabla de diferencias divididas:
    \begin{equation*}
        \begin{array}{c|cccccc}
            x_i & f[x_i] \\
            \\
            -2 & \textbf{-5} \\
            && \textbf{6}\\
            -1 & 1 && \textbf{-3}\\
            && 0&&\mathbf{1}\\
            0 & 1 && 0 && 0\\
            && 0 && 1 && 0\\
            1 & 1 && 3 && 0\\
            && 6 && 1\\
            2 & 7 && 6\\
            && 18  && \\
            3 & 25\\
        \end{array}
    \end{equation*}

    Sabemos que el coeficiente líder de $p_{n}(x)$ es $f[x_0,\dots,x_n]$. Como tenemos que:
    \begin{equation*}
        f[x_0,\dots,x_5] = f[x_0,\dots,x_4] = 0
    \end{equation*}
    Entonces, el coeficiente líder de $p_{5}(x)$ y el de $p_{4}(x)$ es el 0. Por tanto, el polinomio que interpola esos 6 puntos es de grado 3.
\end{ejercicio}

\begin{ejercicio}
    Sea el polinomio de interpolación en forma de Newton
    \begin{equation*}
        p(x) = (x + 3)(x + 2)(x + 1)x - (x + 3)(x + 2)(x + 1) - 3(x + 3)(x + 2) + 17(x + 3) - 26
    \end{equation*}
    Se desea obtener la tabla de valores que generó el polinomio anterior.
    \begin{enumerate}
        \item ¿Cuántos datos de interpolación tenía el problema?

        Como $p(x)\in \bb{P}_4(x)$, como mínimo había 5 datos. Podría haber más datos, pero todos ellos referidos al mismo polinomio, es decir, redundantes.
        
        \item Recupere la tabla completa de diferencias divididas teniendo en cuenta que los nodos son equidistantes, esto es, $x_i - x_{i-1}$, $i = 1, 2, \dots , n$, es constante.

        El término líder sabemos que es:
        \begin{equation*}
            f[x_0,\dots,x_n]\cdot (x-x_0)\dots(x-x_{n-1}) = (x + 3)(x + 2)(x + 1)x
        \end{equation*}
        Por tanto, tenemos que las abcisas de los cuatro primeros datos: $\{0, -1, -2, -3\}$. 

        Para calcular la quinta abcisa, nos fijamos en cómo se ha dado el polinomio. Este tiene la forma de:
        \begin{equation*}
            p(x) = f[x_0] + \sum_{k=1}^n f[x_0,\dots, x_k]\prod_{i=0}^{k-1} (x-x_i)
        \end{equation*}

        Nos fijamos por tanto monomio por monomio.
        \begin{itemize}
            \item Del monomio de grado 0, $-26$ vemos que $f[x_0]=-26$
            \item Del monomio de grado 1, $17(x+3)$ vemos que $x_0=-3$ y $f[x_0,x_1]=17$
            \item Del monomio de grado 2, $-3(x+3)(x+2)$ vemos que $x_1=-2$ y $f[x_0,x_1, x_2]=-3$
            \item Del monomio de grado 3, $-(x+3)(x+2)(x+1)$ vemos que $x_2=-1$ y $f[x_0,x_1, x_2, x_3]=-1$
            \item Del monomio de grado 4, $(x+3)(x+2)(x+1)x$ vemos que $x_3=0$ y $f[x_0,x_1, x_2, x_3, x_4]=1$
        \end{itemize}

        Por tanto, como las abcisas son equidistantes,
        \begin{equation*}
            1 = x_3-x_2 = x_4-x_3 = x_4-0 = x_4 \Longrightarrow x_4 = 1
        \end{equation*}

        Calculamos sus imágenes mediante el algoritmo de Newton-Horner:
        \begin{equation*}
            \begin{split}
                p(x)& = (x + 3)(x + 2)(x + 1)x - (x + 3)(x + 2)(x + 1) - 3(x + 3)(x + 2) + 17(x + 3) - 26\\
                &= -26 +(x+3)[17 + (x+2)[-3 + (x+1)[-1 + x]]]
            \end{split}
        \end{equation*}
        \begin{equation*}
            p(-3) = -26 \qquad p(-2) = -9 \qquad p(-1) =  2 \qquad p(0) = 1 \qquad p(1) = 6
        \end{equation*}

        Por tanto, sabemos:
        \begin{equation*}
            \begin{array}{c|cccccc}
                x_i & f[x_i] \\
                \\
                -3 & \textbf{-26} \\
                && \textbf{17}\\
                -2 & -9 && \textbf{-3}\\
                && -&&\textbf{-1}\\
                -1 & 2 && - && \textbf{1}\\
                && - && -\\
                0 & 1 && -\\
                && -\\
                1 & 6\\
            \end{array}
        \end{equation*}
        
        Completamos, por tanto, la tabla de diferencias divididas.
        \begin{equation*}
            \begin{array}{c|cccccc}
                x_i & f[x_i] \\
                \\
                -3 & \textbf{-26} \\
                && \textbf{17}\\
                -2 & -9 && \textbf{-3}\\
                && 11&&\textbf{-1}\\
                -1 & 2 && -6 && \textbf{1}\\
                && -1 && 3\\
                0 & 1 && 3\\
                && 5\\
                1 & 6\\
            \end{array}
        \end{equation*}
    \end{enumerate}
\end{ejercicio}

\begin{ejercicio}
    Usando aritmética de tres cifras por redondeo, calcule el polinomio de interpolación para los siguientes datos: $f(0.8) = 0.224, f'(0.8) = 2.17, f(1.0)~=~0.658,$ $f'(1.0) = 2.04$. Estime $f(0.9)$ en dicha aritmética, minimizando el error de redondeo.

    \begin{equation*}
        \begin{array}{c|cc}
            x_i & 0.8 & 1.0 \\ \hline
            f_i & 0.224 & 0.658 \\ \hline
            f'_i & 2.17 & 2.04
        \end{array}
    \end{equation*}
    Calculo la tabla de diferencias divididas:
    \begin{equation*}
        \begin{array}{c|cccccc}
            x_i & f[x_i] \\
            \\
            0.8 & \textbf{0.224} \\
            && \textbf{2.17}\\
            0.8 & 0.224 && \textbf{0}\\
            && 2.17&&\mathbf{-3.25}\\
            1 & 0.658 && -0.65 &&\\
            && 2.04\\
            1 & 0.658
        \end{array}
    \end{equation*}

    Por tanto, tengo que:
    \begin{equation*}
        \begin{split}
            p_3(x)&= 0.224 + 2.17(x-0.8) -3.25(x-1)(x-0.8)^2 \\
            &= 0.224 +(x-0.8)[2.17-3.25(x-1)(x-0.8)]
        \end{split}
    \end{equation*}

    Para calcular $f(0.9)$ minimizando el error cometido, empleo el algoritmo de Newton-Horner.
    \begin{equation*}
        \begin{split}
            p_3(0.9)&= 0.224 +(0.9-0.8)[2.17-3.25(0.9-1)(0.9-0.8)] \\
            &= 0.224 +(0.1)[2.17-3.25(-0.1)(0.1)]\\
            &= 0.224 +(0.1)[2.17-3.25(-0.01)] \\
            &= 0.224 +(0.1)[2.17 +0.0325] \\
            &= 0.224 +(0.1)[2.20] = 0.224+0.22 = 0.444 \\
        \end{split}
    \end{equation*}
    
\end{ejercicio}

\begin{ejercicio}
    En este problema se trata de probar mediante interpolación la fórmula
    \begin{equation*}
        0+1+2+\dots+n = \frac{n(n+1)}{2}
    \end{equation*}
    que es válida para todo número natural $n\geq 0$.
    
    \begin{enumerate}
        \item Utilice las diferencias divididas para demostrar que la función $p(n) = 0 + 1 + 2 + 3 + \dots + n$ es un polinomio de grado 2 en la variable $n \geq 0$.
        
        Calculo la tabla de diferencias divididas:
        \begin{equation*}
            \begin{array}{c|ccccccc}
                n & p(n) \\
                \\
                0 & \textbf{0} \\
                && \textbf{1}\\
                1 & 1 && \mathbf{\frac{1}{2}}\\
                && 2&&\textbf{0}\\
                2 & 3 && \frac{1}{2}\\
                && 3 &&\\
                3 & 6\\
                && \\
                \vdots & \\
                n-2 & \sum_{i=0}^{n-2}i\\
                && n-1\\
                n-1 & \sum_{i=0}^{n-1}i && \frac{1}{2}\\
                && n\\
                n & \sum_{i=0}^{n}i
            \end{array}
        \end{equation*}
        Por tanto, tenemos que $f[x_1, x_1, x_3, x_4]=0\;\forall x_i$ consecutivos.Por tanto, tan solo tres puntos son necesarios para interpolarlo, por lo que es de grado 2.

        \item Utilizando la fórmula de Newton para el polinomio de interpolación demuestre que
        \begin{equation*}
            p(n) = \frac{n(n+1)}{2}
        \end{equation*}

        \item Utilice un procedimiento análogo para calcular el valor de la suma
        \begin{equation*}
            0^2 + 1^2 + \dots + n^2
        \end{equation*}
    \end{enumerate}
\end{ejercicio}

\begin{ejercicio}
    Consideremos la función $f(x) = \ln x$ y sea $p(x)$ el polinomio que la interpola en $x_0$ y $x_1$, con $0<x_0<x_1$.

    \begin{enumerate}
        \item Demuestre que el error cometido en cualquier punto del intervalo $[x_0, x_1]$ está acotado por:
        \begin{equation*}
            |e(x)|\leq \frac{(x_1-x_0)^2}{8x_0^2}
        \end{equation*}

        Sabemos que el error cometido en cualquier punto del intervalo $[x_0, x_1]$ viene dado por:
        \begin{equation*}
            |e(x)| = \left| \frac{f''(\xi)}{2!}(x-x_1)(x-x_0)\right| =  \frac{\left|f''(\xi)\right|}{2!}\left|(x-x_1)(x-x_0)\right| \qquad \xi\in [x_0.x_1]
        \end{equation*}

        Como $f''(x)=-\frac{1}{x^2}$ es estrictamente creciente en $\bb{R}^+$, y $x_0,x_1\in \bb{R}^+$ por pertenecer al dominio de $f$, entonces tenemos que:
        \begin{multline*}
            Im \left(f''_{\left|[x_0,x_1] \right.} \right) = \left[f''(x_0), f''(x_1)\right] = \left[-\frac{1}{x_0^2}, -\frac{1}{x_1^2}\right]
            \\ \Longrightarrow
            Im \left(\left|f''_{\left|[x_0,x_1] \right.}\right| \right) = \left[\frac{1}{x_1^2}, \frac{1}{x_0^2}\right]
        \end{multline*}

        Calculamos también la imagen de $|h(x)|=|(x-x_1)(x-x_0)|$.
        \begin{equation*}
            h(x)=(x-x_1)(x-x_0)=x^2 -(x_1+x_0)x + x_1x_0
        \end{equation*}
        \begin{equation*}
            h'(x) = 2x -(x_1+x_0) = 0 \Longleftrightarrow x=\frac{x_1+x_0}{2}
        \end{equation*}
        Como $h''(x) = 2>0 \Longrightarrow x=\frac{x_1+x_0}{2}$ es un mínimo relativo. Al ser una parábola, también es absoluto.
        \begin{equation*}
            h(x_1) = 0 \qquad h(x_0) = 0
        \end{equation*}
        \begin{multline*}
            h\left(\frac{x_1+x_0}{2}\right) = \frac{(x_1+x_0)^2}{4} - \frac{(x_1+x_0)^2}{2} + x_1x_0 = -\frac{(x_1+x_0)^2}{4} + x_1x_0 =\\= -\frac{x_1^2+x_0^2+2x_1x_0-4x_1x_0}{4} =-\frac{x_1^2+x_0^2-2x_1x_0}{4} = -\frac{(x_1-x_0)^2}{4} < 0
        \end{multline*}

        Como la imagen del mínimo es negativa, tenemos,
        \begin{equation*}
            Im \left(\left|h_{\left|[x_0,x_1] \right.}\right| \right) = \left[0, \frac{(x_1-x_0)^2}{4}\right]
        \end{equation*}

        Por tanto,
        \begin{equation*}
            |e(x)| = \frac{\left|f''(\xi)\right|}{2!}\left|(x-x_1)(x-x_0)\right|
            \leq
            \frac{1}{2x_0^2}\cdot \frac{(x_1-x_0)^2}{4}
             = \frac{(x_1-x_0)^2}{8x_0^2}
        \end{equation*}

        \item Si tomamos $x_0 = 1$ ¿hasta donde podremos extender el intervalo asegurando un error menor que $10^{-4}$? ¿Y si partimos de $x_0 = 100$?

        La acotación del error, siendo $M$ la cota, es:
        \begin{equation*}
            |e(x)|\leq \frac{(x_1-x_0)^2}{8x_0^2} = M
        \end{equation*}

        Sabiendo el valor de $x_0$ y de la cota deseada, despejamos $x_1$:
        \begin{equation*}
            \frac{(x_1-x_0)^2}{8x_0^2} = M
            \Longrightarrow
            \frac{x_1-x_0}{\sqrt{8}x_0} = \sqrt{M}
            \Longrightarrow
            x_1 = \sqrt{8M}x_0 + x_0
        \end{equation*}

        Para $M=10^{-4}$ y $x_0=1$, $x_1=\frac{50+\sqrt{2}}{50}\approx 1.02828$.

        Para $M=10^{-4}$ y $x_0=100$, $x_1\approx102.8284$.
        
        \item Se desea tabular $f(x) = \ln x$ para ser capaces de obtener (por interpolación lineal entre puntos adyacentes) cualquier valor de $f(x)$ con un error menor de $10^{-2}$. Dar una expresión para los $x_n$ a utilizar, indicando cuantos serán precisos para cubrir adecuadamente el intervalo $[1,100]$.

        Sabiendo que, desde un extremo inferior del intervalo $x_0$, se puede extender con un error menor que $M$ el intervalo hasta un extremo superior $x_1 = \sqrt{8M}x_0 + x_0$, definimos los siguientes valores:
        \begin{equation*}
            M = 10^{-2} \qquad x_0 = 1
        \end{equation*}

        Por tanto, la expresión para los $x_n$ es:
        \begin{equation*}
            x_n = \sqrt{8M}x_{n-1} + x_{n-1} = \frac{\sqrt{2}}{5}x_{n-1} + x_{n-1} = \left(\frac{\sqrt{2}}{5}+1\right)x_{n-1} \qquad \forall n\in \bb{N}
        \end{equation*}

        De esta forma, vamos construyendo intervalos adyacentes donde, en cada intervalo, el error relativo es menor que la cota establecida.

        Además, tenemos que $\{x_n\}$ es una sucesión geométrica. Por tanto, se tiene que:
        \begin{equation*}
            x_n = \left(\frac{\sqrt{2}}{5}+1\right)^n x_0
        \end{equation*}

        Como se busca que $x_n$ sea $\geq 100$, tenemos:
        \begin{equation*}
            x_n = \left(\frac{\sqrt{2}}{5}+1\right)^n x_0 = \left(\frac{\sqrt{2}}{5}+1\right)^n \geq 100
        \end{equation*}

        Como $n\in \bb{N}$, probando obtenemos que el primer valor que lo cumple es $n=19$, ya que:
        \begin{equation*}
            x_{18}=88.536 \qquad x_{19}=113.578
        \end{equation*}

        Por tanto, se necesita un total de $20$ puntos. De hecho, los puntos son:
        \begin{equation*}
        \begin{array}{c|c}
        n   &  x_n       \\ \hline
        0  & 1           \\
        1  & 1,282842712 \\
        2  & 1,645685425 \\
        3  & 2,111155554 \\
        4  & 2,708280518 \\
        5  & 3,474297926 \\
        6  & 4,456977775 \\
        7  & 5,717601458 \\
        8  & 7,334783364 \\
        9  & 9,409373386 \\
        10 & 12,07074608 \\
        11 & 15,48486864 \\
        12 & 19,86465089 \\
        13 & 25,48322263 \\
        14 & 32,69096644 \\
        15 & 41,93736806 \\
        16 & 53,79904699 \\
        17 & 69,01571537 \\
        18 & 88,53630751 \\
        19 & 113,5781569
        \end{array}
        \end{equation*}

        

        \begin{comment}
        Como se desea saber el número de puntos necesarios, podemos usar los polinomios de Chebyshev. Interpolando con las raíces de $T_{n+1}(x)$, sabemos que el error cometido al interpolar en $[x_0, x_1]$ es:
        \begin{equation*}
            |e(x)| = \frac{|f^{n+1)}(\xi)|}{(n+1)!\cdot 2^n} \qquad \xi \in [x_0,x_1]
        \end{equation*}

        Por tanto, calculamos la derivada $n$-ésima de $ln(x)$. Esta es:
        \begin{equation*}
            f^{n+1)}(x) = (-1)^{n}\frac{n!}{x^{n+1}} \Longrightarrow |f^{n+1)}(x)| = \frac{n!}{x^{n+1}}
        \end{equation*}

        Como $|f^{n+1)}(x)|$ es estrictamente decreciente ($x\in \bb{R}$), tenemos que la imagen de la restricción de esta al intervalo $[x_0,x_1]$ es:
        \begin{equation*}
            Im\left(|f^{n+1)}(x)|_{\left|[x_0,x_1] \right.} \right) = \left[ \frac{n!}{x_1^{n+1}}, \frac{n!}{x_0^{n+1}} \right]
        \end{equation*}

        Por tanto,
        \begin{equation*}
            |e(x)| = \frac{|f^{n+1)}(\xi)|}{(n+1)!\cdot 2^n} \leq \frac{n!}{x_0^{n+1}} \cdot \frac{1}{(n+1)! 2^n} = \frac{1}{(n+1)\cdot 2^n \cdot x_0^{n+1}}
        \end{equation*}

        Como se desea que la cota del error sea menor o igual que $10^{-2}$, y sabiendo que $x_0 = 1$,
        \begin{equation*}
            \frac{1}{(n+1)\cdot 2^n \cdot x_0^{n+1}} \leq 10^{-2} \Longrightarrow (n+1)\cdot 2^n \cdot x_0^{n+1} \geq 100 \Longrightarrow (n+1)\cdot 2^n \geq 100
        \end{equation*}

         Por tanto, podemos ver que para $n=5$ se satisface la inecuación. Por tanto, los puntos para interpolar son las raíces de $T_{6}(x)$. Estas son:
    \begin{equation*}
        x_i = \cos \frac{(2i+1)\pi}{2\cdot 5} \quad i=0,1,\dots,5 
    \end{equation*}

    MAL PQ TAMBIÉN SALEN NEGATIVAS
        \end{comment}
    \end{enumerate}
   
\end{ejercicio}


\begin{ejercicio}
    Estudie la unisolvencia del problema de interpolación consistente en hallar $P\in\bb{P}_3$ tal que verifica
    \begin{gather*}
        p(x_1) = y_1,\qquad p''(x_1)=z_1,\\
        p(x_2) = y_2,\qquad p''(x_2)=z_2,\\
    \end{gather*}
    para cualesquiera puntos $x_1,x_2\in \bb{R}$ con $x_1\neq x_2$, y cualesquiera valores $y_2,y_2,z_1,z_2 \in~\bb{R}$. \\

    Sea $P(x)=a_0+a_1x+a_2x^2+a_3x^3$:
    \begin{equation*}
        P'(x)=a_1+2a_2x+3a_3x^2 \qquad P''(x)=2a_2 +6a_3x    
    \end{equation*}
    
    Tenemos que:
    \begin{equation*}
        \left\{\begin{array}{l}
            a_0+a_1x_1+a_2x_1^2+a_3x_1^3 = y_1 \\
            a_0+a_1x_2+a_2x_2^2+a_3x_2^3 = y_2 \\
            2a_2 +6a_3x_1 = z_1 \\
            2a_2 +6a_3x_2 = z_2 \\
        \end{array}\right.
    \end{equation*}

    El determinante de la matriz de coeficientes es:
    \begin{equation*}\begin{split} \left|
        \begin{array}{cccc}
            1 & x_1 & x_1^2 & x_1^3 \\
            1 & x_2 & x_2^2 & x_2^3 \\
            0 & 0 & 2 & 6x_1 \\
            0 & 0 & 2 & 6x_2 \\
        \end{array}\right|
        = \left|\begin{array}{cc}
            1 & x_1 \\
            1 & x_2
        \end{array}\right|
        \left|\begin{array}{cc}
            2 & 6x_1 \\
            2 & 6x_2
        \end{array}\right| = (x_2-x_1)\cdot 2 \cdot 6(x_2-x_1) = 12(x_2-x_1)\neq 0
    \end{split}\end{equation*}

    Por tanto, por el Teorema de Rouché-Frobenious, la solución es única.
\end{ejercicio}

\begin{ejercicio}
    Aplique el algoritmo de Newton–Horner para aproximar $\sqrt{3}$ con los datos proporcionados por la función $f(x) = 3^x$ en los nodos $x_0 = -2, x_1 = -1, x_2 = 0, x_3 = 1, y x_4 = 2$. Proporcione una cota del error cometido.
    \begin{equation*}
        \begin{array}{c|ccccc}
            x_i & -2 & -1 & 0 & 1 & 2 \\ \hline
            f_i & \frac{1}{9} & \frac{1}{3} & 1 & 3 & 9
        \end{array}
    \end{equation*}

    La tabla de diferencias divididas queda:
    \begin{equation*}
        \begin{array}{c|cccccc}
            x_i & f[x_i] \\
            \\
            -2 & \mathbf{\frac{1}{9}} \\
            && \mathbf{\frac{2}{9}}\\
            -1 & \frac{1}{3} && \mathbf{\frac{2}{9}}\\
            && \frac{2}{3}&&\mathbf{\frac{4}{27}}\\
            0 & 1 && \frac{2}{3} && \mathbf{\frac{2}{27}}\\
            && 2 && \frac{4}{9}\\
            1 & 3 && 2\\
            && 6\\
            2 & 9\\
        \end{array}
    \end{equation*}

    Por tanto, el polinomio de interpolación es:
    \begin{equation*}\begin{split}
        p_4(x)&=\frac{1}{9}+\frac{2}{9}(x+2) +\frac{2}{9}(x+2)(x+1) + \frac{4}{27}(x+2)(x+1)x + \frac{2}{27}(x+2)(x+1)x(x-1) \\
        &= \frac{1}{9}+(x+2)\left[\frac{2}{9} +(x+1)\left[\frac{2}{9} +x\left[\frac{4}{27}+\frac{2}{27}(x-1)\right]\right]\right]
    \end{split}\end{equation*}

    Evaluando mediante el Algoritmo de Newton-Horner:
    \begin{equation*}
        p_4\left(\frac{1}{2}\right) = \frac{41}{24} \approx 1.708\bar{3} \approx \sqrt{3}
    \end{equation*}

    Para acotar el error cometido, sabemos que:
    \begin{equation*}
        e(x) = \frac{f^{n+1)}(\xi)}{(n+1)!}\prod_{i=0}^n (x-x_i)
    \end{equation*}

    Por tanto, como $n=4$ y sustituyendo los nodos:
    \begin{equation*}
        e(x) = \frac{3^\xi \ln^{5}(3)}{5!} (x+2)(x+1)x(x-1)(x-2) \qquad \xi \in [-2,2]
    \end{equation*}

    El error cometido por tanto, al aproximar en $x=\frac{1}{2}$ es:
    \begin{equation*}
        e\left(\frac{1}{2}\right)
        =\frac{3^\xi \ln^5 (3)}{5!} \cdot \frac{5}{2} \cdot \frac{3}{2} \cdot \frac{1}{2} \cdot \frac{-1}{2} \cdot \frac{-3}{2}
        = 3^\xi \ln^5 (3) \cdot \frac{3}{2^8} 
    \end{equation*}

    Además, como $\xi \in [-2, 2]$ y la exponencial es estrictamente creciente, tengo que $3^{-2} \leq 3^\xi \leq 3^2$. Por tanto,
    \begin{equation*}
        e\left(\frac{1}{2}\right)= 3^\xi \ln^5 (3) \cdot \frac{3}{2^8} \leq \ln^5 (3) \cdot \frac{3^3}{2^8} 
    \end{equation*}
\end{ejercicio}

\begin{ejercicio}
    Halle el polinomio $p \in \bb{P}_5$ que verifica
    \begin{gather*}
        p(-1) = 6,\qquad p(0)=2,\qquad p(1)=0,\\
        p'(-1)=-13,\qquad p'(0)=0,\qquad p'(1)=-5
    \end{gather*}

    Usamos el método de interpolación de Hermite:
    \begin{equation*}
        \begin{array}{c|cccccc}
            x_i & p[x_i] \\
            \\
            -1 & \textbf{6} \\
            && \textbf{-13}\\
            -1 & 6 && \textbf{9}\\
            && -4&&\mathbf{-5}\\
            0 & 2 && 4 && \textbf{1}\\
            && 0 && -3 && \textbf{0}\\
            0 & 2 && -2 && 1\\
            && -2 && -1\\
            1 & 0 && -3\\
            && -5  && \\
            1 & 0\\
        \end{array}
    \end{equation*}

    Por tanto, el polinomio queda:
    \begin{equation*}
        p(x)=6-13(x+1)+9(x+1)^2 -5(x+1)^2x +x^2(x+1)^2
    \end{equation*}
\end{ejercicio}

\begin{ejercicio}
    Se desea interpolar la función $f(x) = \ln x$ en los puntos de abcisas 1, 2 y 3 mediante un polinomio de grado adecuado.
    \begin{equation*}
        \begin{array}{c|cccc}
            x_i & 1 & 2 & 3 \\ \hline
            f_i & \ln 1 = 0 & \ln 2 & \ln 3 \\
        \end{array}
    \end{equation*}
    
    \begin{enumerate}
        \item Calcule el polinomio de interpolación utilizando las fórmulas de Lagrange y de Newton.

        Como se dan 3 nodos, $p_n(x)\in \bb{P}_2$.
        
        Empezamos por el método de Lagrange. Calculamos en primer lugar los polinomios básicos de Lagrange.
        \begin{equation*}
            \ell_0(x) = \frac{(x-2)(x-3)}{2}
            \qquad
            \ell_1(x) = \frac{(x-1)(x-3)}{-1}
            \qquad
            \ell_2(x) = \frac{(x-1)(x-2)}{2}
        \end{equation*}

        Por tanto, el polinomio de interpolación queda:
        \begin{equation*}
            p_n (x) = 0\cdot \frac{(x-2)(x-3)}{2} + \ln 2 \cdot \frac{(x-1)(x-3)}{-1} + \ln 3 \frac{(x-1)(x-2)}{2}
        \end{equation*}

        Empleamos ahora el método de Newton. La tabla de diferencias divididas queda:
        \begin{equation*}
            \begin{array}{c|ccc}
                x_i & f[x_i] \\
                \\
                1 & \textbf{0} \\
                && \mathbf{\ln 2}\\
                2 & \ln 2 & & \mathbf{\frac{1}{2} \ln \left(\frac{3}{4} \right) }\\
                && \mathbf{\ln \left(\frac{3}{2} \right)}\\
                3 & \ln 3 \\
            \end{array}
        \end{equation*}

        Por tanto, el polinomio de interpolación queda:
        \begin{equation*}
            p_n (x) = 0 + \ln 2 (x-1) +\frac{1}{2} \ln \left(\frac{3}{4} \right) (x-1)(x-2)
        \end{equation*}

        \item Obtenga una cota lo más ajustada posible del error de interpolación en el intervalo $[1, 3]$

        Ya que $n=2$, el error de interpolación viene dado por:
        \begin{equation*}
            |e(x)| = \frac{|f^{3)}(\xi)|}{(3)!}\left|\prod_{k=0}^2 (x-x_k)\right| \qquad \xi \in [1,3]
        \end{equation*}

        Acoto en primer lugar la tercera derivada.
        \begin{equation*}
            f^{3)}(x) = \frac{2}{x^3} \qquad \forall x\in [1,3]
        \end{equation*}
        Como $f^{3)}(x)$ es continua y estrictamente decreciente en $[1,3]$, tenemos que
        \begin{equation*}
            Im(f^{3)}_{\left|[1,3]\right.}) = \left[\frac{2}{3^3}, 2\right] =  Im(|f^{3)}_{\left|[1,3]\right.}|)
        \end{equation*}

        Acoto ahora el producto $|h(x)| = |(x-1)(x-2)(x-3)| = |x^3-6x^2+11x-6|$.
        \begin{equation*}
            h'(x)=~3x^2-12x+11 = 0 \Longleftrightarrow x = \frac{6\pm \sqrt{3}}{3}
        \end{equation*}
        Calculo las imágenes de los extremos del intervalo y de los extremo relativos.
        \begin{equation*}
            h(1)=h(3) = 0
        \end{equation*}
        \begin{equation*}
            h\left(\frac{6+ \sqrt{3}}{3} \right) = -\frac{2\sqrt{3}}{9}
            \qquad
            h\left(\frac{6- \sqrt{3}}{3} \right) = \frac{2\sqrt{3}}{9} = -h\left(\frac{6+ \sqrt{3}}{3} \right)
        \end{equation*}
        Por tanto,
        \begin{equation*}
            Im(|h|_{\left|[1,3]\right.}) = \left[0,\frac{2\sqrt{3}}{9}\right]
        \end{equation*}

        Por tanto, tenemos que
        \begin{equation*}
            |e(x)| = \frac{|f^{3)}(\xi)|}{(3)!}\left|\prod_{k=0}^2 (x-x_k)\right|
            \leq
            \frac{2}{3!} \cdot \frac{2\sqrt{3}}{9} = \frac{2\sqrt{3}}{27}
        \end{equation*}

        siendo por tanto esa la cota.
    \end{enumerate}
\end{ejercicio}

\begin{ejercicio}
    Dados los valores $f(1.00) = 0.1924,\;f(1.05) = 0.2414,\;f(1.10) =~0.2933,$ $f(1.15) = 0.3492$
    \begin{enumerate}
        \item Calcule el polinomio de interpolación usando la fórmula de Newton utilizando aritmética de cuatro dígitos por redondeo.

        Calculo la tabla de diferencias divididas:
        \begin{equation*}
            \begin{array}{c|cccccc}
                x_i & f[x_i] \\
                \\
                1.00 & \textbf{0.1924} \\
                && \textbf{0.98}\\
                1.05 & 0.2414 && \textbf{0.58}\\
                && 1.038&&\mathbf{1.467}\\
                1.10 & 0.2933 && 0.8 &&\\
                && 1.118\\
                1.15 & 0.3492
            \end{array}
        \end{equation*}

        Por tanto, el polinomio queda:
        \begin{equation*}
            p_3(x) = 0.1924 + (x-1.00)[0.98+(x-1.05)[0.58+1.467(x-1.10)]]
        \end{equation*}
        
        \item Estime el valor de $f(1.09)$.

        Para minimizar los errores de redondeo, empleo el método de Newton-Horner:
        \begin{equation*}
            p_3(1.09) = 0.2827
        \end{equation*}
        \item Proporcione una acotación del error cometido en dicha estimación, sabiendo que los datos proceden de una función cuya derivada de orden 4, en valor absoluto, está acotada por $0.76$. Explique todos los pasos a seguir.

        El error viene dado por:
        \begin{multline*}
            |e(x)| = \frac{|f^{4)}(\xi)|}{4!}|(x-1)(x-1.05)(x-1.10)(x-1.15)|
            \leq \\ \leq
            \frac{0.76}{4!}|(x-1)(x-1.05)(x-1.10)(x-1.15)|
        \end{multline*}

        Evaluando en $x=1.09$ par calcular la cota del error cometido:
        \begin{equation*}
            |e(1.09)| \leq \frac{0.76}{4!}\cdot 2.16\cdot 10^{-6} = 6.84\cdot 10^{-8}
        \end{equation*}

        Como podemos ver, el error cometido ha sido muy bajo.        
    \end{enumerate}
\end{ejercicio}

\begin{ejercicio}
    Se consideran los datos de interpolación $f(0) = 0, f(\pi/2) = 1, f(\pi) =~0$, $f(3\pi/2) = -1, f(2\pi) = 0$.
    \begin{equation*}
        \begin{array}{c|ccccc}
            x_i & 0 & \frac{\pi}{2} & \pi & \frac{3\pi}{2} & 2\pi \\ \hline
            f_i & 0 & 1 & 0 & -1 & 0
        \end{array}
    \end{equation*}
    \begin{enumerate}
        \item Calcule el polinomio de interpolación usando la fórmula de Lagrange.

        Sabiendo que el polinomio de interpolación es:
        \begin{equation*}
            p_4(x) = \sum_{i=0}^4 f_i\ell_i(x) = \ell_1(x) - \ell_3(x)
        \end{equation*}
        Es decir, como $f_0=f_2=f_4 = 0$, no tenemos que calcular $\ell_0(x), \ell_2(x), \ell_4(x)$.
        \begin{equation*}
            \ell_1(x) = \frac{x(x-\pi)(x-\frac{3\pi}{2}) (x-2\pi)}{\left(\frac{\pi}{2}-0\right)\left(\frac{\pi}{2}-\pi\right)\left(\frac{\pi}{2}-\frac{3\pi}{2}\right)\left(\frac{\pi}{2}-2\pi\right)}
            = \frac{x(x-\pi)(x-\frac{3\pi}{2}) (x-2\pi)}{-\frac{3\pi^4}{8}}
        \end{equation*}
        \begin{equation*}
            \ell_3(x) = \frac{x(x-\frac{\pi}{2})(x-\pi) (x-2\pi)}{\left(\frac{3\pi}{2}-0\right)\left(\frac{3\pi}{2}-\frac{\pi}{2}\right)\left(\frac{3\pi}{2}-\pi\right)\left(\frac{3\pi}{2}-2\pi\right)}
            =\frac{x(x-\frac{\pi}{2})(x-\pi) (x-2\pi)}{-\frac{3\pi^4}{8}}
        \end{equation*}

        Por tanto,
        \begin{multline*}
            p_4(x) = \ell_1(x) - \ell_3(x) 
            = \frac{x(x-\pi)(x-\frac{3\pi}{2}) (x-2\pi)}{-\frac{3\pi^4}{8}} - \frac{x(x-\frac{\pi}{2})(x-\pi) (x-2\pi)}{-\frac{3\pi^4}{8}}
            =\\=
            \frac{-x(x-\pi)(x-\frac{3\pi}{2}) (x-2\pi)+x(x-\frac{\pi}{2})(x-\pi) (x-2\pi)}{\frac{3\pi^4}{8}}
        \end{multline*}
        
        \item Usando el apartado anterior, estime el valor de $f(\pi/4)$.
        \begin{multline*}
            p_4\left(\frac{\pi}{4}\right) = \frac{-\frac{\pi}{4}(\frac{\pi}{4}-\pi)(\frac{\pi}{4}-\frac{3\pi}{2}) (\frac{\pi}{4}-2\pi)+\frac{\pi}{4}(\frac{\pi}{4}-\frac{\pi}{2})(\frac{\pi}{4}-\pi) (\frac{\pi}{4}-2\pi)}{\frac{3\pi^4}{8}}
            =\\=
             \frac{-\frac{\pi}{4}(\frac{-3\pi}{4})(\frac{-5\pi}{4}) (\frac{-7\pi}{4})+\frac{\pi}{4}(\frac{-\pi}{4})(\frac{-3\pi}{4}) (\frac{-7\pi}{4})}{\frac{3\pi^4}{8}}
             = \frac{\frac{\pi^4}{4^4}(105-21)}{\frac{3\pi^4}{8}} = \frac{84\cdot 8}{4^4 \cdot 3} = \frac{7}{8}\approx 0.875 
        \end{multline*}


        
        \item Sabiendo que el valor absoluto de la función y de sus derivadas sucesivas está acotado por 1, esto es, $|f^{(n)}(x)| \leq 1$, para todo $x\in \bb{R},x\geq 0$, acote el error cometido en la estimación anterior, detallando todos los pasos que realice.

        Sabemos que:
        \begin{equation*}
            |e(x)| = \left|\frac{f^{(n+1)}(\xi)}{(n+1)!}\prod_{k=0}^n(x-x_n) \right|
        \end{equation*}

        Para $n=4$, y sabiendo los nodos:
        \begin{equation*}
            |e(x)| = \frac{|f^{(5)}(\xi)|}{5!}\left|x\left(x-\frac{\pi}{2}\right)(x-\pi)\left(x-\frac{3\pi}{2}\right)(x-2\pi) \right|
        \end{equation*}

        Como la derivada está acotada por 1,
        \begin{equation*}
            |e(x)| \leq \frac{1}{5!}\left|x\left(x-\frac{\pi}{2}\right)(x-\pi)\left(x-\frac{3\pi}{2}\right)(x-2\pi) \right|
        \end{equation*}

        Evaluando en $x=\frac{\pi}{4}$, tenemos la cota del error cometido.
        \begin{equation*}
            \left|e\left(\frac{\pi}{4}\right)\right| \leq \frac{1}{5!}\cdot \frac{\pi^5}{4^5}\cdot 105 = \frac{7}{2^{13}}\pi ^5 \approx 0.2615
        \end{equation*}
    \end{enumerate}
\end{ejercicio}

\begin{ejercicio}
    Aplique el algoritmo de Newton–Horner para aproximar $\sqrt{3}$ con los datos proporcionados por la función $f(x) =\sqrt{x}$ en los nodos $x_0 = 1, x_1 = 2, x_2 = 4, $ y $ x_3 = 5$. Proporcione una cota del error cometido.

    \begin{equation*}
        \begin{array}{c|cccc}
            x_i & 1 & 2 & 4 & 5 \\ \hline
            f_i & 1 & \sqrt{2} & 2 & \sqrt{5}
        \end{array}
    \end{equation*}

    Calculo, en primer lugar, la tabla de diferencias divididas:
    \begin{equation*}
        \begin{array}{c|cccccc}
            x_i & f[x_i] \\
            \\
            1 & \textbf{1} \\
            && \mathbf{\sqrt{2}}-1\\
            2 & \sqrt{2} && \mathbf{\frac{4-3\sqrt{2}}{6}}\\
            && \frac{2-\sqrt{2}}{2}&&\mathbf{\frac{\sqrt{5}-5+2\sqrt{2}}{12}}\\
            4 & 2 && \frac{2\sqrt{5}-6+\sqrt{2}}{6} &&\\
            && \sqrt{5}-2\\
            5 & \sqrt{5}
        \end{array}
    \end{equation*}

    Por tanto, el polinomio de interpolación es:
    \begin{equation*}
        \begin{split}
            p_4(x) &= 1+(\sqrt{2}-1)(x-1) +\frac{4-3\sqrt{2}}{6}(x-1)(x-2) + \frac{\sqrt{5}-5+2\sqrt{2}}{12}(x-1)(x-2)(x-4) \\
            &= 1+(x-1)\left[\sqrt{2}-1 +(x-2)\left[\frac{4-3\sqrt{2}}{6} +\frac{\sqrt{5}-5+2\sqrt{2}}{12}(x-4)\right]\right]
        \end{split}
    \end{equation*}

    Evalúo en $x=3$ mediante el algoritmo de Newton-Horner:
    \begin{multline*}
            p_4(3) = 1+2\left[\sqrt{2}-1 +\left[\frac{4-3\sqrt{2}}{6} -\frac{\sqrt{5}-5+2\sqrt{2}}{12}\right]\right]
            = 1+2\left[\sqrt{2}-1 +\left[\frac{-\sqrt{5}+13-8\sqrt{2}}{12}\right]\right] =\\
            = 1+2\left[\frac{-\sqrt{5}+1+4\sqrt{2}}{12}\right]
            = \frac{-\sqrt{5}+7+4\sqrt{2}}{6} \approx 1.7368
    \end{multline*}

    Acotamos ahora el error cometido. Sabemos que:
    \begin{equation*}
        |e(x)| = \left|\frac{f^{(n+1)}(\xi)}{(n+1)!}\prod_{k=0}^n(x-x_n) \right|
    \end{equation*}

    Para $n=3$, y sabiendo los nodos:
    \begin{equation*}
        |e(x)| = \frac{|f^{(4)}(\xi)|}{4!}\left|(x-1)(x-2)(x-4)(x-5) \right| \qquad \xi \in [1,5]
    \end{equation*}

    Acotamos ahora la derivada de orden 4 de $\sqrt{x}$:
    \begin{equation*}
        f'(x)=\frac{1}{2}x^{-\frac{1}{2}}
        \qquad f''(x)=-\frac{1}{4}x^{-\frac{3}{2}}
        \qquad f'''(x)=\frac{3}{8}x^{-\frac{5}{2}}
        \qquad f^{4)}(x)=-\frac{15}{16}x^{-\frac{7}{2}}
    \end{equation*}

    Como $f^{4)}(x)$ es estrictamente creciente en $\bb{R}^+$ pero $f^{4)}(x)<0\;\forall x\in \bb{R}^+$, tengo que $|f^{4)}(x)|$ es estrictamente decreciente en $\bb{R}^+$:
    \begin{equation*}
        |f^{4)}(x)|\leq |f^{4)}(1)| = \frac{15}{16} \qquad \forall x\in [1,5]
    \end{equation*}

    Por tanto, tenemos que:
    \begin{equation*}
        |e(x)| \leq \frac{\frac{15}{16}}{4!}\left|(x-1)(x-2)(x-4)(x-5) \right| = \frac{5}{2^7}\left|(x-1)(x-2)(x-4)(x-5) \right|
    \end{equation*}
    

    Evaluando en $x=3$, tenemos la cota del error cometido.
    \begin{equation*}
        \left|e\left(3\right)\right| \leq \frac{5}{2^7}\left|4 \right| = \frac{5}{2^5} \approx 0.15625
    \end{equation*}
\end{ejercicio}

\begin{ejercicio}
    Se consideran los datos $f(-1) = f(1) = 0,\;f(0) = f(2) = 1$.

    \begin{equation*}
        \begin{array}{c|cccc}
            x_i & -1 & 0 & 1 & 2 \\ \hline
            f_i & 0 & 1 & 0 & 1
        \end{array}
    \end{equation*}
    
    \begin{enumerate}
        \item Estime el valor de $f(0.5)$ utilizando el algoritmo de Newton–Horner.        

        Calculo, en primer lugar, la tabla de diferencias divididas:
        \begin{equation*}
            \begin{array}{c|cccccc}
                x_i & f[x_i] \\
                \\
                -1 & \textbf{0} \\
                && \textbf{1}\\
                0 & 1 && \textbf{-1}\\
                && -1&&\mathbf{\frac{2}{3}}\\
                1 & 0 && 1 &&\\
                && 1\\
                2 & 1
            \end{array}
        \end{equation*}

        Por tanto, el polinomio de interpolación es:
        \begin{equation*}
            \begin{split}
                p_4(x) &= (x+1) -x(x+1) +\frac{2}{3}(x+1)x(x-1) \\
                &= (x+1)\left[1+x\left(-1 +\frac{2}{3}(x-1)\right)\right]
            \end{split}
        \end{equation*}

        Usando el algoritmo de Newton-Horner, tenemos:
        \begin{equation*}
            p_4\left(\frac{1}{2}\right) = \frac{3}{2}\cdot \left[1+\frac{1}{2}\left(-1-\frac{2}{3}\cdot \frac{1}{2}\right)\right] 
            = \frac{3}{2}\cdot \left[1+\frac{1}{2}\left(-\frac{4}{3}\right)\right]
            = \frac{3}{2}\cdot \left[\frac{1}{3}\right] = \frac{1}{2}
        \end{equation*}


        
    
        \item Estime el error cometido, sabiendo que $|f^{(k)}(x)| < 0.3$, para todo $x$, y para cualquier orden de derivación $k$.

        Acotamos ahora el error cometido. Sabemos que:
        \begin{equation*}
            |e(x)| = \left|\frac{f^{(n+1)}(\xi)}{(n+1)!}\prod_{k=0}^n(x-x_n) \right|
        \end{equation*}
    
        Para $n=3$, y sabiendo los nodos:
        \begin{equation*}
            |e(x)| = \frac{|f^{(4)}(\xi)|}{4!}\left|(x+1)x(x-1)(x-2) \right| \qquad \xi \in [-1,2]
        \end{equation*}
    
        Como tenemos que $|f^{(k)}(x)|<0.3 \; \forall x,k$, tenemos que:
    
        Por tanto, tenemos que:
        \begin{equation*}
            |e(x)| < \frac{0.3}{4!}\left|(x+1)x(x-1)(x-2) \right|
        \end{equation*}
        
    
        Evaluando en $x=\frac{1}{2}$, tenemos la cota del error cometido.
        \begin{equation*}
            \left|e\left(\frac{1}{2}\right)\right| < \frac{0.3}{4!}\cdot \frac{3^2}{2^4} = \frac{3^2}{2^8\cdot 5} \approx 0.00703
        \end{equation*}
    \end{enumerate}
\end{ejercicio}