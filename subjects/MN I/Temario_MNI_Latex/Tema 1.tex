\chapter{Introducción a los Problemas del Análisis Numérico}

\section{Cálculo de la raíz de 2}
Se define la sucesión convergente siguiente:
$$x_0=2~~x_{n+1}=\dfrac{1}{2}\left(x_n+\dfrac{2}{x_n}\right)$$

\section{Algoritmo de Horner}
Para minimizar las operaciones al evaluar un polinomio:
$$p(x)=2x^5-7x^3+2x^2-x+1 = (((2x^2-7)x+2)x-1)x+1$$

Otra disposición:
$$
    \begin{tabular}{c|cccccc}
          & 2       & 0       & -7      & 2       & -1      & 1        \\
          &         &         &         &         &         &          \\
        x &         & $\cdot$ & $\cdot$ & $\cdot$ & $\cdot$ & $\cdot$  \\
        \hline
          & $\cdot$ & $\cdot$ & $\cdot$ & $\cdot$ & $\cdot$ & $a=p(x)$
    \end{tabular}
$$

\noindent
De forma normal se realizan $n$ sumas y $(n^2+n)/2$ multiplicaciones.\newline
Con el algoritmo de Horner se realizan $n$ sumas y $n$ multiplicaciones.

\section{Errores}
\noindent
Si $p^{*}$ es una aproximación de $p$, definimos:
\begin{itemize}
    \item Error absoluto como $|p-p^{*}|$
    \item Error relativo como $\dfrac{|p-p^{*}|}{|p|}$ (Para $p\neq0$)
\end{itemize}

\section{Aritmética finita}
$$x\oplus y =rd(rd(x)+rd(y))~~x\ominus y=rd(rd(x)-rd(y))$$
$$x\otimes y=rd(rd(x)rd(y))~~x \oslash y = rd\left(\dfrac{rd(x)}{rd(y)}\right)$$

\noindent
Notemos que no se cumplen las propiedades básicas:
$t=3, b=10, x=3410, y=4.87, z=4.92$
$$(x \oplus y) \oplus z = x \oplus z = x$$
$$x \oplus (y \oplus z) = x \oplus 9.79 = 3420$$

\noindent
A la hora de restar cantidades muy similares, es recomendable multiplicar por el conjugado.


\section{Ejercicios}
Los ejercicios relativos a este tema están disponbles en la sección \ref{sec:Rel1}.