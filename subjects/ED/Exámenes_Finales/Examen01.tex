\documentclass[12pt]{article}

% Idioma y codificación
\usepackage[spanish, es-tabla]{babel}       %es-tabla para que se titule "Tabla"
\usepackage[utf8]{inputenc}

% Márgenes
\usepackage[a4paper,top=3cm,bottom=2.5cm,left=3cm,right=3cm]{geometry}

% Comentarios de bloque
\usepackage{verbatim}

% Paquetes de links
\usepackage[hidelinks]{hyperref}    % Permite enlaces
\usepackage{url}                    % redirecciona a la web

% Más opciones para enumeraciones
\usepackage{enumitem}

% Personalizar la portada
\usepackage{titling}

% Paquetes de tablas
\usepackage{multirow}


%------------------------------------------------------------------------

%Paquetes de figuras
\usepackage{caption}
\usepackage{subcaption} % Figuras al lado de otras
\usepackage{float}      % Poner figuras en el sitio indicado H.


% Paquetes de imágenes
\usepackage{graphicx}       % Paquete para añadir imágenes
\usepackage{transparent}    % Para manejar la opacidad de las figuras

% Paquete para usar colores
\usepackage[dvipsnames]{xcolor}
\usepackage{pagecolor}      % Para cambiar el color de la página

% Habilita tamaños de fuente mayores
\usepackage{fix-cm}

% Para los gráficos
\usepackage{tikz}

% Para poder situar los nodos en los grafos
\usetikzlibrary{positioning}


%------------------------------------------------------------------------

% Paquetes de matemáticas
\usepackage{mathtools, amsfonts, amssymb, mathrsfs}
\usepackage[makeroom]{cancel}     % Simplificar tachando
\usepackage{polynom}    % Divisiones y Ruffini
\usepackage{units} % Para poner fracciones diagonales con \nicefrac

\usepackage{pgfplots}   %Representar funciones
\pgfplotsset{compat=1.18}  % Versión 1.18

\usepackage{tikz-cd}    % Para usar diagramas de composiciones
\usetikzlibrary{calc}   % Para usar cálculo de coordenadas en tikz

%Definición de teoremas, etc.
\usepackage{amsthm}
%\swapnumbers   % Intercambia la posición del texto y de la numeración

\theoremstyle{plain}

\makeatletter
\@ifclassloaded{article}{
  \newtheorem{teo}{Teorema}[section]
}{
  \newtheorem{teo}{Teorema}[chapter]  % Se resetea en cada chapter
}
\makeatother

\newtheorem{coro}{Corolario}[teo]           % Se resetea en cada teorema
\newtheorem{prop}[teo]{Proposición}         % Usa el mismo contador que teorema
\newtheorem{lema}[teo]{Lema}                % Usa el mismo contador que teorema

\theoremstyle{remark}
\newtheorem*{observacion}{Observación}

\theoremstyle{definition}

\makeatletter
\@ifclassloaded{article}{
  \newtheorem{definicion}{Definición} [section]     % Se resetea en cada chapter
}{
  \newtheorem{definicion}{Definición} [chapter]     % Se resetea en cada chapter
}
\makeatother

\newtheorem*{notacion}{Notación}
\newtheorem*{ejemplo}{Ejemplo}
\newtheorem*{ejercicio*}{Ejercicio}             % No numerado
\newtheorem{ejercicio}{Ejercicio} [section]     % Se resetea en cada section


% Modificar el formato de la numeración del teorema "ejercicio"
\renewcommand{\theejercicio}{%
  \ifnum\value{section}=0 % Si no se ha iniciado ninguna sección
    \arabic{ejercicio}% Solo mostrar el número de ejercicio
  \else
    \thesection.\arabic{ejercicio}% Mostrar número de sección y número de ejercicio
  \fi
}


% \renewcommand\qedsymbol{$\blacksquare$}         % Cambiar símbolo QED
%------------------------------------------------------------------------

% Paquetes para encabezados
\usepackage{fancyhdr}
\pagestyle{fancy}
\fancyhf{}

\newcommand{\helv}{ % Modificación tamaño de letra
\fontfamily{}\fontsize{12}{12}\selectfont}
\setlength{\headheight}{15pt} % Amplía el tamaño del índice


%\usepackage{lastpage}   % Referenciar última pag   \pageref{LastPage}
\fancyfoot[C]{\thepage}

%------------------------------------------------------------------------

% Conseguir que no ponga "Capítulo 1". Sino solo "1."
\makeatletter
\@ifclassloaded{book}{
  \renewcommand{\chaptermark}[1]{\markboth{\thechapter.\ #1}{}} % En el encabezado
    
  \renewcommand{\@makechapterhead}[1]{%
  \vspace*{50\p@}%
  {\parindent \z@ \raggedright \normalfont
    \ifnum \c@secnumdepth >\m@ne
      \huge\bfseries \thechapter.\hspace{1em}\ignorespaces
    \fi
    \interlinepenalty\@M
    \Huge \bfseries #1\par\nobreak
    \vskip 40\p@
  }}
}
\makeatother

%------------------------------------------------------------------------
% Paquetes de cógido
\usepackage{minted}
\renewcommand\listingscaption{Código fuente}

\usepackage{fancyvrb}
% Personaliza el tamaño de los números de línea
\renewcommand{\theFancyVerbLine}{\small\arabic{FancyVerbLine}}

% Estilo para C++
\newminted{cpp}{
    frame=lines,
    framesep=2mm,
    baselinestretch=1.2,
    linenos,
    escapeinside=||
}

% para minted
\definecolor{LightGray}{rgb}{0.95,0.95,0.92}
\setminted{
    linenos=true,
    stepnumber=5,
    numberfirstline=true,
    autogobble,
    breaklines=true,
    breakautoindent=true,
    breaksymbolleft=,
    breaksymbolright=,
    breaksymbolindentleft=0pt,
    breaksymbolindentright=0pt,
    breaksymbolsepleft=0pt,
    breaksymbolsepright=0pt,
    fontsize=\footnotesize,
    bgcolor=LightGray,
    numbersep=10pt
}


\usepackage{listings} % Para incluir código desde un archivo

\renewcommand\lstlistingname{Código Fuente}
\renewcommand\lstlistlistingname{Índice de Códigos Fuente}

% Definir colores
\definecolor{vscodepurple}{rgb}{0.5,0,0.5}
\definecolor{vscodeblue}{rgb}{0,0,0.8}
\definecolor{vscodegreen}{rgb}{0,0.5,0}
\definecolor{vscodegray}{rgb}{0.5,0.5,0.5}
\definecolor{vscodebackground}{rgb}{0.97,0.97,0.97}
\definecolor{vscodelightgray}{rgb}{0.9,0.9,0.9}

% Configuración para el estilo de C similar a VSCode
\lstdefinestyle{vscode_C}{
  backgroundcolor=\color{vscodebackground},
  commentstyle=\color{vscodegreen},
  keywordstyle=\color{vscodeblue},
  numberstyle=\tiny\color{vscodegray},
  stringstyle=\color{vscodepurple},
  basicstyle=\scriptsize\ttfamily,
  breakatwhitespace=false,
  breaklines=true,
  captionpos=b,
  keepspaces=true,
  numbers=left,
  numbersep=5pt,
  showspaces=false,
  showstringspaces=false,
  showtabs=false,
  tabsize=2,
  frame=tb,
  framerule=0pt,
  aboveskip=10pt,
  belowskip=10pt,
  xleftmargin=10pt,
  xrightmargin=10pt,
  framexleftmargin=10pt,
  framexrightmargin=10pt,
  framesep=0pt,
  rulecolor=\color{vscodelightgray},
  backgroundcolor=\color{vscodebackground},
}

%------------------------------------------------------------------------

% Comandos definidos
\newcommand{\bb}[1]{\mathbb{#1}}
\newcommand{\cc}[1]{\mathcal{#1}}

% I prefer the slanted \leq
\let\oldleq\leq % save them in case they're every wanted
\let\oldgeq\geq
\renewcommand{\leq}{\leqslant}
\renewcommand{\geq}{\geqslant}

% Si y solo si
\newcommand{\sii}{\iff}

% Letras griegas
\newcommand{\eps}{\epsilon}
\newcommand{\veps}{\varepsilon}
\newcommand{\lm}{\lambda}

\newcommand{\ol}{\overline}
\newcommand{\ul}{\underline}
\newcommand{\wt}{\widetilde}
\newcommand{\wh}{\widehat}

\let\oldvec\vec
\renewcommand{\vec}{\overrightarrow}

% Derivadas parciales
\newcommand{\del}[2]{\frac{\partial #1}{\partial #2}}
\newcommand{\Del}[3]{\frac{\partial^{#1} #2}{\partial #3^{#1}}}
\newcommand{\deld}[2]{\dfrac{\partial #1}{\partial #2}}
\newcommand{\Deld}[3]{\dfrac{\partial^{#1} #2}{\partial #3^{#1}}}


\newcommand{\AstIg}{\stackrel{(\ast)}{=}}
\newcommand{\Hop}{\stackrel{L'H\hat{o}pital}{=}}

\newcommand{\red}[1]{{\color{red}#1}} % Para integrales, destacar los cambios.

% Método de integración
\newcommand{\MetInt}[2]{
    \left[\begin{array}{c}
        #1 \\ #2
    \end{array}\right]
}

% Declarar aplicaciones
% 1. Nombre aplicación
% 2. Dominio
% 3. Codominio
% 4. Variable
% 5. Imagen de la variable
\newcommand{\Func}[5]{
    \begin{equation*}
        \begin{array}{rrll}
            #1:& #2 & \longrightarrow & #3\\
               & #4 & \longmapsto & #5
        \end{array}
    \end{equation*}
}

%------------------------------------------------------------------------

\newcommand{\T}[0]{\cc{T}}

\begin{document}

    % 1. Foto de fondo
    % 2. Título
    % 3. Encabezado Izquierdo
    % 4. Color de fondo
    % 5. Coord x del titulo
    % 6. Coord y del titulo
    % 7. Fecha

    
    % 1. Foto de fondo
% 2. Título
% 3. Encabezado Izquierdo
% 4. Color de fondo
% 5. Coord x del titulo
% 6. Coord y del titulo
% 7. Fecha

\newcommand{\portada}[7]{

    \portadaBase{#1}{#2}{#3}{#4}{#5}{#6}{#7}
    \portadaBook{#1}{#2}{#3}{#4}{#5}{#6}{#7}
}

\newcommand{\portadaExamen}[7]{

    \portadaBase{#1}{#2}{#3}{#4}{#5}{#6}{#7}
    \portadaArticle{#1}{#2}{#3}{#4}{#5}{#6}{#7}
}




\newcommand{\portadaBase}[7]{

    % Tiene la portada principal y la licencia Creative Commons
    
    % 1. Foto de fondo
    % 2. Título
    % 3. Encabezado Izquierdo
    % 4. Color de fondo
    % 5. Coord x del titulo
    % 6. Coord y del titulo
    % 7. Fecha
    
    
    \thispagestyle{empty}               % Sin encabezado ni pie de página
    \newgeometry{margin=0cm}        % Márgenes nulos para la primera página
    
    
    % Encabezado
    \fancyhead[L]{\helv #3}
    \fancyhead[R]{\helv \nouppercase{\leftmark}}
    
    
    \pagecolor{#4}        % Color de fondo para la portada
    
    \begin{figure}[p]
        \centering
        \transparent{0.3}           % Opacidad del 30% para la imagen
        
        \includegraphics[width=\paperwidth, keepaspectratio]{assets/#1}
    
        \begin{tikzpicture}[remember picture, overlay]
            \node[anchor=north west, text=white, opacity=1, font=\fontsize{60}{90}\selectfont\bfseries\sffamily, align=left] at (#5, #6) {#2};
            
            \node[anchor=south east, text=white, opacity=1, font=\fontsize{12}{18}\selectfont\sffamily, align=right] at (9.7, 3) {\textbf{\href{https://losdeldgiim.github.io/}{Los Del DGIIM}}};
            
            \node[anchor=south east, text=white, opacity=1, font=\fontsize{12}{15}\selectfont\sffamily, align=right] at (9.7, 1.8) {Doble Grado en Ingeniería Informática y Matemáticas\\Universidad de Granada};
        \end{tikzpicture}
    \end{figure}
    
    
    \restoregeometry        % Restaurar márgenes normales para las páginas subsiguientes
    \pagecolor{white}       % Restaurar el color de página
    
    
    \newpage
    \thispagestyle{empty}               % Sin encabezado ni pie de página
    \begin{tikzpicture}[remember picture, overlay]
        \node[anchor=south west, inner sep=3cm] at (current page.south west) {
            \begin{minipage}{0.5\paperwidth}
                \href{https://creativecommons.org/licenses/by-nc-nd/4.0/}{
                    \includegraphics[height=2cm]{assets/Licencia.png}
                }\vspace{1cm}\\
                Esta obra está bajo una
                \href{https://creativecommons.org/licenses/by-nc-nd/4.0/}{
                    Licencia Creative Commons Atribución-NoComercial-SinDerivadas 4.0 Internacional (CC BY-NC-ND 4.0).
                }\\
    
                Eres libre de compartir y redistribuir el contenido de esta obra en cualquier medio o formato, siempre y cuando des el crédito adecuado a los autores originales y no persigas fines comerciales. 
            \end{minipage}
        };
    \end{tikzpicture}
    
    
    
    % 1. Foto de fondo
    % 2. Título
    % 3. Encabezado Izquierdo
    % 4. Color de fondo
    % 5. Coord x del titulo
    % 6. Coord y del titulo
    % 7. Fecha


}


\newcommand{\portadaBook}[7]{

    % 1. Foto de fondo
    % 2. Título
    % 3. Encabezado Izquierdo
    % 4. Color de fondo
    % 5. Coord x del titulo
    % 6. Coord y del titulo
    % 7. Fecha

    % Personaliza el formato del título
    \pretitle{\begin{center}\bfseries\fontsize{42}{56}\selectfont}
    \posttitle{\par\end{center}\vspace{2em}}
    
    % Personaliza el formato del autor
    \preauthor{\begin{center}\Large}
    \postauthor{\par\end{center}\vfill}
    
    % Personaliza el formato de la fecha
    \predate{\begin{center}\huge}
    \postdate{\par\end{center}\vspace{2em}}
    
    \title{#2}
    \author{\href{https://losdeldgiim.github.io/}{Los Del DGIIM}}
    \date{Granada, #7}
    \maketitle
    
    \tableofcontents
}




\newcommand{\portadaArticle}[7]{

    % 1. Foto de fondo
    % 2. Título
    % 3. Encabezado Izquierdo
    % 4. Color de fondo
    % 5. Coord x del titulo
    % 6. Coord y del titulo
    % 7. Fecha

    % Personaliza el formato del título
    \pretitle{\begin{center}\bfseries\fontsize{42}{56}\selectfont}
    \posttitle{\par\end{center}\vspace{2em}}
    
    % Personaliza el formato del autor
    \preauthor{\begin{center}\Large}
    \postauthor{\par\end{center}\vspace{3em}}
    
    % Personaliza el formato de la fecha
    \predate{\begin{center}\huge}
    \postdate{\par\end{center}\vspace{5em}}
    
    \title{#2}
    \author{\href{https://losdeldgiim.github.io/}{Los Del DGIIM}}
    \date{Granada, #7}
    \thispagestyle{empty}               % Sin encabezado ni pie de página
    \maketitle
    \vfill
}
    \portadaExamen{etsiitA4.jpg}{Estructura\\de Datos\\Examen I}{Estructura de Datos. Examen I}{MidnightBlue}{-8}{28}{2023-2024}{José Juan Urrutia Milán}

    \begin{description}
        \item[Asignatura] Estructura de Datos.
        \item[Curso Académico] 2023-24.
        \item[Grado] Doble Grado en Ingeniería Informática y Matemáticas.
        \item[Grupo] Único.
        \item[Profesor] Joaquín Fernández Valdivia.
        \item[Descripción] Convocatoria Ordinaria.
        %\item[Fecha] 10 de noviembre de 2023.
        \item[Duración] 2 horas y media.
    
    \end{description}
    \newpage
    
    \begin{ejercicio}[1 punto] 
        Elegir en cada caso la opción correcta, de forma justificada.
        \begin{enumerate}[label=(\alph*)]
            \item Si inserto las claves $\{2,4,5,6,12,1,3\}$ en un \textbf{APO} de enteros:
                \begin{enumerate}[label=(a\arabic*)]
                    \item Hay que hacer un solo intercambio padre-hijo.
                    \item Hay que hacer dos intercambios padre-hijo.
                    \item Hay que hacer tres intercambios padre-hijo.
                    \item Todo lo anterior es falso.
                \end{enumerate}
                \textbf{Mostrar el árbol final}

            \item Dadas las siguientes 3 afirmaciones:
                \begin{itemize}
                    \item Es correcto en un esquema de \textbf{hashing cerrado} el uso como función hash de:
                        \begin{equation*}
                            h(k) = [k + 2k] \% M \qquad M \text{\ primo}
                        \end{equation*}
                    \item La declaración \verb|map<list<int>, string> m;| es una declaración válida.
                    \item El elemento de valor máximo en un \verb|ABB<int>| se encuentra en el nodo de más profundidad.
                \end{itemize}
                \begin{enumerate}[label=(b\arabic*)]
                    \item Todas son falsas.
                    \item Hay 2 ciertas y 1 falsa.
                    \item Hay 1 cierta y 2 falsas.
                    \item Todas son ciertas.
                \end{enumerate}

            \item Dados los siguientes recorridos Preorden y Postorden:
                \begin{equation*}
                    Pre = \{A,Z,X,Q,V,Y,L,W,T,R\} \quad Post = \{Q,V,X,Y,L,Z,T,R,W,A\}
                \end{equation*}
                \begin{enumerate}[label=(c\arabic*)]
                    \item Hay exactamente 2 árboles binarios con esos recorridos.
                    \item No hay ningún árbol binario con esos recorridos.
                    \item Hay exactamente 1 árbol binario con esos recorridos.
                    \item Hay más de 2 árboles binarios con esos recorridos.
                \end{enumerate}
                \textbf{Razona la respuesta}

            \item Dados dos nodos $n_1$ y $n_2$ en un árbol binario $T$ y dadas las distancias (longitudes de los camino) $m_1$ y $m_2$ de ambos nodos a su antecesor común más cercano (nodo más profundo que tiene tanto a $n_1$ como a $n_2$ como descendientes):
                \begin{enumerate}[label=(d\arabic*)]
                    \item Si $m_1=m_2=1$ los nodos son el mismo nodo.
                    \item Si $m_1=0$ y $m_2>0$: $n_2$ es sucesor de $n_1$.
                    \item Si $m_1=m_2=2$ los nodos no son hermanos.
                    \item Todo lo anterior es cierto.
                \end{enumerate}
        \end{enumerate}
    \end{ejercicio}

    \begin{ejercicio}[1 punto]
        Supongamos que respresentamos una lista usando un vector de la siguiente forma:
        \begin{listing}[H]
        \begin{minted}[xleftmargin=4cm, linenos]{c++}
class listacursores{
    private:
    struct dato{char elem; int siguiente;};

    vector<dato> elementos;
    int primero;
    int nelems;

    public:
    // ...
};
        \end{minted}
        \end{listing}
    \end{ejercicio}
    donde el campo \verb|elem| es el elemento de cada posición de la lista, y \verb|siguiente| indica la posición dentro del vector en que está el siguiente elemento de la lista. Ejemplo: El vector:
    \begin{table}[H]
    \centering
    \begin{tabular}{|c|c|c|c|c|}
        \hline
        0 & 1 & 2 & 3 & 4 \\
        \hline
        a,4 & d,5 & b,3 & e,1 & c,2 \\
        \hline
    \end{tabular}
    \end{table}
    con: \verb|nelems = 5| \verb|primero = 0|, representa la lista (en orden) \verb|L: <a, c, b, e, d>|.

    Dada dicha representación de listas donde la posición de cada elemento viene determinada por un número entero, \textbf{construir una clase iteradora}, de forma que los elementos listados por el iterador deben aparecer en el orden en que están en la lista (independientemente de cómo estén almacenados en el vector). Para hacerlo correctamente, deben implementarse constructor, \verb|*|, \verb|==|, \verb|!=|, \verb|++|, junto con las funciones \verb|begin()| y \verb|end()| de la clase \verb|listacursores|.
    
    \begin{ejercicio}[1 punto]
        Implementar una función:
        \begin{minted}[xleftmargin=1cm]{c++}
void divide_por_signo(list<int> &L, vector<list<int> > &VL);
        \end{minted}
        que dada una lista \verb|L|, devuelve en el vector de listas \verb|VL| las sublistas contiguas del mismo signo (el 0 se considera junto con los positivos). El algoritmo puede modificar a \verb|L|.

        Ejemplos:
        \begin{gather*}
            L=\{4,-3,-5,-4,-5,-1,4,-1,-5,-5\} \Rightarrow \\ 
            \Rightarrow VL=[\{4\}, \{-3,-5,-4,-5,-1\}, \{4\}, \{-1,-5,-5\}] \\ \\
            L=\{0,4,-2,4,1,-1,-4,-4,-3,-1,-4,4,1\} \Rightarrow \\ 
            \Rightarrow VL=[\{0,4\},\{-2\},\{4,1\},\{-1,-4,-4,-3,-1,-4\},\{4,1\}] \\ \\
            L=\{2,-1,3,-3,3,-3,0,-1,0\} \Rightarrow \\ 
            \Rightarrow VL=[\{2\},\{-1\},\{3\},\{-3\},\{3\},\{-3\},\{0\},\{-1\},\{0\}] \\
        \end{gather*}
        
    \end{ejercicio}

    \begin{ejercicio}[1 punto]
        Implementar la función
        \begin{minted}[xleftmargin=1cm]{c++}
void Fibonacci_Trees(vector<bintree<int>> &v, int n);
        \end{minted}
        que construye la sucesión de árboles binarios de Fibonacci y los almacena en un vector de árboles. La sucesión comienza con un árbol con 1 solo nodo ($T_0$) y un árbol con un solo hijo a la derecha ($T_1$). A partir de ellos, se construye la sucesión construyendo cada árbol binario $T_i$ insertando $T_{i-1}$ a la derecha y $T_{i-2}$ a la izquierda (para $i=2,\ldots,n$). La etiqueta de la raíz del nuevo árbol se obtiene como la suma de las etiquetas de las raíces de los árboles izquierdo y derecho.

        Ejemplo: 
    \begin{figure}[H]
    \centering
    \begin{minipage}{0.45\textwidth}
        \centering
        \begin{tikzpicture}
            \node[draw, circle] (A) at (0,0) {5};
        \end{tikzpicture}
        \caption{T0}
    \end{minipage}\hfill
    \begin{minipage}{0.45\textwidth}
        \centering
        \begin{tikzpicture}
            \node[draw, circle] (A) at (0,1) {2};
            \node[draw, circle] (B) at (1,0) {4};
            \draw (A) -- (B);
        \end{tikzpicture}
        \caption{T1}
    \end{minipage}
\end{figure}

% Segunda fila de figuras
\begin{figure}[H]
    \centering
    \begin{minipage}{0.45\textwidth}
        \centering
        \begin{tikzpicture}
            \node[draw, circle] (A) at (0,2) {7};
            \node[draw, circle] (B) at (-1,1) {5};
            \node[draw, circle] (C) at (1,1) {2};
            \node[draw, circle] (D) at (2,0) {4};
            \draw (A) -- (B);
            \draw (A) -- (C);
            \draw (C) -- (D);
        \end{tikzpicture}
        \caption{T2}
    \end{minipage}\hfill
    \begin{minipage}{0.45\textwidth}
        \centering
        \begin{tikzpicture}
            \node[draw, circle] (A) at (1,3) {9};
            \node[draw, circle] (B) at (0,2) {2};
            \node[draw, circle] (C) at (0.5,1) {4};
            \node[draw, circle] (D) at (2,2) {7};
            \node[draw, circle] (E) at (1.5,1) {5};
            \node[draw, circle] (F) at (3,1) {2};
            \node[draw, circle] (G) at (4,0) {4};
            \draw (A) -- (B);
            \draw (B) -- (C);
            \draw (A) -- (D);
            \draw (D) -- (E);
            \draw (D) -- (F);
            \draw (F) -- (G);
        \end{tikzpicture}
        \caption{T3}
    \end{minipage}
\end{figure}
    \end{ejercicio}

    \begin{ejercicio}[1 punto]
        Dados dos \verb|map|, \verb|M1| y \verb|M2|, definidos como:
        \begin{minted}[xleftmargin=5cm]{c++}
map<string,int> M1, M2;
        \end{minted}
        con el primer campo representando el nombre de una \textbf{persona} (\verb|string|) y el segundo campo su \textbf{número de seguidores} (\verb|int|) en una red social, \textbf{implementar una función:}
        \begin{minted}{c++}
map<string,int> Union (const map<string,int> &M1, const map<string,int> &M2);
        \end{minted}
        que obtenga el \verb|map| correspondiente a la unión de los dos \verb|map| de entrada, en el que el número de seguidores será la suma de los seguidores en \verb|M1| y los seguidores en \verb|M2| para la misma persona que aparece en \verb|M1| y \verb|M2|. En el caso que solamente aparezca en uno de los dos se queda tal cual en el \verb|map| resultado.
        
    \end{ejercicio}

    \begin{ejercicio}[1 punto]\ 
        \begin{enumerate}[label=(\alph*)]
            \item Insertar en el orden indicado (detallando los pasos) los siguientes claves en un \textbf{AVL}: $\{45,30,48,65,49,51,81,37,6,62,52,73\}$. Borrar el elemento $49$ del árbol.
            \item Insertar (detallando los pasos) las claves $\{8,16,12,41,10,62,27,65,13\}$ en una \textbf{Tabla Hash cerrada} de tamaño $13$. A continuación, borrar el $10$ y finalmente insertar el valor $51$. Resolver las colisiones usando hashing doble.
        \end{enumerate}
    \end{ejercicio}

\end{document}
