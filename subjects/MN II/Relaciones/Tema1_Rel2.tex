\subsection{Relación 2}
\setcounter{ejercicio}{0}


\begin{ejercicio}
    Sea la función $f(x) = e^x - ax^2$ con $a \in [3, 4]$.
    \begin{enumerate}
        \item Demuestra que tiene una raíz negativa, otra raíz en $[0, 1]$ y otra mayor que 1.\\
        
        Evaluamos $f$ en los extremos del intervalo $[0,1]$:
        \begin{align*}
            f(0) &= e^0 - a\cdot 0^2 = 1\\
            f(1) &= e^1 - a\cdot 1^2 = e - a<0 \Longleftrightarrow e < a
        \end{align*}

        Calculamos además los límites de $f$ en $\pm \infty$:
        \begin{align*}
            \lim_{x\to -\infty} f(x) &= \lim_{x\to -\infty} e^x - ax^2 = \lim_{x\to -\infty} -ax^2 = -\infty\\
            \lim_{x\to \infty} f(x) &= \lim_{x\to \infty} e^x - ax^2 = \infty
        \end{align*}

        Para demostrar lo pedido, y debido a que $f$ es continua, emplearemos el Teorema de Bolzano.
        \begin{itemize}
            \item Como $\lim\limits_{x\to -\infty} f(x) = -\infty$ y $f(0) = 1>0$, por el Teorema de Bolzano, $f$ tiene una raíz en $\bb{R}^-$.
            \item Como $f(0) = 1>0$ y $f(1) = e - a < 0$, por el Teorema de Bolzano, $f$ tiene una raíz en $[0,1]$.
            \item Como $f(1) = e - a < 0$ y $\lim\limits_{x\to \infty} f(x) = \infty$, por el Teorema de Bolzano, $f$ tiene una raíz en $\left[1, \infty\right[$.
        \end{itemize}
        \item Demuestra que $x = g_1(x) = \sqrt{\frac{e^x}{a}}$ y $x = g_2(x) = -\sqrt{\frac{e^x}{a}}$ son ecuaciones equivalentes a la de partida.
        
        Partiendo de las ecuaciones dadas, elevamos al cuadrado ambos lados, llegando en ambos casos a:
        \begin{align*}
            x^2 &= \frac{e^x}{a} \iff a\cdot x^2 = e^x \iff e^x - ax^2 = 0
        \end{align*}

        Por tanto, efectivamente son equivalentes. La ecuación $x=g_1(x)$ tiene sentido para $x\geq 0$ y la ecuación $x=g_2(x)$ para $x\leq 0$.
        \item Toma $a = 3$. Demuestra la convergencia local hacia la raíz próxima a $-0.5$ partiendo de $x_0 = 0$ usando $g_2(x)$ y realiza dos iteraciones.\\
        
        Trabajaremos en el intervalo $[-1,0]$. Veamos que $g_2\left([-1,0]\right) \subset [-1,0]$. Para ello, como $g_2\in C^\infty(\bb{R})$, calculamos su derivada:
        \begin{align*}
            g_2'(x) = -\frac{1}{2\sqrt{\frac{e^x}{a}}}\cdot \frac{e^x}{a} = -\frac{e^x}{2a\sqrt{\frac{e^x}{a}}}<0\qquad \forall x\in \bb{R}
        \end{align*}
        Por tanto, tenemos que $g_2$ es continua y estrictamente decreciente en $\bb{R}$. Además, evaluamos $g_2$ en los extremos del intervalo:
        \begin{align*}
            g_2(-1) &= -\sqrt{\frac{e^{-1}}{3}} = -\frac{1}{\sqrt{3e}}\approx -0.35\\
            g_2(0) &= -\sqrt{\frac{e^0}{3}} = -\frac{1}{\sqrt{3}}\approx -0.58
        \end{align*}

        Por tanto, $g_2\left([-1,0]\right)=\left[-\frac{1}{\sqrt{3}}, -\frac{1}{\sqrt{3e}}\right]\subset [-1,0]$. Por tanto, podemos considerar $g_2: [-1,0]\to [-1,0]$.
        Veamos ahora que $g_2$ es una contracción en $[-1,0]$:
        \begin{equation*}
            |g_2'(x)|=\left|\frac{e^x}{6\sqrt{\frac{e^x}{3}}}\right|=\frac{e^x}{6\sqrt{\frac{e^x}{3}}}\leq 
            \frac{e^0}{6\sqrt{\frac{e^{-1}}{3}}}\approx 0.47<1\qquad \forall x\in [-1,0]
        \end{equation*}

        Por tanto, $g_2$ es una contracción en $[-1,0]$. Por el Teorema del Punto Fijo, $g_2$ tiene un único punto fijo en $[-1,0]$. Además, la sucesión $x_{n+1}=g_2(x_n)$ converge a dicho punto fijo para cualquier $x_0\in [-1,0]$.\\
        Veamos ahora las primeras dos iteraciones tomando $x_0=0$:
        \begin{equation*}
            \begin{array}{c|c|c}
                n & x_n & g_2(x_n)\\ \hline
                0 & 0 & -0.57735\\
                1 & -0.57735 & -0.4325829\\
                2 & -0.4325829 & -0.4650559
            \end{array}
        \end{equation*}

        
        \item Toma $a = 3$. Demuestra la convergencia local hacia la raíz próxima a 1 partiendo de $x_0 = 0$ usando $g_1(x)$ y realiza dos iteraciones.\\
        
        Trabajaremos en el intervalo $[0,2]$. Veamos que $g_1\left([0,2]\right) \subset [0,2]$. Para ello, como $g_1\in C^\infty(\bb{R})$, calculamos su derivada:
        \begin{align*}
            g_1'(x) = -g_2'(x) = \dfrac{e^x}{2a\sqrt{\frac{e^x}{a}}}>0\qquad \forall x\in \bb{R}
        \end{align*}
        Por tanto, tenemos que $g_1$ es continua y estrictamente creciente en $\bb{R}$. Además, evaluamos $g_1$ en los extremos del intervalo:
        \begin{align*}
            g_1(0) &= -g_2(0) = \sqrt{\frac{e^0}{3}} = \frac{1}{\sqrt{3}}\approx 0.58\\
            g_1(2) &= \sqrt{\frac{e^2}{3}} = \frac{e}{\sqrt{3}}\approx 1.5694
        \end{align*}

        Por tanto, $g_1\left([0,2]\right)=\left[\frac{1}{\sqrt{3}}, \frac{e}{\sqrt{3}}\right]\subset [0,2]$. Por tanto, podemos considerar $g_1: [0,2]\to [0,2]$.
        Veamos ahora que $g_1$ es una contracción en $[0,2]$:
        \begin{equation*}
            |g_1'(x)|=|g_2'(x)|=\dfrac{e^x}{6\sqrt{\frac{e^x}{3}}}<1\Longleftrightarrow
            e^{2x}<36\cdot \dfrac{e^x}{3}
            \iff e^x<12\iff x<\ln 12\approx 2.48
        \end{equation*}
        Por tanto, $g_1$ es una contracción en $[0,2]$. Por el Teorema del Punto Fijo, $g_1$ tiene un único punto fijo en $[0,2]$. Además, la sucesión $x_{n+1}=g_1(x_n)$ converge a dicho punto fijo para cualquier $x_0\in [0,2]$.\\
        Veamos ahora las primeras dos iteraciones tomando $x_0=0$:
        \begin{equation*}
            \begin{array}{c|c|c}
                n & x_n & g_1(x_n)\\ \hline
                0 & 0 & 0.57735\\
                1 & 0.57735 & 0.770565\\
                2 & 0.770565 & 0.848722
            \end{array}
        \end{equation*}
        \item Toma $a = 3$. Comprueba que la raíz mayor que 1 está en $[3, 4]$. Demuestra la no convergencia hacia la raíz próxima a 4 partiendo de $x_0$ muy próximo a ella (pero diferente de ella) usando $g_1(x)$ y encuentra una función para la iteración funcional, alternativa a las anteriores que converja a la raíz cercana a 4. Partiendo de $x_0 = 3.98$ obtén $x_1$ y $x_2$ con el método propuesto.\\
        
        Evaluamos $f$ en los extremos del intervalo $[3,4]$:
        \begin{align*}
            f(3) &= e^3 - 3\cdot 3^2 = e^3 - 3^3<0\\
            f(4) &= e^4 - 3\cdot 4^2 = e^4 - 3\cdot 4^2>0
        \end{align*}

        Por tanto, por el Teorema de Bolzano, $f$ tiene una raíz en $[3,4]$.\\

        Para estudiar la no convergencia hacia la raíz próxima a $4$ partiendo de $x_0$ muy próximo a ella, anteriormente vimos que:
        \begin{equation*}
            |g'(x)|<1\iff x<\ln 12\approx 2.48
        \end{equation*}

        Por tanto, como el punto fijo $s$ está en $[3,4]$, tenemos que $g'(s)>1$. Por tanto, la sucesión $x_{n+1}=g(x_n)$ no converge a $s$ si $x_0$ es muy próximo a $s$.\\

        Para encontrar una función que converja a la raíz próxima a 4, tenemos que:
        \begin{equation*}
            f(x)=0\iff
            e^x=ax^2\iff
            x=\ln(ax^2)
        \end{equation*}

        Por tanto, consideramos la función $h(x)=\ln(3x^2)$. Veamos que $h\left([3,5]\right)\subset [3,5]$. Para ello, calculamos la derivada de $h$:
        \begin{align*}
            h'(x) = \dfrac{6x}{3x^2} = \dfrac{2}{x}>0\qquad \forall x\in \bb{R}^+
        \end{align*}

        Por tanto, $h$ es continua y estrictamente creciente en $[3,5]$. Además, evaluamos $h$ en los extremos del intervalo:
        \begin{align*}
            h(3) &= \ln(3\cdot 3^2) = \ln 27\approx 3.3\\
            h(5) &= \ln(3\cdot 5^2) = \ln 75\approx 4.32
        \end{align*}

        Por tanto, $h\left([3,5]\right)=\left[\ln 27, \ln 75\right]\subset [3,5]$. Por tanto, podemos considerar $h: [3,5]\to [3,5]$.\\
        Veamos ahora que $h$ es una contracción en $[3,5]$:
        \begin{equation*}
            |h'(x)|=\left|\dfrac{2}{x}\right|=\dfrac{2}{x}<\frac{2}{3}<1\qquad \forall x\in [3,5]
        \end{equation*}
        Por tanto, $h$ es una contracción en $[3,5]$. Por el Teorema del Punto Fijo, $h$ tiene un único punto fijo en $[3,5]$. Además, la sucesión $x_{n+1}=h(x_n)$ converge a dicho punto fijo para cualquier $x_0\in [3,5]$.\\
        Veamos ahora las primeras dos iteraciones tomando $x_0=3.98$:
        \begin{equation*}
            \begin{array}{c|c|c}
                n & x_n & h(x_n)\\ \hline
                0 & 3.98 & 3.861176\\
                1 & 3.861176 & 3.800556\\
                2 & 3.800556 & 3.7689
            \end{array}
        \end{equation*}
    \end{enumerate}
\end{ejercicio}

\begin{ejercicio}
    Sea la ecuación $p(x) = x^3 - 8x^2 + 20x - 15.2 = 0$.
    \begin{enumerate}
        \item Prueba que no tiene ninguna raíz menor que 1.
        
        Como $p\in C^\infty(\bb{R})$, podemos calcular la derivada de $p$ y estudiar su signo:
        \begin{multline*}
            p'(x) = 3x^2 - 16x + 20 = 0
            \iff \\ \iff x = \frac{16 \pm \sqrt{16^2 - 4\cdot 3\cdot 20}}{2\cdot 3}
            \iff x = \frac{16 \pm 4}{6}
            \iff x\in \{2, \nicefrac{10}{3}\}
        \end{multline*}

        Por tanto, $p$ es estrictamente creciente en $\left]-\infty, 2\right[$. Tenemos que:
        \begin{align*}
            p(1)=1-8+20-15.2= -2.2 < 0\\
            \lim_{x\to -\infty} p(x) = -\infty
        \end{align*}

        Por tanto, deducimos que $p$ no tiene raíces menores que 1.

        \item Prueba que Newton-Raphson converge partiendo de $x_0 = 0$ hacia la raíz más pequeña y realiza dos iteraciones.\\
        
        Tenemos que:
        \begin{align*}
            p(1)&=-2.2<0\\
            p(2)&=8-32+40-15.2=0.8>0
        \end{align*}

        Por tanto, por el Teorema de Bolzano, $p$ tiene una raíz en $\left[1,2\right]$, y sabemos que es única por ser $p$ estrictamente creciente en $\left]-\infty, 2\right[$. Por último, es la raíz más pequeña por no tener ninguna raíz menor que 1. Comprobemos ahora que cumple las condiciones del Teorema de Convergencia del Método de Newton-Raphson en $\left[1,2\right]$:
        \begin{enumerate}
            \item $p(1)p(2)<0$.
            \item $p'(x)\neq 0$ en $\left]1,2\right[$.
            \item $p''(x)=6x-16$ no cambia de signo en $\left]1,2\right[$, ya que:
            \begin{equation*}
                p''(x)=6x-16=0 \iff x=\frac{16}{6}=\frac{8}{3}\approx 2.67\notin \left]1,2\right[
            \end{equation*}
            \item $p(x_0)p''(x_0)=(-15.2)\cdot (-16)>0$.
        \end{enumerate}

        Por tanto, el método de Newton-Raphson converge en $\left[1,2\right]$ a la raíz más pequeña partiendo desde $x_0=0$. Este método genera la siguiente sucesión:
        \begin{align*}
            x_{n+1} &= x_n - \frac{p(x_n)}{p'(x_n)} = x_n - \frac{x_n^3 - 8x_n^2 + 20x_n - 15.2}{3x_n^2 - 16x_n + 20}=\\
            &= \frac{2x_n^3 - 8x_n^2 + 15.2}{3x_n^2 - 16x_n + 20}
        \end{align*}
        Por tanto, las dos primeras iteraciones son:
        \begin{equation*}
            \begin{array}{c|c}
                n & x_n\\ \hline
                0 & 0 \\
                1 & 0.76\\
                2 & 1.196844
            \end{array}
        \end{equation*}
        \item Calcula la sucesión de Sturm y decide si existen raíces múltiples.
        
        % // TODO: Sturm
        \item Separa las raíces reales de dicha ecuación.
    \end{enumerate}
\end{ejercicio}

\begin{ejercicio}
    Sea la ecuación $f(x) = e^{x-1} - ax^3 = 0$ siendo $a > 1$.
    \begin{enumerate}
        \item Demuestra que tiene al menos una raíz en $[0, 1]$.
        
        Como $f$ es continua, podemos aplicar el Teorema de Bolzano. Evaluamos $f$ en los extremos del intervalo:
        \begin{align*}
            f(0) &= e^{-1} > 0\\
            f(1) &= e^0 - a = 1 - a < 0
        \end{align*}
        Por tanto, por el Teorema de Bolzano, $f$ tiene al menos una raíz en $[0, 1]$.
        \item A partir de ahora considera $a = 2$. Calcula las dos primeras aproximaciones $x_1$ y $x_2$ obtenidas con bisección (siendo $x_0 = 0.5$). Indica el error máximo que se comete con $x_2$.
        \begin{equation*}
            \begin{array}{c|c|c|c|c}
                n & a_n & b_n & x_n & f(x_n) \\ \hline
                0 & 0 & 1 & 0.5 & 0.3565\\
                1 & 0.5 & 1 & 0.75 & -0.0649\\
                2 & 0.5 & 0.75 & 0.625 & 0.1990
            \end{array}
        \end{equation*}

        El error máximo que se comete con $x_2$ es:
        \begin{equation*}
            |e_2| < \dfrac{1-0}{2^{2+1}} = \dfrac{1}{8} = 0.125
        \end{equation*}
        \item Realiza dos iteraciones con el método de la secante tomando como valores iniciales (o semillas) $x_0 = 0$, $x_1 = 1$. Debes calcular $x_2$ y $x_3$.\\
        
        El método de la secante se define como:
        \begin{align*}
            x_{n+1} &= x_n - \frac{f(x_n)(x_n - x_{n-1})}{f(x_n) - f(x_{n-1})}
            =\\&= 
            x_n - \frac{(e^{x_n-1} - 2x_n^3)(x_n - x_{n-1})}{e^{x_n-1} - 2x_n^3 - e^{x_{n-1}-1} + 2x_{n-1}^3}
        \end{align*}

        Por tanto, las dos primeras iteraciones son:
        \begin{equation*}
            \begin{array}{c|c}
                n & x_n\\ \
                0 & 0 \\
                1 & 1 \\
            \end{array}
        \end{equation*}
        % // TODO: Continuar dos iteraciones
        \item Evalúa la función en la segunda aproximación $x_2$ obtenida con bisección e indica, razonadamente con los resultados que se te han pedido, si se puede asegurar, o no, que la segunda aproximación obtenida con bisección está más cerca de la raíz que la segunda aproximación obtenida con la secante.
    \end{enumerate}
\end{ejercicio}

\begin{ejercicio}
    Considera la ecuación $x^2 = a$ siendo $a > 0$.
    \begin{enumerate}
        \item Se pretende usar el método de Newton-Raphson en la ecuación anterior para hallar la raíz cuadrada de $a$. Deduce que el método se puede expresar, en este caso, de la forma
        \[
            x_{n+1} = \frac{1}{2}\left(x_n + \frac{a}{x_n}\right)
        \]

        Sea la función $f(x) = x^2 - a$. Aplicando el método de Newton-Raphson, tenemos:
        \begin{align*}
            x_{n+1} &= x_n - \frac{f(x_n)}{f'(x_n)} = x_n - \frac{x_n^2 - a}{2x_n} = x_n - \frac{x_n}{2} + \frac{a}{2x_n} = \dfrac{1}{2}\left(x_n + \dfrac{a}{x_n}\right)
        \end{align*}
        \item Demuestra que el método es convergente partiendo de $x_0 = \max\{1, a\}$.
        \begin{comment}
        Emplearemos el intervalo $\left[0,a+1\right]$. Calculamos la derivada de $f$:
        \begin{align*}
            f'(x) = 2x\qquad \forall x\in \bb{R}
        \end{align*}

        Además, usaremos el siguiente resultado:
        \begin{equation*}
            (a+1)^2>a \iff a^2 + 2a + 1 > a \iff a^2 + a + 1 > 0
        \end{equation*}
        que es cierto, pues $a > 0$. Demostramos ahora los 4 puntos necesarios para el Teorema de Convergencia del Método de Newton-Raphson:
        \begin{enumerate}
            \item $f(0)f(a+1) < 0$:
            \begin{align*}
                f(0) = -a < 0\\
                f(a+1) = (a+1)^2 - a > 0
            \end{align*}

            \item $f'(x) \neq 0$ en $\left]0, a+1[$.
            \item $f''(x)=2 > 0$ no cambia de signo en $\left]0, a+1\right[$.
            \item $\max\left\{\left|\frac{f(0)}{f'(0)}\right|, \left|\frac{f(a+1)}{f'(a+1)}\right|\right\} < a+1$:
            \begin{align*}
                \dfrac{f(0)}{f'(0)} = \dfrac{-a}{0} = -\infty < a+1\\
            \end{align*}
        \end{enumerate}
        \end{comment}
        \item Apoyándote en la expresión anterior obtén la segunda aproximación $x_2$ de la raíz cuadrada positiva de 13, partiendo de $x_0 = 13$.
        \item Determina la expresión del método de Newton-Raphson para la raíz cúbica de un número diferente de cero y aplícalo dos veces para aproximar la raíz cúbica de 13 partiendo de $x_0 = 13$.
    \end{enumerate}
\end{ejercicio}

\begin{ejercicio}
    Sea $S$ la única solución en el dominio cuadrado $D = [0, 1] \times [0, 1]$ del sistema no lineal
    \[
        \begin{cases}
            xy^2 + 4x - 1 = 0\\
            4yx^2 + 6y - 1 = 0
        \end{cases}
    \]
    ¿Es convergente a $S$ la sucesión de iteraciones del método de iteración funcional definido por
    \[
        \begin{cases}
            x_{n+1} = \frac{1}{4 + y_n^2}\\
            y_{n+1} = \frac{1}{6 + 4x_n^2}
        \end{cases}
    \]
    cualquiera que sea la aproximación inicial $(x_0, y_0) \in [0, 1] \times [0, 1]$?
\end{ejercicio}

\begin{ejercicio}
    Se sabe que $f(x) : [a, b] \to \bb{R}$ es una función continua en $[a, b]$ y posee un único cero en dicho intervalo. ¿Se puede aproximar siempre dicho cero mediante el método de bisección?
\end{ejercicio}

\begin{ejercicio}
    Se considera la ecuación $f(x) = x^3 - x - 1 = 0$. Se pide:
    \begin{enumerate}
        \item Demuestra que la ecuación anterior tiene una única solución real $s$.
        \item Encuentra un intervalo $[a, b]$ en el que al tomar cualquier punto $x_0 \in [a, b]$ como aproximación inicial del método de Newton-Raphson aplicado a $f(x)$ se asegure que la sucesión de iteraciones de dicho método converja a $s$ con convergencia al menos cuadrática y demuestra que eso es así.
        \item Calcula las dos primeras iteraciones del método de Newton-Raphson para resolver la ecuación dada tomando como aproximación inicial $x_0 = 1$.
    \end{enumerate}
\end{ejercicio}

\begin{ejercicio}
    Se pretende estimar el valor de $\sqrt[7]{2}$ usando un método iterativo.
    \begin{enumerate}
        \item Determina justificadamente una función $f$ y un intervalo $[a, b]$ donde se pueda aplicar el método de bisección. ¿Cuántas iteraciones son necesarias para conseguir un error inferior a $10^{-4}$?
        \item Determina justificadamente un intervalo $[a, b]$ y un valor inicial $x_0$ que permita asegurar que el método de Newton-Raphson converge a $\sqrt[7]{2}$ y realiza 3 iteraciones del método.
        \item Se propone el método iterativo
        \[
            x_{n+1} = \frac{8x_n + 3x_n^8}{6 + 4x_n^7}
        \]
        Realiza 3 iteraciones del método empezando en el mismo valor $x_0$ del apartado anterior.
        \item ¿Cuál de los dos métodos converge más rápidamente a la solución? Justifica la respuesta.
    \end{enumerate}
\end{ejercicio}

\begin{ejercicio} Relacionado con la Sucesión de Sturm:
    \begin{enumerate}
        \item Sea $\{f_0(x), f_1(x), \ldots, f_m(x)\}$ una sucesión de Sturm en el intervalo $[a, b]$ y $k_i \in \bb{R}$ con $k_i > 0$ para $i = 0, \ldots, m$. Demuestra que si se define $\tilde{f}_i = k_i f_i$, entonces $\left\{\tilde{f}_0(x), \tilde{f}_1(x), \ldots, \tilde{f}_m(x)\right\}$ es también una sucesión de Sturm en $[a, b]$.
        \item Dado el polinomio $p(x) = x^3 - x + 1$, determina justificadamente un intervalo en el que estén contenidas todas sus raíces.
        \item Construye una sucesión de Sturm para el polinomio $p$ y utilízala para determinar el número de raíces reales así como intervalos de amplitud 1 en los que se encuentran.
        \item Realiza dos iteraciones del método de la secante para calcular de forma aproximada el valor de la raíz positiva más pequeña justificando la convergencia.
    \end{enumerate}
\end{ejercicio}

\begin{ejercicio}
    Contesta razonadamente a las siguientes preguntas:
    \begin{enumerate}
        \item Se pretende resolver la ecuación $f(x) = 0$ utilizando el método de Newton-Raphson sabiendo que es convergente localmente ¿Qué debe cumplir la función $f$ para que dicho método tenga convergencia local al menos cúbica?
        \item Si sabemos que $f$ tiene una única raíz real en el intervalo $[-1, 1]$, ¿Cuántas iteraciones del método de bisección hay que realizar para conseguir un error menor que $10^{-7}$?
        \item ¿Es el método de Newton-Raphson para resolver el sistema $F(X) = 0$ invariante frente a transformaciones lineales de $F$?
        \begin{observacion}
            Que sea invariante frente a transformaciones lineales quiere decir que la secuencia de aproximaciones $\{X_n\}$ es la misma si se aplica el método al sistema $F(X) = 0$ o si se aplica al sistema $AF(X) = 0$, siendo $A$ una matriz no singular, partiendo del mismo vector inicial $X_0$.
        \end{observacion}
    \end{enumerate}
\end{ejercicio}

\begin{ejercicio}
    El problema de trisección de un ángulo consiste en hallar las razones trigonométricas de $\nicefrac{\alpha}{3}$, conociendo las de $\alpha \in \left]0,\nicefrac{\pi}{2}\right[$.
    \begin{enumerate}
        \item Llamando $x = \sen\left(\nicefrac{\alpha}{3}\right)$ y $a = \sen \alpha$, demuestra que $x$ es solución de la ecuación $-4x^3 + 3x - a = 0$.
        \item Construye una sucesión de Sturm de polinomios asociada al polinomio $p(x) = -4x^3 + 3x - a$ y deduce que $p$ tiene exactamente 3 raíces reales.
        \item Demuestra que $\sen\left(\nicefrac{\alpha}{3}\right)$ es la única solución de la ecuación $p(x) = 0$, en el intervalo $\left]0,\nicefrac{a}{2}\right[$ y que, tomando como valores iniciales $x_0 = \nicefrac{a}{3}$ o $x_0 = \nicefrac{a}{2}$, el método de Newton-Raphson converge.
        \item Para resolver la ecuación anterior, se propone el método iterativo
        \[
            x_{n+1} = \frac{a}{3- 4x_n^2}
        \]
        Estudia bajo qué condiciones el método converge localmente a la solución. ¿Cuál de los dos métodos converge más rápidamente?
        \item Tomando $a = \nicefrac{1}{2}$, realiza una iteración del método de Newton-Raphson partiendo de $x_0 = \nicefrac{1}{6}$ para obtener una aproximación de $\sen\left(\nicefrac{\pi}{18}\right)$.
    \end{enumerate}
\end{ejercicio}


