\subsection{Relación 1. Derivación Numérica}
\setcounter{ejercicio}{0}


\begin{ejercicio}\label{ej:2.1.1}~
    \begin{enumerate}
        \item Use el método interpolatorio para obtener la fórmula de diferencia progresiva en dos nodos para aproximar $f'(a)$. ¿Cuál es el grado de exactitud de dicha fórmula? Justifique la respuesta.
        
        \item Utilice la fórmula de Taylor para hallar la fórmula mencionada en el apartado anterior y la expresión de su error cuando $f'' \in C^2[a, a + h]$.
        
        \item Use el método interpolatorio para obtener la fórmula de diferencia centrada en dos nodos para aproximar $f'(a)$. ¿Cuál es el grado de exactitud de dicha fórmula? Justifique la respuesta.
        
        \item Utilice la fórmula de Taylor para hallar la fórmula mencionada en el apartado anterior y la expresión de su error cuando $f'' \in C^3[a - h, a + h]$.
    \end{enumerate}    
\end{ejercicio}

\begin{ejercicio}\label{ej:2.1.2}
    Usando el método de los coeficientes indeterminados deduzca la fórmula de diferencia centrada en tres nodos para aproximar $f'(a)$ y compruebe que coincide con la fórmula de diferencia centrada en dos nodos.
\end{ejercicio}

\begin{ejercicio}\label{ej:2.1.3}
    Obtenga la fórmula de diferencia progresiva en tres nodos para aproximar $f'(a)$ calculando directamente sus coeficientes mediante la base de Lagrange del problema de interpolación unisolvente asociado a dicha fórmula. Halle la expresión del error para esta fórmula cuando $f \in C^4[a, a + 2h]$.
\end{ejercicio}

\begin{ejercicio}\label{ej:2.1.4}
    Se considera la fórmula de tipo interpolatorio en $P_n$ siguiente:
    \[
    f''(c) \approx \alpha_0 f(x_0) + \alpha_1 f(x_1) + \cdots + \alpha_n f(x_n)
    \]
    con $x_0 < x_1 < \cdots < x_n$. Demuestre que si $f \in C^{n+3}[a, b]$ con $[a, b]$ tal que $x_0, x_1, \ldots, x_n, c \in [a, b]$, entonces
    \[
    R(f) = 2\cdot \frac{f^{(n+3)}(\xi_2)}{(n + 3)!}\Pi(c) + 2\cdot \frac{f^{(n+2)}(\xi_1)}{(n + 2)!}\Pi'(c) + \frac{f^{(n+1)}(\xi_0)}{(n + 1)!}\Pi''(c)
    \]
    siendo $\xi_0, \xi_1, \xi_2 \in \left] \min\{x_0, c\}, \max\{x_n, c\} \right[$ y $\Pi(x) = \prod\limits_{i=0}^{n}(x - x_i)$.
\end{ejercicio}

\begin{ejercicio}\label{ej:2.1.5}
    Use el método de los coeficientes indeterminados para obtener la fórmula de diferencia regresiva en tres nodos para aproximar $f''(a)$. Halle la expresión del error de truncamiento para esta fórmula cuando $f \in C^5[a - 2h, a]$.
\end{ejercicio}

\begin{ejercicio}\label{ej:2.1.6}
    Use la fórmula de Taylor para obtener, cuando $f \in C^4[a - h, a + h]$, la fórmula de diferencia centrada en tres nodos para aproximar $f''(a)$ y la expresión de su error. ¿Cuál es el orden de precisión de esta fórmula? ¿Por qué?
\end{ejercicio}

\begin{ejercicio}\label{ej:2.1.7}~
    \begin{enumerate}
        \item Halle la fórmula de tipo interpolatorio en $P_2$ de la forma
        \[
        f'(a) \approx \alpha_0 f(a - h_1) + \alpha_1 f(a) + \alpha_2 f(a + h_2)
        \]
        con $h_1, h_2 > 0$, así como la expresión de su error de truncamiento cuando $f \in C^4[a - h_1, a + h_2]$. ¿Cuál es el grado de exactitud de esta fórmula? Justifique la respuesta.
        
        \item Use la tabla de valores
        \begin{center}
            \begin{tabular}{c|c}
                $x$ & $f(x)$ \\
                \hline
                $0.7$ & $-0.1$ \\
                $1.25$ & $0.2$ \\
                $1.5$ & $0.3$ \\
                $1.75$ & $0.25$
            \end{tabular}
        \end{center}
        para dar valores aproximados de $f'(0.7)$, $f'(1.25)$, $f'(1.5)$ y $f'(1.75)$ utilizando para cada uno de ellos la fórmula de derivación numérica más adecuada. Indique la fórmula usada en cada caso y justifique su uso.
    \end{enumerate}
\end{ejercicio}

\begin{ejercicio}\label{ej:2.1.8}
    Halle una cota del valor absoluto del error que se comete al aproximar la derivada de la función $f(x) = \cos^2 x$ en $x = 0.8$ mediante la correspondiente fórmula de derivación numérica de tipo interpolatorio que usa los valores de $f$ en los puntos $0.6$, $0.8$ y $1$.
\end{ejercicio}