\subsection{Relación 1. Derivación Numérica}
\setcounter{ejercicio}{0}


\begin{ejercicio}\label{ej:2.1.1}~
    \begin{enumerate}
        \item Use el método interpolatorio para obtener la fórmula de diferencia progresiva en dos nodos para aproximar $f'(a)$. ¿Cuál es el grado de exactitud de dicha fórmula? Justifique la respuesta.\\
        
        Supongamos que queremos aproximar $f'(a)$ mediante una fórmula de diferencia progresiva en dos nodos; es decir, sean los nodos $x_0=a, x_1=a+h$. Calculamos los polinomios básicos de Lagrange:
        \begin{align*}
            \ell_0(x) &= \frac{x - x_1}{x_0 - x_1} = \frac{x - (a+h)}{a - (a+h)} = \frac{x - a - h}{-h} = \frac{a+h-x}{h}\\
            \ell_1(x) &= \frac{x - x_0}{x_1 - x_0} = \frac{x - a}{a+h - a} = \frac{x - a}{h}\\
            \ell_0'(x) &= \nicefrac{-1}{h}\\
            \ell_1'(x) &= \nicefrac{1}{h}
        \end{align*}

        Por tanto, sabemos que:
        \begin{align*}
            f(x) &= f(a)\ell_0(x) + f(a+h)\ell_1(x) + E(x)\\
            f'(x) &= f(a)\ell_0'(x) + f(a+h)\ell_1'(x) + E'(x)
        \end{align*}

        Además, el error cometido al interpolar $f$ en $a, a+h$ mediante dos nodos es:
        \begin{align*}
            E(x) &= f[a, a+h, x]\Pi(x)\\
            E'(x) &= f[a, a+h, x, x]\Pi(x) + f[a, a+h, x]\Pi'(x)
        \end{align*}

        Por tanto:
        \begin{align*}
            f'(a) &= \dfrac{f(a+h) - f(a)}{h} + R(f)\\
            R(f) &= E'(a) = f[a, a+h, a]\Pi'(a) =
            -h\cdot \dfrac{f^{(2)}(\xi)}{2!}\qquad \xi\in\left]a, a+h\right[
        \end{align*}

        Por tanto, sabemos que esta fórmula es exacta en $\bb{P}_1$, pueso que en estos casos se anulará la segunda derivada. No es exacta en $\bb{P}_2$ porque el error no se anula.
        
        \item Utilice la fórmula de Taylor para hallar la fórmula mencionada en el apartado anterior y la expresión de su error cuando $f\in C^2[a, a + h]$.
        
        Desaroollamos $f$ en cada uno de los nodos alrededor del punto $a$ hasta el segundo término:
        \begin{align*}
            f(a+h) &= f(a) + hf'(a) + \frac{h^2}{2}f''(\xi_1)\qquad \xi_1\in\left]a, a+h\right[
        \end{align*}

        Por tanto, tenemos que:
        \begin{align*}
            f'(a) &= \frac{f(a+h) - f(a)}{h} - \frac{h}{2}f''(\xi_1)\qquad \xi_1\in\left]a, a+h\right[
        \end{align*}

        
        \item Use el método interpolatorio para obtener la fórmula de diferencia centrada en dos nodos para aproximar $f'(a)$. ¿Cuál es el grado de exactitud de dicha fórmula? Justifique la respuesta.
        
        Supongamos que queremos aproximar $f'(a)$ mediante una fórmula de diferencia centrada en dos nodos; es decir, sean los nodos $x_0=a-h, x_1=a+h$. Calculamos los polinomios básicos de Lagrange:
        \begin{align*}
            \ell_0(x) &= \frac{x - x_1}{x_0 - x_1} = \frac{x - (a+h)}{a-h - (a+h)} = \frac{x - a - h}{-2h}\\
            \ell_1(x) &= \frac{x - x_0}{x_1 - x_0} = \frac{x - (a-h)}{a+h - (a-h)} = \frac{x - a + h}{2h}\\
            \ell_0'(x) &= \nicefrac{-1}{2h}\\
            \ell_1'(x) &= \nicefrac{1}{2h}
        \end{align*}

        Por tanto, sabemos que:
        \begin{align*}
            f(x) &= f(a-h)\ell_0(x) + f(a+h)\ell_1(x) + E(x)\\
            f'(x) &= f(a-h)\ell_0'(x) + f(a+h)\ell_1'(x) + E'(x)
        \end{align*}

        Además, el error cometido al interpolar $f$ en $a-h, a+h$ mediante dos nodos es:
        \begin{align*}
            E(x) &= f[a-h, a+h, x]\Pi(x)\\
            E'(x) &= f[a-h, a+h, x, x]\Pi(x) + f[a-h, a+h, x]\Pi'(x)
        \end{align*}

        Por tanto:
        \begin{align*}
            f'(a) &= \dfrac{f(a+h) - f(a-h)}{2h} + R(f)\\
            R(f) &= E'(a) = f[a-h, a+h, a, a]\Pi(a) =
            - h^2\dfrac{f^{(3)}(\xi)}{3!}
        \end{align*}

        Por tanto, sabemos que esta fórmula es exacta en $\bb{P}_1$ y $\bb{P}_2$, pueso que en estos casos se anulará la tercera derivada. Por tanto, el grado de exactitud de esta fórmula es 2.
        
        \item Utilice la fórmula de Taylor para hallar la fórmula mencionada en el apartado anterior y la expresión de su error cuando $f'' \in C^3[a - h, a + h]$.
        
        Desaroollamos $f$ en cada uno de los nodos alrededor del punto $a$ hasta el segundo término:
        \begin{align*}
            f(a+h) &= f(a) + hf'(a) + \frac{h^2}{2}f''(a) + \frac{h^3}{6}f'''(\xi_1)\qquad \xi_1\in\left]a, a+h\right[\\
            f(a-h) &= f(a) - hf'(a) + \frac{h^2}{2}f''(a) - \frac{h^3}{6}f'''(\xi_2)\qquad \xi_2\in\left]a-h, a\right[
        \end{align*}

        Realizamos una combinación lineal de ambas expresiones, forzando a que el coeficiente de $f'(a)$ sea 1 y el coeficiente de $f(a)$ sea 0:
        \begin{equation*}
            \begin{pmatrix}
                1 & 1\\
                h & -h
            \end{pmatrix}
            \begin{pmatrix}
                \alpha_0\\
                \alpha_1
            \end{pmatrix}
            =
            \begin{pmatrix}
                0\\
                1
            \end{pmatrix}
            \Longrightarrow
            \begin{cases}
                \alpha_0 = \nicefrac{1}{2h}\\
                \alpha_1 = \nicefrac{-1}{2h}
            \end{cases}
        \end{equation*}

        Por tanto, tenemos que:
        \begin{align*}
            f'(a) &= \frac{f(a+h) - f(a-h)}{2h} + R(f)\\
            R(f) &= -\dfrac{h^2}{12}\left(f'''(\xi_1) + f'''(\xi_2)\right)\qquad \xi_1, \xi_2\in\left]a-h, a+h\right[
                \\&\AstIg -\dfrac{h^2}{6}f'''(\xi)\qquad \xi\in\left]a-h, a+h\right[
        \end{align*}
        donde en $(\ast)$ hemos empleado el Teorema del Valor Medio.
    \end{enumerate}    
\end{ejercicio}

\begin{ejercicio}\label{ej:2.1.2}
    Usando el método de los coeficientes indeterminados deduzca la fórmula de diferencia centrada en tres nodos para aproximar $f'(a)$ y compruebe que coincide con la fórmula de diferencia centrada en dos nodos.\\

    Queremos aproximar $f'(a)$ mediante una fórmula de diferencia centrada en tres nodos; es decir, sean los nodos $x_0=a-h, x_1=a, x_2=a+h$. Como lo realizamos mediante el método de los coeficientes indeterminados, debemos imponer exactitud en $\{1, x, x^2\}$:
    \begin{align*}
        0 &= \alpha_0 + \alpha_1 + \alpha_2\\
        1 &= (a-h)\alpha_0 + a\alpha_1 + (a+h)\alpha_2\\
        2a &= (a-h)^2\alpha_0 + a^2\alpha_1 + (a+h)^2\alpha_2
    \end{align*}

    Resolvemos ahora dicho sistema para hallar los coeficientes. Equivalentemente:
    \begin{align*}
        0 &= \alpha_0 + \alpha_1 + \alpha_2\\
        1 &= a\left(\alpha_0 + \alpha_1 + \alpha_2\right) - h\left(\alpha_0 - \alpha_2\right)\\
        2a &= a^2\left(\alpha_0 + \alpha_1 + \alpha_2\right) + h^2\left(\alpha_0 + \alpha_2\right) - 2ah\left(\alpha_0 - \alpha_2\right)
    \end{align*}

    Equivalentemente:
    \begin{align*}
        0 &= \alpha_0 + \alpha_1 + \alpha_2\\
        1 &= - h\left(\alpha_0 - \alpha_2\right)\\
        2a &= h^2\left(\alpha_0 + \alpha_2\right) - 2ah\left(\alpha_0 - \alpha_2\right)
    \end{align*}

    Equivalentemente:
    \begin{align*}
        0 &= \alpha_0 + \alpha_1 + \alpha_2\\
        1 &= - h\left(\alpha_0 - \alpha_2\right)\\
        2a &= h^2\left(\alpha_0 + \alpha_2\right) + 2a
    \end{align*}

    Equivalentemente:
    \begin{align*}
        0 &= \alpha_0 + \alpha_1 + \alpha_2\\
        1 &= - h\left(\alpha_0 - \alpha_2\right)\\
        0 &= h^2\left(\alpha_0 + \alpha_2\right)
    \end{align*}

    De la última ecuación, tenemos que $\alpha_0 = -\alpha_2$, y sustituyendo en la segunda ecuación, obtenemos que $\alpha_1 = 0$. Por tanto, vemos que el peso del nodo central es 0, lo que nos lleva a la fórmula de diferencia centrada en dos nodos. No obstante, comprobémoslo. De la segunda ecuación, tenemos que:
    \begin{equation*}
        1=-2h\alpha_0\quad\Longrightarrow\quad \alpha_0 = -\frac{1}{2h}\Longrightarrow \alpha_2 = \frac{1}{2h}
    \end{equation*}

    Por tanto, la fórmula de diferencia centrada en tres nodos para aproximar $f'(a)$ es:
    \[
    f'(a) \approx \dfrac{f(a+h) - f(a-h)}{2h}
    \]
    que coincide con la fórmula de diferencia centrada en dos nodos.
\end{ejercicio}

\begin{ejercicio}\label{ej:2.1.3}
    Obtenga la fórmula de diferencia progresiva en tres nodos para aproximar $f'(a)$ calculando directamente sus coeficientes mediante la base de Lagrange del problema de interpolación unisolvente asociado a dicha fórmula. Halle la expresión del error para esta fórmula cuando $f \in C^4[a, a + 2h]$.\\

    Queremos aproximar $f'(a)$ mediante una fórmula de diferencia progresiva en tres nodos; es decir, sean los nodos $x_0=a, x_1=a+h, x_2=a+2h$. Calculamos los polinomios básicos de Lagrange:
    \begin{align*}
        \ell_0(x) &= \prod_{\substack{j=0\\j\neq 0}}^{2}\frac{x - x_j}{x_0 - x_j} = \frac{x - (a+h)}{a - (a+h)}\cdot\frac{x - (a+2h)}{a - (a+2h)} = \frac{x - a - h}{-h}\cdot\frac{x - a - 2h}{-2h}\\
        \ell_1(x) &= \prod_{\substack{j=0\\j\neq 1}}^{2}\frac{x - x_j}{x_1 - x_j} = \frac{x - a}{a+h - a}\cdot\frac{x - (a+2h)}{(a+h) - (a+2h)} = \frac{x - a}{h}\cdot\frac{x - a - 2h}{-h}\\
        \ell_2(x) &= \prod_{\substack{j=0\\j\neq 2}}^{2}\frac{x - x_j}{x_2 - x_j} = \frac{x - a}{a+2h - a}\cdot\frac{x - (a+h)}{(a+2h) - (a+h)} = \frac{x - a}{2h}\cdot\frac{x - a - h}{h}
    \end{align*}

    Calculamos ahora las derivadas de los polinomios básicos de Lagrange:
    \begin{align*}
        \ell_0'(x) &= \dfrac{x-a-2h + x-a-h}{2h^2} = \dfrac{2x-2a-3h}{2h^2}\\
        \ell_1'(x) &= \dfrac{x-a-2h + x-a}{-h^2} = \dfrac{2x-2a-2h}{-h^2}\\
        \ell_2'(x) &= \dfrac{x-a-h + x-a}{2h^2} = \dfrac{2x-2a-h}{2h^2}
    \end{align*}

    Como buscamos aproximar $f'(a)$, evaluamos cada una de las derivadas de los polinomios básicos de Lagrange en $a$:
    \begin{align*}
        \ell_0'(a) &= \dfrac{-3h}{2h^2} = -\dfrac{3}{2h}\\
        \ell_1'(a) &= \dfrac{-2h}{-h^2} = \dfrac{2}{h}\\
        \ell_2'(a) &= \dfrac{-h}{2h^2} = -\dfrac{1}{2h}
    \end{align*}

    Por tanto, la fórmula de diferencia progresiva en tres nodos para aproximar $f'(a)$ es la siguiente:
    \[
    f'(a) \approx -\dfrac{3}{2h}f(a) + \dfrac{2}{h}f(a+h) - \dfrac{1}{2h}f(a+2h)
    \]

    Para el error, definimos:
    \begin{align*}
        \Pi(x) &= \prod_{i=0}^{2}(x - x_i) = (x - a)(x - a - h)(x - a - 2h)\\
        \Pi'(x) &= (x-a-h)(x-a-2h) + (x-a)(x-a-2h) + (x-a)(x-a-h)
    \end{align*}

    Por tanto, el error cometido al interpolar $f$ en $a, a+h, a+2h$ mediante tres nodos es:
    \begin{align*}
        E(x) &= f[a, a+h, a+2h, x]\Pi(x)\\
        E'(x) &= f[a, a+h, a+2h, x, x]\Pi(x) + f[a, a+h, a+2h, x]\Pi'(x)
    \end{align*}

    Por tanto, tenemos que:
    \begin{align*}
        R(f) &= E'(a) = f[a, a+h, a+2h, a, a]\Pi(a) + f[a, a+h, a+2h, a]\Pi'(a)
        =\\&= 2h^2f[a, a+h, a+2h, a]=2h^2\cdot \frac{f^{(3)}(\xi)}{3!}=h^2\cdot \frac{f^{(3)}(\xi)}{3}\qquad \xi\in\left]a, a+2h\right[
    \end{align*}
\end{ejercicio}

\begin{ejercicio}\label{ej:2.1.4}
    Se considera la fórmula de tipo interpolatorio en $\bb{P}_n$ siguiente:
    \[
    f''(c) \approx \alpha_0 f(x_0) + \alpha_1 f(x_1) + \cdots + \alpha_n f(x_n)
    \]
    con $x_0 < x_1 < \cdots < x_n$. Demuestre que si $f \in C^{n+3}[a, b]$ con $[a, b]$ tal que $x_0, x_1, \ldots, x_n, c \in [a, b]$, entonces
    \[
    R(f) = 2\cdot \frac{f^{(n+3)}(\xi_2)}{(n + 3)!}\Pi(c) + 2\cdot \frac{f^{(n+2)}(\xi_1)}{(n + 2)!}\Pi'(c) + \frac{f^{(n+1)}(\xi_0)}{(n + 1)!}\Pi''(c)
    \]
    siendo $\xi_0, \xi_1, \xi_2 \in \left] \min\{x_0, c\}, \max\{x_n, c\} \right[$ y $\Pi(x) = \prod\limits_{i=0}^{n}(x - x_i)$.\\

    Sabemos que el error cometido con la interpolación de $f$ en los nodos $x_0, x_1, \ldots, x_n$ mediante $p\in \bb{P}_n$ es:
    \[
    E(x) = f(x)-p(x)= f[x_0, x_1, \ldots, x_n, x]\Pi(x)
    \]

    Calculemos cada una de las derivadas de $E(x)$:
    \begin{align*}
        E'(x) &= f[x_0, x_1, \ldots, x_n, x,x]\Pi(x) + f[x_0, x_1, \ldots, x_n, x]\Pi'(x)\\
        E''(x) &= 2f[x_0, x_1, \ldots, x_n, x,x,x]\Pi(x) + 2f[x_0, x_1, \ldots, x_n, x,x]\Pi'(x) + f[x_0, x_1, \ldots, x_n, x]\Pi''(x)
    \end{align*}

    Por tanto, tenemos que:
    \begin{align*}
        R(f) &= L(f-p)=L(E)=E''(c)
        =\\&= 2f[x_0, x_1, \ldots, x_n, c,c,c]\Pi(c) + 2f[x_0, x_1, \ldots, x_n, c,c]\Pi'(c) + f[x_0, x_1, \ldots, x_n, c]\Pi''(c)
        =\\&= 2\cdot \frac{f^{(n+3)}(\xi_2)}{(n + 3)!}\Pi(c) + 2\cdot \frac{f^{(n+2)}(\xi_1)}{(n + 2)!}\Pi'(c) + \frac{f^{(n+1)}(\xi_0)}{(n + 1)!}\Pi''(c)
    \end{align*}
    con $\xi_0, \xi_1, \xi_2 \in \left] \min\{x_0, c\}, \max\{x_n, c\} \right[$, donde hemos empleado que los nodos están ordenados; es decir, $x_0<x_1<\cdots<x_n$.

\end{ejercicio}

\begin{ejercicio}\label{ej:2.1.5}
    Use el método de los coeficientes indeterminados para obtener la fórmula de diferencia regresiva en tres nodos para aproximar $f''(a)$. Halle la expresión del error de truncamiento para esta fórmula cuando $f \in C^5[a - 2h, a]$.\\
    
    Sabemos que la fórmula de diferencia regresiva en tres nodos para aproximar $f''(a)$ es de la forma:
    \[
    f''(a) \approx \alpha_0 f(a - 2h) + \alpha_1 f(a - h) + \alpha_2 f(a)
    \]
    donde $\alpha_0, \alpha_1, \alpha_2$ son los coeficientes que debemos determinar. Para ello, imponemos exactitud en $\bb{P}_2$:
    \begin{align*}
        0 &= \alpha_0 + \alpha_1 + \alpha_2\\
        0 &= (a-2h)\alpha_0 + (a-h)\alpha_1 + a\alpha_2\\
        2 &= (a-2h)^2\alpha_0 + (a-h)^2\alpha_1 + a^2\alpha_2
    \end{align*}

    Equivalentemente:
    \begin{align*}
        0 &= \alpha_0 + \alpha_1 + \alpha_2\\
        0 &= a\left(\alpha_0 + \alpha_1 + \alpha_2\right) - 2h\alpha_0 - h\alpha_1\\
        2 &= a^2\left(\alpha_0 + \alpha_1 + \alpha_2\right) +4h^2\alpha_0 + h^2\alpha_1 - 4ah\alpha_0 - 2ah\alpha_1
    \end{align*}

    Equivalentemente:
    \begin{align*}
        0 &= \alpha_0 + \alpha_1 + \alpha_2\\
        0 &= h(2\alpha_0 + \alpha_1)\\
        2 &= 2h^2\alpha_0 + h^2(2\alpha_0+\alpha_1) -2ah(2\alpha_0+\alpha_1)
    \end{align*}

    Equivalentemente:
    \begin{equation*}
        \left\{
            \begin{array}{rl}
                0 &= \alpha_0 + \alpha_1 + \alpha_2\\
                0 &= 2\alpha_0 + \alpha_1\\
                1 &= h^2\alpha_0
            \end{array}
        \right\}
        \quad\Longrightarrow\quad
        \left\{
            \begin{array}{rl}
                \alpha_0 &= \nicefrac{1}{h^2}\\
                \alpha_1 &= -2\alpha_0 = \nicefrac{-2}{h^2}\\
                \alpha_2 &= -\alpha_0 - \alpha_1 = \alpha_0 = \nicefrac{1}{h^2}
            \end{array}
        \right\}
    \end{equation*}

    Por tanto, la fórmula de diferencia regresiva en tres nodos para aproximar $f''(a)$ es:
    \[
    f''(a) \approx \frac{1}{h^2}f(a - 2h) - \frac{2}{h^2}f(a - h) + \frac{1}{h^2}f(a)
    \]

    Para el error, definimos:
    \begin{align*}
        \Pi(x) &= \prod_{i=0}^{2}(x - (a - ih)) = (x - a + 2h)(x - a + h)(x - a)\\
        \Pi'(x) &= (x-a+h)(x-a) + (x-a+2h)(x-a) + (x-a+2h)(x-a+h)\\
        \Pi''(x) &= 2(x-a) + 2(x-a+h) + 2(x-a+2h)
    \end{align*}

    Por el Ejercicio~\ref{ej:2.1.4}, puesto que $f \in C^5[a - 2h, a]$, tenemos que:
    \begin{align*}
        R(f) &= 2\cdot \frac{f^{(5)}(\xi_2)}{5!}\cancelto{0}{\Pi(a)} + 2\cdot \frac{f^{(4)}(\xi_1)}{4!}\Pi'(a) + \frac{f^{(3)}(\xi_0)}{3!}\Pi''(a)
        =\\&=2\cdot 2h^2\cdot \frac{f^{(4)}(\xi_1)}{4!} + 6h\cdot \frac{f^{(3)}(\xi_0)}{3!}
        =\\&=h^2\cdot \frac{f^{(4)}(\xi_1)}{6} + h\cdot f^{(3)}(\xi_0)
    \end{align*}
    con $\xi_0, \xi_1, \xi_2 \in \left] a-2h, a \right[$.
\end{ejercicio}

\begin{ejercicio}\label{ej:2.1.6}
    Use la fórmula de Taylor para obtener, cuando se tenga que $f \in C^4[a - h, a + h]$, la fórmula de diferencia centrada en tres nodos para aproximar $f''(a)$ y la expresión de su error. ¿Cuál es el orden de precisión de esta fórmula? ¿Por qué?\\

    Consideramos tres nodos, $x_0=a-h, x_1=a, x_2=a+h$. Desarrollamos $f$ en cada uno de los nodos alrededor del punto $a$. Podríamos desarrollar hasta el tercer término, pero vemos que al sumar el término de tercer orden se anulará. Por tanto, desarrollamos hasta el cuarto término:
    \begin{align*}
        f(a-h) &= f(a) - hf'(a) + \frac{h^2}{2}f''(a) - \frac{h^3}{6}f'''(a) + \frac{h^4}{24}f^{(4)}(\xi_1)\qquad \xi_1\in\left]a-h, a\right[\\
        f(a) &= f(a)\\
        f(a+h) &= f(a) + hf'(a) + \frac{h^2}{2}f''(a) + \frac{h^3}{6}f'''(a) + \frac{h^4}{24}f^{(4)}(\xi_2)\qquad \xi_2\in\left]a, a+h\right[
    \end{align*}

    Multiplicamos la ecuación $i-$ésima por el coeficiente $\alpha_i$ y sumamos:
    \begin{align*}
        \alpha_0f(a-h) &+ \alpha_1f(a) + \alpha_2f(a+h) = \left(\alpha_0+\alpha_1+\alpha_2\right)f(a) -h\left(\alpha_0-\alpha_2\right)f'(a) +\\&\hspace{1cm}+ \frac{h^2}{2}\left(\alpha_0+\alpha_2\right)f''(a) - \frac{h^3}{6}\left(\alpha_0-\alpha_2\right)f'''(a) + \dfrac{h^4}{24}\left(\alpha_0f^{(4)}(\xi_1) + \alpha_2f^{(4)}(\xi_2)\right)
    \end{align*}

    Usando el Teorema del Valor Medio, $\exists \xi\in\left]a-h, a+h\right[$ tal que:
    \begin{align*}
        \alpha_0f(a-h) &+ \alpha_1f(a) + \alpha_2f(a+h) = \left(\alpha_0+\alpha_1+\alpha_2\right)f(a) -h\left(\alpha_0-\alpha_2\right)f'(a) +\\&\hspace{1cm}+ \frac{h^2}{2}\left(\alpha_0+\alpha_2\right)f''(a) - \frac{h^3}{6}\left(\alpha_0-\alpha_2\right)f'''(a) + \dfrac{h^4}{24}f^{(4)}(\xi)\cdot \dfrac{1}{\alpha_0+\alpha_2}
    \end{align*}

    Imponemos ahora las siguientes igualdades, donde vemos por qué hemos tenido que desarrollar hasta el cuarto término:
    \begin{equation*}
        \begin{cases}
            \alpha_0+\alpha_1+\alpha_2 = 0\\
            \alpha_0-\alpha_2 = 0\\
            \alpha_0+\alpha_2 = 1
        \end{cases}
    \end{equation*}

    De la segunda ecuación, tenemos que $\alpha_0 = \alpha_2$, y sustituyendo en la tercera ecuación, obtenemos que $\alpha_0 = \alpha_2 = \nicefrac{1}{2}$. Por tanto, $\alpha_1 = -1$, y la ecuación anterior queda:
    \begin{align*}
        \alpha_0f(a-h) &+ \alpha_1f(a) + \alpha_2f(a+h) = \frac{h^2}{2}f''(a) + \dfrac{h^4}{24}f^{(4)}(\xi)
    \end{align*}

    Por tanto, la fórmula de diferencia centrada en tres nodos para aproximar $f''(a)$ es:
    \begin{align*}
        f''(a) \approx \dfrac{2\alpha_0}{h^2}f(a-h) + \dfrac{2\alpha_1}{h^2}f(a) + \dfrac{2\alpha_2}{h^2}f(a+h) = \dfrac{f(a-h) - 2f(a) + f(a+h)}{h^2}
    \end{align*}

    Respecto al error, tenemos que:
    \begin{equation*}
        R(f) = -\dfrac{h^4}{24}f^{(4)}(\xi)\left(\dfrac{2}{h^2}\right) = -\dfrac{h^2}{12}f^{(4)}(\xi)
    \end{equation*}

    Por tanto, vemos que el orden de precisión de esta fórmula es 3, pues el error se anula en $\bb{P}_3$ pero no en $\bb{P}_4$.
\end{ejercicio}

\begin{ejercicio}\label{ej:2.1.7}~
    \begin{enumerate}
        \item Halle la fórmula de tipo interpolatorio en $\bb{P}_2$ de la forma
        \[
        f'(a) \approx \alpha_0 f(a - h_1) + \alpha_1 f(a) + \alpha_2 f(a + h_2)
        \]
        con $h_1, h_2 > 0$, así como la expresión de su error de truncamiento cuando $f \in C^4[a - h_1, a + h_2]$. ¿Cuál es el grado de exactitud de esta fórmula? Justifique la respuesta.\\

        Imponemos exactitud en $\{1, x, x^2\}$:
        \begin{align*}
            0 &= \alpha_0 + \alpha_1 + \alpha_2\\
            1 &= (a-h_1)\alpha_0 + a\alpha_1 + (a+h_2)\alpha_2\\
            2a &= (a-h_1)^2\alpha_0 + a^2\alpha_1 + (a+h_2)^2\alpha_2
        \end{align*}

        Equivalentemente:
        \begin{align*}
            0 &= \alpha_0 + \alpha_1 + \alpha_2\\
            1 &= a\left(\alpha_0 + \alpha_1 + \alpha_2\right) - h_1\alpha_0 + h_2\alpha_2\\
            2a &= a^2\left(\alpha_0 + \alpha_1 + \alpha_2\right) + \alpha_0h_1\left(h_1-2a\right) + \alpha_2h_2\left(h_2+2a\right)
        \end{align*}

        Equivalentemente:
        \begin{align*}
            0 &= \alpha_0 + \alpha_1 + \alpha_2\\
            1 &= - h_1\alpha_0 + h_2\alpha_2\\
            2a &= \alpha_0h_1\left(h_1-2a\right) + \alpha_2h_2\left(h_2+2a\right)
        \end{align*}

        De la segunda ecuación, obtenemos $h_2\alpha_2=1+h_1\alpha_0$, y sustituyendo en la tercera ecuación, obtenemos:
        \begin{align*}
            2a &= \alpha_0h_1\left(h_1-2a\right) + \left(1+h_1\alpha_0\right)\left(h_2+2a\right)
            = \alpha_0h_1^2-\cancel{2a\alpha_0h_1} + h_2 + 2a + h_1h_2\alpha_0 + \cancel{2a\alpha_0h_1}
        \end{align*}

        Por tanto:
        \begin{equation*}
            0=\alpha_0h_1\left(h_1+h_2\right) + h_2
            \Longrightarrow
            \alpha_0 = -\dfrac{h_2}{h_1\left(h_1+h_2\right)}
            \Longrightarrow
            \alpha_2 = \dfrac{h_1}{h_2\left(h_1+h_2\right)}
        \end{equation*}

        De la primera ecuación, obtenemos:
        \begin{align*}
            \alpha_1 = -\alpha_0 - \alpha_2 = \dfrac{h_2^2-h_1^2}{h_1h_2\left(h_1+h_2\right)}
            = \dfrac{h_2-h_1}{h_1h_2}
        \end{align*}

        Por tanto, la fórmula de tipo interpolatorio en $\bb{P}_2$ para aproximar $f'(a)$ es:
        \[
        f'(a) \approx -\dfrac{h_2}{h_1\left(h_1+h_2\right)}f(a - h_1) + \dfrac{h_2-h_1}{h_1h_2}f(a) + \dfrac{h_1}{h_2\left(h_1+h_2\right)}f(a + h_2)
        \]

        Para el error, definimos:
        \begin{align*}
            \Pi(x) &= \prod_{i=0}^{2}(x - (a + (i-1)h_i)) = (x - a - h_1)(x - a)(x - a + h_2)\\
            \Pi'(x) &= (x-a)(x-a+h_2) + (x-a-h_1)(x-a+h_2) + (x-a-h_1)(x-a)
        \end{align*}

        El error cometido al interpolar $f$ en $a-h_1, a, a+h_2$ mediante tres nodos es:
        \begin{align*}
            E(x) &= f[a-h_1, a, a+h_2, x]\Pi(x)\\
            E'(x) &= f[a-h_1, a, a+h_2, x, x]\Pi(x) + f[a-h_1, a, a+h_2, x]\Pi'(x)
        \end{align*}

        Por tanto, tenemos que:
        \begin{align*}
            R(f) &= E'(a) = f[a-h_1, a, a+h_2, a]\Pi'(a)=-h_1h_2\cdot \frac{f^{(3)}(\xi)}{3!}\qquad \xi\in\left]a-h_1, a+h_2\right[
        \end{align*}

        Por tanto, el grado de exactitud de esta fórmula es 2, pues el error se anula en $\bb{P}_2$ pero no en $\bb{P}_3$.

        
        \item Use la tabla de valores
        \begin{center}
            \begin{tabular}{c|c}
                $x$ & $f(x)$ \\
                \hline
                $0.7$ & $-0.1$ \\
                $1.25$ & $0.2$ \\
                $1.5$ & $0.3$ \\
                $1.75$ & $0.25$
            \end{tabular}
        \end{center}
        para dar valores aproximados de $f'(0.7)$, $f'(1.25)$, $f'(1.5)$ y $f'(1.75)$ utilizando para cada uno de ellos la fórmula de derivación numérica más adecuada. Indique la fórmula usada en cada caso y justifique su uso.
        \begin{itemize}
            \item \ul{Para $f'(0.7)$:}
            
            En este caso, hemos de usar una fórmula progresiva. Empleamos la fórmula progresiva con dos nodos, usando $a=0.7$, $h=1.25-0.7=0.55$. Tenemos que:
            \begin{align*}
                f'(0.7) &\approx \dfrac{f(a+h) - f(a)}{h} = \dfrac{f(1.25) - f(0.7)}{0.55} = \dfrac{0.2 + 0.1}{0.55} = \dfrac{0.3}{0.55} \approx 0.545455
            \end{align*}

            \item \ul{Para $f'(1.25)$:}
            
            Empleamos la fórmula del apartado anterior, con $a=1.25$, $h_1=1.25-0.7=0.55$, $h_2=1.5-1.25=0.25$. Tenemos que:
            \begin{align*}
                f'(1.25) &\approx -\dfrac{0.25}{0.55\cdot 0.8}f(0.7) + \dfrac{0.25-0.55}{0.55\cdot 0.25}f(1.25) + \dfrac{0.55}{0.25\cdot 0.8}f(1.5)\\
                &\approx -\dfrac{0.25}{0.44}(-0.1) + \dfrac{-0.3}{0.1375}(0.2) + \dfrac{0.55}{0.2}(0.3) \approx 0.445455
            \end{align*}

            \item \ul{Para $f'(1.5)$:}
            
            Empleamos la fórmula centrada con 2 nodos (o 3 nodos, puesto que hemos visto que es la misma) con $a=1.5$, $h=0.25$. Tenemos que:
            \begin{align*}
                f'(1.5) &\approx \dfrac{f(a+h) - f(a-h)}{2h} = \dfrac{f(1.75) - f(1.25)}{0.5} = \dfrac{0.25 - 0.2}{0.5} = \dfrac{0.05}{0.5} = 0.1
            \end{align*}

            \item \ul{Para $f'(1.75)$:}
            
            Empleamos la fórmula regresiva con dos nodos, usando $a=1.75$, $h=0.25$. Tenemos que:
            \begin{align*}
                f'(1.75) &\approx \dfrac{f(a) - f(a-h)}{h} = \dfrac{f(1.75) - f(1.5)}{0.25} = \dfrac{0.25 - 0.3}{0.25} = \dfrac{-0.05}{0.25} = -0.2
            \end{align*}
        \end{itemize}
    \end{enumerate}
\end{ejercicio}

\begin{ejercicio}\label{ej:2.1.8}
    Halle una cota del valor absoluto del error que se comete al aproximar la derivada de la función $f(x) = \cos^2 x$ en $x = 0.8$ mediante la correspondiente fórmula de derivación numérica de tipo interpolatorio que usa los valores de $f$ en los puntos $0.6$, $0.8$ y $1$.\\

    Podríamos ver que se trata de una fórmula centrada en tres nodos, pero vamos a calcular directamente el error. Definimos:
    \begin{align*}
        \Pi(x) &= (x-0.6)(x-0.8)(x-1)\\
        \Pi'(x) &= (x-0.8)(x-1) + (x-0.6)(x-1) + (x-0.6)(x-0.8)
    \end{align*}

    El error cometido al interpolar $f$ en $0.6, 0.8, 1$ mediante tres nodos es:
    \begin{align*}
        E(x) &= f[0.6, 0.8, 1, x]\Pi(x)\\
        E'(x) &= f[0.6, 0.8, 1, x, x]\Pi(x) + f[0.6, 0.8, 1, x]\Pi'(x)
    \end{align*}

    Por tanto, tenemos que:
    \begin{align*}
        R(f) &= E'(0.8) = f[0.6, 0.8, 1, 0.8]\Pi'(0.8) = -0.04\cdot \dfrac{f^{(3)}(\xi)}{3!} \qquad \xi\in\left]0.6, 1\right[
    \end{align*}

    Para acotarlo, calculamos $f^{(3)}(x)$:
    \begin{align*}
        f(x) &= \cos^2 x\\
        f'(x) &= -2\cos x\sen x = -\sen 2x\\
        f''(x) &= -2\cos 2x\\
        f^{(3)}(x) &= 4\sen 2x
    \end{align*}

    Por tanto, una cota del error es:
    \begin{align*}
        \left|R(f)\right| &= \left|-\dfrac{0.04}{6}\cdot 4\sen(2\xi)\right|\leq \dfrac{0.04}{6}\cdot 4 = \dfrac{2}{75} \approx 0.0266667
    \end{align*}

\end{ejercicio}