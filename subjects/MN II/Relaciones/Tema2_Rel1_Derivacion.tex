\subsection{Relación 1. Derivación Numérica}
\setcounter{ejercicio}{0}


\begin{ejercicio}\label{ej:2.1.1}~
    \begin{enumerate}
        \item Use el método interpolatorio para obtener la fórmula de diferencia progresiva en dos nodos para aproximar $f'(a)$. ¿Cuál es el grado de exactitud de dicha fórmula? Justifique la respuesta.\\
        
        Supongamos que queremos aproximar $f'(a)$ mediante una fórmula de diferencia progresiva en dos nodos; es decir, sean los nodos $x_0=a, x_1=a+h$. Calculamos los polinomios básicos de Lagrange:
        \begin{align*}
            \ell_0(x) &= \frac{x - x_1}{x_0 - x_1} = \frac{x - (a+h)}{a - (a+h)} = \frac{x - a - h}{-h} = \frac{a+h-x}{h}\\
            \ell_1(x) &= \frac{x - x_0}{x_1 - x_0} = \frac{x - a}{a+h - a} = \frac{x - a}{h}\\
            \ell_0'(x) &= \nicefrac{-1}{h}\\
            \ell_1'(x) &= \nicefrac{1}{h}
        \end{align*}

        Por tanto, sabemos que:
        \begin{align*}
            f(x) &= f(a)\ell_0(x) + f(a+h)\ell_1(x) + E(x)\\
            f'(x) &= f(a)\ell_0'(x) + f(a+h)\ell_1'(x) + E'(x)
        \end{align*}

        Además, el error cometido al interpolar $f$ en $a, a+h$ mediante dos nodos es:
        \begin{align*}
            E(x) &= f[a, a+h, x]\Pi(x)\\
            E'(x) &= f[a, a+h, x, x]\Pi(x) + f[a, a+h, x]\Pi'(x)
        \end{align*}

        Por tanto:
        \begin{align*}
            f'(a) &= \dfrac{f(a+h) - f(a)}{h} + R(f)\\
            R(f) &= E'(a) = f[a, a+h, a]\Pi'(a) =
            -h\cdot \dfrac{f^{(2)}(\xi)}{2!}\qquad \xi\in\left]a, a+h\right[
        \end{align*}

        Por tanto, sabemos que esta fórmula es exacta en $\bb{P}_1$, pueso que en estos casos se anulará la segunda derivada. No es exacta en $\bb{P}_2$ porque el error no se anula.
        
        \item Utilice la fórmula de Taylor para hallar la fórmula mencionada en el apartado anterior y la expresión de su error cuando $f\in C^2[a, a + h]$.
        
        Desaroollamos $f$ en cada uno de los nodos alrededor del punto $a$ hasta el segundo término:
        \begin{align*}
            f(a+h) &= f(a) + hf'(a) + \frac{h^2}{2}f''(\xi_1)\qquad \xi_1\in\left]a, a+h\right[
        \end{align*}

        Por tanto, tenemos que:
        \begin{align*}
            f'(a) &= \frac{f(a+h) - f(a)}{h} - \frac{h}{2}f''(\xi_1)\qquad \xi_1\in\left]a, a+h\right[
        \end{align*}

        
        \item Use el método interpolatorio para obtener la fórmula de diferencia centrada en dos nodos para aproximar $f'(a)$. ¿Cuál es el grado de exactitud de dicha fórmula? Justifique la respuesta.
        
        Supongamos que queremos aproximar $f'(a)$ mediante una fórmula de diferencia centrada en dos nodos; es decir, sean los nodos $x_0=a-h, x_1=a+h$. Calculamos los polinomios básicos de Lagrange:
        \begin{align*}
            \ell_0(x) &= \frac{x - x_1}{x_0 - x_1} = \frac{x - (a+h)}{a-h - (a+h)} = \frac{x - a - h}{-2h}\\
            \ell_1(x) &= \frac{x - x_0}{x_1 - x_0} = \frac{x - (a-h)}{a+h - (a-h)} = \frac{x - a + h}{2h}\\
            \ell_0'(x) &= \nicefrac{-1}{2h}\\
            \ell_1'(x) &= \nicefrac{1}{2h}
        \end{align*}

        Por tanto, sabemos que:
        \begin{align*}
            f(x) &= f(a-h)\ell_0(x) + f(a+h)\ell_1(x) + E(x)\\
            f'(x) &= f(a-h)\ell_0'(x) + f(a+h)\ell_1'(x) + E'(x)
        \end{align*}

        Además, el error cometido al interpolar $f$ en $a-h, a+h$ mediante dos nodos es:
        \begin{align*}
            E(x) &= f[a-h, a+h, x]\Pi(x)\\
            E'(x) &= f[a-h, a+h, x, x]\Pi(x) + f[a-h, a+h, x]\Pi'(x)
        \end{align*}

        Por tanto:
        \begin{align*}
            f'(a) &= \dfrac{f(a+h) - f(a-h)}{2h} + R(f)\\
            R(f) &= E'(a) = f[a-h, a+h, a, a]\Pi(a) =
            - h^2\dfrac{f^{(3)}(\xi)}{3!}
        \end{align*}

        Por tanto, sabemos que esta fórmula es exacta en $\bb{P}_1$ y $\bb{P}_2$, pueso que en estos casos se anulará la tercera derivada. Por tanto, el grado de exactitud de esta fórmula es 2.
        
        \item Utilice la fórmula de Taylor para hallar la fórmula mencionada en el apartado anterior y la expresión de su error cuando $f'' \in C^3[a - h, a + h]$.
        
        Desaroollamos $f$ en cada uno de los nodos alrededor del punto $a$ hasta el segundo término:
        \begin{align*}
            f(a+h) &= f(a) + hf'(a) + \frac{h^2}{2}f''(a) + \frac{h^3}{6}f'''(\xi_1)\qquad \xi_1\in\left]a, a+h\right[\\
            f(a-h) &= f(a) - hf'(a) + \frac{h^2}{2}f''(a) - \frac{h^3}{6}f'''(\xi_2)\qquad \xi_2\in\left]a-h, a\right[
        \end{align*}

        Realizamos una combinación lineal de ambas expresiones, forzando a que el coeficiente de $f'(a)$ sea 1 y el coeficiente de $f(a)$ sea 0:
        \begin{equation*}
            \begin{pmatrix}
                1 & 1\\
                h & -h
            \end{pmatrix}
            \begin{pmatrix}
                \alpha_0\\
                \alpha_1
            \end{pmatrix}
            =
            \begin{pmatrix}
                0\\
                1
            \end{pmatrix}
            \Longrightarrow
            \begin{cases}
                \alpha_0 = \nicefrac{1}{2h}\\
                \alpha_1 = \nicefrac{-1}{2h}
            \end{cases}
        \end{equation*}

        Por tanto, tenemos que:
        \begin{align*}
            f'(a) &= \frac{f(a+h) - f(a-h)}{2h} + R(f)\\
            R(f) &= -\dfrac{h^2}{12}\left(f'''(\xi_1) + f'''(\xi_2)\right)\qquad \xi_1, \xi_2\in\left]a-h, a+h\right[
                \\&\AstIg -\dfrac{h^2}{6}f'''(\xi)\qquad \xi\in\left]a-h, a+h\right[
        \end{align*}
        donde en $(\ast)$ hemos empleado el Teorema del Valor Medio.
    \end{enumerate}    
\end{ejercicio}

\begin{ejercicio}\label{ej:2.1.2}
    Usando el método de los coeficientes indeterminados deduzca la fórmula de diferencia centrada en tres nodos para aproximar $f'(a)$ y compruebe que coincide con la fórmula de diferencia centrada en dos nodos.
\end{ejercicio}

\begin{ejercicio}\label{ej:2.1.3}
    Obtenga la fórmula de diferencia progresiva en tres nodos para aproximar $f'(a)$ calculando directamente sus coeficientes mediante la base de Lagrange del problema de interpolación unisolvente asociado a dicha fórmula. Halle la expresión del error para esta fórmula cuando $f \in C^4[a, a + 2h]$.
\end{ejercicio}

\begin{ejercicio}\label{ej:2.1.4}
    Se considera la fórmula de tipo interpolatorio en $\bb{P}_n$ siguiente:
    \[
    f''(c) \approx \alpha_0 f(x_0) + \alpha_1 f(x_1) + \cdots + \alpha_n f(x_n)
    \]
    con $x_0 < x_1 < \cdots < x_n$. Demuestre que si $f \in C^{n+3}[a, b]$ con $[a, b]$ tal que $x_0, x_1, \ldots, x_n, c \in [a, b]$, entonces
    \[
    R(f) = 2\cdot \frac{f^{(n+3)}(\xi_2)}{(n + 3)!}\Pi(c) + 2\cdot \frac{f^{(n+2)}(\xi_1)}{(n + 2)!}\Pi'(c) + \frac{f^{(n+1)}(\xi_0)}{(n + 1)!}\Pi''(c)
    \]
    siendo $\xi_0, \xi_1, \xi_2 \in \left] \min\{x_0, c\}, \max\{x_n, c\} \right[$ y $\Pi(x) = \prod\limits_{i=0}^{n}(x - x_i)$.\\

    Sabemos que el error cometido con la interpolación de $f$ en los nodos $x_0, x_1, \ldots, x_n$ mediante $p\in \bb{P}_n$ es:
    \[
    E(x) = f(x)-p(x)= f[x_0, x_1, \ldots, x_n, x]\Pi(x)
    \]

    Calculemos cada una de las derivadas de $E(x)$:
    \begin{align*}
        E'(x) &= f[x_0, x_1, \ldots, x_n, x,x]\Pi(x) + f[x_0, x_1, \ldots, x_n, x]\Pi'(x)\\
        E''(x) &= 2f[x_0, x_1, \ldots, x_n, x,x,x]\Pi(x) + 2f[x_0, x_1, \ldots, x_n, x,x]\Pi'(x) + f[x_0, x_1, \ldots, x_n, x]\Pi''(x)
    \end{align*}

    Por tanto, tenemos que:
    \begin{align*}
        R(f) &= L(f-p)=L(E)=E''(c)
        =\\&= 2f[x_0, x_1, \ldots, x_n, c,c,c]\Pi(c) + 2f[x_0, x_1, \ldots, x_n, c,c]\Pi'(c) + f[x_0, x_1, \ldots, x_n, c]\Pi''(c)
        =\\&= 2\cdot \frac{f^{(n+3)}(\xi_2)}{(n + 3)!}\Pi(c) + 2\cdot \frac{f^{(n+2)}(\xi_1)}{(n + 2)!}\Pi'(c) + \frac{f^{(n+1)}(\xi_0)}{(n + 1)!}\Pi''(c)
    \end{align*}
    con $\xi_0, \xi_1, \xi_2 \in \left] \min\{x_0, c\}, \max\{x_n, c\} \right[$, donde hemos empleado que los nodos están ordenados; es decir, $x_0<x_1<\cdots<x_n$.

\end{ejercicio}

\begin{ejercicio}\label{ej:2.1.5}
    Use el método de los coeficientes indeterminados para obtener la fórmula de diferencia regresiva en tres nodos para aproximar $f''(a)$. Halle la expresión del error de truncamiento para esta fórmula cuando $f \in C^5[a - 2h, a]$.\\
    
    Sabemos que la fórmula de diferencia regresiva en tres nodos para aproximar $f''(a)$ es de la forma:
    \[
    f''(a) \approx \alpha_0 f(a - 2h) + \alpha_1 f(a - h) + \alpha_2 f(a)
    \]
    donde $\alpha_0, \alpha_1, \alpha_2$ son los coeficientes que debemos determinar. Para ello, imponemos exactitud en $\bb{P}_2$:
    \begin{align*}
        0 &= \alpha_0 + \alpha_1 + \alpha_2\\
        0 &= (a-2h)\alpha_0 + (a-h)\alpha_1 + a\alpha_2\\
        2 &= (a-2h)^2\alpha_0 + (a-h)^2\alpha_1 + a^2\alpha_2
    \end{align*}

    Equivalentemente:
    \begin{align*}
        0 &= \alpha_0 + \alpha_1 + \alpha_2\\
        0 &= a\left(\alpha_0 + \alpha_1 + \alpha_2\right) - 2h\alpha_0 - h\alpha_1\\
        2 &= a^2\left(\alpha_0 + \alpha_1 + \alpha_2\right) +4h^2\alpha_0 + h^2\alpha_1 - 4ah\alpha_0 - 2ah\alpha_1
    \end{align*}

    Equivalentemente:
    \begin{align*}
        0 &= \alpha_0 + \alpha_1 + \alpha_2\\
        0 &= h(2\alpha_0 + \alpha_1)\\
        2 &= 2h^2\alpha_0 + h^2(2\alpha_0+\alpha_1) -2ah(2\alpha_0+\alpha_1)
    \end{align*}

    Equivalentemente:
    \begin{equation*}
        \left\{
            \begin{array}{rl}
                0 &= \alpha_0 + \alpha_1 + \alpha_2\\
                0 &= 2\alpha_0 + \alpha_1\\
                1 &= h^2\alpha_0
            \end{array}
        \right\}
        \quad\Longrightarrow\quad
        \left\{
            \begin{array}{rl}
                \alpha_0 &= \nicefrac{1}{h^2}\\
                \alpha_1 &= -2\alpha_0 = \nicefrac{-2}{h^2}\\
                \alpha_2 &= -\alpha_0 - \alpha_1 = \alpha_0 = \nicefrac{1}{h^2}
            \end{array}
        \right\}
    \end{equation*}

    Por tanto, la fórmula de diferencia regresiva en tres nodos para aproximar $f''(a)$ es:
    \[
    f''(a) \approx \frac{1}{h^2}f(a - 2h) - \frac{2}{h^2}f(a - h) + \frac{1}{h^2}f(a)
    \]

    Para el error, definimos:
    \begin{align*}
        \Pi(x) &= \prod_{i=0}^{2}(x - (a - ih)) = (x - a + 2h)(x - a + h)(x - a)\\
        \Pi'(x) &= (x-a+h)(x-a) + (x-a+2h)(x-a) + (x-a+2h)(x-a+h)\\
        \Pi''(x) &= 2(x-a) + 2(x-a+h) + 2(x-a+2h)
    \end{align*}

    Por el Ejercicio~\ref{ej:2.1.4}, puesto que $f \in C^5[a - 2h, a]$, tenemos que:
    \begin{align*}
        R(f) &= 2\cdot \frac{f^{(5)}(\xi_2)}{5!}\cancelto{0}{\Pi(a)} + 2\cdot \frac{f^{(4)}(\xi_1)}{4!}\Pi'(a) + \frac{f^{(3)}(\xi_0)}{3!}\Pi''(a)
        =\\&=2\cdot 2h^2\cdot \frac{f^{(4)}(\xi_1)}{4!} + 6h\cdot \frac{f^{(3)}(\xi_0)}{3!}
        =\\&=h^2\cdot \frac{f^{(4)}(\xi_1)}{6} + h\cdot f^{(3)}(\xi_0)
    \end{align*}
    con $\xi_0, \xi_1, \xi_2 \in \left] a-2h, a \right[$.
\end{ejercicio}

\begin{ejercicio}\label{ej:2.1.6}
    Use la fórmula de Taylor para obtener, cuando se tenga que $f \in C^4[a - h, a + h]$, la fórmula de diferencia centrada en tres nodos para aproximar $f''(a)$ y la expresión de su error. ¿Cuál es el orden de precisión de esta fórmula? ¿Por qué?
\end{ejercicio}

\begin{ejercicio}\label{ej:2.1.7}~
    \begin{enumerate}
        \item Halle la fórmula de tipo interpolatorio en $P_2$ de la forma
        \[
        f'(a) \approx \alpha_0 f(a - h_1) + \alpha_1 f(a) + \alpha_2 f(a + h_2)
        \]
        con $h_1, h_2 > 0$, así como la expresión de su error de truncamiento cuando $f \in C^4[a - h_1, a + h_2]$. ¿Cuál es el grado de exactitud de esta fórmula? Justifique la respuesta.
        
        \item Use la tabla de valores
        \begin{center}
            \begin{tabular}{c|c}
                $x$ & $f(x)$ \\
                \hline
                $0.7$ & $-0.1$ \\
                $1.25$ & $0.2$ \\
                $1.5$ & $0.3$ \\
                $1.75$ & $0.25$
            \end{tabular}
        \end{center}
        para dar valores aproximados de $f'(0.7)$, $f'(1.25)$, $f'(1.5)$ y $f'(1.75)$ utilizando para cada uno de ellos la fórmula de derivación numérica más adecuada. Indique la fórmula usada en cada caso y justifique su uso.
    \end{enumerate}
\end{ejercicio}

\begin{ejercicio}\label{ej:2.1.8}
    Halle una cota del valor absoluto del error que se comete al aproximar la derivada de la función $f(x) = \cos^2 x$ en $x = 0.8$ mediante la correspondiente fórmula de derivación numérica de tipo interpolatorio que usa los valores de $f$ en los puntos $0.6$, $0.8$ y $1$.
\end{ejercicio}