\subsection{Relación 2. Derivación Numérica}
\setcounter{ejercicio}{0}


\begin{ejercicio}\label{ej:2.2.1}~
    \begin{enumerate}
        \item Obtén la fórmula progresiva de derivación numérica de tipo interpolatorio clásico para aproximar $f'(a)$ a partir de $f(a)$ y $f(a + h)$, mediante desarrollo de Taylor de $f(a + h)$ en torno a $a$ hasta el cuarto término.
        
        Buscamos obtener $\alpha_0,\alpha_1\in \bb{R}$ tales que:
        \begin{equation*}
            f'(a) = \alpha_0 f(a) + \alpha_1 f(a + h) + R(f)
        \end{equation*}

        Desarrollando en serie de Taylor $f(a)$ y $f(a + h)$ en torno a $a$ hasta el cuarto término, tenemos:
        \begin{align*}
            f(a) &= f(a) \\
            f(a + h) &= f(a) + hf'(a) + \frac{h^2}{2}f''(a) + \frac{h^3}{6}f'''(a) + \frac{h^4}{4!}f^{(4)}(a) + \dfrac{h^5}{5!}f^{(5)}(\xi)
        \end{align*}
        donde $\xi\in\left]a,a+h\right[$. Multiplicando por $\alpha_0$ y $\alpha_1$ respectivamente y sumando, obtenemos:
        \begin{multline*}
            \alpha_0 f(a) + \alpha_1 f(a + h) = (\alpha_0 + \alpha_1)f(a) + \alpha_1hf'(a) + \frac{\alpha_1h^2}{2}f''(a) + \frac{\alpha_1h^3}{6}f'''(a) +\\+ \frac{\alpha_1h^4}{4!}f^{(4)}(a) + \dfrac{\alpha_1h^5}{5!}f^{(5)}(\xi)
        \end{multline*}

        Por tanto, tenemos que:
        \begin{equation*}
            \left\{\begin{aligned}
                \alpha_0 + \alpha_1 &= 0 \\
                \alpha_1\cdot h &= 1
            \end{aligned}\right\}
            \Longrightarrow
            \left\{\begin{aligned}
                \alpha_0 &= -\frac{1}{h} \\
                \alpha_1 &= \frac{1}{h}
            \end{aligned}\right\}
        \end{equation*}

        Por lo tanto, la fórmula progresiva de derivación numérica de tipo interpolatorio clásico para aproximar $f'(a)$ a partir de $f(a)$ y $f(a + h)$ es:
        \begin{equation*}
            f'(a) = \dfrac{f(a + h) - f(a)}{h} + R(f)
        \end{equation*}

        Respecto al error, tenemos que:
        \begin{align*}
            R(f) &= -\dfrac{h}{2}f''(a) -\dfrac{h^2}{6}f'''(a) -\dfrac{h^3}{4!}f^{(4)}(a) - \dfrac{h^4}{5!}f^{(5)}(\xi) \\
            &= -\dfrac{h}{2}f''(\mu)\qquad \text{para algún }\mu\in\left]a,a+h\right[
        \end{align*}

        % // TODO: Ese paso por qué?


        
        \item Si notamos por $F(a, h)$ la aproximación de $f'(a)$ obtenida anteriormente, expresa el valor exacto de $f'(a)$ en función de $F(a, h)$ y los restantes términos en el desarrollo de Taylor.
        
        \item A partir de una combinación de los valores $F(a, h)$ y $F(a, h = 2)$ obtén una fórmula con mayor orden de precisión que $F(a, h)$.
        
        \item Aplica las dos fórmulas obtenidas para aproximar $f'(2)$ con $h = 0.1$ para la función $f(x) = \ln(x)$, $x \in [1, 3]$.
    \end{enumerate}
\end{ejercicio}

\begin{ejercicio}\label{ej:2.2.2}
    Para evaluar el funcional $L(f) = 2f'(a) - f''(a)$ se propone una fórmula del tipo
    \[
    2f'(a) - f''(a) \approx \alpha_0 f(a - h) + \alpha_1 f(a) + \alpha_2 f(a + h) :
    \]
    \begin{enumerate}
        \item Imponiendo exactitud en el espacio correspondiente halla la fórmula anterior para que sea de tipo interpolatorio clásico.
        
        \item Obtén una expresión del error de la fórmula en función de unas o varias derivadas de la función de órdenes superiores a dos.
        
        \item Aplica la fórmula obtenida para aproximar $2f'(2) - f''(2)$ con $h = 0.1$ para la función $f(x) = \ln(x)$, $x \in [1, 3]$.
        
        \item Compara el error real obtenido en en el apartado anterior con respecto a una cota deducida de 2).
        
        \item Aplica la fórmula para obtener $2f'(0) - f''(0)$ suponiendo que tienes la siguiente tabla de valores de $f$:
        \begin{center}
            \begin{tabular}{c|c}
                $x_i$ & $f(x_i)$ \\
                \hline
                $-0.2$ & $9$ \\
                $0$ & $10$ \\
                $0.2$ & $9$ \\
                $0.4$ & $12$
            \end{tabular}
        \end{center}
    \end{enumerate}
\end{ejercicio}

\begin{ejercicio}\label{ej:2.2.3}
    Considera la fórmula de tipo interpolatorio clásico siguiente
    \[
    f'(a) \approx \alpha_0 f(a - h) + \alpha_1 f(a + 3h)
    \]
    \begin{enumerate}
        \item Da una expresión del error de dicha fórmula.
        
        \item Úsala para aproximar la derivada $f'(3)$ siendo $f(x) = x^3$ con $h = 0.1$.
    \end{enumerate}
\end{ejercicio}

\begin{ejercicio}\label{ej:2.2.4}
    Determina razonadamente si es posible diseñar una fórmula numérica de tipo interpolatorio en el espacio generado por $\cc{L}\{1, x, x^2, x^4\}$ para aproximar
    \[
    \int_{-2}^{2} f(x) \ d{x} + \int_{-2}^{2} |x|f(x) \ d{x}
    \]
    usando para ello los datos
    \[
    \int_{-1}^{1} f(x) \ d{x}, \quad \int_{-1}^{1} |x|f(x) \ d{x}, \quad f(0) \quad \text{y} \quad f'(0).
    \]
    En particular determina el peso de $f'(0)$.
\end{ejercicio}

\begin{ejercicio}\label{ej:2.2.5}
    Dada la fórmula de derivación numérica de tipo interpolatorio:
    \[
    f'(0) = \alpha_0 f(0) + \alpha_1 f(x_1) + \alpha_2 f(x_2) + R(f), \quad x_1 \neq 0, x_2 \neq 0, x_1 \neq x_2.
    \]
    \begin{enumerate}
        \item Sin realizar ningún cálculo, ¿puedes indicar el máximo grado de exactitud que puede tener la fórmula? Justifica la respuesta.
        
        \item Determina los valores de $\alpha_0$, $\alpha_1$, $\alpha_2$, $x_1$ y $x_2$ para que la fórmula tenga el mayor grado de exactitud posible. ¿Cuál es ese grado de exactitud?
        
        \item Determina la expresión del error indicando las condiciones sobre derivabilidad de la función $f$. ¿Hay alguna otra conclusión que obtengas respecto a los nodos?
        
        \item Aplica el resultado para la función $x e^{x^2 + 1}$.
        
    \end{enumerate}
\end{ejercicio}

\begin{ejercicio}\label{ej:2.2.6}
    Dada la fórmula de derivación numérica de tipo interpolatorio:
    \[
    f'(0) = \alpha_0 f(-1) + \alpha_1 f(1) + \alpha_2 f(2) + \alpha_3 f(a) + R(f), \quad a \neq -1, 1, 2.
    \]
    \begin{enumerate}
        \item Sin realizar ningún cálculo, ¿puedes indicar el máximo grado de exactitud que puede tener la fórmula? Justifica la respuesta.
        
        \item Determina los valores de $\alpha_0$, $\alpha_1$, $\alpha_2$, $\alpha_3$ y $a$ para que la fórmula tenga el mayor grado de exactitud posible. ¿Cuál es ese grado de exactitud?
        
        \item Determina la expresión del error indicando las condiciones sobre derivabilidad de la función $f$.
        
        \item Aplica el resultado para la función $x e^{x^2 + 1}$.
        
    \end{enumerate}
\end{ejercicio}