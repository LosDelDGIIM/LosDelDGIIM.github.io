\subsection{Relación 2. Derivación Numérica}
\setcounter{ejercicio}{0}


\begin{ejercicio}\label{ej:2.2.1}~
    \begin{enumerate}
        \item Obtén la fórmula progresiva de derivación numérica de tipo interpolatorio clásico para aproximar $f'(a)$ a partir de $f(a)$ y $f(a + h)$, mediante desarrollo de Taylor de $f(a + h)$ en torno a $a$ hasta el cuarto término.
        
        Buscamos obtener $\alpha_0,\alpha_1\in \bb{R}$ tales que:
        \begin{equation*}
            f'(a) = \alpha_0 f(a) + \alpha_1 f(a + h) + R(f)
        \end{equation*}

        Desarrollando en serie de Taylor $f(a)$ y $f(a + h)$ en torno a $a$ hasta el cuarto término, tenemos:
        \begin{align*}
            f(a) &= f(a) \\
            f(a + h) &= f(a) + hf'(a) + \frac{h^2}{2}f''(a) + \frac{h^3}{6}f'''(a) + \frac{h^4}{4!}f^{(4)}(\xi)
        \end{align*}
        donde $\xi\in\left]a,a+h\right[$. Multiplicando por $\alpha_0$ y $\alpha_1$ respectivamente y sumando, obtenemos:
        \begin{equation*}
            \alpha_0 f(a) + \alpha_1 f(a + h) = (\alpha_0 + \alpha_1)f(a) + \alpha_1hf'(a) + \frac{\alpha_1h^2}{2}f''(a) + \frac{\alpha_1h^3}{6}f'''(a) + \frac{\alpha_1h^4}{4!}f^{(4)}(\xi)
        \end{equation*}

        Por tanto, tenemos que:
        \begin{equation*}
            \left\{\begin{aligned}
                \alpha_0 + \alpha_1 &= 0 \\
                \alpha_1\cdot h &= 1
            \end{aligned}\right\}
            \Longrightarrow
            \left\{\begin{aligned}
                \alpha_0 &= -\frac{1}{h} \\
                \alpha_1 &= \frac{1}{h}
            \end{aligned}\right\}
        \end{equation*}

        Por lo tanto, la fórmula progresiva de derivación numérica de tipo interpolatorio clásico para aproximar $f'(a)$ a partir de $f(a)$ y $f(a + h)$ es:
        \begin{equation*}
            f'(a) = \dfrac{f(a + h) - f(a)}{h} + R(f)
        \end{equation*}

        Respecto al error, tenemos que:
        \begin{align*}
            R(f) &= -\dfrac{h}{2}f''(a) -\dfrac{h^2}{6}f'''(a) -\dfrac{h^3}{4!}f^{(4)}(\xi) 
        \end{align*}

        
        \item Si notamos por $F(a, h)$ la aproximación de $f'(a)$ obtenida anteriormente, expresa el valor exacto de $f'(a)$ en función de $F(a, h)$ y los restantes términos en el desarrollo de Taylor.
        
        \begin{equation*}
            f'(a) = F(a, h) - \dfrac{h}{2}f''(a) -\dfrac{h^2}{6}f'''(a) -\dfrac{h^3}{4!}f^{(4)}(\xi)
        \end{equation*}
        
        \item A partir de una combinación de los valores $F(a, h)$ y $F(a, \nicefrac{h}{2})$ obtén una fórmula con mayor orden de precisión que $F(a, h)$.
        
        Sabemos que:
        \begin{align*}
            F(a,h) &= \dfrac{f(a + h) - f(a)}{h}\\
            F(a,\nicefrac{h}{2}) &= \dfrac{f(a + \nicefrac{h}{2}) - f(a)}{\nicefrac{h}{2}} = \dfrac{2\left(f(a + \nicefrac{h}{2}) - f(a)\right)}{h}
        \end{align*}

        Empleando lo obtenido para el error, pero usando ahora $\nicefrac{h}{2}$, tenemos que:
        \begin{align*}
            f'(a) = F(a,\nicefrac{h}{2}) -\dfrac{h}{4}f''(a) -\dfrac{h^2}{4!}f'''(a) -\dfrac{h^3}{8\cdot 4!}f^{(4)}(\xi) 
        \end{align*}

        Multiplicamos por $2$ esta nueva expresión, y le restamos la expresión anterior:
        \begin{align*}
            f'(a) &= 2f'(a)-f'(a)=\\
            &= 2\left(F(a,\nicefrac{h}{2}) -\dfrac{h}{4}f''(a) -\dfrac{h^2}{4!}f'''(a) -\dfrac{h^3}{8\cdot 4!}f^{(4)}(\xi)\right)-\\&\hspace{2cm}-\left(F(a, h) - \dfrac{h}{2}f''(a) -\dfrac{h^2}{6}f'''(a) -\dfrac{h^3}{4!}f^{(4)}(\xi)\right)=\\
            &= \left(2F(a,\nicefrac{h}{2}) -\cancel{\dfrac{h}{2}f''(a)} -\dfrac{h^2}{12}f'''(a) -\dfrac{h^3}{4\cdot 4!}f^{(4)}(\xi)\right)-\\&\hspace{2cm}-\left(F(a, h) - \cancel{\dfrac{h}{2}f''(a)} -\dfrac{h^2}{6}f'''(a) -\dfrac{h^3}{4!}f^{(4)}(\xi)\right)
            =\\&=
            2F(a,\nicefrac{h}{2})-F(a,h)+\dfrac{h^2}{12}f'''(a)-\dfrac{h^3}{4\cdot 4!}f^{(4)}(\xi)+\dfrac{h^3}{4!}f^{(4)}(\xi)
            =\\&=
            2F(a,\nicefrac{h}{2})-F(a,h)+\dfrac{h^2}{12}f'''(a)+\dfrac{3h^3}{4\cdot 4!}f^{(4)}(\xi)
        \end{align*}

        Como vemos, hemos conseguido una fórmula que, en vez de ser exacta en $\bb{P}_1$, es exacta en $\bb{P}_2$.
        
        \item Aplica las dos fórmulas obtenidas para aproximar $f'(2)$ con $h = 0.1$ para la función $f(x) = \ln(x)$, $x \in [1, 3]$.
    \end{enumerate}
\end{ejercicio}

\begin{ejercicio}\label{ej:2.2.2}
    Para evaluar el funcional $L(f) = 2f'(a) - f''(a)$ se propone una fórmula del tipo
    \[
    2f'(a) - f''(a) \approx \alpha_0 f(a - h) + \alpha_1 f(a) + \alpha_2 f(a + h) :
    \]
    \begin{enumerate}
        \item Imponiendo exactitud en el espacio correspondiente halla la fórmula anterior para que sea de tipo interpolatorio clásico.
        
        En este caso, como hay $3$ nodos, es necesario imponer exactitud en $\bb{P}_2$, o equivalentemente, en $\cc{L}\{1, x, x^2\}$. Por tanto, buscamos $\alpha_0,\alpha_1,\alpha_2\in\bb{R}$ tales que:
        \begin{align*}
            2\cdot 0 - 0 &= \alpha_0\cdot 1 + \alpha_1\cdot 1 + \alpha_2\cdot 1 \\
            2\cdot 1 - 0 &= \alpha_0\cdot (a-h) + \alpha_1\cdot a + \alpha_2\cdot (a+h) \\
            2\cdot 2a - 2 &= \alpha_0\cdot (a-h)^2 + \alpha_1\cdot a^2 + \alpha_2\cdot (a+h)^2
        \end{align*}

        Por tanto, se trata de resolver el siguiente sistema de ecuaciones:
        \begin{equation*}
            \begin{pmatrix}
                1 & 1 & 1 \\
                a-h & a & a+h \\
                (a-h)^2 & a^2 & (a+h)^2
            \end{pmatrix}
            \begin{pmatrix}
                \alpha_0 \\
                \alpha_1 \\
                \alpha_2
            \end{pmatrix}=
            \begin{pmatrix}
                0 \\
                2 \\
                4a-2
            \end{pmatrix}
        \end{equation*}

        Planteamos la matriz ampliada, y aplicamos el método de Gauss:
        \begin{align*}
            &\left(\begin{array}{ccc|c}
                1 & 1 & 1 & 0 \\
                a-h & a & a+h & 2 \\
                (a-h)^2 & a^2 & (a+h)^2 & 4a-2
            \end{array}\right) \underrightarrow{F_3'=F_3-(a-h)F_2}
            \\&\left(\begin{array}{ccc|c}
                1 & 1 & 1 & 0 \\
                a-h & a & a+h & 2 \\
                0 & ah & 2h(a+h) & 2(a+h-1)
            \end{array}\right)\underrightarrow{F_2'=F_2-(a-h)F_1}
            \\&\left(\begin{array}{ccc|c}
                1 & 1 & 1 & 0 \\
                0 & h & 2h & 2 \\
                0 & ah & 2h(a+h) & 2(a+h-1)
            \end{array}\right)\underrightarrow{F_3'=F_3-aF_2}
            \\&\left(\begin{array}{ccc|c}
                1 & 1 & 1 & 0 \\
                0 & h & 2h & 2 \\
                0 & 0 & 2h^2 & 2(h-1)
            \end{array}\right)
            \Longrightarrow
            \begin{cases}
                \alpha_2 = \dfrac{h-1}{h^2} \\
                \alpha_1 = \dfrac{2-2\cdot \frac{h-1}{h}}{h} = \dfrac{2}{h^2}\\
                \alpha_0 = -\dfrac{2+h-1}{h^2} = -\dfrac{1+h}{h^2}
            \end{cases}
        \end{align*}

        Por tanto, al ser exacta en $\bb{P}_2$, la siguiente fórmula es de tipo interpolatorio clásico:
        \begin{equation*}
            2f'(a) - f''(a) \approx -\dfrac{1+h}{h^2}f(a-h) + \dfrac{2}{h^2}f(a) + \dfrac{h-1}{h^2}f(a+h)
        \end{equation*}
        
        \item Obtén una expresión del error de la fórmula en función de unas o varias derivadas de la función de órdenes superiores a dos.
        
        Definimos en primer lugar:
        \begin{align*}
            \Pi(x) &= (x-a+h)(x-a)(x-a-h)\\
            \Pi'(x) &= (x-a)(x-a-h) + (x-a+h)(x-a-h) + (x-a+h)(x-a)\\
            \Pi''(x) &= 2(x-a+h) + 2(x-a) + 2(x-a-h)
        \end{align*}
        
        Sabemos que el error del polinomio de interpolación en los $3$ nodos dados es:
        \begin{equation*}
            E(x) = f[a-h, a, a+h, x]\Pi(x)
        \end{equation*}

        Calculemos su derivada primera y segunda:
        \begin{align*}
            \hspace{-1cm}E'(x) &= f[a-h, a, a+h, x, x]\Pi(x) + f[a-h, a, a+h, x]\Pi'(x)\\
            \hspace{-1cm}E''(x) &= 2f[a-h, a, a+h, x, x, x]\Pi(x) + 2f[a-h, a, a+h, x, x]\Pi'(x)+f[a-h, a, a+h, x]\Pi''(x)
        \end{align*}

        Evaluamos cada una de las derivadas en $a$:
        \begin{align*}
            E'(a) &= -f[a-h, a, a+h, a]h^2\\
            E''(a) &= -2f[a-h, a, a+h, a, a]h^2
        \end{align*}

        Por ser de tipo interpolatorio clásico, el error de la fórmula es:
        \begin{align*}
            R(f) &= L(E) = 2E'(a) - E''(a) \\
            &= -2f[a-h, a, a+h, a]h^2+2f[a-h, a, a+h, a, a]h^2
            =\\&= 2h^2\left(f[a-h, a, a+h, a, a]-f[a-h, a, a+h, a]\right)
            =\\&= 2h^2\left(\dfrac{f^{(4)}(\xi_1)}{4!}-\dfrac{f^{(3)}(\xi_2)}{3!}\right)\qquad \text{para algún }\xi_1,\xi_2\in\left]a-h,a+h\right[
        \end{align*}
        
        \item Aplica la fórmula obtenida para aproximar $2f'(2) - f''(2)$ con $h = 0.1$ para la función $f(x) = \ln(x)$, $x \in [1, 3]$.
        \begin{align*}
            2f'(2) - f''(2) &\approx -\dfrac{1+0.1}{0.1^2}f(1.9) + \dfrac{2}{0.1^2}f(2) + \dfrac{0.1-1}{0.1^2}f(2.1)\\
            &= -\dfrac{1.1}{0.01}\ln(1.9) + \dfrac{2}{0.01}\ln(2) + \dfrac{-0.9}{0.01}\ln(2.1) \approx 1.2511476
        \end{align*}
        
        \item Compara el error real obtenido en en el apartado anterior con respecto a una cota deducida de 2).
        
        El valor real es:
        \begin{align*}
            2f'(2) - f''(2) &= 2\left(\dfrac{1}{2}\right) - \left(-\dfrac{1}{2^2}\right) = \dfrac{5}{4} = 1.25
        \end{align*}

        Por tanto, el error real obtenido es:
        \begin{align*}
            \text{Error real} &\approx 1.25 - 1.2511476 \approx -0.147607\cdot 10^{-3}
        \end{align*}

        El error obtenido en 2) es:
        \begin{align*}
            2\cdot 0.1^2\left(\dfrac{f^{(4)}(\xi_1)}{4!}-\dfrac{f^{(3)}(\xi_2)}{3!}\right)=0.02\left(\dfrac{f^{(4)}(\xi_1)}{4!}-\dfrac{f^{(3)}(\xi_2)}{3!}\right)\qquad \xi_1,\xi_2\in\left]1.9,2.1\right[
        \end{align*}
        
        Para poder acotarlo, necesitamos acotar las derivadas cuarta y tercera de $f$. Como $f(x)=\ln(x)$, tenemos que:
        \begin{align*}
            f'(x) &= \dfrac{1}{x} &
            f''(x) &= -\dfrac{1}{x^2} &
            f^{(3)}(x) &= \dfrac{2}{x^3} &
            f^{(4)}(x) &= -\dfrac{6}{x^4}
        \end{align*}
        \begin{observacion}
            Notemos que este ejercicio carece de sentido práctico, puesto que si podemos calcular las funciones derivadas podemos calcular el funcional pedido. No obstante, en la práctica se supone que no podremos calcular estas derivadas, pero sí las tendremos acotadas de alguna forma (por ejemplo, piénsese en las funciones trigonométricas).
        \end{observacion}

        Por tanto, la cota del error es:
        \begin{align*}
            \left|\text{Error Predicho}\right|
            &\leq 0.02\left(\dfrac{|f^{(4)}(\xi_1)|}{4!}+\dfrac{|f^{(3)}(\xi_2)|}{3!}\right)\leq 0.02\left(\dfrac{6}{1.9^4\cdot 4!}+\dfrac{2}{1.9^3\cdot 3!}\right)\\
            &\leq 0.02\left(\dfrac{1}{1.9^4\cdot 4}+\dfrac{1}{1.9^3\cdot 3}\right)\approx 1.355627\cdot 10^{-3}
        \end{align*}
        
        \item Aplica la fórmula para obtener $2f'(0) - f''(0)$ suponiendo que tienes la siguiente tabla de valores de $f$:
        \begin{center}
            \begin{tabular}{c|c}
                $x_i$ & $f(x_i)$ \\
                \hline
                $-0.2$ & $9$ \\
                $0$ & $10$ \\
                $0.2$ & $9$ \\
                $0.4$ & $12$
            \end{tabular}
        \end{center}

        Tomando $h=0.2$, tenemos que:
        \begin{align*}
            2f'(0) - f''(0) &\approx -\dfrac{1+0.2}{0.2^2}f(-0.2) + \dfrac{2}{0.2^2}f(0) + \dfrac{0.2-1}{0.2^2}f(0.2)\\
            &= -\dfrac{1.2}{0.04}\cdot 9 + \dfrac{2}{0.04}\cdot 10 + \dfrac{-0.8}{0.04}\cdot 9 = 50
        \end{align*}
    \end{enumerate}
\end{ejercicio}

\begin{ejercicio}\label{ej:2.2.3}
    Considera la fórmula de tipo interpolatorio clásico siguiente
    \[
    f'(a) \approx \alpha_0 f(a - h) + \alpha_1 f(a + 3h)
    \]
    \begin{enumerate}
        \item Da una expresión del error de dicha fórmula.

        Definimos en primer lugar:
        \begin{align*}
            \Pi(x) &= (x-a+h)(x-a-3h)\\
            \Pi'(x) &= (x-a-3h) + (x-a+h)
        \end{align*}

        Sabemos que el error del polinomio de interpolación en los $2$ nodos dados es:
        \begin{equation*}
            E(x) = f[a-h, a+3h, x]\Pi(x)
        \end{equation*}

        Calculemos su derivada primera:
        \begin{align*}
            E'(x) &= f[a-h, a+3h, x, x]\Pi(x) + f[a-h, a+3h, x]\Pi'(x)
        \end{align*}

        Por ser de tipo interpolatorio clásico, el error de la fórmula es:
        \begin{align*}
            R(f) &= L(E) = E'(a) = f[a-h, a+3h, a,a]\Pi(a) + f[a-h, a+3h, a]\Pi'(a)\\
            &= -3h^2f[a-h, a+3h, a,a]-2hf[a-h, a+3h, a]
            =\\&= -3h^2\left(\dfrac{f^{(3)}(\xi_1)}{3!}\right)-2h\left(\dfrac{f^{(2)}(\xi_2)}{2!}\right)
            =\\&= -h^2\left(\dfrac{f^{(3)}(\xi_1)}{2}\right)-h\left(f^{(2)}(\xi_2)\right)\qquad \text{para algún }\xi_1,\xi_2\in\left]a-h,a+3h\right[
        \end{align*}


        
        \item Úsala para aproximar la derivada $f'(3)$ siendo $f(x) = x^3$ con $h = 0.1$.
        
        Calculamos los polinomios fundamentales de Lagrange:
        \begin{align*}
            \ell_0(x) &= \dfrac{x-(a+2h)}{a-h-(a+3h)} = \dfrac{x-a-2h}{-4h}
            \Longrightarrow \ell_0'(x) = -\dfrac{1}{4h} \\
            \ell_1(x) &= \dfrac{x-(a-h)}{a+3h-(a-h)} = \dfrac{x-a+h}{4h}
            \Longrightarrow \ell_1'(x) = \dfrac{1}{4h}
        \end{align*}

        Por tanto, la fórmula de derivación numérica es:
        \begin{align*}
            f'(3) &\approx -\dfrac{1}{0.4}f(3-0.1) + \dfrac{1}{0.4}f(3+0.3) = 28.87
        \end{align*}
    \end{enumerate}
\end{ejercicio}

\begin{ejercicio}[DGIIM 2023/2024]\label{ej:2.2.5}
    Dada la fórmula de derivación numérica de tipo interpolatorio:
    \[
    f''(0) = \alpha_0 f(0) + \alpha_1 f(x_1) + \alpha_2 f(x_2) + R(f), \quad x_1 \neq 0, x_2 \neq 0, x_1 \neq x_2.
    \]
    \begin{enumerate}
        \item Sin realizar ningún cálculo, ¿puedes indicar el máximo grado de exactitud que puede tener la fórmula? Justifica la respuesta.
        
        Como tiene $3$ nodos y se trata de la segunda derivada, sabemos que no puede ser exacta en $\bb{P}_5$, por lo que el máximo grado de exactitud que puede tener la fórmula es $4$.
        
        \item Determina los valores de $\alpha_0$, $\alpha_1$, $\alpha_2$, $x_1$ y $x_2$ para que la fórmula tenga el mayor grado de exactitud posible. ¿Cuál es ese grado de exactitud?
        
        Imponemos exactitud en $\bb{P}_4$, o equivalentemente, en $\cc{L}\{1, x, x^2,x^3,x^4\}$. Por tanto, buscamos $\alpha_0,\alpha_1,\alpha_2\in\bb{R}$ tales que:
        \begin{align*}
            0 &= \alpha_0 + \alpha_1 + \alpha_2 \\
            0 &= \alpha_1x_1 + \alpha_2x_2 \\
            2 &= \alpha_1x_1^2 + \alpha_2x_2^2 \\
            0 &= \alpha_1x_1^3 + \alpha_2x_2^3\\
            0 &= \alpha_1x_1^4 + \alpha_2x_2^4
        \end{align*}

        Veamos qué es necesario para que el sistema sea compatible:
        \begin{align*}
            \begin{vmatrix}
                x_1 & x_2 & 0 \\
                x_1^2 & x_2^2 & 2 \\
                x_1^3 & x_2^3 & 0
            \end{vmatrix} &= -2\left(x_1x_2^3-x_1^3x_2\right) = -2x_1x_2(x_2^2-x_1^2) = -2x_1x_2(x_2+x_1)(x_2-x_1)
        \end{align*}
        Por tanto, por el Teorema de Rouché-Frobenius, si $x_2\neq -x_1$, el sistema es incompatible y no tiene solución. Por tanto, imponemos $x_2=-x_1$ y el sistema queda:
        \begin{align*}
            0 &= \alpha_0 + \alpha_1 + \alpha_2 \\
            0 &= x_1(\alpha_1- \alpha_2) \\
            2 &= x_1^2(\alpha_1 + \alpha_2) \\
            0 &= x_1^3(\alpha_1 - \alpha_2)\\
            0 &= x_1^4(\alpha_1 + \alpha_2)
        \end{align*}

        Como $x_1\neq 0$, obtenemos que este sistema es incompatible. Por tanto, no es posible imponer exactitud en $\bb{P}_4$. Veamos si es posible imponer exactitud en $\bb{P}_3$:
        \begin{align*}
            0 &= \alpha_0 + \alpha_1 + \alpha_2 \\
            0 &= \alpha_1x_1 + \alpha_2x_2 \\
            2 &= \alpha_1x_1^2 + \alpha_2x_2^2 \\
            0 &= \alpha_1x_1^3 + \alpha_2x_2^3\\
        \end{align*}

        Anteriormente vimos que es necesario imponer $x_2=-x_1$. Por tanto, el sistema queda:
        \begin{align*}
            \left\{
                \begin{array}{rl}
                    0 &= \alpha_0 + \alpha_1 + \alpha_2 \\
                    0 &= x_1(\alpha_1- \alpha_2) \\
                    2 &= x_1^2(\alpha_1 + \alpha_2) \\
                    0 &= x_1^3(\alpha_1 - \alpha_2)\\
                \end{array}
            \right\}
            \Longrightarrow
            \left\{
                \begin{array}{rl}
                    \alpha_0 &= -\dfrac{2}{x_1^2} \\
                    \alpha_1 &= \dfrac{1}{x_1^2} \\
                    \alpha_2 &= \dfrac{1}{x_1^2}
                \end{array}
            \right\}
        \end{align*}

        Por tanto, el grado máximo de exactitud es $3$ y la fórmula de derivación numérica de tipo interpolatorio es:
        \begin{equation*}
            f''(0) = -\dfrac{2}{x_1^2}f(0) + \dfrac{1}{x_1^2}f(x_1) + \dfrac{1}{x_1^2}f(-x_1) + R(f)
        \end{equation*}
        donde notemos que hemos impuesto que los nodos sean simétricos respecto al origen.

        \item Determina la expresión del error indicando las condiciones sobre derivabilidad de la función $f$. ¿Hay alguna otra conclusión que obtengas respecto a los nodos?
        
        Definimos en primer lugar:
        \begin{align*}
            \Pi(x) &= x(x-x_1)(x-x_2)\\
            \Pi'(x) &= (x-x_1)(x-x_2) + x(x-x_2) + x(x-x_1)\\
            \Pi''(x) &= 2(x-x_1) + 2(x-x_2) + 2x
        \end{align*}

        Sabemos que el error del polinomio de interpolación en los $3$ nodos dados es:
        \begin{align*}
            E(x) &= f[0, x_1, x_2, x]\Pi(x)\\
            E'(x) &= f[0, x_1, x_2, x, x]\Pi(x) + f[0, x_1, x_2, x]\Pi'(x)\\
            E''(x) &= 2f[0, x_1, x_2, x, x, x]\Pi(x) + 2f[0, x_1, x_2, x, x]\Pi'(x) + f[0, x_1, x_2, x]\Pi''(x)
        \end{align*}

        Por ser de tipo interpolatorio clásico, el error de la fórmula es:
        \begin{align*}
            R(f) &= L(E) = E''(0) = -2x_1^2f[0, x_1, x_2, 0, 0]+2f[0, x_1, x_2, 0]\left(-x_1+x_1\right)\\
            &= -2x_1^2f[0, x_1, x_2, 0, 0]=\\&= -2x_1^2\left(\dfrac{f^{(4)}(\xi_1)}{4!}\right)\qquad \text{para algún }\xi_1\in\left]\min\{-x_1,x_1\},\max\{-x_1,x_1\}\right[
        \end{align*}

    
        Por tanto, vemos que el error es proporcional a $x_1^2$. Por tanto, si $x_1$ es pequeño, el error será pequeño. Por tanto, es conveniente que $x_1$ sea pequeño.
        
        \item Aplica el resultado para la función $x e^{x^2 + 1}$.
        
        Definimos la función siguiente:
        \Func{f}{\bb{R}}{\bb{R}}{x}{x e^{x^2 + 1}}

        Por tanto, tenemos que:
        \begin{align*}
            f''(0)\approx -\dfrac{2}{x_1^2}f(0) + \dfrac{1}{x_1^2}f(x_1) + \dfrac{1}{x_1^2}f(-x_1)
            = \dfrac{x_1e^{x_1^2+1}-x_1e^{x_1^2+1}}{x_1^2}=0
        \end{align*}
        
    \end{enumerate}
\end{ejercicio}

\begin{ejercicio}\label{ej:2.2.6}
    Dada la fórmula de derivación numérica de tipo interpolatorio:
    \[
    f'(0) = \alpha_0 f(-1) + \alpha_1 f(1) + \alpha_2 f(2) + \alpha_3 f(a) + R(f), \quad a \neq -1, 1, 2.
    \]
    \begin{enumerate}
        \item Sin realizar ningún cálculo, ¿puedes indicar el máximo grado de exactitud que puede tener la fórmula? Justifica la respuesta.
        
        Como tiene $4$ nodos y se trata de la primera derivada, sabemos que no puede ser exacta en $\bb{P}_5$, por lo que el máximo grado de exactitud que puede tener la fórmula es $4$.
        
        \item Determina los valores de $\alpha_0$, $\alpha_1$, $\alpha_2$, $\alpha_3$ y $a$ para que la fórmula tenga el mayor grado de exactitud posible. ¿Cuál es ese grado de exactitud?
        
        Imponemos exactitud en $\bb{P}_4$, o equivalentemente, en $\cc{L}\{1, x, x^2,x^3,x^4\}$. Por tanto, buscamos $\alpha_0,\alpha_1,\alpha_2,\alpha_3, a\in\bb{R}$ tales que:
        \begin{align*}
            0 &= \alpha_0 + \alpha_1 + \alpha_2 + \alpha_3 \\
            1 &= -\alpha_0 + \alpha_1 + 2\alpha_2 + a\alpha_3 \\
            0 &= \alpha_0 + \alpha_1 + 4\alpha_2 + a^2\alpha_3 \\
            0 &= -\alpha_0 + \alpha_1 + 8\alpha_2 + a^3\alpha_3 \\
            0 &= \alpha_0 + \alpha_1 + 16\alpha_2 + a^4\alpha_3
        \end{align*}

        Planteamos el sistema de ecuaciones y aplicamos el método de Gauss:
        \begin{align*}
            &\left(\begin{array}{cccc|c}
                1 & 1 & 1 & 1 & 0 \\
                -1 & 1 & 2 & a & 1 \\
                1 & 1 & 4 & a^2 & 0 \\
                -1 & 1 & 8 & a^3 & 0 \\
                1 & 1 & 16 & a^4 & 0
            \end{array}\right) \xrightarrow[F_4'=F_4+F_1]{F_2'=F_2+F_1}
            \left(\begin{array}{cccc|c}
                1 & 1 & 1 & 1 & 0 \\
                0 & 2 & 3 & a+1 & 1 \\
                1 & 1 & 4 & a^2 & 0 \\
                0 & 2 & 9 & a^3+1 & 0 \\
                1 & 1 & 16 & a^4 & 0
            \end{array}\right)\xrightarrow[F_5'=F_5-F_1]{F_3'=F_3-F_1}
            \\&\left(\begin{array}{cccc|c}
                1 & 1 & 1 & 1 & 0 \\
                0 & 2 & 3 & a+1 & 1 \\
                0 & 0 & 3 & a^2-1 & 0 \\
                0 & 2 & 9 & a^3+1 & 0 \\
                0 & 0 & 15 & a^4-1 & 0
            \end{array}\right)\xrightarrow{F_4'=F_4-F_2}
            \left(\begin{array}{cccc|c}
                1 & 1 & 1 & 1 & 0 \\
                0 & 2 & 3 & a+1 & 1 \\
                0 & 0 & 3 & a^2-1 & 0 \\
                0 & 0 & 6 & a(a^2-1) & -1 \\
                0 & 0 & 15 & a^4-1 & 0
            \end{array}\right)\xrightarrow[F_5'=F_5-5F_3]{F_4'=F_4-2F_3}
            \\&\left(\begin{array}{cccc|c}
                1 & 1 & 1 & 1 & 0 \\
                0 & 2 & 3 & a+1 & 1 \\
                0 & 0 & 3 & a^2-1 & 0 \\
                0 & 0 & 0 & (a^2-1)(a-2) & -1 \\
                0 & 0 & 0 & a^4-5a^2+4 & 0
            \end{array}\right)
        \end{align*}

        Para que el sistema tenga solución, es necesario que:
        \begin{multline*}
            a^4-5a^2+4 = 0 \iff a^2=\dfrac{5\pm\sqrt{25-16}}{2} = \dfrac{5\pm 3}{2} \iff a^2 = 4 \ \vee\  a^2 = 1 \iff\\ \iff a\in \left\{-2,2,-1,1\right\}
        \end{multline*}

        Distinguimos valores:
        \begin{itemize}
            \item Si $a\notin \left\{1,-1,2, -2\right\}$, por la última ecuación $\alpha_3=0$, pero entonces la penúltima ecuación queda $0=-1$, por lo que no hay solución.
            \item Si $a\in \left\{1,-1,2\right\}$, la última penúltima ecuación queda $0=-1$, por lo que no hay solución.
            \item Si $a=-2$, el sistema queda:
            \begin{align*}
                \left(\begin{array}{cccc|c}
                    1 & 1 & 1 & 1 & 0 \\
                    0 & 2 & 3 & -1 & 1 \\
                    0 & 0 & 3 & 3 & 0 \\
                    0 & 0 & 0 & -12 & -1 \\
                    0 & 0 & 0 & 0 & 0
                \end{array}\right)\Longrightarrow
                \begin{cases}
                    \alpha_0 = \nicefrac{-2}{3} \\
                    \alpha_1 = \nicefrac{2}{3} \\
                    \alpha_2 = \nicefrac{-1}{12} \\
                    \alpha_3 = \nicefrac{1}{12}
                \end{cases}
            \end{align*}
        \end{itemize}

        Por tanto, para que la fórmula tenga el mayor grado de exactitud posible, $a=-2$ y el grado máximo de exactitud es $4$. La fórmula es:
        \begin{equation*}
            f'(0) = -\dfrac{2}{3}f(-1) + \dfrac{2}{3}f(1) - \dfrac{1}{12}f(2) + \dfrac{1}{12}f(-2) + R(f)
        \end{equation*}
        
        \item Determina la expresión del error indicando las condiciones sobre derivabilidad de la función $f$.
        
        Definimos en primer lugar:
        \begin{align*}
            \Pi(x) &= (x+1)(x-1)(x-2)(x+2)\\
            \Pi'(x) &= (x-1)(x-2)(x+2) + (x+1)(x-2)(x+2) + (x+1)(x-1)(x+2) +\\&\hspace{2cm}+ (x+1)(x-1)(x-2)
        \end{align*}

        Sabemos que el error del polinomio de interpolación en los $4$ nodos dados es:
        \begin{align*}
            \hspace{-1cm}E(x) &= f[-1, 1, 2, -2, x]\Pi(x)\\
            \hspace{-1cm}E'(x) &= f[-1, 1, 2, -2, x, x]\Pi(x) + f[-1, 1, 2, -2, x]\Pi'(x)
        \end{align*}

        Por ser de tipo interpolatorio clásico, el error de la fórmula es:
        \begin{align*}
            R(f) &= L(E) = E'(0) = f[-1, 1, 2, -2, 0, 0]\Pi(0) + f[-1, 1, 2, -2, 0]\Pi'(0)\\
            &= 4f[-1, 1, 2, -2, 0, 0] + f[-1, 1, 2, -2, 0]\cdot (4-4-2+2)
            =\\&= 4f[-1, 1, 2, -2, 0, 0] = \dfrac{f^{(5)}(\xi)}{30} \qquad \text{para algún }\xi\in\left]-2,2\right[
        \end{align*}
        
        \item Aplica el resultado para la función $x e^{x^2 + 1}$.
        
        Definimos la función siguiente:
        \Func{f}{\bb{R}}{\bb{R}}{x}{x e^{x^2 + 1}}

        Por tanto, tenemos que:
        \begin{align*}
            f'(0) \approx -\dfrac{2}{3}f(-1) + \dfrac{2}{3}f(1) - \dfrac{1}{12}f(2) + \dfrac{1}{12}f(-2)
        \end{align*}

        Tenemos que $f(-x)=-f(x)$ para todo $x\in\bb{R}$, por lo que:
        \begin{align*}
            f'(0) &\approx \dfrac{2}{3}f(1) + \dfrac{2}{3}f(1) - \dfrac{1}{12}f(2) - \dfrac{1}{12}f(-2)
            =\\&= \dfrac{4}{3}f(1)- \dfrac{1}{6}f(2)
            = \dfrac{4}{3}e^2 - \dfrac{1}{6}\cdot \left(2e^5\right)
            = \dfrac{4}{3}e^2 - \dfrac{1}{3}e^5 = \dfrac{e^2(4-e^3)}{3}\approx -39.619
        \end{align*}
        
    \end{enumerate}
\end{ejercicio}



\begin{ejercicio} Razonar por qué es posible calcular los coeficientes de una fórmula de derivación numérica de tipo interpolatorio clásico para aproximar $f'(a)$ mediante el procedimiento de ajuste de polinomios de Taylor, y que siga siendo válida la fórmula obtenida para calcular $R(f)$.
    \begin{observacion}
    Este ejercicio fue propuesto por la profesora en clase para obtener $0.5$ puntos adicionales. El desarrollo hecho por mí fue exhaustivo, y es adecuada su lectura para comprender este métdo para la obtención de los coeficientes $\alpha_i$.
    \end{observacion}


    Fijado $a\in \bb{R}$, consideramos el siguiente funcional lineal objetivo:
    \Func{F}{\bb{F}}{\bb{R}}{f}{f'(a)}

    Además, fijados $x_0,\dots,x_n\in \bb{R}$, consideramos el siguiente funcional lineal para cada $i=0,\dots,n$:
    \Func{F_i}{\bb{F}}{\bb{R}}{f}{f(x_i)}

    Consideramos la siguiente fórmula numérica para aproximar $f'(a)$:
    \begin{align*}
        L(f) = f'(a) &= \sum_{i=0}^n \alpha_i F_i(f) + R(f)\\
        &= \sum_{i=0}^n \alpha_i f(x_i) + R(f)
    \end{align*}

    Veamos una forma posible de calcular los coeficientes $\alpha_i$.
    Consideramos el desarrollo de Taylor de $f$ en torno a cada uno de los nodos alrededor de $a$, donde definimos $h_i=x_i-a$ para cada $i=0,\dots,n$:
    \begin{align*}
        f(x_i) &= f(a) + f'(a)h_i + \frac{f''(a)}{2}h_i^2 + \dots + \frac{f^{(n)}(a)}{n!}h_i^n +\dots + f^{(m)}(\xi_i)\frac{h_i^{m}}{(m)!}
    \end{align*}

    Multiplicamos por $\alpha_i$ en cada término de la ecuación anterior y sumamos:
    \begin{align*}
        \sum_{i=0}^n \alpha_i f(x_i) &= \sum_{i=0}^n \alpha_i f(a) + \sum_{i=0}^n \alpha_i f'(a)h_i + \sum_{i=0}^n \alpha_i \frac{f''(a)}{2}h_i^2 + \dots + \sum_{i=0}^n \alpha_i \frac{f^{(n)}(a)}{n!}h_i^n +\\&\hspace{2cm}+\dots+ \sum_{i=0}^n \alpha_i f^{(m)}(\xi_i)\frac{h_i^{m}}{(m)!}\\
        &= f(a) \sum_{i=0}^n \alpha_i + f'(a) \sum_{i=0}^n \alpha_i h_i + \frac{f''(a)}{2} \sum_{i=0}^n \alpha_i h_i^2 + \dots + \frac{f^{(n)}(a)}{n!} \sum_{i=0}^n \alpha_i h_i^n +\\&\hspace{2cm}+\dots+ \sum_{i=0}^n \alpha_i f^{(m)}(\xi_i)\frac{h_i^{m}}{(m)!}
    \end{align*}

    Por tanto, para poder llegar a la expresión buscada para $L(f)$, necesitamos que se cumpla:
    \begin{align*}
        \sum_{i=0}^n \alpha_i &= 0\qquad 
        \sum_{i=0}^n \alpha_i h_i = 1\qquad
        \sum_{i=0}^n \alpha_i h_i^2 = 0\qquad
        \dots\qquad
        \sum_{i=0}^n \alpha_i h_i^n = 0
    \end{align*}

    Es decir, necesitamos resolver el siguiente sistema de ecuaciones lineales:
    \begin{equation}\label{eq:1}
        \begin{pmatrix}
            1 & 1 & 1 & \dots & 1\\
            h_0 & h_1 & h_2 & \dots & h_n\\
            h_0^2 & h_1^2 & h_2^2 & \dots & h_n^2\\
            \vdots & \vdots & \vdots & \ddots & \vdots\\
            h_0^n & h_1^n & h_2^n & \dots & h_n^n
        \end{pmatrix}
        \begin{pmatrix}
            \alpha_0\\
            \alpha_1\\
            \alpha_2\\
            \vdots\\
            \alpha_n
        \end{pmatrix}
        =
        \begin{pmatrix}
            0\\
            1\\
            0\\
            \vdots\\
            0
        \end{pmatrix}
    \end{equation}

    Una vez obtenidos los coeficientes $\alpha_i$, tenemos que:
    \begin{align*}
        L(f) = f'(a) &= \sum_{i=0}^n \alpha_i f(x_i) + R(f)
    \end{align*}

    Llegados a este punto, buscamos obtener el valor de $R(f)$. Para ello, veamos que la fórmula en cuestión es de tipo interpolatorio; es decir, considerado el polinomio $p\in \bb{P}_n$ único interpolante de $f$ en los nodos $x_0,\dots,x_n$, veamos que:
    \begin{align*}
        L(p) = p'(a) &= \sum_{i=0}^n \alpha_i f(x_i) = \sum_{i=0}^n \alpha_i p(x_i)
    \end{align*}

    Para ello, buscamos obtener $p'(a)$ de una forma similar al desarrollo llevado a cabo hasta ahora. Consideramos el desarrollo de Taylor de $p$ en cada nodo alrededor de $a$, haciendo además uso de que, como $p\in \bb{P}_n$, se tiene que $p^{(n+1)}(x)=0$ para todo $x\in \bb{R}$:
    \begin{align*}
        p(x_i) &= p(a) + p'(a)h_i + \frac{p''(a)}{2}h_i^2 + \dots + \frac{p^{(n)}(a)}{n!}h_i^n + \cancelto{0}{p^{(n+1)}(\xi_i)}\frac{h_i^{n+1}}{(n+1)!}
    \end{align*}

    Repitiendo la idea anterior, multiplicamos por $\beta_i$ en cada término de la ecuación anterior y sumamos:
    \begin{align*}
        \sum_{i=0}^n \beta_i p(x_i) &= \sum_{i=0}^n \beta_i p(a) + \sum_{i=0}^n \beta_i p'(a)h_i + \sum_{i=0}^n \beta_i \frac{p''(a)}{2}h_i^2 + \dots + \sum_{i=0}^n \beta_i \frac{p^{(n)}(a)}{n!}h_i^n
    \end{align*}

    Imponemos que se cumpla:
    \begin{align*}
        \sum_{i=0}^n \beta_i &= 0\qquad 
        \sum_{i=0}^n \beta_i h_i = 1\qquad
        \sum_{i=0}^n \beta_i h_i^2 = 0\qquad
        \dots\qquad
        \sum_{i=0}^n \beta_i h_i^n = 0
    \end{align*}

    Es decir, obtenemos cada valor de $\beta_i$ resolviendo el siguiente sistema de ecuaciones lineales:
    \begin{equation}\label{eq:2}
        \begin{pmatrix}
            1 & 1 & 1 & \dots & 1\\
            h_0 & h_1 & h_2 & \dots & h_n\\
            h_0^2 & h_1^2 & h_2^2 & \dots & h_n^2\\
            \vdots & \vdots & \vdots & \ddots & \vdots\\
            h_0^n & h_1^n & h_2^n & \dots & h_n^n
        \end{pmatrix}
        \begin{pmatrix}
            \beta_0\\
            \beta_1\\
            \beta_2\\
            \vdots\\
            \beta_n
        \end{pmatrix}
        =
        \begin{pmatrix}
            0\\
            1\\
            0\\
            \vdots\\
            0
        \end{pmatrix}
    \end{equation}

    Una vez obtenidos los coeficientes $\beta_i$, tenemos que:
    \begin{align*}
        L(p) = p'(a) &= \sum_{i=0}^n \beta_i p(x_i)
    \end{align*}

    No obstante, tenemos que los sistemas de las Ecuaciones~\eqref{eq:1} y~\eqref{eq:2} son iguales, y la matriz de coeficientes es la matriz de Vandermonde de los nodos $h_0,\dots,h_n$, todos ellos distintos. Por tanto, el sistema es compatible y determinado, por lo que hay una única solución. Así, $\alpha_i=\beta_i$ para todo $i=0,\dots,n$. Por tanto:
    \begin{align*}
        L(p) = p'(a) &= \sum_{i=0}^n \alpha_i p(x_i) = \sum_{i=0}^n \alpha_i f(x_i)
    \end{align*}

    Por tanto, y usando la linealidad de $L$, tenemos que:
    \begin{align*}
        L(f)=L(p)+R(f)\Longrightarrow
        L(f-p)=R(f)
        \Longrightarrow R(f)=L(E)=E'(a)
    \end{align*}
    donde $E$ es el error de interpolación de $f$ en los nodos $x_0,\dots,x_n$, que sabemos que viene dado por:
    \begin{align*}
        E(x) = f(x) - p(x) = \frac{f^{(n+1)}(\xi)}{(n+1)!}\prod_{i=0}^n(x-x_i)
    \end{align*}
    para algún $\xi\in \left[\min\{x_0,\dots,x_n\},\max\{x_0,\dots,x_n\}\right]$.
    

\end{ejercicio}