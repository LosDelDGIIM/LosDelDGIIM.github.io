\chapter{Principio de acotación uniforme y Tª de la gráfica cerrada}
% Resulta que los 3 teoremas que se ven en este capítulo son equivalentes

\begin{definicion}
    Sean $E,F$ espacios normados, definimos:
    \begin{equation*}
        L(E,F) = \{f:E\to F : f \text{\ lineal y continua}\}
    \end{equation*}
    y notaremos normalmente $L(E) := L(E,E)$.
\end{definicion}

\begin{prop}
    Al igual que como sucedía con aplicaciones lineales y continuas de un espacio normado $E$ en $\mathbb{R}$, si $E,F$ son espacios normados y $T\in L(E,F)$ tenemos\footnote{Resultados análogos que se realizan con las mismas pruebas.}:
    \begin{enumerate}
        \item $T$ es continua $\Longleftrightarrow T$ es continua en $0 \Longleftrightarrow \sup\limits_{\|x\|\leq 1}\|Tx\| < \infty$
        \item Si definimos:
            \begin{equation*}
                \|T\| := \sup_{\|x\|\leq 1}\|Tx\| \qquad \forall T\in L(E,F)
            \end{equation*}
            Tenemos que $\|\cdot \|$ es una norma en $L(E,F)$.
        \item Se verifica que:
            \begin{equation*}
                \|T\| = \inf\{M>0 : \|Tx\| \leq M\|x\|, \quad \forall x\in E\}
            \end{equation*}
    \end{enumerate}
\end{prop}

\section{Principio de acotación uniforme}
\noindent
Con vistas a demostrar el Principio de acotación uniforme, demostramos el siguiente Lema:

\begin{lema}\label{lema:acotacion_uniforme}
    Sean $E,F$ espacios normados y $T\in L(E,F)$, entonces:
    \begin{equation*}
        \sup_{\|x-x_0\| \leq r} \|Tx\| \geq r\|T\|\qquad \forall x_0\in E, \quad \forall r>0
    \end{equation*}
    \begin{proof}
        Fijado $r\in \mathbb{R}^+$, sean $x_0,y\in E$ con $\|y\|\leq r$:
        \begin{align*}
            \|Ty\| &= \left\|T\left(\frac{1}{2}\left[x_0 + y - (x_0-y)\right]\right)\right\| = \dfrac{1}{2}\|T(x_0 + y - (x_0-y))\| \\ 
                   &\leq \dfrac{1}{2}\left(\|T(x_0+y)\| + \|T(x_0-y)\|\right) \leq \max\{\|T(x_0+y)\|,\|T(x_0-y\|)\} \\
                   &\leq \sup_{\|y\|\leq r} \max\{\|T(x_0+y)\|, \|T(x_0-y)\|\} \leq \sup_{\|z-x_0\| \leq r} \|Tz\|
        \end{align*}
        Si ahora observamos que:
        \begin{equation*}
            \sup_{\|y\|\leq r}\|Ty\| = r\sup_{\|z\|\leq 1}\|Tz\| = r\|T\|
        \end{equation*}
        Acabamos de probar que $\sup\limits_{\|x-x_0\|\leq r}\|Tx\| \geq r\|T\|$.
    \end{proof}
\end{lema}

\begin{teo}[Principio de acotación uniforme, de Banach-Steinhaus]\label{teo:principio_acotacion_uniforme}
Sea $E$ un espacio de Banach, $F$ un espacio normado y $\cc{F}$ un subconjunto de $L(E,F)$, entonces:
\begin{equation*}
    \sup_{T\in \cc{F}}\|Tx\| < +\infty \quad \forall x\in E \qquad \Longrightarrow \qquad  \sup_{T\in \cc{F}}\|T\| < +\infty
\end{equation*}
    \begin{proof}
        Demostraremos el contrarrecíproco:
        \begin{equation*}
            \sup_{T\in \cc{F}}\|T\| = \infty \qquad  \Longrightarrow \qquad  \sup_{T\in \cc{F}}\|Tx\| = \infty
        \end{equation*}
        Supongamos pues que $\sup\limits_{T\in \cc{F}}\|T\| = \infty$, por lo que existe una sucesión de elementos de $\cc{F}$, llamémosla $\{T_n\}$, de forma que:
        \begin{equation*}
            \|T_n\| \geq 4^n \qquad \forall n\in \mathbb{N}
        \end{equation*}
        Definimos por inducción una sucesión de puntos de $E$:
        \begin{itemize}
            \item $x_0 = 0\in E$.
            \item Tomando $r = \nicefrac{1}{3}$, el Lema~\ref{lema:acotacion_uniforme} nos dice:
                \begin{equation*}
                    \sup_{\|x-x_0\| < \frac{1}{3}}\|T_1x\| \geq \dfrac{1}{3}\|T_1\| > \dfrac{2}{3}\cdot \dfrac{1}{3}\|T_1\|
                \end{equation*}
                Como $\nicefrac{2}{3}\cdot \nicefrac{1}{3}\cdot \|T_1\|$ es estrictamente menor que el supremo de la izquierda, tenemos que no puede ser una cota superior de $\|T_1 x\|$ para $x\in B(x_0,\nicefrac{1}{3})$, con lo que $\exists x_1\in B(x_0,\nicefrac{1}{3})$ de forma que:
                \begin{equation*}
                    \|T_1x_1\| > \dfrac{2}{3}\cdot \dfrac{1}{3}\|T_1\|
                \end{equation*}
            \item Supuesto que hemos construido hasta $x_{n-1}$, veamos cómo construir $x_n$:

                Tomando $r = \nicefrac{1}{3^n}$, el Lema~\ref{lema:acotacion_uniforme} nos dice que:
                \begin{equation*}
                    \sup_{\|x-x_{n-1}\| < \frac{1}{3^n}}\|T_nx\| \geq \dfrac{1}{3^n}\|T_n\| > \dfrac{2}{3}\cdot \dfrac{1}{3^n} \|T_n\|
                \end{equation*}
                Y por el mismo razonamiento de antes podemos encontrar $x_n\in B(x_{n-1},\nicefrac{1}{3^n})$ de forma que:
                \begin{equation*}
                    \|T_nx_n\| > \dfrac{2}{3}\cdot \dfrac{1}{3^n}\|T_n\|
                \end{equation*}
        \end{itemize}
        Veamos ahora que $\{x_n\}$ es de Cauchy. Para ello, buscamos acotar $\|x_m - x_n\|$ para $n,m$ índices bastante avanzados. Supondremos sin pérdida de generalidad que $n,m\in \mathbb{N}$ con $m>n$, donde tendremos:
        \begin{align*}
            \|x_m-x_n\| &= \|x_m - x_{m-1} + x_{m-1} - x_{m-2} + \ldots + x_{n+1} - x_n\| \\
                        &\leq \|x_m - x_{m-1} \| + \| x_{m-1} - x_{m-2} \| + \ldots + \|x_{n+1} - x_n\| \\
                        &\leq \dfrac{1}{3^m} + \dfrac{1}{3^{m-1}} + \ldots + \dfrac{1}{3^{n+1}} = \dfrac{1}{3^n}\left[\dfrac{1}{3^{m-n}} + \ldots + \dfrac{1}{3}\right] \\
                        &\leq \dfrac{1}{3^n} \sum_{j=1}^{+\infty} \dfrac{1}{3^j} = \dfrac{1}{3^n} \dfrac{\frac{1}{3}}{1-\frac{1}{3}} = \dfrac{1}{3^n}\cdot \dfrac{1}{2}
        \end{align*}
        En definitiva, tenemos que:
        \begin{equation*}
            \|x_m-x_n\| = \|x_m - x_{m-1} + x_{m-1} - x_{m-2} + \ldots + x_{n+1} - x_n\| \leq \dfrac{1}{2}\cdot \dfrac{1}{3^n}
        \end{equation*}
        Por lo que $\{x_n\}$ es de Cauchy en $E$, que era de Banach, por lo que $\{x_n\}$ converge a cierto $x\in E$. Observemos que:
        \begin{equation*}
            \dfrac{1}{2}\cdot \dfrac{1}{3^n} \geq \lim_{m\to\infty}\|x_m-x_n\| \AstIg \left\|\lim_{m\to\infty}(x_m-x_n)\right\| = \|x-x_n\|
        \end{equation*}
        donde en $(\ast)$ hemos usado la continuidad de $\|\cdot \|$. Calculamos ahora:
        \begin{align*}
            \|T_nx\| &= \|T_n(x-x_n + x_n)\| \geq \|T_nx_n\| - \|T_n(x-x_n)\| \geq \dfrac{2}{3}\cdot \dfrac{1}{3^n}\|T_n\| - \|T_n\|\|x-x_n\| \\
                     &\geq \dfrac{2}{3}\cdot \dfrac{1}{3^n}\|T_n\| - \|T_n\|\dfrac{1}{2}\cdot \dfrac{1}{3^n} = \left(\dfrac{2}{3}-\dfrac{1}{2}\right) \dfrac{1}{3^n} \|T_n\| = \dfrac{1}{6}\cdot \dfrac{1}{3^n}\|T_n\| \geq \dfrac{1}{6}{\left(\dfrac{4}{3}\right)}^{n} \to \infty
        \end{align*}
        Por tanto:
        \begin{equation*}
            \sup_{T\in \cc{F}}\|Tx\| \geq \sup_{n\in \mathbb{N}}\|T_nx\| = \infty
        \end{equation*}
        Como queríamos probar.
    \end{proof}
\end{teo}

\noindent
Introducimos ahora una serie de Corolarios que nos da el Principio de acotación uniforme:

\begin{coro}
    Sean $E,F$ dos espacios de Banach, sea $\{T_n\}$ una sucesión de elementos de $L(E,F)$ de forma que $\{T_n(x)\} \to T(x)$ para todo $x\in E$. Entonces:
    \begin{enumerate}[label=(\alph*)]
        \item $\sup\limits_{n\in \mathbb{N}}\|T_n\| < \infty$.
        \item $T\in L(E,F)$.
        \item $\|T\| \leq \liminf\limits_{n\to\infty}\|T_n\|$.
    \end{enumerate}
    \begin{proof}
        Demostramos cada apartado:
        \begin{enumerate}[label=(\alph*)]
            \item Dado $x\in E$, como $\{T_n(x)\}\to T(x)$, tenemos que $\exists m\in \mathbb{N}$ de forma que:
                \begin{equation*}
                    T_n(x) \in B(T(x), 1) \qquad \forall n\geq m
                \end{equation*}
                Por lo que $\{T_n(x) : n\in \mathbb{N}\}$ es un conjunto acotado, puesto que podemos verlo como la unión de un conjunto acotado y uno finito:
                \begin{equation*}
                    \{T_n(x) : n\in \mathbb{N}\} = \{T_n(x) : n\geq m\} \cup \{T_n(x) : n < m\} 
                \end{equation*}
                En definitiva, tenemos que $\sup\limits_{n\in \mathbb{N}}\|T_n(x)\| < \infty$ para todo $x\in E$, de donde aplicando el Principio de acotación uniforme tenemos que $\sup\limits_{n\in \mathbb{N}}\|T_n\| < \infty$.
            \item Veamos que $T:E\to F$ es lineal y continua:
                \begin{itemize}
                    \item Es fácil ver que $T$ es lineal:
                        \begin{align*}
                            T(\lm x + y) &= \lim_{n\to\infty}T_n(\lm x + y) = \lim_{n\to\infty}(\lm T_n(x) + T_n(y)) \\ 
                                         &= \lm \lim_{n\to\infty}T_n(x) + \lim_{n\to\infty}T_n(y) = \lm T(x) + T(y) \qquad \forall x,y\in E
                        \end{align*}
                    \item Para ver que $T$ es continua podemos usar el apartado $(a)$:
                        \begin{equation*}
                            \|T_n(x)\| \leq \|T_n\|\|x\| \leq \sup_{n\in \mathbb{N}}\|T_n\|\|x\| \qquad \forall x\in E
                        \end{equation*}
                        Y como $\{\|T_n(x)\|\}\to \|T(x)\|$, tenemos que:
                        \begin{equation*}
                            \|T(x)\| \leq \sup_{n\in \mathbb{N}}\|T_n\|\|x\| \qquad \forall x\in E
                        \end{equation*}
                        lo que nos dice que $T$ es continua.
                \end{itemize}
            \item Para ver que $\|T\| \leq \liminf\limits_{n\to \infty} \|T_n\|$, notemos que:
                \begin{align*}
                    \|T(x)\| &= \left\|\lim_{n\to\infty}T_n(x)\right\| = \lim_{n\to\infty}\left\|T_n(x)\right\| = \liminf \|T_n(x)\| \leq \liminf(\|T_n\|\|x\|) \\
                             &= \sup_{k\in \mathbb{N}}\inf_{n\geq k}(\|T_n\|\|x\|) = \sup_{k\in \mathbb{N}}\inf_{n\geq k}\|T_n\| \cdot \|x\| = \liminf \|T_n\| \cdot \|x\| \qquad \forall x\in E
                \end{align*}
                De donde $\|T\|\leq \liminf \|T_n\|$.
        \end{enumerate}
    \end{proof}
\end{coro}

\begin{coro}\label{coro:entonces_B_acotado}
    Sea $G$ un espacio de Banach y $B\subset G$, si para toda $f\in G^\ast$ el conjunto $f(B)$ está acotado (en $\mathbb{R}$), entonces $B$ está acotado.
    \begin{proof}
        Comenzamos la demostración pensando a lo que queremos llegar, pues así nos será más fácil comenzar la demostración. Queremos probar que $B$ está acotado, es decir, que existe $M>0$ de forma que:
        \begin{equation*}
            \|b\| \leq M \qquad \forall b\in B
        \end{equation*}
        Si recordamos que el Corolario~\ref{coro:calcular_norma_x} nos dice que:
        \begin{equation*}
            \|b\| = \sup_{\|f\|\leq 1}|f(b)|
        \end{equation*}
        observemos que lo queremos es buscar una cota superior de $|f(b)|$, donde $b$ está fija y $f$ se mueve. Para ello, podemos pensar en definir ciertos funcionales $T_b(f)$ de forma que tras aplicar el Principio de Acotación Uniforme obtengamos para $\|f\|\leq 1$:
        \begin{equation*}
            |f(b)| = |T_b(f)| \leq \|T_b\|\|f\| \leq \|T_b\| \leq \sup_{b\in B}\|T_b\| < \infty
        \end{equation*}
        Con lo que nuestra constante $M$ la tomaremos como $\sup\limits_{b\in B}\|T_b\|$. Comenzando ahora con la demostración, fijado $b\in B$, definimos la aplicación $T_b:G^\ast\to \mathbb{R}$ dada por:
        \begin{equation*}
            T_b(f) = f(b) \qquad \forall f\in G^\ast
        \end{equation*}
        Con lo que $T_b\in L(G^\ast,\mathbb{R})$:
        \begin{itemize}
            \item Es claro que $T_b$ es lineal.
            \item $T_b$ es continua, ya que $|T_b(f)| = |f(b)| \leq \|b\|\|f\| \quad \forall f\in G^\ast$.
        \end{itemize}
        Como $f(B)$ es acotado para toda $f\in G^\ast$, tenemos entonces que:
        \begin{equation*}
            \sup_{b\in B}|T_b(f)| = \sup_{b\in B}|f(b)| < \infty \qquad \forall f\in G^\ast
        \end{equation*}
        Con lo que aplicando el Principio de acotación uniforme tenemos: 
        \begin{equation*}
            \sup\limits_{b\in B}\|T_b\| < \infty
        \end{equation*}
        Ahora, si tomamos $f\in G^\ast$ con $\|f\|\leq 1$, buscamos usar que $\|x\| = \sup\limits_{\|f\|\leq 1}|f(x)|$:
        \begin{equation*}
            |f(b)| = |T_b(f)| \leq \|T_b\|\|f\| \leq \sup_{b\in B}\|T_b\|\|f\| \leq \sup_{b\in B}\|T_b\|
        \end{equation*}
        Por lo que $\|b\| \leq \sup\limits_{b\in B}\|T_b\| < \infty \quad \forall b\in B$, lo que nos dice que $B$ está acotado.
    \end{proof}
\end{coro}

\noindent
Este último corolario nos dice que si $B\subset G^\ast$ es un conjunto cualquiera, una forma de estudiar si $B$ es un conjunto acotado, una posibilidad es tratar de calcular su imagen bajo cualquier función $f\in G^\ast$, que es un subconjunto de $\mathbb{R}$.

Recordemos que en $\mathbb{R}^n$ un conjunto era acotado si y solo si cada una de sus proyecciones es un conjunto acotado. Este Corolario hace ese papel en espacios de dimensión infinita, que junto con el siguiente son muy utilizados.
% // TODO: Estudiar la relacion con dicho resultado, cuando G sea un Banach de dimensión finita.

\begin{coro}\label{coro:entonces_Bast_acotado}
    Sea $G$ un espacio de Banach y sea $B^\ast\subset G^\ast$, si el conjunto:
    \begin{equation*}
        B^\ast(x) = \{f(x) : f\in B^\ast\}
    \end{equation*}
    está acotado para todo $x\in G$, entonces $B^\ast$ está acotado.
    \begin{proof}
        Al igual que antes empezamos por el final, pues así nos será más fácil comenzar la demostración. Queremos concluir que $B^\ast$ está acotado, es decir, que:
        \begin{equation*}
            \|f\| \leq M\qquad \forall f\in B^\ast
        \end{equation*}
        para cierta constante $M>0$. Para ello, si recordamos la definición de $\|f\|$, vemos que:
        \begin{equation*}
            \|f\| = \sup_{\|x\|\leq 1}|f(x)|
        \end{equation*}
        donde $f$ está fija y movemos la $x$, con lo que trataremos de definir funcionales $T_f(x)$ de forma que para $\|x\|\leq 1$:
        \begin{equation*}
            |f(x)| = |T_f(x)| \leq \|T_f\|\|x\| \leq \|T_f\| \leq \sup_{f\in B^\ast}\|T_f\|
        \end{equation*}
        Comenzando ahora con la demostración, para cada $f\in B^\ast$ definimos la aplicación $T_f:G\to \mathbb{R}$ dada por:
        \begin{equation*}
            T_f(x) = f(x) \qquad \forall x\in G
        \end{equation*}
        con lo que $T_f\in G^\ast$ para todo $f\in B^\ast$:
        \begin{itemize}
            \item Es fácil ver que $T_f$ es lineal para cualquier $f\in B^\ast$.
            \item No es difícil ver que $T_f$ es continua para $f\in B^\ast$, ya que:
                \begin{equation*}
                    |T_f(x)| = |f(x)| \leq \|f\|\|x\| \qquad \forall x\in G
                \end{equation*}
        \end{itemize}
        Como el conjunto $B^\ast(x)$ está acotado para todo $x\in G$, tenemos que:
        \begin{equation*}
            \sup_{f\in B^\ast}\|T_f(x)\| = \sup_{f\in B^\ast}|f(x)| < \infty
        \end{equation*}
        nos encontramos en las hipótesis del Principio de acotación uniforme, que nos dice que entonces:
        \begin{equation*}
            \sup_{f\in B^\ast}\|T_f\| < \infty
        \end{equation*}

        en cuyo caso, si $\|x\|\leq 1$, entonces:
        \begin{equation*}
            |f(x)| = |T_f(x)| \leq \|T_f\|\|x\|\leq \|T_f\| \leq \sup_{f\in B^\ast}\|T_f\| \qquad \forall f\in B^\ast
        \end{equation*}

        con lo que:
        \begin{equation*}
            \|f\| = \sup_{\|x\|\leq 1}|f(x)| \leq \sup_{f\in B^\ast}\|T_f\| \qquad \forall f\in B^\ast
        \end{equation*}

        de donde deducimos que $B^\ast$ está acotado.
    \end{proof}
\end{coro}

\section{Otra demostración del Principio de acotación uniforme}
\noindent
Repetiremos ahora la demostración del Principio de acotación uniforme de otra forma, usando el Lema de Baire, un resultado clásico que nos da de forma no muy complicada la demostración del Principio.

\begin{lema}[de Baire]\label{lema:Baire}
    Sea $X$ un espacio métrico completo, sea $\{X_n\}$ una sucesión de subconjuntos de $X$ todos ellos cerrados y con interior vacío, entonces:
    \begin{equation*}
        \Int\left(\bigcup_{n\in \mathbb{N}} X_n\right) = \emptyset 
    \end{equation*}
    \begin{proof}
        Tomaremos $O_n = X\setminus X_n$ para cada $n\in \mathbb{N}$, con lo que $O_n$ es abierto y denso para cada $n\in \mathbb{N}$, ya que:
        \begin{equation*}
            \overline{O_n} = \overline{X\setminus X_n} = X\setminus \Int X_n = X\setminus \emptyset  = X \qquad \forall n\in \mathbb{N}
        \end{equation*}
        Y la prueba terminará probando que $G = \bigcap\limits_{n\in \mathbb{N}}O_n$ es denso, ya que en dicho caso tendremos:
        \begin{equation*}
            X = \overline{G} = \overline{\bigcap\limits_{n\in \mathbb{N}}O_n} = \overline{\bigcap_{n\in \mathbb{N}} X\setminus X_n} = \overline{X\setminus\bigcup_{n\in \mathbb{N}}X_n} = X\setminus \Int\left(\bigcup_{n\in \mathbb{N}}X_n\right)
        \end{equation*}
        de donde podremos deducir que $\Int\left(\bigcup\limits_{n\in \mathbb{N}}\right)X_n = \emptyset $. Para probar que $G$ es denso, sea $\omega$ un abierto no vacío de $X$, tenemos que probar que $\omega\cap G \neq \emptyset $. Como $\omega$ es abierto, dado $x_0\in \omega$ podemos encontrar $r_0>0$ de forma que:
        \begin{equation*}
            \overline{B(x_0,r_0)}\subset \omega
        \end{equation*}
        Tras esto, como $O_1$ es abierto y denso, podremos elegir $x_1 \in B(x_0,r_0)\cap O_1$ y $r_1>0$ de forma que:
        \begin{equation*}
            \overline{B(x_1,r_1)}\subset B(x_0,r_0) \cap O_1 \quad \text{y}\quad 0<r_1<\frac{r_0}{2}
        \end{equation*}
        De forma inductiva, como cada $O_n$ es abierto y denso, seremos capaces de encontrar dos sucesiones: $\{x_n\}$ de puntos de $X$ y $\{r_n\}$ de reales positivos de forma que se cumpla:
        \begin{equation*}
            \overline{B(x_{n+1},r_{n+1})}\subset B(x_n,r_n) \cap O_{n+1} \quad \text{y}\quad 0<r_{n+1}<\frac{r_n}{2} \qquad \forall n\in \mathbb{N}\cup \{0\}
        \end{equation*}
        Veamos que $\{x_n\}$ es de Cauchy. Para ello, sean $n,m\in \mathbb{N}$ con $m\geq n$, tendremos que:
        \begin{gather*}
            x_m \in B(x_{m-1},r_{m-1}) \subset \ldots \subset B(x_n,r_n) \\
            r_n < \dfrac{r_{n-1}}{2} < \dfrac{r_{n-2}}{2^2} < \ldots < \dfrac{r_0}{2^n}
        \end{gather*}
        Por lo que:
        \begin{equation*}
            \|x_m - x_n\| < r_n < \dfrac{r_0}{2^n}
        \end{equation*}
        de donde deducimos que $\{x_n\}$ es de Cauchy en $X$, y como $X$ era completo, existe $l\in X$ de forma que $\{x_n\}\to l$. Finalmente, como $x_{n+p}\in B(x_n,r_n)$ para $n,p\in \mathbb{N}\cup\{0\}$, tomando límite cuando $p\to \infty$ obtenemos que:
        \begin{equation*}
            l \in \overline{B(x_n,r_n)} \qquad \forall n\in \mathbb{N}\cup \{0\}
        \end{equation*}
        En particular, $l\in \omega \cap G$, por lo que $G$ es denso, lo que concluye la demostración.
    \end{proof}
\end{lema}

\noindent
Cabe destacar que una de las formas en las que más se utiliza el Lema de Baire es mediante su contrarrecíproco:
\begin{center}
    Sea $X$ un espacio métrico completo, sea $\{X_n\}$ una sucesión de subconjuntos de $X$ todos ellos cerrados, entonces:
    \begin{equation*}
        \Int\left(\bigcup_{n\in \mathbb{N}}X_n\right) \neq \emptyset \quad  \Longrightarrow \quad  \exists n_0\in \mathbb{N} : \Int(X_{n_0}) \neq \emptyset 
    \end{equation*}
\end{center}

\noindent
Ahora, volveremos a demostrar el Principio de acotación uniforme usando el Lema de Baire.

\begin{teo}[Principio de acotación uniforme] % // TODO: Ponerle nombre
    Sea $E$ un espacio de Banach, $F$ un espacio normado y $\cc{F}$ un subconjunto de $L(E,F)$, entonces:
    \begin{equation*}
        \sup_{T\in \cc{F}}\|Tx\| < +\infty \quad \forall x\in E \qquad \Longrightarrow \qquad  \sup_{T\in \cc{F}}\|T\| < +\infty
    \end{equation*}
    \begin{proof}
        Suponiendo que indexamos nuestra familia mediante un conjunto $I$: $\cc{F} = \{T_i\}_{i \in I}$, definimos para cada $n\in \mathbb{N}$:
        \begin{equation*}
            X_n = \{x\in E : \|T_i x\| \leq n, \quad \forall i \in I\}
        \end{equation*}
        que verifica:
        \begin{itemize}
            \item $X_n$ es cerrado para cada $n\in \mathbb{N}$, ya que si tomamos $\{x_m\}$ una sucesión de puntos de $X_n$ convergente a $x\in E$, tenemos entonces que $\|T_ix_m\| \leq n$ para cada $m\in \mathbb{N}$. Usando que $\|\cdot \|$ y que $T_i$ son las dos funciones continuas obtenemos que:
                \begin{equation*}
                    \|T_ix\| \leq n
                \end{equation*}
                con lo que $x\in X_n$.
            \item Usando que $\sup\limits_{T\in \cc{F}}\|Tx\| < \infty$, sabemos entonces que existe $M\in \mathbb{N}$ de forma que $\|Tx\| \leq M$ para todo $x\in E$ y $T\in \cc{F}$, con lo que $X_M = E$, luego:
                \begin{equation*}
                    \bigcup_{n\in \mathbb{N}}X_n = E
                \end{equation*}
        \end{itemize}
        Como $E$ es abierto y es un espacio vectorial, tenemos que $\Int E = E \neq \emptyset $. Por el Lema de Baire tenemos que existe $n_0\in \mathbb{N}$ de forma que $\Int(X_{n_0})\neq \emptyset $, lo que nos permite tomar $x_0\in E$ y $r>0$ de forma que $B(x_0,r)\subset X_{n_0}$, lo que nos dice por la definición de $X_{n_0}$ que:
        \begin{equation*}
            \|T_i(x_0 + rz)\| \leq n_0 \qquad \forall i \in I, \quad \forall z\in B(0,1)
        \end{equation*}

        como:
        \begin{multline*}
            n_0 \geq \|T_i(x_0 + rz)\| \geq \|T_i(rz)\| - \|T_i(x_0)\| \quad \Longrightarrow \quad  r\|T_i(z)\| \leq n_0 + \|T_i(x_0)\| \\
            \qquad \forall i \in I, \quad \forall z\in B(0,1)
        \end{multline*}
        
        tendremos:
        \begin{equation*}
            r\|T_i\| \leq n_0 + \|T_i(x_0)\| \leq n_0 + \sup_{T\in \cc{F}}\|T(x_0)\| < \infty \qquad \forall i \in I, \quad \forall z\in B(0,1)
        \end{equation*}
        
        de donde concluimos que $\sup\limits_{T\in \cc{F}}\|T\| < \infty$
    \end{proof}
\end{teo}


\section{Teorema de la aplicación abierta}
% Demostración distinta al del Brezis

\begin{ejercicio}\label{ej:norman}
    Sean $X,Y$ dos espacios de Banach, $T\in L(X,Y)$, definimos para cada $n\in \mathbb{N}$ y para cada $y \in Y$:
    \begin{equation*}
        \|y\|_n := \inf\{\|u\| + n\|v\| : u\in X, v\in Y \text{\ con\ } y = T(u)+v\}
    \end{equation*}
    Probar que $\|\cdot \|_n$ es una norma en $Y$ para todo $n\in \mathbb{N}$, que verifica:
    \begin{equation*}
        \|y\|_n \leq n\|y\| \qquad \forall y\in Y
    \end{equation*}
    Además, si $y=T(x)$ con $x\in X$, entonces:
    \begin{equation*}
        \|y\|_n \leq \|x\|
    \end{equation*}~\\

    \noindent
    Veamos en primer lugar que $\|\cdot \|_n$ es una norma en $Y$ para cada $n\in \mathbb{N}$. Para ello, fijaremos $n\in \mathbb{N}$ y veremos las propiedades de una norma:
    \begin{itemize}
        \item Para la no degeneración, supongamos que $y\in Y$ con $\|y\|_n = 0$. Por definición del ínfimo, existen sucesiones $\{u_m\}$ de puntos de $X$ y $\{v_m\}$ de puntos de $Y$ de forma que:
            \begin{equation*}
                \{\|u_m\| + n\|v_m\|\} \to 0 \qquad y = T(u_m)+ v_m \qquad \forall m\in \mathbb{N}
            \end{equation*}
            como $\|u_m\|, \|v_m\|\geq 0$ para todo $m\in \mathbb{N}$, tenemos entonces que:
            \begin{equation*}
                \{\|u_m\|\},\{\|v_m\|\}\to 0 \quad \Longrightarrow \quad  \{u_m\},\{v_m\} \to 0
            \end{equation*}
            usando ahora que $y = T(u_m)+v_m$ para todo $m\in \mathbb{N}$, observemos que:
            \begin{equation*}
                \{T(u_m)+v_m\}\to 0
            \end{equation*}
            donde hemos usado que $T$ y la suma son continuas, con lo que $y=0$.
        \item Para la homogeneidad por homotecias, sean $\lm\in \mathbb{R}$ y $y\in Y$:
            \begin{align*}
                |\lm|\|y\|_n &= \inf\{|\lm|(\|u\|+n\|v\|) : u\in X, v\in Y \text{\ con\ } y=T(u)+v\}  \\
                             &= \inf\{\|\lm u\| + n\|\lm v\| : u\in X, v\in Y \text{\ con\ } y=T(u)+v\}  \\
                             &= \inf\left\{\|u\| + n\|v\| : u\in X, v\in Y \text{\ con\ } y=T\left(\frac{u}{\lm}\right)+\frac{v}{\lm}\right\}   \\
                             &= \inf\{\|u\| + n\|v\| : u\in X, v\in Y \text{\ con\ } \lm y=T(u)+v\}  = \|\lm y\|_n
            \end{align*}
        \item Finalmente, para la desigualdad triangular, sean $y_1,y_2\in Y$, para todo $\varepsilon>0$ tenemos por la caracterización del ínfimo que existen $u_1,u_2\in X$, $v_1,v_2\in Y$ de forma que:
            \begin{equation*}
                y_i = T(u_i) + v_i \quad \text{y}\quad \|u_i\| + n\|v_i\| \leq \|y_i\|+\frac{\varepsilon}{2}\qquad \forall i \in \{1,2\}
            \end{equation*}
            de donde $y_1+y_2 = T(u_1)+v_1 + T(u_2)+v_2 = T(u_1+u_2)+v_1+v_2$, por lo que:
            \begin{align*}
                \|y_1 + y_2\| \leq \|u_1+u_2\| + n\|v_1+v_2\| &\leq \|u_1\|+n\|v_1\| + \|u_2\|+n\|v_2\| \\ &\leq \|y_1\|+\|y_2\| + \varepsilon \qquad \forall \varepsilon>0
            \end{align*}
            En definitiva, hemos probado que $\|y_1+y_2\| \leq \|y_1\| + \|y_2\|$.
    \end{itemize}
    Fijado $n\in \mathbb{N}$, sea ahora $y\in Y$, tomando $u=0\in X$ y $v=y\in Y$, tenemos que:
    \begin{equation*}
        y = 0 + y = T(u) + v
    \end{equation*}

    por lo que:
    \begin{equation*}
        \|y\|_n \leq \|u\|+n\|v\| = n\|y\|
    \end{equation*}
    Si tenemos ahora que $y=T(x)$ para $x\in X$, podemos tomar $u=x\in X$ y $v=0\in Y$ con lo que:
    \begin{equation*}
        y = y + 0 = T(x) + v
    \end{equation*}

    por lo que:
    \begin{equation*}
        \|y\|_n \leq \|u\| + n\|v\| = \|x\|
    \end{equation*}
\end{ejercicio}

\begin{ejercicio}
    Si $T:X\to Y$ es una aplicación lineal entre dos espacios normados:
    \begin{equation*}
        T \text{\ es abierta} \quad \Longleftrightarrow \quad  \exists \delta>0 : B(0,\delta)\subset T(B(0,1))
    \end{equation*}
    % \begin{proof}
    %     Lo vemos por doble implicación:
    %     \begin{description}
    %         \item [$\Longrightarrow )$] Como $B(0,1)$ es un conjunto abierto de $X$, entonces $T(B(0,1))$ es un conjunto abierto de $Y$ por ser $T$ abierta. En particular, para $0$ existirá $\delta>0$ de forma que $B(0,\delta)\subseteq T(B(0,1))$.
    %         \item [$\Longleftarrow )$] Para ver que $T$ es abierta basta probar que $T(B(x,r))$ es un conjunto abierto, para todo $x\in X$ y para todo $r\in \mathbb{R}^+$. Sea pues $y \in T(B(x,r))$ y usando que existe $\delta>0$ de forma que $B(0,\delta)\subset T(B(0,1))$:
    %             \begin{equation*}
    %                 B(0,\delta) \subset T(B(0,1)) \Longrightarrow B(0,r\delta) = r\cdot B(0,\delta) \subset r\cdot T(B(0,1)) = T(B(0,r))
    %             \end{equation*}
    %             Ahora:
    %             \begin{equation*}
    %             \end{equation*}
    %     \end{description}
    % \end{proof}
\end{ejercicio}

\begin{teo}[de la aplicación abierta] \ \\
    Sean $X,Y$ espacios de Banach, sea $T\in L(X,Y)$ sobreyectiva, entonces $T$ es una aplicación abierta.
    \begin{proof}
        La demostración se completa en dos pasos:
        \begin{description}
            \item [Paso 1.] Veamos que $\exists r> 0$ de forma que $B(0,r)\subset \overline{T(B(0,1))}$:

                Consideramos en $Y$ la norma:
                \begin{equation*}
                    \|y\|_n = \inf\{\|u\| + n \|v\| : y = T(u)+v, u\in X, v\in Y\} \qquad \forall y\in Y, \quad \forall n\in \mathbb{N}
                \end{equation*}
                Que abreviaremos por:
                \begin{equation*}
                    \|y\|_n = \inf_{y=T(u)+v} \{\|u\| + n\|v\|\}
                \end{equation*}
                Cuyas propiedades fueron vistas en el Ejercicio~\ref{ej:norman}. Consideramos ahora $Z$ como el espacio de todas aquellas sucesiones casi nulas\footnote{Es decir, con un número finito de términos no nulos.} $\{z_n\}$ de puntos de $Y$. Consideraremos en dicho espacio:
                \begin{equation*}
                    \|\{z_n\}\|_\infty = \max_{n\in \mathbb{N}} \|z_n\|_n \qquad \forall \{z_n\}\in Z
                \end{equation*} % // TODO: Comprobar que es una norma
                Para cada $n\in \mathbb{N}$, definimos la aplicación 
                \Func{T_n}{Y}{Z}{y}{T_n(y)}
                dada por:
                \begin{equation*}
                    T_n(y) = \{\delta_{n,k}\cdot y\}_{k\in \mathbb{N}}
                \end{equation*}
                donde usamos la $\delta$ de Dirichlet:
                \begin{equation*}
                    \delta_{n,k} = \left\{\begin{array}{ll}
                        1 & \text{si\ } k=n \\
                        0 & \text{si\ } k\neq n
                    \end{array}\right. 
                \end{equation*}
                Veamos que $T_n\in L(Y,Z)$ para cada $n\in \mathbb{N}$. Para ello, dado $n\in \mathbb{N}$; vemos que:
                \begin{itemize}
                    \item Es lineal, ya que si $y_1,y_2\in Y$:
                        \begin{align*}
                            T_n(y_1+y_2) &= \{\delta_{n,k}\cdot (y_1+y_2)\}_{k\in \mathbb{N}} = \{\delta_{n,k}\cdot y_1 + \delta_{n,k}\cdot y_2\}_{k\in \mathbb{N}} \\ &= \{\delta_{n,k}\cdot y_1\}_{k\in \mathbb{N}} + \{\delta_{n,k}\cdot y_2\}_{k\in \mathbb{N}} = T_n(y_1) + T_n(y_2)
                        \end{align*} % // TODO: Hacer que saca escalares fuera
                    \item Para ver que $T_n:(Y,\|\cdot \|)\to (Z,\|\cdot \|_\infty)$ es continua:
                        \begin{align*}
                            \|T_n(y)\|_\infty &= \|\{\delta_{n,k}\cdot y\}_{k\in \mathbb{N}}\|_\infty = \max_{n\in \mathbb{N}}\|\delta_{n,k}\cdot y\|_n \\ &= \max_{k\in \mathbb{N}} (\delta_{n,k}\cdot \|y\|_n) = \|y\|_n \stackrel{(\ast)}{\leq} n\|y\| \qquad  \forall y\in Y
                        \end{align*}
                        donde en $(\ast)$ hemos usado el Ejercicio~\ref{ej:norman}.
                \end{itemize}
                Veamos ahora que para cada $y\in Y$, la sucesión $\{T_n(y)\}$ está acotada. Para ello, sea $y\in Y$, como $T$ es sobreyectiva $\exists x\in X$ de forma que $T(x) = y$, con lo que:
                \begin{equation*}
                    \|T_n(y)\|_\infty = \|y\|_n  \stackrel{(\ast)}{\leq} \|x\| \qquad \forall n\in \mathbb{N}
                \end{equation*}
                donde en $(\ast)$ hemos vuelto a usar el Ejercicio~\ref{ej:norman}. Aplicando el Principio de acotación uniforme, tenemos que $\{\|T_n\|\}$ está acotada, es decir, existe $M>0$ de forma que:
                \begin{equation*}
                    \|T_n\| \leq M \qquad \forall n\in \mathbb{N}
                \end{equation*}
                con lo que:
                \begin{equation*}
                    \|y\|_n = \|T_n(y)\|_\infty \leq M\|y\| \qquad \forall y\in Y
                \end{equation*}
                Sea $y\in B\left(0,\nicefrac{1}{M}\right)$, queremos deducir que $y\in \overline{T(B(0,1)}$ :
                \begin{equation*}
                    \|y\|_n = \inf_{y=T(u)+v}\{\|u\|+n\|v\|\} \leq M \|y\| < \frac{M}{M} = 1 \qquad \forall n\in \mathbb{N}
                \end{equation*}
                Como el ínfimo de dicho conjunto es menor que 1, han de existir sucesiones $\{u_m\}$ en $X$ y $\{v_m\}$ en $Y$ de forma que:
                \begin{equation*}
                    y = T(u_m) + nv_m \qquad  \|u_m\| + n\|v_m\| < 1 \qquad \forall m\in \mathbb{N}
                \end{equation*}
                Fijado $n\in \mathbb{N}$, tenemos entonces que:
                \begin{equation*}
                    \|u_m\|,\ n\|v_m\| < 1 \qquad \forall m\in \mathbb{N}
                \end{equation*}
                luego $\|v_m\| < \nicefrac{1}{n}$ para todo $m\in \mathbb{N}$. Ahora:
                \begin{equation*}
                    T(u_n) = T(u_n) + v_n - v_n = y - v_n \stackrel{(\ast)}{\to} y
                \end{equation*}
                donde en $(\ast)$ usamos que $\|v_m\| < \nicefrac{1}{n}$. Observamos ahora que $u_n\in B(0,1)$ y que $v_n \to 0$, por lo que ha de ser $y\in \overline{T(B(0,1))}$. Tomando $r=\nicefrac{1}{M}$ tenemos el paso 1.
            \item [Paso 2.] Veamos ahora que $B\left(0,\frac{r}{2}\right)\subset T(B(0,1))$.
               
                Sabemos del paso 1 que $\exists r>0$ de forma que $B(0,r)\subset \overline{T(B(0,1))}$, con lo que:
                \begin{equation*}
                    B\left(0,\frac{r}{2^n}\right) = \frac{1}{2^n}B(0,r) \subset \frac{1}{2^n}\overline{T(B(0,1))} = \overline{T\left(B\left(0,\frac{1}{2^n}\right)\right)} \qquad \forall n\in \mathbb{N}
                \end{equation*}
                \begin{itemize}
                    \item Para $n=1$ tenemos:
                        \begin{equation*}
                            B\left(0,\frac{r}{2}\right)\subset \overline{T\left(B\left(0,\frac{1}{2}\right)\right)}
                        \end{equation*}
                        por lo que si $y\in B\left(0,\frac{r}{2}\right)$, tenemos entonces que existe $x_1\in B\left(0,\frac{1}{2}\right)$ tal que\footnote{Esta distancia podemos acotarla tanto como queramos.} $\|y-T(x_1)\| < \frac{r}{2^2}$, de donde $y-T(x_1)\in B\left(0,\frac{r}{2^2}\right)$
                    \item Para $n=2$:
                        \begin{equation*}
                            B\left(0,\frac{r}{2^2}\right)\subset \overline{T\left(B\left(0,\frac{1}{2^2}\right)\right)}
                        \end{equation*}
                        Como $y-T(x_1)\in B\left(0,\frac{r}{2^2}\right)$, podemos encontrar $x_2\in B\left(0,\frac{1}{2^2}\right)$ tal que $\|y-T(x_1)-T(x_2)\| < \frac{r}{2^3}$, de donde $y-T(x_1)-T(x_2)\in B\left(0,\frac{r}{2^3}\right)$.
                    \item En definitiva, hemos obtenido $x_n \in B\left(0,\frac{1}{2^n}\right)$ tal que:
                        \begin{equation}\label{eq:suma_parcial}
                            \left\|y - \sum_{k=1}^{n}T(x_k)\right\| < \frac{r}{2^{n+1}} \qquad \forall n\in \mathbb{N}
                        \end{equation}
                \end{itemize}
                Con lo que:
                \begin{equation*}
                    \sum_{n=1}^{\infty}\|x_n\| \leq \sum_{n=1}^{\infty}\frac{1}{2^n} < \infty
                \end{equation*}
                Por lo que la serie $\sum\limits_{n\geq 1}x_n$ es convergente en norma, y como $X$ es de Banach, tenemos que $\sum\limits_{n\geq 1}x_n$ es converegente, a cierto $x\in X$. Ahora:
                \begin{equation*}
                    \|x\| \leq \sum_{n=1}^{\infty}\|x_n\| < \sum_{n=1}^{\infty}\frac{1}{2^n} =1
                \end{equation*}
                De la ecuación~(\ref{eq:suma_parcial}) obtenemos que:
                \begin{equation*}
                    \sum_{n=1}^{\infty}T(x_k) = y
                \end{equation*}
                Finalmente:
                \begin{align*}
                    T(x) &= T\left(\sum_{n=1}^{\infty}x_k\right) = T\left(\lim_{n\to\infty}\sum_{k=1}^{n}x_k\right) \AstIg \lim_{n\to\infty}T\left(\sum_{k=1}^{n}x_k\right) \\ &= \lim_{n\to\infty}\sum_{k=1}^{n}T(x_k) = \sum_{n=1}^{\infty}T(x_k)  = y
                \end{align*}
                donde en $(\ast)$ usamos que $T$ es continua. En defintiva, hemos probado que $y\in T(B(0,1))$, ya que $\|x\| < 1$.
        \end{description}
    \end{proof}
\end{teo}

% TODO: Corolarios del teorema 

\begin{coro}
    Sean $E$ y $F$ dos espacios de Banach y sea $T\in L(E,F)$ biyectiva, entonces $T^{-1}$ es también continua.
    \begin{proof}
        Por el Teorema de la aplicación abierta tenemos que $T$ es abierta
        \begin{description}
            \item [Opción 1.] Deducimos que $T$ es un homeomorfismo, luego $T^{-1}$ es continua.
            \item [Opción 2.] Sea $U\subset E$ abierto, observamos que:
                \begin{equation*}
                    {(T^{-1})}^{-1}(U) = T(U)
                \end{equation*}
                y como $T$ es abierta concluimos que $T(U)$ es abierto, con lo que $T^{-1}$ es continua.
        \end{description}
    \end{proof}
\end{coro}

\begin{coro}\label{coro:equivalencia_normas}
    Sea $E$ un espacio vectorial con normas $\|\cdot \|_1$ y $\|\cdot \|_2$. Supongamos que $E$ es de Banach para ambas normas y que existe una constante $C\geq 0$ de forma que:
    \begin{equation*}
        \|x\|_2 \leq C\|x\|_1 \qquad \forall x\in E
    \end{equation*}
    Entonces, las dos normas son equivalentes, es decir, existe una constante $c>0$ de forma que:
    \begin{equation*}
        \|x\|_1 \leq c\|x\|_2 \qquad \forall x\in E
    \end{equation*}
    \begin{proof}
        Sea $T:(E,\|\cdot \|_1)\to (E,\|\cdot \|_2)$ dada por $T(x) = x$, tenemos claramente que $T$ es lineal, así como que:
        \begin{equation*}
            \|T(x)\|_2 = \|x\|_2 \leq C\|x\|_1 \qquad \forall x\in E
        \end{equation*}
        por lo que $T$ es continua. Es claro que también $T\in L(E,E)$ es biyectiva, con lo que por el Ejercicio anterior tenemos que $T^{-1}$ es también continua. Es obvio que $T^{-1}:(E,\|\cdot \|_2)\to (E,\|\cdot \|_1)$ viene dada por $T^{-1}(x) = x$. Es claro que se verifica la igualdad:
        \begin{equation*}
            \|x\|_1 = \|T^{-1}(x)\|_1 \leq \|T^{-1}\|\|x\|_2 \qquad \forall x\in E
        \end{equation*}
        Por lo que tomando $c = \|T^{-1}\|$ se tiene el resultado.
    \end{proof}
\end{coro}

\noindent
Por tanto, si tenemos dos normas completas, es suficiente comprobar una desigualdad para obtener la otra que nos demuestra que las dos normas son equivalentes. 

\section{Teorema de la gráfica cerrada}
\noindent
Sea $T:E\to F$ una aplicación continua con $E,F$ espacios normados, sabemos entonces que $Gr(T)$ es cerrado en $E\times F$.
\begin{proof}
    Sea $\{(x_n,T(x_n))\}$ una sucesión de elementos de $Gr(T)$ convergente a un elemento $(x,y)$ de $E\times F$, tenemos entonces que $\{x_n\}\to x$ y como $T$ es continua, deducimos que:
    \begin{equation*}
        y\leftarrow \{T(x_n)\} \to T(x)
    \end{equation*}
    Por lo que $y = T(x)$, de donde $(x,y)\in Gr(T)$, luego $Gr(T)$ es cerrado.
\end{proof}

\begin{teo}[de la Gráfica Cerrada]
    Sean $E,F$ dos espacios de Banach y sea $T:E\to F$ una aplicación lineal. Si $Gr(T)$ es cerrado, entonces $T$ es continua.
    \begin{proof}
        Definimos una nueva norma en $E$, que depende del funcional $T$:
        \begin{equation*}
            \|x\|_T = \|x\|_E + \|Tx\|_F \qquad \forall x\in E
        \end{equation*}
        Que es una norma:
        \begin{itemize}
            \item Sea $\lm\in \mathbb{R}$, $x\in E$, usando que $T$ es lineal:
                \begin{equation*}
                    \|\lm x\|_T = \|\lm x\|_E + \|T(\lm x)\|_F = |\lm| (\|x\|_E + \|T(x)\|_F) = |\lm| \|x\|_T
                \end{equation*}
            \item Tenemos que:
                \begin{equation*}
                    \|x\|_T = 0 \Longleftrightarrow \|x\|_E + \|Tx\|_F \Longleftrightarrow \left\{\begin{array}{l}
                        \|x\|_E = 0 \\
                        \|Tx\|_F = 0
                    \end{array}\right. \Longleftrightarrow x = 0
                \end{equation*}
            \item Si $x_1,x_2\in E$:
                \begin{align*}
                    \|x_1 + x_2\|_T = \|x_1 + x_2\|_E + \|T(x_1+x_2)\|_F &\leq \|x_1\|_E + \|x_2\|_E + \|Tx_1\|_F + \|Tx_2\|_F \\
                                    &= \|x_1\|_T + \|x_2\|_T
                \end{align*}
        \end{itemize}
        Además, $(E,\|\cdot \|_T)$ es completo. Para ello, si $\{x_n\}$ es una sucesión de puntos de $E$ de Cauchy para $\|\cdot \|_T$, entonces:
        \begin{equation*}
            \forall \varepsilon>0~\exists n_0\in \mathbb{N} : n,m\geq n_0 \Longrightarrow \|x_n-x_m\|_T < \varepsilon
        \end{equation*}
        Como:
        \begin{equation*}
            \|x_n - x_m\|_T = \|x_n - x_m\|_E + \|T(x_n - x_m)\|_F \quad \Longrightarrow \quad  \left\{\begin{array}{l}
                \|x_n - x_m\|_E \\
                \|T(x_n - x_m)\|_F
            \end{array}\right.
        \end{equation*}
        Por lo que $\{x_n\}$ es de Cauchy para $\|\cdot \|_E$ en $E$ y $\{T(x_n)\}$ es de Cauchy en $F$. Como ambos son espacios de Banach, existe $x\in E$ y $y\in F$ de forma que:
        \begin{equation*}
            \|x_n - x\|_E \to 0 \qquad \|T(x_n) -y\|_F \to 0
        \end{equation*}
        Como $(x_n,Tx_n) \in Gr(T)$ para todo $n\in \mathbb{N}$, tenemos entonces que $\{(x_n,Tx_n)\}$ es una sucesión de puntos de $Gr(T)$ convergente a $(x,y)$. Como por hipótesis $Gr(T)$ es cerrado, tenemos entonces que $(x,y)\in Gr(T)$, por lo que $y = T(x)$.\\

        \noindent
        Tenemos por tanto que $\|x_n - x\|_E, \|Tx_n - Tx\|_F \to 0$, de donde:
        \begin{equation*}
            \|x_n - x\|_T = \|x_n - x\|_E + \|Tx_n - Tx\|_F \to 0
        \end{equation*}
        Por lo que $\{x_n\}$ es convergente para $\|\cdot \|_T$ a $x$, de donde $(E,\|\cdot \|_T)$ es completo.\\

        \noindent
        Observemos ahora que $\|x\|_E \leq \|x\|_T$ para todo $x\in E$. Como las dos normales son completas, tenemos por el Corolario~\ref{coro:equivalencia_normas} que son equivalentes. Tenemos entonces que existe $k\geq 0$ de forma que:
        \begin{equation*}
            \|x\|_T = \|x\|_E + \|Tx\|_F \leq k\|x\|_E \qquad \forall x\in E
        \end{equation*}
        de donde $k\geq 1$, con lo que:
        \begin{equation*}
            \|Tx\|_F \leq (k-1)\|x\|_E \qquad \forall x\in E
        \end{equation*}
        Por lo que $T$ es continua en $E$.
    \end{proof}
\end{teo}

% top debiles
% teoria de hilbert y Tª espectral
