% // TODO: Seguimos con teoria, principio de acotacion uniforme
\chapter{Principio de acotación uniforme}
\begin{definicion}
    Sean $E,F$ espacios normados, definimos:
    \begin{equation*}
        L(E,F) = \{f:E\to F : f \text{\ lineal y continua}\}
    \end{equation*}
\end{definicion}

\begin{definicion} % // TODO: Escribir esto bien
    Si $T:E\to F$, es lineal, tenemos que:
    \begin{equation*}
        T \text{\ continua} \Longleftrightarrow T \text{\ continua en\ } 0 \Longleftrightarrow \sup_{\|x\|\leq 1}\|Tx\| < \infty
    \end{equation*}
    Por lo que podemos definir:
    \begin{equation*}
        \|T\| = \sup_{\|x\|\leq 1}\|Tx\|
    \end{equation*}
    o también:
    \begin{equation*}
        \|T\| = \inf\{M>0 : \|Tx\| \leq M\|x\|, \quad \forall x\in E\}
    \end{equation*}
\end{definicion}

% // TODO: Definir L(E,F)

\noindent
Sea $E$ un espacio de Banach, $F$ un espacio normado y $\cc{F}$ una familia de operadores subconjunto de $L(E,F)$. Si 
\begin{equation*}
    \sup_{T\in \cc{F}}\|Tx\| < +\infty \quad \forall x\in E \qquad \Longrightarrow \qquad  \sup_{T\in \cc{F}}\|T\| < +\infty
\end{equation*}

% Demostración distinta a la que viene en Brezis
% La de Brezis se deja como ejercicio mirarla luego, se basa en el Lema de Baire, que puede ser util

\begin{lema}\label{lema:acotacion_uniforme}
    Sean $E,F$ espacios normados y $T\in L(E,F)$, entonces:
    \begin{equation*}
        \sup_{\|x-x_0\| <r} \|Tx\| \geq r\|T\|\qquad \forall x_0\in E, \quad \forall r>0
    \end{equation*}
    \begin{proof}
        Sea $y\in E$: 
        \begin{align*}
            \|Ty\| &= \left\|T\left(\frac{1}{2}\left[x_0 + y - (x_0-y)\right]\right)\right\| = \dfrac{1}{2}\|T(x_0 + y - (x_0-y))\| \\ 
                   &\leq \dfrac{1}{2}\left[\|T(x_0+y)\| + \|T(x_0-y)\|\right] \qquad x_0 \in E \\
                   &\leq \max\{\|T(x_0+y)\|,\|T(x_0-y\|)\} \qquad \forall x_0\in E \\
                   &\leq \sup_{\|y\|\leq r} \max\{\|T(x_0+y)\|, \|T(x_0-y)\|\} \\
                   &\leq \sup_{\|z-x_0\| \leq r} \|Tz\|
        \end{align*}
        La idea es usar que:
        \begin{equation*}
            \sup_{\|y\|\leq r}\|Ty\| = r\sup_{\|z\|\leq 1}\|Tz\| = r\|T\|
        \end{equation*}
    :   Por lo que tenemos ya el Lema probado.
    \end{proof}
\end{lema}

\begin{teo}
    % EL teorema enunciado
    \begin{proof}
        Demostraremos el contrarrecíproco:
        \begin{equation*}
            \sup_{T\in \cc{F}}\|T\| = \infty \qquad  \Longrightarrow \qquad  \sup_{T\in \cc{F}}\|Tx\| = \infty
        \end{equation*}
        Entonces, supongamos que $\sup\limits_{T\in \cc{F}}\|T\| = \infty$, por lo que existe una sucesión de elementos de $\cc{F}$, $\{T_n\}$, de forma que:
        \begin{equation*}
            \|T_n\| \geq 4^n \qquad \forall n\in \mathbb{N}
        \end{equation*}
        Definimos por inducción una sucesión de puntos de $E$:
        \begin{itemize}
            \item $x_0 = 0\in E$.
            \item Tomando $r = \nicefrac{1}{3}$ en el Lema~\ref{lema:acotacion_uniforme}:
                \begin{equation*}
                    \sup_{\|x-x_0\| < \frac{1}{3}}\|T_1x\| \geq \dfrac{1}{3}\|T_1\| > \dfrac{2}{3}\cdot \dfrac{1}{3}\|T_1\|
                \end{equation*}
                Por lo que esta última no puede ser cota superior (está por debajo de todas las cotas superiores), con lo que tiene que $\exists x_1\in B(x_0, \nicefrac{1}{3})$ de forma que:
                \begin{equation*}
                    \dfrac{1}{3}\|T_1\| > \dfrac{2}{3}\cdot \dfrac{1}{3} \|T_1\| 
                \end{equation*}
            \item Supuesto construido hasta $x_{n-1}$, veamos cómo construir $x_n$:

                Tomamos $r = \nicefrac{1}{3^n}$ y aplicamos el Lema~\ref{lema:acotacion_uniforme}, que nos dice:
                \begin{equation*}
                    \sup_{\|x-x_{n-1}\| < \frac{1}{3^n}}\|T_nx\| \geq \dfrac{1}{3^n}\|T_n\| > \dfrac{2}{3}\cdot \dfrac{1}{3^n} \|T_n\|
                \end{equation*}
                Como este número es estrictamente menor que el supremo (que es el ínfimo de todas la s cotas superiores), entonces dicho número no puede ser cota superior, con lo que podemos encontrar $x_n \in B(x_{n-1},\nicefrac{1}{3^n})$ de forma que:
                \begin{equation*}
                    \|T_nx_n\| > \dfrac{2}{3}\cdot \dfrac{1}{3^n}\|T_n\|
                \end{equation*}
        \end{itemize}
        Veamos que $\{x_n\}$ es de Cauchy. Para ello, buscamos acotar $\|x_m - x_n\|$ para $n,m$ índices bastante avanzados. Supondremos sin pérdida de generalidad que $n,m\in \mathbb{N}$ con $m>n$, donde tendremos:
        \begin{align*}
            \|x_m-x_n\| &= \|x_m - x_{m-1} + x_{m-1} - x_{m-2} + \ldots + x_{n+1} - x_n\| \\
                        &\leq \|x_m - x_{m-1} \| + \| x_{m-1} - x_{m-2} \| + \ldots + \|x_{n+1} - x_n\| \\
                        &\leq \dfrac{1}{3^m} + \dfrac{1}{3^{m-1}} + \ldots + \dfrac{1}{3^{n+1}} \AstIg \dfrac{1}{3^n}\left[\dfrac{1}{3^{m-n}} + \ldots + \dfrac{1}{3}\right] \\
                        &\leq \dfrac{1}{3^n} \sum_{j=1}^{+\infty} \dfrac{1}{3^j} = \dfrac{1}{3^n} \dfrac{\frac{1}{3}}{1-\frac{1}{3}} = \dfrac{1}{3^n}\cdot \dfrac{1}{2}
        \end{align*}
        La idea que hay detrás es usar que $\sum_{j\geq 1} \frac{1}{3^j}$ es convergente, con lo que podemos tomar las sumas parciales y justificar que convergen a $0$. Sin embargo, podemos justificarlo de otra forma, usando la serie geométrica, tal y como se ha hecho en $(\ast)$. En definitiva, tenemos que:
        \begin{equation*}
            \|x_m-x_n\| = \|x_m - x_{m-1} + x_{m-1} - x_{m-2} + \ldots + x_{n+1} - x_n\| \leq \dfrac{1}{2}\cdot \dfrac{1}{3^n}
        \end{equation*}
        Por lo que $\{x_n\}$ es de Cauchy, en $E$ (era Banach), por lo que $\{x_n\}$ converge a cierto $x\in E$. Observemos que:
        \begin{equation*}
            \lim_{m\to\infty}\|x_m-x_n\| = \|\lim_{m\to\infty}(x_m-x_n)\| = \|x-x_n\| \leq \dfrac{1}{2}\cdot \dfrac{1}{3^n}
        \end{equation*}
        Calculamos ahora:
        \begin{align*}
            \|T_nx\| &= \|T_n(x-x_n + x_n)\| \geq \|T_nx_n\| - \|T_n(x-x_n)\| \geq \dfrac{2}{3}\cdot \dfrac{1}{3^n}\|T_n\| - \|T_n\|\|x-x_n\| \\
                     &\geq \dfrac{2}{3}\cdot \dfrac{1}{3^n}\|T_n\| - \|T_n\|\dfrac{1}{2}\cdot \dfrac{1}{3^n} = \left(\dfrac{2}{3}-\dfrac{1}{2}\right) \dfrac{1}{3^n} \|T_n\| = \dfrac{1}{6}\cdot \dfrac{1}{3^n}\|T_n\| \geq \dfrac{1}{6}{\left(\dfrac{4}{3}\right)}^{n} \to \infty
        \end{align*}
        Por tanto:
        \begin{equation*}
            \sup_{T\in \cc{F}}\|Tx\| \geq \sup_{n\in \mathbb{N}}\|T_nx\| = \infty
        \end{equation*}
        Como queríamos probar.
    \end{proof}
\end{teo}

\begin{ejercicio}
    Se deja como ejercicio consultar la demostración del libro de Brezis usando el Lema de Baire.
\end{ejercicio}

\begin{lema}[de Baire]
    Sea $X$ un espacio métrico completo, sea $X_n\subset X$ con $X_n$ cerrado y $Int X_n = \emptyset $ para todo $n\in \mathbb{N}$, entonces:
    \begin{equation*}
        Int\left(\bigcup_{n\in \mathbb{N}} X_n\right) = \emptyset 
    \end{equation*}
    % \begin{proof}
    %     Mirar la demostracion y compararla con la que hemos hecho del principio de acotación uniforme.
    %     Se deja como ejercicio hacer la demostracion a partir del Lema de Baire, la dificultad es escoger los X_n
    % \end{proof}
\end{lema}

El contrarrecíproco\footnote{Que es la forma en la que siempre se usa el Lema.} sería que:
\begin{center}
    Si $X$ es un espacio métrico completo, sea $X_n\subset X$ con $X_n$ cerrado para todo $n\in \mathbb{N}$, entonces si:
    \begin{equation*}
        Int\left(\bigcup_{n\in \mathbb{N}}X_n\right) \neq \emptyset  \Longrightarrow \exists n_0\in \mathbb{N} : Int(X_{n_0}) \neq \emptyset 
    \end{equation*}
\end{center}

% // TODO: Ejercicios para mañana
% 3 corolarios despues del principio de acotacion uniforme: 2.3, 2.4, 2.5

% viernes el Tª de app abierta
% falta Tª grafica cerrada
