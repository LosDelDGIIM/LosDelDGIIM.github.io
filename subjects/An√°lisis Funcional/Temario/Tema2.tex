\chapter{Principio de acotación uniforme y Tª de la gráfica cerrada}
\begin{definicion}
    Sean $E,F$ espacios normados, definimos:
    \begin{equation*}
        L(E,F) = \{f:E\to F : f \text{\ lineal y continua}\}
    \end{equation*}
    y notaremos normalmente $L(E) := L(E,E)$.
\end{definicion}

\begin{prop}
    Al igual que como sucedía con aplicaciones lineales y continuas de un espacio normado $E$ en $\mathbb{R}$, si $E,F$ son espacios normados y $T\in L(E,F)$ tenemos\footnote{Resultados análogos que se realizan con las mismas pruebas.}:
    \begin{enumerate}
        \item $T$ es continua $\Longleftrightarrow T$ es continua en $0 \Longleftrightarrow \sup_{\|x\|\leq 1}\|Tx\| < \infty$
        \item Si definimos:
            \begin{equation*}
                \|T\| := \sup_{\|x\|\leq 1}\|Tx\| \qquad \forall T\in L(E,F)
            \end{equation*}
            Tenemos que $\|\cdot \|$ es una norma en $L(E,F)$.
        \item Se verifica que:
            \begin{equation*}
                \|T\| = \inf\{M>0 : \|Tx\| \leq M\|x\|, \quad \forall x\in E\}
            \end{equation*}
    \end{enumerate}
\end{prop}

\section{Principio de acotación uniforme}
Con vistas a demostrar el Principio de acotación uniforme, demostramos el siguiente Lema:

\begin{lema}\label{lema:acotacion_uniforme}
    Sean $E,F$ espacios normados y $T\in L(E,F)$, entonces:
    \begin{equation*}
        \sup_{\|x-x_0\| \leq r} \|Tx\| \geq r\|T\|\qquad \forall x_0\in E, \quad \forall r>0
    \end{equation*}
    \begin{proof}
        Fijado $r\in \mathbb{R}^+$, sean $x_0,y\in E$ con $\|y\|\leq r$:
        \begin{align*}
            \|Ty\| &= \left\|T\left(\frac{1}{2}\left[x_0 + y - (x_0-y)\right]\right)\right\| = \dfrac{1}{2}\|T(x_0 + y - (x_0-y))\| \\ 
                   &\leq \dfrac{1}{2}\left(\|T(x_0+y)\| + \|T(x_0-y)\|\right) \leq \max\{\|T(x_0+y)\|,\|T(x_0-y\|)\} \\
                   &\leq \sup_{\|y\|\leq r} \max\{\|T(x_0+y)\|, \|T(x_0-y)\|\} \leq \sup_{\|z-x_0\| \leq r} \|Tz\|
        \end{align*}
        Si ahora observamos que:
        \begin{equation*}
            \sup_{\|y\|\leq r}\|Ty\| = r\sup_{\|z\|\leq 1}\|Tz\| = r\|T\|
        \end{equation*}
        Acabamos de probar que $\sup\limits_{\|x-x_0\|\leq r}\|Tx\| \geq r\|T\|$.
    \end{proof}
\end{lema}

\begin{teo}[Principio de acotación uniforme]\label{teo:principio_acotacion_uniforme}
Sea $E$ un espacio de Banach, $F$ un espacio normado y $\cc{F}$ un subconjunto de $L(E,F)$, entonces:
\begin{equation*}
    \sup_{T\in \cc{F}}\|Tx\| < +\infty \quad \forall x\in E \qquad \Longrightarrow \qquad  \sup_{T\in \cc{F}}\|T\| < +\infty
\end{equation*}
    \begin{proof}
        Demostraremos el contrarrecíproco:
        \begin{equation*}
            \sup_{T\in \cc{F}}\|T\| = \infty \qquad  \Longrightarrow \qquad  \sup_{T\in \cc{F}}\|Tx\| = \infty
        \end{equation*}
        Supongamos pues que $\sup\limits_{T\in \cc{F}}\|T\| = \infty$, por lo que existe una sucesión de elementos de $\cc{F}$, llamémosla $\{T_n\}$, de forma que:
        \begin{equation*}
            \|T_n\| \geq 4^n \qquad \forall n\in \mathbb{N}
        \end{equation*}
        Definimos por inducción una sucesión de puntos de $E$:
        \begin{itemize}
            \item $x_0 = 0\in E$.
            \item Tomando $r = \nicefrac{1}{3}$, el Lema~\ref{lema:acotacion_uniforme} nos dice:
                \begin{equation*}
                    \sup_{\|x-x_0\| < \frac{1}{3}}\|T_1x\| \geq \dfrac{1}{3}\|T_1\| > \dfrac{2}{3}\cdot \dfrac{1}{3}\|T_1\|
                \end{equation*}
                Como $\nicefrac{2}{3}\cdot \nicefrac{1}{3}\cdot \|T_1\|$ es estrictamente menor que el supremo de la izquierda, tenemos que no puede ser una cota superior de $\|T_1 x\|$ para $x\in B(x_0,\nicefrac{1}{3})$, con lo que $\exists x_1\in B(x_0,\nicefrac{1}{3})$ de forma que:
                \begin{equation*}
                    \|T_1x_1\| > \dfrac{2}{3}\cdot \dfrac{1}{3}\|T_1\|
                \end{equation*}
            \item Supuesto que hemos construido hasta $x_{n-1}$, veamos cómo construir $x_n$:

                Tomando $r = \nicefrac{1}{3^n}$, el Lema~\ref{lema:acotacion_uniforme} nos dice que:
                \begin{equation*}
                    \sup_{\|x-x_{n-1}\| < \frac{1}{3^n}}\|T_nx\| \geq \dfrac{1}{3^n}\|T_n\| > \dfrac{2}{3}\cdot \dfrac{1}{3^n} \|T_n\|
                \end{equation*}
                Y por el mismo razonamiento de antes podemos encontrar $x_n\in B(x_{n-1},\nicefrac{1}{3^n})$ de forma que:
                \begin{equation*}
                    \|T_nx_n\| > \dfrac{2}{3}\cdot \dfrac{1}{3^n}\|T_n\|
                \end{equation*}
        \end{itemize}
        Veamos ahora que $\{x_n\}$ es de Cauchy. Para ello, buscamos acotar $\|x_m - x_n\|$ para $n,m$ índices bastante avanzados. Supondremos sin pérdida de generalidad que $n,m\in \mathbb{N}$ con $m>n$, donde tendremos:
        \begin{align*}
            \|x_m-x_n\| &= \|x_m - x_{m-1} + x_{m-1} - x_{m-2} + \ldots + x_{n+1} - x_n\| \\
                        &\leq \|x_m - x_{m-1} \| + \| x_{m-1} - x_{m-2} \| + \ldots + \|x_{n+1} - x_n\| \\
                        &\leq \dfrac{1}{3^m} + \dfrac{1}{3^{m-1}} + \ldots + \dfrac{1}{3^{n+1}} = \dfrac{1}{3^n}\left[\dfrac{1}{3^{m-n}} + \ldots + \dfrac{1}{3}\right] \\
                        &\leq \dfrac{1}{3^n} \sum_{j=1}^{+\infty} \dfrac{1}{3^j} = \dfrac{1}{3^n} \dfrac{\frac{1}{3}}{1-\frac{1}{3}} = \dfrac{1}{3^n}\cdot \dfrac{1}{2}
        \end{align*}
        En definitiva, tenemos que:
        \begin{equation*}
            \|x_m-x_n\| = \|x_m - x_{m-1} + x_{m-1} - x_{m-2} + \ldots + x_{n+1} - x_n\| \leq \dfrac{1}{2}\cdot \dfrac{1}{3^n}
        \end{equation*}
        Por lo que $\{x_n\}$ es de Cauchy en $E$, que era de Banach, por lo que $\{x_n\}$ converge a cierto $x\in E$. Observemos que:
        \begin{equation*}
            \dfrac{1}{2}\cdot \dfrac{1}{3^n} \geq \lim_{m\to\infty}\|x_m-x_n\| \AstIg \left\|\lim_{m\to\infty}(x_m-x_n)\right\| = \|x-x_n\|
        \end{equation*}
        donde en $(\ast)$ hemos usado la continuidad de $\|\cdot \|$. Calculamos ahora:
        \begin{align*}
            \|T_nx\| &= \|T_n(x-x_n + x_n)\| \geq \|T_nx_n\| - \|T_n(x-x_n)\| \geq \dfrac{2}{3}\cdot \dfrac{1}{3^n}\|T_n\| - \|T_n\|\|x-x_n\| \\
                     &\geq \dfrac{2}{3}\cdot \dfrac{1}{3^n}\|T_n\| - \|T_n\|\dfrac{1}{2}\cdot \dfrac{1}{3^n} = \left(\dfrac{2}{3}-\dfrac{1}{2}\right) \dfrac{1}{3^n} \|T_n\| = \dfrac{1}{6}\cdot \dfrac{1}{3^n}\|T_n\| \geq \dfrac{1}{6}{\left(\dfrac{4}{3}\right)}^{n} \to \infty
        \end{align*}
        Por tanto:
        \begin{equation*}
            \sup_{T\in \cc{F}}\|Tx\| \geq \sup_{n\in \mathbb{N}}\|T_nx\| = \infty
        \end{equation*}
        Como queríamos probar.
    \end{proof}
\end{teo}

\noindent
Introducimos ahora una serie de Corolarios que nos da el Principio de acotación uniforme:

\begin{coro}
    Sean $E,F$ dos espacios de Banach, sea $\{T_n\}$ una sucesión de elementos de $L(E,F)$ de forma que $\{T_n(x)\} \to T(x)$ para todo $x\in E$. Entonces:
    \begin{enumerate}[label=(\alph*)]
        \item $\sup\limits_{n\in \mathbb{N}}\|T_n\| < \infty$.
        \item $T\in L(E,F)$.
        \item $\|T\| \leq \liminf\limits_{n\to\infty}\|T_n\|$.
    \end{enumerate}
    \begin{proof}
        Demostramos cada apartado:
        \begin{enumerate}[label=(\alph*)]
            \item Dado $x\in E$, como $\{T_n(x)\}\to T(x)$, tenemos que $\exists m\in \mathbb{N}$ de forma que:
                \begin{equation*}
                    T_n(x) \in B(T(x), 1) \qquad \forall m\geq n
                \end{equation*}
                Por lo que $\{T_n(x) : n\in \mathbb{N}\}$ es un conjunto acotado, puesto que podemos verlo como la unión de un conjunto acotado y uno finito:
                \begin{equation*}
                    \{T_n(x) : n\in \mathbb{N}\} = \{T_n(x) : n\geq m\} \cup \{T_n(x) : n < m\} 
                \end{equation*}
                En definitiva, tenemos que $\sup_{n\in \mathbb{N}}\|T_n(x)\| < \infty$ para todo $x\in E$, de donde aplicando el Principio de acotación uniforme tenemos que $\sup_{n\in \mathbb{N}}\|T_n\| < \infty$.
            \item Veamos que $T:E\to F$ es lineal y continua:
                \begin{itemize}
                    \item Es fácil ver que $T$ es lineal:
                        \begin{align*}
                            T(\lm x + y) &= \lim_{n\to\infty}T_n(\lm x + y) = \lim_{n\to\infty}(\lm T_n(x) + T_n(y)) \\ 
                                         &= \lm \lim_{n\to\infty}T_n(x) + \lim_{n\to\infty}T_n(y) = \lm T(x) + T(y) \qquad \forall x,y\in E
                        \end{align*}
                    \item Para ver que $T$ es continua podemos usar el apartado $(a)$:
                        \begin{equation*}
                            \|T_n(x)\| \leq \|T_n\|\|x\| \leq \sup_{n\in \mathbb{N}}\|T_n\|\|x\| \qquad \forall x\in E
                        \end{equation*}
                        Y como $\{\|T_n(x)\|\}\to \|T(x)\|$, tenemos que:
                        \begin{equation*}
                            \|T(x)\| \leq \sup_{n\in \mathbb{N}}\|T_n\|\|x\| \qquad \forall x\in E
                        \end{equation*}
                        lo que nos dice que $T$ es continua.
                \end{itemize}
            \item Para ver que $\|T\| \leq \liminf\limits_{n\to \infty} \|T_n\|$, notemos que:
                \begin{align*}
                    \|T(x)\| &= \left\|\lim_{n\to\infty}T_n(x)\right\| = \lim_{n\to\infty}\left\|T_n(x)\right\| \leq \lim_{n\to\infty}(\|T_n\|\|x\|) = \liminf (\|T_n\|\|x\|) \\
                             &= \sup_{k\in \mathbb{N}}\inf_{n\geq k}(\|T_n\|\|x\|) = \sup_{k\in \mathbb{N}}\inf_{n\geq k}\|T_n\| \cdot \|x\| = \liminf \|T_n\| \cdot \|x\| \qquad \forall x\in E
                \end{align*}
                De donde $\|T\|\leq \liminf \|T_n\|$.
        \end{enumerate}
    \end{proof}
\end{coro}

\begin{coro}
    Sea $G$ un espacio de Banach y $B\subset G$, si para toda $f\in G^\ast$ el conjunto $f(B)$ está acotado (en $\mathbb{R}$), entonces $B$ está acotado.
    \begin{proof}
        Definimos para todo $b\in B$ la aplicación $T_b:G^\ast\to \mathbb{R}$ dada por:
        \begin{equation*}
            T_b(f) = f(b) \qquad \forall f\in G^\ast
        \end{equation*}
        Con lo que $T_b\in L(G^\ast,\mathbb{R})$:
        \begin{itemize}
            \item Es claro que $T_b$ es lineal.
            \item $T_b$ es continua, ya que $|T_b(f)| = |f(b)| \leq \|b\|\|f\| \quad \forall f\in G^\ast$.
        \end{itemize}
        Como $f(B)$ es acotado para toda $f\in G^\ast$, tenemos entonces que:
        \begin{equation*}
            \sup_{b\in B}|T_b(f)| = \sup_{b\in B}|f(b)| < \infty \qquad \forall f\in G^\ast
        \end{equation*}
        Con lo que aplicando el Principio de acotación uniforme tenemos: 
        \begin{equation*}
            \sup\limits_{b\in B}\|T_b\| < \infty
        \end{equation*}
        Ahora, si tomamos $f\in G^\ast$ con $\|f\|\leq 1$ y buscamos usar que $\|x\| = \sup\limits_{\|f\|\leq 1}|f(x)|$:
        \begin{equation*}
            |f(b)| = |T_b(f)| \leq \sup_{b\in B}\|T_b\|\|f\| \leq \sup_{b\in B}\|T_b\|
        \end{equation*}
        Por lo que $\|b\| \leq \sup\limits_{b\in B}\|T_b\| < \infty \quad \forall b\in B$, lo que nos dice que $B$ está acotado.
    \end{proof}
\end{coro}

\begin{coro}
    Sea $G$ un espacio de Banach y sea $B^\ast\subset G^\ast$, si el conjunto:
    \begin{equation*}
        B^\ast(x) = \{f(x) : f\in B^\ast\}
    \end{equation*}
    está acotado para todo $x\in G$, entonces $B^\ast$ está acotado.
\end{coro}

\begin{ejercicio}
    Se deja como ejercicio consultar la demostración del libro de Brezis usando el Lema de Baire.
\end{ejercicio}

\begin{lema}[de Baire]
    Sea $X$ un espacio métrico completo, sea $X_n\subset X$ con $X_n$ cerrado y $Int X_n = \emptyset $ para todo $n\in \mathbb{N}$, entonces:
    \begin{equation*}
        Int\left(\bigcup_{n\in \mathbb{N}} X_n\right) = \emptyset 
    \end{equation*}
    % \begin{proof}
    %     Mirar la demostracion y compararla con la que hemos hecho del principio de acotación uniforme.
    %     Se deja como ejercicio hacer la demostracion a partir del Lema de Baire, la dificultad es escoger los X_n
    % \end{proof}
\end{lema}

El contrarrecíproco\footnote{Que es la forma en la que siempre se usa el Lema.} sería que:
\begin{center}
    Si $X$ es un espacio métrico completo, sea $X_n\subset X$ con $X_n$ cerrado para todo $n\in \mathbb{N}$, entonces si:
    \begin{equation*}
        Int\left(\bigcup_{n\in \mathbb{N}}X_n\right) \neq \emptyset  \Longrightarrow \exists n_0\in \mathbb{N} : Int(X_{n_0}) \neq \emptyset 
    \end{equation*}
\end{center}

% // TODO: Ejercicios para mañana
% 3 corolarios despues del principio de acotacion uniforme: 2.3, 2.4, 2.5

% viernes el Tª de app abierta
% falta Tª grafica cerrada
