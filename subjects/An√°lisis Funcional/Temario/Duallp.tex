\section{Dual de $l_p$, para $1\leq p < \infty$}\label{sec:duallp}
\noindent
Consideramos nuevamente el espacio:
\begin{equation*}
    l_p = \{x\in \mathbb{R}^\mathbb{N} : \sum_{n=1}^{\infty}{|x(n)|}^{p}<\infty\}, \qquad \|x\|_p = {\left(\sum_{n=1}^{\infty}{|x(n)|}^{p}\right)}^{\frac{1}{p}}
\end{equation*}
y estaremos interesados en calcular su espacio dual, $l_p^\ast$. Para ello, si para cada $k\in \mathbb{N}$ denotamos por $e_k$ al vector que verifica:
\begin{equation*}
    e_k(n) = \delta_{n,k} = \left\{\begin{array}{ll}
        1 & \text{si\ } n=k \\
        0 & \text{si\ } n\neq k
    \end{array}\right. 
\end{equation*}
claramente tenemos que $e_k\in l_p$, $\forall k\in \mathbb{N}, \forall p\in \left[1,\infty\right[$. Consideraremos el espacio:
\begin{equation*}
    \mathbb{R}^{(\mathbb{N})} = \cc{L}(\{e_i : i \in \mathbb{N}\})\subset l_p
\end{equation*}
Y dado $p\in \left[1,\infty\right[$, consideraremos siempre que $p^\ast$ cumple que:
\begin{equation*}
    \dfrac{1}{p} + \dfrac{1}{p^\ast} = 1, \qquad p^\ast = \infty \text{\ si\ } p = 1
\end{equation*}
Trataremos de probar que \fbox{${(l_p)}^{\ast} \cong l_{p^\ast}$}

\begin{prop}
    Si $p\in \left[1,\infty\right[$, se verifica que:
    \begin{equation*}
        \overline{\mathbb{R}^{(\mathbb{N})}} = l_p 
    \end{equation*}
    \begin{proof}
        Sea $x\in l_p$, definimos para cada $n\in \mathbb{N}$:
        \begin{equation*}
            s_n = \sum_{k=1}^{n} x(k)e_k \in \mathbb{R}^\mathbb{N}
        \end{equation*}
        es decir, $s_n$ será el vector cuyas $n-$ésimas primeras componentes coinciden con las de $x$ y el resto de componentes son $0$:
        \begin{align*}
            s_1 &= (x(1), 0, \ldots) \\
            s_2 &= (x(1), x(2), 0,\ldots) \\
            s_3 &= (x(1), x(2), x(3), 0,\ldots) 
        \end{align*}
        Es evidente que $s_n\in \mathbb{R}^{(\mathbb{N})}$ para cada $n\in \mathbb{N}$, así como que $s_n(k) = x(k)$ siempre que $k\leq n$. De esta última observación deducimos que:
        \begin{equation*}
            {(\|x-s_n\|_p)}^{p} = \sum_{k=n+1}^{\infty} {(|x(k)|)}^{p} \to 0
        \end{equation*}
        por lo que $\{s_n\}\to x$ en $l_p$.
    \end{proof}
\end{prop}

\begin{prop}
    Si $1\leq p < \infty$, el espacio ${(l_p)}^{\ast}$ contiene una copia isométrica de $l_{p^\ast}$.
    \begin{proof}
    Tomaremos $y\in l_{p^\ast}$ y definiremos $\Phi_y:l_p\to \mathbb{R}$ dada por:
    \begin{equation*}
        \Phi_y(x) = \sum_{n=1}^{\infty}x(n)y(n) \qquad \forall x\in l_p
    \end{equation*}
    que está bien definida (la serie es convergente), puesto que:
    \begin{equation*}
        x\in l_p, y\in l_{p^\ast} \Longrightarrow xy \in l_1 \qquad \text{y}\qquad \|xy\|_1 \leq \|x\|_p \|y\|_{p^\ast}
    \end{equation*}
    Veamos que $\Phi_y$ es lineal, continua y que $\|\Phi_y\| = \|y\|_{p^\ast}$, puesto que:
    \begin{itemize}
        \item Si $\lm\in \mathbb{R}$ y $x,z\in l_p$, tenemos que:
            \begin{align*}
                \Phi_y(\lm x + z) &= \sum_{n=1}^{\infty} (\lm x(n) + z(n))y(n) = \lm \sum_{n=1}^{\infty}x(n)y(n) + \sum_{n=1}^{\infty}z(n)y(n) \\ &= \lm\Phi_y(x) + \Phi_y(z)
            \end{align*}
        \item Para ver que $\Phi_y$ es continua, veamos que:
            \begin{equation*}
                |\Phi_y(x)| = \left|\sum_{n=1}^{\infty}x(n)y(n)\right| \leq \sum_{n=1}^{\infty}|x(n)||y(n)| \stackrel{(\ast)}{\leq} \|y\|_{p^\ast}\|x\|_p
            \end{equation*}
            donde en $(\ast)$ usamos la desigualdad de Hölder, con lo que $\Phi_y$ es continua. Además hemos visto ya que, $\|\Phi_y\| \leq \|y\|_{p^\ast}$.
        \item Para la otra desigualdad, distinguimos casos:
            \begin{itemize}
                \item Para $p=1$, tenemos que:
                    \begin{equation*}
                        |y(n)| = |\Phi_y(e_n)| \leq \|\Phi_y\| \|e_n\|_p = \|\Phi_y\| \qquad \forall n\in \mathbb{N}
                    \end{equation*}
                    de donde:
                    \begin{equation*}
                        \|y\|_\infty = \sup_{n\in \mathbb{N}}|y(n)| \leq \|\Phi_y\|
                    \end{equation*}
                \item Para $1<p<\infty$, tomamos:
                    \begin{equation*}
                        x(n) = {(|y_n|)}^{p^\ast-2}y_n \qquad \forall n\in \mathbb{N}
                    \end{equation*}
                    y razonando como en el Ejercicio~\ref{ej:2_rel1} obtenemos la desigualdad:
                    \begin{equation*}
                        \|y\|_{p^\ast} \leq \|\Phi_y\|
                    \end{equation*}
            \end{itemize}
    \end{itemize}
    A partir de eso, veamos que:
    \Func{\Phi}{l_{p^\ast}}{{l_p}^{\ast}}{y}{\Phi_y}
    es lineal, inyectiva y que conserva la norma:
    \begin{itemize}
        \item Si $\lm\in \mathbb{R}$ y $x,y\in l_{p^\ast}$, tenemos que:
            \begin{equation*}
                \Phi(\lm x + y) = \Phi_{(\lm x + y)} \AstIg \lm \Phi_x + \Phi_y = \lm\Phi(x) + \Phi(y)
            \end{equation*}
            donde en $(\ast)$ hemos usado que:
            \begin{align*}
                \Phi_{(\lm x + y)}(z) &= \sum_{n=1}^{\infty}z(n)(\lm x(n) + y(n)) = \lm \sum_{n=1}^{\infty}z(N)x(n) + \sum_{n=1}^{\infty}z(n)y(n) \\ &= \lm\Phi_x(z) + \Phi_y(z)
            \end{align*}
        \item Hemos visto ya que $\|\Phi_y\| = \|y\|_{p^\ast}$ para todo $y\in l_{p^\ast}$.
        \item Como $\Phi$ preserva la norma y es lineal, tenemos que si $x,y\in l_{p^\ast}$ con $\Phi(x) = \Phi(y)$, entonces:
            \begin{equation*}
                0 = \Phi(x-y) \Longrightarrow 0 = \|\Phi(x-y)\| = \|x-y\| \Longrightarrow x = y
            \end{equation*}
            por lo que $\Phi$ es inyectiva.
    \end{itemize}
    \end{proof}
\end{prop}

\begin{prop}
    Si $1\leq p < \infty$, entonces ${l_p}^{\ast}$ es isométrico a $l_{p^\ast}$.
    \begin{proof}
        La demostración se basa en probar que la aplicación $\Phi$ de la Proposición anterior es sobreyectiva. Para ello, fijado $f\in l_p^\ast$, definimos:
        \begin{equation*}
            y(n) = f(e_n) \qquad \forall n\in \mathbb{N}
        \end{equation*}
        \begin{itemize}
            \item Si $p=1$, escribimos: 
                \begin{equation*}
                    |y(k)| = \alpha_k y(k), \qquad \alpha_k = \left\{\begin{array}{cl}
                        1 & \text{si\ } y(k)\geq 0 \\
                         -1& \text{si\ } y(k)<0
                    \end{array}\right. \quad \forall k\in \mathbb{N}
                \end{equation*}
                \begin{equation*}
                    |y(n)| = |f(e_n)| \leq \|f\|\|e_n\| = \|f\| \qquad \forall n\in \mathbb{N}
                \end{equation*}
                Por lo que la sucesión $y$ está acotada, es decir, $y\in l_\infty$, y tenemos que $\Phi(y) = f$.
            \item Si $p>1$, fijado $n\in \mathbb{N}$, tenemos que:
                \begin{align*}
                    \sum_{k=1}^{n} {(|y(k)|)}^{p^\ast} &= \sum_{k=1}^{n} {(\alpha_k y(k))}^{p^\ast-1} = \sum_{k=1}^{n}\alpha_k {(|y(k)|)}^{p^\ast-1} f(e_k) = f\left(\sum_{k=1}^{n}\alpha_k {(|y(k)|)}^{p^\ast-1}e_k\right)  \\
                                                       &\leq \|f\|\left\|\sum_{k=1}^{n}\alpha_k {(|y(k)|)}^{p^\ast-1} e_k\right\|_p  = \|f\| {\left(\sum_{k=1}^{n} {(|y(k)|)}^{(p^\ast-1)p}\right)}^{\frac{1}{p}} \\
                                                       &= \|f\| {\left(\sum_{k=1}^{n}{(|y(k)|)}^{p^\ast}\right)}^{\frac{1}{p}}
                \end{align*}
                Es decir:
                \begin{equation*}
                    {\left(\sum_{k=1}^{n}|y(k)|\right)}^{p^\ast} \leq \|f\| {\left({\left(\sum_{k=1}^{n}|y(k)|\right)}^{p^\ast}\right)}^{\frac{1}{p}}
                \end{equation*}
                de donde:
                \begin{equation*}
                    \|f\| \geq {\left(\sum_{k=1}^{n}{|y(k|)}^{p^\ast}\right)}^{\frac{1}{p^\ast}} \qquad \forall n\in \mathbb{N}
                \end{equation*}
                por lo que $y\in l_{p^\ast}$. Tenemos que:
                \begin{equation*}
                    \Phi_y(e_n) = y(n) = f(e_n) \qquad \forall n\in \mathbb{N}
                \end{equation*}
                de donde $f(x) = \Phi_y(x)$ $\forall x\in \mathbb{R}^{(\mathbb{N})}$. Como es un conjunto denso, han de coincidir por continuidad en todo el espacio, con lo que:
                \begin{equation*}
                    f(x) = \Phi_y(x) \qquad \forall x\in l_p
                \end{equation*}
                de donde $\Phi_y = f$
        \end{itemize}
    \end{proof}
\end{prop}

\noindent
A partir de los resultados vistos en esta sección, se propone:
\begin{ejercicio}
    Calcular el dual de $c_0$, y comprobar que es isométrico con $l_1$.
\end{ejercicio}

\begin{ejercicio}
    Calcular el dual de $c$ y comprobar que es isométrico con $l_1$.
\end{ejercicio}

\begin{ejercicio}
    Demostrar que $c_0$ no es reflexivo.
\end{ejercicio}

\begin{ejercicio}
    Demostrar que $l_p$ es reflexivo para $1<p<\infty$.
\end{ejercicio}

% // TODO: EJERCICIO: CALCULAR EL DUAL DE C0, que es l1
% La misma prueba se puede adaptar para calcular (c0)*, viendo que es isométrico con l1
% Por Hanh-Banach, si tenemos un funcional de l_\infty, al restringirlo dará funcional de c0
% por lo que c0 C linfinito => (linfinito)* C (c0)*

% // TODO: EJERCICIO: CALCULAR EL DUAL DE C, que es l1

% // TODO: Ejercicio: Demostrar que c0 no es reflexivo
% // TODO: Ejercicio: Demostrar que lp 1<p<infty sí son reflexivos


