\chapter{Dual de $l_p$, para $1\leq p < \infty$}
Consideramos:
\begin{equation*}
    \mathbb{R}^{(\mathbb{N})} = \cc{L}(\{e_i : i \in \mathbb{N}\})
\end{equation*}
Supondremos siempre que:
\begin{equation*}
    \dfrac{1}{p} + \dfrac{1}{p^\ast} = 1, \qquad p^\ast = \infty \text{\ si\ } p = 1
\end{equation*}
Y trataremos de probar siempre que ${(l_p)}^{\ast} \cong l_{p^\ast}$.\\

\noindent
Veamos en primer lugar que
\begin{equation*}
    \overline{\mathbb{R}^{(\mathbb{N})}} = l_p \qquad \text{si\ } 1 \leq p < \infty
\end{equation*}
\begin{proof}
    Sea $x\in l_p$, definimos para cada $n\in \mathbb{N}$:
    \begin{equation*}
        s_n = \sum_{k=1}^{n} x(k)e_k
    \end{equation*}
    \begin{align*}
        s_1 &= (x(1), 0, \ldots) \\
        s_2 &= (x(1), x(2), 0,\ldots)
    \end{align*}
    Vemos primero que $s_n \in \mathbb{R}^{(\mathbb{N})}$ para cada $n\in \mathbb{N}$. Además, observamos que $s_n(k) = x(k)$ siempre que $k\leq n$, lo que nos dice que:
    \begin{equation*}
        {(\|x-s_n\|_p)}^{p} = \sum_{k=n+1}^{\infty} {(|x(k)|)}^{p} \to 0
    \end{equation*}
    por lo que $\{s_n\}\to x$ en $l_p$.
\end{proof}

\noindent
Recordemos que si $y\in \mathbb{R}^n$, todos los funcionales lineales y continuos tienen la pinta de un producto escalar:
\begin{equation*}
    \sum_{k=1}^{n}x(k)y(k)
\end{equation*}
Ahora, tomaremos $y\in l_{p^\ast}$ y definiremos $\Phi_y:l_p\to \mathbb{R}$ dada por:
\begin{equation*}
    \Phi_y(x) = \sum_{n=1}^{\infty}x(n)y(n) \qquad \forall x\in l_p
\end{equation*}
que está bien definida (la serie) es convergente, puesto que:
\begin{equation*}
    x\in l_p, y\in l_{p^\ast} \Longrightarrow xy \in l_1 \qquad \text{y}\qquad \|xy\|_1 \leq \|x\|_p \|y\|_{p^\ast}
\end{equation*}
Hay que comprobar que $\Phi_y$ es lineal, continua y preserva la norma.
\begin{itemize}
    \item Es fácil ver que es lineal.
    \item Para ver que $\Phi_y$ es continua, veamos que:
        \begin{equation*}
            |\Phi_y(x)| = \left|\sum_{n=1}^{\infty}x(n)y(n)\right| \leq \sum_{n=1}^{\infty}|x(n)||y(n)| \stackrel{(\ast)}{\leq} \|y\|_{p^\ast}\|x\|_p
        \end{equation*}
        donde en $(\ast)$ usamos la desigualdad de Hölder, con lo que $\Phi_y$ es continua. Además, $\|\Phi_y\| \leq \|y\|_{p^\ast}$.
    \item Para la otra desigualdad, distinguimos dos casos:
        \begin{itemize}
            \item Para $p=1$, tenemos que:
                \begin{equation*}
                    |y(n)| = |\Phi_y(e_n)| \leq \|\Phi_y\| \|e_n\|_1 = \|\Phi_y\| \qquad \forall n\in \mathbb{N}
                \end{equation*}
                de donde:
                \begin{equation*}
                    \|y\|_\infty = \sup_{n\in \mathbb{N}}|y(n)| \leq \|\Phi_y\|
                \end{equation*}
            \item Para $1<p<\infty$, tomamos:
                \begin{equation*}
                    x(n) = {(|y_n|)}^{p^\ast-2}y_n \qquad \forall n\in \mathbb{N}
                \end{equation*}
                y razonando como en el caso de dimensión finita se ve la desigualdad:
                \begin{equation*}
                    \|y\|_{p^\ast} \leq \|\Phi_y\|
                \end{equation*}
        \end{itemize}
\end{itemize}
Tenemos pues que 
\Func{\Phi}{l_{p^\ast}}{{(l_p)}^{\ast}}{y}{\Phi_y}
es lineal continua, y preserva la norma, es decir, es una isometría.


% // TODO: EJERCICIO: CALCULAR EL DUAL DE C0, que es l1
% // TODO: EJERCICIO: CALCULAR EL DUAL DE C, que es l1
