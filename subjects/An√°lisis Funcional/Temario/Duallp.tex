\section{Dual de $l_p$, para $1\leq p < \infty$}
Consideramos:
\begin{equation*}
    \mathbb{R}^{(\mathbb{N})} = \cc{L}(\{e_i : i \in \mathbb{N}\})
\end{equation*}
Supondremos siempre que:
\begin{equation*}
    \dfrac{1}{p} + \dfrac{1}{p^\ast} = 1, \qquad p^\ast = \infty \text{\ si\ } p = 1
\end{equation*}
Y trataremos de probar siempre que ${(l_p)}^{\ast} \cong l_{p^\ast}$.\\

\noindent
Veamos en primer lugar que
\begin{equation*}
    \overline{\mathbb{R}^{(\mathbb{N})}} = l_p \qquad \text{si\ } 1 \leq p < \infty
\end{equation*}
\begin{proof}
    Sea $x\in l_p$, definimos para cada $n\in \mathbb{N}$:
    \begin{equation*}
        s_n = \sum_{k=1}^{n} x(k)e_k
    \end{equation*}
    \begin{align*}
        s_1 &= (x(1), 0, \ldots) \\
        s_2 &= (x(1), x(2), 0,\ldots)
    \end{align*}
    Vemos primero que $s_n \in \mathbb{R}^{(\mathbb{N})}$ para cada $n\in \mathbb{N}$. Además, observamos que $s_n(k) = x(k)$ siempre que $k\leq n$, lo que nos dice que:
    \begin{equation*}
        {(\|x-s_n\|_p)}^{p} = \sum_{k=n+1}^{\infty} {(|x(k)|)}^{p} \to 0
    \end{equation*}
    por lo que $\{s_n\}\to x$ en $l_p$.
\end{proof}

\noindent
Recordemos que si $y\in \mathbb{R}^n$, todos los funcionales lineales y continuos tienen la pinta de un producto escalar:
\begin{equation*}
    \sum_{k=1}^{n}x(k)y(k)
\end{equation*}
Ahora, tomaremos $y\in l_{p^\ast}$ y definiremos $\Phi_y:l_p\to \mathbb{R}$ dada por:
\begin{equation*}
    \Phi_y(x) = \sum_{n=1}^{\infty}x(n)y(n) \qquad \forall x\in l_p
\end{equation*}
que está bien definida (la serie) es convergente, puesto que:
\begin{equation*}
    x\in l_p, y\in l_{p^\ast} \Longrightarrow xy \in l_1 \qquad \text{y}\qquad \|xy\|_1 \leq \|x\|_p \|y\|_{p^\ast}
\end{equation*}
Hay que comprobar que $\Phi_y$ es lineal, continua y preserva la norma.
\begin{itemize}
    \item Es fácil ver que es lineal.
    \item Para ver que $\Phi_y$ es continua, veamos que:
        \begin{equation*}
            |\Phi_y(x)| = \left|\sum_{n=1}^{\infty}x(n)y(n)\right| \leq \sum_{n=1}^{\infty}|x(n)||y(n)| \stackrel{(\ast)}{\leq} \|y\|_{p^\ast}\|x\|_p
        \end{equation*}
        donde en $(\ast)$ usamos la desigualdad de Hölder, con lo que $\Phi_y$ es continua. Además, $\|\Phi_y\| \leq \|y\|_{p^\ast}$.
    \item Para la otra desigualdad, distinguimos dos casos:
        \begin{itemize}
            \item Para $p=1$, tenemos que:
                \begin{equation*}
                    |y(n)| = |\Phi_y(e_n)| \leq \|\Phi_y\| \|e_n\|_1 = \|\Phi_y\| \qquad \forall n\in \mathbb{N}
                \end{equation*}
                de donde:
                \begin{equation*}
                    \|y\|_\infty = \sup_{n\in \mathbb{N}}|y(n)| \leq \|\Phi_y\|
                \end{equation*}
            \item Para $1<p<\infty$, tomamos:
                \begin{equation*}
                    x(n) = {(|y_n|)}^{p^\ast-2}y_n \qquad \forall n\in \mathbb{N}
                \end{equation*}
                y razonando como en el caso de dimensión finita se ve la desigualdad:
                \begin{equation*}
                    \|y\|_{p^\ast} \leq \|\Phi_y\|
                \end{equation*}
        \end{itemize}
\end{itemize}
Tenemos pues que 
\Func{\Phi}{l_{p^\ast}}{{(l_p)}^{\ast}}{y}{\Phi_y}
es lineal continua, y preserva la norma, es decir, es una isometría.

\begin{prop}
    Si $1\leq p < \infty$, entonces $\Phi$ es sobreyectiva.
    \begin{proof}
        Fijado $f\in l_p$, definimos:
        \begin{equation*}
            y(n) = f(e_n) \qquad \forall n\in \mathbb{N}
        \end{equation*}
        \begin{itemize}
            \item Si $p=1$, escribimos: 
                \begin{equation*}
                    |y(k)| = \alpha_k y(k), \qquad \alpha_k = \left\{\begin{array}{cl}
                        1 & \text{si\ } y(k)\geq 0 \\
                         -1& \text{si\ } y(k)<0
                    \end{array}\right. \quad \forall k\in \mathbb{N}
                \end{equation*}
                \begin{equation*}
                    |y(n)| = |f(e_n)| \leq \|f\|\|e_n\| = \|f\| \qquad \forall n\in \mathbb{N}
                \end{equation*}
                Por lo que la sucesión $y$ está acoatada, es decir, $y\in l_\infty$.
            \item Si $p>1$, fijado $n\in \mathbb{N}$, tenemos que:
                \begin{align*}
                    \sum_{k=1}^{n} {(|y(k)|)}^{p^\ast} &= \sum_{k=1}^{n} {(\alpha_k y(k))}^{p^\ast-1} = \sum_{k=1}^{n}\alpha_k {(|y(k)|)}^{p^\ast-1} f(e_k) = f\left(\sum_{k=1}^{n}\alpha_k {(|y(k)|)}^{p^\ast-1}e_k\right)  \\
                                                       &\leq \|f\|\left\|\sum_{k=1}^{n}\alpha_k {(|y(k)|)}^{p^\ast-1} e_k\right\|_p  = \|f\| {\left(\sum_{k=1}^{n} {(|y(k)|)}^{(p^\ast-1)p}\right)}^{\frac{1}{p}} \\
                                                       &= \|f\| {\left(\sum_{k=1}^{n}{(|y(k)|)}^{p^\ast}\right)}^{\frac{1}{p}}
                \end{align*}
                Es decir:
                \begin{equation*}
                    {\left(\sum_{k=1}^{n}|y(k)|\right)}^{p^\ast} \leq \|f\| {\left({\left(\sum_{k=1}^{n}|y(k)|\right)}^{p^\ast}\right)}^{\frac{1}{p}}
                \end{equation*}
                de donde:
                \begin{equation*}
                    \|f\| \geq {\left(\sum_{k=1}^{n}{|y(k|)}^{p^\ast}\right)}^{\frac{1}{p^\ast}} \qquad \forall n\in \mathbb{N}
                \end{equation*}
                por lo que $y\in l_{p^\ast}$. Tenemos que:
                \begin{equation*}
                    \Phi_y(e_n) = y(n) = f(e_n) \qquad \forall n\in \mathbb{N}
                \end{equation*}
                de donde $f(x) = \Phi_y(x)$ $\forall x\in \mathbb{R}^{(\mathbb{N})}$. Como es un conjunto denso, han de coincidir por continuidad en todo el espacio, con lo que:
                \begin{equation*}
                    f(x) = \Phi_y(x) \qquad \forall x\in l_p
                \end{equation*}
                de donde $\Phi_y = f$
        \end{itemize}
    \end{proof}
\end{prop}


% // TODO: EJERCICIO: CALCULAR EL DUAL DE C0, que es l1
% La misma prueba se puede adaptar para calcular (c0)*, viendo que es isométrico con l1
% Por Hanh-Banach, si tenemos un funcional de l_\infty, al restringirlo dará funcional de c0
% por lo que c0 C linfinito => (linfinito)* C (c0)*

% // TODO: EJERCICIO: CALCULAR EL DUAL DE C, que es l1

% // TODO: Ejercicio: Demostrar que c0 no es reflexivo
% // TODO: Ejercicio: Demostrar que lp 1<p<infty sí son reflexivos


