\chapter{El Espacio Dual}
El objetivo de este capítulo es definir el concepto de espacio dual de un espacio normado, así como sus principales propiedades, que nos dotan de muchos ejemplos de espacios de Banach. Para ello, será necesario primero repasar conceptos básicos vistos ya en asignaturas anteriores de Análisis Matemático.

\section{Repaso}
\begin{definicion}[Espacio métrico]
    Un espacio métrico es una tupla $(E, d)$ donde $E$ es un conjunto no vacío y $d:E\times E \to \mathbb{R}$ es una aplicación que verifica:
    \begin{itemize}
        \item \textbf{Desigualdad triangular.} $d(x,z) \leq d(x,y) + d(y,z) \qquad \forall x,y,z\in E$
        \item \textbf{Simetría.} $d(x,y) = d(y,x) \qquad \forall x,y\in E$
        \item \textbf{No degeneración.} $d(x,y) = 0 \Longleftrightarrow x = y$
    \end{itemize}
\end{definicion}

\begin{definicion}[Espacio normado]
    Un espacio normado es una tupla $(E, \|\cdot \|)$ donde $E$ es un espacio vectorial y $\|\cdot \|:E\to \mathbb{R}$ es una aplicación que verifica:
    \begin{itemize}
        \item \textbf{Desigualdad triangular.} $\|x + y\| \leq \|x\| + \|y\| \qquad \forall x,y\in E$
        \item \textbf{Homogeneidad por homotecia.} $\|\lm x\| = |\lm| \|x\| \qquad \forall \lm \in \mathbb{R}, \forall x\in E$
        \item \textbf{No degeneración.} $\|x\| = 0 \Longrightarrow x = 0$
    \end{itemize}
\end{definicion}

A partir de estas propiedades pueden deducirse muchas otras, entre las cuales destacamos:
\begin{prop}
    Si $(E, \|\cdot \|)$ es un espacio normado, entonces:
    \begin{itemize}
        \item $\|0\| = 0$.
        \item $\|x\| \geq 0 \qquad \forall x\in E$.
    \end{itemize}
    \begin{proof}
        Veamos cada propiedad:
        \begin{itemize}
            \item Para la primera: $\|0\| = \|0\cdot 1\| = 0\|1\| = 0$.
            \item Para la segunda, basta observar que si $x\in E$, entonces:
                \begin{equation*}
                    0 = \|0\| = \|x + (-x)\| \leq 2\|x\| \Longrightarrow \|x\| \geq 0
                \end{equation*}
        \end{itemize}
    \end{proof}
\end{prop}

\begin{prop}
    Si $(E,\|\cdot \|)$ es un espacio normado y definimos la aplicación $d:E\times E\to \mathbb{R}$ dada por:
    \begin{equation*}
        d(x,y) = \|y-x\| \qquad \forall x,y\in E
    \end{equation*}
    Se verifica que $(E, d)$ es un espacio métrico.
\end{prop}

\begin{definicion}[Espacio métrico completo]
    Sea $(E,d)$ un espacio métrico, decimos que es completo (o que la distancia $d$ es completa) si toda sucesión de Cauchy para la distancia $d$ es también convergente a un elemento de $E$ para la distancia $d$.
\end{definicion}

Hemos visto ya que cualquier espacio normado puede dotarse de estructura de espacio métrico, así como la definición de espacio métrico completo, ambos conceptos tratados ya en asignaturas previas.

\begin{definicion}[Espacio de Banach]
    Sea $(E,\|\cdot \|)$ un espacio normado, decimos que es de Banach si el espacio métrico $(E,d)$ obtenido de la forma usual a partir de la norma $\|\cdot \|$ es un espacio métrico completo.
\end{definicion}

\begin{definicion}[Espacio prehilbertiano]
    Un espacio prehilbertiano es una tupla $(E,\langle \cdot,\cdot   \rangle )$ donde $H$ es un espacio vectorial y $\langle \cdot ,\cdot  \rangle:E\times E\to \mathbb{R} $ es una aplicación que verifica:
    \begin{itemize}
        \item \textbf{Bilinealidad.} La aplicación $\langle \cdot ,\cdot  \rangle $ es lineal en ambas variables.
        \item \textbf{Simetría.} $\langle x,y \rangle  = \langle y,x \rangle \qquad \forall x,y\in E$
        \item \textbf{Definida positiva.} $\langle x,x \rangle >0 \qquad \forall x\in H\setminus \{0\}$
    \end{itemize}
\end{definicion}

\begin{prop}
    Si $(E, \langle \cdot ,\cdot  \rangle )$ es un espacio prehilbertiano y definimos la aplicación $\|\cdot \|:E\to \mathbb{R}$ dada por:
    \begin{equation*}
        \|x\| = \sqrt{\langle x,x \rangle } \qquad \forall x\in E
    \end{equation*}
    Se verifica que $(E,\|\cdot \|)$ es un espacio normado.
\end{prop}

\begin{definicion}[Espacio de Hilbert]
    Sea $(E,\langle \cdot ,\cdot  \rangle )$ un espacio prehilbertiano, decimos que es de Hilbert si el espacio normado $(E,\|\cdot \|)$ obtenido de la forma usual a partir del producto escalar $\langle \cdot ,\cdot  \rangle $ es un espacio métrico de Banach.
\end{definicion}

\subsection{Ejemplos}
\begin{itemize}
    \item Sea $N\in \mathbb{N}$, en $\mathbb{R}^N$ podemos definir para cada $p\geq 1$ la aplicación\newline $\|\cdot \|_p:\mathbb{R}^N\to \mathbb{R}$ dada por:
        \begin{equation*}
            \|x\|_p = {\left(\sum_{i=1}^{N}|x_i|^p\right)}^{\dfrac{1}{p}} \qquad \forall x\in \mathbb{R}^N
        \end{equation*}
        Que hace que $(\mathbb{R}^N, \|\cdot \|_p)$ sea un espacio normado, que de hecho es de Banach (hágase). % // TODO: Hacer
    \item En el caso anterior, si tomamos $p=2$ se verifica que además si definimos $\langle \cdot ,\cdot  \rangle :\mathbb{R}^N\times \mathbb{R}^N\to \mathbb{R}$ dada por:
        \begin{equation*}
            \langle x,y \rangle  = \sum_{i=1}^{N} x_iy_i \qquad \forall x,y\in \mathbb{R}^N
        \end{equation*}
        Obtenemos que $(\mathbb{R}^N,\langle \cdot ,\cdot  \rangle )$ es un espacio prehilbertiano (compruébese) cuyo espacio normado canónico coincide con $(\mathbb{R}^N,\|\cdot \|_2)$, por lo que es un espacio de Hilbert.
    \item Como otro ejemplo de espacio normado sobre $\mathbb{R}^N$, podemos definir $\|\cdot \|_\infty:\mathbb{R}^N\to \mathbb{R}$ dado por:
        \begin{equation*}
            \|x\|_\infty = \sup \{|x_i| : i \in \{0,\ldots, N\}\}
        \end{equation*}
        Se cumple igualmente que $(\mathbb{R}^N,\|\cdot \|_\infty)$ es un espacio normado que además es de Banach (compruébese).
    \item Como primer ejemplo de espacio normado que no se construye sobre los vectores de un espacio de la forma $\mathbb{R}^N$, si tomamos un conjunto $A\subset \mathbb{R}^N$, y definimos\footnote{El subíndice ``b'' de $\cc{C}_b(A)$ viene de la palabra inglesa ``bounded''.}:
        \begin{equation*}
            \cc{C}_b(A) = \{f:A\to \mathbb{R} : f \text{\ es continua y } f \text{\ es acotada en\ } A\}
        \end{equation*}
        Junto con la aplicación $\|\cdot \|:\cc{C}_b(A)\to \mathbb{R}$ dada por:
        \begin{equation*}
            \|f\| = \sup \{\|f(x)\| : x \in A\}
        \end{equation*}
        Se verifica que $(\cc{C}_b(A), \|\cdot \|)$ es una espacio normado que de hecho es de Banach (compruébese). % // TODO: Hacer
    \item Sea ahora $K\subset \mathbb{R}^N$ un compacto, si definimos:
        \begin{equation*}
            \cc{C}(K) = \{f:K\to \mathbb{R}  : f \text{\ es continua}\}
        \end{equation*}
        resulta que podemos definir una aplicación $\langle \cdot ,\cdot  \rangle:\cc{C}(K)\times \cc{C}(K)\to \mathbb{R} $ dada por:
        \begin{equation*}
            \langle f,g \rangle  = \int_K f(x)g(x)~dx
        \end{equation*}
        que hace que $(\cc{C}(K), \langle \cdot ,\cdot  \rangle )$ sea un espacio prehilbertiano, que nos induce un espacio normado donde la norma es:
        \begin{equation*}
            \|f\|_2 = {\left(\int_{K}{f(x)}^{2}~dx \right)}^{\frac{1}{2}} \qquad \forall f\in \cc{C}(K)
        \end{equation*}
        Sin embargo, este espacio prehilbertiano \textbf{no es de Hilbert}:

        Por ejemplo, si tomamos $K = [0,2]\subset \mathbb{R}$, si tomamos $f_n:K\to \mathbb{R}$ de forma que la gráfica de $f_n$ sea algo parecido a la de la Figura~\ref{fig:no_hilbert}

        \begin{figure}[H]
            \centering
            \begin{tikzpicture}[scale=3]
              % parámetro n (cambiar aquí)
              \def\n{8}
              \pgfmathsetmacro{\a}{1-1/\n} % punto 1 - 1/n

              % ejes
              \draw[-stealth] (0,0) -- (2.2,0) node[right] {$x$};
              \draw[-stealth] (0,-0.15) -- (0,1.3) node[above] {$y$};

              % marcas en eje x
              \foreach \x in {0,1,\a,2}{
                \draw (\x,0.01) -- (\x,-0.01);
              }
              \node[below] at (1,0) {$$};
              \node[below] at (2,0) {$2$};
              \node[below] at (\a,0) {$1-\tfrac{1}{n}$};

              % marcas en eje y
              \draw (0,1) -- (-0.01,1);
              \node[left] at (0,1) {$1$};
              \node[left] at (0,0) {$0$};

              % tramos de la función
              % tramo constante 0 en [0, 1-1/n]
              \draw[line width=1pt] (0,0) -- ({\a},0);

              % tramo lineal en [1-1/n, 1]
              \draw[line width=1pt] ({\a},0) -- (1,1);

              % tramo constante 1 en [1,2]
              \draw[line width=1pt] (1,1) -- (2,1);

              % puntos (para destacar continuidad a la derecha/izquierda)
              \filldraw ({\a},0) circle (0.6pt);
              \filldraw (1,1) circle (0.6pt);

              % etiquetas de los tramos
              \node[below] at ({\a/2},0) {$0$};
              \node[above right] at ({(\a+1)/2},{(1/2)}) {$\displaystyle f_n(x)=n x-(n-1)$};
              \node[above] at (1.5,1) {$1$};

              % leyenda con el valor de n
              \node[anchor=north west] at (0.02,-0.12) { $\displaystyle n=\,$\n };
            \end{tikzpicture}
            \caption{Gráfica de la función $f_n$.}
            \label{fig:no_hilbert}
        \end{figure}
        Si definimos $f=  \chi_{[1,2]}$ la función característica del intervalo $[1,2]$ (que no pertence a $\cc{C}(K)$), tenemos que:
        \begin{equation*}
            \|f-f_n\|_2^2 = \int_{0}^{2} {(f(x)- f_n(x))}^{2}~dx  = \dfrac{1}{2n} \to 0
        \end{equation*}
        Por lo que $f_n$ es una sucesión de Cauchy pero cuyo límite no está en el espacio que consideramos, por lo que no es convergente, luego $\cc{C}(K)$ no es un espacio completo. % // TODO: Afinar más este razonamiento.
\end{itemize}

\section{Espacios de Lebesgue}
Un ejemplo interesante de espacios de Banach son los espacios de Lebesgue, que ya se trabajaron un poco en la asignatura de Análisis Matemático II. En este documento volveremos a definir dicho espacio, puesto que la construcción es importante tenerla clara. En un primer lugar, hemos de repasar ciertas desigualdades para poder construir la estructura de espacio normado.

\subsection{Desigualdades importantes}
Para la primera desigualdad, es conveniente la siguiente motivación, que nos dará una breve justificación del origen de la desigualdad: sean $a,b\in \mathbb{R}^+_0$ dos números reales no negativos, es bien conocido que:
\begin{equation*}
    0 \geq {(a-b)}^{2} = a^2 + b^2 - 2ab \Longrightarrow ab \leq \dfrac{a^2}{2} + \dfrac{b^2}{2}
\end{equation*}

\begin{definicion}
    Sea $p\geq 1$ un número real, definimos su ``exponente conjugado'' por:
    \begin{equation*}
        p' = \left\{\begin{array}{ll}
                \frac{p}{p-1} & \text{si\ } p\neq 1 \\
                \infty & \text{si\ } p = 1
        \end{array}\right.
    \end{equation*}
    De esta forma (admitiendo el convenio de que $0 = \nicefrac{1}{\infty}$ de la recta real extendida), tenemos que:
    \begin{equation*}
        \dfrac{1}{p} + \dfrac{1}{p'} = 1
    \end{equation*}
\end{definicion}
\noindent
Usaremos en esta sección la notación $p'$ para denotar al exponente conjugado de $p$.

\begin{prop}[Desigualdad de Young]
    Sean $a,b\in \mathbb{R}^+_0$ y $p\in \mathbb{R}$ con $p>1$, se verifica que:
    \begin{equation*}
        ab \leq \dfrac{a^p}{p} + \dfrac{b^{p'}}{p'}
    \end{equation*}
    \begin{proof}
        La concavidad\footnote{Recordamos que si $f$ era una función cóncava, entonces $f(tx+(1-t)y) \geq tf(x)+(1-t)f(y)$, para cualquier $t\in [0,1]$, $x$,$y$ en el dominio de definición de $f$.} del logaritmo nos dice:
        \begin{equation*}
            \log\left(\dfrac{a^p}{p} + \dfrac{b^{p'}}{p'}\right) \geq \dfrac{1}{p}\log(a^p) + \dfrac{1}{p'}\log\left(b^{p'}\right) = \log(a) + \log(b) = \log(ab)
        \end{equation*}
        Y si ahora aplicamos la función exponencial y usamos que es creciente obtenemos:
        \begin{equation*}
            ab \leq \dfrac{a^p}{p} + \dfrac{b^{p'}}{p'}
        \end{equation*}
    \end{proof}
\end{prop}

\noindent
Recordemos que en Análisis Matemático I definíamos para cualquier conjunto $\Omega\subset \mathbb{R}$ medible el conjunto de las funciones integrables sobre $\Omega$:
\begin{equation*}
    \cc{L}(\Omega) = \left\{f:\Omega\to \mathbb{R} : \int_\Omega f < \infty\right\}
\end{equation*}
Pues bien, dado $p\geq 1$, podemos definir ahora:
\begin{equation*}
    \cc{L}_p(\Omega) = \left\{f\in \cc{L}(\Omega) : \int_\Omega |f|^p < \infty\right\}
\end{equation*}

\begin{teo}[Desigualdad de Hölder]
    Sea $p>1$, si $f\in \cc{L}_p(\Omega)$ y $g\in \cc{L}_{p'}(\Omega)$, entonces $fg\in \cc{L}(\Omega)$ y además:
    \begin{equation*}
        \int_\Omega |fg| \leq {\left(\int_\Omega |f|^p\right)}^{\frac{1}{p}}{\left(\int_\Omega |g|^{p'}\right)}^{\frac{1}{p'}}
    \end{equation*}
    \begin{proof}
        Si notamos por comodidad:
        \begin{equation*}
            \alpha = {\left(\int_\Omega |f|^p\right)}^{\frac{1}{p}},  \qquad \beta ={\left(\int_\Omega |g|^{p'}\right)}^{\frac{1}{p'}}
        \end{equation*}
        Si $\alpha = 0$, entonces $f^p = 0$ casi por doquier, de donde $|fg| = 0$ casi por doquier, luego:
        \begin{equation*}
            \int_\Omega|fg| = 0
        \end{equation*}
        Si $\beta = 0$ la situación es simétrica. Suponiendo ahora que $\alpha,\beta\in \mathbb{R}^+$, la desigualdad de Young nos dice que:
        \begin{equation*}
            \dfrac{|f(x)|}{\alpha} \dfrac{|g(x)|}{\beta}\leq \dfrac{|f(x)|^p}{p\alpha^p} + \dfrac{|g(x)|^{p'}}{p' \beta^{p'}} \qquad \forall x\in \Omega
        \end{equation*}
        Si ahora aplicamos la integral de Lebesgue a ambos lados usando el crecimiento de dicho funcional, obtenemos que (usando la definición de $\alpha$ y $\beta$):
        \begin{equation*}
            \dfrac{1}{\alpha\beta} \int_\Omega |fg| \leq \dfrac{1}{p\alpha^p} \int_\Omega |f|^p + \dfrac{1}{p'\beta^{p'}} \int_\Omega |g|^{p'} = \dfrac{1}{p} + \dfrac{1}{p'} = 1
        \end{equation*}
        de donde $fg\in \cc{L}(\Omega)$ y despejando de la desigualdad:
        \begin{equation*}
            \dfrac{1}{\alpha\beta} \int_\Omega |fg| \leq 1
        \end{equation*}
        Obtenemos la desigualdad buscada.
    \end{proof}
\end{teo}

\noindent
La desigualdad de Hölder nos proporcionará la desigualdad de Cauchy-Schwartz de la norma del futuro espacio normado, y nos permitirá probar la desigualdad de Minkowski.

\begin{teo}[Desigualdad de Minkowski]
    Para $p\in \mathbb{R}$ con $p\geq 1$ y $f,g\in \cc{L}_p(\Omega)$, se cumple que:
    \begin{equation*}
        {\left(\int_\Omega |f+g|^p\right)}^{\frac{1}{p}} \leq {\left(\int_\Omega |f|^p\right)}^{\frac{1}{p}} + {\left(\int_\Omega |g|^{p'}\right)}^{\frac{1}{p'}}
    \end{equation*}
    \begin{proof}
        Si notamos por comodidad:
        \begin{equation*}
            \alpha = {\left(\int_\Omega |f|^p\right)}^{\frac{1}{p}},  \qquad \beta ={\left(\int_\Omega |g|^{p'}\right)}^{\frac{1}{p'}}, \qquad \gamma = {\left(\int_\Omega |f+g|^p\right)}^{\frac{1}{p}} 
        \end{equation*}
        Si $p=1$, entonces la desigualdad triangular nos dice que $|f+g| \leq |f| + |g|$, donde aplicamos el crecimiento de la integral y ya tenemos el Teorema demostrado. Sabemos por el resultado anterior que $\gamma<\infty$, puesto que $\cc{L}_p(\Omega)\subset \cc{L}(\Omega)$, y la desigualdad buscada es obvia si $\gamma = 0$.  Supuesto ahora que $p>1$ y $\gamma>0$, si tomamos:
        \begin{equation*}
            h = |f+g|^{p-1}
        \end{equation*}
        tenemos entonces que:
        \begin{equation*}
            h^{p'} = |f+g|^{(p-1)p'} = |f+g|^p
        \end{equation*}
        luego:
        \begin{equation*}
            \int_\Omega h^{p'} = \gamma^p < \infty
        \end{equation*}
        Por lo que $h\in \cc{L}_p(\Omega)$. Tenemos:
        \begin{equation*}
            |f+g|^p = |f+g|h \leq |f| h + |g| h
        \end{equation*}
        Y por la desigualdad de Hölder:
        \begin{equation*}
            \gamma^p \leq \int_\Omega |f| h + \int_\Omega |g|h \leq (\alpha+\beta){\left(\int_\Omega h^{p'}\right)}^{\frac{1}{p'}} = (\alpha + \beta)\gamma^{\frac{p}{p'}}
        \end{equation*}
        Y si dividimos por $\gamma^{\frac{p}{p'}}$ tenemos la desigualdad buscada.
    \end{proof}
\end{teo}

\subsection{Definición de los espacios de Lebesgue}
Fijado $p\geq 1$, podemos tratar de dotar a $\cc{L}_p(\Omega)$ de una norma. Pensamos en un principio en la aplicación $\varphi_p:\cc{L}_p(\Omega)\to \mathbb{R}$ dada por:
\begin{equation*}
    \varphi_p(f) = {\left(\int_\Omega |f|^p\right)}^{\frac{1}{p}} \qquad \forall f\in \cc{L}_p(\Omega)
\end{equation*}
Que:
\begin{itemize}
    \item Verifica la desigualdad triangular gracias a la desigualdad de Minkowski.
    \item Verifica la homegeneidad por homotecias, ya que:
        \begin{equation*}
            \varphi_p(\alpha f) = |\alpha| \varphi_p(f) \qquad \forall \alpha\in \mathbb{R}
        \end{equation*}
    \item $\varphi_p(f) = 0 \Longleftrightarrow f = 0$ casi por doquier.
\end{itemize}
Por lo que dicha función \textbf{no es una norma} en $\cc{L}_p(\Omega)$ al no verificar la no degeneración de la norma, puesto que la integral ``es ciega'' a la hora de diferenciar la función constantemente igual a 0 de otras funciones con integral cero.\\

\noindent
Para solucionar el problema con el que nos acabamos de topar (el problema de no poder definir una norma de dicha forma), podemos constuir una relación de equivalencia $\sim$ en $\cc{L}_p(\Omega)$ que identifique a las funciones que son iguales casi por doquier, pudiendo considerar el espacio cociente:
\begin{equation*}
    L_p(\Omega)= \dfrac{\cc{L}_p(\Omega)}{\sim}
\end{equation*}
Donde ya $(L_p(\Omega), \varphi_p)$ sí que es un espacio normado, donde denotaremos normalmente $\varphi_p = \|\cdot \|_p$.

\begin{teo}[Riesz-Fischer]
    Sea $\Omega\subset \mathbb{R}^N$ un conjunto medible y $p\geq 1$, se cumple que $(L_p(\Omega), \|\cdot \|_p)$ es un espacio de Banach.
\end{teo}

\subsection{Más ejemplos de espacios de Banach}
\begin{itemize}
    \item Sea $\Omega\subset \mathbb{R}$ un conjunto medible, si definimos:
        \begin{equation*}
            \sup_\Omega |f| = \inf\{M\geq 0 : |f(x)| \leq M \text{\ casi para todo\ } x\in \Omega\}
        \end{equation*}
        El conjunto:
        \begin{equation*}
            \cc{L}^\infty(\Omega) = \left\{f:\Omega\to \mathbb{R} : f \text{\ es medible y\ } \sup_\Omega |f| < \infty\right\}
        \end{equation*}
        junto con la norma:
        \begin{equation*}
            \|f\|_\infty = \sup_\Omega |f|
        \end{equation*}
        es un espacio de Banach, donde la desigualdad de Hölder se comple considerando que $p=\infty$ y $p'=1$:

        Si $f\in \cc{L}^\infty(\Omega)$ y $g\in \cc{L}(\Omega)$, entonces $fg\in \cc{L}(\Omega)$, con:
        \begin{equation*}
            \|fg\|_1 \leq \|f\|_\infty \|g\|_1
        \end{equation*}
    \item Para $1\leq p < \infty$ podemos considerar otro tipo de espacios:
        \begin{equation*}
            l^p = \left\{x:\mathbb{N}\to \mathbb{R} : \sum_{n=1}^{\infty}|x(n)|^p < \infty\right\}
        \end{equation*}
        que junto con la aplicación:
        \begin{equation*}
            \|x\|_p = {\left(\sum_{n=1}^{\infty}|x(n)|^p\right)}^{\frac{1}{p}} \qquad \forall x\in l^p
        \end{equation*}
        forman un espacio de Banach (compruébese).

        En dichos espacios, se tiene que si $x\in l^p$ y $y\in l^{p'}$, entonces $xy\in l$, con:
        \begin{equation*}
            \|xy\| \leq \|x\|_p \|y\|_{p'}
        \end{equation*}
    \item En el caso anterior, si $p=2$, podemos definir la aplicación:
        \begin{equation*}
            \langle x,y \rangle_2 = \sum_{n=1}^{\infty}x(n)y(n) \qquad \forall x,y\in l^2
        \end{equation*}
        Con lo que $(l^2, \langle \cdot ,\cdot  \rangle _2)$ es un espacio de Hilbert.
    \item Al igual que sucedía con las normas $p-$ésimas en $\mathbb{R}^N$, podemos considerar:
        \begin{equation*}
            l^\infty = \{x:\mathbb{N}\to \mathbb{R} : x \text{\ acotada}\}
        \end{equation*}
        junto con la aplicación $\|\cdot \|:l^{\infty}\to \mathbb{R}$ dada por:
        \begin{equation*}
            \|x\|_\infty = \sup\{|x(n)| : n\in \mathbb{N}\}
        \end{equation*}
        y obtenemos un espacio de Banach.
    \item $C = \{x:\mathbb{N}\to \mathbb{R} : x \text{\ es convergente}\}$ es un subespacio de $l^\infty$.
    \item $C_0 = \{x:\mathbb{N}\to \mathbb{R} : x \text{\ converge a\ }0\}$ es un subespacio de $C$.
\end{itemize}

\section{Espacio dual}
Para introducir la nocíon de espacio dual, nos será necesario primero destacar unos resultados:
\begin{prop}
    Si $H$ es un espacio prehilbertiano, entonces:
    \begin{enumerate}
        \item Se cumple la desigualdad de Cauchy-Schwartz:
            \begin{equation*}
                |\langle u,v \rangle | \leq \|u\|\|v\| \qquad \forall u,v\in H
            \end{equation*}
        \item Se cumple la identidad del paralelogramo:
            \begin{equation*}
                \left\|\dfrac{u+v}{2}\right\| + \left\| \dfrac{u-v}{2}\right\| = \dfrac{1}{2}(\|u\|^2 + \|v\|^2) \qquad \forall u,v\in H
            \end{equation*}
    \end{enumerate}
\end{prop}

\begin{teo}[de la Proyección]
    Sea $H$ un espacio de Hilbert, sea $\emptyset \neq K \subset H$ un conjunto convexo y cerrado, entonces $\forall f\in H~\exists_1 u\in K$ de forma que:
    \begin{equation*}
        \|f-u\| = d(f,K) = \inf \{d(f,v) : v\in K\}
    \end{equation*}
    Además, dicho elemento $u$ está caracterizado por:
    \begin{itemize}
        \item $u\in K$.
        \item $\langle f-u,v-u \rangle \leq 0 \qquad \forall v\in K$.
    \end{itemize}
    Por tanto, a dicho único elemento $u$ lo notaremos por $P_Kf$.
    \begin{proof}
        Como $0\leq d(f,v) \quad \forall v\in K$, tenemos entonces que dicho ínfimo existe. Tenemos por tanto que existe $\{v_n\}$ una sucesión de elementos de $K$ de forma que $\{d(f,v_n)\}\to d(f,K)$. Sean $n,m\in \mathbb{N}$ y usando la identidad del paralelogramo con $f-v_n$ y $f-v_m$, tenemos:
        \begin{gather*}
            \left\| \frac{f-v_n + f-v_m}{2} \right\|^2 + \left\| \frac{f-v_n-(f-v_m)}{2} \right\|^2 = \frac{1}{2}\left(\|f-v_n\|^2 + \|f-v_m\|^2\right)\\
            \left\| f - \frac{v_n+v_m}{2} \right\|^2 + \left\| \frac{v_m - v_n}{2} \right\|^2 = \frac{1}{2}\left(\|f-v_n\|^2 + \|f-v_m\|^2\right)\\
            \frac{\left\| v_m-v_n \right\|^2}{4} = \frac{1}{2}\left(\|f-v_n\|^2 + \|f-v_m\|^2\right) - \left\| f - \frac{v_n+v_m}{2} \right\|^2\\
            \left\| v_m-v_n \right\|^2 = 2\left(\|f-v_n\|^2 + \|f-v_m\|^2\right) - 4\left\| f - \frac{v_n+v_m}{2} \right\|^2
        \end{gather*}
        Como $K$ es convexo, tenemos que $\frac{v_n+v_m}{2}\in K$, por lo que:
        \begin{equation*}
            \left\| f-\dfrac{v_n+v_m}{2}\right\| \geq d(f,K)
        \end{equation*}
        Por lo que:
        \begin{equation*}
            0 \leq \|v_m - v_n\|^2 \leq 2(\|f-v_n\|^2 + \|f-v_m\|^2) - 4d(f,K)^2
        \end{equation*}
        Como $\{\|f-v_n\|^2\} \to d(f,K)^2$ y $\{\|f-v_m\|^2\}\to d(f,K)^2$, tenemos por el Lema del Sandwitch que $\{\|v_n - v_m\|^2\}\to 0$, por lo que $\{v_n\}$ es de Cauchy. Como $H$ es completo, existe $u\in H$ de forma que $\{v_n\}\to u$, pero por ser $K$ cerrado tendremos que $u\in K$.

        Como $\{v_n\}\to u$, tenemos entonces que $\{d(f,v_n)\}\to d(f,v)$, pero $\{d(f,v_n)\}$ convergía también a $d(f,K)$. No queda más salida que $d(f,v) = d(f,K)$.\\

        \noindent
        Una vez probada la existencia de $u$, veamos que:
        \begin{equation*}
            u\in K \text{\ con\ } \|f-u\| = d(f,K) \Longleftrightarrow u\in K \text{\ y\ } \langle f-u,v-u \rangle \leq 0 \quad \forall v\in K
        \end{equation*}
        \begin{description}
            \item [$\Longrightarrow)$] Supongamos que $u\in K$ y sabemos que $\|f-u\|\leq \|f-v\|$ para todo $v\in K$. Tomamos ahora $w\in K$ y consideramos el segmento que une $u$ con $w$. Entonces $\forall w\in K$ y $\forall t \in [0,1]$, al ser $K$ convexo tendremos que
            \begin{gather*}
                (1-t)u + tw \in K\ \  \text{ y }\ \ \|f-u\|^2 \leq \|f-(1-t)u-tw\|^2
            \end{gather*}
            Aplicando la bilinealidad podemos reescribir esta última expresión como 
            \begin{align*}
                \|f-(1-t)u-tw\|^2 &= \langle f-(1-t)u-tw,f-(1-t)u-tw \rangle  =\\
                &=\|f-u\|^2 + t^2\|w-u\|^2-2t(f-u,w-u)
            \end{align*}
            Sustituyendo en la expresión que teníamos anteriormente nos queda que:
            \begin{gather*}
                0\leq t^2\|w-u\|^2-2t\langle f-u,w-u \rangle  \ \ \ \forall t \in (0,1]
            \end{gather*}
            Al dividir entre $t$ nos queda
            \begin{gather*}
                0\leq t\|w-u\|^2-2\langle f-u,w-u \rangle  \ \ \ \forall t \in (0,1]
            \end{gather*}
            y tomando ahora el límite  cuando $t$ tiende a $0$ por la derecha queda que
            \begin{gather*}
                0\leq -2\langle f-u,w-u \rangle  \Rightarrow \langle f-u,w-u \rangle  \leq 0 \qquad \forall w\in K
            \end{gather*}
            \item [$\Longleftarrow)$] 
                \begin{equation*}
                    \|f-v\|^2 = \|f-u+u-v\|^2 = \|f-u\|^2 + 2\langle f-u,u-v \rangle  + \|u-v\|^2 \qquad \forall v\in K
                \end{equation*}
                De donde:
                \begin{equation*}
                    0\geq 2\langle f-u,v-u \rangle  - \|u-v\|^2 = \|f-u\|^2 - \|f-v\|^2
                \end{equation*}
                Luego:
                \begin{equation*}
                    \|f-u\|^2 \leq \|f-v\|^2 \qquad \forall v\in K
                \end{equation*}
        \end{description}
        Para probar finalmente la unicidad, supongamos que existen $u,w\in K$ de forma que:
        \begin{equation*}
            \langle f-u,v-u \rangle ,\langle f-w,v-w \rangle \leq 0 \qquad \forall v\in K
        \end{equation*}
        Entonces:
        \begin{equation*}
            \langle f-u,w-u \rangle , \langle f-w,u-w \rangle = \langle u-f,w-u \rangle  \leq 0
        \end{equation*}
        Por lo que:
        \begin{equation*}
            \langle f-u,w-u \rangle  + \langle w-f, w-u \rangle  = \langle w-u, w-u \rangle  \leq 0
        \end{equation*}
        de donde $\langle w-u,w-u \rangle = 0$, por lo que $\|w-u\|^2 = d(w,u)^2 = 0$, luego $w=u$.
    \end{proof}
\end{teo}

\begin{prop}
    Dado $\emptyset  \neq K\subset H$ un conjunto convexo y cerrado, tenemos que la aplicación
    \Func{P_K}{H}{H}{f}{P_Kf}
    es lipschitziana. De hecho:
    \begin{equation*}
        \|P_Kf_1 - P_Kf_2\| \leq \|f_1 - f_2\| \qquad \forall f_1,f_2\in H
    \end{equation*}
    \begin{proof}
        Sean $f_1, f_2\in H$, $u_1 = P_Kf_1$, $u_2=P_Kf_2$, estos verifican:
        \begin{equation*}
            \langle f_1-u_1, v-u_1 \rangle , \langle f_2-u_2,v-u_2 \rangle  \leq 0 \qquad \forall v\in K
        \end{equation*}
        Por lo que:
        \begin{gather*}
            \langle f_1 - u_1, u_2 - u_1 \rangle  \leq 0 \\
            \langle f_2 - u_2, u_1 - u_2 \rangle \leq 0 \Longrightarrow \langle f_2 - u_2, u_2 - u_1 \rangle \geq 0
        \end{gather*}
        De donde $\langle f_1 - u_2 - f_2 + u_2, u_2 - u_1 \rangle \leq 0$, por lo que:
        \begin{equation*}
            \langle f_1 - f_2 + (u_2 - u_1), (u_2 - u_1) \rangle  = \langle f_1 - f_2, u_2 - u_1 \rangle  + \langle u_2 - u_1, u_2 - u_1 \rangle 
        \end{equation*}
        Luego:
        \begin{equation*}
            \|u_2 - u_1\|^2 = \langle u_2 - u_1, u_2 - u_1 \rangle  \leq - \langle f_1 - f_2, u_2 - u_1 \rangle \stackrel{\text{Cauchy-Schwartz}}{\leq} \|f_1 - f_2\| \|u_2 - u_1\|
        \end{equation*}
        Por lo que:
        \begin{equation*}
            \|u_2 - u_1\| \leq \|f_1 - f_2\|
        \end{equation*}
        Si $\|u_2 - u_1\| \neq 0$, cierto también si $\|u_2 - u_1\| = 0$.
    \end{proof}
\end{prop}

\noindent
Pensemos ahora en un ejemplo de conjuntos convexos con propiedades interesantes, como lo son los espacios vectoriales:

\begin{coro}[Proyección Ortogonal]
    Sea $M\subset H$ un subespacio vectorial cerrado de $H$, un espacio de Hilbert, entonces:
    \begin{equation*}
        \forall f\in H~\exists _1 u \in M \text{\ tal que\ } \|f-u\| = d(f,M)
    \end{equation*}
    Ademś, la caracterización de $u$ puede mejorarse por:
    \begin{equation*}
        u\in M \qquad \text{y} \qquad \langle f-u,w \rangle  = 0 \quad \forall w\in M
    \end{equation*}
\end{coro}

% // TODO: Seguir por el corolario de la proyeccion ortogonal
% AQUI FALTA COPIAR DE LIBRETA
\newpage
% // TODO: Nueva clase

\section{Teorema de Hanh Banach} % //TODO: Revisar nombre
\noindent
El siguiente Teorema tiene la utildad de probar que $E^\ast \neq \{0\}$ para $E$ un espacio normado. En dimensión finita podemos pensarlo, pero el problema es en dimensión infinita. El problema es el siguiente:\\

\noindent
Sea $E$ un espacio de Banach, $G\subset E$ un subespacio suyo y $g:G\to \mathbb{R}$ lineal y continua, ¿podemos garantizar entoences que existe $f:E\to \mathbb{R}$ lineal y continua tal que $f\big|_G = g$?

Que $g$ sea continua significa que $\exists k$ de forma que $|g(x)| \leq k\|x\|$ $\forall x\in G$. Para resolver el problema, necesitamos encontrar $k'$ de forma que:
\begin{equation*}
    |f(x)| \leq k'\|x\| \qquad \forall x\in E
\end{equation*}

\begin{ejercicio} % // TODO: HACER
    Sea $p:(E,\|\cdot \|)\to \mathbb{R}$ dada por:
    \begin{equation*}
        p(x) = k\|x\| \qquad \forall x\in E
    \end{equation*}
    La función $p$ verifica:
    \begin{itemize}
        \item $p(x+y) \leq p(x) + p(y) \qquad \forall x,y\in E$.
        \item $p(\lm x) = \lm p(x) \qquad \forall \lm \in \mathbb{R}^+, \forall x\in E$.
    \end{itemize}
    \begin{proof}
        Sean $x,y\exists E$ y $\lm\in \mathbb{R}^+$:
        \begin{gather*}
            p(x+y)= k\|x+y\| \leq k(\|x\| + \|y\|) = k\|x\| + k\|y\| = p(x) + p(y) \\
            p(\lm x) = k\|\lm x\| = \lm k \|x\| = \lm p(x)
        \end{gather*}
    \end{proof}
\end{ejercicio}

\noindent
El siguitente Teorema es equivalente al axioma de elección. Para realizar la demsotración es conveniente primero repasar el Lema de Zorn:

\subsection{Lema de Zorn}
\begin{definicion}
    Sea $\emptyset \neq P$ un conjunto con una relación $\leq$ de orden, es decir, una relación reflexiva, antisimétrica y transitiva.
    \begin{itemize}
        \item Un subconjunto $Q\subset P$ es totalmente ordenado si:
            \begin{equation*}
                \forall a,b\in Q \Longrightarrow a\leq b \lor b\leq a
            \end{equation*}
        \item Si $Q\subset P$ y $x\in P$, decimos que $x$ es cota superior de $Q$ si y solo si:
            \begin{equation*}
                a\leq x \qquad \forall a\in Q
            \end{equation*}
        \item Si $m\in P$, decimos que $m$ es un elemento maximal de $P$ si y solo si:
            \begin{equation*}
                \{x\in P : m \leq x\} = \{m\}
            \end{equation*}
        \item Diremos que $P$ es inductivo si todo subconjunto $Q\subset P$ que sea totalmente ordenado posee una cota superior.
    \end{itemize}
\end{definicion}

\begin{lema}[de Zorn]
    Si $P$ es un conjunto no vacío con una relación de orden $\leq$ y $P$  es inductivo, entonces $P$ tiene un elemento maximal.
\end{lema}

\subsection{Teorema}

\begin{teo}[Hanh-Banach, versión analítica]
    Sea $E$ un espacio vectorial, sea $p:E\to \mathbb{R}$ tal que: 
    \begin{itemize}
        \item $p(x+y) \leq p(x) + p(y) \qquad \forall x,y\in E$.
        \item $p(\lm x) = \lm p(x) \qquad \forall \lm \in \mathbb{R}^+ ,\forall x\in E$.
    \end{itemize}
    Sea $G\subset E$ un subespacio vectorial y $g:G\to \mathbb{R}$ una aplicación lineal verificando:
    \begin{equation*}
        g(x) \leq p(x) \qquad \forall x\in G
    \end{equation*}
    Entonces, $\exists f:E\to \mathbb{R}$ lineal verificando: 
    \begin{enumerate}
        \item $f(x) \leq p(x) \qquad \forall x\in E$.
        \item $f\big|_G = g$.
    \end{enumerate}
    \begin{proof}
        Sea:
        \begin{equation*}
            P = \left\{h:D(h) \to \mathbb{R} : \left\{\begin{array}{l}
                G\subset D(h) \text{\ subespacio vectorial de\ } E \\
                h \text{\ lineal y\ } h(x) \leq p(x) \quad \forall x\in D(h) \\
                h(x) = g(x) \quad \forall x\in G
            \end{array}\right.\right\}
        \end{equation*}
        es decir, ``el conjunto de extensiones de $g$'' a cualquier otro subespacio más grande que contenga a $G$, buscamos aplicar el Lema de Zorn sobre $D(h)$, buscando $E$ como elemento maximal.\\

        \noindent
        Como $g\in P$, $P\neq \emptyset $. Definiremos una relación de orden en $P$ por:
        \begin{equation*}
            h_1 \leq h_2 \Longleftrightarrow \left\{\begin{array}{l}
                D(h_1) \subset D(h_2) \\
                h_2\big|_{D(h_1)} = h_1
            \end{array}\right. \qquad \forall h_1,h_2\in P
        \end{equation*}
        es decir, $h_1\leq h_2$ si $h_2$ es una extensión de $h_1$. Es fácil ver que esta relación es una relación de orden en $P$ (que no es total).\\ % // TODO: Hacer

        \noindent
        Tratemos de probar que $P$ es inductivo. Para ello, sea $Q\subset P$ totalmente ordenado, para buscar una cota superior, consideramos:
        \begin{equation*}
            V_0 = \bigcup_{h\in Q}D(h)
        \end{equation*}
        Se verifica que $V_0$ es un espacio vectorial (comprobar). % // TODO: Hacer, consecuencia Q totalmente ordenado
        y definimos $h_0:V_0 \to \mathbb{R}$ dada por:
        \begin{equation*}
            h_0(x) = h(x) \quad \text{si\ } x\in D(h)
        \end{equation*}
        que está bien definida, ya que si $x\in D(h_1)\cap D(h_2)$, sucederá bien $h_1 \leq h_2$ bien $h_2 \leq h_1$, luego suponiendo que $h_1\leq h_2$, tendremos que $h_2\big|_{D(h_1)} = h_1$, luego $h_1(x) = h_2(x)$.

        Se cumple además que $h$ es lineal (hacer). % // TODO: Hacer
        Así como que: % // TODO: Hacer
        \begin{equation*}
            h_0(x) \leq p(x) \qquad \forall x\in V_0
        \end{equation*}
        Tenemos:
        \begin{itemize}
            \item $G\subset D(h)$ para toda $h\in P$, luego $G\subset V_0$.
            \item $h$ es lineal.
            \item $h$ verifica $h(x) \leq p(x)$.
            \item $h$ es una extensión de $g$.
        \end{itemize}
        Por tanto, $h\in P$ y cualquier $h\in Q$ verifica que $h\leq h_0$. Por tanto, $h_0$ es cota superior de $P$, luego $P$ es inductivo. Por el Lema de Zorn, existe $f\in P$ que es un elemento maximal de $P$. Por estar en $P$,  $f:D(f)\to \mathbb{R}$ verifica:
        \begin{itemize}
            \item $G\subset D(f)\subset E$ como subespacios vectoriales.
            \item $f$ es lineal y $f(x) \leq p(x) \quad  \forall x\in D(f)$.
            \item $f\big|_G = g$.
        \end{itemize}
        Falta demostrar que $f$ maximal $\Longrightarrow D(f) = E$. Para ello, si $D(f)\neq E$, entonces % // TODO: Terminar, coger un punto de fuera lin. indep y sumar la dimension a D(f)
    \end{proof}
\end{teo}

