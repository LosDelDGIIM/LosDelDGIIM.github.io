\chapter{El Espacio Dual}
El objetivo de este capítulo es definir el concepto de espacio dual de un espacio normado, así como sus principales propiedades, que nos dotan de muchos ejemplos de espacios de Banach. Para ello, será necesario primero repasar conceptos básicos vistos ya en asignaturas anteriores de Análisis Matemático.

\section{Repaso}
\begin{definicion}[Espacio métrico]
    Un espacio métrico es una tupla $(E, d)$ donde $E$ es un conjunto no vacío y $d:E\times E \to \mathbb{R}$ es una aplicación que verifica:
    \begin{itemize}
        \item \textbf{Desigualdad triangular.} $d(x,z) \leq d(x,y) + d(y,z) \qquad \forall x,y,z\in E$
        \item \textbf{Simetría.} $d(x,y) = d(y,x) \qquad \forall x,y\in E$
        \item \textbf{No degeneración.} $d(x,y) = 0 \Longleftrightarrow x = y$
    \end{itemize}
\end{definicion}

\begin{definicion}[Espacio normado]
    Un espacio normado es una tupla $(E, \|\cdot \|)$ donde $E$ es un espacio vectorial y $\|\cdot \|:E\to \mathbb{R}$ es una aplicación que verifica:
    \begin{itemize}
        \item \textbf{Desigualdad triangular.} $\|x + y\| \leq \|x\| + \|y\| \qquad \forall x,y\in E$
        \item \textbf{Homogeneidad por homotecia.} $\|\lm x\| = |\lm| \|x\| \qquad \forall \lm \in \mathbb{R}, \forall x\in E$
        \item \textbf{No degeneración.} $\|x\| = 0 \Longrightarrow x = 0$
    \end{itemize}
\end{definicion}

A partir de estas propiedades pueden deducirse muchas otras, entre las cuales destacamos:
\begin{prop}
    Si $(E, \|\cdot \|)$ es un espacio normado, entonces:
    \begin{itemize}
        \item $\|0\| = 0$.
        \item $\|x\| \geq 0 \qquad \forall x\in E$.
    \end{itemize}
    \begin{proof}
        Veamos cada propiedad:
        \begin{itemize}
            \item Para la primera: $\|0\| = \|0\cdot 1\| = 0\|1\| = 0$.
            \item Para la segunda, basta observar que si $x\in E$, entonces:
                \begin{equation*}
                    0 = \|0\| = \|x + (-x)\| \leq 2\|x\| \Longrightarrow \|x\| \geq 0
                \end{equation*}
        \end{itemize}
    \end{proof}
\end{prop}

\begin{prop}
    Si $(E,\|\cdot \|)$ es un espacio normado y definimos la aplicación $d:E\times E\to \mathbb{R}$ dada por:
    \begin{equation*}
        d(x,y) = \|y-x\| \qquad \forall x,y\in E
    \end{equation*}
    Se verifica que $(E, d)$ es un espacio métrico.
\end{prop}

\begin{definicion}[Espacio métrico completo]
    Sea $(E,d)$ un espacio métrico, decimos que es completo (o que la distancia $d$ es completa) si toda sucesión de Cauchy para la distancia $d$ es también convergente a un elemento de $E$ para la distancia $d$.
\end{definicion}

Hemos visto ya que cualquier espacio normado puede dotarse de estructura de espacio métrico, así como la definición de espacio métrico completo, ambos conceptos tratados ya en asignaturas previas.

\begin{definicion}[Espacio de Banach]
    Sea $(E,\|\cdot \|)$ un espacio normado, decimos que es de Banach si el espacio métrico $(E,d)$ obtenido de la forma usual a partir de la norma $\|\cdot \|$ es un espacio métrico completo.
\end{definicion}

\begin{definicion}[Espacio prehilbertiano]
    Un espacio prehilbertiano es una tupla $(E,\langle \cdot,\cdot   \rangle )$ donde $H$ es un espacio vectorial y $\langle \cdot ,\cdot  \rangle:E\times E\to \mathbb{R} $ es una aplicación que verifica:
    \begin{itemize}
        \item \textbf{Bilinealidad.} La aplicación $\langle \cdot ,\cdot  \rangle $ es lineal en ambas variables.
        \item \textbf{Simetría.} $\langle x,y \rangle  = \langle y,x \rangle \qquad \forall x,y\in E$
        \item \textbf{Definida positiva.} $\langle x,x \rangle >0 \qquad \forall x\in H\setminus \{0\}$
    \end{itemize}
\end{definicion}

\begin{prop}
    Si $(E, \langle \cdot ,\cdot  \rangle )$ es un espacio prehilbertiano y definimos la aplicación $\|\cdot \|:E\to \mathbb{R}$ dada por:
    \begin{equation*}
        \|x\| = \sqrt{\langle x,x \rangle } \qquad \forall x\in E
    \end{equation*}
    Se verifica que $(E,\|\cdot \|)$ es un espacio normado.
\end{prop}

\begin{definicion}[Espacio de Hilbert]
    Sea $(E,\langle \cdot ,\cdot  \rangle )$ un espacio prehilbertiano, decimos que es de Hilbert si el espacio normado $(E,\|\cdot \|)$ obtenido de la forma usual a partir del producto escalar $\langle \cdot ,\cdot  \rangle $ es un espacio métrico de Banach.
\end{definicion}

\subsection{Ejemplos}
\begin{itemize}
    \item Sea $N\in \mathbb{N}$, en $\mathbb{R}^N$ podemos definir para cada $p\geq 1$ la aplicación\newline $\|\cdot \|_p:\mathbb{R}^N\to \mathbb{R}$ dada por:
        \begin{equation*}
            \|x\|_p = {\left(\sum_{i=1}^{N}|x_i|^p\right)}^{\dfrac{1}{p}} \qquad \forall x\in \mathbb{R}^N
        \end{equation*}
        Que hace que $(\mathbb{R}^N, \|\cdot \|_p)$ sea un espacio normado, que de hecho es de Banach (hágase). % // TODO: Hacer
    \item En el caso anterior, si tomamos $p=2$ se verifica que además si definimos $\langle \cdot ,\cdot  \rangle :\mathbb{R}^N\times \mathbb{R}^N\to \mathbb{R}$ dada por:
        \begin{equation*}
            \langle x,y \rangle  = \sum_{i=1}^{N} x_iy_i \qquad \forall x,y\in \mathbb{R}^N
        \end{equation*}
        Obtenemos que $(\mathbb{R}^N,\langle \cdot ,\cdot  \rangle )$ es un espacio prehilbertiano (compruébese) cuyo espacio normado canónico coincide con $(\mathbb{R}^N,\|\cdot \|_2)$, por lo que es un espacio de Hilbert.
    \item Como otro ejemplo de espacio normado sobre $\mathbb{R}^N$, podemos definir $\|\cdot \|_\infty:\mathbb{R}^N\to \mathbb{R}$ dado por:
        \begin{equation*}
            \|x\|_\infty = \sup \{|x_i| : i \in \{0,\ldots, N\}\}
        \end{equation*}
        Se cumple igualmente que $(\mathbb{R}^N,\|\cdot \|_\infty)$ es un espacio normado que además es de Banach (compruébese).
    \item Como primer ejemplo de espacio normado que no se construye sobre los vectores de un espacio de la forma $\mathbb{R}^N$, si tomamos un conjunto $A\subset \mathbb{R}^N$, y definimos\footnote{El subíndice ``b'' de $\cc{C}_b(A)$ viene de la palabra inglesa ``bounded''.}:
        \begin{equation*}
            \cc{C}_b(A) = \{f:A\to \mathbb{R} : f \text{\ es continua y } f \text{\ es acotada en\ } A\}
        \end{equation*}
        Junto con la aplicación $\|\cdot \|:\cc{C}_b(A)\to \mathbb{R}$ dada por:
        \begin{equation*}
            \|f\| = \sup \{\|f(x)\| : x \in A\}
        \end{equation*}
        Se verifica que $(\cc{C}_b(A), \|\cdot \|)$ es una espacio normado que de hecho es de Banach (compruébese). % // TODO: Hacer
    \item Sea ahora $K\subset \mathbb{R}^N$ un compacto, si definimos:
        \begin{equation*}
            \cc{C}(K) = \{f:K\to \mathbb{R}  : f \text{\ es continua}\}
        \end{equation*}
        resulta que podemos definir una aplicación $\langle \cdot ,\cdot  \rangle:\cc{C}(K)\times \cc{C}(K)\to \mathbb{R} $ dada por:
        \begin{equation*}
            \langle f,g \rangle  = \int_K f(x)g(x)~dx
        \end{equation*}
        que hace que $(\cc{C}(K), \langle \cdot ,\cdot  \rangle )$ sea un espacio prehilbertiano, que nos induce un espacio normado donde la norma es:
        \begin{equation*}
            \|f\|_2 = {\left(\int_{K}{f(x)}^{2}~dx \right)}^{\frac{1}{2}} \qquad \forall f\in \cc{C}(K)
        \end{equation*}
        Sin embargo, este espacio prehilbertiano \textbf{no es de Hilbert}:

        Por ejemplo, si tomamos $K = [0,2]\subset \mathbb{R}$, si tomamos $f_n:K\to \mathbb{R}$ de forma que la gráfica de $f_n$ sea algo parecido a la de la Figura~\ref{fig:no_hilbert}

        \begin{figure}[H]
            \centering
            \begin{tikzpicture}[scale=3]
              % parámetro n (cambiar aquí)
              \def\n{8}
              \pgfmathsetmacro{\a}{1-1/\n} % punto 1 - 1/n

              % ejes
              \draw[-stealth] (0,0) -- (2.2,0) node[right] {$x$};
              \draw[-stealth] (0,-0.15) -- (0,1.3) node[above] {$y$};

              % marcas en eje x
              \foreach \x in {0,1,\a,2}{
                \draw (\x,0.01) -- (\x,-0.01);
              }
              \node[below] at (1,0) {$$};
              \node[below] at (2,0) {$2$};
              \node[below] at (\a,0) {$1-\tfrac{1}{n}$};

              % marcas en eje y
              \draw (0,1) -- (-0.01,1);
              \node[left] at (0,1) {$1$};
              \node[left] at (0,0) {$0$};

              % tramos de la función
              % tramo constante 0 en [0, 1-1/n]
              \draw[line width=1pt] (0,0) -- ({\a},0);

              % tramo lineal en [1-1/n, 1]
              \draw[line width=1pt] ({\a},0) -- (1,1);

              % tramo constante 1 en [1,2]
              \draw[line width=1pt] (1,1) -- (2,1);

              % puntos (para destacar continuidad a la derecha/izquierda)
              \filldraw ({\a},0) circle (0.6pt);
              \filldraw (1,1) circle (0.6pt);

              % etiquetas de los tramos
              \node[below] at ({\a/2},0) {$0$};
              \node[above right] at ({(\a+1)/2},{(1/2)}) {$\displaystyle f_n(x)=n x-(n-1)$};
              \node[above] at (1.5,1) {$1$};

              % leyenda con el valor de n
              \node[anchor=north west] at (0.02,-0.12) { $\displaystyle n=\,$\n };
            \end{tikzpicture}
            \caption{Gráfica de la función $f_n$.}
            \label{fig:no_hilbert}
        \end{figure}
        Si definimos $f=  \chi_{[1,2]}$ la función característica del intervalo $[1,2]$ (que no pertence a $\cc{C}(K)$), tenemos que:
        \begin{equation*}
            \|f-f_n\|_2^2 = \int_{0}^{2} {(f(x)- f_n(x))}^{2}~dx  = \dfrac{1}{2n} \to 0
        \end{equation*}
        Por lo que $f_n$ es una sucesión de Cauchy pero cuyo límite no está en el espacio que consideramos, por lo que no es convergente, luego $\cc{C}(K)$ no es un espacio completo. % // TODO: Afinar más este razonamiento.
\end{itemize}

\section{Espacios de Lebesgue}
Un ejemplo interesante de espacios de Banach son los espacios de Lebesgue, que ya se trabajaron un poco en la asignatura de Análisis Matemático II. En este documento volveremos a definir dicho espacio, puesto que la construcción es importante tenerla clara. En un primer lugar, hemos de repasar ciertas desigualdades para poder construir la estructura de espacio normado.

\subsection{Desigualdades importantes}
Para la primera desigualdad, es conveniente la siguiente motivación, que nos dará una breve justificación del origen de la desigualdad: sean $a,b\in \mathbb{R}^+_0$ dos números reales no negativos, es bien conocido que:
\begin{equation*}
    0 \geq {(a-b)}^{2} = a^2 + b^2 - 2ab \Longrightarrow ab \leq \dfrac{a^2}{2} + \dfrac{b^2}{2}
\end{equation*}

\begin{definicion}
    Sea $p\geq 1$ un número real, definimos su ``exponente conjugado'' por:
    \begin{equation*}
        p' = \left\{\begin{array}{ll}
                \frac{p}{p-1} & \text{si\ } p\neq 1 \\
                \infty & \text{si\ } p = 1
        \end{array}\right.
    \end{equation*}
    De esta forma (admitiendo el convenio de que $0 = \nicefrac{1}{\infty}$ de la recta real extendida), tenemos que:
    \begin{equation*}
        \dfrac{1}{p} + \dfrac{1}{p'} = 1
    \end{equation*}
\end{definicion}
\noindent
Usaremos en esta sección la notación $p'$ para denotar al exponente conjugado de $p$.

\begin{prop}[Desigualdad de Young]
    Sean $a,b\in \mathbb{R}^+_0$ y $p\in \mathbb{R}$ con $p>1$, se verifica que:
    \begin{equation*}
        ab \leq \dfrac{a^p}{p} + \dfrac{b^{p'}}{p'}
    \end{equation*}
    \begin{proof}
        La concavidad\footnote{Recordamos que si $f$ era una función cóncava, entonces $f(tx+(1-t)y) \geq tf(x)+(1-t)f(y)$, para cualquier $t\in [0,1]$, $x$,$y$ en el dominio de definición de $f$.} del logaritmo nos dice:
        \begin{equation*}
            \log\left(\dfrac{a^p}{p} + \dfrac{b^{p'}}{p'}\right) \geq \dfrac{1}{p}\log(a^p) + \dfrac{1}{p'}\log\left(b^{p'}\right) = \log(a) + \log(b) = \log(ab)
        \end{equation*}
        Y si ahora aplicamos la función exponencial y usamos que es creciente obtenemos:
        \begin{equation*}
            ab \leq \dfrac{a^p}{p} + \dfrac{b^{p'}}{p'}
        \end{equation*}
    \end{proof}
\end{prop}

\noindent
Recordemos que en Análisis Matemático I definíamos para cualquier conjunto $\Omega\subset \mathbb{R}$ medible el conjunto de las funciones integrables sobre $\Omega$:
\begin{equation*}
    \cc{L}(\Omega) = \left\{f:\Omega\to \mathbb{R} : \int_\Omega f < \infty\right\}
\end{equation*}
Pues bien, dado $p\geq 1$, podemos definir ahora:
\begin{equation*}
    \cc{L}_p(\Omega) = \left\{f\in \cc{L}(\Omega) : \int_\Omega |f|^p < \infty\right\}
\end{equation*}

\begin{teo}[Desigualdad de Hölder]
    Sea $p>1$, si $f\in \cc{L}_p(\Omega)$ y $g\in \cc{L}_{p'}(\Omega)$, entonces $fg\in \cc{L}(\Omega)$ y además:
    \begin{equation*}
        \int_\Omega |fg| \leq {\left(\int_\Omega |f|^p\right)}^{\frac{1}{p}}{\left(\int_\Omega |g|^{p'}\right)}^{\frac{1}{p'}}
    \end{equation*}
    \begin{proof}
        Si notamos por comodidad:
        \begin{equation*}
            \alpha = {\left(\int_\Omega |f|^p\right)}^{\frac{1}{p}},  \qquad \beta ={\left(\int_\Omega |g|^{p'}\right)}^{\frac{1}{p'}}
        \end{equation*}
        Si $\alpha = 0$, entonces $f^p = 0$ casi por doquier, de donde $|fg| = 0$ casi por doquier, luego:
        \begin{equation*}
            \int_\Omega|fg| = 0
        \end{equation*}
        Si $\beta = 0$ la situación es simétrica. Suponiendo ahora que $\alpha,\beta\in \mathbb{R}^+$, la desigualdad de Young nos dice que:
        \begin{equation*}
            \dfrac{|f(x)|}{\alpha} \dfrac{|g(x)|}{\beta}\leq \dfrac{|f(x)|^p}{p\alpha^p} + \dfrac{|g(x)|^{p'}}{p' \beta^{p'}} \qquad \forall x\in \Omega
        \end{equation*}
        Si ahora aplicamos la integral de Lebesgue a ambos lados usando el crecimiento de dicho funcional, obtenemos que (usando la definición de $\alpha$ y $\beta$):
        \begin{equation*}
            \dfrac{1}{\alpha\beta} \int_\Omega |fg| \leq \dfrac{1}{p\alpha^p} \int_\Omega |f|^p + \dfrac{1}{p'\beta^{p'}} \int_\Omega |g|^{p'} = \dfrac{1}{p} + \dfrac{1}{p'} = 1
        \end{equation*}
        de donde $fg\in \cc{L}(\Omega)$ y despejando de la desigualdad:
        \begin{equation*}
            \dfrac{1}{\alpha\beta} \int_\Omega |fg| \leq 1
        \end{equation*}
        Obtenemos la desigualdad buscada.
    \end{proof}
\end{teo}

\noindent
La desigualdad de Hölder nos proporcionará la desigualdad de Cauchy-Schwartz de la norma del futuro espacio normado, y nos permitirá probar la desigualdad de Minkowski.

\begin{teo}[Desigualdad de Minkowski]
    Para $p\in \mathbb{R}$ con $p\geq 1$ y $f,g\in \cc{L}_p(\Omega)$, se cumple que:
    \begin{equation*}
        {\left(\int_\Omega |f+g|^p\right)}^{\frac{1}{p}} \leq {\left(\int_\Omega |f|^p\right)}^{\frac{1}{p}} + {\left(\int_\Omega |g|^{p'}\right)}^{\frac{1}{p'}}
    \end{equation*}
    \begin{proof}
        Si notamos por comodidad:
        \begin{equation*}
            \alpha = {\left(\int_\Omega |f|^p\right)}^{\frac{1}{p}},  \qquad \beta ={\left(\int_\Omega |g|^{p'}\right)}^{\frac{1}{p'}}, \qquad \gamma = {\left(\int_\Omega |f+g|^p\right)}^{\frac{1}{p}} 
        \end{equation*}
        Si $p=1$, entonces la desigualdad triangular nos dice que $|f+g| \leq |f| + |g|$, donde aplicamos el crecimiento de la integral y ya tenemos el Teorema demostrado. Sabemos por el resultado anterior que $\gamma<\infty$, puesto que $\cc{L}_p(\Omega)\subset \cc{L}(\Omega)$, y la desigualdad buscada es obvia si $\gamma = 0$.  Supuesto ahora que $p>1$ y $\gamma>0$, si tomamos:
        \begin{equation*}
            h = |f+g|^{p-1}
        \end{equation*}
        tenemos entonces que:
        \begin{equation*}
            h^{p'} = |f+g|^{(p-1)p'} = |f+g|^p
        \end{equation*}
        luego:
        \begin{equation*}
            \int_\Omega h^{p'} = \gamma^p < \infty
        \end{equation*}
        Por lo que $h\in \cc{L}_p(\Omega)$. Tenemos:
        \begin{equation*}
            |f+g|^p = |f+g|h \leq |f| h + |g| h
        \end{equation*}
        Y por la desigualdad de Hölder:
        \begin{equation*}
            \gamma^p \leq \int_\Omega |f| h + \int_\Omega |g|h \leq (\alpha+\beta){\left(\int_\Omega h^{p'}\right)}^{\frac{1}{p'}} = (\alpha + \beta)\gamma^{\frac{p}{p'}}
        \end{equation*}
        Y si dividimos por $\gamma^{\frac{p}{p'}}$ tenemos la desigualdad buscada.
    \end{proof}
\end{teo}

\subsection{Definición de los espacios de Lebesgue}
Fijado $p\geq 1$, podemos tratar de dotar a $\cc{L}_p(\Omega)$ de una norma. Pensamos en un principio en la aplicación $\varphi_p:\cc{L}_p(\Omega)\to \mathbb{R}$ dada por:
\begin{equation*}
    \varphi_p(f) = {\left(\int_\Omega |f|^p\right)}^{\frac{1}{p}} \qquad \forall f\in \cc{L}_p(\Omega)
\end{equation*}
Que:
\begin{itemize}
    \item Verifica la desigualdad triangular gracias a la desigualdad de Minkowski.
    \item Verifica la homegeneidad por homotecias, ya que:
        \begin{equation*}
            \varphi_p(\alpha f) = |\alpha| \varphi_p(f) \qquad \forall \alpha\in \mathbb{R}
        \end{equation*}
    \item $\varphi_p(f) = 0 \Longleftrightarrow f = 0$ casi por doquier.
\end{itemize}
Por lo que dicha función \textbf{no es una norma} en $\cc{L}_p(\Omega)$ al no verificar la no degeneración de la norma, puesto que la integral ``es ciega'' a la hora de diferenciar la función constantemente igual a 0 de otras funciones con integral cero.\\

\noindent
Para solucionar el problema con el que nos acabamos de topar (el problema de no poder definir una norma de dicha forma), podemos constuir una relación de equivalencia $\sim$ en $\cc{L}_p(\Omega)$ que identifique a las funciones que son iguales casi por doquier, pudiendo considerar el espacio cociente:
\begin{equation*}
    L_p(\Omega)= \dfrac{\cc{L}_p(\Omega)}{\sim}
\end{equation*}
Donde ya $(L_p(\Omega), \varphi_p)$ sí que es un espacio normado, donde denotaremos normalmente $\varphi_p = \|\cdot \|_p$.

\begin{teo}[Riesz-Fischer]
    Sea $\Omega\subset \mathbb{R}^N$ un conjunto medible y $p\geq 1$, se cumple que $(L_p(\Omega), \|\cdot \|_p)$ es un espacio de Banach.
\end{teo}

\subsection{Más ejemplos de espacios de Banach}
\begin{itemize}
    \item Sea $\Omega\subset \mathbb{R}$ un conjunto medible, si definimos:
        \begin{equation*}
            \sup_\Omega |f| = \inf\{M\geq 0 : |f(x)| \leq M \text{\ casi para todo\ } x\in \Omega\}
        \end{equation*}
        El conjunto:
        \begin{equation*}
            \cc{L}^\infty(\Omega) = \left\{f:\Omega\to \mathbb{R} : f \text{\ es medible y\ } \sup_\Omega |f| < \infty\right\}
        \end{equation*}
        junto con la norma:
        \begin{equation*}
            \|f\|_\infty = \sup_\Omega |f|
        \end{equation*}
        es un espacio de Banach, donde la desigualdad de Hölder se comple considerando que $p=\infty$ y $p'=1$:

        Si $f\in \cc{L}^\infty(\Omega)$ y $g\in \cc{L}(\Omega)$, entonces $fg\in \cc{L}(\Omega)$, con:
        \begin{equation*}
            \|fg\|_1 \leq \|f\|_\infty \|g\|_1
        \end{equation*}
    \item Para $1\leq p < \infty$ podemos considerar otro tipo de espacios:
        \begin{equation*}
            l^p = \left\{x:\mathbb{N}\to \mathbb{R} : \sum_{n=1}^{\infty}|x(n)|^p < \infty\right\}
        \end{equation*}
        que junto con la aplicación:
        \begin{equation*}
            \|x\|_p = {\left(\sum_{n=1}^{\infty}|x(n)|^p\right)}^{\frac{1}{p}} \qquad \forall x\in l^p
        \end{equation*}
        forman un espacio de Banach (compruébese).

        En dichos espacios, se tiene que si $x\in l^p$ y $y\in l^{p'}$, entonces $xy\in l$, con:
        \begin{equation*}
            \|xy\| \leq \|x\|_p \|y\|_{p'}
        \end{equation*}
    \item En el caso anterior, si $p=2$, podemos definir la aplicación:
        \begin{equation*}
            \langle x,y \rangle_2 = \sum_{n=1}^{\infty}x(n)y(n) \qquad \forall x,y\in l^2
        \end{equation*}
        Con lo que $(l^2, \langle \cdot ,\cdot  \rangle _2)$ es un espacio de Hilbert.
    \item Al igual que sucedía con las normas $p-$ésimas en $\mathbb{R}^N$, podemos considerar:
        \begin{equation*}
            l^\infty = \{x:\mathbb{N}\to \mathbb{R} : x \text{\ acotada}\}
        \end{equation*}
        junto con la aplicación $\|\cdot \|:l^{\infty}\to \mathbb{R}$ dada por:
        \begin{equation*}
            \|x\|_\infty = \sup\{|x(n)| : n\in \mathbb{N}\}
        \end{equation*}
        y obtenemos un espacio de Banach.
    \item $C = \{x:\mathbb{N}\to \mathbb{R} : x \text{\ es convergente}\}$ es un subespacio de $l^\infty$.
    \item $C_0 = \{x:\mathbb{N}\to \mathbb{R} : x \text{\ converge a\ }0\}$ es un subespacio de $C$.
\end{itemize}

\section{Espacio dual}
Para introducir la nocíon de espacio dual, nos será necesario primero destacar unos resultados:
\begin{prop}
    Si $H$ es un espacio prehilbertiano, entonces:
    \begin{enumerate}
        \item Se cumple la desigualdad de Cauchy-Schwartz:
            \begin{equation*}
                |\langle u,v \rangle | \leq \|u\|\|v\| \qquad \forall u,v\in H
            \end{equation*}
        \item Se cumple la identidad del paralelogramo:
            \begin{equation*}
                \left\|\dfrac{u+v}{2}\right\| + \left\| \dfrac{u-v}{2}\right\| = \dfrac{1}{2}(\|u\|^2 + \|v\|^2) \qquad \forall u,v\in H
            \end{equation*}
    \end{enumerate}
\end{prop}

\begin{teo}[de la Proyección]
    Sea $H$ un espacio de Hilbert, sea $\emptyset \neq K \subset H$ un conjunto convexo y cerrado, entonces $\forall f\in H~\exists_1 u\in K$ de forma que:
    \begin{equation*}
        \|f-u\| = d(f,K) = \inf \{d(f,v) : v\in K\}
    \end{equation*}
    Además, dicho elemento $u$ está caracterizado por:
    \begin{itemize}
        \item $u\in K$.
        \item $\langle f-u,v-u \rangle \leq 0 \qquad \forall v\in K$.
    \end{itemize}
    Por tanto, a dicho único elemento $u$ lo notaremos por $P_Kf$.
    \begin{proof}
        Como $0\leq d(f,v) \quad \forall v\in K$, tenemos entonces que dicho ínfimo existe. Tenemos por tanto que existe $\{v_n\}$ una sucesión de elementos de $K$ de forma que $\{d(f,v_n)\}\to d(f,K)$. Sean $n,m\in \mathbb{N}$ y usando la identidad del paralelogramo con $f-v_n$ y $f-v_m$, tenemos:
        \begin{gather*}
            \left\| \frac{f-v_n + f-v_m}{2} \right\|^2 + \left\| \frac{f-v_n-(f-v_m)}{2} \right\|^2 = \frac{1}{2}\left(\|f-v_n\|^2 + \|f-v_m\|^2\right)\\
            \left\| f - \frac{v_n+v_m}{2} \right\|^2 + \left\| \frac{v_m - v_n}{2} \right\|^2 = \frac{1}{2}\left(\|f-v_n\|^2 + \|f-v_m\|^2\right)\\
            \frac{\left\| v_m-v_n \right\|^2}{4} = \frac{1}{2}\left(\|f-v_n\|^2 + \|f-v_m\|^2\right) - \left\| f - \frac{v_n+v_m}{2} \right\|^2\\
            \left\| v_m-v_n \right\|^2 = 2\left(\|f-v_n\|^2 + \|f-v_m\|^2\right) - 4\left\| f - \frac{v_n+v_m}{2} \right\|^2
        \end{gather*}
        Como $K$ es convexo, tenemos que $\frac{v_n+v_m}{2}\in K$, por lo que:
        \begin{equation*}
            \left\| f-\dfrac{v_n+v_m}{2}\right\| \geq d(f,K)
        \end{equation*}
        Por lo que:
        \begin{equation*}
            0 \leq \|v_m - v_n\|^2 \leq 2(\|f-v_n\|^2 + \|f-v_m\|^2) - 4d(f,K)^2
        \end{equation*}
        Como $\{\|f-v_n\|^2\} \to d(f,K)^2$ y $\{\|f-v_m\|^2\}\to d(f,K)^2$, tenemos por el Lema del Sandwitch que $\{\|v_n - v_m\|^2\}\to 0$, por lo que $\{v_n\}$ es de Cauchy. Como $H$ es completo, existe $u\in H$ de forma que $\{v_n\}\to u$, pero por ser $K$ cerrado tendremos que $u\in K$.

        Como $\{v_n\}\to u$, tenemos entonces que $\{d(f,v_n)\}\to d(f,v)$, pero $\{d(f,v_n)\}$ convergía también a $d(f,K)$. No queda más salida que $d(f,v) = d(f,K)$.\\

        \noindent
        Una vez probada la existencia de $u$, veamos que:
        \begin{equation*}
            u\in K \text{\ con\ } \|f-u\| = d(f,K) \Longleftrightarrow u\in K \text{\ y\ } \langle f-u,v-u \rangle \leq 0 \quad \forall v\in K
        \end{equation*}
        \begin{description}
            \item [$\Longrightarrow)$] Supongamos que $u\in K$ y sabemos que $\|f-u\|\leq \|f-v\|$ para todo $v\in K$. Tomamos ahora $w\in K$ y consideramos el segmento que une $u$ con $w$. Entonces $\forall w\in K$ y $\forall t \in [0,1]$, al ser $K$ convexo tendremos que
            \begin{gather*}
                (1-t)u + tw \in K\ \  \text{ y }\ \ \|f-u\|^2 \leq \|f-(1-t)u-tw\|^2
            \end{gather*}
            Aplicando la bilinealidad podemos reescribir esta última expresión como 
            \begin{align*}
                \|f-(1-t)u-tw\|^2 &= \langle f-(1-t)u-tw,f-(1-t)u-tw \rangle  =\\
                &=\|f-u\|^2 + t^2\|w-u\|^2-2t(f-u,w-u)
            \end{align*}
            Sustituyendo en la expresión que teníamos anteriormente nos queda que:
            \begin{gather*}
                0\leq t^2\|w-u\|^2-2t\langle f-u,w-u \rangle  \ \ \ \forall t \in (0,1]
            \end{gather*}
            Al dividir entre $t$ nos queda
            \begin{gather*}
                0\leq t\|w-u\|^2-2\langle f-u,w-u \rangle  \ \ \ \forall t \in (0,1]
            \end{gather*}
            y tomando ahora el límite  cuando $t$ tiende a $0$ por la derecha queda que
            \begin{gather*}
                0\leq -2\langle f-u,w-u \rangle  \Rightarrow \langle f-u,w-u \rangle  \leq 0 \qquad \forall w\in K
            \end{gather*}
            \item [$\Longleftarrow)$] 
                \begin{equation*}
                    \|f-v\|^2 = \|f-u+u-v\|^2 = \|f-u\|^2 + 2\langle f-u,u-v \rangle  + \|u-v\|^2 \qquad \forall v\in K
                \end{equation*}
                De donde:
                \begin{equation*}
                    0\geq 2\langle f-u,v-u \rangle  - \|u-v\|^2 = \|f-u\|^2 - \|f-v\|^2
                \end{equation*}
                Luego:
                \begin{equation*}
                    \|f-u\|^2 \leq \|f-v\|^2 \qquad \forall v\in K
                \end{equation*}
        \end{description}
        Para probar finalmente la unicidad, supongamos que existen $u,w\in K$ de forma que:
        \begin{equation*}
            \langle f-u,v-u \rangle ,\langle f-w,v-w \rangle \leq 0 \qquad \forall v\in K
        \end{equation*}
        Entonces:
        \begin{equation*}
            \langle f-u,w-u \rangle , \langle f-w,u-w \rangle = \langle u-f,w-u \rangle  \leq 0
        \end{equation*}
        Por lo que:
        \begin{equation*}
            \langle f-u,w-u \rangle  + \langle w-f, w-u \rangle  = \langle w-u, w-u \rangle  \leq 0
        \end{equation*}
        de donde $\langle w-u,w-u \rangle = 0$, por lo que $\|w-u\|^2 = d(w,u)^2 = 0$, luego $w=u$.
    \end{proof}
\end{teo}

\begin{prop}
    Dado $\emptyset  \neq K\subset H$ un conjunto convexo y cerrado, tenemos que la aplicación
    \Func{P_K}{H}{H}{f}{P_Kf}
    es lipschitziana. De hecho:
    \begin{equation*}
        \|P_Kf_1 - P_Kf_2\| \leq \|f_1 - f_2\| \qquad \forall f_1,f_2\in H
    \end{equation*}
    \begin{proof}
        Sean $f_1, f_2\in H$, $u_1 = P_Kf_1$, $u_2=P_Kf_2$, estos verifican:
        \begin{equation*}
            \langle f_1-u_1, v-u_1 \rangle , \langle f_2-u_2,v-u_2 \rangle  \leq 0 \qquad \forall v\in K
        \end{equation*}
        Por lo que:
        \begin{gather*}
            \langle f_1 - u_1, u_2 - u_1 \rangle  \leq 0 \\
            \langle f_2 - u_2, u_1 - u_2 \rangle \leq 0 \Longrightarrow \langle f_2 - u_2, u_2 - u_1 \rangle \geq 0
        \end{gather*}
        De donde $\langle f_1 - u_2 - f_2 + u_2, u_2 - u_1 \rangle \leq 0$, por lo que:
        \begin{equation*}
            \langle f_1 - f_2 + (u_2 - u_1), (u_2 - u_1) \rangle  = \langle f_1 - f_2, u_2 - u_1 \rangle  + \langle u_2 - u_1, u_2 - u_1 \rangle 
        \end{equation*}
        Luego:
        \begin{equation*}
            \|u_2 - u_1\|^2 = \langle u_2 - u_1, u_2 - u_1 \rangle  \leq - \langle f_1 - f_2, u_2 - u_1 \rangle \stackrel{\text{Cauchy-Schwartz}}{\leq} \|f_1 - f_2\| \|u_2 - u_1\|
        \end{equation*}
        Por lo que:
        \begin{equation*}
            \|u_2 - u_1\| \leq \|f_1 - f_2\|
        \end{equation*}
        Si $\|u_2 - u_1\| \neq 0$, cierto también si $\|u_2 - u_1\| = 0$.
    \end{proof}
\end{prop}

\noindent
Pensemos ahora en un ejemplo de conjuntos convexos con propiedades interesantes, como lo son los espacios vectoriales:

\begin{coro}[Proyección Ortogonal]
    Sea $M\subset H$ un subespacio vectorial cerrado de $H$, un espacio de Hilbert, entonces:
    \begin{equation*}
        \forall f\in H~\exists _1 u \in M \text{\ tal que\ } \|f-u\| = d(f,M)
    \end{equation*}
    Además, la caracterización de $u$ puede mejorarse por:
    \begin{equation*}
        u\in M \qquad \text{y} \qquad \langle f-u,w \rangle  = 0 \quad \forall w\in M
    \end{equation*}
    \begin{proof}
        Bajo las hipótesis de que $M$ es un subespacio vectorial cerrado de un espacio de Hilbert $H$, basta probar:
        \begin{equation*}
            u\in M \land \langle f-u,v-u \rangle \leq 0 \quad \forall v\in M \Longleftrightarrow u\in M \land \langle f-u,w \rangle =0 \quad \forall w\in M
        \end{equation*}
        \begin{description}
            \item [$\Longleftarrow )$] Si $v\in M$, tenemos por ser $M$ un espacio vectorial que $v-u\in M$, de donde $\langle f-u,v-u \rangle =0$, por lo que en particular es menor o igual que 0.
            \item [$\Longrightarrow )$] Si tomamos $v\in M$ y $t\in \mathbb{R}^\ast$, como $M$ es un espacio vectorial tendremos que $\nicefrac{v}{t}\in M$, por lo que:
                \begin{equation*}
                    \left\langle f-u,\frac{v}{t}-u \right\rangle  \leq 0 \qquad \forall v\in M, \forall t\in \mathbb{R}^\ast
                \end{equation*}
                \begin{itemize}
                    \item Si $t>0$, entonces $\langle f-u,v-tu \rangle \leq 0$ $\forall v\in M$, de donde tomando límite cuando $t\to 0$, tenemos que $\langle f-u,v \rangle \leq 0$ $\forall v\in M$.
                    \item Si $t<0$, entonces $\langle f-u,v-tu \rangle \geq 0$ $\forall v\in M$, de donde tomando límite cuando $t\to 0$, tenemos que $\langle f-u,v \rangle \geq 0$ $\forall v\in M$.
                \end{itemize}
                En consecuencia, tenemos que $\langle f-u,v \rangle =0$ $\forall v\in M$.
        \end{description}
    \end{proof}
\end{coro}

\begin{prop}
    Sea $M\subset H$ un esubespacio vectorial cerrado de $H$, un espacio de Hilbert, la aplicación
    \Func{P_M}{H}{H}{f}{P_Mf}
    es lineal.
    \begin{proof}
        Sean $f_1,f_2\in H$, $u_1 = P_Mf_1$, $u_2 = P_Mf_2$, $\lm \in \mathbb{R}$, tenemos que:
        \begin{align*}
            \langle \lm f_1+f_2-(\lm u_1+u_2),w \rangle  &= \langle \lm f_1 - \lm u_1 +f_2 - u_2, w \rangle \\ &= \lm \langle f_1 -u_1,w \rangle + \langle f_2-u_2,w \rangle  = 0 \qquad \forall w\in M
        \end{align*}
        Por lo que por el Corolario anterior, tenemos que: 
        \begin{equation*}
            P_M(\lm f_1 + f_2) = \lm u_1 + u_2 = \lm P_M(f_1) + P_M(f_2)
        \end{equation*}
        de donde $P_M$ es lineal.
    \end{proof}
\end{prop}

\begin{definicion}
    Sea $(E,\|\cdot \|)$ un espacio normado, definimos el \underline{espacio dual topológico} de $E$ por:
    \begin{equation*}
        E^\ast = \{f:E\to \mathbb{R} : f \text{\ es lineal y continua}\}
    \end{equation*}
\end{definicion}

\noindent
Nos será necesaria la siguiente Proposición para comprender mejor las propiedades de las aplicaciones lineales. Más concretamente, la relación existente entre la acotación y la continuidad de una aplicación lineal.
\begin{prop}\label{prop:lineal_continuidad_acotacion}
    Sea $T:E\to F$ una aplicación lineal entre dos espacios normados $E$ y $F$, las siguientes afirmaciones son equivalentes:
    \begin{enumerate}[label=(\arabic*)]
        \item $\exists M\in \mathbb{R}^+$ de forma que $\|T(x)\| \leq M\|x\| \quad \forall x\in E$.
        \item $T$ es lipschitziana.
        \item $T$ es continua.
        \item $T$ es continua en $0$.
        \item $T$ es acotada (es decir, si $A\subset E$ es acotado, entonces $T(A)$ es acotado).
        \item $T(\overline{B}(0,1))$  es acotado.
        \item $T(B(0,1))$  es acotado.
    \end{enumerate}
    \begin{proof}
        Veamos la equivalencia entre todas ellas:
        \begin{description}
            \item [$(1)\Longleftrightarrow (2)$]  Por doble implicación:
                \begin{description}
                    \item [$\Longrightarrow )$] Sean $x,y\in E$, entonces $x-y\in E$, de donde:
                        \begin{equation*}
                            \|T(x) - T(y)\| = \|T(x-y)\| \leq M\|x-y\|
                        \end{equation*}
                        Por lo que $T$ es lipschitziana con constante de Lipschitz menor o igual que $M$.
                    \item [$\Longleftarrow )$] Sea $x\in E$, si $M$ es mayor o igual que la constante de Lipschitz de $T$, entonces:
                        \begin{equation*}
                            \|T(x)\| = \|T(2x-x)\| = \|T(2x) - T(x)\| \leq M\|2x-x\| = M\|x\|
                        \end{equation*}
                \end{description}
            \item [$(2)\Longrightarrow (3)$] Es conocida de Cálculo II.
            \item [$(3)\Longrightarrow (4)$] Si $T$ es continua, en particular lo es en $0$.
            \item [$(4)\Longrightarrow (1)$] Supuesto que $T$ es continua en $0$, es decir, que:
                \begin{equation*}
                    \forall \varepsilon>0~\exists \delta>0 : \|T(x)\| < \varepsilon~\forall x\in B(0,\delta)
                \end{equation*}
                Tomando $\varepsilon=1$, la continuidad nos da un $\delta$ cumpliendo la afirmación anterior. Sea $x\in E$ arbitrario, tenemos:
                \begin{equation*}
                    \|T(x)\| = \left\|T\left(\dfrac{x}{\|x\|}\dfrac{\delta}{2}\dfrac{2\|x\|}{\delta}\right)\right\| = \dfrac{2\|x\|}{\delta} \left\|T\left(\dfrac{x}{\|x\|}\dfrac{\delta}{2}\right)\right\| < \dfrac{2}{\delta} \|x\|
                \end{equation*}
                Ya que $\dfrac{x\delta}{2\|x\|}\in B(0,\delta)$, por lo que tomando $M=\frac{2}{\delta}$ tenemos la implicación.
            \item [$(5)\Longrightarrow (6)$] Como $\overline{B}(0,1)$ es acotado, $T(\overline{B}(0,1))$ será acotado por ser $T$ acotada.
            \item [$(6)\Longrightarrow (7)$] Como $B(0,1)\subset \overline{B}(0,1)$, entonces $T(B(0,1))\subset T(\overline{B}(0,1))$.
            \item [$(7)\Longrightarrow (4)$] Si $\exists R\in \mathbb{R}^+$ de forma que $\|T(x)\| \leq R$ para todo $x\in B(0,1)$, dado $\varepsilon>0$, si tomamos $\delta = \frac{\varepsilon}{2R}$, si $x\in B(0,\delta)$, entonces:
                \begin{equation*}
                    \|T(x)\| = \left\|T\left(\dfrac{x}{2\|x\|}2\|x\|\right)\right\| = 2\|x\|\left\|T\left(\dfrac{x}{2\|x\|}\right)\right\| \leq 2\|x\|R < 2\delta R = \varepsilon
                \end{equation*}
            \item [$(1)\Longrightarrow (5)$] Sea $A\subset E$ acotado, entonces $\exists r\in \mathbb{R}^+$ de forma que $A\subset B(0,r)$, por lo que:
                \begin{equation*}
                    \|T(x)\| \leq M\|x\| \leq Mr \qquad \forall x\in A
                \end{equation*}
                De donde $T(A)\subset B(0,Mr)$, por lo que es un conjunto acotado.
        \end{description}
    \end{proof}
\end{prop}

\begin{prop}
    Sea $E$ un espacio normado, observemos que $E^\ast$ es un espacio vectorial, sobre el que definimos la aplicación $\|\cdot \|:E^\ast \to \mathbb{R}$ dada por:
    \begin{equation*}
        \|f\| = \sup_{\|x\| \leq 1} |f(x)| \qquad \forall f\in E^\ast
    \end{equation*}
    Se verifica que:
    \begin{enumerate}
        \item $(E^\ast, \|\cdot \|)$ es un espacio normado.
        \item $(E^\ast, \|\cdot \|)$ es un espacio de Banach.
        \item Sea $f\in E^\ast$, entonces:
            \begin{equation*}
                \sup_{\|x\| \leq 1} |f(x)| = \|f\| = \inf \{M\in \mathbb{R}^+_0 : |f(x)| \leq M\|x\| \quad \forall x\in E\}
            \end{equation*}
    \end{enumerate}
    \begin{proof}
        Veamos cada una de las propiedades:
        \begin{enumerate}
            \item Para la primera, hemos de probar:
                \begin{itemize}
                    \item \textbf{No degeneración.} Sea $f\in E^\ast$ de forma que $\sup\limits_{\|x\|\leq 1}|f(x)| = \|f\| = 0$, entonces:
                        \begin{equation*}
                            0\leq \|f(x)\|\leq 0 \quad \forall x\in \overline{B}(0,1) \Longrightarrow f(x) = 0 \quad \forall x\in \overline{B}(0,1)
                        \end{equation*}
                        de donde:
                        \begin{equation*}
                            f(x) = f\left(\dfrac{x}{\|x\|}\|x\|\right) = \|x\| f\left(\dfrac{x}{\|x\|}\right) = 0 \qquad \forall x\in E
                        \end{equation*}
                        Por lo que $f = 0$.
                    \item \textbf{Homogeneidad por homotecias.} Sea $f\in E^\ast$ y $\lm\in \mathbb{R}A:$
                        \begin{equation*}
                            \|\lm f\| = \sup_{\|x\|\leq 1} |\lm f(x)|  = \sup_{\|x\|\leq 1} |\lm||f(x)| = |\lm| \sup_{\|x\|\leq 1}|f(x)| = |\lm| \|f\|
                        \end{equation*}
                    \item \textbf{Desigualdad triangular.} Sean $f,g\in E^\ast$:
                        \begin{align*}
                            \|f+g\| &= \sup_{\|x\|\leq 1}|f(x) + g(x)| \leq \sup_{\|x\|\leq 1} (|f(x)| + |g(x)|) \\ &\leq \sup_{\|x\|\leqq 1}|f(x)| + \sup_{\|x\|\leq 1} |g(x)| = \|f\| + \|g\|
                        \end{align*}
                \end{itemize}
            \item Sea $\{f_n\}$ una sucesión de Cauchy de elementos de $E^\ast$, sean $\varepsilon,r>0$, la condición de Cauchy para $\nicefrac{\varepsilon}{r}$ nos da $m\in \mathbb{N}$ de forma que si $p,q\geq m$, entonces:
                \begin{equation*}
                    \sup_{\|x\|\leq 1} |f_p(x) - f_q(x)| = \|f_p - f_q\| < \dfrac{\varepsilon}{r}
                \end{equation*}
                de donde:
                \begin{equation*}
                    |f_p(x) - f_q(x)| < \dfrac{\varepsilon}{r}\qquad \forall x\in \overline{B}(0,1)
                \end{equation*}
                pero entonces:
                \begin{equation*}
                    |f_p(rx) - f_q(rx)| = r|f_p(x) - f_q(x)| < \varepsilon \qquad \forall x\in \overline{B}(0,1)
                \end{equation*}
                lo que equivale a que:
                \begin{equation*}
                    |f_p(x) - f_q(x)| < \varepsilon\qquad \forall x\in \overline{B}(0,r)
                \end{equation*}
                Por tanto, la sucesión $\{f_n(x)\}$ es de Cauchy para todo $x\in \overline{B}(0,r)$, pero como $r$ era arbitrario, dicha condición se cumple para todo $r\in \mathbb{R}^+$, tenemos que $\{f_n(x)\}$ es de Cauchy para todo $x\in E$. Como $\mathbb{R}$ es completo, la sucesión $\{f_n(x)\}$ es convergente para todo $x\in E$, lo que nos permite definir $f:E\to \mathbb{R}$ dada por:
                \begin{equation*}
                    f(x) = \lim \{f_n(x)\} \qquad \forall x\in E
                \end{equation*}
                Se verifica que $f$ es lineal, ya que:
                \begin{align*}
                    f(\lm x + y) &= \lim \{f_n(\lm x + y)\} = \lim \{\lm f_n(x)  + f_n(y)\} \\ &= \lm \lim\{f_n(x)\} +  \lim\{f_n(y)\} = \lm f(x) + f(y) \\
                                 & \forall \lm \in \mathbb{R}, \forall x,y\in E
                \end{align*}
                Ahora, como $\{f_n\}$ era de Cauchy, tenemos que fijado $r\in \mathbb{R}^+$ y dado $\varepsilon>0$, $\exists m\in \mathbb{N}$ de forma que para $p,q\geq m$ se tiene:
                \begin{equation*}
                    |f_p(x) - f_q(x)| < \dfrac{\varepsilon}{2} \qquad \forall x\in \overline{B}(0,r)
                \end{equation*}
                Fijado ahora dicho $p$, tenemos:
                \begin{equation*}
                    |f_p(x) - f(x)| = \lim_{q\to \infty} |f_p(x) - f_q(x)| \leq \dfrac{\varepsilon}{2} < \varepsilon
                \end{equation*}
                Por lo que $\{f_n\}$ converge uniformemente a $f$ en $B(0,r)$, para todo $r\in \mathbb{R}^+$. En particular, $\{f_n\}$ converge uniformemente a $f$ en cada conjunto acotado de $E$. Como $\{f_n\}$  es continua $\forall n\in \mathbb{N}$ y para cada $x\in E$ tenemos que $\{f_n\}$ converge uniformemente a $f$ en $B(x,1)$, entonces tenemos que $f $ es continua en $x$, de donde $f$ es continua en $E$. En consecuencia, $f\in E^\ast$.

                Por último, para ver que $\{f_n\}$ converge a $f$, dado $\varepsilon>0$, existe $m\in \mathbb{N}$ de forma que si $n\geq m$, entonces:
                \begin{equation*}
                    |f_n(x) - f(x)| < \dfrac{\varepsilon}{2} \qquad \forall x\in \overline{B}(0,1)
                \end{equation*}
                de donde:
                \begin{equation*}
                    \|f_n - f\| = \sup_{\|x\|\leq 1} |f_n(x) - f(x)| \leq \dfrac{\varepsilon}{2} < \varepsilon
                \end{equation*}
                Por lo que $\{f_n\}\to f$.
            \item Buscamos probar que:
                \begin{equation*}
                    \sup_{\|x\|\leq 1}|f(x)| = \inf\{M\in \mathbb{R}^+_0 : |f(x)| \leq M\|x\| \quad \forall x\in E\}
                \end{equation*}
                \begin{description}
                    \item [$\geq)$] Para ver que el supremo es mayor o igual que el ínfimo, veamos que el supremo pertence al conjunto de la derecha:
                        \begin{equation*}
                            |f(x)| = \|x\| \left|f\left(\dfrac{x}{\|x\|}\right)\right| \leq \|x\| \sup_{\|x\|\leq 1}|f(x)|
                        \end{equation*}
                        Por tanto, $\sup\limits_{\|x\|\leq1}|f(x)| \in \{M\in \mathbb{R}^+_0 : |f(x)|\leq M\|x\| \quad \forall x\in E\}$.
                    \item [$\leq)$] Para ver el el ínfimo es mayor o igual que el supremo, veamos que el ínfimo es un mayorante del conjunto de la izquierda, si tomamos:
                        \begin{equation*}
                            M_0 = \inf\{M\in \mathbb{R}^+_0 : |f(x)| \leq M\|x\| \quad \forall x\in E\}
                        \end{equation*}
                        entonces:
                        \begin{equation*}
                            |f(x)| \leq M\|x\| \leq M  \qquad \forall x\in \overline{B}(0,1)
                        \end{equation*}
                        Por lo que $M_0$ es un mayorante de $\{|f(x)| : \|x\| \leq 1\}$, por lo que es mayor o igual que su supremo.
                \end{description}
        \end{enumerate}
    \end{proof}
\end{prop}

\subsection{Espacio dual de un espacio de Hilbert}
\begin{prop}
    Se verifica que si $v\in H$, entonces la aplicación
    \Func{\varphi_v}{H}{\mathbb{R}}{u}{\langle u,v\rangle}
    verifica que $\varphi_v\in H^\ast$ y en cuyo caso, $\|\varphi_v\| = \|v\|$.\\

    \noindent
    Más aún, podemos definir 
    \Func{\Phi}{H}{H^\ast}{v}{\varphi_v}
    que es una aplicación lineal e inyectiva.
    \begin{proof}
        Como el producto escalar es bilineal es evidente que $\varphi_v$ es lineal. Vemos que:
        \begin{align*}
            |\varphi_v(u) - \varphi_v(w)| &= |\langle u,v \rangle -\langle w,v \rangle | = |\langle u-w,v \rangle | \leq \|u-w\|\|v\| \qquad \forall u,w\in E
        \end{align*}
        Por lo que $\varphi_v$ es lipschitziana, y por la última Proposición tenemos que $\|\varphi_v\| \leq \|v\|$. Si $v=0$ tenemos la igualdad de forma obvia y si $v\neq 0$, entonces:
        \begin{equation*}
            \|v\| = \dfrac{\|v\|}{\|v\|^2} = \dfrac{\langle v,v \rangle }{\|v\|} = \left\langle \dfrac{v}{\|v\|},v \right\rangle  = \varphi_v\left(\dfrac{v}{\|v\|}\right)
        \end{equation*}
        luego:
        \begin{equation*}
            \|v\| \leq \sup_{\|x\|\leq 1}|f(x)| = \|\varphi_v\|
        \end{equation*}
        Para ver que $\Phi$ es lineal, sean $\lm\in \mathbb{R}$ y $u,v\in H$:
        \begin{equation*}
            \Phi(\lm u+v) = \varphi_{\lm u+v} \stackrel{\text{?}}{=} \lm \varphi_u + \varphi_v = \lm \Phi(u) + \Phi(v)
        \end{equation*}
        donde la igualdad puede demostrarse por:
        \begin{equation*}
            \varphi_{\lm u + v}(w) = \langle w,\lm u + v \rangle  = \langle w,\lm u \rangle  + \langle w,v \rangle  = \lm \langle w,u \rangle  + \langle w,v \rangle  = \lm \varphi_u(w) + \varphi_v(w)
        \end{equation*}
        Como $\|\varphi_v\|  = \|v\|$, obtenemos de forma inmediata la continuidad de $\Phi$, por ser una isometría.\\

        \noindent
        Para ver que $\Phi$ es inyectiva, supongamos que $u,v\in H$ con $\Phi(u) = \Phi(v)$, de donde:
        \begin{equation*}
            \langle u,w \rangle  = \langle v,w \rangle  \qquad \forall w\in H
        \end{equation*}
        Luego:
        \begin{equation*}
            \langle u,w \rangle  - \langle v,w \rangle  = \langle u-v,w \rangle  = 0 \qquad \forall w\in H
        \end{equation*}
        En particular, tomando $w= u-v$, tenemos que:
        \begin{equation*}
            \|u-v\|^2 = \langle u-v,u-v \rangle  = 0
        \end{equation*}
        Por lo que $u=v$, de donde $\Phi$ es inyectiva.
    \end{proof}
\end{prop}

\begin{teo}[de Riesz-Fréchet, Representación del dual de un Hilbert]\ \\
    Sea $H$ un espacio de Hilbert, $\forall \varphi\in H^\ast$ $\exists _1 v\in H$ de forma que:
    \begin{equation*}
        \varphi(u) = \langle u,v \rangle  \qquad \forall u\in H
    \end{equation*}

    y además:
    \begin{equation*}
        \|\varphi\| = \|v\|
    \end{equation*}
    \begin{proof}
        Si conseguimos probar la primera parte del Teorema, la segunda la tendremos ya probada gracias a la Proposición anterior. Sea por tanto $f\in H^\ast$, si $f=0$ tomando $v=0$ se tiene la tesis. Suponemos por tanto que $f\neq 0$, por lo que $M = f^{-1}(\{0\})\subsetneq H$ es un espacio vectorial de $H$ distinto del trivial. Como $f$ es continua, tenemos además que $M$ es un conjunto cerrado.\\

        \noindent
        Como $M\subsetneq H$, podemos tomar $z_o\in H\setminus M$. Por el Teorema de la Proyección Ortogonal, tomamos $z_1 = P_Mz_0 \in M$, que verifica:
        \begin{equation*}
            \langle z_0 - z_1, v \rangle  = 0 \qquad \forall v\in M
        \end{equation*}
        Como $z_0\in H\setminus M$ y $z_1 \in M$, tenemos que $z_0 \neq z_1$, lo que nos permite definir:
        \begin{equation*}
            z = \dfrac{z_0 - z_1}{\|z_0 - z_1\|}
        \end{equation*}
        Con esta definición, es claro que $\|z\| = 1$, así como que:
        \begin{equation*}
            \langle z,v \rangle = \dfrac{1}{\|z_0 - z_1\|}\langle z_0 - z_1,v \rangle  = 0 \qquad \forall v\in M
        \end{equation*}
        Como $z_0 \notin M$ la situación $z\in M$ es imposible, por lo que $z\notin M$, luego $f(z) \neq 0$. Veamos ahora que:
        \begin{equation*}
            x-\dfrac{f(x)}{f(z)}z \in M \qquad \forall x\in H
        \end{equation*}
        ya que:
        \begin{equation*}
            f\left(x-\dfrac{f(x)}{f(z)}z\right) = f(x) - \dfrac{f(x)}{f(z)}f(z) = 0
        \end{equation*}
        Por lo que tenemos que:
        \begin{equation*}
            \left\langle z,x-\dfrac{f(x)}{f(z)}z \right\rangle  = 0
        \end{equation*}
        Pero tenemos:
        \begin{equation*}
            0 = \left\langle z,x-\dfrac{f(x)}{f(z)}z \right\rangle  = \langle z,x \rangle - \dfrac{f(x)}{f(z)} \langle z,z \rangle  = \langle z,x \rangle  - \dfrac{f(x)}{f(z)} \|z\|^2
        \end{equation*}
        Por lo que podemos despejar $f(x)$, obteniendo:
        \begin{equation*}
            f(x) = f(z)\langle z,x \rangle = \langle x,zf(z) \rangle   \qquad \forall x\in H
        \end{equation*}
        En consecuencia, tomando $v = zf(z)$ tenemos la existencia probada.\\

        \noindent
        Para la unicidad, supongamos que $\exists v,w\in H$ de forma que:
        \begin{equation*}
            \langle x,v \rangle  = f(x) = \langle x,w \rangle  \qquad \forall x\in H
        \end{equation*}
        En consecuencia:
        \begin{equation*}
            \langle x,v-w \rangle  = 0 \qquad \forall x\in H
        \end{equation*}
        Luego si tomamos $x=v-w$:
        \begin{equation*}
            \|v-w\|^2 = \langle v-w,v-w \rangle  = 0
        \end{equation*}
        Por lo que $v=w$.
    \end{proof}
\end{teo}

\noindent
A partir del Teorema anterior tenemos que $(\mathbb{R}^N, \|\cdot \|_2)$, $l^2$ y $L^2(\Omega)$ son todos isomorfos a sus duales.

\begin{ejercicio} % // TODO: HACER
    Calcular el dual de $(\mathbb{R}^N, \|\cdot \|_p)$, para $p>1$, $p\neq 2$.
\end{ejercicio}

\begin{ejercicio} % // TODO: HACER
    Encontrar una relación entre los duales de $(\mathbb{R}^N, \|\cdot \|_1)$ y $(\mathbb{R}^N, \|\cdot \|_\infty)$.
\end{ejercicio}

\begin{ejercicio} % // TODO: HACER
    Calcular el dual de $l^p$, para $p>1$, $p\neq 2$.
\end{ejercicio}

\section{Teorema de Hahn Banach}
\noindent
El siguiente Teorema tiene la utildad de probar que $E^\ast \neq \{0\}$ para $E$ un espacio normado. Si tenemos un espacio normado $E$ de dimensión finita, es fácil pensar una aplicación lineal y continua $f:E\to \mathbb{R}$, pero en dimensión infinita el problema se complica. Uno de los muchos usos del Teorema de Hahn Banach, que estudiaremos a continuación, es el de justificar que $E^\ast\neq \{0\}$; sin embargo, el Teorema tiene muchas más utilidades. Con el fin de explorar más utilidades del Teorema de Hahn Banach, formularemos la pregunta de $E^\ast\neq \{0\}$ de la siguiente forma:
\\

\noindent
\textbf{Problema}\newline
Sea $E$ un espacio de Banach, $G\subset E$ un subespacio suyo y $g:G\to \mathbb{R}$ lineal y continua, ¿podemos garantizar entonces que existe $f:E\to \mathbb{R}$ lineal y continua tal que $f\big|_G = g$?\\

\noindent
Como ya vimos en la Proposición~\ref{prop:lineal_continuidad_acotacion}, que $g$ sea continua significa que $\exists k\in \mathbb{R}^+$ de forma que $|g(x)| \leq k\|x\|$ $\forall x\in G$. Para resolver el problema, necesitamos encontrar una aplicación $f:E\to \mathbb{R}$ lineal y $k'\in \mathbb{R}^+$ de forma que:
\begin{equation*}
    |f(x)| \leq k'\|x\| \qquad \forall x\in E
\end{equation*}

\begin{ejercicio}\label{ej:aplicacion_p}
    Sea $p:(E,\|\cdot \|)\to \mathbb{R}$ dada por:
    \begin{equation*}
        p(x) = k\|x\| \qquad \forall x\in E
    \end{equation*}
    Demostrar que la función $p$ verifica:
    \begin{itemize}
        \item $p(x+y) \leq p(x) + p(y) \qquad \forall x,y\in E$.
        \item $p(\lm x) = \lm p(x) \qquad \forall \lm \in \mathbb{R}^+, \forall x\in E$.
    \end{itemize}
    \begin{proof}
        Sean $x,y\in  E$ y $\lm\in \mathbb{R}^+$:
        \begin{gather*}
            p(x+y)= k\|x+y\| \leq k(\|x\| + \|y\|) = k\|x\| + k\|y\| = p(x) + p(y) \\
            p(\lm x) = k\|\lm x\| = \lm k \|x\| = \lm p(x)
        \end{gather*}
    \end{proof}
\end{ejercicio}

\noindent
Aunque no lo demostraremos, el Teorema de Hahn Banach resulta ser equivalente al axioma de elección. Para realizar la demostración del Teorema de Hahn Banach es necesario usar el Lema de Zorn, por lo que conviene realizar un breve repaso del mismo. 

\subsubsection{Lema de Zorn}
\begin{definicion}
    Sea $\emptyset \neq P$ un conjunto con una relación $\leq$ de orden, es decir, una relación reflexiva, antisimétrica y transitiva, decimos que:
    \begin{itemize}
        \item Un subconjunto $Q\subset P$ es \underline{totalmente ordenado} si:
            \begin{equation*}
                \forall a,b\in Q \Longrightarrow a\leq b\ \lor\ b\leq a
            \end{equation*}
        \item Si $Q\subset P$ y $x\in P$, decimos que $x$ es una \underline{cota superior} de $Q$ si y solo si:
            \begin{equation*}
                a\leq x \qquad \forall a\in Q
            \end{equation*}
        \item Si $m\in P$, decimos que $m$ es un \underline{elemento maximal} de $P$ si y solo si:
            \begin{equation*}
                \{x\in P : m \leq x\} = \{m\}
            \end{equation*}
        \item Diremos que $P$ es \underline{inductivo} si todo subconjunto $Q\subset P$ que sea totalmente ordenado posee una cota superior.
    \end{itemize}
\end{definicion}

\begin{lema}[de Zorn]
    Si $P$ es un conjunto no vacío con una relación de orden $\leq$ y $P$  es inductivo, entonces $P$ tiene un elemento maximal.
\end{lema}

\begin{teo}[de Hahn-Banach, versión analítica]
    Sea $E$ un espacio vectorial, sea $p:E\to \mathbb{R}$ una aplicación verificando:
    \begin{itemize}
        \item $p(x+y) \leq p(x) + p(y) \qquad \forall x,y\in E$.
        \item $p(\lm x) = \lm p(x) \qquad \forall \lm \in \mathbb{R}^+ , \quad \forall x\in E$.
    \end{itemize}
    Sea $G\subset E$ un subespacio vectorial y $g:G\to \mathbb{R}$ una aplicación lineal tal que
    \begin{equation*}
        g(x) \leq p(x) \qquad \forall x\in G
    \end{equation*}
    Entonces, $\exists f:E\to \mathbb{R}$ lineal de forma que: 
    \begin{enumerate}
        \item $f(x) \leq p(x) \qquad \forall x\in E$.
        \item $f\big|_G = g$.
    \end{enumerate}
    \begin{proof}
        Nos definimos el conjunto $P$ de todas aquellas aplicaciones lineales $h$ que tienen por dominios subespacios vectoriales de $E$ que contienen a $G$ de forma que $h\big|_G = g$ y que cumplen la desigualdad $h(x) \leq p(x) \quad \forall x\in D(h)$ (donde $D(h)$ denota el dominio de $h$); es decir:
        \begin{equation*}
            P = \left\{h:D(h) \to \mathbb{R} : \left.\begin{array}{l}
                G\subset D(h) \text{\ subespacio vectorial de\ } E \\
                h \text{\ lineal y\ } h(x) \leq p(x) \quad \forall x\in D(h) \\
                h(x) = g(x) \quad \forall x\in G
            \end{array}\right.\right\}
        \end{equation*}
        Tendremos entonces en $P$ todas aquellas alicaciones lineales que son extensiones de $g$ y que cumplen la condición de ser menores o iguales que $p$. Buscamos aplicar el Lema de Zorn sobre $P$, obteniendo un elemento maximal que luego probaremos que ha de tener como dominio $E$.\\

        \noindent
        Hemos pues de definir una relación de orden en $P$ que nos permita conseguir lo que queremos. Para ello, definiremos la relación $\leq$ de la siguiente forma:
        \begin{equation*}
            h_1 \leq h_2 \Longleftrightarrow \left\{\begin{array}{l}
                D(h_1) \subset D(h_2) \\
                h_2\big|_{D(h_1)} = h_1
            \end{array}\right. \qquad \forall h_1,h_2\in P
        \end{equation*}
        es decir, $h_1\leq h_2$ si $h_2$ es una extensión de $h_1$. Podemos comprobar que esta efectivamente es una relación de orden en $P$:
        \begin{itemize}
            \item \textbf{Reflexiva.} Si $h\in P$, trivialmente tenemos que $D(h)\subset D(h)$ y $h\big|_D(h) = h$, lo que nos dice que $h\leq h$.
            \item \textbf{Antisimétrica.} Sean $h_1,h_2\in P$ de forma que $h_1\leq h_2$ y $h_2\leq h_1$, entonces:
                \begin{equation*}
                    D(h_1)\subset D(h_2) \ \land\ D(h_2) \subset D(h_1) \Longrightarrow D(h_1) = D(h_2) 
                \end{equation*}
                Y de esta condición junto con $h_2 = h_2\big|_{D(h_2)} = h_2\big|_{D(h_1)} = h_1$ concluimos que $h_2 = h_1$.
            \item \textbf{Transitiva.} Si $h_1,h_2,h_3\in P$ con $h_1\leq h_2$ y $h_2\leq h_3$, tenemos entonces que $D(h_1)\subset D(h_2)$ y que $D(h_2)\subset D(h_3)$. La transitividad de $\subset$ nos dice que $D(h_1)\subset D(h_3)$. Ahora, si tenemos que $h_3\big|_{D(h_2)} = h_2$ y que $h_2\big|_{D(h_1)} = h_1$, obtenemos que:
                \begin{equation*}
                    h_3\big|_{D(h_1)} = h_2\big|_{D(h_1)} = h_1
                \end{equation*}
                De donde $h_1\leq h_3$.
        \end{itemize}

        \noindent
        Tratemos ahora de probar que $P$ es inductivo. Para ello, sea $Q\subset P$ un subconjunto totalmente ordenado, para buscar una cota superior consideraremos:
        \begin{equation*}
            V_0 = \bigcup_{h\in Q}D(h)
        \end{equation*}
        Vemos que $V_0$ es un subespacio vectorial de $E$, ya que si $\alpha\in \mathbb{R}$ y $u,v\in V_0$, tenemos entonces que $\exists h,h'\in Q$ de forma que $u\in D(h), v\in D(h')$. Como $Q$ es totalmente ordenado, tendremos entonces que $h\leq h'$ o que $h'\leq h$. Supondremos sin pérdida de generalidad que $h\leq h'$, lo que nos dice que $D(h)\subset D(h')$, por lo que $u\in D(h')$ y como $D(h')$ es un subespacio vectorial de $E$, tenemos entonces que:
        \begin{equation*}
            \alpha u + v \in D(h') \subset V_0
        \end{equation*}
        Una vez salvada esta cuestión, definimos $h_0:V_0\to \mathbb{R}$ por:
        \begin{equation*}
            h_0(x) = h(x) \qquad \text{si\ } x\in D(h)
        \end{equation*}
        que está bien definida, ya que si $x\in D(h_1)\cap D(h_2)$, sucederá bien $h_1 \leq h_2$ bien $h_2 \leq h_1$, luego suponiendo que $h_1\leq h_2$, tendremos que $h_2\big|_{D(h_1)} = h_1$, luego se cumplirá $h_1(x) = h_2(x)$. Además $h_0$ es lineal, ya que si $x,y\in V_0$, por ser $V_0$ espacio vectorial tendremos que $x+y\in V_0$, de donde $\exists h,h',h'' \in Q$ de forma que $x\in D(h)$, $y\in D(h')$, $x+y\in D(h'')$, con lo que:
        \begin{equation*}
            h_0(x+y) = h''(x+y) = h''(x) + h''(y) = h(x) + h'(y) = h_0(x) + h_0(y)
        \end{equation*}
        Y finalmente es claro que $h_0(x) \leq p(x)\quad \forall x\in V_0$, puesto que si $x\in V_0$, entonces $\exists h\in Q$ de forma que $x\in D(h)$, con lo que:
        \begin{equation*}
            h_0(x) = h(x) \leq p(x)
        \end{equation*}
        En definitiva, tenemos que $h_0$ es una aplicación lineal extensión de $g$ que cumple $h(x) \leq p(x)$ para todo $x\in V_0$ y con $V_0$ un subespacio vectorial de $E$ que contiene a $G$, con lo que $h_0\in P$ y además tenemos que $h\leq h_0$ $\forall h\in Q$, por lo que $h_0$ es una cota superior de $Q$, de donde tenemos que $P$ es inductivo. Por el Lema de Zorn, existe $f\in P$ elemento maximal de $P$.\\

        \noindent
        Para concluir la demostración del Teorema, nos falta probar que si $f$ es un elemento maximal de $P$ entonces $D(f) = E$. Para ello, supongamos por reducción al absurdo que fuese $D(f)\subsetneq E$, luego existe $x_0\in E\setminus D(f)$. Si consideramos\footnote{Aquí hemos usado $\mathbb{R} x_0:= \{x_0 r : r\in \mathbb{R}\}$, que es un subespacio vectorial de $E$ de dimensión 1.}:
        \begin{equation*}
            D(f) \oplus \mathbb{R}x_0
        \end{equation*}
        Tenemos que si $v\in D(f)\oplus \mathbb{R}x_0$, entonces $v$ se escribe como $v = x+tx_0$, con $x\in D(f)$ y $t\in \mathbb{R}$, lo que nos permite definir $\hat{f}:D(f)\oplus \mathbb{R}x_0\to \mathbb{R}$ dada por:
        \begin{equation*}
            \hat{f}(x+tx_0) = f(x) + t\alpha
        \end{equation*}
        Siendo $\alpha$ un número real que por ahora no concretaremos (puesto que necesitamos buscar luego una condición sobre $\alpha$ para garantizar que $\hat{f}\in P$). Veamos que $\hat{f} \in P$:
        \begin{itemize}
            \item Es automático que $\hat{f}\big|_{D(f)} = f$.
            \item $D(f)\oplus \mathbb{R}x_0$ es un subespacio vectorial de $E$ que contiene a $G$.
            \item Es fácil ver que $\hat{f}$ es lineal, ya que si $x,y\in D(f)$ y $t,t'\in \mathbb{R}$:
                \begin{align*}
                    \hat{f}(x+tx_0 + y+t'x_0) &= \hat{f}((x+y) + (t+t')x_0) = f(x+y) + (t+t')\alpha \\
                                              &= f(x) + f(y) + t\alpha + t'\alpha = \hat{f}(x+tx_0) + \hat{f}(y+t'x_0)
                \end{align*}
            \item Tenemos que ver finalmente que 
                \begin{equation}\label{eq:fgorromenorp}
                    \hat{f}(x+tx_0) \leq p(x+tx_0)\qquad \forall x\in D(f),\quad  \forall t\in \mathbb{R}
                \end{equation}
                que sucede si y solo si:
                \begin{equation*}
                    t\hat{f}(z+x_0) = \hat{f}(tz+tx_0) \leq p(tz+tx_0) = p(t(z+x_0)) \qquad \forall z\in D(f), \quad \forall t\in \mathbb{R}
                \end{equation*}
                \begin{itemize}
                    \item En el caso $t=0$ la desigualdad es obvia.
                    \item Si $t>0$, tenemos que:
                        \begin{equation*}
                            t(f(z) + \alpha) = t\hat{f}(z+x_0) \leq p(t(z+x_0)) = tp(z+x_0)  \qquad \forall z\in D(f)
                        \end{equation*}
                        que es equivalente a
                        \begin{equation*}
                            \alpha \leq -f(z) + p(z+z_0) \qquad \forall z \in D(f)
                        \end{equation*}
                    \item Si $t<0$, tenemos:
                        \begin{multline*}
                            -t(-f(z)-\alpha) = -t\hat{f}(-z-x_0) = t\hat{f}(z+x_0) \leq p(t(z+x_0)) = -tp(-z-x_0) \\ \forall z\in D(f)
                        \end{multline*}
                        que es equivalente a
                        \begin{equation*}
                            -f(z) - p(-z-x_0) \leq \alpha \qquad \forall z\in D(f)
                        \end{equation*}
                \end{itemize}
                En definitiva, ver~(\ref{eq:fgorromenorp}) es equivalente a ver que:
                \begin{equation*}
                    \sup\{f(-z)-p(-z-x_0) : z\in D(f)\} \leq \alpha \leq \inf\{-f(z)+p(z+x_0) : z\in D(f)\}
                \end{equation*}
                que a su vez equivale a:
                \begin{equation*}
                    \sup\{f(w)-p(w-x_0) : w\in D(f)\} \leq \alpha \leq \inf\{-f(z)+p(z+x_0) : z\in D(f)\}
                \end{equation*}
                Por tanto, si probamos que el supremo de la izquierda es menor o igual que el ínfimo de la derecha, elegiendo $\alpha$ cualquier valor real comprendido entre ambos (o incluso igual al supremo o al ínfimo) habremos construido una aplicación $\hat{f}$ que cumple con los tres puntos anteriores y con la condición~(\ref{eq:fgorromenorp}), que es la condición que veníamos buscando.

                Para demostrar la desigualdad entre supremo e ínfimo, basta observar que para $z,w\in D(f)$ se verifica:
                \begin{equation*}
                    f(z) + f(w) = f(z+w) \leq p(z+w) = p(z+x_0-x_0+w) \leq p(z+x_0) + p(w-x_0)
                \end{equation*}
                y despejando llegamos a que:
                \begin{equation*}
                    f(w) - p(w-x_0) \leq -f(z) + p(z+x_0) \qquad \forall z,w\in D(f)
                \end{equation*}
                Lo que demuestra que:
                \begin{equation*}
                    \sup\{f(-z)-p(-z-x_0) : z\in D(f)\} \leq \inf\{-f(z)+p(z+x_0) : z\in D(f)\}
                \end{equation*}
                Como hemos comentado anteriormente, tomando por ejemplo:
                \begin{equation*}
                    \alpha = \sup\{f(-z)-p(-z-x_0) : z\in D(f)\} \in \mathbb{R}
                \end{equation*}
                en la definición de $\hat{f}$ nos garantiza la condición~(\ref{eq:fgorromenorp}), que junto con las otras condiciones nos dice que $\hat{f}\in P$. Además, por la definición de $\hat{f}$ es claro que $f\leq \hat{f}$, donde $f$ era un elemento maximal de $P$. Hemos llegado a una \underline{contradicción}, que venía de suponer que $D(f)\subsetneq E$, por lo que $D(f)$ ha de ser igual a $E$, luego hemos encontrado la aplicación que el Teorema enunciaba, lo que concluye la demostración.
        \end{itemize}
    \end{proof}
\end{teo}

\noindent
Volviendo al caso que nos interesaba, tenemos ya respuesta al Teorema anteriormente planteado:

\begin{coro}
    Sea $E$ un espacio de Banach, $G\subset E$ un subespacio suyo y $g:G\to \mathbb{R}$ lineal y continua, existe entonces $f:E\to \mathbb{R}$ lineal y continua de forma que $f\big|_G = g$.
    \begin{proof}
        Por ser $g$ una aplicación lineal, la condición de ser continua equivale a que $\exists k>0$ de forma que:
        \begin{equation*}
            |g(x)| \leq k\|x\| \qquad \forall x\in G
        \end{equation*}
        Si tomamos ahora $p:E\to E$ dada por $p(x) = k\|x\|$ para $x\in E$, vimos en el Ejercicio~\ref{ej:aplicacion_p} que $p$ verificaba:
        \begin{itemize}
            \item $p(x+y)\leq p(x) + p(y) \qquad \forall x,y\in E$
            \item $p(\lm x)=\lm p(x) \qquad \forall \lm \in \mathbb{R}^+, \forall x\in E$
        \end{itemize}
        y la condición que hemos expresado arriba nos dice que $g(x) \leq p(x)$ para todo $x\in G$. Aplicando el Teorema de Hahn Banach, tenemos que existe una aplicación $f:E\to \mathbb{R}$ lineal que verifica:
        \begin{itemize}
            \item $f\big|_G = g$
            \item $f(x) \leq p(x)\qquad \forall x\in E$
        \end{itemize}
        falta ver que $f$ es continua para acabar la demostración. Para ello, observemos que la condición $f(x) \leq p(x) \quad \forall x\in E$ implica:
        \begin{equation*}
            -f(x) = f(-x) \leq p(-x) = k\|- x\| = k\|x\| = p(x) \qquad \forall x\in E
        \end{equation*}
        Por lo que tenemos que $|f(x)| \leq k\|x\| \quad \forall x\in E$, y vimos en la Proposición~\ref{prop:lineal_continuidad_acotacion} que esta condición para una aplicación lineal era equivalente a que la aplicación sea continua.
    \end{proof}
\end{coro}

\subsection{Versiones geométricas del Teorema}
\noindent
Aunque no lo demsotraremos, las sucesivas versiones geométricas del teorema de Hahn Banach son equivalentes a la ya vista. Para realizar la formulación del Teorema será necesario tener claros ciertos conceptos:

\begin{definicion}[Hiperplano afín]
    Sea $E$ un espacio vectorial, un hiperplano afín de $E$ es un subconjunto $H\subset E$ de la forma:
    \begin{equation*}
        H = \{x\in E : f(x) = \alpha\} = f^{-1}(\{\alpha\})
    \end{equation*}
    donde $f:E\to \mathbb{R}$ es una aplicación lineal no nula y $\alpha\in \mathbb{R}$. En dicho caso, escribiremos $H = [f=\alpha]$.
\end{definicion}

\begin{observacion}
    Cuando trabajábamos en asignaturas anteriores en espacios vectoriales de dimensión finita (digamos $n$), para nosotros un hiperplano era un subespacio vectorial de dimensión $n-1$. Ahora, si nos encontramos en un espacio vectorial $E$ genérico (no necesariamente de dimensión finita), el primer Teorema de Isomorfía de aplicaciones lineales aplicado a $f$ nos da el isomorfismo lineal
    \begin{equation*}
        E/\ker f \cong \text{Im}f
    \end{equation*}
    Como $f$ era lineal, ha de ser obligatoriamente $\text{dim}\ \text{Im}f = 1$. Observemos que en el caso $H = [f=0] = \ker f$, tenemos que $\text{dim}(E/H) = 1$, de donde si $E$ es de dimensión finita, tenemos $\text{dim}H = \text{dim}E - 1$. Si consideramos ahora $H=[f=\alpha]$ con $\alpha\neq 0$, tenemos que:
    \begin{equation*}
        H = \{x\in E : f(x) = \alpha\} = \{x+v : x\in E, v\in \ker f, f(x) = \alpha\}
    \end{equation*}
    Por lo que podemos ver $H$ como un trasladado de $\ker f$, como un hiperplano afín.
\end{observacion}

\begin{observacion}
    Notemos además que si $f:E\to \mathbb{R}$ es una aplicación lineal y continua, entonces el hiperplano $H = [f=\alpha]$ para cierto $\alpha\in \mathbb{R}$ es un hiperplano cerrado.
\end{observacion}

\noindent
La condición que nos va a interesar es buscar bajo qué condiciones cuando nos dan dos subconjuntos de un espacio normado vamos a poder separarlos mediante un hiperplano afín. Para ello, es necesario formalizar la idea de ``separar dos subconjuntos de un espacio''.

\begin{definicion}
    Sea $E$ un espacio vectorial, $A,B\subset E$, diremos que el hiperplano $H=[f=\alpha]$ separa $A$ y $B$ si:
    \begin{equation*}
        f(x) \leq \alpha \leq f(y) \qquad \forall x\in A,\quad  \forall y\in B
    \end{equation*}
    Además, diremos que la separación es estricta (o que $H$ separa estrictamente $A$ y $B$) si $\exists \varepsilon>0$ de forma que:
    \begin{equation*}
        f(x) \leq \alpha - \varepsilon < \alpha+\varepsilon \leq f(y) \qquad \forall x\in A, \quad  \forall y\in B
    \end{equation*}
\end{definicion}

\begin{teo}[Hahn Banach, primera versión geométrica]
    Sea $E$ un espacio normado, $\emptyset \neq A,B\subset E$ con $A\cap B = \emptyset $, ambos convexos y uno de ellos (digamos $A$) abierto, entonces existe un hiperplano cerrado\footnote{Luego habrá una aplicación lineal y contina $f:E\to \mathbb{R}$, por lo que $E^\ast \neq \{0\}$.} $H = [f=\alpha]$ que separa $A$ y $B$.
    \begin{proof}
        El Teorema se demuestra en varios pasos:
        \begin{description}
            \item [Paso 1.] Supongamos en una versión más débil que $B$ se reduce a un punto, es decir, existe $x_0\in E$ de forma que $B = \{x_0\}$ y que $A\subset E$ es un conjunto abierto y convexo de forma que $x_0 \notin A$.

                Elegimos $z_0\in A$ y definimos $C = A-z_0$. Se verifica (compruébese) que $C$ es convexo, abierto y $0\in C$. % // TODO: Hacer
                El punto $y_0 = x_ 0 - z_0\notin C$, de donde $y_0 \neq 0$. Por lo que $y_0\mathbb{R}$ es un subespacio vectorial de $E$ de dimensión 1. Definimos $G = \mathbb{R} y_0$ subespacio vectorial de $E$, y tomamos 
                \Func{g}{G}{\bb{R}}{ty_0}{t}
                que es una aplicación lineal y verificando $g(y_0) = 1$. La función $g$ nos permite ``separar'' el corte de $C$ con $G$ y el punto $y_0$. Si consideramos el funcional de Minkowski del conjunto $C$, observamos que:
                \begin{itemize}
                    \item Si $t\geq 0$, como $y_0\notin C$, entonces $p(y_0) \geq 1$, de donde:
                        \begin{equation*}
                            g(ty_0) = t \leq tp(y_0) \stackrel{\text{(1)}}{=} p(ty_0)
                        \end{equation*}
                    \item Si $t<0$, tenemos que:
                        \begin{equation*}
                            g(ty_0) = t < 0 \leq p(ty_0)
                        \end{equation*}
                \end{itemize}
                En cualquier caso, $g(ty_0) \leq p(ty_0)$ $\forall t\in \mathbb{R}$. Reunimos las hipótesis del Teorema de Hahn-Banach, por lo que $\exists f:E\to \mathbb{R}$ lineal tal que:
                \begin{equation*}
                    f\big|_G = g
                \end{equation*}
                y $f(y) \leq p(y) \qquad \forall y\in E$. La propiedad 2 del funcional nos dice que:
                \begin{equation*}
                    f(y) \leq p(y) \leq M\|y\| \qquad \forall y\in E
                \end{equation*}
                Si aplicamos esta propiedad para $-y$:
                \begin{equation*}
                    -f(y) = f(-y) \leq M\| -y\| = M\|y\|
                \end{equation*}
                De donde:
                \begin{equation*}
                    |f(y)| \leq M\|y\| \qquad \forall y\in E
                \end{equation*}
                Como $f$ es lineal, una Proposición que vimos anteriormente nos dice que $f$ es continua. % // TODO: Queda buscar alpha para definir el hiperplano, que será alpha = 1 para ello, habrá que probar que f(y) \leq 1 = f(y_0) ff y ii C
                % // TODO: Terminar el caso sencillo trasladando de nuevo C a A y y_0 a x_0. Para ello, usar que f(y) \leq 1 = f(y_0) ff y ii C y que C = A - z_0


%                 La función que vayamos a definir hemos de definirla en $y_0\mathbb{R}$, por lo que será de la forma:
%                 \Func{g}{y_0\mathbb{R}}{\mathbb{R}}{ty_0}{t}
%                 Buscamos aplicar el Teorema de Hahn Banach para obtener una función $f:E\to \mathbb{R}$, que al evaluarla en $y_0$ obtendremos $1$. Buscaremos que:
%                 \begin{equation*}
%                     f(x) \leq 1 = f(y_0) \qquad \forall x\in C
%                 \end{equation*}
%                 l problema es elegir la función $p$ para que $g(ty_0) \leq p(ty_0)$, con el fin de aplicar Hahn Banach, obteniendo:
%                 \begin{itemize}
%                     \item $f(ty_0) = t \qquad \forall t\in y_0\mathbb{R}$
%                     \item $f(x) \leq p(x) \qquad \forall x\in E$
%                 \end{itemize}
%                 Si buscamos $p$ de forma que $p(x)\leq 1$, tendremos lo que queríamos, por lo que el problema se reduce a buscar dicha función $p$.
            \item [Paso 2.] Volviendo al caso que nos plantea el Teorema siendo $B$ un conjunto convexo y disjunto de $A$. Tomamos:
                \begin{equation*}
                    A - B = \{a-b : a\in A, b\in B\}
                \end{equation*}
                Observemos que $0\notin A-B$, ya que $A\cap B = \emptyset $. Si conseguimos probar que $A-B$ es abierto y convexo, sabemos ya separar $A-B$ de $\{0\}$, y será sencillo terminar la prueba. Para ver que $A-B$ es abierto, se expresa de forma sencilla como unión de abiertos:
                \begin{equation*}
                    A-B = \bigcup_{b\in B} (A-b)
                \end{equation*}
                % // TODO: Hacer la convexidad
                Aplicamos el paso 1 a este caso, lo que nos permite separar $\{0\}$ de $A-B$. Como lo hacemos con una aplicación $f$ lineal, obtendremos que podemos separar $A$ de $B$. % // TODO: Terminar
        \end{description}
    \end{proof}
\end{teo}

\subsubsection{Funcional de Minkowski de un conjunto}
\noindent
En este subapartado definiremos el funcional de Minkowski de un conjunto, una cierta aplicación con propiedades interesantes que nos permite realizar la demostración de la primera versión geométrica del Teorema de Hahn Banach y que además tiene cierto interés fuera de esta demostración, como luego se pondrá de manifiesto en los ejercicios a realizar.

\begin{definicion}[Funcional de Minkowski]
    Sea $E$ un espacio normado y $C\subset E$ un conjunto convexo, abierto y con $0\in C$, definimos el funcional de Minkowski de $C$ como la aplicación $p:E\to \mathbb{R}$ dada por:
    \begin{equation*}
        p(x) = \inf \left\{\alpha\in \mathbb{R}^+ : \dfrac{x}{\alpha}\in C\right\} \qquad \forall x\in E
    \end{equation*}
\end{definicion}

\begin{observacion}
    Bajo las hipótesis de la definición del funcional de Minkowski, observamos que lo que estamos haciendo es, fijado un punto $x\in E\setminus \{0\}$, tomar la recta de origen $0$ que pasa por $x$, y si multiplicamos $x$ por un escalar positivo, nos movemos por dicha recta. En particular, si multiplicamos $x$ por el inverso de un escalar positivo, si aumentamos dicho escalar, nos estaremos acercando a $0$, y si decrementamos dicho escalar, nos alejaremos de $0$. Notemos que lo que estamos haciendo por la definición del funcional de Minkowski es tomar aquel valor más ``pequeño'' para el cual si multiplicamos $x$ por el inverso de un escalar que se queda por encima suya no nos saldremos del conjunto $C$.
    \begin{figure}[H]
        \centering
        \begin{tikzpicture}[scale=1]
          % Ejes
          \draw[-stealth] (-2.5,0) -- (3.5,0) node[right] {$x$};
          \draw[-stealth] (0,-1.5) -- (0,2.5) node[above] {$y$};

          % Conjunto convexo C: elipse centrada en (0,0), borde discontinuo
          \draw[dashed, thick] (0,0) ellipse (1.8cm and 1.0cm);

          % Marcar el origen
          \fill (0,0) circle (1.5pt) node[below left] {$0$};

          % Punto x en el primer cuadrante (fuera de C)
          \coordinate (x) at (2.2,1.5);
          \fill (x) circle (1.5pt) node[below] {$x$};

          % Semirrecta (rayo) desde el origen que pasa por x, trazo discontinuo
          % se extiende más allá de x usando la notación de cálculo
          \draw[dashed, thick, -stealth] (0,0) -- ($(0,0)!1.5!(x)$);

          % Etiqueta para C
          \node[below right] at (0.4,-0.2) {$C$};
        \end{tikzpicture}
    \end{figure}
\end{observacion}

Observemos que $p(0) = 0$. Además, el funcional de Minkowski tiene ciertas propiedades resaltables.

\begin{prop}
    Sea $E$ un espacio normado y $C\subset E$ un conjunto convexo, abierto y con $0\in C$, el funcional de Minkowski verifica:
    \begin{enumerate}
        \item $p(\lm x) = \lm p(x)\qquad \forall x\in E, \quad \forall \lm\in \mathbb{R}^+$
        \item $\exists M>0$ tal que $0\leq p(x) \leq M\|x\|\qquad \forall x\in E$
        \item $C = \{x\in E : p(x) < 1\}$
        \item $p(x+y) \leq p(x) + p(y)\qquad \forall x,y\in E$
    \end{enumerate}
    \begin{proof}
        Demostramos cada una de las propiedades:
        \begin{enumerate}
            \item Para la primera, basta usar que $\lm > 0$ y observar:
                \begin{align*}
                    p(\lm x) &= \inf \left\{\alpha\in \mathbb{R}^+ : \dfrac{x}{\nicefrac{\alpha}{\lm}} = \dfrac{\lm x}{\alpha} \in C\right\} = \inf\left\{\lm \alpha\in \mathbb{R}^+ : \dfrac{x}{\alpha}\in C\right\} \\
                             &= \lm\inf\left\{\alpha\in \mathbb{R}^+ : \dfrac{x}{\alpha}\in C\right\} = \lm p(x) \qquad \forall x\in E
                \end{align*}
            \item Como $0\in C$ es abierto, $\exists r>0$ de forma que $B(0,r)\subset C$. Si tomamos:
                \begin{equation*}
                    \alpha > \dfrac{\|x\|}{r} \Longrightarrow \left\|\dfrac{x}{\alpha}\right\| < r \Longrightarrow \dfrac{x}{\alpha} \in B(0,r)\subset C
                \end{equation*}
                Por tanto:
                \begin{equation*}
                    \left]\dfrac{\|x\|}{r},+\infty\right[ \subset \{\alpha\in \mathbb{R}^+:\frac{x}{\alpha}\in C \}
                \end{equation*}
                de donde el ínfimo de la derecha será menor o igual que el ínfimo de la izquierda:
                \begin{equation*}
                    p(x) \leq \dfrac{\|x\|}{r}
                \end{equation*}
                Tomamos $M = \frac{1}{r}$.
            \item Queremos ver que $C= \{x\in E : p(x) < 1\}$:
                \begin{description}
                    \item [$\supset )$] Sea $x\in E$ con $p(x)<1$, el ínfimo nos garantiza la existencia de $\alpha_0\in \mathbb{R}^+$ de forma que $\alpha_0 < 1$ y $\frac{x}{\alpha_0}\in C$. Como $C$ es convexo y $0\in C$, tenemos entonces que:
                        \begin{equation*}
                            x = \alpha_0\dfrac{x}{\alpha_0}+(1-\alpha_0)\cdot 0 \in C
                        \end{equation*}
                    \item [$\subset )$] % // TODO: Seguir
                \end{description}
        \end{enumerate}
    \end{proof}
\begin{enumerate}
    \item 
        Por doble inclusión:
        \begin{description}
            \item [$\subset )$] EJERCICIO ($(1+\varepsilon)x$). Si $x\in C$, como $C$ es abierto, $\exists r>0$ de forma que $B(x,r)\subset C$. Sea $\varepsilon>0$, vemos que:
                \begin{equation*}
                    \left\|\dfrac{x}{1+\varepsilon} - x\right\| = \left\|\dfrac{-\varepsilon x}{1+\varepsilon}\right\| = \dfrac{\varepsilon}{1+\varepsilon}\|x\|
                \end{equation*}
                Elegimos $\varepsilon_0>0$ de forma que $\frac{\varepsilon_0}{1+\varepsilon_0} < \varepsilon_0 < \frac{r}{\|x\|+1}$ (podría ser $\|x\| = 0$), entonces:
                \begin{equation*}
                    \left\|\dfrac{x}{1+\varepsilon}-x\right\| < r \qquad \forall \varepsilon\in \left]0,\varepsilon_0\right]
                \end{equation*}
                Por tanto, $\frac{x}{1+\varepsilon}\in B(x,r)\subset C$, $\forall \varepsilon\in \left]0,\varepsilon_0\right]$, es decir, en la definición de $p$ podemos tomar $\alpha = 1+\varepsilon_0$, de donde el ínfimo puede ser más pequeño:
                \begin{equation*}
                    p(x) \leq 1+\varepsilon \qquad \forall \varepsilon\in \left]0,\varepsilon_0\right]
                \end{equation*}
            \item [$\supset )$] A partir de la definición de $p$, si $p(x) < 1$, tenemos entonces que:
                \begin{equation*}
                    \exists \alpha_0 < 1 : \dfrac{x}{\alpha_0} \in C
                \end{equation*}
                Como $C$ es convexo y $0\in C$, tenemos entonces que:
                \begin{equation*}
                    x = \alpha_0 \frac{x}{\alpha_0} + (1-\alpha_0)0 \in C
                \end{equation*}
        \end{description}
    \item $p(x+y)\leq p(x) + p(y)$
        Por la definición de $p(x)$, sabemos que el conjunto que usamos para definir $p8x$ es un intervalo desde $p(x)$ (no sabemos si abierto o cerrado) hasta $+\infty$. Lo que sí sabemos, es que:
        \begin{equation*}
            \left\{\alpha>0 : \frac{x}{\alpha}\in C\right\} = \left[p(x),+\infty\right[ \qquad \forall \varepsilon>0
        \end{equation*}
        Es decir:
        \begin{equation*}
            \dfrac{x}{p(x) + \varepsilon}\in C \qquad \forall \varepsilon>0
        \end{equation*}
        Usando el apartado 3:
        \begin{equation*}
            p\left(\dfrac{x}{p(x)+\varepsilon}\right) < 1
        \end{equation*}
        Como $C$ es convexo, si tomamos $\frac{y}{p(y)+\varepsilon}\in C$ y $t = \frac{p(x)+\varepsilon}{p(x) + p(y) + 2\varepsilon} \leq 1$, tenemos entonces que:
        \begin{equation*}
            \dfrac{x+y}{p(x) + p(y) + 2\varepsilon} = t\dfrac{x}{p(x)+\varepsilon} + (1-t)\dfrac{y}{p(y)+\varepsilon}\in C\qquad \forall x,y\in E
        \end{equation*}
        Usando de nuevo la propiedad 3:
        \begin{equation*}
            p(x+y) \leq p(x) + p(y) + 2\varepsilon \qquad \forall x,y\in E, \quad \forall \varepsilon>0
        \end{equation*}
        De donde deducimos la propiedad buscada.
\end{enumerate}    
\end{prop}


\begin{ejemplo}
    Si $C = B(0,1)$, entonces $p(x) = \|x\|\qquad \forall x\in E$.
\end{ejemplo}

\begin{ejercicio}
    Parece ser que $p$ tiene propiedades deseables para ser una norma para cualquier conjunto $C$, ¿qué tiene que verificar un conjunto $C$ para conseguir que $p$ sesa na norma? El conjunto $C$ parece ser la bola unidad a partir de la norma $p$.
\end{ejercicio}

\begin{teo}[Hahn Banach, segunda versión geométrica]
    Sea $E$ un espacio normado, $\emptyset \neq A, B\subset E$, $A\cap B=\emptyset $, ambos convexos, $A$ cerrado y $B$ compacto, entonces existe un hiperplano que separa estrictamente $A$ y $B$:
    ($\exists f:E\to \mathbb{R}$ lineal y continua, $\exists \alpha\in \mathbb{R}, \exists \varepsilon>0$) de forma que:
    \begin{equation*}
        f(a) \leq \alpha-\varepsilon < \alpha < \alpha + \varepsilon \leq f(b) \qquad \forall a\in A, \forall b\in B
    \end{equation*}
    Antes teníamos que: 
    ($\exists f:E\to \mathbb{R}$ lineal y continua, $\exists \alpha\in \mathbb{R}$ tales que $f(a)\leq \alpha \leq f(b) \quad \forall a\in A, b\in B$)
    Es decir, que no estaban separados estrictamente
    \begin{proof}
        Sea $C=A-B$, como ambos son convexos, vimos en la demostración del otro teorema que entonces $C$ es convexo.
        \begin{equation*} % // TODO: HACER
            \left.\begin{array}{l}
                A \text{\ cerrado} \\
                B \text{\ compacto}
            \end{array}\right\} \Longrightarrow C \text{\ cerrado}
        \end{equation*}
        Sabemos además, al igual que antes, que $0\notin C$. Como $C$ es cerrado, $E\setminus C$ es abierto y $0\in E\setminus C$, de donde $\exists r>0$ de forma que $B(0,r)\cap C = \emptyset $. Si usamos la primera versión geométrica del Teorema de Hahn Banach para los conjuntos $B(0,r)$ y $C$, podemos decir entonces que estos dos conjuntos podemos separarlos por un hiperplano (aunque no estrictamente, pero la bola $B(0,r)$ nos dará la propiedad). % // TODO: TERMINAR LA DEMO COMO EJERCICIO
    \end{proof}
\end{teo}

Parece un Teorema muy interesante, pero en dimensión infinita apenas hay conjuntos compactos.

\noindent
En el libro de Brezis después da una serie de consecuencias del Teorema de Hahn Banach. Hacer como ejercicios los corolarios. % // TODO: EJercicios

\section{Ejercicios}

% // TODO: Ejercicios
\begin{ejercicio} % 1.1 de Brezis
    \begin{equation*}
        F(x) = \{f\in E^\ast : \|f\| = \|x\| \text{\ y\ } \langle f,x \rangle =\|x\|^2\}
    \end{equation*}
    $F(x)$ es no vací por el Corolario 1.3.
    \begin{enumerate}[label=\alph*)]
        \item Probar $\tilde{F}(x) = \{f\in E^\ast : \|f\| \leq \|x\| \text{\ y\ } \langle f,x \rangle =\|x\|^2\}$ y que $F(x)\neq \emptyset $, cerrado y convexo.
            \begin{description}
                \item [$\subset )$] 
                \item [$\supset )$] $f\in \tilde{F}(x)$:
                    \begin{equation*}
                        \|x\|^2 = \langle f,x \rangle  \leq \|f\|\|x\| \Longrightarrow \|x\| \leq \|f\|
                    \end{equation*}
                    \begin{itemize}
                        \item $\forall x_0\in E~\exists f_0\in E^\ast$ de forma que $\|f_0\| = \|x_0\|$ y $\langle f_0,x_0 \rangle =\|x_0\|^2$.
                        \item Tomamos $\{f_n\}_{n\in \mathbb{N}} \subseteq F(x)$ de forma que $\{f_n\}\to f$, entonces:
                            \begin{equation*}
                                \|f_n\| = \|x\| \qquad \forall n \Longrightarrow \|x\| =  \lim_{n\to\infty}\|f_n\| = \|\lim_{n\to\infty}f_n\| = \|f\|
                            \end{equation*}
                        \item
                            \begin{equation*}
                                \langle f_n,x \rangle =\|x\|^2\quad \forall n \Longrightarrow \|x\|^2 = \lim_{n\to\infty}\langle f_n,x \rangle  = \langle \lim_{n\to\infty}f_n ,x  \rangle = \langle f,x \rangle 
                            \end{equation*}
                        \item Nos preguntamos si tomado $t\in [0,1]$, entonces:
                            \begin{equation*}
                                (1-t)f + tg \in F(x), \qquad f,g\in F(x)
                            \end{equation*}
                            \begin{gather*}
                                \|(1-t)f + tg\| = (1-t)\|f\| + t\|g\| = \|x\| \\
                                \langle (1-t)f+tg,x \rangle  = (1-t)\langle f,x \rangle  + t\langle g,x \rangle = \|x\|^2
                            \end{gather*}
                            De donde $F(x)$ es convexo.
                    \end{itemize}
            \end{description}
        \item Probar que si $E^\ast$ es estrictamente convexo entonces solo tiene un único elemento. Es decir, si:
            \begin{equation*}
                g_1,g_2\in E^\ast : \|g_1\| = 1 = \|g_2\|, g_1\neq g_2 \Longrightarrow \|(1-t)g_1 + tg_2\| < 1 \qquad \forall t\in \left]0,1\right[
            \end{equation*}
            Suponemos pues que $\exists f_1, f_2 \in F(x)$ con $f_1\neq f_2$, de donde:
            \begin{gather*}
                \langle f_1,x \rangle = \|x\|^2 = \langle f_2,x \rangle  \\
                \|f_1\| = \|x\| = \|f_2\|
            \end{gather*}
            Tomamos los unitarios:
            \begin{equation*}
                g_1 = \dfrac{f_1}{\|x\|}, \qquad g_2 = \dfrac{f_2}{\|x\|}
            \end{equation*}
            Y además tenemos que:
            \begin{equation*}
                \langle g_1,x \rangle  = \|x\| = \langle g_2,x \rangle 
            \end{equation*}
            si tomamos:
            \begin{equation*}
                h_t = (1-t)g_1 + tg_2, \qquad t\in \left]0,1\right[
            \end{equation*}
            tenemos entonces por la concavidad estricta que $\|h_t\| < 1$ y si hacemos el producto escalar:
            \begin{equation*}
                \|x\| = \langle h_t,x \rangle  \leq \|h_t\| \|x\| \Longrightarrow \|h_t\| \geq 1
            \end{equation*}
            Lo que contradice la concavidad estricta, que venía de suponer que $F(x)$ tenía más de un elemento.
    \end{enumerate}
\end{ejercicio}

% // TODO: Todo espacio de hilbert es estrictamente convexo

\begin{coro} % Corolario 1.3
    $\forall x_0\in E$ $\exists f_0\in E^\ast$ de forma que:
    \begin{equation*}
        \|f_0\| = \|x_0\| \quad \text{y} \quad \langle f_0,x_0 \rangle  = \|x_0\|^2
    \end{equation*}
    \begin{proof}
        \begin{itemize}
            \item Si $x_0 = 0$, vale tomar $f_0 = 0$.
            \item Supuesto que $x_0\neq 0$, sea $G = \mathbb{R} x_0$, defino $g:G\to \mathbb{R}$ dada por:
                \begin{equation*}
                    g(tx_0) = t\|x_0\|^2 \qquad \forall t\in \mathbb{R}
                \end{equation*}
                es fácil ver que $g$ es lineal. Definimos $p:E\to \mathbb{R}$ dada por:
                \begin{equation*}
                    p(y) = \|y\| \|x_0\| \qquad \forall y\in E
                \end{equation*}
                Veamos que cumple las propiedades:
                \begin{itemize}
                    \item Si $\lm > 0$, entonces:
                        \begin{equation*}
                            p(\lm y) = \|\lm y\|\|x_0\| = \lm \|y\| \|x_0\| = \lm p(y) \qquad \forall y\in E
                        \end{equation*}
                    \item Ahora:
                        \begin{multline*}
                            p(y_1+y_2) = \|y_1+y_2\|\|x_0\| \leq \|y_1\|\|x_0\| + \|y_2\|\|x_0\| = p(y_1) + p(y_2) \\ \forall y_1,y_2\in E
                        \end{multline*}
                \end{itemize}
                Veamos ahora que:
                \begin{itemize}
                    \item Si $t\geq 0$, entonces $g(tx_0) = t\|x_0\|^2 = p(tx_0)$.
                    \item Si $t<0$, entonces $g(tx_0) = t\|x_0\|^2<0 \leq p(tx_0)$.
                \end{itemize}
                Es decir, $g(tx_0)\leq p(tx_0)$ $\forall t\in \mathbb{R}$. Aplicamos el Teorema de Hahn Banach, por lo que $\exists f:E\to \mathbb{R}$ de forma que $f\big|_G = g$ y $f(y)\leq p(y)\quad \forall y\in E$.\\

                \noindent
                Vemos fácilmente que $f$ es continua, por estar dominada por la función $p$, luego $f\in E^\ast$.
                \begin{itemize}
                    \item $f(x_0) = g(x_0) = \|x_0\|^2$.
                    \item Como $f(y)\leq \|x_0\|\|y\| \quad \forall y\in E$, cambiando $y$ por $-y$ conseguimos:
                        \begin{equation*}
                            |f(y)| \leq \|x_0\|\|y\|
                        \end{equation*}
                        De donde $\|f\| \leq \|x_0\|$. Si ahora vemos:
                        \begin{equation*}
                            f(x_0) = \|x_0\|^2 = \|x_0\|\|x_0\|
                        \end{equation*}
                        deducimos que $\|f\| = \|x_0\|$.
                \end{itemize}
        \end{itemize}
    \end{proof}
\end{coro}

% // TODO: SEguir con estos ejercicios, primero con los corolarios del Tª Hahn Banach
