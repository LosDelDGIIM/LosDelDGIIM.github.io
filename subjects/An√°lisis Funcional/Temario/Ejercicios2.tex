\section{Principio de acotación uniforme y Tª de la gráfica cerrada}
\begin{ejercicio}
    % (Continuidad de funciones convexas)\newline
    % Sea $E$ un espacio de Banach y sea $\varphi:E\to \left]-\infty,+\infty\right]$ 
\end{ejercicio}

\begin{ejercicio}
    Sea $E$ un espacio vectorial y sea $p:E\to \mathbb{R}$ una función que cumple las siguientes propiedades:
    \begin{enumerate}
        \item $p(x+y)\leq p(x)+p(y)\quad \forall x,y\in E$.
        \item Para cada $x\in E$ fijo la función $\lm\longmapsto p(\lm x)$ es continua.
        \item Siempre que una sucesión de puntos de $E$ $\{y_n\}$ verifique que $\{p(y_n)\}\to 0$, entonces $\{p(\lm y_n)\}\to 0$ para cada $\lm \in \mathbb{R}$.
    \end{enumerate}
    Supongamo que $\{x_n\}$ es una sucesión de puntos de $E$ de forma que $\{p(x_n)\}\to 0$ y $\{\alpha_n\}$ es una sucesión de números reales acotada. Probar que $p(0)=0$ y que $\{p(\alpha_n x_n)\}\to 0$.\newline
    (\textbf{Pista:} Dado $\varepsilon>0$, considera los conjuntos:
    \begin{equation*}
        F_n = \{\lm \in \mathbb{R} : |p(\lm x_k)| \leq \varepsilon\quad \forall k\geq n\}
    \end{equation*}
    Deduce que si $\{x_n\}$ es una sucesión de $E$ de forma que $\{p(x_n-x)\}\to 0$ para algún $x\in E$ y $\{\alpha_n\}$ es una sucesión de forma que $\{\alpha_n\}\to \alpha$, entonces $\{p(\alpha_n x_n)\}\to p(\alpha x)$.)\\

    \noindent
    Usaando que $\{x_n\}$ y 3 con $\lm = 0$ obtenemos que: % // TODO: suponer en 2 que \lm \neq 0
    \begin{equation*}
        \{p(0\cdot x_n)\} = \{p(0)\} \to 0
    \end{equation*}
    de donde $p(0) = 0$. Siguiendo la pista, dado $\varepsilon>0$ definimos:
    \begin{equation*}
        F_n = \{\lm \in \mathbb{R} : |p(\lm x_k)| \leq \varepsilon\quad \forall k\geq n\}
    \end{equation*}
    Por reducción al absurdo, supongamos que $\{p(\alpha_n x_n)\}\not\to 0$, de donde existe una parcial con $|p(\alpha_{\sigma(n)}x_{\sigma(n)})|\geq \varepsilon$, para todo $n\in \mathbb{N}$. Como $\{\alpha_n\}$ está acotada, el Teorema de Weierstrass nos permite encontrar una parcial convergente. Supongmos que la parcial $\sigma$ verifica esto, con lo que $\{\alpha_{\sigma(n)}\}\to \alpha$. Observemos que:
    \begin{itemize}
        \item $F_n$ es cerrado para cada $n\in \mathbb{N}$, ya que:
            \begin{equation*}
                F_n = \bigcap_{k\geq n}\{\lm \in \mathbb{R} : |p(\lm x_k)|\leq \varepsilon\}
            \end{equation*}
            Como $(2)$ nos dice que $\lm \longmapsto p(\lm x_k)$ es continua, tenemos que cada uno de dichos conjuntos son cerrados, como preimagen de un conjunto cerrado por una función continua.
        \item Veamos que $\bigcup_{n\geq 1}F_n = \mathbb{R}$. Para ello, tomamos $\lm \in \mathbb{R}$ y como por hipótesis $\{p(\lm x_n)\}\to 0$, con lo que existe $n_0\in \mathbb{N}$ de forma que:
            \begin{equation*}
                |p(\lm x_n)| < \varepsilon \qquad \forall n\geq n_0
            \end{equation*}
            Luego $\lm\in  F_{n_0}$
    \end{itemize}
    Por el contrarrecíproco del Lema de Baire, existe $F_{\overline{n}}$ con $F_{\overline{n}}^\circ \neq \emptyset $, por lo que existen $\lm_0 \in \mathbb{R}$ y $\delta>0$ de forma que
    \begin{equation*}
        B(\lm_0, \delta)\subset F_{\overline{n}}^\circ
    \end{equation*}
    En otras palabras:
    \begin{equation*}
        (|p((\lm_0+t)x_{\sigma(k)})| \leq \varepsilon\quad \forall k\geq n_0 ) \quad \forall t\in \left]-\delta,\delta\right[
    \end{equation*}
    Ahora:
    \begin{equation*}
        p(\alpha_{\sigma(k)}x_{\sigma(k)})\leq p((\lm_0+\alpha_{\sigma(k)}-\alpha)x_{\sigma(k)}) + p((\alpha-\lm_0)x_{\sigma(k)})
    \end{equation*}
    con $\{ p((\alpha-\lm_0)x_{\sigma(k)})\}\to 0$ y podemos acotar el primer sumando en valor absoluto:
    \begin{equation*}
        |p((\lm_0+\alpha_{\sigma(k)}-\alpha)x_{\sigma(k)}) |\leq \varepsilon
    \end{equation*}
    Luego:
    \begin{equation*}
        p((\lm_0 + \alpha_{\sigma(k)} - \alpha)x_{\sigma(k)}) \leq p((\lm_0-\alpha)x_{\sigma(k)}) + p(\alpha_{\sigma(k)}x_{\sigma(k)})
    \end{equation*}
    de donde:
    \begin{equation*}
        p(x_{\sigma(k)}x_{\sigma(k)}) \geq p((\lm_0 + \alpha_{\sigma(k)}-\alpha)x_{\sigma(k)}) - p((\lm_0-\alpha)x_{\sigma(k)})
    \end{equation*}
    de forma que el segundo sumando tiende a 0 y el primero está acotado en valor absoluto por $\varepsilon$, de donde deducimos:
    \begin{equation*}
        p(\alpha_{\sigma(k)}x_{\sigma(k)}) \leq 2\varepsilon
    \end{equation*}
    Seguimos:
    \begin{equation*}
        p(\alpha_n x_n) - p(\alpha x) \leq p(\alpha_n (x_n - x)) + \underbrace{p(\alpha_n x) - p(\alpha  x)}_{\text{tiende a\ } 0}
    \end{equation*}
    Además, como $\{\alpha_n\}$ está acotada y $\{x_n - x\}\to 0$, nos queda simplemente acotar por debajo para aplicar el Lema del Sandwich:
    \begin{equation*}
        p(\alpha_n x) \leq p(\alpha_n (x-x_n)) + p(\alpha_n x_n)
    \end{equation*}

    de donde:
    \begin{equation*}
        p(\alpha_n x) - p(\alpha_n(x_n - x)) - p(\alpha x) \leq p(\alpha_n x_n) - p(\alpha x)
    \end{equation*}
    de donde $\{p(\alpha_n x_n)\}\to p(\alpha x)$.
\end{ejercicio}

\begin{ejercicio}
    Sean $E$ y $F$ dos espacios de Banach y $\{T_n\}$ una sucesión en $L(E,F)$. Supongamos que para todo $x\in E$ se tiene que $\{T_nx\}$ converge a un cierto límite $Tx$. Probar que si $\{x_n\}\to x$ en $E$, entonces $\{T_n(x_n)\}\to Tx$ en $F$.\\

    \noindent
    Dado $x\in E$, como $\{T_nx\}$ es convergente a $Tx$, entonces $\{\|T_nx\|\}\to \|Tx\|$, luego:
    \begin{equation*}
        \sup_{n\in \mathbb{N}}\|T_nx\| < \infty
    \end{equation*}
    Por el Principio de acotación uniforme, tenemos que $C=\sup\limits_{n\in \mathbb{N}}\|Tn\| < \infty$, de donde:
    \begin{align*}
        \|T_nx_n - Tx\| &= \|T_nx_n - T_nx + T_nx - Tx\| \leq \|T_nx_n - T_nx\| + \|T_nx - Tx\| \\
                        &= \|T_n(x_n - x)\| + \|T_nx-Tx\| \leq C\|x_n-x\| + \|T_nx-Tx\|
    \end{align*}
    Como $\|x_n-x\|, \|T_nx-Tx\|\to 0$, tenemos pues que $\|T_nx_n - Tx\| \to 0$, de donde $\{T_n(x_n)\}\to Tx$.
\end{ejercicio}

\begin{ejercicio}\label{ej:4_rel2} % // TODO: Ejercicio de examen
    Sean $E,F$ dos espacios de Banach y sea $a:E\times F\to \mathbb{R}$ una forma bilineal que verifica:
    \begin{enumerate}
        \item Para cada $x\in E$, la aplicación $fx:y\longmapsto a(x,y)$ es continua.
        \item Para cada $y\in F$, la aplicación $f_y:x\longmapsto a(x,y)$ es continua.
    \end{enumerate}
    Probar que existe una constante $C\geq 0$ de forma que:
    \begin{equation*}
        |a(x,y)| \leq C\|x\|\|y\| \qquad \forall x\in E, \quad \forall y\in F
    \end{equation*}
    (\textbf{Pista:} Introduce un operador lineal $T:E\to F^\ast$ y prueba que $T$ está acotada con ayuda del Corolario~\ref{coro:entonces_Bast_acotado}).\\

    \begin{description}
        \item [Opción 1.] Es claro que $f_x$ es lineal, con lo que $f_x\in F^\ast$.
            \Func{T}{E}{F^\ast}{x}{f_x}
            Queremos ver que si $B\subset E$ es acotada entonces $T(B)$ es acotada. Para ello:
            \begin{equation*}
                \langle T(B),y \rangle  = \{\langle f_x,y \rangle :f_x\in T(B)\} = \{a(x,y):x\in B\} = \{f_y(x) : x\in B\}
            \end{equation*}

            de donde:
            \begin{equation*}
                |f_y(x)| \leq M \|x\| \qquad \forall x\in E
            \end{equation*}
            Luego por el Corolario~\ref{coro:entonces_Bast_acotado} tenemos que $T(B)$ está acotado, así como $T$ es continua, por lo que existe $C\geq 0$ tal que $\|T(x)\|\leq C\|x\|\quad \forall x\in E$. Luego:
            \begin{equation*}
                \|f_x\| = \sup_{\|y\|\leq 1}|f_x(y)| \leq C\|x\|
            \end{equation*}

            de donde:
            \begin{equation*}
                \left|a\left(x,\frac{y}{\|y\|}\right)\right| \leq C\|x\| \Longrightarrow a(x,y) \leq C\|x\|\|y\|
            \end{equation*}
        \item [Opción 2.] Definimos:
            \Func{T}{E}{F^\ast}{x}{f_x}
            Es claro que $T$ es lineal por ser $a$ bilineal. Lo que haremos será probar que $T$ es continua tratando de usar el Teorema de la Gráfica cerrada. Recordamos que:
            \begin{equation*}
                Gr(T) = \{(x,Tx) : x\in E\}
            \end{equation*}
            Para ver que $Gr(T)$ es cerrado, tomamos una sucesión de puntos de $Gr(T)$: $\{(x_n,Tx_n)\}$ que suponemso convergente a un punto $(x,w)\in Gr(T)$. Tenemos entonces que:
            \begin{equation*}
                \|x_n-x\|, \|Tx_n - w\| \to 0
            \end{equation*}
            De la segunda podemos deducir que $\langle Tx_n - w,y \rangle\to 0 \quad \forall y\in F$ :
            \begin{equation*}
                \langle Tx_n - w,y \rangle  = \langle Tx_n,y \rangle - \langle w-y \rangle  = a(x_n,y) - \langle w,y \rangle 
            \end{equation*}
            Por lo que $\{a(x_n,y)\}\to \langle w,y \rangle $, pero por la propiedad 2 se tiene que $\{a(x_n,y)\}\to a(x,y)$, de donde deducimos que $\langle T_x,y \rangle= a(x,y) = \langle w,y \rangle $ $\forall y\in F$, de donde deducimos que $w=T_x$, con lo que $(x,w)\in Gr(T)$, luego $Gr(T)$ es cerrada y por el Teorema de la gráfica cerrada concluimos que $T$ es continua.
    \end{description}
\end{ejercicio}

\begin{ejercicio}\label{ej:5_rel2}
    Sea $E$ un espacio de Banach y sea $\{\varepsilon_n\}$ una sucesión de reales positivos de forma que $\{\varepsilon_n\}\to 0$. Además, sea $\{f_n\}$ una sucesión de elementos de $E^\ast$ que cumple la propiedad:
    \begin{center}
        $\exists r>0~\forall x\in E$ con $\|x\|<r$, $\exists C(x)\in \mathbb{R}$ de forma que $\langle f_n,x \rangle \leq \varepsilon_n\|f_n\| + C(x)\qquad \forall n\in \mathbb{N}$
    \end{center}
    Prueba que la sucesión $\{f_n\}$ está acotada.\newline
    (\textbf{Pista:} Introduce $g_n = \frac{f_n}{(1+\varepsilon_n\|f_n\|)}$).
\end{ejercicio}

\begin{ejercicio}\label{ej:6_rel2}
    (Operadores no lineales monótonos y localmente acotados)\newline
    Sea $E$ un espacio de Banach y sea $D(A)$ cualquier subconjunto de $E$. Una aplicación (no lineal) $A:D(A)\subseteq E\to E^\ast$ se dice ``monótona'' si verifica
    \begin{equation*}
        \langle Ax - Ay,x-y \rangle \geq 0 \qquad \forall x,y\in D(A)
    \end{equation*}
    \begin{enumerate}
        \item Sea $x_0\in \Int D(A)$. Prueba que existen dos constantes $R>0$ y $C$ de forma que
            \begin{equation*}
                \|Ax\|\leq C \qquad \forall x\in D(A) \quad \text{con}\quad \|x-x_0\|<R
            \end{equation*}
            (\textbf{Pista:} Razona por reducción al absurdo y construye una sucesión $\{x_n\}$ de puntos de $D(A)$ de forma que $\{x_n\}\to x_0$ y $\{\|Ax_n\|\}\to \infty$. Elije $r>0$ de forma que $B(x_0,r)\subset D(A)$. Usa la monotonía de $A$ en $x_n$ y en $(x_0+r)$ con $\|x\|<r$. Aplica el Ejercicio~\ref{ej:5_rel2}).
        \item Prueba la misma conclusión para un punto $x_0\in \Int(\conv D(A))$.
        \item Extiende la conclusión de la pregunta 1 al caso de que $A$ sea multivaluada, es decir, para cada $x\in D(A)$, $Ax$ es un conjunto no vacío de $E^\ast$, en este caso la monotonía se define como sigue:
            \begin{equation*}
                \langle f-g,x-y \rangle \geq 0 \qquad \forall x,y\in D(A), \quad \forall f\in Ax, \quad \forall g\in Ay
            \end{equation*}
    \end{enumerate}

    \noindent
    \textbf{Solución.}
    \begin{enumerate}
        \item Sea $x_0\in \Int D(A)$. Prueba que existen dos constantes $R>0$ y $C$ de forma que
            \begin{equation*}
                \|Ax\| \leq C \qquad \forall x\in D(A) \quad \text{con}\quad \|x-x_0\|<R
            \end{equation*}

            Por reducción al absurdo, supongamos que:
            \begin{equation*}
                \forall R,C\geq 0\quad \exists x\in D(A) \quad \text{con}\quad \|x-x_0\|<R \quad \text{y}\quad  \|Ax\|>C
            \end{equation*}
            Por tanto, para cada $n\in \mathbb{N}$
            \begin{equation*}
                \exists x_n\in D(A) \quad \text{con}\quad \|x_n-x_0\|<\frac{1}{n} \quad \text{y}\quad  \|Ax_n\|>C
            \end{equation*}
            Por lo que $\{x_n\}\to x_0$ y $\{\|Ax_n\|\}\to \infty$. Como $x_0\in \Int D(A)$, sea $r>0$ de forma que $B(x_0,r)\subset D(A)$, si tomamos $x\in E$ con $\|x\|<r$ tendremos entonces que:
            \begin{equation*}
                \|x_0 + x - x_0\| = \|x\| < r \Longrightarrow x_0+x\in B(x_0,r)\subset D(A)
            \end{equation*}
            Como $A$ es monótona, tenemos que:
            \begin{align*}
                0&\leq \langle Ax_n-A(x_0+x),x_n-x_0-x \rangle  \\
                 &= \langle Ax_n, x \rangle  + \langle Ax_n, x_n-x_0 \rangle  + \langle A(x_0+x),x \rangle  + \langle A(x_0+x),x_0-x_n \rangle 
            \end{align*}

            de donde:
            \begin{align*}
                \langle Ax_n,x \rangle  &\leq \|Ax_n\| \|x_n-x_0\| + \|A(x_0+x)\|\|x\| + \|A(x_0+x)\|\|x_0-x_n\| \\
                                        &= \|Ax_n\| \|x_n - x_0\| + \|A(x_0+x)\|(\|x\| + \|x_0-x_n\|)\\
                                        &\stackrel{(\ast)}{\leq} \|Ax_n\| \underbrace{\|x_n - x_0\|}_{\varepsilon_n} + \underbrace{\|A(x_0+x)\|(\|x\| + 1)}_{C(x)}
            \end{align*}
            Donde en $(\ast)$ hemos usado que $\|x_n-x_0\|<1 \quad \forall n\in \mathbb{N}$. Aplicando el Ejercicio~\ref{ej:5_rel2}, tenemos que $\{Ax_n\}$ está acotada, contradicción con que $\{\|Ax_n\|\}\to \infty$, por lo que tenemos el primer apartado.
        \item Prueba la misma conclusión para un punto $x_0\in \Int(\conv D(A))$.

            Tenemos que $\conv D(A)$ es el cierre convexo de $D(A)$:
            \begin{equation*}
                \conv D(A) = \left\{\sum_{k=1}^{n}\lm_k v_k : v_k \in D(A), \text{\ con\ } \sum_{k=1}^{n}\lm_k = 1\right\}
            \end{equation*}
            Repetiremos la misma prueba que en el apartado 1, cambiando un poco el final. Por Reducción al absurdo, de la misma forma podemos encontrar $\{x_n\}\to x_0$ con $\|x_n-x_0\| \leq 1$ para todo $n\in \mathbb{N}$ y $\{\|Ax_n\|\}\to \infty$. Como $x_0\in \Int(\conv D(A))$, sea $r>0$ de forma que $B(x_0,r)\subset \conv D(A)$, si tomamos $x\in E$ con $\|x\|<r$ tendremos que $x_0+x\in \conv D(A)$, por lo que existen $\lm_1,\ldots,\lm_n\in \mathbb{R}$, $v_1,\ldots, v_n\in D(A)$ con:
            \begin{equation*}
                x_0 + x = \sum_{k=1}^{n} \lm_k v_k \qquad \sum_{k=1}^{n}\lm_k = 1
            \end{equation*}
            Fijado $k\in \{1,\ldots,n\}$, podemos usar la monotonía de $A$:
            \begin{equation*}
                0\leq \langle Ax_n - Av_k, x_n - v_k \rangle  \Longrightarrow \langle Ax_n,x_n-v_k \rangle  \geq \langle Av_k, x_n-v_k \rangle 
            \end{equation*}
            Por lo que:
            \begin{equation*}
                \lm_k \langle Ax_n,x_n-v_k \rangle  \geq \lm_k\langle Av_k, x_n-v_k \rangle \qquad \forall k \in \{1,\ldots,n\}
            \end{equation*}
            de donde:
            \begin{equation*}
                \sum_{k=1}^{n}\lm_k \langle Ax_n,x_n-v_k \rangle  \geq\sum_{k=1}^{n} \lm_k\langle Av_k, x_n-v_k \rangle 
            \end{equation*}
            Vemos que:
            \begin{align*}
                \sum_{k=1}^{n}\lm_k \langle Ax_n,x_n-v_k \rangle   &= \left\langle Ax_n, \sum_{k=1}^{n}(\lm_kx_n-\lm_kv_k) \right\rangle  = \left\langle Ax_n, x_n - \sum_{k=1}^{n}\lm_kv_k \right\rangle  \\ &= \langle Ax_n, x_n - x_0-x\rangle  = \langle Ax_n, x_n-x_0 \rangle  - \langle Ax_n, x \rangle 
            \end{align*}
            Por lo que:
            \begin{align*}
                \langle Ax_n,x \rangle &\leq \langle Ax_n, x_n-x_0 \rangle + \sum_{k=1}^{n}\lm_k\langle Av_k, v_k - x_n \rangle  \\
                                       &\leq \|Ax_n\|\|x_n-x_0\| + \sum_{k=1}^{n}\lm_k \|Av_k\|\|v_k-x_n\|
            \end{align*}
            Pero:
            \begin{equation*}
                \|v_k - x_n\| \leq \|v_k - x_0\| + \|x_0 - x_n\| \leq \|v_k - x_0\| + 1
            \end{equation*}
            Por lo que:
            \begin{align*}
                \langle Ax_n,x \rangle &\leq \langle Ax_n, x_n-x_0 \rangle + \sum_{k=1}^{n}\lm_k\langle Av_k, v_k - x_n \rangle  \\
                                       &\leq \|Ax_n\|\|x_n-x_0\| + \sum_{k=1}^{n}\lm_k \|Av_k\|\|v_k-x_n\| \\
                                       &\leq \|Ax_n\|\underbrace{\|x_n-x_0\|}_{\varepsilon_n} + \underbrace{\sum_{k=1}^{n}\lm_k \|Av_k\|(\|v_k-x_0\|+1)}_{C(x)}
            \end{align*}
            Lo que nuevamente lleva a contradicción.
        \item Extiende la conclusión de la pregunta 1 al caso de que $A$ sea multivaluada.

            En este caso, queremos probar que existen $R>0,C\geq 0$ de forma que:
            \begin{equation*}
                \text{Si}\quad  x\in D(A) \quad \text{con}\quad \|x-x_0\|<R \Longrightarrow \|f\| \leq C \qquad \forall f\in Ax
            \end{equation*}
            Por reducción al absurdo, suponemos que
            \begin{equation*}
                \forall R,C\geq 0 \quad \exists  x\in D(A) \quad \text{con}\quad \|x-x_0\|<R \quad \text{y}\quad f\in Ax \quad \text{con}\quad  \|f\| > C 
            \end{equation*}
            Por lo que para cada $n\in \mathbb{N}$ podemos tomar un elemento $x_n\in D(A)$ con $\|x_n-x_0\| < \frac{1}{n}$, y $f_n\in Ax_n$ con $\|f_n\|>C$. En definitiva, tenemos $\{x_n\}\to x_0$ y $\{\|f_n\|\}\to \infty$. Sea ahora $r>0$ de forma que $B(x_0,r)\subset D(A)$, entonces si $x\in E$ con $\|x\|<r$ tendremos que $x_0+x\in D(A)$. Si tomamos $g\in A(x_0+x)$ y usamos la monotonía de $A$:
            \begin{equation*}
                \langle f_n - g,x_n-x_0-x \rangle \geq 0
            \end{equation*}
            luego:
            \begin{align*}
                \langle f_n,x \rangle &\leq \langle f_n,x_n-x_0 \rangle  + \langle g,x+x_0-x_n \rangle  \leq \|f_n\|\|x_n-x_0\| + \|g\|\|x + x_0 - x_n\| \\
                                      &\leq \|f_n\|\|x_n-x_0\| + \|g\|(\|x\| + \|x_0-x_n\|) \leq \|f_n\|\underbrace{\|x_n-x_0\|}_{\varepsilon_n} + \underbrace{\|g\|(\|x\|+1)}_{C(x)}
            \end{align*}
            Lo que nos lleva a contradicción.
    \end{enumerate}
\end{ejercicio}

\begin{ejercicio}
    Sea $\alpha = \{\alpha_n\}$ una sucesión de números reales y sea $1\leq p \leq \infty$. Supongamos que $\sum_{n=1}^{\infty} |\alpha_n||x_n| < \infty$ para cada elemento $x=\{x_n\}$ de $l_p$. Prueba que $\alpha\in l_{p'}$.
\end{ejercicio}

\begin{ejercicio}
    Sea $E$ un espacio de Banach y sea $T:E\to E^\ast$ un operador lineal verificando que
    \begin{equation*}
        \langle Tx, x \rangle \geq 0 \qquad \forall x\in E
    \end{equation*}
    Prueba que $T$ es acotado.\newline
    (Se pueden aplicar dos métodos, bien usar el Ejercicio~\ref{ej:6_rel2}, bien aplicar el Teorema de la Gráfica Cerrada.)

    \begin{description}
        \item [Usando el Ejercicio~\ref{ej:6_rel2}.] Es fácil ver que $T$ es monótona, ya que:
            \begin{equation*}
                \langle Tx-Ty,x-y \rangle  = \langle T(x-y),x-y \rangle \geq 0 \qquad \forall x,y\in E
            \end{equation*}
            En dicho caso, por el Ejercicio~\ref{ej:6_rel2} tenemos que para todo $x\in E$ existen $R>0$, $C\geq 0$ de forma que:
            \begin{equation*}
                y\in E \quad \text{con}\quad \|y-x\|<R \Longrightarrow \|Ty\| \leq C
            \end{equation*}
            Es decir, si $y\in B(x,R)$ tenemos entonces que $\|Ty\| \leq C$. Sea ahora $z\in B(0,1)$, tenemos que:
            \begin{equation*}
                \|Tz\| = \dfrac{\|T(Rz)\|}{R} \leq \dfrac{C}{R}
            \end{equation*}
            Por lo que $T(B(0,1))$ es un conjunto acotado, y por una proposición vista en teoría concluimos que $T$ es acotada.
        \item [Usando el Teorema de la Gráfica Cerrada.]  % // TODO: 


            % // TODO: VER ESTO
            \noindent
            Sea $y\in E$ fijo, $\forall n\in \mathbb{N}, \quad \forall \lm \in \mathbb{R}$ tenemos:
            \begin{equation*}
                \langle T(x_n+\lm y),x_n + \lm y \rangle  \geq 0
            \end{equation*}
            Por lo que:
            \begin{equation*}
                \langle Tx_n,x_n \rangle  + \langle Tx_n, \lm y \rangle  + \langle T\lm y, x_n \rangle  + \langle T\lm y, \lm y \rangle  \geq 0
            \end{equation*}
            de donde:
            \begin{equation*}
                \lm \langle L,y \rangle  + \lm^2 \langle Ty,y \rangle  \geq 0
            \end{equation*}
            para todo $\lm\in \mathbb{R}$, por lo que si nos quedamos con $\lm>0$:
            \begin{equation*}
                \langle L,y \rangle  + \lm\langle Ty,y \rangle  \geq 0
            \end{equation*}
            y si tendemos $\lm\to0^+$ tenemos:
            \begin{equation*}
                \langle L,y \rangle  \geq 0
            \end{equation*}
            por lo que:
            \begin{equation*}
                \langle L,-y \rangle  = -\langle L,y \rangle \geq 0 \Longrightarrow \langle L,y \rangle \leq 0
            \end{equation*}
            de donde $\langle L,y \rangle = 0 $ para todo $y\in E$, de donde $L=0$, con lo que la gráfica es cerrada, obteniendo que $T$ es continua, luego acotada.
    \end{description}
\end{ejercicio}

\begin{ejercicio} % // TODO: SE puede hacer sin usar el ejercicio 4, pero se repite dentro la demostración
    Sea $E$ un espacio de Banach y sea $T:E\to E^\ast$ un operador lineal verificando 
    \begin{equation*}
        \langle Tx,y \rangle  = \langle Ty,x \rangle  \qquad \forall x,y\in E
    \end{equation*}
    Prueba que $T$ es acotado.\\

    \noindent
    Usaremos el Ejercicio~\ref{ej:4_rel2}, por lo que definimos
    \Func{a}{E\times E}{\bb{R}}{(x,y)}{\langle Tx, y\rangle}
    Fijado $x\in E$, tenemos que $y\longmapsto \langle Tx,y \rangle $ es continua, porque $Tx\in E^\ast$. Veamos que $a$ es bilineal:
    \begin{itemize}
        \item En primera componente, si tomamos $x_1,x_2\in E$ y $\beta_1,\beta_2\in \mathbb{R}$, tenemos que:
            \begin{equation*}
                a(\beta_1x_1+\beta_2x_2) = \langle T(\beta_1x_1 + \beta_2x_2),y \rangle  = \langle Ty , \beta_1x_1 + \beta_2x_2 \rangle  = \beta_1a(x1,y) + \beta_2a(x_2,y)
            \end{equation*}
            Existe $C\geq 0$ tal que:
            \begin{equation*}
                |a(x,y)| \leq C \|x\|\|y\| \qquad \forall x,y\in E
            \end{equation*}
            por lo que:
            \begin{equation*}
                \|Tx\| = \sup_{\|y\|\leq 1}|\langle Tx,y \rangle  \leq C \|x\|\|y\|
            \end{equation*}
    \end{itemize}
    \textbf{También se puede hacer por el Teorema de la gráfica cerrada}.
\end{ejercicio}

\begin{ejercicio}
    Sean $E$ y $F$ dos espacios de Banach y sea $T\in L(E,F)$ una aplicación sobreyectiva.
    \begin{enumerate}
        \item Sea $M\subset E$. Prueba que $T(M)$ es cerrado en $F$ si y solo si $M+N(T)$ es cerrado en $E$.

            Donde $N(T)$ es el núcleo de $T$, por doble implicación tenemos que:
            \begin{description}
                \item [$\Longleftarrow )$] Si $M+N(T)$ es cerrado, entonces $E\setminus(M+N(T))$ es abierto, y como:
                    \begin{equation*}
                        T(E\setminus(M+N(T))) = F\setminus T(M)
                    \end{equation*}
                    ya que:
                    \begin{description}
                        \item [$\supseteq )$] Si $y\in F\setminus T(M)$, por ser $T$ sobreyectiva existe $x\in E$ de forma que $T(x) = y$. Afirmo que $x\notin M + N(T)$, ya que si $x\in M+N(T)$ entonces existen $m\in M$, $n\in N(T)$ de forma que $x = m+ n$, por lo que:
                            \begin{equation*}
                                y = T(x) = T(m+n) = T(m) + T(n) = T(m) \in T(M)
                            \end{equation*}
                            \underline{contradicción}, luego $x\in E\setminus(M+N(T))$, de donde tenemos que $y \in T(E\setminus (M+N(T)))$.
                        \item [$\subseteq )$] Si $y\in T(E\setminus(M+N(T)))$ entonces existe $x\in E\setminus(M+N(T))$ de forma que $T(x) = y$. Si existiera $z\in M$ de forma que $T(z) = y$, tendríamos entonces que:
                            \begin{equation*}
                                0 = y-y = T(x)-T(z) = T(x-z) 
                            \end{equation*}
                            Por lo que existe $v\in N(T)$ de forma que $v = x-z$, de donde:
                            \begin{equation*}
                                x = z+v \in M+N(T)
                            \end{equation*}
                            \underline{contradicción}, luego $y\in F\setminus T(M)$.
                    \end{description}
                    si aplicamos ahora el Teorema de la Aplicación Abierta, obtenemos que $F\setminus T(M)$ es abierto, por lo que $T(M)$ es cerrado.
                \item [$\Longrightarrow )$] Si $T(M)$ es cerrado, por ser $T$ continua tenemos que $T^{-1}(T(M))$ es cerrado, y:
                    \begin{equation*}
                        T^{-1}(T(M)) = M+N(T)
                    \end{equation*}
                    \begin{description}
                        \item [$\supseteq )$] Si $x+n\in M+N(T)$, entonces:
                            \begin{equation*}
                                T(x+n) = T(x)+T(n) = T(x) \in T(M)
                            \end{equation*}
                            Por lo que $x+n\in T^{-1}(T(x))$.
                        \item [$\subseteq )$] Si $x\in T^{-1}(T(M))$, entonces:
                            \begin{equation*}
                                T(x) \in T(M) \Longrightarrow \exists m\in M \text{\ con\ } T(x) = T(m)
                            \end{equation*}
                            de donde $T(x-m) = T(x)-T(m) = 0$. En conclusión, tenemos que:
                            \begin{equation*}
                                x = m + x - m
                            \end{equation*}
                            con $m\in M$, $x-m\in N(T)$.
                    \end{description}
            \end{description}
        \item Deduce que si $M$ es un subespacio vectorial cerrado de $E$ y si $dim N(T)<\infty$, entonces $T(M)$ es cerrado.

            Si $dim N(T)<\infty$ tenemos entonces que $N(T)$ es un espacio vectorial de dimensión finita, luego es un conjunto cerrado. Como Además, $M$ es un espacio vectorial cerrado, tendremos que $M+N(T)$ es cerrado, luego $T(M)$ será cerrado por el primer apartado.
    \end{enumerate}

    En este último apartado hemos usado que:
    \begin{center}
        Si $E$ es un espacio normado y $M,N\subset E$ son subespacios vectoriales con $M$ cerrado y $N$ de dimensión finita, entonces $M+N$ es cerrado.
    \end{center}
    \begin{proof}
        Distingamos ciertos casos triviales que nos facilitan la prueba:
        \begin{itemize}
            \item Si $N\subset M$, entonces $M+ N = M$, por lo que la prueba es trivial. Suponemos pues que $M\cap N \neq N$.
            \item Si $M\cap N \neq \{0\}$, entonces como $N$ es de dimensión finita, podemos tomar una base suya $B$, de forma que $N = \cc{L}(B)$, y como $M\cap N$ es un subespacio de $N$ ha de existir $B'\subsetneq B$ de forma que $M\cap N = \cc{L}(B')$, por lo que si consideramos $N' = \cc{L}(B\setminus B')$ tenemos que $M\cap N' = \{0\}$, con:
                \begin{equation*}
                    M + N = M + N'
                \end{equation*}
                Podemos por tanto suponer que $M\cap N = \{0\}$.
        \end{itemize}
        Si tenemos una sucesión $\{x_n\}$ de puntos de $M+N$ convergente a $x\in E$, queremos ver que $x\in M+N$. Para ello, para cada $n\in \mathbb{N}$ existen $u_n\in M$ y $v_n\in N$ de forma que:
        \begin{equation*}
            x_n = u_n + v_n
        \end{equation*}
        Y la demostración se completa en dos pasos:
        \begin{description}
            \item [Ver que $\{v_n\}$ está acotada.]  Por reducción al absurdo, supongamos que $\{v_n\}$ no está acotada, con lo que podemos encontrar una parcial divergente \newline $\{\|v_{\sigma(n)}\|\}\to \infty$, lo que nos dirá que (usando que $\{x_n\}$ está acotada por ser convergente):
                \begin{equation*}
                    \left\{\frac{u_{\sigma(n)} + v_{\sigma(n)}}{\|v_{\sigma(n)}\|}\right\}= \left\{\frac{x_{\sigma(n)}}{\|v_{\sigma(n)}\|}\right\} \to 0
                \end{equation*}
                Si tomamos:
                \begin{equation*}
                    w_n = \frac{v_n}{\|v_n\|} \qquad \forall n\in \mathbb{N}
                \end{equation*}
                tenemos que $\|w_n\| = 1 \quad \forall n\in \mathbb{N}$ y que $w_n\in N\quad \forall n\in \mathbb{N}$. Es decir, una sucesión acotada en un espacio de dimensión finita, propiedad que también cumple $\{w_{\sigma(n)}\}$, por lo que ha de existir una parcial de esta última, $\{w_{\gamma(n)}\}$ que sea convergente a cierto $w\in E$ (por el Teorema de Bolzano-Weierstrass). Notemos que por ser $N$ cerrado ha de ser $w\in N$, así como $\|w\| = 1$ por ser $\|\cdot \|$ una aplicación continua. Tenemos entonces que:
                \begin{equation*}
                    \left\{\frac{u_{\gamma(n)}}{\|v_{\gamma(n)}\|} + w_{\gamma(n)}\right\} = \left\{\frac{x_{\gamma(n)}}{\|v_{\gamma(n)}\|}\right\} \to 0
                \end{equation*}
                Fijado $\varepsilon>0$, esta última convergencia nos da $n_1\in \mathbb{N}$ de forma que:
                \begin{equation*}
                    \left\|\frac{u_{\gamma(n)}}{\|v_{\gamma(n)}\|} + w_{\gamma(n)}\right\| < \frac{\varepsilon}{2} \qquad \forall n\geq n_1
                \end{equation*}
                y la convergencia $\{w_{\gamma(n)}\}\to w$ nos da $n_2\in \mathbb{N}$ de forma que
                \begin{equation*}
                    \|w_{\gamma(n)} - w\| < \frac{\varepsilon}{2}\qquad \forall n\geq n_2
                \end{equation*}
                Por lo que tomando $n_0=\max\{n_1,n_2\}$ obtenemos que:
                \begin{equation*}
                    \left\|\frac{u_{\gamma(n)}}{\|v_{\gamma(n)}\|} + w\right\| \leq \left\|\frac{u_{\gamma(n)}}{\|v_{\gamma(n)}\|} + w_{\gamma(n)}\right\| + \|w_{\gamma(n)}  - w\| < \varepsilon \qquad \forall n\geq n_0
                \end{equation*}
                Es decir:
                \begin{equation*}
                    \left\{\frac{u_{\gamma(n)}}{\|v_{\gamma(n)}\|} + w\right\} \to 0 \quad \Longrightarrow \quad \left\{\frac{u_{\gamma(n)}}{\|v_{\gamma(n)}\|}\right\} \to -w
                \end{equation*}
                Y como $M$ es cerrado y la sucesión que tenemos es de puntos de $M$ deducimos que $-w\in M$, por lo que $w\in M$ por ser $M$ un espacio vectorial. En definitiva, hemos probado que si la sucesión $\{v_n\}$ no está acotada, entonces podemos encontrar (recordamos que $\|w\|=1$) $0\neq w \in M\cap N$, \underline{contradicción} con que $M\cap N = \{0\}$.
            \item [Ver que $x\in M+N$.] Una vez sabemos que $\{v_n\}$ está acotada, como $dim N<\infty$, por Bolzano-Weierstrass ha de existir una parcial $\{v_{\alpha(n)}\}$ convergente a cierto $v\in E$, y por ser $N$ de dimensión finita tenemos que es cerrado, con lo que $v\in N$. Fijado $\varepsilon>0$, la convergencia $\{u_{\alpha(n)} + v_{\alpha(n)}\}\to x$ nos da $n_1\in \mathbb{N}$ de forma que:
                \begin{equation*}
                    \|u_{\alpha(n)} + v_{\alpha(n)} - x \| < \frac{\varepsilon}{2} \qquad \forall n\geq n_1
                \end{equation*}
                y la convergencia $\{v_{\alpha(n)}\}\to v$ nos da $n_2\in \mathbb{N}$ de forma que:
                \begin{equation*}
                    \|v_{\alpha(n)} - v\| < \frac{\varepsilon}{2}\qquad \forall n\geq n_2
                \end{equation*}
                por lo que tomando $n_0 = \max\{n_1,n_2\}$ tenemos que:
                \begin{align*}
                    \|u_{\alpha(n)} - x + v\| &= \|u_{\alpha(n)} + v_{\alpha(n)} - x - (v_{\alpha(n)} - v) \|\\ & \leq \|u_{\alpha(n)} + v_{\alpha(n)} - x\| + \|v_{\alpha(n)} - v\| < \varepsilon\qquad  \forall n\geq n_0
                \end{align*}
                Por lo que $\{u_{\alpha(n)}\}\to x-v$ y como $M$ es cerrado, $x-v\in M$. En definitiva, tenemos que:
                \begin{equation*}
                    x = x - v + v
                \end{equation*}
                con $x-v\in M$ y $v\in N$, por lo que $x\in M+N$.
        \end{description}
    \end{proof}
\end{ejercicio}
