\section{Ejercicios}
\begin{ejercicio}
    
\end{ejercicio}

\begin{ejercicio}
    Sea $E$ un espacio vectorial y sea $p:E\to \mathbb{R}$ una función que cumple las siguientes propiedades:
    \begin{enumerate}
        \item $p(x+y)\leq p(x)+p(y)\quad \forall x,y\in E$.
        \item Para cada $x\in E$ fijo la función $\lm\longmapsto p(\lm x)$ es continua.
        \item Siempre que una sucesión de puntos de $E$ $\{y_n\}$ verifique que $\{p(y_n)\}\to 0$, entonces $\{p(\lm y_n)\}\to 0$ para cada $\lm \in \mathbb{R}$.
    \end{enumerate}
    Supongamo que $\{x_n\}$ es una sucesión de puntos de $E$ de forma que $\{p(x_n)\}\to 0$ y $\{\alpha_n\}$ es una sucesión de números reales acotada. Probar que $p(0)=0$ y que $\{p(\alpha_n x_n)\}\to 0$.\newline
    (\textbf{Pista:} Dado $\varepsilon>0$, considera los conjuntos:
    \begin{equation*}
        F_n = \{\lm \in \mathbb{R} : |p(\lm x_k)| \leq \varepsilon\quad \forall k\geq n\}
    \end{equation*}
    Deduce que si $\{x_n\}$ es una sucesión de $E$ de forma que $\{p(x_n-x)\}\to 0$ para algún $x\in E$ y $\{\alpha_n\}$ es una sucesión de forma que $\{\alpha_n\}\to \alpha$, entonces $\{p(\alpha_n x_n)\}\to p(\alpha x)$.)\\

    \noindent
    Usaando que $\{x_n\}$ y 3 con $\lm = 0$ obtenemos que: % // TODO: suponer en 2 que \lm \neq 0
    \begin{equation*}
        \{p(0\cdot x_n)\} = \{p(0)\} \to 0
    \end{equation*}
    de donde $p(0) = 0$. Siguiendo la pista, dado $\varepsilon>0$ definimos:
    \begin{equation*}
        F_n = \{\lm \in \mathbb{R} : |p(\lm x_k)| \leq \varepsilon\quad \forall k\geq n\}
    \end{equation*}
    Por reducción al absurdo, supongamos que $\{p(\alpha_n x_n)\}\not\to 0$, de donde existe una parcial con $|p(\alpha_{\sigma(n)}x_{\sigma(n)})|\geq \varepsilon$, para todo $n\in \mathbb{N}$. Como $\{\alpha_n\}$ está acotada, el Teorema de Weierstrass nos permite encontrar una parcial convergente. Supongmos que la parcial $\sigma$ verifica esto, con lo que $\{\alpha_{\sigma(n)}\}\to \alpha$. Observemos que:
    \begin{itemize}
        \item $F_n$ es cerrado para cada $n\in \mathbb{N}$, ya que:
            \begin{equation*}
                F_n = \bigcap_{k\geq n}\{\lm \in \mathbb{R} : |p(\lm x_k)|\leq \varepsilon\}
            \end{equation*}
            Como $(2)$ nos dice que $\lm \longmapsto p(\lm x_k)$ es continua, tenemos que cada uno de dichos conjuntos son cerrados, como preimagen de un conjunto cerrado por una función continua.
        \item Veamos que $\bigcup_{n\geq 1}F_n = \mathbb{R}$. Para ello, tomamos $\lm \in \mathbb{R}$ y como por hipótesis $\{p(\lm x_n)\}\to 0$, con lo que existe $n_0\in \mathbb{N}$ de forma que:
            \begin{equation*}
                |p(\lm x_n)| < \varepsilon \qquad \forall n\geq n_0
            \end{equation*}
            Luego $\lm\in  F_{n_0}$
    \end{itemize}
    Por el contrarrecíproco del Lema de Baire, existe $F_{\overline{n}}$ con $F_{\overline{n}}^\circ \neq \emptyset $, por lo que existen $\lm_0 \in \mathbb{R}$ y $\delta>0$ de forma que
    \begin{equation*}
        B(\lm_0, \delta)\subset F_{\overline{n}}^\circ
    \end{equation*}
    En otras palabras:
    \begin{equation*}
        (|p((\lm_0+t)x_{\sigma(k)})| \leq \varepsilon\quad \forall k\geq n_0 ) \quad \forall t\in \left]-\delta,\delta\right[
    \end{equation*}
    Ahora:
    \begin{equation*}
        p(\alpha_{\sigma(k)}x_{\sigma(k)})\leq p((\lm_0+\alpha_{\sigma(k)}-\alpha)x_{\sigma(k)}) + p((\alpha-\lm_0)x_{\sigma(k)})
    \end{equation*}
    con $\{ p((\alpha-\lm_0)x_{\sigma(k)})\}\to 0$ y podemos acotar el primer sumando en valor absoluto:
    \begin{equation*}
        |p((\lm_0+\alpha_{\sigma(k)}-\alpha)x_{\sigma(k)}) |\leq \varepsilon
    \end{equation*}
    Luego:
    \begin{equation*}
        p((\lm_0 + \alpha_{\sigma(k)} - \alpha)x_{\sigma(k)}) \leq p((\lm_0-\alpha)x_{\sigma(k)}) + p(\alpha_{\sigma(k)}x_{\sigma(k)})
    \end{equation*}
    de donde:
    \begin{equation*}
        p(x_{\sigma(k)}x_{\sigma(k)}) \geq p((\lm_0 + \alpha_{\sigma(k)}-\alpha)x_{\sigma(k)}) - p((\lm_0-\alpha)x_{\sigma(k)})
    \end{equation*}
    de forma que el segundo sumando tiende a 0 y el primero está acotado en valor absoluto por $\varepsilon$, de donde deducimos:
    \begin{equation*}
        p(\alpha_{\sigma(k)}x_{\sigma(k)}) \leq 2\varepsilon
    \end{equation*}
\end{ejercicio}

\begin{ejercicio}
    Sean $E$ y $F$ dos espacios de Banach y $\{T_n\}$ una sucesión en $L(E,F)$. Supongamos que para todo $x\in E$ se tiene que $\{T_nx\}$ converge a un cierto límite $Tx$. Probar que si $\{x_n\}\to x$ en $E$, entonces $\{T_n(x_n)\}\to Tx$ en $F$.\\

    \noindent
    Dado $x\in E$, como $\{T_nx\}$ es convergente a $Tx$, entonces $\{\|T_nx\|\}\to \|Tx\|$, luego:
    \begin{equation*}
        \sup_{n\in \mathbb{N}}\|T_nx\| < \infty
    \end{equation*}
    Por el Principio de acotación uniforme, tenemos que $C=\sup\limits_{n\in \mathbb{N}}\|Tn\| < \infty$, de donde:
    \begin{align*}
        \|T_nx_n - Tx\| &= \|T_nx_n - T_nx + T_nx - Tx\| \leq \|T_nx_n - T_nx\| + \|T_nx - Tx\| \\
                        &= \|T_n(x_n - x)\| + \|T_nx-Tx\| \leq C\|x_n-x\| + \|T_nx-Tx\|
    \end{align*}
    Como $\|x_n-x\|, \|T_nx-Tx\|\to 0$, tenemos pues que $\|T_nx_n - Tx\| \to 0$, de donde $\{T_n(x_n)\}\to Tx$.
\end{ejercicio}

\begin{ejercicio} % // TODO: Ejercicio de examen
    Sean $E,F$ dos espacios de Banach y sea $a:E\times F\to \mathbb{R}$ una forma bilineal que verifica:
    \begin{enumerate}
        \item Para cada $x\in E$, la aplicación $fx:y\longmapsto a(x,y)$ es continua.
        \item Para cada $y\in F$, la aplicación $f_y:x\longmapsto a(x,y)$ es continua.
    \end{enumerate}
    Probar que existe una constante $C\geq 0$ de forma que:
    \begin{equation*}
        |a(x,y)| \leq C\|x\|\|y\| \qquad \forall x\in E, \quad \forall y\in F
    \end{equation*}
    (\textbf{Pista:} Introduce un operador lineal $T:E\to F^\ast$ y prueba que $T$ está acotada con ayuda del Corolario~\ref{coro:entonces_Bast_acotado}).\\

    \begin{description}
        \item [Opción 1.] Es claro que $f_x$ es lineal, con lo que $f_x\in F^\ast$.
            \Func{T}{E}{F^\ast}{x}{f_x}
            Queremos ver que si $B\subset E$ es acotada entonces $T(B)$ es acotada. Para ello:
            \begin{equation*}
                \langle T(B),y \rangle  = \{\langle f_x,y \rangle :f_x\in T(B)\} = \{a(x,y):x\in B\} = \{f_y(x) : x\in B\}
            \end{equation*}

            de donde:
            \begin{equation*}
                |f_y(x)| \leq M \|x\| \qquad \forall x\in E
            \end{equation*}
            Luego por el Corolario~\ref{coro:entonces_Bast_acotado} tenemos que $T(B)$ está acotado, así como $T$ es continua, por lo que existe $C\geq 0$ tal que $\|T(x)\|\leq C\|x\|\quad \forall x\in E$. Luego:
            \begin{equation*}
                \|f_x\| = \sup_{\|y\|\leq 1}|f_x(y)| \leq C\|x\|
            \end{equation*}

            de donde:
            \begin{equation*}
                \left|a\left(x,\frac{y}{\|y\|}\right)\right| \leq C\|x\| \Longrightarrow a(x,y) \leq C\|x\|\|y\|
            \end{equation*}
        \item [Opción 2.] Definimos:
            \Func{T}{E}{F^\ast}{x}{f_x}
            Es claro que $T$ es lineal por ser $a$ bilineal. Lo que haremos será probar que $T$ es continua tratando de usar el Teorema de la Gráfica cerrada. Recordamos que:
            \begin{equation*}
                Gr(T) = \{(x,Tx) : x\in E\}
            \end{equation*}
            Para ver que $Gr(T)$ es cerrado, tomamos una sucesión de puntos de $Gr(T)$: $\{(x_n,Tx_n)\}$ que suponemso convergente a un punto $(x,w)\in Gr(T)$. Tenemos entonces que:
            \begin{equation*}
                \|x_n-x\|, \|Tx_n - w\| \to 0
            \end{equation*}
            De la segunda podemos deducir que $\langle Tx_n - w,y \rangle\to 0 \quad \forall y\in F$ :
            \begin{equation*}
                \langle Tx_n - w,y \rangle  = \langle Tx_n,y \rangle - \langle w-y \rangle  = a(x_n,y) - \langle w,y \rangle 
            \end{equation*}
            Por lo que $\{a(x_n,y)\}\to \langle w,y \rangle $, pero por la propiedad 2 se tiene que $\{a(x_n,y)\}\to a(x,y)$, de donde deducimos que $\langle T_x,y \rangle= a(x,y) = \langle w,y \rangle $ $\forall y\in F$, de donde deducimos que $w=T_x$, con lo que $(x,w)\in Gr(T)$, luego $Gr(T)$ es cerrada y por el Teorema de la gráfica cerrada concluimos que $T$ es continua.
    \end{description}
\end{ejercicio}

\begin{ejercicio}
    
\end{ejercicio}
