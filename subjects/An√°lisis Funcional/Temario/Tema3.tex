\chapter{Topologías Débiles}
\section{Topologías iniciales}
\noindent
Trabajaremos sobre un conjunto $X$ y una familia de espacios topológicos $\{Y_i\}_{i \in I}$ junto con una familia de aplicaciones $\varphi_i:X\to Y_i, \quad \forall i \in I$.\\

\noindent
Observamos que si consideramos en $X$ la topología discreta:
\begin{equation*}
    \tau_d = \{A : A\subset X\}
\end{equation*}
tenemos que $\varphi_i$ es continua, para todo $i \in I$. Sin embargo, hemos sido ``muy brutos'' al considerar esta topología sobre $X$. Nos preguntamos por definir alguna topología en $X$ que haga que todas las funciones de la familia $\{\varphi_i\}_{i \in I}$ sean continuas con el menor número de abiertos.

\begin{observacion}
    Observemos que como pretendemos que las funciones $\varphi_i$ sean continuas, necesitaremos que esta topología $\tau$ buscada verifique que:
    \begin{equation*}
        \tau \supset \{\varphi_i^{-1}(\omega_i) : \omega_i \text{\ es un abierto de\ } Y_i, \quad \forall i \in I\}
    \end{equation*}
\end{observacion}
\noindent
De esta forma, el problema podemos reformularlo como:

\begin{center}
    Dado un conjunto $X$ y una familia $\cc{U} = \{U_\lm\subset X : \lm \in \Lambda\}$, buscar la topología $\tau$ con menor cantidad de abiertos de forma que $\cc{U}\subset \tau$.
\end{center}

\noindent
Para ello, si pretendemos que los $U_\lm$ estén en $\tau$, estos serán abiertos, luego toda intersección finita de ellos lo seguirá siendo, con lo que la intersección finita de los conjuntos $U_\lm$ también tiene que seguir estando en $\tau$. Es decir, si consideramos:
\begin{equation*}
    \cc{V} = \left\{V = \bigcap_{i=1}^n U_{\lm _i} : \lm_1, \ldots, \lm_n \in \Lambda, \quad n\in \mathbb{N}\right\}
\end{equation*}
Tenemos que $\cc{U}\subset \cc{V}$, y nos preguntamos si $\cc{V}$ es una topología. Observamos:
\begin{itemize}
    \item Primero, que $\cc{V}$ es estable por intersecciones finitas.
    \item Sin embargo, la familia no es cerrada por uniones arbitrarias de elementos del conjunto.
\end{itemize}
Para solucionar el segundo problema, consideramos:
\begin{equation*}
    \left\{\bigcup_{\eta \in \Lambda_0} V_\eta : V_\eta \in \cc{V}, \Lambda_0\subset \Lambda\right\}
\end{equation*}
Y tenemos que la topología más pequeña que buscamos debe contener este conjunto. Se demuestra que este conjunto es, de hecho, una topología.

\begin{observacion}
    Observemos que el vacío es resultado de una unión vacía.

    Sin embargo, faltaría unir el total.
\end{observacion}

\noindent
¿Cómo se forma una base de entornos de dicha topología en $X$?

Resulta que una base de entornos es:
\begin{equation*}
    \left\{\bigcap_{i \in J}\varphi_i^{-1}(V_i) : V_i \text{\ entorno de\ } \varphi_i(x)\in Y_i, \quad J\subset I \text{\ finito}\right\}
\end{equation*}

\noindent
Se verifica que el conjunto es una base de entornos.\\

\noindent
Aunque no conozcamos en profundidad la topología (puesto que no hemos dado de forma explícita quiénes son sus abiertos), es sencillo en ocasiones probar ciertas propiedades topológicas, usando para ellos las dos proposiciones siguientes, que nos permiten comprobar propiedades sobre la topología inicial sin tener que usarla, sino tratar de buscar problemas equivalentes realizando composiciones con las aplicaciones $\varphi_i$ de la familia que nos da la topología inicial.

\begin{prop}\label{prop:top_inicial_suc}
    Sea $(X,\tau)$ con $\tau$ la topología inicial asociada a una familia de aplicaciones $\{\varphi_i\}_{i \in I}$, sea $\{x_n\}$ una sucesión de puntos de $X$ y $x\in X$:
    \begin{equation*}
        \{x_n\}\to x \quad \Longleftrightarrow \quad  \{\varphi_i(x_n)\} \to \varphi_i(x) \quad \forall i \in I
    \end{equation*}
    \begin{proof}
        Por doble implicación:
        \begin{description}
            \item [$\Longrightarrow )$] Para cada $i \in I$, $\tau$ hace que $\varphi_i$ sea continua, por lo que:
                \begin{equation*}
                    \{x_n\} \to x \Longrightarrow \{\varphi_i(x_n)\} \to \varphi_i(x)
                \end{equation*}
            \item [$\Longleftarrow )$] Si consideramos un entorno $U$ de $x$, este ha de contener un entorno de la base de entornos, luego existe una familia finita $\{V_i\}_{i \in J}$ de entornos de $\varphi_i(x)$ en cada $Y_i$ con $i \in J$ de forma que:
                \begin{equation*}
                    W = \bigcap_{i \in J}\varphi_i^{-1}(V_i)
                \end{equation*}
                es un entorno básico contenido en $U$. Observemos que para cada $i \in J$ tenemos:
                \begin{equation*}
                    \left.\begin{array}{c}
                        \varphi_i(x) \in V_i \\
                        \{\varphi_i(x_n)\} \to \varphi_i(x)
                    \end{array}\right\} \Longrightarrow \exists N_i \in \mathbb{N} : \varphi_i(x_n)\in V_i \quad \forall n\geq N_i
                \end{equation*}
                Sin embargo, como $J$ es finito, podemos tomar $N = \max\limits_{i \in J}N_i$ y tendremos que:
                \begin{equation*}
                    n\geq N \Longrightarrow \varphi_i(x_n) \in V_i \qquad \forall i \in J
                \end{equation*}
                de donde:
                \begin{equation*}
                    x_n \in W = \bigcap_{i \in J}\varphi_i^{-1}(V_i) \subset U \quad \forall n\geq N
                \end{equation*}
                Por lo que $\{x_n\}\to x$.
        \end{description}
    \end{proof}
\end{prop}

\noindent
Por lo que conociendo la convergencia de las sucesiones en los espacios $Y_i$, estudiar la convergencia de $X$ con la topología inicial se reduce a estudiar las convergencias de sus imágenes por $\varphi_i$, esto hace fácil trabajar con sucesiones en la topología inicial. Sin embargo, no todos los conceptos topológicos se pueden caracterizar por sucesiones.

Otra propiedad útil de las topologías iniciales es la siguiente:

\begin{prop}
    Sea $(X,\tau)$ con $\tau$ la topología inicial asociada a una familia de aplicaciones $\{\varphi_i\}_{i \in I}$. Si $Z$ es un espacio topológico y tenemos una aplicación entre espacios topológicos $\psi:Z\to X$, entonces:
    \begin{equation*}
        \psi \text{\ es continua} \Longleftrightarrow \varphi_i\circ \psi:Z\to Y_i \text{\ es continua}\quad \forall i \in I
    \end{equation*}
    \begin{proof}
        Por doble implicación:
        \begin{description}
            \item [$\Longrightarrow )$] Si $\psi:Z\to X$ es continua, $\tau$ hace que cada $\varphi_i$ sea continua, por lo que $\varphi_i\circ\psi$ es continua.
            \item [$\Longleftarrow )$] Para esta, tenemos que si $U\in \tau$, entonces podemos escribir:
                \begin{equation*}
                    U = \bigcup \bigcap_{J} \varphi_i^{-1}(\omega_i)
                \end{equation*}
                para ciertos conjuntos abiertos $\omega_i$ de $Y_i$, de donde:
                \begin{equation*}
                    \psi^{-1}(U) = \bigcup\bigcap_{J}\psi^{-1}(\varphi^{-1}_i(\omega_i)) = \bigcup\bigcap_J {(\varphi_i \circ \psi_i)}^{-1}(\omega_i)
                \end{equation*}
                como la intersección finita de abiertos en $Z$ es un abierto de $Z$ y la unión arbitraria de abiertos de $Z$ también lo es, tenemos que $\psi^{-1}(U)$ es abierto en $Z$, para cada $U\in \tau$, por lo que $\psi$ es continua.
        \end{description}
    \end{proof}
\end{prop}

\noindent
Y la idea es la misma de la Proposición anterior: aunque no conozcamos con exactitud los abiertos de la topología inicial, estudiar las funciones continuas $Z\to X$ se reduce al problema de estudiar la continuidad de cada una de las funciones resultantes con componer con $\varphi_i$, obteniendo funciones $Z\to Y_i$. Este procedimiento hace que sea muy fácil comprobar qué aplicaciones $Z\to X$ son continuas.

\section{Topología débil}
\noindent
Sea $(E,\|\cdot \|_E)$ un espacio normado, tenemos ya sobre $E$ una topología, la asociada a la norma $\|\cdot \|_E$, que denotaremos a veces por $\tau_{\|\cdot \|_E}$. Definiremos sobre este espacio $E$ otra topología:

\begin{definicion}[Topología débil de un espacio normado]
    Sea $(E,\|\cdot \|_E)$ un espacio normado, definimos la topología débil en $E$ como la topología inicial en $E$ que hace que todas las aplicaciones de la familia $E^\ast$ sean continuas, y denotaremos a esta topología por $\sigma(E,E^\ast)$.
\end{definicion}

\begin{observacion}
    Observaciones que hay que tener en cuenta al trabajar con $\sigma(E,E^\ast)$:
    \begin{itemize}
        \item La notación $\sigma(E,E^\ast)$ hay que pensarla como la ``topología débil en $E$ es la topología inicial en $E$ que hace continuos todos aquellos elementos de $E^\ast$''.
        \item Tenemos $Y_f = \mathbb{R}$ para cada $f\in E^\ast$, donde tomamos como conjunto de índices $I = E^\ast$.
        \item Observemos que:
            \begin{equation*}
                \sigma(E,E^\ast) \subset \tau_{\|\cdot \|_E}
            \end{equation*}
            Ya que toda aplicación $f\in E^\ast$ es continua en $\tau_{\|\cdot \|_E}$ y $\sigma(E,E^\ast)$ es, por definición de topología inicial, la topología más pequeña que hace que las aplicaciones de $E^\ast$ sean continuas.
    \end{itemize}
\end{observacion}

\noindent
Destacaremos a continuación propiedades destacables de la topología débil de un espacio normado $E$, donde siempre que hagamos referencia $\sigma(E,E^\ast)$, estaremos trabajando sobre un espacio normado $E$ arbitrario.

\begin{prop}
    $\sigma(E,E^\ast)$ es Hausdorff.
    \begin{proof}
        Sean $x_1,x_2\in E$ distintos, tomamos:
        \begin{equation*}
            A = \{x_1\}, \qquad B = \{x_2\}
        \end{equation*}
        que son dos conjuntos convexos, disjuntos y cerrados para $\tau_{\|\cdot \|_E}$, lo que nos permite aplicar la segunda versión geométrica del Teorema de Hahn-Banach (Teorema~\ref{teo:hahn-banach_2aversiongeometrica}), obteniendo $f\in E^\ast\setminus\{0\}$ y $\alpha\in \mathbb{R}$ de forma que:
        \begin{equation*}
            \langle f,x_1 \rangle  < \alpha < \langle f,x_2 \rangle 
        \end{equation*}
        de donde tomando:
        \begin{align*}
            x_1 &\in \Theta_1 := \{x\in E : \langle f,x \rangle <\alpha\} = f^{-1}(\left]-\infty,\alpha\right[) \in \sigma(E,E^\ast) \\
            x_2 &\in \Theta_2 := \{x\in E : \langle f,x \rangle >\alpha\} = f^{-1}(\left]\alpha,+\infty\right[) \in \sigma(E,E^\ast) 
        \end{align*}
        Tenemos que $\Theta_1,\Theta_2$ son disjuntos entre sí, con lo que nos dan la condición de Hausdorff que buscábamos.
    \end{proof}
\end{prop}

\noindent
Veamos ahora una base de entornos en $\sigma(E,E^\ast)$, aplicando el procedimiento que hicimos anteriormente al construir la topología inicial.

\begin{prop}
    Dado $x_0\in E$ y $f_1, \ldots, f_k\in E^\ast$, tenemos que:
    \begin{enumerate}
        \item $V = V(f_1, \ldots, f_k;\varepsilon) = \{x\in E : |\langle f_i,x-x_0 \rangle | < \varepsilon, \quad i \in \{1,\ldots, k\}\}$ es un entorno de $x_0$ en $\sigma(E,E^\ast)$, para todo $\varepsilon>0$.
        \item Además, $\cc{V} = \{V(f_1, \ldots, f_k;\varepsilon) : \varepsilon>0, \quad f_1,\ldots,f_k\in E^\ast\}$ es base de entornos de $x_0$ en $\sigma(E,E^\ast)$.
    \end{enumerate}
    \begin{proof}
        Veamos cada apartado:
        \begin{enumerate}
            \item Si definimos:
                \begin{equation*}
                    a_i = \langle f_i ,x_0 \rangle  \qquad \forall i \in \{1,\ldots,k\}
                \end{equation*}
                dado $\varepsilon>0$, tenemos que:
                \begin{equation*}
                    |\langle f_i,x \rangle -\langle f_i,x_0 \rangle | < \varepsilon \Longleftrightarrow \langle f_i,x_0 \rangle -\varepsilon\leq \langle f_i,x \rangle \leq \langle f_i,x_0 \rangle  + \varepsilon
                \end{equation*}
                que a su vez equivale a:
                \begin{equation*}
                    x \in f_i^{-1}\left(\langle f_i,x_0 \rangle -\varepsilon,\langle f_i,x_0 \rangle +\varepsilon\right)
                \end{equation*}
                Por lo que podemos escribir:
                \begin{equation*}
                    V = \bigcap_{i = 1}^k f_i^{-1}(a_i - \varepsilon,a_i + \varepsilon)
                \end{equation*}
                como $f_i$ es continua para la topología débil, el conjunto $V$ ha de ser abierto para $\sigma(E,E^\ast)$, y $x_0\in V$, por lo que $V$ es un entorno abierto de $x_0$.
            \item Para ver que es base de entornos, si tomamos $U$ un entorno abierto de $x_0$ en $\sigma(E,E^\ast)$, tenemos entonces que existe un entorno de la base de entornos de $\sigma(E,E^\ast)$, luego existen $f_1, \ldots, f_k\in E^\ast$ de forma que:
                \begin{equation*}
                    \bigcap_{j=1}^k f_j^{-1}(V_j) \subset U
                \end{equation*}
                como $V_j$ es un entorno de $f_j(x_0)$ en la topología usual en $\mathbb{R}$, ha de existir $\varepsilon>0$ de forma que:
                \begin{equation*}
                    \left]f_j(x_0)-\varepsilon,f_j(x_0)+\varepsilon\right[\subset V_j \qquad \forall j \in \{1,\ldots,k\}
                \end{equation*}

                de donde:
                \begin{equation*}
                    V(f_1, \ldots, f_k;\varepsilon) \subset \bigcap_{j = 1}^k f_j^{-1}(V_j) \subset U
                \end{equation*}
        \end{enumerate}
    \end{proof}
\end{prop}

\noindent
Esta proposición nos permite, tomado $x\in E$ y $U$ un abierto de $x$, han de existir $f_1, \ldots, f_k\in E^\ast$ y $\varepsilon>0$ de forma que:
\begin{equation*}
    x\in V(f_1, \ldots, f_k;\varepsilon)\subset U
\end{equation*}

\begin{ejercicio} % // TODO: PONER BIEN
    Probar que $dim E < \infty \Longrightarrow \sigma(E,E^\ast) = \tau_{\|\cdot \|_E}$.\\

    \noindent
    Si $dim E < \infty$, $E$ ha de ser isométrico a $\mathbb{R}^N$, para $N = dim E$, y tenemos que:
    \begin{equation*}
        E^\ast = \{f:E\to \mathbb{R} \text{\ de forma que\ } f \text{\ es lineal}\}
    \end{equation*}
    Nos preguntamos ahora por la topología que hace que todas estas sean continuas. Si tomamos $\{x_n\}$ una sucesión de $\mathbb{R}^N$ Y $x\in \mathbb{R}^N$, tenemos que:
    \begin{equation*}
        \{x_n\} \stackrel{(\ast)}{\to} x \Longleftrightarrow \{f(x_n)\} \to f(x) \qquad \forall f\in E^\ast
    \end{equation*}
    Además, una base de $E^\ast$ es $\{\pi_1, \ldots, \pi_N\}$, donde $\pi_i$ es la proyección en $i-$ésima coordenada, de donde:
    \begin{equation*}
        \{f(x_n)\} \to f(x) \quad \forall f\in E^\ast \Longleftrightarrow \{x_n(i)\} =  \{\pi_i(x_n)\} \to \pi_i(x) = x(i) \quad \forall i \in \{1,\ldots, k\}
    \end{equation*}
    y esta última condición es equivalente a decir que $\{x_n\}$ converge a $x$ con la norma de $E$.
\end{ejercicio}

\noindent
Resumimos en la siguiente proposición cada una de las relaciones entre las convergencias de sucesiones en los distintos espacios topológicos que manejamos. Antes de ello, introducimos la siguiente notación:

\begin{notacion}
    Si $(E,\|\cdot \|_E)$ es un espacio normado, consideraremos sobre él habitualmente dos topologías posiblemente distintas (por lo que obtendremos distintas convergencias de sucesiones):
    \begin{equation*}
        \tau_{\|\cdot \|_E} \qquad \text{y}\qquad \sigma(E,E^\ast)
    \end{equation*}
    Si tenemos una sucesión $\{x_n\}$ de puntos de $E$ y un punto $x\in E$, será costumbre para nosotros:
    \begin{itemize}
        \item notar por ``$\{x_n\}\to x$'' si la sucesión $\{x_n\}$ es convergente en $\tau_{\|\cdot \|_E}$ al elemento $x$, diciendo en alguna ocasión que la sucesión $\{x_n\}$ ``converge'' o que ``converge fuertemente'' al elemento $x$.
        \item notar por ``$\{x_n\}\rightharpoonup x$'' si la sucesión $\{x_n\}$ es convergente en $\sigma(E,E^\ast)$ al elemento $x$, diciendo en alguna ocasión que la sucesión $\{x_n\}$ ``converge débilmente'' al elemento $x$.
    \end{itemize}
    Todavía no está del todo claro la relación entre estas dos convergencias distintas de sucesiones de puntos de $E$, que aclararemos en la siguiente Proposición, pero ya podremos hablar de convergencia de sucesiones de puntos de $E$ de forma cómoda, sin confundir en ningún momento la convergencia de $\sigma(E,E^\ast)$ con la de $\tau_{\|\cdot \|_E}$.
\end{notacion}

\begin{prop}
    Sea $E$ un espacio normado y $\{x_n\}$ una sucesión de puntos de $E$:
    \begin{enumerate}
        \item $\{x_n\} \rightharpoonup x \Longleftrightarrow \{\langle f,x_n \rangle \}\to \langle f,x \rangle \quad \forall f\in E^\ast$.
        \item $\{x_n\}\to x \Longrightarrow \{x_n\} \rightharpoonup x$.
        \item $\{x_n\}\rightharpoonup x \Longrightarrow \{\|x_n\|\}$ acotada y $\|x\|\leq \liminf \|x_n\|$.
        \item Tenemos:
            \begin{equation*}
                \left.\begin{array}{r}
                    \{x_n\}\rightharpoonup x \\
                    \{f_n\} \to f
                \end{array}\right\} \Longrightarrow \{\langle f_n,x_n \rangle \}\to \langle f,x \rangle 
            \end{equation*}
    \end{enumerate}
    \begin{proof}
        Demostramos cada una de las propiedades:
        \begin{enumerate}
            \item Es la Proposición~\ref{prop:top_inicial_suc} pero usando la notación para la topología débil de $E$.
            \item Si tenemos $\{x_n\} \to x$, entonces para $f\in E^\ast$:
                \begin{equation*}
                    |\langle f,x_n-x \rangle | \leq \|f\|\|x_n-x\|  \qquad \forall n\in \mathbb{N}
                \end{equation*}
                y como $\|x_n-x\| \to 0$, deducimos que $\langle f,x_n-x \rangle \to 0 $, luego tenemos que:
                \begin{equation*}
                    \{\langle f,x_n \rangle \}\to \langle f,x \rangle  \qquad \forall f\in E^\ast
                \end{equation*}
                y usando 1 tenemos que $\{x_n\}\rightharpoonup x$.
            \item Tomamos $B = \{x_n : n\in \mathbb{N}\}$, que verifica para $f\in E^\ast$:
                \begin{equation*}
                    f(B) = \{\langle f,x_n \rangle : n\in \mathbb{N}\}
                \end{equation*}
                Como $\{x_n\}\rightharpoonup x$, el apartado 1 nos dice que $f(B)$ es acotado, $\forall f\in E^\ast$. Por el Corolario~\ref{coro:entonces_B_acotado} deducimos que $B$ es acotado, es decir, que $\{\|x_n\|\}$ está acotada.

                Para la segunda parte, si tomamos $f\in E^\ast$, tenemos que:
                \begin{equation*}
                    |\langle f,x_n \rangle | \leq \|f\|\|x_n\| \qquad \forall n\in \mathbb{N}
                \end{equation*}
                si tomamos límite inferior:
                \begin{equation*}
                    \langle f,x \rangle = \lim_{n\to\infty}\langle f,x_n \rangle  = \liminf \langle f,x_n \rangle  \leq \|f\|\liminf \|x_n\| \qquad \forall f\in E^\ast
                \end{equation*}
                En particular, si tomamos $\|f\| = 1$, tenemos que:
                \begin{equation*}
                    \|x\| = \sup_{\|f\|\leq 1}\langle f,x \rangle  \leq \|f\| \liminf \|x_n\| = \liminf \|x_n\|
                \end{equation*}
            \item Estudiamos la diferencia:
                \begin{align*}
                    |\langle f_n,x_n \rangle  - \langle f,x \rangle | &\leq |\langle f_n-f,x_n \rangle | + |\langle f,x_n-x \rangle |  \\
                                                                      &\leq \|f_n-f\| \|x_n\| + |\langle f,x_n \rangle -\langle f,x \rangle | \qquad \forall n\in \mathbb{N}
                \end{align*}
                Y tenemos que $\|f_n-f\| \to 0$, que $\{\|x_n\|\}$ está acotada, y que $\langle f,x_n \rangle \to \langle f,x \rangle $, de donde deducimos que $|\langle f_n,x_n \rangle -\langle f,x \rangle |\to 0$.
        \end{enumerate}
    \end{proof}
\end{prop}

\noindent
Para entender mejor el punto 3 de esta Proposición, introducimos el siguiente concepto:
\begin{definicion}
    Sea $(E,\tau)$ un espcio topológico, sea $f:(E,\tau)\to \mathbb{R}$ una aplicción, decimos que la función $f$ es \underline{secuencialmente semicontinua inferiormente} si se cumple que:
    \begin{equation*}
        \{x_n\} \to x \Longrightarrow f(x) \leq \liminf f(x_n)
    \end{equation*}
\end{definicion}~\\

\noindent
Notemos que sabíamos que la aplicación 
\begin{equation*}
    \|\cdot \|:(E,\tau_{\|\cdot \|_E})\to \mathbb{R}
\end{equation*}
es continua. Sin embargo, en vista de la Proposición y la Definición anterior, sabemos que la aplicación 
\begin{equation*}
    \|\cdot \|:(E,\sigma(E,E^\ast))\to \mathbb{R}
\end{equation*}
es secuencialmente semicontinua inferiormente.\\

\noindent
Nos preguntamos ahora por el recícproco de la propiedad 3, si tenemos una sucesión $\{\|x_n\|\}$ acotada con $\{x_n\}$ una sucesión de puntos de $E$, ¿será cierto que $\{x_n\}\rightharpoonup x$? La respuesta a esta pregunta es rotundamente negativa, pues sabemos que en dimensión finita $\sigma(E,E^\ast) = \tau_{\|\cdot \|_E}$, y sabemos de la existencia de sucesiones acotadas que no son convergentes en cualquier espacio normado $N-$dimensional.

Sin embargo, si recordamos el Teorema de Bolzano-Weierstrass, en todo espacio normado $N-$dimensional siempre que teníamos una sucesión acotada podríamos extraer una parcial suya convergente. Veremos próximamente que una propiedad similar a esta se cumple en la topología débil de $E$, lo que nos permitirá llegar a un Teorema que relacione los conjuntos compactos de $\sigma(E,E^\ast)$ con los conjuntos cerrados y acotados, brindándonos un espacio topológico con una cantidad abundante de conjuntos compactos, cosa que no sucede en $\tau_{\|\cdot \|_E}$ cuando la dimensión del espacio $E$ no es finita.

Esta propiedad de $\sigma(E,E^\ast)$ es totalmente natural, pues al considerar como $\sigma(E,E^\ast)$ la menor topología sobre $E$ que hace que las aplicaciones de $E^\ast$ sean continuas lo que estamos haciendo es eliminar de $\tau_{\|\cdot \|_E}$ abiertos que no nos interesa considerar en ciertos momentos, haciendo más fácil que un conjunto sea compacto, pues cuantos menos abiertos contenga una topología más fácil será que un conjunto sea compacto, por la propia definición de conjunto compacto en un espacio topológico general.

