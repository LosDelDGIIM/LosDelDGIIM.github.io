\documentclass[12pt]{article}

% Idioma y codificación
\usepackage[spanish, es-tabla]{babel}       %es-tabla para que se titule "Tabla"
\usepackage[utf8]{inputenc}

% Márgenes
\usepackage[a4paper,top=3cm,bottom=2.5cm,left=3cm,right=3cm]{geometry}

% Comentarios de bloque
\usepackage{verbatim}

% Paquetes de links
\usepackage[hidelinks]{hyperref}    % Permite enlaces
\usepackage{url}                    % redirecciona a la web

% Más opciones para enumeraciones
\usepackage{enumitem}

% Personalizar la portada
\usepackage{titling}

% Paquetes de tablas
\usepackage{multirow}


%------------------------------------------------------------------------

%Paquetes de figuras
\usepackage{caption}
\usepackage{subcaption} % Figuras al lado de otras
\usepackage{float}      % Poner figuras en el sitio indicado H.


% Paquetes de imágenes
\usepackage{graphicx}       % Paquete para añadir imágenes
\usepackage{transparent}    % Para manejar la opacidad de las figuras

% Paquete para usar colores
\usepackage[dvipsnames]{xcolor}
\usepackage{pagecolor}      % Para cambiar el color de la página

% Habilita tamaños de fuente mayores
\usepackage{fix-cm}

% Para los gráficos
\usepackage{tikz}

% Para poder situar los nodos en los grafos
\usetikzlibrary{positioning}


%------------------------------------------------------------------------

% Paquetes de matemáticas
\usepackage{mathtools, amsfonts, amssymb, mathrsfs}
\usepackage[makeroom]{cancel}     % Simplificar tachando
\usepackage{polynom}    % Divisiones y Ruffini
\usepackage{units} % Para poner fracciones diagonales con \nicefrac

\usepackage{pgfplots}   %Representar funciones
\pgfplotsset{compat=1.18}  % Versión 1.18

\usepackage{tikz-cd}    % Para usar diagramas de composiciones
\usetikzlibrary{calc}   % Para usar cálculo de coordenadas en tikz

%Definición de teoremas, etc.
\usepackage{amsthm}
%\swapnumbers   % Intercambia la posición del texto y de la numeración

\theoremstyle{plain}

\makeatletter
\@ifclassloaded{article}{
  \newtheorem{teo}{Teorema}[section]
}{
  \newtheorem{teo}{Teorema}[chapter]  % Se resetea en cada chapter
}
\makeatother

\newtheorem{coro}{Corolario}[teo]           % Se resetea en cada teorema
\newtheorem{prop}[teo]{Proposición}         % Usa el mismo contador que teorema
\newtheorem{lema}[teo]{Lema}                % Usa el mismo contador que teorema

\theoremstyle{remark}
\newtheorem*{observacion}{Observación}

\theoremstyle{definition}

\makeatletter
\@ifclassloaded{article}{
  \newtheorem{definicion}{Definición} [section]     % Se resetea en cada chapter
}{
  \newtheorem{definicion}{Definición} [chapter]     % Se resetea en cada chapter
}
\makeatother

\newtheorem*{notacion}{Notación}
\newtheorem*{ejemplo}{Ejemplo}
\newtheorem*{ejercicio*}{Ejercicio}             % No numerado
\newtheorem{ejercicio}{Ejercicio} [section]     % Se resetea en cada section


% Modificar el formato de la numeración del teorema "ejercicio"
\renewcommand{\theejercicio}{%
  \ifnum\value{section}=0 % Si no se ha iniciado ninguna sección
    \arabic{ejercicio}% Solo mostrar el número de ejercicio
  \else
    \thesection.\arabic{ejercicio}% Mostrar número de sección y número de ejercicio
  \fi
}


% \renewcommand\qedsymbol{$\blacksquare$}         % Cambiar símbolo QED
%------------------------------------------------------------------------

% Paquetes para encabezados
\usepackage{fancyhdr}
\pagestyle{fancy}
\fancyhf{}

\newcommand{\helv}{ % Modificación tamaño de letra
\fontfamily{}\fontsize{12}{12}\selectfont}
\setlength{\headheight}{15pt} % Amplía el tamaño del índice


%\usepackage{lastpage}   % Referenciar última pag   \pageref{LastPage}
\fancyfoot[C]{\thepage}

%------------------------------------------------------------------------

% Conseguir que no ponga "Capítulo 1". Sino solo "1."
\makeatletter
\@ifclassloaded{book}{
  \renewcommand{\chaptermark}[1]{\markboth{\thechapter.\ #1}{}} % En el encabezado
    
  \renewcommand{\@makechapterhead}[1]{%
  \vspace*{50\p@}%
  {\parindent \z@ \raggedright \normalfont
    \ifnum \c@secnumdepth >\m@ne
      \huge\bfseries \thechapter.\hspace{1em}\ignorespaces
    \fi
    \interlinepenalty\@M
    \Huge \bfseries #1\par\nobreak
    \vskip 40\p@
  }}
}
\makeatother

%------------------------------------------------------------------------
% Paquetes de cógido
\usepackage{minted}
\renewcommand\listingscaption{Código fuente}

\usepackage{fancyvrb}
% Personaliza el tamaño de los números de línea
\renewcommand{\theFancyVerbLine}{\small\arabic{FancyVerbLine}}

% Estilo para C++
\newminted{cpp}{
    frame=lines,
    framesep=2mm,
    baselinestretch=1.2,
    linenos,
    escapeinside=||
}

% para minted
\definecolor{LightGray}{rgb}{0.95,0.95,0.92}
\setminted{
    linenos=true,
    stepnumber=5,
    numberfirstline=true,
    autogobble,
    breaklines=true,
    breakautoindent=true,
    breaksymbolleft=,
    breaksymbolright=,
    breaksymbolindentleft=0pt,
    breaksymbolindentright=0pt,
    breaksymbolsepleft=0pt,
    breaksymbolsepright=0pt,
    fontsize=\footnotesize,
    bgcolor=LightGray,
    numbersep=10pt
}


\usepackage{listings} % Para incluir código desde un archivo

\renewcommand\lstlistingname{Código Fuente}
\renewcommand\lstlistlistingname{Índice de Códigos Fuente}

% Definir colores
\definecolor{vscodepurple}{rgb}{0.5,0,0.5}
\definecolor{vscodeblue}{rgb}{0,0,0.8}
\definecolor{vscodegreen}{rgb}{0,0.5,0}
\definecolor{vscodegray}{rgb}{0.5,0.5,0.5}
\definecolor{vscodebackground}{rgb}{0.97,0.97,0.97}
\definecolor{vscodelightgray}{rgb}{0.9,0.9,0.9}

% Configuración para el estilo de C similar a VSCode
\lstdefinestyle{vscode_C}{
  backgroundcolor=\color{vscodebackground},
  commentstyle=\color{vscodegreen},
  keywordstyle=\color{vscodeblue},
  numberstyle=\tiny\color{vscodegray},
  stringstyle=\color{vscodepurple},
  basicstyle=\scriptsize\ttfamily,
  breakatwhitespace=false,
  breaklines=true,
  captionpos=b,
  keepspaces=true,
  numbers=left,
  numbersep=5pt,
  showspaces=false,
  showstringspaces=false,
  showtabs=false,
  tabsize=2,
  frame=tb,
  framerule=0pt,
  aboveskip=10pt,
  belowskip=10pt,
  xleftmargin=10pt,
  xrightmargin=10pt,
  framexleftmargin=10pt,
  framexrightmargin=10pt,
  framesep=0pt,
  rulecolor=\color{vscodelightgray},
  backgroundcolor=\color{vscodebackground},
}

%------------------------------------------------------------------------

% Comandos definidos
\newcommand{\bb}[1]{\mathbb{#1}}
\newcommand{\cc}[1]{\mathcal{#1}}

% I prefer the slanted \leq
\let\oldleq\leq % save them in case they're every wanted
\let\oldgeq\geq
\renewcommand{\leq}{\leqslant}
\renewcommand{\geq}{\geqslant}

% Si y solo si
\newcommand{\sii}{\iff}

% Letras griegas
\newcommand{\eps}{\epsilon}
\newcommand{\veps}{\varepsilon}
\newcommand{\lm}{\lambda}

\newcommand{\ol}{\overline}
\newcommand{\ul}{\underline}
\newcommand{\wt}{\widetilde}
\newcommand{\wh}{\widehat}

\let\oldvec\vec
\renewcommand{\vec}{\overrightarrow}

% Derivadas parciales
\newcommand{\del}[2]{\frac{\partial #1}{\partial #2}}
\newcommand{\Del}[3]{\frac{\partial^{#1} #2}{\partial #3^{#1}}}
\newcommand{\deld}[2]{\dfrac{\partial #1}{\partial #2}}
\newcommand{\Deld}[3]{\dfrac{\partial^{#1} #2}{\partial #3^{#1}}}


\newcommand{\AstIg}{\stackrel{(\ast)}{=}}
\newcommand{\Hop}{\stackrel{L'H\hat{o}pital}{=}}

\newcommand{\red}[1]{{\color{red}#1}} % Para integrales, destacar los cambios.

% Método de integración
\newcommand{\MetInt}[2]{
    \left[\begin{array}{c}
        #1 \\ #2
    \end{array}\right]
}

% Declarar aplicaciones
% 1. Nombre aplicación
% 2. Dominio
% 3. Codominio
% 4. Variable
% 5. Imagen de la variable
\newcommand{\Func}[5]{
    \begin{equation*}
        \begin{array}{rrll}
            #1:& #2 & \longrightarrow & #3\\
               & #4 & \longmapsto & #5
        \end{array}
    \end{equation*}
}

%------------------------------------------------------------------------

\usetikzlibrary{er, fit}

\newcommand{\key}[1]{\ul{#1}}

% Definición de estilos
\tikzset{atributo/.style={attribute}}
\tikzset{entidad/.style={entity}}
\tikzset{relacion/.style={relationship}}
\tikzset{atributo derivado/.style={attribute, dashed}}
\tikzset{union discriminante/.style={dashed}}
\tikzset{entidad debil/.style={entidad, double distance=1.5 pt}}
\tikzset{participacion obligatoria/.style={double distance=1.5 pt}}
\tikzset{herencia/.style={draw, isosceles triangle, isosceles triangle apex angle=60, shape border rotate=230, minimum height=1cm, minimum width=1cm, fill=blue!20}}
\tikzset{agregacion/.style={draw, fit=#1, inner sep=0.5em}}


% Estilos estéticos
\tikzset{every entity/.style={draw=orange, fill=orange!20}}
\tikzset{every attribute/.style={draw=purple, fill=purple!20, font=\footnotesize}}
\tikzset{every relationship/.style={draw=green, fill=green!20, minimum height=2cm, minimum width=2cm}}

\begin{comment}
    \begin{tikzpicture}[node distance=6 em]
        \node [entidad](person){Person};
        \node [atributo derivado](pid) at (0.8,-2){\key{ID}} edge (person);
        \node [atributo](name) at (-0.8, -2) {Name} edge (person);
        \node [atributo](phone)[left of=person]{Phone} edge[union discriminante] (person);
        \node [atributo](address)[above left of=person]{Address} edge[-stealth] (person);
        \node [atributo](street)[above left of=address]{Street} edge (address);
        \node [atributo](city)[left of=address]{City} edge (address);
        \node [atributo](age)[above of=person]{Age} edge (person);
        \node [relacion](uses)[right of=person]{Uses} edge[participacion obligatoria] (person);
        \node [entidad debil](tool)[right of=uses]{Tool} edge (uses);
        \node [atributo](tid)[right of=tool]{\key{ID}} edge[Stealth-Stealth] (tool);
        \node [atributo](tname)[below of=tool]{Name} edge (tool);
    
        % Línea desde encima de Name hasta encima de ID
        \draw ([yshift=1em]name.north) -- ([yshift=1em]pid.north) node[circle, fill, inner sep=1.5pt] {};
    
        \node[agregacion=(A) (B) (R)] (box) {};
    
        \node [herencia, below of= B](H){} edge (B);
    \end{tikzpicture}
\end{comment}

% ---------------------------------------------------
\tikzset{
    CP/.style={
        overlay, text opacity=0, fill opacity=0,
        append after command={
            \pgfextra{
                \draw[thick] (\tikzlastnode.south west) -- (\tikzlastnode.south east);
                \node[below=0ex of \tikzlastnode] {CP};
            }
        }
    },
    CC/.style={
        overlay, text opacity=0, fill opacity=0,
        append after command={
            \pgfextra{
                \draw[thick] (\tikzlastnode.south west) -- (\tikzlastnode.south east);
                \node[below=0ex of \tikzlastnode] {CC};
            }
        }
    },
    CE/.style={
        overlay, text opacity=0, fill opacity=0,
        append after command={
            \pgfextra{
                \draw[thick, purple] (\tikzlastnode.north west) -- (\tikzlastnode.north east);
                \node[above=0ex of \tikzlastnode, text=purple] {CE};
            }
        }
    }
}


\begin{document}

    % 1. Foto de fondo
    % 2. Título
    % 3. Encabezado Izquierdo
    % 4. Color de fondo
    % 5. Coord x del titulo
    % 6. Coord y del titulo
    % 7. Fecha

    
    % 1. Foto de fondo
% 2. Título
% 3. Encabezado Izquierdo
% 4. Color de fondo
% 5. Coord x del titulo
% 6. Coord y del titulo
% 7. Fecha

\newcommand{\portada}[7]{

    \portadaBase{#1}{#2}{#3}{#4}{#5}{#6}{#7}
    \portadaBook{#1}{#2}{#3}{#4}{#5}{#6}{#7}
}

\newcommand{\portadaExamen}[7]{

    \portadaBase{#1}{#2}{#3}{#4}{#5}{#6}{#7}
    \portadaArticle{#1}{#2}{#3}{#4}{#5}{#6}{#7}
}




\newcommand{\portadaBase}[7]{

    % Tiene la portada principal y la licencia Creative Commons
    
    % 1. Foto de fondo
    % 2. Título
    % 3. Encabezado Izquierdo
    % 4. Color de fondo
    % 5. Coord x del titulo
    % 6. Coord y del titulo
    % 7. Fecha
    
    
    \thispagestyle{empty}               % Sin encabezado ni pie de página
    \newgeometry{margin=0cm}        % Márgenes nulos para la primera página
    
    
    % Encabezado
    \fancyhead[L]{\helv #3}
    \fancyhead[R]{\helv \nouppercase{\leftmark}}
    
    
    \pagecolor{#4}        % Color de fondo para la portada
    
    \begin{figure}[p]
        \centering
        \transparent{0.3}           % Opacidad del 30% para la imagen
        
        \includegraphics[width=\paperwidth, keepaspectratio]{assets/#1}
    
        \begin{tikzpicture}[remember picture, overlay]
            \node[anchor=north west, text=white, opacity=1, font=\fontsize{60}{90}\selectfont\bfseries\sffamily, align=left] at (#5, #6) {#2};
            
            \node[anchor=south east, text=white, opacity=1, font=\fontsize{12}{18}\selectfont\sffamily, align=right] at (9.7, 3) {\textbf{\href{https://losdeldgiim.github.io/}{Los Del DGIIM}}};
            
            \node[anchor=south east, text=white, opacity=1, font=\fontsize{12}{15}\selectfont\sffamily, align=right] at (9.7, 1.8) {Doble Grado en Ingeniería Informática y Matemáticas\\Universidad de Granada};
        \end{tikzpicture}
    \end{figure}
    
    
    \restoregeometry        % Restaurar márgenes normales para las páginas subsiguientes
    \pagecolor{white}       % Restaurar el color de página
    
    
    \newpage
    \thispagestyle{empty}               % Sin encabezado ni pie de página
    \begin{tikzpicture}[remember picture, overlay]
        \node[anchor=south west, inner sep=3cm] at (current page.south west) {
            \begin{minipage}{0.5\paperwidth}
                \href{https://creativecommons.org/licenses/by-nc-nd/4.0/}{
                    \includegraphics[height=2cm]{assets/Licencia.png}
                }\vspace{1cm}\\
                Esta obra está bajo una
                \href{https://creativecommons.org/licenses/by-nc-nd/4.0/}{
                    Licencia Creative Commons Atribución-NoComercial-SinDerivadas 4.0 Internacional (CC BY-NC-ND 4.0).
                }\\
    
                Eres libre de compartir y redistribuir el contenido de esta obra en cualquier medio o formato, siempre y cuando des el crédito adecuado a los autores originales y no persigas fines comerciales. 
            \end{minipage}
        };
    \end{tikzpicture}
    
    
    
    % 1. Foto de fondo
    % 2. Título
    % 3. Encabezado Izquierdo
    % 4. Color de fondo
    % 5. Coord x del titulo
    % 6. Coord y del titulo
    % 7. Fecha


}


\newcommand{\portadaBook}[7]{

    % 1. Foto de fondo
    % 2. Título
    % 3. Encabezado Izquierdo
    % 4. Color de fondo
    % 5. Coord x del titulo
    % 6. Coord y del titulo
    % 7. Fecha

    % Personaliza el formato del título
    \pretitle{\begin{center}\bfseries\fontsize{42}{56}\selectfont}
    \posttitle{\par\end{center}\vspace{2em}}
    
    % Personaliza el formato del autor
    \preauthor{\begin{center}\Large}
    \postauthor{\par\end{center}\vfill}
    
    % Personaliza el formato de la fecha
    \predate{\begin{center}\huge}
    \postdate{\par\end{center}\vspace{2em}}
    
    \title{#2}
    \author{\href{https://losdeldgiim.github.io/}{Los Del DGIIM}}
    \date{Granada, #7}
    \maketitle
    
    \tableofcontents
}




\newcommand{\portadaArticle}[7]{

    % 1. Foto de fondo
    % 2. Título
    % 3. Encabezado Izquierdo
    % 4. Color de fondo
    % 5. Coord x del titulo
    % 6. Coord y del titulo
    % 7. Fecha

    % Personaliza el formato del título
    \pretitle{\begin{center}\bfseries\fontsize{42}{56}\selectfont}
    \posttitle{\par\end{center}\vspace{2em}}
    
    % Personaliza el formato del autor
    \preauthor{\begin{center}\Large}
    \postauthor{\par\end{center}\vspace{3em}}
    
    % Personaliza el formato de la fecha
    \predate{\begin{center}\huge}
    \postdate{\par\end{center}\vspace{5em}}
    
    \title{#2}
    \author{\href{https://losdeldgiim.github.io/}{Los Del DGIIM}}
    \date{Granada, #7}
    \thispagestyle{empty}               % Sin encabezado ni pie de página
    \maketitle
    \vfill
}
    \portadaExamen{etsiitA4.jpg}{FBD\\Examen IV}{FBD. Examen IV}{MidnightBlue}{-8}{28}{2024-2025}{Arturo Olivares Martos}

    \begin{description}
        \item[Asignatura] Fundamentos de Bases de Datos.
        \item[Curso Académico] 2020-21.
        %\item[Grado] Doble Grado en Ingeniería Informática y Matemáticas.
        %\item[Grupo] Único.
        %\item[Profesor] Nicolás Marín Ruiz.
        \item[Descripción] Convocatoria Ordinaria. Práctico Parcial 2 (Seminarios 3-4).
        %\item[Fecha] 8 de noviembre de 2021.
        % \item[Duración] 60 minutos.
    
    \end{description}
    \newpage

\section{Modelo 1}\label{sec:modelo1}
Considerar el esquema relacional de la Figura~\ref{fig:modelo1}, donde cada paciente tiene asignada una aseguradora que cubre sus gastos médicos. La tabla \verb|Consulta| almacena citas en las que un médico ha atendido a un paciente en una fecha dada.
\begin{figure}[H]
    \centering
    \begin{tikzpicture}[node distance=4 em]
        \node (aseguradora) {Asegura(id\_aseguradora, nombre\_aseguradora, país)};
        \node (paciente) [below of=aseguradora] {Paciente(id\_paciente, nombre\_paciente, id\_aseguradora, fecha\_nacimiento)};
        \node (medico) [below of=paciente] {Médico(id\_medico, nombre\_medico, especialidad, sueldo)};
        \node (consulta) [below of=medico] {Consulta(id\_paciente, id\_medico, fecha, precio\_facturado, numero\_horas)};

        \node[CP, xshift=-9ex] (CPAseguradora) at(aseguradora) {id\_aseguradora};
        \node[CP, xshift=-15ex] (CPMedico) at(medico) {id\_medico};
        \node[CP, xshift=-21ex] (CPPaciente) at(paciente) {id\_paciente};
        \node[CP, xshift=-11ex] (CPConsulta) at(consulta) {id\_paciente, id\_medico, fecha};

        \node[CE, xshift=31ex] (CEPaciente) at(CPPaciente) {id\_aseguradora};
        \node[CE, xshift=-10ex] (CEConsultaPaciente) at(CPConsulta) {id\_paciente};
        \node[CE, xshift=2ex] (CEConsultaMedico) at(CPConsulta) {id\_medico};

        \node[yshift=1.5em, xshift=1.5em, purple] at (CEPaciente) {(1)};
        \node[yshift=1.5em, xshift=1.5em, purple] at (CEConsultaPaciente) {(2)};
        \node[yshift=1.5em, xshift=1.5em, purple] at (CEConsultaMedico) {(3)};

        \node[yshift=-1.5em, xshift=1.5em, purple] at (CPAseguradora) {(1)};
        \node[yshift=-1.5em, xshift=1.5em, purple] at (CPMedico) {(3)};
        \node[yshift=-1.5em, xshift=1.5em, purple] at (CPPaciente) {(2)};
    \end{tikzpicture}
    \caption{Esquema relacional del Modelo~\ref{sec:modelo1}.}
    \label{fig:modelo1}
\end{figure}

\begin{ejercicio}[SQL]
    Encontrar, considerando sólo los médicos que han realizado consultas en el 2014, el sueldo del médico que ha realizado menos consultas durante dicho año. Puede haber más de un médico que cumpla esta condición.
    \begin{minted}{sql}
        SELECT id_medico, sueldo, count(*)
            FROM medico NATURAL JOIN consulta
            WHERE TO_CHAR(fecha, 'YYYY') = '2014'
            GROUP BY id_medico, sueldo
            HAVING COUNT(*) = (
                SELECT MIN(COUNT(*))
                    FROM consulta
                    WHERE TO_CHAR(fecha, 'YYYY') = '2014'
                    GROUP BY id_medico
            );
    \end{minted}
\end{ejercicio}


\begin{ejercicio}[SQL]
    Encontrar, entre los pacientes que han tenido consultas en el 2011, el id de la aseguradora de aquel paciente que ha gastado en total menos dinero sumando sus consultas en dicho año. Puede haber más de un paciente que cumpla esta condición.\\

    En primer lugar, buscamos los pacientes que han tenido consultas en el 2011 y han gastado menos dinero sumando sus consultas en dicho año.
    \begin{minted}{sql}
        SELECT id_paciente, SUM(precio_facturado)
            FROM consulta
            WHERE TO_CHAR(fecha, 'YYYY') = '2011'
            GROUP BY id_paciente
            HAVING SUM(precio_facturado) = (
                SELECT MIN(SUM(precio_facturado))
                    FROM consulta
                    WHERE TO_CHAR(fecha, 'YYYY') = '2011'
                    GROUP BY id_paciente
            );
    \end{minted}

    Una vez obtenidos los pacientes que cumplen la condición, buscamos el id de la aseguradora de cada uno de ellos.
    \begin{minted}{sql}
        SELECT id_aseguradora, id_paciente, SUM(precio_facturado)
            FROM paciente, consulta
            WHERE TO_CHAR(fecha, 'YYYY') = '2011'
            GROUP BY id_aseguradora, id_paciente
            HAVING SUM(precio_facturado) = (
                SELECT MIN(SUM(precio_facturado))
                    FROM consulta
                    WHERE TO_CHAR(fecha, 'YYYY') = '2011'
                    GROUP BY id_paciente
            );
    \end{minted}
\end{ejercicio}

\begin{ejercicio}[SQL]
    Encontrar el nombre y sueldo de los médicos con especialidad \verb|VASCULAR| que, para cada aseguradora que hay en la base de datos, han hecho al menos una consulta con un precio superior a $100$ a un paciente de esa aseguradora.\\

    Buscamos los médicos con especialidad \verb|VASCULAR| para los que no existe una aseguradora para la que no hay una consulta con precio superior a $100$ a un paciente de esa aseguradora.
    \begin{minted}{sql}
        SELECT id_medico, nombre_medico, sueldo
            FROM medico
            WHERE especialidad = 'VASCULAR'
                AND NOT EXISTS (
                    SELECT id_aseguradora
                        FROM aseguradora
                    MINUS
                    SELECT id_aseguradora
                        FROM paciente NATURAL JOIN consulta
                        WHERE precio_facturado > 100
                            AND consulta.id_medico = medico.id_medico
                );
    \end{minted}
\end{ejercicio}


\begin{ejercicio}[SQL]
    Encontrar el \verb|id_paciente| y el \verb|id_aseguradora| de los pacientes que, para cada especialidad de los médicos que tienen un sueldo superior a 4000, han sido atendidos al menos una vez durante menos de una hora por un médico con esa especialidad (gane lo que gane ese médico).\\

    Buscamos los pacientes para los que no existe una especialidad de los médicos con sueldo superior a $4000$ para la que no hay una consulta entre dicho paciente y un médico con esa especialidad que haya durado menos de una hora.
    \begin{minted}{sql}
        SELECT id_paciente, id_aseguradora
            FROM paciente
            WHERE NOT EXISTS (
                SELECT especialidad
                    FROM medico
                    WHERE sueldo > 4000
                MINUS
                SELECT especialidad
                    FROM medico consulta
                    WHERE numero_horas < 1
                    AND consulta.id_paciente = paciente.id_paciente
            );
    \end{minted}
\end{ejercicio}

\begin{ejercicio}[SQL]
    Encontrar, considerando sólo los médicos que han realizado consultas en el año 2017, el nombre del médico que ha facturado en total más dinero sumando sus consultas durante dicho año. Puede haber más de un médico que cumpla esta condición.
    \begin{minted}{sql}
        SELECT id_medico, nombre_medico, SUM(precio_facturado)
            FROM medico NATURAL JOIN consulta
            WHERE TO_CHAR(fecha, 'YYYY')='2017'
            GROUP BY id_medico, nombre_medico
            HAVING SUM(precio_facturado)=(
                SELECT MAX(SUM(precio_facturado))
                    FROM consulta
                    WHERE TO_CHAR(fecha, 'YYYY')='2017'
                    GROUP BY id_medico
            );
    \end{minted}
\end{ejercicio}

\begin{ejercicio}[AR]
    Elaborar un listado con el id de los pacientes atendidos en consulta para los que todas sus consultas son siempre del mismo médico.\\

    Creamos en primer lugar el siguiente alias.
    \begin{equation*}
        \rho(\text{Consulta})=C_1,C_2
    \end{equation*}

    Los pacientes buscados son los que no han sido atendidos por dos médicos distintos:
    \begin{equation*}
        \pi_{\text{id\_paciente}}(\text{Consulta})
        - \pi_{C_1.\text{id\_paciente}}\left(\sigma_{\substack{C_1.\text{id\_paciente}=C_2.\text{id\_paciente}\\\land \\ C_1.\text{id\_médico}\neq C_2.\text{id\_médico}}}(C_1\times C_2)\right)
    \end{equation*}
\end{ejercicio}

\begin{ejercicio}[AR]
    Elaborar un listado de países con una única aseguradora.\\
    Creamos en primer lugar el siguiente alias.
    \begin{equation*}
        \rho(\text{Aseguradora})=A_1,A_2
    \end{equation*}

    Los pacientes buscados son los que no han sido atendidos por dos médicos distintos:
    \begin{equation*}
        \pi_{\text{país}}(\text{Aseguradora})
        - \pi_{A_1.\text{país}}\left(\sigma_{\substack{A_1.\text{país}=A_2.\text{país}\\\land \\ A_1.\text{id\_aseguradora}\neq A_2.\text{id\_aseguradora}}}(A_1\times A_2)\right)
    \end{equation*}
\end{ejercicio}


\newpage
\section{Modelo 2}\label{sec:modelo2}

Considerar el esquema relacional de la Figura~\ref{fig:modelo2}, donde cada proyecto tiene asignada la empresa que lo elabora. La tabla \verb|Revisión| registra revisiones indicando qué revisor revisa qué proyecto en qué fecha.
\begin{figure}[H]
    \centering
    \begin{tikzpicture}[node distance=4 em]
        \node (empresa) {Empresa(id\_empresa, nombre\_empresa, país)};
        \node (proyecto) [below of=empresa] {Proyecto(id\_proyecto, titulo, id\_empresa, fecha\_elaboracion)};
        \node (revisor) [below of=proyecto] {Revisor(id\_revisor, nombre\_revisor, categoria, tarifa)};
        \node (revision) [below of=revisor] {Revisión(id\_revisor, id\_proyecto, fecha, puntuacion)};

        \node[CP, xshift=-7ex] (CPEmpresa) at(empresa) {id\_empresa};
        \node[CP, xshift=-13ex] (CPRevisor) at(revisor) {id\_revisor};
        \node[CP, xshift=-14ex] (CPProyecto) at(proyecto) {id\_proyecto};
        \node[CP, xshift=-2ex] (CPRevision) at(revision) {id\_revisor, id\_proyecto, fecha};

        \node[CE, xshift=19ex] (CEProyecto) at(CPProyecto) {id\_empresa};
        \node[CE, xshift=-10ex] (CERevisionRevisor) at(CPRevision) {id\_revisor};
        \node[CE, xshift=2ex] (CERevisionProyecto) at(CPRevision) {id\_proyecto};

        \node[yshift=1.5em, xshift=1.5em, purple] at (CEProyecto) {(1)};
        \node[yshift=1.5em, xshift=1.5em, purple] at (CERevisionRevisor) {(2)};
        \node[yshift=1.5em, xshift=1.5em, purple] at (CERevisionProyecto) {(3)};

        \node[yshift=-1.5em, xshift=1.5em, purple] at (CPEmpresa) {(1)};
        \node[yshift=-1.5em, xshift=1.5em, purple] at (CPRevisor) {(2)};
        \node[yshift=-1.5em, xshift=1.5em, purple] at (CPProyecto) {(3)};
    \end{tikzpicture}
    \caption{Esquema relacional del Modelo~\ref{sec:modelo2}.}
    \label{fig:modelo2}
\end{figure}

\begin{ejercicio}[SQL]
    Encontrar, considerando sólo los revisores que han hecho revisiones en el 2012, el nombre del revisor que ha otorgado mayor total de puntos sumando todas sus revisiones de ese año. Puede haber más de un revisor que cumpla estacondición.
    \begin{minted}{sql}
        SELECT id_revisor, nombre_revisor, SUM(puntuacion)
            FROM revisor NATURAL JOIN revision
            WHERE TO_CHAR(fecha, 'YYYY') = '2012'
            GROUP BY id_revisor, nombre_revisor
            HAVING SUM(puntuacion) = (
                SELECT MAX(SUM(puntuacion))
                    FROM revision
                    WHERE TO_CHAR(fecha, 'YYYY') = '2012'
                    GROUP BY id_revisor
            );
    \end{minted}
\end{ejercicio}

\begin{ejercicio}[SQL]
    Encontrar el nombre y la tarifa de los revisores con categoría \verb|SENIOR| que, para cada empresa que hay en la base de datos, han hecho al menos una revisión con una puntuación mayor de 75 puntos a un proyecto de esa empresa.\\

    Buscamos los revisores con categoría \verb|SENIOR| para los que no existe una empresa para la que no hay una revisión con puntuación mayor de $75$ a un proyecto de esa empresa.
    \begin{minted}{sql}
        SELECT id_revisor, nombre_revisor, tarifa
            FROM revisor
            WHERE categoria = 'SENIOR'
                AND NOT EXISTS (
                    SELECT id_empresa
                        FROM empresa
                    MINUS
                    SELECT id_empresa
                        FROM proyecto NATURAL JOIN revision
                        WHERE puntuacion > 75
                            AND revision.id_revisor = revisor.id_revisor
                );
    \end{minted}
\end{ejercicio}

\begin{ejercicio}[SQL]
    Encontrar, considerando sólo proyectos que han tenido alguna revisión durante el año 2015, la fecha de elaboración del proyecto que ha tenido más revisiones en dicho año. Puede haber más de un proyecto que cumpla esta condición.
    \begin{minted}{sql}
        SELECT id_proyecto, fecha_elaboracion, COUNT(*)
            FROM proyecto NATURAL JOIN revision
            WHERE TO_CHAR(fecha, 'YYYY') = '2015'
            GROUP BY id_proyecto, fecha_elaboracion
            HAVING COUNT(*) = (
                SELECT MAX(COUNT(*))
                    FROM revision
                    WHERE TO_CHAR(fecha, 'YYYY') = '2015'
                    GROUP BY id_proyecto
            );
    \end{minted}
\end{ejercicio}

\newpage
\section{Modelo 3}\label{sec:modelo3}

Considerar el esquema relacional de la Figura~\ref{fig:modelo3}, donde cada vehículo tiene asignado un modelo de una marca determinada. La tabla \verb|Repara| registra reparaciones indicando qué mecánico repara qué vehículo en qué fecha.
\begin{figure}[H]
    \centering
    \resizebox{1.1\textwidth}{!}{
        \begin{tikzpicture}[node distance=15em]
            \node(modelo) {Modelo(id\_modelo, marca, descripcion)};
            \node(vehiculo)[right of=modelo, xshift=7em] {Vehículo(matrícula, id\_modelo, fecha\_matriculación)};
            \node(mecanico)[below of=modelo, yshift=10em] {Mecánico(id\_mecánico, nombre\_mecánico, cargo, salario)};
            \node(repara)[right of=mecanico, xshift=10em] {Repara(id\_mecánico, matrícula, fecha, número\_horas)};

            \node[CP, xshift=-6ex] (CP_modelo) at(modelo) {id\_modelo};
            \node[CP, xshift=-12ex] (CP_vehiculo) at(vehiculo) {matrícula};
            \node[CP, xshift=-11.5ex] (CP_mecanico) at(mecanico) {id\_mecánico};
            \node[CP, xshift=-3.5ex] (CP_repara) at(repara) {id\_mecánico, matrícula, fecha};

            \node[CE, xshift=11ex] (CE_vehiculo) at(CP_vehiculo) {id\_modelo};
            \node[CE, xshift=-10ex] (CE_repara_mecanico) at(CP_repara) {id\_mecánico};
            \node[CE, xshift=3ex] (CE_repara_vehiculo) at(CP_repara) {matrícula};

            \node[yshift=1.4em, xshift=1.4em, purple] at (CE_vehiculo) {(1)};
            \node[yshift=1.4em, xshift=1.4em, purple] at (CE_repara_mecanico) {(2)};
            \node[yshift=1.4em, xshift=1.4em, purple] at (CE_repara_vehiculo) {(3)};

            \node[yshift=-1.4em, xshift=1.4em, purple] at (CP_modelo) {(1)};
            \node[yshift=-1.4em, xshift=1.4em, purple] at (CP_mecanico) {(2)};
            \node[yshift=-1.4em, xshift=1.4em, purple] at (CP_vehiculo) {(3)};

        \end{tikzpicture}
    }
    \caption{Esquema relacional del Modelo~\ref{sec:modelo3}.}
    \label{fig:modelo3}
\end{figure}

\begin{ejercicio}[SQL]
    Encontrar, de entre los coches que han sufrido reparaciones en el año 2017, la fecha de matriculación del coche al que le han realizado más reparaciones en dicho año. Puede haber más de un coche que cumpla esta condición.
    \begin{minted}{sql}
        SELECT matrícula, fecha_matriculación, COUNT(*)
            FROM vehículo NATURAL JOIN repara
            WHERE TO_CHAR(fecha, 'YYYY') = '2017'
            GROUP BY matrícula, fecha_matriculación
            HAVING COUNT(*) = (
                SELECT MAX(COUNT(*))
                    FROM repara
                    WHERE TO_CHAR(fecha, 'YYYY') = '2017'
                    GROUP BY matrícula
            );
    \end{minted}
\end{ejercicio}

\begin{ejercicio}[SQL]
    Encontrar la matrícula y la fecha de matriculación de los vehículos con \verb|id_modelo| igual a \verb|AURIS| que, por cada cargo de mecánico que han en la BD, han sido atendidos al menos una vez durante más de una hora por un mecánico con ese cargo.\\

    Buscamos los vehículos con \verb|id_modelo| igual a \verb|AURIS| para los que no existe un cargo de mecánico para el que no hay una reparación con duración mayor de una hora a un vehículo de ese modelo.
    \begin{minted}{sql}
        SELECT matrícula, fecha_matriculación
            FROM vehículo
            WHERE id_modelo = 'AURIS'
                AND NOT EXISTS (
                    SELECT cargo
                        FROM mecánico
                    MINUS
                    SELECT cargo
                        FROM mecánico NATURAL JOIN repara
                        WHERE número_horas > 1
                            AND repara.matrícula = vehículo.matrícula
                );
    \end{minted}
\end{ejercicio}

\newpage
\section{Modelo 4}\label{sec:modelo4}

Considerar el esquema relacional de la Figura~\ref{fig:modelo4}, donde cada grupo musical tiene asignado un género musical. La tabla \verb|Actuación| registra
actuaciones indicando qué grupo actúa en qué población y cuándo.
\begin{figure}[H]
    \centering
    \begin{tikzpicture}[node distance=4 em]
        \node (genero) {Género(id\_género, nombre\_género, país\_origen)};
        \node (grupo) [below of=genero] {Grupo(id\_grupo, nombre\_grupo, id\_género, fecha\_creación)};
        \node (poblacion) [below of=grupo] {Población(id\_población, nombre\_población, número, habitantes)};
        \node (actuacion) [below of=poblacion] {Actuación(id\_poblacion, id\_grupo, fecha, duración)};

        \node[CP, xshift=-10ex] (CPGenero) at(genero) {id\_genero};
        \node[CP, xshift=-14.5ex] (CPPoblacion) at(poblacion) {id\_poblacion};
        \node[CP, xshift=-17ex] (CPGrupo) at(grupo) {id\_grupo};
        \node[CP, xshift=0ex] (CPActuacion) at(actuacion) {id\_poblacion, id\_grupo, fecha};

        \node[CE, xshift=25ex] (CEGenero) at(CPGrupo) {id\_genero};
        \node[CE, xshift=-9ex] (CEActuacionPoblacion) at(CPActuacion) {id\_población};
        \node[CE, xshift=4ex] (CEActuacionGrupo) at(CPActuacion) {id\_grupo};

        \node[yshift=1.5em, xshift=1.5em, purple] at (CEGenero) {(1)};
        \node[yshift=1.5em, xshift=1.5em, purple] at (CEActuacionPoblacion) {(2)};
        \node[yshift=1.5em, xshift=1.5em, purple] at (CEActuacionGrupo) {(3)};

        \node[yshift=-1.5em, xshift=1.5em, purple] at (CPGenero) {(1)};
        \node[yshift=-1.5em, xshift=1.5em, purple] at (CPPoblacion) {(2)};
        \node[yshift=-1.5em, xshift=1.5em, purple] at (CPGrupo) {(3)};
    \end{tikzpicture}
    \caption{Esquema relacional del Modelo~\ref{sec:modelo4}.}
    \label{fig:modelo4}
\end{figure}

\begin{ejercicio}[AR]
    Elaborar un listado de los países que han dado origen a un solo género musical.\\

    Creamos en primer lugar el siguiente alias.
    \begin{equation*}
        \rho(\text{Género})=G_1,G_2
    \end{equation*}

    Los países buscados son los que no han dado origen a dos géneros musicales distintos:
    \begin{equation*}
        \pi_{\text{país}}(\text{Género})
        - \pi_{G_1.\text{país}}\left(\sigma_{\substack{G_1.\text{país}=G_2.\text{país}\\\land \\ G_1.\text{id\_género}\neq G_2.\text{id\_género}}}(G_1\times G_2)\right)
    \end{equation*}
\end{ejercicio}

\newpage
\section{Ejercicios Comunes}

\begin{ejercicio}[AR]\label{ej:comunes1}
    Considere el esquema relacional de la Figura~\ref{fig:comunes1}.
    \begin{figure}[H]
        \centering
        \begin{tikzpicture}[node distance=7 em]
            \node (A) {$A(F,B)$};
            \node (D) [right of=A] {$D(E,B)$};
            \node (C) [right of=D] {$C(F,E)$};

            \node[CP, xshift=-1ex] (CPA) at(A) {$F$};
            \node[CP, xshift=-1ex] (CPD) at(D) {$E$};
            \node[CP, xshift=1ex] (CPC) at(C) {$F,E$};

            \node[CE, xshift=-2ex] (CEF) at(CPC) {F};
            \node[CE, xshift=2ex] (CEE) at(CPC) {E};

            \node[yshift=1.5em, xshift=-1.5em, purple] at (CEF) {(1)};
            \node[yshift=1.5em, xshift=1.5em, purple] at (CEE) {(2)};

            \node[yshift=-1.5em, xshift=1.5em, purple] at (CPA) {(1)};
            \node[yshift=-1.5em, xshift=1.5em, purple] at (CPD) {(2)};

        \end{tikzpicture}
        \caption{Esquema relacional del Ejercicio~\ref{ej:comunes1}.}
        \label{fig:comunes1}
    \end{figure}

    Encontrar los valores del atributo $F$ de la tabla $A$ que cumplen la condición de que aparecen relacionados en la tabla $C$ con todos los valores de $E$ que tienen asociado en la tabla $D$ el mismo valor del atributo $B$ que tiene asociado el valor de $F$ en la tabla $A$. Es decir, un valor $x$ debería aparecer en el resultado si existe una tupla $(x,y)\in A$ tal que, para cada tupla $(z,y)\in D$, existe una tupla $(x,z)\in C$ (siendo $x\in F$, $z \in E$ e $y\in B$).\\

    Los valores del atributo $F$ de la tabla $A$ son:
    \begin{equation*}
        \pi_{F}(A)
    \end{equation*}

    Los valores de $E$ que tienen asociado en la tabla $D$ el mismo valor del atributo $B$ que tiene asociado el valor de $F$ en la tabla $A$ son:
    \begin{equation*}
        \pi_{E}(D \bowtie A)
    \end{equation*}

    Por tanto, los valores del atributo $F$ que aparecen relacionados en la tabla $C$ con todos los valores de $E$ que tienen asociado en la tabla $D$ el mismo valor del atributo $B$ que tiene asociado el valor de $F$ en la tabla $A$ son:
    \begin{equation*}
        C \div \pi_E(D \bowtie A)
    \end{equation*}

    Como se pide además que dichos valores sean de $A$, la consulta pedida es:
    \begin{equation*}
        \pi_F(A) \cap \left( C \div \pi_E(D \bowtie A) \right)
    \end{equation*}

    No obstante, como $F$ en $C$ es una clave externa a $A$, se tiene que $\pi_F(C)\subset \pi_F(A)$, por lo que la consulta anterior es equivalente a:
    \begin{equation*}
        C \div \pi_E(D \bowtie A)
    \end{equation*}
\end{ejercicio}



\begin{ejercicio}[AR]\label{ej:comunes2}
    Considere el esquema relacional de la Figura~\ref{fig:comunes2}.
    \begin{figure}[H]
        \centering
        \begin{tikzpicture}[node distance=7 em]
            \node (A) {$B(E,A)$};
            \node (D) [right of=A] {$D(F,A)$};
            \node (C) [right of=D] {$C(E,F)$};

            \node[CP, xshift=-1ex] (CPA) at(A) {$F$};
            \node[CP, xshift=-1ex] (CPD) at(D) {$E$};
            \node[CP, xshift=1ex] (CPC) at(C) {$F,E$};

            \node[CE, xshift=-2ex] (CEF) at(CPC) {F};
            \node[CE, xshift=2ex] (CEE) at(CPC) {E};

            \node[yshift=1.5em, xshift=-1.5em, purple] at (CEF) {(1)};
            \node[yshift=1.5em, xshift=1.5em, purple] at (CEE) {(2)};

            \node[yshift=-1.5em, xshift=1.5em, purple] at (CPA) {(1)};
            \node[yshift=-1.5em, xshift=1.5em, purple] at (CPD) {(2)};

        \end{tikzpicture}
        \caption{Esquema relacional del Ejercicio~\ref{ej:comunes2}.}
        \label{fig:comunes2}
    \end{figure}

    Encontrar los valores del atributo $E$ de la tabla $B$ que cumplen la condición de que aparecen relacionados en la tabla $C$ con todos los valores de $F$ que tienen asociado en la tabla $D$ el mismo valor del atributo $A$ que tiene asociado el valor de $E$ en la tabla $B$. Es decir, un valor $x$ debería aparecer en el resultado si existe una tupla $(x,y)\in B$ tal que, para cada tupla $(z,y)\in D$, existe una tupla $(x,z)\in C$ (siendo $x\in E$, $z \in F$ e $y\in A$).\\

    Los valores del atributo $E$ de la tabla $B$ son:
    \begin{equation*}
        \pi_{E}(B)
    \end{equation*}

    Los valores de $F$ que tienen asociado en la tabla $D$ el mismo valor del atributo $A$ que tiene asociado el valor de $E$ en la tabla $B$ son:
    \begin{equation*}
        \pi_{F}(D \bowtie B)
    \end{equation*}

    Por tanto, los valores del atributo $E$ que aparecen relacionados en la tabla $C$ con todos los valores de $F$ que tienen asociado en la tabla $D$ el mismo valor del atributo $A$ que tiene asociado el valor de $E$ en la tabla $B$ son:
    \begin{equation*}
        C \div \pi_F(D \bowtie B)
    \end{equation*}

    Como se pide además que dichos valores sean de $B$, la consulta pedida es:
    \begin{equation*}
        \pi_E(B) \cap \left( C \div \pi_F(D \bowtie B) \right)
    \end{equation*}

    No obstante, como $E$ en $C$ es una clave externa a $B$, se tiene que $\pi_E(C)\subset \pi_E(B)$, por lo que la consulta anterior es equivalente a:
    \begin{equation*}
        C \div \pi_F(D \bowtie B)
    \end{equation*}
\end{ejercicio}


\end{document}