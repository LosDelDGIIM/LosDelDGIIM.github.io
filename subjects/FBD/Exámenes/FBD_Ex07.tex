\documentclass[12pt]{article}

% Idioma y codificación
\usepackage[spanish, es-tabla]{babel}       %es-tabla para que se titule "Tabla"
\usepackage[utf8]{inputenc}

% Márgenes
\usepackage[a4paper,top=3cm,bottom=2.5cm,left=3cm,right=3cm]{geometry}

% Comentarios de bloque
\usepackage{verbatim}

% Paquetes de links
\usepackage[hidelinks]{hyperref}    % Permite enlaces
\usepackage{url}                    % redirecciona a la web

% Más opciones para enumeraciones
\usepackage{enumitem}

% Personalizar la portada
\usepackage{titling}

% Paquetes de tablas
\usepackage{multirow}


%------------------------------------------------------------------------

%Paquetes de figuras
\usepackage{caption}
\usepackage{subcaption} % Figuras al lado de otras
\usepackage{float}      % Poner figuras en el sitio indicado H.


% Paquetes de imágenes
\usepackage{graphicx}       % Paquete para añadir imágenes
\usepackage{transparent}    % Para manejar la opacidad de las figuras

% Paquete para usar colores
\usepackage[dvipsnames]{xcolor}
\usepackage{pagecolor}      % Para cambiar el color de la página

% Habilita tamaños de fuente mayores
\usepackage{fix-cm}

% Para los gráficos
\usepackage{tikz}

% Para poder situar los nodos en los grafos
\usetikzlibrary{positioning}


%------------------------------------------------------------------------

% Paquetes de matemáticas
\usepackage{mathtools, amsfonts, amssymb, mathrsfs}
\usepackage[makeroom]{cancel}     % Simplificar tachando
\usepackage{polynom}    % Divisiones y Ruffini
\usepackage{units} % Para poner fracciones diagonales con \nicefrac

\usepackage{pgfplots}   %Representar funciones
\pgfplotsset{compat=1.18}  % Versión 1.18

\usepackage{tikz-cd}    % Para usar diagramas de composiciones
\usetikzlibrary{calc}   % Para usar cálculo de coordenadas en tikz

%Definición de teoremas, etc.
\usepackage{amsthm}
%\swapnumbers   % Intercambia la posición del texto y de la numeración

\theoremstyle{plain}

\makeatletter
\@ifclassloaded{article}{
  \newtheorem{teo}{Teorema}[section]
}{
  \newtheorem{teo}{Teorema}[chapter]  % Se resetea en cada chapter
}
\makeatother

\newtheorem{coro}{Corolario}[teo]           % Se resetea en cada teorema
\newtheorem{prop}[teo]{Proposición}         % Usa el mismo contador que teorema
\newtheorem{lema}[teo]{Lema}                % Usa el mismo contador que teorema

\theoremstyle{remark}
\newtheorem*{observacion}{Observación}

\theoremstyle{definition}

\makeatletter
\@ifclassloaded{article}{
  \newtheorem{definicion}{Definición} [section]     % Se resetea en cada chapter
}{
  \newtheorem{definicion}{Definición} [chapter]     % Se resetea en cada chapter
}
\makeatother

\newtheorem*{notacion}{Notación}
\newtheorem*{ejemplo}{Ejemplo}
\newtheorem*{ejercicio*}{Ejercicio}             % No numerado
\newtheorem{ejercicio}{Ejercicio} [section]     % Se resetea en cada section


% Modificar el formato de la numeración del teorema "ejercicio"
\renewcommand{\theejercicio}{%
  \ifnum\value{section}=0 % Si no se ha iniciado ninguna sección
    \arabic{ejercicio}% Solo mostrar el número de ejercicio
  \else
    \thesection.\arabic{ejercicio}% Mostrar número de sección y número de ejercicio
  \fi
}


% \renewcommand\qedsymbol{$\blacksquare$}         % Cambiar símbolo QED
%------------------------------------------------------------------------

% Paquetes para encabezados
\usepackage{fancyhdr}
\pagestyle{fancy}
\fancyhf{}

\newcommand{\helv}{ % Modificación tamaño de letra
\fontfamily{}\fontsize{12}{12}\selectfont}
\setlength{\headheight}{15pt} % Amplía el tamaño del índice


%\usepackage{lastpage}   % Referenciar última pag   \pageref{LastPage}
\fancyfoot[C]{\thepage}

%------------------------------------------------------------------------

% Conseguir que no ponga "Capítulo 1". Sino solo "1."
\makeatletter
\@ifclassloaded{book}{
  \renewcommand{\chaptermark}[1]{\markboth{\thechapter.\ #1}{}} % En el encabezado
    
  \renewcommand{\@makechapterhead}[1]{%
  \vspace*{50\p@}%
  {\parindent \z@ \raggedright \normalfont
    \ifnum \c@secnumdepth >\m@ne
      \huge\bfseries \thechapter.\hspace{1em}\ignorespaces
    \fi
    \interlinepenalty\@M
    \Huge \bfseries #1\par\nobreak
    \vskip 40\p@
  }}
}
\makeatother

%------------------------------------------------------------------------
% Paquetes de cógido
\usepackage{minted}
\renewcommand\listingscaption{Código fuente}

\usepackage{fancyvrb}
% Personaliza el tamaño de los números de línea
\renewcommand{\theFancyVerbLine}{\small\arabic{FancyVerbLine}}

% Estilo para C++
\newminted{cpp}{
    frame=lines,
    framesep=2mm,
    baselinestretch=1.2,
    linenos,
    escapeinside=||
}

% para minted
\definecolor{LightGray}{rgb}{0.95,0.95,0.92}
\setminted{
    linenos=true,
    stepnumber=5,
    numberfirstline=true,
    autogobble,
    breaklines=true,
    breakautoindent=true,
    breaksymbolleft=,
    breaksymbolright=,
    breaksymbolindentleft=0pt,
    breaksymbolindentright=0pt,
    breaksymbolsepleft=0pt,
    breaksymbolsepright=0pt,
    fontsize=\footnotesize,
    bgcolor=LightGray,
    numbersep=10pt
}


\usepackage{listings} % Para incluir código desde un archivo

\renewcommand\lstlistingname{Código Fuente}
\renewcommand\lstlistlistingname{Índice de Códigos Fuente}

% Definir colores
\definecolor{vscodepurple}{rgb}{0.5,0,0.5}
\definecolor{vscodeblue}{rgb}{0,0,0.8}
\definecolor{vscodegreen}{rgb}{0,0.5,0}
\definecolor{vscodegray}{rgb}{0.5,0.5,0.5}
\definecolor{vscodebackground}{rgb}{0.97,0.97,0.97}
\definecolor{vscodelightgray}{rgb}{0.9,0.9,0.9}

% Configuración para el estilo de C similar a VSCode
\lstdefinestyle{vscode_C}{
  backgroundcolor=\color{vscodebackground},
  commentstyle=\color{vscodegreen},
  keywordstyle=\color{vscodeblue},
  numberstyle=\tiny\color{vscodegray},
  stringstyle=\color{vscodepurple},
  basicstyle=\scriptsize\ttfamily,
  breakatwhitespace=false,
  breaklines=true,
  captionpos=b,
  keepspaces=true,
  numbers=left,
  numbersep=5pt,
  showspaces=false,
  showstringspaces=false,
  showtabs=false,
  tabsize=2,
  frame=tb,
  framerule=0pt,
  aboveskip=10pt,
  belowskip=10pt,
  xleftmargin=10pt,
  xrightmargin=10pt,
  framexleftmargin=10pt,
  framexrightmargin=10pt,
  framesep=0pt,
  rulecolor=\color{vscodelightgray},
  backgroundcolor=\color{vscodebackground},
}

%------------------------------------------------------------------------

% Comandos definidos
\newcommand{\bb}[1]{\mathbb{#1}}
\newcommand{\cc}[1]{\mathcal{#1}}

% I prefer the slanted \leq
\let\oldleq\leq % save them in case they're every wanted
\let\oldgeq\geq
\renewcommand{\leq}{\leqslant}
\renewcommand{\geq}{\geqslant}

% Si y solo si
\newcommand{\sii}{\iff}

% Letras griegas
\newcommand{\eps}{\epsilon}
\newcommand{\veps}{\varepsilon}
\newcommand{\lm}{\lambda}

\newcommand{\ol}{\overline}
\newcommand{\ul}{\underline}
\newcommand{\wt}{\widetilde}
\newcommand{\wh}{\widehat}

\let\oldvec\vec
\renewcommand{\vec}{\overrightarrow}

% Derivadas parciales
\newcommand{\del}[2]{\frac{\partial #1}{\partial #2}}
\newcommand{\Del}[3]{\frac{\partial^{#1} #2}{\partial #3^{#1}}}
\newcommand{\deld}[2]{\dfrac{\partial #1}{\partial #2}}
\newcommand{\Deld}[3]{\dfrac{\partial^{#1} #2}{\partial #3^{#1}}}


\newcommand{\AstIg}{\stackrel{(\ast)}{=}}
\newcommand{\Hop}{\stackrel{L'H\hat{o}pital}{=}}

\newcommand{\red}[1]{{\color{red}#1}} % Para integrales, destacar los cambios.

% Método de integración
\newcommand{\MetInt}[2]{
    \left[\begin{array}{c}
        #1 \\ #2
    \end{array}\right]
}

% Declarar aplicaciones
% 1. Nombre aplicación
% 2. Dominio
% 3. Codominio
% 4. Variable
% 5. Imagen de la variable
\newcommand{\Func}[5]{
    \begin{equation*}
        \begin{array}{rrll}
            #1:& #2 & \longrightarrow & #3\\
               & #4 & \longmapsto & #5
        \end{array}
    \end{equation*}
}

%------------------------------------------------------------------------

\usetikzlibrary{er, fit}

\newcommand{\key}[1]{\ul{#1}}

% Definición de estilos
\tikzset{atributo/.style={attribute}}
\tikzset{entidad/.style={entity}}
\tikzset{relacion/.style={relationship}}
\tikzset{atributo derivado/.style={attribute, dashed}}
\tikzset{union discriminante/.style={dashed}}
\tikzset{entidad debil/.style={entidad, double distance=1.5 pt}}
\tikzset{participacion obligatoria/.style={double distance=1.5 pt}}
\tikzset{herencia/.style={draw, isosceles triangle, isosceles triangle apex angle=60, shape border rotate=230, minimum height=1cm, minimum width=1cm, fill=blue!20}}
\tikzset{agregacion/.style={draw, fit=#1, inner sep=0.5em}}


% Estilos estéticos
\tikzset{every entity/.style={draw=orange, fill=orange!20}}
\tikzset{every attribute/.style={draw=purple, fill=purple!20, font=\footnotesize}}
\tikzset{every relationship/.style={draw=green, fill=green!20, minimum height=2cm, minimum width=2cm}}

\begin{comment}
    \begin{tikzpicture}[node distance=6 em]
        \node [entidad](person){Person};
        \node [atributo derivado](pid) at (0.8,-2){\key{ID}} edge (person);
        \node [atributo](name) at (-0.8, -2) {Name} edge (person);
        \node [atributo](phone)[left of=person]{Phone} edge[union discriminante] (person);
        \node [atributo](address)[above left of=person]{Address} edge[-stealth] (person);
        \node [atributo](street)[above left of=address]{Street} edge (address);
        \node [atributo](city)[left of=address]{City} edge (address);
        \node [atributo](age)[above of=person]{Age} edge (person);
        \node [relacion](uses)[right of=person]{Uses} edge[participacion obligatoria] (person);
        \node [entidad debil](tool)[right of=uses]{Tool} edge (uses);
        \node [atributo](tid)[right of=tool]{\key{ID}} edge[Stealth-Stealth] (tool);
        \node [atributo](tname)[below of=tool]{Name} edge (tool);
    
        % Línea desde encima de Name hasta encima de ID
        \draw ([yshift=1em]name.north) -- ([yshift=1em]pid.north) node[circle, fill, inner sep=1.5pt] {};
    
        \node[agregacion=(A) (B) (R)] (box) {};
    
        \node [herencia, below of= B](H){} edge (B);
    \end{tikzpicture}
\end{comment}

% ---------------------------------------------------
\tikzset{
    CP/.style={
        overlay, text opacity=0, fill opacity=0,
        append after command={
            \pgfextra{
                \draw[thick] (\tikzlastnode.south west) -- (\tikzlastnode.south east);
                \node[below=0ex of \tikzlastnode] {CP};
            }
        }
    },
    CC/.style={
        overlay, text opacity=0, fill opacity=0,
        append after command={
            \pgfextra{
                \draw[thick] (\tikzlastnode.south west) -- (\tikzlastnode.south east);
                \node[below=0ex of \tikzlastnode] {CC};
            }
        }
    },
    CE/.style={
        overlay, text opacity=0, fill opacity=0,
        append after command={
            \pgfextra{
                \draw[thick, purple] (\tikzlastnode.north west) -- (\tikzlastnode.north east);
                \node[above=0ex of \tikzlastnode, text=purple] {CE};
            }
        }
    }
}


\begin{document}

    % 1. Foto de fondo
    % 2. Título
    % 3. Encabezado Izquierdo
    % 4. Color de fondo
    % 5. Coord x del titulo
    % 6. Coord y del titulo
    % 7. Fecha

    
    % 1. Foto de fondo
% 2. Título
% 3. Encabezado Izquierdo
% 4. Color de fondo
% 5. Coord x del titulo
% 6. Coord y del titulo
% 7. Fecha

\newcommand{\portada}[7]{

    \portadaBase{#1}{#2}{#3}{#4}{#5}{#6}{#7}
    \portadaBook{#1}{#2}{#3}{#4}{#5}{#6}{#7}
}

\newcommand{\portadaExamen}[7]{

    \portadaBase{#1}{#2}{#3}{#4}{#5}{#6}{#7}
    \portadaArticle{#1}{#2}{#3}{#4}{#5}{#6}{#7}
}




\newcommand{\portadaBase}[7]{

    % Tiene la portada principal y la licencia Creative Commons
    
    % 1. Foto de fondo
    % 2. Título
    % 3. Encabezado Izquierdo
    % 4. Color de fondo
    % 5. Coord x del titulo
    % 6. Coord y del titulo
    % 7. Fecha
    
    
    \thispagestyle{empty}               % Sin encabezado ni pie de página
    \newgeometry{margin=0cm}        % Márgenes nulos para la primera página
    
    
    % Encabezado
    \fancyhead[L]{\helv #3}
    \fancyhead[R]{\helv \nouppercase{\leftmark}}
    
    
    \pagecolor{#4}        % Color de fondo para la portada
    
    \begin{figure}[p]
        \centering
        \transparent{0.3}           % Opacidad del 30% para la imagen
        
        \includegraphics[width=\paperwidth, keepaspectratio]{assets/#1}
    
        \begin{tikzpicture}[remember picture, overlay]
            \node[anchor=north west, text=white, opacity=1, font=\fontsize{60}{90}\selectfont\bfseries\sffamily, align=left] at (#5, #6) {#2};
            
            \node[anchor=south east, text=white, opacity=1, font=\fontsize{12}{18}\selectfont\sffamily, align=right] at (9.7, 3) {\textbf{\href{https://losdeldgiim.github.io/}{Los Del DGIIM}}};
            
            \node[anchor=south east, text=white, opacity=1, font=\fontsize{12}{15}\selectfont\sffamily, align=right] at (9.7, 1.8) {Doble Grado en Ingeniería Informática y Matemáticas\\Universidad de Granada};
        \end{tikzpicture}
    \end{figure}
    
    
    \restoregeometry        % Restaurar márgenes normales para las páginas subsiguientes
    \pagecolor{white}       % Restaurar el color de página
    
    
    \newpage
    \thispagestyle{empty}               % Sin encabezado ni pie de página
    \begin{tikzpicture}[remember picture, overlay]
        \node[anchor=south west, inner sep=3cm] at (current page.south west) {
            \begin{minipage}{0.5\paperwidth}
                \href{https://creativecommons.org/licenses/by-nc-nd/4.0/}{
                    \includegraphics[height=2cm]{assets/Licencia.png}
                }\vspace{1cm}\\
                Esta obra está bajo una
                \href{https://creativecommons.org/licenses/by-nc-nd/4.0/}{
                    Licencia Creative Commons Atribución-NoComercial-SinDerivadas 4.0 Internacional (CC BY-NC-ND 4.0).
                }\\
    
                Eres libre de compartir y redistribuir el contenido de esta obra en cualquier medio o formato, siempre y cuando des el crédito adecuado a los autores originales y no persigas fines comerciales. 
            \end{minipage}
        };
    \end{tikzpicture}
    
    
    
    % 1. Foto de fondo
    % 2. Título
    % 3. Encabezado Izquierdo
    % 4. Color de fondo
    % 5. Coord x del titulo
    % 6. Coord y del titulo
    % 7. Fecha


}


\newcommand{\portadaBook}[7]{

    % 1. Foto de fondo
    % 2. Título
    % 3. Encabezado Izquierdo
    % 4. Color de fondo
    % 5. Coord x del titulo
    % 6. Coord y del titulo
    % 7. Fecha

    % Personaliza el formato del título
    \pretitle{\begin{center}\bfseries\fontsize{42}{56}\selectfont}
    \posttitle{\par\end{center}\vspace{2em}}
    
    % Personaliza el formato del autor
    \preauthor{\begin{center}\Large}
    \postauthor{\par\end{center}\vfill}
    
    % Personaliza el formato de la fecha
    \predate{\begin{center}\huge}
    \postdate{\par\end{center}\vspace{2em}}
    
    \title{#2}
    \author{\href{https://losdeldgiim.github.io/}{Los Del DGIIM}}
    \date{Granada, #7}
    \maketitle
    
    \tableofcontents
}




\newcommand{\portadaArticle}[7]{

    % 1. Foto de fondo
    % 2. Título
    % 3. Encabezado Izquierdo
    % 4. Color de fondo
    % 5. Coord x del titulo
    % 6. Coord y del titulo
    % 7. Fecha

    % Personaliza el formato del título
    \pretitle{\begin{center}\bfseries\fontsize{42}{56}\selectfont}
    \posttitle{\par\end{center}\vspace{2em}}
    
    % Personaliza el formato del autor
    \preauthor{\begin{center}\Large}
    \postauthor{\par\end{center}\vspace{3em}}
    
    % Personaliza el formato de la fecha
    \predate{\begin{center}\huge}
    \postdate{\par\end{center}\vspace{5em}}
    
    \title{#2}
    \author{\href{https://losdeldgiim.github.io/}{Los Del DGIIM}}
    \date{Granada, #7}
    \thispagestyle{empty}               % Sin encabezado ni pie de página
    \maketitle
    \vfill
}
    \portadaExamen{etsiitA4.jpg}{FBD\\Examen VII}{FBD. Examen VII}{MidnightBlue}{-8}{28}{2024-2025}{}

    \begin{description}
        \item[Asignatura] Fundamentos de Bases de Datos.
        \item[Curso Académico] 2024-25.
        \item[Grado] Doble Grado en Ingeniería Informática y Matemáticas.
        \item[Grupo] Único.
        \item[Profesor] Nicolás Marín Ruiz.
        \item[Descripción] Convocatoria Ordinaria. Parcial Práctico 2 (Seminarios 3-4).
        \item[Fecha] 10 de enero de 2025.
        \item[Duración] 90 minutos (durante los cuales también se tenía que hacer la parte de teoría, 20 preguntas tipo test).
    
    \end{description}
    \newpage

    Ante el siguiente esquema de la BD:
    \begin{itemize}
        \item Empresa(id\_empresa, nombre\_empresa, pais)
            \begin{itemize}
                \item Donde ``id\_empresa'' es la clave primaria.
            \end{itemize}
        \item Proyecto(id\_proyecto, id\_empresa, titulo, fecha\_creacion)
            \begin{itemize}
                \item Donde ``id\_proyecto'' es la clave primaria.
                \item ``id\_empresa'' es una clave externa que referencia a Empresa(id\_empresa).
            \end{itemize}
        \item Revisor(id\_revisor, nombre\_revisor, tarifa, categoria)
            \begin{itemize}
                \item Donde ``id\_revisor'' es la clave primaria.
            \end{itemize}
        \item Revision(id\_revisor, id\_proyecto, fecha, puntuacion)
            \begin{itemize}
                \item Donde ``(id\_revisor, id\_proyecto, fecha)'' es la clave primaria.
                \item ``id\_revisor'' es una clave externa que referencia a Revisor(id\_revisor).
                \item ``id\_proyecto'' es una clave externa que referencia a Proyecto(id\_proyecto).
            \end{itemize}
    \end{itemize}

    \subsection*{SQL}
    \begin{ejercicio}
        Actualiza la puntuación de las revisiones del revisor con id ``1234'', sumándole 10 a cada puntuación realizada por él.
    \end{ejercicio}
    \begin{ejercicio}
        Sobre la tabla \verb|Proyecto|:
        \begin{enumerate}[label=\alph*)]
            \item Cree un índice de mapa de bits sobre el campo \verb|titulo|.
            \item ¿Es una buena opción? Justifique su respuesta.
            \item Borre el índice.
        \end{enumerate}
    \end{ejercicio}
    \begin{ejercicio}
        Mostrar para cada revisor la suma de las puntuaciones de las revisiones hechas en 2024 solo de aquellos proveedores que hayan hecho al menos 5 revisiones en 2023.
    \end{ejercicio}
    \begin{ejercicio}
        Mostrar los revisores que han hecho al menos una revisión a cada uno de los proyectos de la empresa con id ``EMP1''. 
    \end{ejercicio}
    \subsection*{AR}
    \begin{ejercicio}
        Haga una lista de \verb|id_revisor|, \verb|fecha| de las revisiones realizadas ppr los revisores de categoría ``SENIOR''.
    \end{ejercicio}
    \begin{ejercicio}
        Mostrar las empresas de los proyectos que han obtenido la mayor puntuación.
    \end{ejercicio}
    \begin{ejercicio}
        Mostrar el id de los revisores cuyo conjunto de puntuaciones otorgadas coincide con el conjunto de puntuaciones que han sido otorgadas por el revisor de id ``1234''.
    \end{ejercicio}

    \newpage
    \setcounter{ejercicio}{0}
    \subsection*{SQL}
    \begin{ejercicio}
        Actualiza la puntuación de las revisiones del revisor con id ``1234'', sumándole 10 a cada puntuación realizada por él.
        \begin{minted}[numbers=none]{sql}
            update revision set puntuacion = puntuacion + 10
            where id_revisor = '1234';
        \end{minted}
    \end{ejercicio}
    \begin{ejercicio}
        Sobre la tabla \verb|Proyecto|:
        \begin{enumerate}[label=\alph*)]
            \item Cree un índice de mapa de bits sobre el campo \verb|titulo|.
                \begin{minted}[numbers=none]{sql}
                    create bitmap index indice_bmp_titulo
                    on proyecto(titulo);
                \end{minted}
            \item ¿Es una buena opción? Justifique su respuesta.\\
                No es una opción buena, ya que los índices de mapas de bits son útiles sobre atributos de relaciones cuyo dominio activo cuenta con pocos valores. En este caso, es probable que el atributo ``titulo'' de la tabla ``Proyecto'' contenga un título distinto por cada proyecto (aunque también existan proyectos que tengan el mismo título, aunque pocos), así como una gran variedad de títulos en el sistema.

                Como contamos una entradad por cada valor y un bit por cada registro, obtendríamos una gran cantidad de enormes mapas de bits en los que cada uno contenga una gran cantidad de ceros.
            \item Borre el índice.
                \begin{minted}[numbers=none]{sql}
                    drop index indice_bmp_titulo;
                \end{minted}
        \end{enumerate}
    \end{ejercicio}
    \begin{ejercicio}
        Mostrar para cada revisor la suma de las puntuaciones de las revisiones hechas en 2024 solo de aquellos proveedores que hayan hecho al menos 5 revisiones en 2023.
        \begin{minted}[numbers=none]{sql}
            select id_revisor, sum(puntuacion)
            from revision
            where to_char(fecha, 'yyyy') = 2024
            group by id_revisor
            having id_revisor in (
                select id_revisor
                from revision
                where to_char(fecha, 'yyyy') = 2023
                group by id_revisor
                having count(*) >= 5
            );
        \end{minted}    
    \end{ejercicio}
    \begin{ejercicio}
        Mostrar los revisores que han hecho al menos una revisión a cada uno de los proyectos de la empresa con id ``EMP1''. 
        \begin{minted}[numbers=none]{sql}
            select distinct id_revisor
            from revision r
            where not exists (
                select * from proyecto p
                where id_empresa = 'EMP1' and not exists (
                    select * from revision 
                    where id_revisor = r.id_revisor
                      and id_proyecto = p.id_proyecto
                )
            );
        \end{minted}    
    \end{ejercicio}
    \subsection*{AR}
    \begin{ejercicio}
        Haga una lista de \verb|id_revisor|, \verb|fecha| de las revisiones realizadas ppr los revisores de categoría ``SENIOR''.
        \begin{equation*}
            \pi_{id\_revisor, fecha}(Revision \bowtie \sigma_{categoria='SENIOR'}(Revisor))
        \end{equation*}
    \end{ejercicio}
    \begin{ejercicio}
        Mostrar las empresas de los proyectos que han obtenido la mayor puntuación.\\

        Primero calculamos los proyectos que han obtenido la mayor puntuación en una revisión:
        \begin{gather*}
            R = \rho(Revision) \\
            \alpha = \rho(\pi_{id\_proyecto}(R) - \pi_{R.id\_proyecto}(R\bowtie_{R.puntuacion<Revision.puntuacion} Revision));
        \end{gather*}   
        Finalmente, nos quedamos con las empresas de dichos proyectos:
        \begin{equation*}
            \pi_{id\_empresa}(\alpha \bowtie Proyecto)
        \end{equation*}
    \end{ejercicio}
    \begin{ejercicio}
        Mostrar el id de los revisores cuyo conjunto de puntuaciones otorgadas coincide con el conjunto de puntuaciones que han sido otorgadas por el revisor de id ``1234''.\\

        \noindent
        Conjunto de puntuaciones otorgadas por el revisor de id ``1234'':
        \begin{equation*}
            P = \rho(\pi_{puntuacion}(\sigma_{id\_revisor='1234'}(Revision)))
        \end{equation*}
        Revisores que han otorgado puntuaciones que no están en $P$:
        \begin{equation*}
            \alpha = \rho (\pi_{id\_revisor}(Revision \bowtie (\pi_{puntuacion}(Revision) - P)))
        \end{equation*}
        Revisores que no han otorgado puntuaciones que no están en $P$ (es decir, el conjunto de revisores que solo otorgan puntuaciones de $P$):
        \begin{equation*}
            \beta = \rho(\pi_{id\_revisor}(Revision) - \alpha)
        \end{equation*}
        Tenemos que asegurarnos de que dichos revisor otorgen \textbf{todas} las puntuaciones que hay en $P$, con lo que habrán otorgado exactamente las puntuaciones que hay en $P$:
        \begin{equation*}
            \pi_{id\_revisor,\ puntuacion}(\beta \bowtie Revision) \div P
        \end{equation*}
    \end{ejercicio}


\end{document}
