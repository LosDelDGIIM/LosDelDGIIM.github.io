\documentclass[12pt]{article}

% Idioma y codificación
\usepackage[spanish, es-tabla]{babel}       %es-tabla para que se titule "Tabla"
\usepackage[utf8]{inputenc}

% Márgenes
\usepackage[a4paper,top=3cm,bottom=2.5cm,left=3cm,right=3cm]{geometry}

% Comentarios de bloque
\usepackage{verbatim}

% Paquetes de links
\usepackage[hidelinks]{hyperref}    % Permite enlaces
\usepackage{url}                    % redirecciona a la web

% Más opciones para enumeraciones
\usepackage{enumitem}

% Personalizar la portada
\usepackage{titling}

% Paquetes de tablas
\usepackage{multirow}


%------------------------------------------------------------------------

%Paquetes de figuras
\usepackage{caption}
\usepackage{subcaption} % Figuras al lado de otras
\usepackage{float}      % Poner figuras en el sitio indicado H.


% Paquetes de imágenes
\usepackage{graphicx}       % Paquete para añadir imágenes
\usepackage{transparent}    % Para manejar la opacidad de las figuras

% Paquete para usar colores
\usepackage[dvipsnames]{xcolor}
\usepackage{pagecolor}      % Para cambiar el color de la página

% Habilita tamaños de fuente mayores
\usepackage{fix-cm}

% Para los gráficos
\usepackage{tikz}

% Para poder situar los nodos en los grafos
\usetikzlibrary{positioning}


%------------------------------------------------------------------------

% Paquetes de matemáticas
\usepackage{mathtools, amsfonts, amssymb, mathrsfs}
\usepackage[makeroom]{cancel}     % Simplificar tachando
\usepackage{polynom}    % Divisiones y Ruffini
\usepackage{units} % Para poner fracciones diagonales con \nicefrac

\usepackage{pgfplots}   %Representar funciones
\pgfplotsset{compat=1.18}  % Versión 1.18

\usepackage{tikz-cd}    % Para usar diagramas de composiciones
\usetikzlibrary{calc}   % Para usar cálculo de coordenadas en tikz

%Definición de teoremas, etc.
\usepackage{amsthm}
%\swapnumbers   % Intercambia la posición del texto y de la numeración

\theoremstyle{plain}

\makeatletter
\@ifclassloaded{article}{
  \newtheorem{teo}{Teorema}[section]
}{
  \newtheorem{teo}{Teorema}[chapter]  % Se resetea en cada chapter
}
\makeatother

\newtheorem{coro}{Corolario}[teo]           % Se resetea en cada teorema
\newtheorem{prop}[teo]{Proposición}         % Usa el mismo contador que teorema
\newtheorem{lema}[teo]{Lema}                % Usa el mismo contador que teorema

\theoremstyle{remark}
\newtheorem*{observacion}{Observación}

\theoremstyle{definition}

\makeatletter
\@ifclassloaded{article}{
  \newtheorem{definicion}{Definición} [section]     % Se resetea en cada chapter
}{
  \newtheorem{definicion}{Definición} [chapter]     % Se resetea en cada chapter
}
\makeatother

\newtheorem*{notacion}{Notación}
\newtheorem*{ejemplo}{Ejemplo}
\newtheorem*{ejercicio*}{Ejercicio}             % No numerado
\newtheorem{ejercicio}{Ejercicio} [section]     % Se resetea en cada section


% Modificar el formato de la numeración del teorema "ejercicio"
\renewcommand{\theejercicio}{%
  \ifnum\value{section}=0 % Si no se ha iniciado ninguna sección
    \arabic{ejercicio}% Solo mostrar el número de ejercicio
  \else
    \thesection.\arabic{ejercicio}% Mostrar número de sección y número de ejercicio
  \fi
}


% \renewcommand\qedsymbol{$\blacksquare$}         % Cambiar símbolo QED
%------------------------------------------------------------------------

% Paquetes para encabezados
\usepackage{fancyhdr}
\pagestyle{fancy}
\fancyhf{}

\newcommand{\helv}{ % Modificación tamaño de letra
\fontfamily{}\fontsize{12}{12}\selectfont}
\setlength{\headheight}{15pt} % Amplía el tamaño del índice


%\usepackage{lastpage}   % Referenciar última pag   \pageref{LastPage}
\fancyfoot[C]{\thepage}

%------------------------------------------------------------------------

% Conseguir que no ponga "Capítulo 1". Sino solo "1."
\makeatletter
\@ifclassloaded{book}{
  \renewcommand{\chaptermark}[1]{\markboth{\thechapter.\ #1}{}} % En el encabezado
    
  \renewcommand{\@makechapterhead}[1]{%
  \vspace*{50\p@}%
  {\parindent \z@ \raggedright \normalfont
    \ifnum \c@secnumdepth >\m@ne
      \huge\bfseries \thechapter.\hspace{1em}\ignorespaces
    \fi
    \interlinepenalty\@M
    \Huge \bfseries #1\par\nobreak
    \vskip 40\p@
  }}
}
\makeatother

%------------------------------------------------------------------------
% Paquetes de cógido
\usepackage{minted}
\renewcommand\listingscaption{Código fuente}

\usepackage{fancyvrb}
% Personaliza el tamaño de los números de línea
\renewcommand{\theFancyVerbLine}{\small\arabic{FancyVerbLine}}

% Estilo para C++
\newminted{cpp}{
    frame=lines,
    framesep=2mm,
    baselinestretch=1.2,
    linenos,
    escapeinside=||
}

% para minted
\definecolor{LightGray}{rgb}{0.95,0.95,0.92}
\setminted{
    linenos=true,
    stepnumber=5,
    numberfirstline=true,
    autogobble,
    breaklines=true,
    breakautoindent=true,
    breaksymbolleft=,
    breaksymbolright=,
    breaksymbolindentleft=0pt,
    breaksymbolindentright=0pt,
    breaksymbolsepleft=0pt,
    breaksymbolsepright=0pt,
    fontsize=\footnotesize,
    bgcolor=LightGray,
    numbersep=10pt
}


\usepackage{listings} % Para incluir código desde un archivo

\renewcommand\lstlistingname{Código Fuente}
\renewcommand\lstlistlistingname{Índice de Códigos Fuente}

% Definir colores
\definecolor{vscodepurple}{rgb}{0.5,0,0.5}
\definecolor{vscodeblue}{rgb}{0,0,0.8}
\definecolor{vscodegreen}{rgb}{0,0.5,0}
\definecolor{vscodegray}{rgb}{0.5,0.5,0.5}
\definecolor{vscodebackground}{rgb}{0.97,0.97,0.97}
\definecolor{vscodelightgray}{rgb}{0.9,0.9,0.9}

% Configuración para el estilo de C similar a VSCode
\lstdefinestyle{vscode_C}{
  backgroundcolor=\color{vscodebackground},
  commentstyle=\color{vscodegreen},
  keywordstyle=\color{vscodeblue},
  numberstyle=\tiny\color{vscodegray},
  stringstyle=\color{vscodepurple},
  basicstyle=\scriptsize\ttfamily,
  breakatwhitespace=false,
  breaklines=true,
  captionpos=b,
  keepspaces=true,
  numbers=left,
  numbersep=5pt,
  showspaces=false,
  showstringspaces=false,
  showtabs=false,
  tabsize=2,
  frame=tb,
  framerule=0pt,
  aboveskip=10pt,
  belowskip=10pt,
  xleftmargin=10pt,
  xrightmargin=10pt,
  framexleftmargin=10pt,
  framexrightmargin=10pt,
  framesep=0pt,
  rulecolor=\color{vscodelightgray},
  backgroundcolor=\color{vscodebackground},
}

%------------------------------------------------------------------------

% Comandos definidos
\newcommand{\bb}[1]{\mathbb{#1}}
\newcommand{\cc}[1]{\mathcal{#1}}

% I prefer the slanted \leq
\let\oldleq\leq % save them in case they're every wanted
\let\oldgeq\geq
\renewcommand{\leq}{\leqslant}
\renewcommand{\geq}{\geqslant}

% Si y solo si
\newcommand{\sii}{\iff}

% Letras griegas
\newcommand{\eps}{\epsilon}
\newcommand{\veps}{\varepsilon}
\newcommand{\lm}{\lambda}

\newcommand{\ol}{\overline}
\newcommand{\ul}{\underline}
\newcommand{\wt}{\widetilde}
\newcommand{\wh}{\widehat}

\let\oldvec\vec
\renewcommand{\vec}{\overrightarrow}

% Derivadas parciales
\newcommand{\del}[2]{\frac{\partial #1}{\partial #2}}
\newcommand{\Del}[3]{\frac{\partial^{#1} #2}{\partial #3^{#1}}}
\newcommand{\deld}[2]{\dfrac{\partial #1}{\partial #2}}
\newcommand{\Deld}[3]{\dfrac{\partial^{#1} #2}{\partial #3^{#1}}}


\newcommand{\AstIg}{\stackrel{(\ast)}{=}}
\newcommand{\Hop}{\stackrel{L'H\hat{o}pital}{=}}

\newcommand{\red}[1]{{\color{red}#1}} % Para integrales, destacar los cambios.

% Método de integración
\newcommand{\MetInt}[2]{
    \left[\begin{array}{c}
        #1 \\ #2
    \end{array}\right]
}

% Declarar aplicaciones
% 1. Nombre aplicación
% 2. Dominio
% 3. Codominio
% 4. Variable
% 5. Imagen de la variable
\newcommand{\Func}[5]{
    \begin{equation*}
        \begin{array}{rrll}
            #1:& #2 & \longrightarrow & #3\\
               & #4 & \longmapsto & #5
        \end{array}
    \end{equation*}
}

%------------------------------------------------------------------------

\usetikzlibrary{er, fit}

\newcommand{\key}[1]{\ul{#1}}

% Definición de estilos
\tikzset{atributo/.style={attribute}}
\tikzset{entidad/.style={entity}}
\tikzset{relacion/.style={relationship}}
\tikzset{atributo derivado/.style={attribute, dashed}}
\tikzset{union discriminante/.style={dashed}}
\tikzset{entidad debil/.style={entidad, double distance=1.5 pt}}
\tikzset{participacion obligatoria/.style={double distance=1.5 pt}}
\tikzset{herencia/.style={draw, isosceles triangle, isosceles triangle apex angle=60, shape border rotate=230, minimum height=1cm, minimum width=1cm, fill=blue!20}}
\tikzset{agregacion/.style={draw, fit=#1, inner sep=0.5em}}


% Estilos estéticos
\tikzset{every entity/.style={draw=orange, fill=orange!20}}
\tikzset{every attribute/.style={draw=purple, fill=purple!20, font=\footnotesize}}
\tikzset{every relationship/.style={draw=green, fill=green!20, minimum height=2cm, minimum width=2cm}}

\begin{comment}
    \begin{tikzpicture}[node distance=6 em]
        \node [entidad](person){Person};
        \node [atributo derivado](pid) at (0.8,-2){\key{ID}} edge (person);
        \node [atributo](name) at (-0.8, -2) {Name} edge (person);
        \node [atributo](phone)[left of=person]{Phone} edge[union discriminante] (person);
        \node [atributo](address)[above left of=person]{Address} edge[-stealth] (person);
        \node [atributo](street)[above left of=address]{Street} edge (address);
        \node [atributo](city)[left of=address]{City} edge (address);
        \node [atributo](age)[above of=person]{Age} edge (person);
        \node [relacion](uses)[right of=person]{Uses} edge[participacion obligatoria] (person);
        \node [entidad debil](tool)[right of=uses]{Tool} edge (uses);
        \node [atributo](tid)[right of=tool]{\key{ID}} edge[Stealth-Stealth] (tool);
        \node [atributo](tname)[below of=tool]{Name} edge (tool);
    
        % Línea desde encima de Name hasta encima de ID
        \draw ([yshift=1em]name.north) -- ([yshift=1em]pid.north) node[circle, fill, inner sep=1.5pt] {};
    
        \node[agregacion=(A) (B) (R)] (box) {};
    
        \node [herencia, below of= B](H){} edge (B);
    \end{tikzpicture}
\end{comment}

% ---------------------------------------------------
\tikzset{
    CP/.style={
        overlay, text opacity=0, fill opacity=0,
        append after command={
            \pgfextra{
                \draw[thick] (\tikzlastnode.south west) -- (\tikzlastnode.south east);
                \node[below=0ex of \tikzlastnode] {CP};
            }
        }
    },
    CC/.style={
        overlay, text opacity=0, fill opacity=0,
        append after command={
            \pgfextra{
                \draw[thick] (\tikzlastnode.south west) -- (\tikzlastnode.south east);
                \node[below=0ex of \tikzlastnode] {CC};
            }
        }
    },
    CE/.style={
        overlay, text opacity=0, fill opacity=0,
        append after command={
            \pgfextra{
                \draw[thick, purple] (\tikzlastnode.north west) -- (\tikzlastnode.north east);
                \node[above=0ex of \tikzlastnode, text=purple] {CE};
            }
        }
    }
}


\begin{document}

    % 1. Foto de fondo
    % 2. Título
    % 3. Encabezado Izquierdo
    % 4. Color de fondo
    % 5. Coord x del titulo
    % 6. Coord y del titulo
    % 7. Fecha

    
    % 1. Foto de fondo
% 2. Título
% 3. Encabezado Izquierdo
% 4. Color de fondo
% 5. Coord x del titulo
% 6. Coord y del titulo
% 7. Fecha

\newcommand{\portada}[7]{

    \portadaBase{#1}{#2}{#3}{#4}{#5}{#6}{#7}
    \portadaBook{#1}{#2}{#3}{#4}{#5}{#6}{#7}
}

\newcommand{\portadaExamen}[7]{

    \portadaBase{#1}{#2}{#3}{#4}{#5}{#6}{#7}
    \portadaArticle{#1}{#2}{#3}{#4}{#5}{#6}{#7}
}




\newcommand{\portadaBase}[7]{

    % Tiene la portada principal y la licencia Creative Commons
    
    % 1. Foto de fondo
    % 2. Título
    % 3. Encabezado Izquierdo
    % 4. Color de fondo
    % 5. Coord x del titulo
    % 6. Coord y del titulo
    % 7. Fecha
    
    
    \thispagestyle{empty}               % Sin encabezado ni pie de página
    \newgeometry{margin=0cm}        % Márgenes nulos para la primera página
    
    
    % Encabezado
    \fancyhead[L]{\helv #3}
    \fancyhead[R]{\helv \nouppercase{\leftmark}}
    
    
    \pagecolor{#4}        % Color de fondo para la portada
    
    \begin{figure}[p]
        \centering
        \transparent{0.3}           % Opacidad del 30% para la imagen
        
        \includegraphics[width=\paperwidth, keepaspectratio]{assets/#1}
    
        \begin{tikzpicture}[remember picture, overlay]
            \node[anchor=north west, text=white, opacity=1, font=\fontsize{60}{90}\selectfont\bfseries\sffamily, align=left] at (#5, #6) {#2};
            
            \node[anchor=south east, text=white, opacity=1, font=\fontsize{12}{18}\selectfont\sffamily, align=right] at (9.7, 3) {\textbf{\href{https://losdeldgiim.github.io/}{Los Del DGIIM}}};
            
            \node[anchor=south east, text=white, opacity=1, font=\fontsize{12}{15}\selectfont\sffamily, align=right] at (9.7, 1.8) {Doble Grado en Ingeniería Informática y Matemáticas\\Universidad de Granada};
        \end{tikzpicture}
    \end{figure}
    
    
    \restoregeometry        % Restaurar márgenes normales para las páginas subsiguientes
    \pagecolor{white}       % Restaurar el color de página
    
    
    \newpage
    \thispagestyle{empty}               % Sin encabezado ni pie de página
    \begin{tikzpicture}[remember picture, overlay]
        \node[anchor=south west, inner sep=3cm] at (current page.south west) {
            \begin{minipage}{0.5\paperwidth}
                \href{https://creativecommons.org/licenses/by-nc-nd/4.0/}{
                    \includegraphics[height=2cm]{assets/Licencia.png}
                }\vspace{1cm}\\
                Esta obra está bajo una
                \href{https://creativecommons.org/licenses/by-nc-nd/4.0/}{
                    Licencia Creative Commons Atribución-NoComercial-SinDerivadas 4.0 Internacional (CC BY-NC-ND 4.0).
                }\\
    
                Eres libre de compartir y redistribuir el contenido de esta obra en cualquier medio o formato, siempre y cuando des el crédito adecuado a los autores originales y no persigas fines comerciales. 
            \end{minipage}
        };
    \end{tikzpicture}
    
    
    
    % 1. Foto de fondo
    % 2. Título
    % 3. Encabezado Izquierdo
    % 4. Color de fondo
    % 5. Coord x del titulo
    % 6. Coord y del titulo
    % 7. Fecha


}


\newcommand{\portadaBook}[7]{

    % 1. Foto de fondo
    % 2. Título
    % 3. Encabezado Izquierdo
    % 4. Color de fondo
    % 5. Coord x del titulo
    % 6. Coord y del titulo
    % 7. Fecha

    % Personaliza el formato del título
    \pretitle{\begin{center}\bfseries\fontsize{42}{56}\selectfont}
    \posttitle{\par\end{center}\vspace{2em}}
    
    % Personaliza el formato del autor
    \preauthor{\begin{center}\Large}
    \postauthor{\par\end{center}\vfill}
    
    % Personaliza el formato de la fecha
    \predate{\begin{center}\huge}
    \postdate{\par\end{center}\vspace{2em}}
    
    \title{#2}
    \author{\href{https://losdeldgiim.github.io/}{Los Del DGIIM}}
    \date{Granada, #7}
    \maketitle
    
    \tableofcontents
}




\newcommand{\portadaArticle}[7]{

    % 1. Foto de fondo
    % 2. Título
    % 3. Encabezado Izquierdo
    % 4. Color de fondo
    % 5. Coord x del titulo
    % 6. Coord y del titulo
    % 7. Fecha

    % Personaliza el formato del título
    \pretitle{\begin{center}\bfseries\fontsize{42}{56}\selectfont}
    \posttitle{\par\end{center}\vspace{2em}}
    
    % Personaliza el formato del autor
    \preauthor{\begin{center}\Large}
    \postauthor{\par\end{center}\vspace{3em}}
    
    % Personaliza el formato de la fecha
    \predate{\begin{center}\huge}
    \postdate{\par\end{center}\vspace{5em}}
    
    \title{#2}
    \author{\href{https://losdeldgiim.github.io/}{Los Del DGIIM}}
    \date{Granada, #7}
    \thispagestyle{empty}               % Sin encabezado ni pie de página
    \maketitle
    \vfill
}
    \portadaExamen{etsiitA4.jpg}{FBD\\Examen VI}{FBD. Examen VI}{MidnightBlue}{-8}{28}{2024-2025}{Arturo Olivares Martos}

    \begin{description}
        \item[Asignatura] Fundamentos de Bases de Datos.
        \item[Curso Académico] 2017-18.
        %\item[Grado] Doble Grado en Ingeniería Informática y Matemáticas.
        %\item[Grupo] Único.
        %\item[Profesor] Nicolás Marín Ruiz.
        \item[Descripción] Convocatoria Extraordinaria. Práctico Parcial 2 (Seminarios 3-4).
        %\item[Fecha] 8 de noviembre de 2021.
        % \item[Duración] 60 minutos.
    
    \end{description}
    \newpage



Disponemos de la BD de la Figura~\ref{fig:diagrama} sobre gestión de varias ligas de agentes inteligentes (bots). Hay varios grupos separados. Cada participante está en un grupo. Cada participante juega con un único bot. Hay varias ligas, identificadas por la fecha de comienzo de la liga y el grupo que juega la liga. En cada partido de liga de un grupo juegan dos bots diferentes (\verb|jLocal| y \verb|jVisit|) y se guarda si ha ganado el bot local (\verb|gLocal|), el visitante (\verb|gVisit|) o si hay empate (\verb|empate|).
\begin{figure}
    \centering
    \begin{tikzpicture}[node distance=15em]
        \node (Grupo) {Grupo(ID, descripción)};
        \node (Participante) [right of=Grupo, xshift=-1em] {Participante(email, \textbf{grupo}, nombre)};
        \node (Bot) [right of=Participante, xshift=1em] {Bot(email, \textbf{nombreBot}, cpp, h)};

        \node (Partido) [below of=Grupo, yshift=9em, xshift=10em] {Partido(fecha, grupo, jLocal, jVisit, \textbf{gLocal}, \textbf{gVisit}, \textbf{empate})};
        \node (Liga) [right of=Partido, xshift=5em] {Liga(fecha, grupo)};

        \node[CP, xshift=-3ex] (CPGrupo) at (Grupo) {ID};
        \node[CP, xshift=-2ex] (CPParticipante) at (Participante) {email};
        \node[CP, xshift=-9ex] (CPBot) at (Bot) {email};
        \node[CC, xshift=11ex] (CCBot) at (CPBot) {nombreBot};
        \node[CP, xshift=-9ex] (CPPartido) at (Partido) {fecha, grupo, jLocal, jVisit};
        \node[CP, xshift=2.5ex] (CPLiga) at (Liga) {fecha, grupo};

        \node[CE, xshift=7ex] (CEParticipante) at (CPParticipante) {grupo};
        \node[CE, xshift=0ex] (CEBot) at (CPBot) {email};
        \node[CE, xshift=-7.5ex] (CEPartidoLiga) at (CPPartido) {fecha, grupo};
        \node[CE, xshift=4ex] (CEPartidoLocal) at (CPPartido) {jLocal};
        \node[CE, xshift=12ex] (CEPartidoVisit) at (CPPartido) {jVisit};
        \node[CE, xshift=3ex] (CELiga) at (CPLiga) {grupo};

        \node[yshift=1.5em, xshift=1.5em, purple] at (CEParticipante) {(1)};
        \node[yshift=1.5em, xshift=1.5em, purple] at (CEBot) {(2)};
        \node[yshift=1.5em, xshift=1.5em, purple] at (CEPartidoLiga) {(3)};
        \node[yshift=1.5em, xshift=1.5em, purple] at (CEPartidoLocal) {(4)};
        \node[yshift=1.5em, xshift=1.5em, purple] at (CEPartidoVisit) {(5)};
        \node[yshift=1.5em, xshift=1.5em, purple] at (CELiga) {(6)};

        \node[yshift=-1.5em, xshift=1.9em, purple] at (CPGrupo) {(1,6)};
        \node[yshift=-1.5em, xshift=1.5em, purple] at (CPParticipante) {(2)};
        \node[yshift=-1.5em, xshift=1.5em, purple] at (CPLiga) {(3)};
        \node[yshift=-1.5em, xshift=1.9em, purple] at (CPBot) {(4,5)};
        
    \end{tikzpicture}
    \caption{Diagrama Entidad-Relación}
    \label{fig:diagrama}
\end{figure}



\begin{ejercicio}
    Escribe las instrucciones en SQL para la creación de las tablas \verb|Partido| y \verb|Bot|. Se presuponen creadas todas las restantes tablas. Además de las restricciones de integridad especificadas en el dibujo, deben considerarse las siguientes:
    \begin{enumerate}
        \item Los campos en negrita no pueden ser nulos.
        \item \verb|gLocal|, \verb|gVisit| y \verb|empate| deben tener valores sólo 0 o 1 (\verb|gLocal=1| gana el local, \verb|gVisit=1| gana el visitante, \verb|empate=1| hay empate).
        \item Solo es posible que gane un único jugador, o que haya empate.
        \item Un bot no puede jugar consigo mismo un partido.
        \item \verb|cpp| y \verb|h| deben ser cadenas largas, pues contendrán el código fuente del bot.
    \end{enumerate}

    \begin{minted}{sql}
        CREATE TABLE Bot (
            email CONSTRAINT Bot_PK PRIMARY KEY CONSTRAINT Bot_FK_Participante REFERENCES Participante(email),
            nombreBot VARCHAR2(50) CONSTRAINT Bot_NombreBot_NN NOT NULL CONSTRAINT Bot_NombreBot_UN UNIQUE,
            cpp VARCHAR2(4000),
            h VARCHAR2(4000)
        );
        CREATE TABLE Partido (
            fecha,
            grupo,
            jLocal CONSTRAINT Partido_FK_Bot_Local REFERENCES Bot(email),
            jVisit CONSTRAINT Partido_FK_Bot_Visit REFERENCES Bot(email),
            gLocal INT CONSTRAINT Partido_GLocal_CK CHECK (gLocal IN (0, 1)),
            gVisit INT CONSTRAINT Partido_GVisit_CK CHECK (gVisit IN (0, 1)),
            empate INT CONSTRAINT Partido_Empate_CK CHECK (empate IN (0, 1)),
            CONSTRAINT Partido_PK PRIMARY KEY (fecha, grupo, jLocal, jVisit),
            CONSTRAINT Partido_FK_Liga FOREIGN KEY (fecha, grupo) REFERENCES Liga(fecha, grupo),
            CONSTRAINT Partido_CK CHECK (gLocal + gVisit + empate = 1),
            CONSTRAINT Partido_CK_Bot_Distinto CHECK (jLocal != jVisit)
        );
    \end{minted}
\end{ejercicio}

\begin{ejercicio}[AR y SQL]
    Mostrar los nombres de los participantes del grupo \verb|A1| cuyos bots han jugado algún partido como local en todas las ligas de su grupo.\\

    Mostraremos en primer lugar el resultado en Álgebra Relacional. Las ligas del grupo \verb|A1| son:
    \[
        \sigma_{\text{grupo}=\text{A1}}(\text{Liga})
    \]

    Buscamos ahora todas las parejas participante del grupo \verb|A1|, liga; en las que dicho participante ha jugado como local en algún partido de la liga:
    \[
        \pi_{\substack{\text{email}\\ \text{fecha}\\ \text{grupo}}}\left(\sigma_{\text{grupo}=\text{A1}}(\text{Participante})\bowtie_{\text{jLocal}=\text{email}}\text{Partido}\right)
    \]

    Por tanto, el email de los participantes del grupo \verb|A1| cuyos bots han jugado algún partido como local en todas las ligas de su grupo es:
    \[
        \pi_{\substack{\text{email}\\ \text{fecha}\\ \text{grupo}}}\left(\sigma_{\text{grupo}=\text{A1}}(\text{Participante})\bowtie_{\text{jLocal}=\text{email}}\text{Partido}\right)
        \div \sigma_{\text{grupo}=\text{A1}}(\text{Liga})
    \]

    Por tanto, los nombres de los participantes del grupo \verb|A1| cuyos bots han jugado algún partido como local en todas las ligas de su grupo son:
    \[
        \pi_{\text{nombre}}\left[\left[\pi_{\substack{\text{email}\\ \text{fecha}\\ \text{grupo}}}\left(\sigma_{\text{grupo}=\text{A1}}(\text{Participante})\bowtie_{\text{jLocal}=\text{email}}\text{Partido}\right)
        \div \sigma_{\text{grupo}=\text{A1}}(\text{Liga})\right]\bowtie \text{Participante}\right]
    \]~\\

    En SQL, la consulta sería:
    \begin{minted}{sql}
        SELECT nombre
        FROM Participante
        WHERE grupo = 'A1'
            AND NOT EXISTS (
                SELECT *
                    FROM Liga
                    WHERE grupo = 'A1'
                MINUS
                    SELECT fecha, grupo
                    FROM Partido
                    WHERE jLocal = email
            );
    \end{minted}
\end{ejercicio}

\begin{ejercicio}[SQL]
    Mostrar nombre y puntuación de aquellos bots que, jugando como local en la liga del 3 de Junio de 2018, obtuvieron un total de puntos superior a 4. La puntuación jugando como local se calcula como $3\times$Partidos ganados como local $+$ partidos empatados.
    \begin{minted}{sql}
        SELECT nombreBot, SUM(3*gLocal + empate) AS puntuacion
            FROM bot JOIN Partido ON email = jLocal
            WHERE TO_CHAR(fecha, 'DD-MM-YYYY') = TO_DATE('03-06-2018', 'DD-MM-YYYY')
            GROUP BY nombreBot
            HAVING SUM(3*gLocal + empate) > 4;
    \end{minted}
\end{ejercicio}

\begin{ejercicio}[AR y SQL]
    Mostrar los emails de los participantes que hayan ganado al bot de nombre \verb|GreedyBot| como local y como visitante.\\

    En Álgebra Relacional, los emails de los participantes que hayan ganado al bot de nombre \verb|GreedyBot| como local y como visitante son:
    \begin{multline*}
        \pi_{\text{email}}\left(\sigma_{\text{nombreBot}=\text{GreedyBot}}(\text{Bot})\bowtie_{\text{email}=\text{jLocal}}\sigma_{\text{gLocal}=1}(\text{Partido})\right)
        \qquad \cap\\
        \cap\qquad \pi_{\text{email}}\left(\sigma_{\text{nombreBot}=\text{GreedyBot}}(\text{Bot})\bowtie_{\text{email}=\text{jVisit}}\sigma_{\text{gVisit}=1}(\text{Partido})\right)
    \end{multline*}

    En SQL, la consulta sería:
    \begin{minted}{sql}
        SELECT email
            FROM Bot JOIN Partido ON email = jLocal
            WHERE nombreBot = 'GreedyBot' AND gLocal = 1
        INTERSECT
        SELECT email
            FROM Bot JOIN Partido ON email = jVisit
            WHERE nombreBot = 'GreedyBot' AND gVisit = 1;
    \end{minted}
\end{ejercicio}


\end{document}