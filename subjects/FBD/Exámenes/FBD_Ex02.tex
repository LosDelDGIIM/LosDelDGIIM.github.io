\documentclass[12pt]{article}

% Idioma y codificación
\usepackage[spanish, es-tabla]{babel}       %es-tabla para que se titule "Tabla"
\usepackage[utf8]{inputenc}

% Márgenes
\usepackage[a4paper,top=3cm,bottom=2.5cm,left=3cm,right=3cm]{geometry}

% Comentarios de bloque
\usepackage{verbatim}

% Paquetes de links
\usepackage[hidelinks]{hyperref}    % Permite enlaces
\usepackage{url}                    % redirecciona a la web

% Más opciones para enumeraciones
\usepackage{enumitem}

% Personalizar la portada
\usepackage{titling}

% Paquetes de tablas
\usepackage{multirow}


%------------------------------------------------------------------------

%Paquetes de figuras
\usepackage{caption}
\usepackage{subcaption} % Figuras al lado de otras
\usepackage{float}      % Poner figuras en el sitio indicado H.


% Paquetes de imágenes
\usepackage{graphicx}       % Paquete para añadir imágenes
\usepackage{transparent}    % Para manejar la opacidad de las figuras

% Paquete para usar colores
\usepackage[dvipsnames]{xcolor}
\usepackage{pagecolor}      % Para cambiar el color de la página

% Habilita tamaños de fuente mayores
\usepackage{fix-cm}

% Para los gráficos
\usepackage{tikz}

% Para poder situar los nodos en los grafos
\usetikzlibrary{positioning}


%------------------------------------------------------------------------

% Paquetes de matemáticas
\usepackage{mathtools, amsfonts, amssymb, mathrsfs}
\usepackage[makeroom]{cancel}     % Simplificar tachando
\usepackage{polynom}    % Divisiones y Ruffini
\usepackage{units} % Para poner fracciones diagonales con \nicefrac

\usepackage{pgfplots}   %Representar funciones
\pgfplotsset{compat=1.18}  % Versión 1.18

\usepackage{tikz-cd}    % Para usar diagramas de composiciones
\usetikzlibrary{calc}   % Para usar cálculo de coordenadas en tikz

%Definición de teoremas, etc.
\usepackage{amsthm}
%\swapnumbers   % Intercambia la posición del texto y de la numeración

\theoremstyle{plain}

\makeatletter
\@ifclassloaded{article}{
  \newtheorem{teo}{Teorema}[section]
}{
  \newtheorem{teo}{Teorema}[chapter]  % Se resetea en cada chapter
}
\makeatother

\newtheorem{coro}{Corolario}[teo]           % Se resetea en cada teorema
\newtheorem{prop}[teo]{Proposición}         % Usa el mismo contador que teorema
\newtheorem{lema}[teo]{Lema}                % Usa el mismo contador que teorema

\theoremstyle{remark}
\newtheorem*{observacion}{Observación}

\theoremstyle{definition}

\makeatletter
\@ifclassloaded{article}{
  \newtheorem{definicion}{Definición} [section]     % Se resetea en cada chapter
}{
  \newtheorem{definicion}{Definición} [chapter]     % Se resetea en cada chapter
}
\makeatother

\newtheorem*{notacion}{Notación}
\newtheorem*{ejemplo}{Ejemplo}
\newtheorem*{ejercicio*}{Ejercicio}             % No numerado
\newtheorem{ejercicio}{Ejercicio} [section]     % Se resetea en cada section


% Modificar el formato de la numeración del teorema "ejercicio"
\renewcommand{\theejercicio}{%
  \ifnum\value{section}=0 % Si no se ha iniciado ninguna sección
    \arabic{ejercicio}% Solo mostrar el número de ejercicio
  \else
    \thesection.\arabic{ejercicio}% Mostrar número de sección y número de ejercicio
  \fi
}


% \renewcommand\qedsymbol{$\blacksquare$}         % Cambiar símbolo QED
%------------------------------------------------------------------------

% Paquetes para encabezados
\usepackage{fancyhdr}
\pagestyle{fancy}
\fancyhf{}

\newcommand{\helv}{ % Modificación tamaño de letra
\fontfamily{}\fontsize{12}{12}\selectfont}
\setlength{\headheight}{15pt} % Amplía el tamaño del índice


%\usepackage{lastpage}   % Referenciar última pag   \pageref{LastPage}
\fancyfoot[C]{\thepage}

%------------------------------------------------------------------------

% Conseguir que no ponga "Capítulo 1". Sino solo "1."
\makeatletter
\@ifclassloaded{book}{
  \renewcommand{\chaptermark}[1]{\markboth{\thechapter.\ #1}{}} % En el encabezado
    
  \renewcommand{\@makechapterhead}[1]{%
  \vspace*{50\p@}%
  {\parindent \z@ \raggedright \normalfont
    \ifnum \c@secnumdepth >\m@ne
      \huge\bfseries \thechapter.\hspace{1em}\ignorespaces
    \fi
    \interlinepenalty\@M
    \Huge \bfseries #1\par\nobreak
    \vskip 40\p@
  }}
}
\makeatother

%------------------------------------------------------------------------
% Paquetes de cógido
\usepackage{minted}
\renewcommand\listingscaption{Código fuente}

\usepackage{fancyvrb}
% Personaliza el tamaño de los números de línea
\renewcommand{\theFancyVerbLine}{\small\arabic{FancyVerbLine}}

% Estilo para C++
\newminted{cpp}{
    frame=lines,
    framesep=2mm,
    baselinestretch=1.2,
    linenos,
    escapeinside=||
}

% para minted
\definecolor{LightGray}{rgb}{0.95,0.95,0.92}
\setminted{
    linenos=true,
    stepnumber=5,
    numberfirstline=true,
    autogobble,
    breaklines=true,
    breakautoindent=true,
    breaksymbolleft=,
    breaksymbolright=,
    breaksymbolindentleft=0pt,
    breaksymbolindentright=0pt,
    breaksymbolsepleft=0pt,
    breaksymbolsepright=0pt,
    fontsize=\footnotesize,
    bgcolor=LightGray,
    numbersep=10pt
}


\usepackage{listings} % Para incluir código desde un archivo

\renewcommand\lstlistingname{Código Fuente}
\renewcommand\lstlistlistingname{Índice de Códigos Fuente}

% Definir colores
\definecolor{vscodepurple}{rgb}{0.5,0,0.5}
\definecolor{vscodeblue}{rgb}{0,0,0.8}
\definecolor{vscodegreen}{rgb}{0,0.5,0}
\definecolor{vscodegray}{rgb}{0.5,0.5,0.5}
\definecolor{vscodebackground}{rgb}{0.97,0.97,0.97}
\definecolor{vscodelightgray}{rgb}{0.9,0.9,0.9}

% Configuración para el estilo de C similar a VSCode
\lstdefinestyle{vscode_C}{
  backgroundcolor=\color{vscodebackground},
  commentstyle=\color{vscodegreen},
  keywordstyle=\color{vscodeblue},
  numberstyle=\tiny\color{vscodegray},
  stringstyle=\color{vscodepurple},
  basicstyle=\scriptsize\ttfamily,
  breakatwhitespace=false,
  breaklines=true,
  captionpos=b,
  keepspaces=true,
  numbers=left,
  numbersep=5pt,
  showspaces=false,
  showstringspaces=false,
  showtabs=false,
  tabsize=2,
  frame=tb,
  framerule=0pt,
  aboveskip=10pt,
  belowskip=10pt,
  xleftmargin=10pt,
  xrightmargin=10pt,
  framexleftmargin=10pt,
  framexrightmargin=10pt,
  framesep=0pt,
  rulecolor=\color{vscodelightgray},
  backgroundcolor=\color{vscodebackground},
}

%------------------------------------------------------------------------

% Comandos definidos
\newcommand{\bb}[1]{\mathbb{#1}}
\newcommand{\cc}[1]{\mathcal{#1}}

% I prefer the slanted \leq
\let\oldleq\leq % save them in case they're every wanted
\let\oldgeq\geq
\renewcommand{\leq}{\leqslant}
\renewcommand{\geq}{\geqslant}

% Si y solo si
\newcommand{\sii}{\iff}

% Letras griegas
\newcommand{\eps}{\epsilon}
\newcommand{\veps}{\varepsilon}
\newcommand{\lm}{\lambda}

\newcommand{\ol}{\overline}
\newcommand{\ul}{\underline}
\newcommand{\wt}{\widetilde}
\newcommand{\wh}{\widehat}

\let\oldvec\vec
\renewcommand{\vec}{\overrightarrow}

% Derivadas parciales
\newcommand{\del}[2]{\frac{\partial #1}{\partial #2}}
\newcommand{\Del}[3]{\frac{\partial^{#1} #2}{\partial #3^{#1}}}
\newcommand{\deld}[2]{\dfrac{\partial #1}{\partial #2}}
\newcommand{\Deld}[3]{\dfrac{\partial^{#1} #2}{\partial #3^{#1}}}


\newcommand{\AstIg}{\stackrel{(\ast)}{=}}
\newcommand{\Hop}{\stackrel{L'H\hat{o}pital}{=}}

\newcommand{\red}[1]{{\color{red}#1}} % Para integrales, destacar los cambios.

% Método de integración
\newcommand{\MetInt}[2]{
    \left[\begin{array}{c}
        #1 \\ #2
    \end{array}\right]
}

% Declarar aplicaciones
% 1. Nombre aplicación
% 2. Dominio
% 3. Codominio
% 4. Variable
% 5. Imagen de la variable
\newcommand{\Func}[5]{
    \begin{equation*}
        \begin{array}{rrll}
            #1:& #2 & \longrightarrow & #3\\
               & #4 & \longmapsto & #5
        \end{array}
    \end{equation*}
}

%------------------------------------------------------------------------

\usetikzlibrary{er, fit}

\newcommand{\key}[1]{\ul{#1}}

% Definición de estilos
\tikzset{atributo/.style={attribute}}
\tikzset{entidad/.style={entity}}
\tikzset{relacion/.style={relationship}}
\tikzset{atributo derivado/.style={attribute, dashed}}
\tikzset{union discriminante/.style={dashed}}
\tikzset{entidad debil/.style={entidad, double distance=1.5 pt}}
\tikzset{participacion obligatoria/.style={double distance=1.5 pt}}
\tikzset{herencia/.style={draw, isosceles triangle, isosceles triangle apex angle=60, shape border rotate=230, minimum height=1cm, minimum width=1cm, fill=blue!20}}
\tikzset{agregacion/.style={draw, fit=#1, inner sep=0.5em}}


% Estilos estéticos
\tikzset{every entity/.style={draw=orange, fill=orange!20}}
\tikzset{every attribute/.style={draw=purple, fill=purple!20, font=\footnotesize}}
\tikzset{every relationship/.style={draw=green, fill=green!20, minimum height=2cm, minimum width=2cm}}

\begin{comment}
    \begin{tikzpicture}[node distance=6 em]
        \node [entidad](person){Person};
        \node [atributo derivado](pid) at (0.8,-2){\key{ID}} edge (person);
        \node [atributo](name) at (-0.8, -2) {Name} edge (person);
        \node [atributo](phone)[left of=person]{Phone} edge[union discriminante] (person);
        \node [atributo](address)[above left of=person]{Address} edge[-stealth] (person);
        \node [atributo](street)[above left of=address]{Street} edge (address);
        \node [atributo](city)[left of=address]{City} edge (address);
        \node [atributo](age)[above of=person]{Age} edge (person);
        \node [relacion](uses)[right of=person]{Uses} edge[participacion obligatoria] (person);
        \node [entidad debil](tool)[right of=uses]{Tool} edge (uses);
        \node [atributo](tid)[right of=tool]{\key{ID}} edge[Stealth-Stealth] (tool);
        \node [atributo](tname)[below of=tool]{Name} edge (tool);
    
        % Línea desde encima de Name hasta encima de ID
        \draw ([yshift=1em]name.north) -- ([yshift=1em]pid.north) node[circle, fill, inner sep=1.5pt] {};
    
        \node[agregacion=(A) (B) (R)] (box) {};
    
        \node [herencia, below of= B](H){} edge (B);
    \end{tikzpicture}
\end{comment}

% ---------------------------------------------------
\tikzset{
    CP/.style={
        overlay, text opacity=0, fill opacity=0,
        append after command={
            \pgfextra{
                \draw[thick] (\tikzlastnode.south west) -- (\tikzlastnode.south east);
                \node[below=0ex of \tikzlastnode] {CP};
            }
        }
    },
    CC/.style={
        overlay, text opacity=0, fill opacity=0,
        append after command={
            \pgfextra{
                \draw[thick] (\tikzlastnode.south west) -- (\tikzlastnode.south east);
                \node[below=0ex of \tikzlastnode] {CC};
            }
        }
    },
    CE/.style={
        overlay, text opacity=0, fill opacity=0,
        append after command={
            \pgfextra{
                \draw[thick, purple] (\tikzlastnode.north west) -- (\tikzlastnode.north east);
                \node[above=0ex of \tikzlastnode, text=purple] {CE};
            }
        }
    }
}

\begin{comment}
\begin{tikzpicture}
    \node (alumno) {Alumno(DNI, kjbbhjbj, Código, jhjghjh)};
    \node (asignatura) [below of=alumno, yshift=-5em] {Asignatura(hjbjhblj, jjgjhb, kgjhh, Código)};

    \node[CP, xshift=17.3ex] (codigoCP) at(asignatura) {Código};
    \node[CP, xshift=-9ex] (dniCP) at(alumno) {DNI};


    \node[CE, xshift=17.3ex] (codigoCE) at(asignatura) {Código};
    \draw[-Stealth, purple] (codigoCE.north west) -- (dniCP.south east);
    
\end{tikzpicture}
\end{comment}


\begin{document}

    % 1. Foto de fondo
    % 2. Título
    % 3. Encabezado Izquierdo
    % 4. Color de fondo
    % 5. Coord x del titulo
    % 6. Coord y del titulo
    % 7. Fecha

    
    % 1. Foto de fondo
% 2. Título
% 3. Encabezado Izquierdo
% 4. Color de fondo
% 5. Coord x del titulo
% 6. Coord y del titulo
% 7. Fecha

\newcommand{\portada}[7]{

    \portadaBase{#1}{#2}{#3}{#4}{#5}{#6}{#7}
    \portadaBook{#1}{#2}{#3}{#4}{#5}{#6}{#7}
}

\newcommand{\portadaExamen}[7]{

    \portadaBase{#1}{#2}{#3}{#4}{#5}{#6}{#7}
    \portadaArticle{#1}{#2}{#3}{#4}{#5}{#6}{#7}
}




\newcommand{\portadaBase}[7]{

    % Tiene la portada principal y la licencia Creative Commons
    
    % 1. Foto de fondo
    % 2. Título
    % 3. Encabezado Izquierdo
    % 4. Color de fondo
    % 5. Coord x del titulo
    % 6. Coord y del titulo
    % 7. Fecha
    
    
    \thispagestyle{empty}               % Sin encabezado ni pie de página
    \newgeometry{margin=0cm}        % Márgenes nulos para la primera página
    
    
    % Encabezado
    \fancyhead[L]{\helv #3}
    \fancyhead[R]{\helv \nouppercase{\leftmark}}
    
    
    \pagecolor{#4}        % Color de fondo para la portada
    
    \begin{figure}[p]
        \centering
        \transparent{0.3}           % Opacidad del 30% para la imagen
        
        \includegraphics[width=\paperwidth, keepaspectratio]{assets/#1}
    
        \begin{tikzpicture}[remember picture, overlay]
            \node[anchor=north west, text=white, opacity=1, font=\fontsize{60}{90}\selectfont\bfseries\sffamily, align=left] at (#5, #6) {#2};
            
            \node[anchor=south east, text=white, opacity=1, font=\fontsize{12}{18}\selectfont\sffamily, align=right] at (9.7, 3) {\textbf{\href{https://losdeldgiim.github.io/}{Los Del DGIIM}}};
            
            \node[anchor=south east, text=white, opacity=1, font=\fontsize{12}{15}\selectfont\sffamily, align=right] at (9.7, 1.8) {Doble Grado en Ingeniería Informática y Matemáticas\\Universidad de Granada};
        \end{tikzpicture}
    \end{figure}
    
    
    \restoregeometry        % Restaurar márgenes normales para las páginas subsiguientes
    \pagecolor{white}       % Restaurar el color de página
    
    
    \newpage
    \thispagestyle{empty}               % Sin encabezado ni pie de página
    \begin{tikzpicture}[remember picture, overlay]
        \node[anchor=south west, inner sep=3cm] at (current page.south west) {
            \begin{minipage}{0.5\paperwidth}
                \href{https://creativecommons.org/licenses/by-nc-nd/4.0/}{
                    \includegraphics[height=2cm]{assets/Licencia.png}
                }\vspace{1cm}\\
                Esta obra está bajo una
                \href{https://creativecommons.org/licenses/by-nc-nd/4.0/}{
                    Licencia Creative Commons Atribución-NoComercial-SinDerivadas 4.0 Internacional (CC BY-NC-ND 4.0).
                }\\
    
                Eres libre de compartir y redistribuir el contenido de esta obra en cualquier medio o formato, siempre y cuando des el crédito adecuado a los autores originales y no persigas fines comerciales. 
            \end{minipage}
        };
    \end{tikzpicture}
    
    
    
    % 1. Foto de fondo
    % 2. Título
    % 3. Encabezado Izquierdo
    % 4. Color de fondo
    % 5. Coord x del titulo
    % 6. Coord y del titulo
    % 7. Fecha


}


\newcommand{\portadaBook}[7]{

    % 1. Foto de fondo
    % 2. Título
    % 3. Encabezado Izquierdo
    % 4. Color de fondo
    % 5. Coord x del titulo
    % 6. Coord y del titulo
    % 7. Fecha

    % Personaliza el formato del título
    \pretitle{\begin{center}\bfseries\fontsize{42}{56}\selectfont}
    \posttitle{\par\end{center}\vspace{2em}}
    
    % Personaliza el formato del autor
    \preauthor{\begin{center}\Large}
    \postauthor{\par\end{center}\vfill}
    
    % Personaliza el formato de la fecha
    \predate{\begin{center}\huge}
    \postdate{\par\end{center}\vspace{2em}}
    
    \title{#2}
    \author{\href{https://losdeldgiim.github.io/}{Los Del DGIIM}}
    \date{Granada, #7}
    \maketitle
    
    \tableofcontents
}




\newcommand{\portadaArticle}[7]{

    % 1. Foto de fondo
    % 2. Título
    % 3. Encabezado Izquierdo
    % 4. Color de fondo
    % 5. Coord x del titulo
    % 6. Coord y del titulo
    % 7. Fecha

    % Personaliza el formato del título
    \pretitle{\begin{center}\bfseries\fontsize{42}{56}\selectfont}
    \posttitle{\par\end{center}\vspace{2em}}
    
    % Personaliza el formato del autor
    \preauthor{\begin{center}\Large}
    \postauthor{\par\end{center}\vspace{3em}}
    
    % Personaliza el formato de la fecha
    \predate{\begin{center}\huge}
    \postdate{\par\end{center}\vspace{5em}}
    
    \title{#2}
    \author{\href{https://losdeldgiim.github.io/}{Los Del DGIIM}}
    \date{Granada, #7}
    \thispagestyle{empty}               % Sin encabezado ni pie de página
    \maketitle
    \vfill
}
    \portadaExamen{ffccA4.jpg}{FBD\\Examen II}{FBD. Examen II}{MidnightBlue}{-8}{28}{2024-2025}{Arturo Olivares Martos}

    \begin{description}
        \item[Asignatura] Fundamentos de Bases de Datos.
        %\item[Curso Académico] 2017-18.
        %\item[Grado] Doble Grado en Ingeniería Informática y Matemáticas.
        %\item[Grupo] Único.
        %\item[Profesor] Nicolás Marín Ruiz.
        \item[Descripción] Práctico Parcial 1 (Seminarios 1-2).
        %\item[Fecha] 8 de noviembre de 2021.
        % \item[Duración] 60 minutos.
    
    \end{description}
    \newpage

Se quiere recoger información sobre la atención a los pacientes en un hospital. Hay que registrar la siguiente información:
\begin{itemize}
    \item Personal sanitario: Número de registro sanitario único, DNI, Nombre, Categoría, Título.
    \item Servicios o especialidades: Código de servicio, Nombre, Ubicación.
    \item Pacientes: NSS, DNI, Nombre, Dirección y Teléfono.
\end{itemize}

Deben cumplirse las siguientes restricciones:
\begin{itemize}
    \item Entre el personal sanitario hay que destacar a los enfermeros y a los médicos; de estos últimos hay que conocer su especialidad y año de finalización del MIR.
    \item Todo el personal sanitario está adscrito obligatoriamente a un servicio.
    \item Todos los servicios tienen un director que tiene que ser médico.
    \item El trabajo se realiza por turnos (mañana, tarde, noche) y, durante un turno, cada enfermero depende de un médico y un médico es responsable de un grupo de enfermeros.
    \item Los pacientes son atendidos por una pareja de enfermero y médico, y hay que registrar el día y la hora.
\end{itemize}

\begin{ejercicio}
    Realizar el diagrama E/R que mejor represente dicha información.

    El esquema Entidad/Relación se encuentra en la Figura~\ref{fig:esquemaER}.
    Notemos que:
    \begin{itemize}
        \item Se podría haber incluido la participación obligatoria de la entidad \emph{Servicio} en la relación \emph{Dirigir}.
        \item El atributo \emph{Fecha} de la relación \emph{Atiende} tiene precisión de hora.
    \end{itemize}
    \begin{figure}
        \centering
        \scalebox{0.9}{
        \begin{tikzpicture}[node distance=6.3 em]
            \node [entidad](personal){Personal sanitario};
            \node [atributo](NRS) [above left of=personal, yshift=1em, xshift=-1em]{\key{NRS}} edge (personal);
            \node [atributo](DNI) [above left of=personal, yshift=-1.1em, xshift=-3em]{\key{DNI}} edge (personal);
            \node [atributo](nombre) [left of=personal, xshift=-1em]{Nombre} edge (personal);
            \node [atributo](categoria) [above of=personal]{Categoría} edge (personal);
            \node [atributo](titulo) [below left of=personal]{Título} edge (personal);

            \node [relacion] (adscrito) [right of=personal, xshift=3.5em]{Adscrito} edge[participacion obligatoria] (personal);
            \node [entidad](servicio) [right of=adscrito, xshift=1.5em]{Servicio} edge[Stealth-] (adscrito);
            \node [atributo](codigo) [above of=servicio]{\key{Código}} edge (servicio);
            \node [atributo](nombreS) [above right of=servicio]{Nombre} edge (servicio);
            \node [atributo](ubicacion) [right of=servicio]{Ubicación} edge (servicio);

            \node [herencia] (tipo_personal) [below of=personal]{Es} edgenode[midway, right] {Disjunta} (personal);
            \node [entidad](enfermero) [below left of=tipo_personal]{Enfermero} edge (tipo_personal);
            \node [entidad](medico) [below right of=tipo_personal]{Médico} edge (tipo_personal);
            \node [atributo](especialidad) [below of=medico, xshift=3em]{Especialidad} edge (medico);
            \node [atributo](MIR) [below right of=medico, yshift=2em]{Año MIR} edge (medico);
            \node [relacion] (dirigir) [below of=servicio, yshift=-4.5em]{Dirigir} edge[-Stealth](servicio) edge[-Stealth] (medico);

            \node [relacion] (responsable) [below of=tipo_personal, yshift=-4em]{Responsable} edge (enfermero) edge[-Stealth] (medico);
            \node [atributo](turno) [left of=responsable]{\key{Turno}} edge[union discriminante] (responsable);

            \node [agregacion=(enfermero) (medico) (responsable) (turno) (especialidad) (MIR)] (agregacion) {};
            \node [relacion](atiende) [right of=agregacion, xshift=8em]{Atiende} edge[-Stealth] (agregacion);
            \node [atributo](fecha) [below of=atiende]{\key{Fecha}} edge[union discriminante] (atiende);
            \node [entidad](paciente) [right of=atiende]{Paciente} edge[Stealth-] (atiende);
            \node [atributo](NSS) [below of=paciente]{\key{NSS}} edge (paciente);
            \node [atributo](DNIp) [above of=paciente]{\key{DNI}} edge (paciente);
            \node [atributo](direccion) [below right of=paciente]{Dirección} edge (paciente);
            \node [atributo](telefono) [right of=paciente]{Teléfono} edge (paciente);
            \node [atributo](nombreP) [above right of=paciente]{Nombre} edge (paciente);
        \end{tikzpicture}
        }
        \caption{Esquema Entidad/Relación.}
        \label{fig:esquemaER}
    \end{figure}
\end{ejercicio}
\begin{ejercicio}
    Obtener el esquema relacional con el menor número de tablas posible, señalando claves candidatas, primarias y externas.

    El esquema relacional se encuentra en la Figura~\ref{fig:esquemaRelacional}.
    Las posibles fusiones se han realizado por debajo de la línea discontinua divisoria. 
    \begin{figure}
        \centering
        \resizebox{1.1\textwidth}{!}{
            \begin{tikzpicture}[node distance=15em]
                \node(personal) {Personal sanitario(DNI, NRS, Nombre, Categoría, Título)};
                \node(enfermero) [right of=personal, xshift=3em] {Enfermero(NRS)};
                \node(adscrito) [right of=enfermero] {Adscrito(NRS, Código)};

                \node(servicio) [below of=personal, yshift=9em, xshift=-4em] {Servicio(Código, Nombre, Ubicación)};
                \node(medico) [right of=servicio, xshift=2em] {Médico(NRS, Especialidad, MIR)};
                \node(paciente) [right of=medico, xshift=4em] {Paciente(NSS, DNI, Nombre, Dirección, Teléfono)};
                

                \node(responsable) [below of=servicio, yshift=9em, xshift=3em] {Responsable(NRS\_Enfermero, Turno, NRS\_Médico)};
                \node(dirigir) [right of=responsable, xshift=5em] {Dirigir(Código, NRS)};
                \node(atiende) [right of=dirigir, xshift=1em] {Atiende(NRS\_Enfermero, Turno, Fecha, NSS)};

                \node[CP, xshift=-1ex] (CP_Personal) at(personal) {NRS};
                \node[CC, xshift=-7ex] (CC_Personal) at(CP_Personal) {DNI};
                \node[CP, xshift=5ex] (CP_Enfermero) at(enfermero) {NRS};
                \node[CP, xshift=0ex] (CP_Adscrito) at(adscrito) {NRS};
                \node[CP, xshift=-6ex] (CP_Servicio) at(servicio) {Código};
                \node[CP, xshift=-6ex] (CP_Medico) at(medico) {NRS};
                \node[CP, xshift=-14ex] (CP_Paciente) at(paciente) {NSS};
                \node[CC, xshift=7ex] (CC_Paciente) at(CP_Paciente) {DNI};
                \node[CP, xshift=0ex] (CP_Responsable) at(responsable) {NRS\_Enfermero, Turno};
                \node[CP, xshift=0ex] (CP_Dirigir) at(dirigir) {Código};
                \node[CC, xshift=8ex] (CC_Dirigir) at(CP_Dirigir) {NRS};
                \node[CP, xshift=16ex] (CP_Atiende) at(atiende) {Fecha, NSS};
                \node[CC, xshift=-15ex,yshift=-1.4em] (CC_Atiende) at(CP_Atiende) {NRS\_Enfermero, Turno, Fecha};

                \node[CE] (CE_Enfermero) at(CP_Enfermero) {NRS};
                \node[CE] (CE_Adscrito_Personal) at(CP_Adscrito) {NRS};
                \node[CE, xshift=8ex] (CE_Adscrito_Servicio) at(CP_Adscrito) {Código};
                \node[CE] (CE_Medico) at(CP_Medico) {NRS};
                \node[CE, xshift=-4ex] (CE_Responsable_Enfermero) at(CP_Responsable) {NRS\_Enfermero};
                \node[CE, xshift=19ex] (CE_Responsable_Medico) at(CP_Responsable) {NRS\_Médico};
                \node[CE] (CE_Dirigir_Servicio) at(CP_Dirigir) {Código};
                \node[CE] (CE_Dirigir_Medico) at(CC_Dirigir) {NRS};
                \node[CE, xshift=-4ex, yshift=1.4em] (CE_Atiende_Enfermero) at(CC_Atiende) {NRS\_Enfermero, Turno};
                \node[CE, xshift=4ex] (CE_Atiende_Paciente) at(CP_Atiende) {NSS};

                \node[yshift=1.4em, xshift=1.4em, purple] at(CE_Enfermero) {$(1)$};
                \node[yshift=1.4em, xshift=1.4em, purple] at(CE_Adscrito_Personal) {$(2)$};
                \node[yshift=1.4em, xshift=1.6em, purple] at(CE_Adscrito_Servicio) {$(10)$};
                \node[yshift=1.4em, xshift=1.4em, purple] at(CE_Medico) {$(3)$};
                \node[yshift=1.4em, xshift=1.4em, purple] at(CE_Responsable_Enfermero) {$(4)$};
                \node[yshift=1.4em, xshift=1.4em, purple] at(CE_Responsable_Medico) {$(5)$};
                \node[yshift=1.4em, xshift=1.4em, purple] at(CE_Dirigir_Servicio) {$(6)$};
                \node[yshift=1.4em, xshift=1.4em, purple] at(CE_Dirigir_Medico) {$(7)$};
                \node[yshift=1.4em, xshift=1.4em, purple] at(CE_Atiende_Enfermero) {$(8)$};
                \node[yshift=1.4em, xshift=1.4em, purple] at(CE_Atiende_Paciente) {$(9)$};

                \node[yshift=-1.4em, xshift=2.6em, purple] at(CP_Personal) {$(1,2,3)$};
                \node[yshift=-1.4em, xshift=1.4em, purple] at(CP_Enfermero) {$(4)$};
                \node[yshift=-1.4em, xshift=1.8em, purple] at(CP_Medico) {$(5,7)$};
                \node[yshift=-1.5em, xshift=2em, purple] at(CP_Servicio) {$(6, 10)$};
                \node[yshift=-1.5em, xshift=1.4em, purple] at(CP_Responsable) {$(8)$};
                \node[yshift=-1.4em, xshift=1.4em, purple] at(CP_Paciente) {$(9)$};

                % Línea divisoria debajo de todo
                \draw[dashed] (current bounding box.south west) ++(-0.5em, -1em) -- ++(1.65\linewidth, 0);

                \node(Personal-Adscrito) [below of=responsable, yshift=8em, xshift=4em] {Personal-Adscrito(DNI, NRS, Nombre, Categoría, Título, Código)};
                \node[CP, xshift=-5.5ex] (CP_Personal-Adscrito) at(Personal-Adscrito) {NRS};
                \node[CC, xshift=-7ex] (CC_Personal-Adscrito) at(CP_Personal-Adscrito) {DNI};
                \node[CE, xshift=34ex] (CE_Personal-Adscrito) at(CP_Personal-Adscrito) {Código};
                \node[yshift=-1.4em, xshift=1.8em, purple] at(CP_Personal-Adscrito) {$(1,3)$};
                \node[yshift=1.4em, xshift=1.6em, purple] at(CE_Personal-Adscrito) {$(10)$};

                \node(Servicio-Dirigir) [right of=Personal-Adscrito, xshift=14em] {Servicio-Dirigir(Código, Nombre, Ubicación, NRS)};
                \node[CP, xshift=-5.5ex] (CP_Servicio-Dirigir) at(Servicio-Dirigir) {Código};
                \node[CC, xshift=28ex] (CC_Servicio-Dirigir) at(CP_Servicio-Dirigir) {NRS};
                \node[yshift=-1.4em, xshift=1.6em] at(CC_Servicio-Dirigir) {$(*)$};
                \node[CE](CE_Servicio-Dirigir) at(CC_Servicio-Dirigir) {NRS};
                \node[yshift=1.4em, xshift=1.6em, purple] at(CE_Servicio-Dirigir) {$(7)$};
                \node[yshift=-1.4em, xshift=1.6em, purple] at(CP_Servicio-Dirigir) {$(10)$};
            \end{tikzpicture}
        }
        \caption{Esquema Relacional.}
        \label{fig:esquemaRelacional}
    \end{figure}
\end{ejercicio}
\begin{ejercicio}
    Contestar de forma justificada a la siguiente cuestión: ¿Se permite que un paciente sea atendido varias veces el mismo día?

    Sí, porque la precisión de la fecha en la relación \emph{Atiende} es de hora (e incluso minuto si es necesario), por lo que se permite que un paciente sea atendido varias veces el mismo día.
\end{ejercicio}

\end{document}