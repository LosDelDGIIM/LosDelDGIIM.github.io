\documentclass[12pt]{article}

% Idioma y codificación
\usepackage[spanish, es-tabla]{babel}       %es-tabla para que se titule "Tabla"
\usepackage[utf8]{inputenc}

% Márgenes
\usepackage[a4paper,top=3cm,bottom=2.5cm,left=3cm,right=3cm]{geometry}

% Comentarios de bloque
\usepackage{verbatim}

% Paquetes de links
\usepackage[hidelinks]{hyperref}    % Permite enlaces
\usepackage{url}                    % redirecciona a la web

% Más opciones para enumeraciones
\usepackage{enumitem}

% Personalizar la portada
\usepackage{titling}

% Paquetes de tablas
\usepackage{multirow}


%------------------------------------------------------------------------

%Paquetes de figuras
\usepackage{caption}
\usepackage{subcaption} % Figuras al lado de otras
\usepackage{float}      % Poner figuras en el sitio indicado H.


% Paquetes de imágenes
\usepackage{graphicx}       % Paquete para añadir imágenes
\usepackage{transparent}    % Para manejar la opacidad de las figuras

% Paquete para usar colores
\usepackage[dvipsnames]{xcolor}
\usepackage{pagecolor}      % Para cambiar el color de la página

% Habilita tamaños de fuente mayores
\usepackage{fix-cm}

% Para los gráficos
\usepackage{tikz}

% Para poder situar los nodos en los grafos
\usetikzlibrary{positioning}


%------------------------------------------------------------------------

% Paquetes de matemáticas
\usepackage{mathtools, amsfonts, amssymb, mathrsfs}
\usepackage[makeroom]{cancel}     % Simplificar tachando
\usepackage{polynom}    % Divisiones y Ruffini
\usepackage{units} % Para poner fracciones diagonales con \nicefrac

\usepackage{pgfplots}   %Representar funciones
\pgfplotsset{compat=1.18}  % Versión 1.18

\usepackage{tikz-cd}    % Para usar diagramas de composiciones
\usetikzlibrary{calc}   % Para usar cálculo de coordenadas en tikz

%Definición de teoremas, etc.
\usepackage{amsthm}
%\swapnumbers   % Intercambia la posición del texto y de la numeración

\theoremstyle{plain}

\makeatletter
\@ifclassloaded{article}{
  \newtheorem{teo}{Teorema}[section]
}{
  \newtheorem{teo}{Teorema}[chapter]  % Se resetea en cada chapter
}
\makeatother

\newtheorem{coro}{Corolario}[teo]           % Se resetea en cada teorema
\newtheorem{prop}[teo]{Proposición}         % Usa el mismo contador que teorema
\newtheorem{lema}[teo]{Lema}                % Usa el mismo contador que teorema

\theoremstyle{remark}
\newtheorem*{observacion}{Observación}

\theoremstyle{definition}

\makeatletter
\@ifclassloaded{article}{
  \newtheorem{definicion}{Definición} [section]     % Se resetea en cada chapter
}{
  \newtheorem{definicion}{Definición} [chapter]     % Se resetea en cada chapter
}
\makeatother

\newtheorem*{notacion}{Notación}
\newtheorem*{ejemplo}{Ejemplo}
\newtheorem*{ejercicio*}{Ejercicio}             % No numerado
\newtheorem{ejercicio}{Ejercicio} [section]     % Se resetea en cada section


% Modificar el formato de la numeración del teorema "ejercicio"
\renewcommand{\theejercicio}{%
  \ifnum\value{section}=0 % Si no se ha iniciado ninguna sección
    \arabic{ejercicio}% Solo mostrar el número de ejercicio
  \else
    \thesection.\arabic{ejercicio}% Mostrar número de sección y número de ejercicio
  \fi
}


% \renewcommand\qedsymbol{$\blacksquare$}         % Cambiar símbolo QED
%------------------------------------------------------------------------

% Paquetes para encabezados
\usepackage{fancyhdr}
\pagestyle{fancy}
\fancyhf{}

\newcommand{\helv}{ % Modificación tamaño de letra
\fontfamily{}\fontsize{12}{12}\selectfont}
\setlength{\headheight}{15pt} % Amplía el tamaño del índice


%\usepackage{lastpage}   % Referenciar última pag   \pageref{LastPage}
\fancyfoot[C]{\thepage}

%------------------------------------------------------------------------

% Conseguir que no ponga "Capítulo 1". Sino solo "1."
\makeatletter
\@ifclassloaded{book}{
  \renewcommand{\chaptermark}[1]{\markboth{\thechapter.\ #1}{}} % En el encabezado
    
  \renewcommand{\@makechapterhead}[1]{%
  \vspace*{50\p@}%
  {\parindent \z@ \raggedright \normalfont
    \ifnum \c@secnumdepth >\m@ne
      \huge\bfseries \thechapter.\hspace{1em}\ignorespaces
    \fi
    \interlinepenalty\@M
    \Huge \bfseries #1\par\nobreak
    \vskip 40\p@
  }}
}
\makeatother

%------------------------------------------------------------------------
% Paquetes de cógido
\usepackage{minted}
\renewcommand\listingscaption{Código fuente}

\usepackage{fancyvrb}
% Personaliza el tamaño de los números de línea
\renewcommand{\theFancyVerbLine}{\small\arabic{FancyVerbLine}}

% Estilo para C++
\newminted{cpp}{
    frame=lines,
    framesep=2mm,
    baselinestretch=1.2,
    linenos,
    escapeinside=||
}

% para minted
\definecolor{LightGray}{rgb}{0.95,0.95,0.92}
\setminted{
    linenos=true,
    stepnumber=5,
    numberfirstline=true,
    autogobble,
    breaklines=true,
    breakautoindent=true,
    breaksymbolleft=,
    breaksymbolright=,
    breaksymbolindentleft=0pt,
    breaksymbolindentright=0pt,
    breaksymbolsepleft=0pt,
    breaksymbolsepright=0pt,
    fontsize=\footnotesize,
    bgcolor=LightGray,
    numbersep=10pt
}


\usepackage{listings} % Para incluir código desde un archivo

\renewcommand\lstlistingname{Código Fuente}
\renewcommand\lstlistlistingname{Índice de Códigos Fuente}

% Definir colores
\definecolor{vscodepurple}{rgb}{0.5,0,0.5}
\definecolor{vscodeblue}{rgb}{0,0,0.8}
\definecolor{vscodegreen}{rgb}{0,0.5,0}
\definecolor{vscodegray}{rgb}{0.5,0.5,0.5}
\definecolor{vscodebackground}{rgb}{0.97,0.97,0.97}
\definecolor{vscodelightgray}{rgb}{0.9,0.9,0.9}

% Configuración para el estilo de C similar a VSCode
\lstdefinestyle{vscode_C}{
  backgroundcolor=\color{vscodebackground},
  commentstyle=\color{vscodegreen},
  keywordstyle=\color{vscodeblue},
  numberstyle=\tiny\color{vscodegray},
  stringstyle=\color{vscodepurple},
  basicstyle=\scriptsize\ttfamily,
  breakatwhitespace=false,
  breaklines=true,
  captionpos=b,
  keepspaces=true,
  numbers=left,
  numbersep=5pt,
  showspaces=false,
  showstringspaces=false,
  showtabs=false,
  tabsize=2,
  frame=tb,
  framerule=0pt,
  aboveskip=10pt,
  belowskip=10pt,
  xleftmargin=10pt,
  xrightmargin=10pt,
  framexleftmargin=10pt,
  framexrightmargin=10pt,
  framesep=0pt,
  rulecolor=\color{vscodelightgray},
  backgroundcolor=\color{vscodebackground},
}

%------------------------------------------------------------------------

% Comandos definidos
\newcommand{\bb}[1]{\mathbb{#1}}
\newcommand{\cc}[1]{\mathcal{#1}}

% I prefer the slanted \leq
\let\oldleq\leq % save them in case they're every wanted
\let\oldgeq\geq
\renewcommand{\leq}{\leqslant}
\renewcommand{\geq}{\geqslant}

% Si y solo si
\newcommand{\sii}{\iff}

% Letras griegas
\newcommand{\eps}{\epsilon}
\newcommand{\veps}{\varepsilon}
\newcommand{\lm}{\lambda}

\newcommand{\ol}{\overline}
\newcommand{\ul}{\underline}
\newcommand{\wt}{\widetilde}
\newcommand{\wh}{\widehat}

\let\oldvec\vec
\renewcommand{\vec}{\overrightarrow}

% Derivadas parciales
\newcommand{\del}[2]{\frac{\partial #1}{\partial #2}}
\newcommand{\Del}[3]{\frac{\partial^{#1} #2}{\partial #3^{#1}}}
\newcommand{\deld}[2]{\dfrac{\partial #1}{\partial #2}}
\newcommand{\Deld}[3]{\dfrac{\partial^{#1} #2}{\partial #3^{#1}}}


\newcommand{\AstIg}{\stackrel{(\ast)}{=}}
\newcommand{\Hop}{\stackrel{L'H\hat{o}pital}{=}}

\newcommand{\red}[1]{{\color{red}#1}} % Para integrales, destacar los cambios.

% Método de integración
\newcommand{\MetInt}[2]{
    \left[\begin{array}{c}
        #1 \\ #2
    \end{array}\right]
}

% Declarar aplicaciones
% 1. Nombre aplicación
% 2. Dominio
% 3. Codominio
% 4. Variable
% 5. Imagen de la variable
\newcommand{\Func}[5]{
    \begin{equation*}
        \begin{array}{rrll}
            #1:& #2 & \longrightarrow & #3\\
               & #4 & \longmapsto & #5
        \end{array}
    \end{equation*}
}

%------------------------------------------------------------------------

\usetikzlibrary{er, fit}

\newcommand{\key}[1]{\ul{#1}}

% Definición de estilos
\tikzset{atributo/.style={attribute}}
\tikzset{entidad/.style={entity}}
\tikzset{relacion/.style={relationship}}
\tikzset{atributo derivado/.style={attribute, dashed}}
\tikzset{union discriminante/.style={dashed}}
\tikzset{entidad debil/.style={entidad, double distance=1.5 pt}}
\tikzset{participacion obligatoria/.style={double distance=1.5 pt}}
\tikzset{herencia/.style={draw, isosceles triangle, isosceles triangle apex angle=60, shape border rotate=230, minimum height=1cm, minimum width=1cm, fill=blue!20}}
\tikzset{agregacion/.style={draw, fit=#1, inner sep=0.5em}}


% Estilos estéticos
\tikzset{every entity/.style={draw=orange, fill=orange!20}}
\tikzset{every attribute/.style={draw=purple, fill=purple!20, font=\footnotesize}}
\tikzset{every relationship/.style={draw=green, fill=green!20, minimum height=2cm, minimum width=2cm}}

\begin{comment}
    \begin{tikzpicture}[node distance=6 em]
        \node [entidad](person){Person};
        \node [atributo derivado](pid) at (0.8,-2){\key{ID}} edge (person);
        \node [atributo](name) at (-0.8, -2) {Name} edge (person);
        \node [atributo](phone)[left of=person]{Phone} edge[union discriminante] (person);
        \node [atributo](address)[above left of=person]{Address} edge[-stealth] (person);
        \node [atributo](street)[above left of=address]{Street} edge (address);
        \node [atributo](city)[left of=address]{City} edge (address);
        \node [atributo](age)[above of=person]{Age} edge (person);
        \node [relacion](uses)[right of=person]{Uses} edge[participacion obligatoria] (person);
        \node [entidad debil](tool)[right of=uses]{Tool} edge (uses);
        \node [atributo](tid)[right of=tool]{\key{ID}} edge[Stealth-Stealth] (tool);
        \node [atributo](tname)[below of=tool]{Name} edge (tool);
    
        % Línea desde encima de Name hasta encima de ID
        \draw ([yshift=1em]name.north) -- ([yshift=1em]pid.north) node[circle, fill, inner sep=1.5pt] {};
    
        \node[agregacion=(A) (B) (R)] (box) {};
    
        \node [herencia, below of= B](H){} edge (B);
    \end{tikzpicture}
\end{comment}

% ---------------------------------------------------
\tikzset{
    CP/.style={
        overlay, text opacity=0, fill opacity=0,
        append after command={
            \pgfextra{
                \draw[thick] (\tikzlastnode.south west) -- (\tikzlastnode.south east);
                \node[below=0ex of \tikzlastnode] {CP};
            }
        }
    },
    CC/.style={
        overlay, text opacity=0, fill opacity=0,
        append after command={
            \pgfextra{
                \draw[thick] (\tikzlastnode.south west) -- (\tikzlastnode.south east);
                \node[below=0ex of \tikzlastnode] {CC};
            }
        }
    },
    CE/.style={
        overlay, text opacity=0, fill opacity=0,
        append after command={
            \pgfextra{
                \draw[thick, purple] (\tikzlastnode.north west) -- (\tikzlastnode.north east);
                \node[above=0ex of \tikzlastnode, text=purple] {CE};
            }
        }
    }
}


\begin{document}

    % 1. Foto de fondo
    % 2. Título
    % 3. Encabezado Izquierdo
    % 4. Color de fondo
    % 5. Coord x del titulo
    % 6. Coord y del titulo
    % 7. Fecha

    
    % 1. Foto de fondo
% 2. Título
% 3. Encabezado Izquierdo
% 4. Color de fondo
% 5. Coord x del titulo
% 6. Coord y del titulo
% 7. Fecha

\newcommand{\portada}[7]{

    \portadaBase{#1}{#2}{#3}{#4}{#5}{#6}{#7}
    \portadaBook{#1}{#2}{#3}{#4}{#5}{#6}{#7}
}

\newcommand{\portadaExamen}[7]{

    \portadaBase{#1}{#2}{#3}{#4}{#5}{#6}{#7}
    \portadaArticle{#1}{#2}{#3}{#4}{#5}{#6}{#7}
}




\newcommand{\portadaBase}[7]{

    % Tiene la portada principal y la licencia Creative Commons
    
    % 1. Foto de fondo
    % 2. Título
    % 3. Encabezado Izquierdo
    % 4. Color de fondo
    % 5. Coord x del titulo
    % 6. Coord y del titulo
    % 7. Fecha
    
    
    \thispagestyle{empty}               % Sin encabezado ni pie de página
    \newgeometry{margin=0cm}        % Márgenes nulos para la primera página
    
    
    % Encabezado
    \fancyhead[L]{\helv #3}
    \fancyhead[R]{\helv \nouppercase{\leftmark}}
    
    
    \pagecolor{#4}        % Color de fondo para la portada
    
    \begin{figure}[p]
        \centering
        \transparent{0.3}           % Opacidad del 30% para la imagen
        
        \includegraphics[width=\paperwidth, keepaspectratio]{assets/#1}
    
        \begin{tikzpicture}[remember picture, overlay]
            \node[anchor=north west, text=white, opacity=1, font=\fontsize{60}{90}\selectfont\bfseries\sffamily, align=left] at (#5, #6) {#2};
            
            \node[anchor=south east, text=white, opacity=1, font=\fontsize{12}{18}\selectfont\sffamily, align=right] at (9.7, 3) {\textbf{\href{https://losdeldgiim.github.io/}{Los Del DGIIM}}};
            
            \node[anchor=south east, text=white, opacity=1, font=\fontsize{12}{15}\selectfont\sffamily, align=right] at (9.7, 1.8) {Doble Grado en Ingeniería Informática y Matemáticas\\Universidad de Granada};
        \end{tikzpicture}
    \end{figure}
    
    
    \restoregeometry        % Restaurar márgenes normales para las páginas subsiguientes
    \pagecolor{white}       % Restaurar el color de página
    
    
    \newpage
    \thispagestyle{empty}               % Sin encabezado ni pie de página
    \begin{tikzpicture}[remember picture, overlay]
        \node[anchor=south west, inner sep=3cm] at (current page.south west) {
            \begin{minipage}{0.5\paperwidth}
                \href{https://creativecommons.org/licenses/by-nc-nd/4.0/}{
                    \includegraphics[height=2cm]{assets/Licencia.png}
                }\vspace{1cm}\\
                Esta obra está bajo una
                \href{https://creativecommons.org/licenses/by-nc-nd/4.0/}{
                    Licencia Creative Commons Atribución-NoComercial-SinDerivadas 4.0 Internacional (CC BY-NC-ND 4.0).
                }\\
    
                Eres libre de compartir y redistribuir el contenido de esta obra en cualquier medio o formato, siempre y cuando des el crédito adecuado a los autores originales y no persigas fines comerciales. 
            \end{minipage}
        };
    \end{tikzpicture}
    
    
    
    % 1. Foto de fondo
    % 2. Título
    % 3. Encabezado Izquierdo
    % 4. Color de fondo
    % 5. Coord x del titulo
    % 6. Coord y del titulo
    % 7. Fecha


}


\newcommand{\portadaBook}[7]{

    % 1. Foto de fondo
    % 2. Título
    % 3. Encabezado Izquierdo
    % 4. Color de fondo
    % 5. Coord x del titulo
    % 6. Coord y del titulo
    % 7. Fecha

    % Personaliza el formato del título
    \pretitle{\begin{center}\bfseries\fontsize{42}{56}\selectfont}
    \posttitle{\par\end{center}\vspace{2em}}
    
    % Personaliza el formato del autor
    \preauthor{\begin{center}\Large}
    \postauthor{\par\end{center}\vfill}
    
    % Personaliza el formato de la fecha
    \predate{\begin{center}\huge}
    \postdate{\par\end{center}\vspace{2em}}
    
    \title{#2}
    \author{\href{https://losdeldgiim.github.io/}{Los Del DGIIM}}
    \date{Granada, #7}
    \maketitle
    
    \tableofcontents
}




\newcommand{\portadaArticle}[7]{

    % 1. Foto de fondo
    % 2. Título
    % 3. Encabezado Izquierdo
    % 4. Color de fondo
    % 5. Coord x del titulo
    % 6. Coord y del titulo
    % 7. Fecha

    % Personaliza el formato del título
    \pretitle{\begin{center}\bfseries\fontsize{42}{56}\selectfont}
    \posttitle{\par\end{center}\vspace{2em}}
    
    % Personaliza el formato del autor
    \preauthor{\begin{center}\Large}
    \postauthor{\par\end{center}\vspace{3em}}
    
    % Personaliza el formato de la fecha
    \predate{\begin{center}\huge}
    \postdate{\par\end{center}\vspace{5em}}
    
    \title{#2}
    \author{\href{https://losdeldgiim.github.io/}{Los Del DGIIM}}
    \date{Granada, #7}
    \thispagestyle{empty}               % Sin encabezado ni pie de página
    \maketitle
    \vfill
}
    \portadaExamen{etsiitA4.jpg}{FBD\\Examen V}{FBD. Examen V}{MidnightBlue}{-8}{28}{2024-2025}{Arturo Olivares Martos}

    \begin{description}
        \item[Asignatura] Fundamentos de Bases de Datos.
        \item[Curso Académico] 2020-21.
        %\item[Grado] Doble Grado en Ingeniería Informática y Matemáticas.
        %\item[Grupo] Único.
        %\item[Profesor] Nicolás Marín Ruiz.
        \item[Descripción] Convocatoria Extraordinaria. Práctico Parcial 2 (Seminarios 3-4).
        %\item[Fecha] 8 de noviembre de 2021.
        % \item[Duración] 60 minutos.
    
    \end{description}
    \newpage

\section{Modelo 1}\label{sec:modelo1}
Considerar el esquema relacional de la Figura~\ref{fig:modelo1}, donde cada paciente tiene asignada una aseguradora que cubre sus gastos médicos. La tabla \verb|Consulta| almacena citas en las que un médico ha atendido a un paciente en una fecha dada.
\begin{figure}[H]
    \centering
    \begin{tikzpicture}[node distance=4 em]
        \node (aseguradora) {Asegura(id\_aseguradora, nombre\_aseguradora, país)};
        \node (paciente) [below of=aseguradora] {Paciente(id\_paciente, nombre\_paciente, id\_aseguradora, fecha\_nacimiento)};
        \node (medico) [below of=paciente] {Médico(id\_medico, nombre\_medico, especialidad, sueldo)};
        \node (consulta) [below of=medico] {Consulta(id\_paciente, id\_medico, fecha, precio\_facturado, numero\_horas)};

        \node[CP, xshift=-9ex] (CPAseguradora) at(aseguradora) {id\_aseguradora};
        \node[CP, xshift=-15ex] (CPMedico) at(medico) {id\_medico};
        \node[CP, xshift=-21ex] (CPPaciente) at(paciente) {id\_paciente};
        \node[CP, xshift=-11ex] (CPConsulta) at(consulta) {id\_paciente, id\_medico, fecha};

        \node[CE, xshift=31ex] (CEPaciente) at(CPPaciente) {id\_aseguradora};
        \node[CE, xshift=-10ex] (CEConsultaPaciente) at(CPConsulta) {id\_paciente};
        \node[CE, xshift=2ex] (CEConsultaMedico) at(CPConsulta) {id\_medico};

        \node[yshift=1.5em, xshift=1.5em, purple] at (CEPaciente) {(1)};
        \node[yshift=1.5em, xshift=1.5em, purple] at (CEConsultaPaciente) {(2)};
        \node[yshift=1.5em, xshift=1.5em, purple] at (CEConsultaMedico) {(3)};

        \node[yshift=-1.5em, xshift=1.5em, purple] at (CPAseguradora) {(1)};
        \node[yshift=-1.5em, xshift=1.5em, purple] at (CPMedico) {(3)};
        \node[yshift=-1.5em, xshift=1.5em, purple] at (CPPaciente) {(2)};
    \end{tikzpicture}
    \caption{Esquema relacional del Modelo~\ref{sec:modelo1}.}
    \label{fig:modelo1}
\end{figure}

\begin{ejercicio}[SQL]
    Encontrar el id del paciente y el id de la aseguradora de los pacientes atendidos en consulta para los que el precio más alto de sus consultas es más bajo que el precio facturado medio del paciente de código \verb|PAC01|.
    \begin{minted}{sql}
        SELECT id_paciente, id_aseguradora, MAX(precio_facturado)
            FROM paciente NATURAL JOIN consulta
            GROUP BY id_paciente, id_aseguradora
            HAVING MAX(precio_facturado) < (
                SELECT AVG(precio_facturado)
                    FROM consulta
                    WHERE id_paciente = 'PAC01'
            );
    \end{minted}
\end{ejercicio}


\begin{ejercicio}[SQL]
    Encontrar el id y el nombre de los médicos que, no habiendo atendido en consulta nunca a un paciente asegurado por la aseguradora de id \verb|MAP01|, han atendido al menos una consulta en los tres primeros meses de 2017.\\
    \begin{minted}{sql}
        SELECT id_medico, nombre_medico
            FROM medico
            WHERE id_medico IN (
                SELECT id_medico
                    FROM consulta
                    WHERE TO_CHAR(fecha, 'MM/YYYY')
                        BETWEEN TO_DATE('01/2017', 'MM/YYYY') AND TO_DATE('03/2017', 'MM/YYYY')
                MINUS
                SELECT id_medico
                    FROM consulta NATURAL JOIN paciente
                    WHERE id_aseguradora = 'MAP01'
            );
    \end{minted}
\end{ejercicio}

\begin{ejercicio}[AR]
    Encontrar, para cada médico que ha atendido en consulta, el precio más alto que ha facturado. Es suficiente con mostrar el id del médico, junto al precio facturado.\\

    Establecemos los siguientes alias:
    \begin{equation*}
        \rho(\text{Consulta})=C_1,C_2
    \end{equation*}
    
    Busquemos en primer lugar, para cada médico, los precios que no son candidatos a ser el precio más alto facturado por un médico.
    \begin{equation*}
        \pi_{\substack{C_1.\text{id\_médico}\\C_1.\text{precio\_facturado}}}\left(\sigma_{\substack{C_1.\text{id\_médico}=C_2.\text{id\_médico}\\\land\\C_1.\text{precio\_facturado}<C_2.\text{precio\_facturado}}}(C_1\times C_2)\right)
    \end{equation*}

    Por tanto, el precio más alto facturado por cada médico es:
    \begin{equation*}
        \pi_{\substack{\text{id\_médico}\\\text{precio\_facturado}}}\left(\text{Consulta}\right)
        -
        \pi_{\substack{C_1.\text{id\_médico}\\C_1.\text{precio\_facturado}}}\left(\sigma_{\substack{C_1.\text{id\_médico}=C_2.\text{id\_médico}\\\land\\C_1.\text{precio\_facturado}<C_2.\text{precio\_facturado}}}(C_1\times C_2)\right)
    \end{equation*}
\end{ejercicio}


\newpage
\section{Modelo 2}\label{sec:modelo2}

Considerar el esquema relacional de la Figura~\ref{fig:modelo2}, donde cada proyecto tiene asignada la empresa que lo elabora. La tabla \verb|Revisión| registra revisiones indicando qué revisor revisa qué proyecto en qué fecha.
\begin{figure}[H]
    \centering
    \begin{tikzpicture}[node distance=4 em]
        \node (empresa) {Empresa(id\_empresa, nombre\_empresa, país)};
        \node (proyecto) [below of=empresa] {Proyecto(id\_proyecto, titulo, id\_empresa, fecha\_elaboracion)};
        \node (revisor) [below of=proyecto] {Revisor(id\_revisor, nombre\_revisor, categoria, tarifa)};
        \node (revision) [below of=revisor] {Revisión(id\_revisor, id\_proyecto, fecha, puntuacion)};

        \node[CP, xshift=-7ex] (CPEmpresa) at(empresa) {id\_empresa};
        \node[CP, xshift=-13ex] (CPRevisor) at(revisor) {id\_revisor};
        \node[CP, xshift=-14ex] (CPProyecto) at(proyecto) {id\_proyecto};
        \node[CP, xshift=-2ex] (CPRevision) at(revision) {id\_revisor, id\_proyecto, fecha};

        \node[CE, xshift=19ex] (CEProyecto) at(CPProyecto) {id\_empresa};
        \node[CE, xshift=-10ex] (CERevisionRevisor) at(CPRevision) {id\_revisor};
        \node[CE, xshift=2ex] (CERevisionProyecto) at(CPRevision) {id\_proyecto};

        \node[yshift=1.5em, xshift=1.5em, purple] at (CEProyecto) {(1)};
        \node[yshift=1.5em, xshift=1.5em, purple] at (CERevisionRevisor) {(2)};
        \node[yshift=1.5em, xshift=1.5em, purple] at (CERevisionProyecto) {(3)};

        \node[yshift=-1.5em, xshift=1.5em, purple] at (CPEmpresa) {(1)};
        \node[yshift=-1.5em, xshift=1.5em, purple] at (CPRevisor) {(2)};
        \node[yshift=-1.5em, xshift=1.5em, purple] at (CPProyecto) {(3)};
    \end{tikzpicture}
    \caption{Esquema relacional del Modelo~\ref{sec:modelo2}.}
    \label{fig:modelo2}
\end{figure}

\begin{ejercicio}[SQL]
    Encontrar el id y el nombre de los revisores que, no habiendo revisado nunca un proyecto de la empresa con id \verb|EMP04|, han hecho al menos una revisión en los primeros seis meses de 2018.
    \begin{minted}{sql}
        SELECT nombre_revisor, id_revisor
            FROM revisor
            WHERE id_revisor IN (
                SELECT id_revisor
                    FROM revisión
                    WHERE TO_CHAR(fecha, 'MM/YYYY')
                        BETWEEN TO_DATE('01/2018', 'MM/YYYY') AND TO_DATE('06/2018', 'MM/YYYY')
                MINUS
                SELECT id_revisor
                    FROM revisión NATURAL JOIN proyecto
                    WHERE id_empresa = 'EMP04'
            );
    \end{minted}
\end{ejercicio}

\begin{ejercicio}[AR]
    Encontrar, para cada proyecto revisado, la puntuación más baja que ha obtenido en sus revisiones. Es suficiente con mostrar el id del proyecto, junto a la puntuación solicitada.\\

    Establecemos los siguientes alias:
    \begin{equation*}
        \rho(\text{Revisión})=R_1,R_2
    \end{equation*}

    Busquemos en primer lugar, para cada proyecto, las puntuaciones que no son candidatas a ser la puntuación más baja obtenida por un proyecto.
    \begin{equation*}
        \pi_{\substack{R_1.\text{id\_proyecto}\\R_1.\text{puntuación}}}\left(\sigma_{\substack{R_1.\text{id\_proyecto}=R_2.\text{id\_proyecto}\\\land\\R_1.\text{puntuación}>R_2.\text{puntuación}}}(R_1\times R_2)\right)
    \end{equation*}

    Por tanto, la puntuación más baja obtenida por cada proyecto es:
    \begin{equation*}
        \pi_{\substack{\text{id\_proyecto}\\\text{puntuación}}}\left(\text{Revisión}\right)
        -
        \pi_{\substack{R_1.\text{id\_proyecto}\\R_1.\text{puntuación}}}\left(\sigma_{\substack{R_1.\text{id\_proyecto}=R_2.\text{id\_proyecto}\\\land\\R_1.\text{puntuación}>R_2.\text{puntuación}}}(R_1\times R_2)\right)
    \end{equation*}
\end{ejercicio}


\newpage
\section{Modelo 3}\label{sec:modelo3}

Considerar el esquema relacional de la Figura~\ref{fig:modelo3}, donde cada vehículo tiene asignado un modelo de una marca determinada. La tabla \verb|Repara| registra reparaciones indicando qué mecánico repara qué vehículo en qué fecha.
\begin{figure}[H]
    \centering
    \resizebox{1.1\textwidth}{!}{
        \begin{tikzpicture}[node distance=15em]
            \node(modelo) {Modelo(id\_modelo, marca, descripcion)};
            \node(vehiculo)[right of=modelo, xshift=7em] {Vehículo(matrícula, id\_modelo, fecha\_matriculación)};
            \node(mecanico)[below of=modelo, yshift=10em] {Mecánico(id\_mecánico, nombre\_mecánico, cargo, salario)};
            \node(repara)[right of=mecanico, xshift=10em] {Repara(id\_mecánico, matrícula, fecha, número\_horas)};

            \node[CP, xshift=-6ex] (CP_modelo) at(modelo) {id\_modelo};
            \node[CP, xshift=-12ex] (CP_vehiculo) at(vehiculo) {matrícula};
            \node[CP, xshift=-11.5ex] (CP_mecanico) at(mecanico) {id\_mecánico};
            \node[CP, xshift=-3.5ex] (CP_repara) at(repara) {id\_mecánico, matrícula, fecha};

            \node[CE, xshift=11ex] (CE_vehiculo) at(CP_vehiculo) {id\_modelo};
            \node[CE, xshift=-10ex] (CE_repara_mecanico) at(CP_repara) {id\_mecánico};
            \node[CE, xshift=3ex] (CE_repara_vehiculo) at(CP_repara) {matrícula};

            \node[yshift=1.4em, xshift=1.4em, purple] at (CE_vehiculo) {(1)};
            \node[yshift=1.4em, xshift=1.4em, purple] at (CE_repara_mecanico) {(2)};
            \node[yshift=1.4em, xshift=1.4em, purple] at (CE_repara_vehiculo) {(3)};

            \node[yshift=-1.4em, xshift=1.4em, purple] at (CP_modelo) {(1)};
            \node[yshift=-1.4em, xshift=1.4em, purple] at (CP_mecanico) {(2)};
            \node[yshift=-1.4em, xshift=1.4em, purple] at (CP_vehiculo) {(3)};

        \end{tikzpicture}
    }
    \caption{Esquema relacional del Modelo~\ref{sec:modelo3}.}
    \label{fig:modelo3}
\end{figure}

\begin{ejercicio}[SQL]
    Encontrar la matrícula y el id del modelo de los vehículos reparados para los que la duración más corta de sus reparaciones es más alta que la duración media de las reparaciones del vehículo con matrícula \verb|1234|.

    \begin{minted}{sql}
        SELECT matrícula, id_modelo
            FROM vehículo NATURAL JOIN repara
            GROUP BY matrícula, id_modelo
            HAVING MIN(número_horas) > (
                SELECT AVG(número_horas)
                    FROM repara
                    WHERE matrícula = '1234'
                    GROUP BY matrícula
            );
    \end{minted}
\end{ejercicio}


\begin{ejercicio}[SQL]
    Encontrar la matrícula y la fecha de matriculación de los vehículos reparados para los que la fecha más reciente de reparación es anterior a la reparación más antigua del vehículo con matrícula \verb|5678| (recuerde que, en SQL, las fechas más recientes son mayores que las más antiguas).
    \begin{minted}{sql}
        SELECT matrícula, fecha_matriculación
            FROM vehículo NATURAL JOIN repara
            GROUP BY matrícula, fecha_matriculación
            HAVING MAX(fecha) < (
                SELECT MIN(fecha)
                    FROM repara
                    WHERE matrícula = '5678'
                    GROUP BY matrícula
            );
    \end{minted}
\end{ejercicio}

\begin{ejercicio}[SQL]
    Encontrar el id y el nombre de los mecánicos que, no habiendo reparado nunca un coche con id de modelo \verb|AURIS|, han hecho al menos una reparación en los últimos seis meses de 2020.
    \begin{minted}{sql}
        SELECT id_mecánico, nombre_mecánico
            FROM mecánico
            WHERE id_mecánico IN (
                SELECT id_mecánico
                    FROM repara
                    WHERE TO_CHAR(fecha, 'MM/YYYY')
                        BETWEEN TO_DATE('07/2020', 'MM/YYYY') AND TO_DATE('12/2020', 'MM/YYYY')
                MINUS
                SELECT id_mecánico
                    FROM repara NATURAL JOIN vehículo
                    WHERE id_modelo = 'AURIS'
            );
    \end{minted}
\end{ejercicio}

\begin{ejercicio}[AR]
    Encontrar, para cada modelo, la fecha de matriculación más reciente de vehículos de ese modelo (considere un orden en las fechas, en las que las más recientes son mayores que las más antiguas). Es suficiente con mostrar el id del modelo, junto a la fecha solicitada.\\

    Establecemos los siguientes alias:
    \begin{equation*}
        \rho\left(\pi_{\substack{\text{id\_modelo}\\\text{fecha\_matriculación}}}\left(\text{Vehículo}\right)\right)=V_1,V_2
    \end{equation*}

    Busquemos en primer lugar, para cada modelo, las fechas que no son candidatas a ser la fecha de matriculación más reciente de un vehículo de ese modelo.
    \begin{equation*}
        \pi_{\substack{V_1.\text{id\_modelo}\\V_1.\text{fecha\_matriculación}}}\left(\sigma_{\substack{V_1.\text{id\_modelo}=V_2.\text{id\_modelo}\\\land\\V_1.\text{fecha\_matriculación}<V_2.\text{fecha\_matriculación}}}(V_1\times V_2)\right)
    \end{equation*}

    Por tanto, la fecha de matriculación más reciente de cada modelo es:
    \begin{equation*}
        \pi_{\substack{\text{id\_modelo}\\\text{fecha\_matriculación}}}\left(\text{Vehículo}\right)
        -
        \pi_{\substack{V_1.\text{id\_modelo}\\V_1.\text{fecha\_matriculación}}}\left(\sigma_{\substack{V_1.\text{id\_modelo}=V_2.\text{id\_modelo}\\\land\\V_1.\text{fecha\_matriculación}<V_2.\text{fecha\_matriculación}}}(V_1\times V_2)\right)
    \end{equation*}
\end{ejercicio}


\newpage
\section{Modelo 4}\label{sec:modelo4}

Considerar el esquema relacional de la Figura~\ref{fig:modelo4}, donde cada grupo musical tiene asignado un género musical. La tabla \verb|Actuación| registra
actuaciones indicando qué grupo actúa en qué población y cuándo.
\begin{figure}[H]
    \centering
    \begin{tikzpicture}[node distance=4 em]
        \node (genero) {Género(id\_género, nombre\_género, país\_origen)};
        \node (grupo) [below of=genero] {Grupo(id\_grupo, nombre\_grupo, id\_género, fecha\_creación)};
        \node (poblacion) [below of=grupo] {Población(id\_población, nombre\_población, número, habitantes)};
        \node (actuacion) [below of=poblacion] {Actuación(id\_poblacion, id\_grupo, fecha, duración)};

        \node[CP, xshift=-10ex] (CPGenero) at(genero) {id\_genero};
        \node[CP, xshift=-14.5ex] (CPPoblacion) at(poblacion) {id\_poblacion};
        \node[CP, xshift=-17ex] (CPGrupo) at(grupo) {id\_grupo};
        \node[CP, xshift=0ex] (CPActuacion) at(actuacion) {id\_poblacion, id\_grupo, fecha};

        \node[CE, xshift=25ex] (CEGenero) at(CPGrupo) {id\_genero};
        \node[CE, xshift=-9ex] (CEActuacionPoblacion) at(CPActuacion) {id\_población};
        \node[CE, xshift=4ex] (CEActuacionGrupo) at(CPActuacion) {id\_grupo};

        \node[yshift=1.5em, xshift=1.5em, purple] at (CEGenero) {(1)};
        \node[yshift=1.5em, xshift=1.5em, purple] at (CEActuacionPoblacion) {(2)};
        \node[yshift=1.5em, xshift=1.5em, purple] at (CEActuacionGrupo) {(3)};

        \node[yshift=-1.5em, xshift=1.5em, purple] at (CPGenero) {(1)};
        \node[yshift=-1.5em, xshift=1.5em, purple] at (CPPoblacion) {(2)};
        \node[yshift=-1.5em, xshift=1.5em, purple] at (CPGrupo) {(3)};
    \end{tikzpicture}
    \caption{Esquema relacional del Modelo~\ref{sec:modelo4}.}
    \label{fig:modelo4}
\end{figure}

\begin{ejercicio}[SQL]
    Encuentre el id y la fecha de creación de los grupos con actuaciones para los que la fecha más reciente de actuación es anterior a la actuación más antigua del grupo de código \verb|GRUP77| (recuerde que, en SQL, las fechas más recientes son mayores que las más antiguas).
    \begin{minted}{sql}
        SELECT id_grupo, fecha_creación
            FROM grupo
            WHERE id_grupo IN (
                SELECT id_grupo
                    FROM actuación
                    GROUP BY id_grupo
                    HAVING MAX(fecha) < (
                        SELECT MIN(fecha)
                            FROM actuación
                            WHERE id_grupo = 'GRUP77'
                            GROUP BY id_grupo
                    )
            );
    \end{minted}
\end{ejercicio}

\begin{ejercicio}[AR]
    Elaborar un listado de los países que han dado origen a un solo género musical.\\

    Creamos en primer lugar el siguiente alias.
    \begin{equation*}
        \rho(\text{Género})=G_1,G_2
    \end{equation*}

    Los países buscados son los que no han dado origen a dos géneros musicales distintos:
    \begin{equation*}
        \pi_{\text{país}}(\text{Género})
        - \pi_{G_1.\text{país}}\left(\sigma_{\substack{G_1.\text{país}=G_2.\text{país}\\\land \\ G_1.\text{id\_género}\neq G_2.\text{id\_género}}}(G_1\times G_2)\right)
    \end{equation*}
\end{ejercicio}

\end{document}