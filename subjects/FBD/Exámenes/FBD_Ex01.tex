\documentclass[12pt]{article}

% Idioma y codificación
\usepackage[spanish, es-tabla]{babel}       %es-tabla para que se titule "Tabla"
\usepackage[utf8]{inputenc}

% Márgenes
\usepackage[a4paper,top=3cm,bottom=2.5cm,left=3cm,right=3cm]{geometry}

% Comentarios de bloque
\usepackage{verbatim}

% Paquetes de links
\usepackage[hidelinks]{hyperref}    % Permite enlaces
\usepackage{url}                    % redirecciona a la web

% Más opciones para enumeraciones
\usepackage{enumitem}

% Personalizar la portada
\usepackage{titling}

% Paquetes de tablas
\usepackage{multirow}


%------------------------------------------------------------------------

%Paquetes de figuras
\usepackage{caption}
\usepackage{subcaption} % Figuras al lado de otras
\usepackage{float}      % Poner figuras en el sitio indicado H.


% Paquetes de imágenes
\usepackage{graphicx}       % Paquete para añadir imágenes
\usepackage{transparent}    % Para manejar la opacidad de las figuras

% Paquete para usar colores
\usepackage[dvipsnames]{xcolor}
\usepackage{pagecolor}      % Para cambiar el color de la página

% Habilita tamaños de fuente mayores
\usepackage{fix-cm}

% Para los gráficos
\usepackage{tikz}

% Para poder situar los nodos en los grafos
\usetikzlibrary{positioning}


%------------------------------------------------------------------------

% Paquetes de matemáticas
\usepackage{mathtools, amsfonts, amssymb, mathrsfs}
\usepackage[makeroom]{cancel}     % Simplificar tachando
\usepackage{polynom}    % Divisiones y Ruffini
\usepackage{units} % Para poner fracciones diagonales con \nicefrac

\usepackage{pgfplots}   %Representar funciones
\pgfplotsset{compat=1.18}  % Versión 1.18

\usepackage{tikz-cd}    % Para usar diagramas de composiciones
\usetikzlibrary{calc}   % Para usar cálculo de coordenadas en tikz

%Definición de teoremas, etc.
\usepackage{amsthm}
%\swapnumbers   % Intercambia la posición del texto y de la numeración

\theoremstyle{plain}

\makeatletter
\@ifclassloaded{article}{
  \newtheorem{teo}{Teorema}[section]
}{
  \newtheorem{teo}{Teorema}[chapter]  % Se resetea en cada chapter
}
\makeatother

\newtheorem{coro}{Corolario}[teo]           % Se resetea en cada teorema
\newtheorem{prop}[teo]{Proposición}         % Usa el mismo contador que teorema
\newtheorem{lema}[teo]{Lema}                % Usa el mismo contador que teorema

\theoremstyle{remark}
\newtheorem*{observacion}{Observación}

\theoremstyle{definition}

\makeatletter
\@ifclassloaded{article}{
  \newtheorem{definicion}{Definición} [section]     % Se resetea en cada chapter
}{
  \newtheorem{definicion}{Definición} [chapter]     % Se resetea en cada chapter
}
\makeatother

\newtheorem*{notacion}{Notación}
\newtheorem*{ejemplo}{Ejemplo}
\newtheorem*{ejercicio*}{Ejercicio}             % No numerado
\newtheorem{ejercicio}{Ejercicio} [section]     % Se resetea en cada section


% Modificar el formato de la numeración del teorema "ejercicio"
\renewcommand{\theejercicio}{%
  \ifnum\value{section}=0 % Si no se ha iniciado ninguna sección
    \arabic{ejercicio}% Solo mostrar el número de ejercicio
  \else
    \thesection.\arabic{ejercicio}% Mostrar número de sección y número de ejercicio
  \fi
}


% \renewcommand\qedsymbol{$\blacksquare$}         % Cambiar símbolo QED
%------------------------------------------------------------------------

% Paquetes para encabezados
\usepackage{fancyhdr}
\pagestyle{fancy}
\fancyhf{}

\newcommand{\helv}{ % Modificación tamaño de letra
\fontfamily{}\fontsize{12}{12}\selectfont}
\setlength{\headheight}{15pt} % Amplía el tamaño del índice


%\usepackage{lastpage}   % Referenciar última pag   \pageref{LastPage}
\fancyfoot[C]{\thepage}

%------------------------------------------------------------------------

% Conseguir que no ponga "Capítulo 1". Sino solo "1."
\makeatletter
\@ifclassloaded{book}{
  \renewcommand{\chaptermark}[1]{\markboth{\thechapter.\ #1}{}} % En el encabezado
    
  \renewcommand{\@makechapterhead}[1]{%
  \vspace*{50\p@}%
  {\parindent \z@ \raggedright \normalfont
    \ifnum \c@secnumdepth >\m@ne
      \huge\bfseries \thechapter.\hspace{1em}\ignorespaces
    \fi
    \interlinepenalty\@M
    \Huge \bfseries #1\par\nobreak
    \vskip 40\p@
  }}
}
\makeatother

%------------------------------------------------------------------------
% Paquetes de cógido
\usepackage{minted}
\renewcommand\listingscaption{Código fuente}

\usepackage{fancyvrb}
% Personaliza el tamaño de los números de línea
\renewcommand{\theFancyVerbLine}{\small\arabic{FancyVerbLine}}

% Estilo para C++
\newminted{cpp}{
    frame=lines,
    framesep=2mm,
    baselinestretch=1.2,
    linenos,
    escapeinside=||
}

% para minted
\definecolor{LightGray}{rgb}{0.95,0.95,0.92}
\setminted{
    linenos=true,
    stepnumber=5,
    numberfirstline=true,
    autogobble,
    breaklines=true,
    breakautoindent=true,
    breaksymbolleft=,
    breaksymbolright=,
    breaksymbolindentleft=0pt,
    breaksymbolindentright=0pt,
    breaksymbolsepleft=0pt,
    breaksymbolsepright=0pt,
    fontsize=\footnotesize,
    bgcolor=LightGray,
    numbersep=10pt
}


\usepackage{listings} % Para incluir código desde un archivo

\renewcommand\lstlistingname{Código Fuente}
\renewcommand\lstlistlistingname{Índice de Códigos Fuente}

% Definir colores
\definecolor{vscodepurple}{rgb}{0.5,0,0.5}
\definecolor{vscodeblue}{rgb}{0,0,0.8}
\definecolor{vscodegreen}{rgb}{0,0.5,0}
\definecolor{vscodegray}{rgb}{0.5,0.5,0.5}
\definecolor{vscodebackground}{rgb}{0.97,0.97,0.97}
\definecolor{vscodelightgray}{rgb}{0.9,0.9,0.9}

% Configuración para el estilo de C similar a VSCode
\lstdefinestyle{vscode_C}{
  backgroundcolor=\color{vscodebackground},
  commentstyle=\color{vscodegreen},
  keywordstyle=\color{vscodeblue},
  numberstyle=\tiny\color{vscodegray},
  stringstyle=\color{vscodepurple},
  basicstyle=\scriptsize\ttfamily,
  breakatwhitespace=false,
  breaklines=true,
  captionpos=b,
  keepspaces=true,
  numbers=left,
  numbersep=5pt,
  showspaces=false,
  showstringspaces=false,
  showtabs=false,
  tabsize=2,
  frame=tb,
  framerule=0pt,
  aboveskip=10pt,
  belowskip=10pt,
  xleftmargin=10pt,
  xrightmargin=10pt,
  framexleftmargin=10pt,
  framexrightmargin=10pt,
  framesep=0pt,
  rulecolor=\color{vscodelightgray},
  backgroundcolor=\color{vscodebackground},
}

%------------------------------------------------------------------------

% Comandos definidos
\newcommand{\bb}[1]{\mathbb{#1}}
\newcommand{\cc}[1]{\mathcal{#1}}

% I prefer the slanted \leq
\let\oldleq\leq % save them in case they're every wanted
\let\oldgeq\geq
\renewcommand{\leq}{\leqslant}
\renewcommand{\geq}{\geqslant}

% Si y solo si
\newcommand{\sii}{\iff}

% Letras griegas
\newcommand{\eps}{\epsilon}
\newcommand{\veps}{\varepsilon}
\newcommand{\lm}{\lambda}

\newcommand{\ol}{\overline}
\newcommand{\ul}{\underline}
\newcommand{\wt}{\widetilde}
\newcommand{\wh}{\widehat}

\let\oldvec\vec
\renewcommand{\vec}{\overrightarrow}

% Derivadas parciales
\newcommand{\del}[2]{\frac{\partial #1}{\partial #2}}
\newcommand{\Del}[3]{\frac{\partial^{#1} #2}{\partial #3^{#1}}}
\newcommand{\deld}[2]{\dfrac{\partial #1}{\partial #2}}
\newcommand{\Deld}[3]{\dfrac{\partial^{#1} #2}{\partial #3^{#1}}}


\newcommand{\AstIg}{\stackrel{(\ast)}{=}}
\newcommand{\Hop}{\stackrel{L'H\hat{o}pital}{=}}

\newcommand{\red}[1]{{\color{red}#1}} % Para integrales, destacar los cambios.

% Método de integración
\newcommand{\MetInt}[2]{
    \left[\begin{array}{c}
        #1 \\ #2
    \end{array}\right]
}

% Declarar aplicaciones
% 1. Nombre aplicación
% 2. Dominio
% 3. Codominio
% 4. Variable
% 5. Imagen de la variable
\newcommand{\Func}[5]{
    \begin{equation*}
        \begin{array}{rrll}
            #1:& #2 & \longrightarrow & #3\\
               & #4 & \longmapsto & #5
        \end{array}
    \end{equation*}
}

%------------------------------------------------------------------------

\usetikzlibrary{er, fit}

\newcommand{\key}[1]{\ul{#1}}

% Definición de estilos
\tikzset{atributo/.style={attribute}}
\tikzset{entidad/.style={entity}}
\tikzset{relacion/.style={relationship}}
\tikzset{atributo derivado/.style={attribute, dashed}}
\tikzset{union discriminante/.style={dashed}}
\tikzset{entidad debil/.style={entidad, double distance=1.5 pt}}
\tikzset{participacion obligatoria/.style={double distance=1.5 pt}}
\tikzset{herencia/.style={draw, isosceles triangle, isosceles triangle apex angle=60, shape border rotate=230, minimum height=1cm, minimum width=1cm, fill=blue!20}}
\tikzset{agregacion/.style={draw, fit=#1, inner sep=0.5em}}


% Estilos estéticos
\tikzset{every entity/.style={draw=orange, fill=orange!20}}
\tikzset{every attribute/.style={draw=purple, fill=purple!20, font=\footnotesize}}
\tikzset{every relationship/.style={draw=green, fill=green!20, minimum height=2cm, minimum width=2cm}}

\begin{comment}
    \begin{tikzpicture}[node distance=6 em]
        \node [entidad](person){Person};
        \node [atributo derivado](pid) at (0.8,-2){\key{ID}} edge (person);
        \node [atributo](name) at (-0.8, -2) {Name} edge (person);
        \node [atributo](phone)[left of=person]{Phone} edge[union discriminante] (person);
        \node [atributo](address)[above left of=person]{Address} edge[-stealth] (person);
        \node [atributo](street)[above left of=address]{Street} edge (address);
        \node [atributo](city)[left of=address]{City} edge (address);
        \node [atributo](age)[above of=person]{Age} edge (person);
        \node [relacion](uses)[right of=person]{Uses} edge[participacion obligatoria] (person);
        \node [entidad debil](tool)[right of=uses]{Tool} edge (uses);
        \node [atributo](tid)[right of=tool]{\key{ID}} edge[Stealth-Stealth] (tool);
        \node [atributo](tname)[below of=tool]{Name} edge (tool);
    
        % Línea desde encima de Name hasta encima de ID
        \draw ([yshift=1em]name.north) -- ([yshift=1em]pid.north) node[circle, fill, inner sep=1.5pt] {};
    
        \node[agregacion=(A) (B) (R)] (box) {};
    
        \node [herencia, below of= B](H){} edge (B);
    \end{tikzpicture}
\end{comment}

% ---------------------------------------------------
\tikzset{
    CP/.style={
        overlay, text opacity=0, fill opacity=0,
        append after command={
            \pgfextra{
                \draw[thick] (\tikzlastnode.south west) -- (\tikzlastnode.south east);
                \node[below=0ex of \tikzlastnode] {CP};
            }
        }
    },
    CC/.style={
        overlay, text opacity=0, fill opacity=0,
        append after command={
            \pgfextra{
                \draw[thick] (\tikzlastnode.south west) -- (\tikzlastnode.south east);
                \node[below=0ex of \tikzlastnode] {CC};
            }
        }
    },
    CE/.style={
        overlay, text opacity=0, fill opacity=0,
        append after command={
            \pgfextra{
                \draw[thick, purple] (\tikzlastnode.north west) -- (\tikzlastnode.north east);
                \node[above=0ex of \tikzlastnode, text=purple] {CE};
            }
        }
    }
}

\begin{comment}
\begin{tikzpicture}
    \node (alumno) {Alumno(DNI, kjbbhjbj, Código, jhjghjh)};
    \node (asignatura) [below of=alumno, yshift=-5em] {Asignatura(hjbjhblj, jjgjhb, kgjhh, Código)};

    \node[CP, xshift=17.3ex] (codigoCP) at(asignatura) {Código};
    \node[CP, xshift=-9ex] (dniCP) at(alumno) {DNI};


    \node[CE, xshift=17.3ex] (codigoCE) at(asignatura) {Código};
    \draw[-Stealth, purple] (codigoCE.north west) -- (dniCP.south east);
    
\end{tikzpicture}
\end{comment}


\begin{document}

    % 1. Foto de fondo
    % 2. Título
    % 3. Encabezado Izquierdo
    % 4. Color de fondo
    % 5. Coord x del titulo
    % 6. Coord y del titulo
    % 7. Fecha

    
    % 1. Foto de fondo
% 2. Título
% 3. Encabezado Izquierdo
% 4. Color de fondo
% 5. Coord x del titulo
% 6. Coord y del titulo
% 7. Fecha

\newcommand{\portada}[7]{

    \portadaBase{#1}{#2}{#3}{#4}{#5}{#6}{#7}
    \portadaBook{#1}{#2}{#3}{#4}{#5}{#6}{#7}
}

\newcommand{\portadaExamen}[7]{

    \portadaBase{#1}{#2}{#3}{#4}{#5}{#6}{#7}
    \portadaArticle{#1}{#2}{#3}{#4}{#5}{#6}{#7}
}




\newcommand{\portadaBase}[7]{

    % Tiene la portada principal y la licencia Creative Commons
    
    % 1. Foto de fondo
    % 2. Título
    % 3. Encabezado Izquierdo
    % 4. Color de fondo
    % 5. Coord x del titulo
    % 6. Coord y del titulo
    % 7. Fecha
    
    
    \thispagestyle{empty}               % Sin encabezado ni pie de página
    \newgeometry{margin=0cm}        % Márgenes nulos para la primera página
    
    
    % Encabezado
    \fancyhead[L]{\helv #3}
    \fancyhead[R]{\helv \nouppercase{\leftmark}}
    
    
    \pagecolor{#4}        % Color de fondo para la portada
    
    \begin{figure}[p]
        \centering
        \transparent{0.3}           % Opacidad del 30% para la imagen
        
        \includegraphics[width=\paperwidth, keepaspectratio]{assets/#1}
    
        \begin{tikzpicture}[remember picture, overlay]
            \node[anchor=north west, text=white, opacity=1, font=\fontsize{60}{90}\selectfont\bfseries\sffamily, align=left] at (#5, #6) {#2};
            
            \node[anchor=south east, text=white, opacity=1, font=\fontsize{12}{18}\selectfont\sffamily, align=right] at (9.7, 3) {\textbf{\href{https://losdeldgiim.github.io/}{Los Del DGIIM}}};
            
            \node[anchor=south east, text=white, opacity=1, font=\fontsize{12}{15}\selectfont\sffamily, align=right] at (9.7, 1.8) {Doble Grado en Ingeniería Informática y Matemáticas\\Universidad de Granada};
        \end{tikzpicture}
    \end{figure}
    
    
    \restoregeometry        % Restaurar márgenes normales para las páginas subsiguientes
    \pagecolor{white}       % Restaurar el color de página
    
    
    \newpage
    \thispagestyle{empty}               % Sin encabezado ni pie de página
    \begin{tikzpicture}[remember picture, overlay]
        \node[anchor=south west, inner sep=3cm] at (current page.south west) {
            \begin{minipage}{0.5\paperwidth}
                \href{https://creativecommons.org/licenses/by-nc-nd/4.0/}{
                    \includegraphics[height=2cm]{assets/Licencia.png}
                }\vspace{1cm}\\
                Esta obra está bajo una
                \href{https://creativecommons.org/licenses/by-nc-nd/4.0/}{
                    Licencia Creative Commons Atribución-NoComercial-SinDerivadas 4.0 Internacional (CC BY-NC-ND 4.0).
                }\\
    
                Eres libre de compartir y redistribuir el contenido de esta obra en cualquier medio o formato, siempre y cuando des el crédito adecuado a los autores originales y no persigas fines comerciales. 
            \end{minipage}
        };
    \end{tikzpicture}
    
    
    
    % 1. Foto de fondo
    % 2. Título
    % 3. Encabezado Izquierdo
    % 4. Color de fondo
    % 5. Coord x del titulo
    % 6. Coord y del titulo
    % 7. Fecha


}


\newcommand{\portadaBook}[7]{

    % 1. Foto de fondo
    % 2. Título
    % 3. Encabezado Izquierdo
    % 4. Color de fondo
    % 5. Coord x del titulo
    % 6. Coord y del titulo
    % 7. Fecha

    % Personaliza el formato del título
    \pretitle{\begin{center}\bfseries\fontsize{42}{56}\selectfont}
    \posttitle{\par\end{center}\vspace{2em}}
    
    % Personaliza el formato del autor
    \preauthor{\begin{center}\Large}
    \postauthor{\par\end{center}\vfill}
    
    % Personaliza el formato de la fecha
    \predate{\begin{center}\huge}
    \postdate{\par\end{center}\vspace{2em}}
    
    \title{#2}
    \author{\href{https://losdeldgiim.github.io/}{Los Del DGIIM}}
    \date{Granada, #7}
    \maketitle
    
    \tableofcontents
}




\newcommand{\portadaArticle}[7]{

    % 1. Foto de fondo
    % 2. Título
    % 3. Encabezado Izquierdo
    % 4. Color de fondo
    % 5. Coord x del titulo
    % 6. Coord y del titulo
    % 7. Fecha

    % Personaliza el formato del título
    \pretitle{\begin{center}\bfseries\fontsize{42}{56}\selectfont}
    \posttitle{\par\end{center}\vspace{2em}}
    
    % Personaliza el formato del autor
    \preauthor{\begin{center}\Large}
    \postauthor{\par\end{center}\vspace{3em}}
    
    % Personaliza el formato de la fecha
    \predate{\begin{center}\huge}
    \postdate{\par\end{center}\vspace{5em}}
    
    \title{#2}
    \author{\href{https://losdeldgiim.github.io/}{Los Del DGIIM}}
    \date{Granada, #7}
    \thispagestyle{empty}               % Sin encabezado ni pie de página
    \maketitle
    \vfill
}
    \portadaExamen{ffccA4.jpg}{FBD\\Examen I}{FBD. Examen I}{MidnightBlue}{-8}{28}{2024-2025}{Arturo Olivares Martos}

    \begin{description}
        \item[Asignatura] Fundamentos de Bases de Datos.
        \item[Curso Académico] 2017-18.
        %\item[Grado] Doble Grado en Ingeniería Informática y Matemáticas.
        %\item[Grupo] Único.
        %\item[Profesor] Nicolás Marín Ruiz.
        \item[Descripción] Convocatoria Extraordinaria. Práctico Parcial 1 (Seminarios 1-2).
        %\item[Fecha] 8 de noviembre de 2021.
        % \item[Duración] 60 minutos.
    
    \end{description}
    \newpage



    Se trata de organizar la información relativa a una plataforma de control de actividad y monitoreo de actividad física. Los datos y restricciones a considerar son:
    \begin{itemize}
        \item Se desea guardar la información sobre los usuarios identificados por su e-mail y de los cuales se desea conocer su password, nombre, apellidos y edad.
        \item Un usuario puede tener uno o más dispositivos asociados, cada uno identificado por un código único.
        \item En particular interesa conocer en detalle los datos de sus móviles (número de teléfono, marca y modelo) y de su pulsera de actividad, de las cual se desea también saber marca y modelo.
        \item Cada pulsera puede monitorear en un día y hora determinados una actividad específica, identificada por un nombre. Hay sólo dos tipos de actividades: pasos y ritmo cardíaco. De los pasos se desea saber su cantidad, mientras que del ritmo cardíaco se desea conocer el número de pulsaciones por minuto.
        \item Se quiere poder registrar además las alertas que emiten las pulseras a un usuario en base a una actividad registrada. Por ejemplo: ``ritmo cardiaco elevado'' o ``meta de cantidad de pasos diaria alcanzada''.
        \item Finalmente se desea saber cuándo se sincroniza una pulsera de actividad. Suponemos que un móvil sólo puede sincronizarse con una pulsera, pero una pulsera sí puede sincronizarse con varios móviles.
    \end{itemize}

\begin{ejercicio}[$6$ puntos] Dibujar el esquema Entidad/Relación que represente adecuadamente dicha información.

    El esquema Entidad/Relación se encuentra en la Figura~\ref{fig:esquemaER}.
    Notemos que hemos hecho las siguientes consideraciones:
    \begin{itemize}
        \item Las alertas no se envían al móvil sincronizado, sino directamente a un usuario mediante la dirección de correo electrónico.
        \item La alerta de una actividad puede enviarse a muchos usuarios distintos, y un usuario puede recibir alertas de muchas actividades.
        \item Hemos modelado la entidad débil \emph{Actividad}, para poder así especificar que cada actividad depende de una pulsera.
        \item Los atributos \emph{Modelo} y \emph{Marca} de \emph{Pulsera} y \emph{Móvil} no se han asignado a la entidad \emph{dispositivo}, ya que no se establece que se quieran saber de todos los dispositivos.
    \end{itemize}
    \begin{figure}
        \centering
        \scalebox{0.9}{
        \begin{tikzpicture}[node distance=6.3 em]
            \node[entidad] (usuario) {Usuario};
            \node[atributo](email) [above left of=usuario] {\key{Email}} edge (usuario);
            \node[atributo](password) [above right of=usuario] {Password} edge (usuario);
            \node[atributo](nombre) [below left of=usuario] {Nombre} edge (usuario);
            \node[atributo](apellidos) [left of=usuario, yshift=-2em] {Apellidos} edge (usuario);
            \node[atributo](fecha_nac) [above of=usuario] {Fecha Nacimiento} edge (usuario);
            \node[atributo derivado](edad) [left of=usuario, yshift=2em] {Edad} edge (usuario);

            \node[relacion] (tiene) [right of=usuario] {Tiene} edge[-Stealth] (usuario);
            \node[entidad] (dispositivo) [right of=tiene] {Dispositivo} edge (tiene);
            \node[atributo](codigo) [above of=dispositivo] {\key{Código}} edge (dispositivo);

            \node[herencia] (tipo_dispositivo) [below of=dispositivo, yshift=-1em] {Tipo} edgenode[midway, right] {Disjunta} (dispositivo);
            \node[entidad] (movil) [below right of=tipo_dispositivo, yshift=-4em, xshift=2em] {Móvil} edge (tipo_dispositivo);
            \node[atributo](telefono) [above of=movil] {\key{Teléfono}} edge[union discriminante] (movil);
            \node[atributo](marca) [below of=movil] {Marca} edge (movil);
            \node[atributo](modelo) [below right of=movil] {Modelo} edge (movil);

            \node[entidad] (pulsera) [below left of=tipo_dispositivo, yshift=-4em, xshift=-2em] {Pulsera} edge (tipo_dispositivo);
            \node[atributo](marca_p) [above of=pulsera, yshift=-1em] {Marca} edge (pulsera);
            \node[atributo](modelo_p) [below of=pulsera, yshift=1em] {Modelo} edge (pulsera);
            
            \node[entidad debil] (actividad) [left of=pulsera, xshift=0.4ex] {Actividad} edge (pulsera);
            \node[atributo](fecha) [above left of=actividad, yshift=-1em] {\key{Fecha}} edge[union discriminante] (actividad);
            \node[atributo](nombre_a) [left of=actividad] {\key{Nombre}} edge[union discriminante] (actividad);

            \node[herencia] (tipo_actividad) [below of=actividad] {Tipo} edge[participacion obligatoria]node[midway, right] {Disjunta} (actividad);
            \node[entidad] (pasos) [below left of=tipo_actividad] {Pasos} edge (tipo_actividad);
            \node[atributo](cantidad) [below of=pasos, yshift=3em] {Cantidad} edge (pasos);

            \node[entidad] (ritmo) [right of=pasos, xshift=2em] {Ritmo Cardíaco} edge (tipo_actividad);
            \node[atributo](pulsaciones) [below of=ritmo, yshift=3em] {Pulsaciones} edge (ritmo);

            \node[relacion] (alerta) [below of=usuario] {Alerta} edge (actividad) edge (usuario);
            \node[atributo](mensaje) [right of=alerta] {Mensaje} edge (alerta);

            \node[relacion] (sincroniza) [below of=tipo_dispositivo, yshift=-2em] {Sincroniza} edge[-Stealth] (pulsera) edge (movil);
            \node[atributo](fecha_s) [below of=sincroniza] {Fecha} edge (sincroniza);
        \end{tikzpicture}
        }
        \caption{Esquema Entidad/Relación.}
        \label{fig:esquemaER}
    \end{figure}
\end{ejercicio}

\begin{ejercicio}[$2.5$ puntos] Elaborar el esquema relacional al que da lugar indicando las claves primarias, candidatas y externas correspondientes.
    
    El esquema relacional se encuentra en la Figura~\ref{fig:esquemaRelacional}.
    \begin{figure}
        \centering
        \resizebox{1.1\textwidth}{!}{
            \begin{tikzpicture}[node distance=15em]
                \node(usuario) {Usuario(Email, Password, Nombre, Apellidos, Fecha\_Nac)};
                \node(alerta) [right of=usuario, xshift=9em] {Alerta(Email, Cód\_Pulsera, Nombre, Mensaje)};

                \node(tiene) [below of=usuario, yshift=9em, xshift=-7em] {Tiene(Email, Código)};
                \node(movil) [right of=tiene] {Móvil(Código, Teléfono, Marca, Modelo)};
                \node(pulsera) [right of=movil, xshift=4em] {Pulsera(Código, Marca, Modelo)};

                \node(sincroniza) [below of=tiene, yshift=9em, xshift=4em] {Sincroniza(Cód\_Pulsera, Cód\_Móvil, Fecha)};
                \node(actividad) [right of=sincroniza, xshift=5em] {Actividad(Fecha, Cód\_Pulsera, Nombre)};
                \node(dispositivo) [right of=actividad] {Dispositivo(Código)};

                \node(pasos) [below of=sincroniza, yshift=9em, xshift=0em] {Pasos(Cód\_Pulsera, Nombre, Cantidad)};
                \node(ritmo) [right of=pasos, xshift=7em] {Ritmo\_Cardíaco(Cód\_Pulsera, Nombre, Pulsaciones)};
                

                \node[CP, xshift=-17ex] (CP_Usuario) at(usuario) {Email};
                \node[CP, xshift=6ex] (CP_Dispositivo) at(dispositivo) {Código};
                \node[CP, xshift=7ex] (CP_Tiene) at(tiene) {Código};
                \node[CP, xshift=-10ex] (CP_Movil) at(movil) {Código};
                \node[CC, xshift=10ex] (CC_Movil) at(CP_Movil) {Teléfono};
                \node[yshift=-1.4em, xshift=1.4em] at(CC_Movil) {($*$)};
                \node[CP, xshift=-5ex] (CP_Pulsera) at(pulsera) {Código};
                \node[CP, xshift=9ex] (CP_Sincroniza) at(sincroniza) {Cód\_Móvil};
                \node[CP, xshift=9ex] (CP_Actividad) at(actividad) {Cód\_Pulsera, Nombre};
                \node[CC, xshift=-8ex, yshift=-1.4em] (CC_Actividad) at(CP_Actividad) {Fecha, Cód\_Pulsera};
                \node[CP, xshift=-2ex] (CP_Pasos) at(pasos) {Cód\_Pulsera, Nombre};
                \node[CP, xshift=2ex] (CP_Ritmo) at(ritmo) {Cód\_Pulsera, Nombre};
                \node[CP, xshift=-1ex] (CP_Alerta) at(alerta) {Email, Cód\_Pulsera, Nombre};

                \node[CE, xshift=-8.5ex] (CE_Tiene_Email) at(CP_Tiene) {Email};
                \node[CE] (CE_Tiene_Codigo) at(CP_Tiene) {Código};
                \node[CE] (CE_Movil) at(CP_Movil) {Código};
                \node[CE] (CE_Pulsera) at(CP_Pulsera) {Código};
                \node[CE, xshift=-14ex] (CE_Sincroniza_Pulsera) at(CP_Sincroniza) {Cód\_Pulsera};
                \node[CE] (CE_Sincroniza_Movil) at(CP_Sincroniza) {Cód\_Móvil};
                \node[CE, xshift=-4ex] (CE_Actividad) at(CP_Actividad) {Cód\_Pulsera};
                \node[CE] (CE_Pasos) at(CP_Pasos) {Cód\_Pulsera, Nombre};
                \node[CE] (CE_Ritmo) at(CP_Ritmo) {Cód\_Pulsera, Nombre};
                \node[CE, xshift=-13ex] (CE_Alerta_Email) at(CP_Alerta) {Email};
                \node[CE, xshift=4ex] (CE_Alerta_Actividad) at(CP_Alerta) {Cód\_Pulsera, Nombre};

                \node[yshift=1.4em, xshift=1.4em, purple] at(CE_Alerta_Email) {($1$)};
                \node[yshift=1.4em, xshift=1.4em, purple] at(CE_Alerta_Actividad) {($2$)};
                \node[yshift=1.4em, xshift=1.4em, purple] at(CE_Tiene_Email) {($3$)};
                \node[yshift=1.4em, xshift=1.4em, purple] at(CE_Tiene_Codigo) {($4$)};
                \node[yshift=1.4em, xshift=1.4em, purple] at(CE_Movil) {($5$)};
                \node[yshift=1.4em, xshift=1.4em, purple] at(CE_Pulsera) {($6$)};
                \node[yshift=1.4em, xshift=1.4em, purple] at(CE_Sincroniza_Pulsera) {($7$)};
                \node[yshift=1.4em, xshift=1.4em, purple] at(CE_Sincroniza_Movil) {($8$)};
                \node[yshift=1.4em, xshift=1.4em, purple] at(CE_Actividad) {($9$)};
                \node[yshift=1.4em, xshift=1.7em, purple] at(CE_Pasos) {($10$)};
                \node[yshift=1.4em, xshift=1.7em, purple] at(CE_Ritmo) {($11$)};
                
                \node[yshift=-1.5em, xshift=1.8em, purple] at(CP_Usuario) {$(1,3)$};
                \node[yshift=-1.5em, xshift=2.8em, purple] at(CP_Actividad) {$(2,10,11)$};
                \node[yshift=-1.5em, xshift=2.4em, purple] at(CP_Dispositivo) {$(4,5,6)$};
                \node[yshift=-1.5em, xshift=1.8em, purple] at(CP_Pulsera) {$(7,9)$};
                \node[yshift=-1.5em, xshift=1.4em, purple] at(CP_Movil) {$(8)$};
            \end{tikzpicture}
        }
        \caption{Esquema Relacional.}
        \label{fig:esquemaRelacional}
    \end{figure}
\end{ejercicio}

\begin{ejercicio}[$1$ punto] Fusionar aquellas tablas que semánticamente lo permitan.

    La fusión de tablas se muestra en la Figura~\ref{fig:fusion}.
    \begin{figure}
        \centering
        \begin{tikzpicture}
            \node(fusion) {Dispositivo-Tiene(Código, Email)};
            \node[CP, xshift=5ex] (CP_Fusion) at(fusion) {Código};
            \node[CE, xshift=8ex] (CE_Fusion) at(CP_Fusion) {Email};

            \node[yshift=-1.5em, xshift=1.8em, purple] at(CP_Fusion) {$(5,6)$};
            \node[yshift=1.5em, xshift=1.4em, purple] at(CE_Fusion) {$(3)$};

            \node(fusion2) [below of=fusion, yshift=-3em] {Móvil-Sincroniza(Código, Teléfono, Marca, Modelo, Cód\_Pulsera, Fecha)};
            \node[CP, xshift=-15ex] (CP_Fusion2) at(fusion2) {Código};
            \node[CC, xshift=10ex] (CC_Fusion2) at(CP_Fusion2) {Teléfono};
            \node[yshift=-1.4em, xshift=1.4em] at(CC_Fusion2) {$(*)$};
            \node[CE, xshift=37ex] (CE_Fusion2) at(CP_Fusion2) {Cód\_Pulsera};
            \node[CE] (CE_Fusion2_2) at(CP_Fusion2) {Código};
            \node[yshift=1.5em, xshift=1.4em, purple] at(CE_Fusion2) {$(7)$};
            \node[yshift=1.5em, xshift=1.4em, purple] at(CE_Fusion2_2) {$(5)$};
        \end{tikzpicture}
        \caption{Fusión en las tablas de la Figura~\ref{fig:esquemaRelacional}.}
        \label{fig:fusion}
    \end{figure}
\end{ejercicio}

\begin{ejercicio}[$0.5$ puntos] ¿El esquema obtenido permite que un pariente pueda recibir las alertas de la pulsera de su hijo/a? Justifique la respuesta.\\

    Sí, ya que la alerta no se envía al móvil sincronizado, sino a los usuarios que se quieran.
    La única complicación es que esta base de datos no tenemos modelado la relación de parentesco, por lo que no podemos saber si un usuario es pariente de otro.
\end{ejercicio}
\end{document}