\section{Continuidad y límite funcional}

\begin{ejercicio}
    Sean $E$ y $F$ espacios métricos y $f:E\to F$ una función. Probar que $f$ es continua si, y sólo si, $f\left(\ol{A}\right)\subset \ol{f(A)}$ para todo conjunto $A \subset E$.\\

    \begin{description}
        \item [$\Longrightarrow)$] Sea $A\subset E$ y sea $y\in f\left(\overline{A}\right)$, entonces $\exists x\in \overline{A}$ de forma que $y = f(x)$. Sea $V \in \cc{U}(y)$, queremos ver que $V\cap f(A)\neq \emptyset $ para concluir que $y\in \overline{f(A)}$. Para ello, usaremos la definición de que $f$ sea continua en $x$, con lo que $f^{-1}(V)\in \cc{U}(x)$. Ahora, como $x\in \overline{A}$, tenemos que $f^{-1}(V)\cap A \neq \emptyset $, por lo que $\exists z\in f^{-1}(V)\cap A$. De esta forma, tendremos que (basta recordar la definición de $f^{-1}(V)$ y aplicar que $z\in A$) $f(z) \in V\cap f(A)$, como queríamos probar.

            En definitiva, $V\cap f(A) \neq \emptyset $ para todo $V\in \cc{U}(y)$, siendo $y\in f\left(\overline{A}\right)$ arbitrario. Concluimos que $f\left(\overline{A}\right) \subset \overline{f(A)}$.
        \item [$\Longleftarrow)$] Como la propiedad enunciada usa el cierre de un conjunto, trataremos de probar que $f$ es continua probando que la preimagen de todo cerrado es un conjunto cerrado.

            Para ello, si $C\subset F$ es un cerrado ($C=\overline{C}$) y $A = f^{-1}(C)$, queremos ver que $A$ es cerrado. En este caso, tenemos que $f(A)\subset C$, y usando la igualdad:
            \begin{equation*}
                f\left(\overline{A}\right) \subset \overline{f(A)} \subset \overline{C} = C
            \end{equation*}
            Por lo que $\overline{A}\subset f^{-1}(C) = A$, de donde $A = \overline{A}$ y $A$ es un cerrado de $E$, por lo que $f$ es continua.
    \end{description}
\end{ejercicio}

\begin{ejercicio}
    Dado un subconjunto $A$ de un espacio métrico $E$, la función característica de $A$ es la función $\chi_A:E\to \bb{R}$ definida por:
    \begin{equation*}
        \chi_A(x)=1~~\forall x\in A
        \hspace{1cm} \text{y} \hspace{1cm}
        \chi_A(x)=0~~\forall x\in E\setminus A
    \end{equation*}
Probar que $\chi_A$ es continua en un punto $x\in E$ si, y sólo si, $x \in A^\circ \cup (E\setminus A)^\circ$. Deducir que $\chi_A$ es continua si, y sólo si, $A$ es a la vez abierto y cerrado.\\

Si $x\in A^\circ \cup {(E\setminus A)}^{\circ}$, tenemos dos casos:
\begin{itemize}
    \item Si $x\in A^\circ$, entonces $\exists \delta\in \mathbb{R}^+$ de forma que $B(x,\delta)\subset A$, por lo que $\chi_{A|_{B(x,\delta)}}$ es constantemente igual a 1, luego es continua en $x$ y como $B(x,\delta)$ es un abierto, el carácter local de la continuidad nos dice que $\chi_A$ es continua en $x$.
    \item Si $x\in {(E\setminus A)}^{\circ}$, de forma análoga, $\exists \delta\in \mathbb{R}^+$ de forma que $B(x,\delta)\subset E\setminus A$, por lo que $\chi_{A|_{B(x,\delta)}}$ es constantemente igual a 0, luego es continua en $x$ y como $B(x,\delta)$ es un abierto, el carácter local de la continuidad nos dice que $\chi_A$ es continua en $x$.
\end{itemize}
Recíprocamente, probaremos que si $x\in Fr(A)$, entonces $\chi_A$ no es continua en $x$. Para ello, si $x\in Fr(A) = \overline{A}\setminus A^\circ$, distinguimos casos:
\begin{itemize}
    \item Si $x\in A$, entonces tomando $V = \left]0.5,2\right[\in \cc{U}(f(x)) = \cc{U}(1)$, si $\chi_A$ fuese continua en $x$ tendríamos que $U = \chi_A^{-1}(V) \in \cc{U}(x)$, pero $\chi_A^{-1}(V) = A$, por lo que $A\in \cc{U}(x)$, contradicción con que $x\in A\setminus A^{\circ}$.
    \item Si $x\notin A$, entonces tomando $V = \left]-0.5,0.5\right[\in \cc{U}(f(x)) = \cc{U}(0)$, si $\chi_A$ fuese continua en $x$ tendríamos que $U = \chi_A^{-1}(V) = E\setminus A \in \cc{U}(x)$, contradicción con que $x\in Fr(A)$.
\end{itemize}


Una vez probado el enunciado, notemos que $\chi_A$ es continua si y solo si $E = A^\circ \cup {(E\setminus A)}^{\circ}$, es decir, si $Fr(A) = \overline{A}\setminus A^\circ = \emptyset $, si y solo si $\overline{A}=A^\circ$, pero como $A^\circ \subset A \subset \overline{A}$, esto equivale a que $A^\circ = A = \overline{A}$, es decir, que $A$ sea abierto y cerrado.
\end{ejercicio}

\begin{ejercicio}
    Si $E$ y $F$ son espacios métricos, se dice que una función $f:E\to F$ es \emph{localmente constante} cuando, para cada $x\in E$, existe $U \in \cc{U}(x)$ tal que $f\big|_{U}$ es constante. Probar que entonces $f$ es continua. Dar un ejemplo de un conjunto $A \subset \bb{R}$ y una función localmente
    constante $f : A\to \bb{R}$, cuya imagen $f(A)$ sea un conjunto infinito.\\

    \noindent
    Tenemos el resultado pedido aplicando simplemente el carácter local de la continuidad: si $x\in E$, existe $U\in \cc{U}(x)$ de forma que $f\big|_{U}$ es constante, luego $f\big|_{U}$ es continua en $x$. Por el carácter local de la continuidad, como $U\in \cc{U}(x)$, tenemos que $f$ es continua en $x$, para todo $x\in E$. Para el ejemplo pedido, podemos tomar por ejemplo $A = \mathbb{R}\setminus \mathbb{Z}$ y $f = E\big|_{A}$, donde $E$ es la función parte entera.
\end{ejercicio}

\begin{ejercicio}\label{ej:rel3_4}
    Sea $E$ un espacio métrico con distancia $d$ y $A$ un subconjunto no vacío de $E$. Probar la continuidad de la función $f : E \to \bb{R}$ dada por
    \begin{equation*}
        f(x)=d(x,A)\stackrel{\text{def}}{=}\inf\{d(x,a)\mid a \in A\} \qquad \forall x\in E
    \end{equation*}
    Si $x,y\in E$, tenemos que:
    \begin{equation*}
        d(x,a) \leq d(x,y) + d(y,a) \qquad \forall a\in A
    \end{equation*}
    Si consideramos los conjuntos formados por dichos elementos y tomamos ínfimos a ambos lados:
    \begin{equation*}
        d(x,A) = \inf\{d(x,a):a\in A\} \leq \inf\{d(x,y)+d(y,a):a\in A\} = d(x,y)+d(y,A)
    \end{equation*}
    De donde deducimos que $d(x,A) - d(y,A) \leq d(x,y)$, para todo $x,y\in E$. Si repetimos los pasos pero sustituyendo $x$ por $y$:
    \begin{equation*}
        d(y,a) \leq d(y,x) + d(x,a) \qquad \forall a\in A
    \end{equation*}
    De donde:
    \begin{equation*}
        d(y,A) \leq d(y,x) + d(x,A) \Longrightarrow d(y,A) - d(x,A) \leq d(y,x) = d(x,y) \qquad \forall x,y\in E
    \end{equation*}
    En definitiva, hemos probado que:
    \begin{equation*}
        |f(x) - f(y)| = \left|d(x,A)-d(y,A)\right| \leq d(x,y) \qquad \forall x,y\in E
    \end{equation*}
    Lo que claramente implica la continuidad de $f$.
\end{ejercicio}

\begin{ejercicio}
    Sea $E$ un espacio métrico con distancia $d$, y consideremos el espacio producto $E \times E$.
    Probar que, $\forall r \in \bb{R}^+_0$, el conjunto $\{(x,y) \in E\times E \mid d(x,y) < r\}$ es abierto, mientras que $\{(x,y) \in E\times E \mid d(x,y) \leq r\}$ es cerrado.
    En particular se tiene que la diagonal $\Delta(E)=\{(x,x)\mid x\in E\}$ es un conjunto cerrado. Deducir que, si $F$ es otro espacio métrico y $f,g: E \to F$ son funciones continuas, entonces $\{x\in E\mid f(x)=g(x)\}$ es un subconjunto cerrado de $E$ .\\

    \noindent
    Hemos probado en teoría que si $E$ es un espacio métrico, entonces su distancia $d:E\times E\to \mathbb{R}^+_0$ es una función continua. Por tanto:
    \begin{itemize}
        \item Como $[0,r[$ es un abierto ($\left[0,r\right[ = \left]-r,r\right[\cap \mathbb{R}^+_0$) en $\mathbb{R}^+_0$, tenemos que:
            \begin{equation*}
                \{(x,y)\in E \mid d(x,y)< r\} = d^{-1}([0,r[) \text{\ es un abierto}
            \end{equation*}
        \item Como $[0,r]$ es un cerrado en $\mathbb{R}^+_0$ (de hecho lo es en $\mathbb{R}$), tenemos que:
            \begin{equation*}
                \{(x,y)\in E \mid d(x,y)\leq r\} = d^{-1}([0,r]) \text{\ es un cerrado}
            \end{equation*}
    \end{itemize}
    En particular, llamando $A$ al primer conjunto y $B$ al segundo: $\Delta(E) = B\setminus A$, por lo que es cerrado, como intersección de dos cerrados ($B\setminus A = B\cap ((E\times E)\setminus A)$).\\

    \noindent
    Si $f,g:E\to F$ son continuas, si consideramos $\Phi:E\to F\times F$ dada por:
    \begin{equation*}
        \Phi(x) = (f(x), g(x)) \qquad \forall x\in E
    \end{equation*}
    Es decir, $\Phi = (f,g)$, tenemos que $\Phi$ es continua $\Longleftrightarrow f$  y $g$ son continuas, por lo que $\Phi$ es continua, y si consideramos $d \circ \Phi:E\to \mathbb{R}^+_0$ dada por:
    \begin{equation*}
        (d\circ \Phi)(x) = d(\Phi(x)) = d(f(x),g(x)) \qquad \forall x\in E
    \end{equation*}
    Tendremos una función continua, como composición de dos funciones continuas. Finalmente, observamos que $\{0\}\subset \mathbb{R}^+_0$ es un conjunto cerrado, por lo que:
    \begin{equation*}
        {(d\circ \Phi)}^{-1}(\{0\}) = \{x\in E \mid d(f(x),g(x)) = 0\} = \{x\in E\mid f(x) = g(x)\}
    \end{equation*}
    Es un conjunto cerrado.
\end{ejercicio}

\begin{ejercicio}
    Sean $E$, $F$ espacios métricos y $f:E\to F$ una función continua. Probar que su gráfica, es decir, el conjunto $\operatorname{Gr} f = \{(x,f(x))\mid x\in E\}$ es un subconjunto cerrado del espacio métrico producto $E \times F$.\\

    \noindent
    Sean $\Phi,\Psi:E\times F \to E\times F$ dadas por:
    \begin{equation*}
        \Phi(x,y) = (x,y), \qquad \Psi(x,y) = (x,f(x)) \qquad \forall (x,y)\in E\times F
    \end{equation*}
    Es decir, $\Phi$ la función identidad en $E\times F$ y $\Psi$ la que tiene como primera coordenada a $\pi_1$ y en la segunda a $f\circ \pi_1$, tenemos que ambas son continuas ($\Psi$ es continua porque sus dos componentes son continuas). Como $E\times F$ es un espacio métrico con la distancia producto, en el ejercicio anterior demostramos que:
    \begin{equation*}
        \{(x,y)\in E\times F \mid \Phi(x,y) = \Psi(x,y)\}
    \end{equation*}
    Es un conjunto cerrado, pero resulta que:
    \begin{equation*}
        \{(x,y)\in E\times F \mid \Phi(x,y) = \Psi(x,y)\} = \{(x,y) \in E\times F \mid (x,y) = (x,f(x))\} = \operatorname{Gr} f
    \end{equation*}
\end{ejercicio}


\begin{ejercicio}
    Sea $E$ un espacio métrico e $Y$ un espacio pre-hilbertiano. Para $f,g \in \cc{F}(E,Y)$, se define una función $h \in \cc{F}(E)$ por $h(x)=\left(f(x)\mid g(x)\right)$ para todo $x\in E$ . Probar que, si $f,g$ son continuas en un punto $a \in E$ , entonces $h$ también lo es.\\

    \noindent
    En el Ejercicio~\ref{ej:Rel1_1} vimos que se cumplía la identidad de polarización:
    \begin{equation*}
        \langle x,y \rangle  = \dfrac{1}{4}\left({\|x+y\|}^{2} - {\|x-y\|}^{2}\right)
    \end{equation*}
    Propiedad que usaremos para probar primero que $\langle \cdot,\cdot \rangle:Y\times Y\to \mathbb{R} $ es continua, como resultado de la composición que ilustramos:
    \begin{equation*}
        (x,y) \longmapsto (x+y,x-y) \longmapsto \left({\|x+y\|}^{2}, {\|x-y\|}^{y}\right) \longmapsto \dfrac{1}{4}\left({\|x+y\|}^{2}-{\|x-y\|}^{2} \right)
    \end{equation*}
    Donde:
    \begin{itemize}
        \item La primera es $\Phi_1:Y\times Y \to Y\times Y$, dada por:
            \begin{equation*}
                \Phi_1(x,y) = (x+y,x-y)
            \end{equation*}
            Que es continua, porque sus dos componentes lo son.
        \item La segunda es $\Phi_2:Y\times Y \to \mathbb{R}\times \mathbb{R}$, dada por:
            \begin{equation*}
                \Phi_2(x,y) = \left({\|x\|}^{2}, {\|y\|}^{2}\right)
            \end{equation*}
            Que es continua, porque sus dos componentes son composición de dos funciones continuas.
        \item La tercera es $\Phi_3:\mathbb{R}\times \mathbb{R}\to \mathbb{R}$ dada por:
            \begin{equation*}
                \Phi_3(x,y) = \dfrac{1}{4}(x-y)
            \end{equation*}
            Que también es continua, como composición de funciones continuas.
    \end{itemize}
    Finalmente, si $f$ y $g$ son continuas en $a\in E$ y tomamos $\Phi = (f,g)$ obtenemos una función continua en $a\in E$, que nos permite escribir:
    \begin{equation*}
        h = \langle \cdot,\cdot \rangle  \circ \Phi
    \end{equation*}
    Por lo que $h$ es continua en $a\in E$.
\end{ejercicio}


\begin{ejercicio}
    Sea $E$ un espacio métrico y $f,g : E \to \bb{R}$ funciones continuas en un punto $a\in E$ . Probar que la función $h : E \to \bb{R}$ definida por $h(x) = \max \{f(x), g(x)\}$ para todo $x\in E$, también es continua en $a$.\\

    \noindent
    La clave es expresar $h$ de la siguiente manera:
    $$h(x)=\max\{f(x),g(x)\}=\frac{1}{2}\left(f(x)+g(x)+|f(x)-g(x)|\right) \qquad \forall x\in E$$
    Como $f,g$ son continuas en $a$, entonces $f+g$ y $f-g$ también lo son. Además, el valor absoluto es una función continua en $\bb{R}$,
    por lo que $|f-g|$ es continua en $a$. Por tanto, $h$ es continua en $a$.
\end{ejercicio}


\begin{ejercicio}
    Probar que si $Y$ es un espacio normado, $E$ un espacio métrico y $f:E \to Y$ una función continua en un punto $a \in E$ , entonces la función $g: E \to \bb{R}$ definida por $g(x)=\|f(x)\|$ para todo $x\in E$ , también es continua en el punto $a$.\\

    \noindent
    Tenemos que $g = \|\cdot\| \circ f$ y como $f$ es continua en $a$ y $\|\cdot\|$ es continua en $f(a)$ (ya que es continua en todo $Y$), tenemos que $g$ es continua en $a$.
\end{ejercicio}

\begin{ejercicio}
    Probar las siguientes igualdades:
    \begin{enumerate}
        \item $\displaystyle \lim_{(x,y)\to (0,0)} \frac{\sen(x^2 + y^2)}{x^2+y^2} = 1$.

            Observando el límite, vemos que se parece a uno bien conocido:
            \begin{equation*}
                \lim_{x\to0}\dfrac{\sen x}{x} = 1
            \end{equation*}
            Por tanto, lo que haremos será intentar aplicar el resultado visto en teoría sobre el límite de una composición de funciones (conviene tener el resultado delante para comprobar todas las hipótesis). De esta forma, si definimos $f:\mathbb{R}^\ast\to \mathbb{R}$ dada por:
            \begin{equation*}
                f(x) = \dfrac{\sen x}{x} \qquad \forall x\in \mathbb{R}^\ast
            \end{equation*}
            Y definimos también $\varphi:\mathbb{R}\times \mathbb{R}\to \mathbb{R}$ dada por:
            \begin{equation*}
                \varphi(x,y) = x^2+y^2 \qquad \forall (x,y)\in \mathbb{R}^2
            \end{equation*}
            Tenemos que $\varphi$ verifica las dos hipótesis que ha de cumplir, puesto que:
            \begin{gather*}
                \lim_{(x,y)\to(0,0)}\varphi(x,y) = \lim_{(x,y)\to(0,0)}x^2+y^2=  0 \\
                \varphi(x,y) = x^2+y^2 = 0 \Longleftrightarrow (x,y) = (0,0)
            \end{gather*}
            Como veníamos anunciando, como tenemos que:
            \begin{equation*}
                \lim_{x\to0}f(x) = \lim_{x\to0}\dfrac{\sen x}{x} = 1
            \end{equation*}
            Concluimos que:
            \begin{equation*}
                \lim_{(x,y)\to(0,0)} f(\varphi(x,y)) = \lim_{(x,y)\to(0,0)}\dfrac{\sen(x^2+y^2)}{x^2+y^2} = 1
            \end{equation*}
        \item $\displaystyle \lim_{(x,y)\to (0,0)} \frac{\log(1+x^4 + y^4)}{x^4+y^4} = 1$.

            Procedemos de forma análoga, en primer lugar definimos $f:\mathbb{R}^+ \to \mathbb{R}$ dada por:
            \begin{equation*}
                f(x) = \dfrac{\log(1+x)}{x} \qquad \forall x\in \mathbb{R}^+
            \end{equation*}
            Y definimos $\varphi:\mathbb{R}\times \mathbb{R} \to \mathbb{R}$ dada por:
            \begin{equation*}
                \varphi(x,y) = x^4+y^4 \qquad \forall (x,y)\in \mathbb{R}^2
            \end{equation*}
            Como se cumple que:
            \begin{gather*}
                \lim_{(x,y)\to(0,0)}\varphi(x,y) = \lim_{(x,y)\to(0,0)}x^4+y^4 =  0 \\
                \varphi(x,y) = x^4+y^4 > 0 \Longleftrightarrow (x,y) \neq (0,0) 
            \end{gather*}
            Y es bien sabido que:
            \begin{equation*}
                \lim_{x\to0}f(x) = \lim_{x\to0}\dfrac{\log(1+x)}{x} = 1
            \end{equation*}
            Concluimos que:
            \begin{equation*}
                \lim_{(x,y)\to(0,0)}f(\varphi(x,y)) = \lim_{(x,y)\to(0,0)}\dfrac{\log(1+x^4+y^4)}{x^4+y^4} = 1
            \end{equation*}
    \end{enumerate}
\end{ejercicio}
