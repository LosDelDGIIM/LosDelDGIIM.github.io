\section{Vector derivada}

\begin{ejercicio}
Sea \( Y \) un espacio pre-hilbertiano, \( \Omega \) un abierto de \( \mathbb{R} \) y \( f,g : \Omega \rightarrow Y \) dos funciones diferenciables en un punto \( a \in \Omega \). Probar que la función \( \varphi : \Omega \rightarrow \mathbb{R} \) definida por
\[
\varphi(t) = \left( f(t) | g(t) \right) \quad \forall t \in \Omega
\]
es derivable en el punto \( a \) y calcular su derivada \( \varphi'(a) \).\\

\noindent
Si $(Y, \langle \cdot,\cdot \rangle )$ es un espacio pre-hilbertiano, probaremos que $\varphi$ es derivable en el punto $a$ definiendo $\Phi=(f,g):\Omega\to Y^2$ dada por:
\begin{equation*}
    \Phi(t) = (f(t), g(t))\qquad \forall t\in \Omega
\end{equation*}
Y definiendo $\Psi:Y^2\to \mathbb{R}$ dada por:
\begin{equation*}
    \Psi(x,y) = \langle x,y \rangle  \qquad \forall (x,y)\in Y^2
\end{equation*}
Con lo que $\varphi = \Psi \circ \Phi$.
\begin{description}
    \item [Diferenciabilidad de $\Phi$.] Como tanto $f$ como $g$ son diferenciables en $a$, tenemos que $\Phi = (f,g)$ es diferenciable en $a$, con:
        \begin{equation*}
            \Phi'(a) = \left(f'(a), g'(a)\right)
        \end{equation*}
        Por lo que $D\Phi(a)(t) = t\Phi'(a)$ $\forall t\in \mathbb{R}$.
    \item [Diferenciabilidad de $\Psi$.] Fijado $(x,y)\in Y^2$, si tomamos $T:Y^2\to \mathbb{R}$ dada por:
        \begin{equation*}
            T(u,v) = \langle x,v \rangle  + \langle u,y \rangle  \qquad \forall (u,v)\in Y^2
        \end{equation*}
        Tendremos que (como trabajamos en $Y\times Y$, consideramos la norma producto):
        \begin{align*}
            &\dfrac{\Psi((x,y)+(u,v)) - \Psi(x,y) - T(u,v)}{\|(u,v)\|_\infty} = \dfrac{\langle x+u,y+v \rangle - \langle x,y \rangle - \langle x,v \rangle - \langle u,y \rangle }{\|(u,v)\|_\infty} \\ &= \dfrac{\langle u,v \rangle }{\max \{\|u\|, \|v\|\}} \stackrel{(\ast)}{\leq} \dfrac{\|u\|\|v\|}{\max\{\|u\|,\|v\|\}} = \min\{\|u\|,\|v\|\} \qquad \forall (u,v)\in Y\times Y
        \end{align*}
        Donde en $(\ast)$ hemos usado la desigualdad de Cauchy-Schwartz. En definitiva, tenemos que:
        \begin{equation*}
            \lim_{(u,v)\to(0,0)}\dfrac{\Psi((x,y)+(u,v))-\Psi(x,y)-T(u,v)}{\|(u,v)\|_\infty} = \lim_{(u,v)\to(0,0)}\min\{\|u\|,\|v\|\} = 0
        \end{equation*}
        Por lo que $\Psi$ es diferenciable en todo punto $(x,y)\in Y^2$, con $D\Psi(x,y):Y^2\to \mathbb{R}$ dada por:
        \begin{equation*}
            D\Psi(x,y)(u,v) = \langle x,v \rangle + \langle u,y \rangle \qquad \forall (u,v)\in Y\times Y
        \end{equation*}
\end{description}
Finalmente, tenemos por la regla de la cadena que $\varphi = \Psi\circ\Phi$ es diferenciable en $a$, con:
\begin{equation*}
    D\varphi(a) = D\Psi(\Phi(a)) \circ D\Phi(a)
\end{equation*}
Por tanto, $\varphi$ es derivable en $a$, y tenemos que:
\begin{align*}
    D\varphi(a)(h) &= D\Psi(\Phi(a))\left(D\Phi(a)(h)\right) = D\Psi(\Phi(a))\left(h\Phi'(a)\right) = D\Psi(f(a),g(a))\left(hf'(a),hg'(a)\right) \\
                   &= \langle f(a),hg'(a) \rangle  + \langle g(a),hf'(a) \rangle = h\left[\langle f(a),g'(a) \rangle + \langle g(a),f'(a) \rangle \right] \qquad \forall h\in \mathbb{R}
\end{align*}
Por lo que finalmente obtenemos:
\begin{equation*}
    \varphi'(a) = D\varphi(a)(1) = \langle f(a),g'(a) \rangle + \langle g(a),f'(a) \rangle 
\end{equation*}
\end{ejercicio}

\begin{ejercicio}
Sean \( Y,Z \) espacios normados y consideremos dos funciones $f,g$, \( f: \Omega \rightarrow Y \) y \( g : U \rightarrow Z \), donde \( \Omega \) es un abierto no vacío de \( \mathbb{R} \) y \( U \) es un abierto no vacío de \( Y \). Supongamos que \( f \) es derivable en un punto \( a \in \Omega \) y que \( g \) es diferenciable en el punto \( b = f(a) \). Calcular el vector derivada de \( g \circ f \) en el punto \( a \).\\

\noindent
Aplicando directamente la regla de la cadena obtenemos que $g\circ f$ es diferenciable en $a$, luego derivable en $a$, con:
\begin{equation*}
    D(g\circ f)(a) = Dg(f(a)) \circ Df(a)
\end{equation*}
Por lo que:
\begin{multline*}
    D(g\circ f)(a)(h) = Dg(f(a))(Df(a)(h)) = Dg(f(a))(hf'(a)) = h\cdot Dg(f(a))(f'(a)) \\ \forall h\in \mathbb{R}
\end{multline*}
De donde:
\begin{equation*}
    (g\circ f)'(a) = D(g\circ f)(a)(1) = Dg(f(a))(f'(a))
\end{equation*}
\end{ejercicio}

\begin{ejercicio}
Sean \( X,Z \) espacios normados, \( \Omega \) un abierto no vacío de \( X \) y \( U \) un abierto no vacío de \( \mathbb{R} \). Sea \( f : \Omega \rightarrow U \) una función diferenciable en un punto \( a \in \Omega \) y \( g : U \rightarrow Z \) una función derivable en el punto \( b = f(a) \). Calcular la diferencial de \( g \circ f \) en \( a \).\\ 

\noindent
Aplicando la regla de la cadena obtenemos que $g\circ f$ es diferenciable en $a$, luego derivable en $a$, con:
\begin{equation*}
    D(g\circ f)(a) = Dg(f(a)) \circ Df(a)
\end{equation*}
Por lo que:
\begin{multline*}
    D(g\circ f)(a)(h) = Dg(f(a))(Df(a)(h)) = Df(a)(h)\cdot g'(f(a)) = h\cdot Df(a)(1)\cdot g'(f(a))\\ \forall h\in \mathbb{R}
\end{multline*}
\end{ejercicio}

\begin{ejercicio}
Probar que, dados \( a,b \in \mathbb{R}^+ \), la hipérbola \[ \{(x,y) \in \mathbb{R}^2 : b^2x^2 - a^2y^2 = a^2b^2\} \] no es una curva paramétrica, pero cada una de sus ramas es una curva explícita.
% // TODO: Toda curva paramétrica es conexa por ser la imagen de un conexo por una funcion continua
\end{ejercicio}

\begin{ejercicio}
Describir geométricamente las curvas cuyas ecuaciones paramétricas son:
\begin{enumerate}
    \item \( x = e^t \), \( y = e^{-t} \) \qquad \( \forall t \in \mathbb{R} \)
    \item \( x = \cosh t \), \( y = \senh t \) \qquad \( \forall t \in \mathbb{R} \)
\end{enumerate}
\end{ejercicio}
    
