\section{Complitud y continuidad uniforme}

    \begin{ejercicio}
    Probar que, en cualquier espacio métrico, toda sucesión de Cauchy está acotada.\\

    \noindent
    Sea $\{x_n\}$ una sucesión de Cauchy de un espacio métrico $(E,d)$, para $\veps = 1~\exists m\in \mathbb{N}$ de forma que si $p,q\geq m$ entonces $d(x_p,x_q)<1$. Podemos escribir:
    \begin{equation*}
        \{x_n : n\in \mathbb{N}\} = \{x_n : n \leq m\} \cup \{x_n : n > m\}
    \end{equation*}
    Y tenemos que:
    \begin{itemize}
        \item $\{x_n : n\leq m\}$ es un conjunto finito de $m$ elementos, luego es un conjunto acotado.
        \item Si $x_n \in \{x_n : n > m\}$, entonces tomando $k = n-m \in \mathbb{N}$ tenemos que $m,m+k\geq m$, por lo que:
            \begin{equation*}
                d(x_m, x_{m+k}) = d(x_m, x_n) < 1
            \end{equation*}
            Luego $x_n\in B(x_m, 1)$. En definitiva, tenemos que:
            \begin{equation*}
                \{x_n : n > m\} \subset B(x_m, 1)
            \end{equation*}
            Por lo que también es un conjunto acotado.
    \end{itemize}
    Finalmente, tenemos que $\{x_n : n\in \mathbb{N}\}$ es unión de dos conjuntos acotados, luego también estará acotado, es decir, la sucesión $\{x_n\}$ está acotada.
    \end{ejercicio}
    
    \begin{ejercicio}
    Probar que todo espacio métrico compacto es completo.\\

    \noindent
    Sea $(E,d)$ un espaciom métrico compacto, para ver que es completo tomamos una sucesión de Cauchy $\{x_n\}$ y queremos ver que es convergente. Como el espcio es compacto, existirá una parcial $\{x_{\sigma(n)}\}$ convergente a cierto $x\in E$. En dicho caso, dado $\veps > 0$:
    \begin{itemize}
        \item Como $\{x_n\}$ es de Cauchy, $\exists m_1\in \mathbb{N}$ de forma que si $p,q \geq m_1$ entonces $d(x_p,x_q)<\nicefrac{\veps}{2}$.
        \item Como $\{x_{\sigma(n)}\}\to x$, $\exists m_2 \in \mathbb{N}$ de forma que si $n\geq m_2$ entonces $d(x_{\sigma(n)},x) < \nicefrac{\veps}{2}$.
    \end{itemize}
    Tomando $m = \max \{m_1,m_2\}$, si tomamos $n\geq m$, tenemos que $\sigma(n)\geq n \geq m$, con lo que:
    \begin{equation*}
        d(x_n,x) \leq d(x_n,x_{\sigma(n)}) + d(x_{\sigma(n)}, x) < \dfrac{\veps}{2}+ \dfrac{\veps}{2} = \veps
    \end{equation*}
    Lo que implica que $\{x_n\} \to x$.\\

    \noindent
    En general, si $\{x_n\}$ es de Cauchy y tenemos una parcial convergente, la sucesión será convergente al mismo límite que su parcial.
    \end{ejercicio}
    
    \begin{ejercicio}
    Sea \( f : \mathbb{R} \rightarrow \mathbb{R} \) una función continua e inyectiva. Probar que definiendo
    \[
    \rho(x,y) = |f(x) - f(y)| \qquad \forall x, y \in \mathbb{R}
    \]
    se obtiene una distancia en \( \mathbb{R} \), equivalente a la usual. ¿Cuándo es \( \rho \) completa?\\

    \noindent
    Tenemos mucho trabajo por hacer:
    \begin{description}
        \item [Probar que $\rho$ es una distancia en $\bb{R}$.] Para ello, hemos de probar:
            \begin{description}
                \item [D1)] $\rho(x,z) = |f(x) - f(z)| \leq |f(x) - f(y)| + |f(y) - f(z)| = \rho(x,y) + \rho(y,z)$ $\forall x,y,z\in \mathbb{R}$.
                \item [D2)] $\rho(x,y) = |f(x) - f(y)| = |f(y) - f(x)| = \rho(y,x)$ $\forall x,y\in \mathbb{R}$.
                \item [D3)] $|f(x) - f(y)| = \rho(x,y) = 0 \Longleftrightarrow f(x) = f(y) \Longleftrightarrow x=y$.
            \end{description}
        \item [Probar que $\rho$ es equivalente a la distancia usual de $\bb{R}$.] Para ello, si $d$ es la distancia usual de $\mathbb{R}$:
            \begin{equation*}
                d(x,y) = |x-y| \qquad \forall x,y\in \mathbb{R}
            \end{equation*}
            Tenemos que:
            \begin{itemize}
                \item Si $\{d(x_n,x)\}\to 0$, como $f$ es continua, tenemos que:
                    \begin{equation*}
                        \{\rho(x_n,x)\} = \{d(f(x_n),f(x))\} \to 0
                    \end{equation*}
                \item Si $\{d(f(x_n),f(x))\} = \{\rho(x_n,x)\} \to 0$, como $f^{-1}$ es continua por ser $f$ continua, inyectiva y estar definida en un intervalo, tenemos que:
                    \begin{equation*}
                        \{\rho(f^{-1}(x_n), f^{-1}(x) )\} = \{d(f^{-1}(f(x_n)), f^{-1}(f(x)))\} = \{d(x_n,x)\} \to 0
                    \end{equation*}
            \end{itemize}
            Y como vimos que la convergencia de sucesiones caracteriza la topología de los espacios métricos, tenemos que $d$ y $\rho$ son distancias equivalentes.
        \item [Caracterizar cuándo $\rho$ es una distancia completa.] Si escribimos la definición de que $\{x_n\}$ sea convergente a $x\in \mathbb{R}$ para $\rho$ o que sea de Cauchy para $\rho$:
            \begin{gather*}
                \forall \veps > 0~\exists m\in \mathbb{N}:n\geq m \Longrightarrow |f(x_n) - f(x)| = \rho(x_n,x) < \veps \\
                \forall \veps > 0~\exists m\in \mathbb{N}:p,q\geq m \Longrightarrow |f(x_p) - f(x_q)| = \rho(x_p,x_q) < \veps
            \end{gather*}
            Vemos que obteneos las definiciones de que $\{f(x_n)\}\to f(x)$ y de que $\{f(x_n)\}$ sea de Cauchy para la distancia usual de $\mathbb{R}$. Por tanto, podemos sospechar a partir de lo visto en teoría que:
            \begin{equation*}
                (\mathbb{R}, \rho) \text{\ es completo} \Longleftrightarrow f(\mathbb{R}) \text{\ es cerrado}
            \end{equation*}
            \begin{description}
                \item [$\Longleftarrow)$] Sea $\{x_n\}$ una sucesión de Cauchy para $\rho$, entonces $\{f(x_n)\}$ es una sucesión de Cauchy para la distancia usual, y como $f(\mathbb{R})$ es cerrado y es subconjunto de un espacio completo ($\mathbb{R}$), tenemos que $f(\mathbb{R})$ es completo, con lo que existe $y\in f(\mathbb{R})$ de forma que $\{f(x_n)\} \to y$, y existirá $x\in \mathbb{R}$ de forma que $y = f(x)$. Que $\{f(x_n)\}\to f(x)$ significa que $\{x_n\}$ converge a $x$ para $\rho$, por lo que toda sucesión de Cauchy en $(\mathbb{R},\rho)$ es convergente, como queríamos probar.
                \item [$\Longrightarrow)$] Si $\{y_n\}\to y$ con $y_n\in f(\mathbb{R})$ para todo $n\in \mathbb{N}$, entonces $\{y_n\}$ es de Cauchy. Por ser $y_n \in f(\mathbb{R})$ para todo $n\in \mathbb{N}$, existe $x_n\in \mathbb{R}$ de forma que $y_n = f(x_n)$ $\forall n\in \mathbb{N}$, con lo que $\{x_n\}$ es de Cauchy para $\rho$. Como $(\mathbb{R},\rho)$ es completo, existe $x\in \mathbb{R}$ de forma que $\{x_n\}$ converge a $x$ para $\rho$, es decir, que $\{f(x_n)\} \to f(x)$. Como tenemos que:
                    \begin{equation*}
                        y \leftarrow \{y_n\} = \{f(x_n)\} \to f(x)
                    \end{equation*}
                    Deducimos que $y = f(x)$, luego $y\in f(\mathbb{R})$; con lo que $f(\mathbb{R})$ es un conjunto cerrado.
            \end{description}
    \end{description}
    \end{ejercicio}
    
    \begin{ejercicio}
    ¿Qué se puede afirmar sobre la composición de dos funciones uniformemente continuas?\\

    \noindent
    Que la composición es uniformemente continua. Formalmente, si $f:X\to Y$ y $g:Y\to Z$ son dos funciones uniformemente continuas, entonces $g\circ f$ es una función uniformemente continua. Para verlo, dado $\veps > 0$, la continuidad uniforme de $g$ nos da $\delta>0$ de modo que:
    \begin{equation*}
        x,y\in Y \text{\ con\ } d(x,y) < \delta \Longrightarrow d(g(x),g(y)) < \veps
    \end{equation*}
    Para $\delta$, la continuidad uniforme de de $f$ nos da $\eta > 0$ de modo que:
    \begin{equation*}
        u,v\in X \text{\ con\ } d(u,v) < \eta \Longrightarrow d(f(u),f(v)) < \delta
    \end{equation*}
    Por tanto, dado $\veps > 0$ hemos encontrado $\eta>0$ de modo que si $u,v\in X$ con $d(u,v) < \eta$, entonces:
    \begin{equation*}
        d(f(u), f(v)) < \delta \Longrightarrow d((g\circ f)(u), (g\circ f)(v)) = d(g(f(u)), g(f(v))) < \veps
    \end{equation*}
    \end{ejercicio}
    
    \begin{ejercicio}
    Dado un espacio normado \( X \neq \{0\} \), probar que la función
    \Func{f}{X\setminus \{0\}}{X}{x}{\dfrac{x}{\|x\|}}
    no es uniformemente continua. Sin embargo, probar también que, para cada \( \delta \in \mathbb{R}^+ \), la restricción de \( f \) al conjunto \( \{ x \in X : \|x\| \geq \delta \} \) es una función lipschitziana.\\

    \noindent
    $f$ no es uniformemente continua, ya que podemos considerar $\{\nicefrac{1}{n}\}$, $\{\nicefrac{-1}{n}\}$, dos sucesiones de forma que:
    \begin{equation*}
        \left\{d\left(\dfrac{1}{n}, \dfrac{-1}{n}\right) \right\}\to 0, \qquad \left\{d\left(f\left(\dfrac{1}{n}\right), f\left(\dfrac{-1}{n}\right)\right)\right\} = \{d(1,-1)\} 
    \end{equation*}
    Para ver que la restricción mencionada es lipschitziana, observemos que si tomamos $x,y\in \{x\in X:\|x\| \geq \delta\}$, tenemos entonces que $\nicefrac{1}{\|x\|}\leq \nicefrac{1}{\delta}$, y si usamos la desigualdad que conseguimos en el Ejercicio~\ref{ej:rel4_11}, tenemos que:
    \begin{equation*}
        \|f(x) - f(y)\| \leq \dfrac{2}{\|x\|}\|x-y\| \leq \dfrac{2}{\delta}\|x-y\|
    \end{equation*}
    Por tanto, tomando $M = \nicefrac{2}{\delta}$, tenemos que $f$ restringida a dicho conjunto es lipschitziana.
    \end{ejercicio}
    
    \begin{ejercicio}
    Sea \( A \) un subconjunto no vacío de un espacio métrico \( E \) y la siguiente función:
    \Func{f}{E}{\mathbb{R}}{x}{\inf \{ d(x,a) : a \in A \}}
    Probar que \( f \) es no expansiva.\\

    \noindent
    Si recordamos el Ejercicio~\ref{ej:rel3_4}, en este demostramos que:
    \begin{equation*}
        |f(x) - f(y)| = |d(x,A) - d(y,A)| \leq d(x,y) \qquad \forall x,y\in E
    \end{equation*}
    Por tanto, $f$ es lipschitziana, con constante de lipschitz menor o igual que 1, es decir, $f$ es no expansiva.
    \end{ejercicio}
    
    \begin{ejercicio}
    Dado \( y \in \mathbb{R}^N \), se define \( T_y \in L(\mathbb{R}^N, \mathbb{R}) \) usando el producto escalar en \( \mathbb{R}^N \):
    \[
    T_y(x) = \left( x | y \right) \hspace{1cm} \forall x \in \mathbb{R}^N 
    \]
    Calcular la norma de la aplicación lineal \( T_y \), considerando en \( \mathbb{R}^N \):
    \begin{enumerate}
        \item La norma euclídea.

            Usando la desigualdad de Cauchy-Schwarz, tenemos que:
            \begin{equation*}
                \|T_y(x)\|_2 = |\langle x,y \rangle| \leq \|x\|_2\|y\|_2 \qquad \forall x\in \mathbb{R}^N
            \end{equation*}
            De donde deducimos que $\|T_y\| \leq \|y\|_2$. Sin embargo, como:
            \begin{equation*}
                \|T_y(y)\|_2 = \langle y,y \rangle  = \|y\|_2^2
            \end{equation*}
            Concluimos que $\|T_y\| = \|y\|_2$.
        \item La norma del máximo.

            Hemos visto en teoría que:
            \begin{equation*}
                \|T_y\| = \max \{\|T_y(x)\| : x \in X \text{\ con\ } \|x\|_\infty = 1\} = \max\left\{\sum_{i \in \Delta_N} x_iy_i : \|x\|_\infty = 1\right\}
            \end{equation*}
            Por tanto, la forma de maximizar el producto es tomando:
            \begin{equation*}
                u_i = sgn(y_i) = \left\{\begin{array}{ll}
                        1 & \text{si\ } y_i > 0 \\
                        -1 & \text{si\ } y_i < 0 \\
                        0 & \text{si\ } y_i = 0
                \end{array}\right.
            \end{equation*}
                Para así tener $\|u\|_\infty = 1$, con lo que al final obtenemos:
                \begin{equation*}
                    \|T_y\| = \sum_{i \in \Delta_n} x_i y_i = \sum_{i \in \Delta_N} |y_i| = \|y\|_1
                \end{equation*}
        \item La norma de la suma.

            De forma análoga, hemos visto que:
            \begin{equation*}
                \|T_y\| = \max \{\|T_y(x)\| : x \in X \text{\ con\ } \|x\|_1 = 1\} = \max\left\{\sum_{i \in \Delta_N} x_iy_i : \|x\|_1 = 1\right\}
            \end{equation*}
            Y la forma de maximiar el producto es tomando:
            \begin{align*}
                u_k = 1 &\text{\ si\ } |y_k| = \max_{i \in \Delta_N} \{|y_i|\} \\
                u_i = 0 &\text{\ si\ } i\neq k
            \end{align*}
            Para así tener $\|u\|_\infty = 1$, con lo que obtenemos:
            \begin{equation*}
                \|T_y\| = \sum_{i \in \Delta_N}x_iy_i = |y_k| = \|y\|_\infty
            \end{equation*}
    \end{enumerate}
    \end{ejercicio}
    
    \begin{ejercicio}
    Consideremos los espacios normados \( X = \mathbb{R}^N \) con la norma de la suma y \( Y = \mathbb{R}^N \) con la norma del máximo. Denotando como siempre por \( \{ e_k : k \in \mathbb{N} \} \) a la base usual de \( \mathbb{R}^N \), probar que, para toda \( T \in L(X,Y) \), se tiene:
    \[
    \|T\| = \max \{ |\left(T(e_j)\mid e_k \right)| : j,k \in \Delta_n \}
    \]
    % // TODO: Hacer
    \end{ejercicio}
    
