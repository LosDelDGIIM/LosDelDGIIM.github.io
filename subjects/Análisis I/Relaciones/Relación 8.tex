\section{Vector gradiente}

\begin{ejercicio}
    Calcular todas las derivadas direccionales en el punto $(-1,0,0)$ de la función $f:\bb{R}^3\to \bb{R}$ definida por
    \begin{equation*}
        f(x,y,z) = x^3  -3xy + z^3 \qquad \forall (x,y,z)\in \bb{R}^3
    \end{equation*}

    \noindent
    Sea $u=(u_1,u_2,u_3)\in S$ una dirección y $a = (-1,0,0)$, tenemos que:
    \begin{align*}
        \dfrac{f(a+tu)-f(a)}{t} &= \dfrac{f(-1+tu_1, tu_2, tu_3) - f(-1,0,0)}{t} \\ &= \dfrac{{(-1+tu_1)}^{3}-3tu_2(-1+tu_1) + t^3u_3^3 + 1}{t} \\
                                &= \dfrac{t^3u_1^3 - 3t^2u_1^2 + 3tu_1 \cancel{- 1} + 3tu_2 - 3t^2u_2u_1 + t^3u_3^3 \cancel{+ 1}}{t} \\
                                &= t^2u_1^3 - 3tu_1^2 + 3u_1 + 3u_2 - 3tu_2u_1 + t^2u_3^3 \qquad \forall t\in \mathbb{R}^\ast
    \end{align*}
    Por lo que:
    \begin{multline*}
        f_u'(-1,0,0) = \lim_{t\to0}\dfrac{f(a+tu)-f(a)}{t} = \lim_{t\to0}\left(t^2u_1^3 - 3tu_1^2 + 3u_1 + 3u_2 - 3tu_2u_1 + t^2u_3^3\right) \\ = 3(u_1 + u_2) \qquad \forall u\in S
    \end{multline*}
\end{ejercicio}

\begin{ejercicio}
    Sea $J$ un intervalo abierto en $\bb{R}$ y $\Omega$ un subconjunto abierto de $\bb{R}^N$. Si $f:J\to \Omega$ es una función derivable en un punto $a\in J$ y $g:\Omega\to \bb{R}$ es diferenciable en el punto $b=f(a)$, probar que la función $h:g\circ f:J\to \bb{R}$ es derivable en punto $a$, con:
    \begin{equation*}
        h'(a) = \langle \nabla g(b), f'(a) \rangle 
    \end{equation*}

    \noindent
    Como $f$ es derivable en $a$, tenemos que $f$ es diferenciable en $a$, con:
    \begin{equation*}
        Df(a)(x) = f'(a)x \qquad \forall x\in \mathbb{R}
    \end{equation*}
    Aplicando la regla de la cadena, tenemos que $h$ es diferenciable en $a$, con:
    \begin{equation*}
        Dh(a) = Dg(b)\circ Df(a) 
    \end{equation*}
    Además, como $g$ es diferenciable en $b$, es parcialmente derivable en $b$, con:
    \begin{equation*}
        Dg(b)(x) = \langle \nabla g(b), x \rangle  \qquad \forall x\in \mathbb{R}^N
    \end{equation*}
    Por tanto:
    \begin{equation*}
        Dh(a)(x) = Dg(b)(Df(a)(x)) = Dg(b)(f'(a)x) = \langle \nabla g(a), f'(a)x \rangle  \qquad \forall x\in \mathbb{R}
    \end{equation*}
    Y finalmente tenemos que:
    \begin{equation*}
        h'(a) = Dh(a)(1) = \langle \nabla g(a), f'(a) \rangle 
    \end{equation*}
\end{ejercicio}

\begin{ejercicio}
    Sea $\Omega$ un subconjunto abierto de $\mathbb{R}^N$ y $f,g:\Omega\to\mathbb{R}$ funciones parcialmente derivables en un punto $a\in \Omega$. Probar que las funciones $f+g$ y $fg$ son parcialmente derivables en $a$ con
    \begin{align*}
        \nabla (f+g)(a) &= \nabla f(a) + \nabla g(a) \\
        \nabla(fg)(a) &= g(a)\nabla f(a) + f(a) \nabla g(a)
    \end{align*}
    Suponiendo que $g(x)\neq 0$ para todo $x\in \Omega$, probar también que $\nicefrac{f}{g}$ es parcialmente derivable en $a$ con
    \begin{equation*}
        \nabla(\nicefrac{f}{g})(a) = \dfrac{g(a) \nabla f(a) - f(a) \nabla g(a)}{{g(a)}^{2}}
    \end{equation*}

    \noindent
    Las demostraciones son análogas a las que siempre se hacen para las reglas de derivación. Para $f+g$, como:
    \begin{equation*}
        \dfrac{(f+g)(a+te_k) - (f+g)(a)}{t} = \dfrac{f(a+te_k)-f(a)}{t}+\dfrac{g(a+te_k)-g(a)}{t} \quad \forall t\in \mathbb{R}^\ast, \forall k\in \Delta_N
    \end{equation*}
    Tenemos entonces que:
    \begin{equation*}
        \dfrac{\partial (f+g)}{\partial x_k}(a) = \dfrac{\partial f}{\partial x_k}(a) + \dfrac{\partial g}{\partial x_k}(a) \qquad \forall k\in \Delta_N
    \end{equation*}
    De donde $\nabla(f+g)(a) = \nabla f(a) + \nabla g(a)$. Para $fg$:
    \begin{align*}
        \dfrac{(fg)(a+te_k) - (fg)(a)}{t} &= \dfrac{f(a+te_k)g(a+te_k) - g(a+te_k)f(a) + g(a+te_k)f(a) - f(a)g(a)}{t} \\
                                          &= g(a+te_k)\cdot \dfrac{f(a+te_k)-f(a)}{t} + f(a) \cdot \dfrac{g(a+te_k)-g(a)}{t} \\ &\forall t\in \mathbb{R}^\ast, \forall k\in \Delta_N
    \end{align*}
    Tenemos entonces que:
    \begin{equation*}
        \dfrac{\partial (fg)}{\partial x_k} (a) = g(a)\dfrac{\partial f}{\partial x_k}(a) + f(a) \dfrac{\partial g}{\partial x_k}(a) \qquad \forall k\in \Delta_N
    \end{equation*}
    De donde $\nabla(fg)(a) = g(a)\nabla f(a) + f(a)\nabla g(a)$. Suponiendo que $g(x) \neq 0$ para todo $x\in \Omega$, tenemos que:
    \begin{align*}
        \dfrac{(\nicefrac{f}{g})(a+te_k) - (\nicefrac{f}{g})(a)}{t} &= \dfrac{\dfrac{f(a+te_k)}{g(a+te_k)} + \dfrac{f(a)}{g(a)}}{t} = \dfrac{f(a+te_k)g(a) + f(a)g(a+te_k)}{g(a+te_k)g(a)t} \\
                                                                    &= \dfrac{f(a+te_k)g(a) - g(a)f(a) + g(a)f(a) - f(a)g(a+te_k)}{g(a+te_k)g(a)t}  \\
                                                                    &= \dfrac{g(a) \cdot \dfrac{f(a+te_k)-f(a)}{t}}{g(a+te_k)g(a)} - \dfrac{f(a) \cdot \dfrac{g(a+te_k)-g(a)}{t}}{g(a+te_k)g(a)} \\
                                                                    & \forall t\in \mathbb{R}^\ast, \forall k\in \Delta_N
    \end{align*}
    De donde:
    \begin{equation*}
        \dfrac{\partial (\nicefrac{f}{g})}{\partial x_k}(a) = \dfrac{g(a) \dfrac{\partial f}{\partial x_k}(a) - f(a)\dfrac{\partial g}{\partial x_k}(a)}{{g(a)}^{2}} \qquad \forall k\in \Delta_N
    \end{equation*}
    Por lo que tenemos también que $\nabla(\nicefrac{f}{g})(a) = \frac{g(a) \nabla f(a) - f(a) \nabla g(a)}{{g(a)}^{2}}$
\end{ejercicio}

\begin{ejercicio}
    Fijado $p\in \bb{R}^\ast$, se considera la función $f:\bb{R}^N\setminus \{0\}\to \bb{R}$ definida por $f(x) = \|x\|^p$ para todo $x\in \bb{R}^N\setminus \{0\}$, donde $\|\cdot\|$ es la norma euclídea. Probar que $f\in C^1(\bb{R}^N\setminus \{0\})$, con
    \begin{equation*}
        \nabla f(x) = p\|x\|^{p-2}x \qquad \forall x\in \bb{R}^N\setminus \{0\}
    \end{equation*}

    \noindent
    Como consecuencia, encontrar una función $g\in C^1(\bb{R}^N)$ que verifique 
    \begin{equation*}
        \nabla g(x)=x \qquad \forall x\in \bb{R}^N
    \end{equation*}

    \begin{description}
        \item [Opción 1.] 
        Desarrollamos la expresión de la norma euclídea:
        \begin{equation*}
            f(x) = \|x\|^p = \left(\sqrt{\sum_{i=1}^N x_i^2}\right)^p \qquad \forall x\in \bb{R}^N\setminus \{(0,0)\}
        \end{equation*}

        Calculemos las derivadas parciales de $f$:
        \begin{equation*}
            \del{f}{x_i}(x) = p\|x\|^{p-1}\cdot \frac{1}{2\|x\|}\cdot 2x_i = p\|x\|^{p-2}x_i \qquad \forall x\in \bb{R}^N\setminus \{(0,0)\}
        \end{equation*}

        Por tanto, $\qquad \forall x\in \bb{R}^N\setminus \{(0,0)\}$ tenemos que:
        \begin{equation*}
            \nabla f(x) = \left(p\|x\|^{p-2}x_1,\dots, p\|x\|^{p-2}x_i, \dots, p\|x\|^{p-2}x_n\right) = p\|x\|^{p-2} \cdot x
        \end{equation*}
        \item [Opción 2.] Si definimos
            \Func{g}{\bb{R}^N\setminus \{0\}}{\bb{R}}{x}{\|x\|}
            \Func{h}{\bb{R}^+}{\bb{R}}{x}{x^p}
            Vemos que $g(\mathbb{R}^N\setminus\{0\})\subset \mathbb{R}^+$ y que $f = h\circ g$. En el Ejercicio~\ref{ej:rel6_2} vimos que $g$ era diferenciable en $\mathbb{R}^N\setminus \{0\}$, con:
            \begin{equation*}
                Dg(a)(x) = \left\langle \dfrac{a}{\|a\|},x \right\rangle  \qquad \forall x\in \mathbb{R}^N, \forall a\in \mathbb{R}^N\setminus\{0\}
            \end{equation*}
            Además, sabemos que $h$ es derivable en $\mathbb{R}^+$, con $h'(x) = px^{p-1}$ $\forall x\in \mathbb{R}^+$, luego es diferenciable en $\mathbb{R}^+$, con:
            \begin{equation*}
                Dh(a)(x) = pa^{p-1}x \qquad \forall x\in \mathbb{R}, \forall a\in \mathbb{R}^+
            \end{equation*}
            Por la regla de la cadena, obtenemos que $f = h\circ g$ es diferenciable en $\mathbb{R}^N\setminus\{0\}$, con $Df(a) = Dh(g(a)) \circ Dg(a)$, por lo que:
            \begin{align*}
                Df(a)(x) &= Dh(g(a))(Dg(a)(x)) = Dh(g(a))\left(\left\langle \frac{a}{\|a\|},x \right\rangle \right) = p\|a\|^{p-1}\left\langle \frac{a}{\|a\|},x \right\rangle  \\
                         &= p\|a\|^{p-2}\langle a,x \rangle  \qquad \forall a\in \mathbb{R}^N\setminus\{0\}, \forall x\in \mathbb{R}^N
            \end{align*}
            De donde:
            \begin{equation*}
                Df(x)(e_k) = p\|x\|^{p-2}\langle x,e_k \rangle  = p\|x\|^{p-2}x_k \qquad \forall k\in \Delta_N
            \end{equation*}
            Finalmente, vemos que:
            \begin{equation*}
                \nabla f(x) = \sum_{k=1}^{N}Df(x)(e_k)e_k = p\|x\|^{p-2}(x_1,x_2,\ldots,x_N) = p\|x\|^{p-2}x \qquad \forall x\in \mathbb{R}^N\setminus\{0\}
            \end{equation*}
    \end{description}

    \noindent
    Como $\nabla f$ es una función claramente continua, $f\in C^1(\bb{R}^N\setminus \{0\})$.

    La función $g$ pedida vemos que es la siguiente:
    \begin{equation*}
        g(x) = \frac{\|x\|^2}{2} = \frac{1}{2} \sum_{i=1}^N x_i^2 \qquad \forall x\in \bb{R}^N
    \end{equation*}
\end{ejercicio}

\begin{ejercicio}
    Calcular la ecuación del plano tangente a la superficie explícita de ecuación $z = x + y^3$ con $(x,y)\in\bb{R}^2$, en el punto $(1,1,2)$.\\

    \noindent
    Sea $f:\bb{R}^2\to \bb{R}$ dada por $f(x,y) = x+y^3$. Tenemos que $f\in C^1(\bb{R}^2)$; y las derivadas parciales de $f$ son:
    \begin{equation*}
        \del{f}{x}(x,y) = 1
        \qquad 
        \del{f}{y}(x,y) = 3y^2
    \end{equation*}
    De donde $\nabla f(1,1)=(1,3)$. Por tanto, la ecuación del plano tangente a dicha superficie en el punto $(1,1,2)$ es:
    \begin{equation*}
        \pi\equiv z-2 = 1(x-1) + 3(y-1) \Longrightarrow \pi\equiv x+3y-z=2
    \end{equation*}
\end{ejercicio}

\begin{ejercicio}
    Sea $\Omega$ un subconjunto abierto y conexo de $\bb{R}^2$ y $f:\Omega\to \bb{R}$ una función diferenciable. Se considera la superficie explícita $S\subset \bb{R}^3$ dada por
    \begin{equation*}
        S=\{(x,f(x,z), z)\mid (x,z)\in \Omega\}
    \end{equation*}
    Calcular la ecuación del plano tangente a $S$ en un punto arbitrario $(x_0,y_0,z_0)\in S$.\\

    \noindent
    Usamos la siguiente notación:
    \begin{equation*}
        y_0=f(x_0,z_0) \qquad \alpha_0 = \del{f}{x}(x_0, z_0) \qquad \beta_0 = \del{f}{z}(x_0, z_0)
    \end{equation*}

    Veamos ahora que dicho plano es el siguiente:
    \begin{equation*}
        \Pi \equiv y-y_0 = \alpha_0(x-x_0) + \beta_0(z-z_0)
    \end{equation*}

    Tenemos que se trata de una superficie explícita, donde $g:\bb{R}^2\to \bb{R}$ es la función definida por:
    \begin{multline*}
        g(x,z) = f(x_0,z_0) + \alpha_0(x-x_0) + \beta_0(z-z_0) = f(x_0,z_0) + (\nabla f(x_0,z_0)\mid [(x,z) - (x_0,z_0)]) \AstIg\\
        \AstIg f(x_0,z_0) + Df(x_0,y_0)[(x,z) - (x_0,z_0)])
    \end{multline*}
    donde en $(\ast)$ he aplicad la relación de la diferencial con su gradiente como producto escalar. Además, por el significado analítico de la diferencial, tenemos que:
    \begin{equation*}
        \lim_{(x,z)\to (x_0,z_0)} \frac{f(x,z) - g(x,z)}{\|(x,z)-(x_0,z_0)\|} = 0
    \end{equation*}
    Es decir, $g$ es una buena aproximación de $f$ cerca del punto buscado. Por tanto, el plano tangente es $\Pi$ mencionado.
\end{ejercicio}


\begin{ejercicio}
    Sea $\Omega$ un subconjunto abierto de $\bb{R}^2$ y $f:\Omega\to \bb{R}^2$ una función parcialmente derivable en todo punto de $\Omega$. Probar que, si la función $\nabla f :\Omega\to \bb{R}^2$ está acotada, entonces $f$ es continua. Usando la función $f:\bb{R}^2\to \bb{R}$ definida por:
    \begin{equation*}
        f(x,y) = \frac{xy}{\sqrt{x^2+y^2}}~~\forall (x,y)\in \bb{R}^2\setminus \{(0,0)\}, \quad f(0,0)=0
    \end{equation*}
    comprobar que, con las mismas hipótesis, no se puede asegurar que $f$ sea diferenciable.\\
\end{ejercicio}
