\chapter{Práctica 3. Imagen función dos variables}

\section{Planteamiento del problema}
\noindent
Dado un conjunto $\emptyset \neq A \subset \R^2$ y una función $f:A \flecha \R$, buscamos estudiar la imagen de $f$.
\vspace{.5cm}

\noindent
El problema así planteado es inabordable, ya que no sabemos nada de $f$. Por tanto, nuestro estudio estará bastante restringido mediante varias hipótesis que han de verificar todas nuestras funciones para poder estudiar en ellas su imagen.
\vspace{.5cm}

\noindent
Por tanto, al empezar cualquier ejercicio relacionado con esta práctica, deberemos comprobar al inicio de esta que la función que queremos estudiar verifica todas y cada una de las hipótesis.

\section{Teoremas relacionados}
\noindent
Repasamos la teoría relacionada con esta práctica:

\begin{definicion}[Arco paramétrico]
    Un arco paramétrico es un conjunto de la forma $C=g(I) \subset \R^2$ donde $I \subset \R$, $I$ intervalo compacto y $g:I\rightarrow \R^2$ es una función continua.
    \vspace{.5cm}

    \noindent
    Es decir, un arco paramétrico es la imagen (en $\R^2$) por una función continua de un intervalo cerrado y acotado de $\R$.
\end{definicion}

\begin{prop}
    $$A \subset \rn \\ compacto \sii A \\ acotado \\ y \\ \overline{A} = A$$
\end{prop}

\begin{teo}[Weierstrass]
    $E,F$ espacios métricos, $\fEF$ continua.
    $$E \\ compacto \implies f(E) \\ compacto$$
\end{teo}

\begin{prop}
    $$A\subset \R \\ conexo \\\sii\\ A \\ intervalo$$
\end{prop}

\begin{teo}[del valor intermedio]
    $E,F$ espacios métricos, $\fEF$ continua.
    $$E \\ conexo \implies f(E) \\ conexo$$
\end{teo}

\begin{coro}
    $E$ espacio métrico compacto y conexo, $f:E\flecha\R$ continua $\implies$ $f(E)$ es un intervalo cerrado y acotado.
\end{coro}

\begin{definicion}[Conjuntos convexos]
    Sea $E\subset X$, $X$ espacio vectorial. $E$ es convexo cuando: 
    $$x,y \in E \implies (1-t)x+ty \in E \\ \forall t \in [0,1]$$
\end{definicion}

\begin{prop}
    Todo subconjunto convexo de un espacio normado es conexo.
\end{prop}

\begin{prop}
    La intersección de dos conjuntos convexos es convexo.
\end{prop}

\begin{observacion}
    La intersección de dos conjuntos conexos no tiene por qué ser conexo.
\end{observacion}

\begin{prop}
    Las bolas de un espacio normado son convexas, luego conexas.
\end{prop}

\begin{prop}
    $C,D \subset E$ conexos, $C\cap D \neq \emptyset \implies C\cup D$ conexo
\end{prop}

\begin{prop}[Condición necesaria de extremo relativo]
    Sea $\emptyset \neq A \subseteq \rn$ y $f:A \rightarrow \R$ un campo escalar. Si $f$ tiene un extremos relativo en un punto $a \in A^o$ y es parcialmente derivable en dicho punto, entonces $\nabla f(a)=0$.
\end{prop}

\section{Hipótesis}
\noindent
Las hipótesis que exigiremos a todas las funciones $f:A \rightarrow \R$ de esta práctica serán:

\begin{itemize}
    \item \textbf{H1)} $A$ es compacto y conexo.
    \item \textbf{H2)} $f$ es continua.
    \item \textbf{H3)} $f$ es parcialmente derivable en $A^o$ (puede no cumplirse a veces).
    \item \textbf{H4)} La frontera de $A$ es unión de arcos paramétricos.
\end{itemize}

\section{El estudio de la función}
\noindent
A continuación, describimos el estudio de la imagen de una función $f:A \rightarrow \R$ como ya hemos descrito anteriormente, justificando la razón de las hipótesis anteriores.

\subsubsection{Simplificación del estudio}
\noindent
El problema es demasiado general, así que exigimos \textbf{H1} y \textbf{H2} para simplificar este: al ser $A$ compacto y conexo por \textbf{H1} y $f$ continua por \textbf{H2}, sabemos que $f(A)$ será compacto y conexo, y por ser subconjunto de $\R$, tenemos que es un intervalo cerrado y acotado. Por tanto, $f(A) = [\max{f(A)}, \min{f(A)}]$. Nuestro estudio ha quedado reducido a encontrar los extremos absolutos de $f$.
\vspace{.5cm}

\noindent
Debemos justificar de alguna forma que se verifican \textbf{H1} y \textbf{H2}. H2 sabemos hacerlo mientras que debemos recordar ciertas estrategias para comprobar H1:
\vspace{.5cm}

\noindent
Para comprobar que $A$ es compacto, simplemente deberemos comprobar que sea cerrado y acotado, lo cual es fácil de hacer en la mayoría de los casos. Para comprobar que $A$ sea conexo:
\begin{itemize}
    \item Podemos comprobar que $A$ es convexo, luego conexo.
    \item Podemos comprobar que $A$ es intersección de convexos, luego es convexo y aplicamos el punto anterior.
    \item Podemos comprobar que $A$ es unión de dos conexos con intersección no vacía, luego conexo.
\end{itemize}

\subsubsection{Estudio del interior de A}
\noindent
Por la condición necesaria de extremo relativo, sabemos que si $f$ presenta un extremo relativo en $a \in A^o$ y es parcialmente derivable en $a$, entonces $\nabla f(a)=(0,0)$. Por tanto, debemos buscar los extremos absolutos en los puntos interiores de $f$ con gradiente 0 (puntos críticos de $f$), en los puntos interiores a $A$ en los que $f$ no sea derivable (si no se cumple \textbf{H3}) y en los puntos de $A\setminus A^o = FrA$.
\vspace{.5cm}

\noindent
Concretando más, debemos hallar los puntos del conjunto:
$$E_0 = \{(x,y) \in A^o \mid \nabla f(x,y) = 0\}$$
De cumplirse \textbf{H3} ya tendremos a todos los candidatos de $A^o$ donde buscar extremos absolutos. De lo contrario, añadiremos a $E_0$ los puntos de $A^o$ donde $f$ no sea parcialmente derivable. Es común que $f(E_0)$ sea finito, y ya tendremos finalizado el estudio en $A^o$. Faltará hallar los puntos de $FrA$ donde podamos encontrar extremos absolutos.

\subsubsection{Estudio de la frontera de A}
\noindent
La razón de \textbf{H4} es la de simplificarnos la frontera del conjunto $A$, ya que la frontera de un conjunto genérico puede ser tan caótica como queramos. Por lo que usaremos \textbf{H4} y tendremos que:
$$FrA = \bigcup_{k=1}^m C_k \mid C_k \mbox{ es un arco paramétrico } \forall k \in \Delta_m$$
Estudiaremos uno a uno los arcos paramétricos que forman $FrA$. Sea $C$ uno de los anteriores arcos paramétricos, nos disponemos a estudiar su imagen:
\vspace{.5cm}

\noindent
Por ser $C$ un arco paramétrico, sabemos que existe un intervalo $I \subset \R$ cerrado y acotado y una función continua $g:I \rightarrow \R^2$ tal que:
$$C = g(I) \subset FrA \subset A$$
Recordemos que nuestro objetivo es estudiar $f(C) \subset \R$, luego intentaremos definir una función $h:I \rightarrow \R$ tal que:
$$f(C) = f(g(I)) = h(I)$$
De forma que el problema se ha reducido a estudiar los extremos relativos de una función real de variable real (cosa que ya sabíamos hacer). Por tanto:
$$f(C) = h(I) = [\min{h(I)}, \max{h(I)}] = [\min{f(C)}, \max{f(C)}]$$
Hágase (recordamos que los extremos absolutos de una función real de variable real pueden alcanzarse en puntos críticos, puntos donde la función no es deriable, o puntos del extremo del intervalo).

\subsection{Formas de concluir el estudio}
\label{formasEstudio}
\noindent
Planteamos dos formas válidas de concluir el estudio:

\subsubsection{Usando los valores de la función}
\noindent
Una vez que disponemos del conjunto $E_0$ y de $\alpha_k = \min{f(C_k)}$, $\beta_k = \max{f(C_k)}$ $\forall k \in \Delta_m$, tomamos:
$$\alpha = \min\{\alpha_1, \alpha_2, \ldots, \alpha_m\}~~~~\beta = \max\{\beta_1, \beta_2, \ldots, \beta_m\}$$
Entonces:
$$\min{f(A)} = \min\{f(E_0), \alpha\}~~~~\max{f(A)} = \max\{f(E_0), \beta\}$$

\subsubsection{Considerando todos los puntos donde se maximiza y minimiza la función}
Una vez que disponemos del conjunto $E_0$ y de ($f(C_k) = h_k(I_k) ~\forall k \in \Delta_m$):
$$U_k = \{t \in I_k \mid h_k \mbox{ no es derivable en } t\}$$
$$V_k = \{t \in I_k \mid h_k'(t) = 0 \}$$
$$T_k = U_k \cup V_k \cup \min{I_k} \cup \max{I_k}~~~~\forall k \in \Delta_m$$
Consideramos:
$$E_k = \{(x,y) \in C_k \mid f(x,y) = h_k(t) \land t \in T_k \}$$
Definimos:
$$S = \bigcup\limits_{k=0}^m E_k$$
Entonces:
$$\min{f(A)} = \min{f(S)} \land \max{f(A)} = \max{f(S)}$$

\section{Resumen del estudio}
\begin{itemize}
    \item Comprobamos que $A$ es compacto y conexo.
    \item Comprobamos que $f$ es continua, luego $f(A) = [\min f(A), \max f(A)]$.
    \item Calculamos $A^o$ y estudiamos la derivabilidad parcial de $f$ en $A^o$.
    \item Encontramos el conjunto $E_0$ formado por los puntos críticos de $f$ y los puntos de $A^o$ en los que $f$ no sea parcialmente derivable.
    \item Comprobamos que $FrA$ es unión de $m$ arcos paramétricos.
    \item Para cada $k \in \Delta_m$, tenemos $f(C_k) = h_k(I_k)$ donde $I_k$ es un intervalo compacto y $h_k:I_k \rightarrow \R$ una función continua.
    \item \nameref{formasEstudio}.
\end{itemize}

\section{Ejemplos}
\subsubsection{a)}
Sea $A = \{(x,y) \in \R^2 \mid x^2 \leq 2y-y^2\}$ y $f:A \rightarrow \R$ dada por:
$$f(x,y) = x^2+y(y^3-4)~~~~\forall (x,y) \in \R^2$$
Se pide calcular la imagen de $f$.
\vspace{.5cm}

\noindent
Comenzamos por dar otra expresión para el conjunto $A$, ya que así escrito no sabemos a qué puntos del plano estamos haciendo referencia.
$$(x,y) \in A \Leftrightarrow x^2 \leq 2y-y^2 \Leftrightarrow x^2+y^2-2y \leq 0 \Leftrightarrow x^2+y^2-2y + 1\leq 1 \Leftrightarrow$$
$$\Leftrightarrow x^2 + (y-1)^2\leq 1 \Leftrightarrow (x,y) \in \overline{B}((0,1),1)$$
Luego $A = \overline{B}((0,1),1)$.
\vspace{.5cm}

\noindent
Continuamos comprobando que se verifican las hipótesis \textbf{H1} y \textbf{H2}:
Sabemos que $A$ es cerrado y acotado por ser una bola cerrada, luego (estamos en $\R^2$) es compacto.\newline
Como $A$ es una bola, es convexo, luego conexo.\newline
$f$ es una función continua al tratarse de un polinomio. Por tanto, sabemos que $f(A)$ es compacto y conexo en $\R$. Es decir, $f(A)$ es un intervalo cerrado y acotado:
$$f(A) = [\min f(A), \max f(A)]$$
Calculamos $A^o$ y comprobamos los puntos es lo que $f$ es parcialmente derivable, calculando sus puntos críticos:
$$A^o = B((0,1),1)$$
$f$ es parcialmente derivable en $A^o$ por tratarse de un polinomio (es diferenciable en todo $A$), luego se verifica \textbf{H3}. Los puntos donde estudiamos los extremos absolutos de $f$ en $A^o$ son los puntos críticos. Los calculamos, sea $(x,y) \in A^o$:
$$\dfrac{\partial f}{\partial x}(x,y) = 2x~~~~\dfrac{\partial f}{\partial y}(x,y) = 4y^3-4$$
$$\nabla f(x,y) = (2x, 4y^3-4) = (0,0) \Leftrightarrow 2x = 0 \land 4y^3-4 = 0 \Leftrightarrow (x,y) = (0,1)$$
Y como $(0,1) \in A^o$:
$$E_0 = \{(0,1)\}~~~~~~f(0,1) = -3$$
Ahora calculamos $FrA$ y para cada arco paramétrico, buscamos una función real de variable real cuya imagen sea la misma que la imagen del arco por $f$. Calculamos los puntos donde dicha función alcanza sus extremos absolutos.
$$FrA = S((0,1),1) = \{(x,y) \in \R^2 \mid x^2 + (y-1)^2 = 1\} = C$$
Sea $(x,y) \in C$:
$$(y-1)^2 \leq 1 \Leftrightarrow |y-1| \leq 1 \Leftrightarrow y \in [0,2]$$
Por la definición de $A$ (mirando la definición de $f$, buscamos cómo expresar $x^2$ en función de $y$):
$$(x,y) \in FrA \Leftrightarrow x^2 = 2y-y^2$$
Por tanto, podemos definir $h:[0,2] \rightarrow \R$
$$h(y) = 2y - y^2 + y(y^3-4) = -2y - y^2 + y^4~~~~\forall y \in [0,2]$$
De forma que $h([0,2]) = f(C)$
\vspace{.5cm}

\noindent
Procedemos ahora a estudiar los extremos absolutos de $h$ en el intervalo $[0,2]$. Por lo pronto, consideramos como puntos donde se pueden encontrar estos extremos 0 y 2, al ser extremos de nuestro intervalo. Calculamos los puntos críticos de $h$:
$$h'(y) = 4y^3 - 2y - 2 = 2(2y^3-y-1) = 2(y-1)(2y^2+2y+1)$$
Observamos que $h'$ tiene una raíz, $y=1$ y que no tiene más raíces en $[0,2]$, ya que si $2y^2+2y+1$ tuviera raíces, estas deberían de ser negativas. Por tanto, $h$ tiene un único punto crítico, $y=1$.
$$E(h) = \{0,1,2\}$$
$$h(0) = 0~~~~h(1)=-2~~~~h(2) = 8$$
$$h([0,2]) = [-2,8] = f(C)$$
Como teníamos que $f(0,1) = -3$, tenemos que:
$$f(A) = [-3,8]$$

\subsubsection{b)}
Sea $A = \{(x,y) \in \R^2 \mid (x-1)^2 + y^2 \leq 4 \land x \geq 0 \}$ y $f:A \rightarrow \R$ dada por:
$$f(x,y) = (x-2)^2+2y^2~~~~\forall (x,y) \in \R^2$$
Se pide calcular la imagen de $f$.
\vspace{.5cm}

\noindent
Notemos que $A = \overline{B}((1,0),2) \cap H^+_0$ donde:
$$H^+_0 = \{(x,y) \in \R^2 \mid x \geq 0\}$$
Continuamos comprobando que se verifican las hipótesis \textbf{H1} y \textbf{H2}:
Sabemos que $A$ es cerrado por ser intersección de dos cerrados. Es acotado por estar contenido en una bola cerrada, luego (estamos en $\R^2$) es compacto.\newline
Como $A$ es intersección de dos conjuntos convexos, es convexo, luego conexo.\newline
$f$ es una función continua al tratarse de un polinomio. Por tanto, sabemos que $f(A)$ es compacto y conexo en $\R$. Es decir, $f(A)$ es un intervalo cerrado y acotado:
$$f(A) = [\min f(A), \max f(A)]$$
Calculamos $A^o$ y comprobamos los puntos es lo que $f$ es parcialmente derivable, calculando sus puntos críticos:
$$A^o = B((1,0),2) \cap H^+ = B((1,0),2) \cap \{(x,y) \in \R^2 \mid x>0\}$$
$f$ es parcialmente derivable en $A^o$ por tratarse de un polinomio (es diferenciable en todo $A$), luego se verifica \textbf{H3}. Los puntos donde estudiamos los extremos absolutos de $f$ en $A^o$ son los puntos críticos. Los calculamos, sea $(x,y) \in A^o$:
$$\dfrac{\partial f}{\partial x}(x,y) = 2(x-2)~~~~\dfrac{\partial f}{\partial y}(x,y) = 4y$$
$$\nabla f(x,y) = (2(x-2), 4y) = (0,0) \Leftrightarrow 2x-4 = 0 \land 4y = 0 \Leftrightarrow (x,y) = (2,0)$$
Y como $(2,0) \in A^o$:
$$E_0 = \{(2,0)\}~~~~~~f(2,0) = 0$$
Notemos que $f(x,y) \geq 0~~~~\forall (x,y) \in A$, luego ya tenemos $\min f(A)$. Falta buscar el punto donde se encuentra el máximo absoluto.
Ahora calculamos $FrA$ y para cada arco paramétrico, buscamos una función real de variable real cuya imagen sea la misma que la imagen del arco por $f$. Calculamos los puntos donde dicha función alcanza su máximo absoluto.
\vspace{.5cm}

\noindent
Notemos que $FrA = C_1 \cup C_2$ donde:
$$C_1 = \{(x,y) \in \R^2 \mid x = 0 \land (x-1)^2 + y^2 \leq 4\} = \{(x,y) \in \R^2 \mid x=0 \land 1 + y^2 \leq 4\} =$$ $$=\{(x,y) \in \R^2 \mid x=0 \land y^2 \leq 3\}$$
Luego:
$$f(C_1) = \{f(x,y) \mid (x,y) \in C_1\} = \{f(x,y) \mid x = 0 \land y^2 \leq 3\} = \{f(0,y) \mid y^2 \leq 3\}$$
$$f(0,y) = 4+2y^2 \Rightarrow \max f(C_1) = f(0,\sqrt{3}) = 4+2 \cdot 3 = 10$$
Pasamos a estudiar $C_2$:
$$C_2 = \{(x,y) \in \R^2 \mid (x-1)^2 + y^2 = 4 \land x \geq 0\}$$
Tratamos de expresar $f$ en función de una variable dentro de $C_2$:
$$(x,y) \in C_2 \Leftrightarrow x \geq 0 \land (x-1)^2 \leq 4 \Leftrightarrow x \geq 0 \land |x-1|\leq 2 \Leftrightarrow$$
$$\Leftrightarrow x \geq 0 \land x \in [-1,3] \Leftrightarrow x \in [0,3]$$
Mirando la definición de $f$, nos damos cuenta de que es cómodo saber qué valor toma $y^2$:
$$(x,y) \in C_2 \Leftrightarrow (x-1)^2 + y^2 = 4 \land x \geq 0 \Leftrightarrow y^2 = 4-(x-1)^2 \land x \geq 0$$
Sustituyendo $y^2$ en $f$, podemos definir $h:[0,3] \Rightarrow \R$ tal que:
$$h(x) = (x-2)^2 + 2(4-(x-1)^2) = (x-2)^2 + 8 - 2(x-1)^2 = 10 - x^2$$
Es fácil ver que $\max h([0,3]) = 10$ y como teníamos que $f(2,0) = 0$:
$$f(A) = [0,10]$$