\documentclass[12pt]{book}

% Idioma y codificación
\usepackage[spanish, es-tabla]{babel}       %es-tabla para que se titule "Tabla"
\usepackage[utf8]{inputenc}

% Márgenes
\usepackage[a4paper,top=3cm,bottom=2.5cm,left=3cm,right=3cm]{geometry}

% Comentarios de bloque
\usepackage{verbatim}

% Paquetes de links
\usepackage[hidelinks]{hyperref}    % Permite enlaces
\usepackage{url}                    % redirecciona a la web

% Más opciones para enumeraciones
\usepackage{enumitem}

% Personalizar la portada
\usepackage{titling}

% Paquetes de tablas
\usepackage{multirow}


%------------------------------------------------------------------------

%Paquetes de figuras
\usepackage{caption}
\usepackage{subcaption} % Figuras al lado de otras
\usepackage{float}      % Poner figuras en el sitio indicado H.


% Paquetes de imágenes
\usepackage{graphicx}       % Paquete para añadir imágenes
\usepackage{transparent}    % Para manejar la opacidad de las figuras

% Paquete para usar colores
\usepackage[dvipsnames]{xcolor}
\usepackage{pagecolor}      % Para cambiar el color de la página

% Habilita tamaños de fuente mayores
\usepackage{fix-cm}

% Para los gráficos
\usepackage{tikz}

% Para poder situar los nodos en los grafos
\usetikzlibrary{positioning}


%------------------------------------------------------------------------

% Paquetes de matemáticas
\usepackage{mathtools, amsfonts, amssymb, mathrsfs}
\usepackage[makeroom]{cancel}     % Simplificar tachando
\usepackage{polynom}    % Divisiones y Ruffini
\usepackage{units} % Para poner fracciones diagonales con \nicefrac

\usepackage{pgfplots}   %Representar funciones
\pgfplotsset{compat=1.18}  % Versión 1.18

\usepackage{tikz-cd}    % Para usar diagramas de composiciones
\usetikzlibrary{calc}   % Para usar cálculo de coordenadas en tikz

%Definición de teoremas, etc.
\usepackage{amsthm}
%\swapnumbers   % Intercambia la posición del texto y de la numeración

\theoremstyle{plain}

\makeatletter
\@ifclassloaded{article}{
  \newtheorem{teo}{Teorema}[section]
}{
  \newtheorem{teo}{Teorema}[chapter]  % Se resetea en cada chapter
}
\makeatother

\newtheorem{coro}{Corolario}[teo]           % Se resetea en cada teorema
\newtheorem{prop}[teo]{Proposición}         % Usa el mismo contador que teorema
\newtheorem{lema}[teo]{Lema}                % Usa el mismo contador que teorema

\theoremstyle{remark}
\newtheorem*{observacion}{Observación}

\theoremstyle{definition}

\makeatletter
\@ifclassloaded{article}{
  \newtheorem{definicion}{Definición} [section]     % Se resetea en cada chapter
}{
  \newtheorem{definicion}{Definición} [chapter]     % Se resetea en cada chapter
}
\makeatother

\newtheorem*{notacion}{Notación}
\newtheorem*{ejemplo}{Ejemplo}
\newtheorem*{ejercicio*}{Ejercicio}             % No numerado
\newtheorem{ejercicio}{Ejercicio} [section]     % Se resetea en cada section


% Modificar el formato de la numeración del teorema "ejercicio"
\renewcommand{\theejercicio}{%
  \ifnum\value{section}=0 % Si no se ha iniciado ninguna sección
    \arabic{ejercicio}% Solo mostrar el número de ejercicio
  \else
    \thesection.\arabic{ejercicio}% Mostrar número de sección y número de ejercicio
  \fi
}


% \renewcommand\qedsymbol{$\blacksquare$}         % Cambiar símbolo QED
%------------------------------------------------------------------------

% Paquetes para encabezados
\usepackage{fancyhdr}
\pagestyle{fancy}
\fancyhf{}

\newcommand{\helv}{ % Modificación tamaño de letra
\fontfamily{}\fontsize{12}{12}\selectfont}
\setlength{\headheight}{15pt} % Amplía el tamaño del índice


%\usepackage{lastpage}   % Referenciar última pag   \pageref{LastPage}
\fancyfoot[C]{\thepage}

%------------------------------------------------------------------------

% Conseguir que no ponga "Capítulo 1". Sino solo "1."
\makeatletter
\@ifclassloaded{book}{
  \renewcommand{\chaptermark}[1]{\markboth{\thechapter.\ #1}{}} % En el encabezado
    
  \renewcommand{\@makechapterhead}[1]{%
  \vspace*{50\p@}%
  {\parindent \z@ \raggedright \normalfont
    \ifnum \c@secnumdepth >\m@ne
      \huge\bfseries \thechapter.\hspace{1em}\ignorespaces
    \fi
    \interlinepenalty\@M
    \Huge \bfseries #1\par\nobreak
    \vskip 40\p@
  }}
}
\makeatother

%------------------------------------------------------------------------
% Paquetes de cógido
\usepackage{minted}
\renewcommand\listingscaption{Código fuente}

\usepackage{fancyvrb}
% Personaliza el tamaño de los números de línea
\renewcommand{\theFancyVerbLine}{\small\arabic{FancyVerbLine}}

% Estilo para C++
\newminted{cpp}{
    frame=lines,
    framesep=2mm,
    baselinestretch=1.2,
    linenos,
    escapeinside=||
}

% para minted
\definecolor{LightGray}{rgb}{0.95,0.95,0.92}
\setminted{
    linenos=true,
    stepnumber=5,
    numberfirstline=true,
    autogobble,
    breaklines=true,
    breakautoindent=true,
    breaksymbolleft=,
    breaksymbolright=,
    breaksymbolindentleft=0pt,
    breaksymbolindentright=0pt,
    breaksymbolsepleft=0pt,
    breaksymbolsepright=0pt,
    fontsize=\footnotesize,
    bgcolor=LightGray,
    numbersep=10pt
}


\usepackage{listings} % Para incluir código desde un archivo

\renewcommand\lstlistingname{Código Fuente}
\renewcommand\lstlistlistingname{Índice de Códigos Fuente}

% Definir colores
\definecolor{vscodepurple}{rgb}{0.5,0,0.5}
\definecolor{vscodeblue}{rgb}{0,0,0.8}
\definecolor{vscodegreen}{rgb}{0,0.5,0}
\definecolor{vscodegray}{rgb}{0.5,0.5,0.5}
\definecolor{vscodebackground}{rgb}{0.97,0.97,0.97}
\definecolor{vscodelightgray}{rgb}{0.9,0.9,0.9}

% Configuración para el estilo de C similar a VSCode
\lstdefinestyle{vscode_C}{
  backgroundcolor=\color{vscodebackground},
  commentstyle=\color{vscodegreen},
  keywordstyle=\color{vscodeblue},
  numberstyle=\tiny\color{vscodegray},
  stringstyle=\color{vscodepurple},
  basicstyle=\scriptsize\ttfamily,
  breakatwhitespace=false,
  breaklines=true,
  captionpos=b,
  keepspaces=true,
  numbers=left,
  numbersep=5pt,
  showspaces=false,
  showstringspaces=false,
  showtabs=false,
  tabsize=2,
  frame=tb,
  framerule=0pt,
  aboveskip=10pt,
  belowskip=10pt,
  xleftmargin=10pt,
  xrightmargin=10pt,
  framexleftmargin=10pt,
  framexrightmargin=10pt,
  framesep=0pt,
  rulecolor=\color{vscodelightgray},
  backgroundcolor=\color{vscodebackground},
}

%------------------------------------------------------------------------

% Comandos definidos
\newcommand{\bb}[1]{\mathbb{#1}}
\newcommand{\cc}[1]{\mathcal{#1}}

% I prefer the slanted \leq
\let\oldleq\leq % save them in case they're every wanted
\let\oldgeq\geq
\renewcommand{\leq}{\leqslant}
\renewcommand{\geq}{\geqslant}

% Si y solo si
\newcommand{\sii}{\iff}

% Letras griegas
\newcommand{\eps}{\epsilon}
\newcommand{\veps}{\varepsilon}
\newcommand{\lm}{\lambda}

\newcommand{\ol}{\overline}
\newcommand{\ul}{\underline}
\newcommand{\wt}{\widetilde}
\newcommand{\wh}{\widehat}

\let\oldvec\vec
\renewcommand{\vec}{\overrightarrow}

% Derivadas parciales
\newcommand{\del}[2]{\frac{\partial #1}{\partial #2}}
\newcommand{\Del}[3]{\frac{\partial^{#1} #2}{\partial #3^{#1}}}
\newcommand{\deld}[2]{\dfrac{\partial #1}{\partial #2}}
\newcommand{\Deld}[3]{\dfrac{\partial^{#1} #2}{\partial #3^{#1}}}


\newcommand{\AstIg}{\stackrel{(\ast)}{=}}
\newcommand{\Hop}{\stackrel{L'H\hat{o}pital}{=}}

\newcommand{\red}[1]{{\color{red}#1}} % Para integrales, destacar los cambios.

% Método de integración
\newcommand{\MetInt}[2]{
    \left[\begin{array}{c}
        #1 \\ #2
    \end{array}\right]
}

% Declarar aplicaciones
% 1. Nombre aplicación
% 2. Dominio
% 3. Codominio
% 4. Variable
% 5. Imagen de la variable
\newcommand{\Func}[5]{
    \begin{equation*}
        \begin{array}{rrll}
            #1:& #2 & \longrightarrow & #3\\
               & #4 & \longmapsto & #5
        \end{array}
    \end{equation*}
}

%------------------------------------------------------------------------

\DeclareMathOperator{\Fr}{Fr}


%Definiciones para ahorrar trabajo
\def\R{\mathbb R}
\def\C{\mathbb C}
\def\N{\mathbb N}
\def\Z{\mathbb Z}
\def\Q{\mathbb Q}
\def\todoi{\forall i \in \{1, \ldots, n\}}
\def\todon{\ \forall n \in \N}
\def\sii{\Longleftrightarrow}

\def\\{\quad}
\def\rn{\R^N}


\begin{document}

    % 1. Foto de fondo
    % 2. Título
    % 3. Encabezado Izquierdo
    % 4. Color de fondo
    % 5. Coord x del titulo
    % 6. Coord y del titulo
    % 7. Fecha
    % 8. Autor

    
    % 1. Foto de fondo
% 2. Título
% 3. Encabezado Izquierdo
% 4. Color de fondo
% 5. Coord x del titulo
% 6. Coord y del titulo
% 7. Fecha

\newcommand{\portada}[7]{

    \portadaBase{#1}{#2}{#3}{#4}{#5}{#6}{#7}
    \portadaBook{#1}{#2}{#3}{#4}{#5}{#6}{#7}
}

\newcommand{\portadaExamen}[7]{

    \portadaBase{#1}{#2}{#3}{#4}{#5}{#6}{#7}
    \portadaArticle{#1}{#2}{#3}{#4}{#5}{#6}{#7}
}




\newcommand{\portadaBase}[7]{

    % Tiene la portada principal y la licencia Creative Commons
    
    % 1. Foto de fondo
    % 2. Título
    % 3. Encabezado Izquierdo
    % 4. Color de fondo
    % 5. Coord x del titulo
    % 6. Coord y del titulo
    % 7. Fecha
    
    
    \thispagestyle{empty}               % Sin encabezado ni pie de página
    \newgeometry{margin=0cm}        % Márgenes nulos para la primera página
    
    
    % Encabezado
    \fancyhead[L]{\helv #3}
    \fancyhead[R]{\helv \nouppercase{\leftmark}}
    
    
    \pagecolor{#4}        % Color de fondo para la portada
    
    \begin{figure}[p]
        \centering
        \transparent{0.3}           % Opacidad del 30% para la imagen
        
        \includegraphics[width=\paperwidth, keepaspectratio]{assets/#1}
    
        \begin{tikzpicture}[remember picture, overlay]
            \node[anchor=north west, text=white, opacity=1, font=\fontsize{60}{90}\selectfont\bfseries\sffamily, align=left] at (#5, #6) {#2};
            
            \node[anchor=south east, text=white, opacity=1, font=\fontsize{12}{18}\selectfont\sffamily, align=right] at (9.7, 3) {\textbf{\href{https://losdeldgiim.github.io/}{Los Del DGIIM}}};
            
            \node[anchor=south east, text=white, opacity=1, font=\fontsize{12}{15}\selectfont\sffamily, align=right] at (9.7, 1.8) {Doble Grado en Ingeniería Informática y Matemáticas\\Universidad de Granada};
        \end{tikzpicture}
    \end{figure}
    
    
    \restoregeometry        % Restaurar márgenes normales para las páginas subsiguientes
    \pagecolor{white}       % Restaurar el color de página
    
    
    \newpage
    \thispagestyle{empty}               % Sin encabezado ni pie de página
    \begin{tikzpicture}[remember picture, overlay]
        \node[anchor=south west, inner sep=3cm] at (current page.south west) {
            \begin{minipage}{0.5\paperwidth}
                \href{https://creativecommons.org/licenses/by-nc-nd/4.0/}{
                    \includegraphics[height=2cm]{assets/Licencia.png}
                }\vspace{1cm}\\
                Esta obra está bajo una
                \href{https://creativecommons.org/licenses/by-nc-nd/4.0/}{
                    Licencia Creative Commons Atribución-NoComercial-SinDerivadas 4.0 Internacional (CC BY-NC-ND 4.0).
                }\\
    
                Eres libre de compartir y redistribuir el contenido de esta obra en cualquier medio o formato, siempre y cuando des el crédito adecuado a los autores originales y no persigas fines comerciales. 
            \end{minipage}
        };
    \end{tikzpicture}
    
    
    
    % 1. Foto de fondo
    % 2. Título
    % 3. Encabezado Izquierdo
    % 4. Color de fondo
    % 5. Coord x del titulo
    % 6. Coord y del titulo
    % 7. Fecha


}


\newcommand{\portadaBook}[7]{

    % 1. Foto de fondo
    % 2. Título
    % 3. Encabezado Izquierdo
    % 4. Color de fondo
    % 5. Coord x del titulo
    % 6. Coord y del titulo
    % 7. Fecha

    % Personaliza el formato del título
    \pretitle{\begin{center}\bfseries\fontsize{42}{56}\selectfont}
    \posttitle{\par\end{center}\vspace{2em}}
    
    % Personaliza el formato del autor
    \preauthor{\begin{center}\Large}
    \postauthor{\par\end{center}\vfill}
    
    % Personaliza el formato de la fecha
    \predate{\begin{center}\huge}
    \postdate{\par\end{center}\vspace{2em}}
    
    \title{#2}
    \author{\href{https://losdeldgiim.github.io/}{Los Del DGIIM}}
    \date{Granada, #7}
    \maketitle
    
    \tableofcontents
}




\newcommand{\portadaArticle}[7]{

    % 1. Foto de fondo
    % 2. Título
    % 3. Encabezado Izquierdo
    % 4. Color de fondo
    % 5. Coord x del titulo
    % 6. Coord y del titulo
    % 7. Fecha

    % Personaliza el formato del título
    \pretitle{\begin{center}\bfseries\fontsize{42}{56}\selectfont}
    \posttitle{\par\end{center}\vspace{2em}}
    
    % Personaliza el formato del autor
    \preauthor{\begin{center}\Large}
    \postauthor{\par\end{center}\vspace{3em}}
    
    % Personaliza el formato de la fecha
    \predate{\begin{center}\huge}
    \postdate{\par\end{center}\vspace{5em}}
    
    \title{#2}
    \author{\href{https://losdeldgiim.github.io/}{Los Del DGIIM}}
    \date{Granada, #7}
    \thispagestyle{empty}               % Sin encabezado ni pie de página
    \maketitle
    \vfill
}
    \portada{ffccA4.jpg}{Resúmenes\\Análisis\\Matemático I}{Resúmenes Analisis Matematico I}{MidnightBlue}{-9}{28}{2023-2024}{José Juan Urrutia Milán ``JJ''} %Irina Kuzyshyn

    \newpage
    \chapter*{Introducción}
    En el presente libro se podrán encontrar resúmenes de lo básico del temario de la asignatura Análisis Matemático I, sin demostracioón alguna. Es ningún caso con esto basta para comprender perfectamente la asignatura, simplemente es un recurso más para no olvidar lo básico.

    \chapter{Práctica 1. Continuidad}

\section{Teoremas relacionados}
\begin{teo}[Carácter local de la continuidad]
    Sean $E$, $F$ espacios topológicos, $\emptyset \neq A \subseteq E$, $x \in A$.

    
    Si $f_{\big| A}$ es continua en $x$ con $A \in \mathbb{U}(x) \Rightarrow f$ es continua en $x$.
\end{teo}

\begin{teo}[Cambio de variable]
    Sean $E$, $F$ espacios métricos, $\emptyset \neq A \subseteq E$, $f:A \rightarrow F$
    y $\alpha \in E$:\\

    
    Si $G$ es un espacio métrico con $T \subseteq G$ y $\varphi:T \rightarrow E$, $z \in T'$ que cumple:
    $$\lim_{t \to z}\varphi(t)=\alpha \in E ~~~~ \varphi(t) \in A\setminus\{\alpha\} ~\forall t \in T\setminus\{z\}$$
    Entonces, $\alpha \in A'$ y se verifica que:
    $$\lim_{x \to \alpha} f(x) = L \Rightarrow \lim_{t \to z}f(\varphi(t)) = L$$
\end{teo}

\vspace{2cm}

Procedemos por tanto a estudiar el siguiente problema:\\

\noindent
Dada $f:E \rightarrow \R^n$ con $E \subseteq \R^m$, $n,m \in \N$. Comprobar que $f$ es continua.

\section{Parte rutinaria del problema}
\begin{itemize}
    \item Definimos $U$ y comprobamos que $U$ sea abierto.
    \item Comprobamos que $f_{\big|U}$ sea continua.
    \item Aplicamos el \underline{carácter local de la continuidad} y tenemos que $f$ es continua en $U$.
\end{itemize}


A continuación se nos presentan distintos puntos problemáticos en los que querremos
estudiar el límite. Nos fereriremos a un punto de estos como $\alpha$.
Calculamos:
$$\lim_{x \to \alpha} f(x)$$
Normalmente, se presentará una indeterminación. A continuación, la intuición nos dirá
si debemos intentar probar que el límite no existe o intentar probar la existencia del límite.


Un camino algo más mecánico es comprobar, en este orden, los límites parciales, límites direccionales,
intentar probar la existencia de límite y, por último, intentar probar que el límite no
existe con un cambio de variable.

\section{El límite no existe}

Si creemos que el límite en $\alpha$ no existe, el procedimiento a seguir es el siguiente:

\subsection{Límites parciales}

Si $e_k$ es el $k$-ésimo vector de la base usual. Sea $t \in \R \mid x \rightarrow \alpha$ si
$t \rightarrow 0$ con $x \neq \alpha$ si $t \neq 0$. Entonces:
$$\lim_{x \to \alpha}f(x) = L \Rightarrow \lim_{t \to 0}f(\alpha + te_k)=L$$
En el caso $n=2$, si $\alpha = (a,b)$. Entonces:
$$\lim_{x \to (a,b)}f(x) = L \Rightarrow \lim_{x \rightarrow a}f(x,b) = \lim_{x \rightarrow b}f(a,x)=L$$

\begin{itemize}
    \item Si uno de los límites parciales no existe, podemos afirmar ($\ast$).
    \item Si existen los dos límites parciales y no son iguales, podemos afirmar ($\ast$).
    \item En caso de que existan y sean iguales, el único candidato a límite será $L$, por lo que si por
          otro método nos sale que el límite no es $L$, podemos afirmar ($\ast$).
\end{itemize}
$$(\ast)\hspace{1cm}\nexists \lim_{x \to \alpha}f(x)$$

\subsection{Límites direccionales}
$$S = \{u \in E\setminus\{0\} \mid \|u\| = 1\}$$


Sea $u \in S$. Entonces, si $t \in \R$:
$$\lim_{x \to \alpha}f(x) = L \Rightarrow \lim_{t \to 0}f(\alpha + tu) = L~~~~~~\forall u \in S$$
El cálculo lo haremos con un $u$ genérico que cumpla estas premisas, de forma que:
\begin{itemize}
    \item Si uno de los límites direccionales no existe, podemos afirmar ($\ast$).
    \item Si el límte direccional depende de $u$, podemos afirmar ($\ast$) al saber que si cambiamos
          $u$ obtenemos distintos valores del límite.
    \item En caso de que existan y sean iguales, el único candidato a límite será $L$, por lo que si por
          otro método nos sale que el límite no es $L$, podemos afirmar ($\ast$).
\end{itemize}

Hay que tener en cuenta que según el $E$ a veces no podemos estudiar ciertos límites direccionales.

\subsubsection{Límites radiales}
$$\lim_{x \to \alpha}f(x) = L \Rightarrow \lim_{t \to 0^{+}}f(\alpha + tu) = L~~~~~~\forall u \in S$$
Hay que tener en cuenta que según el $E$ a veces no podemos estudiar ciertos límites radiales.


Caso $n=2$:
\subsubsection{Coordenadas polares}
$$\alpha = (x,y)$$
$$u \in \R^2 \mid \|u\| = 1 \Leftrightarrow u =(\cos\theta,\sin\theta)~~\theta \in \R$$
$$\rho = \sqrt{x^2+y^2} \in \R$$
\ \\
$$\lim_{t \to 0^{+}}f(x) = \lim_{\rho \to 0}f(a+\rho \cos\theta, b+\rho \sin\theta)$$
$$\lim_{x \to \alpha}=L \Rightarrow \lim_{\rho \rightarrow 0}f(a+\rho \cos\theta, b+\rho \sin\theta)=L~~~
    ~~~ \forall \theta \in \R$$

\subsubsection{Coordenadas cartesianas}
$$u=(u_1, u_2) \in \R^2$$
En vez de normalizar con $\|u\|=1$, tomamos:
$$u_1 = 1 \mbox{ y } u_2 = \lambda \in \R$$
$$\lim_{t \to 0}f(\alpha + tu) = \lim_{t \to 0}f(a+t,b+\lambda t)$$
$$\lim_{x \to \alpha}f(x) = L \Rightarrow \lim_{t \to 0}f(a+t,b+\lambda t) = L~~~~~~ \forall \lambda \in \R$$

\section{Existencia del límite}

La \underline{única} forma de probar que $\lim\limits_{x \to \alpha}f(x) = L \in \R$ es acotando $f$:\\


Necesitamos hallar $r \in \R^{+}$ y $g:B(\alpha,r)\rightarrow \R^{+}$ tal que:
$$0 \leq \left| f(x) - L \right| \leq g(x) ~~~~\forall x \in B(\alpha,r)\setminus\{\alpha\}$$

De tal forma que $$\lim_{x \to \alpha}g(x) = 0$$


Entonces, por el lema del Sándwich, tenemos que:
$$\lim_{x \to \alpha}f(x) = L$$


El estudio fracasado de los límites direccionales puede ayudarnos a la hora de determinar de
forma más fácil una acotación:

\subsection{Acotación por límites direccionales}

Si el estudio de los límites direccionales fracasó fue porque:
$$\lim_{t \to 0}f(\alpha + tu)-L = 0~~~~~~\forall u \in S$$
Luego si hallamos $r \in \R^{+}$ y una función $h:]0,r[\rightarrow \R^{+}$ con $\lim\limits_{t\to 0}h(t) = 0$,
de forma que:
$$0 \leq |f(\alpha + tu)-L| \leq h(t)~~~~~~\forall u \in S~~\forall t \in ]0,r[$$
Tendremos que: $$\lim_{x \to \alpha}f(x) = L$$


En el caso $n=2$:
\subsection{Acotación por uso de coordenadas polares}

Si el estudio de los límites direccionales usando coordenadas polares fracasó fue porque:
$$\lim_{\rho \to 0}f(a + \rho \cos\theta, b + \rho \sin\theta)-L = 0~~~~~~\forall \theta \in \R$$
Luego si hallamos $r \in \R^{+}$ y una función $h:]0,r[\rightarrow \R^{+}$ con $\lim\limits_{\rho\to 0}h(\rho) = 0$,
de forma que:
$$0 \leq |f(a+\rho \cos\theta,b+\rho \sin\theta)-L| \leq h(\rho)~~~~~~\forall \theta \in \R~~\forall \rho \in ]0,r[$$
Tendremos que: $$\lim_{x \to \alpha}f(x) = L$$

\subsection{Acotación por uso de coordenadas cartesianas}

Si el estudio de los límites direccionales usando coordenadas cartesianas fracasó fue porque:
$$\lim_{t \to 0}f(a + t, b + \lambda t)-L = 0~~~~~~\forall \lambda \in \R$$
Luego si hallamos $r \in \R^{+}$ y una función $h:]0,r[\rightarrow \R^{+}$ con $\lim\limits_{t\to 0}h(t) = 0$,
de forma que:
$$0 \leq |f(a+t,b+\lambda t)-L| \leq h(t)~~~~~~\forall t \in \R~~\forall t \in ]-r,r[\setminus\{0\}$$
Tendremos que: $$\lim_{x \to \alpha}f(x) = L$$

\section{Último recurso}

Si no pudimos encontrar ninguna acotación de $f$, deberemos intuir que el límite no existe.
Para probar esto, tenemos que idear un cambio de variable nuevo:\\


Si $L \in \R$ es el único posible límite de $f$ en $\alpha$, podemos probar con un cambio de variable
$x = \varphi(t)$ con $0 < t < r$ tal que:
$$\lim_{t \to 0}\varphi(t) = \alpha ~~~~\mbox{ y } ~~~~ \varphi(t) \neq \alpha ~~\forall t \in ]0,r[$$
$$\lim_{x \to \alpha}f(x) = L \Rightarrow \lim_{t \to 0}f(\varphi(t)) = L$$
Luego buscamos $\varphi$ de forma que $f \circ \varphi$ no tenga límite en 0.\\


Por ejemplo, en el caso $n=2$ con $\alpha = (a,b)$, podemos hacer el cambio de variable:
$$\varphi_p(t)=(a+t, b+t^p)~~~~~~~p \in \R^{+}$$
De forma que calculamos el límite con un $p \in \R^{+}$ cualquiera y luego fijamos un valor de $p$
para el cual el límite no exista.\\


Otro recurso que podemos usar es que si $n=2$ y nuestra función $f$ es un cociente entre dos
términos que contienen $x$ e $y$ de forma que el exponente de $y$ es siempre el doble de $x$, podemos
usar el cambio de variable:
$$\varphi_2(t) = (a+t,b+t^2)$$

\section{Límites famosos}
\begin{equation*}
    \begin{array}{ccc}
        \displaystyle \lim_{t \to 0}\dfrac{\sen t}{t} = 1
        &
        \displaystyle  \lim_{t \to 0}\dfrac{\tan t}{t} = 1
        &
        \displaystyle  \lim_{t \to 0}\dfrac{\arcsen t}{t} = 1\\ \\
        \displaystyle  \lim_{t \to 0}\dfrac{\arctan t}{t} = 1
        &
        \displaystyle  \lim_{t \to 0}\dfrac{1-\cos t}{t^2} = \dfrac{1}{2}
        & 
        \displaystyle  \lim_{t \to 0}\dfrac{e^t -e^0}{t} = 1\\ \\
        &\displaystyle  \lim_{t \to 0}\dfrac{\log(1+t)}{t}=1
    \end{array}
\end{equation*}

\subsection{Cambiar forma de la función}

Dada una función a la que le queremos calcular un límite, es recurrente que nos sepamos a qué tiende
parte del límite ya que nos es conocido y querramos descomponer la función en dos partes, una
de la que conocemos su límite y otra que será más sencillo de calcular. La pregunta es cómo podemos
hacer esto formalmente y sin fallos. Para ello, pondremos el ejemplo de:
$$f(x,y) = \dfrac{x^2 \sen y}{x^2+y^2}~~\mbox{ si } x \in \R^2 \setminus\{(0,0)\}$$
De tal forma que queremos calcular el límite en el punto $(0,0)$. Para ello, uno podría pensar
que podemos hacer:
$$f(x,y) = \dfrac{\sen y}{y} \dfrac{x^2y}{x^2+y^2}$$
Pero debemos tener cuidado, ya que nuestro dominio es $\R^2\setminus\{(0,0)\}$ y al cambiar la
expresión de $f$ estamos dividiendo por cero al considerar cualquier punto del estilo $(a,0)$ con
$a \neq 0$ en nuestro dominio.
Para solucionar este problema, resolveremos el ejercicio de la siguiente forma:\\


Sea $\varphi:\R \rightarrow \R$ una función tal que:
$$\varphi(y) = \dfrac{\sen y}{y}~~~~\forall y \neq 0~~~~~~~~\varphi(0)=1$$
Notemos que $\varphi$ es continua en todo $\R$, al ser $\lim\limits_{y \to 0}\varphi(y) = 1 = \varphi(0)$.


De esta forma, hemos conseguido una función que nos permite hacer lo siguiente:
$$\sen y = y\varphi(y)~~~~~~\forall y \in \R$$
$$f(x,y)=\varphi(y)\dfrac{x^2y}{x^2+y^2}~~~~~~\forall (x,y) \in \R^2\setminus\{(0,0)\}$$
De esta forma, podemos estudiar $f$ en dos partes:\\


Por una lado, sabemos que:
$$\lim_{y \to 0} \varphi(y) = \varphi(0) = 1$$


Y por otro, tenemos que podemos acotar fácilmente la función, haciendo que el otro trozo converja a cero:
$$0 \leq \left| \dfrac{x^2y}{x^2+y^2} \right| = \dfrac{x^2|y|}{x^2+y^2} \leq |y|$$
Con $\lim\limits_{y \to 0}y = 0$. Luego:
$$\lim_{(x,y) \to (0,0)} \dfrac{x^2y}{x^2+y^2}=0$$


Por lo que, finalmente:
$$\lim_{(x,y) \to (0,0)} \dfrac{x^2\sen y}{x^2+y^2} = \lim_{(x,y) \to (0,0)} \varphi(y)\dfrac{x^2y}{x^2+y^2}=0$$


\end{document}