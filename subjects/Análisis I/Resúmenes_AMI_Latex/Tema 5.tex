\chapter{Complitud y continuidad uniforme}
\section{Complitud}

\begin{definicion}[Sucesiones de Cauchy]
    $E$ espacio métrico con distancia $d$. $\xn \subset E$ es una sucesión de Cauchy cuando:
    $$\forall \varepsilon > 0 \\ \exists m \in \N : p,q \geq m \implies d(x_p,x_q) < \varepsilon$$
    Toda sucesión convergente es una sucesión de Cauchy.\newline
    No es una propiedad topológica.
\end{definicion}

\begin{definicion}[Espacios completos]
    Un espacio métrico $E$ es completo, o su distancia $d$ es completa cuando toda sucesión de Cauchy es convergente.
\end{definicion}

\begin{ejemplo}
    Espacio de Banach = espacio normado completo.
\end{ejemplo}

\begin{ejemplo}
    Espacio de Hilbert = espacio pre-hilbertiano completo.
\end{ejemplo}

\begin{}