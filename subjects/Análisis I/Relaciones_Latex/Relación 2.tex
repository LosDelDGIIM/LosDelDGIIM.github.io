\section{Topología de un espacio métrico}


\begin{ejercicio}
    Probar que, en todo espacio métrico, la distancia queda determinada cuando se conocen las bolas abiertas. En el caso particular de un espacio normado, probar que la norma queda determinada cuando se conoce la bola abierta unidad.
\end{ejercicio}


\begin{ejercicio}
    Sea $X$ un espacio normado, $x,y\in X$ y $r,\rho\in \bb{R}^+$. Probar que:
    \begin{enumerate}
        \item $B(x,r)\cap B(y,\rho)=\emptyset \Longleftrightarrow ||y-x|| <r+\rho$.
        \item $B(y,\rho)\subset B(x,r)=\emptyset \Longleftrightarrow ||y-x|| <r-\rho$.
    \end{enumerate}

    ¿Son ciertos los resultados análogos en un espacio métrico cualquiera?
\end{ejercicio}


\begin{ejercicio}
    Dar un ejemplo de una familia numerable de abiertos de $\bb{R}$ cuya intersección no sea un conjunto abierto.
\end{ejercicio}

\begin{ejercicio}
    Si $A$ es un subconjunto no vacío de un espacio métrico $E$ con distancia $d$, se define la \emph{distancia} de un punto $x\in E$ al conjunto $A$ por
    $$d(x,A) = \inf\{ d(x,a) \mid a \in A\}$$

    Probar que $ \ol{A} = \{x \in E \mid d(x,A) = 0\}$.
\end{ejercicio}


\begin{ejercicio}
    Sea $X$ un espacio normado, $x \in X$ y $r \in \bb{R}^+$, probar que
    \begin{enumerate}
        \item $\ol{B(x,r)} = \ol{B}(x,r)$,
        \item $B(x,r) = [\ol{B}(x,r)]^\circ$
    \end{enumerate}

    Deducir que $\Fr(B(x,r)) = \Fr(\ol{B}(x,r)) = S(x, r)$. ¿Son ciertos estos resultados en un espacio métrico cualquiera?
\end{ejercicio}

\begin{ejercicio}
    Para un intervalo $J \subset \bb{R}$, calcular los conjuntos $J^\circ,~\ol{J},~J'$ y $\Fr J$ .
\end{ejercicio}

\begin{ejercicio}
    En el espacio métrico $\bb{R}$ y para cada uno de los conjuntos $\bb{N}, \bb{Z}, \bb{Q}$ y $\bb{R}\setminus \bb{Q}$, calcular su interior y su cierre, sus puntos de acumulación, sus puntos aislados y su frontera.
\end{ejercicio}

\begin{ejercicio}
    Si un subconjunto $A$ de un espacio métrico $E$ verifica que $A'=\emptyset$, probar que la topología inducida por $E$ en $A$ es la discreta. ¿Es cierto el recíproco?
\end{ejercicio}

\begin{ejercicio}
    Sean $\{x_n\}$ e $\{y_n\}$ sucesiones convergentes en un espacio métrico $E$ con distancia $d$. Probar que la sucesión $\{d(x_n, y_n)\}$ es convergente y calcular su límite.
\end{ejercicio}


\begin{ejercicio}
    Sea $E = \prod\limits_{k=1}^N E_k$ un producto de espacios métricos y $A =\prod\limits_{k=1}^N A_k\subset~E$, donde $A_k \subset E_k$ para todo $k \in I_N$ . Probar que $A^\circ =\prod\limits_{k=1}^N A_k^\circ$ y $\ol{A} =\prod\limits_{k=1}^N \ol{A_k}$. Deducir que $A$ es un abierto de $E$ si, y sólo si, $A_k$ es un abierto de $E_k$ para todo $k \in \Delta_N$ , mientras que $A$ es un cerrado de $E$ si, y sólo si, $A_k$ es un cerrado de $E_k$ para todo $k \in \Delta_N$ .
\end{ejercicio}