\section{Compacidad y conexión}


\begin{ejercicio}
Probar que dos normas definidas en un mismo espacio vectorial, que dan lugar a los mismos conjuntos acotados, han de ser equivalentes.
\end{ejercicio}

\begin{ejercicio}
Dado un subconjunto \( A \) de un espacio normado, probar que las siguientes afirmaciones son equivalentes:
    \begin{enumerate}
        \item \( A \) está acotado.
        \item Si \( \{a_n\} \) es una sucesión de puntos de \( A \) y \( \{\lambda_n\} \) una sucesión de números reales tal que \( \{\lambda_n\} \rightarrow 0 \), entonces \( \{\lambda_n  a_n\} \rightarrow 0 \).
        \item Para toda sucesión \( \{a_n\} \) de puntos de \( A \) se tiene que \( \left\{\dfrac{a_n}{n}\right\} \rightarrow 0 \).
    \end{enumerate}
    ¿Significa esto que si \( \left\{\dfrac{a_n}{n}\right\} \rightarrow 0 \), entonces la sucesión \( \{a_n\} \) está acotada?
\end{ejercicio}

\begin{ejercicio}
Probar que todo espacio métrico finito es compacto. Probar también que, en un conjunto no vacío \( E \) con la distancia discreta, todo subconjunto compacto de \( E \) es finito.
\end{ejercicio}

\begin{ejercicio}
Sea \( \{x_n\} \) una sucesión convergente de puntos de un espacio métrico y \( x = \lim\limits_{n \to \infty} x_n \). Probar que el conjunto \( A = \{x_n : n \in \mathbb{N}\} \cup \{x\} \) es compacto.
\end{ejercicio}

\begin{ejercicio}
Probar que, si \( E \) es un espacio métrico compacto, y \( A \) es un subconjunto infinito de \( E \), entonces \( A' \neq \emptyset \).
\end{ejercicio}

\begin{ejercicio}
Sea \( E \) un espacio métrico con distancia \( d \) y \( K \) un subconjunto compacto de \( E \). Probar que, para cada \( x \in E \), existe un punto \( k_x \in K \) tal que \( d(x, k_x) \leq d(x, k) \) para todo \( k \in K \).
\end{ejercicio}

\begin{ejercicio}
Sea \( A \) un subconjunto cerrado de \( \mathbb{R}^N \) y consideremos en \( \mathbb{R}^N \) cualquier distancia \( d \) que genere la topología usual. Probar que, para todo \( x \in \mathbb{R}^N \), se puede encontrar un \( a_x \in A \), tal que \( d(x, a_x) \leq d(x, a) \) para todo \( a \in A \).
\end{ejercicio}

\begin{ejercicio}
Sea \( F \) un conjunto no vacío con la distancia discreta. Probar que si \( E \) es un espacio métrico conexo, toda función continua de \( E \) en \( F \) es constante.
\end{ejercicio}

\begin{ejercicio}
Probar que, en todo espacio métrico, el cierre de un conjunto conexo es conexo.
\end{ejercicio}

\begin{ejercicio}
Probar que el conjunto \( A = \{ (x,y) \in \mathbb{R}^2 : xy \geq 0 \} \) es conexo pero \( A^\circ \) no lo es.
\end{ejercicio}

\begin{ejercicio}
Sea \( X \) un espacio normado. Probar que para cualesquiera \( x, y \in X \setminus \{0\} \) se tiene
\[
    \left\| \frac{x}{\|x\|} - \frac{y}{\|y\|} \right\| \leq \frac{2\left\| x - y \right\|}{\|x\|}
\]
Deducir que la función \( \varphi : X \setminus \{0\} \rightarrow X \), dada por \( \varphi(x) = \dfrac{x}{\|x\|} \) para todo \( x \in X \setminus \{0\} \), es continua.
\end{ejercicio}

\begin{ejercicio}
Probar que, si \( X \) es un espacio normado de dimensión mayor que 1, el conjunto \( X \setminus \{0\} \) es conexo. Deducir que la \textit{esfera unidad} dada porla ecuación \( S(0,1) = \{ x \in X : \|x\| = 1 \} \) es un conjunto conexo.
\end{ejercicio}