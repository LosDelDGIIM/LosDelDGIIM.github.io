\section{Complitud y continuidad uniforme}

\begin{ejercicio}
    Probar que, en cualquier espacio métrico, toda sucesión de Cauchy está acotada.
    \end{ejercicio}
    
    \begin{ejercicio}
    Probar que todo espacio métrico compacto es completo.
    \end{ejercicio}
    
    \begin{ejercicio}
    Sea \( f : \mathbb{R} \rightarrow \mathbb{R} \) una función continua e inyectiva. Probar que definiendo
    \[
    \rho(x,y) = |f(x) - f(y)| \qquad \forall x, y \in \mathbb{R}
    \]
    se obtiene una distancia en \( \mathbb{R} \), equivalente a la usual. ¿Cuándo es \( \rho \) completa?
    \end{ejercicio}
    
    \begin{ejercicio}
    ¿Qué se puede afirmar sobre la composición de dos funciones uniformemente continuas?
    \end{ejercicio}
    
    \begin{ejercicio}
    Dado un espacio normado \( X \neq \{0\} \), probar que la función
    \Func{f}{X\setminus \{0\}}{X}{x}{\dfrac{x}{\|x\|}}
    no es uniformemente continua. Sin embargo, probar también que, para cada \( \delta \in \mathbb{R}^+ \), la restricción de \( f \) al conjunto \( \{ x \in X : \|x\| \geq \delta \} \) es una función lipschitziana.
    \end{ejercicio}
    
    \begin{ejercicio}
    Sea \( A \) un subconjunto no vacío de un espacio métrico \( E \) y la siguiente función:
    \Func{f}{E}{\mathbb{R}}{x}{\inf \{ d(x,a) : a \in A \}}
    Probar que \( f \) es no expansiva.
    \end{ejercicio}
    
    \begin{ejercicio}
    Dado \( y \in \mathbb{R}^N \), se define \( T_y \in L(\mathbb{R}^N, \mathbb{R}) \) usando el producto escalar en \( \mathbb{R}^N \):
    \[
    T_y(x) = \left( x | y \right) \hspace{1cm} \forall x \in \mathbb{R}^N 
    \]
    Calcular la norma de la aplicación lineal \( T_y \), considerando en \( \mathbb{R}^N \):
    \begin{enumerate}
        \item La norma euclídea
        \item La norma del máximo
        \item La norma de la suma
    \end{enumerate}
    \end{ejercicio}
    
    \begin{ejercicio}
    Consideremos los espacios normados \( X = \mathbb{R}^N \) con la norma de la suma y \( Y = \mathbb{R}^N \) con la norma del máximo. Denotando como siempre por \( \{ e_k : k \in \mathbb{N} \} \) a la base usual de \( \mathbb{R}^N \), probar que, para toda \( T \in L(X,Y) \), se tiene:
    \[
    \|T\| = \max \{ |\left(T(e_j)\mid e_k \right)| : j,k \in \Delta_n \}
    \]
    \end{ejercicio}
    