\section{Ecuaciones y sistemas}

% Ejercicio 1
\begin{ejercicio}
    En Teoría del Aprendizaje, se supone que la velocidad a la que se memoriza una materia es proporcional a la
    cantidad que queda por memorizar. Suponemos que \(M\) es la cantidad total de materia a memorizar y \(A(t)\) es la
    cantidad de materia memorizada a tiempo \(t\). Determine una ecuación diferencial para \(A(t)\). Encuentre soluciones
    de la forma \(A(t) = a + be^{\lambda t}\).\\

    Tras interpretar el enunciado, deducimos que:
    \begin{equation*}
        A' = c(M - A),
    \end{equation*}
    donde $c\in \bb{R}$ es la constante de proporcionalidad. Esta es la ecuación diferencial que buscamos.

    % // TODO: Terminar la segunda parte
\end{ejercicio}

% Ejercicio 8
\begin{ejercicio}
    Demuestre que si \(x(t)\) es una solución de la ecuación diferencial
    \begin{equation}\label{eq:ej1.8_1}
        x'' + x = 0,
    \end{equation}
    entonces también cumple, para alguna constante \(c \in \bb{R}\),
    \begin{equation}\label{eq:ej1.8_2}
        (x')^2 + x^2 = c.
    \end{equation}

    Encuentre una solución de $(x')^2 + x^2 = 1$ que no sea solución de \eqref{eq:ej1.8_1}.\\

    \begin{proof}
        Sea $I\subset \bb{R}$ el intervalo de definición de $x(t)$ solución de \eqref{eq:ej1.8_1}. Definimos la función auxiliar
        \Func{f}{I}{\bb{R}}{t}{(x'(t))^2 + x^2(t).}

        Por ser $x$ una solición de una ecuación diferencial de segundo orden, tenemos que $x\in C^2(I)$. Por tanto, $x,~x'\in C^1(I)$ y, por tanto $f$ es derivable. Calculamos su derivada:
        \begin{equation*}
            f'(t) = 2x'(t)x''(t) + 2x(t)x'(t) = 2x'(t)~\left[ x''(t) + x(t) \right] = 2x'(t)\cdot 0 = 0.
        \end{equation*}

        Por tanto, $f'(t)=0$ para todo $t\in I$, lo que implica que $f$ es constante en $I$. Es decir, existe $c\in \bb{R}$ tal que
        \begin{equation*}
            (x'(t))^2 + x^2(t) = c \quad \forall t\in I.
        \end{equation*}

        Por tanto, queda demostrado lo pedido.
    \end{proof}

    Para la segunda parte, sea la solución $x(t) = 1$ para todo $t\in \bb{R}$. Entonces, tenemos que:
    \begin{align*}
        (x'(t))^2 + x^2(t) &= 0^2 + 1^2 = 1,\\
        x''(t) + x(t) &= 0 + 1 = 1
    \end{align*}
\end{ejercicio}
