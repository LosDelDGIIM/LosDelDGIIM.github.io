\chapter{Ejercicios}
% // TODO: Seccion de "algunos ejercicios" son de examenes finales
\begin{ejercicio} % 49
    Sea $F$ cuerpo de descomposición de $f=x^3+x+1\in \bb{F}_2[x]$ y $\alpha\in F$ raíz de $f$. Razonar que $F=\bb{F}_2(\alpha)$. Resolver en $F$ las soluciones de las ecuaciones en función de $\alpha$:
    \begin{equation*}
        x^3+x+1=0,\qquad x^3+x^2+1=0,\qquad x^2+x+1=0
    \end{equation*}

    \noindent
    \textbf{Solución.}\newline
    Para ver que $F=\bb{F}_2(\alpha)$ basta ver que $f$ es irreducible, por lo que entonces $F\leq \bb{F}_2(\alpha)$ será de Galois y lo tenemos. Comprobamos que $f$ es irreducible, ya que se tiene $f(0) = f(1) = 1\neq 0$, y $deg f = 3$.

    \noindent
    Como $\alpha$ es raíz de $f$ y $\bb{F}_2\leq F$ es de Galois, entonces todas las raíces de $f$, que son, $\alpha,\alpha^2, \alpha^4 \in F$, tenemos entonces que $F=\bb{F}_2(\alpha)$. Puede razonarse también por el orden de los cuerpos, ambos es 8.
    \begin{enumerate}
        \item Las soluciones de la primera ecuación son trivialmente $\alpha,\alpha^2$ y $\alpha^4$.
        \item Para la segunda, vamos a repetir un argumento análogo al último ejemplo, es decir, resolver la ecuación en $\bb{F}_8$, y ya hemos descartado 5 raíces (3+2). Como $\bb{F}_8^\times$ tiene por generador $\alpha$, las raíces de $x^3+x^2+1$ son $\alpha^3,\alpha^5$ y $\alpha^6$.

            Esto lo sabemos viendo la factorización de $x^8+x\in \bb{F}_2[x]$ como el producto:
            \begin{equation*}
                x^8 + x = x(x+1)(x^3+x+1)(x^3+x^2+1)
            \end{equation*}
        \item Para la ecuación $x^2+x+1=0$, veamos que no tiene solución en $\bb{F}_8$, ya que:
            \begin{itemize}
                \item Como hemos gastado todas las raíces, tendría que ser una de ellas y \ldots
                \item El polinomio no está en la factorización de $x^8+x$.
                \item Si tuviera solución podríamos tener entonces:
                    \begin{equation*}
                        \bb{F}_2\leq \bb{F}_4\leq \bb{F}_8
                    \end{equation*}
                    Pero no puede ser $\bb{F}_{2^2} \leq \bb{F}_{2^3}$, ya que $2\nmid 3$.
            \end{itemize}
    \end{enumerate}
\end{ejercicio}

\begin{ejercicio} % 51
    Consideramos $f=x^3+x+1\in \bb{F}_4[x]$ y tomamos $K$ cuerpo de descomposición de $f$. Sea $\alpha\in K$ una raíz de $f$, veamos que $K = \bb{F}_4(\alpha)$.\\

    \noindent
    Para ello, usaremos que $\bb{F}_4\leq K$ es de Galois (puesto que son cuerpos finitos), como $f$ es irreducible en $\bb{F}_4[x]$, por varios motivos tenemos que $K = \bb{F}_4(\alpha)$. Sabemos que $f$ es irreducible en $\bb{F}_4[x]$ porque no tienes raíces en $\bb{F}_4$:
    \begin{itemize}
        \item Una opción es tomar $\bb{F}_4 = \{0,1,\gamma,\gamma+1\}$ y comprobarlo.
        \item Otra es por reducción al absurdo, suponer que sí, que $\gamma\in \bb{F}_4$ es una raíz de $f$. Vemos claramente que $\gamma\notin \bb{F}_2$, puesto que $f$ es irreducible en $\bb{F}_2$. De esta forma, observamos que $\bb{F}_4\leq \bb{F}_2(\gamma)$, con lo que $g=\Irr(\gamma,\bb{F}_2) = x^2+x+1\in \bb{F}_2[x]$. Tenemos que $g(\gamma) = 0 = f(\gamma)$, con $g,f\in \bb{F}_2[x]$, por lo que (por definición de $\Irr(\gamma,\bb{F}_2)$) $g\mid f$, pero ambos son irreducibles, por lo que hemos llegado a una contradicción. 
    \end{itemize}
    Por tanto, como $[K:\bb{F}_4] = 3$ y $|\bb{F}_4| = 4$, tenemos que $|K| = 4^3 = {(2^2)}^{3} = 64$.\\

    \noindent
    Nos preguntamos ahora cómo resolver las ecuaciones:
    \begin{equation*}
        x^3+x+1=0, \qquad x^3+x^2+1=0, \qquad x^2+x+1=0
    \end{equation*}
    \begin{description}
        \item [Si no nos acordamos del ejercicio anterior.] Como una solución es $\alpha$ y consideramos el homomorfismo de Frobenius de la extensión (elevar a 4), tenemos entonces que sus raíces en $K$ son $\alpha,\alpha^4$ y $\alpha^{16}$.
        \item [Si nos acordamos del ejercicio de ayer.] Habíamos resuelto la ecuación en $\bb{F}_2$, y es claro que $F=\bb{F}_2(\alpha) \leq K$, con $|\bb{F}_2(\alpha)| = 8$. Las soluciones que encontrábamos eran $\alpha,\alpha^2$ y $\alpha^4$ en $F\leq K$.

            Parece que no coinciden. Sin embargo, $\alpha^{16} = \alpha^2$, porque $\alpha^7 = 1$.
    \end{description}
    De la misma forma y usando el ejercicio anterior, tenemos que las soluciones de la segunda ecuación son $\alpha^3,\alpha^5$ y $\alpha^6$.
    \begin{description}
        \item [Si no nos acordamos.] Podríamos haber pensado en considerar $\bb{F}_2(\alpha)\leq K$ y resolver $f$ en $\bb{F}_2(\alpha)$, que es lo que hicimos en el ejercicio anterior.
    \end{description}
    Para la última ecuación, tomamos $\gamma\in \bb{F}_4\setminus \bb{F}_2$, de donde $\gamma$ es solución de $x^2+x+1=0$, ya que es el único polinomio irreducible de grado 2 sobre $\bb{F}_2[x]$.\\

    \noindent
    Finalmente, se nos pide hacer una base de $K$ sobre $\bb{F}_2$ usando $\alpha$ y $\gamma$.\\

    \noindent
    Para ello, lo que haremos es ver que:
    \begin{equation*}
        \bb{F}_2\leq F \leq K
    \end{equation*}
    con $\{1,\gamma\}$ una base de $\bb{F}_2\leq F$ y $\{1,\alpha,\alpha^2\}$ una base de $F\leq K$. Si repasamos la demostración del Lema de la Torre, tenemos que:
    \begin{equation*}
        \{1,\alpha,\alpha^2,\gamma,\gamma\alpha, \gamma\alpha^2\}
    \end{equation*}
    es una base de $\bb{F}_2\leq K$.\\

    \noindent
    ¿Cuál es el orden multiplicativo de $\gamma\alpha$? ¿Es un elemento primitivo de $K$ sobre $\bb{F}_2$?

    % // TODO: Se toma F_{64}^times, por lo que los ordene son 3, 7 o 63 (63 = 7 * 3²)
\end{ejercicio}

\begin{ejercicio}
    ¿Cuántos polinomios irreducibles de grado 6 hay en $\bb{F}_2[x]$?\\

    % Se puede hacer por algebra 1, listando todos los polinomios (hacerlo de forma inteligente para no hacer mucho) y comprobar que no es producto de grado 2 ni de grado 3
    \noindent
    Usando Álgebra III, estos polinomios aparecen en la descomposición del polinomio $x^{64}-x$ en irreducibles ($64=2^6$, con $Div(6) = \{1,2,3,6\}$)
    \begin{equation*}
        x^{64}-x = x(x+1)(x^2+x+1)(x^3+x+1)(x^3+x^2+1)\prod \text{polinomios grado 6}
    \end{equation*}
    Como la suma de los grados tiene que ser $64$, el último polinomio (el producto) tiene grado:
    \begin{equation*}
        64 - 10 = 54, \qquad \frac{54}{6} = 9
    \end{equation*}
    Por lo que hay $9$ polinomios irreducibles de grado $6$ en $\bb{F}_2[x]$.
\end{ejercicio} % // TODO: probar con distintos cardinales, que a lo mejor cae en examen

\begin{ejercicio} % 53
    Calcular el cardinal del grupo de Galois sobre $\mathbb{Q}$ del polinomio $f=(x^3+x+1)(x^2+1)$.\\

    \noindent
    Sea $K$ el cuerpo de descomposición de $f\in \mathbb{Q}[x]$. Las raíces del segundo están claras pero las del primero no (sabemos calcularlas pero es una cuesta llena de pinchos). Del primero sabemos que bien tiene 3 raíces reales o bien 1 raíz real y 2 complejas. Sea $g$ la función $g(x) = x^3+x+1$, tenemos que:
    \begin{equation*}
        g'(x) = 3x^2+1 > 0
    \end{equation*}
    Por lo que $g$ es estrictamente creciente, por lo que solo tiene una raíz real. Será por tanto:
    \begin{equation*}
        K = \mathbb{Q}(i,-i,r,\alpha,\overline{\alpha}) = \mathbb{Q}(i,r,\alpha)
    \end{equation*}
    donde $r\in \mathbb{R}$, $\alpha\in \mathbb{C}\setminus \mathbb{R}$ con $r,\alpha$ raíces de $x^3+x+1$. Como $\mathbb{Q}\leq K$ es de Galois, nos piden $|\Aut_\mathbb{Q}(K)| = [K:\mathbb{Q}]$. Por el Lema de la Torre:
    \begin{equation*}
        [K:\mathbb{Q}] = [K:\mathbb{Q}(i)][\mathbb{Q}(i):\mathbb{Q}] = [K:\mathbb{Q}(i)]\cdot 2
    \end{equation*}
    Para calcular $[K:\mathbb{Q}(i)]$ vamos a usar que las raíces de $x^3+x+1$ están en $K$. ¿Es cierto que $K$ es cuerpo de descomposición de $g=x^3+x+1\in \mathbb{Q}(i)[x]$? En efecto, si $g$ fuera irreducible en $\mathbb{Q}(i)$ entonces a partir del discriminante de $g$ (si está en $\mathbb{Q}$ o no) tendríamos que $[K:\mathbb{Q}(i)]$ es el cardinal bien de $A_3$ o de $S_3$.\\

    \noindent
    Para ver que $g$ es irreducible en $\mathbb{Q}(i)[x]$ veamos que no tiene raíces en $\mathbb{Q}(i)$: % // TODO: EN los apuntes hay otra forma de resolverlo
    \begin{itemize}
        \item Si $r$ es raíz de $g$ en $\mathbb{Q}(i)$, entonces $r\in \mathbb{Q}$, por lo que es raíz de $x^3+x+1$, pero $x^3+x+1$ es irreducible en $\mathbb{Q}$.
        \item Si $\alpha\in \mathbb{Q}(i)$, como $\alpha\notin \mathbb{Q}$, tendremos entonces que $\alpha$ tiene grado 2 sobre $\mathbb{Q}$, es decir, que $\Irr(\alpha,\mathbb{Q})$ tiene grado 2, y teníamos que $\alpha$ era raíz de $g$, por lo que $\Irr(\alpha,\mathbb{Q})\mid g$, pero $g$ es irreducible, lo que lleva a una contradicción.
        \item Como $\alpha\notin \mathbb{Q}(i)$ tenemos entonces que $\overline{\alpha}\notin \mathbb{Q}(i)$.
    \end{itemize}
    De aquí tenemos que $g$ es irreducible en $\mathbb{Q}(i)$, por lo que su grupo de Galois será bien $A_3$ o $S_3$, en función del discriminante:
    \begin{equation*}
        \Disc(g) = -31
    \end{equation*}
    Por lo que:
    \begin{equation*}
        \Delta = \sqrt{\Disc(g)} = i\sqrt{31} \notin\mathbb{Q}(i)
    \end{equation*}
    Ya que si $i\sqrt{31}\in \mathbb{Q}(i)$  tendríamos entonces que $\sqrt{31}\in \mathbb{Q}(i)$, pero $\sqrt{31}\in \mathbb{R}\setminus \mathbb{Q}$, ya que $31$ es primo ($x^2-31$ es irreducible por Eisenstein para $p=31$).

    \noindent
    En definitiva, como $\Delta\notin \mathbb{Q}(i)$ tenemos entonces que $\Aut_{\mathbb{Q}(i)}(K)\cong S_3$, con $|S_3| = 6$, de donde:
    \begin{equation*}
        [K:\mathbb{Q}] = [K:\mathbb{Q}(i)][\mathbb{Q}(i):\mathbb{Q}] = 6\cdot 2 = 12
    \end{equation*}~\\

    \noindent
    Describir cuál es el grupo $\Aut_\mathbb{Q}(K)$.\\ 

    \noindent
    La conexión de Galois nos dice que hay un subgrupo normal, puesto que su índice es 2. 

    \noindent
    Usamos el Teorema de extensión, por lo que estará $\eta_0,\eta_1,\eta_2$:
    \begin{equation*}
        r \stackrel{\eta_0}{\longmapsto} r ,\qquad
        r \stackrel{\eta_1}{\longmapsto} \alpha ,\qquad
        r \stackrel{\eta_2}{\longmapsto} \overline{\alpha}
    \end{equation*}
    Cada uno de estos lo extendemos llevando $i$ a $i$ o a $-i$. El último paso parece más difícil. % // TODO: ¿Puedo calcular el grupo de automorfismos? Como son sus elementos? Es un problema algo dificil
\end{ejercicio}

\begin{ejercicio}
    Calcular el grupo de Galois de $f=(x^2+x+1)(x^2-3)\in \mathbb{Q}[x]$.\\

    \noindent
    Sea $K$ el cuerpo de descomposición de $f$, tenemos que:
    \begin{equation*}
        K = \mathbb{Q}\left(w,\sqrt{3}\right), \qquad w \text{\ una raíz cúbica primitiva de la unidad}
    \end{equation*}
    Por el Lema de la Torre:
    \begin{equation*}
        [K:\mathbb{Q}] = \left[K:\mathbb{Q}\left(\sqrt{3}\right)\right] \left[\mathbb{Q}\left(\sqrt{3}\right) :\mathbb{Q}\right] = 2\cdot 2 = 4
    \end{equation*}
    Por lo que:
    \begin{equation*}
        |\Aut(K)| = [K:\mathbb{Q}] = 4
    \end{equation*}
    Calculamos sus elementos, por la Proposición de extensión. Calculamos primero los homomorfismos de cuerpos $\mathbb{Q}\left(\sqrt{3}\right)\to K$, que son $\eta_j$ con $j \in \{0,1\}$, con:
    \begin{equation*}
        \eta_j\left(\sqrt{3}\right) = {(-1)}^{j}\sqrt{3}, \qquad j \in \{0,1\}
    \end{equation*}
    Extendemos cada uno de estos homomorfismos, cada uno de ellos se extiende a dos homomorfismos $K\to K$, que son $\eta_{j,k}$, donde:
    \begin{equation*}
        \eta_{j,k}\left(\sqrt{3}\right) = {(-1)}^{j}\sqrt{3}, \qquad \eta_{j,k}(w) = w^k, \qquad j\in \{0,1\},\quad  k \in \{1,2\}
    \end{equation*}
    Como $\eta_{1,1}$ y $\eta_{1,2}$ tienen orden 2 tenemos que el grupo es isomorfo a $C_2\oplus C_2$.
\end{ejercicio}

\begin{ejercicio}
    Calcular el grupo de Galois de $g=(x^2+x+1)(x^2+3)\in \mathbb{Q}[x]$.\\

    \noindent
    Sea $K$ el cuerpo de descomposición de $g$, si tomamos como raíz cúbica primitiva de la unidad:
    \begin{equation*}
        w = \frac{-1}{2} + i\frac{\sqrt{3}}{2}
    \end{equation*}
    Tendremos ahora que:
    \begin{equation*}
        K = \mathbb{Q}\left(i\sqrt{3}\right)
    \end{equation*}
    Con lo que $|\Aut_\mathbb{Q}(K)| = [K:\mathbb{Q}] = 2$, isomorfo a $C_2$. Sus elementos son $\eta_j$, donde:
    \begin{equation*}
        \eta_j\left(i\sqrt{3}\right) = {(-1)}^{j}i\sqrt{3} \qquad j \in \{0,1\}
    \end{equation*} % // TODO: Esto prueba que un mismo cuerpo de descomposicion puede servir para dos polinomios irreducibles distintos, hay mas polinoimos que cuerpos de descomposicion
\end{ejercicio}

\begin{ejercicio} % 55
    Sea $f=x^3-3x+1\in \mathbb{Q}[x]$ y $\alpha$ cualquier raíz real de $f$. Demostrar que el cuerpo de descomposición de $f$ sobre $\mathbb{Q}$ es $\mathbb{Q}(\alpha)$.\\

    \noindent
    Veamos que $f$ es irreducible, puesto que es de grado 3 y sus únicas posibles raíces en $\mathbb{Q}$ son $\pm 1$, y no lo son; luego $f$ es irreducible. De aquí deducimos que el grupo de Galois de la extensión $\mathbb{Q}\leq K$ es un subgrupo transitivo de $S_3$, luego es $A_3$ o $S_3$.\\

    \noindent
    Calculamos ahora:
    \begin{equation*}
        \Disc(f) = -4{(-3)}^{3}-27 = 81 = 9^2 \quad\Longrightarrow\quad \Delta = \sqrt{\Disc(f)} = 9 \in \mathbb{Q}
    \end{equation*}
    por lo que el grupo de Galois de $f$ es $A_3$, de donde:
    \begin{equation*}
        [K:\mathbb{Q}] = |A_3| = 3
    \end{equation*}
    por lo que por el Lema de la Torre tenemos que $\mathbb{Q}\leq \mathbb{Q}(\alpha)\leq K$, por lo que $\mathbb{Q}(\alpha) = K$, ya que $[\mathbb{Q}(\alpha):\mathbb{Q}] = 3$ por ser $f=\Irr(\alpha,\mathbb{Q})$ con $grd(f) = 3$.
\end{ejercicio} % // TODO: Apuntar bien en alguna parte el discriminante de la cubica

\begin{ejercicio} % 56
    Sea $K$ el cuerpo de descomposición de $f=(x^2+3)(x^3-3)\in \mathbb{Q}[x]$. Calcular todos los subcuerpos de $K$. Demostrar que $\mathbb{Q}\left(\sqrt[3]{3} + i\sqrt{3}\right) = K$.\\

    \noindent
    Las raíces de $f$ son:
    \begin{equation*}
        \pm i\sqrt{3}, \quad \sqrt[3]{3}, \quad w\sqrt[3]{3},\quad w^2\sqrt[3]{3}, \qquad w = \frac{-1}{2} + i\frac{\sqrt{3}}{2}
    \end{equation*}
    con lo que $K = \mathbb{Q}\left(i\sqrt{3}, \sqrt[3]{3}, w\sqrt[3]{3}, w^2\sqrt[3]{3}\right)$. Como $w\in \mathbb{Q}\left(i\sqrt{3}\right)$, tenemos pues que:
    \begin{equation*}
        K = \mathbb{Q}\left(i\sqrt{3},\sqrt[3]{3}\right)
    \end{equation*}
    y usando el Lema de la Torre obtenemos que:
    \begin{equation*}
        [K:\mathbb{Q}] = \left[K:\mathbb{Q}\left(\sqrt[3]{3}\right)\right]\left[\mathbb{Q}\left(\sqrt[3]{3}\right):\mathbb{Q}\right] = 2\cdot 3 = 6
    \end{equation*}
    Donde usamos:
    \begin{itemize}
        \item $\Irr\left(\sqrt[3]{3},\mathbb{Q}\right) = x^3-3$ por Eisenstein.
        \item $\Irr\left(i\sqrt{3},\mathbb{Q}\left(\sqrt[3]{3}\right)\right) = x^2+3$, ya que sus raíces son complejas.
    \end{itemize}
    Aplicamos la proposición de extensión, primero a $\mathbb{Q}\leq \mathbb{Q}\left(\sqrt[3]{3}\right)$ para determinar los homomorfismos de cuerpos $\mathbb{Q}\left(\sqrt[3]{3}\right)\to K$, que están en biyección con las raíces de $\Irr\left(\sqrt[3]{3},\mathbb{Q}\right) = x^3-3$ en $K$. Así, aquellos son $\eta_j:\mathbb{Q}\left(\sqrt[3]{3}\right)\to K$ determinados por:
    \begin{equation*}
        \eta_j\left(\sqrt[3]{3}\right) = w^j \sqrt[3]{3}, \qquad j \in \{0,1,2\}
    \end{equation*}
    Aplicando de nuevo la proposición citada a la extensión $\mathbb{Q}\left(\sqrt[3]{3}\right)\leq K = \mathbb{Q}\left(\sqrt[3]{3}\right)\left(i\sqrt{3}\right)$, puesto que $\Irr\left(i\sqrt{3},\mathbb{Q}\left(\sqrt[3]{3}\right)\right) = x^2+3$, obtenemos los homomorfismos $\eta_{j,k}:K\to K$ determinados por:
    \begin{equation*}
        \eta_{j,k}\left(\sqrt[3]{3}\right) = w^j \sqrt[3]{3}, \qquad \eta_{j,k} = {(-1)}^{k}\sqrt{3}, \qquad j \in \{0,1,2\}, \quad k\in \{0,1\}
    \end{equation*}
    De esta forma:
    \begin{equation*}
        \Aut(K) = \{\eta_{j,k} : j=0,1,2, \ k = 0,1\}
    \end{equation*}
    Calculamos los órdenes de los elementos:
    \begin{table}[H]
    \centering
    \begin{tabular}{cccccc}
        $\eta_{0,0} $& $\eta_{0,1} $& $\eta_{1,0} $& $\eta_{1,1} $& $\eta_{2,0} $& $\eta_{2,1} $\\
        \hline
        1 & 2 & 3 & 2 & 3 & 2
    \end{tabular}
    \end{table}
    \noindent
    Por ejemplo:
    \begin{align*}
        \eta_{1,1}\left(\eta_{1,1}\left(\sqrt[3]{3}\right)\right) &= \eta_{1,1}\left(w\sqrt[3]{3}\right) = \eta_{1,1}(w)\eta_{1,1}\left(\sqrt[3]{3}\right) = \overline{w}w \sqrt[3]{3} = \sqrt[3]{3} \\
        \eta_{1,1}\left(\eta_{1,1}\left(i\sqrt{3}\right)\right) &= \eta_{1,1}\left(-i\sqrt{3}\right) = i\sqrt{3}
    \end{align*}
    Y ya sabemos en este punto los subgrupos que tenemos de $\Aut(K)$. 
    \begin{itemize}
        \item Usando la conexión de Galois, $K^{\langle \eta_{1,0} \rangle }$ ha de ser una extensión de grado 2 de $\mathbb{Q}$. Como $i\sqrt{3}\in K^{\langle \eta_{1,0} \rangle }$:
            \begin{equation*}
                \eta_{1,0}\left(i\sqrt{3}\right) = i\sqrt{3}
            \end{equation*}
            resulta que $\mathbb{Q}\left(i\sqrt{3}\right)\leq K^{\langle \eta_{1,0} \rangle }$ con $\left[\mathbb{Q}\left(i\sqrt{3}\right):\mathbb{Q}\right] = 2$, por lo que:
            \begin{equation*}
                \mathbb{Q}\left(i\sqrt{3}\right) = K^{\langle \eta_{1,0} \rangle }
            \end{equation*}
        \item Observemos que $\eta_{0,1}\left(\sqrt[3]{3}\right) = \sqrt[3]{3}$, de donde se deduce que $\mathbb{Q}\left(\sqrt[3]{3}\right)\leq K^{\langle \eta_{0,1} \rangle }$ con:
            \begin{equation*}
                \left[\mathbb{Q}\left(\sqrt[3]{3}\right):\mathbb{Q}\right] = 3
            \end{equation*}
            por lo que $\mathbb{Q}\left(\sqrt[3]{3}\right) = K^{\langle \eta_{0,1} \rangle }$.
        \item Si vemos que:
            \begin{equation*}
                \eta_{1,1}\left(w\sqrt[3]{3}\right) = \eta_{1,1}(w) \eta_{1,1}\left(\sqrt[3]{3}\right) = \overline{w}w\sqrt[3]{3} = \sqrt[3]{3}
            \end{equation*}
            no es fijo, probamos ahora con:
            \begin{equation*}
                \eta_{2,1}\left(w\sqrt[3]{3}\right) = \eta_{2,1}(w)\eta_{2,1}\left(\sqrt[3]{3}\right) = \eta_{2,1}(w) w^2\sqrt[3]{3} = w^2w^2\sqrt[3]{3} = w\sqrt[3]{3}
            \end{equation*}
            Y concluimos que $\mathbb{Q}\left(w\sqrt[3]{3}\right) = K^{\langle \eta_{2,1} \rangle }$.
        \item De forma análoga, observaremos que:
            \begin{equation*}
                \mathbb{Q}\left(w^2\sqrt[3]{3}\right) = K^{\langle \eta_{1,1} \rangle }
            \end{equation*}
    \end{itemize}
    Para ver que:
    \begin{equation*}
        K = \mathbb{Q}\left(\sqrt[3]{3}+i\sqrt{3}\right) 
    \end{equation*}
    Obviamente tenemos que $E=\mathbb{Q}\left(\sqrt[3]{3}+i\sqrt{3}\right)\leq K = \mathbb{Q}\left(\sqrt[3]{3}+i\sqrt{3}\right)$. Para ver la igualdad, la estrategia que seguimos es ver que este subcuerpo no es ninguno de los ya mencionados, que son todos los posibles, descartando trivialmente $\mathbb{Q}$, por lo que tendrá que ser igual a $K$.
    \begin{itemize}
        \item Como $\mathbb{Q}\left(\sqrt[3]{3}\right)\leq \mathbb{R}$ este tampoco puede ser, al igual que $\mathbb{Q}$.
        \item Para $\mathbb{Q}\left(w\sqrt[3]{3}\right) = K^{\langle \eta_{2,1} \rangle }$:
            \begin{equation*}
                \eta_{2,1}\left(\sqrt[3]{3}+i\sqrt{3}\right) = \eta_{2,1}\left(\sqrt[3]{3}\right) + \eta_{2,1}\left(i\sqrt{3}\right) = w^2 \sqrt[3]{3} - i\sqrt{3}
            \end{equation*}
            Si fuese $E=\mathbb{Q}\left(w\sqrt[3]{3}\right)$ entonces tendríamos que:
            \begin{equation*}
                \sqrt[3]{3}+i\sqrt{3} = w^2\sqrt[3]{3}-i\sqrt{3}
            \end{equation*}
            Y comparando partes reales (recordemos que $w^2 = \overline{w}$), obtenemos:
            \begin{equation*}
                \sqrt[3]{3} = -\frac{1}{2}\sqrt[3]{3}
            \end{equation*}
            lo que es una contradicción.
        \item De forma análoga se descartan todos.
    \end{itemize}
\end{ejercicio}

\begin{ejercicio} % 62
    Sean $\sqrt{2},\sqrt[3]{2}\in \mathbb{R}$:
    \begin{enumerate}
        \item Calcular razonadamente $\left[\mathbb{Q}\left(\sqrt{2},\sqrt[3]{2}\right):\mathbb{Q}\right]$.

            Sale 6.
        \item Decidir razonadamente si $K=\mathbb{Q}\left(\sqrt{2},\sqrt[3]{2},i\sqrt{3}\right)$ es una extensión de Galois de $\mathbb{Q}$.

            Vamos a tratar de ver si $K$ es cuerpo de descomposición de un polinomio. Candidato a $f$ para que $K$ sea cuerpo de descomposición de $f$:
            \begin{equation*}
                f = (x^2-2)(x^3-2)
            \end{equation*}
            así metemos las dos primeras, y el cuerpo de descomposición de $f$ $E$ debe contener también las raíces cúbicas primitivas de la unidad, entre ellas:
            \begin{equation*}
                w = -\frac{1}{2}+i\frac{\sqrt{3}}{2}
            \end{equation*}
            Y tenemos así que $E = \mathbb{Q}\left(\sqrt{2},\sqrt[3]{2},w\right)$ y tenemos que $\mathbb{Q}\leq E$ es de Galois, por ser un cuerpo de descomposición de un polinomio en característica $0$. En vista de la forma de $w$, tenemos que $E = K$, por lo que $\mathbb{Q}\leq K$ es de Galois.
        \item Calcular $\Aut(K)$ definiendo explícitamente todos sus elementos.

            Buscamos primero $[K:\mathbb{Q}]$, que por el Lema de la Torre: 
            \begin{equation*}
                [K:\mathbb{Q}] = \left[K:\mathbb{Q}\left(\sqrt{2},\sqrt[3]{2}\right)\right]\left[\mathbb{Q}\left(\sqrt{2},\sqrt[3]{2}\right)\right] = 2\cdot 6 = 12
            \end{equation*}
            Y aplicamos ahora en reiteradas ocasiones la Proposición de extensión, para hayar todos los elementos de $\Aut(K)$, obteniendo:
            \begin{align*}
                \eta_{j,k,l}\left(\sqrt{2}\right) &= {(-1)}^{j}\sqrt{2} \\
                \eta_{j,k.l}\left(\sqrt[3]{2}\right) &= w^k\sqrt[3]{2} \\
                \eta_{j,k,l}\left(i\sqrt{3}\right) &= {(-1)}^{l}i\sqrt{3} \\
                                                   &j \in \{0,1\},\quad  k \in \{0,1,2\},\quad  l\in \{0,1\}
            \end{align*}
            con todos estos números en $K$, el polinomio $x^2-2$ es irreducible sobre $\mathbb{Q}$, el polinomio $x^3-2$ es irreducible sobre $\mathbb{Q}\left(\sqrt{2}\right)$ por el apartado 2.
        \item Calcular razonadamente el grado del polinomio
            \begin{equation*}
                f = \Irr\left(\sqrt{2}+\sqrt[3]{2},\mathbb{Q}\right)
            \end{equation*}
            % Calcular el polinomio es fácil pero luego comprobar si es irreducible no lo es.
            % Trataremos usar la conexión de galois, llevando grado de subcuerpo en índice de subgrupo
            Como $deg \Irr(\alpha,\mathbb{Q}) = [\mathbb{Q}(\alpha):\mathbb{Q}] = (\Aut_{\mathbb{Q}}(K) : \Aut_{\mathbb{Q}(\alpha)}(K)) = \frac{|\Aut_\mathbb{Q}(K)|}{|\Aut_{\mathbb{Q}(\alpha)}(K)|}$. Vemos que:
            \begin{equation*}
                \Aut_{\mathbb{Q}(\alpha)}(K) \supseteq \{\eta_{0,0,0},\eta_{0,0,1}\}
            \end{equation*}
            ya que estos elementos dejan fijo el elemento $\sqrt{2}+\sqrt[3]{2}$. Veamos ahora que $\alpha$ no se queda fijo por el resto de automorfismos:
            \begin{equation*}
                \eta_{j,k,l}(\alpha) = \eta_{j,k,l}\left(\sqrt{2}\right) + \eta_{j,k,l}\left(\sqrt[3]{2}\right) = {(-1)}^{j}\sqrt{2} + w^k \sqrt[3]{2} 
            \end{equation*}
            Tendremos que $\eta_{j,k,l}(\alpha) = \alpha$ si y solo si:
            \begin{equation*}
                \sqrt{2}+{(-1)}^{j+1}\sqrt{2} = (w^k-1)\sqrt[3]{2}
            \end{equation*}
            como el de la izquierda es un número real, tiene que ser $w^k-1\in \mathbb{R}$, y esto sucede si y solo si $k=0$. Por tanto tenemos que:
            \begin{equation*}
                \sqrt{2}+{(-1)}^{j+1}\sqrt{2} = 0 
            \end{equation*}
            de donde tiene que ser $j+1$ impar con $j\in \{0,1\}$, luego tiene que ser $j=0$. Así tenemos que:
            \begin{equation*}
                \Aut_{\mathbb{Q}(\alpha)}(K) = \{\eta_{0,0,0}, \eta_{0,0,1}\}
            \end{equation*}
            Por lo que:
            \begin{equation*}
                deg\Irr(\alpha,\mathbb{Q}) = [\mathbb{Q}(\alpha):\mathbb{Q}] = \frac{|\Aut_\mathbb{Q}(K)|}{|\Aut_{\mathbb{Q}(\alpha)}(K)|} = \frac{12}{2} = 6
            \end{equation*}
        \item Decidir razonadamente quién es el grupo de Galois de $f$. ¿Es $f$ resoluble por radicales? ¿Son las raíces complejas de este polinomio construibles por regla y compás?

            Es de orden 12 porque $\mathbb{Q}(\alpha) = \mathbb{Q}\left(\sqrt{2},\sqrt[3]{2}\right)$. Otra forma de verlo es tomando $E$ el cuerpo de descomposición de $f$, $\mathbb{Q}\leq K$ es de Galois y si la extensión contiene una raíz las tiene todas, por lo que $\mathbb{Q}(\alpha)\leq E\leq K$. Como es un cuerpo de descomposición, tendremos que $\mathbb{Q}\leq E$ es de Galois. Si un polinomio irreducible tiene una raíz en $E$ tiene que tenerlas todas, como por ejemplo $x^3-2$. Tiene que ser $E = K$. 

            Como todo grupo de orden 12 es resoluble tenemos que $f$ es resoluble por radicales, por ser su grupo de Galois resoluble.

            Una raíz es del grado de $deg\Irr(\alpha,\mathbb{Q}) = 6$, y todas estas tienen el mismo grado, que no es potencia de 2, por lo que ninguna de las raíces es constructible.
    \end{enumerate}
\end{ejercicio}

\begin{ejercicio} % 59
    Sea $K$ el cuerpo de descomposición de $f=x^6-3\in \mathbb{Q}[x]$.
    \begin{enumerate}[label=\alph*)]
        \item Calcular $[K:\mathbb{Q}]$.

            Sea $w$ una raíz sextra primitiva de la unidad, tenemos por un Teorema que $K = \mathbb{Q}\left(\sqrt[6]{3},w\right)$. Podemos calcular el $w$ con razones trigonométricas o con el sexto polinomio ciclotómico, que tiene grado 2:
            \begin{equation*}
                \phi_6 = \frac{x^6-1}{\phi_1 \phi_2 \phi_3} = \frac{x^6-1}{(x-1)(x+1)(x^2+x+1)}
            \end{equation*}
            Dividimos:
            \begin{equation*}
                x^6-1 = (x-1)(x+1)(4+x^2+1) = (x^2-1)(x^4+x^2+1) = \phi_1\phi_2\phi_3 (x^2-x+1)
            \end{equation*}
            Si lo hacemos por otra parte:
            \begin{equation*}
                \phi_6 = (x-w)(x-w^5) = (x-w)(x-\overline{w}) = x^2 - 2Rew + 1
            \end{equation*}
            tenemos que usar por tanto trigonometría, con $\cos(30) = \nicefrac{1}{2}$. De aquí:
            \begin{equation*}
                w = \frac{1\pm \sqrt{-3}}{2} = \frac{1}{2} \pm i\frac{\sqrt{3}}{2}
            \end{equation*}
            tomamos $w = \frac{1}{2}+i\frac{\sqrt{3}}{2}$. Y obtenemos así que:
            \begin{equation*}
                K = \mathbb{Q}\left(\sqrt[6]{3},w\right) = \mathbb{Q}\left(\sqrt[6]{3},i\sqrt{3}\right)
            \end{equation*}
            Por el Lema de la Torre obtenemos fácilmente que $[K:\mathbb{Q}] = 2\cdot 6 = 12$.
        \item Demostrar que $i+\sqrt{3}\in K$.

            Tenemos que:
            \begin{equation*}
                \sqrt{3} = {\left(\sqrt[6]{3}\right)}^{3} \in K
            \end{equation*}
            Por lo que:
            \begin{equation*}
                i = \frac{i\sqrt{3}}{\sqrt{3}} \in K
            \end{equation*}
            de donde $i+\sqrt{3}\in K$. 
         \item Calcular, definiendo explícitamente todos sus elementos, el grupo $G= \Aut_{\mathbb{Q}(i+\sqrt{3})}(K)$.

             Del apartado anterior vemos que $K = \mathbb{Q}\left(\sqrt[6]{3},i\right)$, y aplicando la Proposición de extensión vemos que:
             \begin{equation*}
                 \Aut(K) = \{\sigma_{j,k} : j \in \{0,\ldots,5\}, k \in \{0,1\}\}
             \end{equation*}
             donde:
             \begin{equation*}
                 \sigma_{j,k}\left(\sqrt[6]{3}\right) = w^j \sqrt[6]{3}, \qquad \sigma_{j,k}(i) = {(-1)}^{k}i
             \end{equation*}
            Tenemos que:
            \begin{equation*}
                |G| = (G:\{\sigma_{0,0}\}) = \left[K:\mathbb{Q}\left(i+\sqrt{3}\right)\right] = \frac{[K:\mathbb{Q}]}{\left[\mathbb{Q}\left(i+\sqrt{3}\right):\mathbb{Q}\right]}
            \end{equation*}
            y como:
            \begin{equation*}
                \left[\mathbb{Q}\left(i+\sqrt{3}\right):\mathbb{Q}\right] = \left[\mathbb{Q}\left(i,\sqrt{3}\right):\mathbb{Q}\right] = 4
            \end{equation*}
            tenemos entonces que:
            \begin{equation*}
                |G| = 3
            \end{equation*}
            Será:
            \begin{equation*}
                G \supseteq \{\sigma_{0,0}, \sigma_{2,0}, \sigma_{4,0} \}
            \end{equation*}
            ya que:
            \begin{equation*}
                \sigma_{2,0}(\sqrt{3}+i) = \sigma_{2,0}\left({\left(\sqrt[6]{3}\right)}^{3}\right) + \sigma_{2,0}(i) = w^6{\left(\sqrt[6]{3}\right)}^{3}+i = i+\sqrt{3}
            \end{equation*}
            y como tenemos 3 será pues $G = \{\sigma_{0,0}, \sigma_{2,0}, \sigma_{4,0}\}$.
        \item ¿Es $G$ un subgrupo normal del grupo de Galois de $f$?

            Como $\mathbb{Q}\leq \mathbb{Q}(i+\sqrt{3})$ es de Galois por ser cuerpo de descomposición de un polinomio tenemos que el grupo es normal.
    \end{enumerate}
\end{ejercicio}

% // TODO: Examen 3

\begin{ejercicio}
    Calcular el número de polinomios irreducibles mónicos de grado hasta $3$ en $\bb{F}_5[x]$.\\

    \noindent
    \begin{itemize}
        \item De grado 1 tenemos todos los irreducibles mónicos de grado 1:
            \begin{equation*}
                x, \quad x-1, \quad x-2,\quad x-3,\quad x-4
            \end{equation*}
        \item Para los de grado 2 sabemos que van a aparecer en la descomposición del polinomio $x^{25}-x$, y sabemos que ya tenemos 5, nos queda grado 20 y polinomios de grado 2 serán 10.
        \item Para los de grado 3, sabemos que van a aparecer en la descomposición del polinomio $x^{125}-x$, y sabemos que $2$ no divide a 3, por lo que tenemos $5$ y nos queda $120$ entre 3, obteniendo $40$.
    \end{itemize}
    Así, hay 55 polinomios mónicos irreducibles de grado hasta $3$.
\end{ejercicio}

\begin{ejercicio}
    En $\bb{F}_{81} = \bb{F}_3(a)$ con $a$ un elemento primitivo, describir todos los subcuerpos de $\bb{F}_{81}$.\\

    \noindent
    Sabemos que $81 = 3^4$ y que cada divisor del exponente nos da un cuerpo, por lo que tenemos $3^1, 3^2$ y $3^4$.\\

    \noindent
    Sabemos que $\bb{F}_{81}^\times$ es un grupo cíclico de orden $80$:
    \begin{equation*}
        \bb{F}_{81}^\times = \{1,a,\ldots,a^{79}\}
    \end{equation*}
    Por lo que:
    \begin{equation*}
        \bb{F}_{81} = \{1,a,\ldots,a^{80}\}
    \end{equation*}
    Un cuerpo de $9$ elementos seguro que existe, buscamos un elemento que se exprese en términos de $a$ para expresar $\bb{F}_9$, es decir, buscar un elemento de orden 8. Como:
    \begin{equation*}
        {(a^{10})}^{8} = a^80 = 1
    \end{equation*}
    Tenemos que $a^10$ tiene orden 8, por lo que:
    \begin{equation*}
        \bb{F}_9 = \bb{F}_3(a^{10})
    \end{equation*}

    % profe
    \noindent
    Cada subcuerpo de $\bb{F}_{81}$ es subgrupo (quitando el 0) de $\bb{F}_{81}^\times$, pero hay más subgrupos de este que subcuerpos.
\end{ejercicio}

\begin{ejercicio} % Ejercicio 3 del examen 3
    ¿Cuántos subcuerpos tiene $\bb{F}_{256}$?\\

    \noindent
    Tenemos que $256 = 2^8$, con $Div(8) = \{1,2,4,8\}$.
\end{ejercicio}

\begin{ejercicio} % 60
    Sea $F=\bb{F}_3(a)$ con $a^3+a-1=0$:
    \begin{enumerate}
        \item Calcular el cardinal de $F$.

            Tenemos que $f= x^3+x-1$ no es irreducible, lo factorizamos:
            \begin{equation*}
                x^3+x-1 = (x+1)(x^2-x-1)
            \end{equation*}
            y como $a$ es raíz de $f$ tenemos que:
            \begin{itemize}
                \item Bien es raíz de $x+1$, es decir, $a=-1$, con lo que $F = \bb{F}_3(-1) = \bb{F}_3$.
                \item Bien $a^2-a-1=0$, de donde $[F:\bb{F}_3] = 2$, con lo que $|F| = 9$.
            \end{itemize}
        \item Calcular el grado de $\Irr(a^2,\bb{F}_3)$.

            Tenemos que:
            \begin{equation*}
                a^2 -a-1= 0 \quad\Longrightarrow\quad a^2 = a+1\notin \bb{F}_3
            \end{equation*}
            ya que $\{1,a\}$ es una $\bb{F}_3-$base de $F$. Por tanto, $\Irr(a,\bb{F}_3)$ tiene que tener grado 2.
        \item Calcular $\Irr(a^2,\bb{F}_3)$.

            Buscamos en los 3 polinomios mónicos irreducibles de $\bb{F}_3[x]$, buscando dónde es $a^2$ raíz, y podemos decartar $x^2-x-1$, por ser $a$ raíz suya.
            \begin{itemize}
                \item Probamos en $a^2+1$:
                    \begin{equation*}
                        {(a^2)}^{2}+1 = {(a+1)}^{2} + 1 = a^2 + 2a + 1 + 1 = a+1 + 2a + 2 = 3a + 3 = 0
                    \end{equation*}
            \end{itemize}
            Hemos tenido suerte, por lo que $\Irr(a^2,\bb{F}_3) = x^2+1$.
    \end{enumerate}
\end{ejercicio}
