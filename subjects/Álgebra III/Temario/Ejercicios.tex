\chapter{Ejercicios}
% // TODO: Seccion de "algunos ejercicios" son de examenes finales
\begin{ejercicio} % 49
    Sea $F$ cuerpo de descomposición de $f=x^3+x+1\in \bb{F}_2[x]$ y $\alpha\in F$ raíz de $f$. Razonar que $F=\bb{F}_2(\alpha)$. Resolver en $F$ las soluciones de las ecuaciones en función de $\alpha$:
    \begin{equation*}
        x^3+x+1=0,\qquad x^3+x^2+1=0,\qquad x^2+x+1=0
    \end{equation*}

    \noindent
    \textbf{Solución.}\newline
    Para ver que $F=\bb{F}_2(\alpha)$ basta ver que $f$ es irreducible, por lo que entonces $F\leq \bb{F}_2(\alpha)$ será de Galois y lo tenemos. Comprobamos que $f$ es irreducible, ya que se tiene $f(0) = f(1) = 1\neq 0$, y $deg f = 3$.

    \noindent
    Como $\alpha$ es raíz de $f$ y $\bb{F}_2\leq F$ es de Galois, entonces todas las raíces de $f$, que son, $\alpha,\alpha^2, \alpha^4 \in F$, tenemos entonces que $F=\bb{F}_2(\alpha)$. Puede razonarse también por el orden de los cuerpos, ambos es 8.
    \begin{enumerate}
        \item Las soluciones de la primera ecuación son trivialmente $\alpha,\alpha^2$ y $\alpha^4$.
        \item Para la segunda, vamos a repetir un argumento análogo al último ejemplo, es decir, resolver la ecuación en $\bb{F}_8$, y ya hemos descartado 5 raíces (3+2). Como $\bb{F}_8^\times$ tiene por generador $\alpha$, las raíces de $x^3+x^2+1$ son $\alpha^3,\alpha^5$ y $\alpha^6$.

            Esto lo sabemos viendo la factorización de $x^8+x\in \bb{F}_2[x]$ como el producto:
            \begin{equation*}
                x^8 + x = x(x+1)(x^3+x+1)(x^3+x^2+1)
            \end{equation*}
        \item Para la ecuación $x^2+x+1=0$, veamos que no tiene solución en $\bb{F}_8$, ya que:
            \begin{itemize}
                \item Como hemos gastado todas las raíces, tendría que ser una de ellas y \ldots
                \item El polinomio no está en la factorización de $x^8+x$.
                \item Si tuviera solución podríamos tener entonces:
                    \begin{equation*}
                        \bb{F}_2\leq \bb{F}_4\leq \bb{F}_8
                    \end{equation*}
                    Pero no puede ser $\bb{F}_{2^2} \leq \bb{F}_{2^3}$, ya que $2\nmid 3$.
            \end{itemize}
    \end{enumerate}
\end{ejercicio}

\begin{ejercicio} % 51
    Consideramos $f=x^3+x+1\in \bb{F}_4[x]$ y tomamos $K$ cuerpo de descomposición de $f$. Sea $\alpha\in K$ una raíz de $f$, veamos que $K = \bb{F}_4(\alpha)$.\\

    \noindent
    Para ello, usaremos que $\bb{F}_4\leq K$ es de Galois (puesto que son cuerpos finitos), como $f$ es irreducible en $\bb{F}_4[x]$, por varios motivos tenemos que $K = \bb{F}_4(\alpha)$. Sabemos que $f$ es irreducible en $\bb{F}_4[x]$ porque no tienes raíces en $\bb{F}_4$:
    \begin{itemize}
        \item Una opción es tomar $\bb{F}_4 = \{0,1,\gamma,\gamma+1\}$ y comprobarlo.
        \item Otra es por reducción al absurdo, suponer que sí, que $\gamma\in \bb{F}_4$ es una raíz de $f$. Vemos claramente que $\gamma\notin \bb{F}_2$, puesto que $f$ es irreducible en $\bb{F}_2$. De esta forma, observamos que $\bb{F}_4\leq \bb{F}_2(\gamma)$, con lo que $g=\Irr(\gamma,\bb{F}_2) = x^2+x+1\in \bb{F}_2[x]$. Tenemos que $g(\gamma) = 0 = f(\gamma)$, con $g,f\in \bb{F}_2[x]$, por lo que (por definición de $\Irr(\gamma,\bb{F}_2)$) $g\mid f$, pero ambos son irreducibles, por lo que hemos llegado a una contradicción. 
    \end{itemize}
    Por tanto, como $[K:\bb{F}_4] = 3$ y $|\bb{F}_4| = 4$, tenemos que $|K| = 4^3 = {(2^2)}^{3} = 64$.\\

    \noindent
    Nos preguntamos ahora cómo resolver las ecuaciones:
    \begin{equation*}
        x^3+x+1=0, \qquad x^3+x^2+1=0, \qquad x^2+x+1=0
    \end{equation*}
    \begin{description}
        \item [Si no nos acordamos del ejercicio anterior.] Como una solución es $\alpha$ y consideramos el homomorfismo de Frobenius de la extensión (elevar a 4), tenemos entonces que sus raíces en $K$ son $\alpha,\alpha^4$ y $\alpha^{16}$.
        \item [Si nos acordamos del ejercicio de ayer.] Habíamos resuelto la ecuación en $\bb{F}_2$, y es claro que $F=\bb{F}_2(\alpha) \leq K$, con $|\bb{F}_2(\alpha)| = 8$. Las soluciones que encontrábamos eran $\alpha,\alpha^2$ y $\alpha^4$ en $F\leq K$.

            Parece que no coinciden. Sin embargo, $\alpha^{16} = \alpha^2$, porque $\alpha^7 = 1$.
    \end{description}
    De la misma forma y usando el ejercicio anterior, tenemos que las soluciones de la segunda ecuación son $\alpha^3,\alpha^5$ y $\alpha^6$.
    \begin{description}
        \item [Si no nos acordamos.] Podríamos haber pensado en considerar $\bb{F}_2(\alpha)\leq K$ y resolver $f$ en $\bb{F}_2(\alpha)$, que es lo que hicimos en el ejercicio anterior.
    \end{description}
    Para la última ecuación, tomamos $\gamma\in \bb{F}_4\setminus \bb{F}_2$, de donde $\gamma$ es solución de $x^2+x+1=0$, ya que es el único polinomio irreducible de grado 2 sobre $\bb{F}_2[x]$.\\

    \noindent
    Finalmente, se nos pide hacer una base de $K$ sobre $\bb{F}_2$ usando $\alpha$ y $\gamma$.\\

    \noindent
    Para ello, lo que haremos es ver que:
    \begin{equation*}
        \bb{F}_2\leq F \leq K
    \end{equation*}
    con $\{1,\gamma\}$ una base de $\bb{F}_2\leq F$ y $\{1,\alpha,\alpha^2\}$ una base de $F\leq K$. Si repasamos la demostración del Lema de la Torre, tenemos que:
    \begin{equation*}
        \{1,\alpha,\alpha^2,\gamma,\gamma\alpha, \gamma\alpha^2\}
    \end{equation*}
    es una base de $\bb{F}_2\leq K$.\\

    \noindent
    ¿Cuál es el orden multiplicativo de $\gamma\alpha$? ¿Es un elemento primitivo de $K$ sobre $\bb{F}_2$?

    % // TODO: Se toma F_{64}^times, por lo que los ordene son 3, 7 o 63 (63 = 7 * 3²)
\end{ejercicio}

\begin{ejercicio}
    ¿Cuántos polinomios irreducibles de grado 6 hay en $\bb{F}_2[x]$?\\

    % Se puede hacer por algebra 1, listando todos los polinomios (hacerlo de forma inteligente para no hacer mucho) y comprobar que no es producto de grado 2 ni de grado 3
    \noindent
    Usando Álgebra III, estos polinomios aparecen en la descomposición del polinomio $x^{64}-x$ en irreducibles ($64=2^6$, con $Div(6) = \{1,2,3,6\}$)
    \begin{equation*}
        x^{64}-x = x(x+1)(x^2+x+1)(x^3+x+1)(x^3+x^2+1)\prod \text{polinomios grado 6}
    \end{equation*}
    Como la suma de los grados tiene que ser $64$, el último polinomio (el producto) tiene grado:
    \begin{equation*}
        64 - 10 = 54, \qquad \frac{54}{6} = 9
    \end{equation*}
    Por lo que hay $9$ polinomios irreducibles de grado $6$ en $\bb{F}_2[x]$.
\end{ejercicio} % // TODO: probar con distintos cardinales, que a lo mejor cae en examen

\begin{ejercicio} % 53
    Calcular el cardinal del grupo de Galois sobre $\mathbb{Q}$ del polinomio $f=(x^3+x+1)(x^2+1)$.\\

    \noindent
    Sea $K$ el cuerpo de descomposición de $f\in \mathbb{Q}[x]$. Las raíces del segundo están claras pero las del primero no (sabemos calcularlas pero es una cuesta llena de pinchos). Del primero sabemos que bien tiene 3 raíces reales o bien 1 raíz. Sea $g$ la función $g(x) = x^3+x+1$, tenemos que:
    \begin{equation*}
        g'(x) = 3x^2+1 > 0
    \end{equation*}
    Por lo que $g$ es estrictamente creciente, por lo que solo tiene una raíz real. Será por tanto:
    \begin{equation*}
        K = \mathbb{Q}(i,-i,r,\alpha,\overline{\alpha}) = \mathbb{Q}(i,r,\alpha)
    \end{equation*}
    donde $r\in \mathbb{R}$, $\alpha\in \mathbb{C}\setminus \mathbb{R}$ con $r,\alpha$ raíces de $x^3+x+1$. Como $\mathbb{Q}\leq K$ es de Galois, nos piden $|\Aut_\mathbb{Q}(K)| = [K:\mathbb{Q}]$. Por el Lema de la Torre:
    \begin{equation*}
        [K:\mathbb{Q}] = [K:\mathbb{Q}(i)][\mathbb{Q}(i):\mathbb{Q}] = [K:\mathbb{Q}(i)]\cdot 2
    \end{equation*}
    Para calcular $[K:\mathbb{Q}(i)]$ vamos a usar que las raíces de $x^3+x+1$ están en $K$. ¿Es cierto que $K$ es cuerpo de descomposición de $g=x^3+x+1\in \mathbb{Q}(i)[x]$? En efecto, si $g$ fuera irreducible en $\mathbb{Q}(i)$ entonces a partir del discriminante de $g$ (si está en $\mathbb{Q}$ o no) tendríamos que $[K:\mathbb{Q}(i)]$ es el cardinal bien de $A_3$ o de $S_3$.\\

    \noindent
    Para ver que $g$ es irreducible en $\mathbb{Q}(i)[x]$ veamos que no tiene raíces en $\mathbb{Q}(i)$: % // TODO: EN los apuntes hay otra forma de resolverlo
    \begin{itemize}
        \item Si $r$ es raíz de $g$ en $\mathbb{Q}(i)$, entonces $r\in \mathbb{Q}$, por lo que es raíz de $x^3+x+1$, pero $x^3+x+1$ es irreducible en $\mathbb{Q}$.
        \item Si $\alpha\in \mathbb{Q}(i)$, como $\alpha\notin \mathbb{Q}$, tendremos entonces que $\alpha$ tiene grado 2 sobre $\mathbb{Q}$, es decir, que $\Irr(\alpha,\mathbb{Q})$ tiene grado 2, y teníamos que $\alpha$ era raíz de $g$, por lo que $\Irr(\alpha,\mathbb{Q})\mid g$, pero $g$ es irreducible, lo que lleva a una contradicción.
        \item Como $\alpha\notin \mathbb{Q}(i)$ tenemos entonces que $\overline{\alpha}\notin \mathbb{Q}(i)$.
    \end{itemize}
    De aquí tenemos que $g$ es irreducible en $\mathbb{Q}(i)$, por lo que su grupo de Galois será bien $A_3$ o $S_3$, en función del discriminante:
    \begin{equation*}
        \Disc(g) = -31
    \end{equation*}
    Por lo que:
    \begin{equation*}
        \Delta = \sqrt{\Disc(g)} = i\sqrt{31} \notin\mathbb{Q}(i)
    \end{equation*}
    Ya que si $i\sqrt{31}\in \mathbb{Q}(i)$  tendríamos entonces que $\sqrt{31}\in \mathbb{Q}(i)$, pero $\sqrt{31}\in \mathbb{R}\setminus \mathbb{Q}$, ya que $31$ es primo ($x^2-31$ es irreducible por Eisenstein para $p=31$).

    \noindent
    En definitiva, como $\Delta\notin \mathbb{Q}(i)$ tenemos entonces que $\Aut_{\mathbb{Q}(i)}(K)\cong S_3$, con $|S_3| = 6$, de donde:
    \begin{equation*}
        [K:\mathbb{Q}] = [K:\mathbb{Q}(i)][\mathbb{Q}(i):\mathbb{Q}] = 6\cdot 2 = 12
    \end{equation*}~\\

    \noindent
    Describir cuál es el grupo $\Aut_\mathbb{Q}(K)$.\\ 

    \noindent
    La conexión de Galois nos dice que hay un subgrupo normal, puesto que su índice es 2. 

    \noindent
    Usamos el Teorema de extensión, por lo que estará $\eta_0,\eta_1,\eta_2$:
    \begin{equation*}
        r \stackrel{\eta_0}{\longmapsto} r ,\qquad
        r \stackrel{\eta_1}{\longmapsto} \alpha ,\qquad
        r \stackrel{\eta_2}{\longmapsto} \overline{\alpha}
    \end{equation*}
    Cada uno de estos lo extendemos llevando $i$ a $i$ o a $-i$. El último paso parece más difícil. % // TODO: ¿Puedo calcular el grupo de automorfismos? Como son sus elementos? Es un problema algo dificil
\end{ejercicio}

\begin{ejercicio}
    Calcular el grupo de Galois de $f=(x^2+x+1)(x^2-3)\in \mathbb{Q}[x]$.\\

    \noindent
    Sea $K$ el cuerpo de descomposición de $f$, tenemos que:
    \begin{equation*}
        K = \mathbb{Q}\left(w,\sqrt{3}\right), \qquad w \text{\ una raíz cúbica primitiva de la unidad}
    \end{equation*}
    Por el Lema de la Torre:
    \begin{equation*}
        [K:\mathbb{Q}] = \left[K:\mathbb{Q}\left(\sqrt{3}\right)\right] \left[\mathbb{Q}\left(\sqrt{3}\right) :\mathbb{Q}\right] = 2\cdot 2 = 4
    \end{equation*}
    Por lo que:
    \begin{equation*}
        |\Aut(K)| = [K:\mathbb{Q}] = 4
    \end{equation*}
    Calculamos sus elementos, por la Proposición de extensión. Calculamos primero los homomorfismos de cuerpos $\mathbb{Q}\left(\sqrt{3}\right)\to K$, que son $\eta_j$ con $j \in \{0,1\}$, con:
    \begin{equation*}
        \eta_j\left(\sqrt{3}\right) = {(-1)}^{j}\sqrt{3}, \qquad j \in \{0,1\}
    \end{equation*}
    Extendemos cada uno de estos homomorfismos, cada uno de ellos se extiende a dos homomorfismos $K\to K$, que son $\eta_{j,k}$, donde:
    \begin{equation*}
        \eta_{j,k}\left(\sqrt{3}\right) = {(-1)}^{j}\sqrt{3}, \qquad \eta_{j,k}(w) = w^k, \qquad j\in \{0,1\},\quad  k \in \{1,2\}
    \end{equation*}
    Como $\eta_{1,1}$ y $\eta_{1,2}$ tienen orden 2 tenemos que el grupo es isomorfo a $C_2\oplus C_2$.
\end{ejercicio}

\begin{ejercicio}
    Calcular el grupo de Galois de $g=(x^2+x+1)(x^2+3)\in \mathbb{Q}[x]$.\\

    \noindent
    Sea $K$ el cuerpo de descomposición de $g$, si tomamos como raíz cúbica primitiva de la unidad:
    \begin{equation*}
        w = \frac{-1}{2} + i\frac{\sqrt{3}}{2}
    \end{equation*}
    Tendremos ahora que:
    \begin{equation*}
        K = \mathbb{Q}\left(i\sqrt{3}\right)
    \end{equation*}
    Con lo que $|\Aut_\mathbb{Q}(K)| = [K:\mathbb{Q}] = 2$, isomorfo a $C_2$. Sus elementos son $\eta_j$, donde:
    \begin{equation*}
        \eta_j\left(i\sqrt{3}\right) = {(-1)}^{j}i\sqrt{3} \qquad j \in \{0,1\}
    \end{equation*} % // TODO: Esto prueba que un mismo cuerpo de descomposicion puede servir para dos polinomios irreducibles distintos, hay mas polinoimos que cuerpos de descomposicion
\end{ejercicio}

\begin{ejercicio} % 55
    Sea $f=x^3-3x+1\in \mathbb{Q}[x]$ y $\alpha$ cualquier raíz real de $f$. Demostrar que el cuerpo de descomposición de $f$ sobre $\mathbb{Q}$ es $\mathbb{Q}(\alpha)$.
\end{ejercicio}
