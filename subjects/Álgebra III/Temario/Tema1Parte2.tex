\section{Homomorfismos de cuerpos}

\begin{lema}
    Sea $\sigma:F\to A$ un homomorfismo de anillos donde $F$ es un cuerpo y $A$ es no trivial, entonces $\sigma$ es inyectivo y, por tanto, $Im \sigma$ es un cuerpo isomorfo a $F$ y subanillo de $A$.
    \begin{proof}
        Solo hemos de probar que $\ker \sigma = \{0\}$. Para ello, $\ker \sigma$ es un ideal de $F$ que no es $F$ (ya que $\sigma(1) = 1$), de donde $\ker \sigma = \{0\}$. Para ver que $Im \sigma\cong F$, basta aplicar el Primer Teorema de Isomorfía:
        \begin{equation*}
            F = \dfrac{F}{\ker \sigma} \cong Im \sigma
        \end{equation*}
    \end{proof}
\end{lema}

\begin{definicion}[Homomorfismo de cuerpos]
    Sea $F\stackrel{\sigma}{\to} K$ un homomorfismo de anillos entre cuerpos, diremos entonces que es un \underline{homomorfismo de cuerpos}.
\end{definicion}

\begin{observacion}
    Resulta sorprendente que exigir ``buenas propiedades'' a una aplicación entre anillos ya nos da una aplicación con ``buenas propiedades'' entre cuerpos, pero resulta que lo único que nos faltaba era que la aplicación se comporte bien con los inversos, propiedad que queda garantizada al exigir ``buenas propiedades'' sobre anillos:
    \begin{equation*}
        1 = \sigma(1) = \sigma\left(\alpha\alpha^{-1}\right) = \sigma(\alpha)\sigma\left(\alpha^{-1}\right) \Longrightarrow \sigma\left(\alpha^{-1}\right) = {\sigma(\alpha)}^{-1}
    \end{equation*}
    Como por el Lema anterior todo homomorfismo de cuerpos $F\stackrel{\sigma}{\to}K$ es siempre inyectivo, tendremos siempre una copia de $F$ dentro de $K$, que en ocasiones identificaremos con el propio $F$, viendo $\sigma(F)$ como una copia isomorfa de $F$. Como $\sigma(F)\leq K$ es una extensión de cuerpos, podemos ver $K$ como un $\sigma(F)-$espacio vectorial. Además, si identificamos $F$ con $\sigma(F)$, podremos ver $K$ como un $F-$espacio vectorial.
\end{observacion}

\begin{definicion}
    Siempre que tengamos $F\stackrel{\sigma}{\to} K$ y $f\in F[x]$ dada por:
\begin{equation*}
    f = \sum_{i=1}^{n} f_i x^i, \qquad f_i \in F \quad \forall i \in \{1,\ldots,n\}
\end{equation*}
Definiremos:
\begin{equation*}
    f^\sigma = \sum_{i=1}^{n} \sigma(f_i) x^i \in K[x]
\end{equation*}
Se verifica que la correspondencia $f\mapsto f^\sigma$ es un homomorfismo de anillos entre $F[x]$ y $K[x]$.
\end{definicion}

\begin{ejemplo}
    Sea $f\in F[x]$, $f$ no constante, sea $p\in F[x]$ un factor irreducible de $f$, consideramos\footnote{Donde $\langle p \rangle $ es el ideal generado por $p$.}:
    \begin{equation*}
        K = \dfrac{F[x]}{\langle p \rangle }
    \end{equation*}
    como $p$ es irreducible, tenemos que $K$ es un cuerpo. Definimos $\sigma:F\to K$ como:
    \begin{equation*}
        \sigma(a) = a + \langle p \rangle  \qquad \forall a\in F
    \end{equation*}
    que es un homomorfismo de anillos (observemos que es la composición de la proyección al cociente con la inclusión en $F[x]$) entre cuerpos, luego un homomorfismo de cuerpos, con:
    \begin{equation*}
        \sigma(F) = \{a+\langle p \rangle : a\in F\} \cong F
    \end{equation*}
    Sea $\alpha = x+\langle p \rangle \in K$, tenemos que:
    \begin{equation*}
        p^\sigma(\alpha) = \sum_{i=1}^{n} (p_i + \langle p \rangle){(x+\langle p \rangle )}^{i} = \sum_{i=1}^{n} p_ix^i + \langle p \rangle  = p + \langle p \rangle  = 0 + \langle p \rangle 
    \end{equation*}
    Además:
    \begin{equation*}
        \sigma(F)(\alpha) = K
    \end{equation*}
    \begin{description}
        \item [$\subset )$] Basta ver que $K$ contiene a $\sigma(F)$ y a $\alpha$.
        \item [$\supset )$] Si tomamos un elemento de $K$, este será de la forma $g+\langle p \rangle $ para cierta $g\in F[x]$ dada por:
            \begin{equation*}
                \sum_{i=1}^{n}g_ix^i, \qquad g_i \in F, \quad \forall i \in \{1,\ldots,n\}
            \end{equation*}
            Por lo que:
            \begin{equation*}
                g+\langle p \rangle  = \sum_{i=1}^{n}g_ix^i + \langle p \rangle  = \sum_{i=1}^{n}(g_i+\langle p \rangle ){(x+\langle p \rangle )}^{i} \in \sigma(F)(\alpha)
            \end{equation*}
    \end{description}
\end{ejemplo}

\begin{lema}\label{lema:cuerpo_descomposicion}
    Si $f\in F[x]$ es no constante y $p$ es un factor irreducible de $f$, entonces existen $F\stackrel{\sigma}{\to}K$ homomorfismo de cuerpos y $\alpha\in K$ tales que:
    % \begin{equation*}
    %     f^\sigma(\alpha) = 0 \qquad \text{y}\qquad  \sigma(F)(\alpha) = K
    % \end{equation*}
    \begin{equation*}
        p^\sigma(\alpha) = 0 \qquad \text{y}\qquad  K = \sigma(F)(\alpha)
    \end{equation*}
    Bajo estas condiciones, a menudo identificaremos $F$ con $\sigma(F)$ y en dicho caso, escribiremos $K= F(\alpha)$.
    \begin{proof}
        La demostración se deduce del ejemplo anterior.
    \end{proof}
\end{lema}

\begin{prop}\label{prop:existe_cdd}
    Sea $f\in F[x]$ con $degf = n \geq 1$, entonces existe un homomorfismo de cuerpos $\sigma:F\to E$ tal que $E$ es un cuerpo de descomposición de $f^{\sigma}$.
    \begin{proof}
        Suponemos sin pérdida de generalidad que $f$ es mónico. Vamos a ver que existe $F\stackrel{\sigma}{\to}L$ tal que $f^\sigma$ se descompone completamente como producto de factores lineales en $L[x]$. Para ello, descomponemos $f=gh$, donde $g\in F[x]$ es producto de polinomios lineales y $h\in F[x]$ es un polinomio sin raíces en $F$. Por inducción sobre el grado de $h$ (usando el segundo principio de inducción):
        \begin{itemize}
            \item \underline{Si $deg h = 0$}, tomando $L = F$ y $\sigma = id_F$ se tiene.
            \item \underline{Supuesto que $degh>0$ y la hipótesis de inducción}, tomamos $p$ un factor irreducible de $h$, por lo que podemos aplicar el Lema~\ref{lema:cuerpo_descomposicion}, con lo que existen $F\stackrel{\tau}{\to}K$ y $\alpha\in K$ tal que $p^\tau(\alpha)=0$ y $K = F(\alpha)$. Observamos que $h^\tau(\alpha)=0$.

                El polinomio $g$ que habíamos escogido será de la forma:
                \begin{equation*}
                    g = (x-\alpha_1)\cdot  \ldots \cdot (x-\alpha_t), \qquad \alpha_1,\ldots,\alpha_t \in F
                \end{equation*}
                Extraemos ahora los factores lineales de $h^\tau$ en $K[x]$ (sabemos que al menos $s\geq 1$, puesto que $\alpha$ es raíz de $h^\tau$):
                \begin{equation*}
                    h^\tau = (x-\beta_1)\cdot \ldots\cdot (x-\beta_s)k, \qquad k\in K[x], \beta_1,\ldots,\beta_s\in K
                \end{equation*}
                Con uno de los $\beta_i$ es $\alpha$ y $k$ sin raíces en $K$ y de grado menor que el de $h^\tau$. En definitiva, tenemos que:
                \begin{equation*}
                    f^\tau = g^\tau h^\tau = (x-\tau(\alpha_1))\cdot  \ldots \cdot (x-\tau(\alpha_s))(x-\beta_1) \cdot \ldots \cdot (x-\beta_s)k, \qquad deg k < deg h^\tau
                \end{equation*}
                Aplicando la hipótesis de inducción tomando $k$ como $h$, existe un homomorfismo $K\stackrel{\rho}{\to}L$ tal que $k^\rho$  se descompone como producto de polinomios lineales en $L[x]$. En defintiiva, tendremos que ${(f^\tau)}^{\rho}$ se descompone como producto de polinomios lineales en $L[x]$. Si tomamos:
                \begin{equation*}
                    \sigma = \rho\tau : F\to L
                \end{equation*}
                tenemos que $f^\sigma$ se descompone como producto de lineales en $L[x]$. Tomamos ahora como $E$ el subcuerpo de $L$ generado por las raíces de $f^\sigma$ y $\sigma(F)$, con lo que $E$ es un cuerpo de descomposición de $f^\sigma$.
        \end{itemize}
    \end{proof}
\end{prop}

\begin{definicion}
    A un homomorfismo $\sigma:F\to E$ como el de la Proposición anterior se le llama cuerpo de descomposición de $f$.
\end{definicion}

\noindent
Respecto a esta última definición, debemos tener claro que antes hablábamos de cuerpo de descomposición de $f\in F[x]$ a una extensión $F\leq K$ de forma que $K$ cumplía ciertas propiedades relativas a $f$. Ahora, lo que hacemos es ver el homomorfismo $F\stackrel{\sigma}{\to} K$ como una extensión de cuerpos, identificando $F$ con $\sigma(F)$, por lo que al propio homomorfismo (que hace el papel de la extensión) le llamamos ahora cuerpo de descomopsición, si $K$ cumple unas propiedades relativas a $f^\sigma$.

\begin{ejemplo}
    Tomamos $f = x^2+x+1\in \bb{F}_2[x]$, donde $\bb{F}_2 = \{0,1\}$ es el cuerpo que contiene dos elementos. Como $f(0) = f(1) = 1 \neq 0$, tenemos que $f$ no tiene raíces en $\bb{F}_2$. Nuestro objetivo es buscar un cuerpo de descomposición suyo.\\

    \noindent
    Observemos que como $f$ es de grado 2 y no tiene raíces en $\bb{F}_2$, $f$ es irreducible, por lo que repitiendo el ejemplo anterior del que vienen el Lema y la Proposición, podemos tomar el cuerpo:
    \begin{equation*}
        K = \dfrac{\bb{F}_2[x]}{\langle f \rangle }
    \end{equation*}

    y el homomorfismo de cuerpos:
    \Func{\sigma}{\bb{F}_2}{K}{\sigma(y)}{y+\langle f\rangle}

    sabemos ya que:
    \begin{equation*}
        f^\sigma(\alpha) = 0 \quad \text{con}\quad  \alpha = x+\langle f \rangle 
    \end{equation*}
    Si factorizamos $f^\sigma$ (usando que $\alpha^2 + \alpha+1=0$):
    \begin{equation*}
        f^\sigma = (x+\alpha)(x+\alpha^2)
    \end{equation*}
    tenemos que $\sigma$ es un cuerpo de descomposición de $F$. Viendo que tenemos una copia isomorfa de $\bb{F}_2$ dentro de $K$, identificamos $\bb{F}_2$ con $\sigma(\bb{F}_2)$, y tenemos $\bb{F}_2\leq K$, con lo que:
    \begin{equation*}
        K = \bb{F}_2(\alpha), \qquad Irr(\alpha,\bb{F}_2) = x^2+x+1, \quad \text{ó}\quad \alpha^2 + \alpha+1=0
    \end{equation*}
    ¿Cuántos elementos tiene $K$?

    \noindent
    En vista de que $[K:\bb{F}_2] = 2$ y $|\bb{F}_2| = 2$, tenemos que $|K| = 4$. Para listarlos:
    \begin{itemize}
        \item $K = \{1,\alpha,0,1+\alpha\}$. 

            Donde vemos que $1+\alpha$ es distinto del resto porque $\{1,\alpha\}$ es una $\bb{F}_2-$base de $K$. La condición $\alpha^2 + \alpha + 1 = 0$ también nos dice que $\alpha+1= \alpha^2$:
        \item $K = \{0,1,\alpha,\alpha^2\}$.
    \end{itemize}
\end{ejemplo}

\begin{ejemplo}
    Al igual que en el ejemplo anterior, tomamos
    \begin{equation*}
        f = x^3+x+1 \in \bb{F}_2[x]
    \end{equation*}
    que sigue siendo irreducible sobre $\bb{F}_2[x]$, por ser de grado 3 y no tener raíces en $\bb{F}_2[x]$. De la misma forma, un cuerpo de descomposición de $f$ es de la forma $\bb{F}_2(a)$ con $a$ en cierto cuerpo $K$, siendo $a$ una raíz de $f$. Tratamos de factorizar $f$ en $\bb{F}_2(a)$:

    \begin{equation*}
        \begin{array}{r|c}
            \begin{array}{rcrcrcc}
                x^3&+&&+&x&+&1 \\
                x^3&+&ax^2 && &&\\
                \hline
                   &&ax^2&+&x&+&1  \\
                   &&ax^2&+&a^2x&& \\
                   \hline
                   &&&&(a^2 +1)x &+&1 \\
                   &&&&(a^2+1)x &+&a^3 + a \\
                   \hline
                   &&&&&& a^3 + a + 1
            \end{array} & 
            \begin{array}{l}
                x+a \\
                \hline
                x^2+ax+(a^2+1) \\ \\ \\ \\ \\ \\
            \end{array}
        \end{array}
    \end{equation*}
    Y tenemos que $a^3+a+1 = 0$. Buscamos ahora una raíz de $x^2+ax+(a^2+1)$. Probamos con $a^2$ (donde usamos que $a^3 + a + 1 = 0$):
    \begin{equation*}
        {\left(a^2\right)}^{2}+ aa^2 + (a^2+1) = a^4 + a^3 + a^2 + 1 = a^4 + a + a^2 = a(a^3 + a + 1) = 0
    \end{equation*}
    Dividimos ahora entre $x+a^2$:
    \begin{equation*}
        \begin{array}{r|c}
            \begin{array}{rcrcc}
                x^2 &+& ax &+& a^2 +1 \\
                x^2 &+& a^2 x && \\
                \hline
                     a^4x &=& a(a+1)x &+& a^2 + 1 \\
                          && a^4x &+& a^6 \\
                          \hline
                          &&&& a^6+a^2 +1

            \end{array} & 
            \begin{array}{l}
                x+a^2 \\
                \hline
                x+a^4 \\ \\ \\ \\
            \end{array}
        \end{array}
    \end{equation*}
    Y tenemos:
    \begin{align*}
        a^6+a^2+1 = a^6+a^2 +a^3 +a = a(a^5+a+a^2+1) = a(a^5+a^2+a^3) = a^3(a^3 + 1 + a) = 0
    \end{align*}
    En definitiva, la factorización de $f$ en $\bb{F}_2(a)$ es:
    \begin{equation*}
        x^3+x+1 = (x+a)(x+a^2)(x+a^4)
    \end{equation*}
    con lo que $\bb{F}_2(a)$ es un cuerpo de descomposición de $f$, ahora $|\bb{F}_2(a)| = 2^3 = 8$.

    \noindent
    Podríamos haber estudiado también $f=x^3+x^2+1$, obteniendo otro cuerpo de $8$ elementos. Veremos luego que estos dos cuerpos son isomorfos entre sí, e isomorfos con todo otro cuerpo que contenga $8$ elementos, lo que nos permitirá notarlos a todos por $\bb{F}_8$.
\end{ejemplo}

\begin{lema}\label{lema:extension}
    Sea $F\stackrel{\sigma}{\to}{K}$, $p\in F[x]$ irreducible, si $\alpha\in K$ es raíz de $p^\sigma$, entonces se tiene que:
    \Func{\sigma_\alpha}{\dfrac{F[x]}{\langle p\rangle}}{\sigma(F)(\alpha)}{g+\langle p\rangle}{g^\sigma(\alpha)}
    es un isomorfismo de cuerpos.
    \begin{proof}
        Podemos tomar:
        \Func{\overline{\sigma_\alpha}}{F[x]}{\sigma(F)(\alpha)}{g}{g^\sigma(\alpha)}
        En el Lema~\ref{lema:cuerpo_descomposicion} vimos que $g^{\sigma}(\alpha)\in \sigma(F)(\alpha)$ siempre que $g\in F[x]$, por lo que $\overline{\sigma_\alpha}$ está bien definida, y además es un homomorfismo de cuerpos. Como $p^\sigma(\alpha)=0$, tenemos que $\langle p \rangle \subseteq \ker(\overline{\sigma_\alpha})$, pero como $p$ es irreducible, tenemos que $\langle p \rangle $ es maximal, con lo que $\langle p \rangle =\ker(\overline{\sigma_\alpha})$. Si aplicamos ahora el Primer Teorema de Isomorfía para anillos, vemos que:
        \begin{equation*}
            \dfrac{F[x]}{\langle p \rangle } = \dfrac{F[x]}{\ker(\overline{\sigma_\alpha})} \cong Im \overline{\sigma_\alpha} = \sigma(F)(\alpha)
        \end{equation*}
    \end{proof}
\end{lema}

\begin{definicion}
    Si tenemos dos homomorfismos de cuerpos:
    \begin{figure}[H]
        \centering
        \shorthandoff{""}
        \begin{tikzcd}
        F \arrow[r, "\tau"] \arrow[rd, "\sigma"'] & E                    \\
                                                  & K 
        \end{tikzcd}
        \shorthandon{""}
    \end{figure}
    \noindent
    Diremos que un homomorfismo de cuerpos $\eta:K\to E$ es una $\sigma-$extensión de $\tau$ si:
    \begin{equation*}
        \eta\sigma = \tau
    \end{equation*}
    \noindent
    Y notaremos al conjunto de todas las $\sigma-$extensiones de $\tau$ por:
    \begin{equation*}
        Ex(\tau,\sigma) = \{\eta:K\to E \text{\ con\ } \eta \text{\ homomorfismo y\ } \eta \sigma = \tau\}
    \end{equation*}
    Notemos que todos estos hacen que el siguiente diagrama sea conmutativo:
    \begin{figure}[H]
        \centering
        \shorthandoff{""}
        \begin{tikzcd}
        F \arrow[r, "\tau"] \arrow[rd, "\sigma"'] & E                    \\
                                                  & K \arrow[u, "\eta"']
        \end{tikzcd}
        \shorthandon{""}
    \end{figure}
\end{definicion}

\begin{prop}[Extensión de homomorfismos]
    Si tenemos dos homomorfismos de cuerpos:
    \begin{figure}[H]
        \centering
        \shorthandoff{""}
        \begin{tikzcd}
        F \arrow[r, "\tau"] \arrow[rd, "\sigma"'] & E                    \\
                                                  & K 
        \end{tikzcd}
        \shorthandon{""}
    \end{figure}
    \noindent
    Si $p\in F[x]$ irreducible y $\alpha\in K$ con $p^{\sigma}(\alpha)=0$, si $\cc{R}\subseteq E$ es el conjunto de todas las raíces de $p^\tau$ y además $K = \sigma(F)(\alpha)$, tenemos entonces que la aplicación
    \Func{}{Ex(\tau,\sigma)}{\cc{R}}{\eta}{\eta(\alpha)}
    es una biyección.
    \begin{proof}
        Veamos en primer lugar que dicha aplicación está bien definida. Para ello, sea $\eta \in Ex(\tau, \sigma)$:
        \begin{equation*}
            p^\tau(\eta(\alpha)) = p^{\eta \sigma}(\eta (\alpha)) \AstIg \eta(p^\sigma(\alpha)) = \eta(0) = 0
        \end{equation*}
        donde en $(\ast)$ hemos usado que si $p$ es de la forma:
        \begin{equation*}
            p = \sum_i p_i x^i, \qquad p_i \in F
        \end{equation*}

        entonces:
        \begin{equation*}
            p^{\eta\sigma}(\eta(\alpha)) = \sum_i \eta(\sigma(p_i)) \eta(\alpha) = \sum_i \eta(\sigma(p_i)\alpha) = \eta\left(\sum_i \sigma(p_i) \alpha\right) = \eta(p^\sigma(\alpha))
        \end{equation*}
        Esto prueba que\footnote{Notemos que hemos probado además que $Ex(\tau,\sigma)\neq \emptyset \Longrightarrow \cc{R}\neq \emptyset $.} $\eta(\alpha)\in \cc{R}$. Veamos ahora que la aplicación enunciada es sobreyectiva\footnote{Con lo que tendremos $\cc{R}\neq \emptyset \Longrightarrow Ex(\tau,\sigma)\neq \emptyset $}. Para ello, sea $\beta\in \cc{R}$, buscamos una $\sigma-$extensión $\eta$ de forma que $\eta(\alpha) = \beta$. Usando el Lema~\ref{lema:extension}, obtenemos los isomorfismos:
        \Func{\sigma_\alpha }{\dfrac{F[x]}{\langle p \rangle }}{\sigma(F)(\alpha)=K}{g+\langle p\rangle}{g^\sigma(\alpha)}
        \Func{\tau_\beta }{\dfrac{F[x]}{\langle p \rangle }}{\tau(F)(\beta)\leq E}{g+\langle p\rangle}{g^\tau(\beta)}
        Si tomamos:
        \begin{equation*}
            \eta = i \circ \tau_\beta \circ \sigma_\alpha^{-1}
        \end{equation*}
        donde $i$ es la inclusión $\tau(F)(\beta)\leq E$, observamos que:
        \begin{equation*}
            K\stackrel{\sigma_\alpha^{-1}}{\longrightarrow} \frac{F[x]}{\langle p \rangle } \stackrel{\tau_\beta}{\longrightarrow} \tau(F)(\beta) \stackrel{i}{\longrightarrow} E
        \end{equation*}
        Comprobemos que $\eta\in Ex(\tau,\sigma)$, ya que si $a\in F$:
        \begin{equation*}
            (\eta \circ \sigma)(a) = (i\circ \tau_\beta\circ \sigma_\alpha^{-1})(\sigma(a)) = (i\circ \tau_\beta)(\sigma_\alpha^{-1}(\sigma(a))) = (i\circ \tau_\beta)(a+\langle p \rangle ) = i(\tau(a)) = \tau(a)
        \end{equation*}
        donde hemos aplicado que tanto $\sigma_\alpha$ como $\tau_\beta$ aplicado sobre constantes son iguales a $\sigma$ y a $\tau$, respectivamente, lo que prueba que $\eta\in Ex(\tau,\sigma)$. Ahora:
        \begin{equation*}
            \eta(\alpha) = (i\circ \tau_\beta)(\sigma_\alpha^{-1}(\alpha))= (i\circ \tau_\beta)(x+\langle p \rangle ) = i(\beta) = \beta
        \end{equation*}
        Falta probar que la aplicación es inyectiva. Para ello, sean $\eta,\eta' \in Ex(\tau,\sigma)$ de forma que $\eta(\alpha)=\eta'(\alpha)$, entonces como $K=\sigma(F)(\alpha)$, si tomamos un elemento de $K$ este será de la forma (ya que $\alpha$ es algebraico sobre $\sigma(F)$):
        \begin{equation*}
            \sum_{i} \sigma(a_i) \alpha^i \in \sigma(F)(\alpha), \qquad a_i \in F
        \end{equation*}
        con lo que:
        \begin{equation*}
            \eta\left(\sum_{i}\sigma(a_i)\alpha^i\right) = \sum_i \eta(\sigma(a_i)) {\eta(\alpha)}^{i} \AstIg \sum_i \eta'(\sigma(a_i)) {\eta'(\alpha)}^{i} = \eta'\left(\sum_{i}\sigma(a_i)\alpha^i\right) 
        \end{equation*}
        donde en $(\ast)$ usamos que tanto $\eta$ como $\eta'$ son $\sigma-$extensiones de $\tau$, con lo que $\eta\circ \sigma = \tau = \eta'\circ \sigma$. En definitiva, tenemos que $\eta = \eta'$, al ser $\eta(g) = \eta'(g)$ para todo $g\in K$, lo que nos dice que la aplicación es inyectiva.
    \end{proof}
\end{prop}

\noindent
Obsevemos que en esta última proposición hemos probado además que:
\begin{equation*}
    \cc{R}= \emptyset  \Longleftrightarrow Ex(\tau,\sigma) =  \emptyset 
\end{equation*}

\begin{lema}
    Sean tres homomorfismos entre cuerpos:
    \begin{figure}[H]
        \centering
        \shorthandoff{""}
        \begin{tikzcd}
        F \arrow[r, "\tau"] \arrow[d, "\sigma_1"'] & L   \\
        E_1 \arrow[r, "\sigma_2"']                 & E_2
        \end{tikzcd}
        \shorthandon{""}
    \end{figure}
    Se verifica que:
    \begin{equation*}
        Ex(\tau,\sigma_2\sigma_1) = \biguplus_{\eta \in Ex(\tau,\sigma_1)} Ex(\eta,\sigma_2)
    \end{equation*}
    \begin{proof}
        Por doble inclusión:
        \begin{description}
            \item [$\subseteq )$] Si tomamos $\theta \in Ex(\tau,\sigma_2\sigma_1)$, tenemos entonces que:
                \begin{equation*}
                    \theta\sigma_2\sigma_1 = \tau
                \end{equation*}
                Por lo que si tomamos $\eta = \theta\sigma_2$, tenemos que:
                \begin{gather*}
                    \theta\sigma_2\sigma_1 = \eta\sigma_1 = \tau \Longrightarrow \eta \in Ex(\tau,\sigma_1) \\
                    \theta\sigma_2 = \eta \Longrightarrow \theta \in Ex(\eta, \sigma_2)
                \end{gather*}
            \item [$\supseteq )$] Si $\eta \in Ex(\tau,\sigma_1)$ y tomamos $\theta\in Ex(\eta,\sigma_2)$, tendremos entonces que:
                \begin{equation*}
                    \left.\begin{array}{l}
                        \eta\sigma_1 = \tau \\
                        \theta\sigma_2 = \eta
                    \end{array}\right\}\Longrightarrow \theta\sigma_2\sigma_1 = \tau \Longrightarrow \theta\in Ex(\eta,\sigma_1\sigma_2)
                \end{equation*}
        \end{description}
        Hemos probado que
        \begin{equation*}
            Ex(\tau,\sigma_2\sigma_1) = \bigcup_{\eta \in Ex(\tau,\sigma_1)} Ex(\eta,\sigma_2)
        \end{equation*}
        Ahora, si $\eta,\eta'\in Ex(\tau,\sigma_1)$ y tenemos que:
        \begin{equation*}
            \theta \in  Ex(\eta,\sigma_2)\cap Ex(\eta',\sigma_2) \Longrightarrow \left\{\begin{array}{l}
                \theta\sigma_2 = \eta \\
                \theta \sigma_2 = \eta'
            \end{array}\right. \Longrightarrow \eta = \eta'
        \end{equation*}
        por lo que la unión es disjunta.
    \end{proof}
\end{lema}

% Esta es otra proposicion de extension, pero no le daremos nombre porque quiere que si usemos la proposicion de extension nos referiramos a la otra.
\begin{prop}\label{prop:extension_2}
    Sean dos homomorfismos de cuerpos:
    \begin{figure}[H]
        \centering
        \shorthandoff{""}
        \begin{tikzcd}
        F \arrow[r, "\tau"] \arrow[rd, "\sigma"'] & E \\
                                                  & K
        \end{tikzcd}
        \shorthandon{""}
    \end{figure}
    Si $[K:\sigma(F)]<\infty$, entonces $|Ex(\tau,\sigma)| \leq [K:\sigma(F)]$.
    \begin{proof}
        Por inducción sobre $n = [K:\sigma(F)]$ usando el segundo principio de inducción:
        \begin{itemize}
            \item \underline{Si $n=1$}, entonces $\sigma(F) = K$, por lo que $\sigma$ es un isomorfismo, con lo que $Ex(\tau,\sigma) = \{\tau \sigma^{-1}\}$, ya que si $\eta \in Ex(\tau, \sigma)$, entonces:
                \begin{equation*}
                    \eta\sigma = \tau \Longrightarrow \eta = \tau\sigma^{-1}
                \end{equation*}
                En definitiva, $1 = |Ex(\tau,\sigma)| \leq [K:\sigma(F)] = 1$.
            \item \underline{Supuesto que $n>1$ y la hipótesis de inducción}, como $[K:\sigma(F)] = n > 1$, tenemos que existe $\alpha\in K$ de forma que $[\sigma(F)(\alpha):\sigma(F)] > 1$, con lo que el Lema de la Torre nos dice que $[K:\sigma(F)(\alpha)] < n$. 

                Sea ahora $\iota:\sigma(F)(\alpha)\to K$ la inclusión en $K$, podemos tomar:
                \begin{equation*}
                    \sigma = \iota \circ \sigma'
                \end{equation*}
                con $\sigma':F\to \sigma(F)(\alpha)$ la restricción en codominio (o correstricción) de $\sigma$. Nos encontramos en la siguiente situación:
                \begin{figure}[H]
                    \centering
                    \shorthandoff{""}
                    \begin{tikzcd}
                        F \arrow[r, "\tau"] \arrow[rd, "\sigma"] \arrow[d, "\sigma'"'] & E \\
                        \sigma(F)(\alpha) \arrow[r, "\iota"]                           & K
                    \end{tikzcd}
                    \shorthandon{""}
                \end{figure}
                \noindent
                Aplicando el Lema anterior, obtenemos:
                \begin{equation*}
                    Ex(\tau,\sigma) = \biguplus_{\eta \in Ex(\tau,\sigma')}Ex(\eta, \iota)
                \end{equation*}
                Con lo que:
                \begin{equation*}
                    |Ex(\tau,\sigma)| = \sum_{\eta \in  Ex(\tau, \sigma')}|Ex(\eta,\iota)|
                \end{equation*}
                Sea $\eta\in Ex(\tau,\sigma')$, por hipótesis de inducción ($[K:\sigma(F)(\alpha)]<n$) tenemos que:
                \begin{equation*}
                    |Ex(\eta,\iota)| \leq [K:\sigma(F)(\alpha)]
                \end{equation*}
                con lo que:
                \begin{align*}
                    |Ex(\tau,\sigma)| &= \sum_{\eta\in Ex(\tau,\sigma')}|Ex(\eta,\iota)| \leq \sum_{\eta\in Ex(\tau,\sigma')}[K:\sigma(F)(\alpha)] \\
                                      &= |Ex(\tau,\sigma')|[K:\sigma(F)(\alpha)]
                \end{align*}
                Sea $p^\sigma = Irr(\alpha,\sigma(F))$, la Proposición de extensión nos dice que si $\cc{R}_\tau$ es el número de raíces de $p^\tau$ en $E$, entonces:
                \begin{equation*}
                    |Ex(\tau, \sigma')| = |\cc{R}_\tau| \leq degp^\tau = [\sigma(F)(\alpha):\sigma(F)]
                \end{equation*}
                Por lo que aplicando el Lema de la Torre en $(\ast)$:
                \begin{equation*}
                    |Ex(\tau,\sigma)|  \leq [\sigma(F)(\alpha):\sigma(F)][K:\sigma(F)(\alpha)] = [K:\sigma(F)]
                \end{equation*}
        \end{itemize}
    \end{proof}
\end{prop}

\begin{ejercicio} 
    Si $\Pi$ es el cuerpo primo de un cuerpo $K$, entonces el único homomorfismo de cuerpos $\sigma:\Pi\to K$ es la inclusión. % // TODO: Formalizar, es la idea de siempre
\end{ejercicio}

\noindent
Mostramos a continuación un ejemplo básico de la proposición de extensión.

\begin{ejemplo}
    ¿Cuántos homomorfismos de cuerpos hay de $\mathbb{Q}(\sqrt[3]{2})$ en $\mathbb{C}$, y cuáles son?\\

    \noindent
    Es decir, nos preguntamos por cuántos homomorfismos $\eta:\mathbb{Q}\left(\sqrt[3]{2}\right)\to \mathbb{C}$ hay. Para ello, reformularemos el problema en calcular un cierto conjunto de extensiones. Si consideramos las inclusiones de $\mathbb{Q}$ en $\mathbb{C}$ ($\tau$) y de $\mathbb{Q}$ en $\mathbb{Q}\left(\sqrt[3]{2}\right)$ ($\sigma$), tenemos:
    \begin{figure}[H]
        \centering
        \shorthandoff{""}
        \begin{tikzcd}
            \mathbb{Q} \arrow[rd, "\sigma"] \arrow[r, "\tau"] & \mathbb{C}                                   \\
                                                             & {\mathbb{Q}\left(\sqrt[3]{2}\right)} \arrow[u, "\eta"']
        \end{tikzcd}
        \shorthandon{""}
    \end{figure}
    \noindent
    Y lo que queremos hacer es calcular los elementos del conjunto $Ex(\tau,\sigma)$. Sabemos que:
    \begin{equation*}
        Irr\left(\sqrt[3]{2},\mathbb{Q}\right) = x^3-2
    \end{equation*}
    Si tomamos el conjunto de sus raíces en $\mathbb{C}$, sabemos que:
    \begin{equation*}
        \cc{R} = \left\{\sqrt[3]{2}, w\sqrt[3]{2}, w^2\sqrt[3]{2}\right\}
    \end{equation*}
    donde $w$ es una raíz cúbica primitiva de la unidad, tenemos tres homomorfismos de $\mathbb{Q}(\sqrt[3]{2})$ en $\mathbb{C}$. Les damos nombre a cada uno de ellos:
    \begin{equation*}
        Ex(\tau,\sigma) = \{\eta_0,\eta_1,\eta_2\}
    \end{equation*}
    donde $\eta_i$ está determinado (según la proposición de extensión) por:
    \begin{equation*}
        \eta_j\left(\sqrt[3]{2}\right) = w^j \sqrt[3]{2}, \qquad \forall j \in \{0,1,2\}
    \end{equation*}
    Como cada uno de los $\eta_j$ es un homomorfismo definido sobre $\mathbb{Q}\left(\sqrt[3]{2}\right)$ con base $\{1,\sqrt[3]{2}\}$, es suficiente definirlos sobre 1 (todos cumplirán $\eta_j(1)=1$, por ser homomorfismos) y sobre $\sqrt[3]{2}$. Como ejemplo de esto último, observemos que podemos calcular:
    \begin{equation*}
        \eta_2\left(\frac{{\sqrt[3]{2}+(\sqrt[3]{2})}^{2}}{27}\right) = \dfrac{\eta_2(\sqrt[3]{2}) + {\left(\eta_2\left(\sqrt[3]{2}\right)\right)}^{2}}{27} = \dfrac{w^2 \sqrt[3]{2} + {\left(w^2\sqrt[3]{2}\right)}^{2}}{27} 
    \end{equation*} 
\end{ejemplo}

\begin{prop}\label{prop:extension_3}
    Sean $\tau:F\to E$, $\sigma:F\to K$ homomorfismos de cuerpos con $\sigma$ un cuerpo de descomposición de $f\in F[x]$. Si $f^\tau$ se descompone como producto de polinomios lineales en $E[x]$, entonces $Ex(\tau,\sigma)$ es no vacío. Además, si $f^\sigma$ tiene $deg f$ raíces distintas, entonces:
    \begin{equation*}
        |Ex(\tau, \sigma)| = [K:\sigma(F)]
    \end{equation*}
    \begin{proof}
        La idea es similar a la de la Poposición~\ref{prop:extension_2}, por inducción sobre $n=[K:\sigma(F)]$:
        \begin{itemize}
            \item Para $n = 1$, tenemos que $K = \sigma(F)$, con lo que $\sigma$ es un isomorfismo y tendremos por tanto que $Ex(\tau,\sigma) = \{\tau\sigma^{-1}\}$.
            \item Supuesto que $n>1$ y la hipótesis de inducción, tenemos que $f$ tiene un factor irreducible $p\in F[x]$ de grado mayor o igual 1. Tomamos una raíz $\alpha\in K$ de $p^\sigma$, de donde $[K:\sigma(F)(\alpha)]<n$. Si consideramos $\sigma':F\to \sigma(F)(\alpha)$ y la inclusión $\iota:\sigma(F)(\alpha)\to K$, tenemos que:
                \begin{equation*}
                    Ex(\tau,\sigma) = \biguplus_{\eta\in Ex(\tau,\sigma')}Ex(\eta,\iota)
                \end{equation*}
                con lo que:
                \begin{equation*}
                    |Ex(\tau,\sigma)| = \sum_{\eta \in  Ex(\tau, \sigma')}|Ex(\eta,\iota)|
                \end{equation*}
                de la proposición de extensión deducimos que $Ex(\tau,\sigma')$ tiene tantos elementos como raíces de $p^\tau$ hay en $E$. Sin embargo, como $f^\tau$ se factoriza como producto de polinomios de grado 1 en $E[x]$ y $p^\tau$ es un factor de $f^\tau$, en particular $Ex(\tau,\sigma')\neq \emptyset $, lo que nos permite tomar $\eta \in Ex(\tau,\sigma')$, y por hipótesis de inducción obtenemos que $Ex(\eta, \iota)$ es no vacío, con lo que tampoco puede serlo $Ex(\tau,\sigma)$.

                Además, si $f^\sigma$ tiene $degf$ raíces distintas, entonces $p^\sigma$ tiene $degp$ raíces distintas, de donde:
                \begin{equation*}
                    |Ex(\tau,\sigma')| = \cc{R}(p^\sigma) = degp^{\sigma} = [\sigma(F)(\alpha):\sigma(F)]
                \end{equation*}
                Por hipótesis de inducción (como $[K:\sigma(F)(\alpha)]<n$), para cada $\eta \in Ex(\tau,\sigma')$ tenemos que $|Ex(\eta, \iota)| = [K:\sigma(F)(\alpha)]$, con lo que:
                \begin{equation*}
                    |Ex(\tau,\sigma)| = \sum_{\eta \in  Ex(\tau, \sigma')}|Ex(\eta,\iota)| = [K:\sigma(F)(\alpha)][\sigma(F)(\alpha):\sigma(F)] = [K:\sigma(F)]
                \end{equation*}
        \end{itemize}
    \end{proof}
\end{prop}

\begin{ejemplo}
    (Continuación del ejemplo anterior)

    \noindent
    Sea $K = \mathbb{Q}\left(\sqrt[3]{2},w\right)$ con $w$ una raíz cúbica primitiva de la unidad, si queremos calcular todos los homomorfismos de $K$ en $\mathbb{C}$, lo que haremos será considerar las respectivas aplicaciones de inclusión $\tau,\sigma_1$ y $\sigma_2$, con lo que tenemos:
    \begin{figure}[H]
        \centering
        \shorthandoff{""}
        \begin{tikzcd}
        \mathbb{Q} \arrow[d, "\sigma_1"'] \arrow[r, "\tau"]                   & \mathbb{C}           \\
        {\mathbb{Q}(\sqrt[3]{2})} \arrow[ru, "\eta_j"] \arrow[r, "\sigma_2"'] & K \arrow[u, "\eta"']
        \end{tikzcd}
        \shorthandon{""}
    \end{figure}
    \noindent
    Y queremos calcular $Ex(\tau,\sigma_1\sigma_2)$. Para ello, trataremos de usar las aplicaciones $\eta_j$ que ya conocemos, que cumplían:
    \begin{equation*}
        \eta_j\left(\sqrt[3]{2}\right) = w^j \sqrt[3]{2} \qquad \forall j \in \{0,1,2\}
    \end{equation*}
    Calcularemos para cada $j$ todas las $\sigma_2-$extensiones de $\eta_j$, ya que:
    \begin{equation*}
        Ex(\tau,\sigma_2\sigma_1) = \biguplus_{\eta \in  Ex(\tau,\sigma_1)}Ex(\eta, \sigma_2) = Ex(\eta_0,\sigma_2) \cup Ex(\eta_1, \sigma_2) \cup Ex(\eta_2,\sigma_2)
    \end{equation*}
    Para ello, necesitamos calcular el polinomio irreducible de $w$ sobre $\mathbb{Q}\left(\sqrt[3]{2}\right)$ y calcular sus raíces en $\mathbb{C}$, cosa que ya hemos realizado en alguna ocasión:
    \begin{equation*}
        Irr\left(w,\mathbb{Q}\left(\sqrt[3]{2}\right)\right) = x^2+x+1 \qquad \text{con raíces\ } w, w^2
    \end{equation*}
    Por tanto, tendremos 2 $\sigma_2$ extensiones de $\eta_j$ para cada $j \in \{0,1,2\}$:
    \begin{equation*}
        \eta_{j,k}(w) = w^k \qquad k \in \{1,2\}
    \end{equation*}
    \begin{equation*}
        Ex(\tau,\sigma_2\sigma_1) = \{\eta_{j,k} : j\in \{0,1,2\}, k\in \{1,2\}\}
    \end{equation*}
    determinadas por
    \begin{equation*}
        \eta_{j,k}\left(\sqrt[3]{2}\right) = w^j\sqrt[3]{2}, \qquad \eta_{j,k}(w)= w^k
    \end{equation*}
    Sabíamos que teníamos que obtener 6 extensiones, porque todas las raíces obtenidas son distintas.
\end{ejemplo}

\begin{ejercicio}\label{ej:F-lineal}
    Sean $F\stackrel{\tau}{\to}E\stackrel{\rho}{\to}E$ homomorfismos de cuerpos. Sabemos que $E$ es un $\tau(F)-$espacio vectorial, se verifica que: 
    \begin{equation*}
        \rho \text{\ es\ } \tau(F)-\text{lineal} \Longleftrightarrow \rho\tau = \tau
    \end{equation*}

    \begin{description}
        \item [$\Longleftarrow )$] Sea $y\in \tau(F)$ y $z\in E$, tenemos que existe $x\in F$ de forma que $\tau(x)=y$, con lo que:
            \begin{align*}
                \rho(y\cdot z) = \rho(\tau(x)\cdot z) = \rho(\tau(x))\cdot \rho(z) = \rho(x)\cdot \rho(z) = y\cdot \rho(z)
            \end{align*}
        \item [$\Longrightarrow )$] Supuesto que $\rho$ es $\tau(F)-$lineal, tenemos que:
            \begin{equation*} 
                \rho(\tau(x)) = \rho(\tau(x)\cdot 1)= \tau(x)\cdot \rho(1) = \tau(x)\cdot 1 = \tau(x) \qquad \forall x\in E
            \end{equation*}
    \end{description}
\end{ejercicio}

\begin{teo}[Unicidad del cuerpo de descomposición] \ \\
    Sean $\tau:F\to E$ y $\tau':F\to E'$ dos cuerpos de descomposición de $f\in F[x]$. Entonces, existe un isomorfismo de cuerpos $\eta:E\to E'$ tal que $\eta \tau = \tau'$.
    \begin{proof}
        La Proposición~\ref{prop:extension_3} nos dice que como $f^\tau$ y $f^{\tau'}$ se descomponen como producto de polinomios lineales en $E[x]$ y $E'[x]$ de forma respectiva, entonces $Ex(\tau,\tau')$ y $Ex(\tau',\tau)$ son no vacíos, con lo que existen $\eta:E\to E'$ y $\eta':E'\to E$ tales que 
        \begin{equation*}
            \eta'\tau' = \tau\qquad \eta\tau = \tau'
        \end{equation*}

        si observamos que:
        \begin{equation*}
            \eta\eta'\tau' = \tau'
        \end{equation*}
        el Ejercicio~\ref{ej:F-lineal} nos dice que $\eta\eta'$ es $F-$lineal. Ahora, como $E'$ es de dimensión finita sobre $\tau'(F)$ por ser $E'[x]$ cuerpo de descomposición de $f^{\tau'}$; y como tenemos que $\eta\eta':E\to E$ es inyectiva, obtenemos automáticamente que $\eta\eta'$ es biyectiva. De aquí concluimos que $\eta$ es sobreyectiva, pero como era un homomorfismo de cuerpos, concluimos que $\eta$ es biyectiva, con lo que $\eta$ es un isomorfismo.
    \end{proof}
\end{teo}

\begin{ejercicio}
    Sea $\sigma:F\to E$ un homomorfismo de cuerpos tal que la extensión $\sigma(F)\leq E$ es finita. Demostrar que existe un polinomio $f\in F[x]$ y un homomorfismo de cuerpos $\tau:E\to K$ tal que $\tau\sigma:F\to K$ es cuerpo de descomposición de $f$.\\

    \noindent
    Como la extensión $\sigma(F)\leq E$ es finita, sabemos entonces que es algebraica y finitamente generada, con lo que existen $\alpha_1, \ldots, \alpha_n\in E$ algebraicos sobre $\sigma(F)$ de forma que $E = \sigma(F)(\alpha_1, \ldots, \alpha_n)$. Obtenemos para todo $i \in \{1,\ldots,n\}$:
    \begin{equation*}
        g_i = Irr(\alpha_i, \sigma(F))
    \end{equation*}
    con lo que $g_i(\alpha_i)=0\quad \forall i \in \{1,\ldots,n\}$. Como $\sigma:F\to\sigma(F)$ es un isomorfismo, para cada $g_i$ existe un único polinomio $f_i \in F[x]$ de forma que $f_i^\sigma=g_i$. Consideramos:
    \begin{equation*}
        f = \prod_{i=1}^{n} f_i \Longrightarrow f^\sigma =\prod_{i=1}^{n}f_i^\sigma= \prod_{i=1}^{n}g_i \in \sigma(F)[x]
    \end{equation*}
    Por la Proposición~\ref{prop:existe_cdd}, sabemos que podemos encontrar $\theta:\sigma(F)\to K$ cuerpo de descomposición de $f^{\sigma}$. Trataremos ahora de extender $\theta$ a $E$. Para ello, si observamos que:
    \begin{figure}[H]
        \centering
        \shorthandoff{""}
        \begin{tikzcd}
            \sigma(F) \arrow[r, "\theta"] \arrow[rd, "\iota"] & K                   \\
                                                            & \sigma(F)(\alpha_1)
        \end{tikzcd}
        \shorthandon{""}
    \end{figure}
    \noindent
    y recordamos que $g_1\in \sigma(F)[x]$ es irreducible en $\sigma(F)$, la proposición de extensión nos dice que existe existe $\eta_1 \in Ex(\theta,\iota)$ de forma que $\eta_1(\alpha_1)$ es una raíz de $g_1^\theta$ (y por tanto de ${(f^\sigma)}^{\theta}$) en $K$. Supuesto ahora que:
    \begin{figure}[H]
        \centering
        \shorthandoff{""}
        \begin{tikzcd}
            \sigma(F)(\alpha_1, \ldots,\alpha_k) \arrow[r, "\eta_k"] \arrow[rd, "\iota"] & K                   \\
                                                                                         & \sigma(F)(\alpha_1, \ldots, \alpha_{k+1})
        \end{tikzcd}
        \shorthandon{""}
    \end{figure}
    Si tomamos $Irr(\alpha_{k+1},\sigma(F)(\alpha_1, \ldots, \alpha_k))$ (divisor de $g_{k+1}$), la proposición de extensión nos garantiza la existencia de $\eta_{k+1}\in Ex(\eta_k,\iota)$ de forma que $\eta_{k+1}(\alpha_{k+1})$ es una raíz de $g_{k+1}^{\eta_k}$. Tomando ahora $\tau = \eta_{n}$, tenemos $\tau:\sigma(F)(\alpha_1, \ldots, \alpha_n) = E\to K$ de forma que ${(f^{\sigma})}^{\tau}$ se descompone como producto de polinomios lineales en $K[x]$, y si $\cc{R}$ es el conjunto de raíces de ${(f^{\sigma})}^{\tau}$, $K = \sigma(F)(\cc{R})$, con lo que $K$ es cuerpo de descomposición de ${(f^{\sigma})}^{\tau}$.
\end{ejercicio}

\section{Clasificación de los cuerpos finitos} 
\begin{prop}\label{prop:cuerpo_descomposicion}
    Sea $F$ un cuerpo finito con\footnote{Sabemos que es así por el Ejercicio~\ref{ej:cardinal_cuerpo}.} $q=p^n$ elementos (para $p$ la característica de $F$), entonces $F$ es cuerpo de descomposición de $x^q-x\in \bb{F}_p[x]$.
    \begin{proof}
        Llamamos $f=x^q-x$ y consideramos el grupo $F^{\times} = F\setminus\{0\}$, que tiene $q-1$ elementos. Por el Teorema de Lagrange para grupos tenemos que todo $\alpha\in F^{\times}$ satisface que $\alpha^{q-1}=1$, de donde $\alpha^q = \alpha$. Para $0$ es trivial, con lo que:
        \begin{equation*}
            \alpha^q = \alpha \qquad \forall \alpha\in F
        \end{equation*}
        es decir, todo elemento de $F$ es raíz de $x^q-x$. Como su polinomio derivado es $qx^{q-1}-1 = 0$, tenemos entonces que $x^q-x$ tiene exactamente $q$ raíces distintas, que son todos aquellos elementos de $F$, con lo que $F$ es cuerpo de descomposición de $f\in \bb{F}_p[x]$.
    \end{proof}
\end{prop}

\begin{ejercicio}\label{ej:cuerpo_caracteristica}
    Sean $a,b\in F$ con $F$ un cuerpo de característica $p>0$. Si $q=p^n$, comprobar que ${(a-b)}^{q} = a^q - b^q$.\\

    \noindent
    Veamos en primer lugar que:
    \begin{equation*}
        {(a-b)}^{p} = \sum_{k=0}^{p} \binom{p}{k} a^{p-k}{(-b)}^k = a^p - b^p + \sum_{k=1}^{p-1} \dfrac{p!}{k!(p-k)!}a^{p-k}{(-b)}^{k} \AstIg a^p - b^p
    \end{equation*}
    donde en $(\ast)$ usamos que para $1<k<p-1$ tenemos que $\binom{p}{k}$ es múltiplo de $p$. Observemos ahora que:
    \begin{equation*}
        {(a-b)}^{p^2} = {({(a-b)}^{p})}^{p} = {(a^p-b^p)}^{p} = a^{p^2} - b^{p^2}
    \end{equation*}
    Y por un procedimiento inductivo se termina probando que ${(a-b)}^{q} = a^q - b^q$.
\end{ejercicio}

\begin{teo}[Clasificación de cuerpos finitos]
    Para cada número primo $p$ y para cada $n\in \mathbb{N}\setminus\{0\}$ existe un único, salvo isomorfismos, cuerpo de cardinal $p^n$. Además, estos son los únicos cuerpos finitos.
    \begin{proof}
        Sea $q=p^n$, tomamos como $F$ un cuerpo de descomposición del polinomio $f=x^q-x\in \bb{F}_p[x]$. Sea $S$ el conjunto de las raíces de $f$ en $F$, veamos que $S$ es un subcuerpo de $F$, puesto que:
        \begin{itemize}
            \item $1\in S$.
            \item Si $a,b\in S$:
                \begin{equation*}
                    \left.\begin{array}{l}
                        a^q - a = 0 \\
                        b^q - b = 0 
                    \end{array}\right\} \Longrightarrow a^qb^q = ab \Longrightarrow {(ab)}^{q}-ab = 0
                \end{equation*}
                con lo que $ab\in S$, y vemos ahora que:
                \begin{equation*}
                    {(a-b)}^{q}-(a-b) \AstIg a^q - b^q - (a-b) = a-b-(a-b) = 0
                \end{equation*}
                donde en $(\ast)$ usamos el Ejercicio~\ref{ej:cuerpo_caracteristica}, con lo que también $a-b\in S$.
            \item Ahora, si $a\in S\setminus\{0\}$, tenemos que:
                \begin{equation*}
                    {(a^{-1})}^{q} = a^{-q} = {(a^{q})}^{-1} = a^{-1} \Longrightarrow {(a^{-1})}^{q}-a^{-1} = 0
                \end{equation*}
                por lo que $a^{-1}\in S$.
        \end{itemize}
        Finalmente, como $F$ es un cuerpo de descomposición de $f$, ha de ser $S = F$. Además, como el polinomio derivado no comparte raíces con $f$, tenemos que $|F| = q$.\\

        \noindent
    Ahora, si tenemos dos cuerpos del mismo cardinal $q$, la Proposición~\ref{prop:cuerpo_descomposicion} nos dice que ambos cuerpos son cuerpos de descomposición de $x^q-x\in \bb{F}_p[x]$, y aplicando el Teorema de unicidad del cuerpo de descomposición, tenemos que son isomorfos.\\

    \noindent
    Sea ahora $F$ cualquier cuerpo, tenemos por el Ejercicio~\ref{ej:cardinal_cuerpo} que este tiene cardinal $p^n$, por lo que tenemos el resultado por lo que acabamos de probar.
    \end{proof}
\end{teo}

\begin{notacion}
    Si $F$ es un cuerpo de $q=p^n$ elementos, lo notaremos por $\bb{F}_q$, y hablaremos ``del'' cuerpo de $q$ elementos.
\end{notacion}

\begin{ejemplo}
    Sabemos ya que:
    \begin{equation*}
        \dfrac{\mathbb{Z}[i]}{\langle 3 \rangle }, \qquad \dfrac{\bb{F}_3[x]}{\langle x^2+x+2 \rangle }
    \end{equation*}
    son dos cuerpos de $9$ elementos, con lo que el Teorema recién probado nos dice que ambos son isomorfos.
\end{ejemplo}

\section{El grupo de automorfismos de una extensión}
\begin{definicion}[Grupo de automorfismos de un cuerpo]
    Sea $K$ un cuerpo, consideramos el conjunto de todos los automorfismos de $K$:
    \begin{equation*}
        Aut(K) = \{\sigma:K\to K \text{\ homomorfismo de cuerpos biyectivo}\}
    \end{equation*}
    Se verifica que $Aut(K)$ es un grupo con la operación composición de aplicaciones, que recibe el nombre de \underline{grupo de automorfismos de $K$}. \\

    \noindent
    Si $F\leq K$ es una extensión de cuerpos, tomamos:
    \begin{equation*}
        Aut_F(K) = \{\sigma\in Aut(K) : \sigma \text{\ es\ } F-\text{lineal}\}
    \end{equation*}
    y se verifica que $Aut_F(K)$ es un subgrupo de $Aut(K)$, que recibe el nombre de \underline{grupo de automorfismos de $F\leq K$}.\\
\end{definicion}

\begin{ejercicio} % // TODO: HACER, es la idea de siempre
    Si $\Pi$ es el subcuerpo primo de $K$, entonces $Aut_\Pi(K) = Aut(K)$.
\end{ejercicio}

\begin{prop}\label{prop:comienzo_chap_2}
    Si $F\leq K$ es finita, entonces $|Aut_F(K)| \leq [K:F]$
    \begin{proof}
        Si llamamos $F\stackrel{\iota}{\to}K$ al homomorfismo inclusión, entonces:
        \begin{equation*}
            Aut_F(K) = Ex(\iota, \iota)
        \end{equation*}
        \begin{description}
            \item [$\subseteq )$] Si $\sigma\in Aut_F(K)$, tenemos entonces que $\sigma$ es $F-$lineal, y por el Ejercicio~\ref{ej:F-lineal}, tenemos entonces que $\sigma\iota = \iota$, lo que nos dice que $\sigma\in Ex(\iota,\iota)$.
            \item [$\supseteq )$] Si tomamos $\sigma\in Ex(\iota,\iota)$ como es homomorfismo de cuerpos tenemos que es inyectivo, y como es $F-$lineal, ha de ser necesariamente sobreyectivo, con lo que $\sigma\in Aut_F(K)$
        \end{description}
        Finalmente, la segunda proposición de extensión nos dice que:
        \begin{equation*}
            |Aut_F(K)| = |Ex(\iota,\iota)| \leq [K:F]
        \end{equation*}
    \end{proof}
\end{prop}

\begin{prop}\label{prop:cuerpo_desc_aut_f_lin}
    Si $F\leq K$ es cuerpo de descomposición de $f\in F[x]$, entonces: 
    \begin{equation*}
        |Aut_F(K)| \leq [K:F]
    \end{equation*}
    y si todas las raíces de $f$ en $K$ son simples (es decir, $f$ tiene $degf$ raíces distintas), entonces:
    \begin{equation*}
        |Aut_F(K)| = [K:F]
    \end{equation*}
    \begin{proof}
        Si $F\leq K$ es un cuerpo de descomposición de $f\in F[x]$, tenemos entonces que si $\alpha_1, \ldots, \alpha_s$ son las raíces de $f$ en $K$ entonces $K = F(\alpha_1, \ldots, \alpha_s)$ es una extensión algebraica y finitamente generada, luego finita, de donde aplicando la Proposición anterior tenemos que $|Aut_F(K)|\leq [K:F]$.\\

        \noindent
        Si ahora tenemos que todas las raíces de $f$ en $K$ son simples, aplicando que $|Aut_F(K)| = |Ex(\iota,\iota)|$ para $\iota:F\to K$ la aplicación inclusión, tenemos por la tercera propiedad de extensión que:
        \begin{equation*}
            |Aut_F(K)| = |Ex(\iota,\iota)| = [K:F]
        \end{equation*}
    \end{proof}
\end{prop}

\begin{ejemplo}
    Según un ejemplo ya visto, tenemos que:
    \begin{equation*}
        Aut\left(\mathbb{Q}\left(\sqrt[3]{2}, w\right)\right) = Aut_\mathbb{Q}\left(\mathbb{Q}\left(\sqrt[3]{2}, w\right)\right)
    \end{equation*}
    con lo que la Proposición nos dice que:
    \begin{equation*}
        \left|Aut_\mathbb{Q}\left(\mathbb{Q}\left(\sqrt[3]{2},w\right)\right)\right| = 6
    \end{equation*}
    Por Álgebra II, tenemos que este grupo es isomorfo a $C_6$ o a $S_3$, pero en ejemplos anteriores vimos que:
    \begin{equation*}
        Aut\left(\mathbb{Q}\left(\sqrt[3]{2},w\right)\right) = \{\eta_{j,k} : j \in \{0,1,2\}, k\in \{1,2\}\}
    \end{equation*}

    donde:
    \begin{equation*}
        \left\{\begin{array}{ccl}
                \eta_{j,k}\left(\sqrt[3]{2}\right) &=& w^j\sqrt[3]{2} \\
                \eta_{j,k}(w) &=& w^k
        \end{array}\right.
    \end{equation*}
    resulta que tenemos un grupo no conmutativo:
    \begin{align*}
        \sqrt[3]{2} &\stackrel{\eta_{1,1}}{\longmapsto} w\sqrt[3]{2} \stackrel{\eta_{1,0}}{\longmapsto} w\sqrt[3]{2} \\
        \sqrt[3]{2} &\stackrel{\eta_{1,0}}{\longmapsto} w\sqrt[3]{2} \stackrel{\eta_{1,1}}{\longmapsto} w^2\sqrt[3]{2} 
    \end{align*}
    por lo que es isomorfo a $S_3$.
\end{ejemplo}

\begin{teo}
    Sea $\bb{F}_q$ un cuerpo finito con $q=p^n$ elementos, entonces $Aut(\bb{F}_q)$ es un grupo cíclico de orden $n$.
    \begin{proof}
        Sabemos por la Proposición~\ref{prop:cuerpo_descomposicion} que $\bb{F}_q$ es cuerpo de descomposición de $x^q-x\in \bb{F}_q[x]$, así como que las raíces de dicho polinomio son todas distintas (puesto que no comparte raíces con su polinomio derivado). Estamos en las condiciones de aplicar la Proposición~\ref{prop:cuerpo_desc_aut_f_lin}, obteniendo que:
        \begin{equation*}
            |Aut(\bb{F}_q)| = |Aut_{\bb{F}_p}(\bb{F}_q)| = [\bb{F}_q : \bb{F}_p] = n
        \end{equation*}
        Sea $\tau:\bb{F}_q\to \bb{F}_q$ la aplicación:
        \begin{equation*}
            \tau(a) = a^{p} \qquad \forall a\in \bb{F}_q
        \end{equation*}
        tenemos por el Ejercicio~\ref{ej:cuerpo_caracteristica} que es un homomorfismo de cuerpos, luego un automorfismo (que recibe el nombre de automorfismo de Frobenius). Veamos que su oren es $n$. Para ello, sea $m\in \mathbb{N}\setminus \{0\}$ de forma que:
        \begin{equation*}
            \tau^m = id_{\bb{F}_q}
        \end{equation*}
        En el Ejercicio~\ref{ej:subgrupo_finito} vimos que $\bb{F}_q^{\times}$ es cíclico y de orden $q-1$. Tomamos $a$ como su generador, que será de orden $q-1$, lo que nos dice entonces que:
        \begin{equation*}
            a = \tau^m (a)  = a^{p^m}
        \end{equation*}
        Usando que el orden de $a$ es $p^n-1$, deducimos que $p^m -1\geq p^n -1$, luego $m\geq n$, de donde el orden de $\tau\in Aut(\bb{F}_q)$ es $n$, con lo que $Aut(\bb{F}_q)$ está generado por $\tau$.
    \end{proof}
\end{teo}

\section{Ejercicios}

\begin{ejercicio} 
    Sea $F\leq K$ una extensión de cuerpos de grado 2. Mostrar que, si la característica de $F$ es distinta de dos, existe $\beta\in K$ tal que $\beta^2 \in F$ y $K = F(\beta)$.\\

    \noindent
    Sea $\alpha\in K\setminus F$, tenemos que $\alpha$ tiene grado 2 sobre $K$, puesto que si fuera de grado 1, entonces existe un polinómico mónico de grado 1 $x-a$ (con $a\in F$) de forma que $\alpha$ es raíz de dicho polinomio, con lo que ha de ser $a = \alpha\notin F$, contradicción. De esta forma, $deg Irr(\alpha,F)=2$, es decir, existen $a,b\in F$ de forma que $\alpha$ es raíz del polinomio:
    \begin{equation*}
        x^2+ax+b
    \end{equation*}
    Por lo que $\alpha^2 + a\cdot \alpha + b = 0$. Como la característica de $F$ no es dos, tenemos que $1+1 = 2 \neq 0$, con lo que podemos considerar $2^{-1}$. Si tomamos:
    \begin{equation*}
        \beta = \alpha + \frac{a}{2}
    \end{equation*}

    tenemos que:
    \begin{equation*}
        \beta^2 = {\left(\alpha+\frac{a}{2}\right)}^{2} = \alpha^2 + \alpha\cdot a + \frac{a^2}{4} = -b + \frac{a^{2}}{4} \in F
    \end{equation*}
    Y además $\beta\notin F$, pues $\alpha = \beta-\frac{a}{2}$. Como $\beta\in K$, es obvio que $F(\beta)\leq K$, y como $[F(\beta):F] = [K:F]$, ha de ser $K = F(\beta)$.
\end{ejercicio}

\begin{ejercicio}
    Calcular un cuerpo de descomposición de $x^4+16\in \mathbb{Q}[x]$.\\

    \noindent
    Tenemos que:
    \begin{equation*}
        x^4+16= 0 \Longleftrightarrow x=\sqrt[4]{-16} = 2\sqrt[4]{-1}
    \end{equation*}
    Si recordamos que:
    \begin{equation*}
        \sqrt[4]{-1} = \left\{e^{\frac{i}{n}(\pi + 2k\pi)} : k\in \{0,1,2,3\}\right\} = \left\{e^{\frac{i\pi}{4}}, e^{\frac{3i\pi}{4}}, e^{\frac{5i\pi}{4}}, e^{\frac{7i\pi}{4}}\right\}
    \end{equation*}

    con:
    \begin{align*}
        e^{\frac{i\pi}{4}} &= \cos\left(\frac{\pi}{4}\right) + i\sen\left(\frac{\pi}{4}\right) = \frac{\sqrt{2}}{2} + i\frac{\sqrt{2}}{2} \\
        e^\frac{3i\pi}{4} &= \cos\left(\frac{3\pi}{4}\right) + i\sen\left(\frac{3\pi}{4}\right) = -\frac{\sqrt{2}}{2} + i\frac{\sqrt{2}}{2} \\
        e^\frac{5i\pi}{4} &= \cos\left(\frac{5\pi}{4}\right) + i\sen\left(\frac{5\pi}{4}\right) = -\frac{\sqrt{2}}{2} - i\frac{\sqrt{2}}{2} \\
        e^\frac{7i\pi}{4} &= \cos\left(\frac{7\pi}{4}\right) + i\sen\left(\frac{7\pi}{4}\right) = \frac{\sqrt{2}}{2} - i\frac{\sqrt{2}}{2} 
    \end{align*}
    Por lo que:
    \begin{equation*}
        \sqrt[4]{-16} = \left\{\sqrt{2}+i\sqrt{2}, -\sqrt{2}+i\sqrt{2}, -\sqrt{2}-i\sqrt{2},\sqrt{2}-i\sqrt{2}\right\}
    \end{equation*}
    Con lo que $\mathbb{Q}\left(\sqrt{2}+i\sqrt{2}, \sqrt{2}-i\sqrt{2}\right)$ es un cuerpo de descomposición de $x^4+16$, que trataremos de probar que es igual a $\mathbb{Q}\left(i,\sqrt{2}\right)$:
    \begin{description}
        \item [$\subseteq )$] Es claro que $\mathbb{Q}\left(\sqrt{2}+i\sqrt{2}, \sqrt{2}-i\sqrt{2}\right)\leq \mathbb{Q}\left(i,\sqrt{2}\right)$.
        \item [$\supseteq )$] Vemos que:
            \begin{align*}
                \sqrt{2} &= \dfrac{\sqrt{2}+i\sqrt{2}+\sqrt{2}-i\sqrt{2}}{2} \in \mathbb{Q}\left(\sqrt{2}+i\sqrt{2}, \sqrt{2}-i\sqrt{2}\right) \\
                i &= \dfrac{\sqrt{2}+i\sqrt{2}-\sqrt{2}}{\sqrt{2}} \in  \mathbb{Q}\left(\sqrt{2}+i\sqrt{2}, \sqrt{2}-i\sqrt{2}\right)
            \end{align*}
    \end{description}
    En definitiva, $\mathbb{Q}\left(i,\sqrt{2}\right)$ es un cuerpo de descomposición de $x^4+16$.
\end{ejercicio}

\begin{ejercicio} 
    Razonar cuáles de los siguientes números complejos son algebraicos sobre $\mathbb{Q}$, suponiendo conocido que $e$ y $\pi$ son trascendentes:
    \begin{equation*}
        \sqrt[5]{4}, (1+\sqrt[5]{4}){(1-\sqrt[5]{16})}^{-1}, \pi^2, e^2-i, i\sqrt{i}+\sqrt{2}, \sqrt{1-\sqrt[3]{2}}, \sqrt{\pi}, \sqrt{2}{(\sqrt[3]{2}+\sqrt[5]{2})}^{-1}.
    \end{equation*}
    Veamos cada caso:
    \begin{itemize}
        \item $\sqrt[5]{4}$ es algebraico sobre $\mathbb{Q}$, puesto que es raíz de $x^5-4$.
        \item $\left(1+\sqrt[5]{4}\right){\left(1-\sqrt[5]{16}\right)}^{-1}$

            En el apartado anterior hemos visto que $[\mathbb{Q}\left(\sqrt[5]{4}\right) :\mathbb{Q}]\leq 5$, con lo que la extensión $\mathbb{Q}\leq \mathbb{Q}\left(\sqrt[5]{4}\right)$ es finita, luego algebraica y finitamente generada, por lo que todo elemento de este último cuerpo será algebraico sobre $\mathbb{Q}$. Observemos que:
            \begin{equation*}
                (1+\sqrt[5]{4}){\left(1-\sqrt[5]{16}\right)}^{-1} = (1+\sqrt[5]{4}){\left(1-\sqrt[5]{4}\sqrt[5]{4}\right)}^{-1} \in \mathbb{Q}(\sqrt[5]{4})
            \end{equation*}
            Por lo que es algebraico sobre $\mathbb{Q}$.
        \item $\pi^2$%//TODO: HACER
        \item $e^2-i$ % // TODO: HACER
        \item $i\sqrt{i}+\sqrt{2}$ % // TODO: HACER
        \item $\sqrt{1-\sqrt[3]{2}}$

            Buscamos un polinomio con coeficientes en $\mathbb{Q}$ del que este elemento sea raíz. Para ello, lo que haremos será ver que este ha de cumplir que:
            \begin{equation*}
                x^2 = 1-\sqrt[3]{2} \Longrightarrow x^2-1 = -\sqrt[3]{2}
            \end{equation*}
            De donde:
            \begin{equation*}
                {(x^2-1)}^{3} = x^6-3x^4+3x^2-1 = -2
            \end{equation*}
            Por lo que si tomamos $f = x^6-3x^4+3x^2+1\in \mathbb{Q}[x]$, tenemos que $\sqrt{1-\sqrt[3]{2}}$ es raíz de $f$, con lo que es algebraico sobre $\mathbb{Q}$.
        \item $\sqrt{\pi}$ % // TODO: HACER
        \item $\sqrt{2}{(\sqrt[3]{2}+\sqrt[5]{2})}^{-1}$ % // TODO: HACER
    \end{itemize}
\end{ejercicio}

\begin{ejercicio}
    Sea $F\leq K$ una extensión de cuerpos, $\alpha\in K$ y $n$ natural no nulo. Demostrar que $\alpha$ es algebraico sobre $F$ si, y solo si, $\alpha^n$ es algebraico sobre $F$.
\end{ejercicio}

\begin{ejercicio}
    Sea $F\leq K$ una extensión de cuerpos, $\alpha\in K$ y $\beta = 1+\alpha^2 + \alpha^5$. Demostrar que $\alpha$ es algebraico sobre $F$ si, y solo si, $\beta$ es algebraico sobre $F$:
\end{ejercicio}

\begin{ejercicio}
    Calcular $Irr(\alpha,\mathbb{Q})$ para los siguientes valores de $\alpha$:
    \begin{equation*}
        3+\sqrt{2}, \sqrt{3}-\sqrt[4]{3}, \sqrt[3]{2}+\sqrt[3]{4}
    \end{equation*}
\end{ejercicio}

\begin{ejercicio}
    Calcular $[E:\mathbb{Q}]$ y una base de $E$ sobre $\mathbb{Q}$ en los siguientes casos:
    \begin{equation*}
        E = \mathbb{Q}\left(\sqrt{6},i\right), \qquad E= \mathbb{Q}\left(\sqrt[3]{5},\sqrt{-2}\right), \qquad E = \mathbb{Q}\left(\sqrt{18},\sqrt[3]{4}\right)
    \end{equation*}
\end{ejercicio}

\begin{ejercicio}
    Sea $\alpha\in \mathbb{C}$ una raíz del polinomio $x^3+3x+1$. Describir una base de $\mathbb{Q}(\alpha)$ sobre $\mathbb{Q}$ y calcular las coordenadas racionales con respecto de la misma de $(1+\alpha){(1+\alpha+\alpha^2)}^{-1}$.
\end{ejercicio}

\begin{ejercicio}
    Pongamos $\bb{F}_4 = \bb{F}_2(a)$ con $a^2+a+1=0$. Comprobar que $\bb{F}_{16}$ puede presentarse como $\bb{F}_{16}=\bb{F}_2(b)$, donde $b^4+b+1=0$. Determinar todos los homomorfismos de cuerpos $\bb{F}_4\to \bb{F}_{16}$ en función de $a$ y $b$.\\

    \noindent
    Tomamos $x^4+x+1\in \bb{F}_2[x]$, y comprobamos que es irreducible para ello, comporobamos que no tiene raíces y que no puede escribirse como producto de dos polinomios irreducibles de grado 2. Como el único polinomio irreducible de grado 2 en $\bb{F}_2[x]$ es $x^2+x+1$, basta ver que no es cuadrado del mismo. Como consecuencia:
    \begin{equation*}
        \dfrac{\bb{F}_2[x]}{\langle x^4+x+1 \rangle }
    \end{equation*}
    es un cuerpo, que tiene dimensión 4 (el grado de $x^4+x+1$) sobre $\bb{F}_2$, con lo que el cuerpo ``es'' $\bb{F}_{16}$. Tomamos $b = x+\langle x^4+x+1 \rangle $ y se tiene.\\

    \noindent
    Para ver ahora todos los homomorfismos de cuerpos, conocido:
    \begin{equation*}
        p = Irr(a,\bb{F}_2) = x^2+x+1
    \end{equation*}
    \begin{figure}[H]
        \centering
        \shorthandoff{""}
        \begin{tikzcd}
            \mathbb{F}_2 \arrow[r, "\iota"] \arrow[rd, "\iota"] & \mathbb{F}_2(a) \arrow[d, "\eta"] \\
                                                                & \mathbb{F}_2(b)                  
        \end{tikzcd}
        \shorthandon{""}
    \end{figure}
    \noindent
    tenemos que hay tantos homomorfismos entre dichos cuerpos como raíces de $p$. La Propiedd de Extensión nos dice que los homomorfismos que me piden están parametrizados por las raíces de $p$ en $\bb{F}_2(b)$.\\

    \noindent
    Observemos que en este ejercicio (usando el ejercicio siguiente), cada $\eta$ por restricción nos da un homomorfismo de grupos $\eta:\bb{F}_2^\times(a)\to \bb{F}_2^\times(b)$ como los cardinales son 3 y 15 y 3 divide a 15, hay homomorfismos. Sabemos que $\bb{F}_2(a) = \langle a \rangle $ por ser 3 primo. Ahora, no estamos seguros de si $\bb{F}_2(b) = \langle b \rangle $, para lo cual hemos de probar que $O(b) = 15$.

    \begin{itemize}
        \item $b^2\neq 1$, ya que $b^2+1=0$, ya que $\{1,b,b^2, b^3\}$ es una $\bb{F}_2-$base de $\bb{F}_{16}$.
        \item $b^3\neq 1$ por la misma razón.
        \item $b^4 = b+1\neq 1$, ya que si no $b = 0$.
        \item $b^5 = b(b^4) = b(b+1) = b^2 + b \neq 1$, por la misma razón.
    \end{itemize}
    En definitiva, $O(b)=15$, luego $\bb{F}_{16}^{\times} = \langle b \rangle $.

    \noindent
    Buscando ahora homomorfismos de grupos, tenemos que llevar $a$ en un elemento de orden 3. Ahora, los candidatos a elementos de orden 3 de $\bb{F}_{16}^\times$ son los que generan un grupo de orden $5$ y $10$, es decir, $b^5$ y $b^{10}$, y tenemos que comprobar que son raíces de $x^2+x+1$.

    Finalmente, evalúo $p$ en las candidatas para comprobar que sean raíces:
    \begin{align*}
        p(b^5) &= b^{10} + b^{5} + 1 = {(b^2 + b)}^{2} + b^2 + b + 1 = b^4 + b^2 + b^2 + b + 1 \\
               &= b^4 + b + 1 = 0 
    \end{align*}
    Por el Automorfismo de Frobenius, la otra raíz es $b^{10}$. Sabemos que hay un $\eta$ para cada raíz del polinomio, obteniendo $\eta_i:\bb{F}_4\to \bb{F}_{16}$:
    \begin{equation*}
        \eta_1(a) = b^5, \qquad \eta_2(a) = b^{10}
    \end{equation*}
\end{ejercicio}

% // TODO: Dado un un homomorfismo, restringirlo al primo es la inclusioon, poner esto en los ejemlos de propiedad de extension, se consiguen todas

\begin{ejercicio}\label{ej:subgrupo_finito}
    Demostrar que, si $F$ es un cuerpo, entonces cualquier subgrupo finito de $F^\times$ es cíclico. Deducimos que, en particular, $\bb{F}_q^\times$ es un grupo cíclico de orden $q-1$. (Pista: usar la descomposición cíclica de un grupo finito abeliano).\\

    \noindent
    Sea $G$ un subgrupo finito de $F^\times$, tomamos la descomposición cíclica de $G$:
    \begin{equation*}
        G = C_1 \oplus \ldots \oplus C_t
    \end{equation*}
    Con $C_i$ cíclico para cada $i \in \{1,\ldots,t\}$, con $|C_{i+1}|\mid |C_i|$. Sea $m = |C_1|$, para todo $g\in G$ tenemos que $g^m = 1$. De esta forma, cada elemento de $G$ es raíz de $x^m-1\in F[x]$, que a lo mucho tiene $m$ raíces, con lo que $|G| \leq m\leq |G|$, de donde $|G| = m$, por lo que todos los grupos cíclicos en los que $G$ se descompone son triviales salvo $C_1$, de donde $G$ es cíclico.
\end{ejercicio}

\begin{observacion}
    Para $\bb{F}_q$, $\bb{F}_q^\times$ es un grupo cíclico de orden $q-1$.\newline
    A cualquier generador $a$ de $\bb{F}_q^\times$ se le llama elemento primitivo de $\bb{F}_q$, por lo que:
    \begin{equation*}
        \bb{F}_q = \{0, 1, a, \ldots, a^{q-2}\}
    \end{equation*}
    Por lo que $\bb{F}_q = \bb{F}_p(a)$, con $p = \car(\bb{F}_q)$.
\end{observacion}

\begin{ejercicio}
    Demostrar que los anillos $\frac{\mathbb{Z}[i]}{\langle 3 \rangle }$ y $\frac{\bb{F}_3}{\langle x^2+x+2 \rangle }$ son isomorfos sin necesidad de dar un isomorfismo concreto. ¿Serías capaz de darlo? ¿Y de calcularlos todos?
\end{ejercicio}

\begin{ejercicio} % // TODO: es ejercicio de examen, al final del capitulo 1
    Se pide:
    \begin{enumerate}
        \item Comprobar que $\sqrt{3}\in \mathbb{Q}\left(\sqrt{1+2\sqrt{3}}\right)$.

            Llamamos $\alpha = \sqrt{1+2\sqrt{3}}$ y calculamos:
            \begin{equation*}
                \alpha^2 = 1+2\sqrt{3} \quad \Longrightarrow \quad  \sqrt{3} = \dfrac{\alpha^{2}-1}{2}\in \mathbb{Q}(\alpha)
            \end{equation*}
            De donde también deducimos que $\mathbb{Q}(\sqrt{3})\leq \mathbb{Q}(\alpha)$.
        \item Calcular $Irr\left(\alpha, \mathbb{Q}\left(\sqrt{3}\right)\right)$.

            Sabemos que $\alpha$ es raíz de $f=x^2-1-2\sqrt{3}\in \mathbb{Q}\left(\sqrt{3}\right)[x]$, con lo que:
            \begin{equation*}
                \left[\mathbb{Q}(\alpha):\mathbb{Q}\left(\sqrt{3}\right)\right] \leq 2
            \end{equation*}
            Supongamos que $\left[\mathbb{Q}(\alpha):\mathbb{Q}\left(\sqrt{3}\right)\right]=1$, con lo que $\alpha\in \mathbb{Q}\left(\sqrt{3}\right)$, de donde $\alpha = a + b\sqrt{3}$ para ciertos $a,b\in \mathbb{Q}$. Si elevamos al cuadrado:
            \begin{equation*}
                1 + 2\sqrt{3} = \alpha^2 = a^2 + 3b^2 + 2ab\sqrt{3}
            \end{equation*}
            Usando que $\{1,\sqrt{3}\}$ es una base de $\mathbb{Q}\left(\sqrt{3}\right)$, tenemos entonces que:
            \begin{align*}
                \left.\begin{array}{l}
                    1 = a^2 + 3b^2 \\
                    2 = 2 ab
                    \end{array}\right\} &\Longrightarrow \left\{\begin{array}{l}
                    b = \frac{1}{a} \\
                    1 = a^2 + 3\frac{1}{a^2}
            \end{array}\right\} \Longrightarrow a^2 = a^4 + 3 \\ &\Longrightarrow a^2 = \dfrac{1\pm \sqrt{1-12}}{2} \notin \mathbb{Q} \Longrightarrow a \notin \mathbb{Q}
            \end{align*}
            Por lo que no es posible $\left[\mathbb{Q}(\alpha):\mathbb{Q}\left(\sqrt{3}\right)\right]=1$, con lo que $\left[\mathbb{Q}(\alpha):\mathbb{Q}\left(\sqrt{3}\right)\right]=2$, de donde deducimos que:
            \begin{equation*}
                Irr\left(\alpha,\mathbb{Q}\left(\sqrt{3}\right)\right) = x^2-1-2\sqrt{3}
            \end{equation*}
        \item Calcular los homomofismos de $\mathbb{Q}(\alpha)$ en $\mathbb{C}$.

            Queremos calcular los $\eta$ que cumplen:
            \begin{figure}[H]
                \centering
                \shorthandoff{""}
                \begin{tikzcd}
                \mathbb{Q} \arrow[rd, "\iota"'] \arrow[r, "\tau"] & \mathbb{C}                            \\
                                                                  & \mathbb{Q}(\alpha) \arrow[u, "\eta"']
                \end{tikzcd}
                \shorthandon{""}
            \end{figure}
            donde $\tau,\iota$ son la inclusión, es decir, calcular $Ex(\tau,\iota)$.

            No conocemos $Irr(\alpha,\mathbb{Q})$, pero hemos hecho el apartado 2, con lo que calculamos primero los homomorfismos de $\mathbb{Q}(\sqrt{3})$ a $\mathbb{C}$, que son dos por la Proposición de extensión, determinados por:
            \begin{equation*}
                \eta_j(\sqrt{3}) = {(-1)}^{j}\sqrt{3}, \qquad \forall j\in \{0,1\}
            \end{equation*}
            ya que $Irr\left(\sqrt{3},\mathbb{Q}\right) = x^2-3$. Cada uno de ellos da lugar a 2 homomorfismos de $\mathbb{Q}(\alpha)$ en $\mathbb{C}$. Las extensiones de $\eta_0$, digamos $\eta_{0,k}$ con $k\in \{0,1\}$, determinadas por:
            \begin{equation*}
                \eta_{0,k}(\alpha) = {(-1)}^{k}\alpha \qquad \forall k\in \{0,1\}
            \end{equation*}
            Las extensiones de $\eta_1$ vienen dadas por las raíces en $\mathbb{C}$ de $p^{\eta_1} = x^2-1+2\sqrt{3}$, que son $\pm \beta$, con $\beta = \sqrt{1-2\sqrt{3}}$, con lo que tenemos $\eta_{1,k}$ con $k\in \{0,1\}$ dadas por:
            \begin{equation*}
                \eta_{1,k}(\beta) = {(-1)}^{k}\beta
            \end{equation*}
        \item Calcular $Irr(\alpha,\mathbb{Q})$ y sus raíces en $\mathbb{C}$. % // TODO: HACER

            Sabemos ya que el grado es 4, el polinomio se obtiene elevando $\alpha^2=1+2\sqrt{3}$ al cuadrado, y las raíces las sacamos por la bicuadrática, que salen $\alpha,-\alpha,\beta,-\beta$.
    \end{enumerate}
\end{ejercicio}


\begin{ejercicio}
    Sea $\eta = e^{i\frac{2\pi}{5}}\in \mathbb{C}$, ¿$Irr(\eta+\overline{\eta},\mathbb{Q})$?. Llamando $\alpha = \eta+\overline{\eta}$, observamos que $\alpha = \eta + \eta^4$, y ahora:
    \begin{equation*}
        \alpha^2 = \eta^2 + 2+\eta^8 = \eta^2 + 2 + \eta^3
    \end{equation*}
    Y ahora como:
    \begin{equation*}
        \eta^4 + \eta^3 + \eta^2 + \eta + 1 = 0
    \end{equation*}

    tenemos que:
    \begin{equation*}
        \alpha^2 = \eta^2 + 2 + \eta^3 = 2 - 1 -\eta-\eta^4 = 1-\alpha
    \end{equation*}
    Por lo que $\alpha^2 + \alpha - 1 = 0$, le calculamos las raíces:
    \begin{equation*}
        \alpha = \dfrac{-1\pm \sqrt{5}}{2}\notin \mathbb{Q}
    \end{equation*}
    Y como es de grado 2 ha de ser irreducible, con lo que:
    \begin{equation*}
        Irr(\eta+\overline{\eta},\mathbb{Q}) = x^2+x+1
    \end{equation*}
    Y el número $\eta$ es constructible porque $\eta+\overline{\eta}$ es 2 veces su parte real, y $\sqrt{5}$ es constructible, luego su parte real es constructible. La parte imaginaria la obtenemos del Teorema de Pitágoras, como la raíz cuadrada de cierto número constructible.\\

    \noindent
    Este ejercicio demustra que el pentágono regular es constructible.
\end{ejercicio}
