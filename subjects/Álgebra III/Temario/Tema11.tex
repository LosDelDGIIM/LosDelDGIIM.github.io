% // TODO: Pedir a Irina
\begin{teo}[Unicidad de los cuerpos de descomposición]
    Sean $\tau:F\to E$ y $\tau':F\to E'$ cuerpos de descomposición de $f\in F[x]$. Existe un isomorfismo de cuerpos $\eta:E\to E'$ tal que $\eta \tau = \tau'$.
    \begin{proof}
        Por la proposición de hoy, sabemos que existe $\eta:E\to E'$ y $\eta':E'\to E$ tales que 
        \begin{equation*}
            \eta'\tau' = \tau\qquad \eta'\tau = \tau'
        \end{equation*}

        si observamos que:
        \begin{equation*}
            \eta\eta'\tau' = \tau'
        \end{equation*}

        el ejercicio nos dice que $\eta\eta'$ es $F-$lineal. Ahora, como:
        \begin{equation*}
            [E':\tau'(F)] < \infty
        \end{equation*}
        tenemos entonces que $\eta\eta':E\to E$ es inyectiva, con lo que automáticamente obtenemos que $\eta\eta'$ es biyectiva. De aquí concluimos que $\eta$ es sobreyectiva, pero como era un homomorfismo de cuerpos, concluimos que $\eta$ es biyectiva, con lo que $\eta$ es un isomorfismo.
    \end{proof}
\end{teo}

\section{Clasificación de los cuerpos finitos}
\begin{prop}
    Sea $F$ un cuerpo finito con\footnote{Sabemos que es así por el subcuerpo primo.} $q=p^n$ elementos (para $p$ la característica de $F$), entonces $F$ es cuerpo de descomposición de $x^q-x\in \bb{F}_p[x]$.
    \begin{proof}
        Llamamos $f=x^q-x$, tomamos: $F^{\times} = F\setminus\{0\}$, que tiene $q-1$ elementos. Por el Teorema de Lagrange para grupos tenemos que todo $\alpha\in F^{\times}$ satisface que $\alpha^{q-1}=1$, de donde $\alpha^q = \alpha$. Para $0$ es trivial, con lo que:
        \begin{equation*}
            \alpha^q = \alpha \qquad \forall \alpha\in F
        \end{equation*}
        es decir, todo elemento de $F$ es raíz de $x^q-x$. Como su polinomio derivado es $qx^{q-1}-1 = 0$, tenemos entonces que $x^q-x$ tiene exactamente $q$ raíces distintas, que son todos aquellos elementos de $F$, con lo que $F$ es cuerpo de descomposición de $f\in \bb{F}_p[x]$.
    \end{proof}
\end{prop}
