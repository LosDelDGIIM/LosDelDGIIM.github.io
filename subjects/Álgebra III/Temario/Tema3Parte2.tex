\section{Extensiones cíclicas}
\noindent
Nos interesará ahora el estudio del cuerpo de descomposición y del grupo de Galois de polinomios separables de la forma $x^n-a$.

\begin{teo}\label{teo:ext_ciclicas}
    Si $x^n-a\in F[x]$ es separable y $K$ es su cuerpo de descomposición, entonces $K$ contiene una raíz $n-$ésima primitiva de la unidad $\zeta$ y $K = F(\zeta, r)$ para cualquier raíz $r\in K$ de $x^n-a$.

    \noindent
    Además, el grupo de Galois de la extensión $F(\zeta)\leq K$ es cíclico de orden un divisor de $n$.
    \begin{proof}
        En el caso $a = 0$, tenemos que $x^n$ es separable si y solo si $n = 1$, con lo que $K = F$ y una raíz $n-$ésima primitiva de la unidad es 1, se trivializa el enunciado.\\

        \noindent
        Suponemos por tanto que $a\neq 0$, con lo que $n$ no puede ser múltiplo de $\car(F)$, para que $x^n-a$ sea separable. Sea $R$ el conjunto de las raíces en $K$ de $x^n-a$, tenemos que $|R| = n$, puesto que $x^n-a$ es separable. Si tomamos $r,s\in R$, tenemos que $r^{-1}s\in K$ es una raíz $n-$ésima de la unidad: 
        \begin{equation*}
            {(r^{-1}s)}^{n} = r^{-n}s^n = a^{-1}a = 1
        \end{equation*}
        Fijado $r\in R$, entonces el conjunto $\{r^{-1}s : s\in R\}$ contiene $n$ raíces $n-$ésimas de la unidad distintas, por lo que en dicho conjunto las tenemos todas, luego ha de contener al menos una raíz $n-$ésima primitiva de la unidad, llamémosla $\zeta\in K$ .\\

        \noindent
        Para la segunda afirmación, fijado $r\in R$, es claro ahora que:
        \begin{equation*}
            F(\zeta, r)\leq K
        \end{equation*}
        Para la otra inclusión, si $r=r_1,\ldots,r_n$ son las raíces de $x^n-a$ tenemos entonces que $K = F(r_1,\ldots,r_n)$. Hemos visto que:
        \begin{equation*}
            \{\zeta^k : 0\leq k \leq n-1\} = \{r^{-1}s:s\in R\} \quad\Longrightarrow\quad \{r\zeta^k : 0\leq k\leq n-1\} = R = \{r_1,\ldots,r_n\}
        \end{equation*}
        Por lo que ahora tenemos que:
        \begin{equation*}
            K = F(r_1,\ldots,r_n) \leq F(\zeta,r)
        \end{equation*}

        \noindent
        Para ver que el grupo de Galois de la extensión $F(\zeta)\leq K$ es cíclico, vamos a representar el grupo de manera sencilla. Para ello, tomamos $\sigma\in \Aut_{F(\zeta)}(K)$ y vemos que $\sigma$ toma una raíz de $x^n-a$ y la lleva en otra, siendo el conjunto de todas las raíces:
        \begin{equation*}
            R = \{r, r\zeta, \ldots, r\zeta^{n-1}\}
        \end{equation*}
        Por lo que tendremos $\sigma(r) =  r\zeta^j$ para cierto $j\in \mathbb{Z}_n$. Si tuviéramos que $\sigma(r) = r\zeta^{j'}$ para $j'\in \mathbb{Z}_n$, tendríamos entonces que $\zeta^j = \zeta^{j'}$, pero como $\zeta$ es una raíz $n-$ésima primitiva de la unidad, tenemos que $j=j'$. Esto nos permite definir una aplicación $j:\Aut_{F(\zeta)}(K)\to \mathbb{Z}_n$ de forma que $\sigma\mapsto j$ con $\sigma(r) = r\zeta^j$.

        \noindent
        Veamos que $j$ es un homomorfismo de grupos, considerando $\mathbb{Z}_n$ como grupo aditivo, ya que si $\sigma,\tau \in \Aut_{F(\zeta)}(K)$ y consideramos $j(\sigma\tau)\in \mathbb{Z}_n$ dado por la condición:
        \begin{equation*}
            (\sigma\tau)(r) = r\zeta^{j(\sigma\tau)}
        \end{equation*}
        Tenemos entonces que:
        \begin{equation*}
            r\zeta^{j(\sigma\tau)} = (\sigma\tau)(r) = \sigma(\tau(r)) = \sigma(r\zeta^{j(\tau)}) = \zeta^{j(\tau)} \sigma(r) = \zeta^{j(\tau)}r\zeta^{j(\sigma)} = r\zeta^{j(\sigma)+j(\tau)}
        \end{equation*}
        de donde $j(\sigma\tau) = j(\sigma)+j(\tau)$, por lo que $j$ es un homomorfismo de grupos. Vemos además que $j$ es inyectivo, pues si $\sigma \in \Aut_{F(\zeta)}(K)$ con $j(\sigma) = 0$ tenemos entonces que $\sigma(r) = r\zeta^0 = r$, por lo que $\sigma = id$.\\

        \noindent
        En definitiva, tenemos que $\Aut_{F(\zeta)}(K)$ es isomorfo a su imagen por $j$, por lo que es isomorfo a un subgrupo de $\mathbb{Z}_n$, que ha de ser cíclico como subgrupo de un grupo cíclico. El Teorema de Lagrange nos dice que el orden debe ser un divisor de $n$.
    \end{proof}
\end{teo}

\begin{coro}
    Sea $x^n-a\in F(\zeta)[x]$, es irreducible si y solo si $[K:F(\zeta)] = n$, donde $K$ es el cuerpo de descomposición de $x^n-a$.
    \begin{proof}
        Si $r\in K$ es una raíz de $x^n-a$, tenemos que $K = F(\zeta,r)$ a partir del Teorema anterior. Por doble implicación:
        \begin{description}
            \item[$\Longrightarrow )$] Si $x^n-a\in F(\zeta)[x]$ es irreducible tenemos entonces que $\Irr(r,F(\zeta)) = x^n-a$, por lo que $[K:F(\zeta)]= n$.
            \item[$\Longleftarrow )$] Si $[K:F(\zeta)] = n$, vemos que $x^n-a\in F(\zeta)[x]$ es un polinomio mónico de grado $n$ del que $r$ es raíz, por lo que tiene que ser $\Irr(r,F(\zeta)) = x^n-a$, de donde $x^n-a$ es irreducible sobre $F(\zeta)$.
        \end{description}
    \end{proof}
\end{coro}

\begin{definicion}[Extensión cíclica]
    Una extensión $F\leq K$ se dice cíclica si es de Galois y su grupo de Galois $\Aut_F(K)$ es cíclico.
\end{definicion}

\begin{ejemplo}
    Como ejemplos de extensiones cíclicas que ya conocemos:
    \begin{itemize}
        \item Toda extensión de cuerpos finitos es cíclica, ya que si $F\leq K$ es una extensión con $K$ finito entonces $F\leq K$ es de Galois y existen $p$ primo y $n\geq 1$ de forma que:
            \begin{equation*}
                \bb{F}_p \leq F \leq K = \bb{F}_{p^n}
            \end{equation*}
            habíamos visto ya que $\Aut(K)$ es cíclico de orden $n$, por lo que $\Aut_F(K)$ será también un grupo cíclico, como subgrupo de un grupo cíclico.
        \item Si $F$ contiene una raíz $n-$ésima primitiva de la unidad y $x^n-a\in F[x]$ es separable, si tomamos $K$ su cuerpo de descomposición tenemos que $F\leq K$ es cíclica; ya que $F\leq K$ es de Galois al ser un cuerpo de descomposición de un polinomio separable y en el Teorema anterior hemos visto que el grupo de Galois de la extensión $F=F(\zeta) \leq K$  es cíclico.
    \end{itemize}
\end{ejemplo}

\noindent
Nuestro siguiente objetivo es ver cómo las extensiones cíclicas son, bajo ciertas hipótesis, cuerpos de descomposición de polinomios de la forma $x^n-a$. Para ello, primero será necesario ver un Lema con un resultado muy sorprendente:

\begin{lema}[de independencia de Dedekind]
    Sean $\sigma_1,\ldots,\sigma_n:F\to E$ homomorfismos de cuerpos distintos, tenemos entonces que $\sigma_1, \ldots, \sigma_n$  son linealmente independientes. Es decir, si $\lm_1, \ldots, \lm_n\in E$ verifican que:
    \begin{equation*}
        \lm_1 \sigma_1(x) + \ldots + \lm_n \sigma_n(x) = 0 \quad \forall x\in F \qquad \Longrightarrow \qquad \lm_1 = \ldots = \lm_n = 0
    \end{equation*}
    \begin{proof} 
        Para $n=1$ es cierto. Supuesto que $n>1$, razonamos por reducción al absurdo. Para ello, tenemos que existe al menos una lista $\lm_1,\ldots,\lm_n\in E$ de elementos no todos nulos de forma que:
        \begin{equation}\label{eq:dedekind}
            \lm_1\sigma_1(x) + \ldots + \lm_n\sigma_n(x) = 0 \qquad \forall x\in F
        \end{equation}
        de entre todas aquellas listas que verifican esta afirmación tomamos aquella con la mínima cantidad de elementos no nulos, $m>0$. Vemos que $m\geq 2$, pues si $m=1$ llegamos a que la lista no contenía elementos no nulos. Podemos suponer sin pérdida de generalidad que los $m$ elementos no nulos son los $m$ primeros. Como los homomorfismos son todos distintos entre sí, ha de existir $y\in F$ de forma que $\sigma_1(y)-\sigma_m(y)\neq 0$, y por otra parte tenemos que:
        \begin{equation*}
            \lm_1\sigma_1(yx) + \ldots + \lm_n\sigma_n(yx) = 0 \qquad \forall x\in F
        \end{equation*}
        Restándole a esta última igualdad la igualdad~\eqref{eq:dedekind} multiplicada por $\sigma_m(y)$, obtenemos que:
        \begin{equation*}
            \lm_1(\sigma_1(y) - \sigma_m(y))\sigma_1(x) + \ldots + \lm_{m-1}(\sigma_{m-1}(y)-\sigma_m(y))\sigma_{m-1}(x) = 0 \qquad \forall x\in F
        \end{equation*}
        Por lo que obtenemos una lista de elementos de $E$:
        \begin{equation*}
            \lm_1(\sigma_1(y)-\sigma_m(y)),\ldots,\lm_{m-1}(\sigma_{m-1}(y)-\sigma_m(y)),0,\ldots,0
        \end{equation*}
        que verifican la condición~\eqref{eq:dedekind} y con $m-1$ elementos no nulos, hemos llegado a una contradicción, pues la lista $\lm_1,\ldots,\lm_n$ verificaba~\eqref{eq:dedekind} y era la que tenía una menor cantidad de elementos no nulos.
    \end{proof}
\end{lema}

\begin{teo}\label{teo:ext_ciclica}
    Sea $F\leq K$ extensión cíclica tal que $n = [K:F]$ no es múltiplo de $\car(F)$. Si $F$ contiene una raíz $n-$ésima primitiva de la unidad, entonces $K$ es cuerpo de descomposición de un polinomio irreducible de la forma $x^n-a\in F[x]$.

    \noindent
    Además, si $\alpha$ es una raíz de $x^n-a$ entonces $K = F(\alpha)$.
    \begin{proof}
        Suponemos que $\zeta$ es una raíz $n-$ésima primitiva de la unidad con $\zeta \in F$. Como $F\leq K$ es cíclica, el grupo de Galois de la extensión debe ser cíclico de grado $n$, por lo que tendrá un generador $\sigma\in \Aut_F(K)$ de orden $n$. El Lema de independencia de Dedekind para los escalares $1,\zeta,\ldots,\zeta^{n-1}$ nos dice que ha de existir $r\in K$ de forma que:
        \begin{equation*}
            \beta := r + \zeta\sigma(r) + \ldots + \zeta^{n-1}\sigma^{n-1}(r) \neq 0
        \end{equation*}
        Tendremos entonces que:
        \begin{align*}
            \zeta\sigma(\beta) &= \zeta \sigma(r) + \zeta^2 \sigma^2(r) + \ldots + \zeta^{n-1} \sigma^{n-1}(r) +  \zeta^n \sigma^n(r) \\
            &= \zeta \sigma(r) + \zeta^2 \sigma^2(r) + \ldots + \zeta^{n-1} \sigma^{n-1}(r) +  r = \beta
        \end{align*}
        Por lo que:
        \begin{equation*}
            \beta^n = \zeta^n \sigma(\beta)^n = \sigma(\beta)^n = \sigma(\beta^n)
        \end{equation*}
        como $\sigma$ genera todo $\Aut_F(K)$, tenemos que $a:=\beta^n\in K^{\langle \sigma \rangle } = K^{\Aut_F(K)} = F$. Como $\zeta$ es una raíz $n-$ésima de la unidad, tenemos que:
        \begin{equation*}
            \beta, \zeta\beta, \ldots, \zeta^{n-1}\beta
        \end{equation*}
        son todas raíces distintas de $x^n-a$, y tenemos $n$, de donde:
        \begin{equation*}
            x^n-a = (x-\beta)(x-\zeta\beta) \ldots (x-\zeta^{n-1}\beta)
        \end{equation*}
        Y como $\zeta\in F$, tenemos que $F(\beta)$ es cuerpo de descomposición de $x^n-a\in F[x]$. Como $\beta$ es raíz de $x^n-a$, tenemos que $[F(\beta):F] \leq n$, y para obtener la igualdad observamos que:
        \begin{equation*}
            \sigma^k(\beta) = \zeta^{-k}\beta \qquad \forall k\in \mathbb{Z}_n
        \end{equation*}
        Por lo que la acción de $\Aut_F(K)$ sobre las raíces de $x^n-a$ es transitiva, por lo que $x^n-a$ es irreducible, luego $\Irr(\beta,F) = x^n-a$, de donde $[F(\beta):F] = n$. Tenemos en definitiva que $F(\beta) = K$.\\

        \noindent
        La última afirmación es clara, pues si $\alpha$ es raíz de $x^n-a$, entonces $\alpha = \zeta^k\beta$ para $k \in \mathbb{Z}_n$, obteniendo que $K = F(\alpha)$.
    \end{proof}
\end{teo}

\begin{prop}
    El grupo de Galois de un polinomio separable de la forma $x^n-a\in F[x]$ es resoluble.
    \begin{proof}
        Sea $K$ el cuerpo de descommposición de $x^n-a$, sabemos que existe $\zeta\in K$ una raíz $n-$ésima primitiva de la unidad, por lo que tenemos la torre de cuerpos:
        \begin{equation*}
            F\leq F(\zeta)\leq K
        \end{equation*}
        con $F\leq K$ de Galois por ser cuerpo de descommposición de un polinomio separable y $F\leq F(\zeta)$ de Galois por ser una extensión ciclotómica. Por ser $F\leq F(\zeta)$ de Galois, tenemos que:
        \begin{equation*}
            \Aut_{F(\zeta)}(K) \lhd \Aut_F(K)
        \end{equation*}
        y que además:
        \begin{equation*}
            \Aut_F(F(\zeta)) \cong \frac{\Aut_F(K)}{\Aut_{F(\zeta)}(K)}
        \end{equation*}
        Tenemos así la serie normal:
        \begin{equation*}
            \{id_F\} \lhd \Aut_{F(\zeta)}(K) \lhd \Aut_F(K)
        \end{equation*}
        con el primer factor cíclico, puesto que $\Aut_{F(\zeta)}(K)$ es cíclico y con el segundo abeliano, por ser isomorfo a $\Aut_F(F(\zeta))$, que es abeliano por ser isomorfo a un subgrupo de las unidades de $\mathbb{Z}_n$.
    \end{proof}
\end{prop}

\begin{ejemplo}  % // TODO: REvisar este ejemplo
    Si consideramos $x^8-3\in \mathbb{Q}[x]$, si $\zeta$ es una raíz octava primitiva de la unidad y consideramos $K$ el cuerpo de descomposición de $x^8-3\in \mathbb{Q}(\zeta)[x]$, tenemos que el grupo de Galois de la extensión $\mathbb{Q}(\zeta)\leq K$ es cíclico. Sabemos además por el Teorema~\ref{teo:ext_ciclicas} que $K = \mathbb{Q}\left(\sqrt[8]{3}, \zeta\right)$. Podemos calcular $\zeta$ como una raíz cuadrada de $i$, que es una raíz cuarta de la unidad. Tomamos una de ellas:
    \begin{equation*}
        \zeta = e^{i\frac{\pi}{4}} = \frac{1}{\sqrt{2}} + i\frac{1}{\sqrt{2}}
    \end{equation*}
    Como el octavo polinomio ciclotómico tiene grado $\varphi(8) = 4$, tenemos que la extensión $\mathbb{Q}\leq\mathbb{Q}(\zeta)$ es de grado 4. Si consideramos ahora $\zeta+\overline{\zeta}\in \mathbb{Q}(\zeta)$:
    \begin{equation*}
        \zeta + \overline{\zeta} = 2\text{Re}(\zeta) = \sqrt{2} \in  \mathbb{Q}(\zeta)
    \end{equation*}
    De donde $\mathbb{Q}\left(\sqrt{2}\right)\leq \mathbb{Q}(\zeta)$, por lo que $\mathbb{Q}\left(i,\sqrt{2}\right) = \mathbb{Q}(\zeta)$, al tener claramente $\mathbb{Q}\left(i,\sqrt{2}\right)\leq \mathbb{Q}(\zeta)$ y $\left[\mathbb{Q}\left(i,\sqrt{2}\right):\mathbb{Q}\right] = 4$.\\

    \noindent
    Calculamos el grado de la extensión $[K:\mathbb{Q}]$, para saber el cardinal de $\Aut_F(K)$. Por el Lema de la Torre:
    \begin{equation*}
        [K:\mathbb{Q}] = \left[\mathbb{Q}\left(\sqrt[8]{3},\sqrt{2},i\right):\mathbb{Q}\left(\sqrt[8]{3},\sqrt{2}\right)\right] \left[\mathbb{Q}\left(\sqrt{2},\sqrt[8]{3}\right):\mathbb{Q}\left(\sqrt[8]{3}\right)\right] \left[\mathbb{Q}\left(\sqrt[8]{3}\right):\mathbb{Q}\right]
    \end{equation*}
    Donde el último grado es $8$ por ser 3 primo y aplicar Eisenstein. La primera es 2 por ser $i\notin \mathbb{R}$. La segunda es 2 si y solo si $\sqrt{2}\notin \mathbb{Q}\left(\sqrt[8]{3}\right)$. 

    \noindent
    Consideramos:
    \begin{equation*}
        \mathbb{Q}\stackrel{2}{\leq } \mathbb{Q}\left(\sqrt{3}\right) \stackrel{\leq 2}{\leq} \mathbb{Q}\left(\sqrt[4]{3}\right) \stackrel{\leq 2}{\leq} \mathbb{Q}\left(\sqrt[8]{3}\right)
    \end{equation*}
    Y como $\mathbb{Q}\leq \mathbb{Q}\left(\sqrt[8]{3}\right)$ es de grado 8, tienen que ser todas estas de grado 2. Veamos:
    \begin{itemize}
        \item $\sqrt{2}\notin \mathbb{Q}\left(\sqrt{3}\right)$, puesto que si esto fuera así, $\sqrt{2} = a + b\sqrt{3}$, elevamos al cuadrado y sale que $\sqrt{3}$ es racional, que no es posible porque $x^2-3$ es irreducible.
        \item $\sqrt{2}\notin \mathbb{Q}\left(\sqrt[4]{3}\right)$, usando que una $\mathbb{Q}\left(\sqrt{3}\right)-$base es $\left\{1,\sqrt[4]{3}\right\}$, si $\sqrt{2}\in \mathbb{Q}\left(\sqrt[4]{3}\right)$ tendríamos entonces que $\exists a,b\in \mathbb{Q}\left(\sqrt{3}\right)$ de manera que:
            \begin{equation*}
                \sqrt{2} = a+b\sqrt[4]{3}
            \end{equation*}
            Elevando al cuadrado:
            \begin{equation*}
                2 = a^2 + 2ab\sqrt[4]{3} + b^2\sqrt{3} \Longrightarrow \left\{\begin{array}{l}
                    2 = a^2 + b^2 \sqrt{3} \\
                    0 = 2ab
                \end{array}\right.
            \end{equation*}
            igualando coordenadas a coordenadas, por lo que:
            \begin{itemize}
                \item Si $b= 0$, entonces $2 = a^2$, de donde $\sqrt{2} = a \in \mathbb{Q}(\sqrt{3})$, pero ya habíamos visto que este caso no puede ser.
                \item Si $a = 0$, entonces $2=b^2 \sqrt{3}$ para $b = x+y\sqrt{3}$ con $x,y\in \mathbb{Q}$, luego:
                    \begin{equation*}
                        2 = (x^2 + 2xy\sqrt{3} + 3y^2)\sqrt{3}
                    \end{equation*}
                    Y tenemos que $\{1,\sqrt{3}\}$ es una $\mathbb{Q}$-base de $\mathbb{Q}\left(\sqrt{3}\right)$, por lo que igualdando coordenadas:
                    \begin{equation*}
                        0 = x^2 + 3y^2  \Longrightarrow x = 0 = y
                    \end{equation*}
                    Luego $b= 0 \Longrightarrow 2 = 0$ \underline{contradicción}, que viene de suponer que teníamos $\sqrt{2}\in \mathbb{Q}\left(\sqrt[4]{3}\right)$.
            \end{itemize}
        \item Intentamos ver ahora que $\sqrt{2}\notin \mathbb{Q}\left(\sqrt[8]{3}\right)$. Por reducción al absurdo, si $\sqrt{2}\in \mathbb{Q}\left(\sqrt[8]{3}\right)$, tenemos que $\left\{1,\sqrt[8]{3}\right\}$  es una $\mathbb{Q}\left(\sqrt[4]{3}\right)-$base, por lo que existirían $c,d\in \mathbb{Q}\left(\sqrt[4]{3}\right)$ de forma que:
            \begin{equation*}
                \sqrt{2} = c+d\sqrt[8]{3}
            \end{equation*}
            con $d\neq 0$, por el apartado anterior. Elevando al cuadrado:
            \begin{equation*}
                2 = c^2 + 2cd\sqrt[8]{3} + d^2\sqrt[4]{3}
            \end{equation*}
            Igualando coordenadas en la base obtenemos que ($d\neq 0$) $c=0$, por lo que:
            \begin{equation*}
                2 = d^4\sqrt[4]{3} 
            \end{equation*}
            Escribimos las coordenadas de $d$:
            \begin{equation*}
                d = z + t\sqrt[4]{3} \qquad z,t\in \mathbb{Q}\left(\sqrt{3}\right)
            \end{equation*}
            De donde elevando al cuadrado:
            \begin{equation*}
                2 = (z^2 + 2zt\sqrt[4]{3} + t^2 \sqrt{3})\sqrt[4]{3}
            \end{equation*}
            Igualando coordenadas:
            \begin{equation*}
                0 = z^2 + t^2\sqrt{3} \Longrightarrow z = 0 = t
            \end{equation*}
            lo que nos lleva a una \underline{contradicción}.
    \end{itemize} % // TODO: En los apuntes esto está hecho de otra forma
    En definitiva, $[K:\mathbb{Q}] = 32$, de donde el grupo ciclico es de orden 8.
\end{ejemplo}

\begin{ejercicio}
    Supongamos que el polinomio $x^n-a\in F[x]$ es separable, con $a\neq 0$, y sea $K$ su cuerpo de descomposición. Denotemos por $\zeta\in K$ una raíz primitiva $n-$ésima de la unidad, y $\sqrt[n]{a}\in K$ una raíz de $f$. Dado $\sigma\in \Aut_F(K)$, denotemos por $j(\sigma),k(\sigma)\in \mathbb{Z}_n$ determinados por las condiciones $\sigma(\sqrt[n]{a}) = \zeta^{j(\sigma)}\sqrt[n]{a}$, $\sigma(\zeta) = \zeta^{k(\sigma)}$. Demostrar que la aplicación 
    \begin{equation*}
        \Aut_F(K) \to GL_2(\mathbb{Z}_n), \qquad \sigma\longmapsto \left(\begin{array}{cc}
            1 & 0 \\
            j(\sigma) &k(\sigma)  
        \end{array}\right)
    \end{equation*}
    es un homomorfismo inyectivo de grupos. Deducir que $|\Aut_F(K)|$ es un divisor de $n\varphi(n)$. En el caso $F=\mathbb{Q}$, deducir que $|\Aut_F(K)| = n\varphi(n)$ si, y solo si, $x^n-a\in \mathbb{Q}(\zeta)[x]$ es irreducible.
\end{ejercicio}

\begin{ejercicio}
    Sea $K = \mathbb{Q}\left(\sqrt[4]{5},i\right)$.
    \begin{enumerate}
        \item Rzonar que $K$ es una extensión de Galois de $\mathbb{Q}$ y calcular el cardinal de su grupo de Galois.
        \item Describir los elementos del grupo $\Aut(K)$.
        \item Calcular todos los subcuerpos de $K$ que tienen grado 4 sobre $\mathbb{Q}$.
        \item Calcular todos los subcuerpos de $K$.
    \end{enumerate}
\end{ejercicio}

\section{Ecuaciones resolubles por radicales}
\noindent
A lo largo de esta sección trabajaremos solo con cuerpos de característica 0, aunque es posible desarrollar la teoría para cuerpos de cualquier característica, pero añadiendo muchas hipótesis extra. Preferimos simplificar los enunciados. Así mismo, las definiciones que haremos en este apartado dependen mucho del autor.\\

\noindent
La siguiente definición generaliza el concepto de extensión por raíces cuadradas:
\begin{definicion}
    Una extensión de cuerpos $F\leq E$ es una \underline{extensión por radicales} si existe una torre de cuerpos
    \begin{equation*}
        F = E_0 \leq E_1 \leq \ldots \leq E_t = E
    \end{equation*}
    tal que $E_j=E_{j-1}(\alpha_j)$, con $\alpha_j^{n_j}\in E_{j-1}$, para $j \in \{1,\ldots,t\}$.
\end{definicion}

\begin{definicion}
    Un polinomio $f\in F[x]$ se dice \underline{resoluble por radicales} si existe una extensión por radicales $F\leq E$ que contiene al cuerpo de descomposición de $f$.
\end{definicion}

\begin{definicion}
    Una extensión $F\leq K$ es \underline{radical} si $K$ es cuerpo de descomposición de un polinomio separable $x^n-a\in F[x]$.
\end{definicion}

\noindent
Bajo estas condiciones, toda extensión radical es cíclica, y toda cíclica da una radical. Como estamos en $\car(F) = 0$, la hipótesis de que $x^n-a$ sea separable se traduce en que $a\neq 0$.

\begin{definicion}
    Diremos que una extensión $F\leq K$ es \underline{radical iterada} si $F\leq K$ es de Galois y hay una torre de cuerpos
    \begin{equation*}
        F = K_0 \leq K_1 \leq \ldots \leq K_t = K
    \end{equation*}
    de forma que cada $K_{i-1}\leq K_i$ es radical, para $i \in \{1,\ldots,t\}$.
\end{definicion}

\begin{prop}
    Si $F\leq E$ es de Galois y $E\leq E(\alpha)$ para $\alpha$ raíz de $x^n-a\in E[x]$ con $a\neq 0$, entonces existe una extensión radical iterada $E\leq K$ con $E(\alpha)\leq K$ y $F\leq K$ de Galois.
    \begin{proof}
        Consideramos:
        \begin{equation*}
            f = \prod_{\sigma\in \Aut_F(E)} (x^n-\sigma(a)) \in E[x]
        \end{equation*}
        Como $f^\tau = f\quad \forall \tau\in \Aut_F(E)$, tenemos que en realidad $f\in F[x]$. Como $F\leq E$ es de Galois, $E$ es cuerpo de descomposición de cierto polinomio separable $g\in F[x]$. Sea $K$ el cuerpo de descomposición de $fg\in F[x]$, como estamos en característica cero, $F\leq K$ es de Galois. Además, vemos que:
        \begin{itemize}
            \item Como $\alpha$ es raíz de $x^n-a$ y este es un factor de $f$ para $\sigma = id$, tenemos que $\alpha\in K$.
            \item Como $K$ es cuerpo de descomposición de $fg$ y $E$ es cuerpo de descomposición de $g$, tiene que ser $E\leq K$.
        \end{itemize}
        de aquí deducimos que $E(\alpha)\leq K$. Nos falta ver que $E\leq K$ es radical iterada, vemos que $E\leq K$ es de Galois por ser $F\leq K$ de Galois. \\

        \noindent
        Como $x^n-a$ es un factor de $f$, $K$ ha de contener un cuerpo de descomposición de $x^n-a$, por lo que por el Torema~\ref{teo:ext_ciclicas} podemos encontrar $\zeta\in K$ una raíz $n-$ésima primitiva de la unidad. Si enumeramos los elementos de $\Aut_F(E)$:
        \begin{equation*}
            \Aut_F(E) = \{\sigma_1, \ldots, \sigma_s\}
        \end{equation*}
        con $\sigma_1 = id_E$. Tomando $K_{-1} = E$ y $K_0 = E(\zeta)$, para cada $i \in \{1,\ldots,s\}$ tomamos $K_i = K_{i-1}(\alpha_i)$, con $\alpha_i$ raíz de $x^{n}-\sigma_i(a)$.\\

        \noindent
        De esta forma, cada $K_{i-1}\leq K_i$ es radical, para $i \in \{0,\ldots,s\}$.
    \end{proof}
\end{prop}

\noindent
En cierto momento, usaremos la sigiuente observación:

\begin{observacion}
    \noindent
    Sea $F$ un cuerpo, dados $\sigma_1:F\to L_1$, $\sigma_2:F\to L_2$ dos homomorfismos de cuerpos tales que las extensiones $\sigma_i(F)\leq L_i$ son finitas para $i \in \{1,2\}$. Para cada $i \in \{1,2\}$ vamos a construir un polinomio $f_i \in F[x]$ tal que exite un homomorfismo de cuerpos $\tau_i:L_i\to K_i$ de manera que $\tau_i\sigma_i:F\to K_i$ es cuerpo de descomposición de $f_i$. Para ello, como $\sigma_i(F)\leq L_i$ es finita, tenemos que $L_i = \sigma_i(F)(\alpha_1,...,\alpha_t)$ para $\alpha_1,\ldots,\alpha_t$ algebraicos sobre $\sigma_i(F)$. Tomamos $g_j = \Irr(\alpha_j,\sigma_i(F))$ para $j \in \{1,\ldots,t\}$, y definimos $f_i \in F[x]$ de forma que $f_i^{\sigma_i} = g_1\ldots g_t$.\\

    \noindent
    Consideraremos también un cuerpo de descomposición $\tau:F\to E$ de $f_1f_2$. La Tercera Proposición de extensión nos permite encontrar homomorfismos $\eta_i:F\to K_i$ en $Ex(\tau,\sigma_i,\tau_i)$, para $i \in \{1,2\}$. Obtenemos así el diagrama conmutativo:
    \begin{figure}[H]
        \centering
        \shorthandoff{""}
        \begin{tikzcd}
            F \arrow[r, "\sigma_1"] \arrow[d, "\sigma_2"'] \arrow[rrdd, "\tau"'] & L_1 \arrow[r, "\tau_1"] & K_1 \arrow[dd, "\eta_1"'] \\
            L_2 \arrow[d, "\tau_2"']                                             &                         &                           \\
            K_2 \arrow[rr, "\eta_2"]                                             &                         & E                        
        \end{tikzcd}
        \shorthandon{""}
    \end{figure}
    \noindent
    de manera que (cada triángulo conmuta):
    \begin{equation*}
        \tau = \eta_1\tau_1\sigma_1 = \eta_2\tau_2\sigma_2
    \end{equation*}
    De esta forma, como cada homomorfismo de cuerpos es inyectivo:
    \begin{equation*}
        F \cong \tau(F) \leq \eta_i \tau_i(L_i) \leq E \qquad \forall i \in \{1,2\}
    \end{equation*}
\end{observacion}

\begin{prop}
    Sea $F\leq E$ una extensión por radicales, entonces existe una extensión radical iterada $F\leq K$ tal que $E\leq K$.
    \begin{proof}
        Suponemos pues que tenemos una torre:
        \begin{equation*}
            F= E_0\leq E_1 \leq \ldots \leq E_t = E
        \end{equation*}
        tal que $E_j = E_{j-1}(\alpha_j)$ con $\alpha_j$ raíz de $x^{n_j}-a_j\in E_{j-1}[x]$, para cada $j \in \{1,\ldots,t\}$. Razonamos por inducción sobre $t\geq 0$:
        \begin{itemize}
            \item \underline{Para $t=0$}, tomamos $F=E=K$.
            \item \underline{Para $t>0$}, por hipótesis de inducción tenemos que existe una extensión radical iterada 
                \begin{equation*}
                    F=K_0\leq K_1\leq \ldots \leq K_r
                \end{equation*}
                tal que $E_{t-1}\leq K_r$. Tomamos una $F-$extensión común de $K_r$ y $E_t$ (como hemos hecho en la observación anterior), dentro de la cual estará $K_r(\alpha_t)$, pues $\alpha_t \in E_t$. Como tenemos $E_{t-1}\leq K_r$, será $E_{t-1}(\alpha_t) = E_t\leq K_r(\alpha_t)$.\\

                \noindent
                Tenemos que $F\leq K_r$ es de Galois por ser $F\leq K_r$ radical iterada, así como que $K\leq K(\alpha_t)$ con $\alpha_t$ raíz de $x^{n_t}-a_t\in E_{t-1}[x]$ (y $E_{t-1}\leq K_r$), por lo que podemos aplicar la Proposición anterior, obteniendo una extensión radical iterada $K_r\leq K$ tal que $K_r(\alpha_t)\leq K$ y $F\leq K$ de Galois.\\

                \noindent
                Por tanto, tenemos una torre de cuerpos
                \begin{equation*}
                    K_r\leq K_{r+1}\leq \ldots \leq K_s = K
                \end{equation*}
                con $K_{i-1}\leq K_i$ radical para cada $i \in \{r+1,\ldots,s\}$, de donde:
                \begin{equation*}
                    F = K_0 \leq K_1 \leq \ldots \leq K_r \leq K_{r+1}\leq \ldots \leq K_s = K
                \end{equation*}
                con cada $K_{k-1}\leq K_k$ radical para cada $k \in \{1, \ldots, s\}$ y $F\leq K$ de Galois. De aquí tenemos que $F\leq K$ es radical iterada y que $E\leq K$, puesto que $E=E_t \leq K_r(\alpha_t)\leq K$.
        \end{itemize}
    \end{proof}
\end{prop}

\begin{lema}
    Toda extensión radical iterada tiene grupo de Galois resoluble.
    \begin{proof} 
        Veamos primero que si $F\leq K$ es radical entonces $\Aut_F(K)$ es resoluble. Si $F\leq K$ es radical entonces $K$ es cuerpo de descomposición de $x^n-a\in F[x]$ separable, por lo que contiene por el Teorema~\ref{teo:ext_ciclicas} una raíz $n-$ésima primitiva de la unidad $\zeta\in K$. Tenemos por tanto:
        \begin{equation*}
            F\leq F(\zeta) \leq K
        \end{equation*}
        Con $F(\zeta)\leq K$ de Galois por ser $F\leq K$ de Galois (ya que $F\leq K$ es radical) y tenemos además que $F\leq F(\zeta)$ de Galois por ser la $n-$ésima extensión ciclotómica de $F$ (que siempre son de Galois). Usando ahora la conexión de Galois y el Teorema~\ref{teo:normal}, tenemos entonces que:
        \begin{equation*}
            \{id_K\} \lhd \Aut_{F(\zeta)}(K) \lhd \Aut_F(K)
        \end{equation*}

        y además:
        \begin{equation*}
            \frac{\Aut_F(K)}{\Aut_{F(\zeta)}(K)} \cong \Aut_F(F(\zeta))
        \end{equation*}
        Como $\Aut_F(F(\zeta))$ es el grupo de Galois de una extensión ciclotómica, tiene que ser abeliano (ya que es isomorfo a un subgrupo de $\cc{U}(\mathbb{Z}_n)$, y este grupo es abeliano), por lo que el factor $\Aut_{F(\zeta)}(K)\lhd \Aut_F(K)$ da un cociente abeliano. Ahora, tenemos que $\Aut_{F(\zeta)}(K)$ es cíclico, luego el factor $\{id_K\} \lhd \Aut_{F(\zeta)}(K)$ también es abeliano. En definitiva, tenemos que $\Aut_F(K)$ es resoluble, ya que admite una serie normal con factores abelianos.\\

        \noindent
        Si ahora $F\leq K$ es una extensión radical iterada:
        \begin{equation*}
            F=K_0\leq K_1 \leq \ldots \leq K_t = K
        \end{equation*}
        tendremos entonces que $F\leq K$ es de Galois por definición, así como que cada extensión intermedia es de Galois, por ser cuerpo de descomposición de un polinomio separable. Usando al igual que antes la conexión de Galois y el Torema~\ref{teo:normal}, obtenemos una serie normal de grupos:
        \begin{equation*}
            \Aut_F(K) = \Aut_{K_0}(K) \rhd \Aut_{K_1}(K) \rhd \ldots \rhd \Aut_{K_{t-1}}(K) \rhd \{id_K\}
        \end{equation*}
        con factores:
        \begin{equation*}
            \frac{\Aut_{K_{i-1}}(K)}{\Aut_{K_i}(K)} \cong \Aut_{K_{i-1}}(K_i) \qquad \forall i \in \{1,\ldots,t-1\}
        \end{equation*}
        y estos factores son resolubles, ya que la extensión $K_{i-1}\leq K_i$ es radical, y hemos visto ya que estas extensiones tienen grupo de Galois resoluble. En definitiva, obtenemos que $\Aut_F(K)$ es resoluble, ya que: 
        \begin{description}
            \item [Opción 1, argumento recursivo.] Vemos que:
                \begin{itemize}
                    \item Como $K_{t-1}\leq K_t$ es radical, tenemos que $\Aut_{K_{t-1}}(K)$ es resoluble.
                    \item Supuesto que $\Aut_{K_{t-s}}(K)$ es resoluble para $s \in \{1,\ldots,t-1\}$, tenemos que $\Aut_{K_{t-s}}(K)\lhd \Aut_{K_{t-(s+1)}}(K)$ con:
                        \begin{equation*}
                            \Aut_{K_{t-s}}(K), \qquad \frac{\Aut_{K_{t-(s+1)}}(K)}{\Aut_{K_{t-s}}(K)}
                        \end{equation*}
                        resolubles, por lo que $\Aut_{K_{t-(s+1)}}(K)$ es resoluble.
                \end{itemize}
                En definitiva, obtenemos que $\Aut_{K_{0}}(K) = \Aut_F(K)$ es resoluble.
            \item [Opción 2, refinando la serie.] Como cada factor es resoluble, podemos encontrar entre cada dos eslabones de la cadena una serie normal con factores abelianos, y si repetimos este proceso en cada eslabón, obtenemos al final una serie normal con factores abelianos, por lo que $\Aut_F(K)$ es resoluble.
        \end{description}
    \end{proof}
\end{lema}

\begin{ejercicio}
    Sea $f\in F[x]$ y $L$ cuerpo de descomposición de $f\in F[x]$. Demostrar que para cualquier extensión $F\leq E$, si $K$ es un cuerpo de descomposición de $f$ sobre $E$, entonces $\Aut_E(K)$ es isomorfo a un subgrupo de $\Aut_F(L)$.\\

    \noindent
    Es claro que $L\leq K$. Como $L = F(\alpha_1,\ldots,\alpha_n)$ siendo $\alpha_1,\ldots,\alpha_n$ las raíces de $f$, tendremos también que $K = E(\alpha_1,\ldots,\alpha_n)$. Si tomamos $\sigma \in \Aut_E(K)$, vemos que tenemos un automorfismo $\sigma:K\to K$ que permuta las raíces de $f$ y que es $E-$lineal. Como $F\leq E$ es claro que también es $F-$lineal, y como $L\leq K$ podemos restringir $\sigma$ a $L$, obteniendo $\sigma\big|_L:L\to K$. Como $L = F(\alpha_1,\ldots,\alpha_n)$ y $\sigma\big|_L$ deja fijos los elementos de $F$ y permuta las raíces de $f$, vemos que $Im \sigma\big|_L\leq L$. Si consideramos la aplicación $\Aut_E(K)\to \Aut_F(L)$ que restringe cada $\sigma$ a $L$ obtenemos lo buscado.
\end{ejercicio}

\begin{teo}[Gran Teorema de Galois]
    Sea $f\in F[x]$: 
    \begin{center}
        $f$ es resoluble por radicales $\Longleftrightarrow $ el grupo de Galois de $f$ es resoluble
    \end{center}
    \begin{proof} 
        Sea $L$ el cuerpo de descomposición de $f$:
        \begin{description}
            \item [$\Longrightarrow )$] Tenemos una torre $F\leq L\leq E$ tal que $F\leq E$ es una extensión por radicales. Por las Proposiciones anteriores, se ha visto que existe una extensión radical iterada $F\leq K$ tal que $E\leq K$, por lo que $F\leq K$ es de Galois, y como $F\leq L$ también es de Galois (es cuerpo de descomposición en característica cero), aplicando el Torema~\ref{teo:normal} tenemos que $\Aut_L(K)\lhd \Aut_F(K)$. Sabemos además que:
                \begin{equation*}
                    \Aut_F(L) \cong \frac{\Aut_F(K)}{\Aut_L(K)}
                \end{equation*}
                Y en el Lema anterior vimos que $\Aut_F(K)$ es resoluble, por lo que $\Aut_F(L)$ es resoluble, que es el grupo de Galois de $f$.
            \item [$\Longleftarrow )$] Supuesto que $\Aut_F(L)$ es resoluble, tomamos $n=[L:F]$ y consideramos $K=L(\zeta)$ con $\zeta$ una raíz $n-$ésima primitiva de la unidad. El Ejercicio anterior nos dice que $\Aut_{F(\zeta)}(K)$ es isomorfo a un subgrupo de $\Aut_F(L)$. Como $\Aut_F(L)$ es resoluble, tendremos que $\Aut_{F(\zeta)}(K)$ es resoluble y sabemos que $|\Aut_{F(\zeta)}(K)|$ divide a $n$ (por el Teorema de Lagrange). Como $\Aut_{F(\zeta)}(K)$ es resoluble, podemos encontrar una serie de composición con factores primos:
                \begin{equation*}
                    \Aut_{F(\zeta)}(K) = G_0 \rhd G_1 \rhd \ldots \rhd G_{t-1} \rhd G_t = \{id_K\}
                \end{equation*}
                con $G_{i-1}/G_i$ de cardinal $p_i$ primo que divide a $|\Aut_{F(\zeta)}(K)|$, luego también a $n$. Si aplicamos la conexión de Galois a la serie de composición, obtenemos:
                \begin{equation*}
                    F(\zeta) = K^{G_0} \leq K^{G_1} \leq \ldots \leq K^{G_{i-1}} \leq K^{G_t} = K
                \end{equation*}
                Para cada $i \in \{1,\ldots,t\}$ tenemos por el Teorema~\ref{teo:normal} que el grupo de Galois de $K^{G_{i-1}}\leq K^{G_{i}}$ es isomorfo a $G_{i-1}/G_i$, luego es cíclico de orden $p_i$. Además, tenemos que $\zeta^{\nicefrac{n}{p_i}}\in F(\zeta)$ es una raíz $p_i-$ésima de la unidad, que estará contenida también en $K^{G_{i-1}}$ por ser $F(\zeta) = K^{G_0}$. Bajo estas condiciones podemos aplicar el Teorema~\ref{teo:ext_ciclica} a la extensión $K^{G_{i-1}}\leq K^{G_{i}}$, obteniendo que entonces $K^{G_{i}}$ es cuerpo de descomposición de un polinomio de la forma $x^{p_i}-a_i \in K^{G_{i-1}}[x]$, por lo que $K^{G_{i-1}}\leq K^{G_{i}}$ es radical.\\

                Como claramente $F\leq F(\zeta)$ es radical, obtenemos finalmente una cadena de extensiones radicales:
                \begin{equation*}
                    F\leq F(\zeta) = K^{G_0} \leq K^{G_1} \leq \ldots \leq K^{G_{t-1}} \leq K^{G_t} = K
                \end{equation*}
                Es decir, $F\leq K$ es una extensión por radicales, con $L\leq K = L(\zeta)$, por lo que $f$ es resoluble por radicales.
        \end{description}
    \end{proof}
\end{teo}

\subsubsection{Consecuencias}
\begin{enumerate}
    \item Si $deg f\leq 4$, entones $f$ es resoluble por radicales.

        Esto es porque el grupo de Galois de $f$ está dentro de $S_n$ con $n\leq 4$ y estos grupos son resolubles.
    \item Si $degf \geq 5$, entonces $f$ es resoluble por radicales dependiendo de su grupo de Galois.

        Veremos que $x^5-4x-1\in \mathbb{Q}[x]$ tiene grupo de Galois isomorfo a $S_5$, luego NO es resoluble por radicales.
\end{enumerate}

\section{Ecuación general de grado $n$} % // TODO: Qué coño es esta sección
\noindent
Recordamos que si $F$ es un cuerpo, podemos considerar el anillos de polinomios con coeficientes en $F$ y con $n$ indeterminadas, $F[x_1,\ldots,x_n]$.
\begin{itemize}
    \item recordamos que al alterar el orden de las indeterminadas obtenemos anillos isomorfos
    \item como $F$ es un cuerpo, $F[x_1]$ es un DFU, por lo que $F[x_1,x_2]$ también, y en una cantidad finita de pasos llegamos a que $F[x_1,\ldots,x_n]$ también es un DFU, y en particular un dominio de integridad.
\end{itemize}
Si aplicamos la construcción de cuerpo de fracciones a $F[x_1,\ldots,x_n]$, obtenemos $F(x_1,\ldots, x_n)$, el cuerpo de fracciones del dominio de integridad $F[x_1,\ldots, x_n]$:
\begin{equation*}
    F(x_1,\ldots,x_n) = \left\{\frac{f}{g} : f,g\in F[x_1,\ldots,x_n], g\neq 0\right\}
\end{equation*}
Dada una permutación $\sigma\in S_n$ y aplicando $n$ veces la Propiedad Universal del anillo de polinomios, podemos obtener un homomorfismo de anillos $F-$lineal\newline $\overline{\sigma}:F[x_1,\ldots,x_n]\to F[x_1,\ldots,x_n]$ determinado por:
\begin{align*}
    \overline{\sigma}(\alpha) &= \alpha \quad \forall \alpha\in F \\
    \overline{\sigma}(x_i) &= x_{\sigma(i)} \quad \forall i \in \{1,\ldots,n\}
\end{align*}
Que claramente es un isomorfismo, pues $\overline{\sigma^{-1}}$ es su homomorfismo inverso. Usando la Propiedad Universal de $F(x_1,\ldots,x_n)$ para obtener un automorfismo de cuerpos $\overline{\sigma}:F(x_1,\ldots,x_n)\to F(x_1,\ldots,x_n)$ dado por:
\begin{equation*}
    \overline{\sigma}\left(\frac{f(x_1,\ldots,x_n)}{g(x_1,\ldots,x_n)}\right) = \frac{f(x_{\sigma(1)}, \ldots, x_{\sigma(n)})}{g(x_{\sigma(1)}, \ldots, x_{\sigma(n)})}
\end{equation*}
Tenemos así una aplicación $S_n\to \Aut_F(F(x_1,\ldots,x_n))$ que es un homomorfismo de grupos inyectivo.

\begin{notacion}
    Ante estas condiciones, usaremos la siguiente notación a lo largo de esta sección:
    \begin{itemize}
        \item $E = F(x_1,\ldots,x_n)$.
        \item $G$ es la imagen del monomorfismo de grupos $S_n\to \Aut_F(E)$.
    \end{itemize}
\end{notacion}

\begin{definicion}
    Al cuerpo $E^G$ lo llamamos cuerpo de las funciones simétricas racionales en $x_1,\ldots,x_n$ con coeficientes en $F$.\\

    \noindent
    Tenemos que $E^G\leq E$ es de Galois, por el Lema de Artin.
\end{definicion}

% // TODO: Es el ejercicio 3.1.3.
% \begin{ejercicio}
%     Sea $f\in F[x]$ separable, irreducible y de grado primo $p$, entonces su grupo de Galois contiene un $p-$ciclo.\\

%     \noindent
%     Si $f$ es separable e irreducible, entonces $p$ divide al orden del grupo de Galois de $f$. Como es primo, entonces $G$ ha de contener un elemento de orden $p$. Descomponemos el elemento como producto de ciclos disjuntos y su orden ha de ser el minimo común múltiplo de todos los órdenes de los ciclos que aparecen en su descomposición. Si estos números tienen como mínimo común múltiplo un número primo, entonces el orden de todos los ciclos es $p$. En $S_p$, el primero que aparece ha gastado todos los símbolos, luego ha de ser un $p-$ciclo.
% \end{ejercicio}

\begin{notacion}
    Para cada $k \in \{1,\ldots,n\}$ escribimos:
    \begin{equation*}
        s_k = \sum_{1\leq i_1< \ldots < i_k\leq n} x_{i_1} \ldots x_{i_k} \in  F[x_1, \ldots, x_n]
    \end{equation*}
    Por ejemplo, si $n= 4$, tenemos entonces que:
    \begin{align*}
        S_1 &= x_1 + x_2 + x_3 + x_4 \\
        S_2 &= x_1x_2 + x_1x_3 + x_1x_4 + x_2x_3 + x_2x_4 + x_3x_4 \\
        S_3 &= x_1x_2x_3 + x_1x_2x_4 + x_1x_3x_4 + x_2x_3x_4 \\
        S_4 &= x_1x_2x_3x_4
    \end{align*}
    Llamaremos a los polinomios $s_k$ para $1\leq k\leq n$ polinomios simétricos elementales.
\end{notacion}

\begin{prop}
    $E^G = F(s_1,\ldots, s_n)$, es decir, toda función simétrica racional en $x_1,\ldots,x_n$ con coeficientes en $F$ puede expresarse exclusivamente en términos de los polinomios simétricos elementales $s_1,\ldots,s_n$.
    \begin{proof}
        Consideramos el polinomio:
        \begin{equation*}
            f = (x-x_1) \ldots (x-x_n) \in E[x]
        \end{equation*}
        Y si tomamos $\overline{\sigma}\in G$ es claro que $f^{\overline{\sigma}} = f$, luego $f\in E^G[x]$. El Ejercicio~\ref{ej:cardano-vieta} nos dice que:
        \begin{equation*}
            f = x^n-s_1x^{n-1} + \ldots + {(-1)}^{n}s_n 
        \end{equation*}
        por lo que $s_1,\ldots,s_n \in E^G$, de donde $F(s_1,\ldots,s_n) \leq E^G$.\\

        \noindent
        Observemos además que $E$ es cuerpo de descomposición de $f$ sobre $F(s_1,\ldots,s_n)$ y como $f$ es separable vemos que $F(s_1,\ldots,s_n)\leq E$ es de Galois.\\

        \noindent
        Además, $G\leq \Aut_{F(s_1, \ldots, s_n)}(E)$. Pero si $\tau:F\to E$ es un automorfismo de cuerpos $F(s_1, \ldots, s_n)-$lineal, entonces para cada $i \in \{1,\ldots,n\}$ tenemos que:
        \begin{equation*}
            0 = \tau(0) = \tau(f(x_i)) = f(\tau(x_i))
        \end{equation*}
        Por lo que $\tau$ permuta las indeterminadas $x_i$, que son los elementos de $G$, por lo que $\tau \in G$, de donde:
        \begin{equation*}
            G = \Aut_{F(s_1, \ldots, s_n)}(E)
        \end{equation*}
        Y como $F(s_1, \ldots, s_n)\leq E$ es de Galois tenemos por la conexión de Galois que:
        \begin{equation*}
            E^G = F(s_1, \ldots, s_n)
        \end{equation*}
    \end{proof}
\end{prop}

\noindent
Consideramos ahora:
\begin{equation*}
    g = x^n - \lm_1 x^{n-1} + \ldots + {(-1)}^{n}\lm_n \in F(\lm_1, \ldots, \lm_n)[x]
\end{equation*}
con $\lm_1, \ldots, \lm_n$ indeterminadas sobre $F$. La ecuación $g=0$ en $x$ se llama \underline{ecuación} \underline{general sobre $F$ de grado $n$}.

\begin{lema}
    Si tomamos $h\in F[\lm_1, \ldots, \lm_n]$ con $h\neq 0$, tenemos entonces que:
    \begin{equation*}
        h(s_1, \ldots, s_n) \neq 0
    \end{equation*}
    \begin{proof}
        Llamamos $s_0 := 1$, y definimos:
        \begin{equation*}
            s_n(x_1, \ldots, x_{n-1}) := 0
        \end{equation*}
        Estas definiciones dan sentido a la fórmula recursiva:
        \begin{equation*}
            s_k(x_1, \ldots, x_n) = s_k(x_1, \ldots, x_{n-1}) + s_{k-1}(x_1, \ldots, x_{n-1})x_n, \quad k \in \{1,\ldots,n\}
        \end{equation*}
        Por inducción sobre $n$:
        \begin{itemize}
            \item \underline{Para $n=1$}, tenemos que $s_1 = x_1$, y tenemos que $h_1(x_1)\neq 0 \Longrightarrow h_1(x_1)\neq 0$.
            \item \underline{Para $n>1$}, razonamos por inducción al absurdo: supongamos que existe $h\neq 0$ pero $h(s_1, \ldots, s_n) = 0$. Entre todos éstos, tomamos:
                \begin{equation*}
                    0\neq h = h_0 + h_1\lm_n + \ldots + h_m \lm_n^m \quad \text{con} \quad h_i \in F[\lm_1, \ldots, \lm_{n-1}]
                \end{equation*}
                de grado mínimo $m$ en $\lm_n$. Tenemos entonces que:
                \begin{align*}
                    0 &= h(s_1, \ldots, s_n) \\
                      &= h_0(s_1, \ldots, s_{n-1}) + h_1(s_1, \ldots, s_{n-1})s_ n + \ldots + h_m(s-1, \ldots, s_{n-1})s_n^m
                \end{align*}
                Evaluando en $x_n = 0$, obtenemos que $s_n = 0$, por lo que:
                \begin{align*}
                    0 &= h_0(s_1(x_1,\ldots,x_{n-1},0) , \ldots , s_{n-1}(x_1, \ldots, x_{n-1},0)) \\
                      &= h_0(s_1(x_1, \ldots, x_{n-1}), \ldots, s_{n-1}(x_1, \ldots, x_{n-1}))
                \end{align*}
                La hipótesis de inducción nos dice que $h_0(\lm_1,\ldots,\lm_{n-1}) = 0$, y sacando factor común $\lm_n$ de la definición de $h$:
                \begin{equation*}
                    h = (h_1 + h_2 \lm_n + \ldots + h_m \lm_n^{m-1})\lm_n
                \end{equation*}
                Evaluando en $(s_1, \ldots, s_{n-1})$ obtengo 
                \begin{equation*}
                    0 = (h_1(s_1, \ldots, s_{n-1}) + h_2(s_1, \ldots, s_{n-1})s_n + \ldots + h_m(s_1, \ldots, s_{n-1})s_n^{m-1})s_n
                \end{equation*}
                y como $s_n\neq 0$, tenemos que:
                \begin{equation*}
                     0 = h_1(s_1, \ldots, s_{n-1}) + h_2(s_1, \ldots, s_{n-1})s_n + \ldots + h_m(s_1, \ldots, s_{n-1})s_n^{m-1}
                \end{equation*}
                de donde obtendríamos que $h_1 + h_2\lm_n + \ldots + h_m\lm_n^{m-1}$ se anula a $(s_1, \ldots, s_n)$, con grado menor que $h$, lo que nos lleva a una \underline{contradicción}.
        \end{itemize}
    \end{proof}
\end{lema}

\begin{prop}
    El polinomio:
    \begin{equation*}
        g = x^n - \lm_1 x^{n-1} + \ldots + {(-1)}^{n} \lm_n \in F(\lm_1,\ldots,\lm_n)[x]
    \end{equation*}
    es irreducible, separable y su grupo de Galois es isomorfo a $S_n$.
    \begin{proof}
        Tomamos $\varepsilon:F[\lm_1,\ldots,\lm_n]\to F(s_1,\ldots, s_n)$ el anillo de polinomios determinado por $\varepsilon(\alpha) = \alpha\quad \forall \alpha\in F$ y $\varepsilon(\lm_i) = s_i\quad  i \in \{1,\ldots,n\}$, tenemos que $\ker\varepsilon = \{0\}$ por el Lema anterior. La propiedad universal del cuerpo de fracciones $F(\lm_1,\ldots,\lm_n)$ nos da un homomorfismo de cuerpos
        \begin{equation*}
            \overline{\varepsilon} :F(\lm_1,\ldots,\lm_n) \to F(s_1,\ldots,s_n)
        \end{equation*}
        que extiende a $\varepsilon$. Tenemos que $\overline{\varepsilon}$ es $F-$lineal y es un isomorfismo de cuerpos. Cardano-Vieta nos dice que $g^{\overline{\varepsilon}} = f$. Teemos que $E = F(x_1,\ldots,x_n)$ es cuerpo de descomposición de $f$, por lo que al aplicar el isomorfismo $\overline{\varepsilon}$ tenemos que  la correstricción:
        \begin{equation*}
            \overline{\varepsilon}:F(\lm_1,\ldots,\lm_n) \to E
        \end{equation*}
        da un cuerpo de descomposición de $g\in F(\lm_1,\ldots,\lm_n)[x]$. El grupo de Galois de $g$ es isomorfo al de $f$, $G$, que es isomorfo a $S_n$.

        \noindent
        Tenemos además que $g$ es separable, porque $f$ lo es. Como su grupo de Galois es transitivo ($S_n$ es transitivo sobre $1,\ldots,n$) tenemos que el grupo de Galois de $f$ es transitivo, luego $f$ es irreducible.
    \end{proof}
\end{prop}

\begin{teo}[de Abel-Ruffini]
    Si $\car(F) = 0$ y $n\geq 5$, entonces $g$ no es resoluble por radicales sobre $F(\lm_1,\ldots,\lm_n)$.
    \begin{proof}
        El grupo de Galois de $g$ es isomorfo a $S_n$ con $n\geq 5$, que no es resoluble, por lo que $f$ no puede ser resoluble por radicales.
    \end{proof}
\end{teo}

\section{Resolución de ecuaciones de grado hasta 4} % // TODO: repasar seccion
\noindent
El Gran Teorema de Galois nos dice que las ecuaciones de grado hasta 4 son resolubles por radicales en característica cero. Realmente lo son para todas las características, por lo que en esta sección vamos a dar procedimientos clásicos para la resolución de ecuaciones de grado 2, 3 y 4. Sea pues $F$ un cuerpo cualquiera, afrontaremos resolver la ecuación $f(x) = 0$ donde $f\in F[x]$  es un polinomio mónico de grado $degf \in \{2,3,4\}$.

\subsection{Cuadrática}
\noindent
Tendremos $f=x^2+bx+c\in F[x]$. Si $\car(F) \neq 2$, podemos escribir:
\begin{equation*}
    f = {\left(x+\frac{b}{2}\right)}^{2} + c - \frac{b^2}{4}
\end{equation*}
y ahora extendiendo $F$ de forma adecuada mediante $F\leq K$, tratamos de despejar $x$, escribiendo las igualdades con elementos de $K$:
\begin{equation*}
    x + \frac{b}{2} = \pm\sqrt{\frac{b^2}{4}-c} = \pm\sqrt{\frac{b^2-4c}{4}} = \pm\frac{\sqrt{b^2-4c}}{2}
\end{equation*}

de donde:
\begin{equation*}
    x = -\frac{b}{2}\pm \frac{\sqrt{b^2-4c}}{2} = \frac{-b\pm \sqrt{b^2 -4c}}{2}
\end{equation*}
Por lo que la ecuación cuadrática es resoluble por radicales para cualquier cuerpo $F$ con $\car(F)\neq 2$.

\subsection{Cúbica}
\noindent
Tendremos $f=x^3+bx^2+cx+d\in F[x]$. Si\footnote{Distinta de 3 para el truco siguiente y distinta de 2 para poder aplicar luego la resolución de cuadrácticas.} $\car(F)\notin \{2,3\}$, consideramos el polinomio:
\begin{equation*}
    g(x) = f\left(x-\frac{b}{3}\right) = x^3+px+q
\end{equation*}

para ciertos $p,q\in F$, ya que: 
\begin{align*}
    g(x) = f\left(x-\frac{b}{3}\right) &= {\left(x-\frac{b}{3}\right)}^{3} + b{\left(x-\frac{b}{3}\right)}^{2} + c\left(x-\frac{b}{3}\right) + d \\
                                &= x^3 \cancel{-bx^2} + \frac{b^2}{3^2}x - \frac{b^3}{3^3} + \cancel{bx^2} + \frac{b^3}{3^3} - \frac{2}{3}b^2x + cx - \frac{bc}{3} + d \\
                                &= x^3 + x\left(\frac{b^2}{3^2} - \frac{2}{3}b^2 + c\right) + \left(d-\frac{bc}{3}\right)
\end{align*}
Como podemos ver, $g$ no tiene término cuadrático. El polinomio $g$ que se obtiene de esta forma a partir de un polinomio cualquiera $f$ de grado 3 recibe el nombre \underline{cúbica reducida de $f$}. Sea $K$ una extensión de $F$ donde están las raíces de $f$ y una raíz cúbica primitiva de la unidad\footnote{Aquí también necesitamos que $\car(F)\neq 3$, para que $x^3-1$ sea separable.} $\omega$. Sean $\alpha_1,\alpha_2,\alpha_3\in K$ las raíces de $g$. Tenemos por las ecuaciones de Cardano-Vieta que:
\begin{equation*}
    \alpha_1 + \alpha_2 + \alpha_3 = 0
\end{equation*}
tomamos ahora:
\begin{align*}
    \beta :&= \alpha_1 + \omega \alpha_2 + \omega^2\alpha_3 \\
    \gamma :&= \alpha_1 + \omega^2\alpha_2 + \omega\alpha_3
\end{align*}
Sumando vemos que ($\omega$ es raíz de $x^2+x+1=0$):
\begin{align*}
    \beta + \gamma &= 2\alpha_1 + \alpha_2(\omega + \omega^2) + \alpha_3(\omega+\omega^2) = 2\alpha_1 -\alpha_2 - \alpha_3 \\
                   &= 2\alpha_1 + \alpha_1 = 3\alpha_1
\end{align*}
Multiplicamos ahora $\beta$ por $\gamma$, obteniendo (usando propiedades de $\omega$): % // TODO: HACER
\begin{align*}
    \beta\gamma &= \alpha_1^2 + \alpha_2^2 + \alpha_3^2 - \alpha_1\alpha_2 - \alpha_2\alpha_3 - \alpha_1\alpha_3  \\
                &= \cancelto{0}{{(\alpha_1 + \alpha_2 + \alpha_3)}^{2}} - 3 (\alpha_1\alpha_2 + \alpha_2\alpha_3 + \alpha_1\alpha_3) = -3p
\end{align*}
De las condiciones $\beta+\gamma=3\alpha_1$ y $\beta\gamma = -3p$ obtenemos una ecuación cuadrática a resolver. Llamamos para simplificar:
\begin{equation*}
    u = \frac{\beta}{3}, \qquad v = \frac{\gamma}{3}
\end{equation*}
tenemos que:
\begin{equation*}
    \alpha_1 = u+v
\end{equation*}
y observamos ahora que:
\begin{equation*}
    u^3 + v^3 + (3uv+p)(u+v) + q = {(u+v)}^{3} + p(u+v) + q = g(u+v) = g(\alpha_1) = 0
\end{equation*}
y usando ahora que $uv = \nicefrac{-p}{3}$, tenemos que:
\begin{equation*}
    0 = u^3 + v^3 + q \quad\Longrightarrow\quad \left\{\begin{array}{l}
        u^3 + v^3 = -q \\
        u^3 v^3 = \frac{-p^3}{27}
    \end{array}\right.
\end{equation*}
Si tomamos:
\begin{equation*}
    h(z) = (z-u^3)(z-v^3) = z^2 + qz - \frac{p^3}{27}
\end{equation*}
El sistema de ecuaciones es equivalente a $h(z) = 0$, obteniendo las soluciones $u^3$ y $v^3$. Si tomamos raíces cúbicas en un cuerpo que extienda al nuestro obtenemos 6 posibles valores de $u$ y $v$. Sabemos ahora que $\alpha_1$ es suma de dos valores de forma que estos son raíces cúbicas, puesto que no hemos supuesto nada a $\alpha_1$ distinto que a $\alpha_2$ y $\alpha_3$.

\noindent
Elegimos entre aquellas parejas de $u$ y $v$ las que verifican $3uv = -p$ (ya que $3uv = \beta\gamma$).

\begin{ejemplo}
    Tomamos $f = x^3-6x^2-9x+2\in \mathbb{Q}[x]$.\\

    \noindent
    Calculamos primero su cúbica reducida:
    \begin{equation*}
        g(x) = f\left(x-\frac{-6}{3}\right) = f(x+2) = x^3-21x-32
    \end{equation*}
    Consideramos ahora:
    \begin{equation*}
        h(z) = z^2 - 32z +343
    \end{equation*}
    Y las raíces de la resolvente cuadŕatica de la reducida cúbica en $\mathbb{C}$ son $u^3 = 16 + i\sqrt{87}$, $v^3 = 16-i\sqrt{87}$.

    \noindent
    Extraemos las raíces cúbicas, obteniendo:
    \begin{align*}
        u_k &= e^{\nicefrac{ik2\pi}{3}}\sqrt[3]{6+i\sqrt{87}}, \quad k = 0,1,2 \\
        v_k &= e^{\nicefrac{ik2\pi}{3}}\sqrt[3]{16-i\sqrt{87}}, \quad k = 0,1,2
    \end{align*}
    Hemos de elegir de acuerdo con la condición $3uv = -p = 21$, lo que nos da:
    \begin{align*}
        u_0v_0 &= u_1v_2 = u_2v_1 = 7
    \end{align*}
    Por tanto, las raíces de la cúbica reducida son:
    \begin{align*}
        u_0+v_0 &= \sqrt[3]{16 + i\sqrt{87}} + \sqrt[3]{16-i\sqrt{87}} \\
        u_1+v_2 &= e^{\nicefrac{i2\pi}{3}} \sqrt[3]{16+i\sqrt{87}} + e^{\nicefrac{i4\pi}{3}} \sqrt[3]{16-i\sqrt{87}} \\
        u_2+v_1 &= e^{\nicefrac{i4\pi}{3}}\sqrt[3]{16+i\sqrt{87}} + e^{\nicefrac{i2\pi}{3}}\sqrt[3]{16-i\sqrt{87}}
    \end{align*}
    Las raíces para $f$ es sumar $2$ a cada una de ellas.
\end{ejemplo}

\subsection{Cuártica}
\noindent
Consideramos ahora $f=x^4+bx^3+cx^2+dx + e \in F[x]$ con $\car(F) \notin \{2,3\}$.

\noindent
El primer paso es reducir la ecuación con la resolvente cúbica:
\begin{equation*}
    g = f\left(x-\frac{b}{4}\right) = x^4+px^2+qx + r 
\end{equation*}
Llamaremos $\beta_1,\beta_2,\beta_3,\beta_4$ a las raíces de $g$ en una extensión adecuada. Por las relaciones de Cardano-Vieta:
\begin{equation*}
    \beta_1 +  \beta_2 + \beta_3 + \beta_4 = 0
\end{equation*}
Tomamos ahora las expresiones (se han obteniedo pensando en la primera y luego aplicando permutaciones sobre los índices):
\begin{align*}
    \rho_1 &= -(\beta_1+\beta_2)(\beta_3+\beta_4) \\
    \rho_2 &= -(\beta_1+\beta_3)(\beta_2+\beta_4) \\
    \rho_3 &= -(\beta_1+\beta_4)(\beta_2+\beta_3)
\end{align*}
De esta forma, independientemente del grupo de Galois de $f$, tenemos que el polinomio:
\begin{equation*}
    h(x) = (x-\rho_1)(x-\rho_2)(x-\rho_3)
\end{equation*}
tiene coeficientes en $F$. De hecho, calculando con ingenio, se obtiene que:
\begin{equation*}
    h(x) = x^3+2px^2 + (p^2-4r)x - q^2
\end{equation*}
Y este es un polinomio del que ya sabemos calcular sus raíces. Falta ver cómo relacionar $\rho_1,\rho_2,\rho_3$ con $\beta_1,\beta_2,\beta_3,\beta_4$. Observamos por ejemplo que:
\begin{equation*}
    \beta_3 + \beta_4 = \beta_1 + \beta_2
\end{equation*}

de donde:
\begin{equation*}
    \rho_1^2 = {(\beta_1+\beta_2)}^{2} \quad\Longrightarrow\quad\beta_1+\beta_2 = \sqrt{\rho_1}
\end{equation*}
Y así obtenemos:
\begin{equation*}
    \left\{\begin{array}{lll}
        \beta_1 + \beta_2 = \sqrt{\rho_1} && \beta_3 + \beta_4 = -\sqrt{\rho_1} \\
        \beta_1 + \beta_3 = \sqrt{\rho_2} && \beta_2 + \beta_4 = -\sqrt{\rho_2} \\
        \beta_1 + \beta_4 = \sqrt{\rho_3} && \beta_2 + \beta_3 = -\sqrt{\rho_3} 
    \end{array}\right.
\end{equation*}
donde elegimos los signos de acuerdo con $\sqrt{\rho_1}\sqrt{\rho_2}\sqrt{\rho_3} = -q$. Sumando de 3 en 3 las igualdades adecuadas obtenemos:
\begin{align*}
    \beta_1 &= \frac{1}{2}(\sqrt{\rho_1} + \sqrt{\rho_2} + \sqrt{\rho_3}) \\
    \beta_2 &= \frac{1}{2}(\sqrt{\rho_1} - \sqrt{\rho_2} + \sqrt{\rho_3}) \\
    \beta_3 &= \frac{1}{2}(-\sqrt{\rho_1} + \sqrt{\rho_2} - \sqrt{\rho_3}) \\
    \beta_4 &= \frac{1}{2}(-\sqrt{\rho_1} - \sqrt{\rho_2} - \sqrt{\rho_3}) 
\end{align*}
Sumando $\nicefrac{b}{4}$ a cada una de ellas obtenemoas las raíces de $f$.\\

\noindent
Ahora, si consideramos $3\in \bb{F}_5$ y nos preguntamos por $\sqrt{3}$ probando vemos que no está en $\bb{F}_5$; por lo que la raíz estará en $\bb{F}_5(\sqrt{3}) = \bb{F}_{25}$.

\noindent
Un poquillo de como resolver ecuaciones en cuerpos finitos.\\


% // TODO: Clase práctica, ver donde meter
\noindent
Vimos que si teníamos $f\in F[x]$ separable, irreducible y de grado primo $p$ entonces el grupo de Galois contiene un ciclo de orden $p$.
\begin{ejercicio}
    Sea $f\in \mathbb{Q}[x]$ irreducible de grado primo $p$. Se pide demostrar que si $f$ tiene exactamente 2 raíces complejas no reales entonces su grupo de Galois es $S_p$.\\

    \noindent
    Si tomamos $\alpha_1,\ldots,\alpha_{p-2}\in \mathbb{R}$; $\alpha,\overline{\alpha}\in \mathbb{C}\setminus \mathbb{R}$ las raíces de $f$ tenemos que el cuerpo de descomposición de $f$ es:
    \begin{equation*}
        \mathbb{Q}(\alpha_1,\ldots,\alpha_{p-2})(\alpha,\overline{\alpha}) \leq \mathbb{C}
    \end{equation*}
    Además, tenemos que la conjugación compleja deja fijo el cuerpo de descomposición de $f$. Visto como permutaciones tenemos que es una trasposición, por lo que el grupo de Galois de $f$ visto como subgrupo de $S_p$ contiene una trasposición.

    \noindent
    Por el ejercicio mencionado antes tenemos que el grupo de Galois de $f$ contiene además un ciclo de orden $p$. Como estos dos elementos generan $S_p$ ha de ser el grupo de Galois igual a $S_p$. % // TODO: Razonar bien esto
\end{ejercicio}

\begin{ejercicio}
    Sea $f=x^5-4x-1\in \mathbb{Q}[x]$, veamos que el grupo de Galois de $f$ es isomorfo a $S_5$.\\

    \noindent
    Como $f\in \mathbb{Z}[x]$, tenemos que $f$ es irreducible si y solo si $f\in \mathbb{Z}[x]$ es irreducible. Reducimos módulo 3, obteniendo:
    \begin{equation*}
        \overline{f}  = x^5-x-1\in \mathbb{Z}_3[x]
    \end{equation*}
    que no tiene raíces en $\mathbb{Z}_3$. Los posibles factores de grado 2 son todos aquellos de grado 2 irreducibles:
    \begin{equation*}
        x^2+1, \qquad x^2+2x+2 \qquad, x^2+x+2
    \end{equation*}
    con la división euclidiana vemos que al dividir $f$ entre estos ningún resto sale nulo, por lo que $f$ tiene que ser irreducible en $\mathbb{Z}_3[x]$, por ser de grado 5 y no tener factores ni de grado 1 (no tiene raíces) ni de grado 2. Por tanto, $f\in \mathbb{Z}[x]$  es irreducible.\\

    \noindent
    Veamos ahora cuántas raíces en $\mathbb{R}$ tiene $f$. Para ello:
    \begin{equation*}
        f' = 5x^4-4
    \end{equation*}
    imponiendo $f' = 0$ y quedándonos con las reales obtenemos como puntos críticos $\pm \sqrt{\frac{4}{5}}$. 
    Evaluando como en bachiller:
    \begin{equation*}
        f(-2) = -25<0, \qquad f(-1) = 2>0, \qquad f(0) = -1<0
    \end{equation*}
    Vemos que $f$ tiene que tener 3 raíces reales, ya que tiene 2, las raíces complejas van en parejas y por la derivada sabemos que $f$ no puede tener más de 3 raíces reales.\\

    \noindent
    El último ejercicio nos dice que el grupo de Galois de $f$ es $S_5$, que no es resoluble, por lo que $f$ No es resoluble por radicales, por el gran Teorema de Galois.
\end{ejercicio}

% // TODO: El Tª de Dedekind no entra, es como sacar el grupo de Galois a partir de los grupos de las reducciones

\begin{ejercicio} % // TODO: 44
    Sea $f=x^n-a\in F[x]$ separable y $K$ su cuerpo de descomposición. Fijado $\zeta\in K$ una raíz $n-$ésima primitiva de la unidad, fijamos $\sqrt[n]{a}\in K$. Sea $\sigma\in \Aut_F(K)$, denotamos por $j(\sigma),k(\sigma)\in \mathbb{Z}_n$ a los elementos determinados por 
    \begin{equation*}
        \sigma\left(\sqrt[n]{a}\right) = \zeta^{j(\sigma)}\sqrt[n]{a}, \qquad \sigma(\zeta) = \zeta^{k(\sigma)}
    \end{equation*}
    con $k(\sigma)\in \cc{U}(\mathbb{Z}_n)$. Comprobar que la aplicación\footnote{Si se recuerda la demostración de que una matriz es invertible si y solo si su determinante es no nulo se puede hacer también considerando los coeficientes solo sobre un anillo conmutativo.} $\Aut_F(K)\to GL_2(\mathbb{Z}_n)$ dada por:
    \begin{equation*}
        \sigma\longmapsto \left(\begin{array}{cc}
            1 & 0 \\
            j(\sigma) & k(\sigma) 
        \end{array}\right)
    \end{equation*}
    es un homomorfismo inyectivo de grupos.\\

    \noindent
    Vemos que la aplicación está bien definida, pues:
    \begin{equation*}
        det\left(\begin{array}{cc}
            1 & 0 \\
            j(\sigma) &  k(\sigma)
        \end{array}\right) = k(\sigma) \in \cc{U}(\mathbb{Z}_n)
    \end{equation*}
    Si tomamos $\sigma$ de forma que su matriz es la identidad tenemos por la definición de $j(\sigma)$ y de $k(\sigma)$ que $\sigma = id$. Se comprueba fácil que es un homomorfismo. Por tanto, $|\Aut_F(K)|$ es un divisor de $|GL_2(\mathbb{Z}_n)| = n\times \varphi(n)$. Y tenemos que alcanza el máximo si $f$ es irreducible sobre $F$.\\

    \noindent
    Se llaman grupos holomorfos, los de las matrices triangulares.
\end{ejercicio}

\section{Cuerpos finitos}
% Cosas de cuerpos finitos, para k las sepamos

\begin{teo}
    Sean $p$ un primo y $k,n\geq 1$, si consideramos una extensión de cuerpos finitos $\bb{F}_{p^k}\leq \bb{F}_{p^{kn}}$, tenemos entonces que $\bb{F}_{p^{kn}}$ es cuerpo de descomposición de $x^{p^{kn}}-x\in \bb{F}_{p^k}[x]$, así como que $\Aut_{\bb{F}_{p^k}}(\bb{F}_{p^{kn}})$ es cíclico de orden $n$, y está generado por $\phi$, donde:
    \begin{equation*}
        \phi(\alpha) = \alpha^{p^k} \qquad \forall \alpha \in \bb{F}_{p^{kn}}
    \end{equation*}
    \begin{proof}
        Vimos ya que $\bb{F}_{p^{kn}}$ era cuerpo de descomposición del polinomio $x^{p^{kn}}-x\in \bb{F}_p[x]$, y como $\bb{F}_p \leq \bb{F}_{p^k} \leq \bb{F}_{p^{kn}}$, tenemos entonces que $\bb{F}_{p^{kn}}$ es cuerpo de descomposición de $x^{p^{kn}}-x\in \bb{F}_{p^k}[x]$.\\

        \noindent
        Observemos que tenemos la torre de cuerpos finitos:
        \begin{equation*}
            \bb{F}_p \leq \bb{F}_{p^k} \leq \bb{F}_{p^{kn}}
        \end{equation*}
        Por lo que las tres extensiones de cuerpos que aparecen son de Galois. Vimos que el automorfismo de Frobenius $\tau:\bb{F}_{p^{kn}}\to \bb{F}_{p^{kn}}$ nos permitía escribir:
        \begin{equation*}
            \Aut_{\bb{F}_p}(\bb{F}_{p^{kn}}) = \langle \tau \rangle 
        \end{equation*}

        donde:
        \begin{equation*}
            \tau(\alpha) = \alpha^p \qquad \forall \alpha\in \bb{F}_{p^{kn}}
        \end{equation*}
        Como $\Aut_{\bb{F}_{p^k}}(\bb{F}_{p^{kn}})$ es un subgrupo de $G = \Aut_{\bb{F}_p}(\bb{F}_{p^{kn}})$, este último cíclico de orden $kn$, tenemos que el primero es cíclico. Como $\bb{F}_{p^k}\leq \bb{F}_{p^{kn}}$ es de Galois, el orden de $\Aut_{\bb{F}_{p^k}}(\bb{F}_{p^{kn}})$ será igual al grado de la extensión $\bb{F}_{p^k}\leq \bb{F}_{p^{kn}}$, que es:
        \begin{equation*}
            [\bb{F}_{p^{kn}} : \bb{F}_{p^k}] = n 
        \end{equation*}
        En vista de que el orden de $\tau$ es $kn$ y que buscamos un elemento de orden $n$, tomamos $\phi = \tau^k$, que verifica:
        \begin{equation*}
            \phi(\alpha) = \tau^k(\alpha) = \alpha^{p^k} \qquad \forall \alpha \in \bb{F}_{p^{kn}}
        \end{equation*}
        Observemos que $\phi$ es $\bb{F}_{p^k}-$lineal, ya que si $\gamma \in \bb{F}_{p^k}$, tenemos entonces que el orden multiplicativo de $\gamma$ es divisor de $p^k-1$, con lo que:
        \begin{equation*}
            \phi(\gamma) = \gamma^{p^k} = \gamma^{p^k-1}\cdot  \gamma = \gamma
        \end{equation*}
        Así, $\phi$ es un elemento de $\Aut_{\bb{F}_{p^k}}(\bb{F}_{p^{kn}})$ de orden $n$, por lo que ha de generar todo el grupo.
    \end{proof}
\end{teo}

\begin{notacion}
    En vistas el Teorema anterior, bajo sus mismas hipótesis, llamaremos a $\phi$ automorfismo de Frobenius de la extensión $\bb{F}_{p^k}\leq \bb{F}_{p^{kn}}$.
\end{notacion}

\begin{teo}
    Si $f\in \bb{F}_{p^k}[x]$ es un polinomio irreducible de grado $n$, entonces su cuerpo de descomposición  es $\bb{F}_{p^{kn}}$. Además, si $\alpha\in \bb{F}_{p^{kn}}$ es una raíz de $f$, entonces el resto de sus raíces son
    \begin{equation*}
        \alpha^{p^k}, \alpha^{p^{2k}}, \ldots, \alpha^{p^{(n-1)k}}
    \end{equation*}
    \begin{proof}
        Suponemos que $f$ es mónico. Sabemos que $f$ tiene una raíz en alguna extensión de grado $n$ de $\bb{F}_{p^k}$, como por ejemplo en:
        \begin{equation*}
            \bb{F}_{p^{kn}} := \frac{\bb{F}_{p^k}[x]}{\langle f \rangle }
        \end{equation*}
        Sea $\alpha$ una raíz de $f$ en $\bb{F}_{p^{kn}}$, observemos que la extensión $\bb{F}_{p^k}\leq \bb{F}_{p^{kn}}$ es de Galois, por lo que $f$ tiene todas sus raíces en $\bb{F}_{p^{kn}}$ (por ser la extensión normal) y como $f = \Irr(\alpha,\bb{F}_{p^k})$ tenemos también que $f$ es separable (por ser la extensión separable). Vemos además que el grupo de Galois de $f$ es $\Aut_{\bb{F}_{p^k}}(\bb{F}_{p^{kn}})$, generado por $\phi$, el automorfismo de Frobenius de la extensión, por lo que:
        \begin{equation*}
            \alpha^{q^k} = \phi(\alpha)^k  \qquad \forall k \in \{1,\ldots n\}
        \end{equation*}
        son todas raíces de $f$. Además, como tenemos $n$ de ellas, hemos obtenido todas las raíces de $f$.
    \end{proof}
\end{teo}

\begin{teo}
    Un polinomio irreducible $f\in \bb{F}_{p^k}[x]$ de grado $n$ divide al polinomio $x^{p^{km}}-x\in \bb{F}_{p^k}[x]$ si, y solo si, $n$ divide a $m$.\newline
    Como consecuencia, $x^{p^{km}}-x\in \bb{F}_{p^{k}}[x]$ es producto de todos los polinomios irreducibles en $\bb{F}_{p^k}[x]$ cuyo grado divide a $m$.
    \begin{proof}
        Por doble implicación y llamando $g = x^{p^{km}}-x\in \bb{F}_{p^k}[x]$:
        \begin{description}
            \item [$\Longrightarrow )$] Supongamos que $f$ es irreducible y que divide a $g$. Si tomamos los cuerpos de descomposición de ambos polinomios, $\bb{F}_{p^{kn}}$ y $\bb{F}_{p^{km}}$, como $f$ divide a $g$, todas las raíces de $f$ lo serán de $g$, por lo que tendremos que $\bb{F}_{p^{kn}}\leq \bb{F}_{p^{km}}$. El Lema de la Torre aplicado a:
                \begin{equation*}
                    \bb{F}_{p^k} \leq \bb{F}_{p^{kn}} \leq \bb{F}_{p^{km}}
                \end{equation*}
                nos dice que $n$ divide a $m$, ya que:
                \begin{equation*}
                    [\bb{F}_{p^{kn}}:\bb{F}_{p^k}] = n, \qquad [\bb{F}_{p^{km}}:\bb{F}_{p^k}] = m
                \end{equation*}

                Tomamos sendos cuerpos de descomposición sobre $\bb{F}_q$: $\bb{F}_{q^n}\leq \bb{F}_{q^m}$, de donde $n\mid m$ por el Lema de la Torre.
            \item [$\Longleftarrow )$] Si $f$ es irreducible y $n\mid m$ tenemos entonces que $(p^{kn}-1)$ divide a $(p^{km}-1)$, de donde el polinomio $x^{p^{kn}}-x\in \bb{F}_{p^k}[x]$ dividirá al polinomio $x^{p^{km}}-x\in \bb{F}_{p^k}[x]$, por lo que $\bb{F}_{p^{kn}}\leq \bb{F}_{p^{km}}$.

                Como $\bb{F}_{p^{kn}}$ es cuerpo de descomposición de $f$, tenemos que todas sus raíces están en $\bb{F}_{p^{km}}$, pero todas estas son a su vez raíces de $x^{p^{km}}-x\in \bb{F}_{p^{k}}[x]$, por lo que $f$ divide a $x^{p^{km}}-x$.
        \end{description}
        Finalmente, como $x^{p^{km}}-x\in \bb{F}_{p^k}[x]$ es separable, será producto de polinomios irreducibles mónicos distintos, que por la caracterización recién vista, obtenemos que todos sus términos son los polinomios mónicos irreducibles en $\bb{F}_{p^k}[x]$ cuyo grado divide a $m$.
    \end{proof}
\end{teo}

\begin{ejemplo}
    Factoricemos $x^{16}+x\in \bb{F}_2[x]$ como producto de irreducibles.\\

    \noindent
    Vemos que $16 = 2^4$, por lo que según el Teorema anterior, el polinomio factoriza como producto de todos los polinomios mónicos irreducibles de grado un divisor de 4:
    \begin{itemize}
        \item De grado 1 sabemos que solo son 2: $x,x-1\in \bb{F}_2[x]$.
        \item De grado 2 sabemos que solo hay uno, $x^2+x+1\in \bb{F}_2[x]$.
        \item De grado 4. Sabemos ya que los polinomios anteriores aparecerán en la factorización de $x^{16}+x$, y entre ellos sumamos ya hasta grado 4, por lo que solo puede haber 3 polinomios irreducibles de grado 4, y todos ellos dividen a $x^{16}+x$.

            Para buscarlos, buscamos los polinomios de grado 4 en $\bb{F}_{2}[x]$ que no tengan raíces (y que por tanto tengan un número impar de monomios) y que no sean el polinomio:
            \begin{equation*}
                {(x^2+x+1)}^{2} = x^4+x^2+1
            \end{equation*}
            Estos son:
            \begin{equation*}
                x^4+x+1, \qquad x^4+x^3+1, \qquad x^4+x^3+x^2+x+1
            \end{equation*}
    \end{itemize}
    Por lo que la factorización será:
    \begin{equation*}
        x^{16}+x = x(x+1)(x^2+x+1)(x^4+x+1)(x^4+x^3+1)(x^4+x^3+x^2+x+1)
    \end{equation*}
    En esta última factorización vemos que $\bb{F}_{16}$ puede presentarse de 3 formas distintas sobre $\bb{F}_2$, tomando $\bb{F}_{16} = \bb{F}_2(a)$ con $a$ raíz de cualquiera de los 3 polinomios de grado 4. Para cada polinomio tendremos una presentación posible. Es decir:

    \begin{itemize}
        \item Si tomamos $f=x^4+x+1\in \bb{F}_2[x]$, tenemos que su cuerpo de descomposición es $\bb{F}_{16}$, y podemos dar el cuerpo como $\bb{F}_2(\alpha)$, con $\alpha^4 +\alpha+1=0$. Además, las demás raíces de $f$ son $\alpha^2$, $\alpha^4$ y $\alpha^8$
        \item Si tomamos $g=x^4+x^3+1\in \bb{F}_2[x]$, queremos calcular ahora las raíces de $g$ en $\bb{F}_{16}$ en función de $\alpha$.

            En primer lugar, sabemos que todas las raíces de $g$ están en $\bb{F}_{16}$, porque la extensión $\bb{F}_2\leq \bb{F}_{16}$ es de Galois y en las extensiones de Galois bastaba encontrar una solución en $\bb{F}_{16}$ de $g$ (irreducible) para tenerlas todas. 

            Además, por uno de los últimos teoremas sabemos que si encontramos una de ellas el resto vienen dadas por elevar al cuadrado repetidas veces dicha raíz.

            Sabemos que $\bb{F}_{16}$ tiene que tener una raíz de $g$ porque $\bb{F}_{16}$ consiste en exclusivamente las raíces de $x^{16}+x\in \bb{F}_2[x]$, que según la descomposición en factores irreducibles nos dice que todas las raíces de $x^4+x^3+1$ son también raíces de $x^{16}+x$.

            Para calcular sus raíces, calculemos el orden de $\alpha$ en el grupo multiplicativo $\bb{F}_{16}^\times$. Los posibles órdenes de sus elementos son $1,3,5$ y 15:
            \begin{itemize}
                \item Sabemos que $O(\alpha)\neq 1$, ya que $\alpha\neq 1$ porque 1 no es raíz de $f$.
                \item Sabemos que $\{1,\alpha,\alpha^2,\alpha^3\}$ es una $\bb{F}_2-$base de $\bb{F}_{16}$, por lo que $\alpha$ y $\alpha^3$ son linealmente independientes, luego no puede ser $O(\alpha) = 3$.
                \item Vemos ahora que $\alpha^5 = \alpha \alpha^4 = \alpha(\alpha+1) = \alpha^2 + \alpha \neq 1$, ya que $1,\alpha$ y $\alpha^2$ son $\bb{F}_2-$linealmente independientes, luego ha de ser $O(\alpha)\neq 5$.
            \end{itemize}
            Concluimos que ha de ser $O(\alpha) = 15$, por lo que:
            \begin{equation*}
                \bb{F}_{16} = \{0,1,\alpha, \alpha^2, \ldots, \alpha^{14}\}
            \end{equation*}
            Observamos primero que las raíces de $f$ y de $g$ han de ser distintas, porque $f$ y $g$ son dos polinomios irreducibles distintos. Por tanto, las raíces de $g$ no son $\alpha,\alpha^2,\alpha^4,\alpha^8$. Tampoco pueden ser $0$ ni $1$, por lo que nos quedan 8 posibles candidatos a raíz.

            Buscamos heurísticamente generadores de $\bb{F}_{16}^\times$ (ya que hemos obtenido $\bb{F}_{16}$ por $\alpha$, que era raíz de $f$), por lo que creemos que $\alpha^7$ (la primera potencia de $\alpha$ que genera todo el grupo cíclico) es un candidato a raíz de $g$. Lo comprobamos (pensando que $\alpha^{15} = 1$):
            \begin{align*}
                g(\alpha^7) &= {(\alpha^7)}^{4} + {(\alpha^7)}^{3} + 1 = \alpha^{28} + \alpha^{21} + 1 = \alpha^{13} + \alpha^{6} + 1 = {(\alpha^4)}^{3}\alpha + \alpha^4 \alpha^2 + 1 \\
                            &= {(\alpha+1)}^{3}\alpha + (\alpha+1)\alpha^2 + 1 = (\alpha^3 + \alpha^2 + \alpha + 1)\alpha + \alpha^3 + \alpha^2 + 1 \\
                            &= \alpha + 1 + \alpha^3 + \alpha^2 + \alpha + \alpha^3 + \alpha^2 + 1 = 0
            \end{align*}
            En efecto, $\alpha^7$ es una raíz de $g$. Ahora:
            \begin{itemize}
                \item Elevamos $\alpha^7$ al cuadrado varias veces.
                \item Vemos elevar al cuadrado como automorfismo de Frobenius, que restringido a $\bb{F}_{16}^\times$ es un automorfismo de grupos, por lo que lleva generadores en generadores, obteniendo que las raíces de $g$ son:
                    \begin{equation*}
                        \alpha^7, \alpha^{11}, \alpha^{13} \text{\ y\ } \alpha^{14}
                    \end{equation*}
            \end{itemize}
        \item Nos falta clasificar 6 raíces, 2 de $k=x^2+x+1$ y 4 de $h=x^4+x^3+x^2+x+1$.

            Tomamos la potencia más pequeña de $\alpha$ que no hayamos clasificado:
            \begin{equation*}
                k(\alpha^3) = \alpha^6 + \alpha^3 + 1 = (\alpha+1)\alpha^2 + \alpha^3 + 1 = \alpha^3 + \alpha^2 + \alpha^3 + 1 = \alpha^2 + 1 \neq 0
            \end{equation*}
            donde $\alpha^2 + 1 \neq 0$ porque $1$ y $\alpha^2$ son $\bb{F}_2-$linealmente independientes, por lo que $\alpha^3$ es raíz de $h$ y obtenemos todas sus raíces elevando $\alpha^3$ al cuadrado, obteniendo:
            \begin{equation*}
                \alpha^3, \alpha^6, \alpha^{12} \text{\ y\ } \alpha^9
            \end{equation*}
            Las que quedan son $\alpha^5$ y $\alpha^{10}$, que han de ser raíces de $k$.

            Podríamos haberlo hecho también pensando también en que $\alpha^5$ tiene orden 3 y $\alpha^3$ tiene orden 5 sobre $\bb{F}_{16}^{\times}$. Como $x^2+x+1$ tiene cuerpo de descomposición $\bb{F}_4$, tendremos un elemento $\alpha^2 +\alpha+1=0$  con orden 3 en $\bb{F}_4^\times$, por lo que viendo la extensión $\bb{F}_4\leq \bb{F}_{16}$ obtendremos finalmente que las raíces de $k$ son las de orden 3.
    \end{itemize}
\end{ejemplo}


