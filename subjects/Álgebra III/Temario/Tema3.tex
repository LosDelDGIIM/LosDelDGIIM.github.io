\chapter{Teoría de Galois de Ecuaciones}
\section{Grupo de Galois de un polinomio}
\noindent
A lo largo de este capítulo, consideraremos siempre polinomios mónicos.

\begin{definicion} % El discriminante resulta ser caso particular de la resultante de dos polinomios (en particular, de un polinomio y su derivado), que es para ver si dos curvas se tocan o no, viene de la geometría clásica
    Sea $f\in F[x]$ no constante, mónico y sean $\alpha_1, \ldots, \alpha_n$ sus raíces (repetidas tantas veces como indique su multiplicidad) en algún cuerpo $K$ de descomposición de $f$. El \underline{discriminante} de $f$ es:
    \begin{equation*}
        \Disc(f) = \prod_{1\leq i<j \leq n} {(\alpha_i-\alpha_j)}^{2} \in K
    \end{equation*}
\end{definicion}

\noindent
Resulta que $\Disc(f)$ se puede calcular a partir de los coeficientes del polinomio.

\begin{observacion}
    $f$ es separable $\Longleftrightarrow \Disc(f)\neq 0$. % // TODO: Escribir
\end{observacion}

\begin{notacion}
    Notaremos usualmente a la raíz del discriminante $Disc(f)$ por:
    \begin{equation*}
        \Delta(f) = \prod_{1\leq i < j \leq n}(\alpha_i-\alpha_j)
    \end{equation*}
\end{notacion}

\begin{notacion} % // TODO: Estudiar los S_n, y subgrupos de S4
    Dado un conjunto $S = \{\alpha_1, \ldots, \alpha_n\}$, denotamos al grupo de permutaciones de dichos elementos por:
    \begin{equation*}
        \Sim(\alpha_1, \ldots, \alpha_n)
    \end{equation*}
    Observemos que $\Sim(\alpha_1,\ldots, \alpha_n)\cong S_n$.
\end{notacion}

\begin{definicion}
    Si $f\in F[x]$ es separable y $K$ es su cuerpo de descomposición, $\Aut_F(K)$ se llagma grupo de Galois de $f$. Consideraremos la aplicación definida por restricción:
    \begin{align*}
        \Aut_F(K) &\longrightarrow \Sim(\alpha_1, \ldots, \alpha_n) \\
        \sigma&\longmapsto \sigma\big|_{\{\alpha_1, \ldots, \alpha_n\}}
    \end{align*}

    así, como $\Sim(\alpha_1,\ldots,\alpha_n)\cong S_n$, podemos ver $\Sim(\alpha_1,\ldots,\alpha_n)$ como permutar los subíndices, de forma que:
    \begin{equation*}
        \alpha_i \stackrel{\sigma}{\longmapsto} \alpha_{\sigma(i)}
    \end{equation*}
    donde consideramos que $\alpha_{\sigma(i)} := \sigma(\alpha_i)$.
\end{definicion}

% Sera habitual ver Aut_F(K) = Sim(.) = Sn

\begin{observacion}
    Si tomamos $\sigma\in \Aut_F(K)$:
    \begin{itemize}
        \item $\sigma(\Disc(f)) = \Disc(f)$.
        \item $\sigma(\Delta(f)) = sgn(\sigma)\Delta(f)$.
    \end{itemize}
\end{observacion}

\begin{prop}
    Sea $f\in F[x]$ separable con grupo de Galois $G = \Aut_F(K)$. Entonces $\Disc(f) \in F$. Además:
    \begin{equation*}
        K^{G\cap A_n} = F(\Delta(f))
    \end{equation*}
    Por tanto, $\Delta(f) \in F \Longleftrightarrow G\leq A_n$.
    \begin{proof}
        Para lo primero, como $\sigma(\Disc(f)) = \Disc(f)$, tenemos $\Disc(f)\in K^G$, y además $F = K^G$ por ser $F\leq K$ de Galois.\\

        \noindent
        Además, de la segunda observación vemos que $\Delta(f)\in K^{G\cap A_n}$, por lo que:
        \begin{equation*}
            F(\Delta(f)) \leq K^{G\cap A_n}
        \end{equation*}
        Tenemos por la conexión de Galois que:
        \begin{equation*}
            \left[K^{G\cap A_n} : F\right] = (G : G\cap A_n) \leq (S_n : A_n) = 2
        \end{equation*}
        donde en la desigualdad hemos usado el Tercer Teorema de Isomorfía para grupos. Por tanto, bien el grado de la extensión es 1 o 2, en función de si $\Delta(f) \in F$. % // TODO: TERMINAR
    \end{proof}
\end{prop}

\noindent
La condición ``$\Delta(f)\in F$'' se suele decir por ``$\Disc(f)$ es un cuadrado en $F$''.

\begin{ejercicio} % // TODO: HACER
    Sea $f\in \mathbb{R}[x]$ con $degf = 3$, discutir el número de raíces reales de $f$ según el signo de $\Disc(f)$. % Pasar al caso monico
\end{ejercicio}

\begin{ejemplo}
    Consideramos $f = x^n +\sum\limits_{i=0}^{n-1}a_ix^i \in F[x]$ y sean $\alpha_1, \ldots, \alpha_n$ sus raíces (repetidas según multiplicidad), tenemos que:
    \begin{equation*}
        f = \prod_{i=1}^{n}(x-\alpha_i)
    \end{equation*}
    Igualando coeficientes de igual grado, obtenemos las relaciones de Cardano-Vieta\footnote{Hay una teoría desarrollada sobre esto, siempre se obtienen funciones simétricas en las raíces del polinomio.}. Por ejemplo, si $n=2$ se obtiene:
    \begin{equation*}
        a_0 = \alpha_1\alpha_2 \qquad a_1 = -(\alpha_1+\alpha_2)
    \end{equation*}
    Como $\Disc(f) = {(\alpha_1-\alpha_2)}^{2}$, tenemos que $\Disc(f) = a_1^2 - 4a_0$. \newline
    Para $n>2$, la cuenta no es tan sencilla, por lo que se prefiere usar un algoritmo para resolver el sistema de ecuaciones. Por tanto, se puede expresar $\Disc(f)$ en término de los coeficientes de $f$. Para $n=3$, la damos para $f=x^3+px+q$  (cúbica reducida\footnote{Sin término cuadrático.}) es:
    \begin{equation*}
        \Disc(f) = -4p^3 - 27q^2
    \end{equation*}
\end{ejemplo}

\begin{prop}
    Sea $f\in F[x]$ separable con grupo de Galois $G$
    \begin{equation*}
        f\text{\ es irreducible} \Longleftrightarrow G\text{\ actúa transitivamente sobre las raíces de\ } f
    \end{equation*}
    En tal caso, $degf$ divide a $|G|$.
    \begin{proof}
        Sea $K$ el cuerpo de descomposición de $f$, tenemos que $G = \Aut_F(K)$.
        \begin{description}
            \item [$\Longrightarrow )$] Si $f$ es irreducible y $\alpha,\beta\in K$ son raíces de $f$, podemos ($f = \Irr(\alpha,F)$) usar la Proposición de extensión, obteniendo $\sigma:F(\alpha)\to K$ de forma que $\sigma(\alpha) = \beta$.

                La tercera proposición nos dice que $\sigma$ se extiende a un $\eta:K\to K$, con lo que $\eta\in G$ y $\eta(\alpha) = \sigma(\alpha) = \beta$, por lo que la acción es transitiva.
            \item [$\Longleftarrow )$] Sea $g$ un factor irreducible de $f$ (ambos mónicos), tenemos que $g$ no es constante, con lo que sus raíces son también de $f$. Además, $\sigma(\alpha)$ es raíz de $g$, para todo $\sigma\in G$, y como $G$ actúa transitivamente sobre las raíces de $f$; toda raíz de $f$ es de $g$, con lo que $f = g$, de donde $f$ es irreducible.
        \end{description}
        Finalmente, para ver que $degf$ divide a $|G|$, si $\alpha$ es raíz de $f$, tenemos entonces $[F(\alpha):F] = degf$, que divide a $[K:F]$ por el Lema de la Torre, y $|G| = [K:F]$.
    \end{proof}
\end{prop}

\begin{coro}
    Por tanto, a la hora de buscar el grupo de Galois de un polinomio, descartaremos automáticamente los subgrupos de $S_4$ no transitivos.
\end{coro}

\begin{ejemplo}
    Sea $f\in F[x]$ separable e irreducible: 
    \begin{enumerate}
        \item Si $degf = 1$, su grupo de Galois es la identidad.
        \item Si $degf = 2$, la extensión de Galois de $f$ tiene grado 1 o 2. Si $f$ es irreducible, ha de ser de grado 2, con lo que su grupo de Galois es isomorfo a $C_2$ (observemos que $S_2\cong C_2$).
        \item Si $degf = 3$, la Proposición anterior nos dice que bien $G\cong A_3$ o $G\cong S_3$. La Proposición vista antes de eso, tenemos el primer caso si $\Delta(f)\in F$ y el segundo si $\Delta(f)\notin F$.
        \item Si $degf = 4$, la Proposición anterior nos dice que $G$ es isomorfo a un subgrupo transitivo de $S_4$. % // TODO: Aprender la lista
    \end{enumerate}
\end{ejemplo}

\begin{ejemplo}
    Sea $f\in F[x]$ polinomio separable e irreducible de grado $degf = 4$, sean $\alpha_1,\alpha_2,\alpha_3,\alpha_4$ las raíces de $f$ en un cuerpo de descomposición $K$ de $f$, consideramos:
    \begin{align*}
        \beta_1 = \alpha_1\alpha_2 + \alpha_3 \alpha_4 \\
        \beta_2 = \alpha_1\alpha_3 + \alpha_2 \alpha_4 \\
        \beta_3 = \alpha_1\alpha_4 + \alpha_2 \alpha_3 
    \end{align*}

    y definimos:
    \begin{equation*}
        g = (x-\beta_1)(x-\beta_2)(x-\beta_3)\in K[x]
    \end{equation*}
    Veamos que en realidad $g\in F[x]$. Para ello, como $F\leq K$ es de Galois, hemos de ver que el polinomio es fijo por todos los automorfismos del grupo de Galois de $f$ (basta verlo para todas las permutaciones). Concluimos que $g^{\sigma} = g\quad \forall \sigma\in G$, con lo que $g$ es una resolvente cúbica de $f$ (se verá).

    \noindent
    Se puede ver por el algoritmo mencionado anteriormente que si $f=x^4+bx^3+cx^2+dx+e$, entonces:
    \begin{equation*}
        g = x^3-cx^2+(bd-4e)x-b^2e+4ce-d^2
    \end{equation*}~\\

    \noindent
    Consultamos si sus raíces son distintas:
    \begin{equation*}
        \beta_2 - \beta_1 = (\alpha_2 - \alpha_3)(\alpha_4-\alpha_1)
    \end{equation*}
    Por lo que $\beta_2$ y $\beta_1$ son distintas (análogo para el resto de las parejas), con lo que $g$ es separable, luego $E= F(\beta_1,\beta_2,\beta_3)$ es una extensión de Galois de $F$, con $F\leq E \leq K$, de donde el grupo de Galois de $g$, $N = \Aut_E(K)$ es normal en $G$. Por lo que:
    \begin{equation*}
        \Aut_F(E)\cong \frac{G}{N}
    \end{equation*}~\\

    \noindent
    Consideramos $S:\Sim(\alpha_1,\alpha_2,\alpha_3,\alpha_4)\to\Sim(\beta_1,\beta_2,\beta_3)$ una aplicación de forma que:
    \begin{equation*}
        S(\sigma)(\alpha_i\alpha_j + \alpha_k\alpha_l) = \alpha_{\sigma(i)}\alpha_{\sigma(j)} + \alpha_{\sigma(k)} \alpha_{\sigma(l)}
    \end{equation*}
    que es un homomorfimo de grupos y es sobreyectivo (ya que dada una trasposición en el grupo de la derecha, podemos encontrar un elemento en la izquierda cuya imagen vaya a él). Calculamos su núcleo:
    \begin{equation*}
        \ker S = \{(1), (1\ 2)(3\ 4), (1\ 3)(2\ 4), (1\ 4)(2\ 3)\}
    \end{equation*}
    Y sabemos que son todas porque como el grupo de la derecha tiene $6$ elementos y el de la derecha 24; con lo que $\ker S = V$.\\

    \begin{figure}[H]
        \centering
        \shorthandoff{""}
        \begin{tikzcd}
            1 \arrow[r] & V \arrow[r] & {Sim(\alpha_1,\alpha_2,\alpha_3,\alpha_4)} \arrow[rr, "S"]                  &  & {Sim(\beta_1,\beta_2,\beta_3)} \arrow[r] & 1 \\
            1 \arrow[r] & N \arrow[r] & G \arrow[u] \arrow[rr, "r"'] \arrow[rru, bend left] \arrow[rru, bend right] &  & Aut_F(E) \arrow[r] \arrow[u]             & 1
        \end{tikzcd}
        \shorthandon{""}
    \end{figure}
    \noindent
    Por lo que:
    \begin{equation*}
        N = V\cap G
    \end{equation*}
\end{ejemplo}
