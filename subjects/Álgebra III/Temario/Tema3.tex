\chapter{Teoría de Galois de Ecuaciones}
\section{Grupo de Galois de un polinomio}
\noindent
A lo largo de este capítulo, consideraremos siempre polinomios mónicos.

\begin{definicion} % El discriminante resulta ser caso particular de la resultante de dos polinomios (en particular, de un polinomio y su derivado), que es para ver si dos curvas se tocan o no, viene de la geometría clásica
    Sea $f\in F[x]$ no constante, mónico y sean $\alpha_1, \ldots, \alpha_n$ sus raíces (repetidas tantas veces como indique su multiplicidad) en algún cuerpo $K$ de descomposición de $f$. El \underline{discriminante} de $f$ es:
    \begin{equation*}
        \Disc(f) = \prod_{1\leq i<j \leq n} {(\alpha_i-\alpha_j)}^{2} \in K
    \end{equation*}
\end{definicion}

\noindent
Resulta que $\Disc(f)$ se puede calcular a partir de los coeficientes del polinomio.

\begin{observacion}
    $f$ es separable $\Longleftrightarrow \Disc(f)\neq 0$. 
\end{observacion}

\begin{notacion}
    Notaremos usualmente a la raíz del discriminante $\Disc(f)$ por:
    \begin{equation*}
        \Delta(f) = \prod_{1\leq i < j \leq n}(\alpha_i-\alpha_j)
    \end{equation*}
\end{notacion}

\begin{notacion} % // TODO: Estudiar los S_n, y subgrupos de S4
    Dado un conjunto $S = \{\alpha_1, \ldots, \alpha_n\}$, denotaremos normalmente al grupo de permutaciones de dichos elementos por:
    \begin{equation*}
        \Sim(\alpha_1, \ldots, \alpha_n)
    \end{equation*}
    Observemos que $\Sim(\alpha_1,\ldots, \alpha_n)\cong S_n$.
\end{notacion}

\begin{definicion}
    Si $f\in F[x]$ es separable y $K$ es su cuerpo de descomposición, diremos que $\Aut_F(K)$ es el grupo de Galois\footnote{Observemos que por ser $f$ separable y $K$ cuerpo de descomposición suyo tenemos siempre por el Teorema~\ref{teo:piedra_angular} que la exntesión $F\leq K$ es de Galois.} de $f$.
\end{definicion}

\noindent
Si $f\in F[x]$ es separable y $K$ es su cuerpo de descomposición, si consideramos $\{\alpha_1, \ldots, \alpha_n\}$ el conjunto de todas las raíces de $f$ en $K$, podemos siempre definir un homomorfismo de grupos entre el grupo de Galois de $f$ y el grupo de permutaciones de sus raíces:
\begin{align*}
    \Aut_F(K) &\longrightarrow \Sim(\alpha_1, \ldots, \alpha_n) \\
    \sigma&\longmapsto \sigma\big|_{\{\alpha_1, \ldots, \alpha_n\}}
\end{align*}
Tenemos que:
\begin{itemize}
    \item La aplicación está bien definida, pues si consideros $\sigma\in \Aut_F(K)$, tendremos siempre que $\sigma^\ast = \sigma\big|_{\{\alpha_1, \ldots, \alpha_n\}}\in \Sim(\alpha_1, \ldots, \alpha_n)$, pues si $\alpha_i$ es una raíz de $f$ (para $i \in \{1,\ldots,n\}$) tendremos entonces que $\sigma(\alpha_i)$ también es raíz de $f$:
        \begin{equation*}
            f(\sigma(\alpha_i)) = \sum_{i=0}^n f_i {(\sigma(\alpha_i))}^{i} \AstIg \sigma\left(\sum_{i=0}^{n}f_i \alpha_i^i\right) = \sigma(0) = 0
        \end{equation*}
        donde en $(\ast)$ hemos usado que $\sigma\in \Aut_F(K)$ y que $f\in F[x]$.
    \item La aplicación es un homeomorfismo, pues si $\sigma,\tau\in \Aut_F(K)$ tenemos entonces que:
        \begin{equation*}
            (\sigma\tau)\big|_{\{\alpha_1, \ldots, \alpha_n\}} = \sigma\big|_{\{\alpha_1, \ldots, \alpha_n\}}\tau\big|_{\{\alpha_1, \ldots, \alpha_n\}}
        \end{equation*}
\end{itemize}
Además dicho homomorfismo de grupos es siempre inyectivo, pues la Proposición de Extensión nos dice que cada automorfismo del grupo de Galois queda unívocamente determinado por la imagen de cada raíz de $f$, puesto que sabemos que el grupo de Galois de $f$ coincide con las extensiones de la inclusión:
\begin{equation*}
    \Aut_F(K) = Ex(\iota,\iota)
\end{equation*}
Si pensamos en la obtención de todos los elementos del grupo de Galois de $f$ mediante el siguiente procedimiento:
\begin{figure}[H]
    \centering
    \shorthandoff{""}
    \begin{tikzcd}
        F \arrow[r, hook] \arrow[rd, hook] & K                                                             & {F(\alpha_1, ..., \alpha_{i-1})} \arrow[r, hook] \arrow[rd, hook] & K                                                                              & {F(\alpha_1, ..., \alpha_{n-1})} \arrow[r, hook] \arrow[rd, hook] & K                                                 \\
                                           & F(\alpha_1) \arrow[u, "\alpha_1 \longmapsto \eta(\alpha_1)"'] &                                                                   & {F(\alpha_1, ..., \alpha_{i})} \arrow[u, "\alpha_i\longmapsto\eta(\alpha_i)"'] &                                                                   & K \arrow[u, "\alpha_n\longmapsto\eta(\alpha_n)"']
    \end{tikzcd}
    \shorthandon{""}
\end{figure}
\noindent
observamos que cada uno de ellos queda determinado por cada una de las elecciones hechas sobre cada una de las imágenes de cada raíz. De esta forma, si tenemos que dos elementos $\sigma,\tau\in \Aut_F(K)$ coinciden en $\{\alpha_1, \ldots, \alpha_n\}$, tendremos entonces que $\sigma = \tau$, lo que nos prueba la inyectividad del homomorfismo de grupos.

\noindent
De esta forma, como $\Sim(\alpha_1, \ldots, \alpha_n)\cong S_n$, podemos ver siempre el grupo de Galois de $f$ como subgrupo de $S_n$, aquel que permuta los índices de las raíces de $f$:
\begin{equation*}
    \alpha_i \stackrel{\sigma}{\longmapsto} \alpha_{\sigma(i)}
\end{equation*}

\begin{notacion}
    En vista de la relación existente entre $\Aut_F(K)$ (el grupo de Galois de cierto polinomio $f\in F[x]$), $\Sim(\alpha_1, \ldots, \alpha_n)$ (el grupo de permutaciones sobre sus raíces) y $S_n$, será habitual identificar $\Sim(\alpha_1, \ldots, \alpha_n)$ con $S_n$, y ver $\Aut_F(K)$ directamente como subgrupo de $S_n$. Este uso de la notación no debe llevar a errores, pues simplemente es una forma más rápida de enunciar ciertas propiedades sobre $\Aut_F(K)$.
\end{notacion}

\begin{observacion}
    Si tomamos $\sigma\in \Aut_F(K)$, una vez visto que $\sigma$ actuando sobre las raíces del polinomio $f$ simplemente las permuta, vemos fácilmente que:
    \begin{itemize}
        \item $\sigma(\Disc(f)) = \Disc(f)$.
        \item $\sigma(\Delta(f)) = sgn(\sigma)\Delta(f)$.
    \end{itemize}
\end{observacion}

\begin{prop}\label{prop:discf}
    Sea $f\in F[x]$ separable con grupo de Galois $G = \Aut_F(K)$. Entonces $\Disc(f) \in F$. Además:
    \begin{equation*}
        K^{G\cap A_n} = F(\Delta(f))
    \end{equation*}
    Por tanto, $\Delta(f) \in F \Longleftrightarrow G\leq A_n$.
    \begin{proof}
        Para ver que $\Disc(f)\in F$, vimos en el primer punto de la observación superior que:
        \begin{equation*}
            \sigma(\Disc(f)) = \Disc(f) \qquad \forall \sigma\in G
        \end{equation*}
        Por lo que tenemos que $\Disc(f) \in K^G$, pero como $F\leq K$ es de Galois, tenemos que $K^G = F$.\\

        \noindent
        Para ver que $K^{G\cap A_n} = F(\Delta(f))$, en vista del segundo punto de la observación superior:
        \begin{equation*}
            \sigma(\Delta(f)) = sgn(\sigma)\Delta(f) \qquad \forall \sigma\in G
        \end{equation*}
        Tenemos que $\Delta(f) \in K^{G\cap A_n}$, y como todo elementos de $G$ es $F-$lineal es claro que $F(\Delta(f)) \leq K^{G\cap A_n}$. Si estudiamos el índice de este subcuerpo de $K$, la conexión de Galois nos dice que:
        \begin{equation*}
            \left[K^{G\cap A_n} : F\right] = (G : G\cap A_n) \stackrel{(\ast)}{\leq} (S_n : A_n) = 2
        \end{equation*}
        donde en $(\ast)$ hemos usado el Segundo Teorema de Isomorfía para grupos. Por tanto, solo tnemos dos situaciones posibles:
        \begin{equation*}
            F(\Delta(f)) = F \qquad \text{o}\qquad F(\Delta(f)) = K^{G\cap A_n}
        \end{equation*}
        \begin{itemize}
            \item Si $F(\Delta(f)) = F$, tendremos entonces que $\Delta(f)\in F$, así como que:
                \begin{equation*}
                    sgn(\sigma)\Delta(f) = \sigma(\Delta(f)) = \Delta(f)\sigma(1) = \Delta(f)\qquad \forall \sigma\in G
                \end{equation*}
                Por lo que $G\leq A_n$, de donde:
                \begin{equation*}
                    K^{G\cap A_n} =K^G = F = F(\Delta(f))
                \end{equation*}
            \item Si $F(\Delta(f)) = K^{G\cap A_n}$, tendremos entonces que $\Delta(f)\notin F$, por lo que:
                \begin{equation*}
                    sgn(\sigma)\Delta(f) = \sigma(\Delta(f)) \neq \Delta(f)  \qquad \forall \sigma\in G
                \end{equation*}
                Por lo que $sgn(\sigma) = -1$, de donde $G\not\leq A_n$.
        \end{itemize}
    \end{proof}
\end{prop}

\noindent
En relación al enunciado de la Proposición anterior, se suele hacer referencia a la condición ``$\Delta(f)\in F$''  por ``$\Disc(f)$ es un cuadrado en $F$''.

\begin{ejercicio} % // TODO: HACER
    Sea $f\in \mathbb{R}[x]$ con $degf = 3$, discutir el número de raíces reales de $f$ según el signo de $\Disc(f)$. % Pasar al caso monico
\end{ejercicio}

\begin{ejemplo}
    Consideramos $f = x^n +\sum\limits_{i=0}^{n-1}a_ix^i \in F[x]$ y sean $\alpha_1, \ldots, \alpha_n$ sus raíces (repetidas según multiplicidad), tenemos que:
    \begin{equation*}
        f = \prod_{i=1}^{n}(x-\alpha_i)
    \end{equation*}
    Igualando coeficientes de igual grado, obtenemos las relaciones de Cardano-Vieta\footnote{Hay una teoría desarrollada sobre esto, siempre se obtienen funciones simétricas en las raíces del polinomio.}. Por ejemplo, si $n=2$ se obtiene:
    \begin{equation*}
        a_0 = \alpha_1\alpha_2 \qquad a_1 = -(\alpha_1+\alpha_2)
    \end{equation*}
    Como $\Disc(f) = {(\alpha_1-\alpha_2)}^{2}$, tenemos que $\Disc(f) = a_1^2 - 4a_0$. \newline
    Para $n>2$, la cuenta no es tan sencilla, por lo que se prefiere usar un algoritmo para resolver el sistema de ecuaciones. Por tanto, se puede expresar $\Disc(f)$ en término de los coeficientes de $f$. Para $n=3$, la damos para $f=x^3+px+q$  (cúbica reducida\footnote{Sin término cuadrático.}) es:
    \begin{equation*}
        \Disc(f) = -4p^3 - 27q^2
    \end{equation*}
\end{ejemplo}

\begin{prop}
    Sea $f\in F[x]$ separable con grupo de Galois $G$
    \begin{equation*}
        f\text{\ es irreducible} \Longleftrightarrow G\text{\ actúa transitivamente sobre las raíces de\ } f
    \end{equation*}
    En tal caso, $degf$ divide a $|G|$.
    \begin{proof}
        Sea $K$ el cuerpo de descomposición de $f$, tenemos que $G = \Aut_F(K)$.
        \begin{description}
            \item [$\Longrightarrow )$] Si $f$ es irreducible y $\alpha,\beta\in K$ son raíces de $f$, podemos ($f = \Irr(\alpha,F)$) usar la Proposición de extensión, obteniendo $\sigma:F(\alpha)\to K$ de forma que $\sigma(\alpha) = \beta$.

                La tercera proposición de extensión nos dice que $\sigma$ se extiende a un automorfismo $\eta\in G$ y $\eta(\alpha) = \sigma(\alpha) = \beta$, por lo que la acción es transitiva.
            \item [$\Longleftarrow )$] Sea $g$ un factor irreducible de $f$ (ambos mónicos), tenemos que $g$ no es constante, con lo que sus raíces son también de $f$. Además, $\sigma(\alpha)$ es raíz de $g$, para todo $\sigma\in G$, y como $G$ actúa transitivamente sobre las raíces de $f$; toda raíz de $f$ es de $g$, con lo que $f = g$, de donde $f$ es irreducible.
        \end{description}
        Finalmente, para ver que $degf$ divide a $|G|$, si $\alpha$ es raíz de $f$, tenemos entonces $[F(\alpha):F] = degf$, que divide a $[K:F]$ por el Lema de la Torre, y $|G| = [K:F]$.
    \end{proof}
\end{prop}

\begin{coro}
    Por tanto, a la hora de buscar el grupo de Galois de un polinomio irreducible, descartaremos automáticamente los subgrupos de $S_n$ no transitivos.
\end{coro}

\begin{ejemplo}
    Sea $f\in F[x]$ separable e irreducible: 
    \begin{enumerate}
        \item Si $degf = 1$, su grupo de Galois es la identidad, como único elemento de $S_1$.
        \item Si $degf = 2$, el cuerpo de Galois de $f$ tiene grado 1 o 2. Si $f$ es irreducible, ha de ser de grado 2, con lo que su grupo de Galois es isomorfo a $C_2$ (observemos que $S_2\cong C_2$).
        \item Si $degf = 3$, la Proposición anterior nos dice que bien $G\cong A_3$ o $G\cong S_3$. La Proposición~\ref{prop:discf} nos dice que tenemos el primer caso si $\Delta(f)\in F$ y el segundo si $\Delta(f)\notin F$.
        \item Si $degf = 4$, la Proposición anterior nos dice que $G$ es isomorfo a un subgrupo transitivo de $S_4$. % // TODO: Aprender la lista
    \end{enumerate}
\end{ejemplo}

\begin{ejemplo}
    Sea $f\in F[x]$ polinomio separable e irreducible de grado $degf = 4$, sean $\alpha_1,\alpha_2,\alpha_3,\alpha_4$ las raíces de $f$ en un cuerpo de descomposición $K$ de $f$, consideramos:
    \begin{align*}
        \beta_1 = \alpha_1\alpha_2 + \alpha_3 \alpha_4 \\
        \beta_2 = \alpha_1\alpha_3 + \alpha_2 \alpha_4 \\
        \beta_3 = \alpha_1\alpha_4 + \alpha_2 \alpha_3 
    \end{align*}

    y definimos:
    \begin{equation*}
        g = (x-\beta_1)(x-\beta_2)(x-\beta_3)\in K[x]
    \end{equation*}
    Veamos que en realidad $g\in F[x]$. Para ello, como $F\leq K$ es de Galois, hemos de ver que el polinomio es fijo por todos los automorfismos del grupo de Galois de $f$ (basta verlo para todas las permutaciones). Concluimos que $g^{\sigma} = g\quad \forall \sigma\in G$, con lo que $g$ es una resolvente cúbica de $f$ (se verá).

    \noindent
    Se puede ver por el algoritmo mencionado anteriormente que si $f=x^4+bx^3+cx^2+dx+e$, entonces:
    \begin{equation*}
        g = x^3-cx^2+(bd-4e)x-b^2e+4ce-d^2
    \end{equation*}~\\

    \noindent
    Consultamos si sus raíces son distintas:
    \begin{equation*}
        \beta_2 - \beta_1 = (\alpha_2 - \alpha_3)(\alpha_4-\alpha_1)
    \end{equation*}
    Por lo que $\beta_2$ y $\beta_1$ son distintas (análogo para el resto de las parejas), con lo que $g$ es separable, luego $E= F(\beta_1,\beta_2,\beta_3)$ es una extensión de Galois de $F$, con $F\leq E \leq K$, de donde el grupo de Galois de $g$, $N = \Aut_E(K)$ es normal en $G$. Por lo que:
    \begin{equation*}
        \Aut_F(E)\cong \frac{G}{N}
    \end{equation*}~\\

    \noindent
    Consideramos $S:\Sim(\alpha_1,\alpha_2,\alpha_3,\alpha_4)\to\Sim(\beta_1,\beta_2,\beta_3)$ una aplicación de forma que:
    \begin{equation*}
        S(\sigma)(\alpha_i\alpha_j + \alpha_k\alpha_l) = \alpha_{\sigma(i)}\alpha_{\sigma(j)} + \alpha_{\sigma(k)} \alpha_{\sigma(l)}
    \end{equation*}
    que es un homomorfimo de grupos y es sobreyectivo (ya que dada una trasposición en el grupo de la derecha, podemos encontrar un elemento en la izquierda cuya imagen vaya a él). Calculamos su núcleo:
    \begin{equation*}
        \ker S = \{(1), (1\ 2)(3\ 4), (1\ 3)(2\ 4), (1\ 4)(2\ 3)\}
    \end{equation*}
    Y sabemos que son todas porque como el grupo de la derecha tiene $6$ elementos y el de la derecha 24; con lo que $\ker S = V$.\\

    \begin{figure}[H]
        \centering
        \shorthandoff{""}
        \begin{tikzcd}
            1 \arrow[r] & V \arrow[r] & {Sim(\alpha_1,\alpha_2,\alpha_3,\alpha_4)} \arrow[rr, "S"]                  &  & {Sim(\beta_1,\beta_2,\beta_3)} \arrow[r] & 1 \\
            1 \arrow[r] & N \arrow[r] & G \arrow[u] \arrow[rr, "r"'] \arrow[rru, bend left] \arrow[rru, bend right] &  & Aut_F(E) \arrow[r] \arrow[u]             & 1
        \end{tikzcd}
        \shorthandon{""}
    \end{figure}
    \noindent
    Por lo que:
    \begin{equation*}
        N = V\cap G
    \end{equation*}
\end{ejemplo}

\begin{ejemplo}
    Si tenemos $f=x^4+x+1\in \mathbb{Q}[x]$, no tiene raíces en $\mathbb{Q}$ (las únicas posibles son $-1$ y $1$). Como $f\in \mathbb{Z}[x]$ y $f$ es primitivo, reducimos módulo 2, obteniendo:
    \begin{equation*}
        \overline{f} = x^4+x+1 \in \mathbb{Z}_2[x]
    \end{equation*}
    $\overline{f}$ no tiene raíces y si fuera irreducible, tendríamos entonces que es producto del único polinomio irreducible de $\mathbb{Z}_2[x]$ que es $x^2+x+1$, pero no lo es, por lo que $f$ es irreducible sobre $\mathbb{Z}[x]$, luego sobre $\mathbb{Q}[x]$ también. Sea $G$ el grupo de Galois de $f$ sobre $\mathbb{Q}$, tenemos que $G$ es un subgrupo transitivo de $S_4$, así como que $|G|$ es un múltiplo de $degf = 4$. Los transitivos de $S_4$ son:
    \begin{itemize}
        \item Los cíclicos de 4 elementos.
        \item Un Klein, de entre los 3 isomorfos a Klein uno es transitivo y dos no.

            Como ejemplo de esto:
            \begin{equation*}
                V = \{(1\ 2)(3\ 4), (1\ 3)(2\ 4), (1\ 4)(2\ 3), 1\}
            \end{equation*}
            es transitivo pero:
            \begin{equation*}
                \{(1), (1\ 2), (3\ 4), (1\ 2)(3\ 4)\}
            \end{equation*}
            es isomorfo a Klein pero no es transitivo (desde 1 no podemos llegar a 3).
        \item De 8 elementos tenemos los diédricos, que hay varios.
        \item $A_4$.
    \end{itemize}
    Anteriormente vimos que si $f = x^4+bx^3+cx^2+dx+e$ entonces su resolvente tenía el aspecto:
    \begin{equation*}
        g = x^3-cx^2+(bd-4e)x-b^2+4ce-d^2
    \end{equation*}
    Para nuestro $f$ la resolvente cúbica es: 
    \begin{equation*}
        g = x^3-4x -1
    \end{equation*}
    Si $\alpha_1,\alpha_2,\alpha_3,\alpha_4$ son raíces de $f$ y $\beta_1,\beta_2,\beta_3$ son las de $g$, teníamos entonces que:
    \begin{equation*}
        \beta_2-\beta_1 = (\alpha_2-\alpha_3)(\alpha_4-\alpha_1)
    \end{equation*}
    más otras dos relaciones. Usando estas, se demuestra que $\Disc(f) = \Disc(g)$. % // TODO: PROBAR
    Además, $g$ es una cúbica reducida, y teníamos una fórmula para calcular $\Disc(g)$, obtniendo que:
    \begin{equation*}
        \Disc(f) = \Disc(g) = 229
    \end{equation*}
    Y tenemos que $\sqrt{229}\notin \mathbb{Q}$, ya que esto sucede si $x^2-229\in \mathbb{Q}[x]$ es irreducible, porque $229$ es primo (se comprueba tratando de dividir entre primos hasta la parte entera de $\sqrt{229}$, que es 15). Como $\sqrt{229}\notin \mathbb{Q}$, tenemos que $G\not\subseteq A_4$, por lo que no puede ser el isomorfo a Klein ni $A_4$.\\

    \noindent
    En estas condiciones, teníamos una subextensión:
    \begin{equation*}
        \mathbb{Q}\leq E =\mathbb{Q}(\beta_1,\beta_2,\beta_3) \leq K = \mathbb{Q}(\alpha_1, \alpha_2,\alpha_3,\alpha_4)
    \end{equation*}
    Como $\mathbb{Q}\leq K$ es de Galois, tenemos que $E\leq K$ es de Galois, y la conexión nos dice que:
    \begin{equation*}
        \Aut_E(K) \lhd G
    \end{equation*}
    Veamos qué aspecto tiene $\Aut_E(K)$, para reducir las opciones sobre $G$, de hecho:
    \begin{equation*}
        \dfrac{G}{\Aut_E(K)} \cong \Aut_\mathbb{Q}(E)
    \end{equation*}
    Como $g$ no tiene raíces (ya que las únicas posibles raíces son $\pm 1$) y es de grado 3 tenemos que $g$ es irreducible, por lo que $|\Aut_\mathbb{Q}(E)|$ es múltiplo de $degg = 3$, con lo que solo puede ser 3 o 6. Como el único posible grupo $G$ que es divisible entre $3$ es la opción $G\cong S_4$. Buscamos ahora $\Aut_E(K)$, que ha de ser $V$.
\end{ejemplo}

\noindent
Respecto al tema anterior ganamos que no es necesario calcular de forma explícita cada uno de los automorfismos.

\section{Extensiones ciclotómicas}
\noindent
Para $n\geq 1$, a lo largo de este capítulo nos interesará el polinomio $x^n-1\in F[x]$, para $F\leq K$ cualquier extensión.
\begin{prop}
    Si $n\geq 1$ y consideramos como $S$ el conjunto de todas las raíces de $x^n-1\in F[x]$ en $K$ para $F\leq K$ cualquier extensión, tenemos que $S$ es un subgrupo cíclico de $K^\times$ cuyo orden es un divisor de $n$.
    \begin{proof}
        Sean $\alpha,\beta\in S$, tenemos que:
        \begin{equation*}
            \alpha^n - 1 = 0 = \beta^n -1 \quad \Longrightarrow \quad  \alpha^n = 1 = \beta^n
        \end{equation*}
        Y observamos ahora que:
        \begin{equation*}
            {(\alpha\beta^{-1})}^{n} -1  = (\alpha^n \beta^{-n})-1 = (1\cdot 1) - 1 = 0
        \end{equation*}
        Por lo que $\alpha\beta^{-1}\in S$, de donde $S$ es un subgrupo de $K^\times$. Sabemos que $S$ es un subgrupo cíclico de $K^\times$ por el Ejercicio~\ref{ej:subgrupo_finito}. Además, su orden ha de dividir a $n$, pues todos los elementos tienen orden un divisor de $n$: $\alpha^n = 1 \quad \forall \alpha\in S$.
    \end{proof}
\end{prop} 

\noindent
El subgrupo cíclico de las raíces será de orden $n$ si $K$ contiene un cuerpo de descomposición de $x^n-1$ y si $x^n-1$ es separable, es decir, si $n$ no es múltiplo de $\car(F)$. \newline
En este contexto, llamamos a las raíes de $x^n-1$ \underline{raíces $n-$ésimas de la unidad}, que es un grupo cíclico de orden $n$ si $x^n-1$ es separable, por lo que lo supondremos en general. A sus generadores los llamamos \underline{raíces $n-$ésimas primitivas de la unidad}.

\begin{ejemplo}
    En característica cero, podemos suponer sin pérdida de generalidad que $F=\mathbb{Q}$, por lo que obtenemos las conocidas raíces $n-$ésimas primitivas de la unidad en $\mathbb{C}$.
\end{ejemplo}

\noindent
En Álgebra I se vió que\footnote{Donde $\varphi$ es la función de Euler.} $|\cc{U}(\mathbb{Z}_n)| = \varphi(n)$, sea $\zeta$ una raíz $n-$ésima primitiva de la unidad, tenemos que el conjunto de todas las raíces $n-$ésimas de la unidad es:
\begin{equation*}
    \{\zeta^k : k\in \mathbb{Z}_n\}
\end{equation*}
Y las raíces $n-$ésimas primitivas de la unidad son:
\begin{equation*}
    \{\zeta^k : k\in \cc{U}(\mathbb{Z}_n)\}
\end{equation*}
De esta forma, el cuerpo de descomposición de $x^n-1\in F[x]$ es $F(\zeta)$, que recibe el nombre \underline{$n-$ésima extensión ciclotómica de $F$}, y como mucho es:
\begin{equation*}
    [F(\zeta):F] \leq n-1
\end{equation*}
Puesto que $x^n-1$ siempre tiene a 1 como raíz.

\begin{prop}
    Sea $x^n-1\in F[x]$ separable, tiene grupo de Galois $G$ isomorfo a un subgrupo de $\cc{U}(\mathbb{Z}_n)$. Además, $G$ es isomorfo a $\cc{U}(\mathbb{Z}_n)$ si y solo si actúa transitivamente sobre las raíces $n-$ésimas primitivas de la unidad (sobre $F$).
    \begin{proof}
        Sea $\zeta$ una raíz primitiva de la unidad, tenemos que:
        \begin{equation*}
            G = \Aut_F(F(\zeta))
        \end{equation*}
            donde $F(\zeta)$ es la $n-$ésima extensión ciclotómica de $F$. Habíamos visto que las raíces $n-$ésimas primitivas de la unidad son:
            \begin{equation*}
                \{\zeta^k : k\in \cc{U}(\mathbb{Z}_n)\}            
            \end{equation*}
            Sea $\sigma\in G$, tenemos que $\sigma(\zeta) = \zeta^k$ para cierto $k\in \cc{U}(\mathbb{Z}_n)$. Podemos definir 
            \Func{}{G}{\cc{U}(\bb{Z}_n)}{\sigma}{k}
            donde se cumple que:
            \begin{equation*}
                \sigma(\zeta) = \zeta^k
            \end{equation*}
            Si tomamos $\sigma,\tau\in G$ de forma que $\tau(\zeta) = \zeta^l$ con $l\in \cc{U}(\mathbb{Z}_n)$, tenemos que:
            \begin{equation*}
                \sigma\tau(\zeta) = \sigma(\zeta^l) = \sigma(\zeta)^l = {(\zeta^k)}^{l} = \eta^{kl}
            \end{equation*}
            Con lo que la aplicación considerada es un homomorfismo de grupos, que además es inyectivo por su definición. Tenemos pues que $G$ es isomorfo a cierto subgrupo de $\cc{U}(\mathbb{Z}_n)$. Esta aplicación será sobreyectiva si para cada exponente tenemos un automorfismo, con lo que $G$ actuará transitivamente sobre estas raíces.
    \end{proof}
\end{prop}

\begin{definicion}[Polinomio ciclotómico]
    Sea $F$ un cuerpo, definimos el $n-$ésimo polinomio ciclotómico como:
    \begin{equation*}
        \phi_n = \prod_{k\in \cc{U}(\mathbb{Z}_n)} \left(x-\zeta^k\right)
    \end{equation*}
    con $\zeta$ una raíz $n-$ésima primitiva de la unidad.
\end{definicion}

\begin{prop}
    Se tiene que:
    \begin{equation*}
        x^n-1 = \prod_{d \in Div(n)}\phi_d
    \end{equation*}
    \begin{proof}
        Consideramos como $R_n$ el conjunto de todas las raíces de $x^n-1$ (es decir, el conjunto de todas las raíces $n-$ésimas de la unidad). Si consideramos también $P_m$, el conjunto de las raíces $m-$ésimas primitivas de la unidad, tenemos una partición de $R_n$:
        \begin{equation*}
            R_n = \biguplus_{d\in Div(n)}P_d
        \end{equation*}
        Como:
        \begin{equation*}
            x^n - 1 = \prod_{\alpha \in R_n}(x-\alpha)
        \end{equation*}
        y cada $\alpha$ está en un cierto $P_d$, ha de estar en $\phi_d$, con lo que:
        \begin{equation*}
            x^n - 1 = \prod_{\alpha \in R_n}(x-\alpha) = \prod_{d\in Div(n)}\phi_d
        \end{equation*}
    \end{proof}
\end{prop}

\begin{prop}
    Cada $\phi_n$ tiene coeficientes en el subcuerpo primo de $F$ y además, si $\car(F) = 0$, tenemos que $\phi_n\in \mathbb{Z}[x]$.
    \begin{proof}
        Por inducción sobre $n$:
        \begin{itemize}
            \item Si $n=1$, entonces $\phi_1 = x-1$ y se tiene la Proposición.
            \item Si $n>1$, tenemos:
                \begin{equation*}
                    x^n-1 = \phi_n \cdot \prod_{\substack{d \in Div(n)\\d<n}}\phi_d
                \end{equation*}
                Por hipótesis de inducción, tenemos que el producto de la derecha está en el subcuerpo primo de $F$. Tenemos además que $\phi_n$ es cociente de $x^n-1$ entre el producto de la derecha, con lo que $\phi_n$ también ha de tener coeficientes en el subcuerpo primo de $F$.
        \end{itemize}
        Si $\car(F) = 0$, sabemos por lo ya probado que $\phi_n\in \mathbb{Q}[x]$. Si expresamos sus coeficientes como fracciones irreducibles y tomamos $a\in \mathbb{Z}$ el mínimo común múltiplo de sus denominadores, tenemos que $a\phi_n \in \mathbb{Z}[x]$, con todos sus coeficientes coprimos entre sí, luego $a\phi_n$  es primitivo. Tenemos pues que:
        \begin{equation*}
            a(x^n-1) = a\phi_n \prod_{\substack{d\in Div(n)\\d<n}}\phi_d
        \end{equation*}
        de nuevo por inducción, suponemos ahora que los polinomios $\phi_d$ son primitivos (para $n=1$ es claro que $\phi_1$ es primitivo). Por el Lema de Gauss, tenemos que:
        \begin{equation*}
            \prod_{\substack{d\in Div(n)\\d<n}}\phi_d
        \end{equation*}
        es primitivo y con coeficientes en $\mathbb{Z}$, si recordamos que $a\phi_n$ es primitivo, de donde todo el producto es primitivo, es decir, $a(x^n-1)$ es primitivo, luego ha de ser $a = 1$.
    \end{proof}
\end{prop}

\begin{ejemplo}
    En característica cero: 
    \begin{equation*}
        \phi_1 = x-1,\qquad \phi_2 = x+1, \qquad \phi_3 = x^2+x+1, \qquad \phi_4 = x^2+1
    \end{equation*}
    Para $\phi_6$ usamos la fórmula:
    \begin{equation*}
        \phi_6 = \frac{x^6-1}{\phi_1\phi_2\phi_3} = x^2-x+1
    \end{equation*}
\end{ejemplo}

\begin{teo}
    Cada $\phi_n\in \mathbb{Z}[x]$ es irreducible.
    \begin{proof}
        Sea $f$ un factor irreducible de $\phi_n$. Tomamos $\zeta$ una raíz compleja de $f$ en la $n-$ésima extensión ciclotómica. Probemos que si $p$ es primo y no divide a $n$, entonces $\zeta^p$ es raíz de $f$. Por reducción al absurdo, tomamos $g$ con:
        \begin{equation*}
            \phi_n = fg\qquad f,g\in \mathbb{Z}[x]
        \end{equation*}
        Y ha de cumplir $g(\zeta^p) = 0$. Resulta que $\zeta$ es raíz de $h=g(x^p)\in \mathbb{Z}[x]$. De esta forma, $f$ y $h$ tienen una raíz común compleja, y la identidad de Bezout nos dice entonces que $f$ y $h$ no son coprimos. Como $f$ es irreducible, ha de ser $f\mid h$. Como $f$ es primitivo, tenemos que $f\mid h$ en $\mathbb{Z}[x]$. Reducimos módulo $p$: 
        \begin{equation*}
            \overline{\phi_n} = \overline{f}\overline{g}
        \end{equation*}
        y tenemos:
        \begin{equation*}
            \overline{h} = \overline{g(x^p)} \AstIg {\overline{g(x)}}^{p} = \overline{g}^p
        \end{equation*}
        donde $(\ast)$ es porque % // TODO: HACER, a^p = a para a en F_p
        como $f\mid h$  en $\mathbb{Z}[x]$, tenemos que $\overline{f}\mid \overline{h}$, pero como $\overline{h} = \overline{g}^p$. Como $\bb{F}_p[x]$ es DFU, tenemos que $\overline{f}$ y $\overline{g}$ tienen algún factor común no constante. Deducimos que $\overline{x^n-1}$ tiene una raíz múltiple en su cuerpo de descomposición. Sin embargo, $p$ no divide a $n$, por lo que:
        \begin{equation*}
            \overline{x^n-1} = x^n-1
        \end{equation*}
        y este polinomio es separable, \underline{contradicción}; por lo que para cada $p$ primo que no divide a $n$ tenemos que $\zeta^p$ es raíz de $f$. Si tomamos $n$ y lo factorizamos en primos, miramos los primos que no dividen a $n$, con lo que tomando estos primos nos movemos dentro de las unidades de $\mathbb{Z}_n$, con lo que en alguna cantidad finita de pasos rellenamos todas las unidades de $\mathbb{Z}_n$. Por lo que todas las raíces de $\phi_n$ son raíces de $f$. Como $f$ dividía a $\phi_n$, $f = \phi_n$.
    \end{proof}
\end{teo}~\\

\noindent
Hemos visto ya que el $n-$ésimo polinomio ciclotómico $\phi_n\in \mathbb{Z}[x]$ es irreducible. $\mathbb{Q}(\zeta)$, la $n-$ésima extensión ciclotómica con $\zeta\in \mathbb{C}$ la raíz $n-$ésima primitiva de la unidad. Así, $\Irr(\zeta,\mathbb{Q}) = \phi_n$. Sabemos que $\Aut(\mathbb{Q}(\zeta))$ actúa transitivamente sobre las raíces de $\phi_n$.\\

\noindent
Esto significa que $|\Aut(\mathbb{Q}(\zeta))| = [\mathbb{Q}(\zeta):\mathbb{Q}] = deg \phi_n = \varphi(n)$. Además, tenemos que $\Aut(\mathbb{Q}(\zeta))\cong \cc{U}(\mathbb{Z}_n)$.

\begin{ejemplo} % // TODO: HACER
    Para $n=16$: 
    \begin{equation*}
        \deg \phi_n = \varphi(16) = 8
    \end{equation*}
    Por lo que tenemos que:
    \begin{equation*}
        \Aut(\mathbb{Q}(\zeta)) \cong \cc{U}(\mathbb{Z}_{16})
    \end{equation*}
\end{ejemplo}

\section{Construcciones con regla y compás II}
\noindent
A lo largo de la sección, consideraremos siempre que $S$ es un subconjunto de $\mathbb{C}$ con $\{0,1\}\subseteq S$. Además, consideraremos siempre también $F = \mathbb{Q}(S\cup \overline{S})$.

\begin{teo}
    Sea $z\in \mathbb{C}$, tenemos que:
    \begin{center} % // TODO: Arreglar enunciado
        $z$ es constructible a partir de $S \Longleftrightarrow z\in K$, donde $F\leq K$ es de Galois y $[K:F]=2^k$, para cierto $k\in \mathbb{N}$.
    \end{center}
    \begin{proof}
        Por doble implicación:
        \begin{description}
            \item [$\Longrightarrow )$] Sé que existe una torre de subcuerpos de $\mathbb{C}$:
                \begin{equation*}
                    F=F_0\leq F_1\leq \ldots \leq F_s
                \end{equation*}
                tales que $F_{i+1} = F(\alpha_{\alpha_i+1})$, con $\alpha_{i+1}^2 \in F_i$, para todo $i \in \{0,\ldots,s-1\}$; con $z\in F_s$. Por inducción sobre $s$:
                \begin{itemize}
                    \item Para $s=0$, tenemos que $F_s = F_0$, por lo que tomamos $K = F_0$, que trivialmente es de Galois.
                    \item Supongamos como hipótesis de inducción que existe una extensión de Galois $F\leq E$ con grado una potencia de $2$ y tal que $F_{s-1}\leq E$. Llamamos $a_s = \alpha_s^2\in F_{s-1}$, y enumeramos los elementos:
                        \begin{equation*}
                            \Aut_F(E) = \{\sigma_0, \ldots, \sigma_t\}
                        \end{equation*}
                        Y definimos el polinomio:
                        \begin{equation*}
                            f = \prod_{j=0}^{t} (x^2-\sigma_j(a_s))
                        \end{equation*}
                        Que resulta ser un polinomio con coeficientes en $E$, pero queda fijo por cualquier automorfismo de la extensión de Galois, por lo que en realidad tenemos que $f\in F[x]$.

                        \noindent
                        Como $F\leq E$ es de Galois, tenemos que $E$ es cuerpo de descomposición de cierto $g\in F[x]$. Tomamos como $K$ el cuerpo de descomposición de $fg$, por lo que $F\leq K$ es de Galois. Definimos $\alpha_{s+j}$ como la raíz de $x^2-\sigma_j(a_s)$, para cada $j \in \{0, \ldots, t\}$, por lo que $\alpha_{s+j}\in K$.

                        \noindent
                        Tenemos que:
                        \begin{equation*}
                            K = E(\alpha_s, \alpha_{s+1}, \ldots,\alpha_{s+t})
                        \end{equation*}
                        Puesto que las raíces de $g$ ya están en $E$. Como el grado de $\alpha_{s+j}$ es 1 o 2 (al ser raíz de $x^2-\alpha_j(a_s)$), tenemos que $F\leq K$ tiene grado una potencia de 2. Ahora, como $F_s\leq K$, tenemos que $z\in F_s\leq K$, para completar la inducción.
                \end{itemize}
            \item [$\Longleftarrow )$] Tomamos $z\in K$ con $F\leq K$ de Galois y $[K:F]$ una potencia de $2$. Tenemos por tanto que $\Aut_F(K)$ es un $2-$grupo, luego es resoluble\footnote{Por ser un $p-$grupo.}. Podemos por tanto tomar una serie de composición suya, obteniendo:
                \begin{equation*}
                    \Aut_F(K) = G_0 \geq G_1 \geq \ldots \geq G_n = \{id\}
                \end{equation*}
                con factores de composición $2$. La conexión de Galois nos transforma esta cadena en una cadena de extensiones de subcuerpos cuadráticas:
                \begin{equation}\label{eq:torre}
                    F=K_0 \leq K_1 \leq \ldots \leq K_n = K
                \end{equation}
                con $[K_{i+1}:K_i] = 2$, para cada $i \in \{0, \ldots, n-1\}$, por lo que:
                \begin{equation*}
                    K_{i+1} = K_i(\beta_i)\qquad \text{con}\qquad \beta_i = \frac{-b_i\pm \sqrt{b_i^2 - 4c_i}}{2}
                \end{equation*}
                en el caso de que $\Irr(\beta_i,K_i) = x^2+b_ix+c_i$. De esta forma:
                \begin{equation*}
                    K_{i+1} = K_i\left(\sqrt{b_i^2 - 4c_i}\right)
                \end{equation*}
                Por tanto, tenemos que~(\ref{eq:torre}) es una torre por raíces cuadradas, con lo que $z$ es constructible a partir de $S$.
        \end{description}
    \end{proof}
\end{teo}

Sabemos que el heptágono no es constructible, puesto que $\Irr(\sqrt[7]{algo},\mathbb{Q})$ es de grado 6, al ser $\varphi(7) = 6$, que no es una potencia de 2.

\begin{coro}
    Un polígono regular de $n$ lados es constructible (con regla y compás) si y solo si $\varphi(n)$ es una potencia de 2.
    \begin{proof}
        Decir que un polígono regular de $n$ lados es constructible es equivalente a decir que una raíz primitiva $n-$ésima de la unidad es constructible. Por tanto, hemos de ver que $\zeta$ es constructible (como raíz primitiva $n-$ésima de la unidad) si y solo si existe $\mathbb{Q}\leq K$ de forma que $[K:\mathbb{Q}]$ es una potencia de 2.\\

        \noindent
        Sabemos que $[\mathbb{Q}(\zeta):\mathbb{Q}] = \varphi(n)$.
        \begin{description}
            \item [$\Longrightarrow )$] Si $\zeta$ es constructible, existe una extensión de Galois $\mathbb{Q}\leq K$ de grado 2 que contiene a $\zeta$, luego ha de contener a $\mathbb{Q}(\zeta)$: $\mathbb{Q}(\zeta)\leq K$, de donde por el Lema de la Torre ha de ser $\varphi(n)$ una potencia de 2.
            \item [$\Longleftarrow )$] Si tomamos $K = \mathbb{Q}(\zeta)$ se tiene.
        \end{description}
    \end{proof}
\end{coro}

De esta forma:
\begin{table}
\centering
\begin{tabular}{c|c}
    $n$ & Es constructible \\
    \hline
    3 & Sí \\
    4 & Sí \\
    5 & Sí \\
    6 & Sí \\
    7 & No \\
    8 & Sí \\
    9 & No \\
    10 & Sí \\
    11 & No \\
    12 & No \\
    13 & No \\
    14 & No \\
    15 & Sí \\
    16 & Sí \\
    17 & Sí \\
\end{tabular}
\caption{Qué polígonos regulares son constructibles.}
\end{table}

\noindent
Si $n$ es primo, tenemos que $\varphi(n) = n-1$, que es una potencia de 2 si y solo si el primo es de la forma $2^{algo}+1$. Haciendo la cuenta, tiene que ser:
\begin{equation*}
    n = 2^{2^m} + 1
\end{equation*}
\begin{itemize}
    \item $m=0$, 3
    \item $m=1$, 5
    \item $m=2$, 17
    \item $m=3$, algo
    \item $m=4$, 65 mil y pico
\end{itemize}
Sin embargo, todavía no se ha encontrado un primo más de esta forma, los llamados primos de Fermat

\section{Extensiones cíclicas}
\begin{teo}
    Sea $x^n-a\in F[x]$ separable siendo $K$ su cuerpo de descomposición. Entonces $K$ contiene una raíz $n-$ésima primitiva de la unidad $\zeta$ y $K = F(\zeta, r)$ para cualquier raíz $r$ de $x^n-a$.

    \noindent
    Además, el grupo de Galois de la extensión $F(\zeta)\leq K$ es cíclico de orden un divisor de $n$.
    \begin{proof}
        Si $a = 0$, si $x^n$ es separable ha de ser $n = 1$, con lo que $K = F$ y una raíz $n-$ésima primitiva de la unidad es 1, se trivializa el enunciado. Suponemos por tanto que $a\neq 0$, con lo que $n$ puede ser cualquiera distinto de $\car(F)$. Sea $R$ el conjunto de las raíces en $K$ de $x^n-a$, tenemos que $|R| = n$, puesto que $x^n-a$ es separable. Además, si $r,s\in R$, tenemos que $r^{-1}s\in K$ es una raíz $n-$ésima de la unidad. 

        \noindent
        Fijado $r$, entonces el conjunto $\{r^{-1}s : s\in R\}$ contiene $n$ raíces $n-$ésimas de la unidad, por lo que en dicho conjunto las tenemos todas, luego ha de contener al menos una raíz primitiva de la unidad, llamémosla $\zeta\in K$ .\\

        \noindent
        Para la segunda afirmación, tenemos que:
        \begin{equation*}
            K = F(\zeta, r)
        \end{equation*}
        Puesto que $\zeta$ genera $\{r^{-1}s:s\in \mathbb{R}\}$ y al multiplicar por $s$ obtenemos $r$.\\ % // TODO: RAZONAR BIEN

        \noindent
        Para ver que el grupo de Galois es cíclico, representemos el grupo de manera sencilla. Para ello, tomamos $\sigma\in \Aut_{F(\zeta)}(K)$. Como $\sigma$ toma una raíz de $x^n-a$, la lleva en otra raíz y el conjunto de todas las raíces es:
        \begin{equation*}
            R = \{r, \zeta r, \ldots, \zeta^{n-1}r\}
        \end{equation*}
        Por lo que tendremos $\sigma(r) = \zeta^j r$ para cierto $j\in \mathbb{Z}_n$. Si tuviéramos que $\sigma(r) = \zeta^{j'}r$ para $j'\neq j$, como $r$ es una raíz primitiva de la unidad tenemos entonces que $j = j'$, por lo que observamos una dependencia $j\longmapsto \sigma$, que nos da una aplicación $j:\Aut_{F(\zeta)}(K)\to \mathbb{Z}^n$. Podemos ver $j$ como un homomorfismo de grupos, considerando $\mathbb{Z}^n$ como grupo aditivo. Para ello, tomamos $\sigma,\tau \in \Aut_{F(\zeta)}(K)$ y tenemos que $j(\sigma\tau)\in \mathbb{Z}_n$ está determinado por la condición:
        \begin{equation*}
            (\sigma\tau)(r) = \zeta^{j(\sigma\tau)} r
        \end{equation*}
        Y tenemos que:
        \begin{equation*}
            (\sigma\tau)(r) = \sigma(\tau(r)) = \sigma\left(\zeta^{j(\tau)}r\right) = \zeta^{j(\tau)}\sigma(r) = \zeta^{j(\tau)}\zeta^{j(\sigma)}r = \zeta^{j(\tau)+j(\sigma)} r
        \end{equation*}
        De donde $j(\sigma\tau) = j(\sigma) + j(\tau)$, por lo que $j$ es un homomorfismo de grupos. Además tenemos que $j$ es inyectivo, pues si $j(\sigma) = 0$, tendríamos por tanto que $\sigma(r) = \zeta^{0}r$, de donde $\sigma = id$, por lo que $\ker j = \{id\}$.  Tenemos por tanto que $\Aut_{F(\zeta)}(K)$ es isomorfo a un subgrupo de $\mathbb{Z}_n$, por lo que ha de ser cíclico y por el Teorema de Lagrange, de orden menor un divisor de $n$.
    \end{proof}
\end{teo}

\begin{coro}
    Sea $x^n-a\in F(\zeta)[x]$, es irreducible si y solo si $[K:F(\zeta)] = n$.
    \begin{proof}
        Por doble implicación:
        \begin{description}
            \item [$\Longleftarrow )$] En este caso tenemos que $j$ es sobreyectivo, por lo que $j$ es un isomorfimo, de donde todo elemento de $\mathbb{Z}_n$ proviene de un automorfismo, con lo que el conjunto $R$ es transitivo.
            \item [$\Longrightarrow )$]  % // TODO: HACER
        \end{description}
        % otra forma:
        % Tenemos que ese es irreducible si y solo si es el polinomio irreducible de $F(\zeta)(r)$, que enlaza con lo último.
    \end{proof}
\end{coro}

\begin{definicion}[Extensiones cíclica]
    Una extensión $F\leq K$ se dice cíclica si es de Galois y su grupo de Galois es cíclico.
\end{definicion}

\begin{observacion}
    Observemos que tanto las extensiones de cuerpos finitos como las extensiones como las del último Teorema son extensiones cíclicas:
\end{observacion}

\begin{ejemplo}
    \begin{itemize}
        \item Si $F$ contiene una raíz $n-$ésima primitiva de la unidad y $x^{n}-a\in F[x]$ es separable, entonces para $K$ un cuerpo de descomposición de $x^n-a$ tenemos que $F\leq K$ es cíclica.

            Esto se debe a que el $\zeta$ del Teorema está ya en $F$, y tenemos que $\Aut_F(K)$ es subgrupo de un grupo cíclico por el monomorfismo $j$.
            % // TODO: Pensar po que tenemos garantizado que x^n-a es separable con lo primero
        \item Toda extensión de cuerpos finitos es cíclica. % // TODO: razonar
    \end{itemize}
\end{ejemplo}

% Buscamos ahora un resultado que se asemeje a un recícproco del ejemplo

\begin{lema}[de independencia de Dedekind]
    Sean $\sigma_1,\ldots,\sigma_n:F\to E$ homomorfismos de cuerpos distintos. Tenemos entonces que $\sigma_1, \ldots, \sigma_n$  son linealmente independientes. Es decir, si $\lm_1, \ldots, \lm_n\in E$ es tal que:
    \begin{equation*}
        \lm_1 \sigma_1(x) + \ldots + \lm_n \sigma_n(x) = 0 \quad \forall x\in F \qquad \Longrightarrow \qquad \lm_1 = \ldots = \lm_n = 0
    \end{equation*}
    \begin{proof} % // TODO: REvisar esta demo
        Si $n=1$ es cierto. Suponemos que $n>1$ y razonamos por reducción al absurdo. Para ello, tomamos de entre todas las posibles elecciones de $\lm_1, \ldots, \lm_n$ aquella lista de ellos contengan $m$ elementos no nulos. Reordenando, podemos suponer que:
        \begin{gather*}
            \lm_i \neq 0 \qquad \forall i \in \{1,\ldots,m\} \\
            \lm_i = 0 \qquad \forall i >m
        \end{gather*}
        % // TODO: Ver esto por qué:
        Tenemos que $m\geq 2$. Como $\sigma_1 \neq \sigma_m$, existe $y\in F$ de forma que $\sigma_1(y) \neq \sigma_m(y)$, de donde:
        \begin{equation*}
            \lm_1 \sigma_1(yx) + \ldots + \lm_m \sigma_m(yx) = 0
        \end{equation*}
        Y obtenemos restando la cadena anterior con $n=m$:
        \begin{equation*}
            \lm_1(\sigma_1(y) - \sigma_m(y))\sigma_1(x) + \ldots + \lm_{m-1}(\sigma_{m-1}(y)-\sigma_m(y))\sigma_{m-1}(x) = 0 \qquad \forall x\in F
        \end{equation*}
        Y tenemos que $\sigma_m(y)$ no  es cero, pero hemos llegado a una contradicción, pues tenemos $\lm_1, \ldots, \lm_{m-1}$ una lista de números menor.
    \end{proof}
\end{lema}

\begin{teo}
    Sea $F\leq K$ extensión cíclica tal que $n = [K:F]$ no es múltiplo de $\car(F)$. Si $F$ contiene una raíz $n-$ésima primitiva de la unidad, entonces $K$ es cuerpo de descomposición de un polinomio irreducible de la forma $x^n-a\in F[x]$.

    \noindent
    Además, si $\alpha$ es una raíz de $x^n-a$ entonces $K = F(\alpha)$.
    \begin{proof}
        El grupo de automorfismos debe ser cíclico de grado $n$, por lo que tendrá un generador $\sigma\in \Aut_F(K)$ de orden $n$. El Lema de independencia de Dedekind nos dice que ha de existir $r\in K$ de forma que (si $\zeta\in F$ es una raíz $n-$ésima primitiva de la unidad):
        \begin{equation*}
            \beta := r + \zeta\sigma(r) + \ldots + \zeta^{n-1}\sigma^{n-1}(r) \neq 0
        \end{equation*}
        Tendremos entonces que:
        \begin{equation*}
            \zeta\sigma(\beta) = \beta
        \end{equation*}
        Por lo que:
        \begin{equation*}
            \beta^n = \zeta^n \sigma(\beta)^n = \sigma(\beta)^n = \sigma(\beta^n)
        \end{equation*}
        como $\sigma$ genera todo $\Aut_F(K)$ tenemos que $a:=\beta^n\in K^{\Aut_F(K)} = F$. Como $\zeta^n = 1$, tenemos que:
        \begin{equation*}
            \beta, \zeta\beta, \ldots, \zeta^{n-1}\beta
        \end{equation*}
        son todas raíces distintas de $x^n-a$, y tenemos $n$, de donde:
        \begin{equation*}
            x^n-a = (x-\beta)(x-\zeta\beta) \ldots (x-\zeta^{n-1}\beta)
        \end{equation*}
        Y como $\zeta\in F$, tenemos que $F(\beta)$ es cuerpo de descomposición de $x^n-a\in F[x]$. Como el orden de $\beta$ es menor o igual que $n$, $[F(\beta):F]\leq n$, y para conseguir la igualdad tenemos que:
        \begin{equation*}
            \sigma^k(\beta) = \zeta^{-k}\beta
        \end{equation*}
        Por lo que la acción de $\Aut_F(K)$ sobre las raíces de $x^n-a$ es transitiva, por lo que $x^n-a$ es irreducible, de donde $[F(\beta):F] = n$. Tenemos en definitiva que $F(\beta) = K$.
    \end{proof}
\end{teo}

\begin{ejemplo}
    Si consideramos $x^8-3\in \mathbb{Q}[x]$, si metemos una raíz octava primitiva de la unidad tenemos que su grupo de Galois es cíclico. Sea $K\leq \mathbb{C}$ su cuerpo de descomposición, sabemos por el Teorema de ayer que $K = \mathbb{Q}\left(\sqrt[8]{3}, \zeta\right)$, con $\zeta$ una raíz octava primitiva de la unidad. Esta podemos calcularla como una raíz cuadrada de $i$, que es una raíz cuarta de la unidad. Tomamos una de ellas:
    \begin{equation*}
        \zeta = e^{i\frac{\pi}{4}} = \frac{1}{\sqrt{2}} + i\frac{1}{\sqrt{2}}
    \end{equation*}
    Como el octavo polinomio ciclotómico tiene grado $\varphi(8) = 4$, tenemos que la extensión $\mathbb{Q}(\zeta)$ es de grado 4. Si consideramos ahora $\zeta+\overline{\zeta}\in \mathbb{Q}(\zeta)$:
    \begin{equation*}
        \zeta + \overline{\zeta} = 2\text{Re}(\zeta) = \sqrt{2} \in  \mathbb{Q}(\zeta)
    \end{equation*}
    De donde $\mathbb{Q}\left(\sqrt{2}\right)\leq \mathbb{Q}(\zeta)$, por lo que $\mathbb{Q}\left(i,\sqrt{2}\right) = \mathbb{Q}(\zeta)$.\\

    \noindent
    Calculamos el grado de la extensión $[K:\mathbb{Q}]$, para saber el cardinal de $\Aut_F(K)$. Por el Lema de la Torre:
    \begin{equation*}
        [K:\mathbb{Q}] = \left[\mathbb{Q}\left(\sqrt[8]{3},\sqrt{2},i\right):\mathbb{Q}\left(\sqrt[8]{3},\sqrt{2}\right)\right] \left[\mathbb{Q}\left(\sqrt{2},\sqrt[8]{3}\right):\mathbb{Q}\left(\sqrt[8]{3}\right)\right] \left[\mathbb{Q}\left(\sqrt[8]{3}\right):\mathbb{Q}\right]
    \end{equation*}
    Donde el último grado es $8$ por ser 3 primo y aplicar Eisenstein. La primera es 2 por ser $i\notin \mathbb{R}$. La segunda es 2 si y solo si $\sqrt{2}\notin \mathbb{Q}\left(\sqrt[8]{3}\right)$. 

    \noindent
    Consideramos:
    \begin{equation*}
        \mathbb{Q}\stackrel{2}{\leq } \mathbb{Q}\left(\sqrt{3}\right) \stackrel{\leq 2}{\leq} \mathbb{Q}\left(\sqrt[4]{3}\right) \stackrel{\leq 2}{\leq} \mathbb{Q}\left(\sqrt[8]{3}\right)
    \end{equation*}
    Y como $\mathbb{Q}\leq \mathbb{Q}\left(\sqrt[8]{3}\right)$ es de grado 8, tienen que ser todas estas de grado 2. Veamos:
    \begin{itemize}
        \item $\sqrt{2}\notin \mathbb{Q}\left(\sqrt{3}\right)$, puesto que si esto fuera así, $\sqrt{2} = a + b\sqrt{3}$, elevamos al cuadrado y sale que $\sqrt{3}$ es racional, que no es posible porque $x^2-3$ es irreducible.
        \item $\sqrt{2}\notin \mathbb{Q}\left(\sqrt[4]{3}\right)$, usando que una $\mathbb{Q}\left(\sqrt{3}\right)-$base es $\left\{1,\sqrt[4]{3}\right\}$, si $\sqrt{2}\in \mathbb{Q}\left(\sqrt[4]{3}\right)$ tendríamos entonces que $\exists a,b\in \mathbb{Q}\left(\sqrt{3}\right)$ de manera que:
            \begin{equation*}
                \sqrt{2} = a+b\sqrt[4]{3}
            \end{equation*}
            Elevando al cuadrado:
            \begin{equation*}
                2 = a^2 + 2ab\sqrt[4]{3} + b^2\sqrt{3} \Longrightarrow \left\{\begin{array}{l}
                    2 = a^2 + b^2 \sqrt{3} \\
                    0 = 2ab
                \end{array}\right.
            \end{equation*}
            igualando coordenadas a coordenadas, por lo que:
            \begin{itemize}
                \item Si $b= 0$, entonces $2 = a^2$, de donde $\sqrt{2} = a \in \mathbb{Q}(\sqrt{3})$, pero ya habíamos visto que este caso no puede ser.
                \item Si $a = 0$, entonces $2=b^2 \sqrt{3}$ para $b = x+y\sqrt{3}$ con $x,y\in \mathbb{Q}$, luego:
                    \begin{equation*}
                        2 = (x^2 + 2xy\sqrt{3} + 3y^2)\sqrt{3}
                    \end{equation*}
                    Y tenemos que $\{1,\sqrt{3}\}$ es una $\mathbb{Q}$-base de $\mathbb{Q}\left(\sqrt{3}\right)$, por lo que igualdando coordenadas:
                    \begin{equation*}
                        0 = x^2 + 3y^2  \Longrightarrow x = 0 = y
                    \end{equation*}
                    Luego $b= 0 \Longrightarrow 2 = 0$ \underline{contradicción}, que viene de suponer que teníamos $\sqrt{2}\in \mathbb{Q}\left(\sqrt[4]{3}\right)$.
            \end{itemize}
        \item Intentamos ver ahora que $\sqrt{2}\notin \mathbb{Q}\left(\sqrt[8]{3}\right)$. Por reducción al absurdo, si $\sqrt{2}\in \mathbb{Q}\left(\sqrt[8]{3}\right)$, tenemos que $\left\{1,\sqrt[8]{3}\right\}$  es una $\mathbb{Q}\left(\sqrt[4]{3}\right)-$base, por lo que existirían $c,d\in \mathbb{Q}\left(\sqrt[4]{3}\right)$ de forma que:
            \begin{equation*}
                \sqrt{2} = c+d\sqrt[8]{3}
            \end{equation*}
            con $d\neq 0$, por el apartado anterior. Elevando al cuadrado:
            \begin{equation*}
                2 = c^2 + 2cd\sqrt[8]{3} + d^2\sqrt[4]{3}
            \end{equation*}
            Igualando coordenadas en la base obtenemos que ($d\neq 0$) $c=0$, por lo que:
            \begin{equation*}
                2 = d^4\sqrt[4]{3} 
            \end{equation*}
            Escribimos las coordenadas de $d$:
            \begin{equation*}
                d = z + t\sqrt[4]{3} \qquad z,t\in \mathbb{Q}\left(\sqrt{3}\right)
            \end{equation*}
            De donde elevando al cuadrado:
            \begin{equation*}
                2 = (z^2 + 2zt\sqrt[4]{3} + t^2 \sqrt{3})\sqrt[4]{3}
            \end{equation*}
            Igualando coordenadas:
            \begin{equation*}
                0 = z^2 + t^2\sqrt{3} \Longrightarrow z = 0 = t
            \end{equation*}
            lo que nos lleva a una \underline{contradicción}.
    \end{itemize} % // TODO: En los apuntes esto está hecho de otra forma
    En definitiva, $[K:\mathbb{Q}] = 32$, de donde el grupo ciclico es de orden 8.
\end{ejemplo}

\section{Ecuaciones resolubles por radicales}
\noindent
Las definiciones de este apartado dependen mucho del autor.\\

\noindent
La siguiente definición generaliza el concepto de extensión por raíces cuadradas:
\begin{definicion}
    Una extensión de cuerpos $F\leq E$ se llamará una \underline{extensión por radicales} cuando existe una torre de cuerpos
    \begin{equation*}
        F = E_0 \leq E_1 \leq \ldots \leq E_t = E
    \end{equation*}
    tal que $E_j=E_{j-1}(\alpha_j)$, con $\alpha_j^{n_j}\in E_{j-1}$, para $j \in \{1,\ldots,t\}$.
\end{definicion}

\begin{definicion}
    Un polinomio $f\in F[x]$ se dice \underline{resoluble por radicales} si existe una extensión por radicales $F\leq E$ que contiene al cuerpo de descomposición de $f$.
\end{definicion}

\noindent
\textbf{Supondremos} que siempre trabajaremos en $\car(F)= 0$, para simplificar los enunciados siguientes. Es posible hacerlo para $\car(F)\neq 0$, pero entonces los enunciados se complican.

\begin{definicion}
    Una extensión $F\leq K$ es \underline{radical} si $K$ es cuerpo de descomposición de un polinomio $x^n-a\in F[x]$ y $F$ contiene una raíz $n-$ésima primitiva de la unidad.
\end{definicion}

\noindent
Bajo estas condiciones, toda extensión radical es cíclica, y toda cíclica da una radical. Y como estamos en $\car(F) = 0$, no hace falta hipótesis sobre que $x^n-a$ sea separable.

\begin{definicion}
    Diremos que una extensión $F\leq K$ es \underline{radical iterada} si hay una torre de cuerpos
    \begin{equation*}
        F = K_0 \leq K_1 \leq \ldots \leq K_t = K
    \end{equation*}
    de forma que cada $K_{i-1}\leq K_i$ es radical, para $i \in \{1,\ldots,t\}$.
\end{definicion}

\begin{prop}
    Si $F\leq E$ es una extensión de Galois, supongamos una extensión $E\leq E(\alpha)$ para $\alpha$ raíz de $x^n-a\in E[x]$, $a\neq 0$. Entonces existe una extensión radical iterada $E(\zeta)\leq K$ tal que $F\leq K$ es de Galois y $E(\alpha) \leq K$, para $\zeta$ una raíz $n-$ésima primitiva de la unidad. 
    \begin{proof}
        Consideramos:
        \begin{equation*}
            f = \prod_{\sigma\in \Aut_F(E)} (x^n-\sigma(a)) \in E[x]
        \end{equation*}
        Como $f^\tau = f\quad \forall \tau\in \Aut_F(E)$, tenemos que en realidad $f\in F[x]$. $E$ es cuerpo de descomposición de $g\in F[x]$. Sea $K$ el cuerpo de descomposición de $fg\in F[x]$. Como $f\in F[x]$, es claro que $E(\alpha)\leq K$, al tomar $\sigma = id$.\\

        \noindent
        Como $x^n-a$ es un factor de $f$, $K$ ha de contener un cuerpo de descomposición de $x^n-a$; por lo que podemos encontrar en $K$ una raíz $n-$ésima primitiva de la unidad, $\zeta$. Si enumeramos los elementos de $\Aut_F(E)$:
        \begin{equation*}
            \Aut_F(E) = \{\sigma_1, \ldots, \sigma_s\}
        \end{equation*}
        con $\sigma_1 = id_E$. Tomando $K_{-1} = E$ y $K_0 = E(\zeta)$, para cada $i \in \{1,\ldots,s\}$ tomamos $K_i = K_{i-1}(\alpha_i)$, con $\alpha_i$ raíz de $x^{n}-\sigma_i(a)$.\\

        \noindent
        De esta forma, cada $K_{i-1}\leq K_i$ es radical para $i\geq 1$. Además, $F\leq K$ es claramente de Galois.
    \end{proof}
\end{prop}

\noindent
En cierto momento, usaremos la sigiuente observación:

\begin{observacion}
    \noindent
    Sea $F$ un cuerpo, dados $\sigma_1:F\to L_1$, $\sigma_2:F\to L_2$ dos homomorfismos de cuerpos tales que las extensiones $\sigma_i(F)\leq L_i$ para $i \in \{1,2\}$ son finitas. Tomamos los polinomios $f_1,f_2\in F[x]$ tales que el cuerpo de descomposición de cada $f_i^{\sigma_i}$ venga dado por $\tau_i:L_i\to K_i$, para $i \in \{1,2\}$. Tenemos $\tau:F\to E$, cuerpo de descomposición de $f_1f_2$.\\

    \noindent
    La Tercera Proposición de Extensión obtenemos el diagrama de homomorfismos de cuerpos
    \begin{figure}[H]
        \centering
        \shorthandoff{""}
        \begin{tikzcd}
            F \arrow[r, "\sigma_1"] \arrow[d, "\sigma_2"'] \arrow[rrdd, "\tau"'] & L_1 \arrow[r, "\tau_1"] & K_1 \arrow[dd, "\eta_1"'] \\
            L_2 \arrow[d, "\tau_2"']                                             &                         &                           \\
            K_2 \arrow[rr, "\eta_2"]                                             &                         & E                        
        \end{tikzcd}
        \shorthandon{""}
    \end{figure}
    \noindent
    de manera que (cada triángulo conmuta):
    \begin{equation*}
        \tau = \eta_1\tau_1\sigma_1 = \eta_2\tau_2\sigma_2
    \end{equation*}
    De esta forma, como cada homomorfismo de cuerpos es inyectivo:
    \begin{equation*}
        F \cong Im \tau = \tau(F) \leq \eta_i \tau_i(L_i) \leq E \qquad \forall i \in \{1,2\}
    \end{equation*}
\end{observacion}

% // TODO: Cambiar las dos ultimas definiciones y proposicion, por las siguientes:
\begin{definicion}
    Una extensión $F\leq K$ se dice radical si $K$ es cuerpo de descomposición de un polinomio separable de la forma $x^n-a\in F[x]$.\\

    \noindent
    Es decir, como $\car(F) = 0$, es equivalente a decir que $a\neq 0$.\\

    \noindent
    Más aún, se dice que es radical iterada si:
    \begin{equation*}
        F=K_0\leq K_1 \leq \ldots \leq K_t = K
    \end{equation*}
    con $K_{i-1}\leq K_i$ radical, para cada $i \in \{1,\ldots,t\}$ y $F\leq K$ de Galois.
\end{definicion}

\begin{prop}
    Si $F\leq E$ es de Galois y $E\leq E(\alpha)$ para $\alpha$ raíz de $x^n-a\in E[x]$ con $a\neq 0$, entonces existe una extensión radical iterada $E\leq K$ con $E(\alpha)\leq K$ y $F\leq K$ de Galois.
\end{prop}

% // TODO: SEGUIMOS:
\begin{prop}
    Sea $F\leq E$ una extensión por radicales, entonces existe una extensión radical iterada $F\leq K$ tal que $E\leq K$.
    \begin{proof}
        Suponemos pues que tenemos una torre:
        \begin{equation*}
            F= E_0\leq E_1 \leq \ldots \leq E_t = E
        \end{equation*}
        tal que $E_j = E_{j-1}(\alpha_j)$ con $\alpha_j$ raíz de $x^{n_j}-a_j\in E_{j-1}[x]$, $j \in \{1,\ldots,t\}$. Razonamos por inducción sobre $t\geq 0$:
        \begin{itemize}
            \item \underline{Para $t=0$}, tomamos $F=E=K$.
            \item \underline{Para $t>0$}, por hipótesis de inducción tenemos que existe una extensión radical iterada 
                \begin{equation*}
                    F=K_0\leq K_1\leq \ldots \leq K_r
                \end{equation*}
                tal que $E_{t-1}\leq K_r$. Tomamos una $F-$extensión común de $K_r$ y $E_t$, dentro de la cual esté $K_r(\alpha_t)$. Tenemos que $E_t\leq K_r(\alpha_t)$. Por la Proposición anterior aplicada a la extensión $F\leq K_r$ de Galois, tenemos que existe una extensión radical iterada $K_r\leq K$  tal que $K_r(\alpha_t)\leq K$ y $F\leq K$ es de Galois.

                \noindent
                Tenemos una torre de cuerpos
                \begin{equation*}
                    K_r\leq K_{r+1}\leq \ldots \leq K_s
                \end{equation*}
                con $K_{i-1}\leq K_i$ radical para cada $i \in \{r+1,\ldots,s\}$. Tenemos entonces que:
                \begin{equation*}
                    F = K_0 \leq K_1 \leq \ldots \leq K_r \leq K_{r+1}\leq \ldots \leq K_j = K
                \end{equation*}
                con cada $K_{k-1}\leq K_k$ radical para $k \in \{1, \ldots, s\}$ y $F\leq K$ de Galois. De aquí tenemos que $F\leq K$ es radical iterada y $E\leq K$.
        \end{itemize}
    \end{proof}
\end{prop}

\begin{lema}
    Toda extensión radical iterada tiene grupo de Galois resoluble.
    \begin{proof} % Propo 3.31
        Veamos que si $F\leq K$ es radical entonces $\Aut_F(K)$ es resoluble. Si $F\leq K$ es radical entonces $K$ es cuerpo de descomposición de $x^n-a\in F[x]$ separable, por lo que contiene una raíz $n-$ésima primitiva de la unidad $\zeta\in K$. Tenemos por tanto:
        \begin{equation*}
            F\leq F(\zeta) \leq K
        \end{equation*}
        Con $F(\zeta)\leq K$ de Galois por ser $F\leq K$ de Galois ($F\leq K$ es radical) y $F\leq F(\zeta)$ de Galois por ser una extensión ciclotómica (que siempre son de Galois). Usando ahora la conexión de Galois, tenemos entonces que:
        \begin{equation*}
            \{id_K\} \lhd \Aut_{F(\zeta)}(K) \lhd \Aut_F(K)
        \end{equation*}

        y además:
        \begin{equation*}
            \frac{\Aut_F(K)}{\Aut_{F(\zeta)}(K)} \cong \Aut_F(F(\zeta))
        \end{equation*}
        Como $\Aut_F(F(\zeta))$ es el grupo de Galois de una extensión ciclotómica, tiene que ser abeliano, por lo que el factor $\Aut_{F(\zeta)}(K)\lhd \Aut_F(K)$ da un cociente abeliano. Ahora, tenemos que $\Aut_{F(\zeta)}(K)$ es cíclico, luego el factor $\{id_K\} \lhd \Aut_{F(\zeta)}(K)$ también es abeliano. En definitiva, tenemos que $\Aut_F(K)$ es resoluble, ya que admite una serie de composición con factores simples abelianos.\\

        \noindent
        Si ahora:
        \begin{equation*}
            F=K_0\leq K_1 \leq \ldots \leq K_t = K
        \end{equation*}
        es radical iterada, tendremos entonces que $F\leq K$ es de Galois por definición, así como que cada extensión intermedia es de Galois, por ser cuerpo de descomposición de un polinomio separable. Usando ahora la conexión de Galois, obtenemos una serie de grupos:
        \begin{equation*}
            \Aut_F(K) = \Aut_{K_0}(K) \rhd \Aut_{K_1}(K) \rhd \ldots \rhd \Aut_{K_{t-1}}(K) \rhd \{id_K\}
        \end{equation*}
        Veamos ahora que:
        \begin{equation*}
            \frac{\Aut_{K_{i-1}}(K)}{\Aut_{K_i}(K)} \cong \Aut_{K_{i-1}}(K_i)
        \end{equation*}
        con este resoluble, ya que la extensión $K_{i-1}\leq K_i$ es de Galois, y hemos visto que estas tienen grupo de Galois resoluble. En definitiva, obtenemos que $\Aut_F(K)$ es resoluble. % // TODO: Puede hacerse por induccioin sobre t o por refinamiento de Schreier
    \end{proof}
\end{lema}

\begin{ejercicio}
    Sea $f\in F[x]$ y $L$ cuerpo de descomposición de $f\in F[x]$, si $F\leq E$ es una extensión, demostrar que para $K$ cuerpo de descomposición de $f\in E[x]$ se tiene que:
    \begin{center}
        $\Aut_E(K)$ es isomorfo a un subgrupo de $\Aut_F(L)$.
    \end{center}

    \noindent
    Es claro que $L\leq K$, como $K$ está generado sobre $F$ por las raíces de $f$, los generadores que necesitamos obtener para obtener $K$ son los mismos. Si tomamos $\sigma\in \Aut_E(K)$, lleva $\alpha_i$  en $\alpha_j$ y deja fijos a los elementos de $L$. Estamos llevando un elemento de $L$ en otro de $L$. Se considera la aplicación restricción.
\end{ejercicio}

\begin{teo}[Gran Teorema de Galois]
    Sea $f\in F[x]$: 
    \begin{center}
        $f$ es resoluble por radicales $\Longleftrightarrow $ el grupo de Galois de $f$ es resoluble
    \end{center}
    \begin{proof} 
        Sea $L$ el cuerpo de descomposición de $f$:
        \begin{description}
            \item [$\Longrightarrow )$] Tenemos entonces una extensión por radicales $E$ de $f$ que contiene a todas sus raíces, por lo que tenemos la torre $F\leq L\leq E$. Por las Proposiciones anteriores, se ha visto que existe una extensión radical iterada de $F$ tal que $E\leq K$, por lo que $F\leq K$ es de Galois, y como $F\leq L$ también es de Galois, sabemos por la conexión de Galois que $\Aut_F(K)$ contiene como subgrupo normal al grupo de Galois $\Aut_L(K)$. Sabemos además que:
                \begin{equation*}
                    \Aut_F(L) \cong \frac{\Aut_F(K)}{\Aut_L(K)}
                \end{equation*}
                Y en el Lema anterior vimos que $\Aut_F(K)$ es resoluble, por lo que $\Aut_F(L)$ es resoluble, que es el grupo de Galois de $f$.
            \item [$\Longleftarrow )$] Supuesto que $\Aut_F(L)$ es resoluble, tomamos $n=[L:F]$ y consideramos $K=L(\zeta)$ con $\zeta$ una raíz $n-$ésima primitiva de la unidad. El Ejercicio anterior nos dice que $\Aut_{F(\zeta)}(K)$ es isomorfo a un subgrupo de $\Aut_F(L)$. Como $\Aut_F(L)$ es resoluble, tendremos que $\Aut_{F(\zeta)}(K)$ es resoluble y sabemos que $|\Aut_{F(\zeta)}(K)|$ divide a $n$ (por el Teorema de Lagrange). Usando la conexión de Galois, tenemos una serie de composición de $\Aut_{F(\zeta)}(K)$:
                \begin{equation*}
                    \Aut_{F(\zeta)}(K) = G_0 \rhd G_1 \rhd \ldots \rhd G_{t-1} \rhd G_t = \{id_K\}
                \end{equation*}
                con $\frac{G_{i-1}}{G_i}$ de cardinal $p_i$ primo y esto nos da:
                \begin{equation*}
                    F(\zeta) = K^{G_0} \leq K^{G_1} \leq \ldots \leq K^{G_{i-1}} \leq K^{G_t} = K
                \end{equation*}
                con $p_i\mid n\quad \forall i$, ya que el cardinal de cada uno de los factores invariantes dividen al cardinal del grupo del que son factores invariantes, por lo que cada primo divide a $n$. % // TODO: ?

                \noindent
                Resulta que $\zeta$ elevada a una potencia elevada nos da una raíz $p_i-$ésima de la unidad, por lo que cada $K^{G_{i}}$ contiene una raíz $p_i-$ésima primitiva de la unidad. Como los cocientes son cíclicos, cada extensión $K^{G_i}$ es cíclica, por lo que cada $K^{G_{i}}$ es cuerpo de descomposión de cierto $x^{p_i}-a_i\in K_{i-1}[x]$.\\

                \noindent
                Como $K$ contiene todas las raíces de $f$ y $F(\zeta)\leq K$ es radical iterada será una extensión por radicales, con lo que $f$ es resoluble por radicales.
        \end{description}
    \end{proof}
\end{teo}

\subsubsection{Consecuencias}
% // TODO: Dependiendo si f es separable, descomponer en producto de irreducibles por ejemplo
\begin{enumerate}
    \item Si $deg f\leq 4$, entones $f$ es resoluble por radicales.

        Esto es porque el grupo de Galois de $f$ está dentro de $S_n$ con $n\leq 4$ y estos grupos son resolubles.
    \item Si $degf \geq 5$, entonces $f$ es resoluble por radicales dependiendo de su grupo de Galois.

        Veremos que $x^5-4x-1\in \mathbb{Q}[x]$ tiene grupo de Galois isomorfo a $S_5$, luego NO es resoluble por radicales.
\end{enumerate}

\section{Ecuación general de grado $n$}
\noindent
Recordamos que si $F$ un cuerpo, consideramos $F[x_1, \ldots, x_n]$ el anillo de polinomios con $n$ indeterminadas.
\begin{itemize}
    \item recordamos que al alterar el orden de las indeterminadas obtenemos anillos isomorfos
    \item como $F$ es un cuerpo, $F[x_1]$ es un DFU, por lo que $F[x_1,x_2]$ también, \ldots, $F[x_1,\ldots, x_n]$ es un DFU. En particular, un dominio de integridad.
\end{itemize}
Si aplicamos la construcción de cuerpo de fracciones a $F[x_1,\ldots,x_n]$, obtenemos $F(x_1,\ldots, x_n)$, el cuerpo de fracciones del dominio de integridad $F[x_1,\ldots, x_n]$:
\begin{equation*}
    F(x_1,\ldots,x_n) = \left\{\frac{f}{g} : f,g\in F[x_1,\ldots,x_n], g\neq 0\right\}
\end{equation*}
Dada una permutación $\sigma\in S_n$, aplicando $n$ veces la Propiedad Universal del anillo de polinomioa, obtenemos un homomorfismo de anillos $\overline{\sigma}:F[x_1,\ldots,x_n]\to F[x_1,\ldots,xn]$ determinado\footnote{Por la Propiedad Universal del anillo de polinomios.} por:
\begin{align*}
    \overline{\sigma}(\alpha) &= \alpha \quad \forall \alpha\in F \\
    \overline{\sigma}(x_i) &= x_{\sigma(i)} \quad \forall i \in \{1,\ldots,n\}
\end{align*}
Que claramente es un isomorfismo, pues $\overline{\sigma^{-1}}$ es su homomorfismo inverso. Usando la Propiedad Universal de $F(x_1,\ldots,x_n)$ para obtener un automorfismo de cuerpos $\overline{\sigma}:F(x_1,\ldots,x_n)\to F(x_1,\ldots,x_n)$ dado por:
\begin{equation*}
    \overline{\sigma}\left(\frac{f(x_1,\ldots,x_n)}{g(x_1,\ldots,x_n)}\right) = \frac{f(x_{\sigma(1)}, \ldots, x_{\sigma(n)})}{g(x_{\sigma(1)}, \ldots, x_{\sigma(n)})}
\end{equation*}
Tenemos por tanto una aplicación $S_n\to \Aut_F(F(x_1,\ldots,x_n))$ que es un homomorfismo de grupos inyectivo, cuya imagen denotaremos por $G$.

\begin{definicion}
    Al cuerpo $E^G$ (donde $E=F(x_1,\ldots,x_n)$) lo llamamos cuerpo de las funciones simétricas racionales en $x_1,\ldots,x_n$ con coeficientes en $F$.\\

    \noindent
    Tenemos que $E^G\leq E$ es de Galois.
\end{definicion}

% // TODO: VER donde meter esto, es de clase de ejercicios
\begin{ejercicio}
    Sea $f\in F[x]$ separable, irreducible y de grado primo $p$, entonces su grupo de Galois contiene un $p-$ciclo.\\

    \noindent
    Si es separable e irreducible, $p$ divide al orden del grupo de Galoios. Como es primo, entonces $G$ ha de contener un elemento de orden $p$. Descomponemos el elemento como producto de ciclos disjuntos y su orden ha de ser el minimo común múltiplo de todos los órdenes de los ciclos que aparecen en su descomposición. Si estos números tienen como mínimo común múltiplo un número primo, entonces el orden de todos los cicloss es $p$. En $S_p$, el primero que aparece ha gastado todos los símbolos, luego ha de ser un $p-$ciclo.
\end{ejercicio}
