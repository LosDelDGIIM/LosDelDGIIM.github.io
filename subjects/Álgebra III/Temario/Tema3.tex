\chapter{Teoría de Galois de Ecuaciones}
\section{Grupo de Galois de un polinomio}
\noindent
A lo largo de este capítulo, consideraremos siempre polinomios mónicos.

\begin{definicion} % El discriminante resulta ser caso particular de la resultante de dos polinomios (en particular, de un polinomio y su derivado), que es para ver si dos curvas se tocan o no, viene de la geometría clásica
    Sea $f\in F[x]$ no constante, mónico y sean $\alpha_1, \ldots, \alpha_n$ sus raíces (repetidas tantas veces como indique su multiplicidad) en algún cuerpo $K$ de descomposición de $f$. El \underline{discriminante} de $f$ es:
    \begin{equation*}
        \Disc(f) = \prod_{1\leq i<j \leq n} {(\alpha_i-\alpha_j)}^{2} \in K
    \end{equation*}
\end{definicion}

\noindent
Resulta que $\Disc(f)$ se puede calcular a partir de los coeficientes del polinomio.

\begin{observacion}
    $f$ es separable $\Longleftrightarrow \Disc(f)\neq 0$. % // TODO: Escribir
\end{observacion}

\begin{notacion}
    Notaremos usualmente a la raíz del discriminante $Disc(f)$ por:
    \begin{equation*}
        \Delta(f) = \prod_{1\leq i < j \leq n}(\alpha_i-\alpha_j)
    \end{equation*}
\end{notacion}

\begin{notacion} % // TODO: Estudiar los S_n, y subgrupos de S4
    Dado un conjunto $S = \{\alpha_1, \ldots, \alpha_n\}$, denotamos al grupo de permutaciones de dichos elementos por:
    \begin{equation*}
        \Sim(\alpha_1, \ldots, \alpha_n)
    \end{equation*}
    Observemos que $\Sim(\alpha_1,\ldots, \alpha_n)\cong S_n$.
\end{notacion}

\begin{definicion}
    Si $f\in F[x]$ es separable y $K$ es su cuerpo de descomposición, $\Aut_F(K)$ se llagma grupo de Galois de $f$. Consideraremos la aplicación definida por restricción:
    \begin{align*}
        \Aut_F(K) &\longrightarrow \Sim(\alpha_1, \ldots, \alpha_n) \\
        \sigma&\longmapsto \sigma\big|_{\{\alpha_1, \ldots, \alpha_n\}}
    \end{align*}

    así, como $\Sim(\alpha_1,\ldots,\alpha_n)\cong S_n$, podemos ver $\Sim(\alpha_1,\ldots,\alpha_n)$ como permutar los subíndices, de forma que:
    \begin{equation*}
        \alpha_i \stackrel{\sigma}{\longmapsto} \alpha_{\sigma(i)}
    \end{equation*}
    donde consideramos que $\alpha_{\sigma(i)} := \sigma(\alpha_i)$.
\end{definicion}

% Sera habitual ver Aut_F(K) = Sim(.) = Sn

\begin{observacion}
    Si tomamos $\sigma\in \Aut_F(K)$:
    \begin{itemize}
        \item $\sigma(\Disc(f)) = \Disc(f)$.
        \item $\sigma(\Delta(f)) = sgn(\sigma)\Delta(f)$.
    \end{itemize}
\end{observacion}

\begin{prop}
    Sea $f\in F[x]$ separable con grupo de Galois $G = \Aut_F(K)$. Entonces $\Disc(f) \in F$. Además:
    \begin{equation*}
        K^{G\cap A_n} = F(\Delta(f))
    \end{equation*}
    Por tanto, $\Delta(f) \in F \Longleftrightarrow G\leq A_n$.
    \begin{proof}
        Para lo primero, como $\sigma(\Disc(f)) = \Disc(f)$, tenemos $\Disc(f)\in K^G$, y además $F = K^G$ por ser $F\leq K$ de Galois.\\

        \noindent
        Además, de la segunda observación vemos que $\Delta(f)\in K^{G\cap A_n}$, por lo que:
        \begin{equation*}
            F(\Delta(f)) \leq K^{G\cap A_n}
        \end{equation*}
        Tenemos por la conexión de Galois que:
        \begin{equation*}
            \left[K^{G\cap A_n} : F\right] = (G : G\cap A_n) \leq (S_n : A_n) = 2
        \end{equation*}
        donde en la desigualdad hemos usado el Tercer Teorema de Isomorfía para grupos. Por tanto, bien el grado de la extensión es 1 o 2, en función de si $\Delta(f) \in F$. % // TODO: TERMINAR
    \end{proof}
\end{prop}

\noindent
La condición ``$\Delta(f)\in F$'' se suele decir por ``$\Disc(f)$ es un cuadrado en $F$''.
