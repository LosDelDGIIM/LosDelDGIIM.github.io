\chapter{Teoría de Galois de Ecuaciones}
\section{Grupo de Galois de un polinomio}
\noindent
A lo largo de este capítulo, consideraremos siempre polinomios mónicos.

\begin{definicion} % El discriminante resulta ser caso particular de la resultante de dos polinomios (en particular, de un polinomio y su derivado), que es para ver si dos curvas se tocan o no, viene de la geometría clásica
    Sea $f\in F[x]$ no constante, mónico y sean $\alpha_1, \ldots, \alpha_n$ sus raíces (repetidas tantas veces como indique su multiplicidad) en algún cuerpo $K$ de descomposición de $f$. El \underline{discriminante} de $f$ es:
    \begin{equation*}
        \Disc(f) = \prod_{1\leq i<j \leq n} {(\alpha_i-\alpha_j)}^{2} \in K
    \end{equation*}
\end{definicion}

\noindent
Resulta que $\Disc(f)$ se puede calcular a partir de los coeficientes del polinomio.

\begin{observacion}
    $f$ es separable $\Longleftrightarrow \Disc(f)\neq 0$. 
\end{observacion}

\begin{notacion}
    Notaremos usualmente a la raíz del discriminante $\Disc(f)$ por:
    \begin{equation*}
        \Delta(f) = \prod_{1\leq i < j \leq n}(\alpha_i-\alpha_j)
    \end{equation*}
\end{notacion}

\begin{notacion} % // TODO: Estudiar los S_n, y subgrupos de S4
    Dado un conjunto $S = \{\alpha_1, \ldots, \alpha_n\}$, denotaremos normalmente al grupo de permutaciones de dichos elementos por:
    \begin{equation*}
        \Sim(\alpha_1, \ldots, \alpha_n)
    \end{equation*}
    Observemos que $\Sim(\alpha_1,\ldots, \alpha_n)\cong S_n$.
\end{notacion}

\begin{definicion}
    Si $f\in F[x]$ es separable y $K$ es su cuerpo de descomposición, diremos que $\Aut_F(K)$ es el grupo de Galois\footnote{Observemos que por ser $f$ separable y $K$ cuerpo de descomposición suyo tenemos siempre por el Teorema~\ref{teo:piedra_angular} que la exntesión $F\leq K$ es de Galois.} de $f$.
\end{definicion}

\noindent
Si $f\in F[x]$ es separable y $K$ es su cuerpo de descomposición, si consideramos $\{\alpha_1, \ldots, \alpha_n\}$ el conjunto de todas las raíces de $f$ en $K$, podemos siempre definir un homomorfismo de grupos entre el grupo de Galois de $f$ y el grupo de permutaciones de sus raíces:
\begin{align*}
    \Aut_F(K) &\longrightarrow \Sim(\alpha_1, \ldots, \alpha_n) \\
    \sigma&\longmapsto \sigma\big|_{\{\alpha_1, \ldots, \alpha_n\}}
\end{align*}
Tenemos que:
\begin{itemize}
    \item La aplicación está bien definida, pues si consideros $\sigma\in \Aut_F(K)$, tendremos siempre que $\sigma^\ast = \sigma\big|_{\{\alpha_1, \ldots, \alpha_n\}}\in \Sim(\alpha_1, \ldots, \alpha_n)$, pues si $\alpha_i$ es una raíz de $f$ (para $i \in \{1,\ldots,n\}$) tendremos entonces que $\sigma(\alpha_i)$ también es raíz de $f$:
        \begin{equation*}
            f(\sigma(\alpha_i)) = \sum_{i=0}^n f_i {(\sigma(\alpha_i))}^{i} \AstIg \sigma\left(\sum_{i=0}^{n}f_i \alpha_i^i\right) = \sigma(0) = 0
        \end{equation*}
        donde en $(\ast)$ hemos usado que $\sigma\in \Aut_F(K)$ y que $f\in F[x]$.
    \item La aplicación es un homeomorfismo, pues si $\sigma,\tau\in \Aut_F(K)$ tenemos entonces que:
        \begin{equation*}
            (\sigma\tau)\big|_{\{\alpha_1, \ldots, \alpha_n\}} = \sigma\big|_{\{\alpha_1, \ldots, \alpha_n\}}\tau\big|_{\{\alpha_1, \ldots, \alpha_n\}}
        \end{equation*}
\end{itemize}
Además dicho homomorfismo de grupos es siempre inyectivo, pues la Proposición de Extensión nos dice que cada automorfismo del grupo de Galois queda unívocamente determinado por la imagen de cada raíz de $f$, puesto que sabemos que el grupo de Galois de $f$ coincide con las extensiones de la inclusión:
\begin{equation*}
    \Aut_F(K) = Ex(\iota,\iota)
\end{equation*}
Si pensamos en la obtención de todos los elementos del grupo de Galois de $f$ mediante el siguiente procedimiento:
\begin{figure}[H]
    \centering
    \shorthandoff{""}
    \begin{tikzcd}
        F \arrow[r, hook] \arrow[rd, hook] & K                                                             & {F(\alpha_1, ..., \alpha_{i-1})} \arrow[r, hook] \arrow[rd, hook] & K                                                                              & {F(\alpha_1, ..., \alpha_{n-1})} \arrow[r, hook] \arrow[rd, hook] & K                                                 \\
                                           & F(\alpha_1) \arrow[u, "\alpha_1 \longmapsto \eta(\alpha_1)"'] &                                                                   & {F(\alpha_1, ..., \alpha_{i})} \arrow[u, "\alpha_i\longmapsto\eta(\alpha_i)"'] &                                                                   & K \arrow[u, "\alpha_n\longmapsto\eta(\alpha_n)"']
    \end{tikzcd}
    \shorthandon{""}
\end{figure}
\noindent
observamos que cada uno de ellos queda determinado por cada una de las elecciones hechas sobre cada una de las imágenes de cada raíz. De esta forma, si tenemos que dos elementos $\sigma,\tau\in \Aut_F(K)$ coinciden en $\{\alpha_1, \ldots, \alpha_n\}$, tendremos entonces que $\sigma = \tau$, lo que nos prueba la inyectividad del homomorfismo de grupos.

\noindent
De esta forma, como $\Sim(\alpha_1, \ldots, \alpha_n)\cong S_n$, podemos ver siempre el grupo de Galois de $f$ como subgrupo de $S_n$, aquel que permuta los índices de las raíces de $f$:
\begin{equation*}
    \alpha_i \stackrel{\sigma}{\longmapsto} \alpha_{\sigma(i)}
\end{equation*}

\begin{notacion}
    En vista de la relación existente entre $\Aut_F(K)$ (el grupo de Galois de cierto polinomio $f\in F[x]$), $\Sim(\alpha_1, \ldots, \alpha_n)$ (el grupo de permutaciones sobre sus raíces) y $S_n$, será habitual identificar $\Sim(\alpha_1, \ldots, \alpha_n)$ con $S_n$, y ver $\Aut_F(K)$ directamente como subgrupo de $S_n$. Este uso de la notación no debe llevar a errores, pues simplemente es una forma más rápida de enunciar ciertas propiedades sobre $\Aut_F(K)$.
\end{notacion}

\begin{observacion}
    Si tomamos $\sigma\in \Aut_F(K)$, una vez visto que $\sigma$ actuando sobre las raíces del polinomio $f$ simplemente las permuta, vemos fácilmente que:
    \begin{itemize}
        \item $\sigma(\Disc(f)) = \Disc(f)$.
        \item $\sigma(\Delta(f)) = sgn(\sigma)\Delta(f)$.
    \end{itemize}
\end{observacion}

\begin{prop}\label{prop:discf}
    Sea $f\in F[x]$ separable con grupo de Galois $G = \Aut_F(K)$. Entonces $\Disc(f) \in F$. Además:
    \begin{equation*}
        K^{G\cap A_n} = F(\Delta(f))
    \end{equation*}
    Por tanto, $\Delta(f) \in F \Longleftrightarrow G\leq A_n$.
    \begin{proof}
        Para ver que $\Disc(f)\in F$, vimos en el primer punto de la observación superior que:
        \begin{equation*}
            \sigma(\Disc(f)) = \Disc(f) \qquad \forall \sigma\in G
        \end{equation*}
        Por lo que tenemos que $\Disc(f) \in K^G$, pero como $F\leq K$ es de Galois, tenemos que $K^G = F$.\\

        \noindent
        Para ver que $K^{G\cap A_n} = F(\Delta(f))$, en vista del segundo punto de la observación superior:
        \begin{equation*}
            \sigma(\Delta(f)) = sgn(\sigma)\Delta(f) \qquad \forall \sigma\in G
        \end{equation*}
        Tenemos que $\Delta(f) \in K^{G\cap A_n}$, y como todo elementos de $G$ es $F-$lineal es claro que $F(\Delta(f)) \leq K^{G\cap A_n}$. Si estudiamos el índice de este subcuerpo de $K$, la conexión de Galois nos dice que:
        \begin{equation*}
            \left[K^{G\cap A_n} : F\right] = (G : G\cap A_n) \stackrel{(\ast)}{\leq} (S_n : A_n) = 2
        \end{equation*}
        donde en $(\ast)$ hemos usado el Segundo Teorema de Isomorfía para grupos. Por tanto, solo tnemos dos situaciones posibles:
        \begin{equation*}
            F(\Delta(f)) = F \qquad \text{o}\qquad F(\Delta(f)) = K^{G\cap A_n}
        \end{equation*}
        \begin{itemize}
            \item Si $F(\Delta(f)) = F$, tendremos entonces que $\Delta(f)\in F$, así como que:
                \begin{equation*}
                    sgn(\sigma)\Delta(f) = \sigma(\Delta(f)) = \Delta(f)\sigma(1) = \Delta(f)\qquad \forall \sigma\in G
                \end{equation*}
                Por lo que $G\leq A_n$, de donde:
                \begin{equation*}
                    K^{G\cap A_n} =K^G = F = F(\Delta(f))
                \end{equation*}
            \item Si $F(\Delta(f)) = K^{G\cap A_n}$, tendremos entonces que $\Delta(f)\notin F$, por lo que:
                \begin{equation*}
                    sgn(\sigma)\Delta(f) = \sigma(\Delta(f)) \neq \Delta(f)  \qquad \forall \sigma\in G
                \end{equation*}
                Por lo que $sgn(\sigma) = -1$, de donde $G\not\leq A_n$.
        \end{itemize}
    \end{proof}
\end{prop}

\noindent
En relación al enunciado de la Proposición anterior, se suele hacer referencia a la condición ``$\Delta(f)\in F$''  por ``$\Disc(f)$ es un cuadrado en $F$''.

\begin{ejercicio} % // TODO: HACER
    Sea $f\in \mathbb{R}[x]$ con $degf = 3$, discutir el número de raíces reales de $f$ según el signo de $\Disc(f)$. % Pasar al caso monico
\end{ejercicio}

\begin{ejemplo}
    Consideramos $f = x^n +\sum\limits_{i=0}^{n-1}a_ix^i \in F[x]$ y sean $\alpha_1, \ldots, \alpha_n$ sus raíces (repetidas según multiplicidad), tenemos que:
    \begin{equation*}
        f = \prod_{i=1}^{n}(x-\alpha_i)
    \end{equation*}
    Igualando coeficientes de igual grado, obtenemos las relaciones de Cardano-Vieta\footnote{Hay una teoría desarrollada sobre esto, siempre se obtienen funciones simétricas en las raíces del polinomio.}. Por ejemplo, si $n=2$ se obtiene:
    \begin{equation*}
        a_0 = \alpha_1\alpha_2 \qquad a_1 = -(\alpha_1+\alpha_2)
    \end{equation*}
    Como $\Disc(f) = {(\alpha_1-\alpha_2)}^{2}$, tenemos que $\Disc(f) = a_1^2 - 4a_0$. \newline
    Para $n>2$, la cuenta no es tan sencilla, por lo que se prefiere usar un algoritmo para resolver el sistema de ecuaciones. Por tanto, se puede expresar $\Disc(f)$ en término de los coeficientes de $f$. Para $n=3$, la damos para $f=x^3+px+q$  (cúbica reducida\footnote{Sin término cuadrático.}) es:
    \begin{equation*}
        \Disc(f) = -4p^3 - 27q^2
    \end{equation*}
\end{ejemplo}

\begin{prop}
    Sea $f\in F[x]$ separable con grupo de Galois $G$
    \begin{equation*}
        f\text{\ es irreducible} \Longleftrightarrow G\text{\ actúa transitivamente sobre las raíces de\ } f
    \end{equation*}
    En tal caso, $degf$ divide a $|G|$.
    \begin{proof}
        Sea $K$ el cuerpo de descomposición de $f$, tenemos que $G = \Aut_F(K)$.
        \begin{description}
            \item [$\Longrightarrow )$] Si $f$ es irreducible y $\alpha,\beta\in K$ son raíces de $f$, podemos ($f = \Irr(\alpha,F)$) usar la Proposición de extensión, obteniendo $\sigma:F(\alpha)\to K$ de forma que $\sigma(\alpha) = \beta$.

                La tercera proposición de extensión nos dice que $\sigma$ se extiende a un automorfismo $\eta\in G$ y $\eta(\alpha) = \sigma(\alpha) = \beta$, por lo que la acción es transitiva.
            \item [$\Longleftarrow )$] Sea $g$ un factor irreducible de $f$ (ambos mónicos), tenemos que $g$ no es constante, con lo que sus raíces son también de $f$. Además, $\sigma(\alpha)$ es raíz de $g$, para todo $\sigma\in G$, y como $G$ actúa transitivamente sobre las raíces de $f$; toda raíz de $f$ es de $g$, con lo que $f = g$, de donde $f$ es irreducible.
        \end{description}
        Finalmente, para ver que $degf$ divide a $|G|$, si $\alpha$ es raíz de $f$, tenemos entonces $[F(\alpha):F] = degf$, que divide a $[K:F]$ por el Lema de la Torre, y $|G| = [K:F]$.
    \end{proof}
\end{prop}

\begin{coro}
    Por tanto, a la hora de buscar el grupo de Galois de un polinomio irreducible, descartaremos automáticamente los subgrupos de $S_n$ no transitivos.
\end{coro}

\begin{ejemplo}
    Sea $f\in F[x]$ separable e irreducible: 
    \begin{enumerate}
        \item Si $degf = 1$, su grupo de Galois es la identidad, como único elemento de $S_1$.
        \item Si $degf = 2$, el cuerpo de Galois de $f$ tiene grado 1 o 2. Si $f$ es irreducible, ha de ser de grado 2, con lo que su grupo de Galois es isomorfo a $C_2$ (observemos que $S_2\cong C_2$).
        \item Si $degf = 3$, la Proposición anterior nos dice que bien $G\cong A_3$ o $G\cong S_3$. La Proposición~\ref{prop:discf} nos dice que tenemos el primer caso si $\Delta(f)\in F$ y el segundo si $\Delta(f)\notin F$.
        \item Si $degf = 4$, la Proposición anterior nos dice que $G$ es isomorfo a un subgrupo transitivo de $S_4$. % // TODO: Aprender la lista
    \end{enumerate}
\end{ejemplo}

\begin{ejemplo}
    Sea $f\in F[x]$ polinomio separable e irreducible de grado $degf = 4$, sean $\alpha_1,\alpha_2,\alpha_3,\alpha_4$ las raíces de $f$ en un cuerpo de descomposición $K$ de $f$, consideramos:
    \begin{align*}
        \beta_1 = \alpha_1\alpha_2 + \alpha_3 \alpha_4 \\
        \beta_2 = \alpha_1\alpha_3 + \alpha_2 \alpha_4 \\
        \beta_3 = \alpha_1\alpha_4 + \alpha_2 \alpha_3 
    \end{align*}

    y definimos:
    \begin{equation*}
        g = (x-\beta_1)(x-\beta_2)(x-\beta_3)\in K[x]
    \end{equation*}
    Veamos que en realidad $g\in F[x]$. Para ello, como $F\leq K$ es de Galois, hemos de ver que el polinomio es fijo por todos los automorfismos del grupo de Galois de $f$ (basta verlo para todas las permutaciones). Concluimos que $g^{\sigma} = g\quad \forall \sigma\in G$, con lo que $g$ es una resolvente cúbica de $f$ (se verá).

    \noindent
    Se puede ver por el algoritmo mencionado anteriormente que si $f=x^4+bx^3+cx^2+dx+e$, entonces:
    \begin{equation*}
        g = x^3-cx^2+(bd-4e)x-b^2e+4ce-d^2
    \end{equation*}~\\

    \noindent
    Consultamos si sus raíces son distintas:
    \begin{equation*}
        \beta_2 - \beta_1 = (\alpha_2 - \alpha_3)(\alpha_4-\alpha_1)
    \end{equation*}
    Por lo que $\beta_2$ y $\beta_1$ son distintas (análogo para el resto de las parejas), con lo que $g$ es separable, luego $E= F(\beta_1,\beta_2,\beta_3)$ es una extensión de Galois de $F$, con $F\leq E \leq K$, de donde el grupo de Galois de $g$, $N = \Aut_E(K)$ es normal en $G$. Por lo que:
    \begin{equation*}
        \Aut_F(E)\cong \frac{G}{N}
    \end{equation*}~\\

    \noindent
    Consideramos $S:\Sim(\alpha_1,\alpha_2,\alpha_3,\alpha_4)\to\Sim(\beta_1,\beta_2,\beta_3)$ una aplicación de forma que:
    \begin{equation*}
        S(\sigma)(\alpha_i\alpha_j + \alpha_k\alpha_l) = \alpha_{\sigma(i)}\alpha_{\sigma(j)} + \alpha_{\sigma(k)} \alpha_{\sigma(l)}
    \end{equation*}
    que es un homomorfimo de grupos y es sobreyectivo (ya que dada una trasposición en el grupo de la derecha, podemos encontrar un elemento en la izquierda cuya imagen vaya a él). Calculamos su núcleo:
    \begin{equation*}
        \ker S = \{(1), (1\ 2)(3\ 4), (1\ 3)(2\ 4), (1\ 4)(2\ 3)\}
    \end{equation*}
    Y sabemos que son todas porque como el grupo de la derecha tiene $6$ elementos y el de la derecha 24; con lo que $\ker S = V$.\\

    \begin{figure}[H]
        \centering
        \shorthandoff{""}
        \begin{tikzcd}
            1 \arrow[r] & V \arrow[r] & {Sim(\alpha_1,\alpha_2,\alpha_3,\alpha_4)} \arrow[rr, "S"]                  &  & {Sim(\beta_1,\beta_2,\beta_3)} \arrow[r] & 1 \\
            1 \arrow[r] & N \arrow[r] & G \arrow[u] \arrow[rr, "r"'] \arrow[rru, bend left] \arrow[rru, bend right] &  & Aut_F(E) \arrow[r] \arrow[u]             & 1
        \end{tikzcd}
        \shorthandon{""}
    \end{figure}
    \noindent
    Por lo que:
    \begin{equation*}
        N = V\cap G
    \end{equation*}
\end{ejemplo}

\begin{ejemplo}
    Si tenemos $f=x^4+x+1\in \mathbb{Q}[x]$, no tiene raíces en $\mathbb{Q}$ (las únicas posibles son $-1$ y $1$). Como $f\in \mathbb{Z}[x]$ y $f$ es primitivo, reducimos módulo 2, obteniendo:
    \begin{equation*}
        \overline{f} = x^4+x+1 \in \mathbb{Z}_2[x]
    \end{equation*}
    $\overline{f}$ no tiene raíces y si fuera irreducible, tendríamos entonces que es producto del único polinomio irreducible de $\mathbb{Z}_2[x]$ que es $x^2+x+1$, pero no lo es, por lo que $f$ es irreducible sobre $\mathbb{Z}[x]$, luego sobre $\mathbb{Q}[x]$ también. Sea $G$ el grupo de Galois de $f$ sobre $\mathbb{Q}$, tenemos que $G$ es un subgrupo transitivo de $S_4$, así como que $|G|$ es un múltiplo de $degf = 4$. Los transitivos de $S_4$ son:
    \begin{itemize}
        \item Los cíclicos de 4 elementos.
        \item Un Klein, de entre los 3 isomorfos a Klein uno es transitivo y dos no.

            Como ejemplo de esto:
            \begin{equation*}
                V = \{(1\ 2)(3\ 4), (1\ 3)(2\ 4), (1\ 4)(2\ 3), 1\}
            \end{equation*}
            es transitivo pero:
            \begin{equation*}
                \{(1), (1\ 2), (3\ 4), (1\ 2)(3\ 4)\}
            \end{equation*}
            es isomorfo a Klein pero no es transitivo (desde 1 no podemos llegar a 3).
        \item De 8 elementos tenemos los diédricos, que hay varios.
        \item $A_4$.
    \end{itemize}
    Anteriormente vimos que si $f = x^4+bx^3+cx^2+dx+e$ entonces su resolvente tenía el aspecto:
    \begin{equation*}
        g = x^3-cx^2+(bd-4e)x-b^2+4ce-d^2
    \end{equation*}
    Para nuestro $f$ la resolvente cúbica es: 
    \begin{equation*}
        g = x^3-4x -1
    \end{equation*}
    Si $\alpha_1,\alpha_2,\alpha_3,\alpha_4$ son raíces de $f$ y $\beta_1,\beta_2,\beta_3$ son las de $g$, teníamos entonces que:
    \begin{equation*}
        \beta_2-\beta_1 = (\alpha_2-\alpha_3)(\alpha_4-\alpha_1)
    \end{equation*}
    más otras dos relaciones. Usando estas, se demuestra que $\Disc(f) = \Disc(g)$. % // TODO: PROBAR
    Además, $g$ es una cúbica reducida, y teníamos una fórmula para calcular $\Disc(g)$, obtniendo que:
    \begin{equation*}
        \Disc(f) = \Disc(g) = 229
    \end{equation*}
    Y tenemos que $\sqrt{229}\notin \mathbb{Q}$, ya que esto sucede si $x^2-229\in \mathbb{Q}[x]$ es irreducible, porque $229$ es primo (se comprueba tratando de dividir entre primos hasta la parte entera de $\sqrt{229}$, que es 15). Como $\sqrt{229}\notin \mathbb{Q}$, tenemos que $G\not\subseteq A_4$, por lo que no puede ser el isomorfo a Klein ni $A_4$.\\

    \noindent
    En estas condiciones, teníamos una subextensión:
    \begin{equation*}
        \mathbb{Q}\leq E =\mathbb{Q}(\beta_1,\beta_2,\beta_3) \leq K = \mathbb{Q}(\alpha_1, \alpha_2,\alpha_3,\alpha_4)
    \end{equation*}
    Como $\mathbb{Q}\leq K$ es de Galois, tenemos que $E\leq K$ es de Galois, y la conexión nos dice que:
    \begin{equation*}
        \Aut_E(K) \lhd G
    \end{equation*}
    Veamos qué aspecto tiene $\Aut_E(K)$, para reducir las opciones sobre $G$, de hecho:
    \begin{equation*}
        \dfrac{G}{\Aut_E(K)} \cong \Aut_\mathbb{Q}(E)
    \end{equation*}
    Como $g$ no tiene raíces (ya que las únicas posibles raíces son $\pm 1$) y es de grado 3 tenemos que $g$ es irreducible, por lo que $|\Aut_\mathbb{Q}(E)|$ es múltiplo de $degg = 3$, con lo que solo puede ser 3 o 6. Como el único posible grupo $G$ que es divisible entre $3$ es la opción $G\cong S_4$. Buscamos ahora $\Aut_E(K)$, que ha de ser $V$.
\end{ejemplo}

\noindent
Respecto al tema anterior ganamos que no es necesario calcular de forma explícita cada uno de los automorfismos.

\section{Extensiones ciclotómicas}
\noindent
Para $n\geq 1$, a lo largo de este capítulo nos interesará el polinomio $x^n-1\in F[x]$, para $F\leq K$ cualquier extensión.
\begin{prop}
    Si $n\geq 1$ y consideramos como $S$ el conjunto de todas las raíces de $x^n-1\in F[x]$ en $K$ para $F\leq K$ cualquier extensión, tenemos que $S$ es un subgrupo cíclico de $K^\times$ cuyo orden es un divisor de $n$.
    \begin{proof}
        Sean $\alpha,\beta\in S$, tenemos que:
        \begin{equation*}
            \alpha^n - 1 = 0 = \beta^n -1 \quad \Longrightarrow \quad  \alpha^n = 1 = \beta^n
        \end{equation*}
        Y observamos ahora que:
        \begin{equation*}
            {(\alpha\beta^{-1})}^{n} -1  = (\alpha^n \beta^{-n})-1 = (1\cdot 1) - 1 = 0
        \end{equation*}
        Por lo que $\alpha\beta^{-1}\in S$, de donde $S$ es un subgrupo de $K^\times$. Sabemos que $S$ es un subgrupo cíclico de $K^\times$ por el Ejercicio~\ref{ej:subgrupo_finito}. Además, su orden ha de dividir a $n$, pues todos los elementos tienen orden un divisor de $n$: $\alpha^n = 1 \quad \forall \alpha\in S$.
    \end{proof}
\end{prop} 

\noindent
El subgrupo cíclico de las raíces será de orden $n$ si $K$ contiene un cuerpo de descomposición de $x^n-1$ y si $x^n-1$ es separable, es decir, si $n$ no es múltiplo de $\car(F)$. \newline
En este contexto, llamamos a las raíes de $x^n-1$ \underline{raíces $n-$ésimas de la unidad}, que es un grupo cíclico de orden $n$ si $x^n-1$ es separable, por lo que lo supondremos en general. A sus generadores los llamamos \underline{raíces $n-$ésimas primitivas de la unidad}.

\begin{ejemplo}
    En característica cero, podemos suponer sin pérdida de generalidad que $F=\mathbb{Q}$, por lo que obtenemos las conocidas raíces $n-$ésimas primitivas de la unidad en $\mathbb{C}$.
\end{ejemplo}

\noindent
En Álgebra I se vió que\footnote{Donde $\varphi$ es la función de Euler.} $|\cc{U}(\mathbb{Z}_n)| = \varphi(n)$, sea $\zeta$ una raíz $n-$ésima primitiva de la unidad, tenemos que el conjunto de todas las raíces $n-$ésimas de la unidad es:
\begin{equation*}
    \{\zeta^k : k\in \mathbb{Z}_n\}
\end{equation*}
Y las raíces $n-$ésimas primitivas de la unidad son:
\begin{equation*}
    \{\zeta^k : k\in \cc{U}(\mathbb{Z}_n)\}
\end{equation*}
De esta forma, el cuerpo de descomposición de $x^n-1\in F[x]$ es $F(\zeta)$, que recibe el nombre \underline{$n-$ésima extensión ciclotómica de $F$}, y como mucho es:
\begin{equation*}
    [F(\zeta):F] \leq n-1
\end{equation*}
Puesto que $x^n-1$ siempre tiene a 1 como raíz.

\begin{prop}
    Sea $x^n-1\in F[x]$ separable, tiene grupo de Galois $G$ isomorfo a un subgrupo de $\cc{U}(\mathbb{Z}_n)$. Además, $G$ es isomorfo a $\cc{U}(\mathbb{Z}_n)$ si y solo si actúa transitivamente sobre las raíces $n-$ésimas primitivas de la unidad (sobre $F$).
    \begin{proof}
        Sea $\zeta$ una raíz primitiva de la unidad, tenemos que:
        \begin{equation*}
            G = \Aut_F(F(\zeta))
        \end{equation*}
            donde $F(\zeta)$ es la $n-$ésima extensión ciclotómica de $F$. Habíamos visto que las raíces $n-$ésimas primitivas de la unidad son:
            \begin{equation*}
                \{\zeta^k : k\in \cc{U}(\mathbb{Z}_n)\}            
            \end{equation*}
            Sea $\sigma\in G$, tenemos que $\sigma(\zeta) = \zeta^k$ para cierto $k\in \cc{U}(\mathbb{Z}_n)$. Podemos definir 
            \Func{}{G}{\cc{U}(\bb{Z}_n)}{\sigma}{k}
            donde se cumple que:
            \begin{equation*}
                \sigma(\zeta) = \zeta^k
            \end{equation*}
            Si tomamos $\sigma,\tau\in G$ de forma que $\tau(\zeta) = \zeta^l$ con $l\in \cc{U}(\mathbb{Z}_n)$, tenemos que:
            \begin{equation*}
                \sigma\tau(\zeta) = \sigma(\zeta^l) = \sigma(\zeta)^l = {(\zeta^k)}^{l} = \eta^{kl}
            \end{equation*}
            Con lo que la aplicación considerada es un homomorfismo de grupos, que además es inyectivo por su definición. Tenemos pues que $G$ es isomorfo a cierto subgrupo de $\cc{U}(\mathbb{Z}_n)$. Esta aplicación será sobreyectiva si para cada exponente tenemos un automorfismo, con lo que $G$ actuará transitivamente sobre estas raíces.
    \end{proof}
\end{prop}

\begin{definicion}[Polinomio ciclotómico]
    Sea $F$ un cuerpo, definimos el $n-$ésimo polinomio ciclotómico como:
    \begin{equation*}
        \phi_n = \prod_{k\in \cc{U}(\mathbb{Z}_n)} \left(x-\zeta^k\right)
    \end{equation*}
    con $\zeta$ una raíz $n-$ésima primitiva de la unidad.
\end{definicion}

\begin{prop}
    Se tiene que:
    \begin{equation*}
        x^n-1 = \prod_{d \in Div(n)}\phi_d
    \end{equation*}
    \begin{proof}
        Consideramos como $R_n$ el conjunto de todas las raíces de $x^n-1$ (es decir, el conjunto de todas las raíces $n-$ésimas de la unidad). Si consideramos también $P_m$, el conjunto de las raíces $m-$ésimas primitivas de la unidad, tenemos una partición de $R_n$:
        \begin{equation*}
            R_n = \biguplus_{d\in Div(n)}P_d
        \end{equation*}
        Como:
        \begin{equation*}
            x^n - 1 = \prod_{\alpha \in R_n}(x-\alpha)
        \end{equation*}
        y cada $\alpha$ está en un cierto $P_d$, ha de estar en $\phi_d$, con lo que:
        \begin{equation*}
            x^n - 1 = \prod_{\alpha \in R_n}(x-\alpha) = \prod_{d\in Div(n)}\phi_d
        \end{equation*}
    \end{proof}
\end{prop}

\begin{prop}
    Cada $\phi_n$ tiene coeficientes en el subcuerpo primo de $F$ y además, si $\car(F) = 0$, tenemos que $\phi_n\in \mathbb{Z}[x]$.
    \begin{proof}
        Por inducción sobre $n$:
        \begin{itemize}
            \item Si $n=1$, entonces $\phi_1 = x-1$ y se tiene la Proposición.
            \item Si $n>1$, tenemos:
                \begin{equation*}
                    x^n-1 = \phi_n \cdot \prod_{\substack{d \in Div(n)\\d<n}}\phi_d
                \end{equation*}
                Por hipótesis de inducción, tenemos que el producto de la derecha está en el subcuerpo primo de $F$. Tenemos además que $\phi_n$ es cociente de $x^n-1$ entre el producto de la derecha, con lo que $\phi_n$ también ha de tener coeficientes en el subcuerpo primo de $F$.
        \end{itemize}
        Si $\car(F) = 0$, sabemos por lo ya probado que $\phi_n\in \mathbb{Q}[x]$. Si expresamos sus coeficientes como fracciones irreducibles y tomamos $a\in \mathbb{Z}$ el mínimo común múltiplo de sus denominadores, tenemos que $a\phi_n \in \mathbb{Z}[x]$, con todos sus coeficientes coprimos entre sí, luego $a\phi_n$  es primitivo. Tenemos pues que:
        \begin{equation*}
            a(x^n-1) = a\phi_n \prod_{\substack{d\in Div(n)\\d<n}}\phi_d
        \end{equation*}
        de nuevo por inducción, suponemos ahora que los polinomios $\phi_d$ son primitivos (para $n=1$ es claro que $\phi_1$ es primitivo). Por el Lema de Gauss, tenemos que:
        \begin{equation*}
            \prod_{\substack{d\in Div(n)\\d<n}}\phi_d
        \end{equation*}
        es primitivo y con coeficientes en $\mathbb{Z}$, si recordamos que $a\phi_n$ es primitivo, de donde todo el producto es primitivo, es decir, $a(x^n-1)$ es primitivo, luego ha de ser $a = 1$.
    \end{proof}
\end{prop}

\begin{ejemplo}
    En característica cero: 
    \begin{equation*}
        \phi_1 = x-1,\qquad \phi_2 = x+1, \qquad \phi_3 = x^2+x+1, \qquad \phi_4 = x^2+1
    \end{equation*}
    Para $\phi_6$ usamos la fórmula:
    \begin{equation*}
        \phi_6 = \frac{x^6-1}{\phi_1\phi_2\phi_3} = x^2-x+1
    \end{equation*}
\end{ejemplo}

\begin{teo}
    Cada $\phi_n\in \mathbb{Z}[x]$ es irreducible.
    \begin{proof}
        Sea $f$ un factor irreducible de $\phi_n$. Tomamos $\zeta$ una raíz compleja de $f$ en la $n-$ésima extensión ciclotómica. Probemos que si $p$ es primo y no divide a $n$, entonces $\zeta^p$ es raíz de $f$. Por reducción al absurdo, tomamos $g$ con:
        \begin{equation*}
            \phi_n = fg\qquad f,g\in \mathbb{Z}[x]
        \end{equation*}
        Y ha de cumplir $g(\zeta^p) = 0$. Resulta que $\zeta$ es raíz de $h=g(x^p)\in \mathbb{Z}[x]$. De esta forma, $f$ y $h$ tienen una raíz común compleja, y la identidad de Bezout nos dice entonces que $f$ y $h$ no son coprimos. Como $f$ es irreducible, ha de ser $f\mid h$. Como $f$ es primitivo, tenemos que $f\mid h$ en $\mathbb{Z}[x]$. Reducimos módulo $p$: 
        \begin{equation*}
            \overline{\phi_n} = \overline{f}\overline{g}
        \end{equation*}
        y tenemos:
        \begin{equation*}
            \overline{h} = \overline{g(x^p)} \AstIg {\overline{g(x)}}^{p} = \overline{g}^p
        \end{equation*}
        donde $(\ast)$ es porque % // TODO: HACER, a^p = a para a en F_p
        como $f\mid h$  en $\mathbb{Z}[x]$, tenemos que $\overline{f}\mid \overline{h}$, pero como $\overline{h} = \overline{g}^p$. Como $\bb{F}_p[x]$ es DFU, tenemos que $\overline{f}$ y $\overline{g}$ tienen algún factor común no constante. Deducimos que $\overline{x^n-1}$ tiene una raíz múltiple en su cuerpo de descomposición. Sin embargo, $p$ no divide a $n$, por lo que:
        \begin{equation*}
            \overline{x^n-1} = x^n-1
        \end{equation*}
        y este polinomio es separable, \underline{contradicción}; por lo que para cada $p$ primo que no divide a $n$ tenemos que $\zeta^p$ es raíz de $f$. Si tomamos $n$ y lo factorizamos en primos, miramos los primos que no dividen a $n$, con lo que tomando estos primos nos movemos dentro de las unidades de $\mathbb{Z}_n$, con lo que en alguna cantidad finita de pasos rellenamos todas las unidades de $\mathbb{Z}_n$. Por lo que todas las raíces de $\phi_n$ son raíces de $f$. Como $f$ dividía a $\phi_n$, $f = \phi_n$.
    \end{proof}
\end{teo}
