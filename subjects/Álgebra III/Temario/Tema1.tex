\chapter{Extensiones de cuerpos y raíces de polinomios}




¿Qué es un cuerpo (o field, campo)? Es un tipo de anillo conmutativo.\noindent
¿Qué es un anillo?

\begin{definicion}[Anillo]
    Un anillo es un conjunto no vacío $A$ que tiene definidas dos operaciones binarias, $+:A\times A\to A$, que tendrá un elemento destacado, denotado por $0$; y $\cdot:A\times A\to A$, que tiene un elemento destacado, denotado por $1$. Abreviando:
    \begin{equation*}
        (A,+,0,\cdot,1)
    \end{equation*}
    $A$ con $+$ es un grupo aditivo conmutativo con elemento neutro $0$. $A$ con $\cdot $ $A$ es un monoide, es decir, una operación binaria asociativa, que no tiene por qué ser conmutativa.\\

    \noindent
    Para completar el anillo hacen falta las leyes distributivas:
    \begin{equation*}
        a\cdot (b+c) = a\cdot b + a\cdot c \qquad \forall a,b,c\in A
    \end{equation*}
    Y como no exigimos conmutatividad:
    \begin{equation*}
        (a+b)\cdot c = a\cdot c + b\cdot c \qquad \forall a,b,c\in A
    \end{equation*}
\end{definicion}

\begin{definicion}[Cuerpo]
    Un cuerpo es un anillo $A$ al que se le pide además 
    $A\setminus \{0\}$ es un grupo, es decir, $\forall a\in A\setminus \{0\}$ $\exists a^{-1}\in A$ de forma que $a\cdot a^{-1} = 1$, de donde $0\neq 1$.
\end{definicion}

\begin{definicion}[Cuerpo]
    Es un anillo conmutativo $A$ de forma que $A\setminus\{0\}$ es un grupo.
\end{definicion}

\begin{ejemplo}
    Algunos ejemplos de los cuerpos más famosos son:
    \begin{itemize}
        \item $\mathbb{Q}$.
        \item $\mathbb{R}$.
        \item $\mathbb{C}$.
        \item $\mathbb{Z}_p$ con $p$ primo.
    \end{itemize}
\end{ejemplo}

% // TODO: Incluir cuestión de notación

\begin{definicion}[Subanillo]
    Sea $A$ un anillo y $B\subseteq A$, $B$ es un subanillo de $A$ si:
    \begin{itemize}
        \item $1\in B$.
        \item $(B,+)$ es un subgrupo de $(A,+)$.
        \item $a,b\in B\Longrightarrow ab\in B$.
    \end{itemize}
\end{definicion}

\begin{ejemplo}
    Por ejemplo, $\mathbb{Z}$ no tiene subanillos propios.
    \begin{itemize}
        \item $\mathbb{Z}$ es subanillo de $\mathbb{Q}$.
        \item $\mathbb{Q}$ es subanillo de $\mathbb{R}$.
        \item $\mathbb{R}$ es subanillo de $\mathbb{C}$.
        \item $\mathbb{Z}_p$ no puede ser subanillo de $\mathbb{Z}$, $\mathbb{Q}$, $\mathbb{R}$, $\mathbb{C}$, ya que tiene característica $p$.
    \end{itemize}
\end{ejemplo}

\begin{definicion}[Homomorfismo de anillos]
    Sean $A,B$ anillos, un homomorfismo de anillos es una aplicación$f:A\to B$ que verifica:
    \begin{itemize}
        \item $f(1) = 1$.
        \item $f(a+b) = f(a) + f(b)$ (homomorfismo de grupos).
        \item $f(ab) = f(a)f(b)$ (homomorfismo de monoides).
    \end{itemize}
\end{definicion}

\begin{definicion}[Característica de un anillo]
    Sea $A$ un anillo, existe un único homomorfismo de anillos 
    \begin{equation*}
        \chi:\mathbb{Z}\to A
    \end{equation*}
    Ya que:
    \begin{equation*}
        \chi(n) = \sum_{k=1}^{n}\chi(1) = \sum_{k=1}^{n}1_A \qquad \forall n\in \mathbb{N}\setminus\{0\}
    \end{equation*}
    \end{itemize}

    Además, $\ker\chi$ es un ideal en $\mathbb{Z}$, y todos los ideales de $\mathbb{Z}$ eran principales, es decir, de la forma $n\mathbb{Z}$ para cierto $n\in \mathbb{N}$. La característica de $A$ es $\ker\chi$.
\end{definicion}

\begin{definicion}[Subcuerpo]
    Si $K$ es un cuerpo, un subcuerpo de $K$ es un subanillo $F$ de $K$ tal que $F$ es un cuerpo (es decir, que $F$ es cerrado para los inversos de cada elemento no nulo de $K$).
\end{definicion}

\subsection{Cuerpos primos}
Sea $K$ un cuerpo y $\Gamma$ un conjunto no vacío (ya que el propio cuerpo siempre es un subcuerpo del mismo) de subcuerpos de $K$. Es fácil ver que:
\begin{equation*}
    \bigcap_{F\in \Gamma} F \text{\ es un subcuerpo de } K
\end{equation*}

Sea ahora $S\subseteq K$ un subconjunto de un cuerpo $K$ (nada impide que $S\neq \emptyset $), tomamos como $\Gamma$ el conjunto de los subcuerpos de $K$ que contienen a $S$. Para dicho $\Gamma$, consideramos:
\begin{equation*}
    \bigcap_{F\in \Gamma} F
\end{equation*}
Obtenemos el subcuerpo más pequeño de $K$ que contiene a $S$.\\

\noindent
Observemos que si $S = \emptyset $, obtenemos el menor subcuerpo de $K$. Llamaremos a dicho cuerpo ``subcuerpo primo de $K$''.

\\
\noindent
El primer Teorema de Isomorfismo nos dice ($\leq$ para subanillo):
\begin{equation*}
    \dfrac{\mathbb{Z}}{p\mathbb{Z}} = \dfrac{\mathbb{Z}}{\ker\chi} \cong Im\chi \leq K
\end{equation*}
Y $\mathbb{Z}/p\mathbb{Z}$ es un dominio de integridad cuando $p$ es $0$ o primo. (En un dominio en el que todos los ideales son principales, los ideales que dan como cociente un dominio de integridad son el 0 o uno que automáticamente es un cuerpo). % // TODO: Ver en Álgebra I

\begin{prop}
    Sea $K$ un cuerpo de característica $p$, entonces si $p>0$, el subcuerpo primo de $K$ es isomorfo a $\mathbb{Z}_p$, y si $p=0$, el subcuerpo primo de $K$ es isomorfo a $\mathbb{Q}$.
    \begin{proof}
        Sea $\Pi$ el subcuerpo primo de $K$: 
        \begin{itemize}
            \item Si $p>0$, $Im\chi$ es un subcuerpo de $K$, ya que:
                \begin{equation*}
                    \dfrac{\mathbb{Z}}{p\mathbb{Z}} = \dfrac{\mathbb{Z}}{\ker\chi} \cong Im\chi \leq K
                \end{equation*}
                Por tanto, como $\Pi$ es el menor subcuerpo de $K$, tenemos que $\Pi\subseteq Im\chi\cong \mathbb{Z}_p$, y como $\mathbb{Z}_p$ no contiene subcuerpo más que él mismo, tenemos que $\Pi=Im\chi \cong \mathbb{Z}_p$.
            \item Si $p=0$, entonces $\mathbb{Z}\cong Im\chi\leq K$, como cualquier subanillo contiene a $Im\chi$, tenemos que $Im\chi \subseteq \Pi$. 

                Si $Q$ es el cuerpo de fracciones de $Im\chi$, como $Im\chi\cong \mathbb{Z}$, tendremos que $Q\cong \mathbb{Q}$.

                Por la propiedad universal del cuerpo de fracciones, podemos meter una copia isomorfa del cuerpo de fracciones dentro de $\Pi$: $Q\subseteq \Pi$, y como $\Pi$ es el subcuerpo más chico de $K$ y $Q$ es un cuerpo, $\Pi = Q$.
        \end{itemize}
    \end{proof}
\end{prop}

\begin{definicion}[Extensión de cuerpos]
    Sea $F$ subcuerpo de un cuerpo $K$, diremos que $K$ es una extensión de cuerpos de $F$, notado por $F\leq K$ (esta notación se reservará para esto próximamente).
\end{definicion}

\begin{observacion}
    Si $F\leq K$ es una extensión, entonces $K$ es un espacio vectorial sobre $F$.

    \begin{itemize}
        \item $(K,+)$ es un grupo aditivo.
        \item Si $\lm\in F$ y $\alpha\in K$, $\lm \alpha$ es el producto que ya conocemos de $K$.
    \end{itemize}
\end{observacion}

\begin{definicion}
    Si $F\leq K$ es una extensión, la dimensión de $K$ sobre $F$ como espacio vectorial recibe el nombre de ``grado de la extensión $F\leq K$'', denotado por $[K:F]$.
\end{definicion}

\begin{ejemplo}
    Como ejemplo a destacar:
    \begin{itemize}
        \item $\mathbb{R}\leq \mathbb{C}$ tiene grado de extensión 2.
        \item $[\mathbb{R}:\mathbb{Q}]$.
    \end{itemize}
    Si $[\mathbb{R}:\mathbb{Q}]$ fuese finito e igual a $n$, entonces $\mathbb{R}\cong \mathbb{Q}^n$; por lo que $[\mathbb{R}:\mathbb{Q}] = \infty$, ya que $\mathbb{R}$ no es numerable.
\end{ejemplo}

\begin{notacion}
    Si la extensión de un cuerpo no es finita, diremos que es infinita.
\end{notacion}

\begin{ejercicio}
    Demostrar que el cardinal de un cuerpo finito es de la forma $p^n$, con $p$ primo y $n\geq 1$. (algebra lineal)\\

    \noindent
    Como es finito, el primo es de la forma $\mathbb{Z}_p$, luego es un espacio vectorial de dimensión finita $n$ sobre $\mathbb{Z}_p$, isomorfo como espacio vectorial a $\mathbb{Z}_p\times \ldots \times \mathbb{Z}_p$ $n$ veces, luego de cardinal $p^n$.
\end{ejercicio}

Haremos una clasificacion de cuerpos finitos, cada nº primo y numero natural no nulo, existe un cuerpo de $p^n$ elementos y todos ellos serán isomorfos entre sí.

\begin{notacion}
    Sea $F\leq K$ extensión, $S\subseteq K$, el menor subcuerpo de $K$ que contiene a $F\cup S$ lo denoto por $F(S)$, que recibe el nombre de ``extensión de $F$ generada por $S$''.\\

    \noindent
    Si $S = \{s_1,\ldots,s_t\}$, simplifico la notación como $F(s_1,\ldots,s_t)$.
\end{notacion}

\begin{ejemplo}
    $\mathbb{Q}(\sqrt{2})$ es el menor subcuerpo de $\mathbb{R}$ que contiene a $\sqrt{2}$, que puede calcularse:
    \begin{equation*}
        \mathbb{Q}(\sqrt{2}) = \left\{a+b\sqrt{2} : a,b\in \mathbb{Q}\right\}
    \end{equation*}
    \begin{proof}
        Veámoslo:
        \begin{description}
            \item [$\supseteq)$] Si tomo $a+b\sqrt{2}$, tomo 3 elementos de dicho cuerpo y se quedan dentro del cuerpo.
            \item [$\subseteq)$] Si demuestro que dicho subconjunto es un cuerpo, tengo inmediatamente la igualdad, por ser el menor subcuerpo que contiene a $\sqrt{2}$. Que es subanillo se ve facil, para ver que es subcuerpo, se imita lo que pasa con los numeros complejos: dado $a+b\sqrt{2}$, lo multiplicamos por su conjugado, que es distinto de cero y luego lo ponemos en el denominador, usando que $\sqrt{2}$ no es racional, luego denominador no nulo.
        \end{description}
    \end{proof}
    Observemos que tenemos $[\mathbb{Q}(\sqrt{2}): \mathbb{Q}] = 2$.
\end{ejemplo}
