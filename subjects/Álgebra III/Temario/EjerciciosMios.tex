\chapter{Ejercicios}
\setcounter{ejercicio}{50}

\begin{ejercicio}
    Sea $F$ un cuerpo de descomposición de $f=x^3+x+1\in \bb{F}_2[x]$ y $\alpha\in F$ una raíz de $f$. Razonar que $F=\bb{F}_2(\alpha)$. Resolver, en $F$, las siguientes ecuaciones, expresando las soluciones en función de $\alpha$:
    \begin{equation*}
        x^3+x+1 = 0, \qquad x^3+x^2+1=0, \qquad x^2+x+1=0
    \end{equation*}~\\

    \noindent
    Vemos que $f$ es un polinomio irreducible en $\bb{F}_2[x]$, puesto que tiene grado 3 y no tiene raíces en $\bb{F}_2$:
    \begin{equation*}
        f(0) = 1, \qquad f(1) = 1
    \end{equation*}
    Sea $F$ un cuerpo de descomposición de $f$, hemos visto en teoría que:
    \begin{equation*}
        F = \bb{F}_{2^3} = \bb{F}_8
    \end{equation*}
    Por lo que tenemos que $\bb{F}_2\leq \bb{F}_8$ con $[\bb{F}_8:\bb{F}_2] = 3$. Como $\alpha$ es raíz de $f$ tenemos que:
    \begin{equation*}
        \bb{F}_2 \leq \bb{F}_2(\alpha) \leq F
    \end{equation*}
    Y tenemos que $f = \Irr(\alpha,\bb{F}_2)$, de donde $[\bb{F}_2(\alpha):\bb{F}_2] = 3$. En vista de la torre de cuerpos anterior tenemos que $F=\bb{F}_2(\alpha)$.\\

    \noindent
    Sabemos que $\alpha$ es una raíz de $f$. Hemos visto en teoría que el resto se obtienen aplicando el automorfismo de Frobenius de la extensión $\bb{F}_2\leq F$ a $\alpha$, obteniendo así que las raíces de $f$ son:
    \begin{equation*}
        \alpha,\qquad \alpha^2,\qquad \alpha^4
    \end{equation*}
    Por lo que estas son las soluciones en $\bb{F}_2(\alpha) = F$ de la primera ecuación.\\

    \noindent
    Para resolver la segunda ecuación, sabemos que $\bb{F}_8$ es el cuerpo formado por todas y cada una de las raíces del polinomio:
    \begin{equation*}
        x^8-x\in \bb{F}_2[x]
    \end{equation*}
    Además, sabemos que este polinomio se descompone como producto de todos los polinomios mónicos irreducibles de grado (divisores de 3, ya que $8 = 2^3$) 1 y 3. Podemos comprobar que $g=x^3+x^2+1\in \bb{F}_2[x]$ es irreducible, puesto que es de grado 3 y no tiene raíces en $\bb{F}_2$:
    \begin{equation*}
        g(0) = 1, \qquad g(1) = 1
    \end{equation*}
    Tenemos así que $g$ aparece en la descomposición en irreducibles de $x^8-x$, al igual que $f$ y que $x,x-1$, con lo que:
    \begin{equation*}
        x^8-x = x(x-1)(x^3+x+1)(x^3+x^2+1)
    \end{equation*}
    Además, sabemos que $\bb{F}_8^\times$ es un grupo cíclico de orden 7, y como $\alpha\neq 1$ tenemos que:
    \begin{equation*}
        \langle \alpha \rangle  = \bb{F}_8^\times 
    \end{equation*}
    Vemos así que $x^3+x^2+1=0$ tiene solución en $\bb{F}_8$, puesto que aparece en la descomopsición de $x^8-x$, y sus raíces no pueden ser $0,1,\alpha,\alpha^2$ ni $\alpha^4$, por lo que tienen que ser $\alpha^3,\alpha^5$ y $\alpha^7$.\\

    \noindent
    Para la última ecuación, vemos que $h=x^2+x+1\in \bb{F}_2[x]$ es irreducible, al ser de grado 2 y no tener raíces. Como todos los polinomios que aparecen en la descomposición de $x^8-x$ también son irreducibles mónicos concluimos que $x^8-x$ no tiene raíces en común con $h$, y como las raíces de $x^8-x$ son exactamente los elementos de $\bb{F}_8$ concluimos que $h$ no tiene raíces en $\bb{F}_8$, por lo que $x^2+x+1=0$ no tiene solución en $\bb{F}_8$.
\end{ejercicio}

\begin{ejercicio}
    Sea $K$ un cuerpo de descomposición de $f=x^3+x+1\in \bb{F}_4[x]$ y $\alpha\in K$ una raíz de $f$. Razonar que $K=\bb{F}_4(\alpha)$. Resolver, en $K$, las siguientes ecuaciones, expresando las soluciones en función de $\alpha$:
    \begin{equation*}
        x^3+x+1 = 0, \qquad x^3+x^2+1=0, \qquad x^2+x+1=0
    \end{equation*}
    Construir, si es posible, una base de $K$ sobre $\bb{F}_2$ usando $\alpha$ y una solución de la tercera ecuación.\\

    \noindent
    Podemos ver $\bb{F}_4$ como cuerpo de descomposición de $x^2+x+1\in \bb{F}_2[x]$, polinomio irreducible, con lo que será de la forma:
    \begin{equation*}
        \bb{F}_4 = \{0,1,\gamma,\gamma^2\}
    \end{equation*}
    con $\gamma^2 + \gamma + 1 = 0$ y $\{1,\gamma\}$ una $\bb{F}_2-$base de $\bb{F}_4$. Vemos que $f$ es irreducible en $\bb{F}_4$, por ser de grado 3 y no tener raíces en $\bb{F}_4$:
    \begin{gather*}
        f(0) = 1, \qquad f(1) = 1\\
        f(\gamma) = \gamma^3 + \gamma + 1 = \gamma(\gamma+1) + \gamma + 1 = \gamma^2 + 1 = \gamma \neq 0 \\
        f(\gamma^2) = \gamma^6 + \gamma^2 + 1 = {(\gamma+1)}^{3} + \gamma+1 + 1 = \gamma^3 + \gamma^2 + \gamma + 1 + \gamma \\
        = \gamma(\gamma+1) + \gamma^2 + 1 = \gamma^2 + \gamma + \gamma^2 + 1 = \gamma + 1 \neq 0
    \end{gather*}
    Hemos visto en teoría que el cuerpo de descomposición de $f$ es:
    \begin{equation*}
        K = \bb{F}_{2^{2\cdot 3}} = \bb{F}_{2^6} = \bb{F}_{64}
    \end{equation*}
    por lo que tenemos $[K:\bb{F}_4] = 3$, y además $\bb{F}_4\leq \bb{F}_4(\alpha)\leq \bb{F}_{64}$, con:
    \begin{equation*}
        [\bb{F}_4(\alpha):\bb{F}_4] = 3
    \end{equation*}
    ya que $\Irr(\alpha,\bb{F}_4) = f$, de donde deducimos que $K = \bb{F}_4(\alpha)$.\\

    \noindent
    Para resolver la primera ecuación, sabemos por lo visto en teoría que el resto de soluciones se obtienen aplicando a $\alpha$ el automorfismo de Frobenius de la extensión $\bb{F}_4\leq \bb{F}_{64}$, con lo que el resto de raíces son:
    \begin{equation*}
        \alpha, \qquad \alpha^4, \qquad \alpha^{16}
    \end{equation*}
    Para la segunda ecuación, observemos que tenemos la torre:
    \begin{equation*}
        \bb{F}_2\leq \bb{F}_2(\alpha) \leq \bb{F}_4(\alpha)
    \end{equation*}
    con lo que si recordamos el Ejercicio anterior, teníamos que las soluciones de $x^3+x^2+1=0$ en $\bb{F}_2(\alpha)$ eran:
    \begin{equation*}
        \alpha^3, \qquad \alpha^5, \qquad \alpha^7
    \end{equation*}
    Por lo que estas seguirán siendo las raíces de $x^3+x^2+1=0$ en $\bb{F}_{64}$. No nos confundamos con la primera ecuación, que parece que la raíz $\alpha^2$ que teníamos en $\bb{F}_8$ ha cambiado por $\alpha^{16}$ en $\bb{F}_{64}$, lo que sucede es que $\alpha$ es un elemento de orden 7:
    \begin{equation*}
        \alpha^7 = \alpha{(\alpha^3)}^{2} = \alpha{(\alpha+1)}^{2} = \alpha(\alpha^2 + 1) = \alpha^3 + \alpha = 1
    \end{equation*}
    Por lo que $\alpha^2 = \alpha^{16}$. Si ahora tratamos de resolver la última ecuación:
    \begin{equation*}
        x^2+x+1
    \end{equation*}
    Empezamos el ejercicio viendo que $\bb{F}_4$ es cuerpo de descomposición de $x^2+x+1\in \bb{F}_2[x]$, y que además sus raíces eran $\gamma,\gamma^2$.
\end{ejercicio}

\begin{ejercicio}
    Calcular el número de polinomios irreducibles de grado 6 en $\bb{F}_2[x]$. (Nota: hay una fórmula general, si la encuentras en la web no la uses, no se trata de eso).
\end{ejercicio}

\begin{ejercicio}
    Calcular los grupos de Galois sobre $\mathbb{Q}$ de los polinomios $f=(x^2+x+1)(x^2-3)$ y $g=(x^2+x+1)(x^2+3)$.
\end{ejercicio}

\begin{ejercicio}
    Calcular el carindal del grupo de Galois sobre $\mathbb{Q}$ del polinomio $f=(x^3+x+1)(x^2+1)$.
\end{ejercicio}

\begin{ejercicio}
    Tomemos $f=(x^3-2)(x^2-3)\in \mathbb{Q}[x]$ y $K$ el cuerpo de descomposición sobre $\mathbb{Q}$ de $f$.
    \begin{enumerate}[label=\alph*)]
        \item Decidir razonadamente si $i+\sqrt{3}\in K$.
        \item Calcular razonadamente $[K:\mathbb{Q}]$.
        \item Describir los elementos del grupo $\Aut(K)$.
        \item Describir los elementos de $\Aut_{\mathbb{Q}\left(i+\sqrt{3}\right)}(K)$ y decidir si es un subgrupo normal de $\Aut(K)$.
    \end{enumerate}
\end{ejercicio}

\begin{ejercicio}
    Sea $f=x^3-3x+1\in \mathbb{Q}[x]$ y $\alpha$ cualquier raíz real de $f$. Demostrar que el cuerpo de descomposición de $f$ sobre $\mathbb{Q}$ es $\mathbb{Q}(\alpha)$.
\end{ejercicio}

\begin{ejercicio}
    Sea $K$ el cuerpo de descomposición del polinomio $f=(x^2+3)(x^3-3)\in \mathbb{Q}[x]$. Calcular todos los subcuerpos de $K$. Demostrar que $\mathbb{Q}\left(\sqrt[3]{3}+i\sqrt{3}\right)=K$.
\end{ejercicio}

\begin{ejercicio}
    Sea $F$ un cuerpo de descomposición de $f=x^6+x+1\in \bb{F}_2[x]$ y $\alpha\in F$ una raíz de $f$.
    \begin{enumerate}[label=\alph*)]
        \item Razonar que $F=\bb{F}_2(\alpha)$.
        \item Calcular el orden multiplicativo de $\alpha$.
        \item Resolver en $F$, expresando las soluciones en función de $\alpha$, la ecuación $x^2+x+1=0$.
    \end{enumerate}
\end{ejercicio}

\begin{ejercicio}
    Sea $\alpha=\sqrt{2}+i\sqrt{3}\in \mathbb{C}$.
    \begin{enumerate}[label=\alph*)]
        \item Demostrar que $\sqrt{2}\in \mathbb{Q}(\alpha)$ y calcular $[\mathbb{Q}(\alpha):\mathbb{Q}]$.
        \item Calcular, definiendo explícitamente todos sus elementos, el grupo de Galois de $\Irr(\alpha,\mathbb{Q})$.
        \item ¿Son las raíces de $\Irr(\alpha,\mathbb{Q})$ construibles con regla y compás?
    \end{enumerate}
\end{ejercicio}

\begin{ejercicio}
    Sea $K$ el cuerpo de descomposición de $f=x^6-3\in \mathbb{Q}[x]$.
    \begin{enumerate}[label=\alph*)]
        \item Calcular $[K:\mathbb{Q}]$.
        \item Demostrar que $i+\sqrt{3}\in K$.
        \item Calcular, definiendo explícitamente todos sus elementos, el grupo $G=\Aut_{\mathbb{Q}\left(i+\sqrt{3}\right)}(K)$.
        \item ¿Es $G$ un subgupo normal del grupo de Galois de $f$?
    \end{enumerate}
\end{ejercicio}

\begin{ejercicio}
    Sea $F=\bb{F}_3(a)$ un cuerpo con $a$ satisfaciendo la ecuación $a^3+a-1=0$.
    \begin{enumerate}[label=\alph*)]
        \item Calcular el cardinal de $F$.
        \item Calcular el grado de $\Irr(a^2,\bb{F}_3)$.
        \item Calcular $\Irr(a^2,\bb{F}_3)$.
    \end{enumerate}
\end{ejercicio}

\begin{ejercicio}
    Calcular el número de polinomios mónicos irreducibles de grado menor o igual que 3 en $\bb{F}_5[x]$.
\end{ejercicio}

\begin{ejercicio}
    Sean $\sqrt{2},\sqrt[3]{2}\in \mathbb{R}$.
    \begin{enumerate}[label=\alph*)]
        \item Calcular razonadamente $[\mathbb{Q}\left(\sqrt{2},\sqrt[3]{2}\right):\mathbb{Q}]$.
        \item Calcular razonadamente si $K=\mathbb{Q}\left(\sqrt{2},\sqrt[3]{2},i\sqrt{3}\right)$ es una extensión de Galois de $\mathbb{Q}$.
        \item Calcular $\Aut(K)$ definiendo explícitamente todos sus elementos.
        \item Calcular razonadamente el grado del polinomio
            \begin{equation*}
                f= \Irr(\sqrt{2}+\sqrt[3]{2},\mathbb{Q})
            \end{equation*}
        \item Decidir razonadamente quién es el grupo de Galois de f. ¿Es f resoluble por radicales? ¿Son las raíces complejas de este polinomio construibles con regla y compás?
    \end{enumerate}
\end{ejercicio}

\begin{ejercicio}
    Sea $K$ el cuerpo de descomposición de $f=x^{12}-1\in \mathbb{Q}[x]$.
    \begin{enumerate}[label=\alph*)]
        \item Calcular el grupo de Galois $G$ de $f$, definiendo explícitamente todos sus elementos.
        \item Calcular todos los subgrups de $G$.
        \item Calcular todos los subcuerpos de $K$, indicando a qué subgrupo de $G$ corresponde cada uno de ellos mediante la Conexión de Galois.
    \end{enumerate}
\end{ejercicio}

\begin{ejercicio}
    Sea $F=\bb{F}_3(a)$ un cuerpo con $a$ satisfaciendo la ecuación $a^3+a^2-1=0$.
    \begin{enumerate}[label=\alph*)]
        \item Calcular el cardinal de $F$.
        \item Calcular el grado de $\Irr(a^2,\bb{F}_3)$.
        \item Calcular $\Irr(a^2,\bb{F}_3)$.
    \end{enumerate}
\end{ejercicio}

\setcounter{ejercicio}{67}

\begin{ejercicio}
    Realizar las siguientes tareas.
    \begin{enumerate}[label=\alph*)]
        \item Demostrar que existe $a\in \bb{F}_{16}$ tal que $\bb{F}_{16} = \bb{F}_2(a)$ y $a^4+a+1=0$.
        \item Demostrar que $a$ genera el grupo cíclico $\bb{F}_{16}^\times$.
        \item Resolver en $\bb{F}_{16}$, expresando la solución en función de $a$, la ecuación $x^2+x+1=0$.
        \item Calcular el número de homomorfismos de cuerpos $\bb{F}_4\to \bb{F}_{16}$.
    \end{enumerate}
\end{ejercicio}

\begin{ejercicio}
    Calcular $\Aut(E)$, para $E=\mathbb{Q}\left(\sqrt[3]{5},i\sqrt{5}\right)$.
\end{ejercicio}
