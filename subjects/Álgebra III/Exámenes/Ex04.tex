\documentclass[12pt]{article}

% Idioma y codificación
\usepackage[spanish, es-tabla]{babel}       %es-tabla para que se titule "Tabla"
\usepackage[utf8]{inputenc}

% Márgenes
\usepackage[a4paper,top=3cm,bottom=2.5cm,left=3cm,right=3cm]{geometry}

% Comentarios de bloque
\usepackage{verbatim}

% Paquetes de links
\usepackage[hidelinks]{hyperref}    % Permite enlaces
\usepackage{url}                    % redirecciona a la web

% Más opciones para enumeraciones
\usepackage{enumitem}

% Personalizar la portada
\usepackage{titling}

% Paquetes de tablas
\usepackage{multirow}


%------------------------------------------------------------------------

%Paquetes de figuras
\usepackage{caption}
\usepackage{subcaption} % Figuras al lado de otras
\usepackage{float}      % Poner figuras en el sitio indicado H.


% Paquetes de imágenes
\usepackage{graphicx}       % Paquete para añadir imágenes
\usepackage{transparent}    % Para manejar la opacidad de las figuras

% Paquete para usar colores
\usepackage[dvipsnames]{xcolor}
\usepackage{pagecolor}      % Para cambiar el color de la página

% Habilita tamaños de fuente mayores
\usepackage{fix-cm}

% Para los gráficos
\usepackage{tikz}

% Para poder situar los nodos en los grafos
\usetikzlibrary{positioning}


%------------------------------------------------------------------------

% Paquetes de matemáticas
\usepackage{mathtools, amsfonts, amssymb, mathrsfs}
\usepackage[makeroom]{cancel}     % Simplificar tachando
\usepackage{polynom}    % Divisiones y Ruffini
\usepackage{units} % Para poner fracciones diagonales con \nicefrac

\usepackage{pgfplots}   %Representar funciones
\pgfplotsset{compat=1.18}  % Versión 1.18

\usepackage{tikz-cd}    % Para usar diagramas de composiciones
\usetikzlibrary{calc}   % Para usar cálculo de coordenadas en tikz

%Definición de teoremas, etc.
\usepackage{amsthm}
%\swapnumbers   % Intercambia la posición del texto y de la numeración

\theoremstyle{plain}

\makeatletter
\@ifclassloaded{article}{
  \newtheorem{teo}{Teorema}[section]
}{
  \newtheorem{teo}{Teorema}[chapter]  % Se resetea en cada chapter
}
\makeatother

\newtheorem{coro}{Corolario}[teo]           % Se resetea en cada teorema
\newtheorem{prop}[teo]{Proposición}         % Usa el mismo contador que teorema
\newtheorem{lema}[teo]{Lema}                % Usa el mismo contador que teorema

\theoremstyle{remark}
\newtheorem*{observacion}{Observación}

\theoremstyle{definition}

\makeatletter
\@ifclassloaded{article}{
  \newtheorem{definicion}{Definición} [section]     % Se resetea en cada chapter
}{
  \newtheorem{definicion}{Definición} [chapter]     % Se resetea en cada chapter
}
\makeatother

\newtheorem*{notacion}{Notación}
\newtheorem*{ejemplo}{Ejemplo}
\newtheorem*{ejercicio*}{Ejercicio}             % No numerado
\newtheorem{ejercicio}{Ejercicio} [section]     % Se resetea en cada section


% Modificar el formato de la numeración del teorema "ejercicio"
\renewcommand{\theejercicio}{%
  \ifnum\value{section}=0 % Si no se ha iniciado ninguna sección
    \arabic{ejercicio}% Solo mostrar el número de ejercicio
  \else
    \thesection.\arabic{ejercicio}% Mostrar número de sección y número de ejercicio
  \fi
}


% \renewcommand\qedsymbol{$\blacksquare$}         % Cambiar símbolo QED
%------------------------------------------------------------------------

% Paquetes para encabezados
\usepackage{fancyhdr}
\pagestyle{fancy}
\fancyhf{}

\newcommand{\helv}{ % Modificación tamaño de letra
\fontfamily{}\fontsize{12}{12}\selectfont}
\setlength{\headheight}{15pt} % Amplía el tamaño del índice


%\usepackage{lastpage}   % Referenciar última pag   \pageref{LastPage}
\fancyfoot[C]{\thepage}

%------------------------------------------------------------------------

% Conseguir que no ponga "Capítulo 1". Sino solo "1."
\makeatletter
\@ifclassloaded{book}{
  \renewcommand{\chaptermark}[1]{\markboth{\thechapter.\ #1}{}} % En el encabezado
    
  \renewcommand{\@makechapterhead}[1]{%
  \vspace*{50\p@}%
  {\parindent \z@ \raggedright \normalfont
    \ifnum \c@secnumdepth >\m@ne
      \huge\bfseries \thechapter.\hspace{1em}\ignorespaces
    \fi
    \interlinepenalty\@M
    \Huge \bfseries #1\par\nobreak
    \vskip 40\p@
  }}
}
\makeatother

%------------------------------------------------------------------------
% Paquetes de cógido
\usepackage{minted}
\renewcommand\listingscaption{Código fuente}

\usepackage{fancyvrb}
% Personaliza el tamaño de los números de línea
\renewcommand{\theFancyVerbLine}{\small\arabic{FancyVerbLine}}

% Estilo para C++
\newminted{cpp}{
    frame=lines,
    framesep=2mm,
    baselinestretch=1.2,
    linenos,
    escapeinside=||
}

% para minted
\definecolor{LightGray}{rgb}{0.95,0.95,0.92}
\setminted{
    linenos=true,
    stepnumber=5,
    numberfirstline=true,
    autogobble,
    breaklines=true,
    breakautoindent=true,
    breaksymbolleft=,
    breaksymbolright=,
    breaksymbolindentleft=0pt,
    breaksymbolindentright=0pt,
    breaksymbolsepleft=0pt,
    breaksymbolsepright=0pt,
    fontsize=\footnotesize,
    bgcolor=LightGray,
    numbersep=10pt
}


\usepackage{listings} % Para incluir código desde un archivo

\renewcommand\lstlistingname{Código Fuente}
\renewcommand\lstlistlistingname{Índice de Códigos Fuente}

% Definir colores
\definecolor{vscodepurple}{rgb}{0.5,0,0.5}
\definecolor{vscodeblue}{rgb}{0,0,0.8}
\definecolor{vscodegreen}{rgb}{0,0.5,0}
\definecolor{vscodegray}{rgb}{0.5,0.5,0.5}
\definecolor{vscodebackground}{rgb}{0.97,0.97,0.97}
\definecolor{vscodelightgray}{rgb}{0.9,0.9,0.9}

% Configuración para el estilo de C similar a VSCode
\lstdefinestyle{vscode_C}{
  backgroundcolor=\color{vscodebackground},
  commentstyle=\color{vscodegreen},
  keywordstyle=\color{vscodeblue},
  numberstyle=\tiny\color{vscodegray},
  stringstyle=\color{vscodepurple},
  basicstyle=\scriptsize\ttfamily,
  breakatwhitespace=false,
  breaklines=true,
  captionpos=b,
  keepspaces=true,
  numbers=left,
  numbersep=5pt,
  showspaces=false,
  showstringspaces=false,
  showtabs=false,
  tabsize=2,
  frame=tb,
  framerule=0pt,
  aboveskip=10pt,
  belowskip=10pt,
  xleftmargin=10pt,
  xrightmargin=10pt,
  framexleftmargin=10pt,
  framexrightmargin=10pt,
  framesep=0pt,
  rulecolor=\color{vscodelightgray},
  backgroundcolor=\color{vscodebackground},
}

%------------------------------------------------------------------------

% Comandos definidos
\newcommand{\bb}[1]{\mathbb{#1}}
\newcommand{\cc}[1]{\mathcal{#1}}

% I prefer the slanted \leq
\let\oldleq\leq % save them in case they're every wanted
\let\oldgeq\geq
\renewcommand{\leq}{\leqslant}
\renewcommand{\geq}{\geqslant}

% Si y solo si
\newcommand{\sii}{\iff}

% Letras griegas
\newcommand{\eps}{\epsilon}
\newcommand{\veps}{\varepsilon}
\newcommand{\lm}{\lambda}

\newcommand{\ol}{\overline}
\newcommand{\ul}{\underline}
\newcommand{\wt}{\widetilde}
\newcommand{\wh}{\widehat}

\let\oldvec\vec
\renewcommand{\vec}{\overrightarrow}

% Derivadas parciales
\newcommand{\del}[2]{\frac{\partial #1}{\partial #2}}
\newcommand{\Del}[3]{\frac{\partial^{#1} #2}{\partial #3^{#1}}}
\newcommand{\deld}[2]{\dfrac{\partial #1}{\partial #2}}
\newcommand{\Deld}[3]{\dfrac{\partial^{#1} #2}{\partial #3^{#1}}}


\newcommand{\AstIg}{\stackrel{(\ast)}{=}}
\newcommand{\Hop}{\stackrel{L'H\hat{o}pital}{=}}

\newcommand{\red}[1]{{\color{red}#1}} % Para integrales, destacar los cambios.

% Método de integración
\newcommand{\MetInt}[2]{
    \left[\begin{array}{c}
        #1 \\ #2
    \end{array}\right]
}

% Declarar aplicaciones
% 1. Nombre aplicación
% 2. Dominio
% 3. Codominio
% 4. Variable
% 5. Imagen de la variable
\newcommand{\Func}[5]{
    \begin{equation*}
        \begin{array}{rrll}
            #1:& #2 & \longrightarrow & #3\\
               & #4 & \longmapsto & #5
        \end{array}
    \end{equation*}
}

%------------------------------------------------------------------------


\DeclareMathOperator{\Irr}{Irr}
\DeclareMathOperator{\Subgr}{Subgr}
\DeclareMathOperator{\Subex}{Subex}
\DeclareMathOperator{\car}{car}
\DeclareMathOperator{\Aut}{Aut}
\DeclareMathOperator{\Disc}{Disc}
\DeclareMathOperator{\Sim}{Sim}

\begin{document}

    % 1. Foto de fondo
    % 2. Título
    % 3. Encabezado Izquierdo
    % 4. Color de fondo
    % 5. Coord x del titulo
    % 6. Coord y del titulo
    % 7. Fecha

    
    % 1. Foto de fondo
% 2. Título
% 3. Encabezado Izquierdo
% 4. Color de fondo
% 5. Coord x del titulo
% 6. Coord y del titulo
% 7. Fecha

\newcommand{\portada}[7]{

    \portadaBase{#1}{#2}{#3}{#4}{#5}{#6}{#7}
    \portadaBook{#1}{#2}{#3}{#4}{#5}{#6}{#7}
}

\newcommand{\portadaExamen}[7]{

    \portadaBase{#1}{#2}{#3}{#4}{#5}{#6}{#7}
    \portadaArticle{#1}{#2}{#3}{#4}{#5}{#6}{#7}
}




\newcommand{\portadaBase}[7]{

    % Tiene la portada principal y la licencia Creative Commons
    
    % 1. Foto de fondo
    % 2. Título
    % 3. Encabezado Izquierdo
    % 4. Color de fondo
    % 5. Coord x del titulo
    % 6. Coord y del titulo
    % 7. Fecha
    
    
    \thispagestyle{empty}               % Sin encabezado ni pie de página
    \newgeometry{margin=0cm}        % Márgenes nulos para la primera página
    
    
    % Encabezado
    \fancyhead[L]{\helv #3}
    \fancyhead[R]{\helv \nouppercase{\leftmark}}
    
    
    \pagecolor{#4}        % Color de fondo para la portada
    
    \begin{figure}[p]
        \centering
        \transparent{0.3}           % Opacidad del 30% para la imagen
        
        \includegraphics[width=\paperwidth, keepaspectratio]{assets/#1}
    
        \begin{tikzpicture}[remember picture, overlay]
            \node[anchor=north west, text=white, opacity=1, font=\fontsize{60}{90}\selectfont\bfseries\sffamily, align=left] at (#5, #6) {#2};
            
            \node[anchor=south east, text=white, opacity=1, font=\fontsize{12}{18}\selectfont\sffamily, align=right] at (9.7, 3) {\textbf{\href{https://losdeldgiim.github.io/}{Los Del DGIIM}}};
            
            \node[anchor=south east, text=white, opacity=1, font=\fontsize{12}{15}\selectfont\sffamily, align=right] at (9.7, 1.8) {Doble Grado en Ingeniería Informática y Matemáticas\\Universidad de Granada};
        \end{tikzpicture}
    \end{figure}
    
    
    \restoregeometry        % Restaurar márgenes normales para las páginas subsiguientes
    \pagecolor{white}       % Restaurar el color de página
    
    
    \newpage
    \thispagestyle{empty}               % Sin encabezado ni pie de página
    \begin{tikzpicture}[remember picture, overlay]
        \node[anchor=south west, inner sep=3cm] at (current page.south west) {
            \begin{minipage}{0.5\paperwidth}
                \href{https://creativecommons.org/licenses/by-nc-nd/4.0/}{
                    \includegraphics[height=2cm]{assets/Licencia.png}
                }\vspace{1cm}\\
                Esta obra está bajo una
                \href{https://creativecommons.org/licenses/by-nc-nd/4.0/}{
                    Licencia Creative Commons Atribución-NoComercial-SinDerivadas 4.0 Internacional (CC BY-NC-ND 4.0).
                }\\
    
                Eres libre de compartir y redistribuir el contenido de esta obra en cualquier medio o formato, siempre y cuando des el crédito adecuado a los autores originales y no persigas fines comerciales. 
            \end{minipage}
        };
    \end{tikzpicture}
    
    
    
    % 1. Foto de fondo
    % 2. Título
    % 3. Encabezado Izquierdo
    % 4. Color de fondo
    % 5. Coord x del titulo
    % 6. Coord y del titulo
    % 7. Fecha


}


\newcommand{\portadaBook}[7]{

    % 1. Foto de fondo
    % 2. Título
    % 3. Encabezado Izquierdo
    % 4. Color de fondo
    % 5. Coord x del titulo
    % 6. Coord y del titulo
    % 7. Fecha

    % Personaliza el formato del título
    \pretitle{\begin{center}\bfseries\fontsize{42}{56}\selectfont}
    \posttitle{\par\end{center}\vspace{2em}}
    
    % Personaliza el formato del autor
    \preauthor{\begin{center}\Large}
    \postauthor{\par\end{center}\vfill}
    
    % Personaliza el formato de la fecha
    \predate{\begin{center}\huge}
    \postdate{\par\end{center}\vspace{2em}}
    
    \title{#2}
    \author{\href{https://losdeldgiim.github.io/}{Los Del DGIIM}}
    \date{Granada, #7}
    \maketitle
    
    \tableofcontents
}




\newcommand{\portadaArticle}[7]{

    % 1. Foto de fondo
    % 2. Título
    % 3. Encabezado Izquierdo
    % 4. Color de fondo
    % 5. Coord x del titulo
    % 6. Coord y del titulo
    % 7. Fecha

    % Personaliza el formato del título
    \pretitle{\begin{center}\bfseries\fontsize{42}{56}\selectfont}
    \posttitle{\par\end{center}\vspace{2em}}
    
    % Personaliza el formato del autor
    \preauthor{\begin{center}\Large}
    \postauthor{\par\end{center}\vspace{3em}}
    
    % Personaliza el formato de la fecha
    \predate{\begin{center}\huge}
    \postdate{\par\end{center}\vspace{5em}}
    
    \title{#2}
    \author{\href{https://losdeldgiim.github.io/}{Los Del DGIIM}}
    \date{Granada, #7}
    \thispagestyle{empty}               % Sin encabezado ni pie de página
    \maketitle
    \vfill
}
    \portadaExamen{ffccA4.jpg}{Álgebra III\\Examen IV}{Álgebra III. Examen IV}{MidnightBlue}{-8}{28}{2025}{}

    \begin{description}
        \item[Asignatura] Álgebra III.
        \item[Curso Académico] 2023/24.
        \item[Grado] Doble Grado en Ingeniería Informática y Matemáticas.
        \item[Grupo] Único.
        \item[Profesor] José Gómez Torrecillas.
        \item[Descripción] Examen Ordinario de Incidencias.
        % \item[Fecha] 20 de noviembre de 2025.
        % \item[Duración] Una hora.
    \end{description}
    \newpage


    % ------------------------------------
    
    \begin{ejercicio}
        Sea $f=(x^4+1)(x^2-3)\in \mathbb{Q}[x]$ y $K$ el cuerpo de descomposición de $f$ sobre $\mathbb{Q}$:
        \begin{enumerate}[label=\alph*)]
            \item Describe todos los elementos de $\Aut(K)$.
            \item Comprueba que $w\in K$, con $w = \nicefrac{-1}{2}+i\nicefrac{\sqrt{3}}{2}$.
            \item Calcula $\Aut_{\mathbb{Q}(w)}(K)\cap \Aut_{\mathbb{Q}(\sqrt{3})}(K)$.
            \item Calcula los subcuerpos de $K$ de grado 4.
        \end{enumerate}
    \end{ejercicio}

    \begin{ejercicio}
        Sea $f=x^3+3x^2-x+1\in \mathbb{Q}[x]$ con $\alpha,\beta$ raíces reales de $f$. Calcular $[\mathbb{Q}(\alpha+\beta):\mathbb{Q}]$.
    \end{ejercicio}

    \begin{ejercicio}
        Sea $F$ un cuerpo con $\car(F)=2$, $a\in F$ con $F=\bb{F}_2(a)$ y $a^6 = a^5 + 1$.
        \begin{enumerate}[label=\alph*)]
            \item Calcular $\Aut(F)$.
            \item Encontrar un elemento $b$ y expresarlo en función de $a$ para que $|\bb{F}_2(b)| = 8$.
        \end{enumerate}
    \end{ejercicio}

    \newpage
    \setcounter{ejercicio}{0}
    \noindent
    \textbf{Solución.}

    \begin{ejercicio}
        Sea $f=(x^4+1)(x^2-3)\in \mathbb{Q}[x]$ y $K$ el cuerpo de descomposición de $f$ sobre $\mathbb{Q}$:
        \begin{enumerate}[label=\alph*)]
            \item Describe todos los elementos de $\Aut(K)$.

                Las raíces de $f$ son $\pm\sqrt{3},\pm \sqrt{i},\pm i\sqrt{i}$, donde vemos que:
                \begin{equation*}
                    \sqrt{i} = \frac{\sqrt{2}}{2}+i\frac{\sqrt{2}}{2}
                \end{equation*}
                Así, vemos que $K = \mathbb{Q}\left(\sqrt{3},\sqrt{i}\right)\leq \mathbb{Q}\left(\sqrt{3},\sqrt{2},i\right)$, pero esta última inclusión es una igualdad, pues:
                \begin{equation*}
                    i = {\left(\sqrt{i}\right)}^{2}\in \mathbb{Q}\left(\sqrt{3},\sqrt{i}\right) \qquad \sqrt{2} = \frac{\sqrt{i}}{\nicefrac{1}{2}+\nicefrac{i}{2}} \in \mathbb{Q}\left(\sqrt{3},\sqrt{i}\right)
                \end{equation*}
                Tenemos así que $K = \mathbb{Q}\left(\sqrt{3},\sqrt{2},i\right)$ y que $\mathbb{Q}\leq K$ es de Galois. Si calculamos ahora $[K:\mathbb{Q}]$ por el Lema de la Torre:
                \begin{equation*}
                    [K:\mathbb{Q}] = \left[K:\mathbb{Q}\left(\sqrt{3},\sqrt{2}\right)\right]\left[\mathbb{Q}\left(\sqrt{3},\sqrt{2}\right):\mathbb{Q}\left(\sqrt{3}\right)\right]\left[\mathbb{Q}\left(\sqrt{3}\right):\mathbb{Q}\right]
                \end{equation*}
                Vemos que:
                \begin{itemize}
                    \item $\left[\mathbb{Q}\left(\sqrt{3}\right):\mathbb{Q}\right] = 2$, ya que $x^2-3$ es irreducible en $\mathbb{Q}[x]$ por Eisenstein.
                    \item $\left[K:\mathbb{Q}\left(\sqrt{3},\sqrt{2}\right)\right]=2$, ya que $x^2+1$ es irreducible en $\mathbb{Q}\left(\sqrt{3},\sqrt{2}\right)[x]$ por ser sus raíces complejas.
                    \item $\left[\mathbb{Q}\left(\sqrt{3},\sqrt{2}\right):\mathbb{Q}(\sqrt{3})\right]\leq 2$ ya que $x^2-2$ tiene por raíz $\sqrt{2}$. Si fuera $\left[\mathbb{Q}\left(\sqrt{3},\sqrt{2}\right):\mathbb{Q}\left(\sqrt{3}\right)\right]=1$ tendríamos entonces que $\sqrt{2}\in \mathbb{Q}\left(\sqrt{3}\right)$, por lo que existirían $a,b\in \mathbb{Q}$ de forma que:
                        \begin{equation*}
                            \sqrt{2} = a+b\sqrt{3} \quad\Longrightarrow\quad 2 = a^2 + 2ab\sqrt{3} + 3b^2 \quad\Longleftrightarrow\quad \left\{\begin{array}{l}
                                2 = a^2 + 3b^2 \\
                                0 = 2ab
                            \end{array}\right.
                        \end{equation*}
                        De ser $b=0$ tendríamos que $\sqrt{2}\in \mathbb{Q}$, pero $x^2-2$ es irreducible en $\mathbb{Q}[x]$ por Eisenstein, por lo que tiene que ser $a=0$, de donde:
                        \begin{equation*}
                            2 = 3b^2 \quad\Longleftrightarrow\quad b = \sqrt{\frac{2}{3}}
                        \end{equation*}
                        Pero vemos que $\sqrt{\nicefrac{2}{3}}\notin \mathbb{Q}$, ya que el polinomio $3x^2-2$ es irreducible en $\mathbb{Q}[x]$ por ser de grado 2 y no tener raíces en $\mathbb{Q}$, ya que sus únicas posibiles raíces son:
                        \begin{equation*}
                            \pm 1, \pm 2, \pm\frac{1}{3}, \pm\frac{2}{3}
                        \end{equation*}
                        y ninguna lo es. Hemos llegado a una contradicción, que viene de suponer que $\left[\mathbb{Q}\left(\sqrt{3},\sqrt{2}\right):\mathbb{Q}\left(\sqrt{3}\right)\right]=1$, por lo que ha de ser 2.
                \end{itemize}
                En definitiva, tenemos que $x^2-2$ es irreducible en $\mathbb{Q}\left(\sqrt{3}\right)[x]$, de donde tenemos:
                \begin{equation*}
                    [K:\mathbb{Q}] = 2^3 = 8
                \end{equation*}
                Para calcular los elementos de $\Aut(K)$ lo que haremos será aplicar varias veces la Proposición de extensión. Comenzamos calculando los homomorfismos $\eta:\mathbb{Q}\left(\sqrt{3}\right)\to K$:
                \begin{figure}[H]
                    \centering
                    \shorthandoff{""}
                    \begin{tikzcd}
                        \mathbb{Q} \arrow[r, hook] \arrow[rd, hook] & K                    \\
                                                                    & \mathbb{Q}(\sqrt{3})
                    \end{tikzcd}
                    \shorthandon{""}
                \end{figure}
                \noindent
                Como $x^2-3\in \mathbb{Q}[x]$ es irreducible y tiene sus dos raíces $(\pm \sqrt{3})$ en $K$ tenemos que hay dos automorfismos, $\eta_j$ con $j\in \{0,1\}$, que vienen determinados por:
                \begin{equation*}
                    \sqrt{3} \stackrel{\eta_j}{\longmapsto} {(-1)}^{j}\sqrt{3}
                \end{equation*}
                Cada uno de estos homomorfismos puede extenderse a varios homomorfismos $\mathbb{Q}\left(\sqrt{3},\sqrt{2}\right)\to K$:
                \begin{figure}[H]
                    \centering
                    \shorthandoff{""}
                    \begin{tikzcd}
                        \mathbb{Q}(\sqrt{3}) \arrow[r, "\eta_j"] \arrow[rd, hook] & K                               \\
                                                                                  & {\mathbb{Q}(\sqrt{3},\sqrt{2})}
                    \end{tikzcd}
                    \shorthandon{""}
                \end{figure}
                \noindent
                Ya que $x^2-2\in \mathbb{Q}\left(\sqrt{3}\right)[x]$ es irreducible y tiene sus dos raíces en $K$, tenemos que cada $\eta_j$ se extiende a dos homomorfismos $\eta_{j,k}$ con $k \in \{0,1\}$, que vienen determinados por:
                \begin{align*}
                    \sqrt{3} &\stackrel{\eta_{j,k}}{\longmapsto} {(-1)}^{j}\sqrt{3} \\
                    \sqrt{2} &\stackrel{\eta_{j,k}}{\longmapsto} {(-1)}^{k}\sqrt{2}
                \end{align*}
                Cada uno puede extenderse a su vez a homomorfismos $K\to K$:
                \begin{figure}[H]
                    \centering
                    \shorthandoff{""}
                    \begin{tikzcd}
                        {\mathbb{Q}(\sqrt{3},\sqrt{2})} \arrow[r, "\eta_{j,k}"] \arrow[rd, hook] & K \\
                                                                                             & K
                    \end{tikzcd}
                    \shorthandon{""}
                \end{figure}
                \noindent
                Ya que $x^2+1\in \mathbb{Q}\left(\sqrt{3},\sqrt{2}\right)[x]$ es irreducible y tiene sus dos raíces en $K$, obteniendo para cada $j,k\in \{0,1\}\times \{0,1\}$ los automorfismos $\eta_{j,k,l}$ con $l \in \{0,1\}$ determinados por:
                \begin{align*}
                    \sqrt{3} &\stackrel{\eta_{j,k,l}}{\longmapsto} {(-1)}^{j}\sqrt{3} \\
                    \sqrt{2} &\stackrel{\eta_{j,k,l}}{\longmapsto} {(-1)}^{k}\sqrt{2} \\
                    i &\stackrel{\eta_{j,k,l}}{\longmapsto} {(-1)}^{l}i
                \end{align*}
                Así, tenemos que:
                \begin{equation*}
                    \Aut(K) = \left\{\eta_{j,k,l}:j,k,l\in \{0,1\}\right\}
                \end{equation*}
            \item Comprueba que $w\in K$, con $w = \nicefrac{-1}{2}+i\nicefrac{\sqrt{3}}{2}$.

                Como $K = \mathbb{Q}\left(\sqrt{3},\sqrt{2},i\right)$ es claro que $w\in K$.
            \item Calcula $\Aut_{\mathbb{Q}(w)}(K)\cap \Aut_{\mathbb{Q}(\sqrt{3})}(K)$.

                Calculamos primero $\Aut_{\mathbb{Q}(w)}(K)$ y $\Aut_{\mathbb{Q}(\sqrt{3})}(K)$:
                \begin{itemize}
                    \item Para el primero, vemos que:
                        \begin{equation*}
                            |\Aut_{\mathbb{Q}(w)}(K)| = (\Aut_{\mathbb{Q}(w)}(K):\{id\}) = [K:\mathbb{Q}(w)] = \frac{[K:\mathbb{Q}]}{[\mathbb{Q}(w):\mathbb{Q}]} = \frac{8}{2} = 4
                        \end{equation*}
                        Puesto que $\Irr(w,\mathbb{Q}) = \phi_3 = x^2+x+1$. Así como que los automorfismos $\eta_{000}, \eta_{010}, \eta_{101}, \eta_{111}$ dejan fijo el elemento $w$, por lo que:
                        \begin{equation*}
                            \{\eta_{000}, \eta_{010}, \eta_{101}, \eta_{111}\} \subseteq \Aut_{\mathbb{Q}(w)}(K)
                        \end{equation*}
                        Pero como $|\Aut_{\mathbb{Q}(w)}(K)| = 4$ estos son todos.
                    \item Y para el segundo de forma análoga se tiene que:
                        \begin{equation*}
                            \left|\Aut_{\mathbb{Q}\left(\sqrt{3}\right)}(K)\right| = \left(\Aut_{\mathbb{Q}\left(\sqrt{3}\right)}(K):\{id\}\right) = \left[K:\mathbb{Q}\left(\sqrt{3}\right)\right] = \frac{8}{2} = 4
                        \end{equation*}
                        Vemos de la misma forma que:
                        \begin{equation*}
                            \Aut_{\mathbb{Q}\left(\sqrt{3}\right)}(K) = \{\eta_{000},\eta_{001},\eta_{010},\eta_{011}\}
                        \end{equation*}
                \end{itemize}
                En definitiva, tenemos que:
                \begin{equation*}
                    \Aut_{\mathbb{Q}(w)}(K)\cap \Aut_{\mathbb{Q}(\sqrt{3})}(K) = \{\eta_{000}, \eta_{010}\}
                \end{equation*}
            \item Calcula los subcuerpos de $K$ de grado 4.

                Viendo la definición de $\eta_{j,k,l}$ para $j,k,l\in \{0,1\}$ observamos que todos estos elementos son de orden multiplicativo 2. Sea $L\leq K$ con $[L:\mathbb{Q}] = 4$, tenemos entonces que ($G = \Aut(K)$):
                \begin{equation*}
                    4 = [L:\mathbb{Q}] = [G:\Aut_L(K)] = \frac{|G|}{|\Aut_L(K)|} \quad\Longrightarrow\quad |\Aut_L(K)| = 2
                \end{equation*}
                Es decir, que cada subcuerpo de grado 4 da un subgrupo de $\Aut(K)$ de orden 2 y sabemos por la conexión de Galois que al revés también sucede esto, por lo que buscamos calcular los subgrupos de $\Aut(K)$ de orden 2, que sabemos que se corresponden con los generados por los elementos de orden 2, obteniendo así 7 subgrupos de orden 2:
                \begin{equation*}
                    \langle \eta_{001} \rangle , \quad \langle \eta_{010} \rangle , \quad \langle \eta_{011} \rangle , \quad \langle \eta_{100} \rangle , \quad \langle \eta_{101} \rangle , \quad \langle \eta_{110} \rangle , \quad \langle \eta_{111} \rangle 
                \end{equation*}
                Detallaremos la obtención del subcuerpo asociado al primer subgrupo y el resto son análogos.
                \begin{itemize}
                    \item Para $\langle \eta_{001} \rangle $ observamos que $\mathbb{Q}\left(\sqrt{3},\sqrt{2}\right)\leq K^{\langle \eta_{001} \rangle }$, pues $\sqrt{3}$ y $\sqrt{2}$ quedan fijos por este automorfismo. Como tenemos que:
                        \begin{equation*}
                            \left[K^{\langle \eta_{001} \rangle }:\mathbb{Q}\right] = \frac{|G|}{|\langle \eta_{001} \rangle |} = \frac{8}{2} = 4
                        \end{equation*}
                        Y anteriormente vimos que:
                        \begin{equation*}
                            \left[\mathbb{Q}\left(\sqrt{3},\sqrt{2}\right):\mathbb{Q}\right] = 4
                        \end{equation*}
                        Tenemos por tanto que $\mathbb{Q}\left(\sqrt{3},\sqrt{2}\right)=K^{\langle \eta_{001} \rangle }$, por lo que $\mathbb{Q}\left(\sqrt{3},\sqrt{2}\right)$ es un subcuerpo de $K$ de grado 4.
                    \item $K^{\langle \eta_{010} \rangle } = \mathbb{Q}\left(\sqrt{3},i\right)$.
                    \item $K^{\langle \eta_{011} \rangle } = \mathbb{Q}\left(\sqrt{3},i\sqrt{2}\right)$.
                    \item $K^{\langle \eta_{100} \rangle } = \mathbb{Q}\left(i,\sqrt{2}\right)$.
                    \item $K^{\langle \eta_{101} \rangle } = \mathbb{Q}\left(i\sqrt{3},\sqrt{2}\right)$.
                    \item $K^{\langle \eta_{110} \rangle } = \mathbb{Q}\left(\sqrt{6},i\right)$.
                    \item $K^{\langle \eta_{111} \rangle } = \mathbb{Q}\left(i\sqrt{2},\sqrt{6}\right)$.
                \end{itemize}
        \end{enumerate}
    \end{ejercicio}

    \begin{ejercicio}
        Sea $f=x^3+3x^2-x+1\in \mathbb{Q}[x]$ con $\alpha,\beta$ raíces reales de $f$. Calcular $[\mathbb{Q}(\alpha+\beta):\mathbb{Q}]$.\\

        \noindent
        Sea $\gamma$ la tercera raíz de $f$ en un cuerpo de descomposición, en vista de los coeficientes de $f$ y las relaciones de Cardano-Vieta, tenemos que:
        \begin{equation*}
            -3 = \alpha + \beta + \gamma \quad\Longleftrightarrow\quad  \alpha+\beta = -3-\gamma
        \end{equation*}
        De donde deducimos que $\mathbb{Q}(\gamma) = \mathbb{Q}(\alpha+\beta)$. Observemos que $f$ es irreducible, pues es de grado 3 y no tiene raíces en $\mathbb{Q}$, ya que las únicas posibles son $\pm 1$ y ninguna de ellas es raíz:
        \begin{equation*}
            f(1) = 4, \quad f(-1) = 4
        \end{equation*}
        Tenemos así que $f=\Irr(\gamma,\mathbb{Q})$, por lo que:
        \begin{equation*}
            3 = [\mathbb{Q}(\gamma):\mathbb{Q}] = [\mathbb{Q}(\alpha+\beta):\mathbb{Q}]
        \end{equation*}
    \end{ejercicio}

    \begin{ejercicio}
        Sea $F$ un cuerpo con $\car(F)=2$, $a\in F$ con $F=\bb{F}_2(a)$ y $a^6 = a^5 + 1$.
        \begin{enumerate}[label=\alph*)]
            \item Calcular $\Aut(F)$.

                Sea $f=x^6+x^5+1\in \bb{F}_2[x]$, estudiemos la irreducibilidad de $f$:
                \begin{itemize}
                    \item $f$ no tiene raíces en $\bb{F}_2[x]$, pues $f(0) = 1 = f(1)$, de donde no tiene factores de grado 1 ni factores de grado 5.
                    \item El único polinomio irreducible de grado 2 en $\bb{F}_2[x]$ es $x^2+x+1$, si dividimos $f$ entre él vemos que:
                        \begin{equation*}
                            x^6+x^5+1 = (x^2+x+1)(x^4+x^2+1) + x+1
                        \end{equation*}
                        Por lo que $f$ no tiene factores de grado 2, luego tampoco de grado 4.
                    \item $f$ puede tener factores de grado 3, y los únicos polinomios irreducibles de grado 3 en $\bb{F}_2[x]$ son:
                        \begin{equation*}
                            x^3+x^2+1, \quad x^3+x+1
                        \end{equation*}
                        Vemos que:
                        \begin{equation*}
                            x^6+x^5+1 = (x^3+x^2+1)(x^3+1) + x^2
                        \end{equation*}
                        Por lo que $x^3+x^2+1$ no es un factor de $f$, la única posibilidad restante para que $f$ tenga factores de grado 3 es que sea igual a:
                        \begin{equation*}
                            {(x^3+x+1)}^{2} = x^6+x^2+1
                        \end{equation*}
                        pero no es el caso, por lo que $f$ no tiene factores de grado 3.
                \end{itemize}
                Concluimos que $f$ es irreducible. Tenemos así que $f=\Irr(a,\bb{F}_2)$, con lo que:
                \begin{equation*}
                    [\bb{F}_2(a):\bb{F}_2] = 6 \quad\Longrightarrow\quad \bb{F}_2(a) = \bb{F}_{64}
                \end{equation*}
                Sabemos que $\Aut(F)$ es un grupo cíclico de orden 6, generado por el automorfismos de Frobenius $\tau_2:\bb{F}_2(a)\to \bb{F}_2(a)$ determinado por $\tau_2(a) = a^2$, con lo que el grupo es:
                \begin{equation*}
                    \Aut(F) = \{\tau_1,\tau_2,\tau_4,\tau_8,\tau_{16},\tau_{32}\}
                \end{equation*}
                donde $\tau_j$ viene dado por $\tau_j(a) = a^j$, para $j \in \{1,2,4,8,16,32\}$.
            \item Encontrar un elemento $b$ y expresarlo en función de $a$ para que $|\bb{F}_2(b)| = 8$.

                Como $[\bb{F}_2(a):\bb{F}_2] = 6$, tenemos que $\{1,a,a^2,a^3,a^4,a^5\}$ es una $\bb{F}_2-$base de $\bb{F}_2(a)$, $\bb{F}_2(a)^\times$ es un grupo de orden $63 = 3^2\cdot 7$, por lo que $a$ puede tener órden $3,7,9,21$ o $63$:
                \begin{itemize}
                    \item $a^3\neq 1 \Longrightarrow O(a)\neq 3$.
                    \item $a^7 = a(a^5+1) = a^6+a = a^5 + 1 + a \neq 1 \Longrightarrow O(a)\neq 7$.
                    \item $a^9 = a^2a^7 = a^2(a^5+a+1) = a^7 + a^3 + a^2 = a(a^5+1) + a^3+a^2 = a^5+1+a+a^3+a^2\neq 1 \Longrightarrow O(a)\neq 9$.
                    \item $a^{21} = a^3{(a^9)}^{2} = a^3{(a^5+a^3+a^2+a+1)}^{2} = a^3(a^{10} + a^6 + a^4 + a^2 + 1) = a^3(a^4(a^5+1)+a^5+1+a^4+a^2+1) = a^3(a+a^3+1) = a^4 + a^6 + a^3 = a^4 + a^5 + 1 + a^3\neq 1 \Longrightarrow O(a)\neq 21$.
                \end{itemize}
                Vemos así que $O(a)=63$. Buscamos un elemento $b\in \bb{F}_2(a)$ de forma que $\bb{F}_2(b)=\bb{F}_8$. Sabemos que $\bb{F}_8$ es cuerpo de descomposición de un polinomio de grado 3 sobre $\bb{F}_2$, y sabemos que los elementos de $\bb{F}_{64}$ son las raíces de $x^{64}-x$, cuyos factores son polinomios irreducibles cuyo grado divide a $6$, por lo que tanto las raíces de $x^3+x+1$ como de $x^3+x^2+1$ están en $\bb{F}_{64}$. Sea $b\in \bb{F}_2(a)$ la solución a cualquiera de estas ecuaciones, obtendríamos así que $\bb{F}_2(b)$ es cuerpo de descomposición de dicho polinomio, con lo que $\bb{F}_8 = \bb{F}_2(b)$. Observamos que $\bb{F}_2(b)^\times$ es un grupo cíclico de orden 7, por lo que $b$ tiene que tener orden multiplicativo 7. Observamos que elementos de esta forma en $\bb{F}_2(a)$ hay 6 (todos y cada uno de los elementos del grupo cíclico de orden 7, menos 1), 3 corresponden a las soluciones de $x^3+x+1$ y otros 3 a las de $x^3+x^2+1$. De esta forma, al tomar cualquier elemento $b\in \bb{F}_2(a)$ de orden multiplicativo 3 obtenemos que $\bb{F}_2(b) = \bb{F}_8$.\\

                \noindent
                Como $a$ tiene orden $63$, tomando:
                \begin{equation*}
                    b = a^9
                \end{equation*}
                obtenemos que $O(b) = 7$, por lo que por el razonamiento que hemos mostrado tenemos que $\bb{F}_2(b) = \bb{F}_8$.
        \end{enumerate}
    \end{ejercicio}
\end{document}
