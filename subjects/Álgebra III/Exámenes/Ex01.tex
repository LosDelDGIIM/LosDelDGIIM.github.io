\documentclass[12pt]{article}

% Idioma y codificación
\usepackage[spanish, es-tabla]{babel}       %es-tabla para que se titule "Tabla"
\usepackage[utf8]{inputenc}

% Márgenes
\usepackage[a4paper,top=3cm,bottom=2.5cm,left=3cm,right=3cm]{geometry}

% Comentarios de bloque
\usepackage{verbatim}

% Paquetes de links
\usepackage[hidelinks]{hyperref}    % Permite enlaces
\usepackage{url}                    % redirecciona a la web

% Más opciones para enumeraciones
\usepackage{enumitem}

% Personalizar la portada
\usepackage{titling}

% Paquetes de tablas
\usepackage{multirow}


%------------------------------------------------------------------------

%Paquetes de figuras
\usepackage{caption}
\usepackage{subcaption} % Figuras al lado de otras
\usepackage{float}      % Poner figuras en el sitio indicado H.


% Paquetes de imágenes
\usepackage{graphicx}       % Paquete para añadir imágenes
\usepackage{transparent}    % Para manejar la opacidad de las figuras

% Paquete para usar colores
\usepackage[dvipsnames]{xcolor}
\usepackage{pagecolor}      % Para cambiar el color de la página

% Habilita tamaños de fuente mayores
\usepackage{fix-cm}

% Para los gráficos
\usepackage{tikz}

% Para poder situar los nodos en los grafos
\usetikzlibrary{positioning}


%------------------------------------------------------------------------

% Paquetes de matemáticas
\usepackage{mathtools, amsfonts, amssymb, mathrsfs}
\usepackage[makeroom]{cancel}     % Simplificar tachando
\usepackage{polynom}    % Divisiones y Ruffini
\usepackage{units} % Para poner fracciones diagonales con \nicefrac

\usepackage{pgfplots}   %Representar funciones
\pgfplotsset{compat=1.18}  % Versión 1.18

\usepackage{tikz-cd}    % Para usar diagramas de composiciones
\usetikzlibrary{calc}   % Para usar cálculo de coordenadas en tikz

%Definición de teoremas, etc.
\usepackage{amsthm}
%\swapnumbers   % Intercambia la posición del texto y de la numeración

\theoremstyle{plain}

\makeatletter
\@ifclassloaded{article}{
  \newtheorem{teo}{Teorema}[section]
}{
  \newtheorem{teo}{Teorema}[chapter]  % Se resetea en cada chapter
}
\makeatother

\newtheorem{coro}{Corolario}[teo]           % Se resetea en cada teorema
\newtheorem{prop}[teo]{Proposición}         % Usa el mismo contador que teorema
\newtheorem{lema}[teo]{Lema}                % Usa el mismo contador que teorema

\theoremstyle{remark}
\newtheorem*{observacion}{Observación}

\theoremstyle{definition}

\makeatletter
\@ifclassloaded{article}{
  \newtheorem{definicion}{Definición} [section]     % Se resetea en cada chapter
}{
  \newtheorem{definicion}{Definición} [chapter]     % Se resetea en cada chapter
}
\makeatother

\newtheorem*{notacion}{Notación}
\newtheorem*{ejemplo}{Ejemplo}
\newtheorem*{ejercicio*}{Ejercicio}             % No numerado
\newtheorem{ejercicio}{Ejercicio} [section]     % Se resetea en cada section


% Modificar el formato de la numeración del teorema "ejercicio"
\renewcommand{\theejercicio}{%
  \ifnum\value{section}=0 % Si no se ha iniciado ninguna sección
    \arabic{ejercicio}% Solo mostrar el número de ejercicio
  \else
    \thesection.\arabic{ejercicio}% Mostrar número de sección y número de ejercicio
  \fi
}


% \renewcommand\qedsymbol{$\blacksquare$}         % Cambiar símbolo QED
%------------------------------------------------------------------------

% Paquetes para encabezados
\usepackage{fancyhdr}
\pagestyle{fancy}
\fancyhf{}

\newcommand{\helv}{ % Modificación tamaño de letra
\fontfamily{}\fontsize{12}{12}\selectfont}
\setlength{\headheight}{15pt} % Amplía el tamaño del índice


%\usepackage{lastpage}   % Referenciar última pag   \pageref{LastPage}
\fancyfoot[C]{\thepage}

%------------------------------------------------------------------------

% Conseguir que no ponga "Capítulo 1". Sino solo "1."
\makeatletter
\@ifclassloaded{book}{
  \renewcommand{\chaptermark}[1]{\markboth{\thechapter.\ #1}{}} % En el encabezado
    
  \renewcommand{\@makechapterhead}[1]{%
  \vspace*{50\p@}%
  {\parindent \z@ \raggedright \normalfont
    \ifnum \c@secnumdepth >\m@ne
      \huge\bfseries \thechapter.\hspace{1em}\ignorespaces
    \fi
    \interlinepenalty\@M
    \Huge \bfseries #1\par\nobreak
    \vskip 40\p@
  }}
}
\makeatother

%------------------------------------------------------------------------
% Paquetes de cógido
\usepackage{minted}
\renewcommand\listingscaption{Código fuente}

\usepackage{fancyvrb}
% Personaliza el tamaño de los números de línea
\renewcommand{\theFancyVerbLine}{\small\arabic{FancyVerbLine}}

% Estilo para C++
\newminted{cpp}{
    frame=lines,
    framesep=2mm,
    baselinestretch=1.2,
    linenos,
    escapeinside=||
}

% para minted
\definecolor{LightGray}{rgb}{0.95,0.95,0.92}
\setminted{
    linenos=true,
    stepnumber=5,
    numberfirstline=true,
    autogobble,
    breaklines=true,
    breakautoindent=true,
    breaksymbolleft=,
    breaksymbolright=,
    breaksymbolindentleft=0pt,
    breaksymbolindentright=0pt,
    breaksymbolsepleft=0pt,
    breaksymbolsepright=0pt,
    fontsize=\footnotesize,
    bgcolor=LightGray,
    numbersep=10pt
}


\usepackage{listings} % Para incluir código desde un archivo

\renewcommand\lstlistingname{Código Fuente}
\renewcommand\lstlistlistingname{Índice de Códigos Fuente}

% Definir colores
\definecolor{vscodepurple}{rgb}{0.5,0,0.5}
\definecolor{vscodeblue}{rgb}{0,0,0.8}
\definecolor{vscodegreen}{rgb}{0,0.5,0}
\definecolor{vscodegray}{rgb}{0.5,0.5,0.5}
\definecolor{vscodebackground}{rgb}{0.97,0.97,0.97}
\definecolor{vscodelightgray}{rgb}{0.9,0.9,0.9}

% Configuración para el estilo de C similar a VSCode
\lstdefinestyle{vscode_C}{
  backgroundcolor=\color{vscodebackground},
  commentstyle=\color{vscodegreen},
  keywordstyle=\color{vscodeblue},
  numberstyle=\tiny\color{vscodegray},
  stringstyle=\color{vscodepurple},
  basicstyle=\scriptsize\ttfamily,
  breakatwhitespace=false,
  breaklines=true,
  captionpos=b,
  keepspaces=true,
  numbers=left,
  numbersep=5pt,
  showspaces=false,
  showstringspaces=false,
  showtabs=false,
  tabsize=2,
  frame=tb,
  framerule=0pt,
  aboveskip=10pt,
  belowskip=10pt,
  xleftmargin=10pt,
  xrightmargin=10pt,
  framexleftmargin=10pt,
  framexrightmargin=10pt,
  framesep=0pt,
  rulecolor=\color{vscodelightgray},
  backgroundcolor=\color{vscodebackground},
}

%------------------------------------------------------------------------

% Comandos definidos
\newcommand{\bb}[1]{\mathbb{#1}}
\newcommand{\cc}[1]{\mathcal{#1}}

% I prefer the slanted \leq
\let\oldleq\leq % save them in case they're every wanted
\let\oldgeq\geq
\renewcommand{\leq}{\leqslant}
\renewcommand{\geq}{\geqslant}

% Si y solo si
\newcommand{\sii}{\iff}

% Letras griegas
\newcommand{\eps}{\epsilon}
\newcommand{\veps}{\varepsilon}
\newcommand{\lm}{\lambda}

\newcommand{\ol}{\overline}
\newcommand{\ul}{\underline}
\newcommand{\wt}{\widetilde}
\newcommand{\wh}{\widehat}

\let\oldvec\vec
\renewcommand{\vec}{\overrightarrow}

% Derivadas parciales
\newcommand{\del}[2]{\frac{\partial #1}{\partial #2}}
\newcommand{\Del}[3]{\frac{\partial^{#1} #2}{\partial #3^{#1}}}
\newcommand{\deld}[2]{\dfrac{\partial #1}{\partial #2}}
\newcommand{\Deld}[3]{\dfrac{\partial^{#1} #2}{\partial #3^{#1}}}


\newcommand{\AstIg}{\stackrel{(\ast)}{=}}
\newcommand{\Hop}{\stackrel{L'H\hat{o}pital}{=}}

\newcommand{\red}[1]{{\color{red}#1}} % Para integrales, destacar los cambios.

% Método de integración
\newcommand{\MetInt}[2]{
    \left[\begin{array}{c}
        #1 \\ #2
    \end{array}\right]
}

% Declarar aplicaciones
% 1. Nombre aplicación
% 2. Dominio
% 3. Codominio
% 4. Variable
% 5. Imagen de la variable
\newcommand{\Func}[5]{
    \begin{equation*}
        \begin{array}{rrll}
            #1:& #2 & \longrightarrow & #3\\
               & #4 & \longmapsto & #5
        \end{array}
    \end{equation*}
}

%------------------------------------------------------------------------


\DeclareMathOperator{\Irr}{Irr}
\DeclareMathOperator{\Subgr}{Subgr}
\DeclareMathOperator{\Subex}{Subex}
\DeclareMathOperator{\car}{car}
\DeclareMathOperator{\Aut}{Aut}
\DeclareMathOperator{\Disc}{Disc}
\DeclareMathOperator{\Sim}{Sim}

\begin{document}

    % 1. Foto de fondo
    % 2. Título
    % 3. Encabezado Izquierdo
    % 4. Color de fondo
    % 5. Coord x del titulo
    % 6. Coord y del titulo
    % 7. Fecha

    
    % 1. Foto de fondo
% 2. Título
% 3. Encabezado Izquierdo
% 4. Color de fondo
% 5. Coord x del titulo
% 6. Coord y del titulo
% 7. Fecha

\newcommand{\portada}[7]{

    \portadaBase{#1}{#2}{#3}{#4}{#5}{#6}{#7}
    \portadaBook{#1}{#2}{#3}{#4}{#5}{#6}{#7}
}

\newcommand{\portadaExamen}[7]{

    \portadaBase{#1}{#2}{#3}{#4}{#5}{#6}{#7}
    \portadaArticle{#1}{#2}{#3}{#4}{#5}{#6}{#7}
}




\newcommand{\portadaBase}[7]{

    % Tiene la portada principal y la licencia Creative Commons
    
    % 1. Foto de fondo
    % 2. Título
    % 3. Encabezado Izquierdo
    % 4. Color de fondo
    % 5. Coord x del titulo
    % 6. Coord y del titulo
    % 7. Fecha
    
    
    \thispagestyle{empty}               % Sin encabezado ni pie de página
    \newgeometry{margin=0cm}        % Márgenes nulos para la primera página
    
    
    % Encabezado
    \fancyhead[L]{\helv #3}
    \fancyhead[R]{\helv \nouppercase{\leftmark}}
    
    
    \pagecolor{#4}        % Color de fondo para la portada
    
    \begin{figure}[p]
        \centering
        \transparent{0.3}           % Opacidad del 30% para la imagen
        
        \includegraphics[width=\paperwidth, keepaspectratio]{assets/#1}
    
        \begin{tikzpicture}[remember picture, overlay]
            \node[anchor=north west, text=white, opacity=1, font=\fontsize{60}{90}\selectfont\bfseries\sffamily, align=left] at (#5, #6) {#2};
            
            \node[anchor=south east, text=white, opacity=1, font=\fontsize{12}{18}\selectfont\sffamily, align=right] at (9.7, 3) {\textbf{\href{https://losdeldgiim.github.io/}{Los Del DGIIM}}};
            
            \node[anchor=south east, text=white, opacity=1, font=\fontsize{12}{15}\selectfont\sffamily, align=right] at (9.7, 1.8) {Doble Grado en Ingeniería Informática y Matemáticas\\Universidad de Granada};
        \end{tikzpicture}
    \end{figure}
    
    
    \restoregeometry        % Restaurar márgenes normales para las páginas subsiguientes
    \pagecolor{white}       % Restaurar el color de página
    
    
    \newpage
    \thispagestyle{empty}               % Sin encabezado ni pie de página
    \begin{tikzpicture}[remember picture, overlay]
        \node[anchor=south west, inner sep=3cm] at (current page.south west) {
            \begin{minipage}{0.5\paperwidth}
                \href{https://creativecommons.org/licenses/by-nc-nd/4.0/}{
                    \includegraphics[height=2cm]{assets/Licencia.png}
                }\vspace{1cm}\\
                Esta obra está bajo una
                \href{https://creativecommons.org/licenses/by-nc-nd/4.0/}{
                    Licencia Creative Commons Atribución-NoComercial-SinDerivadas 4.0 Internacional (CC BY-NC-ND 4.0).
                }\\
    
                Eres libre de compartir y redistribuir el contenido de esta obra en cualquier medio o formato, siempre y cuando des el crédito adecuado a los autores originales y no persigas fines comerciales. 
            \end{minipage}
        };
    \end{tikzpicture}
    
    
    
    % 1. Foto de fondo
    % 2. Título
    % 3. Encabezado Izquierdo
    % 4. Color de fondo
    % 5. Coord x del titulo
    % 6. Coord y del titulo
    % 7. Fecha


}


\newcommand{\portadaBook}[7]{

    % 1. Foto de fondo
    % 2. Título
    % 3. Encabezado Izquierdo
    % 4. Color de fondo
    % 5. Coord x del titulo
    % 6. Coord y del titulo
    % 7. Fecha

    % Personaliza el formato del título
    \pretitle{\begin{center}\bfseries\fontsize{42}{56}\selectfont}
    \posttitle{\par\end{center}\vspace{2em}}
    
    % Personaliza el formato del autor
    \preauthor{\begin{center}\Large}
    \postauthor{\par\end{center}\vfill}
    
    % Personaliza el formato de la fecha
    \predate{\begin{center}\huge}
    \postdate{\par\end{center}\vspace{2em}}
    
    \title{#2}
    \author{\href{https://losdeldgiim.github.io/}{Los Del DGIIM}}
    \date{Granada, #7}
    \maketitle
    
    \tableofcontents
}




\newcommand{\portadaArticle}[7]{

    % 1. Foto de fondo
    % 2. Título
    % 3. Encabezado Izquierdo
    % 4. Color de fondo
    % 5. Coord x del titulo
    % 6. Coord y del titulo
    % 7. Fecha

    % Personaliza el formato del título
    \pretitle{\begin{center}\bfseries\fontsize{42}{56}\selectfont}
    \posttitle{\par\end{center}\vspace{2em}}
    
    % Personaliza el formato del autor
    \preauthor{\begin{center}\Large}
    \postauthor{\par\end{center}\vspace{3em}}
    
    % Personaliza el formato de la fecha
    \predate{\begin{center}\huge}
    \postdate{\par\end{center}\vspace{5em}}
    
    \title{#2}
    \author{\href{https://losdeldgiim.github.io/}{Los Del DGIIM}}
    \date{Granada, #7}
    \thispagestyle{empty}               % Sin encabezado ni pie de página
    \maketitle
    \vfill
}
    \portadaExamen{ffccA4.jpg}{Álgebra III\\Examen I}{Álgebra III. Examen I}{MidnightBlue}{-8}{28}{2025}{José Juan Urrutia Milán}

    \begin{description}
        \item[Asignatura] Álgebra III.
        \item[Curso Académico] 2025/26.
        \item[Grado] Doble Grado en Ingeniería Informática y Matemáticas.
        \item[Grupo] Único.
        \item[Profesor] José Gómez Torrecillas.
        \item[Descripción] Segundo examen sorpresa.
        \item[Fecha] 20 de noviembre de 2025.
        \item[Duración] Una hora.
    
    \end{description}
    \newpage


    % ------------------------------------
    
    \begin{ejercicio}
        Calcular $\phi_9\in \mathbb{Q}[x]$.\\

        \noindent
        \textbf{Solución.}\newline
        Sabemos que:
        \begin{equation*}
            x^9-1 = \prod_{d \in Div(9)} \phi_d = \phi_1\phi_3\phi_9
        \end{equation*}
        Por lo que:
        \begin{equation*}
            \phi_9 = \frac{x^9-1}{\phi_1\phi_3}
        \end{equation*}
        Tenemos que $\phi_1 = x-1$, y para $\phi_3$ tenemos que las raíces cúbicas de la unidad son:
        \begin{equation*}
            \{1,w,w^2\}
        \end{equation*}
        con $w$ y $w^2$ las raíces cúbicas primitivas de la unidad. $1,w,w^2$ son todas las raíces dle polinomio $x^3-1$, y queremos quitar 1 como raíz, por lo que dividimos entre $x-1$, obteniendo:
        \begin{equation*}
            x^3-1 = (x-1)(x^2+x+1)
        \end{equation*}
        Con $w$ y $w^2$ las raíces de $x^2+x+1$, por lo que este polinomio es $\phi^3$. Así:
        \begin{equation*}
            \phi_9 = \frac{x^9-1}{\phi_1\phi_3} = \frac{x^9-1}{(x-1)(x^2+x+1)} = \frac{x^9-1}{x^3-1}
        \end{equation*}
        Y vemos que:
        \begin{equation*}
            x^9-1 = (x^3-1)(x^6+x^3+1)
        \end{equation*}
        de donde deducimos que $\phi_9 = x^6+x^3+1$.
    \end{ejercicio}

    \begin{ejercicio}
        Sea $\zeta\in \mathbb{C}$ una raíz primitiva novena de la unidad, calcular todos los subcuerpos de $\mathbb{Q}(\zeta)$.\\

        \noindent
        \textbf{Solución.}\newline
        Sabemos que $\mathbb{Q}\leq \mathbb{Q}(\zeta)$ es de Galois, que $\Aut(\mathbb{Q}(\zeta)) \cong \cc{U}(\mathbb{Z}_9)$ y que:
        \begin{equation*}
            [\mathbb{Q}(\zeta):\mathbb{Q}] = \varphi(9) = 6
        \end{equation*}
        Además, vimos que los elementos de $\Aut(\mathbb{Q}(\zeta))$ son:
        \begin{equation*}
            \Aut(\mathbb{Q}(\zeta)) = \{\tau_1, \tau_2, \tau_4, \tau_5, \tau_7, \tau_8\}
        \end{equation*}
        y que vienen determinados por:
        \begin{equation*}
            \zeta \stackrel{\tau_j}{\longmapsto} \zeta^j \qquad \forall j\in \cc{U}(\mathbb{Z}_9) = \{1,2,4,5,7,8\}
        \end{equation*}
        Si calculamos el orden de $\tau_2$ vemos que:
        \begin{equation*}
            \zeta \stackrel{\tau_2}{\longmapsto} \zeta^2 \stackrel{\tau_2}{\longmapsto} \zeta^4 \stackrel{\tau_2}{\longmapsto} \zeta^8
        \end{equation*}
        por lo que el orden de $\tau_2$ no es ni 1 ni 2 ni 3, luego ha de ser un elemento de orden 6, con lo que $\langle \tau_2 \rangle = \Aut(\mathbb{Q}(\zeta)) $. Una vez conocemos el generador del grupo es sencillo calcular los órdenes del resto de elementos:
        \begin{itemize}
            \item Para los elementos de orden 2 (sabemos que solo hay uno, puesto que el grupo es cíclico y por tanto solo puede haber un subgrupo de orden 2, que están en correspondencia biunívoca con los elementos de orden 2), vemos que $\tau_2^3$ es un elemento de orden 2, y como:
                \begin{equation*}
                    \zeta \stackrel{\tau_2^3}{\longmapsto} \zeta^8
                \end{equation*}
                vemos por tanto que $\tau_2^3 = \tau_8$
            \item Para los elementos de orden 3 (sabemos que hay 2, y que una vez obtenido uno el otro es su cuadrado) tenemos que $\tau_2^2$ será un elemento de orden 3, y como:
                \begin{equation*}
                    \zeta \stackrel{\tau_2^2}{\longmapsto} \zeta^4
                \end{equation*}
                tenemos por tanto que $\tau_2^2 = \tau_4$, y el otro elemento de orden 3 será $\tau_4^2$, que verifica:
                \begin{equation*}
                    \zeta \stackrel{\tau_4}{\longmapsto} \zeta^4 \stackrel{\tau_4}{\longmapsto} \zeta^{16} = \zeta^7
                \end{equation*}
            \item Sabemos que el orden de $\tau_1$ es uno, y el resto de elementos que no hemos considerado serán de orden 6.
        \end{itemize}
        Tenemos así que los órdenes de los elementos son:
        \begin{table}[H]
        \centering
        \begin{tabular}{cccccc}
            $\tau_1$ & $\tau_2$ & $\tau_4$ & $\tau_5$ & $\tau_7$ & $\tau_8$  \\
            \hline 
            1 & 6 & 3 & 6 & 3 & 2
        \end{tabular}
        \end{table}
        \noindent
        Y podemos ya por tanto determinar todos los subgrupos de $\Aut(\mathbb{Q}(\zeta))$:
        \begin{figure}[H]
            \centering
            \shorthandoff{""}
            \begin{tikzcd}
                                                                                 & Aut(\mathbb{Q}(\zeta)) = \langle\tau_2\rangle = \langle\tau_5\rangle \arrow[rdd, no head] \arrow[ld, no head] &                                          \\
                \langle\tau_4\rangle = \langle\tau_7\rangle \arrow[rdd, no head] &                                                                                                               &                                          \\
                                                                                 &                                                                                                               & \langle\tau_8\rangle \arrow[ld, no head] \\
                                                                                                                                                  & \langle\tau_1\rangle                                                                                          &                                         
            \end{tikzcd}
            \shorthandon{""}
        \end{figure}
        \noindent
        Para calcular todos los subcuerpos de $\mathbb{Q}(\zeta)$ lo que hacemos es buscar las subextensiones de $\mathbb{Q}\leq \mathbb{Q}(\zeta)$ que son correspondientes mediante la conexión de Galois a los subgrupos $\langle \tau_4 \rangle $ y $\langle \tau_8 \rangle $ de $\Aut(\mathbb{Q}(\zeta))$. Para ello, una opción es buscar los elementos que dejan fijos $\tau_4$ y $\tau_8$:
        \begin{itemize}
            \item Para la subextensión asociada a $\langle \tau_4 \rangle $, como $|\langle \tau_4 \rangle | = 3$ tenemos por tanto que:
                \begin{equation*}
                    (\Aut(\mathbb{Q}(\zeta)) : \langle \tau_4 \rangle ) = 2
                \end{equation*}
                y por la conexión de Galois:
                \begin{equation*}
                    2 = (\Aut(\mathbb{Q}(\zeta)) : \langle \tau_4 \rangle ) = [K^{\langle \tau_4 \rangle } : \mathbb{Q}]
                \end{equation*}
                Por lo que basta buscar una subextensión de $\mathbb{Q}\leq K^{\langle \tau_4 \rangle }$ con grado 2 sobre $\mathbb{Q}$.

                Observemos que si $\zeta$ es una raíz novena primitiva de la unidad tenemos entones que $\zeta^3$ es una raíz cúbica de la unidad, que ha de ser primitiva, pues $\zeta^3\neq 1$. De esta forma, vemos que $\zeta^3$ es raíz del polinomio irreducible:
                \begin{equation*}
                    \phi_3 = x^2+x+1
                \end{equation*}
                de donde $[\mathbb{Q}(\zeta^3):\mathbb{Q}] = 2$. Si aplicamos ahora $\tau_4$ a $\zeta_3$ vemos que:
                \begin{equation*}
                    \zeta^3 \stackrel{\tau_4}{\longmapsto} \zeta^{12} = \zeta^3
                \end{equation*}
                de donde $\mathbb{Q}\leq \mathbb{Q}(\zeta^3)\leq K^{\langle \tau_4 \rangle }$, y viendo el grado de $\mathbb{Q}(\zeta^3)$ la única posibilidad es $\mathbb{Q}(\zeta^3) = K^{\langle \tau_4 \rangle }$.
            \item Para la subextensión asociada a $\langle \tau_8 \rangle $, como $|\langle \tau_8 \rangle | = 2$ tenemos que:
                \begin{equation*}
                    [K^{\langle \tau_8 \rangle }:\mathbb{Q}] = (\Aut(\mathbb{Q}(\zeta)):\langle \tau_8 \rangle ) = 3
                \end{equation*}
                Por lo que basta buscar una subextensión de $\mathbb{Q}\leq K^{\langle \tau_8 \rangle }$ con grado 3 sobre $\mathbb{Q}$.

                \begin{description}
                    \item [Opción 1.] Observemos que:
                        \begin{equation*}
                            \zeta^9 = 1 \quad\Longrightarrow\quad \zeta^8 = \zeta^{-1}
                        \end{equation*}
                        Si calculamos $\tau_8(\zeta^8)$ vemos que:
                        \begin{equation*}
                            \zeta^8 \stackrel{\tau_8}{\longmapsto} \zeta^{64} = \zeta
                        \end{equation*}
                        y recordemos que:
                        \begin{equation*}
                            \zeta \stackrel{\tau_8}{\longmapsto} \zeta
                        \end{equation*}
                        por lo que vemos que el elemento $\zeta+\zeta^8$ queda fijo por $\tau_8$:
                        \begin{equation*}
                            \zeta + \zeta^8 \stackrel{\tau_8}{\longmapsto} \zeta^8 + \zeta
                        \end{equation*}
                        Así, tenemos que:
                        \begin{equation*}
                            \mathbb{Q}\leq \mathbb{Q}(\zeta + \zeta^8) \leq K^{\langle \tau_8 \rangle }
                        \end{equation*}
                        Como $[K^{\langle \tau_8 \rangle }:\mathbb{Q}] = 3$, será $[\mathbb{Q}(\zeta+\zeta^8):\mathbb{Q}] \in \{1,3\}$, de donde bien:
                        \begin{equation*}
                            \mathbb{Q}(\zeta+\zeta^8) = \mathbb{Q} \quad \text{ó} \quad \mathbb{Q}(\zeta+\zeta^8) = K^{\langle \tau_8 \rangle }
                        \end{equation*}
                        Por reducción al absurdo, supuesto que $\mathbb{Q}(\zeta+\zeta^8) = \mathbb{Q}$, tendremos entonces que $\zeta+\zeta^8 \in \mathbb{Q}$. En este caso, usando el ejercicio anterior vimos que $\phi_9 = x^6+x^3+1$, de donde:
                        \begin{equation*}
                            \zeta^6 = -1-\zeta^3
                        \end{equation*}
                        Usando esta relación vemos que:
                        \begin{equation*}
                            \mathbb{Q}\ni \zeta + \zeta^8 = \zeta + \zeta^2\zeta^6 = \zeta + \zeta^2(-1-\zeta^3) = \zeta - \zeta^2 - \zeta^5
                        \end{equation*}
                        Pero como $[\mathbb{Q}(\zeta):\mathbb{Q}] = 6$ tenemos que el conjunto $\{1,\zeta, \zeta^2, \zeta^3, \zeta^4, \zeta^5\}$ son $\mathbb{Q}-$linealmente independientes, por lo que no puede ser $\zeta-\zeta^2-\zeta^5\in \mathbb{Q}$, hemos llegado a una contradicción, de donde teníamos que:
                        \begin{equation*}
                            K^{\langle \tau_8 \rangle } = \mathbb{Q}(\zeta + \zeta^8)
                        \end{equation*}
                    \item [Opción 2.] Como $[\mathbb{Q}(\zeta):\mathbb{Q}] = 6$, tenemos que una $\mathbb{Q}-$base de $\mathbb{Q}(\zeta)$ es:
                        \begin{equation*}
                            \{1,\zeta,\zeta^2,\zeta^3,\zeta^4, \zeta^5\}
                        \end{equation*}
                        Así, buscamos las condiciones que ha de cumplir cualquier elemento de $\mathbb{Q}(\zeta)$ para quedar fijo por $\tau_8$. De esta forma, sean $a,b,c,d,e,f\in \mathbb{Q}$, calculamos:
                        \begin{align*}
                            a + b\zeta + c\zeta^2 + d\zeta^3 + e\zeta^4 + f\zeta^5 &\stackrel{\tau_8}{\longmapsto} a + b\zeta^8 + c\zeta^{16} + d\zeta^{24} + e\zeta^{32} + f\zeta^{40} \\
                                                                                   &= a + b\zeta^8 + c\zeta^7 + d\zeta^6 + e\zeta^5 + f\zeta^4
                        \end{align*}
                        Si imponemos que el elemento sea igual a su imagen, buscamos igualar término a término usando que las potencias de $\zeta$ hasta $5$ son una $\mathbb{Q}-$base. Sin embargo, vemos que nos aparecen potencias de $\zeta$ mayores que $5$. Conseguiremos reducirlas si recordamos el ejercicio anterior, pues sabemos que $\zeta$ es raíz de $\phi_9$, con lo que:
                        \begin{equation*}
                            \zeta^6 + \zeta^3 + 1 = 0 \quad\Longrightarrow\quad \zeta^6 = -1-\zeta^3
                        \end{equation*}
                        así, tenemos que:
                        \begin{align*}
                            &a + b\zeta + c\zeta^2 + d\zeta^3 + e\zeta^4 + f\zeta^5 \\ &= a + b\zeta^8 + c\zeta^{16} + d\zeta^{24} + e\zeta^{32} + f\zeta^{40} \\
                                                                                   &= a + b\zeta^8 + c\zeta^7 + d\zeta^6 + e\zeta^5 + f\zeta^4 \\
                                                                                   &= a + b\zeta^2(-1-\zeta^3) + c\zeta(-1-\zeta^3) + d(-1-\zeta^3) + e\zeta^5 + f\zeta^4 \\
                                                                                   &= (a-d) + \zeta(-c) + \zeta^2(-b) + \zeta^3(-d) + \zeta^4(-c+f) + \zeta^5(-b+e)
                        \end{align*}
                        Y si usamos que el conjunto anterior era una base, podemos igualar coeficiente a coeficiente, obteniendo las relaciones:
                        \begin{equation*}
                            \left.\begin{array}{l}
                                a = a-d \\
                                b = -c \\
                                c = -b \\
                                d = -d \\
                                e = -c+f \\
                                f = -b+e
                        \end{array}\right\} \quad\Longleftrightarrow\quad \left\{\begin{array}{l}
                            b = -c \\
                            d = 0 \\
                            e = -c+f \\
                            f = -b+e
                        \end{array}\right.
                        \end{equation*}
                        Si probamos a tomar $a=0 = d$, $b = 1$, $c=-1$ nos queda:
                        \begin{equation*}
                            \left\{\begin{array}{l}
                                e = 1 + f \\
                                f = -1 + e
                            \end{array}\right.
                        \end{equation*}
                        por lo que una elección es tomar $f = 0$ y $e = 1$. Obtenemos por tanto el elemento:
                        \begin{equation*}
                            \zeta - \zeta^2 + \zeta^4
                        \end{equation*}
                        que ya sabemos que queda fijo por $\tau_8$, pero que comprobamos para ver que no nos hayamos equivocado en ninguna cuenta:
                        \begin{align*}
                            \zeta - \zeta^2 + \zeta^4 &\stackrel{\tau_8}{\longmapsto} \zeta^8 - \zeta^{16} + \zeta^{32} \\
                                                      &= \zeta^8 - \zeta^7 + \zeta^5 \\
                                                      &= \zeta^2(-1-\zeta^3) - \zeta(-1-\zeta^3) + \zeta^5 \\
                                                      &= -\zeta^2 \cancel{-\zeta^5} + \zeta + \zeta^4 \cancel{+\zeta^5}
                        \end{align*}
                        Efectivamente, tenemos así que:
                        \begin{equation*}
                            \mathbb{Q}\leq \mathbb{Q}(\zeta -\zeta^2 + \zeta^4) \leq K^{\langle \tau_8 \rangle }
                        \end{equation*}
                        Con $[K^{\langle \tau_8 \rangle }:\mathbb{Q}] = 3$, por lo que tenemos:
                        \begin{equation*}
                            \mathbb{Q} = \mathbb{Q}(\zeta-\zeta^2+\zeta^4) \quad \text{ó}\quad K^{\langle \tau_8 \rangle } = \mathbb{Q}(\zeta-\zeta^2+\zeta^4)
                        \end{equation*}
                        Por reducción al absurdo, suponemos que $\mathbb{Q}(\zeta-\zeta^2+\zeta^4) = \mathbb{Q}$, con lo que tenemos que:
                        \begin{equation*}
                            \zeta - \zeta^2 + \zeta^4 \in \mathbb{Q}
                        \end{equation*}
                        Pero teníamos que $\{\zeta,\zeta^2,\zeta^4\}$ eran $\mathbb{Q}-$linealmente independientes, por lo que hemos llegado a una contradicción, luego tenemos que:
                        \begin{equation*}
                            K^{\langle \tau_8 \rangle } = \mathbb{Q}(\zeta-\zeta^2+\zeta^4)
                        \end{equation*}
                \end{description}
        \end{itemize}
        Hemos obtenido así todos los subcuerpos de $\mathbb{Q}(\zeta)$, obteniendo:
        \begin{figure}[H]
            \centering
            \shorthandoff{""}
            \begin{tikzcd}
                                                         & \mathbb{Q} \arrow[rdd, no head] \arrow[ld, no head] &                                                       \\
                \mathbb{Q}(\zeta^3) \arrow[rdd, no head] &                                                     &                                                       \\
                                                         &                                                     & \mathbb{Q}(\zeta-\zeta^2+\zeta^4) \arrow[ld, no head] \\
                                                                                                  & \mathbb{Q}(\zeta)                                   &                                                      
            \end{tikzcd}
            \shorthandon{""}
        \end{figure}
    \end{ejercicio}
\end{document}
