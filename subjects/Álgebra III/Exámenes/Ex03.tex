\documentclass[12pt]{article}

% Idioma y codificación
\usepackage[spanish, es-tabla]{babel}       %es-tabla para que se titule "Tabla"
\usepackage[utf8]{inputenc}

% Márgenes
\usepackage[a4paper,top=3cm,bottom=2.5cm,left=3cm,right=3cm]{geometry}

% Comentarios de bloque
\usepackage{verbatim}

% Paquetes de links
\usepackage[hidelinks]{hyperref}    % Permite enlaces
\usepackage{url}                    % redirecciona a la web

% Más opciones para enumeraciones
\usepackage{enumitem}

% Personalizar la portada
\usepackage{titling}

% Paquetes de tablas
\usepackage{multirow}


%------------------------------------------------------------------------

%Paquetes de figuras
\usepackage{caption}
\usepackage{subcaption} % Figuras al lado de otras
\usepackage{float}      % Poner figuras en el sitio indicado H.


% Paquetes de imágenes
\usepackage{graphicx}       % Paquete para añadir imágenes
\usepackage{transparent}    % Para manejar la opacidad de las figuras

% Paquete para usar colores
\usepackage[dvipsnames]{xcolor}
\usepackage{pagecolor}      % Para cambiar el color de la página

% Habilita tamaños de fuente mayores
\usepackage{fix-cm}

% Para los gráficos
\usepackage{tikz}

% Para poder situar los nodos en los grafos
\usetikzlibrary{positioning}


%------------------------------------------------------------------------

% Paquetes de matemáticas
\usepackage{mathtools, amsfonts, amssymb, mathrsfs}
\usepackage[makeroom]{cancel}     % Simplificar tachando
\usepackage{polynom}    % Divisiones y Ruffini
\usepackage{units} % Para poner fracciones diagonales con \nicefrac

\usepackage{pgfplots}   %Representar funciones
\pgfplotsset{compat=1.18}  % Versión 1.18

\usepackage{tikz-cd}    % Para usar diagramas de composiciones
\usetikzlibrary{calc}   % Para usar cálculo de coordenadas en tikz

%Definición de teoremas, etc.
\usepackage{amsthm}
%\swapnumbers   % Intercambia la posición del texto y de la numeración

\theoremstyle{plain}

\makeatletter
\@ifclassloaded{article}{
  \newtheorem{teo}{Teorema}[section]
}{
  \newtheorem{teo}{Teorema}[chapter]  % Se resetea en cada chapter
}
\makeatother

\newtheorem{coro}{Corolario}[teo]           % Se resetea en cada teorema
\newtheorem{prop}[teo]{Proposición}         % Usa el mismo contador que teorema
\newtheorem{lema}[teo]{Lema}                % Usa el mismo contador que teorema

\theoremstyle{remark}
\newtheorem*{observacion}{Observación}

\theoremstyle{definition}

\makeatletter
\@ifclassloaded{article}{
  \newtheorem{definicion}{Definición} [section]     % Se resetea en cada chapter
}{
  \newtheorem{definicion}{Definición} [chapter]     % Se resetea en cada chapter
}
\makeatother

\newtheorem*{notacion}{Notación}
\newtheorem*{ejemplo}{Ejemplo}
\newtheorem*{ejercicio*}{Ejercicio}             % No numerado
\newtheorem{ejercicio}{Ejercicio} [section]     % Se resetea en cada section


% Modificar el formato de la numeración del teorema "ejercicio"
\renewcommand{\theejercicio}{%
  \ifnum\value{section}=0 % Si no se ha iniciado ninguna sección
    \arabic{ejercicio}% Solo mostrar el número de ejercicio
  \else
    \thesection.\arabic{ejercicio}% Mostrar número de sección y número de ejercicio
  \fi
}


% \renewcommand\qedsymbol{$\blacksquare$}         % Cambiar símbolo QED
%------------------------------------------------------------------------

% Paquetes para encabezados
\usepackage{fancyhdr}
\pagestyle{fancy}
\fancyhf{}

\newcommand{\helv}{ % Modificación tamaño de letra
\fontfamily{}\fontsize{12}{12}\selectfont}
\setlength{\headheight}{15pt} % Amplía el tamaño del índice


%\usepackage{lastpage}   % Referenciar última pag   \pageref{LastPage}
\fancyfoot[C]{\thepage}

%------------------------------------------------------------------------

% Conseguir que no ponga "Capítulo 1". Sino solo "1."
\makeatletter
\@ifclassloaded{book}{
  \renewcommand{\chaptermark}[1]{\markboth{\thechapter.\ #1}{}} % En el encabezado
    
  \renewcommand{\@makechapterhead}[1]{%
  \vspace*{50\p@}%
  {\parindent \z@ \raggedright \normalfont
    \ifnum \c@secnumdepth >\m@ne
      \huge\bfseries \thechapter.\hspace{1em}\ignorespaces
    \fi
    \interlinepenalty\@M
    \Huge \bfseries #1\par\nobreak
    \vskip 40\p@
  }}
}
\makeatother

%------------------------------------------------------------------------
% Paquetes de cógido
\usepackage{minted}
\renewcommand\listingscaption{Código fuente}

\usepackage{fancyvrb}
% Personaliza el tamaño de los números de línea
\renewcommand{\theFancyVerbLine}{\small\arabic{FancyVerbLine}}

% Estilo para C++
\newminted{cpp}{
    frame=lines,
    framesep=2mm,
    baselinestretch=1.2,
    linenos,
    escapeinside=||
}

% para minted
\definecolor{LightGray}{rgb}{0.95,0.95,0.92}
\setminted{
    linenos=true,
    stepnumber=5,
    numberfirstline=true,
    autogobble,
    breaklines=true,
    breakautoindent=true,
    breaksymbolleft=,
    breaksymbolright=,
    breaksymbolindentleft=0pt,
    breaksymbolindentright=0pt,
    breaksymbolsepleft=0pt,
    breaksymbolsepright=0pt,
    fontsize=\footnotesize,
    bgcolor=LightGray,
    numbersep=10pt
}


\usepackage{listings} % Para incluir código desde un archivo

\renewcommand\lstlistingname{Código Fuente}
\renewcommand\lstlistlistingname{Índice de Códigos Fuente}

% Definir colores
\definecolor{vscodepurple}{rgb}{0.5,0,0.5}
\definecolor{vscodeblue}{rgb}{0,0,0.8}
\definecolor{vscodegreen}{rgb}{0,0.5,0}
\definecolor{vscodegray}{rgb}{0.5,0.5,0.5}
\definecolor{vscodebackground}{rgb}{0.97,0.97,0.97}
\definecolor{vscodelightgray}{rgb}{0.9,0.9,0.9}

% Configuración para el estilo de C similar a VSCode
\lstdefinestyle{vscode_C}{
  backgroundcolor=\color{vscodebackground},
  commentstyle=\color{vscodegreen},
  keywordstyle=\color{vscodeblue},
  numberstyle=\tiny\color{vscodegray},
  stringstyle=\color{vscodepurple},
  basicstyle=\scriptsize\ttfamily,
  breakatwhitespace=false,
  breaklines=true,
  captionpos=b,
  keepspaces=true,
  numbers=left,
  numbersep=5pt,
  showspaces=false,
  showstringspaces=false,
  showtabs=false,
  tabsize=2,
  frame=tb,
  framerule=0pt,
  aboveskip=10pt,
  belowskip=10pt,
  xleftmargin=10pt,
  xrightmargin=10pt,
  framexleftmargin=10pt,
  framexrightmargin=10pt,
  framesep=0pt,
  rulecolor=\color{vscodelightgray},
  backgroundcolor=\color{vscodebackground},
}

%------------------------------------------------------------------------

% Comandos definidos
\newcommand{\bb}[1]{\mathbb{#1}}
\newcommand{\cc}[1]{\mathcal{#1}}

% I prefer the slanted \leq
\let\oldleq\leq % save them in case they're every wanted
\let\oldgeq\geq
\renewcommand{\leq}{\leqslant}
\renewcommand{\geq}{\geqslant}

% Si y solo si
\newcommand{\sii}{\iff}

% Letras griegas
\newcommand{\eps}{\epsilon}
\newcommand{\veps}{\varepsilon}
\newcommand{\lm}{\lambda}

\newcommand{\ol}{\overline}
\newcommand{\ul}{\underline}
\newcommand{\wt}{\widetilde}
\newcommand{\wh}{\widehat}

\let\oldvec\vec
\renewcommand{\vec}{\overrightarrow}

% Derivadas parciales
\newcommand{\del}[2]{\frac{\partial #1}{\partial #2}}
\newcommand{\Del}[3]{\frac{\partial^{#1} #2}{\partial #3^{#1}}}
\newcommand{\deld}[2]{\dfrac{\partial #1}{\partial #2}}
\newcommand{\Deld}[3]{\dfrac{\partial^{#1} #2}{\partial #3^{#1}}}


\newcommand{\AstIg}{\stackrel{(\ast)}{=}}
\newcommand{\Hop}{\stackrel{L'H\hat{o}pital}{=}}

\newcommand{\red}[1]{{\color{red}#1}} % Para integrales, destacar los cambios.

% Método de integración
\newcommand{\MetInt}[2]{
    \left[\begin{array}{c}
        #1 \\ #2
    \end{array}\right]
}

% Declarar aplicaciones
% 1. Nombre aplicación
% 2. Dominio
% 3. Codominio
% 4. Variable
% 5. Imagen de la variable
\newcommand{\Func}[5]{
    \begin{equation*}
        \begin{array}{rrll}
            #1:& #2 & \longrightarrow & #3\\
               & #4 & \longmapsto & #5
        \end{array}
    \end{equation*}
}

%------------------------------------------------------------------------


\DeclareMathOperator{\Irr}{Irr}
\DeclareMathOperator{\Subgr}{Subgr}
\DeclareMathOperator{\Subex}{Subex}
\DeclareMathOperator{\car}{car}
\DeclareMathOperator{\Aut}{Aut}
\DeclareMathOperator{\Disc}{Disc}
\DeclareMathOperator{\Sim}{Sim}

\begin{document}

    % 1. Foto de fondo
    % 2. Título
    % 3. Encabezado Izquierdo
    % 4. Color de fondo
    % 5. Coord x del titulo
    % 6. Coord y del titulo
    % 7. Fecha

    
    % 1. Foto de fondo
% 2. Título
% 3. Encabezado Izquierdo
% 4. Color de fondo
% 5. Coord x del titulo
% 6. Coord y del titulo
% 7. Fecha

\newcommand{\portada}[7]{

    \portadaBase{#1}{#2}{#3}{#4}{#5}{#6}{#7}
    \portadaBook{#1}{#2}{#3}{#4}{#5}{#6}{#7}
}

\newcommand{\portadaExamen}[7]{

    \portadaBase{#1}{#2}{#3}{#4}{#5}{#6}{#7}
    \portadaArticle{#1}{#2}{#3}{#4}{#5}{#6}{#7}
}




\newcommand{\portadaBase}[7]{

    % Tiene la portada principal y la licencia Creative Commons
    
    % 1. Foto de fondo
    % 2. Título
    % 3. Encabezado Izquierdo
    % 4. Color de fondo
    % 5. Coord x del titulo
    % 6. Coord y del titulo
    % 7. Fecha
    
    
    \thispagestyle{empty}               % Sin encabezado ni pie de página
    \newgeometry{margin=0cm}        % Márgenes nulos para la primera página
    
    
    % Encabezado
    \fancyhead[L]{\helv #3}
    \fancyhead[R]{\helv \nouppercase{\leftmark}}
    
    
    \pagecolor{#4}        % Color de fondo para la portada
    
    \begin{figure}[p]
        \centering
        \transparent{0.3}           % Opacidad del 30% para la imagen
        
        \includegraphics[width=\paperwidth, keepaspectratio]{assets/#1}
    
        \begin{tikzpicture}[remember picture, overlay]
            \node[anchor=north west, text=white, opacity=1, font=\fontsize{60}{90}\selectfont\bfseries\sffamily, align=left] at (#5, #6) {#2};
            
            \node[anchor=south east, text=white, opacity=1, font=\fontsize{12}{18}\selectfont\sffamily, align=right] at (9.7, 3) {\textbf{\href{https://losdeldgiim.github.io/}{Los Del DGIIM}}};
            
            \node[anchor=south east, text=white, opacity=1, font=\fontsize{12}{15}\selectfont\sffamily, align=right] at (9.7, 1.8) {Doble Grado en Ingeniería Informática y Matemáticas\\Universidad de Granada};
        \end{tikzpicture}
    \end{figure}
    
    
    \restoregeometry        % Restaurar márgenes normales para las páginas subsiguientes
    \pagecolor{white}       % Restaurar el color de página
    
    
    \newpage
    \thispagestyle{empty}               % Sin encabezado ni pie de página
    \begin{tikzpicture}[remember picture, overlay]
        \node[anchor=south west, inner sep=3cm] at (current page.south west) {
            \begin{minipage}{0.5\paperwidth}
                \href{https://creativecommons.org/licenses/by-nc-nd/4.0/}{
                    \includegraphics[height=2cm]{assets/Licencia.png}
                }\vspace{1cm}\\
                Esta obra está bajo una
                \href{https://creativecommons.org/licenses/by-nc-nd/4.0/}{
                    Licencia Creative Commons Atribución-NoComercial-SinDerivadas 4.0 Internacional (CC BY-NC-ND 4.0).
                }\\
    
                Eres libre de compartir y redistribuir el contenido de esta obra en cualquier medio o formato, siempre y cuando des el crédito adecuado a los autores originales y no persigas fines comerciales. 
            \end{minipage}
        };
    \end{tikzpicture}
    
    
    
    % 1. Foto de fondo
    % 2. Título
    % 3. Encabezado Izquierdo
    % 4. Color de fondo
    % 5. Coord x del titulo
    % 6. Coord y del titulo
    % 7. Fecha


}


\newcommand{\portadaBook}[7]{

    % 1. Foto de fondo
    % 2. Título
    % 3. Encabezado Izquierdo
    % 4. Color de fondo
    % 5. Coord x del titulo
    % 6. Coord y del titulo
    % 7. Fecha

    % Personaliza el formato del título
    \pretitle{\begin{center}\bfseries\fontsize{42}{56}\selectfont}
    \posttitle{\par\end{center}\vspace{2em}}
    
    % Personaliza el formato del autor
    \preauthor{\begin{center}\Large}
    \postauthor{\par\end{center}\vfill}
    
    % Personaliza el formato de la fecha
    \predate{\begin{center}\huge}
    \postdate{\par\end{center}\vspace{2em}}
    
    \title{#2}
    \author{\href{https://losdeldgiim.github.io/}{Los Del DGIIM}}
    \date{Granada, #7}
    \maketitle
    
    \tableofcontents
}




\newcommand{\portadaArticle}[7]{

    % 1. Foto de fondo
    % 2. Título
    % 3. Encabezado Izquierdo
    % 4. Color de fondo
    % 5. Coord x del titulo
    % 6. Coord y del titulo
    % 7. Fecha

    % Personaliza el formato del título
    \pretitle{\begin{center}\bfseries\fontsize{42}{56}\selectfont}
    \posttitle{\par\end{center}\vspace{2em}}
    
    % Personaliza el formato del autor
    \preauthor{\begin{center}\Large}
    \postauthor{\par\end{center}\vspace{3em}}
    
    % Personaliza el formato de la fecha
    \predate{\begin{center}\huge}
    \postdate{\par\end{center}\vspace{5em}}
    
    \title{#2}
    \author{\href{https://losdeldgiim.github.io/}{Los Del DGIIM}}
    \date{Granada, #7}
    \thispagestyle{empty}               % Sin encabezado ni pie de página
    \maketitle
    \vfill
}
    \portadaExamen{ffccA4.jpg}{Álgebra III\\Examen III}{Álgebra III. Examen III}{MidnightBlue}{-8}{28}{2026}{}

    \begin{description}
        \item[Asignatura] Álgebra III.
        \item[Curso Académico] 2023/24.
        \item[Grado] Doble Grado en Ingeniería Informática y Matemáticas.
        \item[Grupo] Único.
        \item[Profesor] José Gómez Torrecillas.
        \item[Descripción] Examen Ordinario.
        % \item[Fecha] 20 de noviembre de 2025.
        % \item[Duración] Una hora.
    \end{description}
    \newpage


    % ------------------------------------
    
    \begin{ejercicio}
        Sea $f = (x^3-2)(x^2-3)\in \mathbb{Q}[x]$ y $K$ el cuerpo de descomposición de $f$:
        \begin{enumerate}[label=\alph*)]
            \item Comprobar que $i+\sqrt{3}\in K$.
            \item Calcular $[K:\mathbb{Q}]$.
        \end{enumerate}
    \end{ejercicio}

    \begin{ejercicio}
        Sea $f=x^3-3x+1\in \mathbb{Q}[x]$ siendo $\alpha$ una raíz real de $f$. Probar que el cuerpo de descomposición de $f$ sobre $\mathbb{Q}$ es $\mathbb{Q}(\alpha)$.
    \end{ejercicio}

    \begin{ejercicio}
        Sea $F$ un cuerpo con $\car(F) =3$ y con un elemento $a\in F$ con $F=\bb{F}_3(a)$ con $a^4+a-1=0$.
        \begin{enumerate}[label=\alph*)]
            \item Describir $\Aut(F)$ y evaluarlos en $a^2$.
            \item Calcular el cardinal de $\bb{F}_3(a^2)$.
        \end{enumerate}
    \end{ejercicio}

    \begin{ejercicio}
        Responda razonadamente si las siguientes afirmaciones son verdaderas o falsas.
        \begin{enumerate}[label=\alph*)]
            \item Si $F\leq E\leq K$ con $F\leq E$ y $E\leq K$ extensiones de Galois, entonces $F\leq K$ es de Galois.
            \item Si $z\in \mathbb{C}$ tiene grado 4 sobre $\mathbb{Q}$ entonces $z$ es construible.
        \end{enumerate}
    \end{ejercicio}

    \newpage
    \setcounter{ejercicio}{0}
    \noindent
    \textbf{Solución.}

    \begin{ejercicio}
        Sea $f = (x^3-2)(x^2-3)\in \mathbb{Q}[x]$ y $K$ el cuerpo de descomposición de $f$:
        \begin{enumerate}[label=\alph*)]
            \item Comprobar que $i+\sqrt{3}\in K$.

                Las raíces de $f$ son $\pm\sqrt{3},w^k \sqrt[3]{2}$ para $k=0,1,2$ y donde $w$ es una raíz cúbica primitiva de la unidad. Podemos tomar por ejemplo:
                \begin{equation*}
                    w = e^{\frac{2\pi i}{3}} = \frac{-1}{2}+i\frac{\sqrt{3}}{2}
                \end{equation*}
                Por lo que $K = \mathbb{Q}\left(\sqrt{3},\sqrt[3]{2},w\sqrt[3]{2},w^2\sqrt[3]{2}\right)$. Vemos que:
                \begin{equation*}
                    w = \frac{w\sqrt[3]{2}}{\sqrt[3]{2}} \in \mathbb{Q}\left(\sqrt[3]{2},w\right)
                \end{equation*}
                Por lo que $K = \mathbb{Q}\left(\sqrt{3},\sqrt[3]{2},w\right)$. Más aún, vemos que:
                \begin{equation*}
                    K = \mathbb{Q}\left(\sqrt{3},\sqrt[3]{2},i\right)
                \end{equation*}
                Ya que:
                \begin{description}
                    \item [$\subseteq )$] $w = \frac{-1}{2}+i\frac{\sqrt{3}}{2}\in \mathbb{Q}\left(\sqrt{3},\sqrt[3]{2},i\right)$.
                    \item [$\supseteq )$] $i = \frac{2}{\sqrt{3}}\left(w+\frac{1}{2}\right)\in \mathbb{Q}\left(\sqrt{3},\sqrt[3]{2},w\right)$.
                \end{description}
                Con esta descripción de $K$ es claro que:
                \begin{equation*}
                    i+\sqrt{3} \in K = \mathbb{Q}\left(\sqrt{3},\sqrt[3]{2},i\right)
                \end{equation*}
            \item Calcular $[K:\mathbb{Q}]$.
                Por el Lema de la Torre tenemos que:
                \begin{equation*}
                    [K:\mathbb{Q}] = \left[K:\mathbb{Q}\left(\sqrt{3},\sqrt[3]{2}\right)\right]\left[\mathbb{Q}\left(\sqrt{3},\sqrt[3]{2}\right):\mathbb{Q}\right]
                \end{equation*}
                donde:
                \begin{itemize}
                    \item $\left[K:\mathbb{Q}\left(\sqrt{3},\sqrt[3]{2}\right)\right] = 2$ ya que $x^2+1\in \mathbb{Q}\left(\sqrt{3},\sqrt[3]{2}\right)[x]$ es un polinomio irreducible por ser sus dos raíces complejas.
                    \item Comprobamos ahora que $\left[\mathbb{Q}\left(\sqrt{3},\sqrt[3]{2}\right):\mathbb{Q}\right]=6$, ya que el Lema de la Torre nos permite escribir:
                        \begin{equation*}
                            \left[\mathbb{Q}\left(\sqrt{3},\sqrt[3]{2}\right):\mathbb{Q}\right] = \left[\mathbb{Q}\left(\sqrt{3},\sqrt[3]{2}\right):\mathbb{Q}\left(\sqrt{3}\right)\right]\left[\mathbb{Q}\left(\sqrt{3}\right):\mathbb{Q}\right]
                        \end{equation*}
                        donde:
                        \begin{itemize}
                            \item $\left[\mathbb{Q}\left(\sqrt{3},\sqrt[3]{2}\right):\mathbb{Q}\left(\sqrt{3}\right)\right]\leq 3$ ya que $\sqrt[3]{2}$ es raíz de $x^3-2$.
                            \item $\left[\mathbb{Q}\left(\sqrt{3}\right):\mathbb{Q}\right] = 2$, ya que $\Irr\left(\sqrt{3},\mathbb{Q}\right) = x^2-3$, que es irreducible por Eisenstein para $p=3$.

                                Deducimos por tanto que $\left[\mathbb{Q}\left(\sqrt{3},\sqrt[3]{2}\right):\mathbb{Q}\right] \leq 6$ y es múltiplo de 2.
                        \end{itemize}
                        Aplicando el Lema de la Torre en sentido opuesto y usando que $x^3-2$ es irreducible en $\mathbb{Q}[x]$ para $p=2$ por Eisenstein vemos que $\left[\mathbb{Q}\left(\sqrt{3},\sqrt[3]{2}\right):\mathbb{Q}\right]$ es múltiplo de 3 también, por lo que no queda más salida que:
                        \begin{equation*}
                            \left[\mathbb{Q}\left(\sqrt{3},\sqrt[3]{2}\right):\mathbb{Q}\right] = 6
                        \end{equation*}
                \end{itemize}
                Así, tenemos que:
                \begin{equation*}
                    [K:\mathbb{Q}] = \left[K:\mathbb{Q}\left(\sqrt{3},\sqrt[3]{2}\right)\right]\left[\mathbb{Q}\left(\sqrt{3},\sqrt[3]{2}\right):\mathbb{Q}\right] = 2\cdot 6 = 12
                \end{equation*}
        \end{enumerate}
    \end{ejercicio}

    \begin{ejercicio}
        Sea $f=x^3-3x+1\in \mathbb{Q}[x]$ siendo $\alpha$ una raíz real de $f$. Probar que el cuerpo de descomposición de $f$ sobre $\mathbb{Q}$ es $\mathbb{Q}(\alpha)$.\\

        \noindent
        Vemos que $f$ es irreducible en $\mathbb{Q}[x]$ por ser de grado 3 y no tener raíces en $\mathbb{Q}$, ya que las únicas posibles raíces de $f$ en $\mathbb{Q}$ son $\pm 1$ y ninguna de ellas es raíz:
        \begin{equation*}
            f(0) = 1, \qquad f(1) = -1
        \end{equation*}
        Sea $K$ el cuerpo de descomposición de $f$, como $\car(K) = 0$ vemos que $\mathbb{Q}\leq K$ es de Galois y si consideramos $G=\Aut_F(K)$ el grupo de Galois de $f$ vemos que $G$ es un ``subgrupo'' de $S_3$ que actúa de forma transitiva sobre las raíces de $f$ (por ser $f$ irreducible), por lo que $G\cong S_3$ ó $G\cong A_3$. Si calculamos:
        \begin{equation*}
            \Disc(f) = -4p^3 - 27q^2 = 4\cdot 3^3 - 27 = 4\cdot 27 - 27 = 3\cdot 27 = 81
        \end{equation*}
        Vemos que $81 = 3^4$, de donde $\Delta(f) = \sqrt{81} = 3^9 = 9 \in \mathbb{Q}$. Así, vemos que ``$G<A_3$'', por lo que $|G| = 3$. Así, tenemos que:
        \begin{equation*}
            [K:F] = |G| = 3
        \end{equation*}
        Como $\alpha$ es una raíz de $f$ tenemos claramente que $\mathbb{Q}\leq \mathbb{Q}(\alpha)\leq K$, y como $[K:F] = 3$ tenemos por el Lema de la Torre que bien $\mathbb{Q}(\alpha) = \mathbb{Q}$ o bien $\mathbb{Q}(\alpha) = K$. Como $f=\Irr(\alpha,\mathbb{Q})$ tenemos por tanto que $[\mathbb{Q}(\alpha):\mathbb{Q}] = 3$, por lo que la única posibilidad es $K = \mathbb{Q}(\alpha)$.
    \end{ejercicio}

    \begin{ejercicio}
        Sea $F$ un cuerpo con $\car(F) =3$ y con un elemento $a\in F$ con $F=\bb{F}_3(a)$ con $a^4+a-1=0$.
        \begin{enumerate}[label=\alph*)]
            \item Describir $\Aut(F)$ y evaluarlos en $a^2$.

                Sea $f=x^4+x-1\in \bb{F}_3[x]$, vemos que $f$ es irreducible en $\bb{F}_3[x]$, pues:
                \begin{itemize}
                    \item No tiene raíces en $\bb{F}_3$:
                        \begin{equation*}
                            f(0) = -1, \quad f(1) = 1, \quad f(2) = 2
                        \end{equation*}
                        Por lo que no tiene factores de grado 1 ni de grado 3.
                    \item Podría tener factores de grado 2. Para ello, calculamos primero los polinomios mónicos irreducibles de grado 2 en $\bb{F}_3[x]$. Sabemos que:
                        \begin{equation*}
                            x^{3^2}-x = x^9-x\in \bb{F}_3[x]
                        \end{equation*}
                        factoriza como todos y cada uno de los polinomios mónicos irreducibles de grados 1 y 2, de grado 1 hay 3, por lo que de grado 2 hay $\frac{9-3}{2} = 3$. Buscando entre todos los polinomios de grado 2 que parecen no tener raíces, encontramos que estos son:
                        \begin{equation*}
                            x^2+1, \quad x^2+x+2, \quad x^2+2x+2
                        \end{equation*}
                        Veamos si alguno de estos divide a $f$:
                        \begin{align*}
                            x^4+x-1 &= (x^2+1)(x^2+2) + x \\
                            x^4+x-1 &= (x^2+x+2)(x^2+2x+2) + x+1
                        \end{align*}
                        Vemos en este último caso que tampoco es divisible entre $x^2+2x+2$, por lo que $f$ no tiene factores de grado 2.
                \end{itemize}
                Como $f$ es irreducible, vemos que $\Irr(a,\bb{F}_3) = f$, por lo que:
                \begin{equation*}
                    [\bb{F}_3(a):\bb{F}_3] = 4
                \end{equation*}
                y tenemos por tanto que $\{1,a,a^2,a^3\}$ es una $\bb{F}_3-$base de $F$. En vista de esto último y de la extensión $\bb{F}_3\leq \bb{F}_3(a)$, vemos que $\Aut(F)$ es un grupo cíclico de orden 4, que estará generado por el automorfismo de Frobenius de la extensión $\bb{F}_3\leq \bb{F}_3(a)$, que es $\tau:\bb{F}_3(a)\to \bb{F}_3(a)$ determinado por:
                \begin{equation*}
                    \tau(a) = a^3
                \end{equation*}
                Elevando $\tau$ a 2 y 3 obtenemos todos los automorfismos. En definitiva, tenemos que:
                \begin{equation*}
                    \Aut(F) = \{\tau_1, \tau_3, \tau_9, \tau_{27}\}
                \end{equation*}
                donde $\tau_j$ viene dado por $\tau_j(a) = a^j$, para $j \in \{1,3,9,27\}$. Si evaluamos cada uno de ellos en $a^2$ teniendo en cuenta que:
                \begin{equation*}
                    a^4+a-1 = 0 \quad\Longleftrightarrow\quad a^4 = 1-a
                \end{equation*}
                Vemos que:
                \begin{align*}
                    \tau_1(a^2) &= a^2 \\
                    \tau_3(a^2) &= {(a^2)}^{3} = a^6 = a^2(1-a) = a^2 - a^3 \\
                    \tau_9(a^2) &= {(a^2)}^{9} = a^{18} = a^2{(a^4)}^{4} = a^2{(1-a)}^{4} = a^2(1-a-a^3+a^4) \\
                                &= a^2 - a^3 - a^5 + a^6 = a^2 - a^3 - a(1-a) + a^2(1-a) \\ &= a^2 - a^3 -a + a^2 + a^2 - a^3 = a^3 + 2a \\
                    \tau_{27}(a^2) &= {(a^2)}^{27} = {(a^{18})}^{3} = {(a^3+2a)}^{3} = a^9 + 2a^7 + a^5 + 2a^3 \\
                                   &= a{(1-a)}^{2} + 2a^3(1-a) + a(1-a) + 2a^3 \\
                                   &= a \cancel{- 2a^2} + \cancel{a^3} + \cancel{2a^3} - 2a^4 + a \cancel{- a^2} + 2a^3 \\
                                   &= 2a + 2a^3 - 2a^4 = 2a + 2a^3 - 2(1-a) = 2a + 2a^3 -2 +2a \\
                                   &= -2 + a +2a^3
                \end{align*}
            \item Calcular el cardinal de $\bb{F}_3(a^2)$.

                Como $a^2\in \bb{F}_3(a)$ vemos claramente que $\bb{F}_3\leq \bb{F}_3(a^2)\leq \bb{F}_3(a)$ con todas las extensiones de Galois por ser cuerpos finitos, por lo que esta subextensión debe corresponderse por la conexión de Galois con un subgrupo de $\Aut(F)$, concretamente con $\Aut_{\bb{F}_3(a^2)}(\bb{F}_3(a))$. Sin embargo, en el apartado anterior hemos visto que ningún elemento de $\Aut(\bb{F}_3(a))$ distinto de $\tau_1=id$ deja fijo $a^2$, por lo que tiene que ser:
                \begin{equation*}
                    \Aut_{\bb{F}_3(a^2)}(\bb{F}_3(a)) = \{\tau_1\}
                \end{equation*}
                de donde $\bb{F}_3(a^2) = \bb{F}_3(a)$, y tenemos por tanto que:
                \begin{equation*}
                    |\bb{F}_3(a^2)| = |\bb{F}_3(a)| \AstIg 3^4 = 81
                \end{equation*}
                donde en $(\ast)$ usamos que $[\bb{F}_3(a):\bb{F}_3] = 4$.
        \end{enumerate}
    \end{ejercicio}

    \begin{ejercicio}
        Responda razonadamente si las siguientes afirmaciones son verdaderas o falsas.
        \begin{enumerate}[label=\alph*)]
            \item Si $F\leq E\leq K$ con $F\leq E$ y $E\leq K$ extensiones de Galois, entonces $F\leq K$ es de Galois.

                Es falsa, si consideramos $\mathbb{Q}\leq \mathbb{Q}\left(\sqrt{2}\right)\leq \mathbb{Q}\left(\sqrt[4]{2}\right)$, tenemos que $\mathbb{Q}\leq \mathbb{Q}\left(\sqrt{2}\right)$ es de Galois por ser $\mathbb{Q}\left(\sqrt{2}\right)$ el cuerpo de descomposición sobre $\mathbb{Q}$ del polinomio $x^2-2$, que $\mathbb{Q}\left(\sqrt{2}\right)\leq \mathbb{Q}\left(\sqrt[4]{2}\right)$ es de Galois por ser $\mathbb{Q}\left(\sqrt[4]{2}\right)$ cuerpo de descomposición de $x^2-\sqrt{2}$ sobre $\mathbb{Q}\left(\sqrt{2}\right)$ y $\mathbb{Q}\leq \mathbb{Q}\left(\sqrt[4]{2}\right)$ no es de Galois, pues el polinomo $x^4-2\in \mathbb{Q}[x]$ es irreducible por Eisenstein para $p=2$ y sus raíces son $\pm \sqrt[4]{2}, \pm i \sqrt[4]{2}$, vemos que algunas están en $\mathbb{Q}\left(\sqrt[4]{2}\right)$ y otras no, con lo que $\mathbb{Q}\leq \mathbb{Q}\left(\sqrt[4]{2}\right)$ no puede ser una extensión de Galois, al no ser si quiera una extensión normal.
            \item Si $z\in \mathbb{C}$ tiene grado 4 sobre $\mathbb{Q}$ entonces $z$ es construible.

                Es falsa, pensamos en buscar un polinomio de grado 4 irreducible en $\mathbb{Q}[x]$ para así obtener un elemento de grado 4 como una raíz suya. Para ello, como hemos de buscar un polinomio de grado 4 que sea irreducible, lo escogemos bien pensando en el criterio de reducción para $p=2$, en $\bb{F}_2[x]$, sabemos que el polinomio $x^4+x^2+1 = {(x^2+x+1)}^{2}$ no es irreducible y que los polinomios:
                \begin{equation*}
                    x^4+x+1, \qquad x^4+x^3+1
                \end{equation*}
                sí que lo son. Tomamos pues $f=x^4+x+1\in \mathbb{Q}[x]$, que sabemos que es irreducible por el criterio de reducción para $p=2$. Buscamos calcular su grupo de Galois. Como $f$ es irreducible sabemos que su grupo de Galois será isomorfo a $C_4,V,D_4,A_4$ o $S_4$. Para ello, en vista de que $f$ es una cuártica reducida, buscamos su resolvente cúbica, que aplicando la fórmula vista en teoría, sabemos que es (para $x^4+px^2+qx+r$):
                \begin{equation*}
                    g = x^3 + 2px^2 + (p^2-4r)x - q^2 = x^3-4x-1
                \end{equation*}
                Sabemos ya calcular el discriminante de $f$:
                \begin{equation*}
                    \Disc(f) = \Disc(g) = -4p^3 -27q^3 = 4\cdot 4^3 - 27 = 4^4 - 27 = 256 - 27 = 229
                \end{equation*}
                Como $229$ es primo, vemos que $\sqrt{229}\notin\mathbb{Q}$, por lo que $G$ (el grupo de Galois de $f$) no puede ser isomorfo a $A_4$ ni a $V$. Vemos que $g$ es irreducible en $\mathbb{Q}[x]$, por no tener raíces en $\mathbb{Q}$ y que su grupo de Galois es $S_3$ por ser $\Delta(g)\notin \mathbb{Q}$. Así, sea $E$ el cuerpo de descomposición de $g$ y $K$ el de $f$, vemos que $\mathbb{Q}\leq E$ es de Galois, por lo que aplicando un Teorema de teoría vemos que:
                \begin{equation*}
                    \frac{\Aut_F(K)}{\Aut_E(K)} \cong \Aut_F(E) \cong S_3
                \end{equation*}
                Con $|S_3| = 6$, múltiplo de $3$, por lo que $|\Aut_F(K)|$ es múltiplo de $3$, por lo que la única opción es $G\cong S_4$. Así, vemos que:
                \begin{equation*}
                    [K:F] = |G| = |S_4| = 24 = 2^3\cdot 3
                \end{equation*}
                Tenemos la torre:
                \begin{equation*}
                    \mathbb{Q}\leq \mathbb{Q}(\alpha) \leq K
                \end{equation*}
                Sea $L$ un cuerpo de forma que $\mathbb{Q}(\alpha)\leq L$ con $\mathbb{Q}\leq L$ de Galois, tenemos entonces que la extensión es normal y $\alpha$ es una raíz de $f=\Irr(\alpha,\mathbb{Q})$, por lo que todas las raíces de $f$ deben estar en $L$, de donde tiene que ser entonces:
                \begin{equation*}
                    \mathbb{Q}(\alpha) \leq K \leq L
                \end{equation*}
                Así, tenemos que $[L:\mathbb{Q}]$ es múltiplo de 3 por serlo $[K:\mathbb{Q}]$, de donde $[L:\mathbb{Q}]$ no puede ser una potencia de 2, luego $\alpha$ no es constructible.
        \end{enumerate}
    \end{ejercicio}
\end{document}
