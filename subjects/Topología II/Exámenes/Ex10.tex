\documentclass[12pt]{article}

% Idioma y codificación
\usepackage[spanish, es-tabla]{babel}       %es-tabla para que se titule "Tabla"
\usepackage[utf8]{inputenc}

% Márgenes
\usepackage[a4paper,top=3cm,bottom=2.5cm,left=3cm,right=3cm]{geometry}

% Comentarios de bloque
\usepackage{verbatim}

% Paquetes de links
\usepackage[hidelinks]{hyperref}    % Permite enlaces
\usepackage{url}                    % redirecciona a la web

% Más opciones para enumeraciones
\usepackage{enumitem}

% Personalizar la portada
\usepackage{titling}

% Paquetes de tablas
\usepackage{multirow}


%------------------------------------------------------------------------

%Paquetes de figuras
\usepackage{caption}
\usepackage{subcaption} % Figuras al lado de otras
\usepackage{float}      % Poner figuras en el sitio indicado H.


% Paquetes de imágenes
\usepackage{graphicx}       % Paquete para añadir imágenes
\usepackage{transparent}    % Para manejar la opacidad de las figuras

% Paquete para usar colores
\usepackage[dvipsnames]{xcolor}
\usepackage{pagecolor}      % Para cambiar el color de la página

% Habilita tamaños de fuente mayores
\usepackage{fix-cm}

% Para los gráficos
\usepackage{tikz}

% Para poder situar los nodos en los grafos
\usetikzlibrary{positioning}


%------------------------------------------------------------------------

% Paquetes de matemáticas
\usepackage{mathtools, amsfonts, amssymb, mathrsfs}
\usepackage[makeroom]{cancel}     % Simplificar tachando
\usepackage{polynom}    % Divisiones y Ruffini
\usepackage{units} % Para poner fracciones diagonales con \nicefrac

\usepackage{pgfplots}   %Representar funciones
\pgfplotsset{compat=1.18}  % Versión 1.18

\usepackage{tikz-cd}    % Para usar diagramas de composiciones
\usetikzlibrary{calc}   % Para usar cálculo de coordenadas en tikz

%Definición de teoremas, etc.
\usepackage{amsthm}
%\swapnumbers   % Intercambia la posición del texto y de la numeración

\theoremstyle{plain}

\makeatletter
\@ifclassloaded{article}{
  \newtheorem{teo}{Teorema}[section]
}{
  \newtheorem{teo}{Teorema}[chapter]  % Se resetea en cada chapter
}
\makeatother

\newtheorem{coro}{Corolario}[teo]           % Se resetea en cada teorema
\newtheorem{prop}[teo]{Proposición}         % Usa el mismo contador que teorema
\newtheorem{lema}[teo]{Lema}                % Usa el mismo contador que teorema

\theoremstyle{remark}
\newtheorem*{observacion}{Observación}

\theoremstyle{definition}

\makeatletter
\@ifclassloaded{article}{
  \newtheorem{definicion}{Definición} [section]     % Se resetea en cada chapter
}{
  \newtheorem{definicion}{Definición} [chapter]     % Se resetea en cada chapter
}
\makeatother

\newtheorem*{notacion}{Notación}
\newtheorem*{ejemplo}{Ejemplo}
\newtheorem*{ejercicio*}{Ejercicio}             % No numerado
\newtheorem{ejercicio}{Ejercicio} [section]     % Se resetea en cada section


% Modificar el formato de la numeración del teorema "ejercicio"
\renewcommand{\theejercicio}{%
  \ifnum\value{section}=0 % Si no se ha iniciado ninguna sección
    \arabic{ejercicio}% Solo mostrar el número de ejercicio
  \else
    \thesection.\arabic{ejercicio}% Mostrar número de sección y número de ejercicio
  \fi
}


% \renewcommand\qedsymbol{$\blacksquare$}         % Cambiar símbolo QED
%------------------------------------------------------------------------

% Paquetes para encabezados
\usepackage{fancyhdr}
\pagestyle{fancy}
\fancyhf{}

\newcommand{\helv}{ % Modificación tamaño de letra
\fontfamily{}\fontsize{12}{12}\selectfont}
\setlength{\headheight}{15pt} % Amplía el tamaño del índice


%\usepackage{lastpage}   % Referenciar última pag   \pageref{LastPage}
\fancyfoot[C]{\thepage}

%------------------------------------------------------------------------

% Conseguir que no ponga "Capítulo 1". Sino solo "1."
\makeatletter
\@ifclassloaded{book}{
  \renewcommand{\chaptermark}[1]{\markboth{\thechapter.\ #1}{}} % En el encabezado
    
  \renewcommand{\@makechapterhead}[1]{%
  \vspace*{50\p@}%
  {\parindent \z@ \raggedright \normalfont
    \ifnum \c@secnumdepth >\m@ne
      \huge\bfseries \thechapter.\hspace{1em}\ignorespaces
    \fi
    \interlinepenalty\@M
    \Huge \bfseries #1\par\nobreak
    \vskip 40\p@
  }}
}
\makeatother

%------------------------------------------------------------------------
% Paquetes de cógido
\usepackage{minted}
\renewcommand\listingscaption{Código fuente}

\usepackage{fancyvrb}
% Personaliza el tamaño de los números de línea
\renewcommand{\theFancyVerbLine}{\small\arabic{FancyVerbLine}}

% Estilo para C++
\newminted{cpp}{
    frame=lines,
    framesep=2mm,
    baselinestretch=1.2,
    linenos,
    escapeinside=||
}

% para minted
\definecolor{LightGray}{rgb}{0.95,0.95,0.92}
\setminted{
    linenos=true,
    stepnumber=5,
    numberfirstline=true,
    autogobble,
    breaklines=true,
    breakautoindent=true,
    breaksymbolleft=,
    breaksymbolright=,
    breaksymbolindentleft=0pt,
    breaksymbolindentright=0pt,
    breaksymbolsepleft=0pt,
    breaksymbolsepright=0pt,
    fontsize=\footnotesize,
    bgcolor=LightGray,
    numbersep=10pt
}


\usepackage{listings} % Para incluir código desde un archivo

\renewcommand\lstlistingname{Código Fuente}
\renewcommand\lstlistlistingname{Índice de Códigos Fuente}

% Definir colores
\definecolor{vscodepurple}{rgb}{0.5,0,0.5}
\definecolor{vscodeblue}{rgb}{0,0,0.8}
\definecolor{vscodegreen}{rgb}{0,0.5,0}
\definecolor{vscodegray}{rgb}{0.5,0.5,0.5}
\definecolor{vscodebackground}{rgb}{0.97,0.97,0.97}
\definecolor{vscodelightgray}{rgb}{0.9,0.9,0.9}

% Configuración para el estilo de C similar a VSCode
\lstdefinestyle{vscode_C}{
  backgroundcolor=\color{vscodebackground},
  commentstyle=\color{vscodegreen},
  keywordstyle=\color{vscodeblue},
  numberstyle=\tiny\color{vscodegray},
  stringstyle=\color{vscodepurple},
  basicstyle=\scriptsize\ttfamily,
  breakatwhitespace=false,
  breaklines=true,
  captionpos=b,
  keepspaces=true,
  numbers=left,
  numbersep=5pt,
  showspaces=false,
  showstringspaces=false,
  showtabs=false,
  tabsize=2,
  frame=tb,
  framerule=0pt,
  aboveskip=10pt,
  belowskip=10pt,
  xleftmargin=10pt,
  xrightmargin=10pt,
  framexleftmargin=10pt,
  framexrightmargin=10pt,
  framesep=0pt,
  rulecolor=\color{vscodelightgray},
  backgroundcolor=\color{vscodebackground},
}

%------------------------------------------------------------------------

% Comandos definidos
\newcommand{\bb}[1]{\mathbb{#1}}
\newcommand{\cc}[1]{\mathcal{#1}}

% I prefer the slanted \leq
\let\oldleq\leq % save them in case they're every wanted
\let\oldgeq\geq
\renewcommand{\leq}{\leqslant}
\renewcommand{\geq}{\geqslant}

% Si y solo si
\newcommand{\sii}{\iff}

% Letras griegas
\newcommand{\eps}{\epsilon}
\newcommand{\veps}{\varepsilon}
\newcommand{\lm}{\lambda}

\newcommand{\ol}{\overline}
\newcommand{\ul}{\underline}
\newcommand{\wt}{\widetilde}
\newcommand{\wh}{\widehat}

\let\oldvec\vec
\renewcommand{\vec}{\overrightarrow}

% Derivadas parciales
\newcommand{\del}[2]{\frac{\partial #1}{\partial #2}}
\newcommand{\Del}[3]{\frac{\partial^{#1} #2}{\partial #3^{#1}}}
\newcommand{\deld}[2]{\dfrac{\partial #1}{\partial #2}}
\newcommand{\Deld}[3]{\dfrac{\partial^{#1} #2}{\partial #3^{#1}}}


\newcommand{\AstIg}{\stackrel{(\ast)}{=}}
\newcommand{\Hop}{\stackrel{L'H\hat{o}pital}{=}}

\newcommand{\red}[1]{{\color{red}#1}} % Para integrales, destacar los cambios.

% Método de integración
\newcommand{\MetInt}[2]{
    \left[\begin{array}{c}
        #1 \\ #2
    \end{array}\right]
}

% Declarar aplicaciones
% 1. Nombre aplicación
% 2. Dominio
% 3. Codominio
% 4. Variable
% 5. Imagen de la variable
\newcommand{\Func}[5]{
    \begin{equation*}
        \begin{array}{rrll}
            #1:& #2 & \longrightarrow & #3\\
               & #4 & \longmapsto & #5
        \end{array}
    \end{equation*}
}

%------------------------------------------------------------------------


\usetikzlibrary{arrows.meta, decorations.markings} % Cargar las bibliotecas necesarias
% Configuración para las flechas
\tikzset{
    arrow at 1/3/.style={postaction={decorate},
        decoration={markings, mark=at position 0.33 with {\arrow{Stealth}}}},
    arrow at 2/3/.style={postaction={decorate},
        decoration={markings, mark=at position 0.66 with {\arrow{Stealth}}}}
}

\begin{document}

    % 1. Foto de fondo
    % 2. Título
    % 3. Encabezado Izquierdo
    % 4. Color de fondo
    % 5. Coord x del titulo
    % 6. Coord y del titulo
    % 7. Fecha

    
    % 1. Foto de fondo
% 2. Título
% 3. Encabezado Izquierdo
% 4. Color de fondo
% 5. Coord x del titulo
% 6. Coord y del titulo
% 7. Fecha

\newcommand{\portada}[7]{

    \portadaBase{#1}{#2}{#3}{#4}{#5}{#6}{#7}
    \portadaBook{#1}{#2}{#3}{#4}{#5}{#6}{#7}
}

\newcommand{\portadaExamen}[7]{

    \portadaBase{#1}{#2}{#3}{#4}{#5}{#6}{#7}
    \portadaArticle{#1}{#2}{#3}{#4}{#5}{#6}{#7}
}




\newcommand{\portadaBase}[7]{

    % Tiene la portada principal y la licencia Creative Commons
    
    % 1. Foto de fondo
    % 2. Título
    % 3. Encabezado Izquierdo
    % 4. Color de fondo
    % 5. Coord x del titulo
    % 6. Coord y del titulo
    % 7. Fecha
    
    
    \thispagestyle{empty}               % Sin encabezado ni pie de página
    \newgeometry{margin=0cm}        % Márgenes nulos para la primera página
    
    
    % Encabezado
    \fancyhead[L]{\helv #3}
    \fancyhead[R]{\helv \nouppercase{\leftmark}}
    
    
    \pagecolor{#4}        % Color de fondo para la portada
    
    \begin{figure}[p]
        \centering
        \transparent{0.3}           % Opacidad del 30% para la imagen
        
        \includegraphics[width=\paperwidth, keepaspectratio]{assets/#1}
    
        \begin{tikzpicture}[remember picture, overlay]
            \node[anchor=north west, text=white, opacity=1, font=\fontsize{60}{90}\selectfont\bfseries\sffamily, align=left] at (#5, #6) {#2};
            
            \node[anchor=south east, text=white, opacity=1, font=\fontsize{12}{18}\selectfont\sffamily, align=right] at (9.7, 3) {\textbf{\href{https://losdeldgiim.github.io/}{Los Del DGIIM}}};
            
            \node[anchor=south east, text=white, opacity=1, font=\fontsize{12}{15}\selectfont\sffamily, align=right] at (9.7, 1.8) {Doble Grado en Ingeniería Informática y Matemáticas\\Universidad de Granada};
        \end{tikzpicture}
    \end{figure}
    
    
    \restoregeometry        % Restaurar márgenes normales para las páginas subsiguientes
    \pagecolor{white}       % Restaurar el color de página
    
    
    \newpage
    \thispagestyle{empty}               % Sin encabezado ni pie de página
    \begin{tikzpicture}[remember picture, overlay]
        \node[anchor=south west, inner sep=3cm] at (current page.south west) {
            \begin{minipage}{0.5\paperwidth}
                \href{https://creativecommons.org/licenses/by-nc-nd/4.0/}{
                    \includegraphics[height=2cm]{assets/Licencia.png}
                }\vspace{1cm}\\
                Esta obra está bajo una
                \href{https://creativecommons.org/licenses/by-nc-nd/4.0/}{
                    Licencia Creative Commons Atribución-NoComercial-SinDerivadas 4.0 Internacional (CC BY-NC-ND 4.0).
                }\\
    
                Eres libre de compartir y redistribuir el contenido de esta obra en cualquier medio o formato, siempre y cuando des el crédito adecuado a los autores originales y no persigas fines comerciales. 
            \end{minipage}
        };
    \end{tikzpicture}
    
    
    
    % 1. Foto de fondo
    % 2. Título
    % 3. Encabezado Izquierdo
    % 4. Color de fondo
    % 5. Coord x del titulo
    % 6. Coord y del titulo
    % 7. Fecha


}


\newcommand{\portadaBook}[7]{

    % 1. Foto de fondo
    % 2. Título
    % 3. Encabezado Izquierdo
    % 4. Color de fondo
    % 5. Coord x del titulo
    % 6. Coord y del titulo
    % 7. Fecha

    % Personaliza el formato del título
    \pretitle{\begin{center}\bfseries\fontsize{42}{56}\selectfont}
    \posttitle{\par\end{center}\vspace{2em}}
    
    % Personaliza el formato del autor
    \preauthor{\begin{center}\Large}
    \postauthor{\par\end{center}\vfill}
    
    % Personaliza el formato de la fecha
    \predate{\begin{center}\huge}
    \postdate{\par\end{center}\vspace{2em}}
    
    \title{#2}
    \author{\href{https://losdeldgiim.github.io/}{Los Del DGIIM}}
    \date{Granada, #7}
    \maketitle
    
    \tableofcontents
}




\newcommand{\portadaArticle}[7]{

    % 1. Foto de fondo
    % 2. Título
    % 3. Encabezado Izquierdo
    % 4. Color de fondo
    % 5. Coord x del titulo
    % 6. Coord y del titulo
    % 7. Fecha

    % Personaliza el formato del título
    \pretitle{\begin{center}\bfseries\fontsize{42}{56}\selectfont}
    \posttitle{\par\end{center}\vspace{2em}}
    
    % Personaliza el formato del autor
    \preauthor{\begin{center}\Large}
    \postauthor{\par\end{center}\vspace{3em}}
    
    % Personaliza el formato de la fecha
    \predate{\begin{center}\huge}
    \postdate{\par\end{center}\vspace{5em}}
    
    \title{#2}
    \author{\href{https://losdeldgiim.github.io/}{Los Del DGIIM}}
    \date{Granada, #7}
    \thispagestyle{empty}               % Sin encabezado ni pie de página
    \maketitle
    \vfill
}
    \portadaExamen{ffccA4.jpg}{Topología II\\Examen X}{Topología II. Examen X}{MidnightBlue}{-8}{28}{2026}{José Juan Urrutia Milán}

    \begin{description}
        \item[Asignatura] Topología II.
        \item[Curso Académico] 2025/26.
        \item[Grado] Doble Grado en Ingeniería Informática y Matemáticas.
        \item[Grupo] Grupo Único.
        \item[Profesor] José Antonio Gálvez.
        \item[Descripción] Examen Ordinario.
        \item[Fecha] 14 de enero de 2026.
        \item[Duración] 2 horas y media.
    \end{description}
    \newpage


    % ------------------------------------

    \noindent
    \textbf{Responda a la pregunta 1 y elija dos preguntas entre la 2, 3 y 4. Todos los ejercicios valen lo mismo.}

    \begin{ejercicio}
        Razona si son verdaderas o falsas las siguientes afirmaciones.
        \begin{enumerate}[label=\alph*)]
            \item Toda aplicación continua $f:\mathbb{D}\to \mathbb{S}^1\times\mathbb{S}^1$ es homotópicamente nula. Aquí, $\mathbb{D}$ denota el disco abierto unidad de $\mathbb{R}^2$.
            \item Existe una aplicación recubridora desde $\mathbb{S}^1$ en $X=C_1\cup C_2$, donde
                \begin{equation*}
                    C_1 = \{(x,y)\in \mathbb{R}^2: {(x+1)}^{2}+y^2 = 1\}, \qquad C_2 = \{(x,y)\in \mathbb{R}^2 : {(x-1)}^{2}+y^2 = 1\}
                \end{equation*}
            \item Si $f:\mathbb{S}^1\to \mathbb{S}^1$ es una aplicación continua entonces $f$ es sobreyectiva o tiene un punto fijo.
        \end{enumerate}
    \end{ejercicio}

    \begin{ejercicio}
        Sea $X$ el espacio topológico dado por $\mathbb{R}^3$ menos el eje $x$ y el eje $y$, es decir,
        \begin{equation*}
            X = \mathbb{R}^3\setminus\left( \{(x,0,0)\in \mathbb{R}^3 : x\in \mathbb{R}\}\cup \{(0,y,0)\in \mathbb{R}^3 : y\in \mathbb{R}\}\right)
        \end{equation*}
        Calcula el grupo fundamental de $X$ en el punto $(0,0,1)$ y determina generadores de dicho grupo.
    \end{ejercicio}

    \begin{ejercicio}
        Sean $X,Y,Z$ tres espacios topológicos conexos y localmante arcoconexos. Consideremos dos aplicaciones continuas $p_1:X\to Y$ y $f:Y\to Z$ tales que $p_1$ y $p_2=f\circ p_1$ son aplicaciones recubridoras. Demuestra que $f$ también es una aplicación recubridora.\newline
        Utiliza lo anterior para demostrar que si $a,b,c,d$ son cuatro números enteros con $ad-bc\neq 0$ entonces la aplicación (bien definida) $f:\mathbb{S}^1\times \mathbb{S}^1\to \mathbb{S}^1\times \mathbb{S}^1$ dada por
        \begin{equation*}
            f(\cos\theta,\sen\theta,\cos\varphi,\sen\varphi) = (\cos(a\theta + b\varphi), \sen(a\theta + b\varphi), \cos(c\theta + d\varphi), \sen(c\theta + d\varphi))
        \end{equation*}
        es una aplicación recubridora.
    \end{ejercicio}

    \begin{ejercicio}
        Clasifica la superficie compacta $S$ asociada a la presentación poligonal con expresión:
        \begin{equation*}
            abcadefd^{-1}e^{-1}bf^{-1}c^{-1}
        \end{equation*}
        ¿Es homeomorfa a una suma conexa finita de botellas de Klein? ¿Se cumple que $S$ es homeomorfa a la suma conexa $\mathbb{T}_n\# \mathbb{R}\mathbb{P}^2$, para algún $n$ natural?
    \end{ejercicio}
    
    \newpage
    \setcounter{ejercicio}{0}
    \noindent
    \textbf{Solución.}

    \begin{ejercicio}
        Razona si son verdaderas o falsas las siguientes afirmaciones.
        \begin{enumerate}[label=\alph*)]
            \item Toda aplicación continua $f:\mathbb{D}\to \mathbb{S}^1\times\mathbb{S}^1$ es homotópicamente nula. Aquí, $\mathbb{D}$ denota el disco abierto unidad de $\mathbb{R}^2$.

                Es verdadera, si consideramos la aplicación recubridora estándar $p:\mathbb{R}\to \mathbb{S}^1$ dada por:
                \begin{equation*}
                    p(x) = (\cos(2\pi x),\sen(2\pi x))
                \end{equation*}
                Tenemos entonces que $p\times p:\mathbb{R}^2\to \mathbb{S}^1\times\mathbb{S}^1$ es una aplicación recubridora.
                \begin{figure}[H]
                    \centering
                    \shorthandoff{""}
                    \begin{tikzcd}
                                                  & \mathbb{R}\times \mathbb{R} \arrow[d, "p\times p"] \\
                        \mathbb{D} \arrow[r, "f"] & \mathbb{S}^1\times \mathbb{S}^1                   
                    \end{tikzcd}
                    \shorthandon{""}
                \end{figure}
                \noindent
                Fijado $x_0\in \mathbb{D}$ y tomando $r_0\in p^{-1}(\{p(x_0)\})$, como $\mathbb{D}$ es simplemente conexo tenemos que $\pi_1(\mathbb{D},x_0) = \{[\varepsilon_{x_0}]\}$, por lo que tenemos que:
                \begin{equation*}
                    f_\ast(\pi_1(\mathbb{D},x_0)) = \{[\varepsilon_{f(x_0)}]\} \subseteq p_\ast(\pi_1(\mathbb{R}^2,r_0))
                \end{equation*}
                Por lo que $f$ se puede levantar, es decir, existe una aplicación $\hat{f}:\mathbb{R}^2\to \mathbb{S}^1\times\mathbb{S}^1$ continua de forma que
                \begin{equation*}
                    f = \hat{f} \circ p
                \end{equation*}
                Como $\hat{f}$ llega a $\mathbb{R}^2$ y este espacio es contráctil hacia $\{(0,0)\}$, tenemos que existe una homotopía $H:\mathbb{D}\times [0,1]\to \mathbb{R}^2$ de forma que:
                \begin{equation*}
                    H(x,0) = \hat{f}(x), \qquad H(x,1) = (0,0) \qquad \forall x\in \mathbb{D}
                \end{equation*}
                Si consideramos ahora la aplicación $(p\times p)\circ H:\mathbb{D}\times [0,1]\to \mathbb{S}^1\times \mathbb{S}^{1}$, tenemos que $(p\times p)\circ H$ es una aplicación continua con:
                \begin{multline*}
                    ((p\times p) \circ H)(x,0) = p(\hat{f}(x)) = f(x), \qquad ((p\times p)\circ H)(x,1) = (p\times p)(0,0) = (1,0,1,0) \\ \forall x\in \mathbb{D}
                \end{multline*}
                Por lo que $f$ es homotópicamente nula.
            \item Existe una aplicación recubridora desde $\mathbb{S}^1$ en $X=C_1\cup C_2$, donde
                \begin{equation*}
                    C_1 = \{(x,y)\in \mathbb{R}^2: {(x+1)}^{2}+y^2 = 1\}, \qquad C_2 = \{(x,y)\in \mathbb{R}^2 : {(x-1)}^{2}+y^2 = 1\}
                \end{equation*}

                Es falsa, por reducción al absurdo, supongamos que existe $p:\mathbb{S}^1\to X$ una aplicación recubridora. De ser así, como $p$ es sobreyectiva tenemos que existe $x_0\in \mathbb{S}^1$ de forma que $p(x_0) = (0,0)$. Por ser $p$ recubridora, existe $U$ entorno abierto (podemos suponer conexo) de $(0,0)$ y $V$ entorno abierto de $x_0$ de forma que $p\big|_V:V\to U$ es un homeomorfismo.\\

                \noindent
                Podemos suponer ahora sin pérdida de generalidad que:
                \begin{equation*}
                    U\subseteq X\setminus\{x_1,x_2\} \quad \text{con}\quad x_1\in C_1\setminus \{(0,0)\}, \quad x_2\in C_2\setminus \{(0,0)\}
                \end{equation*}
                ya que si no basta tomar $U\setminus \{x_1,x_2\}$ con $x_1\in C_1\setminus \{(0,0)\}$ y $x_2\in C_2\setminus \{(0,0)\}$, y seguirá siendo un conjunto abierto (ya que $\{x_1,x_2\}$ es cerrado) regularmente recubierto. Bajo estas hipótesis, tenemos que ha de ser $V\subseteq \mathbb{S}^1\setminus \{p\}$ para $p\in \mathbb{S}^1\setminus \{x_0\}$, puesto que $(p\big|_V)_\ast$ es un isomorfismo entre los grupos fundamentales de $U$ y $V$, y tenemos que $U$ es contráctil, por lo que no puede ser $V=\mathbb{S}^1$.\\

                \noindent
                Así, vemos que podemos considerar el homeomorfismo $\overline{p}:V\setminus \{x_0\}\to U\setminus \{(0,0)\}$ dado por la restricción de $p\big|_{V}$, pero sin embargo vemos que $V\setminus \{x_0\}$ tiene 2 componentes conexas y que $U\setminus \{(0,0)\}$ tiene 4, hemos llegado a una contradicción, pues $\overline{p}$ debería mantener el número de componentes conexas, por ser un homeomorfismo.
            \item Si $f:\mathbb{S}^1\to \mathbb{S}^1$ es una aplicación continua entonces $f$ es sobreyectiva o tiene un punto fijo.

                Es verdadera, supongamos que $f:\mathbb{S}^1\to \mathbb{S}^1$ es una aplicación continua que no es sobreyectiva. De ser así, tenemos que existe $p\in \mathbb{S}^1\setminus f(\mathbb{S}^1)$, con lo que $f(\mathbb{S}^1)\subseteq \mathbb{S}^1$. Podemos considerar ahora $g:\mathbb{S}^1\setminus \{p\}\to \mathbb{R}$ un homeomorfismo, y vemos que el conjunto:
                \begin{equation*}
                    g(f(\mathbb{S}^1)) \subseteq \mathbb{R}
                \end{equation*}
                es un conjunto compacto y conexo como imagen de un conjunto compacto y conexo por una aplicación continua ($g\circ f$). Por tanto, existen $a,b\in \mathbb{R}$ de forma que:
                \begin{equation*}
                    g(f(\mathbb{S}^1)) = [a,b]
                \end{equation*}
                Si consideramos ahora $h:[a,b]\to \mathbb{S}^1$ dado por:
                \begin{equation*}
                    h(x) = g^{-1}(x)
                \end{equation*}
                Tenemos entonces la aplicación continua $(g\circ f\circ h):[a,b]\to [a,b]$.
                \begin{figure}[H]
                    \centering
                    \shorthandoff{""}
                    \begin{tikzcd}
                        {[a,b]} \arrow[r, "h"] & \mathbb{S}^1 \arrow[r, "f"] & \mathbb{S}^1\setminus\{p\} \arrow[r, "g"] & {[a,b]}
                    \end{tikzcd}
                    \shorthandon{""}
                \end{figure}
                \noindent
                Se ha visto en Cálculo II que este tipo de aplicaciones tienen algún punto fijo\footnote{Piénsese en el conjunto $[a,b]\times [a,b]$ y en la recta $y=x$, que pasa por los vértices inferior izquierdo y superior derecho del cuadrado, la gráfica de cualquier función que dibujemos de $[a,b]$ en $[a,b]$ debe cortar a dicha recta, obteniéndose un punto fijo.}, por lo que existe $x_0\in [a,b]$ de forma que:
                \begin{equation*}
                    g(f(h(x_0))) = x_0 \quad\Longleftrightarrow\quad f(g^{-1}(x_0)) = g^{-1}(x_0) 
                \end{equation*}
                Por lo que tomando $z_0 = g^{-1}(x_0)\in \mathbb{S}^1$ tenemos que $f(z_0) = z_0$.
        \end{enumerate}
    \end{ejercicio}

    \begin{ejercicio}
        Sea $X$ el espacio topológico dado por $\mathbb{R}^3$ menos el eje $x$ y el eje $y$, es decir,
        \begin{equation*}
            X = \mathbb{R}^3\setminus\left( \{(x,0,0)\in \mathbb{R}^3 : x\in \mathbb{R}\}\cup \{(0,y,0)\in \mathbb{R}^3 : y\in \mathbb{R}\}\right)
        \end{equation*}
        Calcula el grupo fundamental de $X$ en el punto $(0,0,1)$ y determina generadores de dicho grupo.\\

        \noindent
        Habría que detallar más la solución, pero $X$ tiene por retracto de deformación a $\mathbb{S}^2$ menos 4 puntos, que es homeomorfo a $\mathbb{R}^3$ menos 3 puntos, que tiene grupo fundamental $\mathbb{Z}\ast\mathbb{Z}\ast\mathbb{Z}$.
    \end{ejercicio}

    \begin{ejercicio}
        Sean $X,Y,Z$ tres espacios topológicos conexos y localmante arcoconexos. Consideremos dos aplicaciones continuas $p_1:X\to Y$ y $f:Y\to Z$ tales que $p_1$ y $p_2=f\circ p_1$ son aplicaciones recubridoras. Demuestra que $f$ también es una aplicación recubridora.\newline
        Utiliza lo anterior para demostrar que si $a,b,c,d$ son cuatro números enteros con $ad-bc\neq 0$ entonces la aplicación (bien definida) $f:\mathbb{S}^1\times \mathbb{S}^1\to \mathbb{S}^1\times \mathbb{S}^1$ dada por
        \begin{equation*}
            f(\cos\theta,\sen\theta,\cos\varphi,\sen\varphi) = (\cos(a\theta + b\varphi), \sen(a\theta + b\varphi), \cos(c\theta + d\varphi), \sen(c\theta + d\varphi))
        \end{equation*}
        es una aplicación recubridora.\\

        \noindent
        La primera parte del ejercicio se corresponde con el Ejercicio 1.3.8. de la relación de ejercicios. Para la segunda parte, buscamos un espacio topológico $X$ y dos aplicaciones recubridoras $p_1$ y $p_2$ para poder aplicar el ejercicio.\\

        \noindent
        Sea $p:\mathbb{R}\to \mathbb{S}^1$ la aplicación recubridora estándar
        \begin{equation*}
            p(x) = (\cos(2\pi x),\sen(2\pi x))
        \end{equation*}
        tenemos entonces que $p\times p:\mathbb{R}^2\to \mathbb{S}^1\times \mathbb{S}^1$ es una aplicación recubridora. Si consideramos ahora la aplicación $h:\mathbb{R}^2\to \mathbb{R}^2$ dada por:
        \begin{equation*}
            h(\theta,\varphi) = (a\theta + b\varphi, c\theta + d\varphi)
        \end{equation*}
        Vemos que $h$ es continua, así como que la condición $ad-bc\neq 0$ implica que el sistema de ecuaciones:
        \begin{equation*}
            \left\{\begin{array}{l}
                \theta = a\theta + b\varphi \\
                \varphi = c\theta + d\varphi
            \end{array}\right.
        \end{equation*}
        Tiene una única solución, por lo que podemos obtener $\theta$ y $\varphi$ de forma continua (mediante suma, resta, multiplicación y división) a partir de $a\theta+b\varphi$ y $c\theta + d\varphi$, con lo que la aplicación $h$ admite una inversa continua, por lo que $h$ es un homeomorfismo. Ante esta situación, tenemos que $(p\times p)\circ h:\mathbb{R}^2\to \mathbb{S}^1\times \mathbb{S}^1$ es una aplicación recubridora, como composición de un homeomorfismo con una aplicación recubridora. Veamos que:
        \begin{equation*}
            (p\times p) \circ h = f \circ (p\times p)
        \end{equation*}
        Para ello, si $(\theta,\varphi)\in \mathbb{R}^2$ tenemos entonces que:
        \begin{align*}
            (f\circ (p\times p))(\theta,\varphi) &= f(\cos\theta,\sen\theta,\cos\varphi,\sen\varphi) \\
                                                 &= (\cos(a\theta + b\varphi), \sen(a\theta + b\varphi), \cos(c\theta + d\varphi), \sen(c\theta + d\varphi)) \\
            ((p\times p)\circ h)(\theta,\varphi) &= (p\times p)(a\theta+b\varphi, c\theta +d\varphi) \\
                                                 &= (\cos(a\theta + b\varphi), \sen(a\theta + b\varphi), \cos(c\theta + d\varphi), \sen(c\theta + d\varphi))
        \end{align*}
        Tenemos por tanto el diagrama conmutativo
        \begin{figure}[H]
            \centering
            \shorthandoff{""}
            \begin{tikzcd}
                \mathbb{R}\times\mathbb{R} \arrow[d, "h"] \arrow[r, "p\times p"] & \mathbb{S}^1\times\mathbb{S}^1 \arrow[r, "f"] & \mathbb{S}^1\times\mathbb{S}^1 \\
                \mathbb{R}\times\mathbb{R} \arrow[rru, "p\times p"']             &                                               &                               
            \end{tikzcd}
            \shorthandon{""}
        \end{figure}
        \noindent
        Por lo que aplicando la primera parte de este ejercicio concluimos que $f$ una aplicación recubridora.
    \end{ejercicio}

    \begin{ejercicio}
        Clasifica la superficie compacta $S$ asociada a la presentación poligonal con expresión:
        \begin{equation*}
            abcadefd^{-1}e^{-1}bf^{-1}c^{-1}
        \end{equation*}
        ¿Es homeomorfa a una suma conexa finita de botellas de Klein? ¿Se cumple que $S$ es homeomorfa a la suma conexa $\mathbb{T}_n\# \mathbb{R}\mathbb{P}^2$, para algún $n$ natural?\\

        \noindent
        Como $S$ tiene una sola expresión en su presentación poligonal vemos que $S$ es conexa. Si calculamos su característica de Euler calculando para ello:
        \begin{itemize}
            \item $C=1$.
            \item $A=6$.
            \item $V=1$.
        \end{itemize}
        Vemos que $S$ solo tiene un vértice:
        \begin{figure}[H]
            \centering
            \begin{tikzpicture}[
                    flechaCCW/.style={postaction={decorate}, decoration={markings, mark=at position 0.7 with {\arrow{Latex[length=3mm]}}}, thick},
                    flechaCW/.style={postaction={decorate}, decoration={markings, mark=at position 0.7 with {\arrow{Latex[length=3mm, reversed]}}}, thick},
                    etiqueta/.style={midway, auto, swap, font=\small}
                ]
                \def\R{1.5} % Radio del hexágono

                \foreach \i in {0,...,11} {
                    \pgfmathsetmacro{\angle}{\i * 30}
                    \coordinate (P\i) at (\angle:\R);
                }

                % Aristas en sentido antihorario
                \draw[flechaCCW] (P0) -- node[etiqueta] {$a$} (P1);
                \draw[flechaCCW] (P1) -- node[etiqueta] {$b$} (P2);
                \draw[flechaCCW] (P2) -- node[etiqueta] {$c$} (P3); 
                \draw[flechaCCW] (P3) -- node[etiqueta] {$a$} (P4); 
                \draw[flechaCCW] (P4) -- node[etiqueta] {$d$} (P5); 
                \draw[flechaCCW] (P5) -- node[etiqueta] {$e$} (P6); 
                \draw[flechaCCW] (P6) -- node[etiqueta] {$f$} (P7); 
                \draw[flechaCW] (P7) -- node[etiqueta] {$d^{-1}$} (P8); 
                \draw[flechaCW] (P8) -- node[etiqueta] {$e^{-1}$} (P9); 
                \draw[flechaCCW] (P9) -- node[etiqueta] {$b$} (P10); 
                \draw[flechaCW] (P10) -- node[etiqueta] {$f^{-1}$} (P11); 
                \draw[flechaCW] (P11) -- node[etiqueta] {$c^{-1}$} (P0); 

                % Colores de los vértices
                \filldraw[fill=yellow, draw=black] (P0) circle (2pt);
                \filldraw[fill=yellow, draw=black] (P1) circle (2pt);
                \filldraw[fill=yellow, draw=black] (P2) circle (2pt);
                \filldraw[fill=yellow, draw=black] (P3) circle (2pt);
                \filldraw[fill=yellow, draw=black] (P4) circle (2pt);
                \filldraw[fill=yellow, draw=black] (P5) circle (2pt);
                \filldraw[fill=yellow, draw=black] (P6) circle (2pt);
                \filldraw[fill=yellow, draw=black] (P7) circle (2pt);
                \filldraw[fill=yellow, draw=black] (P8) circle (2pt);
                \filldraw[fill=yellow, draw=black] (P9) circle (2pt);
                \filldraw[fill=yellow, draw=black] (P10) circle (2pt);
                \filldraw[fill=yellow, draw=black] (P11) circle (2pt);
            \end{tikzpicture}    
        \end{figure}
        \noindent
        Por lo que:
        \begin{equation*}
            \chi(S) = V-A+C = -4
        \end{equation*}
        de done $S\cong \mathbb{T}_3$ ó $S\cong \mathbb{R}\mathbb{P}^2_6$. Como la presentación es no orientada tiene que ser $S\cong \mathbb{R}\mathbb{P}^2_6$.\\

        \noindent
        Sea $K$ una botella de Klein, en teoría se ha visto que $K\cong \mathbb{R}\mathbb{P}^2_2$. Vemos que:
        \begin{equation*}
            S\cong \mathbb{R}\mathbb{P}^2_6 = (\mathbb{R}\mathbb{P}^2_2) \# (\mathbb{R}\mathbb{P}^2_2) \# (\mathbb{R}\mathbb{P}^2_2) \cong K\# K\# K
        \end{equation*}
        Por lo que $S$ es homoemorfa a la suma conexa de 3 botellas de Klein.\\

        \noindent
        Si calculamos la característica de Euler de $\mathbb{T}_n\# \mathbb{R}\mathbb{P}^2$ para cualquier $n$ vemos que:
        \begin{equation*}
            \chi(\mathbb{T}_n\# \mathbb{R}\mathbb{P}^2) = \chi(\mathbb{T}_n) + \chi(\mathbb{R}\mathbb{P}^2) -2 = 2(1-n) + 1 - 2 = -2n+1 = -(2n-1)
        \end{equation*}
        Es un número negativo impar, por lo que $\chi(S)\neq \chi(\mathbb{T}_n\# \mathbb{R}\mathbb{P}^2)$ para todo $n$ natural, de donde nunca podrá ser $S$ homeomorfa a $\mathbb{T}_n\# \mathbb{R}\mathbb{P}^2$ para algún $n$ natural.
    \end{ejercicio}

\end{document}
