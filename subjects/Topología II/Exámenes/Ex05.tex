\documentclass[12pt]{article}

% Idioma y codificación
\usepackage[spanish, es-tabla]{babel}       %es-tabla para que se titule "Tabla"
\usepackage[utf8]{inputenc}

% Márgenes
\usepackage[a4paper,top=3cm,bottom=2.5cm,left=3cm,right=3cm]{geometry}

% Comentarios de bloque
\usepackage{verbatim}

% Paquetes de links
\usepackage[hidelinks]{hyperref}    % Permite enlaces
\usepackage{url}                    % redirecciona a la web

% Más opciones para enumeraciones
\usepackage{enumitem}

% Personalizar la portada
\usepackage{titling}

% Paquetes de tablas
\usepackage{multirow}


%------------------------------------------------------------------------

%Paquetes de figuras
\usepackage{caption}
\usepackage{subcaption} % Figuras al lado de otras
\usepackage{float}      % Poner figuras en el sitio indicado H.


% Paquetes de imágenes
\usepackage{graphicx}       % Paquete para añadir imágenes
\usepackage{transparent}    % Para manejar la opacidad de las figuras

% Paquete para usar colores
\usepackage[dvipsnames]{xcolor}
\usepackage{pagecolor}      % Para cambiar el color de la página

% Habilita tamaños de fuente mayores
\usepackage{fix-cm}

% Para los gráficos
\usepackage{tikz}

% Para poder situar los nodos en los grafos
\usetikzlibrary{positioning}


%------------------------------------------------------------------------

% Paquetes de matemáticas
\usepackage{mathtools, amsfonts, amssymb, mathrsfs}
\usepackage[makeroom]{cancel}     % Simplificar tachando
\usepackage{polynom}    % Divisiones y Ruffini
\usepackage{units} % Para poner fracciones diagonales con \nicefrac

\usepackage{pgfplots}   %Representar funciones
\pgfplotsset{compat=1.18}  % Versión 1.18

\usepackage{tikz-cd}    % Para usar diagramas de composiciones
\usetikzlibrary{calc}   % Para usar cálculo de coordenadas en tikz

%Definición de teoremas, etc.
\usepackage{amsthm}
%\swapnumbers   % Intercambia la posición del texto y de la numeración

\theoremstyle{plain}

\makeatletter
\@ifclassloaded{article}{
  \newtheorem{teo}{Teorema}[section]
}{
  \newtheorem{teo}{Teorema}[chapter]  % Se resetea en cada chapter
}
\makeatother

\newtheorem{coro}{Corolario}[teo]           % Se resetea en cada teorema
\newtheorem{prop}[teo]{Proposición}         % Usa el mismo contador que teorema
\newtheorem{lema}[teo]{Lema}                % Usa el mismo contador que teorema

\theoremstyle{remark}
\newtheorem*{observacion}{Observación}

\theoremstyle{definition}

\makeatletter
\@ifclassloaded{article}{
  \newtheorem{definicion}{Definición} [section]     % Se resetea en cada chapter
}{
  \newtheorem{definicion}{Definición} [chapter]     % Se resetea en cada chapter
}
\makeatother

\newtheorem*{notacion}{Notación}
\newtheorem*{ejemplo}{Ejemplo}
\newtheorem*{ejercicio*}{Ejercicio}             % No numerado
\newtheorem{ejercicio}{Ejercicio} [section]     % Se resetea en cada section


% Modificar el formato de la numeración del teorema "ejercicio"
\renewcommand{\theejercicio}{%
  \ifnum\value{section}=0 % Si no se ha iniciado ninguna sección
    \arabic{ejercicio}% Solo mostrar el número de ejercicio
  \else
    \thesection.\arabic{ejercicio}% Mostrar número de sección y número de ejercicio
  \fi
}


% \renewcommand\qedsymbol{$\blacksquare$}         % Cambiar símbolo QED
%------------------------------------------------------------------------

% Paquetes para encabezados
\usepackage{fancyhdr}
\pagestyle{fancy}
\fancyhf{}

\newcommand{\helv}{ % Modificación tamaño de letra
\fontfamily{}\fontsize{12}{12}\selectfont}
\setlength{\headheight}{15pt} % Amplía el tamaño del índice


%\usepackage{lastpage}   % Referenciar última pag   \pageref{LastPage}
\fancyfoot[C]{\thepage}

%------------------------------------------------------------------------

% Conseguir que no ponga "Capítulo 1". Sino solo "1."
\makeatletter
\@ifclassloaded{book}{
  \renewcommand{\chaptermark}[1]{\markboth{\thechapter.\ #1}{}} % En el encabezado
    
  \renewcommand{\@makechapterhead}[1]{%
  \vspace*{50\p@}%
  {\parindent \z@ \raggedright \normalfont
    \ifnum \c@secnumdepth >\m@ne
      \huge\bfseries \thechapter.\hspace{1em}\ignorespaces
    \fi
    \interlinepenalty\@M
    \Huge \bfseries #1\par\nobreak
    \vskip 40\p@
  }}
}
\makeatother

%------------------------------------------------------------------------
% Paquetes de cógido
\usepackage{minted}
\renewcommand\listingscaption{Código fuente}

\usepackage{fancyvrb}
% Personaliza el tamaño de los números de línea
\renewcommand{\theFancyVerbLine}{\small\arabic{FancyVerbLine}}

% Estilo para C++
\newminted{cpp}{
    frame=lines,
    framesep=2mm,
    baselinestretch=1.2,
    linenos,
    escapeinside=||
}

% para minted
\definecolor{LightGray}{rgb}{0.95,0.95,0.92}
\setminted{
    linenos=true,
    stepnumber=5,
    numberfirstline=true,
    autogobble,
    breaklines=true,
    breakautoindent=true,
    breaksymbolleft=,
    breaksymbolright=,
    breaksymbolindentleft=0pt,
    breaksymbolindentright=0pt,
    breaksymbolsepleft=0pt,
    breaksymbolsepright=0pt,
    fontsize=\footnotesize,
    bgcolor=LightGray,
    numbersep=10pt
}


\usepackage{listings} % Para incluir código desde un archivo

\renewcommand\lstlistingname{Código Fuente}
\renewcommand\lstlistlistingname{Índice de Códigos Fuente}

% Definir colores
\definecolor{vscodepurple}{rgb}{0.5,0,0.5}
\definecolor{vscodeblue}{rgb}{0,0,0.8}
\definecolor{vscodegreen}{rgb}{0,0.5,0}
\definecolor{vscodegray}{rgb}{0.5,0.5,0.5}
\definecolor{vscodebackground}{rgb}{0.97,0.97,0.97}
\definecolor{vscodelightgray}{rgb}{0.9,0.9,0.9}

% Configuración para el estilo de C similar a VSCode
\lstdefinestyle{vscode_C}{
  backgroundcolor=\color{vscodebackground},
  commentstyle=\color{vscodegreen},
  keywordstyle=\color{vscodeblue},
  numberstyle=\tiny\color{vscodegray},
  stringstyle=\color{vscodepurple},
  basicstyle=\scriptsize\ttfamily,
  breakatwhitespace=false,
  breaklines=true,
  captionpos=b,
  keepspaces=true,
  numbers=left,
  numbersep=5pt,
  showspaces=false,
  showstringspaces=false,
  showtabs=false,
  tabsize=2,
  frame=tb,
  framerule=0pt,
  aboveskip=10pt,
  belowskip=10pt,
  xleftmargin=10pt,
  xrightmargin=10pt,
  framexleftmargin=10pt,
  framexrightmargin=10pt,
  framesep=0pt,
  rulecolor=\color{vscodelightgray},
  backgroundcolor=\color{vscodebackground},
}

%------------------------------------------------------------------------

% Comandos definidos
\newcommand{\bb}[1]{\mathbb{#1}}
\newcommand{\cc}[1]{\mathcal{#1}}

% I prefer the slanted \leq
\let\oldleq\leq % save them in case they're every wanted
\let\oldgeq\geq
\renewcommand{\leq}{\leqslant}
\renewcommand{\geq}{\geqslant}

% Si y solo si
\newcommand{\sii}{\iff}

% Letras griegas
\newcommand{\eps}{\epsilon}
\newcommand{\veps}{\varepsilon}
\newcommand{\lm}{\lambda}

\newcommand{\ol}{\overline}
\newcommand{\ul}{\underline}
\newcommand{\wt}{\widetilde}
\newcommand{\wh}{\widehat}

\let\oldvec\vec
\renewcommand{\vec}{\overrightarrow}

% Derivadas parciales
\newcommand{\del}[2]{\frac{\partial #1}{\partial #2}}
\newcommand{\Del}[3]{\frac{\partial^{#1} #2}{\partial #3^{#1}}}
\newcommand{\deld}[2]{\dfrac{\partial #1}{\partial #2}}
\newcommand{\Deld}[3]{\dfrac{\partial^{#1} #2}{\partial #3^{#1}}}


\newcommand{\AstIg}{\stackrel{(\ast)}{=}}
\newcommand{\Hop}{\stackrel{L'H\hat{o}pital}{=}}

\newcommand{\red}[1]{{\color{red}#1}} % Para integrales, destacar los cambios.

% Método de integración
\newcommand{\MetInt}[2]{
    \left[\begin{array}{c}
        #1 \\ #2
    \end{array}\right]
}

% Declarar aplicaciones
% 1. Nombre aplicación
% 2. Dominio
% 3. Codominio
% 4. Variable
% 5. Imagen de la variable
\newcommand{\Func}[5]{
    \begin{equation*}
        \begin{array}{rrll}
            #1:& #2 & \longrightarrow & #3\\
               & #4 & \longmapsto & #5
        \end{array}
    \end{equation*}
}

%------------------------------------------------------------------------


\usetikzlibrary{arrows.meta, decorations.markings} % Cargar las bibliotecas necesarias
% Configuración para las flechas
\tikzset{
    arrow at 1/3/.style={postaction={decorate},
        decoration={markings, mark=at position 0.33 with {\arrow{Stealth}}}},
    arrow at 2/3/.style={postaction={decorate},
        decoration={markings, mark=at position 0.66 with {\arrow{Stealth}}}}
}

\begin{document}

    % 1. Foto de fondo
    % 2. Título
    % 3. Encabezado Izquierdo
    % 4. Color de fondo
    % 5. Coord x del titulo
    % 6. Coord y del titulo
    % 7. Fecha

    
    % 1. Foto de fondo
% 2. Título
% 3. Encabezado Izquierdo
% 4. Color de fondo
% 5. Coord x del titulo
% 6. Coord y del titulo
% 7. Fecha

\newcommand{\portada}[7]{

    \portadaBase{#1}{#2}{#3}{#4}{#5}{#6}{#7}
    \portadaBook{#1}{#2}{#3}{#4}{#5}{#6}{#7}
}

\newcommand{\portadaExamen}[7]{

    \portadaBase{#1}{#2}{#3}{#4}{#5}{#6}{#7}
    \portadaArticle{#1}{#2}{#3}{#4}{#5}{#6}{#7}
}




\newcommand{\portadaBase}[7]{

    % Tiene la portada principal y la licencia Creative Commons
    
    % 1. Foto de fondo
    % 2. Título
    % 3. Encabezado Izquierdo
    % 4. Color de fondo
    % 5. Coord x del titulo
    % 6. Coord y del titulo
    % 7. Fecha
    
    
    \thispagestyle{empty}               % Sin encabezado ni pie de página
    \newgeometry{margin=0cm}        % Márgenes nulos para la primera página
    
    
    % Encabezado
    \fancyhead[L]{\helv #3}
    \fancyhead[R]{\helv \nouppercase{\leftmark}}
    
    
    \pagecolor{#4}        % Color de fondo para la portada
    
    \begin{figure}[p]
        \centering
        \transparent{0.3}           % Opacidad del 30% para la imagen
        
        \includegraphics[width=\paperwidth, keepaspectratio]{assets/#1}
    
        \begin{tikzpicture}[remember picture, overlay]
            \node[anchor=north west, text=white, opacity=1, font=\fontsize{60}{90}\selectfont\bfseries\sffamily, align=left] at (#5, #6) {#2};
            
            \node[anchor=south east, text=white, opacity=1, font=\fontsize{12}{18}\selectfont\sffamily, align=right] at (9.7, 3) {\textbf{\href{https://losdeldgiim.github.io/}{Los Del DGIIM}}};
            
            \node[anchor=south east, text=white, opacity=1, font=\fontsize{12}{15}\selectfont\sffamily, align=right] at (9.7, 1.8) {Doble Grado en Ingeniería Informática y Matemáticas\\Universidad de Granada};
        \end{tikzpicture}
    \end{figure}
    
    
    \restoregeometry        % Restaurar márgenes normales para las páginas subsiguientes
    \pagecolor{white}       % Restaurar el color de página
    
    
    \newpage
    \thispagestyle{empty}               % Sin encabezado ni pie de página
    \begin{tikzpicture}[remember picture, overlay]
        \node[anchor=south west, inner sep=3cm] at (current page.south west) {
            \begin{minipage}{0.5\paperwidth}
                \href{https://creativecommons.org/licenses/by-nc-nd/4.0/}{
                    \includegraphics[height=2cm]{assets/Licencia.png}
                }\vspace{1cm}\\
                Esta obra está bajo una
                \href{https://creativecommons.org/licenses/by-nc-nd/4.0/}{
                    Licencia Creative Commons Atribución-NoComercial-SinDerivadas 4.0 Internacional (CC BY-NC-ND 4.0).
                }\\
    
                Eres libre de compartir y redistribuir el contenido de esta obra en cualquier medio o formato, siempre y cuando des el crédito adecuado a los autores originales y no persigas fines comerciales. 
            \end{minipage}
        };
    \end{tikzpicture}
    
    
    
    % 1. Foto de fondo
    % 2. Título
    % 3. Encabezado Izquierdo
    % 4. Color de fondo
    % 5. Coord x del titulo
    % 6. Coord y del titulo
    % 7. Fecha


}


\newcommand{\portadaBook}[7]{

    % 1. Foto de fondo
    % 2. Título
    % 3. Encabezado Izquierdo
    % 4. Color de fondo
    % 5. Coord x del titulo
    % 6. Coord y del titulo
    % 7. Fecha

    % Personaliza el formato del título
    \pretitle{\begin{center}\bfseries\fontsize{42}{56}\selectfont}
    \posttitle{\par\end{center}\vspace{2em}}
    
    % Personaliza el formato del autor
    \preauthor{\begin{center}\Large}
    \postauthor{\par\end{center}\vfill}
    
    % Personaliza el formato de la fecha
    \predate{\begin{center}\huge}
    \postdate{\par\end{center}\vspace{2em}}
    
    \title{#2}
    \author{\href{https://losdeldgiim.github.io/}{Los Del DGIIM}}
    \date{Granada, #7}
    \maketitle
    
    \tableofcontents
}




\newcommand{\portadaArticle}[7]{

    % 1. Foto de fondo
    % 2. Título
    % 3. Encabezado Izquierdo
    % 4. Color de fondo
    % 5. Coord x del titulo
    % 6. Coord y del titulo
    % 7. Fecha

    % Personaliza el formato del título
    \pretitle{\begin{center}\bfseries\fontsize{42}{56}\selectfont}
    \posttitle{\par\end{center}\vspace{2em}}
    
    % Personaliza el formato del autor
    \preauthor{\begin{center}\Large}
    \postauthor{\par\end{center}\vspace{3em}}
    
    % Personaliza el formato de la fecha
    \predate{\begin{center}\huge}
    \postdate{\par\end{center}\vspace{5em}}
    
    \title{#2}
    \author{\href{https://losdeldgiim.github.io/}{Los Del DGIIM}}
    \date{Granada, #7}
    \thispagestyle{empty}               % Sin encabezado ni pie de página
    \maketitle
    \vfill
}
    \portadaExamen{ffccA4.jpg}{Topología II\\Examen V}{Topología II. Examen V}{MidnightBlue}{-8}{28}{2025}{}

    \begin{description}
        \item[Asignatura] Topología II.
        \item[Curso Académico] 2024/25.
        \item[Grado] Doble Grado en Ingeniería Informática y Matemáticas.
        \item[Grupo] Grupo Único.
        % \item[Profesor] ---.
        \item[Descripción] Examen ordinario.
        \item[Fecha] 10 de enero de 2025.
        \item[Duración] 2 horas y media.
    \end{description}
    \newpage


    % ------------------------------------
    
    \noindent
    \textbf{Responda la pregunta 1 y elija dos preguntas entre la 2, 3 y 4.}

    \begin{ejercicio}
        Razona si son verdaderas o falsas las siguientes afirmaciones:
        \begin{enumerate}[label=\alph*)]
            \item Sea $f:\bb{S}^1\to X$ una aplicación continua y $F:A\to X$ una aplicación continua desde $A=\{x\in \mathbb{R}^2 : |x|\geq 1\}$ tal que $F\big|_{\bb{S}^1} = f$. Entonces $f_\ast$ es trivial.
            \item Sea $X$ un espacio topológico arcoconexo que admite un recubridor universal. Si $Y\subset X$ es arcoconexo entonces $Y$ también tiene un recubridor universal.
            \item Si $S_1$ y $S_2$ son dos superficies compactas y conexas con $S_1$ no orientable entonces $S_1\#S_2$ es no orientable.
        \end{enumerate}
    \end{ejercicio}

    \begin{ejercicio}
        Calcula el grupo fundamental de los espacios topológicos $X,Y$ siguientes:
        \begin{enumerate}[label=\alph*)]
            \item $X=E_1\cup E_2$, donde $E_1,E_2$ son las esferas de $\mathbb{R}^3$ de radio 2 y centradas, respectivamente, en los puntos $(1,0,0)$ y $(-1,0,0)$.
            \item $Y = S_1\cup S_2$, donde $S_1,S_2\subset \mathbb{R}^3$ son dos superficies topológicas conexas que se cortan en un único punto y con grupos fundamentales isomorfos, respectivamente, a los grupos $G_1$ y $G_2$.
        \end{enumerate}
    \end{ejercicio}

    \begin{ejercicio}
        Sean $X=\bb{S}^2\setminus \{N,S\}$ con $N = (0,0,1)$, $S=(0,0,-1)$ y $r:X\to A$ la aplicación continua
        \begin{equation*}
            r(x,y,z) = \frac{1}{\sqrt{x^2+y^2}}(x,y,0)
        \end{equation*}
        donde $A=\bb{S}^2\cap \{(x,y,z) : z=0\}$.
        \begin{enumerate}[label=\alph*)]
            \item Demuestra que $H:X\times [0,1]\to X$ dada por
                \begin{equation*}
                    H((x,y,z),t) = \cos\left(\frac{\pi t}{2}\right)(x,y,z) + \frac{\sen\left(\frac{\pi t}{2}\right)}{\sqrt{x^2+y^2}}(x,y,0)
                \end{equation*}
                es una homotopía entre $r$ y la identidad en $X$.
            \item Si $Y=\mathbb{R}\bb{P}^2\setminus \{[N]\}$, comprueba que $H$ se puede inducir a una homotopía
                \begin{equation*}
                    \overline{H}:Y\times [0,1]\to Y
                \end{equation*}
                    y utiliza esto para calcular el grupo fundamental del plano proyectivo menos un punto.
        \end{enumerate}
    \end{ejercicio}

    \begin{ejercicio}
        Sean $\bb{T}$ el toro y $S_1,S_2$ las superficies siguientes:
        \begin{enumerate}[label=\alph*)]
            \item $S_1$ está dada por la presentación poligonal
                \begin{equation*}
                    \langle a,b,c,d,e;ab^{-1}cde, ad^{-1}e^{-1}bc^{-1} \rangle 
                \end{equation*}
            \item $S_2 = \mathbb{R}\bb{P}^2\# \mathbb{R}\bb{P}^2\#\bb{T}$.
        \end{enumerate}
        ¿Son $S_1$ y $S_2$ homeomorfas?
    \end{ejercicio}

    \newpage
    \setcounter{ejercicio}{0} % Reiniciar contador de ejercicios
    \noindent
    \textbf{Solución.}

    \begin{ejercicio}
        Razona si son verdaderas o falsas las siguientes afirmaciones:
        \begin{enumerate}[label=\alph*)]
            \item Sea $f:\bb{S}^1\to X$ una aplicación continua y $F:A\to X$ una aplicación continua desde $A=\{x\in \mathbb{R}^2 : |x|\geq 1\}$ tal que $F\big|_{\bb{S}^1} = f$. Entonces $f_\ast$ es trivial.

                Es falsa, si consideramos $X = \mathbb{S}^1$ y tomamos $f = Id_{\mathbb{S}^1}$ y $F:A\to \mathbb{S}^1$ dada por:
                \begin{equation*}
                    F(x) = \frac{x}{|x|}
                \end{equation*}
                Tenemos que $f$ y $F$ son continuas, así como que $F\big|_{\mathbb{S}^1} = f$. Como $f$ es un homeomorfismo, tendremos que $f_\ast:\pi_1(\mathbb{S}^1)\to \pi_1(\mathbb{S}^1)$ es un isomorfismo de grupos, por lo que $f_\ast$ no puede ser trivial, al ser $\pi_1(\mathbb{S}^1)\cong \mathbb{Z}$.
            \item Sea $X$ un espacio topológico arcoconexo que admite un recubridor universal. Si $Y\subset X$ es arcoconexo entonces $Y$ también tiene un recubridor universal. 

                Es falsa, si consideramos $X= \mathbb{R}^2$ tenemos que $X$ admite un recubridor universal (él mismo con la aplicación identidad). Si denotamos ahora por $S(x,r)$ a la circunferencia de centro $x$ y radio $r$ y consideramos:
                \begin{equation*}
                    Y = \bigcup_{n\in \mathbb{N}}S\left(\frac{1}{n},\frac{1}{n}\right)
                \end{equation*}
                \begin{figure}[H]
                    \centering
                    \begin{tikzpicture}
                    \draw (2,0) circle(2) node[] {};
                    \draw (1,0) circle(1) node[] {};
                    \draw (0.5,0) circle(0.5) node[] {};
                    \draw (0.25,0) circle(0.25) node[] {};
                    \draw (0.125,0) circle(0.125) node[] {};
                    \draw (0.0625,0) circle(0.0625) node[] {};
                    \draw (0.03175,0) circle(0.03175) node[] {};
                    \draw (0.015875,0) circle(0.015875) node[] {};
                    \end{tikzpicture}
                    \caption{Conjunto $Y$.}
                \end{figure}
                Tenemos que $Y$ es arcoconexo como la unión de conjunto arcoconexos que se interseca en un punto (en el punto $(0,0)$) y que $Y$ no es semilocalmente simplemente conexo, pues para $x_0 = (0,0)$ todo entorno $U$ de $x_0$ ha de contener alguna circunferencia de radio $\nicefrac{1}{n}$, y el arco cuya imagen rodea dicha circunferencia no puede cerrarse en $U$ pero sí en $X$, por lo que $(i_U)_\ast$ no es trivial, lo que prueba que $Y$ no es semilocalmente simplemente conexo, luego no puede tener recubridor universal.
            \item Si $S_1$ y $S_2$ son dos superficies compactas y conexas con $S_1$ no orientable entonces $S_1\#S_2$ es no orientable. 

                Es verdadera, para ambas superficies podemos encontrar representaciones poligonales suyas $\cc{P}_1$ y $\cc{P}_2$ con una única expresión, con lo que serán de la forma:
                \begin{equation*}
                    \cc{P}_1 = \langle a_1,\ldots,a_n;w_1 \rangle , \qquad \cc{P}_2 = \langle b_1,\ldots,b_m;w_2 \rangle 
                \end{equation*}
                Como $S_1$ no es orientable y $\cc{P}_1$ tiene una expresión tendremos que esta expresión contiene dos letras con el mismo exponente. Hemos visto en teoría que la presentación:
                \begin{equation*}
                    \cc{P} = \langle a_1,\ldots,a_n,b_1,\ldots,b_m;w_1w_2 \rangle 
                \end{equation*}
                es una presentación poligonal de $S_1\# S_2$, que contiene una letra con el mismo exponente, al contener la expresión $w_1w_2$ la palabra $w_1$, por lo que $S_1\# S_2$ no es orientable.
        \end{enumerate}
    \end{ejercicio}

    \begin{ejercicio}
        Calcula el grupo fundamental de los espacios topológicos $X,Y$ siguientes:
        \begin{enumerate}[label=\alph*)]
            \item $X=E_1\cup E_2$, donde $E_1,E_2$ son las esferas de $\mathbb{R}^3$ de radio 2 y centradas, respectivamente, en los puntos $(1,0,0)$ y $(-1,0,0)$.

                El conjunto que nos dan es el siguiente:
                \begin{figure}[H]
                    \centering
                    \begin{tikzpicture}[scale=0.8]
                        \shorthandoff{>}
                        % Esfera
                        \draw[thick] (0,0) circle [radius=2];
                        \draw[thick, dashed] (2,0) arc (0:180:2 and 0.5);
                        \draw[thick] (2,0) arc (0:-180:2 and 0.5);

                        \draw[thick] (2,0) circle [radius=2];
                        \draw[thick, dashed] (4,0) arc (0:180:2 and 0.5);
                        \draw[thick] (4,0) arc (0:-180:2 and 0.5);
                    \end{tikzpicture} 
                \end{figure}
                Si consideramos los conjuntos:
                \begin{equation*}
                    U = X\setminus \{(2,0,0)\}, \qquad V = X\setminus \{(-2,0,0)\}
                \end{equation*}
                Tenemos:
                \begin{itemize}
                    \item Claramente $X= U\cup V$ con $U$ y $V$ abiertos.
                    \item $U$, $V$ y $U\cap V$ son arcoconexos como unión de conjuntos arcoconexos que se intersecan.
                    \item $U\cap V$ tiene como retracto de deformación el conjunto:
                        \begin{figure}[H]
                            \centering
                            \begin{tikzpicture}[scale=0.7]
                                \draw[thick, bend right=70] (0,2) to (0,-2);
                                \draw[thick, bend left=70] (0,2) to (0,-2);
                                \draw[thick, bend right=30] (-1.1,0) to (1.1,0);
                                \draw[thick, dashed, bend left=30] (-1.1,0) to (1.1,0);
                            \end{tikzpicture}
                        \end{figure}
                        Que es homeomorfo a $\mathbb{S}^2$, por lo que es simplemente conexo.
                    \item $U$ tiene como retracto de deformación el conjunto $Z$:
                         \begin{figure}[H]
                            \centering
                            \begin{tikzpicture}[scale=0.8]
                                \shorthandoff{>}
                                % Esfera
                                \draw[thick] (0,0) circle [radius=2];
                                \draw[thick, dashed] (2,0) arc (0:180:2 and 0.5);
                                \draw[thick] (2,0) arc (0:-180:2 and 0.5);
                                \draw[thick, bend right=50] (1,1.7) to (1,-1.7);
                                \draw[thick, dashed, bend right=30] (0.25,0) to (2,0);
                                \draw[thick, dashed, bend left=30] (0.25,0) to (2,0);
                            \end{tikzpicture} 
                        \end{figure}   
                        Si consideramos:
                        \begin{equation*}
                            W = Z\setminus \{(-1,0,0)\}, \qquad O = Z\setminus \{(1,0,0)\}
                        \end{equation*}
                        Tenemos que:
                        \begin{itemize}
                            \item $Y = W\cup O$, con $W$ y $O$ abiertos.
                            \item $W,O$ y $W\cap O$ son arcoconexos.
                            \item $W$ tiene a $E_2$ como retracto de deformación, por lo que $\pi_1(W)=\{1\}$.
                            \item $O$ tiene al conjunto:
                                \begin{figure}[H]
                                    \centering
                                    \begin{tikzpicture}[scale=0.8]
                                        \draw[thick, bend right=50] (1,1.7) to (1,-1.7);
                                        \draw[thick, bend left=50] (1,1.7) to (1,-1.7);
                                        \draw[thick] (1,1.7) .. controls (-1.5,1.7) and (-1.5,-1.7) .. (1,-1.7);
                                        \draw[thick, bend right=20] (-0.9,0) to (0.3,-0.5);
                                        \draw[thick, dashed, bend left=30] (-0.9,0) to (1.8,0);
                                        \draw[thick, bend left=35] (0.3,-0.5) to (1.8,0);
                                    \end{tikzpicture}
                                \end{figure}
                                Como retracto de deformación, que es homeomorfo a $\mathbb{S}^2$, por lo que $\pi_1(O) = \{1\}$.
                      \end{itemize}
                    \item Claramente $V$ es homeomorfo a $U$ (basta considerar una rotación), por lo que también será $\pi_1(V) = \{1\}$.
                \end{itemize}
                Aplicando el Teorema de Seifert-van Kampen concluimos que $\pi_1(X) = \{1\}$.

            \item $Y = S_1\cup S_2$, donde $S_1,S_2\subset \mathbb{R}^3$ son dos superficies topológicas conexas que se cortan en un único punto y con grupos fundamentales isomorfos, respectivamente, a los grupos $G_1$ y $G_2$.

                Tenemos que $S_1\cap S_2 = \{x_0\}$. Sabemos que los discos regulares de $S_1$ forman una base de entornos abiertos en $S_1$, y que los discos regulares de $S_2$ forman una base de entornos abiertos de $S_2$. Sean por tanto $D_1$ un disco regular que contiene a $x_0$ en $S_1$ y $D_2$ un disco regular que contiene a $x_0$ en $S_2$, podemos considerar los conjuntos:
                \begin{equation*}
                    U = S_2 \cup D_1, \qquad V = S_1\cup D_2
                \end{equation*}
                Vemos que:
                \begin{itemize}
                    \item $U$ y $V$ son arcoconexos como unión de dos conjuntos arcoconexos con intersección no vacía, pues:
                        \begin{equation*}
                            S_2\cap D_1 = \{x_0\} = S_1\cap D_2
                        \end{equation*}
                    \item $U\cap V$ es también arcoconexo como unión de dos conjuntos arcoconexos como intersección no vacía:
                        \begin{equation*}
                            U\cap V = D_1\cup D_2, \qquad D_1\cap D_2 = \{x_0\}
                        \end{equation*}
                    \item $U$ y $V$ son abiertos.
                    \item Para calcular el grupo fundamental de $U$ vemos que $U$ tiene como retracto de deformación a $S_2$, pues podemos contraer $D_1$ a $\{x_0\}$, obtenido así que:
                        \begin{equation*}
                            \pi_1(U) \cong \pi_1(S_2) \cong G_2
                        \end{equation*}
                    \item Para calcular el grupo fundamental de $V$ podemos hacer un razonamiento análogo, pues tiene a $S_1$ como retracto de deformación, con lo que:
                        \begin{equation*}
                            \pi_1(V) \cong \pi_1(S_1) \cong G_1
                        \end{equation*}
                    \item Tenemos que $U\cap V = D_1\cup D_2$ tiene por retracto de deformación $\{x_0\}$, con lo que $U\cap V$ es simplemente conexo.
                \end{itemize}
                Así, aplicando el Teorema de Seifert-van Kampen vemos que:
                \begin{equation*}
                    \pi_1(X) \cong \pi_1(U)\ast \pi_1(V) \cong G_2\ast G_1
                \end{equation*}
        \end{enumerate}
    \end{ejercicio}

    \begin{ejercicio}
        Sean $X=\bb{S}^2\setminus \{N,S\}$ con $N = (0,0,1)$, $S=(0,0,-1)$ y $r:X\to A$ la aplicación continua
        \begin{equation*}
            r(x,y,z) = \frac{1}{\sqrt{x^2+y^2}}(x,y,0)
        \end{equation*}
        donde $A=\bb{S}^2\cap \{(x,y,z) : z=0\}$.
        \begin{enumerate}[label=\alph*)]
            \item Demuestra que $H:X\times [0,1]\to X$ dada por
                \begin{equation*}
                    H((x,y,z),t) = \cos\left(\frac{\pi t}{2}\right)(x,y,z) + \frac{\sen\left(\frac{\pi t}{2}\right)}{\sqrt{x^2+y^2}}(x,y,0)
                \end{equation*}
                es una homotopía entre $r$ y la identidad en $X$.\\

                Vemos que:
                \begin{equation*}
                    H((x,y,z),t) = \cos\left(\frac{\pi t}{2}\right)(x,y,z) + \sen\left(\frac{\pi t}{2}\right)r(x,y,z) \qquad \forall ((x,y,z),t)\in X\times [0,1]
                \end{equation*}
                La aplicación $H$ es continua, como suma de aplicaciones continuas. Vemos además que:
                \begin{align*}
                    H((x,y,z),0) &= (x,y,z) = Id_X(x,y,z) \qquad \forall (x,y,z)\in X \\
                    H((x,y,z),1) &= \frac{1}{\sqrt{x^2+y^2}}(x,y,0) = r(x,y,z) \in A \qquad \forall (x,y,z)\in X
                \end{align*}
                Por lo que $H$ es una homotopía entre $Id_X$ y $i\circ r$.
            \item Si $Y=\mathbb{R}\bb{P}^2\setminus \{[N]\}$, comprueba que $H$ se puede inducir a una homotopía 
                \begin{equation*}
                    \overline{H}:Y\times [0,1]\to Y
                \end{equation*}
                    y utiliza esto para calcular el grupo fundamental del plano proyectivo menos un punto.\\

                    Definimos la aplicación $\overline{H}:Y\times [0,1]\to Y$ dada por:
                    \begin{equation*}
                        \overline{H}([(x,y,z)],t) = \left[\cos\left(\frac{\pi t}{2}\right)(x,y,z) + \frac{\sen\left(\frac{\pi t}{2}\right)}{\sqrt{x^2+y^2}}(x,y,0)\right]
                    \end{equation*}
                    Que está bien definida, pues si $(x,y,z)$ está relacionado con otro punto mediante la relación de equivalencia:
                    \begin{itemize}
                        \item El caso $(x,y,z)$ es trivial.
                        \item Si está relacionado con $(-x,-y,-z)$, tenemos que:
                            \begin{align*}
                                \overline{H}([(-x,-y,-z)],t) &=  \left[\cos\left(\frac{\pi t}{2}\right)(-x,-y,-z) + \frac{\sen\left(\frac{\pi t}{2}\right)}{\sqrt{x^2+y^2}}(-x,-y,0)\right] \\
                                                             &=  \left[-\left(\cos\left(\frac{\pi t}{2}\right)(x,y,z) + \frac{\sen\left(\frac{\pi t}{2}\right)}{\sqrt{x^2+y^2}}(x,y,0)\right)\right] \\
                                                             &=  \left[\cos\left(\frac{\pi t}{2}\right)(x,y,z) + \frac{\sen\left(\frac{\pi t}{2}\right)}{\sqrt{x^2+y^2}}(x,y,0)\right]
                            \end{align*}
                    \end{itemize}
                    Sea $p:\mathbb{S}^2\to \mathbb{R}\mathbb{P}^2$ la aplicación proyección al cociente, recordamos que vimos en teoría que esta aplicación es recubridora. Vemos además que el siguiente diagrama es conmutativo:
                    \begin{figure}[H]
                        \centering
                        \shorthandoff{""}
                        \begin{tikzcd}
                            {X\times [0,1]} \arrow[d, "p\times Id"'] \arrow[r, "H"] & X \arrow[d, "p"] \\
                            {Y\times [0,1]} \arrow[r, "\overline{H}"]               & Y               
                        \end{tikzcd}
                        \shorthandon{""}
                    \end{figure}
                    \noindent
                    Como $p$ es una aplicación recubridora y $Id:[0,1]\to [0,1]$ también (de hecho es un homeomorfismo), tenemos que $p\times Id$ es una aplicación recubridora, luego será continua, sobreyectiva y abierta, es decir, una identificación. Tenemos así que:
                    \begin{equation*}
                        \overline{H} \text{\ es continua} \Longleftrightarrow \overline{H}\circ (p\times Id) \text{\ es continua}
                    \end{equation*}
                    Pero como $\overline{H}\circ (p\times Id) = p\circ H$ y $p\circ H$ es continua como composición de aplicaciones continuas deducimos que $\overline{H}$ es continua. Como $H$ era una homotopía entre $Id_X$ y $r$, vemos ahora que $\overline{H}$ es una homotopía entre $Id_Y$ y $p\circ r$, por lo que hemos probado que $p(A) = \mathbb{R}\mathbb{P}$ es un retracto de deformación de $Y$, por lo que:
                    \begin{equation*}
                        \pi_1(Y) \cong \pi_1(\mathbb{R}\mathbb{P})
                    \end{equation*}
                    Sin embargo, en Topología I se vió que $\mathbb{R}\mathbb{P}\cong \mathbb{S}^1$, de donde:
                    \begin{equation*}
                        \pi_1(Y) \cong \pi_1(\mathbb{R}\mathbb{P}) \cong \pi_1(\mathbb{S}^1) \cong \mathbb{Z}
                    \end{equation*}
                    Por lo que el grupo fundamental de $Y$ es isomorfo a $\mathbb{Z}$.
        \end{enumerate}
    \end{ejercicio}

    \begin{ejercicio} 
        Sean $\bb{T}$ el toro y $S_1,S_2$ las superficies siguientes:
        \begin{enumerate}[label=\alph*)]
            \item $S_1$ está dada por la presentación poligonal
                \begin{equation*}
                    \langle a,b,c,d,e;ab^{-1}cde, ad^{-1}e^{-1}bc^{-1} \rangle 
                \end{equation*}
            \item $S_2 = \mathbb{R}\bb{P}^2\# \mathbb{R}\bb{P}^2\#\bb{T}$.
        \end{enumerate}
        ¿Son $S_1$ y $S_2$ homeomorfas?\\

        \noindent
        Tratamos de clasificar las superficies $S_1$ y $S_2$:
        \begin{enumerate}[label=\alph*)]
            \item Para $S_1$, tenemos que:
                \begin{itemize}
                    \item $C = 2$.
                    \item $A = 5$.
                    \item $V = 1$.
                \end{itemize}
                Y hemos calculado el número de vértices viendo que:
                \begin{figure}[H]
                    \centering
                    \begin{minipage}{0.24\textwidth}
                        \centering
                    \begin{tikzpicture}[
                            flechaCCW/.style={postaction={decorate}, decoration={markings, mark=at position 0.7 with {\arrow{Latex[length=3mm]}}}, thick},
                            flechaCW/.style={postaction={decorate}, decoration={markings, mark=at position 0.7 with {\arrow{Latex[length=3mm, reversed]}}}, thick},
                            etiqueta/.style={midway, auto, swap, font=\small}
                        ]
                        \def\R{1} % Radio del hexágono

                        \foreach \i in {0,...,4} {
                            \pgfmathsetmacro{\angle}{\i * 72}
                            \coordinate (P\i) at (\angle:\R);
                        }

                        % Aristas en sentido antihorario
                        \draw[flechaCCW] (P0) -- node[etiqueta] {$a$} (P1);
                        \draw[flechaCW] (P1) -- node[etiqueta] {$b^{-1}$} (P2);
                        \draw[flechaCCW] (P2) -- node[etiqueta] {$c$} (P3); 
                        \draw[flechaCCW] (P3) -- node[etiqueta] {$d$} (P4); 
                        \draw[flechaCCW] (P4) -- node[etiqueta] {$e$} (P0); 

                        % Colores de los vértices
                        \filldraw[fill=yellow, draw=black] (P0) circle (2pt);
                        \filldraw[fill=yellow, draw=black] (P1) circle (2pt);
                        \filldraw[fill=yellow, draw=black] (P2) circle (2pt);
                        \filldraw[fill=yellow, draw=black] (P3) circle (2pt);
                        \filldraw[fill=yellow, draw=black] (P4) circle (2pt);
                    \end{tikzpicture}   
                    \end{minipage}\hspace{2em}
                    \begin{minipage}{0.24\textwidth}
                        \centering
                    \begin{tikzpicture}[
                            flechaCCW/.style={postaction={decorate}, decoration={markings, mark=at position 0.7 with {\arrow{Latex[length=3mm]}}}, thick},
                            flechaCW/.style={postaction={decorate}, decoration={markings, mark=at position 0.7 with {\arrow{Latex[length=3mm, reversed]}}}, thick},
                            etiqueta/.style={midway, auto, swap, font=\small}
                        ]
                        \def\R{1} % Radio del hexágono

                        \foreach \i in {0,...,4} {
                            \pgfmathsetmacro{\angle}{\i * 72}
                            \coordinate (P\i) at (\angle:\R);
                        }

                        % Aristas en sentido antihorario
                        \draw[flechaCCW] (P0) -- node[etiqueta] {$a$} (P1);
                        \draw[flechaCW] (P1) -- node[etiqueta] {$d^{-1}$} (P2);
                        \draw[flechaCW] (P2) -- node[etiqueta] {$e^{-1}$} (P3); 
                        \draw[flechaCCW] (P3) -- node[etiqueta] {$b$} (P4); 
                        \draw[flechaCW] (P4) -- node[etiqueta] {$c^{-1}$} (P0); 

                        % Colores de los vértices
                        \filldraw[fill=yellow, draw=black] (P0) circle (2pt);
                        \filldraw[fill=yellow, draw=black] (P1) circle (2pt);
                        \filldraw[fill=yellow, draw=black] (P2) circle (2pt);
                        \filldraw[fill=yellow, draw=black] (P3) circle (2pt);
                        \filldraw[fill=yellow, draw=black] (P4) circle (2pt);
                    \end{tikzpicture}   
                    \end{minipage}
                        
                \end{figure}
                \noindent
                Por lo que:
                \begin{equation*}
                    \chi(S_1) = V-A+C = 1-5+2 = -2
                \end{equation*}
                Luego $S_1\cong \mathbb{T}_2$ ó $S_1\cong \mathbb{R}\mathbb{P}^2_4$. Para distinguir cual es, buscamos una presentación poligonal equivalente pero solo de una expresión, mediante transformaciones elementales vemos que:
                \begin{equation*}
                    ab^{-1}cde, ad^{-1}e^{-1}bc^{-1} \rightsquigarrow ab^{-1}cde, e^{-1}bc^{-1}ad^{-1} \rightsquigarrow ab^{-1}cdbc^{-1}ad^{-1}
                \end{equation*}
                Por lo que una presentación poligonal equivalente para $S_1$ es:
                \begin{equation*}
                    \langle a,b,c,d;ab^{-1}cdbc^{-1}ad^{-1} \rangle 
                \end{equation*}
                Vemos que esta es no orientada, puesto que $a$ aparece con el mismo exponente, por lo que deducimos que $S_1$ es una superficie no orientable, con lo que debe ser:
                \begin{equation*}
                    S_1 \cong \mathbb{R}\mathbb{P}^2_4
                \end{equation*}
            \item Para $S_2$ aplicamos la fórmula (donde $S$ y $T$ son superficies compactas y conexas):
                \begin{equation*}
                    \chi(S\# T) = \chi(S) + \chi(T) - 2
                \end{equation*}
                Vista en teoría, por lo que:
                \begin{align*}
                    \chi(\mathbb{R}\mathbb{P}^2\# \mathbb{R}\mathbb{P}^2\# \mathbb{T}) &= \chi(\mathbb{R}\mathbb{P}^2\# \mathbb{R}\mathbb{P}^2) + \chi(\mathbb{T}) -2 = \chi(\mathbb{R}\mathbb{P}^2) + \chi(\mathbb{R}\mathbb{P}^2) + \chi(\mathbb{T}) - 4 \\
                                                                                       &= 1 + 1 + 0 - 4 = -2
                \end{align*}
                Al igual que antes tenemos que $S_2\cong \mathbb{T}_2$ ó $S_2\cong \mathbb{R}\mathbb{P}^2_4$. Como hemos visto en el Ejercicio 1 c) de este examen, la suma conexa de una superficie no orientable con otra superfice es no orientable, y como:
                \begin{equation*}
                    S_2 = \mathbb{R}\mathbb{P}^2\# \mathbb{R}\mathbb{P}^2\# \mathbb{T} = \mathbb{R}\mathbb{P}^2\# (\mathbb{R}\mathbb{P}^2\# T)
                \end{equation*}
                Vemos que $S_2$ es no orientable, con lo que $S_2\cong \mathbb{R}\mathbb{P}^2_4$. 
        \end{enumerate}
        En definitiva vemos que $S_1$ y $S_2$ sí son homeomorfas, ya que:
        \begin{equation*}
            S_1 \cong \mathbb{R}\mathbb{P}^2_4 \cong S_2
        \end{equation*}
    \end{ejercicio}

\end{document}
