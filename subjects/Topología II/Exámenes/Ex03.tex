\documentclass[12pt]{article}

% Idioma y codificación
\usepackage[spanish, es-tabla]{babel}       %es-tabla para que se titule "Tabla"
\usepackage[utf8]{inputenc}

% Márgenes
\usepackage[a4paper,top=3cm,bottom=2.5cm,left=3cm,right=3cm]{geometry}

% Comentarios de bloque
\usepackage{verbatim}

% Paquetes de links
\usepackage[hidelinks]{hyperref}    % Permite enlaces
\usepackage{url}                    % redirecciona a la web

% Más opciones para enumeraciones
\usepackage{enumitem}

% Personalizar la portada
\usepackage{titling}

% Paquetes de tablas
\usepackage{multirow}


%------------------------------------------------------------------------

%Paquetes de figuras
\usepackage{caption}
\usepackage{subcaption} % Figuras al lado de otras
\usepackage{float}      % Poner figuras en el sitio indicado H.


% Paquetes de imágenes
\usepackage{graphicx}       % Paquete para añadir imágenes
\usepackage{transparent}    % Para manejar la opacidad de las figuras

% Paquete para usar colores
\usepackage[dvipsnames]{xcolor}
\usepackage{pagecolor}      % Para cambiar el color de la página

% Habilita tamaños de fuente mayores
\usepackage{fix-cm}

% Para los gráficos
\usepackage{tikz}

% Para poder situar los nodos en los grafos
\usetikzlibrary{positioning}


%------------------------------------------------------------------------

% Paquetes de matemáticas
\usepackage{mathtools, amsfonts, amssymb, mathrsfs}
\usepackage[makeroom]{cancel}     % Simplificar tachando
\usepackage{polynom}    % Divisiones y Ruffini
\usepackage{units} % Para poner fracciones diagonales con \nicefrac

\usepackage{pgfplots}   %Representar funciones
\pgfplotsset{compat=1.18}  % Versión 1.18

\usepackage{tikz-cd}    % Para usar diagramas de composiciones
\usetikzlibrary{calc}   % Para usar cálculo de coordenadas en tikz

%Definición de teoremas, etc.
\usepackage{amsthm}
%\swapnumbers   % Intercambia la posición del texto y de la numeración

\theoremstyle{plain}

\makeatletter
\@ifclassloaded{article}{
  \newtheorem{teo}{Teorema}[section]
}{
  \newtheorem{teo}{Teorema}[chapter]  % Se resetea en cada chapter
}
\makeatother

\newtheorem{coro}{Corolario}[teo]           % Se resetea en cada teorema
\newtheorem{prop}[teo]{Proposición}         % Usa el mismo contador que teorema
\newtheorem{lema}[teo]{Lema}                % Usa el mismo contador que teorema

\theoremstyle{remark}
\newtheorem*{observacion}{Observación}

\theoremstyle{definition}

\makeatletter
\@ifclassloaded{article}{
  \newtheorem{definicion}{Definición} [section]     % Se resetea en cada chapter
}{
  \newtheorem{definicion}{Definición} [chapter]     % Se resetea en cada chapter
}
\makeatother

\newtheorem*{notacion}{Notación}
\newtheorem*{ejemplo}{Ejemplo}
\newtheorem*{ejercicio*}{Ejercicio}             % No numerado
\newtheorem{ejercicio}{Ejercicio} [section]     % Se resetea en cada section


% Modificar el formato de la numeración del teorema "ejercicio"
\renewcommand{\theejercicio}{%
  \ifnum\value{section}=0 % Si no se ha iniciado ninguna sección
    \arabic{ejercicio}% Solo mostrar el número de ejercicio
  \else
    \thesection.\arabic{ejercicio}% Mostrar número de sección y número de ejercicio
  \fi
}


% \renewcommand\qedsymbol{$\blacksquare$}         % Cambiar símbolo QED
%------------------------------------------------------------------------

% Paquetes para encabezados
\usepackage{fancyhdr}
\pagestyle{fancy}
\fancyhf{}

\newcommand{\helv}{ % Modificación tamaño de letra
\fontfamily{}\fontsize{12}{12}\selectfont}
\setlength{\headheight}{15pt} % Amplía el tamaño del índice


%\usepackage{lastpage}   % Referenciar última pag   \pageref{LastPage}
\fancyfoot[C]{\thepage}

%------------------------------------------------------------------------

% Conseguir que no ponga "Capítulo 1". Sino solo "1."
\makeatletter
\@ifclassloaded{book}{
  \renewcommand{\chaptermark}[1]{\markboth{\thechapter.\ #1}{}} % En el encabezado
    
  \renewcommand{\@makechapterhead}[1]{%
  \vspace*{50\p@}%
  {\parindent \z@ \raggedright \normalfont
    \ifnum \c@secnumdepth >\m@ne
      \huge\bfseries \thechapter.\hspace{1em}\ignorespaces
    \fi
    \interlinepenalty\@M
    \Huge \bfseries #1\par\nobreak
    \vskip 40\p@
  }}
}
\makeatother

%------------------------------------------------------------------------
% Paquetes de cógido
\usepackage{minted}
\renewcommand\listingscaption{Código fuente}

\usepackage{fancyvrb}
% Personaliza el tamaño de los números de línea
\renewcommand{\theFancyVerbLine}{\small\arabic{FancyVerbLine}}

% Estilo para C++
\newminted{cpp}{
    frame=lines,
    framesep=2mm,
    baselinestretch=1.2,
    linenos,
    escapeinside=||
}

% para minted
\definecolor{LightGray}{rgb}{0.95,0.95,0.92}
\setminted{
    linenos=true,
    stepnumber=5,
    numberfirstline=true,
    autogobble,
    breaklines=true,
    breakautoindent=true,
    breaksymbolleft=,
    breaksymbolright=,
    breaksymbolindentleft=0pt,
    breaksymbolindentright=0pt,
    breaksymbolsepleft=0pt,
    breaksymbolsepright=0pt,
    fontsize=\footnotesize,
    bgcolor=LightGray,
    numbersep=10pt
}


\usepackage{listings} % Para incluir código desde un archivo

\renewcommand\lstlistingname{Código Fuente}
\renewcommand\lstlistlistingname{Índice de Códigos Fuente}

% Definir colores
\definecolor{vscodepurple}{rgb}{0.5,0,0.5}
\definecolor{vscodeblue}{rgb}{0,0,0.8}
\definecolor{vscodegreen}{rgb}{0,0.5,0}
\definecolor{vscodegray}{rgb}{0.5,0.5,0.5}
\definecolor{vscodebackground}{rgb}{0.97,0.97,0.97}
\definecolor{vscodelightgray}{rgb}{0.9,0.9,0.9}

% Configuración para el estilo de C similar a VSCode
\lstdefinestyle{vscode_C}{
  backgroundcolor=\color{vscodebackground},
  commentstyle=\color{vscodegreen},
  keywordstyle=\color{vscodeblue},
  numberstyle=\tiny\color{vscodegray},
  stringstyle=\color{vscodepurple},
  basicstyle=\scriptsize\ttfamily,
  breakatwhitespace=false,
  breaklines=true,
  captionpos=b,
  keepspaces=true,
  numbers=left,
  numbersep=5pt,
  showspaces=false,
  showstringspaces=false,
  showtabs=false,
  tabsize=2,
  frame=tb,
  framerule=0pt,
  aboveskip=10pt,
  belowskip=10pt,
  xleftmargin=10pt,
  xrightmargin=10pt,
  framexleftmargin=10pt,
  framexrightmargin=10pt,
  framesep=0pt,
  rulecolor=\color{vscodelightgray},
  backgroundcolor=\color{vscodebackground},
}

%------------------------------------------------------------------------

% Comandos definidos
\newcommand{\bb}[1]{\mathbb{#1}}
\newcommand{\cc}[1]{\mathcal{#1}}

% I prefer the slanted \leq
\let\oldleq\leq % save them in case they're every wanted
\let\oldgeq\geq
\renewcommand{\leq}{\leqslant}
\renewcommand{\geq}{\geqslant}

% Si y solo si
\newcommand{\sii}{\iff}

% Letras griegas
\newcommand{\eps}{\epsilon}
\newcommand{\veps}{\varepsilon}
\newcommand{\lm}{\lambda}

\newcommand{\ol}{\overline}
\newcommand{\ul}{\underline}
\newcommand{\wt}{\widetilde}
\newcommand{\wh}{\widehat}

\let\oldvec\vec
\renewcommand{\vec}{\overrightarrow}

% Derivadas parciales
\newcommand{\del}[2]{\frac{\partial #1}{\partial #2}}
\newcommand{\Del}[3]{\frac{\partial^{#1} #2}{\partial #3^{#1}}}
\newcommand{\deld}[2]{\dfrac{\partial #1}{\partial #2}}
\newcommand{\Deld}[3]{\dfrac{\partial^{#1} #2}{\partial #3^{#1}}}


\newcommand{\AstIg}{\stackrel{(\ast)}{=}}
\newcommand{\Hop}{\stackrel{L'H\hat{o}pital}{=}}

\newcommand{\red}[1]{{\color{red}#1}} % Para integrales, destacar los cambios.

% Método de integración
\newcommand{\MetInt}[2]{
    \left[\begin{array}{c}
        #1 \\ #2
    \end{array}\right]
}

% Declarar aplicaciones
% 1. Nombre aplicación
% 2. Dominio
% 3. Codominio
% 4. Variable
% 5. Imagen de la variable
\newcommand{\Func}[5]{
    \begin{equation*}
        \begin{array}{rrll}
            #1:& #2 & \longrightarrow & #3\\
               & #4 & \longmapsto & #5
        \end{array}
    \end{equation*}
}

%------------------------------------------------------------------------



\begin{document}

    % 1. Foto de fondo
    % 2. Título
    % 3. Encabezado Izquierdo
    % 4. Color de fondo
    % 5. Coord x del titulo
    % 6. Coord y del titulo
    % 7. Fecha

    
    % 1. Foto de fondo
% 2. Título
% 3. Encabezado Izquierdo
% 4. Color de fondo
% 5. Coord x del titulo
% 6. Coord y del titulo
% 7. Fecha

\newcommand{\portada}[7]{

    \portadaBase{#1}{#2}{#3}{#4}{#5}{#6}{#7}
    \portadaBook{#1}{#2}{#3}{#4}{#5}{#6}{#7}
}

\newcommand{\portadaExamen}[7]{

    \portadaBase{#1}{#2}{#3}{#4}{#5}{#6}{#7}
    \portadaArticle{#1}{#2}{#3}{#4}{#5}{#6}{#7}
}




\newcommand{\portadaBase}[7]{

    % Tiene la portada principal y la licencia Creative Commons
    
    % 1. Foto de fondo
    % 2. Título
    % 3. Encabezado Izquierdo
    % 4. Color de fondo
    % 5. Coord x del titulo
    % 6. Coord y del titulo
    % 7. Fecha
    
    
    \thispagestyle{empty}               % Sin encabezado ni pie de página
    \newgeometry{margin=0cm}        % Márgenes nulos para la primera página
    
    
    % Encabezado
    \fancyhead[L]{\helv #3}
    \fancyhead[R]{\helv \nouppercase{\leftmark}}
    
    
    \pagecolor{#4}        % Color de fondo para la portada
    
    \begin{figure}[p]
        \centering
        \transparent{0.3}           % Opacidad del 30% para la imagen
        
        \includegraphics[width=\paperwidth, keepaspectratio]{assets/#1}
    
        \begin{tikzpicture}[remember picture, overlay]
            \node[anchor=north west, text=white, opacity=1, font=\fontsize{60}{90}\selectfont\bfseries\sffamily, align=left] at (#5, #6) {#2};
            
            \node[anchor=south east, text=white, opacity=1, font=\fontsize{12}{18}\selectfont\sffamily, align=right] at (9.7, 3) {\textbf{\href{https://losdeldgiim.github.io/}{Los Del DGIIM}}};
            
            \node[anchor=south east, text=white, opacity=1, font=\fontsize{12}{15}\selectfont\sffamily, align=right] at (9.7, 1.8) {Doble Grado en Ingeniería Informática y Matemáticas\\Universidad de Granada};
        \end{tikzpicture}
    \end{figure}
    
    
    \restoregeometry        % Restaurar márgenes normales para las páginas subsiguientes
    \pagecolor{white}       % Restaurar el color de página
    
    
    \newpage
    \thispagestyle{empty}               % Sin encabezado ni pie de página
    \begin{tikzpicture}[remember picture, overlay]
        \node[anchor=south west, inner sep=3cm] at (current page.south west) {
            \begin{minipage}{0.5\paperwidth}
                \href{https://creativecommons.org/licenses/by-nc-nd/4.0/}{
                    \includegraphics[height=2cm]{assets/Licencia.png}
                }\vspace{1cm}\\
                Esta obra está bajo una
                \href{https://creativecommons.org/licenses/by-nc-nd/4.0/}{
                    Licencia Creative Commons Atribución-NoComercial-SinDerivadas 4.0 Internacional (CC BY-NC-ND 4.0).
                }\\
    
                Eres libre de compartir y redistribuir el contenido de esta obra en cualquier medio o formato, siempre y cuando des el crédito adecuado a los autores originales y no persigas fines comerciales. 
            \end{minipage}
        };
    \end{tikzpicture}
    
    
    
    % 1. Foto de fondo
    % 2. Título
    % 3. Encabezado Izquierdo
    % 4. Color de fondo
    % 5. Coord x del titulo
    % 6. Coord y del titulo
    % 7. Fecha


}


\newcommand{\portadaBook}[7]{

    % 1. Foto de fondo
    % 2. Título
    % 3. Encabezado Izquierdo
    % 4. Color de fondo
    % 5. Coord x del titulo
    % 6. Coord y del titulo
    % 7. Fecha

    % Personaliza el formato del título
    \pretitle{\begin{center}\bfseries\fontsize{42}{56}\selectfont}
    \posttitle{\par\end{center}\vspace{2em}}
    
    % Personaliza el formato del autor
    \preauthor{\begin{center}\Large}
    \postauthor{\par\end{center}\vfill}
    
    % Personaliza el formato de la fecha
    \predate{\begin{center}\huge}
    \postdate{\par\end{center}\vspace{2em}}
    
    \title{#2}
    \author{\href{https://losdeldgiim.github.io/}{Los Del DGIIM}}
    \date{Granada, #7}
    \maketitle
    
    \tableofcontents
}




\newcommand{\portadaArticle}[7]{

    % 1. Foto de fondo
    % 2. Título
    % 3. Encabezado Izquierdo
    % 4. Color de fondo
    % 5. Coord x del titulo
    % 6. Coord y del titulo
    % 7. Fecha

    % Personaliza el formato del título
    \pretitle{\begin{center}\bfseries\fontsize{42}{56}\selectfont}
    \posttitle{\par\end{center}\vspace{2em}}
    
    % Personaliza el formato del autor
    \preauthor{\begin{center}\Large}
    \postauthor{\par\end{center}\vspace{3em}}
    
    % Personaliza el formato de la fecha
    \predate{\begin{center}\huge}
    \postdate{\par\end{center}\vspace{5em}}
    
    \title{#2}
    \author{\href{https://losdeldgiim.github.io/}{Los Del DGIIM}}
    \date{Granada, #7}
    \thispagestyle{empty}               % Sin encabezado ni pie de página
    \maketitle
    \vfill
}
    \portadaExamen{ffccA4.jpg}{Topología II\\Examen III}{Topología II. Examen III}{MidnightBlue}{-8}{28}{2025}{José Juan Urrutia Milán}

    \begin{description}
        \item[Asignatura] Topología II.
        \item[Curso Académico] 2021/22.
        \item[Grado] Doble Grado en Ingeniería Informática y Matemáticas.
        \item[Grupo] Grupo Único.
        % \item[Profesor] ---.
        \item[Descripción] Recopilación de ejercicios de repaso de varios temas.
        % \item[Fecha] 21 de noviembre de 2024.
        % \item[Duración] ---.
    \end{description}
    \newpage


    % ------------------------------------
    
    \begin{ejercicio}
        Sea $M = \frac{I\times I}{\sim}$ con $I = [0,1]$ la banda de Möbius con $(0,y)\sim (1,1-y)$. Probar que $\frac{I\times \{\frac{1}{2}\}}{\sim}$ es un retracto de deformación de $M$ y deducir que $\pi_1(M) \cong \mathbb{Z}$. Probar que el borde $\frac{I\times \{0,1\}}{\sim}$ es un lazo y hallar qué clase da en $\pi_1(M)$.
    \end{ejercicio}

    \begin{ejercicio}
        Hallar el grupo fundamental de:
        \begin{enumerate}[label=\alph*)]
            \item $(\bb{S}^1\times [0,1])\cup (\bb{D}^2\cup \{0,1\})$.
            \item $\bb{S}^1(-1,0)\cup \bb{S}^1(1,0)\cup [(-2,0),(2,0)]$.
            \item Tres esferas de $\mathbb{R}^3$ donde cada una es tangente a las otras dos.
        \end{enumerate}
    \end{ejercicio}

    \begin{ejercicio}
        Sea $G$ un grupo de homeomorfismos de $X$ actuando de manera natural. Si $G$ actúa propia y discontinuamente sobre $X$, probar que la aplicación proyección $p:X\to X/G$ es recubridora.
    \end{ejercicio}

    \begin{ejercicio}
        Si $Y$ es un espacio topológico discreto, probar que $(X\times Y,p_1:X\times Y\to X)$ es recubridor de $X$.
    \end{ejercicio}

    \begin{ejercicio}
        Probar:
        \begin{enumerate}[label=\alph*)]
            \item El grupo fundamental de un plano proyectivo menos un punto.
            \item Un espacio contráctil es arcoconexo.
            \item Si el recubridor de un espacio simplemente conexo también es simplemente conexo, entonces ambos espacios son homeomorfos.
            \item Si $\overline{X}$ es simplemente conexo, entonces $|\pi_1(X)|$ es el número de hojas.
        \end{enumerate}
    \end{ejercicio}

    % // TODO: La solucion esta en un pdf de esta carpeta

    \newpage
    \setcounter{ejercicio}{0} % Reiniciar contador de ejercicios
    \noindent
    \textbf{Solución.}

    \begin{ejercicio}
        Sea $M = \frac{I\times I}{\sim}$ con $I = [0,1]$ la banda de Möbius con $(0,y)\sim (1,1-y)$. Probar que $\frac{I\times \{\frac{1}{2}\}}{\sim}$ es un retracto de deformación de $M$ y deducir que $\pi_1(M) \cong \mathbb{Z}$. Probar que el borde $\frac{I\times \{0,1\}}{\sim}$ es un lazo y hallar qué clase da en $\pi_1(M)$.\\

        \noindent
        Nos piden probar que el conjunto $C=\frac{I\times \{\nicefrac{1}{2}\}}{\sim}$ es un retracto de deformación de $M$:
        \begin{figure}[H]
            \centering
            \begin{tikzpicture}
                \draw[thick] (0,0) -- (2,0);
                \draw[thick] (0,0) -- (0,2);
                \draw[thick] (2,0) -- (2,2);
                \draw[thick] (0,2) -- (2,2);

                \draw[purple, thick] (0,1) -- (2,1);
                \fill[green] (0,1) circle(2pt);
                \fill[green] (2,1) circle(2pt);

                \fill[blue] (0,0.5) circle(2pt);
                \fill[red] (0,1.5) circle(2pt);
                \fill[red] (2,0.5) circle(2pt);
                \fill[blue] (2,1.5) circle(2pt);
            \end{tikzpicture}
        \end{figure}
        \noindent
        Consideramos la aplicación $H:M\times [0,1]\to M$ dada por:
        \begin{equation*}
            H((x,y),t) = (1-t)(x,y) + t(x,\nicefrac{1}{2}) = (x, (1-t)y + \nicefrac{t}{2})
        \end{equation*}
        Que está bien definida, ya que si $(x,y)\sim (u,v)$, entonces:
        \begin{itemize}
            \item Bien $(x,y) = (u,v)$, en cuyo caso es claro que $H(x,y) = H(u,v)$.
            \item Bien $x=0$, $u=1$ y $y = 1-v$. En dicho caso:
                \begin{align*}
                    H(u,v) &= H(1,v) = (1, (1-t)v + \nicefrac{t}{2})  \\
                    H(x,y) &= H(0,1-v) = (0, (1-t)(1-v) + \nicefrac{t}{2}) = (0, (1-v)-t(1-v)+\nicefrac{t}{2}) \\
                           &= (0, 1-t+\nicefrac{t}{2}-(1-t)v) = (0,1-(1-t)v - \nicefrac{t}{2}) \qquad \forall t\in [0,1]
                \end{align*}
                Observamos que $H(u,v)\sim H(x,y)$.
        \end{itemize}
        Vemos que $H$ es continua. % // TODO:
        Vemos además que:
        \begin{itemize}
            \item $H((x,y),0) = (x,y) \quad \forall (x,y)\in M$.
            \item $H((x,y),1) = (x,\nicefrac{1}{2})\in C\quad \forall (x,y)\in M$.
            \item $H((u,v),1) = (u,\nicefrac{1}{2}) = (u,v)\quad \forall (u,v)\in C$.
        \end{itemize}
        Por lo que $C$ es un retracto de deformación de $M$. Ahora, si consideramos la aplicación $f:I\times \{\nicefrac{1}{2}\}\to \mathbb{S}^1$ dada por:
        \begin{equation*}
            f(x,\nicefrac{1}{2}) = (\cos(2\pi x), \sen(2\pi x))
        \end{equation*}
        Tenemos claramente que $f$ es continua y sobreyectiva. Como va de un conjunto compacto en un T2 tenemos también que es cerrada, por lo que $f$ es una identificación. Vemos finalmente que:
        \begin{equation*}
            f(0,\nicefrac{1}{2}) = (1,0) = f(1,\nicefrac{1}{2})
        \end{equation*}
        Por lo que podemos inducir $f$ al cociente $\frac{I\times \{\nicefrac{1}{2}\}}{\sim}$, lo que nos da un homeomorfismo entre $C$ y $\mathbb{S}^1$. De esta forma:
        \begin{equation*}
            \pi_1(M) = \pi_1(C) \cong \pi_1(\mathbb{S}^1) \cong \mathbb{Z}
        \end{equation*} % // TODO: No está hecho lo de que el borde es un lazo
    \end{ejercicio}

    \begin{ejercicio}
        Hallar el grupo fundamental de:
        \begin{enumerate}[label=\alph*)]
            \item $(\bb{S}^1\times [0,1])\cup (\bb{D}^2\cup \{0,1\})$.
            \item $\bb{S}^1(-1,0)\cup \bb{S}^1(1,0)\cup [(-2,0),(2,0)]$.
            \item Tres esferas de $\mathbb{R}^3$ donde cada una es tangente a las otras dos.

                Tenemos que $X = S_1\cup S_2\cup S_3$, con:
                \begin{equation*}
                    S_1\cap S_2 = \{x_{1,2}\}, \qquad S_1\cap S_3 = \{x_{1,3}\}, \qquad S_2\cap S_3 = \{x_{2,3}\}
                \end{equation*}
                \begin{figure}[H]
                    \centering
                    \begin{tikzpicture}[scale=0.8]
                        \shorthandoff{>}
                        % Esfera
                        \draw[thick] (0,0) circle [radius=2];
                        \draw[thick, dashed] (2,0) arc (0:180:2 and 0.5);
                        \draw[thick] (2,0) arc (0:-180:2 and 0.5);
                        \fill (-2,0) node[left] {$S_1$};

                        \draw[thick] (4,0) circle [radius=2];
                        \draw[thick, dashed] (6,0) arc (0:180:2 and 0.5);
                        \draw[thick] (6,0) arc (0:-180:2 and 0.5);
                        \fill (6,0) node[right] {$S_2$};

                        \draw[thick] (2,-3.5) circle [radius=2];
                        \draw[thick, dashed] (4,-3.5) arc (0:180:2 and 0.5);
                        \draw[thick] (4,-3.5) arc (0:-180:2 and 0.5);
                        \fill (4,-3.5) node[right] {$S_3$};

                        \fill (2,0) circle(2pt) node[right] {$x_{1,2}$};
                        \fill (1,-1.75) circle(2pt) node[above left] {$x_{1,3}$};
                        \fill (3,-1.75) circle(2pt) node[above right] {$x_{2,3}$};
                        % \fill (5.75,1) circle(2pt) node[right] {$p$};
                        \fill (2,-5.5) circle(2pt) node[above] {$p$};
                    \end{tikzpicture} 
                \end{figure}
                \noindent
                y consideramos los conjuntos:
                \begin{align*}
                    U &= X\setminus \{x_{1,2}\}\\ 
                    V &=X\setminus \{p\}, \quad p\in S_3\setminus \{x_{1,3},x_{2,3}\}
                \end{align*}
                tenemos que:
                \begin{itemize}
                    \item $X = U\cup V$ con $U,V$ abiertos.
                    \item $U,V$ y $U\cap V$ son arcoconexos.
                    \item $U$ tiene por retracto de deformación el conjunto $S_3$, por lo que $U$ es simplemente conexo.
                    \item $U\cap V$ tiene por retracto de deformación el conjunto $S_3\setminus \{q\}$, que a su vez es contráctil, por lo que $U\cap V$ es contráctil, luego simplemente conexo.
                    \item $V$ tiene como retracto de deformación el conjunto $Y$:
                    \begin{figure}[H]
                        \centering
                        \begin{tikzpicture}[scale=0.8]
                            \shorthandoff{>}
                            % Esfera
                            \draw[thick] (0,0) circle [radius=2];
                            \draw[thick, dashed] (2,0) arc (0:180:2 and 0.5);
                            \draw[thick] (2,0) arc (0:-180:2 and 0.5);
                            \fill (-2,0) node[left] {$S_1$};

                            \draw[thick] (4,0) circle [radius=2];
                            \draw[thick, dashed] (6,0) arc (0:180:2 and 0.5);
                            \draw[thick] (6,0) arc (0:-180:2 and 0.5);
                            \fill (6,0) node[right] {$S_2$};

                            \draw[thick, bend right=23] (3,-1.75) to (1,-1.75);
                            \fill (2,-1.7) node[below] {$R$};
                            \fill (2,-1.5) circle(2pt) node[above] {$q$};
                            \fill (0,2) circle(2pt) node[above] {$t$};
                        \end{tikzpicture} 
                    \end{figure}
                    \noindent
                    Al que tratamos de calcularle el grupo fundalmental, usando para ello nuevamente el Teorema de Seifert-van Kampen. Si denotamos por $R$ al segmento restante de $S_3$ en $Y$ que une $x_{1,3}$ con $x_{2,3}$ (luego $Y = S_1\cup S_2\cup R$), consideramos ahora los conjuntos:
                    \begin{align*}
                        W &= Y\setminus \{q\}, \qquad q\in R\setminus \{x_{1,3}, x_{2,3}\} \\
                        O &= Y\setminus \{t\}, \qquad t\in S_1\setminus \{x_{1,2}, x_{1,3}\}
                    \end{align*}
                    Tenemos que:
                    \begin{itemize}
                        \item $Y=W\cup O$ con $W$ y $O$ abiertos.
                        \item $W,O$ y $O\cap W$ son arcoconexos.
                        \item $W$ tiene a $S_1\cup S_2$ como retracto de deformación, y en teoría vimos que este espacio topológico es simplemente conexo.
                        \item $W\cap O$ tiene a $(S_1\setminus \{t\})\cup S_2$ como retracto de deformación, y este último tiene a su vez a $S_2$ como retracto de deformación, por lo que $W\cap O$ es simplemente conexo.
                        \item $O$ tiene el conjunto:
                        \begin{figure}[H]
                            \centering
                            \begin{tikzpicture}[scale=0.8]
                                \shorthandoff{>}
                                \draw[thick] (4,0) circle [radius=2];
                                \draw[thick, dashed] (6,0) arc (0:180:2 and 0.5);
                                \draw[thick] (6,0) arc (0:-180:2 and 0.5);
                                \fill (6,0) node[right] {$S_2$};

                                \draw[thick, bend right=23] (3,-1.75) to (1,-1.75);
                                \draw[thick, bend left=25] (2,0) to (1,-1.75);
                            \end{tikzpicture} 
                        \end{figure}
                        como retracto de deformación, y en teoría tambien se calculó el grupo fundamental de este espacio topológico, pues podemos ver $R$ como la imagen de cierto arco $\alpha$, por lo que sabemos que dichos espacio topológico tiene grupo fundamental isomorfo a $\mathbb{Z}$, por lo que el grupo fundalmental de $O$ es isomorfo a $\mathbb{Z}$.
                    \end{itemize}
                    Aplicando el Teorema de Seifert-van Kampen obtenemos que:
                    \begin{equation*}
                        \pi_1(V)\cong \pi_1(Y) \cong \pi_1(W) \ast \pi_1(O) \cong \{1\}\ast \mathbb{Z} \cong \mathbb{Z}
                    \end{equation*}
                \end{itemize}
                Aplicando nuevamente el Teorema de Seifert-van Kampen obtenemos:
                \begin{equation*}
                    \pi_1(X)\cong \pi_1(U) \ast \pi_1k(V) \cong \{1\}\ast \mathbb{Z} \cong \mathbb{Z}
                \end{equation*}

        \end{enumerate}
    \end{ejercicio}

    \begin{ejercicio}
        Sea $G$ un grupo de homeomorfismos de $X$ actuando de manera natural. Si $G$ actúa propia y discontinuamente sobre $X$, probar que la aplicación proyección $p:X\to X/G$ es recubridora.
    \end{ejercicio}

    \begin{ejercicio}
        Si $Y$ es un espacio topológico discreto, probar que $(X\times Y,p_1:X\times Y\to X)$ es recubridor de $X$.\\

        \noindent
        Sabemos ya que $p_1$ es una aplicación continua y sobreyectiva, por ser la proyección en primera coordenada del espacio topológico producto $X\times Y$.

        \noindent
        Dado $x\in X$, si consideramos como $O_x$ cualquier entorno abierto de $x$ tendremos entonces que:
        \begin{equation*}
            p_1^{-1}(O_x) = O_x\times Y = \biguplus_{y \in Y}O_x\times \{y\}
        \end{equation*}
        Con cada $O_x\times \{y\}$ abierto para todo $y\in Y$, puesto que $O_x$ es abierto en $X$ y $\{y\}$ abierto en $Y$ por tener $Y$ la topología discreta. Es obvio finalmente que para cada $y\in Y$ tenemos que $p_1\big|_{O_x\times \{y\}}:O_x\times \{y\}\to O_x$ es un homeomorfismo.
    \end{ejercicio}

    \begin{ejercicio}
        Probar:
        \begin{enumerate}[label=\alph*)]
            \item El grupo fundamental de un plano proyectivo menos un punto.
            \item Un espacio contráctil es arcoconexo.

                Si $X$ es un espacio topológico contráctil, existe entonces una aplicación continua $H:X\times [0,1]\to X$ de forma que:
                \begin{equation*}
                    H(x,0) = x, \qquad H(x,1) = x_0 \qquad \forall x\in X
                \end{equation*}
                Por tanto, si consideramos $\alpha_x:[0,1]\to X$ dada por:
                \begin{equation*}
                    \alpha_x(t) = H(x,t)
                \end{equation*}
                Tenemos que $\alpha_x$ es una aplicación continua, luego es un arco que une $\alpha_x(0) = H(x,0) = x$ con $\alpha_x(1) = H(x,1) = x_0$. Como este lazo lo podemos considerar para todo $x\in X$, dados dos puntos arbitrarios de $X$ $x$ e $y$, el lazo $\alpha_x\ast \widetilde{\alpha_y}$ une $x$ con $y$, por lo que $X$ es arcoconexo.
            \item Si el recubridor de un espacio simplemente conexo también es simplemente conexo, entonces ambos espacios son homeomorfos.

                Sea $p:R\to B$ una aplicación recubridora entre dos espacios topológicos simplemente conexos, por ser $p$ una aplicación recubridora tenemos que es continua, sobreyectiva y abierta. Para probar que $p$ es un homeomorfismo, basta ver que $p$ es inyectiva.
                \begin{description}
                    \item [Opción 1.] Como $R$ es simplemente conexo tenemos para cada$ b\in B$ que la aplicación correspondencia del levantamiento $\phi:\pi_1(B,b)\to p^{-1}(b)$ es biyectiva, por lo que:
                        \begin{equation*}
                            |p^{-1}(b)| = |\pi_1(B,b)| = |\{1\}| = 1
                        \end{equation*}
                        de donde cada elemento de $B$ tiene una única preimagen por $p$, por lo que $p$ es inyectiva.
                    \item [Opción 2.] Para ello, dados $x,y\in R$ con $p(x) = p(y)$, como $R$ es arcoconexo (por ser simplemente conexo) tenemos que existe un arco $\alpha$ que une $x$ con $y$, por lo que $p\circ \alpha$ es un arco de $B$ que une $p(x)$ con $p(y) = p(x)$, por lo que es un lazo basado en $p(x)$. Como $B$ es simplemente conexo:
                        \begin{equation*}
                            [p\circ \alpha] = [\varepsilon_{p(x)}]
                        \end{equation*}
                        Ahora, tenemos que $\alpha$ es el único levantamiento de $p\circ \alpha$ que comienza en $x$, así como que $\varepsilon_x$ es el único levantamiento de $\varepsilon_{p(x)}$ que comienza en $x$. Tenemos por tanto que $\alpha$ debe ser un lazo, y además $[\alpha] = [\varepsilon_x]$, de donde $x=y$, por lo que $p$ es inyectiva, luego es un homeomorfismo.
                \end{description}
                Notemos que en este ejercicio solo hemos usado que $R$ es arcoconexo y que $B$ es simplemente conexo. Aunque parezca que la condición ``$R$ simplemente conexo'' es necesaria para hacer el ejercicio según la opción 1, si $R$ es solo arcoconexo tenemos que la correspondencia del levantamiento es sobreyectiva y por tanto que:
                \begin{equation*}
                    1 = |p^{-1}(b)| \leq |\pi_1(B,b)| = 1
                \end{equation*}
                Y habríamos llegado a la misma conclusión
            \item Si $\overline{X}$ es simplemente conexo, entonces $|\pi_1(X)|$ es el número de hojas.
        \end{enumerate}
    \end{ejercicio}

\end{document}
