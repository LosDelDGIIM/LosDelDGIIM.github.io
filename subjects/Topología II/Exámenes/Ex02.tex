\documentclass[12pt]{article}

% Idioma y codificación
\usepackage[spanish, es-tabla]{babel}       %es-tabla para que se titule "Tabla"
\usepackage[utf8]{inputenc}

% Márgenes
\usepackage[a4paper,top=3cm,bottom=2.5cm,left=3cm,right=3cm]{geometry}

% Comentarios de bloque
\usepackage{verbatim}

% Paquetes de links
\usepackage[hidelinks]{hyperref}    % Permite enlaces
\usepackage{url}                    % redirecciona a la web

% Más opciones para enumeraciones
\usepackage{enumitem}

% Personalizar la portada
\usepackage{titling}

% Paquetes de tablas
\usepackage{multirow}


%------------------------------------------------------------------------

%Paquetes de figuras
\usepackage{caption}
\usepackage{subcaption} % Figuras al lado de otras
\usepackage{float}      % Poner figuras en el sitio indicado H.


% Paquetes de imágenes
\usepackage{graphicx}       % Paquete para añadir imágenes
\usepackage{transparent}    % Para manejar la opacidad de las figuras

% Paquete para usar colores
\usepackage[dvipsnames]{xcolor}
\usepackage{pagecolor}      % Para cambiar el color de la página

% Habilita tamaños de fuente mayores
\usepackage{fix-cm}

% Para los gráficos
\usepackage{tikz}

% Para poder situar los nodos en los grafos
\usetikzlibrary{positioning}


%------------------------------------------------------------------------

% Paquetes de matemáticas
\usepackage{mathtools, amsfonts, amssymb, mathrsfs}
\usepackage[makeroom]{cancel}     % Simplificar tachando
\usepackage{polynom}    % Divisiones y Ruffini
\usepackage{units} % Para poner fracciones diagonales con \nicefrac

\usepackage{pgfplots}   %Representar funciones
\pgfplotsset{compat=1.18}  % Versión 1.18

\usepackage{tikz-cd}    % Para usar diagramas de composiciones
\usetikzlibrary{calc}   % Para usar cálculo de coordenadas en tikz

%Definición de teoremas, etc.
\usepackage{amsthm}
%\swapnumbers   % Intercambia la posición del texto y de la numeración

\theoremstyle{plain}

\makeatletter
\@ifclassloaded{article}{
  \newtheorem{teo}{Teorema}[section]
}{
  \newtheorem{teo}{Teorema}[chapter]  % Se resetea en cada chapter
}
\makeatother

\newtheorem{coro}{Corolario}[teo]           % Se resetea en cada teorema
\newtheorem{prop}[teo]{Proposición}         % Usa el mismo contador que teorema
\newtheorem{lema}[teo]{Lema}                % Usa el mismo contador que teorema

\theoremstyle{remark}
\newtheorem*{observacion}{Observación}

\theoremstyle{definition}

\makeatletter
\@ifclassloaded{article}{
  \newtheorem{definicion}{Definición} [section]     % Se resetea en cada chapter
}{
  \newtheorem{definicion}{Definición} [chapter]     % Se resetea en cada chapter
}
\makeatother

\newtheorem*{notacion}{Notación}
\newtheorem*{ejemplo}{Ejemplo}
\newtheorem*{ejercicio*}{Ejercicio}             % No numerado
\newtheorem{ejercicio}{Ejercicio} [section]     % Se resetea en cada section


% Modificar el formato de la numeración del teorema "ejercicio"
\renewcommand{\theejercicio}{%
  \ifnum\value{section}=0 % Si no se ha iniciado ninguna sección
    \arabic{ejercicio}% Solo mostrar el número de ejercicio
  \else
    \thesection.\arabic{ejercicio}% Mostrar número de sección y número de ejercicio
  \fi
}


% \renewcommand\qedsymbol{$\blacksquare$}         % Cambiar símbolo QED
%------------------------------------------------------------------------

% Paquetes para encabezados
\usepackage{fancyhdr}
\pagestyle{fancy}
\fancyhf{}

\newcommand{\helv}{ % Modificación tamaño de letra
\fontfamily{}\fontsize{12}{12}\selectfont}
\setlength{\headheight}{15pt} % Amplía el tamaño del índice


%\usepackage{lastpage}   % Referenciar última pag   \pageref{LastPage}
\fancyfoot[C]{\thepage}

%------------------------------------------------------------------------

% Conseguir que no ponga "Capítulo 1". Sino solo "1."
\makeatletter
\@ifclassloaded{book}{
  \renewcommand{\chaptermark}[1]{\markboth{\thechapter.\ #1}{}} % En el encabezado
    
  \renewcommand{\@makechapterhead}[1]{%
  \vspace*{50\p@}%
  {\parindent \z@ \raggedright \normalfont
    \ifnum \c@secnumdepth >\m@ne
      \huge\bfseries \thechapter.\hspace{1em}\ignorespaces
    \fi
    \interlinepenalty\@M
    \Huge \bfseries #1\par\nobreak
    \vskip 40\p@
  }}
}
\makeatother

%------------------------------------------------------------------------
% Paquetes de cógido
\usepackage{minted}
\renewcommand\listingscaption{Código fuente}

\usepackage{fancyvrb}
% Personaliza el tamaño de los números de línea
\renewcommand{\theFancyVerbLine}{\small\arabic{FancyVerbLine}}

% Estilo para C++
\newminted{cpp}{
    frame=lines,
    framesep=2mm,
    baselinestretch=1.2,
    linenos,
    escapeinside=||
}

% para minted
\definecolor{LightGray}{rgb}{0.95,0.95,0.92}
\setminted{
    linenos=true,
    stepnumber=5,
    numberfirstline=true,
    autogobble,
    breaklines=true,
    breakautoindent=true,
    breaksymbolleft=,
    breaksymbolright=,
    breaksymbolindentleft=0pt,
    breaksymbolindentright=0pt,
    breaksymbolsepleft=0pt,
    breaksymbolsepright=0pt,
    fontsize=\footnotesize,
    bgcolor=LightGray,
    numbersep=10pt
}


\usepackage{listings} % Para incluir código desde un archivo

\renewcommand\lstlistingname{Código Fuente}
\renewcommand\lstlistlistingname{Índice de Códigos Fuente}

% Definir colores
\definecolor{vscodepurple}{rgb}{0.5,0,0.5}
\definecolor{vscodeblue}{rgb}{0,0,0.8}
\definecolor{vscodegreen}{rgb}{0,0.5,0}
\definecolor{vscodegray}{rgb}{0.5,0.5,0.5}
\definecolor{vscodebackground}{rgb}{0.97,0.97,0.97}
\definecolor{vscodelightgray}{rgb}{0.9,0.9,0.9}

% Configuración para el estilo de C similar a VSCode
\lstdefinestyle{vscode_C}{
  backgroundcolor=\color{vscodebackground},
  commentstyle=\color{vscodegreen},
  keywordstyle=\color{vscodeblue},
  numberstyle=\tiny\color{vscodegray},
  stringstyle=\color{vscodepurple},
  basicstyle=\scriptsize\ttfamily,
  breakatwhitespace=false,
  breaklines=true,
  captionpos=b,
  keepspaces=true,
  numbers=left,
  numbersep=5pt,
  showspaces=false,
  showstringspaces=false,
  showtabs=false,
  tabsize=2,
  frame=tb,
  framerule=0pt,
  aboveskip=10pt,
  belowskip=10pt,
  xleftmargin=10pt,
  xrightmargin=10pt,
  framexleftmargin=10pt,
  framexrightmargin=10pt,
  framesep=0pt,
  rulecolor=\color{vscodelightgray},
  backgroundcolor=\color{vscodebackground},
}

%------------------------------------------------------------------------

% Comandos definidos
\newcommand{\bb}[1]{\mathbb{#1}}
\newcommand{\cc}[1]{\mathcal{#1}}

% I prefer the slanted \leq
\let\oldleq\leq % save them in case they're every wanted
\let\oldgeq\geq
\renewcommand{\leq}{\leqslant}
\renewcommand{\geq}{\geqslant}

% Si y solo si
\newcommand{\sii}{\iff}

% Letras griegas
\newcommand{\eps}{\epsilon}
\newcommand{\veps}{\varepsilon}
\newcommand{\lm}{\lambda}

\newcommand{\ol}{\overline}
\newcommand{\ul}{\underline}
\newcommand{\wt}{\widetilde}
\newcommand{\wh}{\widehat}

\let\oldvec\vec
\renewcommand{\vec}{\overrightarrow}

% Derivadas parciales
\newcommand{\del}[2]{\frac{\partial #1}{\partial #2}}
\newcommand{\Del}[3]{\frac{\partial^{#1} #2}{\partial #3^{#1}}}
\newcommand{\deld}[2]{\dfrac{\partial #1}{\partial #2}}
\newcommand{\Deld}[3]{\dfrac{\partial^{#1} #2}{\partial #3^{#1}}}


\newcommand{\AstIg}{\stackrel{(\ast)}{=}}
\newcommand{\Hop}{\stackrel{L'H\hat{o}pital}{=}}

\newcommand{\red}[1]{{\color{red}#1}} % Para integrales, destacar los cambios.

% Método de integración
\newcommand{\MetInt}[2]{
    \left[\begin{array}{c}
        #1 \\ #2
    \end{array}\right]
}

% Declarar aplicaciones
% 1. Nombre aplicación
% 2. Dominio
% 3. Codominio
% 4. Variable
% 5. Imagen de la variable
\newcommand{\Func}[5]{
    \begin{equation*}
        \begin{array}{rrll}
            #1:& #2 & \longrightarrow & #3\\
               & #4 & \longmapsto & #5
        \end{array}
    \end{equation*}
}

%------------------------------------------------------------------------


\usetikzlibrary{decorations.markings}

\begin{document}

    % 1. Foto de fondo
    % 2. Título
    % 3. Encabezado Izquierdo
    % 4. Color de fondo
    % 5. Coord x del titulo
    % 6. Coord y del titulo
    % 7. Fecha

    
    % 1. Foto de fondo
% 2. Título
% 3. Encabezado Izquierdo
% 4. Color de fondo
% 5. Coord x del titulo
% 6. Coord y del titulo
% 7. Fecha

\newcommand{\portada}[7]{

    \portadaBase{#1}{#2}{#3}{#4}{#5}{#6}{#7}
    \portadaBook{#1}{#2}{#3}{#4}{#5}{#6}{#7}
}

\newcommand{\portadaExamen}[7]{

    \portadaBase{#1}{#2}{#3}{#4}{#5}{#6}{#7}
    \portadaArticle{#1}{#2}{#3}{#4}{#5}{#6}{#7}
}




\newcommand{\portadaBase}[7]{

    % Tiene la portada principal y la licencia Creative Commons
    
    % 1. Foto de fondo
    % 2. Título
    % 3. Encabezado Izquierdo
    % 4. Color de fondo
    % 5. Coord x del titulo
    % 6. Coord y del titulo
    % 7. Fecha
    
    
    \thispagestyle{empty}               % Sin encabezado ni pie de página
    \newgeometry{margin=0cm}        % Márgenes nulos para la primera página
    
    
    % Encabezado
    \fancyhead[L]{\helv #3}
    \fancyhead[R]{\helv \nouppercase{\leftmark}}
    
    
    \pagecolor{#4}        % Color de fondo para la portada
    
    \begin{figure}[p]
        \centering
        \transparent{0.3}           % Opacidad del 30% para la imagen
        
        \includegraphics[width=\paperwidth, keepaspectratio]{assets/#1}
    
        \begin{tikzpicture}[remember picture, overlay]
            \node[anchor=north west, text=white, opacity=1, font=\fontsize{60}{90}\selectfont\bfseries\sffamily, align=left] at (#5, #6) {#2};
            
            \node[anchor=south east, text=white, opacity=1, font=\fontsize{12}{18}\selectfont\sffamily, align=right] at (9.7, 3) {\textbf{\href{https://losdeldgiim.github.io/}{Los Del DGIIM}}};
            
            \node[anchor=south east, text=white, opacity=1, font=\fontsize{12}{15}\selectfont\sffamily, align=right] at (9.7, 1.8) {Doble Grado en Ingeniería Informática y Matemáticas\\Universidad de Granada};
        \end{tikzpicture}
    \end{figure}
    
    
    \restoregeometry        % Restaurar márgenes normales para las páginas subsiguientes
    \pagecolor{white}       % Restaurar el color de página
    
    
    \newpage
    \thispagestyle{empty}               % Sin encabezado ni pie de página
    \begin{tikzpicture}[remember picture, overlay]
        \node[anchor=south west, inner sep=3cm] at (current page.south west) {
            \begin{minipage}{0.5\paperwidth}
                \href{https://creativecommons.org/licenses/by-nc-nd/4.0/}{
                    \includegraphics[height=2cm]{assets/Licencia.png}
                }\vspace{1cm}\\
                Esta obra está bajo una
                \href{https://creativecommons.org/licenses/by-nc-nd/4.0/}{
                    Licencia Creative Commons Atribución-NoComercial-SinDerivadas 4.0 Internacional (CC BY-NC-ND 4.0).
                }\\
    
                Eres libre de compartir y redistribuir el contenido de esta obra en cualquier medio o formato, siempre y cuando des el crédito adecuado a los autores originales y no persigas fines comerciales. 
            \end{minipage}
        };
    \end{tikzpicture}
    
    
    
    % 1. Foto de fondo
    % 2. Título
    % 3. Encabezado Izquierdo
    % 4. Color de fondo
    % 5. Coord x del titulo
    % 6. Coord y del titulo
    % 7. Fecha


}


\newcommand{\portadaBook}[7]{

    % 1. Foto de fondo
    % 2. Título
    % 3. Encabezado Izquierdo
    % 4. Color de fondo
    % 5. Coord x del titulo
    % 6. Coord y del titulo
    % 7. Fecha

    % Personaliza el formato del título
    \pretitle{\begin{center}\bfseries\fontsize{42}{56}\selectfont}
    \posttitle{\par\end{center}\vspace{2em}}
    
    % Personaliza el formato del autor
    \preauthor{\begin{center}\Large}
    \postauthor{\par\end{center}\vfill}
    
    % Personaliza el formato de la fecha
    \predate{\begin{center}\huge}
    \postdate{\par\end{center}\vspace{2em}}
    
    \title{#2}
    \author{\href{https://losdeldgiim.github.io/}{Los Del DGIIM}}
    \date{Granada, #7}
    \maketitle
    
    \tableofcontents
}




\newcommand{\portadaArticle}[7]{

    % 1. Foto de fondo
    % 2. Título
    % 3. Encabezado Izquierdo
    % 4. Color de fondo
    % 5. Coord x del titulo
    % 6. Coord y del titulo
    % 7. Fecha

    % Personaliza el formato del título
    \pretitle{\begin{center}\bfseries\fontsize{42}{56}\selectfont}
    \posttitle{\par\end{center}\vspace{2em}}
    
    % Personaliza el formato del autor
    \preauthor{\begin{center}\Large}
    \postauthor{\par\end{center}\vspace{3em}}
    
    % Personaliza el formato de la fecha
    \predate{\begin{center}\huge}
    \postdate{\par\end{center}\vspace{5em}}
    
    \title{#2}
    \author{\href{https://losdeldgiim.github.io/}{Los Del DGIIM}}
    \date{Granada, #7}
    \thispagestyle{empty}               % Sin encabezado ni pie de página
    \maketitle
    \vfill
}
    \portadaExamen{ffccA4.jpg}{Topología II\\Examen II}{Topología II. Examen II}{MidnightBlue}{-8}{28}{2025}{}

    \begin{description}
        \item[Asignatura] Topología II.
        \item[Curso Académico] 2024/25.
        \item[Grado] Doble Grado en Ingeniería Informática y Matemáticas.
        \item[Grupo] Grupo Único.
        % \item[Profesor] ---.
        \item[Descripción] Prueba del Tema 1.
        \item[Fecha] 21 de noviembre de 2024.
        % \item[Duración] ---.
    \end{description}
    \newpage


    % ------------------------------------
    
    \begin{ejercicio}
        Elija uno de los siguientes ejercicios:
        \begin{enumerate}[label=\alph*)]
            \item Sea $f:\bb{S}^2\to \bb{S}^2$ una aplicación continua e inyectiva. Demuestra que $f$ es sobreyectiva.
            \item Prueba que no existe una retracción $r:\bb{S}^2\to E$ donde $E$ es el ecuador de la esfera, es decir
                \begin{equation*}
                    E = \{(x,y,z) \in \bb{S}^2 : z=0\}
                \end{equation*}
        \end{enumerate}
    \end{ejercicio}

    \begin{ejercicio}
        Sean $S_1$ y $S_2$ las esferas de $\mathbb{R}^3$ de radio 1 con centros respectivos en los puntos $(2,0,0)$ y $(-2,0,0)$, calcula el grupo fundamental de
        \begin{equation*}
            X = S_1\cup S_2\cup R
        \end{equation*}
        en el origen, donde $R = \{(x,0,0)\in \mathbb{R}^3 : x\in \mathbb{R}\}$. Determina lazos basados en el origen cuyas clases de equivalencia generen $\pi_1(X,(0,0,0))$.
    \end{ejercicio}

    \newpage
    \setcounter{ejercicio}{0} % Reiniciar contador de ejercicios
    \noindent
    \textbf{Solución.}

    \begin{ejercicio}
        Elija uno de los siguientes ejercicios:
        \begin{enumerate}[label=\alph*)]
            \item Sea $f:\bb{S}^2\to \bb{S}^2$ una aplicación continua e inyectiva. Demuestra que $f$ es sobreyectiva.

                Supuesto que $f:\mathbb{S}^2\to\mathbb{S}^2$  es una aplicación continua, inyectiva y no sobreyectiva, tenemos entonces que existe $y\in \mathbb{S}^2$ de forma que $f(x)\neq y \quad \forall x\in \mathbb{S}^2$. Como $\mathbb{S}^2\setminus \{y\}$ es homeomorfo a $\mathbb{R}^2$, podemos considerar un homeomorfismo $g:\mathbb{S}^2\setminus \{y\}\to \mathbb{R}^2$. De esta forma, tenemos que $(g\circ f):\mathbb{S}^2\to \mathbb{R}^2$ es una aplicación continua e inyectiva (como composición de aplicaciones inyectivas), pero esto contradice el Teorema de Borsuk-Ulam: si $h:\mathbb{S}^2\to \mathbb{R}^2$ es continua entonces existe $x_0\in \mathbb{S}^2$ tal que $h(x_0) = h(-x_0)$.
            \item Prueba que no existe una retracción $r:\bb{S}^2\to E$ donde $E$ es el ecuador de la esfera, es decir
                \begin{equation*}
                    E = \{(x,y,z) \in \bb{S}^2 : z=0\}
                \end{equation*}

                Por reducción al absurdo, si existiera una retracción $r:\bb{S}^2\to E$, tendríamos entonces que la inclusión $i:E\hookrightarrow \mathbb{S}^2$ induce un homomorfismo inyectivo entre grupos fundamentales (no especificamos el punto en los grupos fundamentales, pensando en que como los dos conjuntos son arcoconexos estos serán isomorfos):
                \begin{equation*}
                    i_\ast:\pi_1(E) \to \pi_1\left(\mathbb{S}^2\right)
                \end{equation*}
                Como $\pi_1\left(\mathbb{S}^2\right)=\{1\}$, ha de ser entonces $\pi_1(E)=\{1\}$ para que $i_\ast$ sea inyectiva. Sin embargo, tenemos que:
                \begin{equation*}
                    E = \mathbb{S}^1\times \{0\} \cong \mathbb{S}^1
                \end{equation*}
                Por lo que:
                \begin{equation*}
                    \pi_1(E)\cong \pi_1(\mathbb{S}^1)\cong \mathbb{Z}
                \end{equation*}
                Hemos llegado a una \underline{contradicción}.
        \end{enumerate}
    \end{ejercicio}

    \begin{ejercicio}
        Sean $S_1$ y $S_2$ las esferas de $\mathbb{R}^3$ de radio 1 con centros respectivos en los puntos $(2,0,0)$ y $(-2,0,0)$, calcula el grupo fundamental de
        \begin{equation*}
            X = S_1\cup S_2\cup R
        \end{equation*}
        en el origen, donde $R = \{(x,0,0)\in \mathbb{R}^3 : x\in \mathbb{R}\}$. Determina lazos basados en el origen cuyas clases de equivalencia generen $\pi_1(X,(0,0,0))$.\\

        \noindent
        Tenemos el conjunto:
        \begin{figure}[H]
            \centering
            \begin{tikzpicture}[scale=0.8]
                \shorthandoff{>}
                % Esfera
                \draw[thick] (-4,0) circle [radius=2];
                \draw[thick, dashed] (-2,0) arc (0:180:2 and 0.5);
                \draw[thick] (-2,0) arc (0:-180:2 and 0.5);

                \draw[thick] (4,0) circle [radius=2];
                \draw[thick, dashed] (6,0) arc (0:180:2 and 0.5);
                \draw[thick] (6,0) arc (0:-180:2 and 0.5);

                \draw[thick] (-8,0) -- (-6,0);
                \draw[thick, dashed] (-6,0) -- (-2,0);
                \draw[thick] (-2,0) -- (2,0);
                \draw[thick, dashed] (2,0) -- (6,0);
                \draw[thick] (6,0) -- (8,0);
            \end{tikzpicture} 
        \end{figure}
        Que tiene como retracto de deformación el conjunto:
        \begin{equation*}
            Y = S_1\cup S_2\cup [(-3,0,0), (3,0,0)]
        \end{equation*}
        \begin{figure}[H]
            \centering
            \begin{tikzpicture}[scale=0.8]
                \shorthandoff{>}
                % Esfera
                \draw[thick] (-4,0) circle [radius=2];
                \draw[thick, dashed] (-2,0) arc (0:180:2 and 0.5);
                \draw[thick] (-2,0) arc (0:-180:2 and 0.5);

                \draw[thick] (4,0) circle [radius=2];
                \draw[thick, dashed] (6,0) arc (0:180:2 and 0.5);
                \draw[thick] (6,0) arc (0:-180:2 and 0.5);

                \draw[thick, dashed] (-6,0) -- (-2,0);
                \draw[thick] (-2,0) -- (2,0);
                \draw[thick, dashed] (2,0) -- (6,0);
            \end{tikzpicture} 
        \end{figure}
        \noindent
        Podemos considerar los conjuntos:
        \begin{equation*}
            U = Y\setminus [(-3,0,0), (-2,0,0)], \qquad V = Y\setminus [(2,0,0), (3,0,0)]
        \end{equation*}
        \begin{figure}[H]
            \centering
            \begin{tikzpicture}[scale=0.8]
                \shorthandoff{>}
                % Esfera
                \draw[thick] (-4,0) circle [radius=2];
                \draw[thick, dashed] (-2,0) arc (0:180:2 and 0.5);
                \draw[thick] (-2,0) arc (0:-180:2 and 0.5);

                \draw[thick] (4,0) circle [radius=2];
                \draw[thick, dashed] (6,0) arc (0:180:2 and 0.5);
                \draw[thick] (6,0) arc (0:-180:2 and 0.5);

                \draw[thick, red, dashed] (-6,0) -- (-4,0);
                \draw[thick, dashed] (-4,0) -- (-2,0);
                \draw[thick] (-2,0) -- (2,0);
                \draw[thick, dashed] (2,0) -- (4,0);
                \draw[thick, blue, dashed] (4,0) -- (6,0);
            \end{tikzpicture} 
        \end{figure}
        Y tenemos que:
        \begin{itemize}
            \item Claramente $Y = U\cup V$ con $U$ y $V$ abiertos.
            \item $U$, $V$ y $U\cap V$ son arcoconexos como unión de ciertos conjuntos arcoconexos con intersección no vacía.
            \item $U\cap V$ tiene como retracto de deformación el conjunto $[(-1,0,0), (1,0,0)]$, que claramente es simplemente conexo.
            \item $U$ tiene como retracto de deformación el conjunto $Z = S_2\cup [(1,0,0), (3,0,0)]$:
                \begin{figure}[H]
                    \centering
                    \begin{tikzpicture}[scale=0.8]
                        \shorthandoff{>}
                        % Esfera
                        \draw[thick] (4,0) circle [radius=2];
                        \draw[thick, dashed] (6,0) arc (0:180:2 and 0.5);
                        \draw[thick] (6,0) arc (0:-180:2 and 0.5);
                        \draw[thick, dashed] (2,0) -- (6,0);

                        \fill (4,2) circle(2pt) node[below] {$q$};
                        \fill (4,0) circle(2pt) node[above right] {$p$};
                    \end{tikzpicture} 
                \end{figure}
                Si en este tomamos $p = (2,0,0)$, $q=(2,0,1)$ y consideramos:
                \begin{equation*}
                    W = Z\setminus \{p\}, \qquad K = Z\setminus \{q\}
                \end{equation*}
                Tenemos:
                \begin{itemize}
                    \item $W$ y $K$ son abiertos con $Z=W\cup K$.
                    \item $W$, $K$ y $W\cap K$ son arcoconexos como unión de conjuntos arcoconexos con intersección no vacía.
                    \item $W$ tiene a $S_2$ como retracto de deformación, por lo que $W$ es simplemente conexo.
                    \item $W\cap K$ tiene a $S_2\setminus \{q\}$ como retracto de deformación, que es homeomorfo a $\mathbb{R}^2$, por lo que $W\cap K$ es simplemente conexo.
                    \item $K$ tiene al conjunto:
                        \begin{equation*}
                            [(1,0,0),(3,0,0)] \cup \{(x,0,z)\in S_2 : z\leq 0\}
                        \end{equation*}
                        \begin{figure}[H]
                            \centering
                            \begin{tikzpicture}
                                \draw[thick] (0,0) -- (3,0);
                                % \draw[thick, bend right=100] (0,0) to (3,0);
                                \draw[thick] (0,0) .. controls (0,-2) and (3,-2) .. (3,0);
                            \end{tikzpicture}
                        \end{figure}
                        como retracto de deformación, homeomorfo a $\mathbb{S}^1$, por lo que $\pi_1(K)\cong \mathbb{Z}$.
                \end{itemize}
                Aplicando el Teorema de Seifert-van Kampen obtenemos que $\pi_1(U)\cong \mathbb{Z}$.
            \item $U$ y $V$ son claramente homeomorfos (basta considerar una rotación), por lo que también será $\pi_1(V) \cong \mathbb{Z}$.
        \end{itemize}
        Aplicando el Teorema de Seifert-van Kampen concluimos que:
        \begin{equation*}
            \pi_1(X) = \pi_1(Y) = \pi_1(U)\ast \pi_1(V) \cong \mathbb{Z}\ast\mathbb{Z}
        \end{equation*}
        Ahora, vimos que $\pi_1(U)\cong \mathbb{Z}$, por lo que dicho grupo ha de tener un generador, tal y como lo es la clase del lazo $\beta\in \Omega(U,(0,0,0))$:
        \begin{figure}[H]
            \centering
            \begin{tikzpicture}[scale=0.8]
                \shorthandoff{>}
                % Esfera
                \draw[thick] (-4,0) circle [radius=2];
                \draw[thick, dashed] (-2,0) arc (0:180:2 and 0.5);
                \draw[thick] (-2,0) arc (0:-180:2 and 0.5);

                \draw[thick] (4,0) circle [radius=2];
                \draw[thick, dashed] (6,0) arc (0:180:2 and 0.5);
                \draw[thick] (6,0) arc (0:-180:2 and 0.5);

                \draw[thick, dashed] (-4,0) -- (-2,0);
                \draw[thick] (-2,0) -- (2,0);
                % \draw[thick, dashed] (2,0) -- (4,0);
                % \draw[thick, blue, dashed] (4,0) -- (6,0);

                \draw[thick, blue, ->, >=stealth] (0,0) -- (2,0);
                \draw[
                    thick, dashed, blue,
                    decoration={markings, mark=at position 0.5 with {\arrow{stealth}}},
                    postaction={decorate}
                ]
                (2,0) arc (180:0:2 and 1.5);
                \draw[
                    thick, dashed, blue,
                    decoration={markings, mark=at position 0.5 with {\arrow{stealth reversed}}},
                    postaction={decorate}
                ] 
                (2,0) -- (6,0);
                \node[label={[blue, xshift=100pt, yshift=23pt]left:{$\beta$}}] at (0,0) {};
            \end{tikzpicture} 
        \end{figure}
        \noindent
        Ya que como podemos ver, $\beta$ no puede ser homotópico a $[\varepsilon_{(0,0,0)}]$ y vemos que de $Im \beta$ no podemos extraer otro lazo ``que dé menos vueltas que $\beta$'', por lo que la clase de $\beta$ debe corresponderse mediante el isomorfismo $\pi_1(U)\cong \mathbb{Z}$ bien con $1$ o con $-1$, ambos generadores de $\mathbb{Z}$, luego:
        \begin{equation*}
            \pi_1(U,(0,0,0)) = \langle [\beta]\big|_U \rangle 
        \end{equation*}
        El caso de $V$ es análogo, sin más que considerar el siguiente lazo $\alpha\in \Omega(V,(0,0,0))$:
        \begin{figure}[H]
            \centering
            \begin{tikzpicture}[scale=0.8]
                \shorthandoff{>}
                
                % Parte izquierda
                
                \node[label={[red, xshift=-81pt, yshift=25pt]left:{$\alpha$}}] at (0,0) {};
                
                \draw[thick] (4,0) circle [radius=2];
                \draw[thick, dashed] (6,0) arc (0:180:2 and 0.5);
                \draw[thick] (6,0) arc (0:-180:2 and 0.5);

                % Parte izquierda

                \draw[thick] (-4,0) circle [radius=2];
                \draw[thick, dashed] (-2,0) arc (0:180:2 and 0.5);
                \draw[thick] (-2,0) arc (0:-180:2 and 0.5);

                \draw[
                    thick, dashed, red,
                    decoration={markings, mark=at position 0.5 with {\arrow{stealth}}},
                    postaction={decorate}
                ] 
                (-6,0) -- (-2,0);

                \draw[thick, red, <-, >=stealth] (-2,0) -- (0,0);

                \draw[
                    thick, dashed, red,
                    decoration={markings, mark=at position 0.5 with {\arrow{stealth reversed}}},
                    postaction={decorate}
                ]
                (-6,0) arc (180:0:2 and 1.5);

                \draw[thick] (0,0) -- (2,0);
                \draw[thick, dashed] (2,0) -- (4,0);
            \end{tikzpicture} 
        \end{figure}
        \noindent
        Obteniendo así:
        \begin{equation*}
            \pi_1(V,(0,0,0)) = \langle [\alpha]\big|_V \rangle 
        \end{equation*}

        \begin{figure}[H]
            \centering
            \begin{tikzpicture}[scale=0.8]
                \shorthandoff{>}
                
                % Parte izquierda
                
                \node[label={[red, xshift=-81pt, yshift=25pt]left:{$\alpha$}}] at (0,0) {};
                
                \draw[
                    thick, dashed, blue,
                    decoration={markings, mark=at position 0.5 with {\arrow{stealth reversed}}},
                    postaction={decorate}
                ] 
                (2,0) -- (6,0);
                
                \draw[thick] (4,0) circle [radius=2];
                \draw[thick, dashed] (6,0) arc (0:180:2 and 0.5);
                \draw[thick] (6,0) arc (0:-180:2 and 0.5);
                    
                % Origen

                \node[draw, circle, fill=yellow, yellow, inner sep=0.8pt, label={[xshift = 24pt, yshift = 8pt]left:{$(0,0,0)$}}] at (0,0) {};
                
                % Parte izquierda

                \draw[thick] (-4,0) circle [radius=2];
                \draw[thick, dashed] (-2,0) arc (0:180:2 and 0.5);
                \draw[thick] (-2,0) arc (0:-180:2 and 0.5);

                \draw[
                    thick, dashed, red,
                    decoration={markings, mark=at position 0.5 with {\arrow{stealth}}},
                    postaction={decorate}
                ] 
                (-6,0) -- (-2,0);

                \draw[thick, red, <-, >=stealth] (-2,0) -- (0,0);

                \draw[
                    thick, dashed, red,
                    decoration={markings, mark=at position 0.5 with {\arrow{stealth reversed}}},
                    postaction={decorate}
                ]
                (-6,0) arc (180:0:2 and 1.5);

                \node[label={[blue, xshift=100pt, yshift=23pt]left:{$\beta$}}] at (0,0) {};
                

                \draw[thick, blue, ->, >=stealth] (0,0) -- (2,0);
                \draw[
                    thick, dashed, blue,
                    decoration={markings, mark=at position 0.5 with {\arrow{stealth}}},
                    postaction={decorate}
                ]
                (2,0) arc (180:0:2 and 1.5);

            \end{tikzpicture} 
        \end{figure}
        \noindent
        Por el Teorema de Seifert-van Kampen, tendremos entonces que:
        \begin{align*}
            \pi_1(X,(0,0,0)) &\cong \pi_1(Y,(0,0,0)) \cong \pi_1(U,(0,0,0))\ast \pi_1(V,(0,0,0)) \\
                             &= \left\langle [\beta]\big|_U \right\rangle  \ast \left\langle [\alpha]\big|_V \right\rangle  = \left\langle [\beta]\big|_U, [\alpha]\big|_V \right\rangle 
        \end{align*}
    \end{ejercicio}
\end{document}
