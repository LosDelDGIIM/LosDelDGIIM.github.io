\documentclass[12pt]{article}

% Idioma y codificación
\usepackage[spanish, es-tabla]{babel}       %es-tabla para que se titule "Tabla"
\usepackage[utf8]{inputenc}

% Márgenes
\usepackage[a4paper,top=3cm,bottom=2.5cm,left=3cm,right=3cm]{geometry}

% Comentarios de bloque
\usepackage{verbatim}

% Paquetes de links
\usepackage[hidelinks]{hyperref}    % Permite enlaces
\usepackage{url}                    % redirecciona a la web

% Más opciones para enumeraciones
\usepackage{enumitem}

% Personalizar la portada
\usepackage{titling}

% Paquetes de tablas
\usepackage{multirow}


%------------------------------------------------------------------------

%Paquetes de figuras
\usepackage{caption}
\usepackage{subcaption} % Figuras al lado de otras
\usepackage{float}      % Poner figuras en el sitio indicado H.


% Paquetes de imágenes
\usepackage{graphicx}       % Paquete para añadir imágenes
\usepackage{transparent}    % Para manejar la opacidad de las figuras

% Paquete para usar colores
\usepackage[dvipsnames]{xcolor}
\usepackage{pagecolor}      % Para cambiar el color de la página

% Habilita tamaños de fuente mayores
\usepackage{fix-cm}

% Para los gráficos
\usepackage{tikz}

% Para poder situar los nodos en los grafos
\usetikzlibrary{positioning}


%------------------------------------------------------------------------

% Paquetes de matemáticas
\usepackage{mathtools, amsfonts, amssymb, mathrsfs}
\usepackage[makeroom]{cancel}     % Simplificar tachando
\usepackage{polynom}    % Divisiones y Ruffini
\usepackage{units} % Para poner fracciones diagonales con \nicefrac

\usepackage{pgfplots}   %Representar funciones
\pgfplotsset{compat=1.18}  % Versión 1.18

\usepackage{tikz-cd}    % Para usar diagramas de composiciones
\usetikzlibrary{calc}   % Para usar cálculo de coordenadas en tikz

%Definición de teoremas, etc.
\usepackage{amsthm}
%\swapnumbers   % Intercambia la posición del texto y de la numeración

\theoremstyle{plain}

\makeatletter
\@ifclassloaded{article}{
  \newtheorem{teo}{Teorema}[section]
}{
  \newtheorem{teo}{Teorema}[chapter]  % Se resetea en cada chapter
}
\makeatother

\newtheorem{coro}{Corolario}[teo]           % Se resetea en cada teorema
\newtheorem{prop}[teo]{Proposición}         % Usa el mismo contador que teorema
\newtheorem{lema}[teo]{Lema}                % Usa el mismo contador que teorema

\theoremstyle{remark}
\newtheorem*{observacion}{Observación}

\theoremstyle{definition}

\makeatletter
\@ifclassloaded{article}{
  \newtheorem{definicion}{Definición} [section]     % Se resetea en cada chapter
}{
  \newtheorem{definicion}{Definición} [chapter]     % Se resetea en cada chapter
}
\makeatother

\newtheorem*{notacion}{Notación}
\newtheorem*{ejemplo}{Ejemplo}
\newtheorem*{ejercicio*}{Ejercicio}             % No numerado
\newtheorem{ejercicio}{Ejercicio} [section]     % Se resetea en cada section


% Modificar el formato de la numeración del teorema "ejercicio"
\renewcommand{\theejercicio}{%
  \ifnum\value{section}=0 % Si no se ha iniciado ninguna sección
    \arabic{ejercicio}% Solo mostrar el número de ejercicio
  \else
    \thesection.\arabic{ejercicio}% Mostrar número de sección y número de ejercicio
  \fi
}


% \renewcommand\qedsymbol{$\blacksquare$}         % Cambiar símbolo QED
%------------------------------------------------------------------------

% Paquetes para encabezados
\usepackage{fancyhdr}
\pagestyle{fancy}
\fancyhf{}

\newcommand{\helv}{ % Modificación tamaño de letra
\fontfamily{}\fontsize{12}{12}\selectfont}
\setlength{\headheight}{15pt} % Amplía el tamaño del índice


%\usepackage{lastpage}   % Referenciar última pag   \pageref{LastPage}
\fancyfoot[C]{\thepage}

%------------------------------------------------------------------------

% Conseguir que no ponga "Capítulo 1". Sino solo "1."
\makeatletter
\@ifclassloaded{book}{
  \renewcommand{\chaptermark}[1]{\markboth{\thechapter.\ #1}{}} % En el encabezado
    
  \renewcommand{\@makechapterhead}[1]{%
  \vspace*{50\p@}%
  {\parindent \z@ \raggedright \normalfont
    \ifnum \c@secnumdepth >\m@ne
      \huge\bfseries \thechapter.\hspace{1em}\ignorespaces
    \fi
    \interlinepenalty\@M
    \Huge \bfseries #1\par\nobreak
    \vskip 40\p@
  }}
}
\makeatother

%------------------------------------------------------------------------
% Paquetes de cógido
\usepackage{minted}
\renewcommand\listingscaption{Código fuente}

\usepackage{fancyvrb}
% Personaliza el tamaño de los números de línea
\renewcommand{\theFancyVerbLine}{\small\arabic{FancyVerbLine}}

% Estilo para C++
\newminted{cpp}{
    frame=lines,
    framesep=2mm,
    baselinestretch=1.2,
    linenos,
    escapeinside=||
}

% para minted
\definecolor{LightGray}{rgb}{0.95,0.95,0.92}
\setminted{
    linenos=true,
    stepnumber=5,
    numberfirstline=true,
    autogobble,
    breaklines=true,
    breakautoindent=true,
    breaksymbolleft=,
    breaksymbolright=,
    breaksymbolindentleft=0pt,
    breaksymbolindentright=0pt,
    breaksymbolsepleft=0pt,
    breaksymbolsepright=0pt,
    fontsize=\footnotesize,
    bgcolor=LightGray,
    numbersep=10pt
}


\usepackage{listings} % Para incluir código desde un archivo

\renewcommand\lstlistingname{Código Fuente}
\renewcommand\lstlistlistingname{Índice de Códigos Fuente}

% Definir colores
\definecolor{vscodepurple}{rgb}{0.5,0,0.5}
\definecolor{vscodeblue}{rgb}{0,0,0.8}
\definecolor{vscodegreen}{rgb}{0,0.5,0}
\definecolor{vscodegray}{rgb}{0.5,0.5,0.5}
\definecolor{vscodebackground}{rgb}{0.97,0.97,0.97}
\definecolor{vscodelightgray}{rgb}{0.9,0.9,0.9}

% Configuración para el estilo de C similar a VSCode
\lstdefinestyle{vscode_C}{
  backgroundcolor=\color{vscodebackground},
  commentstyle=\color{vscodegreen},
  keywordstyle=\color{vscodeblue},
  numberstyle=\tiny\color{vscodegray},
  stringstyle=\color{vscodepurple},
  basicstyle=\scriptsize\ttfamily,
  breakatwhitespace=false,
  breaklines=true,
  captionpos=b,
  keepspaces=true,
  numbers=left,
  numbersep=5pt,
  showspaces=false,
  showstringspaces=false,
  showtabs=false,
  tabsize=2,
  frame=tb,
  framerule=0pt,
  aboveskip=10pt,
  belowskip=10pt,
  xleftmargin=10pt,
  xrightmargin=10pt,
  framexleftmargin=10pt,
  framexrightmargin=10pt,
  framesep=0pt,
  rulecolor=\color{vscodelightgray},
  backgroundcolor=\color{vscodebackground},
}

%------------------------------------------------------------------------

% Comandos definidos
\newcommand{\bb}[1]{\mathbb{#1}}
\newcommand{\cc}[1]{\mathcal{#1}}

% I prefer the slanted \leq
\let\oldleq\leq % save them in case they're every wanted
\let\oldgeq\geq
\renewcommand{\leq}{\leqslant}
\renewcommand{\geq}{\geqslant}

% Si y solo si
\newcommand{\sii}{\iff}

% Letras griegas
\newcommand{\eps}{\epsilon}
\newcommand{\veps}{\varepsilon}
\newcommand{\lm}{\lambda}

\newcommand{\ol}{\overline}
\newcommand{\ul}{\underline}
\newcommand{\wt}{\widetilde}
\newcommand{\wh}{\widehat}

\let\oldvec\vec
\renewcommand{\vec}{\overrightarrow}

% Derivadas parciales
\newcommand{\del}[2]{\frac{\partial #1}{\partial #2}}
\newcommand{\Del}[3]{\frac{\partial^{#1} #2}{\partial #3^{#1}}}
\newcommand{\deld}[2]{\dfrac{\partial #1}{\partial #2}}
\newcommand{\Deld}[3]{\dfrac{\partial^{#1} #2}{\partial #3^{#1}}}


\newcommand{\AstIg}{\stackrel{(\ast)}{=}}
\newcommand{\Hop}{\stackrel{L'H\hat{o}pital}{=}}

\newcommand{\red}[1]{{\color{red}#1}} % Para integrales, destacar los cambios.

% Método de integración
\newcommand{\MetInt}[2]{
    \left[\begin{array}{c}
        #1 \\ #2
    \end{array}\right]
}

% Declarar aplicaciones
% 1. Nombre aplicación
% 2. Dominio
% 3. Codominio
% 4. Variable
% 5. Imagen de la variable
\newcommand{\Func}[5]{
    \begin{equation*}
        \begin{array}{rrll}
            #1:& #2 & \longrightarrow & #3\\
               & #4 & \longmapsto & #5
        \end{array}
    \end{equation*}
}

%------------------------------------------------------------------------


\usetikzlibrary{arrows.meta, decorations.markings} % Cargar las bibliotecas necesarias
% Configuración para las flechas
\tikzset{
    arrow at 1/3/.style={postaction={decorate},
        decoration={markings, mark=at position 0.33 with {\arrow{Stealth}}}},
    arrow at 2/3/.style={postaction={decorate},
        decoration={markings, mark=at position 0.66 with {\arrow{Stealth}}}}
}

\begin{document}

    % 1. Foto de fondo
    % 2. Título
    % 3. Encabezado Izquierdo
    % 4. Color de fondo
    % 5. Coord x del titulo
    % 6. Coord y del titulo
    % 7. Fecha

    
    % 1. Foto de fondo
% 2. Título
% 3. Encabezado Izquierdo
% 4. Color de fondo
% 5. Coord x del titulo
% 6. Coord y del titulo
% 7. Fecha

\newcommand{\portada}[7]{

    \portadaBase{#1}{#2}{#3}{#4}{#5}{#6}{#7}
    \portadaBook{#1}{#2}{#3}{#4}{#5}{#6}{#7}
}

\newcommand{\portadaExamen}[7]{

    \portadaBase{#1}{#2}{#3}{#4}{#5}{#6}{#7}
    \portadaArticle{#1}{#2}{#3}{#4}{#5}{#6}{#7}
}




\newcommand{\portadaBase}[7]{

    % Tiene la portada principal y la licencia Creative Commons
    
    % 1. Foto de fondo
    % 2. Título
    % 3. Encabezado Izquierdo
    % 4. Color de fondo
    % 5. Coord x del titulo
    % 6. Coord y del titulo
    % 7. Fecha
    
    
    \thispagestyle{empty}               % Sin encabezado ni pie de página
    \newgeometry{margin=0cm}        % Márgenes nulos para la primera página
    
    
    % Encabezado
    \fancyhead[L]{\helv #3}
    \fancyhead[R]{\helv \nouppercase{\leftmark}}
    
    
    \pagecolor{#4}        % Color de fondo para la portada
    
    \begin{figure}[p]
        \centering
        \transparent{0.3}           % Opacidad del 30% para la imagen
        
        \includegraphics[width=\paperwidth, keepaspectratio]{assets/#1}
    
        \begin{tikzpicture}[remember picture, overlay]
            \node[anchor=north west, text=white, opacity=1, font=\fontsize{60}{90}\selectfont\bfseries\sffamily, align=left] at (#5, #6) {#2};
            
            \node[anchor=south east, text=white, opacity=1, font=\fontsize{12}{18}\selectfont\sffamily, align=right] at (9.7, 3) {\textbf{\href{https://losdeldgiim.github.io/}{Los Del DGIIM}}};
            
            \node[anchor=south east, text=white, opacity=1, font=\fontsize{12}{15}\selectfont\sffamily, align=right] at (9.7, 1.8) {Doble Grado en Ingeniería Informática y Matemáticas\\Universidad de Granada};
        \end{tikzpicture}
    \end{figure}
    
    
    \restoregeometry        % Restaurar márgenes normales para las páginas subsiguientes
    \pagecolor{white}       % Restaurar el color de página
    
    
    \newpage
    \thispagestyle{empty}               % Sin encabezado ni pie de página
    \begin{tikzpicture}[remember picture, overlay]
        \node[anchor=south west, inner sep=3cm] at (current page.south west) {
            \begin{minipage}{0.5\paperwidth}
                \href{https://creativecommons.org/licenses/by-nc-nd/4.0/}{
                    \includegraphics[height=2cm]{assets/Licencia.png}
                }\vspace{1cm}\\
                Esta obra está bajo una
                \href{https://creativecommons.org/licenses/by-nc-nd/4.0/}{
                    Licencia Creative Commons Atribución-NoComercial-SinDerivadas 4.0 Internacional (CC BY-NC-ND 4.0).
                }\\
    
                Eres libre de compartir y redistribuir el contenido de esta obra en cualquier medio o formato, siempre y cuando des el crédito adecuado a los autores originales y no persigas fines comerciales. 
            \end{minipage}
        };
    \end{tikzpicture}
    
    
    
    % 1. Foto de fondo
    % 2. Título
    % 3. Encabezado Izquierdo
    % 4. Color de fondo
    % 5. Coord x del titulo
    % 6. Coord y del titulo
    % 7. Fecha


}


\newcommand{\portadaBook}[7]{

    % 1. Foto de fondo
    % 2. Título
    % 3. Encabezado Izquierdo
    % 4. Color de fondo
    % 5. Coord x del titulo
    % 6. Coord y del titulo
    % 7. Fecha

    % Personaliza el formato del título
    \pretitle{\begin{center}\bfseries\fontsize{42}{56}\selectfont}
    \posttitle{\par\end{center}\vspace{2em}}
    
    % Personaliza el formato del autor
    \preauthor{\begin{center}\Large}
    \postauthor{\par\end{center}\vfill}
    
    % Personaliza el formato de la fecha
    \predate{\begin{center}\huge}
    \postdate{\par\end{center}\vspace{2em}}
    
    \title{#2}
    \author{\href{https://losdeldgiim.github.io/}{Los Del DGIIM}}
    \date{Granada, #7}
    \maketitle
    
    \tableofcontents
}




\newcommand{\portadaArticle}[7]{

    % 1. Foto de fondo
    % 2. Título
    % 3. Encabezado Izquierdo
    % 4. Color de fondo
    % 5. Coord x del titulo
    % 6. Coord y del titulo
    % 7. Fecha

    % Personaliza el formato del título
    \pretitle{\begin{center}\bfseries\fontsize{42}{56}\selectfont}
    \posttitle{\par\end{center}\vspace{2em}}
    
    % Personaliza el formato del autor
    \preauthor{\begin{center}\Large}
    \postauthor{\par\end{center}\vspace{3em}}
    
    % Personaliza el formato de la fecha
    \predate{\begin{center}\huge}
    \postdate{\par\end{center}\vspace{5em}}
    
    \title{#2}
    \author{\href{https://losdeldgiim.github.io/}{Los Del DGIIM}}
    \date{Granada, #7}
    \thispagestyle{empty}               % Sin encabezado ni pie de página
    \maketitle
    \vfill
}
    \portadaExamen{ffccA4.jpg}{Topología II\\Examen VI}{Topología II. Examen VI}{MidnightBlue}{-8}{28}{2025}{}

    \begin{description}
        \item[Asignatura] Topología II.
        \item[Curso Académico] 2024/25.
        \item[Grado] Doble Grado en Ingeniería Informática y Matemáticas.
        \item[Grupo] Grupo Único.
        % \item[Profesor] ---.
        \item[Descripción] Examen extraordinario.
        \item[Fecha] 10 de febrero de 2025.
        \item[Duración] 2 horas y media.
    \end{description}
    \newpage


    % ------------------------------------
    
    \noindent
    \textbf{Responda la pregunta 1 y elija tres preguntas entre la 2, 3, 4 y 5.}

    \begin{ejercicio}
        Razona si son verdaderas o falsas las siguientes afirmaciones:
        \begin{enumerate}[label=\alph*)]
            \item Dadas las circunferencias
                \begin{equation*}
                    C_1 = \{(x,y)\in \mathbb{R}^2 : {(x-1)}^{2}+y^2 = 1\}, \qquad C_2 = \{(x,y)\in \mathbb{R}^2 : {(x+1)}^{2}+y^2 = 1\}
                \end{equation*}
                no existe una aplicación recubridora desde $X=C_1\cup C_2$ en $\bb{S}^1$.
            \item Si $p:X\to Y$ es una aplicación recubridora con $X$ compacto, entonces $p^{-1}(y)$ es finito para todo $y\in Y$.
            \item Toda aplicación continua $f:\bb{D}\to \mathbb{R}^2\setminus \{(0,0)\}$ es homotópicamente nula.
        \end{enumerate}
    \end{ejercicio}

    \begin{ejercicio}
        Consideremos el cilindro vertical $C=\bb{S}^1\times \mathbb{R}\subset \mathbb{R}^3$ y las rectas horizonatales
        \begin{equation*}
            R_1 = \{(x,y,z)\in \mathbb{R}^3 : y=0, z=-1\}, \qquad R_2 = \{(x,y,z)\in \mathbb{R}^3 : y= 0, z=1\}
        \end{equation*}
        Calcula el grupo fundamental de
        \begin{equation*}
            X = C\cup R_1\cup R_2
        \end{equation*}
    \end{ejercicio}

    \begin{ejercicio}
        Sea $f:\overline{\bb{D}}\to \overline{\bb{D}}$ una aplicación continua cumpliendo que $f\big|_{Fr(\overline{\bb{D}})}:Fr(\overline{\bb{D}})\to Fr(\overline{\bb{D}})$ es un homeomorfismo. Demuestra que $f$ es sobreyectiva.
    \end{ejercicio}

    \begin{ejercicio}
        Sea $f:\bb{S}^n\to \bb{S}^1$ una aplicación continua. Demuestra que $f$ es homotópicamente nula si $n\geq 2$.
    \end{ejercicio}

    \begin{ejercicio}
        Determina la superficie compacta $S$ dada por la presentación poligonal
        \begin{equation*}
            \langle a,b,c,d,e,f,g,h,i;afc^{-1}, deb^{-1}, hia^{-1}, dbc, ehg^{-1}, gfi^{-1} \rangle 
        \end{equation*}
    \end{ejercicio}

    \newpage
    \setcounter{ejercicio}{0} % Reiniciar contador de ejercicios
    \noindent
    \textbf{Solución.}

    \begin{ejercicio}
        Razona si son verdaderas o falsas las siguientes afirmaciones:
        \begin{enumerate}[label=\alph*)]
            \item Dadas las circunferencias
                \begin{equation*}
                    C_1 = \{(x,y)\in \mathbb{R}^2 : {(x-1)}^{2}+y^2 = 1\}, \qquad C_2 = \{(x,y)\in \mathbb{R}^2 : {(x+1)}^{2}+y^2 = 1\}
                \end{equation*}
                no existe una aplicación recubridora desde $X=C_1\cup C_2$ en $\bb{S}^1$.\\

                \noindent
                Es falsa: Por reducción al absurdo, supongamos que existe una aplicación recubridora $p:X\to \mathbb{S}^1$. Tenemos entonces por el Teorema de Monodromía que el homomorfismo (como trabajamos con espacios arcoconexos da igual el punto en el que trabajemos) $p_\ast:\pi_1(X)\to \pi_1(\mathbb{S}^1)$ es inyectivo, con $\pi_1(X)\cong \mathbb{Z}\ast\mathbb{Z}$ y $\pi_1(\mathbb{S}^1)\cong \mathbb{Z}$. Tendremos por tanto que $\mathbb{Z}$ contiene un subgrupo isomorfo a $\mathbb{Z}\ast\mathbb{Z}$, pero esto es una contradicción, ya que todo subgrupo de $\mathbb{Z}$ es abeliano por ser $\mathbb{Z}$ abeliano y $\mathbb{Z}\ast\mathbb{Z}$ no es abeliano.

            \item Si $p:X\to Y$ es una aplicación recubridora con $X$ compacto, entonces $p^{-1}(y)$ es finito para todo $y\in Y$.\\

                \begin{description}
                    \item [Demostración directa.]~\\
                Es cierta: por reducción al absurdo, supongamos que existe $y\in Y$ tal que el conjunto $S = p^{-1}(y)$ un conjunto infinito. En dicho caso, por ser $X$ compacto ha de existir $x\in X$ punto de acumulación de $S$. Sea $z=p(x)$, como $p$ es una aplicación recubridora existe $U$ un entorno abierto de $z$ y $V$ un entorno abierto de $x$ de forma que $p:V\to U$ es un homeomorfismo.

                Usaremos el siguiente resultado:

                \begin{center}
                    Sea $X$ un espacio topológico compacto y $S\subseteq X$ un subconjunto infinito, existe un punto $x\in X$ adherente a $S$ tal que $N\cap S$ es un conjunto infinito, para todo $N$ entorno de $x$.
                \end{center}
                \begin{proof}
                    Por reducción al absurdo, supongamos que para todo punto $x\in X$ adherente a $S$ existe $U_x$ entorno abierto de $x$ tal que $U_x\cap S$ es finito. Observemos que podemos escribir:
                    \begin{equation*}
                        \overline{S} \subseteq  \bigcup_{x\in \overline{S}}U_x
                    \end{equation*}
                    Y como $\overline{S}$ es compacto por ser cerrado y subconjunto de un compacto tenemos que existen $x_1,\ldots,x_n\in \overline{S}$ de forma que:
                    \begin{equation*}
                        \overline{S} \subseteq  U_{x_1} \cup \ldots \cup U_{x_n}
                    \end{equation*}
                    Esto nos permite escribir:
                    \begin{equation*}
                        S = (U_{x_1}\cap S) \cup \ldots \cup (U_{x_n}\cap S)
                    \end{equation*}
                    con cada $U_{x_i}\cap S$ finito para cada $i \in \{1,\ldots,n\}$. Sin embargo, tenemos que $S$ era un conjunto infinito, por lo que hemos llegado a contradicción.
                \end{proof}
                Tenemos entonces que $V\cap S$ es un conjunto infinito, con lo que $p$ no es inyectiva en $V$, lo que contradice que $p:V\to U$ sea un homeomorfismo.

                    \item [Asumiendo una hipótesis adicional.]~\\
                Si $Y$ es T1 es cierta, pues para todo $y\in Y$ tendremos que $\{y\}$ es cerrado por ser $Y$ T1, por lo que $p^{-1}(\{y\})\subseteq X$ es un conjunto cerrado. Como $X$ es compacto, tenemos que $p^{-1}(\{y\})$ también será compacto. Como $p$ es una aplicación recubridora, para $y$ existirá $O_y$ un abierto de $Y$ regularmente recubierto, por lo que:
                \begin{equation*}
                    p^{-1}(O_y) = \biguplus_{i \in I} A_i
                \end{equation*}
                con $A_i$ abierto y $p\big|_{A_i}:A_i\to O_y$ homeomorfismo, para cada $i \in I$. Tenemos un recubrimiento por abiertos de $p^{-1}(\{y\})\subseteq p^{-1}(O_y)$, por lo que por ser $p^{-1}(\{y\})$ compacto existirá $J\subseteq I$ finito de forma que:
                \begin{equation*}
                    p^{-1}(\{y\}) \subseteq \biguplus_{i \in J}A_i
                \end{equation*}
                Como los conjuntos $A_i$ son disjuntos tenemos entonces que solo puede haber una cantidad finita de preimágenes de $y$.
                \end{description}
            \item Toda aplicación continua $f:\bb{D}\to \mathbb{R}^2\setminus \{(0,0)\}$ es homotópicamente nula.\\

                Verdadera: Si consideramos el conjunto:
                \begin{equation*}
                    Y = \mathbb{R} \times \left]0,1\right[
                \end{equation*}
                \begin{figure}[H]
                    \centering
                    \begin{tikzpicture}
                        \filldraw[fill=blue!40, draw=blue, opacity=0.3] (0,0) rectangle (3,2);
                        \draw[thick] (0,0) -- (3,0);
                        \fill(1.5,1) node[] {$Y$};
                    \end{tikzpicture}
                \end{figure}
                \noindent
                Tenemos que $Y$ recubre a $X=\mathbb{R}^2\setminus\{(0,0)\}$, ya que para cada circunferencia de centro $(0,0)$  de $X$ podemos considerar la aplicación recubridora estándar desde $\mathbb{R}$ cuyo dominio se encuentra a cierta altura en el conjunto $Y$. Podemos repetir esta idea haciendo corresponder a cada altura del conjunto $Y$ un radio en el conjunto $X$, de forma que los radios tiendan a $0$ cuando la altura tienda a $0$ y que los radios diverjan cuando la altura se acerque a $1$.
                \begin{figure}[H]
                    \centering
                    \begin{tikzpicture}
                        \filldraw[fill=blue!40, draw=blue, opacity=0.3] (0,0) rectangle (3,2);
                        \draw[thick] (0,0) -- (3,0);
                        \fill(1.5,1) node[] {$Y$};
                        % \draw[thick](6,1) circle(1cm);
                        \draw[fill = red!40, draw=red, opacity=0.3](6,1) circle(1cm);
                        \fill(5.5,1) node[] {$X$};
                        \draw[fill=white] (6,1) circle (2pt);

                        \draw[thick, draw=green] (0,0.3) -- (3,0.3);
                        \draw[thick, draw=purple] (6,1) circle(0.8cm);
                        \draw[thick, draw=purple] (0,1.7) -- (3,1.7);
                        \draw[thick, draw=green] (6,1) circle(0.3cm);
                    \end{tikzpicture}
                \end{figure}
                \noindent
                Usando esta idea, si consideramos la aplicación $p:Y\to X$ dada por:
                \begin{equation*}
                    p(x,y) = tg\left(\frac{\pi}{2}y\right) p_0(x)
                \end{equation*}
                donde $p_0:\mathbb{R}\to\mathbb{S}^1$ es la aplicación recubridora estándar, es sencillo probar que $p$ es una aplicación recubridora. Así, estamos en la situación:
                \begin{figure}[H]
                    \centering
                    \shorthandoff{""}
                    \begin{tikzcd}
                                                  & Y \arrow[d, "p"]                 \\
                        \mathbb{D} \arrow[r, "f"] & {\mathbb{R}^2\setminus\{(0,0)\}}
                    \end{tikzcd}
                    \shorthandon{""}
                \end{figure}
                \noindent
                Si observamos ahora que $\bb{D}$ es simplemente conexo tenemos entonces que $f$ se puede levantar, con lo que fijado $x_0\in \bb{D}$ y tomando $r_0\in p^{-1}(\{f(x_0)\})$ tenemos que existe un levantamiento $\hat{f}:\bb{D}\to Y$. Como tenemos ahora que $Y$ es contráctil tenemos que $\hat{f}$ es homotópicamente nula, y razonando como en el Ejercicio~\ref{ej:4} llegamos a que $f$ es homotópicamente nula.
        \end{enumerate}
    \end{ejercicio}

    \begin{ejercicio}
        Consideremos el cilindro vertical $C=\bb{S}^1\times \mathbb{R}\subset \mathbb{R}^3$ y las rectas horizonatales
        \begin{equation*}
            R_1 = \{(x,y,z)\in \mathbb{R}^3 : y=0, z=-1\}, \qquad R_2 = \{(x,y,z)\in \mathbb{R}^3 : y= 0, z=1\}
        \end{equation*}
        Calcula el grupo fundamental de
        \begin{equation*}
            X = C\cup R_1\cup R_2
        \end{equation*}

        \noindent
        Tenemos el conjunto:
        \begin{figure}[H]
            \centering
            \begin{tikzpicture}[scale=0.8]
                % Cilindro
                \draw[thick] (2,4) arc (0:180:2 and 0.5);
                \draw[thick] (2,4) arc (0:-180:2 and 0.5);
                \draw[thick, dashed] (2,-1) arc (0:180:2 and 0.5);
                \draw[thick] (2,-1) arc (0:-180:2 and 0.5);
                \draw[thick] (2,4) -- (2,-1);
                \draw[thick] (-2,4) -- (-2,-1);

                % rectas
                \draw[thick] (-4,2.5) -- (-2,2.5);
                \draw[thick, dashed] (-2,2.5) -- (2,2.5);
                \draw[thick] (2,2.5) -- (4,2.5);

                \draw[thick] (-4,0.5) -- (-2,0.5);
                \draw[thick, dashed] (-2,0.5) -- (2,0.5);
                \draw[thick] (2,0.5) -- (4,0.5);
            \end{tikzpicture}
        \end{figure}
        \noindent
        Que tiene por retracto de deformación el conjunto:
        \begin{figure}[H]
            \centering
            \begin{tikzpicture}[scale=0.8]
                % Cilindro
                \draw[thick] (2,1) arc (0:180:2 and 0.5);
                \draw[thick] (2,1) arc (0:-180:2 and 0.5);
                \draw[thick, dashed] (2,-1) arc (0:180:2 and 0.5);
                \draw[thick] (2,-1) arc (0:-180:2 and 0.5);
                \draw[thick] (2,1) -- (2,-1);
                \draw[thick] (-2,1) -- (-2,-1);

                % rectas
                \draw[thick] (-2,1) -- (2,1);
                \draw[thick, dashed] (-2,-1) -- (2,-1);
            \end{tikzpicture}
        \end{figure}
        \noindent
        Si tomamos como $U$:
        \begin{figure}[H]
            \centering
            \begin{tikzpicture}[scale=0.8]
                % Cilindro
                \draw[thick] (2,1) arc (0:180:2 and 0.5);
                \draw[thick] (2,1) arc (0:-180:2 and 0.5);
                \draw[thick, dashed] (2,-1) arc (0:180:2 and 0.5);
                \draw[thick] (2,-1) arc (0:-180:2 and 0.5);
                \draw[thick] (2,1) -- (2,-1);
                \draw[thick] (-2,1) -- (-2,-1);

                % rectas
                \draw[thick] (-2,1) -- (2,1);
                \draw[thick, dashed] (-2,-1) -- (2,-1);

                \fill[red] (0,1) circle(2pt) node[] {};
                \draw[thick, red] (0.8,0.55) -- (0.8,-1.45);
            \end{tikzpicture}
        \end{figure}
        \noindent
        Y como $V$:
        \begin{figure}[H]
            \centering
            \begin{tikzpicture}[scale=0.8]
                % Cilindro
                \draw[thick] (2,1) arc (0:180:2 and 0.5);
                \draw[thick] (2,1) arc (0:-180:2 and 0.5);
                \draw[thick, dashed] (2,-1) arc (0:180:2 and 0.5);
                \draw[thick] (2,-1) arc (0:-180:2 and 0.5);
                \draw[thick] (2,1) -- (2,-1);
                \draw[thick] (-2,1) -- (-2,-1);

                % rectas
                \draw[thick] (-2,1) -- (2,1);
                \draw[thick, dashed] (-2,-1) -- (2,-1);

                \fill[blue] (0,-1) circle(2pt) node[] {};
            \end{tikzpicture}
        \end{figure}
        \noindent
        Puede probarse que $U\cap V$ es simplemente conexo, que $U$ tiene grupo fundamental $\mathbb{Z}$ y $V$ tiene $\mathbb{Z}\ast \mathbb{Z}$, de donde:
        \begin{equation*}
            \pi_1(X) \cong \mathbb{Z}\ast\mathbb{Z}\ast \mathbb{Z}
        \end{equation*}
    \end{ejercicio}

    \begin{ejercicio}
        Sea $f:\overline{\bb{D}}\to \overline{\bb{D}}$ una aplicación continua cumpliendo que $f\big|_{Fr(\overline{\bb{D}})}:Fr(\overline{\bb{D}})\to Fr(\overline{\bb{D}})$ es un homeomorfismo. Demuestra que $f$ es sobreyectiva.\\

        \noindent
        Por reducción al absurdo suponemos que $f$ no es sobreyectiva, es decir, que existe $y\in \overline{\bb{D}}\setminus f(\overline{\bb{D}})$ (notemos que $y\in \bb{D}$, ya que si no esto contradeciría la existencia del homeomorfismo enunciado). De esta forma, podemos considerar la restricción en codominio $f:\overline{\bb{D}}\to \overline{\bb{D}}\setminus \{y\}$. Ahora, el conjunto $\overline{\bb{D}}\setminus \{y\}$ tiene por retracto de deformación al conjunto $\mathbb{S}^1$. En definitiva, podemos considerar la siguiente cadena de composiciones de aplicaciones continuas:
        \begin{equation*}
            \overline{\bb{D}} \stackrel{f}{\longrightarrow} \overline{\bb{D}}\setminus \{y\}   \stackrel{r}{\longrightarrow}\mathbb{S}^1 \stackrel{f^{-1}}{\longrightarrow} \mathbb{S}^1
        \end{equation*}
        donde usamos que $Fr(\overline{\bb{D}}) = \mathbb{S}^1$, obteniendo así una aplicación continua de $\overline{\bb{D}}$ en $\mathbb{S}^1$ que mantiene fijos los puntos de $\mathbb{S}^1$, \underline{contradicción}.
    \end{ejercicio}

    \begin{ejercicio}\label{ej:4}
        Sea $f:\bb{S}^n\to \bb{S}^1$ una aplicación continua. Demuestra que $f$ es homotópicamente nula si $n\geq 2$.\\

        \noindent
        Si consideramos el recubridor $(\mathbb{R},p)$ de $\mathbb{S}^1$ con $p$ la aplicación recubridora estándar:
        \begin{figure}[H]
            \centering
            \shorthandoff{""}
            \begin{tikzcd}
                                                      & \mathbb{R} \arrow[d, "p"] \\
                \mathbb{S}^n \arrow[r, "f"] & \mathbb{S}^1             
            \end{tikzcd}
            \shorthandon{""}
        \end{figure}
        \noindent
        Como $\pi_1(\mathbb{S}^n,x_0) = \{[\varepsilon_{x_0}]\}$ para $n\geq 2$ tenemos entonces que:
        \begin{equation*}
            f_\ast(\pi_1(\mathbb{S}^n,x_0))\subseteq p_\ast(\pi_1(\mathbb{R},r_0))
        \end{equation*}
        para $r_0 \in p^{-1}(\{f(x_0)\})$, por lo que existe una aplicación $\hat{f}:\mathbb{S}^n\to \mathbb{R}$ levantamiento de $f$ con $\hat{f}(x_0) = r_0$. Como $\hat{f}$ llega a $\mathbb{R}$, que es contráctil, tenemos que $\hat{f}$ es homotópicamente nula, es decir, existe $z_0 \in \mathbb{R}$ y una homotopía $H:\mathbb{S}^n\times [0,1]\to \mathbb{R}$ de forma que:
                \begin{equation*}
                    H(x,0) = \hat{f}(x), \qquad H(x,1) = z_0 \qquad \forall x\in \mathbb{S}^n
                \end{equation*}
                Si consideramos ahora la homotopía $p\circ H:\mathbb{S}^n\times [0,1]\to \mathbb{S}^1$ tenemos que:
                \begin{equation*}
                    (p\circ H)(x,0) = p(\hat{f}(x)) = f(x), \qquad (p\circ H)(x,1)=p(z_0) \qquad \forall x\in \mathbb{S}^n
                \end{equation*}
                Por lo que $f$ es homotópica a la aplicación constantemente igual a $p(z_0)$, como queríamos probar.

    \end{ejercicio}

    \begin{ejercicio}
        Determina la superficie compacta $S$ dada por la presentación poligonal
        \begin{equation*}
            \langle a,b,c,d,e,f,g,h,i;afc^{-1}, deb^{-1}, hia^{-1}, dbc, ehg^{-1}, gfi^{-1} \rangle 
        \end{equation*}
        Calculamos la característica de Euler de $S$:
        \begin{itemize}
            \item $C = 6$.
            \item $A = 9$.
            \item $V = 2$.
        \end{itemize}
        Y para averiguar el número de vértices hemos visto que:
        \begin{figure}[H]
            \centering
            \begin{minipage}{0.30\textwidth}
                \centering
                 \begin{tikzpicture}[
                    flechaCCW/.style={postaction={decorate}, decoration={markings, mark=at position 0.7 with {\arrow{Latex[length=3mm]}}}, thick},
                    flechaCW/.style={postaction={decorate}, decoration={markings, mark=at position 0.7 with {\arrow{Latex[length=3mm, reversed]}}}, thick},
                    etiqueta/.style={midway, auto, swap, font=\small}
                ]
                \def\R{0.5} % Radio del hexágono

                \foreach \i in {0,...,2} {
                    \pgfmathsetmacro{\angle}{\i * 120}
                    \coordinate (P\i) at (\angle:\R);
                }

                % Aristas en sentido antihorario
                \draw[flechaCCW] (P0) -- node[etiqueta] {$a$} (P1);
                \draw[flechaCCW] (P1) -- node[etiqueta] {$f$} (P2);
                \draw[flechaCW] (P2) -- node[etiqueta] {$c^{-1}$} (P0); 

                % Colores de los vértices
				\filldraw[fill=yellow, draw=black] (P0) circle (2pt);
				\filldraw[fill=red, draw=black] (P1) circle (2pt);
				\filldraw[fill=red, draw=black] (P2) circle (2pt);
            \end{tikzpicture}
            \end{minipage}\hfill
            \begin{minipage}{0.30\textwidth}
                
                \centering
                 \begin{tikzpicture}[
                    flechaCCW/.style={postaction={decorate}, decoration={markings, mark=at position 0.7 with {\arrow{Latex[length=3mm]}}}, thick},
                    flechaCW/.style={postaction={decorate}, decoration={markings, mark=at position 0.7 with {\arrow{Latex[length=3mm, reversed]}}}, thick},
                    etiqueta/.style={midway, auto, swap, font=\small}
                ]
                \def\R{0.5} % Radio del hexágono

                \foreach \i in {0,...,2} {
                    \pgfmathsetmacro{\angle}{\i * 120}
                    \coordinate (P\i) at (\angle:\R);
                }

                % Aristas en sentido antihorario
                \draw[flechaCCW] (P0) -- node[etiqueta] {$d$} (P1);
                \draw[flechaCCW] (P1) -- node[etiqueta] {$e$} (P2);
                \draw[flechaCW] (P2) -- node[etiqueta] {$b^{-1}$} (P0); 

                % Colores de los vértices
				\filldraw[fill=red, draw=black] (P0) circle (2pt);
				\filldraw[fill=red, draw=black] (P1) circle (2pt);
				\filldraw[fill=yellow, draw=black] (P2) circle (2pt);
            \end{tikzpicture}
            \end{minipage}\hfill
            \begin{minipage}{0.30\textwidth}
                
                \centering
                 \begin{tikzpicture}[
                    flechaCCW/.style={postaction={decorate}, decoration={markings, mark=at position 0.7 with {\arrow{Latex[length=3mm]}}}, thick},
                    flechaCW/.style={postaction={decorate}, decoration={markings, mark=at position 0.7 with {\arrow{Latex[length=3mm, reversed]}}}, thick},
                    etiqueta/.style={midway, auto, swap, font=\small}
                ]
                \def\R{0.5} % Radio del hexágono

                \foreach \i in {0,...,2} {
                    \pgfmathsetmacro{\angle}{\i * 120}
                    \coordinate (P\i) at (\angle:\R);
                }

                % Aristas en sentido antihorario
                \draw[flechaCCW] (P0) -- node[etiqueta] {$h$} (P1);
                \draw[flechaCCW] (P1) -- node[etiqueta] {$i$} (P2);
                \draw[flechaCW] (P2) -- node[etiqueta] {$a^{-1}$} (P0); 

                % Colores de los vértices
				\filldraw[fill=yellow, draw=black] (P0) circle (2pt);
				\filldraw[fill=red, draw=black] (P1) circle (2pt);
				\filldraw[fill=red, draw=black] (P2) circle (2pt);
            \end{tikzpicture}
            \end{minipage}

            \vspace{1em}

            \begin{minipage}{0.30\textwidth}
                \centering
                 \begin{tikzpicture}[
                    flechaCCW/.style={postaction={decorate}, decoration={markings, mark=at position 0.7 with {\arrow{Latex[length=3mm]}}}, thick},
                    flechaCW/.style={postaction={decorate}, decoration={markings, mark=at position 0.7 with {\arrow{Latex[length=3mm, reversed]}}}, thick},
                    etiqueta/.style={midway, auto, swap, font=\small}
                ]
                \def\R{0.5} % Radio del hexágono

                \foreach \i in {0,...,2} {
                    \pgfmathsetmacro{\angle}{\i * 120}
                    \coordinate (P\i) at (\angle:\R);
                }

                % Aristas en sentido antihorario
                \draw[flechaCCW] (P0) -- node[etiqueta] {$d$} (P1);
                \draw[flechaCCW] (P1) -- node[etiqueta] {$b$} (P2);
                \draw[flechaCCW] (P2) -- node[etiqueta] {$c$} (P0); 

                % Colores de los vértices
				\filldraw[fill=red, draw=black] (P0) circle (2pt);
				\filldraw[fill=red, draw=black] (P1) circle (2pt);
				\filldraw[fill=yellow, draw=black] (P2) circle (2pt);
            \end{tikzpicture}
            \end{minipage}\hfill
            \begin{minipage}{0.30\textwidth}
                
                \centering
                 \begin{tikzpicture}[
                    flechaCCW/.style={postaction={decorate}, decoration={markings, mark=at position 0.7 with {\arrow{Latex[length=3mm]}}}, thick},
                    flechaCW/.style={postaction={decorate}, decoration={markings, mark=at position 0.7 with {\arrow{Latex[length=3mm, reversed]}}}, thick},
                    etiqueta/.style={midway, auto, swap, font=\small}
                ]
                \def\R{0.5} % Radio del hexágono

                \foreach \i in {0,...,2} {
                    \pgfmathsetmacro{\angle}{\i * 120}
                    \coordinate (P\i) at (\angle:\R);
                }

                % Aristas en sentido antihorario
                \draw[flechaCCW] (P0) -- node[etiqueta] {$e$} (P1);
                \draw[flechaCCW] (P1) -- node[etiqueta] {$h$} (P2);
                \draw[flechaCW] (P2) -- node[etiqueta] {$g^{-1}$} (P0); 

                % Colores de los vértices
				\filldraw[fill=red, draw=black] (P0) circle (2pt);
				\filldraw[fill=yellow, draw=black] (P1) circle (2pt);
				\filldraw[fill=red, draw=black] (P2) circle (2pt);
            \end{tikzpicture}
            \end{minipage}\hfill
            \begin{minipage}{0.30\textwidth}
                
                \centering
                 \begin{tikzpicture}[
                    flechaCCW/.style={postaction={decorate}, decoration={markings, mark=at position 0.7 with {\arrow{Latex[length=3mm]}}}, thick},
                    flechaCW/.style={postaction={decorate}, decoration={markings, mark=at position 0.7 with {\arrow{Latex[length=3mm, reversed]}}}, thick},
                    etiqueta/.style={midway, auto, swap, font=\small}
                ]
                \def\R{0.5} % Radio del hexágono

                \foreach \i in {0,...,2} {
                    \pgfmathsetmacro{\angle}{\i * 120}
                    \coordinate (P\i) at (\angle:\R);
                }

                % Aristas en sentido antihorario
                \draw[flechaCCW] (P0) -- node[etiqueta] {$g$} (P1);
                \draw[flechaCCW] (P1) -- node[etiqueta] {$f$} (P2);
                \draw[flechaCW] (P2) -- node[etiqueta] {$i^{-1}$} (P0); 

                % Colores de los vértices
				\filldraw[fill=red, draw=black] (P0) circle (2pt);
				\filldraw[fill=red, draw=black] (P1) circle (2pt);
				\filldraw[fill=red, draw=black] (P2) circle (2pt);
            \end{tikzpicture}
            \end{minipage}
        \end{figure}
        tenemos así que:
        \begin{equation*}
            \chi(S) = V-A+C = 2-9+6 = -1
        \end{equation*}
        Por lo que $S\cong \mathbb{R}\mathbb{P}^2_3$.
    \end{ejercicio}

\end{document}
