\section{Espacios recubridores}

\begin{ejercicio}
    Sea $R=\left]-1,1\right[\subset \mathbb{R}$. Demuestra que existe una aplicación recubridora $p:R\to \mathbb{S}^1$. ¿Se puede levantar al recubridor la aplicación $f:\mathbb{S}^1\to\mathbb{S}^1$ dada por $f(x,y) = (y,x)$?
\end{ejercicio}

\begin{ejercicio}
    Dada la aplicación recubridora estándar $p:\mathbb{R}\to\mathbb{S}^1$ definida por
    \begin{equation*}
        p(x) = (\cos(2\pi x), \sen(2\pi x)),
    \end{equation*}
    determina si la aplicación $f:\mathbb{S}^1\to\mathbb{S}^1$ dada por $f(x,y) = (x,|y|)$ puede ser levantada al recubridora. Y, en tal caso, calcula sus levantamientos.
\end{ejercicio}

\begin{ejercicio}
    ¿Existe una aplicación recubridora desde $\mathbb{S}^1\times \mathbb{S}^1$ en $\mathbb{S}^1$?
\end{ejercicio}

\begin{ejercicio}
    Determina, salvo isomorfismos, todos los recubridores del cilindro $\mathbb{S}^1\times \mathbb{R}$.
\end{ejercicio}

\begin{ejercicio}
    Demuestra que si $p:R\to B$ esun homeomorfismo local con $R$ compacto y $B$ Hausdorff (y conexo), entonces $p$ es una aplicación recubridora.
\end{ejercicio}

\begin{ejercicio}
    Sea $p:\mathbb{R}^n\to \mathbb{R}^n$ un homeomorfismo local tal que para todo $r>0$ se tiene que $p^{-1}(\overline{B}(0,r))$ es compacto. Demuestra que $p$ es un homeomorfismo.
\end{ejercicio}

\begin{ejercicio}
    Sea $X$ conexo y localmente arcoconexo con grupo fundamental finito. Si $f,g:X\to \mathbb{R}$ son aplicaciones continuas cumpliendo que $f(x)^2 + g(x)^2 = 1$ para todo $x\in X$. Prueba que existe $h:X\to\mathbb{R}$ continua tal que $\cos(h(x)) = f(x)$ y $\sen(h(x)) = g(x)$ para cada $x\in X$.
\end{ejercicio}

\begin{ejercicio}
    Sean $p:X\to Y$ y $f:Y\to Z$ dos aplicaciones continuas tales que $p$ y $f\circ p$ son aplicaciones recubridoras. Prueba que $f$ es también una aplicación recubridora.
\end{ejercicio}

\begin{ejercicio}
    Sean $p_1:X\to Y$ y $p_2:Y\to Z$ dos aplicaciones recubridoras. Prueba que si $Z$ tiene recubridor universal, entonces $p_2\circ p_1:X\to Z$ es una aplicación recubridora.
\end{ejercicio}

\begin{ejercicio}
    Sean $p:R\to B$ una aplicación recubridora y $b_0\in B$. Definimos la aplicación (correspondencia del levantamiento generalizada)
    \Func{\phi}{p^{-1}(\{b_0\})\times \pi_1(B,b_0)}{p^{-1}(\{b_0\})}{(r,[\alpha])}{\hat{\alpha}_r(1)}
    donde $\hat{\alpha}_r(s)$ es el levantamiento de $\alpha(s)$ con condición inicial $\alpha(0)=r$. Demuestra que:
    \begin{enumerate}[label=\alph*)]
        \item $\phi$ está bien definida.
        \item $\phi(r,[\varepsilon_{b_0}])=r$, para cualquier $r\in p^{-1}(\{b_0\})$.
        \item $\phi(\phi(r,[\alpha]),[\beta]) = \phi(r,[\alpha]\ast[\beta])$, para cualesquiera $r\in p^{-1}(\{b_0\})$ y $[\alpha],[\beta]\in \pi_1(B,b_0)$.
        \item $\phi$ es sobreyectiva.
        \item $\phi(r,[\alpha]) = r$ si y solo si $[\alpha] \in p_\ast(\pi_1(R,r))$.
        \item El cardinal de $p^{-1}(\{b_0\})$ coincide con el cardinal de $\pi_1(B,b_0)/p_\ast(\pi_1(R,r))$ (es decir, el índice de $p_\ast(\pi_1(R,r))$ como subgrupo de $\pi_1(B,b_0)$).
    \end{enumerate}
\end{ejercicio}

\begin{ejercicio}\label{ej:11_rel3}
    Sea $X$ un espacio topológico (conexo y localmente arcoconexo), $G$ un grupo de homeomorfismos de $X$ y $X/\cc{R}_G$ el espacio topológico cociente dado por la relación de equivalencia:
    \begin{equation*}
        x\cc{R}_G y \quad\Longleftrightarrow\quad\exists \varphi \in G : y = \varphi(x)
    \end{equation*}
    para cualesquiera $x,y\in X$.\newline
    Demuestra que la aplicación proyección $\pi:X\to X/\cc{R}_G$ es recubridora si y solo si para cada $x\in X$ existe un entorno suyo $U_x$ tal que $\varphi(U_x)\cap U_x=\emptyset $ para todo $\varphi\in G\setminus \{Id_X\}$.\newline
    Deduce que, además, $\varphi:(X,\pi)\to (X,\pi)$ es un isomorfismo de recubridores si y solo si $\varphi \in G$.
\end{ejercicio}

\begin{ejercicio}
    Para cada $n\in \mathbb{Z}$ se define $f_n:\mathbb{R}^2\to\mathbb{R}^2$ como $f_n(x,y) = (x+2n,{(-1)}^{n}y)$. Utiliza el ejercicio anterior para demostrar que:
    \begin{enumerate}[label=\alph*)]
        \item $G = \{f_n:n\in \mathbb{Z}\}$ es un grupo de homeomorfismos de $\mathbb{R}^2$ y para cada $x\in \mathbb{R}^2$ existe un entorno suyo $U_x$ tal que $f_n(U_x)\cap U_x = \emptyset $ para todo $n\in \mathbb{Z}\setminus \{0\}$.
        \item La proyección $p:\mathbb{R}^2\to\mathbb{R}^2/\cc{R}_G$ es una aplicación recubridora de $\mathbb{R}^2$ en la cinta de Moebius $\mathbb{R}^2/\cc{R}_G$.
    \end{enumerate}
\end{ejercicio}

\begin{ejercicio}
    Para cada $n,m\in \mathbb{Z}$ se define $f_{n,m}:\mathbb{R}^2\to\mathbb{R}^2$ como:
    \begin{equation*}
        f_{n,m}(x,y) = (x,{(-1)}^{n}y) + 2(n,{(-1)}^{n}m).
    \end{equation*}
    Utiliza el ejercicio~\ref{ej:11_rel3} para demostrar que:
    \begin{enumerate}[label=\alph*)]
        \item $G=\{f_{n,m}:n,m\in \mathbb{Z}\}$ es un grupo de homeomorfismos de $\mathbb{R}^2$ y para cada $x\in \mathbb{R}^2$ existe un entorno suyo $U_x$ tal que $f_{n,m}(U_x)\cap U_x=\emptyset $ para todo $n,m\in \mathbb{Z}$ con $(n,m)\neq (0,0)$.
        \item La proyección $p:\mathbb{R}^2\to\mathbb{R}^2/\cc{R}_G$ es una aplicación recubridora de $\mathbb{R}^2$ en la botella de Klein $\mathbb{R}^2/\cc{R}_G$.
    \end{enumerate}
\end{ejercicio}

\begin{ejercicio}
    Razona si son verdaderas o falsas las siguientes afirmaciones:
    \begin{enumerate}[label=\alph*)]
        \item Existe una aplicación recubridora $p:\mathbb{S}^1\to [0,1]$.
        \item Existe una aplicación recubridora $p:\mathbb{R}\bb{P}^2\to \mathbb{S}^1$.
        \item EL semiplano $X=\{(x,y)\in \mathbb{R}^2 : y\geq 0\}$ es el recubridor universal de la bola cerrada punteada $Y = \{(x,y)\in \mathbb{R}^2 : 0<x^2+y^2 \leq 1\}$.
        \item Si $X$ es un espacio topológicos (conexo y localmente arcoconexo) con grupo fundamental finito y $p:X\to X$ es una aplicación recubridora, entonces $p$ es un homeomorfismo.
    \end{enumerate}
\end{ejercicio}
