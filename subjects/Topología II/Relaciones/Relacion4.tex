\section{Superficies compactas}

\begin{ejercicio}
    Sea $X = \mathbb{S}^2\cup \{x_0\}$, donde $x_0 \in \mathbb{R}^3\setminus \mathbb{S}^2$. En $X$ se considera la topología tal que los entornos de los puntos de $\mathbb{S}^2$ son los usuales, y los de $x_0$ son de la forma $(V\setminus \{N\})\cup \{x_0\}$, donde $N = (0,0,1)$ y $V$ es un entorno de $N$ en $\mathbb{S}^2$. Demuestra que $X$ es localmente euclídeo, es 2AN, pero no es T2.
\end{ejercicio}

\begin{ejercicio}
    Consideremos el espacio producto $X=\mathbb{R}^2\times \mathbb{R}$, donde en $\mathbb{R}^2$ se considera la topología usual y en $\mathbb{R}$ la topología discreta. Demuestra que $X$ es localmente euclídeo, es T2 pero no es 2AN.
\end{ejercicio}

\begin{ejercicio}
    Prueba que los siguientes espacios topológicos no son superficies:
    \begin{enumerate}[label=\alph*)]
        \item $S = \{(x,y,z)\in \mathbb{R}^3 : x^2+y^2-z^2 = 0\}$.
        \item $\mathbb{R}^n$ con $n\neq 2$.
        \item $S = \{(x,y)\in \mathbb{R}^2 : y\geq 0\}$.
    \end{enumerate}
    ¿Es la unión o intersección de dos superficies en $\mathbb{R}^3$ una superficie?
\end{ejercicio}

\begin{ejercicio}
    Sea $(R,p)$ un recubridor de una superficie topológica $S$. Si $R$ es 2AN, demuestra que $R$ es una superficie topológica.
\end{ejercicio}

\begin{ejercicio}
    Sean $S$ una superficie y $f:S\to \mathbb{R}$ una función continua. Definimos el grafo de $f$ como el espacio topológico
    \begin{equation*}
        G(f) = \{(x,t)\in S\times \mathbb{R} : t=f(x)\}
    \end{equation*}
    con la topología inducida por la topología producto en $S\times \mathbb{R}$. Prueba que $G(f)$ es una superficie, que además es compacta si y solo si lo es $S$.
\end{ejercicio}

\begin{ejercicio}
    Prueba que la característica de Euler de la suma conexa de dos superficies compactas es igual a la suma de sus características de Euler menos dos.
\end{ejercicio}

\begin{ejercicio}
    Calcula la característica de Euler de la suma conexa de un plano proyectivo y $n$ toros.
\end{ejercicio}

\begin{ejercicio}
    Estudia la orientabilidad de $S_1 \# S_2$ a partir de la de $S_1$ y de $S_2$.
\end{ejercicio}

\begin{ejercicio}
    Para cada una de las siguientes presentaciones poligonales de superficies compactas, calcula la característica de Euler y determina a cuál de las superficies modelo es homeomorfa:
    \begin{enumerate}[label=\alph*)]
        \item $\langle a,b,c;abacb^{-1}c^{-1} \rangle $.
        \item $\langle a,b,c;abca^{-1}b^{-1}c^{-1} \rangle $.
        \item $\langle a,b,c,d;abcdca^{-1}bd^{-1} \rangle $.
        \item $\langle a,b,c,d,e;aba^{-1}cdb^{-1}c^{-1}ed^{-1}e^{-1} \rangle $.
        \item $\langle a,b,c,d,e,f;abcadb^{-1}efce^{-1}df^{-1} \rangle $.
        \item $\langle a,b,c,d,e,f;abc,bde,c^{-1} d f, e^{-1}fa \rangle $.
        \item $\langle a,b,c,d,e,f,g,h,i,j,k,l,m,n,o;\\ abc, bde,dfg, fhi, haj, c^{-1}kl, e^{-1}mn, g^{-1}ok^{-1}, i^{-1}l^{-1}m^{-1}, j^{-1}n^{-1}o^{-1} \rangle $.
        \item $\langle a,b,c,d,e,f;abc,bde,c^{-1}d f, e^{-1}fa \rangle $.
    \end{enumerate}
\end{ejercicio}

\begin{ejercicio}
    Clasifica la suma conexa de las superficies representadas en los apartados $a)$ y $b)$ del ejercicio anterior.
\end{ejercicio}

\begin{ejercicio}
    Demuestra que toda superficie compacta y conexa es homeomorfa a una y solo una de las siguientes superficies:
    \begin{equation*}
        \mathbb{S}^2,\quad \bb{T}_n, \quad \mathbb{R}\bb{P}^2, \quad K, \quad \bb{T}_n\# \mathbb{R}\bb{P}^2, \quad \bb{T}_n\# K
    \end{equation*}
    donde $K$ denota la botella de Klein.
\end{ejercicio}

\begin{ejercicio}
    Identifica, salvo homeomorfismos, las superficies compactas y conexas con característica de Euler igual a $-2$.
\end{ejercicio}

\begin{ejercicio}
    Sea $S$ una superficie compacta y conexa. Probar que $\chi(S)\geq -2$ si y solo si $S$ tiene una presentación poligonal $\cc{P} = \langle A;W \rangle $ donde $A$ tiene exactamente $4$ elementos.
\end{ejercicio}

\begin{ejercicio}
    Discute de forma razonada si cada par de las siguientes superficies compactas y conexas son homeomorfas entre sí:
    \begin{enumerate}[label=\alph*)]
        \item $S_1$ tiene por presentación poligonal a $\langle a,b,c,d;abcdad^{-1}cb^{-1} \rangle $.
        \item $S_2$ cumple que $\chi(S_2)\geq 0$ y $\pi_1(S_2)$ no es abeliano.
        \item $S_3$ cumple que $\pi_1(S_3)$ es isomorfo al grupo $F(a,b,c)/\langle acbcba^{-1} \rangle_N $.
    \end{enumerate}
\end{ejercicio}

\begin{ejercicio}
    Obten la presentación poligonal canónica de la superficie $S_1$ del ejercicio anterior efectuando para ello las transformaciones elementales que sean necesarias.
\end{ejercicio}

\begin{ejercicio}
    Sea $S$ la superficie compacta y conexa que admite una presentación poligonal de la forma
    \begin{equation*}
        \langle a,b,c,d,e;ab^{-1}c-da^{-1}ebc^{-1}- \rangle 
    \end{equation*}
    donde cada guión - está ocupado por un único símbolo. Completa los guiones correspondientes para que $S$ sea homeomorfa a:
    \begin{enumerate}[label=\alph*)]
        \item $\bb{T}_2$.
        \item $\mathbb{R}\bb{P}^2_4$.
        \item La superficie compacta con grupo fundamental abelianizado isomorfo a $\mathbb{Z}_2\times \mathbb{Z}^4$.
    \end{enumerate}
\end{ejercicio}
