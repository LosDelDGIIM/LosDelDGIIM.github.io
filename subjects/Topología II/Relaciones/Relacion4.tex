\section{Superficies compactas}

\begin{ejercicio}
    Sea $X = \mathbb{S}^2\cup \{x_0\}$, donde $x_0 \in \mathbb{R}^3\setminus \mathbb{S}^2$. En $X$ se considera la topología tal que los entornos de los puntos de $\mathbb{S}^2$ son los usuales, y los de $x_0$ son de la forma $(V\setminus \{N\})\cup \{x_0\}$, donde $N = (0,0,1)$ y $V$ es un entorno de $N$ en $\mathbb{S}^2$. Demuestra que $X$ es localmente euclídeo, es 2AN, pero no es T2.\\

    \noindent
    Para ver que $X$ es localmente euclídeo hemos de probar que todo punto admite un entorno abierto homeomorfo a un abierto de $\mathbb{R}^n$ (en este caso, tendremos $n=2$). Para ello:
    \begin{itemize}
        \item Si $x\in \mathbb{S}^2$, tenemos entonces que $\mathbb{S}^2\setminus\{-x\}$ es un entorno abierto de $x$ homeomorfo a $\mathbb{R}^2$.
        \item Para $x_0\in X$, si consideramos un entorno abierto suyo (donde $V\neq \mathbb{S}^2$) $U = (V\setminus \{N\})\cup \{x_0\}$ podemos considerar la aplicación $f:U\to V$ dada por:
            \begin{equation*}
                f(x) = \left\{\begin{array}{ll}
                    x & \text{si\ } x=x_0 \\
                    N & \text{si\ } x\neq x_0
                \end{array}\right. 
            \end{equation*}
            y obtendremos que $f$ es un homeomorfismo (compruébese), con $V$ un abierto de $\mathbb{S}^2$ distinto de $\mathbb{S}^2$, con lo que este es homeomorfo a un abierto de $\mathbb{R}^2$.
    \end{itemize}
    Para ver que $X$ es 2AN:
    \begin{itemize}
        \item Si $x\in \mathbb{S}^2$ tenemos que:
            \begin{equation*}
                \mathbb{S}^2 \cap \{B(x,r) : r\in \mathbb{Q}\}
            \end{equation*}
            es una base de entornos numerable de $x$.
        \item Para $x_0\in X$ tenemos que:
            \begin{equation*}
                (\mathbb{S}^2 \cap \{B(x,r)\setminus\{N\} : r\in \mathbb{Q}\}) \cup \{x_0\}
            \end{equation*}
            es una base de entornos numerable de $x_0$.
    \end{itemize}
    El espacio topológico $X$ no es T2 porque no podemos ``separar'' $N$ de $x_0$, sea $U$ un entorno abierto de $N$ y $(V\setminus \{N\})\cup \{x_0\}$ un entorno abierto de $x_0$, vemos que $U\cap (V\setminus \{x_0\})$ es no vacío, por lo que no se puede dar la condición para que $X$ sea T2.
\end{ejercicio}

\begin{ejercicio}
    Consideremos el espacio producto $X=\mathbb{R}^2\times \mathbb{R}$, donde en $\mathbb{R}^2$ se considera la topología usual y en $\mathbb{R}$ la topología discreta. Demuestra que $X$ es localmente euclídeo, es T2 pero no es 2AN.\\

    \noindent
    Para ver que $X$ es localmente euclídeo hemos de probar que todo punto admite un entorno abierto homeomorfo a un abierto de $\mathbb{R}^n$ (en este caso, tendremos $n=2$). Para ello, si $(z,x)\in \mathbb{R}^2\times \mathbb{R}$ tenemos que $\mathbb{R}^2\times \{x\}$ es un entorno abierto de $(z,x)$ homeomorfo a $\mathbb{R}^2$, por lo que $X$ es localmente euclídeo.\\

    \noindent
    Para ver que $X$ es T2, dados $(u,x),(v,y)\in \mathbb{R}^2\times \mathbb{R}$ distintos: 
    \begin{itemize}
        \item Si $x\neq y$ vemos que $\mathbb{R}^2\times \{x\}$ es un entorno abierto de $x$, $\mathbb{R}^2\times \{y\}$ es un entorno abierto de $y$ y que:
            \begin{equation*}
                (\mathbb{R}^2\times \{x\})\cap (\mathbb{R}^2\times \{y\}) = \emptyset 
            \end{equation*}
            por ser $x\neq y$, con lo que el espacio topológico $X$ es T2.\\
        \item Si $x=y$ tendremos entonces que $u\neq v$, con lo que $d(u,v)> 0$. Si consideramos:
            \begin{equation*}
                r = \frac{d(u,v)}{2}
            \end{equation*}
            tenemos entonces que los abiertos:
            \begin{equation*}
                B(u,r)\times \{x\}, \qquad B(v,r)\times \{x\}
            \end{equation*}
            son disjuntos, con $(u,x)$ contenido en el primero y $(v,y)$ en el segundo.
    \end{itemize}

    \noindent
    Para ver que $X$ no es 2AN, supongamos por reducción al absurdo que tenemos $\bb{B}$ una base para la topología de $X$ numerable. En dicho caso, para cada $x\in \mathbb{R}$ tenemos que $\mathbb{R}^2\times \{x\}$ es un abierto de $X$, con lo que ha de existir $B_x\in \bb{B}$ de forma que:
    \begin{equation*}
        B_x \subseteq \mathbb{R}^2\times \{x\}
    \end{equation*}
    Esto nos permite construir la aplicación $\Phi:\mathbb{R}\to \bb{B}$ dada por:
    \begin{equation*}
        \Phi(x) = B_x
    \end{equation*}
    Vemos además que $\Phi$ es inyectiva, pues si $x,y\in \mathbb{R}$ son distintos tenemos entonces que $B_x\subseteq \mathbb{R}^2\times \{x\}$, $B_y\subseteq \mathbb{R}^2\times \{y\}$ y que:
    \begin{equation*}
        (\mathbb{R}^2\times \{x\}) \cap (\mathbb{R}^2\times \{y\}) = \emptyset 
    \end{equation*}
    por lo que $B_x\cap B_y = \emptyset $, de donde $B_x \neq B_y$. Esto demuestra que $\Phi$ es inyectiva. Como $\bb{B}$ era numerable tenemos una aplicación $\Psi:\bb{B}\to \mathbb{N}$ inyectiva, con lo que $\Psi\circ\Phi:\mathbb{R}\to \mathbb{N}$ es una aplicación inyectiva, \underline{contradicción} con que $\mathbb{R}$ es no numerable.
\end{ejercicio}

\begin{ejercicio}
    Prueba que los siguientes espacios topológicos no son superficies:
    \begin{enumerate}[label=\alph*)]
        \item $S = \{(x,y,z)\in \mathbb{R}^3 : x^2+y^2-z^2 = 0\}$.
        \item $\mathbb{R}^n$ con $n\neq 2$.
        \item $S = \{(x,y)\in \mathbb{R}^2 : y\geq 0\}$.
    \end{enumerate}
    ¿Es la unión o intersección de dos superficies en $\mathbb{R}^3$ una superficie?
\end{ejercicio}

\begin{ejercicio}
    Sea $(R,p)$ un recubridor de una superficie topológica $S$. Si $R$ es 2AN, demuestra que $R$ es una superficie topológica.
\end{ejercicio}

\begin{ejercicio}
    Sean $S$ una superficie y $f:S\to \mathbb{R}$ una función continua. Definimos el grafo de $f$ como el espacio topológico
    \begin{equation*}
        G(f) = \{(x,t)\in S\times \mathbb{R} : t=f(x)\}
    \end{equation*}
    con la topología inducida por la topología producto en $S\times \mathbb{R}$. Prueba que $G(f)$ es una superficie, que además es compacta si y solo si lo es $S$.
\end{ejercicio}

\begin{ejercicio}
    Prueba que la característica de Euler de la suma conexa de dos superficies compactas es igual a la suma de sus características de Euler menos dos.
\end{ejercicio}

\begin{ejercicio}
    Calcula la característica de Euler de la suma conexa de un plano proyectivo y $n$ toros.
\end{ejercicio}

\begin{ejercicio}
    Estudia la orientabilidad de $S_1 \# S_2$ a partir de la de $S_1$ y de $S_2$.
\end{ejercicio}

\begin{ejercicio}
    Para cada una de las siguientes presentaciones poligonales de superficies compactas, calcula la característica de Euler y determina a cuál de las superficies modelo es homeomorfa:\\

    \noindent
    Recordamos que las superficienes modelo son:
    \begin{equation*}
        \mathbb{S}^2, \qquad \mathbb{T}_n, \qquad \mathbb{R}\bb{P}^2_n
    \end{equation*}

    y tenemos que:
    \begin{equation*}
        \chi(\mathbb{S}^2) = 2, \qquad \chi(\bb{T}_n) = 2(1-n), \qquad \chi(\mathbb{R}\bb{P}^2_n) = 2-n; \qquad n\geq 1
    \end{equation*}
    \begin{enumerate}[label=\alph*)]
        \item $\langle a,b,c;abacb^{-1}c^{-1} \rangle $.

            Sea $|\cc{P}|$ su realización geométrica, tenemos que:
            \begin{equation*}
                \chi(|\cc{P}|) = V-A+C
            \end{equation*}
            Vemos que:
            \begin{itemize}
                \item $C = 1$, pues la presentación solo cuenta con una palabra.
                \item $A = 3$, pues la presentación cuenta con 3 símbolos.
            \end{itemize}
            Para el número de vértices, tenemos que observar la figura:
            \begin{figure}[H]
                \centering
                 \begin{tikzpicture}[
                    flechaCCW/.style={postaction={decorate}, decoration={markings, mark=at position 0.5 with {\arrow{Latex[length=3mm]}}}, thick},
                    flechaCW/.style={postaction={decorate}, decoration={markings, mark=at position 0.5 with {\arrow{Latex[length=3mm, reversed]}}}, thick},
                    etiqueta/.style={midway, auto, swap, font=\small}
                ]
                \def\R{1} % Radio del hexágono

                \foreach \i in {0,...,5} {
                    \pgfmathsetmacro{\angle}{\i * 60}
                    \coordinate (P\i) at (\angle:\R);
                }

                % Aristas en sentido antihorario
                \draw[flechaCCW] (P0) -- node[etiqueta] {$a$} (P1);
                \draw[flechaCCW] (P1) -- node[etiqueta] {$b$} (P2);
                \draw[flechaCCW] (P2) -- node[etiqueta] {$a$} (P3); 
                \draw[flechaCCW] (P3) -- node[etiqueta] {$c$} (P4);
                \draw[flechaCW] (P4) -- node[etiqueta] {$b^{-1}$} (P5);
                \draw[flechaCW] (P5) -- node[etiqueta] {$c^{-1}$} (P0);
            \end{tikzpicture}
            \end{figure}
            \noindent
            Analicemos sus vértices en el espacio topológico cociente:
            \begin{itemize}
                \item El vértice en el que empieza la arista $a$ de la derecha se pega con el vértice que empieza en la arista $a$ de la izquierda, que es el mismo vértice en el que termina $b$, que es el mismo vértice en el que termina $c$, que es el mismo vértice en el que empieza $b$, que es el mismo vértice en el que termina $a$, que es el mismo vértice en el que empieza $c$, que es el mismo vértice en el que empieza $a$.

                    Hemos cerrado el ciclo, por lo que todos estos vértices se identifican en el cociente.
            \end{itemize}
            Hemos dejado al polígono sin vértices, por lo que todos estos son iguales en el cociente:
            \begin{figure}[H]
                \centering
                 \begin{tikzpicture}[
                    flechaCCW/.style={postaction={decorate}, decoration={markings, mark=at position 0.5 with {\arrow{Latex[length=3mm]}}}, thick},
                    flechaCW/.style={postaction={decorate}, decoration={markings, mark=at position 0.5 with {\arrow{Latex[length=3mm, reversed]}}}, thick},
                    etiqueta/.style={midway, auto, swap, font=\small}
                ]
                \def\R{1} % Radio del hexágono

                \foreach \i in {0,...,5} {
                    \pgfmathsetmacro{\angle}{\i * 60}
                    \coordinate (P\i) at (\angle:\R);
                }

                % Aristas en sentido antihorario
                \draw[flechaCCW] (P0) -- node[etiqueta] {$a$} (P1);
                \draw[flechaCCW] (P1) -- node[etiqueta] {$b$} (P2);
                \draw[flechaCCW] (P2) -- node[etiqueta] {$a$} (P3); 
                \draw[flechaCCW] (P3) -- node[etiqueta] {$c$} (P4);
                \draw[flechaCW] (P4) -- node[etiqueta] {$b^{-1}$} (P5);
                \draw[flechaCW] (P5) -- node[etiqueta] {$c^{-1}$} (P0);

                % Colores de los vértices
                \filldraw[fill=yellow, draw=black] (P0) circle (2pt);
                \filldraw[fill=yellow, draw=black] (P1) circle (2pt);
                \filldraw[fill=yellow, draw=black] (P2) circle (2pt);
                \filldraw[fill=yellow, draw=black] (P3) circle (2pt);
                \filldraw[fill=yellow, draw=black] (P4) circle (2pt);
                \filldraw[fill=yellow, draw=black] (P5) circle (2pt);
            \end{tikzpicture}
            \end{figure}
            \noindent
            Tenemos así que $V = 1$, por lo que:
            \begin{equation*}
                \chi(|\cc{P}|) = V-A+C = 1-3+1 = -1
            \end{equation*}
            Vemos que obtenemos un número negativo e impar, por lo que tiene que ser el producto de un plano proyectivo consigo mismo $n$ veces, y como:
            \begin{equation*}
                \chi(\mathbb{R}\mathbb{P}^2_n) = n-2
            \end{equation*}
            vemos entonces que $n=3$, es decir:
            \begin{equation*}
                |\cc{P}| \cong \mathbb{R}\bb{P}^2_3
            \end{equation*}
        \item $\langle a,b,c;abca^{-1}b^{-1}c^{-1} \rangle $.

            Sea $|\cc{P}|$ su realización geométrica, tenemos que:
            \begin{itemize}
                \item $C = 1$.
                \item $A = 3$.
            \end{itemize}
            Para los vértices, vemos que tenemos la figura:
            \begin{figure}[H]
                \centering
                 \begin{tikzpicture}[
                    flechaCCW/.style={postaction={decorate}, decoration={markings, mark=at position 0.5 with {\arrow{Latex[length=3mm]}}}, thick},
                    flechaCW/.style={postaction={decorate}, decoration={markings, mark=at position 0.5 with {\arrow{Latex[length=3mm, reversed]}}}, thick},
                    etiqueta/.style={midway, auto, swap, font=\small}
                ]
                \def\R{1} % Radio del hexágono

                \foreach \i in {0,...,5} {
                    \pgfmathsetmacro{\angle}{\i * 60}
                    \coordinate (P\i) at (\angle:\R);
                }

                % Aristas en sentido antihorario
                \draw[flechaCCW] (P0) -- node[etiqueta] {$a$} (P1);
                \draw[flechaCCW] (P1) -- node[etiqueta] {$b$} (P2);
                \draw[flechaCCW] (P2) -- node[etiqueta] {$c$} (P3); 
                \draw[flechaCW] (P3) -- node[etiqueta] {$a^{-1}$} (P4);
                \draw[flechaCW] (P4) -- node[etiqueta] {$b^{-1}$} (P5);
                \draw[flechaCW] (P5) -- node[etiqueta] {$c^{-1}$} (P0);
            \end{tikzpicture}
            \end{figure}
            \noindent
            Contamos sus vértices:
            \begin{itemize}
                \item El vértice en el que empieza $a$ también es el vértice en el que termina $b$, que también es el vértice en el que empieza $c$, que es el vértice en el que empieza $a$. Hemos cerrado el ciclo.
                \item El vértice en el que termina $a$ es también donde termina $c$, que es donde comienza $b$ y donde termina $a$. Hemos cerrado el ciclo.
            \end{itemize}
            Hemos obtenido así $2$ vértices:
            \begin{figure}[H]
                \centering
                 \begin{tikzpicture}[
                    flechaCCW/.style={postaction={decorate}, decoration={markings, mark=at position 0.5 with {\arrow{Latex[length=3mm]}}}, thick},
                    flechaCW/.style={postaction={decorate}, decoration={markings, mark=at position 0.5 with {\arrow{Latex[length=3mm, reversed]}}}, thick},
                    etiqueta/.style={midway, auto, swap, font=\small}
                ]
                \def\R{1} % Radio del hexágono

                \foreach \i in {0,...,5} {
                    \pgfmathsetmacro{\angle}{\i * 60}
                    \coordinate (P\i) at (\angle:\R);
                }

                % Aristas en sentido antihorario
                \draw[flechaCCW] (P0) -- node[etiqueta] {$a$} (P1);
                \draw[flechaCCW] (P1) -- node[etiqueta] {$b$} (P2);
                \draw[flechaCCW] (P2) -- node[etiqueta] {$c$} (P3); 
                \draw[flechaCW] (P3) -- node[etiqueta] {$a^{-1}$} (P4);
                \draw[flechaCW] (P4) -- node[etiqueta] {$b^{-1}$} (P5);
                \draw[flechaCW] (P5) -- node[etiqueta] {$c^{-1}$} (P0);

                % Colores de los vértices
                \filldraw[fill=yellow, draw=black] (P0) circle (2pt);
                \filldraw[fill=red, draw=black] (P1) circle (2pt);
                \filldraw[fill=yellow, draw=black] (P2) circle (2pt);
                \filldraw[fill=red, draw=black] (P3) circle (2pt);
                \filldraw[fill=yellow, draw=black] (P4) circle (2pt);
                \filldraw[fill=red, draw=black] (P5) circle (2pt);
            \end{tikzpicture}
            \end{figure}
            \noindent
            Tenemos entonces que:
            \begin{equation*}
                \chi(|\cc{P}|) = V-A+C = 2-3+1 = 0
            \end{equation*}
            Por lo que nuestra superficie $|\cc{P}|$ puede ser $\bb{T}$ o $\mathbb{R}\bb{P}^2_2$. Vemos que la presentación poligonal es orientada, por lo que la superficie $|\cc{P}|$ es orientable, luego ha de ser:
            \begin{equation*}
                |\cc{P}| \cong \mathbb{T}
            \end{equation*}
        \item $\langle a,b,c,d;abcdca^{-1}bd^{-1} \rangle $.

            Sea $|\cc{P}|$ la realización geométrica de la presentación, tenemos:
            \begin{itemize}
                \item $C=1$.
                \item $A=4$.
            \end{itemize}
            Tenemos el polígono:
            \begin{figure}[H]
                \centering
                 \begin{tikzpicture}[
                    flechaCCW/.style={postaction={decorate}, decoration={markings, mark=at position 0.7 with {\arrow{Latex[length=3mm]}}}, thick},
                    flechaCW/.style={postaction={decorate}, decoration={markings, mark=at position 0.7 with {\arrow{Latex[length=3mm, reversed]}}}, thick},
                    etiqueta/.style={midway, auto, swap, font=\small}
                ]
                \def\R{1} % Radio del hexágono

                \foreach \i in {0,...,7} {
                    \pgfmathsetmacro{\angle}{\i * 45}
                    \coordinate (P\i) at (\angle:\R);
                }

                % Aristas en sentido antihorario
                \draw[flechaCCW] (P0) -- node[etiqueta] {$a$} (P1);
                \draw[flechaCCW] (P1) -- node[etiqueta] {$b$} (P2);
                \draw[flechaCCW] (P2) -- node[etiqueta] {$c$} (P3); 
                \draw[flechaCCW] (P3) -- node[etiqueta] {$d$} (P4);
                \draw[flechaCCW] (P4) -- node[etiqueta] {$c$} (P5);
                \draw[flechaCW] (P5) -- node[etiqueta] {$a^{-1}$} (P6);
                \draw[flechaCCW] (P6) -- node[etiqueta] {$b$} (P7);
                \draw[flechaCW] (P7) -- node[etiqueta] {$d^{-1}$} (P0);

%                 % Colores de los vértices
%                 \filldraw[fill=yellow, draw=black] (P0) circle (2pt);
%                 \filldraw[fill=yellow, draw=black] (P1) circle (2pt);
%                 \filldraw[fill=yellow, draw=black] (P2) circle (2pt);
%                 \filldraw[fill=yellow, draw=black] (P3) circle (2pt);
%                 \filldraw[fill=yellow, draw=black] (P4) circle (2pt);
%                 \filldraw[fill=yellow, draw=black] (P5) circle (2pt);
            \end{tikzpicture}
            \end{figure}
            \noindent
            Y si calculamos sus vértices obtendremos:
            \begin{equation*}
                V = 2
            \end{equation*}
            ya que:
            \begin{figure}[H]
                \centering
                 \begin{tikzpicture}[
                    flechaCCW/.style={postaction={decorate}, decoration={markings, mark=at position 0.7 with {\arrow{Latex[length=3mm]}}}, thick},
                    flechaCW/.style={postaction={decorate}, decoration={markings, mark=at position 0.7 with {\arrow{Latex[length=3mm, reversed]}}}, thick},
                    etiqueta/.style={midway, auto, swap, font=\small}
                ]
                \def\R{1} % Radio del hexágono

                \foreach \i in {0,...,7} {
                    \pgfmathsetmacro{\angle}{\i * 45}
                    \coordinate (P\i) at (\angle:\R);
                }

                % Aristas en sentido antihorario
                \draw[flechaCCW] (P0) -- node[etiqueta] {$a$} (P1);
                \draw[flechaCCW] (P1) -- node[etiqueta] {$b$} (P2);
                \draw[flechaCCW] (P2) -- node[etiqueta] {$c$} (P3); 
                \draw[flechaCCW] (P3) -- node[etiqueta] {$d$} (P4);
                \draw[flechaCCW] (P4) -- node[etiqueta] {$c$} (P5);
                \draw[flechaCW] (P5) -- node[etiqueta] {$a^{-1}$} (P6);
                \draw[flechaCCW] (P6) -- node[etiqueta] {$b$} (P7);
                \draw[flechaCW] (P7) -- node[etiqueta] {$d^{-1}$} (P0);

                % Colores de los vértices
                \filldraw[fill=yellow, draw=black] (P0) circle (2pt);
                \filldraw[fill=yellow, draw=black] (P1) circle (2pt);
                \filldraw[fill=red, draw=black] (P2) circle (2pt);
                \filldraw[fill=yellow, draw=black] (P3) circle (2pt);
                \filldraw[fill=red, draw=black] (P4) circle (2pt);
                \filldraw[fill=yellow, draw=black] (P5) circle (2pt);
                \filldraw[fill=yellow, draw=black] (P6) circle (2pt);
                \filldraw[fill=red, draw=black] (P7) circle (2pt);
            \end{tikzpicture}
            \end{figure}
            \noindent
            Tenemos así que:
            \begin{equation*}
                \chi(|\cc{P}|) = 2-4+1 = -1
            \end{equation*}
            Por lo que:
            \begin{equation*}
                |\cc{P}|\cong \mathbb{R}\bb{P}^2_3
            \end{equation*}
        \item $\langle a,b,c,d,e;aba^{-1}cdb^{-1}c^{-1}ed^{-1}e^{-1} \rangle $.
            Sea $|\cc{P}|$ su realización geométrica, tenemos que:
            \begin{itemize}
                \item $C=1$.
                \item $A=5$
            \end{itemize}
            El polígono es:
            \begin{figure}[H]
                \centering
                 \begin{tikzpicture}[
                    flechaCCW/.style={postaction={decorate}, decoration={markings, mark=at position 0.7 with {\arrow{Latex[length=3mm]}}}, thick},
                    flechaCW/.style={postaction={decorate}, decoration={markings, mark=at position 0.7 with {\arrow{Latex[length=3mm, reversed]}}}, thick},
                    etiqueta/.style={midway, auto, swap, font=\small}
                ]
                \def\R{1.5} % Radio del hexágono

                \foreach \i in {0,...,9} {
                    \pgfmathsetmacro{\angle}{\i * 36}
                    \coordinate (P\i) at (\angle:\R);
                }

                % Aristas en sentido antihorario
                \draw[flechaCCW] (P0) -- node[etiqueta] {$a$} (P1);
                \draw[flechaCCW] (P1) -- node[etiqueta] {$b$} (P2);
                \draw[flechaCW] (P2) -- node[etiqueta] {$a^{-1}$} (P3); 
                \draw[flechaCCW] (P3) -- node[etiqueta] {$c$} (P4);
                \draw[flechaCCW] (P4) -- node[etiqueta] {$d$} (P5);
                \draw[flechaCW] (P5) -- node[etiqueta] {$b^{-1}$} (P6);
                \draw[flechaCW] (P6) -- node[etiqueta] {$c^{-1}$} (P7);
                \draw[flechaCCW] (P7) -- node[etiqueta] {$e$} (P8);
                \draw[flechaCW] (P8) -- node[etiqueta] {$d^{-1}$} (P9);
                \draw[flechaCW] (P9) -- node[etiqueta] {$e^{-1}$} (P0);
            \end{tikzpicture}
            \end{figure}
            \noindent
            Y tenemos que:
            \begin{equation*}
                V = 2
            \end{equation*}
            ya que:
            \begin{figure}[H]
                \centering
                 \begin{tikzpicture}[
                    flechaCCW/.style={postaction={decorate}, decoration={markings, mark=at position 0.7 with {\arrow{Latex[length=3mm]}}}, thick},
                    flechaCW/.style={postaction={decorate}, decoration={markings, mark=at position 0.7 with {\arrow{Latex[length=3mm, reversed]}}}, thick},
                    etiqueta/.style={midway, auto, swap, font=\small}
                ]
                \def\R{1.5} % Radio del hexágono

                \foreach \i in {0,...,9} {
                    \pgfmathsetmacro{\angle}{\i * 36}
                    \coordinate (P\i) at (\angle:\R);
                }

                % Aristas en sentido antihorario
                \draw[flechaCCW] (P0) -- node[etiqueta] {$a$} (P1);
                \draw[flechaCCW] (P1) -- node[etiqueta] {$b$} (P2);
                \draw[flechaCW] (P2) -- node[etiqueta] {$a^{-1}$} (P3); 
                \draw[flechaCCW] (P3) -- node[etiqueta] {$c$} (P4);
                \draw[flechaCCW] (P4) -- node[etiqueta] {$d$} (P5);
                \draw[flechaCW] (P5) -- node[etiqueta] {$b^{-1}$} (P6);
                \draw[flechaCW] (P6) -- node[etiqueta] {$c^{-1}$} (P7);
                \draw[flechaCCW] (P7) -- node[etiqueta] {$e$} (P8);
                \draw[flechaCW] (P8) -- node[etiqueta] {$d^{-1}$} (P9);
                \draw[flechaCW] (P9) -- node[etiqueta] {$e^{-1}$} (P0);

                % Colores de los vértices
                \filldraw[fill=yellow, draw=black] (P0) circle (2pt);
                \filldraw[fill=red, draw=black] (P1) circle (2pt);
                \filldraw[fill=red, draw=black] (P2) circle (2pt);
                \filldraw[fill=yellow, draw=black] (P3) circle (2pt);
                \filldraw[fill=red, draw=black] (P4) circle (2pt);
                \filldraw[fill=red, draw=black] (P5) circle (2pt);
                \filldraw[fill=red, draw=black] (P6) circle (2pt);
                \filldraw[fill=yellow, draw=black] (P7) circle (2pt);
                \filldraw[fill=red, draw=black] (P8) circle (2pt);
                \filldraw[fill=red, draw=black] (P9) circle (2pt);
            \end{tikzpicture}
            \end{figure}
            \noindent
            En definitiva:
            \begin{equation*}
                \chi(|\cc{P}|) = 2-5+1 = -2
            \end{equation*}
            Por lo que puede ser $\mathbb{T}_2$ o $\mathbb{R}\bb{P}^2_4$. Vemos que la presentación es orientada, por lo que $|\cc{P}|$ es orientable, de donde tiene que ser:
            \begin{equation*}
                |\cc{P}| \cong \mathbb{T}_2
            \end{equation*}

        \item $\langle a,b,c,d,e,f;abcadb^{-1}efce^{-1}df^{-1} \rangle $.
            Sea $|\cc{P}|$ su realización geométrica, tenemos que:
            \begin{itemize}
                \item $C=1$.
                \item $A=6$
            \end{itemize}
            El polígono es:
            \begin{figure}[H]
                \centering
                 \begin{tikzpicture}[
                    flechaCCW/.style={postaction={decorate}, decoration={markings, mark=at position 0.7 with {\arrow{Latex[length=3mm]}}}, thick},
                    flechaCW/.style={postaction={decorate}, decoration={markings, mark=at position 0.7 with {\arrow{Latex[length=3mm, reversed]}}}, thick},
                    etiqueta/.style={midway, auto, swap, font=\small}
                ]
                \def\R{1.5} % Radio del hexágono

                \foreach \i in {0,...,11} {
                    \pgfmathsetmacro{\angle}{\i * 30}
                    \coordinate (P\i) at (\angle:\R);
                }

                % Aristas en sentido antihorario
                \draw[flechaCCW] (P0) -- node[etiqueta] {$a$} (P1);
                \draw[flechaCCW] (P1) -- node[etiqueta] {$b$} (P2);
                \draw[flechaCCW] (P2) -- node[etiqueta] {$c$} (P3); 
                \draw[flechaCCW] (P3) -- node[etiqueta] {$a$} (P4);
                \draw[flechaCCW] (P4) -- node[etiqueta] {$d$} (P5);
                \draw[flechaCW] (P5) -- node[etiqueta] {$b^{-1}$} (P6);
                \draw[flechaCCW] (P6) -- node[etiqueta] {$e$} (P7);
                \draw[flechaCCW] (P7) -- node[etiqueta] {$f$} (P8);
                \draw[flechaCCW] (P8) -- node[etiqueta] {$c$} (P9);
                \draw[flechaCW] (P9) -- node[etiqueta] {$e^{-1}$} (P10);
                \draw[flechaCCW] (P10) -- node[etiqueta] {$d$} (P11);
                \draw[flechaCW] (P11) -- node[etiqueta] {$f^{-1}$} (P0);
            \end{tikzpicture}
            \end{figure}
            \noindent
            Y tenemos que:
            \begin{equation*}
                V =3
            \end{equation*}
            ya que:
            \begin{figure}[H]
                \centering
                 \begin{tikzpicture}[
                    flechaCCW/.style={postaction={decorate}, decoration={markings, mark=at position 0.7 with {\arrow{Latex[length=3mm]}}}, thick},
                    flechaCW/.style={postaction={decorate}, decoration={markings, mark=at position 0.7 with {\arrow{Latex[length=3mm, reversed]}}}, thick},
                    etiqueta/.style={midway, auto, swap, font=\small}
                ]
                \def\R{1.5} % Radio del hexágono

                \foreach \i in {0,...,11} {
                    \pgfmathsetmacro{\angle}{\i * 30}
                    \coordinate (P\i) at (\angle:\R);
                }

                % Aristas en sentido antihorario
                \draw[flechaCCW] (P0) -- node[etiqueta] {$a$} (P1);
                \draw[flechaCCW] (P1) -- node[etiqueta] {$b$} (P2);
                \draw[flechaCCW] (P2) -- node[etiqueta] {$c$} (P3); 
                \draw[flechaCCW] (P3) -- node[etiqueta] {$a$} (P4);
                \draw[flechaCCW] (P4) -- node[etiqueta] {$d$} (P5);
                \draw[flechaCW] (P5) -- node[etiqueta] {$b^{-1}$} (P6);
                \draw[flechaCCW] (P6) -- node[etiqueta] {$e$} (P7);
                \draw[flechaCCW] (P7) -- node[etiqueta] {$f$} (P8);
                \draw[flechaCCW] (P8) -- node[etiqueta] {$c$} (P9);
                \draw[flechaCW] (P9) -- node[etiqueta] {$e^{-1}$} (P10);
                \draw[flechaCCW] (P10) -- node[etiqueta] {$d$} (P11);
                \draw[flechaCW] (P11) -- node[etiqueta] {$f^{-1}$} (P0);

                % Colores de los vértices
                \filldraw[fill=yellow, draw=black] (P0) circle (2pt);
                \filldraw[fill=red, draw=black] (P1) circle (2pt);
                \filldraw[fill=blue, draw=black] (P2) circle (2pt);
                \filldraw[fill=yellow, draw=black] (P3) circle (2pt);
                \filldraw[fill=red, draw=black] (P4) circle (2pt);
                \filldraw[fill=blue, draw=black] (P5) circle (2pt);
                \filldraw[fill=red, draw=black] (P6) circle (2pt);
                \filldraw[fill=yellow, draw=black] (P7) circle (2pt);
                \filldraw[fill=blue, draw=black] (P8) circle (2pt);
                \filldraw[fill=yellow, draw=black] (P9) circle (2pt);
                \filldraw[fill=red, draw=black] (P10) circle (2pt);
                \filldraw[fill=blue, draw=black] (P11) circle (2pt);
            \end{tikzpicture}
            \end{figure}
            \noindent
            Tenemos así que:
            \begin{equation*}
                \chi(|\cc{P}|) = 3-6+1 = -2
            \end{equation*}
            Por lo que puede ser $\mathbb{T}_2$ o $\mathbb{R}\bb{P}^2_4$. Vemos que la presentación es no orientada, por lo que $|\cc{P}|$ es no orientable, de donde será:
            \begin{equation*}
                |\cc{P}| \cong \mathbb{R}\bb{P}^2_4
            \end{equation*}

        \item $\langle a,b,c,d,e,f;abc,bde,c^{-1} d f, e^{-1}fa \rangle $.
            Sea $|\cc{P}|$ la realización geométrica de la presentación, tenemos en este caso que:
            \begin{itemize}
                \item $C=4$, ya que contamos con 4 expresiones.
                \item $A=6$.
            \end{itemize}
            Tenemos en este caso los polígonos:
            \begin{figure}[H]
                \centering
                \begin{minipage}{0.24\textwidth}
                    \centering
                     \begin{tikzpicture}[
                        flechaCCW/.style={postaction={decorate}, decoration={markings, mark=at position 0.7 with {\arrow{Latex[length=3mm]}}}, thick},
                        flechaCW/.style={postaction={decorate}, decoration={markings, mark=at position 0.7 with {\arrow{Latex[length=3mm, reversed]}}}, thick},
                        etiqueta/.style={midway, auto, swap, font=\small}
                    ]
                    \def\R{0.5} % Radio del hexágono

                    \foreach \i in {0,...,2} {
                        \pgfmathsetmacro{\angle}{\i * 120}
                        \coordinate (P\i) at (\angle:\R);
                    }

                    % Aristas en sentido antihorario
                    \draw[flechaCCW] (P0) -- node[etiqueta] {$a$} (P1);
                    \draw[flechaCCW] (P1) -- node[etiqueta] {$b$} (P2);
                    \draw[flechaCCW] (P2) -- node[etiqueta] {$c$} (P0); 
                \end{tikzpicture}
                \end{minipage}\hfill
                \begin{minipage}{0.24\textwidth}
                    
                    \centering
                     \begin{tikzpicture}[
                        flechaCCW/.style={postaction={decorate}, decoration={markings, mark=at position 0.7 with {\arrow{Latex[length=3mm]}}}, thick},
                        flechaCW/.style={postaction={decorate}, decoration={markings, mark=at position 0.7 with {\arrow{Latex[length=3mm, reversed]}}}, thick},
                        etiqueta/.style={midway, auto, swap, font=\small}
                    ]
                    \def\R{0.5} % Radio del hexágono

                    \foreach \i in {0,...,2} {
                        \pgfmathsetmacro{\angle}{\i * 120}
                        \coordinate (P\i) at (\angle:\R);
                    }

                    % Aristas en sentido antihorario
                    \draw[flechaCCW] (P0) -- node[etiqueta] {$b$} (P1);
                    \draw[flechaCCW] (P1) -- node[etiqueta] {$d$} (P2);
                    \draw[flechaCCW] (P2) -- node[etiqueta] {$e$} (P0); 
                \end{tikzpicture}
                \end{minipage}\hfill
                \begin{minipage}{0.24\textwidth}
                    
                    \centering
                     \begin{tikzpicture}[
                        flechaCCW/.style={postaction={decorate}, decoration={markings, mark=at position 0.7 with {\arrow{Latex[length=3mm]}}}, thick},
                        flechaCW/.style={postaction={decorate}, decoration={markings, mark=at position 0.7 with {\arrow{Latex[length=3mm, reversed]}}}, thick},
                        etiqueta/.style={midway, auto, swap, font=\small}
                    ]
                    \def\R{0.5} % Radio del hexágono

                    \foreach \i in {0,...,2} {
                        \pgfmathsetmacro{\angle}{\i * 120}
                        \coordinate (P\i) at (\angle:\R);
                    }

                    % Aristas en sentido antihorario
                    \draw[flechaCW] (P0) -- node[etiqueta] {$c^{-1}$} (P1);
                    \draw[flechaCCW] (P1) -- node[etiqueta] {$d$} (P2);
                    \draw[flechaCCW] (P2) -- node[etiqueta] {$f$} (P0); 
                \end{tikzpicture}
                \end{minipage}\hfill
                \begin{minipage}{0.24\textwidth}
                    
                    \centering
                     \begin{tikzpicture}[
                        flechaCCW/.style={postaction={decorate}, decoration={markings, mark=at position 0.7 with {\arrow{Latex[length=3mm]}}}, thick},
                        flechaCW/.style={postaction={decorate}, decoration={markings, mark=at position 0.7 with {\arrow{Latex[length=3mm, reversed]}}}, thick},
                        etiqueta/.style={midway, auto, swap, font=\small}
                    ]
                    \def\R{0.5} % Radio del hexágono

                    \foreach \i in {0,...,2} {
                        \pgfmathsetmacro{\angle}{\i * 120}
                        \coordinate (P\i) at (\angle:\R);
                    }

                    % Aristas en sentido antihorario
                    \draw[flechaCW] (P0) -- node[etiqueta] {$e^{-1}$} (P1);
                    \draw[flechaCCW] (P1) -- node[etiqueta] {$f$} (P2);
                    \draw[flechaCCW] (P2) -- node[etiqueta] {$a$} (P0); 
                \end{tikzpicture}
                \end{minipage}
            \end{figure}
            \noindent
            Y tenemos que:
            \begin{equation*}
                V = 4
            \end{equation*}
            ya que:
            \begin{figure}[H]
                \centering
                \begin{minipage}{0.24\textwidth}
                    \centering
                     \begin{tikzpicture}[
                        flechaCCW/.style={postaction={decorate}, decoration={markings, mark=at position 0.7 with {\arrow{Latex[length=3mm]}}}, thick},
                        flechaCW/.style={postaction={decorate}, decoration={markings, mark=at position 0.7 with {\arrow{Latex[length=3mm, reversed]}}}, thick},
                        etiqueta/.style={midway, auto, swap, font=\small}
                    ]
                    \def\R{0.5} % Radio del hexágono

                    \foreach \i in {0,...,2} {
                        \pgfmathsetmacro{\angle}{\i * 120}
                        \coordinate (P\i) at (\angle:\R);
                    }

                    % Aristas en sentido antihorario
                    \draw[flechaCCW] (P0) -- node[etiqueta] {$a$} (P1);
                    \draw[flechaCCW] (P1) -- node[etiqueta] {$b$} (P2);
                    \draw[flechaCCW] (P2) -- node[etiqueta] {$c$} (P0); 

                    % Colores de los vértices
                    \filldraw[fill=yellow, draw=black] (P0) circle (2pt);
                    \filldraw[fill=red, draw=black] (P1) circle (2pt);
                    \filldraw[fill=blue, draw=black] (P2) circle (2pt);
                \end{tikzpicture}
                \end{minipage}\hfill
                \begin{minipage}{0.24\textwidth}
                    
                    \centering
                     \begin{tikzpicture}[
                        flechaCCW/.style={postaction={decorate}, decoration={markings, mark=at position 0.7 with {\arrow{Latex[length=3mm]}}}, thick},
                        flechaCW/.style={postaction={decorate}, decoration={markings, mark=at position 0.7 with {\arrow{Latex[length=3mm, reversed]}}}, thick},
                        etiqueta/.style={midway, auto, swap, font=\small}
                    ]
                    \def\R{0.5} % Radio del hexágono

                    \foreach \i in {0,...,2} {
                        \pgfmathsetmacro{\angle}{\i * 120}
                        \coordinate (P\i) at (\angle:\R);
                    }

                    % Aristas en sentido antihorario
                    \draw[flechaCCW] (P0) -- node[etiqueta] {$b$} (P1);
                    \draw[flechaCCW] (P1) -- node[etiqueta] {$d$} (P2);
                    \draw[flechaCCW] (P2) -- node[etiqueta] {$e$} (P0); 

                    % Colores de los vértices
                    \filldraw[fill=red, draw=black] (P0) circle (2pt);
                    \filldraw[fill=blue, draw=black] (P1) circle (2pt);
                    \filldraw[fill=green, draw=black] (P2) circle (2pt);
                \end{tikzpicture}
                \end{minipage}\hfill
                \begin{minipage}{0.24\textwidth}
                    
                    \centering
                     \begin{tikzpicture}[
                        flechaCCW/.style={postaction={decorate}, decoration={markings, mark=at position 0.7 with {\arrow{Latex[length=3mm]}}}, thick},
                        flechaCW/.style={postaction={decorate}, decoration={markings, mark=at position 0.7 with {\arrow{Latex[length=3mm, reversed]}}}, thick},
                        etiqueta/.style={midway, auto, swap, font=\small}
                    ]
                    \def\R{0.5} % Radio del hexágono

                    \foreach \i in {0,...,2} {
                        \pgfmathsetmacro{\angle}{\i * 120}
                        \coordinate (P\i) at (\angle:\R);
                    }

                    % Aristas en sentido antihorario
                    \draw[flechaCW] (P0) -- node[etiqueta] {$c^{-1}$} (P1);
                    \draw[flechaCCW] (P1) -- node[etiqueta] {$d$} (P2);
                    \draw[flechaCCW] (P2) -- node[etiqueta] {$f$} (P0); 

                    % Colores de los vértices
                    \filldraw[fill=yellow, draw=black] (P0) circle (2pt);
                    \filldraw[fill=blue, draw=black] (P1) circle (2pt);
                    \filldraw[fill=green, draw=black] (P2) circle (2pt);
                \end{tikzpicture}
                \end{minipage}\hfill
                \begin{minipage}{0.24\textwidth}
                    
                    \centering
                     \begin{tikzpicture}[
                        flechaCCW/.style={postaction={decorate}, decoration={markings, mark=at position 0.7 with {\arrow{Latex[length=3mm]}}}, thick},
                        flechaCW/.style={postaction={decorate}, decoration={markings, mark=at position 0.7 with {\arrow{Latex[length=3mm, reversed]}}}, thick},
                        etiqueta/.style={midway, auto, swap, font=\small}
                    ]
                    \def\R{0.5} % Radio del hexágono

                    \foreach \i in {0,...,2} {
                        \pgfmathsetmacro{\angle}{\i * 120}
                        \coordinate (P\i) at (\angle:\R);
                    }

                    % Aristas en sentido antihorario
                    \draw[flechaCW] (P0) -- node[etiqueta] {$e^{-1}$} (P1);
                    \draw[flechaCCW] (P1) -- node[etiqueta] {$f$} (P2);
                    \draw[flechaCCW] (P2) -- node[etiqueta] {$a$} (P0); 

                    % Colores de los vértices
                    \filldraw[fill=red, draw=black] (P0) circle (2pt);
                    \filldraw[fill=green, draw=black] (P1) circle (2pt);
                    \filldraw[fill=yellow, draw=black] (P2) circle (2pt);
                \end{tikzpicture}
                \end{minipage}
            \end{figure}
            \noindent
            Por lo que:
            \begin{equation*}
                \chi(|\cc{P}|) = 4-6+4 = 2
            \end{equation*}
            Y la única posibilidad es $|\cc{P}|\cong \mathbb{S}^2$.
        \item $\langle a,b,c,d,e,f,g,h,i,j,k,l,m,n,o;\\ abc, bde,dfg, fhi, haj, c^{-1}kl, e^{-1}mn, g^{-1}ok^{-1}, i^{-1}l^{-1}m^{-1}, j^{-1}n^{-1}o^{-1} \rangle $. % // TODO:
    \end{enumerate}
\end{ejercicio}

\begin{ejercicio}
    Clasifica la suma conexa de las superficies representadas en los apartados $a)$ y $b)$ del ejercicio anterior.
\end{ejercicio}

\begin{ejercicio}
    Demuestra que toda superficie compacta y conexa es homeomorfa a una y solo una de las siguientes superficies:
    \begin{equation*}
        \mathbb{S}^2,\quad \bb{T}_n, \quad \mathbb{R}\bb{P}^2, \quad K, \quad \bb{T}_n\# \mathbb{R}\bb{P}^2, \quad \bb{T}_n\# K
    \end{equation*}
    donde $K$ denota la botella de Klein.
\end{ejercicio}

\begin{ejercicio}
    Identifica, salvo homeomorfismos, las superficies compactas y conexas con característica de Euler igual a $-2$.
\end{ejercicio}

\begin{ejercicio}
    Sea $S$ una superficie compacta y conexa. Probar que $\chi(S)\geq -2$ si y solo si $S$ tiene una presentación poligonal $\cc{P} = \langle A;W \rangle $ donde $A$ tiene exactamente $4$ elementos.
\end{ejercicio}

\begin{ejercicio}
    Discute de forma razonada si cada par de las siguientes superficies compactas y conexas son homeomorfas entre sí:
    \begin{enumerate}[label=\alph*)]
        \item $S_1$ tiene por presentación poligonal a $\langle a,b,c,d;abcdad^{-1}cb^{-1} \rangle $.
        \item $S_2$ cumple que $\chi(S_2)\geq 0$ y $\pi_1(S_2)$ no es abeliano.
        \item $S_3$ cumple que $\pi_1(S_3)$ es isomorfo al grupo $F(a,b,c)/\langle acbcba^{-1} \rangle_N $.
    \end{enumerate}
\end{ejercicio}

\begin{ejercicio}
    Obten la presentación poligonal canónica de la superficie $S_1$ del ejercicio anterior efectuando para ello las transformaciones elementales que sean necesarias.
\end{ejercicio}

\begin{ejercicio}
    Sea $S$ la superficie compacta y conexa que admite una presentación poligonal de la forma
    \begin{equation*}
        \langle a,b,c,d,e;ab^{-1}c-da^{-1}ebc^{-1}- \rangle 
    \end{equation*}
    donde cada guión - está ocupado por un único símbolo. Completa los guiones correspondientes para que $S$ sea homeomorfa a:
    \begin{enumerate}[label=\alph*)]
        \item $\bb{T}_2$.
        \item $\mathbb{R}\bb{P}^2_4$.
        \item La superficie compacta con grupo fundamental abelianizado isomorfo a $\mathbb{Z}_2\times \mathbb{Z}^4$.
    \end{enumerate}
\end{ejercicio}
