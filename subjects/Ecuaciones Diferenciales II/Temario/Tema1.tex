\chapter{Movimiento de partículas en un fluido}
\noindent
Fijado un espacio $\Omega\subseteq \mathbb{R}^3$, supondremos siempre que será abierto\footnote{Queremos estudiar movimientos de partículas en un fluido y pasan cosas no deseadas en la frontera.} y conexo\footnote{Por comodidad.}.

\begin{notacion}
    A cada punto de $\Omega$ lo notaremos normalmente por $p\in \Omega$, que tendrá coordenadas:
    \begin{equation*}
        p = (x,y,z)
    \end{equation*}
\end{notacion}

\noindent
El fluido suele llevar en cada punto una dirección y una velocidad, por lo que en cada punto $p\in \Omega$ tendremos un vector $v$ que determina la velocidad del fluido, la cual supondremos conocida siempre. Este vector no tiene por qué ser constante sino que puede depender del tiempo, por lo que generalmente $v$ será una función $v = v(t,p)$, es decir, $v:\mathbb{R}\times \Omega\to \mathbb{R}^3$ es una aplicación que a cada momento $t$ y posición $p$ le asigna un vector $v(t,p)$.\\

\noindent
Si tenemos una partícula en el fluido que se mueve esta determinará una trayectoria, que podemos modelar como una curva parametrizada $p(t)$.\\

\noindent
Podremos medir la velocidad de la partícula de dos formas: observando la velocidad de la partícula de forma interna (como el cuentakilómetros de un coche) o suponiendo que es una partícula que se deja llevar por la corriente y que su velocidad es la del fluido. La velocidad es la derivada de $p$.\\

\noindent
En el segundo caso:
\begin{equation*}
    \dot{p}(t) = V(t,p(t))
\end{equation*}
Estamos ante una ecuación diferencial. Observemos que es un sistema de primer orden en forma normal, puesto que las coordenadas de $p$ son funciones del tiempo. Si escribimos $V = (u,v,w)$ tenemos:
\begin{equation*}
    \left\{\begin{array}{l}
            \dot{x} = u(t,x,y,z) \\
            \dot{y} = v(t,x,y,z) \\
            \dot{z} = w(t,x,y,z)
    \end{array}\right.
\end{equation*}
Un sistema de 3 ecuaciones y 3 incógnitas, que conocidas las velocidades del fluido determina la trayectoria de la partícula. 

\begin{observacion}
    Observemos que en la notación primera hemos escrito explícitamente que $\dot{p}$ está en función de $t$. En la segunda notación estamos usando la notación habitual en las ecuaciones diferenciales, omitiendo la dependencia de $t$ y pensándola de forma implícita. Podemos así reescribir la primera ecuación como:
    \begin{equation*}
        \dot{p} = V(t,p)
    \end{equation*}
    Una vez denotamos $\dot{p}(t)$ estamos diciendo que es una solución concreta, por lo que será una variable dependiente de $t$. Sin embargo, cuando hablamos de $V(t,p)$ vemos $p$ como una variable independiente.
\end{observacion}

\noindent
Aparecerán sistemas autónomos, que representan fluidos estacionarios, donde la velocidad $V(t,p)$ no depende del tiempo.\\

\begin{ejemplo} % // TODO: Ver folio 1
    Comenzamos primero con dos ejemplos de fluidos estacionarios (el vector en cada posición no depende del tiempo):
    \begin{itemize}
        \item Consideramos $\Omega = \mathbb{R}^3$ y tomamos:
            \begin{equation*}
                V(t,x,y,z) = (0,1,0)
            \end{equation*}
            Observamos que es un sistema autónomo (no depende del tiempo) y en el que la velocidad tampoco depende de la posición:
            \begin{equation*}
                \left\{\begin{array}{l}
                        \dot{x} = 0 \\
                        \dot{y} = 1 \\
                        \dot{z} = 0 
                \end{array}\right.
            \end{equation*}
            Con lo que las soluciones son de la forma:
            \begin{align*}
                &x(t) = c_1 \\
                &y(t) = t + c_2 \\
                &z(t) = c_3
            \end{align*}
            Es una familia que depende de 3 parámetros.
        \item Tomamos la ecuación de un vórtice lineal, $\Omega = \mathbb{R}^3$ y:
            \begin{equation*}
                V(t,x,y,z) = (y, -x, 0)
            \end{equation*}
            Tenemos el sistema:
            \begin{equation*}
                \left\{\begin{array}{l}
                        \dot{x} = y \\
                        \dot{y} = -x \\
                        \dot{z} = 0
                \end{array}\right.
            \end{equation*}
            Es un sistema lineal homogéneo, que nos da las soluciones:
            \begin{align*}
                z(t) = c_3
            \end{align*}
            Reducimos a una ecuación de segundo orden, derivando en la primera:
            \begin{equation*}
                \ddot{x} + \dot{x} = 0
            \end{equation*}
            Por lo que:
            \begin{equation*}
                x(t) = c_1\cos(t) + c_2\sen(t)
            \end{equation*}
            Y también tendremos:
            \begin{equation*}
                y(t) = -c_1\sen(t) + c_2\cos(t)
            \end{equation*}
            Para ciertas condiciones iniciales $c_1,c_2,c_3\in \mathbb{R}$.
            % // TODO: Pensar por qué esto describe circunferencias
    \end{itemize}
\end{ejemplo}~\\

\noindent
Nuestro objetivo ahora es tratar de probar que el sistema:
\begin{equation*}
    \dot{p} = V(t,p)
\end{equation*}
con la condición inicial $p(t_0) = p_0$ tiene una solución. Luego trataremos de ver que en cada condición inicial tenemos una única solución. Demostraremos el Teorema de existencia y unicidad en cualquier número de dimensiones.

\begin{notacion}
    Notaremos a los puntos por $x=(x_1,\ldots,x_d)\in \mathbb{R}^d$ y al campo que define la ecuación diferencial por $X$, que será función de $t$ y de $x$, que estará definido en un conjunto $D\subseteq \mathbb{R}\times \mathbb{R}^d$ y exigiremos que sea abierto y conexo. Así, tendremos
    \Func{X}{D}{\bb{R}^d}{(t,x)}{X(t,x)}
    donde el campo $X$ tiene $d$ coordenadas:
    \begin{equation*}
        X = (X_1, \ldots, X_d)
    \end{equation*}
    donde tratamos de resolver la ecuación $\dot{x} = X(t,x)$, que en realidad es un sistema de ecuaciones:
    \begin{equation*}
        \left\{\begin{array}{l}
                \dot{x_1} = X_1(t,x_1,\ldots,x_d) \\
                \vdots \\
                \dot{x_d} = X_d(t,x_1,\ldots,x_d) 
        \end{array}\right.
    \end{equation*}
    Supondremos siempre que $X$ es una función continua.\\

    \noindent
    Tomaremos $(t_0,x_0) \in D$ y queremos resolver el problema de condiciones iniciales
    \begin{equation*}
        \dot{x} = X(t,x), \qquad x(t_0) = x_0
    \end{equation*}
    que consiste en resolver la ecuación diferencial superior mediante una solución que cumpla la condición enunciada.
\end{notacion}~\\

\section{Problema a resolver}
\noindent
Así, queremos resolver el problema:
\begin{equation}\label{eq:problema}
    \dot{x} = X(t,x)
\end{equation}
con $X:D\to \mathbb{R}^d$ continuo, donde $D\subset \mathbb{R}\times\mathbb{R}^d$ es un subconjunto abierto y conexo. La condición de que $X$ sea continuo es equivalente a que cada una de sus coordenadas
\begin{equation*}
    X = (X_1,\ldots,X_d)
\end{equation*}
sea una aplicación continua.

\begin{definicion}
    Una \textbf{solución} del problema~\eqref{eq:problema} es una aplicación $x:I\to \mathbb{R}^d$ donde  $I\subset \mathbb{R}$ es un intervalo\footnote{No necesariamente abierto.} de forma que $x(t)\subset D$:
    \begin{enumerate}
        \item[$i)$] $x(t)$ es derivable.
        \item[$ii)$] $(t,x(t))\in D\qquad \forall t\in I$.
        \item[$iii)$] $\dot{x}(t) = X(t,x(t)) \qquad \forall t\in I$.
    \end{enumerate}
\end{definicion}

\begin{observacion}
    Observemos que la definición de solución del problema planteado anteriormente es equivalente si sustituimos la condición de que $x$ sea derivable por la condición de que $x$ sea de clase 1, puesto que la condición $iii)$ nos da automáticamente la continuidad de la derivada de $x$. 

    \noindent
    Así, la condición $i)$ se puede sustituir por la condición $i')~x\in C^1(I,\mathbb{R}^d)$.
\end{observacion}

\begin{ejemplo}
    Veamos un par de ejemplos para fijar las ideas:
    \begin{enumerate}
        \item Consideramos $\dot{x} = x^2$ y la aplicación $x(t) = \dfrac{1}{1-t}$. Vamos a analizar el marco del problema.

            Tenemos en este caso $d=1$ y $D=\mathbb{R}^2$, puesto que tenemos el campo vectorial $X(t,x) = x^2$, que es continuo en todo $D=\mathbb{R}\times \mathbb{R}$. 

            Tenemos una primera solución en $I_1 = \left]-\infty,1\right[$, $x_1(t) = \frac{1}{1-t}$.

            Y otra solución en $I_2 = \left]1,+\infty\right[$, $x_2(t) = \frac{1}{1-t}$.

            Son \underline{dos soluciones distintas}, con la misma fórmula.
        \item Consideramos ahora $\dot{x} = \frac{x}{t}$ y la aplicación $x(t) = -7t$.

            Tenemos $d=1$ y el campo es $X(t,x) = \frac{x}{t}$, que tiene dos posibles dominios de definición:
            \begin{equation*}
                D_1 = \mathbb{R}^-\times \mathbb{R}, \qquad D_2 = \mathbb{R}^+\times \mathbb{R}
            \end{equation*}
            Tenemos en realidad dos ecuaciones diferenciales.

            Para discutir una de sus soluciones, si consideramos:
            \begin{equation*}
                I_1 = \mathbb{R}^-, \qquad I_2 = \mathbb{R}^+
            \end{equation*}
            tenemos que $x\big|_{I_1}$ es solución de la ecuación diferencial considerando el dominio $D_1$ y análogamente para $x\big|_{I_2}$ con el dominio $D_2$.
    \end{enumerate}
\end{ejemplo}

\noindent
Para una ecuación $\dot{x} = X(t,x)$ definida en un dominio $D\subset\mathbb{R}\times \mathbb{R}^d$ abierto y conexo, acompañaremos usualmente la ecuación de una \textbf{condición inicial}, que es fijar un punto $(t_0,x_0)\in D$ y obligaremos a la solución $x(t)$ de la ecuación que verifique la condición:
\begin{equation*}
    x(t_0) = x_0
\end{equation*}

\begin{ejemplo}
    En el primer ejemplo anterior tenemos para la condición inicial $x(0) = 1$ que la solución de $\dot{x} = x^2$ es $x:\left]-\infty,1\right[\to \mathbb{R}$ dada por:
    \begin{equation*}
        x(t) = \frac{1}{1-t}
    \end{equation*}
\end{ejemplo}

\begin{teo}[Cauchy-Peano, Existencia de soluciones de ecuaciones diferenciales.]\label{teo:existencia}
    Todo problema de valores iniciales admite una solución definida en algún intervalo\footnote{Por tanto, el Teorema nos da una solución local.} $I$ con $t_0\in I$.
\end{teo}

\begin{observacion}
    El ejepmlo anterior nos demuestra que el Teorema de Cauchy-Peano no puede generalizarse a otro que resuelva el problema de forma global, puesto que tenemos una solución para la ecuación que no puede extenderse a todo $\mathbb{R}$; a pesar de ser el dominio de la ecuación todo $\mathbb{R}^2$.
\end{observacion}

\noindent
Lo que venga a continuación estará destinado a probar el Teorema anterior.\\

\noindent
El primer paso será pasar el problema de valores iniciales a una ecuación integral, puesto que la ecuación diferencial y el problema de condiciones iniciales son muy intuitivos pero difíciles de manipular desde el punto de vista de las demostraciones.

\section{Ecuación integral de Volterra}
La ecuación integral de Volterra es:
\begin{equation}\label{eq:Volterra}
    x(t) = x_0 + \int_{t_0}^{t} X(s,x(s))~ds 
\end{equation}
Es la ecuación que obtenemos al integrar $\dot{x} = X(t,x)$. Así, cada ecuación diferencial tiene una ecuación de Volterra y cada ecuación de Volterra tiene una ecuación diferencial.

\begin{ejemplo}
    El problema de valores iniciales $\dot{x} = 3x$ con la condición inicial $x(2) = -5$ tiene ecuación integral de Volterra:
    \begin{equation}\label{eq:ejmVolterra}
        x(t) = -5 + \int_{2}^{t} 3x(s)~ds 
    \end{equation}
\end{ejemplo}

\begin{definicion}
    Una solución de~\eqref{eq:Volterra} es una función $x:I\to \mathbb{R}^d$ con $I$ un intervalo que contiene a $t_0$ en su interior y:
    \begin{enumerate}
        \item[$i)$] $x$ es continua.
        \item[$ii)$] $(t,x(t)) \in D\qquad \forall t\in I$.
        \item[$iii)$] Se cumple la condición~\eqref{eq:Volterra} para todo $t\in I$.
    \end{enumerate}
\end{definicion}

\begin{ejemplo}
    Una solución de la ecuación integral de Volterra~\eqref{eq:ejmVolterra} es
    \begin{equation*}
        x(t) = -5e^{3(t-2)} \qquad \forall t\in \mathbb{R}
    \end{equation*}
\end{ejemplo}

\noindent
La estrategia será enunciar los Teoremas como ecuaciones diferenciales y realizar sus demostraciones con ecuaciones integrales.

\begin{prop}
    Los problemas~\eqref{eq:problema} y~\eqref{eq:Volterra} son equivalentes\footnote{Es decir, tienen las mismas soluciones.}.
    \begin{proof}
        % // TODO: HACER
    \end{proof}
\end{prop}
